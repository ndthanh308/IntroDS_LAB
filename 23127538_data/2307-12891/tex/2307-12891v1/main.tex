\documentclass[aps,prl,floatfix,10pt,superscriptaddress, twocolumn]{revtex4-2}
 
\usepackage[english]{babel}
\usepackage{subcaption}
\usepackage{amsmath}
\usepackage{graphicx}
\usepackage[colorlinks=true, allcolors=blue]{hyperref}

\begin{document}




\title{Design and sensitivity of a 6-axis seismometer for gravitational wave observatories}
\date{\today}

\author{Leonid Prokhorov}
\email{l.g.prokhorov@bham.ac.uk}
\affiliation{Institute for Gravitational Wave Astronomy, School of Physics and Astronomy, University of Birmingham, Birmingham B15 2TT, United Kingdom}

\author{Sam Cooper}
\affiliation{Institute for Gravitational Wave Astronomy, School of Physics and Astronomy, University of Birmingham, Birmingham B15 2TT, United Kingdom}

\author{Amit Singh Ubhi}
\affiliation{Institute for Gravitational Wave Astronomy, School of Physics and Astronomy, University of Birmingham, Birmingham B15 2TT, United Kingdom}

\author{Conor Mow-Lowry}
\affiliation{Institute for Gravitational Wave Astronomy, School of Physics and Astronomy, University of Birmingham, Birmingham B15 2TT, United Kingdom}
\affiliation{Department of Physics and Astronomy, VU Amsterdam, 1081 HV, Amsterdam, The Netherlands}
\affiliation{Dutch National Institute for Subatomic Physics, Nikhef, 1098 XG, Amsterdam, Netherlands}

\author{John Bryant}
\affiliation{Institute for Gravitational Wave Astronomy, School of Physics and Astronomy, University of Birmingham, Birmingham B15 2TT, United Kingdom}

\author{Artemiy Dmitriev}
\affiliation{Institute for Gravitational Wave Astronomy, School of Physics and Astronomy, University of Birmingham, Birmingham B15 2TT, United Kingdom}

\author{Chiara Di Fronzo}
\affiliation{Institute for Gravitational Wave Astronomy, School of Physics and Astronomy, University of Birmingham, Birmingham B15 2TT, United Kingdom}

\author{Christopher J. Collins}
\affiliation{Institute for Gravitational Wave Astronomy, School of Physics and Astronomy, University of Birmingham, Birmingham B15 2TT, United Kingdom}


\author{Alex Gill}
\affiliation{Institute for Gravitational Wave Astronomy, School of Physics and Astronomy, University of Birmingham, Birmingham B15 2TT, United Kingdom}

\author{Alexandra Mitchell}
\affiliation{Department of Physics and Astronomy, VU Amsterdam, 1081 HV, Amsterdam, The Netherlands}
\affiliation{Dutch National Institute for Subatomic Physics, Nikhef, 1098 XG, Amsterdam, Netherlands}

\author{Joscha Heinze}
\affiliation{Institute for Gravitational Wave Astronomy, School of Physics and Astronomy, University of Birmingham, Birmingham B15 2TT, United Kingdom}

\author{Jiri Smetana}
\affiliation{Institute for Gravitational Wave Astronomy, School of Physics and Astronomy, University of Birmingham, Birmingham B15 2TT, United Kingdom}

\author{Tianliang Yan}
\affiliation{Institute for Gravitational Wave Astronomy, School of Physics and Astronomy, University of Birmingham, Birmingham B15 2TT, United Kingdom}

\author{Alan V. Cumming}
\affiliation{Institute for Gravitational Wave Research, School of Physics and Astronomy, University of Glasgow, Glasgow G12 8QQ, United Kingdom}

\author{Giles Hammond}
\affiliation{Institute for Gravitational Wave Research, School of Physics and Astronomy, University of Glasgow, Glasgow G12 8QQ, United Kingdom}

\author{Denis Martynov}
\affiliation{Institute for Gravitational Wave Astronomy, School of Physics and Astronomy, University of Birmingham, Birmingham B15 2TT, United Kingdom}


\begin{abstract}
We present the design, control system, and noise analysis of a 6-axis seismometer comprising a mass suspended by a single fused silica fibre. We utilise custom-made, compact Michelson interferometers for the readout of the mass motion relative to the table and successfully overcome the sensitivity of existing commercial seismometers by over an order of magnitude in the angular degrees of freedom. We develop the sensor for gravitational-wave observatories, such as LIGO, Virgo, and KAGRA, to help them observe intermediate-mass black holes, increase their duty cycle, and improve localisation of sources. Our control system and its achieved sensitivity makes the sensor suitable for other fundamental physics experiments, such as tests of semiclassical gravity, searches for bosonic dark matter, and studies of the Casimir force.  
\end{abstract}

\maketitle

% \section{Intro}

% For many projects in different areas, seismic motion is one of the key noise sources.
Ground vibrations reduce the duty cycle of interferometric gravitational-wave detectors, such as Advanced LIGO~\cite{Aasi2015} and Advanced Virgo~\cite{Acernese2014}, and limit their sensitivity below 25\,Hz~\cite{Buikema2020, Martynov_Noise_2016}. This frequency band is important for the observation of intermediate-mass black holes, accumulation of signal-to-noise ratio from lighter sources, and localisation and advanced warning of neutron star mergers for multi-messenger observations~\cite{Branchesi_2016, Yu2018}. For future detectors, such as Cosmic Explorer~\cite{Evans_CE_2021} and Einstein Telescope~\cite{Maggiore_2020}, inertial isolation is crucial for achieving the design sensitivity and for observing red-shifted signals from cosmological sources.

The LIGO test masses are suspended from actively controlled platforms, which suppress seismic vibrations in the frequency band from 100\,mHz up to 40\,Hz~\cite{Matichard_2015, MATICHARD2015273, MATICHARD2015287}. LIGO utilises commercially available Trillium 240 seismometers to measure the platform motion in its three translational and three rotational degrees of freedom. The seismometer signals are fed into six feedback loops that actuate on the platform with coil-magnet pairs. This scheme had proved successful through the LIGO detectors' observation of tens of gravitational-wave sources~\cite{LIGO_2019, Abbott2021GWTC2}. However, better inertial sensors are required to further improve the low-frequency sensitivity, duty cycle, and to simplify the lock acquisition process.

% LIGO seismic isolation combines active isolation as an initial stage and passive isolation to reach the ultimate performance~\cite{Matichard_2015, MATICHARD2015273, MATICHARD2015287}. Active isolation uses a number of inertial sensors such as Nanometrics Trillium T240 seismometers and Geotech GS13 geophones to stabilise the platforms from which the main interferometer mirrors and most of the auxiliary optics are suspended. Using a multi-stage pendulum and spring suspensions provides passive vibration isolation at frequencies higher than the resonances, e.g. from several Hz. Lowering the pendulum resonances is limited by the length of the suspension and can hardly be improved for existing detectors. 

%Active stabilisation of the test mass suspension could be improved with better sensors. 
%There are two factors which arise from how seismometers work which make measurements of the low-frequency seismic motion less trivial though. 

%A mechanical inertial sensor measures the displacement of its frame relative to a mass suspended from the frame. The output signal is proportional to the seismic noise at frequencies higher than the suspension eigen mode. At lower frequencies, the seismometer response is significantly suppressed by the common motion of the seismometer frame and the suspended mass. Therefore, it is beneficial to reduce the eigen frequency of the suspended mass if the seismometer sensitivity is limited by its readout noise rather than by unwanted forces on the suspended mass.

There is significant effort in the gravitational-wave community to develop new sensors, including beam rotation sensors~\cite{Venkateswara2014, Ross2020} and accelerometers with optical readout~\cite{Collette2015, Heijningen_2018}. These sensors measure platform motion along one axis and mechanically constrain the suspended mass in the other five degrees of freedom. This approach simplifies the control scheme of the platform but leads to a complex mechanical design of the system due to the need for six sensors for stabilisation.

%This relative motion is equal to the frame motion at high frequencies and falls as $f^2$ below the resonant frequency of the reference mass suspension. One can calculate the ground motion below the resonance by correcting it for a proper plant response, but this correction function quickly grows and determines the low-frequency noise spectra. Thus, for the low-frequency measurements, we need an extremely low-frequency suspension of the seismometer mass. 
%he second problem arises from the fact that typical seismometers can't distinguish tilt and constant acceleration. As softer low-frequency suspension becomes more prone to tilt-to-translation coupling, for low-frequency seismic measurements a tiltmeter is needed near the seismometer. 
%There are several tilt meters in different stages of development with different sensors.

We pursue an alternative approach with a simple mechanical design. Instead of utilising six one-axis sensors, we softly suspend a single mass without any mechanical constraints in all six degrees of freedom~\cite{MowLowry2019, Ubhi2022}. We then measure the position of the active platform relative to the suspended mass. Mechanical simplicity leads to cross-couplings between degrees of freedom; however, real-time processing can reduce the cross-couplings and enable successful stabilisation of the platform~\cite{Ubhi_2022b}. In this Letter, we discuss the design, control system, and sensitivity of the seismometer. 

% An alternative approach to platform stabilisation, which is in development in this paper, suggests using one mass as a reference for all 6 degrees of freedom \cite{MowLowry2019}. 
%With this approach, there is no need for 6 different sensors, which are not colocated.  It benefits from a simpler cross-coupling. The reference mass is designed to have relatively large moments of inertia and soft suspensions for all degrees of freedom to improve low-frequency performance. We use interferometric sensing of the reference mass position, which benefits from stability and low noise.

% Figure environment removed


We develop the seismometer for the gravitational-wave observatories. However, the sensor may be applied in other fundamental physics experiments due to its sensitivity and stability. Particularly, in a separate experiment, we utilise the sensor and a high-finesse optical cavity to search for the signatures of semiclassical gravity~\cite{Yang_2013, Helou_2017, Liu_2023}. The sensor also has the potential to improve limits on the bosonic dark matter fields, which couple to baryon minus lepton number~\cite{Shaw_2022}. Furthermore, the seismometer is optimally set to study the Casimir forces between metamaterials~\cite{Rosa_2008, Zhao_2011}, due to its high sensitivity to torques in all rotational degrees of freedom.

% \section{Experimental design}

%% Figure environment removed

\textit{Seismometer design.}- The central element of the setup is a 1-kg mass suspended with a single fused silica fibre as shown in Fig.~\ref{fig:setup}. The mass consists of three aluminium tubes with a wall thickness of 0.5\,mm and 100-g end masses made of brass. The end masses can be moved vertically to adjust the center of mass position. The mass moments of inertia around X,Y, and Z axes are $I_{zz} \approx 0.14$\,m\,kg$^2$, $I_{xx} = I_{yy} = I_{zz} / 2$. We chose non-magnetic materials for the mass to minimise the coupling of magnetic noise to the seismometer readout.
%In addition to the parts mentioned above, we used titanium screws and peek mirror holders for optical readout.

The fused silica fibre has a diameter of $80-200$\,um and a maximum stress of 2\,GPa. The fibre was pulled at the University of Glasgow with a similar procedure as the one utilised for making the LIGO fibres~\cite{Cumming2012, Cumming2020, Cumming2022}. We started with a fused silica stock (3\,mm in diameter), polished and cut it to the proper length before pulling. In the first design of the seismometer, we attached the 3-mm stock straight to the metal mass with a 1-mm thick layer of epoxy 2216 Scotchweld. We found that the layer caused a significant drift in RX and RY. This drift grew worse over time. After 6 months of operation, the mass drifted by 2\,mrad over 3 days even when the RX and RY resonances were relatively stiff (80\,mHz).

In order to reduce the drift, we welded the fibre ends to fused silica cones using a custom fused silica welding machine at the University of Glasgow. In the second design of the seismometer, we fixed the aluminium mass on the fused silica anchor with a 200\,um thick layer of indium. We chose the material to achieve a high mechanical Q-factor of the RX and RY modes because the loss angle of indium is lower compared to that of epoxy. However, we suffered from creep events~\cite{Levin_2012, Vajente_2017, Popovi_2022} in the indium layer, which was stressed at $\sim 1$\,MPa by the mass. The creep events triggered a non-stationary torque noise on the mass. As a result, the seismometer noise below 50\,mHz was modulated by temperature and fluctuated by an order of magnitude on a time scale of one day.

In the final, and successful, design of the seismometer, which is reported in this paper, we replaced indium with a 30-um thick layer of epoxy Araldite 2014-2. The creep noise has disappeared and we observed the stable and low-noise operation of the sensor. The layer of epoxy, eddy currents induced in the coil frames, and the fibre profile determined the Q-factors of the suspension modes.
%\begin{equation}\label{rx_z_Q}
%\omega_{rx}^2 = \frac{k_f + Mgh}{I_{rx}} \hspace{7mm} \omega_z^2 = \frac{k_z}{M} %\hspace{7mm}
%\frac{Q_z}{Q_{rx}} = \frac{k_f^2}{k_z k_{rx} l_0^2},
%\end{equation}
%\begin{equation}\label{rx_z_Q}
%\omega_{rx, ry}^2 = \frac{k_{rx, ry} + Mgh}{I_{rx}} \hspace{1cm} \omega_z^2 = \frac{k_z}{M}
%\end{equation}
%where $k_{rx}, k_{ry}, k_z$ are the fibre rigidity for RX, RY, and Z, $h$ is the offset between the centre of mass position and the suspension point.
We measured $Q_z = 10^4$ for the vertical (Z) mode of 7.6\,Hz, $Q_{rz} \approx 1000$ for the torsion (RZ) mode of 0.6\,mHz, set a lower limit on the Q-factors for the pendulum (X,Y) modes at 0.64\,Hz of $Q_{x,y} > 10^5$, and measured $Q_{rx,ry}=140$ for tilt modes (RX, RY) of 12--13\,mHz.

%From Eq.~(\ref{rx_z_Q}), we find that $k_f$ is 16 times higher compared to the fibre stiffness.
%However, the fibre bending mostly occurs at its ends where the fibre diameter increases from 200\,um up to 3\,mm. We measured the stiffness of the bending area, $k_f$ by comparing the Q-factors of the Z and RX modes ($Q_z$ and $Q_{rx}$). The eigen modes $\omega_z$ and $\omega_{rx}$ and their Q-factors are given by the equations

We found that the fibre rigidity is not the same for RX and RY because the fibre neck profile is elliptical in the XY plane. The asymmetry prevents the seismometer from reaching very low RX and RY modes because both of them are tuned with the same gravitational antispring~\cite{Ubhi2022}. We tuned the centre-of-mass position relative to the suspension point and achieved RX mode of 13.6\,mHz and RY mode of 12.6\,mHz. The figures imply that the difference in RX and RY fibre rigidity is $\approx 7 \times 10^{-5}$\,Nm and the minimum RX frequency we can achieve while keeping RY stable is 5\,mHz.

% Fig.~\ref{fig:tilt_Q} shows the beat between RX and RY modes during the ringdown of the mass.
%For RX and RY modes set to $\approx 50$\,mHz, we measured the beat period of 22\,mins.
%The period implies that when we tune the centre of mass position, $h$, the softer mode (RY) becomes unstable when the stiffer mode (RX) frequency is $5$\,mHz. Therefore, we tuned $h$ to achieve RX of 13.6\,mHz and RY of 12.6\,mHz in the final version of the seismometer. \dm{Remove 50 mHz.}

% X,Y, RZ resonances and their Q-factors. RZ stiffness.

%The eigen frequencies of the pendulum (X and Y) and torsion (RZ) modes depend on the fibre length $L$ according to the equations
%\begin{equation}
%    \omega_x^2 = \frac{g}{L} \hspace{1cm} \omega_{rz}^2 = \frac{k_{rz}}{ I_{rz}} = \frac{S I_a}{L I_{rz}},
%\end{equation}
%where $S=30$\,GPa is the shear modulus of fused silica and $I_a$ is the second moment of inertia of the fibre.
%The epoxy layer and eddy currents induced in the coil frames limit the Q-factors of the X,Y modes to $10^5$. The eddy currents also limit the Q-factor of the RZ mode to $\approx 10^3$.

%To increase the sensitivity, all the frequencies were set as low as possible (see \ref{tab:params}).
%For pendulum modes (X and Y translation), frequency is determined by the fibre length. Vertical resonance depends on the suspended mass and fibre stiffness.
%In our suspension, it is the stiffest mode. There are two factors which may require a softer $f_{Z}$: first is coupling of vertical seismic motion in reference mass tilt, and the second is an improvement of the sensitivity in Z at low frequencies. We don't see any evidence of the coupling and for this proof of principle experiment having a great sensitivity in Z was not necessary. Instead, we use the vertical degree of freedom as an estimation of the sensor noise as the mass position is effectively fixed at 0.1 Hz and below. If needed, $f_Z$ could be softened by adding a spring suspension of the top fused silica anchor. 

%Tilt resonance depends on the fibre neck profile and the height of the RM centre of mass against the fibre bending point. Increasing the centre of mass height we add a negative stiffness to both tilt modes, $f_{RX}$ and $f_{RY}$. For preliminary fibres we observed a significant difference between elastic restoring coefficients for RX and RY \cite{Ubhi2022}. The fibre used in our final suspension, described in this paper, looks pretty symmetric and the measured difference is below 1 mHz for 12 mHz tilt modes.

%\begin{table*}
%    \caption{Eighenmodes of the suspended reference mass.}
%    \begin{tabular}{clccc}
%    \hline
%    Name & Description & Equation & Value & Measured Q-factor\\
%    \hline
%    $f_{\rm X,Y}$ & Translational resonances & $\frac{1}{2\pi}\sqrt{\frac{g}{L}}$ & 0.65\,Hz & $>10^4$\\
%    \hline
%    $f_{\rm Z}$ & Vertical resonance & & 7.6\,Hz\\
%    \hline
%    $f_{\rm RX, RY}$ & Tilt resonances & $\frac{1}{2\pi}\sqrt{\frac{m g d + k_{el}}{I_x}}$  & 12\,mHz & 150\\
%    \hline
%    $f_{\rm RZ}$ & Torsion resonance & & 0.6\,mHz &\\
%    \hline
%    \hline
%    \end{tabular}
%    \label{tab:params}
%\end{table*}

% One factor which may play a crucial role in the tilt mode is the fibre profile. The tilt mode of the reference mass causes the fibre bending near the lower end of the fibre. As pulled fused silica fibre usually have a long neck, the fibre bending occurs in areas where fibre is significantly thicker. It increases stiffness and loss of the tilt modes. To reduce this, the pulling process was adjusted to make the neck as short as possible at the bottom of the fibre.

%Another factor is the interface between the anchor and reference mass: it should be stable and low loss. Initially, we used indium bonding loaded with RM weight, which provides the pressure of 0.5~MPa. It was found that the mass has extra non-stationary noise in tilt and the bonding strength is too low, so indium bonding was replaced with a thin layer of epoxy. 

%% Figure environment removed


\textit{Sensing and control system.-} We utilised two types of sensors to measure the distance between the active platform and the suspended mass. The first is a shadow sensor~\cite{Cooper2022_BOSEM, Ubhi_2022c, Strain2012} with a linear range of 0.7\,mm and resolution of $\sim$1\,nm. The sensor consists of an LED and a photodiode mounted on the platform and a flag attached to the suspended mass as shown in Fig.~\ref{fig:setup}. Part of the LED light is blocked by the flag and its absolute position is determined by the photodiode signal.

Sensors of the second type measure the position between the mass and the platform interferometrically. We utilised custom-made Homodyne Quadrature Interferometers (HoQI) \cite{Cooper_2018, Cooper2022_L4C}. The sensors operate as Michelson interferometers with polarisation readout. By measuring the light intensity in the vertical and horizontal polarisation separately, we achieved multi-wavelength range~\cite{Cooper_2018} and better resolution compared to the shadow sensors. However, HoQIs cannot measure the absolute position of the mass relative to the platform, are more nonlinear compared to the shadow sensors, and require precise alignment ($\approx 0.2$\,mrad) of the suspended mass.
%The signals from the interferometric sensors were the primary output of the seismometer.

In order to keep the interferometric sensors operational during temperature-induced drifts of the mass, we set up a control scheme (catching servo in Fig.~\ref{fig:feedback}) to stabilise the angular drifts below 1\,mHz. We found that if the mass is not controlled, it drifts in RX and RY by $\approx 1.5$\,mrad per 1\,K of temperature variation in the laboratory. The motion
%, $\theta_{rx,ry}$, and temperature fluctuations, $\Delta T$, are related by the equation
%\begin{equation}
%    \theta_{rx,ry} \approx 1.5\,{\rm mrad} \times \left( \frac{12\,{\rm mHz}}{\omega_{rx,ry}/{2 \pi}} \right)^2 \frac{\Delta T}{1\,{\rm K}}.
%\end{equation}
is caused by the differential thermal expansion of the seismometer arms. The measured figures may either be explained by the imbalance of the arm lengths (by $\approx 1$\%) or by the thermal gradients along the seismometer (on the level of $\approx 40$\,mK) produced by the ambient temperature variations.

% Move to supplementary
%\note{Figure. Drift of the RX and RY tilt and corresponding temperature vs time}
%% Figure environment removed

% Figure environment removed

% Figure environment removed

Initially, we utilised only coil-magnet actuators to control the test mass position relative to the platform. For each actuation site, we mounted two 1~mm$^3$ SmCo magnets with a separation of 2\,mm. We oriented the poles of the magnets in opposite directions to minimise the coupling of ambient magnetic forces on the suspended mass. 
%In this case, forces caused by uniform fields from external sources compensate each other, while the non-uniform magnetic field from the nearby coil produces a reasonable force up to $5\cdot 10^{-6}$N. For coils, we used BOSEM bases without LED and PD in order to minimize the interaction of the magnets with potentially magnetic photo-detectors. Each arm has vertical and horizontal actuation, perpendicular to the arm, which gives us control on all six degrees of freedom of the reference mass.
The coil-magnet actuation kept the shadow sensors and interferometric sensors in the linear regime. However, it also modified the dynamics of the suspended mass due to the parasitic rigidity of magnetic actuation, $k_m = dF_m / dl$, where $F_m$ is the magnetic force and $l$ is the separation between the mass and the coil. In order to achieve $k_m = 0$, the coil must be located exactly at a distance of 10\,mm from the closest magnet, and the central axes of the coil and the magnets must coincide. However, we found the condition difficult to satisfy in practice because the magnets are slightly tilted relative to each other and relative to the coil. Furthermore, the central part of the coil and its axis are poorly defined.

We have reduced the magnetic stiffness, $k_m$, by keeping the time-averaged control force $\langle F_m \rangle = 0$. We reduced the vertical coil signals to zero at frequencies below 0.1\,mHz by installing three heaters along each aluminium arm of the seismometer as shown in Fig.~\ref{fig:setup} (right). The heaters created a temperature distribution along the mass and changed the RX and RY angles by 9\,mrad per 1 Watt of heating power
%\begin{equation}
%    \Delta \theta_{rx,ry} \approx 9\,{\rm mrad} \times \left( \frac{12\,{\rm mHz}}{\omega_{rx,ry}/{2 \pi}} \right)^2 \frac{P}{1\,{\rm W}},
%\end{equation}
provided by the 600\,Ohm resistors.  The heating scheme is non-contact and low-noise because any fluctuations of the current through the resistors are naturally filtered above 0.1\,mHz by the heat exchange process. Furthermore, the scheme does not change the dynamics of the suspended mass.

% Move to supplementary
%The main option we used was a manual balancing of the reference mass by adding and adjusting the position of small masses (1-0.01 g) put near the end masses. Manual balancing at frequencies below 100mHz is difficult due to long periods and the huge effect of the air currents. To check the balance, the vacuum chamber was closed with damping feedback acting on the mass. Manual balancing is necessary, but for low-frequency suspension, in-vacuum balancing is needed as there are effects which change the reference mass balance during a pump down. For the test mass consisting of different materials, buoyancy in the atmosphere may create torque unveiled during pumping. We also faced the temperature distribution change after pump-down: the turbo-pump we use heated up one side of the vacuum chamber and one of the arms resulting in the balance change.

%We considered several options for in-vacuum balancing. The mass adjustment used in compact BRS sensors \cite{RossPhD} rely on having stiff DoFs to make a mechanical connection between a motor and the small movable mass. After adjustment, this mass is not fixed, which may introduce a balance change if the reference mass hit the surroundings. We considered also using some kind of piezo-motors mounted at the reference mass to adjust the centre of mass position. Stick-slip or walking piezo motor is a non-magnetic, vacuum-compatible and reliable option. It needs a removable electrical connection which allows the motor operation when rebalancing is needed. When disconnected, the motor keeps the position of the adjusting mass. The main cons are the need for a removable electrical connection and testing of the drifts needed. We chose the third option over the previous ones. 

The inertial mass is stabilised relative to the active platform at frequencies below 1\,mHz. At higher frequencies, the seismometer signals can be utilised to stabilise the platform motion and to suppress ground vibrations as shown in Fig.~\ref{fig:feedback} (top path).
%The platform is placed on a hexapod with piezo-actuated legs which allow us to control its position at 0.1 -- 10\,Hz.
Conditioning of the seismometer signals before the actuation on the platform and the feedback control scheme in six axes was discussed in Ref.~\cite{Ubhi_2022b}. When the platform servo was off, we utilised the coil-magnet actuation to damp the high-Q suspension modes in translational degrees of freedom down to the Q-factors of $\sim 100$. The damping servos for the angular resonances are always on.

%\section{Interferometric sensing}

%% Figure environment removed
 
\textit{Sensitivity analysis.-}
%After we removed indium from the fused silica - metal interface, we did not observe any more unknown torques on the mass.
Fig.~\ref{fig:asd} shows the spectra of the measured seismometer signals. The RX and RY sensors observe the tilt motion of the ground above 4\,mHz (as witnessed by the Y and X signals) and are limited by the thermal and actuation noise near the resonances at 12-13\,mHz. Below 4\,mHz, ambient temperature variations cause the largest noise in RX and RY. We have suppressed the noise by a factor of 5 around 1\,mHz by installing thermal shields. The shields consist of 3-mm thick aluminium panels that cover the mass from the thermal radiation of the vacuum chamber and the platform. The system passively filters ambient temperature variations at frequencies above 0.2\,mHz.

We estimated the readout noise in RX and RY sensors by measuring the vertical motion of the mass relative to the platform.
Since the Z mode is stiff, the vertical ground motion does not couple to the vertical position sensors below 1\,Hz. At 90\,mHz--1\,Hz the Z spectrum is limited by the coupling of the horizontal motion to the HoQI sensors (on the level of 1.2\%). At lower frequencies, we observe the readout noise of our vertical interferometric sensors. The noise is a factor of 5 larger than the HoQI noise measured with stationary mirrors (gray curve in Fig.~\ref{fig:asd}(b)). The additional noise comes from nonlinearities in the HoQI sensors due to the large ($\sim 0.1$\,um) motion of the platform above 5\,Hz.

The nonlinearities are caused by the ellipticity of the Lissajous figures formed by the two HoQI quadrature signals~\cite{Cooper_2018}. We define the ellipticity of the figures, $\epsilon$, as the ratio of their major and minor axes. We circularised our Lissajous figures by translating, rotating, and stretching the ellipses. However, the figures changed over time due to polarisation drifts in the optical fibres which connect the laser and the in-vacuum compact interferometers. Our long-term ellipticity was on the level of $\epsilon - 1 \approx 0.05$.
Despite noise from nonlinearities, the sensitivity of vertical interferometric sensors was still better than the one of the shadow sensors by an order of magnitude below 100\,mHz.

The noise from nonlinearities in our horizontal interferometric signals was worse than in the vertical ones by a factor of 5 due to stronger polarisation drifts in one of our in-vacuum optical fibres. The estimated readout noise in X and Y (see Fig.~\ref{fig:asd}(c)) is still low enough to measure ground vibrations, including the microseismic peak at 0.2 Hz, but is an order of magnitude worse than the Trillium 240 noise. Below 30\,mHz, the signal is limited by the tilt-to-horizontal coupling. We computed the coupling by multiplying the tilt spectrum RX by $g/\omega^2$, where $\omega$ is the angular frequency. We overestimated the tilt coupling around the RX resonance of the suspended mass because we are limited by the mass own motion rather than by the platform tilt at these frequencies as shown in Fig.~\ref{fig:asd}(a).
% The noise floor of the HoQI and shadow sensors was similar at frequencies below 100\,mHz. The noise of the shadow sensors was better than the HoQI one above the frequency. Therefore, we utilised shadow sensors for the readout of X,Y, and RZ degrees of freedom.
%The X and Y sensors observe horizontal ground vibrations above 50\,mHz, and ground tilt in RY and RX at lower frequencies.
The RZ sensitivity (see Fig.~\ref{fig:asd}(d)) is limited by actuation noise below 10\,mHz and by readout noise at higher frequencies.

% the shot noise floor of $5 \times 10^{-11}\,\frac{\rm m}{\sqrt{\rm Hz}}$ which limits the sensor sensitivity above 10\,Hz. At lower frequencies, the diode current noise is the primary source of position uncertainty. 

%Noise measured for static mirror is the lowest possible noise for these sensors. Large motion of the mirror would add extra noise due to non-linear effects. In our case, signals from sine and cosine photodetectors has a phase shift not equal to 90 degrees, causing an ellipticity in the sin-cos plane. We compensated this measured ellipticity to approx 1\%. The ellipticity slightly evolves with time. More crucial though, sensor signal depends on the position of the mass. Rotation of the mass results in misalignment of sensing and reference beams at the photodetector (PD) and reduced fringe visibility.  It could be compensated in two ways. First one is a smart processing and correction of the fit when mass went through several fringes.We chose a second way - we made a best fit for the optimal position of the test mass and then used actuators to hold this position with a long integration time. Making so, we keep the mass in the optimal position for all our sensors and actuators and have relatively simple and reliable processing of PD data to the mass coordinates.

In conclusion, we have developed a six-axis seismometer that outperforms the current LIGO inertial sensors in measuring angular degrees of freedom.
%The sensor has the potential to enhance the LIGO sensitivity to intermediate-mass black holes, improve their duty cycle, and simplify the lock acquisition procedure.
Our sensitivity in the translational degrees of freedom is worse than the one of Trillium 240 due to the nonlinearities in the interferometric sensors and stiff Z mode. The nonlinearities should become smaller on the LIGO suspended platforms which move over two orders of magnitude less compared to our rigid platform above 5\,Hz. The seismometer can be further improved by (i) replacing the metal mass with a fused silica one for better thermal stability, (ii) installing a soft vertical spring to increase the seismometer response in Z, and (iii) reducing the coupling of fibre polarisation drifts to the interferometric readout by utilising a deep frequency modulation scheme~\cite{Smetana2022compact, Isleif_2019, Gerberding_2021, Gerberding_2017, Armano_2016, Weichert_2012, Rue_1972, Heinzel_2010}.

We acknowledge members of the LIGO Suspension Working Group for useful discussions, the support of the Institute for Gravitational Wave Astronomy at the University of Birmingham, STFC Equipment Call 2018 (Grant No. ST/S002154/1), STFC Consolidated Grant "Astrophysics at the University of Birmingham" (No. ST/S000305/1), and UKRI Quantum Technology for Fundamental Physics scheme (Grant No. ST/T006609/1 and ST/W006375/1). D.M. is supported by the 2021 Philip Leverhulme Prize. 


% \section{Noise budget}

%\section{Results}
%\section*{Bibliography}
%\bibliographystyle{unsrt}
\bibliography{main.bib}


\end{document}











