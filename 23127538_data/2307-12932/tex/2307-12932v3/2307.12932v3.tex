\documentclass{amsart}
%[draft]{article}
%\def\MakeUppercaseUnsupportedInPdfStrings{\scshape}
\usepackage{a4wide,amsmath,amssymb}
\usepackage[toc,page]{appendix}
\usepackage{hyperref}
%\usepackage[notref,notcite]{showkeys}
\usepackage{url}
%\usepackage{geometry}
%\usepackage[mathcal]{euscript}
\usepackage{courier}
\usepackage{easyReview}
%
%\usepackage{amsmath,amssymb,amsthm}
\usepackage{mathtools}
%%\mathtoolsset{showonlyrefs}
%%\usepackage[active]{srcltx}
%\usepackage[notref,notcite]{showkeys}
%%\usepackage[dvips]{graphicx}
%\geometry{a4paper}
%\usepackage{dutchcal}
%\usepackage[left]{showlabels}
%

%\DeclareMathAlphabet{\mathpzc}{OT1}{pzc}{m}{it}
%\DeclareMathOperator{\dist}{dist}
\usepackage{xcolor}
%\usepackage{cases}

\newcommand{\bb}{\color{blue}}
\newcommand{\by}{\color{yellow}}
\newcommand{\br}{\color{red}}
\newcommand{\bvio}{\color{violet}}
\newcommand{\bg}{\color{green}}
\newcommand{\bgr}{\color{gray}}
\newcommand{\bo}{\color{orange}}

\DeclarePairedDelimiter\ceil{\lceil}{\rceil}
\DeclarePairedDelimiter\floor{\lfloor}{\rfloor}

\newcommand{\bigzero}{\mbox{\normalfont\Large\bfseries 0}}
\newcommand{\rvline}{\hspace*{-\arraycolsep}\vline\hspace*{-\arraycolsep}}
\newtheorem{thm}{Theorem}[section]
\newtheorem{lemma}[thm]{Lemma}
\newtheorem{prop}[thm]{Proposition}
\newtheorem{cor}[thm]{Corollary}
\theoremstyle{remark}
\newtheorem{rem}[thm]{Remark}
\newtheorem{opp}[thm]{Open problem}
\newtheorem{ex}[thm]{Example}
\newtheorem{cex}[thm]{Counterexample}
\newtheorem{conj}[thm]{Conjecture}
\theoremstyle{definition}
\newtheorem{defn}[thm]{Definition}
\newtheoremstyle{Claim}{}{}{\itshape}{}{\itshape\bfseries}{:}{ }{#1}
\theoremstyle{Claim}
\newtheorem{Claim}[thm]{Claim}

%\renewcommand{\div}{{\rm div}}

\newcommand{\T}{{\mathbb{T}}}
\newcommand{\Z}{{\mathbb{Z}}}
\newcommand{\E}{\mathbb{E}}
\newcommand{\R}{\mathbb{R}}
\newcommand{\RsetN}{\mathbb{R}^d}
\newcommand{\N}{\mathbb{N}}

\newcommand{\He}{\mathbb{H}}
\newcommand{\X}{\mathcal{X}}

\newcommand{\aac}{\`a}
\newcommand{\Aac}{\`A}
\newcommand{\eac}{\`e}
\newcommand{\iac}{\`i}
\newcommand{\oac}{\`o}
\newcommand{\uac}{\`u}
\newcommand{\virg}{``}

\newcommand{\Fc}{\mathbcal{F}}
\newcommand{\Gc}{\mathbcal{G}}
\newcommand{\Lc}{\mathbcal{L}}
\newcommand{\su}{\mathpzc{u}}
\newcommand{\sm}{\mathpzc{m}}

\newcommand{\eps}{\varepsilon}
\newcommand{\norm}[1]{\left\lVert#1\right\rVert}

\newcommand{\al}{\alpha}
\newcommand{\Tr}{\text{Tr}}

\newcommand{\g}{\gamma}
%\DeclareMathOperator{\dive}{div}
%
%\def\ds{\displaystyle}


\newcommand{\noticina}[1]{\marginpar{\footnotesize\emph{#1}}}

\newcommand{\corr}[2]{\textcolor{red}{#1}} % visualizza in rosso il primo argomento (nuova versione) e nasconde il secondo (vecchia versione)
% \newcommand{\corr}[2]{ {#1} } % uncomment this to accept all the corrections :)





\newcommand{\Om}{\Omega}
\newcommand{\esp}{\varepsilon}
\newcommand{\G}{\mathbb{G}}
\DeclareMathOperator{\VProp}{VProp}
\DeclareMathOperator{\HProp}{HProp}
\DeclareMathOperator{\Prop}{Prop}
\DeclareMathOperator{\supp}{supp}
\DeclareMathOperator{\m}{meas}
\DeclareMathOperator{\jac}{Jac}
\DeclareMathOperator{\Ind}{Ind}
\DeclareMathOperator{\Imm}{Im}
\DeclareMathOperator{\Rea}{Re}
\DeclareMathOperator{\perog}{per\ ogni}
\DeclareMathOperator{\Arg}{Arg}
\DeclareMathOperator{\esssup}{ess\,sup}
\DeclareMathOperator{\osc}{osc}
\DeclareMathOperator{\diam}{diam}
\DeclareMathOperator{\hyp}{hyp}
\DeclareMathOperator{\epi}{epi}
\DeclareMathOperator{\sign}{sign}
\DeclareMathOperator{\LSC}{LSC}
\DeclareMathOperator{\USC}{USC}
\DeclareMathOperator{\bu}{\textbf{u}}
\newcommand{\Sym}{\mathcal{S}}

\theoremstyle{plain}
%\DeclareUrlCommand\uri{}

% SIDE-REMARKS
\def\sideremark#1{\ifvmode\leavevmode\fi\vadjust{% The remark
\vbox to0pt{\hbox to 0pt{\hskip\hsize\hskip1em% will appear only
\vbox{\hsize3cm\tiny\raggedright\pretolerance10000% on the side
\noindent #1\hfill}\hss}\vbox to8pt{\vfil}\vss}}}% in 3cm

% Numerical scheme
\newcommand{\dx}{\Delta x}
\newcommand{\dy}{\Delta y}
\newcommand{\dt}{\Delta t}
\newcommand{\D}{\Delta}
\newcommand{\normdiscrete}[1]{\| #1\|_{W^2(\mathcal{M})}}
\makeatletter
\@namedef{subjclassname@2020}{\textup{2020} Mathematics Subject Classification}
\makeatother

\begin{document}

% \title[short text for running head]{full title}
\title[Quantitative and qualitative properties for Hamilton-Jacobi PDEs]{Quantitative and qualitative properties for Hamilton-Jacobi PDEs via the nonlinear adjoint method}



\author{Fabio Camilli}
\address{SBAI, Sapienza Universit\`{a} di Roma,
	via A.Scarpa 14, 00161 Roma (Italy)}
\curraddr{}
\email{fabio.camilli@uniroma1.it}


\author{Alessandro Goffi}
\address{Dipartimento di Matematica ``Tullio Levi-Civita'', Universit\`a degli Studi di Padova, 
via Trieste 63, 35121 Padova (Italy)}
\curraddr{}
\email{alessandro.goffi@unipd.it}


\author{Cristian Mendico}
\address{Dipartimento di matematica, Universit\`a degli studi di Roma Tor Vergata, Via della Ricerca Scientifica 1, 00133 Roma (Italy)}
\curraddr{}
\email{mendico@mat.uniroma2.it}


 \thanks{%Partially supported by ....
The authors wish to thank Martino Bardi, Jeff Calder, Alekos Cecchin, Samuel Daudin, Lawrence Craig Evans, Philippe Souplet and Eitan Tadmor for several pointers on  references and   fruitful comments on the material of the paper. The authors are member of the Gruppo Nazionale per l'Analisi Matematica, la Probabilit\`a e le loro Applicazioni (GNAMPA) of the Istituto Nazionale di Alta Matematica (INdAM). A. Goffi was partially supported by the INdAM-GNAMPA projects 2022 and 2023. C. Mendico was partially supported by the INdAM-GNAMPA projects 2022 and 2023, by the MIUR Excellence Department Project awarded to the Department of Mathematics, University of Rome Tor Vergata, CUP E83C23000330006, and the project E83C22001720005 “ConDiTransPDE”, Control, diffusion and transport problems in PDEs and applications. The authors were partially supported by the King Abdullah University of Science and Technology (KAUST) project CRG2021-4674 ``Mean-Field Games: models, theory and computational aspects"}

\subjclass[2020]{35F21, 35B65, 49L25}


\keywords{Hamilton-Jacobi equation; vanishing viscosity; nonlinear adjoint method; semiconcavity estimates}


\date{\today}

\begin{abstract}
 We provide some new integral estimates for solutions  to 
 Hamilton-Jacobi equations and we discuss several consequences, ranging from $L^p$-rates of convergence for the vanishing viscosity approximation and homogenization to regularizing effects for the Cauchy problem in the whole Euclidean space and Liouville-type theorems.  Our approach is based on duality techniques \`a la Evans and a careful study of   advection-diffusion equations.  The optimality of the results is discussed by several examples.
\end{abstract}

\maketitle

%    Text of article.

%\tableofcontents
\section{Introduction}
%Aim of this paper is to establish high-order integral estimates for solutions of the  %Hamilton-Jacobi equation
The powerful theory of viscosity solutions, by means of the maximum principle, allows to obtain general existence, uniqueness and regularity results for the first-order Hamilton-Jacobi equation 
\begin{equation}\label{hjintroinv}
	\begin{cases}
		\partial_t u+H(x ,Du)=f(x,t)&\text{ in } Q:=\R^n\times(0,T),\\
		u(x,0)=u_0(x)&\text{ in }\R^n.
	\end{cases}
\end{equation}
(see \cite{CL83tams,CEL,BCD,Barles}).
%and derive from them some lower-order quantitative and qualitative properties of %solutions. 
Nonetheless, a recent approach known as the {\em nonlinear adjoint method} has been developed by L.C. Evans in \cite{Evansadjoint}  (see also \cite{K65,LinTadmor} for related results) to capture finer properties of the solution $u$ for a nonconvex  Hamiltonian left open by the viscosity theory. The core idea of this strategy is the study of  the linear (backward) problem
\begin{equation}\label{adjointintro}
	\begin{cases}
		-\partial_{t} \rho - \eps \Delta \rho - \textrm{div} (D_pH(x, Du_\eps) \rho) = 0 &\text{ in }Q,
		\\
		\rho(x, \tau) = \rho_{\tau}(x) &\text{ in }\R^n,
	\end{cases}
\end{equation}
where 
\[
\rho_\tau\geq0\text{ and }\int_{\R^n}\rho_\tau(x)\,dx=1,
\]
which is the adjoint of the linearized of the  viscous regularization of \eqref{hjintroinv}, i.e.
\begin{equation}\label{hjintro}
	\begin{cases}
		\partial_t u_\eps-\eps\Delta u_\eps+ H(x,Du_\eps)=f(x,t)&\text{ in }Q,\\
		u(x,0)=u_0(x)&\text{ in }\R^n.
	\end{cases}
\end{equation}
It makes little use of the maximum principle (only to show that $\rho\geq0$) and provides a better understanding of the convergence $u_\eps$ to $u$, along with the gradient shock structures of the solution $u$. \\
Here, for various choices of the regularity properties of the terminal condition $\rho_\tau$, we derive estimates for the solution $\rho$ to \eqref{adjointintro} to retrieve useful properties of $u_\eps$, sometimes stable with respect to the viscosity parameter $\eps$. \\ 
We first exploit  this approach to investigate some second-order regularity properties of solutions to Hamilton-Jacobi equations in $L^p$-spaces, $1\leq p\leq\infty$. Examples of such properties are the one-side estimates
\begin{equation}\label{semicintro}
D^2u\xi\cdot \xi\leq D^2u_0(x)\xi\cdot \xi +\int_0^Tc_f(t)\,dt,\ \xi\in\R^n,\ |\xi|=1,\ D^2f(x,t)\xi\cdot \xi\leq c_f(t)\in L^1(0,T)
\end{equation}
and  
\[
\Delta u\leq C(x,t)\in L^1_t(L^p_x),\, 1\leq p\leq\infty ,
\]
which are motivated by uniqueness and stability properties initiated in \cite{L82book,LinTadmor}. As a byproduct, we address the following issues, referring to the introductory part within each section for a detailed comparison and the improvements with respect to the literature:
\begin{itemize}
	\item Convergence rates for the vanishing viscosity approximation \eqref{hjintro} in $L^p$ norms.
	These are done in Sections \ref{sec;rate1} and \ref{sec;rate2};
	\item Stability estimates for Hamilton-Jacobi equations, including the rate of convergence for homogenization and numerical schemes of Godunov-type. These are done in Sections \ref{sec;hom} and \ref{sec;num};
	\item Regularizing effects for the Hamilton-Jacobi equation \eqref{hjintroinv} posed on the whole space. These are the subject of Section \ref{sec;regeff};
	\item Liouville-type properties for first- and second-order equations, see Section \ref{sec;lio}.
\end{itemize}
%Moreover, we extend some known sup-norm estimates for \eqref{hjintroinv} to a more general setting, allowing also to encompass possibly nonconvex Hamiltonians.
%
In perspective, the previous results can  lead to new developments in the context of Mean Field Games, nonlocal problems, fully nonlinear equations, subelliptic equations, and Porous-Medium equations, that will be the matter of future research.
As a further advance, our duality method reveals, by means of \eqref{semicintro}, a connection among second-order properties for first-order Hamilton-Jacobi equations and   integrability conditions on the advection term for continuity equations arising from the Ambrosio-Diperna-Lions theory \cite{Ambrosio,Figalli,LBL}. Indeed, when $D^2u\leq c(t)\in L^1_t$ and the velocity field of \eqref{adjointintro} with $\eps=0$ is $b(x,t)=-D_pH(x,Du)$, as in Mean Field Games theory, we have from \eqref{semicintro}
\[
D^2u\leq c(t)\in L^1_t\implies [\mathrm{div}(b)]^-\leq c_Hc(t)\in L^1_t,\text{ namely }[\mathrm{div}(b)]^-\in L^1_t(L^\infty_x).
\]
This is a classical condition which, along with a suitable growth of the coefficients in the equation,   ensures the well-posedness and the validity of stability estimates in Lebesgue spaces for inviscid transport and continuity equations. We also outline in the course of the paper several connections with estimates for conservation laws, see e.g. Remarks \ref{conlaws}, \ref{system} and \ref{Nwave}, for which we propose a new proof by means of duality methods.\\
To conclude, we recall that since its introduction by L.C. Evans \cite{Evansadjoint}, the nonlinear adjoint method has been intensively applied  to several different contexts in the field of nonlinear PDEs, mostly of Hamilton-Jacobi type. To mention a few:  weak KAM theory \cite{EvansWeakKAM,TranBook},  infinity Laplacian equations \cite{EvansSmart}, large-time behavior of Hamilton-Jacobi equations and  Mean Field Games \cite{CGMT,CirantPorretta}, uniqueness principles for weak solutions of viscous Hamilton-Jacobi equations \cite{CGpoinc} and, more recently, maximal regularity properties for viscous Hamilton-Jacobi equations in $L^p$ spaces, in connections with a P.L. Lions' conjecture, and  classical regularity of solutions to Mean Field Games \cite{GPV,CGpoinc,CGpar,G23}. 
%%%%%%%%%%%%%%%%%%%%%%%%%%
%                        %
%%%%%%%%%%%%%%%%%%%%%%%%%%

\section{Some preliminary notions and definitions}
In this section, we    introduce various notions of unilateral second-order bounds which arise in the study of Hamilton-Jacobi equations. 
%% Semiconcavity and SSH  %%%
We start with the classical  definition of semiconcavity.
\begin{defn}\label{semic}\hfill
	\begin{itemize}
		\item[(i)] 
		A function $u\in C(Q)$  is said to be semiconcave with constant $C$ if it satisfies
		\[
		u(x+z,t)-2u(x,t)+u(x-z,t)\leq C|z|^2,\ x,z\in\R^n, t>0.
		\]
		\item[(ii)]	A function $u\in C(Q)$ is said to be semiconcave for positive times if there exists a constant $C$ such that
		\[
		u(x+z,t)-2u(x,t)+u(x-z,t)\leq C\left(1+\frac1t\right)|z|^2,\ x,z\in\R^n, t>0.
		\]
	\end{itemize}
\end{defn}
Some remarks on the previous definition are in order
\begin{rem}
	Definition \ref{semic}-(i) is equivalent to the existence of a constant $C>0$ such that $u(x,t)-\frac12 C|x|^2$ is concave on $\R^n$, see e.g. Proposition 5.2 of \cite{BardiDragoni} or \cite{CannarsaSinestrari}. Moreover, a result of P.-L. Lions \cite{Lions83} and H. Ishii \cite{IshiiFE} shows that the semiconcavity is also equivalent to the validity of $D^2u\leq CI_n$ for $t>0$ in the viscosity sense.\\
	% Then, the statement follows from Proposition 5.1 of \cite{BardiDragoni}.   \\
Definitions \ref{semic}-(i) and (ii) are also equivalent to the following properties, respectively:
\begin{itemize}
		\item[(i)] 
		A function $u\in C(Q)$  is  semiconcave if and only if
		\[
		u_{\xi\xi}:=D^2u\xi\cdot \xi\leq C\text{ in } \mathcal{D}'(Q)\,,\ \forall\xi\in\R^n,
		\]
		namely when $D^2u\leq CI_n$ in the sense of distributions.
		\item[(ii)]	A function $u\in C(Q)$ is   semiconcave for positive times if and only if there exists a constant $C$ such that
		\[
		u_{\xi\xi}\leq C\left(1+\frac1t\right)\text{ in } \mathcal{D}'(Q),\ \forall\xi\in\R^n,\ t>0.
		\]
	\end{itemize}
	We refer to \cite{BardiDragoni} for other equivalent characterizations of convexity/concavity in terms of  fully nonlinear Hessian PDEs.
\end{rem}
%%
\begin{rem}
	Definition \ref{semic}-(i) will be used if the initial datum of the problem is semiconcave,  see  for instance  Theorem \ref{semiconc1}. Instead, the semiconcavity for positive times (ii) is related to non-semiconcave initial and source term data and it is connected with   mild regularization effects of solutions to Hamilton-Jacobi equations, see   Theorem \ref{semic>0}  and  Section 3.3 of \cite{EvansBook}.
\end{rem}
Following \cite{LinTadmor}, we introduce a generalization of the notion of semiconcavity connected with the  $L^p$ properties of solutions to Hamilton-Jacobi equations.
\begin{defn}\label{semicmix}
	 A function $u\in C(Q)$ is said to be $L^q_t(L^p_x)$-semiconcave if there exists a function $k(x,t)$ in  $L^q(0,T;L^p_{\mathrm{loc}}(\R^n))$, for   $1\leq q,p<\infty$,    and  in $L^q(0,T;L^\infty_{\mathrm{loc}}(\R^n))$, for $p=\infty$, such that
	\[
	u_{\xi\xi}\leq k\text{ in } \mathcal{D}'(Q),\qquad \forall\xi\in\R^n,|\xi|=1.
	\]
\end{defn}
%The previous definition is particularly relevant for equations with $L^q_t(L^p_x)$-semiconcave data.
When $p=\infty$ and $q=1$, the above definition reduces to the semiconcave stability in Definition 2.1 of \cite{LinTadmor}. \\
Following   \cite[Remark 3.6]{L82book}, we introduce
a weaker second-order regularity notion for of solutions of Hamilton-Jacobi equations:
\begin{defn}\label{SSHp}
	A function $u\in C(Q)$ is said to be $L^q_t(L^p_x)$-semisuperharmonic ($L^q_t(L^p_x)$-SSH in short) if there exists a function $k(x,t)$ as in Definition \ref{semicmix} such that 
	\[
	\Delta u\leq k(x,t)\text{ in } \mathcal{D}'(Q).
	\]
\end{defn}
	A result of H. Ishii (see \cite{IshiiFE}) shows that,  for $p=q=\infty$, Definition \ref{SSHp} is equivalent 
	to $u$ being a viscosity supersolution of $C-\Delta u=0$ in $Q$.\\
We now introduce the notions of solution for the Hamilton-Jacobi equations. In all the paper, we assume that $H$, $f$ and $u_0$ are at least continuous. For $\eps>0$, we consider classical solutions to \eqref{hjintro} (existence and uniqueness results can be found in \cite{L82book}).
\begin{defn}
	A classical solution to \eqref{hjintro} is a function $u\in C^{2,1}_{x,t}(Q)$ solving the problem in pointwise sense.
\end{defn} 
When $H(x,Du)$ is bounded, $C^{2,1}$ regularity can be achieved by classical maximal regularity for heat equations, see \cite{CGpar}.\\
To study some properties of solutions in the limit $\eps=0$, we  will consider two notions of solutions: viscosity solutions and generalized solutions.  We start with that of viscosity solution, cf. \cite{CEL,CL83tams}.
\begin{defn}\label{visco}
	A continuous function  $u:Q\to\R$ is said to be a viscosity subsolution (respectively, supersolution) of \eqref{hjintroinv}
	if for any $(x_0,t_0)\in Q$ and for any   $\phi\in C^1(Q)$ such that $u-\phi$
	has a local maximum (respectively, minimum) point at $(x_0,t_0)$, then we have
	\begin{align*} 
		&\partial_t\phi(x_0,t_0)+  H(x_0, D\phi(x_0,t_0))\le f(x_0,t_0)\\
	\Big(\text{respectively,}\quad &\partial_t\phi(x_0,t_0)+  H(x_0, D\phi(x_0,t_0))\ge  f(x_0,t_0)  \Big)
	\end{align*}
and $u(x)\le u_0(x)$ (respectively, $u(x)\ge u_0(x)$) in $\R^n$.
A continuous function $u:Q\to \R$ is said to be a viscosity solution of \eqref{hjintroinv}
if it is a viscosity subsolution and supersolution. 
\end{defn}
The second one is that of generalized solution in the sense of S.N. Kruzhkov \cite{K67II}.
\begin{defn}\label{def;weakae}
	A Lipschitz continuous function $u:Q\to\R$ is a generalized solution of \eqref{hjintroinv}
% the Cauchy problem
%	\begin{equation}\label{HJfirst}
%		\begin{cases}
%			\partial_t u+H(Du)=0&\text{ in }Q,\\
%			u(x,0)=g(x)&\text{ in }\R^n.
%	\end{cases}\end{equation}
	provided that
	\begin{itemize}
		\item[(i)] $u(x,0)=u_0(x)$ for $x\in\R^n$, with $u_0$ Lipschitz;
		\item[(ii)] the equation $\partial_t u+H(x,Du)=f(x,t)$ holds a.e. on $Q$;
		\item[(iii)] $u$ is semiconcave for positive times on $Q$.
	\end{itemize}
\end{defn}
\begin{rem}
	Both the definitions of solutions for \eqref{hjintroinv} satisfy  existence and  uniqueness results. As for Definition \ref{visco} we refer to \cite{BCD,CEL} for general Hamiltonians $H$, while corresponding results for generalized solutions are due to S.N. Kruzhkov \cite{Kruzhkov}, but only in the case  $H\in C^2$ and convex in $p$. \\
	In P.-L. Lions \cite[Theorems 3.1 and 10.1]{L82book},  assumption (iii) in Definition \ref{def;weakae} was relaxed to $L^\infty_{x,t}$-SSH, obtaining existence and  uniqueness results for  SSH solutions to Hamilton-Jacobi equations. When $u$ is $L^1_t(L^p_x)$-SSH and solves a stationary Hamilton-Jacobi equation with $H$ convex, P.-L. Lions proved in \cite{L82book} a stability estimate in $L^\infty$. This result is however conditional to the validity of unilateral bounds in $L^p$ of $\Delta u$ and cannot be reached, as the author in \cite{L82book} explains, using the sole maximum principle. We provide a way to derive these bounds in Section \ref{sec;semiLp}.
\end{rem}
These two notions of solutions for \eqref{hjintroinv} are intimately connected as shown in the next proposition, cf. \cite{L82book} and Proposition III.3 in \cite{CL83tams}.
\begin{prop}
	Let $H$ be convex in $p$. If $u$ is a weak a.e. subsolution of \eqref{hjintroinv}  (i.e., in the sense of Definition \ref{def;weakae} but without assuming (iii)), then $u$ is a viscosity subsolution of \eqref{hjintroinv}. If $u$ is a generalized supersolution of \eqref{hjintroinv}, then $u$ is a viscosity supersolution of \eqref{hjintroinv}. Hence, if $u$ is a generalized solution to \eqref{hjintroinv},   then $u$ is a viscosity solution to the same problem.
\end{prop}

%%%%%%%%%%%%%%%%%%%%%%%%%

% Preliminary estimates%

%%%%%%%%%%%%%%%%%%%%%%%%
%%%%%%%%%%%%%%%%%%%%%%%%%%% 
% second-order estimate   %
%%%%%%%%%%%%%%%%%%%%%%%%%%%
\section{Second-order estimates for Hamilton-Jacobi equations}
We start reviewing the literature on (one-side) second-order bounds  for solution to Hamilton-Jacobi equations. The following list summarizes the main approaches to derive semiconcavity estimates and convexity preserving properties for  semilinear equations of Hamilton-Jacobi type: 
\begin{itemize}
\item[(a)] The first one consists in approximating the first-order problem adding a viscosity term $\eps\Delta u_\eps$. Using the maximum principle   for elliptic and parabolic equations to derive a semiconcavity estimate independent of $\eps>0$, one brings to the limit as $\eps\to 0$ the properties of the solution of the second-order equation to the first-order one;
\item[(b)] The second one is based on control theoretic techniques working at the level of representation formulas, an example being the Hopf's formula.
\item[(c)] Another method exploits the doubling of variable argument in the viscosity solutions theory, which avoids the differentiation of the equation. Indeed, the low regularity of viscosity solutions to first-order problems prevents from the formal differentiation of the problem as in the viscosity regularization described in (a). 
\item[(d)] Finally, spatial convexity preserving properties can be proved by the analysis of the convex envelope if the comparison principle holds.
\end{itemize}
All the above methods usually require convexity-type assumptions of the Hamiltonian. The parabolic regularization to prove semiconcavity (or SSH) properties was first used by S. N. Kruzhkov \cite{K66I,K67II}, see also P.-L. Lions \cite{L82book}. The method in (c) is due to H. Ishii and P.-L. Lions \cite{IL} for second-order fully nonlinear equations, see also \cite{Barles,Calder,CannarsaSinestrari} for a proof for first-order equations and \cite{Giga} for convexity preserving properties. These works have their roots in \cite{Korevaar}, where a concavity maximum principle was established. We do not focus on the approach in (b), referring to \cite{CannarsaSinestrari} for further details, mentioning, however, that it allows weaker differentiability properties on $H$ and, sometimes, it encompasses also some non-convex cases, cf.  \cite[Theorem 5.3.9]{CannarsaSinestrari}. The approach in (d) is based on the properties of the convex envelope of solutions to elliptic equations, was developed in \cite{ALL} and applies even to fully nonlinear equations. A survey on the previous approaches can be found in Section 5.3 of \cite{CannarsaSinestrari}, while other  recent results based on different  techniques can be found in \cite{Liu}.\\
The recent introduction of the nonlinear adjoint method by L.C. Evans \cite{Evansadjoint} provides an alternative to the use of the maximum principle in the method   (a) and allows to get  most of the known results  by means of stability properties of the (dual) Fokker-Planck equations, see also \cite{GPV,TranBook,Tran2011} and the more recent   \cite{CGsima}. \\
Exploting the adjoint method, we  obtain second-order bounds for solutions to Hamilton-Jacobi equations under suitable assumptions on $H$ that do not necessarily involve convexity or strict convexity. This will be done first assuming semiconcavity properties of  the data  and then removing this assumption, showing thus a kind of regularizing effect. Moreover,  via a refinement of the   method, we will   prove some new one-side second-order bounds in $L^p$ spaces inspired by \cite{L82book,LinTadmor} that provide new results for the rate of convergence of the vanishing viscosity method. This will also complement the results in \cite{L82book} about uniqueness and stability for convex Hamilton-Jacobi equations.

We consider  the viscous Hamilton-Jacobi equation \eqref{hjintro} and suppose that $H$ is $C^2(\R^n\times\R^n)$, $H(x,p)\geq H(x,0)=0$ and there exist constants $\gamma > 1$ and $C_{H,i},\tilde{C}_{H,i}\geq0$ such that
\begin{align}
\tag{H1}\label{H1} & D_pH(x,p)\cdot p-H(x,p)\geq C_{H,1}|p|^{\gamma}-\tilde{C}_{H,1}\ , \\
\tag{H2}\label{H2} & |D_{xx}^2H(x,p)|\leq C_{H,2}|p|^{\gamma}+\tilde{C}_{H,2} \ , \\
\tag{H3}\label{H3} & |D_{px}^2H(x,p)|\leq C_{H,3}|p|^{\gamma-1}+\tilde{C}_{H,3} \ , \\
\tag{H4}\label{H4} & D_{pp}^2H(x,p)\xi\cdot \xi\geq C_{H,4}|p|^{\gamma-2}|\xi|^2-\tilde{C}_{H,4}. 
\end{align}
Note that \eqref{H1} and \eqref{H4} are convexity-type assumptions, but weaker than convexity.



\subsection{Second-order one-side bounds under semiconcavity assumptions}
We start with the following example, which motivates the main result of the section on the conservation of semiconcavity properties for Hamilton-Jacobi equations
\begin{ex}[Example (ii) in Section 3.3 of \cite{EvansBook}] Consider  the initial-value problem
\[
\begin{cases}
\partial_t u+\frac12|Du|^2=0&\text{ in }\R^n\times(0,\infty).\\
u(x,0)=-|x|&\text{ in }\R^n.
\end{cases}
\]
 The unique viscosity solution of the problem can be find  by means of the Hopf-Lax formula
\[u(x,t)= \inf_{y\in\R^n}\left\{u_0(y)+tL\left(\frac{x-y}{t}\right)\right\},\]
where $L=H^*$ is the Fenchel conjugate of $H$, which in this case gives
\[
u(x,t)=
%\min_{y\in\R^n}\left\{\frac{|x-y|^2}{2t}-|y|\right\}=
-|x|-\frac{t}{2},\ t\geq0.
\]
The initial condition is semiconcave,  see   \cite[Example 2.2.5]{CannarsaSinestrari}, and the above explicit formula shows that the solution preserves the semiconcavity for positive times.
\end{ex}
We start by proving the preservation of semiconcavity properties following the lines of \cite{Evansadjoint,CGsima} in the case of more general Hamiltonians and under   weaker assumptions on the data. We premise the following
\begin{lemma}\label{cross}
Let $u_\eps$ be a solution to \eqref{hjintro} with $H$ satisfying \eqref{H1}. There exists a constant $K$ depending on $\|u\|_\infty$, $C_{H,1}$, $\widetilde{C}_{H,1}$ and independent of $\eps$ such that
\[
\iint_{Q}|Du|^\gamma\rho\,dxdt\leq K,
\]
where $\rho$ is the solution to the linear advection-diffusion equation 
\begin{equation}\label{FK_test}
\begin{cases}
-\partial_{t} \rho - \eps \Delta \rho - \mathrm{div} (D_pH(x, Du) \rho) = 0, & \text{ in }\R^n\times(0,\tau),
\\
\rho(x, \tau) = \rho_{\tau}(x), & \text{ in } \R^n.
\end{cases}
\end{equation}
with $\rho_{\tau} \in C^{\infty}_{c}(\R^n)$, $\rho_{\tau} \geq 0$ and $\|\rho_{\tau}\|_{L^1(\R^n)} = 1$. 
\end{lemma}
\begin{proof}
 To show the  estimate, we test the solution $\rho$ to \eqref{FK_test} against the solution of the Hamilton-Jacobi equation, and use the solution $u$ as a test function for the transport equation \eqref{FK_test}. We thus have 
\begin{multline*}
\int_{\R^n} u(x, \tau) \rho_{\tau}(x)\,dx - \int_{\R^n} u(x, 0) \rho_{0}(x)\,dx -\int_{0}^{\tau}\int_{\R^n}f(x,t)\rho(x,t)\,dxdt
\\= \int_{0}^{\tau} \int_{\R^n} (D_pH(x, Du(x, \tau)) \cdot Du(x, \tau) - H(x, Du(x, \tau))\rho(x, \tau)\;dxd\tau.
\end{multline*}
and, by \eqref{H1}, we deduce 
\begin{multline*}
\int_{0}^{\tau} \int_{\R^n} (D_pH(x, Du(x, \tau)) \cdot Du(x, \tau) - H(x, Du(x, \tau))\rho(x, \tau)\;dxd\tau \\
\geq C_{H,1}\int_{0}^{\tau} \int_{\R^n} |Du|^{\gamma}\rho\;dxdt-\widetilde{C}_{H,1}\int_0^\tau\int_{\R^n}\rho\,dxdt.
\end{multline*}
We conclude  that
\begin{multline*}
	\int_{0}^{\tau} \int_{\R^n} |Du|^{\gamma}\rho\;dxdt \leq \frac{1}{C_{H,1}}\left[\int_{\R^n} u(x, \tau) \rho_{\tau}(x)\,dx - \int_{\R^n} u(x, 0) \rho_{0}(x)\,dx\right.\\
	\left.-\int_{0}^{\tau}\int_{\R^n}f(x,t)\rho(x,t)\,dxdt-\widetilde{C}_{H,1}\int_0^\tau\int_{\R^n}\rho\,dxdt\right]=:K.
\end{multline*}
Note that $K<\infty$ since $\rho\in L^\infty([0,\tau];L^1(\R^n))$ by Lemma \ref{well}.
\end{proof}
We are now ready to prove the main result of the section.
\begin{thm}\label{semiconc1}
Assume that $H$ satisfies \eqref{H1}-\eqref{H4} 
%(no convexity assumption is required except the convexity-type hypothesis \eqref{H3})
 with $u_0:\R^n\to\R$ bounded and semiconcave, i.e. $D^2u_0\leq c_0I_n$, and $f\in C(Q)$, bounded and $L^1_t(L^\infty_x)$-semiconcave, i.e. $D^2f\leq c_f(t)I_n$ with $c_f\in L^1(0,T)$. Then, the classical solution to \eqref{hjintro}  is semiconcave, i.e. 
\begin{equation}\label{sem}
D^2u_\eps\xi\cdot \xi\leq C,
\end{equation}
where $C$ depends on the constants $C_{H,i}$, $c_0$, $\int_0^Tc_f(t)\,dt$, $\|u_\eps\|_{\infty}$ and $\|f\|_\infty$.  In particular, if $H=H(p)$ satisfies \eqref{H4}
%satisfies merely \eqref{H3}, without being convex,
 we have the explicit bound
\begin{equation}\label{sem2}
D^2u_\eps(\tau)\xi\cdot \xi\leq c_0+\int_0^\tau c_f(t)\,dt+\widetilde{C}_{H,4}\tau, \quad \forall \xi\in\R^n,|\xi|=1, \tau in (0,T).
\end{equation}
\end{thm}
\begin{rem}
A similar result continues to hold if the semiconcavity assumptions on the data are replaced by weaker SSH bounds, i.e. if $\Delta u_0\leq c_0$ and $\Delta f\leq c_f(t)$, $c_f\in L^1(0,T)$. These assumptions imply estimates on $\Delta u_\eps$ instead of the full Hessian (in the sense of measures) using the same method, providing a different proof than those obtained in \cite{L82book} exploiting  the maximum principle.
\end{rem}


\proof
Let $\eta \in C^{\infty}_c(\R^n)$, let $\eta_{\delta}(x) = \frac{1}{\delta^n}\eta(\frac{x}{\delta})$  and set $f_{\delta}(x, t)= \eta_{\delta} \star f(x, t)$. Then,  $f_{\delta}$   uniformly converges to $f$ as $\delta \downarrow 0$ and $\{f_{\delta}\}_{\delta >0}$ is a family of semiconcave functions with the same modulus $c_{f}(t)$, indeed given $\xi \in \R^n$ with $|\xi| = 1$
\begin{equation*}
D^2 f_{\delta}(x, t) \xi \cdot \xi = \int_{\R^n} \eta_{\delta}(y)D^2f(x-y)\xi \cdot \xi \;dy \leq \int_{\R^n} \eta_{\delta}(y)c_f(t)\;dy = c_f(t).
\end{equation*}
Let us consider the classical solution $u^{\delta}_{\eps}$ to the regularized Hamilton-Jacobi equation
\begin{equation}\label{HJ_reg}
\begin{cases}
\partial_t u_\eps^{\delta}-\eps\Delta u^{\delta}_\eps+H(x,Du^{\delta}_\eps)=f_{\delta}(x,t)&\text{ in }Q,\\
u^{\delta}(x,0)=u_0(x)&\text{ in }\R^n.
\end{cases}
\end{equation}
Observe that, since $f_{\delta}(\cdot, t)$ belongs to $C^{2}(\R^n)$ then we have $u^{\delta}_{\eps}(\cdot, t) \in C^4(\R^n)$. 
For simplicity of notation we drop the indices $\eps, \delta$ in $u$ and we set $u=u^{\delta}_{\eps}$. Differentiating twice the equation, we get for $v=u_{\xi\xi}$
\begin{multline}\label{xixi}
\partial_{t}v-\eps\Delta v+D^2_{pp}H(x, Du)Du_\xi\cdot Du_\xi +D_pH(x, Du)\cdot Dv
\\
+ 2 D_{px}^{2} H(x, Du)Du_{\xi} + D^{2}_{xx}H(x, Du) =(f_{\delta})_{\xi\xi}.
\end{multline}
Multiplying \eqref{xixi} by $\rho$ and integrating over $\R^n \times [0, \tau]$, we get 
\begin{multline*}
\int_{\R^n} v(x, \tau) \rho_{\tau}(x)\,dx + \int_{\tau}^{T} \int_{\R^n} Du_{\xi} \cdot D^{2}_{pp}H(x,Du) Du_{\xi} \rho\, dxdt = 
\\
- \int_{0}^{\tau} \int_{\R^n} \left(2D_{px}^{2} H(x, Du) Du_{\xi} + D^{2}_{xx}H(x, Du) \right)\rho\,dxdt\  + \int_{\R^n} v(x, 0)\rho_0(x)\,dxdt +\int_{0}^{\tau}\int_{\R^n}(f_{\delta})_{\xi\xi}\rho\,dxdt. 
\end{multline*}

On the  one hand, by \eqref{H4} we have 
\begin{equation*}
\int_{0}^{\tau} \int_{\R^n}  D^2_{pp}H(x, Du)Du_{\xi} \cdot Du_{\xi}\rho\,dxdt \geq C_{H,4} \int_{0}^{\tau} \int_{\R^n} |Du|^{\gamma-2} |Du_{\xi}|^{2}\rho\, dxdt - \widetilde{C}_{H,4} \int_{0}^{\tau} \int_{\R^n} \rho\,dxdt
\end{equation*}
Moreover, by \eqref{H2} and \eqref{H3},  we obtain 
\begin{multline*}
\int_{\R^n} v(x, \tau) \rho_{\tau}(x)\,dx +C_{H,4} \int_{0}^{\tau} \int_{\R^n} |Du|^{\gamma-2} |Du_{\xi}|^{2}\rho\, dxdt - \widetilde{C}_{H,4} \int_{0}^{\tau} \int_{\R^n} \rho\,dxdt
\\
\leq  C_{H, 2} \int_{0}^{\tau}\int_{\R^n} |Du|^{\gamma}\rho\;dxdt\ + C_{H, 3} \int_{0}^{\tau} \int_{\R^n} |Du|^{\gamma-1} |Du_{\xi}|\rho\;dxdt\ + (\widetilde{C}_{H,2} + \widetilde{C}_{H,3}) \int_{0}^{\tau} \int_{\R^n} \rho\;dxdt\
\\
+  \int_{\R^n} v(x, 0)\rho_0(x)\,dxdt +\int_{0}^{\tau}\int_{\R^n}(f_{\delta})_{\xi\xi}\rho\,dxdt. 
\end{multline*}
By the weighted Young's inequality we get for all $\sigma>0$
\begin{equation*}
C_{H,3}\int_{0}^{\tau} \int_{\R^n} |Du|^{\gamma-1} |Du_{\xi}|\rho\;dxdt \leq \frac{\sigma^2 C_{H,3}}{2} \int_{0}^{\tau} \int_{\R^n} |Du|^{\gamma-2} |Du_{\xi}|^{2}\rho\;dxdt\ + \frac{C_{H,3}}{\sigma^2} \int_{0}^{\tau} \int_{\R^n} |Du|^{\gamma}\rho\;dxdt.
\end{equation*}
We choose $\frac{\sigma^2 C_{H,3}}{2}=C_{H,4}$, and conclude that
\begin{equation}\label{stima-1}
	\begin{split}
\int_{\R^n} v(x, \tau) \rho_{\tau}(x)\,dx &\leq \int_{\R^n} v(x, 0)\rho_0(x)\,dxdt +\int_{0}^{\tau}\int_{\R^n}(f_{\delta})_{\xi\xi}\rho\,dxdt 
\\
&+ \left(\frac{C_{H,3}^2}{4C_{H,4}} + C_{H,2}  \right) \int_{0}^{\tau} \int_{\R^n} |Du|^{\gamma}\rho\;dxdt.
\end{split}
\end{equation}
By Lemma \ref{cross}, we have
\begin{equation*}
\int_{0}^{\tau} \int_{\R^n} |Du|^{\gamma}\rho\;dxdt\leq K,
\end{equation*} 
and in view of Lemma \ref{well} 
\begin{multline*}
\int_{\R^n} u(x, \tau) \rho_{\tau}(x)\,dx - \int_{\R^n} u(x, 0) \rho_{0}(x)\,dx-\int_{0}^{\tau}\int_{\R^n}f(x,t)\rho(x,t)\,dxdt\\
+\widetilde{C}_{H,1}\int_{0}^{\tau}\int_{\R^n}\rho\,dxdt\leq 2\|u\|_\infty+(\|f\|_{\infty}+\widetilde{C}_{H,1})\tau.
\end{multline*}
Therefore, passing to the supremum over $\rho_{\tau}$ in \eqref{stima-1}, we deduce for $u=u^{\delta}_{\eps}$
\begin{equation*}
D^2u^{\delta}_{\eps}\xi\cdot \xi\leq c_0+\int_0^T\int_{\R^n}\partial_{\xi\xi}f_{\delta}\rho\,dxdt +\left(\frac{C_{H,3}^2}{4C_{H,4}} + C_{H,2}  \right)K\leq c_0+\int_0^Tc_f(t)\,dt+\left(\frac{C_{H,3}^2}{4C_{H,4}} + C_{H,2}  \right)K,
\end{equation*}
where we used that the  $\{f_{\delta}\}_{\delta > 0}$ preserves the modulus of semiconcavity $c_f(t)$. Finally, passing to the limit as $\delta \downarrow 0$ we get \eqref{sem}.  Piecing together all the estimates we finally conclude 
\begin{multline*}
\int_{\R^n}v(x, \tau) \rho_{\tau}(x)\,dx\leq c_0+\int_0^Tc_f(t)\,dt+\frac{1}{C_{H,1}}\left(\frac{C_{H,3}^2}{4C_{H,4}}+C_{H,2}\right)[2\|u\|_\infty+(\|f\|_{\infty}+\widetilde{C}_{H,1})\tau]\\
+(\widetilde{C}_{H,2} + \widetilde{C}_{H,3}+\widetilde{C}_{H,4})\tau. \hfill\square
\end{multline*}

\begin{rem}
The same approach of Theorem \ref{semiconc1} also leads to a discrete semiconcavity bound, as studied in Lemma 5.2 of \cite{LinTadmor}, that can be achieved without differentiating the equation and, thus, requires less regularity assumptions on the solution. Indeed, we can prove that
\begin{equation*}
\omega_h(u_\eps(x,t)):=u_\eps(x+h\xi,t)-2u_\eps(x,t)+u_\eps(x-h\xi,t) \leq C |h|^2
\end{equation*}
observing, as in \cite{LinTadmor}, that $\omega_h$ solves
\begin{align*}
\frac{1}{h^2}\partial_t\omega_h (u_\eps (x, t))& + \frac{1}{h^2} \big(H(Du_\eps(x+h\xi, t) ) + H(Du_\eps(x-h\xi, t) - 2H(Du_\eps(x, t) )  \big) - \eps \Delta \omega_h (u_\eps(x, t))
\\
&=\frac{1}{h^2}\partial_t\omega_h (u_\eps (x, t))  + \frac{1}{h^2} \big(D_pH(Du_\eps(x,t))\cdot (Du_\eps(x, t)^+ - Du_\eps(x, t)^- )
\\
&+ Du_\eps(x, t)^+D^2_{pp}H(Du_\eps(x,t))\cdot Du_\eps(x, t)^+ \\ &+  Du_\eps(x, t)^-D^2_{pp}H(Du_\eps(x,t))\cdot Du_\eps(x, t)^-\big) -\eps \Delta \omega_h (u_\eps (x, t))  = 0,
\end{align*}
where we applied a Taylor expansion and we set
\begin{align*}
D u_\eps(x, t)^+ =\ & D u_\eps(x+h\xi, t) - Du_\eps (x,t),
\\
D u_\eps (x, t)^-=\ &  Du_\eps(x, t) - D u_\eps(x-h\xi,t).
\end{align*}
Since $Du_\eps(x, t)^+ - Du_\eps(x, t)^- = D\omega_h(u_\eps(x, t))$ we obtain 
\begin{align*}
\frac{1}{h^2}\partial_t \omega_h(u_\eps(x, t)) u_\eps (x, t) &+ \frac{1}{h^2} \big(D_pH(Du_\eps(x,t))\cdot D\omega(u_\eps(x, t))
\\
&+ Du_\eps(x, t)^+D^2_{pp}H(Du_\eps(x,t))\cdot Du_\eps(x, t)^+ \\&+  Du_\eps(x, t)^-D^2_{pp}H(Du_\eps(x,t))\cdot Du_\eps(x, t)^-\big) - \eps \Delta \omega_h (u_\eps (x, t)) = 0
\end{align*}
So, following the same reasoning applied in Theorem \ref{semiconc1} we conclude a semiconcavity bound under less restrictive assumptions on $H$ than \cite{LinTadmor}. 
\end{rem}

\begin{rem}
The dependence of the constant  on $\|u_\eps\|_\infty$ in \eqref{sem} can be removed, and the semiconcavity constant can be bounded only in terms of $\|u_0\|_\infty$. To show this, one can argue by duality as in Proposition 3.7 of \cite{CGpoinc} using that the solution of the adjoint problem belongs to $L^\infty([0,\tau];L^1(\R^n))$, the only difference with \cite{CGpoinc} being that $f\in L^\infty(Q)$ instead of $f\in L^q(Q)$.
\end{rem}
\begin{rem}\label{Hconvex}
When $H=H(p)$ is  convex, the equation satisfied by $u_{\xi\xi}$ reads as
\[
\partial_{t}v-\eps\Delta v+D^2_{pp}H(Du)Du_\xi\cdot Du_\xi + D_pH(Du)\cdot Dv=f_{\xi\xi}.
\]
By duality one gets the semiconcavity estimate
\[
u_{\xi\xi}(x,t)\leq u_{\xi\xi}(x,0)+\int_0^T\|(f_{\xi\xi}(x))^+\|_{L^\infty(\R^n)}\,dt.
\]
\end{rem}


We conclude with an application of Lemma \ref{cross} to prove the conservation of Lipschitz regularity for \eqref{hjintro}. This is already known by means of the maximum principle, cf. \cite{Barles}, using slightly different hypotheses on $H$. 
\begin{lemma}\label{grad}
Let $H\in W^{1,\infty}_{\mathrm{loc}}(\R^n)$ be satisfying \eqref{H1}. Then any solution $u_\eps$ to \eqref{hjintro} with $u_0\in W^{1,\infty}(\R^n)$ satisfies
\[
\|Du_\eps\|_{L^\infty(Q)}\leq \|Du_0\|_{L^\infty(\R^n)}+\|Df\|_{L^\infty(Q)}T.
\]
In particular, the estimate does not depend on $\eps$.
\end{lemma}
\begin{proof}
We proceed by the adjoint method. After a regularization argument we may assume that $u_\eps$ is sufficiently smooth to perform a differentiation procedure. We set $v=u_\xi$, $\xi\in\R^n$, $|\xi|=1$, to find
\[
\partial_t v-\eps \Delta v+D_pH(Du)\cdot Dv=f_{\xi}
\]
By duality, using $\rho$ solving \eqref{FK_test} we get
\[
\int_{\R^n}v(\tau)\rho_\tau(x)\,dx\leq \int_{\R^n}v(0)\rho(0)\,dx+\int_0^\tau\int_{\R^n}f_\xi\rho\,dxdt
\]
We now use that $\|\rho(t)\|_{L^1}=1$ by Lemma \ref{well} and Lemma \ref{cross} to conclude the estimate, recalling that $u_0\in W^{1,\infty}(\R^n)$.
\end{proof}


\begin{rem}
%\textcolor{red}{Scrivere bene che anche $u$ al limite di viscosita' preserva lo stesso bound di semiconcavita'}
Note that the semiconcavity estimates \eqref{sem} and \eqref{sem2} are independent of the viscosity parameter $\eps$. So, appealing to  \cite[Theorem 3.3.3]{CannarsaSinestrari} and using the global uniform bound of $Du_{\eps}$ (see, for instance, \cite{L82book} or Lemma \ref{grad}) we deduce that the vanishing viscosity limit $u$ is a semiconcave solution to \eqref{hjintroinv}
%\begin{equation*}
%\begin{cases}
%\partial_t u + H(x, Du) = f(x, t) & \textrm{in}\; \R^n \times (0,T),
%\\
%u(x, 0) = u_0(x) & \textrm{in}\; \R^n
%\end{cases}
%\end{equation*}
with the same modulus of semiconcavity $C$ as in \eqref{sem}. 
\end{rem}

\begin{rem}
The obtainment of Lipschitz estimates typically requires to impose some mild coercivity assumptions on $H$ with respect to $Du$. This is done for instance in \cite{Barles,L82book} in the theory of viscosity or generalized solutions. Here what we really need is the conservation of mass for the dual problem solved by $\rho$. In the periodic setting this property is automatically satisfied by using the test function identically equal to $1$, and a locally Lipschitz $H$ is enough to run the argument. The whole space $\R^n$ requires more care, and some additional assumptions,   as for instance \eqref{H1}, than the sole local Lipschitz continuity are needed, cf. Lemma \ref{well}.
\end{rem}

\subsection{Second-order regularizing effects for equations with Lipschitz data}
In this section we focus on Hamiltonians depending only on $p$ without source terms in the equation. We show, on the line of \cite{EvansBook} or Proposition 2.2.6 in \cite{CannarsaSinestrari}, that in this case solutions to Hamilton-Jacobi equations satisfy a mild regularization effect even though the initial datum is not semiconcave, provided that the Hamiltonian satisfies convexity-type hypotheses. The next is an explicit example of such a phenomenon and motivates   Theorem \ref{semic>0}. 
\begin{ex}[Example (i) in Section 3.3 of \cite{EvansBook}]\label{ex2} The initial-value problem
\[
\begin{cases}
\partial_t u+\frac12|Du|^2=0&\text{ in }\R^n\times(0,\infty).\\
u(x,0)=|x|&\text{ in }\R^n.
\end{cases}
\]
admits the viscosity solution given by the Hopf formula
\[
u(x,t)=\min_{y\in\R^n}\left\{\frac{|x-y|^2}{2t}+|y|\right\}.
\]
In particular, one has 
\[
u(x,t)=\begin{cases}
|x|-\frac{t}{2}&\text{ if }|x|\geq t\\
\frac{|x|^2}{2t}&\text{ if }|x|\leq t.
\end{cases}
\]
 The initial condition is not semiconcave, see  \cite[Example 3.3.9]{CannarsaSinestrari}, but the solution becomes semiconcave for positive times. 
\end{ex}
The next  result prove a semiconcavity result for positive times, weakening the requirement of uniform convexity in \cite{EvansBook,Evansadjoint,CannarsaSinestrari}. A similar result was obtained by W. Fleming \cite{FlemingJDE} using different methods under an assumption similar to \eqref{H4}. Moreover, it provides a first step towards a Lipschitz regularization effect for first-order Hamilton-Jacobi equations  which will be discussed in Section \ref{sec;regeff}. 
\begin{thm}\label{semic>0}
Assume that $H$ satisfies \eqref{H4} and $f\equiv0$ (no further hypotheses are assumed on $u_0$). Then, any  solution to \eqref{hjintro} with $Du\in L^\infty_{x,t}$ satisfies 
%for any $\delta\geq0$ the one-side estimate
\[
u_{\xi\xi}\leq\frac{C_1}{t}\|Du\|_\infty^{2-\gamma}+C_2t\text{ when }\gamma\leq 2,
\]
where $C_1,C_2$ depend on $C_{H,4},\widetilde{C}_{H,4}$, and do not depend on $\eps$. The constant $C_2=0$ if $\widetilde{C}_{H,4}=0$.
\end{thm}
\begin{proof}
%	\alert{Mandare $\delta$ a zero nella stima del caso subquadratico?}\\
We follow the route of an idea introduced by L.C. Evans in Theorem 4.2 of \cite{Evansadjoint}. We differentiate the equation twice with respect to an arbitrary unitary direction $\xi\in\R^n$ to find for $w=u_{\xi\xi}$
\begin{equation}
\partial_t w-\eps\Delta w+D^2_{pp}H(Du)Du_\xi\cdot Du_\xi+D_pH(Du)\cdot Dw=0\text{ in }Q.
\end{equation}
Assume for the moment a regularized version of \eqref{H4}, i.e.,
%We assume a regularized version of \eqref{H4}, i.e.
\[
D^2_{pp}H(p)\xi\cdot\xi\geq C_{H,4}(\delta+|p|^2)^{\frac{\gamma-2}{2}}|\xi|^2-\widetilde{C}_{H,4}.
\]
The result will follow by letting $\delta$ to 0. Taking a smooth function $\chi:[0,T]\to\R$ to be chosen later  and setting $z=\chi w$ we have
\[
\partial_t z-\eps\Delta z+\chi D^2_{pp}H(Du)D u_\xi\cdot D u_\xi+D_pH(Du)\cdot Dz=\chi'(t)u_{\xi\xi}.
\]
We consider the adjoint problem
\[
\begin{cases}
-\partial_t \rho-\eps\Delta \rho-\mathrm{div}(D_pH(Du)\rho)=0&\text{ in }\R^n\times(0,\tau)\\
\rho(x,\tau)=\rho_\tau(x)&\text{ in }\R^n
\end{cases}
\]
and we choose $\chi(t)=t^2$ on $[0,\tau]$, observing that this implies $\int_{\R^n}z(0)\rho(0)\,dx=0$. This choice is crucial to shift the time horizon away from the initial time $t=0$, and avoids to require second-order properties on the initial datum. We test the equation of $z$ by $\rho$ to get
\[
\int_{\R^n}z(\tau)\rho_\tau(x)\,dx+\iint_{Q_\tau}t^2 D^2_{pp}H(Du)Du_\xi\cdot Du_\xi\rho\,dxdt=\iint_{Q_\tau}2tu_{\xi\xi}\rho\,dxdt.
\]
On the one hand, we have
\[
\iint_{Q_\tau}t^2 D^2_{pp}H(Du)Du_\xi\cdot Du_\xi\rho\,dxdt\geq C_{H,4}\iint_{Q_\tau}t^2 |D^2u|^2(\delta+|Du|^2)^{\frac{\gamma-2}{2}}\rho\,dxdt-\widetilde{C}_{H,4}\int_0^\tau\int_{\R^n}t^2\rho\,dxdt.
\]
On the other side, by the Young inequality we conclude
\[
\iint_{Q_\tau}2 t u_{\xi\xi}\rho\,dxdt\leq C_{H,4}\iint_{Q_\tau}t^2|D^2u|^2(\delta+|Du|^2)^{\frac{\gamma-2}{2}}\rho\,dxdt+\frac{1}{C_{H,4}}\iint_{Q_\tau}\rho(\delta+|Du|^2)^{-\frac{\gamma-2}{2}}\,dxdt.
\]
This implies
\[
\tau^2 u_{\xi\xi}\leq \frac{1}{C_{H,4}}\|(\delta+|Du|^2)^{\frac{2-\gamma}{2}}\|_\infty\tau+\widetilde C_{H,4}\frac{\tau^{3}}{3},
\]
which implies the assertion in the subquadratic case by letting $\delta \downarrow 0$. 
\end{proof}
Some remarks on the optimality of the constants are in order.
\begin{rem}
We observe that the modulus of semiconcavity $\frac{C}{t}$ in Theorem \ref{semic>0} cannot be in general improved. We show this for the model case of uniformly convex $H$, i.e. $\gamma=2$ in \eqref{H4}. Indeed, let $u$ be a solution to the problem 
\begin{equation*}
\begin{cases}
\partial_t u + H(Du) = 0
\\
u(x, 0)= |x|
\end{cases}
\end{equation*}
with $H$ satisfying $D^2_{pp}H(p)\xi \cdot \xi \geq \theta |\xi|^2 $ for some $\theta > 0$ (so $C_{H,4}=\theta$ and $\widetilde{C}_{H,4}=0$). Then, we have that $L$,   the Fenchel conjugate of $H$, is semiconcave with modulus $\frac{1}{\theta}$, and, from the Hopf formula, we have
\begin{multline*}
u(x+z, t) + u(x-z,t) - 2u(x, t) \leq t\left(L\left(\frac{x+z-y}{t}\right) + L\left(\frac{x-z-y}{t}\right) -2 L\left(\frac{x-y}{t}\right) \right)
\\
\leq\ \frac{t}{\theta} \left|\frac{z}{t} \right|^2 = \frac{1}{\theta t}|z|^2.
\end{multline*}
It is easy to see that the semiconcavity estimate in Theorem \ref{semic>0} now reads
\[
u_{\xi\xi}\leq \frac{1}{\theta t}.
\]
\end{rem}

\begin{rem}
In the subquadratic case, when $\widetilde{C}_{H,4}=0$, we recover by a different method estimate 3.2 in Proposition 3.2 of \cite{BKL}.
\end{rem}


\begin{rem}\label{conlaws}
When $n=1$, equation \eqref{hjintro} reduces to 
\[
\partial_t u-\eps u_{xx}+H(u_x)=0.
\]
In the special case $H(u_x)=|u_x|^\gamma$, one has that $U=u_x$ solves the regularized conservation law
\[
\partial_t U-\eps U_{xx}+(|U|^\gamma)_x=0.
\]
The estimate in Theorem \ref{semic>0} leads to the Oleinik-type one-side Lipschitz estimate, cf. \cite{EVZ},
\[
U_x\leq \frac{C}{t}\|U\|_\infty^{2-\gamma}.
\]
In general dimension $n$ one can apply a similar duality argument as that in Theorem \ref{semic>0} to the multidimensional conservation law $\partial_t u+\mathrm{div}(F(u))=\eps\Delta u$ with flux $F:\R\to\R^n$,  written in nondivergence form
\[
\partial_t u+F'_i(u)u_{x_i}=\eps\Delta u,
\]
obtaining an estimate as that in \cite{Hoff}. Some other related results by duality for multidimensional scalar conservation laws can be found in Section 7 of \cite{Evansadjoint}.
\end{rem}

%%%%%%%%%%%%%%%%
% L^p bounds   %
%%%%%%%%%%%%%%%%
\subsection{Second-order $L^p$ one-side bounds}\label{sec;semiLp}
In this section, we extend the $L^\infty$ bounds on second-order derivatives of the two previous subsections to $L^p$ bounds on the same quantities.
\begin{thm}\label{aho}
Assume that $H\in W^{1,\infty}_{\mathrm{loc}}(\R^n)$ satisfies \eqref{H1}-\eqref{H4}
and let $u_0:\R^n\to\R$ be $L^p_x$-SSH and $f$ be $L^1_t(L^\infty_x)$-SSH. Then, any classical solution to \eqref{hjintro} satisfies the one-side bound
\begin{equation}\label{semp}
\|(\Delta u_\eps)^+(t)\|_{L^p_{\mathrm{loc}}(\R^n)}\leq C,\ t\in(0,T).
\end{equation}
\end{thm}
\begin{proof}
The proof is the same as that in Theorem \ref{semiconc1}, the only difference being that we consider the solution of the adjoint problem
\[
\begin{cases}
-\partial_t \rho-\eps\Delta \rho-\mathrm{div}(D_pH(x,Du)\rho)=0&\text{ in }\R^n\times(0,\tau),\\
\rho(x,\tau)=\rho_\tau(x)&\text{ in }\R^n.
\end{cases}
\]
with $\rho_\tau\in C_c^\infty(\R^n)$, $\rho_\tau\in L^1(\R^n)\cap L^{p'}(\R^n)$, $\rho_\tau\geq0$ and $\|\rho_\tau\|_{L^1(\R^n)\cap L^{p'}(\R^n)}=1$, $p'>1$. This implies that
\[
\int_{\R^n}\rho_\tau(x)\,dx\leq 1
\] 
Moreover, we have
\begin{equation}\label{conserv}
\int_{\R^n}\rho(x,t)\,dxdt\leq 1,
\end{equation}
using the same localization argument of Lemma \ref{well}. Hence the proof continues along the same lines of that in Theorem \ref{semiconc1}. Note that by duality one gets a global estimate of $(u_{\xi\xi})^+$ in the space $L^\infty(\R^n)+L^p(\R^n)$, which is embedded into $L^p_{\mathrm{loc}}(\R^n)$ by Lemma \ref{emb+}.
\end{proof}


\section{Quantitative properties of Hamilton-Jacobi equations}
%%%%%%%%%%%%%%%%%%%%%%%
% Vanishing viscosity %
%%%%%%%%%%%%%%%%%%%%%%%

\subsection{A survey on the rate of convergence for the vanishing viscosity approximation of Hamilton-Jacobi equations}
It is well-known that the viscosity solution to the first order Hamilton-Jacobi equation \eqref{hjintroinv} can be  obtained as the limit as $\eps\to0$ of the solutions to \eqref{hjintro}, see e.g. Chapter VI in \cite{BCD}. This limiting procedure is indeed fundamental to select   a solution of the first-order problem, as in general uniqueness for a.e. solutions  is not always expected, cf. Example in Section 3.3.3 of \cite{EvansBook}. The uniform convergence of $u_\eps$ to $u$ has been proved in Theorem 3.1 in \cite{CEL} and Theorem VI.3.1-3.2 in \cite{BCD}. We note that the global convergence requires extra regularity hypotheses on the solution, as discussed in Chapter VI of \cite{BCD}. Moreover, one can prove more refined quantitative properties such as the rate of convergence (with respect to $\eps$) of the vanishing viscosity process. The first results in this direction appeared in \cite{K65} for stationary problems with $H$ convex, and later refined in \cite{CL84}  and \cite{Souganidis} using doubling variables methods: for $W^{1,\infty}_{\mathrm{loc}}(\R^n)$ Hamiltonians, not necessarily convex, and an initial datum $u_0\in W^{1,\infty}(\R^n)$, it is proved the following rate for Lipschitz viscosity solutions
\[
\sup_{[0,T]}\sup_{\R^n}|u_\eps-u|\leq c\sqrt{\eps},
\] 
where $c$ depends on $H$, $u_0$ and $T$. Another proof of such a rate uses smoothing arguments through sup-inf convolutions (which are semiconvex-semiconcave), cf. p.76 of \cite{Calder}. The same rate for Lipschitz solutions has been proved via the adjoint method in \cite{Evansadjoint}, see also \cite{Tran2011}, \cite{GPV} and Theorem \ref{ev1}. We further emphasize that the $\mathcal{O}(\sqrt{\epsilon})$ rate is in general optimal, as the following example with $H=0$ (i.e. when there is no control) adapted from \cite{PerthameSanders} shows:
\begin{ex}\label{exrate1}
The function
\[
u_\eps(x)=\sqrt{\eps}\frac{ \cosh\left(\frac{x-1/2}{\sqrt{\eps}}\right)}{\sinh\left(\frac{1}{2\sqrt{\eps}}\right)}
\]
solves
\[
-\eps u''_\eps(x)+u_\eps(x)=0 \quad\text{in } (-1,1)
\]
with boundary conditions $u(0)=u(1)=\cosh(1/\sqrt{\eps})/\sinh(1/\sqrt{\eps})$ and
\[
|u_\eps-u|\leq C\sqrt{\eps}
\]
where $u\equiv0$ is the solution of the  problem with $\eps=0$ and null boundary datum.
\end{ex}
The rate of convergence is sensitive to the regularity assumptions on the solution and on the Hamiltonian. Indeed, if one assumes that $u\in C^{0,\alpha}(\R^n)$, $\alpha\in(0,1]$, still with locally Lipschitz Hamiltonians, the rate becomes
\[
\sup_{\R^n}|u_\eps-u|\leq c\eps^{\frac{\alpha}{2}},
\]
see \cite[Theorem 3.2]{BCD}. The rate would become slower if the Hamiltonian is  less regular, say only H\"older continuous  with exponent $\beta\in(0,1)$ would depend also on this new parameter. This latter point can be seen by a direct inspection of the proof in \cite[Theorem 3.2]{BCD}. 
Nonetheless, sometimes better rates are expected under additional assumptions, as in the next

\begin{ex} Following \cite{Calder}, one can observe that the solution to
\[
-\eps u''_\eps(x)+|u'_\eps(x)|=1\text{ for }x\in(-1,1)\subset\R
\]
satisfying $u_\eps(-1)=u_\eps(1)=0$ is
\[
u_\eps(x)=1-|x|-\eps(e^{-\frac{|x|}{\eps}}-e^{-\frac1\eps})
\]
and $|u-u_\eps|\leq C\eps$, where $u(x)=1-|x|$ is the  viscosity solution to same problem with $\eps=0$. The difference with Example \ref{exrate1} is related to the additional properties satisfied by the solution of this problem. Indeed, such solutions are semi-superharmonic with a constant independent of $\eps$ or, better, semiconcave.
\end{ex}
In general, knowing  that $\Delta u_\eps\leq C$ (note that this condition is much weaker than the semiconcavity condition) independently of $\eps>0$, one can show  the one-side rate
\[
u_\eps-u\leq c\eps.
\]
The previous bound holds  for nonconvex $W^{1,\infty}_{\mathrm{loc}}$ Hamiltonians, but it is conditional to the unilateral bound on $u_\eps$ which  usually requires convexity type assumptions. This improved rate has been first proved in Section 11 of \cite{L82book} using probabilistic methods under the assumption that $H$ is convex and the initial datum is SSH (i.e. $\Delta u_0\leq c_0$), and in \cite{BCD} using techniques from viscosity solutions' theory.\\
 A related two-side $\mathcal{O}(\eps)$ rate has been proved by S.N. Kruzhkov in Lemma 2 of \cite{K65} for semiconcave solutions in $L^1$ and $L^\infty$, and by C.-T. Lin and E.Tadmor in $L^1$-norms under the assumption that $u$ is semiconcave stable (i.e. $L^1_tL^\infty_x$ semiconcave in our notation), $H$ is uniformly convex and in the case of periodic boundary conditions. Both these works exploit   duality arguments. Moreover, P.-L. Lions proved the convergence of $u_\eps$ to $u$ in $L^p$ for any $p$ in Chapter 6 of \cite{L82book}, so it is natural to determine the rate of convergence in Lebesgue norms.\\
 We also mention that H. V. Tran \cite[Theorem 1.43]{TranBook} proved an average rate of order $\mathcal{O}(\eps)$ when the Hamiltonian is uniformly convex.\\
 Recent works have considered the problem of establishing the rate of the vanishing viscosity process in the context of Mean Field Games. The work \cite{TangZhang} proved the rate of convergence of the vanishing viscosity process both in the case of local and nonlocal coupling among the equations using duality methods, while \cite{DDJ} shows that the convergence problem in mean field control can be reduced to a problem of vanishing viscosity for finite dimensional Hamilton-Jacobi equations, as studied in the present paper.\\
 
 In this section we provide a unifying method for proving rates of convergence in any $L^p$ norm $1\leq p\leq\infty$ using duality methods and properties of transport equations, extending all the previous results under   and weaker assumptions on $H$. We further mention that the approach is flexible enough to cover various boundary conditions (periodic, Cauchy-Dirichlet, Neumann, whole space,...) as well as stationary problems. We will also give precise results on the size of the estimates taking care of the constants in the bounds.
  
  
  
  \subsection{Rate of convergence: the non-compact case}\label{sec;rate1}
  
\subsubsection{$L^\infty$ rate of convergence}
We consider, for simplicity, the viscous Cauchy problem
\begin{equation}\label{hjsemi}
\begin{cases}
\partial_t u_\eps-\eps\Delta u_\eps+H(Du_\eps)=f(x,t)&\text{ in }Q\\
u(x,0)=u_0(x)&\text{ in }\R^n.
\end{cases}
\end{equation}
and the first-order equation
\begin{equation}\label{first}
	\begin{cases}
		\partial_t u+H(Du)=f(x,t)&\text{ in }Q\\
		u(x,0)=u_0(x)&\text{ in }\R^n.
	\end{cases}
\end{equation}
From now on, we will mainly consider $H\in W^{1,\infty}_{\mathrm{loc}}(\R^n)$ and exploit the Lipschitz estimates
\[
\|Du\|_{L^\infty(Q)}\leq \|Du_0\|_{L^\infty(\R^n)}+T\|Df\|_{L^\infty(Q)}
\]
to make the gradient of $u$ globally bounded and ensure the conservation of mass for Fokker-Planck equations. Lipschitz estimates are in general known under rather general conditions of coercivity, an example being \eqref{H1}, those in Corollary 4.1 p.100 in \cite{L82book}, Section 8 in \cite{Barles} or Chapter 1 in \cite{TranBook}.\\
We start proving a two-side rate of convergence for the vanishing viscosity for Lipschitz solutions. The result is already known from \cite{Evansadjoint}, we only slightly reword the proof and take care of the constants in the estimates to compare it with the corresponding results obtained in \cite{L82book}.
\begin{thm}\label{ev1}
Let $H\in W^{1,\infty}_{\mathrm{loc}}(\R^n)$, $u_0\in W^{1,\infty}(\R^n)$ and $u_\eps$, $u_\eta$ be two solutions to \eqref{hjsemi} with $f=0$. Then
\[
\|u_\eps-u_\eta\|_{L^\infty(Q)}\leq 2(\sqrt{\eps}-\sqrt{\eta})\sqrt{2nT\|Du_0\|_\infty},\ \forall \eps,\eta\geq0.
\]
Moreover $u_\eps$ converges in $L^\infty(\R^n)$ to the viscosity solution $u\in W^{1,\infty}(Q)$ of the first-order Hamilton-Jacobi equation
\eqref{first} and we have the rate 
\[
\|u_\eps-u\|_{L^\infty(Q)}\leq 2\sqrt{2nT\|Du_0\|_\infty}\sqrt{\eps}.
\]
\end{thm}
\begin{proof}
We can assume that $H\in W^{1,\infty}(\R^n)$ by the global Lipschitz estimate in Lemma \ref{grad}. Indeed, one can consider (since $f=0$), setting $R=\|Du_0\|_\infty$, the truncated Hamiltonian
\[
\widetilde{H}(p)=H(p)\text{ if }|p|\leq R,\ \widetilde{H}(p)=H\left(\frac{R}{|p|}p\right)\text{ if }|p|\geq R,
\]
and argue with $\widetilde{H}\in W^{1,\infty}(\R^n)$ instead of $H$, which is only locally Lipschitz. 
We first estimate
\[
\eps\iint_{Q}|D^2u|^2\rho\,dxdt\leq 2\|Du_0\|_\infty.
\]
We use the B\"ochner's identity to find for $g=|Du_\eps|^2$
\[
\partial_t g-\eps\Delta g+2\eps|D^2u_\eps|^2+D_pH(Du_\eps)\cdot Dg=0.
\]
We test the above equation with the adjoint variable $\rho$ solving 
\begin{equation}\label{adjrate1}
\begin{cases}
\partial_t \rho-\eta\Delta \rho-\mathrm{div}(b(x,t)\rho)=0&\text{ in }\R^n\times(0,\tau),\\
\rho(x,\tau)=\rho_\tau(x)&\text{ in }\R^n.
\end{cases}
\end{equation}
with $b(x,t)=D_pH(Du_\eps)$, $\rho_\tau\in C_c^\infty(\R^n)$, $\rho_\tau\geq0$, $\rho_\tau\in L^1(\R^n)$, $\|\rho_\tau\|_{L^1}=1$, to find
\[
\eps\iint_{Q}|D^2u_\eps|^2\rho\,dxdt\leq -\int_{\R^n}g(\tau)\rho(\tau)\,dx+\int_{\R^n}g(0)\rho(0)\,dx\leq  2\|Du_0\|_\infty.
\]
We now consider the equation satisfied by $w=u_\eps-u_\eta$, that is,
\[
\partial_t w-\eta\Delta w+H(Du_\eps)-H(Du_\eta)=(\eps-\eta)\Delta u_\eps.
\]
Since $H\in W^{1,\infty}_{\mathrm{loc}}(\R^n)$ we have
\begin{equation}\label{adjrate}
\partial_t w-\eta\Delta w+\underbrace{\left(\int_0^1 D_pH(sDu_\eps+(1-s)Du_\eta)\,ds\right)}_{b(x,t)}\cdot Dw= (\eps-\eta)\Delta u_\eps.
\end{equation}
By duality, using that $\rho\geq0$ and $w(0)=0$, along with the conservation of mass $\int_{\R^n}\rho\,dx=1$ (note that the drift is now globally bounded) and the Cauchy-Schwarz inequality, we obtain
\begin{align*}
\int_{\R^n}w(\tau)\rho_\tau(x)\,dx&= (\eps-\eta)\iint_{Q}\Delta u_\eps\rho\,dxdt\leq (\eps-\eta)\sqrt{n}\iint_{Q}|D^2u_\eps|\rho\,dxdt\\
&\leq (\eps-\eta)\sqrt{n}\left(\iint_{Q}|D^2u_\eps|^2\rho\,dxdt\right)^\frac12\left(\iint_Q \rho\,dxdt\right)^\frac12\\&\leq (\eps-\eta)\frac{1}{\sqrt{\eps}}\sqrt{2nT\|Du_0\|_\infty}\leq 2(\sqrt{\eps}-\sqrt{\eta})\sqrt{2nT\|Du_0\|_\infty}.
\end{align*}
This gives the estimate on $u_\eps-u_\eta$ from above. The estimate on the negative part is similar.
\end{proof}
\begin{rem}
The rate of convergence in Theorem \ref{ev1} can be proved also for equations with a globally Lipschitz right-hand side $f$, but the estimates will depend also on $\|Df\|_\infty$. Indeed one would have
\begin{multline*}
\eps\iint_{Q}|D^2u_\eps|^2\rho\,dxdt\leq -\int_{\R^n}g(\tau)\rho(\tau)\,dx+\int_{\R^n}g(0)\rho(0)\,dx+\|Df\|_\infty\|Du_\eps\|_\infty T\\
\leq \|Du_0\|_\infty+\|Du_\eps\|_\infty (1+T\|Df\|_{\infty}).
\end{multline*}
One can also require $H\in W^{1,\infty}(\R^n)$ directly as the conservation of mass for the adjoint problem continues to hold. $H$ can also be taken as a function of $x,t$ under the assumptions of \cite{Tran2011}, and the above proof shows the same rate.
\end{rem}
We now turn to SSH solutions, and propose a new proof by PDE methods of a result obtained by P.-L. Lions in \cite{L82book} through a related probabilistic argument, see also \cite{BCD} for a proof by doubling variables' methods. Differently from \cite{L82book}, we require on $u_\eps$ the weaker condition $L^1_t(L^\infty_x)$-SSH.

\begin{thm}\label{inftyrate1s}
Let $H\in W^{1,\infty}_{\mathrm{loc}}(\R^n)$,  $u_\eps$ be a $L^1_t(L^\infty_x)$-SSH solution to \eqref{hjsemi} and $u_\eta$ another solution of \eqref{hjsemi} with viscosity $\eps$ replaced by $\eta$ and the same initial condition as $u_\eps$. Then
\[
\|(u_\eps-u_\eta)^+\|_{L^\infty(Q)}\leq \|(\Delta u_\eps)^+\|_{L^1(0,T;L^\infty(\R^n))}(\eps-\eta),\ \eps\geq\eta\geq0.
\]
\end{thm}


\proof
We start again with the difference $w=u_\eps-u_\eta$ satisfying
\[
\partial_t w-\eta\Delta w+H(Du_\eps)-H(Du_\eta)=(\eps-\eta)\Delta u_\eps
\]
As above, we have
\[
\partial_t w-\eta\Delta w+\left(\int_0^1 D_pH(sDu_\eps+(1-s)Du_\eta)\,ds\right)\cdot Dw= (\eps-\eta)\Delta u_\eps.
\]
Using the solution $\rho\geq0$ to \eqref{adjrate} and arguing by duality we obtain
\[
\int_{\R^n}w(\tau)\rho_\tau(x)\,dx\leq\int_{\R^n}w(0)\rho(0)+(\eps-\eta)\int_0^\tau\|(\Delta u_\eps)^+\|_{L^\infty(\R^n)}\int_{\R^n}\rho\,dxdt.
\]
Since $\int_{\R^n}\rho\,dx=1$ due to the standing assumptions on $H$ (and using the same truncation argument of Theorem \ref{ev1}) and the fact that $w(0)=0$, we get
\begin{equation*}
\|(u_\eps-u_\eta)^+\|_{L^\infty(\R^n)}\leq \|(\Delta u_\eps )^+\|_{L^1(0,T;L^\infty(\R^n))}(\eps-\eta)\eqno\square
\end{equation*}

\begin{rem}
Theorem \ref{inftyrate1s} can be extended to more general Hamiltonians depending also on $(x,t)$ as soon as the regularity of $D_pH$ ensures the well-posedness and the conservation of mass for transport equations with degenerate diffusion, see e.g. \cite{LBL}. Under such assumptions, we would conclude the same rate of convergence.
\end{rem}
Combining the previous result with the second-order bounds of Theorem \ref{semiconc1} we get the following one-side $\mathcal{O}(\eps)$ rate of convergence (under additional hypotheses on $H$).
\begin{cor}\label{rateSSH}
Assume that  $H\in W^{1,\infty}_{\mathrm{loc}}(\R^n)$ satisfies \eqref{H1} and \eqref{H4}, $f$ is $L^1_t(L^\infty_x)$-SSH and $u_0$ is $L^\infty_x$-SSH. Then, the unique solution $u_\eps$ of \eqref{hjsemi} converges to the unique bounded viscosity solution $u\in W^{1,\infty}(Q)$ of \eqref{first}.
In addition, we have the following bound for all $t\in[0,T]$
\[
\|(u_\eps-u_\eta)^+(t)\|_{L^\infty(\R^n)}\leq (\|(\Delta f)^+\|_{L^1(0,T;L^\infty(\R^n))}+\|(\Delta u_0)^+\|_{L^\infty(\R^n)})T(\eps-\eta),\ \eps\geq\eta\geq0.
\]
and, consequently, we have for $f=0$ the two-side rate 
\[-\sqrt{2n}\sqrt{T}\|Du_0\|_{L^\infty(\R^n)}\sqrt{\eps}\leq u_\eps-u\leq \|(\Delta u_0)^+\|_{L^\infty(\R^n)})T\eps.  \]
\end{cor}


\begin{proof}
The convergence of $u_\eps$ towards $u$ has been already discussed in the introduction. We prove the bound on $u_\eps-u_\eta$. Since $\Delta f\leq c(t)$ with $c\in L^1(0,T)$, this implies by Theorem \ref{semiconc1} and Remark \ref{Hconvex} the following bound
\[
\Delta u_\eps\leq \|(\Delta f)^+\|_{L^1(0,T;L^\infty(\R^n))}+\|(\Delta u_0)^+\|_{L^\infty(\R^n)}.
\]
Then, the result follows immediately by Theorem \ref{inftyrate1s} using that
\[
\|(\Delta u_\eps)^+\|_{L^1(0,T;L^\infty(\R^n))}\leq \|(\Delta u_\eps)^+\|_{L^\infty(Q)}T\leq (\|(\Delta f)^+\|_{L^1(0,T;L^\infty(\R^n))}+\|(\Delta u_0)^+\|_{L^\infty(\R^n)})T.
\]
When $f=0$ the second statement follows from the above estimate combined with Theorem \ref{ev1}.
\end{proof}
\begin{rem}
If $f=0$ and $H=H(p)\in W^{1,\infty}_{\mathrm{loc}}(\R^n)$ is convex, one has by Corollary \ref{rateSSH}
\[
\|(\Delta u)^+(t)\|_{L^\infty(\R^n)}\leq \|(\Delta u_0)^+\|_{L^\infty(\R^n)}.
\]
Consequently, the estimate for the one-side rate of convergence becomes
\[
u_\eps-u_\eta\leq \|(\Delta u_0)^+\|_{L^\infty(\R^n)}T(\eps-\eta),\ \eps\geq\eta\geq0.
\]
This is the same estimate stated by P.-L. Lions in Proposition 11.2 of \cite{L82book}. The corresponding estimate for the stationary problem has been proved in Section 6.2 of \cite{L82book} using methods from probability. Our proofs and the results in the previous Theorems \ref{inftyrate1s} and Corollary \ref{rateSSH} are new.
\end{rem}


\subsubsection{$L^p$ rate of convergence}
In the next proposition, we extend the $L^\infty$ estimate in Theorem \ref{ev1} to $L^p_{\mathrm{loc}}$ norms for Lipschitz continuous solutions.
\begin{thm}\label{prate1}
Let $H\in W^{1,\infty}_{\mathrm{loc}}(\R^n)$, $u_0\in W^{1,\infty}(\R^n)$ and $u_\eps$ be a solution to \eqref{hjsemi}. Then, $u_\eps$ converges in $L^\infty(Q)$ to the viscosity solution $u\in W^{1,\infty}(Q)$ of the first-order Hamilton-Jacobi equation \eqref{first}. We have the rate
\[
\|u_\eps-u\|_{L^p_{\mathrm{loc}}(Q)}\leq C\sqrt{\eps}, \quad 1 \leq p < \infty,
\]
where $C$ depends on $n,T,\|Du_0\|_{L^\infty(\R^n)},p$.
\end{thm}
\begin{proof}
The proof is the same as that in Theorem \ref{ev1}, but we have to introduce the adjoint problem \eqref{adjrate} with terminal data $\rho_\tau\in L^1\cap L^{p'}$, as in Theorem \ref{pSSHrate}.
\end{proof}

We next prove a one side estimate by duality for SSH solutions on the whole space.
\begin{thm}\label{pSSHrate}
Let $H\in W^{1,\infty}_{\mathrm{loc}}(\R^n)$, $u_\eps$ be a $L^1_t(L^\infty_x)$-SSH solution to \eqref{hjsemi}, and $u_\eta$ be another solution (with viscosity $\eps$ replaced by $\eta$) with the same initial condition. Then, there exists a constant $C>0$
\[
\|(u_\eps-u_\eta)^+(t)\|_{L^p_{\mathrm{loc}}(\R^n)}\leq C\|(\Delta u_\eps)^+\|_{L^1(0,T;L^\infty(\R^n))}(\eps-\eta),\ \eps\geq\eta\geq0,\ p\geq1.
\]
\end{thm}
\begin{proof}
The difference $w=u_\eps-u_\eta$ satisfies
\[
\partial_t w-\eta\Delta w+H(Du_\eps)-H(Du_\eta)=(\eps-\eta)\Delta u_\eps, w(0)=0.
\]
As in Theorem \ref{inftyrate1s} we linearize the equation and then introduce the adjoint problem \eqref{adjrate} with terminal datum $\rho_\tau\in C_0^\infty(\R^n)$, $\rho_\tau\geq0$, $\rho_\tau\in L^1(\R^n)\cap L^{p'}(\R^n)$, $\|\rho_\tau\|_{L^1\cap L^{p'}}=1$. Note that, arguing as in Theorem \ref{aho}, we have
\begin{equation}\label{L1p}
\int_{\R^n}\rho_\tau(x)\,dx\leq 1\quad \textrm{and}\quad  \int_{\R^n}\rho(x,t)\,dx\leq 1,\ \forall t\in[0,\tau).
\end{equation} 
Moreover, $\rho\geq0$ by the maximum principle. By duality, we obtain
\[
\int_{\R^n}w(\tau)\rho_\tau(x)\,dx\leq\int_{\R^n}w(0)\rho(0)+(\eps-\eta)\int_0^T\|(\Delta u_\eps)^+\|_{L^\infty(\R^n)}\int_{\R^n}\rho\,dxdt.
\]
Using that $w(0)=0$ and \eqref{L1p},  we have
\[
\|(u_\eps-u_\eta)^+(t)\|_{L^\infty(\R^n)+L^p(\R^n)}\leq C_1\|(\Delta u_\eps)^+\|_{L^1(0,T;L^\infty(\R^n))}(\eps-\eta)
\]
Moreover, appealing to Lemma \ref{emb+} we have that, for all $K\subset\subset\R^n$, there exists a constant $C_2$ depending on $C_1,p,K$ such that
\[
\|(u_\eps-u_\eta)^+(t)\|_{L^p(K)}\leq C_2\|(\Delta u_\eps)^+\|_{L^1(0,T;L^\infty(\R^n))}(\eps-\eta).
\]
\end{proof}


\begin{rem}
One can remove the bound on $\|(\Delta u_\eps)^+\|_{L^1(0,T;L^\infty(\R^n))}$ in Theorem \ref{prate1}, as in Corollary \ref{rateSSH}, using the one-side bounds in Theorem   \ref{semiconc1} and obtain a more precise estimate.
\end{rem}

The next result shows instead that a one-side rate in $L^p$ for $p>1$ holds globally in $\R^n$ under the additional assumption that $u_\eta$ is semiconcave  and $H$ is convex. These further assumptions  are fundamental to apply the stability estimates for transport equations in Theorem \ref{Krylov}. However, we weaken the requirement on $u_\eps$, which will now be assumed in $L^p$-SSH, $p>1$. Note that uniqueness and stability for $L^p$-SSH solutions require the restriction $p\geq n$, cf. Remark 3.6 of \cite{L82book}, because of regularity estimates for parabolic equations due to N.V. Krylov \cite{Krylov}.
\begin{thm}
Let $H\in W^{1,\infty}_{\mathrm{loc}}(\R^n)$ be convex, $u_\eps$ be a $L^1_t(L^p_x)$-SSH solution to \eqref{hjsemi}, and $u_\eta$ be another $L^1_t(L^\infty_x)$-semiconcave solution (with viscosity $\eps$ replaced by $\eta$) having the same initial condition. Then, there exists a constant $C>0$
\[
\|(u_\eps-u_\eta)^+(t)\|_{L^p(\R^n)}\leq C\|(\Delta u_\eps)^+\|_{L^1(0,T;L^p(\R^n))}(\eps-\eta),\ \eps\geq\eta\geq0,\ p>1.
\]
where $C$ depends on the data of the problem and also on $\|(D^2u_\eta)^+\|_{L^1_t(L^\infty_x)}$.
\end{thm}
\begin{proof}
The proof is similar to the previous result, since the difference $w^+=(u_\eps-u_\eta)^+$ satisfies the inequality
\[
\partial_t w^+-\eta\Delta w^++D_pH(Du_\eta)\cdot Dw\chi_{\{w>0\}}\leq (\eps-\eta)\Delta u_\eps \chi_{\{w>0\}}, w(0)=0.
\]
Now, we consider the adjoint problem 
\begin{equation*}
\begin{cases}
\partial_t \rho-\eta\Delta \rho-\mathrm{div}(D_pH(Du_\eta)\rho \chi_{\{w>0\}})=0&\text{ in }\R^n\times(0,\tau)\\
\rho(x,\tau)=\rho_\tau(x)&\text{ in }\R^n.
\end{cases}
\end{equation*}
with terminal datum $\rho_\tau:=(w^+(\tau))^{p-1}/\|w^+\|^{p-1}_{L^p}$ that belongs only to $L^{p'}$ having $\|\rho_\tau\|_{p'}\leq1$ (and not on the intersection $L^1\cap L^{p'}$). This yields a (global) bound on $(u_\eps-u_\eta)^+(t)\in L^p(\R^n)$ for all $t\in(0,T)$. However, one has to estimate the term involving $\Delta u_\eps$ on the right-hand side of the equation satisfied by $w$ by the H\"older's inequality as follows:
\[
(\eps-\eta)\int_0^T\|(\Delta u_\eps)^+\|_{L^p(\R^n)}\left(\int_{\R^n}|\rho|^{p'}\right)^{\frac{1}{p'}}\,dxdt.
\]
Then, one applies the $L^r$ stability estimates in Theorem \ref{Krylov} with $r=p'$ to bound $\|\rho(t)\|_{p'}$ in terms of $\|(D^2 u_\eta)^+\|_{L^1_t(L^\infty_x)}$.
\end{proof}
\subsection{Rate of convergence: an improved estimate for compact state spaces}\label{sec;rate2}
We now address $L^p$ rates of convergence using duality methods, as in \cite{LinTadmor}. In this setting we need to strengthen the requirement both on $u_\eps$ and $u_\eta$, and work in a compact state space as the flat torus, but we are able to get a two side control on the difference $u_\eps-u_\eta$ in any $L^p$ space and for semiconcave or semi-superharmonic solutions. We premise the following lemma concerning superharmonic and semiconcave functions to study the $L^1$ case.
\begin{lemma}\label{laplL1}
The following properties hold:
\begin{itemize}
\item[(i)] Let $u$ be such that $\Delta u\leq C$ in the viscosity sense in $\R^n$, then $\Delta u\in L^1_{\mathrm{loc}}(\R^n)$. This in particular holds in the periodic case of the flat torus $\T^n$. 
\item[(ii)] If $u$ is semiconcave in $\R^n$, then $D^2u\in L^1_{\mathrm{loc}}(\R^n)$.
\end{itemize}
\end{lemma}
\begin{proof}
The validity of (i) has been pointed out in \cite{L82book} and \cite{LinTadmor}. The proof is also   a step towards the validity of $C^{1,1}$ estimates for fully nonlinear second-order concave equations. The main ideas can be found in Chapter 6 of \cite{CC}: indeed, up to subtracting a paraboloid with opening $C/n$, one may assume that $\Delta u\leq0$ in the viscosity sense, so that one can apply the mean value inequalities and argue as in p. 55-56 of \cite{CC}. \\
To prove (ii), we observe that a function is semiconcave if it is so in the variables $x_1,...,x_n$, cf. Example 5.1 in \cite{BardiDragoni}. Using these properties, we can follow an argument of \cite{K65}, see also \cite{K66d}. We prove that for a $C^2$ function $v$ on the bounded interval $[a-h,b+h]$, $a,b\in\R$, $h>0$, which satisfies $|v'(x)|\leq K_0$ and $v''(x)\leq K_1$ (recall that a semiconcave function is also Lipschitz continuous, cf. \cite{CannarsaSinestrari}), we have
\[
\int_a^b\left|\frac{v(x+h)-2v(x)+v(x-h)}{h^2}\right|\,dx\leq c,
\]
where $c$ depends on the size of the interval, $K_0,K_1$. Indeed, using that 
\[
v(x+h)-2v(x)+v(x-h)=\int_{x}^{x+h}(v'(t)-v'(t-h))\,dt=h\int_{x}^{x+h}\int_0^1v''(t-h\xi)\,d\xi dt
\]
we conclude
\[
v(x+h)-2v(x)+v(x-h)=h^2\int_0^1\int_0^1v''(x+sh+zh)\,dzds.
\]
Therefore, using that for $|y|\leq h$
\[
\int_a^b|v''(x+y)|\,dx\leq \int_a^b(K_1-v''(x+y))\,dx\leq K_1(b-a)+K_0(b-a),
\]
the proof follows combining the last two inequalities.
\end{proof}
\begin{rem}
A related result of part (ii) in Lemma \ref{laplL1} is due to A. D. Aleksandrov, see Lemma 1 in Section 2 of \cite{IshiiFE}, who proved that if $u$ is $C(\R^n)$ and semiconvex, then there are matrices $X\in L^{1}_{\mathrm{loc}}(\R^n)$ and $Y\in \mathcal{M}(\R^n)$ (the space of Radon measures) such that $D^2u=X+Y$ in $D'(\R^n)$ (the space of distributions), $Y\geq0$ in $\mathcal{M}(\R^n)$. The recent work \cite{FRT} addressed a similar property for the symmetric case of subharmonic functions in the sense of distributions, proving result $(D^2u)_{-}\in L^1_{\mathrm{loc}}$, which is sharp since $(D^2u)_{-}$ cannot be in $L^p_{\mathrm{loc}}$, $p>1$. Property (ii) in Lemma \ref{laplL1} shows the stronger property that both the positive and negative part of $D^2u$ are in $L^1_{\mathrm{loc}}$ provided that one has a one-side control on the whole Hessian (and not only on its trace, as in \cite{FRT}).
\end{rem}
The next is the main result of the section. It contains an estimate on the rate of convergence of the solution of the viscous equation $u_\eps$ towards the inviscid solution $u$ in the space $L^\infty_t(L^1_x)$ under the assumption that both of them are $L^1_t(L^\infty_x)$-semiconcave. It also provides a second rate of convergence in the stronger norm $L^\infty_t(L^p_x)$ at the expenses of assuming a two-side a priori bound on the solution.
\begin{thm}\label{rateL1}
Let $u_\eps, u_\eta$ be $L^1_t(L^\infty_x)$-semiconcave solutions of \eqref{hjsemi} with viscosity $\eps$ and, respectively, $\eta$. Let also $H\in W^{1,\infty}_{\mathrm{loc}}(\R^n)$ be such that \eqref{H4} holds. Then, there exists a constant $C>0$ such that
\[
\|u_\eps-u_\eta\|_{L^\infty(0,T;L^1(\T^n))}\leq C(\eps-\eta),\ \eps\geq\eta\geq0.
\]
where $C$ depends on the semiconcavity constant of $u_\eps$ and $u_\eta$. If, in addition, $-u_\eta$ is $L^1_t(L^\infty_x)$-SSH we conclude
\[
\|u_\eps-u_\eta\|_{L^\infty(0,T;L^p(\T^n))}\leq C(\eps-\eta),\ \eps\geq\eta\geq0,\;\; 1 \leq p \leq \infty,
\]
\end{thm}
\begin{proof}
Consider $w=u_\eps-u_\eta$ satisfying the equation
\[
\partial_t w-\eta\Delta w+H(Du_\eps)-H(Du_\eta)= (\eps-\eta)\Delta u_\eps, w(0)=0.
\]
We thus have
\[
\partial_t w-\eta\Delta w-b(x,t)\cdot Dw= (\eps-\eta)\Delta u_{\eps}, w(0)=0,
\]
where $b$ is the average velocity
\[
b(x,t)=-\int_0^1 D_pH(sDu_\eps+(1-s)Du_\eta)\,ds.
\]
We now recall that since $u_\eps,u_\eta$ are semiconcave stable we have, using \eqref{H4}, the following estimate for the average velocity
\[
\mathrm{div}(b)=-\int_0^1 s\sum_{i,j}D^2_{p_ip_j}H\partial_{x_ix_j}u_\eps+(1-s)\sum_{i,j}D^2_{p_ip_j}H\partial_{x_ix_j}u_\eta\,ds\geq -c(t)\in L^1.
\]

%Next, setting
%\[
%\widetilde w(x, t) = w(x, t) - (\eps - \eta) \int_{0}^{t} \Delta u_\eps(s)\;ds,
%\]
%we have that $\widetilde w$ satisfies
%\begin{equation*}
%\partial_t \widetilde w-\eta\Delta \widetilde w-b(x,t)\cdot D\widetilde w= 0, \quad \widetilde w(x, 0) = 0.
%\end{equation*}
%We multiply the above equation by $\beta'(\widetilde w)\geq0$, where $\beta(\widetilde w)$ will be defined later, and obtain the identity
%\[
%\partial_t \beta(\widetilde w)-\eta \beta'(\widetilde w)\Delta \widetilde w-(b(x,t)\cdot D\widetilde w)\beta'(w)=  0.
%\]
%By integrating on the torus
%\[
%\frac{d}{dt}\int_{\T^n}\beta(\widetilde w)\,dx+\eta\int_{\T^n}\beta''(\widetilde w)|D\widetilde w|^2\,dx-\int_{\T^n}b\cdot D\beta(\widetilde w)\,dx=  0
%\]
%and integrating by parts
%\[
%\frac{d}{dt}\int_{\T^n}\beta(\widetilde w)\,dx+\eta\int_{\T^n}\beta''(\widetilde w)|D\widetilde w|^2\,dx= -\int_{\T^n}\mathrm{div}(b)\beta(\widetilde w)\,dx\]
%Choosing $\beta(r)=r^k$, with $k\in\N$ being any natural even number $k\geq2$, we have
%\[
%\frac{d}{dt}\int_{\T^n}\widetilde w^k\,dx+\eta k(k-1)\int_{\T^n}\beta''(\widetilde w)|D\widetilde w|^2\,dx\leq \|[\mathrm{div}(b(\cdot, t))]^-\|_{L^\infty(\T^n)}\int_{\T^n}\widetilde w^k\,dx.
%\]
%Thus, observing that $\beta''(w) \geq 0$, we obtain 
%\[
%\frac{d}{dt}\int_{\T^n}\widetilde w^k\,dx\leq \|[\mathrm{div}(b(\cdot, t))]^-\|_{L^\infty(\T^n)}\int_{\T^n}\widetilde w^k\,dx.
%\]
%We now recall that since $u_\eps,u_\eta$ is semiconcave stable we have, using \eqref{H4}, the following estimate for the average velocity
%\[
%\mathrm{div}(b)=-\int_0^1 s\sum_{i,j}D^2_{p_ip_j}H\partial_{x_ix_j}u_\eps+(1-s)\sum_{i,j}D^2_{p_ip_j}H\partial_{x_ix_j}u_\eta\,ds\geq -c(t)\in L^1.
%\]
%By Gronwall's inequality, since $\widetilde w(x,0)=0$, we conclude
%\[
%\int_{\T^n} \widetilde w^k\;dx = 0, \quad \textrm{equivalently} \quad w(x, t) - (\eps - \eta) \int_{0}^{t} \Delta u_{\eps}(s)\;ds= 0, \;\;\textrm{a.e.}\;\;\text{ in } \T^n.
%\]
%Thus, using that $\Delta u_\eps\leq c_\eps(t)\in L^1_t$ (recall that $u_\eps$ is $L^1_t(L^\infty_x)$ semiconcave) we get
%\begin{equation*}
%w(x, t) \leq (\eps - \eta)\int_{0}^{t} c_{\eps}(s)\;ds
%\end{equation*}
%and
%\begin{equation*}
%\|(u_{\eps} - u_{\eta})(t)\|_{L^k(\T^n)} \leq (\eps - \eta) \int_{0}^{t} c_\eps (s)\;ds. 
%\end{equation*}
%Finally, observing that the constant on the right-hand side is stable as $k \uparrow \infty$ we get also 
%\begin{equation*}
%\|u_{\eps} - u_{\eta}\|_{L^\infty(\T^n)} \leq (\eps - \eta) \int_{0}^{t} c_{\eps}(s)\;ds. 
%\end{equation*}
%We then recover any $L^p$ with $p\geq 2$ by interpolation.
%
We start with the case $p=1$ as in Theorem \ref{pSSHrate}, by using the adjoint method. Here, we consider the adjoint problem
\begin{equation}\label{adjrate2}
\begin{cases}
\partial_t \rho-\eta\Delta \rho+\mathrm{div}(b(x,t)\rho)=0&\text{ in }\T^n\times(0,\tau),\\
\rho(x,\tau)=\rho_\tau(x)&\text{ in }\T^n,
\end{cases}
\end{equation}
but with terminal datum $\rho(\tau)=\mathrm{sgn}(w(\tau))$ on $\T^n$. Note that
\[
-1\leq\rho(\tau)\leq 1\implies \|\rho(\tau)\|_{L^\infty(\T^n)}\leq 1.
\]
Arguing by duality and using Lemma \ref{laplL1} we get
\begin{multline*}
\int_{\T^n}w(\tau)\rho_\tau(x)\,dx=\int_{\R^n}w(0)\rho(0)\,dx+(\eps-\eta)\int_0^T\int_{\T^n}\Delta u_\eps\rho\,dxdt\\
\leq\|w(0)\|_{L^1(\T^n)}\|\rho(0)\|_{L^\infty(\T^n)}+ (\eps-\eta)\|\Delta u_\eps\|_{L^1(Q_\tau)}\|\rho\|_{L^\infty(Q_\tau)}\leq C\|\rho\|_{L^\infty(Q_\tau)}(\eps-\eta).
\end{multline*}
It remains to estimate $\|\rho\|_{L^\infty(Q_\tau)}$ in terms of the final datum of the adjoint problem, since $w(0)=0$. A classical stability estimate for continuity equations that follows from the Gronwall's inequality or the Feynman-Kac formula \cite[Theorem 4.12]{Figalli} (see also p.710 in \cite{LinTadmor})  leads to
\[
\|\rho(t)\|_{L^\infty(\T^n)}\leq \|\rho(\tau)\|_{L^\infty(\T^n)}e^{\int_0^T\|(\mathrm{div}(b))^-\|_{L^\infty(\T^n)}},\ t\in[0,\tau).
\]
To prove the general case for $p=\infty$, we argue by duality as in Theorem \ref{inftyrate1s} to find the bound on $u_\eps-u_\eta$ from above. The bound from below follows noting that $z=u_\eta-u_\eps$ solves
\[
\partial_t z-\eps\Delta z+H(Du_\eta)-H(Du_\eps)=-(\eps-\eta)\Delta u_\eta,
\]
and hence
\[
\partial_t z-\eps\Delta z-b(x,t)\cdot Dz=-(\eps-\eta)\Delta u_\eta,
\]
where
\[
b(x,t)=-\int_0^1 D_pH(sDu_\eta+(1-s)Du_\eps)\,ds.
\]
We can now proceed again by duality using that $\Delta u_\eta\geq -c_\eta(t)$, and hence testing equation solved by $z$ against the solution of \eqref{adjrate2} having $\rho_\tau\in L^1$. This implies, since $\int_{\T^n}\rho(t)\,dx=1$ and $\rho\geq0$, the inequality
\[
\int_{\T^n}z(\tau)\rho_\tau(x)\,dx=(\eps-\eta)\iint_{Q}(-\Delta u_\eta)\rho\,dxdt\leq (\eps-\eta)\iint_{Q}c_\eta(t)\rho\,dxdt=(\eps-\eta)\int_0^\tau c_\eta(t)\,dt.
\]
The case $p\in(1,\infty)$ follows by the compactness of the flat torus or by interpolation.
\end{proof}

\begin{cor}
Under the assumptions of Theorem \ref{semiconc1}, let $u_\eps$ and $u_\eta$ be $L^1_t(L^\infty_x)$-semiconcave with viscosity $\eps$ and, respectively, $\eta$. Then
\[
\|u_\eps-u_\eta\|_{L^\infty(0,T;L^1(\T^n))}\leq C(\eps-\eta).
\]
where $C$ depends on $c_f(t)\in L^1(0,T)$, $c_0$ and the constants $C_{H,4}$ appearing in \eqref{H4}.
\end{cor}
\begin{rem}
Quite sharp assumptions ensuring that $u_\eta$ is in $W^{2,\infty}$ can be found in Proposition 7.1 of \cite{L82book}. In the second part of Theorem \ref{rateL1} for the $L^p$ case we are requiring only a control on the trace of the Hessian of $u_\eta$ from below, in addition to the semiconcavity of $u_\eps,u_\eta$.
\end{rem}
%We now give a slightly different proof of a two-side rate of order $\mathcal{O}(\eps)$ for a uniformly convex $H$.
%\begin{thm}\label{new}
%\alert{Let $u_\eps,u_\eta$ be semiconcave solutions of \eqref{hjsemi} with viscosity $\eps$ and, respectively, $\eta$. Let also $H$ be uniformly convex. Then
%\[
%\|u_\eps-u_\eta\|_{L^\infty(0,T;(L^\infty(\T^n))')}\leq C(\eps-\eta)
%\]
%where $X'$ is the dual of $X$.}
%\end{thm}
%\begin{proof}
%We proceed similarly to Theorem 1.43 in \cite{TranBook} and Theorem \ref{ev1}. We start by estimating 
%\[
%\iint_Q|D^2u_\eps|^2\rho\,dxdt\leq C,
%\]
%where $C$ now does not depend on $\eps$ and depends only on the semiconcavity constant of $u_\eps$, $\rho$ being the solution of \eqref{adjrate2} with $\eta=\eps$ and $b=-D_pH(Du_\eps)$. We have that $z=(u_\eps)_{x_ix_i}$ solves the equation
%\[
%\partial_t z-\eps\Delta z+D_pH(Du_\eps)\cdot Dz+D^2_{pp}H(Du_\eps)D(u_\eps)_{x_i}\cdot D(u_\eps)_{x_i}=0
%\]
%By duality, taking the solution of \eqref{adjrate2} with $\rho_\tau\in L^\infty(\T^n)$ such that $\|\rho_\tau\|_{L^\infty}=1$, we conclude
%\begin{align*}
%\iint_Q D^2_{pp}H(Du_\eps)(Du_\eps)_{x_i}\cdot (Du_\eps)_{x_i}\rho\,dxdt&\leq -\int_{\T^n}(u_\eps)_{x_ix_i}(\tau)\rho_\tau(x)\,dx+\int_{\T^n}(u_\eps)_{x_ix_i}(0)\rho(0)\,dx\\
%&\leq \|(u_\eps)_{x_ix_i}\|_{L^1(\T^n)}\|\rho_\tau\|_{L^\infty(\T^n)}+c_\eps\|\rho(0)\|_{L^1(\T^n)}\\
%&\leq \|(u_{\eps})_{x_ix_i}\|_{L^1(\T^n)}\|\rho_\tau\|_{L^\infty(\T^n)}+c_\eps\|\rho(0)\|_{L^\infty(\T^n)}\leq C,
%\end{align*}
%where we used again that $\rho\geq0$, $\|\rho(t)\|_{L^\infty(\T^n)}\leq \|\rho(\tau)\|_{L^\infty(\T^n)}e^{\int_0^T\|(\mathrm{div}(-D_pH(Du_\eps)))^-\|_{L^\infty(\T^n)}},\ t\in[0,\tau)$ and also (ii) in Lemma \ref{laplL1}. Since $H$ is uniformly convex, we have $D^2_{pp}H(Du_\eps)\xi\cdot \xi\geq \theta|\xi|^2$ for $\xi\in\R^n$ and $\theta>0$, and summing over $i$ we find the estimate
%\[
%\iint_Q |D^2u_\eps|^2\rho\,dxdt\leq \tilde C,
%\]
%where $\tilde C$ depends on the semiconcavity constant of $u_\eps$ and $\theta$, but not on $\eps$. Note that the uniform convexity of $H$ and the semiconcavity of $u_\eps$ imply $-\mathrm{div}(D_pH(Du_\eps))=-D^2_{p_ip_j}H(Du_\eps)\partial_{x_ix_j}u_\eps\geq -c$. The same scheme of the proof of Theorem \ref{ev1} gives for $w=u_\eps-u_\eta$
%\begin{align*}
%\int_{\R^n}w(\tau)\rho_\tau(x)\,dx&= (\eps-\eta)\iint_{Q}\Delta u_\eps\rho\,dxdt\leq (\eps-\eta)\sqrt{n}\iint_{Q}|D^2u_\eps|\rho\,dxdt\\
%&\leq (\eps-\eta)\sqrt{n}\left(\iint_{Q}|D^2u_\eps|^2\rho\,dxdt\right)^\frac12\left(\iint_Q \rho\,dxdt\right)^\frac12\\&\leq (\eps-\eta)\sqrt{\tilde C T n\|\rho\|_{L^\infty(Q)}}\leq \bar C(\eps-\eta),
%\end{align*}
%where $\bar C$ depends on $\tilde C, n, T$ and on the semiconcavity constants of $u_\eps$ and $u_\eta$ (because of the estimate on $\|\rho\|_\infty$). By duality this gives an estimate of $w^+$ on $(L^\infty)'$. The estimate from below of $w$ is the same, as we can use the bound
%\[
%\int_{\R^n}w(\tau)\rho_\tau(x)\,dx= (\eps-\eta)\iint_{Q}\Delta u_\eps\rho\,dxdt\geq -(\eps-\eta)\iint_{Q}|\Delta u_\eps|\rho\,dxdt,
%\]
%and proceed applying (from below) the H\"older's inequality.
%\end{proof}
We now provide an improvement of the rate for Lipschitz solutions of Hamilton-Jacobi equations with semisuperharmonic initial condition. Such an enhancement is due to the uniform convexity of the Hamiltonian. This extends a result of \cite{TranBook} to the time-dependent setting:
\begin{thm}\label{new2}
Let $u_\eps,u_\eta$ be Lipschitz solutions of \eqref{hjsemi} with viscosity $\eps$ and, respectively, $\eta$. Let also $H$ be uniformly convex with constant $\theta$ and $\Delta u_0\leq c$. Then
\[
\|u_\eps-u_\eta\|_{L^\infty(0,T;(W^{1,1}(\T^n))')}\leq C(\eps-\eta)
\]
where $X'$ is the dual of $X$. Equivalently, one obtains the following weighted rate of convergence
\[
\esssup_{t\in(0,T)}\left|\int_{\T^n}(u_\eps-u_\eta)(t)\psi(x)\,dx\right|\leq C(\eps-\eta),
\]
where $\psi\geq0$, $\psi\in W^{1,1}(\T^n)$ with $\|\psi\|_{W^{1,1}(\T^n)}=1$.
\end{thm}
\begin{proof}
We start again by estimating 
\[
\iint_Q|D^2u_\eps|^2\rho\,dxdt\leq C,
\]
where $C$ now does not depend on $\eps$, $\rho$ being the solution of \eqref{adjrate2} with $\eta=\eps$ and $b=-D_pH(Du_\eps)$. We have that $z=(u_\eps)_{x_ix_i}$ solves the equation
\[
\partial_t z-\eps\Delta z+D_pH(Du_\eps)\cdot Dz+D^2_{pp}H(Du_\eps)D(u_\eps)_{x_i}\cdot D(u_\eps)_{x_i}=0
\]
By duality, taking the solution of \eqref{adjrate2} with $\rho_\tau\in W^{1,1}(\T^n)$, $\rho_\tau\geq0$, such that $\|\rho_\tau\|_{W^{1,1}(\T^n)}=1$, we conclude
\begin{multline}\label{D2ppH}
\iint_Q D^2_{pp}H(Du_\eps)(Du_\eps)_{x_i}\cdot (Du_\eps)_{x_i}\rho\,dxdt\leq -\int_{\T^n}(u_\eps)_{x_ix_i}(\tau)\rho_\tau(x)\,dx+\int_{\T^n}(u_\eps)_{x_ix_i}(0)\rho(0)\,dx\\
\leq \|(u_\eps)_{x_i}\|_{L^\infty(\T^n)}\|(\rho_\tau)_{x_i}\|_{L^1(\T^n)}+\int_{\T^n}(u_\eps)_{x_ix_i}(0)\rho(0)\,dx
\end{multline}
where we used the integration by parts on the first integral. Since $H$ is uniformly convex and $\rho\geq0$, we have $D^2_{pp}H(Du_\eps)\xi\cdot \xi\geq \theta|\xi|^2$ for $\xi\in\R^n$ and $\theta>0$, and summing over $i$ we find the estimate
\begin{align*}
\theta\iint_Q |D^2u_\eps|^2\rho\,dxdt&\leq \tilde C\|D\rho_\tau\|_{L^1(\T^n)}+\int_{\T^n}\Delta u_\eps(0) \rho(0)\,dx\\
&\leq \tilde C\|D\rho_\tau\|_{L^1(\T^n)}+\|(\Delta u_0)^+\|_{L^\infty(\T^n)}\|\rho_\tau\|_{L^1(\T^n)}\\
&\leq (\tilde C+\|(\Delta u_0)^+\|_{L^\infty(\T^n)})\|\rho_\tau\|_{W^{1,1}(\T^n)},
\end{align*}
where $\tilde C$ depends on the Lipschitz constant of $u_\eps$, but not on $\eps$. The same scheme of the proof of Theorem \ref{ev1} gives for $w=u_\eps-u_\eta$
\begin{align*}
\int_{\R^n}w(\tau)\rho_\tau(x)\,dx&= (\eps-\eta)\iint_{Q}\Delta u_\eps\rho\,dxdt\leq (\eps-\eta)\sqrt{n}\iint_{Q}|D^2u_\eps|\rho\,dxdt\\
&\leq (\eps-\eta)\sqrt{n}\left(\iint_{Q}|D^2u_\eps|^2\rho\,dxdt\right)^\frac12\left(\iint_Q \rho\,dxdt\right)^\frac12\\&\leq (\eps-\eta)\sqrt{n\frac1\theta(\tilde C+\|(\Delta u_0)^+\|_{L^\infty(\T^n)})\|\rho_\tau\|_{W^{1,1}(\T^n)}}\sqrt{\|\rho_\tau\|_{L^1(\T^n)}}\\
&\leq \bar C\|\rho_\tau\|_{W^{1,1}(\T^n)}(\eps-\eta).
\end{align*}
where $\bar C$ depends on $\tilde C, \theta,n, \|(\Delta u_0)^+\|_{L^\infty(\T^n)}$. By duality this gives an estimate of $w^+$ on $(W^{1,1})'$. The estimate from below of $w$ is the same, as we can use the bound from below
\[
\int_{\R^n}w(\tau)\rho_\tau(x)\,dx= (\eps-\eta)\iint_{Q}\Delta u_\eps\rho\,dxdt\geq -(\eps-\eta)\iint_{Q}|\Delta u_\eps|\rho\,dxdt.
\]
\end{proof}

\begin{rem}
A slight modification of the proof of Theorem \ref{new2} gives an alternative proof to Theorem \ref{rateL1}. It is enough to take $\rho_\tau\in L^\infty(\T^n)$ such that $\|\rho_\tau\|_{L^\infty(\T^n)}=1$ and avoid the integration by parts in the estimate \eqref{D2ppH}.
\end{rem}

\begin{rem}
The estimate in Theorem \ref{rateL1} was stated for semiconvex solutions of some stationary equations in Theorem 4 of \cite{K75}, eq. (16), in $L^1$ norms with order $\mathcal{O}(\eps^\nu)$, $\nu\in(0,1)$, and in eq. (17) of  \cite{K75} in $L^p$ norms for $p$ sufficiently large, with order $\mathcal{O}(\eps)$. The $L^1$ rate of order $\mathcal{O}(\eps)$ for semiconcave solutions is the subject of Lemma 2 in \cite{K66d} for uniformly convex $H$.
\end{rem}

\begin{rem}\label{rateDu}
It is worth remarking that Lemma \ref{laplL1} combined with the rates $\|u_\eps-u_\eta\|_{L^\infty}\leq C(\eps-\eta)$ or $\|u_\eps-u_\eta\|_{L^\infty}\leq C(\sqrt{\eps}-\sqrt{\eta})$ obtained in Theorems \ref{rateL1} and \ref{ev1} imply the  integral   estimates
\[
\int_0^T\int_{\T^n}|D(u_\eps-u_\eta)|^2\,dxdt\leq C(\eps-\eta)\text{ or }\int_0^T\int_{\T^n}|D(u_\eps-u_\eta)|^2\,dxdt\leq C(\sqrt{\eps}-\sqrt{\eta}).
\]
This slightly improves Remark 6.9 in \cite{L82book}, being valid for possibly nonconvex $H$ (i.e. under \eqref{H1}-\eqref{H4}). Indeed, an integration by parts and Lemma \ref{laplL1} give
\begin{align*}
\int_0^T\int_{\T^n}|D(u_\eps-u_\eta)|^2\,dxdt&=-\int_0^T\int_{\T^n}(u_\eps-u_\eta)\Delta(u_\eps-u_\eta)\,dxdt\\
&\leq \|u_\eps-u_\eta\|_{L^\infty}\|\Delta(u_\eps-u_\eta)\|_{L^1}\leq C\|u_\eps-u_\eta\|_{L^\infty}.
\end{align*}
This implies
\[
\|Du_\eps-Du_\eta\|_{L^2(Q)}\leq C\sqrt{\eps-\eta}.
\]
S.N. Kruzhkov obtained in \cite{K66d} a $L^1$ rate of convergence of the gradient with order $\mathcal{O}(\eps^\frac12)$ using Gagliardo-Nirenberg inequalities and (ii) in Lemma \ref{laplL1}. A direct integration in the one-side interpolation argument of Lemma \ref{onesideinterp} or the Gagliardo-Nirenberg inequality combined with the estimate on $D^2u\in L^1(\T^n)$ shows
\[
\|Du_\eps-Du_\eta\|_{L^1(Q)}\leq C\sqrt{\eps-\eta},
\]
which extends the results in \cite{K66d} for more general Hamiltonians.
\end{rem}

\begin{rem}\label{system}
As noticed in Section 16.1 of \cite{L82book}, starting with a solution $u$ of
\[
\partial_t u+H(Du)=0\text{ in }Q,
\]
one obtains that $v=Du$ (with $v_i=u_{x_i}$) solves the hyperbolic quasilinear system
\[
\partial_t v_i+(H(v))_{x_i}=0\text{ in }Q.
\]
Existence, stability and further properties of solutions for such special systems can be thus obtained following the lines of Theorem 16.1 in \cite{L82book} or Theorem 8 of \cite{K67II}. In particular, S.N. Kruzhkov in Theorem 8 \cite{K67II} obtained the convergence of the solution of the viscous quasilinear system 
\[
\partial_t v^\eps_i+(H(v^\eps))_{x_i}=\eps\Delta v^\eps_i\text{ in }Q
\]
towards the inviscid system solved by $v_i$, which arise from a Hamilton-Jacobi equation with uniformly convex $H$. In this case, as already observed in Remark \ref{conlaws}, the semi-superharmonic condition on $u$ becomes the classical Oleinik one-side Lipschitz condition on $v$ ensuring uniqueness of entropy solutions with convex fluxes, cf. \cite{EvansBook}. \\
By means of Remark \ref{rateDu} one obtains for instance a new $\mathcal{O}(\sqrt{\eps})$ rate in $L^1$ (or even $L^2$) for the convergence of $v^\eps$ to $v$. This kind of relation between Hamilton-Jacobi equations and hyperbolic systems has been used in \cite{K67II} and also recently in the context of Mean Field Games \cite{CecchinDelarue}, while $L^1$ rates for the vanishing viscosity approximation of hyperbolic systems were considered in \cite{Bressan} for $n=1$.
\end{rem}


\subsubsection{Extensions to stationary problems}
Throughout this section we briefly discuss how to extend the previous rate of convergence results for the stationary problem
\begin{equation}\label{hj_staz}
	\lambda u(x)+H(Du(x))=f(x)\text{ in }\R^n.
\end{equation}
As before, we consider the regularized problem
\begin{equation}\label{HJstat}
-\eps\Delta u_\eps(x)+\lambda u_\eps(x)+H(Du_\eps(x))=f(x)\text{ in }\R^n.
\end{equation}
We state a model result that extends the rate of convergence for Lipschitz solutions obtained in Theorem 2.1 of \cite{Tran2011} to norms weaker than $L^\infty$. This holds for non-convex $H$.
\begin{thm}
Assume that $H,f\in W^{1,\infty}_{\mathrm{loc}}(\R^n)$. Let $u$, $u_\eps$ be a solution of \eqref{hj_staz} and, respectively, of \eqref{HJstat}. Then
\[
\|u_\eps-u\|_{L^\infty(\R^n)}\leq \frac{\sqrt{n}}{\lambda^2}\|Df\|_{L^\infty(\R^n)}\sqrt{\eps}.
\]
In addition, there exists a constant $C>0$ independent of $\eps$ such that
\[
\|u_\eps-u\|_{L^p_{\mathrm{loc}}(\R^n)}\leq C\sqrt{\epsilon}.
\]
\end{thm}
\begin{proof}
The proof can be done as in \cite{Tran2011}, using the transformation $w(x,t)=e^t u(x)$, which is a solution of the parabolic problem. One can then exploit the proof of the parabolic case and then go back to the elliptic case as in Theorem 2.7 of \cite{Tran2011}.
\end{proof}
One can formulate similar results with rate $\mathcal{O}(\varepsilon)$ up to $L^1$ for SSH solutions as it is done in the parabolic case. An example is the following
\begin{thm}
Assume that $H\in W^{1,\infty}_{\mathrm{loc}}(\R^n)$, $f$ being $L^\infty$-SSH. Assume that $u_\eps$ is $L^\infty$-SSH. Then
\[
u_\eps-u\leq \frac{1}{\lambda}\|(\Delta f)^+\|_{L^\infty(\R^n)}\eps
\]
In addition, for $p>1$ there exists a constant $C>0$ independent of $\eps$ such that
\[
\|(u_\eps-u)^+\|_{L^p_{\mathrm{loc}}(\R^n)}\leq C\|(\Delta f)^+\|_{L^\infty(\R^n)}\eps.
\]
\end{thm}


\subsection{Rate of convergence in homogenization}\label{sec;hom}
In this section, we  give an   estimate of the rate of convergence for the ergodic approximation to the effective Hamiltonian arising in homogenization of Hamilton-Jacobi equation. These estimates are used in the study of the numerical approximation of the ergodic homogenized Hamilton-Jacobi equation, cf. Section 3.1 in  \cite{acd}.\\
Given the periodic Hamiltonian $H:\mathbb{T}^n\times\R^n\to\R$, for $\delta>0$ consider   the ergodic approximation 
\begin{equation}\label{hjergodic}
	\delta v^\delta+H(x,Dv^\delta+p)= 0\qquad\text{ in }\mathbb{T}^n.
\end{equation}
Then, for $\delta\to 0$, $\delta v^\delta\to-\bar H(p)$, $v^\delta-\min_{\mathbb{T}^n}v^\delta\to v$  and $v$ is a viscosity solution of the cell problem
\begin{equation}\label{HJcell}
	H(x,Dv+p)=\bar H(p).
\end{equation}
In \cite{cdi}, via the maximum principle, and in \cite{Tran2011}, via the adjoint method but for a different approximation, it was given an estimate  for  $\|\delta v^\delta+\bar H(p)\|_{L^\infty}$ of order $O(\delta)$. This is an intermediate step towards the $\mathcal{O}(\eps^\frac13)$ rate of $\|u_\eps-u\|_\infty$ obtained in Theorem 1.1 of \cite{cdi}.  Exploiting the result of the previous section, we prove a similar  intermediate estimate in $L^r$-norm that retrieves Lemma 2.3-(d) of \cite{cdi} for $p=\infty$.
\begin{prop}
	Assume that $H\in W^{1,\infty}_{\mathrm{loc}}(\R^n)$,   coercive in $p$ uniformly with respect to $x\in \T^n$ and $|H(x,0)|\le C_1$ for $x\in\T^n$.

	Let $v^\eps$ be the solution of 
	\begin{equation}\label{hjergvisco}
		-\eps \Delta v^\eps+ H(x, Dv^\eps+p)+\eps v^\eps=0 \qquad \text{ in }\mathbb{T}^n.
	\end{equation}
	Then, for $r\in [1,\infty]$, we have
	\begin{equation}\label{stima}
		\|\eps v^\eps+\bar H(p)\|_{L^r(\mathbb{T}^n)}\le C(1+|p|)\eps.
	\end{equation}
	where $C$ depends only on the Hamiltonian $H$. 
\end{prop}
\begin{proof}
	Let $v^\eps$ be the solution of \eqref{hjergvisco}, $v$ a solution of \eqref{HJcell}  and set $w^\eps=\eps v^\eps+\bar H(p)$. Subtracting \eqref{HJcell} from \eqref{hjergvisco}, we get
	\[-\Delta w^\eps+ H(x,Dv^\eps + p )-H(x, p)+w^\eps=H(x, Dv+p)- H(x,p)\]
	and therefore
	\begin{equation}\label{HJcelleps}
		-\eps \Delta w^\eps+b(x) \cdot Dw^\eps+\eps w^\eps=\eps[H(x, Dv+p)- H(x,p)], 
	\end{equation} 
	where $b(x)=\int_0^1 D_pH(x, s(Dv^\eps+p)+(1-s)p)ds$. Let $\rho_\eps$ be the solution of 
	\[-\eps\Delta\rho_\eps-\mathrm{div}(b(x)\rho_\eps)+\eps\rho_\eps=\psi(x)\]
	with $\psi\in L^{r'}(\mathbb{T}^n)$. Then, multiplying \eqref{HJcelleps} by $\rho_\eps$ and integrating by parts, we get
	\begin{equation}\label{stimahom1}
		\int_{\mathbb{T}^n}w_\eps \psi dx\le \eps \|H(x,Dv+p)-H(x,p)\|_{L^\infty}\int_{\mathbb{T}^n}\rho_\eps dx\leq C\eps \|Dv\|_{L^\infty(\T^n)}.%(1+|p|).
	\end{equation}
Take $\psi(x)=|w_\eps|^{r-1}\mathrm{sgn}(w_\eps)/\|w_\eps\|^{r-1}_{L^r}$ and observe that
$\|\psi\|_{L^{r'}}\le 1$. Moreover recall that, by \cite[Lemma 2.3-(b)]{cdi}, $\|Dv\|_{L^\infty}\le C(1+|p|)$. Using the previous estimate in \eqref{stimahom1}, we get \eqref{stima}.
\end{proof}
Let now $u^\eps$ be the solution of the oscillating problem 
\begin{equation}\label{HJoscill}
	H\left(\frac{x}{\eps} , Du^\eps\right)+	u^\eps(x)=0\qquad \text{ in }\mathbb{T}^n.
\end{equation}
Following \cite[Theorem 1.2]{cdi}, we also give a  $L^r$-estimate of the rate of convergence of $u^\eps$ to $u$, where $u$ is the solution of the homogenized equation  
\begin{equation}\label{HJhom}
	\bar H(Du)+u(x)=0\qquad \text{ in }\mathbb{T}^n.
\end{equation}
\begin{thm}\label{homsimple}
	Let $u^\eps, u$ be, respectively, the solutions of \eqref{HJoscill} and \eqref{HJhom}. Then,  for $r\in [1,\infty]$ we have
	\begin{equation}\label{estimatehom}
		\|u^\eps-u\|_{L^r(\mathbb{T}^n)}\le 2K\eps.
	\end{equation}
	for $K$ independent of $\eps\in(0,1)$.
\end{thm}
\proof
	First observe that the solution of \eqref{HJhom}	is given by $u(x)=-\bar H(0)$. Let $v$ be the solution of the cell problem \eqref{HJcell} with $p=0$. Then $\bar u^\eps(x)=u(x)+\eps v(\frac{x}{\eps})$ satisfies
	\[  H\left(\frac{x}{\eps}, D\bar u^\eps\right)+\bar u^\eps=\eps v\left(\frac{x}{\eps}\right).\]
Let us so consider the function $w^{\eps} = u^{\eps} - \bar{u}^{\eps}$ that solves the equation 
	\begin{equation*}
	w^{\eps} + H\left(\frac{x}{\eps}, Du^{\eps}\right) - H\left(\frac{x}{\eps}, D\bar{u}^{\eps}\right) = \eps v\left(\frac{x}{\eps}\right).
	\end{equation*}
	in the viscosity sense. In order to get  an  estimate for $w^{\eps}$ we proceed by approximation considering 
	\begin{equation}\label{sigma}
	-\sigma \Delta w^{\eps}_{\sigma} + b(x)\cdot  Dw^{\eps}_{\sigma}+w^{\eps}_\sigma = \eps v\left(\frac{x}{\eps}\right)
	\end{equation}
	where 
	\begin{equation*}
	b(x) = \int_{0}^{1} D_pH\left(\frac{x}{\eps}, sDu^{\eps} + (1-s)D\bar{u}^{\eps}\right)\;ds. 
	\end{equation*}
	Let $\rho_\sigma$ be the solution of 
	\[-\sigma\Delta\rho_\sigma-\mathrm{div}(b(x)\rho_\sigma)+\rho_\sigma=\psi(x)\]
	with $\psi(x)=|w^{\eps}_{\sigma}|^{r-1}\mathrm{sgn}(w^{\eps}_{\sigma})/\|w^{\eps}_{\sigma}\|^{r-1}_{L^r}$, such that $\|\psi\|_{L^{r'}} \leq 1$, $r>1$. So, multiplying \eqref{sigma} by $\rho_{\sigma}$ we get
	\begin{equation*}
	\|w^{\eps}_{\sigma} \|_{L^r(\T^n)} \leq \eps \|v\|_{\infty}.
	\end{equation*}
	Since the constant on the right-hand side is independent of $\sigma$ and letting $\sigma \downarrow 0$ we obtain $\|w^{\eps} \|_{L^r} \leq \eps \|v\|_{\infty}$ and therefore
	\begin{equation*}
	\|u^\eps-u\|_{L^r}\le \|u^\eps-\bar u^\eps\|_{L^r} +\eps \|v\|_{L^r} \le 2\eps \|v\|_{\infty}. \eqno\square
	\end{equation*}
\begin{rem}
If we consider the parabolic problem
\[
\partial_t u^\eps+H\left(\frac{x}{\eps} , Du^\eps\right)=0\text{ in }\T^n\times(0,\infty),
\]
and the homogenized equation
\[
\partial_t u+\bar H(Du)=0\text{ in }\T^n\times(0,\infty),
\]
equipped with an affine initial condition of the form $u_0(x)=\alpha+p\cdot x,\ \alpha\in\R,\ p\in\R^n$, the solution of the homogenized parabolic problem is given by $u(x,t)=\alpha+p\cdot x-t\overline{H}(p)$, cf. Section I.2 of \cite{LPV}. To obtain and estimate of $u^\eps-u$, we first consider the formal asymptotic expansion given by
\[
\bar u^{\eps}(x,t)=u(x,t)+\eps v\left(\frac{x}{\eps}\right),
\]
and, as in \cite{LPV}, one finds that $\bar u^{\eps}$ solves in the viscosity sense
\[
\partial_t \bar u^\eps+H\left(\frac{x}{\eps},D\bar u^\eps\right)=0.
\]
The approach of Theorem \ref{homsimple} finally leads to the following formal estimate
\[
\|u^\eps-u\|_{L^r(\T^n)}\leq 2\eps\|v\|_\infty,
\]
where $v$ is the solution of the cell problem \eqref{HJcell}, noting that $w^\eps_\sigma=u^\eps-\bar u^\eps$ now solves the viscous approximation
\[
\begin{cases}
\partial_t w^\eps_\sigma+H\left(\frac{x}{\eps} , Du^\eps\right)-H\left(\frac{x}{\eps},D\bar u^\eps\right)=\sigma \Delta w^\eps_\sigma&\text{ in }\T^n\times(0,\infty),\\
w^\eps_\sigma(0)=\eps v\left(\frac{x}{\eps}\right)&\text{ in }\T^n.
\end{cases}
\]
\end{rem}
%%%%%%%%%%%%%%%%
%              %
%%%%%%%%%%%%%%%%
\subsection{Rate of convergence for numerical methods: the Godunov scheme}\label{sec;num}
In this part, we exploit the results of the previous sections in order to obtain 
a $L^1$ rate of convergence for   Godunov-type  approximation schemes for
Hamilton-Jacobi equations. This part improves the $L^1$ estimate obtained   in \cite[Theorem 2.3]{LinTadmor} where the same rate was obtained for uniformly convex Hamiltonians. \\
For simplicity, we consider  Hamilton-Jacobi equation of the type
\begin{equation}\label{num:HJ}
	\left\{	\begin{array}{ll}
		\partial_tu+H(Du)=0\qquad& \text{in}\,\mathbb{T}^n\times(0,\infty),\\
		u(x,0)=u_0(x)&\text{in}\,\mathbb{T}^n.
	\end{array}
	\right.	
\end{equation}
with $n=2$. 
We  fix a time grid $t^n=n\dt$, $n\in \mathbb{N}$,  and a  rectangular grid of cells of size $\D=\dx\times \dy$ which satisfies the non degeneracy condition
$0<c_0\le \dx/\dy\le C_0$ and  the CFL condition
$L_H \dt/\max\{\D x,\D y\}	 <1/4$ where $L_H$ is the Lipschitz constant of $H$.
A Godunov scheme reads a
\begin{equation}\label{num:Godunov}
	u^\D(\cdot,t)=
	\begin{cases}
		E(t-t^{n-1})u^\D(\cdot,t^{n-1})\quad& t\in(t^{n-1}, t^n)\\
		P^\D u^\D(\cdot, t^{n,-})&t=t^n
	\end{cases}
	\qquad n=1,2,\dots
\end{equation}
with $u^\D(\cdot, 0)= P^\D u_0(\cdot)$, 
where $E(\cdot)$ is the exact solution operator associated to the Hamilton-Jacobi equation \eqref{num:HJ}, $P^\D$ a projection operator on the grid and $u^\D(\cdot, t^{n,-})=E(t^n-t^{n-1})u^\D(\cdot,t^{n-1})$.\\
We need two preliminary results. The first one gives an estimate of the truncation error in terms of the $L^1$-norm of the error introduced by the projection operator $P^\D$
(see \cite[Lemma 2.1]{LinTadmor}).
%%%%
\begin{lemma}
	Let $u^\D$ be a family of   functions given by the scheme \eqref{num:Godunov}. Then
	\begin{equation}\label{num:trunc_error}
		\|\partial_tu^\D +H(Du^\D)\|_{L^1_x}\le\frac{T}{\dt}
		\max_{0<t^n<T}\|(I-P^\D)u^\D(\cdot, t^{n,-})\|_{L^1_x}
	\end{equation}	
	($I$ denotes the identity operator).
\end{lemma}
The second lemma is a stability result that can be proved in the same way of the $L^1$ estimate in Theorem \ref{rateL1} via the adjoint method.
\begin{lemma}\label{stab}
Assume that $H$ satisfies \eqref{H1}-\eqref{H4}. For $i=1,2$, let $u_i$  be $L^1_t(L^\infty_x)$-semiconcave  solution  to
\begin{equation}\label{stability}
	\begin{cases}
		\partial_t u_i+H(Du_i)=f_i&\text{ in }\T^n\times (0,T)\\
		u_i(x,0)=u^i_0(x)&\text{ in }\T^n.
	\end{cases}
\end{equation}
Then 
\begin{equation}\label{God_est1}
\|(u_1-u_2)(t)\|_{L^1_x}\leq C (\|(u^1_0-u^2_0\|_{L^1_x}+\|f_1-f_2\|_{L^1_t(L^1_x)}),\quad t\in (0,T).
\end{equation}
\end{lemma}
Given $w:\mathbb{T}^2\times [0,T]\to\R$, $\xi\in\mathbb{R}^2$ with $|\xi|=1$ and $h>0$, we define second-order finite difference operator
\[
D^2_{h,\xi}w(x,t)=\frac{w(x+h\xi,t)+w(x-h\xi,t)-2w(x,t)}{h^2}
\]
and the norm 
\[
\normdiscrete{w(t)}:=\sup_{h>0,|\xi|=1}\|D^2_{h,\xi}w(x,t)\|_{L^1_x}.
\]
A family $\{\psi^\D\}$, $\D>0$, is said to be uniformly semiconcave in $L^1_t(L^\infty_x)$ if it  satisfies Definition \ref{semicmix} with the same function $k\in L^1_t(L^\infty_x)$ for any $\D$.
In the following result, we give an abstract $L^1$ estimate for the rate of convergence of Godunov schemes. 
\begin{thm}\label{num:theo_est}
Assume that $H$ satisfies \eqref{H1}-\eqref{H4}.	Let $u^\D$ be a family of  functions given by the scheme \eqref{num:Godunov} and assume that
	\begin{itemize}
		\item[(i)] For any $t>0$,
		\begin{equation}\label{num:consistent}
			\|(I-P^\D)u^\D(t)\|_{L^1_x}\le C\D^2 \normdiscrete{u^\D(t)},\quad t\in (0,T) .
		\end{equation}
		\item[(ii)] There exists a family of functions $\psi^\D$  such that
		\begin{align}
			&\text{$\psi^\D$ is  uniformly  semiconcave in $L^1_t(L^\infty_x)$ for $\D>0$} \label{num:nearby_1},\\
			&\|\partial_t(u^\D(t)-\psi^\D(t))\|_{L^1_x}+\|D(u^\D(t)-\psi^\D(t))\|_{L^1_x}\le C\D  \normdiscrete{\psi^\D(t)}.\label{num:nearby_2}
		\end{align}
		Then $u^\D$ converges to the viscosity solution $u$ of \eqref{num:HJ} and 
		\begin{equation}\label{num:est_Lp}
			\|u(t) -u^\D(t)\|_{L^1_{x}}\le C\D, \qquad t\in [0,T]
		\end{equation}
	\end{itemize}
\end{thm}
\begin{proof}
	By  \eqref{num:nearby_1}, it follows that	$\normdiscrete{\psi^\D(t)}\le C$, $t\in [0,T]$,  and by \eqref{num:nearby_2}	
	\begin{align*}
		&\|u( t)-u^\D(t)\|_{L^1_x}	\le \|u( t)-\psi^\D(t)\|_{L^1_x}+ \|\psi^\D(t)-u^\D( t)\|_{L^1_x}\\
		&\le \|u(  t)-\psi^\D( t)\|_{L^1_x}+  C\D  
	\end{align*}
		By Lemma \ref{stab} and \eqref{num:nearby_1}, we have for $t\in [0,T]$  
		\begin{equation}\label{num:proof1}
			\|u(  t)-\psi^\D( t)\|_{L^1_x}\le C\left(	\|u( 0)-\psi^\D(0)\|_{L^1_x}+\|\partial_t\psi^\D+H(D\psi^\D)\|_{L^1_x}\right).
	\end{equation}
%	By \cite[Cor. 2.4]{lin-tadmor}, we have $\normdiscrete{u^\Delta(\cdot,t)}\le \normdiscrete{u^\Delta(\cdot,0)}$ for $t\in [0,T]$. 
	By \eqref{num:consistent} and \eqref{num:nearby_2}, we   estimate 
	\begin{equation}\label{num:proof2}
		\begin{split}
			\|u( 0)-\psi^\D( 0)\|_{L^1_x}&\le \|u( 0)-u^\D( 0)\|_{L^1_x}+\|u^\D( 0)-\psi^\D(0)\|_{L^1_x}\le  C\D 	.
		\end{split}
	\end{equation}
	Moreover, by \eqref{num:trunc_error}, \eqref{num:consistent} and \eqref{num:nearby_2} we have 
	\begin{equation}\label{num:proof3}
		\begin{split}
			\|\partial_t\psi^\D +H(D\psi^\D)\|_{L^1_x}&\le \|\partial_tu^\D +H(Du^\D)\|_{L^1_x}+	\|\partial_t(u^\D-\psi^\D)\|_{L^1_x}\\
			&+L_H\|D(u^\D-\psi^\D)\|_{L^1_x}\le C\D
		\end{split}
	\end{equation}
	Replacing \eqref{num:proof2} and \eqref{num:proof3} in \eqref{num:proof1}, we get
	\eqref{num:est_Lp}.
\end{proof}
Note that, for the estimate \eqref{num:est_Lp}, the family of approximate solutions $\{u^\D\}$ is not required to satisfy a  uniform semiconcavity estimate, as it fails in general for numerical schemes. Instead, it is sufficient to find a family $\{\psi^\D\}$, uniformly semiconcave in $\D$, which is ``close" in a certain sense to that given by the Godunov scheme.
In this context, the key property is the following discrete semiconcavity  introduced in \cite{LinTadmor}: a family ${u^\D}$, $\D>0$, is said semiconcave stable if there exists $k(t)\in L^1(0,T)$ such that for all $h\ge h_0(\D)>0$, there holds
\[
D^2_{h,\xi}u(x,t)\le k(t)\qquad \forall |\xi|=1.
\]
The previous property allows the construction of a family $\{\psi^\D\}$ satisfying   \eqref{num:nearby_1}-\eqref{num:nearby_2} for several examples of Godunov schemes as described in \cite[Section 3]{LinTadmor} to which we refer for further details.









%%%%%%%%%%%%%%%%%%%%
%  Applications    %
%%%%%%%%%%%%%%%%%%%%



\section{Qualitative properties for Hamilton-Jacobi equations}
\subsection{Regularizing effects for Cauchy problems in the whole space $\R^n$}\label{sec;regeff}
We consider in this section Lipschitz regularizing effects for  the Hamilton-Jacobi equation
\begin{equation}\label{hjregularizing}
\begin{cases}
\partial_t u-\eps \Delta u_\eps+H(Du_\eps)=0&\text{ in }\R^n\times(0,\infty),\\
u(x,0)=u_0(x)&\text{ in }\R^n,
\end{cases}
\end{equation}
namely we inquire whether the solution of the Cauchy problem for  the Hamilton-Jacobi equation is smoother, as time evolves, than the initial condition $u_0$. We prove the following
\begin{thm}\label{regLip}
Let $u_0$ be bounded and assume that $H$ satisfies (H4) with $\widetilde{C}_{H,4}=0$, $\gamma\leq2$ and $f\equiv 0$. Let $u_\eps$ be any bounded classical solution to \eqref{hjregularizing}. Then, for any $\delta>0$ we have $u_\eps\in W^{1,\infty}(\R^n\times(\delta,T))$ and, in particular, the following a priori estimate holds
\[
\|Du(t)\|_{L^\infty(\R^n)}\leq C\|u\|_\infty^{\frac{1}{\gamma}}t^{-\frac1\gamma}
\]
for a positive constant $C>0$.
\end{thm}
\begin{proof}
	We drop the subscript $\eps$ for simplicity of notation. By Thereom \ref{semic>0}, we have for all $\tau\in(0,\infty)$
	\[
	u_{\xi\xi}\leq \frac{C(\delta +\|Du\|_{\infty}^2)^\frac{2-\gamma}{2}}{\tau}=:C_0(\tau).
	\]
	Lemma \ref{onesideinterp} implies for a.e. $t\in(0,T)$
	\[
	\|Du(t)\|_{L^\infty(\R^n)}\leq (4\|u(t)\|_{L^\infty(\R^n)}C_0(t))^\frac12=\sqrt{4C}\|u\|_\infty^\frac12(\delta +\|Du\|_{\infty}^2)^\frac{2-\gamma}{4}t^{-\frac12}.
	\] 
	Rearranging the terms this implies
	\[
	\|Du(t)\|_{L^\infty(\R^n)}\leq 2^\frac2\gamma C^\frac{1}{\gamma}\|u\|_\infty^{\frac{1}{\gamma}}t^{-\frac1\gamma}
	\]
	and concludes the proof.
\end{proof}
\begin{rem}
The previous result applies, for instance, to convex Hamiltonians and, in particular, to those having power-like growth. The method of proof combines a semiconcavity estimate for positive times and a one-side interpolation result due to P.-L. Lions, see Proposition 13.1 in \cite{L82book} and Lemma \ref{onesideinterp} below. However, the result in \cite{L82book} requires a uniformly convex Hamiltonian depending only on $p$. Related results have been obtained in \cite{Lions85aa} and also in \cite{Laurencot} for first-order equations with power-like gradient dependent terms, under the general assumption (corresponding roughly speaking to \eqref{H1})
\[
D_pH(p)\cdot p-H(p)\to+\infty\text{ as }|p|\to+\infty,
\]
using different Bernstein arguments and without relying on second-order estimates. Here, we mix the hypotheses from \cite{Lions85aa} and the approach in \cite{L82book} via the new semiconcavity estimate in Theorem \ref{semic>0}. This gives a shorter and new proof of most of the aforementioned results.
\end{rem}

\begin{rem}
The above result transfers to first-order equation as studied in \cite{Laurencot}, and the time-decay gradient estimate is sharp in view of Remark (iii) p. 286 of \cite{Lions85aa}. 
\end{rem}

\begin{rem}\label{Nwave}
Recalling the relation among Hamilton-Jacobi equations and conservation laws, one can exploit the one side interpolation inequality due to E. Tadmor \cite{LinTadmor}, along with the Oleinik one side-Lipschitz condition from Remark \ref{conlaws}, to conclude for the solution $u$ of the conservation law
\[
\partial_t u+(F_i(u))_{x_i}=0
\]
the time-decay in sup-norm for large times due to R. Diperna
\[
\|u(\cdot,t)\|_{\infty}\lesssim t^{-\frac12},
\]
see e.g. Theorem 5 and 6 in Section 3.4 of \cite{EvansBook}, and the references therein.
\end{rem}

\subsection{Liouville-type theorems for first-order Hamilton-Jacobi equations}\label{sec;lio}
In this section we establish some non-existence properties for generalized (ancient) solutions to the parabolic problem
\begin{equation}\label{ancient}
\partial_t u+H(Du)=0\text{ in }\R^n\times(-\infty,0),
\end{equation}
when $H$ satisfies (H4) with $\gamma\in(1,2]$.  Our main result is inspired from the Liouville type property appeared in Section 1 of \cite{K67II}, and extends it to more general Hamiltonians. These results are rather unnatural due to the absence of diffusive terms in the equation. We may assume without loss of generality that $H(0)=0$, otherwise replace $u$ with $\bar u$ solving
\[
\bar u=u-tH(0)
\]
The result reads as follows:
\begin{thm}\label{lio}
Let $u$ be a solution of \eqref{ancient} with $H(0)=0$ and satisfying the one-side decay condition
\[
u(x,t)\geq -|x|\mu(|x|)+K(t),\ \mu(r)\geq0, \mu(r)\to0\text{ as }r\to\infty,
\]
for a bounded function $K$. Then $u$ must be constant.
\end{thm}
\begin{rem}
The previous decay condition is satisfied if $u$ is bounded from below or has   sublinear decay at infinity. In such a case $\mu(r)=r^{\alpha-1}$, $r=|x|$, $\alpha\in(0,1)$, satisfies the condition $\mu(r)\to0\text{ as }r\to\infty$ with $\mu(r)\geq0$.
\end{rem}
To prove the vanishing property, we premise the following lemma of convex analysis taken from Lemma 2 in Section 1 of \cite{K67II}. It extends the classical result saying that a concave function bounded from below must be constant (cf. Lemma \ref{onesideinterp} with $C_0=0$), allowing for a more general unilateral decay condition.
\begin{lemma}
Let $u:\R\to\R$ be a concave function. If 
\[
u(s)\geq -|s|\mu(|s|)+C,\ \mu(r)\geq0,\ \mu(r)\geq0,\ \mu(r)\to0\text{ as }r\to\infty
\]
then $u$ is constant.
\end{lemma}
\begin{proof}[Proof of Theorem \ref{lio}]
The proof relies on showing that in the long-time regime the solution of \eqref{ancient} becomes concave. We can view a solution to \eqref{ancient} as a solution of the problem on the layer $\{-T\leq t\leq0\}$ by letting $T\to\infty$ with initial condition $u(-T)=u(x,-T)$. By Theorem \ref{semic>0} we have
\[
D^2u\xi\cdot \xi\leq \frac{C}{T+t}.
\]
We fix $x_0,t_0$ and restrict $u$ on the line $u(x_0,t_0+\xi s)$. Letting $T\to\infty$ in the second-order estimate we conclude that $u$ is a concave function satisfying the growth conditions of the lemma. This implies that $u$ must be constant in space, i.e. $u(x,t)=k(t)$. Then, by the equation we have
\[
\partial_t u=-H(Dk(t))=-H(0)=0
\]
which implies that $u$ is also constant in the time-variable.
\end{proof}
\begin{rem}
The Liouville-type result for the evolutive problem \eqref{ancient} is determined by the nonlinearity. Indeed, even for the simplest heat equation $\partial_t u-u_{xx}=0$ in $\R^2$ or $\R\times(-\infty,0]$ the Liouville property does not hold for solutions satisfying only one-side bounds: the function $u(x,t)=e^{x+t}$ is bounded from below and it is not a constant.\\
Furthermore, Remark 2 in \cite{K67II} shows that the lower bound on $u$ cannot be replaced by an upper bound (e.g. $u\leq0$). Other polynomial Liouville theorems can be obtained following the lines of \cite{K67II}.
\end{rem}
\begin{rem}
Since the semiconcavity estimates of Theorem \ref{semic>0} are independent of $\eps>0$, one can prove with the same proof of Theorem \ref{lio} a Liouville theorem for ancient solutions to the following model viscous problem
\[
\partial_t u-\Delta u+|Du|^\gamma=0\quad \text{ in }\R^n\times(-\infty,0).
\]
P. Souplet and Q.S. Zhang \cite[Theorem 3.3]{SZ} proved by the Bernstein method that any classical solution to the above equation such that $|u(x,t)|=o(|x|+|t|^\frac1\gamma)$, $\gamma\in(1,2]$, as $|x|+|t|^\frac1\gamma\to\infty$, must be a constant. In particular, any bounded solution to the above equation is a constant. Our result, instead, requires only a one-side condition in space, but asks an a priori sublinear decay in $x$. Clearly, if $u$ is bounded, Theorem \ref{lio} leads to the same conclusion as that in \cite[Theorem 3.3]{SZ}.
\end{rem}


\appendix
\section{Auxiliary results from functional analysis}
Let $X,Y$ be two real Banach spaces. The couple of Banach space $(X,Y)$ is said to be an interpolation couple if both $X$ and $Y$ are continuously embedded into a Hausdorff topological vector space $\mathcal{V}$. Therefore, $X\cap Y$ is a linear subspace of $\mathcal{V}$ and it is a Banach space under the norm
\[
\|u\|_{X\cap Y}:=\max\{\|u\|_X,\|u\|_Y\}.
\]
Moreover, the sum $X+Y:=\{x+y:x\in X,y\in Y\}$ is a linear subspace of $\mathcal{V}$ and it is a Banach space equipped with the norm
\[
\|u\|_{X+ Y}:=\inf_{x\in X,y\in Y,x+y=u}\|x\|_X+\|y\|_Y.
\] 
The pair $(L^{p_0}(\R^n),L^{p_1}(\R^n))$, $1\leq p_0,p_1\leq\infty$ is an interpolation couple as a consequence of Theorem I.1.4 of \cite{BennetSharpley}. We denote by $X'$ the dual space of $X$. We have the following
\begin{lemma}\label{emb+}
The dual of the space $X\cap Y$, denoted with $(X\cap Y)'$, is isomorphic to the space $X'+Y'$. Conversely, $(X+Y)'=X'\cap Y'$ isometrically. In particular, the dual of $L^1(\R^n)\cap L^{p'}(\R^n)$ is isomorphic to $L^\infty(\R^n)+L^p(\R^n)$, and we have the embedding
\[
L^\infty(\R^n)+L^p(\R^n)\hookrightarrow L^p_{\mathrm{loc}}(\R^n).
\]
\end{lemma}
\begin{proof}
The result can be found in Theorem 2.7.1 of \cite{BerghLofstrom}, see also properties 2-3-6 p.175 in \cite{BennetSharpley}. The proof of the last embedding follows from the definition of the norms. 
\end{proof}
The next is a one-side interpolation result of Gagliardo-Nirenberg type in the endpoint case.
\begin{lemma}\label{onesideinterp} Let $u:Q\to\R$ be bounded and such that $u\in C^2(Q)$ and for a.e. $t\in (0,T)$
\[
D^2u(t)\leq C_0(t)I_n\text{ in }D'(\R^n).
\]
Then, we have for all $t>0$
\[
\|Du(t)\|_{L^\infty(\R^n)}\leq (2\|u(t)\|_{L^\infty(\R^n)}C_0(t))^\frac12.
\]
\end{lemma}
\begin{proof}
By a Taylor expansion around a point $x$ we get
\[
-\|u\|_\infty\leq u(x+h)\leq u(x)+Du(x)\cdot h+\frac{C_0}{2}|h|^2, h\in\R^n.
\]
For fixed $x$, define $p(h)=u(x)+Du(x)\cdot h+\frac{C_0}{2}|h|^2$, which attains its minimum at $h_m=-\frac{1}{C_0}Du(x)$. Hence,
\[
p(h_m)=u(x)-\frac{1}{2C_0}|Du(x)|^2\geq -\|u\|_\infty,
\]
which implies the desired inequality.
\end{proof}
\section{Well-posedness and stability estimates for equations with divergence-type terms}
\subsection{Some useful properties of advection-diffusion equations with bounded drifts}
We consider here some preliminary properties of the Fokker-Planck equation
\begin{equation}\label{adjoint}
	\begin{cases}
		-\partial_{t} \rho - \eps \Delta \rho +\textrm{div} (b(x,t) \rho) = 0, &\text{ in }\R^n\times(0,\tau),
		\\
		\rho(x, \tau) = \rho_{\tau}(x), &\text{ in }\R^n.
	\end{cases}
\end{equation}
We have the following
\begin{lemma}\label{well}
Assume $b\in L^\infty_{\mathrm{loc}}(\R^n\times(0,\tau))$, $|b|\in L^{k}(\rho\,dxdt)$, $k>1$, $\eps>0$. Then there exists a weak (energy) solution of \eqref{adjoint} which satisfies $\rho\in L^\infty([0,\tau];L^1(\R^n))$. Moreover, if $\|\rho_\tau\|_1=1$ and $\rho_\tau\geq0$, we have $\|\rho(t)\|_1=1$ for all $t\in[0,\tau)$ and $\rho\geq0$ in $Q$.
\end{lemma}
\begin{proof}
The existence and uniqueness of a weak solution satisfying $\rho\in L^\infty(0,\tau;L^1(\R^n))$ follows from Theorem 3.7 of \cite{PorrettaUMI}. We prove the conservation of mass by a cut-off argument. Let $\chi \in C^{\infty}_{c}(\R^n)$ be such that $\chi(x) = 1$ for any $x \in B_1$ and $\chi(x) = 0$ for any $x \in \R^n\backslash B_2$. Then, setting $\chi_R(x) = \chi\left(\frac{x}{R}\right)$ and testing the above equation against such a function $\chi_R$ we get
\[
\int_{\R^n}\rho(t)\chi_R(x)\,dx+\iint_{Q}\eps\rho\Delta \chi_R(x)+\rho b\cdot D\chi_R(x)\,dxdt=\int_{\R^n}\rho(\tau)\chi_R(x)\,dx.
\]
We thus have
\[
\iint_{Q}\rho\Delta \chi_R(x)\,dxdt\leq \frac{1}{R^2}\iint_Q\rho\Delta\chi\,dxdt\leq \frac{C}{R^2}
\]
and
\[
\iint_{Q}\rho b\cdot D\chi_R(x)\,dxdt\leq \frac{1}{R}\iint_{Q}|b|\rho|D\chi|\,dxdt\leq \frac{C}{R}\left(\iint_Q|b|^k\rho\,dxdt\right)^\frac{1}{k}\left(\iint_Q\rho\,dxdt\right)^\frac{1}{k'}. 
\]
The conservation of the $L^1$ norm follows then from the dominated convergence theorem and the above estimates.
\end{proof}
\begin{rem}
When $b=-D_pH(Du_\eps)\in L^\infty_{\mathrm{loc}}(\R^n\times(0,\tau))$, where $u_\eps$ solves \eqref{hjintro}, the condition $|b|\in L^{k}(\rho\,dxdt)$ with $k=\gamma'$ is satisfied under the assumption \eqref{H1}. This condition is verified in view of Lemma \ref{cross}.
\end{rem}
\begin{rem}
If $b\in L^\infty(\R^n\times(0,\tau))$ the further integrability condition $|b|\in L^{k}(\rho\,dxdt)$ is no longer needed to prove the preservation of $L^1$ norms. \end{rem}

\subsection{$L^r$ stability estimates}
In this section we consider the Cauchy problem
\begin{equation}\label{fp}
\begin{cases}
\partial_t \rho-\partial_{x_ix_j}(a_{ij}(x,t)\rho)+\mathrm{div}(b(x,t)\rho)=0&\text{ in }Q\\
\rho(x,0)=\rho_0(x)&\text{ in }\R^n.
\end{cases}
\end{equation}
when $\mathrm{div}(b)$ is not divergence-free, and find sufficient conditions that guarantee the validity of $L^p$-stability estimates. 

The next result concerns the degenerate case, cf. Theorem 6.7.4 of \cite{BKRS}.
\begin{thm}\label{Krylov}
Let $q>1$, $b\in L^q(0,T;L^q_{\mathrm{loc}}(\R^n))$, $a_{ij}\in L^\infty(0,T;W^{1,q}_{\mathrm{loc}}(\R^n))$ (no strict parabolicity is needed). Let $r=\frac{q}{q-1}$ and suppose that
\begin{equation}
\left[(r-1)\left(\mathrm{div}(b)-\sum_{i,j}\partial_{x_ix_j}a_{ij}\right)\right]^-\in L^1(0,T;L^\infty(\R^n))
\end{equation}
along with $\rho_0\in L^r(\R^n)$. Then, there exists a solution $\rho$ of \eqref{fp} in $L^\infty(0,T;L^r(\R^n))$ and it holds the estimate.
\begin{equation*}
\sup_{t \in [0,T]} \int_{\R^n} |\rho(t, x)|^r\ dx \leq\|\rho_0\|_{L^r(\R^n)}e^{t\|(r-1)[\mathrm{div}(b)-\sum_{i,j}\partial_{ij}a_{ij}]^-\|_{L^1(0,T;L^\infty(\R^n))}}. 
\end{equation*}
\end{thm}

\proof

Assume that the data of the equation are locally smooth function, then the conclusion follows by an approximation argument from the local integrability. Moreover, in the following we use the summation index notation to simplify the presentation. 

First, for $k \in \N$ let $\varphi_k \in C^{\infty}_{c}(\R^n)$ be such that $\varphi_k \to \rho_0$ in $L^r(\R^d)$.  Note that, such a sequence exists: it is enough to set $\varphi_k = \rho_0 \star \Gamma(\cdot, 1/k)(x)$ where $\Gamma(x, t)$ is the fundamental solution of the heat equation on $\R^n \times (0, \infty)$. Then, it is not difficult to show that  $\varphi_k \to \rho_0$ in $L^r(\R^d)$. Let $\psi \in C^{\infty}_{c}(\R^d)$ be such that $\psi(x) = 1$ for any $|x| \leq 1$, $0 \leq \psi(x) \leq 1$ and set $\psi_k(x)= \psi(\frac{x}{k})$. So, consider the approximating Cauchy problem of \eqref{fp}
\begin{equation}\label{Cau}
\begin{cases}
\partial_t u_k = L^*_k u_k
\\
u_k(0, x) = \varphi_k(x)
\end{cases}
\end{equation}
where, for simplicity of notation, we set the operator
\begin{equation*}
L_k = (\psi_k a_{ij} + \frac{1}{k}\delta_{ij}) \partial_{x_ix_j} + \psi_k b_i\partial_{x_i} + \frac{r-1}{r} \big((2\partial_{x_j} a_{ij} - b_i)\partial_{x_i} \psi_k + a_{ij} \partial_{x_ix_j}\psi_k \big).
\end{equation*}
Notice that the operator $L^*_k$ is obtained from \eqref{fp} by multiplying it by $\psi_k$, while $L_k$ is the formal adjoint of $L^*_k$ that has been also regularized by a viscosity $\frac1k\delta_{ij}$.
By \cite{PorrettaUMI} we have that there exists a bounded solution $u_k \in L^{\infty}([0,T]; L^{1}(\R^d))$. Moreover, multiplying \eqref{Cau} by $|u_k|^{r-2}u_k$ and integrating by parts we obtain that the following inequaltiy is satisfied
\begin{equation*}
\partial_t |u_k|^{r} \leq \partial_{x_ix_j}\big(|u_k|^r(\psi_k a_{ij} + \frac{1}{k}\delta_{ij}) \big) - \partial_{x_i}(\psi_k b |u_k|^r) + \psi_k (r-1) \partial_{x_i} (\partial_{x_j} a_{ij} - b_i)|u_k|^r. 
\end{equation*}
Thus, by Gronwall's inequality we infer
\begin{equation*}
\sup_{t \in [0,T]} \int_{\R^n} |u_k|^r\ dx \leq \|\varphi_k\|_{L^r(\R^n)}e^{t\|(r-1)[\mathrm{div}(b)-\frac\sigma2\sum_{i,j}\partial_{ij}a_{ij}]^-\|_{L^1(0,T;L^\infty(\R^n))}}
\end{equation*}
and the constant on the right-hand side is independent of $k \in \N$. This yields the existence of subsequence, still denoted by $u_k$, such that $u_k \to \rho$ with $\rho \in L^{\infty}([0,T]; L^r(\R^d))$ and the following estimate holds
\begin{equation*}
\sup_{t \in [0,T]} \int_{\R^n} |\rho(t, x)|^r\ dx \leq \|\rho_0\|_{L^r(\R^n)}e^{t\|(r-1)[\mathrm{div}(b)-\frac\sigma2\sum_{i,j}\partial_{x_ix_j}a_{ij}]^-\|_{L^1(0,T;L^\infty(\R^n))}}. \eqno\square
\end{equation*} 

\par\bigskip

\textbf{Disclosure statement}. The authors report there are no competing interests to declare.

%\bibliography{stability}
%\bibliographystyle{abbrv}

\begin{thebibliography}{10}

\bibitem{acd}
Y.~Achdou, F.~Camilli, and I.~Capuzzo~Dolcetta.
\newblock Homogenization of {H}amilton-{J}acobi equations: numerical methods.
\newblock {\em Math. Models Methods Appl. Sci.}, 18(7):1115--1143, 2008.

\bibitem{Ambrosio}
L.~Ambrosio.
\newblock Transport equation and {C}auchy problem for {$BV$} vector fields.
\newblock {\em Invent. Math.}, 158(2):227--260, 2004.

\bibitem{ALL}
O.~Alvarez, J.-M. Lasry, and P.-L. Lions.
\newblock Convex viscosity solutions and state constraints.
\newblock {\em J. Math. Pures Appl. (9)}, 76(3):265--288, 1997.

\bibitem{BCD}
M.~Bardi and I.~Capuzzo-Dolcetta.
\newblock {\em Optimal control and viscosity solutions of
  {H}amilton-{J}acobi-{B}ellman equations}.
\newblock Systems \& Control: Foundations \& Applications. Birkh\"{a}user
  Boston, Inc., Boston, MA, 1997.

\bibitem{BardiDragoni}
M.~Bardi and F.~Dragoni.
\newblock Convexity and semiconvexity along vector fields.
\newblock {\em Calc. Var. Partial Differential Equations}, 42(3-4):405--427,
  2011.

\bibitem{Barles}
G.~Barles.
\newblock An introduction to the theory of viscosity solutions for first-order
  {Hamilton}-{Jacobi} equations and applications.
\newblock In {\em Hamilton-Jacobi equations: approximations, numerical analysis
  and applications.}, pages 49--109. Berlin: Springer;
  Florence: Fondazione CIME, 2013.

\bibitem{Laurencot}
S.~Benachour, M.~Ben-Artzi, and P.~Lauren\c{c}ot.
\newblock Sharp decay estimates and vanishing viscosity for diffusive
  {H}amilton-{J}acobi equations.
\newblock {\em Adv. Differential Equations}, 14(1-2):1--25, 2009.

\bibitem{BKL}
S.~Benachour, G.~Karch, and P.~Lauren{\c{c}}ot.
\newblock Asymptotic profiles of solutions to viscous {Hamilton}--{Jacobi}
  equations.
\newblock {\em J. Math. Pures Appl. (9)}, 83(10):1275--1308, 2004.

\bibitem{BennetSharpley}
C.~Bennett and R.~Sharpley.
\newblock {\em Interpolation of operators}, volume 129 of {\em Pure and Applied
  Mathematics}.
\newblock Academic Press, Inc., Boston, MA, 1988.

\bibitem{BerghLofstrom}
J.~Bergh and J.~L\"{o}fstr\"{o}m.
\newblock {\em Interpolation spaces. {A}n introduction}.
\newblock Grundlehren der Mathematischen Wissenschaften, No. 223.
  Springer-Verlag, Berlin-New York, 1976.

\bibitem{BKRS}
V.~I. Bogachev, N.~V. Krylov, M.~R\"{o}ckner, and S.~V. Shaposhnikov.
\newblock {\em Fokker-{P}lanck-{K}olmogorov equations}, volume 207 of {\em
  Mathematical Surveys and Monographs}.
\newblock American Mathematical Society, Providence, RI, 2015.

\bibitem{Bressan}
A.~Bressan and T.~Yang.
\newblock On the convergence rate of vanishing viscosity approximations.
\newblock {\em Commun. Pure Appl. Math.}, 57(8):1075--1109, 2004.

\bibitem{CC}
L.~A. Caffarelli and X.~Cabr\'{e}.
\newblock {\em Fully nonlinear elliptic equations}, volume~43 of {\em American
  Mathematical Society Colloquium Publications}.
\newblock American Mathematical Society, Providence, RI, 1995.

\bibitem{CGMT}
F.~Cagnetti, D.~Gomes, H.~Mitake, and H.~V. Tran.
\newblock A new method for large time behavior of degenerate viscous
  {Hamilton}-{Jacobi} equations with convex {Hamiltonians}.
\newblock {\em Ann. Inst. Henri Poincar{\'e}, Anal. Non Lin{\'e}aire},
  32(1):183--200, 2015.

\bibitem{Calder}
J.~Calder.
\newblock {L}ecture notes on viscosity solutions.
\newblock available at
  \url{https://www-users.cse.umn.edu/~jwcalder/viscosity_solutions.pdf}, 2018.

\bibitem{CannarsaSinestrari}
P.~Cannarsa and C.~Sinestrari.
\newblock {\em Semiconcave functions, {H}amilton-{J}acobi equations, and
  optimal control}, volume~58 of {\em Progress in Nonlinear Differential
  Equations and their Applications}.
\newblock Birkh\"{a}user Boston, Inc., Boston, MA, 2004.

\bibitem{cdi}
I.~Capuzzo-Dolcetta and H.~Ishii.
\newblock On the rate of convergence in homogenization of {H}amilton-{J}acobi
  equations.
\newblock {\em Indiana Univ. Math. J.}, 50(3):1113--1129, 2001.

\bibitem{CecchinDelarue}
A.~Cecchin and F.~Delarue.
\newblock Selection by vanishing common noise for potential finite state mean
  field games.
\newblock {\em Commun. Partial Differ. Equations}, 47(1):89--168, 2022.

\bibitem{CGsima}
M.~Cirant and A.~Goffi.
\newblock On the existence and uniqueness of solutions to time-dependent
  fractional {MFG}.
\newblock {\em SIAM J. Math. Anal.}, 51(2):913--954, 2019.

\bibitem{CGpoinc}
M.~Cirant and A.~Goffi.
\newblock Lipschitz regularity for viscous {H}amilton-{J}acobi equations with
  {$L^p$} terms.
\newblock {\em Ann. Inst. H. Poincar\'{e} C Anal. Non Lin\'{e}aire},
  37(4):757--784, 2020.

\bibitem{CGpar}
M.~Cirant and A.~Goffi.
\newblock Maximal {$L^q$}-regularity for parabolic {H}amilton-{J}acobi
  equations and applications to {M}ean {F}ield {G}ames.
\newblock {\em Ann. PDE}, 7(2):Paper No. 19, 40, 2021.

\bibitem{CirantPorretta}
M.~Cirant and A.~Porretta.
\newblock Long time behavior and turnpike solutions in mildly non-monotone mean
  field games.
\newblock {\em ESAIM Control Optim. Calc. Var.}, 27:Paper No. 86, 40, 2021.

\bibitem{CEL}
M.~G. Crandall, L.~C. Evans, and P.-L. Lions.
\newblock Some properties of viscosity solutions of {H}amilton-{J}acobi
  equations.
\newblock {\em Trans. Amer. Math. Soc.}, 282(2):487--502, 1984.

\bibitem{CL83tams}
M.~G. Crandall and P.-L. Lions.
\newblock Viscosity solutions of {H}amilton-{J}acobi equations.
\newblock {\em Trans. Amer. Math. Soc.}, 277(1):1--42, 1983.

\bibitem{CL84}
M.~G. Crandall and P.-L. Lions.
\newblock Two approximations of solutions of {H}amilton-{J}acobi equations.
\newblock {\em Math. Comp.}, 43(167):1--19, 1984.

\bibitem{DDJ}
S.~Daudin, F.~Delarue, and J.~Jackson.
\newblock On the optimal rate for the convergence problem in mean field
  control.
\newblock arXiv:2303.08423, 2023.

\bibitem{EVZ}
M.~Escobedo, J.~L. Vazquez, and E.~Zuazua.
\newblock Asymptotic behaviour and source-type solutions for a
  diffusion-convection equation.
\newblock {\em Arch. Ration. Mech. Anal.}, 124(1):43--65, 1993.

\bibitem{EvansWeakKAM}
L.~C. Evans.
\newblock Further {PDE} methods for weak {KAM} theory.
\newblock {\em Calc. Var. Partial Differential Equations}, 35(4):435--462,
  2009.

\bibitem{Evansadjoint}
L.~C. Evans.
\newblock Adjoint and compensated compactness methods for {H}amilton-{J}acobi
  {PDE}.
\newblock {\em Arch. Ration. Mech. Anal.}, 197(3):1053--1088, 2010.

\bibitem{EvansBook}
L.~C. Evans.
\newblock {\em Partial differential equations}, volume~19 of {\em Graduate
  Studies in Mathematics}.
\newblock American Mathematical Society, Providence, RI, second edition, 2010.

\bibitem{EvansSmart}
L.~C. Evans and C.~K. Smart.
\newblock Adjoint methods for the infinity {L}aplacian partial differential
  equation.
\newblock {\em Arch. Ration. Mech. Anal.}, 201(1):87--113, 2011.

\bibitem{FRT}
X.~Fern\'andez-Real and R.~Tione.
\newblock Improved regularity for second derivatives of subharmonic functions.
\newblock arXiv:2110.02602, to appear in Proc. Amer. Math. Soc., 2023.

\bibitem{Figalli}
A.~Figalli.
\newblock Existence and uniqueness of martingale solutions for {SDE}s with
  rough or degenerate coefficients.
\newblock {\em J. Funct. Anal.}, 254(1):109--153, 2008.

\bibitem{FlemingJDE}
W.~H. Fleming.
\newblock The {C}auchy problem for a nonlinear first order partial differential
  equation.
\newblock {\em J. Differential Equations}, 5:515--530, 1969.

\bibitem{Giga}
Y.~Giga, S.~Goto, H.~Ishii, and M.-H. Sato.
\newblock Comparison principle and convexity preserving properties for singular
  degenerate parabolic equations on unbounded domains.
\newblock {\em Indiana Univ. Math. J.}, 40(2):443--470, 1991.

\bibitem{G23}
A.~Goffi.
\newblock On the optimal ${L}^q$-regularity for viscous {H}amilton-{J}acobi
  equations with subquadratic growth in the gradient.
\newblock {\em Comm. Cont. Math.}, 2350019, 2023.

\bibitem{GPV}
D.~A. Gomes, E.~A. Pimentel, and V.~Voskanyan.
\newblock {\em Regularity theory for mean-field game systems}.
\newblock SpringerBriefs in Mathematics. Springer, [Cham], 2016.

\bibitem{Hoff}
D.~Hoff.
\newblock The sharp form of {Oleinik}'s entropy condition in several space
  variables.
\newblock {\em Trans. Am. Math. Soc.}, 276:707--714, 1983.

\bibitem{IshiiFE}
H.~Ishii.
\newblock On the equivalence of two notions of weak solutions, viscosity
  solutions and distribution solutions.
\newblock {\em Funkcial. Ekvac.}, 38(1):101--120, 1995.

\bibitem{IL}
H.~Ishii and P.-L. Lions.
\newblock Viscosity solutions of fully nonlinear second-order elliptic partial
  differential equations.
\newblock {\em J. Differential Equations}, 83(1):26--78, 1990.

\bibitem{Korevaar}
N.~J. Korevaar.
\newblock Convex solutions to nonlinear elliptic and parabolic boundary value
  problems.
\newblock {\em Indiana Univ. Math. J.}, 32:603--614, 1983.

\bibitem{K66I}
S.~N. Kruzhkov.
\newblock Generalized solutions of nonlinear equations of the first order with
  several variables. {I}.
\newblock {\em Mat. Sb. (N.S.)}, 70 (112):394--415, 1966.

\bibitem{K66d}
S.~N. Kruzhkov.
\newblock Solutions of first-order nonlinear equations.
\newblock {\em Sov. Math., Dokl.}, 7:376--379, 1966.

\bibitem{K67II}
S.~N. Kruzhkov.
\newblock Generalized solutions of nonlinear equations of the first order with
  several independent variables. {II}.
\newblock {\em Mat. Sb. (N.S.)}, 72 (114):108--134, 1967.

\bibitem{K65}
S.~N. Kruzhkov.
\newblock The method of finite differences for a first-order non-linear
  equation with many independent variables.
\newblock {\em U.S.S.R. Comput. Math. Math. Phys.}, 6(5):136--151, 1968.

\bibitem{K75}
S.~N. Kruzhkov.
\newblock The {Cauchy}-{Dirichlet} problem for {Hamilton}-{Jacobi} equations of
  eikonal type.
\newblock {\em Sov. Math., Dokl.}, 16:1344--1348, 1975.

\bibitem{Kruzhkov}
S.~N. Kruzhkov.
\newblock Generalized solutions of {H}amilton-{J}acobi equations of eikonal
  type. {I}. {S}tatement of the problems; existence, uniqueness and stability
  theorems; certain properties of the solutions.
\newblock {\em Mat. Sb. (N.S.)}, 98(140)(3(11)):450--493, 496, 1975.

\bibitem{Krylov}
N.~V. Krylov.
\newblock {\em Controlled diffusion processes}, volume~14 of {\em Appl. Math. (N. Y.)}.
\newblock Berlin: Springer, reprint of the 1980 ed. edition, 2009.

\bibitem{Bris}
C.~Le~Bris and P.-L. Lions.
\newblock Existence and uniqueness of solutions to {F}okker-{P}lanck type
  equations with irregular coefficients.
\newblock {\em Comm. Partial Differential Equations}, 33(7-9):1272--1317, 2008.

\bibitem{LBL}
C.~Le~Bris and P.-L. Lions.
\newblock {\em Parabolic equations with irregular data and related
  issues---applications to stochastic differential equations}, volume~4 of {\em
  De Gruyter Series in Applied and Numerical Mathematics}.
\newblock De Gruyter, Berlin, 2019.

\bibitem{LinTadmor}
C.-T. Lin and E.~Tadmor.
\newblock {$L^1$}-stability and error estimates for approximate
  {H}amilton-{J}acobi solutions.
\newblock {\em Numer. Math.}, 87(4):701--735, 2001.

\bibitem{L82book}
P.-L. Lions.
\newblock {\em Generalized solutions of {H}amilton-{J}acobi equations},
  volume~69 of {\em Research Notes in Mathematics}.
\newblock Pitman (Advanced Publishing Program), Boston, Mass.-London, 1982.

\bibitem{Lions83}
P.-L. Lions.
\newblock Optimal control of diffusion processes and
  {H}amilton-{J}acobi-{B}ellman equations. {II}. {V}iscosity solutions and
  uniqueness.
\newblock {\em Comm. Partial Differential Equations}, 8(11):1229--1276, 1983.

\bibitem{Lions85aa}
P.-L. Lions.
\newblock Regularizing effects for first-order {H}amilton-{J}acobi equations.
\newblock {\em Applicable Anal.}, 20(3-4):283--307, 1985.

\bibitem{LPV}
P.-L. Lions, G.~Papanicolau, and S.~R.~S.~Varadhan.
\newblock Homogenization of Hamilton-Jacobi equations.
\newblock unpublished manuscript, 1986.

\bibitem{Liu}
Q.~Liu.
\newblock Semiconvexity of viscosity solutions to fully nonlinear evolution
  equations via discrete games.
\newblock In {\em Geometric properties for parabolic and elliptic {PDE}s},
  volume~47 of {\em Springer INdAM Ser.}, pages 205--231. Springer, Cham, 2021.

\bibitem{PerthameSanders}
B.~Perthame and R.~Sanders.
\newblock The {N}eumann problem for nonlinear second order singular
  perturbation problems.
\newblock {\em SIAM J. Math. Anal.}, 19(2):295--311, 1988.

\bibitem{PorrettaUMI}
A.~Porretta.
\newblock On the weak theory for mean field games systems.
\newblock {\em Boll. Unione Mat. Ital.}, 10(3):411--439, 2017.

\bibitem{Souganidis}
P.~E. Souganidis.
\newblock Existence of viscosity solutions of {H}amilton-{J}acobi equations.
\newblock {\em J. Differential Equations}, 56(3):345--390, 1985.

\bibitem{SZ}
P.~Souplet and Q.~S. Zhang.
\newblock Global solutions of inhomogeneous {H}amilton-{J}acobi equations.
\newblock {\em J. Anal. Math.}, 99:355--396, 2006.

\bibitem{TangZhang}
W.~Tang and Y.~P. Zhang.
\newblock The convergence rate of vanishing viscosity approximations for mean
  field games.
\newblock arXiv:2303.14560, 2023.

\bibitem{TranBook}
H.~Tran.
\newblock {\em Hamilton-Jacobi equations: {T}heory and {A}pplications}.
\newblock AMS Graduate Studies in Mathematics. American Mathematical Society,
  2021.

\bibitem{Tran2011}
H.~V. Tran.
\newblock Adjoint methods for static {H}amilton-{J}acobi equations.
\newblock {\em Calc. Var. Partial Differential Equations}, 41(3-4):301--319,
  2011.

\end{thebibliography}



\end{document}
