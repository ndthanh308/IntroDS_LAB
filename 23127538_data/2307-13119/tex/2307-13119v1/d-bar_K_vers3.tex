\documentclass[10pt,a4paper]{article}
\usepackage[utf8]{inputenc}
\usepackage{cite}
\usepackage[english]{babel}
\usepackage{graphicx}
\usepackage{geometry}
\geometry{margin=3cm}
\usepackage{amsbsy,todonotes,cancel, mathrsfs}
\usepackage[colorlinks=true, pdfstartview=FitV, linkcolor=blue, citecolor=blue, urlcolor=blue]{hyperref}
\usepackage{cancel,ulem}
\usepackage{amsmath,framed}
%\usepackage{showkeys}
\usepackage{amssymb}
\usepackage{amsthm}
%\usepackage{biblatex}
\usepackage{comment}
\usepackage{hyperref}
\def \nn{\nonumber}
\definecolor {darkgreen}{rgb}{0,0.6,0}
\def\le{\left}
\def\eqref#1{(\ref{#1})}
\def\ri{\right}
\def\M{\mathcal M}
\def\m{\ds \mathop}
\def\st{\hbox { s.t. }}
\def \QED{\hfill $\blacksquare$\par \vskip 5pt}
\def\ds{\displaystyle}
\def\un{\underline}
\def\res{\mathop{\mathrm {res}}\limits_}
\def\Eta{{\boldsymbol\eta}}
\def \bftau{ {\boldsymbol \tau}}
\def \bft{\mathbf t}
\def\be{\begin{eqnarray}}
\def\ee{\end{eqnarray}}
\def \e{\rm e}
\def \im{\, {\rm Im}}

\def\wt{\widetilde}
\newtheorem{lemma}{Lemma}[section]
\newtheorem{theorem}[lemma]{Theorem}
\newtheorem{corollary}[lemma]{Corollary}
\newtheorem{proposition}[lemma]{Proposition}
\newtheorem{definition}[lemma]{\mathscr{D}efinition}
\newtheorem{conjecture}[lemma]{Conjecture}
\newtheorem{remark}[lemma]{Remark}
\newtheorem{problem}[lemma]{Problem}
\numberwithin{equation}{section}
\def \Tr{ {\rm Tr}}
\def\ov{\overline}
\def\id{\mathrm {Id}\,}
\def \pa{\partial}
\definecolor{shadecolor}{rgb}{0.95, 0.95, 0.46}
\def\bd{
\definecolor{shadecolor}{rgb}{0.99, 0.9, 0.86}
\begin{shaded}
\begin{definition}}
\def\ed{\end{definition}
\end{shaded}}
\def\bp{
\definecolor{shadecolor}{rgb}{0.9, 0.99, 0.86}
\begin{shaded}
\begin{proposition}}
\def\ep{\end{proposition}
\end{shaded}}

\def\bea#1\eea{\begin{align}#1\end{align}}


\def\beas{\begin{eqnarray*}}
\def\eeas{\end{eqnarray*}}
\def \pa{\partial}
\def\C{{\mathbb C}}
\def\L{\mathcal L}
\def\R{{\mathbb R}}
\def\Q{{\mathbb Q}}
\def\N{{\mathbb N}}
\def\z {{\mathbf z}}
\def\wh{\widehat}
\def\H{{\cal H}}
\def\Z{{\mathbb Z}}
\def\u{\mathscr A}
\def\a{\alpha}
\def\d{\mathrm d}
\def\K{\mathcal K}
\def\l{\lambda}

\def\1{{\bf 1}}



\title{Integrable operators,   $\ov\pa$-Problems,  KP and NLS hierarchy}

%\author{Marco Bertola, Tamara Grava and Giuseppe Orsatti\\ \textit{\small SISSA}}

%\date{\today}



\begin{document}
\maketitle
\begin{center}
M. Bertola$^{\dagger\star}$ \footnote{Marco.Bertola@concordia.ca} 
T. Grava $^{\ddagger\diamondsuit}$ \footnote{grava@sissa.it}
G. Orsatti $^{\ddagger}$ \footnote{gorsatti@sissa.it}
\\
\bigskip
\begin{minipage}{0.7\textwidth}
\begin{small}
\begin{enumerate}
\item [${\dagger}$] {\it  Department of Mathematics and
Statistics, Concordia University\\ 1455 de Maisonneuve W., Montr\'eal, Qu\'ebec,
Canada H3G 1M8} 
\item[${\ddagger}$] {\it SISSA, International School for Advanced Studies, via Bonomea 265, Trieste, Italy  and INFN sezione di Trieste }
\item[${\star}$] {\it Centre de recherches math\'ematiques,
Universit\'e de Montr\'eal\\ C.~P.~6128, succ. centre ville, Montr\'eal,
Qu\'ebec, Canada H3C 3J7}
\item[${\diamondsuit}$] {\it School of Mathematics, University of Bristol, Fry Building, Bristol,
BS8 1UG, UK}
\end{enumerate}
\end{small}
\end{minipage}
\end{center}



\begin{abstract}
We develop the theory of integrable operators $\mathcal{K}$ acting on a domain 
%$\mathscr{D}$  
of the complex plane with smooth boundary
in analogy with  the theory of integrable operators acting on contours of the complex plane. 
 We show how the resolvent operator is obtained from the solution of a
 $\ov\pa$-problem in the complex plane. 
 When such a   $\ov\pa$-problem depends on auxiliary parameters we define its Malgrange one form   in analogy with the theory of  isomonodromic problems. 
 We  show that the Malgrange   one form is closed and  coincides with the exterior logarithmic differential of the  Hilbert-Carleman determinant of the operator $\mathcal{K}$.
 With suitable choices of the setup  we show that the Hilbert-Carleman determinant is a $\tau$-function of the  Kadomtsev-Petviashvili (KP) or  nonlinear Schr\"odinger   hierarchies.
\end{abstract}
\section{Introduction}
\label{sec:intro}
A key notion in the theory of solvable integrable systems is that of $\tau$--function, (see \cite{Harnad2021} for a comprehensive historical perspective) and  in many instances such $\tau$--functions coincide with  the Fredholm determinant of some 
integral  operator. 

 A  distinguished class of integral operators  are the so called  {\it integrable operators}:
the theory of such operators has its roots in the work of Jimbo et al. \cite{JMMS80}  that ultimately led to the construction by   Its, Izergin, Korepin and Slavnov \cite{IIKS} of a Riemann-Hilbert   problem to express the kernel of their resolvent operators.
Their original motivation  for studying these operators comes  from the theory of
 quantum integrable models. Later the theory of integrable operators found  applications in many fields of mathematics like   random matrices and integrable partial differential equations: for example the {\it gap probabilities} in determinantal random point processes (and more generally the generating function of occupation numbers) are expressible as  a Fredholm determinant \cite{Soshnikov}, \cite{Deift} and this is at the core of the celebrated Tracy-Widom distribution for fluctuations of the largest eigenvalue of a random matrix in a Gaussian Unitary Ensemble \cite{TracyWidom}.
An integrable operator is an  integral operator acting on  $L^2(\Sigma,|\d w|)\otimes \mathbb{C}^n$ of the form
\[
\mathcal{K}[v](z)=\int_{\Sigma}K(z,w)v(w)dw
\]
where $\Sigma$ is some oriented contour in  the complex plane  and  the  kernel $K(z,w)\in gl(n,\C)$ has a special form
\bea
  \label{eq:K_ker0}
  K(z,w) :&= \frac{f^T(z) \, g(w)}{z - w} , \quad f(z),g(z) \in Mat(r \times n, \mathbb{C}), 
\eea
where $f$ and $g$ are rectangular $r\times n$ matrices and for the time being we only assume that $f$ and $g$ are smooth
along the connected components of $\Sigma$.
The condition for $K$ to be nonsingular requires 
\[
   f^{T}(z) \, g(z) \equiv 0.
 \]
In the  most relevant applications, we refer to the operators of the form~\eqref{eq:K_ker0} as trace class operators.  
%We observe that if $ \mathcal{K}$ is a Hilbert–Schmidt operator with a well–defined
%and continuous diagonal in $\Sigma\times \Sigma$ then the Fredholm   determinant is well defined \cite{Simon_2015}.
 An  important observation in \cite{IIKS} is that the resolvent operator
  \begin{equation}
  \label{Res}
  \mathcal{R}=\mathcal{K}(\id-\mathcal{K})^{-1} = (\id-\mathcal K)^{-1} - \id,
  \end{equation}
  where $\id$ is the identity operator, is in the same class, namely
   \[
 \mathcal{R}[v](z)=\int_{\Sigma}R(z,w)v(w)dw
 \]  
 where the resolvent kernel has also the form of an integrable operator:
 \bea
  \label{eq:R_ker0}
  R(z,w) :&= \frac{F^T(z) \, G(w)}{z - w} , \quad F(z),G(z) \in Mat(r \times n, \mathbb{C}).
\eea
Here  $F^T(z)=(\id-\mathcal{K})^{-1}f^T$ and $G=g(\id-\mathcal{K})^{-1}$
where $(\id-\mathcal{K})^{-1}$ in the first relation is acting to the right while in the second relation its action is  to the left.
  Another crucial observation of \cite{IIKS} (see also the introduction of \cite{Its_Harnad}) is that the determination of $\mathcal{R}$ is equivalent to the solution of an associated Riemann-Hilbert   (RH) problem for a $r\times r$ matrix $\Gamma(z)$ 
    analytic in $\C\backslash \Sigma$ that satisfies the boundary value relation (sometimes referred to as ``jump relation'')
    \begin{equation}
    \label{RH0}
    \begin{split}
   & \Gamma_+(z)=\Gamma_-(z)M(z), \;\;z\in\Sigma, \quad M(z)=\1+2\pi i f(z)g^T(z)\\
    &\Gamma(z)\to\1,\quad \mbox{as $|z|\to\infty$}.
    \end{split}
    \end{equation}
    Here $\Gamma_\pm(z)$ denote the boundary values of the matrix $\Gamma(z)$ as $z$ approaches from the left and right   the oriented contour $\Sigma$ and $\1$ is the identity matrix in $\mbox{Mat}(r\times r,\C)$. The matrices $F$ and $G$ that define the resolvent kernel \eqref{eq:R_ker0} are related to the solution  $\Gamma$ of the RH problem \eqref{RH0}  by the relation 
    \begin{equation}
    \label{FG}
    F(z)=\Gamma(z) f(z),\quad G(z)=(\Gamma(z)^T)^{-1}g(z).
     \end{equation}
   This  connection between the Fredholm determinant  and  the RH problem    has been exploited in several contexts where the kernel depends on large parameters and the study of the asymptotic behaviour  of the Fredholm determinant 
   is obtained via  the Deift–Zhou nonlinear steepest descent method of the corresponding  RH problem \cite{DZ}. 
    This analysis has been  successfully implemented  for $n=1$ and $r=2$  in a large  class of kernels     originating in random matrices, orthogonal polynomials, probability and partial differential equations see for example (see e.g. \cite{DZ,Harnad2021,BorodinDeift, DeiftItsKrasovsky, MR1060387}).

 The knowledge of the resolvent operator allows to write variational formulæ for the
  Fredholm determinant of the operator
$\id - \mathcal{K}$  as
\be
\label{eq:Jvar}
\delta  \log \det (\id - \mathcal{K}) =-\mbox{Tr} \bigl((\id - \mathcal{K})^{-1} \circ \delta \mathcal{K}\bigr)= -\mbox{Tr} \bigl((\id + \mathcal{R}) \circ \delta \mathcal{K}\bigr),
\ee
 where here and below  $\delta$ stands for  exterior total  differentiation  in the space of parameters.

There exist  more general Riemann Hilbert problems then \eqref{RH0},  that  describe the inverse monodromy problem of a linear system of first order  ODEs in the complex plane: in those cases the deformation parameters can be introduced in such a way that the monodromy data do not depend on them. 
In this context,
the Kyoto school headed by Jimbo, Miwa and Ueno \cite{JMU} introduced the concept of   isomonodromic $\tau$ function starting from a differential  one form  $\omega$  (the Malgrange one form)  defined on the space of isomonodromic deformation parameters.
In several situations the isomonodromic tau function can be identified with the Fredholm determinant (possibly up to multiplication by explicit factors) of an operator of integrable form \cite{Bertola},\cite{CGL},\cite{GL}.
%
% $L^2(\Sigma)$   of the form $Id_{L^2(\Sigma)}+\mathcal{K}$ where $\Sigma $ is some contour on the complex plane $\mathbb{C}$ and the kernel $K(z,w)$, $z,w\in\Sigma$   of the operator $\mathcal{K}$ takes the form
%\begin{equation}
%\label{int1}
%K(z,w)=\dfrac{\sum_{\ell=1}^Nf_\ell(z)g_\ell(w)}{z-w}\quad \mbox{ with  $\quad \sum_{\ell=1}^Nf_\ell(z)g_\ell(z)=0$}.
%\end{equation}
%Here $N$ is a positive integers, and in many applications it turns out to be equal to  two.
% It has been show in \cite{IIKS}   that the study of such integral operators  namely the  calculation of their
% their resolvent, variational formula for the determinant, etc, can be reduced to the  solution of an $N \times  N$ 
% Riemann–Hilbert problem.   
 
  
 An enlargement of the class of integrable operators~\eqref{eq:K_ker0} was studied  by Bertola and Cafasso   \cite{BC} who 
considered Hankel composition operators that have been reduced  to integrable operators in Fourier space.
 
  Recently Bothner \cite{BB} and A. Krajenbrink \cite{KlD}  enlarged the class of Hankel composition operators that    can be studied via Riemann-Hilbert problems.  Applications are obtained in \cite{BCT}, \cite{CCR}.
  
The common feature of all these works  is the appearance, in one way or another, of a Riemann--Hilbert problem, namely, a boundary value problem of a matrix with discontinuities across a contour (or union thereof) with boundary values related multiplicatively by a group-like element $M$ (the 	``jump matrix'') as  in \eqref{RH0}.
  
  The goal of the present manuscript is to enlarge the class of integrable operators by considering  operators acting on a bounded  domain  $\mathscr{D}$  of the complex plane   with a  matrix  kernel $K(z,\ov z,w,\ov w)\in  Mat(n  \times n, \mathbb{C})$, namely 

  \bea
  \label{eq:K_ker}
&\mathcal{K}[v](z)=\iint_{\mathscr{D}}K(z,\ov z,w,\ov w)v(w)\frac{\d\ov z \wedge \d{z}}{2 i},\\
  &K(z,\ov z,w,\ov w) := \frac{f^T(z,\ov z) \, g(w,\ov w)}{z - w} , \nn \\
  \label{fTg}
   & f^{T}(z,\ov z) \, g(z,\ov z) \equiv 0 \equiv (\pa_{\bar{z}} f(z,\ov z))^{T} \, g(z,\ov z),\;\;\;\;f,g\in \mathcal{C}^\infty(\mathscr D, Mat(r  \times n, \mathbb{C})).
\eea
 Here and below,  instead of $\ov\pa$, we use the symbol $\pa_{\bar{z}}$ to specify the derivation respect to $\ov z$.   The dependence of $f$ and $g$ on $z$ and $\ov z$ is to remind the reader that $f$ and $g$ are  in  general smooth matrix functions on the complex plane. 
The kernel $ K(z,\ov z,w,\ov w)$ and the corresponding integral   operator $ \mathcal{K}$ is a Hilbert–Schmidt operator with a well–defined
and continuous diagonal in $\mathscr{D}\times \mathscr{D}$  {and therefore  its  Fredholm   determinant is well defined \cite{Simon_2015}.
 Our results are the following.
 \begin{itemize}
 \item    In Section~\ref{sec:K}  we show that  the resolvent of the integral operator $\id-\mathcal{K}$ is obtained through the solution  of a
   $\overline{\partial}$-Problem   (instead of a Riemann-Hilbert problem)
   for a matrix function $\Gamma$
   \be
   \begin{split}
   \label{dbar0}
&\pa_{\bar{z}} \Gamma(z,\ov z) = \Gamma(z,\ov z) M(z,\ov z);\qquad \Gamma(z,\ov z) \underset{z \to \infty}{ \to } \1,\\
&M(z,\ov z)=\pi f(z,\ov z)g^T(z,\ov z)\chi_{\mathscr{D}}(z),
\end{split}
\ee
where $\chi_{\mathscr{D}}(z)$  is the characteristic function   of the domain $\mathscr{D}$.  Note that the matrix $M(z,\ov z)$ is nilpotent because of \eqref{fTg}.
  We show that the $\overline{\partial}$-Problem is solvable if and only if the operator $\id-\mathcal{K}$
 is invertible.  Further we show, in analogy with  integrable operators defined on contours, 
  that  the kernel of the resolvent is
  \begin{align*}
 R(z,\ov z,w,\ov w)= \frac{F(z,\ov z)^T\, G(w,\ov w)}{z - w}, \quad F(z, \ov z) = \Gamma(z, \ov z) f(z, \ov z),\  \ G(z, \ov z) = \Gamma^{-1}(z) g(z, \ov z)
\end{align*}
where $\Gamma$ solves the $\ov\pa$-problem \eqref{dbar0}.
 
 \item In Section~\ref{sec:fred}  using the Jacobi variational formula~\eqref{eq:Jvar}  we show that 
 \bea
 \delta\log&\det(\id-\mathcal{K})=\omega - \delta\Tr\left(\mathcal{K}\right),
 \\
\label{defomega}&\omega:= -\iint_{\mathscr{D}}\Tr\left(\Gamma^{-1}(z)\pa_z\Gamma(z)\delta M(z)\right)\frac{\d\ov z \wedge \d{z}}{2\pi i},
 \eea
%\todo{Forse dire due parole perche' omega e' chiusa}
 where $\Tr(\mathcal{K})$ is (formal) operator trace, defined as
 \be
 \Tr(\mathcal{K}):= \iint_{\mathscr{D}}\Tr(K(z,z))\frac{\d\ov z \wedge \d{z}}{2  i},
 \ee
 and $\Tr(\Gamma(z))$ is matrix trace. The one form $\omega$ is shown to be closed. In analogy with the literature on Riemann--Hilbert problems on contours \cite{Bert10, Bertola} we call $\omega$   the Malgrange one form of the $\overline{\partial}$-Problem. The corresponding $\tau $ function of the $\overline{\partial}$-Problem is henceforth defined by 
  \[
  \delta \log \tau=\omega\,.%+ \delta_{t} \Tr_{L^{2}}[\mathcal{K}]
  \]
  Using the relation between  the  Fredholm determinant and  the  \textit{Hilbert-Carleman determinant}  ~\cite{Gohberg_2000}
  \[
  \text{{\rm det}}_2(\id-\mathcal{K}):=\det(\id-\mathcal{K}){\rm e}^{\Tr(\mathcal{K})}
  \]
we conclude that the $\tau$-function of the $\ov\pa$-problem coincides with the Hilbert-Carleman determinant  of the operator $\mathcal{K}, $  namely
\begin{equation}
\label{tau0}
   \tau=\text{{\rm det}}_2(\id-\mathcal{K}).
   \end{equation}
We also show (Subsection \ref{secMal})  that the formula \eqref{defomega} defines a closed one--form under the less restrictive assumption that $M(z,\ov z)$  is traceless  but not nilpotent, and this allows us to define a $\tau$--function (up to multiplicative constants) of the $\ov\pa$-problem 
\begin{equation}
\label{dbar1}
\pa_{\bar{z}} \Gamma(z,\ov z) = \Gamma(z,\ov z) M(z,\ov z);\qquad \Gamma(z,\ov z) \underset{z \to \infty}{ \to } \1, 
\end{equation}
by the relation 
\be
\label{tau1}
\delta \log \tau =
 \omega.
\ee
Note that in this more general case the $\tau$-function of  the  $\ov\pa$-problem is well defined but it is not in general related to a
Hilbert-Carleman determinant  of some integral operator.
   \item Finally in Section~\ref{sect:tau}
 we use the results of the previous section  by considering  the    $\ov\pa$-problem \eqref{dbar1} 
where  $M$    is  a $2\times 2$ matrix of    the form
\[
  M(z,\bar{z},\pmb{t})= {\rm e}^{\frac{\xi(z,\pmb{t})}{2}\sigma_{3}}M_{0}(z,\bar{z}){\rm e}^{-\frac{\xi(z,\pmb{t})}{2}\sigma_{3}}\;\quad \sigma_3=\begin{bmatrix}1&0\\0&-1\end{bmatrix},
\]
where $\xi(z,\pmb{t})= \sum_{j=1}^{+\infty}z^{j}t_{j}$ and $M_{0}(z,\bar{z})$ is a traceless matrix compactly supported on $\mathscr{\mathscr{D}}$; we show that   the corresponding $\tau$-function \eqref{tau1}  of the $\ov\pa$-problem  \eqref{dbar1}  is a Kadomtsev-Petviashvili  (KP)  $\tau$-function, 
 namely it satisfies  Hirota bilinear relations for  the  KP hierarchy   (see e.g. \cite{Harnad2021}).
 
% \item We prove that if $f$ and $g$ are two dimensional vector such that the matrix $f(z, \ov z)\, g^{T}(z,\ov z)$ satisfy the Schwarz symmetry, its diagonal is antiholomorphic and also depends on a finite (or infinite) number of parameters $\pmb{t} = (t_1, t_2, \dots, t_n)$, then the solution of the corresponding $\ov\pa$-problem generates the NLS hierarchy~\cite{Guil1994}; which is a special case of the Ablowitz-Kaup-Newell-Segur hierarchy (AKNS)~\cite{\mathscr{D}ickey1997,Matveev2018}.
  


Further , we specialize  the matrix $M$  of the $\ov\pa$-problem in  \eqref{dbar1} to the nilpotent and traceless  form 
$$M(z,\ov z;x,t)=\pi  {\rm e}^{-i(zx +z^2 t)\sigma_3} f(z,\ov z)g^T(z,\ov z){\rm e}^{i(zx +z^2 t)\sigma_3},\quad x\in \R,\;\; t\geq 0  $$ with   \begin{equation}
  f(z,\ov z)= \frac{1}{\sqrt{\pi}}
  \begin{bmatrix}
    \sqrt{\beta(z, \ov z)} \chi_{\mathscr{D}}(z)\\
    -\sqrt{\beta^{*}(z, \ov z)}\chi_{\ov{\mathscr{D}}}(\ov z)
  \end{bmatrix}\;\;\;
  g(z, \ov z)= \frac{1}{\sqrt{\pi}}
  \begin{bmatrix}
    \sqrt{\beta^{*}(z, \ov z)}\chi_{\ov{\mathscr{D}}}(\ov z)\\
    \sqrt{\beta(z, \ov z)} \chi_{\mathscr{D}}(z)
  \end{bmatrix},
\end{equation}
where $\beta^{*}(z,\ov z)=\overline{\beta(\ov z,z)}$ is a smooth function  and $\chi_{\mathscr{D}}$, $\chi_{\ov{\mathscr{D}}}$ are respectively the characteristic functions of a simply connected domain $\mathscr{D}\subset \C^+$ and its conjugate $\ov{\mathscr{D}}$.
% the {\color{red}  flux tensor $F_{jk}$ [Ecchee' 'sto flux tensor?]}
% \be
% F_{jk}=\frac{\pa^{2}}{\pa t_{j} \pa t_{k}}\ln \tau(\pmb{t}).
% \ee
% This property was proved by Flachka,Newell and Ratiu in~\cite{Flaschka1983}.
Then we show that  the  $\tau$-function  of   the  $\ov\pa$-problem  \eqref{dbar1} is  the  $\tau$-function  for the focusing Nonlinear Schr\"odinger   (NLS) equation and 
 coincides with  Hilbert-Carleman determinant of the operator $\mathcal{K}$ with integrable kernel $K(z,\ov z,w,\ov w) = \frac{f^T(z,\ov z) \, g(w,\ov w)}{z - w} $, namely 
\begin{equation}
\label{solNLS}
\pa^{2}_{x} \log \tau(x,t) =\pa^{2}_{x}\log\text{{\rm det}}_2(\id-\mathcal{K})= |\psi(x,t)|^2, \nn
\end{equation}
where  the complex function  $\psi(x,t)$ solves the focusing Nonlinear Schr\"odinger equation (NLS)
 \be
  \label{eq:NLS}
  i \pa_t   \psi + \frac{1}{2} \pa^{2}_x \psi + |\psi|^{2}\psi=0. 
 \ee
While  we can write the solution  to the NLS equation in the form \eqref{solNLS},  the analytical properties of such  family of initial data and solutions (e.g. the long-time behaviour)  have  still to be  explored. 
It is shown in \cite{BGO2023} that  such family of initial data  naturally emerge in the limit of  an infinite number of solitons. For some choices of the density $\beta$ and the domain $\mathscr{D}$,  the corresponding initial data are  solitons or  step-like   oscillatory initial data.
  \end{itemize}
  }
   
\section{Integrable operators and $\ov\pa$-problems}
\label{sec:K}

Let $\mathscr{D} \subset \mathbb{C}$  be a  compact union of domains with smooth boundary  and denote by  $\mathcal{K}$ the integral operator acting on the space $L^{2}(\mathscr{D},\d^2 z)\otimes \C^n$ with a kernel $K(z,w)$ of the form
\begin{align}
  \label{eq:K_ker2}
  &K(z,\ov z,w,\ov w) := \frac{f^T(z,\ov z)  g(w,\ov w)}{z - w} , \quad f(z, \ov z),g(z,\ov z) \in {\rm Mat}(r \times n, \mathbb{C}),  \\
   \label{eq:K_ker3}
   & f^T(z,\ov z)\, g(z,\ov z) \equiv 0  \quad \mbox{and }\quad   (\pa_{\bar{z}} f(z,\ov z))^{T} \, g(z,\ov z) \equiv 0.
\end{align}
Here the matrix-valued functions $f,g$ are assumed to be sufficiently smooth on $\mathscr D$ but no analyticity is required and for this  reason we indicate the dependence on both variables $z$ and $\ov z$. 
The vanishing requirements along the locus $z=w$ are sufficient to guarantee that  the kernel $K$ admits a well-defined value on the diagonal and it is continuous on $ \mathscr{D}\times  \mathscr{D}$ 
\be
\lim_{w\to z} K(z,\ov z,w,\ov w)=K(z,\ov z, z,\ov z) = \pa_{z} f^T(z,\ov z)\, g(z,\ov z).
\ee
We have emphasized that the kernel and the functions are not holomorphically dependent on the variables; that said, from now on we omit the explicit dependence on $\ov z$, trusting that the class of functions we are dealing with  will be clear by the context each time.
The operator $\mathcal{K}$  acts as follows on functions
\be
\label{Koper}
\mathcal{K}[\varphi](z):= \iint_{\mathscr{D}} K(z,w) \varphi (w) \frac {\d \ov w \wedge \d w}{2i}, \ \ \varphi\in L^{2}(\mathscr{D},\d^2 z)\otimes \C^n.
\ee
%The matrix functions $f,g$ are supposed to be smooth on the closure of $ \mathscr{D}$; we  write $f,g$ for the same functions extended to zero on the complement of $ \mathscr{D}$.
We introduce the following $\ov\pa$-problem for an $r\times r$ matrix-valued function $\Gamma(z,\ov z)$.  
\begin{problem}
\label{dbarproblem}Find a matrix-valued function $\Gamma(z,\ov z)\in GL_r(\mathbb C)$ such that 
\be
\label{dbarpb}
\pa_{\bar{z}} \Gamma(z) = \Gamma(z) M(z);\qquad \Gamma(z) \underset{z \to \infty}{ \to } \1
\ee
where $\1$ is the identity in $GL_r(\mathbb C)$ and 
\begin{equation}
  \label{eq:M}
M(z):= 
\left\{
\begin{array}{ll}
{\pi} f(z)g^T(z),& \mbox{for $z\in \mathscr{D}$},\\
&\\
0& \mbox{for $z\in\C\backslash \mathscr{D}$}.
\end{array}\right.
\end{equation}

\end{problem}
We first show that 
\begin{lemma}
  \label{lemma_1}
If a solution of the $\ov\pa$-problem \ref{dbarproblem} exists, it is unique. Furthermore $\det \Gamma(z) \equiv 1$. 
\end{lemma}
\noindent {\bf Proof.}
 If $\Gamma$ is a solution of the $\ov\pa$-problem \ref{dbarproblem}  then
%
% $\det\Gamma\neq 0$ and 
%\bea
%\pa_{\bar{z}}  \log  \det \Gamma =\frac{\pa_{\bar{z}}    \det \Gamma}{\det\Gamma}= \Tr \Big( \Gamma^{-1} \Gamma M\Big)= \Tr M=0,
%\eea
%%or 

\bea
\pa_{\bar{z}} \det \Gamma = \Tr \Big({\rm adj} (\Gamma)\pa_{\bar{z}} \Gamma\Big)=\Tr \Big({\rm adj} (\Gamma) \Gamma M\Big)
\eea
where ${\rm adj}(\Gamma)$ denotes the adjugate matrix (the co-factor, transposed). Now the product in the last formula yields  ${\rm adj}(\Gamma)\Gamma =( \det \Gamma)\1 $, 
so that 
\be
\pa_{\bar{z}} \det \Gamma =\det(\Gamma) \Tr ( M) = 0
\ee
where the last identity follows from the fact that $M$ is traceless because $\Tr ( M) =\Tr ( M^T)=0$.  Thus $\det \Gamma$ is an entire function  which tends to $\1$ at infinity, and hence  it  is identically equal to $1$ by Liouville's theorem. 

Now, if $\Gamma_1,\Gamma_2$ are two solutions, it follows easily that $R(z):= \Gamma_1 \Gamma_2^{-1}$ is an entire matrix-valued function which tends to the identity matrix $\1$ at infinity. This proves that $R(z)\equiv \1$ and hence shows the uniqueness.
\QED
\begin{theorem}
The operator $\id-\mathcal K$   with $\mathcal K$ as in \eqref{Koper} is invertible in $L^2(\mathscr D, \d^2 z) \otimes \C^n$ if and only if the $\ov\pa$-problem \ref{dbarproblem} admits a solution.   The resolvent $\mathcal{R}$ of $\mathcal{K}$ has  kernel  given by:
  \begin{equation}
    \label{eq:Resol}
    R(z,w) := \frac{f^{T}(z)\Gamma^{T}(z)\le(\Gamma^{T}(w)\ri)^{-1}\!\!\! g(w)}{z - w}
  \end{equation}
  where $\Gamma(z)$ is a $r \times r$ matrix  that solves the $\ov\pa$-problem \ref{dbarproblem}.
%\begin{equation}
%  \label{eq:d-bar}
%  \begin{split}
%    &\bar{\partial}\Gamma(z) =  \Gamma(z)M(z)\\
%    &\Gamma(z) \underset{z \to \infty}{ \to } \mathbb{I}_{r}. 
%  \end{split}
%\end{equation}
\end{theorem}

\noindent {\bf Proof.}
Suppose that  the $\ov\pa$-problem~\ref{dbarproblem} is solved by $\Gamma(z)$; we now show that the operator  $(\id- \mathcal{K})$ is invertible.
 Let us define the resolvent operator
  \begin{equation*}
    \mathcal{R} : L^{2}(\mathscr{D},\d^2 z)\otimes \C^n\to L^{2}(\mathscr{D},\d^2 z)\otimes \C^n
  \end{equation*}
  with the kernel given by \eqref{eq:Resol}.
 To verify that $\mathcal R$ is indeed the resolvent, we need to check that  the following condition is satisfied
  \begin{equation}
    \label{eq:resol-cond}
    \begin{split}
      (\id+ \mathcal{R}) \circ  &(\id- \mathcal{K})= \id \\
      &\Downarrow\\
      \mathcal{R} \circ \mathcal{K} &= \mathcal{R} - \mathcal{K}.
    \end{split}
  \end{equation}
To this end we compute the kernel of $\mathcal{R} \circ \mathcal{K}$  namely  
\bea
(R\circ K)(z,w)&:= \iint_{\mathscr{D}}R(z,\zeta)K(\zeta,w) \frac {\d \ov \zeta \wedge \d \zeta}{2i}
\nn\\
&= \iint_{\mathscr{D}}\frac{f^{T}(z)\Gamma^{T}(z)\overbrace{(\Gamma^{T}(\zeta))^{-1}g(\zeta)f^{T}(\zeta)}^{=-\frac 1\pi  \pa_{\bar{\zeta}} (\Gamma^T(\zeta))^{-1}}g(w)}{(z - \zeta)(\zeta - w)} \frac {\d \ov \zeta \wedge \d \zeta}{2i}
\nn \\
 \nn \\
&=
-\frac{f^{T}(z)\Gamma^{T}(z)}{z-w}\iint_{\mathscr{D}} \pa_{\bar{\zeta}} (\Gamma^T(\zeta))^{-1}\le(\frac{ 1}{z - \zeta}+\frac1{\zeta - w} \ri)\frac {\d \ov \zeta \wedge \d \zeta}{2i\pi}g(w).
%      &=\frac{1}{(z - w)}\left[ \iint_{\mathcal{\mathscr{D}}}\frac{f^{T}(z)g(x)f^{T}(x)\Gamma^{T}(x)\Gamma^{-T}(w)g(w)}{z - x} \frac {\d \ov x \wedge \d x}{2i}  + \right. \\
%     &\left. - \iint_{\mathcal{\mathscr{D}}}\frac{f^{T}(z)g(x)f^{T}(x)\Gamma^{T}(x)\Gamma^{-T}(w)g(w)}{w - x} \frac {\d \ov x \wedge \d x}{2i} \right] \\
  \label{21231}
\eea  
 If we consider the generalized Cauchy-Pompeiu formula for the matrix $(\Gamma^T(z))^{-1}$ we can express it as the integral equation
% \todo[inline]{M:
% Here is the point; if $\pa_{\bar{z}}\Gamma$ is integrable, the Cauchy Pompeiu equation does not have a boundary term. But, if $\Gamma$ had some discontinuity, then $\pa_{\bar{z}}\Gamma$ would have a delta-contribution along the discontinuity and then we would have a boundary term.  In turn, this boundary term would cause the $\omega$ to be, in general, not closed.  For this to happen, the equation for $\Gamma$ would have to be a mixed $\pa_{\bar{z}}$ and RHP problem. 
% 
% So, it seems to me that if $M$ is bounded (I think it is enough $L^2$), then the equation $\pa \Gamma = \Gamma M$ should have a solution that is $L^2_{loc}$. Whom can we ask?
% 
% }
 \begin{equation}
   \label{eq:Cauchy_G}
   (\Gamma^T(z))^{-1} =   \1 - \iint\limits_{\mathscr{D}}\frac{\pa_{\bar{\zeta}}(\Gamma^T(\zeta))^{-1}}{\zeta - z} \frac{\d \ov \zeta \wedge d \zeta}{2 \pi i} 
 \end{equation}
% Since $\Gamma$ is analytic in $\C$ outside $\mathscr{D}$, the boundary term gives 
% \todo[inline]{M: I am a bit unsure of the analysis here; is $\Gamma$ smooth? why there is no boundary contribution?\\
% G: The boundary condition is $\1$ in the C-P formula}
We substitute \eqref{eq:Cauchy_G} into \eqref{21231}:
\bea     
(R\circ K)(z,w)&=
-\frac{f^{T}(z)\Gamma^{T}(z)}{z-w}\Big(\left( (\Gamma^T(z))^{-1} - \1 \right)
-\left( (\Gamma^T(w))^{-1} - \1 \right)\Big)g(w)
\nn\\
&=-\frac{f^{T}(z)\Gamma^{T}(z)}{z-w} 
\Big((\Gamma^T(z))^{-1}-(\Gamma^T(w))^{-1} \Big)g(w)
\nn\\
&= \frac{f^{T}(z)\Gamma^T(z) (\Gamma^{T}(w))^{-1}g(w) }{z-w} -\frac{f^{T}(z)g(w) }{z-w} = R(z,w)-K(z,w).
\eea
This shows that indeed $\mathcal R$ satisfies the resolvent equation \eqref{eq:resol-cond} and hence the operator $\id - \mathcal{K}$ is invertible.
\\[10pt]
 Viceversa, let us now suppose that the operator $\id - \mathcal{K}$ is invertible and denote 
 $$
 \mathcal{R} =\Big(\id - \mathcal{K}\Big)^{-1}-\id.
 $$
 We now verify that $\mathcal R$ has kernel
 \begin{equation}
   \label{eq:resolv_2}
   R(z,w)= \frac{F(z)^T\, G(w)}{z - w}
 \end{equation}
 where the matrices $F(z)$ and $G(z)$ are defined as
 \begin{equation}
   \label{eq:def_2}
   \begin{split}
     F(z)&:= (\id - \mathcal{K}^{T})^{-1}[f](z)\\
     G(z)&:=  (\id - \mathcal{K})^{-1}[g](z),\\
   \end{split} 
 \end{equation}
 with the inverse applied to each entry (and the transposition $T$ acts on the matrix indices).
 Indeed we verify  the condition~\eqref{eq:resol-cond} with $R$ given by~\eqref{eq:resolv_2}:
 \begin{equation*}
   \begin{split}
(R\circ K)(z,w)=    & \iint_{\mathscr{D}} \frac{F^{T}(z)G(\zeta)f^{T}(\zeta)g(w)}{(z - \zeta)(\zeta - w)} 
\frac{\d \ov \zeta \wedge \d {\zeta}}{2 i}
 =\frac{1}{z - w} \left(\iint_{\mathscr{D}}  \frac{F^{T}(z)G(\zeta)f^{T}(\zeta)g(w)}{(z - \zeta)} 
 \frac{\d \ov \zeta \wedge \d {\zeta}}{2 i} + \right. \\
    & \left. + \iint_{\mathscr{D}}  \frac{F^{T}(z)G(\zeta)f^{T}(\zeta)g(w)}{(\zeta - w)} \frac{\d \ov \zeta \wedge \d {\zeta}}{2 i} \right)\\
    &= \frac{1}{z - w}\Big(\mathcal{R}[f^{T}](z)g(w) + F^{T}(z)\mathcal{K}[G](w)\Big).\\      
   \end{split}
 \end{equation*}
Adding and subtracting the kernels $K(z,w)$ and $R(z,w)$, we obtain
 \bea
(R\circ K)(z,w)=\frac{1}{z - w}\Big((\id +& \mathcal{R})[f^{T}](z)g(w) - F^{T}(z)(\id - \mathcal{K})[G](w) \Big) + \nn \\
    & + R(z,w) - K(z,w) .
    \label{212}
\eea
With the definitions~\eqref{eq:def_2} the contributions in the first line of \eqref{212}  cancel out and the condition~\eqref{eq:resol-cond} is satisfied.
To conclude the proof we need to verify that 
\begin{align}
F(z) = \Gamma(z) f(z),\  \ G(z) = \Gamma^{-1}(z) g(z)
\end{align}
where the matrix $\Gamma$  solves the $\ov\pa$-problem \ref{dbarproblem}. 
To this end, let us define the matrix $\wt {\Gamma}(z)$
 \begin{equation}
   \label{eq:tild}
   \wt{\Gamma}(z) := \1 - \iint_{\mathscr{D}} \frac{F(\zeta) \, g^T(\zeta)}{\zeta - z} \frac {\d \ov \zeta \wedge \d \zeta}{2i} .
 \end{equation}
From this definition  it follows that 
 \begin{equation}
   \begin{split}
     f^{T}(z)\wt{\Gamma}^{T}(z) &= f^{T}(z) - \iint_{\mathscr{D}}\frac{f^{T}(z)g(\zeta) F^{T}(\zeta)}{\zeta - z} \frac {\d \ov \zeta \wedge \d \zeta}{2i}\\
     &= f^{T}(z) + \mathcal{K}[F^{T}](z) \\
     &=f^{T}(z) + F^{T}(z) - (\id-\mathcal{K})[F^{T}](z)\\
     &=F^{T}(z)\\
   \end{split}
 \end{equation}
 which implies
 \begin{equation}
   \label{eq:res_2}
   F(z)= \wt{\Gamma}(z)f(z).
 \end{equation}
 We now substitute \eqref{eq:res_2} in the definition \eqref{eq:tild}:
 \begin{equation}
   \wt{\Gamma}(z) = \1 - \iint_{\mathscr{D}}\frac{\wt{\Gamma}(\zeta) f(\zeta)g^{T}(\zeta)}{\zeta -z} \frac {\d \ov \zeta \wedge \d \zeta}{2i}.
 \end{equation}
 Then, following the general Cauchy formula~\eqref{eq:Cauchy_G}, we find that the matrix $\tilde{\Gamma}(z)$ satisfies
 \begin{equation}
  \pa_{\bar{z}}\tilde{\Gamma}(z)= \pi\wt {\Gamma}(z)f(z)g^{T}(z).
\end{equation}
Finally, since the support $ \mathscr{D}$ of $M$ is compact, the equation  \eqref{eq:tild} implies that $\wt\Gamma$ is analytic outside of $ \mathscr{D}$ and tends to $\1$ as $|z|\to \infty$. Thus $\wt \Gamma$ solves the same $\ov\pa$-problem \ref{dbarproblem} and since the solution is unique, it must coincide with  $\Gamma$.
\QED

\section{The Fredholm determinant}
\label{sec:fred}

In Section \ref{sec:K} we have  linked the solution of the $\ov\pa$-problem \ref{dbarproblem} with the existence of the inverse  of $\id -\mathcal{K}$. 
From the conditions~\eqref{eq:K_ker2} we conclude  that $\mathcal K$ is a Hilbert-Schmidt operator with a well-defined and continuous diagonal in $\mathscr{D} \times \mathscr{D}$: according to \cite{Simon_2015} this is sufficient to  define the Fredholm determinant for the operator $ \id - \mathcal{K}$, as explained in the following remark.

\begin{remark}
  In general, for a Hilbert-Schmidt operator $\mathcal{A}$, the Fredholm determinant is not defined but we can still define a regularization  of it, called the \textit{Hilbert-Carleman determinant}

\begin{equation}
  \label{eq:HC_det}
  \text{{\rm det}}_2(\id - \mathcal{A}):= \text{{\rm det}}\left((\id - \mathcal{A})\e^{\mathcal{A}}\right)= 1 + \sum_{n=2}^{\infty}\frac{(-1)^n}{n!}\Psi_{n}(\mathcal{A})
\end{equation}
where $\Psi_n(\mathcal{A})$ is given by the \textit{Plemelj-Smithies formula}
\begin{equation*}
  \Psi_{n}(\mathcal{A})=\det
  \begin{bmatrix}
    0 & n-1 & 0 & \dots & 0 &0\\
    \Tr(\mathcal{A}^{2})&0& n-2 &\dots &0 &0\\
    \Tr(\mathcal{A}^{3}) & \Tr(\mathcal{A}^2) &0&\dots &0 &0\\
    . & . & . & . & . & .\\
    \Tr(\mathcal{A}^{n}) & \Tr(\mathcal{A}^{n-1}) & \Tr(\mathcal{A}^{n-2}) &\dots& \Tr(\mathcal{A}^2)&0\\ 
  \end{bmatrix}.
\end{equation*}
It is shown  that if $\mathcal{A}$ is Hilbert-Schmidt then~\eqref{eq:HC_det} converges (\cite{Gohberg_2000}, Chapter 10, Theorem 3.1).
If $\mathcal{A}$ has a well define trace, then we can rewrite it as
\begin{equation}
\label{det2}
\text{{\rm det}}_2(\id - \mathcal{A})= \det(\id-\mathcal{A}){\rm e}^{\Tr(\mathcal{A})}.
\end{equation}
 Suppose now that $\mathcal A$ is a Hilbert--Schmidt operator with kernel $A(x,y)$ on $L^2(X,\d\mu)$, for some space $X$ and measure $\d\mu$, and such that $\int_X A(x,x)\d\mu(x)$ is well defined. Then  the formula can be turned on its head  to ``force'' a definition of Fredholm determinant even if the operator itself may not be of trace class. The resulting formula has all the properties of the standard Fredholm determinant. See \cite {Simon_2015} for more details.
\end{remark}



Let us now  assume that $\mathcal{K}$ depends smoothly on parameters $\pmb{t}=(t_1,t_2,\dots,t_j,\dots) \text{ with } t_j \in \C \text{ , } \forall \, j\ge 1$:  we want to relate solutions of the $\ov\pa$-problem~\ref{dbarproblem} with the variational equations for the determinant.

\begin{proposition}
\label{prop3.2}
Let us suppose that the matrix $M(z,\ov z)$   in the $\ov\pa$-Problem~\ref{dbarproblem},  depends smoothly on  some parameters $\pmb{t}$, while remaining identically nilpotent.  Then the solution $\Gamma(z)$   of the $\ov\pa$-problem~\ref{dbarproblem}   is related to the logarithmic derivative of the Fredholm determinant   of $\id -\mathcal K$ as follows:
  \begin{equation}
   \label{eq:lemma_3}
  \delta \log\Big[\det(\id - \mathcal{K})\Big]= -\iint_{\mathscr{D}}\Tr\le(\Gamma^{-1}(z)\pa_z\Gamma(z)\delta M(z)\ri)\frac{\d\ov z \wedge \d{z}}{2\pi i}  - \delta\Tr_{L^{2}}\le(\mathcal{K}\ri),
  \end{equation}
  where $\delta$  stands for the total differential in the space of parameters $\pmb{t}$.
\end{proposition}

\noindent {\bf Proof.}
Using the Jacobi variational formula~\eqref{eq:Jvar}, we can rewrite the LHS of~\eqref{eq:lemma_3} as

\begin{equation}
  \label{eq:J-var}
  \delta \log\Big[\det(\id - \mathcal{K})\Big]= - \Tr\le((\id + \mathcal{R})\circ \delta \mathcal{K}\ri).
\end{equation}
The last term in \eqref{eq:lemma_3} comes from the identity term in \eqref{eq:J-var}.  Let us now  compute the term $\Tr\le((\mathcal{R} \circ  \delta \mathcal{K})\ri)$.
The compositions of the two operator produces the kernel
\bea
 (R\circ\delta K)(z,w) &=\iint_{\mathscr{D}}\frac{f^{T}(z)\Gamma^{T}(z)(\Gamma^{T}(\zeta))^{-1}g(\zeta)\delta(f^{T}(x)g(w))}{(z-\zeta)(\zeta-w)} \frac {\d \ov \zeta \wedge \d \zeta}{2i}
    \nn
    \\
    &=\iint_{\mathscr{D}}\frac{f^{T}(z)\Gamma^{T}(z)(\Gamma^{T}(\zeta))^{-1}g(\zeta)f^{T}(\zeta)\delta g(w)}{(z-\zeta)(\zeta-w)} \frac {\d \ov \zeta \wedge \d \zeta}{2i} + \label{eq:I}
   \\
   &+\iint_{\mathscr{D}}\frac{f^{T}(z)\Gamma^{T}(z)(\Gamma^{T}(\zeta))^{-1}g(\zeta)\delta  f^{T}(\zeta)g(w)}{(z-\zeta)(\zeta-w)} \frac {\d \ov \zeta \wedge \d \zeta}{2i} \label{eq:II}
   \eea
   where we have omitted explicit notation of the dependence on $\pmb t$ of the functions $f,g, F, G, \Gamma$.

   We focus on the term in \eqref{eq:I}.
   Using the identity $\frac 1{(z-\zeta)(\zeta-w)} = \frac 1{z-w} \le(\frac 1 {z-\zeta}+\frac 1{\zeta-w}\ri)$, we obtain 
\bea
\eqref{eq:I}= & \frac{f^{T}(z)\Gamma^{T}(z)}{z - w} \left(\iint_{\mathscr{D}} (\Gamma^{T}(\zeta))^{-1}g(\zeta)f^{T}(\zeta)\le(\frac 1 {z-\zeta}+\frac 1{\zeta-w}\ri)\frac {\d \ov \zeta \wedge \d \zeta}{2i}\right)\delta   g(w) 
%+\nn \\
   % &+\frac{f^{T}(z)\Gamma^{T}(z)}{z - w} \left(\iint_{\mathscr{D}}\frac{ (\Gamma^{T}(\zeta))^{-1}g(\zeta)f^{T}(\zeta)}{\zeta - w} \frac {\d \ov \zeta \wedge \d \zeta}{2i}\right)\delta   g(w).
    \label{32}
\eea  
In order to compute the trace we need to compute the kernel \eqref{32} along the diagonal $z=w$ and hence we consider  $\lim_{w\to z} \eqref{32}$.
Observe  that $(\Gamma^{T}(\zeta))^{-1}g(\zeta)f^{T}(\zeta) = -\frac 1 \pi \pa_{\bar{\zeta}} (\Gamma^T(\zeta))^{-1}$, and hence we can apply  the formula \eqref{eq:Cauchy_G} to eliminate the integral and rewrite \eqref{32} as follows
\bea
\eqref{32}&=-\frac{f^{T}(z)\Gamma^{T}(z)\Big((\Gamma^{T}(z))^{-1}-\1 \Big)\delta   g(w)}{z - w} +\frac{f^{T}(z)\Gamma^{T}(z)\Big( (\Gamma^{T}(w))^{-1}-\1 \Big)\delta  g(w)}{z - w}\\
&=\frac{f^{T}(z)\Big(\Gamma^{T}(z)(\Gamma^{T}(w))^{-1} -\1 \Big)\delta   g(w)}{z-w} \label{39}.
\eea
 We can now easily compute the expansion of~\eqref{39}  along the diagonal $w\to z$ by Taylor's formula, keeping in mind that $\Gamma$ is not a holomorphic function inside $\mathscr D$: 
\bea
\eqref{39}&=f^{T}(z) \pa_{z}\Gamma^T(z) (\Gamma^{T}(z))^{-1}\delta   g(z) + \frac{\bar{z} - \bar{w}}{z - w}\overbrace{f^{T}(z)g(z)}^{\equiv \, 0}f^{T}(z)\delta  g(w) + \mathcal{O}(|z - w|) \\
\nn \\
&=f^{T}(z) \pa_{z}\Gamma^T(z) (\Gamma^{T}(z))^{-1}\delta   g(z) + \mathcal{O}(|z - w|). 
 \eea
 Using the above expression we conclude the  trace in $L^{2}(\mathscr{D},\d^2 z)\otimes \C^n$ of~\eqref{eq:I} is 
\begin{equation}
  \label{eq:Tr_I}
\Tr(\eqref{eq:I})=  \iint_{\mathscr{D}}
\Tr
\big(f^{T}(z) \partial_z\Gamma(z)^{T}(\Gamma^{T}(z))^{-1}\delta  g(z)\big)
\frac {\d \ov z\wedge \d z}{2i}.
\end{equation} 
Using the cyclicity of the trace and its invariance under transposition of the arguments, we reorder the terms~\eqref{eq:Tr_I} in to the form 
\begin{equation}
  \label{eq:eq:Tr_I_fin}
\Tr(\eqref{eq:I})=\iint_{\mathscr{D}}\Tr\big(
\Gamma^{-1}(z)\partial_z\Gamma(z)f(z)\delta  g^{T}(z)
\big)
\frac {\d \ov z\wedge \d z}{2i}.
\end{equation}
We now consider the term \eqref{eq:II}.  Taking its trace yields:
\bea
  \label{eq:Tr_II}
    &\Tr(\eqref{eq:II})=\nn
    \\
   & =-\iint_{\mathscr{D}}\iint_{\mathscr{D}}\frac{\Tr\big(f^{T}(z)\Gamma^{T}(z)(\Gamma^{T}(\zeta))^{-1}g(\zeta)\delta  f^{T}(\zeta)g(z)\big)}{(z - \zeta)^{2}}\frac {\d \ov \zeta \wedge \d \zeta}{2i} \frac {\d \ov z \wedge \d z}{2i} 
      \nn \\
  & = -\iint_{\mathscr{D}}\iint_{\mathscr{D}}\frac{\Tr\big(g(z)f^{T}(z)\Gamma^{T}(z)(\Gamma^{T}(\zeta))^{-1}g(\zeta)\delta  f^{T}(\zeta)\big)}{(z - \zeta)^{2}}\frac {\d \ov \zeta \wedge \d \zeta}{2i} \frac {\d \ov z \wedge \d z}{2i} 
\eea
 We observe that the integrand is  in  $L^2_{loc}$ because the numerator vanishes to order $\mathcal O(|z-\zeta|)$ along the diagonal
\bea
\Tr\Big(g(z)f^{T}(z)\Gamma^{T}(z)(\Gamma^{T}(\zeta))^{-1}g(\zeta)\delta  f^{T}(\zeta)\Big)
=
\Tr\Big(g(\zeta)\overbrace{f^{T}(\zeta)g(\zeta)}^{=0}\delta  f^{T}(\zeta)\Big)
+\mathcal O(|z-\zeta|),
\eea
and hence the integrand is $\mathcal O(|z-\zeta|^{-1})$ which is locally integrable with respect to the area measure.
 We can now relate this integral to  $\pa_z\Gamma$ as follows.
Using the formula~\eqref{eq:Cauchy_G} and the $\ov\pa$-problem \ref{dbarproblem}  we can rewrite $\Gamma^{T}(\zeta)$ as
\begin{equation}
  \label{eq:G-1}
  \Gamma^{T}(\zeta)= \1 - 
  \iint_{\mathscr{D}}\frac{\pa_{\bar{z}}(\Gamma^{T}(z))}{z - \zeta} \frac{\d\ov z\wedge \d{z}}{2 \pi i} =  \1
  - \iint_{\mathscr{D}}\frac{M^{T}(z)\Gamma^{T}(z)}{z - \zeta} \frac{\d\ov z\wedge \d{z}}{2 \pi i}
\end{equation}
%It's easy to see that, since $\Gamma(z)$ satisfy~\eqref{eq:d-bar}, than also $\Gamma^{-1}(z)$ must satisfy a $\ov\pa$-problem:
%\begin{equation*}
%  \pa_{\bar{z}}(\Gamma^{-1}(z))=-\Gamma^{-1}(z)\pa_{\bar{z}}(\Gamma(z))\Gamma^{-1}(z)=-M(z)\Gamma^{-1}(z)
%\end{equation*}
%
%and so, the equation~\eqref{eq:G-1} becomes
%\begin{equation}
%  \label{eq:G-2}
%   \Gamma^{-1}(z)= \mathbb{I} - \iint_{\mathcal{\mathscr{D}}}\frac{M(w)\Gamma^{-1}(w)}{w - z} \frac{dw\wedge d\bar{w}}{2 \pi i}.
% \end{equation}
 Taking the holomorphic derivative  with respect to $\zeta$ we get
 \begin{equation*}
   \partial_\zeta\Gamma^{T}(\zeta)= -\iint_{\mathscr{D}}\frac{g(z)f^{T}(z)\Gamma^{T}(z)}{(z-\zeta)^{2}}\frac{\d \ov z\wedge \d {z}}{2 i}
 \end{equation*}
% and the transposition give us
% \begin{equation}
%   \partial\Gamma^{-T}(z)=\iint_{\mathcal{\mathscr{D}}}\frac{\Gamma^{-T}(w)g(w)f^{T}(w)}{(w - z)^{2}}dw \wedge d\bar{w}.
% \end{equation}
Plugging the result  into \eqref{eq:Tr_II} we obtain
 \begin{equation}
   \begin{split}
     \eqref{eq:Tr_II}=&-\iint_{\mathscr{D}}\Tr\le((\Gamma^{T}(\zeta))^{-1}g(\zeta)\delta   f(\zeta)\left(\iint_{\mathscr{D}}\frac{g(z)f^{T}(z)\Gamma^{T}(z)}{(z -\zeta)^{2}}\frac {\d \ov z \wedge \d z}{2i}\right)\ri) \frac {\d \ov \zeta \wedge \d \zeta}{2i}\\
     = &\iint_{\mathscr{D}}\Tr\big((\Gamma^{T}(\zeta))^{-1}g(\zeta)\delta  f^{T}(\zeta)\partial_\zeta(\Gamma^{T}(\zeta))\big)\frac {\d \ov \zeta \wedge \d \zeta}{2i} \\
     =&\iint_{\mathscr{D}}\Tr\big(g(\zeta)\delta  f^{T}(\zeta)\partial_\zeta\Gamma^{T}(\zeta)(\Gamma^{-1}(\zeta))^{T}\big)\frac {\d \ov \zeta \wedge \d \zeta}{2i}\\
     = &\iint_{\mathscr{D}}\Tr\big(\Gamma^{-1}(\zeta)\partial_\zeta\Gamma(\zeta)\delta  f(\zeta)g^{T}(\zeta)\big)\frac {\d \ov \zeta \wedge \d \zeta}{2i},
   \end{split}
 \end{equation}
 so that 
 \begin{equation}
 \label{part2}
  \Tr(\eqref{eq:II})=\iint_{\mathscr{D}}\Tr\big(\Gamma^{-1}(\zeta)\partial_\zeta \Gamma(\zeta)\delta  f(\zeta)g^{T}(\zeta)\big)\frac {\d \ov \zeta \wedge \d \zeta}{2i},
  \end{equation}
Combining \eqref{eq:eq:Tr_I_fin} and \eqref{part2} we have obtained that
\bea
   \label{eq:Fr-det}
    & -\Tr ((\id + \mathcal{R})\circ \delta  \mathcal{K})= 
    -\Tr(\eqref{eq:I} + \eqref{eq:II}) -\delta  \Tr(\mathcal{K}) =
    \nn\\
     &=-\iint_{\mathscr{D}}\Tr(\Gamma^{-1}(z)\partial_z\Gamma(z)\delta  (f(z)g^{T}(z)))\frac{\d\ov z \wedge \d{z}}{2i} -\delta  \Tr(\mathcal{K})=
     \nn\\
     &=-\iint_{\mathscr{D}}\Tr(\Gamma^{-1}(z)\partial_z\Gamma(z)\delta  M(z))\frac{\d\ov z \wedge \d{z}}{2\pi i} -\delta  \Tr(\mathcal{K})
     \eea
This concludes the proof of Proposition~\ref{prop3.2}.
     \QED
\begin{remark}
\label{remark3.3}
Comparing the fomula  \eqref{eq:lemma_3} in Proposition~\ref{prop3.2} with   formula  \eqref{det2}
we can conclude that the first term in the r.h.s. of   \eqref{eq:lemma_3}  coincides with the  logarithmic derivative of the  \textit{Hilbert-Carleman determinant}, namely
\begin{equation}
  \delta   \log\text{{\rm det}}_2(\id - \mathcal{K})=
  -\iint_{\mathscr{D}}\Tr\le(\Gamma^{-1}(z)\partial_z\Gamma(z)\delta  M(z)\ri)\frac{\d\ov z \wedge \d{z}}{2\pi i}.
    \end{equation}
In analogy to the theory of isomonodromic RH problems,  we call the r.h.s. of the above relation the  Malgrange one form  ( see e.g. \cite{Bert10}).
\end{remark}





 \subsection{Malgrange one form and $\tau$-function}
\label{secMal}

 Combining the Proposition~\ref{prop3.2} and the remark~\ref{remark3.3}, we  define the following one form on the space of deformations, which we call {\it Malgrange one form} following the terminology in \cite{Bert10}: 
\begin{equation}
  \label{eq:Malg_form}
  \omega:=-\iint_{\mathscr{D}}\Tr\Big(\Gamma^{-1}(z)\partial_z\Gamma(z)\delta M(z)\Big)\frac{\d \ov z\wedge \d {z}}{2 \pi i}, %-\delta \Tr_{L^2}[\mathcal{K}]
\end{equation}
where $\Gamma(z)$ is solution of the $\ov\pa$-problem~\ref{dbarproblem} and $M(z)$ is defined in~\eqref{eq:M}.
For  the operator  $\mathcal K$ defined in \eqref{Koper}, the Proposition~\ref{prop3.2} implies that 
\bea
\omega = \delta \log \text{{\rm det}}_2(\id -\mathcal{K}), 
\eea
and hence $\omega$ is an exact (and hence closed) one form in the space of deformation parameters the operator $\mathcal K$ may depend upon. The form  $\omega$ can be shown to be closed under weaker assumptions on the matrix $M$ then the ones that appears in the $\ov\pa$-problem \ref{dbarproblem} as the following theorem show.
%{\color{red} when
%\[
%\delta^2M=2\sum_{i<j}\left[\dfrac{\partial f}{\partial t_i}\dfrac{\partial g}{\partial t_j}-\dfrac{\partial f}{\partial t_j}\dfrac{\partial g}{\partial t_i}\right]\d t_i\wedge \d t_j=0\,.
%\]
%}

% \todo[inline]{M: It seems to me that Thm. \ref{th_2} is completely subsumed by Lemma \ref{lemma3.1}, since the lemma shows clearly that the malgrange form is exact. The only added value of the Thm. is if we dispose of the condition $M^2=0$, which I think is not needed. It is only important that $\Tr M=0$ so that $\det \Gamma\equiv 1$. Lemma is now upgraded to Proposition.}
\begin{theorem}
  \label{th_2}
Suppose that the $r \times r$  matrix $M = M(z,\ov z; \pmb{t})$ is smooth and compactly supported in $\mathscr{D}$  (uniformly with respect to the parameters $\pmb{t}$), depends smoothly on $\pmb{t}$ and  $\Tr (M)\equiv 0$.
Let $\Gamma(z,\ov z;\pmb{t})$ be  the solution of the $\ov\pa$-problem \ref{dbar1}.  Then   the exterior differential of the one-form   $\omega $   defined in  \eqref{eq:Malg_form} vanishes:
\be
\delta \omega=0.
\ee
\end{theorem}

\noindent {\bf Proof.} 
From the $\ov\pa$-problem we obtain
\be
\label{deltaGamma}
\delta( \pa_{\bar{z}} \Gamma) = \Gamma \delta M + \delta \Gamma M \ \ \ \Rightarrow \ \ \ 
\delta \Gamma(z) = \iint_{\mathscr{D}}
   \frac{\Gamma(w)\delta M(w)\Gamma^{-1}(w)}{(w - z)^2}\frac{\d \ov w \wedge \d w}{2 \pi i}\Gamma(z).
\ee
Using \eqref{deltaGamma} we can compute 
\bea
\label{eq:proof}
\delta \omega &= -\iint_{\mathscr{D}}
   \Tr\Big(\delta(\Gamma^{-1}\partial_z\Gamma \wedge\delta  M )\Big)\frac{\d \ov z \wedge \d{z}}{2 \pi i} %- \delta^{2}\Tr_{L^2}[\mathcal{K}]
=\nn\\
&= \iint_{\mathscr{D}}
\Tr
\Big(\Gamma^{-1}\delta \Gamma \Gamma^{-1}\partial_z \Gamma \wedge\delta  M \Big)\frac{\d \ov z \wedge \d{z}}{2 \pi i}
-
 \iint_{\mathscr{D}}\Tr
 \Big(
 \Gamma^{-1}\delta \partial_z\Gamma \wedge \delta M\Big )\frac{\d \ov z \wedge \d{z}}{2 \pi i}
 \eea
 %the second term is identically zero because $\delta^2 \equiv 0$.
 
From \eqref{deltaGamma} we deduce
\begin{equation}
  \label{eq:proof_G1}
  \delta \partial_z\Gamma(z) =
   -\iint_{\mathscr{D}}
   \frac{\Gamma(w)\delta M(w)\Gamma^{-1}(w)}{(w - z)^2}\frac{\d \ov w \wedge \d w}{2 \pi i}\Gamma(z) + \delta \Gamma(z) \Gamma(z)^{-1} \partial_z\Gamma(z).
\end{equation}
Substituting~\eqref{eq:proof_G1} in the equation~\eqref{eq:proof} we obtain:  

\begin{equation}
  \label{eq:proof_om}
\delta \omega = \iint_{\mathscr{D}}
 \Tr
 \left(
\Gamma^{-1}(z)\left(\iint_{\mathscr{D}}\frac{\Gamma(w)\delta  M(w)\Gamma^{-1}(w)}{(w - z)^{2}}\frac{\d \ov w \wedge d{w}}{2 \pi i}\right)\Gamma(z) \wedge \delta M(z)
\right)
\frac{\d \ov z \wedge \d  z}{2 \pi i}.
\end{equation}
The crux of the proof is now the correct evaluation of the iterated integral:
\bea
\delta \omega  &= \iint_{\mathscr{D}} \frac{ \d^2 z}\pi\iint_{\mathcal{\mathscr{D}}} \frac{ \d^2 w}\pi
\frac{ F(z,w)}{(z-w)^2},\nn \\
&  F(z,w):= \Tr
 \Big(
\Gamma(w)\delta  M(w)\Gamma^{-1}(w)  \wedge \Gamma(z)\delta M(z)\Gamma^{-1}(z)
\Big)
\eea
By applying Fubini's theorem, since the integrand is antisymmetric in the exchange of the variables $z\leftrightarrow w$, we quickly conclude that the integral is zero. However the integrand is singular along the diagonal $\Delta := \{ z=w\}\subset  \mathscr{D}\times \mathscr{D}$ and we need to make sure that the integrand is absolutely summable. 

Recalling that  $F(z,w)=-F(w,z)$, so that $F(z,z)\equiv 0$, we now compute the Taylor expansion of $F(z,w)$ with respect to $w$ near $z$;
\be
F(z,w) = 0+  \pa_{w}F(z,z) (w-z) + \pa_{\bar{w}}F(z,z) (\ov w- \ov z)  + \mathcal O(|z-w|^2).
\ee
Thus $\frac{|F(z,w)|}{|z-w|^2} = \mathcal O(|z-w|^{-1})$ which is integrable with respect to the area measure. Hence application of Fubini's theorem is justified. 
\QED
From this theorem, we can define a $\tau$-function associated to the the $\ov\pa$-problem~\ref{dbar1} by %the following PDE
%\begin{equation}
%  \label{eq:tau}
%  \omega_{\pa_{\bar{z}}}+\delta \Tr[\mathcal{K}]=\delta_t \ln \tau(t).
%\end{equation}
\begin{equation}
   \label{eq:tau}
   \tau(\pmb{t})= \exp\left(\int \omega\right). %+ \Tr_{L^{2}}[\mathcal{K}]\right)
\end{equation}
 In general the  above  $\tau$-function is defined only up to scalar multiplication and hence should be rather thought of as a section of an appropriate line bundle over the space of deformation parameters, depending on the context. However, for $M$ in the form specified in \eqref{eq:M} we know from Proposition~\ref{prop3.2} that we can {\it identify} the tau function with a regularized determinant: 
\begin{equation}
  \label{eq:taudet}
  \tau(\pmb{t})= \exp\left(\int \omega\right)= \text{{\rm det}}_2(\id -\mathcal{K}). %+ \Tr_{L^{2}}[\mathcal{K}]\right)
\end{equation}
In the next section, by choosing a specific dependence on the parameters $\pmb{t}$  in the more general setting of $M$ as in Thm. \ref{th_2}  we are going to show that $ \tau(\pmb{t})$ is a  KP $\tau$-function in the sense that it satisfies Hirota bilinear relations \cite{Hirota1986}.

\section{$\tau (\pmb{t})$ as a KP $\tau$-function}
\label{sect:tau}

In this section we   consider a specific  type of dependence of $M$ on the ``times'': let  $M(z,\pmb{t})$   be a $2\times 2$ matrix that depends on $ \pmb{t}$ in  the following form
\begin{equation}
  \label{eq:M-matrix}
  M(z,\pmb{t})= {\rm e}^{\frac{\xi(z,\pmb{t})}{2}\sigma_{3}}M_{0}(z){\rm e}^{-\frac{\xi(z,\pmb{t})}{2}\sigma_{3}}\;,
\end{equation}
with \begin{equation}
\label{xi}
\xi(z,\pmb{t})= \sum_{j=1}^{+\infty}z^{j}t_{j}
\end{equation}
and $M_{0}(z,\bar{z})$ a traceless matrix compactly supported on $\mathscr{\mathscr{D}}$. 
A $\tau$-function of the Kadomtsev-Petviashvili  hierarchy, $\tau(\pmb{t})$,  can be characterized as a function of  (formally) an infinite number of variable which satisfies the Hirota Bilinear relation
\begin{equation}
  \label{eq:Hirota}
  \text{Res}_{z=\infty} (\tau(\pmb{t} -[z^{-1}])\tau(\pmb{s} + [z^{-1}]){\rm e}^{\xi(z,\pmb{t}) - \xi(z,\pmb{s})}=0
\end{equation}
where $\pmb{t} \pm [z^{-1}]$ is the \textit{Miwa Shift}, defined as:
\begin{equation}
  \label{eq:Miwa}
 \pmb{t} \pm [z^{-1}]:= \le(t_1 \pm \frac{1}{z}, t_2 \pm \frac{1}{2 z^{2}}, \dots, t_j \pm \frac{1}{j z^{j}}, \dots\ri).
\end{equation}
 The residue in \eqref{eq:Hirota} is meant in the formal sense, namely by considering the coefficient of $z^{-1}$ in the expansion at infinity and can be thought of as 
 the limit of $\oint_{|z|=R}$ as $R\to+\infty$.  If the functions of $z$  intervening  in \eqref{eq:Hirota} can be written as analytic functions in a deleted neighbourhood of  $\infty$, then the residue is  a genuine integral; this is the case of interest below.

 As described in~\cite{JMD}, the equation~\eqref{eq:Hirota} implies that the tau function satisfy an equation of the Hirota type
 \be
 P(\mathcal{D}_1, \mathcal{D}_2, \dots)\, \tau^2=0
 \ee
 where $\mathcal{D}_j$ is the Hirota derivative respect to $t_j$, defined as
 \be
 \mathcal{D}_j \, p(\pmb{t}) q(\pmb{t}) := (\pa_{t_j} - \pa_{t'_j})(p(\pmb{t}) q(\pmb{t}'))|_{\pmb{t}=\pmb{t}'},
 \ee
 and $P(\mathcal{D}_1, \mathcal{D}_2, \dots)$ is a polynomial in $(\mathcal{D}_1, \mathcal{D}_2, \dots)$. In particular, if we consider the first three times $t_1,t_2 \text{ and } t_3$,  and $t_k=0$ for $k\geq 3$ the equation~\eqref{eq:Hirota} is equivalent to the KP equation in Hirota's form
 \be
 ( 3 \mathcal{D}^2_2 -4 \mathcal{D}_1 \mathcal{D}_3 +\mathcal{D}^{4}_1) \, \tau^2 =0.
 \ee
 Putting 
 \[
   \pa^2_{t_1}\log \tau(t_1,t_2,t_3)=\frac{1}{2}u(t_1,t_2,t_3)
 \]
 one obtains the celebrated KP equation
   \begin{equation}
   \label{KP}
    3   \partial^2_{t_2}u =   \partial_{t_1}(4  \partial_{t_3} u  -   \partial^3_{t_1}  u  - 6 u  \partial_{t_1} u )
    \end{equation}
    

The rest of this section is devoted to the  verification  of   the Hirota bilinear relation \eqref{eq:Hirota}  for the  KP tau function.
\subsection{Hirota bilinear relation for  the KP hierarchy}
The main result is the following.
\begin{theorem}
    \label{th-tau}
    Let $\Gamma(z,\pmb{t})$ be the solution of the $\ov\pa$-problem 
    \[
    \pa_{\bar{z}} \Gamma(z,\pmb{t}) = \Gamma(z,\pmb{t}) {\rm e}^{\frac{\xi(z,\pmb{t})}{2}\sigma_{3}}M_{0}(z){\rm e}^{-\frac{\xi(z,\pmb{t})}{2}\sigma_{3}},\;\qquad \Gamma(z,\pmb{t}) \underset{z \to \infty}{ \to } \1
    \] 
    with the  traceless   matrix $M_0(z)$   compactly supported on   a bounded domain  $\mathscr{D}$ of the complex plane and the function 
    $\xi$ given by the formal sum $\xi(z,\pmb{t})= \sum_{j=1}^{+\infty}z^{j}t_{j}$.
    Then the  function 
    \begin{equation}
    \label{KPtau}
    \tau(\pmb{t})=\exp\left(\int\omega\right),
    \end{equation}
    with $\omega$  defined in \eqref{eq:Malg_form}
     is a KP $\tau$-function; i.e. it satisfies the Hirota Bilinear relation~\eqref{eq:Hirota}.
  \end{theorem}
 
\begin{remark}  In this setting the KP $\tau$-function  is in general complex--valued.  Under appropriate additional symmetry constraints for the matrix $M_0$   and the domain $\mathscr{D}$ we can obtain  a real--valued $\tau$-function.
\end{remark}

We prove the theorem in several steps. We first analyse the effect of the Miwa shifts on the $\tau$-function.
%The main logical steps:
%\begin{itemize}
%\item the determination of the action of the Miwa shifts on $\tau(\pmb{t})$;
%\item the costruction of a matrix $H(z,\ov z)$ from the matrix $\Gamma(z,\ov z)$ which is analytic in $z \in \C$;
%\item the connection between the relation~\eqref{eq:Hirota} with the fact that $(H(z,\ov z))_{11}$ has no poles at infinity.
%\end{itemize}
%\paragraph{Effect of the Miwa shifts on $\tau$.}
For this purpose we need to determine how the Miwa shift acts on the matrices $\Gamma(z,\bar{z},t)$ and $M(z,\bar{z},\pmb{t})$. We consider $M(z,\bar{z},\pmb{t} \pm [\zeta^{-1}])$ first. 
\begin{equation*}
   M(z,\pmb{t} \pm [\zeta^{-1}])= {\rm e}^{\xi(z,\pmb{t}\pm [\zeta^{-1}])\sigma_{3}}M_{0}(z){\rm e}^{-\xi(z,\pmb{t} \pm [\zeta^{-1}])\sigma_{3}}
 \end{equation*}
 from the definition of $\xi(z,t)$~\eqref{xi}
\begin{equation*}
  \begin{split}
   &\xi(z,\pmb{t} \pm [\zeta^{-1}])= \sum^{+\infty}_{j=1}z^{j}\left(t_{j} \pm \frac{1}{j \zeta^{j}} \right)= \sum^{+\infty}_{j=1} z^{j}t_j \pm \sum^{+\infty}_{j=1}\frac{z^{j}}{j \zeta^{j}}= \xi(z,\pmb{t}) \mp \ln\left( 1 - \frac{z}{\zeta}\right)\\
  \end{split}
\end{equation*}
 and we have that
\begin{equation}
  \label{eq:M_s}
  M(z,\pmb{t} \pm [\zeta^{-1}])=\left(1 - \frac{z}{\zeta}\right)^{\mp \frac{\sigma_{3}}{2}}M(z,\pmb{t})\left(1 - \frac{z}{\zeta}\right)^{\pm \frac{\sigma_{3}}{2}}.
\end{equation}
For the matrices $\Gamma(z,\ov z,\pmb{t} \pm [\zeta^{-1}])$ we need to consider the two case separately. Let us start with the negative shift $\Gamma(z,\ov z,\pmb{t} -[\zeta^{-1}])$.
\begin{equation}
  \label{eq:G-shift}
  \begin{split}
    \partial_{\bar{z}}\Gamma(z,\pmb{t} -[\zeta^{-1}])&=\Gamma(z,\pmb{t} -[\zeta^{-1}])M(z,\pmb{t} - [\zeta^{-1}])\\
    &=\Gamma(z,\pmb{t} -[\zeta^{-1}])\left(1 - \frac{z}{\zeta}\right)^{+ \frac{\sigma_{3}}{2}}M(z,\pmb{t})\left(1 - \frac{z}{\zeta}\right)^{- \frac{\sigma_{3}}{2}}\\
    &=\Gamma(z,\pmb{t} -[\zeta^{-1}])D(z,\zeta)M(z,\pmb{t})D^{-1}(z,\zeta)\\
  \end{split}
\end{equation}
where
\be
D(z,\zeta)=
    \begin{bmatrix}
      1 - \frac{z}{\zeta} & 0\\
      0 & 1
    \end{bmatrix}. \nn
\ee
From~\eqref{eq:G-shift}, we notice that the matrix $\Gamma(z,t - [\zeta^{-1}])D(z,\zeta)$ satisfies the $\ov\pa$-problem~\ref{dbarproblem}, i.e. there exists a connection matrix $C(z)$ such that
\begin{equation}
  \label{eq:Gamma-1}
  \Gamma(z,\pmb{t} -[\zeta^{-1}])= C(z)\Gamma(z,\pmb{t})D(z,\zeta)^{-1},
\end{equation}
where obviously $C(z)$ depends also on $\zeta$ and $\pmb{t}$.

 The matrix $C(z)$ is determined by the conditions that both $\Gamma(z,\pmb{t})$ and $\Gamma(z,\ov z, \pmb{t}-[\zeta^{-1}])$ must tend to $\1$ for $z\to \infty$ and are regular at $z=\zeta$
\begin{equation}
  \label{eq:R-cond}
  \begin{split}
    \lim_{z\to \infty}\left(1 - \frac{z}{\zeta}\right)^{-1}&C(z)
    \begin{bmatrix}
      \Gamma_{1 1}(z,\pmb{t})\\
      \Gamma_{1 2}(z,\pmb{t})
    \end{bmatrix}=
    \begin{bmatrix}
      1\\
      0
    \end{bmatrix}
\quad  \lim_{z\to \infty}C(z)
    \begin{bmatrix}
     \Gamma_{2 1}(z,\pmb{t})\\
      \Gamma_{2 2}(z,\pmb{t})
    \end{bmatrix}=
    \begin{bmatrix}
      0\\
      1
    \end{bmatrix}\\
    &\lim_{z \to \zeta} \left(1 - \frac{z}{\zeta}\right)^{-1}C(z)
    \begin{bmatrix}
      \Gamma_{1 1}(z,\pmb{t})\\
      \Gamma_{1 2}(z,\pmb{t})
    \end{bmatrix}=
    \begin{bmatrix}
      \Gamma_{1 1}(\zeta,\pmb{t})\\
      0
    \end{bmatrix}
  \end{split}
\end{equation}
Solving the  system~\eqref{eq:R-cond}, we obtain that the matrix $C(z)$ has the following form
\begin{equation}
  \label{eq:R-mat}
  C(z)=
  \begin{bmatrix}
    \left(1 - \frac{z}{\zeta}\right) + \frac{\partial_z\Gamma_{12}(\infty)\Gamma_{21}(\zeta)}{\zeta \Gamma_{11}(\zeta)} & -\frac{\partial_z\Gamma_{12}(\infty)}{\zeta}\\
    -\frac{\Gamma_{21}(\zeta)}{\Gamma_{11}(\zeta)} & 1
  \end{bmatrix}.
\end{equation}

Following the same ideas, we can find a similar formula for $\Gamma(z,\ov z,t + [\zeta^{-1}])$
\begin{equation}
  \label{eq:Gamma-2}
  \Gamma(z,\pmb{t} + [\zeta^{-1}])= \tilde{C}(z)\Gamma(z,\pmb{t})\tilde{D}(z,\zeta)^{-1}
\end{equation}
with
\begin{equation*}
  \tilde{D}(z,\zeta)=
    \begin{bmatrix}
      1 & 0\\
      0 & 1 - \frac{z}{\zeta}
    \end{bmatrix}.
  \end{equation*}
  Also in this case, we have three conditions similar to~\eqref{eq:R-cond} :
  \begin{equation}
  \label{eq:R-cond2}
  \begin{split}
    \lim_{z\to \infty}\tilde{C}(z)
    &\begin{bmatrix}
      \Gamma_{1 1}(x,\pmb{t})\\
      \Gamma_{1 2}(x,\pmb{t})
    \end{bmatrix}=
    \begin{bmatrix}
      1\\
      0
    \end{bmatrix}
\quad  \lim_{z\to \infty} \left(1 - \frac{z}{\zeta}\right)^{-1}\tilde{R}(z,\zeta)
    \begin{bmatrix}
      \Gamma_{2 1}(x,\pmb{t})\\
      \Gamma_{2 2}(x,\pmb{t})
    \end{bmatrix}=
    \begin{bmatrix}
      0\\
      1
    \end{bmatrix}\\
    &\lim_{z \to x} \left(1 - \frac{z}{\zeta}\right)^{-1}\tilde{C}(z)
    \begin{bmatrix}
      \Gamma_{2 1}(x,\pmb{t})\\
      \Gamma_{2 2}(x,\pmb{t})
    \end{bmatrix}=
    \begin{bmatrix}
      0\\
      \Gamma_{22}(\zeta,\pmb{t})
    \end{bmatrix}
  \end{split}
\end{equation}
and we find out that $\tilde{C}(z)$ has the following form:
\begin{equation}
  \label{eq:R-mat2}
 \tilde{C}(z)=
  \begin{bmatrix}
    1  & -\frac{\Gamma_{12}(\zeta)}{\Gamma_{22}(\zeta)}\\
    -\frac{\partial_z\Gamma_{21}(\infty)}{\zeta} & \left(1 - \frac{z}{\zeta}\right) + \frac{\partial_z\Gamma_{21}(\infty)\Gamma_{12}(\zeta)}{\zeta \Gamma_{22}(\zeta)}
  \end{bmatrix}.
\end{equation}

We need to show how the Miwa shift acts on the Malgrange one form. We define $\delta_{[\zeta]}$ the differential deformed including the external parameter $\zeta$
\begin{equation}
  \label{eq:def-delt}
    \delta_{[\zeta]} := \sum^{+\infty}_{j=1} \d t_{j} \partial_{t_{j}} + \d \zeta \partial_{\zeta}= \delta + \delta_{\zeta}.
  \end{equation}

\begin{lemma}
  \label{lemm-2}
  For $\zeta \notin \mathscr{D}$, then the Miwa shift~\eqref{eq:Miwa} acts on the Malgrange one form~\eqref{eq:Malg_form} in the following way
  \be
  \label{eq:lemm-2}
 \omega(\pmb{t} \pm [\zeta^{-1}])= \omega(\pmb{t}) + \delta_{[\zeta]}\ln\le((\Gamma^{\mp 1}(\zeta))_{11}\ri) \mp \delta_{[\zeta]}\gamma(\zeta)
  \ee
  where $\Gamma(z)$ solves the $\ov\pa$-problem \ref{dbarproblem} and $\gamma(\zeta)$ is a $\pmb{t}$ independent function defined as
  \be
  \label{eq:gam}
  \gamma(\zeta):= \iint_{\mathscr{D}}\log\left(\frac{\zeta}{\zeta-z}\right)(\partial_z M_{0}(z))_{11}\frac{ \d \ov{z} \wedge \d z}{2\pi i}
 \ee
  which is analytic for $\zeta \notin \mathscr{D}$ and goes to zero as $\zeta \to \infty$.
\end{lemma}
Observe that since $\Tr M_0=0$ we may express the formula in terms of the $(2,2)$ entry instead. 
The proof of this lemma is presented in the Appendix~\ref{app-II}.
Now we can state the following proposition:
% Define the \textit{Baker-Akhiezer function} and its dual as
% \be
% \Phi(\zeta,\pmb{t}):= \frac{\tau(\pmb{t} -[\zeta^{-1}])}{\tau(\pmb{t})} \quad \Phi^{*}(\zeta,\pmb{t}):=\frac{\tau(\pmb{t} + [\zeta^{-1}])}{\tau(\pmb{t})}
% \ee
% then we can prove the following proposition:

\begin{proposition}
  For $\zeta \notin \mathscr{D}$, then the following relations holds: 
  \begin{equation}
    \label{eq:tau-M}
    \frac{\tau(\pmb{t} -[\zeta^{-1}])}{\tau(\pmb{t})} = \Gamma_{11}(\zeta,\pmb{t}){\rm e}^{\gamma(\zeta)}  \qquad \frac{\tau(\pmb{t} + [\zeta^{-1}])}{\tau(\pmb{t})} = \Gamma^{-1}_{11}(\zeta,\pmb{t}){\rm e}^{-\gamma(\zeta)}
  \end{equation}
  where $\tau(t)$ is defined in~\eqref{eq:tau},
  $\Gamma(z)$ solves the $\ov\pa$-problem \ref{dbarproblem} and $\gamma(\zeta)$ is defined in~\eqref{eq:gam}
\end{proposition}


 
 %  From~\eqref{eq:tau},~\eqref{eq:M_s} and~\eqref{eq:Gamma-1}, we get that
%   \begin{eqnarray}
%       & &\delta_{[\zeta]}\ln\tau(\pmb{t}-[\zeta^{-1}])= \omega(\pmb{t} -[\zeta^{-1}])= \nn \\ %+ \delta_{[\zeta]}\Tr_{L^2}[\mathcal{K}]= \nonumber\\
%      & &=-\iint_{\mathscr{D}}\Tr_{\mathbb{C}^{2}}[D(z)\Gamma^{-1}(z)C^{-1}(z)\partial(C(z)\Gamma(z)D^{-1}(z))\delta_{[\zeta]}(D(z)M(z,\pmb{t})D^{-1}(z))]\frac{ \d \ov{z} \wedge \d z}{2\pi i}= \nonumber\\
%       & &=-\iint_{\mathscr{D}}\Tr_{\mathbb{C}^{2}}[D(z)\Gamma^{-1}(z)\partial\Gamma(z)D^{-1}(z)\delta_{[\zeta]}(D(z)M(z,\pmb{t})D^{-1}(z))]\frac{ \d \ov{z} \wedge \d z}{2\pi i}  + \label{eq:A}\\
%      & &-\iint_{\mathscr{D}}\Tr_{\mathbb{C}^{2}}[D(z)\Gamma^{-1}(z)C^{-1}(z)\partial C(z)\Gamma(z)D^{-1}(z)\delta_{[\zeta]}(D(z)M(z,\pmb{t})D^{-1}(z))]\frac{ \d \ov{z} \wedge \d z}{2\pi i} + \label{eq:B}\\
%      & &-\iint_{\mathscr{D}}\Tr_{\mathbb{C}^{2}}[D(z)\partial(D^{-1}(z)\delta_{[\zeta]}(D(z)M(z,\pmb{t})D^{-1}(z))]\frac{ \d \ov{z} \wedge \d z}{2\pi i} \label{eq:C}
%   \end{eqnarray}
 
%   We now consider the three parts \eqref{eq:A}, \eqref{eq:B}, \eqref{eq:C}, separately.
% Let us start by considering \eqref{eq:A}:
%   \begin{equation*}
%     \begin{split}
%       \eqref{eq:A}&= -\iint_{\mathscr{D}}\Tr_{\mathbb{C}^{2}}[\Gamma^{-1}(z)\partial\Gamma(z)\delta M(z,\pmb{t})]\frac{ \d \ov{z} \wedge \d z}{2\pi i} + \\
%       &-\iint_{\mathscr{D}}\Tr_{\mathbb{C}^{2}}[\Gamma^{-1}(z)\partial\Gamma(z)D^{-1}(z)\delta_{\zeta} D(z) M(z,\pmb{t})]\frac{ \d \ov{z} \wedge \d z}{2\pi i}+ \\
%       &+\iint_{\mathscr{D}}\Tr_{\mathbb{C}^{2}}[\Gamma^{-1}(z)\partial\Gamma(z)M(z,\pmb{t})D^{-1}(z)\delta_{\zeta} D(z)]\frac{ \d \ov{z} \wedge \d z}{2\pi i}=\\
%       &=\omega(\pmb{t}) %+ \delta\Tr_{L^{2}}[\mathcal{K}]
%       - \iint_{\mathscr{D}}\Tr_{\mathbb{C}^{2}}[\Gamma^{-1}(z)\partial\Gamma(z)D^{-1}(z)\delta_{\zeta} D(z) M(z,\pmb{t})]\frac{ \d \ov{z} \wedge \d z}{2\pi i}+\\
%        &+ \iint_{\mathscr{D}}\Tr_{\mathbb{C}^{2}}[\Gamma^{-1}(z)\partial\Gamma(z)M(z,\pmb{t})D^{-1}(z)\delta_{\zeta} D(z)]\frac{ \d \ov{z} \wedge \d z}{2\pi i}.
%     \end{split}
%   \end{equation*}
%   Since $\zeta \notin \mathscr{D}$, the matrix $D^{-1}(z)$ is analytic in $\mathscr{D}$ and we can rewrite the two integrals as
%   \begin{equation}
%     \label{eq:A_mid}
%     \begin{split}
%       \eqref{eq:A} &= \omega(\pmb{t}) %+ \delta\Tr_{L^{2}}[\mathcal{K}]
%       + \iint_{\mathscr{D}}\pa_{\bar{z}}\Tr_{\mathbb{C}^{2}}[\Gamma^{-1}(z)\partial\Gamma(z) D^{-1}(z)\delta_{\zeta} D(z)]\frac{ \d \ov{z} \wedge \d z}{2\pi i} +\\
%       &- \iint_{\mathscr{D}}\Tr_{\mathbb{C}^{2}}[\partial M(z,\pmb{t})D^{-1}(z)\delta_{\zeta} D(z)]\frac{ \d \ov{z} \wedge \d z}{2\pi i}.
%     \end{split}
%   \end{equation}
%   We now observe that the last integral is independent of $\pmb{t}$, due to the fact that $D(z)$ is diagonal. Moreover, using
%   \begin{equation*}
%     D^{-1}(z)\delta D(z)= -\frac{z}{\zeta(z-\zeta)} E_{11}\d \zeta,
%   \end{equation*}
%   where $E_{11}= \begin{bmatrix} 1 & 0\\ 0 & 0\end{bmatrix}$, we find 
%   \begin{equation}
%     \label{eq:LI}
%  - \iint_{\mathscr{D}} \Tr_{\mathbb{C}^{2}}[\partial M(z,\pmb{t}) D^{-1}(z)\pa_{\zeta} D(z)]\frac{ \d \ov{z} \wedge \d z}{2\pi i}= \iint_{\mathscr{D}}\frac{z}{\zeta(z-\zeta)}(\partial M_{0}(z))_{11} \frac{ \d \ov{z} \wedge \d z}{2\pi i}.
%   \end{equation}
%  The RHS of~\eqref{eq:LI} equals $\pa_{\zeta} \gamma(\zeta)$.
%  %  \begin{equation}
% %     \label{eq:LI2}
% %  - \iint_{\mathscr{D}} \Tr_{\mathbb{C}^{2}}[\partial M(z,\pmb{t}) D^{-1}(z)\delta D(z)]\frac{ \d \ov{z} \wedge \d z}{2\pi i}= \delta_{\zeta} \gamma(\zeta).
% % \end{equation}

%   Since the integrand of the remaining integral in~\eqref{eq:A_mid} does not have a pole in $\mathscr{D}$, we can use Stokes' Theorem 
%   \begin{equation*}
%     \oint_{\partial \mathscr{D}}\Tr_{\mathbb{C}^{2}}[\Gamma^{-1}(z)\partial\Gamma(z)D^{-1}(z)\partial_{\zeta} D(z)]\frac{\d z}{2\pi i}=\oint_{-\partial \mathscr{D}}\frac{z}{\zeta(z-\zeta)}(\Gamma^{-1}(z)\partial\Gamma(z))_{11}\frac{\d z}{2 \pi i}
%   \end{equation*}
%   where $-\pa \mathscr{D}$ is the border of $\mathscr{D}$ oriented clockwise.
  
%   % Since $\det(\Gamma)=1$, the inverse of $\Gamma(z)$ is simply ${\rm adj}(\Gamma(z))$. This means that
%   % \begin{equation*}
%   %   (\Gamma^{-1}(z)\partial\Gamma(z))_{11}= \Gamma_{22}(z)\partial\Gamma_{11}(z) -\partial\Gamma_{21}(z)\Gamma_{12}(z)
%   % \end{equation*}
%   % and
%   Since $\Gamma(z)$ is analytic outside $\mathscr{D}$, we can apply Cauchy's residue Theorem
%   \begin{equation*}
%  \oint_{-\partial \mathscr{D}}\frac{z}{\zeta(z-\zeta)}(\Gamma_{22}(z)\partial\Gamma_{11}(z) -\partial\Gamma_{21}(z)\Gamma_{12}(z))\frac{\d z}{2 \pi i}= \Gamma_{22}(\zeta)\partial\Gamma_{11}(\zeta) -\partial\Gamma_{21}(\zeta)\Gamma_{12}(\zeta)   
% \end{equation*}
% so that
% \begin{equation}
%   \label{eq:Af}
%   \eqref{eq:A}= \omega(\pmb{t}) %+ \delta\Tr_{L^{2}}[\mathcal{K}]
%   +\left( \Gamma_{22}(\zeta)\partial_{\zeta}\Gamma_{11}(\zeta) - \partial_{\zeta}\Gamma_{21}(\zeta)\Gamma_{12}(\zeta)\right) \d \zeta + \delta_{\zeta} \gamma(\zeta).
% \end{equation}
% Let us consider \eqref{eq:B}: 
% \begin{equation*}
%   \begin{split}
%    \eqref{eq:B} &=-\iint_{\mathscr{D}}\Tr_{\mathbb{C}^{2}}[\Gamma^{-1}(z)C^{-1}(z)\partial C(z)\Gamma(z)\delta M(z,\pmb{t})]\frac{ \d \ov{z} \wedge \d z}{2\pi i}+\\
%     &-\iint_{\mathscr{D}}\Tr_{\mathbb{C}^{2}}[M(z,\pmb{t})\Gamma^{-1}(z)C^{-1}(z)\partial C(z)\Gamma(z)D^{-1}(z) \delta_{\zeta} D(z)]\frac{ \d \ov{z} \wedge \d z}{2\pi i} +\\
%     &+\iint_{\mathscr{D}}\Tr_{\mathbb{C}^{2}}[\Gamma^{-1}(z)C^{-1}(z)\partial C(z)\Gamma(z)M(z,\pmb{t})D^{-1}(z)\delta_{\zeta} D(z)]\frac{ \d \ov{z} \wedge \d z}{2\pi i}=\\
%     &=-\iint_{\mathscr{D}}\Tr_{\mathbb{C}^{2}}[\Gamma^{-1}(z)C^{-1}(z)\partial C(z)\Gamma(z)\delta M(z,\pmb{t})]\frac{ \d \ov{z} \wedge \d z}{2\pi i}+\\
%     &+\iint_{\mathscr{D}}\Tr_{\mathbb{C}^{2}}[\ov{\pa}\Gamma^{-1}(z)C^{-1}(z)\partial C(z)\Gamma(z)D^{-1}(z) \delta_{\zeta} D(z)]\frac{ \d \ov{z} \wedge \d z}{2\pi i}+\\
%     &+\iint_{\mathscr{D}}\Tr_{\mathbb{C}^{2}}[\Gamma^{-1}(z)C^{-1}(z)\partial C(z)\ov{\pa}\Gamma(z)D^{-1}(z)\delta_{\zeta} D(z)]\frac{ \d \ov{z} \wedge \d z}{2\pi i}. 
%  \end{split}
% \end{equation*}
% Since the only singularity is at $z=\zeta$, which is outside the domain $\mathscr{D}$, we can apply Stokes' Theorem to the integration and we get
% \begin{equation*}
%   \begin{split}
%    \eqref{eq:B} &=-\iint_{\mathscr{D}}\Tr_{\mathbb{C}^{2}}[C^{-1}(z)\partial C(z)\Gamma(z)\delta M(z,\pmb{t})\Gamma^{-1}(z)]\frac{ \d \ov{z} \wedge \d z}{2\pi i} +\\
%     &+\oint_{\pa \mathscr{D}}\Tr_{\mathbb{C}^{2}}[\Gamma^{-1}(z)C^{-1}(z)\partial C(z)\Gamma(z)D^{-1}(z)\delta_{\zeta} D(z)]\frac{\d z}{2\pi i}.
% \end{split}
% \end{equation*}
% Now observe that
% \begin{equation}
%   \label{eq:I_int_B}
%   \Gamma(z,\ov z,\pmb{t}) \delta M(z,\ov z,\pmb{t}) \Gamma^{-1}(z,\ov z,\pmb{t})=\pa_{\bar{z}}[(\delta \Gamma(z,\ov z,\pmb{t})) \Gamma^{-1}(z,\ov z,\pmb{t})]
% \end{equation}
% % \begin{equation*}
% %   \delta \pa_{\bar{z}} \Gamma(z,\ov z,\pmb{t})= \delta \Gamma(z,\ov z,\pmb{t}) M(z,\ov z,\pmb{t}) + \Gamma(z,\ov z,\pmb{t})\delta M(z,\ov z,\pmb{t})
% % \end{equation*}

% % we multiply it on the right by $\Gamma^{-1}(z,\ov z,\pmb{t})$ and we get
% % \begin{equation*}
% %   \begin{split}
% %     (\pa_{\bar{z}} \delta \Gamma(z,\ov z,\pmb{t}))\Gamma^{-1}(z,\ov z,\pmb{t})&=\delta \Gamma(z,\ov z,\pmb{t}) M(z,\ov z,\pmb{t})\Gamma^{-1}(z,\ov z,\pmb{t}) +\\
% %     &+ \Gamma(z,\ov z,\pmb{t})\delta M(z,\ov z,\pmb{t})\Gamma^{-1}(z, \ov z,\pmb{t}) =\\
% %     &= -\delta \Gamma(z,\ov z,\pmb{t})\pa_{\bar{z}}\Gamma^{-1}(z,\ov z,\pmb{t}) + \Gamma(z,\ov z,\pmb{t})\delta M(z,\ov z,\pmb{t})\Gamma^{-1}(z, \ov z,\pmb{t})
% %   \end{split}
% % \end{equation*}
% Using~\eqref{eq:I_int_B} in the first integral of \eqref{eq:B}, we can rewrite it as a contour integral
% \begin{equation*}
%   \begin{split}
%     &-\iint_{\mathscr{D}}\Tr_{\mathbb{C}^{2}}[\Gamma^{-1}(z)C^{-1}(z)\partial C(z)\Gamma(z)\delta M(z,\pmb{t})]\frac{ \d \ov{z} \wedge \d z}{2\pi i} =\\
%     &=-\iint_{\mathscr{D}}\pa_{\bar{z}}\Tr_{\mathbb{C}^{2}}[C^{-1}(z)\partial C(z)\delta \Gamma(z) \Gamma^{-1}(z)]\frac{ \d \ov{z} \wedge \d z}{2\pi i}=\\
%     &= \oint_{-\partial \mathscr{D}}\Tr_{\mathbb{C}^{2}}[C^{-1}(z)\partial C(z)\delta \Gamma(z) \Gamma^{-1}(z)]\frac{\d z}{2\pi i}
%     \end{split}
%   \end{equation*}
%   and now we can compute it. We now use that
%   \begin{equation*}
%     C^{-1}(z)= \frac{1}{\det C(z)} {\rm adj}(C(z))= \frac{1}{\left(1 - \frac{z}{\zeta}\right)}
%     \begin{bmatrix}
%       1 & -\frac{\partial\Gamma_{12}(\infty)}{\zeta}\\
%       \frac{\Gamma_{21}(\zeta)}{\Gamma_{11}(\zeta)} & \left(1 - \frac{z}{\zeta}\right) - \frac{\partial \Gamma_{12}(\infty) \Gamma_{12}(\zeta)}{\zeta\Gamma_{11}(\zeta)}
%     \end{bmatrix}
%   \end{equation*}
%   \begin{equation*}
%     \partial C(z) = -\frac{1}{\zeta} E_{11}
%   \end{equation*}
%   and we expand the trace below  
%   \begin{equation*}
%     \begin{split}
%       &\Tr_{\mathbb{C}^{2}}[C^{-1}(z)\partial C(z)\delta\Gamma(z) \Gamma^{-1}(z)]=\frac{1}{(z-\zeta)}\left[(\delta \Gamma(z)\Gamma^{-1}(z))_{11} + \frac{\Gamma_{21}(\zeta)}{\Gamma_{11}(\zeta)}(\delta \Gamma(z)\Gamma^{-1}(z))_{12}\right]=\\
%       &=\frac{\delta\Gamma_{11}(z)\Gamma_{22}(z) - \delta \Gamma_{12}(z)\Gamma_{21}(z)}{(z - \zeta)} + \frac{\Gamma_{21}(\zeta)}{\Gamma_{11}(\zeta)}\left[\frac{\delta \Gamma_{12}(z)\Gamma_{11}(z) - \delta \Gamma_{11}(z) \Gamma_{12}(z)}{(z - \zeta)}\right].
%     \end{split}
%   \end{equation*}
%   At the end, the first integral in~\eqref{eq:B} becomes
%   \begin{equation*}
%     \begin{split}
%       \oint_{-\partial \mathscr{D}}\Tr_{\mathbb{C}^{2}}[C^{-1}(z)\partial C(z)&\delta \Gamma(z) \Gamma^{-1}(z)]\frac{\d z}{2\pi i} = \delta \Gamma_{11}(\zeta)\Gamma_{22}(\zeta) - \frac{\delta \Gamma_{11}(\zeta)}{\Gamma_{11}(\zeta)}\Gamma_{12}(\zeta)\Gamma_{21}(\zeta)\\
%       &= \frac{\delta \Gamma_{11}(\zeta)}{\Gamma_{11}(\zeta)}= \delta \ln \Gamma_{11}(\zeta).
%      \end{split}
%   \end{equation*}
%   The second integral is a little bit more complex. Indeed, expanding the trace we find that 
%   \begin{equation*}
%     \Tr_{\mathbb{C}^{2}}[\Gamma^{-1}(z)C^{-1}(z)\partial C(z)\Gamma(z)D^{-1}(z)\partial_{\zeta} D(z)]= -\frac{z}{\zeta(z-\zeta)^{2}}\Gamma_{11}(z)\left[\Gamma_{22}(z) - \frac{\Gamma_{12}(z)\Gamma_{21}(\zeta)}{\Gamma_{11}(\zeta)}\right].
%   \end{equation*}
%   So we have a contour integral with a double pole in $z=\zeta$ and a simple pole in $z=\infty$:
%   \begin{equation*}
%     \begin{split}
%       &\oint_{\pa \mathscr{D}}\Tr_{\mathbb{C}^{2}}[\Gamma^{-1}(z)C^{-1}(z)\partial C(z)\Gamma(z)D^{-1}(z)\partial_{x} D(z)]\frac{\d z}{2\pi i}=\\
%       &=\oint_{-\partial \mathscr{D}}\frac{z}{\zeta(z-\zeta)^{2}}\Gamma_{11}(z)\left[\Gamma_{22}(z) - \frac{\Gamma_{12}(z)\Gamma_{21}(\zeta)}{\Gamma_{11}(\zeta)}\right]\frac{\d z}{2\pi i}=\\
%       &=\lim_{z\to \zeta}\partial \left[\frac{z}{\zeta}\Gamma_{11}(\zeta)\left(\Gamma_{22}(z) - \frac{\Gamma_{12}(z)\Gamma_{21}(\zeta)}{\Gamma_{11}(\zeta)}\right)\right] +\\
%       &\quad+ \text{Res}_{z = \infty}\frac{z}{\zeta(z-\zeta)^{2}}\Gamma_{11}(z)\left[\Gamma_{22}(z) - \frac{\Gamma_{12}(z)\Gamma_{21}(\zeta)}{\Gamma_{11}(\zeta)}\right]=\\
%       &=\lim_{z \to \zeta}\frac{\Gamma_{11}(z)}{\zeta}\left(\Gamma_{22}(z) - \frac{\Gamma_{12}(z)\Gamma_{21}(\zeta)}{\Gamma_{11}(\zeta)}\right) + \lim_{z \to \zeta}\frac{z}{\zeta}\partial\Gamma_{11}(z)\left(\Gamma_{22}(z) - \frac{\Gamma_{12}(z)\Gamma_{21}(\zeta)}{\Gamma_{11}(\zeta)}\right)+\\
%       &\quad+\lim_{z \to \zeta}\frac{z}{\zeta}\Gamma_{11}(\zeta)\left(\partial\Gamma_{22}(z) - \frac{\partial\Gamma_{12}(z)\Gamma_{21}(\zeta)}{\Gamma_{11}(\zeta)}\right)+\\
%       &\quad +\text{Res}_{z = \infty}\left[\frac{z}{\zeta(z-\zeta)^{2}}\Gamma_{11}(z)\left(\Gamma_{22}(z) - \frac{\Gamma_{12}(z)\Gamma_{21}(\zeta)}{\Gamma_{11}(\zeta)}\right)\right]=\\
%       &= \frac{1}{\zeta} + \partial_{\zeta}\ln(\Gamma_{11}(\zeta)) + \Gamma_{11}(\zeta)\partial_{\zeta}\Gamma_{11}(\zeta) - \Gamma_{21}(\zeta)\partial_{\zeta} \Gamma_{12}(\zeta) +\\
%       &\quad+\text{Res}_{z = \infty}\left[\frac{z}{\zeta(z-\zeta)^{2}}\Gamma_{11}(z)\left(\Gamma_{22}(z) - \frac{\Gamma_{12}(z)\Gamma_{21}(\zeta)}{\Gamma_{11}(\zeta)}\right)\right].
%     \end{split}
%   \end{equation*}
%   For the residue at $z=\infty$ we need to expand the integrand for $k =1/z$ as $k$ goes to zero:
%   \begin{equation*}
%      \frac{z}{\zeta(z-\zeta)^{2}}\Gamma_{11}(z)\left(\Gamma_{22}(z) - \frac{\Gamma_{12}(z)\Gamma_{21}(\zeta)}{\Gamma_{11}(\zeta)}\right)\d z=-\frac{\Gamma_{11}(k)}{\zeta k(1 -\zeta k)^{2}} \left(\Gamma_{22}(k) - \frac{\Gamma_{12}(k)\Gamma_{21}(\zeta)}{\Gamma_{11}(\zeta)}\right) dk
%    \end{equation*}
%    considering that, as $k \sim 0$, the matrix $\Gamma(k)$ tends to the identity the integrand follows this expansion:
%    \begin{equation*}
%     \left(-\frac{1}{k\zeta} + \mathcal{O}(1)\right)dk
%   \end{equation*}
%   this means that the residue at infinity is $-\frac{1}{\zeta}$.

%   So we have that \eqref{eq:B} is:
%   \begin{equation}
%     \label{eq:Bf}
%     \eqref{eq:B}= \delta_{[\zeta]}(\ln(\Gamma_{11}(\zeta))) +\left( \Gamma_{11}(\zeta)\partial_{\zeta}\Gamma_{22}(\zeta) - \Gamma_{21}(\zeta)\partial _{\zeta}\Gamma_{12}(\zeta) \right)\d \zeta.
%   \end{equation}
%   Let us consider \eqref{eq:C}
%   \begin{equation*}
%     \begin{split}
%       \eqref{eq:C}&= \iint_{\mathscr{D}}\Tr_{\mathbb{C}^{2}}[D^{-1}(z)\partial D(z)\delta M(z,\pmb{t})]\frac{ \d \ov{z} \wedge \d z}{2\pi i}+\\
%       &\quad-\iint_{\mathscr{D}}\Tr_{\mathbb{C}^{2}}[D^{-1}(z)\partial D(z)[M(z,\pmb{t});D^{-1}(z)\delta_{\zeta} D(z)]]\frac{ \d \ov{z} \wedge \d z}{2\pi i}=\\
%       &= -\iint_{\mathscr{D}}\Tr_{\mathbb{C}^{2}}[\delta\xi(z,t)D^{-1}(z)\partial D(z) [M(z,\pmb{t});\sigma_{3}]]\frac{ \d \ov{z} \wedge \d z}{2\pi i} +\\
%       &\quad-\iint_{\mathscr{D}}\Tr_{\mathbb{C}^{2}}[D^{-1}(z)\partial D(z)[M(z,\pmb{t});D^{-1}(z)\delta_{\zeta} D(z)]]\frac{ \d \ov{z} \wedge \d z}{2\pi i}
%       \end{split}
%     \end{equation*}
%     and those integrals are identically zero. Indeed
%     \begin{equation*}
%       \partial D(z)= -\frac{1}{\zeta}E_{11} \implies D^{-1}(z)\partial D(z)= \frac{1}{(z-\zeta)}E_{11}
%     \end{equation*}
%     so the traces give us the first diagonal elements of the commutators $[M(z,\pmb{t});\sigma_{3}]$ and $[M(z,\pmb{t});D^{-1}(z)\delta D(z)]$, which are zeros.
%     In the end, adding the equations~\eqref{eq:Af} and~\eqref{eq:Bf}, we get that:
%     \begin{equation}
%       \label{eq:res-p1}
%       \delta_{[\zeta]}\ln\tau(\pmb{t}-[\zeta^{-1}])=\omega(\pmb{t}) %+ \delta\Tr_{L^{2}}[\mathcal{K}]
%       + \delta_{[\zeta]} \ln(\Gamma_{11}(\zeta)) + \delta_{[\zeta]} \gamma(\zeta) .
%     \end{equation}
%     Substituting $C(z)$ and $D(z)$ with $\tilde{C}(z)$ and $\tilde{D}(z)$ respectively and using the nonsingular condition for $K$~\eqref{eq:K_ker2}, we find $\omega(\pmb{t} + [\zeta^{-1}])$ with similar calculations and we get the following result
%     \begin{equation}
%       \label{eq:res-p2}
%       \delta_{[\zeta]}\ln\tau(\pmb{t}+[\zeta^{-1}])= \omega(\pmb{t})+\delta\Tr_{L^{2}}[\mathcal{K}] + \delta_{[\zeta]}\ln(\Gamma^{-1}_{11}(\zeta)) - \delta_{[\zeta]}\gamma(\zeta).
%     \end{equation}
\noindent {\bf Proof.}
    From Lemma~\ref{lemm-2} and the equation~\eqref{eq:tau}, we  rewrite~\eqref{eq:lemm-2} as 
    \begin{equation}
      \label{eq:lnt}
      \delta_{[\zeta]} \ln \tau(\pmb{t} \pm [\zeta^{-1}])= \delta_{[\zeta]} \ln\tau(\pmb{t}) + \delta_{[\zeta]}\ln((\Gamma^{\mp 1}(\zeta))_{11}) \mp \delta_{[\zeta]}\gamma(\zeta)
    \end{equation}
    an then, from the properties of the  logarithm the statement~\eqref{eq:tau-M} is proved.
  \QED

\begin{remark}
  The exponential term ${\rm e}^{\gamma(\zeta)}$ could be absorbed by a gauge transformation in the formalism of the infinite dimensional Grassmannian manifold of Segal-Wilson (\cite{Harnad2021}, Chapter 4). Such  gauge transformations have no effect on the Hirota bilinear relation~\eqref{eq:Hirota}.
\end{remark}

  
%  \begin{remark}
%    If $\zeta \in \mathscr{D}$ the $\ov{\pa}$-Problem~\ref{eq:G-shift} is still well defined, but the Malgrange one-form~\eqref{eq:Malg_form} is not. Indeed, the $\delta_{[\zeta]}$ operator gives rise to a double pole in $z=\zeta$, which is not integrable with respect to the measure $\d^2 z$. \todo{M: where is this double pole?}
%  \end{remark}
%  \todo[inline]{G: Because $D^{-1}(z)\delta_{\zeta}D(z)$ and $C^{-1}(z)\pa_{z}C(z)$ are not analytic for $\zeta \in \mathscr{D}$, so when we try to use Stokes' theorem we need to consider also it's $\pa_{\bar{z}}$, and this creates some probems.}
  
 Let us now  define the matrix $H(z)$ as
  \begin{equation}
      \label{eq:lemm-1}
      H(z):=H(z; \pmb{t}, \pmb{s}):= \Gamma(z,\pmb{t}){\rm e}^{(\xi(z,\pmb{t}) -\xi(z,\pmb{s}))E_{11}}\Gamma^{-1}(z,\pmb{s})
  \end{equation}
  where $E_{11}= \begin{bmatrix} 1 & 0\\ 0 & 0\end{bmatrix}$, $\Gamma(z,\ov z, \pmb{t})$ solves the $\ov\pa$-problem~\ref{dbarproblem} and $\pmb{s}= (s_1, s_2, \dots, s_j,\dots)$ denotes another set of values for the deformation parameters. 
  
  \begin{lemma}
    \label{lemma-1}
    %  Given $\Gamma(z,\ov z, \pmb{t})$ solution of the $\ov\pa$-problem~\ref{dbarproblem}, then the matrix
    % \begin{equation}
    %   \label{eq:lemm-1}
    %   H(z,\ov z):=\Gamma(z,\ov z,\pmb{t}){\rm e}^{(\xi(z,\pmb{t}) -\xi(z,\pmb{s}))E_{11}}\Gamma^{-1}(z,\ov z,\pmb{s})
    % \end{equation}
    
    The matrix $H(z)$ defined in~\eqref{eq:lemm-1} is analytic for all $z \in \mathbb{C}$.
  \end{lemma}

  \noindent {\bf Proof.}
    For $z \notin \mathscr{D}$ the statement is trivial, so we consider the case of $z \in \mathscr{D}$.
    
    We apply the operator $\pa_{\bar{z}}$ to the matrix~\eqref{eq:lemm-1}
    \begin{equation*}
      \begin{split}
        \pa_{\bar{z}} H(z)&=\pa_{\bar{z}}\Gamma(z,\pmb{t}){\rm e}^{(\xi(z,\pmb{t}) -\xi(z,\pmb{s}))E_{11}}\Gamma^{-1}(z,\pmb{s}) +\Gamma(z,\pmb{t}){\rm e}^{(\xi(z,\pmb{t}) -\xi(z,\pmb{s}))E_{11}}\pa_{\bar{z}}\Gamma^{-1}(z,\pmb{s})\\
        &= \Gamma(z,\pmb{t})M(z,\pmb{t}){\rm e}^{(\xi(z,\pmb{t}) -\xi(z,\pmb{s}))E_{11}}\Gamma^{-1}(z,\pmb{s})  +\\
        &\quad -\Gamma(z,\pmb{t}){\rm e}^{(\xi(z,\pmb{t}) -\xi(z,\pmb{s}))E_{11}}M(z,\pmb{s})\Gamma^{-1}(z,\pmb{s})\\
        &=\Gamma(z,\pmb{t})\left({\rm e}^{\frac{\xi(z,\pmb{t})}{2}\sigma_{3}}M_{0}(z){\rm e}^{-\frac{\xi(z,\pmb{t})}{2}\sigma_{3}}{\rm e}^{(\xi(z,\pmb{t}) -\xi(z,\pmb{s}))E_{11}} +\right. \\
        &\quad \left.-{\rm e}^{(\xi(z,\pmb{t}) -\xi(z,\pmb{s}))E_{11}}{\rm e}^{\frac{\xi(z,\pmb{s})}{2}\sigma_{3}}M_{0}(z){\rm e}^{-\frac{\xi(z,\pmb{s})}{2}\sigma_{3}}\right)\Gamma^{-1}(z,\pmb{s})\\
        &=\Gamma(z,\pmb{t})\left({\rm e}^{\frac{\xi(z,\pmb{t})}{2}\sigma_{3}}{\rm e}^{\frac{\xi(z,\pmb{t})}{2}\mathbb{I}}M_{0}(z){\rm e}^{-\xi(z,\pmb{s})E_{11}} +\right. \\
        &\quad \left. -{\rm e}^{\xi(z,\pmb{t})E_{11}}M_{0}(z){\rm e}^{\frac{\xi(z,\pmb{s})}{2}\mathbb{I}}{\rm e}^{-\frac{\xi(z,\pmb{s})}{2}\sigma_{3}}\right)\Gamma^{-1}(z,\pmb{s})\\
        &=\Gamma(z,\pmb{t})\left({\rm e}^{\xi(z,\pmb{t})E_{11}}M_{0}(z){\rm e}^{-\xi(z,\pmb{s})E_{11}} -{\rm e}^{\xi(z,\pmb{t})E_{11}}M_{0}(z){\rm e}^{-\xi(z,\pmb{s})E_{11}} \right)\Gamma^{-1}(z,\pmb{s})\\
        &= 0
      \end{split}
    \end{equation*}
    and this proves the statement.
  \QED

  From the Lemma~\ref{lemma-1} we can also state the following corollary, whose proof follows at once from Cauchy's residue theorem and the fact that $H$ is entire:
  \begin{corollary}
    \label{coroll}
    Given $H(z)$ defined in~\eqref{eq:lemm-1}, then
    \begin{equation}
      \oint_{|z|=R} H(z)\frac{\d z}{2 \pi i} = 0_{2 \times 2}
    \end{equation}
  \end{corollary}

  
  % \begin{lemma}
  %   \label{lemma-2}
  %   \begin{equation}
  %     \label{eq:lemm-2}
  %     \left(\oint_{|z|=R} H(z)\frac{\d z}{2\pi i}\right)_{11}=\oint_{|z|=R}\Gamma_{11}(z,\pmb{t}){\rm e}^{\xi(z,\pmb{t}) -\xi(z,\pmb{s})}(\Gamma^{-1}(z,\pmb{s}))_{11}\frac{\d z}{2\pi i}
  %   \end{equation}
  % \end{lemma}

  % \noindent {\bf Proof.}
  %   From a straightforward computation, it is easy to show that:
  %   \begin{equation*}
  %     H(z)_{11}= \Gamma_{11}(z,\pmb{t}){\rm e}^{\xi(z,\pmb{t}) -\xi(z,\pmb{s})}(\Gamma^{-1}(z,\pmb{s}))_{11} - \Gamma_{12}(z,\pmb{t})\Gamma_{21}(z,\pmb{s})
  %   \end{equation*}
  %   so the we get
  %   \begin{equation*}
  %     \begin{split}
  %       \left(\oint_{|z|=R} H(z)\frac{\d z}{2\pi i}\right)_{11} &=\oint_{|z|=R}\Gamma_{11}(z,\pmb{t}){\rm e}^{\xi(z,\pmb{t}) -\xi(z,\pmb{s})}(\Gamma^{-1}(z,\pmb{s}))_{11}\frac{\d z}{2\pi i} +\\
  %       &\quad - \oint_{|z|=R}\Gamma_{12}(z,\pmb{t})\Gamma_{21}(z,\pmb{s})\frac{\d z}{2\pi i}.
  %       \end{split}
  %     \end{equation*}
  %     Since, for $R$ large enough, both $\Gamma_{12}(z,\pmb{t})$ and $\Gamma_{21}(z,\pmb{s})$ are analytic outside the disk $|z|=R$ and since, for $z \sim \infty$
  %     \begin{equation*}
  %       \Gamma(z,\pmb{t}) \sim \mathbb{I} + \mathcal{O}(z^{-1})
  %     \end{equation*}
  %     then the product $\Gamma_{12}(z)\Gamma_{21}(z)$ is of order $\mathcal{O}(z^{-2})$, which means that  we do not have a pole at  $\infty$. So, by Cauchy's residue Theorem, the Lemma is proved. 
  %     \QED

  We are now ready to  prove the main result of the section,  namely  Theorem~\ref{th-tau}.
  
  

    \noindent {\bf Proof   of Theorem~\ref{th-tau}}
      Let us compute the residue
      \bea
      \label{eq:proof-hir}
         & \text{Res}_{z=\infty}(\tau(\pmb{t} -[z^{-1}])\tau(\pmb{s} + [z^{-1}]){\rm e}^{\xi(z,\pmb{t}) - \xi(z,\pmb{s})}) \nn\\
         & = \tau(\pmb{t})\tau(\pmb{s})\text{Res}_{z=\infty}(\frac{\tau(\pmb{t} -[z^{-1}])}{\tau(\pmb{t})}\frac{\tau(\pmb{s} + [z^{-1}])}{\tau(\pmb{s})}{\rm e}^{\xi(z,\pmb{t}) - \xi(z,\pmb{s})})\nn\\
         & =\tau(\pmb{t})\tau(\pmb{s})\text{Res}_{z=\infty}(\Gamma_{11}(z,\pmb{t})(\Gamma^{-1}(z,\pmb{s}))_{11}{\rm e}^{\xi(z,\pmb{t}) - \xi(z,\pmb{s})}) \nn\\
         & = \tau(\pmb{t})\tau(\pmb{s})\lim_{R\to \infty}\oint_{|z|=R}\Gamma_{11}(z,\pmb{t}){\rm e}^{\xi(z,\pmb{t}) - \xi(z,\pmb{s})}(\Gamma^{-1}(z,\pmb{s}))_{11}\frac{\d z}{2\pi i}
        \eea
        Consider the first diagonal element of the matrix $H(z)$. From the corollary~\ref{coroll} we get that 
        \begin{equation}
          \begin{split}
        \left(\oint_{|z|=R} H(z)\frac{\d z}{2\pi i}\right)_{11} &=\oint_{|z|=R}\Gamma_{11}(z,\pmb{t}){\rm e}^{\xi(z,\pmb{t}) -\xi(z,\pmb{s})}(\Gamma^{-1}(z,\pmb{s}))_{11}\frac{\d z}{2\pi i} +\\
        &\quad - \oint_{|z|=R}\Gamma_{12}(z,\pmb{t})\Gamma_{21}(z,\pmb{s})\frac{\d z}{2\pi i}=0.
        \end{split}
        \end{equation}
        So, we can rewite~\eqref{eq:proof-hir} as
        \be
        \label{eq:proof-hir-2}
        ~\eqref{eq:proof-hir}=\tau(\pmb{t})\tau(\pmb{s})\lim_{R\to \infty}\oint_{|z|=R}\Gamma_{12}(z,\pmb{t})\Gamma_{21}(z,\pmb{s})\frac{\d z}{2\pi i}
        \label{eerw}
        \ee
         Since both $\Gamma_{12}(z,\pmb{t})$ and $\Gamma_{21}(z,\pmb{s})$ are analytic for $|z|$ sufficiently large (given that $\mathscr D$ is compact)  we can omit the limit $\lim_{R\to\infty}$ since the integrals only depend on the homotopy class of the contour.  Moreover  for $z \sim \infty$
      \begin{equation*}
        \Gamma(z,\pmb{t}) \sim \mathbb{I} + \mathcal{O}(z^{-1})
      \end{equation*}
      so that  the product $\Gamma_{12}(z)\Gamma_{21}(z)$ is  $\mathcal{O}(z^{-2})$. Therefore the integral in which means that the integrand does not have a pole as $ z \to \infty$.
      So the integral in~\eqref{eerw} is zero and the statement is proved.
    \QED%
%
    \subsection{The case of focusing Nonlinear Schr\"odinger equation}
    \label{NonSCH}
    In this subsection we  make a specific choice of the matrix $M_0$ 
         \[
     M_0(z)=\begin{bmatrix}
     0&\beta(z)\chi_{\mathscr{D}}\\
     -\overline{\beta(\overline{z})}\chi_{\overline{\mathscr{D}}}&0
     \end{bmatrix},
     \]
     where $\beta(z)$ is a smooth function on $\mathscr{D}\subset\C_+$  and $\chi_{\mathscr{D}}$  ($\chi_{\overline{\mathscr{D}}}$) is the characteristic function of       $\mathscr{D}$   ($\overline{\mathscr{D}}$).
     We observe that $M_0$  satisfies the Schwarz symmetry
     \be
     \label{eq:Sch_sym}
      \overline{M_0(\ov z)}=\sigma_2 M_0(z)\sigma_2 , \quad \text{ where } \sigma_2=
      \begin{bmatrix}
        0 & -i\\
        i & 0
      \end{bmatrix}.
      \ee
Let us consider the $\ov\pa$-problem
        \bea
     \label{eq:NLShdbar}
     &  \pa_{\bar{z}} \Gamma(z,\pmb{t}) = \Gamma(z,\pmb{t}){\rm e}^{-i\xi(z,\pmb{t})\sigma_3} M_0(z){\rm e}^{i\xi(z,\pmb{t})\sigma_3} & \text{for } z \in {\mathscr D} \cup \ov  {\mathscr D}  \\
   &  \Gamma(z,\pmb{t}) \underset{z \to \infty}{ \to } \1 & \nn 
     \eea
     with  $\xi(z,\pmb{t})$ as in \eqref{xi}.  Here we have sent $t_j\to -2i t_j$ with respect to the normalization in the KP-hieararchy.

     
 % We will show that the solution $\Gamma$ of the above problem     generates (also)  the focusing nonlinear Schr\"odinger hierarchy.     
 \begin{theorem}
      \label{th-NLSh}
       Let  $\Gamma(z, \pmb{t})$ be the solution of the $\ov\pa$-problem~\eqref{eq:NLShdbar}  and let
       \[
      \psi(\pmb{t}):=2i\lim_{z\to\infty}z(\Gamma(z, \pmb{t})-\1)_{12}\,.
       \]
       Then the function $\psi=\psi(\pmb{t})$  satisfies the nonlinear Schr\"odinger hierarchy~\cite{FadTak,Matveev2018}  written in the recursive form
        \bea
        & i\pa_{t_m} \psi_1 = 2 \psi_{m+1},\quad      \psi_1:=\psi \label{eq:thi} \\
        & \psi_{m} = \frac{i}{2}\pa_{t_1} \psi_{m-1} + \psi_1  h_{m-1}, \quad \pa_{t_1} h_{m}= 2 \im(\psi_1 \ov \psi_{m}),
          \label{eq:thf}
        \eea
        where $\psi_{m}$ and $h_{m}$ are functions of $\pmb{t}$.
      \end{theorem}
      The proof of this theorem is classical and  is deferred to  Appendix~\ref{app:I}.
      In particular the second flow gives the focusing NLS equation
\[
        i \pa_{t_2}   \psi + \frac{1}{2} \pa^{2}_{t_1} \psi + |\psi|^{2}\psi=0,\quad \psi:=\psi_1,
  \]    
  where comparing with the notation in the  introduction $t_2=t$ and $t_1=x$.  The third flows gives the so called complex modified KdV equation
   \be
    \pa_{t_3} \psi + \frac{\pa^{3}_{t_1} \psi}{4} + \frac{3}{2}|\psi|^2\pa_{t_1} \psi=0. \nn
    \ee
  Setting $t_k=0$ for $k\geq 4$ one obtains that $v(t_1,t_2,t_3):=2|\psi_1(t_1,t_2,t_3)|^2$ satisfies  the KP equation 
  \eqref{KP} after the  rescalings   $v=-4u$ and $t_j\to \frac{i}{2}t_j$.
    
    \section{Conclusions}
     
    The $\ov\pa$-problems treated in this manuscript differ from the $\ov\pa$  introduced in \cite{MM},\cite{MDM} to study asymptotic behaviour of orthogonal polynomials or PDEs with non analytic initial data respectively.
    In those cases the $\ov\pa$-problem  is a by-product of the steepest descent Deift-Zhou method extended to the case where the jump-matrix is not analytic but otherwise the initial problem is an ordinary RHP;   in our case, the 
    initial data is defined from the solution of the $\ov\pa$-problem and is encoded in the domain $\mathscr{D}$ 
    and in the matrix $M$  of the $\ov\pa$-problem \eqref{dbar0}.
    An equation similar to~\eqref{eq:NLShdbar} was also studied by Zhu et al.~\cite{ZJW}, with the aim to find solutions for the defocusing/focusing NLS with nonzero boundary conditions.  
    
    
    A generalization that could be considered is one where instead of the ``pure'' $\ov\pa$-problem \eqref{dbarproblem} one has a mixed $\ov\pa$ and Riemann--Hilbert problem; this would correspond to an operator for example acting on $L^2(\mathscr D, \d^2z) \oplus L^2(\Sigma, |\d z|)$ (typically with $\pa \mathscr D\subseteq \Sigma$); this type of problems would use, in the computation of the exterior derivative of the Malgrange form, the full Cauchy--Pompeiu formula. We defer this investigation to future efforts.    
    
 \vspace{1cm}  
\noindent {\bf Acknowledgements}

\noindent
 TG  and GO acknowledge the support of   the European Union's H2020 research and innovation programme under the Marie 
 Sklodowska-Curie grant No. 778010 {\it   IPaDEGAN}, the GNFM-INDAM group and the  research project Mathematical Methods in NonLinear Physics (MMNLP), Gruppo 4-Fisica Teorica of INFN.  
The work of MB was supported in part by the Natural Sciences and Engineering Research Council of Canada (NSERC) grant RGPIN-2023-04747.


 \appendix
 \section{Connection between the $\ov\pa$-Problem and the Inverse Scattering Theory}
 \label{app:I}
In this section we   prove Theorem~\ref{th-NLSh}  by deriving the corresponding   Zakharov-Shabat Lax pair \cite{ZS} for the solution  of   the $\ov\pa$-problem~\eqref{eq:NLShdbar}.
 To simplify the presentation, we restrict only to the first flow, namely we set $t_1=x$, $t_2=t$ and $t_j=0$ for $j\geq 3$.  The general case can be treated in a similar way.
Let us consider the matrix
\begin{equation}
  \label{eq:psi_1}
  \Psi(z;x,t)=\Gamma(z;x,t){\rm e}^{-i(zx  +z^2 t)\sigma_{3}}.
\end{equation}
{ where $\Gamma$ is a solution of the $\ov\pa$-problem \ref{eq:NLShdbar}}  so that we obtain  the $\ov\pa$-problem
\begin{equation}
  \label{eq:new_d_bar}
  \left\{
    \begin{split}
      \pa_{\bar{z}}\Psi(z)&=\Psi(z)M_{0}(z)\\
      \Psi(z)&=\left( \1 + \mathcal{O}\left( z^{-1}\right) \right){\rm e}^{-i(zx  +z^2 t)\sigma_{3}}  \quad \text{as } z \to \infty\,.
    \end{split} \right.
\end{equation}
We denote the terms of the expansion of $\Psi$ near $z=\infty$ as follows:
\be
\label{psiexp}
\Psi(z;x,t) = \le(\1 +\sum_{\ell=1}^\infty \frac {\Gamma_\ell(x,t)}{z^\ell} \ri){\rm e}^{-i(zx  +z^2 t)\sigma_{3}}\,.
\ee
{ The first observation is that $\Psi$ satisfies the Schwartz-like symmetry 
\be
\label{SchwartzPsi}
{\Psi(z;x,t)}^\dagger \Psi(\ov z;x,t) \equiv \1,
\ee
which follows from the uniqueness of the solution after observing that the matrix $\Phi(z;x,t):= {\Psi(z;x,t)}^\dagger $ solves the same $\ov\pa$-problem, thanks to the property $M(z;x,t) =- M(\ov z;x,t)^\dagger$. Given that $\det \Psi\equiv 1$ we can rewrite the symmetry as 
\be
\Psi(z;x,t) = \sigma_2 \Psi(\ov z;x,t)^\dagger \sigma_2. \label{ShcwartzPsi2}
\ee
This translates to the following symmetry for the matrices $\Gamma_\ell(x,t)$:
\be
\label{SchwartzGamma}
\Gamma_\ell(x,t) = \sigma_2 \ov{\Gamma_\ell(x,t)}\sigma_2.
\ee
}
Since the operators $\partial_{x}$ and $\pa_{\bar{z}}$ commute, we can see that $\partial_{x}\Psi$ satisfies the problem \eqref{eq:new_d_bar}
\begin{equation}
  \pa_{\bar{z}}\pa_x \Psi= \pa_x\Psi M_{0}(z).
\end{equation}
It now follows that the matrix   $U(z;x,t):= \pa_x\Psi(\Psi^{-1})$ is an entire function in $z$. Indeed
\begin{equation*}
  \begin{split}
    \partial_{\bar{z}}\left( \pa_x \Psi\Psi^{-1}\right)&= (\partial_{\bar{z}}\pa_x\Psi)\Psi^{-1} + \pa_x\Psi(\pa_{\bar{z}}\Psi^{-1})\\
    &=(\pa_x \Psi) M_{0}\Psi^{-1} -(\pa_x \Psi)\Psi^{-1}\partial_{\bar{z}}\Psi (\Psi^{-1})\\
    &=(\pa_x \Psi) M_{0}\Psi^{-1} - (\pa_x \Psi) M_{0}\Psi^{-1}=0.
  \end{split}
\end{equation*}
Thus we obtain the   following equation:
\begin{equation}
  \label{eq:Lax_1}
  \pa_x \Psi(z;x,t)=U(z;x,t)\Psi(z;x,t).
\end{equation}
In order to  determine the $z$--dependence of $U(z;x,t)$ we  consider the asymptotic of $\Psi(z;x,t)$ for $z \to \infty$  by differentiation of the asymptotic behaviour specified in \eqref{eq:new_d_bar}  
\begin{equation*}
  \begin{split}
\pa_x \Psi \sim -\left( \1 +\frac{\Gamma_{1}(x,t)}{z} + \mathcal{O}\left(z^{-2}\right) \right)iz\sigma_{3}{\rm e}^{-i(zx  +z^2 t) \sigma_{3}} +\left(\frac{\partial_{x}\Gamma_{1}(x,t)}{z}  + \mathcal{O}\left( z^{-2}\right)\right){\rm e}^{-i(zx  +z^2 t)\sigma_{3}}   .
  \end{split}
\end{equation*}
Upon substitution in~\eqref{eq:Lax_1} we get
\begin{equation}
  \label{eq:U}
  \begin{split}
    U(z;x,t) &= \pa_x \Psi (\Psi^{-1} )\sim\\
    &\sim -iz\left( \mathbb{I} + \frac{\Gamma_{1}(x,t)}{z} \right)\sigma_{3}\left( \1 -\frac{\Gamma_{1}(x,t)}{z} \right) + \mathcal{O}\left(z^{-1}\right)\\
    &= -iz\sigma_{3} -i\left[\Gamma_{1}(x,t),\sigma_{3}\right] + \mathcal{O}\left(z^{-1}\right).
  \end{split}
\end{equation}
and since we know that $U(z;x,t)$ is entire, we conclude that $U$ is the polynomial in $z$ of first degree obtained by dropping the $\mathcal O(z^{-1})$ in \eqref{eq:U}. Due to the symmetry \eqref{SchwartzGamma} the matrix $\Gamma_{1}(x,t)$ has the  form
\begin{equation}
  \label{eq:Gamma_1}
  \Gamma_{1}(x,t)=
  \begin{bmatrix}
    a(x,t) & b(x,t)\\
    -\bar{b}(x,t) & \bar{a}(x,t)
  \end{bmatrix}, 
\end{equation}
from which we find  
\begin{equation*}
  \left[\Gamma_{1}(x,t),\sigma_{3}\right]=
  \begin{bmatrix}
0 & -2b(x,t)\\
    -2\bar{b}(x,t) & 0
  \end{bmatrix}\,.
\end{equation*}
We thus conclude that the matrix $U(z;x,t)$ has the form
\begin{equation}
  \label{eq:U_1}
  U(z;x,t)=
  \begin{bmatrix}
    -iz & 2ib(x,t)\\
    2i\bar{b}(x,t) & iz
  \end{bmatrix}.
\end{equation}
The same arguments can be applied  for the parameter $t$. In that case, $\Psi(z;x,t)$ satisfy the ODE
\begin{equation}
  \pa_t \Psi(z;x,t)=V(z;x,t)\Psi(z;x,t)
\end{equation}
where $V(z;x,t)$ is an entire function in $z$.
Following the same idea as before, we expand $\pa_t \Psi(z;x,t)$ for $z\to \infty$
\begin{equation}
  \pa_t \Psi \sim- iz^2\left(\mathbb{I} + \frac{\Gamma_{1}(x,t)}{z} + \frac{\Gamma_{2}(x,t)}{z^{2}} + \mathcal{O}\left(z^{-3}\right)\right)\sigma_{3}{\rm e}^{-i(zx  +z^2 t)\sigma_{3}} + \mathcal{O}\left(z^{-1}\right){\rm e}^{-i(zx  +z^2 t)\sigma_{3}}
\end{equation}
and we have
\begin{equation}
  \begin{split}
    \pa_t \Psi(\Psi^{-1})&=V(z;x,t) \sim -\left(\mathbb{I} +\frac{\Gamma_{1}(x,t)}{z} + \frac{\Gamma_{2}(x,t)}{z^{2}} +\mathcal{O}\left(z^{-3}\right)  \right)\times\\
    & \times iz^{2}\sigma_{3}\left( \mathbb{I} -\frac{\Gamma_{1}(x,t)}{z} -\frac{\Gamma_{2}(x,t)}{z^{2}} +\frac{(\Gamma_{1}(x,t))^{2}}{z^{2}} + \mathcal{O}\left(z^{-3}\right)  \right)\\
    &= -iz^{2}\sigma_{3} - iz\left[\Gamma_{1}(x,t),\sigma_{3}\right] - i\left[\Gamma_{2}(x,t),\sigma_{3}\right] +i\left[\Gamma_{1}(x,t),\sigma_{3}\right]\Gamma_{1} (x,t)+ \mathcal{O}\left(z^{-1}\right).
  \end{split}
\end{equation}
We similarly conclude that $V(z;x,t)$ is the polynomial part of the above expression, a quadratic polynomial in $z$.
To complete the calculation we need to relate  the matrix $\Gamma_{2}(x,t)$ to the $\pa_x$ derivative of $\Gamma_1$ by taking  the expansion of both sides of the Lax equation \eqref{eq:Lax_1} as $z\to\infty$, and using the explicit expression of $U$ given in \eqref{eq:U}. 
The term $\mathcal{O}(z^{-1})$ in~\eqref{eq:Lax_1} provides the equation:
\begin{equation}
\label{eq:z_1}
  \pa_{x} \Gamma_{1}(x,t)= i [\Gamma_{2}(x,t),\sigma_{3}] -i[\Gamma_{1}(x,t),\sigma_3]\Gamma_{1}(x,t) .
\end{equation}
% Since $U(z,x,t)$ is entire, we get that this term must be null
%
%\begin{equation}
%  
%  \left[\Gamma_{2};\sigma_{3}\right]=-i\partial_{x}\Gamma_{1} + \left[\Gamma_{1};\sigma_3\right]\Gamma_{1}.
%\end{equation}
The $(1,1)$ entry of~\eqref{eq:z_1} yields the relation
\begin{equation}
  \label{eq:rec_comp1}
  \pa_{x} a(x,t)=-2i|b(x,t)|^{2}
\end{equation}
while the off diagonal give
\begin{equation}
  (\left[\Gamma_{2}(x,t),\sigma_{3}\right])_{12}=(\overline{\left[\Gamma_{2}(x,t),\sigma_{3}\right]})_{21}= -2{b}\ov {a} - i \pa_xb.
\end{equation}
In conclusion,  the matrix $V(z;x,t)$ is
\begin{equation}
  \begin{split}
    V(z;x,t)&= -iz^{2}\sigma_{3} - iz\left[\Gamma_{1}(x,t),\sigma_{3}\right] - i\left(\left[\Gamma_{2}(x,t),\sigma_{3}\right] - \left[\Gamma_{1}(x,t),\sigma_{3}\right]\Gamma_1(x,t)\right)\\
    &= -iz^{2}\sigma_{3} - iz\left[\Gamma_{1}(x,t),\sigma_{3}\right] -\partial_{x}\Gamma_{1}(x,t)\\
    &=
    \begin{bmatrix}
      -iz^{2}+2i|b|^{2} & 2zb -\pa_xb\\
      2z\bar{b} +\pa_x\bar{b} & iz^{2} -2i|b|^{2}
    \end{bmatrix}.
  \end{split}
\end{equation}
Summarizing, the matrix $\Psi(z;x,t)$  solves the $\bar{\partial}$-Problem~\ref{eq:new_d_bar} as well as the two linear PDEs 
\begin{equation}
  \label{eq:Lax_pair}
  \begin{split}
    \pa_x \Psi &= U(z;x,t) \Psi = 
    \begin{bmatrix}
      -iz & \psi\\
      -\bar{\psi} & iz
    \end{bmatrix}\Psi\\
    \pa_t \Psi &= V(z;x,t)\Psi=
    \begin{bmatrix}
      -iz^{2} +\frac{i}{2}|\psi|^{2} & z\psi  +\frac{i}{2}\pa_x\psi\\
      -z\bar{\psi} +\frac{i}{2}\pa_x\bar{\psi} & iz^{2} - \frac{i}{2}|\psi|^{2}
    \end{bmatrix}\Psi
  \end{split}
\end{equation}
where we have set $\psi(x,t):=2ib(x,t)$. We   can see that the matrices $U(z;x,t)$ and $V(z;x,t)$ are in the form of the Lax pair of the NLS~\eqref{eq:NLS},  namely, the zero curvature equations  \cite{ZS}
\be
\pa_x\pa_t\Psi \equiv \pa_t\pa_x\Psi \ \ \Leftrightarrow\ \ \ 
\pa_t U- \pa_x V + [U,V]\equiv 0
\ee
and the latter is equivalent to the NLS equation~\eqref{eq:NLS}.
%
%
%
\section{Proof of Lemma~\ref{lemm-2}}
\label{app-II}
In this section we give the proof of Lemma~\ref{lemm-2}.
Since the computations of  $\omega(\pmb{t} \pm [\zeta^{-1}])$ are the same, we give the proof only for $\omega(\pmb{t} - [\zeta^{-1}])$.

From~\eqref{eq:Malg_form},~\eqref{eq:M_s} and~\eqref{eq:Gamma-1}, we get
  \bea
       \omega(&\pmb{t} -[\zeta^{-1}])=-\iint_{\mathscr{D}}\Tr\le(\Gamma^{-1}\le(z,\pmb{t} -[\zeta^{-1}]\ri)\pa_z\Gamma\le(z,\pmb{t} -[\zeta^{-1}]\ri)\delta_{[\zeta]} M\le(z,\pmb{t} -[\zeta^{-1}]\ri)\ri)\frac{\d \ov z\wedge \d {z}}{2 \pi i} \nn \\ %+ \delta_{[\zeta]}\Tr_{L^2}[\mathcal{K}]= \nonumber\\
      =&-\iint_{\mathscr{D}}\Tr\le(D(z)\Gamma^{-1}(z)C^{-1}(z)\pa_z
     \le(C(z)\Gamma(z)D^{-1}(z)\ri)\delta_{[\zeta]}\!\!\le(D(z)M(z,\pmb{t})D^{-1}(z)\ri)\ri)\frac{ \d \ov{z} \wedge \d z}{2\pi i} \nonumber\\
     =&-\iint_{\mathscr{D}}\Tr\le(D(z)\Gamma^{-1}(z)\pa_z\Gamma(z)D^{-1}(z)\delta_{[\zeta]}\le(D(z)M(z,\pmb{t})D^{-1}(z)\ri)\ri)\frac{ \d \ov{z} \wedge \d z}{2\pi i}  + \label{eq:A}\\
     &-\iint_{\mathscr{D}}\Tr\le(D(z)\Gamma^{-1}(z)C^{-1}(z)\pa_z C(z)\Gamma(z)D^{-1}(z)\delta_{[\zeta]}\le(D(z)M(z,\pmb{t})D^{-1}(z)\ri)\ri)\frac{ \d \ov{z} \wedge \d z}{2\pi i} + \label{eq:B}\\
      &-\iint_{\mathscr{D}}\Tr\le(D(z)\pa_zD^{-1}(z)\delta_{[\zeta]}\le(D(z)M(z,\pmb{t})D^{-1}(z)\ri)\ri)\frac{ \d \ov{z} \wedge \d z}{2\pi i} \label{eq:C}
  \eea
 
  We now consider the three parts \eqref{eq:A}, \eqref{eq:B}, \eqref{eq:C}, separately.
\paragraph{Computation of \eqref{eq:A}.}
We find:
  \begin{equation*}
    \begin{split}
      \eqref{eq:A}&= -\iint_{\mathscr{D}}\Tr\le(\Gamma^{-1}(z)\pa_z\Gamma(z)\delta M(z,\pmb{t})\ri)\frac{ \d \ov{z} \wedge \d z}{2\pi i} + \\
      &-\iint_{\mathscr{D}}\Tr\le(\Gamma^{-1}(z)\pa_z\Gamma(z)
      \bigg[D^{-1}(z)\delta_{\zeta} D(z), M(z,\pmb{t})\bigg]
      \ri)\frac{ \d \ov{z} \wedge \d z}{2\pi i}%
\\
%
      &=\omega(\pmb{t}) +%+ \delta\Tr_{L^{2}}[\mathcal{K}]
     \iint_{\mathscr{D}}\Tr\le(D^{-1}(z)\delta_{\zeta} D(z)
      \bigg[ \Gamma^{-1}(z)\pa_z\Gamma(z),M(z,\pmb{t})\bigg]
      \ri)\frac{ \d \ov{z} \wedge \d z}{2\pi i}.
    \end{split}
  \end{equation*}
  Since $\zeta \notin \mathscr{D}$, the matrix $D^{-1}(z)$ in \eqref{eq:G-shift} is analytic in $\mathscr{D}$ and  using the $\ov\pa$-problem for $\Gamma$ we can rewrite the two integrals as
  \begin{equation}
    \label{eq:A_mid}
    \begin{split}
      \eqref{eq:A} &= \omega(\pmb{t}) %+ \delta\Tr_{L^{2}}[\mathcal{K}]
      + \iint_{\mathscr{D}}\pa_{\bar{z}}\Tr\bigg(\Gamma^{-1}(z)\pa_z\Gamma(z) D^{-1}(z)\delta_{\zeta} D(z)\bigg)\frac{ \d \ov{z} \wedge \d z}{2\pi i} +\\
      &- \iint_{\mathscr{D}}\Tr\le(\pa_z M(z,\pmb{t})D^{-1}(z)\delta_{\zeta} D(z)\ri)\frac{ \d \ov{z} \wedge \d z}{2\pi i}.
    \end{split}
  \end{equation}
  We now observe that the last integral is independent of $\pmb{t}$, due to the fact that $D(z)$ is diagonal. Moreover, using
  \begin{equation*}
    D^{-1}(z)\delta_{\zeta} D(z)= -\frac{z}{\zeta(z-\zeta)} E_{11}\d \zeta,
  \end{equation*}
   where $E_{11}= \begin{bmatrix} 1 & 0\\ 0 & 0\end{bmatrix}$, we find 
  \begin{equation}
    \label{eq:LI}
 - \iint_{\mathscr{D}} \Tr\le(\pa_z M(z,\pmb{t}) D^{-1}(z)\pa_{\zeta} D(z)\ri)\frac{ \d \ov{z} \wedge \d z}{2\pi i}= \iint_{\mathscr{D}}\frac{z}{\zeta(z-\zeta)}(\pa_z M_{0}(z))_{11} \frac{ \d \ov{z} \wedge \d z}{2\pi i}.
  \end{equation}
 The RHS of~\eqref{eq:LI} equals $\pa_{\zeta} \gamma(\zeta)$.
 Now, the integrand of the remaining integral in~\eqref{eq:A_mid} does not have a pole in $\mathscr{D}$ and  we can use Stokes' Theorem 
  \begin{equation*}
    \oint_{\pa \mathscr{D}}\Tr
    \le(\Gamma^{-1}(z)\pa_z\Gamma(z)D^{-1}(z)\pa_{\zeta} D(z)\ri)
    \frac{\d z}{2\pi i}=\oint_{-\pa \mathscr{D}}\frac{z}{\zeta(z-\zeta)}(\Gamma^{-1}(z)\pa_z\Gamma(z))_{11}\frac{\d z}{2 \pi i}
  \end{equation*}
  where $-\pa \mathscr{D}$ is the border of $\mathscr{D}$ oriented clockwise.
   Since $\Gamma(z)$ is analytic outside $\mathscr{D}$, we can apply Cauchy's residue Theorem  and pick up the residues at $z=\zeta$ (there is no residue at $z=\infty$ because the integrand is $\mathcal O(z^{-2})$):
  \begin{equation*}
 \oint_{-\pa \mathscr{D}}\frac{z}{\zeta(z-\zeta)}\big(\Gamma_{22}(z)\pa_z\Gamma_{11}(z) -\pa_z\Gamma_{21}(z)\Gamma_{12}(z)\big)\frac{\d z}{2 \pi i}= \Gamma_{22}(\zeta)\pa_\zeta\Gamma_{11}(\zeta) -\pa_\zeta\Gamma_{21}(\zeta)\Gamma_{12}(\zeta)   
\end{equation*}
so that
\begin{equation}
  \label{eq:Af}
  \eqref{eq:A}= \omega(\pmb{t}) %+ \delta\Tr_{L^{2}}[\mathcal{K}]
  +\left( \Gamma_{22}(\zeta)\pa_{\zeta}\Gamma_{11}(\zeta) - \pa_{\zeta}\Gamma_{21}(\zeta)\Gamma_{12}(\zeta)\right) \d \zeta + \delta_{\zeta} \gamma(\zeta).
\end{equation}
\paragraph{Computation of \eqref{eq:B}.}
Let us consider \eqref{eq:B}: 
\bea
\nn   \eqref{eq:B} &=-\iint_{\mathscr{D}}
\Tr
\le(\Gamma^{-1}(z)C^{-1}(z)\pa_z C(z)\Gamma(z)\delta M(z,\pmb{t})\ri)\frac{ \d \ov{z} \wedge \d z}{2\pi i}+\\
 \nn
    &-\iint_{\mathscr{D}}\Tr
    \le(M(z,\pmb{t})\Gamma^{-1}(z)C^{-1}(z)\pa_z C(z)\Gamma(z)D^{-1}(z) \delta_{\zeta} D(z)\ri)\frac{ \d \ov{z} \wedge \d z}{2\pi i} +\\
 \nn
    &+\iint_{\mathscr{D}}\Tr
    \le(\Gamma^{-1}(z)C^{-1}(z)\pa_z C(z)\Gamma(z)M(z,\pmb{t})D^{-1}(z)\delta_{\zeta} D(z)\ri)\frac{ \d \ov{z} \wedge \d z}{2\pi i}\\
 \nn
    &=-\iint_{\mathscr{D}}\Tr
    \le(\Gamma^{-1}(z)C^{-1}(z)\pa_z C(z)\Gamma(z)\delta M(z,\pmb{t})\ri)
    \frac{ \d \ov{z} \wedge \d z}{2\pi i}+\\
 \nn
    &+\iint_{\mathscr{D}}\Tr\le(\pa_{\bar{z}}\Gamma^{-1}(z)C^{-1}(z)\pa_z C(z)\Gamma(z)D^{-1}(z) \delta_{\zeta} D(z)\ri)
    \frac{ \d \ov{z} \wedge \d z}{2\pi i}+\\
    &+\iint_{\mathscr{D}}\Tr
    \le(\Gamma^{-1}(z)C^{-1}(z)\pa_z C(z)\ov{\pa}\Gamma(z)D^{-1}(z)\delta_{\zeta} D(z)\ri)\frac{ \d \ov{z} \wedge \d z}{2\pi i}. 
    \label{B7}
\eea
Since the only singularity is at $z=\zeta$, which is outside the domain $\mathscr{D}$, we can apply Stokes' Theorem to the integration and we get
\bea\nn
   \eqref{eq:B} &=-\iint_{\mathscr{D}}\Tr
   \le(C^{-1}(z)\pa_z C(z)\Gamma(z)\delta M(z,\pmb{t})\Gamma^{-1}(z)\ri)\frac{ \d \ov{z} \wedge \d z}{2\pi i} +\\
    &+\oint_{\pa \mathscr{D}}\Tr\le(
    \Gamma^{-1}(z)C^{-1}(z)\pa_z C(z)\Gamma(z)D^{-1}(z)\delta_{\zeta} D(z)\ri)\frac{\d z}{2\pi i}.
\label{B8}
\eea
Now observe that
\begin{equation}
  \label{eq:I_int_B}
  \Gamma(z,\pmb{t}) \delta M(z,\pmb{t}) \Gamma^{-1}(z,\pmb{t})=\pa_{\bar{z}}[(\delta \Gamma(z,\pmb{t})) \Gamma^{-1}(z,\pmb{t})]
\end{equation}
Using~\eqref{eq:I_int_B} in the first integral of \eqref{B8}, we can rewrite it as a contour integral
\begin{equation*}
  \begin{split}
    &-\iint_{\mathscr{D}}\Tr\le(
    \Gamma^{-1}(z)C^{-1}(z)\pa_z C(z)\Gamma(z)\delta M(z,\pmb{t})\ri)
    \frac{ \d \ov{z} \wedge \d z}{2\pi i} \\
    &=-\iint_{\mathscr{D}}\pa_{\bar{z}}\Tr
    \le(C^{-1}(z)\pa_z C(z)\delta \Gamma(z) \Gamma^{-1}(z)\ri)\frac{ \d \ov{z} \wedge \d z}{2\pi i}\\
    &= \oint_{-\pa \mathscr{D}}\Tr\le(C^{-1}(z)\pa_z C(z)\delta \Gamma(z) \Gamma^{-1}(z)\ri)\frac{\d z}{2\pi i}.
    \end{split}
  \end{equation*}
From the explicit expression of $C$ in \eqref{eq:R-mat} we obtain
  \begin{equation*}
    C^{-1}(z)= \frac{1}{\det C(z)} {\rm adj}(C(z))= \frac{1}{\left(1 - \frac{z}{\zeta}\right)}
    \begin{bmatrix}
      1 & -\frac{\pa_z\Gamma_{12}(\infty)}{\zeta}\\
      \frac{\Gamma_{21}(\zeta)}{\Gamma_{11}(\zeta)} & \left(1 - \frac{z}{\zeta}\right) - \frac{\pa_z \Gamma_{12}(\infty) \Gamma_{12}(\zeta)}{\zeta\Gamma_{11}(\zeta)}
    \end{bmatrix}
  \end{equation*}
  \begin{equation*}
    \pa_z C(z) = -\frac{1}{\zeta} E_{11}
  \end{equation*}
  and
  \begin{equation*}
    \begin{split}
      &\Tr\le(C^{-1}(z)\pa_z C(z)\delta\Gamma(z) \Gamma^{-1}(z)\ri)=\frac{1}{(z-\zeta)}\left((\delta \Gamma(z)\Gamma^{-1}(z))_{11} + \frac{\Gamma_{21}(\zeta)}{\Gamma_{11}(\zeta)}(\delta \Gamma(z)\Gamma^{-1}(z))_{12}\right)\\
      &=\frac{\delta\Gamma_{11}(z)\Gamma_{22}(z) - \delta \Gamma_{12}(z)\Gamma_{21}(z)}{(z - \zeta)} + \frac{\Gamma_{21}(\zeta)}{\Gamma_{11}(\zeta)}\left(\frac{\delta \Gamma_{12}(z)\Gamma_{11}(z) - \delta \Gamma_{11}(z) \Gamma_{12}(z)}{(z - \zeta)}\right).
    \end{split}
  \end{equation*}
  We thus conclude that  the first integral in~\eqref{B8} is given by 
\bea
      \oint_{-\pa \mathscr{D}}\Tr
      \big(C^{-1}(z)\pa_z C(z)&\delta \Gamma(z) \Gamma^{-1}(z)\big)\frac{\d z}{2\pi i} = \delta \Gamma_{11}(\zeta)\Gamma_{22}(\zeta) - \frac{\delta \Gamma_{11}(\zeta)}{\Gamma_{11}(\zeta)}\Gamma_{12}(\zeta)\Gamma_{21}(\zeta)\nn
      \\
      &= \frac{\delta \Gamma_{11}(\zeta)}{\Gamma_{11}(\zeta)}= \delta \ln \Gamma_{11}(\zeta).\label{B9}
\eea
  To compute the second integral in \eqref{B8} we expand the trace and obtain 
  \begin{equation*}
    \Tr\big(\Gamma^{-1}(z)C^{-1}(z)\pa_z C(z)\Gamma(z)D^{-1}(z)\pa_{\zeta} D(z)\big)= -\frac{z}{\zeta(z-\zeta)^{2}}\Gamma_{11}(z)\left(\Gamma_{22}(z) - \frac{\Gamma_{12}(z)\Gamma_{21}(\zeta)}{\Gamma_{11}(\zeta)}\right).
  \end{equation*}
  So we are left with a contour integral with a double pole at $z=\zeta$ and a simple pole at $z=\infty$. Using the explicit expression \eqref{eq:R-mat} for the matrix $C$ we obtain: 
  \bea\nn
        \oint_{\pa \mathscr{D}}&\Tr
        \le(\Gamma^{-1}(z)C^{-1}(z)\pa_z C(z)\Gamma(z)D^{-1}(z)\pa_\zeta D(z)\ri)\frac{\d z}{2\pi i}\\
      \nn
      &=\oint_{-\pa \mathscr{D}}\frac{z}{\zeta(z-\zeta)^{2}}\Gamma_{11}(z)\left(\Gamma_{22}(z) - \frac{\Gamma_{12}(z)\Gamma_{21}(\zeta)}{\Gamma_{11}(\zeta)}\right)\frac{\d z}{2\pi i}\nn
\\
\nn
&=-\frac 1 \zeta 
  + \frac {\det \Gamma(\zeta)}{\zeta} +\pa_\zeta(\Gamma_{11}(\zeta)\Gamma_{22}(\zeta))  
  -\frac{  \pa_\zeta\big(\Gamma_{11}(\zeta)\Gamma_{12}(\zeta)\big)\Gamma_{21}(\zeta)}{\Gamma_{11}(\zeta)}
  \\
  &=\pa_\zeta\ln \Gamma_{11}(\zeta)
  +  \Gamma_{11}(\zeta) \pa_\zeta\Gamma_{22}(\zeta)  - \pa_\zeta\Gamma_{12}(\zeta) \Gamma_{21}(\zeta).
  \label{B10}
      \eea
Combining \eqref{B9} with \eqref{B10}  we have 
  \begin{equation}
    \label{eq:Bf}
    \eqref{eq:B}= \delta_{[\zeta]}(\ln(\Gamma_{11}(\zeta))) +\left( \Gamma_{11}(\zeta)\pa_{\zeta}\Gamma_{22}(\zeta) - \Gamma_{21}(\zeta)\pa_{\zeta}\Gamma_{12}(\zeta) \right)\d \zeta.
  \end{equation}
  \paragraph{Computation of \eqref{eq:C}.}
 This term turns out to vanish; indeed
  \begin{equation*}
    \begin{split}
      \eqref{eq:C}&= \iint_{\mathscr{D}}\Tr
      \le(D^{-1}(z)\pa_z D(z)\delta M(z,\pmb{t})\ri)
      \frac{ \d \ov{z} \wedge \d z}{2\pi i}+\\
      &\quad-\iint_{\mathscr{D}}\Tr
      \le(D^{-1}(z)\pa_z D(z)\le[M(z,\pmb{t}), D^{-1}(z)\delta_{\zeta} D(z)\ri]\ri)\frac{ \d \ov{z} \wedge \d z}{2\pi i}\\
      &= -\iint_{\mathscr{D}}\Tr
      \le(\delta\xi(z,t)D^{-1}(z)\pa_z D(z) \le[M(z,\pmb{t}),\sigma_{3}\ri]\ri)\frac{ \d \ov{z} \wedge \d z}{2\pi i} +\\
      &\quad-\iint_{\mathscr{D}}\Tr
      \le(D^{-1}(z)\pa_z D(z)\le[M(z,\pmb{t}),D^{-1}(z)\delta_{\zeta} D(z)\ri]\ri)\frac{ \d \ov{z} \wedge \d z}{2\pi i}
      \end{split}
    \end{equation*}
 and the integrand vanishes identically because of the cyclicity of the trace and the fact that $D$ is a diagonal matrix.
%    \begin{equation*}
%      \pa_z D(z)= -\frac{1}{\zeta}E_{11} \implies D^{-1}(z)\pa_z D(z)= \frac{1}{(z-\zeta)}E_{11}
%    \end{equation*}
%    so the traces give us the first diagonal elements of the commutators $[M(z,\pmb{t}),\sigma_{3}]$ and $[M(z,\pmb{t}),D^{-1}(z)\delta D(z)]$, which are zeros.
    In conclusion,  adding the equations~\eqref{eq:Af} and~\eqref{eq:Bf}, we obtain
    \begin{equation}
      \label{eq:res-p1}
      \omega(\pmb{t}-[\zeta^{-1}])=\omega(\pmb{t}) %+ \delta\Tr_{L^{2}}[\mathcal{K}]
      + \delta_{[\zeta]} \ln(\Gamma_{11}(\zeta)) + \delta_{[\zeta]} \gamma(\zeta) .
    \end{equation}
    Substituting $C(z)$ and $D(z)$ with $\tilde{C}(z)$ and $\tilde{D}(z)$ respectively and using the nonsingular condition for $K$~\eqref{eq:K_ker2}, we find $\omega(\pmb{t} + [\zeta^{-1}])$ with similar calculations and we get the following result
    \begin{equation}
      \label{eq:res-p2}
      \omega(\pmb{t}+[\zeta^{-1}])= \omega(\pmb{t}) + \delta_{[\zeta]}\ln(\Gamma^{-1}_{11}(\zeta)) - \delta_{[\zeta]}\gamma(\zeta).
    \end{equation}
    and this proves the Lemma~\ref{lemm-2}.\QED


\bibliographystyle{siam}

\begin{thebibliography}{10}

\bibitem{Bert10}
{\sc M.~Bertola}, {\it  The dependence on the monodromy data of the
  isomonodromic tau function}, Commun. Math. Phys.,  (2010), pp.~539--579.
\bibitem{Bertola}{\sc M. Bertola}, {\it  The Malgrange form and Fredholm determinants}, SIGMA 13, (2017), 046; arXiv:1703.00046
\bibitem{BC}
{\sc M.~Bertola and M.~Cafasso}, {\it  Fredholm {D}eterminants and {P}ole-free
  {S}olutions to the {N}oncommutative {P}ainlev{\'e} {I}{I} {E}quation},
  Commun. Math. Phys.,  (2012), pp.~793--833.

\bibitem{BGO2023}
{\sc M.~Bertola, T.~Grava, and G.~Orsatti}, {\it  Soliton Shielding of the
  Focusing Nonlinear Schr\"odinger Equation}, Phys. Rev. Lett., 130 (2023),
  p.~127201.
\bibitem{BorodinDeift}
{\sc A. Borodin and P. Deift}
{\it  Fredholm determinants, {J}imbo-{M}iwa-{U}eno {$\tau$}-functions, and
  representation theory}, 
\newblock { Comm. Pure Appl. Math.}, 55(9):1160--1230, 2002.
\bibitem{BB}
{\sc T.~Bothner and R.~Buckingham}, {\it  Large {D}eformations of the
  {T}racy--{W}idom {D}istribution {I}: {N}on-oscillatory {A}symptotics},
  Commun. Math. Phys,  (2018), pp.~223--263.

\bibitem{BCT}
{\sc T.~Bothner, M.~Cafasso, and S.~Tarricone}, {\it  Momenta spacing
  distributions in anharmonic oscillators and the higher order finite
  temperature {A}iry kernel}, Annales de l'Institut Henri Poincar{\'e},
  Probabilit{\'e}s et Statistiques, 58 (2022), pp.~1505 -- 1546.

\bibitem{CCR}
{\sc M.~Cafasso, T.~Claeys, and G.~Ruzza}, {\it  Airy {K}ernel {D}eterminant
  {S}olutions of the {K}d{V} {E}quation and {I}ntegro-{D}ifferential
  {P}ainlev{\'e} {E}quation}, Commun. Math. Phys.,  (2021), pp.~1107--1153.

\bibitem{CGL}
{\sc M.~Cafasso, P.~Gavrylenko, and O.~Lisovyy}, {\it  Tau {F}unctions as
  {W}idom {C}onstants}, Commun. Math. Phys.,  (2019), pp.~741--772.


\bibitem{Deift} {\sc P. Deift, }
Integrable operators.  Differential operators and spectral theory, 69–84.
Amer. Math. Soc. Transl. Ser. 2, 189
Adv. Math. Sci., 41
American Mathematical Society, Providence, RI, 1999

\bibitem{DeiftItsKrasovsky}
{\sc P.~Deift, A.~Its, I.~Krasovsky, and X.~Zhou}
{\it  The {W}idom-{D}yson constant for the gap probability in random matrix
  theory}, {J. Comput. Appl. Math.}, 202(1):26--47, 2007.
  
\bibitem{DZ}
{\sc P.~Deift and X.~Zhou}, {\it  A steepest descent method for oscillatory
  {R}iemann-{H}ilbert problems. {A}symptotics for the {MK}d{V} equation}, Ann.
  of Math. (2), 137 (1993), pp.~295--368.

\bibitem{FadTak}
{\sc L.~D. Faddeev and L.~A. Takhtajan}, 
\newblock {\it Hamiltonian methods in the theory of solitons}.
\newblock Classics in Mathematics. Springer, Berlin, english edition, 2007.
\newblock Translated from the 1986 Russian original by Alexey G. Reyman.


\bibitem{GL} {\sc Gavrylenko, P., Lisovyy}, {\it  O. Fredholm Determinant and Nekrasov Sum Representations of Isomonodromic Tau Functions}. Commun. Math. Phys. 363, 1 - 58 (2018). 
\bibitem{Gohberg_2000}
{\sc I.~Gohberg, S.~Goldberg, and N.~Krupnik}, {\it  Traces and Determinants of
  Linear Operators}, Birkhäuser Basel, 2000.

\bibitem{Harnad2021}
{\sc J.~Harnad and F.~Balogh}, {\it  Tau functions and their applications},
  Camb. Monogr. Math. Phys., Cambridge: Cambridge University Press, 2021.

\bibitem{Hirota1986}
{\sc R.~Hirota}, {\it  Reduction of soliton equations in bilinear form}, Physica
  D, 18 (1986), pp.~161--170.

\bibitem{Its_Harnad}
{\sc A.~Its and J.~Harnad}, {\it  {Integrable Fredholm Operators and Dual
  Isomonodromic Deformations}}, Commun. Math. Phys.,  (2002), pp.~497--530.

\bibitem{MR1060387}
{\sc A.~R. Its, A.~G. Izergin, and V.~E. Korepin}, {\it  Long-distance asymptotics of temperature correlators of the
  impenetrable {B}ose gas}, { Comm. Math. Phys.}, 130(3):471--488, 1990.


\bibitem{IIKS}
{\sc A.~Its, A.~Izergin, V.~Korepin, and N.~Slavnov}, {\it  {Differential
  Equations for Quantum correlation functions}}, International Journal of
  Modern Physics B, 04 (1990), pp.~1003--1037.

\bibitem{IIKS21}
{\sc A.~Its, A.~Izergin, V.~Korepin, and N.~Slavnov}, {\it  {
The quantum correlation function as the $\tau$  function of classical differential equations.}} {\it Important developments in soliton theory}, 407–417.
Springer Ser. Nonlinear Dynam.
Springer-Verlag, Berlin, 1993
ISBN:3-540-55913-2

\bibitem{ItsTakht}
{\sc A.~Its, L. A. Takhtajan}, 
{\it Normal matrix models, dbar-problem, and orthogonal polynomials on the complex plane}, arXiv:0708.3867.


\bibitem{JMD}
  {\sc M.~Jimbo, T.~Miwa, E.~Date}, {\it {Solitons: Differential Equations, Symmetries and Infinite Dimensional Algebras}},
  Cambridge Tracts on Mathematics.
  Cambridge University Press, Cambridge, 2000
  ISBN: 9780521561617
  
\bibitem{JMMS80}
{\sc M.~Jimbo, T.~Miwa, Y.~M{\^o}ri, and M.~Sato}, {\it  {Density matrix of an
  impenetrable Bose gas and the fifth Painlev{\'e} transcendent}}, Physica D:
  Nonlinear Phenomena, 1 (1980), pp.~80--158.

\bibitem{JMU}
{\sc M. Jimbo, T. Miwa, and K. Ueno},
{\it Monodromy preserving deformation of linear ordinary differential
  equations with rational coefficients. {I}. {G}eneral theory and {$\tau
  $}-function}, { Phys. D}, 2(2):306--352, 1981.

\bibitem{KlD}
{\sc A.~Krajenbrink and P.~L. Doussal}, {\it  Replica Bethe Ansatz solution to
  the Kardar-Parisi-Zhang equation on the half-line}, SciPost Phys., 8 (2020),
  p.~035.


\bibitem{Krichever} {\sc I. Krichever}, {\it 
General rational reductions of the Kadomtsev-Petviashvili hierarchy and their symmetries} 
Funct. Anal. Appl. 29 (1995), no. 2, 75–80.

\bibitem{Matveev2018}
  {\sc V.B.~Matveev, A.O.~Smirnov}, {\it {AKNS and NLS hierarchies, MRW solutions, $P_ n$ breathers, and beyond}},{J. Math. Phys.}, 59, 091419, 2018.

\bibitem{MM} {\sc  K. T.-R. McLaughlin and  P.D.Miller}. 
The $\overline{\partial}-$steepest descent method for orthogonal polynomials on the real line with varying weights
 Int. Math. Res. Not. IMRN(2008), Art. ID rnn 075, 66 pp.

\bibitem{MDM}  {\sc  D. Momar, K. McLaughlin, P.D.Miller}, Dispersive asymptotics for linear and integrable equations by the 
$\overline{\partial}-$steepest descent method.  Fields Inst. Commun., 83
Springer, New York, 2019, 253–291.

\bibitem{TracyWidom}
{\sc C.~A. Tracy and H. Widom}, 
{\it Level-spacing distributions and the {A}iry kernel}, 
\newblock { Comm. Math. Phys.}, 159(1):151--174, 1994.
\bibitem{Simon_2015}
{\sc B.~Simon}, {\it  Operator Theory}, American Mathematical Society, nov 2015.
\bibitem{Soshnikov}
{\sc A.~Soshnikov}, {\it  Determinantal random point fields}, 
{ Uspekhi Mat. Nauk}, 55(5(335)):107--160, 2000.

\bibitem{ZS}
  {\sc Zakharov, V. E. and Šabat, A. B.} Integration of the nonlinear equations of mathematical physics by the method of the inverse scattering problem. II, Functional Analysis and Its Applications, Volume 13, Issue 3, Pages 166–174, 1979. 

\bibitem{ZJW} {\sc Zhu, J., Jiang, X. and Wang, X. } Nonlinear Schr\"odinger equation with nonzero boundary conditions revisited: Dbar approach. Anal. Math. Phys. 13, 51 (2023). https://doi.org/10.1007/s13324-023-00816-8

\end{thebibliography}
    
 \end{document}
