\documentclass[pra,twocolumn,showpacs,preprintnumbers,superscriptaddress]{revtex4-2}
\usepackage{xcolor}
\usepackage{times}
\usepackage{bm}
\usepackage{float}
\usepackage{graphicx}
\usepackage{amsbsy}
\usepackage{amsmath}
\usepackage{amsfonts}
\usepackage{amsthm}
\usepackage{placeins}
%\usepackage{kbordermatrix}
\usepackage{autobreak}
%\newcommand\scalemath[2]{\scalebox{#1}{\mbox{\ensuremath{\displaystyle #2}}}}
\usepackage{booktabs}
\usepackage{multirow}
\usepackage[margin=1in]{geometry}

%\setlength{\topmargin}{.5cm}

%\setlength{\textwidth}{7.00in}

%\setlength{\textheight}{8in}


\begin{document}
	
\allowdisplaybreaks
	
\theoremstyle{plain}
\newtheorem{theorem}{Theorem}
\newtheorem{lemma}[theorem]{Lemma}
\newtheorem{corollary}[theorem]{Corollary}
\newtheorem{proposition}[theorem]{Proposition}
\newtheorem{conjecture}[theorem]{Conjecture}
	
\theoremstyle{definition}
\newtheorem{definition}[theorem]{Definition}
	
\theoremstyle{remark}
\newtheorem*{remark}{Remark}
\newtheorem{example}{Example}
\title{Estimation of Power in the Controlled Quantum Teleportation through the Witness Operator}
\author{Anuma Garg, Satyabrata Adhikari}
\email{anumagarg\_phd2k18@dtu.ac.in, satyabrata@dtu.ac.in} \affiliation{Delhi Technological University, Delhi-110042, Delhi, India}
	
\begin{abstract}
\centerline{Abstract}
Controlled quantum teleportation (CQT) can be considered as a variant of quantum teleportation in which three parties are involved where one party acts as the controller. The usability of the CQT scheme depends on two types of fidelities viz. conditioned fidelity and non-conditioned fidelity. The difference between these fidelities may be termed as power of the controller and it plays a vital role in the CQT scheme. Thus, our aim is to estimate the power of the controller in such a way so that its estimated value can be obtained in an experiment. To achieve our goal, we have constructed a witness operator and have shown that its expected value may be used in the estimation of the lower bound of the power of the controller. Furthermore, we have shown that it is possible to make the standard W state useful in the CQT scheme if one of its qubits either passes through the amplitude damping channel or the phase damping channel. We have also shown that the phase damping channel performs better than the amplitude damping channel in the sense of generating more power of the controller in the CQT scheme.      
\end{abstract}
\pacs{03.67.Hk, 03.67.-a} 
\maketitle
%\textbf{Keywords:}
	
\section{introduction}
\noindent The process of transferring an
unknown quantum state between two parties at two distant locations without transferring the physical information about the unknown quantum state
itself is known as quantum teleportation \cite{nielsen,wilde,bennett2}. This means that neither any physical information about the state is transferred nor a swap-operation between the sender and the receiver is performed. Teleportation protocol makes use of the non-local correlations generated by using an entangled pair between the sender and the
receiver, and the exchange of classical information between them. This concept plays a
central role in quantum communication using quantum repeaters \cite{gisin1, briegel} and can also be used to implement logic
gates for universal quantum computation \cite{gottesman}. Quantum teleportation was also demonstrated experimentally \cite{bouwmeister,kwiat,michler,boschi}. \\
In standard quantum teleportation prtocol, Alice the sender and Bob the receiver share a maximally two-qubit entangled state. Alice is then supposed to send an arbitrary single qubit state to Bob using the shared maximally entangled state. To execute the quantum teleportation protocol, Alice makes Bell state measurement on the composite system of two qubit with her possession. After the measurement, a state will be projected at Bob's location. Alice communicates her measurement result to Bob by sending the two classical bits $\{00,01,10,11\}$ using a classical channel. According to the received two classical bits, Bob may apply the appropriate unitary operator on his qubit to retrieve the state sent by Alice. The faithfulness of this quantum teleportation scheme is quantified by the fidelity, which may be defined as the maximum overlap between the state to be teleported by Alice and the state at Bob's site. In case of conventional teleportation scheme, the average fidelity is unity. But if either the non-maximally entangled pure two-qubit state or a mixed two-qubit entangled state is used as a resource in the teleportation protocol then the average fidelity of teleportation will always be less than unity, provided the filtering operation is not allowed.\\ 
Quantum teleportation using three qubit state as a resource state was introduced by Karlsson et al. \cite{karlsson}. It is a variant of teleportation in which three members such as Alice (A), Bob (B) and Charlie (C) are participating with one qubit each. Later, this type of quantum teleportation protocol is popularly called as the controlled quantum teleportation (CQT). The CQT scheme proposed by Karlsson et al. \cite{karlsson}, have used maximally entangled three-qubit GHZ state as a shared three-qubit state while on the other hand, Gao et al. \cite{gao} have shown that non-maximally pure three-qubit entangled states such as maximal slice state can be used in CQT with unit probability and unit fidelity. In 2014, Li et al. \cite{li2014} have focussed on the role of controller in the CQT scheme and also defined the power of the controller. Later, they have generalised their CQT protocol for multiqubit pure system \cite{li2015}. K. Jeong et al. \cite{jeong} also have considered the $n-$ qubit GHZ and W state as a channel for CQT and studied it using minimal control power. They have shown that the maximal values of the minimal control power may be obtained for $n-$ qubit GHZ and W class of states. In another work, Artur et al. \cite{Artur} have derived the lower and upper bound of the control power in terms of the three-tangle of any pure three-qubit states. Moreover, Paulson et al. also studied the CQT protocol using the symmetric mixed state such as $X-$ matrix states. Interestingly, it was found that the optimum controlled quantum teleportation fidelity has been obtained for non-maximally entangled mixed $X-$ states \cite{paulson}. The CQT scheme also has been investigated using high-dimensional tripartite standard GHZ and other GHZ classes of states \cite{wang}. Recently, controlled quantum teleportation was  experimentally realized using cluster states \cite{kumar}.    
The potential application of controlled quantum teleportation may be found in quantum computing algorithms, quantum communication protocols, and quantum error correction schemes \cite{rau}. The concept of CQT may also be used in quantum networks \cite{hamdoun}, entanglement swapping \cite{jun}, quantum reapters \cite{sango}, and quantum key distribution \cite{sayan}.\\
We are now in a position to discuss the detailed scheme of CQT. In the CQT scheme, we may consider that Alice, Bob and Charlie shared a three-qubit pure/mixed state described by the density operator $\rho_{ABC}$. We assume throughout the paper that Charlie act as a controller who perform von-Neumann measurement on his qubit. A single qubit von-Neumann measurement in the computational basis may be described as \{$\pi_{k}=|k\rangle\langle k|,k=0,1$\}. In general, a single qubit measurement operator in an arbitrary basis can be described by \{$B_{k}=V\pi_{k}V^{\dagger}:k=0,1$\}, where  
$V$ denote the single qubit unitary operator which may be expressed as \cite{luo}
\begin{eqnarray}
V=tI+i\overrightarrow{y}.\overrightarrow{\sigma},~~t^{2}+y_{1}^{2}+y_{2}^{2}+y_{3}^{2}=1
\label{vnm}
\end{eqnarray} 
where $t \in \mathbb{R}$ and $\overrightarrow{y}=(y_{1},y_{2},y_{3}) \in \mathbb{R}^{3} $.\\
Therefore, when Charlie perform measurement $B_{k}$ on his qubit, the three-qubit state $\rho_{ABC}$ projected onto the two-qubit state 
\begin{eqnarray}
\rho_{AB}^{(k)}=\frac{1}{p_{k}}(I\otimes I\otimes B_{k})\rho(I\otimes I\otimes B_{k}^{\dagger}),~~k=0,1
\end{eqnarray} 
where $p_{k}=tr((I\otimes I\otimes B_{k})\rho_{ABC}(I\otimes I\otimes B_{k}^{\dagger}))$ denote the probability of collapsing the three-qubit state to two-qubit state after the measurement performed on the third qubit. The two-qubit state $\rho_{AB}^{(k)}$ shared between Alice and Bob may be used as a resource state when teleporting an arbitrary single qubit state possessed by Alice. We further assume that in the process of single qubit teleportation using a shared two-qubit state, Alice act as a sender and Bob, a receiver. In CQT scheme, the faithfulness of the teleportation may be quantified by the conditioned fidelity denoted by $f_{CT}(\rho_{AB}^{(k)})$ and the non-conditioned fidelity $f_{NC}(\rho_{AB})$.\\
Alternatively, we may also describe the above equivalent situation with the reduced two-qubit state described by the density operator $\rho_{AB}=Tr_{C}(\rho_{ABC})$.  The resulting two-qubit state described by the density operator $\rho_{AB}$ may also be used in transmitting an arbitrary single qubit state through conventional teleportation scheme.\\
If we are not allowing any filtering operation then we may observe that the two-qubit state obtained either through the von-Neumann measurement or through the application of partial trace operation, may or may not be useful as a resource state in quantum teleportation. In this scenario, the controlled teleportation scheme may be helpful in the sense that by controlling the measurement parameter, the controller may be able to increase the teleportation fidelity in the conventional teleportation scheme. Therefore, the enhancement of the teleportation fidelity may be measured by a quantity known as controller's power ($P_{CT}^{(k)}$) of the controlled quantum teleportation. It may be defined as the difference between the conditioned fidelity ($f_{CT}(\rho_{AB}^{(k)})$) and the non-conditined fidelity ($f_{NC}(\rho_{AB})$)
\begin{eqnarray}
	P_{CT}^{(k)}&=&f_{CT}(\rho_{AB}^{(k)})-f_{NC}(\rho_{AB}),~~k=0,1
	\label{def} 
\end{eqnarray}
In the controlled quantum teleportation scheme, there are two basic assumptions: (i) $f_{CT}(\rho_{AB}^{(k)})>\frac{2}{3}$ and (ii) $f_{NC}(\rho_{AB})\leq \frac{2}{3}$. If these two conditions are satisfied by any three-qubit states then we say that the given three-qubit state is useful in CQT scheme.\\
%Initially, the scheme for the controlled quantum teleportation was proposed as a simple teleportation protocol with the pure three-qubit GHZ state as a resource state. Later, Li et al. studied Controlled quantum teleportation in the view of controller. They defined a new term known as Controller's Power. The faithfulness of Controlled Quantum Teleportation is estimated with the help of Controller's power, more the controller power better will be Controlled quantum teleportation\cite{li}.\\ 
The rest of the paper is organized as follows: In Section-II, we construct an witness operator to detect two-qubit entangled state and then discussed its importance by deriving the lower and upper bound of the expectation value of it with respect to any two-qubit entangled state. In section-III, we estimate the conditioned fidelity and non-conditioned fidelity and thus power of the controller in terms of the expectation value of the witness operator. In secion-IV, we study the controlled teleportation in noisy environment and analyze the relationship between the noisy parameter and the power of the controller. Lastly, we conclude the work.\\
%% Figure environment removed

	
\section{Witness operators}
\noindent An operator $W$ is said to be a witness operator if it satisfies the following two conditions \cite{Guhne}
\begin{eqnarray}
C1.~~ Tr[W\sigma_{sep}]\geq 0, \forall~ \text{separable state}~\sigma_{sep}~~~~~~~~~~~~~~~~~~~~~
\label{c1}\\
C2.~~ Tr[W\sigma_{ent}] < 0, \text{for at least one entangled}~ \text{state}~\sigma_{ent}
\label{c2}
\end{eqnarray}
In this section, our task is to construct witness operator, and study the relationship between the expected value of the constructed witness operator and the Bell-CHSH inequality. We have shown that the constructed witness operator may detect the two-qubit entangled state even when Bell-CHSH inequality unable to detect it. Moreover, we find that the two-qubit entangled states, which are not detected by Bell-CHSH inequality but detected by witness operator, are useful for teleportation.\\
In general, it has already been shown in the literature \cite{horo3} that any two-qubit state described by the density matrix $\rho_{AB}$ violates CHSH inequality if and only if $M(\rho_{AB})>1$. The quantity $M(\rho_{AB})$ may be defined as 
\begin{eqnarray}
	M(\rho_{AB})=u_{1}+u_{2}
	\label{mrho}
\end{eqnarray}
where $u_{1}$ and $u_{2}$ are the two maximum eigenvalues of $T^{\dagger}T$. T denote the $ 3 \otimes 3$ correlation matrix of $\rho_{AB}$ and its entries $t_{ij}$ can be calculated by the formula
\begin{eqnarray}
t_{ij}=Tr[\rho_{AB}(\sigma_{i}\otimes\sigma_{j})],~~ i,j=\{1,2,3\}
\end{eqnarray}
\subsection{Construction of witness operator $W^{(1)}_{ij}$}  
\noindent
 To start with, let us first recall different Bell-CHSH operator defined in $xy-$, $yz-$ and $zx-$ plane, which are collectively denoted as $B^{(ij)}_{CHSH}~(i,j=x,y,z;i\neq j)$ and it is given by \cite{hyllus}
\begin{eqnarray}
B^{(ij)}_{CHSH}&=&\sigma_{i}\otimes \frac{\sigma_{i}+\sigma_{j}}{\sqrt{2}}+ \sigma_{i}\otimes \frac{\sigma_{i}-\sigma_{j}}{\sqrt{2}}\nonumber\\&+&
\sigma_{j}\otimes \frac{\sigma_{i}+\sigma_{j}}{\sqrt{2}}- \sigma_{j}\otimes \frac{\sigma_{i}-\sigma_{j}}{\sqrt{2}}
\label{bchsh}
\end{eqnarray}
Afterward, we will use the short form $B_{ij}$ instead of using the long form $B^{(ij)}_{CHSH}$ throughout the paper. The four Bell states in the computational basis are denoted by $|\phi^{\pm}\rangle$, $|\psi^{\pm}\rangle$ and can be expressed as
\begin{eqnarray}
|\phi^{+}\rangle = \frac{|00\rangle +|11\rangle}{\sqrt{2}}
\label{bellbasis1}\\
|\phi^{-}\rangle = \frac{|00\rangle -|11\rangle}{\sqrt{2}}
\label{bellbasis2}\\
|\psi^{+}\rangle = \frac{|01\rangle +|10\rangle}{\sqrt{2}}
\label{bellbasis3}\\
|\psi^{-}\rangle = \frac{|01\rangle -|10\rangle}{\sqrt{2}}
\label{bellbasis4}
\end{eqnarray}
Now we are in a position to construct the operator $W_{ij}$ that may be expressed in the form as
\begin{equation}
W_{ij}= (\frac{1}{2}+2a)I-A-aB_{ij} ,  i,j=x,y,z~\&~ i\neq j
\label{witdef1}
\end{equation}
where $a$ is a positive real number. The operator $A$  given in Eq. (\ref{witdef1}) may take any form of two-qubit Bell states and other operators have their usual meaning.
In particular, if we take $A=|\phi^{+}\rangle\langle \phi^{+}|$ then the operator $W_{ij}$ reduces to $W^{(\phi^{+})}_{ij}$, where $W^{(\phi^{+})}_{ij}$ is given by
\begin{equation}
W^{(\phi^{+})}_{ij}= (\frac{1}{2}+2a)I-|\phi^{+}\rangle \langle \phi^{+}|-aB_{ij},  i,j=x,y,z~\&~ i\neq j
\label{witdef}
\end{equation}
\textbf{Theorem 1:} The operator $W^{(\phi^{+})}_{ij}$ given in Eq. (\ref{witdef}) is a witness operator.\\
\textbf{Proof:} We call the operator $W^{(\phi^{+})}_{ij}$, a witness operator if it satisfies the conditions $C1$ and $C2$ given in (\ref{c1}) and  (\ref{c2}) respectively.\\
\textbf{(a)} To show the validity of condition $C1$, take the operator $W^{(\phi^{+})}_{ij}$ and consider an arbitrary two-qubit separable state described by the density operator $\sigma_{sep}$. The expectation value of the operator $W^{(\phi^{+})}_{ij}$ with respect to  $\sigma_{sep}$ is given by 
\begin{equation}
 Tr[W^{(\phi^{+})}_{ij}\sigma_{sep}]= (\frac{1}{2}+2a)-\langle
 \phi^{+}|\sigma_{sep}|\phi^{+}\rangle -aTr[B_{ij}\sigma_{sep}]
\label{expecval1}
 \end{equation}
If $F(\sigma_{sep})$ denote the singlet fraction \cite{Bennett} of the state $\sigma_{sep}$ then we have 
\begin{eqnarray}
F(\sigma_{sep}) \geq \langle \phi^{+}|\sigma_{sep}|\phi^{+}\rangle
\label{singfrac}
\end{eqnarray}
Using Eq. (\ref{singfrac}) in Eq. (\ref{expecval1}), we get
\begin{eqnarray}
Tr[W^{(\phi^{+})}_{ij}\sigma_{sep}] &\geq& (\frac{1}{2}+2a)-F(\sigma_{sep}) -aTr[B_{ij}\sigma_{sep}]\nonumber\\
%\nonumber\\&\geq& \frac{1}{2}-F(\sigma_{sep})\nonumber\\&\geq& 0
\label{expecval}
\end{eqnarray}
For any separable state $\sigma_{sep}$, we have $-2\leq Tr[B_{ij}\sigma_{sep}] \leq 2$. Thus, for $a>0$, the inequality (\ref{expecval}) reduces to
\begin{equation}
Tr[W^{(\phi^{+})}_{ij}\sigma_{sep}] \geq \begin{cases}
\frac{1}{2}-F(\sigma_{sep})+2a, & Tr[B_{ij}\sigma_{sep}]\in [-2,0]\\
\frac{1}{2}-F(\sigma_{sep}), & Tr[B_{ij}\sigma_{sep}]\in [0,2]	\end{cases}
\end{equation}
Since the singlet fraction of any separable state $\sigma_{sep}$ satisfies the inequality $F(\sigma_{sep}) \leq  \frac{1}{2}$ so for any $a>0$ and for any separable state $\sigma_{sep}$, we have $Tr[W^{(\phi^{+})}_{ij}\sigma_{sep}] \geq 0$. Hence $C1$ is verified.\\
\textbf{(b)} To prove the validity of the condition $C2$, it is enough to show that there exist an entangled state $\sigma_{ent}$ for which $Tr[W^{(\phi^{+})}_{ij}\sigma_{ent}]<0$. For this, let us consider an entangled state $\sigma_{ent}$, which may be defined as \cite{Mhorodecki}
\begin{eqnarray}
\sigma_{ent}&=& p|\phi^{+}\rangle \langle \phi^{+}|+ \frac{1-p}{4}I , \frac{1}{3}<p \leq 1 
\label{state}
\end{eqnarray}
Let us now consider the operator $B_{yz}$ which is defined in the interval $\frac{1}{3}<p \leq 1$. In this interval, we find that the state $\sigma_{ent}$ satisfy the Bell-CHSH inequality as it is clearly evident from the equation given below
\begin{eqnarray}
Tr[B_{yz}\sigma_{ent}]= 0
\label{byz}
\end{eqnarray}
%Therefore, the following observation we can made for the state $\sigma_{ent}$ with respect to the Bell-CHSH inequality:\\
%(i) If the Bell operator $B_{xy}$ and $B_{yz}$ is defined in the whole entangled region i.e. when $\frac{1}{3}<p \leq 1$ then the entangled state $\sigma_{ent}$ satisfy the Bell-CHSH inequality.\\
%(ii) If the Bell operator $B_{xz}$ is defined in the interval $\frac{1}{3}<p\leq \frac{1}{\sqrt{2}}$ then the entangled state $\sigma_{ent}$ satisfy the Bell-CHSH inequality.\\
%(iii) If the Bell operator $B_{xz}$ is defined for the region $ \frac{1}{\sqrt{2}}< p \leq 1$ then the entangled state $\sigma_{ent}$ violate the Bell-CHSH inequality.\\
The operator $W^{(\phi^{+})}_{yz}$ may be expressed as 
\begin{equation}
W^{(\phi^{+})}_{yz}= (\frac{1}{2}+2a)I-|\phi^{+}\rangle \langle \phi^{+}|-aB_{yz},~~a>0
\label{witdefyz}
\end{equation}
The expectation value of the operator $W^{(\phi^{+})}_{yz}$ with respect to the state $\sigma_{ent}$ is given by 
\begin{eqnarray}
Tr[W^{(\phi^{+})}_{yz}\sigma_{ent}]&=& \frac{1}{2}+ 2a - \langle \phi^{+}|\sigma_{ent}|\phi^{+}\rangle -aTr[B_{yz}\sigma_{ent}]\nonumber \\
&=& \frac{1}{2}+ 2a -\frac{1+3p}{4}\nonumber \\
&=& \frac{1-3p}{4}+ 2a \nonumber \\
&<&  0,~ a \in [0,0.001],~\frac{1}{3}<p\leq 1 \nonumber
\label{expecval3}
\end{eqnarray} 
%Taking $a=0.001$ in (\ref{expecval3}), we get 
%\begin{eqnarray}
%Tr[W_{yz}\sigma_{ent}]=\frac{3}{4}(0.336-p) <0,~~\frac{1}{3}<p\leq 1
%\label{expecval4}
%\end{eqnarray}
Thus, there exist an entangled state $\sigma_{ent}$ for which $Tr[W^{(\phi^{+})}_{yz}\sigma_{ent}]<0$. Therefore, $C2$ is verified.\\
Thus, we can now say that the operator $W^{(\phi^{+})}_{yz}$ may serve as a valid entanglement witness operator. Similarly, it can be shown that there exist a range of the parameter $a$ for which $Tr[W^{(\phi^{+})}_{xy}\sigma_{ent}]<0$ and  $Tr[W^{(\phi^{+})}_{zx}\sigma_{ent}]<0$. Hence, the witness operator $W^{(\phi^{+})}_{ij}$ for any $i,j=x,y,z;i\neq j$ is a witness operator.\\
Moreover, if we replace the operator $A$ by other Bell states like $|\phi^{-}\rangle\langle \phi^{-}|$ or $|\psi^{\pm}\rangle\langle \psi^{\pm}|$ then it can be shown that the corresponding operators $W^{(\phi^{-})}_{ij}$ or $W^{(\psi^{\pm})}_{ij}$ may serve as witness operator for any $i,j=x,y,z;i\neq j$. Therefore, the witness operators $W^{(\phi^{-})}_{ij}$, $W^{(\psi^{\pm})}_{ij}$ may be expressed in the following way:\\
\begin{equation}
W^{(\phi^{-})}_{ij}= (\frac{1}{2}+2a)I-|\phi^{-}\rangle \langle \phi^{-}|-aB_{ij} ,  i,j=x,y,z~~ i\neq j 
\label{witdef11}
\end{equation}
\begin{equation}
W^{(\psi^{\pm})}_{ij}= (\frac{1}{2}+2a)I-|\psi^{\pm}\rangle \langle \psi^{\pm}|-aB_{ij} ,  i,j=x,y,z~~ i\neq j 
\label{witdef2}
\end{equation}


%It may be noted that during the construction of witness operator $W_{ij}$, one may use any pair of Bell operators and Bell states such as $(B_{ij},|\phi^{\pm}\rangle),i,j=x,y,z;i\neq j$ or $(B_{ij},|\psi^{\pm}\rangle),i,j=x,y,z;i\neq j$, where the Bell states $|\phi^{\pm}\rangle$ and $|\psi^{\pm}\rangle$ are given by
%\begin{eqnarray}
%|\phi^{\pm}\rangle= \frac{|00\rangle \pm|11\rangle}{\sqrt{2}},|\psi^{\pm}\rangle=\frac{|01\rangle \pm|10\rangle}{\sqrt{2}}
%\end{eqnarray}
%Therefore, the witness operator $W_{ij}$ may also be defined in the following way
%\begin{equation}
%W_{ij}= (\frac{1}{2}+2a)I-A-aB_{ij} ,  i,j=x,y,z~\&~ i\neq j
%\label{witdef1}
%\end{equation}
%where the operator $A$ may take any Bell states of the following form 
%\begin{equation}
%|\phi^{-}\rangle \langle \phi^{-}|,|\psi^{+}\rangle \langle \psi^{+}|,|\psi^{-}\rangle \langle \psi^{-}|
%\end{equation}
\subsection{Characteristic of the introduced witness operator}
\noindent In this section, we may take into account the Bell state $|\phi^{+}\rangle$ and then discuss the relation between the three quantities such as (i) $M(\rho)$, which determine whether the quantum state violating the Bell-CHSH inequality (ii) $F(\rho)$ denoting the singlet fraction of the state $\rho$ that determine whether the state is useful as a resource in quantum teleportation \cite{horo4} and (iii) the expectation value of the witness operator $W^{(\phi^{+})}_{ij}$ that detect the signature of the entanglement. Specifically, we derived here the lower and upper bound of the the expectation value of the witness operator $W^{(\phi^{+})}_{ij}$. Using these bounds, we have obtained few results that focusses on the condition for which the witness operator may detect or may not detect the entangled state. Furthermore, we note that all the results obtained by considering the operator $|\phi^{+}\rangle\langle \phi^{+}|$ may also be obtained by considering the other three Bell operators such as $|\phi^{-}\rangle\langle \phi^{-}|$, $|\psi^{\pm}\rangle\langle \psi^{\pm}|$.\\ %namely, $|\phi^{-}\rangle\langle \phi^{-}|$ or $|\psi^{+}\rangle\langle \psi^{+}|$ or $|\psi^{-}\rangle\langle \psi^{-}|$. One may also verify that the corresponding operators $W^{(\phi^{-})}_{ij}$ or $W^{(\psi^{+})}_{ij}$ or $W^{(\psi^{-})}_{ij}$ may serve as witness operator for any $i,j=x,y,z;i\neq j$.\\  
\textbf{Result-1:} Consider an entangled state $\rho_{ent}$ such that $M(\rho_{ent})\leq 1$. Then the lower and upper bound of the expectation value of the witness operator $W^{(\phi^{+})}_{ij}$ with respect to an entangled state $\rho_{ent}$ is given by
\begin{eqnarray}
U(a) \leq	Tr[W^{(\phi^{+})}_{ij}\rho_{ent}] \leq L(a) ,~~a>0
\label{p1}
\end{eqnarray}
where $U(a)=(\frac{1}{2}-F(\rho_{ent}))+2a(1-\sqrt{M(\rho_{ent})})$ and $L(a)=\frac{1}{2}- \langle \phi^{+}|\rho_{ent}|\phi^{+}\rangle +4a$.\\
The proof is given in the Appendix-A.\\
The inequality (\ref{p1}) estimates the lower and upper bound of  the expectation value of the witness operator $W^{(\phi^{+})}_{ij}$ with respect to any two-qubit entangled state. Further, we note that the lower bound of $Tr[W^{(\phi^{+})}_{ij}\rho_{ent}]$ depends on two inequalities such as (i) the singlet fraction $F(\rho_{ent})$ and (ii) a quantity $M(\rho_{ent})$. Based on these two quantities, %that fails to determine the non-locality of the two-qubit entangled state $\rho_{ent}$. Thus,
we can make the following observations from the inequality (\ref{p1}):\\
\textbf{Observation-1:} If there exist any two-qubit entangled state $\rho_{ent}$ such that $M(\rho_{ent})\leq 1$ and $F(\rho_{ent})\leq \frac{1}{2}$ then it is clear from Eq. (\ref{p1}) that the witness operator $W^{(\phi^{+})}_{ij}$ cannot detect the entangled state $\rho_{ent}$.\\
This observation may be illustrated by the following example: Let us consider the two-qubit state \cite{versatrate}
\begin{eqnarray}
	\rho_{F}&=& F|\phi^{+}\rangle \langle \phi^{+}|+ (1-F)|01\rangle \langle 01| , ~~\frac{1}{3}<F \leq \frac{1}{2} \nonumber \\
\end{eqnarray}
where $F$ denotes the singlet fraction of the state. One may easily verify that the state $\rho_{F}$ is an entangled state when $\frac{1}{3}<F \leq \frac{1}{2}$.\\
The expectation value of the Bell operators $B_{xy}$, $B_{yz}$ and $B_{xz}$ in different setting with respect to the state $\rho_{F}$ is given by
\begin{eqnarray}
	%Tr[W\rho_{F}]=\frac{F}{2}>0,~\frac{1}{3}<F \leq 1
	%\label{f1}\\
	\langle B_{xy} \rangle_{\rho_{F}}=0,
	\label{f1}\\
	\langle B_{yz} \rangle_{\rho_{F}}\in (-0.9428,-0.7071), 
	\label{f21}\\
	\langle B_{xz} \rangle_{\rho_{F}} \in (0,0.707107)
	\label{f31}
\end{eqnarray}
%Thus, it can be easily seen from Eq. (\ref{f1}), Eq. (\ref{f2}), and Eq. (\ref{f3}) that the state $\rho_{F}$ defined in $0.8047< F \leq 1$ violate the Bell-CHSH inequality by the Bell operator $B_{xz}$, while in the remaining region, it does not violate the Bell-CHSH inequality. On the other hand, the state $\rho_{F}$ does not violate the Bell CHSH inequality by any of the Bell operator $B_{xy}$ or $B_{yz}$ when $F \in (\frac{1}{3},1]$. Eq. (\ref{f3}) also illustrate the fact that the state violate Bell-CHSH inequality when it is useful for teleportation \cite{horo4}.\\
Therefore, using using (\ref{f1}), (\ref{f21}) and (\ref{f31}), we can find that the state $\rho_{F}$ satisfies the Bell-CHSH inequality. Let us now calculate the expectation value of the corresponding witness operators $W^{(\phi^{+})}_{xy}$, $W^{(\phi^{+})}_{yz}$, and $W^{(\phi^{+})}_{zx}$ with respect to the state $\rho_{F}$. For positive 'a', the expectation values are given by 
\begin{eqnarray}
	Tr[W^{(\phi^{+})}_{xy}\rho_{F}]&=& \frac{1}{2}+2a-F>0 %~~~~\text{for}~ a>0,
	\label{wx1}\\
	Tr[W^{(\phi^{+})}_{yz}\rho_{F}]&=& (\frac{1}{2}-F)+a[2+\sqrt{2}(1-F)]>0 \nonumber\\ %&& \text{for}~ a>0
	\label{wy2}\\
	Tr[W^{(\phi^{+})}_{zx}\rho_{F}]&=& (\frac{1}{2}-F)+a[2+\sqrt{2}(1-3F)]>0 \nonumber\\ %&& \text{for}~ a>0
	\label{wz3}
\end{eqnarray}
Thus, it is clear from (\ref{wx1}), (\ref{wy2}) and (\ref{wz3}) that the entangled state $\rho_{F}$,% which does not violate the Bell-CHSH inequality and not useful for teleportation, 
is not detected by the witness operator $W^{(\phi^{+})}_{xy}$. The observation-1 is now verified for a particular quantum state described by the density operator $\rho_{F}$. But, in general, from the inequality (\ref{p1})  we can conclude that if any quantum entangled state $\rho$ satisfies $M(\rho)\leq 1$ and $F(\rho)\leq \frac{1}{2}$ then the witness operator $W^{(\phi^{+})}_{ij},~i,j=x,y,z,~i\neq j $ does not detect the entangled state $\rho$.\\
\textbf{Observation-2:} If there exist any two-qubit entangled state $\rho_{ent}$ which is useful in teleportation i.e. $F(\rho_{ent})> \frac{1}{2}$ then the witness operator $W^{(\phi^{+})}_{ij}$ may detect the entangled state $\rho_{ent}$ when the parameter $a$ lies in some specific range. This observation may be written in the form of another result, which is stated below:\\
\textbf{Result-2:} Let us consider a two-qubit entangled state described by a density operator $\rho_{ent}$. If $F(\rho_{ent}) > \frac{1}{2}$
and if the parameter $a$ lies in the range $0<a\leq \frac{\langle \phi^{+}|\rho_{ent}| \phi^{+}\rangle-\frac{1}{2}}{4}$ then the witness operator $W_{ij}^{(\phi^{+})}$ detect the entangled state $\rho_{ent}$. \\
\textbf{Proof:} The expectation value of the witness operator $W_{ij}^{(\phi^{+})}$ $(i,j= x,y,z; i\neq j)$ with respect to the entangled state $\rho_{ent}$ can be written as
\begin{eqnarray}
Tr[W_{ij}^{(\phi^{+})}\rho_{ent}] &=& \frac{1}{2}- \langle \phi^{+}|\rho_{ent}| \phi^{+}\rangle +a\times\nonumber\\&&(2-Tr[B_{ij}\rho_{ent}]) 
\label{w1}
\end{eqnarray}
Using Eq. (\ref{p1}), it can be easily shown that if $F(\rho_{ent})>\frac{1}{2}$ and if whether the state $\rho_{ent}$ satisfies the Bell-CHSH inequality or violate it, the upper and lower bound of $Tr[W_{ij}^{(\phi^{+})}\rho_{ent}]$ will be a negative quantity. Thus, we have
\begin{eqnarray}
Tr[W_{ij}^{(\phi^{+})}\rho_{ent}]= \text{a negative quantity}
\label{w1neg}	
\end{eqnarray}
Therefore, (\ref{w1neg}) clearly indicate the fact that the witness operator $W_{ij}^{(\phi^{+})}$  $(i,j= x,y,z; i\neq j)$ detect the entangled state $\rho_{ent}$. Hence proved the result.\\
We will now verify Result-2 by considering the Bell state $|\psi^{-}\rangle$ instead of taking $|\phi^{+}\rangle$. To verify Result-2, let us consider the two-qubit state described by the density operator $\rho(\theta)$
\begin{eqnarray}
	\rho(\theta)= \frac{1}{2}\begin{pmatrix} 
		a(\theta) & 0& 0& 0\\
		0 & b(\theta) & c(\theta) & 0\\
		0 & c(\theta) & d(\theta) & 0\\
		0 & 0& 0& e(\theta) \\
	\end{pmatrix},~ 0\leq \theta \leq 0.4175\pi \nonumber \\
\end{eqnarray}
where $a(\theta)=(3-2\sqrt{2})Sin^{2}\theta $, $b(\theta)=(3-2\sqrt{2})Cos^{2}\theta$, $c(\theta)=(1-\sqrt{2})Cos\theta$, $d(\theta)=1+(2\sqrt{2}-2)Sin^{2}\theta$ and $e(\theta)=(2\sqrt{2}-2)Cos^{2}\theta$.\\ 
It can be easily verified that $\rho(\theta)$ is an entangled state and $M(\rho(\theta))<1$  for $\theta \in [0,0.4175\pi]$. Thus, the entangled state $\rho(\theta)$ will satisfy Bell-CHSH inequality for $\theta \in [0,0.4175\pi]$ and thus it is undetected by Bell-CHSH operator. Further, the singlet fraction of $\rho(\theta)$, i.e., $F(\rho(\theta))$ can be calculated as 
\begin{eqnarray}
F(\rho(\theta))= \frac{1}{8}(3 + 4 (-1 + \sqrt{2}) Cos\theta+ (5 - 4\sqrt{2}) Cos(2\theta)) \nonumber
\end{eqnarray}
We can verify that $F(\rho(\theta))> \frac{1}{2}$ when $\theta \in [0, 0.4175\pi]$ and $a \in (0,0.00560188]$.\\ 
By direct calculation, we obtain the value of the following expressions in terms of the state parameter $\theta$ as
\begin{eqnarray}
\langle B_{xy} \rangle_{\rho(\theta)}&=&2 (-2 + \sqrt{2}) Cos\theta%~ 0\leq \theta \leq 0.4175\pi 
\nonumber \\
\langle B_{yz} \rangle_{\rho(\theta)}=\langle B_{xz} \rangle_{\rho(\theta)}&=& \frac{1}{\sqrt{2}}(-1 - 2 (-1 + \sqrt{2}) Cos\theta \nonumber \\ &+&  (-5 + 4\sqrt{2} ) Cos2\theta)  \nonumber\\
Tr[\rho(\theta) |\psi^{-} \rangle\langle \psi^{-}|]&=& \frac{1}{8}(3 + 4 (-1 + \sqrt{2}) Cos\theta \nonumber \\ &+& (5 - 4\sqrt{2}) Cos(2\theta)) 
\end{eqnarray}
We are now in a position to calculate the expectation value of the witness operatr $W_{ij}^{(\psi^{-})}$ with respect to the state $\rho(\theta)$. It is given by 
\begin{eqnarray}
	Tr[W_{xy}^{(\psi^{-})}\rho(\theta)]&=& \frac{1}{8} (1 + 16 a - 
	4 (-1 + \sqrt{2} + 4 (-2 \nonumber \\ &+& \sqrt{2}) a) Cos\theta + (-5 + 4\sqrt{2}) Cos2\theta) \nonumber \\
	Tr[W_{yz}^{(\psi^{-})}\rho(\theta)]&=& \frac{1}{8} (1 + 16a + 4 \sqrt{2} a -
	4 (-1 + \sqrt{2} +\nonumber \\ &&  2 (-2 + \sqrt{2}) a) Cos\theta +(-5 + 4 \sqrt{2}+ \nonumber \\&& 4 (-8 + 5\sqrt{2}) a) Cos2\theta) \nonumber\\
Tr[W_{xz}^{(\psi^{-})}\rho(\theta)]&=& Tr[W_{yz}^{(\psi^{-})}\rho(\theta)] 
\end{eqnarray}
We find that the witness operator $W_{xy}^{(\psi^{-})}$ detect the state $\rho(\theta)$ when $a\in (0,0.00032]$ \& $\theta \in [0,0.4175\pi]$. We also find that witness operator $W_{yz}^{(\psi^{-})}$ \& $W_{xz}^{(\psi^{-})}$ detect the state $\rho(\theta)$ when $a\in (0,0.00016]$ \& $0\leq \theta < 0.4175\pi $. Therefore, there exist a range of the parameter $a$ for which the entangled state $\rho(\theta)$ is detected by the witness operator $W_{ij}^{(\psi^{-})}$ when $i,j=x,y,z;i\neq j$.\\
Now, we are in a position to derive the non-trivial lower bound of the teleportation fidelity when $\rho_{ent}$ is used as a resource state in quantum teleportation. It may be expressed in terms of the expectation value of the witness operator and M($\rho_{ent}$),\\
\textbf{Result-3:} If there exist an entangled state described by the density operator $\rho_{ent}$, which satisfies the Bell-CHSH inequality but detected by the witness operator $W^{(\phi^{+})}_{ij}$, then the entangled state $\rho_{ent}$ is useful in teleportation with teleportation fidelity $f(\rho_{ent})$, which satisfies the inequality
\begin{eqnarray}
f(\rho_{ent}) \geq	\frac{2}{3}\{1-Tr[W^{(\phi^{+})}_{ij}\rho_{ent}]+2a(1-\sqrt{M(\rho_{ent})})\}\nonumber\\
\label{fid1}
\end{eqnarray} 
where $a\in (0,\frac{\frac{1}{2}+Tr[W^{(\phi^{+})}_{ij}\rho_{ent}]}{2(1-\sqrt{M(\rho_{ent})})}]$.\\
The proof of Result-3 is given in Appendix-B.
\section{Estimation of controller's power} 
\noindent We have assumed here that the controlled teleportation scheme involve three parties namely Alice (A), Bob (B) and Charlie (C), who have shared a three-qubit state. In this protocol, the measurement is performed by Charlie (acting as a controller) on his qubit. As a result of the measurement, the two-qubit state will be shared between Alice and Bob described by the density operator $\rho_{AB}$. The shared state $\rho_{AB}$ may or may not violate the Bell-CHSH inequality and accordingly the state may or may not be useful in the conventional teleportation scheme \cite{horo4}. Therefore, the study of the violation of Bell-CHSH inequality is important in this scenario and thus we consider it here as the CHSH game \cite{jon}. In the CHSH game, we assume that the two distant players, Alice (A) and Bob (B) receive binary questions $s, t \in \{0, 1\}$ respectively, and similarly their answers $a, b \in \{0, 1\}$ are single bits. Alice
and Bob win the CHSH game if their answers satisfy $a \oplus b = s ·t$. Thus, CHSH game can be considered as a particular example of XOR games. In this game, the non-locality of the shared state $\rho_{AB}$ may be determined when Alice and Bob perform
measurements on their respective qubit and the outcomes
of their measurements are correlated. Therefore, the maximum probability $P_{ij}$ of winning the game overall strategy is given by \cite{jon} 
\begin{eqnarray}
P_{ij}&=&\frac{1}{2}[1+\frac{\langle B_{ij}\rangle _{\rho_{AB}}}{4}]
\label{pmax}
\end{eqnarray}
where $\langle B_{ij}\rangle _{\rho_{AB}}={Tr\rm}[(A_{0}\otimes B_{0}+A_{0}\otimes B_{1}+A_{1}\otimes B_{0}-A_{1}\otimes B_{1})\rho_{AB}]$. Since the maximum probability of winning the game depends on the expectation value of the Bell operator $B_{ij}$, so $P^{max}$ is somehow related to the non-locality of the state $\rho_{AB}$. Adding the known fact that the state $\rho_{AB}$ violate the Bell-CHSH inequality if $\langle B_{ij}\rangle _{\rho_{AB}}>2$ and thus, we find that the state $\rho_{AB}$ is non-local when $P^{max}>\frac{3}{4}$. Hence, the shared state $\rho_{AB}$ may be useful for teleportation when $P^{max}>\frac{3}{4}$.\\
It may be easily shown that the winning probability $P_{ij}$ may be estimated in terms of the expectation value of the witness operator $W_{ij}$ with respect to the state $\rho_{AB}$. The result may be stated as\\
\textbf{Lemma:} The probability $P_{ij}$ of the CHSH game may be estimated as\\
(i) When $F(\rho_{AB})\leq \frac{1}{2}$
\begin{eqnarray}
\frac{3}{4}-\frac{Tr[W_{ij}\rho_{AB}]}{8a} \leq P_{ij} \leq 1
\label{pmaxest1}
\end{eqnarray}
(i) When $F(\rho_{AB})> \frac{1}{2}$
\begin{eqnarray}
0\leq P_{ij} < \frac{3}{4}-\frac{Tr[W_{ij}\rho_{AB}]}{8a}
\label{pmaxest2}
\end{eqnarray}
where $F(\rho_{AB})$ denote the singlet fraction of the state $\rho_{AB}$.\\
\textbf{Proof:} Let us recall the witness operator $W_{ij}^{(\phi^{+})}$ given in (\ref{witdef}). Therefore, Using (\ref{witdef}) and ($\ref{pmax}$), the expression for $P_{ij}$ may be re-written as
\begin{eqnarray}
P_{ij}= \frac{3}{4}+\frac{1}{8a}[\frac{1}{2}- \langle\phi^{+}|\rho_{AB}|\phi^{+}\rangle-  Tr[W_{ij}^{(\phi^{+})}\rho_{AB}]] \nonumber \\
\label{pmaxexp}
\end{eqnarray}
Using the fact that $\langle\phi^{+}|\rho_{AB}|\phi^{+}\rangle \leq F(\rho_{AB})$ and considering the two different cases i) $F(\rho_{AB}) \leq \frac{1}{2}$ and ii) $F(\rho_{AB})> \frac{1}{2}$ separately, we can easily obtain the above estimation given in (\ref{pmaxest1}) and (\ref{pmaxest2}). The above result may be proved in the same way for $W_{ij}^{(\phi^{-})}$, $W_{ij}^{(\psi^{+})}$ and $W_{ij}^{(\psi^{-})}$. Hence proved.\\
In this section, we aim to derive the lower and upper bound of the controller's power using the winning probability of the CHSH game. We found that the controller's power ($P_{CT}^{(k)}$) is always upper bounded by $\frac{1}{2}$ while the lower bound may be estimated in terms of a quantity $M(\rho^{(k)}_{AB})$ and the expectation value of the witness operator $W_{ij}^{(\phi^{+})}$ with respect to the state $\rho^{(k)}_{AB}$. The state $\rho^{(k)}_{AB}$ has been obtained when charlie perform measurement $B_{k}$ on his qubit.\\
\subsection{Estimation of non-conditioned teleportation fidelity}
\noindent Let us suppose that the three-qubit state shared between Alice (A), Bob (B) and Charlie (C) is described by the density operator $\rho_{ABC}$. The reduced two-qubit state shared between Alice and Bob is described by the density operator $\rho_{AB}=Tr_{C}(\rho_{ABC})$. If $\rho_{AB}$ is used as a resource state in quantum teleportation then the faithfulness of the teleportation is determined by the non-conditioned teleportation fidelity which is denoted by $f_{NC}(\rho_{AB})$. %when Alice and Bob adopt the conventional teleportation scheme to transmit information. 
The non-conditioned fidelity can be expressed in terms of the correlation tensor $T_{AB}$ as \cite{horo4} 
\begin{eqnarray}
f_{NC}(\rho_{AB})= \frac{3+||T_{AB}||_{1}}{6}
\label{fnctensor}
\end{eqnarray}
where $||.||_{1}$ denote the trace norm.\\
To express $f_{NC}(\rho_{AB})$ in terms of witness operator, we recall the expression of $P_{ij}$ given in (\ref{pmaxexp}). It may be re-written as
\begin{eqnarray}
P_{ij}= \frac{3}{4}+\frac{1}{8a}(\frac{1}{2}- \langle\phi^{+}|\rho_{AB}|\phi^{+}\rangle-  Tr[W_{ij}^{(\phi^{+})}\rho_{AB}])\nonumber
\end{eqnarray}
Using $\langle\phi^{+}|\rho_{AB}|\phi^{+}\rangle \leq F(\rho_{AB})$ in the expression of $P_{ij}$, we get the inequality as
\begin{eqnarray}
Tr[W_{ij}^{(\phi^{+})}\rho_{AB}]\geq 8a(\frac{3}{4}-P_{ij})+\frac{1}{2}-F(\rho_{AB})
\end{eqnarray}
One of the assumption to execute the controlled quantum teleportation scheme is that the non-conditioned teleportation fidelity must be less than $\frac{2}{3}$. Thus, considering $F(\rho_{AB})\leq \frac{1}{2}$ and $P_{ij}\leq \frac{3}{4}$, we get 
\begin{eqnarray}
\frac{1}{2}-Tr[W_{ij}^{(\phi^{+})}\rho_{AB}] \leq	F(\rho_{AB})\leq \frac{1}{2}
\label{fnc}
\end{eqnarray}
Using the relation between singlet fraction $(F(\rho_{AB}))$ and non-conditioned teleportation fidelity $(f_{NC}(\rho_{AB}))$, the inequality (\ref{fnc}) may be expressed in terms of $(f_{NC}(\rho_{AB}))$. Therefore, the inequality (\ref{fnc}) may be re-expressed as 
\begin{eqnarray}
\frac{2}{3}(1-Tr[W_{ij}^{(\phi^{+})}\rho_{AB}])\leq f_{NC}(\rho_{AB})\leq \frac{2}{3}
\label{fnc1}
\end{eqnarray} 
While constructing the witness operator $W_{ij}^{(\phi^{+})}$, we should be careful in choosing the positive value of the parameter $a$. The value of $a$ is choosen in such a way that $Tr[W_{ij}^{(\phi^{+})}\rho_{AB}]\geq 0$.
\subsection{Estimation of the conditioned teleportation fidelity}
\noindent In the controlled teleportation protocol, when the controller charlie measures on his qubit, the three-qubit state $\rho_{ABC}$ reduces to $\rho_{AB}^{(k)}$ according to the measurement outcome $k=0,1$. If Alice and Bob uses the shared state $\rho_{AB}^{(k)}$ as a resource state in the teleportation protocol then the fidelity of the teleportation may be termed as conditioned teleportation fidelity and it is denoted by $f_{C}(\rho_{AB}^{(k)})$. There is an interesting relationship between the conditioned teleportation fidelity and the partial tangle $\tau_{AB}$ and it is given by \cite{lee}
\begin{eqnarray}
f_{C}(\rho_{AB}^{(k)})=\frac{2+\tau_{AB}^{(k)}}{3}, k=0,1
\label{confidtau}
\end{eqnarray} 
To implement the controlled quantum teleportation, it is assumed that $f_{C}(\rho_{AB}^{(k)})>\frac{2}{3}$.
Therefore, the conditioned teleportation fidelity $f_{C}(\rho_{AB}^{(k)})$ may be estimated by using the $Result-3$
\begin{eqnarray}
L_{C} \leq f_{C}(\rho_{AB}^{(k)}) 
\label{fid12}
\end{eqnarray} 
where $L_{C}=\frac{2}{3}\{1-Tr[W^{(\phi^{+})}_{ij}\rho_{AB}^{(k)}]+2a(1-\sqrt{M(\rho_{AB}^{(k)})})\}$.\\
The condition of controlled teleportation will be met when the witness operator detect the entangled state $\rho_{AB}^{(k)}$ and also when the entangled state $\rho_{AB}^{(k)}$ satisfies the Bell-CHSH inequality. The value of the parameter $a$ involved in the witness operator will be chosen in such a way that the witness operator detect $\rho_{AB}^{(k)}$.    

\subsection{Lower and Upper bound of the controller's power}
\noindent The power of the controlled quantum teleportation for the $k^{th}$ measurement outcome may be defined as
\begin{eqnarray}
P_{CT}^{(k)}= f_{C}(\rho_{AB}^{(k)})- f_{NC}(\rho_{AB}), k=0,1
\label{power1}
\end{eqnarray}
Using (\ref{fnctensor}) and (\ref{confidtau}), the expression of the power given in (\ref{power1}) reduces to
\begin{eqnarray}
P_{CT}^{(k)}=\frac{1}{6}+\frac{1}{6}(2\tau_{AB}^{(k)}-\lVert T_{AB}\rVert_{1})
\label{power2} 
\end{eqnarray}
Our task is now to estimate the value of $||T_{AB}||_{1}$ and $\tau_{AB}^{(k)}$.\\
\textbf{(i) Estimation of $||T_{AB}||_{1}$:} Let us recall (\ref{fnc1}) and using (\ref{fnctensor}) in it, we get the estimation of $||T_{AB}||_{1}$ which is given by
\begin{eqnarray}
1-4Tr[W_{ij}^{NC}\rho_{AB}]\leq ||T_{AB}||_{1} \leq 1
\label{esttensor} 
\end{eqnarray}
In this case, the parameter $a$ is chosen in such a way that the witness operator $W_{ij}^{NC}$ does not detect the state $\rho_{AB}$. Therefore, we can put the restriction on $Tr[W_{ij}^{NC}\rho_{AB}]$ as
\begin{eqnarray}
0\leq Tr[W_{ij}^{NC}\rho_{AB}]\leq \frac{1}{4}
\label{restriction} 
\end{eqnarray} 
The upper bound of $Tr[W_{ij}^{NC}\rho_{AB}]$ is obtained by using the condition $||T_{AB}||_{1}\geq 0$.\\
\textbf{(ii) Estimation of $\tau_{AB}^{(k)}$:} Using (\ref{confidtau}) and (\ref{fid12}) and simplifying the inequality, we get
\begin{eqnarray}
4a(1-\sqrt{M(\rho_{AB}^{(k)})})-2Tr[W_{ij}^{C}\rho_{AB}^{(k)}] \leq \tau_{AB}^{(k)} \leq 1
\label{esttau} 
\end{eqnarray} 
The parameter $a$ in the LHS of above inequality (\ref{esttau}) can be chosen in such a way that the witness operator $W_{ij}^{C}$ detect the state $\rho_{AB}^{k}$.\\
Now, we are in a position to derive the lower and upper bound of the power $P_{CT}^{(k)}$. To start with, let us
use the upper bound of $||T_{AB}||_{1}$ and the lower bound of $\tau_{AB}^{(k)}$ in the expression (\ref{power2}) of the power of the controlled teleportation. Therefore, it reduces the power given in (\ref{power2}) to the inequality that gives the lower bound as 
\begin{eqnarray}
\frac{4a}{3} (1-\sqrt{M(\rho_{AB}^{(k)})})-\frac{2}{3}Tr[W_{ij}^{(C)}\rho_{AB}^{(k)}] \leq P_{CT}^{(k)}
\label{plb}
\end{eqnarray}
Similarly, using the lower bound of $||T_{AB}||_{1}$ and the upper bound of $\tau_{AB}^{(k)}$ in the expression of the power of the controlled teleportation, we get the upper bound of the power which is given by 
\begin{eqnarray}
P_{CT}^{(k)} \leq \frac{1}{3}+\frac{2}{3}Tr[W_{ij}^{NC}\rho_{AB}]
\label{power3}
\end{eqnarray} 
Further, if we use the restriction given in (\ref{restriction}) then the inequality (\ref{power3}) reduces to %the inequality that gives the upper bound of the power $P_{CT}^{(k)}$. The upper bound of the controlled teleportation is given by
\begin{eqnarray}
P_{CT}^{(k)} \leq \frac{1}{2}
\label{pub}
\end{eqnarray} 
Combining (\ref{plb}) and (\ref{pub}), we get 
\begin{eqnarray}
\frac{4a}{3} (1-\sqrt{M(\rho_{AB}^{(k)})})-\frac{2}{3}Tr[W_{ij}^{(C)}\rho_{AB}^{(k)}] \nonumber \\ \leq P_{CT}^{(k)} \leq \frac{1}{2}
\label{ublb}
\end{eqnarray}
\subsection{Estimation of the lower bound of the power for pure three-qubit states}
\noindent In this section, we study the controlled quantum teleportation protocol by considering the pure three-qubit states such as standard GHZ state, Maximally Slice State (MSS) and a W class of states. Then we estimate the lower bound of the power of the controller for all the above mentioned states.\\
\noindent Let us consider a three-qubit standard GHZ state of the form 
\begin{eqnarray}
|\psi^{(1)}\rangle_{CAB}&=& \lambda_{0}|000\rangle +\lambda_{4}|111\rangle,~~\lambda_{0}^{2}+\lambda_{4}^{2}=1  
\label{ghzstate}
\end{eqnarray}
Now, to execute the controlled teleportation scheme with the three-qubit state described by the density operator $\rho_{CAB}^{(1)}= |\psi^{(1)}\rangle_{CAB}\langle\psi^{(1)}|$, the assumptions on the non-conditioned fidelity and conditioned fidelity must be fulfilled. Therefore, we need to calculate the non-conditioned fidelity and conditioned fidelity and thus the power of the controller.\\
(i) Non-conditioned fidelity: We trace out system C from the three-qubit state $\rho_{CAB}^{(1)}$. The resulting two qubit state $\rho_{AB}^{(1)}$ is given by
\begin{eqnarray}
\rho_{AB}^{(1)}=Tr_{C}(\rho_{CAB}^{(1)})=\lambda_{0}^{2}|00\rangle \langle00| + \lambda_{4}^{2}|11\rangle \langle11|
\end{eqnarray}
Using $\rho_{AB}^{(1)}$ as a resource state in quantum teleportation, the non-conditioned fidelity can be calculated as
\begin{eqnarray}
f_{NC}(\rho_{AB}^{(1)})=\frac{2}{3}
\label{ncfghz}
\end{eqnarray}
(ii) Conditioned fidelity: Charlie, performed measurement on his qubit in the  single qubit generalised basis $\{B_{0},B_{1}\}$. After the measurement, the state collapses either to $\rho_{AB}^{G(0)}$ or  $\rho_{AB}^{G(1)}$, where  
\begin{eqnarray}
\rho_{AB}^{G(0)}&=&\frac{1}{p_{0}}\bigg( (t^{2}+y_{3}^{2})\lambda_{0}^{2}|00\rangle \langle00|+ \lambda_{0}\lambda_{4}(-ty_{2}+y_{1}y_{3}\nonumber \\ &+&\iota(ty_{1}+y_{2}y_{3}))|00\rangle \langle11| + \lambda_{0}\lambda_{4}(-ty_{2}+y_{1}y_{3}\nonumber \\ &-&\iota(ty_{1}+y_{2}y_{3}))|11\rangle \langle00|+ \lambda_{4}^{2}(y_{1}^{2}+y_{2}^{2}) |11\rangle \langle11|\bigg), \nonumber \\&&
\text{where}~~~~ p_{0}=(t^{2}+y_{3}^{2})\lambda_{0}^{2}+(y_{1}^{2}+y_{2}^{2})\lambda_{4}^{2}
\end{eqnarray}
\begin{eqnarray}
\rho_{AB}^{G(1)}&=&\frac{1}{p_{1}}\bigg( (y_{1}^{2}+y_{2}^{2})\lambda_{0}^{2}|00\rangle \langle00|+ \lambda_{0}\lambda_{4}(ty_{2}-y_{1}y_{3}\nonumber \\ &-&i(ty_{1}+y_{2}y_{3}))|00\rangle \langle11| + \lambda_{0}\lambda_{4}(ty_{2}-y_{1}y_{3}\nonumber \\ &+&i(ty_{1}+y_{2}y_{3}))|11\rangle \langle00|+ \lambda_{4}^{2}(t^{2}+y_{3}^{2}) |11\rangle \langle11|\bigg), \nonumber \\&&
\text{where}~~~~p_{1}=(y_{1}^{2}+y_{2}^{2})\lambda_{0}^{2}+(y_{3}^{2}+t^{2})\lambda_{4}^{2}
\end{eqnarray}
Let us now use the state $\rho_{AB}^{G(0)}$ for the teleportation of a single qubit. We choose the normalized measurement parameters $(y_{1},y_{2},y_{3},t)$ in such a way that the conditioned fidelity is greater than $\frac{2}{3}$ i.e. $f_{C}(\rho_{AB}^{G(0)})>\frac{2}{3}$ and also the normalization condition (\ref{vnm}) holds. Therefore, choosing the measurement parameters $y_{1}=-0.25$, $y_{2}=-0.49$, $y_{3}=0.39$ and $t=-0.74$, we can calculate the conditioned fidelity in terms of the state parameter $\lambda_{4}$ as 
\begin{eqnarray}
f_{C}(\rho_{AB}^{G(0)})&=& \frac{1}{1.79958-\lambda_{4}^{2}}(1.19972 - 0.666667\lambda_{4}^{2} \nonumber\\ &+& 0.799676\lambda_{4}\sqrt{1-\lambda_{4}^{2}})
\label{cfghz}
\end{eqnarray}
We may observe that $f_{C}(\rho_{AB}^{G(0)})$ varies from $[0.66667,0.99997]$ when $\lambda_{4}$ varies from $[0,1]$.\\
Thus, the assumptions on non-conditioned fidelity and conditioned fidelity are met. It can be easily verified that these assumptions still hold if we consider the state  $\rho_{AB}^{G(1)}$. This means that the GHZ state described by the density operator $\rho_{CAB}^{(1)}$ is useful for the controlled quantum teleportation.\\
Now, our task is to calculate the power of the controller when three-qubit GHZ state (\ref{ghzstate}) is shared between Alice, Bob and Charlie (controller). To estimate the power of the controller, we again consider the state $\rho_{AB}^{G(0)}$ and proceed toward the calculation of the lower bound of the power that need the following information:\\
(i) $M(\rho_{AB}^{G(0)})>1$ for $\lambda_{4}\in [0,1]$. This indicate that the state $\rho_{AB}^{G(0)}$ violate the Bell-CHSH inequality and therefore, the state is useful in conventional quantum teleportation \cite{horo4}.\\
(ii) The expectation value of the witness operator $W_{xy}^{(\phi^{+})}$ with respect to the state $\rho_{AB}^{G(0)}$ is given by
\begin{eqnarray}
Tr[W_{xy}^{(\phi^{+})}(\rho_{AB}^{G(0)})]= \frac{1}{2}-2a -\frac{\Lambda}{1.8-\lambda_{4}^{2}}
\end{eqnarray}
where $\Lambda=0.90 - 0.5\lambda_{4}^{2} + 1.2\lambda_{4}\sqrt{1-\lambda_{4}^{2}}$.\\
The value of $a~(>0)$ is chosen in such a way that the witness operator $W_{xy}^{(\phi^{+})}$ detects the state $\rho_{AB}^{G(0)}$. Thus, we find that when $a\in (0,0.232)$ \& $0.598\leq \lambda_{4} \leq 0.95 $, the witness operator detect the state $\rho_{AB}^{G(0)}$.  Therefore, the lower bound of the controller's power can be estimated as
\begin{eqnarray}
P_{CT}^{G(0)} \geq \frac{4a}{3} (1-\sqrt{M(\rho_{AB}^{G(0)})})-\frac{2}{3}Tr[W_{ij}^{(\phi^{+})}\rho_{AB}^{G(0)}] 
\end{eqnarray}
It can be easily verified that lower bound of power lies in the interval (0.001472,0.333) for $a\in (0,0.1555)$ \& $0.598\leq \lambda_{4} \leq 0.95$.\\
Also, We note that the calculation of the power of the controller for the state $\rho_{AB}^{G(1)}$ may be done in a similar way.\\
Moreover, we may consider other pure three-qubit states such as maximally slice state $|\psi^{(2)}\rangle_{ABC}= \lambda_{0}|000\rangle +\lambda_{1}|100\rangle+\frac{1}{\sqrt{2}}|111\rangle$ given in \cite{li2014} and W class states $|W_{n}\rangle=\frac{1}{\sqrt{2+2n}}(|100\rangle+\sqrt{n}|010\rangle+\sqrt{n+1}|001\rangle)$ introduced in \cite{patiagrawal}. We have analyzed the power of the controller for these classes of states, which is given in Table-I\label{t1}. We also find that the pure three-qubit W class of state described by $|W_{1}\rangle$ is more useful in CQT scheme than all other W class of states such as $|W_{2}\rangle$, $|W_{3}\rangle$ etc. $|W_{1}\rangle$ is more useful in CQT scheme in the sense that when $|W_{1}\rangle$ is used, the power of the controller is greater than all the power calculated over the states $|W_{2}\rangle$, $|W_{3}\rangle$ etc. This finding is shown in Fig.1.
% Figure environment removed 
\begin{table*}[!htbp]
\begin{tabular} {|c|c|c|c|}\hline 
			%\multicolumn{7}{c}{text}\hline 
Three-qubit State & Non-Conditioned Fidelity & Conditioned Fidelity & Estimated lower bound of the power \\  \hline 
$|\psi^{(2)}\rangle$ & [0.5,0.6667] & \begin{tabular}{c}$f_{C}(\rho_{AB}^{MSS(0)})$=$f_{C}(\rho_{AB}^{MSS(1)})$=[0.667538,1] \\$\lambda_{4} \in [0,0.643]$  \end{tabular}  & \begin{tabular}{c}
(0,0.3333), $a \in (0,0.0402]$	\\ $\lambda_{4} \in [0,0.6]$
\end{tabular} \\ \hline 
			
\multirow{2}{*}{$|W_{1}\rangle$} & \multirow{2}{*}{$\frac{2}{3}$} & $f_{C}(\rho_{AB}^{W_{1}(0)})$= 0.9999999999968732 & [0.00146455,0.33333), a$\in$(0,0.1767]  \\ \cline{3-4}
&  & $f_{C}(\rho_{AB}^{W_{1}(1)})$=0.99999999976277 & [0.00146455,0.33333), a$\in$(0,0.1767]  \\ \hline  
$|W_{2}\rangle$ & 0.657135 &$f_{C}(\rho_{AB}^{W_{2}(0)})$= $f_{C}(\rho_{AB}^{W_{2}(1)})$=0.980937 & [0.0001815,0.31427), a$\in$(0,0.1665]  \\ \hline  	
$|W_{3}\rangle$ & 0.644337 & $f_{C}(\rho_{AB}^{W_{3}(0)})$=$f_{C}(\rho_{AB}^{W_{3}(1)})$=0.955342 & [0.000427,0.288675), a$\in$(0,0.1525]   \\ \hline		
\end{tabular}
\caption{In this table, we have estimated the lower bound of controller’s power when various three-qubit pure state such as maximally slice state $|\psi^{(2)}\rangle$, and $|W_{n}\rangle,~n=1,2,3$ states. We have found that all the three-qubit states are useful for controlled teleportation and furthermore, we obtain that $|W_{1}\rangle$ is more useful in controlled teleportation in comparison to $|W_{2}\rangle$ and $|W_{3}\rangle$ state. }
\label{t1}
\end{table*}

\section{Controlled teleportation in noisy environment}
\noindent The standard W state is given by
\begin{eqnarray}
|\psi^{(W)}\rangle_{BAC}&=& \frac{1}{\sqrt{3}}(|000\rangle +|101\rangle+ |110\rangle) 
\label{wstate}
\end{eqnarray}
It is known that the standard W state is not useful in controlled quantum teleportation \cite{Artur}. Therefore,  in this section, we investigate for the possibility of using the standard W state in CQT protocol when one of the qubit passes through the noisy environment. To execute our protocol, we assume that a source generate three-qubit entangled state $\rho^{(W)}_{BAC}=|\psi^{(W)}\rangle_{BAC}\langle \psi^{(W)}|$, where $|\psi^{(W)}\rangle_{BAC}$ is given in the form (\ref{wstate}). In this protocol, let us further assume that the two party Alice and Charlie are in one place while Bob is another distant partner. Alice possesses the two qubits $A$ and $B$ respectively. On the other hand, Charlie have the qubit $C$. Thus, initially, the qubit $B$ is also with Alice. Since Alice would like to send some information to Bob via a shared quantum state so she need to construct an entangled channel between them. Thus, Alice has to send a qubit (suppose, a qubit B) involved in the three-qubit entangled state $\rho^{(W)}_{BAC}$ to Bob through the noisy environment. The noisy environment may be described by either as (i) Amplitude Damping Channel or (ii) Phase Damping Channel. The qubit $B$ then interact with the noisy environment while travelling to Bob's place and assuming that finally it reaches to Bob. In this way, a channel is constructed between Alice and Bob through which Alice can send her information to Bob using quantum teleportation protocol. Since the qubit $B$ has interacted with the noisy environment so there may be a possibility of the degradation of the entanglement of the established channel between Alice and Bob. Thus, the teleportation fidelity may become less than $\frac{2}{3}$. In this scenario, Charlie may play a major role as controller to enhance the teleportation fidelity. Hence, we can calculate the power of the controller in this version of controlled teleportation.    
\subsection{Amplitude Damping Channel}
\noindent  Recalling the standard W state given in (\ref{wstate}) and following the above described protocol where the qubit $B$ is interacting with the noisy environment. Let us consider first the amplitude damping channel as the noisy environment through which the qubit $B$ is passing. Amplitude damping channel is described by the Kraus operators defined as \cite{ban}
\begin{eqnarray}
K_{1}=|0\rangle \langle0|+\sqrt{1-p}|1\rangle \langle1|,  
K_{2}=\sqrt{p}|0\rangle \langle1|,~~0\leq p\leq 1 \nonumber\\
\label{krauss}
\end{eqnarray} 
The Kraus operator satisfies $K_{1}^{\dagger}K_{1}+K_{2}^{\dagger}K_{2}=I$.\\
When the qubit $B$ passes through the amplitude damping channel, the state $\rho_{BAC}^{(W)}$ reduces to 
\begin{eqnarray}
\rho_{BAC}^{(W1)}&=&(K_{1}\otimes I\otimes I)\rho_{BAC}^{(W)}(K_{1}^{\dagger}\otimes I\otimes I)\nonumber \\ &+& (K_{2}\otimes I\otimes I)\rho_{BAC}^{(W)}(K_{2}^{\dagger}\otimes I\otimes I)\nonumber \\
&=& \frac{1}{3}(|000\rangle \langle000|+ p(|001\rangle \langle001|+|001\rangle \langle010|\nonumber \\ &+&|010\rangle \langle001|+|010\rangle \langle010|)+\sqrt{1-p}(|000\rangle \langle101|\nonumber \\&+&|000\rangle \langle110|+|101\rangle \langle000|+|110\rangle \langle000|) \nonumber \\&+&(1-p)(|101\rangle \langle101|+|101\rangle \langle110|+|110\rangle \langle101|\nonumber \\ &+&|110\rangle \langle110|)) 
\end{eqnarray}
Now, our task is to see whether the channel generated between Alice and Bob is useful for conventional teleportation. To verify this, we trace out the system $C$ from the state described by the density operator $\rho_{BAC}^{(W1)}$. The resulting two-qubit state is then given by
\begin{eqnarray}
\rho_{BA}^{(W1)}&=&\frac{1}{3}(|00\rangle \langle00| + p(|00\rangle \langle00|+|01\rangle \langle01|)\nonumber \\ &+&\sqrt{1-p}(|00\rangle \langle11|+|11\rangle \langle00|)\nonumber \\ &+& (1-p)(|10\rangle \langle10|+|11\rangle \langle11|))
\end{eqnarray}
The non-conditioned fidelity of teleportation when $\rho_{BA}^{(W1)}$ is used as a resource state, is given by
\begin{eqnarray}
f_{NC}(\rho_{BA}^{(W1)})&=&\frac{5+2\sqrt{1-p}}{9}\nonumber \\
&\leq& \frac{2}{3}, ~~ \text{for}~~\frac{3}{4}\leq p\leq 1
\end{eqnarray}
Therefore, we find here that there exist a range of the noisy parameter $p$ $(\frac{3}{4}\leq p \leq 1)$ for which $f_{NC}(\rho_{BA}^{(W1)})\leq \frac{2}{3}$. Thus, in this case, the controller (Charlie) has to use his power to enhance the fidelty of teleportation. To do this task, let us recall again the three-qubit state $\rho_{BAC}^{(W1)}$ ad then charlie performs von-Neumann measurement $\{B_{k},k=0,1\}$ on his qubit $C$. According to the measurement result, the resulting two-qubit states are given by  
\begin{eqnarray}
\rho_{BA}^{W1(0)}&=&\frac{1}{3N_{0}}\bigg( (t^{2}+y_{3}^{2})(|00\rangle \langle00|+p|01\rangle \langle01|\nonumber\\ &+&\sqrt{1-p}(|00\rangle \langle11|+|11\rangle \langle00|)+(1-p)|11\rangle \langle11|)\nonumber \\ &+& (-ty_{2}+y_{1}y_{3}+i(ty_{1}+y_{2}y_{3}))(p|01\rangle \langle00|\nonumber \\&+& \sqrt{1-p}|00\rangle \langle10|+ (1-p)|11\rangle \langle10| )\nonumber \\ &+& (-ty_{2}+y_{1}y_{3} -i(ty_{1}+y_{2}y_{3}))(p|00\rangle \langle01|\nonumber \\ &+& \sqrt{1-p}|10\rangle \langle00|+ (1-p)|10\rangle \langle11|)\nonumber \\ &+& (y_{1}^{2}+y_{2}^{2}) (p|00\rangle \langle00|+(1-p)|10\rangle \langle10|)\bigg) 
\end{eqnarray}
\begin{eqnarray}
	\rho_{BA}^{W1(1)}&=&\frac{1}{3N_{1}}\bigg( (y_{1}^{2}+y_{2}^{2})(|00\rangle \langle00|+p|01\rangle \langle01|\nonumber\\ &+&\sqrt{1-p}(|00\rangle \langle11|+|11\rangle \langle00|)+(1-p)|11\rangle \langle11|)\nonumber \\ &+& (ty_{2}-y_{1}y_{3}-i(ty_{1}+y_{2}y_{3}))(p|01\rangle \langle00|\nonumber \\&+& \sqrt{1-p}|00\rangle \langle10|+ (1-p)|11\rangle \langle10| )\nonumber \\ &+& (ty_{2}-y_{1}y_{3}+i(ty_{1}+y_{2}y_{3}))(p|00\rangle \langle01|\nonumber \\ &+& \sqrt{1-p}|10\rangle \langle00|+ (1-p)|10\rangle \langle11|)\nonumber \\ &+& (t^{2}+y_{3}^{2}) (p|00\rangle \langle00|+(1-p)|10\rangle \langle10|)\bigg) 
\end{eqnarray}
where $N_{0}=\frac{2(t^{2}+y_{3}^{2})+(y_{1}^{2}+y_{2}^{2})}{3}$and $N_{1}=\frac{(t^{2}+y_{3}^{2})+2(y_{1}^{2}+y_{2}^{2})}{3}$.\\
In the first case, we consider the two-qubit state $\rho_{BA}^{W1(0)}$ shared between Alice and Bob. We now choose the measurement parameter $(t,y_{1},y_{2},y_{3})$ in such a way that the fidelity of teleportation would be greater than $\frac{2}{3}$. Therefore, the measurement parameters may be chosen as
\begin{eqnarray}
&&t=0.9615239544277027\nonumber\\&& y_{1}=-0.00000006450287021375004\nonumber\\&& y_{2}=-0.000000029154369318260298\nonumber\\&& y_{3}=0.2747211110648374 
\end{eqnarray}
The conditioned fidelity is then given by
\begin{eqnarray}
f_{C}(\rho_{BA}^{W1(0)})&=& 0.666667 + 0.333333\sqrt{1-p} - 0.166667p\nonumber\\&& 0.75\leq p \leq 0.82842
\label{cfam} 
\end{eqnarray}
We may observe that the conditioned fidelity $f_{C}(\rho_{BA}^{W1(0)})$ is greater than $\frac{2}{3}$ when $0.75\leq p \leq 0.82842$. In all other range of the parameter $p$, either $f_{NC}> \frac{2}{3}$ or $f_{C}\leq \frac{2}{3}$. Thus, we will consider $0.75\leq p \leq 0.82842$ where all conditions of controlled teleportation are met. In a similar way, the condition for the controlled teleportation can be studied by considering the second case when the measurement on the charlie's qubit generate a two-qubit state described by the density operator $\rho_{AB}^{W1(1)}$. In any case, we find that both the states $\rho_{AB}^{W1(0)}$ and $\rho_{AB}^{W1(1)}$ are useful in the controlled quantum teleportation scheme. 
\subsection{Phase Damping Channel}
\noindent The phase damping channel is described by the 
Kraus Operator, which may be defined as \cite{adhikari}
\begin{eqnarray}
K_{1}&=&\sqrt{1-p}(|0\rangle \langle0|+|1\rangle \langle1|),  K_{2}=\sqrt{p}|0\rangle \langle0|,\nonumber \\
K_{3}&=&\sqrt{p}|1\rangle \langle1|,~~0\leq p\leq 1 
\end{eqnarray} 
Let us recall the standard W state described by the density operator $\rho_{BAC}^{(W)}=|\psi^{(W)}\rangle_{BAC}\langle\psi^{(W)}|$ where $|\psi^{(W)}\rangle_{BAC}$ is given in (\ref{wstate}) and follow the same protocol, as we did for amplitude damping channel. When a qubit $B$ interacted with the phase damping channel, the state $\rho_{BAC}^{(W)}$ reduces to   
\begin{eqnarray}
\rho_{BAC}^{(W2)}&=&(K_{1}\otimes I\otimes I)\rho_{BAC}^{(W)}(K_{1}^{\dagger}\otimes I\otimes I)\nonumber \\ &+& (K_{2}\otimes I\otimes I)\rho_{BAC}^{(W)}(K_{2}^{\dagger}\otimes I\otimes I)\nonumber \\  &+& (K_{3}\otimes I\otimes I)\rho_{BAC}^{(W)}(K_{3}^{\dagger}\otimes I\otimes I)\nonumber \\
&=& \frac{1}{3}\big(|000\rangle \langle000|+ |101\rangle \langle101|+|110\rangle \langle101|\nonumber \\ &+&|101\rangle \langle110|+|110\rangle \langle110|+|101\rangle \langle000|\nonumber \\ &+&|110\rangle \langle000|+|000\rangle \langle101|+|000\rangle \langle110|\nonumber \\&-& p(|101\rangle \langle000|+|110\rangle \langle000|+ |000\rangle \langle101|\nonumber \\&+&|000\rangle \langle110|)\big) 
\end{eqnarray}
To verify whether the controlled teleportation scheme is applicable for the state $\rho_{BAC}^{(W2)}$, we need to calculate non-conditioned fidelity and conditioned fidelity.\\ 
(i) Non-conditioned fidelity: The non-conditioned fidelity can be calculated as
\begin{eqnarray}
f_{NC}(\rho_{BA}^{(W2)})&=&\frac{7-2p}{9}\nonumber \\
&\leq& \frac{2}{3}, ~~ \text{for}~~\frac{1}{2}< p\leq 1
\label{ncfw2}
\end{eqnarray} 
where $\rho_{BA}^{(W2)}=Tr_{C}(\rho_{BAC}^{(W2)})$.\\
(ii) Conditioned fidelity: To calculate it, Charlie apply the measurement on his qubit in the basis $\{B_{0},B_{1}\}$. According to the measurement result, the resulting two-qubit states are given by 
\begin{eqnarray}
\rho_{BA}^{W2(0)}&=&\frac{1}{3N_{2}}\bigg( (t^{2}+y_{3}^{2})(|00\rangle \langle00|+|11\rangle \langle11|\nonumber \\ &+&|11\rangle \langle00|+|00\rangle \langle11|-p(|11\rangle \langle00|+|00\rangle \langle11|))\nonumber \\ &+&+ (-ty_{2}+y_{1}y_{3}+\iota(ty_{1}+y_{2}y_{3}))(|11\rangle \langle10|\nonumber \\ &+&|00\rangle \langle10|-p|00\rangle \langle10|)+ (-ty_{2}+y_{1}y_{3} \nonumber \\ &-&\iota(ty_{1}+y_{2}y_{3}))(|10\rangle \langle11|+|10\rangle \langle00|\nonumber \\ &-&p|10\rangle \langle00|)+ (y_{1}^{2}+y_{2}^{2}) |10\rangle \langle10|\bigg) 
\end{eqnarray}
\begin{eqnarray}
	\rho_{BA}^{W2(1)}&=&\frac{1}{3N_{3}}\bigg( (y_{1}^{2}+y_{2}^{2})(|00\rangle \langle00|+|11\rangle \langle11|\nonumber \\ &+&|11\rangle \langle00|+|00\rangle \langle11|-p(|11\rangle \langle00|+|00\rangle \langle11|))\nonumber \\ &+& (ty_{2}-y_{1}y_{3}-i(ty_{1}+y_{2}y_{3}))(|11\rangle \langle10|\nonumber \\ &+&|00\rangle \langle10|-p|00\rangle \langle10|)+ (ty_{2}-y_{1}y_{3} \nonumber \\ &+&i(ty_{1}+y_{2}y_{3}))(|10\rangle \langle11|+|10\rangle \langle00|\nonumber \\ &-&p|10\rangle \langle00|)+ (t^{2}+y_{3}^{2}) |10\rangle \langle10|\bigg) 
\end{eqnarray}
where $N_{2}=\frac{2(t^{2}+y_{3}^{2})+(y_{1}^{2}+y_{2}^{2})}{3}$and $N_{3}=\frac{(t^{2}+y_{3}^{2})+2(y_{1}^{2}+y_{2}^{2})}{3}$.\\
If the measurement parameters are given by
\begin{eqnarray}
&& t = 0.9615239543413954 \nonumber\\&&
y1 = 0.000002698965323056848 \nonumber\\&&
y2 = -0.000000004258892841348826 \nonumber\\&&
y3 = 0.2747211523762679
\end{eqnarray}
then the conditioned fidelity $f_{C}(\rho_{BA}^{W2(0)})$ is given by
\begin{eqnarray}
f_{C}(\rho_{BA}^{W2(0)})=1- 0.333333 p,~~\frac{1}{2}\leq p \leq 1
\end{eqnarray}
It may be easily verified that $f_{C}(\rho_{BA}^{W2(0)}) \in [0.66667,0.833333]$ for $p \in [0.5,1]$.
Thus, the controlled teleportation protocol may be implemented using the state $\rho_{BA}^{W2(0)}$. In the similar fashion, it may be shown that the state $\rho_{BA}^{W2(1)}$ is useful in controlled teleportation. 

\subsection{Comparision analysis of the power of the controlled teleportation}
\noindent In this section, we compare the power of the controlled teleportation when the standard W state given by (\ref{wstate}) is evolved under amplitude damping channel and phase damping channel. We will show here that the power of the conrolled teleportation in case of phase damping channel is greater than the power in case of amplitude damping channel.\\
\textbf{(a) Power of the controlled teleportation when standard W state is evolved under amplitude damping channel:} Since we find that both the state $\rho_{AB}^{W1(0)}$ and $\rho_{AB}^{W1(1)}$ are useful in the controlled teleportation scheme so we can consider any one of the state $\rho_{AB}^{W1(0)}$ or $\rho_{AB}^{W1(1)}$ to calculate the power of the controller. Let us consider the two-qubit state $\rho_{AB}^{W1(0)}$ for the estimation of the power of the controller. To estimate it, We need to calculate the following:\\
(i) The quantity $M(\rho_{AB}^{W1(0)})$ which is found out to be less than one.\\
(ii) The expectation value of the constructed witness operator $W_{ij}^{(\phi+)}$ with respect to the state $\rho_{BA}^{W1(0)}$, which is given by
\begin{eqnarray}
	Tr[W_{xy}^{(\phi^{+})}(\rho_{BA}^{W1(0)})]&=& \frac{p}{4}
	-\frac{\sqrt{1-p}}{2}+2a,\nonumber\\&&
	0.75\leq p \leq 0.82842
	\label{witam}
\end{eqnarray}
The value of $a>0$ is chosen in such a way that the witness operator $W_{xy}^{(\phi^{+})}$ detect the state $\rho_{BA}^{W1(0)}$. We find that the witness operator $W_{xy}^{(\phi^{+})}$ detects the state $\rho_{BA}^{W1(0)}$ when $a\in (0,0.005]$. \\
With all the above information, we can estimate the power ($P_{CT}^{W1(0)}$), which is given by
\begin{eqnarray}
\frac{4a}{3} (1-\sqrt{M(\rho_{BA}^{W1(0)})})-\frac{2}{3}Tr[W_{ij}^{(\phi^{+})}\rho_{BA}^{W1(0)}] \nonumber \\ \leq P_{CT}^{W1(0)} \leq \frac{1}{2}
\end{eqnarray}
We calculate the lower limit of the power $P_{CT}^{W1(0)}$ and found that the lower limit varies in the interval $[0.0056075,0.041667)$ when $a\in (0,0.005]$ \& $0.75\leq p \leq 0.8164$. \\
\textbf{(b) Power of the controlled teleportation when standard W state is evolved under phase damping channel:} In this scenario also, we find that both the state $\rho_{AB}^{W2(0)}$ and $\rho_{AB}^{W2(1)}$ are useful in the controlled teleportation scheme so we can consider any one of the state $\rho_{AB}^{W2(0)}$ or $\rho_{AB}^{W2(1)}$ to calculate the power of the controller. Let us consider the two-qubit state $\rho_{AB}^{W2(0)}$ for the estimation of the power of the controller.  The power of the controller can be estimated by 
\begin{eqnarray}
	\frac{4a}{3} (1-\sqrt{M(\rho_{BA}^{W2(0)})})-\frac{2}{3}Tr[W_{ij}^{(\phi^{+})}\rho_{BA}^{W2(0)}] \nonumber \\ \leq P_{CT}^{W2(0)} \leq \frac{1}{2} 
\end{eqnarray}
where the expectation value of the witness operatr $W_{xy}^{(\phi^{+})}$ with respect to the state $\rho_{BA}^{W2(0)}$ is given by
\begin{eqnarray}
	Tr[W_{xy}^{(\phi^{+})}\rho_{BA}^{W2(0)}]= \frac{1}{2}+2a-(1 - 0.5p) <0 ~ \text{for} \nonumber\\ a\in (0,0.035], p\in [0.5,0.859] \nonumber \\
\end{eqnarray}
Also, the quantity M($\rho_{BA}^{W2(0)}$) can be easily calculated and found out to be greater than 1. Therefore, the lower bound of power $P_{W2}^{(0)}$ lying in the interval $[0,0.16667)$ for $a\in (0,0.005]$ \& $0.5\leq p \leq 0.859$.
% Figure environment removed 
It can be clearly seen from Fig.-{\ref{f2}} that controller's power is more for standard $W$ state when it is under phase damping channel. Though standard $W$ state with both amplitude damping channel and phase damping channel are useful in controlled quantum channel but standard $W$ state with phase damping channel is more useful in controlled quantum teleportation. 
%% Figure environment removed 
%Clearly from the figure-(\ref{f3}), we can see that the power obtained by our work is more than the power obtained by Artur's work \cite{Artur} and at a=0.1325825 our power coincides with there power.
\section{Conclusion}
To summarize, we have considered the problem of  estimation of the power of the controller in CQT scheme. To investigate it, we have constructed a witness operator and have shown that the entangled state will be useful for teleportation as a resource state if the same entangled state is detected by the constructed witness operator and if it satisfies the Bell-CHSH inequality. Thus, at least for some cases, we need not have to use the filtering operation \cite{versatrate} to increase the teleportation fidelity. On the other hand, the study of the violation of Bell-CHSH inequality is equally important in the CQT scheme and thus we have considered the CHSH game for the estimation of the probability of success of the game through the constructed witness operator. The estimated probability of success helps in the derivation of the lower bound of the conditioned and non-conditioned fidelity in terms of the expectation value of the witness operator. Therefore, we are now able to estimate the lower and upper bound of the power of the controller in terms of the witness operator. Thus, this can pave a way to estimate the power of the controlled teleprtation in an experiment. Moreover, we have found that the state $W_{1}$ is not only useful for conventional teleportation between two parties but also useful in the CQT scheme and performs better than all the other W-class of states described by $W_{n},~n=2,3,...$ \cite{patiagrawal}. We have also studied the CQT scheme using the standard W state under a noisy environment. When one of the qubits of the standard W state passes either through the amplitude damping channel or the phase damping channel, the resulting state will be a mixed state which will be useful in controlled quantum teleportation protocol. We also found that the phase damping channel makes the controller power more positive than the amplitude damping channel. Thus, we may conclude that phase damping channel is more useful than amplitude damping channel while performing CQT protocol with the standard W state.      
\section{Acknowledgement}
\noindent A. G. would like to acknowledge the financial support from CSIR. This work is supported by CSIR File No. 08/133(0035)/2019-EMR-1.

\section{DATA AVAILABILITY STATEMENT}
Data sharing not applicable to this article as no datasets were generated or analysed during the current study.				
				
\begin{thebibliography}{90}
\bibitem{nielsen} M. A. Nielsen, and I. L. Chuang, Quantum computation and quantum information (Cambridge University Press, Cambridge, 2000).
%\bibitem{horn} R. A. Horn, and C. R. Johnson, Matrix Analysis (Cambridge University Press, 1999); S. Adhikari, and S. Banerjee, Phys. Rev. A \textbf{86}, 062313 (2012).
\bibitem{wilde} M. M. Wilde, Quantum information theory (Cambridge University Press, Cambridge, 2013).
\bibitem{bennett2} C. H. Bennett, G. Brassard, C. Crepeau, R. Jozsa, A. Peres, and
W. K. Wootters, Phys. Rev.
Lett. \textbf{70}, 1895 (1993).
\bibitem{gisin1} N. Gisin, G. Ribordy, W. Tittel, and H. Zbinden, Rev. Mod. Phys. \textbf{74}, 145 (2002).
\bibitem{briegel} H.-J. Briegel, W. Dur, J. I. Cirac, and P. Zoller, Phys. Rev. Lett. \textbf{81}, 5932 (1998).
\bibitem{gottesman} D. Gottesman, and I. L. Chuang, Nature \textbf{402}, 390 (1999).
\bibitem{bouwmeister}D. Bouwmeister, J. W. Pan, K. Mattle, M. Eible, H. Weinfurther,
and A. Zeilinger, Nature \textbf{390}, 575 (1997).
\bibitem{boschi}D. Boschi, S. Branca, F. De Martini, L. Hardy, and S. Popescu,
Phys. Rev. Lett. \textbf{80}, 1121 (1998).
\bibitem{kwiat}P. G. Kwiat, K. Mattle, H. Weinfurther, and A. Zeilinger,
Phys. Rev. Lett. \textbf{75}, 4337 (1995).
\bibitem{michler} M. Michler, K. Mattle, M. Eible, H. Weinfurther, and A.
Zeilinger, Phys. Rev. A \textbf{53}, R1209 (1996).
\bibitem{karlsson} A. Karlsson, and M. Bourennane, Phys. Rev. A \textbf{58}, 4394 (1998).
\bibitem{gao}T. Gao, F. L. Yan, and Y. C. Li, Euro. Phys. Lett. \textbf{84}, 50001 (2008).
\bibitem{li2014}X. Li, and S. Ghose, Phys. Rev. A \textbf{90}, 052305 (2014).
\bibitem{li2015} X. Li, and S. Ghose, Phys. Rev. A \textbf{91}, 012320 (2015).
\bibitem{jeong} K. Jeong, J. Kim, and S. Lee, Phys. Rev. A \textbf{93}, 032328 (2016).
\bibitem{Artur} A. Barasinski, and J. Svozilik, Phys. Rev. A \textbf{99}, 012306 (2019).
\bibitem{paulson} K. G. Paulson, and P. K. Panigrahi,  Phys. Rev. A \textbf{100}, 052325 (2019).
%\bibitem{coffman} V. Coffman, J. Kundu, and W. K. Wootters, Phys. Rev. A \textbf{61},052306 (2000).
\bibitem{wang} T.-J. Wang, G.-Q. Yang, and C. Wang, Phys. Rev. A \textbf{101}, 012323 (2020).
\bibitem{kumar} A. Kumar, S. Haddadi, M. R. Pourkarimi,
B. K. Behera, and P. K. Panigrahi, Sci. Rep. \textbf{10}, 13608 (2020).
\bibitem{rau}R. Raussendorf, and H. J. Briegel, Phys. Rev. Lett. \textbf{86}, 5188 (2001).
\bibitem{hamdoun} H. Hamdoun, and A. Sagheer, Dig. Commun. Net. \textbf{6}, 463 (2020).
\bibitem{jun}  Z. Zhan-Jun, L. Yi-Min, and M. Zhong-Xiao, Commun. Theor. Phys. \textbf{44}, 847 (2005).
\bibitem{sango}N. Sangouard, C. Simon, H.  Riedmatten, and N. Gisin,
Rev. Mod. Phys. \textbf{83}, 33 (2011).
\bibitem{sayan} S. Gangopadhyay, T. Wang, A. Mashatan, and S. Ghose,
Phys. Rev. A \textbf{106}, 052433 (2022).
\bibitem{luo} S. Luo, Phys. Rev. A \textbf{77}, 042303 (2008).  
\bibitem{Guhne}O. Guhne, and G. Toth, Phys. Rep. \textbf{474}, 1 (2009).
\bibitem{horo3} R. Horodecki, P. Horodecki, and M. Horodecki, Phys. Lett. A \textbf{200}, 340 (1995).
\bibitem{hyllus} P. Hyllus, O. Guhne, D. Brub, and M. Lewenstein, Phys. Rev. A \textbf{72}, 012321 (2005).
\bibitem{Bennett}C. H. Bennett, D. P. Di Vincenzo, J. Smolin, and W. K. Wootters,
Phys. Rev. A \textbf{54}, 3824 (1997).
\bibitem{Mhorodecki}M. Horodecki, and P. Horodecki, Phys. Rev. A \textbf{59}, 4206 (1999).
\bibitem{horo4} R. Horodecki, M. Horodecki, and P. Horodecki, Phys. Lett. A \textbf{222}, 21 (1996). 

\bibitem{versatrate}F. Verstraete, and H. Verschelde, Phys. Rev. Lett. \textbf{90}, 097901 (2003).

%\bibitem{lasserre} J. B. Lasserre, IEEE Trans. on Automatic Control \textbf{40}, 1500 (1995); A. Kumari, and S. Adhikari, Phys. Rev. A \textbf{100}, 052323 (2019).
%\bibitem{karlson} A. Karlsson, and M. Bourenname, Phys. Rev. A \textbf{58}, 6(1998).
%\bibitem{Shoini} X. H. Li, and S.Ghose, Phys. Rev. A \textbf{90}, 052305(2014).
%\bibitem{eisert}J. Eisert, M. Wilkens, and M. Lewenstein, Phys. Lett. A \textbf{83}, 15(1999).
%\bibitem{archan} A. S. Majumdar, and T. Pramanik, Int. Jour. of Quant. Info. \textbf{14}, 06(2016).
\bibitem{jon} J. Oppenheim, and S. Wehner, Science \textbf{330}, 1072 (2010).
\bibitem{lee} S. Lee, J. Joo, and J. Kim, Phys. Rev. A \textbf{72}, 024302 (2005). 
\bibitem{patiagrawal}P. Agrawal, and A. Pati, Phys. Rev. A \textbf{74}, 062320 (2006).
\bibitem{ban} S. Bandyopadhyay, Phys. Rev. A \textbf{65}, 022302 (2002).
\bibitem{adhikari} S. Adhikari, I. Chakrabarty, and P. Agrawal, Quan. Inf. Comp. \textbf{12}, 0253 (2012).
%\bibitem{Badzaig} P. Badziag, M. Horodecki, P. Horodecki, and R. Horodecki, Phys. Rev. A \textbf{62}, 012311 (2000).

%\bibitem{Adhikari}S. Adhikari, N. Ganguly, and A. S. Majumdar, Phys. Rev. A \textbf{86}, 032315 (2012).



%\bibitem{voll}K. G. H. Vollbrecht, and F. Verstraete, Phys. Rev. A \textbf{71}, 062325 (2005).	
%\bibitem{ishizaka}S. Ishizaka, Phys. Rev. A \textbf{69}, 020301(R) (2004).	


\bibitem{horo5} M. Horodecki, P. Horodecki, and R. Horodecki, Phys. Rev. A \textbf{60}, 1888 (1999). 
\end{thebibliography}

\section{Appendix}


\subsection{Proof of Result-1}
\noindent To derive the required lower bound of the expectation value of the witness operator $W^{(\phi^{+})}_{ij}$, let us recall the witness operator defined in (\ref{witdef}). The expectation value of $W^{(\phi^{+})}_{ij}$ with respect to an entangled state $\rho_{ent}$, is given by
\begin{eqnarray}
	Tr[W^{(\phi^{+})}_{ij}\rho_{ent}]&=&(\frac{1}{2}+2a)-\langle \phi^{+}|\rho_{ent}|\phi^{+}\rangle-a\times\nonumber\\&&
	Tr[B_{ij}\rho_{ent}]\nonumber\\&\geq& (\frac{1}{2}+2a)-F(\rho_{ent})-aTr[B_{ij}\rho_{ent}]\nonumber\\&\geq&
	(\frac{1}{2}-F(\rho_{ent}))+2a(1-\sqrt{M(\rho_{ent})})\nonumber\\
	\label{lb}
\end{eqnarray}
In the second step, we have used $\langle \phi^{+}|\rho_{ent}|\phi^{+}\rangle \leq F(\rho_{ent})$. In the third step, we use the following  $Tr[B_{ij}\rho_{ent}]=\langle B_{ij}\rangle_{\rho_{ent}}\leq max_{B_{ij}}\langle B_{ij}\rangle_{\rho_{ent}}=2\sqrt{M{(\rho_{ent}})}$ for any $(i,j)$, where $i,j=x,y,z;i\neq j$ \cite{horo3}.\\ 
Let us now derive the upper bound of the expectation value of the witness operator $W^{(\phi^{+})}_{ij}$. Again, the expectation value of $Tr[W_{ij}^{(\phi^{+})}(\rho_{ent})]$ can be expressed as
\begin{eqnarray}
Tr[W_{ij}^{(\phi^{+})}\rho_{ent}] &=& \frac{1}{2}- \langle \phi^{+}|\rho_{ent}| \phi^{+}\rangle +a \times \nonumber \\&& (2-Tr[B_{ij}\rho_{ent}]) 
\label{w1}
\end{eqnarray}
Let us assume that the two qubit entangled state $\rho_{ent}$ satisfies the Bell-CHSH inequality, i.e., $Tr[B_{ij}\rho_{ent}] \in [-2,2]$ for any $i,j=x,y,z;i\neq j$. Let us split the interval $[-2,2]$ into two subintervals $[-2,0)$ and $[0,2]$. Therefore, we have the following two cases:\\
(i) If $Tr[B_{ij}\rho_{ent}] \in [0,2]$ then we get  
\begin{eqnarray}
	Tr[W_{ij}^{(\phi^{+})}\rho_{ent}] \leq \frac{1}{2}- \langle \phi^{+}|\rho_{ent}| \phi^{+}\rangle +2a  
	\label{w2}
\end{eqnarray} 
(ii) If $Tr[B_{ij}\rho_{ent}] \in [-2,0]$, we get 
\begin{eqnarray}
	Tr[W_{ij}^{(\phi^{+})}\rho_{ent}] \leq \frac{1}{2}- \langle \phi^{+}|\rho_{ent}| \phi^{+}\rangle +4a
	\label{w3}
\end{eqnarray}
Thus, combining (\ref{w2}) and (\ref{w3}) and since $a>0$, we get 
\begin{eqnarray}
	Tr[W_{ij}^{(\phi^{+})}\rho_{ent}] \leq \frac{1}{2}- \langle \phi^{+}|\rho_{ent}| \phi^{+}\rangle +4a
	\label{ub}
\end{eqnarray}
Hence, if $M(\rho_{ent})\leq 1$ then the lower and upper bound of the expectation value of the witness operator $W^{(\phi^{+})}_{ij}$ is given by
\begin{eqnarray}
&&(\frac{1}{2}-F(\rho_{ent}))+2a(1-\sqrt{M(\rho_{ent})}) \leq	Tr[W^{(\phi^{+})}_{ij}\rho_{ent}]\nonumber\\ && \leq \frac{1}{2}- \langle \phi^{+}|\rho_{ent}|\phi^{+}\rangle +4a,~~a>0
\end{eqnarray}

\subsection{Proof of Result-3}
\noindent Let us start with the lower bound of the expectation value of the witness operator $W_{ij}^{(\phi^{+})}$ $(i,j= x,y,z; i\neq j)$. Therefore, the inequality (\ref{lb}) can be re-expressed as 
\begin{eqnarray}
F(\rho_{ent}) \geq \frac{1}{2}- Tr[W_{ij}^{(\phi^{+})}\rho_{ent}] +2a(1-\sqrt{M(\rho_{ent})})\nonumber\\ 
\label{lb1}
\end{eqnarray}
The relation between the teleportation fidelity $f(\rho_{ent})$ and singlet fraction $ F(\rho_{ent})$ of an entangled state $\rho_{ent}$ is given by \cite{horo5}
\begin{eqnarray}
f(\rho_{ent})=\frac{2F(\rho_{ent})+1}{3}
\label{rel}
\end{eqnarray} 
Using (\ref{lb1}) and (\ref{rel}), we get 
\begin{eqnarray}
f(\rho_{ent}) \geq \frac{2}{3}[1- Tr[W_{ij}^{(\phi^{+})}\rho_{ent}] +2a(1-\sqrt{M(\rho_{ent})})] \nonumber\\
\label{reltel}
\end{eqnarray}
Using the fact that $M(\rho_{ent})\leq 1$ and the witness operator $W^{(\phi^{+})}_{ij}$ detect the entangled state $\rho_{ent}$, it can be easily verified that $f(\rho_{ent})>\frac{2}{3}$. Further, imposing the condition that $f(\rho_{ent}) \leq 1$, we can obtain  the upper bound of the parameter $a$, which is given by   
\begin{eqnarray}
a \leq \frac{\frac{1}{2}+Tr[W_{ij}^{(\phi^{+})}(\rho_{ent})]}{2(1-\sqrt{M(\rho_{ent})})}]
\end{eqnarray}
Therefore, the interval of the parameter $a$ for which the entangled state $\rho_{ent}$ satisfies the inequality $M(\rho_{ent})\leq 1$ and useful for teleportation is given by
\begin{eqnarray}
a \in(0,\frac{\frac{1}{2}+Tr[W_{ij}^{(\phi^{+})}(\rho_{ent})]}{2(1-\sqrt{M(\rho_{ent})})}]
\end{eqnarray}




\end{document} 






%Since $Tr[W\rho_{AB}]\leq 1.5606$ so the expected value of $ W $ with respect to the state $\rho_{AB}$ may take negative value also. Thus, it is the indication of the existence of the entangled state that may satisfy the Bell inequality.  If this happens, we can say that we can measure the strength of the non-locality of those entangled states that satisfy the Bell inequality. \\
%\textbf{Result-2}:- If the state $\rho$ satisfies the inequality $0\leq \langle B_{CHSH}\rangle_{\rho}\leq 2$, then $Tr[W\rho] \geq 0.78033.$\\
%\textbf{Proof}:- 
%The maximum probability of winning the game of a state $\rho$ shared between Alice and Bob w.r.t $W$ operator is given as \cite{jon} 
%\begin{equation}
%	Tr[W\rho]=\frac{1}{4}[1+\frac{1}{2\sqrt{2}}\sum_{\substack{i,j \\ i\neq j}} \langle B_{ij}\rangle_{\rho}]
%	\label{4}
%\end{equation}
%Since, we have assumed that $0\leq \langle B_{CHSH}\rangle_{\rho}\leq 2$ \\
%$\Rightarrow$  $ \langle B_{xy}\rangle_{\rho},\langle B_{xz}\rangle_{\rho}$ and  $\langle B_{yz}\rangle_{\rho} \geq 0$
%\\Thus, we are in a position to apply A.M $\geq$ G.M. condition on  $ \langle B_{xy}\rangle_{\rho},\langle B_{xz}\rangle_{\rho}$ and  $\langle B_{yz}\rangle_{\rho}$. \\ 
%Therefore, 
%\begin{eqnarray}
%	\frac{\langle B_{xy}\rangle_{\rho}+ \langle B_{xz}\rangle_{\rho}+  \langle B_{yz}\rangle_{\rho}}{3}\nonumber \\
%	\geq (\langle B_{xy}\rangle_{\rho}\langle B_{xz}\rangle_{\rho} \langle B_{yz}\rangle_{\rho})^{\frac{1}{3}}
%	\label{5}
%\end{eqnarray}
%From (\ref{4}) and (\ref{5}), we have 
%\begin{eqnarray}
%	Tr[W\rho]\geq \frac{1}{4}[1+\frac{1}{2\sqrt{2}}.3(\langle B_{xy}\rangle_{\rho}\langle B_{xz}\rangle_{\rho} \langle B_{yz}\rangle_{\rho})^{\frac{1}{3}}]
%	\label{6}
%\end{eqnarray}
%From the hypothesis, we have $\langle B_{i,j}\rangle\geq 2$ $\forall$ i,j \& $i\neq j$.\\ 
%So by maximizing the R.H.S. of (\ref{6}) which is maximized when $\langle B_{ij}\rangle_{\rho}$=2 
%$\forall$ i,j \& $i\neq j$, we get 

%\begin{eqnarray}
%		Tr[W\rho]\geq \frac{1}{4}[1+\frac{3.(8)^{\frac{1}{3}}}{2\sqrt{2}}] \label{7}
%\end{eqnarray}

%After simplifying (\ref{7}), we get 
%\begin{eqnarray}
%	Tr[W\rho]\geq \frac{1}{4}[1+\frac{3}{\sqrt{2}}] =0.78033
%\end{eqnarray}
%which is the desired condition.\\ \\
%\textbf{Result-3}:- When any two qubit arbitrary state $\rho$ satisfy\\ $-2 \leq \langle B_{CHSH}\rangle_{\rho}\leq 0$, then  $-0.2803 \leq Tr[W\rho]\leq 0.25.$\\ 
%\textbf{Proof}:- When any arbitatry two qubit state satisfies $-2 \leq \langle B_{CHSH}\rangle_{\rho}\leq 0$, we have $\langle B_{ij}\rangle_{\rho}=-k_{ij}$, $\forall$ i,j \& $i\neq j$, $k_{ij}\geq 0$\\
%Since, all $k_{i,j}$ ($i\neq j$) are non-negative. so, let we can apply A.M$ \geq $ G.M. condition on $ k_{ij}. $ 
%\begin{eqnarray}
%	k_{xy}+k_{xz}+k_{yz}\geq 3(k_{xy}.k_{xz}.k_{yz})^{\frac{1}{3}}
%	\label{18}
%\end{eqnarray}
%Recalling $Tr[W\rho]$ from (\ref{4}) and applying (\ref{18}) to it, we get 
%\begin{eqnarray}
%	Tr[W\rho]&=&\frac{1}{4}[1-\frac{1}{2\sqrt{2}}(k_{xy}+k_{yz}+k_{zx})]\nonumber \\ 
%	&& \leq \frac{1}{4}[1-\frac{1}{2\sqrt{2}}*3(k_{xy}+k_{yz}+k_{zx})^{\frac{1}{3}}]
%	\label{8}
%\end{eqnarray}
%To minimize the R.H.S. of (\ref{8}), we can take $k_{xy}=k_{xz}=k_{yz}=0$. Therefore, 
%\begin{equation}
%	Tr[W\rho]\leq\frac{1}{4}
%	\label{9}
%\end{equation}
%Also from \cite{hyllus}, Hyllus et.al. have shown  the lower bound of $W$ operator for any arbitrary state $\rho$  which is given by 
%\begin{equation}
%	Tr[W\rho]\geq-0.2803
%	\label{10}
%\end{equation}
%After combining (\ref{9}) and (\ref{10}), we get \\ 
%$$-0.2803\leq Tr[W\rho] \leq 0.25.$$
%Hence, we get the desired result.\\ \\
%Now, we will illustrate the above results by an entangled state which satisfies the Bell's inequality for every setting of pauli matrices.\\ \\
%Further, one can easily calculate the singlet fraction $(F)$ of the state $\sigma$ and $F^{*}$ which is fidelity of the state after applying the local filtering operation $(X)$. We have divided that into 3 subcases which are listed below according to the condition on state parameters $(b,c,d)$ and fitering operation $(e)$\\ \\
%\textbf{Case-1}:- $F>\frac{1}{2}$ when \\
% \textbf{A}) $0<d<\frac{1}{2}$
% \begin{itemize}
	%	\item $ -\sqrt{1-d^{2}}< b\leq -\sqrt{1-2d}+d $\\
	%	     \& $ \frac{1-b-d^{2}}{-1+b}<c<b $
	%	\item $ -\sqrt{1-2d}+d < b\leq \sqrt{1-2d}+d $  \\
	%	     \& $b-2d<c<b$  
	%	     \item $\sqrt{1-2d}+d< b< \sqrt{1-d^{2}}$\\
	%	     \& $ \frac{1-b-d^{2}}{-1+b}<c<b $
	%	 \end{itemize}     
%\textbf{B}) $\frac{1}{2}\leq d<1  $,  $ -\sqrt{1-d^{2}}< b< \sqrt{1-d^{2}} $ \& $ \frac{1-b-d^{2}}{-1+b}<c<b $\\ 
%\textbf{C}) $-1<d\leq \frac{1}{2}  $,  $ -\sqrt{1-d^{2}}< b< \sqrt{1-d^{2}} $ \& $ \frac{1-b-d^{2}}{-1+b}<c<b $\\
%\textbf{D}) $-\frac{1}{2}\leq d<0  $, 
 %\begin{itemize}
%	 	\item  $ -\sqrt{1-d^{2}}< b<-d-\sqrt{1+2d} $\\  \& $ \frac{1-b-d^{2}}{-1+b}<c<b$
%	 	\item $ -d-\sqrt{1+2d}< b\leq -d+\sqrt{1+2d} $\\  \& $  b+2d<c<b$
%	 	\item  $ -d-\sqrt{1+2d}<b<\sqrt{1-d^{2}}  $\\  \& $ \frac{1-b-d^{2}}{-1+b}<c<b$
%	 \end{itemize}
%\noindent \textbf{Case-2}:- $F< \frac{1}{2}$
%\begin{itemize}
%	\item $0<d<1$, $-\sqrt{1-d^{2}}<b<\sqrt{1-d^{2}}$ \& $ \frac{1-b-d^{2}}{-1+b}<c<b$
%	\item $-1<d<0$, $-\sqrt{1-d^{2}}<b<\sqrt{1-d^{2}}$ \& $\frac{1-b-d^{2}}{-1+b}<c<b$
%	\item  $-\frac{1}{2}<d<0$, $-d-\sqrt{1+2d}<b<-d+sqrt{1+2d}$ \& $ \frac{1-b-d^{2}}{-1+b}<c<b+2d$
%	\item  $0<d<\frac{1}{2}$, $d-\sqrt{1-2d}<b<d+\sqrt{1-2d}$ \& $ \frac{1-b-d^{2}}{-1+b}<c<b-2d$
%\end{itemize}
%\noindent \textbf{Case-3}:- $F^{\ast}> \frac{1}{2}$\\
%\textbf{A})  -$1<b<1$
%\begin{itemize}
%	\item $0<d<\sqrt{1-b+c-bc}$, $-1<c<2-2\sqrt{2}\sqrt{1-b}-b$ \& $ 0<e<\frac{2d}{b-c}$
%	\item $0<d<\frac{b-c}{2}$, $0<e<\frac{2d}{b-c}$ \& $2-2\sqrt{2}\sqrt{1-b}-b<c<b$
%	\item $2-2\sqrt{2}\sqrt{1-b}-b<c<b$, $0<e<1$ \& $\frac{b-c}{2}<d<\sqrt{1-b+c-bc}$
%\end{itemize} 
%\textbf{B})  -$1<b<1$
%\begin{itemize}
%	\item $-\sqrt{1-b+c-bc}<d<0$, $-1<c<2-2\sqrt{2}\sqrt{1-b}-b$ \& $\frac{2d}{b-c}<e<0$
%	\item $-\sqrt{1-b+c-bc}<d<\frac{c-b}{2}$, $-1<e<0$ \& $2-2\sqrt{2}\sqrt{1-b}-b<c<b$
%	\item $2-2\sqrt{2}\sqrt{1-b}-b<c<b$, $\frac{2d}{b-c}<e<0$ \& $\frac{c-b}{2}<d<0$
%\end{itemize} 
%where $F^{\ast}$ is the fidelity after applying filtering operation on the state. \\

%Out of many cases mentioned above where $F^{\ast}> \frac{1}{2}$,  we have studied two cases, one case where  $(Tr[X\sigma^{\tau}] -Tr[A\sigma])$ satisfes (C1) criteria and in another case where it fullfills (C2) criteria.\\ \\
%\textbf{Case-1}:- This case satisfies \textbf{(C1)} criteria and hence, we show  $P_{succ}^{mod}$ $\geq$ $\frac{3}{4}$. \\ 
%So, $Tr[X\rho^{\tau}] -Tr[A\rho]>$0 ~and~ $P_{succ}^{mod}$ $\geq$ $\frac{3}{4}$ for 
%\begin{itemize}
%	\item $a \in (0.0144423,0.0149646)$
%	\item $b \in (0.954359,0.96)$
%	\item $c \in (0.765,0.771)$
%	\item $d \in (0.0503,0.053)$
%	\item $e \in (0.258063,0.344127)$
%\end{itemize}
%\smallskip
%\textbf{Case-2}:-This case satisfies \textbf{(C2)} criteria and hence $P_{succ}^{mod}$ $\geq$ $\frac{3}{4}$. \\ 
%So, $Tr[X\rho^{\tau}] -Tr[A\rho]<$0 ~and~ $P_{succ}^{mod}$ $\geq$ $\frac{3}{4}$ for 
%\begin{itemize}
%	\item $a \in [-1,-0.318099]$
%	\item $b \in (0.9526,0.978171)$
%	\item $c \in (0.827911,0.879)$
%	\item $d \in (-0.49,-0.46)$
%	\item $e=-0.1$
%\end{itemize}
%So, from the above cases we can say that if we consider any general matrix of the form $\sigma$ we will always be able to satisfy Result-4 criteria and hence we always will be able to increase Probability $>\frac{3}{4}$.
%It can also be verified that $F(\rho_{B})=1-3p$ and \\
%So, to conclude from above two cases we can say that for any general 2-qubit entangled state we can always increase the probability of winning. We can now easily say that we can calculate the strength of non-locality of any entangled state with the help of our new modified probability success. 

%\subsection{Illustrations}
%We have illustrated many different classes of states where they may or may not be detected by the operator $A$.
%\begin{equation}
%	A= (\frac{1}{2}+2a)I-|\phi^{+}\rangle \langle \phi^{+}|-aB_{i,j} ,  i,j=x,y,z \& i\neq j
%\end{equation}
%but by our modified formula of probability $P_{succ}^{mod}$ will always be able to calculate the non-locality of the state. $P_{succ}^{mod}$ will always be greater than $\frac{3}{4}$  i.e. $P_{succ}^{mod}>\frac{3}{4}.$\\
%Here, in Table-1 we have considered few states which are stated below  \cite{luo},\cite{Badzaig},\cite{Mhorodecki},\cite{versatrate} :-\\
%\begin{eqnarray*}
%	\rho_{F}&=& F|\phi^{+}\rangle \langle \phi^{+}|+ (1-F)|01\rangle \langle 01| , \frac{1}{3}<F \leq 1 \nonumber \\
%\end{eqnarray*}
%where $F$ denotes the singlet fraction of the state.
%\begin{eqnarray*}
%	\rho_{MMS}= \frac{1}{4}[I\otimes I + \sum c_{i}\sigma_{i} \otimes \sigma_{i}]  \nonumber \\
%\end{eqnarray*}  
%\begin{eqnarray*}
%	\rho_{p}= p|\phi^{+}\rangle \langle \phi^{+}|+ \frac{1-p}{4}I , \frac{1}{3}<p \leq 1
%\end{eqnarray*}
% \begin{eqnarray}
	%	\rho_{BA}=\frac{1}{2}\begin{pmatrix} 
		%		0 & 0& 0& 0\\
		%		0 & 3-2\sqrt{2} & 1-\sqrt{2} & 0\\
		%		0 & 1-\sqrt{2} & 1 & 0\\
		%		0 & 0& 0& 2\sqrt{2}-2 \\
		%	\end{pmatrix}
	%\end{eqnarray}
	
	%\begin{table*}[!htbp]\centering
	%	\begin{tabular} {|c|c|c|c|c|c|c|}\hline 
		%\multicolumn{7}{c}{text}\hline 
		%		State & $\langle B_{yz} \rangle$ & Tr[A$\rho$] & a &Parameter of state & Filtering operator &  $P_{succ}^{mod}$\\ \hline
		%		$\rho_{F}$ & $ -\sqrt{2}(1-F) $ & Negative & (-0.292,-0.24) & F$\in$ (0.9013,0.92) & N.A. & $>\frac{3}{4}$\\ \hline 
		%			$\rho_{F}$ & $-\sqrt{2}(1-F)$ & Positive & (-0.054,-0.0365) & F$\in$ ($ \frac{1}{3} $,0.342747) & N.A. & $>\frac{3}{4}$\\ \hline 
		%				$ \rho_{BA} $ & -0.828427 & Negative & (-0.125,-0.07) & N.A. & 0.5 & $>\frac{3}{4}$\\ \hline 
		%			$ \rho_{BA} $ & -0.828427 & Positive & (-0.1035,-0.065) & N.A. & (0,0.114) & $>\frac{3}{4}$\\ \hline 
		%			$\rho_{MMS} $& $ \sqrt{2}(c_{2}+c_{3}) $ & Negative & (-0.098,-0.09715) & $ c_{1}=-0.3,c_{2}=-0.5,c_{3} \in (-0.56,-0.449) $ & 0.5 & $>\frac{3}{4}$\\ \hline 
		%				$\rho_{MMS}$&$ \sqrt{2}(c_{2}+c_{3}) $ & Positive & (-.09715,-0.07) & $ c_{1}=-0.3,c_{2}=-0.5,c_{3} \in (-0.604,-0.449) $  & 0.5 & $>\frac{3}{4}$\\ \hline 
		%			$\rho_{p}$  & 0 & Negative & (0.0299,0.052) & (0.57733,0.62) & 0.5 & $>\frac{3}{4}$\\ \hline 					
		%	\end{tabular}
	%	\caption{Illustrations to show that$P_{succ}^{mod}>\frac{3}{4}$ }
	%\end{table*}
	%\textbf{Note-5}:- One can also check the result for different settings of Bell operator for any entangled state $\rho$. So, to conclude in all settings of Bell operator $P_{succ}^{mod}>\frac{3}{4}$ for any general 2-qubit entangled state.
	
	
	
	%\section{Extra}
	%The expression of the maximum probability that measures the amount of non-locality is given by 
	%\begin{equation}
	%		P^{max}=\frac{3}{4}-\frac{Tr[W_{CHSH}\rho]}{8} 
	%		\label{26}
	%\end{equation}
	%It is clear from (\ref{26}) that if $\rho$ satisfy the Bell inequality, i.e. if $\rho$ is not detected by $W_{CHSH}$ then $P^{max}$ $\leq \frac{3}{4}$.\\
	%Let us now consider the witness operator for different settings such as $(x,z), (y,z)$ and $(z,x)$ settings and they are given by 
	%\begin{eqnarray*}
	%	W_{CHSH}^{(i,j)}&=& 2I-B_{CHSH}^{(i,j)} ,i,j=x,y,z \& i\neq j\nonumber \\
	%	&=& 2I-(A_{0}\otimes B_{0}-A_{0}\otimes B_{1}+A_{1}\otimes B_{0}+A_{1}\otimes B_{1})
	%\end{eqnarray*}
	%Choosing $A_{0}=\sigma_{j}$, $B_{0}=\frac{\sigma_{i}+\sigma_{j}}{\sqrt{2}}$ \\
	%$~~~~~~~~~~~~~~~~~A_{1}=\sigma_{i}$, $B_{1}=\frac{\sigma_{i}-\sigma_{j}}{\sqrt{2}}$ \\
	% Therefore, using (\ref{26}), we have 
	% \begin{equation*}
		%	Tr[	W_{CHSH}^{(i,j)}\rho] =6-8P^{max}_{i,j}, i,j=x,y,z \& i\neq j
		%\end{equation*}
		%Let us consider the witness operator 
		%\begin{eqnarray*}
		%	W&=& \frac{1}{4}[I+ \frac{1}{2\sqrt{2}}(B_{CHSH}^{(x,y)}+B_{CHSH}^{(x,z)}+B_{CHSH}^{(y,z)})]\nonumber \\
		%	&=& \frac{1}{4}[I+ \frac{1}{2\sqrt{2}}(6I-W_{CHSH}^{(x,y)}-W_{CHSH}^{(x,z)}-W_{CHSH}^{(y,z)})]\nonumber \\
		%	&=& \frac{1}{4}[\frac{3+\sqrt{2}}{\sqrt{2}}I- \frac{1}{2\sqrt{2}}(W_{CHSH}^{(x,y)}+W_{CHSH}^{(x,z)}+W_{CHSH}^{(y,z)})]
		%\end{eqnarray*}
		%For any two-qubit state $\rho$, we have 
		%\begin{eqnarray*}
		%	Tr[W\rho] &=& \frac{3+\sqrt{2}}{4\sqrt{2}}- \frac{1}{8\sqrt{2}}[18-8(P_{x,y}^{max}+P_{x,z}^{max}+P_{y,z}^{max})]\nonumber \\
		%	&=& \frac{\sqrt{2}-6}{4\sqrt{2}}+ \frac{1}{\sqrt{2}}[P_{x,y}^{max}+P_{x,z}^{max}+P_{y,z}^{max}]
		%\end{eqnarray*}
		%If the state $\rho$ is an entangled state and not detected by W then $Tr[W\rho] \geq 0$.\\
		%\begin{eqnarray*}
		%	\Rightarrow P_{x,y}^{max}+P_{x,z}^{max}+P_{y,z}^{max} \geq \frac{6-\sqrt{2}}{4} =1.14644 
		%\end{eqnarray*}
		
		%\subsection{Condition that may indicate the presence of entanglement in two-qubit state} 
		%\noindent Can we have any criterion by which it is possible to identify class of two-qubit entangled states described by the density operator $\rho_{AB}$ that can be used as a resource in a quantum game? In other words, the above question may be posed in this way also: Can we have any criterion for the identification of a class of two-qubit states $\rho_{AB}$ that can be used as a resource in a quantum game and for which the maximum probability $P_{max}$ of success of winning the game will be greater than $\frac{3}{4}$? In this subsection, we have investigated this question by developing an inequality that may specify the presence of two-qubit entanglement.\\
		%There are many entanglement detection schemes that may detect the entangled states. The partial transposition (PT) criterion can be considered as a very strong entanglement detection criterion in comparison to other criterion for $2\otimes 2$ system. PT criterion based on the minimum eigenvalue of the partial transposition of a quantum state. Let us denote the minimum eigenvalue of the partial transposition of a quantum state $\rho_{AB}$ as $\lambda_{min}(\rho_{AB}^{T_{B}})$. If $\lambda_{min}(\rho_{AB}^{T_{B}})>0$ then the state is a separable state. Otherwise the state is entangled. For any specific state, it is very easy to find the value of $\lambda_{min}(\rho_{AB}^{T_{B}})$ but the task of obtaining the analytical expression of $\lambda_{min}(\rho_{AB}^{T_{B}})$ will be very complicated for a general two-qubit state. Thus, in this section, we will provide the lower bound of $\lambda_{min}(\rho_{AB})$ and $\lambda_{min}(\rho_{AB}^{T_{B}})$ for any general two-qubit state $\rho_{AB}$, that may help in detecting the entanglement.\\
		%Let us consider the general two-qubit state which is of the form \cite{luo}
		%\begin{equation}
		%	\rho_{AB}= \frac{1}{4}[I\otimes I+ \overrightarrow{a}.\overrightarrow{\sigma}\otimes I +I \otimes \overrightarrow{b}.\overrightarrow{\sigma}+ \sum c_{j} \sigma_{j} \otimes \sigma_{j}]
		%	\label{gen2qbitst}
		%\end{equation} 
		%where $\vec{a}=(a_{1},a_{2},a_{3})\in R^{3}$, $\vec{b}=(b_{1},b_{2},b_{3})\in R^{3}$, $c_{i}\in R$ and $\sigma_{i}$ denote the Pauli matrices.\\
		%Taking the partial transpose on the second subsystem of $\rho_{AB}$, we get 
		%\begin{equation}
		%	\rho_{AB}^{T_{B}}= \frac{1}{4}[I\otimes I+ \overrightarrow{a}.\overrightarrow{\sigma}\otimes I +I \otimes \overrightarrow{b}.\overrightarrow{\sigma}^{T}+ \sum c_{j} \sigma_{j} \otimes \sigma_{j}^{T}]
		%	\label{gen2qbitst1}
		%\end{equation}
		%Using weyl's inequality \cite{horn}, the lower bound of $\rho_{AB}$ and $\rho_{AB}^{T_{B}}$ can be obtained as
		%\begin{eqnarray}
		%	\lambda_{min}(\rho_{AB})&\geq& \frac{1}{4}[1-\sum_{i=1}^{3}a_{i}-\sum_{i=1}^{3}b_{i}+\lambda_{min}(\sum_{i=1}^{3}c_{i}\sigma_{i}\otimes \sigma_{i})]\nonumber\\&\equiv&  (\lambda_{min}(\rho_{AB}))_{lb}
		%	\label{lb1}
		%\end{eqnarray}
		%\begin{eqnarray}
		%	\lambda_{min}(\rho_{AB}^{T_{B}})&\geq& \frac{1}{4}[1-\sum_{i=1}^{3}a_{i}-\sum_{i=1}^{3}b_{i}+\lambda_{min}(\sum_{i=1}^{3}c_{i}\sigma_{i}\otimes \sigma_{i}^{T})]\nonumber\\&\equiv& (\lambda_{min}(\rho_{AB}^{T_{B}}))_{lb}
		%	\label{lb2}
		%\end{eqnarray}
		%Now, our task is to choose the state parameters in such a way so that $(\lambda_{min}(\rho_{AB}))_{lb}\geq 0$ and $(\lambda_{min}(\rho_{AB}^{T_{B}}))_{lb}<0$ holds simultaneously. The two-qubit state $\rho_{AB}$ is an entangled state if and only if $\lambda_{min}(\rho_{AB}^{T_{B}}) <0$. Thus, using the condition $(\lambda_{min}(\rho_{AB}))_{lb}\geq 0$ and $(\lambda_{min}(\rho_{AB}^{T_{B}}))_{lb}<0$ in (\ref{lb1}) and (\ref{lb2}), we can say that there is a possibility that the two-qubit state $\rho_{AB}$ is an entangled state. Hence, the state parameter may be chosen according to the condition given in Table-I and Table-II. We should note that all the cases are not included in the above two tables so one may consider the above procedure to discuss the other cases.\\
		%\begin{tabular}{|c|c|c|c|}\hline
		%	Cases & Nature of $c_{i}$'s & lower bound of & lower bound of  \\    &  & $\lambda_{min}(\rho_{AB})$ & $\lambda_{min}(\rho_{AB}^{T_{B}})$ \\\hline
		%	\textbf{Case-I} & $c_{i}\geq 0$  & $\frac{1}{4}(k_{1}-c_{1}$ & $\frac{1}{4}(k_{1}-c_{1}$\\ & $(i=1,2,3)$ & $-c_{2}-c_{3})$ & $-c_{2}+c_{3})$\\\hline
		%	\textbf{Case-II} & $c_{i}\leq 0$  & $\frac{1}{4}(k_{1}-c_{1}$ & $\frac{1}{4}(k_{1}+c_{1}$\\ & $(i=1,2,3)$ & $+c_{2}+c_{3})$ & $+c_{2}+c_{3})$ \\\hline
		%	\textbf{Case-IIIa} &$c_{1},c_{2}\geq 0$  & $\frac{1}{4}(k_{1}-c_{1}$ & $\frac{1}{4}(k_{1}-c_{1}$\\  & $c_{3}<0$ & $+c_{2}+c_{3})$ & $-c_{2}+c_{3})$ \\\hline  
		%	\textbf{Case-IIIb} &$c_{1},c_{2}\geq 0$  & $\frac{1}{4}(k_{1}-c_{1}$ & $\frac{1}{4}(k_{1}-c_{1}$\\ & $c_{3}<0$ & $-c_{2}-c_{3})$ & $-c_{2}+c_{3})$ \\\hline
		%	\textbf{Case-IVa} & $c_{1}\geq 0, c_{2}< 0$  & $\frac{1}{4}(k_{1}-c_{1}$ & $\frac{1}{4}(k_{1}-c_{1}$\\ & $c_{3}<0$ & $+c_{2}+c_{3})$ & $-c_{2}+c_{3})$ \\\hline
		%	\textbf{Case-IVb} & $c_{1}\geq 0, c_{2}< 0$  & $\frac{1}{4}(k_{1}-c_{1}$ & $\frac{1}{4}(k_{1}+c_{1}$\\ & $c_{3}<0$ & $+c_{2}+c_{3})$ & $+c_{2}+c_{3})$ \\\hline
		%\end{tabular}
		%\begin{tabular}{|c|c|c|c|}\hline
		%	Cases & Nature of $c_{i}$'s & lower bound of & lower bound of  \\  &  & $\lambda_{min}(\rho_{AB})$ & $\lambda_{min}(\rho_{AB}^{T_{B}})$ \\\hline
		%	\textbf{Case-I} & $c_{i}\geq 0$  & $\frac{1}{4}(k_{1}-c_{1}$ & $\frac{1}{4}(k_{1}-c_{1}$\\ & $(i=1,2,3)$ & $-c_{2}-c_{3})$ & $-c_{2}+c_{3})$\\\hline
		%	\textbf{Case-II} & $c_{i}\leq 0$  & $\frac{1}{4}(k_{1}+c_{1}$ & $\frac{1}{4}(k_{1}+c_{1}$\\ & $(i=1,2,3)$ & $-c_{2}+c_{3})$ & $+c_{2}+c_{3})$ \\\hline
		%	\textbf{Case-IIIa} & $c_{1},c_{2}\geq 0$  & $\frac{1}{4}(k_{1}-c_{1}$ & $\frac{1}{4}(k_{1}-c_{1}$\\ & $c_{3}<0$ & $-c_{2}-c_{3})$ & $-c_{2}+c_{3})$ \\\hline  
		%	\textbf{Case-IIIb} & $c_{1},c_{2}\geq 0$  & $\frac{1}{4}(k_{1}+c_{1}$ & $\frac{1}{4}(k_{1}-c_{1}$\\ & $c_{3}<0$ & $-c_{2}+c_{3})$ & $-c_{2}+c_{3})$ \\\hline
		%	\textbf{Case-IVa} & $c_{1}\geq 0, c_{2}< 0$  & $\frac{1}{4}(k_{1}-c_{1}$ & $\frac{1}{4}(k_{1}-c_{1}$\\ & $c_{3}<0$ & $+c_{2}+c_{3})$ & $-c_{2}+c_{3})$ \\\hline
		%	\textbf{Case-IVb} & $c_{1}\geq 0, c_{2}< 0$  & $\frac{1}{4}(k_{1}-c_{1}$ & $\frac{1}{4}(k_{1}+c_{1}$\\ & $c_{3}<0$ & $+c_{2}+c_{3})$ & $+c_{2}+c_{3})$ \\\hline
		%\end{tabular}\\\\\\
		%where $k_{1}=1-a_{1}-a_{2}-a_{3}-b_{1}-b_{2}-b_{3}$.\\ 
		%For instance, consider the two-qubit state which is given by
		%\begin{equation}
		%	\rho_{AB}^{(c)}= \frac{1}{4}[I\otimes I+ 0.2 I \otimes \sigma_{3}+ 0.5 \sigma_{1} \otimes \sigma_{1}+ 0.3 \sigma_{2} \otimes \sigma_{2}-0.19 \sigma_{3} \otimes \sigma_{3}]
		%	\label{2qubitcr1}
		%\end{equation} 
		%Let us now apply our criterion to check whether the state $\rho_{AB}^{(c)}$ is entangled or not. Firstly, we can observe that in the above example, we have $b_{3}=0.2$, $c_{1}=0.5$, $c_{2}=0.3$ and $c_{3}=-0.19$. Therefore, it comes under case-III in the Table-I. Thus, $(\lambda_{min}(\rho_{AB})_{lb}=0.0475$ $(\lambda_{min}(\rho_{AB}^{T_{B}}))_{lb}=-0.0475$. These values indicate the fact that $\rho_{AB}^{(c)}$ indeed represent a quantum state and it may be an entangled state. The actual value of $(\lambda_{min}(\rho_{AB})$ and $(\lambda_{min}(\rho_{AB}^{T_{B}}))$ can be calculated as $0.0913447$ and $-0.00365528$ respectively.
		%\subsection{Strength of non-locality of two-qubit entangled system measured through witness operator $W_{CHSH}$}
	
	%Since the state $\rho_{AB}^{(1)}$ has the parameters $c_{1}=0.7$, $c_{2}=0.2$, $c_{3}=-0.5$ so this falls under case-IIIb of table-I and it can be easily checked that the lower bound of $\lambda_{min}(\rho_{AB}^{(1)})$ is positive and the lower bound of $\lambda_{min}((\rho_{AB}^{(1)})^{T_{B}})$ is negative and thus it is an implication of the fact that the state $\rho_{AB}^{(1)}$ may be an entangled state.\\

%The state $\rho_{AB}^{(2)}$ constructed here has the parameters $a_{1}=0.001$, $c_{1}=0.8$, $c_{2}=0.89$, $c_{3}=-0.9$ so this falls under case-IIIb of table-II and it can be easily checked that the lower bound of $\lambda_{min}(\rho_{AB}^{(2)})$ is positive and the lower bound of $\lambda_{min}((\rho_{AB}^{(2)})^{T_{B}})$ is negative. This shows that the state $\rho_{AB}^{(2)}$ may be an entangled state.\\

%In other words, can the players play the game using shared entangled state for which $P^{max} \in [0,\frac{3}{4}]$? 

%If the answer is in affirmative then can we identify the structure of two-qubit entangled state for which $P^{max} \leq \frac{3}{4}$?\\

%When $Tr[B_{ij}\sigma_{sep}] \in [-2,0)$, we may take $a<0$ and When $Tr[B_{ij}\sigma_{sep}] \in [0,2]$, we may consider $a\geq 0$.

%then we will use the expression of newly constructed witness operator to modify the maximum probability of winning the game. If $B^{(ij)}_{CHSH}$ denote the Bell operator given by
%\begin{eqnarray}
%B^{(ij)}_{CHSH}&=&\sigma_{i}\otimes \frac{\sigma_{i}+\sigma_{j}}{\sqrt{2}}+ \sigma_{i}\otimes \frac{\sigma_{i}-\sigma_{j}}{\sqrt{2}}\nonumber\\&+&
%\sigma_{j}\otimes \frac{\sigma_{i}+\sigma_{j}}{\sqrt{2}}- \sigma_{j}\otimes \frac{\sigma_{i}-\sigma_{j}}{\sqrt{2}},\nonumber\\&& i,j=x,y,z~~ \&~~ i\neq j
%\label{bchsh}
%\end{eqnarray}
%then we have shown that the constructed witness operator $W_{ij}$ that may detect entangled states which satisfy the Bell-CHSH inequality.\\
%\textbf{Result-1:} Let us consider a two-qubit entangled state described by the density operator $\rho_{ent}$. If\\
%(i) $M(\rho_{ent})\leq 1$\\
%(ii) $F(\rho_{ent})> \frac{1}{2}$ and \\
%(iii) $0<a<\frac{F(\rho_{ent})-\frac{1}{2}}{2(1-\sqrt{M(\rho_{ent})})}$
%\\
%then the witness operator $W_{ij}$ may detect the entangled state $\rho_{ent}$. \\
%\textbf{Proof:} Recalling the witness operator $W^{(1)}_{ij}~(i,j=x,y,z;i\neq j)$ given in (\ref{witdef}), it may be easily shown that the expression $(\frac{1}{2}-F(\rho_{ent}))+2a(1-\sqrt{M(\rho_{ent})})$ will be negative when $a\in (0,\frac{F(\rho_{ent})-\frac{1}{2}}{2(1-\sqrt{M(\rho_{ent})})})$.
%Therefore, when $a\in (0,\frac{F(\rho_{ent})-\frac{1}{2}}{2(1-\sqrt{M(\rho_{ent})})})$, the inequality (\ref{r1}) may be expressed as
%\begin{eqnarray}
%Tr[W^{(1)}_{ij}\rho_{ent}]> \text{a negative quantity}
%\label{r12}
%\end{eqnarray} 
%The inequality (\ref{r12}) tells us that there is a possibility  for $W^{(1)}_{ij}$ to take the negative value. Hence, if the parameter $a$ lies in the interval given in (iii) then the witness operator $W^{(1)}_{ij}$ may detect the entangled state $\rho_{ent}$, which satisfies the condition (i) and (ii).\\
%\textbf{Observation-1:} If the two-qubit entangled state $\rho_{ent}$ violate the Bell-CHSH inequality then it is useful in teleportation \cite{horo4}. Therefore, if $M(\rho_{ent})>1$ then $F(\rho_{ent})>\frac{1}{2}$. Thus, if $M(\rho_{ent})>1$ then the witness operator $W^{(1)}_{ij}$ also detect the entangled state $\rho_{ent}$.\\ 
%\textbf{Example-2} Let us consider the two-qubit state described by the density operator $\rho_{1}$
%\begin{eqnarray}
%	\rho_{1}= \begin{pmatrix} 
	%		\frac{1-x}{2} & 0& 0& 0.35\\
	%		0 & \frac{x}{2} & 0 & 0\\
	%		0 & 0 & \frac{x}{2} & 0\\
	%		0.35 & 0& 0& \frac{1-x}{2} \\
	%	\end{pmatrix},~~ 0\leq x \leq 0.3
%\end{eqnarray}
%It can be easily verified that the state $\rho_{1}$ is an entangled state for $x \in [0,0.3]$ and also useful as a resource state in quantum teleportation in the same interval of $x$.  
%The expectation value of the Bell operators $B_{xy}$, $B_{yz}$ and $B_{xz}$ with respect to the state $\rho_{1}$ can be calculated as
%\begin{eqnarray}
%Tr[\langle \phi^{+} |\rho_{2}| \phi^{+} \rangle ]&=&0.35+ \frac{1-x}{2},~0\leq x \leq 0.3 \nonumber \\ &\in& [0.7,0.85], ~0\leq x \leq 0.3 \nonumber \\
\langle B_{xy} \rangle_{\rho_{1}}=0,~ 0\leq x \leq 0.3 
%\label{b1}\\
\langle B_{yz} \rangle_{\rho_{1}}\in [-0.41,0.41], %\text{when}~0\leq x \leq 0.3 
%\label{b2}\\
%\langle B_{xz} \rangle_{\rho_{1}} \in \begin{cases}
	%	(2,2.4] & \text{when $0\leq x < 0.143 $}\\
	%	(1.55,2) & \text{when $0.143 <x \leq 0.3$}\\
	%\end{cases}
	%\label{b3}
	%\end{eqnarray}
	%We may note from Eq. (\ref{b3}) that the state $\rho_{1}$ violate the Bell-CHSH inequality by the Bell operator $B_{xz}$ when $x \in [0,0.143]$. On the other hand, Eq. (\ref{b1}), Eq. (\ref{b2}), and Eq. (\ref{b3}) shows that the state $\rho_{1}$ does not violate the Bell-CHSH inequality by the Bell operators $B_{xy}$ and $B_{yz}$ when $x \in [0,0.143]$ but the expectation value of the Bell operator $B_{xz}$ with respect to the state $\rho_{1}$ does not violate the Bell-CHSH inequality when $x \in (0.143,0.3)$.\\ 
	%Let us now calculate the expectation value of the witness operator $W^{(\phi^{+})}_{ij}$ with respect to the state $\rho_{1}$ in the settings $xy-$, $yz-$ and $zx-$ plane. The expectation values of the witness operators are given by 
	%\begin{eqnarray}
	% Tr[W_{xy}^{(\phi^{+})}\rho_{1}]&=& 0.15+2a-(\frac{1-x}{2})
	%\label{wxy}\\
	%Tr[W_{yz}^{(\phi^{+})}\rho_{1}]&=& 0.15+1.575736a-(\frac{1-x}{2})+a(2.82843x)\nonumber\\ \label{wyz}\\
	%Tr[W_{xz}^{(\phi^{+})}\rho_{1}]&=&0.15-(\frac{1-x}{2})-a (0.40416-2.82843x)\nonumber\\
	%\label{wxz} 
	%\end{eqnarray}
	%existence and uniqueness strum lioville theorem weakness theorem
	%From Eq. (\ref{wxy}), Eq. (\ref{wyz}) and Eq. (\ref{wxz}),  we find that (i) the witness operator $W_{xy}^{(\phi^{+})}$ detects the state $\rho_{1}$, when $a\in (0,0.094]$ \& $ x \in [0,0.3]$, (ii) the witness operator $W_{yz}^{(\phi^{+})}$ detects the state $\rho_{1}$, when $a\in (0,0.082]$ \& $ x \in [0,0.3]$, and (iii) the witness operator $W_{xz}^{(\phi^{+})}$ detecs the state $\rho_{1}$, when $a\in (0,0.444]$ \& $ x \in [0.143,0.3]$, and when $a\in (0,1]$ \& $ x \in [0,0.143)$.
	
	
	%Therefore, the state $\rho_{2}$ detected by the witness operator $W_{ij}$ as an entangled state for all the three settings and even it detects the state in that range also where $M(\rho_{2})$ is less than 1.\\
	
	%is not detected by $M(\rho_{2})$ for state parameter $x \in [0.143,0.3]$  and also the state $\rho_{2}$ does not violate the Bell-CHSH inequality in any setting for a given range of the state parameter $x  \in [0.143,0.3]$.\\ 
	
	%when  M($\rho_{2}$) is less than 1 for $x \in [0.143,0.3]$ and M($\rho_{2}$) is greater than 1 for $x \in [0,0.143)$. Hence it can be concluded that for $x \in [0.143,0.3]$ state $\rho_{2}$ will satisfy Bell-CHSH operator and will violate Bell-CHSH operator for $x \in [0,0.143)$. \\
	
	
	%In the second example, we found that there exist a range of the parameter $a$ and $x$ where the state $\rho_{1}$ is not useful as a resource state in quantum teleportation but the witness operator $W_{xz}$ detect the entangled state in the same domain of the parameter $a$ and $x$. On the other hand, in the first example, we found that there exist an entangled state useful for teleportation and also detected by the witness operators $W_{xy}$, $W_{yz}$ and $W_{zx}$ respectively. Thus, these observations tempt us to investigate that whether there exist any range of the parameter $a$ for which the entangled state is useful in teleportation and also detected by all the witness operators $W_{xy}$, $W_{yz}$ and $W_{zx}$.\\
	%To achieve this aim, 
%By direct calculation, we obtain the expectation value of the Bell operators $B_{xy}$, $B_{xz}$, $B_{yz}$ and in different setting with respect to the state $\rho_{AB}^{new1}$ as
%\begin{eqnarray}
%	&&Tr[\langle \phi^{+} |\rho_{AB}^{new1}| \phi^{+} \rangle]\nonumber \\&=& \frac{0.899789 - 0.5\lambda_{4}^{2} + 1.19951\lambda_{4}\sqrt{1-\lambda_{4}^{2}}}{1.79958-\lambda_{4}^{2}} \nonumber \\
%	&&\langle B_{xy} \rangle_{\rho_{AB}^{new1}}=0%~ 0\leq \theta \leq 0.4175\pi 
%	\nonumber \\
%	&&\langle B_{yz} \rangle_{\rho_{AB}^{new1}}=\langle B_{xz} \rangle_{\rho_{AB}^{new1}}=\nonumber\\ &&\frac{2.54499-1.41421\lambda_{4}^{2}+3.39274\lambda_{4}\sqrt{1-\lambda_{4}^{2}}}{1.79958-\lambda_{4}^{2}} \nonumber 
%\end{eqnarray}
%we need to first calculate the expectation value of Bell-CHSH operator and the expectation value of the constructed witness $W_{ij}^{(\phi+)}$ such that for which values of a state  $\rho_{BA}^{new3'}$ is detected.\\
%By direct calculation, we obtain the expectation value of the Bell operators $B_{xy}$, $B_{xz}$, $B_{yz}$ and in different setting with respect to the state $\rho_{BA}^{new3'}$ as
%\begin{eqnarray}
%	&&Tr[\langle \phi^{+} |\rho_{BA}^{new3'}| \phi^{+} \rangle]\nonumber \\&=&  0.5 + 0.5 \sqrt{1-p} - 0.25 p \nonumber \\
%	&&\langle B_{xy} \rangle_{\rho_{BA}^{new3'}}=0%~ 0\leq \theta \leq 0.4175\pi 
%	\nonumber \\
%	&&\langle B_{yz} \rangle_{\rho_{BA}^{new3'}}=\langle B_{xz} \rangle_{\rho_{BA}^{new3'}}=\nonumber\\ &&1.41421 + 1.41421\sqrt{1-p}- 1.41421 p \nonumber 
%\end{eqnarray}
%Let us now calculate the expectation value of the witness operatr $W_{xy}^{(\phi^{+})}$ with respect to the state $\rho_{BA}^{new3'}$
%\begin{eqnarray}
%	Tr[W_{xy}^{(\phi^{+})}(\rho_{BA}^{new3'})]&=& \frac{1}{2}-2a\nonumber \\ &-&(0.5 + 0.5 \sqrt{1-p} - 0.25 p)
%\end{eqnarray}
%As it can be seen that after the measurement on the controller's qubit we are able to increase the fidelity of the state and hence we get non-zero controller's power. So, we can say that standard W state  under amplitude damping channel is useful in Controlled Teleportation.

%It can be easily verified that by putting the above considered values $\rho_{BA}^{new3''}$ is an entangled state for $p \in [0.5,1]$ and also . 

% Hence, it can be concluded that after applying measuremnt on the third qubit of three qubit state $\rho_{BAC}^{(3'')}$ fidelity of the obtained state becomes greater than $\frac{2}{3}$ which was less than $\frac{2}{3}$ before measuremnt. Now to calculate the power of the controller as defined in Equation-(\ref{p}) we need to first calculate the expectation value of Bell-CHSH operator and the expectation value of the constructed witness $W_{ij}^{(\phi+)}$ such that for which values of a state  $\rho_{BA}^{new3''}$ is detected.\\
%By direct calculation, we obtain the expectation value of the Bell operators $B_{xy}$, $B_{xz}$, $B_{yz}$ and in different setting with respect to the state $\rho_{BA}^{new3''}$ as
%\begin{eqnarray}
%	&&Tr[\langle \phi^{+} |\rho_{BA}^{new3''}| \phi^{+} \rangle]=  1- 0.5p  \nonumber \\
%	&&\langle B_{xy} \rangle_{\rho_{BA}^{new3'}}=0%~ 0\leq \theta \leq 0.4175\pi 
%	\nonumber \\
%	&&\langle B_{yz} \rangle_{\rho_{BA}^{new3'}}=1.41421 p\nonumber\\ &&
%	\langle B_{xz} \rangle_{\rho_{BA}^{new3'}}= 2.82843- 1.41421 p\nonumber\\ &&
%\end{eqnarray}
%Let us now calculate  Now, we are in a postion to calculate the lower limit of power obtained in Equation-(\ref{p}) 


%As it can be seen that after the measurement on the controller's qubit we are able to increase the fidelity of the state and hence we get non-zero controller's power. So, we can say that standard W state  under phase damping channel is useful in Controlled Teleportation.

%\section{Comparision Analysis}
%\noindent In this section, we will compare our results of the states analyzed above with Artur et al. work. Firstly, let us consider $|W_{1}\rangle$ which is of the form \begin{eqnarray}
	%	|W_{1}\rangle=\frac{1}{\sqrt{4}}(|100\rangle+|010\rangle+\sqrt{2}|001\rangle)
	%\end{eqnarray}
	%and now we will calculate the controller's power of  $|W_{1}\rangle$ using Artur et al. defined power and compared it to our obtained power in Equation-(\ref{pw1}). In Artur et al. paper, when third system is traced out non-conditioned fidelity is defined as $F_{NC}= \frac{3+\lVert T_{AB}\rVert_{1}}{6}$ and conditoned fidelity is defined as $F_{C}=\frac{2+\sqrt{\tau+C_{AB}^{2}}}{3}$. Now, calculating $F_{NC}$ and $F_{C}$ for $|W_{1}\rangle$, we get $F_NC(	W_{AB}^{(1)})=\frac{2}{3}$ and $F_C(	W_{AB}^{(1)})=0.75$  and power is calculated as the difference between $F_{C}$ and $F_{NC}$, which is calculated as P($W_{AB}^{(1)}$)=0.083333. 

%Suppose we have a three qubit enatngled state $\rho_{klm}$ shared between Charlie, Alice and Bob. It should be noted here that, $\rho_{ent}$ is a two qubit state obtained after measurement on kth qubit, which can also be denoted as $\rho_{CT}^{k}$ and $\rho_{G}$ is a two qubit state obtained after tracing out system k, which can also be denoted as $\rho_{lm}$.



%Teleportation has played a very crucial role in quantum information theory for the last few years\cite{nielsen,wilde}. Teleportation provides 100 percent efficiency in 2 qubit system but when a third party is involved teleportation process is not that much efficient. So, for a 3-qubit state, another process was developed by Karlson et. al. which is called controlled quantum teleportation\cite{karlson}. In this process, he has shown that Alice wants to send a qubit to Bob with the help of controller Charlie. The controller plays a very important role in it, he can solely control the type of state Bob will receive after teleportation. So to measure his power another term was defined by Shoini Et. al. called Controllers Power in terms of fidelity\cite{Shoini}.

%Initially Classical games were used to calculate the probabilities of winning but scientists came up with an idea that why not use quantum tools like superposition, quantum computing, entanglement, etc. to classical games and see their effects\cite{eisert}.  Using quantum tools in classical games gave rise to a new field known as quantum games with a lot of scope and improvement. In classical games, we can get the maximum probability of 0.75 for winning a game, but by using quantum tools we can increase the maximum probability up to  0.8535 for winning the game.\cite{archan}. %Still it is not shown that we will always get probability greater than 0.75 with the help of quantum mechanics. But in this paper we will show that by the introduction of controller we will always get probability of wining greater than equal to 0.75.

%In this paper, we would like to introduce the role of the controller in controlling the probability of winning the game. We will observe how the controller can increase or decrease the probabilities of winning. We will analyze different cases to observe the controllers role in winning the game of Alice and Bob. Further, we will also define controllers power in terms of the probability of winning, which was defined by Shohini. et.al.\cite{Shoini} but it was in terms of fidelity. 
%Here, we will consider the Non-Local Quantum game there are four persons Alice, Bob, Charlie(the controller), and Referee. The Referee randomly selects questions and sends those questions to Alice and Bob directly. Here, Alice, Bob, and Charlie are playing the CHSH game and they will follow this set of rules, which are given below: 
%"In this type of game, Alice and Bob do not know the questions but they know the probability distribution by which the Referee is selecting the questions. After receiving the questions they are not allowed to communicate, but before the game starts Alice, Bob and Charlie are allowed to communicate to decide the quantum state they want to share with them but this quantum state will be decided by these three mutually so that they can get the maximum probability." The state shared between Alice, Bob and Charlie are decided by these three-person mutually before starting the game. Now, Charlie will apply measurement on his qubit, and an entangled state is shared between Alice and Bob, of which they both are unaware. Now, the state shared between Alice and Bob is decided by Charlie, who is controlling the state shared between Alice and Bob. If the controller wishes, he can increase or decrease the probability by sharing a suitable state between Alice and Bob. Then Alice and Bob both will apply the suitable measurement to questions, that they have received from the referee and submit their answers to the referee. Then the referee will apply the suitable predicate to calculate the probability of winning or losing the game.
%The conditioned fidelity $F_{CT}(\rho^{(k)})$ may be obtained when the shared two-qubit state is formed after controller Charlie performing the Von nuemann measurement on his qubit, and non-conditioned fidelity denoted by $F_{NC}(\rho_{lm})$ is the teleportation fidelity of the two-qubit state $\rho_{lm}$ when Charlie's qubit is traced out from the initially shared three-qubit state $\rho_{ABC}$.\\
%In controlled quantum teleportation scheme, there are three parties Alice (sender), Bob (receiver), and Charlie(the controller). Alice the sender wants to teleport an arbitrary qubit to a Bob in the presence of Charlie. Alice, Bob, and Charlie shares a quantum channel known as resource state between them. In standard Controlled quantum teleportation Alice will perform Bell-state measurement on her qubits and after her measurement she uses classical channel to communicate the  reasult obtained after the measurement performed by her to Bob in the form of two classical bits. And also after Alice's measurement, an entangled state will be projected between Charlie and Bob. Charlie will do von neumann mesurement on his particle, after the measurement a quantum state will be projected at Bob's location in terms of Charlie's parameter. This teleportation scheme is controlled by Charlie in the sense that, he can adjust his parameter such that which can correspondingly increase or decrease the fidelity of the teleportation. For that he will send one classical bit to tell him about his measurement. Then Bob will reconstruct the quantum state at his place by applying suitable phase shift on the recevied qubit according to the measurement outcome sent by the Charlie. This process of teleportating a qubit in three party scenario is called Controlled Quantum Teleportation\cite{karlsson}.\\

%\subsubsection{Maximally Slice State}
%noindent Let us consider a pure three qubit entangled state known as maximally slice state, which is of the form 
%\begin{eqnarray}
%	|\psi^{(2)}\rangle_{ABC}&=& \lambda_{0}|000\rangle +\lambda_{1}|100\rangle+\frac{1}{\sqrt{2}}|111\rangle \nonumber 
%	\label{msstate}
%\end{eqnarray}
%where due to the normalization condition $N^{2}=\lambda_{0}^{2}+\lambda_{1}^{2}=\frac{1}{2}$.\\
%The pure state described by the density operator $\rho_{ABC}^{(2)}= |\psi^{(2)}\rangle_{ABC}\langle\psi^{(2)}|$ is an entangled state. After tracing out system C from $\rho_{ABC}^{(2)}$ we get a two qubit state $\rho_{AB}^{(2)}$ of the form 
%\begin{eqnarray}
%	\rho_{AB}^{(2)}&=&\lambda_{0}^{2}|00\rangle \langle00| + \lambda_{1}^{2}|10\rangle \langle10|+ \lambda_{0}\lambda_{1}|10\rangle \langle00|\nonumber \\&+&\lambda_{0}\lambda_{1}|00\rangle \langle10|+\frac{1}{2}|11\rangle \langle11|
%\end{eqnarray}
%The non-conditioned fidelity can be easily calculated as $F_{NC}(\rho_{AB}^{(2)})=0.666667- 0.333333\lambda_{4}^{2}$ which is less than $\frac{2}{3}$ for $\lambda_{4} \in (0,\sqrt{0.5}]$. Now, after applying the measurement on the third qubit we get a state $\rho_{AB}^{new2}$, which is of the form 
%\begin{eqnarray}
%	\rho_{AB}^{new2}&=&\frac{1}{p_{0}N^{2}}\bigg( (t^{2}+y_{3}^{2})(\lambda_{0}^{2}|00\rangle \langle00|+ \lambda_{0}\lambda_{1}|10\rangle \langle00|\nonumber \\&+& \lambda_{0}\lambda_{1}|00\rangle \langle10|+\lambda_{1}^{2}|10\rangle \langle10|) +(-ty_{2}+y_{1}y_{3}\nonumber \\ &+&\iota(ty_{1}+y_{2}y_{3}))\frac{1}{\sqrt{2}}(\lambda_{0}|00\rangle \langle11|+\lambda_{1}|10\rangle \langle11|)\nonumber \\&+& (-ty_{2}+y_{1}y_{3}-\iota(ty_{1}+y_{2}y_{3}))\frac{1}{\sqrt{2}}(\lambda_{0}|00\rangle \langle11|\nonumber \\ &+&\lambda_{1}|10\rangle \langle11|)+ \frac{1}{2}(y_{1}^{2}+y_{2}^{2}) |11\rangle \langle11|\bigg) 
%\end{eqnarray}
%where $N^{2}$ is the normalization condition and
%$p_{0}=\frac{2(t^{2}+y_{3}^{2})(\lambda_{0}^{2}+\lambda_{1}^{2})+y_{1}^{2}+y_{2}^{2}}{2N^{2}}$\\
%It can be verified that maximum values of conditioned fidelity is acheived by taking the values y1 = 0.6911100024636817,
%y2 = -0.1495924553372768,
%y3 = 0.6910959318830716,
%and t = 0.1495847413687177, also we have considered $\lambda_{0}=\sqrt{0.5-{\lambda_{1}}^{2}}$ 
%It can be easily verified that by putting the above considered values $\rho_{AB}^{new2}$ is an entangled state for $\lambda_{1} \in [0,\sqrt{0.5}]$ and also M($\rho_{AB}^{new2}$) is greater than 1. And the Fidelity calculated after the controller's measurement is expressed as 
%\begin{eqnarray}
%	F_{CT}^{3}(\rho_{AB}^{new2})&=&  0.333333(1 + 2(0.5 - 0.499989\lambda_{1}^{2} \nonumber \\&+& 0.707107\sqrt{0.5 -\lambda_{1}^{2}}))
%\end{eqnarray}
%And it can be easily verified that $F_{CT}^{3}(\rho_{AB}^{new2})$ varies from [0.66667,1) when $\lambda_{1}$ varies from (0,0.6435]. Hence, it can be concluded that after applying measuremnt on the first qubit of three qubit state $\rho_{ABC}^{2}$ fidelity of the obtained state becomes greater than $\frac{2}{3}$ which was less than $\frac{2}{3}$ before measuremnt. Now to calculate the power of the controller as defined in Equation-(\ref{p}) we need to first calculate the expectation value of Bell-CHSH operator and the expectation value of the constructed witness $W_{ij}^{(\phi+)}$ such that for which values of a state  $\rho_{AB}^{new2}$ is detected.\\
%By direct calculation, we obtain the expectation value of the Bell operators $B_{xy}$, $B_{xz}$, $B_{yz}$ and in different setting with respect to the state $\rho_{AB}^{new2}$ as
%\begin{eqnarray}
%	&&Tr[\langle \phi^{+} |\rho_{AB}^{new2}| \phi^{+} \rangle]\nonumber \\&=& 0.5 - 0.499989\lambda_{1}^{2} + 0.707107\sqrt{0.5-\lambda_{1}^{2}}\nonumber \\
%	&&\langle B_{xy} \rangle_{\rho_{AB}^{new2}}=0%~ 0\leq \theta \leq 0.4175\pi 
%	\nonumber \\
%	&&\langle B_{yz} \rangle_{\rho_{AB}^{new2}}=1.41421 - 2.82837\lambda_{1}^{2}- 2\sqrt{0.5-\lambda_{1}^{2}}\nonumber\\ &&
%	\langle B_{xz} \rangle_{\rho_{AB}^{new2}}=1.41421 - 2.82837\lambda_{1}^{2} + 2\sqrt{0.5-\lambda_{1}^{2}}\nonumber 
%\end{eqnarray}
%Let us now calculate the expectation value of the witness operatr $W_{xy}^{(\phi^{+})}$ with respect to the state $\rho_{AB}^{new2}$
%\begin{eqnarray}
%	&&Tr[W_{xy}^{(\phi^{+})}(\rho_{AB}^{new1})]= \frac{1}{2}-2a\nonumber \\ &-&\big( 0.5 - 0.499989\lambda_{1}^{2} + 0.707107\sqrt{0.5-\lambda_{1}^{2}}\big)
%\end{eqnarray}
%Witness operator $W_{xy}^{(\phi^{+})}$ detects the state $\rho_{AB}^{new2}$ when $a\in (0,0.042]$ \& $0< \lambda_{1} < 0.6 $. Now, we are in a postion to calculate the lower limit of power obtained in Equation-(\ref{p}) 
%\begin{eqnarray}
%	\frac{4a}{3} (1-\sqrt{M(\rho_{AB}^{new2})})-\frac{2}{3}Tr[W_{ij}^{(\phi^{+})}\rho_{AB}^{new2}] \leq P \leq \frac{1}{2}
%\end{eqnarray}
%and it can be easily verified that lower bound of power varies from [0.0001,0.333) for $a\in (0,0,0402]$ \& $0< \lambda_{1}< 0.6$.\\
%As it can be seen that after the measurement on the controller's qubit we are able to increase the fidelity of the state and hence we get non-zero controller's power. So, we can say that Maximally Slice state is useful in Controlled Teleportation.
%\subsubsection{$W_{n}$ State}
%\noindent Let us consider a class of three qubit pure entangled state of the form 
%\begin{eqnarray}
%	|W_{n}\rangle=\frac{1}{\sqrt{2+2n}}(|100\rangle+e^{\iota\gamma}\sqrt{n}|010\rangle+e^{\iota\delta}\sqrt{n+1}|001\rangle)
%	\label{ws}
%\end{eqnarray}
%This is class of W state and it is a very special state since it is useful in teleportation shown by A. K. Pati and P. Aggarwal. We will check whether it will be useful for controlled quantum teleportation or not ?\\
%Here, we have studied the class of $|W_{n}\rangle$ for n=1,2,3,4 and for simplicity we have considered $\delta=\gamma=0$.\\
%Firstly, we traced out the third system from the density operator $W_{ABC}^{(n)}=|W_{n}\rangle \langle W_{n}|$ and we get $W_{AB}^{n}$ is expressed as 
%\begin{eqnarray}
%	W_{AB}^{(n)}&=&\frac{1}{2+2n}\big((n+1)|00\rangle \langle00|+ |10\rangle \langle10|\nonumber \\&+&\sqrt{n}(|10\rangle \langle01|+|01\rangle \langle10|)+n|01\rangle \langle01|\big)
%	\label{wnc}
%\end{eqnarray}
%and the two-qubit state received between Alice and Bob after applying the measurement operator on the third qubit, is expressed as 
%\begin{eqnarray}
%	W_{BC}^{(modn)}&=&\frac{1}{2+2n}\big[(t^{2}+y_{3}^{2})(|10\rangle \langle10|+\sqrt{n}(|01\rangle \langle10|\nonumber \\&+&|10\rangle \langle01|)+n|01\rangle \langle01|)+(-ty_{2}+y_{1}y_{3}\nonumber \\&-&\iota(ty_{1}+y_{2}y_{3}))(\sqrt{n+1}|00\rangle \langle10|\nonumber \\&+&\sqrt{n}\sqrt{n+1}|00\rangle \langle01|)+ (-ty_{2}+y_{1}y_{3}\nonumber \\&+&\iota(ty_{1}+y_{2}y_{3}))(\sqrt{n+1}|10\rangle \langle00|+\sqrt{n}\sqrt{n+1}\nonumber \\ &&|01\rangle \langle00|) (y_{1}^{2}+y_{2}^{2})(n+1)|00\rangle \langle00| \big]
%	\label{wc}
%\end{eqnarray}
%In particular, let us study the case for $n=1$. Let us consider a three qubit pure entangled state of the form 
%\begin{eqnarray}
%	|W_{1}\rangle=\frac{1}{\sqrt{4}}(|100\rangle+|010\rangle+\sqrt{2}|001\rangle)
%\end{eqnarray}
%The pure state described by the density operator $W_{ABC}^{(1)}= |W_{1}\rangle_{ABC}\langle W_{1}|$ is an entangled state. After tracing out system C from $W_{ABC}^{(1)}$, we get a two qubit state $W_{AB}^{(1)}$ of the form by putting n=1 in Equation-(\ref{wnc}) 
%\begin{eqnarray}
%	W_{AB}^{(1)}=\frac{|10\rangle \langle10|+|01\rangle \langle10|+|10\rangle \langle01|+|01\rangle \langle01|+2|00\rangle \langle00|}{4}
%\end{eqnarray}
%The non-conditioned fidelity of	$W_{AB}^{(1)}$ can be easily calculated as $F_{NC}(W_{AB}^{(1)})=\frac{2}{3}$. Now, after applying the measurement on the third qubit and putting the values of measurements in Equation-(\ref{wc}) as y1 = -0.0000017683054274818906,
%y2 = -0.000000008832301660331127, 
%y3 = 0.27472117488661846, 
%t = 0.9615240567844199, and
%n = 1 we get a two-qubit state $W_{AB}^{(mod1)}$. 
%It can be easily verified that $W_{AB}^{(mod1)}$ is an entangled state and also M($W_{AB}^{(mod1)}$)=2 and the Fidelity calculated after the controller's measurement is $F_{CT}^{3}(W_{AB}^{(mod1)})=0.9999999999968732$
%Hence, it can be concluded that after applying measuremnt on the third qubit of three qubit state $W^{(1)}_{ABC}$ fidelity of the obtained state becomes greater than $\frac{2}{3}$, which was less than $\frac{2}{3}$ before measuremnt. Now to calculate the power of the controller as defined in Equation-(\ref{p}) we need to first calculate the expectation value of Bell-CHSH operator and the expectation value of the constructed witness $W_{ij}^{(\psi+)}$ such that for which values of a state  $W_{AB}^{(mod1)}$ is detected.\\
%By direct calculation, we obtain the expectation value of the Bell operators $B_{xy}$, $B_{xz}$, $B_{yz}$ and in different setting with respect to the state $W_{AB}^{(mod1)}$ as
%\begin{eqnarray}
%	&&Tr[\langle \psi^{+} |W_{AB}^{(mod1)}| \psi^{+} \rangle]= 0.9999999999968732\nonumber \\
%	&&\langle B_{xy} \rangle_{W_{AB}^{(mod1)}}=2.82843%~ 0\leq \theta \leq 0.4175\pi 
%	\nonumber \\
%	&&\langle B_{yz} \rangle_{W_{AB}^{(mod1)}}=\langle B_{xz} \rangle_{W_{AB}^{(mod1)}}= 0.0000000000044222 \nonumber\\ 
%\end{eqnarray}
%Let us now calculate the expectation value of the witness operatr $W_{xz}^{(\psi^{+})}$ with respect to the state $W_{AB}^{(mod1)}$
%\begin{eqnarray}
%	&&Tr[W_{xz}^{(\psi^{+})}(W_{AB}^{(mod1)})]= -\frac{1}{2}+2a
%\end{eqnarray}
%Witness operator $W_{xz}^{(\psi^{+})}$ detects the state $W_{AB}^{(mod1)}$ when $a\in (0,0.249]$. Now, we are in a postion to calculate the lower limit of power obtained in Equation-(\ref{p}) 
%\begin{eqnarray}
%	\frac{4a}{3} (1-\sqrt{M(W_{AB}^{(mod1)})})-\frac{2}{3}Tr[W_{xz}^{(\psi^{+})}W_{AB}^{(mod1)}] \leq P \leq \frac{1}{2}
%	\label{pw1}
%\end{eqnarray}
%and it can be easily verified that lower bound of power varies from [0.00146455,0.333) for $a\in (0,0.1767]$.\\
%As it can be seen that after the measurement on the controller's qubit we are able to increase the fidelity of the state and hence we get non-zero controller's power. So, we can say that $|W_{1}\rangle$ state is useful in Controlled Teleportation.\\


%Now, we will analyze the Controller's power obtained for different n in $W_{n}$ state. Since all states $W_{1}$, $W_{2}$, $W_{3}$, and $W_{4}$ are useful for controlled teleportation since we get non-zero controlled power for each state, so further we analyze all the states together to conclude which state is more useful for controlled teleportation. 
%\noindent As, it can be seen in the above figure $W_{1}$ is more useful than $W_{2}$, $W_{3}$, and $W_{4}$ state in controlled teleportation. After $W_{1}$ state than $W_{2}$ is useful in controlled telportation. So, we can conclude that as n increases in $W_{n}$ the controller's power decreases.
