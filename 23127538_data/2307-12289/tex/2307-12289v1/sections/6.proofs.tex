%!TeX root = ../lmcs-gandalf22.tex

\section{Appendix}
%TODO CAPIRE CHE ALTRE DEFINIZIONI DI CAP.2 DI VALENTINO AGGIUNGERE
% DEFINIZIONI MANCANTI:
% - matching function per rule graphs e relazione \models
% - graph concatenation \graphconcat

\runfuncmap*

\begin{proof}
  \proofif%
  We proceed by induction on the length of the event sequence $\evseq =
  \seq{\event_1, \ldots, \event_n}$.
  \begin{description}[before={\renewcommand\makelabel[1]{\bfseries ##1.}},
    labelsep=*, leftmargin=*]
  \item[Base case] % base case: empty sequence
    If $n = 0$, the only well defined function on an empty codomain is the
    function $\gamma_0: \emptyset \to \emptyset$ with an empty domain, which
    vacuously satisfies the definition of matching function and
    \Cref{lemma:function-matching-run:entire-atom,lemma:function-matching-run:partial-atom}.
    Then, the only run of $\M_\E = (V, D, \emptyset, 0)$ on an empty event
    sequence $\evseq$ is the empty run $\matchseq$ yielding $\M_\E$ itself,
    which vacuously satisfies the definition of run.
    % Inductive hypothesis: if there exists a matching function $\gamma:M_{n-1}
    % \to [1,\ldots,n-1]$ satisfying
    % \Cref{lemma:function-matching-run:entire-atom,lemma:function-matching-run:partial-atom},
    % then there exists a run $\seq{I_1, \ldots, I_{n-1}}$ of $\M_\E$ on
    % $\seq{\event_1, \ldots, \event_{n-1}}$, yielding a matching structure
    % $\M_{n-1}$.
  \item[Inductive step] % inductive step for n
    Let $\gamma:M_{n} \to [1,\ldots, n]$ be a matching function satisfying
    \Cref{lemma:function-matching-run:entire-atom,lemma:function-matching-run:partial-atom},
    and let $\restrict{\gamma}^{<n}:M_{n-1}\to[1,\ldots,n-1]$ be the restriction
    of $\gamma$ on the domain $M_{n-1}$ defined as the inverse image of
    $[1,\ldots,n-1]$ under $\gamma$, \ie, $M_{n-1} =
    \gamma^{-1}([1,\ldots,n-1])$. $\restrict{\gamma}^{<n}$ is a matching
    function for the event sequence $\slice\evseq_{1,n-1}$ and satisfies
    \cref{lemma:function-matching-run:entire-atom,lemma:function-matching-run:partial-atom}.
    %
    % If $T' \in M_{n-1}$, then $T' \in M_{n}$ and $\gamma(T') < n$, so that $T
    % \in M_{n}$, $\gamma(T) \leq \gamma(T') < n$, and $T \in M_{n-1}$.
    % Furthermore, $\restrict{\gamma}^{<n}(T) \le \restrict{\gamma}^{<n}(T')$
    % and $l \le
    % \delta(\slice\evseq_{\restrict{\gamma}^{<n}(T),\restrict{\gamma}^{<n}(T')})
    % \le u$, because $\restrict{\gamma}^{<n}$ is a restriction of $\gamma$. If
    % $T' \not\in M_{n-1}$ there can be two cases, either $T'$ belongs to the
    % domain of $\gamma$, \ie, $T'\in M_n$, or not. In the first case,
    % $\gamma(T')$ must be equal to $n$ and
    % $\delta(\slice\evseq_{\restrict{\gamma}^{<n}(T),n-1}) \leq
    % \delta(\slice\evseq_{\restrict{\gamma}^{<n}(T),\gamma(T')}) \leq u$, by
    % \Cref{lemma:function-matching-run:entire-atom} for $\gamma$. In the latter
    % case, $\delta(\slice\evseq_{\restrict{\gamma}^{<n}(T),n-1}) \leq
    % \delta(\slice\evseq_{\restrict{\gamma}^{<n}(T),n}) \leq u$, by
    % \Cref{lemma:function-matching-run:partial-atom} for $\gamma$.
    %
    By the inductive hypothesis, there exists a run $\seq{I_1, \ldots, I_{n-1}}$
    of $\M_\E$ on $\slice\evseq_{1,n-1}$, yielding a matching structure
    $\M_{n-1} = (V, D_{n-1}, M_{n-1}, t_{n-1})$. Let $I_n = \gamma^{-1}(n)$, and
    note that $I_n \subseteq \overline{M_{n-1}}$. We show that $\event_n = (A_n,
    \delta_n)$ is an \emph{$I_n$-match} event for $\M_{n-1}$ by breaking the
    proof in steps.

    \statement{$\event_n$ is an \emph{admissible} event for $\M_{n-1}$} Let $T
    \in M_{n-1}$ and $T' \not\in M_{n-1}$. If $D_{n-1}[T',T] = +\infty$,
    $\delta_n \le D_{n-1}[T',T]$ trivially holds. Otherwise, there exists an
    atom $T \before_{l,u} T'$ in $\clause$ and $D_{n-1}[T',T] = u -
    \delta(\slice\evseq_{\restrict\gamma^{<n}(T),n-1})$. We consider two cases
    based on whether $T'$ belongs to the domain of $\gamma$, or not. In the
    first case, $\gamma(T') = n$ and $\delta(\slice\evseq_{\gamma(T),n-1}) +
    \delta_n = \delta(\slice\evseq_{\gamma(T),\gamma(T')}) \le u$, by
    \Cref{lemma:function-matching-run:entire-atom}. In the second case,
    $\delta(\slice\evseq_{\gamma(T),n-1}) + \delta_n =
    \delta(\slice\evseq_{\gamma(T),n}) \le u$, by
    \Cref{lemma:function-matching-run:partial-atom}. In either case, $\delta_n
    \leq u - \delta(\slice\evseq_{\gamma(T),n-1}) = D_{n-1}[T',T]$.

    \statement{\Cref{def:match-event:good-match:start} of
      \Cref{def:match-event}} Let $a[x = v]$ be a quantified token of $\E$. If
    $\tokstart(a) \in I_n$, then $\gamma(\tokstart(a)) = n$ and by definition of
    matching function $\tokstart(x,v)\in A_n$.

    \statement{\Cref{def:match-event:good-match:end} of
      \Cref{def:match-event}}\proofif Let $\tokend(a) \not\in M_{n-1}$ be a
    possible candidate for inclusion in $I_n$. If $\tokstart(a) \in M_{n-1}$ and
    $\tokend(x, v) \in A_n$, then $\tokend(x, v)$ ends the token started at
    $\event_{\restrict\gamma^{<n}(\tokstart(a))}$; otherwise, there would exist
    $\event_i = \pair{A_i, \delta_i}$ prior to $\event_n$ such that $\tokend(x,
    v) \in A_i$, contradicting that $\restrict\gamma^{<n}$ is undefined on
    $\tokend(a)$. By definition of matching function, since $\tokend(x, v) \in
    A_n$ ends the token started at $\event_{\gamma(\tokstart(a))}$, we have
    $\gamma(\tokend(a)) = n$ and $\tokend(a) \in I_n$.
    % Versione di Valentino:
    %
    % . If this was not the case, there would exist an event $\event_i =
    % \pair{A_i, \delta_i}$ in
    % $\slice\evseq_{\restrict\gamma^{<n}(\tokstart(a)),n-1}$ (prior to
    % $\event_n$), such that $\tokend(x, v) \in A_i$. Since a matching function
    % correctly identifies the endpoints of the same tokens
    % (\Cref{def:matching-function}), $\restrict\gamma^{<n}$ would be defined on
    % $\tokend(a)$ as $\restrict\gamma^{<n}(\tokend(a)) = i$, contradicting the
    % hypothesis that $\tokend(a) \not\in M_{n-1}$. Again by definition of
    % matching function, since $\tokend(x, v) \in A_n$ ends the token started at
    % $\event_{\restrict\gamma^{<n}(\tokstart(a))} =
    % \event_{\gamma(\tokstart(a))}$, $\gamma$ is such that $\gamma(\tokend(a))
    % = n$ and $\tokend(a) \in I_n$.

    \statement{\Cref{def:match-event:good-match:end} of
      \Cref{def:match-event}}\proofonlyif If $\tokend(a) \in I_n$, then by
    definition of matching function $\tokend(x,v) \in A_n$. Furthermore, since
    $\tokend(a) \in M_n$, \Cref{lemma:function-matching-run:entire-atom} gives
    $\gamma(\tokstart(a)) \le \gamma(\tokend(a))$ for the atom $\tokstart(a)
    \before_{l,u} \tokend(a)$ in $\clause$. By definition of event sequence,
    $\tokstart(x,v)$ and $\tokend(x,v)$ cannot appear in the same event; hence,
    $\gamma(\tokstart(a)) < \gamma(\tokend(a)) = n$ and $\tokstart(a) \in
    M_{n-1}$.

    \statement{\Cref{def:match-event:preceding-terms} of \Cref{def:match-event}}
    Let $T$ be a term in $I_n$, and let $T'\in V$ be any other term such that
    $D_{n-1}[T',T] \leq 0$. Then, $D_{n-1}[T',T]$ can either be the lower bound
    of an atom $T' \before_{l,u} T$, or the upper bound of an atom $T
    \before_{l,u} T'$ in $\clause$. In the first case, we can directly conclude
    that $T' \in M_{n-1} \cup I_n$, because $T' \in M_n$ by
    \Cref{lemma:function-matching-run:entire-atom} of $\gamma$ and $M_n =
    M_{n-1} \cup I_n$ by definition of $M_{n-1}$ and $I_n$. In the second case,
    note that $D_{n-1}[T',T] = u$, \ie, it has never been decremented because $T
    \not\in M_{n-1}$, and that upper bounds $u$ can never be negative. Thus, $u$
    is equal to $0$ and $\gamma$ satisfies $0 \le
    \delta(\slice\evseq_{\gamma(T),\gamma(T')}) \le 0$
    (\Cref{lemma:function-matching-run:entire-atom}), meaning that $\gamma(T') =
    \gamma(T)$ and $T' \in I_n$.
    % Alternativa scritta da Renato:
    %
    % Let $T' \in V$ be any other term such that $D_{n-1}[T',T] \le 0$. If
    % $D_{n-1}[T',T]$ is the lower bound of an atom $T' \before_{l,u} T$ or the
    % upper bound of an atom $T \before_{l,u} T'$, then $T' \in M_n$ by Lemma 1
    % and $M_n = M_{n-1} \cup I_n$. Otherwise, $u = D_{n-1}[T',T] = 0$ and Lemma 1
    % gives $\gamma(T') = \gamma(T)$; hence $T' \in I_n$.

    \statement{\Cref{def:match-event:lower-bounds} of \Cref{def:match-event}}
    Let $T \in I_n$ and $T' \in M_{n-1}$. $D_{n-1}[T',T]$ cannot be the upper
    bound of an atom $T \before_{l,u} T'$; otherwise,
    \Cref{lemma:function-matching-run:entire-atom} would imply $T \in M_{n-1}$,
    contradicting $T \in I_n$. Thus, $D_{n-1}[T',T]$ must either represent the
    lower bound of an atom $T' \before_{l,u} T$ in $\clause$, or be equal to
    $+\infty$. In the latter case, $\delta_n \ge -D_{n-1}[T',T]$ trivially
    holds. In the former case, $D_{n-1}[T',T] = -l +
    \delta(\slice\evseq_{\restrict\gamma^{<n}(T'),n-1})$. Since $\gamma(T) = n$,
    we have $\delta(\slice\evseq_{\gamma(T'), \gamma(T)}) =
    \delta(\slice\evseq_{\gamma(T'), n}) =
    \delta(\slice\evseq_{\gamma(T'),n-1})+\delta_n \ge l$. Hence, $\delta_n \ge
    l - \delta(\slice\evseq_{\gamma(T'),n-1}) = -D_{n-1}[T',T]$.

    \statement{\Cref{def:match-event:zero-no-bounds} of \Cref{def:match-event}}
    Let $T, T' \in I_n$ be two distinct terms. Then, $\gamma(T') = \gamma(T)$
    and $\delta(\slice\evseq_{\gamma(T'),\gamma(T)}) = 0 $. If $T \before_{l,u}
    T'$ (resp., $T' \before_{l,u} T$) belongs to $\clause$, then $D_{n-1}[T,T']$
    (resp., $D_{n-1}[T',T]$) is the lower bound $l$ and equals 0 by
    \Cref{lemma:function-matching-run:entire-atom}. Otherwise, $D_{n-1}[T,T'] =
    D_{n-1}[T',T] = +\infty$.
    % Versione di Valentino:
    %
    % Let $T'$ be any other term in $I_n$ different from $T$, and note that
    % $\gamma(T') = \gamma(T)$, so that
    % $\delta(\slice\evseq_{\gamma(T'),\gamma(T)}) =
    % \delta(\slice\evseq_{\gamma(T),\gamma(T')}) = 0 $. Since neither $T$ nor
    % $T'$ belongs to $M_{n-1}$, both $D_{n-1}[T', T]$ and $D_{n-1}[T,T']$ are
    % equal to their initial value. If the atom $T \before_{l,u} T'$ belongs to
    % $\clause$, then $D_{n-1}[T,T']$ is the lower bound $l$, and since $\gamma$
    % satisfies $l \le \delta(\slice\evseq_{\gamma(T'), \gamma(T)}) = 0$, it
    % must be the case that $D_{n-1}[T,T'] = 0$. If instead is the atom $T'
    % \before_{l,u} T$ that belongs to $\clause$, $D_{n-1}[T',T]$ is the lower
    % bound $l$ and $D_{n-1}[T',T] = 0$ for the same argument. Lastly, if there
    % is no atom in $\clause$ relating $T$ and $T'$, then $D_{n-1}[T,T'] =
    % D_{n-1}[T',T] = +\infty$.

    Hence, $\M_{n-1} \stepm[n] \M_n$ is well defined and $\seq{I_1, \ldots,
      I_n}$ is a run of $\M_\E$ on $\evseq$ yielding $\M_n$.
  \end{description}

  
  \proofonlyif We proceed by induction on the length of the event sequence $\evseq =
  \seq{\event_1, \ldots, \event_n}$.

  \begin{description}[before={\renewcommand\makelabel[1]{\bfseries ##1.}},
    labelsep=*, leftmargin=*]
  \item[Base case] An empty run $\matchseq$ yields $\M_\E = (V, D, \emptyset,
    0)$ itself. Then the function $\gamma_0: \emptyset \to \emptyset$ vacuously
    satisfies the definition of matching function and
    \Cref{lemma:function-matching-run:entire-atom,lemma:function-matching-run:partial-atom}.
  \item[Inductive step]
    Let $\matchseq = \seq{I_1, \ldots, I_n}$ be a run of $\M_\E$ on $\evseq$,
    yielding a matching structure $\M_n = (V, D_n, M_n, t_n)$. Note that
    $\slice\matchseq_{1,n-1}$ is a run of $\M_\E$ on $\slice\evseq_{1,n-1}$
    yielding a matching structure $\M_{n-1} = (V, D_{n-1}, M_{n-1}, t_{n-1})$.
    By the inductive hypothesis, there exists a matching function $\gamma_{<n}:
    M_{n-1} \to [1, \ldots, n-1]$ satisfying
    \Cref{lemma:function-matching-run:entire-atom,lemma:function-matching-run:partial-atom}.
    Let $\gamma: M_{n} \to [1,\ldots, n]$ be the extension of $\gamma_{<n}$ to
    $M_n$, such that $\gamma(T) = n$, for all $T\in I_n$.

    \statement{$\gamma$ is a matching function}
    \Cref{def:matching-function:nodes,def:matching-function:tokens} hold for all
    the terms already present in the domain of $\gamma_{<n}$. For every term in
    $I_n$, \Cref{def:matching-function:nodes} for $\gamma$ follows from
    \Cref{def:match-event:good-match} of $I_n$-match event. Let $\tokstart(a),
    \tokend(a) \in M_n$ be two terms not both already present in $M_{n-1}$,
    meaning that $\tokstart(a) \in M_{n-1}$ and $\tokend(a) \in I_n$, for some
    quantified token $a[x=v]$ in $\E$. By definition of $I_n$-match event,
    $\event_n = (A_n, \delta_n)$ is such that $\tokend(x,v) \in A_n$ and no
    other event in $\slice\event_{\gamma_{<n}(T),n-1}$ contains an action
    $\tokend(x, v)$, otherwise $\tokend(a)$ would already belong to $M_{n-1}$
    (by \Cref{def:match-event:good-match:end} of $I$-match event).
    % according to \Cref{def:match-event:good-match:end} of the definition of
    % $I$-match event.
    Thus, $\tokend(x, v) \in A_n$ ends the token started at
    $\event_{\gamma(\tokstart(a))}$, and $\gamma(\tokstart(a))$ and
    $\gamma(\tokend(a))$ correctly identify the endpoints of such token.

    \statement{\Cref{lemma:function-matching-run:entire-atom} of
      \Cref{lemma:function-matching-run}} Let $T \before_{l, u} T'$ be an atom
    in $\clause$, and note that $\gamma$ already satisfies
    \Cref{lemma:function-matching-run:entire-atom} for every $T' \in M_{n-1}$.
    If $T'\in I_n$ instead, consider the entry $D_{n-1}[T,T']$ representing the
    lower bound $l$ of the aforementioned atom. If $D_{n-1}[T,T'] \le 0$,
    \Cref{def:match-event:preceding-terms} of $I$-match event gives $T \in
    M_{n-1} \cup I_n = M_n$. If $D_{n-1}[T,T'] > 0$, $D_{n-1}[T,T']$ no longer
    stores its initial value $-l \leq 0$, meaning that $T$ must have been
    previously matched and $T \in M_{n-1} \subseteq M_n$. In either case, $T\in
    M_n$ and $\gamma(T) \le \gamma(T')$, because $\gamma(T) \le n$.
    % proof of lower and upper bounds of \gamma
    If $T \in I_n$, then $\delta(\slice\evseq_{\gamma(T),\gamma(T')}) = 0 \le
    u$, is trivially satisfied by any upper bound $u$. Furthermore, by
    \Cref{def:match-event:zero-no-bounds} of $I$-match event, either the lower
    bound $D_{n-1}[T,T'] = 0$ or the upper bound $D_{n-1}[T',T] = 0$, and they
    both equal their initial values $l$ and $u$. Note that the former case is
    also implied by the latter, so that $l = 0$ and $l \le
    \delta(\slice\evseq_{\gamma(T),\gamma(T')})$. If $T \in M_{n-1}$, by
    \Cref{def:match-event:lower-bounds} of $I$-match event, $\delta_n \ge
    -D[T,T'] = l - \delta(\slice\evseq_{\gamma(T),n-1})$. Hence, $l \le
    \delta(\slice\evseq_{\gamma(T),n-1}) + \delta_n =
    \delta(\slice\evseq_{\gamma(T),\gamma(T')})$. While $\delta_n \le
    D_{n-1}[T',T] = u - \delta(\slice\evseq_{\gamma(T),n-1})$, since $\event_n$
    is an admissible event for $\M_{n-1}$. Hence,
    $\delta(\slice\evseq_{\gamma(T),n-1}) + \delta_n =
    \delta(\slice\evseq_{\gamma(T),\gamma(T')}) \le u$.

    \statement{\Cref{lemma:function-matching-run:partial-atom} of
      \Cref{lemma:function-matching-run}} Let $T \before_{l,u} T'$ be an atom in
    $\clause$ such that $T \in M_n$ and $T' \not\in M_n$. Since $\event_n$ is an
    admissible event for $\M_{n-1}$, $\delta_n \le D_{n-1}[T',T] = u -
    \delta(\slice\evseq_{\gamma(T),n-1})$. Hence,
    $\delta(\slice\evseq_{\gamma(T),n-1}) + \delta_n =
    \delta(\slice\evseq_{\gamma(T),n}) \le u$.
  \end{description}
\end{proof}

\superset*
  %Let $\evseq =\langle\event_1,\dots,\event_n\rangle$ be an event sequence
  %(closed to the left??), let $\M_\E$ be the initial matching structure of some
  %existential statement $\E$ of a rule $\Rule$, and let $\M_r$ be an active
  %matching structure for the trigger event $\event_r$ of $\Rule$ resulting from
  %a run $\M_\E \runm*[r] \M_r$. If there exists an active matching structure
  %$\M_s$, resulting from a run $\M_\E \runm*[s] \M_s$, for a trigger event of
  %$\Rule$ prior to $\event_r$, then there exists an active matching structure
  %$\M$ for the same trigger event of $\M_s$ resulting from a run $\M_\E \runm*
  %\M$ and matching at least as many tokens as $\M_r$.

\begin{proof}
  Let $\M_\E \runm*[r]\M_r = (V, D_r, M_r, t_r)$ and $\M_\E \runm*[s] \M_s = (V,
  D_s, M_s, T_s)$, with $\gamma_s(\tokstart(a_0)) \le \gamma_r(\tokstart(a_0))$.
  Let $M = M_r \cup M_s$ and $\gamma: M \to [1,\ldots,n]$ be a function defined
  as:
  \[
    \gamma(T) =
    \begin{cases}
      \gamma_s(T) &\text{if } T\in M_s\cap M_r\text{ and }\gamma_s(T)\leq\gamma_r(T)\\
      \gamma_r(T) &\text{if } T \in M_s \cap M_r \text{ and }\gamma_s(T)>\gamma_r(T)\\
      \gamma_s(T) &\text{if } T \in M_s \setminus M_r\\
      \gamma_r(T) &\text{if } T \in M_r \setminus M_s
    \end{cases}
  \]

  \statement{$\gamma$ is a matching function} \Cref{def:matching-function:nodes}
  of \Cref{def:matching-function} for $\gamma$ follows from our hypothesis on $\gamma_s$
  and $\gamma_r$. Regarding \Cref{def:matching-function:tokens}, let
  $\tokstart(a), \tokend(a) \in M$ for some quantified token $a[x=v]$ in $\E$.
  If $\gamma_s$ and $\gamma_r$ map the endpoints of $a$ to the same token in
  $\evseq$, then $\gamma(\tokstart(a))$ and $\gamma(\tokend(a))$ correctly
  identify the endpoints of that token. If instead $\gamma_s$ and $\gamma_r$ map
  $a$ to two distinct tokens in $\evseq$, then $\gamma$ would match $a$
  according to the function whose token comes first, correctly identifying the
  endpoints of such token.

  \statement{$\gamma$ satisfies
    \Cref{lemma:function-matching-run:entire-atom,lemma:function-matching-run:partial-atom}
    of \Cref{lemma:function-matching-run}} Let $T \before_{l,u} T'$ be an atom
  in $\clause$. If $T' \in M$, then either $T' \in M_s$, and $T \in M_s
  \subseteq M$, or $T' \in M_r$, and $T \in M_r \subseteq M$. If $\gamma$ maps
  both terms with either $\gamma_s$ or $\gamma_r$, then $\gamma(T) \leq
  \gamma(T')$ and $l \leq \delta(\slice\evseq_{\gamma(T),\gamma(T')}) \leq u$
  immediately follows. If instead
  % \begin{itemize}
  % \item
  $\gamma(T) = \gamma_s(T)$ and $\gamma(T') = \gamma_r(T')$, then $T' \in
    M_r$ and $T \in M_s \cap M_r$. By definition of $\gamma$, $\gamma_s(T) \leq
    \gamma_r(T)$, and, by \Cref{lemma:function-matching-run:entire-atom} for
    $\gamma_r$, $\gamma_r(T) \leq \gamma_r(T')$. Hence, $\gamma(T) \leq
    \gamma(T')$. If $T' \in M_s$, then $\gamma_s(T') > \gamma_r(T')$, and:
    \begin{align*}
      l &\leq\delta_{\gamma_r(T),\gamma_r(T')}&&\text{\Cref{lemma:function-matching-run:entire-atom} for } \gamma_r\\
        &\leq \delta_{\gamma_s(T),\gamma_r(T')}&&\text{}\gamma_s(T)\leq\gamma_r(T)\\
        &< \delta_{\gamma_s(T),\gamma_s(T')} &&\text{} \gamma_s(T') > \gamma_r(T')\\
        &\leq u &&\text{\Cref{lemma:function-matching-run:entire-atom} for } \gamma_s
    \end{align*}
    otherwise:
    \begin{align*}
      l &\leq \delta_{\gamma_r(T),\gamma_r(T')} &&\text{\Cref{lemma:function-matching-run:entire-atom} for } \gamma_r\\
        &< \delta_{\gamma_s(T),\gamma_r(T')}&&\text{} \gamma_s(T) \leq \gamma_r(T)\\
        &\leq \delta_{\gamma_s(T),n} &&\text{} \gamma_r(T') \leq n\\
        &\leq u &&\text{\Cref{lemma:function-matching-run:partial-atom} for } \gamma_s
    \end{align*}
    The case for $\gamma(T) = \gamma_r(T)$ and $\gamma(T') = \gamma_s(T')$ is
    completely symmetrical.
    % \item SYMMETRICAL CASE:
    % $\gamma(T) = \gamma_r(T)$ and $\gamma(T') = \gamma_s(T')$, then $T' \in
    % M_s$ and $T \in M_s \cap M_r$. By definition of $\gamma$, $\gamma_r(T) <
    % \gamma_s(T)$, and, by \Cref{lemma:function-matching-run:entire-atom} for
    % $\gamma_s$, $\gamma_s(T) \leq \gamma_s(T')$. Hence, $\gamma(T) \leq
    % \gamma(T')$. If $T' \in M_r$, then $\gamma_s(T') \leq \gamma_r(T')$, and:
    % \begin{align*}
    %   l &\leq \delta_{\gamma_s(T),\gamma_s(T')} &&\text{\Cref{lemma:function-matching-run:entire-atom} for } \gamma_s\\
    %     &< \delta_{\gamma_r(T),\gamma_s(T')} &&\text{} \gamma_r(T) < \gamma_s(T)\\
    %     &\leq \delta_{\gamma_r(T),\gamma_r(T')} &&\text{} \gamma_s(T') \leq \gamma_r(T')\\
    %     &\leq u &&\text{\Cref{lemma:function-matching-run:partial-atom} for } \gamma_r
    % \end{align*}
    % If $T' \not\in M_t$, then
    % \begin{align*}
    %   l &\leq \delta_{\gamma_s(T),\gamma_s(T')} &&\text{\Cref{lemma:function-matching-run:entire-atom} for } \gamma_s\\
    %     &< \delta_{\gamma_r(T),\gamma_s(T')} &&\text{} \gamma_r(T) < \gamma_s(T)\\
    %     &\leq \delta_{\gamma_r(T),n} &&\text{} \gamma_s(T') \leq n\\
    %     &\leq u &&\text{\Cref{lemma:function-matching-run:partial-atom} for } \gamma_r
    % \end{align*}
  % \end{itemize}

    Lastly, if $T'\not\in M$, but $T \in M$, then either $\gamma(T)=\gamma_s(T)$
    or $\gamma(T) = \gamma_r(T)$, and
    \Cref{lemma:function-matching-run:partial-atom} for $\gamma$ follows from
    \Cref{lemma:function-matching-run:partial-atom} for $\gamma_s$ and
    $\gamma_r$.
\end{proof}

\residualexist*

\begin{proof}
  Let $\gamma(\tokstart(a_0)) = s$, assuming there is no residual matching
  structure $\M_k$ in the sequence $\seq{\M_s, \ldots, \M_{n-1}}$, then for
  every matching structure $\M_i=(V, D_i, M_i, t_i)$, where $s \leq i < n$,
  there exists a pair of terms $\pair{T, T'}$ such that $T \in M_i$ and
  $T'\not\in M_i$, and their distance $D_i[T',T]$ has a finite upper bound. Let
  $E \subseteq V \times V$ be the set that collects all pairs $(T, T')$ for the
  matching structures $\M_i$. We define $\delta_{T, T'}$ as
  $\delta(\slice\evseq_{\gamma(T),\gamma(T')})$ if $\gamma(T')$ is defined, or
  as $\delta(\slice\evseq_{\gamma(T),n})$ otherwise. Let $\delta_E =
  \sum_{\pair{T,T'} \in E} \delta_{T,T'}$ and note that $\delta_E \ge
  \delta(\slice\evseq_{\gamma(\tokstart(a_0)),n})$, because every position in
  $\slice\evseq_{\gamma(\tokstart(a_0)),n}$ is covered by some distance
  $\delta_{T,T'}$. Moreover, each pair $(T, T') \in E$ corresponds to an atom of
  the form $T \before_{l,u} T'$ in $\clause$. According to
  \Cref{lemma:function-matching-run}, we have $\delta_{T, T'} \leq u$, and
  therefore, $\delta_E \leq \window(P)$. Hence, we have
  $\delta(\slice\evseq_{\gamma(\tokstart(a_0)),n}) \leq \delta_E \leq
  \window(P)$: a contradiction hence proving the existence of a residual
  matching structure $M_k$. 
\end{proof}

%-----------------------------------------------------------------------------
% PROOF DEL TEOREMA 3.6 DI LMCS GANDALF (todo check la definizione di run)
%-----------------------------------------------------------------------------
\soundnessCompleteness*

\begin{proof}
  \proofonlyif Let $\evseq = \seq{\event_1,\ldots,\event_n}$ be a solution plan
  for $P$, and let $\stateseq = \seq{q_0,\ldots,q_n}$ be the run of $\A_P$ on
  $\evseq$. We first show that the sink state is never reached, and then that
  $q_n$ is a final state.

  Let $\event_s = (A_s, \delta_s)$ be the trigger event of a rule $\Rule \equiv
  a_0[x_0 = v_0] \implies \E_1 \lor \dots \lor \E_m$, \ie, $\tokstart(x_0,v_0)
  \in A_s$. Since $\evseq$ is a solution plan, there exist tokens satisfying an
  existential statement $\E$ of $\Rule$ for the trigger $\event_s$. Hence, by
  \cref{lemma:function-matching-run,obs:matching-functions} there exists a run
  $\M_\E \runm* \M_n$, yielding a \emph{closed} matching structure $\M_n$, such
  that $\gamma(\tokstart(a_0)) = s$.

  Let $\overline{\M} = \seq{\M_\E, \M_1,\ldots,\M_{n}}$ be the sequence of all
  the matching structures involved in such run. Note that, by construction
  (\cref{sec:automata-construction}), the states of $\stateseq$ induce all the
  possible runs for the \emph{initial} matching structures of $P$ that can be
  defined on $\evseq$. In particular, the run $\gamma$ must be one of them.
  However, only a subsequence of the matching structures $\overline{M}$ will
  appear in the states of the run $\stateseq$. Indeed, we can identify three key
  points for the sequence $\overline{M}$: the least position $s$ such that
  $\M_s$ is \emph{active} (corresponding to $\gamma(\tokstart(a_0))$), the least
  position $h$ following $s$ such that $\M_h$ no longer belongs to the component
  $\Upsilon$ of the states in $\slice\stateseq_{h, n}$, either because $\M_h$ is
  \emph{closed} or because $\delta(\slice\evseq_{s,h}) > \window(P)$, and the
  least position $k$ following $h$ such that $\M_k$ no longer belongs to the
  component $\Delta$ of the states in $\slice\stateseq_{k,n}$, either because
  $\M_k$ is \emph{closed} or because it gets discarded in favour of the matching
  structures of a later trigger event.

  Every matching structure in $\slice{\overline{M}}_{1,h-1}$ belongs to the
  component $\Upsilon$ of a corresponding state in $\slice\stateseq_{1,h-1}$, so
  the set $\Upsilon$ of the state $q_{s-1}$ is such that $\M_s \in
  \step_{\event_s}(\Upsilon_\bot)$, satisfying condition
  \ref{dfa:delta:trigger-capture} of \cref{sec:automata-construction} for the
  trigger event $\event_s$. Matching structures $\slice{\overline{M}}_{s+1,h}$
  instead belong to the set $\step_\event(\Upsilon^\Rule_t)$, for the partition
  $\Upsilon^\Rule_t$ tracking the satisfaction of the trigger event $\event_s$
  of every state $\slice\stateseq_{s,h-1}$. Hence, all such states satisfy
  condition \ref{dfa:delta:no-failed-step} of \cref{sec:automata-construction}.

  We now show that no \emph{active} matching structures for the trigger event
  $\event_s$ exists after some state $q_h$, following $q_s$ in $\stateseq$. Note
  that the run $\gamma$ yields a \emph{closed} matching structure, and if it
  does so within $\window(P)$ time units from the event
  $\event_{\tokstart(a_0)}$, we identified such position as the closed matching
  structure $\M_h$. So that the state $q_{h-1}$ is such that $\M_h \in
  \step_{\event_h}(\Upsilon^\Rule_t)$, for the partition $\Upsilon^\Rule_t$
  tracking the trigger event $\event_s$, and
  $\step_{\event_h}(\Upsilon^\Rule_t)$ is discarded from $q_h$.

  If instead $\gamma$ yields a \emph{closed} matching structure after
  $\window(P)$ time units from the event $\event_{\tokstart(a_0)}$, lets
  identify such position as $\M_j$, with $j \le n$. If $\M_{j-1}$ belongs to the
  set $\Delta(\E)$ of the state $q_{j-1}$, then $\M_j \in
  \step_{\event_j}(\Delta(\E))$, so that $\step_{\event_j}(\Delta(\E))$ is
  \emph{closed} and discarded from $q_j$, alongside all the other matching
  structures in $\Delta(\E')$, for every $E' \in \Phi(\E)$, \ie, for every other
  existential statement $\E'$ of $\Rule$ still tracking the trigger $\event_s$.
  If instead $\M_{j-1}$ for the trigger event $\event_s$ does not belong to the
  set $\Delta(\E)$ of the state $q_{j-1}$, by construction
  (\cref{sec:automata-construction}), there exist a state $q_h$ in which the
  matching structures tracking $\event_s$ have been replaced by those of a later
  event, and they no longer appear in $\Delta(\E)$ from $q_h$ onwards.

  Since $\evseq$ is a solution plan, the previous argument holds for all the
  trigger events in $\evseq$ of any rules in $S$. Hence, conditions
  \labelcref{dfa:delta:trigger-capture,dfa:delta:no-failed-step} are always met,
  \ie, the sink state is never reached, and no active matching structures belong
  to $q_n$, making it a final state.

  \proofif Let $\evseq = \seq{\event_1,\ldots,\event_n}$ be an event sequence
  accepted by $\A_P$ and let $\rho =\seq{q_0,\ldots,q_n}$ be its accepting run.
  We have to show that the plan corresponding to $\evseq$ is a solution plan for
  $P$, \ie, for every event triggering a rule $\Rule$ in $S$, at least one of
  the existential statements of $\Rule$ is satisfied by $\evseq$.

  Let $\event_s = (A_s, \delta_s)$ be an event in $\evseq$ triggering a rule
  $\Rule \equiv a_0[x_0 = v_0] \implies \E_1 \lor \dots \lor \E_m$, \ie,
  $\tokstart(x_0,v_0) \in A_s$. Since the sink state is never visited in an
  accepting run, the state $q_s$, reached upon reading the event $\event_s$, is
  such that the partition $\Upsilon^\Rule_0$, tracking the satisfaction of the
  trigger event $\event_s$, is not empty. For the same reason, the partition
  $\Upsilon^\Rule_t$ tracking $\event_s$ in every state following $q_s$ can
  never be empty as a result of the function $\step_\event$. However, since the
  final state $q_n$ does not contain any \emph{active} matching structure, there
  must exists a state $q_h$ in $\stateseq$ whose partition
  $\step_{\event_{h+1}}(\Upsilon^\Rule_t)$ gets discarded from $q_{h+1}$. This
  can happen either because $\step_{\event_{h+1}}(\Upsilon^\Rule_t)$ is a
  \emph{closed} set, or because the matchings structures in
  $\step_{\event_{h+1}}(\Upsilon^\Rule_t)$ get promoted to the component
  $\Delta$. In the first case, we can conclude that there exists a run $\M_\E
  \runm* \M_n$ for the initial matching structure $\M_\E$ of an existential
  statement $\E$ of $\Rule$ such that $\M_n$ is \emph{closed} and
  $\gamma(\tokstart(a_0)) = s$, hence, by
  \cref{obs:matching-functions,lemma:function-matching-run}, the trigger event
  $\event_s$ satisfies $\Rule$.

  In the second case, let $\Psi$ be the set of existential statements having an
  active matching structure in $\step_{\event_{h+1}}(\Upsilon^\Rule_t)$, so that
  we can identify them as the sets $\Delta(\E)$, for $\E \in \Psi$, in the
  states from $q_{h+1}$ onwards. By \cref{lemma:residual-matching-structure},
  every such set contains a \emph{residual} matching structure. Hence, by
  \cref{obs:residual-run}, they can become empty only if, at some state $q_{k}$
  following $q_h$, $\step_{\event_{k+1}}(\Delta(\E))$ contains a \emph{closed}
  matching structure for some existential statement $\E \in \Psi$. Note that
  the run $\stateseq$ is an accepting run, so every non-empty set
  $\Delta(\E)$ must become empty before the end of the run. Hence, $q_k$ is
  guaranteed to exist.

  However, it may be the case that, by the time
  $\step_{\event_{k+1}}(\Delta(\E))$ is \emph{closed}, $\Delta(\E)$ no longer
  contains the matching structures for the trigger event $\event_s$, but those
  for a later trigger event $\event_r$ of $\Rule$. Since the sets
  $\Delta(\E)$ store only the matching structures tracking the most recent
  trigger event older than $\window(P)$. Thus, if
  $\step_{\event_{k+1}}(\Delta(\E))$ contains a \emph{closed} matching structure
  for $\event_s$, we can directly assert the existence of a run for $\M_\E$
  implying the satisfaction of $\Rule$ for the trigger event $\event_s$. If
  instead $\step_{\event_{k+1}}(\Delta(\E))$ contains a \emph{closed} matching
  structure $\M_r$ for a later event $\event_r$, there exists a run $\M_\E
  \runm*[r] \M_r$, such that $\gamma_r(\tokstart(a_0)) = r$. Furthermore, by a
  previous consideration on $q_{h+1}$, there exists a run $\M_\E
  \xlongrightarrow{\slice\evseq_{1,h+1},\gamma_s} \hat\M_{h+1}$, yielding a
  \emph{residual} matching structure $\hat\M_{h+1}$, and, by
  \cref{obs:residual-run}, such run can be extended on the entire event sequence
  $\M_\E \runm*[s] \hat\M_n$, to yield a \emph{residual} matching structure
  $\hat\M_n$. Given $\M_\E \runm*[r] \M_r$ and $\M_\E \runm*[s] \hat\M_n$, with
  $\gamma_s(\tokstart(a_0)) \le \gamma_r(\tokstart(a_0))$, by
  \cref{lemma:matching-structure-superset}, there exists a run $\M_\E \runm*[s]
  \M_n$ yielding a matching structure $\M_n$ matching as many terms as $\M_r$
  and such that $\gamma_s(\tokstart(a_0)) = s$. Hence, $\M$ is a \emph{closed}
  matching structure for the existential statement $\E$, and, by
  \cref{obs:matching-functions,lemma:function-matching-run}, $\Rule$ satisfies
  the trigger event $\event_s$.

  Furthermore, all the value duration functions are satisfied by
  the tokens in $\evseq$, being encoded as synchronisation rules by the
  automaton $S_P$. Meanwhile, the automaton $T_P$ guarantees the fulfilment of
  the value transition functions. Hence, we can conclude that $\evseq$ is a
  solution plan for $P$, because every rule in $S$ is satisfied, as well as the
  value duration and value transition functions of every state variable.
\end{proof}

%%% Local Variables:
%%% TeX-master: "../lmcs-gandalf22.tex"
%%% End: