%!TeX root = ../lmcs-gandalf22.tex

\section{Conclusions and Future Work}
\label{sec:conclusions}

Our paper presents an effective procedure for synthesizing controllers for timeline-based games, whereas previously, only a proof of the \EXPTIME[2]-completeness of the problem of determining the existence of a strategy was available in the literature. We use a novel construction of a \emph{deterministic} automaton of doubly-exponential (thus optimal) size, which is then adapted to serve as the arena for the game. Then, with standard methods, we solve a reachability game on the arena to effectively compute the winning strategy for the game, if it exists.

This work paves the way for future developments. First, the procedure provided in this paper can be realistically implemented and tested. It is conceivable, though, that to avoid the state explosion problem due to the doubly-exponential size of the automaton, it will be necessary to apply \emph{symbolic techniques}. Moreover, an implementation would also need a concrete syntax to specify timeline-based games. Existing languages supported by timeline-based systems (\eg NDDL~\cite{CestaO96} or ANML~\cite{SmithFC08}) might be inadequate for this purpose. Next, as in the case of \LTL, the high complexity makes one wonder whether simpler but still expressive fragments can be found. One possibility might be restricting the synchronization rules to only talk about the \emph{past} concerning the rule's trigger. For co-safety properties (\ie properties expressing the fact that something good will eventually happen) expressed in pure-past \LTL, the realizability problem goes down to being \EXPTIME-complete, and by analogy, this might happen to pure-past timeline-based games as well.