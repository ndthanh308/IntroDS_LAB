%!TeX root = ../lmcs-gandalf22.tex

\section{Introduction}
\label{sec:introduction}
Automated planning is the field of \emph{artificial intelligence} that studies the development of autonomous agents capable of reasoning about how to reach some goals, starting from a high-level description of their operating environment. It is one of the most studied fields of AI, with early work going several decades back~\cite{McCarthyH69,FikesN71}.
%
Most of the research by the planning community focuses on the
\emph{action-based} approach, where planning problems are modeled in terms of
\emph{actions} that an agent has to perform to suitably change its \emph{state}.
The task is to devise a sequence of such actions that lead to the goal when
executed starting from a given initial state~\cite{FikesN71,FoxL03}.

In this paper, we focus on the different paradigm of \emph{timeline-based planning}, an approach born and developed in the space sector~\cite{Muscettola94}. In timeline-based planning, there is no explicit separation among actions, states, and goals. Planning domains are represented as systems of independent but interacting components, whose behavior over time, the \emph{timelines}, is governed by a set of temporal constraints, called \emph{synchronization rules}.

Over the years, timeline-based planning systems have been developed and successfully used by space agencies on both sides of the Atlantic~\cite{CestaCFOP06, CestaCDDFOPRS07, FrankJ03, BernardiniS07, ChienRKSEMESFBST00},  for short- to long-term mission planning~\cite{ChienRTTDNASVGA15} as well as on-board autonomy~\cite{FratiniCORD11}. The main advantage of such a paradigm in these contexts is the ability of these systems of handling both planning and \emph{execution} in a uniform way: by the use of \emph{flexible timelines}, timeline-based planners can produce robust plans that, during execution, can be adapted to the current contingency.

However, flexible timelines currently employed in timeline-based systems only
handle \emph{temporal uncertainty}, where the precise timings of events in the
plan are unknown, but the causal sequence of the events is determined.
In particular, they cannot generate robust plans against an environment empowered with
general \emph{nondeterminism}. To overcome this limitation, the concept of
\emph{timeline-based games} was introduced~\cite{GiganteMOCR20}. In
timeline-based games, state variables belong either to the controller or to the
environment. The controller aims at satisfying its set of \emph{system} rules, while
the environment can make arbitrary moves, as long as the \emph{domain} rules 
that define the game arena are satisfied. A controller's strategy is
winning if it guarantees that the controller wins, regardless of the choices
made by the environment. The moves available to the two players can determine
both \emph{what} happens and \emph{when} it happens, thus handling temporal uncertainty
and general nondeterminism in a uniform way.

Determining whether a winning strategy exists for timeline-based games has been
proved to be \EXPTIME[2]-complete~\cite{GiganteMOCR20}. However, there is
currently no effective way to synthesize a controller that implements such
strategies. A necessary condition for synthesizing a finite-state strategy and
the corresponding controller is the availability of a \emph{deterministic}
arena. Two methods to obtain such an arena have been followed in the literature, 
but both have limitations and turn out to be inadequate. On the one hand, 
the complexity result of~\cite{GiganteMOCR20} relies on the construction 
of a (doubly exponential) \emph{concurrent game structure} used to model
check some Alternating-time Temporal Logic (ATL) formulas~\cite{AlurHK02}. 
Even though such a structure is deterministic and theoretically suitable 
to solve a reachability game and synthesize a controller, its construction 
relies on theoretical nondeterministic procedures that are not realistically 
implementable. On the other hand, Della~Monica~\etal~\cite{DellaMonicaGMS18} 
have devised an automata-theoretic solution that provides a concrete and effective 
way to construct an automaton that accepts a word if and only if the original 
planning problem has a solution plan. Unfortunately, the size of the resulting 
\emph{nondeterministic} automaton is already doubly exponential, and its
determinization would result in a further blowup and thus in a non-optimal procedure.

The present paper fills the gap by developing an effective and computationally
optimal approach to synthesizing controllers for timeline-based games. The
proposed method addresses the limitations of previous techniques by directly
constructing a \emph{deterministic} finite-state automaton of an optimal
doubly-exponential size, that recognizes solution plans. Such an automaton can be
turned into the arena for a reachability game, for which many controller
synthesis techniques are available in the literature.

The paper is a significantly revised and extended version of~\cite{Acampora_2022}. 
It provides a detailed account of the general framework, gives some illustrative 
examples, and fully works out all the proofs. 

The rest of the paper is organized as follows. After discussing related work in 
\cref{sec:related}, \cref{sec:preliminaries} introduces timeline-based planning and games.
\Cref{sec:automaton} presents the main technical contribution of the paper,
namely, the construction of the deterministic automaton that recognizes solution
plans. \Cref{sec:games} shows how to turn such an automaton into the arena of a
suitable game from which the controller can be synthesized. \cref{sec:conclusions} 
summarizes the main contributions of the work and suggest future research directions. 
All the technical proofs are included in the appendix.

