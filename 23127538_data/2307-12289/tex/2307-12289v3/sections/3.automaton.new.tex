%!TeX root = ../lmcs-gandalf22.tex

%TODO controllare le definizioni
\section{A deterministic automaton for timeline-based planning}
\label{sec:automaton}

In this section, we define an encoding of timeline-based planning problems into \emph{deterministic} finite state automata (DFA).
Given a timeline-based planning problem, the corresponding automaton recognizes all and only those \emph{event sequences} that represent solution plans for the problem. In the next section, we will use such an automaton as the game arena for a timeline-based game.

\subsection{Plans as words}
Let $P=(\SV, S)$ be a timeline-based planning problem and, as already stated in the previous section, let $d = \max(L, U)+1$, where $L$ and $U$ are the maximum lower and (finite) upper bounds appearing in any rule of $P$. We restrict our attention to event sequences where the distance between two consecutive events is at most $d$. Such a restriction guarantees us the finiteness of the considered alphabet, and it does not cause any loss in generality, as proved by Lemma 4.8 of~\cite{Gigante19}. Moreover, it agrees with the notion of move of a timeline-based game (see \cref{def:games:moves}).

We define the symbols of the alphabet $\Sigma$ as \emph{events} of the form $\event = \pair{A,
\delta}$, where $A \subseteq \actions_\SV$ and $1\le\delta\le d$. Formally,
$\Sigma=2^{\actions_\SV}\times\ar{d}$, where $\ar{d}=\set{1,\ldots,d}$. Note that the size of $\Sigma$ is exponential in the size of the problem. Moreover,
we define $\window(P)$ as the sum of all the
coefficients appearing as upper bounds in the rules of $P$. This value represents the
maximum amount of time a rule can ``count'' far away from the
occurrence of the quantified
tokens. Consider, for instance, the following rule: 
\begin{align}
  \label{eq:example-3var-synch}
  a_0[x_0=v_0]\to{} &\exists a_1[x_1=v_1] a_2[x_2=v_2] a_3[x_3=v_3] \suchdot \\
  &\tokstart(a_1)\before_{4,14}\tokend(a_0)
  \land \tokend(a_0)\before_{0,+\infty}\tokend(a_2) \land \tokstart(a_2)\before_{0,3}\tokend(a_3)\tag*{}
\end{align}

In this case, assuming the above rule  to be the only one in the problem,
$\window(P)$ would be $3 + 14=17$. Thus, the rule can account for what happens
at most $17$ time points from the occurrence of its quantified tokens. For
instance, if the token $a_1$ appears at a specific distance from $a_0$, it has
to be within less than $17$ time points, and any modification of the plan that
alters this distance can break the rule's satisfaction. However, what occurs
further away from $a_0$ only affects the fulfillment of the rule
\emph{qualitatively}. Suppose that the tokens $a_2$ and $a_3$ are, together, at
$100$ time points from $a_0$. Changing this distance while maintaining the
qualitative order between tokens does not break the rule's satisfaction. For
$\window(P)$'s properties refer to \cite{Gigante19}.

\subsection{Matching structures}
A key insight underlying the construction we are going to outline is that every atomic temporal
relation $T \before_{l,u} T^\prime$ can be rewritten as the conjunction of two upper 
bound constraints $T^\prime - T \leq u$ and $T - T^\prime \leq -l$, where we
represent a lower bound constraint $T^\prime - T \geq l$ as an upper bound one.
This allows us to rewrite the clause \clause of an existential statement \E as a constraint system
$\nu(\clause)$ with constraints of the form $T - T^\prime \leq n$, for $n \in
\Z \cup \{+\infty\}$.

The  constraint system $\nu(\clause)$ can be represented by a difference bound matrix $D$ indexed by terms, where the entry $D[T, T']$ gives the upper bound $n$ on $T - T^\prime$.
In building $D$, we ensure the right duration of tokens
by augmenting the system with constraints
of the kind $\tokstart(a_i)-\tokend(a_i)\leq-\dmin^{x_i=v_i}$ and
$\tokend(a_i)-\tokstart(a_i)\leq\dmax^{x_i=v_i}$, for any quantified token
$a_i[x_i=v_i]$ of $\E$.
As an example, the constraint system and the DBM for the above rule are the ones in \cref{fig:rule-constraint-system,fig:dbm}, respectively.

% Figure environment removed

% Figure environment removed

On top of DBMs, we define the concept of \emph{matching structure}, a data
structure that allows us to monitor and update the fulfillment of atomic
temporal relations among terms throughout the execution of the plan. More
precisely, it allows us to manipulate and reason about existential statements of
which only a portion of the requests has been satisfied by the word read so far,
while the rest is potentially satisfiable in the future.

\begin{defi}[Matching Structure]
  \label{def:matching-structure}
  Let $\E\equiv \exists a_1[x_1 = v_1] \dots a_m[x_m = v_m] \,.\, \clause$ be
  an existential statement of a synchronization rule $\Rule \equiv a_0[x_0 =
  v_0] \rightarrow \E_1 \lor \dots \lor \E_k$ over the set of state variables
  \SV. The \emph{matching structure} for $\E$ is a tuple $\M_{\E} = (V, D, M, t)$,
  where:
  \begin{itemize}
  \item $V$ is the set of terms $\tokstart(a)$ and $\tokend(a)$, for
    $a\in\set{a_0, \dots, a_m}$;
  \item %$D \in \Z_{+\infty}^{|V|^2}$ is a DBM indexed by terms of $V$, 
    $D$ is a DBM of size $|V| \times |V|$, indexed by terms of $V$, whose
    entries take value over $\Z \cup \{+\infty\}$, where
  \begin{align*}
      \begin{cases}
          D[T,T']=n &\quad if  \; T-T'\le n \ \in \nu(\clause), \\
          D[T,T']=0 &\quad if \; T=T', \\
          D[T,T']=+\infty &\quad otherwise;
      \end{cases}
  \end{align*}    
  \item $M \subseteq V$ and $0\le t \le \window(P)$.
  \end{itemize}
\end{defi}

The set $M$ contains the set of terms from $V$ correctly seen in the sequence so
far. We say these terms have been \emph{matched} by the matching structure. We
use $\overline{M} = V \setminus M$ to refer to terms yet to be matched. We say a
matching structure $\M$ to be \emph{closed} if $M = V$, \emph{initial} if $M =
\emptyset$, and \emph{active} if $\tokstart(a_0) \in M$ and it is not closed.
The component $t$ represents the time elapsed since matching $\tokstart(a_0)$.
As time progresses, we update a matching structure as follows.

In the DBMs of a matching structure, the bounds between any pair of terms $T$
and $T'$, with one in $M$ while the other not, are tightened by the elapsing of
time. When $T\in M$ and $T'\in\overline{M}$, $D[T,T']$ is a lower bound loosened
by adding the elapsed time $\delta$. When $T\in\overline{M}$ and $T'\in M$,
$D[T,T']$ is an upper bound tightened by subtracting $\delta$. Consider the DBM
in \Cref{fig:dbm} and the pair of terms $\tokstart(a_1)$ and $\tokend(a_0)$. We
have $D[\tokstart(a_1),\tokend(a_0)]=-4$, implying that
$\tokstart(a_1)-\tokend(a_0)\le -4$ must hold. Suppose that $\tokstart(a_1)\in
M$ (it has been matched), and that $\tokend(a_0)\in\overline{M}$ (it needs to be
matched). Now, in a time step, the entry in the DBM is incremented and updated
to $-4+1=-3$ reflecting the fact that we now have $3$ time steps left to match
$\tokend(a_0)$. A similar analysis leads us to the conclusion that the entry
$D[\tokend(a_0),\tokstart(a_1)]=14$ has to be decremented by $1$ and updated to
$14-1=13$. This intuition is formalized as follows.

\begin{defi}[Time shifting]
  \label{def:time-shift}
  Let $\delta > 0$ be a positive amount of time, and let $\M = (V, D, M, t)$ be a
  matching structure. The result of shifting $\M$ by $\delta$ time units,
  written $\M + \delta$, is a matching structure $\M^\prime = (V, D^\prime, M,
  t')$, where:
  \begin{itemize}
  \item for all $T, T' \in V$:
    \begin{equation*}
      D^\prime[T,T'] =
      \begin{cases}
        D[T,T'] + \delta &\text{if } T \in M \text{ and } T' \in
        \overline{M}\\% D[Tj] è lower bound
        D[T,T'] - \delta &\text{if } T \in \overline{M} \text{ and } T' \in
        M\\% D[ij] è upper bound
        D[T,T'] &\text{otherwise}
      \end{cases}
    \end{equation*}
  \item and
    \[
      t' =
      \begin{cases}
        t+\delta & \text{if } \M \text{ is \emph{active}}\\
        t & \text{otherwise}
      \end{cases}
    \]
  \end{itemize}
\end{defi}
\Cref{def:time-shift} specifies how to update the entries of $D$ and how to update $t$ to the trigger occurrence of an active matching structure.

\begin{defi}[Matching]
  \label{def:matching}
  Let $\M = (V, D, M, t)$ be a matching structure and $I \subseteq \overline{M}$
  a set of matched terms. A matching structure $\M^\prime = (V, D, M^\prime, t)$
  is the result of matching the set $I$, written $\M \cup I$, with $M^\prime = M
  \cup I$.
\end{defi}

To correctly match an existential statement while reading an event sequence, a matching structure is updated only as long as one witnesses no violation of temporal constraints. As such, we deem an event as \emph{admissible} or not.

\begin{defi}[Admissible Event]\label{def:admissible-event}
  An event $\event = (A, \delta)$ is \emph{admissible} for a matching structure
  $\M_{\E} = (V, D, M, t)$ if and only if, for every $T \in M$
  and $T' \in \overline{M}$, $\delta \leq D[T',T]$, \ie the elapsing of $\delta$
  time units does not exceed the upper bound of some term $T'$ not yet
matched by $\M_{\E}$.
\end{defi}

Each admissible event $\event$ that is read can be matched with a subset of terms from the matching structure. However, there can be multiple ways to match events and terms. To make this choice explicit, we introduce the following definition.

\begin{defi}[$I$-match Event]\label{def:match-event}% I for indexes
  Let $\M_{\E} = (V, D, M, t)$ be a matching structure and  $I \subseteq
  \overline{M}$. An $I$\emph{-match event} is an admissible event $\event = (A,
  \delta)$ for $\M_{\E}$ such that:
  \begin{enumerate}
  \item for all token names $a \in \mathsf{N}$ quantified as $a[x = v]$ in $\E$
    we have that:\label{def:match-event:good-match}
    \begin{enumerate}
    \item if $\tokstart(a) \in I$, then $\tokstart(x, v) \in A$;
      \label{def:match-event:good-match:start}
    \item $\tokend(a) \in I$ if and only if $\tokstart(a) \in M$ and $\tokend(x,v) \in
      A$;\label{def:match-event:good-match:end}
    \end{enumerate}
  \item and for all $T \in I$ it holds that:\label{def:match-event:relations}
    \begin{enumerate}
    \item \label{def:match-event:preceding-terms} for every other term $T' \in
      V$, if $D[T',T] \leq 0$, then $T' \in M \cup I$;
    \item \label{def:match-event:lower-bounds} for all $T' \in M$, $\delta \geq
      -D[T',T]$, \ie all the lower bounds on $T$ are satisfied;
    \item \label{def:match-event:zero-no-bounds} for each other term $T' \in I$,
      either $D[T',T] = 0$, $D[T,T'] = 0$, or $D[T',T] = D[T, T'] = +\infty$.
    \end{enumerate}
  \end{enumerate}
\end{defi}

We consider an event $\event$ an $I$-match event if its actions correspond to the terms in $I$. The definition in \Cref{def:match-event:good-match} ensures the correct matching of each term to an action it represents and that the endpoints of a quantified token precisely identify the endpoints of a token in the event sequence. Meanwhile, \Cref{def:match-event:relations} guarantees that matching the terms in $I$ does not violate any atomic temporal relation. In addition, \Cref{def:match-event:preceding-terms} deals with the qualitative aspect of a ``happens before'' relation, while \Cref{def:match-event:lower-bounds,def:match-event:zero-no-bounds} address the quantitative aspects of the lower bounds of these relations. It is worth noting that an $\emptyset$-event is also considered admissible.

Let $\matchstructs_P$ denote the set of all matching structures for a planning
problem $P$, and let $\I$ be the set of all possible terms built from token
names in $\toknames$. To describe the evolution of a matching structure, we
define a quaternary relation
$S\subseteq\matchstructs_P\times\Sigma\times\I\times\matchstructs_P$ as
$(\M,\event,I,\M')\in{S}$, for an event $\event = (A, \delta)$, if and only if
$\event$ is an $I$-match event for $\M$, and $\M'=(\M+\delta)\cup I$. We also
write $\M \stepm \M'$ in place of $(\M,\event,I,\M')\in{S}$.
Note that, from \Cref{def:match-event}, a single event can represent multiple
$I$-match events for a matching structure. Therefore, given a matching structure
$\M$ and an event $\event$, automaton states will collect all the matching
structures $\M'$ resulting from the relation $S$, for some set of terms $I$.
Given a set of matching structures $\Upsilon$, this notion is best described by
the function $\step_\event(\Upsilon)=\set{\M' \mid (\M,\event,I,\M')\in S,
  \text{ for some } \M\in\Upsilon \text{ and } I \in \I}$. Furthermore, we
define $\Upsilon^\Rule_t\subseteq\Upsilon$ as the set of all the \emph{active}
matching structures $\M\in\Upsilon$ with timestamp $t$, associated with any
existential statement of $\Rule$. Matching structures in $\Upsilon^\Rule_t$
contribute to fulfilling the same triggering event of $\Rule$, regardless of their
existential statement. We also define $\Upsilon_\bot\subseteq\Upsilon$ as the
set of \emph{non-active} matching structures of $\Upsilon$. Lastly, we say that
$\Upsilon$ is \emph{closed} if there exists $\M\in\Upsilon$ such that $\M$ is
\emph{closed}.

% Figure environment removed

We conclude this section by providing an example of updating a matching structure $\M=(V,D,M,t)$ for the rule discussed at the beginning of the section. Consider the set of timelines in \cref{fig:timelines-0-1-2-3}. Before matching any term $\M$ is initial with $M = \emptyset$, $t = 0$, $D$ as the DBM in \cref{fig:dbm}, and $V$ as the set of term $\tokstart(a)$ and $\tokend(a)$ for $a \in \{a_0, a_1, a_2, a_3\}$.
We begin by matching the terms $\tokstart(a_0)$ and $\tokstart(a_3)$ from the event $\event = (\{\tokstart(x_0, v_0),\tokstart(x_3, v_3)\}, 0)$ (we do not consider $\tokstart(x_1,v'_1)$ and $\tokstart(x_2,v'_2)$ since they are not in $V$). Such event is an $I$-match event for $I = \{ \tokstart(a_0), \tokstart(a_3)\}$: it is an admissible event (\cref{def:admissible-event}), \cref{def:match-event:good-match:start} holds, for both 
$\tokstart(a_0)$ and $\tokstart(a_3)$, there are no terms that should appear before them (\cref{def:match-event:preceding-terms}), there are no related lower bounds (\cref{def:match-event:lower-bounds}), and $D[\tokstart(a_0), \tokstart(a_3)] = D[\tokstart(a_3), \tokstart(a_0)] = +\infty$ (\cref{def:match-event:zero-no-bounds}). Hence, we update $M = M \cup I = \{\tokstart(a_0), \tokstart(a_3)\}$ and $t = t + \delta = 0$; now $\M$ is active. The next term to consider is $\tokstart(a_2)$, which occurs after $\delta = 5$ time steps.

First, we ensure that the event $\event = (\tokstart(x_2, v_2), 5)$ is admissible. We show that by examining the DBM in \cref{fig:dbm}, we see that the elapsing of time $\delta$ does not exceed any upper bound related to terms $T \in M$ and $T' \in \overline{M}$. Next, the set $I$ in the current state appears as $I = \{ \tokstart(a_2) \}$. Notice that we are in the case of \cref{def:match-event:good-match:start}, and \cref{def:match-event:relations} holds because no constraint involves the term $\tokstart(a_2)$ (\cref{def:match-event:preceding-terms}), no lower bounds are related to $\tokstart(a_2)$ (\cref{def:match-event:lower-bounds}), and $\tokstart(a_2)$ is the only term in $I$ (\cref{def:match-event:zero-no-bounds}). Therefore, from \cref{def:matching,def:time-shift}, we update $\M$ as follows: $\M = (\M + \delta) \cup I$. Each entry of the DBM will remain unchanged since the third update case of \cref{def:time-shift} applies, $M = M \cup I = \{\tokstart(a_0), \tokstart(a_3), \tokstart(a_2)\}$, and $t = t + \delta = 5$.

Similarly, for the next event is $\event = (\tokstart(x_1, v_1), 1)$, we check if such an event is admissible, and indeed it is since the upper bound  $D[\tokend(a_0), \tokstart(a_1)] = 9 \ge \delta$. It is also an $I$-match event for $I = \{\tokstart(a_1)\}$, since it respects \cref{def:match-event:good-match:start} and all the relations in \cref{def:match-event:relations}; thus we update $\M$. We decrement $D[\tokend(a_3), \tokstart(a_2)]$ and increment $D[\tokstart(a_2), \tokend(a_3)]$ by 1 (see \cref{def:time-shift}), update $M$ like follows $M = \{\tokstart(a_0), \tokstart(a_3), \tokstart(a_2), \tokstart(a_1)\}$, and $t = t + \delta = 5 + 1 = 6$.

The next event is $\event = (\tokend(x_3, v_3), 3)$ after 2 time steps. Note that it is an admissible event and also an $I$-match event for $I = \{\tokend(a_3)\}$. In this case, we emphasize that \cref{def:match-event:good-match:end,def:match-event:relations} are respected. We update the DBM as follows: $D[\tokend(a_0), \tokstart(a_1)] = 14 - 2 = 12$, $D[\tokstart(a_1), \tokend(a_0)] = -4 + 2 = -2$, $D[\tokend(a_3), \tokstart(a_2)] = 2 - 2 = 0$, $D[\tokstart(a_2), \tokend(a_3)] = 1 + 2 = 3$. Then, we update $M = M \cup I = \{\tokstart(a_0), \tokstart(a_3), \tokstart(a_2),\\ \tokstart(a_1), \tokend(a_3)\}$ and $t = t + \delta = 6 + 2 = 8$. Notice that if we did not match $\tokend(a_3)$ now, at the next time step, the timeline would have violated the rule above because the upper bound $D[\tokend(a_3),\tokstart(a_2)] = 0$.

The subsequent event is $\event = (\tokend(x_1, v_1), \tokstart(x_1 = v'_1), 6)$ for which $I = \tokend(a_1)$. Since there is no constraint involving $\tokend(a_1)$, this event is admissible and an $I$-match event. The DBM is shifted by 6 time steps, and $M = \{\tokstart(a_0), \tokstart(a_3), \tokstart(a_2), \tokstart(a_1), \tokend(a_1)\}$.

The last event $\event = (\{\tokstart(x_0, v'_0), \tokstart(x_2, v^{''}_2), \tokend(x_0, v_0), \tokend(x_2, v_2)\}, 2)$ is admissible and an $I$-match for $I = \{\tokend(a_0), \tokend(a_2)\}$, note that there is not an upper bound between $\tokend(a_0)$ and $\tokend(a_2)$ and that \cref{def:match-event:good-match:end,def:match-event:relations} of the definition of $I$-match event are respected.

\subsection{Building the automaton}\label{sec:automata-construction}
We can now define the automaton. First, given an existential statement $\E$, let $\mathbb{E}_\E$ be the set of all existential statements in the same rule of $\E$. Next, let $\mathbb{F}_P$ be the set of functions that map each existential statement of $P$ to a set of existential statements and let $\mathbb{D}_P$ be the set of functions that map each existential statement to a set of matching structures $\Upsilon$.
%TODO rivedere questa definizione o magari definire l'automa TV_P
An automaton $\TV_P$ that checks the transition functions of the variables is easy to define. Then, given a timeline-based planning problem $P=(\SV, S)$, we can characterize the corresponding automaton as $A_P=\TV_P\cap\S_P$. Here, $\S_P$ checks the fulfillment of the synchronization rules, and we define it as $\S_P = (Q, \Sigma, q_0, F, \tau)$ where
\begin{enumerate} %todo considerare se cambiarlo in itemize
\item $Q = 2^{\matchstructs_P} \times \mathbb{D}_P \times \F_P \cup \set{\bot}$ is the
  finite set of states, \ie states are tuples of the form $\langle \Upsilon,
  \Delta, \Phi \rangle\in2^{\matchstructs_P} \times \mathbb{D}_P \times \F_P$, plus a
  sink state $\bot$;
\item $\Sigma$ is the input alphabet defined above;
  % in $\mathcal{A}_P \times [K]$,
  % where $K = \operatorname{window}(P)$;
\item the initial state $q_0 = \langle \Upsilon_0, \Delta_0, \Phi_0 \rangle$ is
  such that $\Upsilon_0$ is the set of initial matching structures of the
  existential statements of $P$ and, for all existential statements $\E$ of $P$,
  we have $\Delta_0(\E) = \emptyset$ and $\Phi_0(\E) = \mathbb{E}_\E$;
  % $\Upsilon = \Set{\M_{\E} | \E \in S \text{ and} \M_{\E}\text{ is
  % \emph{initial}}}$ and, for all
  % $\E \in S$, $\Delta(\E) = \emptyset$ and $\Phi(\E)
  % = \emptyset$;
\item $F \subseteq Q$ is the set of final states defined as:
  \[
    F = \Set{ \langle \Upsilon, \Delta, \Phi \rangle \in Q |
      \begin{gathered}
        \M \text{ is not \emph{active} for all } \M \in
        \Upsilon\\
        \text{and }\Delta(\E)=\emptyset\text{ for all }\E\text{ of } P
      \end{gathered}}
  \]
\item $\tau : Q \times \Sigma \rightarrow Q$ is the transition function that
  given a state $q=\langle \Upsilon, \Delta, \Phi \rangle$ and a symbol $\event
  = (A, \delta)$ computes the new state $\tau(q,\event)$. Let
  $\step^\E_\event(\Upsilon^\Rule_t)=\set{\M_\E \mid
  \M_\E\in\step_\event(\Upsilon^\Rule_t)}$. Moreover, let $\Psi^\Rule_t = \set{ \E |
  \M_{\E} \in \step_\event(\Upsilon^\Rule_t)}$. Then, the updated components of
  the state are based on what follows, where $W = \window(P)$:
  \begin{align*}
    \Upsilon' &= \step_\event(\Upsilon_\bot) \cup \bigcup \Set{
      \step_\event(\Upsilon^\Rule_t) |
      \text{$t\le W-\delta$ and 
      $\step_\event(\Upsilon^\Rule_t)$ is not \emph{closed}}} \\
    \Delta'(\E) &=\begin{cases}
        \step^\E_\event(\Upsilon^\Rule_t) & \text{where $t$ is the minimum such that $t> W-\delta$ and $\step^\E_\event(\Upsilon^\Rule_t)\ne\emptyset$} \\
        \step_\event(\Delta(\E)) & \text{if such $t$ does not exist}
      \end{cases}\\
    \Phi'(\E) &= \begin{cases}
      \mathbb{E}_\E\quad\text{if $\E\in\Psi(\E')$ for some $\E'$ such that 
      $\Delta'(\E')$ is \emph{closed}}  \\
      \Phi(\E) \setminus 
        \set{
          \E'\mid \exists t> W-\delta \suchdot \E'\in\Psi^\Rule_t 
          \land \E\not\in\Psi^\Rule_t
        } \quad \text{otherwise}
    \end{cases}
  \end{align*}

  Let $\Delta''(\E)=\Delta'(\E)$ unless there is an $\E'$ with $\E\in\Phi'(\E')$
  such that $\Delta'(\E')$ is \emph{closed}, in which case
  $\Delta''(\E)=\emptyset$. Then, $\tau(q,\event)=\seq{\Upsilon', \Delta'',
  \Phi'}$ if the following holds:
  \begin{enumerate}
  \item for every $\Upsilon^\Rule_t$, $\step_\event(\Upsilon^\Rule_t) \neq
    \emptyset$, and \label{dfa:delta:no-failed-step}
  \item for every synchronization rule $\Rule \equiv a_0[x_0=v_0] \rightarrow
    \E_1 \lor \dots \lor \E_n$ in $S$, if $\tokstart(x_0, v_0) \in A$, then
    there exists $\M_{\E_i} = (V,D,M,0) \in \Upsilon'$, 
    with $i \in \{1\dots n\}$, such that $\tokstart(a_0) \in M$;\label{dfa:delta:trigger-capture}
  \end{enumerate}
  Otherwise, $\tau(q,\event)=\bot$.
\end{enumerate}

The first component $\Upsilon$ of an automaton's state $q$ is a set of matching structures that keeps track of the occurred events in the last $\window(P)$ time points. The timestamp $t$ of any matching structure in $\Upsilon$ satisfies $t<\window(P)$. These matching structures evolve using the $\step_\event$ function until they become closed or their timestamp reaches $\window(P)$. 

Matching structures that reach $\window(P)$ get promoted to a new role where they record the pieces of existential statements not yet matched to satisfy all the trigger events of $\Rule$ that occurred before the last $\window(P)$ time points. However, the automaton does not store these matching structures in $\Upsilon$. Instead, it uses the function $\Delta$ mapping each existential statement $\E$ of a rule $\Rule$ to the set of matching structures for $\E$ with $t=\window(P)$. Thus, effectively summarizing events happening before this window to keep size under control.

When a set $\Upsilon^\Rule_t$ exceeds the bound $\window(P)$, the $\Delta$ function needs to be updated by merging the information from $\Upsilon^\Rule_t$ with the information already stored in $\Delta$. However, closing a set $\Delta(\E)$ does not necessarily mean that every event that triggered $\Rule$ satisfies $\Rule$. This is because there may be other sets, say $\Delta(\E')$, responsible for fulfilling the same rule $\Rule$, but for different trigger events. Therefore, closing $\Delta(\E)$ alone does not imply that $\Rule$ has been satisfied. Conversely, there may be cases where $\Delta(\E)$ and $\Delta(\E')$ contribute to match the same trigger events, and closing either set is enough to satisfy $\Rule$.

To address the issue of lost information when adding a set of matching structures to $\Delta$, we introduce the $\Phi$ function, mapping existential statements to sets of existential statements, as the third component of the automaton states. For an existential statement $\E$ and for every existential statement $\E' \in \Phi(\E)$, it holds that the set of matching structures $\Delta(\E')$ tracks the satisfaction of the same trigger events as the set $\Delta(\E)$. This way, when a set $\Delta(\E)$ is closed, we can discard its matching structures as well as the matching structures in $\Delta(\E')$.

In \cref{sec:soundness:completeness} we state and prove soundness and
completeness of the automaton construction. Now, instead, let us address the
size of the automaton.

Let us recall that we assumed that the timestamp of each event in an event sequence is bounded. However, it is worth noting that since events may have an empty set of actions, \cref{thm:soundness-completeness} can handle arbitrary event sequences as well, provided that we add suitable empty events. Let us now analyze the size of the automaton. 

\begin{thm}[Size of the automaton]
  Let $P=(\SV, S)$ be a timeline-based planning problem and let $\A_P$ be the
  associated automaton. Then, the size of $A_P$ is at most doubly-exponential in
  the size of $P$.
\end{thm}

\begin{proof}
We define $E$ as the overall number of existential statements in $P$, which is linear in the size of $P$. We can then observe that $\abs{\mathbb{D}_P} \in \O({(2^{\abs{\matchstructs_P}})}^E)= \O(2^{E\cdot\abs{\matchstructs_P}})$, thus the number of $\Delta$ functions is doubly exponential in the size of $P$.
Next, note that $\lvert\mathbb{F}_P\rvert \in \mathcal{O}({(2^E)}^E) = \mathcal{O}(2^{E^2})$. Then, $\abs{\S_P} \in \O(\abs{\Sigma}\cdot 2^{\abs{\matchstructs_P}})$ indicating that the size of $\S_P$ is at most exponential in the number of possible matching structures.
To bound this number, we define $N$ as the largest finite constant appearing in $P$ in any atom or value duration and $L$ as the length of the longest existential prefix of an existential statement occurring inside a rule of $P$. Note that $N$ is exponential in the size of $P$ since constants are expressed in binary, while $L \in \O(\abs{P})$.
We can then observe that the entries of a DBM for $P$, of which the number is quadratic in $L$, are constrained to take values within the interval $\ar{-N, N}$ (excluding the value $+\infty$), which size is linear in $N$. By \Cref{def:matching-structure}, it follows that $\abs{\matchstructs_P} \in \O(N^{L^2} \cdot 2^L \cdot \window(P))$ indicating that the number of matching structures is at most exponential in the size of $P$.
\end{proof}
Note that our automaton is the same size as the automaton built by Della Monica et al. in \cite{DellaMonicaGMS18}. However, while their automaton is nondeterministic, ours is deterministic: an essential property to achieve the \EXPTIME[2] optimal asymptotic complexity for the synthesis procedure.

\subsection{Soundness and Completeness}
\label{sec:soundness:completeness}
In the following, we present auxiliary notation, definitions, and essential lemmas for establishing the soundness and completeness of the automaton construction. For readability, we have included proofs in the appendix.

\begin{defi}[Run of a matching structure]
  Let $\evseq=\seq{\event_1,\ldots,\event_n}$ be a (possibly open) event
  sequence, and let $\M_\E$ be the initial matching structure of an existential
  statement $\E$. A \emph{run} of $\M_\E$ on $\evseq$ yielding a matching
  structure $\M_n$ is a sequence $\matchseq = \seq{\match_1, \ldots, \match_n}$
  of $I$-match events for the matching structures $\seq{\M_\E, \M_1, \ldots,
    \M_{n-1}}$, such that for every $i \in [1,\ldots,n]$, $\M_{i-1} \stepm[i]
  \M_i$. We write $\M_\E \runm \M_n$ when such run exists, or $\M_\E
  \xlongrightarrow{\evseq} \M_n$, if $\matchseq$ is not relevant.
\end{defi}

To link matching structures with the semantics of synchronization rules we
establish a connection between matching functions (\cref{def:matching-function})
and runs.

\begin{restatable}[Correspondence between runs and matching functions]{lem}{runfuncmap}
  \label{lemma:function-matching-run}
  Let $\evseq=\seq{\event_1,\ldots,\event_n}$ be a (possibly open) event
  sequence, and let $\M_\E$ be the initial matching structure of an existential
  statement $\E\equiv \exists a_1[x_1=v_1]\ldots a_k[x_k=v_k]\suchdot\clause$,
  with $\clause$ augmented with atoms $\tokstart(a_i) \before_{\dmin^{x_i=v_i},
    \dmax^{x_i=v_i}} \tokend(a_i)$, for every $0\leq i \leq k$. Then, there
  exists a run $\matchseq=\seq{\match_1, \ldots, \match_n}$ of $\M_\E$ on
  $\evseq$, yielding a matching structure $\M_n = \tuple{V, D_n, M_n, t_n}$, if
  and only if there exists a matching function $\gamma:M_n \to[1,\ldots,n]$ such
  that, for every atom of the form $T\before_{l,u} T'$ in $\clause$:
  \begin{enumerate}[label=(\Roman*)]
  \item \label{lemma:function-matching-run:entire-atom} if $T' \in M_n$, then also $T
    \in M_n$, $\gamma(T) \le \gamma(T')$, and $l \le
    \delta(\slice\evseq_{\gamma(T),\gamma(T')}) \le u$;
  \item \label{lemma:function-matching-run:partial-atom} if $T' \not\in M_n$, but $T \in
    M_n$, then $\delta(\slice\evseq_{\gamma(T),n}) \le u$.
  \end{enumerate}
  Furthermore, $\gamma$ and $\matchseq$ are such that for every $T \in M_n$, $T
  \in I_{\gamma(T)}$, \ie, they agree on the matching of the terms of $\M_n$. We
  write $M_\E \runm* M_n$, if $\gamma$ corresponds to a run of $\M_\E$, or
  $\evseq,\gamma\models \M_n$, if $\M_\E$ is clear from the context.
\end{restatable}

\begin{observation}
    \label{obs:matching-functions}
    Note that the existence of the matching function $\gamma$ stated by 
    \cref{lemma:function-matching-run}, if the corresponding matching structure is 
    closed, implies the satisfaction of the given existential statement, and 
    \viceversa.
\end{observation}

We now state the core technical result of the completeness proof, which ensures
no important details are lost when matching structures are discarded.

\begin{restatable}{lem}{superset}\label{lemma:matching-structure-superset}
  Let $\evseq =\langle\event_1,\dots,\event_n\rangle$ be an event sequence , let
  $\M_\E$ be the initial matching structure of some existential statement $\E$
  of a rule $\Rule$, and let $\M_r$ be an active matching structure resulting
  from a run $\M_\E \runm*[r] \M_r$, such that $\gamma_r(\tokstart(a_0)) = r$.
  If there exists a run $\M_\E \runm*[s] \M_s$, such that
  $\gamma_s(\tokstart(a_0)) < r$, then there exists a run $\M_\E \runm* \M$,
  such that $\gamma(\tokstart(a_0)) = \gamma_s(\tokstart(a_0))$ and $\M$ matches
  at least as many tokens as $\M_r$.
\end{restatable}

The last needed notion is that of \emph{residual} matching structure, which is
an active matching structure with only infinite bounds.

\begin{defi}[Residual matching structure]\label{def:residual-matching-structure}
  A matching structure $\M = (V, D, M, t)$ is \emph{residual} if it is
  \emph{active} and for every $T \in M$ and $T' \in \overline{M}$, $D[T',T] = +\infty$.
\end{defi}

In other words, $\M$ does not impose any finite upper bound on the distance at
which terms yet to be matched may appear relative to those already matched. The
definition implies that for any residual matching structure, denoted as $\hat\M
= (V, D, M, t)$, every event $\event = (A, \delta)$ is admissible. Additionally,
it is never the case that $\tokstart(a) \in M$ and $\tokend(a) \in \overline{M}$
for any quantified token $a[x = v]$ of $\E$, given that such terms always have a
finite upper bound in $D$ that is at least as strict as the value $\dmax^{x=v}$.
As a result, the ``if'' direction of \Cref{def:match-event:good-match:end} in
the \Cref{def:match-event} of $I$-match never applies to $\hat\M$ for any event
$\event$. Therefore, every event is a valid $\emptyset$-match event for
$\hat\M$.

\begin{observation}\label{obs:residual-run}
  Let $\M_\E \xlongrightarrow{\evseq_1,\matchseq_1}\hat\M$ be a run of the
  \emph{initial} matching structure $\M_\E$, on an event sequence $\evseq_1$,
  yielding a \emph{residual} matching structure $\hat\M$. Then, for any event
  sequence $\evseq_2$, there exists a run $\M_\E
  \xlongrightarrow{\evseq_1\evseq_2,\matchseq_1\matchseq_2} \hat\M'$ such that
  every $I$-match event in $\matchseq_2$ is an $\emptyset$-match event and
  $\hat\M'$ differs from $\hat\M$ by at most the value of the component $t$.
\end{observation}

Consequently, whenever a residual matching structure appears in a run, it has
the potential to remain there indefinitely, which is why it is called
\emph{residual}.

\begin{restatable}[Existence of residual matching structure]{lem}{residualexist}
  \label{lemma:residual-matching-structure}
  Let $\evseq = \seq{\event_1, \ldots, \event_n}$ be an event sequence, and let
  $\M_n$ be an \emph{active} matching structure such that $\evseq, \gamma
  \models \M_n$ and
  $\delta(\slice\evseq_{\gamma(\tokstart(a_0)),n})>\window(P)$. If we consider
  the intermediate matching structures $\seq{\M_1, \ldots, \M_{n-1}}$ of the run
  $\M_\E \runm* \M_n$, then there exists a position $\gamma(\tokstart(a_0)) \leq
  k < n$ such that $\M_k$ is a \emph{residual} matching structure.
\end{restatable}

We are now ready to prove the final result.

\begin{restatable}[Soundness and completeness]{thm}{soundnessCompleteness}
  \label{thm:soundness-completeness}
  Let $P=(\SV, S)$ be a timeline-based planning problem and let $\A_P$ be the
  associated automaton. Then, any event sequence $\evseq$ is a solution plan for
  $P$ if and only if $\evseq$ is accepted by $\A_P$.
\end{restatable}

%%% Local Variables:
%%% TeX-master: "../lmcs-gandalf22.tex"
%%% End:
