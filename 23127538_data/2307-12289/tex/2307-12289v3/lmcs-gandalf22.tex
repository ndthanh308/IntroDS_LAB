
\documentclass{lmcs}

\keywords{Planning, automata, synthesis}

\usepackage{todonotes}
\usepackage{comment}
\usepackage{amsmath,amssymb,amsthm,thmtools,thm-restate}
\usepackage[all]{foreign}
\usepackage{multirow}         % multi-row cells in tables
\usepackage{bigdelim}         % delimiters spanning multiple rows in tables
\usepackage{extarrows}        % \xlongrightarrow
\usepackage{braket}           % \Set
\usepackage{lipsum}
\usepackage{hyperref}
\usepackage[capitalise]{cleveref}
\usepackage{etoolbox}
\usepackage{booktabs}

\usepackage{tikz}
\usepackage{xcolor}
\usepackage{packages/graphics}
\usepackage{packages/timelines}
\usepackage{packages/commands}
\usepackage{packages/symbols}
\definecolor{primary}{HTML}{87cefa}

\Crefname{thm}{Theorem}{Theorems}
\Crefname{lem}{Lemma}{Lemmas}
\Crefname{defi}{Definition}{Definitions}

\newtheorem{observation}{Observation}
\Crefname{observation}{Observation}{Observations}


\begin{document}

\title{Controller Synthesis for Timeline-based Games\rsuper*}
\titlecomment{{\lsuper*}A preliminary version of this article has appeared in the proceedings of GandALF 2022~\cite{Acampora_2022}.}
%\thanks{thanks, optional.} %optional

\author[R.~Acampora]{Renato Acampora}[a]
\author[L.~Geatti]{Luca Geatti\lmcsorcid{0000-0002-7125-787X}}[a]
\author[N.~Gigante]{Nicola Gigante\lmcsorcid{0000-0002-2254-4821}}[b]
\author[A.~Montanari]{Angelo Montanari\lmcsorcid{0000-0002-4322-769X}}[a]
\author[V.~Picotti]{Valentino Picotti\lmcsorcid{0000-0001-7713-1461}}[c]

\address{University of Udine, Italy}
\email{%
  acampora.renato@spes.uniud.it, luca.geatti@uniud.it, 
  angelo.montanari@uniud.it%
}

\address{Free University of Bozen-Bolzano, Italy}
\email{nicola.gigante@unibz.it}

\address{University of Southern Denmark}
\email{picotti@imada.sdu.dk}

\begin{abstract}
In the timeline-based approach to planning, the evolution over time of a set of
state variables (the timelines) is governed by a set of temporal constraints.
Traditional timeline-based planning systems excel at the integration of planning
with execution by handling \emph{temporal uncertainty}. In order to handle
general nondeterminism as well, the concept of \emph{timeline-based games} has
been recently introduced. It has been proved that finding whether a winning
strategy exists for such games is \EXPTIME[2]-complete. However, a concrete
approach to synthesize controllers implementing such strategies is missing. This
article fills the gap by providing an effective and computationally optimal
approach to controller synthesis for timeline-based games.
\end{abstract}

\maketitle

% ######################################################
% Cycling
% ######################################################
To promote sustainability, cities worldwide are promoting a transition to public transportation and active transportation. From these, cycling has proven to provide numerous advantages, including benefits to health \cite{gotschi2016cycling}, economy \cite{clifton2013examining}, and reduction of carbon emissions \cite{NEVES2019130}. Despite these benefits, cycling numbers remain predominantly low in some cities. In contrast, barriers to cycling include hilliness, lack of cycling infrastructure, or appropriate bike storage or parking. Yet, the main deterrent to cycling relates to safety concerns \cite{aldred2015investigating, lawson2013perception, felix2019maturing}. If cyclists feel unsafe or are afraid to cycle, they will prefer other means of transportation. 


% ######################################################
% Perception of Safety
% ######################################################
Thus, for cities aiming to boost cycling numbers and the effectiveness of such strategies, it is increasingly important to understand what affects individuals' perceptions. Perception of cycling safety research explores how individuals subjectively experience cycling accident risk and what fears and events negatively impact one's perception of being involved in a cycling accident. Current research shows that infrastructure layout, fear of traffic, and distracted cycling are some aspects that influence this perception \cite{heinen2010commuting}. Most research focuses on surveys and in-loco and post-riding interviews to compare factors influencing perceptions \cite{sanders2015perceived}. Even though these approaches are vital to understanding cycling perception of safety, they need to be more scalable over space or time due to their high cost (human resources, time, and money). This prevents any analysis of perceptions over time, and qualitative non-scalable data analysis hampers any comparative study across cities or countries.


% !BIB TS-program =
\documentclass[12pt]{article}
\usepackage{color}
\usepackage{amsfonts,amssymb,amsmath}
\usepackage[export]{adjustbox}
\makeatletter
\setlength{\@fptop}{0pt}
 \makeatother
\usepackage{graphicx}
\usepackage[T1]{fontenc}
\usepackage[numbers,sort&compress]{natbib}
\graphicspath{ {./images/} } \textheight 9in \textwidth  6.5in
\topmargin -1cm \oddsidemargin -0.1in \evensidemargin -0.1in
\marginparwidth 17.57mm
%\renewcommand{\baselinestretch}{1.55}
\newcounter{tempeq}
\begin{document}
\title{\textsf{Enhancing the performance of an open quantum battery by adjusting its velocity}}
\author{B. Mojaveri\thanks{Email: bmojaveri@azaruniv.ac.ir; bmojaveri@gmail.com},
\hspace{2mm}R. Jafarzadeh Bahrbeig\thanks{Email:
r.jafarzadeh86@gmail.com},\hspace{2mm}M. A. Fasihi
\thanks{Email: ma-fasihi@azarunic.ac.ir}, and S. Babanzadeh\thanks{Email: s.babanzadeh@azaruniv.ac.ir}\\
{\small {Department of Physics, Azarbaijan Shahid Madani University,
PO Box 51745-406, Tabriz, Iran \,}}} \maketitle
\begin{abstract}
The performance of open quantum batteries (QBs) is severely limited
by decoherence due to the interaction with the surrounding
environment. So, protecting the charging processes against
decoherence is of great importance for realizing QBs. In this work
we address this issue by developing a charging process of a
qubit-based open QB composed of a qubit-battery and a qubit-charger,
where each qubit moves inside an independent cavity reservoir. Our
results show that, in both the Markovian and non-Markovian dynamics,
the charging characteristics, including the charging energy,
efficiency and ergotropy, regularly increase with increasing the
speed of charger and battery qubits. Interestingly, when the charger
and battery move with higher velocities, the initial energy of the
charger is completely transferred to the battery in the Markovian
dynamics. In this situation, it is possible to extract the total
stored energy as work for a long time. Our findings show that open
moving-qubit systems are robust and reliable QBs, thus making them a
promising candidate for experimental implementations.\\\\
{\bf Keywords:} Open quantum batter, Markovian and non-Markovian
charging process, Ergotropy, Atomic motion.
\end{abstract}
\section{introduction}
In recent years, with advancements in quantum thermodynamics, there
has been a radical change of perspective in the framework of energy
manipulation based on the electrochemical principles. The
possibility to create an alternative and efficient energy storage
device at small scale introduces the concept of the quantum battery
(QB), which was proposed by Alicki and Fennes in the 2013's
\cite{Alicki}, and  subsequently became into a significant field of
research. As their name indicates, QBs are finite dimensional
quantum systems that are able to temporarily store energy in their
quantum degrees of freedom for later use. The fundamental strategy
for developing the idea of QBs is based on their non-classical
features such as quantum coherence, entanglement and many-body
collective behaviors that can be cleverly exploited to achieve more
efficient and faster charging processes than the macroscopic
counterpart \cite{Strasberg, Vinjanampathy, Goold, Campisi,
Gelbwaser, Horodecki}. A QB is charged based on an interaction
protocol between QB itself with either an external field or a
quantum system which serves as a charger. It is then discharged into
a consumption hub based on the same protocol. When the battery
enters into an interaction with the charger, it transitions from a
lower energy level into the higher ones and will be charged. So far,
a variety of powerful charging protocols have been proposed in
different platforms, including two-level systems \cite{Farin, Zhang,
Fus}, harmonic oscillators \cite{Cata}, and hybrid light-matter
systems \cite{Maze, Manzo, Cond}. Some proposals have been also
devoted to implement QBs based on the two-level systems such as
trapped ions \cite{Forn, Lv}, cold atoms \cite{Bau} and
superconducting qubits \cite{Devoret}.

 Due to the fact that a real quantum system inevitably interacts with
its environment, studying QBs from the open quantum systems
perspective is attracting considerable interest. The interaction of
a QB with its surrounding environments causes the leakage of the
coherence of battery to the environment, leading to decoherence
effect in the battery. Such an adverse effect often plays a negative
role in the charging and discharging performance of QBs \cite{Camp,
Farin1, Carega}. Decoherence brought during the charging process
tends to lead QBs to a non-active (passive) equilibrium state in
which work extracting from the QBs is often impossible \cite{Barra}
in a cyclic unitary process. The environmental-induced noises also
affect QBs that are disconnected from both charger and consumption
hub and cause self-discharging of that QBs \cite{San0, Pedro,
Salimi}. Therefore, designing a more robust battery against the
environmental dissipations is valuable step for implementation of
QBs in the real-life. Recently, researchers have devoted efforts not
only to studying the effect of the environment on QBs, but also to
exploit non-classical effect as well as to developing open system
protocols to stabilize the charging cycle performance through
quantum control techniques. For example, Kamin et al \cite{Kamin1}
studied the charging performance of a qubit-based QB charged by the
mediation of a non-Markovian environment. They revealed the
non-Markovian property is beneficial for improving charging cycle
performance. In Ref. \cite{Squeezing}, the authors studied dynamics
of a continuous variable QB coupled weakly to the squeezed thermal
reservoir and managed to control the performance of the charging
process by boosting the quantum squeezing of reservoir. A feasible
route for harnessing loss-free dark states for stabilizing the
stored energy of a qubit-based open QB has been introduced in
\cite{Dark}. In addition to the above considerations, several other
protocols have been developed to protect the charging cycle of QBs
such as feedback control method \cite{Mitch, Shao, Ios}, convergent
iterative algorithm \cite{Borhan}, Bang-Bang modulation of the
intensity of an external Hamiltonian \cite{Franc}, inhiring an
auxiliary quantum system \cite{Behzadi}, modulating the detuning
between system and reservoir \cite{Yu0}, stimulated Raman adiabatic
passage technique \cite{Baris}, engineering quantum environments
\cite{Segal}, etc.

 On the other hand, according to the previous studies on
the Markovian and non-Markovian dynamics of open two-qubit systems,
translational motion of qubits provides novel insights for
stabilizing qubit-qubit entanglement against the environmental
induced dissipations by suitably adjusting the velocities of the
qubits \cite{Epjp0, morteza0, Chao0, sare0, Golkar1, Epjp1, MPLA,
Wang00}. We want here to use this safeguard capability of the
motional properties to improve the charging cycle performance of the
open qubit-based QBs. For this end, we consider a moving-biparticle
system composed of a qubit-battery and a qubit-charger that
independently interacts with their local environments. The battery
qubit here is charged with the help of the dipole-dipole interaction
with the charger qubit. We will investigate how the translational
motion of qubits affects the charging process of QB. Our results
show that translational motion of qubits always plays a constructive
role in protecting QB from decay induced by the environment. This
work is organized as follows: in Sec. 2, we introduce and describe
several figures of merit for characterizing the performance of QBs.
In Sec. 3, we illustrate our model and obtain explicit expressions
for the reduced density matrix of the QB and the charger. In Sec. 4
we present the results of our numerical simulations in the context
of their physical significance. Finally, Sec. 5 concludes this
paper.
\section{Figures of Merit}
Let us consider a QB modeled as a quantum system with d-dimensional
Hilbert space $\mathcal{H}$ and Hamiltonian $H_B$ such that
\renewcommand\theequation{\arabic{tempeq}\alph{equation}}
\setcounter{equation}{-1}
\addtocounter{tempeq}{1}\begin{eqnarray}\label{Bat}
H_B=\sum_{i=1}^{d} \varepsilon_i
|\varepsilon_i\rangle\langle\varepsilon_i|,
\end{eqnarray}
with non-degenerate energy levels $\varepsilon_i \leq
\varepsilon_{i+1}$. Internal energy of QB is given by $Tr(\rho_B
H_B)$, where $\rho_B$ is the state of the battery. Charging a QB
means brings the quantum system from a lower energy state $\rho_B$
to a higher energy state $\rho_B^\prime$, while discharging refers
to the inverse process, i.e., brings the quantum system from a
higher energy state $\rho_B^\prime$ to a lower one
$\rho_B^{\prime\prime}$:
\renewcommand\theequation{\arabic{tempeq}\alph{equation}}
\setcounter{equation}{-1}
\addtocounter{tempeq}{1}\begin{eqnarray}\label{den}
\texttt{Tr}\left\{\left(\rho_B^\prime-\rho_B\right) H_B\right\}\geq0,\qquad\qquad\qquad\qquad charging \nonumber \\
\texttt{Tr}\left\{\left(\rho_B^{\prime\prime}-\rho_B^\prime\right)
H_B\right\}\geq0.\qquad\qquad\qquad\quad \;\;discharging
\end{eqnarray}
Therefore, in a charging process, the actual stored energy of QB at
time $t$, regarding the initial energy, can be expressed as follows
\cite{Alicki}
\renewcommand\theequation{\arabic{tempeq}\alph{equation}}
\setcounter{equation}{-1} \addtocounter{tempeq}{1}\begin{equation}
\Delta E_B=\texttt{Tr}\{\rho_B(t) H_B\}-\texttt{Tr}\{\rho_B(0)
H_B\}.
\end{equation}
A complete converting the stored energy into valuable work is
impossible without dissipation of heat according to the second law
of thermodynamics. The maximum amount of energy extracted from a
given quantum state $\rho_B=\sum_{i} r_i |r_i\rangle\langle r_i|$,
($ r_i \geq r_{i+1}$) through a cyclic unitary operation is called
ergotropy \cite{Allahverdyan}. This quantity can be defined as
\cite{Allahverdyan, Franc0, Cakmak0}
\renewcommand\theequation{\arabic{tempeq}\alph{equation}}
\setcounter{equation}{-1}
\addtocounter{tempeq}{1}\begin{equation}\label{ergo}
\mathcal{W}=\texttt{Tr}\{\rho_B
H_B\}-\texttt{min}_U\,\texttt{Tr}\{U\rho_B U^{\dagger} H_B\},
\end{equation}
where the minimization is taken over all possible unitary
transformations acting locally on such system. It has been shown in
\cite{Allahverdyan} that no work can be extracted from the passive
counterpart of $\rho_B$ with the form $\sigma_{\rho_B}=\sum_{i} r_i
|\varepsilon_i\rangle\langle\varepsilon_i|$. The unique unitary
transformation $U=\sum_i |\varepsilon_i\rangle\langle r_i|$ on the
$\rho$ minimizes $\texttt{Tr}(U\rho_B U^{\dagger} H_B)$, and when
inserted in Eq. (\ref{ergo}) yields the following expression for the
ergotropy
\renewcommand\theequation{\arabic{tempeq}\alph{equation}}
\setcounter{equation}{-1} \addtocounter{tempeq}{1}\begin{equation}
\mathcal{W}=\sum_{i,j} r_j \varepsilon_i\left(|\langle
r_j|\varepsilon_i\rangle|^2-\delta_{ij}\right).
\end{equation}
In order to quantify the amount of extractable energy, the
efficiency $\eta$ is defined as the ratio between the ergotropy
$\mathcal{W}$ and the total charging energy $\Delta E_B$
\renewcommand\theequation{\arabic{tempeq}\alph{equation}}
\setcounter{equation}{-1} \addtocounter{tempeq}{1}\begin{equation}
\eta=\frac{\mathcal{W}}{\Delta E_B}.
\end{equation}% Figure environment removed
\section{Open Moving-Quantum Battery}
The open QB under consideration is composed of an atomic two-qubit
system, the qubit $A$ as a charger and the qubit $B$ as a quantum
battery, coupled to each other trough the dipole-dipole interaction.
The battery and charger qubits coupled locally to two independent
zero-temperature cavity reservoirs (see Fig. 1). We assume that each
qubit moves along the $z$-axis of its cavity at a constant
non-relativistic speed $v$. For simplicity we neglect here any
scattering \cite{Engl} or trapping \cite{Haro} effects and consider
the translational motion of the atom qubits being classically. Under
the dipole and rotating wave approximation, the entire system is
ruled by Hamiltonian (setting $\hbar=1$)
\renewcommand\theequation{\arabic{tempeq}\alph{equation}}
\setcounter{equation}{-1} \addtocounter{tempeq}{1}\begin{equation}
H=H_0+H_{int},
\end{equation}
with
\renewcommand\theequation{\arabic{tempeq}\alph{equation}}
\setcounter{equation}{-1}
\addtocounter{tempeq}{1}\begin{eqnarray}\label{Ham}
&&\hspace{-1.15cm}
H_0=H_A+H_B+H_{R_A}+H_{R_B}=\sum_{j=A,B}\left(\frac{\omega_0}{2}
\sigma_{z}^{j} + \sum_{k}\omega_{k}^j a_{k}^{j\dag} a_{k}^j\right),\nonumber\\
&&\hspace{-1.2cm}H_{int}=H_{A-B}+H_{A-R_A}+H_{B-R_B}=D\left(\sigma_{+}^{A}\sigma_{-}^{B}+\sigma_{-}^{A}
\sigma_{+}^{B}\right) +\sum_{j=A,B}\sum_{k} f_k^j(z)
\left(\mathfrak{g}_{k}^j \sigma_{+}^{j} a^j_k +H.c.\right).
\end{eqnarray}
Here, H.c. stands for Hermitian conjugate, $\sigma_z^j$,
$\sigma_+^j$, and $\sigma_-^j$ $(j=A,B)$ are, respectively, the
population inversion, raising and lowering operators of the $j$th
qubit with transition frequency $\omega_0$. $a_k^{j\dagger}$ and
$a^j_k$ are, respectively, the creation and annihilation operators
of the $k$th mode of the cavity reservoir $j$ with the frequency
$\omega_k^j$. Also, $D$ is coupling constant of the dipole-dipole
interaction between the battery and charger qubits, and
$\mathfrak{g}_{k}^j$ is the coupling constant between the $j$th
qubit and $k$th mode of in the cavity reservoir $j$. The effect of
translation motion of the battery and charger qubits has been
included in the model by introducing the $z$-dependent shape
function $f_k^j(z)$ in the Hamiltonian $H_{int}$. When the battery
and charger qubits are moving with same constant velocity $v$, the
shape function $f_k^j(z=vt)$ can be taken into account as
\renewcommand\theequation{\arabic{tempeq}\alph{equation}}
\setcounter{equation}{-1} \addtocounter{tempeq}{1}\begin{equation}
f_k^j(z)=\sin[\omega_k^j(\beta t-\Gamma)],\qquad\qquad j=A,B
\end{equation}
where, $\Gamma=L/c$ with $L$ being the size of the cavity. Also,
$\beta=v/c$ where $c$ refers to the speed of light in the vacuum
space. This particular form of the shape function can be obtained by
imposing an appropriate boundary condition on the cavity reservoirs
\cite{Lenard, morteza0}. Here we describe the translational motion
of both battery and charger qubits by classical mechanics ($z=vt$).
To this end, we will choose the values of the parameters in such a
way that the de Broglie wavelength of qubit $\lambda_B$ is
significantly smaller than the wavelength $\lambda_0$ associated
with the resonant transition $\omega_0=\omega_n$ ($\omega_n$ is the
central frequency of the cavity field mode) \cite{mortezapour,
Cook}. Furthermore, we consider a situation in which the photon
momentum is relatively small than the atomic momentum and thus we
neglect the atomic recoil caused by the interaction with the
electric field \cite{Wilkens}. In the optical regime, to ignore the
atomic recoil and consider the translational motion of atoms as
classical, the velocity of qubits should be $v\gg 10^{-3}$
\cite{morteza0}.

In the interaction picture (IP) generated by the unitary
transformation $U=e^{-iH_0t}$, the Hamiltonian (\ref{Ham}) can be
written as follows
\renewcommand\theequation{\arabic{tempeq}\alph{equation}}
\setcounter{equation}{-1}
\addtocounter{tempeq}{1}\begin{eqnarray}\label{HIP}
&&\hspace{-1.5cm}H_{IP}=D\left(\sigma_{+}^{A}
\sigma_{-}^{B}+\sigma_{-}^{A} \sigma_{+}^{B}\right)+
\sum_{j=A,B}\sum_{k} f_k^j(z)\left(\mathfrak{g}_{k}^j \sigma_{+}^{j}
a_k^{j} e^{i(\omega_0-\omega_k^j)t}+\mathfrak{g}_k^{j \ast}
\sigma_{-}^{j}a_{k}^{j\dag} e^{-i(\omega_0-\omega_k^j)t}\right).
\end{eqnarray}
It is straightforward to show that the total excitation operator
$\hat{\mathcal{N}}=\sum_{j=A,B}\left(\sum_k\hat{a_k}^{j\dagger}\hat{a_k}^j+
\frac{1}{2}\hat{\sigma}_{z}^{j}\right)+1$, commutes with the total
Hamiltonian, i.e. $[H,\hat{\mathcal{N}}]=0$ and therefor it is the
constant of the motion. This allows us to decompose Hilbert space of
the entire qubit-cavity system,
$\mathcal{H}=\mathcal{H}_q\otimes\mathcal{H}_R$ spanned by the basis
$\{\left|i_A,j_B\right\rangle\otimes\left|n_1,n_2, ...,n_k,
...\right\rangle_{R_A}|_{n_1,n_2,...=0}^{\infty}
\otimes\left|m_1,m_2, ...,m_k,
...\right\rangle_{R_B}|_{m_1,m_2,...=0}^{\infty}\}$
$\left(i,j=e,g\right)$ into the excitation subspaces, as follows
\renewcommand\theequation{\arabic{tempeq}\alph{equation}}
\setcounter{equation}{-1} \addtocounter{tempeq}{1}
\begin{eqnarray}
&&\hspace{-14mm} \mathcal{H}=\oplus_{n=0}^{\infty} \mathcal{H}_{n}.
\end{eqnarray}
As a result of this decomposition, the dynamics of the entire
qubit-reservoir system can be restricted to the excitation subspaces
labeled by the total excitation number $n$. Here we are interested
to explore dynamics of the entire system in the single-excitation
subspace $\mathcal{H}_1$ spanned by vectors
$\{\left|g_A,g_B\right\rangle\otimes\left|1_k\right\rangle_{R_A}\left|0_k\right\rangle_{R_B}|_{k=0}^\infty,
\left|g_A,g_B\right\rangle\otimes\left|0_k\right\rangle_{R_A}\left|1_k\right\rangle_{R_B}|_{k=0}^\infty,
\left|e_A,g_B\right\rangle\otimes\left|0_k\right\rangle_{R_A}\left|0_k\right\rangle_{R_B},
\left|g_A,e_B\right\rangle\otimes\left|0_k\right\rangle_{R_A}\left|0_k\right\rangle_{R_B}\}$
in which the single excitation is either in one of the qubits or in
the k-th mode of one of cavity reservoirs. We consider a normalized
initial state of entire qubit-reservoir as a superposition of
$\left|e_A,g_B\right\rangle\left|0_k\right\rangle_{R_A}\left|0_k\right\rangle_{R_B}$
and
$\left|g_A,e_B\right\rangle\left|0_k\right\rangle_{R_A}\left|0_k\right\rangle_{R_B}$
states with the following form
\renewcommand\theequation{\arabic{tempeq}\alph{equation}}
\setcounter{equation}{-1}
\addtocounter{tempeq}{1}\begin{eqnarray}\label{sai0}
|\Psi(0)\rangle=\big[c_1(0) |e_{A},g_{B}\rangle +c_2(0)
|g_{A},e_{B}\rangle\big]\otimes |0\rangle_{R_A}|0\rangle_{R_B}.
\end{eqnarray}
For times $t>0$, we expand the state vector $|\Psi(t)\rangle$ in
terms of the vector basis of the single-excitation subspace
$\mathcal{H}_1$ as
\renewcommand\theequation{\arabic{tempeq}\alph{equation}}
\setcounter{equation}{-1}
\addtocounter{tempeq}{1}{\footnotesize\begin{eqnarray}\label{sai}
&&\hspace{-3.5cm}\left|\Psi(t)\right\rangle=\big[c_1(t)\left |e_{A},
g_B\right\rangle +c_2(t) \left|g_A, e_B\right\rangle\big] \otimes
\left|0_k\right\rangle_{R_A}\left|0_k\right\rangle_{R_B}
\nonumber\\
&&\hspace{-2.35cm}+\left|g_A, g_B\right\rangle\otimes\sum_{k}
\big[d_{k}(t)\left|1_k\right\rangle_{R_A}\left|0_k\right\rangle_{R_B}+d_{k}^{\prime}(t)
\left|0_k\right\rangle_{R_A}\left|1_k\right\rangle_{R_B}\big],
\end{eqnarray}}
where the time-dependent amplitudes satisfy the normalization
requirement
\renewcommand\theequation{\arabic{tempeq}\alph{equation}}
\setcounter{equation}{-1} \addtocounter{tempeq}{1}\begin{eqnarray}
\sum_{i=1}^2|c_i(t)|^2+\sum_k(|d_{k}(t)|^2+|d_{k}^{\prime}(t)|^2)=1.
\end{eqnarray}
By taking the partial traces over the field modes and subsystem A
(B), the reduced time-dependent density operator for the battery
(charger) in the $\{\left|e\right\rangle, \left|g\right\rangle\}$
basis is obtained as
\renewcommand\theequation{\arabic{tempeq}\alph{equation}}
\setcounter{equation}{0} \addtocounter{tempeq}{1}\begin{eqnarray}
&&\hspace{-2cm}\rho_A(t)=|c_1(t)|^2\left|e_A\right\rangle\left\langle
e_A\right|-\left(1-|c_1(t)|^2\right)\left|g_A\right\rangle\left\langle
g_A\right|\label{rob2},\\
&&\hspace{-2cm}\rho_B(t)=|c_2(t)|^2\left|e_B\right\rangle\left\langle
e_B\right|-\left(1-|c_2(t)|^2\right)\left|g_B\right\rangle\left\langle
g_B\right|\label{rob1}.
\end{eqnarray}

 Inserting Eq. (\ref{sai}) into the time dependent Schr\"{o}dinger
equation $H_{IP}|\Psi(t)\rangle=i\frac{d}{d t}|\Psi(t)\rangle$, with
$H_{IP}$ given in (\ref{HIP}), leads to the following set of
differential equations for time-dependent amplitudes
\renewcommand\theequation{\arabic{tempeq}\alph{equation}}
\setcounter{equation}{0} \addtocounter{tempeq}{1}\begin{eqnarray}
&&\hspace{-4cm}i\dot{c_1}(t)=D c_2(t)+\sum_{k} \mathfrak{g}_{k}^A
f_k^A(z)d_{k}(t)e^{i(\omega_0-\omega_{k}^A)}\label{c1t},\\
&&\hspace{-4cm}i\dot{c_2}(t)= D c_1(t)+ \sum_{k} \mathfrak{g}_{k}^B
f_k^B(z)d_{k}^{\prime}(t)e^{i(\omega_0-\omega_{k}^B)}\label{c2t},\\
&&\hspace{-4cm}i\dot{d}_{k}(t)=\mathfrak{g}_k^{A\ast}f_k^A(z)
c_1(t)e^{-i(\omega_0-\omega_{k}^A)t},\label{d1t}\\
&&\hspace{-4cm}i\dot{d}_{k}^{\prime}(t)=
\mathfrak{g}_k^{B\ast}f_k^B(z)
c_2(t)e^{-i(\omega_0-\omega_{k}^B)t}\label{d2t}.
\end{eqnarray}
By integrating Eqs. (\ref{d1t}) and (\ref{d2t}) with the initial
condition $d_{k}(0)=0$ and $d_{k}^{\prime}(0)=0$ and putting their
solutions, respectively, in Eqs. (\ref{c1t}) and (\ref{c2t}), we get
the following integro-differential equations for the amplitudes
$c_1(t)$ and $c_2(t)$
\renewcommand\theequation{\arabic{tempeq}\alph{equation}}
\setcounter{equation}{0} \addtocounter{tempeq}{1}\begin{eqnarray}
&&\hspace{-2cm}\dot{c_1}(t)=-iDc_2(t)+\int_{0}^{t}F_A(t-t^\prime)c_1(t^\prime)dt^\prime,\label{mt}\\
&&\hspace{-2cm}\dot{c_2}(t)=-iDc_1(t)+\int_{0}^{t}F_B(t-t^\prime)c_2(t^\prime)dt^\prime,\label{nt}
\end{eqnarray}
where
\renewcommand\theequation{\arabic{tempeq}\alph{equation}}
\setcounter{equation}{0} \addtocounter{tempeq}{1}\begin{eqnarray}
&&\hspace{-2cm}F_{A}(t-t^\prime)=\sum_{k} |\mathfrak{g}_{k}^A|^2
e^{i(\omega_0-\omega_{k}^A)(t-t^\prime)}\sin[\omega_k^A(\beta^A
t-\Gamma)]\sin[\omega_k^A(\beta^A t^\prime-\Gamma)],\\
&&\hspace{-2cm}F_{B}(t-t^\prime)=\sum_{k} |\mathfrak{g}_{k}^B| ^2
e^{i(\omega_0-\omega_{k}^B)(t-t^\prime)}\sin[\omega_k^B(\beta^B
t-\Gamma)]\sin[\omega_k^B(\beta^B t^\prime-\Gamma)],
\end{eqnarray}
are the memory correlation function of the reservoirs $A$ and $B$,
respectively. For simplicity, we suppose
$F_{A}(t-t^\prime)=F_{B}(t-t^\prime)=F(t-t^\prime)$. In the limit of
a large number of modes ( in the continuum limit ), the correlation
function $F(t-t^\prime)$ takes the following form
\renewcommand\theequation{\arabic{tempeq}\alph{equation}}
\setcounter{equation}{-1}
\addtocounter{tempeq}{1}\begin{equation}\label{kernel}
F(t-t^\prime)=\int d\omega J(\omega)
e^{i(\omega_0-\omega)(t-t^\prime)}\sin[\omega(\beta
t-\Gamma)]\sin[\omega(\beta t^\prime-\Gamma)],
\end{equation}
in which $J(\omega)$ is the spectral density of the cavity
reservoirs and has the Lorentzian form \cite{Lenard, Breuer0}
\renewcommand\theequation{\arabic{tempeq}\alph{equation}}
\setcounter{equation}{-1}
\addtocounter{tempeq}{1}\begin{equation}\label{lorentz}
J(\omega)=\frac{1}{2\pi}\frac{\gamma\lambda^2}{(\omega_0-\omega-\Delta)^2+\lambda^2},
\end{equation}
where $\lambda$ defines the spectral width of the coupling which is
connected to the memory time $\tau_E$ by the relation
$\tau_E=\lambda^{-1}$ and $\gamma$ refers to the qubit-environment
coupling strength which is related to the relaxation time scale
$\tau_R$ by $\tau_R \approx \gamma^{-1}$. Also $\Delta$ is the
detuning of $\omega_0$ and the central frequency of the cavity. The
weak and strong coupling regimes can be distinguished by comparing
$\tau_E$ and  $\tau_R$, in other words with an increasing
$\frac{\tau_E}{\tau_R}=\frac{\gamma}{\lambda}$ ratio, the
interaction will transition into a strong coupling or a non-Markovian regime \cite{Breuer0}.\\
By inserting the Eq. (\ref{lorentz}) into the Eq. (\ref{kernel}) and
after some calculations, in the continuum limit ($\Gamma \rightarrow
\infty$), the correlation function is simplified as
\renewcommand\theequation{\arabic{tempeq}\alph{equation}}
\setcounter{equation}{-1}
\addtocounter{tempeq}{1}\begin{equation}\label{ft}
F(t-t^\prime)=\frac{\gamma \lambda}{4} \cosh[\beta
\overline{\lambda}(t-t^\prime)] e^{-(\lambda-i\Delta) |t-t^\prime|}
\end{equation}
with $\overline{\lambda}=\lambda+i(\omega_0-\Delta)$.\\
In view of (\ref{ft}), taking the Laplace transformations of both
sides of the differential Eqs. (\ref{mt}) and (\ref{nt}) and using
the convolution property
$\mathcal{L}[\int_{0}^{t}\mathbf{A}(t-t^\prime) \mathbf{B}(t^\prime)
dt^\prime]=\mathbf{A}(s)\mathbf{B}(s)$ yields
\renewcommand\theequation{\arabic{tempeq}\alph{equation}}
\setcounter{equation}{0} \addtocounter{tempeq}{1}\begin{eqnarray}
&&\hspace{-2cm}sc_1(s)-c_1(0)=-iDc_2(s)-F(s)c_1(s),\label{ms}\\
&&\hspace{-2cm}sc_2(s)-c_2(0)=-iDc_1(s)-F(s)c_2(s),\label{ns}
\end{eqnarray}
where the functions $c_1(s)$ and $c_2(s)$ are the Laplace
transformations of the $c_1(t)$ and $c_2(t)$, respectively, and
$F(s)$ is the Laplace transforms of $F(t-t^\prime)$ which has the
following explicit form
\renewcommand\theequation{\arabic{tempeq}\alph{equation}}
\setcounter{equation}{-1} \addtocounter{tempeq}{1}\begin{eqnarray}
F(s)=\frac{\gamma\lambda}{4}\frac{s+\overline{\lambda}}{(s+\overline{\lambda})^2-\beta^2\overline{\lambda}\,^2}.
\end{eqnarray}
By reformulating the Eqs. (\ref{ms}) and (\ref{ns}), we get a
general solution for $c_1(s)$ and $c_2(s)$ as follows
\renewcommand\theequation{\arabic{tempeq}\alph{equation}}
\setcounter{equation}{0} \addtocounter{tempeq}{1}\begin{eqnarray}
&&\hspace{-2cm}c_1(s)=\frac{s+F(s)}{\big(s+F(s)\big)^2+D^2}c_1(0)-i\frac{D}{(s+F(s))^2+D^2}c_2(0),\\
&&\hspace{-2cm}c_2(s)=\frac{s+F(s)}{\big(s+F(s)\big)^2+D^2}c_2(0)-i\frac{D}{(s+F(s))^2+D^2}c_1(0).
\end{eqnarray}
In continuation, by applying the inverse Laplace transformation on
the both side of the above equations, we obtain finally $c_1(t)$ and
$c_2(t)$, as
\renewcommand\theequation{\arabic{tempeq}\alph{equation}}
\setcounter{equation}{0} \addtocounter{tempeq}{1}\begin{eqnarray}
&&\hspace{-2cm}c_1(t)=\frac{1}{2}\bigg(c_1(0)\Re(\mathcal{M}(t))-ic_2(0)\Im(\mathcal{M}(t))\bigg)\label{ct12},\\
&&\hspace{-2cm}c_2(t)=\frac{1}{2}\bigg(c_2(0)\Re(\mathcal{M}(t))-ic_1(0)\Im(\mathcal{M}(t))\bigg)\label{ct122},
\end{eqnarray}
where, $\Re(x)$ ($\Im(x)$) is real (imaginary) part of $x$, and
\renewcommand\theequation{\arabic{tempeq}\alph{equation}}
\setcounter{equation}{-1} \addtocounter{tempeq}{1}\begin{equation}
\mathcal{M}(t)=\sum_{i,j,k=1}^3\varepsilon_{ijk}\frac{ e^{q_it}
(q_j-q_k)\bigg((q_i+\overline{\lambda})^2-\beta
^2\overline{\lambda}^2\bigg)}{\prod_{i=1}^{3}\prod_{j=i+1}^{3}(q_i-q_j)},
\end{equation}
with $\varepsilon_{ijk}$ is the Levi-Civita symbol and $q_i (i=  1,
2, 3)$ are the roots of
\renewcommand\theequation{\arabic{tempeq}\alph{equation}}
\setcounter{equation}{-1} \addtocounter{tempeq}{1}\begin{equation}
q^3+q^2(2 \overline{\lambda}-i \text{D} )+q \left(\frac{\gamma
\lambda }{4}+\overline{\lambda} (\overline{\lambda}-2 i
\text{D})-\beta ^2\overline{\lambda}^2\right)+\frac{\gamma  \lambda
\overline{\lambda}}{4}+i \text{D} \overline{\lambda}^2\left(\beta
^2-1\right)=0.
\end{equation}

 With substitution (\ref{ct12}) and (\ref{ct122}), respectively, into the reduced density matrices
(\ref{rob1}) and (\ref{rob2}), and then using the $\Delta
E_{A(B)}=\texttt{Tr}\{\rho_{A(B)}(t)
H_{A(B)}\}-\texttt{Tr}\{\rho_{A(B)}(0) H_{A(B)}\}$ , the internal
energy of the charger and battery are deduced as
\renewcommand\theequation{\arabic{tempeq}\alph{equation}}
\setcounter{equation}{-1} \addtocounter{tempeq}{1}\begin{equation}
\Delta
E_A=\omega_0\left(|c_1(t)|^2-|c_1(0)|^2\right),\quad\quad\Delta
E_B=\omega_0\left(|c_2(t)|^2-|c_2(0)|^2\right).
\end{equation}
On the other hand, one can obtain ergotropy of the battery by
substitution Eq. (\ref{rob1}) with Eq. (\ref{ergo}). So, we have
\renewcommand\theequation{\arabic{tempeq}\alph{equation}}
\setcounter{equation}{-1} \addtocounter{tempeq}{1}\begin{equation}
 W_B=\omega_0\left(2|c_2(t)|^2-1\right)\Theta
\left(|c_2(t)|^2-\frac{1}{2}\right),
\end{equation}
where $\Theta(x-x_0)$ is the Heaviside function, which satisfies
$\Theta(x-x_0)=0$ for $x<x_0$, $\Theta(x-x_0)=\frac{1}{2}$ for
$x=x_0$ and $\Theta(x-x_0)=1$ for $x>x_0$.
% Figure environment removed
% Figure environment removed
\section{Numerical Results and Discussion}
In this section, we will analyze the charging dynamics of the
introduced open moving-battery in the weak and strong coupling
regimes. In particular, we explore the role of the movement of QB on
the dynamical behavior of performance indicators including stored
energy, ergotropy and efficiency. In our following analysis, we
choose the optical regime parameters \cite{Hood, Pinkse} and
consider that qubit transition frequency as
$\omega_0=1.5\times10^{9}\lambda$. In what follows, we consider an
initial condition in which the battery is initially empty and the
charger has the maximum energy, i.e. $c_1(0)=0$, $c_2(0)=1$.
% Figure environment removed

 In Fig. 2, we plot the Markovian and non-Markovian dynamics of the stored energy $\Delta E_B$
for the initial state
$\left|\Psi(0)\right\rangle=\left|g\right\rangle_{A}\left|e\right\rangle_{B}\otimes
\left|0\right\rangle_{R_B}\left|0\right\rangle_{R_B}$, by
considering different values of the QB speed $\beta$. In panel (a),
the battery is charged in the Markovian dynamics with
$(\gamma=0.1\lambda)$, while in panel (b), it is charged in a
non-Markovian dynamics with $(\gamma=20\lambda)$. Here we consider a
situation at which the charger and battery's qubits are both in
resonance with the reservoir modes by setting $\Delta=0$. According
to this figure, the positive impact of the translational motion of
the charger and batter's qubits in controlling the stored energy of
battery is clearly visible in both Markovian and non-Markovian
charging processes. As can be seen in both Figs. 2(a) and (b), when
the charger and battery's qubits are at rest inside their cavity
reservoirs, the stored energy in the battery $\Delta E_B$ decays
into zero at sufficiently long times. However the rate of these
decays decreases regularly by gradual growth of the qubit velocity,
and therefore the energy stored in the battery and consequently the
charging process is strongly protected from the environmental
noises. Comparing Fig. 2(a) with Fig. 2(b) clearly reveals a
fundamental difference between Markovian and non-Markovian charging
processes. The maximal amount of stored energy in the Markovian
charging process is more than those of the non-Markovian charging
process. The reason stems from the nature of the qubit-cavity
coupling. In the non-Markovian charging process, the coupling
strength of charger's qubit to the cavity modes is greater than its
coupling to the battery's qubit, therefore, the initial internal
energy of charger has more tendency to evolve toward the reservoir
than to the battery. Moreover, since the motional effect of QB has
been included in battery-cavity and charger-cavity coupling
strength, it seems that increasing speed of QB decreases the
charger-cavity coupling strength in favor of to charger-battery
coupling strength, which increases the energy stored in the battery.

In order to get more insight to this area and a deeper understanding
of the relationship between the charger and battery energy, in Fig.
2 we have illustrated the energy stored in the battery at the end of
charging process as well as the energy that the charger loses at the
same time. Here $\Delta E_B$ and $|\Delta E_A|$ have been plotted as
a function of the dimensionless time $\lambda t$ for the qubit
velocities $\beta=0$ and $\beta=0.7\times 10^{-9}$ in the Markovian
and non-Markovian regimes. In the non-Markovian charging process,
$|\Delta E_A|$ is much more than $\Delta E_B$ for a given $\beta$ as
shown in Fig. 3(b). This implies that the internal energy of the
charger is not completely transferred to the battery. Fig. 3(b) also
shows that, when the charger and battery's qubits are at rest inside
their cavity reservoirs, the charger's qubit immediately loses a
large amount of its initial energy without being transferred to the
battery. However, increasing the qubit velocity (decreasing the
ratio of charger-cavity coupling strength to charger-battery
coupling strength) during the non-Markovian process, decreases the
initial loss-rate of the charger, and therefore improves the energy
transfer in the charging processes.

The relationship between the charger and battery energy in the
Markovian charging process is drastically different from that in the
non-Markovian charging process. One can infer from Fig. 3(a) that,
for the static battery-charger system ($\beta=0$), the total energy
of the charger can be transferred to the battery in the Markovian
short-charging process, where we have $|\Delta E_A|=\Delta E_B$.
Interestingly, when the qubits move with the velocity
$\beta=0.7\times10^{-9}$, $|\Delta E_A|=\Delta E_B$ holds at any
charging time. So, we conclude again that a robust Markovian
charging against the arisen dissipation can be achieved, when the
qubits move with higher velocities.
% Figure environment removed

 In the following, we examine the influence of translational motion
of the battery-charger system on the dynamics of ergotropy. In Fig.
4, we plot $W/W_{max}$ as a function of $\lambda t$ for the
different values of $\beta$ in the Markovian (Fig. 4(a)) and
non-Markovian (Fig. 4(b)) regimes. Our numerical results in Fig.
4(a) and (b) illustrate that, the effect of translational motion of
QB on the ergotropy is also constructive in both Markovian and
non-Markovian regimes. Fig. 4(b) shows that, in the non-Markovian
regime, in the cases of stationary ($\beta=0$) and slowly moving
($\beta=3\times10^{-9}$) qubits, we are not able to extract useful
work from the QB, but in this regime a considerable work can be
extracted, as the qubits move with a higher velocity
($\beta=0.8\times10^{-9}$). Our numerical results in Fig. 4(a)
illustrate that, the effect of translational motion of QB on the
ergotropy is more considerable in the Markovian case. We observe
that, in the Markovian regime, increasing the speed of QB $\beta$
(decreasing the qubit-reservoir coupling) not only boosts the
ergotropy, but also increases the number of time zones in which work
can be extracted. Accordingly, a strong robust charging process can
be established in the higher speed limit, in which the extractable
work approaches to its maximum value.

 Finally, we examine the effect of translational motion
of QB on the Markovian and non-Markovian charging efficiency. The
results for Markovian and non-Markovian charging processes are
presented in Fig. 5(a) and 5(b), respectively. Here we consider the
same parameter values as Fig. 4. Comparing Figs. 4 and 3 reveals
that both ergotropy and efficiency are positively affected by the
translational motion of QB. However the efficiency is influenced
more than the ergotropy; the amount of increment in efficiency is
more than the ergotropy in both Markovian and non-Markovian charging
processes.
\section{Outlook and summary}
To summarize, we proposed a mechanism for robust charging process of
an open qubit-based quantum battery (QB) whose robustness can be
well controlled by the translational motion of the charger and
battery in both Markovian and non-Markovian dynamical regimes. Both
the battery and charger's qubits move with a same speed inside two
separated identical environments, and are directly coupled by the
dipole-dipole interaction. We showed that the stored energy,
ergotropy and efficiency of the moving QB regularly increased with
the gradual growth of the charger and battery speed, thereby
improving its charging performance. The constructive role of the
translational movement of QB in controlling the charging process
arises from the attachment of qubits velocities to the
qubit-reservoir coupling strength (see Eq. (\ref{Ham})). According
to the adopted charging protocol, a weak qubit-reservoir coupling is
required for a strongly robust charging process which can be
fulfilled by adjusting $\beta$ to the higher velocities.

 Our results represent a further control strategy to have a robust QB with
a natural implementation in cavity-QED context. The strategy can be
easily implemented also in the circuit-QED setups where the qubit
position slowly varies linearly with time and also the qubit-cavity
interaction is tuned through a sinusoidal position-dependent
coupling \cite{Jones}.

  In perspective, we believe that this strategy can be used
to control the performance of the discharging of a qubit-based QB to
an available consumption hub. Further efforts in this field can be
devoted to use the proposed strategy for improving the performance
of the two-photon based charging process where the moving-QB is
coupled with a cavity reservoir by means of a two-photon
relaxation.\\\\
\textbf{\large{Data availability}}\\ The datasets used and analysed
during the current study available from the corresponding author on
reasonable request.
\begin{thebibliography}{99}
\bibitem{Alicki} R. Alicki and M. Fannes, Entanglement boost for extractable work from ensembles of quantum batteries, Phys. Rev. E 87, 042123 (2013).
\bibitem{Strasberg} P. Strasberg, G. Schaller, T. Brandes, and M. Esposito, Quantum and information thermodynamics: A unifying framework based on repeated interactions, Phys. Rev. X 7, 021003 (2016).
\bibitem{Vinjanampathy} S. Vinjanampathy and J. Anders, Quantum thermodynamics, Cont. Phy. 57, 545 (2016).
\bibitem{Goold} J. Goold, M. Huber, A. Riera, L. del Rio, and P. Skrzypczyk, The role of quantum information in thermodynamics: a topical review, J. Phys. A: Math. Theor. 49, 143001 (2016).
\bibitem{Campisi} M. Campisi, P. H\"{a}nggi, and P. Talkner, Colloquium: Quantum fluctuation relations: Foundations and applications, Rev. Mod. Phys. 83, 1653 (2011).
\bibitem{Gelbwaser} D. Gelbwaser-Klimovsky, W. Niedenzu and G. Kurizki, Thermodynamics of quantum systems under dynamical control, Adv. At. Mol. Opt. Phys., 64, 329 (2015).
\bibitem{Horodecki} M. Horodecki and J. Oppenheim,Fundamental limitations for quantum and nanoscale thermodynamics, Nature Comm. 4, 2059 (2013).
\bibitem{Farin} D. Farina, G. M. Andolina, A. Mari, M. Polini and V. Giovannetti, powerful charging of quantum batteries, Phys. Rev. B 99, 035421 (2019).
\bibitem{Zhang} Y-Y. Zhang, T-R. Yang, L. Fu and X. Wang, Powerful harmonic charging in a quantum battery, Phys. Rev. E 99, 052106 (2019).
\bibitem{Fus} L. Fusco, M. Paternostro, and G. D. Chiara, Work extraction and energy storage in the Dicke model, Phys. Rev. E 94, 052122 (2016).
\bibitem{Cata} R. R. Rodriguez, B. Ahmadi, P. Mazurek, S. Barzanjeh, R. Alicki and P. Horodecki, catalysis in charging quantum batteries, Phys. Rev. A 107, 042419 (2023).
\bibitem{Maze} J. Carrasco, J. R. Maze, C. Hermann-Avigliano and F. Barra, collective enhancement in dissipative quantum batteries, Phys. Rev. E. 105, 064119 (2022).
\bibitem{Manzo} M. Gumberidze, M. Kol\'{a}r and R. filip, Measurement induced Synthesis of coherent Quantum Batteries, Sci. Rep 9, 19628 (2019).
\bibitem{Cond} D. Ferraro, M. Campisi, G. M. Andolina, V. Pellegrini and M. Polini, High-power collective charging of a solid-state quantum battery, Phys. Rev. Lett. 120, 117702 (2018).
\bibitem{Forn} P. Forn-D\'{\i}laz, J. J. Garc\'{\i}la-Ripoll, B. Peropadre, J.-L. Orgiazzi, M. A. Yurtalan, R. Belyansky, C. M. Wilson, and A. Lupascu, Ultrastrong coupling of a single artificial atom to an electromagnetic continuum in the nonperturbative regime, Nat. Phys. 13, 39 (2016).
\bibitem{Lv} Bruzewicz, C.D.; Chiaverini, J.; McConnell, R.; Sage, J.M. Trapped-Ion Quantum Computing: Progress and Challenges. Appl. Phys. Rev. 2019, 6, 021314..
\bibitem{Bau} K. Baumann, C. Guerlin, F. Brennecke, and T. Esslinger, The dicke quantum phase transition with a superfluid gas in an optical cavity, Nature (London) 464, 1301 (2010)
\bibitem{Devoret} Devoret, M.H.; Schoelkopf, R. J. Superconducting Circuits for Quantum Information: An Outlook. Science 2013, 339, 1169
\bibitem{Farin1} D. Farina, G. M. Andolina, A. Mari, M. Polini, and V. Giovannetti, Charger-mediated energy transfer for quantum batteries: Anopen-system approach. Phys. Rev. B 99, 035421 (2019).
\bibitem{Camp} C. Ou, R. V. Chamberlin and S. Abe, Lindbladian operators, von Neumann entropy and energy conservation in time-dependent quantum open systems, Physica A 466, 450 (2017).
\bibitem{Carega} M. Carrega, A. Crescente, D. Ferraro, and M. Sassetti, Dissipative dynamics of an open quantum battery. New J. Phys. 22, 083085 (2020).
\bibitem{Barra} F. Barra, Dissipative charging of a quantum battery, Phys. Rev. Lett. 122, 210601 (2019).
\bibitem{San0} A. C. Santos, Quantum advantage of two-level batteries in
self-discharging process, Phys. Rev. E 103, 042118 (2021).
\bibitem{Pedro} L. P. Garcia-Pintos, A. Hamma, A. del Campo, Fluctuations in extractable work bound the charging power of quantum batteries. Phys. Rev. Lett. 125, 040601 (2020).
\bibitem{Salimi} F. H. Kamian, F. T. Tabesh, S. Salimi, F. Kheirandish, and A. C. Santos, Non-markovian effects on charging and selfdischarging processes of quantum batteries, New J. Phys. 22, 083007 (2020).
\bibitem{Kamin1} F. T. Tabesh, F. H. Kamin, and S. Salimi, Environmentmediated charging process of quantum batteries, Phys. Rev. A 102, 052223 (2020).
\bibitem{Squeezing} F. Centrone, L. Mancino, M. Paternostro, Charging batteries with quantum squeezing, https://doi.org/10.48550/arXiv.2106.07899.
\bibitem{Dark} J. Q. Quach and W. J. Munro, Using dark states to charge and stabilise open quantum batteries, Phys. Rev. Applied 14, 024092 (2020).
\bibitem{Mitch} M. T. Mitchison, J. Goold and J. Prior, Charging a quantum battery with linear feedback control, Quantum 5, 500 (2021).
\bibitem{Shao} Y. Yao and X. Q. Shao, Phys. Rev. E Optimal charging of open spin-chain quantum batteries via homodyne-based feedback control, 106, 014138 (2022).
\bibitem{Ios} S. Borisenok, Ergotropy of quantum battery controlled via target attractor feedback, J. Appl. Phys. 12, 43 (2020).
\bibitem {Borhan} R. R. Rodriguez, B. Ahmadi, G. Suarez, P. Mazurek, S. Barzanjeh, P. Horodecki, Optimal quantum control of charging quantum batteries, arXiv:2207.00094 [quant-ph].
\bibitem{Franc} F. Mazzoncini, V. Cavina, G. M. Andolina, P. A. Erdman and V. Giovannetti, Optimal control methods for quantum batteries, Phys. Rev. A 107 (2023) 032218.
\bibitem{Behzadi} N. Behzadi and H. Kassani, Mechanism of controlling robust and stable charging of open quantum batteries, J. Phys. A: Math. Theor. 55, 425303 (2022).
\bibitem{Yu0} J. L. Li, H. Z. Shen and X. X. Yi, Quantum batteries in non-Markovian reservoirs, Opt. Lett 21, 5614 (2022).
\bibitem{Baris} A. C. Santos, B. \c{C}akmak, S. Campbell and N.T. Zinner, Stable adiabatic quantum batteries, Phys. Rev. E 100, 032107 (2019).
\bibitem{Segal} J. Liu, D. Segal, Boosting quantum battery performance by structure engineering, arXiv:2104.06522 [quant-ph].
\bibitem{Epjp0} J. Taghipour, B. Mojaveri and A. Dehghani, Witnessing entanglement between two two-level atoms coupled to a leaky cavity via two-photon relaxation, Eur. Phys. J. Plus 137, 772 (2022).
\bibitem{morteza0} A. Mortezapour, M. A. Borji, and R. L. Franco, Protecting entanglement by adjusting the velocities of moving qubits inside non-Markovian environments, Laser Phys. Lett 14, 055201 (2017).
\bibitem{Chao0} W. Chao and F. Mao-Fa, The entanglement of two moving atoms interacting with a single-mode field via a three-photon process, Chin. Phys. B 19, 020309 (2010).
\bibitem{sare0} S. Golkar and M. K. Tavassoly and A. Nourmandipour, Entanglement dynamics of moving qubits in a common environment, J. Opt. Soc. Am. B 37, 400 (2020).
\bibitem{Golkar1} S. Golkar and M. K. Tavassoly And A. Nourmandipour, Qubit movement-assisted entanglement swapping, Chin. Phys. B. 29, 050304 (2020).
\bibitem{Epjp1} B. Mojaveri, A. Dehghani and J. Taghipour, Control of entanglement, single excited-state population and memory-assisted entropic uncertainty of two qubits moving in a cavity by using a classical driving field, Eur. Phys. J. Plus 137, 1065 (2022).
\bibitem{MPLA} J. Taghipour, B. Mojaveri and A. Dehghani, Witnessing entanglement between two two-level atoms moving inside a leaky cavity under classical control, Mod. Phys. Lett. A 37, 2250141 (2022).
\bibitem{Wang00} Q. Wang, R. Liu, H. M. Zou, D. Long and J. Wang, Entanglement dynamics of an open moving-biparticle system driven by classical-field, Phys. Scr. 97, 055101, (2022).
\bibitem{Allahverdyan} A. E. Allahverdyan, R. Balian and T. M. Nieuwenhuizen, Maximal work extraction from finite quantum systems. Eur. phys. Lett 67, 565 (2004).
\bibitem{Franc0} G. Francica, J. Goold, F. Plastina, and M. Paternostro, Daemonic ergotropy: enhanced work extraction from quantum correlations, npj Quantum Inf. 3, 12 (2017).
\bibitem{Cakmak0} B. \c{C}akmak, Ergotropy from coherences in an open quantum system, Phys. Rev. E 102, 042111 (2020).
\bibitem{Engl} B.G. Englert, J. Schwinger, A.O. Barut and M.O. Scully, Reflecting slow atoms from a micromaser field, Eur. Phys. Lett 14, 25 (1991).
\bibitem{Haro} S. Haroche, M. Brune and J.M. Raimond, Trapping atoms by the vacuum field in a cavity, Eur. Phys. Lett 14, 19 (1991).
\bibitem{Lenard} C. Leonardi and A. Vagliea, Non-markovian dynamics and spectrum of a moving atom strongly coupled to the field in a damped cavity, Opt. Commun 97, 130 (1993).
\bibitem{mortezapour} F. Nosrati, A. Mortezapour and R. Lo Franco, Validating and controlling quantum enhancement against noise by the motion of a qubit, Phys. Rev. A. 101, 012331 (2020).
\bibitem{Cook} R. J. Cook, Atomic motion in resonant radiation: An application of Ehrenfest's theorem, Phys. Rev. A. 20, 224 (1979).
\bibitem{Wilkens} M. Wilkens, Z. Bialynicka-Birula and P. Meystre, Spontaneous emission in a Fabry-P\'{e}rot cavity: The effects of atomic motion, Phys. Rev. A. 45, 477 (1992).
\bibitem{Breuer0} H. P. Breuer and F. Petruccione, \textit{The Theory of Open Quantum Systems} (Oxford University Press, Oxford, New York, 2002).
\bibitem{Hood} C. J. Hood et al., The Atom-Cavity Microscope: Single Atoms Bound in Orbit by Single Photons, Science 287, 1447 (2000).
\bibitem{Pinkse} P. W. H. Pinkse et al., Trapping an atom with single photons, Nature 404, 365 (2000).
\bibitem{Jones} P. J. Jones, J. A. M. Huhtam\"{a}ki, K. Y. Tan and M. M\"{o}tt\"{o}nen, Tunable electromagnetic environment for superconducting quantum bits, Sci. Rep. 3, 1987 (2013).
\end{thebibliography}
\end{document}

% ######################################################
% Pairwise Comparisons
% ######################################################
Studying such perceptions has traditionally been carried out using direct rating methods (users assign a score to each event or situation). This procedure requires a well-defined scale and user training and is particularly difficult to conduct when events or conditions substantially differ from one another \cite{perez2017practical}, which is the case when analyzing real-world environments. In contrast, using pairwise comparisons (users compare two situations and choose one of the two) is often simpler and faster to set up, well-suited for non-expert participants \cite{perez2017practical}, and presents lower measurement error compared to direct ratings \cite{shah2015estimation}. With this in mind, we employ pairwise comparisons to analyze cycling safety perceptions. Moreover, we draw current practice and knowledge from other research areas (e.g., sports outcome prediction and preference learning) about pairwise comparisons and how algorithms can be used to study cycling safety perceptions, something unexplored in cycling safety research. This paves the way to scale safety perception studies and ubiquitously understand how individuals perceive cycling risk.


% ######################################################
% Gap, Objectives & Contributions
% ######################################################
The main contributions of this paper are as follows. First, we draw knowledge from other research areas about pairwise comparisons and apply them to studying cycling safety perceptions. This novel approach
uses a survey showcasing images of two road environments and asking users which one they find safer, if any. % We use respondents' answers to compare different methodologies previously applied to sports prediction and preference learning, showcasing how these can be directly applied to our main goal: understanding cycling perception of safety.
With the respondents' answers, we compare different methodologies, previously applied to sports prediction and preference learning, and show how these can be directly applied to our main goal: understanding cycling perception of safety. Lastly, we draw from these results to objectively classify cycling environments based on urban characteristics and cycling environments. 


% ######################################################
% Outline of article
% ######################################################
We divide the article as follows. In the next section, we explore the current literature on pairwise comparisons and how traditional rating methods unravel such data. In Section \ref{sec:survey}, we detail our pairwise comparison survey and present different algorithms to rate cycling environments. Next, in Section \ref{sec:ranking}, we present the methodology, overviewing all pairwise ranking algorithms and environment classification. Section \ref{sec:results} presents the results and highlights what environments are perceived as safer or riskier. Finally, Section \ref{sec:conclusions} concludes the paper and draws possible paths forward.
%!TeX root = ../lmcs-gandalf22.tex


\section{Related work}
\label{sec:related}

The paradigm of timeline-based planning has been first introduced to plan and schedule scientific operations of the Hubble space telescope~\cite{Muscettola94}. In the following two decades, many timeline-based planning systems have been developed both at NASA and ESA, including EUROPA~\cite{BedraxWeissGBEI05}, ASPEN~\cite{ChienRKSEMESFBST00}, and APSI~\cite{DonatiPCFOCPSRNS08}. Such systems have been used both for short- to long-term mission planning, \eg for the renowned Rosetta mission~\cite{ChienRTTDNASVGA15}, and for onboard autonomy~\cite{FratiniCORD11}. Elements of the timeline-based and the action-based paradigm have been combined into the Action Notation Modeling Language (ANML)~\cite{SmithFC08}, extensively used at NASA since then.

Despite the real-world success, the timeline-based planning paradigm lacked a
thorough foundational understanding in contrast to the action-based paradigm, which
has been extensively studied from a theoretical perspective from the
start~\cite{McCarthyH69,Bylander94}. To enable theoretical investigations into
timeline-based planning, Cialdea Mayer \etal~\cite{CialdeaMayerOU16} laid down
the core features of the paradigm, describing them in a uniform formalism, which
has  been later studied in several contributions. The formalism was compared to
traditional action-based languages like STRIPS, and it was proved that the
latter are expressible by timeline-based languages~\cite{GiganteMMO16}. The
timeline-based plan existence problem was proved to be
\EXPSPACE-complete~\cite{GiganteMMO17} over discrete time in the general case,
and \PSPACE-complete with qualitative constraints \cite{DellaMonicaGTM20}. On
dense time, the problem goes from being \NP-complete to undecidable, depending
on the applied syntactic restrictions~\cite{BozzelliMMPW20}. Additionally,
logical~\cite{DellaMonicaGMSS17} and automata-theoretic~\cite{DellaMonicaGMS18}
counterparts have been investigated to study the expressiveness of timeline-based
languages.

The above body of work focuses on \emph{deterministic} timeline-based planning domains. However, the paradigm also fits to \emph{uncertain} domains requiring robust plans. Current timeline-based planning systems employ the concept of \emph{flexible timelines}, described as including uncertainty in the timings of events, representing envelopes of possible executions of the plan. Planners, when possible, produce \emph{strongly controllable} flexible plans, whose execution is then robust for the given temporal uncertainty. In order to obtain controllers for executing strongly controllable flexible plans, the problem can be simplified by reducing it to \emph{timed game automata}~\cite{OrlandiniFCF11}.

While the current approach works fairly well in handling temporal uncertainty, it does not support scenarios where the environment is fully nondeterministic. Furthermore, as pointed out in \cite{GiganteMOCR20}, the language of timeline-based planning as formalized in \cite{CialdeaMayerOU16} allows one to write domains that are not solvable by strongly controllable flexible plans, but that may easily be by strategies coping with general nondeterminism. For this reason, \cite{GiganteMOCR20} introduced the concept of \emph{timeline-based game}, which is the focus of this work. Timeline-based games adopt a game-theoretic point of view, where the controller and the environment play by constructing timelines, with the controller trying to fulfill its synchronization rules independently from the choices of the environment. This setting allows one to handle both temporal uncertainty and general nondeterminism, thus strictly generalizing previous approaches based on flexible timelines. In \cite{GiganteMOCR20}, the problem of deciding the existence of a winning strategy for a given timeline-based game has been proved to be \EXPTIME[2]-complete. The proof is based on the construction of a \emph{concurrent game structure} where a suitable \emph{alternating-time temporal logic} formula is model checked~\cite{AlurHK02}. However, the construction relies on nondeterministic procedures that are not effectively implementable, and thus it does not solve the problem of synthesizing actual controllers for timeline-based games. This work fills the gap by providing a constructive and effective synthesis algorithm.

The devised algorithm builds on classical results in the field of \emph{reactive synthesis}, which studies how to build correct-by-construction controllers satisfying high-level logical specifications. The original formulation of the problem of reactive synthesis is due to Church~\cite{church1962logic}. The problem  for \emph{S1S} specifications was later solved by Büchi and Landweber using a non-elementary complexity algorithm~\cite{buchi1990solving}. As for Linear Temporal Logic (LTL) specifications, the problem is \EXPTIME[2]-complete \cite{pnueli1989synthesis,rosner1992modular}, which, interestingly,  is the same complexity of timeline-based games. In both cases, the core of the synthesis algorithm is the construction of a \emph{deterministic} arena, where the game can be solved with a fix-point computation. This work focuses on constructing such an arena (\cref{sec:automaton,sec:games}).
%!TeX root = ../lmcs-gandalf22.tex

\section{Preliminaries}
\label{sec:preliminaries}


In this section, we provide an overview of the general framework that underpins
our work. We begin by introducing the general features of timeline-based planning, and then we
discuss timeline-based games. Next, we introduce the reactive synthesis problem.
Finally, we recall the concept of \emph{difference bound matrices}
(DBMs)~\cite{dill1989timing,peron2007abstract}, which are the data structures that we will use
to represent the temporal constraints of a system.

\subsection{Timeline-based planning}
The first basic notion is that of \emph{state variable}.
\begin{defi}[State variable]
  \label{def:timelines:state-variable}
  A \emph{state variable} is a tuple $x=(V_x,T_x,D_x,\gamma)$, where:
  \begin{itemize}
  \item $V_x$ is the \emph{finite domain} of $x$;
  \item $T_x:V_x\to2^{V_x}$ is the \emph{value transition function} of $x$,
    which maps each value $v\in V_x$ to the set of values that can immediately
    follow it; %the value $v$;
  \item $D_x:V_x\to\N\times\N$ is the \emph{duration function} of $x$, mapping
    each value $v\in V_x$ to a pair $(\dmin, \dmax)$ specifying respectively the
    minimum and maximum duration of any interval where $x=v$;
  \item $\gamma:V_x\to\set{\mathsf{c},\mathsf{u}}$ is the 
    \emph{controllability tag}, that, for each value $v\in V_x$, specifies whether it is
    \emph{controllable} $\left(\gamma\left(v\right)=\mathsf{c}\right)$ or \emph{uncontrollable}
    $\left(\gamma\left(v\right)=\mathsf{u}\right)$.
  \end{itemize}
\end{defi}

A state variable  $x$ takes its values from a finite domain and represents a
finite state machine with a transition function $T_x$. The behavior over time of
a state variable $x$ is modeled by a timeline. Intuitively, a \emph{timeline}
for a state variable $x$ is a finite sequence of \textit{tokens}, that is,
contiguous time intervals where $x$ holds a given value.

Following the approach described in~\cite{GiganteMOCR20}, instead of formally
defining timelines in terms of tokens, we represent executions of timeline-based
systems as single words, called \emph{event sequences}, where each event
describe the start/end of some token in a given time point.

To this end, we first define the notion of action.

\begin{defi}
  Let $\SV$ be a set of state variables. An \emph{action} is a term of the form $\tokstart(x,v)$ or $\tokend(x,v)$, where $x\in\SV$ and $v\in V_x$.
\end{defi}

Actions of the form $\tokstart(x,v)$ are \emph{starting} actions, and those of the form $\tokend(x,v)$ are \emph{ending} actions. We denote by $\actions_\SV$ the set of all the actions definable over a set of state variables $\SV$.

\begin{defi}[Event sequence~\cite{GiganteMOCR20}]
  \label{def:event-sequence}
  %
  Let $\SV$ be a set of state variables and $\actions_\SV$ be the set of all
  the \emph{actions} $\tokstart(x,v)$ and $\tokend(x,v)$, for $x\in\SV$ 
  and $v\in V_x$. An \emph{event sequence} over $\SV$ is a sequence
  $\evseq=\seq{\event_1,\ldots,\event_n}$ of pairs $\event_i=(A_i,\delta_i)$,
  called \emph{events}, where $A_i\subseteq\actions_\SV$ %is a set of actions
  and $\delta_i\in\N^+$, such that, for any $x\in\SV$:
  \begin{enumerate}
  \item \label{def:event-sequence:start}
        for all $1\le i\le n$, if $\tokstart(x,v)\in A_i$, for some $v\in V_x$,
        then there is no $\tokstart(x,v')$ in any $\event_j$ before the
        closest event $\event_k$, with $k > i$, such that $\tokend(x,v)\in A_k$ (if
        any);
  \item \label{def:event-sequence:end}
        for all $1\le i\le n$, if $\tokend(x,v)\in A_i$, for some $v\in V_x$,
        then there is no $\tokend(x,v')$ in any $\event_j$ after the
        closest event $\event_k$, with $k < i$, such that $\tokstart(x,v)\in A_k$ (if
        any);
  \item \label{def:event-sequence:gaps-right}
        for all $1\le i < n$, if $\tokend(x,v)\in A_i$, for some $v\in V_x$, then
        $\tokstart(x,v')\in A_i$, for some $v'\in V_x$;
  \item \label{def:event-sequence:gaps-left}
        for all $1< i \le n$, if $\tokstart(x,v)\in A_i$, for some $v\in V_x$,
        then $\tokend(x,v')\in A_i$, for some $v'\in V_x$.
  \end{enumerate}
\end{defi}

The first two conditions guarantee correct parenthesis placement by identifying
the start and the end of each token in the sequence. Condition 1 prevents a
token from starting before the end of the previous one, while condition 2
prevents the occurrence of two consecutive ends not interleaved by a start.
Conditions 3 and 4 ensure seamless continuity: each token's end (resp., start)
is consistently followed (preceded) by the start (resp., end) of another, except
for the first (resp., last) event in the sequence. These latter conditions
prevent gaps in the timeline description of the represented plan.

In event sequences, a \texttt{token} for a variable $x$ is a maximal interval
with at most one occurrence of events $\event_i=(A_i,\delta_i)$ and
$\event_j=(A_j,\delta_j)$, where $\tokstart(x,v)\in A_i$ and $\tokend(x,v)\in
A_j$, for some $v\in V_x$. We say such a token \emph{starts} at position $i$ and
\emph{ends} at position $j$. Note that \cref{def:event-sequence} implies that a
token that has started is not required to end before the end of the sequence and
that it can end without the corresponding starting action ever appearing. If
this is the case, we say that an event sequence is \emph{open} either to the right or
to the left. Otherwise, it is said to be \emph{closed}. An event sequence closed
to the left and open to the right is called a \emph{partial plan}. Notice that
the empty event sequence $\epsilon$ is closed on both sides for any variable.
Furthermore, in closed event sequences, the first event contains only start
actions, while the last one contains only end actions, one for each variable
$x$.

% Each event $\event_i=(A_i,\delta_i)$ consists of a set $A_i$ of actions describing the start or the end of tokens. The actions occur $\delta_i$ time steps after the previous event.
%
% \Cref{def:event-sequence} implies that a token that has started is not required to end before the end of the sequence and that it can end without the corresponding starting action ever appearing. If this is the case, we say an event sequence is \emph{open} either on the right or left. Otherwise, it is said to be \emph{closed}. An event sequence closed on the left and open on the right is %also 
% called a \emph{partial plan}. Notice that the empty event sequence $\epsilon$ is closed on both sides for any variable. Furthermore, in closed event sequences, the first event contains only $\tokstart(x,v)$ actions, while the last event only $\tokend(x,v)$ actions, one for each variable $x$.
Given an event sequence $\evseq=\seq{\event_1,\ldots,\event_n}$ over a set of state variables $\SV$, with $\event_i=(A_i,\delta_i)$, we define $\delta(\evseq)$ as $\sum_{1<i\le n}\delta_i$, that is, $\delta(\evseq)$ is the time elapsed from the start to the end of the event sequence (its duration). For any subsequence $\seq{\event_i, \ldots,\event_j}$ of $\evseq$, abbreviated $\slice\evseq_{i,j}$, we denote by $\delta_{i,j}$ (or, equivalently, $\delta(\slice\evseq_{i,j})$) the amount of time spanning that subsequence. Notice that $\delta_{i,j}$ is defined as $\sum_{i<k\le j}\delta_k$. Finally, given an event sequence $\evseq=\seq{\event_1,\ldots,\event_n}$, we define $\evseq_{<i}$ as $\seq{\event_1,\ldots,\event_{i-1}}$, for each $1<i\le n$.

\smallskip

In timeline-based planning, the objective is to satisfy a set of \emph{synchronization rules}, that specify the desired behavior of the system (constraints and goal). These rules relate tokens, possibly belonging to different timelines, through temporal relations among their endpoints. Let \SV be a set of state variables and $\toknames = \set{a,b,\ldots}$ be 
%an arbitrary 
a set of \emph{token names}. 

\begin{defi}[Atom]
    \label{def:atom}
    An atom is a temporal relation between tokens' endpoints of the form $\production{term} \before_{l,u} \production{term}$, where $l\in\N$, $u\in\N\cup\set{+\infty}$, $l \le u$, and a \emph{term} is either $\tokstart(a)$ or $\tokend(a)$, for some $a\in\toknames$.
\end{defi}

As an example, the atom $\tokstart(a) \before_{3,7} \tokend(b)$ constrains token $a$ to start at least $3$ and at most $7$ time units before the end of token $b$, while the atom $\tokstart(a) \before_{0,+\infty} \tokstart(b)$ simply constrains  token $a$ to start before token $b$.
 
%In such a notation $l$ and $u$ are the lower and the upper bound we impose on the temporal relation between the tokens' 
%endpoints. Meaning that $\langle term_2 \rangle$ is at least $l$ and at most $u$ time units after $\langle term_1 \rangle$.


\begin{defi}[Synchronization rule]
    A synchronization rule \Rule has one of the following two forms:
\begin{gather*}\label{eq:synchronisation-rules}
  \begin{array}{rcl}
  \centering
    \langle rule \rangle &:=& a_0[x_0=v_0]\implies \langle body \rangle\quad\\
    \langle rule \rangle &:=& \top\implies \langle body \rangle\quad \\
    \langle body \rangle &:=& \E_1 \lor \E_2 \lor \dots \lor \E_k\quad \\
    \E_j&\bydef&\exists a_1[x_1=v_1] a_2[x_2=v_2]\ldots a_n[x_n=v_n] \suchdot \clause_j, \ for \ 1 \leq j \leq k,
  \end{array}  
\end{gather*} 
where $a_i \in \toknames$, $x_i \in \SV$,  $v_i \in V_{x_i}$, and $\clause_j$ is a conjunction of atoms, for $0 \leq i\leq n$.
%$a_0, \ldots, a_n \in \toknames$, $x_0,\ldots,x_n \in \SV$, with $v_0,\ldots, v_n$ satisfying $v_i \in V_{x_i}$, for $0 \leq i\leq n$, and $\clause_i$ being a conjunction of atoms.
\end{defi} 
Terms $a_i[x_i=v_i]$ are referred to as \emph{quantifiers}. The term $a_0[x_0=v_0]$ is called the \emph{trigger}. The disjuncts in the body are called \emph{existential statements}. Quantifiers refer to tokens with the corresponding variable and value. The intuitive semantics of a synchronization rule can be given as follows: for every token satisfying the trigger, at least one of the existential statements must be satisfied as well. Each existential statement $\E_j$ requires the existence of tokens that satisfy the quantifiers in its prefix and the clause $\clause_j$. A token that satisfies the trigger of a rule is said to \emph{trigger} that rule. The trigger of a rule can be empty ($\top$). In such a case, the rule is referred to as \emph{triggerless} and it requires the satisfaction of its body without any precondition.

Let $a$ and $b$ be token names.
%An atom of the form $\tokstart(a) \before_{l,u} \tokend(b)$ constrains token $a$ to start before the end of token $b$, with 
%the distance between the two endpoints being at least $l$ and at most $u$.
Here are two examples of synchronization rules (relations $=$ and $\before$ are syntactic sugar for $\before_{0,0}$ and $\before_{0,+\infty}$, respectively):
\begin{align*}
  a[x_s=\mathsf{Comm}] \implies {}
  & \exists b[x_g=\mathsf{Available}] \suchdot
    \tokstart(b) \before \tokstart(a) \land \tokend(a) \before \tokend(b)\\
  a[x_s=\mathsf{Science}] \implies {}
  & \exists b[x_s=\mathsf{Slewing}] \ c[x_s=\mathsf{Earth}] \ d[x_s=\mathsf{Comm}]
    \suchdot {}\\
  & \tokend(a) = \tokstart(b) \land \tokend(b) = \tokstart(c) \land
    \tokend(c) = \tokstart(d)
\end{align*}
where variables $x_s$ and $x_g$ represent the state of a spacecraft and the visibility of the communication ground station, respectively. The first synchronization rule requires the satellite and the ground station to coordinate their communications so that when the satellite is transmitting, the ground station is available for reception. The second one instructs the system to send data to Earth after every measurement session, interleaved by the required slewing operation. Triggerless rule can be used to state the \emph{goal} of the system. As an example, the following rule ensures that the spacecraft performs some scientific measurement:
% 
\begin{equation*}
  \true \implies\exists a[x_s=\mathsf{Science}]
\end{equation*}
Triggerless rules only require the existence of tokens specified by the existential statements, being their universal quantification trivial. 
In fact, 
%While triggerless rules can describe the goals of a planning problem, 
they are syntactic sugar, 
%on top of the syntax mentioned earlier. 
as it is possible to translate them into triggered rules, as shown in~\cite{GiganteMOCR20}.
From now on, we will not consider them anymore.

We now formalise the above intuitive account of the semantics of synchronization
rules.
\begin{defi}[Matching functions~\cite{Gigante19}]
  \label{def:matching-function}
  %
  Let $\evseq=\seq{\event_1,\ldots,\event_n}$ be a (possibly open) event
  sequence, $\E\equiv \exists a_1[x_1=v_1]\ldots
  a_k[x_k=v_k]\suchdot\clause$ be one of the existential statements of a
  synchronization rule $\Rule\equiv a_0[x_0=v_0]\implies
  \E_1\lor\ldots\lor\E_m$, and $V$ be a set of terms such that $\tokstart(a)\in
  V$ or $\tokend(a) \in V$ only if $a \in \set{a_0,\ldots, a_k}$. A
  \emph{matching function} $\gamma:V\to[1,\ldots,n]$ maps each term $T\in V$
  to an event $\event_{\gamma(T)}$ in $\evseq$, such that:
  \begin{enumerate}
  \item \label{def:matching-function:nodes}
        for each $T\in V$, with $T=\tokstart(a)$ (resp., $T=\tokend(a)$),
        if $a$ is quantified as $a[x=v]$ in $\E$, then the event
        $\event_{\gamma(T)}=(A_T,\delta_T)$ is such that $\tokstart(x,v)\in A_T$
        (resp.,~$\tokend(x,v)\in A_T$);
      \item \label{def:matching-function:tokens} if both
        $T=\tokstart(a)$ and $T'=\tokend(a)$ belong to $V$ for some token name
        $a\in\toknames$, then $\gamma(T)$ and $\gamma(T')$ identify the
        endpoints of the same token.
  \end{enumerate}
\end{defi}
As a matter of fact, in \cite{Gigante19}, matching functions are defined in terms of \emph{rule graphs}, a data structure that we do not use here. For this reason, we reformulated the original definition in terms of event sequences.

The following definition gives a formal account of the semantics of synchronization rules.
\begin{defi}[Semantics of synchronization rules]
  Let $\Rule\equiv a_0[x_0=v_0]\implies \E_1\lor\ldots\lor\E_m$ and let
  $\evseq=\seq{\event_1,\ldots,\event_n}$ be an event sequence. We say that $\Rule$
  is \emph{satisfied} by $\evseq$ if, \emph{for each} event $\event_i=(A_i,
  \delta_i)$ such that $\tokstart(x_0,v_0)\in A_i$, there exist an existential
  statement $\E_j\equiv \exists a_1[x_1=v_1]\ldots a_k[x_k=v_k]\suchdot\clause$
  and a matching function $\gamma$ such that if $T \le_{[l,u]} T'$ appears in
  $\clause$, then $l\le \gamma(T')-\gamma(T)\le u$, for any pair of terms $T$ and $T'$.
\end{defi}

\emph{Timeline-based planning problems} can be defined as follows.
\begin{defi}[Timeline-based planning problem]
  A \emph{timeline-based planning problem} is a pair $P=(\SV,\S)$, where $\SV$ is
  a set of state variables and $\S$ is a set of synchronization rules over
  $\SV$. An event sequence $\evseq$ over $\SV$ is a solution plan for $P$ if all
  the rules in $\S$ are satisfied by $\evseq$.
\end{defi}

\subsection{Timeline-based games.} We are now ready to introduce the notion of \emph{timeline-based game}, that subsumes that of \emph{timeline-based planning with uncertainty} given in~\cite{CialdeaMayerOU16}.
%, which have been designed to be strictly
%more general than \emph{timeline-based %planning with %uncertainty}~\cite{CialdeaMayerOU16} while %being able to capture its semantics precisely.

\begin{defi}[Timeline-based game]
  \label{def:games:game}
  A \emph{timeline-based game} is a tuple $G=(\SV_C,\SV_E,\S,$ $\D)$,
  where $\SV_C$ and $\SV_E$ are the sets of \emph{controlled}
  and \emph{external} state variables, respectively, and $\S$ and $\D$ are the sets of
  \emph{system}  and \emph{domain} synchronization rules, respectively, both 
  involving variables from $\SV_C$ and $\SV_E$.
\end{defi}

A partial plan for $G$ is a partial plan over the variables
$\SV_C\cup\SV_E$. Let $\partialplans_G$ be the set of all possible partial plans
for $G$, simply $\partialplans$ when there is no ambiguity.
Since the empty event sequence $\epsilon$ is closed and $\delta(\epsilon)=0$, the
\emph{empty} partial plan $\epsilon$ is a good starting point for the game.
Players incrementally build onto a partial plan, starting from $\epsilon$, by
playing actions that specify which tokens to start and (or) to end, adding an
event that extends the event sequence, or complementing the existing
last one.  

Formally, we partition the set  of all the available actions $\actions_\SV$ into those that are playable by either of the two players.\fitpar

\begin{defi}[Partition of player actions]
  \label{def:games:actions-partition}
  %
  Let $\SV=\SV_C\cup\SV_E$. The set $\actions_\SV$ of available actions over
   $\SV$ is partitioned into the sets $\actions_C$ of \charlie's actions and  
   $\actions_E$ of \eve's actions, which are defined as follows:\fitpar
  \begin{align}
    \actions_C = {} &
      \underbrace{%
        \set{\tokstart(x,v)\suchthat x\in\SV_C,\; v\in V_x}%
      }_{\text{start tokens on \charlie's timelines}}\;\cup\;
      \underbrace{%
        \set{\tokend(x,v)\suchthat x\in\SV,\; v\in V_x,\; \gamma_x(v)=\ctag}%
      }_{\text{end controllable tokens}}\\
    \actions_E = {} &
      \underbrace{%
        \set{\tokstart(x,v)\suchthat x\in\SV_E,\; v\in V_x}%
      }_{\text{start tokens on \eve's timelines}}\;\cup\;
      \underbrace{%
        \set{\tokend(x,v)\suchthat x\in\SV,\; v\in V_x,\; \gamma_x(v)=\utag}%
      }_{\text{end uncontrollable tokens}}
  \end{align}
  %
\end{defi}

Hence, players can start tokens for owned variables and end them for values that
they control. Let $d=\max(L, U)+1$, where $L$ and $U$ are the maximum lower and
(finite) upper bounds appearing in any rule of $G$. Note that, by
\cref{def:games:actions-partition}, we may have $x \in \SV_E$ and $\gamma_x(v)=
c$ for some $v \in V_x$. This means that Charlie may control the duration of a
variable that belongs to Eve. This situation is symmetrical to the more common one where Eve controls the duration of a variable that belongs to Charlie, that is, uncontrollable tokens. As an example, Charlie may decide to start a task, without
being able to foresee how long it will take. Similarly, the environment may
trigger the start of a process, \eg fixing a plant fault, but Charlie may be
able to control, to some extent, how long it will take to end it, \eg we can
decide to fix it today or tomorrow. 

Actions combine into \emph{moves} starting (resp., ending) multiple tokens simultaneously.


\begin{defi}[Move]
  \label{def:games:moves}
  %
  A \emph{move} $\move_C$ for \charlie is a term of the form $\wait(\delta_C)$
  or $\play(A_C)$, where $1 \le \delta_C \le d$ and $\emptyset\ne
  A_C\subseteq\actions_C$ is  either a set of \emph{starting} actions or 
  a set of \emph{ending} actions. A \emph{move} $\move_E$ for \eve is a term of the form $\play(A_E)$ or
  $\play(\delta_E,A_E)$, where $1 \le \delta_E \le d$ and $A_E\subseteq\actions_E$ is
  either a set of \emph{starting} actions or a set of \emph{ending} actions.
  %
\end{defi}
By \cref{def:games:moves}, moves like $\play(A_C)$ and $\play(\delta_E,A_E)$ can
play either $\tokstart(x,v)$ actions only or $\tokend(x,v)$ actions only. A move
of the former kind is called a \emph{starting} move, while a move of the latter
kind is called an \emph{ending} move. We consider $\wait$ moves as \emph{ending}
moves. Starting and ending moves must alternate during the game.

Let us denote the sets of \charlie's and \eve's moves by $\moves_C$ and $\moves_E$ , respectively. A round of the game is defined as follows.

\begin{defi}[Round]
  \label{def:games:round}
  %
  A \emph{round} $\round$ is a pair
  $(\move_C,\move_E)\in\moves_C\times\moves_E$ of moves such that:
  \begin{enumerate}
  \item \label{def:games:round:alternation}
        $\move_C$ and $\move_E$ are either both \emph{starting} or both
        \emph{ending} moves;
  \item \label{def:games:round:paring}
        either $\round=(\play(A_C),\play(A_E))$, or
        $\round=(\wait(\delta_C),\play(\delta_E,A_E))$, with
        $\delta_E\le\delta_C$;
  \end{enumerate}
  %
\end{defi}

A \emph{starting} (resp., \emph{ending}) round is one made of starting (resp., ending) moves.
Since \charlie cannot play empty moves and $\wait$ moves are ending moves, each round is unambiguously either a starting or an ending round. Moreover, since $\play(\delta_E,A_E)$ moves are always paired with 
%played only in rounds together with 
$\wait(\delta_C)$ ones, which are ending moves, then $\play(\delta_E,A_E)$ moves are necessarily ending moves (item \ref{def:games:round:alternation} of Definition 
\ref{def:games:round}).

 We can now specify how to apply a round to the current partial plan to obtain the new one. The game always starts with a single starting round.

\begin{defi}[Outcome of rounds]
  \label{def:games:round-outcome}
  %
  Let $\evseq=\seq{\event_1,\ldots,\event_n}$ be an event sequence, with
  $\event_n=(A_n,\delta_n)$ ($\event_n=(\emptyset,0)$ if $\evseq=\epsilon$). Let
  $\round=(\move_C,\move_E)$ be a round, $A_E$ and $A_C$ be the sets of actions
  of the two moves ($A_C$ is empty if $\move_C$ is a $\wait$ move), and
  $\delta_E$ and $\delta_C$ be the time increments of the moves. We define
  $\delta_C=1$ (resp., $\delta_E=1$) for $\play(A_C)$  
  (resp., $\play(A_E)$).
  
  The \emph{outcome} of the application of $\round$ on $\evseq$ is the event sequence
  $\round(\evseq)$ defined as follows:
  %
  \begin{enumerate}
  \item \label{def:games:round-outcome:starting}
        if $\round$ is a starting round, then $\round(\evseq)=\evseq_{< n}\event_n'$,
        where $\event_n'\nobreak=\nobreak(A_n\cup A_C\cup A_E,\delta_n)$;
  \item \label{def:games:round-outcome:ending}
        if $\round$ is an ending round, then $\round(\evseq)=\evseq\event'$, where
        $\event'=(A_C\cup A_E,\delta_E)$;
  \end{enumerate}
  
  We say that $\round$ is \emph{applicable} to $\evseq$ if:
  \begin{enumerate}[label=\alph*)]
  \item \label{def:games:round-outcome:integrity}
        $\round(\evseq)$ %is a well defined event sequence by 
        complies with \cref{def:event-sequence};
  \item \label{def:games:round-outcome:alternation}
        $\round$ is an ending round if and only if $\evseq$ is open for all variables that appear in the moves.
  \end{enumerate}
  %
\end{defi}

A single move by either player is applicable to $\evseq$ if there is
a move for the other player such that the resulting round is applicable to
$\evseq$.
The game starts from the empty partial plan $\epsilon$, and players play in
turn, composing a round from the move of each one, which is applied to the
current partial plan to obtain the new one.
We can now define the notion of \emph{strategy} for each player and that of \emph{winning strategy} for \charlie.

\begin{defi}[Strategy]
  \label{def:games:strategies}
  %
  A \emph{strategy for Charlie} is a function
  $\strategy_C:\partialplans\to\moves_C$ that maps any given partial plan
  $\evseq$ into a move $\move_C$ applicable to $\evseq$.
%
  A \emph{strategy for Eve} is a function
  $\strategy_E:\partialplans\times\moves_C\to\moves_E$ that maps a partial
  plan $\evseq$ and a move $\move_C\in\moves_C$ applicable to $\evseq$ into a move
  $\move_E$ such that the round $\round=(\move_C,\move_E)$ is applicable to
  $\evseq$.
  %
\end{defi}

A sequence $\rounds=\seq{\round_0,\ldots,\round_n}$ of rounds is called a
\emph{play} of the game. A play is said to be \emph{played according to} some
strategy $\strategy_C$ for \charlie, if, starting from the initial partial plan
$\evseq_0=\epsilon$, it holds that $\round_i=(\strategy_C(\Pi_{i-1}),
\move_E^i)$, for some $\move_E^i$, for all $0<i\le n$, and to be played
according to some strategy $\strategy_E$ for \eve if $\round_i=(\move_C^i,
\strategy_E(\Pi_{i-1},\move_C^i))$, for all $0<i\le n$. It can be easily seen that for
any pair of strategies $(\strategy_C,\strategy_E)$ and any $n\ge0$, there is a
unique play $\rounds_n(\strategy_C,\strategy_E)$ of length $n$ played according to both $\strategy_C$ and $\strategy_E$.

Then, we say that a partial plan $\evseq$ and the play $\rounds$ such that
$\evseq=\rounds(\epsilon)$ are \emph{admissible}, if the partial plan satisfies
the domain rules, and that they are \emph{successful} if the partial plan satisfies the
system rules.

\begin{defi}[Admissible strategy for \eve]
  \label{def:games:admissible-strategy}
  %
  A strategy $\strategy_E$ for \eve is \emph{admissible} if for each strategy
  $\strategy_C$ for \charlie, there is $k\ge 0$ such that the play
  $\rounds_k(\strategy_C,\strategy_E)$ is admissible.
  %
\end{defi}

\charlie wins if, \emph{assuming} that domain rules are respected, he manages to
satisfy the system rules no matter how \eve plays.

\begin{defi}[Winning strategy for \charlie]
  \label{def:games:winning-strategy}
  %
  Let $\strategy_C$ be a strategy for \charlie. We say that $\strategy_C$ is a
  \emph{winning strategy} for \charlie if for any \emph{admissible} strategy
  $\strategy_E$ for \eve, there exists $n\ge0$ such that the play
  $\rounds_n(\strategy_C,\strategy_E)$ is successful.
  %
\end{defi}

We say that \charlie \emph{wins} the game $G$ if he has a winning strategy,
while \eve \emph{wins} the game if a winning strategy for \charlie does not
exist.




\subsection{Synthesis}
The synthesis problem is the problem of devising an implementation that satisfies a formal specification of an input-output relation~\cite{PnueliRosner89}. Such an implementation may be a transducer, a Mealy machine, a Moore machine, a circuit, or the like. In the following, we give a short account of the roles of games and strategies in game-based synthesis.

\begin{defi}[Game Graph] A finite game graph $G$ is a triple $\left(Q, Q_C, E\right)$, where $Q$ is a finite set of nodes, $Q_C \subseteq Q$ is the subset of \charlie's nodes, and $E \subseteq Q \times Q$ is a transition relation. The relation $E$ must satisfy the condition: $\forall q \exists q' : \left(q,q'\right) \in E$ (totality). \end{defi} 

A \emph{play} on a game graph $G$ starting from the initial state $q_0$ is an infinite sequence $p = q_0 q_1 q_2\ldots$, where $(q_i, q_{i+1}) \in E$, for all  $i \ge 0$. A game is  a pair $(G, \mathcal{W})$, where $G$ is a game graph and $\mathcal{W}$ 
%\subseteq 2^Q$ 
is the winning condition of the game. In the general case, $\mathcal{W}$ consists of the set of plays won by  \charlie.



Here, we focus on reachability winning conditions, which are expressed as $\mathcal{W} \bydef \{ R \subseteq Q \mid R \cap F \neq \emptyset \}$, for a given set $F \subseteq Q$. A play $p$ is said to satisfy 
%the reachability winning condition 
$\mathcal{W}$ if the set of states visited by $p$, denoted by $occ(p) = \{q \in Q \mid \exists i \mathrel{.} p(i) = q\}$, intersects $\mathcal{W}$, that is, \charlie wins the play $p$ if $p$ visits at least one state in $F$.

\begin{defi}[Reachability game] A reachability game is a pair $(G, \mathcal{W})$, where $G = (Q, Q_C, E)$ is a game graph and $\mathcal{W}$ is a reachability winning condition. \end{defi}

A strategy for \charlie is a function $f : Q^*\cdot Q_C \rightarrow Q$. A play
$p$ adheres to strategy $f$ if, for each $q_i \in Q_C$, $q_{i+1} = f(q_0 \ldots
q_i)$. Given an initial state $q$, a strategy for \charlie is a winning strategy
if \charlie wins any play from $q$ that follows the strategy $f$. The same holds
for \eve.
%principle applies to \eve. 
\charlie (resp., \eve) wins if a winning strategy exists from $q$. 

Given a game $(G, \mathcal{W})$, with $G = (Q, Q_C, E)$,
the winning region of \charlie is defined as $W_C \bydef \{ q \in Q \, \vert \,
\charlie \text{ wins from q } \}$. The winning region $W_E$ for \eve
is defined in an analogous way.
The two sets are clearly disjoint ($W_C \cap W_E = \emptyset$). The game is
said to be \emph{determined} if $W_C \cup W_E = Q$. It is well known that
reachability games are determined~\cite{Thomas2008}. 

\smallskip

The next step is to build a Controller starting from a winning strategy $f$ such that the specification is met. We use Moore machines as \charlie plays first. 

\begin{defi}[Moore machine]
    \label{def:moore-machine}
    A Moore machine is a tuple $M = (Q, \Sigma, \Gamma, q_0, \delta, \tau)$, where $Q$ is a finite set of states, $\Sigma$ is a finite input alphabet, $\Gamma$ is a finite output alphabet, $q_0 \in Q$ is the initial state, $\delta : Q \times \Sigma \rightarrow Q$ is the transition function, and $\tau: Q \rightarrow \Gamma$ it the output function.
\end{defi}

By suitably tying $\delta$ and $\tau$ to $f$, one can effectively implement $f$. We refer the reader to \cref{def:controller-implementation} for the details on how we do it.

\subsection{Difference Bound Matrices}
\emph{Difference bound matrices} (DBMs) were introduced by
Dill~\cite{dill1989timing} as a pragmatic representation of constraints $(x - y
\leq c)$. Later on, Péron et al.~\cite{peron2007abstract} suitably expanded
the formalism. The following short account of the formalism is basically borrowed from
the latter work,

Let $\mathit{Var} = \{v_0, v_1, \ldots v_n\}$ be a finite set of variables, $\bar{V} = \mathbb{Z} \cup
\{+\infty\}$ be a set of values that variables and constants can take, and $C$ be  a set of constraints of the form $v_i - v_j \leq c$, where $v_i,v_j \in \mathit{Var}$ and $c \in \bar{V}$. 
%
The DBM that represents $C$ is an $(n+1)\times(n+1)$ matrix defined as follows: 
\begin{equation*}
    M_{ij} = \text{inf}\{c \, \mid \, (v_i - v_j \leq c) \in C\},
\end{equation*}
where $\text{inf}(\emptyset) = +\infty$. 

$M_{ij}$ equals the tightest value of
$c$ if there is some constraint $(v_i - v_j \leq c)$ in $C$; otherwise, it is
$+\infty$. The variable $v_0 \in \mathit{Var}$ is always valued to $0$, and it is
used to express bounds on variables, that is, $v_i \leq c$ is written as
$v_i - v_0 \leq c$.  In \cref{sec:automaton}, we use DBMs to conveniently
represent atoms (see \cref{def:atom}).

%%% Local Variables:
%%% TeX-master: "../lmcs-gandalf22.tex"
%%% End:
%!TeX root = ../lmcs-gandalf22.tex

%TODO controllare le definizioni
\section{A deterministic automaton for timeline-based planning}
\label{sec:automaton}

In this section, we define an encoding of timeline-based planning problems into \emph{deterministic} finite state automata (DFA).
Given a timeline-based planning problem, the corresponding automaton recognizes all and only those \emph{event sequences} that represent solution plans for the problem. In the next section, we will use such an automaton as the game arena for a timeline-based game.

\subsection{Plans as words}
Let $P=(\SV, S)$ be a timeline-based planning problem and, as already stated in the previous section, let $d = \max(L, U)+1$, where $L$ and $U$ are the maximum lower and (finite) upper bounds appearing in any rule of $P$. We restrict our attention to event sequences where the distance between two consecutive events is at most $d$. Such a restriction guarantees us the finiteness of the considered alphabet, and it does not cause any loss in generality, as proved by Lemma 4.8 of~\cite{Gigante19}. Moreover, it agrees with the notion of move of a timeline-based game (see \cref{def:games:moves}).

We define the symbols of the alphabet $\Sigma$ as \emph{events} of the form $\event = \pair{A,
\delta}$, where $A \subseteq \actions_\SV$ and $1\le\delta\le d$. Formally,
$\Sigma=2^{\actions_\SV}\times\ar{d}$, where $\ar{d}=\set{1,\ldots,d}$. Note that the size of $\Sigma$ is exponential in the size of the problem. Moreover,
we define $\window(P)$ as the sum of all the
coefficients appearing as upper bounds in the rules of $P$. This value represents the
maximum amount of time a rule can ``count'' far away from the
occurrence of the quantified
tokens. Consider, for instance, the following rule: 
\begin{align}
  \label{eq:example-3var-synch}
  a_0[x_0=v_0]\to{} &\exists a_1[x_1=v_1] a_2[x_2=v_2] a_3[x_3=v_3] \suchdot \\
  &\tokstart(a_1)\before_{4,14}\tokend(a_0)
  \land \tokend(a_0)\before_{0,+\infty}\tokend(a_2) \land \tokstart(a_2)\before_{0,3}\tokend(a_3)\tag*{}
\end{align}

In this case, assuming the above rule  to be the only one in the problem,
$\window(P)$ would be $3 + 14=17$. Thus, the rule can account for what happens
at most $17$ time points from the occurrence of its quantified tokens. For
instance, if the token $a_1$ appears at a specific distance from $a_0$, it has
to be within less than $17$ time points, and any modification of the plan that
alters this distance can break the rule's satisfaction. However, what occurs
further away from $a_0$ only affects the fulfillment of the rule
\emph{qualitatively}. Suppose that the tokens $a_2$ and $a_3$ are, together, at
$100$ time points from $a_0$. Changing this distance while maintaining the
qualitative order between tokens does not break the rule's satisfaction. For
$\window(P)$'s properties refer to \cite{Gigante19}.

\subsection{Matching structures}
A key insight underlying the construction we are going to outline is that every atomic temporal
relation $T \before_{l,u} T^\prime$ can be rewritten as the conjunction of two upper 
bound constraints $T^\prime - T \leq u$ and $T - T^\prime \leq -l$, where we
represent a lower bound constraint $T^\prime - T \geq l$ as an upper bound one.
This allows us to rewrite the clause \clause of an existential statement \E as a constraint system
$\nu(\clause)$ with constraints of the form $T - T^\prime \leq n$, for $n \in
\Z \cup \{+\infty\}$.

The  constraint system $\nu(\clause)$ can be represented by a difference bound matrix $D$ indexed by terms, where the entry $D[T, T']$ gives the upper bound $n$ on $T - T^\prime$.
In building $D$, we ensure the right duration of tokens
by augmenting the system with constraints
of the kind $\tokstart(a_i)-\tokend(a_i)\leq-\dmin^{x_i=v_i}$ and
$\tokend(a_i)-\tokstart(a_i)\leq\dmax^{x_i=v_i}$, for any quantified token
$a_i[x_i=v_i]$ of $\E$.
As an example, the constraint system and the DBM for the above rule are the ones in \cref{fig:rule-constraint-system,fig:dbm}, respectively.

% Figure environment removed

% Figure environment removed

On top of DBMs, we define the concept of \emph{matching structure}, a data
structure that allows us to monitor and update the fulfillment of atomic
temporal relations among terms throughout the execution of the plan. More
precisely, it allows us to manipulate and reason about existential statements of
which only a portion of the requests has been satisfied by the word read so far,
while the rest is potentially satisfiable in the future.

\begin{defi}[Matching Structure]
  \label{def:matching-structure}
  Let $\E\equiv \exists a_1[x_1 = v_1] \dots a_m[x_m = v_m] \,.\, \clause$ be
  an existential statement of a synchronization rule $\Rule \equiv a_0[x_0 =
  v_0] \rightarrow \E_1 \lor \dots \lor \E_k$ over the set of state variables
  \SV. The \emph{matching structure} for $\E$ is a tuple $\M_{\E} = (V, D, M, t)$,
  where:
  \begin{itemize}
  \item $V$ is the set of terms $\tokstart(a)$ and $\tokend(a)$, for
    $a\in\set{a_0, \dots, a_m}$;
  \item %$D \in \Z_{+\infty}^{|V|^2}$ is a DBM indexed by terms of $V$, 
    $D$ is a DBM of size $|V| \times |V|$, indexed by terms of $V$, whose
    entries take value over $\Z \cup \{+\infty\}$, where
  \begin{align*}
      \begin{cases}
          D[T,T']=n &\quad if  \; T-T'\le n \ \in \nu(\clause), \\
          D[T,T']=0 &\quad if \; T=T', \\
          D[T,T']=+\infty &\quad otherwise;
      \end{cases}
  \end{align*}    
  \item $M \subseteq V$ and $0\le t \le \window(P)$.
  \end{itemize}
\end{defi}

The set $M$ contains the set of terms from $V$ correctly seen in the sequence so
far. We say these terms have been \emph{matched} by the matching structure. We
use $\overline{M} = V \setminus M$ to refer to terms yet to be matched. We say a
matching structure $\M$ to be \emph{closed} if $M = V$, \emph{initial} if $M =
\emptyset$, and \emph{active} if $\tokstart(a_0) \in M$ and it is not closed.
The component $t$ represents the time elapsed since matching $\tokstart(a_0)$.
As time progresses, we update a matching structure as follows.

In the DBMs of a matching structure, the bounds between any pair of terms $T$
and $T'$, with one in $M$ while the other not, are tightened by the elapsing of
time. When $T\in M$ and $T'\in\overline{M}$, $D[T,T']$ is a lower bound loosened
by adding the elapsed time $\delta$. When $T\in\overline{M}$ and $T'\in M$,
$D[T,T']$ is an upper bound tightened by subtracting $\delta$. Consider the DBM
in \Cref{fig:dbm} and the pair of terms $\tokstart(a_1)$ and $\tokend(a_0)$. We
have $D[\tokstart(a_1),\tokend(a_0)]=-4$, implying that
$\tokstart(a_1)-\tokend(a_0)\le -4$ must hold. Suppose that $\tokstart(a_1)\in
M$ (it has been matched), and that $\tokend(a_0)\in\overline{M}$ (it needs to be
matched). Now, in a time step, the entry in the DBM is incremented and updated
to $-4+1=-3$ reflecting the fact that we now have $3$ time steps left to match
$\tokend(a_0)$. A similar analysis leads us to the conclusion that the entry
$D[\tokend(a_0),\tokstart(a_1)]=14$ has to be decremented by $1$ and updated to
$14-1=13$. This intuition is formalized as follows.

\begin{defi}[Time shifting]
  \label{def:time-shift}
  Let $\delta > 0$ be a positive amount of time, and let $\M = (V, D, M, t)$ be a
  matching structure. The result of shifting $\M$ by $\delta$ time units,
  written $\M + \delta$, is a matching structure $\M^\prime = (V, D^\prime, M,
  t')$, where:
  \begin{itemize}
  \item for all $T, T' \in V$:
    \begin{equation*}
      D^\prime[T,T'] =
      \begin{cases}
        D[T,T'] + \delta &\text{if } T \in M \text{ and } T' \in
        \overline{M}\\% D[Tj] è lower bound
        D[T,T'] - \delta &\text{if } T \in \overline{M} \text{ and } T' \in
        M\\% D[ij] è upper bound
        D[T,T'] &\text{otherwise}
      \end{cases}
    \end{equation*}
  \item and
    \[
      t' =
      \begin{cases}
        t+\delta & \text{if } \M \text{ is \emph{active}}\\
        t & \text{otherwise}
      \end{cases}
    \]
  \end{itemize}
\end{defi}
\Cref{def:time-shift} specifies how to update the entries of $D$ and how to update $t$ to the trigger occurrence of an active matching structure.

\begin{defi}[Matching]
  \label{def:matching}
  Let $\M = (V, D, M, t)$ be a matching structure and $I \subseteq \overline{M}$
  a set of matched terms. A matching structure $\M^\prime = (V, D, M^\prime, t)$
  is the result of matching the set $I$, written $\M \cup I$, with $M^\prime = M
  \cup I$.
\end{defi}

To correctly match an existential statement while reading an event sequence, a matching structure is updated only as long as one witnesses no violation of temporal constraints. As such, we deem an event as \emph{admissible} or not.

\begin{defi}[Admissible Event]\label{def:admissible-event}
  An event $\event = (A, \delta)$ is \emph{admissible} for a matching structure
  $\M_{\E} = (V, D, M, t)$ if and only if, for every $T \in M$
  and $T' \in \overline{M}$, $\delta \leq D[T',T]$, \ie the elapsing of $\delta$
  time units does not exceed the upper bound of some term $T'$ not yet
matched by $\M_{\E}$.
\end{defi}

Each admissible event $\event$ that is read can be matched with a subset of terms from the matching structure. However, there can be multiple ways to match events and terms. To make this choice explicit, we introduce the following definition.

\begin{defi}[$I$-match Event]\label{def:match-event}% I for indexes
  Let $\M_{\E} = (V, D, M, t)$ be a matching structure and  $I \subseteq
  \overline{M}$. An $I$\emph{-match event} is an admissible event $\event = (A,
  \delta)$ for $\M_{\E}$ such that:
  \begin{enumerate}
  \item for all token names $a \in \mathsf{N}$ quantified as $a[x = v]$ in $\E$
    we have that:\label{def:match-event:good-match}
    \begin{enumerate}
    \item if $\tokstart(a) \in I$, then $\tokstart(x, v) \in A$;
      \label{def:match-event:good-match:start}
    \item $\tokend(a) \in I$ if and only if $\tokstart(a) \in M$ and $\tokend(x,v) \in
      A$;\label{def:match-event:good-match:end}
    \end{enumerate}
  \item and for all $T \in I$ it holds that:\label{def:match-event:relations}
    \begin{enumerate}
    \item \label{def:match-event:preceding-terms} for every other term $T' \in
      V$, if $D[T',T] \leq 0$, then $T' \in M \cup I$;
    \item \label{def:match-event:lower-bounds} for all $T' \in M$, $\delta \geq
      -D[T',T]$, \ie all the lower bounds on $T$ are satisfied;
    \item \label{def:match-event:zero-no-bounds} for each other term $T' \in I$,
      either $D[T',T] = 0$, $D[T,T'] = 0$, or $D[T',T] = D[T, T'] = +\infty$.
    \end{enumerate}
  \end{enumerate}
\end{defi}

We consider an event $\event$ an $I$-match event if its actions correspond to the terms in $I$. The definition in \Cref{def:match-event:good-match} ensures the correct matching of each term to an action it represents and that the endpoints of a quantified token precisely identify the endpoints of a token in the event sequence. Meanwhile, \Cref{def:match-event:relations} guarantees that matching the terms in $I$ does not violate any atomic temporal relation. In addition, \Cref{def:match-event:preceding-terms} deals with the qualitative aspect of a ``happens before'' relation, while \Cref{def:match-event:lower-bounds,def:match-event:zero-no-bounds} address the quantitative aspects of the lower bounds of these relations. It is worth noting that an $\emptyset$-event is also considered admissible.

Let $\matchstructs_P$ denote the set of all matching structures for a planning
problem $P$, and let $\I$ be the set of all possible terms built from token
names in $\toknames$. To describe the evolution of a matching structure, we
define a quaternary relation
$S\subseteq\matchstructs_P\times\Sigma\times\I\times\matchstructs_P$ as
$(\M,\event,I,\M')\in{S}$, for an event $\event = (A, \delta)$, if and only if
$\event$ is an $I$-match event for $\M$, and $\M'=(\M+\delta)\cup I$. We also
write $\M \stepm \M'$ in place of $(\M,\event,I,\M')\in{S}$.
Note that, from \Cref{def:match-event}, a single event can represent multiple
$I$-match events for a matching structure. Therefore, given a matching structure
$\M$ and an event $\event$, automaton states will collect all the matching
structures $\M'$ resulting from the relation $S$, for some set of terms $I$.
Given a set of matching structures $\Upsilon$, this notion is best described by
the function $\step_\event(\Upsilon)=\set{\M' \mid (\M,\event,I,\M')\in S,
  \text{ for some } \M\in\Upsilon \text{ and } I \in \I}$. Furthermore, we
define $\Upsilon^\Rule_t\subseteq\Upsilon$ as the set of all the \emph{active}
matching structures $\M\in\Upsilon$ with timestamp $t$, associated with any
existential statement of $\Rule$. Matching structures in $\Upsilon^\Rule_t$
contribute to fulfilling the same triggering event of $\Rule$, regardless of their
existential statement. We also define $\Upsilon_\bot\subseteq\Upsilon$ as the
set of \emph{non-active} matching structures of $\Upsilon$. Lastly, we say that
$\Upsilon$ is \emph{closed} if there exists $\M\in\Upsilon$ such that $\M$ is
\emph{closed}.

% Figure environment removed

We conclude this section by providing an example of updating a matching structure $\M=(V,D,M,t)$ for the rule discussed at the beginning of the section. Consider the set of timelines in \cref{fig:timelines-0-1-2-3}. Before matching any term $\M$ is initial with $M = \emptyset$, $t = 0$, $D$ as the DBM in \cref{fig:dbm}, and $V$ as the set of term $\tokstart(a)$ and $\tokend(a)$ for $a \in \{a_0, a_1, a_2, a_3\}$.
We begin by matching the terms $\tokstart(a_0)$ and $\tokstart(a_3)$ from the event $\event = (\{\tokstart(x_0, v_0),\tokstart(x_3, v_3)\}, 0)$ (we do not consider $\tokstart(x_1,v'_1)$ and $\tokstart(x_2,v'_2)$ since they are not in $V$). Such event is an $I$-match event for $I = \{ \tokstart(a_0), \tokstart(a_3)\}$: it is an admissible event (\cref{def:admissible-event}), \cref{def:match-event:good-match:start} holds, for both 
$\tokstart(a_0)$ and $\tokstart(a_3)$, there are no terms that should appear before them (\cref{def:match-event:preceding-terms}), there are no related lower bounds (\cref{def:match-event:lower-bounds}), and $D[\tokstart(a_0), \tokstart(a_3)] = D[\tokstart(a_3), \tokstart(a_0)] = +\infty$ (\cref{def:match-event:zero-no-bounds}). Hence, we update $M = M \cup I = \{\tokstart(a_0), \tokstart(a_3)\}$ and $t = t + \delta = 0$; now $\M$ is active. The next term to consider is $\tokstart(a_2)$, which occurs after $\delta = 5$ time steps.

First, we ensure that the event $\event = (\tokstart(x_2, v_2), 5)$ is admissible. We show that by examining the DBM in \cref{fig:dbm}, we see that the elapsing of time $\delta$ does not exceed any upper bound related to terms $T \in M$ and $T' \in \overline{M}$. Next, the set $I$ in the current state appears as $I = \{ \tokstart(a_2) \}$. Notice that we are in the case of \cref{def:match-event:good-match:start}, and \cref{def:match-event:relations} holds because no constraint involves the term $\tokstart(a_2)$ (\cref{def:match-event:preceding-terms}), no lower bounds are related to $\tokstart(a_2)$ (\cref{def:match-event:lower-bounds}), and $\tokstart(a_2)$ is the only term in $I$ (\cref{def:match-event:zero-no-bounds}). Therefore, from \cref{def:matching,def:time-shift}, we update $\M$ as follows: $\M = (\M + \delta) \cup I$. Each entry of the DBM will remain unchanged since the third update case of \cref{def:time-shift} applies, $M = M \cup I = \{\tokstart(a_0), \tokstart(a_3), \tokstart(a_2)\}$, and $t = t + \delta = 5$.

Similarly, for the next event is $\event = (\tokstart(x_1, v_1), 1)$, we check if such an event is admissible, and indeed it is since the upper bound  $D[\tokend(a_0), \tokstart(a_1)] = 9 \ge \delta$. It is also an $I$-match event for $I = \{\tokstart(a_1)\}$, since it respects \cref{def:match-event:good-match:start} and all the relations in \cref{def:match-event:relations}; thus we update $\M$. We decrement $D[\tokend(a_3), \tokstart(a_2)]$ and increment $D[\tokstart(a_2), \tokend(a_3)]$ by 1 (see \cref{def:time-shift}), update $M$ like follows $M = \{\tokstart(a_0), \tokstart(a_3), \tokstart(a_2), \tokstart(a_1)\}$, and $t = t + \delta = 5 + 1 = 6$.

The next event is $\event = (\tokend(x_3, v_3), 3)$ after 2 time steps. Note that it is an admissible event and also an $I$-match event for $I = \{\tokend(a_3)\}$. In this case, we emphasize that \cref{def:match-event:good-match:end,def:match-event:relations} are respected. We update the DBM as follows: $D[\tokend(a_0), \tokstart(a_1)] = 14 - 2 = 12$, $D[\tokstart(a_1), \tokend(a_0)] = -4 + 2 = -2$, $D[\tokend(a_3), \tokstart(a_2)] = 2 - 2 = 0$, $D[\tokstart(a_2), \tokend(a_3)] = 1 + 2 = 3$. Then, we update $M = M \cup I = \{\tokstart(a_0), \tokstart(a_3), \tokstart(a_2),\\ \tokstart(a_1), \tokend(a_3)\}$ and $t = t + \delta = 6 + 2 = 8$. Notice that if we did not match $\tokend(a_3)$ now, at the next time step, the timeline would have violated the rule above because the upper bound $D[\tokend(a_3),\tokstart(a_2)] = 0$.

The subsequent event is $\event = (\tokend(x_1, v_1), \tokstart(x_1 = v'_1), 6)$ for which $I = \tokend(a_1)$. Since there is no constraint involving $\tokend(a_1)$, this event is admissible and an $I$-match event. The DBM is shifted by 6 time steps, and $M = \{\tokstart(a_0), \tokstart(a_3), \tokstart(a_2), \tokstart(a_1), \tokend(a_1)\}$.

The last event $\event = (\{\tokstart(x_0, v'_0), \tokstart(x_2, v^{''}_2), \tokend(x_0, v_0), \tokend(x_2, v_2)\}, 2)$ is admissible and an $I$-match for $I = \{\tokend(a_0), \tokend(a_2)\}$, note that there is not an upper bound between $\tokend(a_0)$ and $\tokend(a_2)$ and that \cref{def:match-event:good-match:end,def:match-event:relations} of the definition of $I$-match event are respected.

\subsection{Building the automaton}\label{sec:automata-construction}
We can now define the automaton. First, given an existential statement $\E$, let $\mathbb{E}_\E$ be the set of all existential statements in the same rule of $\E$. Next, let $\mathbb{F}_P$ be the set of functions that map each existential statement of $P$ to a set of existential statements and let $\mathbb{D}_P$ be the set of functions that map each existential statement to a set of matching structures $\Upsilon$.
%TODO rivedere questa definizione o magari definire l'automa TV_P
An automaton $\TV_P$ that checks the transition functions of the variables is easy to define. Then, given a timeline-based planning problem $P=(\SV, S)$, we can characterize the corresponding automaton as $A_P=\TV_P\cap\S_P$. Here, $\S_P$ checks the fulfillment of the synchronization rules, and we define it as $\S_P = (Q, \Sigma, q_0, F, \tau)$ where
\begin{enumerate} %todo considerare se cambiarlo in itemize
\item $Q = 2^{\matchstructs_P} \times \mathbb{D}_P \times \F_P \cup \set{\bot}$ is the
  finite set of states, \ie states are tuples of the form $\langle \Upsilon,
  \Delta, \Phi \rangle\in2^{\matchstructs_P} \times \mathbb{D}_P \times \F_P$, plus a
  sink state $\bot$;
\item $\Sigma$ is the input alphabet defined above;
  % in $\mathcal{A}_P \times [K]$,
  % where $K = \operatorname{window}(P)$;
\item the initial state $q_0 = \langle \Upsilon_0, \Delta_0, \Phi_0 \rangle$ is
  such that $\Upsilon_0$ is the set of initial matching structures of the
  existential statements of $P$ and, for all existential statements $\E$ of $P$,
  we have $\Delta_0(\E) = \emptyset$ and $\Phi_0(\E) = \mathbb{E}_\E$;
  % $\Upsilon = \Set{\M_{\E} | \E \in S \text{ and} \M_{\E}\text{ is
  % \emph{initial}}}$ and, for all
  % $\E \in S$, $\Delta(\E) = \emptyset$ and $\Phi(\E)
  % = \emptyset$;
\item $F \subseteq Q$ is the set of final states defined as:
  \[
    F = \Set{ \langle \Upsilon, \Delta, \Phi \rangle \in Q |
      \begin{gathered}
        \M \text{ is not \emph{active} for all } \M \in
        \Upsilon\\
        \text{and }\Delta(\E)=\emptyset\text{ for all }\E\text{ of } P
      \end{gathered}}
  \]
\item $\tau : Q \times \Sigma \rightarrow Q$ is the transition function that
  given a state $q=\langle \Upsilon, \Delta, \Phi \rangle$ and a symbol $\event
  = (A, \delta)$ computes the new state $\tau(q,\event)$. Let
  $\step^\E_\event(\Upsilon^\Rule_t)=\set{\M_\E \mid
  \M_\E\in\step_\event(\Upsilon^\Rule_t)}$. Moreover, let $\Psi^\Rule_t = \set{ \E |
  \M_{\E} \in \step_\event(\Upsilon^\Rule_t)}$. Then, the updated components of
  the state are based on what follows, where $W = \window(P)$:
  \begin{align*}
    \Upsilon' &= \step_\event(\Upsilon_\bot) \cup \bigcup \Set{
      \step_\event(\Upsilon^\Rule_t) |
      \text{$t\le W-\delta$ and 
      $\step_\event(\Upsilon^\Rule_t)$ is not \emph{closed}}} \\
    \Delta'(\E) &=\begin{cases}
        \step^\E_\event(\Upsilon^\Rule_t) & \text{where $t$ is the minimum such that $t> W-\delta$ and $\step^\E_\event(\Upsilon^\Rule_t)\ne\emptyset$} \\
        \step_\event(\Delta(\E)) & \text{if such $t$ does not exist}
      \end{cases}\\
    \Phi'(\E) &= \begin{cases}
      \mathbb{E}_\E\quad\text{if $\E\in\Psi(\E')$ for some $\E'$ such that 
      $\Delta'(\E')$ is \emph{closed}}  \\
      \Phi(\E) \setminus 
        \set{
          \E'\mid \exists t> W-\delta \suchdot \E'\in\Psi^\Rule_t 
          \land \E\not\in\Psi^\Rule_t
        } \quad \text{otherwise}
    \end{cases}
  \end{align*}

  Let $\Delta''(\E)=\Delta'(\E)$ unless there is an $\E'$ with $\E\in\Phi'(\E')$
  such that $\Delta'(\E')$ is \emph{closed}, in which case
  $\Delta''(\E)=\emptyset$. Then, $\tau(q,\event)=\seq{\Upsilon', \Delta'',
  \Phi'}$ if the following holds:
  \begin{enumerate}
  \item for every $\Upsilon^\Rule_t$, $\step_\event(\Upsilon^\Rule_t) \neq
    \emptyset$, and \label{dfa:delta:no-failed-step}
  \item for every synchronization rule $\Rule \equiv a_0[x_0=v_0] \rightarrow
    \E_1 \lor \dots \lor \E_n$ in $S$, if $\tokstart(x_0, v_0) \in A$, then
    there exists $\M_{\E_i} = (V,D,M,0) \in \Upsilon'$, 
    with $i \in \{1\dots n\}$, such that $\tokstart(a_0) \in M$;\label{dfa:delta:trigger-capture}
  \end{enumerate}
  Otherwise, $\tau(q,\event)=\bot$.
\end{enumerate}

The first component $\Upsilon$ of an automaton's state $q$ is a set of matching structures that keeps track of the occurred events in the last $\window(P)$ time points. The timestamp $t$ of any matching structure in $\Upsilon$ satisfies $t<\window(P)$. These matching structures evolve using the $\step_\event$ function until they become closed or their timestamp reaches $\window(P)$. 

Matching structures that reach $\window(P)$ get promoted to a new role where they record the pieces of existential statements not yet matched to satisfy all the trigger events of $\Rule$ that occurred before the last $\window(P)$ time points. However, the automaton does not store these matching structures in $\Upsilon$. Instead, it uses the function $\Delta$ mapping each existential statement $\E$ of a rule $\Rule$ to the set of matching structures for $\E$ with $t=\window(P)$. Thus, effectively summarizing events happening before this window to keep size under control.

When a set $\Upsilon^\Rule_t$ exceeds the bound $\window(P)$, the $\Delta$ function needs to be updated by merging the information from $\Upsilon^\Rule_t$ with the information already stored in $\Delta$. However, closing a set $\Delta(\E)$ does not necessarily mean that every event that triggered $\Rule$ satisfies $\Rule$. This is because there may be other sets, say $\Delta(\E')$, responsible for fulfilling the same rule $\Rule$, but for different trigger events. Therefore, closing $\Delta(\E)$ alone does not imply that $\Rule$ has been satisfied. Conversely, there may be cases where $\Delta(\E)$ and $\Delta(\E')$ contribute to match the same trigger events, and closing either set is enough to satisfy $\Rule$.

To address the issue of lost information when adding a set of matching structures to $\Delta$, we introduce the $\Phi$ function, mapping existential statements to sets of existential statements, as the third component of the automaton states. For an existential statement $\E$ and for every existential statement $\E' \in \Phi(\E)$, it holds that the set of matching structures $\Delta(\E')$ tracks the satisfaction of the same trigger events as the set $\Delta(\E)$. This way, when a set $\Delta(\E)$ is closed, we can discard its matching structures as well as the matching structures in $\Delta(\E')$.

In \cref{sec:soundness:completeness} we state and prove soundness and
completeness of the automaton construction. Now, instead, let us address the
size of the automaton.

Let us recall that we assumed that the timestamp of each event in an event sequence is bounded. However, it is worth noting that since events may have an empty set of actions, \cref{thm:soundness-completeness} can handle arbitrary event sequences as well, provided that we add suitable empty events. Let us now analyze the size of the automaton. 

\begin{thm}[Size of the automaton]
  Let $P=(\SV, S)$ be a timeline-based planning problem and let $\A_P$ be the
  associated automaton. Then, the size of $A_P$ is at most doubly-exponential in
  the size of $P$.
\end{thm}

\begin{proof}
We define $E$ as the overall number of existential statements in $P$, which is linear in the size of $P$. We can then observe that $\abs{\mathbb{D}_P} \in \O({(2^{\abs{\matchstructs_P}})}^E)= \O(2^{E\cdot\abs{\matchstructs_P}})$, thus the number of $\Delta$ functions is doubly exponential in the size of $P$.
Next, note that $\lvert\mathbb{F}_P\rvert \in \mathcal{O}({(2^E)}^E) = \mathcal{O}(2^{E^2})$. Then, $\abs{\S_P} \in \O(\abs{\Sigma}\cdot 2^{\abs{\matchstructs_P}})$ indicating that the size of $\S_P$ is at most exponential in the number of possible matching structures.
To bound this number, we define $N$ as the largest finite constant appearing in $P$ in any atom or value duration and $L$ as the length of the longest existential prefix of an existential statement occurring inside a rule of $P$. Note that $N$ is exponential in the size of $P$ since constants are expressed in binary, while $L \in \O(\abs{P})$.
We can then observe that the entries of a DBM for $P$, of which the number is quadratic in $L$, are constrained to take values within the interval $\ar{-N, N}$ (excluding the value $+\infty$), which size is linear in $N$. By \Cref{def:matching-structure}, it follows that $\abs{\matchstructs_P} \in \O(N^{L^2} \cdot 2^L \cdot \window(P))$ indicating that the number of matching structures is at most exponential in the size of $P$.
\end{proof}
Note that our automaton is the same size as the automaton built by Della Monica et al. in \cite{DellaMonicaGMS18}. However, while their automaton is nondeterministic, ours is deterministic: an essential property to achieve the \EXPTIME[2] optimal asymptotic complexity for the synthesis procedure.

\subsection{Soundness and Completeness}
\label{sec:soundness:completeness}
In the following, we present auxiliary notation, definitions, and essential lemmas for establishing the soundness and completeness of the automaton construction. For readability, we have included proofs in the appendix.

\begin{defi}[Run of a matching structure]
  Let $\evseq=\seq{\event_1,\ldots,\event_n}$ be a (possibly open) event
  sequence, and let $\M_\E$ be the initial matching structure of an existential
  statement $\E$. A \emph{run} of $\M_\E$ on $\evseq$ yielding a matching
  structure $\M_n$ is a sequence $\matchseq = \seq{\match_1, \ldots, \match_n}$
  of $I$-match events for the matching structures $\seq{\M_\E, \M_1, \ldots,
    \M_{n-1}}$, such that for every $i \in [1,\ldots,n]$, $\M_{i-1} \stepm[i]
  \M_i$. We write $\M_\E \runm \M_n$ when such run exists, or $\M_\E
  \xlongrightarrow{\evseq} \M_n$, if $\matchseq$ is not relevant.
\end{defi}

To link matching structures with the semantics of synchronization rules we
establish a connection between matching functions (\cref{def:matching-function})
and runs.

\begin{restatable}[Correspondence between runs and matching functions]{lem}{runfuncmap}
  \label{lemma:function-matching-run}
  Let $\evseq=\seq{\event_1,\ldots,\event_n}$ be a (possibly open) event
  sequence, and let $\M_\E$ be the initial matching structure of an existential
  statement $\E\equiv \exists a_1[x_1=v_1]\ldots a_k[x_k=v_k]\suchdot\clause$,
  with $\clause$ augmented with atoms $\tokstart(a_i) \before_{\dmin^{x_i=v_i},
    \dmax^{x_i=v_i}} \tokend(a_i)$, for every $0\leq i \leq k$. Then, there
  exists a run $\matchseq=\seq{\match_1, \ldots, \match_n}$ of $\M_\E$ on
  $\evseq$, yielding a matching structure $\M_n = \tuple{V, D_n, M_n, t_n}$, if
  and only if there exists a matching function $\gamma:M_n \to[1,\ldots,n]$ such
  that, for every atom of the form $T\before_{l,u} T'$ in $\clause$:
  \begin{enumerate}[label=(\Roman*)]
  \item \label{lemma:function-matching-run:entire-atom} if $T' \in M_n$, then also $T
    \in M_n$, $\gamma(T) \le \gamma(T')$, and $l \le
    \delta(\slice\evseq_{\gamma(T),\gamma(T')}) \le u$;
  \item \label{lemma:function-matching-run:partial-atom} if $T' \not\in M_n$, but $T \in
    M_n$, then $\delta(\slice\evseq_{\gamma(T),n}) \le u$.
  \end{enumerate}
  Furthermore, $\gamma$ and $\matchseq$ are such that for every $T \in M_n$, $T
  \in I_{\gamma(T)}$, \ie, they agree on the matching of the terms of $\M_n$. We
  write $M_\E \runm* M_n$, if $\gamma$ corresponds to a run of $\M_\E$, or
  $\evseq,\gamma\models \M_n$, if $\M_\E$ is clear from the context.
\end{restatable}

\begin{observation}
    \label{obs:matching-functions}
    Note that the existence of the matching function $\gamma$ stated by 
    \cref{lemma:function-matching-run}, if the corresponding matching structure is 
    closed, implies the satisfaction of the given existential statement, and 
    \viceversa.
\end{observation}

We now state the core technical result of the completeness proof, which ensures
no important details are lost when matching structures are discarded.

\begin{restatable}{lem}{superset}\label{lemma:matching-structure-superset}
  Let $\evseq =\langle\event_1,\dots,\event_n\rangle$ be an event sequence , let
  $\M_\E$ be the initial matching structure of some existential statement $\E$
  of a rule $\Rule$, and let $\M_r$ be an active matching structure resulting
  from a run $\M_\E \runm*[r] \M_r$, such that $\gamma_r(\tokstart(a_0)) = r$.
  If there exists a run $\M_\E \runm*[s] \M_s$, such that
  $\gamma_s(\tokstart(a_0)) < r$, then there exists a run $\M_\E \runm* \M$,
  such that $\gamma(\tokstart(a_0)) = \gamma_s(\tokstart(a_0))$ and $\M$ matches
  at least as many tokens as $\M_r$.
\end{restatable}

The last needed notion is that of \emph{residual} matching structure, which is
an active matching structure with only infinite bounds.

\begin{defi}[Residual matching structure]\label{def:residual-matching-structure}
  A matching structure $\M = (V, D, M, t)$ is \emph{residual} if it is
  \emph{active} and for every $T \in M$ and $T' \in \overline{M}$, $D[T',T] = +\infty$.
\end{defi}

In other words, $\M$ does not impose any finite upper bound on the distance at
which terms yet to be matched may appear relative to those already matched. The
definition implies that for any residual matching structure, denoted as $\hat\M
= (V, D, M, t)$, every event $\event = (A, \delta)$ is admissible. Additionally,
it is never the case that $\tokstart(a) \in M$ and $\tokend(a) \in \overline{M}$
for any quantified token $a[x = v]$ of $\E$, given that such terms always have a
finite upper bound in $D$ that is at least as strict as the value $\dmax^{x=v}$.
As a result, the ``if'' direction of \Cref{def:match-event:good-match:end} in
the \Cref{def:match-event} of $I$-match never applies to $\hat\M$ for any event
$\event$. Therefore, every event is a valid $\emptyset$-match event for
$\hat\M$.

\begin{observation}\label{obs:residual-run}
  Let $\M_\E \xlongrightarrow{\evseq_1,\matchseq_1}\hat\M$ be a run of the
  \emph{initial} matching structure $\M_\E$, on an event sequence $\evseq_1$,
  yielding a \emph{residual} matching structure $\hat\M$. Then, for any event
  sequence $\evseq_2$, there exists a run $\M_\E
  \xlongrightarrow{\evseq_1\evseq_2,\matchseq_1\matchseq_2} \hat\M'$ such that
  every $I$-match event in $\matchseq_2$ is an $\emptyset$-match event and
  $\hat\M'$ differs from $\hat\M$ by at most the value of the component $t$.
\end{observation}

Consequently, whenever a residual matching structure appears in a run, it has
the potential to remain there indefinitely, which is why it is called
\emph{residual}.

\begin{restatable}[Existence of residual matching structure]{lem}{residualexist}
  \label{lemma:residual-matching-structure}
  Let $\evseq = \seq{\event_1, \ldots, \event_n}$ be an event sequence, and let
  $\M_n$ be an \emph{active} matching structure such that $\evseq, \gamma
  \models \M_n$ and
  $\delta(\slice\evseq_{\gamma(\tokstart(a_0)),n})>\window(P)$. If we consider
  the intermediate matching structures $\seq{\M_1, \ldots, \M_{n-1}}$ of the run
  $\M_\E \runm* \M_n$, then there exists a position $\gamma(\tokstart(a_0)) \leq
  k < n$ such that $\M_k$ is a \emph{residual} matching structure.
\end{restatable}

We are now ready to prove the final result.

\begin{restatable}[Soundness and completeness]{thm}{soundnessCompleteness}
  \label{thm:soundness-completeness}
  Let $P=(\SV, S)$ be a timeline-based planning problem and let $\A_P$ be the
  associated automaton. Then, any event sequence $\evseq$ is a solution plan for
  $P$ if and only if $\evseq$ is accepted by $\A_P$.
\end{restatable}

%%% Local Variables:
%%% TeX-master: "../lmcs-gandalf22.tex"
%%% End:

%!TeX root = ../lmcs-gandalf22.tex

\section{Controller synthesis}
\label{sec:games}

We leverage the deterministic automaton constructed in the previous section to establish a deterministic arena that enables us to solve a reachability game and determine whether a controller exists. If a controller exists, our procedure allows its synthesis.

\subsection{From the automaton to the arena}

Let $G=(\SV_C, \SV_E, \S, \D)$ be a timeline-based game. The automaton construction we used considered a planning problem with a single set of synchronization rules, while in $G$, we have to account for the roles of both $\S$ and $\D$.

To address this, we define $A_\S$ and $A_\D$ as the deterministic automata constructed over the timeline-based planning problems $P_\S=(\SV_C\cup\SV_E, \S)$ and $P_\D=(\SV_C\cup\SV_E, \D)$, respectively. We then construct the automaton $A_G$ by taking the union of $A_\S$ with the complement of $A_\D$ ($\overline{A_\D}$). Note that these are standard automata-theoretic operations over DFAs.
An accepting run of $A_G$ represents either a plan that violates the domain rules or a plan that adheres to domain and system rules, according to the definition of winning strategy in \cref{def:games:winning-strategy}.
Furthermore, $A_G$ is deterministic, and its size only polynomially increases when built from $A_\D$ and $A_\S$. 

The $A_G$ automaton is not immediately applicable as a game arena since its transitions' labels only reflect events, not game moves. In $A_G$, a single transition can correspond to various combinations of rounds due to the absence of $\wait(\delta)$ moves in the transition's label. For example, an event $\event = (A, 5)$ can arise from either a $\wait(5)$ move by \charlie, followed by a $\play(5, A)$ move by $\eve$, or any $\wait(\delta)$ move with $\delta > 5$ followed by a $\play(5, A)$ move. To obtain a suitable game arena, we need to modify $A_G$ further. 

Let $A_G = (Q,\Sigma, q_0, F, \tau)$ be the automaton constructed as described above. Formally, we define a new automaton $A_G' = (Q,\Sigma,q_0, F,\tau')$ where $\tau'$ is a partial transition function, meaning that the automaton is now incomplete. The function $\tau'$ agrees with $\tau$ on all transitions except those of the form $\tau(q,(\actions,\delta))$, where $\delta>1$ and $\actions$ contains a $\tokend(x,v)$ action with $x\in\SV_C$. In such cases, the transition is undefined in $A_G'$.
An example is shown in Figure \ref{fig:constructions} (left). Note that this removal does not alter the set of plans accepted by the automaton since for each transition $\tau(q,(\actions,\delta))=q'$ with $\delta > 1$, there exist two transitions $\tau(q,(\emptyset,\delta-1))=q''$ and $\tau(q'',(\actions,1))=q'$ in $A_G'$.

% Figure environment removed

To make the game rounds and moves explicit, we can transform the automaton by splitting each transition into four transitions representing the four moves of the two rounds. Starting from the incomplete automaton $A_G'=(Q,\Sigma, q_0, F, \tau')$, we define a new automaton $A_G^a=(Q^a,\Sigma^a, q_0^a, F^a, \tau^a)$ as the game arena.

\begin{enumerate}
  \item The set of states $Q^a$ is given by $Q^a=Q\cup\set{q_\delta\mid 1\le\delta\le d}\cup\set{q_{\delta,A}\mid 1\le\delta\le d, A\subseteq \mathsf{A}}$.
  \item The alphabet $\Sigma^a$ is defined as $\Sigma^a=\moves_C\cup\moves_E$, which corresponds to the set of moves of the two players.
  \item The initial and final states of $A_G^a$ are $q_0^a=q_0$ and $F^a=F$, respectively.
  \item The partial transition function $\tau^a$ is defined as follows. Let $w=\tau(q,\event)$ with $\event=(\actions,\delta)$. We distinguish the cases where $\delta=1$ or $\delta>1$.
    \begin{enumerate}
      \item if $\delta=1$, let $\actions_C\subseteq\actions$ and
      $\actions_E\subseteq\actions$ be the set of actions in $\actions$ playable
      by \charlie and by \eve, respectively. Then:
      \begin{enumerate}
        \item $\tau(q,\play(\actions_C^e))=q_{1,\actions_C^e}$, where 
          $\actions_C^e$ is the set of \emph{ending} actions in $\actions_C$;
        \item $\tau(q_{1,\actions_C^e},\play(\actions_E^e))=q_{1,\actions_C^e\cup\actions_E^e}$, where 
          $\actions_E^e$ is the set of \emph{ending} actions in $\actions_E$;
        \item $\tau(q_{1,\actions_C^e\cup\actions_E^e},\play(\actions_C^s))=q_{1,\actions_C^e\cup\actions_E^e\cup\actions_C^s}$, where 
          $\actions_C^s$ is the set of \emph{starting} actions in $\actions_C$;
        \item $\tau(q_{1,\actions_C^e\cup\actions_E^e\cup\actions_C^s},\play(\actions_E^s))=w$, where 
          $\actions_E^s$ is the set of \emph{starting} actions in $\actions_E$;
      \end{enumerate}
      Here, the states mentioned are added to $Q^a$ as needed.
      \item if $\delta>1$, let $\actions_C\subseteq\actions$ and
      $\actions_E\subseteq\actions$ be the set of actions in $\actions$ playable
      by \charlie and by \eve, respectively. Note that by construction,
      $\actions_C$ only contains \emph{starting} actions. Then:
      \begin{enumerate}
        \item $\tau(q,\wait(\delta_C))=q_{\delta_C}$ for all 
          $\delta\le\delta_C\le d$;
        \item $\tau(q_{\delta_C},\play(\delta, \actions_E^e))=q_{\delta,\actions_E^e}$
          where $\actions_E^e$ is the set of \emph{ending} actions in
          $\actions_E$;
        \item $\tau(q_{\delta,\actions_E^e},\play(\actions_C))=q_{\delta,\actions_E^e\cup\actions_C}$;
        \item $\tau(q_{\delta,\actions_E^e\cup\actions_C},\play(\actions_E^s))=w$ where
          $\actions_E^s$ is the set of \emph{starting} actions in $\actions_E$;
      \end{enumerate}
      where the mentioned states are added to $Q^a$ as needed.
    \end{enumerate}
    All the transitions not explicitly defined above are undefined.
\end{enumerate}

We present a graphical illustration of the above construction in \cref{fig:constructions}. It is worth noting that the automaton preserves the structure of the original automaton $A_G$. For any state, $q\in Q$ and event $\event=(A,\delta)$, any sequence of moves that would result in appending $\event$ to the partial plan (see \cref{def:games:round-outcome}) reaches the same state $w$ in $A^a_G$ as it does in $A_G$ by reading $\event$. Therefore, we can consider $A^a_G$ as being able to read event sequences, even though its alphabet is different. We use the notation $[\evseq]$ to represent the state $q\in Q^a$ reached by reading $\evseq$ in $A^a_G$.
Furthermore, note that, with a slight abuse of notation, any play $\bar\rho$ in the game $G$ is a readable word by the automaton $A_G^a$. Thus, we can establish the following result.
\begin{thm}
  \label{thm:arena-soundness}
  If $G$ is a timeline-based game, for any play $\bar\rho$ for $G$, $\bar\rho$
  is successful if and only if it is accepted by $A_G^a$.
\end{thm}

\subsection{Computing the Winning Strategy and Building the Controller}
Let us define $Q^a_C \subset Q^a$ as the set of states in which \charlie can make a move, and $Q^a_E = Q^a \setminus Q^a_C$ as the set of states where \eve can make a move. Additionally, we define $E=\{(q, q') \in Q^a\times Q^a \mid \exists \event \mathrel{.} \tau^a(q, \event) = q'\}$ as the set of edges in $A^a_G$. By solving the reachability game $(G_R, \mathcal{W})$, where $G_R = (Q^a, Q^a_C, E^a)$ and $\mathcal{W} = \{R \subseteq Q^a \mid R \cap F^a \neq \emptyset \}$, we aim to determine the winning region $W_C$ and the winning strategy $s_C$ for \charlie, provided they exist. In the following discussion, we will show that the winning strategy $\sigma_C$ for the timeline-based game $G$ is derivable from strategy $s_C$ when $q^a_0 \in W_C$.

To determine the winning region $W_C$, we use the well-known \emph{attractor} construction. We are interested to the attractor set of $F^a$ for \charlie, written $Attr_C(F^a)$, thus given $i \ge 0$ we compute the set of states from which \charlie can reach a state $q \in F^a$ within $i$ moves, defined as $Attr_C^i(F^a)$:
\begin{align*} % Indented the last two lines. If you do not agree with this, let us know.
  Attr^0_C (F^a) ={}& F^a \\
  Attr^{i+1}_C (F^a) ={}& Attr^i_C (F^a) \\
  &\cup \set{ q^a \in Q^a_C \, | \, \exists r \big((q^a, r) \in E \land r \in Attr^i_C (F^a)\big) } \\
  &\cup \set{ q^a \in Q^a_E \, | \, \forall r \big((q^a, r) \in E \implies r \in Attr^i_C (F^a) \big) }.
\end{align*}

The sequence $Attr^0_C (F^a) \subseteq Attr^1_C (F^a) \subseteq Attr^2_C (F^a) \subseteq \ldots$ eventually becomes stationary for some index $k \leq \lvert Q^a \rvert$, hence we can define $Attr_C (F^a) = \bigcup^{\lvert Q^a \rvert}_{i=0} Attr^i_C(F^a)$ as the attractor set. Note that $W_C = Attr_C(F^a)$ is a known fact for which proof is available in \cite{Thomas2008}.
Next, we want that $q_0^a\in W_C$ since we are interested in a winning strategy $\sigma_C$ for the timeline-based game $G$. If it is the case, by defining $s_C(q) = \mu$ for any $\mu$ such that $\tau^a(q,\mu)=q'$ with $q,q'\in W_C$, which is guaranteed to exist by the attractor construction, we can define $\sigma_C$ for \charlie in $G$ as $\sigma_C(\evseq) = s_C([\evseq])$ for any event sequence $\evseq$. We prove this claim in the following:

\begin{thm}
    \label{thm:winning-region-soundness-completeness}
  Given $A_G^a=(Q^a,\Sigma^a, q_0^a, F^a, \tau^a)$, $q_0^a\in W_C$ if and only
  if $\sigma_C$ is a winning strategy for \charlie for $G$.
\end{thm}
\begin{proof}

  \proofif From the definition of a winning strategy for \charlie in $G$ (\cref{def:games:winning-strategy}), we know that for every admissible strategy $\sigma_E$ for \eve, there exists $n \ge 0$ such that the play $\rounds_n(\strategy_C,\strategy_E)$ is successful. By the soundness of the arena construction (\cref{thm:arena-soundness}), we know that the event sequence $\evseq_n$ representing $\rounds_n(\strategy_C,\strategy_E)$, when seen as a word over $\Sigma^a$, is accepted by $A_G^a$. Therefore, $\evseq_n$ reaches a state in the set $F^a$ starting from $q_0^a$. By the definition of the reachability game, this means that $q_0^a\in W_C$. Thus, we have proved that if $\sigma_C$ is a winning strategy \charlie in $G$, then $q_0^a\in W_C$.
  
  \proofonlyif If $q_0^a\in W_C$, then by definition, $s_C$ is a winning strategy for \charlie in the reachability game over the arena $A_G^a$. Hence, any word over $\Sigma^a$ obtained by playing with $s_C$ is accepted by $A_G^a$, and therefore, by the soundness of the arena construction (\cref{thm:arena-soundness}), any corresponding play $\rounds$ is successful in $G$. Now, recall that $\sigma_C(\evseq)=s_C([\evseq])$ for any event sequence $\evseq$. Hence, $\rounds=\rounds(\sigma_C,\sigma_E)$ for some strategy $\sigma_E$ of \eve. As a result, we can conclude that $\sigma_C$ is a winning strategy for \charlie in $G$.
\end{proof}

Finally, we build a Controller that implements the winning strategy $\sigma_C$, 
provided it exists. First, by \cref{thm:winning-region-soundness-completeness}, 
the existence of $\sigma_C$ implies that $q^a_0 \in W_C$. Next, we define the 
following Moore machine (\cref{def:moore-machine}) based on $s_C$:

\begin{defi}[Controller]
    \label{def:controller-implementation}
Given $A_G^a=(Q^a,\Sigma^a, q_0^a, F^a, \tau^a)$, we define a Controller as $\mathcal{M} = (Q, \Sigma, \Gamma, q_0, \delta, \tau)$, where $Q = Q^a_C \cap W_C$ represents the set of states, $q_{0} = q_0^a$ is the initial state, $\Sigma = \moves_E$ is the input alphabet, $\Gamma = \moves_C$ is the output alphabet, $\delta : Q \times \Sigma \rightarrow Q$ is the transition function, and $\tau : Q \rightarrow \Gamma$ is the output function. The transition function $\delta$ and the output function $\tau$ are defined as follows:
    \begin{align*} 
        \delta(q_C, \move_E) &= \tau^a(s_C(q_C), \move_E) \\ \tau(q_C) &= s_C(q_C).
    \end{align*} 
\end{defi}

Note that by construction the states of $\mathcal{M}$ belong to the winning
region $W_C$ of $A_G^a$, and $\delta$ follows the transition function $\tau^a$
of $A^a_G$. Hence, the output of $\mathcal{M}$ after reading a word $\evseq$ is
exactly $\sigma_C(\evseq)=s_C([\evseq])$ and $\mathcal{M}$ implements
$\sigma_C$, which is a winning strategy by
\cref{thm:winning-region-soundness-completeness}.

%!TeX root = ../lmcs-gandalf22.tex

\section{Conclusions and Future Work}
\label{sec:conclusions}

Our paper presents an effective procedure for synthesizing controllers for timeline-based games, whereas previously, only a proof of the \EXPTIME[2]-completeness of the problem of determining the existence of a strategy was available in the literature. We use a novel construction of a \emph{deterministic} automaton of doubly-exponential (thus optimal) size, which is then adapted to serve as the arena for the game. Then, with standard methods, we solve a reachability game on the arena to effectively compute the winning strategy for the game, if it exists.

This work paves the way for future developments. First, the procedure provided in this paper can be realistically implemented and tested. It is conceivable, though, that to avoid the state explosion problem due to the doubly-exponential size of the automaton, it will be necessary to apply \emph{symbolic techniques}. Moreover, an implementation would also need a concrete syntax to specify timeline-based games. Existing languages supported by timeline-based systems (\eg NDDL~\cite{CestaO96} or ANML~\cite{SmithFC08}) might be inadequate for this purpose. Next, as in the case of \LTL, the high complexity makes one wonder whether simpler but still expressive fragments can be found. One possibility might be restricting the synchronization rules to only talk about the \emph{past} concerning the rule's trigger. For co-safety properties (\ie properties expressing the fact that something good will eventually happen) expressed in pure-past \LTL, the realizability problem goes down to being \EXPTIME-complete, and by analogy, this might happen to pure-past timeline-based games as well.


\section*{Acknowledgements}
Luca Geatti and Angelo Montanari acknowledge the support from the 2024 Italian
INdAM-GNCS project ``Certificazione, monitoraggio, ed interpretabilità in
sistemi di intelligenza artificiale'', ref. no. CUP E53C23001670001 as well as
that from the Interconnected Nord-Est Innovation Ecosystem (iNEST), which
received funding from the European Union Next-GenerationEU (PIANO NAZIONALE DI
RIPRESA E RESILIENZA (PNRR) -- MISSIONE 4 COMPONENTE 2, INVESTIMENTO 1.5 -- D.D.
1058 23/06/2022, ECS00000043). In addition, Angelo Montanari acknowledges the
support from the MUR PNRR project FAIR - Future AI Research (PE00000013) also
funded by the European Union Next-GenerationEU. This manuscript reflects only
the authors’ views and opinions, neither the European Union n or the European
Commission can be considered responsible for them. Nicola Gigante acknowledges
the support of the PURPLE project, 1st Open Call for Innovators of the AIPlan4EU
H2020 project, a project funded by EU Horizon 2020 research and innovation
programme under GA n.\ 101016442.

\bibliographystyle{alphaurl}
\bibliography{biblio}

%!TeX root = ../lmcs-gandalf22.tex

\section{Appendix}
%TODO CAPIRE CHE ALTRE DEFINIZIONI DI CAP.2 DI VALENTINO AGGIUNGERE
% DEFINIZIONI MANCANTI:
% - matching function per rule graphs e relazione \models
% - graph concatenation \graphconcat

\runfuncmap*

\begin{proof}
  \proofif%
  We proceed by induction on the length of the event sequence $\evseq =
  \seq{\event_1, \ldots, \event_n}$.
  \begin{description}[before={\renewcommand\makelabel[1]{\bfseries ##1.}},
    labelsep=*, leftmargin=*]
  \item[Base case] % base case: empty sequence
    If $n = 0$, the only well defined function on an empty codomain is the
    function $\gamma_0: \emptyset \to \emptyset$ with an empty domain, which
    vacuously satisfies the definition of matching function and
    \Cref{lemma:function-matching-run:entire-atom,lemma:function-matching-run:partial-atom}.
    Then, the only run of $\M_\E = (V, D, \emptyset, 0)$ on an empty event
    sequence $\evseq$ is the empty run $\matchseq$ yielding $\M_\E$ itself,
    which vacuously satisfies the definition of run.
    % Inductive hypothesis: if there exists a matching function $\gamma:M_{n-1}
    % \to [1,\ldots,n-1]$ satisfying
    % \Cref{lemma:function-matching-run:entire-atom,lemma:function-matching-run:partial-atom},
    % then there exists a run $\seq{I_1, \ldots, I_{n-1}}$ of $\M_\E$ on
    % $\seq{\event_1, \ldots, \event_{n-1}}$, yielding a matching structure
    % $\M_{n-1}$.
  \item[Inductive step] % inductive step for n
    Let $\gamma:M_{n} \to [1,\ldots, n]$ be a matching function satisfying
    \Cref{lemma:function-matching-run:entire-atom,lemma:function-matching-run:partial-atom},
    and let $\restrict{\gamma}^{<n}:M_{n-1}\to[1,\ldots,n-1]$ be the restriction
    of $\gamma$ on the domain $M_{n-1}$ defined as the inverse image of
    $[1,\ldots,n-1]$ under $\gamma$, \ie, $M_{n-1} =
    \gamma^{-1}([1,\ldots,n-1])$. $\restrict{\gamma}^{<n}$ is a matching
    function for the event sequence $\slice\evseq_{1,n-1}$ and satisfies
    \cref{lemma:function-matching-run:entire-atom,lemma:function-matching-run:partial-atom}.
    %
    % If $T' \in M_{n-1}$, then $T' \in M_{n}$ and $\gamma(T') < n$, so that $T
    % \in M_{n}$, $\gamma(T) \leq \gamma(T') < n$, and $T \in M_{n-1}$.
    % Furthermore, $\restrict{\gamma}^{<n}(T) \le \restrict{\gamma}^{<n}(T')$
    % and $l \le
    % \delta(\slice\evseq_{\restrict{\gamma}^{<n}(T),\restrict{\gamma}^{<n}(T')})
    % \le u$, because $\restrict{\gamma}^{<n}$ is a restriction of $\gamma$. If
    % $T' \not\in M_{n-1}$ there can be two cases, either $T'$ belongs to the
    % domain of $\gamma$, \ie, $T'\in M_n$, or not. In the first case,
    % $\gamma(T')$ must be equal to $n$ and
    % $\delta(\slice\evseq_{\restrict{\gamma}^{<n}(T),n-1}) \leq
    % \delta(\slice\evseq_{\restrict{\gamma}^{<n}(T),\gamma(T')}) \leq u$, by
    % \Cref{lemma:function-matching-run:entire-atom} for $\gamma$. In the latter
    % case, $\delta(\slice\evseq_{\restrict{\gamma}^{<n}(T),n-1}) \leq
    % \delta(\slice\evseq_{\restrict{\gamma}^{<n}(T),n}) \leq u$, by
    % \Cref{lemma:function-matching-run:partial-atom} for $\gamma$.
    %
    By the inductive hypothesis, there exists a run $\seq{I_1, \ldots, I_{n-1}}$
    of $\M_\E$ on $\slice\evseq_{1,n-1}$, yielding a matching structure
    $\M_{n-1} = (V, D_{n-1}, M_{n-1}, t_{n-1})$. Let $I_n = \gamma^{-1}(n)$, and
    note that $I_n \subseteq \overline{M_{n-1}}$. We show that $\event_n = (A_n,
    \delta_n)$ is an \emph{$I_n$-match} event for $\M_{n-1}$ by breaking the
    proof in steps.

    \statement{$\event_n$ is an \emph{admissible} event for $\M_{n-1}$} Let $T
    \in M_{n-1}$ and $T' \not\in M_{n-1}$. If $D_{n-1}[T',T] = +\infty$,
    $\delta_n \le D_{n-1}[T',T]$ trivially holds. Otherwise, there exists an
    atom $T \before_{l,u} T'$ in $\clause$ and $D_{n-1}[T',T] = u -
    \delta(\slice\evseq_{\restrict\gamma^{<n}(T),n-1})$. We consider two cases
    based on whether $T'$ belongs to the domain of $\gamma$, or not. In the
    first case, $\gamma(T') = n$ and $\delta(\slice\evseq_{\gamma(T),n-1}) +
    \delta_n = \delta(\slice\evseq_{\gamma(T),\gamma(T')}) \le u$, by
    \Cref{lemma:function-matching-run:entire-atom}. In the second case,
    $\delta(\slice\evseq_{\gamma(T),n-1}) + \delta_n =
    \delta(\slice\evseq_{\gamma(T),n}) \le u$, by
    \Cref{lemma:function-matching-run:partial-atom}. In either case, $\delta_n
    \leq u - \delta(\slice\evseq_{\gamma(T),n-1}) = D_{n-1}[T',T]$.

    \statement{\Cref{def:match-event:good-match:start} of
      \Cref{def:match-event}} Let $a[x = v]$ be a quantified token of $\E$. If
    $\tokstart(a) \in I_n$, then $\gamma(\tokstart(a)) = n$ and by definition of
    matching function $\tokstart(x,v)\in A_n$.

    \statement{\Cref{def:match-event:good-match:end} of
      \Cref{def:match-event}}\proofif Let $\tokend(a) \not\in M_{n-1}$ be a
    possible candidate for inclusion in $I_n$. If $\tokstart(a) \in M_{n-1}$ and
    $\tokend(x, v) \in A_n$, then $\tokend(x, v)$ ends the token started at
    $\event_{\restrict\gamma^{<n}(\tokstart(a))}$; otherwise, there would exist
    $\event_i = \pair{A_i, \delta_i}$ prior to $\event_n$ such that $\tokend(x,
    v) \in A_i$, contradicting that $\restrict\gamma^{<n}$ is undefined on
    $\tokend(a)$. By definition of matching function, since $\tokend(x, v) \in
    A_n$ ends the token started at $\event_{\gamma(\tokstart(a))}$, we have
    $\gamma(\tokend(a)) = n$ and $\tokend(a) \in I_n$.
    % Versione di Valentino:
    %
    % . If this was not the case, there would exist an event $\event_i =
    % \pair{A_i, \delta_i}$ in
    % $\slice\evseq_{\restrict\gamma^{<n}(\tokstart(a)),n-1}$ (prior to
    % $\event_n$), such that $\tokend(x, v) \in A_i$. Since a matching function
    % correctly identifies the endpoints of the same tokens
    % (\Cref{def:matching-function}), $\restrict\gamma^{<n}$ would be defined on
    % $\tokend(a)$ as $\restrict\gamma^{<n}(\tokend(a)) = i$, contradicting the
    % hypothesis that $\tokend(a) \not\in M_{n-1}$. Again by definition of
    % matching function, since $\tokend(x, v) \in A_n$ ends the token started at
    % $\event_{\restrict\gamma^{<n}(\tokstart(a))} =
    % \event_{\gamma(\tokstart(a))}$, $\gamma$ is such that $\gamma(\tokend(a))
    % = n$ and $\tokend(a) \in I_n$.

    \statement{\Cref{def:match-event:good-match:end} of
      \Cref{def:match-event}}\proofonlyif If $\tokend(a) \in I_n$, then by
    definition of matching function $\tokend(x,v) \in A_n$. Furthermore, since
    $\tokend(a) \in M_n$, \Cref{lemma:function-matching-run:entire-atom} gives
    $\gamma(\tokstart(a)) \le \gamma(\tokend(a))$ for the atom $\tokstart(a)
    \before_{l,u} \tokend(a)$ in $\clause$. By definition of event sequence,
    $\tokstart(x,v)$ and $\tokend(x,v)$ cannot appear in the same event; hence,
    $\gamma(\tokstart(a)) < \gamma(\tokend(a)) = n$ and $\tokstart(a) \in
    M_{n-1}$.

    \statement{\Cref{def:match-event:preceding-terms} of \Cref{def:match-event}}
    Let $T$ be a term in $I_n$, and let $T'\in V$ be any other term such that
    $D_{n-1}[T',T] \leq 0$. Then, $D_{n-1}[T',T]$ can either be the lower bound
    of an atom $T' \before_{l,u} T$, or the upper bound of an atom $T
    \before_{l,u} T'$ in $\clause$. In the first case, we can directly conclude
    that $T' \in M_{n-1} \cup I_n$, because $T' \in M_n$ by
    \Cref{lemma:function-matching-run:entire-atom} of $\gamma$ and $M_n =
    M_{n-1} \cup I_n$ by definition of $M_{n-1}$ and $I_n$. In the second case,
    note that $D_{n-1}[T',T] = u$, \ie, it has never been decremented because $T
    \not\in M_{n-1}$, and that upper bounds $u$ can never be negative. Thus, $u$
    is equal to $0$ and $\gamma$ satisfies $0 \le
    \delta(\slice\evseq_{\gamma(T),\gamma(T')}) \le 0$
    (\Cref{lemma:function-matching-run:entire-atom}), meaning that $\gamma(T') =
    \gamma(T)$ and $T' \in I_n$.
    % Alternativa scritta da Renato:
    %
    % Let $T' \in V$ be any other term such that $D_{n-1}[T',T] \le 0$. If
    % $D_{n-1}[T',T]$ is the lower bound of an atom $T' \before_{l,u} T$ or the
    % upper bound of an atom $T \before_{l,u} T'$, then $T' \in M_n$ by Lemma 1
    % and $M_n = M_{n-1} \cup I_n$. Otherwise, $u = D_{n-1}[T',T] = 0$ and Lemma 1
    % gives $\gamma(T') = \gamma(T)$; hence $T' \in I_n$.

    \statement{\Cref{def:match-event:lower-bounds} of \Cref{def:match-event}}
    Let $T \in I_n$ and $T' \in M_{n-1}$. $D_{n-1}[T',T]$ cannot be the upper
    bound of an atom $T \before_{l,u} T'$; otherwise,
    \Cref{lemma:function-matching-run:entire-atom} would imply $T \in M_{n-1}$,
    contradicting $T \in I_n$. Thus, $D_{n-1}[T',T]$ must either represent the
    lower bound of an atom $T' \before_{l,u} T$ in $\clause$, or be equal to
    $+\infty$. In the latter case, $\delta_n \ge -D_{n-1}[T',T]$ trivially
    holds. In the former case, $D_{n-1}[T',T] = -l +
    \delta(\slice\evseq_{\restrict\gamma^{<n}(T'),n-1})$. Since $\gamma(T) = n$,
    we have $\delta(\slice\evseq_{\gamma(T'), \gamma(T)}) =
    \delta(\slice\evseq_{\gamma(T'), n}) =
    \delta(\slice\evseq_{\gamma(T'),n-1})+\delta_n \ge l$. Hence, $\delta_n \ge
    l - \delta(\slice\evseq_{\gamma(T'),n-1}) = -D_{n-1}[T',T]$.

    \statement{\Cref{def:match-event:zero-no-bounds} of \Cref{def:match-event}}
    Let $T, T' \in I_n$ be two distinct terms. Then, $\gamma(T') = \gamma(T)$
    and $\delta(\slice\evseq_{\gamma(T'),\gamma(T)}) = 0 $. If $T \before_{l,u}
    T'$ (resp., $T' \before_{l,u} T$) belongs to $\clause$, then $D_{n-1}[T,T']$
    (resp., $D_{n-1}[T',T]$) is the lower bound $l$ and equals 0 by
    \Cref{lemma:function-matching-run:entire-atom}. Otherwise, $D_{n-1}[T,T'] =
    D_{n-1}[T',T] = +\infty$.
    % Versione di Valentino:
    %
    % Let $T'$ be any other term in $I_n$ different from $T$, and note that
    % $\gamma(T') = \gamma(T)$, so that
    % $\delta(\slice\evseq_{\gamma(T'),\gamma(T)}) =
    % \delta(\slice\evseq_{\gamma(T),\gamma(T')}) = 0 $. Since neither $T$ nor
    % $T'$ belongs to $M_{n-1}$, both $D_{n-1}[T', T]$ and $D_{n-1}[T,T']$ are
    % equal to their initial value. If the atom $T \before_{l,u} T'$ belongs to
    % $\clause$, then $D_{n-1}[T,T']$ is the lower bound $l$, and since $\gamma$
    % satisfies $l \le \delta(\slice\evseq_{\gamma(T'), \gamma(T)}) = 0$, it
    % must be the case that $D_{n-1}[T,T'] = 0$. If instead is the atom $T'
    % \before_{l,u} T$ that belongs to $\clause$, $D_{n-1}[T',T]$ is the lower
    % bound $l$ and $D_{n-1}[T',T] = 0$ for the same argument. Lastly, if there
    % is no atom in $\clause$ relating $T$ and $T'$, then $D_{n-1}[T,T'] =
    % D_{n-1}[T',T] = +\infty$.

    Hence, $\M_{n-1} \stepm[n] \M_n$ is well defined and $\seq{I_1, \ldots,
      I_n}$ is a run of $\M_\E$ on $\evseq$ yielding $\M_n$.
  \end{description}

  
  \proofonlyif We proceed by induction on the length of the event sequence $\evseq =
  \seq{\event_1, \ldots, \event_n}$.

  \begin{description}[before={\renewcommand\makelabel[1]{\bfseries ##1.}},
    labelsep=*, leftmargin=*]
  \item[Base case] An empty run $\matchseq$ yields $\M_\E = (V, D, \emptyset,
    0)$ itself. Then the function $\gamma_0: \emptyset \to \emptyset$ vacuously
    satisfies the definition of matching function and
    \Cref{lemma:function-matching-run:entire-atom,lemma:function-matching-run:partial-atom}.
  \item[Inductive step]
    Let $\matchseq = \seq{I_1, \ldots, I_n}$ be a run of $\M_\E$ on $\evseq$,
    yielding a matching structure $\M_n = (V, D_n, M_n, t_n)$. Note that
    $\slice\matchseq_{1,n-1}$ is a run of $\M_\E$ on $\slice\evseq_{1,n-1}$
    yielding a matching structure $\M_{n-1} = (V, D_{n-1}, M_{n-1}, t_{n-1})$.
    By the inductive hypothesis, there exists a matching function $\gamma_{<n}:
    M_{n-1} \to [1, \ldots, n-1]$ satisfying
    \Cref{lemma:function-matching-run:entire-atom,lemma:function-matching-run:partial-atom}.
    Let $\gamma: M_{n} \to [1,\ldots, n]$ be the extension of $\gamma_{<n}$ to
    $M_n$, such that $\gamma(T) = n$, for all $T\in I_n$.

    \statement{$\gamma$ is a matching function}
    \Cref{def:matching-function:nodes,def:matching-function:tokens} hold for all
    the terms already present in the domain of $\gamma_{<n}$. For every term in
    $I_n$, \Cref{def:matching-function:nodes} for $\gamma$ follows from
    \Cref{def:match-event:good-match} of $I_n$-match event. Let $\tokstart(a),
    \tokend(a) \in M_n$ be two terms not both already present in $M_{n-1}$,
    meaning that $\tokstart(a) \in M_{n-1}$ and $\tokend(a) \in I_n$, for some
    quantified token $a[x=v]$ in $\E$. By definition of $I_n$-match event,
    $\event_n = (A_n, \delta_n)$ is such that $\tokend(x,v) \in A_n$ and no
    other event in $\slice\event_{\gamma_{<n}(T),n-1}$ contains an action
    $\tokend(x, v)$, otherwise $\tokend(a)$ would already belong to $M_{n-1}$
    (by \Cref{def:match-event:good-match:end} of $I$-match event).
    % according to \Cref{def:match-event:good-match:end} of the definition of
    % $I$-match event.
    Thus, $\tokend(x, v) \in A_n$ ends the token started at
    $\event_{\gamma(\tokstart(a))}$, and $\gamma(\tokstart(a))$ and
    $\gamma(\tokend(a))$ correctly identify the endpoints of such token.

    \statement{\Cref{lemma:function-matching-run:entire-atom} of
      \Cref{lemma:function-matching-run}} Let $T \before_{l, u} T'$ be an atom
    in $\clause$, and note that $\gamma$ already satisfies
    \Cref{lemma:function-matching-run:entire-atom} for every $T' \in M_{n-1}$.
    If $T'\in I_n$ instead, consider the entry $D_{n-1}[T,T']$ representing the
    lower bound $l$ of the aforementioned atom. If $D_{n-1}[T,T'] \le 0$,
    \Cref{def:match-event:preceding-terms} of $I$-match event gives $T \in
    M_{n-1} \cup I_n = M_n$. If $D_{n-1}[T,T'] > 0$, $D_{n-1}[T,T']$ no longer
    stores its initial value $-l \leq 0$, meaning that $T$ must have been
    previously matched and $T \in M_{n-1} \subseteq M_n$. In either case, $T\in
    M_n$ and $\gamma(T) \le \gamma(T')$, because $\gamma(T) \le n$.
    % proof of lower and upper bounds of \gamma
    If $T \in I_n$, then $\delta(\slice\evseq_{\gamma(T),\gamma(T')}) = 0 \le
    u$, is trivially satisfied by any upper bound $u$. Furthermore, by
    \Cref{def:match-event:zero-no-bounds} of $I$-match event, either the lower
    bound $D_{n-1}[T,T'] = 0$ or the upper bound $D_{n-1}[T',T] = 0$, and they
    both equal their initial values $l$ and $u$. Note that the former case is
    also implied by the latter, so that $l = 0$ and $l \le
    \delta(\slice\evseq_{\gamma(T),\gamma(T')})$. If $T \in M_{n-1}$, by
    \Cref{def:match-event:lower-bounds} of $I$-match event, $\delta_n \ge
    -D[T,T'] = l - \delta(\slice\evseq_{\gamma(T),n-1})$. Hence, $l \le
    \delta(\slice\evseq_{\gamma(T),n-1}) + \delta_n =
    \delta(\slice\evseq_{\gamma(T),\gamma(T')})$. While $\delta_n \le
    D_{n-1}[T',T] = u - \delta(\slice\evseq_{\gamma(T),n-1})$, since $\event_n$
    is an admissible event for $\M_{n-1}$. Hence,
    $\delta(\slice\evseq_{\gamma(T),n-1}) + \delta_n =
    \delta(\slice\evseq_{\gamma(T),\gamma(T')}) \le u$.

    \statement{\Cref{lemma:function-matching-run:partial-atom} of
      \Cref{lemma:function-matching-run}} Let $T \before_{l,u} T'$ be an atom in
    $\clause$ such that $T \in M_n$ and $T' \not\in M_n$. Since $\event_n$ is an
    admissible event for $\M_{n-1}$, $\delta_n \le D_{n-1}[T',T] = u -
    \delta(\slice\evseq_{\gamma(T),n-1})$. Hence,
    $\delta(\slice\evseq_{\gamma(T),n-1}) + \delta_n =
    \delta(\slice\evseq_{\gamma(T),n}) \le u$.
  \end{description}
\end{proof}

\superset*
  %Let $\evseq =\langle\event_1,\dots,\event_n\rangle$ be an event sequence
  %(closed to the left??), let $\M_\E$ be the initial matching structure of some
  %existential statement $\E$ of a rule $\Rule$, and let $\M_r$ be an active
  %matching structure for the trigger event $\event_r$ of $\Rule$ resulting from
  %a run $\M_\E \runm*[r] \M_r$. If there exists an active matching structure
  %$\M_s$, resulting from a run $\M_\E \runm*[s] \M_s$, for a trigger event of
  %$\Rule$ prior to $\event_r$, then there exists an active matching structure
  %$\M$ for the same trigger event of $\M_s$ resulting from a run $\M_\E \runm*
  %\M$ and matching at least as many tokens as $\M_r$.

\begin{proof}
  Let $\M_\E \runm*[r]\M_r = (V, D_r, M_r, t_r)$ and $\M_\E \runm*[s] \M_s = (V,
  D_s, M_s, T_s)$, with $\gamma_s(\tokstart(a_0)) \le \gamma_r(\tokstart(a_0))$.
  Let $M = M_r \cup M_s$ and $\gamma: M \to [1,\ldots,n]$ be a function defined
  as:
  \[
    \gamma(T) =
    \begin{cases}
      \gamma_s(T) &\text{if } T\in M_s\cap M_r\text{ and }\gamma_s(T)\leq\gamma_r(T)\\
      \gamma_r(T) &\text{if } T \in M_s \cap M_r \text{ and }\gamma_s(T)>\gamma_r(T)\\
      \gamma_s(T) &\text{if } T \in M_s \setminus M_r\\
      \gamma_r(T) &\text{if } T \in M_r \setminus M_s
    \end{cases}
  \]

  \statement{$\gamma$ is a matching function} \Cref{def:matching-function:nodes}
  of \Cref{def:matching-function} for $\gamma$ follows from our hypothesis on $\gamma_s$
  and $\gamma_r$. Regarding \Cref{def:matching-function:tokens}, let
  $\tokstart(a), \tokend(a) \in M$ for some quantified token $a[x=v]$ in $\E$.
  If $\gamma_s$ and $\gamma_r$ map the endpoints of $a$ to the same token in
  $\evseq$, then $\gamma(\tokstart(a))$ and $\gamma(\tokend(a))$ correctly
  identify the endpoints of that token. If instead $\gamma_s$ and $\gamma_r$ map
  $a$ to two distinct tokens in $\evseq$, then $\gamma$ would match $a$
  according to the function whose token comes first, correctly identifying the
  endpoints of such token.

  \statement{$\gamma$ satisfies
    \Cref{lemma:function-matching-run:entire-atom,lemma:function-matching-run:partial-atom}
    of \Cref{lemma:function-matching-run}} Let $T \before_{l,u} T'$ be an atom
  in $\clause$. If $T' \in M$, then either $T' \in M_s$, and $T \in M_s
  \subseteq M$, or $T' \in M_r$, and $T \in M_r \subseteq M$. If $\gamma$ maps
  both terms with either $\gamma_s$ or $\gamma_r$, then $\gamma(T) \leq
  \gamma(T')$ and $l \leq \delta(\slice\evseq_{\gamma(T),\gamma(T')}) \leq u$
  immediately follows. If instead
  % \begin{itemize}
  % \item
  $\gamma(T) = \gamma_s(T)$ and $\gamma(T') = \gamma_r(T')$, then $T' \in
    M_r$ and $T \in M_s \cap M_r$. By definition of $\gamma$, $\gamma_s(T) \leq
    \gamma_r(T)$, and, by \Cref{lemma:function-matching-run:entire-atom} for
    $\gamma_r$, $\gamma_r(T) \leq \gamma_r(T')$. Hence, $\gamma(T) \leq
    \gamma(T')$. If $T' \in M_s$, then $\gamma_s(T') > \gamma_r(T')$, and:
    \begin{align*}
      l &\leq\delta_{\gamma_r(T),\gamma_r(T')}&&\text{\Cref{lemma:function-matching-run:entire-atom} for } \gamma_r\\
        &\leq \delta_{\gamma_s(T),\gamma_r(T')}&&\text{}\gamma_s(T)\leq\gamma_r(T)\\
        &< \delta_{\gamma_s(T),\gamma_s(T')} &&\text{} \gamma_s(T') > \gamma_r(T')\\
        &\leq u &&\text{\Cref{lemma:function-matching-run:entire-atom} for } \gamma_s
    \end{align*}
    otherwise:
    \begin{align*}
      l &\leq \delta_{\gamma_r(T),\gamma_r(T')} &&\text{\Cref{lemma:function-matching-run:entire-atom} for } \gamma_r\\
        &< \delta_{\gamma_s(T),\gamma_r(T')}&&\text{} \gamma_s(T) \leq \gamma_r(T)\\
        &\leq \delta_{\gamma_s(T),n} &&\text{} \gamma_r(T') \leq n\\
        &\leq u &&\text{\Cref{lemma:function-matching-run:partial-atom} for } \gamma_s
    \end{align*}
    The case for $\gamma(T) = \gamma_r(T)$ and $\gamma(T') = \gamma_s(T')$ is
    completely symmetrical.
    % \item SYMMETRICAL CASE:
    % $\gamma(T) = \gamma_r(T)$ and $\gamma(T') = \gamma_s(T')$, then $T' \in
    % M_s$ and $T \in M_s \cap M_r$. By definition of $\gamma$, $\gamma_r(T) <
    % \gamma_s(T)$, and, by \Cref{lemma:function-matching-run:entire-atom} for
    % $\gamma_s$, $\gamma_s(T) \leq \gamma_s(T')$. Hence, $\gamma(T) \leq
    % \gamma(T')$. If $T' \in M_r$, then $\gamma_s(T') \leq \gamma_r(T')$, and:
    % \begin{align*}
    %   l &\leq \delta_{\gamma_s(T),\gamma_s(T')} &&\text{\Cref{lemma:function-matching-run:entire-atom} for } \gamma_s\\
    %     &< \delta_{\gamma_r(T),\gamma_s(T')} &&\text{} \gamma_r(T) < \gamma_s(T)\\
    %     &\leq \delta_{\gamma_r(T),\gamma_r(T')} &&\text{} \gamma_s(T') \leq \gamma_r(T')\\
    %     &\leq u &&\text{\Cref{lemma:function-matching-run:partial-atom} for } \gamma_r
    % \end{align*}
    % If $T' \not\in M_t$, then
    % \begin{align*}
    %   l &\leq \delta_{\gamma_s(T),\gamma_s(T')} &&\text{\Cref{lemma:function-matching-run:entire-atom} for } \gamma_s\\
    %     &< \delta_{\gamma_r(T),\gamma_s(T')} &&\text{} \gamma_r(T) < \gamma_s(T)\\
    %     &\leq \delta_{\gamma_r(T),n} &&\text{} \gamma_s(T') \leq n\\
    %     &\leq u &&\text{\Cref{lemma:function-matching-run:partial-atom} for } \gamma_r
    % \end{align*}
  % \end{itemize}

    Lastly, if $T'\not\in M$, but $T \in M$, then either $\gamma(T)=\gamma_s(T)$
    or $\gamma(T) = \gamma_r(T)$, and
    \Cref{lemma:function-matching-run:partial-atom} for $\gamma$ follows from
    \Cref{lemma:function-matching-run:partial-atom} for $\gamma_s$ and
    $\gamma_r$.
\end{proof}

\residualexist*

\begin{proof}
  Let $\gamma(\tokstart(a_0)) = s$, assuming there is no residual matching
  structure $\M_k$ in the sequence $\seq{\M_s, \ldots, \M_{n-1}}$, then for
  every matching structure $\M_i=(V, D_i, M_i, t_i)$, where $s \leq i < n$,
  there exists a pair of terms $\pair{T, T'}$ such that $T \in M_i$ and
  $T'\not\in M_i$, and their distance $D_i[T',T]$ has a finite upper bound. Let
  $E \subseteq V \times V$ be the set that collects all pairs $(T, T')$ for the
  matching structures $\M_i$. We define $\delta_{T, T'}$ as
  $\delta(\slice\evseq_{\gamma(T),\gamma(T')})$ if $\gamma(T')$ is defined, or
  as $\delta(\slice\evseq_{\gamma(T),n})$ otherwise. Let $\delta_E =
  \sum_{\pair{T,T'} \in E} \delta_{T,T'}$ and note that $\delta_E \ge
  \delta(\slice\evseq_{\gamma(\tokstart(a_0)),n})$, because every position in
  $\slice\evseq_{\gamma(\tokstart(a_0)),n}$ is covered by some distance
  $\delta_{T,T'}$. Moreover, each pair $(T, T') \in E$ corresponds to an atom of
  the form $T \before_{l,u} T'$ in $\clause$. According to
  \Cref{lemma:function-matching-run}, we have $\delta_{T, T'} \leq u$, and
  therefore, $\delta_E \leq \window(P)$. Hence, we have
  $\delta(\slice\evseq_{\gamma(\tokstart(a_0)),n}) \leq \delta_E \leq
  \window(P)$: a contradiction hence proving the existence of a residual
  matching structure $M_k$. 
\end{proof}

%-----------------------------------------------------------------------------
% PROOF DEL TEOREMA 3.6 DI LMCS GANDALF (todo check la definizione di run)
%-----------------------------------------------------------------------------
\soundnessCompleteness*

\begin{proof}
  \proofonlyif Let $\evseq = \seq{\event_1,\ldots,\event_n}$ be a solution plan
  for $P$, and let $\stateseq = \seq{q_0,\ldots,q_n}$ be the run of $\A_P$ on
  $\evseq$. We first show that the sink state is never reached, and then that
  $q_n$ is a final state.

  Let $\event_s = (A_s, \delta_s)$ be the trigger event of a rule $\Rule \equiv
  a_0[x_0 = v_0] \implies \E_1 \lor \dots \lor \E_m$, \ie, $\tokstart(x_0,v_0)
  \in A_s$. Since $\evseq$ is a solution plan, there exist tokens satisfying an
  existential statement $\E$ of $\Rule$ for the trigger $\event_s$. Hence, by
  \cref{lemma:function-matching-run,obs:matching-functions} there exists a run
  $\M_\E \runm* \M_n$, yielding a \emph{closed} matching structure $\M_n$, such
  that $\gamma(\tokstart(a_0)) = s$.

  Let $\overline{\M} = \seq{\M_\E, \M_1,\ldots,\M_{n}}$ be the sequence of all
  the matching structures involved in such run. Note that, by construction
  (\cref{sec:automata-construction}), the states of $\stateseq$ induce all the
  possible runs for the \emph{initial} matching structures of $P$ that can be
  defined on $\evseq$. In particular, the run $\gamma$ must be one of them.
  However, only a subsequence of the matching structures $\overline{M}$ will
  appear in the states of the run $\stateseq$. Indeed, we can identify three key
  points for the sequence $\overline{M}$: the least position $s$ such that
  $\M_s$ is \emph{active} (corresponding to $\gamma(\tokstart(a_0))$), the least
  position $h$ following $s$ such that $\M_h$ no longer belongs to the component
  $\Upsilon$ of the states in $\slice\stateseq_{h, n}$, either because $\M_h$ is
  \emph{closed} or because $\delta(\slice\evseq_{s,h}) > \window(P)$, and the
  least position $k$ following $h$ such that $\M_k$ no longer belongs to the
  component $\Delta$ of the states in $\slice\stateseq_{k,n}$, either because
  $\M_k$ is \emph{closed} or because it gets discarded in favour of the matching
  structures of a later trigger event.

  Every matching structure in $\slice{\overline{M}}_{1,h-1}$ belongs to the
  component $\Upsilon$ of a corresponding state in $\slice\stateseq_{1,h-1}$, so
  the set $\Upsilon$ of the state $q_{s-1}$ is such that $\M_s \in
  \step_{\event_s}(\Upsilon_\bot)$, satisfying condition
  \ref{dfa:delta:trigger-capture} of \cref{sec:automata-construction} for the
  trigger event $\event_s$. Matching structures $\slice{\overline{M}}_{s+1,h}$
  instead belong to the set $\step_\event(\Upsilon^\Rule_t)$, for the partition
  $\Upsilon^\Rule_t$ tracking the satisfaction of the trigger event $\event_s$
  of every state $\slice\stateseq_{s,h-1}$. Hence, all such states satisfy
  condition \ref{dfa:delta:no-failed-step} of \cref{sec:automata-construction}.

  We now show that no \emph{active} matching structures for the trigger event
  $\event_s$ exists after some state $q_h$, following $q_s$ in $\stateseq$. Note
  that the run $\gamma$ yields a \emph{closed} matching structure, and if it
  does so within $\window(P)$ time units from the event
  $\event_{\tokstart(a_0)}$, we identified such position as the closed matching
  structure $\M_h$. So that the state $q_{h-1}$ is such that $\M_h \in
  \step_{\event_h}(\Upsilon^\Rule_t)$, for the partition $\Upsilon^\Rule_t$
  tracking the trigger event $\event_s$, and
  $\step_{\event_h}(\Upsilon^\Rule_t)$ is discarded from $q_h$.

  If instead $\gamma$ yields a \emph{closed} matching structure after
  $\window(P)$ time units from the event $\event_{\tokstart(a_0)}$, lets
  identify such position as $\M_j$, with $j \le n$. If $\M_{j-1}$ belongs to the
  set $\Delta(\E)$ of the state $q_{j-1}$, then $\M_j \in
  \step_{\event_j}(\Delta(\E))$, so that $\step_{\event_j}(\Delta(\E))$ is
  \emph{closed} and discarded from $q_j$, alongside all the other matching
  structures in $\Delta(\E')$, for every $E' \in \Phi(\E)$, \ie, for every other
  existential statement $\E'$ of $\Rule$ still tracking the trigger $\event_s$.
  If instead $\M_{j-1}$ for the trigger event $\event_s$ does not belong to the
  set $\Delta(\E)$ of the state $q_{j-1}$, by construction
  (\cref{sec:automata-construction}), there exist a state $q_h$ in which the
  matching structures tracking $\event_s$ have been replaced by those of a later
  event, and they no longer appear in $\Delta(\E)$ from $q_h$ onwards.

  Since $\evseq$ is a solution plan, the previous argument holds for all the
  trigger events in $\evseq$ of any rules in $S$. Hence, conditions
  \labelcref{dfa:delta:trigger-capture,dfa:delta:no-failed-step} are always met,
  \ie, the sink state is never reached, and no active matching structures belong
  to $q_n$, making it a final state.

  \proofif Let $\evseq = \seq{\event_1,\ldots,\event_n}$ be an event sequence
  accepted by $\A_P$ and let $\rho =\seq{q_0,\ldots,q_n}$ be its accepting run.
  We have to show that the plan corresponding to $\evseq$ is a solution plan for
  $P$, \ie, for every event triggering a rule $\Rule$ in $S$, at least one of
  the existential statements of $\Rule$ is satisfied by $\evseq$.

  Let $\event_s = (A_s, \delta_s)$ be an event in $\evseq$ triggering a rule
  $\Rule \equiv a_0[x_0 = v_0] \implies \E_1 \lor \dots \lor \E_m$, \ie,
  $\tokstart(x_0,v_0) \in A_s$. Since the sink state is never visited in an
  accepting run, the state $q_s$, reached upon reading the event $\event_s$, is
  such that the partition $\Upsilon^\Rule_0$, tracking the satisfaction of the
  trigger event $\event_s$, is not empty. For the same reason, the partition
  $\Upsilon^\Rule_t$ tracking $\event_s$ in every state following $q_s$ can
  never be empty as a result of the function $\step_\event$. However, since the
  final state $q_n$ does not contain any \emph{active} matching structure, there
  must exists a state $q_h$ in $\stateseq$ whose partition
  $\step_{\event_{h+1}}(\Upsilon^\Rule_t)$ gets discarded from $q_{h+1}$. This
  can happen either because $\step_{\event_{h+1}}(\Upsilon^\Rule_t)$ is a
  \emph{closed} set, or because the matchings structures in
  $\step_{\event_{h+1}}(\Upsilon^\Rule_t)$ get promoted to the component
  $\Delta$. In the first case, we can conclude that there exists a run $\M_\E
  \runm* \M_n$ for the initial matching structure $\M_\E$ of an existential
  statement $\E$ of $\Rule$ such that $\M_n$ is \emph{closed} and
  $\gamma(\tokstart(a_0)) = s$, hence, by
  \cref{obs:matching-functions,lemma:function-matching-run}, the trigger event
  $\event_s$ satisfies $\Rule$.

  In the second case, let $\Psi$ be the set of existential statements having an
  active matching structure in $\step_{\event_{h+1}}(\Upsilon^\Rule_t)$, so that
  we can identify them as the sets $\Delta(\E)$, for $\E \in \Psi$, in the
  states from $q_{h+1}$ onwards. By \cref{lemma:residual-matching-structure},
  every such set contains a \emph{residual} matching structure. Hence, by
  \cref{obs:residual-run}, they can become empty only if, at some state $q_{k}$
  following $q_h$, $\step_{\event_{k+1}}(\Delta(\E))$ contains a \emph{closed}
  matching structure for some existential statement $\E \in \Psi$. Note that
  the run $\stateseq$ is an accepting run, so every non-empty set
  $\Delta(\E)$ must become empty before the end of the run. Hence, $q_k$ is
  guaranteed to exist.

  However, it may be the case that, by the time
  $\step_{\event_{k+1}}(\Delta(\E))$ is \emph{closed}, $\Delta(\E)$ no longer
  contains the matching structures for the trigger event $\event_s$, but those
  for a later trigger event $\event_r$ of $\Rule$. Since the sets
  $\Delta(\E)$ store only the matching structures tracking the most recent
  trigger event older than $\window(P)$. Thus, if
  $\step_{\event_{k+1}}(\Delta(\E))$ contains a \emph{closed} matching structure
  for $\event_s$, we can directly assert the existence of a run for $\M_\E$
  implying the satisfaction of $\Rule$ for the trigger event $\event_s$. If
  instead $\step_{\event_{k+1}}(\Delta(\E))$ contains a \emph{closed} matching
  structure $\M_r$ for a later event $\event_r$, there exists a run $\M_\E
  \runm*[r] \M_r$, such that $\gamma_r(\tokstart(a_0)) = r$. Furthermore, by a
  previous consideration on $q_{h+1}$, there exists a run $\M_\E
  \xlongrightarrow{\slice\evseq_{1,h+1},\gamma_s} \hat\M_{h+1}$, yielding a
  \emph{residual} matching structure $\hat\M_{h+1}$, and, by
  \cref{obs:residual-run}, such run can be extended on the entire event sequence
  $\M_\E \runm*[s] \hat\M_n$, to yield a \emph{residual} matching structure
  $\hat\M_n$. Given $\M_\E \runm*[r] \M_r$ and $\M_\E \runm*[s] \hat\M_n$, with
  $\gamma_s(\tokstart(a_0)) \le \gamma_r(\tokstart(a_0))$, by
  \cref{lemma:matching-structure-superset}, there exists a run $\M_\E \runm*[s]
  \M_n$ yielding a matching structure $\M_n$ matching as many terms as $\M_r$
  and such that $\gamma_s(\tokstart(a_0)) = s$. Hence, $\M$ is a \emph{closed}
  matching structure for the existential statement $\E$, and, by
  \cref{obs:matching-functions,lemma:function-matching-run}, $\Rule$ satisfies
  the trigger event $\event_s$.

  Furthermore, all the value duration functions are satisfied by
  the tokens in $\evseq$, being encoded as synchronisation rules by the
  automaton $S_P$. Meanwhile, the automaton $T_P$ guarantees the fulfilment of
  the value transition functions. Hence, we can conclude that $\evseq$ is a
  solution plan for $P$, because every rule in $S$ is satisfied, as well as the
  value duration and value transition functions of every state variable.
\end{proof}

%%% Local Variables:
%%% TeX-master: "../lmcs-gandalf22.tex"
%%% End:

\end{document}
