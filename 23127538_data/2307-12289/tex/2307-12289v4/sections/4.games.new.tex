%!TeX root = ../lmcs-gandalf22.tex

\section{Controller synthesis}
\label{sec:games}

We leverage the deterministic automaton constructed in the previous section to establish a deterministic arena that enables us to solve a reachability game and determine whether a controller exists. If a controller exists, our procedure allows its synthesis.

\subsection{From the automaton to the arena}

Let $G=(\SV_C, \SV_E, \S, \D)$ be a timeline-based game. The automaton construction we used considered a planning problem with a single set of synchronization rules, while in $G$, we have to account for the roles of both $\S$ and $\D$.

To address this, we define $A_\S$ and $A_\D$ as the deterministic automata constructed over the timeline-based planning problems $P_\S=(\SV_C\cup\SV_E, \S)$ and $P_\D=(\SV_C\cup\SV_E, \D)$, respectively. We then construct the automaton $A_G$ by taking the union of $A_\S$ with the complement of $A_\D$ ($\overline{A_\D}$). Note that these are standard automata-theoretic operations over DFAs.
An accepting run of $A_G$ represents either a plan that violates the domain rules or a plan that adheres to domain and system rules, according to the definition of winning strategy in \cref{def:games:winning-strategy}.
Furthermore, $A_G$ is deterministic, and its size only polynomially increases when built from $A_\D$ and $A_\S$. 

The $A_G$ automaton is not immediately applicable as a game arena since its transitions' labels only reflect events, not game moves. In $A_G$, a single transition can correspond to various combinations of rounds due to the absence of $\wait(\delta)$ moves in the transition's label. For example, an event $\event = (A, 5)$ can arise from either a $\wait(5)$ move by \charlie, followed by a $\play(5, A)$ move by $\eve$, or any $\wait(\delta)$ move with $\delta > 5$ followed by a $\play(5, A)$ move. To obtain a suitable game arena, we need to modify $A_G$ further. 

Let $A_G = (Q,\Sigma, q_0, F, \tau)$ be the automaton constructed as described above. Formally, we define a new automaton $A_G' = (Q,\Sigma,q_0, F,\tau')$ where $\tau'$ is a partial transition function, meaning that the automaton is now incomplete. The function $\tau'$ agrees with $\tau$ on all transitions except those of the form $\tau(q,(\actions,\delta))$, where $\delta>1$ and $\actions$ contains a $\tokend(x,v)$ action with $x\in\SV_C$. In such cases, the transition is undefined in $A_G'$.
An example is shown in Figure \ref{fig:constructions} (left). Note that this removal does not alter the set of plans accepted by the automaton since for each transition $\tau(q,(\actions,\delta))=q'$ with $\delta > 1$, there exist two transitions $\tau(q,(\emptyset,\delta-1))=q''$ and $\tau(q'',(\actions,1))=q'$ in $A_G'$.

% Figure environment removed

To make the game rounds and moves explicit, we can transform the automaton by splitting each transition into four transitions representing the four moves of the two rounds. Starting from the incomplete automaton $A_G'=(Q,\Sigma, q_0, F, \tau')$, we define a new automaton $A_G^a=(Q^a,\Sigma^a, q_0^a, F^a, \tau^a)$ as the game arena.

\begin{enumerate}
  \item The set of states $Q^a$ is given by $Q^a=Q\cup\set{q_\delta\mid 1\le\delta\le d}\cup\set{q_{\delta,A}\mid 1\le\delta\le d, A\subseteq \mathsf{A}}$.
  \item The alphabet $\Sigma^a$ is defined as $\Sigma^a=\moves_C\cup\moves_E$, which corresponds to the set of moves of the two players.
  \item The initial and final states of $A_G^a$ are $q_0^a=q_0$ and $F^a=F$, respectively.
  \item The partial transition function $\tau^a$ is defined as follows. Let $w=\tau(q,\event)$ with $\event=(\actions,\delta)$. We distinguish the cases where $\delta=1$ or $\delta>1$.
    \begin{enumerate}
      \item if $\delta=1$, let $\actions_C\subseteq\actions$ and
      $\actions_E\subseteq\actions$ be the set of actions in $\actions$ playable
      by \charlie and by \eve, respectively. Then:
      \begin{enumerate}
        \item $\tau(q,\play(\actions_C^e))=q_{1,\actions_C^e}$, where 
          $\actions_C^e$ is the set of \emph{ending} actions in $\actions_C$;
        \item $\tau(q_{1,\actions_C^e},\play(\actions_E^e))=q_{1,\actions_C^e\cup\actions_E^e}$, where 
          $\actions_E^e$ is the set of \emph{ending} actions in $\actions_E$;
        \item $\tau(q_{1,\actions_C^e\cup\actions_E^e},\play(\actions_C^s))=q_{1,\actions_C^e\cup\actions_E^e\cup\actions_C^s}$, where 
          $\actions_C^s$ is the set of \emph{starting} actions in $\actions_C$;
        \item $\tau(q_{1,\actions_C^e\cup\actions_E^e\cup\actions_C^s},\play(\actions_E^s))=w$, where 
          $\actions_E^s$ is the set of \emph{starting} actions in $\actions_E$;
      \end{enumerate}
      Here, the states mentioned are added to $Q^a$ as needed.
      \item if $\delta>1$, let $\actions_C\subseteq\actions$ and
      $\actions_E\subseteq\actions$ be the set of actions in $\actions$ playable
      by \charlie and by \eve, respectively. Note that by construction,
      $\actions_C$ only contains \emph{starting} actions. Then:
      \begin{enumerate}
        \item $\tau(q,\wait(\delta_C))=q_{\delta_C}$ for all 
          $\delta\le\delta_C\le d$;
        \item $\tau(q_{\delta_C},\play(\delta, \actions_E^e))=q_{\delta,\actions_E^e}$
          where $\actions_E^e$ is the set of \emph{ending} actions in
          $\actions_E$;
        \item $\tau(q_{\delta,\actions_E^e},\play(\actions_C))=q_{\delta,\actions_E^e\cup\actions_C}$;
        \item $\tau(q_{\delta,\actions_E^e\cup\actions_C},\play(\actions_E^s))=w$ where
          $\actions_E^s$ is the set of \emph{starting} actions in $\actions_E$;
      \end{enumerate}
      where the mentioned states are added to $Q^a$ as needed.
    \end{enumerate}
    All the transitions not explicitly defined above are undefined.
\end{enumerate}

We present a graphical illustration of the above construction in \cref{fig:constructions}. It is worth noting that the automaton preserves the structure of the original automaton $A_G$. For any state, $q\in Q$ and event $\event=(A,\delta)$, any sequence of moves that would result in appending $\event$ to the partial plan (see \cref{def:games:round-outcome}) reaches the same state $w$ in $A^a_G$ as it does in $A_G$ by reading $\event$. Therefore, we can consider $A^a_G$ as being able to read event sequences, even though its alphabet is different. We use the notation $[\evseq]$ to represent the state $q\in Q^a$ reached by reading $\evseq$ in $A^a_G$.
Furthermore, note that, with a slight abuse of notation, any play $\bar\rho$ in the game $G$ is a readable word by the automaton $A_G^a$. Thus, we can establish the following result.
\begin{thm}
  \label{thm:arena-soundness}
  If $G$ is a timeline-based game, for any play $\bar\rho$ for $G$, $\bar\rho$
  is successful if and only if it is accepted by $A_G^a$.
\end{thm}

\subsection{Computing the Winning Strategy and Building the Controller}
Let us define $Q^a_C \subset Q^a$ as the set of states in which \charlie can make a move, and $Q^a_E = Q^a \setminus Q^a_C$ as the set of states where \eve can make a move. Additionally, we define $E=\{(q, q') \in Q^a\times Q^a \mid \exists \event \mathrel{.} \tau^a(q, \event) = q'\}$ as the set of edges in $A^a_G$. By solving the reachability game $(G_R, \mathcal{W})$, where $G_R = (Q^a, Q^a_C, E^a)$ and $\mathcal{W} = \{R \subseteq Q^a \mid R \cap F^a \neq \emptyset \}$, we aim to determine the winning region $W_C$ and the winning strategy $s_C$ for \charlie, provided they exist. In the following discussion, we will show that the winning strategy $\sigma_C$ for the timeline-based game $G$ is derivable from strategy $s_C$ when $q^a_0 \in W_C$.

To determine the winning region $W_C$, we use the well-known \emph{attractor} construction. We are interested to the attractor set of $F^a$ for \charlie, written $Attr_C(F^a)$, thus given $i \ge 0$ we compute the set of states from which \charlie can reach a state $q \in F^a$ within $i$ moves, defined as $Attr_C^i(F^a)$:
\begin{align*} % Indented the last two lines. If you do not agree with this, let us know.
  Attr^0_C (F^a) ={}& F^a \\
  Attr^{i+1}_C (F^a) ={}& Attr^i_C (F^a) \\
  &\cup \set{ q^a \in Q^a_C \, | \, \exists r \big((q^a, r) \in E \land r \in Attr^i_C (F^a)\big) } \\
  &\cup \set{ q^a \in Q^a_E \, | \, \forall r \big((q^a, r) \in E \implies r \in Attr^i_C (F^a) \big) }.
\end{align*}

The sequence $Attr^0_C (F^a) \subseteq Attr^1_C (F^a) \subseteq Attr^2_C (F^a) \subseteq \ldots$ eventually becomes stationary for some index $k \leq \lvert Q^a \rvert$, hence we can define $Attr_C (F^a) = \bigcup^{\lvert Q^a \rvert}_{i=0} Attr^i_C(F^a)$ as the attractor set. Note that $W_C = Attr_C(F^a)$ is a known fact for which proof is available in \cite{Thomas2008}.
Next, we want that $q_0^a\in W_C$ since we are interested in a winning strategy $\sigma_C$ for the timeline-based game $G$. If it is the case, by defining $s_C(q) = \mu$ for any $\mu$ such that $\tau^a(q,\mu)=q'$ with $q,q'\in W_C$, which is guaranteed to exist by the attractor construction, we can define $\sigma_C$ for \charlie in $G$ as $\sigma_C(\evseq) = s_C([\evseq])$ for any event sequence $\evseq$. We prove this claim in the following:

\begin{thm}
    \label{thm:winning-region-soundness-completeness}
  Given $A_G^a=(Q^a,\Sigma^a, q_0^a, F^a, \tau^a)$, $q_0^a\in W_C$ if and only
  if $\sigma_C$ is a winning strategy for \charlie for $G$.
\end{thm}
\begin{proof}

  \proofif From the definition of a winning strategy for \charlie in $G$ (\cref{def:games:winning-strategy}), we know that for every admissible strategy $\sigma_E$ for \eve, there exists $n \ge 0$ such that the play $\rounds_n(\strategy_C,\strategy_E)$ is successful. By the soundness of the arena construction (\cref{thm:arena-soundness}), we know that the event sequence $\evseq_n$ representing $\rounds_n(\strategy_C,\strategy_E)$, when seen as a word over $\Sigma^a$, is accepted by $A_G^a$. Therefore, $\evseq_n$ reaches a state in the set $F^a$ starting from $q_0^a$. By the definition of the reachability game, this means that $q_0^a\in W_C$. Thus, we have proved that if $\sigma_C$ is a winning strategy \charlie in $G$, then $q_0^a\in W_C$.
  
  \proofonlyif If $q_0^a\in W_C$, then by definition, $s_C$ is a winning strategy for \charlie in the reachability game over the arena $A_G^a$. Hence, any word over $\Sigma^a$ obtained by playing with $s_C$ is accepted by $A_G^a$, and therefore, by the soundness of the arena construction (\cref{thm:arena-soundness}), any corresponding play $\rounds$ is successful in $G$. Now, recall that $\sigma_C(\evseq)=s_C([\evseq])$ for any event sequence $\evseq$. Hence, $\rounds=\rounds(\sigma_C,\sigma_E)$ for some strategy $\sigma_E$ of \eve. As a result, we can conclude that $\sigma_C$ is a winning strategy for \charlie in $G$.
\end{proof}

Finally, we build a Controller that implements the winning strategy $\sigma_C$, 
provided it exists. First, by \cref{thm:winning-region-soundness-completeness}, 
the existence of $\sigma_C$ implies that $q^a_0 \in W_C$. Next, we define the 
following Moore machine (\cref{def:moore-machine}) based on $s_C$:

\begin{defi}[Controller]
    \label{def:controller-implementation}
Given $A_G^a=(Q^a,\Sigma^a, q_0^a, F^a, \tau^a)$, we define a Controller as $\mathcal{M} = (Q, \Sigma, \Gamma, q_0, \delta, \tau)$, where $Q = Q^a_C \cap W_C$ represents the set of states, $q_{0} = q_0^a$ is the initial state, $\Sigma = \moves_E$ is the input alphabet, $\Gamma = \moves_C$ is the output alphabet, $\delta : Q \times \Sigma \rightarrow Q$ is the transition function, and $\tau : Q \rightarrow \Gamma$ is the output function. The transition function $\delta$ and the output function $\tau$ are defined as follows:
    \begin{align*} 
        \delta(q_C, \move_E) &= \tau^a(s_C(q_C), \move_E) \\ \tau(q_C) &= s_C(q_C).
    \end{align*} 
\end{defi}

Note that by construction the states of $\mathcal{M}$ belong to the winning
region $W_C$ of $A_G^a$, and $\delta$ follows the transition function $\tau^a$
of $A^a_G$. Hence, the output of $\mathcal{M}$ after reading a word $\evseq$ is
exactly $\sigma_C(\evseq)=s_C([\evseq])$ and $\mathcal{M}$ implements
$\sigma_C$, which is a winning strategy by
\cref{thm:winning-region-soundness-completeness}.
