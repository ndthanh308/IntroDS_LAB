% ****** Start of file apssamp.tex ******
%
%   This file is part of the APS files in the REVTeX 4.1 distribution.
%   Version 4.1r of REVTeX, August 2010
%
%   Copyright (c) 2009, 2010 The American Physical Society.
%
%   See the REVTeX 4 README file for restrictions and more information.
%
% TeX'ing this file requires that you have AMS-LaTeX 2.0 installed
% as well as the rest of the prerequisites for REVTeX 4.1
%
% See the REVTeX 4 README file
% It also requires running BibTeX. The commands are as follows:
%
%  1)  latex apssamp.tex
%  2)  bibtex apssamp
%  3)  latex apssamp.tex
%  4)  latex apssamp.tex
%

\documentclass[
reprint,
superscriptaddress,
amsmath,amssymb,
aps,
pra,
longbibliography,
]{revtex4-1}
\usepackage{float}
\usepackage{xcolor}

\usepackage{graphicx}% Include figure files
\usepackage{dcolumn}% Align table columns on decimal point
\usepackage{bm}% bold math
%\usepackage{hyperref}% add hypertext capabilities
\usepackage[colorlinks=true, pdfstartview=FitV, linkcolor=blue, citecolor=blue, urlcolor=blue]{hyperref}
\usepackage{dsfont}

\usepackage[separate-uncertainty=true]{siunitx}
\usepackage[siunitx]{circuitikz}


\graphicspath{{figures/}}

\begin{document}

\preprint{APS/123-QED}


\title{A biased Ising model using two coupled Kerr parametric oscillators with external force}


\author{Pablo \'Alvarez}
\affiliation{Laboratory for Solid State Physics, ETH Z\"{u}rich, CH-8093 Z\"urich, Switzerland.}
\author{Davide Pittilini}
\affiliation{Laboratory for Solid State Physics, ETH Z\"{u}rich, CH-8093 Z\"urich, Switzerland.}
\author{Filippo Miserocchi}
\affiliation{Laboratory for Solid State Physics, ETH Z\"{u}rich, CH-8093 Z\"urich, Switzerland.}
\author{Sathyanarayanan Raamamurthy}
\affiliation{Laboratory for Solid State Physics, ETH Z\"{u}rich, CH-8093 Z\"urich, Switzerland.}
\author{Gabriel Margiani}
\affiliation{Laboratory for Solid State Physics, ETH Z\"{u}rich, CH-8093 Z\"urich, Switzerland.}
\author{Orjan Ameye}
\affiliation{Department of Physics, University of Konstanz, D-78457 Konstanz, Germany.}
%\author{...}
%\affiliation{...}
\author{Javier del Pino}
\affiliation{Institute for Theoretical Physics, ETH Z\"{u}rich, CH-8093 Z\"urich, Switzerland.}
\author{Oded Zilberberg}
\affiliation{Department of Physics, University of Konstanz, D-78457 Konstanz, Germany.}
\author{Alexander Eichler}
\affiliation{Laboratory for Solid State Physics, ETH Z\"{u}rich, CH-8093 Z\"urich, Switzerland.}
\affiliation{Quantum Center, ETH Zurich, CH-8093 Zurich, Switzerland}

\date{\today}% It is always \today, today,
             %  but any date may be explicitly specified


\begin{abstract}
  Networks of coupled Kerr parametric oscillators (KPOs) are a leading physical platform for analog solving of complex optimization problems. These systems are colloquially known as ``Ising machines''. We experimentally and theoretically study such a network under the influence of an external force. The force breaks the collective phase-parity symmetry of the system and competes with the intrinsic coupling in ordering the network configuration, similar to how a magnetic field biases an interacting spin ensemble. Specifically, we demonstrate how the force can be used to control the system, and highlight the crucial role of the phase and symmetry of the force. Our work thereby provides a method to create Ising machines with arbitrary bias, extending even to exotic cases that are impossible to engineer in real spin systems.
\end{abstract}

%\pacs{Valid PACS appear here}% PACS, the Physics and Astronomy
                             % Classification Scheme.
%\keywords{Suggested keywords}%Use showkeys class option if keyword
                              %display desired
	\maketitle

%\tableofcontents

% parameters we know from the Arnold tongue: 
% omega_- = 2.6415*2*pi (MHz)
% omega_+ = 2.6485*2*pi (MHz)
% Q = 300
% V_{th} = 2.3
% lambda_{th} = 2/Q

The Kerr parametric oscillator (KPO) is a nonlinear resonator with a time-dependent harmonic potential term~\cite{Ryvkine_2006, Mahboob_2008, Wilson_2010, Eichler_2011_NL, eichler2018parametric, Gieseler_2012, Lin_2014, Puri_2017, Eichler_2018, Nosan_2019, Frimmer_2019, Grimm_2019, wang_2019, Puri_2019_PRX, Miller_2019_phase, yamaji_2022}. %Specifically, the potential is pumped with a modulation depth $\lambda$ at a frequency $f_p$ that is close to twice the bare resonator frequency $f_0$.
In a certain range of parameters, this potential modulation renders the zero-amplitude solution unstable. There, the system undergoes a spontaneous period-doubling $Z_2$ symmetry-breaking phase transition and assumes a large-amplitude oscillation at a so-called phase state. Importantly, a KPO features two such phase states with equal amplitude but phases separated by $\pi$, shown as black dots in Fig.~\ref{fig:fig1}(a).

The KPO is at the focus of much research work because its two phase states are analogous to the two polarization states of an Ising spin, ``up'' and ``down''. Consequently, it was proposed that networks of KPOs can be used to find the ground state of Ising Hamiltonians, that is, the energetically preferred configuration of a spin network~\cite{Ising_1925}. Such resonator-based Ising solvers~\cite{Mahboob_2016, Goto_2016, Puri_2019_PRX, Bello_2019, Okawachi_2020} are of high interest because the corresponding calculations are NP-hard to tackle with conventional computers~\cite{mohseni2022ising}, and yet they map to many other key optimization problems, such as the travelling salesman problem~\cite{Lucas_2014}, the MAX-CUT problem~\cite{Inagaki_2016_Science, Goto_2019}, and the number partitioning problem~\cite{Nigg_2017}. In Fig.~\ref{fig:fig1}(b), we sketch a network of two spins, where each spin takes the form of a double-well potential whose wells corresponds the two levels (``spin up'' or ``spin down'').

% Figure environment removed

In the low-amplitude limit, the oscillations of a resonator network are approximately harmonic and can be understood in terms of their normal modes. A two-KPO system starting from the zero-amplitude (Z) solution can therefore ring up to symmetric (S) or antisymmetric (A) phase states that map to ferromagnetic or antiferromagnetic Ising configurations, respectively, see Fig.~\ref{fig:fig1}(d)~\cite{Heugel_2019_TC}. At large amplitudes, the nonlinearities become significant and can lead to deviations from a normal-mode basis. The deviations manifest for example as oscillations of mixed symmetry (M)~\cite{Heugel_2022}, which have no clear Ising counterpart. Nevertheless, the normal-mode basis remains useful as a frame of reference to study both strongly~\cite{Heugel_2022} and weakly coupled KPOs~\cite{Margiani_2023}.

% Figure environment removed

Previous experimental and theoretical studies of KPO-based Ising simulators with bilinear coupling considered the case of unbiased Ising Hamiltonians, where the solutions are only defined up to a global sign~\cite{Goto_2016,Puri_2017_NC,Nigg_2017,Goto_2018,Dykman_2018,Rota_2019,Strinati2019,Heugel_2022,Margiani_2023}. For example, the state ``down-up'' shown in Fig.~\ref{fig:fig1}(b) can be identically replaced by ``up-down'', as the individual spin levels are degenerate. Many Ising problems, however, require breaking of this degeneracy; the archetypal case being a magnetic field applied to a spin ensemble. In Fig.~\ref{fig:fig1}(c), a magnetic field $B$ biases the potential to compete with the spin-spin coupling $J$. A strong field can overcome the coupling-induced ordering and dramatically change the configuration that emerges as the optimal solution of the spin system. This functionality was recently included in optical parametric systems with dissipative coupling~\cite{Takesue_2020}, which constitute an alternative route to KPO networks~\cite{Inagaki_2016,gershenzon2020exact}.

In this paper, we demonstrate experimentally how the functionality of an external $B$ field can be introduced in a classical KPO network. By applying an external force $F$ to each resonator, we can displace the network's solutions in their phase space, as indicated in Fig.~\ref{fig:fig1}(a). Our experiment features two coupled KPOs, but the concepts we present are easily extended to larger networks. We provide a general framework to understand the role of the external force term for coupled KPOs and resonator-based Ising solvers. Importantly, while this forcing can emulate the effect of a simple bias field, we show that its consequences are much richer due to the freedom of selecting the driving amplitude and phase for each resonator individually. Our study will therefore not only provide a practical guide for applications, but also motivate further fundamental research on Ising networks with inhomogeneous magnetic fields, including quantum implementations~\cite{Puri_2017_NC,Grimm_2019,Dykman_2018,Goto_2016}.

Our experiment consists of two electrical resonators that feature a nonlinear capacitance~\cite{Nosan_2019}. The system can be described by the coupled differential equations
\begin{align}%\label{eq:nonlinear_EOM}
	&\ddot{x}_i + \omega_0^2\left[1-\lambda\cos\left(4\pi f_d t\right)\right]x_i + \beta x_i^3 + \Gamma \dot{x}_i - Jx_i = F_i(t)\,,\label{eq:EOM}
\end{align}
where the displacement $x_{i}$ is the measured voltage signal of resonator $i$~\cite{Nosan_2019,Heugel_2022} as a function of time $t$. The angular resonance frequency $\omega_0/2\pi = f_0 = \SI{2.646}{\mega\hertz}$, the damping rate $\Gamma = \frac{\omega_0}{Q} = \SI{12}{\kilo\hertz}$, and the nonlinearity $\beta$ are approximately equal for both resonators, see section S1 in the supplemental material~\cite{Supplement}. Note that the capacitive diodes used to generate the nonlinearity $\beta$ mainly possess a quadratic force component (equivalent to the three-wave mixing enabled by $\chi^{(2)}$ in nonlinear optics), which allows us to implement parametric pumping $\lambda$ by driving the system with a voltage $U_p$ at the frequency $2f_d$~\cite{Nosan_2019}. The effective model we obtain in a frame rotating at $2\pi f_d$ is equivalent to that resulting from the well-known form in Eq.~\eqref{eq:EOM}~\cite{Eichler_Zilberberg_book}. Finally, $J$ quantifies the coupling between the resonators ($j\neq i$), and the $F_i(t)$ are external forces applied to the resonators individually. In the following, we will study the competition between these two effects ($J$ versus $F_i$) in ordering the phases of the two KPOs.

In Fig.~\ref{fig:fig2}(a), we demonstrate a frequency sweep by slowly increasing $f_d$ at a constant pump amplitude $U_p$ with $F_i = 0$. We use lock-in amplifiers (Zurich Instruments MFLI) to measure the amplitudes $A_i = (u_i^2 + v_i^2)^{1/2}$ and phases $\phi_i = \arctan(v_i/u_i)$ of the two resonators, where $x=u\cos(2\pi f_d t) - v\sin(2\pi f_d t)$. We observe that both resonators jump from zero to a finite amplitude around \SI{2.63}{\mega\hertz}. The phase of the two resonators assumes a well-defined value after the jump, with $\phi_1 - \phi_2 = \pi$. In our spin analogy, this corresponds to an antisymmetric ordering (``up-down''). Around \SI{2.645}{\mega\hertz}, the amplitudes jump again, accompanied by a $\pi$-shift of $\phi_1$. The resulting state is analogous to a symmetric spin state (``down-down''). Roughly at \SI{2.66}{\mega\hertz}, the amplitudes drop to zero and the phases become random again.

We repeat frequency sweeps at different values of $U_p$. In Fig.~\ref{fig:fig2}(b), the measured $\phi_2$ as a function of $f_d$ and $U_d$ is plotted. We clearly see that in each sweep, the phase assumes a well-defined value in a certain frequency range described by two overlapping, rounded triangles often called ``Arnold tongues''~\cite{Heugel_2022,Eichler_Zilberberg_book}. Importantly, the phase randomly assumes one of two values separated by $\pi$ in each sweep. This randomness is a consequence of the symmetry illustrated in Fig.~\ref{fig:fig1}(b), and the resulting spontaneous time-translation symmetry breaking at each jump from zero to finite amplitude. It is a fundamental feature of a KPO, and of networks thereof, in the absence of an external force.

To map the different symmetry phases of the system, we employ the symmetric and antisymmetric quadratures $v_S = \frac{1}{\sqrt{2}}(v_1 + v_2)$ and $v_A = \frac{1}{\sqrt{2}}(v_1 - v_2)$. Note that $u_{S,A}$ can be defined analogously and yield qualitatively similar results. In Fig.~\ref{fig:fig2}(c), we represent the total amplitude $X = (v_S^2 + v_A^2)^{1/2}$ as a brightness contrast, while the relative phase of the two resonators is shown in color code. As in a previous work~\cite{Heugel_2022}, we observe $S$, $A$ and $M$ phases inside the overlapping Arnold tongues. The center frequencies and phase symmetries of the two Arnold tongues are inherited from the normal modes of the system in the linear regime. The precise shapes of these zones are understood to originate from an interplay of the nonlinearity $\beta$ and the coupling $J$, and can be reproduced precisely with numerical simulations and with an analytical solver~\cite{kovsata2022harmonicbalance,Supplement}.

In a next step, we additionally apply an external force to each resonator. For simplicity, we select $F_i(t) = F \cos(2\pi f_d t + \theta)$ for both resonators, with $\theta$ a global phase. In Fig.~\ref{fig:fig2}(d), we see a striking change in the measured $\phi_2$ compared to Fig.~\ref{fig:fig2}(b): the phase now exhibits a well-defined value over the entire parameter space, and the jump between solutions follows a deterministic pattern. The spontaneous time-translation symmetry breaking observed in Fig.~\ref{fig:fig2}(b) is entirely replaced by a force-induced bias, as sketched in Fig.~\ref{fig:fig1}(c). The force therefore allows us to control the ``spin polarization'' of the individual KPOs in each sweep.

% Figure environment removed


% Figure environment removed

When plotting the system states in Fig.~\ref{fig:fig2}(e), we find several important differences to the unbiased example in Fig.~\ref{fig:fig2}(c). First, the outlines of the Arnold tongues are shifted by the force. This effect arises due to changes in the stability conditions of the parametric oscillators, which was previously studied for the case of a single KPO~\cite{Ryvkine_2006,Rhoads_2010,Papariello_2016,Leuch_2016,eichler2018parametric,Nosan_2019}. The most prominent manifestation of this effect is a jump (of both phase and amplitude) at the right border of the $S$ phase. This jump is caused by the termination of the selected symmetric phase state~\cite{Supplement}. For a symmetric force, we mainly observe a shift of the $S$ region. Second, the antisymmetric state labelled $A$ is no longer visible, as it is not favored by the symmetric force. Numerical simulations indicate that the $A$ state still appears at higher values of $U_p$, where the impact of the external force is reduced relative  to the parametric pump. The symmetric state $S$ therefore fills a greater portion of the diagram, in agreement with the intuition that a symmetric force $F_1 = F_2$ should favor this phase configuration. Third, the $S$ state now only comprises an ``up-up'' component, instead of allowing both ``up-up'' and ``down-down'' as in Fig.~\ref{fig:fig2}(c). The force therefore fulfills the role of a potential bias, as anticipated.

In Fig.~\ref{fig:fig2}, we employ frequency sweeps with a fixed force phase $\theta$ to study the solution space of our system. In many applications, however, it may be more useful to start from the equilibrium state ($u_i = 0$ and $v_i = 0$), and then directly access a particular solution through the correct choice of $\theta$. We demonstrate this capability experimentally in Fig.~\ref{fig:fig3}, where the system rings up to four different symmetric and mixed-symmetry states, depending on the phase $\theta$ of the symmetric external force. In the spirit of the two-spin analogy that we employ throughout the paper, we label the final states with corresponding symbols. We find that small changes in $\theta$ can cause the system to attain an entirely different state. For instance, $\theta = \SI{36}{\degree}$ leads to a symmetric state while $\theta = \SI{54}{\degree}$ leads to a mixed-symmetry state. Rotating $\theta$ by \SI{180}{\degree} inverts the final state of both resonators, leaving the relative configuration intact (e.g. ``up-up'' becomes ``down-down''). 

The deterministic experiments we reported so far always reach the same stable solutions for the same initial conditions and driving parameters. This leads to a limited understanding of our system, which possesses several stable solutions for certain positions in the $f_d$-$U_p$ diagram~\cite{Supplement}. Stochastic sampling allows us to explore these different solutions, and to assess their dwell times in the long-time limit~\cite{Margiani_2023}. In Fig.~\ref{fig:fig4}(a), we show how the system jumps between different solutions when activated by white noise and in the absence of external forcing ($F_i = 0$). For most of the time, the system jumps between the two symmetric solutions, which are more stable than the antisymmetric solutions due to their higher amplitude. A high amplitude imposes a larger ``momentum barrier'' and makes the state more stable against jumps~\cite{Margiani_2023}. In a so-called symmetry space spanned by $v_S$ and $v_A$, this dynamics result in the plot shown in Fig.~\ref{fig:fig4}(b). Here, symmetric and antisymmetric states appear at the corners of a diamond at $v_A = 0$ and $v_S = 0$, respectively (plots for the $u_S$ and $u_A$ look very similar). The edges of the diamond result from points measured during transitions between the states. Below the graph, we sketch the corresponding picture of two spins without an external field, allowing both symmetric and antisymmetric states (and fluctuations between them).

When applying an external force, we can change the relative weight (occupation probability) of these four states. In Fig.~\ref{fig:fig4}(c) and (d), we apply symmetric forces whose phase $\theta$ is tuned to favor either of the two symmetric states. As a consequence, we observe the system only in the corresponding symmetric state, and all jumps are suppressed. The spin picture with external field below each graph shows how the spins are forced to align by the homogeneous external field.

In contrast to a real (nanoscale) spin ensemble, our system allows us to arbitrarily tune the amplitude and phase of the applied force (corresponding to the strength and direction of the external field) for each KPO individually. As a demonstration example, we show in Fig.~\ref{fig:fig4}(e) and (f) the results for an antisymmetric force $F_i(t) = F \cos(2\pi f_d t + \theta_i)$ with $\theta_2 = -\theta_1$. Here, we find that the system occupies almost exclusively the selected antisymmetric state. However, the suppression of the symmetric states is weaker than in the opposite case in Fig.~\ref{fig:fig4}(c) and (d). Again, this is due to the fact that the symmetric state is generally more stable than the antisymmetric state for $J/\beta >0$~\cite{Margiani_2023}. Even larger forces would be necessary to entirely overcome this intrinsic bias.

In our experiments and in the theory analysis, we find that the external force can bias our system of coupled KPOs. In the simplest case, this bias is analogous to an external magnetic field acting on an ensemble of coupled spins.
%While we limited our demonstration to a network of two KPOs, the principle of force-induced bias can be applied to systems of arbitrary size.
The tunable phase $\theta$ of the external force assumes the role of the magnetic field angle, which can be different for each KPO, cf. Fig.~\ref{fig:fig3} and Fig.~\ref{fig:fig4}. This freedom in selecting the phases will be crucial in future experiments that go beyond conventional spin systems, as it allows access to unexplored, exotic networks that have no counterpart in solid state physics. Such novel networks include, for instance, the Ising chain in the presence of a tunable local impurity~\cite{Falk_1966}, the random-field Ising model~\cite{Grinstein_1976,belanger1991random,Bingham_2021,yao2023thermal} and corresponding avalanche models~\cite{Percovic_1995,im2009direct,Field_1995,Lahini_2017}, and the magnetic Bose polaron \cite{mistakidis2022inducing}. We believe that controlled experimental realizations of these elusive phenomena will spur new developments in theory in many directions. At the same time, our work demonstrates new strategies to control Ising machines, and to program such systems to solve complex optimization tasks~\cite{mohseni2022ising,Lucas_2014,Inagaki_2016_Science, Goto_2019,Nigg_2017}.

\appendix

\section{Setup Calibration}

Before starting any experiments with the two coupled resonators, we calibrate their characteristics individually. The resonance frequencies of our circuits can be tuned via the bias voltages applied to the diodes across \SI{47}{\mega\ohm} resistors that act as a low-pass filters~\cite{Nosan_2019}. We can apply a voltage to one resonator while keeping the other resonator at zero bias, which results in a large detuning between the resonance frequencies. Driving the biased resonator with a small external drive enables us to extract its resonance frequency, while a sweep with a large drive allows us to assess its nonlinear coefficient. This procedure is then repeated with the second resonator after tuning the bias voltage until the resonance frequencies match within the experimental uncertainty, which is on the order of \SI{100}{\hertz}. We noted, however, that the device characteristics can change slightly when they are resonantly coupled. For this reason, we extract the precise parameters of the coupled system from the measured data, as described below.

\section{Numerical Simulations}

%To confirm that the core observed effects are captured by our theory, as well as to extract parameters from the system, we performed numerical frequency sweeps of the equation of motion (eq.~\ref{eq:nonlinear_EOM}) with varying parameters. This allowed us to adjust the parameters to the measured data and select a well-matching parameter set for further theoretical analysis.

The boundary of the large-amplitude region in Fig.~2(c) corresponds to two overlapping Arnold tongues~\cite{Heugel_2022,Eichler_Zilberberg_book}. As an initial guess, we assume the two resonators to be identical. By fitting a line to this boundary, we can extract the parametric threshold $U_\mathrm{th} = \SI{1.78}{\volt}$ (the lowest tip of the tongues) and the quality factor $Q = \num{227}$, as well as a resonance frequency $f_0 = \SI{2.646}{\mega\hertz}$ (the middle between the two tongues) and frequency splitting $\Delta f = \SI{9.4}{\kilo\hertz}$ (the separation between the tongues). The coupling constant $J$ can then be calculated as $J = (2\pi)^2 f_0 \Delta f$, while the threshold $U_\mathrm{th}$ and the quality factor $Q$ allow us to transform the parametric drive amplitudes $U_d$ to unit-less parametric modulation depths $\lambda = \frac{2 U_d}{Q U_\mathrm{th}}$.

% Figure environment removed

In a second step, we use these parameters to perform numerical simulations of frequency sweeps for different modulation depths $\lambda$. We find that the phase transitions between antisymmetric, symmetric and mixed dynamical phases  ($A$, $S$ and $M$, respectively) are highly sensitive to non-degeneracies bewteen the resonator parameters. We therefore recursively modify the individual resonance frequencies, quality factors, and nonlinear coefficients slightly to optimize the agreement with the measured results. This procedure yielded the parameters used for Fig.~\ref{fig:figS1}(a).

Finally, an external force term is added to the simulations. The applied force strength and phase depends on the induction parameters between the driving coils and the resonators coils, while the measurement gain and phase depend on the induction parameters between the resonator coils and the pickup coils. As these two sets of parameters could not be calibrated independently, we decided to determine the effective strength and phase of the drive $F$ by fine-tuning until the observed features are comparable to the measured results. This leads us to the values used in Fig.~\ref{fig:figS1}(b).



\section{Number and stability of Solutions}

For a numerical analysis of the KPO network, we employ the open-source software package HarmonicBalance.jl~\cite{kovsata2022harmonicbalance}. The package allows us to transform the coupled equations from Eq.~(1) into slow-flow equations for the quadratures $u_i$ and $v_i$:
\begin{align}
    \label{eq:slow_flow_2_coupled}
    \dot{u}_{1,2}^{\phantom\dagger} &= -\frac{\Gamma  u_{1,2}^{\phantom\dagger}}{2}- \left(\frac{3 \beta}{8 \omega }X_{1,2}^2 +\frac{\omega_0^2 - \omega^2}{2\omega}+\frac{\lambda \omega_0^2}{4 \omega }\right) v_{1,2}^{\phantom\dagger} \nonumber\\&+\frac{J  v_{2,1}^{\phantom\dagger}}{2 \omega} + \frac{F_{u,1,2}}{m}\,,\nonumber\\
    \dot{v}_{1,2}^{\phantom\dagger} &= -\frac{\Gamma  v_{1,2}^{\phantom\dagger}}{2} + \left(\frac{3 \beta}{8 \omega }X_{1,2}^2 + \frac{\omega_0^2 - \omega^2 }{2\omega} -\frac{\lambda \omega_0^2}{4 \omega }\right)u_{1,2}^{\phantom\dagger}\nonumber\\&-\frac{J  u_{2,1}^{\phantom\dagger}}{2 \omega} + \frac{F_{v,1,2}}{m}\,,
\end{align}
with $X_{i} = (u_i^2+v_i^2)^{1/2}$ the total amplitude of resonator $i$ and $F_{u,i}$, and $F_{v,i}$ the quadrature components of the force acting on it, respectively. Our main focus lies in examining the system's steady states, specifically the solutions of $u_{1,2}$ and $v_{1,2}$ when $\dot{u}_{1,2}=\dot{v}_{1,2}=0$. Therefore, the task of identifying all the steady states reduces to determining all the roots of the polynomial system described by Eq.~\eqref{eq:slow_flow_2_coupled}. To accomplish this, HarmonicBalance.jl utilizes Homotopy Continuation~\cite{HomotopyContinuation.jl}, a technique that enables the computation of these roots.

We can evaluate all possible roots of Eq.~\eqref{eq:slow_flow_2_coupled} as a function of $\omega$ and $\lambda$, and assess their stability with standard methods~\cite{Eichler_Zilberberg_book}.
%Such maps provide an important basis to start the theory analysis. For instance, we can see that in the regions that we assigned to $S$ in Fig.~1, there exist only two solutions which must be the two symmetric phase states ``up-up'' and ``down-down''. The negative nonlinearity leads to a proliferation of the number of solutions towards lower frequencies, with an upper limit of $3^N$ for $N$ resonators~\cite{breiding2022algebraic}.
%% Figure environment removed
Comparing these solutions as a function of $f_d$ to an experimental sweep provides additional insights. An example comparison in the case with external force is plotted in Fig.~\ref{fig:figS3}. There, we can follow the measured system state (thick grey line) from low to high $f_d$, starting in a state near zero amplitude. Roughly at \SI{2.631}{\mega\hertz}, the system jumps to one out of four solutions with mixed symmetry (see arrow labelled $\textbf{i}$). These solutions are all different for non-identical resonators and respond sensitively to any tuning of the parameters, making it difficult to obtain a good agreement between theory and experiment without heuristic fine-tuning (which we avoided here).

% Figure environment removed

At \SI{2.636}{\mega\hertz}, the system jumps from the mixed-symmetry solution to a symmetric solution ($\textbf{ii}$). From the theory analysis, we understand that a jump is necessary because the mixed-symmetry solutions all terminate in bifurcations where they merge with unstable solutions. In principle, the system offers two symmetric ($S$) and two antisymmetric ($A$) stable solutions that the system could jump to. We interpret the selected symmetric state as dictated by the external force, as already discussed in Fig.~2. Future analysis will be dedicated to analyzing the precise rules governing the state selection.

At \SI{2.66}{\mega\hertz}, the system jumps a third time, with both resonators undergoing a phase change of $\pi$ ($\textbf{iii}$). This situation is reminiscent to what was found in a single KPO in the presence of a symmetry-breaking force~\cite{Rhoads_2010,Papariello_2016,Leuch_2016,eichler2018parametric,Nosan_2019}. There, the external force resulted in a local amplitude minimum associated to a bifurcation, and to a jump in a frequency sweep. In the present case, we observe the same phenomenon for the symmetric normal mode of the system. The third jump at $\textbf{iii}$ is therefore a direct manifestation of the interplay between the parametric pump and the external force applied to a parametric oscillator network.

%\bibliography{aipsamp}% Produces the bibliography via BibTeX.
%merlin.mbs apsrev4-1.bst 2010-07-25 4.21a (PWD, AO, DPC) hacked
%Control: key (0)
%Control: author (0) dotless jnrlst
%Control: editor formatted (1) identically to author
%Control: production of article title (0) allowed
%Control: page (1) range
%Control: year (0) verbatim
%Control: production of eprint (0) enabled
\providecommand{\noopsort}[1]{}\providecommand{\singleletter}[1]{#1}%
\begin{thebibliography}{54}%
\makeatletter
\providecommand \@ifxundefined [1]{%
 \@ifx{#1\undefined}
}%
\providecommand \@ifnum [1]{%
 \ifnum #1\expandafter \@firstoftwo
 \else \expandafter \@secondoftwo
 \fi
}%
\providecommand \@ifx [1]{%
 \ifx #1\expandafter \@firstoftwo
 \else \expandafter \@secondoftwo
 \fi
}%
\providecommand \natexlab [1]{#1}%
\providecommand \enquote  [1]{``#1''}%
\providecommand \bibnamefont  [1]{#1}%
\providecommand \bibfnamefont [1]{#1}%
\providecommand \citenamefont [1]{#1}%
\providecommand \href@noop [0]{\@secondoftwo}%
\providecommand \href [0]{\begingroup \@sanitize@url \@href}%
\providecommand \@href[1]{\@@startlink{#1}\@@href}%
\providecommand \@@href[1]{\endgroup#1\@@endlink}%
\providecommand \@sanitize@url [0]{\catcode `\\12\catcode `\$12\catcode
  `\&12\catcode `\#12\catcode `\^12\catcode `\_12\catcode `\%12\relax}%
\providecommand \@@startlink[1]{}%
\providecommand \@@endlink[0]{}%
\providecommand \url  [0]{\begingroup\@sanitize@url \@url }%
\providecommand \@url [1]{\endgroup\@href {#1}{\urlprefix }}%
\providecommand \urlprefix  [0]{URL }%
\providecommand \Eprint [0]{\href }%
\providecommand \doibase [0]{http://dx.doi.org/}%
\providecommand \selectlanguage [0]{\@gobble}%
\providecommand \bibinfo  [0]{\@secondoftwo}%
\providecommand \bibfield  [0]{\@secondoftwo}%
\providecommand \translation [1]{[#1]}%
\providecommand \BibitemOpen [0]{}%
\providecommand \bibitemStop [0]{}%
\providecommand \bibitemNoStop [0]{.\EOS\space}%
\providecommand \EOS [0]{\spacefactor3000\relax}%
\providecommand \BibitemShut  [1]{\csname bibitem#1\endcsname}%
\let\auto@bib@innerbib\@empty
%</preamble>
\bibitem [{\citenamefont {Ryvkine}\ and\ \citenamefont
  {Dykman}(2006)}]{Ryvkine_2006}%
  \BibitemOpen
  \bibfield  {author} {\bibinfo {author} {\bibfnamefont {D}~\bibnamefont
  {Ryvkine}}\ and\ \bibinfo {author} {\bibfnamefont {M~I}\ \bibnamefont
  {Dykman}},\ }\bibfield  {title} {\enquote {\bibinfo {title} {Resonant
  symmetry lifting in a parametrically modulated oscillator},}\ }\href@noop {}
  {\bibfield  {journal} {\bibinfo  {journal} {Physical Review E}\ }\textbf
  {\bibinfo {volume} {74}},\ \bibinfo {pages} {061118} (\bibinfo {year}
  {2006})}\BibitemShut {NoStop}%
\bibitem [{\citenamefont {Mahboob}\ and\ \citenamefont
  {Yamaguchi}(2008)}]{Mahboob_2008}%
  \BibitemOpen
  \bibfield  {author} {\bibinfo {author} {\bibfnamefont {I.}~\bibnamefont
  {Mahboob}}\ and\ \bibinfo {author} {\bibfnamefont {H.}~\bibnamefont
  {Yamaguchi}},\ }\bibfield  {title} {\enquote {\bibinfo {title} {Bit storage
  and bit flip operations in an electromechanical oscillator},}\ }\href
  {\doibase 10.1038/nnano.2008.84} {\bibfield  {journal} {\bibinfo  {journal}
  {Nature Nanotechnology}\ }\textbf {\bibinfo {volume} {3}},\ \bibinfo {pages}
  {275--279} (\bibinfo {year} {2008})}\BibitemShut {NoStop}%
\bibitem [{\citenamefont {Wilson}\ \emph {et~al.}(2010)\citenamefont {Wilson},
  \citenamefont {Duty}, \citenamefont {Sandberg}, \citenamefont {Persson},
  \citenamefont {Shumeiko},\ and\ \citenamefont {Delsing}}]{Wilson_2010}%
  \BibitemOpen
  \bibfield  {author} {\bibinfo {author} {\bibfnamefont {C.~M.}\ \bibnamefont
  {Wilson}}, \bibinfo {author} {\bibfnamefont {T.}~\bibnamefont {Duty}},
  \bibinfo {author} {\bibfnamefont {M.}~\bibnamefont {Sandberg}}, \bibinfo
  {author} {\bibfnamefont {F.}~\bibnamefont {Persson}}, \bibinfo {author}
  {\bibfnamefont {V.}~\bibnamefont {Shumeiko}}, \ and\ \bibinfo {author}
  {\bibfnamefont {P.}~\bibnamefont {Delsing}},\ }\bibfield  {title} {\enquote
  {\bibinfo {title} {Photon generation in an electromagnetic cavity with a
  time-dependent boundary},}\ }\href {\doibase 10.1103/PhysRevLett.105.233907}
  {\bibfield  {journal} {\bibinfo  {journal} {Phys. Rev. Lett.}\ }\textbf
  {\bibinfo {volume} {105}},\ \bibinfo {pages} {233907} (\bibinfo {year}
  {2010})}\BibitemShut {NoStop}%
\bibitem [{\citenamefont {Eichler}\ \emph {et~al.}(2011)\citenamefont
  {Eichler}, \citenamefont {Chaste}, \citenamefont {Moser},\ and\ \citenamefont
  {Bachtold}}]{Eichler_2011_NL}%
  \BibitemOpen
  \bibfield  {author} {\bibinfo {author} {\bibfnamefont {Alexander}\
  \bibnamefont {Eichler}}, \bibinfo {author} {\bibfnamefont {Julien}\
  \bibnamefont {Chaste}}, \bibinfo {author} {\bibfnamefont {Joel}\ \bibnamefont
  {Moser}}, \ and\ \bibinfo {author} {\bibfnamefont {Adrian}\ \bibnamefont
  {Bachtold}},\ }\bibfield  {title} {\enquote {\bibinfo {title} {Parametric
  amplification and self-oscillation in a nanotube mechanical resonator},}\
  }\href {\doibase 10.1021/nl200950d} {\bibfield  {journal} {\bibinfo
  {journal} {Nano Letters}\ }\textbf {\bibinfo {volume} {11}},\ \bibinfo
  {pages} {2699--2703} (\bibinfo {year} {2011})}\BibitemShut {NoStop}%
\bibitem [{\citenamefont {Eichler}\ \emph
  {et~al.}(2018{\natexlab{a}})\citenamefont {Eichler}, \citenamefont {Heugel},
  \citenamefont {Leuch}, \citenamefont {Degen}, \citenamefont {Chitra},\ and\
  \citenamefont {Zilberberg}}]{eichler2018parametric}%
  \BibitemOpen
  \bibfield  {author} {\bibinfo {author} {\bibfnamefont {Alexander}\
  \bibnamefont {Eichler}}, \bibinfo {author} {\bibfnamefont {Toni~L}\
  \bibnamefont {Heugel}}, \bibinfo {author} {\bibfnamefont {Anina}\
  \bibnamefont {Leuch}}, \bibinfo {author} {\bibfnamefont {Christian~L}\
  \bibnamefont {Degen}}, \bibinfo {author} {\bibfnamefont {Ramasubramanian}\
  \bibnamefont {Chitra}}, \ and\ \bibinfo {author} {\bibfnamefont {Oded}\
  \bibnamefont {Zilberberg}},\ }\bibfield  {title} {\enquote {\bibinfo {title}
  {A parametric symmetry breaking transducer},}\ }\href@noop {} {\bibfield
  {journal} {\bibinfo  {journal} {Applied Physics Letters}\ }\textbf {\bibinfo
  {volume} {112}},\ \bibinfo {pages} {233105} (\bibinfo {year}
  {2018}{\natexlab{a}})}\BibitemShut {NoStop}%
\bibitem [{\citenamefont {Gieseler}\ \emph {et~al.}(2012)\citenamefont
  {Gieseler}, \citenamefont {Deutsch}, \citenamefont {Quidant},\ and\
  \citenamefont {Novotny}}]{Gieseler_2012}%
  \BibitemOpen
  \bibfield  {author} {\bibinfo {author} {\bibfnamefont {Jan}\ \bibnamefont
  {Gieseler}}, \bibinfo {author} {\bibfnamefont {Bradley}\ \bibnamefont
  {Deutsch}}, \bibinfo {author} {\bibfnamefont {Romain}\ \bibnamefont
  {Quidant}}, \ and\ \bibinfo {author} {\bibfnamefont {Lukas}\ \bibnamefont
  {Novotny}},\ }\bibfield  {title} {\enquote {\bibinfo {title} {Subkelvin
  parametric feedback cooling of a laser-trapped nanoparticle},}\ }\href
  {\doibase 10.1103/PhysRevLett.109.103603} {\bibfield  {journal} {\bibinfo
  {journal} {Phys. Rev. Lett.}\ }\textbf {\bibinfo {volume} {109}},\ \bibinfo
  {pages} {103603} (\bibinfo {year} {2012})}\BibitemShut {NoStop}%
\bibitem [{\citenamefont {Lin}\ \emph {et~al.}(2014)\citenamefont {Lin},
  \citenamefont {Inomata}, \citenamefont {Koshino}, \citenamefont {Oliver},
  \citenamefont {Nakamura}, \citenamefont {Tsai},\ and\ \citenamefont
  {Yamamoto}}]{Lin_2014}%
  \BibitemOpen
  \bibfield  {author} {\bibinfo {author} {\bibfnamefont {Z.R.}\ \bibnamefont
  {Lin}}, \bibinfo {author} {\bibfnamefont {K.}~\bibnamefont {Inomata}},
  \bibinfo {author} {\bibfnamefont {K.}~\bibnamefont {Koshino}}, \bibinfo
  {author} {\bibfnamefont {W.~D.}\ \bibnamefont {Oliver}}, \bibinfo {author}
  {\bibfnamefont {Y.}~\bibnamefont {Nakamura}}, \bibinfo {author}
  {\bibfnamefont {J.~S.}\ \bibnamefont {Tsai}}, \ and\ \bibinfo {author}
  {\bibfnamefont {T.}~\bibnamefont {Yamamoto}},\ }\bibfield  {title} {\enquote
  {\bibinfo {title} {Josephson parametric phase-locked oscillator and its
  application to dispersive readout of superconducting qubits},}\ }\href
  {\doibase 10.1038/ncomms5480} {\bibfield  {journal} {\bibinfo  {journal}
  {Nature Communications}\ }\textbf {\bibinfo {volume} {5}},\ \bibinfo {pages}
  {4480} (\bibinfo {year} {2014})}\BibitemShut {NoStop}%
\bibitem [{\citenamefont {Puri}\ \emph
  {et~al.}(2017{\natexlab{a}})\citenamefont {Puri}, \citenamefont {Boutin},\
  and\ \citenamefont {Blais}}]{Puri_2017}%
  \BibitemOpen
  \bibfield  {author} {\bibinfo {author} {\bibfnamefont {S.}~\bibnamefont
  {Puri}}, \bibinfo {author} {\bibfnamefont {S.}~\bibnamefont {Boutin}}, \ and\
  \bibinfo {author} {\bibfnamefont {A.}~\bibnamefont {Blais}},\ }\bibfield
  {title} {\enquote {\bibinfo {title} {Engineering the quantum states of light
  in a kerr-nonlinear resonator by two-photon driving},}\ }\href {\doibase
  10.1038/s41534-017-0019-1} {\bibfield  {journal} {\bibinfo  {journal} {npj
  Quantum Information}\ }\textbf {\bibinfo {volume} {3}},\ \bibinfo {pages}
  {18} (\bibinfo {year} {2017}{\natexlab{a}})}\BibitemShut {NoStop}%
\bibitem [{\citenamefont {Eichler}\ \emph
  {et~al.}(2018{\natexlab{b}})\citenamefont {Eichler}, \citenamefont {Heugel},
  \citenamefont {Leuch}, \citenamefont {Degen}, \citenamefont {Chitra},\ and\
  \citenamefont {Zilberberg}}]{Eichler_2018}%
  \BibitemOpen
  \bibfield  {author} {\bibinfo {author} {\bibfnamefont {Alexander}\
  \bibnamefont {Eichler}}, \bibinfo {author} {\bibfnamefont {Toni~L.}\
  \bibnamefont {Heugel}}, \bibinfo {author} {\bibfnamefont {Anina}\
  \bibnamefont {Leuch}}, \bibinfo {author} {\bibfnamefont {Christian~L.}\
  \bibnamefont {Degen}}, \bibinfo {author} {\bibfnamefont {R.}~\bibnamefont
  {Chitra}}, \ and\ \bibinfo {author} {\bibfnamefont {Oded}\ \bibnamefont
  {Zilberberg}},\ }\bibfield  {title} {\enquote {\bibinfo {title} {{A
  parametric symmetry breaking transducer}},}\ }\href {\doibase
  10.1063/1.5031058} {\bibfield  {journal} {\bibinfo  {journal} {Applied
  Physics Letters}\ }\textbf {\bibinfo {volume} {112}},\ \bibinfo {pages}
  {233105} (\bibinfo {year} {2018}{\natexlab{b}})},\ \Eprint
  {http://arxiv.org/abs/https://pubs.aip.org/aip/apl/article-pdf/doi/10.1063/1.5031058/14512731/233105\_1\_online.pdf}
  {https://pubs.aip.org/aip/apl/article-pdf/doi/10.1063/1.5031058/14512731/233105\_1\_online.pdf}
  \BibitemShut {NoStop}%
\bibitem [{\citenamefont {Nosan}\ \emph {et~al.}(2019)\citenamefont {Nosan},
  \citenamefont {M\"arki}, \citenamefont {Hauff}, \citenamefont {Knaut},\ and\
  \citenamefont {Eichler}}]{Nosan_2019}%
  \BibitemOpen
  \bibfield  {author} {\bibinfo {author} {\bibfnamefont {Z.}~\bibnamefont
  {Nosan}}, \bibinfo {author} {\bibfnamefont {P.}~\bibnamefont {M\"arki}},
  \bibinfo {author} {\bibfnamefont {N.}~\bibnamefont {Hauff}}, \bibinfo
  {author} {\bibfnamefont {C.}~\bibnamefont {Knaut}}, \ and\ \bibinfo {author}
  {\bibfnamefont {A.}~\bibnamefont {Eichler}},\ }\bibfield  {title} {\enquote
  {\bibinfo {title} {Gate-controlled phase switching in a parametron},}\ }\href
  {\doibase 10.1103/PhysRevE.99.062205} {\bibfield  {journal} {\bibinfo
  {journal} {Phys. Rev. E}\ }\textbf {\bibinfo {volume} {99}},\ \bibinfo
  {pages} {062205} (\bibinfo {year} {2019})}\BibitemShut {NoStop}%
\bibitem [{\citenamefont {Frimmer}\ \emph {et~al.}(2019)\citenamefont
  {Frimmer}, \citenamefont {Heugel}, \citenamefont {Nosan}, \citenamefont
  {Tebbenjohanns}, \citenamefont {H\"alg}, \citenamefont {Akin}, \citenamefont
  {Degen}, \citenamefont {Novotny}, \citenamefont {Chitra}, \citenamefont
  {Zilberberg},\ and\ \citenamefont {Eichler}}]{Frimmer_2019}%
  \BibitemOpen
  \bibfield  {author} {\bibinfo {author} {\bibfnamefont {Martin}\ \bibnamefont
  {Frimmer}}, \bibinfo {author} {\bibfnamefont {Toni~L.}\ \bibnamefont
  {Heugel}}, \bibinfo {author} {\bibfnamefont {Ziga}\ \bibnamefont {Nosan}},
  \bibinfo {author} {\bibfnamefont {Felix}\ \bibnamefont {Tebbenjohanns}},
  \bibinfo {author} {\bibfnamefont {David}\ \bibnamefont {H\"alg}}, \bibinfo
  {author} {\bibfnamefont {Abdulkadir}\ \bibnamefont {Akin}}, \bibinfo {author}
  {\bibfnamefont {Christian~L.}\ \bibnamefont {Degen}}, \bibinfo {author}
  {\bibfnamefont {Lukas}\ \bibnamefont {Novotny}}, \bibinfo {author}
  {\bibfnamefont {R.}~\bibnamefont {Chitra}}, \bibinfo {author} {\bibfnamefont
  {Oded}\ \bibnamefont {Zilberberg}}, \ and\ \bibinfo {author} {\bibfnamefont
  {Alexander}\ \bibnamefont {Eichler}},\ }\bibfield  {title} {\enquote
  {\bibinfo {title} {Rapid flipping of parametric phase states},}\ }\href
  {\doibase 10.1103/PhysRevLett.123.254102} {\bibfield  {journal} {\bibinfo
  {journal} {Phys. Rev. Lett.}\ }\textbf {\bibinfo {volume} {123}},\ \bibinfo
  {pages} {254102} (\bibinfo {year} {2019})}\BibitemShut {NoStop}%
\bibitem [{\citenamefont {Grimm}\ \emph {et~al.}(2019)\citenamefont {Grimm},
  \citenamefont {Frattini}, \citenamefont {Puri}, \citenamefont {Mundhada},
  \citenamefont {Touzard}, \citenamefont {Mirrahimi}, \citenamefont {Girvin},
  \citenamefont {Shankar},\ and\ \citenamefont {Devoret}}]{Grimm_2019}%
  \BibitemOpen
  \bibfield  {author} {\bibinfo {author} {\bibfnamefont {Alexander}\
  \bibnamefont {Grimm}}, \bibinfo {author} {\bibfnamefont {Nicholas~E.}\
  \bibnamefont {Frattini}}, \bibinfo {author} {\bibfnamefont {Shruti}\
  \bibnamefont {Puri}}, \bibinfo {author} {\bibfnamefont {Shantanu~O.}\
  \bibnamefont {Mundhada}}, \bibinfo {author} {\bibfnamefont {Steven}\
  \bibnamefont {Touzard}}, \bibinfo {author} {\bibfnamefont {Mazyar}\
  \bibnamefont {Mirrahimi}}, \bibinfo {author} {\bibfnamefont {Steven~M.}\
  \bibnamefont {Girvin}}, \bibinfo {author} {\bibfnamefont {Shyam}\
  \bibnamefont {Shankar}}, \ and\ \bibinfo {author} {\bibfnamefont {Michel~H.}\
  \bibnamefont {Devoret}},\ }\bibfield  {title} {\enquote {\bibinfo {title}
  {Stabilization and operation of a kerr-cat qubit},}\ }\href {\doibase
  10.1038/s41586-020-2587-z} {\bibfield  {journal} {\bibinfo  {journal}
  {Nature}\ }\textbf {\bibinfo {volume} {584}},\ \bibinfo {pages} {205--209}
  (\bibinfo {year} {2019})}\BibitemShut {NoStop}%
\bibitem [{\citenamefont {Wang}\ \emph {et~al.}(2019)\citenamefont {Wang},
  \citenamefont {Pechal}, \citenamefont {Wollack}, \citenamefont
  {Arrangoiz-Arriola}, \citenamefont {Gao}, \citenamefont {Lee},\ and\
  \citenamefont {Safavi-Naeini}}]{wang_2019}%
  \BibitemOpen
  \bibfield  {author} {\bibinfo {author} {\bibfnamefont {Zhaoyou}\ \bibnamefont
  {Wang}}, \bibinfo {author} {\bibfnamefont {Marek}\ \bibnamefont {Pechal}},
  \bibinfo {author} {\bibfnamefont {E.~Alex}\ \bibnamefont {Wollack}}, \bibinfo
  {author} {\bibfnamefont {Patricio}\ \bibnamefont {Arrangoiz-Arriola}},
  \bibinfo {author} {\bibfnamefont {Maodong}\ \bibnamefont {Gao}}, \bibinfo
  {author} {\bibfnamefont {Nathan~R.}\ \bibnamefont {Lee}}, \ and\ \bibinfo
  {author} {\bibfnamefont {Amir~H.}\ \bibnamefont {Safavi-Naeini}},\ }\bibfield
   {title} {\enquote {\bibinfo {title} {Quantum dynamics of a few-photon
  parametric oscillator},}\ }\href {\doibase 10.1103/PhysRevX.9.021049}
  {\bibfield  {journal} {\bibinfo  {journal} {Phys. Rev. X}\ }\textbf {\bibinfo
  {volume} {9}},\ \bibinfo {pages} {021049} (\bibinfo {year}
  {2019})}\BibitemShut {NoStop}%
\bibitem [{\citenamefont {Puri}\ \emph {et~al.}(2019)\citenamefont {Puri},
  \citenamefont {Grimm}, \citenamefont {Campagne-Ibarcq}, \citenamefont
  {Eickbusch}, \citenamefont {Noh}, \citenamefont {Roberts}, \citenamefont
  {Jiang}, \citenamefont {Mirrahimi}, \citenamefont {Devoret},\ and\
  \citenamefont {Girvin}}]{Puri_2019_PRX}%
  \BibitemOpen
  \bibfield  {author} {\bibinfo {author} {\bibfnamefont {Shruti}\ \bibnamefont
  {Puri}}, \bibinfo {author} {\bibfnamefont {Alexander}\ \bibnamefont {Grimm}},
  \bibinfo {author} {\bibfnamefont {Philippe}\ \bibnamefont {Campagne-Ibarcq}},
  \bibinfo {author} {\bibfnamefont {Alec}\ \bibnamefont {Eickbusch}}, \bibinfo
  {author} {\bibfnamefont {Kyungjoo}\ \bibnamefont {Noh}}, \bibinfo {author}
  {\bibfnamefont {Gabrielle}\ \bibnamefont {Roberts}}, \bibinfo {author}
  {\bibfnamefont {Liang}\ \bibnamefont {Jiang}}, \bibinfo {author}
  {\bibfnamefont {Mazyar}\ \bibnamefont {Mirrahimi}}, \bibinfo {author}
  {\bibfnamefont {Michel~H.}\ \bibnamefont {Devoret}}, \ and\ \bibinfo {author}
  {\bibfnamefont {S.~M.}\ \bibnamefont {Girvin}},\ }\bibfield  {title}
  {\enquote {\bibinfo {title} {Stabilized cat in a driven nonlinear cavity: A
  fault-tolerant error syndrome detector},}\ }\href {\doibase
  10.1103/PhysRevX.9.041009} {\bibfield  {journal} {\bibinfo  {journal} {Phys.
  Rev. X}\ }\textbf {\bibinfo {volume} {9}},\ \bibinfo {pages} {041009}
  (\bibinfo {year} {2019})}\BibitemShut {NoStop}%
\bibitem [{\citenamefont {Miller}\ \emph {et~al.}(2019)\citenamefont {Miller},
  \citenamefont {Shin}, \citenamefont {Kwon}, \citenamefont {Shaw},\ and\
  \citenamefont {Kenny}}]{Miller_2019_phase}%
  \BibitemOpen
  \bibfield  {author} {\bibinfo {author} {\bibfnamefont {James~M.L.}\
  \bibnamefont {Miller}}, \bibinfo {author} {\bibfnamefont {Dongsuk~D.}\
  \bibnamefont {Shin}}, \bibinfo {author} {\bibfnamefont {Hyun-Keun}\
  \bibnamefont {Kwon}}, \bibinfo {author} {\bibfnamefont {Steven~W.}\
  \bibnamefont {Shaw}}, \ and\ \bibinfo {author} {\bibfnamefont {Thomas~W.}\
  \bibnamefont {Kenny}},\ }\bibfield  {title} {\enquote {\bibinfo {title}
  {Phase control of self-excited parametric resonators},}\ }\href {\doibase
  10.1103/PhysRevApplied.12.044053} {\bibfield  {journal} {\bibinfo  {journal}
  {Phys. Rev. Applied}\ }\textbf {\bibinfo {volume} {12}},\ \bibinfo {pages}
  {044053} (\bibinfo {year} {2019})}\BibitemShut {NoStop}%
\bibitem [{\citenamefont {Yamaji}\ \emph {et~al.}(2022)\citenamefont {Yamaji},
  \citenamefont {Kagami}, \citenamefont {Yamaguchi}, \citenamefont {Satoh},
  \citenamefont {Koshino}, \citenamefont {Goto}, \citenamefont {Lin},
  \citenamefont {Nakamura},\ and\ \citenamefont {Yamamoto}}]{yamaji_2022}%
  \BibitemOpen
  \bibfield  {author} {\bibinfo {author} {\bibfnamefont {T.}~\bibnamefont
  {Yamaji}}, \bibinfo {author} {\bibfnamefont {S.}~\bibnamefont {Kagami}},
  \bibinfo {author} {\bibfnamefont {A.}~\bibnamefont {Yamaguchi}}, \bibinfo
  {author} {\bibfnamefont {T.}~\bibnamefont {Satoh}}, \bibinfo {author}
  {\bibfnamefont {K.}~\bibnamefont {Koshino}}, \bibinfo {author} {\bibfnamefont
  {H.}~\bibnamefont {Goto}}, \bibinfo {author} {\bibfnamefont {Z.~R.}\
  \bibnamefont {Lin}}, \bibinfo {author} {\bibfnamefont {Y.}~\bibnamefont
  {Nakamura}}, \ and\ \bibinfo {author} {\bibfnamefont {T.}~\bibnamefont
  {Yamamoto}},\ }\bibfield  {title} {\enquote {\bibinfo {title} {Spectroscopic
  observation of the crossover from a classical duffing oscillator to a kerr
  parametric oscillator},}\ }\href {\doibase 10.1103/PhysRevA.105.023519}
  {\bibfield  {journal} {\bibinfo  {journal} {Phys. Rev. A}\ }\textbf {\bibinfo
  {volume} {105}},\ \bibinfo {pages} {023519} (\bibinfo {year}
  {2022})}\BibitemShut {NoStop}%
\bibitem [{\citenamefont {Ising}(1925)}]{Ising_1925}%
  \BibitemOpen
  \bibfield  {author} {\bibinfo {author} {\bibfnamefont {Ernst}\ \bibnamefont
  {Ising}},\ }\bibfield  {title} {\enquote {\bibinfo {title} {Beitrag zur
  theorie des ferromagnetismus},}\ }\href {\doibase 10.1007/BF02980577}
  {\bibfield  {journal} {\bibinfo  {journal} {Zeitschrift f{\"u}r Physik}\
  }\textbf {\bibinfo {volume} {31}},\ \bibinfo {pages} {253--258} (\bibinfo
  {year} {1925})}\BibitemShut {NoStop}%
\bibitem [{\citenamefont {Mahboob}\ \emph {et~al.}(2016)\citenamefont
  {Mahboob}, \citenamefont {Okamoto},\ and\ \citenamefont
  {Yamaguchi}}]{Mahboob_2016}%
  \BibitemOpen
  \bibfield  {author} {\bibinfo {author} {\bibfnamefont {Imran}\ \bibnamefont
  {Mahboob}}, \bibinfo {author} {\bibfnamefont {Hajime}\ \bibnamefont
  {Okamoto}}, \ and\ \bibinfo {author} {\bibfnamefont {Hiroshi}\ \bibnamefont
  {Yamaguchi}},\ }\bibfield  {title} {\enquote {\bibinfo {title} {An
  electromechanical ising hamiltonian},}\ }\href
  {https://advances.sciencemag.org/content/2/6/e1600236} {\bibfield  {journal}
  {\bibinfo  {journal} {Science Advances}\ }\textbf {\bibinfo {volume} {2}},\
  \bibinfo {pages} {e1600236} (\bibinfo {year} {2016})}\BibitemShut {NoStop}%
\bibitem [{\citenamefont {Goto}(2016)}]{Goto_2016}%
  \BibitemOpen
  \bibfield  {author} {\bibinfo {author} {\bibfnamefont {H.}~\bibnamefont
  {Goto}},\ }\bibfield  {title} {\enquote {\bibinfo {title} {Bifurcation-based
  adiabatic quantum computation with a nonlinear oscillator network},}\ }\href
  {\doibase 10.1038/srep21686} {\bibfield  {journal} {\bibinfo  {journal}
  {Scientific Reports}\ }\textbf {\bibinfo {volume} {6}},\ \bibinfo {pages}
  {21686} (\bibinfo {year} {2016})}\BibitemShut {NoStop}%
\bibitem [{\citenamefont {Bello}\ \emph {et~al.}(2019)\citenamefont {Bello},
  \citenamefont {Calvanese~Strinati}, \citenamefont {Dalla~Torre},\ and\
  \citenamefont {Pe'er}}]{Bello_2019}%
  \BibitemOpen
  \bibfield  {author} {\bibinfo {author} {\bibfnamefont {Leon}\ \bibnamefont
  {Bello}}, \bibinfo {author} {\bibfnamefont {Marcello}\ \bibnamefont
  {Calvanese~Strinati}}, \bibinfo {author} {\bibfnamefont {Emanuele~G.}\
  \bibnamefont {Dalla~Torre}}, \ and\ \bibinfo {author} {\bibfnamefont {Avi}\
  \bibnamefont {Pe'er}},\ }\bibfield  {title} {\enquote {\bibinfo {title}
  {Persistent coherent beating in coupled parametric oscillators},}\ }\href
  {\doibase 10.1103/PhysRevLett.123.083901} {\bibfield  {journal} {\bibinfo
  {journal} {Phys. Rev. Lett.}\ }\textbf {\bibinfo {volume} {123}},\ \bibinfo
  {pages} {083901} (\bibinfo {year} {2019})}\BibitemShut {NoStop}%
\bibitem [{\citenamefont {Okawachi}\ \emph {et~al.}(2020)\citenamefont
  {Okawachi}, \citenamefont {Yu}, \citenamefont {Jang}, \citenamefont {Ji},
  \citenamefont {Zhao}, \citenamefont {Kim}, \citenamefont {Lipson},\ and\
  \citenamefont {Gaeta}}]{Okawachi_2020}%
  \BibitemOpen
  \bibfield  {author} {\bibinfo {author} {\bibfnamefont {Yoshitomo}\
  \bibnamefont {Okawachi}}, \bibinfo {author} {\bibfnamefont {Mengjie}\
  \bibnamefont {Yu}}, \bibinfo {author} {\bibfnamefont {Jae~K.}\ \bibnamefont
  {Jang}}, \bibinfo {author} {\bibfnamefont {Xingchen}\ \bibnamefont {Ji}},
  \bibinfo {author} {\bibfnamefont {Yun}\ \bibnamefont {Zhao}}, \bibinfo
  {author} {\bibfnamefont {Bok~Young}\ \bibnamefont {Kim}}, \bibinfo {author}
  {\bibfnamefont {Michal}\ \bibnamefont {Lipson}}, \ and\ \bibinfo {author}
  {\bibfnamefont {Alexander~L.}\ \bibnamefont {Gaeta}},\ }\bibfield  {title}
  {\enquote {\bibinfo {title} {Demonstration of chip-based coupled degenerate
  optical parametric oscillators for realizing a nanophotonic spin-glass},}\
  }\href {\doibase 10.1038/s41467-020-17919-6} {\bibfield  {journal} {\bibinfo
  {journal} {Nature Communications}\ }\textbf {\bibinfo {volume} {11}},\
  \bibinfo {pages} {4119} (\bibinfo {year} {2020})}\BibitemShut {NoStop}%
\bibitem [{\citenamefont {Mohseni}\ \emph {et~al.}(2022)\citenamefont
  {Mohseni}, \citenamefont {McMahon},\ and\ \citenamefont
  {Byrnes}}]{mohseni2022ising}%
  \BibitemOpen
  \bibfield  {author} {\bibinfo {author} {\bibfnamefont {Naeimeh}\ \bibnamefont
  {Mohseni}}, \bibinfo {author} {\bibfnamefont {Peter~L}\ \bibnamefont
  {McMahon}}, \ and\ \bibinfo {author} {\bibfnamefont {Tim}\ \bibnamefont
  {Byrnes}},\ }\bibfield  {title} {\enquote {\bibinfo {title} {Ising machines
  as hardware solvers of combinatorial optimization problems},}\ }\href@noop {}
  {\bibfield  {journal} {\bibinfo  {journal} {Nature Reviews Physics}\ ,\
  \bibinfo {pages} {1--17}} (\bibinfo {year} {2022})}\BibitemShut {NoStop}%
\bibitem [{\citenamefont {Lucas}(2014)}]{Lucas_2014}%
  \BibitemOpen
  \bibfield  {author} {\bibinfo {author} {\bibfnamefont {Andrew}\ \bibnamefont
  {Lucas}},\ }\bibfield  {title} {\enquote {\bibinfo {title} {Ising
  formulations of many np problems},}\ }\href@noop {} {\bibfield  {journal}
  {\bibinfo  {journal} {Frontiers in Physics}\ }\textbf {\bibinfo {volume}
  {2}},\ \bibinfo {pages} {5} (\bibinfo {year} {2014})}\BibitemShut {NoStop}%
\bibitem [{\citenamefont {Inagaki}\ \emph
  {et~al.}(2016{\natexlab{a}})\citenamefont {Inagaki}, \citenamefont
  {Haribara}, \citenamefont {Igarashi}, \citenamefont {Sonobe}, \citenamefont
  {Tamate}, \citenamefont {Honjo}, \citenamefont {Marandi}, \citenamefont
  {McMahon}, \citenamefont {Umeki}, \citenamefont {Enbutsu}, \citenamefont
  {Tadanaga}, \citenamefont {Takenouchi}, \citenamefont {Aihara}, \citenamefont
  {Kawarabayashi}, \citenamefont {Inoue}, \citenamefont {Utsunomiya},\ and\
  \citenamefont {Takesue}}]{Inagaki_2016_Science}%
  \BibitemOpen
  \bibfield  {author} {\bibinfo {author} {\bibfnamefont {Takahiro}\
  \bibnamefont {Inagaki}}, \bibinfo {author} {\bibfnamefont {Yoshitaka}\
  \bibnamefont {Haribara}}, \bibinfo {author} {\bibfnamefont {Koji}\
  \bibnamefont {Igarashi}}, \bibinfo {author} {\bibfnamefont {Tomohiro}\
  \bibnamefont {Sonobe}}, \bibinfo {author} {\bibfnamefont {Shuhei}\
  \bibnamefont {Tamate}}, \bibinfo {author} {\bibfnamefont {Toshimori}\
  \bibnamefont {Honjo}}, \bibinfo {author} {\bibfnamefont {Alireza}\
  \bibnamefont {Marandi}}, \bibinfo {author} {\bibfnamefont {Peter~L.}\
  \bibnamefont {McMahon}}, \bibinfo {author} {\bibfnamefont {Takeshi}\
  \bibnamefont {Umeki}}, \bibinfo {author} {\bibfnamefont {Koji}\ \bibnamefont
  {Enbutsu}}, \bibinfo {author} {\bibfnamefont {Osamu}\ \bibnamefont
  {Tadanaga}}, \bibinfo {author} {\bibfnamefont {Hirokazu}\ \bibnamefont
  {Takenouchi}}, \bibinfo {author} {\bibfnamefont {Kazuyuki}\ \bibnamefont
  {Aihara}}, \bibinfo {author} {\bibfnamefont {Ken-ichi}\ \bibnamefont
  {Kawarabayashi}}, \bibinfo {author} {\bibfnamefont {Kyo}\ \bibnamefont
  {Inoue}}, \bibinfo {author} {\bibfnamefont {Shoko}\ \bibnamefont
  {Utsunomiya}}, \ and\ \bibinfo {author} {\bibfnamefont {Hiroki}\ \bibnamefont
  {Takesue}},\ }\bibfield  {title} {\enquote {\bibinfo {title} {A coherent
  ising machine for 2000-node optimization problems},}\ }\href {\doibase
  10.1126/science.aah4243} {\bibfield  {journal} {\bibinfo  {journal}
  {Science}\ }\textbf {\bibinfo {volume} {354}},\ \bibinfo {pages} {603--606}
  (\bibinfo {year} {2016}{\natexlab{a}})}\BibitemShut {NoStop}%
\bibitem [{\citenamefont {Goto}\ \emph {et~al.}(2019)\citenamefont {Goto},
  \citenamefont {Tatsumura},\ and\ \citenamefont {Dixon}}]{Goto_2019}%
  \BibitemOpen
  \bibfield  {author} {\bibinfo {author} {\bibfnamefont {Hayato}\ \bibnamefont
  {Goto}}, \bibinfo {author} {\bibfnamefont {Kosuke}\ \bibnamefont
  {Tatsumura}}, \ and\ \bibinfo {author} {\bibfnamefont {Alexander~R}\
  \bibnamefont {Dixon}},\ }\bibfield  {title} {\enquote {\bibinfo {title}
  {Combinatorial optimization by simulating adiabatic bifurcations in nonlinear
  hamiltonian systems},}\ }\href {\doibase 10.1126/sciadv.aav2372} {\bibfield
  {journal} {\bibinfo  {journal} {Science advances}\ }\textbf {\bibinfo
  {volume} {5}},\ \bibinfo {pages} {eaav2372} (\bibinfo {year}
  {2019})}\BibitemShut {NoStop}%
\bibitem [{\citenamefont {Nigg}\ \emph {et~al.}(2017)\citenamefont {Nigg},
  \citenamefont {L{\"o}rch},\ and\ \citenamefont {Tiwari}}]{Nigg_2017}%
  \BibitemOpen
  \bibfield  {author} {\bibinfo {author} {\bibfnamefont {Simon~E.}\
  \bibnamefont {Nigg}}, \bibinfo {author} {\bibfnamefont {Niels}\ \bibnamefont
  {L{\"o}rch}}, \ and\ \bibinfo {author} {\bibfnamefont {Rakesh~P.}\
  \bibnamefont {Tiwari}},\ }\bibfield  {title} {\enquote {\bibinfo {title}
  {Robust quantum optimizer with full connectivity},}\ }\href {\doibase
  10.1126/sciadv.1602273} {\bibfield  {journal} {\bibinfo  {journal} {Science
  Advances}\ }\textbf {\bibinfo {volume} {3}},\ \bibinfo {pages} {e1602273}
  (\bibinfo {year} {2017})}\BibitemShut {NoStop}%
\bibitem [{\citenamefont {Heugel}\ \emph {et~al.}(2019)\citenamefont {Heugel},
  \citenamefont {Oscity}, \citenamefont {Eichler}, \citenamefont {Zilberberg},\
  and\ \citenamefont {Chitra}}]{Heugel_2019_TC}%
  \BibitemOpen
  \bibfield  {author} {\bibinfo {author} {\bibfnamefont {Toni~L.}\ \bibnamefont
  {Heugel}}, \bibinfo {author} {\bibfnamefont {Matthias}\ \bibnamefont
  {Oscity}}, \bibinfo {author} {\bibfnamefont {Alexander}\ \bibnamefont
  {Eichler}}, \bibinfo {author} {\bibfnamefont {Oded}\ \bibnamefont
  {Zilberberg}}, \ and\ \bibinfo {author} {\bibfnamefont {R.}~\bibnamefont
  {Chitra}},\ }\bibfield  {title} {\enquote {\bibinfo {title} {Classical
  many-body time crystals},}\ }\href {\doibase 10.1103/PhysRevLett.123.124301}
  {\bibfield  {journal} {\bibinfo  {journal} {Phys. Rev. Lett.}\ }\textbf
  {\bibinfo {volume} {123}},\ \bibinfo {pages} {124301} (\bibinfo {year}
  {2019})}\BibitemShut {NoStop}%
\bibitem [{\citenamefont {Heugel}\ \emph {et~al.}(2022)\citenamefont {Heugel},
  \citenamefont {Zilberberg}, \citenamefont {Marty}, \citenamefont {Chitra},\
  and\ \citenamefont {Eichler}}]{Heugel_2022}%
  \BibitemOpen
  \bibfield  {author} {\bibinfo {author} {\bibfnamefont {Toni~L.}\ \bibnamefont
  {Heugel}}, \bibinfo {author} {\bibfnamefont {Oded}\ \bibnamefont
  {Zilberberg}}, \bibinfo {author} {\bibfnamefont {Christian}\ \bibnamefont
  {Marty}}, \bibinfo {author} {\bibfnamefont {R.}~\bibnamefont {Chitra}}, \
  and\ \bibinfo {author} {\bibfnamefont {Alexander}\ \bibnamefont {Eichler}},\
  }\bibfield  {title} {\enquote {\bibinfo {title} {Ising machines with strong
  bilinear coupling},}\ }\href {\doibase 10.1103/PhysRevResearch.4.013149}
  {\bibfield  {journal} {\bibinfo  {journal} {Phys. Rev. Research}\ }\textbf
  {\bibinfo {volume} {4}},\ \bibinfo {pages} {013149} (\bibinfo {year}
  {2022})}\BibitemShut {NoStop}%
\bibitem [{\citenamefont {Margiani}\ \emph {et~al.}(2023)\citenamefont
  {Margiani}, \citenamefont {del Pino}, \citenamefont {Heugel}, \citenamefont
  {Bousse}, \citenamefont {Guerrero}, \citenamefont {Kenny}, \citenamefont
  {Zilberberg}, \citenamefont {Sabonis},\ and\ \citenamefont
  {Eichler}}]{Margiani_2023}%
  \BibitemOpen
  \bibfield  {author} {\bibinfo {author} {\bibfnamefont {Gabriel}\ \bibnamefont
  {Margiani}}, \bibinfo {author} {\bibfnamefont {Javier}\ \bibnamefont {del
  Pino}}, \bibinfo {author} {\bibfnamefont {Toni~L.}\ \bibnamefont {Heugel}},
  \bibinfo {author} {\bibfnamefont {Nicholas~E.}\ \bibnamefont {Bousse}},
  \bibinfo {author} {\bibfnamefont {Sebasti\'an}\ \bibnamefont {Guerrero}},
  \bibinfo {author} {\bibfnamefont {Thomas~W.}\ \bibnamefont {Kenny}}, \bibinfo
  {author} {\bibfnamefont {Oded}\ \bibnamefont {Zilberberg}}, \bibinfo {author}
  {\bibfnamefont {Deividas}\ \bibnamefont {Sabonis}}, \ and\ \bibinfo {author}
  {\bibfnamefont {Alexander}\ \bibnamefont {Eichler}},\ }\bibfield  {title}
  {\enquote {\bibinfo {title} {Deterministic and stochastic sampling of two
  coupled kerr parametric oscillators},}\ }\href {\doibase
  10.1103/PhysRevResearch.5.L012029} {\bibfield  {journal} {\bibinfo  {journal}
  {Phys. Rev. Res.}\ }\textbf {\bibinfo {volume} {5}},\ \bibinfo {pages}
  {L012029} (\bibinfo {year} {2023})}\BibitemShut {NoStop}%
\bibitem [{\citenamefont {Puri}\ \emph
  {et~al.}(2017{\natexlab{b}})\citenamefont {Puri}, \citenamefont {Andersen},
  \citenamefont {Grimsmo},\ and\ \citenamefont {Blais}}]{Puri_2017_NC}%
  \BibitemOpen
  \bibfield  {author} {\bibinfo {author} {\bibfnamefont {Shruti}\ \bibnamefont
  {Puri}}, \bibinfo {author} {\bibfnamefont {Christian~Kraglund}\ \bibnamefont
  {Andersen}}, \bibinfo {author} {\bibfnamefont {Arne~L.}\ \bibnamefont
  {Grimsmo}}, \ and\ \bibinfo {author} {\bibfnamefont {Alexandre}\ \bibnamefont
  {Blais}},\ }\bibfield  {title} {\enquote {\bibinfo {title} {Quantum annealing
  with all-to-all connected nonlinear oscillators},}\ }\href {\doibase
  10.1038/ncomms15785} {\bibfield  {journal} {\bibinfo  {journal} {Nature
  Communications}\ }\textbf {\bibinfo {volume} {8}},\ \bibinfo {pages} {15785}
  (\bibinfo {year} {2017}{\natexlab{b}})}\BibitemShut {NoStop}%
\bibitem [{\citenamefont {Goto}\ \emph {et~al.}(2018)\citenamefont {Goto},
  \citenamefont {Lin},\ and\ \citenamefont {Nakamura}}]{Goto_2018}%
  \BibitemOpen
  \bibfield  {author} {\bibinfo {author} {\bibfnamefont {H.}~\bibnamefont
  {Goto}}, \bibinfo {author} {\bibfnamefont {Z.}~\bibnamefont {Lin}}, \ and\
  \bibinfo {author} {\bibfnamefont {Y.}~\bibnamefont {Nakamura}},\ }\bibfield
  {title} {\enquote {\bibinfo {title} {Boltzmann sampling from the ising model
  using quantum heating of coupled nonlinear oscillators},}\ }\href {\doibase
  10.1038/s41598-018-25492-8} {\bibfield  {journal} {\bibinfo  {journal}
  {Scientific Reports}\ }\textbf {\bibinfo {volume} {8}},\ \bibinfo {pages}
  {7154} (\bibinfo {year} {2018})}\BibitemShut {NoStop}%
\bibitem [{\citenamefont {Dykman}\ \emph {et~al.}(2018)\citenamefont {Dykman},
  \citenamefont {Bruder}, \citenamefont {L\"orch},\ and\ \citenamefont
  {Zhang}}]{Dykman_2018}%
  \BibitemOpen
  \bibfield  {author} {\bibinfo {author} {\bibfnamefont {M.~I.}\ \bibnamefont
  {Dykman}}, \bibinfo {author} {\bibfnamefont {Christoph}\ \bibnamefont
  {Bruder}}, \bibinfo {author} {\bibfnamefont {Niels}\ \bibnamefont {L\"orch}},
  \ and\ \bibinfo {author} {\bibfnamefont {Yaxing}\ \bibnamefont {Zhang}},\
  }\bibfield  {title} {\enquote {\bibinfo {title} {Interaction-induced
  time-symmetry breaking in driven quantum oscillators},}\ }\href {\doibase
  10.1103/PhysRevB.98.195444} {\bibfield  {journal} {\bibinfo  {journal} {Phys.
  Rev. B}\ }\textbf {\bibinfo {volume} {98}},\ \bibinfo {pages} {195444}
  (\bibinfo {year} {2018})}\BibitemShut {NoStop}%
\bibitem [{\citenamefont {Rota}\ \emph {et~al.}(2019)\citenamefont {Rota},
  \citenamefont {Minganti}, \citenamefont {Ciuti},\ and\ \citenamefont
  {Savona}}]{Rota_2019}%
  \BibitemOpen
  \bibfield  {author} {\bibinfo {author} {\bibfnamefont {Riccardo}\
  \bibnamefont {Rota}}, \bibinfo {author} {\bibfnamefont {Fabrizio}\
  \bibnamefont {Minganti}}, \bibinfo {author} {\bibfnamefont {Cristiano}\
  \bibnamefont {Ciuti}}, \ and\ \bibinfo {author} {\bibfnamefont {Vincenzo}\
  \bibnamefont {Savona}},\ }\bibfield  {title} {\enquote {\bibinfo {title}
  {Quantum critical regime in a quadratically driven nonlinear photonic
  lattice},}\ }\href {\doibase 10.1103/PhysRevLett.122.110405} {\bibfield
  {journal} {\bibinfo  {journal} {Phys. Rev. Lett.}\ }\textbf {\bibinfo
  {volume} {122}},\ \bibinfo {pages} {110405} (\bibinfo {year}
  {2019})}\BibitemShut {NoStop}%
\bibitem [{\citenamefont {Calvanese~Strinati}\ \emph
  {et~al.}(2019)\citenamefont {Calvanese~Strinati}, \citenamefont {Bello},
  \citenamefont {Pe'er},\ and\ \citenamefont {Dalla~Torre}}]{Strinati2019}%
  \BibitemOpen
  \bibfield  {author} {\bibinfo {author} {\bibfnamefont {Marcello}\
  \bibnamefont {Calvanese~Strinati}}, \bibinfo {author} {\bibfnamefont {Leon}\
  \bibnamefont {Bello}}, \bibinfo {author} {\bibfnamefont {Avi}\ \bibnamefont
  {Pe'er}}, \ and\ \bibinfo {author} {\bibfnamefont {Emanuele~G.}\ \bibnamefont
  {Dalla~Torre}},\ }\bibfield  {title} {\enquote {\bibinfo {title} {Theory of
  coupled parametric oscillators beyond coupled ising spins},}\ }\href
  {\doibase 10.1103/PhysRevA.100.023835} {\bibfield  {journal} {\bibinfo
  {journal} {Phys. Rev. A}\ }\textbf {\bibinfo {volume} {100}},\ \bibinfo
  {pages} {023835} (\bibinfo {year} {2019})}\BibitemShut {NoStop}%
\bibitem [{\citenamefont {Takesue}\ \emph {et~al.}(2020)\citenamefont
  {Takesue}, \citenamefont {Inaba}, \citenamefont {Inagaki}, \citenamefont
  {Ikuta}, \citenamefont {Yamada}, \citenamefont {Honjo}, \citenamefont
  {Kazama}, \citenamefont {Enbutsu}, \citenamefont {Umeki},\ and\ \citenamefont
  {Kasahara}}]{Takesue_2020}%
  \BibitemOpen
  \bibfield  {author} {\bibinfo {author} {\bibfnamefont {Hiroki}\ \bibnamefont
  {Takesue}}, \bibinfo {author} {\bibfnamefont {Kensuke}\ \bibnamefont
  {Inaba}}, \bibinfo {author} {\bibfnamefont {Takahiro}\ \bibnamefont
  {Inagaki}}, \bibinfo {author} {\bibfnamefont {Takuya}\ \bibnamefont {Ikuta}},
  \bibinfo {author} {\bibfnamefont {Yasuhiro}\ \bibnamefont {Yamada}}, \bibinfo
  {author} {\bibfnamefont {Toshimori}\ \bibnamefont {Honjo}}, \bibinfo {author}
  {\bibfnamefont {Takushi}\ \bibnamefont {Kazama}}, \bibinfo {author}
  {\bibfnamefont {Koji}\ \bibnamefont {Enbutsu}}, \bibinfo {author}
  {\bibfnamefont {Takeshi}\ \bibnamefont {Umeki}}, \ and\ \bibinfo {author}
  {\bibfnamefont {Ryoichi}\ \bibnamefont {Kasahara}},\ }\bibfield  {title}
  {\enquote {\bibinfo {title} {Simulating ising spins in external magnetic
  fields with a network of degenerate optical parametric oscillators},}\ }\href
  {\doibase 10.1103/PhysRevApplied.13.054059} {\bibfield  {journal} {\bibinfo
  {journal} {Phys. Rev. Appl.}\ }\textbf {\bibinfo {volume} {13}},\ \bibinfo
  {pages} {054059} (\bibinfo {year} {2020})}\BibitemShut {NoStop}%
\bibitem [{\citenamefont {Inagaki}\ \emph
  {et~al.}(2016{\natexlab{b}})\citenamefont {Inagaki}, \citenamefont {Inaba},
  \citenamefont {Hamerly}, \citenamefont {Inoue}, \citenamefont {Yamamoto},\
  and\ \citenamefont {Takesue}}]{Inagaki_2016}%
  \BibitemOpen
  \bibfield  {author} {\bibinfo {author} {\bibfnamefont {T.}~\bibnamefont
  {Inagaki}}, \bibinfo {author} {\bibfnamefont {K}~\bibnamefont {Inaba}},
  \bibinfo {author} {\bibfnamefont {R.}~\bibnamefont {Hamerly}}, \bibinfo
  {author} {\bibfnamefont {K}~\bibnamefont {Inoue}}, \bibinfo {author}
  {\bibfnamefont {Y}~\bibnamefont {Yamamoto}}, \ and\ \bibinfo {author}
  {\bibfnamefont {H.}~\bibnamefont {Takesue}},\ }\bibfield  {title} {\enquote
  {\bibinfo {title} {Large-scale ising spin network based on degenerate optical
  parametric oscillators},}\ }\href {\doibase 10.1038/nphoton.2016.68}
  {\bibfield  {journal} {\bibinfo  {journal} {Nature Photonics}\ }\textbf
  {\bibinfo {volume} {10}},\ \bibinfo {pages} {415--420} (\bibinfo {year}
  {2016}{\natexlab{b}})}\BibitemShut {NoStop}%
\bibitem [{\citenamefont {Gershenzon}\ \emph {et~al.}(2020)\citenamefont
  {Gershenzon}, \citenamefont {Arwas}, \citenamefont {Gadasi}, \citenamefont
  {Tradonsky}, \citenamefont {Friesem}, \citenamefont {Raz},\ and\
  \citenamefont {Davidson}}]{gershenzon2020exact}%
  \BibitemOpen
  \bibfield  {author} {\bibinfo {author} {\bibfnamefont {Igor}\ \bibnamefont
  {Gershenzon}}, \bibinfo {author} {\bibfnamefont {Geva}\ \bibnamefont
  {Arwas}}, \bibinfo {author} {\bibfnamefont {Sagie}\ \bibnamefont {Gadasi}},
  \bibinfo {author} {\bibfnamefont {Chene}\ \bibnamefont {Tradonsky}}, \bibinfo
  {author} {\bibfnamefont {Asher}\ \bibnamefont {Friesem}}, \bibinfo {author}
  {\bibfnamefont {Oren}\ \bibnamefont {Raz}}, \ and\ \bibinfo {author}
  {\bibfnamefont {Nir}\ \bibnamefont {Davidson}},\ }\bibfield  {title}
  {\enquote {\bibinfo {title} {Exact mapping between a laser network loss rate
  and the classical xy hamiltonian by laser loss control},}\ }\href@noop {}
  {\bibfield  {journal} {\bibinfo  {journal} {Nanophotonics}\ }\textbf
  {\bibinfo {volume} {9}},\ \bibinfo {pages} {4117--4126} (\bibinfo {year}
  {2020})}\BibitemShut {NoStop}%
\bibitem [{Sup()}]{Supplement}%
  \BibitemOpen
  \href@noop {} {\enquote {\bibinfo {title} {See supplemental material at [url
  will be inserted by publisher] for additional setup descriptions, more
  measurement data.}}\ }\BibitemShut {NoStop}%
\bibitem [{\citenamefont {Eichler}\ and\ \citenamefont
  {Zilberberg}(2023)}]{Eichler_Zilberberg_book}%
  \BibitemOpen
  \bibfield  {author} {\bibinfo {author} {\bibfnamefont {Alexander}\
  \bibnamefont {Eichler}}\ and\ \bibinfo {author} {\bibfnamefont {Oded}\
  \bibnamefont {Zilberberg}},\ }\href@noop {} {\emph {\bibinfo {title}
  {Classical and Quantum Parametric Phenomena}}}\ (\bibinfo  {publisher}
  {Oxford University Press},\ \bibinfo {year} {2023})\BibitemShut {NoStop}%
\bibitem [{\citenamefont {Košata}\ \emph {et~al.}(2022)\citenamefont
  {Košata}, \citenamefont {del Pino}, \citenamefont {Heugel},\ and\
  \citenamefont {Zilberberg}}]{kovsata2022harmonicbalance}%
  \BibitemOpen
  \bibfield  {author} {\bibinfo {author} {\bibfnamefont {Jan}\ \bibnamefont
  {Košata}}, \bibinfo {author} {\bibfnamefont {Javier}\ \bibnamefont {del
  Pino}}, \bibinfo {author} {\bibfnamefont {Toni~L.}\ \bibnamefont {Heugel}}, \
  and\ \bibinfo {author} {\bibfnamefont {Oded}\ \bibnamefont {Zilberberg}},\
  }\bibfield  {title} {\enquote {\bibinfo {title} {{HarmonicBalance.jl: A Julia
  suite for nonlinear dynamics using harmonic balance}},}\ }\href {\doibase
  10.21468/SciPostPhysCodeb.6} {\bibfield  {journal} {\bibinfo  {journal}
  {SciPost Phys. Codebases}\ ,\ \bibinfo {pages} {6}} (\bibinfo {year}
  {2022})}\BibitemShut {NoStop}%
\bibitem [{\citenamefont {Rhoads}\ and\ \citenamefont
  {Shaw}(2010)}]{Rhoads_2010}%
  \BibitemOpen
  \bibfield  {author} {\bibinfo {author} {\bibfnamefont {Jeffrey~F.}\
  \bibnamefont {Rhoads}}\ and\ \bibinfo {author} {\bibfnamefont {Steven~W.}\
  \bibnamefont {Shaw}},\ }\bibfield  {title} {\enquote {\bibinfo {title} {The
  impact of nonlinearity on degenerate parametric amplifiers},}\ }\href
  {\doibase 10.1063/1.3446851} {\bibfield  {journal} {\bibinfo  {journal}
  {Applied Physics Letters}\ }\textbf {\bibinfo {volume} {96}},\ \bibinfo
  {pages} {234101} (\bibinfo {year} {2010})},\ \Eprint
  {http://arxiv.org/abs/https://doi.org/10.1063/1.3446851}
  {https://doi.org/10.1063/1.3446851} \BibitemShut {NoStop}%
\bibitem [{\citenamefont {Papariello}\ \emph {et~al.}(2016)\citenamefont
  {Papariello}, \citenamefont {Zilberberg}, \citenamefont {Eichler},\ and\
  \citenamefont {Chitra}}]{Papariello_2016}%
  \BibitemOpen
  \bibfield  {author} {\bibinfo {author} {\bibfnamefont {Luca}\ \bibnamefont
  {Papariello}}, \bibinfo {author} {\bibfnamefont {Oded}\ \bibnamefont
  {Zilberberg}}, \bibinfo {author} {\bibfnamefont {Alexander}\ \bibnamefont
  {Eichler}}, \ and\ \bibinfo {author} {\bibfnamefont {R.}~\bibnamefont
  {Chitra}},\ }\bibfield  {title} {\enquote {\bibinfo {title} {Ultrasensitive
  hysteretic force sensing with parametric nonlinear oscillators},}\ }\href
  {\doibase 10.1103/PhysRevE.94.022201} {\bibfield  {journal} {\bibinfo
  {journal} {Phys. Rev. E}\ }\textbf {\bibinfo {volume} {94}},\ \bibinfo
  {pages} {022201} (\bibinfo {year} {2016})}\BibitemShut {NoStop}%
\bibitem [{\citenamefont {Leuch}\ \emph {et~al.}(2016)\citenamefont {Leuch},
  \citenamefont {Papariello}, \citenamefont {Zilberberg}, \citenamefont
  {Degen}, \citenamefont {Chitra},\ and\ \citenamefont {Eichler}}]{Leuch_2016}%
  \BibitemOpen
  \bibfield  {author} {\bibinfo {author} {\bibfnamefont {Anina}\ \bibnamefont
  {Leuch}}, \bibinfo {author} {\bibfnamefont {Luca}\ \bibnamefont
  {Papariello}}, \bibinfo {author} {\bibfnamefont {Oded}\ \bibnamefont
  {Zilberberg}}, \bibinfo {author} {\bibfnamefont {Christian~L.}\ \bibnamefont
  {Degen}}, \bibinfo {author} {\bibfnamefont {R.}~\bibnamefont {Chitra}}, \
  and\ \bibinfo {author} {\bibfnamefont {Alexander}\ \bibnamefont {Eichler}},\
  }\bibfield  {title} {\enquote {\bibinfo {title} {Parametric symmetry breaking
  in a nonlinear resonator},}\ }\href {\doibase 10.1103/PhysRevLett.117.214101}
  {\bibfield  {journal} {\bibinfo  {journal} {Phys. Rev. Lett.}\ }\textbf
  {\bibinfo {volume} {117}},\ \bibinfo {pages} {214101} (\bibinfo {year}
  {2016})}\BibitemShut {NoStop}%
\bibitem [{\citenamefont {Falk}(1966)}]{Falk_1966}%
  \BibitemOpen
  \bibfield  {author} {\bibinfo {author} {\bibfnamefont {H.}~\bibnamefont
  {Falk}},\ }\bibfield  {title} {\enquote {\bibinfo {title} {Ising chain with a
  spin impurity},}\ }\href {\doibase 10.1103/PhysRev.151.304} {\bibfield
  {journal} {\bibinfo  {journal} {Phys. Rev.}\ }\textbf {\bibinfo {volume}
  {151}},\ \bibinfo {pages} {304--311} (\bibinfo {year} {1966})}\BibitemShut
  {NoStop}%
\bibitem [{\citenamefont {Grinstein}(1976)}]{Grinstein_1976}%
  \BibitemOpen
  \bibfield  {author} {\bibinfo {author} {\bibfnamefont {G.}~\bibnamefont
  {Grinstein}},\ }\bibfield  {title} {\enquote {\bibinfo {title} {Ferromagnetic
  phase transitions in random fields: The breakdown of scaling laws},}\ }\href
  {\doibase 10.1103/PhysRevLett.37.944} {\bibfield  {journal} {\bibinfo
  {journal} {Phys. Rev. Lett.}\ }\textbf {\bibinfo {volume} {37}},\ \bibinfo
  {pages} {944--947} (\bibinfo {year} {1976})}\BibitemShut {NoStop}%
\bibitem [{\citenamefont {Belanger}\ and\ \citenamefont
  {Young}(1991)}]{belanger1991random}%
  \BibitemOpen
  \bibfield  {author} {\bibinfo {author} {\bibfnamefont {DP}~\bibnamefont
  {Belanger}}\ and\ \bibinfo {author} {\bibfnamefont {AP}~\bibnamefont
  {Young}},\ }\bibfield  {title} {\enquote {\bibinfo {title} {The random field
  ising model},}\ }\href@noop {} {\bibfield  {journal} {\bibinfo  {journal}
  {Journal of magnetism and magnetic materials}\ }\textbf {\bibinfo {volume}
  {100}},\ \bibinfo {pages} {272--291} (\bibinfo {year} {1991})}\BibitemShut
  {NoStop}%
\bibitem [{\citenamefont {Bingham}\ \emph {et~al.}(2021)\citenamefont
  {Bingham}, \citenamefont {Rooke}, \citenamefont {Park}, \citenamefont
  {Simon}, \citenamefont {Zhu}, \citenamefont {Zhang}, \citenamefont {Batley},
  \citenamefont {Watts}, \citenamefont {Leighton}, \citenamefont {Dahmen},\
  and\ \citenamefont {Schiffer}}]{Bingham_2021}%
  \BibitemOpen
  \bibfield  {author} {\bibinfo {author} {\bibfnamefont {N.~S.}\ \bibnamefont
  {Bingham}}, \bibinfo {author} {\bibfnamefont {S.}~\bibnamefont {Rooke}},
  \bibinfo {author} {\bibfnamefont {J.}~\bibnamefont {Park}}, \bibinfo {author}
  {\bibfnamefont {A.}~\bibnamefont {Simon}}, \bibinfo {author} {\bibfnamefont
  {W.}~\bibnamefont {Zhu}}, \bibinfo {author} {\bibfnamefont {X.}~\bibnamefont
  {Zhang}}, \bibinfo {author} {\bibfnamefont {J.}~\bibnamefont {Batley}},
  \bibinfo {author} {\bibfnamefont {J.~D.}\ \bibnamefont {Watts}}, \bibinfo
  {author} {\bibfnamefont {C.}~\bibnamefont {Leighton}}, \bibinfo {author}
  {\bibfnamefont {K.~A.}\ \bibnamefont {Dahmen}}, \ and\ \bibinfo {author}
  {\bibfnamefont {P.}~\bibnamefont {Schiffer}},\ }\bibfield  {title} {\enquote
  {\bibinfo {title} {Experimental realization of the 1d random field ising
  model},}\ }\href {\doibase 10.1103/PhysRevLett.127.207203} {\bibfield
  {journal} {\bibinfo  {journal} {Phys. Rev. Lett.}\ }\textbf {\bibinfo
  {volume} {127}},\ \bibinfo {pages} {207203} (\bibinfo {year}
  {2021})}\BibitemShut {NoStop}%
\bibitem [{\citenamefont {Yao}\ and\ \citenamefont
  {Jack}(2023)}]{yao2023thermal}%
  \BibitemOpen
  \bibfield  {author} {\bibinfo {author} {\bibfnamefont {Liheng}\ \bibnamefont
  {Yao}}\ and\ \bibinfo {author} {\bibfnamefont {Robert~L}\ \bibnamefont
  {Jack}},\ }\bibfield  {title} {\enquote {\bibinfo {title} {Thermal vestiges
  of avalanches in the driven random field ising model},}\ }\href@noop {}
  {\bibfield  {journal} {\bibinfo  {journal} {Journal of Statistical Mechanics:
  Theory and Experiment}\ }\textbf {\bibinfo {volume} {2023}},\ \bibinfo
  {pages} {023303} (\bibinfo {year} {2023})}\BibitemShut {NoStop}%
\bibitem [{\citenamefont {Perkovi\ifmmode~\acute{c}\else \'{c}\fi{}}\ \emph
  {et~al.}(1995)\citenamefont {Perkovi\ifmmode~\acute{c}\else \'{c}\fi{}},
  \citenamefont {Dahmen},\ and\ \citenamefont {Sethna}}]{Percovic_1995}%
  \BibitemOpen
  \bibfield  {author} {\bibinfo {author} {\bibfnamefont {Olga}\ \bibnamefont
  {Perkovi\ifmmode~\acute{c}\else \'{c}\fi{}}}, \bibinfo {author}
  {\bibfnamefont {Karin}\ \bibnamefont {Dahmen}}, \ and\ \bibinfo {author}
  {\bibfnamefont {James~P.}\ \bibnamefont {Sethna}},\ }\bibfield  {title}
  {\enquote {\bibinfo {title} {Avalanches, barkhausen noise, and plain old
  criticality},}\ }\href {\doibase 10.1103/PhysRevLett.75.4528} {\bibfield
  {journal} {\bibinfo  {journal} {Phys. Rev. Lett.}\ }\textbf {\bibinfo
  {volume} {75}},\ \bibinfo {pages} {4528--4531} (\bibinfo {year}
  {1995})}\BibitemShut {NoStop}%
\bibitem [{\citenamefont {Im}\ \emph {et~al.}(2009)\citenamefont {Im},
  \citenamefont {Fischer}, \citenamefont {Kim},\ and\ \citenamefont
  {Shin}}]{im2009direct}%
  \BibitemOpen
  \bibfield  {author} {\bibinfo {author} {\bibfnamefont {Mi-Young}\
  \bibnamefont {Im}}, \bibinfo {author} {\bibfnamefont {Peter}\ \bibnamefont
  {Fischer}}, \bibinfo {author} {\bibfnamefont {Dong-Hyun}\ \bibnamefont
  {Kim}}, \ and\ \bibinfo {author} {\bibfnamefont {Sung-Chul}\ \bibnamefont
  {Shin}},\ }\bibfield  {title} {\enquote {\bibinfo {title} {Direct observation
  of individual barkhausen avalanches in nucleation-mediated magnetization
  reversal processes},}\ }\href@noop {} {\bibfield  {journal} {\bibinfo
  {journal} {Applied Physics Letters}\ }\textbf {\bibinfo {volume} {95}}
  (\bibinfo {year} {2009})}\BibitemShut {NoStop}%
\bibitem [{\citenamefont {Field}\ \emph {et~al.}(1995)\citenamefont {Field},
  \citenamefont {Witt}, \citenamefont {Nori},\ and\ \citenamefont
  {Ling}}]{Field_1995}%
  \BibitemOpen
  \bibfield  {author} {\bibinfo {author} {\bibfnamefont {Stuart}\ \bibnamefont
  {Field}}, \bibinfo {author} {\bibfnamefont {Jeff}\ \bibnamefont {Witt}},
  \bibinfo {author} {\bibfnamefont {Franco}\ \bibnamefont {Nori}}, \ and\
  \bibinfo {author} {\bibfnamefont {Xinsheng}\ \bibnamefont {Ling}},\
  }\bibfield  {title} {\enquote {\bibinfo {title} {Superconducting vortex
  avalanches},}\ }\href {\doibase 10.1103/PhysRevLett.74.1206} {\bibfield
  {journal} {\bibinfo  {journal} {Phys. Rev. Lett.}\ }\textbf {\bibinfo
  {volume} {74}},\ \bibinfo {pages} {1206--1209} (\bibinfo {year}
  {1995})}\BibitemShut {NoStop}%
\bibitem [{\citenamefont {Lahini}\ \emph {et~al.}(2017)\citenamefont {Lahini},
  \citenamefont {Gottesman}, \citenamefont {Amir},\ and\ \citenamefont
  {Rubinstein}}]{Lahini_2017}%
  \BibitemOpen
  \bibfield  {author} {\bibinfo {author} {\bibfnamefont {Yoav}\ \bibnamefont
  {Lahini}}, \bibinfo {author} {\bibfnamefont {Omer}\ \bibnamefont
  {Gottesman}}, \bibinfo {author} {\bibfnamefont {Ariel}\ \bibnamefont {Amir}},
  \ and\ \bibinfo {author} {\bibfnamefont {Shmuel~M.}\ \bibnamefont
  {Rubinstein}},\ }\bibfield  {title} {\enquote {\bibinfo {title} {Nonmonotonic
  aging and memory retention in disordered mechanical systems},}\ }\href
  {\doibase 10.1103/PhysRevLett.118.085501} {\bibfield  {journal} {\bibinfo
  {journal} {Phys. Rev. Lett.}\ }\textbf {\bibinfo {volume} {118}},\ \bibinfo
  {pages} {085501} (\bibinfo {year} {2017})}\BibitemShut {NoStop}%
\bibitem [{\citenamefont {Mistakidis}\ \emph {et~al.}(2022)\citenamefont
  {Mistakidis}, \citenamefont {Koutentakis}, \citenamefont {Grusdt},
  \citenamefont {Schmelcher},\ and\ \citenamefont
  {Sadeghpour}}]{mistakidis2022inducing}%
  \BibitemOpen
  \bibfield  {author} {\bibinfo {author} {\bibfnamefont {SI}~\bibnamefont
  {Mistakidis}}, \bibinfo {author} {\bibfnamefont {GM}~\bibnamefont
  {Koutentakis}}, \bibinfo {author} {\bibfnamefont {F}~\bibnamefont {Grusdt}},
  \bibinfo {author} {\bibfnamefont {P}~\bibnamefont {Schmelcher}}, \ and\
  \bibinfo {author} {\bibfnamefont {HR}~\bibnamefont {Sadeghpour}},\ }\bibfield
   {title} {\enquote {\bibinfo {title} {Inducing spin-order with an impurity:
  phase diagram of the magnetic bose polaron},}\ }\href@noop {} {\bibfield
  {journal} {\bibinfo  {journal} {New Journal of Physics}\ }\textbf {\bibinfo
  {volume} {24}},\ \bibinfo {pages} {083030} (\bibinfo {year}
  {2022})}\BibitemShut {NoStop}%
\bibitem [{\citenamefont {Breiding}\ and\ \citenamefont
  {Timme}(2018)}]{HomotopyContinuation.jl}%
  \BibitemOpen
  \bibfield  {author} {\bibinfo {author} {\bibfnamefont {Paul}\ \bibnamefont
  {Breiding}}\ and\ \bibinfo {author} {\bibfnamefont {Sascha}\ \bibnamefont
  {Timme}},\ }\bibfield  {title} {\enquote {\bibinfo {title}
  {{H}omotopy{C}ontinuation.jl: {A} {P}ackage for {H}omotopy {C}ontinuation in
  {J}ulia},}\ }in\ \href@noop {} {\emph {\bibinfo {booktitle} {International
  Congress on Mathematical Software}}}\ (\bibinfo {organization} {Springer},\
  \bibinfo {year} {2018})\ pp.\ \bibinfo {pages} {458--465}\BibitemShut
  {NoStop}%
\end{thebibliography}%


\end{document}
%
% ****** End of file apssamp.tex ******

