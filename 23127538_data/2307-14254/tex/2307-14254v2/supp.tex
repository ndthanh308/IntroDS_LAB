% ****** Start of file apssamp.tex ******
\documentclass[prb,onecolumn,superscriptaddress]{revtex4-2}
\usepackage{amsmath,amssymb,bm}
\usepackage{graphicx}
\usepackage{epstopdf}
\usepackage{latexsym}
\usepackage{subfigure}
\usepackage[usenames, dvipsnames]{color}
\usepackage[usenames, dvipsnames]{xcolor}
\usepackage{natbib}
\usepackage{braket}
\usepackage{float}
\usepackage[normalem]{ulem}
\usepackage{comment}
\usepackage{mathtools}
\usepackage{array}
\usepackage{tabu}
\usepackage{multirow}
\usepackage{chemformula}
\usepackage{svg}
\usepackage{tabularx,ragged2e}
%\usepackage{dblfloatfix} %%for putting table at bottom of the page
%\usepackage{stfloats} %%for putting table at bottom of the page
\usepackage{dcolumn}% Align table columns on decimal point
%\sisetup{table-format=1.3}

\usepackage{hyperref}
\hypersetup{
                    colorlinks=true,                
                    breaklinks=true,                
                    urlcolor= red,                
                    linkcolor= red,                
                    bookmarksopen=false,
                    filecolor=red,
                    citecolor=blue,
                    linkbordercolor=green
}


%\preprint{APS/123-QED}

\newcommand{\SP}[1]{{\color{red}{\bf SP: #1}}}
\newcommand{\SPhide}[1]{{}}

\begin{document}

\title{Supplementary information for ``Deconfined pseudocriticality in a 
model spin-1 quantum antiferromagnet"}

\author{Vikas Vijigiri}
\altaffiliation[vikas.v@iitb.ac.in]{}\affiliation{Department of Physics, Indian Institute of Technology Bombay, Powai, Mumbai, 400076, India}
\author{Nisheeta Desai}
\affiliation{Department of Theoretical Physics, Tata Institute of Fundamental Research,  Mumbai, 400005, India}
\author{Sumiran Pujari}
\altaffiliation[sumiran@phy.iitb.ac.in]{}\affiliation{Department of Physics, Indian Institute of Technology Bombay, Powai, Mumbai, 400076, India}

\date{\today}


\begin{abstract}
Here we present additional data and plots to  
document the data sets collected during this work
for the various Heisenberg exchange coupling ($J_H$) values 
studied in the main text.
\end{abstract}



\maketitle
\tableofcontents



\section{Model and Observables}
\label{sec:model}
We recapitulate again the various definitions
related to the model and observables introduced in the main text 
for convenience.
The $S=1$ Hamiltonian (Eq.~1 and Eq.~2 of the main text) is given by,
\begin{eqnarray}
    H &=& J_H\sum_{\langle ij \rangle} 
    \{ \mathbf{S}_i\cdot \mathbf{S}_j - \mathbb{I} \}
    - 
    \frac{J_B}{3} \sum_{\langle ij \rangle} 
    \{(\mathbf{S}_i\cdot \mathbf{S}_j)^2  - \mathbb{I} \}
    \nonumber \\
    & & - \; \frac{Q_B}{9} \sum_{\langle ijkl \rangle} 
    \left[ \{(\mathbf{S}_i\cdot \mathbf{S}_j)^2 - \mathbb{I}\}
    \{(\mathbf{S}_k\cdot \mathbf{S}_l)^2 - \mathbb{I}\}
    +
    \{(\mathbf{S}_i\cdot \mathbf{S}_l)^2 - \mathbb{I}\}
    \{(\mathbf{S}_j\cdot \mathbf{S}_k)^2 - \mathbb{I}\}
    \right]
\end{eqnarray}
where $J_H, J_B$ are the coefficients of the Heisenberg and Biquadratic 
exchange terms, and $Q_B$ is the coefficient of the designer $Q$-term composed
of biquadratic exchanges respectively. 

%%%Observables
%\subsection{Observables}
  The following structure factors  
  \begin{eqnarray}
  \mathcal{A}(\mathbf{q}) &=&\frac{1}{N^2} \sum\limits_{\mathbf{r},\mathbf{r}^\prime} e^{i(\mathbf{r}-\mathbf{r}^\prime)\cdot \mathbf{q}}\langle S^z_{\mathbf{r}} S^z_{\mathbf{r}^\prime} \rangle \\   
  \mathcal{B}(\mathbf{q}) &=& \frac{1}{N^2} \sum\limits_{\mathbf{r},\mathbf{r}^\prime} e^{i(\mathbf{r}-\mathbf{r}^\prime)\cdot \mathbf{q}}
  \langle \left( \mathbf{S}_{\mathbf{r}}\cdot \mathbf{S}_{\mathbf{r}^\prime} -\mathbb{I} \right) \rangle  \\  
  \mathcal{C}^{\mu}(\mathbf{q}) &=& \frac{1}{N^2} \sum\limits_{\mathbf{r},\mathbf{r}^\prime} e^{i(\mathbf{r}-\mathbf{r}^\prime)\cdot \mathbf{q}}\langle 
  (\mathbf{S}_{\mathbf{r}}\cdot \mathbf{S}_{\mathbf{r}+\mathbf{e}_\mu}-\mathbb{I}) 
  \times
 (\mathbf{S}_{\mathbf{r}^\prime} \cdot \mathbf{S}_{\mathbf{r}^\prime+\mathbf{e}_\mu}-\mathbb{I}) \rangle  \\
  \mathcal{D}^\mu(\mathbf{q}) &=& \frac{1}{9N^2} \sum\limits_{\mathbf{r},\mathbf{r}^\prime} e^{i(\mathbf{r}-\mathbf{r}^\prime)\cdot \mathbf{q}}\langle ((\mathbf{S}_{\mathbf{r}}\cdot \mathbf{S}_{\mathbf{r}+\mathbf{e}_\mu})^2-\mathbb{I}) \times
  ((\mathbf{S}_{\mathbf{r}^\prime}\cdot \mathbf{S}_{\mathbf{r}^\prime+\mathbf{e}_\mu})^2-\mathbb{I})) \rangle 
  \end{eqnarray}
  made out of real-space correlators can serve as 
  order parameters for N\'eel and VBS ordering.
As a side remark, $\mathcal{B}(\mathbf{q})$ can be measured during loop update
in SSE. However, we do not need to do this as $\mathcal{A}(\mathbf{q})$ 
which also detects N\'eel ordering 
can measured in  a much simpler way. Furthermore, in presence of 
$SU(2)$ symmetry they are linearly related to each other.
Therefore, we focused on $\mathcal{A}(\mathbf{q})$ to probe N\'eel order
at the antiferromagnetic ordering wavevector on the square lattice
($(\pi,\pi)$ when lattice constants are set to unity)
as mentioned in the main text as well.
The corresponding N\'eel correlation ratio $R_N$ is defined to be
%%Neel ratio def
\begin{eqnarray}
R_N \equiv \frac{R_x + R_y}{2}
\end{eqnarray}
where,
\begin{eqnarray}
R_x & \equiv& 1 - \frac{ \mathcal{A}(\pi + \frac{2\pi}{L},\pi)}
{\mathcal{A}(\pi,\pi)}  \\
R_y &\equiv& 1 - \frac{\mathcal{A}(\pi, \pi + \frac{2\pi}{L})}
{\mathcal{A}(\pi,\pi)}
\end{eqnarray}

Similarly, both $\mathcal{C}(\mathbf{q})$ and $\mathcal{D}(\mathbf{q})$
made out of bond-bond correlators of Heisenberg and biquadratic couplings
can be used to probe VBS order at its ordering wavevector on the square lattice
($(\pi,0)$ and $(0,\pi)$ when lattice constants are set to unity).
Either one would thus suffice for the study of N\'eel-VBS transitions.
The Heisenberg bond energy based VBS correlation ratio $R_{V'}$ is 
defined to be
\begin{eqnarray}
R_{V'} \equiv \frac{R_{V'_x} + R_{V'_y}}{2}
\end{eqnarray}
where, 
%%VBS ratio def
\begin{eqnarray}
R_{V'_x} & \equiv & 1 - \frac{\mathcal{C}(\pi + \frac{2\pi}{L},0)}
{\mathcal{C}(\pi,0)} \\
R_{V'_y} & \equiv & 1 - \frac{\mathcal{C}(0, \pi + \frac{2\pi}{L})}
{\mathcal{C}(0,\pi)}
\end{eqnarray}
We have rather focused on the biquadratic bond energy based VBS 
observables in the paper due to better statistics while estimating
it during QMC compared to the Heisenberg bond energy based observables
as discussed also in a footnote in the main text.
The corresponding VBS correlation ratio is thus defined to
\begin{eqnarray}
R_V \equiv \frac{R_{V_x} + R_{V_y}}{2}
\end{eqnarray}
where, 
\begin{eqnarray}
R_{V_x} &=& 1 - \frac{\mathcal{D}(\pi + \frac{2\pi}{L},0)}{D(\pi,0)} \\
R_{V_y} &=& 1 - \frac{\mathcal{D}(0, \pi + \frac{2\pi}{L})}{\mathcal{D}(0,\pi)}
\end{eqnarray}
Finally, from the above we can also see the correspondence between the notation
of the main text and that used here for the order parameters as
\begin{eqnarray}
 O_N & = & \mathcal{A}(\pi,\pi)   \\
 O_V & = & \mathcal{D}^x(\pi,0) + \mathcal{D}^y(0,\pi)
\end{eqnarray}

\clearpage
\pagebreak[4]

\section{Data sets}
\label{sec:datasets}

\subsection{Convergence with inverse temperature $\beta$}
\label{subsec:beta_convergence}


% Figure environment removed

% Figure environment removed

% Figure environment removed




\clearpage
\pagebreak[4]

\subsection{Correlation Ratios}
\label{subsec:corr_ratio}

% Correlation ratio plot page
% Figure environment removed
% Figure environment removed




\clearpage
\pagebreak[4]
\subsection{Staggered magnetization histograms}
% Figure environment removed

\clearpage
\pagebreak[4]
\subsection{cVBS order parameter histograms}
% Figure environment removed

\clearpage
\pagebreak[4]

\subsection{Staggered magnetization time series}
% Figure environment removed

\clearpage
\pagebreak[4]

% Figure environment removed

\clearpage
\pagebreak[4]

\subsection{cVBS order parameter time series}
% Figure environment removed

\newpage
% Figure environment removed

\newpage
% Figure environment removed


\clearpage
\pagebreak[4]
\subsection{Scaling collapse of correlation ratios and order parameters}
\label{subsec:collapse}

%%%%%%%%%%%%%% Collapse plots %%%%%%%%%
% Figure environment removed

% Figure environment removed


\clearpage
\pagebreak[4]
\subsection{Binder Ratios}
\label{subsec:corr_ratio}

%%%%%%%%%%%%%% Binder ratio plots %%%%%%%%%
% Figure environment removed


\clearpage
\pagebreak[4]
\section{Tables}
\label{sec:exponent_tables}

\subsection{Critcal exponents and critical point estimates}

%% Exponent tables
\begin{table*}[h]
\caption{Critical exponents ($\nu_N,\nu_V$) and critical point estimates
($g_{cN}$, $g_{cV}$) as obtained after performing a
scaling collapse on the N\'eel and cVBS correlation ratios ($R_{N}$ and $R_B$). 
System sizes used for collapses are in
the range of $L=24,32,40,48,64$. $\beta=\frac{L}{4}$ throughout.
Neither $(\nu_N,\nu_V)$ nor $(g_{cN},g_{cV})$ were set equal.
}
\begin{tabularx}{\textwidth}{*{9}{>{\centering\arraybackslash}X}}
\toprule
$J_H$ &  $\nu_{N}$ & $\nu_{V}$  & $g_{cN}$ & $g_{cV}$ & $\chi^2_N$ & $\chi^2_V$ \\
\toprule
0.0 &  0.49(5) & 0.63(1)  & 0.168(1) & 0.167(1) & 0.9-1.56 & 1.14-1.81\\
0.01 &   0.41(2) & 0.56(4)  & 0.175(1) & 0.171(1) & 1.31-2.27 & 1.32-1.84 \\
0.025 &  0.40(3) & 0.51(3)  & 0.182(1) & 0.18(1) & 0.97-2.43 & 1.15-1.61 \\
0.04 & 0.37(4) & 0.49(1)  & 0.191(1) & 0.188(1) & 1.12-2.64  & 1.14-1.86\\
0.05 &   0.40(3) & 0.46(3)  & 0.196(1) & 0.195(1) & 0.91-1.78 & 1.15-1.84 \\
0.07 &   0.34(5) & 0.48(5)  & 0.207(1) & 0.206(1)  & 1.52-2.54 & 1.04-1.77\\
0.1 &   0.31(1) & 0.40(1)  & 0.225(1) & 0.223(1) & 1.55-2.27 & 1.03-2.26\\
0.15 &   0.32(4) & 0.44(2)  & 0.254(1) & 0.252(1) & 1.37-3.39 & 0.74-1.91\\

\toprule
\end{tabularx}
\end{table*}
\begin{table*}[h]
\caption{Critical exponents ($\eta_V,\eta_N$, $\nu_N,\nu_V$) 
and critical point estimates ($g_{cN}$, $g_{cV}$)
as obtained after performing a scaling collapse of the 
N\'eel and cVBS order parameters ($O_{N},O_{B}$). 
System sizes used for collapses are in
the range of $L=24,32,40,48,64$.
$\beta=\frac{L}{4}$ throughout.
$(\nu_N,\nu_V)$ were not set equal to each other, while
$g_{cN}$ and $g_{cV}$ values were fixed while performing the
scaling collapse to the values obtained from
the scaling collapse of correlation ratios as shown the preceding table.}
\begin{tabularx}{\textwidth}{*{9}{>{\centering\arraybackslash}X}}
\toprule
$J_H$ &   $\nu_{N}$ & $\nu_{V}$  & $\eta_{N}$ & $\eta_{V}$ & $g_{cN}$ & $g_{cV}$ & $\chi^2_N$ & $\chi^2_V$ \\
\toprule
0.0 &   0.53(3) & 0.63(1) & 0.41(1) & 0.50(2) & 0.168 & 0.167 & 1.01-1.71 & 1.67-2.53\\
0.01 &  0.45(3) & 0.54(3) & 0.26(2) & 0.46(2) & 0.174 & 0.171 & 1.45-1.66 & 1.74-2.29\\
0.025 &   0.43(3) & 0.46(4) & 0.21(3) & 0.33(1)  & 0.182 & 0.180 & 1.04-1.55 & 0.86-1.37\\
0.04 &   0.40(2) & 0.43(5) & 0.24(2) & 0.23(2)  & 0.191 & 0.189 & 1.99-2.28 & 1.49-2.77\\
0.05 &   0.39(4) & 0.38(5) & 0.16(3) & 0.17(3)  & 0.196 & 0.195 & 1.28-1.54 & 0.98-1.9\\
0.07 &   0.38(2) & 0.36(2) & 0.05(1) & 0.12(3)  & 0.207 & 0.206 & 1.67-2.35 & 2.2-2.68\\
0.1 &   0.33(2) & 0.41(2) & 0.12(2) & 0.28(7)  & 0.225 & 0.223 & 2.48-2.81 & 2.55-2.86\\
0.15 &   ?? & ?? & ?? & ??  & 0.254 & 0.252 & ?? & ??\\

\toprule
\end{tabularx}
\end{table*}
\begin{table*}[h]
\caption{Critical exponents ($\eta_V,\eta_N$, $\nu_N,\nu_V$) 
and critical point estimates ($g_{cN}$, $g_{cV}$)
as obtained after performing a scaling collapse of 
N\'eel and cVBS order parameters ($O_{N},O_{B}$) 
for eight sets of the Heisenberg strength, $J_H=0.,0.01,0.025,0.04,0.05,0.07,0.1,0.15$. 
System sizes used for collapses are in
the range of $L=24,32,40,48,64$. $\beta=\frac{L}{4}$ throughout.
Neither $(\nu_N,\nu_V)$ nor $(g_{cN},g_{cV})$ were set equal.
This table was presented in the main text as well, and the
estimates of various fitting parameters below are corroborated well
by the estimates from the preceding tables.}
\begin{tabularx}{\textwidth}{*{9}{>{\centering\arraybackslash}X}}
\toprule
$J_H$ &  $\nu_{N}$ & $\nu_{V}$  & $\eta_{N}$ & $\eta_{V}$ & $g_{cN}$ & $g_{cV}$ & $\chi^2_N$ & $\chi^2_V$ \\
\toprule
0.0 &   0.53(3) & 0.63(1) & 0.44(5) & 0.49(2) & 0.168(1) & 0.167(1) & 1.08-1.68 & 1.69-2.46\\
0.01 &   0.45(2) & 0.54(3) & 0.23(3) & 0.42(4) & 0.174(1) & 0.171(1) & 1.19-1.63 & 1.38-1.73 \\
0.025 &   0.43(3) & 0.46(4) & 0.15(9) & 0.38(2)  & 0.182(1) & 0.180(1) & 0.75-1.46 & 0.8-1.4\\
0.04 &   0.40(2) & 0.43(5) & 0.13(7) & 0.30(8)  & 0.19(1) & 0.189(1) & 1.06-1.67 & 1.09-1.5\\
0.05 &   0.39(4) & 0.38(5) & 0.20(9) & 0.29(6)  & 0.196(1) & 0.195(1) & 0.87-1.31 & 0.87-1.96\\
0.07 &   0.38(2) & 0.39(3) & 0.10(4) & 0.10(4)  & 0.207(1) & 0.206(1) & 1.52-2.54 & 1.04-1.77\\
0.1 &   0.35(4) & 0.35(3) & -0.03(5) & -0.03(2)  & 0.224(1) & 0.224(1) & 1.24-3.28 & 0.99-1.97\\
0.15 &   0.33(2) & 0.33(1) & 0.00(8) & -0.12(8)  & 0.253(1) & 0.253(1) & 1.42-1.79 & 1.15-1.63\\

\toprule
\end{tabularx}
\end{table*}



\clearpage
\pagebreak[4]
\subsection{Benchmarking with Exact Diagonalization}
%\label{subsec:benchmarking}
%% Benchmarking tables
\begin{table*}[h]
\caption{This benchmarking table shows the values of total energy ($E$), 
N\'eel observables ($O_N$, $R_N$) obtained 
by Exact diagonalization (ED) and 
SSE-QMC on a $2\times2$ square plaquette at 
$\beta=10$.} 
% and for $10^7$ equilibration and measurements. For SSE, errors are also estimated and are shown.
\begin{tabularx}{\textwidth}{*{7}{>{\centering\arraybackslash}X}}
\toprule
$\left(J_{B},Q_B,J_{H}\right)$ 
& $E^{\text{ED}}$
& $E^{\text{SSE}}$ 
& $O^{\text{ED}}_N$ 
& $O^{\text{SSE}}_N$ 
& $R^{\text{ED}}_N$ 
& $R^{\text{SSE}}_N$  \\
\toprule
(1.0,1.0,0.2)   & -5.4524 & -5.452(5) & 1.6980  & 1.6980(2)  & 0.35739 & 0.3573(9)\\
(0.9,0.4,0.3)   & -3.6835 & -3.683(2) & 1.7386 & 1.738(5) & 0.36655 & 0.3665(4)\\
(0.6,0.5,0.5)   & -3.8492 & -3.849(0) & 1.7849 & 1.785(2) & 0.37651 & 0.3765(3)\\
(0.3,0.8,0.2)   & -3.4539 & -3.4539(6) & 1.7170 & 1.717(1) & 0.36172 & 0.3617(1)\\
(1.15,0.88,0.12)  & -5.2108 & -5.210(7) & 1.6861 & 1.686(2) & 0.35463 & 0.3546(0)\\
\toprule
\end{tabularx}
\end{table*}

\begin{table*}[h]
\caption{This benchmarking table shows the values of 
VBS observables 
($O_V$,  $R_V$, $O_B$, $R_B$) obtained by Exact diagonalization (ED) and SSE-QMC on a $2\times2$ square plaquette at $\beta=10$.}
%and for $10^7$ equilibration and measurements. For SSE, errors are also estimated and are shown.
\begin{tabularx}{\textwidth}{*{9}{>{\centering\arraybackslash}X}}
\toprule
$\left(J_{B},Q_B,J_{H}\right)$ 
& $O^{\text{ED}}_V$ 
& $O^{\text{SSE}}_V$ 
& $R^{\text{ED}}_V$
& $R^{\text{SSE}}_V$ 
& $O^{\text{ED}}_B$ 
& $O^{\text{SSE}}_B$ 
& $R^{\text{ED}}_B$ 
& $R^{\text{ED}}_B$ \\
\toprule
(1.0,1.0,0.2)  & 1.63654 & 1.6365(4) & 0.5 & 0.4999(5) & 1.49930 & 1.4993(3) & 0.5 & 0.5000(1)\\
(0.9,0.4,0.3)  & 1.65070 & 1.6507(4) & 0.5 & 0.4999(8) & 1.49615 & 1.496(2) & 0.5 & 0.49999(8)\\
 (0.6,0.5,0.5)  & 1.66564 & 1.665(5) & 0.5 & 0.49999(7) & 1.48881 & 1.488(8) & 0.5 & 0.4999(9)\\
(0.3,0.8,0.2)  & 1.64326 & 1.643(4) & 0.5 & 0.4999(9) & 1.49817 & 1.498(0) & 0.5 & 0.5000(3)\\
\scriptsize{(1.15,0.88,0.12)} & 1.63223 & 1.632(1) & 0.5 & 0.4999(5) & 1.49973 & 1.499(8) & 0.5 & 0.49999(3)\\
\toprule
\end{tabularx}
\end{table*}


% \end{acknowledgments}


\pagebreak[4]



%\bibliography{main}
\end{document}
