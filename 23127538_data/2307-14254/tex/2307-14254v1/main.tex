\documentclass[prl,twocolumn,superscriptaddress]{revtex4-2}
\usepackage{amsmath,amssymb,bm}
\usepackage{hyperref}
\usepackage{graphicx}
\usepackage{epstopdf}
\usepackage{latexsym}
\usepackage{subfigure}
\usepackage[usenames, dvipsnames]{color}
\usepackage[usenames, dvipsnames]{xcolor}
\usepackage{natbib}
\usepackage{braket}
\usepackage{float}
\usepackage[normalem]{ulem}
\usepackage{comment}
\usepackage{mathtools}
\usepackage{array}
\usepackage{tabu}
\usepackage{multirow}
\usepackage{chemformula}
\usepackage{svg}
\usepackage{tabularx,ragged2e}
%\usepackage{dblfloatfix} %%for putting table at bottom of the page
%\usepackage{stfloats} %%for putting table at bottom of the page
\usepackage{dcolumn}% Align table columns on decimal point
%\sisetup{table-format=1.3}

%for strike-outs
\newcommand\redsout{\bgroup\markoverwith{\textcolor{red}{\rule[0.5ex]{2pt}{0.4pt}}}\ULon}
\newcommand\bluesout{\bgroup\markoverwith{\textcolor{blue}{\rule[0.5ex]{2pt}{0.4pt}}}\ULon}

%hand sectioning
\newcommand{\prlsec}[1]{{{\it #1:--}}}

%some annotation macros
\newcommand{\SP}[1]{{\color{ForestGreen}{\bf SP: #1}}}
\newcommand{\SPhide}[1]{{}}
\newcommand{\SPEDITED}[2]{{\redsout{#1}}{ \textcolor{red}{#2}}}
\newcommand{\SPEDITOKAY}[2]{{}{\textcolor{black}{#2}}}

\newcommand{\ND}[1]{{\color{Blue}{\bf #1}}}
\newcommand{\NDhide}[1]{{}}
\newcommand{\NDEDITED}[2]{{\bluesout{#1}}{ \textcolor{blue}{#2}}}
\newcommand{\NDEDITOKAY}[2]{{}{\textcolor{black}{#2}}}




\begin{document}
\title{Deconfined pseudocriticality in a model spin-1 quantum antiferromagnet}
\author{Vikas Vijigiri}
\affiliation{Department of Physics, Indian Institute of
Technology Bombay, Powai, Mumbai, MH 400076, India}
\author{Sumiran Pujari}
\email{sumiran.pujari@iitb.ac.in}
\affiliation{Department of Physics, Indian Institute of
Technology Bombay, Powai, Mumbai, MH 400076, India}
\author{Nisheeta Desai}
\affiliation{Department of Theoretical Physics, Tata Institute
of Fundamental Research, Colaba, Mumbai, MH 400005, India}


\begin{abstract}
Berry phase interference arguments that underlie the theory of 
deconfined quantum criticality (DQC) for $S=1/2$ antiferromagnets have 
also been invoked to allow for 
continuous transitions in $S=1$ magnets
including a N\'eel to (columnar) valence bond solid (cVBS)
transition. 
We provide a microscopic model realization of this transition
on the square lattice consisting of 
Heisenberg exchange ($J_H$) and biquadratic exchange ($J_B$) that
favor a N\'eel phase, and
 a designed $Q$-term ($Q_B$) interaction which favors a cVBS
 through large-scale quantum Monte Carlo (QMC) simulations. 
 For $J_H=0$, this model is equivalent to the $SU(3)$ 
 $JQ$ model 
 with a N\'eel-cVBS transition that has been argued to be 
 DQC through QMC.
 Upon turning on $J_H$ which brings down the symmetry to $SU(2)$, we find 
 multiple signatures -- a single critical point,
high quality collapse of correlation ratios and order parameters,
``$U(1)$-symmetric" cVBS histograms and
lack of double-peak in order parameter histograms 
for largest sizes studied
near the critical point --
that are highly suggestive of a continuous transition scenario.
However, Binder analysis finds negative dips that grow sub-extensively
that we interpret as these
transitions rather being pseudocritical.
This along with recent results on spin-$\frac{1}{2}$ models
suggests that deconfined pseudocriticality is 
the more generic scenario. 



\end{abstract}
\maketitle



The theory of deconfined quantum criticality (DQC) has been of great interest
since its 
inception~\cite{Senthil_etal_Science_2004,Senthil_etal_PRB_2004,levin_senthil_prb2004} because it proposes a concrete theoretical scenario 
\NDEDITOKAY{where the criticality is 
}{of a phase transition that is} beyond the Landau-Ginzburg-Wilson paradigm.
\NDEDITOKAY{}{A continuous transition between two symmetry unrelated phases} 
-- N\'eel and VBS -- was posited for spin-$\frac{1}{2}$
degrees of freedom in $2+1d$. Numerical hints of such a transition were seen 
in an earlier work~\cite{Sandvik_Scalapino_etal_PRL_2002}.
The Berry phase interference
effects~\cite{Read_Sachdev_prl1989,Read_Sachdev_prb1990,Read_Sachdev_nucphysB,Haldane_prl1988} 
of the quantum degrees of freedom is crucial for the beyond-LGW nature.
DQC is often described in terms of a gauge theory of fractionalized 
spinon degrees of freedom that deconfine only at the critical point.
The Higgs condensation of the spinons leads to antiferromagnetic N\'eel order
on one side, while on the other side the confinement of the 
associated $U(1)$ gauge field leads to valence bond crystallization.
Given its appeal, it has subsequently undergone a 
great deal of scrutiny in various spin-$1/2$ models~\cite{Sandvik_prl2007, Melko_Kaul_prl2008, Sandvik_prl2010, Banerjee_Damle_Alet_prb2010, Banerjee_Damle_Alet_prb2011, Sandvik_prb2012, Kaul_prb2014, Pujari_Damle_Alet_prl2013, Pujari_Alet_Damle_prb2015,Ma_etal_prl2019,Zhao_Takahashi_Sandvik_prl2020,Sandvik_Zhao_2020} and their SU($N$) generalizations~\cite{Beach_etal_prb2009, Lou_Sandvik_Kawashima_PRB_2009, Kaul_prb2012, Kaul_Sandvik_prl2012, Block_Melko_Kaul_prl2013, Harada_etal_prb2013},
e.g. see the review~\cite{Kaul_Melko_Sandvik_2013}.
Evidence for many proposed features of DQC have been numerically
seen at the transition in these studies,
including classical loop models and dimer models in 3$d$~\cite{charrier_prl2008,Powell_prl2008,Powell_prb2009,Chen_prb2009},
certain 1$d$ spin-$\frac{1}{2}$ extensions~\cite{Roberts_Jiang_Motrunich_prb2019,Huang_etal_prb2019},
and fermionic models~\cite{DaLiao_etal_prbAug2022,DaLiao_etal_prbOct2022,Sato_etal_prl2017,Li_natcomm2017,Li_arxiv2019} argued to be in the same universality class. 
The issue is contentious though due to the various
scaling violations observed~\cite{Kuklov_2006, Jiang_JStatMech2008,  Kaul_prb2011,  Kun_prl2013, Bartosch_PRB_2013, Shao_Guo_Sandvik_Science2016, nahum_etal_prx2015}, 
the origins of which are still being debated. 
Not much is however known for higher spins
owing both to the emphasis of the
original proposal and to the available recipes for constructing 
numerically accessible spin models that could host this 
physics~\cite{Kaul_prb2014}.


In this work, our focus will be on spin-$1$ for which there 
are only a handful of papers discussing possible DQC
and none which have shown DQC behavior in a microscopic model.
In Ref.~\onlinecite{Grover_Senthil_PRL_2007}, Grover and Senthil 
pointed out the possibility of a DQC from a spin-nematic state
to a VBS state based on analogous field-theoretic arguments
as in Ref.~\onlinecite{Senthil_etal_PRB_2004}.
In a later work~\cite{Wang_Kivelson_Lee_NatPhys_2015},
Wang, Kivelson and Lee proposed a possible DQC from a
N\'eel state to a bond-nematic or Haldane-nematic state 
that may be caricatured as
a symmetry broken stack of AKLT chains.
They further conjectured possible DQC between a N\'eel
to a cVBS state as a ``doubled" version of the spin-$\frac{1}{2}$
DQC theory~\cite{doubled_theory_footnote}.
We are motivated to investigate this latter scenario 
due to the heuristic explained below.

The class of $JQ$ models originated by 
Sandvik~\cite{Sandvik_prl2007} 
have been crucial in probing DQC physics on bipartite lattices
in large scale simulations. The basic units of these
models are the singlet projectors $P_{ij} = \left(\frac{1}{4} -
\mathbf{S}_i \cdot \mathbf{S}_j \right)$ on the bond $\langle ij \rangle$.
A nice aspect of the projector formulation is that they can naturally be 
lifted~\cite{sublattice_footnote} from $SU(2)$ to $SU(N)$ as 
$P_{ij} = \sum_{\alpha,\beta} |\alpha_i;\alpha_j\rangle\langle\beta_i;\beta_j|$.
This has been exploited to study DQC in $N>2$ microscopic models.
The fluctuations of the $U(1)$-gauge field get
suppressed as $N$ increases which makes for a closer match between
perturbation theory estimates and numerical estimates 
of critical exponents~\cite{Fig5_footnote}. 
For our purposes, we note that the $SU(3)$ $JQ$ model can be
recasted as a spin-$1$ model in terms of biquadratic exchanges
due to an exact mapping between the $SU(3)$ singlet projector
and the biquadratic exchange on a bond,
$P_{ij} =\frac{1}{3} \{(\mathbf{S}_i \cdot \mathbf{S}_j)^2 - \mathbb{I}\}$~\cite{Beach_etal_prb2009}. 
The $SU(3)$ $JQ$ model
was first studied by Lou, Sandvik and Kawashima in 
Ref.~\onlinecite{Lou_Sandvik_Kawashima_PRB_2009} where DQC behavior
between N\'eel and VBS was observed for system sizes up to $L=48$.
As a spin-$1$ Hamiltonian, this maps to biquadratic
exchange and a $Q$-term made out of them 
\begin{equation}
H_{SU(3)} = -J_B\sum_{\langle ij \rangle} P_{ij}
-
Q_B\sum_{\langle ijkl \rangle} \left( P_{ij} P_{kl} + P_{il} P_{jk} \right)
\label{eq:SU3Qterm}
\end{equation}
where $\langle ijkl \rangle$ stands for an elementary plaquette of
the square lattice with (say) a clockwise indexing of the sites on
it.
The first term favors N\'eel order. The $Q$-term favors a cVBS of
spin-$1$ valence bonds (which are also $SU(3)$ singlets
for Eq.~\ref{eq:SU3Qterm}).

%%%FIG 1 : ratio crossings
% Figure environment removed



Here, we
add an $SU(2)$-symmetric perturbation in the form of the 
spin-$1$ Heisenberg exchange which also favors N\'eel order
to study the same transition, i.e.
\begin{equation}
H_{SU(2)} = H_{SU(3)} + J_{H} \sum_{\langle ij \rangle} 
\{\mathbf{S}_i \cdot \mathbf{S}_j - \mathbb{I}\}
\label{eq:SU2Ham}
\end{equation}
We study the N\'eel-cVBS quantum phase transtion by varying 
$g \equiv \frac{Q_B}{J_B}$ for different values of $J_H$.
With the recent 
introduction of new designer spin models for $S>1/2$ 
systems~\cite{Desai_Kaul_prl2019}, the above model has been 
rendered simulatable with the same powerful deterministic 
non-local loop updates as in the standard stochastic series expansion based 
finite-temperature QMC~\cite{Sandvik_1991,Sandvik_1992,Sandvik_2010_review} for
$S=1/2$ models. Briefly, this is achieved by working in an expanded Hilbert 
space of two ``split $S=\frac{1}{2}$"s per site with the addition of 
only one non-deterministic loop move at a single ``projection" time slice 
to implement symmetrization over the two split $S=\frac{1}{2}$s
for all sites that restricts us to the physical $S=1$ 
subspace~\cite{Kawashima_1995,Todo_2001} in the QMC ensemble.
An earlier work~\cite{Wildeboer_etal_prb2020}
had instead studied the $J_{B}=0$ case and found 
strong first-order behavior. 
Thus, we will focus on the vicinity of $H_{SU(3)}$
as specified in Eq.~\ref{eq:SU2Ham} using standard notation
($i$ refers to sites, $\langle ij\rangle$ refers to nearest
neighbor bonds, $\mathbf{S}_i$ refers to spin-$1$ operators
on site $i$). 
We work in the units where $J_B=1.0$. 
In presence of the $SU(2)$-symmetric perturbation, the following scenarios 
may be expected: (1) $SU(3)$ criticality becomes first-order right upon turning 
on $J_H$ and we see some cross-over physics for small values of the 
perturbation, (2) there is a regime of $J_H$ for which the doubled 
$S=1/2$ DQC scenario obtains, or (3) there is weakly first-order or 
pseudocritical behaviour in this
regime. For the second scenario, we expect to see stable exponents that are 
either $SU(3)$ exponents or a new set of $SU(2)$ exponents in this regime. 

%%% FIG 2 : Histograms and time series
% Figure environment removed

%%%Table : Exponents
\begin{table*}
\caption{\label{tab:exponents}
Critical exponents ($\eta_V,\eta_N$, $\nu$) as obtained from 
scaling collapse analysis of the order parameters ($O_{N,B}$) 
for different $J_H$.
System sizes used for this analysis are in
the range of $L=24,32,40,48,64$. Additional corroborating 
analysis of correlation ratios and order parameters are shown 
in the Ref.~\cite{suppinfo}.
}
\begin{tabularx}{\textwidth}{*{10}{>{\centering\arraybackslash}X}}
\toprule
$J_H$ &  $\nu_{N}$ & $\nu_{V}$  & $\eta_{N}$ & $\eta_{V}$ & $g_{cN}$ & $g_{cV}$ & $\chi^2_N$ & $\chi^2_V$ \\
\toprule
0.0 & 0.53(3) & 0.63(1) & 0.44(5) & 0.49(2) & 0.168(1) & 0.167(1) & 1.08-1.68 & 1.69-2.46\\
0.01 & 0.45(2) & 0.54(3) & 0.23(3) & 0.42(4) & 0.174(1) & 0.171(1) & 1.19-1.63 & 1.38-1.73 \\
0.025 &  0.43(3) & 0.46(4) & 0.15(9) & 0.38(2)  & 0.182(1) & 0.180(1) & 0.75-1.46 & 0.8-1.4\\
0.04 & 0.40(2) & 0.43(5) & 0.13(7) & 0.30(8)  & 0.19(1) & 0.189(1) & 1.06-1.67 & 1.09-1.5\\
0.05 & 0.39(4) & 0.38(5) & 0.20(9) & 0.29(6)  & 0.196(1) & 0.195(1) & 0.87-1.31 & 0.87-1.96\\
0.07 & 0.38(2) & 0.39(3) & 0.10(4) & 0.10(4)  & 0.207(1) & 0.206(1) & 1.52-2.54 & 1.04-1.77\\
0.1 &  0.35(4) & 0.35(3) & -0.03(5) & -0.03(2)  & 0.224(1) & 0.224(1) & 1.24-3.28 & 0.99-1.97\\
0.15 &  0.33(2) & 0.33(1) & 0.00(8) & -0.12(8)  & 0.253(1) & 0.253(1) & 1.42-1.79 & 1.15-1.63\\
\toprule
\end{tabularx}
\label{dynamic}
\end{table*}

We probe the system by measuring intensive order parameters for the 
two phases. For the N\'eel phase, the 
staggered magnetization order parameter
$O_N \equiv \langle m^2 \rangle$, where 
$m = \frac{1}{N} \sum_{\mathbf r} e^{i (\pi,\pi).\mathbf{r}} S^z_{\mathbf{r}}$. 
For the columnar VBS phase, the cVBS order parameter 
$O_V \equiv \langle \left( \phi^2_x + \phi^2_y \right) \rangle$, 
where $\phi_\mu = \frac{1}{N} \sum_{\mathbf r} 
e^{i \pi \,\mathbf{e}_{\mu} \cdot \mathbf{r}} 
\{ (\mathbf{S}_{\mathbf r} \cdot \mathbf{S}_{\mathbf {r + e_\mu}})^2
- \mathbb{I} \}/3$~\cite{cvbs_footnote}.
Throughout, the inverse temperature is set as 
$\beta = \frac{L}{4}$ to study ground state 
properties~\cite{beta_convergence_footnote}.
Correlation ratios ($R$) of these order
parameters~\cite{corr_ratio_footnote} 
are plotted as a function of the coupling $g=\frac{Q_B}{J_B}$ 
in Fig~\ref{fig:Ratios_gc}.
The ratios plotted for different system sizes show a crossing at the 
transition as shown for a representative value of $J_H$ in 
Fig.~\ref{fig:Ratios_gc}a,b. The crossing points of both the N\'eel and VBS 
ratios converge as a function of $1/L$ (Fig.~\ref{fig:Ratios_gc}c)
indicating a direct  transition between the two phases with no 
intermediate phase. 



Having established a direct transition, we now take a look at 
histograms of the order parameters to probe the nature of the transition. 
No signature of two-peak behavior in the histograms of the 
order parameters near the transition is 
seen as shown for staggered magnetization 
in Fig~\ref{fig:histograms}a. 
The characteristic telegraphic switching between the two 
order parameters at a first order transition is also absent near 
the transition (Fig~\ref{fig:histograms}c). This
rules out the first scenario of a strongly first-order transition.
Furthermore, ``$U(1)$-symmetric"
$(\phi_x,\phi_y)$-histograms are also seen near the transition
for the largest system size studied
(Fig~\ref{fig:histograms}b). In $S=\frac{1}{2}$ studies,
this has been considered a key evidence of DQC.
This is associated with the dangerous irrelevancy of the 
the operators in the DQC theory that capture the 
dominant quantum fluctuations during the transition 
out of the N\'eel phase.
In our $S=1$ model, the appropriate operators
will be those of the doubled-DQC theory~\cite{monopole_footnote}.
We therefore perform scaling collapses of the order parameters and
the correlation ratios as shown in Fig.~\ref{fig:scalingcollapse}.
The scaling collapses are of high quality throughout 
with the $\chi^2$ per degree of freedom being close to 1
for all value of $J_H$ studied. 
The exponents extracted from 
these finite-size scaling analysis are summarized in
Table~\ref{tab:exponents}. We find the exponents to be
stable to various protocols involving 
the range of the tuning parameter $g$ and system sizes 
used for the collapse analysis.

The first thing of note is that the anomalous exponents $\eta_N$ and
$\eta_V$ are markedly different as soon as $J_H \neq 0$.
For $J_H = 0$, we obtain $SU(3)$ exponents in overall agreement 
with earlier work of Lou \textit{et al}~\cite{Lou_Sandvik_Kawashima_PRB_2009}
though our best estimate for the N\'eel correlation exponent $\nu_N$ 
is different.
The collapse quality when $\nu_N$ is set same as $\nu_V$ ($\sim 0.63$) is not
significantly worse. The equality $\nu_N = \nu_V$ is one
of the expectations in the theory of DQC.
This marked difference from $J_H=0$ suggests that $SU(3)$ criticality 
is not obtained when $J_H \neq 0$,
i.e. the $SU(2)$ perturbation changes the universality class. 
This is not entirely unexpected and can be taken as evidence for the 
doubled spin-$1/2$ DQC scenario of 
Wang \textit{et al}~\cite{Wang_Kivelson_Lee_NatPhys_2015}.
However, there is a slow drift in the exponents as $J_H$ increases
which argues against a stable set of 
exponents~\cite{exponent_stability_footnote} 
as expected in the second scenario.
The drift is noticeable even within the accuracy levels in the exponent
estimates achieved by us. This level of accuracy
in the estimation of critical exponents is not unusual
in numerical studies of DQC. Similarly, the best estimates of
$\nu_N$ and $\nu_V$ for each value of $J_H$ do not match in all
cases, but again setting them equal does not lead to significant
loss of collapse quality similar to $J_H=0$.
Nevertheless, we certainly see
that the anomalous exponents $\eta_N$, $\eta_V$ 
are not small. This is one of the 
expectations in the theory of DQC which is 
strikingly different than conventional
second-order critical points, and seen in 
previous $S=\frac{1}{2}$ studies.
Eventually, for large enough $J_H=0.1$ and $0.15$, 
the anomalous exponents go negative indicating first-order behavior.
This is to be expected since first-order
behaviour has been seen for $J_B=0$~\cite{Wildeboer_etal_prb2020},
i.e. when $J_H$ becomes large enough.
The finite-size scaling analysis was performed using the
Bayesian scheme implemented by Harada~\cite{Harada_PRE2011}. 

%%%FIG 3 : Collapses
% Figure environment removed

%%% FIG 4 : Binder ratios
% Figure environment removed




From the preceding discussions, 
the N\'eel-cVBS transition in Eq.~\ref{eq:SU2Ham} appears continuous 
up to $J_H \sim 0.07$.
However, due to the observed drifts in the exponents, we further examine
the Binder ratios of the magnetization order 
parameter across this transition whose behavior would be expected to
be similar to the correlation ratios. 
Fig~\ref{fig:binder_crossings} shows this ratio, 
defined as 
$\frac{5}{2}(3-\frac{\langle m^4 \rangle}{\langle m^2 \rangle^2})$, 
with a clearly visible crossing at the transition point. 
Equally noteworthy is the clear dip below zero near the transition.
This is seen for all values of $J_H$~\cite{suppinfo}. 
A characteristic of first order transitions is that this dip 
grows extensively 
as the system size. In Ref~\cite{Kaul_prb2011}, where the $J_H=0$ case was
studied in the $SU(3)$ language, Kaul had noted that this dip grows 
subextensively as system size and interpreted it as evidence for a 
continuous transition. We similarly find that this negative dip grows 
slower than extensive in the regime of small $J_H$
as shown in Fig~\ref{fig:binderdips_vs_L}. 
As $J_H$ grows 
larger, it eventually grows as $L^2$ as expected for a first order 
transition~\cite{Vollmayr_etal_1993,Sen_Sandvik_PRB2010}.


Given the lack of stability of the exponents 
seen earlier that would be expected for a bonafide DQC scenario,
we rather interpret the subextensive growth of the Binder dip near
the transition as evidence for deconfined pseudocriticality.
In other words, the scenario of a doubled $S=\frac{1}{2}$ DQC 
provides a natural framework to understand the above set of numerical results,
but the spinons get weakly confined at much larger length scales (dependendent
on $J_H$)
than the lattice scale leading to the observed pseudocritical behavior 
as discussed more below. 
In terms of our
simulations, we may imagine it as pseudo-DQC ensembles occuring in both the 
split $S=\frac{1}{2}$ Hilbert spaces in (QMC) space-time which then gets 
``inherited" by the $S=1$ system under projection back to the
fully symmetric subspace.
This also implies a revision of the earlier interpretation of DQC 
in $SU(3)$ $JQ$ model~\cite{Kaul_prb2011}.



%%% FIG 5 : Binder minimum
% Figure environment removed



We conclude by situating our interpretation of deconfined pseudocriticality 
in our $S=1$ microscopic model 
in the light of recent developments that have thrown open 
the issue of the putative second-order nature of 
DQCs~\cite{demidio_etal_arxiv2021,yuan_etal_arxiv2023, Song_etal_Arxiv2023}. 
These arose in the context of earlier works 
regarding emergent symmetry expectations at these
transitions when interpreted in a $SO(5)$ framework with a combined
order parameter built out of the N\'eel and VBS order parameters.
The \NDEDITOKAY{DQC can then}{original $S=\frac{1}{2}$ 
DQC proposal can}  be re-expressed 
\NDEDITOKAY{as a relevant perturbation}{in terms of this combined 
order parameter with a single relevant perturbation} tuning us 
across the transition\NDEDITOKAY{}{~\cite{Tanaka_Hu_prl2005,Senthil_Fisher_prb2006}}, which 
also implies an enhanced emergent symmetry between the two symmetry unrelated
order parameters. This basic expectation has been numerically studied in
various cases~\cite{nahum_etal_prl2015,nahum_etal_prx2015}, in 
certain unconventional transitions~\cite{Zhao_natphys2019} 
including  in one dimension~\cite{patil_katz_sandvik_prb2018,Xi_2022} 
and classical dimer models~\cite{sreejith_prl2019}. System size 
restrictions however makes interpretation of emergent symmetry tricky. 
From the theory side, conformal
bootstrap results by Nakayama and Ohtsuki~\cite{Nakayama_Ohtsuki_PRL_2016} 
pointed out strong constraints on the scaling dimensions
that apparently rule out emergent symmetry in $2+1d$ DQC.
Recent quantum entanglement based results 
by Diao \emph{et al}~\cite{yuan_etal_arxiv2023} 
and Song \emph{et al}~\cite{Song_etal_Arxiv2023} have significantly 
taken forward this line of argument.
Ma, Wang~\cite{Ma_Wang_PRB_2020}, and Nahum~\cite{Nahum_PRB_2020}
had thus conjectured the notion of DQC as pseudocriticality
in order to reconcile with the conformal bootstrap result.
The idea is that the RG fixed point for DQC does not reside
in the (physical) $2+1d$, but slightly below it 
(see Fig.~1 of Ref.~\cite{Ma_Wang_PRB_2020}) such
that pseudocritically slow flows obtain near the critical point
$g_c$. 
For a more detailed discussion on these issues, see this recent
review~\cite{Senthil_review_2023}.
Such slow RG flows may account for 
the good scaling collapse seen in numerical data for 
accessible system sizes \emph{along} with the observed
drifts in the critical exponents.


It is noteworthy that even before the emergent symmetry point
of view gained currency, numerical studies of DQC had seen
anomalous corrections to scalings whose origin
was unclear. 
Bartosch gave an explanation based on scaling corrections
inherent in the effective $U(1)$ gauge theory of the 
deconfined spinons~\cite{Bartosch_PRB_2013}.
A more recent explanation based on two different diverging length scales
associated with spinon correlations and VBS domain wall sizes
being governed by different exponents
has been proposed by Shao, Guo and Sandvik as another possible resolution~\cite{Shao_Guo_Sandvik_Science2016}. 
In the context of our interpretation, pseudocriticality rather
implies the confinement of the spinons
of the $U(1)$ gauge theory framework for all values of the tuning
parameter, and the confinement length scale becomes very large
($L \gtrsim O(100)$ given the high quality of scaling collapses
seen) compared the lattice scale~\cite{Sandvik_youtube_2022} when
the RG flow is pseudocritically slow near the apparent transition.
The present consensus seems to be veering towards 
deconfined pseudocriticality
based on recent $S=\frac{1}{2}$ results, and our work
provides a microscopic $S=1$ model for this
scenario as yet not seen in the literature.
This opens a mystery in the context of $SU(N)$ DQC where,
on one hand, theoretical expectations based on the suppression
of gauge fluctuations with increasing $N$ make the case for
bonafide DQC, but on the other hand, no negative Binder dips
has been numerically seen in $SU(2)$ $JQ$ models for 
largest sizes studied
making the pattern of Binder dip growth with respect to
$N$ non-monotonic~\cite{fig1_Kaulprb2011_footnote}.
It will indeed be an useful theoretical advance and 
strong evidence  for the pseudocriticality scenario if the 
subextensive growth of negative Binder ratio dips can be shown to 
be a generic consequence of the pseudocritcal RG
flows of Refs.~\cite{Nahum_PRB_2020,Ma_Wang_PRB_2020}
that may further clarify the present state of affairs~\cite{Song_etal_Arxiv2023}.



\prlsec{Acknowledgements}
We acknowledge useful discussions with Fabien Alet, Subhro Bhattacharjee,
Kedar Damle, Prashant Kumar and Adam Nahum.
VVi was supported by the institute post-doctoral
fellowship program at IIT Bombay,
and in part by the International Centre for Theoretical Sciences (ICTS) 
during the program - 8th Indian Statistical Physics 
Community Meeting (code: ICTS/ISPCM2023/02).
SP acknowledges funding support from SERB-DST, India 
via Grants No. SRG/2019/001419 and MTR/2022/000386.
Partial support by Grant No. CRG/2021/003024 is also
acknowledged.
ND was initially supported by National Postdoctoral Fellowship of 
SERB, DST, Govt. of India (PDF/2020/001658) at the department of 
Theoretical Physics, TIFR and presently by the TIFR postdoctoral 
fellowship.
The numerical results were obtained using
the computational facilities of the Department of Physics,
Indian Institute of Technology (IIT) Bombay.















\bibliographystyle{apsrev}
\bibliography{Seq1_refs}


%\section{Appendix}

\end{document}
