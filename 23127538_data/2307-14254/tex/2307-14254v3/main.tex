\documentclass[prl,twocolumn,superscriptaddress]{revtex4-2}
\usepackage{amsmath,amssymb,bm}
\usepackage{hyperref}
\usepackage{graphicx}
\usepackage{epstopdf}
\usepackage{latexsym}
\usepackage{subfigure}
\usepackage[usenames, dvipsnames]{color}
\usepackage[usenames, dvipsnames]{xcolor}
\usepackage{natbib}
\usepackage{braket}
\usepackage{float}
\usepackage[normalem]{ulem}
\usepackage{comment}
\usepackage{mathtools}
\usepackage{array}
\usepackage{tabu}
\usepackage{multirow}
\usepackage{chemformula}
\usepackage{svg}
\usepackage{tabularx,ragged2e}
%\usepackage{dblfloatfix} %%for putting table at bottom of the page
%\usepackage{stfloats} %%for putting table at bottom of the page
\usepackage{dcolumn}% Align table columns on decimal point
%\sisetup{table-format=1.3}

%for strike-outs
\newcommand\redsout{\bgroup\markoverwith{\textcolor{red}{\rule[0.5ex]{2pt}{0.4pt}}}\ULon}
\newcommand\bluesout{\bgroup\markoverwith{\textcolor{blue}{\rule[0.5ex]{2pt}{0.4pt}}}\ULon}

%hand sectioning
\newcommand{\prlsec}[1]{{{\it #1:--}}}

%some annotation macros
\newcommand{\SP}[1]{{\color{ForestGreen}{\bf SP: #1}}}
\newcommand{\SPhide}[1]{{}}
\newcommand{\SPEDITED}[2]{{\redsout{#1}}{ \textcolor{red}{#2}}}
\newcommand{\SPEDITOKAY}[2]{{}{\textcolor{black}{#2}}}

\newcommand{\ND}[1]{{\color{Blue}{\bf #1}}}
\newcommand{\NDhide}[1]{{}}
\newcommand{\NDEDITED}[2]{{\bluesout{#1}}{ \textcolor{blue}{#2}}}
\newcommand{\NDEDITOKAY}[2]{{}{\textcolor{black}{#2}}}




\begin{document}
\title{Deconfined pseudocriticality in a model spin-1 quantum antiferromagnet}
\author{Vikas Vijigiri}
\affiliation{Department of Physics, Indian Institute of
Technology Bombay, Powai, Mumbai, MH 400076, India}
\author{Sumiran Pujari}
\email{sumiran.pujari@iitb.ac.in}
\affiliation{Department of Physics, Indian Institute of
Technology Bombay, Powai, Mumbai, MH 400076, India}
\author{Nisheeta Desai}
\affiliation{Department of Theoretical Physics, Tata Institute
of Fundamental Research, Colaba, Mumbai, MH 400005, India}


\begin{abstract}
Berry phase interference arguments that underlie the theory of 
deconfined quantum criticality (DQC) for $S=1/2$ antiferromagnets have 
also been invoked to allow for 
continuous transitions in $S=1$ magnets
including a N\'eel to (columnar) valence bond solid (cVBS)
transition. 
We provide a microscopic model realization of this transition
on the square lattice consisting of 
Heisenberg exchange ($J_H$) and biquadratic exchange ($J_B$) that
favor a N\'eel phase, and
 a designed $Q$-term ($Q_B$) interaction which favors a cVBS
 through large-scale quantum Monte Carlo (QMC) simulations. 
 For $J_H=0$, this model is equivalent to the $SU(3)$ 
 $JQ$ model 
 with a N\'eel-cVBS transition that has been argued to be 
 DQC through QMC.
 Upon turning on $J_H$ which brings down the symmetry to $SU(2)$, we find 
 multiple signatures -- a single critical point,
high quality collapse of correlation ratios and order parameters,
``$U(1)$-symmetric" cVBS histograms and
lack of double-peak in order parameter histograms 
for largest sizes studied
near the critical point --
that are highly suggestive of a continuous transition scenario.
However, Binder analysis finds negative dips that grow sub-extensively
that we interpret as these
transitions rather being pseudocritical.
This along with recent results on spin-$\frac{1}{2}$ models
suggests that deconfined pseudocriticality is 
the more generic scenario. 



\end{abstract}
\maketitle



The theory of deconfined quantum criticality 
(DQC)~\cite{Senthil_etal_Science_2004,Senthil_etal_PRB_2004,levin_senthil_prb2004,
Sandvik_Scalapino_etal_PRL_2002} 
has been of great interest 
as it lies beyond the Landau-Ginzburg-Wilson-Fisher paradigm.
It posits a continuous transition between two symmetry unrelated phases
-- N\'eel and valence bond solid (VBS) -- for spin-$\frac{1}{2}$
moments in $2+1d$. 
DQC is described as a gauge theory of fractionalized 
spinon degrees of freedom that deconfine only at the critical point.
Their Higgs condensation 
leads to antiferromagnetic N\'eel order,
while on the other side the confinement of the 
associated $U(1)$ gauge field~\cite{Haldane_prl1988,Read_Sachdev_prl1989,Read_Sachdev_prb1990,Read_Sachdev_nucphysB} 
leads to the VBS.
It has undergone a 
great deal of scrutiny~\cite{Kaul_Melko_Sandvik_2013} 
in various spin-$1/2$ models~\cite{Sandvik_prl2007, Melko_Kaul_prl2008, Sandvik_prl2010, Banerjee_Damle_Alet_prb2010, Banerjee_Damle_Alet_prb2011, Sandvik_prb2012, Kaul_prb2014, Pujari_Damle_Alet_prl2013, Pujari_Alet_Damle_prb2015,Ma_etal_prl2019,Zhao_Takahashi_Sandvik_prl2020,Sandvik_Zhao_2020} and their SU($N$) generalizations~\cite{Beach_etal_prb2009, Lou_Sandvik_Kawashima_PRB_2009, Kaul_prb2012, Kaul_Sandvik_prl2012, Block_Melko_Kaul_prl2013, Harada_etal_prb2013}.
Evidence for many features of DQC have been numerically
seen in these studies,
including classical loop models and dimer models in 3$d$~\cite{charrier_prl2008,Powell_prl2008,Powell_prb2009,Chen_prb2009},
certain $1+1d$ spin-$\frac{1}{2}$ extensions~\cite{Roberts_Jiang_Motrunich_prb2019,Huang_etal_prb2019,
Zhang_Levin_PRL2023},
and fermionic models~\cite{DaLiao_etal_prbAug2022,DaLiao_etal_prbOct2022,Sato_etal_prl2017,Li_natcomm2017,Li_arxiv2019}.However scaling violations 
have also been seen~\cite{Kuklov_2006, Jiang_JStatMech2008,  Kaul_prb2011,  Kun_prl2013, Bartosch_PRB_2013, Shao_Guo_Sandvik_Science2016, nahum_etal_prx2015} 
that are still under debate. 
Not much is known though for $S>\frac{1}{2}$.

Our focus will be on spin-$1$ here.
There are only a handful of works discussing possible DQC
and none which have shown DQC behavior 
in microscopic model realizations.
Previously, Ref.~\cite{Grover_Senthil_PRL_2007}
argued for a possible DQC from a spin-nematic state
to a VBS state
based on field-theoretic arguments.
Ref.~\cite{Wang_Kivelson_Lee_NatPhys_2015} 
similarly conjectured a possible DQC from
N\'eel to a bond-nematic or Haldane-nematic state
which has been numerically investigated in
Refs.~\cite{Jiang_etal_PRB_2009,Niesen_Corboz_PRB_2017, Niesen_Corboz_SciPost_2017,
Hu_etal_PRB_2019}.
It further conjectured possible DQC 
from N\'eel to cVBS as a ``doubled" 
spin-$\frac{1}{2}$ DQC theory~\cite{doubled_theory_footnote}.
Briefly, the theory is formulated by taking 
two copies of an $SO(5)$ field theory with
$k=1$ Wess-Zumino-Witten term
for a combined 5-component order parameter field made
from the N\'eel and columnar-VBS 
fields~\cite{Tanaka_Hu_prl2005,Senthil_Fisher_prb2006}
that has been used to describe $S=\frac{1}{2}$ DQC. 
Upon
invoking a strong ferromagnetic coupling between the two
copies to be consistent with spin-$1$ at low energies, 
it reduces to a ``single" $SO(5)$ field theory
now with a doubled Wess-Zumino-Witten term. It is
the presence of this topological term which can allow for a
DQC betwen N\'eel to cVBS (for details, see the
supplementary of Ref.~\cite{Wang_Kivelson_Lee_NatPhys_2015}).



We computationally investigate this latter scenario 
based on the following heuristic:
$JQ$ models~\cite{Sandvik_prl2007} 
have been crucial in probing DQC physics 
in large-scale simulations. The basic units are 
$SU(2)$ singlet projectors $P^2_{ij} = \left(\frac{1}{4} -
\mathbf{s}_i \cdot \mathbf{s}_j \right)$ for $S=\frac{1}{2}$
on bond $\langle ij \rangle$.
One can extend~\cite{sublattice_footnote} them 
to $SU(N)$ as 
$P^N_{ij} = \sum^N_{\alpha,\beta=1} |\alpha_i;\alpha_j\rangle\langle\beta_i;\beta_j|$.
The $U(1)$-gauge fluctuations get suppressed as $N$ increases. 
This gives closer match between perturbation theory 
and numerical estimates 
of critical exponents in $N>2$ microscopic models~\cite{Fig5_footnote}. 
The $SU(3)$ $JQ$ model 
\begin{equation}
H_{SU(3)} = -J_B \sum_{\langle ij \rangle} P^3_{ij}
-
Q_B\sum_{\langle ijkl \rangle} \left( P^3_{ij} P^3_{kl} + P^3_{il} P^3_{jk} \right)
\label{eq:SU3Qterm}
\end{equation}
can also be
recasted as a spin-$1$ model 
since $P_{ij} = \frac{1}{3} \{(\mathbf{S}_i \cdot \mathbf{S}_j)^2 - \mathbb{I}\}$
biquadratic exchange for $S=1$~\cite{Beach_etal_prb2009}. 
$\langle ijkl \rangle$ indexes elementary plaquettes of
the square lattice with a clockwise indexing of the sites $i,j,k,l$.
The first term favors N\'eel order. The $Q$-term favors a cVBS of
spin-$1$ valence bonds (also $SU(3)$ singlets
for Eq.~\ref{eq:SU3Qterm}).
DQC behavior
between N\'eel and VBS has been observed in the $SU(3)$ $JQ$ model
up to system sizes $L=48$~\cite{Lou_Sandvik_Kawashima_PRB_2009}.

%%%FIG 1 : ratio crossings
% Figure environment removed


We now
add $S=1$ Heisenberg exchange favoring N\'eel order 
as an $SU(2)$-symmetric perturbation 
to study the same transition, 
i.e.
\begin{equation}
H_{SU(2)} = H_{SU(3)} + J_{H} \sum_{\langle ij \rangle} 
\{\mathbf{S}_i \cdot \mathbf{S}_j - \mathbb{I}\}
\label{eq:SU2Ham}
\end{equation}
by varying 
$g \equiv \frac{Q_B}{J_B}$ for different
$J_H$ using  
quantum Monte Carlo (QMC) methods~\cite{Desai_Kaul_prl2019,Sandvik_1991,Sandvik_1992,Sandvik_2010_review}.
We note for later discussion that 
our QMC method work with 
a (``doubled") Hilbert space of two ``split $S=\frac{1}{2}$"s per site
and a symmetrization step~\cite{Kawashima_1995,Todo_2001} to restrict
to the physical $S=1$ subspace quite in analogy with doubled $S=\frac{1}{2}$
DQC theory. 
An earlier work~\cite{Wildeboer_etal_prb2020}
had studied the $J_{B}=0$ case and found 
strong first-order behavior. 
We will thus focus on the vicinity of $H_{SU(3)}$,
i.e. Eq.~\ref{eq:SU2Ham} 
where $i$ refers to sites, $\langle ij\rangle$ to nearest
neighbor bonds, and $\mathbf{S}_i$ to spin-$1$ operators.
We work in the units where $J_B=1.0$. 
The following scenarios 
may be expected: (1) $SU(3)$ criticality becomes first-order right upon turning 
on $J_H$ and we see some cross-over physics for small values of the 
perturbation, (2) there is a regime of $J_H$ for which the doubled 
$S=1/2$ DQC scenario obtains, or (3) there is weakly first-order or 
pseudocritical behaviour in this
regime. For the second scenario, we expect to see stable exponents that are 
either $SU(3)$ exponents or a new set of $SU(2)$ exponents. 


We probe the system by measuring intensive order parameters for the 
two phases. For the N\'eel phase, the 
staggered magnetization order parameter is
$O_N \equiv \langle m^2 \rangle$, where 
$m = \frac{1}{N} \sum_{\mathbf r} e^{i (\pi,\pi).\mathbf{r}} S^z_{\mathbf{r}}$. 
For the columnar VBS phase, the cVBS order parameter is
$O_V \equiv \langle \left( \phi^2_x + \phi^2_y \right) \rangle$, 
where $\phi_\mu = \frac{1}{N} \sum_{\mathbf r} 
e^{i \pi \,\mathbf{e}_{\mu} \cdot \mathbf{r}} 
\{ (\mathbf{S}_{\mathbf r} \cdot \mathbf{S}_{\mathbf {r + e_\mu}})^2
- \mathbb{I} \}/3$~\cite{cvbs_footnote}.
The inverse temperature 
$\beta$ is set equal to $L/4$ to study ground state 
properties~\cite{beta_convergence_footnote}.
Fig~\ref{fig:Ratios_gc} shows correlation ratios ($R$) of these order
parameters~\cite{corr_ratio_footnote} 
versus $g=\frac{Q_B}{J_B}$
for a representative value of $J_H=0.05$.
We see a clear crossing at the transition 
in Fig.~\ref{fig:Ratios_gc}a,b. The crossing points of both the N\'eel and VBS 
ratios converge 
versus $1/L$ (Fig.~\ref{fig:Ratios_gc}c)
implying a direct  transition between the two phases with no 
intermediate phase. 

%%% FIG 2 : Histograms and time series
% Figure environment removed

We now look at order parameter histograms to probe 
the nature of this direct transition. 
No signature of two-peak behavior is seen
near the transition  as shown for staggered magnetization 
in Fig~\ref{fig:histograms}a. 
Telegraphic switching between the two 
order parameters is also absent near 
the transition (Fig~\ref{fig:histograms}c). This
rules out the first scenario of a strongly first-order transition.
Furthermore, ``$U(1)$-symmetric"
$(\phi_x,\phi_y)$-histograms are also seen near the transition
for the largest system size studied
(Fig~\ref{fig:histograms}b). In $S=\frac{1}{2}$ studies,
this has been considered a key evidence of DQC.
This is associated with the dangerous irrelevancy of the 
the operators in the DQC theory that capture the 
dominant quantum fluctuations out of the N\'eel phase.
For $S=1$, the appropriate operators
are those of the doubled-DQC theory~\cite{monopole_footnote}.
We therefore perform scaling collapses~\cite{Harada_PRE2011} 
of the order parameters and
the correlation ratios as shown in Fig.~\ref{fig:scalingcollapse}.
The scaling collapses are of high quality for all $J_H$
with the $\chi^2$ per degree of freedom being close to 1
throughout.
Table~\ref{tab:exponents} lists the exponents extracted from 
this finite-size scaling analysis.
The exponents are stable to various protocols involving 
the range of the tuning parameter $g$ and system sizes 
used for the collapses. 

%%%Table : Exponents
\begin{table*}
\caption{\label{tab:exponents}
Critical exponents ($\eta_V,\eta_N$, $\nu$) as obtained from 
scaling collapse analysis of the order parameters ($O_{N,B}$) 
for different $J_H$.
% System sizes used for this analysis are in
% the range of 
$L=24,32,40,48,64$ used here. 
Additional corroborating 
analysis with correlation ratios 
%and order parameters 
is shown in the Ref.~\cite{suppinfo}.
}
\begin{tabularx}{\textwidth}{*{10}{>{\centering\arraybackslash}X}}
\toprule
$J_H$ &  $\nu_{N}$ & $\nu_{V}$  & $\eta_{N}$ & $\eta_{V}$ & $g_{cN}$ & $g_{cV}$ & $\chi^2_N$ & $\chi^2_V$ \\
\toprule
0.0 & 0.53(3) & 0.63(1) & 0.44(5) & 0.49(2) & 0.168(1) & 0.167(1) & 1.08-1.68 & 1.69-2.46\\
0.01 & 0.45(2) & 0.54(3) & 0.23(3) & 0.42(4) & 0.174(1) & 0.171(1) & 1.19-1.63 & 1.38-1.73 \\
0.025 &  0.43(3) & 0.46(4) & 0.15(9) & 0.38(2)  & 0.182(1) & 0.180(1) & 0.75-1.46 & 0.8-1.4\\
0.04 & 0.40(2) & 0.43(5) & 0.13(7) & 0.30(8)  & 0.19(1) & 0.189(1) & 1.06-1.67 & 1.09-1.5\\
0.05 & 0.39(4) & 0.38(5) & 0.20(9) & 0.29(6)  & 0.196(1) & 0.195(1) & 0.87-1.31 & 0.87-1.96\\
0.07 & 0.38(2) & 0.39(3) & 0.10(4) & 0.10(4)  & 0.207(1) & 0.206(1) & 1.52-2.54 & 1.04-1.77\\
0.1 &  0.35(4) & 0.35(3) & -0.03(5) & -0.03(2)  & 0.224(1) & 0.224(1) & 1.24-3.28 & 0.99-1.97\\
0.15 &  0.33(2) & 0.33(1) & 0.00(8) & -0.12(8)  & 0.253(1) & 0.253(1) & 1.42-1.79 & 1.15-1.63\\
\toprule
\end{tabularx}
\label{dynamic}
\end{table*}

The first thing of note is that the anomalous exponents $\eta_N$ and
$\eta_V$ are markedly different as soon as $J_H \neq 0$.
For $J_H = 0$, we obtain $SU(3)$ exponents in overall agreement 
with earlier work~\cite{Lou_Sandvik_Kawashima_PRB_2009}
though our best estimate for the N\'eel correlation exponent $\nu_N$ 
is different.
The collapse quality when $\nu_N$ is set same as $\nu_V$ ($\sim 0.63$) is not
significantly worse. The equality $\nu_N = \nu_V$ is 
expected in the theory of DQC.
This marked difference of $\eta_N, \eta_V$
from $J_H=0$ suggests that $SU(3)$ criticality 
is not obtained when $J_H \neq 0$,
i.e. the $SU(2)$ perturbation changes the universality class. 
This is not entirely unexpected and can be taken as evidence for the 
doubled spin-$1/2$ DQC scenario~\cite{Wang_Kivelson_Lee_NatPhys_2015}.
However, there is a slow drift in the exponents as $J_H$ increases
which argues against a stable set of 
exponents~\cite{exponent_stability_footnote} 
as expected in the second scenario.
The drift is noticeable even within the accuracy levels
achieved by us. This level of accuracy
in the estimation of critical exponents is not unusual
in numerical studies of DQC. Similarly, the best estimates of
$\nu_N$ and $\nu_V$ for each value of $J_H$ do not match in all
cases, but again setting them equal does not lead to significant
loss of collapse quality. 
Nevertheless, we certainly see
that the anomalous exponents $\eta_N$, $\eta_V$ 
are not small. This is one of the 
expectations in the theory of DQC which is 
strikingly different than conventional
second-order critical points, and seen in 
previous $S=\frac{1}{2}$ studies.
Eventually for large enough $J_H=0.1$ and $0.15$, 
the anomalous exponents go negative indicating first-order behavior.
The correlation exponents $\nu_N$, $\nu_V \sim \frac{1}{2+1}$ for these
$J_H$ values as well.
This is to be expected when $J_H$ becomes large enough
since first-order
behaviour has been seen for $J_B=0$~\cite{Wildeboer_etal_prb2020}.
 

%%%FIG 3 : Collapses
% Figure environment removed

From the preceding discussions, 
the N\'eel-cVBS transition in Eq.~\ref{eq:SU2Ham} appears continuous 
up to $J_H \sim 0.07$.
However, due to the observed drifts in the exponents, we further examine
the Binder ratios of the magnetization order parameter. 
We would expect them to
behave similar to the correlation ratios. 
Fig~\ref{fig:binder_crossings} shows this ratio, 
defined as 
$\frac{5}{2}(3-\frac{\langle m^4 \rangle}{\langle m^2 \rangle^2})$, 
with a clearly visible crossing at the transition point. 
Equally noteworthy is the clear dip below zero near the transition.
This is seen for all values of $J_H$~\cite{suppinfo}. 
A characteristic of first order transitions is that this dip 
grows extensively with system size~\cite{Vollmayr_etal_1993}. 
In Ref~\cite{Kaul_prb2011}, where the $J_H=0$ case was
studied, it had been noted that this dip grows 
sub-extensively with system size and interpreted as evidence for a 
continuous transition. We similarly find 
sub-extensively growing dips for small $J_H$
as shown in Fig~\ref{fig:binderdips_vs_L}. 
As $J_H$ grows 
larger, it eventually grows as $L^2$ as expected for a first order 
transition~\cite{Sen_Sandvik_PRB2010}
concomitantly with $\nu_N$, $\nu_V \sim \frac{1}{2+1}$.


Given the drifting exponents 
seen earlier that argues against a bonafide DQC scenario~\cite{Demidio_footnote},
we rather interpret the sub-extensive growing Binder dips 
along with the $U(1)$-symmetric VBS histograms
as evidence for deconfined pseudocriticality.
In other words, the scenario of a doubled $S=\frac{1}{2}$ DQC 
provides a framework to understand the above set of numerical results,
but the deconfinement gets curtailed at much larger length scales (dependent
on $J_H$)
than the lattice scale leading to the observed pseudocritical behavior. 
In terms of QMC simulations, we 
imagine it as pseudo-DQC ensembles occurring in both the 
split $S=\frac{1}{2}$ Hilbert spaces in (QMC) space-time which then gets 
``inherited" by the $S=1$ system under projection back to the
fully symmetric subspace.
This also implies a revision of the earlier interpretation of DQC 
in $SU(3)$ $JQ$ model~\cite{Kaul_prb2011}.

%%% FIG 4 : Binder ratios
% Figure environment removed


We 
%conclude by 
situate our interpretation of deconfined pseudocriticality 
in our $S=1$ microscopic model 
in the light of recent developments that have thrown open 
the issue of the second-order nature of 
DQCs~\cite{demidio_etal_arxiv2021,Zhao_etal_PRL2022,yuan_etal_arxiv2023,Song_etal_Arxiv2023}. 
These arose in the context of earlier works 
regarding emergent symmetry expectations at these
transitions when interpreted in a $SO(5)$ framework with a combined
order parameter built out of the N\'eel and VBS order parameters
~\cite{Tanaka_Hu_prl2005,Senthil_Fisher_prb2006}.
%The original $S=\frac{1}{2}$ 
%DQC proposal can be re-expressed 
%in terms of this combined 
%order parameter with a single relevant perturbation tuning us 
%across the putative continuous transition~\cite{Tanaka_Hu_prl2005,Senthil_Fisher_prb2006}.
This would imply an enhanced emergent symmetry between the two symmetry unrelated
order parameters. This basic expectation has been numerically studied in
various cases~\cite{nahum_etal_prl2015,nahum_etal_prx2015,Fuzzy_Sphere_SO5_arxiv}, in 
certain unconventional transitions~\cite{Zhao_natphys2019} 
including  in one dimension~\cite{patil_katz_sandvik_prb2018,Xi_2022} 
and classical dimer models~\cite{sreejith_prl2019}. System size 
restrictions however can make interpretation of emergent symmetry tricky. 
From the theory side, conformal
bootstrap results~\cite{Nakayama_Ohtsuki_PRL_2016}
pointed out strong constraints on the scaling dimensions
that apparently rule out emergent symmetry in $2+1d$ DQC.
The notion of DQC as pseudocriticality was thus conjectured~\cite{Ma_Wang_PRB_2020,Nahum_PRB_2020} 
in order to reconcile with the conformal bootstrap result.
The idea is that the renormalization group (RG) 
fixed point for DQC does not reside
in the (physical) $2+1d$, but slightly below it 
(see Fig.~1 of Ref.~\cite{Ma_Wang_PRB_2020}) such
that pseudocritically slow flows obtain near the critical point
$g_c$. 
For more discussion on these issues, see this recent
review~\cite{Senthil_review_2023}.
Such slow RG flows may account for 
the good scaling collapse seen in numerical data for 
accessible system sizes \emph{along} with the observed
drifts in the critical exponents.
Recent quantum entanglement based 
results~\cite{Zhao_etal_PRL2022,yuan_etal_arxiv2023,Song_etal_Arxiv2023} 
have taken forward a similar line of argument for $SU(N)$ models
in general,
however another work~\cite{Demidio2024entanglement} 
provides a counterargument.

%%% FIG 5 : Binder minimum
% Figure environment removed


It is noteworthy that even before the emergent symmetry point
of view gained currency, numerical studies of DQC had seen
anomalous scaling corrections whose origin was unclear. 
Ref.~\onlinecite{Bartosch_PRB_2013} gave an explanation based on scaling corrections
inherent in the effective $U(1)$ gauge theory of the 
deconfined spinons.
Another explanation based on two different 
length scales
associated with spinon correlations and VBS domain wall sizes
diverging with different exponents
has also been proposed~\cite{Shao_Guo_Sandvik_Science2016}.
In the context of our interpretation, pseudocriticality 
indicates at the confinement of the putative spinons
of the $U(1)$ gauge theory framework for all values of the tuning
parameter as mentioned previously. 
Near the transition, the confinement length scale 
must become very large
($L \gtrsim O(100)$ given the high quality of scaling collapses
seen) compared to the lattice scale~\cite{Sandvik_youtube_2022} but
remain finite due to
pseudocritical nature of the RG flows.
The present consensus seems to be veering towards 
deconfined pseudocriticality
based on recent $S=\frac{1}{2}$ results, and our work
provides a microscopic $S=1$ model for this
scenario. 
This opens a question in the context of $SU(N)$ DQC. 
On one hand, theoretical expectations based on the suppression
of gauge fluctuations with increasing $N$ make the case for
bonafide DQC. On the other hand, no negative Binder dips
has been numerically seen in $SU(2)$ $JQ$ models for 
largest sizes studied.
This makes the pattern of Binder dip growth with respect to
$N$ mysteriously non-monotonic~\cite{fig1_Kaulprb2011_footnote}.
It will be an useful theoretical advance and 
strong evidence for the pseudocriticality scenario if the 
sub-extensive growth of negative Binder ratio dips can be 
linked to the pseudocritcal RG
flows of Refs.~\cite{Nahum_PRB_2020,Ma_Wang_PRB_2020}.



\prlsec{Acknowledgements}
We acknowledge useful discussions with Fabien Alet, Subhro Bhattacharjee,
Kedar Damle, Prashant Kumar and Adam Nahum.
VVi was supported by the institute post-doctoral
fellowship program at IIT Bombay,
and in part by the International Centre for Theoretical Sciences (ICTS) 
during the program - 8th Indian Statistical Physics 
Community Meeting (code: ICTS/ISPCM2023/02).
SP acknowledges funding support from SERB-DST, India 
via Grants No. SRG/2019/001419 and MTR/2022/000386.
Partial support by Grant No. CRG/2021/003024 is also
acknowledged.
ND was initially supported by National Postdoctoral Fellowship of 
SERB, DST, Govt. of India (PDF/2020/001658) at the department of 
Theoretical Physics, TIFR and presently by the TIFR postdoctoral 
fellowship.
The numerical results were obtained using
the computational facilities of the Department of Physics,
Indian Institute of Technology (IIT) Bombay.















\bibliographystyle{apsrev}
\bibliography{Seq1_refs}


%\section{Appendix}

\end{document}
