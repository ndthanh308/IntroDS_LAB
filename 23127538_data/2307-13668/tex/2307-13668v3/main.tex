\documentclass[prx,twocolumn,superscriptaddress,showpacs,notitlepage,floatfix,bibnotes,nofootinbib]{revtex4-2}
\usepackage{amsmath,amsthm,amsfonts,amssymb}
\usepackage{graphicx}
\usepackage{floatrow}
\usepackage[caption=false]{subfig}

\usepackage{bm}
\usepackage{float}
\usepackage{xcolor}
\usepackage[pdfencoding=auto, psdextra]{hyperref}
\hypersetup{linktocpage}
\hypersetup{colorlinks=true,citecolor=blue,linkcolor=blue, urlcolor=blue}

\makeatletter %No subsubsections in TOC
\def\l@subsubsection#1#2{}
\makeatother
\newcommand{\ZZ}{\mathbb{Z}}
\newcommand{\ra}{\rangle}
\newcommand{\la}{\langle}
\newcommand{\PC}{\mathcal{P}}

\newcounter{NoTableEntry}
\renewcommand*{\theNoTableEntry}{NTE-\the\value{NoTableEntry}}


\begin{document}
\title{Engineering Floquet codes by rewinding}


\author{Arpit Dua}
\email{adua@caltech.edu}
\affiliation{Department of Physics, California Institute of Technology, Pasadena, CA 91125, USA}
\affiliation{Institute for Quantum Information and Matter, California Institute of Technology, Pasadena, California 91125, USA}
\author{Nathanan Tantivasadakarn}
\affiliation{Walter Burke Institute for Theoretical Physics, California Institute of Technology, Pasadena, CA 91125, USA}
\affiliation{Department of Physics, California Institute of Technology, Pasadena, CA 91125, USA}
\affiliation{Department of Physics, Harvard University, Cambridge, MA 02138, USA}
\author{Joseph Sullivan}
\affiliation{Stewart Blusson Quantum Matter Institute, University of British Columbia, Vancouver, BC, Canada V6T 1Z1}
\affiliation{Department of Physics, Yale University, New Haven, CT 06511, USA}
\author{Tyler~D. Ellison}
\email{tyler.ellison@yale.edu}
\affiliation{Department of Physics, Yale University, New Haven, CT 06511, USA}

\begin{abstract}
Floquet codes are a novel class of quantum error-correcting codes with dynamically generated logical qubits, which arise from a periodic schedule of non-commuting measurements. We engineer new examples of Floquet codes with measurement schedules that \textit{rewind} during each period. The rewinding schedules are advantageous in our constructions for both obtaining a desired set of instantaneous stabilizer groups and for constructing boundaries. Our first example is a Floquet code that has instantaneous stabilizer groups that are equivalent -- via finite-depth circuits -- to the 2D color code and exhibits a $\mathbb{Z}_3$ automorphism of the logical operators. Our second example is a Floquet code with instantaneous stabilizer codes that have the same topological order as the 3D toric code(s). 
This Floquet code exhibits a splitting of the topological order of the 3D toric code under the associated sequence of measurements i.e., an instantaneous stabilizer group of a single copy of 3D toric code in one round transforms into an instantaneous stabilizer group of two copies of 3D toric codes up to nonlocal stabilizers, in the following round. We further construct boundaries for this 3D code and argue that stacking it with two copies of 3D subsystem toric code allows for a transversal implementation of the logical non-Clifford $CCZ$ gate.  We also show that the coupled-layer construction of the X-cube Floquet code can be modified by a rewinding schedule such that each of the instantaneous stabilizer codes is finite-depth-equivalent to the X-cube model up to toric codes; the X-cube Floquet code exhibits a splitting of the X-cube model into a copy of the X-cube model and toric codes under the measurement sequence. Our final example is a generalization of the honeycomb code to 3D, which has instantaneous stabilizer codes with the same topological order as the 3D fermionic toric code. 




\end{abstract}

\maketitle
\date{}

\tableofcontents

\section{Introduction}

Quantum error-correcting codes are a key ingredient for fault-tolerant quantum computation.
There is an active effort to develop new error-correcting codes with better code properties, such as encoding rate, code distance, and circuit-level thresholds. 
For every such error-correcting code, there is an associated quantum dynamics involving quantum gates, errors, and the repeated extraction of the syndrome for decoding. Naturally, the goal 
of developing new error-correcting codes
is to optimize the quantum dynamics to reduce overheads and minimize the noise on the logical information. 

Recently, Hastings and Haah introduced the first example of what has emerged as a new class of codes, dubbed Floquet codes, which exhibit dynamically generated logical qubits~\cite{HH_dynamic_2021}. In their example, the dynamics of the system is governed by a periodic sequence of non-commuting 2-qubit Pauli measurements and exhibit instantaneous stabilizer codes that are a sequence of topological quantum error-correcting codes. Importantly, the schedule is such that the logical information is preserved from one instantaneous code space to the next.  
Given the low-weight parity checks needed to operate the code and its relatively high error threshold~\cite{Gidney2021faulttolerant, Gidney2022benchmarkingplanar, Paetznick2023Performance}, Floquet codes offer compelling alternatives to the surface code. The current understanding of Floquet codes is still under active development, thus, underscoring the importance of introducing new examples and formalizing the tools to develop new Floquet codes.

In this paper, we introduce three new examples of topological Floquet codes, referred to as:
{(1)} the Floquet color code, {(2)} the 3D Floquet toric code (TC), and {(3)} the 3D Floquet fermionic toric code (fTC). An essential tool employed in our constructions is the concept of ``rewinding'' a measurement schedule, where at some point within a period, the sequence of measurements is reversed. A similar strategy was used in Ref.~\cite{Haah2022boundarieshoneycomb} to adapt the honeycomb code to a system with boundary. We further use rewinding in this work to ensure that the ISGs are characterized by a desired topological order and to avoid inadvertently measuring logical operators. 


\textbf{(1)} Our first example is the Floquet color code, which exhibits instantaneous stabilizer groups (ISGs) that are equivalent under a finite-depth local quantum circuit with ancilla (FDLQC-equivalent) to the 2D color code. This should not be confused with the CSS honeycomb code of Refs.~\cite{Davydova2022,Kesselring2022condensation,bombin2023unifying}, which also has been referred to as the Floquet color code in Ref.~\cite{Kesselring2022condensation}\footnote{In contrast to our Floquet color code, the ISGs of the CSS honeycomb code are FDLQC-equivalent to the 2D TC.}.  We consider two different measurement schedules for the Floquet color code. The first exhibits an order three automorphism of the logical operators, while the second is a rewound version with a trivial automorphism -- in principle enabling the construction of boundaries. Notably, one of the ISGs is equivalent to the conventional color code up to concatenation with a 3-qubit repetition code -- thus allowing for the transversal implementation of certain logical Clifford gates. 

\textbf{(2)} Our second example is the 3D Floquet TC, which has ISGs that are FDLQC-equivalent to the usual 3D TC (or two copies of it). Our construction is inspired by the coupled layer construction of Ref.~\cite{Ma_2017}, in the sense that we prepare an instantaneous state of stacks of 2D TCs along orthogonal directions, then perform measurements that ``condense''  pairs of $e$ anyons 
along the intersection of two orthogonal layers. In the subsequent rounds of measurements, the stabilizers responsible for the condensation generically evolve into higher-weight stabilizers. By making use of a schedule that rewinds, we ensure that not all evolved condensation operators are non-local stabilizers, and that in turn ensures that the system does not evolve back into a stack of 2D TCs. This could be useful for specialized decoding tasks such as single-shot decoding of loop-like excitations in the 3D toric code ISGs. We note that Ref.~\cite{bauer2023topological} presents a construction of a 3D Floquet toric code using the path-integral framework. 

Due to the evolution of condensation terms into higher-weight operators, we observe an exotic splitting of the 3D TC ISG obtained in the preceding round to two copies of 3D TC up to nonlocal stabilizers. We also construct a planar variant of the 3D Floquet TC which has boundaries that condense point-like or loop-like excitations for each ISG. Unlike the planar variant of the 2D Floquet TC, the boundaries of the 3D Floquet TC do not undergo an automorphism. 
By stacking this planar variant of the 3D Floquet TC with two copies of the planar variant of ``subsystem toric code''~\cite{Kubica_2022} (with 3-qubit checks), we can prepare an instantaneous state of cubic lattice 3D TC stacked with two copies of 3D checkerboard lattice TC.  
Such a stacked code allows for an implementation of the logical non-Clifford $CCZ$ gate. ~\cite{Vasmer2019transversal}. 

We note that the construction of the X-cube Floquet code in Ref.~\cite{XcubeFloquet_2022} also uses a coupled layer construction. However, in their construction, the ISGs evolve back into decoupled layers of 2D TCs (up to nonlocal stabilizers). We show in Sec.~\ref{sec:Xcube} that this can be avoided by rewinding the schedule. We also work out explicit FDLQCs to map the ISGs to the X-cube model up to 3D toric code and stacks of 2D toric codes. This in turn leads to a result of the splitting of topological order in the X-cube Floquet code under the sequence of measurements: the X-cube model ISG in one round splits into a copy of the X-cube model and toric codes in the subsequent rounds. 



\textbf{(3)} Lastly, we construct a Floquet code with ISGs that are FDLQC-equivalent to the 3D fermionic TC, i.e., with the same topological order as a 3D $\ZZ_2$ gauge theory with an emergent fermion~\cite{Levin2003Fermions}. Our construction is based on the 3D generalization of Kitaev's honeycomb model of Ref.~\cite{Mandal_2009}. In this construction, we use {rewinding} to avoid inadvertently measuring logical operators throughout the schedule. 

% We also suggest a rewinding schedule for the 3D fermionic TC using the coupled layer construction. However, in this construction, decoupled stacks of 2D TCs (up to nonlocal stabilizers) appear as an intermediate ISG.   
% \begin{center}
% \rule{0.5\columnwidth}{0.6pt}
% \end{center}

Our work establishes new examples of Floquet codes and formalizes rewinding as a tool for designing Floquet codes with beneficial code properties, such as desired ISGs which could be relevant for decoding and transversal logical gates. We expect that constructing examples such as these will be an important step towards developing a more comprehensive classification and characterization of Floquet codes. 

The paper is organized as follows. In Sec.~\ref{sec:3DbosonicFloquet}, we review the Floquet code of Ref.~\cite{Paetznick2023Performance}, referred to as the Floquet TC, and state general properties of rewinding schedules -- such as trivial logical automorphisms after a measurement cycle. In Secs.~\ref{sec:Floquetcolorcode}, \ref{sec:3DbosonicFloquet}, and \ref{sec:3DfermionicFloquet}, we define the Floquet color code, the 3D Floquet TC, and the 3D Floquet fTC, respectively.



\section{Rewinding Floquet code schedules}
\label{sec:review}

In this section, we formalize the notion of a rewinding measurement schedule for a Floquet code. We argue that a Floquet code with a rewinding schedule exhibits a trivial automorphism of the logical operators after a single period and that the measurement quantum cellular automata (MQCAs) defined at the boundary have a trivial index~\cite{aasen2023measurement}. We begin with some preliminary definitions.  

\subsubsection{Preliminaries}

In general, a Floquet code is defined by two pieces of data: (i) a set of operators that are measured throughout the dynamics, known as the check operators (or more simply, the checks), and (ii) a periodic measurement schedule, which dictates when the check operators should be measured. We find it convenient to further define the check group as the group generated by the checks. We call the center of the check group, i.e., the subgroup of operators in the check group which commute with every element of the check group, the stabilizer group of the check group. We note that the stabilizers can be interpreted as conserved quantities of the dynamics since their measurement outcomes are not affected by the measurement of the checks.


We say that a measurement schedule rewinds, or that the Floquet code is a rewinding Floquet code, if the sequence of measurements are performed in reverse order at some point in the schedule. For example, we consider a set of measurements labeled 0, 1, and 2. If the measurements are performed periodically in the sequence 012021, as in Ref.~\cite{Haah2022boundarieshoneycomb}, then the schedule is in fact rewinding. This can be seen by writing a few periods of the schedule as: 
\begin{align} 
    \ldots0120\text{-}0210\text{-}0120\text{-}0210\text{-}0120\text{-}0210\ldots,
\end{align}
where we have written a single 0-round with repeated 0s i.e., as 0-0 to explicitly demonstrate the rewinding. The sequence 0120 is then explicitly followed by the reverse sequence 0210.
In our examples below, we find that rewinding is a useful tool for constructing boundaries of Floquet codes, as acknowledged in Refs.~\cite{Haah2022boundarieshoneycomb,Gidney2022benchmarkingplanar,Paetznick2023Performance} and for obtaining a desired set of ISGs. This may in turn be useful for developing Floquet codes with beneficial decoding properties and transversal implementations of logical gates. 

\subsubsection{Trivial automorphism and MQCA index}
\label{subsec:trivial_automorph_rewinding}

We now state the general properties of such rewinding schedules -- specifically, we emphasize that there is a trivial automorphism of the logical operators, implying that a planar variant of the rewinding Floquet code defines a measurement quantum cellular automata at its boundary with a trivial index \cite{aasen2023measurement, sullivan2023floquet}.

Consider a Floquet code with a rewinding schedule of form 012021 on a torus. We assume that this Floquet code is reversible in the sense of Ref.~\cite{aasen2023measurement}. This means that for each consecutive pair of ISGs, there exist a complete set of shared logical operator representations for every logical operator. That is, there is a complete set of operators that commute with both stabilizer groups. In between rounds, we must update the logical operators by multiplying instantaneous stabilizers of the current round, to obtain a logical that also commutes with the ISG of the next round. Intuitively speaking, the rewinding ensures that any instantaneous stabilizers that were multiplied to the logical operators are removed when the schedule is run backwards. Thus, we arrive back at the same logical representative after the full rewound cycle. 

To see this explicitly, we denote the representations of a logical operator that are shared by ISGs of consecutive rounds $r$ and $r'$ as $\text{LOR}(r,r')$ where round $r'$ follows round $r$. For the 6-round schedule 012021, the logical operator representations of rounds 0 and 1 can be chosen to be LOR(0,1). To evolve to round 2, we can multiply by instantaneous stabilizers of the 1-ISG to obtain LOR(1,2). To go to the second instance of round 0,  we can multiply by instantaneous stabilizers of the 2-ISG to obtain LOR(2,0). We now begin the rewinding process. $\text{LOR}(2,0) = \text{LOR}(0,2)$ is a valid representation for the second instance of round 2. The next rounds are 1 and 0 so we multiply by the same elements of the 2-ISG to obtain LOR(2,1) and the same elements of the 1-ISG to obtain $\text{LOR}(1,0)=\text{LOR}(0,1)$. The net result is that we are back to the original representation LOR(0,1) that we started with. Hence, the automorphism is trivial. Later in Sec.~\ref{sec:square-octagon_RGBRBGschedule}, we explicitly show the shared logical operators for the rewinding sequence of the 2D Floquet TC in Fig.~\ref{fig:square-octagonsubsystemcode}.


The argument that the logical operators “rewind” also implies that the MQCA for the full period, as defined in Ref.~\cite{aasen2023measurement}, is trivial. If the Floquet code is put on a system with a boundary, one can consider the evolution of the logical operators at the boundary as defining an MQCA. Then, since the automorphism of logical operator representations is trivial for the full period, the MQCA is also trivial for the full period. We note however that the MQCA is nontrivial for half a period of the rewinding schedule for the 2D Floquet toric code~\cite{aasen2023measurement}.


\section{2D Floquet color code} \label{sec:Floquetcolorcode}

In this section, we construct the 2D Floquet color code, whose ISGs are FDLQC-equivalent to the 2D color code. We then discuss the $\mathbb{Z}_3$ automorphism of logical operators and the construction of a rewinding schedule that trivializes the automorphism. 

The Floquet color code is based on the topological subsystem code of Ref.~\cite{Bombin2010Subsystem}, which is defined 
on the ruby lattice with a qubit at each vertex, as depicted in Fig.~\ref{fig:colorcode_schedule_automorphism}(a). The group of check operators is generated by two-qubit operators, specified by assigning a label $x$, $y$, or $z$ to each edge of the lattice. The corresponding check operators on the $x$-, $y$-, and $z$-edges are the two-qubit Pauli operators, $XX$, $YY$, and $ZZ$, respectively.\footnote{Through out this paper, we omit the tensor product between the Pauli operators for simplicity of notation. For example, we write $X\otimes X$ as $XX$.}  

The stabilizers of the check group are generated by two types of operators, both of which are supported on each ``inflated'' hexagon of the ruby lattice, as shown in Fig.~\ref{fig:colorcode_schedule_automorphism}(a). We call the generator that is a product of $Z$ operators around a hexagon, the hexagon stabilizer, and the generator that is a product of $X$ and $Y$ operators the inflated hexagon stabilizer. 







% Figure environment removed
\subsection{3-round measurement schedule}
The measurement schedule of the Floquet color code consists of three rounds, labeled by 0, 1, and 2, with the checks of each round shown in Fig.~\ref{fig:colorcode_schedule_automorphism}(b). We note that this schedule is also given in Ref.~\cite{Sarvepalli_2012} in the context of the topological subsystem code.\footnote{Although the schedule has appeared before in the literature, here, we view the code as a Floquet code. The important difference is that the Floquet code has dynamically generated logical qubits in addition to the logical qubits of the subsystem code.} This schedule ensures that the stabilizers of the check group are inferred once every cycle. 

In each round, the ISG is FDLQC-equivalent to the color code. We demonstrate this for round 2, by showing that the ISG is precisely the color code concatenated with the 3-qubit repetition code. 
In round 2, the measured checks consist of $ZZ$ operators on every edge of the triangles. Since there are three qubits and two independent checks per triangle, an effective qubit (3-qubit repetition code) lives on each triangle with effective Pauli operators
\begin{align}
  X_{\text{eff}} &= XYY \equiv XXX &   Z_{\text{eff}} = ZII \equiv ZZZ
  \label{eq:concatcolorcode}
\end{align}
where $\equiv$ is the equivalence up to the 2-qubit $ZZ$ checks. In terms of the effective Pauli operators, the instantaneous stabilizers reduce to those of the color code. Therefore, the 2-round ISG is the color code up to concatenation with 3-qubit repetition codes. For rounds 0 and 1, the FDLQCs that map the ISGs to the color code are given in the \texttt{Mathematica} file.


As an aside, the logical gates that can be implemented transversally for the (CSS) color code~\cite{Yoshida_2015,Kubica_2015}, can still be implemented transversally in round 2 when the ISG is the color code up to concatenation with 3-qubit repetition codes. If we label the four logical qubits 1 through 4, then the transversal logical gates for a single copy of color code (in particular choice of basis) are $\overline{\text{CNOT}_{12}\text{CNOT}_{34}}$ and $\overline{\text{SWAP}_{12}\text{SWAP}_{34}}$.
To realize the former, we act with $S_{\text{eff}} = SII$ on one orientation of triangles (say, left-pointing) and $S^\dagger_{\text{eff}} = S^\dagger II$ on the other orientation (right-pointing) after the second round of measurements. Similarly, we can act with an effective Hadamard gate $H_{\text{eff}} = (X_{\text{eff}}+ Z_{\text{eff}})/\sqrt{2}$ on all triangles to realize the latter logical gate.

Since the logical information is preserved under the Floquet code dynamics, these gates can be done in 2-round and the processed logical information carries over to subsequent rounds. We note that a planar realization of the color code would allow for a transversal implementation of all logical Clifford gates~\cite{bombin2015gauge,Kubica_2015}. We leave the construction of a planar variant of our Floquet color code to future work. 











\subsection{$\mathbb{Z}_3$ automorphism of logical operators}
Given that the ISGs are FDLQC-equivalent to the color code, the Floquet color code encodes four logical qubits on a torus. This implies that the Floquet code has two dynamically generated logical qubits, since the topological subsystem code of Ref.~\cite{Bombin2010Subsystem} hosts only two logical qubits. We refer to the logical operators of the subsystem code as the static logical operators and the logical operators that are not static as the dynamically generated logical operators~\footnote{The static subsystem code is characterized by the 3-fermion anyon theory~\cite{Bombin2009fermions,Bombin2012universal,Bombin2014structure, Ellison2022subsystem}, thus, the dynamically generated logical operators are associated with a second copy of the 3-fermion theory. The two 3-fermion theories together are equivalent to the color code.}. Indeed, since the static logical operators have representations that commute with all of the check operators, they do not evolve under the dynamics. As discussed below, there is a nontrivial automorphism of the dynamically generated logical operators after a full period. This automorphism happens through multiplication with dynamically generated logical operators that belong to the check group but do not commute with all the elements of the check group. This is in contrast to the 2D Floquet TC, where the automorphism happens due to multiplication with a dynamically generated logical operator that belongs to the center of the check group and has been referred to as the inner logical in Ref.~\cite{HH_dynamic_2021}.


For the 3-round Floquet cycle, the dynamically generated logical operators exhibit a $\mathbb{Z}_3$ automorphism, as shown in Fig.~\ref{fig:colorcode_schedule_automorphism}(d). The logical operator representation in each round $r$ commutes with the ISGs of rounds $r$ and $r+1$. The logical operator representation in each round $r$ is related to the one in the preceding round $r-1$ via multiplication with the ISG stabilizers of round $r$. Besides the checks measured in round $r$, we used the instantaneous square stabilizers shown in  Fig.~\ref{fig:colorcode_schedule_automorphism}(c) to work out the representations. 
Since there are no nonlocal stabilizer generators of the check group, the automorphism occurs due to multiplication with products of dynamically generated logical operators belonging to the check group. 
Due to the $\mathbb{Z}_3$ automorphism that occurs every 3 rounds, it takes 9 rounds for a trivial automorphism. 

    
To see the automorphism explicitly, we can apply the mapping to effective qubits in round 2 (Eq.~\eqref{eq:concatcolorcode}) to the string operators shown in Fig.~\ref{fig:colorcode_schedule_automorphism}(d). In the first instance of round 2, we find that the logical consists of $Z_{\text{eff}} =ZII$ on each triangle connecting the red plaquettes. If we truncate such a string in the effective color code pictures, the $X$ and $Y$ red plaquettes of the effective color code will be violated. In the second instance of round 2, we have a string of $X_{\text{eff}} = XYY$ on each triangle connecting blue plaquettes. Thus, truncating such a string will violate the $Y$ and $Z$ blue plaquettes of the effective color code. Lastly, in the third instance of round 2, the strings are a product of $Y_{\text{eff}} =XXY$ connecting green plaquettes. Truncating these strings will violate the $X$ and $Z$ green plaquettes of the effective color code.



\subsection{6-round rewinding schedule}
Instead of trivializing the automorphism through 9 measurement rounds, we can use the 6-round rewinding schedule 012102, in which the $\mathbb{Z}_3$ automorphism action after 012 is rewound back. The evolution of logical operator representations is the same as shown in Fig.~\ref{fig:colorcode_schedule_automorphism}(c) for any three consecutive rounds with 0-, 1- and 2-checks. In principle, this rewinding schedule enables the construction of the Floquet color code on a triangular geometry, such as in Refs.~\cite{bombin2015gauge,Kubica_2015}. 
In the appendix, we write a different 6-round schedule for the Floquet color code, which exhibits a trivial automorphism. It would be interesting to see if that 6-round schedule is equivalent to the above rewinding schedule using some notion of equivalence, such as simple equivalence discussed in Ref.~\cite{aasen2023measurement}. 







\section{3D Floquet toric code}

\label{sec:3DbosonicFloquet}


We now present the 3D Floquet TC whose ISGs are FDLQC-equivalent to the usual 3D TC (or two copies). The construction is inspired by the coupled layer construction of the 3D TC presented in Ref.~\cite{Ma_2017}. 
More specifically, our strategy for building the 3D Floquet TC is to start with layers of 2D Floquet codes and add measurements that implement the condensation procedure of Ref.~\cite{Ma_2017} at the level of the ISGs. We use 2D Floquet codes on square-octagon lattices, which are stacked along the $x$-, $y$-, and $z$-axes, as our building blocks. 

Before going into the detailed construction of the 3D Floquet TC, we review the coupled layer construction of the 3D TC and the construction of the 2D Floquet TC with a rewinding schedule on the square octagon lattice. Readers who are familiar with the details of the coupled layer construction of 3D TC and the 2D Floquet TC are welcome to skip directly to Sec.~\ref{subsec:rewinding_schedule_3D_Floquet_bTC}.


% Figure environment removed

% Figure environment removed

% Figure environment removed
\subsection{Review: coupled layer construction}
\label{subsec:coupledlayerreview}

The coupled layer construction of the 3D TC in Ref.~\cite{Ma_2017} starts with stacks of 2D TCs, along the $x$-, $y$-, and $z$-axes, as shown in Fig.~\ref{fig:foliation_3DTC}(a). The 2D TCs here are defined on square lattices with qubits on the edges. 
Taken together, the layers of 2D TCs define a cubic lattice with two qubits on each edge. The edges parallel to the $z$-axis, for example, have one qubit from a 2D TC in a $yz$-plane and another from a 2D TC in an $xz$-plane. 

The 2D layers are then coupled together by forcing an interlayer $ZZ$ operator at each edge to be a stabilizer.\footnote{We remark that if the 2D TC layers are coupled together by an interlayer $XX$ stabilizer at each edge, then we obtain an effective X-cube model~\cite{Ma_2017}. This is because the $Z$-type stabilizers that remain in the stabilizer group are the products of 2D TC plaquette stabilizers around a cube, corresponding to the cube term of the X-cube model. This is used to build the X-cube Floquet code in Ref.~\cite{XcubeFloquet_2022}.
} This requires removing the 2D TC stabilizers that fail to commute with the $ZZ$ operators. Note that the vertex terms of the 2D TCs do not commute with the interlayer $ZZ$ checks, but the product of three vertex stabilizers, one from each intersecting plane, does commute. Therefore, this product of $X$-type stabilizers remains in the stabilizer group.

Operationally, the $ZZ$ stabilizers define a single effective qubit at each edge. The effective stabilizer group is then equivalent to the usual 3D TC up to concatenation with two-qubit repetition codes; the product of three vertex stabilizers becomes the vertex stabilizer on the cubic lattice and the plaquette interlayer $ZZ$ checks, the logical operators of the 3D TC can be represented by $e$ string operators along non-contractible paths and membranes built from stacks of $m$ string operators. 

Intuitively, the interlayer $ZZ$ operators create pairs of $e$ anyons. By adding the $ZZ$ operators to the stabilizer group, we have condensed pairs of $e$ anyons from intersecting 2D TCs. Along the $z$-axis, for example, the $ZZ$ operator creates a pair of $e$ anyons with one from the $yz$-plane and the other from the $xz$-plane. Heuristically, the condensation of interlayer pairs of $e$ anyons implies that the $e$ anyons can transfer without any energy cost between layers at an intersection, while the $m$ anyons in individual layers become confined.  


% Figure environment removed


\subsection{Review: 2D Floquet toric code}
\label{sec:square-octagon_RGBRBGschedule}
We now review the 2D Floquet toric code, which is essential to our 3D construction. 
Following Ref.~\cite{Paetznick2023Performance}, the 2D Floquet TC can be defined on a two-dimensional square-octagon lattice with periodic boundary conditions and a qubit at each vertex. We color the edges of the lattice red, green, and blue, as shown in Fig.~\ref{fig:square-octagonsubsystemcode}. The red, green, and blue colors determine the 2-qubit Pauli check operators $XX$, $YY$, or $ZZ$, respectively, associated to an edge.
We refer to the checks on the red, green, and blue edges as R-, G-, and B-checks, respectively. The stabilizers of the check group are generated by three types of operators, supported on either a square or an octagon, as shown in Fig.~\ref{fig:square-octagonsubsystemcode}(a). We refer to these as the square stabilizers and the octagon stabilizers.

To initialize the code, we measure the checks RBGR in sequence. Subsequently, in each period, we measure the sequence GBRBGR. As it is essential in the construction of the 3D Floquet TC, we note that this schedule is rewinding, so the logical automorphism is trivial under the full period. We also note that this is the schedule used in Refs.~\cite{Gidney2022benchmarkingplanar} and \cite{Paetznick2023Performance} to define the 2D Floquet TC on a system with a boundary.

The ISGs obtained upon measuring the R-, G-, and B-checks are referred to as the R-, G-, and B-ISGs respectively. These ISGs are generated by the stabilizers of the check group and the check operators measured in the round. The G-ISG, for example, is generated by the square and octagon stabilizers as well as the two-qubit $YY$ stabilizers on the green edges.

The G-ISG is precisely the 2D TC on the square lattice concatenated with a 2-qubit repetition code on the green edges. More specifically, the 2-qubit repetition code on each green edge is defined by a $YY$ stabilizer and logical operators 
\begin{align} \label{eq:GISGrepetition}
    X_{\text{eff}}&=XX\equiv ZZ, & Z_{\text{eff}}&=YI\equiv IY.
\end{align}
That is, in the subspace where all the green checks $YY$ are satisfied, we may define an effective qubit on each edge according to the above equation. In this subspace, the octagon stabilizers can be recast as products of $X_{\text{eff}}$'s on four effective qubits, allowing them to be interpreted as the vertex terms of the square lattice 2D TC. Similarly, the square stabilizers can be written as a product of $Z_{\text{eff}}$'s on four effective qubits and hence correspond to the plaquette terms of the 2D TC.
We define the violations of vertex and plaquette stabilizers to be $e$ anyons and $m$ anyons, respectively.
The $e$ anyons are created by the application of $Z_{\text{eff}}$ on the green edges and the $m$ anyons are created by the application of $X_{\text{eff}}$ on the green edges. 

The R- and B-ISGs are precisely the toric code on a rotated lattice up to concatenation with a 4-qubit code. Since the B- and R-ISGs are symmetric, we make this explicit for the B-ISG.
Each square of the lattice consists of four qubits and three instantaneous stabilizers, i.e., one $YYYY$ stabilizer and an additional two $ZZ$ stabilizers from the measurements of the B-checks. Due to these stabilizers, an effective qubit can be defined on each square with the following effective (logical) operators,
\begin{align}
    X_{\text{eff}}&=ZIZI, & Z_{\text{eff}}&=XXII,
\end{align}
where the first two and last two qubits come from the two blue edges on the square respectively. The above operators commute with the three instantaneous stabilizers on the square. In other words, each square supports a [[4,1,2]] code. 
In the logical subspace of the [[4,1,2]], the octagon stabilizers reduce to a product $X_{\text{eff}}^{\otimes 4}$ or $Z_{\text{eff}}^{\otimes 4}$ depending on the sublattice, as shown in Fig.~\ref{fig:effective_toric_codes_2DFloquet}. Thus, on the effective qubits defined by the [[4,1,2]] code, we have a 2D rotated TC.













\subsection{6-round rewinding schedule}
\label{subsec:rewinding_schedule_3D_Floquet_bTC}

To build the 3D Floquet TC, we start with layers of 2D Floquet TCs on square-octagon lattices, which are stacked along the $x$-, $y$-, and $z$-axes. The stacks of square-octagon lattices define a 3D truncated cubic lattice with two qubits at each vertex, see also Ref.~\cite{XcubeFloquet_2022}. For clarity, we resolve the vertices in Fig.~\ref{fig:foliation_3DTC2} to show the connectivity of each square-octagon foliation. 

The check group of the 3D Floquet TC is generated by the check operators of the 2D Floquet TCs and 2-qubit interlayer $YY$ check operators that couple the layers together as in Fig.~\ref{fig:foliation_3DTC2}. There are three interlayer $YY$ checks for each octahedron of the truncated cubic lattice. These correspond to the three ways of pairing the $xy$-, $yz$-, and $xz$-planes. The stabilizers of the check group are generated by the square stabilizers of the 2D Floquet TCs and the product of three octagon stabilizers sharing a truncated cube, as shown in Fig.~\ref{fig:static_stabilizers_3D_Floquet}.


% Figure environment removed




\begin{table}[]
    \centering
    \begin{tabular}{|c|l|c|}
    \hline
Round   & Measurement  & ISG\\
\hline
0& G (Green $YY$+ interlayer $YY$) & 3D TC\\
1& B (Blue $ZZ$) & 3D TC $\times$ 3D TC\\
2& R (Red $XX$) & 3D TC $\times$ 3D TC\\
3& B (Blue $ZZ$) & 3D TC $\times$ 3D TC\\
4& G (Green $YY$+ interlayer $YY$) & 3D TC \\
5& R (Red $XX$) & 3D TC $\times$ 3D TC\\
\hline  
\end{tabular}
\caption{The GBRBGR schedule of measurements for the 3D Floquet TC. The measured checks and the instantaneous stabilizer groups (ISGs) in each round are written. We note that in the ISG with two copies of 3D TC, the two copies are in an entangled logical state due to nonlocal stabilizers of the form $\overline{Z}_{1,i}\overline{Z}_{2,i}$ where $\overline{Z}_{1,i}$ and $\overline{Z}_{2,i}$ are logical string operators of the two 3D TCs along nontrivial cycles $i$ of the associated 3D tori.}
\label{tab:3DTCschedule}
\end{table}


% Figure environment removed
% Figure environment removed

% Figure environment removed

% Figure environment removed
To initialize the 3D Floquet TC, we first initialize the stack of 2D Floquet codes with the four-round measurement sequence: RBGR. At this point, the ISG is equivalent to a stack of 2D TCs, up to concatenation with two-qubit repetition codes. We then measure the G-checks and the interlayer $YY$ checks simultaneously (round 0 listed in Table~\ref{tab:3DTCschedule}). In terms of the effective Pauli operators $X_\text{eff}$ and $Z_\text{eff}$ in the G-ISG of the 2D Floquet TC [Eq.~\eqref{eq:GISGrepetition}], the interlayer $YY$ checks are precisely the $Z_\text{eff}Z_\text{eff}$ terms of the coupled layer construction as described in Sec.~\ref{subsec:coupledlayerreview} and Fig.~\ref{fig:foliation_3DTC}. Hence, the interlayer $YY$ checks create pairs of $e$ anyons on intersecting layers. Therefore, the interlayer $YY$ checks condense pairs of $e$ anyons, and after measuring the $YY$ checks, the ISG is the 3D TC, up to concatenation with 2-qubit repetition codes on the green edges. 

We continue with a periodic measurement schedule of the checks, according to the sequence: GBRBGR, where the G round implicitly includes the interlayer $YY$ checks. We note that the interlayer $YY$ checks fail to commute with the subsequent B- and R-checks, so they are removed from the B- and R-ISGs. However, products of the interlayer $YY$ checks and possibly checks measured in the preceding rounds survive as stabilizers. We refer to these new instantaneous stabilizers as the evolved condensation terms and these are shown in shown in Fig.~\ref{fig:evolution_condensation_op_3DTC}.On each octahedron of the truncated cubic lattice, only one of the three interlayer $YY$ checks (depending on the octahedron) evolves to a constant-weight stabilizer.
The other two evolve into nonlocal stabilizers. An example of a non-local evolved condensation operator is shown in Fig.~\ref{fig:nonlocalstab}.

The generators of each of the ISGs thus consist of (i) the checks measured in that round, (ii) the condensation operators or evolved condensation operators, (iii) some octagon stabilizers (which octagon terms are in ISG depends on the precise round), and (iv) the stabilizers of the check group. 
We list the topological order of the ISGs in each round of the schedule in Table~\ref{tab:3DTCschedule}. 
Explicit circuits to map the ISGs to the canonical form of the 3D TC (or two copies of it) are given in the supplementary \texttt{Mathematica} file. The counting of logical qubits is described in Appendix~\ref{subsec:counting}. 

We note that the rewinding GBRBGR schedule ensures that, in each round, there is an extensive number of constant-weight evolved condensation terms. This is not the case for the 3-round schedule RGB, in which all the condensation terms evolve into non-local stabilizers leading to an ISG of a stack of 2D TCs with some logical operators fixed as non-local stabilizers. Intuitively, rewinding the schedule prevents the possibility in which all the condensation terms are nonlocal.
We also note that the automorphism of logical operators is trivial, which follows from the rewinding property, discussed in Sec.~\ref{subsec:trivial_automorph_rewinding}. 





% Figure environment removed

% Figure environment removed

\subsubsection{Splitting into two copies of 3D toric code}

Interestingly, the B-ISGs and the R-ISGs are FDLQC-equivalent to two copies of the 3D TC, up to nonlocal stabilizers. Here, we elaborate on the mechanism by which the single 3D TC splits into two copies of the 3D TC via measurements. 
In short, the constant-weight evolved condensation terms can be interpreted as short string operators that create pairs of $e$ anyons on the next-nearest neighbor octagons of the 2D square-octagon lattices. The configuration of evolved condensation terms is such that the ISG is FDLQC-equivalent to two copies of the 3D TC with a constraint on the logical subspace given by the condensation terms that evolve into nonlocal stabilizers. 

The fact that two copies of 3D TC arise in the B-ISG is best understood in the effective picture of TC layers. In Sec.~\ref{sec:square-octagon_RGBRBGschedule}, we showed that in the B- and R-ISGs of the 2D Floquet TC, we get TCs on rotated lattices with an effective qubit on each square plaquette of the square-octagon lattice. The B-ISG of the 3D Floquet TC can hence be understood as a coupled layer construction starting from rotated 2D TC layers stacked along three orthogonal directions as shown in Fig.~\ref{fig:sublattices_BISG}. 

In the effective description of 2D rotated TC layers, the evolved condensation operators act as 4-qubit Pauli $X$ operators which create pairs of $e$-anyons across intersecting layers, with the violations of $Z$-stabilizers in the 2D rotated TCs corresponding to the $e$-anyons. The evolved condensation terms are illustrated on the effective lattice using thick red edges in Fig~\ref{fig:sublattices_BISG}(b). The fact that these evolved condensation terms belong to the stabilizer group implies that only certain
% or the existence of these operators as stabilizers implies that on each cube of the 3-foliated lattice shown in Fig.~\ref{fig:sublattices_BISG}(b), the 
products of $Z$-stabilizers of the 2D layers, as shown in  Fig.~\ref{fig:sublattices_BISG}(c), survive as stabilizers after condensation.
The violations of these products of $Z$-stabilizers correspond to the $e$-charges of the two 3D TCs. 

Given the structure of the evolved condensation terms, the $e$-charges on nearest-neighboring cubes, in fact, belong to inequivalent superselection sectors. Hence, we label them as $e_1$ and $e_2$ corresponding to the two copies of the 3D TC. This illustrates that the two 3D TCs ``live'' on two sublattices of the full cubic lattice. 


We now consider the two R-ISGs appearing in the GBRBGR schedule. We label these two consecutive R-ISGs as the $\text{R}^{(1)}$-ISG and the $\text{R}^{(2)}$-ISG. 
The key difference in the $\text{R}^{(1)}$-ISG, compared to the preceding B-ISG, is that we now have 4-qubit Pauli $Z_{\text{eff}}$ condensation operators in the effective description of 2D rotated TC layers. The effective 2D rotated TC $X_{\text{eff}}$ ($Z_{\text{eff}}$) stabilizers are also now changed to $Z_{\text{eff}}$ ($X_{\text{eff}}$) stabilizers, since the [[4,1,2]] codes on the squares now have $XX$ checks as stabilizers; see Fig.~\ref{fig:effective_toric_codes_2DFloquet}for our convention of $X_{\text{eff}}$ and $Z_{\text{eff}}$ in the R-ISGs of the 2D Floquet toric code. Thus, this R-ISG is equivalent to the B-ISG up to a basis change, and we again get two copies of 3D TC (up to nonlocal stabilizers). The full configuration of evolved condensation terms in the unit cell is shown in Fig.~\ref{fig:gapped_boundaries_3D_Floquet_bTC}(b). The nonlocal stabilizers, in this case, are the same as those of the B-ISG in Fig.~\ref{fig:nonlocalstab}, but with Pauli $Y$ operators replaced by Pauli $X$ operators. These can be obtained by multiplying the nonlocal stabilizers of the B-ISG by the B-checks.

In the $\text{R}^{(2)}$-ISG, the condensation operators are 4-qubit $X_{\text{eff}}^{\otimes 4}$ and 3-qubit $X_{\text{eff}}^{\otimes 3}$; the microscopic representations are shown in Fig.~\ref{fig:evolution_condensation_op_3DTC}(d) and the effective representations are shown in Fig.~\ref{fig:gapped_boundaries_3D_Floquet_bTC} along with the configuration of condensation operators in the unit cell. We again get two copies of 3D TCs (up to nonlocal stabilizers) as the ISG. The nonlocal stabilizers, in this case, are similar to those of B-ISG but shifted in space. 


% Figure environment removed


\subsection{Boundary construction}
\label{subsec:3dFbTC_boundaries}

We now consider a boundary construction for the 3D Floquet TC. Since the 3D Floquet TC is based on a coupled layer construction, we start by reviewing the boundaries for the 2D Floquet TC on the square-octagon lattice~\cite{Gidney2022benchmarkingplanar,Paetznick2023Performance}. We consider in particular the truncation of the square-octagon lattice shown in Fig.~\ref{fig:truncation_layers}(a). Here, any check that is truncated is included in the check group as a single-qubit check.
In the first R-ISG of the schedule GBRBGR, we get $m$-boundaries (i.e., smooth boundaries)
where the truncation cuts through the blue and red edges, and $e$-boundaries (i.e., rough boundaries) where the truncation cuts through the green edges. After half a period of the rewinding schedule, the boundaries switch types. 

We now use this construction of boundaries for the 2D Floquet code to determine the boundaries of the 3D Floquet TC. The truncation for the three foliations of square-octagon layers is specified in Fig.~\ref{fig:truncation_layers}(b). For one of the foliations, the truncation goes through only the blue and red edges while for the other two, the truncation goes through the red and blue edges on the top and bottom and green edges on the left and right sides. The resulting 3D Floquet TC is such that, for each ISG, we have charge-condensing boundaries on the left and right and the loop-condensing boundaries on the remaining four sides, as in Fig.~\ref{fig:truncation_layers}(c). 

We note that the particular choice to truncate the blue and red edges on the lower halves of the square plaquettes ensures that in the 3D Floquet TC lattice, the two-qubit condensation checks in the G-ISG are never truncated. This choice can nonetheless lead to truncated evolved condensation terms. The truncated condensation terms survive as stabilizers along the boundary with truncated green edges, while they do not survive along the boundaries with truncated blue and red edges, since they do not commute with the single-qubit blue and red checks. 

The result of these truncations is that, for each ISG, the charges (possibly two types $e_1$ and $e_2$, depending on the round) are condensed on the right and left boundaries, while the loop-like excitations are condensed on the remaining four sides. 

It is straightforward to understand this result from the perspective of the 2D TCs on the effective lattice. In the G-ISG, the $e$-charge is associated with the product of octagonal plaquettes, which also correspond to $e$-anyons of the layers. Since the truncation through green checks creates an $e$-condensing boundary for the 2D G-ISGs, the $e$-charge of the 3D code also condenses at that boundary. Similarly, the loop excitations are condensed at the truncation through the blue and red edges. 

In both of the B-ISGs, we have the effective picture of the 2D rotated TCs, and the bulk condensation operators are of the form $X_{\text{eff}}^{\otimes 4}$, which implies that the $e$-charges are associated with products of $Z_{\text{eff}}$-stabilizers. For the B-ISG of the 2D Floquet code, the truncation through green checks condenses the violations of the $Z_{\text{eff}}$-stabilizers supported on the octagonal plaquettes consisting of red and green edges. This is illustrated in Fig.~\ref{fig:boundaries_2D_Floquet_code}(c). Thus, for the B-ISG of the 3D Floquet code, at the truncation through green edges, we condense the violations of these $Z_{\text{eff}}$-stabilizers, corresponding to point-like excitations, and at the truncation through blue and red edges, we condense the violations of the $X_{\text{eff}}$-stabilizers corresponding to loop-like excitations. 

After the B-round, we have the $\text{R}^{(1)}$-ISG. For both of the R-ISGs, we can again use the effective picture of rotated toric layers along three foliations and the action of evolved condensation terms. For the 2D Floquet TC, the truncation through green edges gives an $e$- ($m$-) boundary in the $\text{R}^{(1)}$-ISG ($\text{R}^{(2)}$-ISG). This is illustrated in Fig.~\ref{fig:boundaries_2D_Floquet_code}, where the stabilizers at the boundary are shown in terms of the effective Pauli operators. The condensation operators in the $\text{R}^{(1)}$-ISG and $\text{R}^{(2)}$-ISG are given by $Z_{\text{eff}}^{\otimes 4}$ and $X_{\text{eff}}^{\otimes 4}$, respectively, and are shown in Fig.~\ref{fig:gapped_boundaries_3D_Floquet_bTC}. 
In the $\text{R}^{(1)}$-ISG, since the condensation operators are given by $Z_{\text{eff}}^{\otimes 4}$, the stabilizer products corresponding to $e$-charges ($m$-loops) are given by the $X_{\text{eff}}$ ($Z_{\text{eff}}$) stabilizers. Since in the $\text{R}^{(1)}$-ISG of the 2D Floquet code, the truncation through green edges condenses $X_{\text{eff}}$ stabilizers; see Fig.~\ref{fig:boundaries_2D_Floquet_code}. Hence, in the $\text{R}^{(1)}$-ISG of the 3D Floquet toric code, the truncation through green edges condenses the point-like excitations. Similarly, the truncation through the blue and green edges condenses the loop-like excitations of the 3D TCs.   

Similarly, in the $\text{R}^{(2)}$-ISG, the condensation operators are given by $X_{\text{eff}}^{\otimes 4}$ and $X_{\text{eff}}^{\otimes 3}$ as shown in Fig.~\ref{fig:gapped_boundaries_3D_Floquet_bTC} and the stabilizer products corresponding to $e$-charges ($m$-loops) are given by the $Z_{\text{eff}}$ ($X_{\text{eff}}$) stabilizers. As shown in Fig.~\ref{fig:boundaries_2D_Floquet_code}, the truncation through green edges condenses the excitations of the $Z_{\text{eff}}$-stabilizers in the $\text{R}^{(2)}$-ISG. Hence, in the $\text{R}^{(2)}$-ISG of the 3D Floquet toric code, the truncation through green edges again condenses the point-like excitations, and the truncation through the blue and green edges condenses the loop-like excitations of the 3D TCs. 
To conclude, even though the 2D Floquet code layers undergo a boundary transformation from $e$-type to $m$-type, the 3D Floquet TC boundaries of given types condense the same type of excitations in each ISG respectively. 

\subsection{Transversal non-Clifford gate}

One key computational advantage of the 3D TC is that it allows for an implementation of the transversal logical non-Clifford gate~\cite{Vasmer2019transversal}. Such an advantage is retained for our 3D Floquet TC. This is because the 3D TC in the G-ISG is precisely the conventional cubic lattice 3D TC up to concatenation with a 2-qubit repetition code. We can stack the 3D Floquet TC with two copies of the 3D TC on the checkerboard lattice to yield the transversal $CCZ$ gate as an on-site symmetry of the stabilizer group~\cite{Vasmer2019transversal}. More specifically, on a system with both the 3D Floquet TC and two 3D checkerboard lattice TCs, we can perform the transversal logical $CCZ$ gate in the round of the Floquet cycle in the G-ISG. Since the logical information is preserved under subsequent measurements for both the 3D Floquet TC and the 3D checkerboard lattice TCs, one can wait to do the $CCZ$ gate transversally in the G-ISG and then proceed with subsequent rounds.

The checkerboard lattice surface code (planar variant) can be obtained as instantaneous stabilizer codes of the 3D subsystem TC, which uses measurements of three-qubit checks.~\cite{Kubica_2022}. Hence, one can stack the planar variant of our 3D Floquet code with two copies of the 3D subsystem TC to do the non-Clifford $CCZ$ gate in the G-round of the 3D Floquet TC.



% Figure environment removed

% Figure environment removed


\section{Rewinding X-cube Floquet code}
\label{sec:Xcube}





The X-cube Floquet code of Ref.~\cite{XcubeFloquet_2022} has ISGs that are for some rounds, stacks of 2D TC, and for other rounds, FDLQC-equivalent to the X-cube model or another fracton model. In this section, we propose a rewinding schedule for the X-cube model such that each ISG is a fracton model, and we do not go to an intermediate ISG of stacks of TC. 

We consider a rewinding schedule of the form GB$\text{R}^{(1)}$BG$\text{R}^{(2)}$ where $1$ and $2$ label the position of the two R-rounds in the sequence. The on-site condensation checks, as shown in Fig.~\ref{fig:Xcubeevolution}(a) are measured along with the green checks in the G-round. The rewinding schedule ensures that in each subsequent round, not all condensation operators evolve into nonlocal stabilizers. 
The evolution of the condensation operators under the rewinding schedule is shown in Fig.~\ref{fig:Xcubeevolution}. If we had used the schedule GBR, then the R-round would be followed by the G-round and all the condensation operators would grow into non-local stabilizers as shown in Fig.~\ref{fig:Xcubeevolution} and result in an ISG of stacks of 2D TC. 

The ISGs in the rewinding X-cube Floquet code are FDLQC-equivalent to those listed in Table~\ref{tab:XCschedule}. 
Explicit FDLQCs to map the ISGs to these models are given in the supplementary \texttt{Mathematica} files. The G-ISG is exactly the canonical X-cube model concatenated with 4-qubit repetition codes on the composite green edges. The B-ISG is FDLQC-equivalent to a product of decoupled models which include the X-cube model, 3D TC, and a 3-foliated stack of 2D TC. 
Similar to the 3D Floquet TC, the ISG in the $\text{R}^{(1)}$-round is related to that of the B-rounds by a basis change and hence is FDLQC-equivalent to the same models. The ISG in the $\text{R}^{(2)}$-round, which follows immediately after the G-round, has different evolved condensation operators as shown in Fig.~\ref{fig:Xcubeevolution}. The $\text{R}^{(2)}$-ISG is FDLQC-equivalent to a product of the X-cube model and a 3-foliated stack of 2D TCs.


\begin{table}[]
    \centering
    \begin{tabular}{|c|l|c|}
    \hline
Round   & Measurement  & ISG\\
\hline
0& G (Green $YY$+ interlayer $YY$) & XC\\
1& B (Blue $ZZ$) & XC $\times$ 3D TC $\times$ stacks\\
2& R (Red $XX$) & XC $\times$ 3D TC $\times$ stacks\\
3& B (Blue $ZZ$) & XC $\times$ 3D TC $\times$ stacks\\
4& G (Green $YY$+ interlayer $YY$) & XC \\
5& R (Red $XX$) & XC $\times$ stacks\\
\hline  
\end{tabular}
\caption{The ISGs in the (rewinding) GBRBGR schedule of measurements for the X-cube Floquet code. The measured checks and the instantaneous stabilizer groups (ISGs) in each round are written in the Measurement column. In the ISG column, XC denotes the X-cube model, 3D TC denotes 3D TC, and ``stacks'' denote a 3-foliated stack of 2D TCs.  For even system sizes $L=2n$, the 3-foliated stack consists of $n$ layers of 2D TC along each of the 3 orthogonal lattice directions.}
\label{tab:XCschedule}
\end{table}

We now discuss the counting of logical qubits in the rewinding X-cube Floquet code. 
As mentioned, the G-ISG is exactly the canonical X-cube model concatenated with 4-qubit repetition codes on the composite green edges. Hence, we have $6L-3$ logical qubits in the G-ISG~\cite{Vijay_2016}. To count the number of logical qubits in the B-ISGs and $\text{R}^{(1/2)}$-ISG, we consider even system sizes $L=2n$ for simplicity. The FDLQC-equivalent model, as stated in Table~\ref{tab:XCschedule} implies a total of $6L$ logical qubits as we get $6n-3$ logical qubits from the X-cube model, 3 logical qubits from the 3D TC, and $6n$ logical qubits from the 3-foliated stack of 2D TCs. The B-ISG has three independent non-local stabilizers and an example is shown in Fig.~\ref{fig:Xcube_nonlocal_stabilizer}. Due to these three non-local stabilizers, the number of logical qubits the B-ISG is $6L-3$ which in turn implies that the $6L-3$ qubits from the G-ISG are all preserved. For an effective description of the B-ISG in terms of rotated 2D TC stack and the counting of logical qubits using that, see Appendix~\ref{sec:eff_description_and_counting_BISG_XC}. The $\text{R}^{(1)}$-ISG is the same as the B-ISG up to a basis change and hence, we have $6L-3$ logical qubits. Due to rewinding, the second B-ISG is exactly the same as the fist B-ISG. The $\text{R}^{(2)}$-ISG is FDLQC-equivalent to the X-cube model and a 3-foliated stack of 2D TCs; hence, the number of logical qubits is $6L-3$. Overall, $6L-3$ logical qubits are preserved in the rewinding X-cube Floquet code.

% Figure environment removed

% Figure environment removed





% Figure environment removed

% Figure environment removed

% Figure environment removed


% Figure environment removed

\section{3D Floquet fermionic toric code}
\label{sec:3DfermionicFloquet}

In this section, we present a Floquet code that has ISGs that are FDLQC-equivalent to the 3D fermionic TC (fTC) ~\cite{Levin2003Fermions,Yuan2019bosonization3D}. This construction is based on the 3D generalization of Kitaev's honeycomb model introduced in Ref.~\cite{Mandal_2009}. This Floquet code, however, only encodes a single logical qubit on a system with periodic boundary conditions, due to inadvertently measuring a subset of the logical operators. We make use of a rewinding schedule to avoid measuring the logical operators for the remaining logical qubit. 


\subsection{16-round measurement schedule} \label{sec: 3D Floquet}



Our first example is defined on the trivalent lattice in Fig.~\ref{fig:3DFloquetlattice} with periodic boundary conditions.\footnote{We note that any 2D lattice can be coarse-grained into a square lattice with a constant number of vertices per unit cell. Since the square lattice is 4-valent, the model can be generalized to a 3D diamond lattice by replacing the sites of the diamond lattice with the unit cell of the square lattice model. The lattice in Fig.~\ref{fig:3DFloquetlattice} can be constructed in this way starting from the honeycomb lattice.} 
We place a qubit at each vertex and label the edges with $x$, $y$, and $z$, as in Fig.~\ref{fig:3DFloquetlattice}. The 2-body check operators on the edges $x$, $y$, and $z$ are $XX$, $YY$, and $ZZ$, respectively. 




Similar to the 2D Floquet TC in Section~\ref{sec:square-octagon_RGBRBGschedule}, the stabilizers of the check group are generated by products of the 2-body checks along closed paths, which includes stabilizers supported on non-contractible paths. The local generators of the stabilizer group are 10-body products of check operators around a plaquette. Given the shape of these plaquettes, we refer to these stabilizers as the `armchair' stabilizers. 
There are four possible orientations for the armchairs: front-, back-, left-, or right-facing [see Fig.~\ref{fig:armchairorientations}(a)-(d)]. For each 3-cell, there exists a local relation between four armchair stabilizers, with one facing in each direction as pictured in Fig.~\ref{fig:armchairorientations}(e). Thus, we only need to infer the measurement outcomes of three out of the four orientations of armchair stabilizers. 




Our measurement schedule is designed to extract the syndrome for one orientation of armchair stabilizers at a time. We use a separate set of five rounds of measurements to infer the front-, back-, and right-facing armchair stabilizers. We do not need to infer the measurement of the left-facing armchairs, given the local relation in Fig.~\ref{fig:armchairorientations}(e). The five rounds of measurements are shown in Fig.~\ref{fig:3Dschedule} and are labeled as 0, 1, 2, 3, and 0, where the 0 round is repeated at the end of the cycle. If we label the armchairs of a unit cell as $A$, $B$, $C$, and $D$, as in Fig.~\ref{fig:ABCDlabels}, we see that after applying consecutive $01$, $12$, $23$, and $30$ checks, the $A$, $B$, $C$, and $D$ armchair stabilizers are inferred, respectively. Therefore, the above schedule measures all the armchair stabilizers for a given orientation, as desired. We proceed similarly for the back- and right-facing armchair stabilizers. 


Naively, a potential measurement schedule is to periodically measure the sequence FBR, where F, B, and R stand for the 5-round measurement schedules for extracting the front-, back-, and right-facing armchair syndromes. This schedule, however, does not exhibit any dynamically generated logical qubits. This is because, in transitioning between the orientations, e.g., F to B, we inadvertently measure the stabilizers supported along non-contractible paths around the torus. Specifically, in transitioning from F to B, B to R, and R to F, we measure the logical operators supported on non-contractible paths along the $(1,0,0)$-, $(0,1,1)$-, and $(1,0,1)$-directions, respectively (see Fig.~\ref{fig:innerlogicalsmeasured}). As such, the FBR schedule gives a Floquet code that does not encode any qubits. 


To rectify this, we rewind the schedule at the level of the armchair orientations, i.e., we periodically measure the sequence FBFR. This sequence avoids the transition from B to R, implying that we do not inadvertently measure the stabilizer supported on a non-contractible path along the $(0,1,1)$-direction. We note that the second 5-round sequence F can be replaced with a single round of 0 measurements for the front-facing armchair. In summary, our schedule consists of repeating the following sequence of 16 rounds of measurements:
\begin{align}
    (0,1,2,3,0)_\text{F}(0,1,2,3,0)_\text{B}0_\text{F}(0,1,2,3,0)_\text{R},
\end{align}
where the subscripts denote the orientations of the armchairs. This modification to the FBR schedule ensures that we retain a single dynamically generated logical qubit.

After each round of measurements, the ISG is FDLQC-equivalent to the 3D fTC. 
To verify this, we use ERG to construct an explicit circuit to map the ISGs to the canonical form of the 3D fTC, as shown in the supplementary \texttt{Mathematica} file. 
We note that, at a high level, the stabilizers of the check group generate a so-called anomalous 2-form symmetry~\cite{Yuan2019bosonization3D}, i.e., the symmetry operators (the stabilizers) are supported on loops and the point-like excitation at the endpoint of a truncated string operator have nontrivial (in this case fermionic) exchange statistics. This implies that each ISG has an anomalous 2-form symmetry, which is sufficient to guarantee that the ISGs necessarily have an emergent fermion. 


The 3D fTC has three logical qubits on a system with periodic boundary conditions. However, given that our schedule inadvertently infers the measurement of the stabilizers supported on non-contractible paths along the $(1,0,0)$- and $(1,0,1)$-directions, our Floquet 3D fTC encodes a single logical qubit. The logical Pauli $Z$ operator can be represented by a product of 2-body check operators along the $(0,1,0)$-direction, as in Fig.~\ref{fig:3Dlogicals}(a). This can be interpreted as the nonlocal stabilizer of the check group that remains unmeasured throughout the measurement schedule. The logical Pauli $X$ operator can be represented by an operator supported on a membrane perpendicular to the $(0,1,0)$-direction [Fig.~\ref{fig:3Dlogicals}(b)]. 
We find that the sequence $(0,1,2,3,0)_\text{F}$ implements the trivial automorphism (and similarly for the B and R sequences). Thus, due to the rewinding nature of the FBFR schedule, the full 16-round period undergoes a trivial automorphism.


%  \begin{center}
% \rule{0.5\columnwidth}{0.6pt}
% \end{center}
% \subsection{A candidate rewinding schedule using coupled layer construction}






\section{Discussion}
\label{sec:discussion}
In this paper, we constructed examples of topological Floquet codes with ISGs that are FDLQC-equivalent to the 2D color code, the 3D TC, and the 3D fermionic TC. To construct the 3D Floquet codes, we used the idea of rewinding the measurement schedule.
Below, we comment on some aspects of our Floquet codes and discuss directions for future work. 




\subsubsection{Local reversibility and possible fault-tolerance}
In Ref.~\cite{aasen2023measurement}, it was suggested that the local reversibility of a topological Floquet code could imply the existence of a non-zero threshold. We show below that our schedules for the 2D Floquet color code are locally reversible and we expect the same to hold for our 3D Floquet codes. We first state the definitions: An ISG is locally generated \textit{above} a subgroup if there is a set of local operators that along with the subgroup elements form a generating set of the ISG. The ISGs in our Floquet codes are all locally generated above a subgroup, with the subgroup being the intersection of two ISGs. Now, two ISGs, let's say 0-ISG and 1-ISG, form a locally reversible pair if there is a choice of local generating sets above the intersection 0-ISG$\cap$1-ISG, 0-LGI and 1-LGI respectively such that one generator from 
0-LGI (1-LGI) anticommutes with exactly one generator from 1-LGI (0-LGI). Such anticommuting pairs are referred to as conjugate pairs. 

In the Floquet color code, the intersection 0-ISG$\cap$1-ISG contains the product of 0-checks and the product of 1-checks around each hexagon. Hence, we can remove one 0-check and 1-check from our local generating sets. 
The removed checks and conjugate pairs for 0- and 1-ISGs are illustrated in Fig.~\ref{fig:localreversibility}(a). 
The pairs of ISGs in rounds 1 and 2 as well as rounds 2 and 0 are also locally reversible. We now explain this for pair of rounds 1 and 2. The intersection 1-ISG$\cap$2-ISG includes the product of 1-checks on the inner and outer boundary of the red-inflated hexagons and the stabilizers of the check group on each of the three inflated hexagons. Hence we remove one 1-check and three 2-checks on the red-inflated hexagon from our consideration because we need to form the conjugate pairs only above the intersection. Then, the conjugate pairs are the products of 2-checks in a chain whose one end intersects with its conjugate partner 1-check in one qubit and the other end intersects the removed 1-check. The removed checks and conjugate pairs for 1- and 2-ISGs are illustrated in Fig.~\ref{fig:localreversibility}(b). 

% Figure environment removed

We expect that the schedules for the 3D Floquet TC and 3D Floquet fTC are also locally reversible. There are non-local stabilizers in the ISGs but they exist as stabilizers in the intersection of any consecutive pair of ISGs and do not preclude the existence of locally conjugate generating sets. Here, we discuss only the reversibility of ISGs in the 3D Floquet TC. We can consider the pair of ISGs, 3D TC in the G-round, and a subspace of two copies of 3D TCs in the B-round. Any string logical operator of G-ISG 3D TC is indeed also a logical string operator of the ISG of two copies of 3D TCs up to nonlocal stabilizers. Hence, the two ISGs share their full set of logical operators and hence, as defined in Ref.~\cite{aasen2023measurement}, this is a sufficient criterion for reversibility. 

\subsubsection{3D Floquet toric code on a fractal}

In Sec.~\ref{subsec:3dFbTC_boundaries}, we discussed the planar realization of the 3D Floquet TC. 
It is known that in the 3D TC, one can punch holes with smooth boundaries to build a fractal lattice code embedded in 3D with fractal Hausdorff dimension $2<D_H<3$~\cite{zhu_fractal_2022,dua2022quantum}. Starting with the G-ISG, we can, in principle, punch holes with smooth boundaries by truncating the checks across the red and blue edges. In subsequent rounds, as discussed in Sec.~\ref{subsec:3dFbTC_boundaries}, these boundaries always remain loop-condensing boundaries, although they may become loop-condensing boundaries of two copies of the 3D surface code depending on the round. Thus, we expect our construction with holes to yield a Floquet code with ISGs that are fractal surface codes embedded in 3D~\cite{dua2022quantum,zhu_fractal_2022}. It would be interesting to further explore such Floquet codes on fractal lattices for both memory and computation. 



\subsubsection{Bifurcation of topological order under measurements}

In the 3D Floquet TC, we observed that as the condensation operators evolved into 4-qubit operators, we obtained the ISG of two copies of the 3D TC, up to nonlocal stabilizers. Starting from stacks of TC, if we are allowed measurement of higher-but-constant weight condensation operators, we can obtain a higher constant number of copies of 3D TC. It would be interesting to see if there is a Floquet code arising from coupled layers, such that the condensation operators always evolve by a \textit{constant} factor so that topological order self-bifurcates under every round of measurements~\footnote{Here, we distinguish between the splitting of topological order in our Floquet code examples and bifurcation of topological order because the former did not happen recursively under the associated dynamics.}. A similar question holds for the bifurcation of the Floquet codes of fracton models under measurements as in the X-cube Floquet code, we observed a splitting of the X-cube topological order into the X-cube model and toric codes.  
Bifurcation is already known to occur for fracton topological order under entanglement renormalization group~\cite{Haah_2014_ERG,Shirley_2019_ERG,Dua_2020_ER}. However, topological orders are conventional fixed points under the entanglement renormalization group. Thus, it is curious if, under specialized measurement sequences, conventional topological order such as the 3D toric code can exhibit bifurcation. 
On the other hand, it would also be interesting if there is a fundamental obstruction to having such measurement dynamics. 

\subsubsection{Classification of measurement schedules}

One fundamental question that arose in this work is what schedules exist for a given check group and what automorphisms of the logical information are exhibited by them. For instance, we wrote three schedules for the Floquet color code, one which exhibited a $\mathbb{Z}_3$ automorphism of logical operators, a rewound version that exhibited a trivial automorphism and another 6-round schedule that exhibits a trivial automorphism. It would be interesting whether there is a notion of equivalence between the two schedules with a trivial automorphism, such as the notion of simple equivalence discussed in Ref.~\cite{aasen2023measurement}. It would also be interesting to have a recipe to construct the inequivalent Floquet codes, given a set of checks on a lattice. For the 3D Floquet fTC, we presented a schedule with 16 rounds that preserved one logical qubit. There could exist a more efficient schedule and also one that preserves all logical qubits. However, our attempts with different schedules and on different trivalent lattices were unsuccessful. Hence, we leave this as a future direction. 

One potential source of inspiration for new schedules, although, not periodic, come from Refs.~\cite{sriram2022topology} and \cite{lavasani2022monitored}, which
considered weighted random measurements of $XX$, $YY$, and $ZZ$ checks in the honeycomb Floquet code. In particular, they found a regime in which the topological information is protected. One could consider such schedules for our Floquet code examples, especially, for the 3D Floquet fTC. We were not able to preserve all three logical qubits, but this may be possible with an appropriate random measurement schedule. Similar to the 2D case, we expect that there is a regime in which all three logical qubits are preserved under the dynamics.

\subsubsection{Condensation picture}

% Figure environment removed

Following Ref.~\cite{Kesselring2022condensation}, it may be insightful to understand the Floquet codes in this work in terms of sequences of condensations in a parent stabilizer code. In general, it may be challenging to identify a suitable parent stabilizer code, but for our examples, the stabilizer group of the check group is nontrivial. This means that a natural generating set for the stabilizer group of the parent stabilizer group is the stabilizers of the check group and a maximal set of local Pauli operators that commute with them. The measurements of the check operators can then be interpreted as condensing an excitation of this parent stabilizer code, resulting in the ISGs of the Floquet code. For the Floquet color code, we write down a parent stabilizer code, as shown in Fig.~\ref{fig:parent_color_Floquet}.
This code is FDLQC-equivalent to two copies of color code or four copies of TC. The exact circuit to map this code to four copies of the TC is given in the supplementary \texttt{Mathematica} files. The complete pictures of condensation in this parent code for our three variants of the Floquet color code still need investigation.
For the 3D Floquet TC, a natural choice of parent stabilizer code is a stack of color codes, since the color code is a parent stabilizer code for the 2D Floquet TC~\cite{kesselring2022anyon}.  

\vspace{3mm}

\vspace{2mm}
\begin{center}
    \small{\MakeUppercase{\textbf{Acknowledgement}}}
\end{center}
\vspace{2mm}
We thank the authors of Ref.~\cite{davydova2023quantum} and Ref.~\cite{townsendteague2023floquetifying} for informing us about their works. N.T., J.S., and T.D.E. acknowledge the 2023 Boulder Summer School on Non-Equilibrium Quantum Dynamics, where part of this work was completed. 
A.D. is supported by the Simons Foundation through the collaboration on Ultra-Quantum Matter (651438, AD) and the Institute for Quantum Information and Matter, an NSF Physics Frontiers Center (PHY-1733907). N.T. is supported by the Walter Burke Institute for Theoretical Physics at Caltech. J.S. is supported by DOE DE-SC0022102. The $\texttt{Mathematica}$ codes used in this paper are available on GitHub at https://github.com/dua-arpit/floquetcodes. 

\vspace{10mm}

\bibliography{bib}
\onecolumngrid
\newpage
\clearpage
\twocolumngrid
\appendix
\vspace{5mm}


\section{Counting of logical qubits in the 3D Floquet toric code}
\label{subsec:counting}


The counting of logical qubits in the G-ISG is straightforward due to its mapping to the cubic lattice 3D TC up to concatenation with 2-qubit repetition codes. The local relations among plaquette stabilizers on the cubes in the cubic lattice 3D toric code map to the local relation among checks as shown in Fig.~\ref{fig:localrelation}. Here, we state the counting of logical qubits in the cubic lattice 3D toric code, which implies the same for the G-ISG. On a torus with linear size $L$, there are $3L^3$ physical qubits, $3L^3$ plaquette stabilizers with $L^3-2$ independent local relations among them, $L^3$ vertex stabilizers with $1$ global relation among them. Thus, we get 3 logical qubits.   

For the B-ISG, the evolved condensation operators are either local operators as shown in Fig.~\ref{fig:evolution_condensation_op_3DTC} or nonlocal stabilizers as shown in Fig.~\ref{fig:nonlocalstab}. We also have blue-green plaquettes as stabilizers in the ISG. Besides that, we have the stabilizers of the check group and the blue checks as stabilizers of the ISG. It is again useful to work in the effective picture of a 3-foliated stack of 2D rotated TCs and consider the condensation operators and nonlocal stabilizers on top of that. We start with the stack of 2D rotated TCs with periodic boundary conditions and we now have the $X^{\otimes 4}$ condensation operators in the configuration as shown in Fig.~\ref{fig:sublattices_BISG}(a).
On every cube of this lattice, we have a $Z$ stabilizer as described above i.e., it is the product of $Z$ stabilizers among three plaquettes sharing the canonical corner associated with the cube. These $Z$ stabilizers correspond to the vertex operators of the two copies of 3D TCs. We assume the linear system size $L$ to be even here and in the discussion on automorphism below. This is because in the B-ISG, having an odd number of cubes on a torus will lead to a spatial defect in the configuration of condensation terms shown in Fig.~\ref{fig:sublattices_BISG}(a) and we avoid such a scenario for simplicity. There are $L^3$ cubes and thus $L^3$ such $Z$ stabilizer operators. There are two relations among these $Z$ stabilizers, one on each sublattice, leaving us with $L^3-2$ independent $ Z$ stabilizers. There are $3L^2$ $X$ plaquette stabilizers and $3L \times L^2/4$ condensation stabilizers where $3L$ is the number of planes. The number of relations among the $X$ stabilizers is $L^3/4+4$. The form of relations is complicated and is shown in the supplementary \texttt{Mathematica} file. 
Considering the 3 nonlocal stabilizers, we get $3$ logical qubits for the B-ISG. The B-ISG can be mapped via an explicit circuit to two copies of 3D TCs up to nonlocal stabilizers; see supplementary \texttt{Mathematica} file. The counting of logical qubits in the other ISGs is similar to these the counting for G-ISG or B-ISG. 

\section{Effective description of B-ISG of the X-cube Floquet code}
\label{sec:eff_description_and_counting_BISG_XC}
To calculate the number of logical qubits explicitly using the relations of stabilizers in the B-ISG, we use the effective description of a 3-foliated stack of rotated 2D TCs. The stabilizers of the B-ISG including the evolved condensation operators are shown in Fig.~\ref{fig:B-ISG_stabilizers}. 
The X-stabilizers of the rotated 2D TC layers are also $X$-stabilizers of the B-ISG. The product of Z-stabilizers of 2D TC layers around each cube of the 3D lattice forms a stabilizer of the B-ISG. Besides these, on every plaquette corresponding to the $Z$-stabilizer of the 2D TC layer, there lives a $X$-stabilizer acting on the qubits of the orthogonal foliations. These are the evolved condensation operators. 

For counting of logical qubits, we consider linear system size $L=2n$, where $L$ is the number of cubes along each lattice direction since the unit cell as shown in Fig.~\ref{fig:B-ISG_stabilizers}(b) consists of even number of cubes along each direction. We have $3L^3/2$ $X$-plaquette stabilizers, $L^3$ $Z$-cubic stabilizers and $3L^3/2$ evolved condensation stabilizers. This gives a total of $4L^3$ stabilizers. 
For the $X$-plaquette stabilizers of the 2D TCs, there is a relation in each plane and hence there are $3L$ planar relations. Besides these, we have $L^3$ local relations among the $X$-plaquette stabilizers and evolved condensation operators which are also $X$-type stabilizers. There is a global relation formed from the product of cubic $Z$-stabilizers of each type (A, B, C, and D respectively). However, only 3 of them are independent. There is a relation among the cubic stabilizers on every dual lattice plane. Considering the aforementioned global relations, there are $3(L-1)$ such independent planar relations. Thus, there are $3L+3+3(L-1)$ global relations and $L^3$ local relations giving overall $6L+L^3$ relations among the stabilizers. 
Considering $3L^3$ physical qubits, we have $6L$ logical qubits. 

Due to the three non-local stabilizers as described in the main text, we have $6L-3$ logical qubits that are actually preserved. This is expected since the preceding ISG (G-ISG) is the X-cube model (up to concatenation) and has $6L-3$ logical qubits. 

% Figure environment removed

% Figure environment removed

\vspace{-3mm}
\section{Alternative 6-round Floquet color code schedule with trivial automorphism} 

In this appendix, we specify a variant of the Floquet color code that exhibits a trivial automorphism of the logical operator representations. The schedule is depicted in Fig.~\ref{fig:6step_color_code_schedule}. This schedule is designed so that the hexagon and inflated hexagon stabilizers of the blue plaquettes can be inferred from rounds 0 and 1. Similarly, the stabilizers associated with the green and red hexagons and inflated hexagons can be inferred from the 2,3 and 4,5 measurement rounds, respectively.



% Figure environment removed


The ISGs after rounds 0 and 1 are shown in Fig.~\ref{fig:effcolorcode}. The ISGs for rounds 2,3,4,5 can be determined by applying the lattice symmetries. Using entanglement renormalization, we find that each ISG is FDLQC-equivalent to the color code. The explicit FDLQCs, written using the polynomial representation~\cite{haah2013lattice,Dua_2020_ER}, can be found in the supplementary \texttt{Mathematica} files~\cite{Floquet_SM}. 


% Figure environment removed




\end{document}
