\documentclass[12pt]{article}
\usepackage{amsmath,amsthm,amsfonts,amssymb}
\usepackage[margin=1.1in]{geometry}
\usepackage{xcolor}

\begin{document}
% To be copy and pasted as text (no latex): 
\section*{}
\vspace{-14mm}

\noindent Dear PRX Quantum Editor,\\

\noindent We are excited to present our latest research paper titled ``Engineering Floquet codes by rewinding'' for your consideration and potential publication in PRX Quantum. \\

\noindent A significant and ongoing endeavor is to design quantum error-correcting codes that optimize code parameters such as encoding rate, code distance, and circuit-level thresholds while concurrently minimizing space and time overheads. Recently, Hastings and Haah made a breakthrough in this direction by introducing a novel class of codes referred to as Floquet codes, which exhibit dynamically generated logical qubits. 
Floquet codes present a compelling alternative to the surface code due to their low-weight parity checks and relatively high error threshold. Despite their promise, our collective understanding of Floquet codes is still under active development. This underscores the significance of introducing new examples and formalizing tools for the creation and classification of Floquet codes.\\


\noindent Our research provides a significant advancement towards this goal, as we introduce novel examples of Floquet codes such as \textbf{(i)} the Floquet color code, which features a $\mathbb{Z}_3$ automorphism of the logical operators, \textbf{(ii)} a 3D Floquet toric code, which allows for a transversal implementation of a non-Clifford gate and is created by augmenting stacks of 2D Floquet codes with additional 2-qubit measurements, and \textbf{(iii)} a 3D Floquet fermionic toric code obtained by extending the 2D Floquet code framework to a trivalent lattice in three dimensions. Central to our constructions is the technique of ``rewinding'' the measurement schedule. This enables us to attain desired instantaneous stabilizer groups (ISGs) and facilitates \textbf{(a)} the design of schedules with trivial automorphisms, \textbf{(b)} the creation of boundaries, and \textbf{(c)} the preservation of logical qubits. As an exhibition of rewinding, we also adapt an earlier construction of an X-cube Floquet code to a rewinding version, such that each ISG is the X-cube model up to copies of 2D and 3D toric codes, unlike the earlier construction. \\


\noindent Our investigation has also unveiled an intriguing phenomenon, wherein a topological order splits into copies of itself during sequences of measurements in both the 3D Floquet toric code and the X-cube Floquet code. While bifurcations of topological order under the entanglement renormalization group have been established, our research brings forth a novel perspective by revealing similar splittings through measurement sequences. Furthermore, our measurement sequences bifurcate the 3D toric code, which is known to be a conventional fixed point under entanglement renormalization group transformations.\\

\noindent We firmly believe that our work is exceptionally well-suited for the readership of PRX Quantum. These findings offer not only a deeper collection of examples of Floquet codes but also insights into the strategic design of Floquet codes with specific ISGs, hints at potential directions for classifying measurement schedules, and uncovers the prospect of the bifurcation of topological order through measurements.\\

\noindent We extend our gratitude for your valuable time and consideration. The prospect of sharing our insights and contributing to the scientific discourse in PRX Quantum is both exciting and motivating. We eagerly anticipate your response.\\

\vspace{5mm}
\noindent Sincerely,\\
\noindent Authors\\


Popular summary:

Quantum error correcting codes specify dynamics in a quantum computer such that the stored quantum information remains preserved over time. A significant and ongoing endeavor is to design quantum error-correcting codes that optimize code parameters such as encoding rate, code distance, and circuit-level thresholds while concurrently minimizing space and time overheads. Recently introduced Floquet codes present a compelling alternative to the surface code due to their low-weight parity checks and relatively high error threshold. This work uses condensation and rewinding to create novel 3D Floquet codes, including the 3D Floquet toric code, 3D Floquet fermionic toric code, and the rewinding X-cube Floquet code. Imagine snow being condensed in patches (local condensation) on your car's windshield and wipers cleaning it (rewinding) before it collects together to form a non-local patch affecting visibility. In Floquet codes, each measurement can be considered as performing a condensation. Under the sequence of measurements, these checks can evolve into bigger checks. Before all checks evolve into non-local operators, we perform rewinding to obtain desired instantaneous states or prevent logical information loss. 


Despite their promise, our collective understanding of Floquet codes is still under active development. This underscores the significance of introducing new examples and formalizing tools for the creation and classification of Floquet codes. Our research provides a significant advancement towards this goal, as we introduce novel examples of Floquet codes such as \textbf{(i)} the Floquet color code, which features a $\mathbb{Z}_3$ automorphism of the logical operators, \textbf{(ii)} a 3D Floquet toric code, which allows for a transversal implementation of a non-Clifford gate and is created by augmenting stacks of 2D Floquet codes with additional 2-qubit measurements, and \textbf{(iii)} a 3D Floquet fermionic toric code obtained by extending the 2D Floquet code framework to a trivalent lattice in three dimensions. Central to our constructions is the technique of ``rewinding'' the measurement schedule. This enables us to attain desired instantaneous stabilizer groups (ISGs) and facilitates \textbf{(a)} the design of schedules with trivial automorphisms, \textbf{(b)} the creation of boundaries, and \textbf{(c)} the preservation of logical qubits. As an exhibition of rewinding, we also adapt an earlier construction of an X-cube Floquet code to a rewinding version, such that each ISG is the X-cube model up to copies of 2D and 3D toric codes, unlike the earlier construction. Our investigation has also unveiled an intriguing phenomenon, wherein a topological order splits into copies of itself during sequences of measurements in both the 3D Floquet toric code and the X-cube Floquet code. While bifurcations of topological order under the entanglement renormalization group have been established, our research brings forth a novel perspective by revealing similar splittings through measurement sequences. Furthermore, our measurement sequences bifurcate the 3D toric code, which is known to be a conventional fixed point under entanglement renormalization group transformations. These findings offer not only a deeper collection of examples of Floquet codes but also insights into the strategic design of Floquet codes with specific ISGs, hints at potential directions for classifying measurement schedules, and uncovers the prospect of the bifurcation of topological order through measurements.

\end{document}




