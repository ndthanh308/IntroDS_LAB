%% ****** Start of file apstemplate.tex ****** %
%%
%%
%%   This file is part of the APS files in the REVTeX 4.2 distribution.
%%   Version 4.2a of REVTeX, January, 2015
%%
%%
%%   Copyright (c) 2015 The American Physical Society.
%%
%%   See the REVTeX 4 README file for restrictions and more information.
%%
%
% This is a template for producing manuscripts for use with REVTEX 4.2
% Copy this file to another name and then work on that file.
% That way, you always have this original template file to use.
%
% Group addresses by affiliation; use superscriptaddress for long
% author lists, or if there are many overlapping affiliations.
% For Phys. Rev. appearance, change preprint to twocolumn.
% Choose pra, prb, prc, prd, pre, prl, prstab, prstper, or rmp for journal
%  Add 'draft' option to mark overfull boxes with black boxes
%  Add 'showkeys' option to make keywords appear
%\documentclass[aps,prb,twocolumn,groupedaddress]{revtex4-2}
\documentclass[aps,prl,preprint,groupedaddress]{revtex4-2}
%\documentclass[aps,prl,preprint,superscriptaddress]{revtex4-2}
%\documentclass[aps,prl,reprint,groupedaddress]{revtex4-2}

% You should use BibTeX and apsrev.bst for references
% Choosing a journal automatically selects the correct APS
% BibTeX style file (bst file), so only uncomment the line
% below if necessary.
%\bibliographystyle{apsrev4-2}
%==========================================================
% Packages added by author
\usepackage{graphicx}
%\usepackage{graphicx, caption}
%\usepackage{cite}
\usepackage{float}
%\usepackage{pdfpages}
%==================================================

\begin{document}

% Use the \preprint command to place your local institutional report
% number in the upper righthand corner of the title page in preprint mode.
% Multiple \preprint commands are allowed.
% Use the 'preprintnumbers' class option to override journal defaults
% to display numbers if necessary
%\preprint{}

%Title of paper
\title{Layer number and stacking order-dependent thermal transport in molybdenum disulfide with sulfur vacancies}

% repeat the \author .. \affiliation  etc. as needed
% \email, \thanks, \homepage, \altaffiliation all apply to the current
% author. Explanatory text should go in the []'s, actual e-mail
% address or url should go in the {}'s for \email and \homepage.
% Please use the appropriate macro foreach each type of information

% \affiliation command applies to all authors since the last
% \affiliation command. The \affiliation command should follow the
% other information
% \affiliation can be followed by \email, \homepage, \thanks as well.
\author{Ranjuna M K}
\email[]{ranjuna.mk@gmail.com}
\author{Jayakumar Balakrishnan}
\email[]{jayakumar@iitpkd.ac.in}
%\homepage[]{Your web page}
%\thanks{}
%\altaffiliation{}
\affiliation{Department of Physics, Indian Institute of Technology Palakkad, Palakkad-678623, Kerala, India}

%Collaboration name if desired (requires use of superscriptaddress
%option in \documentclass). \noaffiliation is required (may also be
%used with the \author command).
%\collaboration can be followed by \email, \homepage, \thanks as well.
%\collaboration{}
%\noaffiliation

\date{\today}

\begin{abstract}
% insert abstract here
Recent theoretical works on two-dimensional molybdenum disulfide, MoS$_2$, with sulfur vacancies predict that the suppression of thermal transport in MoS$_2$ by point defects is more prominent in monolayers and becomes negligible as layer number increases. %predict a significant decrease in the thermal conductivity of monolayer MoS$_2$ from its pristine values while in higher layer number samples the thermal conductivity remains unchanged from its pristine values. 
Here, we investigate experimentally the thermal transport properties of two-dimensional molybdenum disulfide crystals with inherent sulfur vacancies. We study the first-order temperature coefficients of interlayer and intralayer Raman modes of MoS$_2$ crystals with different layer numbers and stacking orders. The in-plane thermal conductivity ($\kappa$) and total interface conductance per unit area ($ g $) across the 2D material-substrate interface of mono-, bi- and tri-layer MoS$_2$ samples are measured using the micro-Raman thermometry. Our results clearly demonstrate that the thermal conductivity  is significantly suppressed by sulfur vacancies in monolayer MoS$_2$. However, this reduction in $\kappa$ becomes less evident as the layer number increases, confirming the theoretical predictions. No significant variation is observed in the $\kappa$ and $ g $ values of 2H and 3R stacked bilayer MoS$_2$ samples. 


\end{abstract}

% insert suggested keywords - APS authors don't need to do this
%\keywords{}

%\maketitle must follow title, authors, abstract, and keywords
\maketitle

% body of paper here - Use proper section commands
% References should be done using the \cite, \ref, and \label commands
%=======================================================================
%\section{Introduction}

% Put \label in argument of \section for cross-referencing
%\section{\label{}}
\section{Introduction}
Two-dimensional transition metal dichalcogenides (TMDs), an advancing class of layered materials, have recently become a fertile ground for investigating fundamental properties and emergent device applications. Among these, 2D crystals of molybdenum disulfide (MoS$_2$) have gained significant importance due to their unique properties, such as direct bandgap in monolayer\cite{mak2010atomically}, high carrier mobility, significant light-matter interactions\cite{liu2015strong}, reasonable spin-orbit interactions\cite{xiao2012coupled}, good valley selectivity\cite{mak2012control,zeng2012valley,chakrabarti2022enhancement} and high Seebeck coefficient\cite{buscema2013large,wu2014large,huang2016metallic}. This makes them potential materials for applications in electronics\cite{radisavljevic2011single,chen2018tuning,cheng2014few}, optoelectronics\cite{fontana2013electron,pham2019mos2,bernardi2013extraordinary}, Valleytronics\cite{zeng2012valley,mak2012control,sun2019separation} and energy harvesting.\cite{huang2016metallic,xu2022frenkel} However, the number and stacking order of layers and the defects and impurities in MoS$_2$ can significantly impact its properties and the device's performance.\cite{qiu2013hopping} 

Monolayer MoS$_2$ comprises the hexagonal arrangement of molybdenum (Mo) and sulfur (S) atoms sandwiched to form an S-Mo-S structure. These monolayers can be stacked in different configurations to form different polytypes of multilayer MoS$_2$. Previous studies on the most common configurations, the 2H (hexagonal) and 3R (rhombohedral) phases, reveal that stacking order is vital in various applications like piezoelectricity, nonlinear optics, and catalytic activity.\cite{toh20173r,shi20173r,hallil2022strong} Similarly, 
structural defects in MoS$_2$, such as vacancies, substitutions, dislocations, grain boundaries, and edges, can significantly influence its properties due to their impact on the atomic structure and electronic behavior.\cite{qiu2013hopping,chen2018tuning,zhou2013intrinsic,tongay2013defects} However, depending on the targeted application, their impact can be either detrimental or beneficial. For example, the mobility of carriers in electronic devices can get reduced by the collision with defects. In contrast, the presence of defects can enhance the performance of MoS$_2$ in various fields, including catalysis, energy storage, and sensing.\cite{xu2022frenkel}

%Structural defects such as sulfur vacancies can create localized states in the bandgap and alter the electronic and optical properties of the material.\cite{qiu2013hopping,chen2018tuning,zhou2013intrinsic,tongay2013defects} Defect engineering is an effective strategy to improve the catalytic activity of molybdenum disulfide for energy harvesting and storage applications.\cite{xu2022frenkel}
%............Give some examples of Effect of structural defects on different properties and device performance ..........
%Structural defects such as vacancies, substitutions, dislocations, grain boundaries and edges can be either formed during growth or introduced by controlled processes like electron or ion beam irradiation.[cite] The presence of such defects can be either detrimental or beneficial, depending on the targeted application.
%It has been shown that the defects in MoS$_2$ cause the degradation of electronic transport by creating defect-induced localized states and increasing the scattering of carriers.\cite{najmaei2014electrical} %reducing carrier mobility
%Thermal transport studies are essential for understanding fundamental properties, thermoelectric applications, and efficient thermal management in various devices. Even though understanding the effect of stacking order and structural defects on thermal transport in MoS$_2$ is crucial for optimizing its thermal properties and designing efficient thermal management strategies in various applications, there are limited experimental attempts for this.

Understanding the effect of stacking order and structural defects on thermal transport in MoS$_2$ is crucial for optimizing its thermal properties and thermoelectric applications and designing efficient thermal management strategies in various applications. Thermal conductivity of monolayer and multilayer MoS$_2$ in its suspended and substrate-bound form has been reported.\cite{yan2014thermal, sahoo2013temperature, zhang2015measurement,yuan2018nonmonotonic} However, most of these works does not account for the possibility of structural defects or difference in the stacking order of MoS$_2$ crystals. Recently there have been theoretical studies on the thermal transport in MoS$_2$ with various defects like grain boundaries, vacancies, and substitutions.\cite{ding2015manipulating,peng2016beyond, saha2016theoretical,wang2016remarkable,lin2019phonon, polanco2020defect,xu2022grain,gabourie2020reduced} A recent experimental work explored the phonon thermal transport in few-layer MoS$_2$ flakes with various point defect concentrations enabled by helium ion (He$^+$) irradiation\cite{zhao2021modification}. They observed that Mo vacancies significantly reduce the thermal conductivity of few-layer MoS$_2$ compared to S vacancies. Similarly, recent work on thermal transport in CVD-grown monolayer MoS$_2$ with grain boundaries also showed a lower thermal conductivity of the sample.\cite{yarali2017effects} 
%However, a comprehensive experimental study of thermal transport in MoS$_2$ with sulfur deficiency is lacking. 

In this letter, we report a systematic study of thickness-dependent thermal transport in supported MoS$_2$ crystals with intrinsic sulfur vacancies, including the 2H and 3R stacked bilayers. The in-plane thermal conductivity ($\kappa$) and total interface conductance per unit area ($g$) across the 2D material-substrate interface are obtained using Raman thermometry. For this, the first-order temperature coefficients of the intra-layer and inter-layer Raman modes are estimated. As per our understanding, this is the first report on the evolution of first-order temperature coefficients of low-frequency interlayer Raman modes of MoS$_2$ with layer number. Experimental results show that the $\kappa$ decreases as the layer number increases, whereas the $g$ increases. Our results also demonstrate that the thermal conductivity is significantly suppressed by sulfur vacancies in monolayer MoS$_2$. However, this reduction in $\kappa$ becomes less evident as the layer number increases. Hence our results confirm the recent theoretical predictions that the suppression of thermal transport in MoS$_2$ by point defects is more prominent in monolayers and becomes negligible as layer number increases. No significant variation is observed in the $\kappa$ and $ g $ values of 2H and 3R stacked bilayer MoS$_2$ samples.

%In this letter, we report a systematic study of thermal transport in mono-, bi-, and tri-layer MoS$_2$ crystals with intrinsic sulfur vacancies, including the 2H and 3R stacked bilayers. 
%Our results demonstrate that the thermal conductivity is significantly suppressed by sulfur vacancies in monolayer MoS$_2$. However, this reduction in $\kappa$ becomes less evident as the layer number increases. Our results confirm the recent theoretical predictions that the suppression of thermal transport in MoS$_2$ by point defects is more prominent in monolayers and becomes negligible as layer number increases. No significant variation is observed in the $\kappa$ and $ g $ values of 2H and 3R stacked bilayer MoS$_2$ samples.

%============================================================================
\section{RESULTS AND DISCUSSIONS}

 % Figure environment removed

The 2D crystals of MoS$_2$ samples are prepared from bulk crystal by mechanical exfoliation and transferred on silicon wafers covered by 285 nm thick SiO$_2$. The numbers of MoS$_2$ layers are first identified by optical contrast and then confirmed by Raman spectroscopy. The stacking order of bi- and trilayer samples is determined using interlayer phonon modes in their Raman spectra. The stoichiometry of the molybdenum disulfide (Mo:S ratio) is obtained in the range of 1:1.71 to 1:1.76 from the Scanning Electron Microscopy Energy Dispersive Spectroscopy (SEM-EDS) and X-ray Photoelectron Spectroscopy (XPS) measurement which confirms the presence of sulfur vacancies in the sample. Previous works also identified sulphur vacancy as the dominant category of defects in mechanically exfoliated MoS$_2$ samples.\cite{hong2015exploring} 
For additional experimental details, see the supplementary information file. The thermal transport properties of 2D crystals of MoS$_2$ are measured using Raman thermometry.\cite{cai2010thermal} This method benefits from its non-contact and non-destructive nature and relatively simple implementation. The experiment involves the application of a localized heat source to the material and measuring the resulting temperature changes. Later a steady-state heat conduction model is used to extract the $\kappa$ and $g$ values of the 2D materials bound to a substrate. 

Initially, a low-power laser beam is used to acquire the Raman spectrum of uniformly heated MoS$_2$ samples at different temperatures. The Figure \ref{Figure1} (a)-(e) represents the Raman spectra of %monolayer (1L), 2H stacked bilayer (2L 2H), 3R stacked bilayer (2L 3R), 2H stacked trilayer (3L 2H) and fewlayer (FL)
MoS$_2$ samples recorded at different temperatures. The peaks around 384 cm$^{-1}$ and 406 cm$^{-1}$ correspond to the intralayer E$^1_{2g}$ and A$_{1g}$ modes, respectively. The E$_{2g}^1$ mode originates from the in-plane vibrations of the molybdenum (Mo) and sulfur (S) atoms, and the A$_{1g}$ mode originates from the out-of-plane vibrations of S atoms.\cite{li2012bulk} The variation in peak position with temperature is displayed in Figure \ref{Figure1}(f). In all samples, the  Raman mode frequency redshifts with increased temperature. This observed redshift in phonon frequency is mainly due to the contribution from thermal expansion and anharmonic effects at higher temperatures.\cite{sahoo2013temperature} The data points are fitted with a linear function ($ \omega (T) = \omega_0 + \chi_T \times T$) to determine the first-order temperature coefficient ($ \chi_{T}$) of each phonon mode. The comparison of first-order temperature coefficient of E$^1_{2g}$ and A$_{1g}$ modes of MoS$_2$ flakes with different layer number and stacking order is provided in the Section S3 of the supporting information file. For all layer numbers and stacking orders under study, the magnitude of $ \chi_{T}$ value of E$^1_{2g}$ mode is higher than that of A$_{1g}$ mode which agrees with previous reports.\cite{sahoo2013temperature, kim2021temperature} This difference in temperature coefficients between the E$^1_{2g}$ and A$_{1g}$ Raman modes in MoS$_2$ can be attributed to their different symmetries and vibrational characteristics. Since the E$^1_{2g}$ mode corresponds to the in-plane vibration of S and Mo atoms, it is particularly sensitive to the thermal expansion and anharmonic effects in the in-plane direction, contributing to its higher temperature coefficient. We observed that the $\chi_{T}$ values of the E$^1_{2g}$ mode exhibited a monotonous decrease in its magnitude as the layer number increased, whereas that of the A$_{1g}$ mode exhibited a sharp reduction from monolayer to bilayer and then increased with layer number.

%Raman thermometry is a widely used experimental technique to study the in-plane thermal conductivity ($\kappa$ and interface conductance ($g$) of van der Waals 2D materials bound to a substrate. When subjected to uniform heating, a low power laser beam probes the Raman active phonon mode frequency at each temperature. Later the laser beam of different power is used to heat the sample, and the corresponding phonon mode frequency is recorded. The average sample temperature corresponding to the absorbed laser power can be determined by comparing the observations of these two independent sets of measurements. The steady state heat conduction model is used to extract the $\kappa$ and $g$ values. This method benefits from its non-contact and non-destructive nature and relatively simple implementation.

Other than E$^1_{2g}$ and A$_{1g}$ modes, the Raman spectrum of MoS$_2$ samples contains interlayer phonon modes that appear in the low-frequency range (below 50 cm$^{-1}$). Shear mode (SM) and layer breathing mode (LBM) originate from in-plane and out-of-plane vibrations of both S and Mo atoms, respectively, as shown in Figure \ref{Figure2}(a). These modes are absent in the monolayer. The temperature coefficients of these interlayer modes in MoS$_2$ are not extensively studied as other Raman modes, primarily due to their relatively weaker intensity and the requisite of a more complex experimental setup with access to ultra-low frequencies and better spectral resolution.\cite{maity2018temperature,kim2021temperature} Here, we analyze the temperature dependence of low-frequency Raman modes of bi-, tri- and few-layer MoS$_2$ flakes. Figure \ref{Figure2}(b) displays the variation in the peak position of low-frequency modes as a function of temperature. Both shear and layer breathing modes show redshift due to anharmonicity and increased interlayer separation at higher temperatures.\cite{maity2018temperature} The temperature coefficients of low-frequency modes are given in Table SI. The $ \chi_T$ values of interlayer modes are much smaller than that of the high-frequency intralayer modes. The $ \chi _T$ value of the shear mode reduces their magnitude as the layer number increases. %Add more analysis.

%As per our understanding this is the first report on the evolution of first-order temperature coefficients of low-frequency Raman modes of MoS$_2$ with layer number.

  % Figure environment removed

Since both the E$^1_{2g}$ and A$_{1g}$ modes have higher temperature coefficients, it allows the measurement of the local temperature rise of the supported MoS$_2$ crystals due to a change in incident laser power by employing changes in the phonon frequency. 
Raman measurements are performed with different laser power for two different spot-size objectives. The incident power is measured with a power meter, and the absorbed power is calculated by considering the layer number-dependent optical absorption for the 532 nm wavelength. The variation of the peak position of E$^1_{2g}$ and A$_{1g}$ phonon modes with absorbed laser power is displayed in Figure \ref{Figure3}. For all samples, the peak position redshifts with increased absorbed laser power. The fitting of data points with a linear function gives the power coefficient ($d\omega / dP=\chi_p$) of each Raman mode. The $\chi_p$ of the A$_{1g}$ mode appears more sensitive to variations in layer number than that of the E$^1_{2g}$ mode. Since the A$_{1g}$ mode is insensitive to in-plane strain and its power coefficient is more sensitive to layer number, the temperature-induced softening of the A$_{1g}$ phonon mode is used to estimate local temperature rise and extract the thermal parameters. 

 % Figure environment removed
 
Modeling of thermal transport in supported 2D materials subjected to laser heating has been reported previously\cite{cai2010thermal,judek2015high,goushehgir2021simple}. Under the steady-state condition, laser-induced heating is compensated by the heat flow towards the low-temperature heat sink. By assuming diffusive phonon transport, the heat conduction equation for a supported 2D material can be written as \cite{cai2010thermal}

\begin{equation}\label{Eq1}
\frac{1}{r}\frac{d}{dr}\biggl(r\frac{dT(r)}{dr}\biggr) - \frac{g}{\kappa t}(T(r)-T_{{a}}) + \frac{{q}_{0}^{\prime \prime}}{\kappa t }\exp\biggl(-\frac{r^{2}}{r_{0}^{ 2}} \biggr) = 0
\end{equation}

where % $T$ is the temperature of the 2D material at a radial position $r$ measured from the center of the laser beam, 
$T_{a} $ is the temperature of the substrate, $ t $ is the thickness of the 2D material, 
%$ \kappa$ is the effective in-plane thermal conductivity of the supported 2D material, $ g $ is the total interface thermal conductance per unit area across the 2D material-substrate interface 
and $ q_{0}^{\prime \prime} $ is the peak absorbed laser power per unit area at the center of the beam.
The absorbed laser power $Q={q}_{0}^{\prime \prime}\, \pi r_0^2 $ where $r_0 $ is the laser beam radius. Laser beam radius for 50x and 100x objectives is measured using Modified Knife edge method\cite{taube2015temperature} and the values are 0.495 $\mu$m and 0.297 $\mu$m, respectively. Details of laser beam size measurement is included in Section S5 of the supporting information. 
The local temperature rise in the 2D crystal is defined as $  \Theta(r) = T(r) - T_{a} $, and the average temperature rise is given by 

\begin{equation}\label{Eq2}
\Theta_{{m}}(\kappa_{s}, g, r_0,Q) =  \frac{\int_{0}^{\infty}\Theta(r)\exp\biggl(-\frac{r^{2}}{r_{0}^{ 2}} \biggr)rdr }{\int_{0}^{\infty}\exp\biggl(-\frac{r^{2}}{r_{0}^{ 2}} \biggr)rdr }
\end{equation}

The thermal resistance which impedes the flow of heat in the system is obtained as $R_{m}=\Theta_{m}$/$Q$. The ratio of $R_{m}$ for two distinct laser beam sizes depends only on the material parameters $\kappa$ and $ g $. Hence on solving Equation (\ref{Eq1}) for two beam sizes, we are able to estimate  $\kappa$ and $ g $ uniquely. The estimated in-plane thermal conductivity and interface conductance values, along with the $R_{m}$ ratios, are presented in Table \ref{Table1}. The in-plane thermal conductivity decreases as the layer number increases, whereas the total interface conductance per unit area across the 2D material-substrate interface increases. %This opposite trend in in-plane thermal conductivity and interface conductance with an increase in layer number agrees with the recent report on layer number-dependent thermal transport of WSe$_2$ crystals.\cite{easy2021experimental} 
The reduction in in-plane thermal conductivity for increased layer number is primarily attributed to the intrinsic scattering mechanism of phonons.\cite{easy2021experimental} 

\begin{table}[ht]
	\caption{\label{Table1}{Thermal resistance ratio, in-plane thermal conductivity and interface thermal conductance (at room-temperature) of mono-, bi- and tri-layer MoS$_2$ crystals on $ \text{SiO}_2/ \text{Si}$ substrate}}
%	\scalebox{0.95}{
\begin{ruledtabular}
	\begin{tabular}{cccc}
%	\hline
	Sample &  $R_{m}$ ratio & $\kappa $ (W/mK) & $g $ (MW/m$^2$K)\\[0.2em]
	\hline
        1L &   2.282 $\pm$ 0.037 & 40 +8/-6 & 1.02 $\pm$ 0.03 \\
%        \hline
	2L (2H)  &   2.137 $\pm$ 0.084 & 33 +10/-7   & 1.09 +0.04/-0.05 \\
%	\hline
	2L (3R)  &  2.15 $\pm$ 0.13 & 32 +13/-10  & 1.07 +0.13/-0.15\\
%        \hline
	3L (2H) &  2.082 $\pm$ 0.068 & 28 $\pm$6 & 1.23 $\pm$0.05\\
 %1L\footnote{MoS$_2$ on Au/SiO$_2$/Si Ref. \cite{zhang2015measurement}} & - & 55 $\pm$ 20 &  0.44 $\pm $ 0.07\\
% 2L\footnote{MoS$_2$ on Au/SiO$_2$/Si Ref. \cite{zhang2015measurement}} & - & 35 $\pm$ 7 &   0.74 $\pm $ 0.05\\
%    \hline
    \end{tabular}
\end{ruledtabular}
%}
\end{table}

In this work, the measured value of $\kappa $ of the monolayer MoS$_2$ sample is less than the previously reported value of 55 $\pm$ 20 W/mK in samples without any structural defects.\cite{zhang2015measurement} However, in the case of bilayer samples, our measured value of $\kappa $ is close to their reported value of 35 $\pm$ 7 W/mK.\cite{zhang2015measurement} %In this context, it is essential to note that our MoS$_2$ samples contain sulfur vacancies which can alter their thermal transport properties. Hence 
The reduction in the measured $\kappa $ value of 1L MoS$_2$ can be attributed to the suppression of thermal conductivity by sulfur vacancies and this reduction in $\kappa$ becomes less evident as the layer number increases. Our observations are in agreement with the recent theoretical predictions that the presence of sulfur vacancies significantly suppresses the thermal conductivity of monolayer MoS$_2$ due to the strong localization and increased scattering of phonons by defects.\cite{polanco2020defect, wang2016remarkable} However, the thermal conductivity of bulk MoS$_2$ is expected to be less dependent on the presence of sulfur vacancies.\cite{polanco2020defect}. Additionally, no significant variation is observed in the $\kappa $ and $g$ values of 2H and 3R stacked bilayer MoS$_2$ samples. It is also worth noting that the specific impact of sulfur vacancies on thermal transport in MoS$_2$ can depend on additional factors such as vacancy type, concentration and distribution.

%The stacking order has a greater impact on the thermal conductivity of bilayer samples, as per our data. However, the larger error bar in the estimated values suggests a more accurate measurement method is required to distinguish the effect of stacking order on the thermal conductivity of bilayer samples.
%Theoretical reports suggest suppression of thermal conductivity by sulfur vacancies in monolayer MoS$_2$.\cite{polanco2020defect, wang2016remarkable} The thermal conductivity of bulk MoS$_2$ is expected to be less dependent on the presence of sulfur vacancies. This is attributed to the suppression of thermal conductivity by sulfur vacancies in the sample. The thermal conductivity reduction is not evident as the layer number increases. 


%Since the interfacial resistance, which limits the heat transfer between MoS$_2$ and the substrate, is not high, the in-plane thermal conductivity is sensitive to the interface conductance, leading to their reduced thermal conductivity compared to free-standing MoS$_2$. 
 %The supporting information file provides additional details on thermal transport modeling and $\kappa$ and $ g $ estimation. 
\section{Conclusions}
 In conclusion, we investigated the first-order temperature coefficients of interlayer and intralayer Raman modes of MoS$_2$ crystals with inherent sulfur vacancies exfoliated on Si/SiO$_2$ substrate. Later the $\kappa$ and $ g $ values of mono-, bi-, and tri-layer MoS$_2$ crystals are measured using Raman thermometry. The $\kappa$ value decreases, and the $ g $ value decreases as the layer number increases. Importantly, we observed that the thermal conductivity is significantly suppressed by sulfur vacancies in monolayer MoS$_2$. However, this reduction in $\kappa$ becomes less evident as the layer number increases. Our results confirm previous theoretical predictions that the suppression of thermal transport in MoS$_2$ by point defects is more prominent in monolayers and becomes negligible as layer number increases. No significant variation is observed in the $\kappa$ and $ g $ values of 2H and 3R stacked bilayer MoS$_2$ samples.   
%==================================================================


%%%%%%%%%%%%%%%%%%%%%%%%%%%%%%%%%%%%%%%%%%%%%%%%%%%%%%%%%%%%%%%%%%%%%%%%%%%%%%%%%%%%%%%
% If in two-column mode, this environment will change to single-column
% format so that long equations can be displayed. Use
% sparingly.
%\begin{widetext}
% put long equation here
%\end{widetext}

% figures should be put into the text as floats.
% Use the graphics or graphicx packages (distributed with LaTeX2e)
% and the \includegraphics macro defined in those packages.
% See the LaTeX Graphics Companion by Michel Goosens, Sebastian Rahtz,
% and Frank Mittelbach for instance.
%
% Here is an example of the general form of a figure:
% Fill in the caption in the braces of the \caption{} command. Put the label
% that you will use with \ref{} command in the braces of the \label{} command.
% Use the figure* environment if the figure should span across the
% entire page. There is no need to do explicit centering.

% % Figure environment removed

% Surround figure environment with turnpage environment for landscape
% figure
% \begin{turnpage}
% % Figure environment removed
% \end{turnpage}

% tables should appear as floats within the text
%
% Here is an example of the general form of a table:
% Fill in the caption in the braces of the \caption{} command. Put the label
% that you will use with \ref{} command in the braces of the \label{} command.
% Insert the column specifiers (l, r, c, d, etc.) in the empty braces of the
% \begin{tabular}{} command.
% The ruledtabular enviroment adds doubled rules to table and sets a
% reasonable default table settings.
% Use the table* environment to get a full-width table in two-column
% Add \usepackage{longtable} and the longtable (or longtable*}
% environment for nicely formatted long tables. Or use the the [H]
% placement option to break a long table (with less control than 
% in longtable).
% \begin{table}%[H] add [H] placement to break table across pages
% \caption{\label{}}
% \begin{ruledtabular}
% \begin{tabular}{}
% Lines of table here ending with \\
% \end{tabular}
% \end{ruledtabular}
% \end{table}

% Surround table environment with turnpage environment for landscape
% table
% \begin{turnpage}
% \begin{table}
% \caption{\label{}}
% \begin{ruledtabular}
% \begin{tabular}{}
% \end{tabular}
% \end{ruledtabular}
% \end{table}
% \end{turnpage}

% Specify following sections are appendices. Use \appendix* if there
% only one appendix.
%\appendix
%\section{}

% If you have acknowledgments, this puts in the proper section head.
\begin{acknowledgments}
The authors thank the Central Instrumentation Facility (CIF) and Central Micro-Nano Fabrication Facility (CMFF), Indian Institute of Technology Palakkad and Central Instrumentation Facility (CIF) IISER Thiruvananthapuram for the experimental facilities.
\end{acknowledgments}

% Create the reference section using BibTeX:

\newpage
\section{Supporting information}

%\includepdf[pages=-]{MoS2ThermalSupporting.pdf}

%% ****** Start of file template.aps ****** %
%%
%%
%%   This file is part of the APS files in the REVTeX 4 distribution.
%%   Version 4.0 of REVTeX, August 2001
%%
%%
%%   Copyright (c) 2001 The American Physical Society.
%%
%%   See the REVTeX 4 README file for restrictions and more information.
%%
%
% This is a template for producing manuscripts for use with REVTEX 4.0
% Copy this file to another name and then work on that file.
% That way, you always have this original template file to use.
%
% Group addresses by affiliation; use superscriptaddress for long
% author lists, or if there are many overlapping affiliations.
% For Phys. Rev. appearance, change preprint to twocolumn.
% Choose pra, prb, prc, prd, pre, prl, prstab, or rmp for journal
%  Add 'draft' option to mark overfull boxes with black boxes
%  Add 'showpacs' option to make PACS codes appear
%  Add 'showkeys' option to make keywords appear
%\documentclass[aps,prl,preprint,groupedaddress]{revtex4}
%\documentclass[aps,prl,twocolumn,superscriptaddress,showpacs,longbibliography]{revtex4}  % for PRX
%\documentclass[aps,prl,twocolumn,superscriptaddress,showpacs]{revtex4}  % for PRL
%\documentclass[aps,prl,superscriptaddress]{revtex4-2}  % for PRL



%\documentclass[aps,prl,twocolumn,groupedaddress]{revtex4}

% You should use BibTeX and apsrev.bst for references
% Choosing a journal automatically selects the correct APS
% BibTeX style file (bst file), so only uncomment the line
% below if necessary.
%\bibliographystyle{apsrev}
%\usepackage{graphicx}
%\usepackage{epstopdf}
%\usepackage{color}
%\definecolor{orange}{RGB}{255,127,0}
%\definecolor{blue2}{RGB}{33,114,173}
%\begin{document}
% Use the \preprint command to place your local institutional report
% number in the upper righthand corner of the title page in preprint mode.
% Multiple \preprint commands are allowed.
% Use the 'preprintnumbers' class option to override journal defaults
% to display numbers if necessary
%\preprint{}
%Title of paper
%\title{Supporting Information \\~\\ Layer number and stacking order-dependent thermal transport in molybdenum disulfide with sulfur vacancies}
%\author{Ranjuna M K}
%\email[]{ranjuna.mk@gmail.com}
%\author{Jayakumar Balakrishnan}
%\email[]{jayakumar@iitpkd.ac.in}
%\homepage[]{Your web page}
%\thanks{}
%\altaffiliation{}
%\affiliation{Department of Physics, Indian Institute of Technology Palakkad, Palakkad-678623, Kerala, India}
%\maketitle

\section{S1. Layer number and stacking order}
Raman measurements are carried out in a HORIBA LabRAM HR Evolution Raman spectrophotometer setup using an Nd-YAG laser of wavelength 532 nm off-resonance excitation and grating of 1800 lines/mm. All the Raman modes are fitted with Lorentzian function to extract spectral parameters. The relative separation of E$^1_{2g}$ mode to A$_{1g}$ mode increases as the layer number increases for the first few-layers. 

% Figure environment removed


% Figure environment removed

The position of shear mode for bilayer and trilayer MoS$_2$ is $\sim$22 cm$^{-1}$ and $\sim$28 cm$^{-1}$, respectively. The 3R polytype has lower frequency of the layer breathing mode compared to 2H polytype and is mainly due to the lower interlayer coupling strength in 3R stacking.\cite{van2019stacking} Additionally, the integrated intensity ratios of the layer breathing mode (LBM) to the shear mode (SM) strongly depends on the stacking order and interlayer interaction strength. I(LBM)/I(SM) of 3R polytype is nearly 5-6 times that of the 2H polytype. 
%\newpage
%==================================================
\section{S2. Stoichiometry determination}

SEM-EDS mapping performed on different multilayer flakes on the Si/SiO$_2$ substrate. XPS measurement carried out with $\sim$6 mm diameter Mg K$\alpha$ radiation (1253.6eV).
% Figure environment removed

%===========================================

\section{S3. Raman measurements and thermometry}
 For all temperature-dependent measurements, excitation laser power is kept below 200 $\mu$W to avoid local heating. To perform measurements at higher temperatures a Linkam HFS600E-PB4 temperature-controller stage is used. Raman measurements are performed over the spectral range of 10-600 cm$^{-1}$. All laser power dependent measurements are carried out using a step variable neutral density filter at ambient conditions. 
%---------------------------------------------------
\subsection{S3. 1. Temperature coefficient of high-frequency Raman modes}
The FWHM and first-order temperature coefficient of high frequency Raman modes obtained for different samples are summarized in Figure \ref{TempCoeffHFMs} The $\chi_T$ value of 2H and 3R stacked bilayer MoS$_2$ matches well.
%The FWHM and first-order temperature coefficient of A$_{1g}$ mode shows opposite trend as layer number increases. The FWHM variation (at room-temperature) with layer number is provided for comparison. 

% Figure environment removed
\newpage
%-----------------------------------------------
\subsection{S3. 2. Temperature coefficient of low-frequency Raman modes}
The first-order temperature coefficients of low frequency Raman modes obtained for different samples are summarized in Table \ref{TempCoeffLFMs}. The $\chi_T$ values of shear modes of 2H stacked bilayer is much smaller than that of 3R stacked bilayer. This variation can be attributed to the difference in the coupling strength of near by layers in different stacking configurations. For trilayer samples both SM and LBM has the same phonon frequency. The temperature coefficient of layer breathing mode and higher orders of low-frequency modes of few-layer samples are not obtained due to less access to ultra-low frequencies.

\begin{table}[ht]
\renewcommand{\thetable}{SI}
	\caption{\label{TempCoeffLFMs}Temperature coefficients of Low-frequency interlayer modes of bi-, tri and few-layer MoS$_2$ crystals on $ \text{SiO}_2/ \text{Si}$ substrate}
%	\scalebox{0.5\textwidth}{
\begin{ruledtabular}
	\begin{tabular}{ccc}
%	\hline
& \multicolumn{2}{c}{First-order temperature coefficient $\chi_T$, (cm$^{-1}$/K) } \\
	Sample &  Shear mode (SM) & Layer breathing mode (LBM)\\
 %[0.2em]
	\hline
	2L (3R) MoS$_2$ & -0.00956 $\pm$ 0.00082 & -0.0049 $\pm$ 0.0018 \\
%        \hline 
 	2L (2H) MoS$_2$ &  -0.00506 $\pm$ 0.00046 &  -0.0051 $\pm$ 0.0027 \\
%        \hline
	3L (2H) MoS$_2$ & -0.00411 $\pm$ 0.00055 &   \\ 
 	FL MoS$_2$ & -0.00417 $\pm$ 0.00030 &  \\ 
%    \hline
    \end{tabular}
\end{ruledtabular}
%}
\end{table}

In 3R polytype the observed temperature coefficient of layer breathing mode is smaller than that of shear mode. This might be due to the absence of contribution from thermal expansion in layer breathing mode.
\newpage
%---------------------------------------------------
\subsection{S3. 3. Laser power dependent Raman measurements}
The power coefficients of E$^1_{2g}$ mode is less sensitive to variations in layer number. The magnitude of power coefficient of E$^1_{2g}$ and A$_{1g}$ Raman mode decreases with layer number as given in Figure \ref{PowerCoeffHFMs}. 

% Figure environment removed
%\newpage


%=================================================
\section{S4. Thermal transport modelling and estimation of $\kappa$ and $g$}

\subsection{S4. 1. Heat conduction model}
% Figure environment removed

A laser beam with Gaussian intensity distribution is used for local heating of the sample. Under steady-state the local heating is compensated by the conduction of heat to surrounding. The differential equation in cylindrical coordinates corresponding to heat diffusion in the 2D material and across the 2D material-substrate interface is\cite{cai2010thermal}
%Equation (\ref{Eq1_heatequation})

\begin{equation}
\label{Eq1_heatequation}
\frac{1}{r}\frac{d}{dr}\biggl(r\frac{dT(r)}{dr}\biggr) - \frac{g}{\kappa t}(T(r)-T_{{a}}) + \frac{{q}_{0}^{\prime \prime}}{\kappa t }\exp\biggl(-\frac{r^{2}}{r_{0}^{ 2}} \biggr) = 0
\end{equation}

The temperature distribution $T(r)$ in the plane of 2D material has radial symmetry and Gaussian profile. Through out the thickness ($t$) of the 2D material same temperature profile is assumed. The substrate is assumed to be at ambient temperature ($T_a)$ and acts as a heat sink. %Under steady-state the local heating is compensated by the conduction of heat to surrounding.
The temperature rise in the 2D crystal is defined as $  \Theta(r) = T(r) - T_{a} $, and the average temperature rise is given by 

\begin{equation}\label{Eq2}
\Theta_{{m}}(\kappa_{s}, g, r_0,Q) =  \frac{\int_{0}^{\infty}\Theta(r)\exp\biggl(-\frac{r^{2}}{r_{0}^{ 2}} \biggr)rdr }{\int_{0}^{\infty}\exp\biggl(-\frac{r^{2}}{r_{0}^{ 2}} \biggr)rdr }
\end{equation}
where  $Q={q}_{0}^{\prime \prime}\, \pi r_0^2 $ is the total absorbed power and $r_0 $ is the laser beam radius. The thermal resistance is defined as $R_{m}=\Theta_{m}$/$Q$. The optical absorption coefficients of mono-, bi- and tri-layer samples are taken as 5.8\%, 12.1\% and 17\%, respectively. The layer thickness of mono-, bi- and tri-layer samples used for the calculations are 0.65 nm, 1.4 nm and 2.4 nm, respectively. Since the convection through air accounts for less than $0.15\%$ of the total heat conduction, its effect is ignored in the estimation of thermal parameters.\cite{zhang2015measurement}
%-------------------------------------------------------------

\subsection{S4. 2. Estimation of $\kappa$ and $g$}

To determine the in-plane thermal conductivity ($\kappa$) and the interface conductance per unit area across the 2D material-substrate interface ($g$), we used the boundary conditions proposed by Cai et al.\cite{cai2010thermal} and solved Equation (\ref{Eq2}) using numerical integration; Estimated parameters are given in Table I in the manuscript.   

To confirm further thermal parameters are calculated by using the boundary conditions and analytical solution proposed by Goushehgir \cite{goushehgir2021simple}; Estimated parameters are given in Table SII. These values are in good agreement with the values obtained from numerical integration. When MoS$_2$ is supported on a substrate with a silicon dioxide (SiO$_2$) top layer, the interfacial thermal resistance at the MoS$_2$-substrate interface can become a significant factor in determining the overall thermal conductivity of the system. 

\begin{table}[ht]
\renewcommand{\thetable}{SII}
	\caption{\label{Table1}{In-plane thermal conductivity and interface thermal conductance (at room-temperature) of MoS$_2$ crystals}}
%	\scalebox{0.95}{
\begin{ruledtabular}
	\begin{tabular}{ccc}
%	\hline
	Sample &  $\kappa $ (W/mK) & $g $ (MW/m$^2$K)\\[0.2em]
	\hline
        1L &   40 +8/-6 & 1.02 $\pm$ 0.03 \\
%        \hline
	2L (2H)  & 33 +10/-7   & 1.09 +0.04/-0.05 \\
%	\hline
	2L (3R)  & 32 +13/-10  & 1.07 +0.13/-0.15\\
%        \hline
	3L (2H)& 28 $\pm$6 & 1.23 $\pm$0.05\\ 
%    \hline
    \end{tabular}
\end{ruledtabular}
%}
\end{table}

%Absorption values used. Uncertainty in $\kappa$ and $g$  coming from uncertainty in absorption values
\newpage
%----------------------------------------------------------
\subsection{S4. 3. Layer number and stacking order dependence of thermal properties}
Variation of thermal parameters with layer number and stacking order given in Figure \ref{ThermalParameters}.
% Figure environment removed
%\newpage
%---------------------------------------------
\subsection{S4. 4. Substrate heating}
Variation of Si peak position with stage temperature and laser power is given in Figure \ref{Si_Temp_power} (a) and (b). No measurable shift in Si peak position is observed in the local laser heating. Due to high specific heat of SiO$_2$ layer it acts as a good heat sink and results in negligible heating of SiO$_2$ and underlying Si. 

% Figure environment removed
\newpage
%================================================
\section{S5. Laser beam size determination}
The laser beam radius ($ r_0 $) is defined as the distance from the center of the beam at which intensity reduces to $1/e$ times its peak value. Laser beam size for both 50x and 100x objectives are measured by modified knife-edge method.\cite{taube2015temperature} A thin film of gold with sharp-edges is deposited on a Si/SiO$ _2 $ substrate. Linear Raman mapping is performed across the sharp edge of the gold film to detect the Raman signal of silicon. The acquisition time set as 2s, accumulations 2 and spectral range 420 cm$ ^{-1} $ to 600 cm$ ^{-1} $. Figure \ref{SpotSize} (a) shows the laser beam spot size measurement set-up and the optical image of the sharp edge. Figure \ref{SpotSize} (b) shows the normalized intensity profile of the Si peak at different locations across the sharp edge measured using 50x and 100x objectives. The normalized intensity profile is fitted with the function: 

 \begin{equation}\label{eq_intensityfit}
I(x) = \frac{I_0}{2}\biggl(1+erf\biggl(\frac{x-x_0}{w}\biggr)\biggr)
\end{equation}

The fitted normalized intensity profile is differentiated with respect to the distance x. The $ dI/dx  $ data has a Gaussian behavior with a functional form $ A . \exp{(-\frac{(x-x_0)^2}{w^2})} $. Since the source term in the heat conduction equation has the form $ \exp{(-\frac{r^2}{r_0^2})} $, the $w$ value can be identified as the laser beam radius, $ r_0 $. The measurements are repeated and the mean value for each objective is used for the estimation of thermal transport parameters.

% Figure environment removed
%\newpage
%==================================================

%\bibliography{MoS2ThermalSupporting.bib}

%\begin{bibliography}{99}
%\bibitem{van2019stacking} Van Baren, Jeremiah and Ye, Gaihua and Yan, Jia-An and Ye, Zhipeng and Rezaie, Pouyan and Yu, Peng and Liu, Zheng and He, Rui and Lui, Chun Hung, "Stacking-dependent interlayer phonons in 3R and 2H MoS$_2$, \textit{2D Materials} \textbf{6}, 025022 (2019).
%%
%\bibitem{cai2010thermal} Cai, Weiwei and Moore, Arden L and Zhu, Yanwu and Li, Xuesong and Chen, Shanshan and Shi, Li and Ruoff, Rodney S, "Thermal transport in suspended and supported monolayer graphene grown by chemical vapor deposition", \textit{Nano Letters} \textbf{10}, 1645(2010).
%%
%\bibitem{zhang2015measurement} Zhang, Xian and Sun, Dezheng and Li, Yilei and Lee, Gwan-Hyoung and Cui, Xu and Chenet, Daniel and You, Yumeng and Heinz, Tony F and Hone, James C, "Measurement of lateral and interfacial thermal conductivity of single-and bilayer MoS2 and MoSe2 using refined optothermal Raman technique", \textit{ACS applied materials \& interfaces} \textbf{7}, 25923(2015).
%%
%\bibitem{goushehgir2021simple} Goushehgir, SM Hossein, "Simple exact analytical solution of laser-induced thermal transport in supported 2D materials", \textit{International Communications in Heat and Mass Transfer} \textbf{128}, 105592(2021).
%%
%\bibitem{taube2015temperature} Taube, Andrzej and Judek, Jaros{\l}aw and {\L}apinska, Anna and Zdrojek, Mariusz, "Temperature-dependent thermal properties of supported MoS2 monolayers", \textit{ACS applied materials \& interfaces} \textbf{7}, 5061(2015).

%\end{document}
%%%%%%%%%%%%%%%%%%%%%%%%%%%%%%%%%%%%%%%%%%%%%%%%%%%%%%%%%%%%%%%

\bibliography{MoS2Thermal.bib,MoS2ThermalSupporting.bib}
\end{document}
%
% ****** End of file apstemplate.tex ******

