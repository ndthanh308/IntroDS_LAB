%% ****** Start of file template.aps ****** %
%%
%%
%%   This file is part of the APS files in the REVTeX 4 distribution.
%%   Version 4.0 of REVTeX, August 2001
%%
%%
%%   Copyright (c) 2001 The American Physical Society.
%%
%%   See the REVTeX 4 README file for restrictions and more information.
%%
%
% This is a template for producing manuscripts for use with REVTEX 4.0
% Copy this file to another name and then work on that file.
% That way, you always have this original template file to use.
%
% Group addresses by affiliation; use superscriptaddress for long
% author lists, or if there are many overlapping affiliations.
% For Phys. Rev. appearance, change preprint to twocolumn.
% Choose pra, prb, prc, prd, pre, prl, prstab, or rmp for journal
%  Add 'draft' option to mark overfull boxes with black boxes
%  Add 'showpacs' option to make PACS codes appear
%  Add 'showkeys' option to make keywords appear
%\documentclass[aps,prl,preprint,groupedaddress]{revtex4}
%\documentclass[aps,prl,twocolumn,superscriptaddress,showpacs,longbibliography]{revtex4}  % for PRX
%\documentclass[aps,prl,twocolumn,superscriptaddress,showpacs]{revtex4}  % for PRL
%\documentclass[aps,prl,superscriptaddress]{revtex4-2}  % for PRL



%\documentclass[aps,prl,twocolumn,groupedaddress]{revtex4}

% You should use BibTeX and apsrev.bst for references
% Choosing a journal automatically selects the correct APS
% BibTeX style file (bst file), so only uncomment the line
% below if necessary.
%\bibliographystyle{apsrev}
%\usepackage{graphicx}
%\usepackage{epstopdf}
%\usepackage{color}
%\definecolor{orange}{RGB}{255,127,0}
%\definecolor{blue2}{RGB}{33,114,173}

%\begin{document}
% Use the \preprint command to place your local institutional report
% number in the upper righthand corner of the title page in preprint mode.
% Multiple \preprint commands are allowed.
% Use the 'preprintnumbers' class option to override journal defaults
% to display numbers if necessary
%\preprint{}
%Title of paper
%%%%%%%%%%%%%%%%%%%%%%%%%%%%%%%%%%%%%%%%%%%%%%%%%%%%%%%%%%%%%%%%%%%%%
\title{Supporting Information \\~\\ Layer number and stacking order-dependent thermal transport in molybdenum disulfide with sulfur vacancies}
%\author{Ranjuna M K}
%\email[]{ranjuna.mk@gmail.com}
%\author{Jayakumar Balakrishnan}
%\email[]{jayakumar@iitpkd.ac.in}
%\homepage[]{Your web page}
%\thanks{}
%\altaffiliation{}
%\affiliation{Department of Physics, Indian Institute of Technology Palakkad, Palakkad-678623, Kerala, India}
\maketitle

\section{S1. Layer number and stacking order}
Raman measurements are carried out in a HORIBA LabRAM HR Evolution Raman spectrophotometer setup using an Nd-YAG laser of wavelength 532 nm off-resonance excitation and grating of 1800 lines/mm. All the Raman modes are fitted with Lorentzian function to extract spectral parameters. The relative separation of E$^1_{2g}$ mode to A$_{1g}$ mode increases as the layer number increases for the first few-layers. 

% Figure environment removed


% Figure environment removed

The position of shear mode for bilayer and trilayer MoS$_2$ is $\sim$22 cm$^{-1}$ and $\sim$28 cm$^{-1}$, respectively. The 3R polytype has lower frequency of the layer breathing mode compared to 2H polytype and is mainly due to the lower interlayer coupling strength in 3R stacking.\cite{van2019stacking} Additionally, the integrated intensity ratios of the layer breathing mode (LBM) to the shear mode (SM) strongly depends on the stacking order and interlayer interaction strength. I(LBM)/I(SM) of 3R polytype is nearly 5-6 times that of the 2H polytype. 

%%%%%%%%%%%%%%%%%%%%====================%%%%%%%%%%%%%%%%%%%%%%%%%

\section{S2. Raman measurements and thermometry}
 For all temperature-dependent measurements, excitation laser power is kept below 200 $\mu$W to avoid local heating. To perform measurements at higher temperatures a Linkam HFS600E-PB4 temperature-controller stage is used. Raman measurements are performed over the spectral range of 10-600 cm$^{-1}$. All laser power dependent measurements are carried out using a step variable neutral density filter at ambient conditions. 
%%%%%%%%%%%%%%%%%%----------------------%%%%%%%%%%%%%%%%%%%%%
\subsection{S2. 1. Temperature coefficient of low-frequency Raman modes}
The first-order temperature coefficients of low frequency Raman modes obtained for different samples are summarized in Table \ref{TempCoeffLFMs}. For trilayer samples both SM and LBM has the same phonon frequency. The temperature coefficient of layer breathing mode and higher orders of low-frequency modes of few-layer samples are not obtained due to less access to ultra-low frequencies.

\begin{table}[ht]
\renewcommand{\thetable}{SI}
	\caption{\label{TempCoeffLFMs}Temperature coefficients of Low-frequency interlayer modes of bi-, tri and few-layer MoS$_2$ crystals on $ \text{SiO}_2/ \text{Si}$ substrate}
%	\scalebox{0.5\textwidth}{
\begin{ruledtabular}
	\begin{tabular}{ccc}
%	\hline
& \multicolumn{2}{c}{First-order temperature coefficient $\chi_T$, (cm$^{-1}$/K) } \\
	Sample &  Shear mode (SM) & Layer breathing mode (LBM)\\
 %[0.2em]
	\hline
	2L (3R) MoS$_2$ & -0.0096 $\pm$ 0.0008 & -0.005 $\pm$ 0.002 \\
%        \hline 
 	2L (2H) MoS$_2$ &  -0.0051 $\pm$ 0.0005 &  -0.005 $\pm$ 0.003 \\
%        \hline
	3L (2H) MoS$_2$ & -0.0041 $\pm$ 0.0006 &  - \\ 
 	FL MoS$_2$ & -0.0042 $\pm$ 0.0003 & - \\ 
%    \hline
    \end{tabular}
\end{ruledtabular}
%}
\end{table}

%%%%%%%%%%%-----------------------%%%%%%%%%%%%%%%
\subsection{S2. 2. Laser power dependent Raman measurements}
The power coefficients of E$^1_{2g}$ mode is less sensitive to variations in layer number. The magnitude of power coefficient of E$^1_{2g}$ and A$_{1g}$ Raman mode decreases with layer number as given in Figure \ref{PowerCoeffHFMs}. 

% Figure environment removed

%%%%%%%%%%%%%%%%%%===============%%%%%%%%%%%%%%%%%%%%%%%%%
\section{S3. Estimation of $\kappa$ and $g$}

To determine the in-plane thermal conductivity ($\kappa$) and the interface conductance per unit area across the 2D material-substrate interface ($g$), we used the boundary conditions proposed by Cai et al.\cite{cai2010thermal} and solved Equation (2) in the manuscript using numerical integration; Estimated parameters are given in Table I in the manuscript.   

To confirm further thermal parameters are calculated by using the boundary conditions and analytical solution proposed by Goushehgir \cite{goushehgir2021simple}; Estimated parameters are given in Table SII. These values are in good agreement with the values obtained from numerical integration. When MoS$_2$ is supported on a substrate with a silicon dioxide (SiO$_2$) top layer, the interfacial thermal resistance at the MoS$_2$-substrate interface can become a significant factor in determining the overall thermal conductivity of the system. 

\begin{table}[ht]
\renewcommand{\thetable}{SII}
	\caption{\label{Table1}{In-plane thermal conductivity and interface thermal conductance (at room-temperature) of MoS$_2$ crystals}}
%	\scalebox{0.95}{
\begin{ruledtabular}
	\begin{tabular}{ccc}
%	\hline
	Sample &  $\kappa $ (W/mK) & $g $ (MW/m$^2$K)\\[0.2em]
	\hline
        1L &   40 +8/-6 & 1.02 $\pm$ 0.03 \\
%        \hline
	2L (2H)  & 33 +10/-7   & 1.09 +0.04/-0.05 \\
%	\hline
	2L (3R)  & 32 +13/-10  & 1.07 +0.13/-0.15\\
%        \hline
	3L (2H)& 28 $\pm$6 & 1.23 $\pm$0.05\\ 
%    \hline
    \end{tabular}
\end{ruledtabular}
%}
\end{table}

%----------------------------------------------------------
\section{S4. Layer number and stacking order dependence of thermal properties}
Variation of thermal parameters with layer number and stacking order given in Figure \ref{ThermalParameters}.
% Figure environment removed
%---------------------------------------------
\section{S5. Substrate heating}
Variation of Si peak position with stage temperature and laser power is given in Figure \ref{Si_Temp_power} (a) and (b). No measurable shift in Si peak position is observed in the local laser heating. Due to high specific heat of SiO$_2$ layer it acts as a good heat sink and results in negligible heating of SiO$_2$ and underlying Si. 

% Figure environment removed
\newpage
%%%%%%%%%%%%%%%%%%%%%%%%=======%%%%%%%%%%%%%%%%%
\section{S6. Laser beam size determination}
The laser beam radius ($ r_0 $) is defined as the distance from the center of the beam at which intensity reduces to $1/e$ times its peak value. Laser beam size for both 50x and 100x objectives are measured by modified knife-edge method.\cite{taube2015temperature} A thin film of gold with sharp-edges is deposited on a Si/SiO$ _2 $ substrate. Linear Raman mapping is performed across the sharp edge of the gold film to detect the Raman signal of silicon. The acquisition time set as 2s, accumulations 2 and spectral range 420 cm$ ^{-1} $ to 600 cm$ ^{-1} $. Figure \ref{SpotSize} (a) shows the laser beam spot size measurement set-up and the optical image of the sharp edge. Figure \ref{SpotSize} (b) shows the normalized intensity profile of the Si peak at different locations across the sharp edge measured using 50x and 100x objectives. The normalized intensity profile is fitted with the function: 

 \begin{equation}\label{eq_intensityfit}
I(x) = \frac{I_0}{2}\biggl(1+erf\biggl(\frac{x-x_0}{w}\biggr)\biggr)
\end{equation}

The fitted normalized intensity profile is differentiated with respect to the distance x. The $ dI/dx  $ data has a Gaussian behavior with a functional form $ A . \exp{(-\frac{(x-x_0)^2}{w^2})} $. Since the source term in the heat conduction equation has the form $ \exp{(-\frac{r^2}{r_0^2})} $, the $w$ value can be identified as the laser beam radius, $ r_0 $. The measurements are repeated and the mean value for each objective is used for the estimation of thermal transport parameters.

% Figure environment removed
\newpage

\section{S7. Transmission Electron Microscopy characterization}

Transmission Electron Microscopy (TEM) experiments were conducted on our MoS$_2$ samples before annealing to determine the inherent sulfur vacancy concentration. MoS$_2$ flakes were dispersed in ethanol and transferred to a carbon film supported copper grid. Figure \ref{fig:TEMresults}(a) shows a low-magnification image of the suspended flakes, while Figure \ref{fig:TEMresults}(b) presents a high-resolution image of a flake, with uniformly distributed bright spots corresponding to Mo atoms. The inset displays a zoomed image of the region in the white rectangular box. The Selected Area Electron Diffraction (SAED) pattern in Figure \ref{fig:TEMresults}(c) illustrates the high crystalline nature and trigonal prismatic coordination of the MoS$_2$ crystal. Additionally, the Transmission Electron Microscopy - Energy Dispersive Spectroscopy (TEM-EDS) spectrum in Figure \ref{fig:TEMresults}(d) and corresponding atomic percentages confirm the Mo:S ratio as 1:1.80, providing evidence for the presence of sulfur vacancies in the samples. The additional peaks for Cu, C, and O originate from the TEM grid.\\

 % Figure environment removed
\newpage

%%%%%%%===================%%=============%%%%%%%%%%
%\bibliography{MoS2ThermalSupporting.bib}
%\begin{thebibliography}{48}%
%	\makeatletter
%	\providecommand \@ifxundefined [1]{%
%		\@ifx{#1\undefined}
%	}%
%	\providecommand \@ifnum [1]{%
%		\ifnum #1\expandafter \@firstoftwo
%		\else \expandafter \@secondoftwo
%		\fi
%	}%
%	\providecommand \@ifx [1]{%
%		\ifx #1\expandafter \@firstoftwo
%		\else \expandafter \@secondoftwo
%		\fi
%	}%
%	\providecommand \natexlab [1]{#1}%
%	\providecommand \enquote  [1]{``#1''}%
%	\providecommand \bibnamefont  [1]{#1}%
%	\providecommand \bibfnamefont [1]{#1}%
%	\providecommand \citenamefont [1]{#1}%
%	\providecommand \href@noop [0]{\@secondoftwo}%
%	\providecommand \href [0]{\begingroup \@sanitize@url \@href}%
%	\providecommand \@href[1]{\@@startlink{#1}\@@href}%
%	\providecommand \@@href[1]{\endgroup#1\@@endlink}%
%	\providecommand \@sanitize@url [0]{\catcode `\\12\catcode `\$12\catcode
%		`\&12\catcode `\#12\catcode `\^12\catcode `\_12\catcode `\%12\relax}%
%	\providecommand \@@startlink[1]{}%
%	\providecommand \@@endlink[0]{}%
%	\providecommand \url  [0]{\begingroup\@sanitize@url \@url }%
%	\providecommand \@url [1]{\endgroup\@href {#1}{\urlprefix }}%
%	\providecommand \urlprefix  [0]{URL }%
%	\providecommand \Eprint [0]{\href }%
%	\providecommand \doibase [0]{https://doi.org/}%
%	\providecommand \selectlanguage [0]{\@gobble}%
%	\providecommand \bibinfo  [0]{\@secondoftwo}%
%	\providecommand \bibfield  [0]{\@secondoftwo}%
%	\providecommand \translation [1]{[#1]}%
%	\providecommand \BibitemOpen [0]{}%
%	\providecommand \bibitemStop [0]{}%
%	\providecommand \bibitemNoStop [0]{.\EOS\space}%
%	\providecommand \EOS [0]{\spacefactor3000\relax}%
%	\providecommand \BibitemShut  [1]{\csname bibitem#1\endcsname}%
%	\let\auto@bib@innerbib\@empty
%	%</preamble>	%
%	\bibitem [{\citenamefont {Van~Baren}\ \emph {et~al.}(2019)\citenamefont {Van~Baren}, \citenamefont {Ye}, \citenamefont {Yan}, \citenamefont {Ye}, \citenamefont {Rezaie}, \citenamefont {Yu}, \citenamefont {Liu}, \citenamefont {He},\ and\ \citenamefont {Lui}}]{van2019stacking}%
%  \BibitemOpen
%  \bibfield  {author} {\bibinfo {author} {\bibfnamefont {J.}~\bibnamefont {Van~Baren}}, \bibinfo {author} {\bibfnamefont {G.}~\bibnamefont {Ye}}, \bibinfo {author} {\bibfnamefont {J.-A.}\ \bibnamefont {Yan}}, \bibinfo {author} {\bibfnamefont {Z.}~\bibnamefont {Ye}}, \bibinfo {author} {\bibfnamefont {P.}~\bibnamefont {Rezaie}}, \bibinfo {author} {\bibfnamefont {P.}~\bibnamefont {Yu}}, \bibinfo {author} {\bibfnamefont {Z.}~\bibnamefont {Liu}}, \bibinfo {author} {\bibfnamefont {R.}~\bibnamefont {He}},\ and\ \bibinfo {author} {\bibfnamefont {C.~H.}\ \bibnamefont {Lui}},\ }\bibfield  {title} {\bibinfo {title} {Stacking-dependent interlayer phonons in 3r and 2h mos2},\ }\href@noop {} {\bibfield  {journal} {\bibinfo  {journal} {2D Materials}\ }\textbf {\bibinfo {volume} {6}},\ \bibinfo {pages} {025022} (\bibinfo {year} {2019})}\BibitemShut {NoStop}%
% %%%%%%%%%%%%%%%%%%%%%%%%%%%%%%%%%%%%%%%%%%%%%%%%%%%%%%%%%%%%%%%%%%%%%%%%%%% 
%	\bibitem [{\citenamefont {Cai}\ \emph {et~al.}(2010)\citenamefont {Cai},
%		\citenamefont {Moore}, \citenamefont {Zhu}, \citenamefont {Li}, \citenamefont
%		{Chen}, \citenamefont {Shi},\ and\ \citenamefont {Ruoff}}]{cai2010thermal}%
%	\BibitemOpen
%	\bibfield  {author} {\bibinfo {author} {\bibfnamefont {W.}~\bibnamefont
%			{Cai}}, \bibinfo {author} {\bibfnamefont {A.~L.}\ \bibnamefont {Moore}},
%		\bibinfo {author} {\bibfnamefont {Y.}~\bibnamefont {Zhu}}, \bibinfo {author}
%		{\bibfnamefont {X.}~\bibnamefont {Li}}, \bibinfo {author} {\bibfnamefont
%			{S.}~\bibnamefont {Chen}}, \bibinfo {author} {\bibfnamefont {L.}~\bibnamefont
%			{Shi}},\ and\ \bibinfo {author} {\bibfnamefont {R.~S.}\ \bibnamefont
%			{Ruoff}},\ }\bibfield  {title} {\bibinfo {title} {Thermal transport in
%			suspended and supported monolayer graphene grown by chemical vapor
%			deposition},\ }\href@noop {} {\bibfield  {journal} {\bibinfo  {journal} {Nano
%				Letters}\ }\textbf {\bibinfo {volume} {10}},\ \bibinfo {pages} {1645}
%		(\bibinfo {year} {2010})}\BibitemShut {NoStop}%
%		
%\bibitem [{\citenamefont {Goushehgir}(2021)}]{goushehgir2021simple}%
%	\BibitemOpen
%	\bibfield  {author} {\bibinfo {author} {\bibfnamefont {S.~H.}\ \bibnamefont
%			{Goushehgir}},\ }\bibfield  {title} {\bibinfo {title} {Simple exact
%			analytical solution of laser-induced thermal transport in supported 2d
%			materials},\ }\href@noop {} {\bibfield  {journal} {\bibinfo  {journal}
%			{International Communications in Heat and Mass Transfer}\ }\textbf {\bibinfo
%			{volume} {128}},\ \bibinfo {pages} {105592} (\bibinfo {year}
%		{2021})}\BibitemShut {NoStop}%		
%
%\bibitem [{\citenamefont {Taube}\ \emph {et~al.}(2015)\citenamefont {Taube},
%		\citenamefont {Judek}, \citenamefont {{\L}apinska},\ and\ \citenamefont
%		{Zdrojek}}]{taube2015temperature}%
%	\BibitemOpen
%	\bibfield  {author} {\bibinfo {author} {\bibfnamefont {A.}~\bibnamefont
%			{Taube}}, \bibinfo {author} {\bibfnamefont {J.}~\bibnamefont {Judek}},
%		\bibinfo {author} {\bibfnamefont {A.}~\bibnamefont {{\L}apinska}},\ and\
%		\bibinfo {author} {\bibfnamefont {M.}~\bibnamefont {Zdrojek}},\ }\bibfield
%	{title} {\bibinfo {title} {Temperature-dependent thermal properties of
%			supported mos2 monolayers},\ }\href@noop {} {\bibfield  {journal} {\bibinfo
%			{journal} {ACS Applied Materials \& Interfaces}\ }\textbf {\bibinfo {volume}
%			{7}},\ \bibinfo {pages} {5061} (\bibinfo {year} {2015})}\BibitemShut
%	{NoStop}%
%\end{thebibliography}		
%\end{document}
%%%%%%%%%%%%%%%%%%%%%%%%%%%%%%%%%%%%%%%%%%%%%%%%%%%%%%%%%%%%%%%
