%_____ Math envs ____________________________________________________________

\newcommand\numberthis{\addtocounter{equation}{1}\tag{\theequation}}
\newtheorem{theorem}{Theorem}[section]
\newtheorem{lemma}{Lemma}[section]
\newtheorem{proposition}{Proposition}[section]
\newtheorem{definition}{Definition}[section]
\newtheorem{conj}{Conjecture}[section]

%_____ Wick ____________________________________________________________

%\usepackage{wick}
%\newcommand{\wickbold}{}% Check that \wickbold is undefined
%\protected\def\wickbold#1{\bm{#1}}
%% example 
%%\wick{1}{<1 {\wickbold{x}} >1 {\wickbold{y}}}


%_____ Colors ____________________________________________________________

\newcommand{\red}[1]{\textcolor{red}{#1}}
\newcommand{\magenta}[1]{\textcolor{magenta}{#1}}
\newcommand{\blue}[1]{\textcolor{blue}{#1}}
\newcommand{\green}[1]{\textcolor{green}{#1}}

\definecolor{myred}{RGB}{233, 33, 45}

% full list at https://en.wikibooks.org/wiki/LaTeX/Colors
\newcommand{\colemph}[1]{\textcolor{Mahogany}{#1}} % alternative is NavyBlue
\newcommand{\colbad}[1]{\textcolor{red}{#1}}
\newcommand{\colgood}[1]{\textcolor{PineGreen}{#1}}


%_____ Matteo's commands _________________________________________________

\newcommand{\bs}{\begin{shaded}}
\newcommand{\es}{\end{shaded}\noindent}
\def\ba#1\ea{\begin{align}#1\end{align}}		% very clever way to bypass the known problem...
\newcommand{\be}{\begin{equation}}
\newcommand{\ee}{\end{equation}}
\newcommand{\mc}{\mathcal }
\newcommand{\mb}{\mathbb }
\newcommand{\la}{\label}
\newcommand{\eps}{\varepsilon}

\newcommand{\lp}{\notag \\ & }

\DeclareMathOperator{\sign}{\text{sign}}
\DeclareMathOperator{\tr}{\text{tr}}
\DeclareMathOperator{\sech}{\text{sech}}

\newcommand{\wt}{\widetilde}
\newcommand{\wl}{\widehat{\lambda}}

\newcommand{\cf}{\textit{cf.} }
\newcommand{\ie}{\textit{i.e.} }
\newcommand{\eg}{\textit{e.g.} }

\newcommand{\N}{\mathcal N}

\newcommand{\separator}{\medskip\begin{center}\red{\rule{\textwidth}{0.04cm}}\end{center}\medskip}
\newcommand{\blackseparator}{\begin{center}{\rule{\textwidth}{0.04cm}}\end{center}}
\newcommand{\fix}[1]{\textcolor{myred}{$\star$ #1 $\star$}}
\newcommand{\link}[1]{\href{https://arxiv.org/pdf/#1.pdf}{[#1]}}


%_____ Angular brackets _______________

\makeatletter
\DeclareFontFamily{OMX}{MnSymbolE}{}
\DeclareSymbolFont{MnLargeSymbols}{OMX}{MnSymbolE}{m}{n}
\SetSymbolFont{MnLargeSymbols}{bold}{OMX}{MnSymbolE}{b}{n}
\DeclareFontShape{OMX}{MnSymbolE}{m}{n}{
<-6>  MnSymbolE5
   <6-7>  MnSymbolE6
   <7-8>  MnSymbolE7
   <8-9>  MnSymbolE8
   <9-10> MnSymbolE9
  <10-12> MnSymbolE10
  <12->   MnSymbolE12
}{}
\DeclareFontShape{OMX}{MnSymbolE}{b}{n}{
<-6>  MnSymbolE-Bold5
   <6-7>  MnSymbolE-Bold6
   <7-8>  MnSymbolE-Bold7
   <8-9>  MnSymbolE-Bold8
   <9-10> MnSymbolE-Bold9
  <10-12> MnSymbolE-Bold10
  <12->   MnSymbolE-Bold12
}{}

\let\llangle\@undefined
\let\rrangle\@undefined
\DeclareMathDelimiter{\llangle}{\mathopen}%
 {MnLargeSymbols}{'164}{MnLargeSymbols}{'164}
\DeclareMathDelimiter{\rrangle}{\mathclose}%
 {MnLargeSymbols}{'171}{MnLargeSymbols}{'171}
\makeatother

%_____ Hypergeometric _______________


\newcommand*\pFqskip{8mu}
\catcode`,\active
\newcommand*\pFq{\begingroup
        \catcode`\,\active
        \def ,{\mskip\pFqskip\relax}%
        \dopFq
}
\catcode`\,12
\def\dopFq#1#2#3#4#5{%
        {}_{#1}F_{#2}\biggl[\genfrac..{0pt}{}{#3}{#4};#5\biggr]%
        \endgroup
}

%_____ cut Cauchy integral  _______________


\def\Xint#1{\mathchoice
   {\XXint\displaystyle\textstyle{#1}}%
   {\XXint\textstyle\scriptstyle{#1}}%
   {\XXint\scriptstyle\scriptscriptstyle{#1}}%
   {\XXint\scriptscriptstyle\scriptscriptstyle{#1}}%
   \!\int}
\def\XXint#1#2#3{{\setbox0=\hbox{$#1{#2#3}{\int}$}
     \vcenter{\hbox{$#2#3$}}\kern-.5\wd0}}
\def\ddashint{\Xint=}
\def\dashint{\Xint-}


%_____ curly nice paragraph separators  _______________


%\usepackage[object=vectorian]{pgfornament}						
%\newcommand{\sectionline}{%
%  \noindent
%  \begin{center}
%  {%\color{NiceMagentaColor}
%    \resizebox{0.5\linewidth}{1ex}
%    {{%
%    {\begin{tikzpicture}
%    \node  (C) at (0,0) {};
%    \node (D) at (9,0) {};
%    \path (C) to [ornament=88] (D);
%    \end{tikzpicture}}}}}%
%    \end{center}
%  }
%


