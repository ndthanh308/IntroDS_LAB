\section{Results}
\label{sec:results}


%\subsection{Bands/Dos}

% Figure environment removed


GPNL exhibits lower electric conductivity compared to
graphene due to its distinct physical properties, particularly
its formation energies and band gaps. The 
influence of pore size and quantity on these properties is 
depicted in Fig. 
\ref{fig:DOS}. Our DFTB calculations reveal that GPNL 
possesses a bandgap of 
approximately 0.96 eV, while graphene lacks a bandgap altogether.
The porosity of GPNL has the potential to significantly modify its 
electronic characteristics and catalytic performance by increasing
its surface area. The 
electronic band structures, illustrated in Fig. \ref{fig:DOS}, reveal 
that the top of 
the valence bands for carbon atoms is located at the $\Gamma$ point, 
while the bottom 
of the conduction bands is found at the $K$--point.

The band gap of GPNL is structure-dependent and can range from
zero to a few electron volts. Theoretical studies have predicted
band gaps for GPNL ranging from 0 eV to approximately 1.2 eV,
depending on the specific structure and calculation 
method employed \cite{balandin2011thermal}. 
It should be noted that the semi-local functionals tend to 
underestimate the band gaps of GPNL structures. 
It is also noticed the selected path show the characteristic
gaps at the M and $\Gamma$ point reported by DFT calculations
\cite{C2TC00006G}.
To validate our findings, density functional theory (DFT) 
calculations were conducted using the PBE exchange-correlation 
functional. The calculations were carried out under periodic 
boundary conditions, and the Brillouin zone integration was 
performed with the $\Gamma$ point considered. Kohn--Sham orbitals 
were employed as plane waves up to an energy cutoff of 90 Ry to 
ensure convergence in the structural properties of the systems. 
The Quantum-ESPRESSO ab--initio package with relativistic-
corrected pseudo-potentials was utilized for computing the 
density of states, system energies, and band structures. The 
exchange-correlation energy was evaluated using the generalized 
approximation (GGA).
The total electronic density of states (DOS) for GPNL
reveals significant overlaps between the C--2s and C--2p curves, 
indicating the presence of strong sp$^3$ hybridized covalent
bonding states. GPNL consists of interconnected benzene 
rings arranged in a hexagonal lattice, similar to graphene. 
These results demonstrate that the SCC-DFTB method can 
effectively model graphenylene and accurately capture
its properties, showing good agreement with DFT calculations. 
This supports the application of SCC-DFTB in studying gas
separation processes, which is crucial for the production
and utilization of clean fuels. 
Different paths for the GPNL sheet was considered 
and shown in the supplementary material (SM) of this work.

%%%%%%%%%%%%%%%%%%%%%%%%%%%%%%%%%%%%%%%%%%%%%%%
\subsection{Binding Energy and Optical Absorption Spectra}

Figure \ref{fig:PECs} illustrates the binding energies as a
function of separation distance for graphene (a) and GPNL (b), considering different adsorbate sites as
labeled in the inset figure. We performed adsorption 
calculations for isolated H$_2$, CH$_4$, and CO$_2$ molecules
on both systems.
To ensure accurate calculations, we included a 50 \AA{}
vacuum section above the sample to minimize boundary effects.
Periodic boundary conditions were applied in the $x$--$y$
directions to simulate a semi-infinite surface. 
For the $k$--point sampling, we employed a $4\times4\times1$
Monkhorst--Pack set throughout all calculations.


% Figure environment removed

In the case of GPNL, the adsorbate sites associated with
the hexagonal and square holes exhibit a lower likelihood for 
molecule adsorption, with a binding energy of approximately 
-0.04 eV. On the other hand, molecules are more likely to bond
to the top of the carbon atoms, with a binding energy of -0.2 
$\pm$ 0.05 eV showing excellent conditions for gas separation for 
this energy barrier 
\cite{C2TC00006G,D0RA04286B}. It is worth noting that dispersion 
corrections
play a significant role in these calculations, particularly due
to the hybridization of the system and the presence of CH$_4$
molecules.
Similar trends are observed for graphene, where the adsorbate site
located at the top of the carbon atoms is more favorable for both 
physisorption and chemisorption mechanisms confirming reported 
results for hydrogen trapping by graphene 
\cite{10.1063/1.1329889}, 
indicating a higher likelihood of attracting molecules.

%%%%%%%%%%%%%%%%%%%%%%%%%%%%%%%%%%
% Figure environment removed


Figure \ref{fig:optical} presents the optical absorption
spectrum of graphene (a) and GPNL (b) obtained using
the Liouville--von Neumann equation. The spectra are calculated
for different cases of molecular adsorption: 
1) For hydrogen molecules, we considered a single H$_2$ 
molecule positioned above a carbon atom, as well as two hydrogen 
atoms positioned above two different carbon atoms, where the 
interaction between them is minimal; the obtained results 
are in a good agreement with experimental and ab--initio data where 
the absorption intensities of the interband transitions occurring
in the Dirac band (mid-IR and visible) 
\cite{LEE2016109,Keith,ZHANG2018137}.
2) For CO$_2$ and its split CO+O molecule, we positioned the CO$_2$ 
molecule above a carbon atom, and for the resulting CO+O system, 
we placed it above two different carbon atoms. 
The effect of carbon dioxide on the graphene sheet is similar
to that observed for the H$_2$ molecule. For GPNL, 
the influence decreases compared to the pristine case. 
However, the split of CO+O enhances the absorbance of the graphene sheet 
at the same wavelength as the CO$_2$ molecule. For the GPNL case, 
the absorbance rate is lower than that observed in the carbon
dioxide case. 
3) For methane molecules, we investigated several 
configurations: CH$_4$, CH$_2$+H$_2$ (molecular hydrogen and 
methylene as a main dissociation channel), CH$_2$+H+H, and CH$_2$,
as they are crucial for sensor development 
\cite{GUI2020113959, zhu2016theoretical}. Among these
configurations, we observed that the optical absorbance is
maximized in the 300--450 wavelength range for the 
CH$_2$+H+H case, attributed to the bonding between H 
atoms and the C atom of graphene. In contrast, in the 
case of GPNL, CH$_2$ increases the optical 
absorbance in the range of 400 to 550 wavelengths due to the 
system's hybridization.
Before performing the optical absorption calculations, we fully 
optimized the entire system.


%CH$_4$, CO$_2$ and H$_2$ represents
%there are several new peaks in visible light range 

% Describe figures


%\subsection{CH$_4$ and CO$_2$ adsorption mechanisms}
\subsection{Dynamical adsorbtion of CH$_4$ and CO$_4$ molecules}


\begin{table}[b!]
    \centering
     \caption{probabilities of different cases observed in the MD 
     simulations, including transmission, reflection, and adsorption, 
     for both graphene and GPNL. The results clearly demonstrate 
     that GPNL exhibits superior performance as a material for 
     gas separation compared to graphene.}
    \begin{tabular}{llll| lll}
    \hline
    \multicolumn{4}{c}{Graphene} & \multicolumn{3}{c}{Graphelyne} \\
    \hline
    Probability    &   H$_2$ & CO$_2$ & CH$_4$ & H$_2$ & CO$_2$ & CH$_4$ \\
    \hline
     Transmission  & 0 & 0   & 0    & 5.28   & 1.28  & 49.12 \\
     Adsorbation   & 0 & 0   & 16.8 & 0      &  9.66 & 1.60 \\
     Reflection    & 100 & 100 & 83.2 & 94.72  & 89.06 & 49.28 \\
    \hline
    \end{tabular}
    \label{tab:prob}
\end{table}



After conducting MD simulations at room temperature 
and an impact energy of 8 eV, the emission of hundreds of H$_2$, CO$_2$, 
and CH$_4$ molecules is analyzed by calculating the 
probabilities of adsorption, reflection, and transmission, which are 
defined as:
\begin{equation}
    P = 100 \times \frac{N_{x}}{N_{\rm Tot}},
\end{equation}
where $N_{\rm Tot}$ is the total number of MD simulations, while 
$N_x$ is the number of cases for adsorption, reflection, and 
transmission calculated by using the following conditions based 
on the last frame of the MD simulations: 1) Adsorbed cases: Molecules 
with a final position between a sphere centered at the material with a 
radius of 0.35 nm and the direction of the velocity vector points towards 
the surface is considered as adsorbed; 2) Reflection cases: Molecules 
with a final position larger than 0.35 nm and a velocity vector
oriented in the opposite direction
to the surface normal is counted as reflected; 3) 
Transmission cases: Molecules with a final position below the
surface and a distance larger than -0.35 nm are counted as
transmitted. 
The counts for these cases are tabulated in Table \ref{tab:prob}.

For graphene, we did not observe dissociation of H$_2$ and CO$_2$ 
molecules among the reflected molecules. To observe this mechanism, a 
higher impact energy would be required. However, it is worth noting that 
such high-energy collisions could result in the creation of vacancies in 
the graphene sheet by displacing a carbon atom. On the other hand, some 
of the reflected CH$_4$ molecules underwent dissociation, and a few of 
them were able to attach to the graphene sheet. In order to bond more 
CH$_2$ molecules to graphene, a lower impact energy would be sufficient.
In the case of GPNL, its inherent porosity allowed for a higher 
number of transmitted cases for both H$_2$ and CO$_2$ molecules. We 
observed that some carbon dioxide molecules dissociated into CO+O, with 
the CO molecules bonding to the GPNL sheet. This highlights the 
advantage of GPNL's porous structure in facilitating gas 
transmission and reactivity compared to graphene.



% Figure environment removed

From our MD simulation results, we have observed that CH$_4$ molecules 
can undergo reflection and dissociation, leading to the formation of 
CH$_2$ and H$_2$ as the main process. Figure \ref{fig:dynamics} displays 
the final frames of the simulations for graphene (a) and GPNL 
(b), providing a visual representation of the observed dynamics. In the 
figures, C atoms forming graphene and GPNL are represented as 
gray spheres, adsorbed CH$_2$ molecules are depicted as black spheres 
for C atoms and white spheres for hydrogen atoms, while reflected 
molecules are represented by turquoise spheres for molecular hydrogen 
and purple spheres for C atoms.
We have found that CH$_4$ molecules are unable to transmit through 
graphene, whereas in the case of GPNL, there is a probability of 
approximately 50\% for transmission, accompanied by dissociation 
and the 
production of molecular hydrogen. This behavior can be attributed 
to the 
porous nature of the materials and the impact energy of the 
emitted 
molecules. Additionally, we have observed that H$_2$ molecules 
travel 
faster than CH$_2$ molecules after reflection, owing to the 
distribution 
of kinetic energy.


Furthermore, we have noted that CH$_2$ molecules can form bonds 
with GPNL by infiltrating the porous structure and binding to the 
underlying C atoms. This property distinguishes GPNL from 
graphene, as the latter typically requires the presence of defects to 
enable molecule transmission. This mechanism highlights the potential of GPNL is a promising material for gas separation, particularly 
for CH$_4$, as supported by our MD simulations for CO$_2$ purification 
as well.


% Figure environment removed


Figure \ref{fig:reflected} presents the analysis of reflected molecules 
from the surfaces of graphene and GPNL at an impact energy of 8 
eV. The analysis involves determining the internuclear distance between 
atoms in the final frame of the simulations.
In the case of H$_2$ molecules (Figure \ref{fig:reflected}a), we 
observe a uniform distribution of the internuclear distance around the 
bond length of 0.74 Å. This distribution arises from the excitation of 
vibrational and rotational states due to the exchange of kinetic energy 
during a collision with the surfaces.
For CO$_2$ molecules, we observe the splitting of the molecules into CO 
and O, with a majority exhibiting a homogeneous distribution around 
1.43 Å for the internuclear distance which correspons to the CO 
bond length. Oxygen atoms are identified with 
an internuclear distance larger than 2.5 Å.
Finally, CH$_4$ molecules undergo splitting into CH$_2$ and H$_2$ 
molecules. Molecular hydrogen is characterized by an internuclear 
distance of around 0.74 Å and a higher degree of excitation in 
vibrational states. 
We identify CH$_2$ (methylene) molecules as the main dissociation
channel with a peak in the histogram at $~1.8$\AA{} being the 
internuclear distance between H atoms.
The porosity of GPNL makes it a more promising 
candidate for H$_2$ production compared to graphene, as demonstrated by 
our molecular dynamics simulations.
The histogram illustrates the distribution of reflected 
molecules with different bond lengths corresponding to 
their excited vibrational states. For instance, when 
H$_2$ molecules are emitted onto the 2D materials, 
they tend to vibrate towards states close to their 
ground state. On the other hand, H$_2$ molecules originating
from the dissociated methane exhibit various bond lengths
due to their excited vibrational and rotational states. 
This concept also applies to CO$_x$ and CH$_x$ molecules.


%%%%%%%%%%%%%%%%%%%%%%%%%%%%%%%%%%
\subsection{Electron transport and sensitivity}
\label{subsec:sensitivity}

Advancements in 2D materials have expanded the scope of potential 
applications, particularly in energy harvesting and storage, due 
to their high electron transport efficiency and extensive surface 
area with numerous active sites \cite{KUMAR2023232256}. In our 
study, we adopted a computational approach based on a $\pi$-
orbital tight-binding Hamiltonian to simulate the electrical 
transport phenomena of graphene and GPNL. We employed the 
SCC-DFTB method along with Non--equilibrium Green's functions 
\cite{Pecchia_2008}.
In the tight--binding representation, the interaction between 
atoms is limited to a finite range. 
We can solve the contact self--energy function, also known 
as the surface Green's function, for 
the matrix block corresponding to the atoms near the extended 
device region using a recursive algorithm \cite{D0CP04188B}. 
This allows us to 
accurately describe the electrical transport properties of the 
materials and analyze their behavior under various conditions.

Figure \ref{fig:transmissionProb} illustrates our findings 
that are in good agreement with reported results for 
pristine graphene in the literature and by Villegas 
et al.c\cite{D0CP04188B} for the pristine GPNL; 
by adding different molecules it is revealed the preferential 
adsorption of extrinsic chemical 
species like CH$_4$, CO$_2$, and H$_2$ at interdomain sites, 
leading to a significant enhancement of scattering effects in 
both graphene and GPNL. To delve into this intriguing 
behavior, we employed the non-pulsating direct current (NPDC) 
waveform, known for its ability to maintain the stability of 
adsorbates even under ultrahigh--vacuum conditions. Unlike PDC 
waves, which exhibit continuous voltage fluctuations, NPDC waves 
maintain a constant voltage, ensuring the stability of the 
adsorbed species within the scope of our computational approach.


% Figure environment removed


%Smith et al. \cite{smi} predicted that the band gap of 1.22 eV, similar to that of silicon, would render GPNL as a possible replacement in silicon electronic devices; for example, GPNL is a better candidate than graphene for use in field effect transistors. Froudakis investigated the possibility of using hydrogenation to tailor the electronic properties of GPNL.

Fig \ref{fig:I-U} shows the logarithm of the 
tunneling currents for graphene in a) and GPNL in b) 
as a function of the inverse of the voltage for various 
adsorbed molecules with I–-U characteristics stay linear for 
voltages below 300 mV, at all cases.
In the inset, we present results for the difference 
of the current for each molecule $X$ as a function of
the voltage is calculated as:
\begin{equation}
    S = 100 \% \frac{\big| I_{X}-I_g \big|}{I_{X}},
\end{equation}
with $I_{X}$ of each molecule and the surface and 
$I_g$ the current of the surface (graphene or graphenelyne).
Noticing that the adsorption of CH$_x$ compounds 
increases the sensitivity of the surfaces.
Our results are in a qualititive good agrement with reported 
experimental data \cite{SHABAN20194510}. Thus, graphenylene
is a better candidate than graphene for use in field effect 
transistors, as Zener diodes, and faster sensors.

% Figure environment removed
