\documentclass[%
 reprint,
superscriptaddress,
%groupedaddress,
%unsortedaddress,
%runinaddress,
%frontmatterverbose, 
%preprint,
%preprintnumbers,
%nofootinbib,
%nobibnotes,
%bibnotes,
% amsmath,amssymb,
% aps,
%pra,
prb,
%rmp,
%prstab,
%prstper,
%floatfix,
]{revtex4-1}

%\documentclass[5p,11pt,a4paper]{elsarticle}

\usepackage{xcolor}
\usepackage{lineno}
\usepackage{graphicx}% Include figure files
\usepackage{hyperref}% add hypertext capabilities
\usepackage{amsmath}
%\usepackage{grffile} %needed when image file name has multiple dots
%\usepackage{dcolumn}
%\usepackage{bm,amssymb,amsmath}
%\usepackage[pdftex]{hyperref}
%\hypersetup{colorlinks=true,citecolor=blue,linkcolor=red,urlcolor=blue}
%\usepackage[all]{hypcap}
\usepackage{color}
\usepackage{graphics}
\usepackage{adjustbox}
\definecolor{red}{rgb}{0.8, 0.0, 0.0}
\definecolor{blue}{rgb}{0.06, 0.2, 0.65}
\definecolor{green}{rgb}{0,0.6,0}
\newcommand{\red}{\color{red}}
\newcommand{\blue}{\color{blue}}
\newcommand{\green}{\color{green}}
%\thickmuskip=0.5\thickmuskip %reduces spaces in equations
%\usepackage{xr}


%Uncomment next line if AMS fonts required
%\usepackage{iopams}  
\begin{document}

\title{CH$_4$ and CO$_2$ Adsorption Mechanisms on Monolayer Graphenylene and their Effects on Optical and Electronic Properties}

\author{A. Aligayev}
\affiliation{University of Science and Technology of China, Hefei, 230026, China.}
\affiliation{%
NOMATEN Centre of Excellence, National Center for Nuclear Research, 
05-400 Swierk/Otwock, Poland
 }%

\author{F. J. Dominguez-Gutierrez}
\affiliation{%
NOMATEN Centre of Excellence, National Center for Nuclear Research, 
05-400 Swierk/Otwock, Poland
 }%
\affiliation{Institute for Advanced Computational Science
Stony Brook University Stony Brook, NY 11749, USA}

\author{M. Chourashiya}
\affiliation{%
NOMATEN Centre of Excellence, National Center for Nuclear Research, 
05-400 Swierk/Otwock, Poland
 }%
\affiliation{Guangdong Technion Israel Institute of Technology, Shantou, 515063, China}
\affiliation{Technion, Israel Institute of Technology, Haifa, 32000, Israel}

\author{S. Papanikolaou}
\affiliation{%
NOMATEN Centre of Excellence, National Center for Nuclear Research, 
05-400 Swierk/Otwock, Poland
 }%
\author{Q. Huang}
\affiliation{University of Science and Technology of China, Hefei, 230026, China.}

%\author[a]{F. J. Dom\'inguez-Guti\'errez\corref{author}}
%\cortext[author] {Corresponding author: javier.dominguez@ncbj.gov.pl}
%\author[a]{S. Bonfanti}
%\author[a,b]{A. Naghdi}
%\author[a]{S. Papanikolaou}
%\author[a,c]{M. J. Alava}
%\address[a]{NOMATEN Centre of Excellence, National Centre for Nuclear Research, ul. A. Sołtana 7, 05-400 Otwock, Poland}
%\address[b]{IDEAS NCBR, ul. Chmielna 69, 00-801 Warsaw, Poland}
%\address[c]{Department of Applied Physics, Aalto University, P.O. Box 11000, 00076 Aalto, Espoo, Finland}
%\ead{submissions@iop.org}
\vspace{10pt}
%\begin{indented}
%\item[]August 2017
%\end{indented}

\begin{abstract}

Graphenylene (GPNL) is a two--dimensional
carbon allotrope with a hexagonal lattice structure containing 
periodic pores. The unique arrangement of GPNL offers potential 
applications in electronics, optoelectronics, energy storage, and 
gas separation. Specifically, its advantageous 
electronic and optical properties, make it a promising candidate for 
hydrogen production and advanced electronic devices. 
In this study, we employ 
a computational chemistry--based modeling approach to investigate
the adsorption mechanisms of CH$_4$ and CO$_2$ on monolayer
GPNL, with a specific focus on their effects on optical
adsorption and electrical transport properties at room
temperature.
To simulate the adsorption dynamics as closely as possible to 
experimental conditions, we utilize the self--consistent charge 
tight--binding density functional theory (SCC--DFTB). 
Through semi--classical molecular dynamics (MD) simulations,
we observe the formation of H$_2$ molecules from the
dissociation of CH$_4$ and the formation of CO+O species 
from carbon dioxide molecules. 
This provides insights into the adsorption and dispersion
mechanisms of CH$_4$ and CO$_2$ on GPNL.
Furthermore, we explore the impact of molecular 
adsorption on optical absorption properties. 
Our results demonstrate that CH$_4$ and CH$_2$ affects drastically
the optical adsorption of GPNL, while CO$_2$ does not significantly 
affect the optical properties of the two--dimensional material. 
To analyze 
electron transport, we employ the open--boundary non--equilibrium 
Green's function method. By studying the conductivity of GPNL and 
graphene under voltage bias up to 300 mV, 
we gain valuable insights into the electrical transport 
properties of GPNL under optical absorption conditions.
The findings from our computational modeling approach
might contribute to 
a deeper understanding of the potential applications of GPNL in 
hydrogen production and advanced electronic devices.
\end{abstract}

%
% Uncomment for keywords
\vspace{2pc}
%\noindent{\it Keywords}:
\keywords{
%\begin{keyword}
W--Mo alloy, W--V alloy, nanoindentation, plasticity }
%\end{keyword}
%
% Uncomment for Submitted to journal title message
%\submitto{\PPCF}
%
% Uncomment if a separate title page is required
\maketitle
% 
% For two-column output uncomment the next line and choose [10pt] rather than [12pt] in the \documentclass declaration
%\ioptwocol
%
%\linenumbers

\section{Introduction}
Current quantum hardware is unable to carry out universal quantum computations due to the buildup of errors that occur during the computation. 
The magnitude of the individual error is currently above the value that the Threshold Theorem requires in order to kick-start quantum error correction and fault-tolerant quantum computation~\cite[Section 10.6]{nielsen_chuang_2010}. 
Although the experimentally achieved fidelity rates are promising and the error bounds are inching closer to the required threshold, we will have to work for the foreseeable future with quantum hardware with errors that build-up during the computation.  This implies that we can only do a limited number of steps before the output of the computation has become completely uncorrelated with the intended one.

For fault-tolerant quantum computing, we repeat four steps: 
1) We apply a number of single and two-qubit quantum gates, in parallel whenever possible; 
2) We perform a syndrome measurement on a subset of the qubits; 
3) We perform fast classical computations to determine which errors have occurred and how to correct them; 
and, 4) We apply correction terms based on the classical computations.
We then repeat these four steps with a next sequence of gates. 
These four steps are essential to fault-tolerant quantum computing. 


The starting point of this work is to use the four steps outlined above, not to carry out error correction and fault-tolerant computation, but to enhance short, constant-depth, {\em uncorrected} quantum circuits that perform single qubit gates and {\em nearest-neighbor} two qubit gates. 
Since in the long run we will have to implement error-correction and fault-tolerant computation anyhow, and this is done by such a four-step process, why not make other use of this architecture? Moreover, on some of the quantum hardware platforms, these operations are already in place.
Embracing this idea we naturally arrive at the question: what is the computational power of \textit{low-depth} quantum-classical circuits organized as in the four steps outlined above? 
We thus investigate circuits that execute a small, ideally constant, number of stages, where at each stage we may apply, in parallel, single qubit gates and {\em nearest-neighbor} two qubit gates, followed by measurements, followed by low-depth classical computations of which the outcome can control quantum gates in later stages. 
It is not clear, at first, whether such circuits, especially with constant depth, can do anything remotely useful. 
But we will see that this is indeed the case: many quantum computations can be done by such circuits in constant depth. 
By parallelizing quantum computations in this way, we improve the overall computational capabilities of these circuits, as we do not incur errors on qubits that are idle, simply because qubits are not idle for a very long time. 
Furthermore, reducing the depth of quantum circuits, at the cost of increasing width, allows the circuit to be run faster even if errors occur.

The first usage of such a four-step layout, not to do error correction, but to perform computations, can be found in the paradigm of measurement-based quantum computing~\cite{gottesman1999demonstrating,raussendorf2001one,jozsa2006introduction,clark2007generalised}: 
A universal form of quantum computing where a quantum state is prepared and operations are performed by measuring qubits in different bases, depending on previous measurements and intermediate measurements.

\citeauthor{PhamSvore2013} were the first to formalize the four-step protocol for performing computations~\cite{PhamSvore2013}. They included specific hardware topologies by considering two-dimensional graphs for imposing constraints on qubit interactions. In their model, they develop circuits for particularly useful multi-qubit gates, including specifying costs in the width, number of qubits, depth, number of concurrent time steps, size, and total number of non-Identity operations.
As a result, they find an algorithm that factors integers in polylogarithmic depth.
\citeauthor{Browne:2011} showed that the main tool in the work by \citeauthor{PhamSvore2013}, the fan-out gate, can also be replaced by additional log-depth classical computations in the measurement-based quantum computing setting~\cite{Browne:2011}.

More recently, \citeauthor{Cirac:2021} introduced a scheme to implement unitary operations involving quantum circuits combined with Local Operations and Classical Communication ($\mathsf{LOCC}$) channels: $\mathsf{LOCC}$-assisted quantum circuits~\cite{Cirac:2021}. Similarly to the four-step scheme we just described, they allow for a short depth circuit to be run on the qubits, followed by one round of $\mathsf{LOCC}$, in which ancilla qubits are measured and local unitaries are applied based on the measurement outcomes. They show that in this model any 1D transitionally invariant matrix-product state (MPS) with fixed bond dimension is in the same phase of matter as the trivial state. Similar ideas can be found in~\cite{TVV_NonAbelianTopologicalOrder_2022, tantivasadakarn2021long}.

In this work, we introduce a new model, called \textit{Local Alternating Quantum-Classical Computations} ($\LAQCC$). In this model we alternate between running quantum circuits (constrained by locality), ending in the measurement of a subset of qubits, and fast classical computations based on the measurement results. The outcome of the classical computations are then used to control future quantum circuits. We allow for flexibility in this model, by giving different constraints to the power of both the quantum circuits and the classical circuits as well as the number of alternations between them. 
Most attention will be given to $\LAQCC$ containing quantum circuits of constant depth, classical circuits of logarithmic depth and at most a constant number of alternations between them. 
Any circuit constructed in this model is considered to be of constant depth. 
We restrict ourselves to logarithmic depth classical computations, as this is the first natural and non-trivial extension beyond constant-depth classical computations. 
Constant-depth classical computations do however also have an equivalent constant-depth quantum implementation.

The definition of $\LAQCC$ sharpens the original definition of \citeauthor{PhamSvore2013} by adding constraints to the intermediate classical computations. This allows us to bound the power of $\LAQCC$ from above. 

The main result of \citeauthor{Cirac:2021}, that 1D translational invariant MPS with fixed bond dimension can be prepared by $\mathsf{LOCC}$-assisted circuits, relies on local symmetries of the MPS. These symmetries allow them to prepare local states (on a constant number of qubits) and glue them together by doing one round of the appropriate entangling measurement and corrections, after which they run a round of local unitaries to get the desired result. This general scheme for preparing states that exhibit an MPS description with the appropriate local symmetries requires only geometrically local unitaries and one round of measurement and corrections an therefore is accessible in $\LAQCC$. Studying different local symmetries, known as Symmetry Protected Topological (SPT) phases of matter, to find measurement-based constant depth circuits for states is a broad ongoing field of research~\cite{TVV_NonAbelianTopologicalOrder_2022, tantivasadakarn2021long, smith2023deterministic}. 
All these schemes have a $\LAQCC$ implementation.

%$\LAQCC$-circuits also exist for general schemes of preparing local states, based on the local tensors, and gluing them together using one round of entangled measurement and corrections, based on the local symmetry. 
%The main result of \citeauthor{Cirac:2021}, that 1D translational invariant MPS with fixed bond dimension can be prepared by $\mathsf{LOCC}$-assisted circuits, relies heavily on local symmetries of the MPS and as a result also has an equivalent $\LAQCC$ implementation. 
%The corrections applied after the measurement round are local unitaries depending on the local symmetries of the MPS. 

 

%This general scheme of preparing local states, based on the local tensors, and gluing it together by doing one round of entangled measurement and corrections, based on the local symmetry, is accessible in $\LAQCC$.
Note however that \citeauthor{Cirac:2021} also suggest a circuit for the $W$-state.
This circuit uses sequentially and dependent measurement-based corrections of the ancilla qubits. 
These dependent measurements translate to sequential alternations between the quantum and classical circuits and therefore increase the total depth to linear depth, exceeding the constant-depth constraints imposed by $\LAQCC$-circuits. 

We study the power of the $\LAQCC$ model with respect to state preparation, showing that even with only constant quantum-depth and logarithmic classical depth it remains possible to prepare states with long-range entanglement.
Another surprising result is that it is unlikely that $\LAQCC$ circuits are classically simulatable. We show that any instantaneous quantum polynomial-time (IQP) circuit~\cite{Bremner2010,Shepherd2009} has an $\LAQCC$ implementation.
Classical simulation of IQP circuits implies the collapse of the polynomial hierarchy to the third level, which is not believed to be true~\cite{Bremner2017}. Therefore, we expect that $\LAQCC$ circuits are unlikely to be classically simulatable. We bound the power of $\LAQCC$ by showing that it is contained in $\QNC^1$, the class of polynomial-size, log-depth circuits.

Next, we also study the power that intermediate classical calculations can add to quantum computations, by considering a new model that alternates between polynomially many polynomial-depth quantum circuits and unbounded classical computations
We study this model by doing a complexity theoretical analysis, where we draw inspiration from the notions of complexity given by \citeauthor{RosenthalYuen:2022}, \citeauthor{MetgerYuen:2023}, and \citeauthor{Aaronson:2004}.
All three complexity notions are based on the notion of state preparation, instead of more traditional definition of complexity such as the decidability of a computational problem. 
The first two consider classes based on sequences of quantum states preparable by a polynomial-sized quantum circuit, where the circuits are uniformly generated by a computational class, for instance, the class $\mathsf{PSPACE}$, which results in the complexity class $\mathsf{StatePSPACE}$~\cite{RosenthalYuen:2022,MetgerYuen:2023}.
The third notion considers a relative complexity, where the complexity is measured between two given states, and is measured by the number of gates, from a given gate-set, required to transform one state in another state~\cite{Aaronson:2004}. 
For our definition of state preparation complexity, we drop the uniformity constraint from~\cite{RosenthalYuen:2022,MetgerYuen:2023} and define a class as $\mathsf{StateX}$, which refers to states preparable by circuits of type $\mathsf{X}$. 
As an example, if $\mathsf{X} = \QNC^0$, this results in the class $\mathsf{StateQNC^0}$, which is the set of states preparable from the $\ket{0}^n$ state by poly-size constant-depth circuits. 
This notion is similar to the relative complexity from~\cite{Aaronson:2004}, where one state is the  $\ket{0}^n$ state and instead of counting the number of gates we consider the set of states preparable by a fixed number of gates. Using this notion of complexity we show that any state preparable by an $\LAQCC^*$ circuit is also preparable by a $\mathsf{PostQPoly}$ circuit, the class of circuits of polynomial depth with an additional post-selection gate. 

All Clifford circuits have a constant-depth $\LAQCC$ implementation, implying that any stabilizer state can be implemented by a constant-depth $\LAQCC$ circuit, see Section~\ref{sec:clifford_circuits} for a proof of this statement. 
Efficient circuits for stabilizer states have been known already through measurement-based quantum computing. Therefore this paper focuses on the preparation of non-stabilizer states, and as a surprising result we find novel constant-depth protocols for four very natural classes of non-stabilizer states.
Despite the extensive research into these four classes of non-stabilizer states and the many applications of them, no efficient constant- or low-depth state preparation protocols are known yet. We specifically consider these four classes as they are all often used as initial states in other algorithms.

The first state is a uniform superposition over an arbitrary number of states. 
This state finds applications in many quantum algorithms, as they often start with a uniform superposition over multiple states. 
This superposition is often achieved by applying Hadamard gates to every qubit due to its simplicity to prepare. 
Yet, the analysis of many algorithms, such as Shor's algorithm~\cite{Shor:1997}, would benefit from a different initial superposition. 
The circuit to prepare the uniform superposition over an arbitrary number of states uses an exact version of Grover search as a subroutine, that turns a probabilistic circuit, with a known constant probability of success, into a deterministic circuit. 
We use the circuit for preparing a uniform superposition over an arbitrary number of states as a subroutine in the next two quantum state preparation protocols. 

The second state is the $W$-state, the uniform superposition over all computational basis states of Hamming-weight~$1$, a natural long-ranged entangled state that displays a fundamentally nonequivalent type of entanglement from the Greenberger–Horne–Zeilinger state~\cite{WState:2000}, for which $\LAQCC$-type constant-depth circuits were previously known~\cite{PhamSvore2013, Cirac:2021}. 
The $W$-state is often used as benchmark for new quantum hardware~\cite{Haffner2005,Neeley2010,GarciaPerez:2021}. 
A novel way to prepare the $W$-state therefore gives a new way to benchmark different quantum devices with each other. 
A circuit for preparing the $W$-state was given in~\cite{Cirac:2021}, but this implementation requires sequentially alternating measurements followed by local unitaries, which in the $\LAQCC$ model is not considered to be of constant depth. 
We improve this protocol by giving an $\LAQCC$ implementation of the $W$-state, based on a compress-uncompress method that links the one-hot and binary encoding of integers.

The third state considered is the Dicke state, a generalization of the $W$-state, a superposition over all computational basis states with Hamming-weight $k$~\cite{Dicke:1954}. 
Dicke states have relevance in various practical settings.
For instance, for quantum game theory~\cite{zdemir2007}, quantum storage~\cite{Bacon_Compress:2006,Plesch:2010}, quantum error correction~\cite{ouyang2014permutation}, quantum metrology~\cite{toth2012multipartite}, and quantum networking~\cite{prevedel2009experimental}. 
Dicke states have been used as a starting state for variational optimization algorithms, most notably Quantum Alternating Operator Ansatz (QAOA)~\cite{Hadfield2019}, to find solutions to problems such as Maximum k-vertex Cover~\cite{Brandhofer2022,cook2020quantum}.
The ground states of physical Hamiltonians describing one-dimensional chains tend to show a resemblance to Dicke states such as states resulting from the Bethe ansatz, making them an ideal starting state when investigating the ground state behavior of these Hamiltonians~\cite{TDL_BetheAnsatzDerivation:2010,B_ExcitedStateQuantumPhaseTransitions:2013,DickeTransitions:2021}. 
For instance, the algorithm by \citeauthor{van2021preparing}, who give an algorithm to prepare the Bethe ansatz eigenstates of the spin-1/2 XXZ spin chain, starts by first preparing a Dicke state~\cite{van2021preparing}. 
A Dicke-state preparation protocol based on the compress-uncompress methodology used in the $W$-state furthermore finds applications in entanglement distillation, where the entanglement of a large state is concentrated on only a few qubits. 
Efficient deterministic circuits for preparing Dicke states have been proposed by \citeauthor{bartschi2019deterministic}~\cite{bartschi2019deterministic, bartschi2022deterministic_short_depth}. 
They provide a quantum circuit of depth $\mathO(k \log(\frac{n}{k}))$, allowing arbitrary connectivity, to prepare a Dicke state, which they conjecture to be optimal when $k$ is constant. 
In this work, we provide a constant-depth $\LAQCC$ circuit below their conjectured bound already for constant $k$. 
However, this does not directly disprove their conjecture, as we allow for intermediate measurements and classical computations. 
More significantly, we even construct constant-depth $\LAQCC$ circuits for $k = \mathO(\sqrt{n})$ greatly improving their bound.
This construction extends the compress-uncompress method for the $W$-state combined with additional subroutines. 

We continue with a log-depth state preparation protocol for the Dicke-state for arbitrary $k$. 
This protocol implements an efficient transformation between the factoradic number representation and the combinatorial number representation of a positive integer. 
The combinatorial number representation relates directly to the Dicke state. 
The provided efficient transformation between number representation systems might be of independent interest. 

We conclude by modifying our protocol for preparing a Dicke-state to a protocol that prepares quantum many-body scar states in constant-depth. 
These states have low entanglement and longer coherence times than states with similar energy density.
These characteristics make many-body scar states interesting to analyze and relevant within physics.
Many-body scar states appear for instance in the AKLT model~\cite{AKLT:1987,MRBAR:2018,MRB:2018} and different spin models~\cite{SI:2019,MOBFR:2020}.
Known methods for preparing these states have polynomial-depth~\cite{Gustafson:2023}, whereas our circuit has constant depth. 

% We conclude by studying the power that intermediate classical calculations can add to quantum computations. 
% In this study, we define a new model that relaxes constant-depth quantum circuits to polynomial depth quantum circuits, log-depth classical calculations to unbounded classical computations and a constant number of alternations to a polynomial number of alternations. 
% We call this model $\LAQCC^*$. 
% We study this model by doing a complexity theoretical analysis, where we draw inspiration from the notions of complexity given by \citeauthor{RosenthalYuen:2022}, \citeauthor{MetgerYuen:2023}, and \citeauthor{Aaronson:2004}.
% All three complexity notions are based on the notion of state preparation, instead of more traditional definition of complexity such as the decidability of a computational problem. 
% The first two consider classes based on sequences of quantum states preparable by a polynomial-sized quantum circuit, where the circuits are uniformly generated by a computational class, for instance, the class $\mathsf{PSPACE}$, which results in the complexity class $\mathsf{StatePSPACE}$~\cite{RosenthalYuen:2022,MetgerYuen:2023}.
% The third notion considers a relative complexity, where the complexity is measured between two given states, and is measured by the number of gates, from a given gate-set, required to transform one state in another state~\cite{Aaronson:2004}. 
% For our definition of state preparation complexity, we drop the uniformity constraint from~\cite{RosenthalYuen:2022,MetgerYuen:2023} and define a class as $\mathsf{StateX}$, which refers to states preparable by circuits of type $\mathsf{X}$. 
% As an example, if $\mathsf{X} = \QNC^0$, this results in the class $\mathsf{StateQNC^0}$, which is the set of states preparable from the $\ket{0}^n$ state by poly-size constant-depth circuits. 
% This notion is similar to the relative complexity from~\cite{Aaronson:2004}, where one state is the  $\ket{0}^n$ state and instead of counting the number of gates we consider the set of states preparable by a fixed number of gates. Using this notion of complexity we show that any state preparable by an $\LAQCC^*$ circuit is also preparable by a $\mathsf{PostQPoly}$ circuit, the class of circuits of polynomial depth with an additional post-selection gate. 

\paragraph{Summary of results}
\begin{itemize}
    \item We give a new definition of a computational model that captures the power of the four step process: applying a constant number of layers of one- and two-qubit gates; performing a syndrome measurement; perform a fast classical computation determining corrections; apply corrections. We call this model \emph{Local Alternating Quantum Classical Computations}, or $\LAQCC$ for short. In this model we bound the allowed quantum operations, intermediate classical calculations, and number of rounds separately. In Section~\ref{sec:LAQCC_model} we define this model and give a list of operations based on results from literature contained in this computational model. In some of these operations we explicitly use that we allow for multiple, but at most constant, rounds  of corrections.
    \item  We show show that there exist $\LAQCC$ circuits that can not be weakly simulated in Section~\ref{sec:IQP_in_LAQCC}. We further show that for every $\LAQCC$ circuit there exists a $\QNC^1$ circuit simulating it perfectly, in Section~\ref{sec:LAQCC_in_QNC1}.
    \item We introduce a new type computational complexity for preparing states and show that the extension of $\LAQCC$ where we allow a polynomial number of rounds and unbounded classical computation, is contained in $\mathsf{PostQPoly}$, the class of polynomial circuits with post-selection, in Section~\ref{sec:Complexity results}.
    \item We show a protocol to prepare the uniform superposition state of size $q$ in $\LAQCC$ using $\mathO(\ceil{\log_2(q)}^2)$ qubits in Section~\ref{sec:superposition_modulo_q}. 
    \item We show a protocol to prepare the $W_n$ state in $\LAQCC$ using $\mathO(n\log(n))$ qubits in Section~\ref{sec:W_state_in_LAQCC}.
    \item We show two ways of preparing the Dicke-$(n,k)$ state. The first method is in $\LAQCC$, works up to $k = \mathO(\sqrt{n})$, uses $\mathO(n^2\log(n))$ qubits, and is found in Section~\ref{sec:dicke:small_k}. The second method is in $\LAQCC\text{-}\mathsf{LOG}$ (an extension of $\LAQCC$ allowing for logarithmic number of alterations instead of constant), works for any $k$, uses $\mathO(\text{poly}(n))$ qubits, and is found in Section~\ref{sec:Dicke_in_LAQCC_LOG}. 
    \item We extend on our $\LAQCC$ method of generating Dicke-$(n,k)$ states for $k = \mathO(\sqrt{n})$ and show a protocol to generate many-body scar states for a particular Hamiltonian in $\LAQCC$ (Section~\ref{sec:many_body_scar}). 
\end{itemize}
Summarized in a table, we provide the following state generation protocols:
\begin{table}[htb]
\centering
\begin{tabular}{l|l|l|l}
\textbf{State description} & \textbf{Width} & \textbf{Depth} & \textbf{Implementation}\\
\hline 
Uniform superposition mod $q$: $\frac{1}{\sqrt{q}} \sum_{i = 0}^{q-1}\ket{i}$ & $\mathO(\ceil{\log^2 q})$ & $\mathO(1)$ & Section~\ref{sec:superposition_modulo_q}\\

$W$-state: $\frac{1}{\sqrt{n}}\sum_{i = 0}^{n-1}\ket{e_i}$ & $\mathO(n \log n)$ & $\mathO(1)$ & Section~\ref{sec:W_state_in_LAQCC}\\

Dicke-$(n,k)$, $k = \mathO(\sqrt{n})$: $\binom{n}{k}^{-1/2}\sum_{x \in \{0,1\}^n: |x| = k} \ket{x}$ &  $\mathO(n^2\log n)$ & $\mathO(1)$ 
&Section~\ref{sec:dicke:small_k}\\

Dicke-$(n,k)$: $\binom{n}{k}^{-1/2}\sum_{x \in \{0,1\}^n: |x| = k} \ket{x}$ & $\mathO(\text{poly}(n))$ & $\mathO(\log n)$ &Section~\ref{sec:Dicke_in_LAQCC_LOG}\\

QMBS: $\ket{S_k} = \frac{1}{k! \sqrt{\mathcal N(n,k)}}(Q^\dagger)^k \ket{\Omega}$ &  $\mathO(n^2\log n)$ & $\mathO(1)$  &  Section~\ref{sec:many_body_scar}
\end{tabular}
\caption{Summary of state preparation protocols given in this paper.}
\label{tab:sate_prep}
\end{table}
In the entry for the quantum many-body scar state $Q$ denotes the raising operator and $\mathcal N(n,k)=\binom{n-k-1}{k}$. 
Section~\ref{sec:many_body_scar} will provide more details on the variables and the implementation. 

\paragraph{Organization of the paper}
\noindent We first introduce relevant preliminaries in Section~\ref{sec:preliminaries}. 
In Section~\ref{sec:LAQCC_model} we formally define the class of Local Alternating Quantum-Classical Computations ($\LAQCC$). We also show that any Clifford circuit can be implemented in constant depth $\LAQCC$ (a result based on a result from measurement-based quantum computing~\cite{jozsa2006introduction}). 
This result allows us to give many useful multi-qubit gates and routines in Section~\ref{sec:gates_created_in_LAQCC}. 
Beyond that we show that constant depth $\LAQCC$ circuits are contained in $\QNC^1$ and that any $\mathsf{IQP}$ circuit has an $\LAQCC$ implementation.
We conclude this section with an analysis of a more powerful instantiation of $\LAQCC$ and show an inclusion with respect to the class $\mathsf{PostQPoly}$, which is the class of circuits of polynomial depth with one additional post-selection gate. 
In Section~\ref{sec:state_prep_in_LAQCC} we give $\LAQCC$ circuit implementations for preparing the uniform superposition over an arbitrary number of states, the $W$-state and the Dicke state up to $k = \mathO(\sqrt{n})$. We furthermore give a log-depth circuit implementation for preparing the Dicke state for any $k$. We conclude by showing a $\LAQCC$ circuit for generating many body scar states of a particular type of Hamiltonian.


\section{Computational Methods}
\label{sec:methods}

The SCC--DFTB method is a computational approach that approximates 
traditional Density Functional Theory (DFT) by considering 
valence electron interactions in MD simulations. 
It serves as a valuable tool for accurately predicting structures 
and thermodynamic properties prior to synthesis, providing 
insights into the gas adsorption properties of 2D carbon--based 
materials and their potential applications in various gas 
adsorption environments.
The SCC--DFTB method involves solving Kohn--Sham equations to 
obtain total valence electronic densities and energies for each 
atom utilizing a Hamiltonian functional based on a two-center 
approximation and optimized pseudo--atomic orbitals as the basis 
functions \cite{DFTBplus, Qiang}. Slater--Koster parameter files 
are utilized to provide tabulated Hamiltonian matrix elements, 
overlap integrals, and repulsive splines fitted to DFT 
dissociation curves. 
%These parameter files are read into the 
%computer memory only once at the start of the simulation, as 
%described in Koskinen et al. (2009) \cite{KOSKINEN2009237}. 
These parameters describe the overlap and hopping integrals 
between pairs of atoms in the tight--binding Hamiltonian. 
The optimal set of Slater-–Koster parameters have two main 
requirements: a good reproduction of the structure of the 
relevant electronic bands, and faithful representation of the 
orbital contribution along such bands.
Therefore, in the scope of this approach the total energy
of the system is expressed as
\begin{equation}
    E^{\rm DFTB} = E_{\rm band}+E_{\rm rep}+E_{\rm SCC},
\end{equation}
with the band structure energy, $E_{\rm band}$, defined
from the summation of the orbital energies $\epsilon_i$
over all occupied orbitals $\Psi_i$; 
the repulsive energy $E_{\rm rep}$ for the core--core
interactions related to the exchange--correlation energy 
and other contributions in the form of a 
set of distance--dependent pairwise terms;
and an SCC contribution, $E_{\rm SCC}$, as the contributions
given by charge--charge interactions in the system.
Therefore, the electronic energy is calculated by summing 
the occupied Kohn-Sham (KS) single--particle energies and the 
contributions from repulsive energies between diatomic atoms. 
To account for self--consistent charge (SCC) effects during 
the dynamics, an iterative procedure is used. 
%with SCC 
%corrections implemented in the DFTB$+$ code 
%version 22.2 \cite{DFTBplus}. 
%This procedure leads to convergence to a new electron density 
%at every time step during the simulation. The convergence is 
%enhanced by using an electronic temperature of 1000 K.


%%%%%%%%%%%%%%%%%%%%%%%%%%%%%%%%%%%%%%%%%%%%%%%%%%%%%%%%%%%%%

\subsection{Structures and binding energies}

GPNL is a two--dimensional carbon allotrope 
that possesses a hexagonal lattice structure with periodic
pores and its structure as reported 
%It is derived from 
%graphene by introducing line defects in the form of carbon--carbon 
%triple bonds along its lattice. The GPNL structure reported 
by Balaban et al. \cite{balaban1968chemical}  and Martins et al. 
\cite{Martins2022}, consists of three types of symmetrically 
distributed rings: dodecagon (C$_{12}$), hexagon (C$_6$), and
square (C$_4$), which forms a tiling 
of the Euclidean plane. The unit cell of GPNL, determined by 
Fabris et al. using DFT\cite{fabris2018theoretical}, belongs
to the $P6/mmm$ space group and contains a single
irreducible atom, which is considered in our SCC--DFTB 
calculations.
In our study, we performed optimization of the GPNL unit 
cell, resulting in lattice parameters 
$\vec{ \text{a}} = \vec{b} = 
6.735$ \AA{} and bond lengths of 1.50 Å for the square ring and 1.48 
Å for the hexagon ring identifying seven points of 
high symmetry \cite{martins2021new} which is 
in good agreement with experimental measurements for 1.42--1.46 \AA{} for the 6--C rings and 1.50-1.52 \AA{} for the two bonds 
joining the 6--C rings \cite{du2017new},  as shown in Fig. \ref{fig:structureGPNL}a). 
The central nanopore (dodecagon ring) in the unit cell has a diameter 
of 5.66 Å, in good agreement with DFT data  
\cite{fabris2018theoretical,C2TC00006G} 
and experimental measure of 5.8 \AA{} \cite{du2017new}. 
%Martins et al. \cite{martins2021new}  identified in the GPNL structure. 
To compare the adsorption capabilities of GPNL with graphene, 
we also optimized the unit cell of graphene using a well--known 
lattice parameters and bond lengths, as depicted in Fig. 
\ref{fig:structureGPNL}b).

% Figure environment removed

%%%%%%%%%%%%%%%%%%%%%%%%%%%%%%%%%%%%
%\subsection{Binding energies}

The interaction potentials between H$_2$, CO$_2$, and CH$_4$ 
molecules with graphene and GPNL are investigated using the 
DFTB method. To avoid interactions with periodic replicas, the unit 
cell of the optimized GPNL structure is replicated by 
3$\times$4$\times$1 and the unit cell of graphene is replicated by 
5$\times$5$\times$1 along the $x$ and $y$ directions. 
The larger cells are initially optimized using SCC-DFTB. 
Adiabatic calculations follow to determine potential energy curves. 
These curves depict molecule interactions with fully relaxed periodic 
sheets at different distances and adsorbate sites, incorporating 
dispersion corrections through van der Waals interactions.
\cite{10.1063/1.1329889}.
Thus, the total energies, $E(z)$, of the molecule--2D material 
system with a separation $z$ between the adsorbate 
sites and the center of mass of the molecules
are varied above the surface in a range of 0.5 to 7 \AA{}, 
which defines the computation of the adsorption potential 
as a function of the distance separation.
The total energy is then computed as:
\begin{equation}
    E(z) = E_{\rm Tot} - \left( E_{\rm 2D material} + E_{\rm Molecule}
    \right),
\end{equation}
where $E_{\rm Surface}$ is the total energy of the 
2D material; $E_{\rm Molecule}$ is the total energy of the 
isolated molecule: H$_2$, CH$_4$, 
and CO$_2$; and $E_{\rm Tot}$ is the energy of the 
interacting system at every $z$--distance. 
Thus, the binding energy is defined as 
$E_b = E(z_{\rm min})$ with $z_{\rm min}$ as the 
equilibrium molecule--surface distance. 
Total energy calculations are performed for the 
molecule--2D material system, varying the distance between 
the surface and the center of mass of the molecules 
along the $z$--axis. 
We consider 3 different adsorption sites for graphene and 
5 sites for GPNL based on unit cells of 
the materials.
The molecular symmetry planes are consider 
with respect to the surface plane in the calculations. 
The repulsive potential is cut off at
a distance below the second nearest--neighbor interaction
region for numerical stability. 
%However, this approximation may not always provide 
%satisfactory dissociation curves. 
%The SCC--DFTB framework addresses this limitation by 
%shifting the repulsive energy functions downward.



%%%%%%%%%%%%%%%%%%%%%%%%%%%%%%%%%%%%%%%%%%%%%%%
\subsection{Semi--Classical Molecular Dynamics Simulations}

We conducted semi--classical molecular dynamics simulations
to investigate the adsorption dynamics of H$_2$, CH$_4$,
and CO$_2$ molecules on graphene and GPNL. 
For graphene, a $5\times5\times1$ supercell 
was used, while for GPNL, a $3\times3\times1$ supercell was 
employed. The surfaces were optimized and equilibrated to a 
temperature of 300 K using a Nose--Hoover thermostat.
To simulate the adsorption dynamics, we defined a target area of 1 
nm$^2$ on the surface, and molecules were randomly distributed on it 
using the velocity Verlet algorithm. 
The impact energy of the molecules was 8 eV, 
and 650 independent trajectories were generated for each molecule. 
A time step of 0.25 fs was used, and the molecules were emitted 
vertically with random orientations at an initial distance of 
0.6 nm above the surface. The simulations were 
performed for a duration of 350 fs,  
This timeframe was meticulously chosen to ensure convergence
in our MD simulations. It provided ample time for the molecules
to travel away from the carbon sheets while also guaranteeing
that the attached molecules remained bonded to the carbon atoms, 
preventing any detachment from the sheets.
%The calculations were executed on a computer cluster with 
%160--240 cores, and the simulations typically had a wall time of 
%approximately 20 minutes. 
We have previously employed this approach to study hydrogenation 
mechanisms of fullerene cages \cite{DOMINGUEZGUTIERREZ2018189}, 
electronic properties 
of borophene \cite{C7TC00976C}, and dynamic physisorption pathways of 
molecules on alumina surfaces \cite{aluminadftb}, demonstrating 
excellent agreement with first principles DFT calculations.


%%%%%%%%%%%%%%%%%%%%%%%%%%%%%%%%%%%%%%5
\subsection{Optical absorbance and electron transport calculations}
%
%%%%%%%%%%%%%%%%%%%%%%%%%%%%%%%%%%%%%%%%%%%%%%%
%\subsection{Optical absorbance}


The optical absorption is investigated within the DFTB framework as 
an electronic dynamic process in response to an external electric
field \cite{C8CP04625E,B926051J}. 
The conventional adiabatic approximation gives the time evolution 
of the electron density matrix by time integration of the 
Liouville--von Newmann equation expressed as
\begin{equation}
    i \hbar \frac{\partial \hat \rho}{\partial t} = 
    S^{-1}\hat H \hat \rho - \hat \rho S^{-1},
\end{equation}
where $\hat \rho$ is the single electron density matrix, 
$\hat S$ is the overlap matrix, and $\hat H$ is the system Hamiltonian 
that includes the external electric field as 
$\hat H = \hat H_0 + E_0 \delta (t-t_0) \hat e$ with 
$E_0$, the magnitude of the electric field, 
and $\hat e$, its direction. 
Under the framework of linear response, the absorbance $I(\omega)$ is
calculated as the imaginary part of the Fourier transform of the
induced dipole moment caused by an external field. 
In this study, the external field strength was set to 
$E_0 = 0.001$ V/\AA{}. The induced dipole moment was evaluated over
a $200$ fs time period using a time step of $\Delta t$ = 0.01 fs. 
The Fourier transform was performed with an exponential damping
function (using a 5 fs damping constant) to eliminate noise.



 The Non--Equilibrium Green's Functions formalism (NEGF) 
 is a robust  theoretical framework commonly used for 
 modeling electron transport in  nano--scale devices and
 is implemented in the DFTB code  \cite{Pecchia_2008}. 
In Fig. \ref{fig:transport_}, we provide a detailed illustration
of the geometric configuration of the graphene and GPNL 
structures, highlighting the specific regions involved in the 
electron transport calculations. To ensure accurate and reliable
results, several steps are followed: 1) The structures are carefully
divided into distinct sections, including the principal layers, 
two electrode contacts (drain and source), and the device region. 
This partitioning enables a systematic analysis of electron 
transport within the designated "scattering region."; 2) 
The drain section, represented by red spheres, corresponds to 
the region where electrons exit the device, while the source 
section, depicted by blue spheres, represents the region where 
electrons enter the device; 3) To simulate realistic conditions and 
investigate the impact of specific molecules on electron transport,
CO$_2$ molecules are introduced into the graphene device section,
and CH$_4$ molecules are added to the GPNL device section.
This allows us to study the interaction between the adsorbates
and the carbon-based materials and observe their influence on
the electron transport properties; and 4) Before performing the 
electron transport calculations, the entire system undergoes an 
optimization process. This optimization involves adjusting the 
positions and orientations of the atoms to find the most 
energetically favorable configuration for the combined 
graphene/GPNL--adsorbate system.


% Figure environment removed

\section{Results}
\label{sec:results}



% Figure environment removed



%\subsection{Bands/Dos}

GPNL exhibits lower electric conductivity compared to
graphene due to its distinct physical properties, particularly
its formation energies and band gaps. The 
influence of pore size and quantity on these properties is 
depicted in Fig. 
\ref{fig:DOS}. Our DFTB calculations reveal that GPNL 
possesses a bandgap of approximately 0.96 eV from 
the DOS calculations, while graphene lacks a bandgap altogether.
The porosity of GPNL has the potential to significantly modify its 
electronic characteristics and catalytic performance by increasing
its surface area, in good agreement with reported 
results by G. Brunetto et al \cite{brunetto2012nonzero}. The 
electronic band structures, illustrated in Fig. \ref{fig:DOS},
reveal that the valence and conductance
bands for carbon atoms are located at the $\Gamma$ point.
%while the bottom 
%of the conduction bands is found at the $K$--point.
The band gap of GPNL is structure--dependent and can range from
zero to a few electron volts. Theoretical studies have predicted
band gaps for GPNL ranging from 0 eV to approximately 3.3 eV,
depending on the specific structure and calculation 
method employed \cite{balandin2011thermal,brunetto2012nonzero}.
It should be noted that the semi--local functionals tend to 
underestimate the band gaps of GPNL structures. 
It is also noticed the selected path show the characteristic
gaps at the M and $\Gamma$ point reported by DFT 
calculations \cite{C2TC00006G} and our results reported in the SM. 
The slight discrepancy in negative energies can be attributed to 
variations in the SK parameters applied in our calculations and the 
pseudo potentials used in our DFT calculations. Nevertheless, this 
fair agreement still provides validation for our results.



To validate our findings, density functional theory (DFT) 
calculations were conducted using the PBE exchange--correlation 
functional. The calculations were carried out under periodic 
boundary conditions, and the Brillouin zone integration was 
performed with the $\Gamma$ point considered. Kohn--Sham orbitals 
were employed as plane waves up to an energy cutoff of 90 Ry to 
ensure convergence in the structural properties of the systems. 
The Quantum-ESPRESSO ab--initio package with relativistic-
corrected pseudo-potentials was utilized for computing the 
density of states, system energies, and band structures.
%The 
%exchange-correlation energy was evaluated using the generalized 
%approximation (GGA).
The total electronic density of states (DOS) for GPNL
reveals significant overlaps between the C--2s and C--2p curves, 
indicating the presence of strong sp$^3$ hybridized covalent
bonding states.

GPNL consists of interconnected benzene 
rings arranged in a hexagonal lattice, similar to graphene.
The slight discrepancy in negative energies for the GPNL sheet
can be attributed to variations in the SK parameters applied in our 
calculations and the pseudo potentials used in our DFT calculations. 
Nevertheless, this fair agreement still provides validation for our 
results.
This supports the application of SCC--DFTB in studying gas
separation processes, which is crucial for the production
and utilization of clean fuels. 
Different paths for the GPNL sheet was considered 
and shown in the supplementary material (SM) of this work.



%%%%%%%%%%%%%%%%%%%%%%%%%%%%%%%%%%%%%%%%%%%%%%%
%\subsection{Binding Energy }

Figure \ref{fig:PECs} illustrates the binding energies as a
function of separation distance for graphene (a) and GPNL (b), considering different adsorbate sites as
labeled in the inset figure. We performed adsorption 
calculations for isolated H$_2$, CH$_4$, and CO$_2$ molecules
on both systems.
To ensure accurate calculations, we included a 50 \AA{}
vacuum section above the sample to minimize boundary effects.
Periodic boundary conditions were applied in the $x$--$y$
directions to simulate a semi-infinite surface. 
For the $k$--point sampling, we employed a $4\times4\times1$
Monkhorst--Pack set throughout all calculations.


% Figure environment removed


Fig. \ref{fig:PECs} present results for the physisorption 
pathways for H$_2$, CH$_4$, and CO$_2$ molecules on both
graphene and GPNL sheets. In the case of graphene, we 
observe that the bond length for molecular hydrogen and 
methane molecules is approximately 2.2 \AA{}, with binding 
energies of -0.052 eV (in good agreement with Lee et al. \cite{10.1063/5.0116092}) and -0.165 eV, respectively. 
Meanwhile, the carbon dioxide molecule exhibits a bond 
length of 2.6 \AA{} and a binding energy of -0.151 eV.
For GPNL, we find that H$_2$ and CH$_4$ molecules have bond 
lengths of 2.25 \AA{}, with binding energies of -0.043 eV 
and -0.132 eV, respectively. CO$_2$ on GPNL has a bond 
length of 2.62 \AA{}, accompanied by a binding energy of 
-0.125 eV. 
In the SM, we present the physisorption pathways of all 
the adsorbate site of both carbon sheets, showing that 
adsorbate sites associated with
the hexagonal and square holes exhibit a lower likelihood for 
molecule adsorption; specially methane molecules are more 
likely to bond to the top of the carbon atoms, with a binding
energy of -0.2 $\pm$ 0.05 eV showing excellent conditions
for gas separation for this energy barrier 
\cite{C2TC00006G,D0RA04286B}. 


These results serve as the basis for configuring
the initial conditions in our 
MD simulations. We ensure an initial distance greater
than 8 \AA{} to prevent interactions between the 
molecules and the carbon sheet at the outset of the
simulation.
These findings offer valuable insights into the interactions 
between carbon sheets and these molecules, which are essential 
for understanding the physical processes in our MD simulations.
It is worth noting that dispersion 
corrections
play a significant role in these calculations, particularly due
to the hybridization of the system and the presence of CH$_4$
molecules.
Similar trends are observed for graphene, where the adsorbate site
located at the top of the carbon atoms is more favorable for both 
physisorption and chemisorption mechanisms confirming reported 
results for hydrogen trapping by graphene 
\cite{10.1063/1.1329889}, 
indicating a higher likelihood of attracting molecules.



%\subsection{CH$_4$ and CO$_2$ adsorption mechanisms}
\subsection{Dynamical adsorbtion of CH$_4$ and CO$_4$ molecules}


After conducting MD simulations at room temperature 
and an impact energy of 8 eV, the emission of hundreds of H$_2$, 
CO$_2$, 
and CH$_4$ molecules is analyzed by calculating the 
probabilities of adsorption, reflection, and transmission, which are 
defined as:
\begin{equation}
    P = 100 \times \frac{N_{x}}{N_{\rm Tot}},
\end{equation}
where $N_{\rm Tot}$ is the total number of MD simulations, while 
$N_x$ is the number of cases for adsorption, reflection, and 
transmission calculated by using the following conditions based 
on the last frame of the MD simulations: 
1) Adsorbed cases: Molecules with a final position between
a sphere centered at the material with a 
radius of 3.5 \AA{} and the direction of the velocity vector
points towards the surface is considered as adsorbed; 
2) Reflection cases: Molecules 
with a final position larger than 3.5 \AA{} and a velocity vector
oriented in the opposite direction
to the surface normal is counted as reflected; 3) 
Transmission cases: Molecules with a final position below the
surface and a distance larger than -3.5 \AA{} are counted as
transmitted. 
The counts for these cases are tabulated in Table \ref{tab:prob}.



\begin{table}[t!]
    \centering
     \resizebox{\columnwidth}{!}{%
    \begin{tabular}{llll| lll}
    \hline
    \multicolumn{4}{c}{Graphene} & \multicolumn{3}{c}{Graphelyne} \\
    \hline
    Probability    &   H$_2$ & CO$_2$ & CH$_4$ & H$_2$ & CO$_2$ & CH$_4$ \\
    \hline
     Transmission  & 0 & 0   & 0    & 5.28   & 1.28  & 49.12 \\
     Adsorption   & 0 & 0   & 16.8 & 0      &  9.66 & 1.60 \\
     Reflection    & 100 & 100 & 83.2 & 94.72  & 89.06 & 49.28 \\
    \hline
    \end{tabular}}
    \caption{probabilities of different cases observed in the MD 
     simulations, including transmission, reflection, and adsorption, 
     for both graphene and GPNL. The results clearly demonstrate 
     that GPNL exhibits superior performance as a material for 
     gas separation compared to graphene.}
    \label{tab:prob}
\end{table}



For graphene, we did not observe dissociation of H$_2$ and CO$_2$ 
molecules among the reflected molecules. 
The molecule initiates its trajectory with a 
kinetic energy (KE) of 8eV. However, it undergoes a gradual 
deceleration as it comes into contact with the charge
cloud of the carbon sheets. Upon colliding with a carbon
sheet at various adsorbate sites, the molecule is reflected, 
promoting its vibrational and rotational movements.
To observe dissociation mechanism, a
higher impact energy than 10 eV would be required.
However, it is worth noting that 
such high--energy collisions could result in the creation of 
vacancies in the graphene sheet by displacing a carbon atom, 
which is not observed in our MD simulations at 
8 eV. On the other hand, some 
of the reflected CH$_4$ molecules underwent dissociation, 
and a few of them were able to attach to the graphene sheet.
In order to bond more 
CH$_2$ molecules to graphene, a lower impact energy would be 
sufficient.
In the case of GPNL, its inherent porosity allowed for a higher 
number of transmitted cases for both H$_2$ and CO$_2$ molecules. We 
observed that some carbon dioxide molecules dissociated into CO+O, 
with 
the CO molecules bonding to the GPNL sheet. This highlights the 
advantage of GPNL's porous structure in facilitating gas 
transmission and reactivity compared to graphene.



% Figure environment removed

From our MD simulation results, we have observed that CH$_4$ molecules 
can undergo reflection and dissociation, leading to the formation of 
CH$_2$ and H$_2$ as the main process. Figure \ref{fig:dynamics} 
displays a histogram of the counts of the final positions 
of the C and H atoms for the final frames of the simulations for 
graphene (a) and GPNL 
(b), providing a visual representation of the observed dynamics. 
This plot is used to count for the number of events 
with a probability of transmission, reflection, or absorption.
In the figures, we correlated the visualization 
all the events to the histogram by shown the last frame of all the 
MD simulations in a single image where 
C atoms forming graphene and GPNL are represented as 
gray spheres, adsorbed CH$_2$ molecules are depicted as black 
spheres 
for C atoms and white spheres for hydrogen atoms, while reflected 
molecules are represented by turquoise spheres for molecular 
hydrogen 
and purple spheres for C atoms. 
Within our observations, we find that CH$_2$ 
molecules are more likely to binding to the carbon sheets. In 
alternative scenarios, both CH$_2$ and H$_2$ molecules are seen
to dissociate after collision with the carbon sheets. A notable 
observation is the differential behavior of hydrogen molecules in 
graphene compared to GPNL. 

The GPNL's porous structure facilitates 
the transmission of hydrogen molecules, making it a promising 
candidate for applications such as hydrogen production in the 
context of energy generation. While graphene sheets present 
high scattering of H$_2$ molecules.
We have found that CH$_4$ molecules are unable to transmit through 
graphene, whereas in the case of GPNL, there is a probability of 
approximately 50\% for transmission, accompanied by dissociation 
and the 
production of molecular hydrogen. This behavior can be attributed 
to the 
porous nature of the materials and the impact energy of the 
emitted. 
In addition, we analyze the velocity 
distributions 
of carbon (C), oxygen (O), and hydrogen (H) atoms at the last
frame considering all the MD simulations for each case. 
Notably, molecular hydrogen exhibits the highest velocities, 
while CH$_2$ molecules exhibit the lowest 
velocities after being reflected by the carbon sheets. 
This observation provides an initial indication of the bonding 
behavior of the ejected molecules, which is discussed in more
detail within the SM.


Furthermore, we have noted that CH$_2$ molecules can form bonds 
with GPNL by infiltrating the porous structure and binding to the 
underlying C atoms. This property distinguishes GPNL from 
graphene, as the latter typically requires the presence of
defects to enable molecule transmission. 
Here, the distinctive arrangement of 
sp$^2$--carbon atoms in GPNL creates a two--dimensional 
lattice with regularly spaced sized pores which is notably
larger than the kinetic diameters of H$_2$, CO$_2$, and CH$_4$, 
facilitating the favorable diffusion of these molecules. 
This mechanism highlights the potential of GPNL is a 
promising material for gas separation, particularly 
for CH$_4$, as supported by our MD simulations for CO$_2$ 
purification as well. 



Figure \ref{fig:reflected} presents the analysis of reflected 
molecules from the surfaces of graphene and GPNL at an 
impact energy of 8 eV. 
The analysis involves determining the internuclear distance
between atoms in the final frame of the simulations.
In the case of H$_2$ molecules (Figure \ref{fig:reflected}a), we 
observe a uniform distribution of the internuclear distance
around the bond length of 0.74 Å. 
This distribution arises 
from the excitation of vibrational and rotational states 
due to the exchange of kinetic energy during a collision
with the surfaces, as previously discussed.
For CO$_2$ molecules, we observe the splitting of the molecules
into CO and O, with a majority exhibiting a homogeneous distribution 
around 
1.43 Å for the internuclear distance which corresponds to the CO 
bond length. Oxygen atoms are identified with 
an internuclear distance larger than 2.5 Å.
Finally, CH$_4$ molecules undergo splitting into CH$_2$ and H$_2$ 
molecules with the highest probabilities, with CH$_3$+H dissociation 
with a low probability. 
Molecular hydrogen is characterized by an internuclear 
distance of around 0.74 Å and a higher degree of excitation in 
vibrational states. 
We identify CH$_2$ (methylene) molecules as the main dissociation
channel with a peak in the histogram at $~1.8$\AA{} being the 
internuclear distance between H atoms.
The porosity of GPNL makes it a more promising 
candidate for H$_2$ production compared to graphene, 
as demonstrated by MD simulations.
The histogram illustrates the distribution of reflected 
molecules with different bond lengths corresponding to 
their excited vibrational states. For instance, when 
H$_2$ molecules are emitted onto the 2D materials, 
they tend to vibrate towards states close to their 
ground state. On the other hand, H$_2$ molecules originating
from the dissociated methane exhibit various bond lengths
due to their excited vibrational and rotational states. 
This concept also applies to CO$_x$ and CH$_x$ molecules.


% Figure environment removed



%%%%%%%%%%%%%%%%%%%%%%%%%%%%%%%%%%
\subsection{Optical Absorption Spectra, electron transport and sensitivity}
\label{subsec:sensitivity}

%%%%%%%%%%%%%%%%%%%%%%%%%%%%%%%%%%
Before performing the optical absorption 
calculations, We conducted unconstrained optimizations of the
carbon sheet 
in the presence of various molecules and atoms, each in different 
scenarios. In instances where a physisorption pathway was relevant, 
constrained optimization calculations were performed to establish 
the optimal bond lengths. This approach was adopted to simplify the 
computational process, allowing for a more efficient and accurate 
determination of the atomic configurations.

The optical spectra are calculated
for different cases of molecular adsorption: 
1) For hydrogen molecules, we considered a single H$_2$ 
molecule positioned above a carbon atom, as well as two hydrogen 
atoms positioned above two different carbon atoms, where the 
interaction between them is minimal; the obtained results 
are in a good agreement with experimental and ab--initio data where 
the absorption intensities of the interband transitions occurring
in the Dirac band (mid-IR and visible) 
\cite{LEE2016109,Keith,ZHANG2018137}.
2) For CO$_2$ and its split CO+O molecule, we positioned the CO$_2$ 
molecule above a carbon atom, and for the resulting CO+O system, 
we placed it above two different carbon atoms. 
The effect of carbon dioxide on the graphene sheet is similar
to that observed for the H$_2$ molecule. For GPNL, 
the influence decreases compared to the pristine case. 
However, the split of CO+O enhances the absorbance of the graphene sheet 
at the same wavelength as the CO$_2$ molecule. For the GPNL case, 
the absorbance rate is lower than that observed in the carbon
dioxide case. 
3) For methane molecules, we investigated several 
configurations: CH$_4$, CH$_2$+H$_2$ (molecular hydrogen and 
methylene as a main dissociation channel), CH$_2$+H+H, and CH$_2$,
as they are crucial for sensor development 
\cite{GUI2020113959, zhu2016theoretical}. 


Figure \ref{fig:optical} presents the normalized optical absorption
spectrum of graphene (a) and GPNL (b) obtained using
the Liouville--von Neumann equation for the hydrogen molecule, 
methane, and carbon dioxide and the rest of the results are 
displayed in the SM. These findings reveal that pristine 
graphene doesn't interact with visible light, as expected
\cite{Hashemi_2013}. 
However, when it's paired with a hydrogen molecule or carbon 
dioxide, graphene becomes more effective at absorbing visible
light. 
In contrast, the addition of a methane molecule primarily
enhances its ultraviolet absorption capabilities. 
Furthermore, when it comes to the GPNL, the presence of attached 
molecules doesn't exert any discernible effect on the optical 
properties of the carbon sheet. The structural characteristics of 
the pores do not significantly enhance light absorption, and its 
band gap doesn't cause excessive scattering of visible light. This 
underscores the material's stability in maintaining its optical 
properties across diverse conditions. 
In the SM, we noticed that among all the configurations, 
the optical absorbance is
maximized in the 300--450 wavelength range for the 
CH$_2$+H+H case, attributed to the bonding between H 
atoms and the C atom of graphene. In contrast, in the 
case of GPNL, CH$_2$ increases the optical 
absorbance in the range of 400 to 550 wavelengths due to the 
system's hybridization and modifying 
the DOS.


% Figure environment removed


%CH$_4$, CO$_2$ and H$_2$ represents
%there are several new peaks in visible light range 

% Describe figures


Advancements in 2D materials have expanded the scope of potential 
applications, particularly in energy harvesting and storage, due 
to their high electron transport efficiency and extensive surface 
area with numerous active sites \cite{KUMAR2023232256}. In our 
study, we adopted a computational approach based on a $\pi$-
orbital tight-binding Hamiltonian to simulate the electrical 
transport phenomena of graphene and GPNL. We employed the 
SCC-DFTB method along with Non--equilibrium Green's functions 
\cite{Pecchia_2008}.
In the tight--binding representation, the interaction between 
atoms is limited to a finite range. 
We can solve the contact self--energy function, also known 
as the surface Green's function, for 
the matrix block corresponding to the atoms near the extended 
device region using a recursive algorithm \cite{D0CP04188B}. 
This allows us to 
accurately describe the electrical transport properties of the 
materials and analyze their behavior under various conditions.


% Figure environment removed


Figure \ref{fig:transmissionProb} illustrates our findings 
for graphene in a) and GPNL in b)
that are in good agreement with reported results for 
pristine graphene in the literature and by Villegas 
et al. \cite{D0CP04188B} for the pristine GPNL; 
by adding different molecules it is revealed the preferential 
adsorption of extrinsic chemical 
species like CH$_4$, CO$_2$, and H$_2$ at interdomain sites, 
leading to a significant enhancement of scattering effects in 
both graphene and GPNL. To delve into this intriguing 
behavior, we employed the non--pulsating direct current (NPDC) 
waveform, known for its ability to maintain the stability of 
adsorbates even under ultrahigh--vacuum conditions. Unlike PDC 
waves, which exhibit continuous voltage fluctuations, NPDC waves 
maintain a constant voltage, ensuring the stability of the 
adsorbed species within the scope of our computational approach. 
For graphene, it is noteworthy that the 
adsorption of individual hydrogen, carbon dioxide, and methane 
molecules leads to a reduction in the transmission probability. 
This reduction has a direct impact on the density of states
within the altered system. Additionally, it is observed that 
the effects on the 
DOS of the armchair--wise graphene sheet is associated 
with the bond formed between carbon atoms and these molecules; 
which are reflected in the transmission probability of two symmetric 
minima around the Fermi.
On the other hand, when considering the GPNL sheet, the presence of 
singly bonded molecules to carbon atoms does not appear to induce 
substantial modifications in the transmission probability. 
However, there is a decrease in the transmission probability 
within the energy range of approximately ±1.5 to 1.0 eV. This
observation highlights the potential of GPNL as a promising 
candidate for the development of gas sensors, emphasizing its 
sensitivity to changes in its surrounding gaseous environment.
Since, the strong mechanical properties play a key role in 
maintaining the structural integrity of porous frameworks, 
preventing their shrinkage or collapse. Therefore, the 
presence of channels and pores facilitates rapid electrolyte 
diffusion, leading to an augmentation in electrical conductivity 
as shown by our results. 


% Figure environment removed


%Smith et al. \cite{smi} predicted that the band gap of 1.22 eV, similar to that of silicon, would render GPNL as a possible replacement in silicon electronic devices; for example, GPNL is a better candidate than graphene for use in field effect transistors. Froudakis investigated the possibility of using hydrogenation to tailor the electronic properties of GPNL.

Fig \ref{fig:I-U} shows the difference 
of the current for each molecule $X$ as a function of
the voltage as:
\begin{equation}
    S = 100 \% \frac{\big| I_{X}-I_g \big|}{I_{X}},
\end{equation}
with $I_{X}$ of each molecule and the surface and 
$I_g$ the current of the graphene in a) 
and GPNL in b). Noticing that the adsorption of CH$_x$ compounds 
increases the sensitivity of the surfaces.
The tunneling currents for the sheets as a function of 
the voltage for various adsorbed molecules are shown in 
the inset plots by I–-U characteristics graphs for 
voltages below 300 mV.
Our results are in a qualititive good agrement with reported 
experimental data \cite{SHABAN20194510} showing 
a decrease of 25 mV for GPNL and an efficiency increas of 
4\% for methane single molecules. 
Thus, the porous structure of GPNL offers distinct 
advantages, including increased surface area, reduced density, 
and improved accessibility to guest objects. 
This structure is highly suitable for applications involving 
light absorption and electron/ion transport. 
Specifically, it shortens the migration path of charge carriers
from the point of generation to the active surface, 
thereby facilitating electron migration to the surface.
Therefore, we suggest that GPNL based materials can be 
better candidates than graphene ones for use in field effect 
transistors, as Zener diodes, and faster sensors.

\section{Conclusion}\label{sec:conclusion}

This paper presents our empirical domain knowledge distillation framework using ChatGPT and discusses our observations from the framework application experiments in the autonomous driving domain. The key finding is that: 1) with proper design of prompt engineering and execution flow, fully automated domain knowledge (in the ontology format) distillation is possible. However, due to the randomness in the response and the butterfly effect, the quality of fully automated distillation results is not guaranteed. To address this, we develop a web-based assistant to enable manual supervision and early intervention at runtime. We hope our findings and tools inspire future research toward revolutionizing the engineering processes of knowledge-based systems across domains.
%\appendix*
%\begin{comment}
\section{System Architecture}
\label{appendix:architecture}
\system has a novel modularized system architecture with three key components: 
\emph{StreamManager}, 
\emph{TxnManager} and \emph{TxnScheduler}. 
These components are instantiated in each thread locally.
The execution outline of \system is presented in Algorithm~\ref{alg:algo}.
Transactional stream processing is continuous and potentially never ends (Line 1$\sim$8).
The dependency resolution and execution of state transactions are separated into two non-overlapping phases by punctuations~\cite{Tucker:2003:EPS:776752.776780} (Line 2 and 5), which guarantees that no subsequent input event will have a smaller timestamp. 
Effectively, a batch of state transactions is collected during the first phase, and processed during the second phase.

In the first phase (i.e., stream processing phase), 
the \emph{StreamManager} conducts preprocessing for every input event ($e$). Similar to some prior works~\cite{tstream}, state transactions may be issued but not immediately processed during preprocessing (Line 3).
The \emph{pre\_processing} and \emph{post\_processing} functions are exposed as APIs to users.
The \emph{TxnManager} handles dependency resolution (Line 4) among state transactions and insert decomposed operations to construct a \tpg. We discuss the detailed two-phase \tpg construction process in Section~\ref{subsec:construction}.

In the second phase  (i.e., transaction processing phase), 
the \emph{TxnManager} is first involved again to refine (Line 6) the constructed \tpg with further dependency resolution.
The \emph{TxnScheduler} 
schedules operations for concurrent execution based on the constructed \tpg according to the three dimensions of scheduling decisions (Line 7). 
In particular, a scheduling decision model $M$ is instantiated based on the constructed \tpg (Line 14).
\textbf{\circled{1}} Guided by $M$, execution threads adopt an exploration strategy (Section~\ref{subsec:explore}) to explore the constructed \tpg for operations available to be scheduled constrained by dependencies. 
\textbf{\circled{2}} 
During exploration, one or multiple operations may be treated as the 
% basic 
unit of scheduling (Section~\ref{subsec:granularity}). 
Subsequently, \textbf{\circled{3}} every thread executes operation(s) in the unit of scheduling with various abort handling mechanisms (Section~\ref{subsec:abort_handling}).
Only when state transactions are processed (i.e., committed or aborted) can the associated input events be postprocessed (Line 8) by the \emph{StreamManager} based on transaction processing results.
\end{comment}

\begin{comment}
\begin{algorithm}
\footnotesize
    \KwData{$e$ \tcp{Input event}}
    \KwData{$txn_{ts}$ \tcp{State transaction}}
    \KwData{$G$ \tcp{The currently constructed TPG}}
    \While{!finish processing of input streams}{
        \eIf(\tcp*[h]{Phase 1}){\text{$e$ is not a $punctuation$}}{
                $txn_{ts}$ $\gets$ PRE\_Processing($e$)\;
                \textbf{TPG\_Construction}($G$, $txn_{ts}$)\; 
          }(\tcp*[h]{Phase 2}){
                \textbf{TPG\_Refinement}($G$)\; 
                \textbf{TXN\_Scheduling}($G$)\; 
                POST\_Processing()\;
          }
    }
    
    \SetKwFunction{FMain}{TPG\_Construction}
    \SetKwProg{Fn}{Function}{:}{}
    \Fn{\FMain{$G$, $txn_{ts}$}}{
        $O_{1..k}$ $\gets$ \textbf{Partition} $txn_{ts}$\;
        \ForEach{\text{operation $O_{i}$ $\in$ $O_{1..k}$}}{
            \textbf{Identify} its \ld\;
            $G$ $\gets$ $G$ + $O_{i}$ \;
        }
    }
    \SetKwFunction{FMain}{TPG\_Refinement}
    \SetKwProg{Fn}{Function}{:}{}
    \Fn{\FMain{$G$}}{
        \ForEach{\text{vertex $e_{i}$ $\in$ $G$}}{
            \textbf{Identify} its \td, \pd\;
        }
    }
    
    \SetKwFunction{FMain}{TXN\_Scheduling}
    \SetKwProg{Fn}{Function}{:}{}
    \Fn{\FMain{$G$}}{
        $M$ $\gets$ Instantiated with $G$;\tcp{A decision model}
        \While{!finish scheduling of $G$
        }{
          \textbf{\circled{2}} $Scheduling Unit$ $\gets$ \textbf{\circled{1}} \emph{Explore}($G$, $M$)\; 
            \textbf{\circled{3}} \emph{Execute with Abort Handling} ($Scheduling Unit$)\; 
        }
    }
  \caption{Execution Outline of \system}
  \label{alg:algo}
\end{algorithm}
\end{comment}

\section*{Acknowledgements}
We acknowledge support from the European Union Horizon 2020 
research
and innovation program under grant agreement no. 857470,
from the European Regional Development Fund via 
the Foundation for Polish Science International 
Research Agenda PLUS program grant No. MAB PLUS/ 2018/8, 
and  INNUMAT project (Grant Agreement No. 101061241). 
We acknowledge the computational resources 
 provided by the High Performance Cluster at the National Centre 
 for Nuclear Research and 
 the Interdisciplinary Centre for Mathematical and
 Computational Modelling (ICM) University of Warsaw under 
 computational allocation no g91--1427.

%%%%%%%=4=============
%\section*{References}
\bibliography{references}
\bibliographystyle{iopart-num}

\end{document}

