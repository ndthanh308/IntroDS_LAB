\section{Introduction}
\label{sec:intro}

%In the face of the depletion of fossil fuel reserves and increasing ecological 
%contamination, the imperative to harness renewable and eco-friendly energy sources 
%has become apparent. Hydrogen (H$_2$) has garnered significant recognition as a 
%replacement for conventional fossil fuels due to its high chemical energy per 
%mass, eco-friendly combustion by-products, abundant natural reserves, and 
%renewable nature \cite{van2008materials, song2011improvement}. However, the issue 
%of hydrogen storage remains a significant hurdle to accomplishing efficient, cost-
%effective, and secure on-board hydrogen storage. Achieving the ability to store 
%substantial quantities of H$_2$ at or near room temperature would constitute a 
%game-changing technology, and is thus a desirable albeit formidable objective for 
%researchers in this domain \cite{graetz2009new}.


%%%%%%%%%%%%%%%%bura kimi copya elemisem: I have no clue what you wrote
Sensors detect gases through the physical 
adsorption of gas molecules onto a surface. 
These sensors use a gas--sensitive material, such as 
carbon--based ones, that change their properties 
upon gas adsorption, offering high sensitivity and 
selectivity \cite{JANA2022108543}. 
The high sensitivity of graphene to the local environment
has shown to be highly advantageous in sensing applications,
where ultralow concentrations of adsorbed molecules induce
a significant response to the electronic properties of
graphene 
\cite{balandin2011thermal,Wang_2022,PhysRevB.77.125416,C6CP07654H,SHABAN20194510,ALIGAYEV2022355}. 
Additionally, carbon--based materials can be tailored by
varying their surface chemistry, porosity, and morphology.
The hybridization of carbon atoms into sp--, sp$^2$--, and
sp$^3$--orbitals can create diverse carbon allotropes
exhibiting distinct dimensionalities \cite{shanmugam2022review,gao2023tunable}. 
Ongoing research focuses on improving the performance and 
reliability of gas adsorption sensors and exploring 
new materials and sensing mechanisms for enhanced gas sensing 
capabilities 
\cite{hirsch2010era, karfunkel1992new, baughman1987structure, li2009superhard, sheng2011t,C5CP04422G}.

%%%%%%%%%%%%%%%%%%%%%%%%%%%%burdan asagisi pdf 14 dendir ve
Graphenylene (GPNL), an intriguing carbon allotrope sharing
the same point group (D$_{6h}$) as graphene, is composed of 
sp$^2$--hybridized carbon atoms arranged in hexatomic and
tetratomic rings \cite{balaban1968chemical, balaban1994diamond,C2TC00006G, D0CP04188B}. 
Qi-Shi Du et al. successfully synthesized layers of graphenylene, also referred
to as biphenylene--carbon (BPC), by dehydrating and polymerizing 1,3,
\newline
5--trihydroxybenzene \cite{du2017new,zhang2019art}. 
The process involved the removal of three water molecules
from a 1,3,5--trihydroxybenzene molecule using dehydrant aluminum oxide, leading to the
amalgamation of bare 6C rings (benzynes) and the formation of a small segment of
the 2D carbon crystal. Furthermore, polymerization could also take place through
intermolecular dehydration between 1,3,5-trihydroxybenzene molecules, 
where the fragments of the 2D carbon crystal grew rapidly. The experimental
construction of GPNL involved utilizing planar 4-carbon rings and 6-carbon rings
with sp$^2$ electron configuration, experiencing slight distortions that ultimately
resulted in the formation of a large planar conjugated $\pi$-system \cite{brunetto2012nonzero}.
It possesses a hexagonal lattice structure with periodic 
pores, offering a high surface area and pore volume. 
These characteristics make GPNL a promising 
material for gas adsorption and separation applications.
The unique topology of GPNL allows for selective 
adsorption of specific gas molecules, making it a potential 
candidate for highly efficient and selective gas separation
and  storage processes 
\cite{KOCHAEV2021109999, Martins2022,D0RA04286B}. 
The identification of hollow adsorption 
sites in GPNL is of great interest, as these sites hold 
significant potential for various applications, including gas 
separation \cite{xu2017inorganic, zhu2016theoretical, 
rezaee2020graphenylene, motallebipour2021graphenylene}.
%GPNL can be functionalized with alkali and alkaline earth 
%metals, which further enhances its potential as a high-capacity 
%hydrogen storage material \cite{Hussain2017,gmurek}. 


The growing importance of air quality and 
safety has created a demand for advanced 
gas sensors. Porous carbon--based materials 
have emerged as promising candidates for 
such sensors due to their comparable 
electronic mobility and mechanical 
properties to graphene. Additionally, 
these materials offer the added advantage 
of enabling the dispersion of single atoms 
within acetylenic pores.
Building upon the research progress in graphene, 
investigations into post--graphene 2D carbon--based materials 
have swiftly demonstrated diverse electronic devices
and emerging charge transport phenomena. 
However, despite the growing understanding of electronic 
transport in individual 2D materials, practical 
wafer--scale implementation faces significant challenges 
\cite{Sangwan,D1CP01890F}. 
Therefore, the development of reliable techniques for 
wafer--scale growth, ensuring uniformity and predictable 
thickness poses considerable hurdles from a materials
science perspective. 
Thus, carbon materials exhibit versatile bonding abilities,
ranging from sp$^1$ to sp$^3$ hybridization, and encompass
a variety of allotropes, including fullerene, graphite,
diamond, graphene, carbon nanotubes, and fibers 
\cite{CHUNG2023215066,WandOganov,Keith}. 
Porous carbons can be obtained through the carbonization
of natural or synthetic precursors, followed by activation,
enabling tunability of pore sizes across a wide range, 
from micropores ($<$ 2 nm) to mesopores (2-50 nm) and
macropores ($>$ 50 nm). 
Diverse synthesis strategies, such as template methods,
etching of metal carbides, and sol--gel processing, 
have been explored to create porous carbon materials
with controlled pore structures at both the 
micropore and mesopore levels \cite{CHUNG2023215066}. 
These porous carbons find applications in crucial fields
such as adsorption, separation, and electrode materials.

GPNL, with its exceptional porous architecture and 
remarkable electronic features hold great promise as a 
material for the development of high--performance molecular
gas sensors. In order to save financial resources and avoid 
exhaustive experimental trials, detailed atomistic simulations 
are essential to complement practical exploration.
In this study, we employ the Self--Consistent--Charge Density--
Functional Tight--Binding (SCC--DFTB) method \cite{DFTBplus, Qiang} 
to investigate the potential applications of GPNL in the 
detection of important molecular gases such as CH$_4$ and CO$_2$ 
and their species, which have significant environmental
implications \cite{Santos_2021, ARUNRAGSA2020107790}. 
Here we investigate the hydrogen production through methane 
dissociation and CO$_2$ reduction mechanisms by emitting 
CH$_4$, and CO$_2$ molecules with impact energies close to
their dissociation energies and studying their interactions with 
GPNL. Additionally, we also compare our results with the findings 
obtained for graphene. Further computational research that 
closely emulates dynamic mechanisms observed in experiments is 
necessary to fully explore the potential of graphenylene, 
including its effects on optical absorption, electron transport 
performance, and enhancement of material sensitivity. Our primary 
objective is to contribute to the characterization of 
GPNL as a promising material for future research and the 
development of materials for ultrafast gas sensors and gas separation 
applications.