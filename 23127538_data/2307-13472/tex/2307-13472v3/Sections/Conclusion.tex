\section{Conclusion}
\label{sec:Conclusion}

In this study, we utilize computer simulations to explore the gas 
separation mechanism and its impact on the optical and electronic 
properties of graphene and graphenelyne sheets. Our focus is 
specifically 
on the emission behavior of CH$_4$ and CO$_2$ molecules. To 
conduct these 
simulations, we employ a quantum-classical molecular dynamics 
approach, 
utilizing the SCC--DFTB method with van der Waals corrections. 
These 
corrections are essential to accurately capture the dissociation, 
chemisorption, and molecule formation processes involved in the 
dynamics.

We analyze the probabilities of transmission, reflection, and 
adsorption 
of the emitted molecules. Our results highlight that GPNL 
exhibits significant advantages in gas separation compared to graphene. 
Specifically, we find that GPNL enables efficient separation of 
CO$_2$ into CO+O and CH$_4$ into CH$_2$+H$_2$ with the highest probabilities to be dissociated. The porosity of GPNL enhances gas 
separation rates, facilitates CO$_2$ purification, and promotes hydrogen 
production from methane.
By conducting electron transport calculations with the 
non--equilibrium green function method, we noticed that 
hydrocarbon like CH$_2$, and CH$_4$ 
have the most effects on the electron 
transport mechanisms for both 2D materials. 