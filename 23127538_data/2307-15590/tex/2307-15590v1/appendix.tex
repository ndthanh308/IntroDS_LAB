\appendix

\section{Proof of~\texorpdfstring{\Cref{thm:optimality-system}}{Theorem~\ref{thm:optimality-system}}}\label{app:proof-optimality-system}
For simplicity, we omit the dependence of the involved quantities on the parameter~$\mu\in\params$.
\begin{proof}
    Let~$u^*\in G$ denote an optimal control, $x^*\in H$ the corresponding state trajectory. We are going to prove that the first variation of~$\mathcal{J}$ vanishes if~$x^*$, $\varphi^*$ and~$u^*$ solve the boundary value problem stated above. To this end, let~$v\in G$, and consider the perturbation~$u\in G$ of~$u^*$ defined as
    \[
        u(t) \coloneqq u^*(t) + \varepsilon v(t)
    \]
    for~$\varepsilon\in\setR$. Hence, the state equation for the control~$u$ reads
    \[
        \dot{x}(t) = Ax(t)+Bu^*(t)+\varepsilon Bv(t)\qquad\text{for }t\in[0,T].
    \]
    The solution is explicitly given by
    \begin{align*}
        x(t) &= e^{At}x^0 + \int\limits_0^t e^{A(t-s)}\big(Bu^*(s)+\varepsilon Bv(s)\big)\d{s} \\
        &= x^*(t) + \varepsilon\int\limits_0^t e^{A(t-s)}Bv(s)\d{s} \\
        &= x^*(t) + \varepsilon z(t)
    \end{align*}
    for~$z\in H$ defined as~$z(t)\coloneqq\int_0^t e^{A(t-s)}Bv(s)\d{s}$. It holds that~$z$ satisfies the ordinary differential equation
    \[
        \dot{z}(t) = Az(t)+Bv(t),\qquad z(0)=0.
    \]
    Introducing the adjoint state~$\varphi\in H$ and the Hamiltonian function~$\mathcal{H}\colon\X\times\U\times\X\to\setR$ given as
    \[
        \mathcal{H}(x(t),u(t),\varphi(t)) = \frac{1}{2} \langle u(t), Ru(t) \rangle + \langle \varphi(t), \big(Ax(t)+Bu(t)\big)\rangle ,
    \]
    we can rewrite the functional~$\mathcal{J}$ as
    \[
        \mathcal{J}(u) = \frac{1}{2}\langle x(T)-x^T, M\left(x(T)-x^T\right)\rangle + \int\limits_0^T \mathcal{H}(x(t),u(t),\varphi(t)) - \langle \varphi(t), \dot{x}(t)\rangle \d{t},
    \]
    since~$x\in H$ solves the state equation. This holds for any adjoint state~$\varphi\in H$. Similarly, for the optimal control~$u^*$ and corresponding state trajectory~$x^*$ it holds
    \[
        \mathcal{J}(u^*) = \frac{1}{2}\langle x^*(T)-x^T ,M\left(x^*(T)-x^T\right)\rangle + \int\limits_0^T \mathcal{H}(x^*(t),u^*(t),\varphi(t)) - \langle \varphi(t), \dot{x}^*(t)\rangle \d{t}.
    \]
    We now consider the difference~$\mathcal{J}(u)-\mathcal{J}(u^*)$, which is given as
    \begin{equation}\label{equ:difference-cost-functional}
        \begin{aligned}
            \mathcal{J}(u)-\mathcal{J}(u^*) &= \frac{1}{2}\left[\langle x(T)-x^T, M\left(x(T)-x^T\right)\rangle - \langle x^*(T)-x^T, M\left(x^*(T)-x^T\right)\rangle \right] \\
            & \hphantom{==}+ \int\limits_0^T \mathcal{H}(x(t),u(t),\varphi(t))-\mathcal{H}(x^*(t),u^*(t),\varphi(t))\d{t} \\
            & \hphantom{==}+ \int\limits_0^T \langle \varphi(t), \dot{x}^*(t)-\dot{x}(t)\rangle \d{t}.
        \end{aligned}
    \end{equation}
    We obtain for the first term in~\cref{equ:difference-cost-functional} the identity
    \begin{align*}
        & \frac{1}{2}\left[\langle x(T)-x^T, M\big(x(T)-x^T\big) \rangle-\langle x^*(T)-x^T, M\big(x^*(T)-x^T\big)\rangle \right] \\
        &= \frac{1}{2}\left[\langle x^*(T)+\varepsilon z(T)-x^T, M\big(x^*(T)+\varepsilon z(T)-x^T\big) \rangle-\langle x^*(T)-x^T, M\big(x^*(T)-x^T\big)\rangle \right] \\
        &= \varepsilon \langle z(T), M\big(x^*(T)-x^T\big)\rangle +\mathcal{O}(\varepsilon^2),
    \end{align*}
    where we used that~$M$ is self-adjoint. For the difference of the Hamiltonians in the second term in~\cref{equ:difference-cost-functional} it holds
    \begin{align*}
        & \mathcal{H}(x(t),u(t),\varphi(t))-\mathcal{H}(x^*(t),u^*(t),\varphi(t)) \\
        &= \frac{1}{2} \langle u(t), Ru(t)\rangle +\langle \varphi(t),Ax(t)+Bu(t)\rangle - \frac{1}{2} \langle u^*(t), Ru^*(t)\rangle - \langle \varphi(t), Ax^*(t)+Bu^*(t)\rangle \\
        &= \frac{1}{2}\langle u^*(t) + \varepsilon v(t), R\big(u^*(t) + \varepsilon v(t)\big)\rangle +\langle \varphi(t), Ax^*(t)+\varepsilon Az(t)+Bu^*(t)+\varepsilon Bv(t)\rangle\\
        & \hphantom{==}- \frac{1}{2}\langle u^*(t), Ru^*(t)\rangle - \langle\varphi(t), Ax^*(t)+Bu^*(t)\rangle \\
        &= \varepsilon \langle u^*(t), Rv(t)\rangle+\langle \varphi(t), \varepsilon Az(t)+\varepsilon Bv(t)\rangle + \mathcal{O}(\varepsilon^2) \\
        &= \varepsilon\Big[\langle u^*(t), Rv(t)\rangle+\langle \varphi(t), Az(t)\rangle +\langle \varphi(t), Bv(t)\rangle \Big] + \mathcal{O}(\varepsilon^2) \\
        &= \varepsilon\Big[\langle Ru^*(t)+B^*\varphi(t), v(t)\rangle  + \langle \varphi(t), Az(t)\rangle\Big] + \mathcal{O}(\varepsilon^2)
    \end{align*}
    for all~$t\in[0,T]$, where we used that~$R$ is a self-adjoint operator. Further, recall that it holds~$x(t)=x^*(t)+\varepsilon z(t)$ and therefore~$\dot{x}^*(t)-\dot{x}(t)=-\varepsilon\dot{z}(t)$ for all~$t\in[0,T]$. For the last term in~\cref{equ:difference-cost-functional} we thus obtain via integration by parts
    \begin{align*}
        \int\limits_0^T \langle \varphi(t), \dot{x}^*(t)-\dot{x}(t)\rangle \d{t} &= -\varepsilon\int\limits_0^T \langle \varphi(t), \dot{z}(t)\rangle\,\d{t} \\
        &= \big[-\varepsilon\langle\varphi(t), z(t)\rangle\big]_0^T + \varepsilon\int\limits_0^T \langle \dot{\varphi}(t), z(t) \langle \d{t} \\
        &= -\varepsilon\langle \varphi(T), z(T)\rangle + \varepsilon\int\limits_0^T \langle \dot{\varphi}(t), z(t)\rangle \d{t},
    \end{align*}
    where we used that it holds~$z(0)=0$. Bringing everything together gives
    \begin{align*}
        \mathcal{J}(u)-\mathcal{J}(u^*) &= \varepsilon\left[\langle z(T), M\big(x^*(T)-x^T\big)\rangle \vphantom{\int\limits_0^T} + \int\limits_0^T \langle Ru^*(t)+B^*\varphi(t), v(t) \rangle + \langle\varphi(t), Az(t)\rangle \d{t} \right. \\
        & \hphantom{=\varepsilon=}\left. + \int\limits_0^T \langle \dot{\varphi}(t), z(t) \rangle \d{t} - \langle \varphi(T), z(T)\rangle \right] + \mathcal{O}(\varepsilon^2) \\
        &= \varepsilon\left[ \langle z(T), M(x^*(T)-x^T-\varphi(T)\rangle \vphantom{\int\limits_0^T} + \int\limits_0^T \langle Ru^*(t)+B^*\varphi(t), v(t) \rangle \d{t} \right. \\
        & \hphantom{=\varepsilon=}\left. + \int\limits_0^T \langle \dot{\varphi}(t)+A^*\varphi(t), z(t)\rangle \d{t}\right] + \mathcal{O}(\varepsilon^2).
    \end{align*}
    Since~$u^*$ is assumed to be an optimal control, it has to hold for all~$v\in \U$ that
    \begin{align*}
        0 &= \lim\limits_{\varepsilon\to 0}\frac{\mathcal{J}(u^*+\varepsilon v)-\mathcal{J}(u^*)}{\varepsilon} \\
        &= \lim\limits_{\varepsilon\to 0}\frac{\mathcal{J}(u)-\mathcal{J}(u^*)}{\varepsilon} \\
        &= \langle z(T), M(x^*(T)-x^T)-\varphi(T)\rangle \vphantom{\int\limits_0^T} \\
        & \hphantom{==}+ \int\limits_0^T \langle Ru^*(t)+B^*\varphi(t), v(t) \rangle \d{t} + \int\limits_0^T \langle \dot{\varphi}(t)+A^*\varphi(t), z(t)\rangle \d{t}.
    \end{align*}
    This yields the claimed necessary conditions for~$u^*$, $x^*$ and~$\varphi^*$.
\end{proof}