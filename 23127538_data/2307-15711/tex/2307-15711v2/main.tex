\documentclass[a4paper,11pt]{article}
\pdfoutput=1 
\usepackage{jcappub} 
\usepackage[T1]{fontenc}

\usepackage{comment}


\usepackage{subfig}
\newcommand{\gsim}{\raisebox{-0.13cm}{~\shortstack{$>$ \\[-0.07cm]
      $\sim$}}~}
\newcommand{\lsim}{\raisebox{-0.13cm}{~\shortstack{$<$ \\[-0.07cm]
      $\sim$}}~}
      
\newcommand{\MC}[1]{{\color{blue}{[{\bf Matteo:~#1}]}}}
\newcommand{\VC}[1]{{\color{teal}{[{\bf Viviana:~#1}]}}}

\usepackage{comment}
\usepackage{graphicx}
\usepackage{amsmath}
\usepackage{amssymb}
\usepackage{lipsum}
\usepackage{comment}
\usepackage{multirow}
\usepackage{xcolor}
\usepackage{hyperref}
\usepackage{multicol}
\usepackage{float}
\usepackage{soul}
%\usepackage{mathtools}
%\usepackage{latexsym}
%\usepackage{pifont}
%\usepackage{adjustbox}
%\usepackage{ae,aecompl}
%\usepackage{subfigure}
%\usepackage{soul}
%\usepackage{bm}
%\usepackage[normalem]{ulem}
\setlength\parindent{10pt}

%================= MACROS ==================
\usepackage{aas_macros}
%===========================================

%============= USEFUL COMMANDS =============
\newcommand{\LCDM}{$\Lambda$CDM }
\newcommand{\nuLCDM}{$\nu\Lambda$CDM }
%===========================================
\newcommand{\CC}[1]{\textbf{\textcolor{magenta}{[Melita: #1]}}}
\newcommand{\cc}[1]{\textbf{\textcolor{magenta}{#1}}}
\newcommand{\MM}[1]{\textbf{\textcolor{orange}{[MM suggerisce: #1]}}}

\title{\centering{\boldmath{\fontsize{19}{11}\selectfont DEMNUni: cross-correlating the nonlinear ISWRS effect with CMB-lensing and galaxies\\ \vspace{-0.5cm} in the presence of massive neutrinos}}}


\author[a,b]{Viviana Cuozzo}%,\note{Corresponding author.}}
\author[c]{Carmelita Carbone}
\author[d]{Matteo Calabrese}
\author[c,e]{Elisabetta Carella}
\author[a,b]{and Marina Migliaccio}

\affiliation[a]{Università degli Studi di Roma Tor Vergata, via della Ricerca Scientifica 1, I-00133 Roma, Italy}
\affiliation[b]{INFN, Sezione di Roma 2, Università degli Studi di Roma Tor Vergata, via della Ricerca Scientifica 1, I-00133 Roma, Italy }
\affiliation[c]{INAF -- Istituto di Astrofisica Spaziale e Fisica cosmica di Milano (IASF-MI), via Alfonso Corti 12, I-20133 Milano, Italy}
\affiliation[d]{Astronomical Observatory of the Autonomous Region of the Aosta Valley (OAVdA), Loc. Lignan 39, I-11020, Nus (Aosta Valley), Italy}
\affiliation[e]{Dipartimento di Fisica ``Aldo Pontremoli'', Universit\`{a} degli Studi di Milano, via Celoria 16, I-20133 Milano, Italy}

% e-mail addresses: one for each author, in the same order as the authors
\emailAdd{vcuozzo@roma2.infn.it}


\emergencystretch 6pt

\abstract{
We present an analytical modelling to study the cross-correlations between the Integrated Sachs Wolfe--Rees Sciama (ISWRS) effect and large-scale structure tracers in the presence of massive neutrinos. Our method has been validated against large N-body simulations with a massive neutrino particle component, namely the DEMNUni suite. We investigate the impact of different neutrino masses on the cross-correlations between the ISWRS and both the galaxy clustering and the lensing of the Cosmic Microwave Background (CMB). We show that the position of the sign inversion due to nonlinear effects is strongly related to the neutrino mass and we model such a dependence. While these nonlinear cross-correlation signals may not be able alone to constrain the neutrino mass, our approach paves the way to the detection of such cross-spectra on small scales for their exploitation in combination with the main probes from future galaxy surveys and CMB experiments.
}

\begin{document}

\maketitle 

%\flushbottom

\section{Introduction}
\label{sec:intro}
Since its discovery in 1965~\cite{Penzias_1965}, the CMB has provided the most significant evidence for the Big Bang Theory and the study of its anisotropies has given us a unique window onto the primordial Universe. Measurements from WMAP~\cite{Bennett_2003,Spergel_2003,Komatsu_2011} and  Planck~\cite{PlanckXVI_2014, PlanckXIII_2016, PlanckVI_2020} have consistently been in line with the predictions of the $\Lambda$CDM model, the so-called Standard Cosmological Model, and have been fundamental in inferring constraints on the cosmological parameters that characterise it. However, there are still many open questions in cosmology, and several tensions exist in the Standard Cosmological Model. One of the most investigated topics today is the nature of Dark Energy (DE), which appears to dominate the energy density of the Universe at late times (see eg ~\cite{Linder_2003}), counteracting the gravitational collapse and inducing a background accelerated expansion. Another example is the value of the sum of the three neutrino masses, $M_{\nu} = \sum m_{\nu}$, whose best bounds come from cosmological data~\cite{Lesgourgues_2012, Arcidiacono_2020, PlanckVI_2020}. 

Among several cosmological probes, both these issues can be addressed also by measuring and analysing the late Integrated Sachs Wolfe (ISW) effect~\cite{Sachs_1967}, which is the temperature variation that CMB photons experience when they propagate in a time-varying gravitational potential, and the Rees Sciama (RS) effect~\cite{Rees_1968}, which is its nonlinear counterpart. DE is mainly responsible for the combined ISWRS signal \cite{Ferraro_2015}, but it is not the unique source of such effect. The presence of massive neutrinos induces a slow decay of the gravitational potential~\cite{Bond_1980} that generates ISWRS even in the absence of a background expansion. Therefore, a full reconstruction of the ISWRS effect would provide new information on the physics of neutrinos and of DE, allowing to help constraining the DE Equation of State (EoS) and the total neutrino mass. However, in this respect thre has not been much progress as this signal is extremely faint compared to CMB primary anisotropies. Moreover, these measurements are limited in the linear regime by cosmic variance and in the nonlinear regime by the lack of data. The only way to reconstruct the ISWRS is to use its cross-correlation with the Large Scale Structure (LSS), in order to exploit the tracers that follow the dark matter field and the variation of the gravitational potential produced by DE and massive neutrinos~\cite{Crittenden_1996, Kneissl_1997, Boughn_1998, Boughn_2001, Boughn_2004, Giannantonio_2008, Lesgourgues_2008, Douspis_2008}.  

In this work, we focus on the cross-correlations of the ISWRS with the CMB-lensing potential and the galaxy distribution. As the RS effect anti-correlates with both of them~\cite{Cai_2009, Nishizawa_2014, Carbone_2016}, these cross-spectra are characterised by the presence of a sign inversion that occurs when passing from the linear (ISW) to the nonlinear (RS) regime. Moreover, the position of the sign inversion, which indicates the appearance of nonlinearities, shifts towards smaller cosmological scales due to the suppression of the matter power spectrum resulting from the presence of massive neutrinos~\cite{Cai_2009, Carbone_2016}. 

We develop an analytical method to produce the ISWRS spectra using the nonlinear modelling of the matter power spectrum implemented in the Boltzmann solver code \texttt{CAMB}\footnote{\url{https://CAMB.readthedocs.io/en/latest/}}~\cite{Lewis_2011}. Given the large redshift extension investigated in this work, we consider only two of the revised \texttt{Halofit}~\cite{Smith_2003, Cooray_2002} models implemented: the Takahashi2012+Bird2014 model~\cite{Takahashi_2012}, extended to massive neutrino cosmologies by including Bird correction\footnote{As reported in the Readme of the \texttt{CAMB} webpage, on March 2014 modified massive-neutrino parameters were implemented in the nonlinear fitting of the total matter power spectrum to improve the accuracy with the updated version of halofit from~\cite{Takahashi_2012}. The fitting parameters that account for nonlinear corrections in the presence of massive-neutrinos are different from the ones
implemented by~\cite{Bird_2012} in the original HF version from Smith et al. (2003).}; the Mead2020 model~\cite{Mead_2021}, which is an improvement over the Mead2016 model~\cite{Mead_2015} concerning the treatment of Baryon Acoustic Oscillations (BAO) nonlinear damping. 
We then validate this analytical approach by comparing it with signals extracted from the ``Dark Energy and Massive Neutrino Universe'' (DEMNUni) N-body numerical simulations~\cite{Carbone_2016}.

These studies are particularly relevant in light of ongoing and forthcoming galaxy surveys, such as Euclid~\cite{Laureijs_2011, EuclidVII_2020, EuclidI_2022} and SKA~\cite{SKA_cosmology, SKA_cross2, SKA_cross, SKA_LS, SKA_WL, SKA_fundamentalPhys,SKA_Cosmo_Fund}, and CMB surveys, such as CMB-S4~\cite{Abazajian_2016, Abazajian_2022}, which aim to achieve measurements of small scales with high precision and accuracy, such that the possibility of detecting the RS effect could become concrete~\cite{Ferraro_2022}. 

This paper is organised as follows. In Section~\ref{sec:theory} we recap the theoretical framework of the cross-correlations of the ISWRS signal with the CMB-lensing potential and the galaxy distribution in the presence of massive neutrinos. In Section~\ref{sec:simulations} we present how we model ISWRS nonlinearities using measurements from the DEMNUni simulations. In Section~\ref{sec:XC-TP} we test the analytical method for the cross-correlation between the ISWRS signal and the CMB-lensing potential against the DEMNUni simulated cross-spetrum. In Section~\ref{sec:XC-TG} we validate the cross-correlation between the ISWRS signal and the galaxy distribution against the cross-spectrum measured from the DEMNUni mock maps. In  Section~\ref{sec:XC-TG} we evaluate the predictability of $M_{\nu}$ via the estimation of the sign inversion position and model the ISWRS-galaxy cross-spectrum for different redshift ranges. Finally, in Section~\ref{sec:conclusion} we draw our conclusions.

\section{The Integrated Sachs Wolfe and the Rees Sciama effect}
\label{sec:theory}
When CMB photons pass through a time-varying gravitational potential well along their path from the last scattering surface to us, they 
undergo a variation in energy that translates into a variation in temperature known as the Integrated Sachs Wolfe effect:
\begin{equation}
    \frac{\Delta T_{\rm ISW}}{T_{0}} (\hat{\boldsymbol{n}}) = \frac{2}{c^{2}} \int^{t_{0}}_{t_{ls}} {\rm d}t \, \dot{\Phi}(\hat{\boldsymbol{n}}, \chi, t) \,,
\end{equation}
where $\hat{\boldsymbol{n}}$ is a unit direction vector on the sphere, $T_{0}$ is today ($t_{0}$) CMB temperature, $t_{ls}$ is the time at the last scattering surface, $\chi$ the comoving distance and $\dot\Phi$ is the time derivative of the gravitational potential.
This effect is mainly due to the presence of DE that induces an accelerated background expansion of the Universe and counteracts the gravitational potential $\Phi$ inducing a not vanishing $\dot{\Phi}$~\cite{Giannantonio_2008, Giannantonio_2012, Watson_2014, Naidoo_2021}. The nonlinear growth of density perturbations produces additional temperature perturbations, leading to the RS effect. The RS contribution to temperature fluctuations is much more slow over time than the late ISW one, which means that at high redshifts RS predominates over the late ISW~\cite{Cai_2010}. Consequently, we are able to observe $\dot{\Phi}\neq$0 even when the DE effect becomes negligible.
Moreover, the presence of massive neutrinos induces a not vanishing derivative of the gravitational potential even during matter domination era.
This is because, after becoming non-relativistic, neutrinos free-stream with large thermal velocities that suppress the growth of neutrino density perturbations on scales smaller than the so-called ``free-streaming length''~\cite{Lesgourgues_2008,Lesgourgues_2012}:
\begin{equation}
    \lambda_{\rm FS}(z, m_{\nu}) \simeq 8.1 \frac{H_{0}(1+z)}{H(z)} \frac{1\text{ eV}}{ m_{\nu}}\, h^{-1}\text{Mpc} \,,
\end{equation}
where $m_{\nu}$ is the mass of the single neutrino species (i.e. $\nu_{e}, \nu_{\mu} \text{ or } \nu_{\tau}$), $H(z)$ is the Hubble parameter as a function of the redshift $z$ and $H_{0} \equiv H(z=0)$ is the Hubble constant.  Moreover, because of the gravitational back reaction effects, the evolution of cold dark matter (CDM) and baryon densities is affected as well by the presence of neutrinos, and the total matter power spectrum is suppressed at scales $\lambda \ll \lambda_{\rm FS}$~\cite{Rossi_2014}. Consequently, on small cosmological scales the free-streaming of neutrinos induces a slow decay of the gravitational potential, acting during both the matter and DE dominated eras, and this effect depends on the total neutrino mass $M_{\nu} = \sum m_{\nu}$. The more neutrinos are massive, the more the matter power spectrum will be suppressed wit respect to the massless neutrino case, and thefore nonlinearities will appear on smaller scales. 

One of the effects of massive neutrinos, which has been poorly investigated so far, is the shift they induce in the sign inversion that characterises the cross-power spectra between the ISWRS signal and the CMB-lensing potential ($P_{\dot\Phi\Phi}$), and between the ISWRS signal and the galaxy distribution ($P_{\dot\Phi\delta}$). These sign inversions appear because of the anti-correlation between the RS and both the CMB-lensing potential and the galaxy distribution, that depend on the gravitational potential $\Phi$. While on large scales (i.e. the linear regime) the gravitational potential decays because of the Universe expansion ($\Phi<0$, $\dot\Phi<0$), on small scales (i.e. the nonlinear regime) $\Phi$ grows because of nonlinear structures formation ($\Phi<0$, $\dot\Phi>0$). Consequently, the time derivative of the gravitational potential associated to the ISWRS signal is negative in the linear regime and positive in the nonlinear one. The net result is that $\langle \Phi\dot{\Phi}\rangle > 0$ in the late ISW regime, producing a positive correlation, and $\langle \Phi\dot{\Phi}\rangle < 0$ in the RS regime, producing an anti-correlation~\cite{Nishizawa_2014, Nishizawa_2008}. 

Following the example of~\cite{Seljak_1996, Cai_2009, Smith_2009}, we report below the dimensionless and scaled (i.e. divided by ${3}/{2} [H_0/(ck)]^2 \Omega_{m}$) forms of $P_{\dot{\Phi}\Phi}$ and $P_{\dot{\Phi}\delta}$ in  Limber approximation~\cite{Smith_2009, Lesgourgues_2008}: 
\begin{equation}
    \Delta^{2}_{\dot\Phi\Phi} = \frac{4\pi}{(2\pi)^{3}} \frac{k^{3}P_{\dot\Phi\Phi}(k,z)}{[F(k)H(z)/ca(z)]}\, \text{, with } P_{\dot\Phi\Phi}(k,z) = - \frac{1}{2} \frac{H(z)}{a(z)} [F(k)]^{2} \Big(\partial_{z}\frac{P_{\delta\delta}(k, z)}{a^{2}(z)} \Big) \,,
 \label{eq:delta_ptp}
\end{equation}
\begin{equation}
 \label{eq:delta_ptg}
     \Delta^{2}_{\dot\Phi\delta} = \frac{4\pi}{(2\pi)^{3}} \frac{k^{3}P_{\dot\Phi\delta}(k,z)}{[F(k)H(z)/ca(z)]} \, \text{, with } P_{\dot\Phi\delta}(k,z) = \frac{1}{2} H(z)F(k) \Big(\partial_{z}\frac{P_{\delta\delta}(k, z)}{a^{2}(z)} \Big) \,,
 \end{equation}
where $a(z)$ is the scale factor, $P_{\delta\delta}$ is the matter power spectrum, $k$ is the wavenumber, and
\begin{equation*}
    F(k) = \frac{3 \Omega_{m} H_{0}^{2}}{2 c^{2} k^{2}}\,,
\end{equation*}
with $\Omega_{m}$ the matter density parameter and $c$ the speed of light.

We compute each term of Equations~(\ref{eq:delta_ptp}) and (\ref{eq:delta_ptg}) with \texttt{CAMB} and use both the nonlinear modellings chosen for the matter power spectrum (i.e. Takahashi2012+Bird2014 and Mead2020). The absolute values of the results are shown in Figure~\ref{spectra_comp1} and Figure~\ref{spectra_comp2}, for both the Takahashi2012+Bird2014 and Mead2020 nonlinear modelling, at different redshift values, for the four total neutrino masses considered in this work.
% Figure environment removed 
 % Figure environment removed
The panels of both Figures~\ref{spectra_comp1}-\ref{spectra_comp2} show how the presence of nonlinearities varies with time. As already mentioned, the more we travel back in time, the more the RS effect will dominate over the late ISW. Considering that at large redshifts the RS traces not only nonlinearities at small scales, but even the production of filamentary structures (visible at scales of few hundreds Mpc)~\cite{Cai_2010}, we see the shifts of the sign inversion towards smaller $k$ (i.e. larger cosmological scales) from $z=0.5$ to $z=1.89$. In the bottom panels, the zoom in of the sign inversion region highlights the effect of massive neutrinos that we expected to find: the more neutrinos are massive, the more nonlinearities (and therefore the sign inversion) appear on smaller cosmological scales. The only exception, for both $\Delta^{2}_{\dot\Phi \Phi}$ and $\Delta^{2}_{\dot\Phi\delta}$, is the trend obtained for $M_{\nu} = 0.17$ eV using the Takahashi2012+Bird2014 model at $z = 0.5$, where the power spectra show the sign inversion at scales slightly larger than the massless neutrino case. This effect is probably due to inaccuracies in the nonlinear modelling (see Appendix~\ref{appendix:a} for further details).

These results also highlight the different behaviour of the two models. At small redshifts the Takahashi2012+Bird2014 model predicts an higher level of nonlinearities than the Mead2020 model. Indeed, in the panels of Figures~\ref{spectra_comp1}-\ref{spectra_comp2} corresponding to $z=0.5$, the wavenumbers $k$ where Takahashi2012+Bird2014 sign inversions occur are smaller than Mead2020 ones. However, in both Figures it is easy to notice that at $z=0.9$ and $z=1.4$ there is a switch of trend between Takahashi2012+Bird2014 and Mead2020, and at $z=1.89$ the Mead2020 model predicts an higher level of nonlinearities with respect to Takahashi2012+Bird2014. 

The presence of massive neutrinos is expected to affect similarly also the angular power-spectra of both these cross-correlations, that actually depend on the redshift derivative of the matter power spectrum $P_{\delta\delta}$, and can be computed via~\cite{Mangilli_2009}:
\begin{equation}
    C_{\ell}^{\dot\Phi\Phi} = \frac{1}{2} \Big(\frac{3 \Omega_{m} H_{0}^{2}}{2c^{2}k^{2}}\Big)^{2} \int_{z_{\rm min}}^{z_{\rm max}} {\rm d}z\, \frac{\chi(z_{ls}) - \chi(z)}{\chi(z_{ls})\chi^{3}(z)} \Big[ \partial_{z} \frac{P_{\delta\delta}(k,z)}{a^{2}(z)}\Big] \,,
     \label{ctp}
\end{equation}
where $z_{ls} = 1100$, and~\cite{Lesgourgues_2008, Smith_2009, Ferraro_2022}:
\begin{equation}
    C_{\ell}^{\dot\Phi g} = \frac{3\Omega_{m}H_{0}^{2}}{2c^{3}(\ell + 1/2)^{2}}  \int_{z_{\rm min}}^{z_{\rm max}} {\rm d}z\, n(z)b(z)H(z)a(z)\Big[\partial_{z}\frac{P_{\delta\delta}(k,z)}{a^{2}(z)}\Big] \,, 
    \label{ctg}
\end{equation} 
where $n(z)$ and $b(z)$ are the galaxy selection function and the galaxy bias, respectively. The integral limits used in this work are: $z_{\rm min} = 0.02$ and $z_{\rm max} = 1.89$. 
This is because we use Equations~(\ref{ctp}) and (\ref{ctg})  to compute the theoretical predictions that are compared against the cross-spectra measured from the DEMNUni maps, that cover the redshift range $[0.02, 1.89]$ (see Section~\ref{sec:simulations}). Apart from $n(z)$ and $b(z)$, all the ingredients of Equations~(\ref{ctp}) and (\ref{ctg}) have been obtained via \texttt{CAMB}.

\section{Nonlinear modelling with N-body simulations}
\label{sec:simulations}
The choice to investigate the nonlinear regime requires the use of large N-body simulations for validation tests. For this work, we use the ``Dark Energy and Massive Neutrino Universe'' (DEMNUni)~\cite{Carbone_2016} N-body cosmological simulations to model at small scales the ISWRS signal and their cross-correlations with the CMB-lensing and the galaxy distributions. 

\subsection{The Dark Energy and Massive Neutrino Universe simulations}
The DEMNUni simulations have been produced with the aim of investigating large-scale structures in the presence of massive neutrinos and dynamical dark energy (DDE), and they were conceived for the nonlinear analysis and modelling of different probes, including dark matter, halo, and galaxy clustering~\cite{Castorina_2015,Moresco_2017,Zennaro_2018,Ruggeri_2018,Bel_2019,Parimbelli_2021,Parimbelli_2022, Baratta_2022, Guidi_2022, Gouyou_Beauchamps_2023, Carella_in_prep}, weak lensing, CMB-lensing, SZ and late ISW effects~\cite{Roncarelli_2015,Carbone_2016,Fabbian_2018, Beatriz_2023}, cosmic void statistics~\cite{Kreisch_2019,Schuster_2019,Verza_2019,Verza_2022a,Verza_2022b}, and cross-correlations among these probes~\cite{Vielzeuf_2023}.
To this end, they combine a good mass resolution with a large volume to include perturbations both at large and small scales. In fact, these simulations follow the evolution of 2048$^3$ cold dark matter (CDM) and, when present, 2048$^3$ neutrino particles in a box of side $L = 2 \, h^{-1} {\rm Gpc}$. The fundamental frequency of the comoving particle snapshot is, therefore, $k_{\rm F} \approx 3\times 10^{-3} \ h/$Mpc, while the chosen softening length is 20 kpc/$h$. The simulations are initialised at $z_{\rm ini}=99$ with Zel'dovich initial conditions. The initial power spectrum is rescaled to the initial redshift via the rescaling method developed in Ref.~\cite{Zennaro_2017}. Initial conditions are then generated with a modified version of the \texttt{N-GenIC} software, assuming Rayleigh random amplitudes and uniform random phases. 
The DEMNUni simulations have been performed using the tree particle mesh-smoothed particle hydrodynamics (TreePM-SPH) code \texttt{GADGET-3}, an improved version of the code described in~\cite{Springel_2005}, specifically
modified in~\cite{Viel_2010} to account for the presence of massive neutrinos. This version of \texttt{GADGET-3} follows the evolution of CDM and neutrino particles, treating them as two distinct collisionless fluids. Given the large amount of memory required by the simulations, baryon physics is not included. This choice, however, does not affect the results of this work~\cite{Bird_2012}. The DEMNUni suite is composed by a total of sixteen simulations accounting for different combinations of the total neutrino mass $M_{\nu}$ and the DE EoS. In this project, we have worked with four of them, characterised by a baseline Planck 2013 cosmology~\cite{PlanckXVI_2014},  namely a flat $\Lambda$CDM model with $\tau = 0.0925$, $n_{s} =0.96$ and $A_{s}=2.13 \times 10^{-9}$, generalised to a $\nu\Lambda$CDM by varying only the sum of the neutrino masses over the values $M_{\nu} = 0,0.17,0.30,0.53\text{ eV}$ (whence the corresponding values of $\Omega_{\nu}$ and $\Omega_{cdm}$, keeping fixed $\Omega_{m}$ = 0.32 and $\Omega_{b}$ = 0.05), considering a degenerate neutrino mass hierarchy. As aforementioned, in this work we use maps that cover a range of redshift that goes from $z_{\rm min} = 0.02$ to $z_{\rm max} = 1.89$.
For each simulation, 63 outputs have been produced, logarithmically equispaced in the scale factor $a = 1/(1 + z)$, in the redshift interval $z = 0-99$, 49 of which lay between $z = 0$ and $z = 10$. For each of the 63 output times, it has been produced on-the-fly one particle snapshot, composed of both CDM and neutrino particles, one 3D grid of the gravitational potential, $\Phi$, and one 3D grid of its time derivative, $\dot\Phi$, with a mesh of dimension $4096^{3}$ that covers a comoving volume of  $(2 \, h^{-1} {\rm Gpc} )^{3}$.
% Figure environment removed
\clearpage
The map-making procedure has been developed taking into account the standard pixelisation approach introduced by the Hierarchical Equal-Area isoLatitude Pixelization~\cite{Gorski_2005}
(\texttt{HEALPix}\footnote{\url{http://healpix.sourceforge.net}}).
%\clearpage

\subsection{Map-making procedure: ISWRS and CMB-lensing}
\label{chap:melita_map}
The procedure followed to build ISWRS and CMB-lensing potential maps is an adaptation of the one developed in~\cite{Carbone_2008}. To make ISWRS maps (left column of Figure~\ref{fig:maps}), 
CMB photons are ray-traced along the undeflected line of sight through the 3D field $\dot\Phi$. An analogous kind of ray-tracing has been applied to the 3D $\Phi$-grids as well, in order to produce CMB-lensing potential maps (right column of Figure~\ref{fig:maps}) from the same realisation of the Universe as for the ISWRS maps, and compute the cross-correlation between the two maps. Comparing the left and right columns of Figure~\ref{fig:maps}, it is possible to appreciate at a glance how the two signals are highly correlated. Comparing the scales of each rows, the slight difference due to the different $M_{\nu}$ considered emerges.
To this aim, the $\dot\Phi$- and $\Phi$-grids have been stacked around the observer, located at $z = 0$, and it has been applied the replication and randomisation procedure designed by~\cite{Carbone_2008}. This particular 3D tessellation scheme is required to avoid both the repetition of the same structures along the line of sight
and the generation of artifacts like ripples in the simulated deflection-angle field, which can be avoided only if the peculiar gravitational potential is continuous transversely to each line of sight. 


\subsubsection{Simulated ISWRS and CMB-lensing auto-spectra }
The power spectra extracted from the maps of  Figure~\ref{fig:maps} are reported in Figure~\ref{maps1_isw}.  
For comparison, the orange squares and the orange dashed lines represent theoretical predictions in the linear regime obtained with \texttt{CAMB} for the massless neutrino case, in order to highlight the differences from the nonlinear modelling.
The subpanels represent the percentage relative differences of the $\nu\Lambda$CDM models with respect to the massless neutrino one. There is a good agreement between the simulated ISWRS signals and the ones predicted with \texttt{CAMB} for $\ell <100$ (which roughly corresponds to the transition between the linear and the nonlinear regimes~\cite{Schafer_2006,Schafer_2008}).
Past the linear limit, \texttt{CAMB} predictions fail, given that the RS contribution is not  implemented in the code.
The CMB-lensing potential linear prediction for the massless neutrino case, similarly, perfectly agrees with DEMNUni mocks up to the linear limit and then shows a lack of power with respect to the nonlinear ones. 


Focusing on spectra extracted from maps, it is easy to appreciate at small scales the differences produced in the ISWRS induced temperature anisotropies by free-streaming neutrinos with different total masses. Similar effects are noticeable also in the lensing potential power spectra, considering the percentage relative differences with respect to the massless neutrino case (see subpanels of Figure~\ref{maps1_isw}). In particular, as expected, the effect of massive neutrinos produces a suppression of power in the CMB-lensing potential which increases with the neutrino mass and decreases
with the angular scale (at low $\ell$). This effect is reasonably due to the nonlinear excess of suppression of the total matter power spectra with respect to linear expectations in the presence of massive neutrinos, that counteract the nonlinear evolution~\cite{Carbone_2016}. 
% Figure environment removed

\subsection{Map-making procedure: projected galaxy mocks}
\label{chap:galaxy_map}
The procedure followed to construct galaxy maps is an adaptation of the one developed in~\cite{Calabrese_2015,Fabbian_2018} for convergence maps. 
The convergence maps are extracted with a postprocessing procedure acting on the N-body particle snapshots to create matter particle full-sky lightcones~\cite{Fosalba_2008,Hilbert_2020}. The standard way to build lightcones is to pile up high-resolution boxes within 
concentric cells to fill the lightcone to the maximum desired source redshift.  With the observer placed at $z=0$, the volume of the lightcone is sliced into full-sky spherical shells of the desired thickness in redshift. All the particles distributed within each of these 3D matter shells are
% Figure environment removed
then projected onto 2D spherical maps, assigning a specific sky pixel to each particle, via \texttt{HEALPix} pixelisation procedure. Then integration along the undeflected line-of-sight is applied to obtain the convergence maps accounting for the lensing efficiency in the desired redshift range. A very similar method has been applied to the DEMNUni galaxies to construct projected galaxies mocks~\cite{Parimbelli_2022} (see Figure~\ref{fig:galaxy_maps}). 
Specifically, in order to reproduce the galaxy distribution in the DEMNUni simulations, the SubHalo Abundance Matching (SHAM) technique~\cite{Carella_in_prep} has been used. The SHAM method assumes a one-to-one relation between a physical property of a dark matter halo/subhalo and an observational property of the galaxy that it hosts. In particular, it assumes that the most massive galaxies form in the dark matter haloes characterised by the deepest potential wells. Galaxies are linked to the corresponding dark matter structures using stellar mass or luminosity as galaxy property and a measure of halo mass or the circular velocity, as halo property (i.e. a proxy of the depth of the local potential well). Here we link the stellar mass to the halo mass using the Moster et al. (2010)~\cite{Moster_2010} parameterised stellar-to-halo mass relation, fitted against the DEMNUni simulations. As the SHAM galaxy catalogues are fitted against observations, they guarantee an high-degree of agreement with real data from galaxy surveys~\cite{Shankar_2006, Girelli_2020}.

Applying to the SHAM galaxy catalogues the same stacking technique as for particle comoving snapshots, we have generated full-sky mock galaxy catalogues from the DEMNUni simulations in different cosmological models. 

\subsubsection{Galaxy bias and selection function from DEMNUni simulations}
The galaxy bias and selection function necessary to compute the analytical predictions of the cross-spectra between ISWRS and galaxies (see Equation~(\ref{ctg})) have been directly extracted from the DEMNUni simulations. This guarantees a more accurate comparison between mocks and analytical predictions. 
\paragraph{Galaxy selection function}
The galaxy selection function extracted from DEMNUni $\Lambda$CDM map can be modelled via the following equation:
\begin{equation}
\label{eq:demnuni_n}
    n_{\rm DEMNUni}(z) = A\left(\frac{z}{z_{0}}\right)^{B} \exp\Big[ -\Big(\frac{z}{z_{0}} \Big)^{C}\Big]\,,
\end{equation}
where A = 1.89, B = 1.95, C = 1.41 and $z_{0} = 0.88$. 

This is the $n(z)$ used to compute all the theoretical predictions to be compared with DEMNUni data, also for massive neutrino cases. This is because galaxy counts vary mostly because of the variation of the simulated cosmological volume, which depends on $H(z)$ via $\Omega_{m}$ - that has been fixed to 0.32 for all the simulations (and consequently for all the theoretical predictions) - and via the DE EoS, whose presence has not been taken into account in this work. As a result, the normalised galaxy counts do not undergo any significant change due to the variation of $M_{\nu}$, and consequently it is possible to exploit the same $n(z)$ to compute all the theoretical predictions. We have verified this measuring the galaxy counts from the DEMNUni catalogues with different values of $M_\nu$, obtained via the SHAM technique applied to COSMOS2020 and SDSS observations ~\cite{Carella_in_prep}.
% Figure environment removed 
\\In Figure~\ref{n_b}, together with DEMNUni $n(z)$, we show the galaxy selection function expected for the photometric Euclid survey~\cite{Laureijs_2011, EuclidVII_2020}:
\begin{equation}
\label{eq:Euclid_n}
    n_{\rm Euclid}(z) = \frac{3 \cdot (z/z_{0})^{\alpha}}{2 \cdot z_{0}} \text{exp} \Big[-\Big(\frac{z}{z_{0}}\Big) ^{\beta} \Big] \,,
\end{equation}
where $\alpha$ = 2, $\beta$ = 1.5, $z_{0}$ = $z_{m}/\sqrt{2}$, $z_{m}$ = 0.9, and its extensions to a larger redshift range setting $z_{m}$ = 2,3~\cite{Lesgourgues_2008}. These functions have been used here to produce forecasts for future galaxy surveys, whose aim is to reach $z \ge 2$.
\paragraph{Scale-independent bias}
The galaxy scale-independent bias has been obtained for each cosmology from the square root of the ratio between the galaxy power spectrum, $P_{gg}(k,z)$,  and the matter power spectrum, $P_{mm}(k,z)$, measured from DEMNUni comoving snaphots:
\begin{equation}
\label{eq:bgg}
    b_{gg} (z) = \Bigg\langle\sqrt{\frac{P_{gg}(k,z)}{P_{mm}(k,z)}}  \Bigg\rangle_{k} \,,
\end{equation}
where the mean has been computed for large linear scales in the range $0.05<k<0.12$ in unit of  $\text{Mpc}^-1$. 
Differently from the galaxy selection function, the neutrino mass affects the bias. The $b_{gg}$ functions for $M_{\nu} = 0, 0.17, 0.30, 0.53$ eV shown in the right  panel of Figure~\ref{n_b} have been fitted with:
\begin{equation}
\label{eq:euclid_bias}
    b(z) = A + \frac{B}{1 + \exp[-(z-D){C}]} \,,
\end{equation}
where A = 1.0, B = 2.5, C = 2.8, D = 1.6. This is the bias currently assumed for the Euclid Flagship simulation, reported in~\cite{Tutusaus_2020}. From now on, we will refer to Equation~(\ref{eq:euclid_bias}) as the Tutusaus bias.

For the massless neutrino case alone, also the bias obtained from the square root of the ratio between the cross power spectrum between galaxy and matter has been computed:
\begin{equation}
    b_{gm} (z) = \Bigg\langle\sqrt{\frac{P_{gm}(k,z)}{P_{mm}(k,z)}}  \Bigg\rangle_{k} \,,
\end{equation}
(where the mean is again over large linear scales in the range $0.05<k<0.12$ in unit of  $\text{Mpc}^-1$) and it is shown in Figure~\ref{n_b} after the fit with Tutusaus bias. Since it is affected by a lower level of shot noise, $b_{gm}(z)$ should allow us to describe better the connection between matter and galaxies, and consequently to improve the cross-correlation reconstruction. However, the results obtained do not differ significantly from those obtained using $b_{gg}(z)$, as we will show in Section~\ref{test_bias_mnu0}. Finally, we show also the $\sqrt{1+z}$ bias, used in~\cite{EuclidVII_2020}.

\paragraph{Scale-dependent bias}
A fit of scale-dependent bias has been computed from DEMNUni massless neutrino mocks as~\cite{Modi_2020}:
\begin{equation}
    b(k,z) = b_{0}(z)+b_{1}(z)k+b_{2}(z)k^2 \,,
    \label{eq:bnl}
\end{equation}
where $b_{0}$ is the scale-independent bias, $b_{2}$ is a correction due to peaks theory and $b_{1}$ is physically meaningless, but necessary to improve the fit. These parameters have been obtained via the fit of $P_{gg}(k,z)$ up to $k=1\text{ Mpc}^{-1}$.  
In Figure~\ref{bnl}, we show the evolution of the $b_{i}$ parameters on the left, while on the right we show the variation of the $b(k,z)$ at some chosen wavenumbers in the range of scales used to measure the scale-averaged the $b_{gg}(z)$ ($0.05 < k < 0.12$ in unit of  $\text{Mpc}^-1$).
% Figure environment removed

\subsubsection{Simulated projected galaxy auto-spectra}
In Figure~\ref{c_gg}, we report the spectra extracted from the galaxy maps of Figure~\ref{fig:galaxy_maps} after subtracting the shot-noise value.

In addition to the DEMNUni maps spectra, we report the theoretical predictions of the $C_{\ell}^{gg}$ computed in both the linear and nonlinear regimes, for the massless neutrino case, via~\cite{Calabrese_in_prep}:
 \begin{equation}
     C_{\ell}^{gg} = \int_{z_{\rm min}}^{z_{\rm max}} {\rm d}z\, \frac{H(z)}{c \chi^{2}(z)} [W^{g}(z)]^{2}P_{\delta\delta}(k,z) \,,
     \label{c_l_gg}
 \end{equation}
 where:
 \begin{equation}
   \label{W_eq}
     W^{g}(z) = n(z) \cdot b(z)
     \,,
 \end{equation}
and for $n(z)$ and $b(z)$ we use Equation~(\ref{eq:demnuni_n}) and Equation~(\ref{eq:bgg}), respectively.

Both linear and nonlinear predictions do not perfectly agree with simulated spectra on large scales. This is due to the limited dimension of the simulation box, which does not allow to represent those scales. Conversely, on small scales there is a good agreement with the nonlinear prediction and the expected disagreement with the linear one. 

Despite the presence of shot-noise in DEMNUni maps, its effect is negligible when cross-correlating these maps with ISWRS maps (since the noises of the two fields do not correlate). Furthermore, this analysis does not aim to reveal a perfect match between the simulations and the analytical predictions also because inaccuracies in the adopted nonlinear modelling.
 % Figure environment removed

The subpanel in Figure~\ref{c_gg} represents the percentage relative differences of the $\nu\Lambda$CDM models with respect to the massless neutrino one. Here the larger $M_{\nu}$ is, the smaller the difference with respect to the massless neutrino case is. This is because the matter perturbation suppression due to more massive neutrinos is compensated by a stronger clustering~\cite{Marulli_2011}.

\section{The ISWRS cross-correlation with the CMB-lensing potential}
\label{sec:XC-TP} 
The presence of the sign inversion at large multipoles and its shift due to the variation of $M_{\nu}$ in the cross-power spectrum between the ISWRS effect and the CMB-lensing was already verified in~\cite{Carbone_2016}, where they use DEMNUni maps obtained via ray-tracing from $z_{\rm min}=0.02$ to $z_{\rm max}=20.88$. Here, we verify the presence of the same effects in the cross-spectra of DEMNUni maps obtained up to $z_{\rm max}=1.89$ and use these results to validate the analytical predictions computed using Equation~(\ref{ctp}).\\
The cross-correlations of the DEMNUni ISWRS and CMB-lensing potential maps for the four neutrino masses are shown in Figure~\ref{cross_TP}, compared with the theoretical nonlinear prediction computed for the Takahashi2012+Bird2014 (left panels) and the Mead2020 (right panels) models. For completeness, also the theoretical cross-correlation predictions for the massless neutrino case in the linear regime have been reported.

The agreement between the simulation data and both the theoretical predictions obtained with \texttt{CAMB} up to $\ell \sim 100$ is quite impressive.
Once exceeded $\ell \sim 100$, the anti-correlation between the RS effect and the CMB-lensing potential takes over, resulting in the characteristic sign inversion that appears in both simulations (as already shown in~\cite{Carbone_2016}) and nonlinear predictions.
Moreover, both of them validate the expected result, because the more neutrinos are massive, the more they counteract nonlinear evolution (i.e. structure formation) and nonlinear effects appear at smaller cosmological scales. In other words, the more neutrinos are massive, the more the sign inversion shifts towards larger multipoles. 

Focusing on the sign inversion position (subpanels), on the one hand the analytical predictions computed with the Mead2020 model show the sign inversions at smaller multipoles with respect to mocks and with respect to the Takahashi2012+Bird2014 model. On the other hand, the Takahashi2012+Bird2014 model almost perfectly reproduces the results of DEMNUni mocks for $M_{\nu} = 0.17, 0.30$ eV, while overestimating the $\ell$ correspondent to the sign inversion in the massless neutrino case and underestimating it in the more massive case. The differences between the two nonlinear models are strictly due to how nonlinearities are implemented in the two codes~\cite{Takahashi_2012, Bird_2012, Mead_2021, Parimbelli_2022}. The achieved results disprove the counter-intuitive effect reported in~\cite{Xu_2016}, where for larger $M_{\nu}$ values the sign inversion position moves towards smaller multipoles.
% Figure environment removed

\section{The ISWRS cross-correlation with galaxies}
\label{sec:XC-TG}
The cross-spectra between the ISWRS signal and the galaxy distribution is the most used tool to investigate the ISWRS, because of its largest signal-to-noise ratio with respect to the spectra analysed in the previous Section. It is then the main focus of this work, which consists of verifying the presence of the sign inversion due to the anti-correlation between the RS effect and the galaxy distribution in the cross-correlation angular power spectra, showing that the position of this sign inversion depends on $M_{\nu}$, as in the case of the cross-correlation between the ISWRS signal and the CMB-lensing potential~\cite{Carbone_2016}. 
The validation of these results via DEMNUni maps represents an improvement with respect to previous works: on the one hand,~\cite{Lesgourgues_2008} computed this cross-correlation signal in massless and massive neutrino models, focusing only on the effects that neutrinos have in the linear regime; on the other hand, in the nonlinear regime,~\cite{Smith_2009} focused only on the massless neutrino case, finding in the cross-spectrum the sign inversion due to the anti-correlation between the RS effect and the galaxy distribution. 

The cross-correlations of the DEMNUni ISWRS and the galaxy distribution maps for the four neutrino masses are shown in Figure~\ref{cross_TG}, compared with the theoretical nonlinear predictions, computed with Equation~(\ref{ctg}). For completeness, also the theoretical predictions for the massless neutrino case in the linear regime have been reported. All the theoretical predictions have been computed using $n(z)$ given by Equation~(\ref{eq:demnuni_n}) and the $b_{gg}(z)$ of Equation~(\ref{eq:bgg}) associated with the cosmology analysed.
% Figure environment removed

The results shown prove that also in the cross-correlation spectrum between the ISWRS and the galaxy distribution there is a sign inversion due to the anti-correlation between the two signals on small scales and that the position of the sign inversion is strongly affected by the presence of massive neutrinos. Also in this case, because of the suppression that massive neutrinos induce in the matter power spectrum, the larger $M_{\nu}$ is, the more nonlinearities appear on smaller scales and consequently the sign inversion position is shifted towards larger multipoles.

Moreover, the comparison of the simulated spectra with the theoretical predictions for both linear and nonlinear regimes is again quite impressive at large cosmological scales, while once passed $\ell \simeq 100$ nonlinear spectra dramatically differ from the linear one, because of the presence of the sign inversion. This result validates again the analytical method against mocks and makes it an extremely powerful tool to predict cross-spectra of future surveys in both linear and nonlinear regimes.

Focusing on the sign inversion position (subpanels), the Takahashi2012+Bird2014 model shows inversions at larger multipoles with respect to DEMNUni spectra, and Mead2020 at smaller ones.

\subsection{Nonlinear ISWRS-galaxy cross-spectra: effects of different bias modellings}
As already mentioned, differently from the galaxy selection function, the bias strongly depends on the neutrino mass. 
We perform several tests to evaluate how the usage of different 
bias functions affects the comparison between predictions and DEMNUni simulations. First, we produce theoretical predictions for the massive neutrino cases using both the corresponding bias and the massless neutrino bias functions and we verify that the more neutrinos are massive, the more the massless neutrino case bias fails in predicting the spectra (further details in Section~\ref{subsec:b1}). Second, we check, for the massless neutrino case alone, that using a scale-dependent bias function improves the agreement between theoretical predictions and simulations, as already verified in~\cite{Smith_2009}
(further details in Section~\ref{subsec:b2}).
The results of these tests are shown in Figure~\ref{fig:table_bias} and  Figure~\ref{bias_z_comparison_lcdm}, respectively. 

\subsubsection{\texorpdfstring\boldmath{}{}Massive neutrino impact on the scale-independent bias}
\label{subsec:b1}

First, we test how the analytical predictions for massive neutrino cosmologies vary with respect to mock data when using the bias fitted against the simulations for the $M_{\nu} > 0\text{ eV}$ case and the bias of the massless neutrino case.

In Figure~\ref{fig:table_bias}, the simulated cross-correlations are compared with linear predictions and with nonlinear predictions computed with both the massless neutrino galaxy bias and the galaxy bias associated with the specific neutrino model. The results for $\ell \le 10$ can be neglected because of the inaccuracies of mocks on those scales.
% Figure environment removed
\clearpage
For the $M_{\nu}=0.17\text{ eV}$ case, reported in top panels of Figure~\ref{fig:table_bias}, the slight difference with respect to the other cases (and to the predictions), noticeable in the simulation data for very small and very large multipoles, induces a difference $>35\%$ for $\ell<30$ and $\ell>1000$. Before the sign inversion range, the differences do not overcome $25\%$ for the $M_{\nu}=0.17$ eV bias and $15\%$ for the massless neutrino case bias, for the Takahashi2012+Bird2014 modelling. In the case of the Mead2020 modelling, the difference is slightly smaller with the massive cosmology bias ($< 20\%$). The differences in the sign inversion positions consist of $\Delta \ell_{\rm DEMNUni-(Takahashi2012+Bird2014)} \sim - 18$ and $\Delta \ell_{\rm DEMNUni-Mead} \sim + 18$. Both
nonlinear modellings show a difference between the predictions obtained with the two bias recipes at a level of 
 $\sim 5\%$ and the sign inversions occur at the same multipole in the Mead2020 model, while there is a $\Delta \ell = 1$ for the Takahashi2012+Bird2014 one.
The fact that the massless neutrino cosmology bias in the theoretical predictions leads to smaller differences with respect to the DEMNUni spectra can be explained considering the inaccuracies of the Takahashi2012+Bird2014 and Mead2020 models in reproducing the $M_{\nu} = 0.17$ eV case, as already shown in Section~\ref{sec:theory} (see Appendix~\ref{appendix:a} for further details).


In the $M_{\nu}=0.30\text{ eV}$ case, shown in centre panels of  Figure~\ref{fig:table_bias}, both nonlinear predictions, obtained with the Mead2020 and Takahashi2012+Bird2014 models, differ from the simulated data by no more than $15\%$ in the $M_{\nu}=30$ eV bias case and no more than $25\%$ in the massless neutrino case. The differences in the sign inversion positions consist of $\Delta \ell_{\rm DEMNUni-(Takahashi2012+Bird2014)} \sim -10$ and $\Delta \ell_{\rm DEMNUni-Mead} \sim +20$. The difference between the two is larger than the previous case ($\lsim 10\%$) and the sign inversions occur at the same multipole in the Mead2020 model, while there is a $\Delta \ell = 1$ for the Takahashi2012+Bird2014 one.

Conversely to the previous cases, in the $M_{\nu}=0.53\text{ eV}$ one, shown in bottom panels of Figure~\ref{fig:table_bias},  the difference between the predictions is definitely more significant ($\sim 20\%$). The difference with respect to DEMNUni data is $\sim 10\%$ in the $M_{\nu}=0.53$ eV bias case and no more than $25\%$ in the massless neutrino case. The differences in the sign inversion positions consist of $\Delta \ell_{\rm DEMNUni-(Takahashi2012+Bird2014)} \sim -10$ and $\Delta \ell_{\rm DEMNUni-Mead} \sim +10$. Conversely to previous cases, here the sign inversions occur in the same multipole when using both \texttt{Halofit} models.

\subsubsection{\texorpdfstring\boldmath{}{}
\label{test_bias_mnu0}
Scale-independent and scale-dependent bias effects in the neutrino massless neutrino case}
\label{subsec:b2}
For the massless neutrino case, both the $b_{mg}(z)$ and the $b(k,z)$ bias functions have been calibrated against the DEMNUni simulations. In Figure~\ref{bias_z_comparison_lcdm} we compare the effects of $b_{mg}(z)$, $b(k,z)$ and $b_{gg}(z)$ in the modelling of the theoretical nonlinear predictions.
In agreement with the work of~\cite{Smith_2009}, we find that the amplitude of the predictions obtained with $b(k,z)$ better reproduces the simulated signals, with a percentage relative difference of $\sim 10\%$, lower than that obtained with $b_{gg}(z)$ and $b_{mg}(z)$. 
This is true up to $\ell = 200$, while for smaller scales the trend is inverted. With regard to the sign inversion position, in the theoretical spectra it occurs at the same multipole for all bias recipes with both Takahashi2012+Bird2014 and Mead2020 model.  The differences with respect to the sign inversion in the simulated signals is $\Delta \ell_{\rm DEMNUni-(Takahashi2012+Bird2014)} \sim -20$ and $\Delta \ell_{\rm DEMNUni-Mead} \sim +20$.  
% Figure environment removed 
% Figure environment removed
% Figure environment removed
\subsection{\texorpdfstring\boldmath{}{}Nonlinear ISWRS-galaxy cross-spectra: detecting the neutrino mass}
The results shown in the previous sections demonstrate the validity of the analytical method against DEMNUni simulations and that both mocks and predictions confirm what we expect from theory: more massive neutrinos induce a larger suppression in the matter power spectrum and shift the appearance of nonlinearities on smaller cosmological scales. Moreover, Figures~\ref{cross_TP} and~\ref{cross_TG} highlight the strict connection between the sign inversion position, $\ell_{inv}$, and $M_{\nu}$. In light of this, we decide to test if the detection of $\ell_{inv}$ in the cross-power spectrum between the ISWRS and the galaxy distribution computed with Equation~(\ref{ctg}) can constrain $M_{\nu}$. 

In Figure~\ref{fig:mnu} we report the neutrino masses as a function of the sign inversion positions predicted by the Takahashi2012+Bird2014 and Mead2020 models and the DEMNUni maps. 
The $1\sigma$ error bars have been estimated as follows: we generate 5000 Gaussian realisations of the DEMNUni maps and estimate the errors as the standard deviation of the 5000 cross-spectra obtained from them.

For relevant values of the neutrino mass ($M_{\nu} < 0.12$ eV at $95\%$ C.L.~\cite{PlanckVI_2020}), we find differences between the Takahashi2012+Bird2014 and Mead2020 models and the DEMNUni simulations at the $\sim 1.5-2\sigma$ level. This implies that robust constraints on the neutrino mass via the sign inversion detection will require improving the nonlinear modelling.  Furthermore, because of cosmic variace at the multipole range of interest (see Appendix~\ref{appendix:b}) the possibility of detecting $\ell_{inv}$ appears to be unlikely. Nevertheless, we expect to be able to disentangle different $M_{\nu}$ values via the estimation of the overall amplitude of the signal on very small cosmological scales. As shown in~\cite{Ferraro_2022}, the accuracy of future CMB surveys will improve the detection of the T-galaxy cross-spectrum on extremely large multipoles. In future works, we plan to investigate this possibility.

\subsection{Nonlinear ISWRS-galaxy cross-spectra:  the case of different galaxy surveys}
The last part of this work consists of computing predictions of the cross-spectra that can be observed on higher redshift ranges. 
First of all, we compute the ISWRS-galaxy distribution cross-power spectra that should be observable in a redshift range that goes up to $z=2$, which is Euclid photometric range~\cite{EuclidI_2022}, setting $n(z)$ from Equation~(\ref{eq:Euclid_n}) with $z_{m}$ = 0.9, shown in the upper panels of Figure~\ref{fig:table_forecast}.
In the middle and bottom panels we report the cross-spectra obtained using the same $n(z)$, but with $z_{m}$ = 2, 3 respectively (see Figure~\ref{n_b}). 
Differently from all the previous theoretical spectra showed, here all the predictions have been computed integrating up to $z_{\rm max} = 8$. It is evident at first glance that the higher is the redshift, the larger is the difference between the four $M_{\nu}$ values, with both the \texttt{Halofit} models considered so far. This is because the late Universe (i.e. small $z$) is characterised by the presence of DE \cite{Giannantonio_2006}, that affects the appearance of nonlinearities and is expected to have similar effects as neutrinos on this cross-spectrum. Hence, the further we go back in time, the less DE effects will be present, and the shift we measure will be due only to the mass of neutrinos. 

\section{Conclusions}
\label{sec:conclusion}
The aim of this work is to develop and validate, against the DEMNuni cosmological simulations, an analytical modelling able to reproduce the cross-power spectra of the ISWRS signal with the CMB-lensing potential and the galaxy distribution in the presence of massive neutrinos. 

We consider both the linear (ISW) and nonlinear (RS) effects, i.e. we account also for the nonlinear regime where a sign inversion appears in the cross-spectra due to the anti-correlation between the RS effect and the CMB-lensing/galaxy-distribution. We find that the presence of massive neutrinos affects the position of such a sign inversion, and specifically, the more massive neutrinos are, the more it shifts towards larger multipoles $\ell$. This is because the more massive neutrinos, the more they suppress the matter power spectrum, shifting the appearance of nonlinearities towards smaller cosmological scales. 

There is already evidence of this effect at the level of the matter power spectra, $P_{\dot{\Phi}\Phi}$ and $P_{\dot{\Phi}\delta}$, associated to these cross-correlations, as we show in Section~\ref{sec:theory}. We compute them assuming a Planck-like baseline cosmology to which we add neutrinos with total mass $M_{\nu} = 0.17, 0.30, 0.53 \text{ eV}$ and using two \texttt{Halofit} models implemented in \texttt{CAMB} (Takahashi2012+Bird2014 and Mead2020) that allow us to investigate also high redshifts. We verify that a sign inversion appears in the transition from the linear to the nonlinear regime and that the presence of massive neutrinos affects its position. Moreover, we find that the lower the redshift is, the larger the scale is where nonlinearities appear, because the RS effect dominates over the late ISW one, as we go back in time. 

Focussing on the nonlinear modelling, we find that Takahashi2012+Bird2014 predicts an higher level of nonlinearities  with respect to Mead2020 at small redshifts, but this trend is inverted at $z\ge  1.1$ (see Figures~\ref{spectra_comp1} and ~\ref{spectra_comp2}).


The main result of this work is checking and modelling the presence of the same effects in the corresponding angular cross-power spectra, obtained from $P_{\dot{\Phi}\Phi}$ and $P_{\dot{\Phi}\delta}$, using Equation~(\ref{ctp}) and Equation~(\ref{ctg}).

Concerning the cross-correlation between the ISWRS signal and the CMB-lensing potential, we find that both DEMNUni maps and the theoretical predictions, computed using the Takahashi2012+Bird2014 and Mead2020 models, show the sign inversion due to the anti-correlation between the RS and the CMB-lensing, and the shift of this sign inversion moves towards larger multipoles as $M_{\nu}$ increases (see Figure~\ref{cross_TP}).
%\clearpage

Furthermore, we observe that, while the modelling implementing Mead2020  tends to overestimate nonlinearities (and the sign inversions appear always on smaller multipoles with respect to the DEMNUni simulated signal), the Takahashi2012+Bird2014 modelling gives different results depending on $M_{\nu}$, and the corresponding sign inversion locations are indeed extremely close to those of the simulations for $M_{\nu} = 0.17,0.30$ eV.

The same results have been achieved for the cross-correlation between the ISWRS and the galaxy distribution with both DEMNUni maps and the theoretical predictions (see Figure~\ref{cross_TG}). This represents an improvement with respect to~\cite{Smith_2009}, where only the massless neutrino case was investigated, and extends the work of~\cite{Lesgourgues_2008} properly treating the impact of massive neutrinos on this cross-correlation in the nonlinear regime. We find that the Takahashi2012+Bird2014 modelling consistently underestimates nonlinearities compared to simulations, resulting in sign inversions occurring on smaller cosmological scales, i.e. larger multipoles. Conversely, the Mead2020 modelling overestimates nonlinearities. This could be due to the fact that such models are not correctly able to recover the time variation of the gravitational potentials which give rise to the ISWRS effect. 

Considering the dependence of this cross-correlation on galaxy bias, we have performed two further tests to quantify the effects of different bias modellings. First, we test how the analytical predictions fit the simulated signal when considering a bias (measured from the DEMNUni simulations) dependent or not on the neutrino mass. We observe that, for small values of $M_{\nu}$, the differences between the modelling with massless and massive neutrino bias are negligible (Figure~\ref{fig:table_bias}). Conversely, when considering large values of $M_{\nu}$, the difference cannot be neglected, and a bias model specific to that cosmology is required. However, in view of the fact that the most recent constraints on neutrino masses suggest $M_{\nu} < 0.12$ eV~\citep{PlanckVI_2020}, a bias associated to a massless neutrino case cosmology would be appropriate to produce accurate forecasts, at least in a $\Lambda$CDM scenario. Second, only in the massless neutrino case, we test the effect of scale-independent and scale-dependent bias as measured from the DEMNUni. We verify that, both for Takahashi2012+Bird2014 and Mead modellings, choosing a scale-dependent bias $b(k,z)$ results in spectra whose amplitudes better fit the simulated signal (Figure~\ref{bias_z_comparison_lcdm}), as suggested in~\cite{Smith_2009}.

Overall in this work, we have developed an analytical modelling of the nonlinear ISWRS-LSS cross-correlations that exploits the Boltzmann solver code \texttt{CAMB} and has been validated against the DEMNUni simulations~\cite{Carbone_2016}. It is a useful and computationally non-expensive tool to investigate these cross-correlations in different cosmological scenarios without the need to run large N-body simulations for each cosmology. 

We have verified that these cross-spectra change with redshift because of the different role played by dark energy at different epochs. Therefore, a straightforward extension of this work would be to include DDE models in the analysis, and study the dependency on the DE equation of state of the ISWRS-LSS cross-spectra. This could help to distinguish between different DE models and provide insights on the properties and the evolution of DE.

Finally, our results highlight the strict connection between the sign inversion location, $\ell_{inv}$, and the value of neutrino masses. However, improving current nonlinear modellings of the matter power spectra is needed to predict $\ell_{inv}$ accurately.  In a future work we will investigate the detectability of the neutrino mass via ISWRS-LSS cross-correlations. We expect future CMB experiments, such as The Simons Observatory~\cite{Lee_2019}, LiteBIRD\cite{LiteBird}, CMB-S4~\cite{Abazajian_2016}, and galaxy surveys such as Euclid~\cite{EuclidI_2022}, the Roman Space telescope~\cite{Eifler_2021}, the Vera Rubin Observatory~\cite{Ivezic_2019}, will help us exploiting such nonlinear signals with the aim of measuring, in combination with more traditional probes, the neutrino mass and the DE equation of state.


\acknowledgments 
VC and MM are partially supported by the INFN project ``InDark''.
MC is partially supported by a 2021/22 ``Research and Education'' grant from Fondazione CRT. The OAVdA is managed by the Fondazione Cl\'ement Fillietroz-ONLUS, which is supported by the Regional Government of the Aosta Valley, the Town Municipality of Nus and the ``Unit\'e des Communes vald\^otaines Mont-\'Emilius''.
The DEMNUni simulations were carried out in the framework of ``The Dark Energy and Massive-Neutrino Universe" project, using the Tier-0 IBM BG/Q Fermi machine and the Tier-0 Intel OmniPath Cluster Marconi-A1 of the Centro Interuniversitario del Nord-Est per il Calcolo Elettronico (CINECA). We acknowledge a generous CPU and storage allocation by the Italian Super-Computing Resource Allocation (ISCRA) as well as from the coordination of the ``Accordo Quadro MoU per lo svolgimento di attività congiunta di ricerca Nuove frontiere in Astrofisica: HPC e Data Exploration di nuova generazione'', together with storage from INFN-CNAF and INAF-IA2.

\bibliographystyle{JHEP}
\bibliography{bibliography.bib}


\appendix

\section{Test nonlinearities in the dark matter power spectra}
\label{appendix:a}
As showed in Section~\ref{sec:theory} (Figure~\ref{spectra_comp1} and Figure~\ref{spectra_comp2}), there are inaccuracies in the Takahashi2012+Bird2014 nonlinear model in the representation of both $P_{\dot{\Phi}\Phi}$ and $P_{\dot{\Phi}\delta}$ at $z=0.5$, where the sign inversion for the $M_{\nu} = 0.17$ eV case occurs at larger multipoles than the massless neutrino one. We decided then to analyse the trend of both Takahashi2012+Bird2014 and Mead2020 model for masses close to 0.17 eV to check possible irregularities. The results are shown in Figure~\ref{test1} for $P_{\dot{\Phi}\Phi}$
and in Figure~\ref{test2} for $P_{\dot{\Phi}\delta}$.
It is easy to notice how neither Takahashi2012+Bird2014 nor Mead2020 are able to  reproduce the expected trend in the $\ell_{inv}$ position for all $M_{\nu}$ at all redshifts. An improvement in the \texttt{Halofit} models is then required to correctly represent these signals. 
% Figure environment removed 
 % Figure environment removed

\section{Cosmic Variance (massless neutrino case)}
\label{appendix:b}
% Figure environment removed
We assess the uncertainty associated to the cross-spectra in terms of cosmic variance~\cite{Lesgourgues_2008,Seljak_2009}:
\begin{equation}
    \sigma_{C_{\ell}} = \sqrt{\frac{2(C_{\ell})^{2}}{2\ell + 1}}\,.
\end{equation}
For the computation of the cosmic variance of the cross-correlation of the ISWRS signal with the CMB-lensing potential and the galaxy distribution, we divide the signal variance ($\sigma$) in logarithmic equispaced bins $\Delta_{\ell_{bin_i}}$:
\begin{equation}
    \sigma_{C_{\ell}^{\dot\Phi\Phi}, \, bin_{i}} = \sqrt{\frac{(C_{\ell}^{\dot\Phi\Phi})^{2} + C_{\ell}^{TT}C_{\ell}^{\Phi\Phi}}{(2\ell + 1)\Delta\ell_{bin_{i}}}} \,,
    \label{varianceTP}
\end{equation}
\begin{equation}
    \sigma_{C_{\ell}^{\dot\Phi g}, \hspace{1mm}bin_{i}} = \sqrt{\frac{(C_{\ell}^{\dot\Phi g})^{2} + C_{\ell}^{TT}(C_{\ell}^{gg} + N^{gg})}{(2\ell + 1) \Delta\ell_{bin_{i}}}} \,.
    \label{varianceTG}
\end{equation}
The $C_{\ell}^{TT}$ that appears in  Equation~(\ref{varianceTP}) and Equation~(\ref{varianceTG}) is the auto-power spectrum of CMB temperature primary anisotropies. Working with $C_{\ell}^{TT}$ instead of $C_{\ell}^{\dot\Phi\dot\Phi}$ arises from the fact that any CMB measurement is  dominated by primary anisotropies, which are the main obstacle for a correct reconstruction of the ISWRS signal, and of secondary anisotropies in general.
 The $N^{gg}$ term in Equation~(\ref{varianceTG}) is the shot-noise spectrum.
In this analysis, only the cosmic variance for the massless neutrino case of both the cross-spectra analysed has been computed. Moreover, we compared it with the standard deviation of 5000 realisations we computed for the two signals. Results are shown in Figure~\ref{cosmic_variance}. The significant difference between the cosmic variance and the mocks variance is due to the fact that the 5000 gaussian realizations represent the same sky realization of the mocks, and consequently their scatter is small.
Anyway, it is easy to notice that, in both panels, the larger variance values are in the larger multipole range, therefore
the detection of $\ell_{inv}$ is highly unlikely.


\end{document}
