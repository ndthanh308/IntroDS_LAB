\section{Introduction}\label{sec:intro}

This position paper aims to complement a recent white paper on designing future intelligent ecosystems where autonomous Active InFerence (AIF) agents learn purposeful behavior through situated interactions with other AIF agents \cite{friston_designing_2022}. The white paper states that these agents ``... can be realized via (variational) message passing or belief propagation on a factor graph'' \cite[][abstract]{friston_designing_2022}. Here, we discuss the computational requirements for a factor graph software toolbox that supports this vision. Noting that the steep rise of commercialization opportunities for deep learning systems was greatly facilitated by the availability of professional-level toolboxes such as TensorFlow and successors, we claim that a high-quality AIF software toolbox is needed to realize the proposition in \cite{friston_designing_2022}. Therefore, in this paper, we ask the question: what properties should a factor graph toolbox possess that enable a competent engineer to develop relevant AIF agents? The question is important since the number of applications for autonomous AIF agents is expected to vastly outgrow the number of world-class experts in AIF and robotics. 

As an illustrating example, consider an engineer (Sarah) who needs to design a quad-legged robot that is tasked to enter a building and switch off a valve. We assume that Sarah is a competent engineer with an MS degree and a few years of experience in coding and control systems. She has some knowledge of probabilistic modeling but is not a top expert in those fields. 

%During the design period, Sarah does not know which building the robot needs to enter, so the robot possibly needs to adapt its skills (e.g., walking, stair-climbing in a burning building, etc.), and even learn new skills under situated conditions. 



In order to relieve Sarah from designing every detail of the robot, we expect that the robot possesses some ``intelligent'' adaptation capabilities. Firstly, the robot should be able to define sub-tasks and solve these tasks autonomously. Secondly, since we do not know a-priori the inside terrain of the building, the robot should be capable of adapting its walking and other locomotive skills under situated conditions. Thirdly, we expect that the robot performs robustly, in real-time, and cleverly manages the consumption of its computational resources. 

 All these robot properties should be supported seamlessly by Sarah's AIF software toolbox. For instance, she should not need to know the specifics of how to implement robustness in her algorithms or how many time steps the robot needs to look ahead in any  given situation for effective planning purposes. We want a toolbox that enables competent engineers to develop effective AIF agents, not a toolbox for a select group of world-class machine learning experts. We do expect that Sarah is capable of describing her beliefs about desired robot behavior through the high-level specification of a probabilistic (world or generative) model or, at least, the prior preferences or constraints that underwrite behavior.

After reviewing some motivating agent properties that follow immediately from committing to free energy minimization (section~\ref{sec:FEP-and-AIF}), we proceed to discuss why message passing in a factor graph is the befitting framework for implementing AIF agents (section ~\ref{sec:why-MP}). More specifically, we argue that a reactive programming-based implementation of message passing will be the standard in professional-level AIF tools (section~\ref{sec:Reactive-vs-Procedural}). In comparison to the usual procedural coding style, reactive message passing leads to increased robustness (section~\ref{sec:robustness}), lower power consumption (section~\ref{sec:low-power-consumption}), hard real-time processing (section~\ref{sec:real-time-processing}), and support for continual model structure adaptation (section~\ref{sec:structural-adaptation}). In section~\ref{sec:rxinfer} we introduce \texttt{RxInfer}, a toolbox-in-progress for developing AIF agents that robustly minimize free energy in real-time by reactive message passing. 

  

%In a similar way that Tensorflow and its successors enabled Sarah to design commercially valuable deep learning-based applications, we investigate the question of which kind of software tool would enable her to design the robot. 



%Since this paper is inspired by and aimed to complement \cite{friston_designing_2022}, we omit much of the argumentation for choosing Active Inference (AIF) as the underlying computational process. We also do not claim that every opinion in this paper is fully supported by experimental or theoretical evidence. Still, we contend that it is relevant to think about which engineering tools will be needed for competent mere mortals to design commercially interesting AIF agents. 





%the future of Artificial Intelligence (AI) technology for the design of complex signal processing and control algorithms. Roughly speaking, by signal processing algorithms, we mean perceptual tasks that people excel at, such as speech and visual scene understanding. Similarly, the type of control algorithms that we are interested in relate to complex tasks such as driving a car, riding a bike, or dribbling with a soccer ball.

%The design of these algorithms is notoriously difficult for engineers. Yet, people learn to perform these tasks automatically, just by interacting with their environment. In this paper, we discuss the technology that underlies the brain's capacity to develop very complex algorithms without any apparent supervision. We follow the guidance of the Free Energy Principle (FEP), which states that all processing in the brain can be interpreted as variational Free Energy (FE) minimization \parencite{friston_free_2012, friston_life_2013, friston_free_2019}. The essential questions we ask are (1) how is FE minimization organized in natural systems, and (2) is this technology transferable to engineering systems? 


%This is a position paper and our view is biased by the research ambitions of our own academic team\footnote{Our team, BIASlab (\url{http://biaslab.org}), is a research subgroup of the Signal Processing Systems group in the Department of Electrical Engineering at the Eindhoven University of Technology.} to automate the design processes for complex signal processing and control algorithms. This paper contains no equations but assumes some familiarity with the FEP. We aim to paint a vision on FEP-inspired AI technology and a path for how to get there. 
