
\section{Discussion}\label{sec:discussion}

\subsection{Review of arguments}

We shortly summarize our view on a professional-level supporting software toolbox for the design of relevant AIF agents, see also Table~\ref{tab:technology}. In section~\ref{sec:FEP-and-AIF}, we discussed a few extraordinary features that follow straightaway from committing to free energy minimization as the sole computational mechanism for a future AI ecosystem as proposed in \textcite{friston_designing_2022}. First, the FE functional in an AIF agent can be interpreted as a universal performance criterion that applies in principle to all problems. If FEM can be extended to structural model adaptation, then an AIF agent is naturally able to create and solve sub-problems. Moreover, by virtue of the decomposition of EFE into a sum of information- and goal-seeking costs, AIF agents naturally seek out small "smart" data sets.

In terms of FEM implementation, we asserted that useful models are highly factorized and sparse. Efficient inference in factorized models can always be described as message passing in a factor graph. In particular, nearly all known variants of   highly efficient message passing algorithms for FEM can be formulated in a single framework as minimizing a Constrained Bethe Free Energy (CBFE). 



\begin{table}[!b]
\centering
\begin{tabular}{ || c | c | c  || }
\hline 
\hspace{1pt} &
 \large{\textbf{realization technology}} & \large{\textbf{benefits}}  \\ 
 \hline   \hline 
   1  & 
    FEP, AIF
  & one solution approach; \\ & & smart data   \\
 \hline  
2  &  reactive message passing  & low power; \\ & & robustness; \\ & & real-time  \\
\hline 
3  & structural adaptation & problem refinement; \\ & & clever inference   \\   
 \hline 
\end{tabular}
    \caption{Summary of benefits for supporting reactive message passing and structural adaptation in an AIF agent.}
    \label{tab:technology}
\end{table}


We then claimed that a \emph{reactive} rather than procedural processing strategy is essential. Reactive message passing-based (RMP) inference is always interruptible with an inference result, thus supporting guaranteed real-time processing, which is a hard requirement for AIF agents in the real world. In comparison to the more common procedural programming approach to FEM, reactive processing also improves robustness, resource consumption, and the capability to make structural changes without the need for resetting the inference process. 

This latter feature, support for online structural adaptation is also a vital feature of a high-quality AIF toolbox. Online structural adaptation leads to both continual problem representation refinement (by lowering surprise) and to a more efficient inference process.  

\subsection{Review of existing tools}
Currently, there exists a small but vibrant research community on the development of open-source tools for simulating synthetic AIF agents. In this community, a few supporting packages have been released, including SPM \parencite{friston_et_al._spm12_2014}, PyMDP \parencite{heins_pymdp_2022} and \texttt{ForneyLab} \parencite{cox_factor_2019}. The SPM toolbox was originally written by Karl Friston and colleagues, and has developed into a very large set of tools and demonstrations for experimental validation of the scientific output of the UCL team and collaborators. PyMDP is a more recent Python package for simulating discrete-state POMDP models by Conor Heins, Alexander
Tschantz and a team of collaborators. \texttt{ForneyLab.jl} is a Julia package from BIASlab (\url{http://biaslab.org}) for simulating FE minimization by message passing in Forney-style factor graphs. Unfortunately, none of the above-mentioned tools support \emph{reactive} message passing-based inference. Therefore, we believe that these tools will serve the community well as AIF prototyping and validation tools, but they will not scale to support real-time, robust simulation of AIF agents with commercializable value. 

\subsection{Reactive message passing with \texttt{RxInfer}}\label{sec:rxinfer}

More recently, BIASlab has released the open-source Julia package \texttt{RxInfer} (\url{http://rxinfer.ml}) to support an engineer at Sarah's level to develop commercially relevant AIF agents that minimize FE by automated reactive message passing in a factor graph \cite{bagaevReactiveMessagePassing2023}. Julia is a modern open-source scientific programming language with roughly the syntax of MATLAB and out-of-the-box speed of C \cite{bezanson_julia_2017}.

The development process of \texttt{RxInfer} focuses on the following priorities:
\begin{enumerate}
    \item model space coverage
    \begin{itemize}
        \item \texttt{RxInfer} aims to support reactive message passing-based FEM for a very large set of freely definable relevant probabilistic models. 
    \end{itemize}
    \item user experience
    \begin{itemize}
        \item \texttt{RxInfer} aims to support a busy, competent researcher or developer who understands probabilistic modeling (but doesn't know Julia) to design and deploy an AIF agent into the world. In particular, a user-friendly specification of nested AIF agents should be supported. 
    \end{itemize}
    \item adaptation
    \begin{itemize}
        \item \texttt{RxInfer} aims to support continual adaptation by automated FEM over all movable parts of the CBFE functional, including states, parameters, structure, and variational constraints.
    \end{itemize}
    \item real-time
    \begin{itemize}
        \item \texttt{RxInfer} aims to process data streams in ``hard'' real-time, under situated conditions, even for large models. Larger models may lead to less accurate inference (in terms of KL-divergence between variational and Bayesian posteriors), but no crashes.  
    \end{itemize}
    \item low-power
    \begin{itemize}
        \item \texttt{RxInfer} aims to process data streams on any, possibly time-varying, power budget. Lower power budgets may lead to less accurate inference but no crashes.
    \end{itemize}
\end{enumerate}

At the time of writing this paper, \texttt{RxInfer} supports fast and robust automated CBFE minimization by reactive message passing for states and parameters in a large set of freely definable models. \texttt{RxInfer} processes streaming data very fast, but not yet guaranteed in hard real-time. User-friendly specifications of AIF agents will be released this summer. Model structure adaptation is supported by NUV priors (normal priors with unknown variance) \cite{loeliger_sparsity_2016}, but not yet by online Bayesian model reduction \cite{beckers_principled_2022, friston_bayesian_2018}. \texttt{RxInfer} comes with a large set of examples and is slated to support the above priority list in the future.


%In this position paper, we propose to realize automated design processes by synthetic active inference agents, as described under the umbrella of the Free Energy Principle. Table~\ref{tab:technology} extends Table~\ref{tab:natural-vs-engineered-design} by an extra column for technological factors that underlie the advantages of natural design over engineered design. 


%We have discussed AIF and the FEP in general as a design methodology that leads to simultaneous learning of both the problem representation and solution proposals from small, highly informative data sets. The FE functional in an AIF agent has been identified as a universal performance criterion that applies in principle to all problems. 

%The weakest link in an AIF agent-based design cycle is often the human engineer who is tasked to specify the generative models. We discussed how nature circumvents this issue by FE minimization over model structures over evolutionary time spans. FE minimization for structural adaptation is not yet well-developed in the engineering literature so we identify it as a research issue.  

%We also discussed how to implement an automated FE minimization process in a synthetic AIF agent. The FE minimization task is extremely challenging since AIF agents require real-time optimization of a time-varying functional that usually holds a very high-dimensional latent variable space. We have argued that inference in an AIF agent should proceed by MP-based FE minimization on a graph representation of the model. In particular, we claim that the entire MP-based inference engine should be implemented as a \emph{reactive} system, where the behavior of the system components is described by a set of declarative rather than imperative rules. Reactive MP is interruptable at any time with virtually no loss of intermediate results and no need for a system reset. As a result, RMP-based inference leads to several important qualities, including robustness to malfunction and adversarial attacks, and real-time processing guarantees on multiple different computational platforms without the need to modify the code. Since RMP-based inference seems to be a novel idea in the Bayesian inference literature, we identify the development of a high-quality RMP toolbox as a research priority as well. 


%We now return to our vision as stated in section~\ref{sec:vision-mission}: ``... in the long term (say, roughly 10 years from now), an engineer needs to write just a few pages of code for an agent that autonomously designs satisfying personalized hearing enhancement algorithms for the entire hearing industry.'' 

%For a moment, let us assume that \texttt{ReactiveMP.jl} indeed develops into a professional quality package for RMP-based FE minimization in freely specifiable probabilistic models. Assume that \texttt{ReactiveMP.jl} supports state, parameter and structure adaptation, that ongoing FE minimizing processes are  interruptible at any time to guarantee hard real-time and robust inference processing. What's left for the engineer to do?

%Note that an AIF agent contains a generative model for predicting both actual and desired future observations. To a large degree, the model for predicting actual future observations can in principle be learned over time through parameter and structural adaptation. However, the model cannot learn everything about desired future observations from its environment. In particular at the higher abstraction levels of the model, there must be some hard-coded knowledge about the purpose of the system. For instance, in the case of designing hearing enhancement algorithms for the hearing impaired, the purpose of the system might be to maximize user satisfaction or to maximize speech understanding.  Or, for a walking robot, the unchangeable prior might be the goal position and the desired action at that location (e.g., turn a valve). In either case, it is up to the engineer to encode these goals with unchangeable priors in the generative model. So what is left for the engineer is to \emph{encode an adaptive generative model with a fixed (high-level) prior for desired behavior}. In all likelihood, such a model specification should comfortably fit on a few pages of code. Everything else will be automated by real-time FE minimization in the generative model while it exchanges information with its environment. The generative model may adapt over time to optimize its structure, but the goal prior will remain unchanged. 


%\dmitry{Some thoughts about the paper contents. From a high level overview point of view, I like the content and the overall vision, but of course I left some comments. However, the paper has some very very strong claims, in my opinion, which we do not know the answer to yet (these claims might be true at the end although). Some of these claims are:
%%\\ - natural design is fully automated (see Table 2). (Why cannot we incorporate some prior knowledge (see my previous comments in learning how to bike section) from an engineer to enhance or speedup the structural adaptation of the model. Or maybe I got this claim wrong?)
%\\ - brain solves all problems with \emph{only one} single approach (I am not a neuroscientist, but I think brain has different regions that may operate slightly differently? Why would we make such a strong claim that there is only one ultimate to \emph{all} problems? Do we already know it for sure? I understand the idea, but still I would weaken this claim from \emph{all} to \emph{many} at least. There are some human activities that are very hard to formulate under FEP, e.g., making an art, like music or painting.)

%I am not comfortable to make such strong claims without discussing the alternatives in more detail (which as I understand this paper is not about). Thus, I would at least outline that we believe that the answers to these questions are true (or maybe under some mild assumptions), but it requires further research.
%}
%\bdv{There are indeed a lot of claims with too little backup for a scientific peer-reviewed paper. However, this is a \emph{position} paper so the aim is a bit different.}

%\albert{In general, I think it's a good story. I will be happy if the roadmap for \texttt{ReactiveMP.jl} will play out. On the other hand, this overview hides many research and engineering traps. We should indeed be careful with promises we make (to ourselves in the first place). In my opinion, if we want to achieve the goals outlined in Section \ref{sec:agenda} (which aren't impossible), we need more people to be involved in the active development/discussions of \texttt{ReactiveMP.jl}}
