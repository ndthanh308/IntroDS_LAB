\documentclass[]{article}

\usepackage[T1]{fontenc}
\usepackage{amsmath,amssymb}
\usepackage{graphicx}

\usepackage{hyperref}
\usepackage{color}
\usepackage{footmisc}

% algorithm
\usepackage{algorithm}
\usepackage{algpseudocode}

\usepackage[backend=biber,style=numeric]{biblatex}
\addbibresource{bert-zotero.bib}


\begin{document}
%
\title{Toward Design of \\ Synthetic Active Inference Agents \\ by Mere Mortals}
%

\author{Bert de Vries \\ 
Eindhoven University of Technology \\ Eindhoven, the Netherlands \\
\texttt{bert.de.vries@tue.nl}} 

\maketitle   

\begin{abstract}
The theoretical properties of active inference agents are impressive, but how do we realize effective agents in working hardware and software on edge devices? This is an interesting problem because the computational load for policy exploration explodes exponentially, while the computational resources are very limited for edge devices. In this paper, we discuss the necessary features for a software toolbox that supports a competent non-expert engineer to develop working active inference agents. We introduce a toolbox-in-progress that aims to accelerate the democratization of active inference agents in a similar way as TensorFlow propelled applications of deep learning technology.

\end{abstract}


\section{Introduction}
Current quantum hardware is unable to carry out universal quantum computations due to the buildup of errors that occur during the computation. 
The magnitude of the individual error is currently above the value that the Threshold Theorem requires in order to kick-start quantum error correction and fault-tolerant quantum computation~\cite[Section 10.6]{nielsen_chuang_2010}. 
Although the experimentally achieved fidelity rates are promising and the error bounds are inching closer to the required threshold, we will have to work for the foreseeable future with quantum hardware with errors that build-up during the computation.  This implies that we can only do a limited number of steps before the output of the computation has become completely uncorrelated with the intended one.

For fault-tolerant quantum computing, we repeat four steps: 
1) We apply a number of single and two-qubit quantum gates, in parallel whenever possible; 
2) We perform a syndrome measurement on a subset of the qubits; 
3) We perform fast classical computations to determine which errors have occurred and how to correct them; 
and, 4) We apply correction terms based on the classical computations.
We then repeat these four steps with a next sequence of gates. 
These four steps are essential to fault-tolerant quantum computing. 


The starting point of this work is to use the four steps outlined above, not to carry out error correction and fault-tolerant computation, but to enhance short, constant-depth, {\em uncorrected} quantum circuits that perform single qubit gates and {\em nearest-neighbor} two qubit gates. 
Since in the long run we will have to implement error-correction and fault-tolerant computation anyhow, and this is done by such a four-step process, why not make other use of this architecture? Moreover, on some of the quantum hardware platforms, these operations are already in place.
Embracing this idea we naturally arrive at the question: what is the computational power of \textit{low-depth} quantum-classical circuits organized as in the four steps outlined above? 
We thus investigate circuits that execute a small, ideally constant, number of stages, where at each stage we may apply, in parallel, single qubit gates and {\em nearest-neighbor} two qubit gates, followed by measurements, followed by low-depth classical computations of which the outcome can control quantum gates in later stages. 
It is not clear, at first, whether such circuits, especially with constant depth, can do anything remotely useful. 
But we will see that this is indeed the case: many quantum computations can be done by such circuits in constant depth. 
By parallelizing quantum computations in this way, we improve the overall computational capabilities of these circuits, as we do not incur errors on qubits that are idle, simply because qubits are not idle for a very long time. 
Furthermore, reducing the depth of quantum circuits, at the cost of increasing width, allows the circuit to be run faster even if errors occur.

The first usage of such a four-step layout, not to do error correction, but to perform computations, can be found in the paradigm of measurement-based quantum computing~\cite{gottesman1999demonstrating,raussendorf2001one,jozsa2006introduction,clark2007generalised}: 
A universal form of quantum computing where a quantum state is prepared and operations are performed by measuring qubits in different bases, depending on previous measurements and intermediate measurements.

\citeauthor{PhamSvore2013} were the first to formalize the four-step protocol for performing computations~\cite{PhamSvore2013}. They included specific hardware topologies by considering two-dimensional graphs for imposing constraints on qubit interactions. In their model, they develop circuits for particularly useful multi-qubit gates, including specifying costs in the width, number of qubits, depth, number of concurrent time steps, size, and total number of non-Identity operations.
As a result, they find an algorithm that factors integers in polylogarithmic depth.
\citeauthor{Browne:2011} showed that the main tool in the work by \citeauthor{PhamSvore2013}, the fan-out gate, can also be replaced by additional log-depth classical computations in the measurement-based quantum computing setting~\cite{Browne:2011}.

More recently, \citeauthor{Cirac:2021} introduced a scheme to implement unitary operations involving quantum circuits combined with Local Operations and Classical Communication ($\mathsf{LOCC}$) channels: $\mathsf{LOCC}$-assisted quantum circuits~\cite{Cirac:2021}. Similarly to the four-step scheme we just described, they allow for a short depth circuit to be run on the qubits, followed by one round of $\mathsf{LOCC}$, in which ancilla qubits are measured and local unitaries are applied based on the measurement outcomes. They show that in this model any 1D transitionally invariant matrix-product state (MPS) with fixed bond dimension is in the same phase of matter as the trivial state. Similar ideas can be found in~\cite{TVV_NonAbelianTopologicalOrder_2022, tantivasadakarn2021long}.

In this work, we introduce a new model, called \textit{Local Alternating Quantum-Classical Computations} ($\LAQCC$). In this model we alternate between running quantum circuits (constrained by locality), ending in the measurement of a subset of qubits, and fast classical computations based on the measurement results. The outcome of the classical computations are then used to control future quantum circuits. We allow for flexibility in this model, by giving different constraints to the power of both the quantum circuits and the classical circuits as well as the number of alternations between them. 
Most attention will be given to $\LAQCC$ containing quantum circuits of constant depth, classical circuits of logarithmic depth and at most a constant number of alternations between them. 
Any circuit constructed in this model is considered to be of constant depth. 
We restrict ourselves to logarithmic depth classical computations, as this is the first natural and non-trivial extension beyond constant-depth classical computations. 
Constant-depth classical computations do however also have an equivalent constant-depth quantum implementation.

The definition of $\LAQCC$ sharpens the original definition of \citeauthor{PhamSvore2013} by adding constraints to the intermediate classical computations. This allows us to bound the power of $\LAQCC$ from above. 

The main result of \citeauthor{Cirac:2021}, that 1D translational invariant MPS with fixed bond dimension can be prepared by $\mathsf{LOCC}$-assisted circuits, relies on local symmetries of the MPS. These symmetries allow them to prepare local states (on a constant number of qubits) and glue them together by doing one round of the appropriate entangling measurement and corrections, after which they run a round of local unitaries to get the desired result. This general scheme for preparing states that exhibit an MPS description with the appropriate local symmetries requires only geometrically local unitaries and one round of measurement and corrections an therefore is accessible in $\LAQCC$. Studying different local symmetries, known as Symmetry Protected Topological (SPT) phases of matter, to find measurement-based constant depth circuits for states is a broad ongoing field of research~\cite{TVV_NonAbelianTopologicalOrder_2022, tantivasadakarn2021long, smith2023deterministic}. 
All these schemes have a $\LAQCC$ implementation.

%$\LAQCC$-circuits also exist for general schemes of preparing local states, based on the local tensors, and gluing them together using one round of entangled measurement and corrections, based on the local symmetry. 
%The main result of \citeauthor{Cirac:2021}, that 1D translational invariant MPS with fixed bond dimension can be prepared by $\mathsf{LOCC}$-assisted circuits, relies heavily on local symmetries of the MPS and as a result also has an equivalent $\LAQCC$ implementation. 
%The corrections applied after the measurement round are local unitaries depending on the local symmetries of the MPS. 

 

%This general scheme of preparing local states, based on the local tensors, and gluing it together by doing one round of entangled measurement and corrections, based on the local symmetry, is accessible in $\LAQCC$.
Note however that \citeauthor{Cirac:2021} also suggest a circuit for the $W$-state.
This circuit uses sequentially and dependent measurement-based corrections of the ancilla qubits. 
These dependent measurements translate to sequential alternations between the quantum and classical circuits and therefore increase the total depth to linear depth, exceeding the constant-depth constraints imposed by $\LAQCC$-circuits. 

We study the power of the $\LAQCC$ model with respect to state preparation, showing that even with only constant quantum-depth and logarithmic classical depth it remains possible to prepare states with long-range entanglement.
Another surprising result is that it is unlikely that $\LAQCC$ circuits are classically simulatable. We show that any instantaneous quantum polynomial-time (IQP) circuit~\cite{Bremner2010,Shepherd2009} has an $\LAQCC$ implementation.
Classical simulation of IQP circuits implies the collapse of the polynomial hierarchy to the third level, which is not believed to be true~\cite{Bremner2017}. Therefore, we expect that $\LAQCC$ circuits are unlikely to be classically simulatable. We bound the power of $\LAQCC$ by showing that it is contained in $\QNC^1$, the class of polynomial-size, log-depth circuits.

Next, we also study the power that intermediate classical calculations can add to quantum computations, by considering a new model that alternates between polynomially many polynomial-depth quantum circuits and unbounded classical computations
We study this model by doing a complexity theoretical analysis, where we draw inspiration from the notions of complexity given by \citeauthor{RosenthalYuen:2022}, \citeauthor{MetgerYuen:2023}, and \citeauthor{Aaronson:2004}.
All three complexity notions are based on the notion of state preparation, instead of more traditional definition of complexity such as the decidability of a computational problem. 
The first two consider classes based on sequences of quantum states preparable by a polynomial-sized quantum circuit, where the circuits are uniformly generated by a computational class, for instance, the class $\mathsf{PSPACE}$, which results in the complexity class $\mathsf{StatePSPACE}$~\cite{RosenthalYuen:2022,MetgerYuen:2023}.
The third notion considers a relative complexity, where the complexity is measured between two given states, and is measured by the number of gates, from a given gate-set, required to transform one state in another state~\cite{Aaronson:2004}. 
For our definition of state preparation complexity, we drop the uniformity constraint from~\cite{RosenthalYuen:2022,MetgerYuen:2023} and define a class as $\mathsf{StateX}$, which refers to states preparable by circuits of type $\mathsf{X}$. 
As an example, if $\mathsf{X} = \QNC^0$, this results in the class $\mathsf{StateQNC^0}$, which is the set of states preparable from the $\ket{0}^n$ state by poly-size constant-depth circuits. 
This notion is similar to the relative complexity from~\cite{Aaronson:2004}, where one state is the  $\ket{0}^n$ state and instead of counting the number of gates we consider the set of states preparable by a fixed number of gates. Using this notion of complexity we show that any state preparable by an $\LAQCC^*$ circuit is also preparable by a $\mathsf{PostQPoly}$ circuit, the class of circuits of polynomial depth with an additional post-selection gate. 

All Clifford circuits have a constant-depth $\LAQCC$ implementation, implying that any stabilizer state can be implemented by a constant-depth $\LAQCC$ circuit, see Section~\ref{sec:clifford_circuits} for a proof of this statement. 
Efficient circuits for stabilizer states have been known already through measurement-based quantum computing. Therefore this paper focuses on the preparation of non-stabilizer states, and as a surprising result we find novel constant-depth protocols for four very natural classes of non-stabilizer states.
Despite the extensive research into these four classes of non-stabilizer states and the many applications of them, no efficient constant- or low-depth state preparation protocols are known yet. We specifically consider these four classes as they are all often used as initial states in other algorithms.

The first state is a uniform superposition over an arbitrary number of states. 
This state finds applications in many quantum algorithms, as they often start with a uniform superposition over multiple states. 
This superposition is often achieved by applying Hadamard gates to every qubit due to its simplicity to prepare. 
Yet, the analysis of many algorithms, such as Shor's algorithm~\cite{Shor:1997}, would benefit from a different initial superposition. 
The circuit to prepare the uniform superposition over an arbitrary number of states uses an exact version of Grover search as a subroutine, that turns a probabilistic circuit, with a known constant probability of success, into a deterministic circuit. 
We use the circuit for preparing a uniform superposition over an arbitrary number of states as a subroutine in the next two quantum state preparation protocols. 

The second state is the $W$-state, the uniform superposition over all computational basis states of Hamming-weight~$1$, a natural long-ranged entangled state that displays a fundamentally nonequivalent type of entanglement from the Greenberger–Horne–Zeilinger state~\cite{WState:2000}, for which $\LAQCC$-type constant-depth circuits were previously known~\cite{PhamSvore2013, Cirac:2021}. 
The $W$-state is often used as benchmark for new quantum hardware~\cite{Haffner2005,Neeley2010,GarciaPerez:2021}. 
A novel way to prepare the $W$-state therefore gives a new way to benchmark different quantum devices with each other. 
A circuit for preparing the $W$-state was given in~\cite{Cirac:2021}, but this implementation requires sequentially alternating measurements followed by local unitaries, which in the $\LAQCC$ model is not considered to be of constant depth. 
We improve this protocol by giving an $\LAQCC$ implementation of the $W$-state, based on a compress-uncompress method that links the one-hot and binary encoding of integers.

The third state considered is the Dicke state, a generalization of the $W$-state, a superposition over all computational basis states with Hamming-weight $k$~\cite{Dicke:1954}. 
Dicke states have relevance in various practical settings.
For instance, for quantum game theory~\cite{zdemir2007}, quantum storage~\cite{Bacon_Compress:2006,Plesch:2010}, quantum error correction~\cite{ouyang2014permutation}, quantum metrology~\cite{toth2012multipartite}, and quantum networking~\cite{prevedel2009experimental}. 
Dicke states have been used as a starting state for variational optimization algorithms, most notably Quantum Alternating Operator Ansatz (QAOA)~\cite{Hadfield2019}, to find solutions to problems such as Maximum k-vertex Cover~\cite{Brandhofer2022,cook2020quantum}.
The ground states of physical Hamiltonians describing one-dimensional chains tend to show a resemblance to Dicke states such as states resulting from the Bethe ansatz, making them an ideal starting state when investigating the ground state behavior of these Hamiltonians~\cite{TDL_BetheAnsatzDerivation:2010,B_ExcitedStateQuantumPhaseTransitions:2013,DickeTransitions:2021}. 
For instance, the algorithm by \citeauthor{van2021preparing}, who give an algorithm to prepare the Bethe ansatz eigenstates of the spin-1/2 XXZ spin chain, starts by first preparing a Dicke state~\cite{van2021preparing}. 
A Dicke-state preparation protocol based on the compress-uncompress methodology used in the $W$-state furthermore finds applications in entanglement distillation, where the entanglement of a large state is concentrated on only a few qubits. 
Efficient deterministic circuits for preparing Dicke states have been proposed by \citeauthor{bartschi2019deterministic}~\cite{bartschi2019deterministic, bartschi2022deterministic_short_depth}. 
They provide a quantum circuit of depth $\mathO(k \log(\frac{n}{k}))$, allowing arbitrary connectivity, to prepare a Dicke state, which they conjecture to be optimal when $k$ is constant. 
In this work, we provide a constant-depth $\LAQCC$ circuit below their conjectured bound already for constant $k$. 
However, this does not directly disprove their conjecture, as we allow for intermediate measurements and classical computations. 
More significantly, we even construct constant-depth $\LAQCC$ circuits for $k = \mathO(\sqrt{n})$ greatly improving their bound.
This construction extends the compress-uncompress method for the $W$-state combined with additional subroutines. 

We continue with a log-depth state preparation protocol for the Dicke-state for arbitrary $k$. 
This protocol implements an efficient transformation between the factoradic number representation and the combinatorial number representation of a positive integer. 
The combinatorial number representation relates directly to the Dicke state. 
The provided efficient transformation between number representation systems might be of independent interest. 

We conclude by modifying our protocol for preparing a Dicke-state to a protocol that prepares quantum many-body scar states in constant-depth. 
These states have low entanglement and longer coherence times than states with similar energy density.
These characteristics make many-body scar states interesting to analyze and relevant within physics.
Many-body scar states appear for instance in the AKLT model~\cite{AKLT:1987,MRBAR:2018,MRB:2018} and different spin models~\cite{SI:2019,MOBFR:2020}.
Known methods for preparing these states have polynomial-depth~\cite{Gustafson:2023}, whereas our circuit has constant depth. 

% We conclude by studying the power that intermediate classical calculations can add to quantum computations. 
% In this study, we define a new model that relaxes constant-depth quantum circuits to polynomial depth quantum circuits, log-depth classical calculations to unbounded classical computations and a constant number of alternations to a polynomial number of alternations. 
% We call this model $\LAQCC^*$. 
% We study this model by doing a complexity theoretical analysis, where we draw inspiration from the notions of complexity given by \citeauthor{RosenthalYuen:2022}, \citeauthor{MetgerYuen:2023}, and \citeauthor{Aaronson:2004}.
% All three complexity notions are based on the notion of state preparation, instead of more traditional definition of complexity such as the decidability of a computational problem. 
% The first two consider classes based on sequences of quantum states preparable by a polynomial-sized quantum circuit, where the circuits are uniformly generated by a computational class, for instance, the class $\mathsf{PSPACE}$, which results in the complexity class $\mathsf{StatePSPACE}$~\cite{RosenthalYuen:2022,MetgerYuen:2023}.
% The third notion considers a relative complexity, where the complexity is measured between two given states, and is measured by the number of gates, from a given gate-set, required to transform one state in another state~\cite{Aaronson:2004}. 
% For our definition of state preparation complexity, we drop the uniformity constraint from~\cite{RosenthalYuen:2022,MetgerYuen:2023} and define a class as $\mathsf{StateX}$, which refers to states preparable by circuits of type $\mathsf{X}$. 
% As an example, if $\mathsf{X} = \QNC^0$, this results in the class $\mathsf{StateQNC^0}$, which is the set of states preparable from the $\ket{0}^n$ state by poly-size constant-depth circuits. 
% This notion is similar to the relative complexity from~\cite{Aaronson:2004}, where one state is the  $\ket{0}^n$ state and instead of counting the number of gates we consider the set of states preparable by a fixed number of gates. Using this notion of complexity we show that any state preparable by an $\LAQCC^*$ circuit is also preparable by a $\mathsf{PostQPoly}$ circuit, the class of circuits of polynomial depth with an additional post-selection gate. 

\paragraph{Summary of results}
\begin{itemize}
    \item We give a new definition of a computational model that captures the power of the four step process: applying a constant number of layers of one- and two-qubit gates; performing a syndrome measurement; perform a fast classical computation determining corrections; apply corrections. We call this model \emph{Local Alternating Quantum Classical Computations}, or $\LAQCC$ for short. In this model we bound the allowed quantum operations, intermediate classical calculations, and number of rounds separately. In Section~\ref{sec:LAQCC_model} we define this model and give a list of operations based on results from literature contained in this computational model. In some of these operations we explicitly use that we allow for multiple, but at most constant, rounds  of corrections.
    \item  We show show that there exist $\LAQCC$ circuits that can not be weakly simulated in Section~\ref{sec:IQP_in_LAQCC}. We further show that for every $\LAQCC$ circuit there exists a $\QNC^1$ circuit simulating it perfectly, in Section~\ref{sec:LAQCC_in_QNC1}.
    \item We introduce a new type computational complexity for preparing states and show that the extension of $\LAQCC$ where we allow a polynomial number of rounds and unbounded classical computation, is contained in $\mathsf{PostQPoly}$, the class of polynomial circuits with post-selection, in Section~\ref{sec:Complexity results}.
    \item We show a protocol to prepare the uniform superposition state of size $q$ in $\LAQCC$ using $\mathO(\ceil{\log_2(q)}^2)$ qubits in Section~\ref{sec:superposition_modulo_q}. 
    \item We show a protocol to prepare the $W_n$ state in $\LAQCC$ using $\mathO(n\log(n))$ qubits in Section~\ref{sec:W_state_in_LAQCC}.
    \item We show two ways of preparing the Dicke-$(n,k)$ state. The first method is in $\LAQCC$, works up to $k = \mathO(\sqrt{n})$, uses $\mathO(n^2\log(n))$ qubits, and is found in Section~\ref{sec:dicke:small_k}. The second method is in $\LAQCC\text{-}\mathsf{LOG}$ (an extension of $\LAQCC$ allowing for logarithmic number of alterations instead of constant), works for any $k$, uses $\mathO(\text{poly}(n))$ qubits, and is found in Section~\ref{sec:Dicke_in_LAQCC_LOG}. 
    \item We extend on our $\LAQCC$ method of generating Dicke-$(n,k)$ states for $k = \mathO(\sqrt{n})$ and show a protocol to generate many-body scar states for a particular Hamiltonian in $\LAQCC$ (Section~\ref{sec:many_body_scar}). 
\end{itemize}
Summarized in a table, we provide the following state generation protocols:
\begin{table}[htb]
\centering
\begin{tabular}{l|l|l|l}
\textbf{State description} & \textbf{Width} & \textbf{Depth} & \textbf{Implementation}\\
\hline 
Uniform superposition mod $q$: $\frac{1}{\sqrt{q}} \sum_{i = 0}^{q-1}\ket{i}$ & $\mathO(\ceil{\log^2 q})$ & $\mathO(1)$ & Section~\ref{sec:superposition_modulo_q}\\

$W$-state: $\frac{1}{\sqrt{n}}\sum_{i = 0}^{n-1}\ket{e_i}$ & $\mathO(n \log n)$ & $\mathO(1)$ & Section~\ref{sec:W_state_in_LAQCC}\\

Dicke-$(n,k)$, $k = \mathO(\sqrt{n})$: $\binom{n}{k}^{-1/2}\sum_{x \in \{0,1\}^n: |x| = k} \ket{x}$ &  $\mathO(n^2\log n)$ & $\mathO(1)$ 
&Section~\ref{sec:dicke:small_k}\\

Dicke-$(n,k)$: $\binom{n}{k}^{-1/2}\sum_{x \in \{0,1\}^n: |x| = k} \ket{x}$ & $\mathO(\text{poly}(n))$ & $\mathO(\log n)$ &Section~\ref{sec:Dicke_in_LAQCC_LOG}\\

QMBS: $\ket{S_k} = \frac{1}{k! \sqrt{\mathcal N(n,k)}}(Q^\dagger)^k \ket{\Omega}$ &  $\mathO(n^2\log n)$ & $\mathO(1)$  &  Section~\ref{sec:many_body_scar}
\end{tabular}
\caption{Summary of state preparation protocols given in this paper.}
\label{tab:sate_prep}
\end{table}
In the entry for the quantum many-body scar state $Q$ denotes the raising operator and $\mathcal N(n,k)=\binom{n-k-1}{k}$. 
Section~\ref{sec:many_body_scar} will provide more details on the variables and the implementation. 

\paragraph{Organization of the paper}
\noindent We first introduce relevant preliminaries in Section~\ref{sec:preliminaries}. 
In Section~\ref{sec:LAQCC_model} we formally define the class of Local Alternating Quantum-Classical Computations ($\LAQCC$). We also show that any Clifford circuit can be implemented in constant depth $\LAQCC$ (a result based on a result from measurement-based quantum computing~\cite{jozsa2006introduction}). 
This result allows us to give many useful multi-qubit gates and routines in Section~\ref{sec:gates_created_in_LAQCC}. 
Beyond that we show that constant depth $\LAQCC$ circuits are contained in $\QNC^1$ and that any $\mathsf{IQP}$ circuit has an $\LAQCC$ implementation.
We conclude this section with an analysis of a more powerful instantiation of $\LAQCC$ and show an inclusion with respect to the class $\mathsf{PostQPoly}$, which is the class of circuits of polynomial depth with one additional post-selection gate. 
In Section~\ref{sec:state_prep_in_LAQCC} we give $\LAQCC$ circuit implementations for preparing the uniform superposition over an arbitrary number of states, the $W$-state and the Dicke state up to $k = \mathO(\sqrt{n})$. We furthermore give a log-depth circuit implementation for preparing the Dicke state for any $k$. We conclude by showing a $\LAQCC$ circuit for generating many body scar states of a particular type of Hamiltonian.


\section{The Free Energy Principle and Active Inference}\label{sec:FEP-and-AIF}


\subsection{FEP for synthetic AIF agents}

The Free Energy Principle (FEP) describes  self-organizing behavior in persistent natural agents (such as a brain) as the minimization of an information-theoretic functional that is known as the variational Free Energy (FE).\footnote{For reference, we use the following abbreviations in this paper: Active Inference (AIF), Constrained Bethe Free Energy (CBFE), Expected Free Energy (EFE), (variational) Free Energy (FE), Free Energy Principle (FEP), Free Energy Minimization (FEM), Message Passing (MP), Reactive Message Passing (RMP).} Essentially, the FEP is a commitment to describing adaptive behavior by Hamilton's Principle of Least Action \cite{ lanczos_variational_1986}. The process of executing FE minimization in an agent that interacts with its environment through both active and sensory states is called \emph{Active Inference} (AIF). Crucially, the FEP claims that, in natural agents, FE minimization is \emph{all that is going on}. While engineering fields such as signal processing, control, and machine learning are considered different disciplines, in nature these fields all relate to the same computational mechanism, namely FE minimization. 

For an engineer, this is good news. If we wish to design a synthetic AIF agent that learns purposeful behavior solely through self-directed environmental interactions, we can focus on two tasks:
\begin{enumerate}
    \item Specification of the agent's model and inference constraints. This is equivalent to the specification of a (constrained) FE functional. 
    \item A recipe to continually minimize the FE in that model under situated conditions, driven by environmental interactions.
\end{enumerate}

We are interested in the development of an engineering toolbox to support these two tasks. 


%In this context, the FE is a performance function of beliefs, i.e., probabilistic functions over appropriate values for the variables in the model (the ``brain'') under consideration. FE minimization leads to maximizing the performance of the brain's model as a prediction engine for future sensory inputs. The process of FE minimization seeks to maximize both prediction accuracy and model simplicity, thus leading to \emph{inference algorithms that maximize data prediction accuracy while minimizing the amount of required computations}.

%This FE decomposition into accuracy plus simplicity underlies the brain's capacity to attend to many tasks simultaneously at a very low power consumption rate. 

%FE minimization in an AIF agent is equivalent to performing variational Bayesian inference in that agent, which involves updating the agent's beliefs about the state of the world. The brain continually updates beliefs about the state of the world but it will do so with minimal required adaptations to previously held beliefs.  Perception (signal processing in engineering terms) is just FE minimization with respect to synaptic firing rates that represent the state of the world. Learning (in engineering terms called: machine learning or parameter estimation) minimizes FE in relation to the synaptic efficacies, and movement (in engineering terms: control) is due to FE minimization with respect to firing rates in synapses that connect to muscle spindles. Updating beliefs by FE minimization about any aspect of the model is always accomplished continually and simultaneously. Thus, in a FEM agent, we cannot consider signal processing, control, and machine learning separately as they are part of the same process in the same model. \\


%The application of FEP and AIF to algorithm design leads by itself (i.e., independent of the implementation methods) to at least two of the six remarkable benefits of the natural design approach, namely, (1) learning from small data sets, and (2) a one-solution approach to all problems. Next, we shortly recapitulate these ideas.  


\subsection{FEM for simultaneous refinement of problem representation and solution proposal}\label{sec:one-solutiona-approach}

An important quality of the robot will be to define tasks for itself and solve these tasks autonomously. Here, we shortly discuss how the FEP supports this objective.  

Consider a generative model $p(x,s,u)$, where $x$ are observed sensory inputs, $u$ are latent control signals and $s$ are latent internal states. For notational ease, we collect the latent variables by $z = \{s,u\}$. The variational FE for model $p(x,z)$ and variational posterior $q(z)$ is then given by 
\begin{subequations}
\begin{align}
    F[q,p] 
    &= \underbrace{-\log p(x)}_{\text{surprise}} + \underbrace{\sum_z q(z) \log \frac{q(z)}{p(z|x)}}_{\text{bound}}  \label{eq:bound-evidence}\\
    &= \underbrace{\sum_z q(z) \log \frac{q(z)}{p(z)}}_{\text{complexity}} - \underbrace{\sum_z q(z) \log p(x|z)}_{\text{accuracy}} \label{eq:complexity-accuracy}\,.
\end{align}
\end{subequations}
The FE functional in \eqref{eq:bound-evidence} can be interpreted as the sum of surprise (negative log-evidence) and a non-negative bound that is the Kullback-Leibler divergence between the variational and the optimal (Bayesian) posterior. The first term, surprise, can be interpreted as a performance score for the problem representation in the model. This term is completely independent of any inference performance issues. The second term (the bound) scores how well actual solutions are inferred, relative to optimal (Bayesian) inference solutions. In other words, the FE functional is a universal cost function that can be interpreted as the sum of problem representation and solution proposal costs. FE minimization leads toward improving both the problem representation and solving the problem through inference over latent variables. In particular, FE minimization over a particular model structure $p$ should lead to nested sub-models that reflect the causal structure of the sensory data. Sub-tasks are solved by FE minimization in these sub-models. Hence, both creation of subtasks and solving these subtasks are driven solely by FE minimization. 

%serves as a universal cost function that is only low when both the problem representation and the inference process are "good enough".   

In conclusion, a high-end toolbox should be capable to minimize FE both over (beliefs over) latent variables through adaptation of $q(z)$ (leading to better solution proposals for the current model $p$), and over the model structure $p$ (leading to a better problem representation). 

As an aside, an interesting consequence of the FE decomposition into problem plus solution costs is that a relatively poor problem representation with a superior inference process may be preferred (evidenced by lower FE), over a model with a good problem representation (high Bayesian evidence) where inference costs are high. The notion that the model with the largest Bayesian evidence may not be the most useful model in a practical application, casts an interesting light on the common interpretation of FE as a mere upper bound on Bayesian evidence. We argue here that FE is actually a more principled performance score for a model, since in addition to Bayesian model evidence, FE also scores the performance loss in a model due to an inaccurate inference process. 

\subsection{AIF for smart data sets and resource management}

If we want the robot to cope with unknown physical terrain conditions, it is not sufficient to pre-train the robot offline on a large set of relevant examples. The robot must be able to acquire relevant new data and update its model under real-world conditions.  

FE minimization in the generative model's roll-out to the future results in the minimization of a cost functional known as the Expected Free Energy (EFE). It can be shown that the EFE decomposes into a sum of pragmatic (goal-driven, exploitation) and epistemic (information-seeking, exploration) costs \cite{friston_active_2015}. As a result, inferred actions balance the need to acquire informative data (to learn a better predictive model) with the goal to reach desired future behavior. 

In contrast to the current AI direction towards training larger models on larger data sets, an active inference process elicits an optimally informative, small (``smart'') data set for training of just ``good-enough'' models to achieve a desired behavior. AIF agents adapt enough to accomplish the task at hand while minimizing the consumption of resources such as energy, data, and time. The trade-off between data accuracy and resource consumption is driven by the decomposition in \eqref{eq:complexity-accuracy} of FE as a measure of complexity minus accuracy. According to this decomposition, more accurate models are only pursued if the increase in accuracy outweighs the resource consumption costs.   

In short, AIF agents that are driven solely by FE minimization will inherently manage their computational resources. These agents automatically infer actions that elicit appropriately informative data to upgrade their skills toward good-enough performance levels. Since both the agent and environment mutually affect each other in a real-time information processing loop, it would not be possible to acquire the same data set through the sampling of the environment without the agent's participation. 




%, just enough to  FE minimization leads to models that are good enough for the task at hand, adapt in real-time,  drives There is an inherent drive toward This is nature's technological edge that underlies learning from actively in-situ selected smart data sets, rather than training from large data bases in an offline fashion. 


\section{FE Minimization by Reactive Message Passing}\label{sec:RMP}


\subsection{Why message passing-based inference?}\label{sec:why-MP}

Up to this point, our arguments strongly supported AIF as an information processing engine for the robot. Unfortunately, the computational demands for simulating a non-trivial synthetic AIF agent are extreme. For comparison, consider the human brain that minimizes in real-time, for less than 20 watts, a highly time-varying FE functional (visual data rate about of about a million bits per second) over about $100$ trillion latent variables (synapses). It has been estimated that the human brain consumes about a million times less energy than a high-tech silicon computer on quantitatively comparable information processing tasks. \cite{smirnova_organoid_2023}. 

Clearly, the human brain minimizes FE in a very different way than is available in standard optimization toolboxes. In this section, we will argue for developing a FE minimization toolbox based on reactive message passing in a factor graph.   

% Figure environment removed

First, we shortly recapitulate why message passing in factor graphs is an effective inference method for large models. Consider a factorized multivariate function
\begin{align}\label{eq:factorized-model}
p(x_1,&x_2,\ldots,x_7) \notag \\
&= f_a(x_1) f_b(x_2) f_c(x_1,x_2,x_3) f_d(x_4) f_e(x_3,x_4,x_5) f_f(x_6) f_g(x_5,x_6,x_7)
\end{align}
Assume that we are interested in inferring (the so-called marginal distribution) 
\begin{equation}\label{eq:x3-marginal}
  p(x_3) = \sum_{x_1}\sum_{x_2}\sum_{x_4}\sum_{x_5}\sum_{x_6}\sum_{x_7} p(x_1,x_2,\ldots,x_7)  
\end{equation}
If each variable $x_i$ in \eqref{eq:x3-marginal} has about $10$ possible values, then the sum contains about $1$ million terms. However, making use of the factorization \eqref{eq:factorized-model} and the distributive law \cite{noauthor_distributive_2022}, we can rewrite this sum as 
\begin{align}\label{eq:x3-marginal-by-mp}
p(&x_3) = 
\bigg( \overbrace{\sum_{x_1}\sum_{x_2} f_a(x_1) f_b(x_2) f_c(x_1,x_2,x_3)}^{\overrightarrow{\mu}_3(x_3)} \bigg) \cdot \notag \\
&\cdot \bigg( \underbrace{\sum_{x_4} \sum_{x_5} f_d(x_4) f_e(x_3,x_4,x_5) \big( \overbrace{\sum_{x_6} \sum_{x_7} f_f(x_6) f_g(x_5,x_6,x_7)}^{\overleftarrow{\mu}_5(x_5)}\big)}_{\overleftarrow{\mu}_3(x_3)}\bigg)
\end{align}

The computation in \eqref{eq:x3-marginal-by-mp}, which requires only a few hundred summations and multiplications, is clearly preferred from a computational load viewpoint. To execute \eqref{eq:x3-marginal-by-mp}, we need to compute intermediate results $\overrightarrow{\mu}_{i}(x_i)$ and $\overleftarrow{\mu}_{i}(x_i)$ that afford an interpretation of local messages in a Forney-style Factor Graph (FFG) representation of the model, see Fig.~\ref{fig:example-ffg}.

Variational FE minimization can also be executed by message passing in a factor graph. In fact, nearly all known effective variational inference methods on factorized models can be interpreted as minimization of a so-called ``constrained Bethe Free Energy'' (CBFE) functional \cite{senoz_variational_2021}. In this formulation, posterior variational beliefs are factorized into beliefs over both the nodes and the edges of the graph. It is possible to add constraints to these local beliefs such as requiring that a particular variational posterior is expressed by a Gaussian distribution. In general, CBFE minimization by message passing in a factor graph supports local adaptation of a plethora of constraints to optimize accuracy vs resource consumption. \cite{senoz_variational_2021, akbayrak_extended_2021}   

Useful dynamic models for real-time processing of data streams with a large number of latent variables are necessarily sparsely connected because otherwise, real-time inference would not be tractable. In sparse models, the computational complexity of inference can be vastly reduced by message passing in a factor graph representation of the model. In particular, automated CBFE minimization by message passing in a factor graph supports refined optimization of the accuracy vs resource consumption balance.


\subsection{Reactive vs procedural coding style}\label{sec:Reactive-vs-Procedural}

Next, we discuss a key technological component for a synthetic AIF agent, namely the requirement to execute FE minimization through a \emph{reactive} programming paradigm. 

%This is the real issue that hinders progress in applying autonomous AIF agents to serious design tasks. As previously discussed, the potential advantages of automated algorithm design by AIF agents are extremely impressive, but how are we going to implement real-time FE optimization in a high-dimensional AIF agent that samples the world at a sub-millisecond sampling rate?  

%In order to tackle this huge computational challenge, it is important to realize that high-dimensional models are almost per definition sparse. It is certainly true for the brain and most likely also for any useful high-dimensional model. For instance, a first-order Markov assumption in any dynamic model implies that future states are independent of past states, given the present states. Similarly, in an image or video processing model, pixel coloration almost always depends only on a local neighborhood of pixels. If we were to represent such a sparse model by a graph where independent processing modules\footnote{By independent processing modules we mean modules that do not share any variables.} are not connected by an edge, then variational FE minimization in that graph leads always to an inference algorithm that admits an interpretation as Message Passing (MP) between connected modules on that graph. Formally, a graph representation of a factorized model with missing connections between independent processing modules is called a factor graph, and FE minimization on a factor graph leads to well-known inference algorithms such as belief propagation, variational message passing and expectation propagation \parencite{winn_variational_2005,loeliger_factor_2007, chen_factor_2018, senoz_variational_2021}. 

A crucial feature of all MP-based inference is that the inference process consists entirely of a (parallelizable) series of small steps (messages) that individually and independently contribute to FE minimization. As a result, a message passing-based FE minimization process can be interrupted \emph{at any time} without loss of important intermediate computational results.    

In a practical setting, it is very important that an ongoing inference process can be robustly (without crashing) interrupted at any time with a result. These intermediate inference results can only be reliably retrieved if the inference process iteratively updates its beliefs in small steps, or, in other words, by message passing. Moreover, the inference process should not be subject to a prescribed control flow that contains for-loops. Rather, if we were to write code for an anytime-interruptable inference process in a programming language, we should use a \emph{reactive} rather than the more common \emph{procedural} programming style. In a reactively coded inference engine, there is no code for control flow, such as ``do first this, then that'', but instead only a description of how a processing module (a factor graph node) should react to changes in incoming messages. We will call this process \emph{Reactive Message Passing} (RMP) \cite{bagaevReactiveMessagePassing2023}.  In an RMP inference process, there is no prescribed schedule for passing messages such as the Viterbi or Bellman algorithm. Rather, an RMP inference process just \emph{reacts} by FE minimization whenever FE increases due to new observations. 

In Fig.~\ref{fig:AIF-algorithms}, we display the consequences of choosing a reactive programming style for an application engineer like Sarah. The procedural programming style in Algorithm-1 requires Sarah to provide the control flow (the ``procedure'') for the inference process. Sarah needs to write code for when to collect observations, when to update states, etc. The specific control flow in Algorithm-1 is just an example and there exists literature that aims to improve the efficiency of the control flow \cite{champion_realizing_2021, friston_sophisticated_2021}. In order to write an efficient inference control flow recipe for a complex AIF agent, Sarah needs to be an absolute expert in this field. 

Consider in contrast the code for reactive inference in Algorithm-2. In a reactive programming paradigm, there is no control flow. Rather, the only inference instruction is for the agent to react to any opportunity to minimize FE. When FE minimization is executed by a reactive message passing toolbox, the application engineer only needs to specify the model.

Aside from lowering the competence bar for application engineers to design effective AIF agents, the procedural style of implementing FE minimization is fundamentally inappropriate. The control flow in Algorithm-1 necessarily contains many design choices that only become known during deployment. For instance, how far should the agent roll out its model to the future for computing the EFE? This kind of information is highly contextual and not available to the application engineer. In contrast, the application engineer's code for reactive inference ("react to any FEM opportunity") works for any model in any context. In a reactive inference setting, the appropriate planning horizon is going to be continually updated (inferred) with contextual information. In other words, it is the reactive FEM process itself that leads to optimizing the inference control flow.

% Figure environment removed


%These interrupts may be caused by unintentional failures such as a sensor or internal module crash, or they may be of intentional origin. For example, an edge or node in a graph can be pruned for the purpose of model structure adaptation. Alternatively, an ongoing MP-based FE minimization process may be interrupted because new observations change the FE patterns in the graph in such a way that further iterations of the old FE minimization process are no longer appropriate. 



%\footnote{If brain tissue is not fed by sensory signals, it will die \wk{This is quite a strong statement. Do you have a reference?}\bdv{No, I dont. Is it not true?} since the FE functional, which can be decomposed into ``model complexity minus prediction accuracy'', simplifies to the model complexity term when there is no signal to predict. Minimal model complexity is obtained when there is no processing and hence the brain will die.} 
%The actual algorithm that the biological neural network executes, in the form of firing rates of action potentials, %\wk{An action potential is just a digital $1$ (as opposed to "no spike = $0$"). I would relate messages to "firing rate" or "membrane potential oscillation".}\bdv{OK fair point. phrasing updated}
%depends on sensory signals, which are actively selected by the brain's control signals (like turning the head to look into a different direction), which in turn depend on past sensory inputs etc. In other words, the executed inference algorithm in the brain is a result of interactions between the brain and its environment, and \emph{cannot be predicted from a specification of the brain alone}. 

%In Table~\ref{tab:natural-vs-engineered-design}, we included robustness, real-time processing and low power consumption as distinct advantages of natural design systems over engineered systems. We will shortly discuss how these features emerge as a consequence of using RMP for FE minimization. 

\subsection{RMP for robustness}\label{sec:robustness}

Since an AIF agent executes under situated conditions, it must perform the FE minimization process robustly in real-time.  Consider an agent whose computational resources are represented by a graph and FE minimization results from executing MP-based inference on that graph. Any MP schedule that visits the nodes in the graph in a prescribed fixed order (as would be the case in a procedural approach to FE minimization) is vulnerable to malfunction in any of the nodes in the schedule. In principle, the FE minimization process needs to stop after such a malfunction and proceed to compute a new MP schedule. Since FE minimization is the \emph{only} ongoing computational process, the robot basically moves blindfolded after a reset. Clearly, for robustness, we need a system that continues to minimize FE, even after parts of the graph break down over time. In a reactive inference framework, collapse of a component is simply a switch to an alternative model structure. The new model may perform better or worse at FE minimization, but there is no reason to stop processing. %In fact, in this framework, collapsing a tiny fraction of nodes on an ongoing basis may be considered a model exploration feature. 

\subsection{RMP for real-time, situated processing}\label{sec:real-time-processing}

An ongoing RMP process can always be interrupted when computational resources have run out on a given platform. In this way, by trading computational complexity (i.e., the number of messages) for accuracy, any RMP-based inference process can be scaled down to a real-time processing procedure, where of course a prediction accuracy price may have to be paid, depending on the available computational resources. In short, FE minimization in any model can be executed in real-time on any computational platform if we implement inference by RMP in a factor graph. 

%MP is resilient to adversarial attacks and in principle supports both intentional and unintentional interrupts of the inference process.

%Active inference agents process sensory data in real-time. Perceptual processing relates to Bayesian filtering of sensory data following a Kalman filtering update scheme. 


\subsection{RMP for low power consumption}\label{sec:low-power-consumption}

Similarly, an ongoing RMP process can always be terminated if the expected improvement in accuracy does not outweigh the expected computational load that additional messages would incur.\footnote{The computational load and complexity can only be equated in the absence of a Von Neumann bottleneck (i.e., with mortal computation or in-memory processing). This is because energy and time are ‘wasted’ by reading and writing to memory. \label{fnlabel}} Note that, since FE decomposes as computational complexity minus accuracy, interrupting an RMP-based inference process for this reason is fully consistent with the goal of FE minimization.

%As previously discussed, the FE functional decomposes into a computational cost term (often named the ``complexity'', defined as the KLD between posterior and prior distributions over the latent variables) minus prediction accuracy (i.e., the expected log-likelihood). Minimizing FE therefore always aims to maximize data accuracy for any given amount of processing costs. 
Interrupting an ongoing MP process by any of the above-mentioned reasons (e.g., node malfunction, running out of computational resources, expected processing costs outweighing expected accuracy gains, etc.), in principle always leads to sacrificing some prediction accuracy in favor of saving computational costs. Crucially, these interrupts will not cause a system-wide crash in a reactive system.

%The possibility to interrupt the FE minimization process at any time opens the door to solving another serious engineering dilemma. Many complex algorithms are first developed on powerful desktop computers, followed by a ``porting" phase to the actual low-power product platform. Commonly, porting of a complex engineered algorithm to a platform with much lower computational resources requires deep insights into the algorithm itself, since some modules will have to be deleted or re-designed to behave similarly for lower costs. In contrast, if the algorithm would be designed as an inference task in a probabilistic model, then the same RMP-based declarative inference code would work without modifications on both platforms. On the low power platform, the RMP process would, due to power constraints, get interrupted earlier than the RMP process on the high power platform, leading naturally to a solution with lower accuracy. Hence, the same model with the same declarative code for RMP leads to different solutions for different computational platforms. Importantly, in this framework, \emph{porting algorithms  between platforms with very different computational constraints does not require any human expert knowledge}. 

%As a final comment on low power consumption of natural design methods, we note that RMP-based realizations of AIF agents tend to consume less power as time moves on. In the initial stages of algorithm design, the generative model will not yet be very accurate and lots of large prediction errors will be generated. These prediction errors will generate energy-consuming messages to realize FE minimization over states, parameters and even the model structure. As time moves on, the structural and parameter adaptation processes will converge to accurate-enough models and only state inference in the lower layers will be necessary. In the context of learning how to ride a bike, in the beginning stages, full (power-hungry) attention is needed to learn the model, whereas once we have learned a proper model, riding the bike is an almost unconscious exercise that can easily be combined with other cognitive tasks.   



\section{Model Structure Adaptation}\label{sec:structural-adaptation}

In section~\ref{sec:one-solutiona-approach}, we touched upon the notion that FE minimization should ideally drive the generative model $p$ to evolve to structurally segregated but communicating sub-models that reflect the causal structure of the environment. Technically, this is due to the drive for a lower surprise ($-\log p(x)$).

There is another reason why online structural adaptation is important. Free energy minimization over the structure of $p$ should also lead to a model structure for which inference costs $D_{\text{KL}}[q(z)|| p(z|x)]$ are lower by moving $p(z|x)$ closer to $q(z)$. Consider again the procedural and reactive inference code in Fig.~\ref{fig:AIF-algorithms}. The control flow in the procedural code aims to cleverly steer the inference process toward maximal inference accuracy for minimal computational costs. In contrast, the reactive code just declares that the system should react (by message passing) to any FE minimization opportunity. In the reactive framework, \emph{clever} inference is learned over time by continual minimization over all movable parts of the CBFE, i.e., by FEM over states, parameters, structure (adaptation of $p$), and constraints (adaptation of the structure of $q$). To learn the most effective paths for inference, the toolbox should support structural adaptation over both $p$ and $q$. 

%In short, there are multiple reasons why we should demand that our AIF toolbox supports slow, continual structural adaptation of the generative model. 

Unfortunately, online structural adaptation during the deployment of the robot is still an ongoing research issue, e.g., \cite{friston_bayesian_2018, loeliger_sparsity_2016, beckers_principled_2022}. One technical difficulty is that an efficient inference control flow (which states are updated at what time, etc.) may change if the structure of the generative model changes. In a procedural programming style, we would need to reset the system and reprogram the inference code in Algorithm-1 (in Fig.~\ref{fig:AIF-algorithms}). This is incompatible with the demand that the agent adapts during deployment. As discussed above, a reactive programming style solves this issue since the application inference code (Algorithm-2 in Fig.~\ref{fig:AIF-algorithms}) is independent of the model structure.   


%Hence, we demand that our toolbox supports simultaneous and continual inference over multiple temporal abstraction levels, including fhttps://www.overleaf.com/project/646486ae97ed6e28fbb7798bast updating for states (measured in milliseconds to seconds), slower updating for parameters (seconds to days), and even slower model structure adaptation (days to weeks). %In principle, yet even slower updating of prior preferences can be considered. 


%In principle, we distinguish at least four levels of  in an AIF agent. Inference for internal and active states relates to executing the perceptual and control tasks, respectively. If the model's predictions for sensory inputs are inaccurate, then state inference will also be inaccurate and the intended (perceptual or control) task will not be executed well. An active inference agent corrects this problem automatically by extending FE minimization to the states of the next higher abstraction layer in a hierarchical model structure. This process of extending FE minimization to the states in the next higher layer if prediction errors continue to be present, continues until all layers are actively updating their states. At the lowest layer, state adaptation would likely proceed over a time span of milliseconds, whereas in the highest layer, the adaptation rate would be measured in hours to days. Technically, we often call the states of higher layers the ``parameters'' of the model. If these parameter update processes in the higher layers would still not lead to sufficient model prediction accuracy, then an active inference agent (ought to) improve automatically by extending FE minimization to its generative model structure $p$, for instance by extending the number of layers and columns in the model. In this way, an active inference agent systematically and incrementally improves the performance of its intended task to the desired accuracy levels as time passes. 

%The prior for desired behavior should be less adaptable than the generative model. 
 



\section{Discussion}
\label{sec: discussion}
\kmsdelete{In this work} We study \kmsreplace{Fairness-Aware PAC learning}{Fair-ERM} in the malicious noise model, and  in some cases allow 
the learner to maintain optimal overall accuracy despite the signal in Group $B$ being almost entirely washed out.
%when we allow learners to use the
%$\PQ$ randomized expansion of the hypothesis class $\mathcal{H}$
In particular we show that different fairness constraints have fundamentally different behavior in the presence of Malicious Noise, in terms of the amount of accuracy loss that a given level of Malicious Noise could cause a fairness-constrained learner to incur. 
The key to achieving our results, which are more optimistic than those in \cite{lampert}, is allowing for improper learners using the (P,Q)-randomized expansions of the given class $\mathcal{H}$.
%We \kmsreplace{present a picture of the}{prove upper and lower bounds on}
%accuracy loss for a range of fairness notions, given \kmsreplace{this simple randomization step.}{learning over $\PQ$.
%In general our results indicate Fair-ERM (given learning over $\PQ$) is more robust than claimed in \cite{lampert}.
The type of smoothness we create by using $\PQ$ seems to be a natural property that is likely shared by many natural hypothesis classes.

Fairness notions are motivated as a response to learned disparities when there is \kmsdelete{data corruption or} systemic error affecting \kmsdelete{the data for}
one group. 
Fairness notions are supposed to mitigate this by ruling out classifiers that have worse performance on a sub-group. 
This can peg both classifiers at a lower level of performance \kmsdelete{(e.g that the lower subgroup)} in order to \emph{motivate} \cite{hardt16} improving the data collection or labelling process to obtain more reliable performance. 
%So in \kmsreplace{some}{a} sense, sensitivity of the fairness notion to poor sub-group performance caused by malicious noise is the \textit{point} of fairness constraints! 
However, it also desirable that fairness constraints perform gracefully when subject to Malicious Noise because fairness constraints will be used in contexts where the data is unreliable and noisy and this might not be known to the learner.
This tension, exposed by our work, motivates 
%a revisiting of fairness notions from first principles approach and trying to axiomatize the 
%desired properties of a fairness intervention a la cryptography and privacy. \footnote{Work in multi-calibration \cite{multicalib} is a viable direction for this problem but it is unclear how 
%that and related notions behave with unreliable data. }
on going work studying the sensitivity level of fairness constraints. 
%If we we are to take a view, if a classifier is deployed 

%% -*- mode: LaTeX; fill-column: 78; -*-

\section{Concluding Remarks}
\label{sec:conclusions}

In this paper, we presented a novel SMC algorithm, \EventDPOR, tailored to the
characteristics of event-driven multi-threaded programs running under the SC
semantics. The algorithm was proven correct and optimal for event-driven
programs in which the variable accesses of events do not depend on how their
execution is interleaved with other threads.

We have implemented \EventDPOR in the \Nidhugg tool, and we will open-source
our implementation.
%
With a wide range of event-driven programs, we have shown that \EventDPOR
incurs only a moderate constant overhead over its baseline implementation
(\OptimalDPOR), it is exponentially faster than existing state-of-the-art SMC
algorithms in time and number of traces examined on programs where events'
actions do not conflict, and does not suffer from performance degradation
caused by having to examine
% a significant number of
non-serializable executions.
%
%% \bjcom{Should we include:
%% Moreover, in our benchmarks, also those that are not non-branching,
%% \EventDPOR explores only the optimal number of executions, and never
%% had to resort to a potentially expensive decision procedure.}

\EventDPOR assumes that handlers can process their events in arbitrary order.
Directions for future work include to retarget \EventDPOR for event-driven
programs with other policies (e.g., FIFO), and for specific event-driven
execution models.

This work was partially supported by CONICET (Argentina) (PUE 0015-2016), by the Santa Fe province (Argentina) Government under Grant PEICID-2021-170, by the Spanish Government under Grants PGC2018-096367-B-I00 and
PID2021-127685NB-I00 and by the Aragon Government under Grant DGA T45 17R/FSE.
\printbibliography

\end{document}
