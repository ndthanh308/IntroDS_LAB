\documentclass[11pt]{amsart}
\usepackage[T1]{fontenc}
\numberwithin{equation}{section}

%Package block 

%\usepackage{natbib}
\usepackage{float}
\usepackage{tikz-cd}
\usepackage{amsmath}%
\usepackage{amsfonts}%
\usepackage{amsthm}
\usepackage{amssymb}%
\usepackage{graphicx}
%\usepackage{titling}
\usepackage{hyperref}


%\usepackage{natbib}
\usepackage{float}
\usepackage{tikz-cd}
\usepackage{amsmath}%
\usepackage{amsfonts}%
\usepackage{amsthm}
\usepackage{amssymb}%
\usepackage{graphicx}
%\usepackage{titling}
\usepackage{hyperref}
\usepackage{enumerate}
\usepackage{comment}



\usepackage{geometry}
%\usepackage{a4wide}
\usepackage{layout}
\usepackage[utf8]{inputenc}
\usepackage{amsmath}
\usepackage{graphicx}
\usepackage{amsmath}
\usepackage{amssymb}
\usepackage{amsfonts}
\usepackage{graphicx}
\usepackage{url}
\usepackage{MnSymbol}
\usepackage{arydshln}
\usepackage{outlines}
\usepackage{tikz-cd}
\newcommand{\gp}{\mathfrak{p}}
\usepackage{subfig}
\usepackage{amsthm}

\newcommand{\Fbar}{\overline{\mathbb{F}}}
\newcommand{\gc}{\mathfrak{c}}
\newcommand{\gm}{\mathfrak{m}}
\newcommand{\gr}{\mathfrak{r}}
\newcommand{\Z}{\mathbb{Z}}
\newcommand{\Q}{\mathbb{Q}}
\newcommand{\CC}{\mathbb{C}}
\newcommand{\cA}{\mathcal{A}}
\newcommand{\cH}{\mathcal{H}}
\newcommand{\bZ}{\overline{Z}}
\newcommand{\cM}{\mathcal{M}}
\newcommand{\cS}{\mathcal{S}}
\newcommand{\cJ}{\mathcal{J}}
\newcommand{\F}{\mathbb{F}}
\newcommand{\PGL}{\mathrm{PGL}}
\renewcommand{\O}{\mathcal{O}}
\renewcommand{\epsilon}{\varepsilon}
\newcommand{\R}{\mathbb{R}}
\renewcommand{\P}{\mathbb{P}}
\renewcommand{\H}{\mathbb{H}}
\newcommand{\cMbar}{\overline{\mathcal{M}}}
\renewcommand{\cJ}{\mathcal{J}}
\newcommand{\Jac}{\mathcal{J}}
\newcommand{\codim}{\mathrm{codim}}
\newcommand{\dirlim}{\varinjlim}
\newcommand{\invlim}{\varprojlim}
\renewcommand{\o}{\varnothing}
\newcommand{\Dmod}{Dieudonn\'e module}
\renewcommand{\o}{\varnothing}
\newcommand{\Car}{\mathcal{C}}
\newcommand{\Cl}{\mathrm{Cl}}
\newcommand{\GL}{\mathrm{GL}}
\newcommand{\Pic}{\mathrm{Pic}}
\newcommand{\Spec}{\mathrm{Spec}}
\newcommand{\Stab}{\mathrm{Stab}}
\newcommand{\Orbit}{\mathrm{Orbit}}
\newcommand{\rank}{\mathrm{rank}}
\newcommand{\coker}{\mathrm{coker}}
\newcommand{\im}{\mathrm{im}}
\newcommand{\Div}{\mathrm{Div}}
\newcommand{\Aut}{\mathrm{Aut}}
\newcommand{\End}{\mathrm{End}}
\newcommand{\Hom}{\mathrm{Hom}}
\newcommand{\anum}{a\mathrm{-num}}
\newcommand{\prank}{p\text{-}\rank}
\newcommand{\tworank}{2\text{-}\rank}
\newcommand{\dd}{\mathrm{d}}
\usepackage{multirow}


\newcommand{\floor}[1]{\lfloor #1 \rfloor}
\newcommand{\blokje}{\hfill $\Box$\\}
\newcommand{\<}{\left \langle}
\renewcommand{\>}{\right \rangle}

\usepackage{xcolor}

%Preamble

\newtheorem{thm}{Theorem}[section]

\newtheorem{thmx}{Main Theorem}
%\renewcommand{\thethmx}{\Alph{thmx}}

\newtheorem{lem}[thm]{Lemma}
\newtheorem{prop}[thm]{Proposition}
\newtheorem{cor}[thm]{Corollary}
\newtheorem{conj}[thm]{Conjecture}
\newtheorem{que}[thm]{Question}


\theoremstyle{definition}
\newtheorem{dfn}[thm]{Definition}
\newtheorem{ntn}[thm]{Notation} 
\newtheorem{ass}[thm]{Assumption}
\newtheorem{cons}[thm]{Consequence}

\theoremstyle{remark}
\newtheorem{rem}[thm]{Remark}
\newtheorem{exmp}[thm]{Example}


\begin{document}


\author{Du\v san Dragutinovi\'c}
\keywords{Curves, Ekedahl-Oort type, $\prank$, $a$-number}
\address{ Mathematical Institute, Utrecht University,
P.O. Box 80010, 3508 TA Utrecht, The Netherlands}
\email{d.dragutinovic@uu.nl}

\title[Ekedahl-Oort types of stable curves]{Ekedahl-Oort types of stable curves}

\maketitle

\begin{abstract}
We extend Moonen's definition of Ekedahl-Oort types of smooth curves and abelian varieties in terms of Hasse-Witt triples to all stable curves and semi-abelian varieties and show that it matches Ekedahl and van der Geer's definition of Ekedahl-Oort types of semi-abelian varieties. Using this intrinsic insight, we can compute the dimensions of certain Ekedahl-Oort loci of curves and generalize some previously known results about the dimensions of the $\prank$ and $a$-number loci of curves.
\end{abstract}

\section{Introduction}

Suppose $k$ is an algebraically closed field of characteristic $p>0$ and $g\geq 2$. 
For a principally polarized $g$-dimensional abelian variety $A$ over $k$, one can define the \textit{$\prank$} of $A$ as $$\prank(A) = \dim_{\F_p}\Hom(\mu_p, A[p]),$$ while the \textit{$a$-number} of $A$ is $$a(A) = \dim_k\Hom(\alpha_p, A[p]), $$ where $\mu_p$ and $\alpha_p$ are the kernels of the Frobenius morphism on respectively the multiplicative group $\mathbb{G}_m$ and the additive group $\mathbb{G}_a$.  The isomorphism class of the $p$-torsion group scheme $A[p]$ bijectively corresponds to some Young diagram $$\mu = [\mu_1, \mu_2, \ldots, \mu_n], $$ with $g\geq \mu_1 > \ldots > \mu_n > 0$. We call the $\mu$ associated to $A$ the \textit{Ekedahl-Oort type} of $A$ and write $\mu = \mu(A)$.
It turns out that one can read off the $\prank$ and the $a$-number of $A$ from $\mu$ as $\prank(A) = g - \mu_1$ and $a(A) = n$. 
 For a smooth curve $C$, we can define these invariants in terms of the ones of its Jacobian $\Jac_C$, i.e., $\prank(C) = \prank(\Jac_C)$, $a(C) = a(\Jac_C)$, and $\mu(C) = \mu(\Jac_C)$. In this note, we explain how one can define the Ekedahl-Oort type of stable curves of genus $g$ over $k$ in an intrinsic way and use that to describe the geometry of Ekedahl-Oort loci of stable curves.  In that way, we can generalize the previously known results about the $\prank$ and $a$-number loci of curves and produce theoretically new ones.

Let $\cMbar_g$, $\cM^{ct}_g$, and $\cM_g$ be the moduli spaces of respectively stable genus-$g$ curves, stable genus-$g$ curves of compact type (i.e., stable genus-$g$ curves whose dual graph is a tree), and smooth genus-$g$ curves; we have the following inclusions $$\cMbar_g \supset \cM^{ct}_g\supset \cM_g.$$ 
The boundary $\cMbar_g - \cM_g$ consists of the components $\Delta_0, \Delta_1, \ldots, \Delta_{\left \lfloor \frac{g }{2} \right \rfloor}$ and $\cM_g^{ct}$ equals $\cMbar_g - \Delta_0$. The generic point of $\Delta_0$ corresponds to an irreducible curve $C_0$ with an ordinary double point, whose normalization $\Tilde{C}_0$ is a genus-$(g - 1)$ curve, while the generic point of $\Delta_i$ corresponds to a reducible curve with components of genus $i$ and $g - i$ for any $1\leq i \leq \left \lfloor \frac{g }{2} \right \rfloor$.  

Let $\cA_g$ be the moduli space of principally polarized $g$-dimensional abelian varieties. Attaching to a genus-$g$ stable curve of compact type $C$ its Jacobian variety $\Jac_C$ induces the Torelli morphism $j: \cM^{ct}_g \to \cA_g$. Lastly, we consider a fixed smooth toroidal compactification $\Tilde{\cA}_g$ of $\cA_g$, for which the Torelli morphism extends, e.g, as in \cite[Theorem 4.1]{alexeev}, to 
\begin{equation}
 j: \cMbar_g \to \Tilde{\cA}_g.
 \label{eqn:torelli}
\end{equation}


An equivalent way to define the Ekedahl-Oort type of a principally polarized abelian variety $A$ (or a smooth curve) to the above-mentioned is via Hasse-Witt triples as Moonen did in \cite{moonen}. In \cite{ekedahlvdgeer}, Ekedahl and van der Geer define the Ekedahl-Oort types of semi-abelian varieties, and one can define the Ekedahl-Oort type of a stable curve $C$ as one of its (generalized) Jacobian, i.e., $\mu(C) = \mu(\Jac_C)$. In Section \ref{sec:bdry}, for an arbitrary stable curve (or a finite union of smooth curves) $C$, we analyze the action of the Frobenius operator 
\begin{equation}
F: H^{1}(C, \O_C) \to H^{1}(C, \O_C)    
\label{eqn:Frob}
\end{equation} and observe that we can extend Moonen's definition to any such $C$. This gives us an intrinsic and equivalent way to define the Ekedahl-Oort type of $C$. We show the following.

\begin{thmx}(Theorem \ref{thm:muCnorm})
Let $h: \Tilde{C} \to C$ be the normalization of a stable curve $C$ defined over an algebraically closed field $k$ of characteristic $p>0$. The definition of the Ekedahl-Oort type of $C$ via the Hasse-Witt triples extends the definition for smooth curves from Section \ref{sec:eo_types} and  we have  $$\mu(C) = \mu(\Tilde{C}).$$ 
Furthermore, under the Torelli morphism $j: \cMbar_g \to \Tilde{\cA}_g$, sending a stable genus-$g$ curve to its (generalized) Jacobian, we get $\mu(C) = \mu(\Jac_C)$.
\end{thmx}


For any Young diagram $\mu = [\mu_1, \ldots, \mu_n]$ with $g\geq \mu_1$, we denote by $Z_{\mu}$ the locus of principally polarized abelian (resp. semi-abelian) varieties $A$ with $\mu(A) = \mu$ and by $\bZ_{\mu}$ its closure in $\cA_g$ (resp. in $\Tilde{\cA}_g$). This introduces the Ekedahl-Oort stratification of $\cA_g$ (resp. of $\Tilde{\cA}_g$); see \cite[Proposition 12.5]{oort} (resp. \cite[Section 5]{ekedahlvdgeer}). We can define the Ekedahl-Oort stratification of $\cMbar_g$ as the pullback of the one on $\Tilde{\cA}_g$ via the Torelli morphism \eqref{eqn:torelli}; we denote the corresponding Ekedahl-Oort strata $Z_{\mu}\cMbar_g$ (and by $\bZ_{\mu}\cMbar_g$ their closures). On the set of all Young types, there is a partial order relation $\leq$ as we explain in Section \ref{sec:eo_types}. Due to our intrinsic description of that stratification, in Section \ref{sec:inductiveres}, we get the following result, which generalizes \cite[Theorem 2.3]{fabervdgeer} and  \cite[Proposition 3.7]{pries_a_number}.

\begin{thmx}(Theorem \ref{thm:main1})
Let $\mu = [\mu_1, \ldots, \mu_n]$ and $g \geq \mu_1$ be such that $\bZ_{\mu} = \cup_{\mu'\leq \mu} Z_{\mu'}$ in $\cA_g$. Assume that $\bZ_{\mu}\cMbar_{g}$ is pure of codimension $d$ in $\cMbar_g$ and that
$\Gamma$ is a component of $\bZ_{\mu}\cMbar_{g + 1}$ that meets $\Delta_0$ (e.g., if $d \leq g$). Then $\Gamma$ is of codimension at least $d$ in $\cMbar_{g + 1}$.
\end{thmx}


\subsection*{Acknowledgment}
The author is grateful to his supervisor Carel Faber for all the discussions and valuable comments and to Valentijn Karemaker for the feedback and helpful remarks.  The author is supported by the Mathematical Institute of Utrecht University. 

\section{Ekedahl-Oort type of abelian varieties}
\label{sec:eo_types}
In this section, we first recall the definition of the Ekedahl-Oort type of an arbitrary principally polarized abelian variety following \cite{oort} and \cite{vdgeercycle}. Then, we present a traditional definition of the Ekedahl-Oort type of a smooth curve via its Jacobian and using the Hodge-de Rham short exact sequence. Finally, we describe everything by means of Moonen's work in \cite{moonen}, which offers an intrinsic way of defining the Ekedahl-Oort type of smooth curves.

Let $A$ be a $g$-dimensional principally polarized abelian variety over an algebraically closed field $k$ of characteristic $p>0$, and let $\sigma: k \to k$ be the Frobenius morphism on $k$. Its first de Rham cohomology $H^1_{dR}(A)$ is equipped with $\sigma$- and $\sigma^{-1}$- linear operators $F$ and $V$, such that $FV = VF = 0$ and they are adjoints with respect to the symplectic form $\<, \>$ on  $H^1_{dR}(A)$, i.e., $\<V\omega, \omega'\> = \<\omega, F\omega'\>.$ For any subspace $N \subseteq H_{dR}^1(A)$, it holds that $(VN)^{\perp} = F^{-1}(N^{\perp})$. Starting from the filtration $0 \subset H^1_{dR}(A)$ and adding the repeated images of $V$ as well as their orthogonal complements gives us the so-called \textit{canonical filtration}
$$0 \subset N_1' \subset \ldots \subset N_r' \subset N_{r + 1}' \subset \ldots \subset N_{2r}' = H^1_{dR}(A)$$
which stabilizes, and by construction satisfies $N_r' = V(N_{2r}')$ and $N_{r-i}'^{\perp} = N_{r+i}'$ for any $i$. One can further refine this to a \textit{final filtration} by choosing a maximal filtration that is stable under $V$ and $\perp$,
$$0 \subset N_1 \subset \ldots \subset N_g \subset N_{g + 1} \subset \ldots \subset N_{2g} = H^1_{dR}(A), $$
i.e., such that $V(N_{2g}) = N_g$ and $N_{g-i}^{\perp} = N_{g+i}$ for all $i$. Then one introduces the \textit{final type} of $A$ as the increasing surjective map $\{0, 1, \ldots, 2g\} \to \{0, 1, \ldots, g\}$ defined by 
\begin{equation}
\nu(i) = \dim V(N_i) 
\label{eqn:nu_def}
\end{equation}
and $\nu(0) = 0$ satisfying $\nu(2g - i) = \nu(i) - i + g$ for $0\leq i\leq g$. Using these properties, we can see that $\nu$ is determined by its values $\nu(i)$ for $1\leq i\leq g$, and hence, we will write it as $\nu = (\nu(1), \ldots, \nu(g))$. The final type of $A$ is unique. Considering all $A$ in $\cA_g$ such that $\nu(A) = \nu$ defines the locally closed subsets $Z_{\nu}$ of $\cA_g$ called the Ekedahl-Oort strata.  An alternative combinatorial description of the same property uses Young diagrams $\mu = \mu(A) = [\mu_1, \mu_2, \ldots, \mu_n]$ with $g\geq \mu_1 > \ldots \mu_n > 0$ defined in terms of $\nu(A)$ by $$\mu_j = \#\{i: 1\leq i\leq g \text{ and } \nu(i)\leq i - j\}.$$ 
By the Ekedahl-Oort type of $A$, we mean either its final type $\nu = \nu(A)$ or its Young diagram $\mu = \mu(A)$, and for such $\nu$ and $\mu$, we write $Z_{\mu} = Z_{\nu}$.  We can read off the $\prank$ and the $a$-number of $A$ from $\mu(A)$ as $\prank(A) = g - \mu_1$ and $a(A) = n$.  It is well known by \cite[Corollary 11.2]{oort} that the pure codimension of $Z_{\mu}$ in $\cA_g$ equals $\sum_{i = 1}^n \mu_i$, and by \cite[Proposition 11.1]{oort}, if we introduce a partial ordering by $$\mu = [\mu_1, \ldots, \mu_n] \geq \mu'= [\mu'_1, \ldots, \mu'_m] \text{ if } n\leq m \text{ and } \mu_i \leq \mu_i' \text{ for all } 1\leq i\leq n, $$ we get 
\begin{equation}
\mu' \leq \mu \quad \implies Z_{\mu'} \subseteq \bZ_{\mu}.
\label{eq:finaltypes_EOclosure}    
\end{equation}
We note this does not depend on $g$ as long as $g  \geq \max\{\mu_1, \mu_1'\}$. Furthermore, note that the reverse implication  does not hold in general; see \cite[14.3]{oort}.  


Now, let $C$ be a smooth curve. By, e.g., \cite[Proposition 3.1]{kocktait}, we have the Hodge-de Rham short exact sequence $$0 \to H^0(C, \Omega_C^1) \to H_{dR}^1(C) \to H^{1}({C, \O_C}) \to 0, $$
and the operator $V$ on $H_{dR}^1(C)$ essentially acts as the Cartier operator on $ H^0(C, \Omega_C^1)$. By that, we mean that $V(H_{dR}^1(C)) = H^0(C, \Omega_C^1)$ and $V\mid_{H^0(C, \Omega_C^1)} = \Car$. Using the preceding definitions, we can naturally define the Ekedahl-Oort type of such a $C$ using $V$ and $\perp$. 



Another approach to describe the Ekedahl-Oort type can be found in  Moonen's work \cite{moonen}. Note that the Dieudonn\'e  module of $A[p]$ equals $\mathbb{D}(A[p]) = H^1_{dR}(A)$, where $A$ is a $g$-dimensional principally polarized abelian variety; we also write $\mathbb{D}_A = \mathbb{D}(A[p])$. Therefore, an equivalent way to define the Ekedahl-Oort type of $A$ is as the isomorphism class of its Dieudonn\'e module $(M, F, V, b), $
where \begin{itemize}
    \item $M = \mathbb{D}(A[p])$ is a $2g$-dimensional $k$-vector space;
    \item $F: M \to M$ is a $\sigma$-linear operator, while $V: M \to M$ is a $\sigma^{-1}$-linear operator;
    \item $\ker(F) = \im(V)$ and $\ker(V) = \im(F)$;
    \item $b: M\times M \to k$ is a polarization, i.e., a non-degenerate alternating bilinear form such that $b(F(x), y) = b(x, V(y))^p$ for all $x, y \in M$.
\end{itemize}
An important notion that Moonen introduced is the \textit{Hasse-Witt triple} over a perfect field $K$ of characteristic $p>0$, which is a triple $(Q, \Phi, \Psi)$ such that \begin{itemize}
    \item $Q$ is a finite-dimensional $K$-vector space;
    \item $\Phi: Q \to Q$ is a $\sigma$-linear map;
    \item $\Psi: \ker(\Phi) \overset{\cong}{\to} \im(\Phi)^{\perp}$ is a $\sigma$-linear bijective map; 
\end{itemize} 
where $\im(\Phi)^{\perp} = \{\lambda \in Q^{\vee}: \lambda(q) = 0 \text{ for all }q \in \im(\Phi)\}$ and $\sigma: K \to K$ is the Frobenius morphism on $K$. In practice, we choose $K$ to be $k$ as above.
In \cite[Theorem 2.8]{moonen}, Moonen shows the following result. 

\begin{prop}
Let $K$ be a perfect field of characteristic $p$. Then, there is a bijection $$\left\{ \begin{matrix} 
\text{ isomorphism classes of }\\ (M, F, V, b),\text{ } \dim_K(M) = 2g \end{matrix} \right \} \longleftrightarrow \left\{ \begin{matrix} 
\text{ isomorphism classes of }\\ (Q, \Phi, \Psi),\text{ } \dim_K(Q) = g \end{matrix} \right \}, $$
given by $(M, F, V, b) \mapsto (Q, \Phi, \Psi)$ where $Q = M/\ker F$, $\Phi = \Bar{F}:  Q\to Q$ is the map induced by $F$, and $\Psi(x) = b(-, F(x))$.
\label{prop:moonen}
\end{prop}
\begin{proof}
This is \cite[Theorem 2.8]{moonen}. See \cite[2.5 and 2.6]{moonen} for constructing this bijection and its inverse.    
\end{proof}

For a smooth curve $C$ over $k$, we can take $Q$ to be $H^1(C, \O_C)$, $\Phi = F_C$ the Frobenius operator on $H^1(C, \O_C)$, and take $\Psi$ as above to define its Ekedahl-Oort type via the Hasse-Witt triple $(Q, \Phi, \Psi)$ using the equivalence from Proposition \ref{prop:moonen}. 

\section{On Hasse-Witt partitions}
\label{sec:hwpart}
Let $k$ be an algebraically closed field of characteristic $p>0$, and write $\sigma: k \to k$ for the Frobenius morphism on $k$. Given a $g\times g$ matrix $H \in \mathrm{Mat}_{g\times g}(k)$ let us denote $$\rho_i = \rho_i(H) = \rank_k (H\cdot H^{\sigma}\cdot \ldots \cdot H^{\sigma^{i-1}}), $$ for any $i\geq 0$ (where $\rho_0 = g$). Motivated by \cite[Satz 11]{hassewitt}, we define $\delta_i = \rho_{i-1} - \rho_i$ for any $i\geq 1$. Note that $\delta_1\geq \delta_2\geq \ldots \delta_g \geq 0$. If $\delta_n \neq 0, \delta_{n+1} = 0$, we say that the partition $$\delta = \delta(H) = \delta_1\ldots \delta_n$$ is the Hasse-Witt partition associated with $H$. 
In the rest of this section, to the objects parametrized by $\cA_g$ and $\cMbar_g$, we will associate the partitions $\delta$ just as above, which we call the \textit{Hasse-Witt partition} $\delta$. Then, in the next section, following \cite[Section 5]{ekedahlvdgeer}, we consider a natural extension of this definition to $\Tilde{\cA}_g$ and observe that it behaves well with respect to $j: \cMbar_g \to \Tilde{\cA}_g$. 


\subsection{On Hasse-Witt partitions and $\cA_g$}
\label{subsection:hwAg}
Let $\delta = \delta_1\ldots\delta_n$, with $\delta_1\geq \delta_2\geq \ldots \geq \delta_n >0$, be a partition of $g - f$ into $n$ parts, for $g\geq 1$ and $0\leq f\leq g$, and $\rho_j$ for $0\leq j \leq n$ such that $$  \rho_{j - 1} - \rho_{j} = \delta_j \quad \text{ and } \quad \rho_0 = g;$$ explicitly, $\rho_j = f + \sum_{i = j + 1}^n \delta_i$. Consider a $g$-dimensional principally polarized abelian variety $A$ whose final type $\nu$  satisfies 
\begin{equation}
\nu(f + \sum_{i = j}^n\delta_i) = f + \sum_{i = j + 1}^n\delta_i,
\label{eq:deltanu}
\end{equation}
for any $j = 1, \ldots, n+1$; in other words, the final type of $A$ is of the form: \begin{equation*}
     (\underbrace{1, 2, \ldots, f}_{f}, \underbrace{*, \ldots, *, f}_{\delta_n}, \underbrace{*, \ldots, *, f + \delta_n}_{\delta_{n-1}}, \underbrace{*, \ldots, *, f + \sum_{i = n-1}^n\delta_i}_{\delta_{n-2}}, \ldots,  \underbrace{*, \ldots, *, f + \sum_{i = 2}^n \delta_i}_{\delta_1}).
\end{equation*} 

We say that a $g$-dimensional principally polarized abelian variety $A$ has \textit{Hasse-Witt partition} $\delta = \delta_1\ldots\delta_n$ if its final type $\nu = \nu(A)$ satisfies \eqref{eq:deltanu}. In other words, writing $N_g = V(H^{1}_{dR}(A))$ and using \eqref{eqn:nu_def}, we see that $$\delta(A) = \delta_1\ldots\delta_n \Longleftrightarrow \dim_k V^{j}(N_g) = \rho_j \text{ for all }0\leq j\leq n.$$

In Table \ref{tab:HWEO}, we show how the Hasse-Witt partitions $\delta$ of abelian varieties correspond to their Ekedahl-Oort types $\mu$ in some simple cases. For $g\geq 3$, not every Ekedahl-Oort type $\mu$ of a $g$-dimensional abelian variety $A$ can be completely determined only in terms of $\rank_k V^{j}|_{N_g}$ for $0\leq j \leq g$. 

\begin{table}[H]
    \centering
    \begin{tabular}{cc}
    \begin{minipage}{.4\linewidth}
        \begin{tabular}{ |c|c|}
    \hline
H-W type $\delta$ & E-O type(s)  $\mu$\\

 \hline
 $\delta = 1$ & $\mu = [1]$\\
 \hline
 \hline
 $\delta = 11$ & $\mu = [2]$\\
 \hline
 $\delta = 2$ & $\mu = [2, 1]$\\
 \hline
 \hline
 $\delta = 111$ & $\mu = [3]$\\
 \hline
 $\delta = 21$ & $\mu = [3, 1]$ or $[3, 2]$\\
 \hline
 $\delta = 3$ & $\mu = [3, 2, 1]$\\
 \hline
 
 \end{tabular}
    \end{minipage} &

    \begin{minipage}{.5\linewidth}
        \begin{tabular}{ |c|c|}
    \hline
H-W type $\delta$ & E-O type(s)  $\mu$\\
 \hline
 $\delta = 1111$ & $\mu = [4]$\\
\hline
 $\delta = 211$ & $\mu = [4, 1]$ or $[4, 2]$\\
 \hline
 $\delta = 22$ & $\mu = [4, 3]$\\
 \hline
 $\delta = 31$ & $\mu = [4, 2, 1], [4, 3, 1],$ or $[4, 3, 2]$\\
 \hline
 $\delta = 4$ & $\mu = [4, 3, 2, 1]$\\
 \hline
  \end{tabular}
    \end{minipage} 
\end{tabular}

    \caption{Correspondence between some Hasse-Witt and Ekedahl-Oort types}
    \label{tab:HWEO}
\end{table}



\subsection{On Hasse-Witt partitions and $\cMbar_g$}
\label{subsection:hwMg}
For $C$ either a stable curve or a union of smooth curves over $k$, let $F$ be the Frobenius operator on $H^{1}(C, \O_C)$ as in \eqref{eqn:Frob} and let $\rho_i = \rho_i(C)$  be the rank (over $k$) of $F^i = F^{\circ i}$ for any $i\geq 1$ and $\rho_0 = \rho_0(C) = \dim_k H^{1}(C, \O_C)$. We define the \textit{Hasse-Witt partition} $\delta$ of $C$ as $$\delta(C) = \delta_1\ldots\delta_n, $$ where $\delta_i = \rho_{i - 1} - \rho_i$ for $1\leq i \leq g$ and $n$ such that $\delta_n \neq 0$ and $\delta_{n + 1} = 0$. 

Let us now consider the case when $C$ is a smooth curve of genus $g$ over $k$ and show that, in this case, the notions introduced here and in \ref{subsection:hwAg} match, i.e., that $\delta(C) = \delta(\Jac_C)$. Denote by  $H = H_C = [F]$ its Hasse-Witt matrix, i.e., the matrix of Frobenius $F$ on $H^1(C, \O_C)$ and $M = [\Car]$ its Cartier-Manin matrix, i.e., the matrix of the Cartier operator $\Car$ acting on $H^0(C, \Omega^1_C)$ as in Section \ref{sec:eo_types}. The matrix of $F^m$ on $H^1(C, \O_C)$ is given by $$H^{(m)} = H\cdot H^{\sigma} \ldots H^{\sigma^{m -1}},$$ where $H^{\sigma^{i}}$ is the matrix obtained by raising each of the entries of $H$ to the power $p^{i}$. If we denote $\rho_m = \rho_m(C) = \rank_k H^{(m)}$ for $1\leq m \leq g$, and $\delta_i = \rho_{i - 1} - \rho_i$ for $0\leq i\leq g$, then the Hasse-Witt partition of $C$ equals $$\delta(C) = \delta_1\ldots\delta_n,$$ with $n$ such that $\delta_n \neq 0$ and $\delta_{n + 1} = 0$.  
Since the operators $F$ and $\Car$ are adjoint, we have $$M = (H^{\sigma^{-1}})^t \quad \text{and}\quad H = (M^{\sigma})^t;$$ for more details see \cite{achterhowe}. Similarly, if $H^{(i)} = [\mathcal{F}^{\circ i}] = H\cdot H^{\sigma}\cdot \ldots\cdot H^{\sigma^{i-1}}$ and $M^{(i)} = [\Car^{\circ i}]$, then $$M^{(i)} = ({H^{(i)}}^{(\sigma^{-1})^i})^t\quad \text{ and }\quad H^{(i)} = ({M^{(i)}}^{\sigma^i})^t.$$ In particular, for any $i$ we have $$\rank_k M^{(i)} = \rank_k H^{(i)}.$$ Therefore, the partition $\delta$ associated with $C$ is exactly the one associated with $\Jac_C$, so the introduced notions match. 




\section{Going to the boundary}
\label{sec:bdry}
In this section, we first define the Hasse-Witt partition and then the Ekedahl-Oort type of an arbitrary semi-abelian variety or an arbitrary stable curve in terms of Hasse-Witt triples. We carefully prove that these definitions are equivalent to the ones in the literature.   

\subsection{Hasse-Witt partitions of stable curves}

Let first $A$ be a semi-abelian variety of dimension $g$ that is an extension of an abelian variety $A_0$ by a torus of toric rank $l$. In \cite[Section 5]{ekedahlvdgeer}, Ekedahl and van der Geer defined the final type $\nu_A$ of $A$ by $$\nu_A(i + l) = \nu_{A_0}(i) + l, \quad \text{for any}\quad  0\leq i\leq g - l.$$ Essentially, this introduces the Ekedahl-Oort loci on $\Tilde{\cA}_g$ that they study. Here, we define the Hasse-Witt partition of $A$ by $$\delta(A) = \delta(A_0), $$ so that the definition naturally extends the one for abelian varieties and fits into Ekedahl and van der Geer's description. Further, recall that in \ref{subsection:hwMg} we introduced $\delta(C)$ for $C$ either a stable curve or a finite union of smooth curves. 

Now, let $C$ be a stable curve with $n$ irreducible components and $m$ singular points, and let $h: \Tilde{C} \to C$ be its normalization. Similarly as in \cite{bouw}, we get the short exact sequence of sheaves $$0 \to \O_C \to h_{*}\O_{\Tilde{C}} \to Q \to 0, $$ with $Q = h_{*}\O_{\Tilde{C}}/\O_C $ the quotient sheaf; note that $Q$ is constant and supported at the nodes of $C$. The maps above are given by $$f \mapsto f\circ h \quad \text{and}\quad f' \mapsto (f'(P_1) - f'(P_2))_Q,$$ for  $P_1$ and $P_2$ the points over a node $Q$ of $C$. Taking the long exact sequence in cohomology, we get $$0 \to H^0(C, \O_C) \to H^0(\Tilde{C}, \O_{\Tilde{C}}) \to H^{0}(C, Q) \to H^1(C, \O_C) \to H^1(\Tilde{C}, \O_{\Tilde{C}}) \to H^{1}(C, Q) = 0. $$ Since $H^0(C, \O_C) = k$, $H^0(\Tilde{C}, \O_{\Tilde{C}}) = k^n$ and $H^{0}(C, Q) = k^m$, and noting that $l = 1 - n + m$ is the number of loops in $C$, we get the short exact sequence 
\begin{equation}
0 \to k^l \to H^1(C, \O_C) \to H^1(\Tilde{C}, \O_{\Tilde{C}}) \to 0,    
\label{eq:shortseq}
\end{equation}
which gives us the isomorphism of $k$-vector spaces $H^1(C, \O_C) \cong k^l\oplus H^1(\Tilde{C}, \O_{\Tilde{C}})$. Even more is true: consider the following diagram
\begin{center}
\begin{tikzcd}
0\arrow[r] & \O_C \arrow[r] \arrow[d, "F"] &  h_{*}\O_{\Tilde{C}} \arrow[r] \arrow[d, "F"] &  Q \arrow[r] \arrow[d, "F"] &  0\\
0\arrow[r] & \O_C \arrow[r]  &  h_{*}\O_{\Tilde{C}} \arrow[r]  &  Q \arrow[r]  &  0
\end{tikzcd}
\end{center}
and let us show that all the squares in it commute. Once we consider the stalks, we see that $F$ commutes with the integral closure for the left square since both $\O_{C, P}$ and $(h_*\O_{\Tilde{C}})_P$ are subrings of $k(C)$, and with the evaluation for the right one. Therefore, we obtain that 
\begin{equation}
H^1(C, \O_C) \cong k^l\oplus H^1(\Tilde{C}, \O_{\Tilde{C}})    
\label{eq:kFmods}
\end{equation}
is an isomorphism of $k[F]$-modules so that the iterations of the Frobenius respect sequence \eqref{eq:shortseq}. Since the Frobenius acts as a bijection on $k$, for any $i\in \Z_{>0}$ we get $$\rho_i(C) = \rho_i(\Tilde{C}) + l.$$ In the special cases $i = 1$ and $i = g$, we recover $$a(C) = a(\Tilde{C})\quad  \text{and}\quad  \prank(C) = \prank(\Tilde{C}) + l.$$ 

We have proven above the first part of the following observation, while the second one follows after we use \cite[Corollary 4.2]{oort:genjac} for the description of the (generalized) Jacobian of $C$ as in the proof of Theorem \ref{thm:muCnorm}.

\begin{prop}
Let $h: \Tilde{C} \to C$ be the normalization of a stable curve $C$ defined over an algebraically closed field $k$ of characteristic $p>0$. Then the Hasse-Witt partitions of $C$ and $\Tilde{C}$ coincide:  $$\delta(C) = \delta(\Tilde{C}).$$ 
Furthermore, under the Torelli morphism $j: \cMbar_g \to \Tilde{\cA}_g$, sending a stable curve of genus $g$ to its (generalized) Jacobian, we get $\delta(C) = \delta(\Jac_C)$.
\label{prop:deltaCnorm}
\end{prop}

Hence, from now on, for either a semi-abelian $g$-dimensional variety or a stable genus-$g$ curve $X$ of Hasse-Witt type $\delta = \delta_1\ldots\delta_n$ defined as above, we can say that $$a(X) = \delta_1 \quad \text{and} \quad \prank(X) = g - \sum_{i = 1}^n\delta_i.$$

\subsection{Ekedahl-Oort types of stable curves}
Below, we describe the Ekedahl-Oort types intrinsically for stable curves in a way that matches Ekedahl and van der Geer's extension of the Ekedahl-Oort types to the boundary of $\Tilde{\cA}_g$. 
We consider Moonen's definition of the Ekedahl-Oort type of a smooth curve as in \cite{moonen}, described in Section \ref{sec:eo_types}, and observe that it extends to all stable curves. This gives us a nice description of the Ekedahl-Oort type of a stable curve $C$ in terms of its normalization $\Tilde{C}$.  

As we mentioned, one can follow Ekedahl and van der Geer's work in \cite{ekedahlvdgeer} and define $$\nu_A(i + l) = \nu_{A_0}(i) + l \quad \text{for any}\quad  0\leq i\leq g - l$$ as the final type of a semi-abelian variety $A$, which is an extension of an abelian variety $A_0$ by a torus of toric rank $l$; another way to express this is $$\mu(A) = \mu(A_0).$$


In terms of the Dieudonn\'e modules, one can define the Ekedahl-Oort type of a semi-abelian variety $A$ which is an extension of an abelian variety $A_0$ by a torus of toric rank $l$, as the isomorphism class of $\mathbb{D}_A = (M, F, V, b),$ with $M = M_{l, \text{ord}}\oplus M_{A_{0}}$, $F = (F_{l, \text{ord}}, F_{A_{0}})$, $V = (V_{l, ord}, V_{A_{0}})$ and $b = b_{l, ord} + b_{A_0}$, where $$(M_{l, \text{ord}}, F_{l, \text{ord}}, V_{l, \text{ord}}, b_{l, \text{ord}})\cong (k\oplus k^{\vee}, \sigma, \sigma^{-1}, b_{ord})^{\oplus l}$$ is the polarized Dieudonn\'e module of any ordinary $l$-dimensional principally polarized abelian variety, and $(M_{A_0}, F_{A_0}, V_{A_0}, b_{A_0})$ the Dieudonn\'e module of $A_0$. We also write $$\mathbb{D}_A = \mathbb{D}_{l, \text{ord}}\oplus \mathbb{D}_{A_0}$$ to indicate that $\mathbb{D}_A$ is the direct sum of the Dieudonn\'e modules defined by the torus and $A_0$. It is not hard to see that this definition of the Ekedahl-Oort type of $A$ coincides with the one given by Ekedahl and van der Geer, which we point out below.

\begin{prop} Let $A$ be a $g$-dimensional semi-abelian variety which is an extension of an abelian variety $A_0$ by a torus of toric rank $l$. If $\mathbb{D}_A$ is defined as above, then the final type of $A$ is given by $\nu_A(i + l) = \nu_{A_0}(i) + l$, for any $0\leq i\leq g - l$. 
\label{prop:semi_eo}
\end{prop}

We define the \textit{Hasse-Witt triple of }$C$, where $C$ is either a stable curve or a finite union of smooth curves over $k$, as the triple $(Q, \Phi, \Psi)$ over $k$ with 
\begin{itemize}
    \item $Q = H^{1}(C, \O_C)$,
    \item $\Phi = F_C$ the Frobenius operator on $H^{1}(C, \O_C)$, and
    \item  $\Psi: \ker(F) \overset{\cong}{\to} \im(F)^{\perp}$ a $\sigma$-linear bijective map. 
\end{itemize}
We define the \textit{Ekedahl-Oort type} of $C$ as the isomorphism class of the Dieudonn\'e module $\mathbb{D}_C$ associated with the Hasse-Witt triple of $C$ by Proposition \ref{prop:moonen} (i.e., by \cite[Theorem 2.8]{moonen}). Finally, by $\mu(C)$ we denote the Young diagram associated with $\mathbb{D}_C$ as in Section \ref{sec:eo_types}. 
\\

Using the equivalence of the notions, we also refer to $\mu(A)$ and $\mu(C)$ as the Ekedahl-Oort types of respectively $A$  and $C$, for $A$ and $C$ as above.

We get the following description. 

\begin{thm}
Let $h: \Tilde{C} \to C$ be the normalization of a stable curve $C$ defined over an algebraically closed field $k$ of characteristic $p>0$. The definition of the Ekedahl-Oort type of $C$ via the Hasse-Witt triples extends the definition for smooth curves from Section \ref{sec:eo_types} and we have $$\mu(C) = \mu(\Tilde{C}).$$ 
Furthermore, under the Torelli morphism $j: \cMbar_g \to \Tilde{\cA}_g$, sending a stable genus-$g$ curve to its (generalized) Jacobian, we get $\mu(C) = \mu(\Jac_C)$.
\label{thm:muCnorm}
\end{thm}

\begin{proof}
First, for $C$ a smooth curve, we know that the Ekedahl-Oort type of $C$ and its Jacobian $\Jac_C$ coincide, and Moonen's work in \cite{moonen} tells us thus that the Dieudonn\'e module $(\mathbb{D}(\Jac_C[p]), F, V, b)$ bijectively corresponds to the Hasse-Witt triple $$(H^1(C, \O_C), \Phi_C, \Psi_C)$$ for $\Phi_C = F_C$ the Frobenius operator on $H^1(C, \O_C)$ and $\Psi_C: \ker(F_C) \to \im(F_C)^{\perp}$ the $\sigma$-linear bijection. As Moonen observes in \cite[2.5]{moonen}, $\im(F_C)^{\perp} = \coker (F_C)^{\vee}$, so we can and will think of $\Psi_C$ as a fixed $\sigma$-linear bijection $\Psi_C: \ker(F_C) \to  \coker (F_C)^{\vee}$.

Let $C$ now be an arbitrary stable curve with $n$ irreducible components and $m$ singular points, let $l = 1 - n + m$ (be the number of \textit{loops} in $C$), and recall that we understand $H^1(C, \O_C)$ as a $k[F_C]$-module by \eqref{eq:kFmods} and Proposition \ref{prop:semi_eo}. Giving a $\sigma$-linear bijection $\Psi_C:  \ker(F_C) \to \coker(F_C)^{\vee}$ only depends on $H^1(\Tilde{C}, \O_{\Tilde{C}})$ and $\Psi_{\Tilde{C}}$: since $F_C$ on $k$ coincides with $\sigma$ and is a bijection, we can consider the following diagram
\begin{center}
\begin{tikzcd}
\ker(F_C) = 0^{\oplus l}\oplus \ker(F_{\Tilde{C}})   \arrow[rr, "\cong"] \arrow[d, "\Psi_{C}"]                                                                                          &  & \ker(F_{\Tilde{C}}) \arrow[ "\Psi_{\Tilde{C}}", d]                                                                                       \\
{\coker(F_C) = \frac{k^{\oplus l}\oplus H^{1}(\Tilde{C}, \O_{\Tilde{C}})}{\sigma(k)^{\oplus l}\oplus F_{\Tilde{C}}(H^{1}(\Tilde{C}, \O_{\Tilde{C}}))} } \arrow[rr, "\cong"] &  & {\frac{H^{1}(\Tilde{C}, \O_{\Tilde{C}})}{F_{\Tilde{C}}(H^{1}(\Tilde{C}, \O_{\Tilde{C}}))} = \coker(F_{\Tilde{C}}) }
\end{tikzcd}    
\end{center} 
where the top and the bottom isomorphisms exist and are fixed, so that $\Psi_{\Tilde{C}}$ defines $\Psi_{C}$ in a unique way. 
Now, note that $\Psi_{C_i}$ exists as a $\sigma$-linear bijection for each component $C_i$ of $\Tilde{C}$ and that $\Psi_{\Tilde{C}}$ is defined by the collection $(\Psi_{C_i})_i$. In particular, $\Psi_{C}$ exists so that the Ekedahl-Oort type of $C$ is well-defined and the given description tells us that the Dieudonn\'e module of $C$ is the direct sum $$\mathbb{D}_{l, \text{ord}}\oplus \bigoplus_{i = 1}^n \mathbb{D}_{i},$$ where $\mathbb{D}_i$ is the Dieudonn\'e module of $C_i$, so that $$\mu(C) = \mu(\Tilde{C}).$$ Finally, by \cite[Corollary 4.2]{oort:genjac}, we have that $\Jac_C$ is an extension of $\prod_i\Jac_{C_i}$ by a torus of toric rank $l$, so that $\mu(C) = \mu(\Jac_C)$ follows as well.   
\end{proof}

\begin{rem}
Let us point out that the information about the loops of a stable curve (or equivalently, the toric part of its generalized Jacobian) in the Hasse-Witt triple is fully described by the operator $\Phi = F$, while $\Psi$ only depends on its normalization  since $F$ acts as a bijection on $k$. Taking into account Proposition \ref{prop:deltaCnorm}, one can also see Theorem \ref{thm:muCnorm} as its refinement by the  map $\Psi$. In other words, one can distinguish different $\mu$'s defined by a  fixed type $\delta$ as in \eqref{eq:deltanu}, which indeed was one of the intentions of the definition of a Hasse-Witt triple in \cite{moonen}. 
\end{rem}

\begin{rem}
Let $A = \prod_{i = 1}^nA_i$ be a principally polarized abelian variety, e.g., $A = \prod_{i = 1}^n\Jac_{C_i}$, as in the proof of Theorem \ref{thm:muCnorm}. 

For any  $1\leq i\leq n$, let $\delta(A_i) = \delta^{(i)}_1\ldots\delta^{(i)}_{N_i}$ be the Hasse-Witt partition of $A_i$ and let $N = \max\{N_i: 1\leq i\leq n\}$. The description from Section \ref{sec:hwpart} gives us that the Hasse-Witt partition of $A$ equals $$\delta(A) = \delta_1\ldots\delta_N,$$ where $\delta_j = \sum_{i = 1}^n\Tilde{\delta}^{(i)}_j$, with $\Tilde{\delta}^{(i)}_j = \delta_{j}^{(i)}$ if $1\leq j\leq N_i$ and $\Tilde{\delta}^{(i)}_j = 0$ if $N_i + 1\leq j\leq N$. 

In general, we are not aware of a straightforward combinatorial description of $\mu(A)$ in terms of $\mu(A_i)$ for $1\leq i \leq n$. However, note that the collection of $\mu(A_i)$ for $1\leq i\leq n$ uniquely determines $\mu(A)$ and that $\mu(A)$ can be computed using \cite[9.1]{oort}. Similarly, for $C = \cup_{i = 1}^n C_i$ a finite union of smooth curves $C_i$, we can compute $\mu({C})$ in terms of $\mu(C_i)$ for $1\leq i\leq n$. 
\end{rem}

\section{Inductive results}
\label{sec:inductiveres}

Using what was observed in Section \ref{sec:bdry}, we get an inductive result telling us how the dimensions of components of certain $\bZ_{\mu}\cMbar_{g+1}$ that intersect the divisor $\Delta_0$ depend on the dimensions of the components of $\bZ_{\mu}\cMbar_{g}$. We present it below and show how it generalizes some well-known inductive results about the $\prank$ and $a$-number stratifications of $\cMbar_g$.  

Throughout this section, we use that $\cMbar_g$ and $\cA_g$ are smooth stacks so that the codimension of an intersection of two of their closed subvarieties is the sum of the codimensions at most.  We denote $j(\cM_g^{ct}) = \cJ_g$ and $j(\cM_g) = \cJ_g^0 \subset \cJ_g$. Finally, we denote by $\cMbar_{g, n}$ the moduli space of $n$-pointed stable curves of genus $g$, and we use $\kappa$ as the notation for any of the finite clutching morphisms; see \cite[Section 3]{knudsen}. For example,  $$\kappa_{1, 1}: \cMbar_{g, 1} \times \cMbar_{g', 1} \to \cMbar_{g + g'}$$ is obtained by identifying the labeled points on a stable genus-$g$ curve $C$ and a stable genus-$g'$ curve $C'$, while $$\kappa_{2}: \cMbar_{g - 1, 2}  \to \cMbar_{g}$$ is obtained by identifying the labeled points on a stable genus-$(g - 1)$ curve $C$.


The Young types $\mu = [\mu_1, \ldots, \mu_n]$ that satisfy the technical condition $\bZ_{\mu} = \cup_{\mu'\leq \mu} Z_{\mu'}$ in $\cA_g$ for some $g\geq \mu_1$ will be of our main interest in this section. This condition is non-trivial, as we mentioned in \eqref{eq:finaltypes_EOclosure}. However, we can use \cite[Proposition 12.5]{oort} to see that it is satisfied at least for any $$\mu = [\mu_1, \underbrace{\mu_2, \mu_2 - 1, \ldots, 2, 1}_{\mu_2}]$$ with $\mu_1 > \mu_2 \geq 0$ and any $\cA_g$ with $g\geq \mu_1$, because $\bZ_{\mu}$ defines the closed locus of principally polarized abelian varieties with $\prank$ at most $\mu_1$ and $a$-number at least $1 + \mu_2$; see \cite[Corollary 1.5]{oort_subv} and \cite[page 44]{koblitz}.


For any $n\geq 1$, by $Z_{\mu}\cMbar_{g, n}$ we denote the inverse image of $Z_{\mu}\cMbar_g$ under the forgetful morphism $\cMbar_{g, n} \to \cMbar_g$. Next, we write $Z_{\mu}\cM_g = Z_{\mu}\cMbar_g \cap \cM_g$. Finally, by $V_f\cMbar_g$, we denote the locus of stable curves of genus $g$ with $\prank$ at most $f$. Note that $V_f\cMbar_g = \bZ_{[g - f]}\cMbar_g$, while its sublocus consisting of stable curves with $a$-number at least $2$ is $\bZ_{[g - f, 1]}\cMbar_g$.


\begin{thm}
Let $\mu = [\mu_1, \ldots, \mu_n]$ and $g \geq \mu_1$ be such that $\bZ_{\mu} = \cup_{\mu'\leq \mu} Z_{\mu'}$ in $\cA_g$. Assume that $\bZ_{\mu}\cMbar_{g}$ is pure of codimension $d$ in $\cMbar_g$ and that
$\Gamma$ is a component of $\bZ_{\mu}\cMbar_{g + 1}$ that meets $\Delta_0$ (e.g., if $d \leq g$). Then $\Gamma$ is of codimension at least $d$ in $\cMbar_{g + 1}$.
\label{thm:main1}
\end{thm}
\begin{proof}
Let $C$ be any stable curve that corresponds to a point of $\Gamma\cap \Delta_0 \neq \o$ and note that $\Delta_0 = \kappa_2(\cMbar_{g, 2})$. Theorem \ref{thm:muCnorm} gives us that the normalization of any such singular and irreducible curve $C$ of type $\leq\mu$ has type $\leq\mu$ as well, so that, $$\Gamma\cap \Delta_0\subseteq \kappa_2(\bZ_{\mu}\cMbar_{g, 2}).$$ In particular, this gives us that $\dim \Gamma \leq (3g - 3 + 2) - d + 1 = 3(g + 1) - 3  - d$, i.e., that $\Gamma$ is of codimension at least $d$ in $\cMbar_{g + 1}$.    
\end{proof}


As a consequence of the preceding theorem, we reprove some results obtained by Faber and van der Geer and by Pries in a universal way. 

\begin{cor} The following results hold. 
\begin{enumerate}
    \item For any $g\geq 2$ and $0\leq f \leq g$, the locus $V_f\cMbar_g$ of $\cMbar_g$ is pure of codimension $g - f$.
    \item For any $g \geq 2$ (resp.~$g \geq 3$), the locus $V_{g - 2}\cMbar_g$ (resp.~$V_{g - 3}\cMbar_g$) has generically $a$-number $1$.
\end{enumerate}
\label{cor:fvgdp}
\end{cor}

\begin{proof}
See \cite[Theorem 2.3]{fabervdgeer} and \cite[Proposition 3.7]{pries_a_number} for the original proofs of respectively part 1 and part 2. In our unified framework, we proceed as follows.

For part $1$, take $\mu = [g - f]$ and note that $V_0\cMbar_{g- f} = \bZ_{\mu}\cMbar_{g - f}$ is pure of codimension $g - f$ in $\cMbar_{g - f}$ (as a complete subset of $\cM_{g - f}^{ct}$ and since $V_0\cA_{g - f}$ is of codimension ${g - f}$ in $\cA_{g - f}$) for any $0 \leq f \leq g$. Therefore, Theorem \ref{thm:main1} implies that $V_f\cMbar_{g} = \bZ_{\mu}\cMbar_{g}$ is pure of codimension at least $g - f$ in $\cMbar_{g}$, while a purity result (\cite[Lemma 2.1]{fabervdgeer}) gives that its codimension is at most $g - f$. 

For part $2$, we take $\mu = [2, 1]$ and observe that  $\bZ_{[2, 1]}\cMbar_2$  is pure of codimension equal to $3$ in $\cMbar_2$. By induction, we get for any $g\geq 2$, that $\bZ_{\mu}\cMbar_g$ is of codimension at least $3$ in $\cMbar_g$, while $\bZ_{[2]}\cMbar_g = V_{g - 2}\cMbar_g$ is pure of codimension $2$. 
The claim for $V_{g-3}\cMbar_g$ follows similarly, using that $\mu = [3, 1]$ defines a codimension $4$ locus $\bZ_{[3, 1]}\cMbar_3$ in $\cMbar_3$ and noting that $\bZ_{[3]}\cMbar_g = V_{g - 3}\cMbar_g$ is of codimension $3$ in $\cMbar_g$.
\end{proof}

Let us now see an application of how one can compute the dimension of certain components of $\bZ_{\mu}\cM_g$. With $\Delta_{0, l}$ we denote the codimension $l$ locus of stable curves with $l$ loops in $\cMbar_g$, i.e., $\Delta_{0, l} = \kappa_{2l}(\cMbar_{g - l, 2l})$; for example, $\Delta_{0, 1} = \Delta_0$. 

\begin{cor}
Let $g_0 \geq 2$ and let $\mu = [\mu_1, \ldots, \mu_n]$ with $\mu_1 = g_0$ be such that $\bZ_{\mu} = \cup_{\mu'\leq \mu} Z_{\mu'}$ in $\cA_{g_0}$. Assume that $\bZ_{\mu}\cMbar_{g_0}$ is pure of codimension $d = \sum \mu_i$ in $\cMbar_{g_0}$.
For any $g\geq g_0$, if $\Gamma$ is a component of $\bZ_{\mu}\cM_{g}$ whose closure $\overline{\Gamma}$ in $\cMbar_g$ contains a stable curve with $l = g - g_0$ loops, then the codimension of $\Gamma$ in $\cM_g$ equals $d$.
\label{cor:smootheoloci}
\end{cor}

\begin{proof}
Since $\overline{j(\Gamma)} = \cJ_g \cap \bZ_{\mu}\cA_g$ is of codimension at most $d$ in $\cJ_g$, we get that $\Gamma$ is of codimension at most $d$ in $\cM_g$. On the other hand, the dimension of $\Gamma$ is at most $$\dim (\Gamma \cap \Delta_{0, l}) + l = \dim \kappa_{2l}(\bZ_{\mu}\cMbar_{g_0, 2l}) + l = 3g_0 - 3 + 2l - d + l = 3g - 3 - d,$$ i.e., $\Gamma$ is of codimension at least $d$ in $\cM_g$.   
\end{proof}

\begin{rem}
In addition to the preceding corollary, and keeping the notation, note that Pries in \cite[Theorem 6.4]{pries_current_results}, obtains that the existence of a non-empty component of $\bZ_{\mu}\cM_{g_0}$ of codimension $d = \sum \mu_i$ in $\cM_{g_0}$ implies, for any $g\geq g_0$, the existence of a non-empty component of $\bZ_{\mu}\cM_{g}$, which is of codimension $d = \sum \mu_i$ in $\cM_g$.  
\end{rem}

To indicate when the conditions of the preceding corollary are satisfied, we consider the following result, which one can find in \cite[Lemma 2.4, Lemma 2.5]{fabervdgeer} in case $l = 1$. 

\begin{prop}
Let $\Gamma$ be an irreducible, closed subset of $\cMbar_{g, n}$ of dimension $2g - 3 + n + l$, for $l\geq 1$. Then $\Gamma \cap \Delta_{0, l} \neq 0$. Furthermore, $\Gamma$ contains a point corresponding to a stable curve with $m \leq g - l$ elliptic curves as irreducible components and $g - m \geq l$ loops.
\label{prop:loopscomplete}
\end{prop}
\begin{proof}
Similarly as in \cite[Lemma 2.4]{fabervdgeer}, it is enough to prove the claim for $g \geq 2$ and $n = 0$. We use induction on $l$ with \cite[Lemma 2.4]{fabervdgeer} with $l = 1$ as the base case. Let $l \geq 2$, then $\Gamma$ intersects $\Delta_0$ by \cite[Lemma 2.4]{fabervdgeer}, and any component $\Gamma'$ of $\Gamma \cap \Delta_0$ is of dimension $$\dim \Gamma' \geq 2g - 3 + l - 1.$$
Let $\Gamma''$ be a component of its preimage under $\kappa_2$ in $\cMbar_{g - 1, 2}$. Since $\kappa_2$ is a finite map, we have
$$\dim \Gamma'' = \dim \Gamma' \geq 2g - 3 + l - 1,$$ so \cite[Lemma 2.4]{fabervdgeer} applied to $\Gamma''$ gives us that $\Gamma''\cap \Delta_0 \neq \o$. Now we have $\kappa_2(\Gamma'') \subseteq \Gamma$, and by induction $\Gamma'' \cap \Delta_{0, l - 1}\neq \o$, so we find that $$\Gamma \cap \Delta_{0, l} \neq \o.$$ Lastly, \cite[Lemma 2.5]{fabervdgeer} with the preceding discussion implies the second claim.
\end{proof}

We illustrate the general approach for computing the dimensions of certain Ekedahl-Oort strata of $\cM_g$ by the following result and the examples in the coming section. 

    
\begin{cor}
If non-empty, the $a$-number $2$ locus of $V_{f}\cM_g$ is pure of codimension $1$ in $V_{f}\cM_g$ for $f = g - 2$, $g \geq 2$ or $f = g - 3$, $g\geq 3$.
\end{cor}
\begin{proof}
See \cite[Corollary 4.5]{pries_a_number} for a similar result when $f = g - 2$.
The $a$-number $2$ locus of $V_{g - 2}\cM_g$ equals $\bZ_{[2, 1]}\cM_g$, while the $a$-number $2$ locus of $V_{g - 3}\cM_g$ is $\bZ_{[3, 1]}\cM_g$. Dimension computations and Proposition \ref{prop:loopscomplete} tell us that any component $\Gamma$ of these Ekedahl-Oort loci satisfies the conditions of Corollary \ref{cor:smootheoloci}, so the result follows.  
\end{proof}

\begin{rem} The assumption of the previous Corollary is satisfied when $g\geq 3$ or in the case $g = 2$, $p>3$. More generally, for $\mu = [\mu_1, \ldots, \mu_n]$ and any $g\geq \max\{2, \mu_1\}$, recall  that Pries in \cite[3.3.1, 3.3.2, and Theorem 6.4]{pries_current_results} obtains that $Z_{\mu}\cM_g$ is non-empty in characteristic $p$ in the following cases:
\begin{enumerate}
    \item $\mu  \in \{[1], [2], [3], [3, 1]\}$ for any prime $p>0$, 
    \item $\mu = [2, 1]$, for $g = 2$, $p>3$  or $g \geq 3$, $p>0$, or
    \item $\mu  \in \{[3, 2], [3, 2, 1]\}$ and $p>2$.
\end{enumerate} 
\end{rem} 

\newpage
\section{Applications and examples}
\label{sec:exm}

In this section, we collect some results about the Ekedahl-Oort stratification of the moduli spaces of genus-$g$ curves for $g\geq 4$ in characteristics $2$ and $3$. Furthermore, we present a criterion that can potentially be used to discuss the question of whether $V_0\cMbar_4$ generically has $a$-number $1$ in fixed characteristic $p>0$. It essentially relies on counting the number of smooth hyperelliptic curves with a prescribed Ekedahl-Oort type. We finish the section by describing the approach to compute the upper bound on the dimensions of certain Ekedahl-Oort loci in $\cMbar_g$ given by Theorem \ref{thm:main1}.


First, for $p = 2$, we recall the result we obtained in \cite[Corollary 6.6]{dd2}. 

\begin{prop} In characteristic $2$, the Ekedahl-Oort strata $Z_{[4]}\cap \cJ_4$, $Z_{[4, 1]}\cap \cJ_4$, and $Z_{[4, 2]}\cap \cJ_4$ are respectively of the expected codimensions $4, 5,$ and $6$ in $\cJ_4$, while $Z_{\mu}\cap \cJ_4^0 = \o$ exactly for $\mu \in \{[4, 3], [4, 2, 1], [4, 3, 1], [4, 3, 2], [4, 3, 2, 1]\}$.  
\label{prop:eo_conclusion_char2}
\end{prop}



Let us now consider the case $p = 3$. Zhou has studied the Ekedahl-Oort loci of hyperelliptic genus-$4$ curves in characteristic $3$ in \cite{zhou_genus4}. One can either use that or explicitly obtain that there is a unique such curve with Ekedahl-Oort type $[4, 3]$ as we do in the following example.

\begin{exmp}
Any hyperelliptic genus-$4$ curve $C$ over an algebraically closed field $k$ with characteristic $\neq 2$ can be written in the following normal form $$y^2 = x^9 + a_8x^8 + \ldots + a_2x^2 + x,$$ with $a_i \in k, 2\leq i\leq 8$, obtained by choosing $0$ and $\infty$ to be the branch points and by scaling. Let here $\mathrm{char}(k) = 3$, when we can write the Hasse-Witt matrix of $C$ in quite a simple form: $$\begin{pmatrix}
a_2 & a_5 & a_8 &0 \\ 
1 & a_4 & a_7 & 0\\ 
0 & a_3 & a_6 & 1\\ 
0 & a_2 & a_5 & a_8
\end{pmatrix}.$$  The condition that $C$ has Ekedahl-Oort type $[4, 3]$ is equivalent to the fact that its Hasse-Witt partition is $\delta = 22$, i.e., that $$\rank_k H = 2 \quad \text{and} \quad HH^{(3)} = 0.$$ It is not hard to check that the first condition is equivalent to $a_2 = a_5 = a_8 = 0$, while the second one gives us $a_3 = a_4 = a_6 = a_7 = 0$. Hence, there is a unique hyperelliptic curve $$y^2 = x^9 + x$$ with Ekedahl-Oort type $[4, 3]$ up to isomorphism. In particular, there is a zero-dimensional component of $$\bZ_{[4, 3]}\cMbar_4 \cap \overline{\cH}_4,$$ where $\overline{\cH}_4$ is the (closed) locus of stable hyperelliptic curves of genus $4$, which is of codimension $4 - 2 = 2$ in $\cMbar_4$.
\label{exmp:he43}
\end{exmp}

\begin{prop}
In characteristic $3$, for any $\mu = [\mu_1, \ldots, \mu_n]$ with $\mu > [4, 3]$, the Ekedahl-Oort stratum $Z_{\mu}\cap \cJ_4$ is pure of the expected codimension $\sum \mu_i$.
\label{prop:eo_conclusion_char3}
\end{prop}

\begin{proof}
By \cite[Theorem 1.1]{zhou_genus4}, $\bar{Z}_{\mu}\cap \cJ_4^0$ is non-empty in characteristic $3$ if $\mu > [4, 3]$, and so is $\bar{Z}_{\mu}\cap \cJ_4$. Since $\cJ_4$ is of codimension $1$ in $\cA_4$, 
$\bZ_{\mu}\cap \cJ_4$ is either pure of the expected codimension in $\cJ_4$, or it is of codimension one less, irreducible by \cite[Theorem 11.5]{ekedahlvdgeer}, and thus coincides with $\bZ_{\mu}$ in $\cA_4$. 

Assume the latter is true and let $\Gamma$ and $\Gamma'$ be components of $\bZ_{\mu}\cMbar_4$ and $\bZ_{[4, 3]}\cMbar_4$ containing the point corresponding to $C$ from Example \ref{exmp:he43} and such that $\Gamma' \subseteq \Gamma$. By assumption, we get that $j(\Gamma) = \bZ_{\mu}$ and that $\dim\Gamma' = 3$ since $j(\Gamma')$ is a component of the locus $\bZ_{[4, 3]}$, which is pure of dimension $3$. Since it is non-empty, any component of the locus $\Gamma' \cap \overline{\cH}_{4}$ must be of dimension at least  $3 - 2 = 1$. However, as we noted, there is a component of $\Gamma' \cap \overline{\cH}_{4}$ which is $0$-dimensional, namely, the one consisting of the point corresponding to the curve $C$. 
\end{proof}


Furthermore, we can use the proof of Proposition \ref{prop:eo_conclusion_char3} to get the following criterion for computing dimensions of certain loci $\Gamma \cap \cJ_4$ in  $\cA_4$ such that $Z_{[4, 3]} \subset \overline{\Gamma}$. We present it in case $\Gamma = Z_{\mu}$ with $\mu > [4, 3]$. However, note that it can be used as well to compute the dimension of any non-supersingular Newton polygon locus since $Z_{[4, 3]}$ is completely contained in the supersingular locus of $\cA_4$.  

\begin{prop} Let $k$ be an algebraically closed field of characteristic $p>2$ and let $\mu = [\mu_1, \ldots, \mu_n]$ be such that $\mu> [4, 3]$. If the set of smooth hyperelliptic curves of genus $4$ over $k$ with Ekedahl-Oort type $[4, 3]$ is non-empty and finite, then $Z_{\mu}\cap \cJ_4$ is pure of the expected codimension $\sum \mu_i$ in $\cJ_4$ if non-empty. In particular, then $V_0\cMbar_4$ has generically $a$-number $1$, i.e., all generic points of $V_0\cMbar_4$ have $a$-number $1$.
\label{prop:he_implies_eo}
\end{prop}

Now we use our inductive procedure to get results about some larger $\prank$ loci in case when $p = 2$ or $3$. 

\begin{cor}
In characteristics $p \in \{ 2, 3\}$, $V_{g - 4}\cMbar_g$ generically  has $a$-number $1$,  i.e., all generic points of $V_{g - 4}\cMbar_g$ have $a$-number $1$, for any $g\geq 4$.     
\end{cor}
\begin{proof}
Use Theorem \ref{thm:main1} with $\mu = [4, 1]$ together with Proposition \ref{prop:eo_conclusion_char2} and Proposition \ref{prop:eo_conclusion_char3} to obtain the result.    
\end{proof}


Another way to get the previous result is to use \cite[Proposition 3.7]{pries_a_number} and the conclusions about the $\prank$ zero locus of $\cMbar_4$ in characteristics $p = 2, 3$. Let us finish this section by presenting how we can use Theorem \ref{thm:main1} to make some new observations. 

\begin{exmp}[{Ekedahl-Oort type $\left [3, 2\right ]$.}]
Let $\mu = [3, 2]$ and let $\Gamma$ be a component of $\bZ_{\mu}\cMbar_4$ in characteristic $p>0$. 

If ${\Gamma}\cap \Delta_0 \neq \o$, Theorem \ref{thm:main1} gives us that $$\dim \Gamma \leq \dim \kappa_{2}(\bZ_{\mu}\cMbar_{3, 2}) + 1 = 4.$$ 

Assume now that $\Gamma$ generically consists of smooth curves and $\Gamma \subset \cM_4^{ct}$. It follows that $\dim \Gamma = \dim j({\Gamma}) \geq 4$, and thus $$4\leq \dim \Gamma \leq 5$$ since $\Jac_4$ is of codimension $1$ in $\cA_4$. If $\dim \Gamma = 5$, then $\Gamma$ is a complete subvariety of $\cM_4^{ct}$ of dimension $5$ and thus contains a point corresponding to a stable curve that is a chain of elliptic curves by \cite[Lemma 2.5]{fabervdgeer}. In particular, it intersects $\Delta_1$ and $\Delta_2$, and we get that ${\Gamma}\cap \Delta$ consists of the union of some components of:
\begin{itemize}
    \item  $4$-dimensional $\kappa_{1, 1} (\bZ_{[1]}\cMbar_{1, 1} \times \bZ_{[2, 1]}\cMbar_{3, 1})$,
    \item  $3$-dimensional $\kappa_{1, 1}(\cMbar_{1, 1} \times \bZ_{[3, 2]}\cMbar_{3, 1})$,
    \item  $4$-dimensional $\kappa_{1, 1} (\bZ_{[2, 1]}\cMbar_{2, 1} \times \bZ_{[1]}\cMbar_{2, 1})$, and 
    \item  $4$-dimensional $\kappa_{1, 1} (\bZ_{[2]}\cMbar_{2, 1} \times \bZ_{[2]}\cMbar_{2, 1})$.
\end{itemize}
Because of its irreducibility, if $\dim {\Gamma} = 5$, it follows by \cite[Theorem 11.5]{ekedahlvdgeer} that $$j({\Gamma}) = \bZ_{[3, 2]} = Z_{[3, 2]} \cup \bZ_{[4, 2]}.$$ 
Finally, note that we also obtained $\dim \Gamma \leq 4$ in the remaining case $\Gamma \subseteq \Delta_1 \cup \Delta_2$. 
In particular, if there are no components of $\bZ_{[4, 3]}\cMbar_{4}$ of dimension $3$ (e.g., if $p \in \{2, 3\}$), all the components $\Gamma$ of $\bZ_{[3, 2]}$ have $\dim \Gamma \leq 4$.

Note that $\mu = [3, 2]$ is such that $\bZ_{\mu} = \cup_{\mu' \leq \mu}Z_{\mu'}$ in $\cA_4$. That follows from the discussion at the beginning of Section \ref{sec:inductiveres} and the fact that $Z_{[4, 1]} \not \subset \bZ_{[3, 2]}$ since $Z_{[3, 2]}$ and $Z_{[4, 1]}$ have the same dimension in $\cA_4$ and are both irreducible by \cite[Theorem 11.5]{ekedahlvdgeer}.

Let $\Gamma$ be a component of $\bZ_{[3, 2]}\cMbar_5$. If ${\Gamma}\cap \Delta_0 = \o$, then ${\Gamma} \subset \cM_5^{ct}$ and $\dim \Gamma \geq 7$ since $\cJ_{5}\cap \bZ_{[3, 2]}$ is of codimension at most $5$ in $\cJ_5$. By \cite[Lemma 2.4 and Lemma 2.5]{fabervdgeer}, it follows that $\dim \Gamma = 7$ and one can use that $\Gamma$ contains a point corresponding to a chain of elliptic curves to find an upper bound on its dimension. 

Otherwise ${\Gamma}\cap \Delta_0 \neq \o$, and one can compute an upper bound on its dimension using Theorem \ref{thm:main1} applied to $\bZ_{[3, 2]}\cMbar_4$.
\label{exmp:32inGenus4}
\end{exmp}



\begin{exmp}[{Ekedahl-Oort type $\left [3, 2, 1\right ]$.}]
Let $\mu = [3, 2, 1]$ and let $\Gamma$ be a component of $\bZ_{\mu}\cMbar_4$. If ${\Gamma}\cap \Delta_0 \neq \o$, Theorem \ref{thm:main1} gives us that $$\dim \Gamma \leq \dim \kappa_{2}(\bZ_{\mu}\cMbar_{3, 2}) + 1 = 3.$$ 

Otherwise, $\Gamma \subset \cM_4^{ct}$ and if its generic point is smooth, we have $\dim \Gamma = \dim j({\Gamma}) \geq 3$. Similarly as in Example \ref{exmp:32inGenus4}, if $\dim \Gamma = 4$, then it has to be $$j({\Gamma}) = \bZ_{[3, 2, 1]} = Z_{[3, 2, 1]} \cup \bZ_{[4, 2, 1]}.$$
Using that and  Theorem \ref{thm:main1}, one could get an upper bound on the dimension of any component of $$\bZ_{[3, 2, 1]}\cMbar_g\text{ } \text{ for } g \geq 7$$ if we would know more about the dimensions of the components of $\bZ_{[3, 2, 1]}\cMbar_5$ and $\bZ_{[3, 2, 1]}\cMbar_6$. 
\end{exmp}

\normalsize

\newpage
\begin{thebibliography}{vdGvdV95}


\bibitem[Ale05]{alexeev}
{Alexeev, V.}, \emph{Compactified Jacobians and Torelli map.},
Publ. Res. Inst. Math. Sci., 40(4), pp.~1241–1265, 2004.

\bibitem[AH19]{achterhowe}
{Achter, J. and Howe, E.}, \emph{Hasse-Witt and Cartier-Manin matrices: a warning and a request.},
Arithmetic geometry: computation and applications, pp.~1–-18, 
Contemp. Math., 722, Amer. Math. Soc., 2019.

\bibitem[Bou98]{bouw}
{Bouw, I.}, \emph{The p-rank of curves and covers of curves.},
Progr. Math 187, pp. ~403--412, 1998.


\bibitem[Dra23]{dd2}
{Dragutinovi\'c, D.}, \emph{Supersingular curves of genus four in characteristic two},
arxiv:2301.12897.   


\bibitem[EvdG09]{ekedahlvdgeer}
{Ekedahl, T. and van der Geer, G.}, \emph{Cycle Classes of the E-O Stratification on the Moduli of Abelian Varieties.}, In: Algebra, Arithmetic, and Geometry: Volume I: In Honor of Yu. I. Manin, pp. ~567--636. Birkhäuser Boston, Boston, 2009.


\bibitem[FvdG04]{fabervdgeer}
{Faber, C. and van der Geer, G.}, \emph{Complete subvarieties of moduli spaces and the Prym map},
 J. Reine Angew. Math., 573, pp. 117-–137, 2004.

\bibitem[HW36]{hassewitt}
{Hasse, H. and Witt, E.}, \emph{Zyklische unverzweigte Erweiterungskörper vom Primzahlgrade p über einem algebraischen Funktionenkörper der Charakteristik p.},
 Monatsh. Math. Phys. 43, no. 1, pp. 477–492., 1936.  


\bibitem[Knu83]{knudsen}
{Knudsen, F.}, \emph{The projectivity of the moduli space of stable curves. II. The stacks $M_{g,n}$}, Math. Scand. 52, no. 2, pp.~161–-199., 1983.

\bibitem[Kob75]{koblitz}
{Koblitz, N.}, \emph{$p$-adic variation of the zeta-function over families of varieties defined over
finite fields.}, Compositio Math. 31, pp.~119--218., 1975.

\bibitem[KT18]{kocktait}
{Köck, B. and Tait, J.}, \emph{On the de-Rham cohomology of hyperelliptic curves.},
Res. Number Theory, 4, Paper No. 19, 17 pp., 2018.

\bibitem[Moo22]{moonen}
{Moonen, B.}, \emph{Computing discrete invariants of varieties in positive characteristic I. Ekedahl-Oort types of curves.},
Journal of Pure and Applied Algebra, Volume 226, Issue 11, Paper No. 107100, 19 pp.,  2022.



\bibitem[Oor62]{oort:genjac}
Oort, F., \emph{A construction of generalized Jacobian varieties by group extensions}, Mathematische Annalen,
vol. 147, pp.~277--286., 1962.

\bibitem[Oor74]{oort_subv}
Oort, F., \emph{Subvarieties of moduli spaces}, Invent. Math., 24, pp.~95--119. 1974.

\bibitem[Oor01]{oort}
Oort, F., \emph{ A stratification of a moduli space of abelian varieties},
Moduli of abelian varieties
(Texel Island, 1999), vol. 195 of Progr. Math., pp.~345–-416. Birkh\"auser, Basel, 2001.


\bibitem[Pri09]{pries_a_number}
Pries, R., \emph{The $p$-torsion of curves with large $p$-rank},
International Journal of Number Theory 5, 6: pp.~1103-1115, 2009.  

\bibitem[Pri19]{pries_current_results}
Pries, R., \emph{Current results on Newton polygons of curves.},
Open Problems in Arithmetic Algebraic Geometry, editor Oort, Advanced Lectures in Mathematics, 46, chapter 6, pp.~179--208, 2019.

\bibitem[vdG99]{vdgeercycle}
{van der Geer, G.}, \emph{Cycles on the moduli space of abelian varieties},
 Aspects Math., E33. pp.~65--89, 1999.


\bibitem[Zho20]{zhou_genus4}
Zhou, Z., \emph{Ekedahl-Oort strata on the moduli space of curves of genus four.},
Rocky Mountain J. Math. 50, no. 2, pp. 747–761, 2020.

\end{thebibliography} 
\end{document}