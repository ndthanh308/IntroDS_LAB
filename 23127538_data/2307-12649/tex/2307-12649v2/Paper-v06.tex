\documentclass[prl,superscriptaddress, amsfonts, amssymb, amsmath, reprint, showkeys, nofootinbib, twoside]{revtex4-2}
%\documentclass[aps,prl, twoside]{revtex4-2}
%\usepackage[style=phys]{bibtex}
\usepackage[english]{babel}
\usepackage[utf8]{inputenc}
\usepackage{amsthm}
\usepackage{mathtools}
\usepackage{physics}
\usepackage{xcolor}
\usepackage{graphicx}
\usepackage[T1]{fontenc}
\usepackage[pdftitle={Article}, pdfauthor={Author}]{hyperref} % For hyperlinks in the PDF
\usepackage{physics}
\usepackage[colorinlistoftodos, color=green!40, prependcaption]{todonotes}
\usepackage{dsfont}
\usepackage{chemformula}
%\usepackage{siunitx}
\usepackage{amsmath}
%\usepackage{subcaption}
\usepackage{relsize}
\usepackage{enumitem}
\usepackage{epsfig}
\usepackage[amssymb]{SIunits}
\usepackage[normalem]{ulem}
\usepackage{esint}
\usepackage{multirow}
\usepackage{csquotes}

\renewcommand{\unit}[1]{\,\mathrm{#1}}
\addto\captionsenglish{\renewcommand{\figurename}{FIG.}}

\begin{document}
%\title{Electron-magnon scattering in complex systems from first principles} 
%\title{Magnon damping mechanisms in ultra thin metallic layers}
%\title{Frequencies and decay channels of terahertz magnons in ultrathin magnetic films: interplay of correlations, disoder, and multi-magnon processes}
\title{Correlations, disorder, and multi-magnon processes in terahertz spin dynamics of magnetic nanostructures: A first-principles investigation}
%\title{Frequencies and decay channels of terahertz magnons in ultrathin magnetic films: the role of correlations, disoder, and multi-magnon processes} 
%\author{Sebastian and the gang}
\author{Sebastian Paischer} \email{sebastian.paischer@jku.at} 
\affiliation{Institute for  Theoretical Physics, Johannes Kepler  University Linz, Altenberger  Stra{\ss}e 69, 4040 Linz} 
\affiliation{Department of Engineering and Computer Sciences, Hamburg  University of Applied Sciences, Berliner Tor 7, 20099 Hamburg, Germany}
\author{David Eilmsteiner} 
\affiliation{Institute for  Theoretical Physics, Johannes Kepler  University Linz, Altenberger  Stra{\ss}e 69, 4040 Linz} 
\affiliation{Department of Engineering and Computer Sciences, Hamburg  University of Applied Sciences, Berliner Tor 7, 20099 Hamburg, Germany}
\author{Igor Maznichenko} 
\affiliation{Department of Engineering and Computer Sciences, Hamburg  University of Applied Sciences, Berliner Tor 7, 20099 Hamburg, Germany} 
\author{Nadine Buczek} 
\affiliation{Department of Applied Natural Sciences, L\"ubeck  University of Applied Sciences, M\"onkhofer Weg 239, 23562 L\"ubeck,  Germany} 
\author{Khalil Zakeri} 
\affiliation{Heisenberg Spin-Dynamics Group, Physikalisches Institut, Karlsruhe Institute of Technology, Wolfgang-Gaede-Strasse 1, D-76131 Karlsruhe, Germany}
\author{Arthur Ernst}
\affiliation{Institute for Theoretical Physics, Johannes Kepler
  University Linz, Altenberger  Stra{\ss}e 69, 4040 Linz}
\affiliation{Max Planck Institute of Microstructure Physics, Weinberg
  2, D-06120 Halle, Germany}
\author{Pawe\l{} A. Buczek}
\affiliation{Department of Engineering and Computer Sciences, Hamburg  University of Applied Sciences, Berliner Tor 7, 20099 Hamburg, Germany} 
\date{\today}

\begin{abstract}
%The role of correlation, disorder, and multi-magnon processes in terahertz spin dynamics is systematically addressed in a first-principles scheme for magnetic nanostructures on the example of ultrathin Co films deposited on Cu and Ir surfaces. We show that a substantial part of the electron self-energy beyond adiabatic local spin density approximation originates from the interaction of electrons with the gas of virtual magnons leading to a major correction of exchange splitting and renormalization of spin-wave energies, both in full agreement with experimental data. An \emph{ab initio} quantitative hierarchy of attenuation processes is established. The damping of magnons is dominated by the Landau mechanism and disorder induced scattering with multi-magnon processes contributing weakly.
%============================================================================
%Gaining a comprehensive understanding of the profound impact of correlation effects and other factors such as crystal imperfections is essential for achieving an accurate description of solids. 
%The investigation of THz magnons provides a promising pathway to not only address these effects but also unravel their fundamental origins. 
%In this study, we systematically study the role of correlation, disorder, and multi-magnon processes in THz spin dynamics using a first-principles approach. We focus on magnetic nanostructures, specifically ultrathin Co films deposited on Cu and Ir surfaces, as an example. Our findings reveal that a significant portion of the electron self-energy, which goes beyond the adiabatic local spin density approximation, arises from the interaction between electrons and a virtual magnon gas. This interaction leads to a substantial correction of the exchange splitting and a renormalization of spin-wave energies, both of which drastically enhance the agreement with experimental data. Moreover, we establish a quantitative hierarchy of attenuation processes using an \emph{ab initio} method. The damping of magnons is dominated by the Landau mechanism and disorder-induced scattering with multi-magnon processes making only minor contributions.
%============================================================================
Understanding the profound impact of correlation effects and crystal
imperfections is essential for an accurate description of solids. Here
we study the role of correlation, disorder, and multi-magnon processes
in THz magnons. Our findings reveal that a significant part of the
electron self-energy, which goes beyond the adiabatic local spin
density approximation, arises from the interaction between electrons
and a virtual magnon gas. This interaction leads to a substantial
modification of the exchange splitting and a renormalization of magnon
energies, in agreement with the experimental data. We establish a
quantitative hierarchy of magnon relaxation processes based on first
principles.   
\end{abstract}

\keywords{}

\maketitle

Magnetic nanostructures and their intricate spin dynamics have fueled
remarkable developments in experimental, applied, and theoretical
quantum many-body physics over the recent years. On the applied side,
nanostructures, beginning with thin
films %\cite{Jourdan2015,Huang2018,Yang2021}
\cite{Yang2021} and proceeding down to single magnetic
atoms %\cite{Miyamachi2013,Natterer2017}
\cite{Natterer2017}, constitute the basis for spintronic and magnonic
information storage and processing
devices %\cite{Chumak2012,Gertz2015,Guo2019}
\cite{Guo2019}. A substantial body of research has been devoted to the
realization of quantum logical gates using magnetic degrees of
freedom %\cite{Chumak2015,Jia2021}
\cite{Jia2021}. Optical magnetization
switching %\cite{Kimel2004,Kirilyuk2010,Guyader2015,Hadri2016}
\cite{Hadri2016} allows to control the magnetization dynamics on the femtosecond timescale.  For ultrathin magnetic films, recent experiments have yielded highly resolved spectra of the electronic (using, e.g., angle-resolved photoemission \cite{Sobota2021}) and, by means of spin-polarized high-resolution electron energy-loss spectroscopy (SPHREELS) \cite{Qin2019,Zakeri2021}, magnonic (spin-wave) band structures across the entire Brillouin zone.\\
On the other hand, the current first principles theoretical
description of spin dynamics lags clearly behind these spectacular
experimental developments. In this Letter we attempt to narrow this
gap. The task is of considerable interest, as the spectrum of
collective spin excitations (called spin-waves or magnons) determines
the thermodynamic properties of magnets, including the phase
transition temperatures. Additionally, the excitations contribute to
the specific heat as well as to the thermal and electric
conductivities. % \cite{Nolting2009}.
Furthermore, their coupling to electronic degrees of freedom can give
rise to a superconducting state \cite{Essenberger2016} and, in
general, influences the electronic band structure \cite{Tusche2018},
leading to a finite lifetime of excited electronic states
\cite{Schmidt2010a}. Last but not least, the damping of the spin
dynamics is of paramount practical importance in spintronic
applications %\cite{Krawczyk2014,Chumak2017}
\cite{Chumak2017} and, as we show here, constitutes an additional experimental probe sensitive to the atomic and electronic structure of nanostructures.\\
The density functional theory in a local spin density approximation
(LSDA) is able to provide a qualitatively correct picture of
electronic band structures of films and surfaces, including effects
like the formation of electronic surface and quantum well
states %\cite{Heinrichsmeier1993,Varykhalov2005}
\cite{Varykhalov2005}, but misses important corrections arising from
correlation effects \cite{Kotliar2006}, notably predicting a
substantially wrong value of the Stoner exchange splitting
\cite{Monastra2002}, among others. In order to remain specific, we
address Co films on different substrates for which a sufficient body
of experimental evidence and theoretical studies concerning the electronic structure and spin
dynamics is available
\cite{Zakeri2021,Taroni2011}.  %\cite{Miranda1983,Zakeri2021}.
These deficiencies are straightforwardly reflected in the \textit{ab
  initio} descriptions of spin excitations. Without a substantial
\textit{ad hoc} \enquote{negative $U$} correction of the occupied
majority spin bands, the predicted spin-wave energies are too high
compared to their experimental values \cite{Chen2017}. Different
descriptions of the band structure renormalization have been proposed,
including three-body scattering \cite{Monastra2002}, tight-binding
approaches \cite{Costa2004}, and a sophisticated dynamical mean field
treatment \cite{Janas2022}. We have recently put forward a fully
\textit{ab initio} scheme allowing to compute the electronic
self-energy for complex systems, including two dimensional films,
within Hedin's many-body perturbation scheme
\cite{Paischer2023}. Here, we show that in the considered Co films a
substantial part of these corrections arises indeed due to the
interaction of electrons with the gas of virtual magnons.\\ 
Likewise, the damping mechanism of THz spin excitations has not yet
been fully understood. While the Landau damping, arising due to the
interactions of collective spin-waves with single particle (Stoner)
excitations, is known to be an important decay channel in conducting
systems \cite{Buczek2011a, Qin2015}, it does not explain the entire
experimentally observed magnon peak width \cite{Zakeri2021},
especially in the high frequency range. Hence, other conceivable
damping mechanisms may arise due to spin dynamics occurring beyond the
linear response regime (expressible in the language of multi-magnon
processes \cite{Kaganov1987}), relativistic spin-orbit coupling (SOC)
\cite{Bergman2010,Bergqvist2013}, and the presence of structural
disorder \cite{Paischer2021a}. None of these effects have been systematically studied
within a realistic \textit{ab initio} framework so far.  Here, we
introduce a methodology capabale of accounting for all these
effects. Thus, for the first time
%Here we will provide a solution to all these problems and introduce a methodology which can account for all these effects. 
we provide a clear \textit{quantitative} hierarchy of spin dynamics damping mechanisms in itinerant magnetic films dominated by Landau mechanism and disorder with multi-magnon processes contributing weakly to the magnon linewidth.\\
%The text is organized as follows. First, the electronic ground state of the films is investigated taking into account many-body effects beyond the LSDA. Next, the spin dynamics is addressed within the framework of the linear response time-dependent density functional theory (TDDFT) in order to recover the energies and Landau damping of the magnon modes. Finally, we quantitatively describe the decay channels opening due to disorder and multi-magnon processes.
\textbf{Correlated ground state.} 
%COUPLE OF SENTENCES ON THE METHOD? DYSON EQUATION? A PICTURE OF ELECTRON-MAGNON COLLISION?
We consider a three monolayers (ML) thick Co film grown on Cu(001) and
Ir(001) surfaces. Their respective electronic structures are obtained
using a first-principles Green's function method \cite{Hoffmann2020},
fully taking into account the effects associated with the
semi-infinite substrates and, if necessary, the disorder on the level
of the coherent potential approximation (CPA). The band structures are
shown in Fig. \ref{fig_3Co_Cu} for Co/Cu and supplementary
note II in \cite{supplement} for Co/Ir.  For Co grown on
the Cu surface a major deficiency of the LSDA is the resulting
location of the occupied majority states too far below the Fermi
level. In turn, this yields a too large Stoner exchange splitting
between majority and minority bands. While the experimental value for
the exchange splitting for similar systems is reported to be around
0.8$\unit{eV}$ \cite{Miranda1983} the LSDA predicts values between
1.7$\unit{eV}$ and 2$\unit{eV}$ depending on the position in the
Brillouin zone. As shown below, this leads to substantial overestimations of
magnon energies. The origin of this shortcoming is that the LSDA
systematically fails to reproduce important correlation effects in the
band structure of magnetic 3d transition metals
\cite{SanchezBarriga2012}, influencing the values of the bandwidth and
exchange splitting, as well as the presence of satellite states. In
order to account for these effects, one must evaluate the electronic
self-energy, e.g. using Hedin's approach (many-body perturbation
theory (MBPT) framework) \cite{Nabok2021,Paischer2023}, evaluating
selected classes of Feynman diagrams. In particular, the possibility
of an electron or hole decay associated with the emission of a virtual
magnon accounting for the conservation of the spin angular momentum
(\enquote{electron-magnon interaction}), turns out to be essential in
the description of 3d magnets \cite{Paischer2023}. While being state
of the art, such calculations are computationally demanding and so far
have hardly been applied to complex solids and nanostructures. In our
recently proposed computational scheme \cite{Paischer2023}, we
successfully approximate the corresponding series of ladder diagrams
in Hedin's theory \cite{Hedin1965} with less expensive response
functions and kernels available in the time-dependent density functional
theory \cite{Buczek2011a} allowing us to address systems as complex as
Co films considered here.
 % Figure environment removed
Figure \ref{fig_3Co_Cu} shows the impact of spin-fluctuations on the
band structure of the 3$\unit{ML}$ Co/Cu(001) film. The
magnon-electron interaction shifts the occupied majority bands towards
the Fermi energy. The energy shift of the majority bands amounts to
approximately 0.8$\unit{eV}$, in agreement with results from hcp
cobalt \cite{Monastra2002,SanchezBarriga2012}. The quasiholes in the
majority spin channel acquire a finite lifetime being dressed now in
the gas of virtual magnons. The impact of the spin fluctuations on the
minority band is much weaker due to fewer electron partners in the
unoccupied spin-up channel for the exchange of the magnons. We remark
in passing that our calculations for bulk fcc Co (not shown here)
reproduce experimental observations as well \cite{Tusche2018}. This
fully \textit{ab initio} treatment provides a justification for the
application of the \enquote{negative $U$} \cite{Zakeri2021},
cf. Fig. \ref{fig_3Co_Cu}. It turns out that the shift of the band
(but not the hole lifetimes) can indeed be modeled upon the
application of $U=-1.6\unit{eV}$ on the 3d bands of Co. While being
methodologically limited, the latter approach allows nevertheless to
determine the band structure self-consistently which is still beyond
the computational reach of the current many-body approach. The
self-consistency is important for the evaluation of the spin-wave
spectra being sensitive to the value of the Fermi level. 

Interestingly, in the Co/Cu system (but not Co/Ir), the LDA+$U$
correction results in the ferromagnetic ground state becoming
unstable. At first glance this is not surprising, as the shift of the
bands towards the Fermi level results in long-range exchange
interactions between magnetic moments with oscillating sign. However,
this contradicts the experimental findings. This hints at a missing
element in the theoretical description.  According to the experimental
evidence \cite{Heinz2009,Nouvertne1999} both films feature a certain
degree of disorder, generally weaker for Co/Cu compared to Co/Ir. The
primary effect of disorder in Co/Cu is the smearing of majority bands
below the Fermi level as shown in Fig. \ref{fig_3Co_Cu}, the finite
lifetime corresponding to the collisions of electrons with the lattice
imperfections, and the stabilization of the ferromagnetic ground
state. Nevertheless, the smearing is comparable with the electronic
lifetime acquired due to the exchange of virtual magnons, missing in
the LDA+$U$ picture. Thus, it is likely that disorder is not decisive
in the description of Co/Cu. However, as we shall show, it is crucial
in the description of the spin-wave \textit{damping} in Co/Ir.
\\
The impact of spin fluctuations on the band structure of the
3$\unit{ML}$ Co/Ir(001) film is significant as well, see supplementary
note II in \cite{supplement}. However, the impact of spin
fluctuations is not sufficient to explain the value of
$U = - 1.6\unit{eV}$ necessary for reproducing the experimental magnon
energies. This suggests that the exchange of virtual magnons is an
important but not the only correlation effect missed by the LSDA in
this system. The contrast between these two seemingly similar films
placed on the Cu and Ir substrates reveals the rich many-body physics
yet to be unveiled for nanostructures.


\textbf{Spin-wave energies and Landau damping.} The magnon spectra are
evaluated using time dependent density function theory (TDDFT)
\cite{Buczek2011a,Gorni2018}. The scheme involves the solution of the
\textit{susceptibility Dyson equation}
$ \chi^{\pm}=\chi_{\text{KS}}^{\pm}+ \chi_{\text{KS}}^{\pm}
K_{\text{xc}}\chi^{\pm}$ where $\chi_{\text{KS}}$ is the Kohn-Sham
susceptibility and $K_{\text{xc}}$ represents the exchange-correlation
kernel. Our computational scheme is described in detail elsewhere
\cite{Buczek2011a}.

Figure \ref{fig_Co_Cu_magnons} shows the spin-wave dispersion for
Co/Cu. %As mentioned above, the many-body corrections yield magnon energies in good agreement with experiment but taking into account the disorder is necessary to reproduce the stable FM ground state.
While the LSDA leads to magnon energies which are much too high (as
also discussed in \cite{Chen2017,Zakeri2021}), the inclusion of
electron-magnon interactions and disorder in the system leads to a
good agreement with the experimental results obtained by means of
SPHREELS \cite{Chen2017,Zakeri2021}. Note that the result for the
disordered system in Fig.\ \ref{fig_Co_Cu_magnons} has four magnon
modes while in the experiment only three modes were
observed. Theoretically, this is expected, as the magnetic Co atoms
are spread across four layers in the disordered system,
cf. supplementary note I in \cite{supplement}. The
spectral density of the almost dispersionless mode at
$E\approx300\unit{meV}$ is much smaller than for the other mode, which
we suspect to be the reason for its absence in the experimental
data. The situation is similar for Co/Ir with a few exceptions. First,
the ferromagnetic ground state is stable also after the
\enquote{negative $U$} correction. Second, one may note that the
disordered Co/Ir system (as given in the supplementary note I in \cite{supplement}) actually only has 2.8 layers of Co while the Co/Cu system
has 3 layers. This setup was chosen from several different disorder
configurations as it had the best resemblance to experimental
data. %In particular the system with 3 layers (with Co concentration of 40\% on the surface) had two high energy modes with very weak spectral weight.

% Figure environment removed

The TDDFT is capable of natively describe one of the dominating magnon
decay channels in metallic magnets, the Landau damping. It involves
the collision of the collective spin-wave with single particle
spin-flip, the Stoner excitation. This decay channel is believed to be
dominating in the conducting nanostructures (except for half-metals
\cite{Buczek2009}) and reproduces the SPHREELS data for Co/Cu
\cite{Zakeri2021}.
%For Co/Cu, the theoretical data for the Landau damping even slightly overestimates the magnon linewidth, as reported in \cite{Zakeri2021}, which might be attributed to the failure of the LDA+U to account for the damping of electronic states. (ICH VERSTEHE DIESEN LETZTEN SATZ NICHT.)
However, in the case of Co/Ir(001), the damping rate for optical
terahertz magnonic bands is clearly underestimated
\cite{Zakeri2021}. This hints at another important spin-wave
attenuation mechanism operative in ultrathin magnetic films.
%It should be mentioned at this point that the values for the Landau
%damping are calculated for an ordered system. 

% Figure environment removed

\textbf{Non-Landau magnon decay channels.} Only few studies on other than Landau magnon damping have been published thus far. For instance, some research has been conducted at an analytical level regarding scattering on impurities \cite{Arias1999,McMichael2004,Zakeri2007}. 
%Only little attention has been paid in the literature to other than Landau-like decay channels of spin-waves in ultra-thin magnetic films. 
In this report, we quantify them in an \textit{ab initio}
scheme. Conceivable non-Landau damping channels are depicted
schematically in Fig. \ref{fig_damping_channels} and discussed in
detail in the following.

Let us consider magnon-magnon interaction contribution to the lifetime
first. We discuss the process on the level of the Heisenberg
Hamiltonian. In general, a magnon can decay into one or more magnons
\cite{Kaganov1987}. However, in the absence of the spin-orbit coupling
(SOC), in ferromagnets, the magnons are elementary excitations and
eigenstates of the magnetic system. When there is no excited gas of
spin-wave bosons to interact with, this channel is inactive. The
observation pertains to the TDDFT as well. In half-metals, under weak
SOC assumption, there is no Landau damping and the spin-waves do not
decay \cite{Buczek2009}. An excitation of multiple magnons in a
certain process corresponds to large precession amplitudes of magnetic
moments and is a non-linear effect which cannot be grasped in the
linear response theory. However, the SPHREELS involves a continuous
generation of magnons caused by the electron bombardment of the
sample. Thus, even at low temperatures of the experiment when there
are no thermally excited spin-waves, a magnon gas can form,
facilitating the decay. A direct magnon-to-magnon decay cannot occur
(the magnons being eigenstates) but a given magnon coupled with
another magnon of the gas can decay into new pair of magnons
(\enquote{four-magnon-process}). With SOC present, a magnon can
furthermore decay into two magnon states with the lattice absorbing
the excess angular momentum (\enquote{SOC-} or
\enquote{three-magnon-process}).

In order to quantify these contributions we write the Heisenberg Hamiltonian as follows:
\begin{align}\label{eqn_H}
	H = \sum_{\vb*{k}}\sum_{ij}a_i^\dagger(\vb*{k})T_{ij}(\vb*{k})a_j(\vb*{k}) + H_4 + H_{SOC}
\end{align}
Here, the first term describes the non-interacting bosons with the torque matrix $T$ while $H_4$ is the first term beyond the linearization of the Holstein-Primakoff transformation describing the four magnon processes involving the interaction with the magnon bath, cf. Fig. \ref{fig_damping_channels}~c. $H_\text{SOC}$ represents the three-magnon processes enabled by the SOC, cf. Fig. \ref{fig_damping_channels}~d. For the latter Hamiltonian, the exemplary  Dzyaloshinskii-Moriya interaction (DMI) form is utilized. Both $H_4$ as well as $H_{\text{SOC}}$ are treated as perturbations and the corresponding magnon decay rates can be calculated using Fermi's golden rule
$
	\Gamma_{sc}^{i\rightarrow f}  = 2\pi \rho(E_f) \abs{\matrixel{f_{sc}}{H_{sc}}{i_{sc}}}^2 
$
with $sc\in\{4,\text{SOC}\}$. 
Here, both the initial and final state for the four-magnon process ($\ket{i_4}$ and $\ket{f_4}$) are two-magnon states. For the three-magnon process, the initial state $\ket{i_\text{SOC}} = \mathcal{A}^\dagger\ket{0}$ is a single magnon state ($\mathcal{A}^\dagger$ being magnon creation operator acting on the vacuum state) and the final state $\ket{f_\text{SOC}}$ is a two-magnon state. A detailed exposition of the formalism and the results is given in the supplementary note I \cite{supplement}. The main observation is that the magnon-magnon induced decay rate is orders of magnitude smaller than the Landau damping: $\Gamma_{4} < 5\unit{meV}$ and $\Gamma_{SOC} < 1\unit{meV}$.\\
%To obtain the damping of one specific magnon mode, we sum over all possible final states that satisfy the momentum and energy conservation. % conservation exactly and the energy conservation up to a small energy tolerance $\epsilon$.
%For the four-magnon processes an additional summation over the scattering partner is needed. As there is no full \emph{ab initio} treatment of the SPHREELS process so far, we have to model the occupation probability for the magnon bath created in the SPHREELS process. We assume that all possible magnon modes are generated at an equal constant rate in the experiment but account for the higher decay rates of more energetic modes. The order of magnitude of the resulting magnon damping is independent of the details occupation function. A detailed exposition of the formalism and the results is given in the supplementary material. The major observation is that the magnon-magnon induced decay rate is orders of magnitude smaller than the Landau damping: $\Gamma_{4} < 5$meV and $\Gamma_{SOC} < 1$meV.
Lastly, we show this is not the case with the disorder induced damping \cite{Paischer2021a}. We find that the disorder induced damping is sensitive to the shape of the magnon mode and can differently affect two modes of similar energy and momentum, leading to a \enquote{mode selective damping} and constituting an attractive linewidth engineering approach, cf. supplementary note II in \cite{supplement}. %A spin-wave is associated with a Bloch wave delocalized over the entire lattice. Random impurity atoms break the translational symmetry of the system and the Bloch wave can decay into waves of other momenta with the same energy. In our recent work  \cite{Buczek2016,Paischer2021a,Paischer2021}, we have generalized the CPA formalism to systematically account for the non-diagonal disorder of the Heisenberg Hamiltonian in low dimensional systems featuring complex unit cells in order to describe the spin-wave propagation and scattering in disordered nanostructures. In general, the magnons below 200meV are weakly affected by the disorder and the attenuation increases with the magnon energy. However, we find that the disorder induced damping is sensitive to the shape of the magnon mode and can differently affect two modes of similar energy and momentum, leading to a \enquote{mode selective damping} and constituting an attractive linewidth engineering approach, cf. supplementary material.
%
The total magnon damping for Co/Ir compared with the experimental data
is shown in Fig. \ref{fig_damp}. The Landau damping fully explains the
line width of the low energy magnons. For the high energy optical
modes, the lifetime is clearly underestimated and the inclusion of the
non-Landau damping channels, dominated by the disorder induced
scattering, accounts for a substantial part of the missing line
width. The damping is still underestimated which we belive might
result from the coupling of the transverse magnons to the longitudinal
spin dynamics \cite{Buczek2020} arising due to the SOC but not
included in this work (The results for Co/Cu with weaker SOC and
disorder can be fully explained in terms of the Landau damping
\cite{Zakeri2021}). 
%While the agreement of theory and experiment for magnons with low energy is not changed, the higher energy magnons suffer visibly from non-Landau damping mechanisms, primarily disorder induced scattering. While for the latter, the gap between the experimental and theoretical values is narrowed, there is still a sizable disagreement. There are a few possible reasons for this. First, one can see that the agreement between electronic spectra obtained from LDA+U and the LDA+electron-magnon scattering differ in Co/Ir much more than in the case of Co/Cu. This might be a reason for the fact that Co/Cu is well described while this is not quite the case for Co/Ir. Second, there might yet be other damping mechanisms at play in Co/Ir, most likely due to the higher spin-orbit coupling caused by the Ir substrate. The SOC enters in our approach only through the DMI parameters used for the multi-magnon processes whose impact is minor, as discussed. However, the SOC allows for another damping channel not considered so far, which are electronic excitations between states with equal spin projection.
% Figure environment removed
\newline In summary, using a first principles approach, we uncovered
an intricate picture of spin dynamics and its damping in itinerant 3d
magnetic nanostructures governed by a fine interplay between
non-trivial correlation effects dominated by electron-magnon
scattering and disorder. We believe that the electron-magnon interaction might be a generic effect and an indispensable ingredient in the description of all itinerant magnets. Furthermore, we established a quantitative
hierarchy of attenuation mechanisms dominated by the Landau channel
and disorder with the multi-magnon processes playing a secondary
role. 

S.P. is recipient of a DOC Fellowship of the Austrian Academy of
Sciences at the Institute of mathematics, physics, space research and
materials sciences. I.M. and P.B. gratefully acknowledge financial
support from the DFG-LAV grant \enquote{SPINELS} and HSP grant
\enquote{DEUM}. A.E. acknowledges funding by Fonds zur Förderung der
Wissenschaftlichen Forschung (FWF) Grant No. I 5384. The research of Kh.Z. has financially been supported by DFG through grants ZA 902/7-1 and ZA 902/8-1. We thank A. Marmodoro for interesting discussions and comments.
\bibliographystyle{apsrev4-2}
%\end{thebibliography}
\bibliography{GW,magnetism, Quellen}



\end{document}
