\subsection{Space $S$}
\label{sec:generalform:space}


A possible space can be to set as: 
% 
\begin{equation}
\label{eq:space}
S(t) = <H(t),W> 
\end{equation}
%
composed of a tuple of the set of houses $H(t)$ distributed within the set of towns $W$. 
Here is the set of towns are meant to be 

Resoundingly from Section \ref{sec:general:definitions}, the space $ S $ is composed of a tuple 
$$S ~\equiv~ S(t) ~=~ <H(t)~,~W> $$ corresponding to the set of all houses $H(t)$ and towns $W$ implying that: 
\begin{enumerate}[label=\textbf{S. \arabic*}]
\item 
the space is not necessarily static and particularly the set of houses can vary along the simulation time span 
\item \label{ass:space:staticTowns} the set of towns is constant during a simulation, i.e.\ no town vanishes nor new ones get constructed 
\item 
each town $w \in W$ contains a dynamic set of of houses $H_w \equiv H_w(t)$ 
\end{enumerate}
Furthermore,
\begin{enumerate}[label=\textbf{S. \arabic*}]
\setItemnumber{4}
\item 
each house $h \in H(t)$ is located in one and only one town $w \in W$, i.e. 
$$town(h) = w \in W$$ 
\item 
the location of each house $h \in H_w$ is given in xy-coordinate of the town
$$location(h_{x,y} \in H_w) = (x,y)_w$$
\item 
\label{ass:space:housingUniformLocations}
the houses within a town are uniformly distributed along the x and y axes
\item 
\label{ass:space:inhabitable}
a house never get demolished and remains always inhabitable 
\end{enumerate} 
