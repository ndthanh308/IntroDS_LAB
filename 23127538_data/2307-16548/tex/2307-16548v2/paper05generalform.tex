\section{General form}
\label{sec:generalform}


\subsection{General definitions}
\label{sec:general:definitions}

This article is concerned with proposing a terminology for the mathematical specification of demographic ABMs based on a single-clocked fixed-step simulation formally defined via the tuple 
\begin{equation}
  \label{eq:abmDefinition}<\alpha_{sim},\mathcal{F},\mathcal{M},\mathcal{M}^{t_0},\mathcal{E},\mathcal{A},\mathcal{A}^{t_0}>  
\end{equation}
where 
\begin{itemize}
\item 
$\alpha_{sim} = (\Delta t, t_0, t_{final}, \alpha_{meta})^T$:
simulation parameters including a fixed step size and final time-step after which the simulation process ends  
\begin{itemize}
\item $\alpha_{meta}$:
Implementation-dependent simulation parameters, e.g.\ simulation seed for random number generation 
\end{itemize}
%
\item 
$\mathcal{F}$: a finite set of elementary population features, cf.\ Section \ref{sec:generalform:features}
% 
\item 
$\mathcal{M}$:  a mathematical model representation corresponding to (demographic) ABM, cf. Section \ref{sec:generalform:abm}  
%
\item 
$\mathcal{M}^{t_0}$: a mathematical model that evaluates the initial model state at time $t_0$, cf.\ Section \ref{sec:generalform:initialstate}  
% 
\item 
$\mathcal{E}$: a finite set of events that takes place in the population. They transient the states of the (featured sub-)population(s), cf.\ Section \ref{sec:generalform:events}
\end{itemize}
%
Establishing a mathematical model that corresponds to reality till the tiniest details is impossible. Therefore, a set of (typically non-realistic) assumptions has to be included in order to simplify the model specification process:
%
\begin{itemize}
    \item 
$\mathcal{A}$: a set of model assumptions that should not be violated during the simulation course between $t_0$ and $t_{final}$, cf.\ Section \ref{sec:generalform:Assumptions}
\item 
$\mathcal{A}^{t_0}$: a set of initial model assumptions that should not violate the initial model state $\mathcal{M}^{t_0}$, cf.\ Section \ref{sec:generalform:initialAssumptions} 
\end{itemize}



\subsection{Population features $~\mathcal{F}~$} 
\label{sec:generalform:features}


\underline{$\mathcal{F}= \{f_1,f_2,f_3,...f_k\}$} describes a finite set of elementary features each distinguishes a featured sub-population $$P_f(t) \subseteq P(t)  ~ , ~ f \in \mathcal{F}$$
as defined in Equations \ref{eq:fp} and \ref{eq:Pf}, cf.\ Section \ref{sec:terminologies:fsubpopulations} for examples of population features.



\subsection{Demographic agent-based model $\mathcal{M}$}
\label{sec:generalform:abm}

\underline{$\mathcal{M}$} corresponds to a demographic ABM formally defined as:  
\begin{align}
\label{eq:abmModel}
& \mathcal{M} \equiv \mathcal{M}^t \equiv \mathcal{M}(P,S,\alpha,D,t)   \\
\label{eq:abmModelInitialConditions}
& \text{ associated with } \mathcal{M}(P(t_0),S(t_0),\alpha,D(t_0),t_0) =  M^{t_0} 
\end{align}
%
where 
%
\begin{itemize} 
\item \underline{$P \equiv P(t)$}: a given population of agents (i.e.\ individuals) at time $t$ evaluated via the model $\mathcal{M}(t)$
%
\item 
\underline{$S \equiv S(t)$}: 
the space on which individuals $p \in P$ are operating  
\item 
\underline{$\alpha$}: 
time-independent model parameters %, cf.\ Section \ref{sec:parametersdata:parameters} 
% 
\item 
\underline{$D(t)$}: 
input data integrated into the model as (possibly smoothed) input trajectories %, cf.\ Section \ref{sec:parametersdata:data}
%
\end{itemize} 
%
In \cite{MiniDemographicABMjlMisc}, the space is set as: 
\begin{equation}
\label{eq:spaceExample}
S(t) = <H(t), W> 
\end{equation}
where $H(t)$ stands for a set of houses distributed within the the set of towns $W$. 


\subsubsection*{Featured sub-populations (via $ \mathcal{M}_{f^*} ~,~ f^* \in \bigcup \mathcal{F}~$)} 

This subsection concerned with featured sub-populations used to distinguish sub-populations needed for specification of the transient processes via events $\mathcal{E}$ within the agent-based modeling simulation process. 
A featured sub-population $P_f, f \in \mathcal{F}$ is evaluated by the sub-model $M_{f}$ concerned only with the elementary features $f$:
%
\begin{equation}
\label{eq:Mf}
f(\mathcal{M}) \equiv f(\mathcal{M}^t) \equiv \mathcal{M}_f^t = \mathcal{M}_{f}(P_{f},S,\alpha,D,t)     
\end{equation}
%
evaluating or predicting the sub-population 
\begin{equation}
\label{eq:generalform:fP}
f(P(t)) \equiv P_f(t) ~\text{ s.t.\ }~ \forall p \in P_f(t) \implies f(p) = true     
\end{equation}
For a given closed set of elementary features as given in Equation \ref{eq:closedSubsetElemFeatures}, the overall population is the union of these elementary features, 
cf.\ Equation \ref{eq:PUnionPf}. 
In that case, the comprehensive model $\mathcal{M}$ constitutes of the sum of its elementary featured sub-models:
%
\begin{equation}
\label{eq:MSumMf}
\mathcal{M} \equiv \sum_{f' \in \mathcal{F'}} \mathcal{M}_{f'}  
\end{equation}
% 
Note that this terminology extends to composite features $\bigcup \mathcal{F}$ as well by rather assuming $f \in \mathcal{\bigcup F}$ in Equation \eqref{eq:Mf}.



\subsection{Initial population and space (via $\mathcal{M}^{t_0}$)}
\label{sec:generalform:initialstate}

\underline{$\mathcal{M}^{t_0}$} is a model that evaluates the initial population $P(t_0)$ and the initial space $S(t_0)$ at a proposed simulation start time $t_0$ via Equation \eqref{eq:abmModelInitialConditions}. 
Consequently, the state of the model, i.e.\ $P(t)$ and $S(t)$, depends on the initial conditions expressed as $M^{t_0}$.  \\ 

Additionally, the initial state of the model $M^{t_0}$ combined with the elementary population features evaluate elementary as well as composite featured sub-populations:  
\begin{equation}
\label{eq:Mft0}
f(M^{t_0}) = M_f^{t_0}, \forall f \in \bigcup \mathcal{F}    
\end{equation}
Consequently, both the corresponding initial population and featured sub-populations:
\begin{equation}
\label{eq:Pt0Pft0}
P(t_0) \text{ and } P_f(t_0), ~~~ \forall f \in \mathcal{F}
\end{equation}
%
 are specified, e.g.\ the initial ratio of females and males, the age distribution of the population, the spatial distribution of the population, among others. \\ 

In the model described by the package implementing the documentation given in \cite{MiniDemographicABMjlMisc}, the initial space is set as  
\begin{equation}
\label{eq:St0}
    S(t_0) ~=~ <H(t_0)~,~W> 
\end{equation}
composed of a pair of an initial set of houses $H(t_0)$ distributed within a set of towns $W$, in conformance with the space setting in Equation \eqref{eq:spaceExample}. 


\subsection{Events $\mathcal{E}$}
\label{sec:generalform:events}


\underline{$\mathcal{E} = \{e_1, e_2, e_3, ..., e_n\} $} is a finite set of events, each of which transients a particular featured sub-population $P_{f^*}(t)$ is evaluated by the sub-model:  
\begin{equation}
\label{eq:fstarM}
f_1^*(\mathcal{M}^t) \equiv \mathcal{M}_{f_1^*}(P_{f_1^*},S,\alpha,D,t) \text{ with } f_1^* \in \bigcup \mathcal{F} 
\end{equation}
to another modified sub-population predicted by 
\begin{equation}
f_2^*(M^{t+\Delta t}) \equiv \mathcal{M}_{f_2^*}(P_{f_2^*},S,\alpha,D,t + \Delta t)    
\end{equation}
% 
Formally,
\begin{equation}
\label{eq:events}
e(\mathcal{M}_{f_1^*}^{t}) = \mathcal{M}_{f_2^*}^{t + \Delta t} ~~~\text{ for some } \{f_1^* , f_2^*\} \in \bigcup \mathcal{F} ~~\text{ where } e \in \mathcal{E}    
\end{equation}
%
The application of all events transients the model to the next state: 
% 
\begin{equation}
\label{eq:stepping}
\prod_{i=1}^n e_i(\mathcal{M}^t) = \mathcal{M}^{t + \Delta t} 
\end{equation}

As an example, the model specification provided in \cite{Elsheikh2023a} specifies the set of events as
$$
\mathcal{E} = \{ ageing, birth, death, divorce, marriage \} 
$$
The death event specification transforms a given population of alive individuals to the same population with some individuals possibly dead: 
\begin{align}
\label{eq:event:deaths}
    death \left( P_{isAlive }^t \right)  ~=~  P_{isAlive - age=0}^{t+\Delta t}  ~\bigcup~  P_{just(\neg isAlive)}^{t+\Delta t} 
    % \\ & ~=  \left(P_{isAlive}^{t+\Delta t} - P_{age=0}^{t+\Delta t} \right) & \cup~ &  P_{\neg isAlive}^{t+\Delta t} - P^t_{isAlive}
\end{align}
The first phrase in the right hand side stands for the alive population except neonates and the second stands for those who just became dead. 



\subsection{Assumptions $\mathcal{A}(\mathcal{M}^{[t_0,t_{final}]})$}
\label{sec:generalform:Assumptions}


\underline{$\mathcal{A} = \{a_1, a_2, a_3, \dots, a_s\}$} is a finite set of assumptions each represented as a Boolean condition  
\begin{equation}
\label{eq:assumptions}
    a_i(\mathcal{M}^t) = 
    a_i(\mathcal{M}(P(t),S(t),\alpha,D(t),t)) = b \in \{true,false\}  
\end{equation}
% 
if 
$a_i(M^t) = true$ one says that assumption $a_i$ satisfies $M^t$ expressed as follows: 
\begin{equation}
\label{eq:assumptionSatisfy}
a_i \vdash \mathcal{M}^t  ~~~,~~~ \forall i \in \{ 1,2,\dots,s \} 
\end{equation}
A particular assumption $a_j$ violates $\mathcal{M}$ if at any simulation time point 
$$t \in [t_0,t_{final}]_{\Delta t} \implies a_j(M^t) = false$$ 
expressed as: 
\begin{equation}
\label{eq:assumptionViolation}
a_j \nvdash \mathcal{M}^{[t_0,t_{final}]_{\Delta t}}   ~~~,~~~ \text{ where } j \in \{1,2,\dots,s\}    
\end{equation}
%
That is, the violation of an assumption may also depend on the simulation parameter resolution, i.e.\ $\Delta t$. 
Typically, an assumption would be usually rather concerned with a featured sub-population, i.e.\ 
\begin{equation}
\label{eq:assumptionRelatedToSubpopulation}
    a_i(\mathcal{M}^t_{f^*}) = 
    a_i(\mathcal{M}(P_{f^*}(t),S(t),\alpha,D(t),t)) = b \in \{true,false\}  ~,~f^* \in \bigcup \mathcal{F}
\end{equation}
A given assumption could be purely related to the space, the population (a sub-population), or both the space and the (or a sub-)population.  
% 
Examples of assumptions are listed as follows: 

\begin{center}
\graybox{(Space-related assumption) 
Once a new house is built (e.g.\ to host an adult moving out of his parent's house), it never gets demolished and remains always inhabitable 
\begin{equation}
\label{eq:ass:HouseAlywasRemain}
    \text{ if } h \in H(t) \text{ and } t < t'  ~\implies h \in H(t')
\end{equation}
}
% 
\graybox{
(Population-related assumption)
Only a married female\footnote{This was assumed in the lone parent model and obviously the marriage / partnership concept needs to be re-defined in the context of realistic studies} under age of 45 gives birth
\begin{align}
\label{eq:ass:pop:onlyMarriedGivesBirth}
f \in & F_{just(gaveBirth)}^t \implies \nonumber \\ & 
isMarried(f^t) = true \text{ and } age(f^t) < 45
\end{align}
}
\graybox{(mixed-assumption) 
A neonate's house is his mother house:
\begin{align}
p \in~ &  P_{age=0}   
\implies \nonumber \\ 
& house(p) = house(mother(p))    
\end{align}
}  
\end{center}
In the accompanied model specification, assumptions are numbered and labeled according to their types, namely,  $a_{s*}$ for space-, $a_{p*}$ population-related assumptions and $a_{*}$ for mixed assumptions. 

\subsection{Assumptions $\mathcal{A}^{t_0}(\mathcal{M}^{t_0})$}
\label{sec:generalform:initialAssumptions}


\underline{$\mathcal{A}^{t_0} = \{a^0_1, a^0_2, a^0_3, \dots, a^0_r\}$} is a finite set of assumptions concerned only with initial assumptions at the initial model state at time $t_0$. All equations from the previous sections apply only to the initial model state $M^{t_0}$. For instance, Equation \ref{eq:assumptionSatisfy} maps to 
\begin{equation}
\label{eq:t0assumptionSatisfy}
a^0_i \vdash \mathcal{M}^t_0  ~~~,~~~ \forall i \in \{ 1,2,\dots,r \} 
\end{equation}
% 
Examples of initial assumptions are listed as follows:

\begin{center}
\graybox{
All adult persons have no parents
\begin{equation}
\label{eq:ass:adultHasNoParents}
    p \in P_{age \geq 18}^{t_0} \implies \nexists q \in P^{t_0} \text{ s.t.\ } q \in parents(p)
\end{equation}
}
% 
%\begin{equation}
%\label{eq:ass:nograndChildren}
%p \in P^{t_0} \text{ s.t.\ } grandchild(q) = p     
%\end{equation}
%}
% 
\graybox{
All children have alive parents, i.e. 
\begin{equation}
\label{eq:ass:noOrphanChild} 
p \in P_{age<18}^{t_0} \implies parents(p) \subset P_{isAlive}^{t_0}
\end{equation}
}
\end{center}
% 
with initial assumptions labeled as $a^0_{*}$. 



