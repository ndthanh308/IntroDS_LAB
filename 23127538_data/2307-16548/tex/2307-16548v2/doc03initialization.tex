\section{Model initialization $\mathcal{M}^{t_0}$} 
\label{sec:initialization}


\setvalue{\popassnum}{6} \todo{to remove}


%This section provides a detailed description of the initial model state $M^{t_0}$. % given that Section \ref{sec:parametersdata} demonstrates potential case studies specifying possible simulation parameter values for $t_0,~t_{final} \text{ and } \Delta t$. 
%Further initialization assumptions are proposed by demand. 
%, distinguished by the labels \boldmath\initassnum{*} for initial population assumptions or \textbf{S0} for initial space assumptions.




\subsection{Initial population size and distribution $|P^{t_0}_{w}|$}  
\label{sec:initialization:initialPopulation} 

The initial population size is given by the parameter $\alpha_{initialPop}$, cf.\ Section \ref{sec:model:parameters}. 
The matrix $M$ and assumption \sassnum{\ref{ass:space:houseInArbitraryTown}} provide together a stochastic ad-hoc estimate of the initial population distribution within the UK. 
That is, the initial population size of a town $w \in W$ is estimated as
\begin{equation}
\label{eq:Pwt0}
|P_w(t_0)| \approx \left\lceil \alpha_{initialPop} \times M_{y,x} / 48 \right\rceil  \text{~~ where ~~location}(w) = (x,y) 
\end{equation}
where 48 is the number of nonzero entries in $M$. 


\subsection{Gender $P^{t_0} = M^{t_0} \cup F^{t_0}$}  
\label{sec:initialization:gender} 

%The parameter $\alpha_{initialPop}$ specifies the size of initial population, cf.\ Section \ref{sec:parametersdata} for potential values. 
The gender ratio distribution is specified via the following (clearly non-realistic) population-related assumption %\footnote{In reality, worldwide there is a tiny higher number of females births over males births}: 
%
%\begin{enumerate}[label=\textbf{P. \arabic*}]
%\setItemnumber{\popassnum}
%\item 
\graypopass{
\label{ass:pop:gender}
An individual can be equally a male or a female, i.e.
\begin{equation}
\label{eq:genderProbability}
Pr(p \in P^{t_0}_{male}) \approx 0.5
\end{equation}
}
This assumption is employed in the specification of the initial population as well as the specification of the birth event, cf.\ Section \ref{sec:events:births}, and this it is not classified as an initial assumption. 

%Gender assignment to the initial population is established according to a uniform distribution, i.e.
%\setvalue{\popassnum}{7}
%This is not only an initial assumption but also relates to gender ratio in births event, cf.\ Section \ref{sec:events:births}.


\subsection{Age distribution $P^{t_0} = \bigcup_{r} P_{age=r}^{t_0}$}

The proposed non-negative age distribution of population individuals in years follows a normal distribution: 
\begin{equation}
\label{eq:ageDistribution}
Dist \left( \frac{age(P^{t_0})}{N_{\Delta t}} \in \mathbb{Q}^{\alpha_{initialPop}}_+ \right) ~\propto~ \left| \lfloor \mathcal{N}(0,  \frac{100}{4} \cdot N_{\Delta t}) \rfloor \right|
\end{equation}
where $\mathbb{Q}_+$ stands for the set of positive rational numbers and  $\mathcal{N}$ stands for a normal distribution with mean value 0 and standard deviation depending on 
\begin{equation}
\label{eq:NDeltaT}
    N_{\Delta t} ~=~ \left\{ 
\begin{array}{ll}
~ \dots \\
~ 12 &  \text{ if }~ \Delta t = month \\ 
365 & \text{ if }~ \Delta t = day \\ 
365 \cdot 24 & \text{ if }~ \Delta t = hour \\ 
~ \dots 
\end{array}
\right. 
\end{equation}
A possible outcome of the distribution of ages in an initial population of size 1,000,000 is shown in Figure \ref{fig:initialPopulationAges}.
% Figure environment removed
Obviously, in reality, the shape of the age's distribution depends on the initial simulation time $t_0$.








\subsection{Partnership $P_{married}^{t_0} = M_{married}^{t_0} \cup partner(M_{married}^{t_0})$}\todo{Check this}
\label{sec:initialization:partnership}

Initially the following population assumption concerned with marriage age is considered
\graypopass{
\label{ass:pop:marrigeaAge} 
A married person is an adult person  
\begin{align}
p \in P_{married} \implies age(p) \geq 18 
\end{align}
}

%\end{enumerate}
%\setvalue{\popassnum}{8}
%
The ratio of married adults (males or females) is statistically approximated according to 
\begin{equation}
\label{eq:initialMarriedProb}
Pr(p \in P_{isSingle ~\cup~ \neg adult }^{t_0}) \approx 1 - 2 \cdot \alpha_{startMarriedRatio} 
\end{equation}
%
That is, $\alpha_{startMarriedRate}$ is the ratio of married males (females) among adults. 
% 
Partnership initialization is established according to Algorithm \ref{alg:partnerInitialization}. Initially lines 1-3 initialize \begin{enumerate}
    \item the set of males randomly selected for marriage (line \ref{line:alg:partner:line1})
    \item the set of females eligible to marriage (line 2)
    \item  and the number of female candidates for marriage each male has to select from\footnote{This is just an abstract algorithm that does not necessarily reflect the reality} (line 3). 
\end{enumerate} 
%
For every male selected for marriage, a corresponding female is selected (lines 4-11): 
\begin{enumerate}
    \item a set of candidate females is initialized (line 5) 
    \item for every candidate female (line 6), a weight is calculated according to a weight function based on age difference in Equation \eqref{eq:marriageWeight} (line 7) 
    \item 
    a female partner is selected according to a random weighted function (line 9) 
    \item 
    the set of females illegible for marriage is updated (line 10)
\end{enumerate}
% 
Using the calculated weights, a partner 
\begin{algorithm}
\label{alg:partnerInitialization}
\caption{Partnership initialization}
\begin{algorithmic}[1]
\State \label{line:alg:partner:line1}
Select $M_{married}$ randomly according to Equation \eqref{eq:initialMarriedProb} 
\State
Set $F_{marriageEligible}^{t_0} = F_{marEli}^{t_0} = F_{adult ~\cap~ single}^{t_0}$
\State 
Set $n_{candidates} = max\left( \alpha_{maxNumMarrCand} ~,~ \frac{|F_{marEli}|}{10}\right)$ 
\For{$m \in M_{married}$} 
\State Set $ F_{candidates} = random(F_{marEli}, n_{candidates})$ 
    \For{$f \in F_{candidates}$} 
        \State \label{line:alg:partner:weightFunc} Set $w_{m,f} = weight(m,f) = ageFactor(m,f)$ where
    \begin{align}
    \label{eq:alg:ageFactor}
        & ageFactor(m,f)  =~  \nonumber \\ & ~~~ \left\{ 
\begin{array}{ll}
~ 1 / (age(m) - age(f) - 5 + 1) &  \text{ if }~ age(m) - age(f) \geq 5 \\ 
~ -1 / (age(m) - age(f) + 2 - 1) &  \text{ if }~ age(m) - age(f) \leq -2 \\ 
~ 1 \text{ ~~~~otherwise } \\ 
\end{array}
\right.   
\end{align}
    \EndFor
\State Set 
\begin{align}
& f_{partner(m)}  =  weightedSample(F_{candidates}, W_m) 
\nonumber \\ 
& ~~~~~ \text{ where } ~ W_m = \left\{ w_i : w_i = weight(m,f_i) ~,~ f_i \in F_{marEli} \right\}     
\end{align}
\State 
Set $F_{marEli} = F_{marEli} - \{ f_{partner(m)}\}$
\EndFor 
\end{algorithmic}
\end{algorithm}








\subsection{Children and parents} 
\label{sec:initialization:parents}


The following assumptions are assumed only in the context of the initial population: 
% 
%
%\begin{enumerate}[label=\textbf{P. \arabic*}]
%\setItemnumber{\popassnum}
%\item 
\grayinitass{
\label{ass:init:adultsNoParents}
All adult persons have no parents
\begin{equation}
\label{eq:ass:adultHasNoParents}
    p \in P_{age \geq 18}^{t_0} \implies \nexists q \in P^{t_0} \text{ s.t.\ } q \in parents(p)
\end{equation}
} 
%\begin{equation}
%\label{eq:ass:nograndChildren}
%p \in P^{t_0} \text{ s.t.\ } grandchild(q) = p     
%\end{equation}
%}
% 
\grayinitass{
\label{ass:init:childrenAliveParents}
All children have alive parents, i.e. 
\begin{equation}
\label{eq:ass:noOrphanChild} 
p \in P_{age<18}^{t_0} \implies parents(p) \subset P_{alive}^{t_0}
\end{equation}
}
% 
Based on the previous two assumptions, one can may also deduce that there is no individual in the initial population who has a grandpa or grandma. 
% 
\grayinitass{
\label{ass:init:ArbitraryAgeDiffSiblings}
There is no age difference restriction among siblings, i.e.\ age difference can be less than 9 months
}
% 
Moreover, the following population assumption proposes conditions for women who can give birth:
\graypopass{
\label{ass:pop:marriedGivesBirth}
Only a married female\footnote{This was assumed in the lone parent model and obviously the marriage / partnership concept needs to be re-defined in the context of realistic studies} under age of 45 gives birth
\begin{align}
\label{eq:ass:pop:onlyMarriedGivesBirth}
f \in & F_{just(gaveBirth)}^t \implies \nonumber \\ & 
f \in F_{married}^t \text{ and } age(f) < 45
\end{align}
}
% 
Assumptions \passnum{\ref{ass:pop:marriedGivesBirth}} and \passnum{\ref{ass:pop:marrigeaAge}} imply that only an adult person can become a parent. 
Children are assigned to married couples as parents in the following way. 
For any child $c  \in P_{age<18}^{t_0}$, the set of potential fathers is established as follows: 
\begin{align}
\label{eq:fathershipCanddiates}
M_{candidates} = ~ & \{~ m \in M_{married} \text{ s.t.\ } \nonumber  \\ &   min\left(~age(m),age(wife(m))~\right) \geq age(c) + 18 + \frac{9}{12} \text{ and } \nonumber \\ & 
age(wife(m)) < 45 + age(c) ~\}  
\end{align}
out of which a random father is selected for the child: 
\begin{align*}
    father(c) =  & random(M_{candidates}) \text{ and } \\
    & mother(c) = wife(father(c)) 
\end{align*}





\subsection{Housing assignments}
\label{sec:initialization:housing}


Before specifying the housing assignments to initial population, related model assumptions are listed:  
\grayass{
\label{ass:mixed:homeless} There are no homeless individuals:
\begin{equation}
\label{eq:mixed:pop:noHomeless}
    \text{ if } p \in P^t_{alive} ~\implies~ house(p^t) \in H(t)
\end{equation}
}
% 
The last line in the previous equations indicates that a dead person does not need to be associated to any house. 
%
Furthermore, there is no classification of houses according to their capacities:
\grayass{
\label{ass:mixed:numOfOccupants}
A house can be occupied by an arbitrary number of individuals
}
%
Moreover, the following assumption is considered for housing association:
\grayinitass{
\label{ass:init:familyTogether}
A family, i.e.\ a married male and female, and their children live together, that is 
\begin{align}
\label{eq:ass:familyTogether}
    |occupants(h^{t_0})| > 1 & \text{ with } p,q \in occupants(h^{t_0}) \text{ and } p \neq q ~\implies~ \nonumber \\ &  p \in firstDegRelatives(q)  
\end{align}
} 
% 
The previous assumption implies that a single person in the initial population shall be assigned a house alone.  
% 
The assignment of newly established houses to initial population considers the assumptions \sassnum{\ref{ass:space:selectOrCreateEmptyHouse}} and \sassnum{\ref{ass:space:houseInArbitraryTown}}. That is, 
the location of new houses in $H(t_0)$ is specified according to to Equation \eqref{eq:ass:hometowLocation}, each is assigned to a single person or a family as previously stated. 
%
%Assignments of new houses to the initial population is conducted as follows: 
%\begin{align}
%    & p^{t_0} \in P_{isSingle}^{t_0} \implies occupants(house(p)) = \{ p \} \nonumber \text{ otherwise } \\  
%    & m^{t_0} \in M_{married}^{t_0} \implies \nonumber \\ &~~~~~ occupants(house(m)) = \{ m , wife(m) \} \cup children(m) 
%\end{align}
%Overall, all houses in $H(t_0)$ are occupied, i.e.\ $|occupants(h^{t_0})| \geq 1$. 
