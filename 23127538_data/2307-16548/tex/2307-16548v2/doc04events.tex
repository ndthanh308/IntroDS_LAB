\section{Events}
\label{sec:events}


This section provides compact algorithmic specification of events serving as a comprehensive demonstration of the proposed terminology presented in \cite{Elsheikh2023a}.   
%



\subsection{Execution order of events}
\label{sec:events:order}

Despite Equation \eqref{eq:example:events} specifying the set of events, the considered events are just alphabetically listed without enforcing a certain appliance order, except for the ageing event which should proceed any other events.
That is,
$$e_1 = ageing$$ 
This is reasonable since if an event s.a.\ death or birth proceeds ageing, then this implies that the population size and features may not remain consistent with input data $D(t)$, model parameters $\alpha$ and initial model states $\mathcal{M}^{t_0}$. \\ 

%
The execution order of the rest of the events as well as the order of the agents subject to such events, whether sequential or random, remains an implementation detail.  
Nevertheless, since many of the events are following a random stochastic process, probably, the higher the resolution of the simulation becomes (e.g.\ weekly step-size instead of monthly, or daily instead of weekly), the less influential the execution order of the events becomes. \\ 

The combination of event transitions on the population does not preserve the initial assumption \initassnum{\ref{ass:init:familyTogether}} regarding the occupants of houses. Therefore, assumption regarding the occupants of houses is further relaxed to: 
\grayass{
\label{ass:mixed:housingKinship} Any two individuals living in a single house are either a 1-st degree relatives, step-parent, step-child, step-siblings or partners. An exception to the previous assumption occurs when an orphan's oldest sibling is married in which case they also lives together as a family.  
}







\subsection{Ageing}
\label{sec:events:ageing}

The ageing process of a population can be described as follows:
%
\begin{equation}
\label{eq:event:ageing}
ageing \left(
P_{alive(age=a)}^t\right) 
= 
P_{alive(age=a+\Delta t)}^{t+\Delta t} ~~ ,~~ \forall a \in \{0, \Delta t, 2 \Delta t, ... \}
\end{equation}
%
That is, the age of any individual as long as he remains alive is incremented by $\Delta t$ for each simulation step. Furthermore, the following assumption concerned with individuals becoming adults is considered: 
\grayass{
\label{ass:mixed:adultToOwnHouse}
In case a teenager becomes an adult and he/she is not the oldest orphan, he/she gets re-allocated to an empty house in the same town.
}
Formally:
%
\begin{align}
\label{eq:AdultMovesToAnEmptyHouse}
ageing & \left(P_{alive(age = 18 - \Delta t) ~\cap~ \neg \left( orphan ~\cap~ oldestSibling \right) }  \right)  =  \nonumber \\ 
 & P_{alive(age = 18 ) ~\cap~ livesAlone } 
\end{align}
Moreover, 
\begin{align}
\label{eq:AdultMoveToTheSameTown}
 \text{If } p \in P^{t}_{alive(age=18)} \implies 
 town(p) = pre(town(p))   
\end{align}
The re-allocation to an empty house should be according to assumption \sassnum{\ref{ass:space:selectOrCreateEmptyHouse}}.



\subsection{Births}
\label{sec:events:births}

For simplification purpose, from now on, it is implicitly assumed (unless specified) that only the alive population is involved in event-based transition of population features. 
% 
%
Given assumption \ref{ass:pop:marriedGivesBirth}, let the set of reproducible females be defined as:
% 
\begin{align}
F_{reproducible} & ~=~ F_{married ~\cap~ age < 45} ~\bigcap~ \nonumber \\ &  F_{age(youngestChild) > 1 ~\cup~ \neg hasChildren} 
\end{align} 
%
That is, the set of all married females in a reproducible age and either do not have children or those with youngest child older than one.  
%
The birth event produces new children from reproducible females specified as follows
% 
\begin{align}
birth & \left(F^t_{reproducible} \right) ~=~ \nonumber \\ 
&~  \left( F_{reproducible}^{t+\Delta t} ~ - ~ F_{just(reproducible)}^{t+\Delta t}  \right) ~ \bigcup  \nonumber \\ 
&~ F_{just(\neg reproducible) }^{t+\Delta t}  ~\bigcup \nonumber \\ 
 &~  P_{age=0}^{t+\Delta t}
\end{align} 
The previous equation states that the birth event transients the individuals within the set of reproducible females to: 
\begin{itemize}
\item those females who remained reproducible (first line in the rhs) 
\item those who just gave births (second line) and 
\item new neonates are produced (third line)  
\end{itemize}
Note that the following set is subtracted from those who remained reproducible 
\begin{equation}
F_{just(reproudicble)} ~=~ 
F_{just(married)} ~\cup~ 
F_{married(age(youngestChild)=1)}    
\end{equation}
and (given Assumption \passnum{\ref{ass:pop:marriedGivesBirth}}) those who became non-reducible include also who got divorced and widowed 
%
\begin{align}
& F_{just (\neg reproducible)} ~=~ \nonumber \\  
    & ~~~ F_{age(youngestChild)=0} ~ \cup ~ F_{just(\neg married)} ~\cup ~ F_{married(age=45)}    
\end{align}
% 
%Unsurprisingly, the following assumption related to the house assignment of neonates is considered:
%\grayass{
%A neonate's house is his mother house:
%\begin{align}
%p \in~ &  P_{age=0}   
%\implies \nonumber \\ 
%& house(p) = house(mother(p))    
%\end{align}
%}  
%
Employing assumptions \assnum{\ref{ass:mixed:housingKinship}} and \assnum{\ref{ass:mixed:adultToOwnHouse}}, one can deduce that a neonate is assigned to his parents house, i.e. 
\begin{align}
p \in~ &  P_{age=0}   
\implies \nonumber \\ 
& house(p) = house(mother(p))    
\end{align}
%
The yearly-rate of births produced by the sub-population $F_{reproducible,(age=a)}^t$ i.e.\ reproducible females of age $a$ years old with actual simulation time $t$, depends on the yearly-basis fertility rate data:  
%
\begin{equation}
    R_{birth,yearly} (F_{reproducible,(age=a)}^t)  ~\propto~ D_{fertility}(a-16,currentYear(t))   
\end{equation}
%
This implies that the instantaneous probability that a reproducible female $f \in F_{reproducible}^t$ gives birth to a new individual $p \in P^{t + \Delta t}_{age=0}$ depends on $D_{fertility}(a-16,currentYear(t))$ and is given by Equation \eqref{eq:instantaneous}, cf.\ Appendix \ref{sec:terminology:probability} for a conceptual review regarding rates and instantaneous probabilities. 


\subsection{Deaths} 
\label{sec:events:deaths}




The death event transforms a given population of alive individuals as follows: 
\begin{align}
\label{eq:event:deaths}
    death \left( P_{alive }^t \right)  ~=~  P_{alive - age=0}^{t+\Delta t}  ~\bigcup~  P_{just(\neg alive)}^{t+\Delta t} 
    % \\ & ~=  \left(P_{alive}^{t+\Delta t} - P_{age=0}^{t+\Delta t} \right) & \cup~ &  P_{\neg alive}^{t+\Delta t} - P^t_{alive}
\end{align}
where
\begin{itemize}
    \item the first phrase in the right hand side stands for the alive population except neonates as they don't belong to $P^t_{alive}$
    \item the second phrase stands for those who just became dead
\end{itemize}
The following simplification assumptions are considered: 
%
\graypopass{
\label{ass:pop:adoption} 
No adoption or parent re-assignment to orphans is established after their parents die 
\begin{align}
    \label{eq:noAdoption}
    p \in P_{child} \text{ and }  parents(p) & \subset P_{\neg alive} \implies \nonumber \\ 
    & post(parents(p)) \subset P_{\neg alive}
\end{align}
} 
\grayass{
\label{ass:mixed:deadHaveNoHouse}
Those who just became dead they leave their houses, i.e.\ 
\begin{align}
\label{eq:deadsLeaveToGrave}
\text{ if } p \in P_{just(\neg alive)}^{t+\Delta t}~  &
\text{ and }~ pre(house(p)) = h 
\nonumber \\ 
\implies ~ & p \notin P^{t+\Delta t}_h  ~\text{ and }~ house(p) \notin H^{t + \Delta t}    
\end{align}
}
Note that $P_h$ stands for the population (or occupants) of a house $h$. 
% 
The amount of population deaths depends on the yearly-rate given by:  
\begin{align}
\label{eq:probDeath}
R_{death,yearly}&(p \in P) = \alpha_{basicDeathRate} ~+~ \nonumber \\ & \left\{   
\begin{array}{cc}
   \left( e^{\frac{age(p)}{\alpha_{maleAgeScaling}}} \right)  \times \alpha_{maleAgeDeathRate}   \text{ if } male?(p) \\
       \left( e^{\frac{age(p)}{\alpha_{femaleAgeScaling}}} \right)  \times \alpha_{femaleAgeDeathRate}
    \text{ if } female?(p) \\
\end{array}
\right. 
\end{align}
%
from which instantaneous probability of the death of an individual is derived as illustrated in Appendix \ref{sec:probability}.






\subsection{Divorces} 
\label{sec:events:divorces}

The divorce event causes that a subset of married population becomes divorced: 
\begin{align}
& divorce (M_{married}^t) ~ = ~  \nonumber \\ 
 &~~~~~   M_{married - just(married)}^{t+\Delta t}  ~ \bigcup ~ M_{just(isDivorced)}^{t + \Delta t}  %~=  \\ 
% & ( P_{married}^{t+1} - P_{single}^t ) ~ \cup ~ (P_{divorced}^{t+1} - P_{married}^t) 
\end{align}
The first phrase in the right hand side refers to the set of married individuals who remained married excluding those who just got married.
The second phrase refers to the population subset who just got divorced in the current iteration. 
Note that it is sufficient to only apply the divorce event to  either the male or female sub-populations. 
After divorce takes place, the housing's assignment is specified according to the following assumption: 
%
\grayass{
\label{ass:mixed:divorceMaleToOwnHouse}
Any male who just got divorced moves to an empty house within the same town (in conformance with Assumption \sassnum{\ref{ass:space:selectOrCreateEmptyHouse}}):
\begin{align}
 m \in & M_{just(isDivorced}^t) \implies \nonumber \\  
& pre(location(m)) \neq location(m) ~\text{ and }~ livesAlone(m) ~\text{ and }~ \nonumber \\  
& town(m) = pre(town(m))  
\end{align}
}
%
The re-allocation to an empty house is in conformance with assumption \sassnum{\ref{ass:space:selectOrCreateEmptyHouse}}.
The amount of yearly divorces in married male populations can be estimated upon the yearly divorce rate given by
\begin{align}
\label{eq:prob:divorce}
 R_{divorce,yearly}(m \in M_{single}^t) & ~=~ \nonumber \\ \alpha_{basicDivorceRate}  ~\cdot~ &  D_{divorceModifierByDecade}(\lceil age(m) / 10 \rceil)   
\end{align}
That is, the instantaneous probability of a divorce event to a married man $m \in M_{married}$ depends on $D_{divorceModifierByDecade}(\lceil age(m) / 10 \rceil)$, cf.\ Equation \eqref{eq:instantaneous}.




\subsection{Marriages}
\label{sec:events:marriages}

Similar to the divorce event, it is sufficient to apply the marriage event to a sub-population of single males. 
% 
Assuming that 
\begin{equation}
\label{eq:MisMarEli}
M_{marEli} ~=~ M_{marriageEligible} ~=~ M_{single ~\cup~ age \geq 18}    
\end{equation}
% 
the marriage event updates the state of few individuals within a sub-population to married males, formally: 
%
\begin{align}
\label{eq:event:marriage}
& marriage (M_{marEli}^t)  ~ = ~  \nonumber \\ 
 & ~~~~~   M_{marEli - just(isDivorced) - age=18}^{t+\Delta t}  
 ~ \bigcup ~ 
 M_{just(married)}^{t+\Delta t}  
\end{align} 
% 
The amount of yearly marriages can be statistically estimated by
\begin{align}
\label{eq:prob:marriage}
& R_{marraige,yearly}(m \in M_{single}^t) ~=~  \nonumber \\ & ~ \alpha_{basicMaleMarriageRate}  ~\cdot~ D_{maleMarriageModifierByDecade}(\lceil age(m) / 10 \rceil)   
\end{align}
from which simulation-relevant instantaneous probability is calculated as given in Equation \eqref{eq:instantaneous}. 
% 
For an arbitrary just married male $m \in M_{just(married)}^{t + \Delta t}$, 
his wife was selected according to a slight modification of Algorithm \ref{alg:partnerInitialization}. Namely line \ref{line:alg:partner:weightFunc} is modified to: 
\begin{align}
\label{eq:marriageWeight}
        & weight(m,f) ~=~ \nonumber \\  
        & ~~~ geoFactor(m,f) ~\cdot~ childrenFactor(m,f) ~\cdot~ ageFactor(m,f) 
    \end{align}
        \text{ and } %~&~ \\
    \begin{align}
        &        geoFactor(m,f)  =~ \nonumber \\ &~~~~~ 1 / e^{(4 \cdot \text{manhattan-distance}(town(m), town(f)))} \\ 
        & childrenFactor(m,f)  =~ \nonumber \\ 
        & ~~~~~ 1/e^{|children(m)|} \cdot 1/e^{|children(f)|} \cdot e^{|children(m)| \cdot |children(f)|}  
\end{align}
Where $ageFactor(m,f)$ is given in Equation \eqref{eq:alg:ageFactor}. Note that the just married male and his female partner don't then belong to the set of marriage eligible population $P_{marEli}^{t+\Delta 
t}$. \comment{It is thinkable to reverse the genders in the algorithm}
The following assumption specifies the housing's assignment of the new couple. 
\grayass{
\label{ass:mixed:marriedHousing}
When two individuals get married, the wife and the occupants of actual house (i.e.\ children and non-adult orphan siblings) moves to the husband's house unless there are fewer occupants in his house. In the later case, the husband and the occupants of his house move to the wife's house.  
}
Formally, suppose that $m \in P^{t+\Delta t}_{just(married)}$ if
\begin{align*}
 |P_{house(m^t)}| \geq & |P_{house(f^t)}|   \implies   house(p^{t+\Delta t}) = house(m), ~\forall p^t \in P_{house(f^t)}    
\end{align*}
Otherwise
\begin{align}
 house(p^{t+\Delta t}) = house(f),   ~\forall p^t \in P_{house(m^t)}      
\end{align}
