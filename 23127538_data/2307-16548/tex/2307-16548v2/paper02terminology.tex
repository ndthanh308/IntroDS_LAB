\section{Terminology}
\label{sec:terminology}

This section introduces the basic terminological foundation on which fundamental terminology is established for the specification of ABMs and their simulation process following fixed-step single-clocked scheduling.  

\subsection{Populations $P = M \cup F$}
\label{sec:terminology:population}

Given that $F(t) \equiv F^t  \equiv F~(M(t) \equiv M^t$) is the set of all females (males) in a given population $P(t) \equiv P^t$ at an arbitrary time point $t$ where 
\begin{equation}
\label{eq:population}
    P(t) = M(t) \cup F(t)
\end{equation}
%
$t$ is sometimes omitted for the sake of simplification i.e.\ $P \equiv P^t$. 
%
In this case, $P$ refers to the set of all individuals at an implicitly known time point $t$ (e.g.\ the current simulation iteration). \\ 

Analogously, 
$t$ is sometimes placed as a superscript (e.g.\ $F^t$) purely for readability purpose, e.g.\ in the context of algorithm specification. 
Similarly, an individual $p \in P(t)$ when attributed with time, i.e.\ $p^t$, refers to that individual at time point $t$.  \\ 

Furthermore, 
explicit specification of a time interval is realizable as follows: 
\begin{equation}
\label{eq:population:interval} 
P([t_1,t_2]) \equiv P^{[t_1,t_2]}
\end{equation}
indicating the set of all population individuals between $t_1$ till $t_2$ inclusive.
If the interval $[t_1,t_2]$ is associated with a step-size, i.e.\ $[t_1,t_2]_{\Delta t}$, then this time interval is discretized in equidistant time points of length $\Delta t$ starting from $t_1$ till the last possible time point.  


\subsection{Population features $\mathcal{F}$}
\label{sec:features}

Every individual $p \in P$ is attributed by a set of features $f \in F$.    
%
Examples of elementary population features can span but not limited to the following: 
\begin{itemize}
\item \underline{age}, e.g.\ particular age group e.g.\ neonates, children, teenagers, thirties, etc. or specific age e.g.\ 25 years old 
\item \underline{alive status}, i.e. whether alive or dead
\item \underline{gender}, e.g.\ male or female 
\item \underline{space}, e.g.\ inhabitant of a particular town 
\item \underline{kinship status} or relationship, e.g. father-ship, parents, orphans, divorcee, singles etc. 
\end{itemize}




\subsection{Features expressed as predicates}
\label{sec:terminology:predicates}

Population or a person features are explicitly expressed in terms of predicates of various types depending on the outcomes s.a.\ 
\begin{enumerate}
    \item Boolean predicates s.a. $isMale?$, $age>45?$ or $livesInGlasgow?$ %or $hasChildren?$
    \item Individual predicates s.a.\ $fatherOf$  
    \item Grouping predicates s.a.\ $siblingsOf$ or $childrenOf$
\end{enumerate}
The naming choice of the predicate mostly indicates the type of the predicate. However, while nothing prevents the syntactical exchange the predicate "$fatherOf$" with "$father$", the context may lead to an ambiguity in case "$father$" is interpreted as "$isFather?$". \\ 

In this case, it is recommended that the type of the predicate to be explicitly specified by exploiting the post-fix "?" s.a. "$father?$". Similarly, there is nothing against removing the postfix "?" in the predicate $hasSiblings?$ since it is obviously excessive. \\

Nevertheless, the confusion may go further due to the interpretation of $hasSiblings$ whether it implies has more than one sibling or any sibling. Here if the context is not clear, then further precisions need to be employed s.a.\ $hasASibling$ and $hasMoreThanOneSibling$.  \\ 

Formally, a given feature 
\begin{equation}
\label{eq:featureDef}
    f : P \mapsto X
\end{equation}
intuitively maps a given individual $p \in P$ to an element or a subgroup in another subset $X$. The values of the set $X$ depends on the predicate type correspondingly as follows:
\begin{enumerate}
    \item $\{ true, false \}$ for Boolean predicates: e.g.\ $isMale(Jonas \in P) = true$
    \item $X \subset P$ with $|X| = 1$ for individual predicates, e.g.\  $mother(Jonas) = \{Jasmine\} $ ($|*|$ stands for the size)
    \item $X \subseteq P$ for group predicates, e.g.\ $parents(Jonas) = \{ Jahua, Johanna \} $     
\end{enumerate}
The case that Jonas has no siblings is formulated as 
$$siblings(Jonas) = \phi$$ 
with $\phi$ standing for the empty group. 




\subsection{Featured sub-populations via Boolean predicates $P_f$ } 
\label{sec:terminologies:fsubpopulations}

Boolean predicates are further exploited for the specification of featured sub-populations. 
% 
Namely, let $P_{f}$ correspond to the set of all individuals who satisfy a given feature $f$. That is, if 
% 
\begin{equation}
\label{eq:fp}
f(p \in P) = b \in \{true,false\}    
\end{equation}
% 
Then
\begin{equation}
\label{eq:Pf}
P_f = \{ p \in P \text{ s.t.\ } f(p) = true \}    
\end{equation}
%

\subsubsection*{Examples}
\begin{itemize}
    \item $M = P_{male}$ \footnote{it is pre-known that "$male$" indicates a Boolean predicate and thus no need to employ "$isMale?$"} 
    \item $W_{married}$ %\footnote{possibly also $W_{isMarried}$ or $W_{married?}$ but it is already known that sub-populations are featured with Boolean predicates.} 
    corresponds to the set of all married women
    \item $P_{age \geq 65}$ corresponds to all individuals of age older than 65
\end{itemize} 

\graybox{
\begin{definition}[Closed set of features]
\label{def:closedSetOfFeatures}
For the given set of features specified in Section \ref{sec:features}, a subset of features
% 
\begin{equation}
\label{eq:closedSubsetElemFeatures}
F' = \{f'_1,f'_2,\dots \,f'_m\} ~\subset F    
\end{equation}
is called \emph{a closed subset of elementary features}, if 
the overall population constitutes of the union of the underlying elementary featured sub-populations, i.e.\ 
\begin{align}
\label{eq:PUnionPf}
P = P_{\mathcal{F'}} \equiv P_{f'_1 \cup f'_2 \cup \dots \cup f'_m} = P_{f'_1} \cup P_{f'_2} \cup \dots \cup P_{f'_m}     
\end{align} 
where $P_{f'_i} \cap P_{f'_j} = \phi$ for any valid subscripts $i,j$ and $i \neq j$.
\end{definition}
}
For example, male and female gender features constitute a closed set of elementary features as indicated by Equation \eqref{eq:population}. 




\subsection{Featured sub-populations using group predicates $g(P)$}
\label{sec:terminology:gsubpopulations}

Analogously, as implied by Equation \ref{eq:featureDef}, group predicates are employed for specification of sub-populations: 
\begin{equation}
    \label{eq:gP}
    g(P) = \{ p \in P \text{ s.t.\ } \exists~ q \in P \text{ with } g(q) = p \} 
\end{equation}

\subsubsection*{Examples}

\begin{itemize}
\item 
    $children(\{Jad, Jasmine\})$ correspond to children of Jad and Jasmine 
\item 
    $mother(P)$
describes the set of all mothers of a given population 
\end{itemize}
Note that the last example corresponds to individual predicate which is indeed a special case of group predicates. The last example can be equivalently  expressed in terms of Boolean predicates:
$$
mother(P) \equiv P_{isMother} \equiv P_{mother}
$$
This may sound excessive, but the following is a example demonstrating the descriptive power when combining Boolean and group (or individual) predicates together for enabling a concise specification: 
$$
mother(P_{age \leq 3}) \equiv P_{motherWithChildrenOfAgeLessThanOrEqualToThree}
$$
