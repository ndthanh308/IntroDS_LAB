\section{Temporal operators}
\label{sec:temporal}

This section introduces further operators, inspired by the field of temporal logic. 
These operators provide powerful capabilities for algorithmic specification of complex phrases with temporal elements in a compact manner. 
The demonstrated temporal operators are included in the set of composite features $\bigcup \mathcal{F}$. %, cf.\ Section \ref{sec:terminology}.


\subsection{just operator}
\label{sec:temporal:just}

A special operator is 
\begin{equation*}
just(P_f) \equiv P_{just(f)} \subseteq P_f ~,~ f \in \bigcup \mathcal{F} 
\end{equation*}
standing for a featured sub-population established by an event that has just occurred (in the current simulation iteration).
% 
For instance,
$$
just(P_{married}) \equiv 
P_{just(married)}^{t + \Delta t}  ~~~\text{ where } ~~~ \Delta t: \text{ a simulation fixed step size } 
$$ 
stands for those individuals who just got married in the current simulation iteration but they were not married in the previous iteration, i.e.
$$
P_{just(married)}^{t+\Delta t} ~ = ~  P_{married}^{t+\Delta t} ~ - ~ P_{married}^t 
$$
% 
Formally, 
\begin{equation}
\label{eq:just}
P_{just(f)}^{t + \Delta t} = 
P_{f}^{t+\Delta t} ~ - ~ P_{f}^t     
\end{equation}
%
The just operator provides powerful capabilities for concise specification when combined with the negation operator. 
For example, 
$$P_{just(\neg married)}^{t+\Delta t}$$ 
stands for those who "just" got divorced or widowed. 



\subsection{pre operator}
\label{sec:temporal:pre}


Another distinguishable operator is 
$$pre(P_f) \equiv P_{pre(f)} ~,~ f \in \bigcup \mathcal{F}$$ 
standing for a featured $f-$sub-population in the "previous" iteration. 
% 
So for instance,
$$P_{pre(married)}^{t+\Delta t}$$ 
stands for those individuals who were married (and not necessarily just got married) in the previous simulation iteration
% 
$$
pre(P_{married}^{t + \Delta t}) ~ \equiv ~ 
P_{pre(married)}^{t+\Delta t} ~ \equiv ~ P_{married}^t 
$$
Formally, 
\begin{equation}
\label{eq:pre}
   P_{pre(f)}^{t + \Delta t} =  P_f^t  
\end{equation}
Temporal operators can also be applied to individuals and their attributes. For instance 
$$pre(location(p \in P^t)) = Glasgow $$ 
stands for the location of a person in the previous iteration (which does not need to be either similar or different in the current iteration). \\ 

This operator may look unnecessary excessive, however the demographic model specification \cite{Elsheikh2023} of the package implemented in \cite{MiniDemographicABMjlMisc} makes use of the $pre$ operator several times. For instance, a model assumption related to a divorce event is informally and formally described as follows: 
\begin{center}
\graybox{
Any male who just got divorced moves to an empty house within the same town:
\begin{align}
 m \in & M_{just(isDivorced}^t) \implies \nonumber \\  
& pre(location(m)) \neq location(m) ~\text{ and }~ livesAlone(m) ~\text{ and }~ \nonumber \\  
& town(m) = pre(town(m))  
\end{align}
}
\end{center}




%\subsection{apre operator}
%A related operator is  $$ apre(P_f) ~,~ f \in \bigcap \mathcal{F} $$ which stands for "a previous".  For instance $$apre(P_{divorced})$$ stands for those who got divorced in the past, i.e. 
% $$ p \in apre(P_{divorced}^t) \implies \exists ~t_1 < t \text{ s.t.\ } divorced(p^{t_1}) = true $$
% A pre operator is also useful when combined with the negation operator. 
%  For example 
% $$P_{single(apre(\neg divorced))}$$  stands for singles who were never divorced (i.e. either never got married or widows and windowers). 

\section{post operator} 

The operator $post$ is similar to the $pre$ operator, but it is rather concerned with the "post" iteration. 
Without loss of information, the formal description is analgous to the $pre$ operator.  
The following is an example from the accompained demonstrated model for formulating a model assumption making use of the $post$ operator: 

%\grayspass{
%\label{ass:space:inhabitable}
\begin{center}
\graybox{
Once a new house is built (e.g.\ to host an adult moving out of his parent's house), it never gets demolished and remains always inhabitable 
\begin{equation}
\label{eq:ass:HouseAlywasRemain}
    \text{ if } h \in H(t) ~\implies h \in post(H(t)) 
    \text{ where } t_0 \leq t < t_{final}
\end{equation}
}
\end{center}