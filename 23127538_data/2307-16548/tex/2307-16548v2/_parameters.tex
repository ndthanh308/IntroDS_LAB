
The following is a table of parameters employed for events specification, cf.\ Section \ref{sec:events}. The values are set in an ad-hoc manner as they are not calibrated to actual data. The choice of data rather depends on the simulation parameters, e.g.\ the start and final simulation times, as well as the underlying case study.
% 
\begin{center}
\begin{tabular}{|l|c|l|} 
\hline 
$\alpha_{x}$ & Value & Usage  \\
\hline 
$basicDivorceRate$ & 0.06 & Equation \ref{eq:prob:divorce} \\ 
$basicDeathRate$ & 0.0001 & Equation \ref{eq:probDeath} \\ 
$basicMaleMarriageRate$ & 0.7 & Equation \ref{eq:prob:marriage} \\ 
$femaleAgeDeathRate$ & 0.00019 & Equation \ref{eq:probDeath} \\ 
$femaleAgeScaling$ & 15.5 & Equation \ref{eq:probDeath} \\
$initialPop$ & 10000 & Section \ref{sec:initialization:initialPopulation} \\ 
$maleAgeDeathRate$ & 0.00021 &  Equation \ref{eq:probDeath} \\ 
$maleAgeScaling$ & 14.0 & Equation \ref{eq:probDeath} \\ 
$maxNumMarrCand$ & 100 & Sections \ref{sec:initialization:partnership} \& \ref{sec:events:marriages} \\
${startMarriedRatio}$ & 0.8 & Equation \ref{eq:initialMarriedProb} \\
\hline
\end{tabular}
\end{center}
% 
The value of the initial population size is just an experimental value and can be selected, for instance, from the set $\left\{ 10^4, 10^5, 10^6 , 10^7, 10^8 \right\}$ to examine the runtime performance of specific implementation and/or whether it is possible to enable a realistic demographic simulation with an actual population size.  


%This shall be hopefully accompanied in the documentation of the model provided as a pdf-file within the package. 

