\section{The model $\mathcal{M}$} 
\label{sec:model}


\subsection{The space $S$}
\label{sec:model:space}


\subsubsection{Space-related model assumptions $a_{s_*}$} 

Before diving into the specification of the space $S$, it makes sense to list some related set of space-oriented assumptions. The space $ S $, cf.\ Equation \eqref{eq:space} is composed of a tuple $<H(t)~,~W>$  corresponding to the set of all houses $H(t)$ and towns $W$ implying that: 
%
\grayspass{
\label{ass:space:staticTowns} The set of towns is constant during a simulation, i.e.\ no town vanishes nor new ones get constructed:
\begin{equation}
\label{eq:ass:staticTowns}
    \text{ if } t_0 \leq t_1 \neq t_2 \leq t_{final} \implies W(t_1) \equiv W(t_2) \equiv W   
\end{equation}
}
\grayspass{
\label{ass:space:inhabitable}
Once a new house is built (e.g.\ to host an adult moving out of his parent's house), it never gets demolished and remains always inhabitable 
\begin{equation}
\label{eq:ass:HouseAlywasRemain}
    \text{ if } h \in H(t) ~\implies h \in post(H(t)) 
    \text{ where } t_0 \leq t < t_{final}
\end{equation}
}
%
Based on the space definition and the previous Equation, the following assumption is considered:
\grayspass{
%\begin{spass}
\label{ass:space:dynamicSpace}
The space is not necessarily static and particularly the set of houses can vary along the simulation time span, i.e.
\begin{equation}
\label{eq:dynamicHouses}
\text{ if } t_0 \leq t < t_{final} \implies  H(t) \subseteq post(H(t)) 
\end{equation}
%\end{spass}
}
%\item 
Consequently,
\grayspass{
\label{ass:space:dynamicHouses}
Each town $w \in W$ contains a dynamic set of of houses
\begin{equation}
\label{eq:ass:dynamicSetOfHousesInATown}
w(H^t)  \equiv  H_w^t \equiv H_w(t)  
\end{equation}   
}
%\end{enumerate}
%\begin{enumerate}[label=\boldmath$a_{s_\arabic*}$]
%\setItemnumber{4}
%\item 
%\grayspass{
%\label{ass:space:HouseBelongToATown}
%Each house $h \in H(t)$ is located in one and only one town $w \in W$%, i.e. $$town(h) = w \in W \text{ and } h \in H_w$$ 
%}
%\item 


\subsubsection{The space $S$ -- description}

In the sake of simplifying the implementation of the space,  
%\begin{spass}
%\label{ass:space:12x8}
the static set of towns of UK, cf.\ \sassnum{\ref{ass:space:staticTowns}}, is projected as a rectangular $12 \times 8$ grid with each point in the grid corresponding to a town \cite{Gostoli2020}.
%\end{spass}
% 
Formally, assuming that 
$$location(w_{(x,y)}) = (x,y) $$ 
then
% 
\begin{itemize}%[label=\textbf{S. \arabic*}]
%\setItemnumber{9}
\item the town $w_{(1,1)}$ corresponds to the north-est west-est town of UK whereas 
\item the town $w_{(12,8)}$ corresponds to the south-est east-est town of UK 
\item  the distances between towns are commonly defined, e.g.\ 
\begin{equation}
\label{eq:manhattan}
\text{manhattan-distance}(w_{(x_1,y_1)} , w_{(x_2,y_2)}) = \mid x_1 - x_2 \mid + \mid y_1 - y_2 \mid
\end{equation}
%\item 
%\label{ass:space:adjacency}
%two towns $w_{(x_1,y_1)}$ and $ w_{(x_2,y_2)}$ are adjacent (or neighbors) if 
%\begin{equation}
%\mid x_1 - x_2 \mid \leq 1  ~\text{and}~  \mid y_1 - y_2 \mid \leq 1 
%\end{equation}
\end{itemize}

The (initial) population and houses distribution within UK towns are approximated by an ad-hoc pre-given UK population density map.
The map is projected as a rectangular matrix  
%
\begin{equation}
\label{eq:M}
M  \in R^{12 \times 8} \approx 
\begin{bmatrix} 
0.0 & 0.1 & 0.2 & 0.1 & 0.0 & 0.0 & 0.0 &  0.0 \\
0.1 & 0.1 & 0.2 & 0.2 & 0.3 & 0.0 & 0.0 & 0.0 \\
0.0 & 0.2 & 0.2 & 0.3 & 0.0 & 0.0 & 0.0 & 0.0 \\
0.0 & 0.2 & 1.0 & 0.5 & 0.0 & 0.0 & 0.0 & 0.0 \\ 
0.4 & 0.0 & 0.2 & 0.2 & 0.4 & 0.0 & 0.0 & 0.0 \\
0.6 & 0.0 & 0.0 & 0.3 & 0.8 & 0.2 & 0.0 & 0.0 \\ 
0.0 & 0.0 & 0.0 & 0.6 & 0.8 & 0.4 & 0.0 & 0.0 \\ 
0.0 & 0.0 & 0.2 & 1.0 & 0.8 & 0.6 & 0.1 & 0.0 \\ 
0.0 & 0.0 & 0.1 & 0.2 & 1.0 & 0.6 & 0.3 & 0.4 \\ 
0.0 & 0.0 & 0.5 & 0.7 & 0.5 & 1.0 & 1.0 & 0.0 \\ 
0.0 & 0.0 & 0.2 & 0.4 & 0.6 & 1.0 & 1.0 & 0.0 \\ 
0.0 & 0.2 & 0.3 & 0.0 & 0.0 & 0.0 & 0.0 & 0.0
\end{bmatrix}
\end{equation} 
%

It can be observed for instance that
\begin{itemize}%[label=\textbf{S. \arabic*}]
%\setItemnumber{13}
\item 
cells with density $0$ (i.e.\ realistically, with very low-population density) don't correspond to inhabited towns  
\item 
the towns in UK are merged into 48 towns
\item e.g. the center of the capital London spans the cells $(10,6), (10,7), (11,6)$ and $ (11,7)$ 
\end{itemize}


\subsubsection{Further space-related assumptions}

Further assumptions are needed for specification of houses, their creation and their distributions within the UK: 
% 
\grayspass{
\label{ass:space:houseXYTown}
The static location of a house $h \in H^{[t,t_{final}]}_w$ is given in xy-coordinate of the town
\begin{equation}
location(h) = (x,y)_w \text{ where } 1 \leq x,y \leq 25    
\end{equation}
}
\grayspass{
\label{ass:space:housingUniformLocations}
The locations of houses $H_w$ within a town $w \in W$ are uniformly distributed along the x- and y- axes  
\begin{equation}
\label{eq:uniformDistHouses}
Dist(location(H_w)) \propto   (U^{[1,25]},U^{[1,25]})
\end{equation}
}
The previous equation reads the distribution is proportional to. Furthermore,
%
\grayspass{
%\begin{spass}
\label{ass:space:selectOrCreateEmptyHouse}
If an empty house $h$ is demanded in a particular town $w \in W$, an empty house is randomly selected from the set of existing empty houses $H_{w(empty)}$ in that town $w$
\begin{equation}
\label{eq:}
h_w = random(H_{w(empty)}) 
\end{equation}
If no empty house exists, a new empty house is established according to assumptions \sassnum{\ref{ass:space:housingUniformLocations}}
and \sassnum{\ref{ass:space:houseXYTown}}
%\end{spass}
}
\grayspass{
\label{ass:space:houseInArbitraryTown}
If an empty house $h$ is demanded in an arbitrary town, a town is selected via a random weighted selection, say: 
\begin{align}
\label{eq:ass:hometowLocation}
 town(h) =  random(W,M) % \text{ where } 
   % W_{xy} = \{w_{(x,y)} | w_{(x,y)} \in W \text{ and } M_{(y,x)} > 0 \}, \nonumber \\  
\end{align}
an empty house is selected or established according to the previous assumption 
} 


\subsection{Model parameters $\alpha_*$}
\label{sec:model:parameters}


The following is a table of parameters employed for events specification, cf.\ Section \ref{sec:events}. The values are set in an ad-hoc manner as they are not calibrated to actual data. The choice of data rather depends on the simulation parameters, e.g.\ the start and final simulation times, as well as the underlying case study.
% 
\begin{center}
\begin{tabular}{|l|c|l|} 
\hline 
$\alpha_{x}$ & Value & Usage  \\
\hline 
$basicDivorceRate$ & 0.06 & Equation \ref{eq:prob:divorce} \\ 
$basicDeathRate$ & 0.0001 & Equation \ref{eq:probDeath} \\ 
$basicMaleMarriageRate$ & 0.7 & Equation \ref{eq:prob:marriage} \\ 
$femaleAgeDeathRate$ & 0.00019 & Equation \ref{eq:probDeath} \\ 
$femaleAgeScaling$ & 15.5 & Equation \ref{eq:probDeath} \\
$initialPop$ & 10000 & Section \ref{sec:initialization:initialPopulation} \\ 
$maleAgeDeathRate$ & 0.00021 &  Equation \ref{eq:probDeath} \\ 
$maleAgeScaling$ & 14.0 & Equation \ref{eq:probDeath} \\ 
$maxNumMarrCand$ & 100 & Sections \ref{sec:initialization:partnership} \& \ref{sec:events:marriages} \\
${startMarriedRatio}$ & 0.8 & Equation \ref{eq:initialMarriedProb} \\
\hline
\end{tabular}
\end{center}
% 
The value of the initial population size is just an experimental value and can be selected, for instance, from the set $\left\{ 10^4, 10^5, 10^6 , 10^7, 10^8 \right\}$ to examine the runtime performance of specific implementation and/or whether it is possible to enable a realistic demographic simulation with an actual population size.  


%This shall be hopefully accompanied in the documentation of the model provided as a pdf-file within the package. 




\subsection{Input data $D(t)$}
\label{sec:model:data}

In the archived Julia package MiniDemographicABM.jl \cite{MiniDemographicABMjlMisc}, fertility data is given as: 
\begin{align*}
& D_{fertility} \in R^{35 \times 360} ~= \nonumber \\ 
& ~~~~~ \left[ d_{ij} : \text{ fertility rate of women of age } i-16 \text{ in year } j - 1950   \right]    
\end{align*}
This matrix, taken from the Python implementation of the Lone Parent Model \cite{Gostoli2020}, reveals (the forecast of) the fertility rate for woman of ages 17 till 51 between the years 1951 and 2050, cf.\ Figure \ref{fig:fertilityData}. 
% Figure environment removed

Furthermore, the following (ad-hoc) data values, subject to tuning,: 
\begin{align*}
   & D_{divorceModifierByDecade} \in R^{16} ~= \nonumber \\ 
   &~~~~~ (0,1.0,0.9,0.5,0.4,0.2,0.1,0.03,0.01,0.001,0.001,0.001,0,0,0,0)^T  
\end{align*}
\begin{align*}
   & D_{maleMarriageModifierByDecade} \in R^{16}  ~= \nonumber \\ 
   &~~~~~ (0,0.16,0.5,1.0,0.8,0.7,0.66,0.5,0.4,0.2,0.1,0.05,0.01,0,0,0)^T   
\end{align*}
are employed in event specification, cf.\ Equations \eqref{eq:prob:divorce} and \eqref{eq:prob:marriage} for their contextual interpretation. 


% Plot was produced by the following commands 
% julia> plt = plot() 
% julia> plt = plot(plt,1966:2050,fertility[20-16,16:100],label="woman age 20", show=true, reuse=true)
% julia> plt = plot(plt,1966:2050,fertility[25-16,16:100],label="woman age 25", show=true, reuse=true)
% julia> plt = plot(plt,1966:2050,fertility[30-16,16:100],label="woman age 30", show=true, reuse=true)
%  julia> plt = plot(plt,1966:2050,fertility[35-16,16:100],label="woman age 35", show=true, reuse=true)
% julia> plt = plot(plt,1966:2050,fertility[40-16,16:100],label="woman age 40", show=true, reuse=true)
% julia> plt = plot(plt,1966:2050,fertility[44-16,16:100],label="woman age 44", show=true, reuse=true)


