\section{Introduction} 
\label{sec:introduction}


\subsection{Motivation}

\todo[inline]{Suggestion: Motivation that paves the way to why is it beneficial to have a formal terminology for specification for ABM in the field of social computational sciences}. \todo[inline]{An argumentation to sell the proposed terminology from social computational science  agent-based modeling perspective. 
}  

E.g. 
\begin{itemize}
    \item Informal model specification is ambiguous 
    \item Large-scale models need to be carried out by various scientists, modelers and developers each 
    \item Serve as a formal model documentation
    \item easier to detect dis-agreements 
\end{itemize}

I don't think that this paper should really consider argumentation regarding ODD's (this can be actually another paper), but still here are critical issues regarding ODD (This is in appendix) 
\begin{itemize}
    \item 
    The model specification is not following the ODD standard 
    \item 
    Standard gives the impression that ODD follows a formal pattern, though a significant portion of an ODD more or less constitutes of informal model description 
    \item 
    A standard should expresses an exact specification that is capable of replicating the same model using different frameworks and languages where as ODD rather focuses on information exchange and communication among domain experts and modelers and therefore the correctness is not guaranteed 
    \item 
    hence there is no guarantee that what is being described in an ODD is actually what is being implemented 
    \item
    On the other hand 
    a Modelica model (and ModelingToolkit.jl likewise) is an exact specification of a mathematical model that can be simulated as well
    \item 
    The proposed terminology is not suggested as an alternative for ODD but actually it could be integrated into ODD concept, in order to inspire solutions for the existing challenges. However, explicit suggestions are out of the scope of this work \comment{An outlook}
    \item the limited availability of guidance on how to use ODD
    \item limitations of ODD for highly complex models
    \item and the lack of provision for sections in the document structure covering model design rationale, the model’s underlying narrative, and the means by which the model’s fitness for purpose is evaluated
    \item  lack of sufficient details of many ODDs to enable re-implementation without access to the model code;
    \item simply different goals are served 
    \item The current model does not address emergent properties 
    \item It is quite an overhead to write an ODD especially if the implementation of the model, whether with NetLogo, Python or Agents.jl, is well-structured and well-documented following common software engineering techniques regarding coding guidelines, source file documentation, code structuring, design patterns, unit testing among others     
    \item 
    It seems that everyone who writes an O.D.D.\ would suggest a further update or modification to its contents  
\end{itemize}


\todo[inline]{Further improve what the contribution is about}

\subsection{Contribution}

This contribution is primarily concerned with proposing a novel terminology for specifying the mathematical description of agent-based models within a demographic context. 
% 
Particularly, a formal description of fixed-step single-clocked simulation process is presented. 
% 
It is assumed that
\begin{itemize}
    \item all agents are simultaneously progressed according to a \textit{single clock} 
    \item the time clock is progressed along \textit{fixed-step size}
    \item there is no subgroup of agents or subsystems subject to their own clock
\end{itemize} 
The terminology does not span multi-agents models (i.e.\ model with multiple types of agents), though it enables the specification of (or subgroups of agents, i.e.\ a single-level sub-population), cf.\  citation \comment{Citation to a multilevel-modeling Eric paper / Book} for the concept of multi-level agent-based models which is beyond the scope of this work. \\ 

The proposed terminology is illustrated and demonstrated on a simplified demographic model targeting agent-based simulation processes. 
The example model is a simplified version of the model presented in X \comment{Citation to the LPM} concerned with evaluating the impact of social care policies in the UK on social care demand in a socio-economic context. \\

As highlighted in Section \ref{sec:outlook}, the example model has been implemented in the Julia language using state-of-the-art agent-based modeling package Agents.jl \comment{citation to Agents.jl} and enhanced with a decent set of unit tests for effective incremental programming \comment{reference incremental programming / extreme programming}.  
%
The code is published within the model code base following FAIR principles \comment{citation} and is citable according to recommended software citations principles \comment{citation to MiniDemographicABM.jl}. 
%
Nevertheless, this contribution neither favors a particular programming language nor proposes particular agent-based modeling frameworks. While the presented model can actually serve as base model for comparative studies among programming langauges and modeling libraries, implementation-details are out of the scope of this work. \\

The rest of the work is structured as follows. Section \ref{sec:terminology} introduces the fundamental terminologies employed throughout this contribution. 
% 
Section \ref{sec:temporal} proposes a set of temporal operators, inspired by temporal logic \comment{reference}, with which sophisticated phrases can be algorithmically specified in a simple and compact manner.     
% 
Section \ref{sec:generalform} formulates a general specification of agent-based models and their simulation processes.  
% 
The proposed formal specification is then illustrated throughout a simplified demographic model informally introduced in Section \ref{sec:examplemodel} together with a list of simplification assumptions. 
% 
Afterwords, more detailed insights into the model space on which agents are operating is given in Section \ref{sec:space}.  
Subsequently, the set of model events influencing the states of agents is specified in Section \ref{sec:events}. 
Finally, section \ref{sec:outlook} provides further insights into the implementation of the demonstrated model and the aims behind. 

