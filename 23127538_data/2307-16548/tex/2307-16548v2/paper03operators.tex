\section{Composite features $\bigcup \mathcal{F}$}
\label{sec:compositeFeatures}

\subsection{Non-elementary features $f_i ~o~ f_j$}
\label{sec:terminologies:nonelemfeatures}

%\greybox{
%\begin{definition}[non-elementary features]
For a given set of elementary features $\mathcal{F}$ (informally an elementary feature demands only one descriptive predicate), the set of all non-elementary features $\bigcup \mathcal{F}$ is defined as follows:

\graybox{
\begin{definition}
A non-elementary feature:
$$f^* \in \bigcup \mathcal{F} 
~\text{ where }~ \mathcal{F} \subset \bigcup \mathcal{F}
$$ 
is recursively established from a finite number of arbitrary elementary features $$f_i,f_j,f_k, ... \in \mathcal{F}$$ by (but not limited to)
%
\begin{itemize}
    \item union (e.g.\ $f_i \cup f_j$  ),
    \item intersection (e.g.\ $f_i \cap f_j$)
    \item negation (e.g. $ \neg f_i$)
    \item exclusion or difference (e.g.\ $f_i - f_j$ ) 
\end{itemize}
\end{definition}
}
Formally, if 
$$f^* = f_i ~ o ~ f_j ~~~\text{ where }~~~ o \in \{ \cup, \cap , -\} \text{ and } f_i,f_j \in \bigcup \mathcal{F} $$
then 
\begin{equation}
\label{eq:Pfog}
P_{f^*} = P_{f_i~o~f_j} = \{ p \in P ~ s.t.\ p \in (P_{f_i} ~o~ P_{f_j}) \}  
\end{equation}
%
Analogously,
\begin{equation}
\label{eq:neg}
P_{\neg f}  = \{ p \in P ~ s.t.\ p \notin P_f   \} ~\text{ with }  f \in  \bigcup \mathcal{F}     
\end{equation}


Generally, any of the features $f_i$ and $f_j$ in Equation \eqref{eq:Pfog} or $f$ in \eqref{eq:neg} can be either elementary or non-elementary and the definition is recursive allowing the construction of an arbitrary set of non-elementary features. 

\subsubsection*{Examples}

Negation operator s.a.\ ($\neg$) is beneficial for sub-population specification, e.g. 
\begin{equation}
\label{eq:intersectionNegationEx}
F_{married ~\cap~ \neg hasChildren}     
\end{equation}
corresponds to all married females without children. This sub-population can be equivalently described using the difference operator: 
\begin{equation}
\label{eq:differenceEx}
F_{married ~-~ hasChildren}    
\end{equation}
which entails to be a matter of style unless algorithmic execution details of the operators are assumed\footnote{e.g.\ if assumed that within an intermediate computation the ($-$) operator is executed directly on the set of married females rather than the set of all females as the case when employing ($\cap$) instead}. \\

As another example, the sub-population  
\begin{equation}
\label{eq:example:nonelementary}
M_{divorced ~ \cap ~ hasChildren ~ \cap  ~ age>45 ~ - ~ hasSiblings}
\end{equation}
corresponds to the set of all divorced men of age older than 45 who has no siblings but they have children. 
Equation \eqref{eq:example:nonelementary} can be re-written as:
\begin{equation}
\label{eq:example:nonelem:readible}
M_{divorced} ~ \cap ~ M_{hasChildren} ~ \cap ~ M_{age>45} ~ - ~ M_{hasSiblings}     
\end{equation}
Both styles can be mixed together purely for readability or formatting purposes. %, cf.\ Section \ref{sec:events:deaths}.


\subsection{Composition of Boolean predicates $P_{f_1(f_2)}$} 
\label{sec:compositionBooleanOperators}


Another beneficial operator is the composition operator analogously defined as 
\begin{equation}
\label{eq:Pfg}
P_{f_1(f_2)} = \{ p \in P_{f_1} \text{ s.t.\ } f_2(p) = true \}
\end{equation}
where both $f_1$ and $f_2$ correspond to Boolean predicates. 
The composition operator can be regarded to be more  computationally efficient in comparison with the intersection operator\footnote{In this work, the main purpose behind the composition operator mainly remains in the context of algorithmic specification rather than enforcing any implementation details regarding computational efficiency}. \\

The desired sub-population specification in the example given by Equation \ref{eq:example:nonelem:readible} may not correspond to the desired specification. Namely, desired is to specify the alive divorced male population older than 45 years with alive children and alive siblings. 
In this case, the employment of the composition operator is relevant:
\begin{equation}
\label{eq:whatelse}
 M_{alive(divorced ~\cap~ hasAliveChildren ~\cap~ age>45 ~-~ hasAliveSibling)}   
\end{equation}


\subsection{Composition of group and Boolean $g(P_f)$ or ${g(P)}_f$}
\label{sec:compositionGroupBooleanOperators}


Alternatively, if $g$ corresponds to a group predicate , and $f$ corresponds to a Boolean predicate, then composition expresses the following:  
\begin{align}
\label{eq:fPg}
g(P_f) ~=~ & \{ p  \in P \text{ s.t.\ } \exists q \in P_f~ (\text{i.e.\ } f(q) = true) \text{ with } p \in g(q)\}  
\end{align}
For example, $children(M_{alive})$ corresponds to the children of the alive male populations (or equivalently population with alive fathers). Similarly, 
\begin{align}
\label{eq:fPUg}
g(P)_{f} ~=~ & \{ p \in g(P) \text{ s.t.\ } f(p) = true \} \\
& \text{ where } g(P) = \{ p  \in P \text{ s.t.\ } \exists q \in P \text{ with } p \in g(q) \} \nonumber 
\end{align}
Here, $children(M)_{alive}$ corresponds to the alive children of the male population, that is the composition operator should be interpreted with care as
% 
\begin{equation}
\label{eq:non-symmetry}
children(M_{alive})  ~\not \equiv ~ children(M)_{alive}     
\end{equation}
The alive population with alive fathers is expressed as $children\left(M_{alive}\right)_{alive}$



\subsection{Composition of group operators $g_1(g_2(P))$}
\label{sec:compositionGroupOperators}

As a matter of completeness and as an illustrative example, the following phrase expresses the grandchildren of all population of age larger than 65
\begin{equation}
\label{eq:groupOperators}
    children(children(P_{age>65})) \equiv grandChildren(P_{age>65}) 
\end{equation}
However, nothing is against employing the predicate $grandChildren$ directly. 

\todo{Section special cases for $borther(p_{alive})$ or  }

