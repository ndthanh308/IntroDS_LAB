\section{Events rates and instantaneous probability}
\label{sec:terminology:probability}
\label{sec:probability}


Pre-given data, e.g. mortality and fertility rates, are usually given in the form of finite rates (i.e.\ cumulative rate) normalized by sub-population length. 
In other words, the rate 
$$R_{event, period(t,t+\Delta t)}(X^t) ~,~  X^t \subseteq P^t $$ 
corresponds to the number of occurrences that a certain $event$ within a sub-population $X$ (e.g.\ marriage) takes place in the time range between $(t,t+\Delta t)$, e.g.\ a daily, weekly, monthly or yearly rate, etc. normalized by the sub-population length. 
That is, say if a pre-given typically yearly rate is given as input data:
\begin{equation}
\label{eq:yearlyrateAsData}
R_{event,yearly}(X^t) ~=~ D_{event,yearly} \in R^{N \times M}
\end{equation} 
where $M$ corresponds to a given number of years and $N$ corresponds to the number of particular features of interest, for examples:
\begin{itemize}
    \item $M = 100$ for mortality or fertility yearly rate data between the years 1921 and 2020 
    \item $N = 28$ for fertility rate data for women of ages between 18 and 45 years old, i.e.\ $28 = 45-18+1$ 
\end{itemize}
The yearly probability that an event takes place for a particular individual $x^t \in X^t$ is:
\begin{equation}
\label{eq:yearlyProbability}
Pr_{event,yearly}(x^t \in X^t) ~=~ D_{event,yearly}(yearsold(x^t),currentYear(t)) 
\end{equation} 

Pre-given data in such typically yearly format desires adjustments in order to employ them within a single clocked agent-based-model simulation of a fixed step size typically smaller than a year. 
Namely, the occurrences of such events need to be estimated at equally-distant time points with the pre-given constant small simulation step size $\Delta t$. 
For example, if we have population of 1000 individuals with a (stochastic) monthly mortality rate of $0.05$, then after
\begin{itemize}
    \item one month (about) 50 individuals die with 950 left (in average)
    \item two months, about $902.5$  individuals are left
    \item $\dots$
    \item one year, 540 individuals are left resulting in a yearly finite rate of $0.46$
\end{itemize}
\comment{may be a figure}
\comment{better example based on daily rate , e.g. daily rate of 0.001 for a population of 1000}
Typically mortality rate in yearly forms of various age classes are given, but a daily or monthly estimate of the rates shall be applied within an agent-based simulation. \\ 


The desired simulation-adjusted probability is approximated by rather evaluating the desired rate per very short period regardless of the simulation step-size, assumed to be reasonably small (e.g.\ hourly, daily, weekly or monthly at maximum). Formally, the so called instantaneous probability is evaluated as follows:
%
\begin{equation}
\label{eq:instantaneous}
Pr_{event,instantaneous}(x^t \in X^t,\Delta t) ~=~ - \frac{ln(1-Pr_{event,yearly}(x^t))}{N_{\Delta t}} 
\end{equation}
where $N_{\Delta}$ is given as in Equation \eqref{eq:NDeltaT}.
