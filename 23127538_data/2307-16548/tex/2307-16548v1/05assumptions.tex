\section{Model example}
\label{sec:examplemodel}
\label{sec:assumptions}

In this and the following sections, a model example is introduced to demonstrate the descriptive capabilities of the proposed formal terminology.

\subsection{Overview}
\label{subsec:informal_description}

The model is concerned with demographic agent-based model, a simplified demographic-only version of the lone parent model introduced in \cite{Gostoli2020}.
The presented model evolves an initial population of the UK through a combination of events listed in alphabetical order as follows:
\begin{itemize}
    \item ageing, cf.\ Section \ref{sec:events:ageing}
    \item births, cf.\ Section \ref{sec:events:births}
    \item deaths, cf.\ Section \ref{sec:events:deaths}
    \item divorces, cf.\ Section \ref{sec:events:divorces} 
    \item marriages, cf.\ Section \ref{sec:events:marriages} 
\end{itemize}
The population evolution follows Equation \ref{eq:evolution}. \\

Establishing a mathematical model that corresponds to reality till the tiniest details is impossible. 
Therefore, initially a set of (potentially non-realistic) assumptions has to be made in order to simplify the model specification process.
There are mainly two set of assumptions: 
%
\begin{itemize}
    \item population-based assumptions (to be labeled with $P$) 
    \item space-based assumptions (to be labeled with $S$) 
\end{itemize}
% 

\subsection{Population assumptions}


The population assumptions are summarized as follows:
\begin{enumerate}[label=\textbf{P. \arabic*}]
\item \label{ass:pop:homeless} There are no homeless individuals:
\begin{equation}
\label{eq:ass:pop:noHomeless}
    \text{ if } p \in P^t_{isAlive} ~\implies~ house(p) \in H(t)
\end{equation}
\item \label{ass:pop:noImmigration} (In- and out-) immigration is not included: 
    \begin{align}
    \label{eq:ass:pop:noImmigration}
    \text{ if } p \in P_{isAlive}^{t'} 
    & \text{ where }  t_0 < t' \leq t_{final} ~\implies~ \nonumber \\ 
    & pre(town(p)) \in W \text{ and } town(p^{t'}) \in W
\end{align}
\item \label{ass:pop:demographicEvents} Major demographic events s.a.\ world wars and pandemics are not considered
\item \label{ass:pop:housingKinship} Any two individuals living in a single house are either a 1-st degree relatives, step-parent, step-child, step-siblings or partners 
\item An exception to the previous assumption occurs when an orphan's oldest sibling is married 
\end{enumerate}
\setvalue{\popassnum}{6}

\subsection{Space assumptions}

Resoundingly from Section \ref{sec:general:definitions}, the space $ S $ is composed of a tuple 
$$S ~\equiv~ S(t) ~=~ <H(t)~,~W> $$ corresponding to the set of all houses $H(t)$ and towns $W$ implying that: 
\begin{enumerate}[label=\textbf{S. \arabic*}]
\item 
the space is not necessarily static and particularly the set of houses can vary along the simulation time span 
\item \label{ass:space:staticTowns} the set of towns is constant during a simulation, i.e.\ no town vanishes nor new ones get constructed 
\item 
each town $w \in W$ contains a dynamic set of of houses $H_w \equiv H_w(t)$ 
\end{enumerate}
Furthermore,
\begin{enumerate}[label=\textbf{S. \arabic*}]
\setItemnumber{4}
\item 
each house $h \in H(t)$ is located in one and only one town $w \in W$, i.e. 
$$town(h) = w \in W$$ 
\item 
the location of each house $h \in H_w$ is given in xy-coordinate of the town
$$location(h_{x,y} \in H_w) = (x,y)_w$$
\item 
\label{ass:space:housingUniformLocations}
the houses within a town are uniformly distributed along the x and y axes
\item 
\label{ass:space:inhabitable}
a house never get demolished and remains always inhabitable 
\end{enumerate} 
