\section{Temporal operators}
\label{sec:temporal}

This section introduces further operators, inspired by the field of temporal logic. 
These operators provide powerful capabilities for algorithmic specification of time-dependent complex phrases in a compact manner. 
This section is concerned with defining those operators employed within the context of the demonstrated example model described starting from Section \ref{sec:examplemodel}. 
The demonstrated operators in this section shall be included in the set of non-elementary features $\bigcup \mathcal{F}$ defined in Equations \ref{eq:Pfog} and \ref{eq:neg}.


\subsection{just operator}
\label{sec:temporal:just}

A special operator is 
\begin{equation*}
just(P_f) \subseteq P_f ~,~ f \in \bigcup \mathcal{F} 
\end{equation*}
standing for a featured subpopulation established by an event that has just occurred (in the current simulation iteration).
% 
So for instance,
$$P_{just(married)}^{t + \Delta t}  $$ 
stands for those individuals who just got married in the current simulation iteration with a fixed step size $\Delta t$ but they were not married in the previous iteration, i.e.
$$
P_{just(married)}^{t+\Delta t} ~ = ~  P_{married}^{t+\Delta t} ~ - ~ P_{\neg married}^t 
$$
% 
Formally, 
\begin{equation}
\label{eq:just}
P_{just(f)}^{t + \Delta t} = 
P_{f}^{t+\Delta t} ~ - ~ P_{\neg f}^t     
\end{equation}
%
The just operator provides capabilities for powerful specification when combined with the negation operator. For example, 
$$P_{just(\neg married)}^{t+\Delta t}$$ 
stands for those who "just" got divorced or widowed. 



\subsection{pre operator}
\label{sec:temporal:pre}


Another distinguishable operator is 
$$pre(P_f) ~,~ f \in \bigcup \mathcal{F}$$ 
standing for "the previous iteration". 
% 
So for instance,
$$P_{pre(married)}^{t+\Delta t}$$ 
stands for those individuals who were married (and not necessarily just got married) in the previous simulation iteration
% 
$$
P_{pre(married)}^{t+\Delta t} ~ = ~ P_{married}^t 
$$
Formally, 
\begin{equation}
\label{eq:pre}
   P_{pre(f)}^{t + \Delta t} =  P_f^t  
\end{equation}
%
This operator may look unnecessary excessive, however cf.\ Section \ref{sec:events:deaths} as an example for the usefulness of the $pre$ operator. \\

In this work, temporal operators is assumed to extend their applicability to individuals and their attributes. For instance 
$$pre(location(p \in P)$$ 
stands for the location of a person in the previous iteration (which can be the same in the current iteration), e.g.\ cf.\ Section \ref{sec:events:deaths}. 


%\subsection{apre operator}
%A related operator is  $$ apre(P_f) ~,~ f \in \bigcap \mathcal{F} $$ which stands for "a previous".  For instance $$apre(P_{divorced})$$ stands for those who got divorced in the past, i.e. 
% $$ p \in apre(P_{divorced}^t) \implies \exists ~t_1 < t \text{ s.t.\ } divorced(p^{t_1}) = true $$
% A pre operator is also useful when combined with the negation operator. 
%  For example 
% $$P_{single(apre(\neg divorced))}$$  stands for singles who were never divorced (i.e. either never got married or widows and windowers). 