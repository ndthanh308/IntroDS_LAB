\section{Events}
\label{sec:events}


This section provides a compact algorithmic specification as a demonstration of the proposed terminology.   
%
The employed set of parameters and input data trajectories is given in Appendix \ref{sec:parametersdata}. 
%
The algorithmic specification makes use of rates and instantaneous probabilities conceptually reviewed in Appendix \ref{sec:probability}. \\  

The considered events are alphabetically listed in this section without enforcing a certain appliance order, except for the ageing event which should proceed any other events.
That is, Equations \ref{eq:stepping} and \ref{eq:evolution} are constrained by setting 
$$e_1 = ageing$$  
The execution order of the events as well as the order of the agents subject to such events, whether sequential or random, remains an implementation detail.  
Nevertheless, since many of the events, specified in the following subsections, are following a random stochastic process, probably, the higher the resolution of the simulation becomes (e.g.\ weekly step-size instead of monthly, or daily instead of weekly), the less influential the execution order of the events becomes.



\subsection{Ageing}
\label{sec:events:ageing}

Following the terminology introduced so far, ageing process of a population can be described as follows
%
\begin{equation}
\label{eq:event:ageing}
ageing \left(
P_{isAlive(age=a)}^t\right) 
= 
P_{isAlive(age=a+\Delta t)}^{t+\Delta t} ~~ ,~~ \forall a \in \{0, \Delta t, 2 \Delta t, ... \}
\end{equation}
%
The age of any individual as long as he remains alive is incremented by $\Delta t$ for each simulation step. Furthermore, the following assumption is considered
\begin{enumerate}[label=\textbf{P. \arabic*}]
\setItemnumber{\popassnum}
\item \label{ass:pop:teenager}
In case a teenager orphan becomes an adult and he/she is not the oldest sibling, the orphan gets re-allocated to an empty house within the same town, formally:
\end{enumerate}
\setvalue{\popassnum}{7}
%
\begin{align}
\label{eq:AdultMovesToAnEmptyHouse}
ageing & \left(P_{isAlive(age = 18) ~\cap~ isOrphan ~\cap~ hasOlderSibling } \right)  =  \nonumber \\ 
 & P_{isAlive(age = 18 + \Delta t) ~\cap~ isOrphan ~\cap~ hasOlderSibling ~\cap~ livesAlone } 
\end{align}
Moreover, 
\begin{align}
\label{eq:AdultMoveToTheSameTown}
 \text{If } p \in P^{t+\Delta t}_{isAlive(age=18)}  & \text{ and } pre(house(p)) \neq house(p) \implies \nonumber \\ 
 & town(p) = pre(town(p))   
\end{align}
The re-allocation to an empty house is in conformance with assumption \ref{ass:space:selectOrCreateEmptyHouse}.



\subsection{Births}
\label{sec:events:births}

For simplification purpose, from now on it is implicitly assumed (unless specified) that only the alive population is involved in event-based transition of population specification. 
%
Let the set of reproducible females be defined asi.e.  
% 
\begin{align}
F_{reproducible} & ~=~ F_{isMarried ~\cap~ age < 45} ~\bigcap~ \nonumber \\ &  F_{ youngestChild(age > 1) ~\cup~ \neg hasChildren}     
\end{align} 
That is,  the set of all married females in a reproducible age and either do not have children or those with youngest child older than one.  
%
The specification of a birth event demands enhancing the population-related assumptions as follows:
\begin{enumerate}[label=\textbf{P. \arabic*}]
\setItemnumber{\popassnum}
\item 
a neonate's house is his mother house:
\begin{align}
f \in~ &  F_{youngestChild(age=0)}   
\implies \nonumber \\ 
& house(youngestChild(f)) = house(f)    
\end{align}
\item 
\label{ass:pop:marriedGivesBirth}
only a married female\footnote{This was already assumed in the lone parent model and obviously the marriage concept needs to be re-defined in the context of realistic studies} gives birth
\begin{align}
\label{eq:ass:pop:onlyMarriedGivesBirth}
age(&youngestChild(f \in F)) = 0 \implies % \nonumber \\ & 
isMarried(f) = true
\end{align}
\item 
\label{ass:pop:marrigeaAge} 
a married person is not a teenager 
\begin{align}
isMarried(p \in P) = true \implies age(p) \geq 18 
\end{align}
\end{enumerate}
\setvalue{\popassnum}{10}
% 
Assumptions \ref{ass:pop:marriedGivesBirth} and \ref{ass:pop:marrigeaAge} implies that only an adult person can become a parent.  
 %
The birth event produces new children from reproducible females: 
% 
\begin{align}
birth & \left(F^t_{reproducible} \right) ~=~ \nonumber \\ 
&~  \left( F_{reproducible}^{t+\Delta t} ~ - ~ F_{just(reproducible)}^{t+\Delta t}  \right) ~ \bigcup  \nonumber \\ 
&~ F_{just(\neg reproducible) }^{t+\Delta t}  ~\bigcup \nonumber \\ 
 &~  P_{age=0}^{t+\Delta t}
\end{align} 
As an illustration based on the \textit{just} operator illustrated in Section \ref{sec:temporal:just} 
\begin{equation}
F_{just(reproudicble)} ~=~ 
F_{just(isMarried)} ~\cup~ 
F_{isMarried(youngestChild(age=1))}    
\end{equation}
and (given Assumption \ref{ass:pop:marriedGivesBirth})
%
\begin{align}
& F_{just (\neg reproducible)} ~=~ \nonumber \\  
    & ~~~ F_{youngestChild(age=0)} ~ \cup ~ F_{just(devorced)} ~\cup ~ F_{isMarried(age=45)}    
\end{align}
% 
The yearly-rate of births produced by the sub-population $F_{reproducible,(age=a)}^t$ i.e.\ reproducible females of age $a$ years old with actual simulation time $t$, depends on the yearly-basis fertility rate data:  
%
\begin{equation}
    R_{birth,yearly} (F_{reproducible,(age=a)}^t)  ~\propto~ D_{fertility}(a,currentYear(t))   
\end{equation}
% 
cf.\ fertility rate data in \ref{sec:parametersdata:data}.  
%
This implies that the instantaneous probability that a reproducible female $f \in F_{reproducible}(t)$ gives birth to a new individual $p \in P^{t + \Delta t}$ depends on $D_{fertility}(a,currentYear(t))$ and is given by Equation \ref{eq:instantaneous}, cf.\ Appendix \ref{sec:terminology:probability}. 




\subsection{Deaths} 
\label{sec:events:deaths}

The death event transforms a given population of alive individuals as follows: 
\begin{align}
\label{eq:event:deaths}
    death \left( P_{isAlive }^t \right)  ~=~  P_{isAlive - age=0}^{t+\Delta t}  ~\bigcup~  P_{just(\neg isAlive)}^{t+\Delta t} 
    % \\ & ~=  \left(P_{isAlive}^{t+\Delta t} - P_{age=0}^{t+\Delta t} \right) & \cup~ &  P_{\neg isAlive}^{t+\Delta t} - P^t_{isAlive}
\end{align}
The first phrase in the right hand side stands for the alive population except neonates and the second stands for those who just became dead. The following simplification assumptions are considered: 
%
\begin{enumerate}[label=\textbf{P. \arabic*}]
\setItemnumber{\popassnum} 
 \item \label{ass:pop:adoption} No adoption or parent re-assignment to orphans is established after their parents die 
\item 
Those who just became dead they leave their houses, i.e.
\begin{align}
\label{eq:deadsLeaveToGrave}
\text{ if } p \in P_{just(\neg isAlive)}^{t+\Delta t}~  &
\text{ and }~ pre(house(p)) = h 
\nonumber \\ 
\implies ~ & p \notin P_h  ~\text{ and }~ house(p) = grave    
\end{align}
\end{enumerate}
\setvalue{\popassnum}{12}
% 
The amount of population deaths depends on the yearly probability given by:  
\begin{align}
\label{eq:probDeath}
Pr_{death,yearly}&(p \in P) = \alpha_{baseDieRate} + \nonumber \\ & \left\{   
\begin{array}{cc}
   \left( e^{\frac{age(p)}{\alpha_{maleAgeScaling}}} \right)  \times \alpha_{maleAgeDieProb}   \text{ if } isMale(p) \\
       \left( e^{\frac{age(p)}{\alpha_{femaleAgeScaling}}} \right)  \times \alpha_{femaleAgeDieProb}
    \text{ if } isFemale(p) \\
\end{array}
\right. 
\end{align}
%
from which instantaneous probability of the death of an individual is derived as illustrated in Appendix \ref{sec:probability}.






\subsection{Divorces} 
\label{sec:events:divorces}

The divorce event causes that a subset of married population becomes divorced: 
\begin{align}
& divorce (M_{isMarried}^t) ~ = ~  \nonumber \\ 
 &~~~~~   M_{isMarried - just(isMarried)}^{t+\Delta t}  ~ \bigcup ~ M_{just(\neg isMarried)}  %~=  \\ 
% & ( P_{isMarried}^{t+1} - P_{single}^t ) ~ \cup ~ (P_{divorced}^{t+1} - P_{isMarried}^t) 
\end{align}
The first phrase in the right hand side refers to the set of married individuals who remained married excluding those who just got married.
The second phrase refers to the population subset who just got divorced in the current iteration. 
Note that it is sufficient to only apply the divorce event to  either the male or female sub-populations. 
After divorce takes place, the housing is specified according to the following assumption: 
%
\begin{enumerate}[label=\textbf{P. \arabic*}]
\setItemnumber{\popassnum} 
\item 
Any male who just got divorced moves to an empty house within the same town (in conformance with assumption \ref{ass:space:selectOrCreateEmptyHouse}):
\begin{align}
& location(m \in M_{just(isDivorced}) = h   ~\text{ and }~ pre(location(m)) = h'   
\implies \nonumber \\ 
& ~~~~~ | occupants(h)| = 1 ~\text{ and }~  town(h) = town(h')  
\end{align}
\end{enumerate} 
\setvalue{\popassnum}{13}
%
The re-allocation to an empty house is in conformance with assumption \ref{ass:space:selectOrCreateEmptyHouse}.
The amount of yearly divorces in married male populations depends on the yearly probability given by
\begin{align}
\label{eq:prob:divorce}
 Pr_{divorce,yearly}(m \in M_{isSingle}^t) & ~=~ \nonumber \\ \alpha_{basicDivorceRate}  ~\cdot~ &  D_{divorceModifierByDecade}(\lceil age(m) / 10 \rceil)   
\end{align}
That is, the instantaneous probability of a divorce event to a married man $m \in M_{isMarried}$ depends on $D_{divorceModifierByDecade}(\lceil age(m) / 10 \rceil)$, cf.\ Equation \ref{eq:instantaneous}.




\subsection{Marriages}
\label{sec:events:marriages}

Similar to the divorce event, it is sufficient to apply the marriage event to a sub-population of single males. 
% 
Assuming that 
\begin{equation}
\label{eq:MisMarEli}
M_{isMarEli} ~=~ M_{isMarriageEligible} ~=~ M_{isSingle ~\cup~ age \geq 18}    
\end{equation}
% 
the marriage event updates the state of few individuals within a sub-population to married males, formally: 
%
\begin{align}
\label{eq:event:marriage}
& marriage (M_{isMarEli}^t)  ~ = ~  \nonumber \\ 
 & ~~~~~   M_{isMarEli - just(isDivorced) - age=18}^{t+\Delta t}  
 ~ \bigcup ~ 
 M_{just(isMarried)}^{t+\Delta t}  
\end{align} 
% 
The amount of yearly marriages is estimated by
\begin{align}
\label{eq:prob:marriage}
& Pr_{marraige,yearly}(m \in M_{isSingle}^t) ~=~  \nonumber \\ & ~ \alpha_{basicMaleMarriageRate}  ~\cdot~ D_{maleMarriageModifierByDecade}(\lceil age(m) / 10 \rceil)   
\end{align}
from which simulation-relevant instantaneous probability is calculated as given in Equation \ref{eq:instantaneous}. 
% 
For an arbitrary just married male $m \in M_{just(married)}^{t + \Delta t}$, 
his partner was selected according to following steps: 
\begin{itemize}
    \item set $n_{candidates} = max\left( \alpha_{maxNumMarrCand} ~,~ \frac{|F_{isMarEli}|}{10}\right)$ 
    \item establish a random set of $n_{candidates}$ female candidates:
    $$ F_{candidates} = random(F_{isEliMar}, n_{candidates}) $$
    \item for $m \in M_{isMarEli} , ~ f \in F_{isMarEli}$,  
    set a marriage weight function:
    \begin{align}
    \label{eq:marriageWeight}
        & weight(m,f) ~=~ \nonumber \\  
        & ~~~ geoFactor(m,f) ~\cdot~ childrenFactor(m,f) ~\cdot~ ageFactor(m,f) 
    \end{align}
        \text{ where } %~&~ \\
    \begin{align}
        &        geoFactor(m,f)  =~ \nonumber \\ &~~~~~ 1 / e^{(4 \cdot \text{manhattan-distance}(town(m), town(f)))} \\ 
        & childrenFactor(m,f)  =~ \nonumber \\ 
        & ~~~~~ 1/e^{|children(m)|} \cdot 1/e^{|children(f)|} \cdot e^{|children(m)| \cdot |children(f)|}  \\
        & ageFactor(m,f)  =~  \nonumber \\ & ~~~ \left\{ 
\begin{array}{ll}
~ 1 / (age(m) - age(f) - 5 + 1) &  \text{ if }~ age(m) - age(f) \geq 5 \\ 
~ -1 / (age(m) - age(f) + 2 - 1) &  \text{ if }~ age(m) - age(f) \leq -2 \\ 
~ 1 \text{ ~~~~otherwise } \\ 
\end{array}
\right.   
\end{align}
\item select a random female according to the weight function  
\begin{align}
& f_{partner(m)}^{t+\Delta t}  =  weightedSample(F_{isMarEli}^t, W_m) 
\nonumber \\ 
& ~~~~~ \text{ where } ~ W_m = \left\{ w_i : w_i = weight(m,f_i) ~,~ f_i \in F_{isMarEli} \right\}     
\end{align}
\item $F_{candidates} = F_{candidates} - \{ f_{partner(m)}\}$
\end{itemize} 
Note that the just married male and his selected female don't belong to the set of marriage  eligible population $P_{isMarEli}^{t+\Delta 
t}$ any more. \comment{It is thinkable to reverse the genders in the algorithm}
The following assumption specifies the house of the new couple 
\begin{enumerate}[label=\textbf{P. \arabic*}]
\setItemnumber{\popassnum}
\item \label{ass:pop:teenager}
When two individuals get married, the wife and the occupants of actual house (i.e.\ children and non-adult orphan siblings) moves to the husband's house unless there are fewer occupants in his house. In the later case, the husband and the occupants of his house move to the wife's house.  
\end{enumerate}
\setvalue{\popassnum}{14}
%
Formally, suppose that $m \in P^{t+\Delta t}_{just(isMarried)}$ and $f^{t+\Delta t} = partner(m^{t + \Delta t})$, if
\begin{align*}
 |P_{house(m^t)}| \geq & |P_{house(f^t)}| \text{ and } house(p^t) = house(f^t)    \implies  \\ &  house(p^{t+\Delta t}) = house(m)     
\end{align*}
Otherwise
\begin{align}
 |P_{house(m^t)}| <  & |P_{house(f^t)}|\text{ and } house(p^t) = house(m^t)    \implies  \nonumber \\ & house(p^{t+\Delta t}) = house(f)     
\end{align}
