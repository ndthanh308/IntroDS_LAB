\section{Model initialization $\mathcal{M}^{t_0}$} 
\label{sec:initialization}

This section provides a detailed description of the initial model state as evaluated by $M^{t_0}$ given that Section \ref{sec:parametersdata} demonstrates potential case studies specifying possible simulation parameter values for $t_0,~t_{final} \text{ and } \Delta t$. Further initialization assumptions are proposed by demand, distinguished by the labels \textbf{P0} for initial population assumptions or \textbf{S0} for initial space assumptions.



\subsection{Initial population size $|P^{t_0}_{w}|$}  
\label{sec:initialization:initialPopulation} 

The initial population size is given by the parameter $\alpha_{initialPop}$, cf. Section \ref{sec:parametersdata:parameters}. 
The matrix $M$, defined in Equation \ref{eq:M}, provides a stochastic ad-hoc estimate of the initial population $P(t_0)$ distribution within the UK as well as the initial set of given houses $H(t_0)$. 
% 
That is, the initial population size of a town $w \in W$ is approximated by
\begin{equation}
\label{eq:Pwt0}
|P_w(t_0)| \approx \alpha_{initialPop} \times M_{y,x} / 48  \text{~~ where ~~location}(w) = (x,y)     
\end{equation}
where 48 is the number of nonzero entries in $M$. 


\subsection{Gender $P^{t_0} = M^{t_0} \cup F^{t_0}$}  
\label{sec:initialization:initialPopulation} 

The parameter $\alpha_{initialPop}$ specifies the size of initial population, cf.\ Section \ref{sec:parametersdata} for potential values. The gender ratio distribution is unrealistically specified via the following non-realistic assumption: 
%
\begin{enumerate}[label=\textbf{P. \arabic*}]
\setItemnumber{\popassnum}
\item An individual can be equally a male or a female\footnote{In reality, worldwide there is a tiny higher number of females births over males births}
\end{enumerate}
 Gender assignment is established according to a uniform distribution, i.e.
\begin{equation}
\label{eq:genderProbability}
Pr(isMale(p \in P^{t_0})) \approx 0.5
\end{equation}
\setvalue{\popassnum}{7}



\subsection{Age distribution $P^{t_0} = \bigcup_{r} P_{age=r}^{t_0}$}
The proposed non-negative age distribution of population individuals in years follows a normal distribution: 
\begin{equation}
\label{eq:ageDistribution}
\frac{age(P^{t_0})}{N_{\Delta t}} \in \mathbb{Q}^{\alpha_{initialPop}}_+ ~\propto~ \left| \lfloor \mathcal{N}(0,  \frac{100}{4} \cdot N_{\Delta t}) \rfloor \right|
\end{equation}
where $\mathbb{Q}_+$ stands for the set of positive rational numbers and  $\mathcal{N}$ stands for a normal distribution with mean value 0 and standard deviation depending on 
\begin{equation}
\label{eq:NDeltaT}
    N_{\Delta t} ~=~ \left\{ 
\begin{array}{ll}
~ \dots \\
~ 12 &  \text{ if }~ \Delta t = month \\ 
365 & \text{ if }~ \Delta t = day \\ 
365 \cdot 24 & \text{ if }~ \Delta t = hour \\ 
~ \dots 
\end{array}
\right. 
\end{equation}
A possible outcome of the distribution of ages in an initial population of size 1,000,000 is shown in Figure \ref{fig:initialPopulationAges}.
% Figure environment removed






\subsection{Partnership $P_{isMarried}^{t_0} = M_{isMarried}^{t_0} \cup M_{partners}^{t_0}$}
\label{sec:initialization:partnership}

Initially the following population assumption is proposed: 
%
\begin{enumerate}[label=\textbf{P. \arabic*}]
\setItemnumber{\popassnum}
\item 
There is no  grandpa or grandma for any individual in the initial relationship, i.e. 
\begin{equation}
\label{eq:ass:nograndChildren}
p \in P_{isChild}^{t_0} \equiv  P_{age \leq 18}^{t^0} \implies \nexists q \in P^{t_0} \text{ s.t.\ } grandchild(q) = p     
\end{equation}
\end{enumerate}
\setvalue{\popassnum}{8}
%
The ratio of married adults (males or females) is stochastically approximated according to 
\begin{equation}
\label{eq:initialMarriedProb}
Pr(p \in P_{isAdult}^{t_0}) \approx \alpha_{startMarriedRate} 
\end{equation}
%
Before the partnership initialization, the following variables are initialized: 
\begin{enumerate}
    \item set $F_{isMarriageEligible} = F_{isMarEli} = F_{isAdult}^{t_0}$
    \item set $n_{candidates} = max\left( \alpha_{maxNumMarrCand} ~,~ \frac{|F_{isMarEli}|}{10}\right)$ 
\end{enumerate}
For every male $m \in M_{isMarried}^{t_0}$ selected for marriage,\comment{use algorithmic} his wife is selected according to following steps: 
\begin{enumerate}
    \setItemnumber{3}
    \item establish a random set of $n_{candidates}$ female candidates:
    $$ F_{candidates} = random(F_{isMarEli}, n_{candidates}) $$
    \item for every candidate, $~ f \in F_{isMarEli}$,  
    evaluate a marriage weight function:
    \begin{align}
    \label{eq:initMarriageWeight}
        & weight(m,f) ~=~ ageFactor(m,f) 
    \end{align}
        \text{ where } %~&~ \\
    \begin{align}
        & ageFactor(m,f)  =~  \nonumber \\ & ~~~ \left\{ 
\begin{array}{ll}
~ 1 / (age(m) - age(f) - 5 + 1) &  \text{ if }~ age(m) - age(f) \geq 5 \\ 
~ -1 / (age(m) - age(f) + 2 - 1) &  \text{ if }~ age(m) - age(f) \leq -2 \\ 
~ 1 \text{ ~~~~otherwise } \\ 
\end{array}
\right.   
\end{align}
\item select a random female associated with the evaluated weights
\begin{align}
& f_{partner(m)}  =  weightedSample(F_{isMarEli}, W_m) 
\nonumber \\ 
& ~~~~~ \text{ where } ~ W_m = \left\{ w_i : w_i = weight(m,f_i) ~,~ f_i \in F_{isMarEli} \right\}     
\end{align}
\item $F_{isMarEli} = F_{isMarEli} - \{ f_{partner(m)}\}$
\end{enumerate} 










\subsection{Children and parents} 

The following assumptions are assumed only in the context of the initial population: 
% 
\begin{enumerate}[label=\textbf{P0. \arabic*}]
\setItemnumber{\popassnum}
    \item There is no orphan
    \item There is no age difference restrictions among siblings, i.e. age difference can be less than 9 months
\end{enumerate}
\setvalue{\popassnum}{10}
% 
Children are assigned to married couples as parents in the following way. 
For any child $c  \in P_{age<18}^{t_0}$, the set of potential fathers is established as follows: 
\begin{align}
\label{eq:fathershipCanddiates}
M_{candidates} = ~ & \{~ m \in M_{isMarried} \text{ s.t.\ } \nonumber  \\ &   min(age(m),age(wife(m)) \geq age(c) + 18 + \frac{9}{12} \text{ and } \nonumber \\ & 
age(wife(m)) < 45 + age(c) ~\}  
\end{align}
out of which a random father is selected for the child: 
\begin{align*}
    father(c) =  & random(M_{candidates}) \text{ and } \\
    & mother(c) = wife(father(c)) 
\end{align*}



\subsection{Spatial distribution}
\label{sec:initialization:spatial}
 
% 
The assignment of newly established houses to initial population considers the assumptions \ref{ass:space:selectOrCreateEmptyHouse} and \ref{ass:space:houseInArbitraryTown}. That is, 
the location of new houses in $H(t_0)$ is specified according to to Equation \ref{eq:ass:hometowLocation}. 
Furthermore, the following assumption is proposed: 
% 
\begin{enumerate}[label=\textbf{P0. \arabic*}]
\setItemnumber{\popassnum}
\item any house in $H(t_0)$ either occupied by a single person or a family
\begin{align*}
    |occupants(h^{t_0})| > 1 & \text{ with } p,q \in occupants(h^{t_0}) \text{ and } p \neq q ~\implies~ \\ &  p \in firstDegRelatives(q)  
\end{align*}
\end{enumerate}
\setvalue{\popassnum}{9}
%
%
Assignments of new houses to the initial population is conducted as follows: 
\begin{align}
    & p^{t_0} \in P_{isSingle}^{t_0} \implies occupants(house(p)) = \{ p \} \nonumber \text{ otherwise } \\  
    & m^{t_0} \in M_{isMarried}^{t_0} \implies \nonumber \\ &~~~~~ occupants(house(m)) = \{ m , wife(m) \} \cup children(m) 
\end{align}
Overall, all houses in $H(t_0)$ are occupied, i.e.\ $|occupants(h^{t_0})| \geq 1$. 
