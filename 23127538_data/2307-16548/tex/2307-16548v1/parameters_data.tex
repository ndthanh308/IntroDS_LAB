\section{Parameters and input data} 
\label{sec:parametersdata}

\subsection{Parameters}
\label{sec:parametersdata:parameters}




\subsubsection*{Model parameters}

The following is a table of parameters employed for events specification. The values are set in an ad-hoc manner and they are not calibrated to actual data. Actual data is rather dependent on simulation parameters, e.g.\ the start and final simulation times.
% 
\begin{center}
\begin{tabular}{|l|c|l|} 
\hline 
$\alpha_{x}$ & Value & Usage  \\
\hline 
$basicDivorceRate$ & 0.06 & Equation \ref{eq:prob:divorce} \\ 
$basicDeathRate$ & 0.0001 & Equation \ref{eq:probDeath} \\ 
$basicMaleMarriageRate$ & 0.7 & Equation \ref{eq:prob:marriage} \\ 
$femaleAgeDieRate$ & 0.00019 & Equation \ref{eq:probDeath} \\ 
$femaleAgeScaling$ & 15.5 & Equation \ref{eq:probDeath} \\
$initialPop$ & 10000 & Section \ref{sec:initialization:initialPopulation} \\ 
$maleAgeDieRate$ & 0.00021 &  Equation \ref{eq:probDeath} \\ 
$maleAgeScaling$ & 14.0 & Equation \ref{eq:probDeath} \\ 
$maxNumMarrCand$ & 100 & Sections \ref{sec:initialization:partnership} \& \ref{sec:events:marriages} \\
${startMarriedRate}$ & 0.8 & Equation \ref{eq:initialMarriedProb} \\
\hline
\end{tabular}
\end{center}
% 
The value of the initial population size is just an experimental value and can be selected from the set $\left\{ 10^4, 10^5, 10^6 , 10^7 \right\}$ to examine the performance of the implementation or to enable a realistic demographic simulation with actual population size.  

\subsubsection*{Simulation parameters}

In version 1.1 of the package MiniDemographicABM.jl \cite{Gostoli2020}, the following ad-hoc values of the simulation parameters are selected: 
\begin{center}
\begin{tabular}{|l|c|} 
\hline 
$\alpha_{x}$ & Value \\ 
\hline 
$t_0$ & 2020 \\ 
$\Delta_t$ & Daily \\ 
$t_{final}$ & 2030 \\ 
\hline
\end{tabular}
\end{center}
It is beneficial in future to further propose several case studies with specific simulation parameter values for each case. \comment{Proposals here?} 
This shall be hopefully accompanied in the documentation of the model provided as a pdf-file within the package. 


\subsection{Data}
\label{sec:parametersdata:data}

The data values are set as follows: 
\begin{align*}
   & D_{divorceModifierByDecade} \in R^{16} ~= \nonumber \\ 
   &~~~~~ (0,1.0,0.9,0.5,0.4,0.2,0.1,0.03,0.01,0.001,0.001,0.001,0,0,0,0)^T \\  
\end{align*}
\begin{align*}
   & D_{maleMarriageModifierByDecade} \in R^{16}  ~= \nonumber \\ 
   &~~~~~ (0,0.16,0.5,1.0,0.8,0.7,0.66,0.5,0.4,0.2,0.1,0.05,0.01,0,0,0)^T \\  
\end{align*}
In the archived Julia package MiniDemographicABM.jl Version 1.1 \comment{citation} and originally taken from the lone parent model implemented in Python the fertility data
\begin{align*}
& D_{fertility} \in R^{35 \times 360} ~= \nonumber \\ 
& ~~~~~ \left[ d_{ij} : \text{ fertility rate of women of age } i-16 \text{ in year} j - 1950   \right]    
\end{align*}
This matrix reveals and forecast the fertility rate for woman of ages 17 till 51 between the years 1951 and 2050, cf.\ Figure \ref{fig:fertilityData}. 
% Figure environment removed


% Plot was produced by the following commands 
% julia> plt = plot() 
% julia> plt = plot(plt,1966:2050,fertility[20-16,16:100],label="woman age 20", show=true, reuse=true)
% julia> plt = plot(plt,1966:2050,fertility[25-16,16:100],label="woman age 25", show=true, reuse=true)
% julia> plt = plot(plt,1966:2050,fertility[30-16,16:100],label="woman age 30", show=true, reuse=true)
%  julia> plt = plot(plt,1966:2050,fertility[35-16,16:100],label="woman age 35", show=true, reuse=true)
% julia> plt = plot(plt,1966:2050,fertility[40-16,16:100],label="woman age 40", show=true, reuse=true)
% julia> plt = plot(plt,1966:2050,fertility[44-16,16:100],label="woman age 44", show=true, reuse=true)


