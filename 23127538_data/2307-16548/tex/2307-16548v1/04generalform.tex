\section{General form}
\label{sec:generalform}


\subsection{Definitions}
\label{sec:general:definitions}

This article is concerned with formalizing an agent-based model simulation formally defined via the tuple 
\begin{equation}
  \label{eq:abmDefinition}<\mathcal{M},\alpha_{sim},\mathcal{F},\mathcal{M}^{t_0},\mathcal{E}>  
\end{equation}
%
based on a demographic time-dependent model: 
\begin{equation}
\label{eq:abmModel}
\mathcal{M} \equiv \mathcal{M}(t) \equiv \mathcal{M}^t \equiv \mathcal{M}(P,S,\alpha,D,t)    
\end{equation}
%
where 
%
\begin{itemize} 
\item \underline{$P = P(t)$}: a given population of agents (i.e.\ individuals) at time $t$ evaluated via the model $\mathcal{M}(t)$
%
\item \underline{$S = S(t) = <H(t),W>$}: 
the space on which individuals $p \in P $ are operating,  i.e.\ the set of houses $H(t)$ distributed within the set of towns $W$, cf.\ Section \ref{sec:space} for further detailed insights 
% 
\item 
\underline{$\alpha$}: 
time-independent model parameters, , cf.\ Section \ref{sec:parametersdata:parameters} 
% 
\item 
\underline{$D(t)$}: 
input data integrated into the model as (possibly smoothed) input trajectories, cf.\ Section \ref{sec:parametersdata:data}
%
\item 
\underline{$\alpha_{sim} = (\Delta t, t_0, t_{final}, \alpha_{meta})^T$}:
simulation parameters including a fixed step size and final time-step after which simulation process stops 
\item 
\underline{$\alpha_{meta}$}:
Implementation-dependent simulation parameters, e.g.\ simulation seed for random number generation 
\end{itemize} 
The rest of mathematical symbols are defined in the following subsections.


\subsection{Featured sub-populations (via $\mathcal{M}_{f^*})$} 
\label{sec:generalform:features}

\underline{$\mathcal{F}= \{f_1,f_2,f_3,...\}$}: 
a finite set of elementary features each distinguishes a featured sub-population $$P_f(t) \subseteq P(t)  ~ , ~ f \in \mathcal{F}$$
as defined in Equations \ref{eq:fp} and \ref{eq:Pf} , cf.\ Section \ref{sec:terminologies:fsubpopulations}. 
%
Each featured sub-population $P_f(t)$ is evaluated by the submodel 
%
\begin{equation}
\label{eq:Mf}
f(\mathcal{M}) \equiv f(\mathcal{M}^t) \equiv \mathcal{M}_f^t = \mathcal{M}_{f}(P_{f},S,\alpha,D,t)     
\end{equation}
%
evaluating or predicting the sub-population 
\begin{equation}
\label{eq:Pf}
f(P(t)) \equiv P_f(t) ~\text{ s.t.\ }~ \forall p \in P_f(t) \implies f(p) = true     
\end{equation}
Note that this definition extends to non-elementary features  as well:  
\begin{equation}
\label{eq:fstarM}
f^*(\mathcal{M}) \equiv \mathcal{M}_{f^*}(P_{f^*},S,\alpha,D,t) \text{ for any } f^* \in \bigcup \mathcal{F}     
\end{equation}
%
Such non-elementary features, cf.\ Section \ref{sec:terminologies:nonelemfeatures}, are used to distinguish sub-populations needed for describing the transient processes in agent-based modeling simulation process, cf.\ Section \ref{sec:events}. \\ 

For a given closed set of elementary features as given in Equation \ref{eq:closedSubsetElemFeatures}, the overall population is the union of the elementary features, 
cf.\ Equation \ref{eq:PUnionPf}. 
In that case the comprehensive model $\mathcal{M}$ constitutes of the sum of its elementary featured submodels:
\begin{equation}
\label{eq:MSumMf}
\mathcal{M} \equiv \sum_{f' \in \mathcal{F'}} \mathcal{M}_f'  
\end{equation}



\subsection{Initial population and space (via $\mathcal{M}^{t_0}$)}

\underline{$\mathcal{M}^{t_0}$}: a model that evaluates an initial space and a population at a proposed simulation start time $t_0$. 
% 
The initial model also specifies featured sub-populations via: 
\begin{equation}
\label{eq:Mft0}
\mathcal{M}_f^{t_0}, ~~~ \forall f \in \mathcal{F}    
\end{equation}
%
Consequently, both the corresponding initial population and featured sub-populations:
\begin{equation}
\label{eq:Pt0Pft0}
P(t_0) \text{ and } P_f(t_0), ~~~ \forall f \in \mathcal{F}   
\end{equation}
%
 are specified as well as the initial space: 
\begin{equation}
\label{eq:St0}
    S(t_0) ~=~ <H(t_0)~,~W> 
\end{equation}
i.e., the distribution of an initial set of houses $H(t_0)$ within a set of towns $W$. 


\subsection{Events $\mathcal{E}$}

\underline{$\mathcal{E} = \{e_1, e_2, e_3, ..., e_n\} $}: a finite set of events that transients a particular set of sub-populations evaluated by 
$$ \mathcal{M}_{f^*}(t), ~~  \text{ for some } f^* \in \bigcup \mathcal{F} $$ 
to modified sub-populations predicted by 
$$\mathcal{M}_{f^*}(t + \Delta t)$$ 
% 
Formally,
\begin{equation}
\label{eq:events}
e(\mathcal{M}_{f^*}(t)) = \mathcal{M}_{f^*}(t + \Delta t) \text{ for some } {f^*} \in \bigcup \mathcal{F} ~,~ e \in \mathcal{E}    
\end{equation}
%
The appliance of all events transients the model to the next state 
% 
\begin{equation}
\label{eq:stepping}
\prod_{i=1}^n e_i(\mathcal{M}(t)) = \mathcal{M}(t + \Delta t) 
\end{equation}


\subsection{Single-clocked fixed-step simulation process}

An agent-based simulation process follows the following pattern:
%
\begin{equation}
\label{eq:evolution}
\sum_{t=t_0}^{t_{final}} \prod_{i=1}^{n}  e_i ( \mathcal{M}(t) )  
\end{equation}
% 
Illustratively, the evolution of the population and its featured sub-populations is defined as a sequential application of the events transitions: 
\begin{align*}
\mathcal{M}(t_0) & \text{ evaluating } (P(t_0), P_f(t_0)) ~ \forall f \in \mathcal{F} 
\xRightarrow{\mathcal{E}} \\  
\mathcal{M}(t_0 + \Delta t) &  \text{ evaluating } (P(t_0 + \Delta t), P_f(t_0 + \Delta t)) ~ 
\xRightarrow{\mathcal{E}} \\  
\mathcal{M}(t_0 + 2 \Delta t) &  \text{ evaluating } (P(t_0 + 2\Delta t), P_f(t_0 + 2\Delta t)) ~
\xRightarrow{\mathcal{E}} \\ 
\dots & \dots \\ 
\mathcal{M}(t_{final}) & \text{ evaluating } (P(t_{final}), P_f(t_{final})) 
\end{align*}






