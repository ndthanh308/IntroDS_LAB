\section{Outlook}
\label{sec:outlook}


In this article \comment{Attempt to sell and advocate the presented model despite being a simplified one, why is it still useful to have such a representative model}, the presented terminology is illustrated and demonstrated based on a simple agent-based demographic model of the United Kingdom, intentionally simplified to a superficial level. 
The model is mathematically constructed based on major simplified abstraction from the informal description provided in [citation to LPM paper] \comment{Lone Parent Model citation} associated with the corresponding large-scale code repository implemented in the Python programming language. \\ 

The purpose of this simplification and abstraction is
to have a formal model that serves as  an illustrative example for
\begin{enumerate}
\item 
 a compact but powerful terminology for formal specification of such agent-based demographic models 
\item 
an initial implementation as a starting point to build upon when modeling and simulating a specific and sophisticated case study imposing large-scale demographic context 
\end{enumerate}

The first goal is the main contribution presented in this paper whereas the second goal has been largely 
accomplished 

depending on the context, (e.g.\ pandemics, Socio-economics, immigration, social care etc.), the existing building blocks of the simplified model (or new ones) can be subject to tuning and  modification or insertion) according to the process of model tuning. 
Tuning a pre-given model rather than building it from scratch towards a specific case study may provide the possibility to examine and justify whether a particular tuning step, e.g. enhancing further details, is justifiable possible with qualitative metrics s.a.\ model predictive power. \\ 

While successful appliance of software engineering techniques \comment{Citations to multiple dispatch and others, e.g. High-performance symbolic-numerics via multiple dispatch } shall enhance such incremental model development process, this is not claimed within the scope of this document since implementation details are out of the scope of this work. This remains a promising achievement out of the many future potential extensions based on this work. Namely, the introduced simplified model provides  opportunities to examine: 
%
\begin{itemize}
\item 
Existing state-of-the-art modeling frameworks
\item 
Existing state-of-the-art mathematical tools including but not limited to calibration, sensitivity analysis, stochastic analysis and other mathematical tools for model-based applications among others
\item 
Machine learning based applications s.a.\ surrogate modeling 
\item 
Infrastructure tools for visualization and cloud computing 
\end{itemize}


\subsection*{Materials}

Nevertheless, based on this work, the authors are \comment{probably this paragraph belongs to Outlook} particularly targeting the employment of existing state-of-the-art packages provided in the Julia language. The aim is to identify the overall advantages and the shortages of the Julia-language ecosystem in the context of large-scale sophisticated demographic-based modeling applications. 
It is not desired to establish new tools or novel algorithms from scratch but rather to identify their tuning- or extension-potentials for the purpose of prototyping realistic sophisticated demographic models. 