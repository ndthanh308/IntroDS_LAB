\section{Terminology}
\label{sec:terminology}

This section introduces the basic formal terminologies employed throughout the article for the specification of agent-based models and their simulation process. 

\subsection{Populations $P, M$ and $F$}
\label{sec:terminology:population}

Given that $F(t) = F^t  \equiv F~(M(t) = M^t \equiv M$) is the set of all females (males) in a given population $P(t) = P^t = \equiv P$ at an arbitrary time point $t$ where 
\begin{equation}
\label{eq:population}
    P = M \cup F
\end{equation}
%
$t$ is omitted for the sake of simplification. $t$ is sometimes placed as a superscript, i.e.\ $F^t$, purely for algorithmic specification readability purpose. Similarly, individuals $p \in P(t)$ can be attributed with time, i.e.\ $p^t$, referring to that individual at a particular time point.  


\subsection{Population features $\mathcal{F}$}
\label{sec:features}

Every individual $p \in P$ is attributed by a set of features related to gender, location, age among others.   
%
The elementary population features $f$ considered in the presented model are related to: 
\begin{itemize}
\item \underline{age}, e.g.\ particular age group e.g.\ neonates, children, teenagers, renters, thirties, etc. 
\item \underline{alive status}, i.e. whether alive or dead
\item \underline{gender}, i.e.\ male or female 
\item \underline{location}, e.g.\ sub-population of a particular town 
\item \underline{kinship status} or relationship, e.g. father-ship, parents, orphans, divorcee, singles etc. 
\end{itemize}
%Non-elementary features can be constructed, as already highlighted in Section \ref{sec:terminology} and the submodel evaluating sub-populations can be established as illustrated in Subsection \ref{sec:generalform:features}. 


\subsection{Featured sub-populations $P_f ~ , ~  f \in \bigcup \mathcal{F}$} 
\label{sec:terminologies:fsubpopulations}

Let $P_{f}$ corresponds to the set of all individuals who satisfy a given feature $f$ where 
\begin{equation}
\label{eq:fp}
f(p \in P) = b \in \{true,false\}    
\end{equation}
That is
\begin{equation}
\label{eq:Pf}
P_f = \{ p \in P \text{ s.t.\ } f(p) = true \}    
\end{equation}
%
For example
\begin{itemize}
    \item $M = P_{male}$
    \item $W_{married}$ corresponds to the set of all married women
    \item $P_{age > 65}$ corresponds to all individuals of age older than 65
\end{itemize} 

For the given set of features specified in Section \ref{sec:features}, a subset of features
% 
\begin{equation}
\label{eq:closedSubsetElemFeatures}
F' = \{f'_1,f'_2,\dots \,f'_m\} ~\subset F    
\end{equation}
%
is called \underline{a closed subset of elementary features}, if 
the overall population constitutes of the union of the underlying elementary featured sub-populations, i.e.\ 
\begin{equation}
\label{eq:PUnionPf}
P = P_{\mathcal{F'}} \equiv P_{f'_1 \cup f'_2 \cup \dots \cup f'_m} = P_{f'_1} \cup P_{f'_2} \cup \dots \cup P_{f'_m}     
\end{equation}
For example, male and female gender features constitute a closed set of elementary features.  

\subsection{Non-elementary features $\bigcup \mathcal{F}$}
\label{sec:terminologies:nonelemfeatures}

For a set of elementary features $\mathcal{F}$ (informally those which demands only one descriptive predicate), the set of all non-elementary features $\bigcup \mathcal{F}$ is defined as follows.    
A non-elementary feature
$$f^* \in \bigcup \mathcal{F} ~\text{ where }~ \mathcal{F} \subset \bigcup \mathcal{F}$$ 
can be recursively established from a finite number of arbitrary elementary features $$f_i,f_j,f_k, ... \in \mathcal{F}$$ by 
%
\begin{itemize}
    \item union (e.g.\ $f_i \cup f_j$  ),
    \item intersection (e.g.\ $f_i \cap f_j$)
    \item negation (e.g. $ \neg f_i$)
    \item exclusion or difference (e.g.\ $f_i - f_j$ ) 
\end{itemize}
%
Formally, if 
$$f^* = f_i ~ o ~ f_j \text{ where } o \in \{ \cup, \cap , -\} \text{ and } f_i,f_j \in \bigcup \mathcal{F} $$
then 
\begin{equation}
\label{eq:Pfog}
P_{f^*} = P_{f_i~o~f_j} = \{ p \in P ~ s.t.\ p \in P_{f_i} ~o~ P_{f_j} \}  
\end{equation}
%
Analogously,
\begin{equation}
\label{eq:neg}
P_{\neg f}  = \{ p \in P ~ s.t.\ p \notin P_f   \} ~\text{ with }  f \in  \bigcup \mathcal{F}     
\end{equation}
%
Negation operator s.a.\ ($\neg$) is beneficial for sub-population specification, e.g. 
\begin{equation}
\label{eq:intersectionNegationEx}
F_{married ~\cap~ \neg hasChildren}     
\end{equation}
corresponds to all married females without children. This sub-population can be equivalently described using the difference operator: 
\begin{equation}
\label{eq:differenceEx}
F_{married ~-~ hasChildren}    
\end{equation}
which entails to be a matter of style unless algorithmic execution details of the operators are assumed. For instance, it can be assumed that in intermediate computation the $-$ operator is operated directly on the set of married females rather than the set of all females. \\

Generally, any of the features $f_i$ and $f_j$ in Equation \ref{eq:Pfog} can be either elementary or non-elementary and the definition is recursive allowing the construction of an arbitrary set of non-elementary features. 
For example, the sub-population  
\begin{equation}
\label{eq:example:nonelementary}
M_{divorced ~ \cap ~ hasChildren ~ \cap  ~ age>45 ~ - ~ hasSiblings}
\end{equation}
corresponds to the set of all divorced men of age older than 45 who has no siblings but they have children. 
In order to improve readability, 
equation \ref{eq:example:nonelementary} can be re-written as:
\begin{equation}
\label{eq:example:nonelem:readible}
M_{divorced} ~ \cap ~ M_{hasChildren} ~ \cap ~ M_{age>45} ~ - ~ M_{hasSiblings}     
\end{equation}
Both styles can be mixed together for readability purpose, cf.\ Section \ref{sec:events:deaths}.


\subsection{Composition operator $f(g)$} 

Another beneficial operator is the composition operator analogously defined as 
\begin{equation}
\label{eq:Pfg}
P_{f(g)} = \{ p \in P_f \text{ s.t.\ } g(p) = true \}
\end{equation}

%\begin{equation}
%\label{eq:Pfg}
%P_{f(g)} = \{ p \in P_g \text{ s.t.\ } f(p) = true \}
%\end{equation}

Based on the definition, the composition operator is 
not symmetric as the case with the intersection operator. 
For example, 
$$ M_{isAlive(isSibling)}  ~\neq~ M_{isSibling(isAlive)}$$
The left hand side refers to the set of siblings of the alive male population, while the right hand side refers to alive siblings of the male populations. 
Moreover, the composition operator can be regarded to be more  computationally efficient in comparison with the intersection operator\footnote{In this work, the main purpose behind the composition operator mainly remains in the context of algorithmic specification rather than enforcing any implementation details regarding computational efficiency}. \\ 

The desired sub-population specification in the example given by equation \ref{eq:example:nonelem:readible} may not correspond to the desired specification. Namely, desired is to specify the alive divorced male population older than 45 years with alive children and alive siblings. 
In this case, the employment of the composition operator is relevant:
\begin{equation}
\label{eq:whatelse_complex}
 M_{isAlive(isDivorced)~} ~\cap~ M_{|children(isAlive)| > 0 ~\cap~ age>45 ~-~ |sibling(isAlive)| > 0}   
\end{equation}
Nevertheless, to retain the desired simplicity, the previous equation can be rather rewritten as  
\begin{equation}
\label{eq:whatelse}
 M_{isAlive(isDivorced ~\cap~ hasAliveChildren ~\cap~ age>45 ~-~ hasAliveSibling)}   
\end{equation}






 
