\section{The space -- detailed description} 
\label{sec:space}

%\newcommand\setItemnumber[1]{\setcounter{enumi}{\numexpr#1-1\relax}}

In the sake of comprehensive description of the space, further assumptions are listed as follows: 
\begin{enumerate}[label=\textbf{S. \arabic*}]
    \setItemnumber{8}
    \item 
    The static set of towns of UK, cf.\ Assumption \ref{ass:space:staticTowns}, are projected as a rectangular $12 \times 8$ grid with each point in the grid corresponding to a town
\end{enumerate}
% 
Formally, assuming that 
$$location(w_{(x,y)}) = (x,y) $$ 
then
% 
\begin{enumerate}[label=\textbf{S. \arabic*}]
\setItemnumber{9}
\item the town $w_{(1,1)}$ corresponds to the north-est west-est town of UK whereas 
\item the town $w_{(12,8)}$ corresponds to the south-est east-est town of UK 
\item  the distances between towns are commonly defined, e.g.\ 
\begin{equation}
\label{eq:manhattan}
\text{manhattan-distance}(w_{(x_1,y_1)} , w_{(x_2,y_2)}) = \mid x_1 - x_2 \mid + \mid y_1 - y_2 \mid
\end{equation}
%\item 
%\label{ass:space:adjacency}
%two towns $w_{(x_1,y_1)}$ and $ w_{(x_2,y_2)}$ are adjacent (or neighbors) if 
%\begin{equation}
%\mid x_1 - x_2 \mid \leq 1  ~\text{and}~  \mid y_1 - y_2 \mid \leq 1 
%\end{equation}
\end{enumerate}

The (initial) population and houses distribution within UK towns are approximated by am ad-hoc pre-given UK population density map.
The map is projected as a rectangular matrix  
%
\begin{equation}
\label{eq:M}
M  \in R^{12 \times 8} \approx 
\begin{bmatrix} 
0.0 & 0.1 & 0.2 & 0.1 & 0.0 & 0.0 & 0.0 &  0.0 \\
0.1 & 0.1 & 0.2 & 0.2 & 0.3 & 0.0 & 0.0 & 0.0 \\
0.0 & 0.2 & 0.2 & 0.3 & 0.0 & 0.0 & 0.0 & 0.0 \\
0.0 & 0.2 & 1.0 & 0.5 & 0.0 & 0.0 & 0.0 & 0.0 \\ 
0.4 & 0.0 & 0.2 & 0.2 & 0.4 & 0.0 & 0.0 & 0.0 \\
0.6 & 0.0 & 0.0 & 0.3 & 0.8 & 0.2 & 0.0 & 0.0 \\ 
0.0 & 0.0 & 0.0 & 0.6 & 0.8 & 0.4 & 0.0 & 0.0 \\ 
0.0 & 0.0 & 0.2 & 1.0 & 0.8 & 0.6 & 0.1 & 0.0 \\ 
0.0 & 0.0 & 0.1 & 0.2 & 1.0 & 0.6 & 0.3 & 0.4 \\ 
0.0 & 0.0 & 0.5 & 0.7 & 0.5 & 1.0 & 1.0 & 0.0 \\ 
0.0 & 0.0 & 0.2 & 0.4 & 0.6 & 1.0 & 1.0 & 0.0 \\ 
0.0 & 0.2 & 0.3 & 0.0 & 0.0 & 0.0 & 0.0 & 0.0
\end{bmatrix}
\end{equation} 
%

It can be observed for instance that
\begin{itemize}%[label=\textbf{S. \arabic*}]
%\setItemnumber{13}
\item 
cells with density $0$ (i.e.\ realistically, with very low-population density) don't correspond to inhabited towns  
\item 
the towns in UK are merged into 48 towns
\item e.g. the center of the capital London spans the cells $(10,6), (10,7), (11,6)$ and $ (11,7)$ 
\end{itemize}
\begin{enumerate}[label=\textbf{S. \arabic*}]
\setItemnumber{12}
\item 
\label{ass:space:selectOrCreateEmptyHouse}
if an empty house $h$ is demanded in a particular town $w \in W$, an empty house is randomly selected from the set of existing houses $W_w$ in that town. If no empty house exists, a new empty house is established in conformance with assumption \ref{ass:space:housingUniformLocations}
\item 
\label{ass:space:houseInArbitraryTown}
if an empty house $h$ is demanded in an arbitrary town, a town is selected via a random weighted selection: 
\begin{align}
\label{eq:ass:hometowLocation}
 town(h) =  random(W,M^T) % \text{ where } 
   % W_{xy} = \{w_{(x,y)} | w_{(x,y)} \in W \text{ and } M_{(y,x)} > 0 \}, \nonumber \\  
\end{align}
an empty house is selected or established according to the previous assumption 
\end{enumerate}
%
Further details on the initial set of houses is given in Section \ref{sec:initialization:spatial}. 