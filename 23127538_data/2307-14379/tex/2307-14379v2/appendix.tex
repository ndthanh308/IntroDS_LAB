\section{Appendix A: The Proper Detector Frame}
\label{app:PDF}
The proper detector frame (PDF) is a coordinate system that describes the local Lorentz frame of an observer in curved spacetime \cite{MARZ1994}. With respect to the cavity, it properly describes the response of the EM-field to a small metric perturbation in its local lorentz frame. We therefore need it to describe the gauge-dependent Gertsenshtein current of a passing GW. \\
In general, the PDF is a combination of Fermi Normal Coordinates (FNC) \cite{MTW1973, MANMIS1963} and the acceleration and rotation of the observer \cite{NIZIM1978}. As shown by Marzlin in 1994 \cite{MARZ1994}, the metric can be then written as
\begin{align}
    g_{00}&=-(1+\vb{a}\cdot\vb{x})^2+(\vb{\omega}\times\vb{x})^2-\gamma_{00}-2(\vb{\omega}\times\vb{x})^i\gamma_{0i}-(\vb{\omega}\times\vb{x})^i(\vb{\omega}\times\vb{x})^j\gamma_{ij}\notag, \\
    \label{eqn:GeneralFermiNormalCoords}
    g_{0i}&=(\vb{\omega}\times\vb{x})_i-\gamma_{0i}-(\vb{\omega}\times\vb{x})^j\gamma_{ij}, \\
    g_{ij}&=\delta_{ij}-\gamma_{ij},\notag
\end{align}
where the coefficients are given by the series expansions
\begin{align*}
    \gamma_{00}&=\sum_{n=0}^\infty\frac{2}{(n+3)!}x^kx^lx^{k_1}\cdots x^{k_n}(\partial_{k_1}\cdots\partial_{k_n}R_{0k0l})(g)\cdot\Big[ (n+3)+2(n+2)\vb{a}\vb{x}+(n+1)(\vb{a}\vb{x})^2 \Big], \\
    \gamma_{0i}&=\sum_{n=0}^\infty\frac{2}{(n+3)!}x^kx^lx^{k_1}\cdots x^{k_n}(\partial_{k_1}\cdots\partial_{k_n}R_{0kil})(g)\cdot\Big[(n+2)+(n+1)\vb{a}\vb{x}\Big], \\
    \gamma_{ij}&=\sum_{n=0}^\infty\frac{2}{(n+3)!}x^kx^lx^{k_1}\cdots x^{k_n}(\partial_{k_1}\cdots\partial_{k_n}R_{ikjl})(g)\cdot\Big[ n+1 \Big],
\end{align*}
where $\vb{a}$ and $\vb{\omega}$ are the acceleration and rotation and $g$ the geodesics of the observer (reference geodesic). 
Note that the Riemann tensor is gauge independent. For GWs, it is therefore possible to compute it in the more convenient TT-gauge. \\
With SRF experiments, the goal is to measure GWs with frequencies 
in the range $\sim\mathcal{O}(\text{kHz-MHz})$, which is much above the typical variations of the gravitational field on earth with values of $f\lesssim 0.1\,\text{Hz}$ \cite{CAP2020}. We can therefore well separate the GWs from the background field and set $\vb{a}=0$ and $\vb{\omega}=0$. The resulting simplified expansion for the GW strain reads
\begin{align}
    \label{eqn:FullResult00Component}
    h_{00}&=-2\sum_{n=0}^\infty\frac{n+3}{(n+3)!}x^kx^lx^{k_1}\cdots x^{k_n}(\partial_{k_1}\cdots\partial_{k_n}R_{0k0l})(g), \\
    \label{eqn:FullResult0iComponent}
    h_{0i}&=-2\sum_{n=0}^\infty\frac{n+2}{(n+3)!}x^kx^lx^{k_1}\cdots x^{k_n}(\partial_{k_1}\cdots\partial_{k_n}R_{0kil})(g), \\
    \label{eqn:FullResultijComponent}
    h_{ij}&=-2\sum_{n=0}^\infty\frac{n+1}{(n+3)!}x^kx^lx^{k_1}\cdots x^{k_n}(\partial_{k_1}\cdots\partial_{k_n}R_{ikjl})(g).
\end{align}
We refer the reader to \cite{RAKH2014} for a more detailed discussion of FNC. In our case where the expected GW frequency is below the GHz-regime, we can apply the long wavelength approximation. That means, the Riemann tensor is independent of the spatial coordinates and the expansion can be cut off at second order. This leads to a vastly simplified metric (eqn. \ref{eqn:PDFMetric}), which can be used for calculating the Gertsenshtein current. Note, however, that the full expansion is needed in the GHz-regime and above. More details can be found in \cite{BERLIN2023, BERLIN2021}.

\section{Appendix B: Cavity Perturbation Theory}
\label{app:CavityPerturbationTheory}
When a GW propagates through a cavity, it changes the boundary conditions of the electromagnetic field. The eigenmodes of the deformed cavity are in general different from the eigenmodes of the unperturbed one. However, the GW strains are very small ($\lesssim\mathcal{O}(10^{-21})$), so cavity perturbation theory can be applied. That means, we can express the perturbed modes as series expansions of the unperturbed modes. We are then interested in the resulting overlap given by the coefficients of the expansion. An important result of this procedure is that the perturbed mode $\vb{E}'_n$ appears to be strongly coupled to its unperturbed counterpart $\vb{E}_n$, but also has contributions from other modes $\vb{E}_m$ with $m\neq n$. \\
There are several approaches to construct such an expansion. We will use the method given in \cite{GOUBAU1961} as it is consistent with the method applied in \cite{BERN2002}. The main idea is to find an expression for the deformed boundary conditions at the position of the unperturbed shell. The advantage of this approach is that we do not have to deal with a perturbed volume $V'_{\text{cav}}$ and can therefore work with $V_{\text{cav}}$ throughout the calculation. \\
We discuss the formalism in detail here, since \cite{GOUBAU1961} contains some inconsistencies. We further present the arguments in a new and improved way using a modern notation.

\subsection{The Perturbed Boundary Condition}
The unperturbed shell has surface $S$ while $S'$ denotes the surface of the perturbed shell. We note that the electromagnetic field in both cavities is described by the boundary value problem (BVP)
\begin{align}
    \label{eqn:PerturbedandUnperturbedBVP}
    \nabla\times\vb{E}_n&=ck_n\vb{B}_n, \qquad & \qquad \nabla\times\vb{E}'_n&=ck'_n\vb{B}'_n,\notag\\
    \nabla\times\vb{B}_n&=\frac{k_n}{c}\vb{E}_n, \qquad & \qquad \nabla\times\vb{B}'_n&=\frac{k'_n}{c}\vb{E}'_n, \\
    \vb{n}\times\vb{E}_n\vert_S&=0, \qquad & \qquad \vb{n}'\times\vb{E}'_n\vert_{S'}&=0. \notag 
\end{align}
We will use $\omega_n=ck_n$ instead of $k_n$ from now on. Our goal is to find the equivalent of the boundary condition $\vb{n}'\times\vb{E}'_n\vert_{S'}=0$ on $S$. Since $S$ is supposed to be a (at least piecewise) smooth manifold, we can parameterise it with two variables $\lambda_1$ and $\lambda_2$. We then define two differentiable curves
\begin{align*}
    \vb{u}_1 &:= \vb{u}_{\lambda_2}(\lambda_1):=\vb{S}(\lambda_1,\lambda_2)\vert_{\lambda_2\text{ fixed}}, \\
    \vb{u}_2 &:= \vb{u}_{\lambda_1}(\lambda_2):=\vb{S}(\lambda_1,\lambda_2)\vert_{\lambda_1\text{ fixed}},
\end{align*}
such that the tangential vectors
% Figure environment removed
\begin{equation*}
    \vb{t}_1:=\firstdevp{\vb{u}_1}{\lambda_1}, \qquad \qquad \vb{t}_2=\firstdevp{\vb{u}_2}{\lambda_2}
\end{equation*}
define a right-handed orthonormal system ($\vb{t}_1,\vb{t}_2,\vb{n}$), where $\vb{n}$ is the surface normal. The displacement is described by $\Delta(\vb{x})$, which gives the absolute value of the shell deformation at a point $\vb{x}$ on the surface. We set $\Delta<0$ for inward and $\Delta>0$ for outward deformations. Throughout the following discussion, we will assume that $|\Delta(\vb{x})|\ll 1$. \\
We start by going the infinitesimal distances
\begin{equation*}
    \dd\vb{u}_1 = \vb{t}_1\dd\lambda_1, \qquad \qquad \qquad \dd\vb{u}_2 = \vb{t}_2\dd\lambda_2
\end{equation*}
on the unperturbed surface $S$. We then move along a closed path using $\vb{n}\Delta$ to jump on the perturbed surface $S'$. In figure \ref{fig:BoundaryConditions}, it is shown for inward and outward deformation. We consider the surface elements within the path, which are given by
\begin{equation*}
    \dd\vb{A}_1=\pm\vb{t}_2\dd \lambda_1\Delta, \qquad \qquad \dd\vb{A}_2=\mp\vb{t}_1\dd \lambda_2 \Delta.
\end{equation*}
Note that the upper sign corresponds to the inward direction and the lower sign to the outward direction. The key idea now is to apply Stokes theorem. It is useful to look at fig. \ref{fig:BoundaryConditions} to track the signs correctly. Weighting the surface elements $\dd\vb{A}_1$ and $\dd\vb{A}_2$ with $\nabla\times\vb{E}'_n$ leads to
\begin{align*}    
    \nabla\times\vb{E}'_n\cdot\dd\vb{A}_1&= \pm\nabla\times\vb{E}'_n\cdot\vb{t}_2\Delta\dd \lambda_1 \\
    &= \pm\vb{E}'_n\Delta\vb{n}\mp\vb{E}'_n\Delta\vb{n} \mp \dd\lambda_1\firstdevp{ }{\lambda_1}(\vb{E}'_n\vb{n}\Delta)\mp\vb{E}'_n\vb{t}_1\dd \lambda_1 \\
    &= \mp\vb{E}'_n\vb{t}_1\dd \lambda_1 \mp \dd\lambda_1\firstdevp{ }{\lambda_1}(\vb{E}'_n\vb{n}\Delta), \\                
   \nabla\times\vb{E}'_n\cdot\dd\vb{A}_2&= \mp\nabla\times\vb{E}'_n\cdot\vb{t}_1\Delta\dd\lambda_2 \\
    &= \pm\vb{E}'_n\Delta\vb{n}\mp\vb{E}'_n\Delta\vb{n} \mp \dd\lambda_2\firstdevp{ }{\lambda_2}(\vb{E}'_n\vb{n}\Delta)\mp\vb{E}'_n\vb{t}_2\dd \lambda_2 \\
    &= \mp\vb{E}'_n\vb{t}_2\dd \lambda_2 \mp \dd \lambda_2\firstdevp{ }{\lambda_2}(\vb{E}'_n\vb{n}\Delta),
\end{align*}
where we expanded $(\vb{E}'_n\vb{n}\Delta)(\vb{u}_{1,2}+\dd\lambda_{1,2}\vb{t}_{1,2})$ up to first order and used that $\vb{E}'_n\vert_S'=0$ on the perturbed surface (see fig. \ref{fig:BoundaryConditions}).
By eliminating $\dd \lambda_1$ and $\dd\lambda_2$, we find
\begin{align*}
    \vb{E}'_n\cdot\vb{t}_1&=-\nabla\times\vb{E}'_n\cdot\vb{t}_2\Delta-\firstdevp{ }{\lambda_1}(\vb{E}'_n\vb{n}\Delta), \\
    \vb{E}'_n\cdot\vb{t}_2&=\nabla\times\vb{E}'_n\cdot\vb{t}_1\Delta-\firstdevp{ }{\lambda_2}(\vb{E}'_n\vb{n}\Delta).
\end{align*}
These results can be now combined to
\begin{align*}
    \vb{E}'_n&=(\vb{E}'_n\vb{t}_1)\cdot\vb{t}_1+(\vb{E}'_n\vb{t}_2)\cdot\vb{t}_2+(\vb{E}'_n\vb{n})\cdot\vb{n} \\
    &= -(\nabla\times\vb{E}'_n\vb{t}_2\Delta)\cdot\vb{t}_1+(\nabla\times\vb{E}'_n\vb{t}_1\Delta)\cdot\vb{t}_2 \\
    &\qquad\qquad -\firstdevp{ }{\lambda_1}(\vb{E}'_n\vb{n}\Delta)\cdot\vb{t}_1-\firstdevp{ }{\lambda_2}(\vb{E}'_n\vb{n}\Delta)\cdot\vb{t}_2+(\vb{E}'_n\vb{n})\cdot\vb{n}.
\end{align*}
On the shell of the unperturbed cavity, this expression reads
\begin{equation*}
    \vb{E}'_n\vert_S=\vb{n}\times(\nabla\times\vb{E}'_n)\Delta\vert_S-\nabla(\vb{E}'_n\vb{n}\Delta)\vert_S+(\vb{E}'_n\vb{n})\cdot\vb{n}\vert_S,
\end{equation*}
where we used the standard gradient in the coordinate system ($\vb{t}_1$, $\vb{t}_2$, $\vb{n}$) together with the identity $\vb{a}\times(\vb{b}\times\vb{c})=\vb{b}\cdot(\vb{a}\cdot\vb{c})-\vb{c}\cdot(\vb{a}\cdot\vb{b})$. Inserting eqn.  \ref{eqn:PerturbedandUnperturbedBVP} finally yields the perturbed version of the boundary condition $\vb{n}\times\vb{E}_n\vert_S=0$. The result is 
\begin{equation*}           
   \vb{n}\times\vb{E}'_n\vert_S= \Delta(\omega_n\vb{B}_n\times\vb{n})\times\vb{n}\vert_S+\nabla(\vb{E}_n\vb{n}\Delta)\times\vb{n}\vert_S,
\end{equation*}
so the perturbed electric field does not vanish in the unperturbed shell. This will now help us to find a series expansion for $\vb{E}'_n$ (or $\vb{B}'_n$) in terms of $\vb{E}_n$ (or $\vb{B}_n$). Note that we have dropped the primes on the right hand side as we assumed $\Delta$ to be small. It is therefore sufficient to consider leading order terms only.

\subsection{Solving the Boundary Value Problem}
According to the general idea of perturbation theory, we can decompose the perturbed eigenmodes as \cite{BERN2002}
\begin{align*}
    \vb{E}'_n &= \vb{E}_n+\cormode{\vb{E}}+\mathcal{O}(\sigma^2), \\
    \vb{B}'_n &= \vb{B}_n+\cormode{\vb{B}}+\mathcal{O}(\sigma^2), \\
    \omega'_n &= \omega_n+\cormode{\omega}+\mathcal{O}(\sigma^2).
\end{align*}
Substituting this into the perturbed BVP and using the unperturbed BVP (see eqn. \ref{eqn:PerturbedandUnperturbedBVP}), we obtain a BVP for the first order corrections $\cormode{\vb{E}}$ and $\cormode{\vb{B}}$. It can be written as 
\begin{align}
    \label{eqn:PerturbedDE1}
    \nabla\times\cormode{\vb{E}}-\omega_n\cormode{\vb{B}}&=\cormode{\omega}\vb{B}_n, \\
    \label{eqn:PerturbedDE2}
    \nabla\times\cormode{\vb{B}}-\frac{\omega_n}{c^2}\cormode{\vb{E}}&=\frac{\cormode{\omega}}{c^2}\vb{E}_n, \\
    \label{eqn:PerturbedBC}
    \vb{n}\times\cormode{\vb{E}}\vert_S=\omega_n\vb{V_n}\vert_S,
\end{align}
where we again consider leading order terms in $\Delta$ and $\sigma$ only. In order to abbreviate notation, we have defined 
\begin{equation}
    \label{eqn:BoundaryAbbreviation}
    \vb{V}_n:=\Delta (\vb{B}_n\times\vb{n})\times\vb{n}\vert_S+\frac{1}{\omega_n}\nabla(\vb{E}_n\vb{n}\Delta)\times\vb{n}\vert_S
\end{equation}
here. We can now expand the first order corrections in terms of the unperturbed modes, i.e. 
\begin{align}
    \label{eqn:EExpansion}
    \cormode{\vb{E}}&=\sum_m\alpha_{nm}\vb{E}_m, \\
    \label{eqn:BExpansion}
    \cormode{\vb{B}}&=\sum_m\beta_{nm}\vb{B}_m, \\
    \label{eqn:kExpansion}
    \cormode{\omega}&=\sum_m\kappa_{nm}\omega_m.
\end{align}
The remaining task then is to find the coefficients $\alpha_{nm}$, $\beta_{nm}$ and $\kappa_{nm}$.  We start by integrating equation \ref{eqn:PerturbedDE1} over $\vb{B}_m$ such that
\begin{align}
    \label{eqn:PerturbedPrelResult1}
    \intcav\dd^3x\vb{B}_m\cdot\nabla\times\cormode{\vb{E}}-\omega_n\intcav\dd^3x\vb{B}_m,\cdot\cormode{\vb{B}} &= \cormode{\omega}\delta_{nm}\intcav\dd^3x\vb{B}^2_n.
\end{align}
Equivalently, we can integrate equation \ref{eqn:PerturbedDE2} over $\vb{E}_m$ which leads to a similar expression with $\vb{B}$ and $\vb{E}$ exchanged. 
Using standard nabla identities and Gauss's law, we can rewrite the first integral of eqn. \ref{eqn:PerturbedPrelResult1} as
\begin{align*}
    \intcav\dd^3x\vb{B}_m\cdot\nabla\times\cormode{\vb{E}}&=-\intcavbound\dd\vb{S}(\vb{B}_m\times\cormode{\vb{E}})+\intcav\dd^3x\cormode{\vb{E}}\nabla\times\vb{B}_m. 
\end{align*}
To evaluate the surface integral, we can use the boundary conditions in \ref{eqn:PerturbedandUnperturbedBVP} and \ref{eqn:PerturbedBC}. Note that there is now a difference between the E-field and B-field because
\begin{align*}
    \dd\vb{S}(\vb{B}_m\times\cormode{\vb{E}})&=\vb{n}\cdot(\vb{B}_m\times\cormode{\vb{E}})\dd S = \vb{B}_m\cdot(\cormode{\vb{E}}\times\vb{n})\dd S = -\omega_n\vb{B}_m\cdot\vb{V}_n\dd S \\
    \dd\vb{S}(\vb{E}_m\times\cormode{\vb{B}})&=\vb{n}\cdot(\vb{E}_m\times\cormode{\vb{B}})\dd S=\cormode{\vb{B}}\cdot(\vb{n}\times\vb{E}_m)\dd S = 0.
\end{align*}
With these results and using eqn. \ref{eqn:PerturbedandUnperturbedBVP}, we can write equation \ref{eqn:PerturbedPrelResult1} as
\begin{align*}
    \frac{\omega_m}{c^2}\intcav\dd^3x\cormode{\vb{E}}\vb{E}_m&- \omega_n\intcav\dd^3x\cormode{\vb{B}}\cdot\vb{B}_m \\ &= \cormode{\omega}\delta_{nm}\intcav\dd^3x\vb{B}^2_n-\omega_n\intcavbound\dd S \vb{B}_m\cdot\vb{V}_n, \\
    \omega_m\intcav\dd^3x\cormode{\vb{B}}\vb{B}_m&-\frac{\omega_n}{c^2}\intcav\dd^3x\cormode{\vb{E}}\vb{E}_m \\ &= \frac{\cormode{\omega}}{c^2}\delta_{nm}\intcav\dd^3x\vb{E}^2_n,
\end{align*}
where we also gave the corresponding expression for the B-field. The next step is to insert the expansions eqn. \ref{eqn:EExpansion}-\ref{eqn:BExpansion}. We can use eqn. \ref{eqn:Normalization} to simplify the notation and arrive at
\begin{align}
    \label{eqn:DefiningEquation1}
    \frac{\omega_m}{c^2}\alpha_{nm}\frac{2U_m}{\epsilon_0}-\omega_n\beta_{nm}2\mu_0U_m &= \delta_{nm}\cormode{\omega}2\mu_0U_m+\frac{2U_m}{\epsilon_0}\frac{\omega_n}{c^2}\mathcal{C}_{nm}, \\
    \label{eqn:DefiningEquation2}
    \omega_m\beta_{nm}2\mu_0U_m-\frac{\omega_n}{c^2}\alpha_{nm}\frac{2U_m}{\epsilon_0}&=\frac{\cormode{\omega}}{c^2}\delta_{nm}\frac{2U_n}{\epsilon_0},
\end{align}
where a new coupling coefficient is defined by
\begin{equation}
    \label{eqn:OriginalCouplingCoefficient}
    \mathcal{C}_{nm}:=-\frac{c^2}{2U_m}\intcavbound\dd S\epsilon_0\vb{B}_m\vb{V}_n.
\end{equation}
To find the coefficients $\alpha_{nm}$, $\beta_{nm}$ and $\kappa_{nm}$, we have to solve eqn. \ref{eqn:DefiningEquation1} and \ref{eqn:DefiningEquation2}. Therefore, we have to distinguish between the cases $n=m$ and $n\neq m$. We start with the latter, which yields
\begin{align}
    \alpha_{nm}&=\frac{\omega_m \omega_n}{\omega^2_m-\omega^2_n}\mathcal{C}_{nm}, \\ \label{eqn:BetaCoefficients}\beta_{nm}&=\frac{\omega^2_n}{\omega^2_m-\omega^2_n}\mathcal{C}_{nm}.
\end{align}
The case $n=m$ needs a bit more work. From eqn. \ref{eqn:DefiningEquation1} and \ref{eqn:DefiningEquation2}, we directly find
\begin{equation*}
    \cormode{\omega}=-\frac{1}{2}\omega_n\mathcal{C}_{nn}.
\end{equation*}
This leads to a solution for $\kappa_{nm}$ and an expression for $\alpha_{nn}$ and $\beta_{nn}$, which read
\begin{align*}
    \kappa_{nm}&=-\frac{1}{2}\delta_{nm}\mathcal{C}_{nm}, \\
    \alpha_{nn}&=\beta_{nn}+\frac{1}{2}\mathcal{C}_{nn}.
\end{align*}
However, we have to fix another degree of freedom to get a final result for the remaining coefficients. That is because we have not yet chosen a normalization for the perturbed fields. An appropriate choice is to define
\begin{equation}
    \label{eqn:PerturbedNormalization}
    \intcav\dd^3x\vb{E}'^2_n:=\frac{2U_n}{\epsilon_0}=\intcav\dd^3x\vb{E}^2_n.
\end{equation}
By observing that
\begin{equation*}
    \frac{2U_n}{\epsilon_0}=\intcav\dd^3x(\vb{E}_n+\cormode{\vb{E}})^2 = (1+2\alpha_{nn})\frac{2U_n}{\epsilon_0},
\end{equation*}
we find that the diagonal coefficients are asymmetric and given by
\begin{equation}
    \label{eqn:CoefficientsDiagonalComponents}
    \alpha_{nn}=0, \qquad \qquad \qquad \beta_{nn}=-\frac{1}{2}\mathcal{C}_{nn}.
\end{equation}
Finally, we write the perturbed solutions in terms of the time modes $e_n(t)$ and $b_n(t)$. By substituting the expansions \ref{eqn:EExpansion}-\ref{eqn:kExpansion} into eqn. \ref{eqn:TimeExpansion}, we end up with
\begin{align}
    \label{eqn:PerturbedTimeModee}
    e'_n(t)&=e_n(t)+\sum_{m\neq n}\alpha_{nm}\frac{U_m}{U_n}e_m(t),  & \qquad  \alpha_{nm}&=\frac{\omega_n \omega_m}{\omega^2_m-\omega^2_n}\mathcal{C}_{nm}, \\
     \label{eqn:PerturbedTimeModeb}
     b'_n(t)&=b_n(t)-\frac{1}{2}\mathcal{C}_{nn}b_n(t)+\sum_{m\neq n}\frac{U_m}{U_n}\beta_{nm}b_m(t), & \qquad  \beta_{nm}&=\frac{\omega^2_n}{\omega^2_m-\omega^2_n}\mathcal{C}_{nm}, \\
    \omega'_n&=\omega_n-\frac{1}{2}\omega_n\mathcal{C}_{nn}.
\end{align}
The remaining task is to determine the connection coefficients $\mathcal{C}_{nm}$.

\subsection{The Connection Coefficient}
We return to eqn. \ref{eqn:BoundaryAbbreviation} and eqn. \ref{eqn:OriginalCouplingCoefficient}, which define the connection coefficient $\mathcal{C}_{nm}$. The full expression reads
\begin{equation*}
    \mathcal{C}_{nm}=-\frac{c^2}{2U_m}\intcavbound\dd S\epsilon_0\vb{B}_m\Big[ (\Delta\vb{B}_n\times\vb{n})\times\vb{n}+\frac{1}{\omega_n}\nabla(\vb{n}\vb{E}_n \Delta)\times\vb{n} \Big].
\end{equation*}
We can write this in a shorter form by using the boundary condition $\vb{B}_n\cdot\vb{n}\vert_S=0$ for the unperturbed cavity. The left integral can be then written as
\begin{equation*}
    \intcavbound\dd S\cdot\vb{B}_m(\vb{B}_n\times\vb{n})\times\vb{n}\Delta = -\intcavbound\dd S\cdot\vb{B}_m\vb{B}_n\Delta.
\end{equation*}
For the right integral, we use that $\dd\vb{S}=\vb{n}\dd S$ and eqn. \ref{eqn:PerturbedandUnperturbedBVP} to find
\begin{align*}
    \intcavbound\dd S&\cdot\epsilon_0\vb{B}_m\nabla(\vb{n}\vb{E}_n\Delta)\times\vb{n} \\
    &= \omega_m\intcavbound\dd S\cdot\frac{\epsilon_0}{c^2}\Delta(\vb{n}\vb{E}_n)(\vb{n}\vb{E}_m)-\intcavbound\dd\vb{S}\cdot\nabla\times((\vb{n}\vb{E}_n)\vb{B}_m\Delta).
\end{align*}
Using the boundary condition $\vb{E}_{n,m}\times\vb{n}\vert_S=0$, we can write 
\begin{equation*}
    (\vb{n}\cdot\vb{E}_n)(\vb{n}\cdot\vb{E}_m)\vert_S=\vb{E}_n\cdot\vb{E}_m\vert_S.
\end{equation*}
in the first integral. The second integral vanishes due to Stoke's law. Combining all results leads to the relation
\begin{equation*}
    \mathcal{C}_{nm}=\frac{1}{2U_m}\intcavbound\dd S\cdot\Delta\Big[ \frac{1}{\mu_0}\vb{B}_n\vb{B}_m-\frac{\omega_m}{\omega_n}\epsilon_0\vb{E}_n\vb{E}_m \Big].
\end{equation*}
Note that in cases where $\omega_m\approx \omega_n$ like in heterodyne cavity experiments, we can write the simplified form 
\begin{equation}    
    \label{eqn:GeneralConnectionCoefficient}
    \mathcal{C}_{nm}\approx\frac{1}{2U_m}\intcavbound\dd S\cdot\Delta\Big[ \frac{1}{\mu_0}\vb{B}_n\vb{B}_m-\epsilon_0\vb{E}_n\vb{E}_m \Big].
\end{equation}