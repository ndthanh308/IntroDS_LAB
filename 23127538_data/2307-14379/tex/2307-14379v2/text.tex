\section{Introduction}
Since the first detection of gravitational waves (GWs) in 2016 by the LIGO and Virgo collaboration \cite{ABBOTT2016}, there is a rising interest on GW experiments probing the ultra high frequency regime beyond $10\,\text{kHz}$ \cite{AGG2021}. Because no source in this regime is known in the standard model of particle physics and cosmology, a detection would point towards new physics. Recent studies \cite{BERLIN2023} showed that heterodyne experiments using superconducting radio frequency (SRF) cavities are able to push the sensitivity towards a promising window for new sources. 
In particular, black hole superradiance would produce a very suitable signal for this approach \cite{BRIT2020}. The idea was initially worked out in the 1970s by several studies \cite{BRAG1971, BRAG1974, CAVES1979, BERN1978, PEGO1978, PEGO1980}. First experiments began in 1984 \cite{REECE1984} which led to further efforts by the MAGO collaboration at INFN in the late 1990s \cite{BALLA2005, BERN2001, BERNARD1998, BERN1998, BERN2002}. The concept is based on two eigenmodes of an electromagnetic cavity which are nearly degenerate. One eigenmode is excited by an external oscillator (pump mode) whereas the other (signal mode) is coupled to a readout system that measures the electromagnetic field power. When a GW passes by, it can induce a transition of photons from the pump mode into the signal mode, leading to an enhanced power loss at the readout. The signal reaches a maximum when the GW is resonant to the frequency difference between both levels. This allows using superconducting radio frequency (SRF) cavities to scan over a wide frequency range from $1\,\text{kHz}$ to several GHz. Measurements in the superconducting state of the cavity allow for very high electromagnetic quality factors ($Q\sim10^{10}$) \cite{JACK2006, BERLIN2020, BERLIN2021} which are mandatory to distinguish the levels at small frequency differences. One should note that the heterodyne approach is comparable to classical Webar Bar detectors \cite{WEB1960, AST1993, AST1997, MAU1996, VIN2006}, however, it was shown in \cite{BERLIN2023} that higher sensitivities are reached, in particular for GWs above $10\,\text{kHz}$. For more details, see e.g. \cite{TOBAR1995, BLAIR1995, TOBAR2000} and references therein. \\
For the coupling of a GW to the EM field of the cavity, there are two possible channels. One is a direct coupling via the Gertsenshtein effect \cite{GERTS1962, ZEL1973}, the other is an indirect mechanical coupling where the GW leads to a deformation of the cavity boundaries inducing an overlap between the initial eigenmodes. In previous studies, the direct coupling was neglected since it is much weaker than the mechanical coupling at low frequencies. However, it becomes dominant at high frequencies above $1\,\text{GHz}$, which was already investigated in detail in \cite{BERLIN2021} for static B-field setups used in Axion experiments \cite{KHAT2021, ZHONG2018}. Although our main interest focuses on the lower frequency regime between $1\,\text{kHz}-10\,\text{MHz}$, we add it for a complete picture of the coupling phenomena. For simplicity, we apply the long wavelength approximation which allows to describe the coupling by two distinct coupling constants $\cgembfield_{01}$ and $\cgemefield_{01}$ where 0 and 1 refer to the pump and signal mode respectively. The mechanical coupling via the l-th mechanical eigenmode consists of a mechanical-EM part with constant $\cmem$ and a GW-mechanical part. In the monochromatic case, the latter decomposes into two constants $\cgmplus$ and $\cgmtimes$. They correspond to the two possible polarisations of a GW. A sketch explaining the principle of the heterodyne approach is shown in fig. \ref{fig:GWCouplings}.\\
Although the theoretical details were already worked out by the MAGO collaboration \cite{ BALLA2005, BERN2002} as well as in recent studies \cite{BERLIN2023}, one goal of our research is to provide a concise and full description of the experiment from first principles. This allows us to point out an important difference between the old and new publications. In \cite{BERLIN2023}, the field back-action of the EM-modes to the mechanical modes was not taken into account. This effect is commonly known as Lorentz force detuning and can be made plausible because the EM field counteracts the external changes induced by the GW, which is comparable to Lenz's law. In particular, it leads to a signal damping which depends on $\cmem$ and becomes dominant in the sub-MHZ regime and close to the resonance $\omega_l=\omega_g$. Thus, optimising the coupling to $|\cmem|\sim 1$ as it was suggested in \cite{BERLIN2023} does in general not provide the strongest signal in the cavity. 
% Figure environment removed
\newline
The paper is organised as follows: In chapter \ref{sec:GertsenshteinEffect}, we introduce our implementation of the Gertsenshtein effect using the long-wavelength approximation. In chapter \ref{sec:WallDeformation} and \ref{sec:TidalForceDensity}, we introduce concepts from elasticity theory to describe the mechanical coupling via wall deformation, i.e. the cavity boundaries. The change of the EM-modes due to the change of boundaries can be described with cavity perturbation theory, which is introduced in chapter \ref{sec:PerturbationTheory}. With these ingredients, we can derive the equations of motion in chapter \ref{sec:EquationsOfMotion} which are solved for a monochromatic GW in z-direction in chapter \ref{sec:SolutionsForMonoGW}. Finally, we provide a detailed analysis of the damping term in chapter \ref{sec:DampingTerm}. This study has grown out of a Master's thesis \cite{LOEW2023}, where more details can be found.

\section{The Gertsenshtein-Effect}
\label{sec:GertsenshteinEffect}
Gravitational waves (GW) are usually described in the framework of linearised theory of general relativity \cite{FORTINI1990}, \cite{HA2003}, \cite{MTW1973}. In this regime, the metric decomposes into a minkowskian part and a small strain tensor $h_{\mu\nu}$, i.e. $g_{\mu\nu}=\eta_{\mu\nu}+h_{\mu\nu}$. In particular, we assume that $\abs{h_{\mu\nu}}\ll 1$ and $\abs{\partial_\alpha h_{\mu\nu}}\ll 1$. Throughout this study, we use the convention $(-,+,+,+)$ for $\eta_{\mu\nu}$. \\
The coupling between a GW and the electromagnetic field is governed by the Einstein-Maxwell action given by \cite{BERLIN2021}
\begin{equation*}
    S_{\text{EM}}=\int\dd^4x\sqrt{-g}\big( -\frac{1}{4}g^{\mu\alpha}g^{\nu\beta}F_{\mu\nu}F_{\alpha\beta} - g^{\mu\nu}j_\mu A_\nu\big).
\end{equation*}
In vacuum, where $j_\mu=0$, this equation leads to a Lagrangian of the form
\begin{equation}
    \label{eqn:GertsenshteinLagrangian}
    \mathcal{L}=-\frac{1}{4}F_{\mu\nu}F^{\mu\nu}-j^{\mu}_{\text{eff}}A_\mu,
\end{equation}
where the effective current $j_{\text{eff}}$ is induced by the strain $h_{\mu\nu}$. Considering the explicit form \cite{BERLIN2021}
\begin{equation}
    \label{eqn:GertsenshteinCurrent}
    j^{\mu}_{\text{eff}}=\partial_\nu\big( \frac{{h^\alpha}_\alpha}{2}F^{\mu\nu}+{h^\nu}_\alpha F^{\alpha\mu}-{h^\mu}_\alpha F^{\alpha\nu}\big)
\end{equation}
of this current, we see that it does not transform covariantly like a four-vector. Hence, it is not invariant under coordinate transformations and we must therefore carefully think about the reference frame when evaluating the strain. It turns out \cite{BERLIN2021, RAKH2014} that the best choice is given by the proper detector frame. It is an example for a local Lorentz frame and therefore encodes the physical change of the detector due to a passing GW. Since we focus on GWs in the kHz-MHz regime, we apply the long-wavelength approximation which allows us to use the relatively simple metric
\begin{equation}
    \label{eqn:PDFMetric}
    \dd s^2=-\dd t^2\big( 1-\frac{1}{2}\ddot{h}^{\text{TT}}_{ij}(g)x^i x^j\big)+\dd x^i\dd x^j\delta_{ij},
\end{equation}
where $h^{TT}_{ij}$ is the strain in the Transverse Traceless (TT) and $g$ denotes the reference geodesic, i.e. the worldline of the detector. More details can be found in \cite{RAKH2014, CAP2020, MARZ1994, MAG2007} and in Appendix \ref{app:PDF}. \\
For simplicity, we will often refer to a monochromatic GW with frequency $\omega_g$ travelling in z-direction. In TT gauge and using complex notation, it can be written as 
\begin{equation}
    \label{eqn:StrainMonoGW}
    h^{TT}_{ij}(t)=\begin{pmatrix} h_+ & h_\times & 0 \\ h_\times & -h_+ & 0 \\ 0 & 0 & 0 \end{pmatrix}e^{i\omega_g t},
\end{equation}
where the spatial dependence are neglected in the long wavelength approximation. From eqn. \ref{eqn:PDFMetric} we can obtain the only non-vanishing component $h_{00}$ of the strain in the proper detector frame. It yields
\begin{equation}
    \label{eqn:MonochromaticGWZDirection}
    h_{00}(t,\vb{x})=-\frac{\omega^2_g}{2}(h_+(x^2-y^2)+2xyh_{\times})e^{i\omega_g t}=:-H_0(\vb{x})e^{i\omega_g t},
\end{equation}
where we have defined the function $H_0(\vb{x}):=\omega^2_g/2(h_+(x^2-y^2)+2xyh_{\times})$.\\
Finally, we can derive the equations of motion from eqn. \ref{eqn:GertsenshteinLagrangian} and \ref{eqn:GertsenshteinCurrent}. The resulting modified Maxwell equations have the form
\begin{equation}
    \label{eqn:MaxwellEinsteinEquations}
    \begin{array}{rl} \nabla\cdot\vb{E}&=\rho_{\text{eff}}, \\
    \nabla\times\vb{B}-\partial_t\vb{E}&=\vb{j}_{\text{eff}}, \end{array}
\end{equation}
with the effective charge density $\rho_{\text{eff}}$ and current density $\vb{j}_{\text{eff}}$. In the long wavelength regime, they yield
\begin{align}
    \label{eqn:EffCharge}
    \rho_{\text{eff}}&:=\frac{1}{2}\nabla(h_{00}), \\
    \label{eqn:EffCurrent}
    \vb{j}_{\text{eff}}&:=-\frac{1}{2}\partial_t(h_{00}\vb{E}_0)-\frac{1}{2}\nabla\times(h_{00}\vb{B}_0),
\end{align}
where $\vb{E}_0$ and $\vb{B}_0$ are the electromagnetic fields of the pump mode.

\section{Wall Deformation}
\label{sec:WallDeformation}
Since GWs change the spacetime metric, the cavity boundaries get modified. A GW couples to the mechanical modes of the resonator which shifts the EM eigenmodes inside the cavity. In order to describe this effect properly, we need to apply the framework of classical elasticity theory \cite{LALIF1987, THBLAN2021}. A concise formalism was already derived in \cite{BERN2002, LOBO1995} and we only give a short review here. \\
The starting point is the equation of motion for an isotropic elastic solid under the influence of an external force density $\vb{f}(t,\vb{x})$, i.e. 
\begin{equation}
    \label{eqn:EOMDeformationField}
     \rho(\vb{x})\seconddevp{\vb{u}(t,\vb{x})}{t} - (\lambda+\mu)\nabla(\nabla\vb{u}(t,\vb{x}))-\mu\nabla^2\vb{u}(t,\vb{x}) = \vb{f}(t,\vb{x}).
\end{equation}
Here, $\rho(\vb{x})$ denotes the material density and $\lambda$ and $\mu$ are the materials first and second Lam\'{e} parameters \cite{LOBO1995}. For the initial conditions $\vb{u}(\vb{x},0)=0=\firstdevp{\vb{u}}{t}(\vb{x},0)$, we can use the ansatz
\begin{equation}
    \label{eqn:Decomposition}
    \vb{u}(\vb{x},t)=\sum_{l=1}^{\infty}\vb{\xi}_l(\vb{x})q_l(t).
\end{equation}
This leads to a set of equations of motion for the $q_l(t)$ of the form 
\begin{equation}
    \label{eqn:IdealizedVibration}
    \ddot{q}_l(t)+\omega_l^2 q_k(t)=\frac{f_l(t)}{M},
\end{equation}
where M is the cavity mass and $f_l(t)$ the generalised force density. It is defined via the integral
\begin{equation}
    \label{eqn:GeneralizedForceDensity}
    f_l(t):=\intcav\dd^3x\vb{f}(t,\vb{x})\vb{\xi}_l(\vb{x})
\end{equation}
over the cavity volume $V_{\text{cav}}$. Finally, we note that the spatial modes $\vb{\xi}_l(\vb{x})$ are the eigensolutions of the equation
\begin{equation*}
    \label{eqn:VibrationEigenequation}
    \omega_l^2\rho(\vb{x})\vb{\xi}_l(\vb{x})+(\lambda+\mu)\nabla(\nabla\vb{\xi}_l(\vb{x})) +\mu\nabla^2\vb{\xi}_l(\vb{x})=0
\end{equation*}
and are therefore independent of $\vb{f}(t,\vb{x})$. They are normalised as
\begin{equation*}
    \intcav\dd^3x\vb{\xi}_k(\vb{x})\vb{\xi}_l(\vb{x})\rho(\vb{x})=M\delta_{kl}.
\end{equation*}
In general, those modes have to be determined numerically. An analytic solution for a spherical geometry can be found in \cite{LOBO1995}. 

\section{Tidal Force Density for Monochromatic Gravitational Waves}
\label{sec:TidalForceDensity}
Here, we want to state a simple example of the tidal force density $\vb{f}(t,\vb{x})$ induced by a passing GW. It can be derived from the equation of geodesic deviation \cite{MTW1973} and yields \cite{BERN2002, LOBO1995}
\begin{equation*}
    \vb{f}(t,\vb{x}) = -\rho(\vb{x}) R_{0i0j}(t)x_j\vb{e}_i,
\end{equation*}
where $\rho(\vb{x})$ is again the material density and $R_{0i0j}$ the Riemann curvature tensor. In case of a monochromatic GW in z-direction (eqn. \ref{eqn:StrainMonoGW}), we can further evaluate this expression and plug it into the generalised force density in eqn. \ref{eqn:GeneralizedForceDensity}. We get
\begin{equation}
    \label{eqn:GeneralizedForceDensiteZ}
    f_l(t)=-\frac{1}{2}\omega_g^2M V^{1/3}_{\text{cav}}\big(h_{+}\cgmplus+h_{\times}\cgmtimes\big)e^{i\omega_g t}=:F_l(t)e^{i\omega_g t}.
\end{equation}
The dimensionless coupling coefficients $\cgmplus$ and $\cgmtimes$ encode the coupling strength between the GW and the mechanical mode $l$. They are defined by
\begin{align*}
    \cgmplus&:=\frac{V^{-1/3}_{\text{cav}}}{M}\intcav\dd^3x\rho(\vb{x})\big( x\xi_{l,x}(\vb{x})-y\xi_{l,y}(\vb{x}) \big), \\
    \cgmtimes&:=\frac{V^{-1/3}_{\text{cav}}}{M}\intcav\dd^3x\rho(\vb{x})\big( x\xi_{l,y}(\vb{x})+y\xi_{l,x}(\vb{x}) \big).
\end{align*}

\section{Mode Decomposition and Cavity Perturbation Theory}
\label{sec:PerturbationTheory}
The electromagnetic field in an evacuated cavity is, without any perturbation, given by the wave equations
\begin{equation}
    \label{eqn:Eigensolutions}
    \Delta\vb{E}=\frac{1}{c^2}\seconddevp{\vb{E}}{t} \qquad \qquad \Delta\vb{B}=\frac{1}{c^2}\seconddevp{\vb{B}}{t}.
\end{equation}
with the boundary conditions \cite{JACK2006}
\begin{equation}
    \label{eqn:BoundaryConditions}
    \vb{n}\times\vb{E}|_S = 0, \qquad \vb{n}\cdot\vb{B}|_S=0,
\end{equation}
where $\vb{n}$ is the normal vector of the cavity shell $S$. The eigensolutions of this boundary value problem can be separated into a dimensionless time-dependent part $e_n(t)$, $b_n(t)$ and a spatial part $\vb{E}_n(\vb{x})$, $\vb{B}_n(\vb{x})$. The general solution can then be decomposed as
\begin{equation}
    \label{eqn:ModeExpansions}
    \vb{E}(t,\vb{x}) = \sum_n e_n(t)\vb{E}_n(\vb{x}), \qquad \qquad \vb{B}(t,\vb{x}) = \sum_n b_n(t)\vb{B}_n(\vb{x}).
\end{equation}
The normalisation of the spatial modes is chosen accordingly to \cite{BERLIN2019},
\begin{equation}
    \label{eqn:Normalization}
    \intcav\dd^3x\varepsilon_0\vb{E}_n\vb{E}_m = 2U_n\delta_{nm} = \intcav\dd^3x\frac{1}{\mu_0}\vb{B}_n\vb{B}_m,
\end{equation}
where $U_n$ is the average energy in mode $n$. Correspondingly, the time-dependent modes are given by
\begin{equation}
    \label{eqn:TimeExpansion}
    e_n(t)=\frac{\varepsilon_0}{2U_n}\intcav\dd^3x\vb{E}(t,\vb{x})\vb{E}_n(\vb{x}), \qquad \qquad b_n(t)=\frac{1}{2\mu_0 U_n}\intcav\dd^3x\vb{B}(t,\vb{x})\vb{B}_n(\vb{x}).
\end{equation}
When a GW is passing through the cavity, the geometry and therefore the boundary conditions for the electromagnetic field change. In particular, that means we have to change the set of boundary conditions. However, since the GW strain is very small ($\lesssim\mathcal{O}(10^{-21})$), cavity perturbation theory (CPT) can be applied. That means, the perturbed modes can be expanded in terms of the unperturbed modes. \\
It should be noted that there are some pitfalls when applying CPT to the spatial modes. The main problem is that such a series expansion would generally not fulfill the boundary conditions
\begin{equation*}
    \vb{n}\times\vb{E}(\vb{x})|_{S'} = 0 \qquad \vb{n}\cdot\vb{B}(\vb{x})|_{S'}=0.
\end{equation*}
Hence, in order to obtain a consistent theory, a perturbation theory is used only for time dependent modes. Applying CPT to the spatial modes in eqn. \ref{eqn:ModeExpansions} leads to wrong signs in the final result \cite{BERN2002}.
The CPT method we use has been derived in \cite{GOUBAU1961}. Further details can be found in Appendix \ref{app:CavityPerturbationTheory}. Note that we consider GWs with frequencies much smaller than the mode frequencies, so we can treat the shell displacement in adiabatic approximation. Applying the CPT formalism for the time-dependent modes $e_n(t)$ and $b_n(t)$ leads to
\begin{align}
    \label{eqn:EFieldExpansion}
    e'_n(t)&=e_n(t)+\sum_{m\neq n}\frac{U_m}{U_n}\alpha_{nm}e_m, \\
    \label{eqn:BFieldExpansion}
    b'_n(t)&=b_n(t)-\frac{1}{2}\mathcal{C}_{nn}b_n+\sum_{m\neq n}\frac{U_m}{U_n}\beta_{nm}b_m, \\
    \label{eqn:FrequencyExpansion}
    \omega'_n&=\omega_n-\frac{1}{2}\omega_n \mathcal{C}_{nn}.
\end{align}
The expansion coefficients are given by
\begin{equation*}
    \alpha_{nm}=\frac{\omega_n\omega_m}{\omega_m^2-\omega_n^2}\frac{U_m}{U_n}\mathcal{C}_{nm}, \qquad \qquad \beta_{nm}=\frac{\omega_n^2}{\omega_m^2-\omega_n^2}\frac{U_m}{U_n}\mathcal{C}_{nm},
\end{equation*}
where $\mathcal{C}_{nm}$ encodes all geometric and electromagnetic properties of the cavity. In a heterodyne setup, we can decompose this factor as
\begin{equation}
    \label{eqn:PureConnectionCoefficient}
    \mathcal{C}_{nm}=V^{-1/3}_{\text{cav}}\sqrt{\frac{U_n}{U_m}}\sum_l q_l(t)\eta^l_{nm},
\end{equation}
where the symmetric dimensionless coupling coefficient $\eta^l_{nm}$ can shown to be \cite{BERN2002}
\begin{equation}
    \label{eqn:ConnectionCoefficient}
    \eta^l_{nm}=\frac{V^{1/3}_{\text{cav}}}{2\sqrt{U_n U_m}}\intcavbound\dd\vb{S}\vb{\xi}_l(\vb{x})\Big[ \frac{1}{\mu_0}\vb{B}_n\vb{B}_m - \epsilon_0\vb{E}_n\vb{E}_m\Big].
\end{equation}

\section{The Equations of Motion}
\label{sec:EquationsOfMotion}
In order to derive the equations of motion (EoM), we use a similar formalism as in \cite{BERN2002}, however adding the direct coupling to the EM field via the Gertsenshtein effect. Although this coupling is subdominant at low frequencies, it becomes important at frequencies beyond $1\,\text{GHz}$. Since it could be necessary to extend the experimental search into this regime in the future, we include it already in this study. Our starting point is the extended Lagrangian 
\begin{equation}
    \label{eqn:FullLagrangian}
    L=\intcav\dd V\Big[ -\frac{1}{4}F'_{\mu\nu}F'^{\mu\nu}-\frac{1}{2}j^\mu_{\text{eff}}A'_\mu \Big]+\sum_l\big(\frac{1}{2}M\dot{q}_l^2(t)-\frac{1}{2}M\omega_lq^2_l(t)+q_l(t)f_l(t) \big),
\end{equation}
where the prime denotes the perturbed fields and $j^\mu_{\text{eff}}$ is given in eqn. \ref{eqn:EffCharge}-\ref{eqn:EffCurrent}. \\
We can now split the Lagrangian into two parts, $L= L_{\text{em}}+L_{\text{mech}}$, where $L_{\text{em}}$ describes dynamics of the EM-field and is given by
\begin{equation}
    \label{eqn:EMLagrangian}
    L_{\text{em}}=\intcav\dd V\Big[-\frac{1}{4} F'_{\mu\nu}F'^{\mu\nu}-\frac{1}{2}j_{\text{eff}}^{\mu}A'_{\mu}\Big].
\end{equation}
The Lagrangian $L_{\text{mech}}$ governs the physics of the mechanical displacement field and yields
\begin{equation}
    \label{eqn:MechLagrangian}
    L_{\text{mech}}=\sum_n 2U_n\big(e'^2_n(t)-b'^2_n(t)\big)+\sum_l\Big( \frac{1}{2}M\dot{q}_l^2(t)-\frac{1}{2}M\omega_l q_l^2(t)+q_l(t)f_l(t) \Big).
\end{equation}
Note that both $j^\mu_{\text{eff}}$ and the corrections to $A'_{\mu}$ are of order $\mathcal{O}(h)$, so we can drop the prime of the vector field in leading order and neglect the term here. With the techniques described in chapter \ref{sec:GertsenshteinEffect} to \ref{sec:PerturbationTheory}, it is straightforward to derive the equations of motion. For simplicity, we will assume that only one mechanical mode $l$ contributes to the dynamics throughout this study. Then, adding dissipative terms to account for the energy losses through the walls and by the external oscillator driving the pump mode, we find from eqn. \ref{eqn:EMLagrangian} that
\begin{align}
    \label{eqn:EOMPump}
    \ddot{b}_0+\frac{\omega_0}{Q_0}\dot{b}_0+\omega^2_0 b_0 &= \omega^2_0 V^{-1/3}_{\text{cav}}q_l\Big(\eta^l_{00}b_0+\sqrt{\frac{U_1}{U_0}}\cmem b_1 \Big) + J_0 + \frac{\omega_0}{Q_0}\sqrt{\frac{U_d}{U_0}}\dot{b}_{d}, \\
    \label{eqn:EOMSignal}
    \ddot{b}_1+\frac{\omega_1}{Q_1}\dot{b}_1+\omega^2_1 b_1 &= \omega^2_1 V^{-1/3}_{\text{cav}}q_l\Big( \eta^l_{11}b_1+\sqrt{\frac{U_0}{U_1}}\cmem b_0 \Big) 
    +J_1+\epsilon\frac{\omega_1}{Q_1}\sqrt{\frac{U_d}{U_1}}\dot{b}_d,
\end{align}
where $n=0$ refers to the pump mode and $n=1$ to the signal mode. Here, $Q_0$ and $Q_1$ are the quality factors \cite{JACK2006} of the eigenmodes and $b_d$ denotes the oscillator which is also coupled to the signal mode with a constant\footnote{For the MAGO cavity, this coupling could be reduced to $\epsilon\sim10^{-7}$.} $\epsilon$. The Gertsenshtein current shows up as a projected current $J_n$ which can be expressed as
\begin{equation}
    \label{eqn:MonoProjectedCurrent}
    J_n(\omega):= H\omega_g^2\sqrt{\frac{U_0}{U_n}}\big( \kappa_n\cgemefield_{0n} + \lambda_n\cgembfield_{0n} \big)2\pi\delta(\omega-(\omega_0+\omega_g))
\end{equation}
under the assumption that $\vb{n}\times\vb{j}_{\text{eff}}(t,\vb{x})\vert_S=0$. Here we have introduced another two coupling coefficients
\begin{align}
    \eta^{\text{E}}_{0n}&:=\frac{1}{H\sqrt{U_0 U_n}}\intcav\dd^3x H_0(\vb{x})\varepsilon_0\vb{E}_0(\vb{x})\vb{E}_n(\vb{x}) \\
    \eta^{\text{B}}_{0n}&:=\frac{1}{H\sqrt{U_0 U_n}}\intcav\dd^3x H_0(\vb{x})\frac{1}{\mu_0}\vb{B}_0(\vb{x})\vb{B}_n(\vb{x}).
\end{align}
for the E-field and the  B-field respectively. The parameters $\kappa_n$ and $\lambda_n$ are given by
\begin{align}
    \label{eqn:Kappa}
    \kappa_n &:= i\frac{\omega_n}{8c^2}(\omega_0+\omega_g), \\
    \label{eqn:Lambda}
    \lambda_n &:= \frac{\omega_n^2}{8c^2}.
\end{align}
Note that they are different in the long wavelength regime. If $\omega_0+\omega_g<\omega_n$, the GW couples stronger to the B-field than to the E-field, and vice versa for $\omega_0+\omega_g>\omega_n$. Since we work with $\omega_0\approx\omega_1$ and $\omega_g\ll\omega_0,\omega_1$ throughout this study, we assume $|\kappa_n|\approx|\lambda_n|$, i.e. that the couplings are equally strong. If $\omega_g$ becomes comparable to $\omega_0$ and $\omega_1$, the long wavelength approximation may break down and eqn. \ref{eqn:MonoProjectedCurrent} is no longer valid. Finally, we have defined the normalised GW strain as
\begin{equation}
    \label{eqn:StrainNorm}
    H:=\sqrt{\frac{1}{V_{\text{cav}}}\intcav\dd^3x H^2_0(\vb{x})}.
\end{equation}
For a MAGO-like cavity, we found that a reasonable value for the normalised strain is given by $H\sim h_0\times 10\,\text{m}^2$, where $h_0$ is the characteristic strain strength of the GW. \\
Similarly, we can find the EoM for the mechanical modes from eqn. \ref{eqn:MechLagrangian}. Adding again a dissipative term, we end up with
\begin{equation}
    \label{eqn:EOMQl}
    \ddot{q}_l+\frac{\omega_l}{Q_l}\dot{q}_l+\omega_l^2 q_l = \frac{1}{M}\Big(f_l+f_l^{\text{ba}}\Big),
\end{equation}
where $Q_l$ is the mechanical quality factor. In contrast to \cite{BERLIN2023}, we automatically obtain the field back-action 
\begin{equation*}
    f_l^{\text{ab}}(t):=V^{-1/3}_{\text{cav}}\Big(U_0 \eta^l_{00}b_0^2(t)+U_1\eta^l_{11}b_1^2(t)+2\sqrt{U_0 U_1}\cmem b_0(t)b_1(t)\Big),
\end{equation*}
which is responsible for an additional deformation of the cavity walls, commonly known as Lorentz Force Detuning. The first two terms lead to a constant shift that can be absorbed into the definition of the eigenmodes. However, the last term leads to a damping of the signal strength which is particularly strong close to the resonances. Effectively, we therefore get
\begin{equation}
    \label{eqn:BackAction}
    f_l^{\text{ab}}(t)=2V^{-1/3}_{\text{cav}}\sqrt{U_0 U_1}\cmem b_0(t)b_1(t).
\end{equation}
Note that this term already appeared in the studies of the MAGO collaboration, see e.g. \cite{BALLA2005, BERN2002}. However, in chapter \ref{sec:DampingTerm}, we will investigate its impact on the detector sensitivity in greater detail.

\section{Solution for Monochromatic Gravitational Waves}
\label{sec:SolutionsForMonoGW}
In general, the coupled differential equations \ref{eqn:EOMPump}, \ref{eqn:EOMSignal} and \ref{eqn:EOMQl} can only be solved by numerical methods. However, if we assume a monochromatic GW travelling in z-direction, i.e.
\begin{align*}
    q_l(t)&=\real{Q_l(t)e^{i\omega_gt}}, \\
    f_l(t)&=\real{F_l(t)e^{i\omega_gt}}, \\
    J_1(t)&=\real{K_1(t)e^{i(\omega_0+\omega_g)t}},
\end{align*}
it is possible to find an analytic expression for the signal power. We further assume that the pump mode can be stabilised to be monochromatic. Therefore, an appropriate ansatz for the fields is
\begin{align}
    \label{eqn:bFieldPump}
    b_0(t)=b_d(t)&=\real{e^{i\omega_0t}}, \\
    \label{eqn:bFieldSignal}
    b_1(t)&=\real{A_1(t)e^{i(\omega_0+\omega_g)t}}.
\end{align}
The remaining calculation is greatly inspired by \cite{BERN2002}. If $Q_l(t)$ and $A_1(t)$ are small such that only leading terms in these functions are relevant, the EoM (eqn. \ref{eqn:EOMSignal} and eqn. \ref{eqn:EOMQl}) can be written as
\begin{align}
    \label{eqn:TimeEquationForA1}
    \ddot{A}_1(t)+\alpha_1\dot{A}_1(t)+\beta_1A_1(t) &=\sqrt{\frac{U_0}{U_1}}\gamma_1Q_l(t)+K_1(t)+\epsilon i\omega_0\frac{\omega_1}{Q_1}\sqrt{\frac{U_d}{U_1}}e^{-i\omega_gt}, \\
    \label{eqn:TimeEquationForQl}
    \ddot{Q}_l(t)+\alpha_l\dot{Q}_l(t)+\beta_l\omega_l^2 Q(t) &= \frac{F_l(t)}{M} + \sqrt{\frac{U_1}{U_0}}\gamma_l A_1(t),
\end{align}
where we also neglected all fast oscillating terms. That means, we assumed that terms containing $e^{i\omega_0 t}$ or $e^{i\omega_1 t}$ vanish in the time average compared to terms containing $e^{i\omega_g t}$. The constants introduced in eqn. \ref{eqn:TimeEquationForA1} and \ref{eqn:TimeEquationForQl} are given by
\begin{align*}
    \alpha_l&:=2i\omega_g+\frac{\omega_l}{Q_l}, \\
    \beta_l&:=\omega_l^2-\omega_g^2+i\omega_g\frac{\omega_l}{Q_l}, \\
    \gamma_l&=\frac{1}{M}\cavvol^{-1/3}U_0\cmem, \\
    \alpha_1&:=\frac{\omega_1}{Q_1}+2i(\omega_0+\omega_g), \\
    \beta_1&:=\omega_1^2-(\omega_0+\omega_g)^2+i\frac{\omega_1}{Q_1}(\omega_0+\omega_g), \\
    \gamma_1&:=\cavvol^{-1/3}\omega^2_1\cmem.
\end{align*}
We can now perform Fourier transformations to solve eqn. \ref{eqn:TimeEquationForA1} and \ref{eqn:TimeEquationForQl} for $A_1(t)$. By again neglecting fast oscillating terms, the time average of $b_1(t)$, eqn. \ref{eqn:bFieldSignal}, can be expressed as $\av{b^2_1(t)}=\frac{1}{2}\av{|A_1(t)|^2}$. Thus, we can write the signal in terms of a power spectral density (PSD) 
\begin{equation*}
    S_{\text{sig}}(\omega):=2U_1\omega_1/Q_{\text{cpl}} S_{b_1}(\omega),
\end{equation*}
where $\av{b^2_1(t)}=(2\pi)^{-2}\int\dd\omega S_{b_1}(\omega)$. Note the coupling quality factor $Q_{\text{cpl}}$ has to be used here, since it parameterises the energy transfer into the readout system. It is related to the full quality factor $Q_1$ via
\begin{equation*}
    \frac{1}{Q_1}=\frac{1}{Q_{\text{cpl}}}+\frac{1}{Q_{\text{int}}},
\end{equation*}
where $Q_{\text{int}}$ is the internal quality factor neglecting the readout loss. More details on that can be found in \cite{BERLIN2023, BERLIN2020}. \\
The final result for the signal PSD is then given by 
\begin{equation}
    \label{eqn:SignalPSD}
    S_{\text{sig}}(\omega)=\frac{\omega_1}{Q_{\text{cpl}}}\omega_g^4U_0\Bigg| \underbrace{\frac{1}{2}\frac{\omega_1^2\cmem(h_+\cgmplus + h_{\times}\cgmtimes)}{\Lambda_{1}(\omega-(\omega_0+\omega_g))}}_{\text{Mechanical Coupling}}-\underbrace{\frac{H(\kappa_1\cgemefield_{01}+\lambda_1\cgembfield_{01})}{\Lambda_2(\omega-(\omega_0+\omega_g))}}_{\text{Gertsenshtein Coupling}}\Bigg|^24\pi^2\delta(\omega-(\omega_0+\omega_g)).
\end{equation}
where we have introduced two functions $\Lambda_1(\omega)$ and $\Lambda_2(\omega)$ in the denominators which we call \textit{resonance functions}. They are given by
\begin{align}
    \label{eqn:Lambda1}
    \Lambda_1(\omega)&:=\big( \beta_1-\omega^2+i\omega\alpha_1 \big)\big( \beta_l-\omega^2+i\omega\alpha_l \big)-\gamma_1\gamma_l, \\
    \label{eqn:Lambda2}
    \Lambda_2(\omega)&:=\Lambda_1(\omega)\big( \beta_l-\omega^2+i\omega\alpha_l \big)^{-1}.
\end{align}
For completeness, we note that an additional term appears from eqn. \ref{eqn:TimeEquationForA1}, describing the coupling to the external oscillator. It yields
\begin{equation}
    \label{eqn:OscillatorPSD}
    S_{\text{osc}}(\omega)=\epsilon^2 \frac{Q_1}{Q_{\text{cpl}}}\frac{\omega^3_1}{Q^3_1}\omega_0^2\frac{U_d S_{b_d}(\omega)}{|\Lambda_2(\omega-(\omega_0+\omega_g))|^2},
\end{equation}
where we have defined $S_{b_d}(\omega):=4\pi^2\delta(\omega-\omega_0)$. For a monochromatic oscillator, this PSD can be well separated from the signal. However, there is some irreducible phase noise which leads to a power leakage into frequency range of the signal mode. It turns out that, for $\epsilon\sim10^{-7}$, the oscillator phase noise is negligible compared to more dominant sources such as mechanical or thermal noise. For a detailed discussion, see e.g. \cite{BERLIN2023, BERLIN2019}. \\
Finally, we can integrate eqn. \ref{eqn:SignalPSD} to obtain the total signal power. It yields
\begin{equation}
    \label{eqn:SignalPower}
    P_{\text{sig}}=\frac{1}{(2\pi)^2}\int\dd\omega S_{\text{sig}}(\omega) = \frac{\omega_1}{Q_{\text{cpl}}}\omega_g^4U_0\Bigg| \frac{1}{2}\frac{\omega_1^2\cmem(h_+\cgmplus + h_{\times}\cgmtimes)}{\beta_1\beta_l-\gamma_1\gamma_l}-\frac{\beta_lH(\kappa_1\cgemefield_{01}+\lambda_1\cgembfield_{01})}{\beta_1\beta_l-\gamma_1\gamma_l} \Bigg|^2.
\end{equation}
This is the main result of our study. In contrast to the results of \cite{BERLIN2023}, it shows an additional damping factor
\begin{equation}
    \label{eqn:DampingTerm}
    \gamma_1\gamma_l=\frac{1}{M}\cavvol^{-2/3}U_0(\omega_1\cmem)^2
\end{equation}
in the denominator. In the following, we will investigate its impact on the detector sensitivity.

\section{Impact of the Damping Term}
\label{sec:DampingTerm}
% Figure environment removed
In this section we want to investigate the effects of the damping term $\gamma_1\gamma_l$ which follows from the back-action $f^{\text{ab}}_l(t)$, eqn. \ref{eqn:BackAction}. The main reason is that recent studies, such as \cite{BERLIN2023}, do not consider this term, although it has an important influence on the results, as explained in the following. The MAGO collaboration, however, mentioned it, but without investigating it in greater detail. \\
We found that the term has a great influence on the signal since it depends quadratically on the mechanical coupling. Therefore, we propose that a mechanical coupling of $|\cmem|=1$, cf. eqn. \ref{eqn:ConnectionCoefficient}, does not always lead to the strongest signal. To show this, we calculate eqn. \ref{eqn:SignalPower} explicitly and take $\omega_l$, $\omega_g$ and $\cmem$ as free parameters. In order to fix the remaining values, we mostly follow \cite{BERLIN2023, BALLA2005}. That means, concerning the cavity parameters, we choose $M=5\,\text{kg}$, $\cavvol=10\,\text{L}$, $\omega_0=1.8\,\text{GHz}$ and $Q_0=Q_{\text{cpl}}=Q_{\text{int}}=10^{10}$. The electromagnetic field in the cavity and the temperature of the boundaries should not exceed the quenching limit of niobium, which was used for MAGO. According to \cite{BERLIN2023}, we therefore assume a typical E-field of $30\,\text{MV/m}$ which corresponds to a total pump mode energy of $U_0\sim 40\,\text{J}$. For the GW, we assume a typical strain of $h_0=h_+=h_{\times}=10^{-20}$ and a GW-mechanical coupling of $\cgmplus=\cgmtimes=1$, which are both rather optimistic\footnote{For MAGO-like cavities, we found values of $\cgmplus,\cgmtimes\sim\mathcal{O}(10^{-2})$. Note, however, that our qualitative results do not depend on these values.}. The calculations are conducted for a scanning experiment where $\omega_1=\omega_0+\omega_g$. Broadband detection is in principle possible as well \cite{BERLIN2023, BERLIN2020}, but typically leads to a low sensitivity far from the resonance. \\
In the first analysis, we assume a lowest mechanical quadrupole mode at $\omega_l=5\,\text{kHz}$ which is in good agreement with numerical simulations of MAGO-like cavity spectra. We then investigated how the signal power depends on $|\cmem|$ for eight different frequencies $\omega_g$ between $100\,\text{Hz}$ and $10\,\text{MHz}$. For larger frequencies, the long wavelength approximation may break down and eqn. \ref{eqn:SignalPower} has to be adjusted. The results are shown in fig. \ref{fig:SignalCouplingRelation} both with and without the damping term. When the term is neglected (fig. \ref{fig:SignalCouplingRelation}, left panel), it is obvious that the highest signal is achieved for the maximum coupling constant $|\cmem|=1$. However, in case it is included (fig. \ref{fig:SignalCouplingRelation}, right panel), $|\cmem|=1$ leads to the strongest signal only for high frequencies in the MHz-regime. Below, we find that the best coupling is achieved for $|\cmem|<1$. In particular close to the resonance $\omega_g=\omega_l$, the damping term leads to an ideal coupling of $|\cmem|\sim\mathcal{O}(10^{-6})$. The results therefore clearly show that SRF cavity experiments should in general not be optimised to $|\cmem|\sim\mathcal{O}(1)$. This is particularly important for low frequencies and frequencies close to the resonance. \\
We also provide a more general analysis of the best choice for $|\cmem|$ in fig. \ref{fig:OptimisationCurve}. It shows the value of $|\cmem|$ with the largest signal powers in the $\omega_g$-$\omega_l$-plane. The result could be used as a template for optimising future gravitational wave experiments. We point out that the parameters $\omega_l$, $\omega_g$ and $\cmem$ can be controlled via the cavity geometry.
% Figure environment removed
Finally, we note that the optimal coupling does also depend on the noise sources which determine the cavity sensitivity. Applying a qualitative analysis of the sources discussed in \cite{BERLIN2023} using the damping term showed that the coupling still has to be optimised similar to fig. \ref{fig:OptimisationCurve}. However, the results largely depend on the noise parameters and we postpone a more detailed study to future work.

\section{Conclusion}
\label{sec:Conclusions}
Heterodyne cavity experiments provide a promising tool for detecting or excluding new sources of GWs in the future. Recent studies showed that the sensitivity of modern cavities can already approach the regime for new physics. A promising candidate in the measurable regime is, for instance, black hole superradiance. In this study, we refined the theoretical formalism proposed by the former MAGO collaboration \cite{BALLA2005, BERN2002} and added another signal source from the Gertsenshtein effect. The resulting signal power for a monochromatic GW in z-direction is shown in eqn. \ref{eqn:SignalPower}. We note that the Gertsenshtein effect is subdominant in the considered kHz-MHz-regime. At higher frequencies above MHz, however, the effect becomes dominant and the coupling has to be taken into account. In that case, the long wavelength approximation breaks down and the full metric expansion should be used (see appendix \ref{app:PDF}). First studies can be found in \cite{BERLIN2023, BERLIN2021}. \\
An important difference to the recent results \cite{BERLIN2023} is that we included the back-action of the EM field, leading to Lorentz Force Detuning and causing a damping term $\gamma_1\gamma_l$, eqn. \ref{eqn:DampingTerm}, which depends on the coupling $\cmem$ of the EM field to the mechanical cavity modes. This term was already described in \cite{BALLA2005, BERN2002}, but its influence was not investigated further. We found that an important consequence is that choosing $|\cmem|\sim\mathcal{O}(1)$ does in general not lead to the strongest signal power. In particular close to the resonance $\omega_g=\omega_l$, the maximum signal is achieved for much lower couplings of order $|\cmem|\sim\mathcal{O}(10^{-6})$. \\
Altogether, we recommend that future heterodyne cavity experiments do not choose a design where always $|\cmem|\sim 1$. Instead, the coupling constant should be adjusted such that it matches the optimal coupling for the values $\omega_l$ and $\omega_g$ of the experiment.


\section{Acknowledgements}
We gratefully acknowledge helpful discussions with Marc Wenskat, Krisztian Peters, Andreas Ringwald, Lars Fischer, Michel Paulsen, Sebastian Ellis, Raffaele Tito D’Agnolo and Jan Schütte-Engel. GMP is supported by the Deutsche Forschungsgemeinschaft (DFG, German Research Foundation)
under Germany’s Excellence Strategy EXC 2121 ``Quantum Universe'' - 390833306.

