\documentclass[a4paper]{amsart}
\usepackage{graphicx}
\usepackage{amssymb} 
\usepackage{stmaryrd}
\usepackage[mathcal]{euscript}
\usepackage{tikz}
\usepackage{tikz-cd}
\usepackage{hyperref}
\hypersetup{%
  bookmarksnumbered=true,%
  colorlinks=true,%
  linkcolor=blue,%
  citecolor=blue,%
  filecolor=blue,%
  menucolor=blue,%
  urlcolor=blue,%
  bookmarksopen=true,%
  bookmarksdepth=2,%
  pageanchor=true}


\title{The finitistic dimension of a triangulated category}

\author{Henning Krause}
\address{Fakult\"at f\"ur Mathematik\\
Universit\"at Bielefeld\\ D-33501 Bielefeld\\ Germany}
\email{hkrause@math.uni-bielefeld.de}

%%%%%%%%%%%%% theorem styles

\theoremstyle{plain}
\newtheorem{thm}{Theorem}%[section]
\newtheorem{prop}[thm]{Proposition}
\newtheorem{lem}[thm]{Lemma} 
\newtheorem{conj}[thm]{Conjecture}
\newtheorem*{cor}{Corollary}


\theoremstyle{definition}
\newtheorem{defn}[thm]{Definition}
\newtheorem{exm}[thm]{Example}
\newtheorem*{probl}{Problem}


\theoremstyle{remark}
\newtheorem{rem}[thm]{Remark} 
\newtheorem{notation}[thm]{Notation}

\numberwithin{equation}{thm}

%%%%%%%%%%%%% hyphenation

\hyphenation{Grothen-dieck} 
\hyphenation{Auslan-der} 
\hyphenation{com-mu-ta-tive}
\hyphenation{uni-serial}
\hyphenation{ubi-qui-tous}

%%%%%%%%%%%%% math operators

\newcommand{\Abfree}{\operatorname{Ab}}
\newcommand{\Ac}{\operatorname{Ac}}
\newcommand{\add}{\operatorname{add}}
\newcommand{\Add}{\operatorname{Add}}
\newcommand{\amp}{\operatorname{amp}}
\newcommand{\Ann}{\operatorname{Ann}}
\newcommand{\art}{\operatorname{art}}
\newcommand{\card}{\operatorname{card}}
\newcommand{\Cau}{\operatorname{Cauch}}
\newcommand{\Ch}{\operatorname{Ch}}
\newcommand{\cl}{\operatorname{cl}}
\newcommand{\coh}{\operatorname{coh}}
\newcommand{\cohom}{\operatorname{cohom}}
\newcommand{\coind}{\operatorname{coind}}
\newcommand{\Coker}{\operatorname{Coker}}
\newcommand{\colim}{\operatorname*{colim}}
\newcommand{\cone}{\operatorname{cone}}
\newcommand{\Cores}{\operatorname{Cores}}
\newcommand{\cosupp}{\operatorname{cosupp}}
\renewcommand{\dim}{\operatorname{dim}}
\newcommand{\domdim}{\operatorname{dom.dim}}
\newcommand{\eff}{\operatorname{eff}}
\newcommand{\Eff}{\operatorname{Eff}}
\newcommand{\End}{\operatorname{End}}
\newcommand{\END}{\operatorname{\mathcal{E}\!\!\;\mathit{nd}}}
\newcommand{\Ext}{\operatorname{Ext}}
\newcommand{\fibre}{\operatorname{fibre}}
\newcommand{\Filt}{\operatorname{Filt}}
\newcommand{\findim}{\operatorname{fin{.}dim}}
\newcommand{\fl}{\operatorname{fl}}
\newcommand{\fp}{\operatorname{fp}}
\newcommand{\Fp}{\operatorname{Fp}}
\newcommand{\Fun}{\operatorname{Fun}}
\newcommand{\Func}{\overrightarrow{\operatorname{Fun}}}
\newcommand{\GL}{\operatorname{GL}}
\newcommand{\gldim}{\operatorname{gl{.}dim}}
\newcommand{\grmod}{\operatorname{grmod}}
\newcommand{\GrMod}{\operatorname{GrMod}}
\newcommand{\height}{\operatorname{ht}}
\newcommand{\hocolim}{\operatorname*{hocolim}}
\newcommand{\holim}{\operatorname*{holim}}
\newcommand{\Hom}{\operatorname{Hom}}
\newcommand{\HOM}{\operatorname{\mathcal{H}\!\!\;\mathit{om}}}
\newcommand{\id}{\operatorname{id}}
\newcommand{\Id}{\operatorname{Id}}
\newcommand{\idim}{\operatorname{idim}}
\newcommand{\Idl}{\operatorname{Idl}}
\renewcommand{\Im}{\operatorname{Im}}
\newcommand{\ind}{\operatorname{ind}}
\newcommand{\Ind}{\operatorname{Ind}}
\newcommand{\inj}{\operatorname{inj}}
\newcommand{\Inj}{\operatorname{Inj}}
\newcommand{\Irr}{\operatorname{Irr}}
\newcommand{\Ker}{\operatorname{Ker}}
\newcommand{\KGdim}{\operatorname{KGdim}}
\newcommand{\Lcolim}{\operatorname*{\mathbf{L}colim}}
\newcommand{\lend}{\operatorname{endol}}
\newcommand{\length}{\operatorname{length}}
\newcommand{\Lex}{\operatorname{Lex}}
\newcommand{\Loc}{\operatorname{Loc}}
\renewcommand{\min}{\operatorname{min}}
\renewcommand{\mod}{\operatorname{mod}}
\newcommand{\Mod}{\operatorname{Mod}}
\newcommand{\Mor}{\operatorname{Mor}}
\newcommand{\Ob}{\operatorname{Ob}}
\newcommand{\oEnd}{\operatorname{\overline{End}}}
\newcommand{\oHom}{\operatorname{\overline{Hom}}}
\newcommand{\omod}{\operatorname{\overline{mod}}}
\newcommand{\oMod}{\operatorname{\overline{Mod}}}
\newcommand{\pcoh}{\operatorname{pcoh}}
\newcommand{\pdim}{\operatorname{proj{.}dim}}
\newcommand{\Perf}{\operatorname{Perf}}
\newcommand{\Ph}{\operatorname{Ph}}
\newcommand{\Prod}{\operatorname{Prod}} 
\newcommand{\proj}{\operatorname{proj}} 
\newcommand{\Proj}{\operatorname{Proj}}
\newcommand{\Qcoh}{\operatorname{Qcoh}}
\newcommand{\rad}{\operatorname{rad}}
\newcommand{\Rad}{\operatorname{Rad}}
\newcommand{\rank}{\operatorname{rank}}
\newcommand{\rep}{\operatorname{rep}}
\newcommand{\Rep}{\operatorname{Rep}}
\newcommand{\REnd}{\operatorname{REnd}}
\newcommand{\Res}{\operatorname{Res}}
\newcommand{\RHom}{\operatorname{RHom}}
\newcommand{\Rlim}{\operatorname*{\mathbf{R}lim}}
\newcommand{\RHOM}{\operatorname{\mathbf{R}\mathcal{H}\!\!\;\mathit{om}}}
\newcommand{\Sh}{\operatorname{Sh}}
\newcommand{\shift}{\operatorname{shift}}
\newcommand{\sHom}{\underline{\Hom}}
\newcommand{\soc}{\operatorname{soc}}
\newcommand{\Sp}{\operatorname{Sp}}
\newcommand{\Spec}{\operatorname{Spec}}
\newcommand{\St}{\operatorname{St}}
\newcommand{\StMod}{\operatorname{StMod}}
\newcommand{\stmod}{\operatorname{stmod}}
\newcommand{\sub}{\operatorname{sub}}
\newcommand{\supp}{\operatorname{supp}}
\newcommand{\Supp}{\operatorname{Supp}}
\newcommand{\thick}{\operatorname{thick}}
\newcommand{\Thick}{\operatorname{Thick}}
\renewcommand{\top}{\operatorname{top}}
\newcommand{\Tor}{\operatorname{Tor}}
\newcommand{\tot}{\operatorname{tot}}
\newcommand{\Tr}{\operatorname{Tr}}
\newcommand{\uEnd}{\operatorname{\underline{End}}}
\newcommand{\uHom}{\operatorname{\underline{Hom}}}
\newcommand{\umod}{\operatorname{\underline{mod}}}
\newcommand{\uMod}{\operatorname{\underline{Mod}}}
\newcommand{\Zsp}{\operatorname{Zsp}}

%%%%%%%%%%%%% mathroman

\newcommand{\Ab}{\mathrm{Ab}} 
\newcommand{\ac}{\mathrm{ac}} 
\newcommand{\can}{\mathrm{can}}
\newcommand{\cent}{\mathrm{cent}}
\newcommand{\Frm}{\mathbf{Frm}} 
\newcommand{\inc}{\mathrm{inc}} 
\newcommand{\init}{\mathrm{init}} 
\newcommand{\op}{\mathrm{op}}
\newcommand{\perf}{\mathrm{perf}}
\newcommand{\qcoh}{\mathrm{qcoh}}
\newcommand{\Qis}{\mathrm{Qis}}
\newcommand{\Set}{\mathrm{Set}}
\newcommand{\term}{\mathrm{term}} 
\newcommand{\Top}{\mathbf{Top}} 

%%%%%%%%%%%%% macros


\newcommand{\two}{\mathbb I}%\mathbf{2}}
\newcommand{\col}{\colon}
\newcommand{\da}{{\downarrow}}
\newcommand{\ges}{{\scriptscriptstyle\geqslant}}
\newcommand{\iso}{\xrightarrow{\raisebox{-.4ex}[0ex][0ex]{$\scriptstyle{\sim}$}}}
\newcommand{\leftiso}{\xleftarrow{\raisebox{-.4ex}[0ex][0ex]{$\scriptstyle{\sim}$}}}
\newcommand{\les}{{\scriptscriptstyle\leqslant}}
\newcommand{\longiso}{\xrightarrow{\ \raisebox{-.4ex}[0ex][0ex]{$\scriptstyle{\sim}$}\ }}
\newcommand{\Lotimes}{\otimes^{\mathbf L}}
%\newcommand{\Lotimes}{\overset{\mathbf L}{\otimes}}
\newcommand{\Lra}{\Leftrightarrow}
\newcommand{\lras}{\leftrightarrows} 
\newcommand{\lto}{\longrightarrow}
\newcommand{\smatrix}[1]{\left[\begin{smallmatrix}#1\end{smallmatrix}\right]}
\newcommand{\ua}{{\uparrow}}
\newcommand{\wt}{\widetilde}
\newcommand{\xla}{\xleftarrow}
\newcommand{\xto}{\xrightarrow}
\newcommand*{\intref}[2]{\def\tmp{#1}\ifx\tmp\empty\hyperref[#2]{\ref*{#2}}\else\hyperref[#2]{#1~\ref*{#2}}\fi}

\DeclareRobustCommand\longtwoheadrightarrow
     {\relbar\joinrel\twoheadrightarrow}

%%%%%%%%%%%%% change parameters

\renewcommand{\labelenumi}{\normalfont (\arabic{enumi})}

%%%%%%%%%%%%% single characters 

\def\A{\mathcal A} 
\def\B{\mathcal B} 
\def\C{\mathcal C}
\def\D{\mathcal D} 
\def\E{\mathcal E} 
\def\F{\mathcal F} 
\def\G{\mathcal G}
\def\I{\mathcal I}
\def\K{\mathcal K} 
\def\L{\mathcal L} 
\def\M{\mathcal M}
\def\P{\mathcal P}
\def\Q{\mathcal Q}
\def\calS{\mathcal S} 
\def\T{\mathcal T} 
\def\U{\mathcal U}
\def\V{\mathcal V}
\def\X{\mathcal X}
\def\Y{\mathcal Y}

\def\bfi{\mathbf i} 
\def\bfm{\mathbf m} 
\def\bfo{\mathbf o} 
\def\bfL{\mathbf L}
\def\bfR{\mathbf R}

\def\bfC{\mathbf C} 
\def\bfD{\mathbf D} 
\def\bfK{\mathbf K}
\def\bfL{\mathbf L} 
\def\bfP{\mathbf P} 
\def\bfr{\mathbf{r}}

\def\bbA{\mathbb A} 
\def\bbF{\mathbb F} 
\def\bbN{\mathbb N}
\def\bbP{\mathbb P}
\def\bbQ{\mathbb Q} 
\def\bbR{\mathbb R} 
\def\bbU{\mathbb U} 
\def\bbX{\mathbb X} 
\def\bbY{\mathbb Y} 
\def\bbZ{\mathbb Z} 


\newcommand{\fra}{\mathfrak{a}} 
\newcommand{\frb}{\mathfrak{b}}
\newcommand{\frJ}{\mathfrak{J}} 
\newcommand{\frI}{\mathfrak{I}} 
\newcommand{\frakm}{\mathfrak{m}} 
\newcommand{\frako}{\mathfrak{0}} 
\newcommand{\frp}{\mathfrak{p}}
\newcommand{\frq}{\mathfrak{q}} 
\newcommand{\frr}{\mathfrak{r}}

\newcommand{\frS}{\mathfrak{S}}

\def\a{\alpha}
\def\b{\beta}
\def\e{\varepsilon}
\def\d{\delta}
\def\g{\gamma}
\def\p{\phi}
\def\s{\sigma}
\def\t{\tau}
\def\th{\theta}
\def\m{\mu}
\def\n{\nu}
\def\k{\kappa}
\def\la{\lambda}

\def\De{\Delta}
\def\Ga{\Gamma}
\def\La{\Lambda}
\def\Si{\Sigma}
\def\Up{\Upsilon}
\def\Om{\Omega}


\begin{document}

\keywords{finitistic dimension, triangulated category}

\subjclass[2020]{18G80 (primary), 16E10 (secondary)}

\begin{abstract}
The finitistic dimension of a triangulated category is introduced. For
the category of perfect complexes over a ring it is shown that this
dimension is finite if and only if the small finitistic dimension of
the ring is finite.
\end{abstract}


\date{\today}

\maketitle

The finitistic dimension of a ring is a homological invariant which is
conjectured to be finite for any finite dimensional algebra over a
field \cite{Ba1960}. It is well known that the finiteness of this
dimension is a derived invariant, even though the integer values of this
dimension in a derived equivalence class may differ \cite{PX2009}. The purpose of this
note is to describe the finiteness of the finitistic dimension as a
property of the triangulated category of perfect complexes. To this
end we introduce for any triangulated category its finitistic
  dimension. This is in the spirit of Rouquier's dimension
\cite{Ro2008}; for a wide class of rings its finiteness characterises
regularity. Note that we focus on the small finitistic
dimension. For the big finitistic dimension and a characterisation of
its finiteness via the the unbounded derived category we refer to
recent work of Rickard \cite{Ri2019}.  \smallskip

\begin{center}
  \textasteriskcentered \qquad \textasteriskcentered \qquad \textasteriskcentered
\end{center}\smallskip

Let $\T$ be a triangulated category with suspension $\Si\colon\T
\iso\T$. From \cite{BV2003,Ro2008} we recall the following
definition. For an object $X$ in $\T$ and $n\ge 0$ set
\[\thick^n(X):=\begin{cases}
  \add\varnothing&n=0,\\
  \add\{\Si^i X\mid i\in\bbZ\}&n=1,\\
  \add\{\cone\p\mid\p\in\Hom(\thick^1(X), \thick^{n-1}(X))&n>1,
\end{cases}\]
where $\add\X$ denotes the smallest full subcategory containing $\X$
that is closed under finite direct sums and direct summands. Also, set
\[\thick(X):=\bigcup_{n\ge
    0}\thick^n(X).\] The triangulated category $\T$ is
\emph{finitely generated}  when $\T=\thick(X)$ for some object $X$, and $\T$ is  \emph{strongly
  finitely generated} if its \emph{dimension}
\[\dim\T:=\inf\{n\ge 0\mid \exists X\in\T\colon\thick^{n+1}(X)=\T\}\]  
is finite.

We continue with some further notation. For a pair of objects $X,Y$ in
$\T$ and $n\ge 0$ we write  $h(X,Y)\le n$  when for any pair $i,j\in\bbZ$
\[\Hom(X,\Si^iY)\neq 0\neq \Hom(X,\Si^jY) \;\;\implies\;\; |i-j|< n.\]
We set \[\hom^n(X):=\{Y\in\T\mid h(X,Y) \le n\}.\]
The \emph{amplitude} of $X$ is
\[\amp(X):=\sup\{|n|\ge 0\mid\Hom(X,\Si^n X)\neq 0\}.\]

We record some elementary properties of $\hom^n(X)$.

\begin{lem}
    \pushQED{\qed}
    For $X\in\T$ and $n\ge 0$ we have
\begin{enumerate}
\item $\hom^0(X)=\{Y\in\T\mid \Hom(X,\Si^iY)=0\text{ for all }i\in\bbZ\}$,      
\item $\hom^1(X)=\{Y\in\T\mid \Hom(X,\Si^iY)\neq 0\text{ for at most one }i\in\bbZ\}$,      
\item $\hom^n(X)=\hom^{n}(\Si^i X)$ for each $i\in\bbZ$,   
\item $\hom^n(X)=\hom^{n+|i|}(X\oplus\Si^i X)$ for each $i\in\bbZ$,
\item $\hom^n(X)\subseteq\hom^n(Y)$ for each $Y\in\add X$,
\item $\hom^n(X)\subseteq\hom^n(Y)$ for each exact triangle $Y'\to Y\to
  Y''\to\Si Y'$ with $Y',Y''\in\add X$.\qedhere
  \end{enumerate}
\end{lem}

\begin{lem}\label{le:hom-thick}
  Let $Y\in\thick(X)$. For each $p\ge 0$ there are inclusions
  \[\thick^p(Y)\subseteq \thick^q(X) \quad\text{and}\quad
    \hom^p(X)\subseteq \hom^q(Y) \quad\text{for}\quad q\gg 0.\]
\end{lem}  
\begin{proof}
  Let  $Y\in\thick^n(X)$. Then  $\thick^p(Y)\subseteq \thick^{p+n}(X)$. 
The second inclusion follows from the previous lemma by an induction on the
number $n$ of steps needed to build $Y$ from $X$.
\end{proof}


The following definition provides a finiteness condition for
triangulated categories which lies in between `finitely generated' and
`strongly finitely generated'.

Recall from \cite{Or2006} that an object $X$ is \emph{homologically
  finite} if it satisfies
$h(X,Y)<\infty$ for each object $Y$.

\begin{defn}
  An object $X$ of a triangulated category $\T$ is a \emph{finitistic
    generator} of $\T$ if $X$ is homologically finite
  and
  \[\hom^p(X)\subseteq\thick^p(X)\quad\text{for all}\quad p\ge 0.\]
  The \emph{finitistic dimension}\footnote{While the term `finitistic
    dimension' goes back to Bass \cite{Ba1960}, the term `finitistic'
    is often attributed to Hilbert's Programme.  Hilbert intended to
    formalise mathematics based on the use of finite methods. This was
    meant as a contribution towards the foundations of mathematics,
    but his syzygy theorem (about invariants, and in modern language
    about modules of finite projective dimension) may well be
    considered as an early instance for this line of thought. In
    fact, Hilbert mentions `the use or the knowledge of a syzygy' as a
    model for the notion of simplicity when he formulated his
    unpublished 24th problem that asks for `criteria of simplicity, or
    proof of the greatest simplicity of certain proofs'.  For
    Hilbert's 24th problem we refer to \cite{MR2019,Th2005}, and for an excellent
    exposition of Hilbert's finitism, see \cite{Ta2013}.} of $\T$ is
\[\findim\T:=\inf\{\amp(X)\mid X \text{ is a finitistic generator of
  }\T\}.\]
As usual, $\findim\T=\infty$ if no finitistic generator exists.
  \end{defn}

\begin{rem}\label{re:bounded}
  (1) If an object $X$ is homologically finite, then
  $\T=\bigcup_{p\ge 0}\hom^p(X)$. Thus $X$ is a generator of $\T$ when
  in addition $\hom^p(X)\subseteq\thick^p(X)$ for all $p\ge 0$.

(2) If $\T$ admits a finitistic generator, then all objects in $\T$ are
homologically finite.

(3) If $\thick^{n+1}(X)=\T$ for some $X\in\T$, then
$\hom^p(Y)\subseteq\thick^{p}(Y)$ for $Y=X\oplus\Si^n X$ and all
$p\ge 0$.  Thus strong finite generation implies that there is
a finitistic generator, provided that all objects are homologically
finite.
\end{rem}  
  
Let $A$ be a ring and let $\P(A)$ denote the class of $A$-modules $M$
that admit a finite resolution
\[0\lto P_n\lto \cdots \lto P_1 \lto P_0\lto M\lto 0\]
with all $P_i$ finitely generated projective. We denote  
\[\findim A:=\sup\{\pdim M\mid M\in\P(A)\}\]
the \emph{small finitistic dimension} of $A$; this is a slight
variation of the usual definition which seems natural as the
modules in $\P(A)$ are precisely the ones which become compact (or
perfect) when viewed as an object in the derived category of $A$.

We are now in the position to characterise $\findim A<\infty$ in terms
of the category of perfect complexes. We will use the following  well
known fact, which is a consequence of the ghost lemma (cf.\ \cite[Lemma~2.4]{KK2006}).

\begin{lem}\label{le:pdim}
\pushQED{\qed}  For $M\in\P(A)$ and $n > 0$ we have 
  \[\pdim M< n\;\;\iff\;\; M\in\thick^{n}(A).\qedhere\]
\end{lem}

Let $\Perf(A)$ denote the category of perfect complexes over
$A$.

\begin{thm}
  For a ring $A$ we have
  \[\findim A<\infty\;\;\iff\;\;\findim\Perf(A)<\infty.\]
More precisely,  if $X$ is a finitistic generator of $\Perf(A)$ satisfying
\[\hom^1(A)\subseteq\hom^p(X)\quad\text{and}\quad\thick^1(X)\subseteq\thick^q(A)\]
for some $p,q\ge 0$, then
\[\findim \Perf(A)\le\findim A<p+q.\]
\end{thm}
\begin{proof}
The crucial observation is that taking the cohomology of a complex
identifies the objects in $\hom^1(A)$  with the modules in $\P(A)$.
  
Suppose first that $d=\findim A<\infty$. We show by induction on $p\ge 0$ that
  $\hom^p(A)\subseteq\thick^{p+d}(A)$. The case $p=0$ is clear and we
  may assume $p>0$. Let $X\in \hom^p(A)$ and
  suppose its cohomology \[H^n(X)=\Hom(A,\Si^nX)\] is concentrated in
  degrees $0,\ldots,p-1$. Then we may assume $X^n=0$ for $n\ge p$. For $p=1$
the complex $X$ identifies with $H^0(X)\in\P(A)$
and then Lemma~\ref{le:pdim} implies $X\in\thick^{d+1}(A)$. For $p>1$
we may write $X$ as an extension of the truncation
\[X'\colon\quad\cdots \lto X^{p-4}\lto X^{p-3}\lto X^{p-2}\lto
  0\lto\cdots\] and a complex concentrated in
degree $p-1$. We have \[X'\in \hom^{p-1}(A)\subseteq\thick^{p-1+d}(A)\] and therefore
$X\in\thick^{p+d}(A)$. For $A'=A\oplus\Si^dA$ we have 
\[\hom^{p+d}(A')=\hom^p(A)\subseteq\thick^{p+d}(A)=\thick^{p+d}(A'),\]
and we note that $\amp(A')=d$. Thus $\findim \Perf(A)\le\findim A$.

Suppose now that $X$ is a finitistic generator of $\Perf(A)$.  Using
Lemma~\ref{le:hom-thick} we have
\[\hom^1(A)\subseteq\hom^p(X)\subseteq \thick^p(X)\subseteq \thick^n(A)\] for some $n,p\ge 0$.
Thus $\findim A<n$ by Lemma~\ref{le:pdim}.
\end{proof}

\begin{rem}
  There are examples of rings such that $\findim A=\infty$ and
  $\findim A^\op=0$; cf.\ \cite{Kr2022}.  We have
  $\Perf(A^\op)\simeq\Perf(A)^\op$, and therefore finiteness of 
  finitistic dimension is not a symmetric notion for triangulated categories.
\end{rem}

\begin{rem}
  For a noetherian ring $A$, strong finite generation of $\Perf(A)$
  implies that $A$ is \emph{regular}, so each finitely generated
  $A$-module has finite projective dimension
  \cite[Proposition~7.25]{Ro2008}. In general (and despite such claims
  in the literature), the converse is not true even when $A$ is
  commutative \cite[Appendix, Example~1]{Na1962}.  However when $A$ is
  in addition semilocal, then $A$ is regular if and only if $A$ has
  finite global dimension, and this implies that $\Perf(A)$ is
  strongly finitely generated \cite[Proposition~2.6]{KK2006}. This
  provides many examples where $\Perf(A)$ has a finitistic generator
  but no strong generator.
\end{rem}

\begin{exm}
  Let $\T$ be an idempotent complete algebraic triangulated
  category. If $\findim\T=0$, then there exists a ring $A$ with
  $\findim A=0$ and a triangle equivalence $\T\iso \Perf(A)$.
\end{exm}
\begin{proof}
  A generator $X$ of $\T$ with $\amp(X)=0$ is a tilting object. Thus
  for $A=\End(X)$ the functor $\RHom(X,-)$ provides an equivalence
  $\T\iso \Perf(A)$; see
  \cite[Proposition~9.1.20]{Kr2023}. Moreover,
  $\hom^1(A)\subseteq\thick^1(A)$ implies $\findim A=0$ by
  Lemma~\ref{le:pdim}.
\end{proof}


\subsection*{Acknowledgement}

This work was supported by the Deutsche
Forschungsgemeinschaft (SFB-TRR 358/1 2023 - 491392403).
\begin{thebibliography}{10}

\bibitem{Ba1960} H. Bass, \emph{Finitistic dimension and a homological
    generalization of semi-primary rings},
  Trans. Amer. Math. Soc. \textbf{95} (1960) 466--488.

\bibitem{BV2003} A.~I. Bondal\ and\ M. Van~den~Bergh, \emph{Generators and
  representability of functors in commutative and noncommutative
  geometry}, Mosc. Math. J. {\bf 3} (2003), no.~1, 1--36, 258.

\bibitem{Kr2023} H. Krause, \emph{Homological theory of
    representations}, Cambridge Studies in Advanced Mathematics, 195,
  Cambridge University Press, Cambridge, 2022.

\bibitem{Kr2022} H. Krause, \emph{On the symmetry of the finitistic
  dimension}, C. R. Math. Acad. Sci. Paris, to appear.

\bibitem{KK2006} H. Krause\ and\ D. Kussin, \emph{Rouquier's theorem on
  representation dimension}, in {\it Trends in representation theory of
    algebras and related topics}, 95--103, Contemp. Math., 406,
  Amer. Math. Soc., Providence, RI, 2006.

\bibitem{MR2019} A. Malheiro, J. F. Reis, \emph{Identification of proofs via syzygies},
Philos. Trans. Roy. Soc. A \textbf{377} (2019), no.~2140, 20180275, 11 pp.  

\bibitem{Na1962} M. Nagata, \emph{Local rings}, Interscience Tracts in
  Pure and Applied Mathematics, 13, Interscience Publishers, New York,
  1962.

\bibitem{Or2006}  D.~O. Orlov, \emph{Triangulated categories of
    singularities, and equivalences between Landau-Ginzburg models}, Sb. Math. {\bf 197} (2006), no.~11-12,
  1827--1840; translated from Mat. Sb. {\bf 197} (2006), no. 12,
  117--132.

\bibitem{PX2009} S.~Y. Pan\ and\ C.~C. Xi, \emph{Finiteness of finitistic
  dimension is invariant under derived equivalences}, J. Algebra {\bf
    322} (2009), no.~1, 21--24.

\bibitem{Ri2019} J. Rickard, \emph{Unbounded derived categories and the
  finitistic dimension conjecture}, Adv. Math. {\bf 354} (2019),
  106735, 21 pp.  
  
\bibitem{Ro2008} R. Rouquier, \emph{Dimensions of triangulated categories},
  J. K-Theory {\bf 1} (2008), no.~2, 193--256.

\bibitem{Ta2013} C. Tapp, \emph{An den Grenzen des Endlichen},
  Mathematik im Kontext, Springer-Verlag Berlin Heidelberg, 2013.

\bibitem{Th2005} R. Thiele, \emph{Hilbert's twenty-fourth problem},
  Amer. Math. Monthly \textbf{110} (2003), no.~1, 1–24.

\end{thebibliography}

\end{document}
      

