\section{Preliminaries}

\section{Notation Table}
\begin{table}[h]
% \setlength{\abovecaptionskip}{0.cm}
    \caption{Frequently used notation. }
    \label{tab:notations}
    \begin{tabular}{m{1.3cm}<{\centering}|m{6.3cm}}
      \toprule
      \textbf{Notations} & \textbf{Definitions}\\
       \midrule
    %   \hline
      $T$ & The length of history traffic data. \\
      $T_f$ & The length of future traffic data.\\
      $N$ & The number of time series. \\
      $C$ & The number of feature channels in a traffic sensor.\\
      $\mathcal{X}$ & History data of shape $\mathbb{R}^{T\times N\times C}$\\
      $\mathcal{Y}$ & Future data of shape $\mathbb{R}^{T_f\times N\times C}$\\
      $\mathbf{X}^i$ & History data of sensor $i$.\\
      $\mathbf{Y}^i$ & Future data of sensor $i$.\\
      $\hat{\mathbf{Y}}^i$ & Prediction data of sensor $i$.\\
      $L$ & The segment size.\\
      $P$ & The number of segments. $T=P\times L$.\\
      $d$ & The hidden dimension.\\
      $\mathbf{W}$ & Parameter matrix of the fully connected layer.\\
      $\mathbf{b}$ & Parameter of the bias of the fully connected layer.\\
      $\mathbf{E}$ & Spatial embeddings of shape $\mathbb{R}^{N\times d_1}$.\\
      $\mathbf{T}$ & Temporal embeddings.\\
      % $\mathbf{T}_j^{DiW}$
      $N_D$ & The number of time slots of a day.\\
      $N_W$ & The number of days in a week (7).\\
      $\mathbf{S}$ & Embeddings of each segment after segment embedding.\\
      $\mathbf{U}$ & Embeddings of each segment after spatial-temporal positional encoding.\\
      $\mathbf{H}$ & Hidden states.\\
      \bottomrule
    \end{tabular}
\end{table}


% In this 
In this section, we define the notions of traffic data and traffic forecasting task. Frequently used notations are summarized in Table \ref{tab:notations}.
% Traffic Data
% Traffic Forecasting
\begin{definition}
\textbf{Traffic Data} $\mathcal{X}\in\mathbb{R}^{T\times N\times C}$ denotes the observation from all sensors on the traffic network, 
where $T$ is the number of time steps, $N$ is the number of traffic sensors, and $C$ is the number of features collected by sensors.
We additionally denote the data from the sensor $i$ as $\mathbf{X}^{i}\in\mathbb{R}^{T\times C}$.
% and the data of time step $t$ as $\mathbf{X}_{t}\in\mathbb{R}^{N\times C}$.
\end{definition}

\begin{definition}
\textbf{Traffic Forecasting} aims to predict the traffic values $\mathcal{Y}\in\mathbb{R}^{T_f\times N\times C}$ of the $T_f$ nearest future time steps based on the given historical traffic data $\mathcal{X}\in\mathbb{R}^{T_h\times N\times C}$ from the past $T_h$ time steps.
In this paper, we focus on long-term traffic forecasting, e.g., forecasting for a day in the future.

\end{definition}
