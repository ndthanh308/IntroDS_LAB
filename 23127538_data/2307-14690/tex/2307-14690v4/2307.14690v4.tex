
\documentclass{article}
%\usepackage{CJK}
% \usepackage{leftidx}
\usepackage[a4paper,left=3cm,top=3.5cm]{geometry}
\usepackage{verbatim}
%\usepackage{graphicx}
%\usepackage{syntonly}
%\syntaxonly
\usepackage[all,cmtip]{xy}

\usepackage{mathtools}

%\usepackage{CJK,CJKspace,CJKnumb}

\usepackage{latexsym,bm,easymat}
\usepackage{amsmath,amscd,amssymb,amsfonts,mathdots,ntheorem}
\usepackage{indentfirst}
\usepackage[colorlinks,linkcolor=blue,citecolor=red]{hyperref}
%\usepackage{xcolor,graphicx,blkarray}
\usepackage{latexsym}
\usepackage{longtable}
%\usepackage{colortab}
\usepackage{cases}
\usepackage{slashed}
\usepackage{tikz-cd}


\usepackage{shorttoc}
%%%

%%%
\begin{document}
	\makeatletter
	\DeclareRobustCommand\widecheck[1]{{\mathpalette\@widecheck{#1}}}
	\def\@widecheck#1#2{%
		\setbox\z@\hbox{\m@th$#1#2$}%
		\setbox\tw@\hbox{\m@th$#1%
			\widehat{%
				\vrule\@width\z@\@height\ht\z@
				\vrule\@height\z@\@width\wd\z@}$}%
		\dp\tw@-\ht\z@
		\@tempdima\ht\z@ \advance\@tempdima2\ht\tw@ \divide\@tempdima\thr@@
		\setbox\tw@\hbox{%
			\raise\@tempdima\hbox{\scalebox{1}[-1]{\lower\@tempdima\box
					\tw@}}}%
		{\ooalign{\box\tw@ \cr \box\z@}}}
	\makeatother
	
	%%%%%%%%%%%define of the proof=============
	\def\comp{\ensuremath\mathop{\scalebox{.6}{$\circ$}}}
	\def\QEDclosed{\mbox{\rule[0pt]{1.3ex}{1.3ex}}} %
	\def\QEDopen{{\setlength{\fboxsep}{0pt}\setlength{\fboxrule}{0.2pt}\fbox{\rule[0pt]{0pt}{1.3ex}\rule[0pt]{1.3ex}{0pt}}}}
	\def\QED{\QEDopen} %
	\def\pf{\noindent{\bf Proof}} %
	\def\endpf{\hspace*{\fill}~\QED\par\endtrivlist\unskip \hfill}
	\def\Dirac{\slashed{D}}
	%%%%%%%%%%%%%%%% Define the symbol and notation =====
	\def\Hom{\mbox{Hom}}
	\def\ch{\mbox{ch}}
	\def\Ch{\mbox{Ch}}
	\def\Tr{\mbox{Tr}}
	\def\End{\mbox{End}}
	\def\Re{\mbox{Re}}
	\def\cs{\mbox{Cs}}
	\def\spinc{\mbox{spin}^\mathbb C}
	\def\spin{\mbox{spin}}
	\def\Ind{\mbox{Ind}}
	\def\Hom{\mbox{Hom}}
	\def\ch{\mbox{ch}}
	%\def\trs{\mbox{Tr}_s}
	%\def\Re{\mbox{Re}}
	\def\cs{\mbox{cs}}
	\def\nTbe{\nabla^{T,\beta,\epsilon}}
	\def\nbe{\nabla^{\beta,\epsilon}}
	\def\mfa{\mathfrak{a}}
	\def\mfb{\mathfrak{b}}
	\def\mfc{\mathfrak{c}}
	\def\mcM{\mathcal{M}}
	\def\mcC{\mathcal{C}}
	\def\mcB{\mathcal{B}}
	\def\mcG{\mathcal{G}}
	\def\mcE{\mathcal{E}}
	\def\mfs{\mathfrak s}
	\def\mcWs{\mathcal W_{\mathfrak s}}
	\def\umcF{\underline{\mathcal F}}
	\def\mce{\mathfrak e}
	\def\mbT{\mathbf T}
	\def\mbc{\mathbf c}
	\def\spinc{$\mbox{spin}^c~$}
	
	%%%%%%%%%%
	%email-address
	\newcommand{\at}{\makeatletter @\makeatother}
	%%%%%%%%%%%%%%%%%%%%%%%%%%%%%\theoremstyle{mystyle}
	%\newcommand{\qed}{\hfill$\Box$}
	\newtheorem{thm}{Theorem}[section] 
	
	\newtheorem{defi}[thm]{Definition} 
	\newtheorem{lemma}[thm]{Lemma} 
	
	\newtheorem{cor}[thm]{Corollary}
	\newtheorem{exam}{Example}[section]
	\newtheorem{prop}[thm]{Proposition}
	\newtheorem{rmk}[thm]{Remark}
	\newtheorem*{ack}{Acknowledgement}
	\newtheorem*{emp}{~~~}
	\newtheorem*{que}{Question}
	\numberwithin{equation}{section}
	\newtheorem{exa}[thm]{Example}
	
	%%%%%%%%-----------------title
	
	\title{A Kodaira type conjecture on almost complex 4 manifolds}
	\date{}
	\author{Dexie Lin}
	\maketitle
	College of Mathematics and Statistics, Chongqing University,
	Huxi Campus, Chongqing, 401331, P. R. China
	
	Chongqing Key Laboratory of Analytic Mathematics and Applications, Chongqing University, Huxi Campus, Chongqing, 401331, P. R.
	China
	
	E-mail: lindexie@126.com	
	
	%%%%%%%%%%%%%%%%%%%%%%%%%%%%%%%%%%%%%
	\begin{abstract}
		%For compact complex surfaces, the existence of K\"ahler structure is a topological condition, i.e., $b_1\equiv0\bmod2$. However, by the exotic phenomenon in smooth $4$ manifolds, the existence of symplectic structure can not be a topological obstruction for almost complex $4$ manifolds. 
		Not long ago, Cirici and Wilson defined a Dolbeault cohomology on almost complex manifolds to answer Hirzebruch's problem.
		In this paper, we define  a refined Dolbeault cohomology on   almost complex manifolds. We show that the condition $\tilde h^{1,0}=\tilde h^{0,1}$ implies  a symplectic structure on a compact almost complex $4$ manifold, where $\tilde h^{1,0}$ and  $\tilde h^{0,1}$ are the dimensions of the refined Dolbeault cohomology groups with bi-degrees $(1,0)$ and $(0,1)$ respectively. Combining the partial answer to Donaldson's tameness conjecture, we offer a sufficient condition for a compact almost complex $4$ manifold to become  an  almost K\"ahler one.  % We also show that the refined Hodge number $\tilde{h}^{0,1}$ is finite on any compact almost complex $4$ manifold, which is quite different to the Dolbeault cohomology defined by Cirici and Wilson. 
		 Moreover, we prove that the condition $\tilde{h}^{1,0}=\tilde h^{0,1}$ is equivalent to the generalized $\partial\bar\partial$-lemma. This can be regarded as an analogue of the Kodaira's conjecture on almost complex $4$ manifolds. As an application, we   show that the Kodaira-Thurston manifold satisfies % the identity $\tilde h^{1,0}=\tilde h^{0,1}$ and 
		 the  $\partial\bar\partial$-lemma. Meanwhile, we show that   the Fr\"olicher-type equality does not hold on a  general almost complex $4$ manifold, which is  different to the case of  compact complex surfaces. % The main results of this paper can be regarded as the symplectic proof of the Kodaira's conjecture. 
	\end{abstract}
	\noindent
	{\bf Keywords}: almost complex manifold, Dolbeault cohomology, symplectic structure
	
	\noindent
	{\bf AMS classification}: 57R30, 53A45, 53C23, 53D35
	%\tableofcontents
	
	%%%%%%%%%%%%%%%
	
	\section{Introduction}
	%A complex manifold is a real $2n$ dimensional manifold $M$   whose coordinate charts are open subsets of $\mathbb C^n$ and the transition functions between charts are holomorphic, c.f.,   \cite[Section 2-Chapter 0]{GH} and \cite[Chapter 2]{Huy}.  The holomorphic structure induces a complex structure $J$ on the tangent bundle $TM$.
	On   complex manifolds, the deRham differential has the decomposition	
	$d=\partial+\bar\partial,$
	where $\partial$ and $\bar\partial$ are first order differential operators with   bi-degrees $(1,0)$ and $(0,1)$ respectively, c.f., \cite[Definition 2.6.9 and Proposition 2.6.15]{Huy}. %Expanding $d^2=0$, one gets that 	\[\partial^2=0,~\bar\partial^2=0,\mbox{ and }\partial\bar\partial+\bar\partial\partial=0.\]
	The formula $\bar\partial^2=0$ allows a natural definition of the Dolbeault cohomology,   whose dimensions provide  complex invariants
	of  compact complex manifolds, c.f.,  \cite[Section 2 and 3-Chapter 0]{GH}. In this paper, a compact manifolds means a compact manifolds without boundary. % On compact complex manifolds, it turns out that all dimensions of $H^{*,*}_{Dol}$ are finite. 
	Similar to Riemannian manifolds,
	a  complex manifold $M$ equipped with a $J$-invariant metric is called an Hermitian manifold, where $J$ is the natural complex structure on $M$. On a compact Hermitian  manifold, the Hodge decomposition shows that $H^{*,*}_{Dol}$ are isomorphic to  ${\bar\partial}$-harmonic spaces with respect to the same bi-degrees, c.f., \cite[Section 6-Chapter 0]{GH}. 
	
	From the viewpoint of metrics,
	a K\"ahler manifold  is an Hermitian manifold of complex dimension $n$ such that for every point, there is a holomorphic coordinate chart around this point in which the metric agrees with the standard metric on $\mathbb C^n$ up to order $2$, c.f.,  \cite[Chapter 2 and Chapter 3]{Huy}. On   compact K\"ahler manifolds,   the K\"ahler identities(\cite[Proposition 3.1.12]{Huy}) and the Hodge decomposition yield
	%\[\Delta_d=2\Delta_{\partial}=2\Delta_{\bar\partial}.\]
	%Hence, by Hodge-decomposition  one gets
	\[ b_r =\sum_{j+k=r}h^{j,k}\mbox{ and }
	h^{j,k}=h^{k,j}.
	\]
	Here $h^{j,k}=\dim_{\mathbb C}H^{j,k}_{Dol}$ is the Hodge number of bi-degree $(j,k)$, c.f., \cite[Section 7-Chapter 0]{GH}. Consequently,  all
	Betti numbers with odd degrees   must be  even. 
	In \cite[Page 95]{KM},  Kodaira conjectured that: 
	\begin{emp}
		Every compact complex surface  admits a K\"ahler structure if and only if its first Betti number is even.
	\end{emp}
	By the Enrique-Kodaira's   classification \cite[Chapter VI]{BHPV} on compact complex surfaces,
	only the
	cases of elliptic surfaces and $K3$ surfaces remained open at that time.
	\begin{itemize}
		\item[(1)] Miyaoka \cite{Miy74}  proved that the
		conjecture holds for elliptic surfaces by using the current technique.   
		\item[(2)] Siu \cite{Siu}  overcame the difficulties in Todorov's proof \cite{Tod} and  proved 
		that every $K3$ surface is K\"ahler, hence completing the proof of Kodaira's	conjecture.
		\item[(3)] Buchdahl \cite{Buch} and Lamari \cite{Lamari} independently gave a unified proof of the Kodaira's conjecture.
	\end{itemize} 
	In this paper, we want to consider a variant of the  Kodaira's conjecture in the case of almost complex $4$ manifolds. Here, an almost complex manifold is  a manifold whose tangent bundle admits a complex structure.
	In a big picture, the Kodaira's  conjecture fills  in the condition for the upper-left arrow of the following diagram.
	\[\begin{tikzcd}	    
		& \mbox{K\"ahler surface  }          &              \\
		\mbox{complex surface}\arrow[ru,"{\boxed{b_1\mbox{ even}} }"] &              & \mbox{symplectic 4-manifold} \arrow[ul,"\boxed{\mbox{Nijenhuis tensor}=0}"'] \\
		&\mbox{ almost complex 4-manifold}\arrow["\boxed{\mbox{Nijenhuis tensor}=0}",lu] \arrow[ru,""'] &  	 
		%	&\mbox{oriented compact smooth  4 manifold}\arrow[u,"\boxed{\mbox{purelly topological obstruction}}"',]&      
	\end{tikzcd}\]	 
	%In 4 manifold, one popular topic is to determine when a manifold admits a symplectic structure.
	%the following:\begin{emp}    For a non-compact  manifold, every nondegenerate $2$-form  can be homotoped to a nondegenerate closed form   through nondegenerate forms. %Moreover, one can specify the cohomology class of this  nondegenerate closed form   in advance. \end{emp}This implies that 
	
	From the above diagram,
	it is natural to ask: What is a proper condition for a compact almost complex $4$ manifold   to admit  a symplectic structure? Namely, what should be a proper generalization of the Kodaira's conjecture on compact almost complex $4$ manifolds? To start with,  the naive generalization is invalid. 
	In fact,   Gompf  \cite{Gom95} showed that: 
	\begin{emp}
		Every finitely presented group $G$ can be realized as  the fundamental group of some compact symplectic $4$ manifold.
	\end{emp} 
	By using Seiberg-Witten gauge theory, there are many  almost complex $4$ manifolds with  even first Betti numbers, but do not admit any symplectic structure, e.g., $(2k+1)\mathbb{C}P^2\# l\overline{\mathbb{C}P^2}$ with integers $k>0$ and $l\geq0$.
	Back to the case of  compact complex surfaces, the existence of a K\"ahler structure is equivalent to one of the following conditions,  c.f., \cite[Section 2 and 3-Chapter IV]{BHPV} and Gauduchon's work\cite{Gau76}:
	\begin{equation}
		(1)~b_1\mbox{ is even};~~~~(2)~\mbox{condition on }h^{1,0}\mbox{ and } h^{0,1};~~~~(3)~\partial\bar\partial-\mbox{lemma}.\label{eqn-kahler-condition}%b_1=2h^{1,0}\mbox{ or }b_1=2h^{0,1}
	\end{equation}
	Here, the  $\mbox{condition on }h^{1,0}\mbox{ and } h^{0,1}$ means  one of the following equivalent equalities:
	\begin{equation}
		h^{1,0}=h^{0,1},~b_1=2h^{1,0}\mbox{ and }  b_1=2h^{0,1}.	\label{eqn-three-identities}
	\end{equation} 
	%	Thus, we can  excludes the first choice in \eqref{eqn-kahler-condition} for an almost complex $4$ manifold admitting a  symplectic structure, by 
	%	\end{thm}
%\begin{itemize}		\item[(1)] $b_1$\mbox{ is even};		\item[(2)] $h^{1,0}= h^{0,1}$;		\item[(3)] $b_1=2h^{1,0}$ or $b_1=2h^{0,1}$.	\end{itemize}
%Gompf's result excludes the first choice in \eqref{eqn-kahler-condition} for almost complex manifolds admitting  symplectic structure. 
%On an almost complex manifold, we have  two additional terms for the decomposition, \[d=\partial+\bar\partial+\mu+\bar\mu,\] where $\bar\mu$  and $\mu$  arise from the Nijenhuis tensor.
%Not long ago, ,  %In Section 2, we will review  their definition of the Dolbeault cohomology   for almost complex manifolds.  We still denote the Dolbeault cohomology groups   by $H^{*,*}_{Dol}$ and their dimensions by $h^{*,*}$. 
%By using  the Dolbeault cohomology groups 
Hence, it is natural to ask whether there is a suitable generalization of condition (2) or (3) in \eqref{eqn-kahler-condition} that serves as a sufficient condition for   the existence of a symplectic structure. 

To answer Hirzebruch's $20$-th. problem \cite{Hir54}, Cirici and Wilson \cite{CW21} defined a Dolbeault type cohomology for compact almost complex manifolds. However, Coelho,  Placini and Stelzig \cite{CPS} showed that the Hodge number $h^{0,1}$ is  finite if and only if such a compact almost complex $4$-manifold is a compact complex surface. In other words, the condition $h^{1,0}=h^{0,1}$ is equivalent to the K\"ahler condition on the compact almost complex $4$ manifolds.  Here $h^{1,0}$ and $h^{0,1}$ are the Hodge numbers of the  Dolbeault cohomology  defined by Cirici and Wilson.  
On the other hand, there are many compact symplectic $4$ manifolds without any K\"ahler structure  \cite{TO97}. Therefore, it raises a natural motivation  to expect another cohomology on compact almost complex manifold to study the symplectic structure. %on an almost complex manifold. 
In this paper,  we define the  refined Dolbeault cohomology by  the cohomology of a complex $$\cdots\mathcal{A}^{*,*-1}_{Dol}\overset{\bar\partial}{\to}\mathcal{A}^{*,*}_{Dol}
\overset{\bar\partial}{\to}\mathcal{A}^{*,*+1}_{Dol}\cdots,$$
where $\mathcal{A}^{*,*}_{Dol}$ is a  certain subspace of the complex-valued forms, see Definition \ref{defi-tilde-Dolbeault}.  In particular, the identities $\partial^2|_{\mathcal{A}^{*,*}_{Dol}}=0=\bar\partial^2|_{\mathcal{A}^{*,*}_{Dol}}$ hold. This refined cohomology is denoted by $\tilde{H}^{*,*}_{Dol}$. % We set $\tilde{h}^{*,*}=\dim_{\mathbb C}\tilde{H}^{*.*}_{Dol}$.
 We give the refined Hodge numbers %$\tilde{h}^{*,*}=\dim_{\mathbb C}\tilde{H}^{*,*}_{Dol}$ 
 on the Kodaira-Thurston manifold by the following diamond: 
\[\tilde{h}^{*,*}:=\dim_{\mathbb C}\Tilde{H}^{*,*}_{Dol}=\begin{array}{ccccc}
	& &1&& \\
	&\infty &&1&\\
	0&&\infty&&0\\
	&1&&1&\\
	&&1&&
\end{array}.\]
It is clear that this non-K\"ahlerian symplectic $4$ manifold
satisfies the identity $\Tilde{h}^{1,0}=\Tilde{h}^{0,1}$, see Example \ref{exa-kt} for more details. 
We start from the condition (2) in \eqref{eqn-kahler-condition} by  giving a sufficient condition for a compact almost complex $4$ manifold admitting a symplectic structure in terms of the refined Dolbeault cohomology. 
\begin{thm}\label{thm-main-1}
	Let $(X,J)$ be a compact almost complex $4$ manifold. Suppose that $\tilde h^{1,0}=\tilde h^{0,1}$. 
	Then, $X$ admits a $J$-taming  symplectic form $$ \omega'=F+d^{-}_J(v+\bar v),$$
	where $F$ is a positive $J-(1,1)$ form, $v$ is a $J-(0,1)$-form and $d^-_J:=\frac{d-J\comp d}{2}$ denotes the anti-$J$-invariant image of $d$.
\end{thm}
%Note that  $J$ is $\omega$-tame by the proof of the above theorem. 
Recall that  Donaldson \cite{Don06} posed a question: 
 If
an almost complex structure is tamed by a given symplectic form, must it be compatible
with a new symplectic form ? Such a question naturally arises from the calibrated geometry. 
Combining  the partial answer of Tan et al \cite{TWZZ22} and  Wang et al \cite[Theorem 4.3]{WWZ} to Donaldson's question,  one has the following corollary. 
\begin{cor}
	Let $(X,J)$ be a compact almost complex $4$ manifold. Suppose that $\tilde h^{1,0}=\tilde h^{0,1}$ and $h^-_J=b^+-1$ hold, where $h^-_J$ is the dimension of $J$-anti-invariant second cohomology. Then, $(X,J)$ admits a $J$-compatible symplectic form $\omega=\omega'-d(v+\bar v)=F-d^+_J(v+\bar v)$, where %$v$ is a $(0,1)$-form and 
	$d^+_J=\frac{d+J\comp d}{2}$ denotes the  $J$-invariant image of $d$. 
\end{cor}
 %one can find another symplectic form $\omega'$ such that $J$ is $\omega'$-compactible under the condition $h^-_J=b^+-1$, where $h^-_J$ is the dimension of $J$-anti-invariant second cohomology. 
%To answer  Hirzebruch's $20$-th  problem \cite{Hir54}, Cirici and Wilson \cite{CW21} defined the Dolbeault cohomology $H^{*,*}_{Dol}$ by using the relations between $\bar\partial$ and $\bar\mu$ from the decomposition  \[d=\mu+\partial+\bar{\mu}+\bar{\partial},\]on almost complex manifolds. 
Moreover, on compact almost complex $4$ manifolds, we show the following inequalities
\begin{equation}
	\tilde{h}^{1,0}=h^{1,0}\leq\tilde{h}^{0,1} \leq h^{0,1} ,	\label{eqn-tilde-Dolbeault-ineq}
\end{equation}
%where  $h^{p,q}=\dim_{\mathbb{C}}H^{p,q}_{Dol}$,
%\mbox{on compact almost complex $4$ manifolds, }
c.f.,    Lemma \ref{lemma-h1} and Corollary \ref{coro-betti-inequality}. 


%\noindent
Next,   consider the  condition (3) in \eqref{eqn-kahler-condition}. 
Recall that on compact complex surfaces, Gauduchon \cite{Gau76} proved  that the  $\partial\bar\partial$-lemma is equivalent to any one of the three conditions in \eqref{eqn-three-identities}. Hence, the  $\partial\bar\partial$-lemma is equivalent to the existence of K\"ahler structure on compact complex surfaces. %By importing   new terms  $\hat h^{0,1}$ and $\hat h^1$, c.f., Definition \ref{defi-hat-Dolbeault}, 
We  obtain an equivalent condition to the generalized $\partial\bar\partial$-lemma  on compact almost complex $4$ manifolds. 
\begin{thm}[generalized $\partial\bar\partial$-lemma]\label{thm-ddbar}
	Let $(X,J)$ be a compact almost complex $4$ manifold. Then,  any  $d$-exact $(1,1)$ form $\psi$ can be written as $\psi=\partial\bar\partial f$ for some    function $f$, if and only if the equality  $\tilde h^{1,0}=\tilde h^{0,1}$ holds.
\end{thm}
\noindent
Note that
	on compact K\"ahler manifolds, it is well-known that each $d$-exact form is $\partial\bar\partial$-exact, which is called the $\partial\bar\partial$-lemma.  Gauduchon's work \cite{Gau76} tells us that on the case of compact complex surfaces, the $\partial\bar\partial$-lemma is equivalent to the condition $h^{1,0}=h^{0,1}$. Hence, by the proof of Kodaira conjecture, the $\partial\bar\partial$-lemma is equivalent to the K\"ahler condition on  compact complex surfaces. %Since it holds that 	$\tilde h^{0,1}=h^{0,1}=\bar h^{0,1}$ for compact complex surfaces, the above lemma can be regarded as an alternative proof of Gauduchon's work.
%\end{rmk} 
%Since it well-known that  Gromov \cite{Gromov86} showed the following:\begin{emp}    For a non-compact  manifold, every nondegenerate $2$-form  can be homotoped to a nondegenerate closed form   through nondegenerate forms. Moreover, one can specify the cohomology class of this  nondegenerate closed form   in advance. \end{emp}

The above results partially fill into the top arrow of the following  diagram.
\begin{center}
	\begin{tikzcd}	    
		& \mbox{compact symplectic 4-manifold}        &              \\
		&    \mbox{compact almost complex 4-manifold  }   \arrow[u]    &  \\
		&\mbox{oriented compact smooth  4 manifold}\arrow[u,"\boxed{\mbox{purelly topological obstruction}}"',]&      
	\end{tikzcd}
\end{center}
%On compact oriented smooth $4$ manifolds without  boundary, it is well-known that  the existence of an  almost complex structure is a  topological obstruction, 
Here, this topological obstruction is the existence of a cohomology class satisfying a certain condition, c.f., {\cite[Proposition 9.3-Chapter:  IV]{BHPV}}.
For the  convenience of readers' understanding, we give the following diagram to illustrate the conditions for compact almost complex $4$ manifolds admitting the symplectic structure.

\begin{center}
	\begin{tikzcd}[column sep=0.02, row sep=small] 
		& {\begin{array}{l}
				\mbox{symplectic structure}
		\end{array}} &                                                   &                                                    \\
		{\tilde h^{1,0}=\tilde h^{0,1}} \arrow[rr, Leftrightarrow,"\begin{array}{c}
			~
			\\
			\mbox{Almost Complex 4 Manifolds}
		\end{array}"'] \arrow[ru, Rightarrow] &                           & \begin{array}{l}
		\mbox{generalized }	\partial\bar\partial- 
			\mbox{lemma}
		\end{array}  \arrow[lu, Rightarrow]                           
	\end{tikzcd}
\end{center} 
 Note that on compact complex surfaces, the three conditions are equivalent to each other, but for higher dimension there are non-K\"ahlerian complex manifolds satisfying the $\partial\bar\partial$-lemma, e.g., Moishezon manifolds.  Hence, it is natural to consider the following question. 
%	\usetikzlibrary {graphs}
%	\tikz \graph { a -> {b, c} -> d };

%	\begin{center}\begin{tikzcd}[column sep=0.02, row sep=small] 	& {\begin{array}{l}		\mbox{symplectic} \\\mbox{structure}		\end{array}}&                                                   &  &                                                                 &  {\begin{array}{l}		\mbox{K\"ahler} \\\mbox{structure}	\end{array}} &                                                   \\	{h^{0,1}=\tilde h^{0,1}} \arrow[rr, Leftrightarrow,"\begin{array}{c}		~		\\		\mbox{ almost complex }\\		4 \mbox{ manifolds}	\end{array}"'] \arrow[ru, Rightarrow] &                           & \begin{array}{l}	 \partial\bar\partial-\\	 \mbox{lemma}	\end{array}  \arrow[lu, Rightarrow] &  & {h^{0,1}=h^{1,0}} \arrow[ru, Leftrightarrow] \arrow[rr, Leftrightarrow,"\begin{array}{l}	~\\\mbox{complex surfaces}	\end{array}"'] &                  & \begin{array}{l}	\partial\bar\partial-\\	\mbox{lemma}	\end{array} \arrow[lu, Leftrightarrow]                                               \end{tikzcd}\end{center}

\begin{que}
	Let $(X,\omega)$ be a compact symplectic $4$ manifold. Can we find a taming almost complex structure $J$ on $X$,  such that  the equality
	\[
	\tilde h^{1,0}=\tilde{h}^{0,1},\]
	holds?
\end{que}

%By the proof, one can find that the existent symplectic form is tamed by the almost complex structure $J$, under the condition $h^-_J=b^+-1$, we can find a compatible symplectic form, c.f. \cite{TWZZ22}.
%\begin{emp}	Let $X$ be a compact  oriented smooth $4$ manifold without boundary. Then, $X$ admits an almost complex structure if and only if there is a class $c\in H^2(X;\mathbb Z)$ such that 	\[c\equiv w_2(X) \bmod2\mbox{ and }	c^2=2\chi(X)+3\tau(X),\]	where $w_2(X)$ is the second Stiefel-Whitney class of $X$,  $\chi(X)$  and $\tau(X)$ are the Euler number and the signature of $X$ respectively. \end{emp}


In the study   of $4$ manifolds, to determine whether a smooth manifold is symplectic or not is a popular topic. One can show that any non-compact almost complex $4$ manifold admits a symplectic structure by using Gromov's $h$-principle \cite[Page 84]{Gromov86}. Hence, the interesting topic is about the case of  compact $4$ manifolds.  There are topological constraints   for a certain kind of compact almost complex $4$ manifolds admitting a  symplectic structure by using the Seiberg-Witten gauge theory, see  Bauer's work \cite{Bau06} and Li's work \cite{Li06}. On the other hand, the exotic phenomenon  in smooth $4$ manifolds implies that there are also non-topological obstructions for the existence of a symplectic structure on a compact almost complex $4$ manifold, see   Fintushel and Stern's work \cite{FS98} on exotic $K3$
surfaces and Taubes' work \cite{Taub94}, \cite{Taub95}. % On the case of non-compact  $4$ manifolds, Gromov \cite{Gromov86} showed the following:
%\begin{emp}    For a non-compact  manifold, every nondegenerate $2$-form  can be homotoped to a nondegenerate closed form   through nondegenerate forms. Moreover, one can specify the cohomology class of this  nondegenerate closed form   in advance. \end{emp}
%In this paper, we mainly focus on compact manifolds.
A consequence of Theorem \ref{thm-main-1} and Theorem \ref{thm-ddbar}  is a sufficient condition for a compact  oriented smooth  $4$ manifold admitting a symplectic structure.
%Theorem \ref{thm-main-1} and Theorem \ref{thm-ddbar} are  analogous to the case of compact complex surfaces.  It is natural to wonder: what is not analogous about the Dolbeault cohomology?
%We begin with the three identities in \eqref{eqn-three-identities}.
%By using the term $\hat h^{0,1}$, 

Similar to the case of compact complex manifolds, we give a Fr\"olicher type inequality on $b_1$, see Theorem \ref{thm-b1-inequality}. However, the equality can not hold in general, see the end of this paper.  %, which is similar to compact complex surfaces.
%\begin{thm}\label{thm-b1-inequality}	Let $(X,J)$ be a compact almost complex $4$ manifold. Then, it holds that $b_1\leq \tilde   h^{1,0}+\hat h^{0,1}\leq \tilde  h^{1,0}+  h^{0,1}$. 	Moreover, when the equality holds $b_1=\tilde h^{1,0}+\hat h^{0,1}$, 	we have that $\hat h^{0,1}=\dim_{\mathbb C}E^{0,1}_2$, where $E^{0,1}_2$ is the second stage of the Hodge filtration. \end{thm}Here, $\hat{h}^{0,1}$ is a new term, c.f., Definition \ref{defi-hat-Dolbeault}, which plays a crucial role in the proofs of Theorem \ref{thm-main-1} and Theorem \ref{thm-ddbar}.

%However, we can show that $\tilde{h}^{0,1}$, $\hat{h}^{0,1}$ and $\hat h^1$ are finite on compact almost complex $4$ manifolds, see Proposition \ref{prop-finite}. Such a result is quite different to the Dolbeault cohomology defined by Ciric and Wilson. 
% the Fr\"olicher-type inequality holds $$b_1\leq h^{1,0}+ \hat h^{0,1}\leq h^{1,0}+h^{0,1},$$	for compact almost complex manifolds, see Proposition \ref{thm-b1-inequality} and Definition \ref{defi-hat-Dolbeault} of $\hat h^{0,1}$.%

%Moreover, on compact almost complex $4$ manifolds, we show the following relations, c.f., Corollary \ref{cor-b1-ineq-4-mfld}:\begin{equation}	2\tilde{h}^{1,0}=2h^{1,0}\leq b_1 \mbox{ and }\tilde{h}^{1,0}\leq\tilde{h}^{0,1}\leq\hat{h}^{0,1}\leq h^{0,1}.	\label{eqn-b1-ineq-4mfld}\end{equation}  Our definition of the  refined Dolbeault cohomology can also be used to detect some difference between the compact complex surfaces and the general almost complex $4$ manifolds.
%see Lemma \ref{lemma-hodge-inequality} and Corollary \ref{cor-b1-ineq-4-mfld}. %  and also hold on the case of almost complex manifolds. 
%It is clear   that on compact complex surfaces  the equality \begin{equation}	b_1=h^{1,0}+h^{0,1},\label{eqn-b1-equality}\end{equation}always holds by using the Hodge decomposition and the Atiyah-Singer-Hirzebruch index theory, c.f., \cite[Theorem 2.7-Chapter IV]{BHPV}. By  definition,  the equalities $\tilde h^{0,1}=\hat{h}^{0,1}=h^{0,1}$ also trivially hold on  compact complex surfaces. Hence, it is natural to wonder  whether an analog of \eqref{eqn-b1-equality} in terms of $\tilde{h}^{1,0},~\tilde{h}^{0,1},~\hat{h}^{0,1},~h^{0,1}$ and $b_1$ holds for compact  almost complex $4$ manifolds. However, by Theorem \ref{thm-main-1} and Inequalities \eqref{eqn-b1-ineq-4mfld}, neither of  the equalities\begin{equation}	b_1=\tilde h^{1,0}+ \tilde h^{0,1},~b_1=\tilde h^{1,0}+ \hat h^{0,1},\mbox{ }b_1=\tilde  h^{1,0}+{h}^{0,1}, \label{eqn-b1-hodge-complex-surface}\end{equation}  holds on general compact  almost complex $4$  manifolds. Moreover, this implies that the three identities in \eqref{eqn-three-identities} are not equivalent to each other on compact almost complex $4$ manifolds. % and the equality \eqref{eqn-b1-hodge-complex-surface} does not hold, For example, the  manifold $(2k+1)\mathbb{C}P^2\#l\overline{\mathbb{C}P^2}$ with $k>0$,  $l\geq0$ and  any almost complex structure satisfies the relations $$b_1=2h^{1,0}=2\tilde{h}^{1,0}=0\mbox{ and  } \tilde h^{0,1}>0,$$hence violating all the equalities in \eqref{eqn-b1-hodge-complex-surface}. 
\noindent
In summary, we give a list of comparisons between the conditions for an almost complex $4$ manifold  admitting a  symplectic structure and the corresponding ones for a complex surface admitting a K\"ahler structure. 
\begin{center}
	\begin{tabular}{|c||c|}\hline  
		Compact    Complex  Surfaces& Compact Almost Complex $4$ Manifolds \\
		\hline
		\begin{tabular}{c}
			$b_1\equiv0\bmod2$  is equivalent  to  \\
			K\"ahler condition 
		\end{tabular}
		& \begin{tabular}{c}
			$b_1\equiv0\bmod2$  is unrelated to\\
			symplectic condition
		\end{tabular}   \\
		\hline
		\renewcommand{\arraystretch}{1.1}
		\begin{tabular}{c}
			$b_1=2h^{1,0}$   is equivalent  to  \\
			K\"ahler condition 
		\end{tabular}
		&\begin{tabular}{c}
			$b_1=2\tilde h^{1,0}=2h^{1,0}$  is unrelated to\\
			symplectic condition
		\end{tabular}\\\hline
		\renewcommand{\arraystretch}{1.1}
		\begin{tabular}{c}
			$h^{1,0}=h^{0,1}$  is equivalent  to  \\
			K\"ahler condition 
		\end{tabular}
		&\begin{tabular}{c}
			$\tilde h^{1,0}=\tilde h^{0,1}$  induces \\
			a symplectic structure
		\end{tabular} \\\hline
		\renewcommand{\arraystretch}{1.1}
		\begin{tabular}{c}
			$b_1=2h^{0,1}$  is equivalent  to  \\
			K\"ahler condition 
		\end{tabular} & \begin{tabular}{c}
			$b_1=\tilde h^{0,1}+\hat h^{0,1}$ induces\\
			a symplectic structure
		\end{tabular}
		\\\hline
		\renewcommand{\arraystretch}{1.1}
		\begin{tabular}{c}
			$\partial\bar\partial$-lemma  is equivalent  to  \\
			K\"ahler condition 
		\end{tabular} 
		&\begin{tabular}{c} generalized
			$\partial\bar\partial$-lemma induces  \\
			a symplectic structure
		\end{tabular}
		\\\hline
	\end{tabular}
\end{center}


%\begin{center}    \begin{tabular}{|c||c|}\hline    compact complex surfaces& compact almost complex $4$ manifolds\\\hline   {\small  \begin{tabular}{l|l|l}       $b_1$ even  & equivalent  & K\"ahler structure \\\hline       $b_1=2h^{1,0}$&equivalent  & K\"ahler structure \\\hline       $h^{1,0}=h^{0,1}$&equivalent  & K\"ahler structure \\\hline       $b_1=2h^{0,1}$&equivalent  & K\"ahler structure \\\hline       $\partial\bar\partial$-lemma&equivalent  & K\"ahler structure \\\hline    \end{tabular}    } & {\small \begin{tabular}{l|l|l}       $b_1$ even  & unrelated  & symplectic structure \\\hline       $b_1=2h^{1,0}$&unrelated & symplectic structure \\\hline       $\tilde h^{1,0}=\tilde h^{0,1}$&induces  & symplectic structure\\\hline       $b_1=\tilde h^{0,1}+\hat h^{0,1}$&induces  & symplectic structure\\\hline       $\partial\bar\partial$-lemma&induces  & symplectic structure \\\hline    \end{tabular} }       \end{tabular}\end{center}


%\noindent 	In the study   of $4$ manifolds, there are topological constrains   for a closed  oriented smooth $4$ manifold admitting a  symplectic structure by using the Seiberg-Witten gauge theory, e.g.,  Bauer's work \cite{Bau06} and Li's work \cite{Li06} about the $b^+$. On the other hand,    the exotic phenomenon in smooth $4$ manifolds implies that the existence of symplectic structure can not be a topological obstruction for almost complex $4$ manifolds. For example,  Fintushel and Stern \cite{FS98} proved that there are infinitely many exotic $K3$surfaces which do not admit any symplectic structure. The main purpose of this paper is  to  givr a sufficient condition for a compact  oriented smooth  $4$ manifold admitting a symplectic structure.
% In the case of almost complex $4$ manifolds, the non-integrablity of $\bar\partial$ raises a main difficulty to study this parallel problem coming from compact complex surfaces.   In this paper, we   use the refined  Dolbeault cohomology and the method of solving a certain partial differential equation    to give a  sufficient condition for a compact almost complex $4$ manifold admitting a symplectic structure. By importing two term $\hat{h}^1$ and $\hat h^{0,1}$, we establish the Fr\"olicher type inequality on $b_1$ and give the $\partial\bar\partial$-lemma. 
\noindent
This paper is organized as follows: In Section 2, we review the related differential operators on almost complex manifolds, and the Dolbeault cohomology defined by Cirici and Wilson \cite{CW21}; In Section 3, we give the definition of  the refined Dolbeault cohomology and discuss some   relations between  the refined Dolbeault cohomology and the ones defined by Cirici and Wilson, we give an example to help readers' understanding; In Subsection 4.1, we show  some analogous  relations about the dimensions of the Dolbeault cohomology groups  on compact almost complex $4$ manifolds and prove the Fr\"olicher inequality; In Subsection 4.2, we give the proofs to Theorem \ref{thm-main-1} and Theorem \ref{thm-ddbar}.
\begin{ack}
	The author would like to express his gratitude to  Xiangdong Yang for introducing the Dolbeault cohomology  on almost complex manifold  and sharing   the work \cite{CW21} and \cite{ST23}, to Feng Wang for sharing \cite{Gom95} and Masaya Kwamura for explaining his work and Gauduchon's result \cite{Gau77}. The author also thanks Hongyu Wang for the helpful discussion and suggestion.  Part of the writing of this paper was completed at  Zhuhai Campus of Sun Yat-sen University, the author also thanks Yuan Wei for the hospitality. This paper is partially supported by NSFC No. 12301061. 
\end{ack}

\section{Operators on almost complex manifolds}

In this section, we give some necessary notions on almost complex  manifold, and introduce the related differential operators on  almost complex manifolds. We begin from the definition of almost complex manifold. 
\begin{defi}
	For a smooth manifold $M$ of real dimension $2n$. If there exists  a smooth section $J\in\Gamma(End(TM))$ on $TM$  satisfying $J^2=-1$, then we call
	$(M,J)$ an almost complex manifold with almost complex structure $J$.
\end{defi}
A celebrated theorem of  Newlander and Nirenburg  \cite{NN} states that an almost structure comes from a complex structure if and only if the associated Nijenhuis tensor 
\[N(v,w)=\frac{1}{4}\left([Jv,Jw]-[v,w]-J[v,Jw]-J[Jv,w]\right),\]
vanishes, where $v,w$ are vector fields on $M$.
Let  $T_{\mathbb C}M=TM\otimes_{\mathbb R}\mathbb C$ be the complexified tangent bundle. One has the decomposition with respect to $J$,
\[T_{\mathbb C}M=T^{1,0}M\oplus T^{0,1}M,\]
where $T^{1,0}M$ and $T^{0,1}M$ are the eigenspaces of $J$ corresponding to eigenvalues $i$ and $-i$ respectively.
The almost complex structure $J$ acts on the cotangent bundle $T^* M$ by $J  \alpha(v)=$ $-\alpha(J v)$, where $\alpha$ is a $1$-form and $v$ a vector field on $M$. This $J$  action  can be extended to any $p$-form $\psi$ by 
$$(J\psi)\left(v_1, \cdots, v_p\right)=(-1)^p \psi\left(J v_1, \cdots, J v_p\right).$$
Similarly, we  decompose $T^*_{\mathbb C}M:=T^{*}M\otimes_{\mathbb R}\mathbb C$ as
$T^*_{\mathbb C}M=T^{*,(1,0)}_{\mathbb C}M\oplus T^{*,(0,1)}_{\mathbb C}M$.
%where $T'^*_{\mathbb C}M$ denote the eigenspace of $\sqrt{-1}$ and $T''^*_{\mathbb C}M$ denote the eigenspace of $-\sqrt{-1}$.
Denoting $ \mathcal A^*(M)$ by the space of sections of  $\bigwedge^*T^*_{\mathbb C}M$,   the decomposition holds:
\[ \mathcal A^r(M)=\bigoplus_{p+q=r} \mathcal A^{p,q}(M),\]
where $\mathcal{A}^{p,q}(M)$ denotes the space  of   forms of $\bigwedge^{p,q}T^*_{\mathbb C}M:=\bigwedge^pT^{*,(1,0)}_{\mathbb C}M\otimes\bigwedge^qT^{*,(0,1)}_{\mathbb C}M$.
By 
the formula \[d\psi\in   \mathcal{A}^{p+2,q-1}(M)+   \mathcal{A}^{p+1,q}(M)+
\mathcal{A}^{p,q+1}(M)+   \mathcal{A}^{p-1,q+2}(M),\]
for any $\psi\in   \mathcal{A}^{p,q}(M)$,
the deRham differential can be decomposed  as follows:
\[
d=\mu+\partial+\bar{\mu}+\bar{\partial} \mbox{ on } \mathcal{A}^{p,q},
\]
where   each  component is a derivation. The bi-degrees of the four components are  given by
\[|\mu|=(2,-1)
,~|\partial|=(1,0),~|\bar{\partial}|=(0,1),  ~|\bar{\mu}|=(-1,2).
\]
Observe that the operators $\partial,~\bar\partial$ are of the first order and the operators $\mu,~\bar\mu$   are of the zero order. 
Expanding  the relation $d^2=0$  gives the following: 
\begin{equation}
	\begin{aligned}
		\mu^{2} &=0, \\
		\mu \partial+\partial \mu &=0, \\
		\mu \bar{\partial}+\bar{\partial} \mu+\partial^{2} &=0, \\
		\mu \bar{\mu}+\partial \bar{\partial}+\bar{\partial} \partial+\bar{\mu} \mu &=0,\\
		\bar{\mu} \partial+\partial \bar{\mu}+\bar{\partial}^{2}&=0, \\
		\bar{\mu} \bar{\partial}+\bar{\partial} \bar{\mu}&=0, \\
		\bar{\mu}^{2}&=0.
	\end{aligned}\label{eqn-bar-operator}
\end{equation}
Similar to the case of Riemannian manifolds, we introduce the definitions of almost Hermitian manifold and almost K\"ahler manifold.
\begin{defi}
	A metric $g$   on an almost complex manifold $(M,J)$ is called    Hermitian, if $g$ is $J$-invariant, i.e., $g(J-,J-)=g(-,-)$.
	The imaginary component $\omega$ of $g$ defined by $\omega=g(J-,-)$, is a real non-degenerate $(1,1)$-form. 
	The quadruple   $(M,J,g,\omega)$ is called an \textit{almost Hermitian} manifold. 
	Moreover, when $\omega$ is $d$-closed, the quadruple   $(M,J,g,\omega)$ is called  an
	\textit{almost  K\"ahler} manifold. 
\end{defi}
% i.e. a Riemannian metric satisfying $g(J-,J-)=g(-,-)$,

We give some local calculation to help the understanding. 
Given an Hermitian metric $g$ on $(M,J)$, we set
$\{Z_r\}$ as a local $(1,0)$-frame with respect to  $g$ and  $\left\{\theta^r\right\}$ as the local associated coframe, i.e., $\theta^j(Z_k)=\delta^{j}_{k}$. Similar to the case of complex manifolds, we locally write $g_{k \bar j}=g\left(Z_k, \bar Z_j\right)$. The imaginary component $\omega$ of $g$   is  locally written as $\omega=i\sum_{k,j}g_{k \bar j}/2\theta^k \wedge \theta^{\bar{j}}$, c.f., \cite[Chapter 0-Section 6]{GH}. %Since $g$ is   Hermitian, one has $g_{i j}=g_{i j}=0$ and $g_{i j}=g_{j i}=\bar{g}_{i j}$.
The coefficients of   Nijenhuis tensor $N$ is   given by
$$
N\left(Z_{\bar{j}}, Z_{\bar{k}}\right)=-\left[Z_{\bar{j}}, Z_k\right]^{(1,0)}= \sum_t N_{j k}^t Z_t,~ N\left(Z_j, Z_k\right)=-\left[Z_j, Z_k\right]^{(0,1)}=\sum_t\overline{N_{jk}^t} Z_{\bar{t}},
$$
and   the structure coefficients of the Lie bracket is expressed as $$[Z_j,\bar Z_{k}]=\sum_k C^r_{j\bar k}Z_r+
\sum_lC^{\bar l}_{i\bar j}Z_{\bar l},$$
c.f.,   \cite[2.8 and 2.9]{Kaw22}. Hence, we get the local expression of the Nijenhuis tensor
$$
N=\frac{1}{2}\sum_{j,k,l} \overline{N_{l j}^k} Z_{\bar{k}} \otimes\left(\theta^l \wedge \theta^j\right)+\frac{1}{2}\sum_{j,k,l} N_{l j}^k Z_k \otimes\left(\theta^{\bar{l}} \wedge \theta^{\bar{j}}\right) .
$$
The actions of the operators $\bar\mu,~\bar\partial$ and $\partial$ on $\theta^s$ take the following form:
$$\bar\mu \theta^s=\frac12\sum_{k,j}N^s_{\bar k\bar j}\bar\theta^k\wedge\bar\theta^j,~\bar\partial \theta^s=-\sum_{k,j}C^{s}_{k\bar j}\theta^k\wedge \bar\theta^j,~
\partial \theta^s=-\frac{1}{2}\sum_{k,j}C^s_{kj}\theta^k\wedge\theta^j.$$
By  conjugation, one has the similar expressions for $\mu\bar\theta^s,~\partial\theta^s$ and $\bar\partial\bar\theta^s$. By \cite[Lemma 2.1]{CW21}, it holds that 
\[
\mu+\bar\mu=N\otimes(Id_{\mathbb{C}}).\]	

The   Hermitian metric $g$ induces  a unique $\mathbb{C}$-linear Hodge-$*$ operator 
$$
*: \Lambda^{p, q} \rightarrow \Lambda^{n-q, n-p},
$$
$\mbox{defined by }
g\left(\phi_1, \phi_2\right) d V=\phi_1 \wedge *\overline{ \phi_2}$, 
where $dV$ is the volume form $\frac{\omega^n}{n!}$ and $\phi_1, \phi_2 \in \Lambda^{p, q}$. For any two forms $\varphi_1\mbox{ and }\varphi_2$,  we define their  product  by, 
$$(\varphi_1,\varphi_2)=\int_M\varphi_1\wedge *\overline{\varphi_2}.$$  
Let  $\delta^*$ denote the formal-$L^2$-adjoint of $\delta$ for $\delta\in\{\mu,\partial,\bar\partial,\bar\mu,d\}$ with respect to the metric $g$. 
By the integration by part, one has the identities 
\[\delta^*=-*\bar\delta*  \mbox{ for }\delta\in\{\mu,\partial,\bar\partial,\bar\mu,d\}.\] 
We also define the  Lefschetz operator
$$
L: \mathcal{A}^{p, q} \longrightarrow \mathcal{A}^{p+1, q+1},$$ 
by $ L(\alpha):=\omega \wedge \alpha$ for any form $\alpha\in\mathcal{A}^{p,q}$.
Its adjoint operator is $\Lambda=L^*=\star^{-1} L \star$. It is well known that the triple $\{L, \Lambda,  [L, \Lambda]\}$ defines a representation of $\mathfrak{s l}(2, \mathbb{C})$ and there is a Lefschetz decomposition on complex $k$-forms
$$
\mathcal{A}^k=
\bigoplus_{j \geq 0} 
L^jP^{k-2j},
$$
where $P^j=\ker(\Lambda) \cap \mathcal{A}^j$, c.f., \cite[Section 6, Chapter 0]{GH}. 

On almost  K\"ahler manifolds, Cirici and Wilson proved the  Hard Lefschetz Duality \cite[Theorem 5.1]{CW20} and the following two results. 
% and $\mathcal{A}^{p,q}$ denotes the space of smooth $(p,q)$-forms. We also have the formula $\Delta_\delta=\ker(\delta)\cap\ker(\delta^*)$.  For almost K\"ahler manifold, one has the following identities. 

\begin{prop}[Cirici and Wilson {\cite[Proposition 3.1]{CW20}}]\label{prop-kahler-identities}
	On an   almost K\"ahler manifold $(M,J,g,\omega)$, i.e. $d\omega=0$, the following identities hold:
	\begin{itemize}
		\item[$(1)$]$[L, \bar{\mu}]=[L, \mu]=0$ and $\left[\Lambda, \bar{\mu}^*\right]=\left[\Lambda, \mu^*\right]=0$.
		\item[$(2)$] $[L, \bar{\partial}]=[L, \partial]=0$ and $\left[\Lambda, \bar{\partial}^*\right]=\left[\Lambda, \partial^*\right]=0$.
		\item[$(3)$]  $\left[L, \bar{\mu}^*\right]=i \mu,\left[L, \mu^*\right]=-i \bar{\mu}$ and $[\Lambda, \bar{\mu}]=i \mu^*,[\Lambda, \mu]=-i \bar{\mu}^*$.
		\item[$(4)$]  $\left[L, \bar{\partial}^*\right]=-i \partial,\left[L, \partial^*\right]=i \bar{\partial}$ and $[\Lambda, \bar{\partial}]=-i \partial^*,[\Lambda, \partial]=i \bar{\partial}^*$.
	\end{itemize}
\end{prop}

%	By taking the conjugates and duality,	the above identities imply the following isomorphism. 

\begin{thm}[Cirici and Wilson {\cite[Theorem 4.1]{CW20}}]\label{thm-CW-02}
	On any compact almost K\"ahler manifold with dimension $2 n$, it holds
	$$
	\mathcal{H}_d^{p, q}=\mathcal{H}_{\bar\partial}^{p, q} \cap \mathcal{H}_\mu^{p, q}=\mathcal{H}_{\partial}^{p, q} \cap \mathcal{H}_{\bar{\mu}}^{p, q},
	$$
	for all $(p, q)$. Moreover,
	we also have the following identities:
	\begin{itemize}
		\item[$(1)$](Complex conjugation)%. We get equalities
		$$
		\mathcal{H}_{\bar{\partial}}^{p, q} \cap \mathcal{H}_\mu^{p, q}=\mathcal{H}_{\bar{\partial}}^{q, p} \cap \mathcal{H}_{{\mu}}^{q, p} .
		$$
		\item[$(2)$]  (Hodge duality) %The Hodge-$*$ operator induces isomorphisms
		$$
		*: \mathcal{H}_{\bar{\partial}}^{p, q} \cap \mathcal{H}_\mu^{p, q} \rightarrow \mathcal{H}_{\bar{\partial}}^{n-q, n-p} \cap \mathcal{H}_\mu^{n-q, n-p} .
		$$
		\item[$(3)$] (Serre duality)%. There are isomorphisms
		$$
		\mathcal{H}_{\bar\partial}^{p, q} \cap \mathcal{H}_\mu^{p, q} \cong \mathcal{H}_{\bar\partial}^{n-p, n-q} \cap \mathcal{H}_\mu^{n-p, n-q} .
		$$
	\end{itemize}
\end{thm}

For the convenience of  later arguments in the next section, we set $$\mathcal{H}^{p,q}_{\delta,\delta'}:=\mathcal{H}^{p,q}_\delta\cap\mathcal{H}^{p,q}_{\delta'},$$
for $\delta,\delta'\in\{\partial,\bar\partial,\mu,\bar\mu\}$ and $\ell^{p,q}=\dim_{\mathbb C} \mathcal{H}^{p,q}_{\bar\partial,\mu}$. 
%Similar to the case of complex manifolds, we define an operator\[d^c:=i\frac{\bar\partial+\mu-\partial-\bar\mu}{2}.\]On  almost complex manifolds,  the following identities \cite{ST23} hold:\[d^2=0=(d^c)^2\mbox{ and }\frac{i}{2}(dd^c+d^cd)=\partial^2-\bar\partial^2.\]
%In particular, in the case of  dimension $4$, we have the following identity on $(1,1)$-forms,\begin{equation}	dd^c|_{\mathcal{A}^{1,1}}=i\partial\bar\partial|_{\athcal{A}^{1,1}}.	\label{eqn-partial-barpartial}\end{equation}
%Using the unitary coframe $\left\{\phi_i\right\}$, we can write out the $*$ operator directly. Let $\hat{\alpha}, \hat{\beta}$ be the ordered set of the complement multi-indices of $\alpha, \beta$ in $\{1,2, \cdots, n\}$. As $\omega=i \sum_{i=1}^n \phi_i \wedge \bar{\phi}_i$ and $d V=\frac{\omega^n}{n !}$, if we define on the basis$$*\left(\phi_\alpha \wedge \bar{\phi}_\beta\right)=2^{(p+q-n)}(-i)^n \epsilon_{\alpha \beta \hat{\beta} \hat{\alpha}} \phi_{\hat{\beta}} \wedge \bar{\phi}_{\hat{\alpha}}$$where $\epsilon_{\alpha \beta \hat{\beta} \hat{\alpha}}$ is the sign of permutation of$$\left(i_1, \cdots, i_p, j_1, \cdots, j_q, \hat{j}_1, \cdots, \hat{j}_{n-q}, \hat{i}_1, \cdots, \hat{i}_{n-p}\right) \rightarrow\left(1,1^{\prime}, 2,2^{\prime} \cdots, n, n^{\prime}\right).$$

\section{Cohomology groups on almost complex manifolds}
First, we review the Dolbeault cohomology $H^{*,*}_{Dol}$ defined by Cirici and Wilson on   almost complex manifolds. 

%\noindent
Recall that 
the  last three identities in \eqref{eqn-bar-operator} imply  the following identities:
\[
\bar{\mu} \bar{\partial}+\bar{\partial} \bar{\mu}=0, ~
\bar{\mu}^{2}=0, \mbox{ and }\bar \partial^2|_{\ker(\bar\mu)}\equiv0(\bmod ~ im(\bar\mu)). \]
These identities induce a  Fr\"olicher-type spectral sequence on $\mathcal{A}^{*,*}$. 
Hence, one can define the Dolbeault cohomology for an  almost  complex manifold
$(M,J)$.   % with bi-degree $(p,q)$. 
%To answer Hirzebruch's 20th. problem(\cite{Hir54}),  Cirici and Wilson defined the following.
 If there is no ambiguity, we omit the underline manifold and almost complex structure and  simply write  the cohomology as $H^{*,*}_{Dol}$
\begin{defi}[Cirici and Wilson {\cite[Definition 3.1]{CW21}}]
	The Dolbeault cohomology of an almost  complex $2n$-dimensional manifold
	$(M,J)$ is given  by
	\[
	H^{p,q}_{Dol}=H^q(H^{p,*}_{\bar\mu},\bar\partial)=
	\frac{\ker(\bar\partial:H^{p,q}_{\bar\mu}\to H^{p,q+1}_{\bar\mu})}
	{im(\bar\partial:H^{p,q-1}_{\bar\mu}\to H^{p,q}_{\bar\mu})},
	\]
	where $H^{p,q}_{\bar\mu}=\frac{\ker(\bar\mu:\mathcal{A}^{p,q}\to \mathcal{A}^{p-1,q+2})}{im(\bar\mu:\mathcal{A}^{p+1,q-2}\to \mathcal{A}^{p,q})}$.
\end{defi}
% of the above sequence \eqref{eqn-tilde-complex} by $\tilde H^{p,q}_{Dol}$. Similarly, we also  omit the manifolds and almost complex structures for the cohomology groups  $\tilde H^{p,q}_{Dol}$	and $H^{p,q}_{Dol}$. 	
Considering the presence of the operator $\bar\mu$,
Cirici and Wilson gave the definition of the Hodge-type filtration.

\begin{defi}[Cirici and Wilson {\cite[Definition 3.2]{CW21}}]
	The Hodge filtration of $\mathcal{A}^{*}$ of an almost complex manifold $(M,J)$ is given by the following decreasing filtration 
	$$
	F^p \mathcal{A}^n:=\operatorname{Ker}(\bar{\mu}) \cap \mathcal{A}^{p, n-p} \oplus \bigoplus_{i>p} \mathcal{A}^{i, n-i}.
	$$
\end{defi}
Cirici and Wilson  showed that the Dolbeault cohomology groups are isomorphic to the first page of the above Hodge filtration, and the spaces of  $(\Delta_{\bar\partial}+\Delta_{\bar\mu})$-harmonic forms  can be embedded into  the Dolbeault cohomology groups with respect to the same bi-degree. 

\begin{thm}[Cirici and Wilson  {\cite[Theorem 3.8 and Proposition 4.10]{CW21}}]\label{thm-CW-21}
	Let $(M,J)$ be  a $2n$-dimensional compact almost complex  manifold. One has the following:
	\begin{itemize}
		\item[$(1)$] \[
		H^{p,q}_{Dol}\cong E^{p,q}_1:=
		\frac{\{\omega\in\ker(\bar\mu)\cap\mathcal{A}^{p,q}|~\bar\partial\omega\in im(\bar\mu)\}}
		{\{im({\bar\partial})  \cap\ker(\bar\mu)+im(\bar\mu)\}\cap\mathcal{A}^{p,q}}\Rightarrow H^{p+q}_{dR}, ~0\leq p,q\leq n,
		\]
		where $E^{p,q}_1$ is the first stage of  the spectral sequence of the above Hodge filtration;
		\item[$(2)$] \[
		\ker(\Delta_{\bar\partial}+\Delta_{\bar\mu})\cap\mathcal{A}^{p,q}\subseteq H^{p,q}_{Dol} \mbox{ for any }p,~q,
		\]
		and the equality holds for $p\in\{0,...,n\}$ and $q\in\{0,n\}$.
		%\item[$(3)$] $H^{p,q}_{Dol}=E^{p,q}_1\Rightarrow H^{p+q}_{dR}$.
	\end{itemize}
\end{thm}
%Moreover, Cirici and Wilson also showed that $H^{p,q}_{Dol}=E^{p,q}_1\Rightarrow H^{p+q}_{dR}$

Inspired by the work of Sillari and  Tomassini \cite{ST23}, we define the certain subspaces of $\mathcal{A}^{*,*}$ and the associated Dolbeault type cohomology. Here, we also omit the underline manifold and the almost complex, just write $\tilde{H}^{*,*}_{Dol}$. 
\begin{defi}\label{defi-tilde-Dolbeault}
	On an almost complex manifold $(M,J)$,	we define the subspace $\mathcal{A}^{*,*}_{Dol}$ of $\mathcal{A}^{*,*}$ by
	\[
	\mathcal{A}^{*,*}_{Dol}=\ker(\mu)\cap \ker(\bar\mu)
	\cap \ker((\bar\partial+\mu)\bar\partial)\cap
	\mathcal{A}^{*,*}.
	\]
	We define the refined Dolbeault cohomology  $\tilde{H}^{*,*}_{Dol}$ by
	\[\tilde{H}^{*,*}_{Dol}=\frac{\ker(\bar\partial)\cap\mathcal{A}^{*,*}_{Dol}}{\bar\partial(\mathcal{A}^{*,*-1}_{Dol})}.\]
\end{defi}
One can check that the identities $\partial^2|_{\mathcal{A}^{*,*}_{Dol}}=0\mbox{ and }
\bar\partial^2|_{\mathcal{A}^{*,*}_{Dol}}=0$ hold, and $\tilde{H}^{*,*}_{Dol}$ is the cohomology of the complex sequence 
\begin{equation}
	\cdots\to \mathcal{A}^{p,q-1}_{Dol}\overset{\bar\partial}{\to}
	\mathcal{A}^{p,q}_{Dol}\overset{\bar\partial}{\to}\mathcal{A}^{p,q+1}_{Dol}\to\cdots. \label{eqn-tilde-complex}
\end{equation}
%is well-defined for all $p,q$.

\noindent
By observation, we have the following relations on compact almost Hermitian manifolds. 
\begin{lemma}   On a  compact almost Hermitian manifold $(M,J,g,\omega)$ with dimension $2n$, we have the following relations:
	\begin{itemize}
		\item[$(1)$] $\tilde H^{p,0}_{Dol}\cong H^{p,0}_{Dol}\cong \mathcal{H}^{p,0}_{\bar\partial,\bar\mu}$, for $0\leq p\leq n$.
		\item[$(2)$] $\mathcal{H}^{0,1}_{\bar\partial,\mu}\subseteq \tilde H^{0,1}_{Dol} \subseteq H^{0,1}_{Dol}$; 
		%	\item[$(3)$] $\tilde H^{0,n}_{Dol}\cong\mathcal{H}^{0,n}_{\bar\partial,\mu}\cong H^{0,n}_{Dol}$. 
	\end{itemize}
\end{lemma}
\begin{pf}  
	By the identification $$H^{p,q}_{Dol}\cong     \frac{\{\sigma\in\ker(\bar\mu)\cap\mathcal{A}^{p,q}|~\bar\partial\sigma\in im(\bar\mu)\}}    {\{im({\bar\partial})  \cap\ker(\bar\mu)+im(\bar\mu)\}\cap\mathcal{A}^{p,q}},$$
	the linear map $\alpha\in \ker(\mu)\cap\ker(\bar\mu)\cap \ker((\bar\partial+\mu)\bar\partial)\cap\ker(\bar\partial)\cap\mathcal{A}^{p,q} \mapsto [\alpha] \in H^{p,q}_{Dol}$   gives a well-defined  linear map $\tilde H^{p,q}_{Dol}\to H^{p,q}_{Dol}$ for each  $(p,q)$. 
	\begin{itemize}
		\item[$(1) $] The isomorphism $\tilde H^{p,0}_{Dol}\cong H^{p,0}_{Dol}$ follows from 
		the facts that the operators $\mu$ and $\bar\mu^*$ act  trivially on $\mathcal{A}^{p,0}$ and the intersection $im(\bar\partial)\cap\mathcal{A}^{p,0}$ is trivial.
		\item[$(2)$] When a class $[\alpha]\in\tilde H^{0,1}_{Dol}$ is trivial in $H^{0,1}_{Dol}$,  we write $\alpha=\bar\partial f$ for some function $f$. By the definition of $\alpha$,  one gets that $f\in\mathcal{A}^0_{Dol}$, i.e., $[\alpha]$ is trivial in $\tilde H^{0,1}_{Dol}$. Similarly, if a form $\alpha\in\mathcal{H}^{0,1}_{\bar\partial,\mu}$ is trivial in $\tilde H^{0,1}_{Dol}$, then $\alpha=\bar\partial f$.
		Applying the maximum principle to the equation $\bar\partial^*\bar\partial f=0$,   $f$ is a constant function and $\alpha=0$.
		%	\item[$(3)$] The proof follows from the facts		that $\ker(\bar\mu^*)\cap\mathcal{A}^{0,n}=\ker(\mu)\cap\mathcal{A}^{0,n}$, and the operators $\mu$, $\bar\partial^*$ act trivially on $\mathcal{A}^{n,0}$. 
	\end{itemize}	
	% By the identification $H^{p,q}_{Dol}\cong     \frac{\{\omega\in\ker(\bar\mu)\cap\mathcal{A}^{p,q}|~\bar\partial\omega\in im(\bar\mu)\}}    {\{im({\bar\partial})  \cap\ker(\bar\mu)+im(\bar\mu)\}\cap\mathcal{A}^{p,q}}$, the map $H^{p,q}_{\bar\partial,\mu}\to H^{p,q}_{Dol}$, induced by $\alpha\in \ker(\mu)\cap \ker((\bar\partial+\mu)\bar\partial)\cap\ker(\bar\partial)\cap\mathcal{A}^{p,q} \mapsto [\alpha] \in H^{p,q}_{Dol}$ is well-defined. Moreover, if $[\alpha] \in H^{p,q}_{Dol}$ is trivial, then we write $\alpha=\bar\partial \beta+\bar\mu\gamma$,  where $\beta\in\ker(\bar\mu)$. By the definition of $\alpha$, we have \[\mu(\bar\partial\beta+\bar\mu\gamma)=0,~\bar\partial(\bar\partial\beta+\bar\mu\gamma)=0,~\bar\mu\partial(\bar\partial\beta+\bar\mu\gamma)=0.\]
\end{pf}

\noindent
To compare the dimensions of  Dolbeault cohomology groups with the first Betti number,   we introduce the following  two terms.
\begin{defi}\label{defi-hat-Dolbeault}
	We define two terms $\hat{H}^1$ and $\hat{H}^{0,1}$ by
	\[\hat{H}^1=\frac{\{u_1+u_2\in\mathcal{A}^{1,0}\oplus\mathcal{A}^{0,1}\mid~
		\bar\mu(u_1)+\bar\partial(u_2) =0=\mu (u_2)+\partial (u_1)\}}
	{\{u'_1+u'_2\in( \partial(\mathcal{A}^0)\oplus \bar\partial(\mathcal{A}^0))\mid
		\bar\mu(u'_1)+\bar\partial(u'_2)=\mu (u'_2)+\partial (u'_1) =0	\}},\]
	%	\[0\to	\mathcal{A}^{0}\overset{d\partial\oplus\bar\partial}{\longrightarrow}	\mathcal{A}^{1,0}+\overset{d^{0,2}\oplus d^{2,0}}{\longrightarrow}\mathcal{A}^{0,2}\oplus\mathcal{A}^{2,0},	\]	whose middle cohomology is defined by	$\hat H^{1}:=\frac{\ker(d^{0,2}\oplus d^{2,0})\cap\mathcal{A}^{1}}{im(d)\cap \mathcal{A}^{1}}$.
	and  \[\hat H^{0,1}=\frac{\{u''\in\mathcal{A}^{0,1}|~d^{2,0}(u'+u'')=0=d^{0,2}(u'+u'')\mbox{ for some }u'\in\mathcal{A}^{1,0}\}}{im(\bar\partial)\cap\mathcal{A}^{0,1}}.\]
\end{defi} 
By  definition, it is clear that $\hat H^{0,1}\subseteq H^{0,1}_{Dol}$. For two functions $f_1$ and $f_2$, the condition $\bar\mu(\partial f_1)+\bar\partial(\partial f_2) =\mu (\partial f_2)+\partial (\partial f_1)=0$ is equivalent to the condition $\partial^2(f_1-f_2)=\bar\partial^2(f_1-f_2)=0$, which  trivially holds on complex manifolds. 
We denote their dimensions by 
\[\hat h^{0,1}=\dim_{\mathbb C} \hat H^{0,1},~\hat h^1=\dim_{\mathbb C}\hat H^1, \tilde h^{p,q}=\dim_{\mathbb C}\tilde H^{p,q}_{Dol} \mbox{ and }
h^{p,q}=\dim_{\mathbb C}  H^{p,q}_{Dol}.
\] 	

Now, we give a relation between $H^1_{dR}$ and $\hat{H}^1$ on compact almost complex manifolds. 
\begin{lemma}
	On a compact almost complex manifold, there is an inclusion \[
	H^{1}_{dR}\hookrightarrow \hat{H}^1.\]
\end{lemma}
\begin{pf}
	Notice that the composition of the  inclusion and the projection $$\ker(d)\cap(\mathcal{A}^{1,0}\oplus\mathcal{A}^{0,1})\to\ker(d^{2,0})\cap \ker(d^{0,2})\cap(\mathcal{A}^{1,0}\oplus\mathcal{A}^{0,1})\to\mathcal{A}^{0,1},$$ 
	gives a canonical map $H^{1}_{dR}\to \hat{H}^1$, where $d^{2,0}$ and $d^{0,2}$ are  $(2,0)$ and $(0,2)$-components of $d$ respectively. 
	
	It suffices to show that this map is an injection. 
	If  a class $[\alpha]\in H^{1}_{dR}$ maps to a trivial class in $\hat{H}^1$. Then, $\alpha=\partial f_1+\bar\partial f_2$ for   two functions $f_1$ and $f_2$, where $\alpha$ is a representative of this class. Since $d\alpha=0$, one gets that 
	\[\bar\partial\partial f_1+\partial\bar\partial f_2=0, \]
	hence $\partial\bar\partial(f_1-f_2)=0$ by the formula $\partial\bar\partial+\bar\partial\partial=0$ on functions. Applying the maximum principle to the equation $i\partial\bar\partial(f_1-f_2)=0$,  one obtains that $f_1-f_2$ is a constant. Therefore, we write $\alpha=\partial f_1+\bar\partial f_2=\partial f_1+\bar\partial f_1=df_1$, which is trivial in $H^{1}_{dR}$.
\end{pf}
%Recall the conjugate map $\Lambda^{1,0}\to \Lambda^{0,1}$ is an involution map  coming from the map $i\mapsto -i$, which is $\mathbb R$-linear the anti-$\mathbb C$ linear. Similarly, we define a map $J\mapsto -J$. This induces a $\mathbb{C}$-linear map $\Lambda^{0,1}\to \Lambda^{1,0}$, which is denoted by In particular, This involution map is formula on the space of complex-valued functions. 

%\noindent
For the convenience of the later application in the next section, we end this section by the following lemma.

\begin{lemma}\label{lemma-b1-h1}
	The  equality
	\begin{equation} \hat h^{1}= \hat h^{0,1}+\tilde h^{0,1},\label{eqn-b1-h1}
	\end{equation}	holds
	on any compact almost complex manifold.
\end{lemma}
\begin{pf}
	First, we define a linear map 
	\[
	\ker(d^{0,2})\cap\ker(d^{2,0})\cap\mathcal{A}^{1}\to 
	\mathcal{A}^{0,1},
	\]
	by $u'+u''\mapsto u''$,  where $u'\in\mathcal{A}^{1,0}$ and  $u''\in\mathcal{A}^{0,1}$. 
	Indeed, the above map induces a well-defined surjective linear map $$\pi:\hat H^1\to \hat H^{0,1},$$ 
	$\mbox{defined by }[u'+u'']\mapsto[u'']$. 
	
	If a class $[u'+u'']$ belongs to the kernel of $\pi$,  then
	$u''=\bar\partial f$ for some function $f$.  
	Since $[u'+\bar\partial f]\in\hat H^{1}$, one gets 
	\[ \bar\partial^2 f+\bar\mu u'=0\mbox{ and }\partial u'+\mu\bar\partial f=0.
	\]
	Substituting  the identities $\partial^2+\mu\bar\partial=0$ and $\bar\partial^2+\bar\mu\partial=0$ on functions into the above two equations gives that 
	\[u'-\partial f\in\ker(\bar\mu)\cap\ker(\partial),
	\]
	i.e., $[{\bar u'-\bar \partial \bar f}]\in \tilde H^{0,1}_{Dol}$.
	It is clear that $\overline{\tilde H^{0,1}_{Dol}}=\tilde{H}^{1,0}_{\partial}:=\frac{\{v\in\mathcal{A}^{1,0}\mid \partial(v)=0=\bar\mu(v)\}}{ \partial(\mathcal{A}^0_{D0l})\cap\mathcal{A}^{1,0}_{Dol}}$. 	On the other hand, we define a map
	\[
	\tilde{H}^{1,0}_{\partial}\to \hat H^1,\]
	$\mbox{by }
	[v]\mapsto[v+0]$. This map is an injection, because if a class  $[v]\in\tilde{H}^{1,0}_{\partial}$ maps to a trivial class in $\hat{H}^1$, then $v=\partial f$ with $\partial f\in\ker(\partial)\cap\ker(\bar\mu)$ by $\partial(v)=\bar\mu(v)=0$, which is trivial in $\tilde{H}^{1,0}_{\partial}$.
	Hence, for any class $[v]\in {\tilde H^{0,1}_{\partial}}$, the class $[ v+ u'+\bar\partial  f]$ belongs to $\ker(\pi)$. Therefore, we get the exact sequence
	\[0\to\overline{\tilde{H}^{0,1}_{Dol}}=\tilde{H}^{1,0}_{\partial}\to \hat H^1\to \hat H^{0,1}\to 0,
	\]
	i.e., $\hat h^{1}=\hat h^{0,1}+\tilde h^{0,1}$. 
\end{pf}

%Now, we show that these numbers $\hat h^1,\tilde{h}^{0,1}$ and $\hat{h}^{0,1}$ are of finite in dimension $4$. 
% Here, we give the following theorem. 
%\begin{prop}\label{prop-finite}	Let $(X,J)$ be a compact almost complex $4$ manifold. Then,  $\tilde{h}^{0,1}$ and $\hat{h}^{0,1}$ are finite. Moreover,  $\tilde{h}^{0,1}=h^{0,1}$ if and only if $(M,J)$ is a compact complex surface. \end{prop}
%In Section 3,  one will see the proof of the above theorem. 

%\begin{pf} %{\bf of Theorem \ref{thm-finite}}	By the direct calculus of  \cite[Appendix]{CW21}, it is known that $\hat{H}^{0,1}=E^{0,1}_2=E^{0,1}_\infty$ in real dimension $4$, which is finite by the first statement of Theorem \ref{thm-CW-21}. 	By the relation 	\[\tilde{H}^{0,1}_{Dol}\subseteq \hat H^{0,1},\]	and Lemma \ref{lemma-b1-h1}, $\tilde{h}^{0,1}$, $\hat{h}^{0,1}$ and $\hat{h}^1$ are finite. 	Since the condition $h^{0,1}<\infty$ is equivalent to  $(X,J)$ is a compact complex surface, this also gives that  $\tilde{H}^{0,1}_{Dol}=H^{0,1}_{Dol}$  is equivalent to  $(X,J)$ is a compact complex surface. Thus, we proved the theorem. \end{pf} 

Recall that on a compact almost complex $4$ manifold with a non-trivial almost complex structure, $H^{2,0}_{Dol} = H^{0,2}_{Dol}=0$ by Cirici and Wilson \cite{CW20}. 
We show  $\tilde{H}^{1,0}_{Dol}$ and $\tilde{H}^{0,1}_{Dol}$ are  trivial on  some compact almost complex manifolds with $\dim_{\mathbb R}\geq6$. 

\begin{lemma}\label{lemma-trivial}
	Let $(M,J)$ be a   compact   almost complex  $2n(\geq6)$-manifold   such that the associated  Nijenhuis tensor is not vanishing and takes the maximal rank at some point. Then, $\tilde{H}^{1,0}_{Dol}=\{0\}$. Moreover, if the Nijenhuis tensor takes the maximal rank anywhere, then $\tilde{H}^{0,1}_{Dol}=\{0\}$.  
\end{lemma}
\begin{pf}
	An easy   calculus yields $$rank_{\mathbb C}\mathcal{A}^{1,0}=rank_{\mathbb C}\mathcal{A}^{0,1}=n,~rank_{\mathbb C}\mathcal{A}^{2,0}=rank_{\mathbb C}\mathcal{A}^{0,2}=\frac{n(n-1)}{2}.$$ 
	For an Hermitian metric $g$, it is clear that $\bar\partial^* u=0$ and $u\in\ker(\bar\partial)$ is equivalent to $u\in\ker(\Delta_{\bar{\partial}})$ for any $u\in \mathcal{A}^{1,0}$. 
	Hence, all elements in $\tilde{H}^{1,0}$ are $\Delta_{\bar{\partial}}$-harmonic. Since $N\not\equiv0$, there exists a small open subset $U$ such that the restriction $N|_U$ is nowhere vanishing. 
	When $n\geq3$, it is clear that $\bar\mu|_U:\mathcal{A}^{1,0}(U)\to\mathcal{A}^{0,2}(U)$ is either  isomorphic($n=3$) or injective($n>3$) by the assumption on the maximal rank. The equation $\bar\mu(u)=0$ gives that $u|_U=0$ for any $u\in\mathcal{A}^{1,0}$.
	Hence,    the unique-continuity-property for harmonic forms shows that $u=0$ for any $u\in\tilde{H}^{1,0}_{Dol}$. 
	
	Similarly, 
	when $N$ is nontrivial and takes the maximal rank everywhere,  $\mu:\mathcal{A}^{0,1}\to\mathcal{A}^{2,0}$ is either  isomorphic($n=3$) or injective($n>3$). Therefore, $\mu(v)=0$ implies $v=0$ for any $v\in\mathcal{A}^{0,1}$, which proves the last statement. 
\end{pf}
\def\bp{\bar\theta}
% New results in almost Kahler

%By the definition of the refined Dolbeault cohomology, it is not clear whether $\Tilde{h}^{0,1}$ is finite or not. 
We end this section by giving an explicit calculation of a compact almost complex  manifold. 

\def\mbc{\mathbb{C}}
\begin{exa}\label{exa-kt}
	% Recall that the definition of the Kodaira-Thurston manifold and define a family of non-integrable almost complex structures on itself.
	The Kodaira-Thurston manifold $\mathrm{KT}^4$ is defined to be the  product manifold $S^1 \times  H_3(\mathbb{Z}) \backslash H_3(\mathbb R) $, where $H_3(\mathbb{R})\subset G L(3, \mathbb{R})$ denotes the Heisenberg group %$$H_3(\mathbb{R})=\left\{\left(\begin{array}{lll}1 & x & z \\0 & 1 & y \\0 & 0 & 1\end{array}\right) \in G L(3, \mathbb{R})\right\},$$
	and $H_3(\mathbb{Z})=H_3(\mathbb{R}) \cap G L(3, \mathbb{Z})$ acts on $H_3(\mathbb{R})$ by left multiplication. %We call and $H_3(\mathbb{Z}) \backslash H_3(\mathbb{R})$ the Heisenberg manifold. It is also useful to consider
	%The universal covering of 
	This manifold 
	%$\mathbb{R}^4$
	can be  given by identifying points in $\mathbb R^4$ with the relation
	$$(t,x,y,z)\sim (t+t_0 ,
	x+x_0,
	y+y_0,
	z+z_0+x_0 y),%\left(\begin{array}{l}t \\x \\y \\z\end{array}\right)\in \mathbb{R}^4\sim\left(\begin{array}{c}t+t_0 \\x+x_0 \\y+y_0 \\z+z_0+x_0 y\end{array}\right)\in \mathbb{R}^4
	$$
	for each element $(t_0, x_0, y_0, z_0) \in \mathbb{Z}^4$.
	Choose an almost complex structure given by
	\[J\frac{\partial}{\partial t}=\frac{\partial}{\partial x},~J\frac{\partial}{\partial x}=-\frac{\partial}{\partial t},~J(\frac{\partial}{\partial y}+x\frac{\partial}{\partial z})=\frac{\partial}{\partial z},~J\frac{\partial}{\partial z}=-(\frac{\partial}{\partial y}+x\frac{\partial}{\partial z}).\]
%	$$	J=\left(\begin{array}{cccc}		0 & -1 & 0 & 0 \\		1 & 0 & 0 & 0 \\		0 & 0 & 0 & -1 \\		0 & 0 & 1 & 0	\end{array}\right).	$$
	%acting on the basis. %, with $a, b \in \mathbb{R}, b \neq 0$ and $c=-\frac{a^2+1}{b}$. We can then
	The vector fields$$v_1=\frac{1}{2}\left(\frac{\partial}{\partial t}-i \frac{\partial}{\partial x}\right)\mbox{ and } v_2=\frac{1}{2}\left(\left(\frac{\partial}{\partial y}+x \frac{\partial}{\partial z}\right)-i\frac{\partial}{\partial z}\right),$$ span  $T^{1,0} M$ at each point. %, along with
	The  $1$-forms
	$$
	\theta_1=dt+i dx \mbox{ and }\theta_2=dy+i (dz-x dy) ,
	$$ are dual to $v_1$ and $v_2$ respectively. It is clear that %$Jv_l=iv_l$ for $l=1,2$ 
	$\theta_1$ and $\theta_2$ satisfy the structure equations
	$$
	\begin{gathered}
		d \theta_1=0,
		\partial \theta_2=-\frac{1}{4} (\theta_1 \wedge \theta_2),~\bar\partial\theta_2=-\frac{1}{4}(\theta_1 \wedge \bar{\theta}_2+\theta_2 \wedge \bar{\theta}_1),~\bar\mu\theta_2=\frac{1}{4}(\bar{\theta}_1 \wedge \bar{\theta}_2 ) .
	\end{gathered}
	$$
	$J$ has a compactible symplectic form $\omega=i/2(\theta_1\wedge\bar\theta_1+\theta
	_2\wedge\bar \theta_2)$, i.e. $(KT^4, g_J,J,\omega)$ is an almost K\"ahler $4$ manifold, where $g_J=\omega(J,)$. It is known that this symplectic manifold has no K\"ahlerian structure by $b_1=3$. 
	
	We start from $\tilde{H}^{0,1}_{Dol}$ to compute all  refined Hodge numbers. 
	For any $(0,1)$-form $u=f_1\bp_1+f_2\bp_2$ with smooth functions $f_1$ and $f_2$, the requirement $\mu(u)=0=\bar\partial(u)$ implies that $$f_2=0,~\bar\partial f_1\wedge\bp_1=-\frac{1}{2}\left(\left(\partial_y+x  \partial_z\right)f_1+i  \partial_zf_1\right)\bp_1\wedge\bp_2=0,$$
	where $\partial_y=\frac{\partial}{\partial y}$ and $\partial_z=\frac{\partial}{\partial z}$.
	%By the equation $v_2\bar v_2f=0$, $f$ depends only on $\{t,x\}$.
	On the other hand, for any smooth function $f$,we write $\partial f=f_1\theta_1+f_2\theta_2$, where $f_i=v_i(f)$ for $i=1,2$. The requirement $\bar\partial^2 f=0$ derives
	$v_2(f)=0$, and the requirement $\partial^2 f=0$ derives $\bar v_2f=0$. 
	We regard $f$ as a smooth bounded function on $\mathbb R^4$, still denoted by $f$. Setting $$f'(t,x,y,z)=f(t,x,y,xy+z),$$ the equation $\bar v_2f=0$ derives 
	\[\partial_y f'+i\partial_zf'=0. \]
	Namely, the function
	$f'_{t,x}(y,z):=f'(t,x,y,z)$ is  holomorphic and bounded for any $(t,x)$. This implies that $f'$ is independent of the coordinates $\{y,z\}$ by the Liouville theorem. In particular, it holds that $\partial_yf'=\partial_zf'=0$, i.e. $$\partial_y f=\partial_zf=0.$$ %Hence, $f$ descend to a smooth function depending only on $\{t,x\}$.
	Thus, the equation $\bar v_2f=0$ implies that $f$  depends only on  $\{t,x\}$.
	In fact, one can use the Fourier expansion(c.f. \cite[Proposition 3.1 and 3.2]{HZ20}) on $KT^4$ to obtain the same result.  
%	Since any smooth function on $KT^4$ raises a bounded smooth function on $\mathbb R^4$
	
%	  Recall that on $KT^4$, one has the Fourier expansion for functions, c.f., 	\cite[Proposition 3.1 and Proposition 3.2]{HZ20}, 	\[C^\infty(KT^4)=\widehat\bigoplus_{0\leq m<|n|}\mathcal H^{k, m,n}\oplus \widehat\bigoplus\mathcal{H}^{k,l,m,0}, \]	where	$$	\mathcal{H}^{k, m, n}= \{\sum_{\xi \in \mathbb{Z}} F(x+\xi) e^{2 \pi i(k t+(m+n \xi) y+n z)} \mid F \mbox{ is Schwarian } \}  ,	$$	$$	\mathcal{H}^{k, l, m, 0}:= \{G e^{2 \pi i(k t+l x+m y)} \mid G \in \mathbb{C} \}  ,	$$	and $\hat{\oplus}$ denotes the direct sum followed by the closure with respect to the $L^2$ norm. For a smooth function $f$, we write 	\[f=\sum_{m,n\neq0,k,0\leq m<|n|}\sum_{\xi\in\mathbb Z}a_{m,n,k}(x+\xi)e^{2i\pi(kt+(m+n\xi)y+nz)}+\sum_{m,l,k}b_{m,l,k}e^{2i\pi(mx+ly+kt)},\]	and 	\begin{eqnarray*}		-v_2\bar v_2f&=&4\pi^2		\sum_{m,n\neq0,k,0\leq m<|n|}\sum_{\xi\in\mathbb Z}		((m+n\xi)^2+x^2n^2+2(m+n\xi)xn+n^2)\\		&&		\cdot a_{m,n,k}(x+\xi)e^{2i\pi(kt+(m+n\xi)y+nz)}+4\pi^2\sum_{m,l,k}		l^2b_{m,l,k}e^{2i\pi(mx+ly+kt)}.	\end{eqnarray*}	
	
	Therefore,
	we reduce the calculation of the cohomology to the following quotient on the torus $T^2$, 
	\[\Tilde{H}^{0,1}_{Dol}\cong \frac{\{C^\infty(T^2)\}}{\{im(v_1)\}}.\]
	The right-hand-side part of the above identity has an inclusion 
	\[\frac{\{C^\infty(T^2)\}}{\{im(v_1)\}}\subseteq \frac{\{C^\infty(T^2)\}}{im(v_1\bar v_1)}.\]
	Since the operator $v_1\bar v_1=\frac{\partial^2}{\partial t^2}+\frac{\partial^2}{\partial x^2}$ is a formal self-adjoint elliptic operator on the torus, its cokernel is of $\dim=1$ by its index and the maximum-principle. Hence, the relations $0<\Tilde{h}^{1,0}\leq b_1/2$ and $\dim_\mbc\Tilde{H}^{0,1}_{Dol}\geq \Tilde{h}^{1,0}$ imply that $ \Tilde{h}^{0,1}=\tilde{h}^{1,0}=1$.
	
	To calculate $\Tilde{H}^{1,1}_{Dol}$, we write 
	\[w=a_1\theta_1\wedge\bar\theta_1+a_2\theta_2\wedge\bar\theta_2+b_1\theta_1\wedge \bar\theta_2+b_2\theta_2\wedge\bar\theta_1,\]for any $(1,1)$-form $w$.
	The equation $\bar\partial w=0$ gives 
	\[\bar v_2a_1\bar\theta_2\wedge \theta_1\wedge\bar\theta_1+\bar v_1a_2\bar\theta_1\wedge\theta_2\wedge \bar\theta_2+
	\bar v_1b_1 \bar \theta_1\wedge \theta_1\wedge \bar\theta_2-\frac{b_1}{4}\theta_1\wedge\bar\theta_1\wedge\bar\theta_2+\bar v_2b_2\bar\theta_2\wedge\theta_2\wedge\bar\theta_1-\frac{b_2}{4}\theta_1\wedge\bar\theta_2\wedge\bar\theta_1=0,\]
	i.e. \[\begin{cases}
		\bar v_2a_1-\bar v_1b_1-\frac{b_1}{4}+\frac{b_2}{4}=0,\\
		-\bar v_1a_2+\bar v_2b_2=0.
	\end{cases}\]
	Consider   $u=f_1\theta_1+f_2\theta_2\in\mathcal{A}^{1,0}\cap\ker(\bar\mu)\cap\ker(\bar\partial^2)$ for some functions $f_1$ and $f_2$. By the similar arguments, one has $f_2=0$ and 
	$ v_2f_1=0$, i.e.,  $f_1$ is a function depending only on $\{t,x\}$. % by $\bar v_2v_2=v_2\bar v_2$. %and the same arguments to $\tilde{h}^{0,1}$.
	This establishes the following inclusion
	\[
	\tilde{H}^{1,1}_{Dol}\supset
	\frac{\{a(t,x)\theta_1\wedge\bar\theta_1+b_1(t,x)\theta_1\wedge\bar\theta_2+b_2(t,x)\theta_2\wedge\bar\theta_1\mid ~b_2=b_1 +4\bar v_1b_1 \}}{\{im(\bar  v_1(C^\infty(T^2)))\theta_1\wedge\bar\theta_1\}},
	\]
	where the functions $a(t,x)$,  $b_1(t,x)$ and $b_2(t,x)$ depend only on $\{t,x\}$.
	It is clear that \[
	\dim_\mbc\frac{\{a(t,x)\theta_1\wedge\bar\theta_1+b_1(t,x)\theta_1\wedge\bar\theta_2+b_2(t,x)\theta_2\wedge\bar\theta_1\mid ~b_2=b_1+4\bar v_1b_1 \}}{\{im(\bar  v_1(C^\infty(T^2)))\theta_1\wedge\bar\theta_1\}}=\infty,
	\]
	which implies   $\tilde h^{1,1}=\infty$.
	
	To calculate $\Tilde{H}^{2,0}_{Dol}$ and $\tilde{H}^{0,2}_{Dol}$ is not hard. Since the requirement 
	$u=f\theta_1\wedge\theta_2\in\ker(\bar\mu)$ shows that $f\equiv0$, i.e. $\tilde h^{2,0}=0$. Similarly, one also has $\Tilde{h}^{0,2}=0$.  
	
	Now, we calculate $\tilde{H}^{2,1}_{Dol}$ and $\tilde{H}^{1,2}_{Dol}$.
	For any $w=a
	\theta_1\wedge\theta_2\wedge\bar\theta_1+b\theta_1\wedge\theta_2\wedge\bar\theta_2\in\ker(\bar\partial)$, one has $-\bar v_2a+\bar v_1b=0$. By the identity $\ker(\bar\mu)\cap \mathcal{A}^{2,0}=\{0\}$, one has 
	\[\Tilde{H}^{2,1}_{Dol}\supset 
	\{w=a(t,x)\theta_1\wedge\theta_2\wedge\bar\theta_1\mid a\mbox{ depends only on }(t,x)\}, \]
	i.e., $\Tilde{h}^{2,1}=\infty$.
	
	\noindent 
	Since $\Tilde{H}^{1,2}_{Dol}=\frac{\mathcal{A}^{1,2}}{im(\bar\partial)}$, one has $\Tilde{H}^{1,2}_{Dol}\cong \ker(\bar\partial^*)\cap\mathcal{A}^{1,2}=\ker(\Delta_{\bar\partial})\cap\mathcal{A}^{1,2}$ by the Hodge decomposition of $\mathcal{A}^{1,2}$ with respect to $\Delta_{\bar\partial}$. Hence, we get 
	\[\Tilde{h}^{1,2}=\dim(\ker(\Delta_{\bar\partial}\cap\mathcal{A}^{1,2}))=\dim(\ker(\partial)\cap\mathcal{A}^{0,1})=\Tilde{h}^{1,0}=1.\]
	Similarly, one also has $\Tilde{h}^{2,2}=1$.
	Summarizing the above results, we write 
	\[\dim_\mbc\Tilde{H}^{*,*}_{Dol}(KT^4,J)=\begin{array}{ccccc}
		& &1&& \\
		&\infty &&1&\\
		0&&\infty&&0\\
		&1&&1&\\
		&&1&&
	\end{array}.\]
	
	%In other words, the $\partial\bar\partial$-lemma holds on the Kodira-Thurston manifold. 
	%Notice that $d \theta_2$ has a component with bidegree $(0,2)$. This demonstrates that the exterior derivative cannot be written as the sum $d=\partial+\bar{\partial}$ and therefore the almost complex structure $J_{a, b}$ is non-integrable for all $a \in \mathbb{R}, b \in \mathbb{R} \backslash\{0\}$.
\end{exa}


%\begin{prop}\label{prop-h(0,1)=h(1,0)-almost-Kahler-4}    Let $(X,g,J,\omega)$ be a compact almost K\"ahler $4$-manifold. Then, it holds that     \[\tilde{h}^{1,0}=\ell^{1,0}=\ell^{0,1}=\Tilde{h}^{0,1}.\]\end{prop}
%\begin{pf}    We use the $L^2$-method to show the proposition.     Set $H_1=\{f\in L^2_1(M)\mid~\mu\bar\partial f=\bar\mu\partial f=0\}$ as a closed subspace of $L^2_2$.     The continuity of $\mu\bar\partial:L^2_1(M)\to L^2(\mathcal{A}^{2,0})$ shows that $\mathcal{A}^0_{Dol}$ is dense in $H_1$.     Similarly, set $H_2=\{a\in L^2(\mathcal{A}^{0,1})\mid~\mu(a)=0\}$ as a closed subspace of $L^2(\mathcal{A}^{0,1})$  and $H_3=L^2(\mathcal{A}^{0,2})$.  $\mathcal{A}^{0,1}_{Dol}$ is also dense in $H_2$ by the continuity of $\mu:L^2(\mathcal{A}^{0,1})\to L^2(\mathcal{A}^{2,0})$. Consider the complex \[H_1\overset{T}{\longrightarrow}H_2\overset{S}{\longrightarrow}H_3,\]where $T=\bar\partial$ and $S=\bar\partial$, whose domains $D(T)$ and $D(S)$  are $\mathcal{A}^0_{Dol}$ and $\mathcal{A}^{0,1}$ respectively. It is clear that $S\comp T=0$ and $\frac{\ker(S)}{Im(T)}$ is of finite dimension by Lemma \ref{lemma-b1-h1}.  Using \cite[Proposition 2.4 of Section 2.1.2]{Oh18},   $T$ has a closed range, i.e., $im(T)=\overline{im(T)}$.  Hence, there is a representative $a$ in each class $\mathfrak{a}\in\tilde{H}^{0,1}_{Dol}$ such that $a\perp im(T)$, i.e.,   \[\int_X\alpha\wedge \bar*\partial f=0, \]  for any $f\in \mathcal{A}^0_{Dol}$.


%  On the other hand, for any $u\in \ker(\bar\partial)\cap\mathcal{A}^{0,1}_{Dol}$,   it is clear that $\partial^2u=0$.  The identity $[\Lambda,\partial]=i\bar\partial^*$ of Proposition \ref{prop-kahler-identities} derives  \[(\partial u)^+=\Lambda(\partial u)=i\bar\partial^* u.\]  We write $(\partial u)^+=f_u\omega$ for some smooth function $f_u$. \end{pf}
% By \cite[Lemma 5.4]{Taub96}, for each $p\in X$ there exists a neighborhood $U$ and a $J$-fiber-diffeomorphism $f:D\times D\to U$ such that $f(0\times D)$ and $f(D\times w)$ are pseudo-holomorphic discs for each $w\in D$, where $D\subset \mathbb C$ is a small disc.  For a pump function $\rho$ on $D$, consider $h(z,w)=(f^{-1})^*\rho(z)\rho(w)$, which is supported in $U$. We claim $h\in\mathcal{A}^0_{Dol}$ by the direct  calculation \begin{eqnarray*}  \bar\partial^2h=(\bar\partial^2\rho(z))\rho(w)+\rho(z) 	(\bar\partial^2\rho(w)) \end{eqnarray*} \begin{itemize} 	\item For all $w'\in D'$, $f(D'_{w'})$ is a $J_1$-holomorphic submanifold containing $(0,w')$. 	\item For all $w'\in D'$, $dist((\xi',w'),f'(\xi',w'))\leq z\cdot\rho'\cdot|\xi'|$.  	\item For all $w'\in D'$, the derivative of order $m$ of $f'$ are bounded by  	$z_m\cdot\rho'$. \end{itemize} In particular, all the discs $f'(D'_{w'})$ are transverse to $f(D_0)=f'(0\times D')=0\times D'$. 

\section{Almost complex 4 manifolds}

Throughout this section, all almost complex manifolds are compact and $4$ dimensional, unless otherwise specified.   Recall that for any Hermitian metric $g$ on an almost complex $4$ manifold, one has the identities, c.f., \cite[Lemma 2.1.57]{Don86},
\[\Lambda^+=\Lambda^{2,0}\oplus\Lambda^{0,2}\oplus\mathbb C\langle\omega\rangle,~\Lambda^-=\ker(L)\cap\Lambda^{1,1},\]
where $\omega$ is the imaginary part of $g$ and $\Lambda^+$ and $\Lambda^-$ are the $1$ and $-1$-eigenspaces of the Hodge-$*$ operator  with respect to $g$ respectively. 

\subsection{First Betti number on almost complex $4$ manifolds}
In this subsection, we show some inequalities about   the dimensions of Dolbeault cohomology groups  on compact almost complex $4$ manifolds, which are analogous to the case of compact complex surfaces. 




%First, we give some analogous properties to the case of compact complex surfaces. 
\begin{lemma}\label{lemma-barpartial}
Let $(X,J)$ be  a compact  almost complex $4$ manifold. Then, it holds that 
\[\ker(\bar\partial)\cap \mathcal{A}^{1,0}=\ker(d)\cap \mathcal{A}^{1,0}.\]
%	which shows that $\tilde{H}^{1,0}_{Dol}=H^{1,0}_{Dol}$.
\end{lemma}

\begin{pf}
For any form $\alpha\in \mathcal{A}^{1,0}(X)$, it si clear that
\[d\alpha=\partial\alpha+\bar\partial\alpha+\bar\mu\alpha.\]
The inclusion $\ker(d)\cap \mathcal{A}^{1,0}\subset \ker(\bar\partial)\cap \mathcal{A}^{1,0}$ is obvious.
Assume that $\bar\partial\alpha=0$. 
The  two identities $$*|_{\mathcal{A}^{2,0}}=Id \mbox{ and }*|_{\mathcal{A}^{0,2}}=Id,$$ 
shows that
\[d\alpha\wedge \overline{d\alpha}=\partial\alpha\wedge\overline{*\partial\alpha}+
\bar\mu\alpha\wedge\overline{*\bar\mu\alpha}
=|\partial\alpha|^2+|\bar\mu\alpha|^2.
\]
Integral the above formula   over $X$ gives that $\ker(\bar\partial)\cap \mathcal{A}^{1,0}\subset \ker(d)\cap \mathcal{A}^{1,0}$, which finishes the proof.
\end{pf}

Consequently, the above lemma and \cite[Lemma 2.3]{CW20} imply the following corollary.

\begin{cor}\label{cor-ell-h(1,0)}
On a compact almost complex $4$ manifold, it holds that
\[\ker(\Delta_{\bar\partial})\cap\ker(\Delta_{\bar\mu})
\cap\mathcal{A}
^{1,0}=\ker(\Delta_{\bar\partial}+\Delta_\mu)\cap \mathcal{A}^{1,0}=\ker(\bar\partial)\cap \mathcal{A}^{1,0}=\ker(\Delta_{\bar \partial})\cap\mathcal{A}^{1,0}.\]
\end{cor}

Similar to compact complex surfaces, one has the following lemmas. 


\begin{lemma}\label{lemma-h1}
It holds that   $$0\to\tilde  H^{1,0}_{Dol}=H^{1,0}_{Dol}\to\tilde  H^{0,1}_{Dol}\subseteq\hat{H}^{0,1}\subseteq H^{0,1}_{Dol},$$
on any compact almost complex $4$ manifold.
\end{lemma}

\begin{pf}
The inclusions $\tilde  H^{0,1}_{Dol}\subseteq \hat H^{0,1}\subseteq H^{0,1}_{Dol}$ follow from the definitions. We just show the first map is an injection.
If an  element $u\in\ker(\bar\partial)\cap \mathcal{A}^{1,0}$  can be written as $u=\partial f$ for some  function $f\in\mathcal{A}^0_{Dol}$, then   $\bar\partial \partial f=0$. By applying the maximum principle to the equation $i\partial\bar\partial f=0 $,    $f$ is a constant function and $u=0$.  Hence, the first map is injective by the conjugation.
\end{pf}

\begin{lemma}\label{lemma-b1-equiality}
Let $(X,J)$ be a compact almost complex $4$ manifold. Then, it holds that $b_1\geq 2\tilde h^{1,0}=2h^{1,0}$. % Moreover, if $(M,J,g)$ is a compact almost K\"ahler $4$ manifold, then we have that $b_1(X)=2h^{1,0}$.
\end{lemma}

\begin{pf}
It is clear that $\tilde H^{1,0}_{Dol}\cap \overline{\tilde H^{1,0}_{Dol}}=\{0\}$ and $\tilde H^{1,0}_{Dol}=\ker(d)\cap\mathcal{A}^{1,0}$. The  fact  $$\tilde H^{1,0}_{Dol}\cap im(\partial)=\tilde H^{1,0}_{Dol}\cap im(d)=\{0\},$$ 
implies  the    desired inequality.
\end{pf}


%In Section 3, we define the refined Dolbeault cohomology on almost complex manifolds.  On real dimension  $4$, we have the following relations among the dimensions of the Dolbeault and the refined Dolbeault  cohomology groups.
%Similar to the proofs of Lemma \ref{lemma-barpartial} and Proposition \ref{lemma-h1}, one has the following.

%\begin{lemma}\label{lemma-hodge-inequality}	It holds $\tilde h^{1,0}=h^{1,0}\leq \tilde h^{0,1}\leq\hat{h}^{0,1}\leq h^{0,1}$. \end{lemma}Here we skip the proof, since it is a consequence of Lemma \ref{lemma-h1}. This shows the inequality \eqref{eqn-tilde-Dolbeault-ineq}.
\noindent
Now, we consider the relations on the Dolbeault and the refined Dolbeault  cohomologies with bi-degrees $(2,0)$ and $(0,2)$.
First, we give the following lemma.

\begin{lemma}\label{lemma-(2,0)-trivial}
Let $(X,J)$ be  a compact almost complex $4$ manifold.
If  $\sigma\in H^{2,0}_{Dol}$ can be represented by $\sigma=\partial \alpha+\mu\beta$ for some $\alpha\in \mathcal{A}^{1,0}$ and $\beta\in\mathcal{A}^{0,1}$, then $\sigma=0$.
\end{lemma}

\begin{pf}
Clearly,  the formulas $$\bar\partial\alpha\wedge\bar\sigma=0,~\bar\mu\alpha\wedge\bar\sigma=0,$$ and
\[\bar\partial\beta\wedge\bar\sigma=0, ~\partial\beta\wedge\bar\sigma=0,\] 
imply that
$$\sigma\wedge\bar\sigma=d(\alpha+\beta)\wedge\bar\sigma.$$ By   the Stokes lemma and the formula $d\sigma=\bar\partial\sigma+\bar\mu\sigma=0$, one obtains
\[
\int_X|\sigma|^2=\int_X\sigma\wedge \bar \sigma=0,
\]
i.e.,  $\sigma=0$.
\end{pf}

%	The above induces a linear map	$H^{1,0}_{Dol}(X)\to H^{0,1}_{Dol}$.	
%\begin{lemma}		Let $(X,J)$ be  a compact almost complex manifold. We have an injective map  $$H^{1,0}_{Dol}(X)\to H^{0,1}_{Dol},$$		by taking the conjugate.	\end{lemma}	\begin{pf}		Clearly, the conjugate map is well-defined.		Suppose that there exists $		\alpha\in H^{1,0}_{Dol}$ such that $\alpha=\partial f$ for some function $f$. One has the  equation $\bar\partial\partial f=0$. By maximum principle of the operator $i\bar\partial\partial$, $f$ is a constant, i.e. $\alpha=0$. This implies that the conjugate map is injective.	\end{pf}
\noindent
%Together with  
The following lemmas  are  similar to the case of compact complex surfaces.

\begin{lemma}\label{lemma-iso-(2,0)-(0,2)}
The conjugation map $$H^{2,0}_{Dol}\to H^{0,2}_{Dol},$$
is an  isomorphism on a   compact almost complex $4$ manifold $(X,J,g,\omega)$.
\end{lemma}

\begin{pf}
%	Since solving  equation $\Delta_{\delta}\alpha=0$ is equivalent to solving $\delta\alpha=0$ and $\delta^*\alpha=0$ for any symbol $\delta\in\{\mu,\partial,\bar\partial,\bar\mu\}$ and any form $\alpha\in\mathcal{A}^{p,q}$.
Recall	the two isomorphisms in Theorem \ref{thm-CW-21} $$\ker(\Delta_{\bar\partial}+\Delta_{\bar\mu})\cap\mathcal{A}^{2,0}=H^{2,0}_{Dol}\mbox{ and }
\ker(\Delta_{\bar\partial}+\Delta_{\bar\mu})\cap\mathcal{A}^{0,2}\cong H^{0,2}_{Dol}.$$ 
By using % the identification $H^{0,2}_{Dol}\cong\ker(\Delta_{\bar{\partial}}+\Delta_{\bar \mu})\cap\mathcal{A}^{0,2}$ and
the facts that the operators $\bar\partial$ and $\bar\mu$ act trivially on $\mathcal{A}^{0,2}$,  
$\sigma\mbox{ belongs to }\ker(\Delta_{\bar{\partial}}+\Delta_{\bar \mu})\cap\mathcal{A}^{0,2}$
if and only if $\sigma\in\ker(\bar\partial\bar\partial^*+\bar\mu\bar\mu^*)\cap\mathcal{A}^{0,2}$. 
For any $\sigma\in\ker(\Delta_{\bar{\partial}}+\Delta_{\bar \mu})\cap\mathcal{A}^{0,2}$, we get
\begin{eqnarray*}
	0=\int_X(\bar\partial\bar\partial^*\sigma+\bar\mu\bar\mu^*\sigma)\wedge\bar\sigma
	=\int_X|\bar\partial^*\sigma|^2+|\bar\mu^*\sigma|^2
	=\int_X|\partial\sigma|^2+|\mu\sigma|^2,
\end{eqnarray*}
where we use the self-duality of $\mathcal{A}^{0,2}$ for the last equality. On the other hand, the inclusion $$\ker(\bar\partial^*)\cap\ker(\bar\mu^*)\cap\mathcal{A}^{0,2}=\ker(\partial)\cap\ker(\mu)\cap\mathcal{A}^{0,2}\subseteq\ker(\Delta_{\bar{\partial}}+\Delta_{\bar \mu})\cap\mathcal{A}^{0,2},$$ 
is clear. 
Hence, by $\ker(\Delta_{\bar{\partial}})\cap\mathcal{A}^{0,2}=\ker(\partial)\cap\mathcal{A}^{0,2}$, 
$\ker(\Delta_{\mu})\cap\mathcal{A}^{0,2}=\ker(\mu)\cap\mathcal{A}^{0,2}$ and \cite[Lemma 2.3]{CW20}, the space of $(\Delta_{\bar{\partial}}+\Delta_{\bar \mu})$-harmonic $(0,2)$-forms is equal to the space $\mathcal{H}^{0,2}_{\bar\partial,\mu}$. 
%	c.f., \cite[Lemma 2.3]{CW20} and, one has   that

By using the above arguments  to $(2,0)$-forms, % and the formulas in \cite[Lemma 2.3]{CW20}, 
it also holds 
$$H^{2,0}_{Dol}=\ker(\Delta_{\bar\partial}+\Delta_{\bar\mu})\cap\mathcal{A}^{2,0}=\ker(\bar\partial)\cap\ker(\bar\mu)\cap\mathcal{A}^{2,0}.$$
%	for any $(2,0)$-form $\sigma$ belongs to $\mathcal{H}^{2,0}_{\bar\partial,\bar\mu}$ if and only if $\bar\partial\sigma=0$ and $\bar\mu\sigma=0$. This implies the identities	$$H^{2,0}_{Dol}\cong	\ker(\Delta_{\bar\partial})\cap \ker(\Delta_{\bar\mu})\cap\mathcal{A}^{2,0}\cong \ker(\Delta_{\partial})\cap \ker(\Delta_{\mu})\cap\mathcal{A}^{2,0}.$$
Thus, the conjugation
map
\[H^{2,0}_{Dol}\cong\ker( {\bar\partial})\cap \ker( {\bar\mu})\cap\mathcal{A}^{2,0}\to
\ker( {\partial})\cap \ker( {\mu})\cap\mathcal{A}^{0,2}
\cong H^{0,2}_{Dol},\]
is an isomorphism. This proves the lemma.
\end{pf}




\begin{lemma}
It holds that 
\[H^{2,0}_{Dol}=\tilde H^{2,0}_{Dol}=\mathcal{H}^{2,0}_{\bar\partial,\mu}
\cong\mathcal{H}^{0,2}_{\bar\partial,\mu}=H^{0,2}_{Dol}\subseteq \tilde{H}^{0,2}_{Dol},\]
on any compact almost Hermitian $4$ manifold.
\end{lemma}
\begin{pf}
%Clearly, $\partial$ acts trivially on $\mathcal{A}^{2,0}$ and $\bar\partial$ acts trivially on $\mathcal{A}^{0,2}$. 	The  identities  \[\ker(\bar\partial^*)\cap\mathcal{A}^{2,0}=\ker(\partial)\cap\mathcal{A}^{2,0}\mbox{ and }\ker(\Delta_{\bar\mu})\cap\mathcal{A}^{2,0}=\ker(\Delta_\mu)\cap\mathcal{A}^{2,0}=\ker(\bar\mu),	\]	show that $\mathcal{H}^{2,0}_{\bar\partial,\mu}=\ker(\bar\partial)\cap\ker(\bar\mu)\cap\mathcal{A}^{2,0}=\tilde H^{2,0}_{Dol}=H^{2,0}_{Dol}$. Similarly, we also get $\mathcal{H}^{0,2}_{\bar\partial,\mu}=\ker(\partial)\cap\ker(\mu)\cap\mathcal{A}^{0,2}$, which implies the isomorphism 	$\mathcal{H}^{2,0}_{\bar\partial,\mu}	\cong\mathcal{H}^{0,2}_{\bar\partial,\mu}$. Now, we show the last inequality. 
By the first paragraph of the proof to Lemma \ref{lemma-iso-(2,0)-(0,2)}, one obtains 
$H^{0,2}_{Dol}\cong \ker(\partial)\cap\ker(\mu)\cap\mathcal{A}^{0,2}=\mathcal{H}^{0,2}_{\bar\partial,\mu}$. Similarly, we also get $\mathcal{H}^{2,0}_{\bar\partial,\mu}=\ker(\bar\partial)\cap\ker(\bar\mu)\cap\mathcal{A}^{2,0}$, which implies the isomorphism 
$\mathcal{H}^{2,0}_{\bar\partial,\mu}
\cong\mathcal{H}^{0,2}_{\bar\partial,\mu}$. This also gives a well-defined map 
$$H^{0,2}_{Dol}\cong\mathcal{H}^{0,2}_{\bar{\partial},\mu}\to \tilde{H}^{0,2}_{Dol}.$$ 
For any $\bar{\partial},\mu$-harmonic $(0,2)$-form $\sigma$, if it is trivial in $\tilde{H}^{0,2}_{Dol}$, then  $\sigma=\bar\partial\alpha$ for some $(0,1)$-form $\alpha$.  By taking the conjugation on $\sigma$ and using Lemma \ref{lemma-(2,0)-trivial}, we get $\sigma=0$, i.e., $\mathcal{H}^{0,2}_{\bar{\partial},\mu}\to \tilde{H}^{0,2}_{Dol}$ is an injection. % m On the other hand, for any representative $\alpha$ of a class $[\alpha]\in\Tilde{H}^{0,2}_{Dol}$, one gets $\mu(\alpha)$
%	By the identification $H^{0,2}_{Dol}\cong\ker(\Delta_{\bar{\partial}}+\Delta_{\bar \mu})\cap\mathcal{A}^{0,2}$ and the facts $\bar\partial$, $\bar\mu$ act trivially on $\mathcal{A}^{0,2}$, it holds that 
\end{pf}

\noindent
We summarize the results of the above lemmas by the following corollary.

\begin{cor}\label{coro-betti-inequality}
Let $(X,J,g,\omega)$ be  a compact almost Hermitian $4$ manifold.
The following inequalities establish
\begin{itemize}
	\item[$(1)$] $\ell^{1,0}= h^{1,0}=\tilde h^{1,0}\leq \tilde h^{0,1} \leq \hat h^{0,1}\leq h^{0,1}$ and $ \ell^{0,1}  \leq \tilde h^{0,1} \leq\hat h^{0,1}\leq h^{0,1}$.
	\item[$(2)$] $\ell^{2,0}=h^{2,0}=\tilde{h}^{2,0}=\ell^{0,2}=
	h^{0,2}\leq\tilde{h}^{0,2}$.
\end{itemize}
\end{cor}
The first inequalities are the consequences of Lemma \ref{lemma-h1}, which prove \eqref{eqn-tilde-Dolbeault-ineq}. Here we only give the proof for $\ell^{0,1}\leq\tilde h^{0,1}$. 

\begin{pf}
We define  
$$\mathcal{H}^{0,1}_{\bar\partial}\cap \mathcal{H}^{0,1}_{\mu}\to\tilde H^{0,1}_{Dol},$$ 
by $u\mapsto [u]$. 
When $u$ maps to a trivial class, we write $u=\bar\partial f$ for some function $f\in\mathcal{A}^0_{Dol}$. Applying the maximum principle to the  equation $\bar\partial^*\bar\partial f=0$,   $f$ is a constant function and $u=0$.  
\end{pf}

\noindent
Now we show the relation between $b_1$ and $\tilde h^{1,0}+\hat h^{0,1}$. 

\begin{thm}\label{thm-b1-inequality}
	Let $(X,J)$ be a compact almost complex $4$ manifold. Then, it holds that $b_1\leq \tilde   h^{1,0}+\hat h^{0,1}\leq \tilde  h^{1,0}+  h^{0,1}$. 
	Moreover, when the equality holds $b_1=\tilde h^{1,0}+\hat h^{0,1}$, 
	we have that $\hat h^{0,1}=\dim_{\mathbb C}E^{0,1}_2$, where $E^{0,1}_2$ is the second stage of the Hodge filtration. 
\end{thm}
%Here, $\hat{h}^{0,1}$ is a new term, c.f., Definition \ref{defi-hat-Dolbeault}, which plays a crucial role in the proofs of Theorem \ref{thm-main-1} and Theorem \ref{thm-ddbar}.
%However, we can show that $\tilde{h}^{0,1}$, $\hat{h}^{0,1}$ and $\hat h^1$ are finite on compact almost complex $4$ manifolds, see Proposition \ref{prop-finite}. Such a result is quite different to the Dolbeault cohomology defined by Ciric and Wilson. 
% the Fr\"olicher-type inequality holds $$b_1\leq h^{1,0}+ \hat h^{0,1}\leq h^{1,0}+h^{0,1},$$	for compact almost complex manifolds, see Proposition \ref{thm-b1-inequality} and Definition \ref{defi-hat-Dolbeault} of $\hat h^{0,1}$.%


%\begin{thm}\label{thm-b1-inequality}		Let $(X,J)$ be a closed almost complex  manifold. Then, it holds that $b_1(X)\leq h^{0,1}+\hat h^{0,1}\leq  h^{0,1}+  h^{0,1}$. 		Moreover, when the equality holds $b_1=h^{1,0}+h^{0,1}$, 		the differential map $\delta_1: H^{0,1}_{Dol}\to H^{1,1}_{Dol}$ is trivial. 	\end{thm}

\begin{pf} %\textbf{of Theorem \ref{thm-b1-inequality}} 
	By the relation $\hat H^{0,1}\subseteq H^{0,1}_{Dol}$, we only need to prove the first inequality
$
b_1\leq\tilde  h^{1,0}+\hat h^{0,1}. 
$
% We define a map $p :\frac{\ker(d^{0,2})\cap \mathcal{A}^1}{im(d)}\to H^{0,1}_{Dol}$ by setting each class $[u]\in\frac{\ker(d^{0,2})\cap \mathcal{A}^1}{im(d)}$ to its $(0,1)$-component class $[u^{0,1}]$, where $d^{0,2}=\pi^{0,2}\comp d$ is the $(0,2)$-component of the deRham differential. % is a homomorphism. If a class $[u'+u]$ maps to a trivial class $[u]$ in $H^{0,1}_{Dol}$, then $u=\bar\partial f$ .  One can check that $p$ is well-defined and surjective.  By  the deRham-Hodge decomposition, it holds $H^{0,1}_{Dol}\cong \left( \ker(d^*)\cap\ker(d^{0,2})\cap \mathcal{A}^{1}\right)$. For any $u'+u$ in $\ker(d^*)\cap \mathcal{A}^{1}\cap\ker(d^{0,2})$, we have that $\bar\partial u=0$. This implies that $im(\bar\partial )\cap im (\bar\mu)\cap\ker(d^*)=0$. By the identifications $H^{0,1}_{Dol}=\ker(d)\cap \ker(d^*)\cap\mathcal{A}^{1,0}$ and $H^1(X;\mathbb C)\cong \ker(d)\cap\ker(d^*)\cap\mathcal{A}^1$, there exists a short exact sequence  \[ 0\to H^{1,0}_{Dol}(X)\to H^1(X;\mathbb R)\to H^{0,1}_{Dol}(X), \] which yields the inequality. 
Similar to the first paragraph in the proof of Lemma \ref{lemma-b1-h1}, we define a linear map 
\begin{equation}
	\ker(d)\cap\mathcal{A}^1 \to \mathcal{A}^{0,1},\label{eqn-b1-map}
\end{equation} 
$ \mbox{by } u'+u''\mapsto u''$, where $u'\in\mathcal{A}^{1,0}$ and $u''\in\mathcal{A}^{0,1}$. 

One can check that the above map  \eqref{eqn-b1-map} induces a   well-defined map  $$p:H^1_{dR}\to \hat H^{0,1}.$$ 
If an image $[u'']$ is trivial in $\hat H^{0,1}$, then we write $u''=\bar\partial f$ for some function $f$. Combining the equation
\[0=d(u'+u'')=\partial u'+\mu u''+\bar\partial u''+\bar\mu u'+\bar\partial u'+\partial u'',\] 
for some $u'\in\mathcal{A}^{1,0}$ with the two identities $\bar\partial^2+\bar\mu\partial=0\mbox{ and }\partial\bar\partial+\bar\partial\partial=0$ on functions, we obtain the formulas
\[\partial(u'-\partial f)=0,~\bar\mu(u'-\partial f)=0\mbox{ and }\bar\partial (u'-\partial f)=0\]
i.e., $u'-\partial f\in\tilde  H^{1,0}_{Dol}=H^{1,0}_{Dol}$. On the other hand, we have   a linear injection 
\[0\to \tilde{H}^{1,0}_{Dol}=H^{1,0}_{Dol}\to H^1_{dR},\]
defined by $v\mapsto [v+0]$.
Hence, we establish the short exact sequence $$0\to\tilde  H^{1,0}_{Dol}=H^{1,0}_{Dol}\to H^1_{dR}\to \hat H^{0,1},$$
via the identification $\ker(p)\cong\tilde  H^{1,0}_{Dol}$, which yields the inequality.
%We set $p:H^1(X,\mathbb C)\to H^{0,1}_{Dol}$ by $[u'+u'']\mapsto [u]$, where $u'\in\mathcal{A}^{1,0}$ and $u''\in\mathcal{A}^{0,1}$. One can check that the map is well-defined. We   show that  $\ker(p)\cong H^{1,0}_{Dol}$.    If an image $[u'']$ is trivial, then we have $u''=\bar\partial f$ for some function $f$. We write  $0=du=\partial u'+\mu u''+\bar\partial u''+\bar\mu u'+\bar\partial u'+\partial u''$, where $u'\in\mathcal{A}^{1,0}$ and $u''\in\mathcal{A}^{0,1}$. By the identities $\bar\partial^2+\bar\mu\partial=0$ and $\partial\bar\partial+\bar\partial\partial=0$ on $\mathcal{A}^0$, one deduces that  $\bar\mu(u'-\partial f)=0$ and $\bar\partial (u'-\partial f)=0$. This gives $u'-\partial f\in H^{1,0}_{Dol}$, and establishes the short exact sequence $$0\to H^{1,0}_{Dol}(X)\to H^1(X,\mathbb C)\to H^{0,1}_{Dol},$$ via the identification $\ker(p)\cong H^{1,0}_{Dol}$, which gives the inequality $b_1\leq h^{1,0}+h^{0,1}$. 
%

By the same arguments of the above paragraph, one also obtains an exact sequence
$$0\to\tilde  H^{1,0}_{Dol}= H^{1,0}_{Dol}\to H^1_{dR}\overset{p'}{\to }H^{0,1}_{Dol}.$$
When the equality $b_1=\tilde h^{1,0}+h^{0,1}$ holds, one gets   the exact sequence 
\begin{equation}
	0\to\tilde  H^{1,0}_{Dol}=H^{1,0}_{Dol}\to H^1_{dR}\to H^{0,1}_{Dol}\to0.
	\label{eqn-exact-b1}
\end{equation}
Recall that as in \cite[Appendix]{CW21} $E^{0,1}_2=E^{0,1}_\infty$ and $E^{1,0}_1=E^{1,0}_\infty$ for dimension $4$. By the equality $\dim E^{1,0}_\infty+\dim E^{0,1}_\infty=b_1$ of Theorem \ref{thm-CW-21}, one has the desired formula.
%$[w]\mapsto [\partial w+\bar\partial w_1]$, where $\bar\mu w_1+\bar\partial w=0$ for some $w_1$. 	By the exact sequence \eqref{eqn-exact-b1}, for any $[u]\in H^{0,1}_{Dol}$, there is some $v\in\mathcal{A}^{1,0}$ satisfying $d(u+v)=0$. This means, 	\[\partial u+\bar\partial v=0\mbox{ and }\bar\partial u+\bar\mu v=0,\]	for any  $[u]\in H^{0,1}_{Dol}$. 	Therefore, $\delta_1([u])=[\partial u+\bar\partial v]=0$ is trivial.
\end{pf}

\begin{rmk}
When $(X,J)$ is a compact  complex surface,  one can establish a similar inequality   by using the sheaf theory,  c.f., \cite[Lemma 2.4-Chapter IV]{BHPV}. %Proposition  \ref{thm-b1-inequality} just %and the following Theorem \ref{thm-betti-equality}  		can prove the inequality.  In the next section, we give an example that the equality does not hold in general. 
\end{rmk}%To show the equality holds, one    uses the index theory, c.f. \cite[Chapter IV-Theorem 2.7]{BHPV}. 
% by  using the deRham-Hodge decomposition. 

% On dimension $4$,  one has the estimate about the first Betti number.
By Theorem \ref{thm-b1-inequality}, we  have the following corollary.
\begin{cor}\label{cor-b1-ineq-4-mfld}
The inequalities 
\[2\tilde h^{1,0}=2h^{1,0}\leq b_1\leq \tilde h^{1,0}+\hat h^{0,1}\leq \tilde h^{1,0}+h^{0,1},\]
hold on any compact almost complex $4$ manifold. Moreover, one has the following: 
\begin{itemize}
	\item[$(1)$] The equality $\tilde h^{1,0}=\hat{h}^{0,1}$ implies 
	$
	b_1=\hat h^1=2\tilde h^{1,0}%+\hat h^{0,1}.
	$.
	\item[$(2)$]  The equality   $b_1=\hat h^1$ implies 
	$\tilde h^{1,0}=\tilde h^{0,1}.$
	\item[$(3)$] The equality  $\tilde h^{1,0}=h^{0,1}$ implies the identities  
	$h^{1,0}=\ell^{1,0}=\hat h^{0,1}=\tilde{h}^{0,1}.$
\end{itemize}

\end{cor}

\begin{pf}
The three inequalities follows from Theorem \ref{thm-b1-inequality} and Lemma \ref{lemma-b1-equiality}. 
\begin{itemize}
	\item[$(1)$]  The equality $\tilde h^{1,0}=\hat h^{0,1}$ is equivalent to the  condition $\tilde h^{1,0}=\tilde h^{0,1}=\hat{h}^{0,1}$.  Thus,  the result follows from the inequalities
	$$2\tilde h^{1,0}\leq b_1\leq \tilde h^{1,0}+\hat h^{0,1}\leq \tilde h^{0,1}+\hat h^{0,1}=\hat h^1.$$
	\item[$(2)  $]   Recall that there are  two inequalities
	$$b_1\leq\tilde h^{1,0}+\hat{h}^{0,1}\mbox{ and }\tilde h^{1,0}\leq \tilde{h}^{0,1}.$$ 
	Hence, the equality $b_1= \hat h^1=\tilde{h}^{0,1}+\hat{h}^{0,1}$ shows $	\tilde h^{1,0}=\tilde{h}^{0,1}$. 
	%Similarly, by using  	\[ \hat{h}^1=\tilde{h}^{0,1}+\hat{h}^{0,1} \mbox{ and }b_1\leq \hat h^1,\]	the equality $b_1=\hat h^{1}$ implies $	\tilde h^{1,0}=\tilde{h}^{0,1}$. 
	\item[$(3)$] It follows from Corollary \ref{coro-betti-inequality}.
\end{itemize}
\end{pf}
%	For any compact almst complex manifold, we have the following lemma.
%\begin{lemma}\label{lemma-(1,0)}  For any closed almost complex manifold $(M,J)$, we have the inequality $\mathcal{H}^{0,1}_{\bar\partial,\mu}\subseteq H^{0,1}_{Dol}$.\end{lemma}
%\begin{pf}  The formula $\ker(\Delta_{\bar\partial})\cap\ker(\Delta_{\bar\mu})\cap\mathcal{A}^{0,1}=\ker(\Delta_{\bar\partial})\cap\mathcal{A}^{0,1}$ implies that  $\ker(\Delta_{\bar\partial}+\Delta_\mu)\cap\mathcal{A}^{0,1}=\mathcal{H}_{\bar\partial}^{0, 1} \cap \mathcal{H}_\mu^{0, 1}\subseteq \ker(\Delta_{\bar\partial})\cap\ker(\Delta_{\bar\mu})\cap\mathcal{A}^{0,1}$, which proves the lemma.\end{pf}

%We set $h^{p,q}=\dim_{\mathbb C}H^{p,q}_{Dol}(X)$.

%\noindent
On compact almost K\"ahler manifolds, Cirici and Wilson \cite{CW20} gave  the equality $\ell^{p,q}=\ell^{q,p}$ by using  the relation $$\Delta_{\bar\partial}+\Delta_{\mu}=\Delta_\partial+\Delta_{\bar\mu},$$ and the  complex conjugate relation in Theorem \ref{thm-CW-02}. 
For the completeness of this paper, we  show the equality $\ell^{1,0}=\ell^{0,1}$ on compact almost K\"ahler $4$ manifolds via a simple method rather than using $\bar\partial,\mu$-Laplacian operator. 
\begin{lemma}
On a compact almost K\"ahler $4$ manifold $(X,J,g,\omega)$, % $(X,J,g,\omega)$, 
it holds that
$\tilde h^{1,0}=\ell^{1,0}=\ell^{0,1}$.
\end{lemma}
\begin{pf} 
The first equality is provided by Corollary \ref{cor-ell-h(1,0)}. We just show the second one here. The formula 	$[\Lambda, \bar{\partial}]=-i \partial^*$ in Proposition \ref{prop-kahler-identities} implies $\partial^*v=0$ for $v\in\tilde{H}^{1,0}_{Dol}$, which gives an injection $0\to\tilde{H}^{1,0}_{Dol}\to \mathcal{H}^{0,1}_{\bar\partial,\mu}$ by the conjugation map.
%	The formulas   $J u=*(\omega\wedge u)$ and 

Similarly, the formula $[\Lambda, \partial]=i \bar{\partial}^*$ in Proposition \ref{prop-kahler-identities} induces  that
$i\bar\partial^* u=\Lambda(\partial u)$ for any $u\in \mathcal{A}^{0,1}$. 
For any $u\in\mathcal{H}^{0,1}_{\bar\partial,\mu}$, the conditions $\bar\partial^*u=0$ and $\bar\partial u=\mu u=0$  imply that $d^+u=0$. Since on a compact oriented $4$ smooth manifold without boundary,  the formulas
\[0=\int_Xdu\wedge d\bar u=\int_Xd^+ u\wedge d^+\bar u+\int_Xd^-u\wedge d^-\bar u=\int_X|d^+u|^2-|d^-u|^2,\]
and $\|du\|^2_{L^2}=\int_X|d^+u|^2+|d^-u|^2$ hold.  %Similar to the  argument of Lemma \ref{lemma-(1,0)}, we have $u\cap im(\bar\partial)=0$, thus $\bar\partial^*u=d^*u=0$.
Hence,  we get that  $du=0$ and  the conjugation map $\tilde{H}^{1,0}_{Dol}\to \mathcal{H}^{0,1}_{\bar\partial,\mu}$ is an isomorphism. 
\end{pf}

%Let $(X,J)$ be a closed almost complex $4$ manifold. We define the map $H^1(X,\mathbb C)$ by $[u]\mapsto[u^{0,1}]$, where $u''$ denotes the $(0,1)$-component of $u$.

%One might check thatthe map is well-defined. 


%and equality.  % By using the filtered complex, we show that the   
%When the equality holds,  for any class $[u]\in H^{0,1}_{Dol}$, there exists $u'\in\mathcal{A}^{1,0}$ such that $d(u'+u)=0$, i.e.  $\bar\partial u+\bar\mu u'=0,~\partial u'+\mu u=0$ and $\partial u+\bar\partial u'=0$.we want to find a function $f$ such that $d(\partial f+u)=0$. This implies that $\bar\partial (u-\bar\partial f)=0$ and $\mu(u-\bar\partial f)=0$ and $\partial(u-\bar\partial f)=0$,


%\begin{thm}\label{thm-betti-equality}     Let $(X,J)$ be a closed almostcomplex $4$ manifold.  It holds that $b_1 =h^{1,0}+h^{0,1}$.\end{thm}\begin{pf}We  need to show that the map $p$ is subjective. Consider the   filtered complex, c.f. \cite[Appendix]{CW21}   $$F^i \mathcal{A}^p:=\operatorname{ker}(\bar{\mu}) \cap \mathcal{A}^{i, p-i}\oplus \bigoplus_{q>i} \mathcal{A}^{q, p-q} .$$ The  second stage of bi-degree-$(0,1)$ for the associated spectral sequenceis given by the quotient$$E_2^{0, 1} \cong \frac{\left\{u \in \mathcal{A}^{0, 1} |~\bar{\partial} u+\bar{\mu} u'=0, \partial u+\bar{\partial} u'=0,\mbox{ for some }u'\in\mathcal{A}^{1,0}\right\}}{im(\bar\partial)\cap \mathcal{A}^{0,1}} $$Therefore, for  a representative $u$ in  a random  class$[u]\inH^{0,1}_{Dol}\cong E^{0,1}_1$, by the map $d_1:E^{0,1}_1\toE^{0,1}_2$  there exists a form $u'+u\inker(d)\cap\mathcal{A}^1$.  \end{pf}

%When the equanlity $b_1=h^{1,0}+h^{0,1}$, one has the following lemma. 


%\vspace{4mm}
%\noindent When the condition $\ell^{0,1}=h^{0,1}$ holds, we have that each element in $ H^{0,1}_{Dol}$ has a $\bar\partial-\mu$-harmonic representative.  We show that $H^{0,1}_{Dol}\cong ker(d^{0,2})\cap\ker(d^*)\cap\mathcal{A}^1$, where $d^{0,2}=\pi^{0,2}\comp d$ denotes the composition of the projection andthe deRham differential operator.

% We define a map $ :\frac{\ker(d^{0,2})\cap \mathcal{A}^1}{im(d)}\to H^{0,1}_{Dol}$ by setting each class $[u]\in\frac{\ker(d^{0,2})\cap \mathcal{A}^1}{im(d)}$ to its $(0,1)$-component class $[u^{0,1}]$, where $d^{0,2}=\pi^{0,2}\comp d$ is the $(0,2)$ part of the deRham differential. % is a homomorphism. If a class $[u'+u]$ maps to a trivial class $[u]$ in $H^{0,1}_{Dol}$, then $u=\bar\partial f$ .  One can check that $p$ is well-dfined and surjective.  By  the deRham-Hodge decomposition, it holds $H^{0,1}_{Dol}\cong \left( \ker(d^*)\cap\ker(d^{0,2})\cap \mathcal{A}^{1}\right)^{0,2}$. For any $u'+u$ in $\ker(d^*)\cap \mathcal{A}^{1}\cap\ker(d^{0,2})$, we have that $\bar\partial u=0$. This implies that $im(\bar\partial )\cap im (\bar\mu)\cap\ker(d^*)=0$. By the identitications $H^{0,1}_{Dol}=\ker(d)\cap \ker(d^*)\cap\mathcal{A}^{1,0}$ and $H^1(X;\mathbb C)\cong \ker(d)\cap\ker(d^*)\cap\mathcal{A}^1$, there exists short exact sequence  \[ 0\to H^{1,0}_{Dol}(X)\to H^1(X;\mathbb R)\to H^{0,1}_{Dol}(X), \] which yields the inequality. 

%We set the cohomology\[\bar H^{0,1}=\frac{\{u\in\mathcal{A}^{0,1}|~d^{0,1}(v+u)=0=d^{2,0}(v+u)\mbox{ for some }v\in\mathcal{A}^{1,0}\}}{im(\bar\partial)\cap\mathcal{A}^{0,1}}. \]



%\noindent
Recall that
on a compact symplectic $4$ manifold, its first Betti number is not even  by Gompf's theorem, in other words the equality $b_1=2\tilde h^{1,0}$ does not hold for a general compact  almost K\"ahler $4$ manifold. At the end of this subsection, we give an analytic condition for a compact  almost K\"ahler $4$ manifold admitting an even first Betti number.

\begin{lemma}\label{lemma-b1-almost-kahler}
Suppose that   the inequality
\[
\lambda_1>4\|\Delta_{\bar\mu}|_{\mathcal{A}^1}\|,\]
holds on a compact almost K\"ahler $4$ manifold $(X,J,g,\omega)$,
where $\lambda_1$ denotes the first non-zero eigenvalue of the Hodge-Laplacian operator $\Delta_d$ restricted on $1$-forms and 
$\|\Delta_{\bar\mu}|_{\mathcal{A}^1}\|$ denotes the operator norm of $\Delta_{\bar \mu}:L^2(\mathcal{A}^1)\to L^2(\mathcal{A}^1)$.
Then,  $b_1=2\tilde h^{1,0}=2\ell^{1,0}=2\ell^{0,1}$.
\end{lemma}
\begin{pf}
%Since $\ker(\Delta_d)$ for $\alpha=\alpha'+\alpha''\in \ker(d^*)$, we have that $\Lambda(\bar\partial\alpha'-\partial \alpha'')=0$.
Recall that $\Delta_d$ is a real operator, and each real-valued $1$-form in $\mathcal{A}^1_{\mathbb R}$ can be written as $u+\bar u$ for a unique element $u\in\mathcal{A}^{1,0}$, where $\mathcal{A}^1_{\mathbb R}$ is the space of real-valued $1$-forms.  By Lemma \ref{lemma-b1-equiality}, it suffices to show that each element $u+\bar u\in\ker(d)\cap\ker(d^*)\cap\mathcal{A}^1_{\mathbb R}$ corresponds to a unique element $u\in\tilde H^{1,0}_{Dol}$ on $(X,J,g,\omega)$.  

Combining  the condition $d^*(u+\bar u)=0 ,$   
\mbox{with the K\"ahler identities  }$[\Lambda,\partial]=i\bar\partial^*\mbox{ and }[\Lambda,\bar\partial]=-i\partial^*$
in Proposition \ref{prop-kahler-identities}, one obtains  $\Lambda(\bar\partial u-\partial\bar u)=0$. 
Together with the equation  $(\bar\partial u+\partial \bar u)=0$, we get  
\begin{equation}
	\Lambda\bar\partial u=\Lambda\partial \bar u=0,\label{eqn-asd}
\end{equation}
i.e., $\partial\bar u$ is an imaginary-valued anti-self-dual $2$-form. 
%	On the other hand,   	the almost K\"ahler identities $\left[L, \bar{\partial}^*\right]=-i \partial,\left[L, \partial^*\right]=i \bar{\partial}$ in Proposition \ref{prop-kahler-identities}  deduce the following identities       \[    \omega\wedge \bar\partial^*\bar u-\bar\partial^*(\omega\wedge\bar  u)=-i\partial \bar u, ~\omega\wedge\partial^* u-\partial^*(\omega\wedge  u)=i\bar\partial   u.    \]    Since $\bar\partial^*\bar u=\Lambda(\partial\bar u)=0$,    Taking the sum of the above two  yields          \begin{eqnarray*}          	-i\partial\bar u+i\bar\partial u&=&        *\partial*(\omega\wedge\bar u)+*\bar\partial*(\omega\wedge  u)\\         	 &=&        i*(\partial \bar u)-i*(\bar\partial u),      \end{eqnarray*}       i.e., $(\partial\bar  u)^+$ is a real valued $2$ form.
We have the formula   
\begin{eqnarray*}       
	\int_X\partial u\wedge\bar\partial\bar u&=&        -\int_X\bar\partial\partial u\wedge\bar u\\        &=&-\int_X(\partial^2\bar u+\mu\bar\partial\bar u)\wedge \bar u\\       
	&=&\int_X\partial\bar u\wedge\partial\bar u+\int_X\bar\partial\bar u\wedge\mu\bar u\\        &=&\int_X\partial\bar u\wedge\partial\bar u-\int_X\bar\partial\bar u\wedge \partial u,   
\end{eqnarray*}   
where we use  the formula $\bar\partial\partial u=-\partial\bar\partial u-\bar\mu\mu u-\mu\bar\mu u=    \partial^2\bar u+\mu\bar\partial\bar u$ for the second equality and the formula $\partial u+\mu\bar u=0$ for the last one.   
The above formula and the anti-self-duality of $\bar\partial u$ yield 	\begin{equation}
	\|\bar\partial u\|^2_{L^2}=\int_X\partial\bar u\wedge*\bar\partial u=-\int_X\bar\partial u\wedge\partial\bar u=\int_X\partial\bar u\wedge\partial\bar u=2\|\partial u\|^2_{L^2}.\label{eqn-parital-d}
\end{equation}  
%By   the identity\[    \Delta_d=2\left(\Delta_{\bar{\partial}}+\Delta_\mu+\left[\bar{\mu}, \partial^*\right]+\left[\mu, \bar{\partial}^*\right]+\left[\partial, \bar{\partial}^*\right]+\left[\bar{\partial}, \partial^*\right]\right),    \]  in \cite[Proposition 3.4]{CW20}, 
On the other hand, we compute  
\begin{eqnarray*}     
	\int_X(\Delta_du, u)&=&\int_X(du, du) +(d^*u,d^*u) \\
	&=&\int_X(\bar\partial u,\bar\partial u)+(\partial u,\partial u)+(\bar\mu u,\bar\mu u)  = 4\int_X(\Delta_{\bar\mu} u,u), 
\end{eqnarray*}    
where we use the formulas $ d^*u=\partial^*u=0$, \eqref{eqn-asd} for the second equality and  the formulas $\bar\partial \bar u+\bar\mu u=0$, \eqref{eqn-parital-d} for the last one. Therefore, by the hypothesis of $\lambda_1$, it holds that $u\in\ker(d)\cap\mathcal{A}^{1,0}$, i.e., $u\in\tilde  H^{1,0}_{Dol}$.
\end{pf}

\subsection{Symplectic condition on almost complex $4$ manifolds}

%Recall that on almost complex $4$ manifolds, we have the formula \eqref{eqn-partial-barpartial}. %$$dd^c|_{\mathcal{A}^{1,1}}=i\partial\bar\partial|_{\mathcal{A}^{1,1}}.$$
First, we show that the equality $\tilde h^{1,0}=\tilde h^{0,1}$ is equivalent to the condition that each $dd^c$-closed $(1,1)$-form generates a unique  $d$-closed $2$-form. 
%For the later convenience, we verify the following  identities for almost complex $4$ manifolds.	
%On the other hand, by the Stokes theorem shows that    \begin{eqnarray*}        0=\int_Xdu\wedge d\bar u&=&        \int_Xd^+u\wedge d^+\bar u+\int_Xd^-u\wedge d^-\bar u\\        &=&\int_X(\partial u+\bar\mu u)\wedge(\bar\partial\bar u+\mu\bar u)+        \int_X\bar\partial u\wedge \partial\bar u\\        &=&2\|\partial u\|^2_{L^2}-\|\bar\partial u\|^2_{L^2},    \end{eqnarray*}    i.e $u\in H^{1,0}_{Dol}$, which finishes the proof.     the equation  $\partial u+\mu\bar u=0$, yields the identity \begin{eqnarray*}       \bar\partial\partial \bar u&=&-\partial \bar\partial\bar u-\mu\bar\mu\bar u-\bar\mu\mu \bar u       =\partial\bar\mu u-\bar\mu\mu\bar u=\partial \bar\mu u+\bar\mu\partial u=-\bar\partial^2 u.   \end{eqnarray*}   This induces that    \begin{eqnarray*}       -\int_X\bar\partial u\wedge \bar\partial u=\int_X\partial\bar u\wedge\bar\partial u=-\int_X\bar\partial\partial \bar u\wedge u       =\int_X\bar\partial^2 u\wedge u=-\int_X \bar\partial u\wedge \bar\partial u.   \end{eqnarray*}  Hence, it holds that $\|\bar\partial u\|^2_{L^2}=\int_X\bar\partial u\wedge\bar\partial u=0$, i.e $u\in H^{1,0}_{Dol}$.
%Let $f$ satisfy $\Delta_d f=\bar\partial^*u$. , i.e. $f=G d^*u$\begin{eqnarray*}    \bar\partial^2f=d\end{eqnarray*}	
%Since $\Delta_{\bar\partial}$ is an elliptic operator on a compact manifold,  it is well-known that $\Delta_{\bar\partial}$ is Fredholm and admits a Green-function $G_{\bar\partial}$, i.e.  outside the ${\bar\partial}$-harmonic forms,  $G_{\bar\partial}$ is inverse to $\Delta_{\bar\partial}$. Before, moving on, we show an elementary result in functional analysis. 	
%\begin{lemma}\label{lemma-commutator}    Let $(M,J,g,\omega)$ be a compact almost Hermitian manifold.     Under the above notations, we have that $\bar\partial G_{\bar\partial}=G_{\bar\partial}\bar\partial$     and     $G_{\bar\partial}\bar\partial^*=\bar\partial^*G_{\bar\partial}$. \end{lemma}	
%\begin{pf}    By the decomposition $\mathcal{A}^{p,q}=\mathcal{H}^{p,q}_{\bar\partial}\oplus (\mathcal{H}^{p,q}_{\bar\partial})^\perp$, it suffices to check the formula for any $\phi\not\in \mathcal{H}^{p,q}_{\bar\partial}$.     We  find a form $\alpha\in (\mathcal{H}^{p,q}_{\bar\partial})^{\perp}$ such that $\Delta_{\bar\partial}\alpha=\phi$. One has that     \[G\bar\partial\phi=G\bar\partial(\bar\partial\bar\partial^*+\bar\partial^*\bar\partial)\alpha=G\bar\partial^2\bar\partial^*\alpha+G\bar\partial\bar\partial^*\bar\partial\alpha=G(\bar\partial^2\bar\partial^*-\bar\partial^*\bar\partial^2)\alpha+\bar\partial\alpha=\bar\partial G\phi-G(\bar\partial^*\bar\partial^2f)    \]    \begin{eqnarray*}        \Delta_d|_{\mathcal{A}^{0,1}}=        (\partial+\bar\partial)(\bar\partial^*)+(\bar\partial^*+\partial^*+\mu^*+\bar\mu^*)(\mu+\bar\partial+\partial)\\        =\bar\partial\bar\partial^*+\partial\bar\partial^*+\bar\partial^*\bar\partial+\bar\partial^*\partial+\partial^*\mu+\partial^*\partial+\mu^*\mu+\bar\mu^*\bar\partial\\=\Delta_{\bar\partial}+\Delta_{\mu}+\Delta_{\partial}+(\partial\bar\partial^*+\bar\partial^*\partial+\bar\mu^*\bar\partial)    \end{eqnarray*}\end{pf}

%\begin{lemma}\label{lemma-1}   Let $(X,J,g)$ be a compact almost Hermitian   $4$ manifold equipped with a smooth positive $\bar{\partial} \partial$-closed $(1,1)$-form $\omega$. Then the operator $d^{1,1}: L^2_1(\mathcal{A}_{\mathbb{R}}^1 )  \rightarrow  L^2(\mathcal{A}_{\mathbb{R}}^{1,1}))$ has a closed range.\end{lemma}


%\begin{pf}    Let $\{v_i\}$ be a sequence in  $L_1^2(\mathcal{A}^1_{\mathbb{R}})$ such that the sequence $\{\psi_i=d^{1,1} v_i\}$ is converging in $L^2(\mathcal{A}^{1,1}_{\mathbb R})$ to some $\psi \in L^2(\mathcal{A}_{\mathbb{R}}^{1,1})$. Write $v_i=u_i+\bar{u}_i$ for some  $u_i\in\mathcal{A}^{0,1}$, so $\psi_i=\partial u_i+\bar{\partial} \bar{u}_i$. Without loss of generality, we assume  that $u_i$ is smooth for each $i$.  The Stokesi Lemma implies that $$\begin{aligned}\left\|\psi_i\right\|^2 & =\int_X\left(\Lambda \psi_i\right)^2 \omega^2-\int_X \psi_i^2 \\& =\int_X\left(\Lambda \psi_i\right)^2 \omega^2+2 \int_X \bar{\partial} u_i \wedge \partial \bar{u}_i=2\left\|\Lambda \psi_i\right\|^2+2\left\|\bar{\partial} u_i\right\|^2\end{aligned}$$iso it follows that $d v_i=d^{1,1} v_i+\bar{\partial} u_i+\partial \bar{u}_i$ is bounded in $L^2$.Let $\tilde{v}_i$ be the $L^2$ projection of $v_i$ perpendicular to the kernel of $d$, so $d^* \tilde{v}_i=0$ and $\tilde{v}_i$ is perpendicular to the harmonic 1-forms. Hence, there is a constant $C$ such that $\left\|\tilde{v}_i\right\|_{L_1^2} \leq C\left(\left\|d \tilde{v}_i\right\|+\left\|d^* \tilde{v}_i\right\|\right) \leq$ Const., so a subsequence of the sequence $\left\{\tilde{v}_i\right\}$ converges weakly in $L_1^2$ to some $\tilde{v} \in  L_1^2(\mathcal{A}_{\mathbb{R}}^1 )$. Since $d^{1,1} \tilde{v}_i=\psi_i$ it follows $d^{1,1} \tilde{v}=\psi$, proving the claim.\end{pf}  

%Let $d^{0,2}:\mathcal{A}^1_{\mathbb R}\to \mathcal{A}^{0,2}$ 

%Let $V$ be a real vector space and $W$ be a complex vector space. For a real linear map $F:V\to W$,  We set $V_c$ as the complexified vector space  and $F_c:V_c\to W$ as the complexified vector space of $F$. Then, we have  $\ker(F)_c\cong\ker(F_c)$.

%\begin{lemma}\label{lemma-2} Let $(X,J,g,\omega)$ be a compact almost Hermitian $4$ manifold.      If $\psi \in L^2(\mathcal{A}_{\mathbb{R}}^{1,1})$ is weakly $\bar{\partial} \partial$-closed. Then, there exists $u \in  L_1^2(\mathcal{A}^{0,1})$ such that $\psi+\partial u+\bar{\partial} \bar{u}$ is smooth.\end{lemma}

%\begin{pf}     Let ${\psi}'$ be the $L^2$ projection of $\psi$ perpendicular to the image of $d^{1,1}$, so ${\psi}'=\psi+\partial u+\bar{\partial} u$ for some $u \in  L_1^2(\mathcal{A}^{0,1})$ by Lemma \ref{lemma-1}. By Lemma \ref{thm-ddbar-zero},   $\psi'$ is also weakly $\partial\bar\partial$-closed.  The remaining arguments are parallelable     to \cite[Lemma 2]{Buch}. \end{pf}

% Then $\int_X {\psi}' \wedge *(\partial v+\bar{\partial} \bar{v})=0=\int_X \tilde{\psi} \wedge *(i \partial v-i \bar{\partial} v)$ for every $(0,1)$-form $v$, implying $\partial(* {\psi}')=0=\bar{\partial}(* {\psi})'$ weakly. Hence $\bar{\partial} \partial(* \tilde{\psi})=0=\bar{\partial} \partial \psi$ weakly, so $\bar{\partial} \partial((\Lambda \tilde{\psi}) \omega)=0$ in the sense of distributions. Standard regularity theorems imply that $\Lambda \tilde{\psi}$ is constant, and then the equations $d^*{\psi}'=0, d {\psi}'=(\Lambda {\psi}') d \omega$ together with elliptic regularity imply ${\psi}'$ is smooth.

%	By the parallelable argument as in \cite[Lemma 3]{Buch}, one obtains  the following lemma. 	\begin{lemma}		Let $(X,J,g)$ be a closed almost Hermitian $4$ manifold. 		If $\psi\in L^2(\mathcal{A}^{1,1}_{\mathbb R})$ is weakly $\partial\bar\partial$-closed, then there exists a sequence of  smooth $\partial\bar\partial$-closed real $(1,1)$-forms $\{\psi_i\}$ converging to $\psi$ in $L^2$ norm. 	\end{lemma}


%\begin{lemma}   Let $(X,J,g,\omega)$ be a closed almost Hermitian $4$ manifold satisfying $\bar\partial\partial\omega=0$.  If, $h^{1,0}=\ell^{0,1}$, then for any $d$-exact form  $\psi\in L^2(\mathcal{A}^{1,1})$, we have $\psi=\bar\partial\partial f$ for some function $f$. \end{lemma}

%\begin{pf}    We write $\psi=d\alpha$, for some $\alpha\in \mathcal{A}^1$. Then, the $(0,1)$-component $\alpha^{0,1}$ satisfies  $\bar\partial \alpha^{0,1}=0,~\mu \alpha^{0,1}=0$. The condition $h^{1,0}=\ell^{0,1}$, implies that there exists a function $f'$, such that $\alpha- \bar\partial f'\in \ker(d)$. Similarly, for the $(1,0)$-component $\alpha^{1,0}$, one can find a smooth function $f''$     such that $\alpha-\partial f''\in\ker(\bar\partial)$. Thus, $\psi=d\alpha=d(\alpha^{0,1}-\bar\partial f'+\alpha^{1,0}-\partial f'')+d(\bar\partial f'+\partial f'')=\partial\bar\partial (f'-f'')$, i.e., $f=f'-f''$ is the desired function. \end{pf}

%	\begin{lemma}\label{lemma-weak-partial}		Let $X$ be a compact complex surface, and let $\omega$ be its K\"ahler form. Suppose that 		$\psi\in L^2(\mathcal{A}^{1,1}_{\mathbb R})$ is a weakly $\partial\bar\partial$-closed. Then, we have the  inequality 		$(\int_X\omega\wedge\psi)^2\geq \int_X\psi^2\cdot \int_X\omega^2$. Moreover, the equality holds if and only if there exists a function satisfying $\psi=c\omega+i\bar\partial\partial f$ for some $c\in\mathbb R$. 	\end{lemma}	\begin{pf}		We define $c:=\int_X\psi\wedge \omega/(\int_X\omega^2)$. Assume that  $\psi$ is smooth. We consider the equation for  a smooth function $f$ such that 		\[\omega\wedge(\psi-c\omega+i\bar\partial\partial f)=0.\]		The equation is equivalent to $\Lambda\psi-2c-Pf=0$, where $P:=i\Lambda \bar\partial\partial$. The hypothesis on $\psi$ implies that 		$\Lambda\psi-2c\perp \ker(P^*)=const$. Hence, there exists  a smooth  function $f$ satisfying the above equation. 		\[\|\psi-c\omega -i\bar\partial\partial f\|^2=-\int_X\psi^2+c.\]		When the equality holds, we  choose a sequence of smooth $\partial\bar\partial$-		closed forms $\{\psi_j\}$ converging to $\psi$ in $L^2$. Let  $f_j$ satisfy		$\omega\wedge(\psi_j-c\omega+i\bar\partial\partial f_j)=0$, where $c_j$ is defined as above. 			\end{pf}

%\begin{lemma}\label{lemma-partial-equation}     Let $(X,J,g)$ be a closed almost Hermitian $4$ manifold. Then, $\ell^{1,0}=\ell^{0,1}$ if and only if for any $u\in\mathcal{A}^{0,1}$ satisfying $\bar\partial u=0,~\mu u=0$, the equation $\partial u=\partial\bar\partial f$ has a solution $f$. \end{lemma}
%\begin{pf}When the condition $h^{0,1}=\ell^{0,1}$ holds. % The deRham-Hodge decomposition shows that $\mathcal{A}^{0,1}\oplus\overline{\mathcal{A}^{0,1}}= \mathcal{A}^1_{\mathbb R}=im(d)\oplus im(d^*)\oplus Harm^1$, where $Harm^1$ denotes the  space of all $\Delta_d$-harmonic $1$-forms. For any $v\in\mathcal{A}^{0,1}$, we have that  $v+\bar v-df\in \ker(d^*)$  up a real-valued smooth function $f$, i.e. $d^*(v+\bar v-df)=\bar\partial^*(v-\bar\partial f)+\partial^*(\bar v -\partial f)=0$. Since $\int_X\bar\partial^*vdV=0$ for any  $v\in\mathcal{A}^{0,1}$, i.e. $\bar\partial^* v$ is vertical to the kernel of the adjoint operator for $\bar\partial^*\bar\partial$, we have that the equation $\bar\partial^*=\bar\partial^*\bar\partial$ is solvable for some smooth function $f$. It is clear that the equation $\bar\partial v=\partial\bar\partial f$ can be solved for some function $f$, if and only if $\int_X\bar\partial v\wedge\psi=0$ for any $\partial\bar\partial$-closed $(1,1)$ form $\psi$. By the Stokes lemma, we rewrite the above equation as \[ \int_X\bar\partial v\wedge\psi=-\int_Xv\wedge\bar\partial \psi =\int_X v\wedge\bar\partial\partial  u+\mu\bar\mu u=(\partial v+\bar\mu v,\overline{\bar\partial u+\bar\mu u})=0.\]This implies that each element in $\ker(\partial)\cap\ker(\bar\mu)$  is $\bar\partial$-closed up to some element in $im(\partial)$. Hence, $\ell^{0,1}=h^{1,0}$.     \end{pf}

%\noindent
For the $(2,0)$ and $(0,2)$-components of the form
$d(u+\bar u)=\partial u+\bar\partial \bar u+\mu u+\bar\mu \bar u+\bar\partial u+\partial \bar u$, one has  
$$d(\partial\bar u+\bar\partial u)=\bar\partial\partial \bar u+\partial\bar\partial  u+\bar\mu\partial\bar u+\mu\bar\partial u,$$
and
\[
d(\mu u+\bar\mu\bar u)=\bar\mu\mu u+\mu\bar\mu\bar u+\bar\partial \mu u+\partial\bar\mu\bar u
\]where $u\in\mathcal{A}^{0,1}$.
%\begin{eqnarray*}    d^2(\partial u+\bar\partial\bar u)=d(\partial u+\bar\partial \bar u+\mu u+\bar\mu \bar u+\bar\partial u+\partial \bar u)\\    =\partial^2u+\bar\partial\partial u+\partial\bar\partial \bar u+\bar\partial^2\bar u+\bar\partial\mu u+\bar\mu\mu u+\partial\bar \mu \bar u+\partial\bar\partial u+\mu\bar\partial u+\bar\partial\partial \bar u+\mu\partial\bar u\\    =(\partial^2+\bar\partial\mu +\mu\bar\partial)u+(\bar\partial^2+\partial\bar\mu+\bar\mu\partial)\bar u+    (\bar\partial\partial+)\end{eqnarray*}s
For a $dd^c$-closed real $(1,1)$-form $\psi$, we want to find some  $u\in\mathcal{A}^{0,1}$  satisfying the equation,
$$d\psi=d(\bar\partial u+\partial\bar u+\mu u+\bar\mu\bar u).$$ 
This is equivalent to solving $\bar\partial\psi=\partial\bar\partial u+\bar\mu\partial\bar u+\partial\bar\mu \bar u+\bar\mu\mu u$. The existence of such $u$   is provided by the following lemma.  

%The equation $\bar\partial\psi=\partial\bar\partial u+\bar\mu \partial\bar u$ has a solution $u\in\mathcal{A}^{0,1}$, if and only if $\bar\partial \psi$ is vertical to the kernel of the adjoint operator. 

\begin{lemma}\label{lemma-partial-solution}
Let $(X,J,g,\omega)$ be a compact almost Hermitian $4$ 
manifold. Then, the condition $\tilde h^{1,0}=\tilde h^{0,1}$ holds  if and only if for any $\psi\in\mathcal{A}^{1,1}_{\mathbb R}$ satisfying $\partial\bar\partial\psi=0$, there exists a solution  $u\in\mathcal{A}^{0,1}$ to the equation
\begin{equation}
	\bar\partial \psi=\partial\bar\partial u+\bar\mu\partial\bar u+\partial\bar\mu \bar u+\bar\mu\mu u,\label{eqn-ddbar}
\end{equation}
where $\mathcal{A}^{1,1}_{\mathbb R}$ denotes the space of real-valued $(1,1)$-forms.
\end{lemma}
%For the convenience of statement, we use the $dd^c$-closed condition. One can replace by  $\partial\bar\partial$-closed condition. 
\begin{pf}
% We  solve the equation  $\bar\partial\psi=\partial \bar\partial u+\bar\mu\mu u+\bar\mu\partial \bar u$ for some  $u\in\mathcal{A}^{0,1}$ with a given $\partial\bar\partial$-closed $\psi\in\mathcal{A}^{1,1}$. 
%We rewrite the right-hand side as $ \bar\partial\partial u+\bar\mu\partial u'=(\bar\partial+\bar\mu)\partial(u+u')$.
Consider the adjoint operator,
\begin{eqnarray*}
	(\partial\bar\partial +\bar\mu\mu+(\partial\bar\mu+\bar\mu\partial)\comp c)^*&=&*(\partial\bar\partial)*+*(\bar\mu\mu)*+c *(\bar\partial\mu+\mu\bar\partial)*\\
	&=&*((\partial\bar\partial)+(\bar\mu\mu)+c\comp  (\bar\partial\mu+\mu\bar\partial))*,
\end{eqnarray*}
where $c$ denotes the conjugation map.
%=*\bar\partial(\partial+\mu)*$  $((\bar\partial +\bar\mu)\comp\partial)^*=*(\bar\partial\partial)*+*\overline{(\bar\partial\mu)*}$. 
The equation has a solution if and only if the form $\bar\partial\psi$ is vertical to $\ker((\partial\bar\partial)*+(\bar\mu\mu)*+c\comp(\bar\partial\mu+\mu\bar\partial)*)$. 
This is equivalent to the condition that the formula
$\int_X  *\bar w\wedge \bar\partial\psi=0$
holds  
for all $w\in\mathcal{A}^{1,2}$ with the property $\partial\bar\partial*w+{\bar\mu\mu *w}+c\comp(\bar\partial\mu+\mu\bar\partial)*w=0$.
Setting $v=*  w$, 
we rewrite
\begin{equation}
	\partial\bar\partial v+\bar\mu\mu v+c\comp(\mu\bar\partial+\bar\partial\mu)v=0.\label{eqn-ddbar-c}
\end{equation}%satisfying  $\int_X\bar v\wedge \bar\partial \psi=0$.
By the Stokes lemma, the above equation \eqref{eqn-ddbar-c} implies the following:
\begin{eqnarray*}
	0&=&\int_X (\partial\bar\partial v+\bar\mu\mu v+c\comp(\mu\bar\partial+\bar\partial\mu)v)\wedge \bar v\\
	&=&
	\int_Xd(\bar\partial v+\mu v+\partial \bar v+\bar\mu\bar v)\wedge\bar v\\
	&=&-\int_X(\bar\partial v+\mu v+\partial \bar v+\bar\mu\bar v)\wedge d(\bar v)\\
	&=&-\int_X (\bar\partial v+\mu v+\partial \bar v+\bar\mu\bar v)\wedge 
	(\partial\bar v+\bar\mu\bar v)\\
	&=&\int_X-(\bar\partial v,\bar\partial v)-(\bar\mu \bar v,\bar\mu \bar v)-(\bar\partial  v,\bar\mu \bar v)-(\bar\mu \bar v,\bar\partial v)\\
	&=& 
	-\|\bar\partial v+\bar\mu\bar v\|^2_{L^2},
\end{eqnarray*}
i.e.,  $[v]\in\hat{H}^{0,1}$ and $[v+\bar v]\in \hat{H}^1$.
%$$\int_X (\partial\bar\partial v+\bar\mu\mu v+c\comp(\mu\bar\partial+\bar\partial\mu)v)\wedge \bar v=\|\bar\partial v+\bar\mu \bar v\|^2_{L^2}=0.$$ 
The condition $\tilde h^{1,0}=\tilde h^{0,1}$ establishes an  exact sequence $$0\to\tilde  H^{1,0}_{\partial}\cong\tilde{H}^{1,0}_{Dol}\to\hat H^1\to\hat H^{0,1}\to0.$$
Recall that $d^{2,0}$ and $d^{0,2}$ denote the $(2,0)$ and $(0,2)$ components of $im(d)$ respectively. 
By the above exact sequence and the condition $v+\bar v\in \ker(d^{0,2})\cap\ker(d^{2,0})$, there are two functions $f$ and $f'$ with the property $$d^{2,0}(\partial f+\bar\partial f')=\partial^2(f'-f'')=0,~d^{0,2}(\partial f+\bar\partial f')=\bar\partial^2(f''-f')=0,$$
such that
the $(1,0)$-component of $v+\bar v-\partial f-\bar\partial f'$ belongs to $\tilde{H}^{1,0}_{Dol}=\ker(d)\cap\mathcal{A}^{1,0}$, i.e., $\bar v-\partial  f\in\ker(d)$.
%exists a smooth function $f$.  $\bar\partial f\in(\mathcal{H}^{0,1}_{\bar\partial,\mu})^\perp\cap\ker(\bar\partial)\cap\ker(\mu)$ 
The equation $dd^c\psi=i\partial\bar\partial\psi=0$  implies that
\[\int_X\partial f\wedge\bar\partial\psi=-\int_Xf\wedge\partial\bar\partial\psi=0.\]
Therefore, it holds that
\[\int_X \bar v \wedge \bar\partial \psi%=\int_X(\bar v-\partial f)\wedge\bar\partial \psi
=\int_Xd(\bar v-\partial f) \wedge \psi=0,\]
i.e., $\int_X *\bar w\wedge \bar\partial\psi=0$. 
%Recall that $(\bar\partial+\bar\mu)^2=0$ on $\mathcal{A}^1$



Conversely,  assume that for any $dd^c$-closed form $\psi\in\mathcal{A}^{1,1}_{\mathbb R}$  the equation $$\bar\partial \psi=\partial\bar\partial u+\bar\mu\partial\bar u+\bar\mu\partial \bar u+\bar\mu\mu u,$$
can be solved for some $u\in\mathcal{A}^{0,1}$. Recall that for any $\partial\bar\partial$-closed $(1,1)$-form $\psi$, it can be written as $\psi=\psi'+i\psi''$, where $\psi'$ and $i\psi''$ are the real  and the imaginary parts of $\psi$ respectively.
Since the operator $i\partial\bar\partial$ is real, the $\partial\bar\partial$-closeness of $\psi$ is equivalent to  those  of both $\psi'$ and $\psi''$.
Take the sum of the solutions $u'$ and $u''$ to the equations $$	\bar\partial \psi'=\partial\bar\partial u'+\bar\mu\partial\bar u'+\partial\bar\mu \bar u'+\bar\mu\mu u'\mbox{ and }	\bar\partial \psi''=\partial\bar\partial u''+\bar\mu\partial\bar u''+\partial\bar\mu \bar u''+\bar\mu\mu u''\mbox{ respectively}.$$ 
The assumption implies that for any $dd^c$-closed  $(1,1)$-form $\psi$, there are  solutions $u_1\in\mathcal{A}^{0,1}$ and $u_2\in\mathcal{A}^{1,0}$ to the equation 
\[\bar\partial\psi=\partial\bar\partial u_1+\bar\mu\mu u_1+\bar\mu\partial u_2+\partial\bar\mu u_2.\]
%	where  
We need to show that each solution to the equations $\partial v=0$ and $\bar\mu v=0$ for $v\in\mathcal{A}^{1,0}_{Dol}$ generates a unique form in $\ker(\bar\partial)\cap\mathcal{A}^{1,0}$. %up to an element in $im(\bar\partial)$. 
Similar to the first paragraph, 
the equation $$\bar\partial v=\bar\partial\partial f,$$
can be solved for some function $f$, if and only if the formula $\int_X \bar\partial v\wedge \psi=0$ holds for any $dd^c$-closed $(1,1)$-form $\psi$. 
Again by the Stokes lemma,   the above formula is rewritten as follows:
\begin{eqnarray*}
	\int_X\bar\partial v\wedge \psi&=&\int_X v\wedge\bar\partial \psi\\
	&=&\int_X v\wedge(\partial\bar\partial u_1+\partial\bar\mu u_2+\bar\mu\partial u_2+\bar\mu\mu u_1)\\
	&=&\int_X
	(\partial v\wedge ({\bar\partial u_1+\bar\mu u_2})+\bar\mu v\wedge(\partial u_2+\mu u_1))=0.
\end{eqnarray*}
Consequently,   by the conjugation map,  each element in $\ker(\bar\partial)\cap\ker(\mu)\cap\mathcal{A}^{0,1}_{Dol}$  is $\partial$-closed up to some element in $im(\bar\partial)\cap\mathcal{A}^{0,1}_{Dol}$, i.e.,  $\tilde h^{1,0}=\tilde{h}^{0,1}$. 
\end{pf}

\begin{rmk}
For the case of compact complex surfaces, i.e. $\mu=\bar\mu=0$, Buchdahl \cite[Lemma 8]{Buch} considered the    equation for complex-valued $(1,1)$-form $\psi$.
\end{rmk}
%\begin{lemma}    Under the above notations,  suppose that $(X,J,g,\omega)$ is a compact almost K\"ahler  $4$ manifold. Then,  it holds that ${H}^{1}_{dR}\cong\bar H^1$.\end{lemma}
%\begin{pf}
% The idea is similar to prove  Proposition \ref{thm-b1-almost-kahler}.   Since the integral of the function $\bar\partial^* u$ is zero, we can find a function $f$, such that for a representative form $u$ of the  class satisfying $\bar\partial^*u=0$. By the   almost K\"ahler identities $(4)$ in Proposition \ref{prop-kahler-identities}, $\partial u$ is anti-self-dual.   The equations $\bar\partial u+\bar\mu v=0$ and $\partial v+\mu u=0$ imply that   \begin{eqnarray*}      \int_X\bar\partial u\wedge \partial\bar u&=&-\int_X\partial\bar\partial u\wedge \bar u=\int_X\partial\bar\partial\bar u\wedge  u-\int_X\bar\mu\partial v\wedge\bar u\\      &=&-\int_X\bar\partial\bar u\wedge \partial u+\int_X\partial v\wedge\bar\mu \bar u\\      &=&      -\int_X\partial u\wedge \bar\partial\bar u      -\int_X\partial v\wedge \bar\partial \bar v,  \end{eqnarray*}  where we use  the identity $\partial\bar\partial  u=-\bar\partial\partial\bar u-\mu\bar\mu u-\bar\mu\mu u=-\bar\partial\partial\bar u +\bar\mu\partial v$ for the second identity
% \begin{eqnarray*}      \int_X\partial\bar u\wedge \bar\partial u&=&-\int_X\bar\partial \partial\bar u\wedge u=\int_X\bar\partial\partial u\wedge \bar u-\int_X\bar\mu\partial v\wedge\bar u\\      &=&-\int_X\partial u\wedge \bar\partial\bar u+\int_X\partial v\wedge\bar\mu \bar u=      -\int_X\partial u\wedge \bar\partial\bar u      -\int_X\partial v\wedge \bar\partial \bar v,  \end{eqnarray*}  where we use the identity $\partial$
%  It suffices to show that $\bar H^1(X,\mathbb R)\cong H^1_{dR}(X,\mathbb R)$.    By  the deRham-Hodge decomposition $\mathcal{A}^{1}_{\mathbb R}=im(d)\oplus \ker(\Delta_d)\oplus im(d^*)$, we  have the identity    \[    \bar H^{1}\cong \ker(d^{0,2}\oplus d^{2,0})\cap\ker(d^*)\cap\mathcal{A}^{1}_{\mathbb R}.    \]    Recall that on an almost Hermitian $4$ manifold, we have the identity    \[    J^bu=*(\omega\wedge u),    \]    where $u$ is any form in $\mathcal{A}^{1}_{\mathbb C}$ and $J^b$ is the associated almost complex structure on the forms with respect to $J$.  Any element in $\bar H^{1}$ can be written as $u+\bar u$, for some $u\in \mathcal{A}^{0,1}$,  satisfying the equations     \[    \partial^*\bar u+\bar\partial^*u=0,~\bar\partial u+\bar\mu \bar u=0.    \]    By the almost K\"ahler identities $\left[L, \bar{\partial}^*\right]=-i \partial,\left[L, \partial^*\right]=i \bar{\partial}$ in Proposition \ref{prop-kahler-identities}, we have     \[    \omega\wedge \bar\partial^*u-\bar\partial^*(\omega\wedge u)=-i\partial u, ~\omega\wedge\partial^*\bar u-\partial^*(\omega\wedge \bar u)=i\bar\partial \bar u.    \]    To take the sum of the above two identities yields     \begin{eqnarray*}        -i\partial u+i\bar\partial\bar u&=&        *\partial*(\omega\wedge u)+*\bar\partial*(\omega\wedge \bar u)\\        &=&        i*(\partial u)-i*(\bar\partial\bar u),    \end{eqnarray*}    i.e., $(\partial u)^+=(\bar\partial \bar u)^+$ is a real valued $2$ form.    On the other hand,  one gets that     $\partial u-\bar\partial\bar u$ is an imaginary valued anti-self-dual $2$ form, which implies that     $\Lambda(\partial u)-\Lambda(\bar\partial\bar u)=0$ and $d^+(u+\bar u)=0$. Therefore, by the identity $\ker(d^+)=\ker(d)$ on the closed smooth oriented $4$ manifolds, one has $d(u+\bar u)=0$, which yields the identification     $\bar H^1\cong H^1_{dR}(X;\mathbb R).$\end{pf}
Before proceeding, we verify the following  two relations on almost complex $4$ manifolds.
\begin{lemma}\label{lemma-ddbar-zero}
Let  $(X,J)$ be an almost complex $4$ manifold. We have the following formulas: 
\begin{itemize}
\item[$(1)$] $\partial\bar\partial (\partial u+\bar\partial v)=0$, for any $u\in\mathcal{A}^{0,1}$ and $v\in\mathcal{A}^{1,0}$;
\item[$(2)$] $\partial\bar\partial(\partial \bar\partial f)=0$, for any function $f$. 
\end{itemize}
\end{lemma}
\begin{pf}
From the formula \[\partial\bar\partial f=\partial(\bar\partial f/2)+\bar\partial(-\partial f/2),\] we see that (1) implies (2). Therefore, it suffices  to show (1). 
We formulate 
\begin{eqnarray*}\partial\bar\partial (\partial u+\bar\partial v)&=&-
\bar\partial\partial^2 u+\bar\partial\partial\bar\partial v=
(\bar\partial \bar\partial \mu u+\bar\partial \mu\partial u)+
(\bar\partial^2\partial v+\bar\partial \mu\bar\mu v)\\
&=&(\bar\partial^2 d(u+v)+\bar\partial \mu(d(u+v)))\\
&=&  -(\bar\partial^2+\bar\partial\mu+\mu\bar\partial)(d(u+v))=0.
\end{eqnarray*}
Here we use  the fact that $\mu,~\bar\mu $ act trivially on $\mathcal{A}^{1,1}$ for the first equality, $\bar\partial^2$ acts trivially on $\mathcal{A}^{1,1}$ and $\mathcal{A}^{0,2}$, $\bar\partial\mu$ acts trivially on  $\mathcal{A}^{1,1}$ and $\mathcal{A}^{2,0}$ for the second one, and $\mu$ acts trivially on $\mathcal{A}^3$ for the third one. 
\end{pf}

\noindent
Combining the above Lemma \ref{lemma-ddbar-zero} with Lemma \ref{lemma-partial-solution}, one has the following. 
\begin{lemma}\label{lemma-s-map}
Let $(X,J,g,\omega)$ be a compact almost Hermitian $4$ manifold.     
The condition  $\tilde h^{1,0}=\tilde h^{0,1}$ holds, if and only if there exists a linear map $s:Z^{1,1}_{dd^c}(X,\mathbb R)\to Z^{2}_{d}(X,\mathbb R)$, such that
\begin{enumerate}
\item $s(\psi)=\psi+\bar\partial u+\partial\bar u+\mu u+\bar\mu\bar u$, where $u\in\mathcal{A}^{0,1}$;
\item $s(\partial u+\bar\partial \bar u)=d(u+\bar u)$, for any $u\in \mathcal{A}^{0,1}$;
\end{enumerate}
where $Z^{1,1}_{dd^c}(X,\mathbb R)$ and $Z^2_d(X,\mathbb R)$ are the spaces of all $dd^c$-closed real-valued $(1,1)$-forms and    $d$-closed real-valued $2$-forms respectively. 
\end{lemma}
\begin{pf}
%	Combining the condition $b_1=\hat{h}^1$ with Lemma \ref{lemma-b1-h1}, we have 	\[\tilde{h}^{0,1}+\hat{h}^{0,1}=\hat{h}^1= b_1\leq h^{1,0}+\hat{h}^{0,1}.\]	This implies that $\tilde h^{0,1}=h^{1,0}$. 
%We only need to show that the map $s$ is well-defined.
It suffices to prove that  any two solutions to the equation  \eqref{eqn-ddbar} induce the same $d$-closed $2$-form.

Assume that there are two forms $u_1,u_2\in\mathcal{A}^{0,1}$ satisfying  
\[\bar\partial \psi=\partial\bar\partial u_i+\bar\mu\partial\bar u_i+\partial\bar\mu \bar u_i+\bar\mu\mu u_i,\]
for some $dd^c$-closed real-valued $(1,1)$-form $\psi$ with $i=1,2$. 
Setting $\gamma=u_1-u_2$, it holds that
\begin{equation}\partial\bar\partial \gamma+\bar\mu\partial\bar \gamma+\partial\bar\mu \bar \gamma+\bar\mu\mu \gamma=0.\label{eqn-ddbar-zero}
\end{equation}
By the elementary identity $Re(a+b)=Re(a+\bar b)$ for any complex numbers $a$ and $b$, 
we compute
\begin{eqnarray*}
\int_X|\partial\bar\gamma+\mu\gamma|^2
&=&
Re\int_X(|\partial\bar\gamma|^2+|\mu\gamma|^2)+Re(\int_X(\partial\bar\gamma,\mu\gamma)+(\mu\gamma,\partial\bar\gamma))\\
&=&   Re\int_X\partial\bar\gamma\wedge\bar\partial \gamma+\mu\gamma\wedge\bar\mu\bar\gamma+
Re \int_X(\partial\bar\gamma\wedge\bar\mu\bar\gamma+
\overline{(\mu\gamma\wedge\bar\partial\gamma)}\\
&=&-Re(\int_X\partial\bar\partial\gamma\wedge\bar\gamma+
\bar\mu\mu\gamma\wedge\bar\gamma+ \bar\mu\partial\bar\gamma\wedge\bar\gamma+\partial\bar\mu\bar\gamma\wedge\bar\gamma)=0.
\end{eqnarray*}
This shows $\bar\partial\gamma+\bar\mu\bar\gamma=0=\partial\bar\gamma+\mu\gamma$. Substituting $\gamma=u_1-u_2$ into these  two equations, it is clear that \[
\bar\partial u_1+\bar\mu \bar u_1=\bar\partial u_2+\bar \mu\bar u_2
\mbox{ and }
\partial\bar u_1+\mu u_1=\partial\bar u_2+\mu u_2.
\]
%$[\gamma]\in \hat{H}^{0,1}$,where we use the equation \eqref{eqn-ddbar-zero} for the last equality. By the exact sequence, \[0\to H^{1,0}_{Dol}=\tilde H^{0,1}_{Dol}\to \hat H^1 \to \hat H^{0,1}\to0,	\] we can find an element $\gamma'\in\tilde H^{1,0}_{Dol}\subset\ker(d)\cap\mathcal{A}^{1,0}$ such that $d^{0,2}(\gamma'+\gamma)=0=d^{2,0}(\gamma'+\gamma)$.  $$\bar\partial\gamma=0,~\mu\gamma=0,$$and 
Therefore, it yields $$
\bar\partial u_1+\partial\bar u_1+\mu u_1+\bar\mu\bar u_1=\bar\partial u_2+\partial\bar u_2+\mu u_2+\bar\mu\bar u_2,
$$ which shows that the map $s$ is well-defined. 
%We find a function $f\in\mathcal{A}^0_{Dol}$, such that $\gamma-\bar\partial f\in\ker(d)$, which implies that $\bar\partial\gamma=0$ and 
\end{pf}

\noindent
By Lemma \ref{lemma-ddbar-zero} and Lemma \ref{lemma-s-map},
one    has the following corollary.
\begin{cor}
Let $(X,J)$ be a compact almost complex $4$ manifold. Suppose that $\tilde h^{1,0}=\tilde h^{0,1}$. Then, the above map $s$ descends to an injective map
\[
S: H^{1,1}_{dd^c}(X;\mathbb R)\to H^2_{dR}(X;\mathbb R),
\]
where $H^{1,1}_{dd^c}:=\frac{\ker(dd^c:\mathcal{A}^{1,1}_{\mathbb R}\to\mathcal{A}^4_{\mathbb R})}{im(d^{1,1}:\mathcal{A}^1_{\mathbb R}\to\mathcal{A}^{1,1}_{\mathbb R})}$.
\end{cor}



Now we are ready to show our main theorem.

\begin{pf}\textbf{ of Theorem \ref{thm-main-1}}
By Gauduchon's result  \cite{Gau77}, one can find an Hermitian metric $g$ on $(X,J)$, whose imaginary part $\omega$  is  $dd^c$-closed. Lemma \ref{lemma-partial-solution} provides that there is a form $u\in\mathcal{A}^{0,1}$ such that the real-valued $2$-form $\omega'=\omega+\bar\partial u+\partial\bar u+ \mu u+\bar\mu\bar u$ is $d$-closed. 
By using the following Lemma \ref{lemma-nondegenerate},    $\omega'$ is $J$-taming, i.e., $(X,\omega')$ is a compact $J$-taming symplectic $4$ manifold. 
\end{pf}
%By the following lemma, and Gauduchon's result, one has a symplectic form on a compact almost complex $4$ manifold. 

\begin{lemma}\label{lemma-nondegenerate}
Let $\psi$ be a real-valued positive $(1,1)$-form of an almost complex $4$ manifold $(X,J)$ and let $\sigma$ be a $(2,0)$-form. Then, the real form 
$\psi'=\psi+\sigma+\bar\sigma$ is non-degenerate. 
\end{lemma}
\begin{pf}
We consider the Hermitian metric $g=\psi(J,)$. For a point $x$ of $X$, we choose a unitary basis $(Z_1,Z_2)$  with respect to $g$. Locally, one has the expressions 
\[\psi=i\sum_{k=1,2}\theta^k\wedge\bar\theta^k\mbox{ and }\sigma=a\theta^1\wedge\theta^2,\] 
where $(\theta^1,\theta^2)$ is dual to $(Z_1,Z_2)$ at $x$.
It suffices to show that for any  real-valued vector field $v\in\Gamma(TX)$ the contraction $\iota_v\psi'$ is zero at $x$ if and only if $v$ is zero at $x$. 

Again, we have the local expressions \[v=v_1Z_1+v_2Z_2+\bar v_1\bar Z_1+\bar v_2\bar Z_2,\]
and 
$$\iota_v\psi'=av_1\theta^2+iv_1\bar\theta^1-
av_2\theta^1+iv_2\bar\theta^2-i\bar v_1\theta^1+\bar a\bar v_1\bar\theta^2-i\bar v_2\theta^2-\bar a\bar v_2\bar\theta^1.$$
Therefore, the form $\iota_v\psi'$ is zero at $x$  if and only if
\[
av_1-i\bar v_2=0,~i  v_1-\bar a\bar v_2=0.
\]
The non-singularity of the matrix $\left(\begin{array}{cc}
a&-i  \\
i & -\bar a
\end{array}\right)$ shows that $v_1=\bar v_1=v_2=\bar v_2=0$, i.e., $v$ is zero at $x$. %the form $\psi'$ is nondegenerate at each point. 
\end{pf}

%In general for $v\in\mathcal{A}^{1,0},~u\in\mathcal{A}^{0,1}$, we have \[d(\partial v+\bar\partial u)=\bar\partial\partial v+\partial\bar\partial u+\bar\mu \partial v+\mu\bar\partial u,~d(\mu u+\bar\mu v)=\bar\mu\mu u+\mu\mu\bar v+\bar\partial\mu u+\partial\bar\mu v.\]For a given $\partial\bar\partial$-closed $(1,1)$-form $\psi$, we consider the equation \[\bar\partial\psi=\partial\bar\partial u+\bar\mu\mu u+\bar\mu\partial v+\partial\bar\mu v.\]The equation has a solution if and only if the term $\bar\partial\psi$ is vertical to the subspace $\ker((\partial\bar\partial)*+(\bar\mu\mu)*+(\bar\partial\mu+\mu\bar\partial)*)$. For any $w\in\ker((\partial\bar\partial)*+(\bar\mu\mu)*+(\bar\partial\mu+\mu\bar\partial)*)$, we have \[\partial\bar\partial v'+\bar\mu\mu v'+\mu\bar\partial v'+\bar\partial\mu v'=0,\]where $v'=*w\in\mathcal{A}^{0,1}$. For grading reason, we rewrite the above equation as \begin{equation}	\partial\bar\partial v'+\bar\mu\mu v'=0,~\mu\bar\partial v'+\bar\partial\mu v'=0.\label{eqn-ddbar}\end{equation}The Stokes lemma and the second equation of \eqref{eqn-ddbar}  give \begin{eqnarray*}	0&=&\int_X\partial\bar\partial v'+\bar\mu\mu v'\wedge \bar v'=\int_Xd(\bar\partial v'+\mu v')\wedge \bar v'\\	&=&\int_X(\bar\partial v'+\mu v')\wedge d(\bar v')	=\int_X|\bar\partial v'+\mu v'|^2=\int_{X}|\bar\partial v'|^2+|\mu v'|^2.\end{eqnarray*}The condition $h^{1,0}=\ell^{0,1}$ implies that there is a form $v''\in \ker(\bar\partial)\cap \ker(\mu)\cap (\mathcal{H}^{0,1}_{\bar\partial,\mu})^\perp$ such that $v'-v''\in\ker(\Delta_{\bar\partial}+\Delta_{\mu})=\ker(d)$, hence \[\int_X\bar\partial \psi\wedge  v'=0.\]



%We have the decomposition $L^2(\mathcal{A}^{p,q})=L^2(\mathcal{A}^{p,q}_{\bar\partial,\mu})\oplus L^2(\mathcal{A}^{p,q}_{\bar\partial,\mu})^{\perp}$.For any $\alpha\in \ker(\bar\partial)\cap \mathcal{A}^{0,1}_{\bar\partial,\mu}$ with \[\int_X(\alpha,\bar\partial f)dV=0,\]for any $f\in \mathcal{A}^{0,0}_{\bar\partial,\mu}$, we claim that \[\int_X(\alpha,\bar\partial f)dV=0,\]for any $f\in \mathcal{A}^{0,0}$.Since for any  smooth function $f$, we write $f=f_0+f_1$, where $f_0\in L^2(\mathcal{A}^{0,0}_{\bar\partial,\mu})$ and $f_1\perp \mathcal{A}^{0,0}_{\bar\partial,\mu}$.$\bar \partial f=\alpha_0+\alpha_1$, where $\alpha_0 \mathcal{A}^{0,1}_{\bar\partial,\mu}$ i.e. $\mu\alpha_0=0$.\begin{eqnarray*}    \partial u=(du)^{1,1}=(\sum_ie^*_i\wedge \nabla_{e_i}u)^{1,1}\\    =\end{eqnarray*}


%We have the Hodge decomposition on $H^{p,q}_{\bar\partial,\mu}$. 

%For any form $\alpha $ and any function $f$, one has \[\bar\partial^2(f\alpha)=\bar\partial(\bar\partial f\wedge \alpha+f\bar\partial\alpha)=\bar\partial^2f\wedge\alpha+f\bar\partial^2\alpha,\]\[\bar\mu\partial(f\alpha)=\bar\mu(\partial f\wedge \alpha+f\partial\alpha)=\bar\mu\partial f\wedge \alpha-\partial f\wedge\bar\mu\alpha+f\bar\mu\partial \alpha\]and\[\partial\bar\mu(f\alpha)=\partial(f\bar\mu\alpha)=\partial f\wedge\bar\mu\alpha+f\partial\bar\mu\alpha\]

%We want to find a form  $\alpha\in\mathcal{A}^1$ such that $\psi+d^{0,2}\alpha+d^{2,0}\alpha$ is closed for each $\partial\bar\partial$-closed $(1,1)$-form $\psi$. To expand the $d$ action, we have \[\bar\partial \psi=\bar\mu d^{2,0}\alpha+\partial d^{0,2}\alpha.\]The equation has a solution if and only if the term $\bar\partial\psi$ is vertical to the kernel of the adjoint operator $(\bar\mu d^{2,0}+\partial d^{0,2})^*$. For any $w\in\ker((\bar\mu d^{2,0}+\partial d^{0,2})^*)$, we set $v=*w$.  We have $(d^{2,0})^**\mu*v+(d^{0,2})^**\bar\partial*v=0$, \[\int_X\bar\partial\psi\wedge \bar v=0\]


\hfill

%\noindent
Now, we show the generalized  $\partial\bar\partial$-lemma on compact almost complex $4$ manifolds without assuming the vanishing of Nijenhuis tensor. 

%\begin{lemma}	Let $(X,J,g,\omega)$ be a compact almost Hermitian $4$manifold. Then, for any  $d$-exact $(1,1)$ form $\psi$, we can write $\psi=\bar\partial\partial f$ for some    function $f$, if and only if the equality  $h^{1,0}=\tilde h^{0,1}$ holds.\end{lemma}
\begin{pf} {\bf of Theorem \ref{thm-ddbar}}
First,   assume that the equality $\tilde h^{1,0}=\tilde h^{0,1}$ holds. Recall that this establishes the exact sequence
\[0\to \tilde{H}^{1,0}_{\partial}\cong\tilde H^{1,0}_{Dol}\to \hat H^1\to \hat H^{0,1}\to0.	\] 
For any $d$-exact $(1,1)$-form $\psi$,   we write $\psi=d\alpha$ for some $\alpha\in \mathcal{A}^1$.  
The condition that $d\alpha$ is a $(1,1)$-form shows that the $(2,0)$ and $(0,2)$ components of $d\alpha$ vanish, i.e.,
 \[\bar\partial \alpha''+\bar\mu\alpha'=0\mbox{ and }\partial\alpha'+\mu\alpha''=0,\]
where $\alpha'$  and $\alpha''$ are  the $(1,0)$ and  $(0,1)$-component of $\alpha$ respectively. 
%	Since the condition is equivalent to the exact sequence	\[0\to H^{1,0}\to\hat H^1\to \hat H^{0,1}\to0.	\] %$H^1_{dR}=\bar H^1\to \bar H^{0,1}$ is surjective and

Since
$[\alpha'+\alpha'']\in\hat H^{1}$, by definition there are two functions  $ f'$ and $f''$ satisfying 
\begin{equation}
 (\partial^2 f'-\partial^2 f'')=0= (\bar\partial^2 f''-\bar\partial^2 f'),\label{eqn-d-0}
\end{equation}
such that
\begin{equation}
\alpha'-\partial f' \in\ker(d)\cap\mathcal{A}^{1,0}=\tilde{H}^{1,0}_{Dol}.\label{eqn-d-1}
\end{equation}% for some $(1,0)$-form $v'$.
Together with the equations \[\bar\partial (\alpha''-\bar\partial f'')+\bar\mu (\alpha'-\partial f')=0,\]
and
\[\partial (\alpha'-\partial f')+\mu (\alpha''-\bar\partial f'')=0,\] it derives 
\[\alpha''-\bar\partial f''\in\ker(\mu)\cap\ker(\bar\partial),\]
i.e., $[\alpha''-\bar\partial f'']\in\tilde H^{0,1}_{Dol}$.

Again,   the condition $\tilde h^{1,0}=\tilde{h}^{0,1}$ implies that there is a function 
\begin{equation}
f\in\ker(\partial^2)\cap\ker(\bar\partial^2),\label{eqn-d-f}
\end{equation}
such that 
\begin{equation}
\alpha''-\bar\partial f''-\bar\partial f\in\ker(d)\cap\mathcal{A}^{0,1}. \label{eqn-d-2}
\end{equation}
Hence, we obtain
% \[d\alpha=\partial(\alpha''-\bar\partial f'')+\partial\bar\partial f''+ \bar\partial(\alpha'-\partial f')+\bar\partial f'=\partial\bar\partial(f''-f').    \]
\begin{eqnarray*}
\psi&=&d(\alpha''-\bar\partial f''-\bar\partial f)+d(\alpha'-\partial f')+d(\bar\partial f''+\partial f'+\bar\partial f)\\
&=&d^{2,0}(\bar\partial f''+\partial f'+\bar\partial f)+d^{0,2}(\bar\partial f''+\partial f'+\bar\partial f)+d^{1,1}(\bar\partial f''+\partial f'+\bar\partial f)\\
&=&\partial \bar\partial(f''-f'+f) ,
\end{eqnarray*}
where we use \eqref{eqn-d-1}, \eqref{eqn-d-2} for the second equality and \eqref{eqn-d-0}, \eqref{eqn-d-f} for the last equality.   %$d=\partial+\bar\partial $ on functions for the last equality. 
% i.e., .

% This implies that $\int_X\omega\wedge\psi=0$ and $\omega\wedge(dv+\bar\partial\partial f)=0$.    Hence, $dv+\bar\partial\partial f=d(v+\partial f)$ is of anti-self-dual. By the fact that $\{d^+\mbox{-clsoed  forms}\}=\{ d-\mbox{closed forms}\}$ on closed oriented $4$ manifolds, we have that $dv=\bar\partial\partial f$. 

Conversely,    assume that each $d$-exact  $(1,1)$-form $\psi $ can be  written as $\psi=\partial\bar\partial f$ for some   function $f$. We consider a representative form $u$ in a class $[u]\in \tilde H^{0,1}_{Dol}$. Clearly, the form $\partial u=du$ is a $d$-exact $(1,1)$-form. By the assumption, there is a function $f$ such that 
$$\partial (u-\bar\partial f)=0.$$
We obtain $u-\bar\partial f\in \ker(d)\cap\mathcal{A}^{0,1}$ by Lemma \ref{lemma-barpartial}, which implies that $f\in\ker(\partial^2)\cap\ker(\bar\partial^2)$ by $\mu(u)=\bar\partial(u)=0$. 
Therefore,  the conjugation map 
\[ 0\to  H^{1,0}_{Dol}=\tilde H^{1,0}_{Dol}\to \tilde H^{0,1}_{Dol},\]
is an isomorphism, i.e., $\tilde h^{1,0}=\tilde h^{0,1}$.  
%The canonical embedding map $ H^1_{dR}\to \bar H^1$ is an isomorphism. Since     $\bar h^1=h^{1,0}+\tilde h^{0,1}=b_1\leq h^{1,0}+\bar h^{0,1}$. 
%It holds that $\partial v=d v$, i.e. $\partial v$ is a $d$-exact $(1,1)$ form. This, implies that $\partial v=\bar\partial \partial f$ for some $L^2_2$-function $f$. Similarly, we have that $v-\bar\partial f\in\ker(\bar \partial)\cap\mathcal{A}^{0,1}$. 
\end{pf}

%\begin{rmk}On compact K\"ahler manifolds, it is well-known that each $d$-exact form is $\partial\bar\partial$-exact, which is called the $\partial\bar\partial$-lemma.  Gauduchon's work \cite{Gau76} tells us that on the case of compact complex surfaces, the $\partial\bar\partial$-lemma is equivalent to the condition $h^{1,0}=h^{0,1}$. Hence, by the proof of Kodaira conjecture, the $\partial\bar\partial$-lemma is equivalent to the K\"ahler condition on  compact complex surfaces. %Since it holds that 	$\tilde h^{0,1}=h^{0,1}=\bar h^{0,1}$ for compact complex surfaces, the above lemma can be regarded as an alternative proof of Gauduchon's work.\end{rmk}  


\noindent
%At last, we give 
One also has the following corollary. % to end this paper.  
\begin{cor}
Let $(X,J)$ be a compact almost complex $4$ manifold. Then, the condition $\tilde h^{1,0}=\tilde h^{0,1}$ is equivalent to the 
condition
$$H^{1,1}_{dR}(X;\mathbb C)\cong H^{1,1}_{BC}(X;\mathbb C),$$ where 
%$H^{1,1}_{dR}=\frac{\ker(d)\cap\mathcal{A}^{1,1}}{	im(d)\cap\mathcal{A}^{1,1}}$ and 
$H^{1,1}_{BC}(X;\mathbb C):=\frac{\ker(d)\cap\mathcal{A}^{1,1}}{
im(dd^c)\cap\mathcal{A}^{1,1}
}$ and 
$H^{1,1}_{dR}(X;\mathbb C):=\frac{\ker(d)\cap\mathcal{A}^{1,1}}{
im(d)\cap\mathcal{A}^{1,1}
}$.
\end{cor}
%The proof is to use Theorem \ref{thm-ddbar}. Here we omit it.  

%Moreover,
At the end of this paper, we give an application. 
Recall that on compact almost complex $4$ manifolds, we show the following relations, c.f., Corollary \ref{cor-b1-ineq-4-mfld}:\begin{equation}	2\tilde{h}^{1,0}=2h^{1,0}\leq b_1 \mbox{ and }\tilde{h}^{1,0}\leq\tilde{h}^{0,1}\leq\hat{h}^{0,1}\leq h^{0,1}.	\label{eqn-b1-ineq-4mfld}\end{equation}  
Our definition of the  refined Dolbeault cohomology can also be used to detect some difference between the compact complex surfaces and the general almost complex $4$ manifolds.
%see Lemma \ref{lemma-hodge-inequality} and Corollary \ref{cor-b1-ineq-4-mfld}. %  and also hold on the case of almost complex manifolds. 
It is clear   that on compact complex surfaces  the equality 
\begin{equation}
	b_1=h^{1,0}+h^{0,1},\label{eqn-b1-equality}
\end{equation}
always holds by using the Hodge decomposition and the Atiyah-Singer-Hirzebruch index theory, c.f., \cite[Theorem 2.7-Chapter IV]{BHPV}. By  definition,  the equalities $\tilde h^{0,1}=\hat{h}^{0,1}=h^{0,1}$ also trivially hold on  compact complex surfaces. Hence, it is natural to wonder  whether an analog of \eqref{eqn-b1-equality} in terms of $\tilde{h}^{1,0},~\tilde{h}^{0,1},~\hat{h}^{0,1},~h^{0,1}$ and $b_1$ holds for compact  almost complex $4$ manifolds. However, by Theorem \ref{thm-main-1} and Inequalities \eqref{eqn-b1-ineq-4mfld}, neither of  the equalities
\begin{equation}
	b_1=\tilde h^{1,0}+ \tilde h^{0,1},~b_1=\tilde h^{1,0}+ \hat h^{0,1},\mbox{ }b_1=\tilde  h^{1,0}+{h}^{0,1}, \label{eqn-b1-hodge-complex-surface}
\end{equation}  holds on general compact  almost complex $4$  manifolds. Moreover, this implies that the three identities in \eqref{eqn-three-identities} are not equivalent to each other on compact almost complex $4$ manifolds. % and the equality \eqref{eqn-b1-hodge-complex-surface} does not hold, 
For example, the  manifold $(2k+1)\mathbb{C}P^2\#l\overline{\mathbb{C}P^2}$ with $k>0$,  $l\geq0$ and  any almost complex structure satisfies the relations $$b_1=2h^{1,0}=2\tilde{h}^{1,0}=0\mbox{ and  } \tilde h^{0,1}>0,$$
hence violating all the equalities in \eqref{eqn-b1-hodge-complex-surface}. 

%\begin{prop}\label{prop-5}    Let $\psi_1,~\psi_2\in L^2(\mathcal{A}^{1,1}_{\mathbb R})$ be weakly     $\partial\bar\partial$-closed and satisfy $\int_X\psi^2_i\geq0$ and    $\int_X\psi_i\wedge\omega\geq0$, for $i=1,2$.    Then, $\int_X\psi_1\wedge\psi_2\geq (\int_X\psi^2_1)^{1/2}     (\int_X\psi^2_2)^{1/2}$,  with equality if and only if $\psi_1$ and $\psi_2$     are linearly dependent modulo the image of $i\bar\partial\partial$. \end{prop}
%\begin{pf}By Lemma \ref{lemma-weak-partial}, we assume that $c_j=\int_X\omega\wedge \psi_j>0$. To prove the inequality, we may replace $\psi_j$ by $\psi_j+\epsilon\omega$ for a small positive $\epsilon$ and assume $\int_X\psi^2_j>0$.Since $\int_X\omega\wedge (c_2\psi_1-c_1\psi_2)=0$, the form $c_2\psi_1-c_1\psi_2+i\bar\partial\partial f$ is of anti-self-dual for some function $f$. We have \begin{eqnarray*}    0\geq\int_X(c_2\psi_1-c_1\psi-2)^2\geq2c_1c_2(\int_X\psi^2_1)^{1/2}    (\int_X\psi^2_2)^{1/2}-2c_1c_2\int_X\psi_1\wedge\psi_2.\end{eqnarray*}\end{pf}

%\begin{cor}\label{cor-6}    If $\phi\in L^2(\mathcal{A}^{1,1}_{\mathbb R})$ is weakly $\partial\bar\partial$-closed, and satisfies $\int_X\phi^2>0$ and $\int_X\phi\wedge\chi>0$ for any form $\chi$ with the condition, $\int_X\chi^2\geq0$ and $\int_X\chi\wedge\omega>0$. \end{cor}

%\begin{lemma}\label{lemma-7}    Let $\chi\in L^2(\mathcal{A}^{1,1}_{\mathbb R})$ be $\partial\bar\partial$-closed, and satisfy     $\int_X\chi^2\geq0$ and $\int_X\chi\wedge \omega\geq0$. Then,     for each $\epsilon>0$, there is a positive $(1,1)$-form $p_\epsilon$ and a function $f_\epsilon$ such that     \[\|\chi-p_\epsilon+i\bar\partial\partial f_\epsilon\|_{L^2}<\epsilon.\]\end{lemma}
%\begin{pf}    If $\int_X\chi\wedge \omega=0$, by Lemma \ref{lemma-weak-partial}, $\chi$ is $\partial\bar\partial$-exact.     We assume $\int_X\chi\wedge \omega=1$ by scaling.     Let     \[\mathcal P=\{p\in L^2(\mathcal{A}^{1,1}_{\mathbb R})|~p\geq 0,~a.e.,~    \int_X\omega\wedge p>0\},\]    \[\mathcal{P}_\epsilon=\{p'\in  L^2(\mathcal{A}^{1,1}_{\mathbb R})|~    \|p'-p\|_{L^2}<\epsilon,~\mbox{for some }p\in\mathcal{P}\},\]    and\[    \mathcal{H}:=\{\chi+i\bar\partial\partial f|~f\in L^2_2(X)\}.    \]    $\mathcal P_\epsilon$ is an open convex subset of $L^2(\mathcal{A}^{1,1}_{\mathbb R})$ and $\mathcal H$ is a closed convex subset.     If $\mathcal{P}_\epsilon\cap \mathcal{H}=\emptyset$, the Hahn-Banach Theorem implies that there exists $\phi\in L^2(\mathcal{A}^{1,1}_{\mathbb R})$ and a constant $c\in \mathbb R$ such that $\int_X\phi\wedge h\leq c$ and $\int_X\phi\wedge p>c$ for each $h\in\mathcal{H}$ and each $p\in\mathcal{P}_\epsilon$. We have $\phi$ is weakly $\partial\bar\partial$-closed.     We set $\phi_0=\phi-c\omega$, which is weakly $\partial\bar\partial$-closed.     We have that     $\int_X\wedge_0\wedge \chi\leq 0$, and $\int_X\phi_0\wedge p_0>0$, i.e. $\phi_0$ is positive almost everywhere. Hence, $\int_X\phi^2_0>0$ and     $\int_X\phi_0\wedge \omega>0$, i.e $\int_X\phi_0\wedge\chi>0$ by Corollary \ref{cor-6}. This is a contradiction.     Therefore, $\mathcal{P}_\epsilon\cap \mathcal{H}\neq\emptyset$, which implies the existence of $p_\epsilon$ and $f_\epsilon$.\end{pf}


%The straightforward calculation gives that \begin{eqnarray*}    \int_{X}(\partial u+\bar\partial \bar u+\bar\mu\bar u+\mu  u)^2    =\int_X\partial u\wedge\partial u+\int_X{\bar\partial \bar u}\wedge{\bar\partial\bar u}+\int_X\partial u\wedge \bar\partial \bar u+\int_X\bar\partial \bar u\wedge     \partial u+2\int_X{\mu u}\wedge{\bar\mu \bar u}\\    =\int_X\partial^2 u\wedge u+\bar\partial^2\bar u\wedge \bar u+  2\bar\partial\partial u\wedge \bar u+\bar\mu\mu u\wedge \bar u\end{eqnarray*}

%Let $\omega$ be a fixed positive $\partial\bar\partial$-closed $(1,1)$, and let $u\in\mathcal{A}^{0,1}$ satisfy $\omega'=\omega+\partial u+\bar\partial \bar u$ is $d$-closed. We assume that $\omega\wedge \omega'=c \omega^2$ for some constant $c$. We have \begin{eqnarray*}    \int_X\omega\wedge \partial u&=&-\int_X\partial \omega\wedge u    =\int_X(\partial\bar\partial \bar u+\partial\partial u)\wedge u=    \|\bar \partial u\|^2_{L^2}\\    &=&-\int_X\bar\partial \partial \bar u\wedge u+\mu\bar\mu\bar u\wedge u    -\int_X \mu\bar\partial u\wedge u+\bar\partial\mu u\wedge u\\    &=&\|\partial\bar u+\mu u\|^2_{L^2}
%\int_X\omega'^2&=&\int_X(\omega+d(u+\bar u)-\bar\partial u-\mu u-\partial \bar u-\bar\mu \bar u)^2\\\end{eqnarray*}
% and $\int_X\omega\wedge \mu\bar u=\int_X\omega\wedge\bar\mu u=0$. The closed  condition on $\omega'$ and Stokes lemma show that for each term $\sigma\in\{\partial u,\bar\partial \bar u,\bar\mu\bar u,\mu  u\}$, it holds that $\int_X\omega'\wedge \sigma=0$. Hence, \[\int_X\omega'^2=\int_X\omega'\wedge\omega=\int_X\omega^2+2(\|\partial\bar u+\mu u\|^2_{L^2}).\]We set the volume $\int_X\omega\wedge\omega/2=1$, so $c=1+\|\partial\bar u+\mu u\|^2_{L^2}$. We set $a_0=|\partial \bar u+\mu u\|^2$  and $\omega\wedge \partial u=a_0\omega^2$.  Let $\chi$ be a smooth real $\partial\bar\partial$-closed $(1,1)$-form satisfying $\int_X\chi^2>0$ and $\int_X\chi\wedge \omega>0$. We assume that there exists a smooth form $u\in\mathcal{A}^{0,1}$ satisfying that  $\chi'=\chi+\partial u+\bar\partial \bar u$ is $d$-closed. We  set $a_1=\|\partial \bar u+\mu u\|^2$. By Proposition \ref{prop-5}, the number $b:=\frac12\int_X\omega\wedge\ch'i$ satisfies  $b^2\geq (1+a)(1+a_0)$. For $t_0=b-\sqrt{b^2-(1+a)}$, the form $\phi=\chi'-t_0\omega$ satisfies $\partial\bar\partial \phi=0$, $\int_X\phi^2=0$ and $\int_X\omega\wedge \phi=2(b-t_0)\geq0$. with equality if and only if $\omega\in\ker(d)$ and $\chi'$ is $\partial\bar\partial$-homologous to a multiple of $\omega$. 

%If $b-t_0>0$, then for each $n\in\mathbb N$  there is a positive $(1,1)$-form $p_n$ and a function $f_n$ such that $\|\phi-i\partial\bar\partial f_n-p_n\|<\frac 1n$ by  Lemma \ref{lemma-7}.Since $\int_Xp_n\wedge \omega$ is converging to $2(b-t_0)$, the  sequence $p_n/\Lambda p_n$ are bounded in $L^\infty$. By the bootstrapping arguments, $\{p_n\}$ converges to a positive $(1,1)$-current $p$ with $\phi+i\partial\bar\partial f=p$, i.e.,  $P=p+t_0\omega$  is a closed positive $(1,1)$-current satisfying $P\geq t_0\omega$. 

%\begin{eqnarray*}    dd^*=(\partial+\bar\partial+\mu+\bar\mu)(\partial^*+\bar\partial^*+\mu^*+\bar\mu^*)\\=\bar\partial\bar\partial^*+\partial\bar\partial^*+\bar\partial\mu^*+\bar\partial\bar\mu^*+\partial\bar\partial^*+\partial\partial^*+\partial\mu^*+\partial\bar\mu^*\\+\mu\bar\partial^*+\mu\partial^*+\mu\mu^*+\mu\bar\mu^*+\bar\mu\bar\partial^*+\bar\mu\bar\partial^*+\bar\mu\partial^+\bar\mu\mu^*+\bar\mu\bar\mu^*,\end{eqnarray*}
%\begin{eqnarray*}    d^*d=(\partial^*+\bar\partial^*+\mu^*+\bar\mu^*)(\partial+\bar\partial+\mu+\bar\mu)\\=\bar\partial^*\bar\partial+\bar\partial^*\partial+\bar\partial^*\mu+\bar\partial\bar\mu^*+\partial^*\partial+\partial^*\bar\partial+\partial^*\mu+\partial^*\bar\mu\\    +\mu^*\mu+\mu^*\bar\mu+\mu^*\partial+\mu^*\bar\partial+\bar\mu^*\bar\mu+\bar\mu\mu^*+\bar\mu^*\partial+\bar\mu^*\bar\partial,\end{eqnarray*}Hence\begin{eqnarray*}    \Delta_d=\Delta_{\partial}+\Delta_{\bar\partial}+\Delta_\mu+\Delta_{\bar\mu}\\    +[\bar\partial,\partial^*]+[\bar\partial,\mu^*]+[\bar\partial,\bar\mu^*]+[\partial,\bar\partial^*]+[\partial,\mu^*]+    [\partial,\bar\mu^*]\\    +[\mu,\partial^*]+[\mu,\bar\partial^*]+[\mu,\bar\mu^*]+    [\bar\mu,\partial^*]+[\bar\mu,\bar\partial^*]+[\bar\mu,\mu^*].\end{eqnarray*}









%%%%%%%%%%%%%
\begin{thebibliography}{99}

%\bibitem{ADN}S. Agmon, A. Douglis, L. Nirenberg, Estimates near the boundary for solutions of elliptic partial differential equations satisfying general boundary conditions I, Comm. Pure Appl. Math. 12 (1959), 623-727.


%\bibitem{APS} M. F. Atiyah, V. K. Patodi and I. M. Singer, Spectral asymmetry and Riemannian Geometry. III, Math. Proc. Cambridge Philos. Soc. 79, 76-99 (1976)


%\bibitem{BGN1} C. P. Boyer, K. Galicki, and M. Nakamaye, Einstein metrics on rational homology 7-spheres, Ann. Inst. Fourier (Grenoble), vol. 52 (2002), no. 5, 1569-1584.

%\bibitem{BGN2}C. P. Boyer, K. Galicki, and M. Nakamaye,On the geometry of Sasakian-Einstein 5-manifolds, Math. Ann., vol. 325 (2003), no. 3, 485-524.

%\bibitem{BGN3}C. P. Boyer, K. Galicki, and M. Nakamaye,Sasakian geometry, homotopy spheres and positive Ricci curvature, Topology42 (2003), no. 5, 981–1002

\bibitem{BHPV}
Barth, Wolf P.; Hulek, Klaus; Peters, Chris A. M.; Van de Ven, Antonius: Compact Complex Surfaces, 2nd edition, Ergebnisse der Mathematik und ihrer Grenzgebiete. 3. Folge. A Series of Modern Surveys in Mathematics 4. Springer-Verlag, Berlin, 2004. xii+436 pp. ISBN: 3-540-00832-2.

\bibitem{Bau06}
Bauer, Stefan: Almost complex 4-manifolds with vanishing first Chern class, J. Differential Geom.,  \textbf{79} (2008), no.1, 25-32.

\bibitem{Buch}
Buchdahl, Nicholas:
On compact Kähler surfaces. Ann. Inst. Fourier (Grenoble) \textbf{49}(1999), no.1, vii, xi, 287–302.

\bibitem{CPS}	
Coelho, R.; Placini, G. and Stelzig, J.: 
Maxially non-integrable almost complex strucutres: an $h$-principle and cohomological properties, preprint arxiv:2105.12113v3

\bibitem{CW20}
Cirici, Joana; Wilson, Scott O.: Topology and geometric aspects of almost K\"ahler manifolds via harmonic theory,  Selecta Math. (N.S.) \textbf{26} (2020), no. 3, Paper No. 35, 27 pp.


\bibitem{CW21}
Cirici, Joana; Wilson, Scott O.:  Dolbeault cohomology for almost complex manifolds, Adv. Math., 391 (2021) Paper No. 107970, 52 pp. 

\bibitem{Don86}
Donaldson, Simon Kirwan; Kronheimer, Peter Benedict: The geometry of four manifolds, Oxford Math. Monogr.
Oxford Sci. Publ.
The Clarendon Press, Oxford University Press, New York, 1990. x+440 pp.
ISBN:0-19-853553-8

\bibitem{Don06}
Donaldson, Simon Kirwan: Two-forms on four-manifolds and elliptic equations, 
Inspired by S. S. Chern, A Memorial Volume in Honor of A Great Mathematician, Nankai Tracts in Mathematics vol. 11 (2006), pp. 153-172. 

\bibitem{FS98}
Fintushel, Ronald; Stern, Ronald J.: Knots, links, and $4$-manifolds, Invent. Math., \textbf{134} (1998), 363-400.





\bibitem{Gau76}
Gauduchon, Paul:
La classe de Chern pluriharmonique d'un fibr\'e en droites. (English summary)
C. R. Acad. Sci. Paris Sér. A-B 282 (1976), no. 9, Aii, A479-A482. avialable at https://gallica.bnf.fr/ark:/12148/bpt6k6235850q/f41.item

\bibitem{Gau77}
Gauduchon, Paul: Le th\`eor\`eme de l'excentrici\'e nulle, C. R. Acad. Sci. Paris S\'er. A-B 285 (1977), no. 5, A387-A390.

\bibitem{Gom95}
Gompf, Robert E.: A new construction of symplectic manifolds, Annals of
Mathematics, \textbf{142}  (1995), 527-595.

\bibitem{GH}
Griffiths, Phillip; Harris, Joseph: Principles of algebraic geometry, Pure and Applied Mathematics. Wiley-Interscience, New York, 1978. xii+813 pp. ISBN: 0-471-32792-1.


\bibitem{Gromov86} Gromov, Mikhael: Partial Differential Relations, Ergebnisse der Mathematikund ihrer Grenzgebiete, Vol. 9, Springer-Verlag, Berlin, Heidelberg, NewYork, London, Paris and Tokyo, 1986, be + 363 pp.,  ISBN 0-387-12177-3
%\bibitem{BF} J.M.Bismut and D.S. Freed, The analysis of elliptic families I: Metrics and connections on determinant bundles, II: Dirac operators, eta invariants, and the holonomy theorem, Comm. Math. Phys., 107, P.103 163 (1986)
%\bibitem{BL}J.-M. Bismut and G. Lebeau, Complex immersions and Quillen metricsPubl. Math. Inst. Hautes études Sci., 74 (1991), p. ii+298
%\bibitem{KM}P.  Kronheimer, T.  Mrowka,   Monopoles and three-manifolds, volume 10. Cambridge University Press, 2007
%\bibitem{JZhang}J.Yua and W. Zhang, Positive scalar curvature and the Euler class, Journal of Geometry and Physics Volume 126, March 2018, Pages 193-203
%\bibitem{TZhang}Y. Tian and W. Zhang, Quantization formula for symplectic manifolds with boundary, GAFA,   9(1999), 596--640, 1016-443X/99/030596-45

\bibitem{Hir54}
Hirzebruch, Friedrich: Some problems on differentiable and complex manifolds,  Ann. of Math., \textbf{2} (1954) 60, 213–236.

\bibitem{HZ20}
Holt, Tom ; Zhang, Weiyi:
Harmonic forms on the Kodaira-Thurston manifold.Adv. Math. \textbf{400}(2022), Paper No. 108277, 30 pp.

\bibitem{Huy}
Huybrechts, Daniel: Complex Geometry-An Introduction, Universitext
Springer-Verlag, Berlin, 2005. xii+309 pp.
ISBN:3-540-21290-6

\bibitem{Kaw22}
Kawamura, Masaya: On a $k$-th Gauduchon almost Hermitian manifold, Complex Manifolds, \textbf{9} (2022), no.1, 65–77.

\bibitem{KM}
Kodaira, Kunihiko; Morrow, James: Complex manifolds. Holt, Rinehart and Winston, Inc., New York-Montreal, Que.-London, 1971. vii+192 pp.

\bibitem{Lamari}
Lamari, Ahc\'ene: Courants k\"ahl\'eriens et surfaces compactes, Annales de l'institut Fourier, tome 49, no. 1 (1999), p. 263-285.

\bibitem{Li06}
Li, Tian-Jun:  Quaternionic bundles and Betti numbers of symplectic $4$-manifolds with Kodaira dimension zero, Int. Math. Res. Not. (2006), Art. ID 37385, 28 pp.

%\bibitem{McSal99}	McDuff, Dusa; Salamon, Dietmar: Introduction to symplectic topology. 2nd. edition	Oxford Math. Monogr.	The Clarendon Press, Oxford University Press, New York, 1998. x+486 pp.	ISBN:0-19-850451-9

\bibitem{Miy74}
Miyaoka, Yoichi: K\"ahler metrics on elliptic surfaces, Proc. Japan Acad., \textbf{50} (1974), 533-536.

%\bibitem{Molino}P. Molino, Riemannian foliations, Progress in Mathematics, vol. 73, Birkh\"auser, Boston, 1988.
%\bibitem{Morgan} J. W. Morgan, The Seiberg-Witten equations and applications to the topology of smoothfour-manifolds, Mathematical Notes 44, Princeton University Press, Princeton, NJ, 1996.

\bibitem{NN}
Newlander, August; Nirenberg, Louis: Complex analytic coordinates in almsot complex manifolds,
Ann. of Math. \textbf{65} (1957), pp. 391-404.

%\bibitem{Oh18}Ohsawa, Takeo:$L^2$ approaches in several complex variables.Towards the Oka-Cartan theory with precise bounds. Second edition,Springer Monogr. Math.Springer, Tokyo, 2018. xi+258 pp.ISBN:978-4-431-56851-3ISBN:978-4-431-56852-0.

\bibitem{ST23}
Sillari, Lorenzo; Tomassini, Adriano: On Bott-Chern and Aeppli cohomologies of almost complex manifolds and related spaces of harmonic forms, preprint  available at arXiv:2303.17449v1

\bibitem{TWZZ22}
Tan, Qiang; Wang, Hongyu; Zhou, Jiuru; Zhu, Peng:On tamed almost complex four-manifolds. Peking Math. J. \textbf{5}(2022), no.1, 37-152.

\bibitem{Siu}
Siu, Yum Tong: Every K3 surface is K\"ahler. Invent. Math. \textbf{73} (1983), no. 1, 139-150. 

\bibitem{Taub94}
Taubes, Clifford Henry:
The Seiberg-Witten invariants and symplectic forms.Math. Res. Lett. \textbf{1}(1994), no.6, 809-822.

\bibitem{Taub95}
Taubes, Clifford Henry:
More constraints on symplectic forms from Seiberg-Witten invariants.Math. Res. Lett. \textbf{2}(1995), no.1, 9–13.


%\bibitem{Taub96}Taubes, Clifford Henry: SW$\Rightarrow$Gr: From the Seiberg–Witten equations to pseudoholomorphiccurves. Journ. Amer. Math. Soc., \textbf{9}(1996), 845-918.

%\bibitem{Taub96}  Taubes, Clifford Henry:SW$\Rightarrow$Gr: from the Seiberg-Witten equations to pseudo-holomorphic curves. J. Amer. Math. Soc. \textbf{9}(1996), no.3, 845–918

\bibitem{Tod}
Todorov, Andrei N.:
Applications of the K\"ahler-Einstein-Calabi-Yau metric to moduli of K3 surfaces.Invent. Math. \textbf{61} (1980), no.3, 251-265.

\bibitem{TO97}
Tralle, Aleksy; Oprea, John:  Symplectic manifolds with no K\"ahler structure,
Lecture Notes in Math., 1661
Springer-Verlag, Berlin, 1997. viii+207 pp.
ISBN:3-540-63105-4

\bibitem{WWZ}
Wang, Hongyu; Wang, Ken; Zhu, Peng: On closed almost complex four manifolds, preprint arXiv:2305.09213v2

%\bibitem{Nicola} L.I. Nicolaescu, Notes on Seiberg-Witten Theory, Graduate Studies in Mathematics,vol.28, Amer. Math. Soc., 2000
%\bibitem{Pe1} G.Perelman, The entropy formula for the Ricci flow and its geometric applications, arXiv:math/0211159.
% \bibitem{PR} I. Prokhorenkov and K. Richardson,Perturbations of basic Dirac operators on Riemannian foliations, Internat. J. Math.   24 (2013), no. 9, 1350072, 26 pp.
%58J20 (53C12 58E40 58J37) https://doi.org/10.1142/S0129167X13500729
%\bibitem{Reinhart} B. Reinhart, Foliated manifolds with bundle-like metrics, Ann. Math., vol.  69 (1959)(2), 119-132.
%\bibitem{Rummler} H. Rummler, Quelques notions simples en geometrie riemannienne et leursapplications aux feuilletages compacts, Comment. Math. Helv., vol.  54(1979), no. 2, 224-239.

%\bibitem{Ton} Ph. Tondeur, Foliations on Riemannian manifolds, Springer, New York, 1988

%\bibitem{SC}E. Sanmartin-Carbon, The manifold of bundle-like metrics of a Riemannian foliation, Quart. J. Math. Oxford Ser. (2), vol. 48 (1997), no. 190, 243-254.


%\bibitem{Zh}W. Zhang,Positive scalar curvature on foliations, Annals of Mathematics, 185 (2017), 1035-1068 https://doi.org/10.4007/annals.2017.185.3.9

%%%%%%%%%%
\end{thebibliography}




\end{document}

