

\section{Proposed Routing Mechanism}



In this section we present a novel routing mechanism that enables node to sink communication in the network described above. Firstly, we need to describe the topology of the network, which is dynamic over time. Then, we describe the process through which nodes maintain and spread information about the local topology introducing a congestion-avoidance mechanism. Finally, we describe the process through which each node selects a relay.

\subsection{Topology}
The network proposed in this work has a dynamic topology. Let $V$ be the set of nodes: $V = W \cup S \cup A$, namely, the underwater nodes, the surface nodes and the aerial nodes. 
For each node $u \in V$, we know the communication range $r_u$; moreover, let us define the node's state at time $t \in \mathbb
{Z}$ as the following tuple: 
\begin{equation}
    s_{u}(t) = \langle x_{u}(t), v_{u}(t), y_{u}(t), ?\rangle
\end{equation}

Considering the states of a node over time, we can describe its motion in a 3D space, where $x_{u}(t)$ represents the current coordinate of the node at time $t$, $v_{u}(t)$ represents the speed at which the node is moving at time $t$, and $y_{u}(t)$ represents coordinate of the next way-point (or direction) towards which the node is moving at time $t$.


Based on this model, it is possible to predict the future trajectory of the node. 
We assume that the node moves at a constant speed along a straight line without obstacles towards its next way-point. 
Note that the assumption of no obstacles in the environment is reasonable due to the fact that we are deploying the network in a marine environment.\todo{are we sure?} 
Given such assumption, at any given time $t$ and each time interval $\Delta t$,  we can compute the expected position at time $t + \Delta t$ of each node.
\begin{equation}
    x_{u}(t + \Delta t) = 
    \begin{cases}
    x_{u}(t) + x'_{u}(t) \Delta t & \text{if} \\
    y_{u}(t) & \text{otherwise}
    \end{cases}
\end{equation}

\todo{controlla formula, qui ci serve che $x_u(t + \Delta t)$ sia un punto sulla retta che congiunge x(t) e y(t). se lo spostamento è maggiore della distanza tra x e y allora questo punto è y altrimenti è a tale distanza da x }

This state formulation if applicable to both fixed and mobile nodes. Indeed, if a node is fixed, its speed is 0 and its coordinate never changes.

\todo{Assumption of connectivity over time, check}

%where: $x_{\psi}(t) \in \bathbb{R}^3$ is the coordinate of node ${\psi}$ at time $t$, $v_{\psi}(t) \in \bathbb{R}^+$ is the speed at which ${\psi}$ is moving, $y_{\psi}(t) \in \bathbb{R}^3$ is the next way-point (i.e., the direction towards $\psi$ is moving at time $t$).


\subsection{Network setup}
To being able to select a relay, each node has to be aware of which are the nodes in its communication range. To do so, we envisage a mechanism that allows nodes to set up and maintain forwarding tables.
Each node can enter the network broadcasting an \textit{hello packet}. Each node in the communication range which receives the hello packet stores  in its forwarding table node's state information contained in the hello packet. Each node can also decide to temporarily, or permanently, leave the network broadcasting a \textit{goodbye packet}. Each node should take this decision autonomously considering 
%its residual energy and 
the expected congestion considering its neighbourhood at a given time.

In the following we will describe in more details the hello packets, the forwarding table, the goodbye packets and the process through which a node decide whether to broadcast or not a goodbye packet. 

\subsubsection{Hello packet}
The \textit{hello packets} are used to signal node's availability and to exchange information between neighbours. Given a node $u$, the information contained are: (i) \textit{current position} $x_u(t)$ that express the 3-dimensional position of each node in the area, (ii) \textit{current speed} $v_u(t)$ that describe the instant speed at which the node is moving, (iii) \textit{next way-point} $y_u(t)$, namely, the next point in the node's trajectory if the node is mobile, (iv) \textit{buffer occupancy} that points out the current buffer occupation. 
\todo{forse ridondante, pensiamoci}
The node transmit in broadcast the hello packets periodically.

\todo{GIULIO riscrivere}
Let $t_0$ be the time at which the last hello packet has been sent, and let $t_1$ be the time at which the new packet has to be sent.


%Each node transmit a new hello packets if one of the following condition holds: i) , and let $\delta_{hello}$ be the maximum time a node can wait before broadcasting a new hello packet. Then, $t_1\le t_0 + \delta_{hello$;
 


Let us define $\delta_{hello}$ as the maximum time a node can wait before broadcasting a new hello packet. To deal with diverse mobilities of nodes \todo{(and maybe also other stuff)}, a node must send a new hello packet also whenever its state changes significantly from the last hello packet sent.

Thus let $t$ be the time at which $u$ has sent its last hello packet, $t' = min(\{\delta_{hello}\} \cup \{t'' : |s_u(t) - s_u(t'')| > \lambda \})$, where $\lambda$ is a threshold indicating the maximum change between two states before sending again the hello packet is necessary.
\todo{GIULIO fine}


\subsubsection{Neighbourhood}
The neighbourhood of a given node is computed using the hello packets received and then filtering out nodes that are not suitable to be selected as relays. The criteria we set up to exclude a node from the neighborhood are: \textit{age of information}, \textit{distance from the sender node}, \textit{belonging layer}. 
%Our goal is to exclude nodes with old information, that are too far to have a reliable connection quality,  denying the possibility to pass through the underwater layer once the data goes to the surface layer
The set of neighbours $N(\psi, t)$ of a node $\psi$ at time $t$ is defined using Equation \ref{eq:neghset}
    \todo{insert age of information condition for packets in eq \ref{eq:neghset}, modify u and place instead $\overline{\psi}$}
\begin{equation}
    N(\psi, t) = \{ u \in \Psi |\quad ||x_u(t) - x_{\psi}(t)|| < \sigma \wedge \psi \in S \cup A \rightarrow u \notin W \} 
    \label{eq:neghset}
\end{equation}
\todo{Insert details about equation 18}

\subsubsection{Goodbye packet}
Our network have to be efficient, for these reasons it is useful to manage congestion, avoiding packet loss and re-transmissions. To do that we include a control mechanism to forecast buffer saturation in a pre-defined temporal horizon. Given a node $u$ we define as $b_{u}(t)$ the buffer occupancy of the node $u$ at time $t$. We estimate $\Delta$ as the remaining time before buffer saturation \todo{add here}.
If the estimation $\Delta$ is below a certain threshold the node $u$ start sending in broadcast \textit{goodbye packets} telling the neighbours to exclude $u$ from the forwarding table.

\subsubsection{Table update}
\todo{Check if it can be included above}
Since we are modeling a protocol aimed at hybrid-mobility  networks (both static and mobile nodes) it is necessary to keep this table updated both to tackle node mobility and node availability. Thus, each entry has an expiration date that is updated each time the node receives a new hello packet from the node that corresponds to the entry. Moreover, the new state may change the goodness of the relay, requiring to reorder such entry in the table. To make this mechanism works, it is necessary that every node periodically sends a new hello packet containing its state. 

\subsection{Relay selection}
\todo{Make an example figure about the selection}
Whenever a node needs to transmit data, it may need to select the next hop so as to reach the ground station. 
We propose two selection methods: \textit{backbone relay selection}, and \textit{bubble relay selection}.

\paragraph{Backbone relay selection}: this mechanism is meant to be implemented in fixed wing UAVs, which constitute the top layer of the network topology. Once packets reached these nodes they should be forwarded only to other fixed wing UAVs or directly to the ground station. Note that this is possible due to the high-power long-range transmitter that this type of UAVs can be equipped with. Therefore, the relay selection is either the top layer node with the greatest advancement towards the ground station, or the ground station itself if it is in the communication range.

\paragraph{Bubble relay selection}: this mechanism has as objective of bringing the packets to the top layer of the network topology. Thus, each node selects the node in its neighbourhood with the greatest advancement towards the sky (i.e., the z axis). As a tie breaking rule the node with the greatest advancement towards the ground station is picked as a relay.
\todo{another TBR may be congestion avoidance}




