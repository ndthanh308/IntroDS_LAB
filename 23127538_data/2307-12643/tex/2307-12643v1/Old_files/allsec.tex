\section{Introduction}
\label{sec:intro}

Underwater acoustic communication has gained a significant attention due to its promising applications in marine life exploration, natural resources finding, underwater navigation and military operations\cite{song2019editorial}. A plethora of research has been reported in the recent past covering the aspects of modem designs to signal processing techniques of underwater acoustic communication \cite{fattah2020survey}.  The broadcast nature of underwater acoustic communication makes the networks vulnerable to many types of malicious attacks. With passage of time, different attacks were studied with different counter mechanisms \cite{aman2022security}. One of the prominent attacks is the spoofing/impersonation attack, where a malicious node aims to mimic one of the legitimate nodes of the network in order to get access and destroy the integrity of the system  \cite{Ammar:VTC:2017}, \cite{waqas:Sensors:2021}. To counter such attacks, recently, the idea of physical layer authentication is explored for underwater acoustic communication networks \cite{Waqas:Access:2018,xiao2018learning,khalid2020physical,9676618}, where a physical layer feature serves as device fingerprint (can be thought as authentication key). In this regard, we report our investigation on the use of distance and angle of arrival as device finger prints for physical layer authentication, where closed form expressions for error probabilities are derived \cite{Waqas:Access:2018}. Next, \cite{xiao2018learning} study power delay profile of underwater acoustic channel using deep reinforcement learning approach to detect impersonation. Following our work, authors in \cite{khalid2020physical} report two angles of arrival (azimuth and elevation) to counter impersonation in line of sight underwater acoustic communication. Recently, authors in \cite{9676618}  report time reversal resonating strength based physical layer authentication in underwater acoustic sensor networks.

In this work, for the first time, we study position of transmitting node as device fingerprint to decide for impersonation attacks at the physical layer. The detailed contributions of this paper are given below:
\begin{itemize}
    \item We used Time of Arrival (ToA) based localization to extract the coordinates of transmitting node. For this purpose, we exploit our previously proposed best unbiased estimator for ToA estimation. At the end, we find the distribution of the inherit uncertainty in the estimation process.
    
    \item Next, We build a test statistic for binary hypothesis testing in order to decide whether the estimated position belongs to legitimate node or malicious. We derive the distributions of conditional events (i.e., test statistic given legitimate or malicious transmissions) and provide closed form expressions for the two error probabilities (i.e., false alarm and missed detection).
    
\end{itemize}






{\bf Outline.} The rest of this paper is organized as follows. Section \ref{sec:sys-model} presents system model, Section \ref{sec:methods} provides the proposed physical layer authentication mechanism, Section \ref{sec:results}  discusses the simulation results and  Section \ref{sec:conclusion} concludes the paper.
\section{System model}
\label{sec:sys-model}

We consider static two nodes setup (i.e., legitimate Autonomous Underwater Vehicle (AUV) and a malicious AUV) with three anchored/reference nodes, which are used to estimate the position of transmitter AUVs, as shown in Fig. \ref{fig:sysmodel}. We assume that the anchored nodes are perfectly synchronized and connected to ground station/surface ship via secured channel. We consider a one-way authentication system. Specifically, the considered malicious AUV is an intruder which sends malicious packets to the system from time to time while trying to impersonate legitimate AUV. Therefore, a systematic framework is needed to authenticate the sender of every packet it receives. This way, ground station  can reject packets from malicious node. We consider range-based localization technique (i.e., Time-of-Arrival (ToA)) to extract position coordinates from the received signals at the reference nodes. 



We assume that the one-way authentication channel in Fig. \ref{fig:sysmodel} is time-slotted. That is, packets arrive at reference nodes at discrete-time instants $t_m$ where $t_m-t_{m-1}=T$ is the time-gap between two successively received packets at reference nodes. Moreover, each received packet is $T_p < T$ seconds long. 

% Figure environment removed

\section{Proposed PHY layer authentication framework}
\label{sec:methods}

\subsection{Position estimation}
\subsubsection{Distance Estimation}
Let $t_i$ is the arrival time of a signal (ToA) at the $i$-th reference node. We exploit our previous work \cite{Waqas:Access:2018} to estimate ToA of the transmitter nodes  in the presence of heavily frequency dependent pathloss and colored nose using best unbiased estimator (i.e., meeting Crammer Rao bound (CRB)). Specifically, the distance of transmitter node from $i-$th anchored node is estimated using distance equation, given below:
\begin{align}
\hat{d_i}=c\hat{t}_i    
\end{align} 
where $c$ is the speed of sound in water, $\hat{t}_i \sim N(t_i,\hat{\sigma}_i^2)$ is the estimated ToA with $t_i$ is the actual time of arrival and $\sigma_i^2$ is the variance of the best unbiased estimator \cite{Waqas:Access:2018}, which can be expressed as \cite{Waqas:Access:2018}: 
\begin{align}
    \hat{\sigma}_i^2 =  \frac{ PL_i(f)}{4P\mathbf{\Hat{s}^TC^{-1}\Hat{s}}},
\end{align}
where $P$ is the transmit power, $c$ is the underwater speed of sound, \mathbf{C} is the covariance matrix of colored noise. $\mathbf{\Hat{s}}$ is the partial derivative of pseudo random sequence $\mathbf{s}$\footnote{A random sequence or message a transmitter needs to transmit for ToA estimation \cite{Waqas:Access:2018} at the anchored nodes.}. The estimated distance can be written as: $\hat{d}_i^2=d_i+n_i$, where
where $\hat{d_i}$ is the estimated distance, $d_i$ is the true distance of transmitter to the $i$-th anchored node, $n_i\sim N(0,\sigma^2)$  is the uncertainty/noise in the estimator with variance $\sigma^2= \frac{c^2 PL_i(f)}{4P\mathbf{\Hat{s}^TC^{-1}\Hat{s}}}$. The pathloss $PL_i(f)$ of a transmitter to $i$-th anchored node is given as \cite{stojanovic2007relationship}:
\begin{align}
    PL_i(f)_{[dB]}= \nu 10\log_{10}(d)+d\alpha(f)_{[dB]},
\end{align}
where $d\alpha(f)_{[dB]}=\frac{0.11f^2}{1+f^2}+\frac{44f^2}{4100+f^2}+2.75\times 10^{-4}f^2+0.003$ with $f$ as operating frequency.
\subsubsection{Coordinates Extraction}
Using the definition of the standard Euclidean distance, $d_i$ can be written as
\begin{align}
d_i^2=(x-x_i)^2+(y-y_i)^2    
\end{align}
Assuming high SNR regime we can write $\hat{d_i}^2\approx(d_i+n_i)^2=d_i^2+2n_id_i$.
Now  $\hat{d_i}^2$ can be expressed as
\begin{align}
\hat{d_i}^2=(x-x_i)^2+(y-y_i)^2+2n_i((x-x_i)^2+(y-y_i)^2)^2    
\end{align}
Now for all ''$i_s$'' the equation set resulted from the above equation can be written in matrix-vector form as
\begin{align}
AX+N=b
\end{align}
where 
\begin{align}
A=-2
\begin{bmatrix}
    x_1 & y_1 & -0.5   \\
    . & . & .   \\
     . & . & .  \\
    x_L & y_L & -0.5      
  \end{bmatrix} ,  N=2
  \begin{bmatrix}
    n_1d_1   \\
    .    \\
    . \\
     n_Ld_L     
  \end{bmatrix}, \nonumber \\ X=\begin{bmatrix} 
    x   \\
    y  \\
     x^2+y^2     
  \end{bmatrix} 
\ and 
\ b=\begin{bmatrix}
      \hat{d_1}^2-x_1^2-y_1^2 \\
    .   \\
     .\\
       \hat{d_L}^2-x_L^2-y_L^2   
  \end{bmatrix} \nonumber 
\end{align}
we can verify that $E[N]=0$, therefore above can be approximated as 
\begin{align}
\label{eq:aprox}
AX\approx b
\end{align}
now Eq. \ref{eq:aprox} is a least square problem, the solution to $S$ can be obtained as
\begin{align}
\label{eq:LS}
&\min_{X}\Vert b-AX \Vert_2^2\\
&=\min_X(b-AX)^T(b-AX) \nonumber
\end{align}
one can verify that Eq. \ref{eq:LS} is a convex function. To find the minimum we take the gradient $\nabla_X$ and equate it to zero gives us
 \begin{align}
X=(A^TA)^{-1}A^Tb
\end{align}
where $(A^TA)^{-1}A^T=A^{\dagger}$ also known as pseudo inverse of $A$.
Now, we are only interested in $x$ and $y$ components of $X$. so, the desired vector can be written as
\begin{align}
\hat{X}=[X(1) \ X(2)]^T
\end{align}
Next, we derive the distribution of $\hat{X}$, we can write Eq. () as:
From here we avoid to use the index $k$ for simplicity,
As $A^{\dagger}$ is a $3*L$ dimension matrix and we know that our desired component of $X$ lies in $\hat{X}$. Now let $\hat{A^{\dagger}}=A_{2*L}^{\dagger}$, then (7) can also be written as
\begin{align}
\hat{X}=\hat{A^{\dagger}}b
\end{align}
Or more precisely
\begin{align}
\hat{X}=\underbrace{\hat{A^{\dagger}}\begin{bmatrix}
      {d_1}^2-x_1^2-y_1^2 \\
    .   \\
     .\\
       {d_L}^2-x_L^2-y_L^2   
  \end{bmatrix}}_{True \ Co-ordinates} + \underbrace{2\hat{A^{\dagger}}\begin{bmatrix}
      n_1d_1\\
    .   \\
     .\\
       n_Ld_L   
  \end{bmatrix}}_{Noise}
\end{align} 

%\subsection{Kalman Filter}
%The state space equation of Kalman filter can be expressed as

\subsection{Hypothesis Testing}
 Let $\hat{
 X}_A$ the actual coordinates vector of Alice and $\hat{
 X}_E$ of Eve. $\text{H}_0$ is the hypothesis (also known as Null hypothesis) that Alice is the transmitter while $H_1$ is the hypothesis (also known as alternate hypothesis) that Eve is the transmitter  then we define test statistics as:
 
 \begin{align}
    \text{TS}=\Vert (\hat{A^{\dagger}}^T\hat{A^{\dagger}})^{-1}\hat{A^{\dagger}}^T\left( \hat{X}-\hat{
 X}_A \right)\Vert_2^2.
 \end{align}
Now, Binary hypothesis test can be defined as 
\begin{equation}
	\label{eq:H0H1}
	 \begin{cases} \text{H}_0 (\text{no impersonation}): & \text{TS}=\Vert \hat{A^{\dagger\dagger}}\left( \hat{X}-\hat{
 X}_A \right) \Vert_2^2. < \epsilon_{th} \\ 
                   H_1 (\text{impersonation}): & \text{TS}=\Vert \hat{A^{\dagger\dagger}} \left( \hat{X}-\hat{
 X}_A \right) \Vert_2. > \epsilon_{th} \end{cases}
\end{equation}
where $\hat{A^{\dagger\dagger}}=(\hat{A^{\dagger}}^T\hat{A^{\dagger}})^{-1}\hat{A^{\dagger}}^T$, $\epsilon_{th}$ is a predefined threshold. 
Equivalently, we have:
\begin{align} 
\label{eq:bht}
\text{TS} \gtrless_{\text{H}_0}^{H_1} {\epsilon_{th}}.
\end{align}
At this stage, we need to find the error probabilities (i.e., false alarm and missed detection). False alarm can be defined as the probability that BHT decides legitimate node as malicious node while missed detection is the probability that BHT decides malicious node as legitimate node. The false alarm $P_{fa}$ can be expressed as:
\begin{align}
    P_{fa}=Pr(\text{TS} \mid \text{H}_0 > \epsilon_{th})
\end{align}
To compute the above probability we need to find the distribution of the conditional event $\text{TS} \mid \text{H}_0$, which can be expressed as
\begin{align}
\label{eq:ts_H_0}
    \text{TS} \mid \text{H}_0&=\Vert \hat{A^{\dagger\dagger}}\left(\hat{
 X}_A+\mathbf{n}_A-\hat{
 X}_A\right) \Vert_2^2=\Vert \hat{A^{\dagger\dagger}} \mathbf{n}_A\Vert_2^2 \\
 &=4\left( (d_1^An_1^A)^2+(d_2^An_2^A)^2+...+(d_L^An_L^A)^2 \right) \nonumber \\
 &= \sum_{i=1}^L (2 d_i^An_i^A)^2 = \sum_{i=1}^L  (\hat{n}_i^A)^2  \nonumber
\end{align}
where $\mathbf{n}_A={2\hat{A^{\dagger}}\begin{bmatrix}
      n_1^Ad_1^A \
    .   \
     . \
       n_L^Ad_L^A   
  \end{bmatrix}} ^T$ with $d_i^A$ where $i \in \{1...L\}$ denotes the distance of legitimate node from $i$-th reference node and $\hat{n}_i^A \sim N(0,4(d_i^A)^2\sigma^2_i), \forall i$.  We can write $(2d_i^An_i^A)^2\sim \Gamma(1/2, 8(d_i^A)^2\sigma_i^2)$.
  So, the probability of false alarm is given as \cite{ansari2017new}:
  
  \begin{align}
    P_{fa}  =\sqrt{1/2} \prod_{i=1}^L \sqrt{\frac{1}{\kappa_i}} \ H_{L+1,L+1}^{0,L+1} \left[ e^{\epsilon_{th}} \mid \begin{matrix}
\Xi_L^1, (1,1,1)\\
\Xi_L^2, (0,1,1)
\end{matrix} \right],
  \end{align}
  where $\kappa_i=8(d_i)^2\sigma_i^2$, $ H_{.,.}^{.,.}$ is Fox-H function, $\Xi_L^1$ and $\Xi_L^2$ represent the bracket terms as: $\Xi_L^1= (1-\frac{0.5}{8(d_1^A)^2\sigma_1^2},1,0.5),...(1-\frac{0.5}{8(d_L^A)^2\sigma_L^2},1,0.5)$, $\Xi_L^2= (-\frac{0.5}{8(d_1^A)^2\sigma_1^2},1,0.5),...(-\frac{0.5}{8(d_L^A)^2\sigma_L^2},1,0.5)$.
  
  Now, to find the probability of missed detection, we need to find $\text{TS} \mid H_1$, which can be expressed as 
  \begin{align}
      \text{TS} \mid H_1&=\Vert \hat{A^{\dagger\dagger}}\left(\hat{
 X}_E+\mathbf{n_E}-\hat{
 X}_A\right) \Vert_2^2=\Vert \hat{A^{\dagger\dagger}} \mathbf{n_E}+\mathbf{d_{EA}}\Vert_2^2 \\
 &=4( (d_1^En_1+(d_1^E)^2-(d_1^A)^2)^2+... \nonumber \\ &..+((d_L^En_L)+(d_L^E)^2-(d_L^A))^2)
 = \sum_{i=1}^L (2 d_in_i)^2,  \nonumber
  \end{align}
  where $\mathbf{n}_E={2\hat{A^{\dagger}}\begin{bmatrix}
      n_1^Ed_1^E \
    .   \
     . \
       n_L^Ed_L^E   
  \end{bmatrix}} ^T$ with $d_i^E$ where $i \in \{1...L\}$ denotes the distance of malicious node from $i$-th reference node, $\hat{n}_i^E \sim N(0,4(d_i^E)^2\sigma^2_i), \forall i$ and $\mathbf{d_{EA}}={\begin{bmatrix}
      (d_1^E)^2-(d_1^A)^2 \
    .   \
     . \
      (d_L^E)^2-(d_L^A)^2  
  \end{bmatrix}} ^T$.
  Now, probability of missed detection can be expressed as \cite{castano2005distribution}:
  
 \begin{align}
     P_{md}= \frac{e^{-\frac{\epsilon_{th}}{2\beta}}}{(2\beta)^{\frac{L}{2}}} \frac{\epsilon_{th}^{\frac{L}{2}}}{\Gamma (\frac{L}{2}+1)} \sum_{k\geq 0} \frac{k!\zeta_k}{(\frac{L}{2}+1)_k}\mathbf{L}_k^{\frac{L}{2}}(\frac{(L+2)\epsilon_{th}}{4\beta \mu_0}),
 \end{align}
  with $\beta >0$, $\mu_0 >0$, $\zeta_k=\frac{\sum_{j=0}^{k-1} \zeta_j \xi_{k-j}}{k}$, \\ $\zeta_0=2(\frac{L}{2}+1)^{\frac{L}{2}+1}\exp{-\frac{1}{2}\sum_{i=1}^L \frac{\lambda_i \gamma_i (\frac{L}{2}+1-\mu_0)}{\beta \mu_0+\gamma_i (\frac{L}{2}+1-\mu_0)}} \bullet\frac{\beta^{\frac{L}{2}+1}}{\frac{L}{2}+1-\mu_0)}\prod_{i=1}^L (\beta \mu_0+\gamma_i (\frac{L}{2}+1-\mu_0))^{-\frac{1}{2}}$, \\
  $\xi_j=-\frac{j\beta(\frac{L}{2}+1)}{2\mu_0}\sum_{i=1}^L \lambda_i \gamma_i(\beta-\gamma_i)^{j-1} (\frac{\mu_0}{(\beta \mu_0+\gamma_i (\frac{L}{2}+1-\mu_0))})^{j+1}+(\frac{-\mu_0}{\frac{L}{2}+1-\mu_0})^j+\frac{1}{2}(\frac{\mu_0(\beta-\gamma_i)}{(\beta \mu_0+\gamma_i (\frac{L}{2}+1-\mu_0))})^{j+1}$, $\lambda_i=\frac{(d_1^E)^2-(d_1^A)^2}{\sigma_i}$, $\gamma_i=4(d_i^E)^2\sigma_i$, $\Gamma(.)$ indicates gamma function, $\mathbf{L}_k^{.}$ is the $k$-th  generalized Laguerre polynomial. 


\section{Simulation results}
\label{sec:results}
We use Python for simulations. We consider a rectangular area of $1000\times1000$ m$^2$, $L=3$ total number of anchored nodes with positions $[0,500]$, $[-500,-500]$, $[-500,500]$, legitimate AUV is fixed at origin (i.e., $X_A=[0,0]$ and position of malicious AUV is uniformly distributed. Further, we set center frequency $f=10$KHz, speed of sound $c=1500$m/s and spreading factor $\nu=1.5$. 

We choose transmit power $P$ (in dB Pascal (pa)) as controlled parameter or independent variable\footnote{Typically, the inverse of estimator variance is chosen as independent variable in the literature but as we can see that this parameter is different for different transmitters and anchored nodes. Therefore, we need a common parameter for fair comparison. } to generate the figures. We sweep $P$ from $0$ to $100$ dB pa because of its adoption  in the literature for underwater acoustic transmitters \cite{stojanovic2007relationship}.
% Figure environment removed
Fig. \ref{fig:fa} demonstrates the trend of false alarm resulted from binary hypothesis testing Eq. \ref{eq:bht} with increase in the transmit power lowers the false alarm. This is due to the fact that increase in $P$ shrinks the variances $\sigma_i^2$ of the estimator  which makes the estimated distances close to the actual distances. On the other hand, the increase in the threshold $\epsilon_{th}$ produces low false alarm but it has a affect on missed detection.
% Figure environment removed
Fig. \ref{fig:md} demonstrates the behaviour of missed detection. We observe that missed detection $P_{md}$ decreases with increase in $P$ and the hypothesis testing threshold $\epsilon_{th}$ has a negative impact on $P_{md}$. This is due to the fact that $\epsilon_{th}$ extends the accepting range around actual position of legitimate AUV which makes some of the noisy estimates of malicious AUV appears in the accepting range of legitimate AUV. 















