\documentclass[9pt,conference]{IEEEtran}



% --- Headings --- %
\usepackage[utf8x]{inputenc}
\usepackage{amscd}
\usepackage{amsmath}
%\usepackage{amssymb}
\usepackage{amsthm}

\usepackage[colorinlistoftodos]{todonotes}

\usepackage{thmtools}
\usepackage{thm-restate}
\usepackage{mathtools}
\usepackage[full]{complexity}
\usepackage{longtable}

%\usepackage[usenames,dvipsnames]{xcolor}
\usepackage{xcolor}
% % tables
 \usepackage{array}

\usepackage{bbm}
\usepackage{comment}
\usepackage{enumerate}
\usepackage{floatrow}


\usepackage{parallel,enumitem}

\usepackage{xspace}
\usepackage{paralist}
\usepackage{xifthen}
\usepackage{url}
\usepackage{csquotes}
% \usepackage{graphicx}
\usepackage{wrapfig}
\usepackage{multirow}
\usepackage[binary-units=true]{siunitx}

\usepackage{tikz}
\usetikzlibrary{trees,decorations,arrows,automata,shadows,positioning,plotmarks,backgrounds,shapes}
\usetikzlibrary{calc,matrix,fit,petri,decorations.markings,decorations.pathmorphing,patterns,intersections,decorations.text}
\usepackage{pgfplots}
\usepackage{pgfplotstable}

\tikzstyle{mystate}=[state,inner sep=3pt,minimum size=20pt,line width=0.2mm]
\tikzstyle{fstate}=[state,accepting,inner sep=2pt,minimum size=3pt]
\tikzstyle{istate}=[state,initial,inner sep=2pt,minimum size=3pt]
\tikzstyle{mysquare}=[inner sep=3pt,minimum size=15pt,line width=0.2mm]
\tikzstyle{fmysquare}=[inner sep=3pt,minimum size=15pt,line width=0.5mm,accepting]
\newcommand{\SFSAutomatEdge}[5]{\path[->](#1) edge[#4,line width=0.2mm] node[#5] {\ensuremath{#2}} (#3);}
\usepackage{subcaption}
\usepackage{tabularx}
\usepackage{booktabs}
\usepackage{xfrac}

\usepackage{etoc}
\etocsettocdepth{3}

% \usepackage{minitoc}

% \usepackage{titletoc}
% 
% \newcommand\DoToC{%
%   \startcontents
%   \printcontents{}{2}{\textbf{Contents}\vskip3pt\hrule\vskip5pt}
%   \vskip3pt\hrule\vskip5pt
% }

% Special Font
\newcommand{\cc}{\centering}
\newcommand{\bs}{\boldsymbol}
\newcommand{\fs}{\mathsf}
\newcommand{\bb}{\mathbb}
\newcommand{\cl}{\mathcal}
\newcommand{\bcl}[1]{\boldsymbol{\mathcal{#1}}}

% Abbreviations
\newcommand{\ts}{\textstyle}
\newcommand{\ie}{\emph{i.e.}, }
\newcommand{\eg}{\emph{e.g.}, }
\newcommand{\etal}{\emph{et al.}}
\newcommand{\st}{\ensuremath{\mathrm{s.t.}}\xspace}

% text
\newcommand{\ong}{\emph{on-the-grid} }
\newcommand{\ofg}{\emph{off-the-grid} }

\newcommand{\onestep}{one-step }
\newcommand{\twosteps}{two-steps }

% Operators
\newcommand{\argmin}{\mathrm{arg\,min}}
\newcommand{\argmax}{\mathrm{arg\,max}}
\newcommand{\amin}[1]{\underset{#1}{\argmin}\xspace}
\newcommand{\amax}[1]{\underset{#1}{\argmax}\xspace}

\newcommand{\diag}{\mathrm{diag}}
\newcommand{\sign}{\mathrm{sign}}
\newcommand{\prox}[1]{\mathrm{prox}_{#1}\;}

\newcommand{\sps}[3]{#3\langle#1,\,#2 #3\rangle}
\newcommand{\pder}[1]{\frac{\partial }{\partial #1}}

\newcommand{\mathif}{\mathrm{if}\;\;}
\newcommand{\tr}{^\top}

\newcommand{\trace}{\mathrm{Tr}}

% Misc.
\newcommand{\cel}{{\sf c}}
\newcommand{\imsymb}{\mathrm{j}}
\newcommand{\cn}[1]{\bb C \cl N(0, #1)}

% comments
\newcommand{\GM}[1]{ 
\ifthenelse{\boolean{comments}}
{{\footnotesize \color{blue} [\textbf{GM:} #1]}}
{}
}
\newcommand{\DRAFT}[1]{
\ifthenelse{\boolean{comments}}
{{\color{orange}{#1}}}
{}
}
\usepackage{amsmath}
\usepackage{amssymb}
\usepackage{amsthm}
\usepackage{mathtools}
\usepackage{comment}
\usepackage{todonotes}
\usepackage{float}
\usepackage[algo2e,ruled,linesnumbered]{algorithm2e}

% for restatable
\usepackage{thmtools,thm-restate}
\usepackage{physics}
% %% >> for restatable links
% \usepackage{xpatch}
% \usepackage{xcolor}
% \usepackage{scalerel}

% % a flag to turn on and off
% \newif\ifmarginprooflinks
%     \marginprooflinkstrue
%     % \marginprooflinksfalse


% %% STEP 1: patch restatable so there are backward links on recall
% \makeatletter
% \xpatchcmd{\thmt@restatable}% Edit \thmt@restatable
%    {\csname #2\@xa\endcsname\ifx\@nx#1\@nx\else[{#1}]\fi}% Replace this code
%    {\ifthmt@thisistheone%
%     \csname #2\@xa\endcsname\ifx\@nx#1\@nx\else[{#1}]\fi% same as before
%     %except with also marginparbox
%    \else\fi} {}{\typeout{FIRST PATCH TO THM RESTATE FAILED}}
% \xpatchcmd{\thmt@restatable}% A second edit to \thmt@restatable
%    {\csname end#2\endcsname}
%    {\ifthmt@thisistheone\csname end#2\endcsname\else\fi}
%    {}{\typeout{FAILED SECOND THMT RESTATE PATCH}}


% \newcommand{\recall}[1]{\medskip\par\noindent{\bf \Cref{thmt@@#1}.} \begingroup\em \noindent
%    \expandafter\csname#1\endcsname* \endgroup\par\smallskip}

% %% STEP 2: make forward links to restatable.
% \setlength\marginparwidth{1.55cm}
% \let\oldmarginpar\marginpar
% \renewcommand{\marginpar}[1]{%
%     \leavevmode%
%     \oldmarginpar{#1}%
%     \ignorespacesafterend\ignorespaces}
% \newsavebox\marginprooflinkbox
% \newenvironment{linked}[3][]{%
%     \def\linkedproof{#3}%
%     \def\linkedtype{#2}%
%     \ifmarginprooflinks%
%     \sbox\marginprooflinkbox{%
%         \centering%
%         \hyperref[proof:\linkedproof]{%
%         \color{blue!30!white}%
%         \scaleleftright{$\Big[$}{\,\mbox{\footnotesize\centering\tt\begin{tabular}{@{}c@{}}
%             link to\\[-0.15em]
%             proof
%         \end{tabular}}\,}{$\Big]$}}~}
%     \fi
%         \restatable[#1]{#2}{#2:#3}\label{#2:#3}%
%     \reversemarginpar	\ifmarginprooflinks\marginpar{\vspace{-1ex}\usebox\marginprooflinkbox}\fi
%     }%
%     {\sbox\marginprooflinkbox{}\endrestatable}
% \newcounter{proofcntr}
% \newenvironment{lproof}{\begin{proof}\refstepcounter{proofcntr}}{\end{proof}}

\newcommand{\vect}[1]{\ensuremath{\mathbf{#1}}}

%% Useful
\newcommand{\p}[1]{\left( #1 \right)}
\newcommand{\br}[1]{\left[ #1 \right]}


%\newcommand{\ev}[1]{\mathbb{E}\left[{#1}\right]}
\newcommand{\evd}[2]{\mathbb{E}_{#1}\left[{#2}\right]}

%% Algortihm notations
\newcommand{\bigO}[1]{O \left( #1 \right )}

%% Calibration
\newcommand{\I}[1]{\mathbb{I}\left[#1\right]}       % Indicator
\newcommand{\calerr}{\mathrm{calerr}}   % Calibration error

\newcommand{\A}{\mathcal{A}}    % Algorithm
\newcommand{\Ber}{\mathrm{Ber}}
\newcommand{\Ecover}{\event^{\textrm{cover}}}   % Event that covered epochs exist
\newcommand{\Enegl}{\event^{\textrm{negl}}}     % Event that negligible epochs exist
\newcommand{\Epoch}{\mathsf{Epoch}}
\newcommand{\eps}{\epsilon}     % epsilon
\newcommand{\Etruth}{\event^{\textrm{truth}}}   % Event that all epochs are truthful
\newcommand{\event}{\mathcal{E}}    % Events
\newcommand{\Ex}[2]{\operatorname*{\mathbb{E}}_{#1}\left[#2\right]}  % Expectation
\newcommand{\Int}{\mathcal{I}}      % Interval
\newcommand{\poly}{\operatorname*{poly}}    % Polynomial
\newcommand{\pr}[1]{\Pr\left[#1\right]}     % Probability
%\newcommand{\red}[1]{{\color{red} #1}}

\newcommand{\red}[1]{\textcolor{red}{#1}}
\newcommand{\blue}[1]{\textcolor{blue}{#1}}

\newcommand{\SPinner}{\mathsf{SP}^{\textrm{inner}}}
\newcommand{\SPouter}{\mathsf{SP}^{\textrm{outer}}}
\newcommand{\Tact}{T^{\mathrm{actual}}}     % Actual stopping time
\newcommand{\prodspace}{\mathcal{X}\times A \times \mathcal{Y}}


%% Fair ERM Notation
\newcommand{\error}[1]{ \left| \mathbb{E}_{(x,y) \sim \mathcal{D}} \ [\one (#1(x) \neq y)] - \ \mathbb{E}_{(x,y) \sim \mathcal{D}} \ [\one (h^*(x) \neq y)] \right|}
\newcommand{\htilde}{\tilde{h}}
\newcommand{\hhat}{\hat{h}}
\newcommand{\hstar}{h^*}
\newcommand{\hclass}{\mathcal{H}}
\newcommand{\posrate}[1]{ P_{(x,y) \sim \DA} [#1 (x)=1]}

\newcommand{\DAC}{\widetilde{\mathcal{D}}_A}
\newcommand{\DBC}{\widetilde{\mathcal{D}}_B}
\newcommand{\RA}{P_{(x,y) \sim \dist } [x \in A]}
\newcommand{\RB}{P_{(x,y) \sim \dist} [x \in B]}
\newcommand{\normalF}{F}
\newcommand{\corruptF}{\widetilde{F}}


\newboolean{comments}
\setboolean{comments}{True}


\begin{document}



% --- Acronyms --- %
\newacronym{bomp}{BOMP}{Block Orthogonal Matching Pursuit}
\newacronym{fmcw}{FMCW}{Frequency-Modulated Continuous Wave}
\newacronym{tx}{TX}{transmitter}
\newacronym{rx}{RXs}{receivers}
\newacronym{fft}{FFT}{Fast Fourier Transforms}


%% Comments and edition commands
\newcommand{\LJc}[2][\footnotesize]{{#1 \color{red}{\bf [LJ: #2]}}}
\newcommand{\GMc}[2][\footnotesize]{{#1 \color{blue}{\bf [GM: #2]}}}
\newcommand{\TFc}[2][\footnotesize]{{#1 \color{magenta}{\bf [TF: #2]}}}
\newcommand{\LVc}[2][\footnotesize]{{#1 \color{green}{\bf [LV: #2]}}}
\newcommand{\MDc}[2][\footnotesize]{{#1 \color{purple}{\bf [MD: #2]}}}
\newcommand{\edt}[1]{\textcolor{blue}{#1}}




%\title{Accelerating direct signal processing for multistatic FMCW radar with \emph{Grid Hopping}}
\title{\vspace{-3mm} Grid Hopping: Accelerating Direct Estimation Algorithms for Multistatic FMCW Radar}
\author{\IEEEauthorblockN{
        Gilles Monnoyer \IEEEauthorrefmark{1}, 
        Thomas Feuillen\IEEEauthorrefmark{2}, 
        Maxime Drouguet\IEEEauthorrefmark{1}, 
        Laurent Jacques\IEEEauthorrefmark{1}, 
        Luc Vandendorpe\IEEEauthorrefmark{1}
    }
    \IEEEauthorblockA{
        \IEEEauthorrefmark{1} ICTEAM UCLouvain (Belgium).
        \IEEEauthorrefmark{2} SPARC, SnT, UniLu (Luxembourg) \vspace{-3mm}}
}
\maketitle
\begin{abstract} 
    %\LJc{I have adapted this abstract in its formulation. Please proofread}
    This paper presents a novel signal processing technique, coined grid hopping, as well as an active multistatic \gls{fmcw} radar system designed to evaluate its performance. 
    The design of grid hopping is motivated by two existing estimation algorithms.  
    %\GMc{Je veux plutot dire par la que le grid hopping est un algo qui est une sorte d'hybridation entre les méthodes directe et indirecte}\LJc{Ok. Je parlerais pas encore de comparaison à ce stade alors, ça vient à la fin de l'abstract.}
    The first one is the \emph{indirect} algorithm estimating ranges and speeds separately for each received signal, before combining them to obtain location and velocity estimates.
    The second one is the \emph{direct} method jointly processing the received signals to directly estimate target location and velocity.
    While the direct method is known to provide better performance, it is seldom used because of its high computation time.
    Our grid hopping approach, which relies on interpolation strategies, offers a reduced computation time while its performance stays on par with the direct method.
    We validate the efficiency of this technique on actual \gls{fmcw} radar measurements and compare it with other methods.
\end{abstract}
\vspace{-3mm}
\section{Introduction}
\vspace{-.5mm}
Multistatic radar systems are being increasingly used in multiple applications such as low-cost monitoring or automotive systems~\cite{lin2018, saponara2018, capobianco2018}. % in civil, military and imaging applications, 
%\emph{multistatic} or \emph{distributed MIMO} radar systems have taken the front of the scene to push radar capabilities to the next level. 
A multistatic radar is composed of one or more \gls{tx} and at least two \gls{rx} with widely spread locations. %around an area to be observed.
This enables the estimation of the location and velocity of one or multiple targets with an increased diversity by viewing the targets from different angles~\cite{haimovich2008, stinco2014, sun2015}.
% While passive bistatic and multistatic radars are vastly studied in the literature~\cite{}, active multistatic radar have more recently emerged for shorter range applications such as the netRad system which is used for the detection of UAV~\cite{}.
In this work, we developed an active K-band \gls{fmcw} multistatic radar 
%working in the K band and 
designed for 
%the estimation of the location and velocity 
detection of cars in civil applications. 
The focus of this paper is the comparison 
of the performance 
of different estimation algorithms by means of actual measurements. 
%This paper focuses on the comparison of the performance of different estimation algorithms by means of actual measurements. 
%In this paper we provide measurement results from which we compare multiple signal processing techniques for multistatic radar.
Moreover, we propose a novel processing methodology, called grid hopping, which appears as an efficient trade off between the computational complexity and the performance when compared to existing algorithms.

Conventional processing algorithms for multistatic radars can be classified into two categories. 
Indirect methods estimate ranges and speeds of the targets separately for each received signal.
Then they combine the estimates to provide location and velocity estimates through multilateration. 
Direct methods, on the other hand, process all received signals jointly to directly estimate locations and velocities, commonly by maximizing a decision function evaluated over a grid of values~\cite{eldar2010, gogineni2010, berger2011}.
%Signal processing algorithms for multistatic radars can be separated in two main classes. 
%On the one hand, \emph{indirect} methods perform first the estimation of the ranges and speeds of the targets separately for each of the received signals.
%In a second step, the indirect method combines those estimates in a multilateration process to provide location and velocity estimates.
%The speeds are similarily combined to obtain the velocity estimates.
%On the other hand, direct methods gather all received signals and process them jointly to directly provide estimates for the locations and velocities.
%This is performed by maximizing a decision function which is evaluated in a set of candidate values for the parameters of interest, thereby defining a grid of values.
In short, the indirect algorithms are faster as they require the evaluation of a few 2D \gls{fft} that correspond to correlations in the range-Doppler domain.
The direct algorithms require the discretization of the 4D location-velocity domain which makes them 
%\GMc{Plutot them car je parle des algo directs. Direct étant vu comme un classe de plusieurs algorithms possibles qui suivent la même philosophie.} 
computationally heavier.
Yet, the results of a direct processing are more robust to non-idealities. 
%such as noise, reverberation and target occlusion. 
%Moreover, unlike indirect methods they do not suffer from the challenging data-association problem which is sometimes solve with tracking~\cite{liu2020}. 
%This issue can be solved with tracking~\cite{} but the direct method enables to estimate multiple target without ambiguity prior the tracking, enhancing its robustness.
%The reader can refer to~\cite{} for application of the \gls{bomp} algorithm~\cite{eldar2010} to multistatic radars. 

Grid hopping is inspired by contributions for sound source localization applications~\cite{cobos2011, cobos2017, dietzen2021}.
%In brief, we use interpolation of waveforms 
%(\eg \emph{polar}~\cite{simoncelli2011, duarte2013}) 
%to approximate the evaluation of the decision function used by the direct method.
%The interpolation is performed on the output of the 2D \gls{fft} used in the indirect method.
In brief, we approximate the evaluation of the decision function used by the direct method with an interpolation performed on the output of the 2D \gls{fft} used in the indirect method.
Thereby, the grid hopping enables us to keep the philosophy of the direct method with a computation time similar to the indirect method, and hence provides a compromise between the two strategies.

\textbf{Notations:} Vectors and matrices are denoted by lowercase and uppercase bold letters, respectively. Given a matrix $\bs A$, we use $\bs A^*$, $\bs A^\top$ and $\bs A[i]$ to respectively denote the conjugate, the transpose and the $i$-th row of $\bs A$. The scalar product and the outer product between two vectors respectively read $\sps{\cdot}{\cdot}{}$ and $\otimes$. Finally $\imsymb = \sqrt{-1}$.

%Hopping is ...

%Measurement, we present here a preview of the results. only one target and we already see the benefit of hopping. 
%We describe the setup.




\section{Radar Model}

This section introduces the simplified model for the signals acquired by a multistatic \gls{fmcw} radar. 
%which is used to design the processing algorithms presented in the next section. 
The model is similar to the one derived in~\cite{monnoyer2019}.
The system uses a modulation made of linear chirps
%such as depicted in Fig.~\ref{} 
characterized by a carrier frequency $\carrier$, and bandwidth $\band$ and a chirp duration $\tchirp$. 
A processed frame is composed of the acquisition of $\nchirp$ chirps with $\nsamp$ uniform samples per chirp. 
We restrict here the model to one target.
%The application of algorithms for multiple targets to our measurements will be addressed in future work. 

We consider a multistatic radar composed of one \gls{tx} located in $\loctx$ and $\nobs$ \gls{rx} located in $\listobs{\locrx_\obsindex}$.
The whole system observes a scene that contains a single target assumed to be characterized by a single couple of location-velocity vectors ($\loc$, $\vel$), namely the \emph{parameters of interest}.
We consider subspaces $\domloc$ and $\domvel$ from which $\loc$ and $\vel$ are respectively known to be taken.
Each receiver indexed by $\obsindex$ provides a signal that depends on the range and the radial speed of the target, respectively denoted by $\range_{\obsindex}$ and $\speed_{\obsindex}$.
Those are the \emph{sensed parameters} and are defined with respect to the given receiver  
%More precisely, we denote respectively by $\range_{\obsindex}$ and $\speed_{\obsindex}$ the range and the speed of the target as viewed by the $\obsindex$-th receiver.
% and are related to the parameters of interest 
trough the sensing functions $\rfunsen_{\obsindex}$ and $\ufunsen_{\obsindex}$.
These function are defined as 
%for all $\loc\in\domloc$ and $\vel\in\domvel$ as
\begin{align}
    \range_{\obsindex} 
    & = \rfunsen_\obsindex (\loc) 
     = \ts \frac{2 \band}{\cel} \,\big( \|\loc - \loctx\|_2 + \|\loc - \locrx_\obsindex\|_2\big), 
    \\
    \speed_{\obsindex} 
    & = \ufunsen_\obsindex (\loc, \vel) 
     = \ts \frac{2 f_0 T_c}{\cel} \,\sps{\frac{\loc - \loctx}{\|\loc - \loctx\|_2} + \frac{\loc - \locrx_\obsindex}{\|\loc - \locrx_\obsindex\|_2}}{\vel}{\big}. 
\end{align}
%In the above, note that the expressions of the range and radial speed have arbitrarily been normalized with respect to their respective radar resolutions, \ie $\frac{2 \band}{\cel}$ and $\frac{2 f_0 T_c}{\cel}$ which in turns simplifies subsequent equations. 


%The signal provided by a single receiver, indexed by $\obsindex$, is a linear combination of the waveforms, sometimes called \emph{atoms} and denoted by $\atom$, originating from each of the $\sparsity$ targets.
The measurement provided by the $\obsindex$-th \gls{rx} is reshaped into a matrix denoted by $\matmeas_\obsindex \in \bb C^{\nchirp \times \nsamp}$ which can, under a few assumption described in~\cite{bao2014, feuillen2016, monnoyer2019}, be written as
\begin{equation}
    \label{eq:general-model}
    \ts \matmeas_\obsindex = \scatter_{\obsindex} \btom(\speed_\obsindex) \otimes \atom(\range_\obsindex) + \Noise_\obsindex,
\end{equation}
%\LJc{$\otimes \to$ Notations?} GM : ajouté dans les notations
where $\scatter_{\obsindex} \in \bb C$ is a scattering coefficient of amplitude and phase, and where $\Noise_\obsindex$ is a noise term.
The functions $\atom(\range_\obsindex)$ and $\btom(\speed_\obsindex)$ are waveforms or \emph{atoms} whose $\measindex$-th components are respectively given by 
$\atomi_\measindex(\range_{\obsindex}) = \exp({\imsymb 2\pi \,\range_{\obsindex}\, \measindex})$ 
and
$\btomi_\measindex(\speed_\obsindex) = \exp({\imsymb 2\pi \,\speed_{\obsindex}\, \measindex})$.
%The reader can refer to~\cite{bao2014, feuillen2016} for a study of these simplifications. 


\section{Direct and Indirect methods}
We summarize now the two classes of existing strategies to recover $\loc$ and $\vel$ from the set of measurements $\listobs{\matmeas_\obsindex}$, namely the indirect and the direct methods.
The indirect method starts by computing the estimates $\mestim\range_\obsindex$ and $\mestim\speed_\obsindex$ independently for each index $\obsindex$. 
This is done by maximizing the range-Doppler map obtained by computing the modulo of the 2D \gls{fft} of $\matmeas_\obsindex$. % which has been reshaped as a matrix of size $\nsamp \times \nchirp$.
Then, the location $\loc$ is estimated from $\mestim\loc^\mathrm{Ind}$, this quantity being computed by this multilateration
\begin{equation}
    \label{eq:multilat}
    \mestim\loc^\mathrm{Ind} = \ts \arg\,\max_{\loc' \in \domloc} \obsum |\rfunsen_\obsindex(\loc') - \mestim\range_\obsindex|^2.
\end{equation}
The velocity $\vel$ is estimated similarly from $\listobs{\mestim\speed_\obsindex}$ and $\ufunsen_\obsindex$.
Note that the computation time for~\eqref{eq:multilat} is negligible when compared to the computation time required for the $\nobs$ \gls{fft}s computed in the first step.  

The direct algorithm works on grids discretizing the spaces $\domloc$ and $\domvel$ and denoted by respectively $\grid\domloc$ and $\grid\domvel$. 
The factorized methodology for the direct method that is used in~\cite{monnoyer2019} leads to
\begin{equation}
    \label{eq:direct-locestim}
    \mestim\loc^\mathrm{Dir} = \ts \arg\max_{\loc \in \grid\domloc} \obsum \sum_{\chirpindex = 1}^{\nchirp} 
    \big| \sps{\matmeas_\obsindex [\chirpindex]} { \atom(\rfunsen_\obsindex(\loc)) } {\big} \big|^2
\end{equation}
for the direct estimation of the location.
Then, defining ${\bs g}_\obsindex = \matmeas^* \atom(\rfunsen_\obsindex(\mestim\loc^\mathrm{Dir}))$, we compute the velocity estimate as
\begin{equation}
    \label{eq:direct-velestim}
    \mestim\vel^\mathrm{Dir} = \ts \arg\,\max_{\vel \in \grid\domvel} \obsum |\sps{{\bs g}_\obsindex}{\atom(\ufunsen_\obsindex(\mestim\loc^\mathrm{Dir}, \vel))}{}|^2.
\end{equation}

% Figure environment removed

\section{Grid Hopping}
\label{sec:hopping}
Our grid hopping technique relies on an interpolation strategy enabling the approximation of the scalar products on which relies the direct method.
We formulate it for the estimation of location~\eqref{eq:direct-locestim}.
Given a receiver index $\obsindex$  and a column $\chirpindex$,
let us denote by $\casualvar$ the output of the \gls{fft} of the column $\matmeas_\obsindex [\chirpindex]$. 
Then, 
%by construction, 
the $i$-th component of $\casualvar$ corresponds to the correlation between $\matmeas_\obsindex [\chirpindex]$ and $\atom(\bar r_i)$ where $\bar r_i$ is the $i$-th range taken from the grid of frequencies (and hence ranges) that is implicitly used by the \gls{fft}.
The grid hopping aims at approximating the scalar products in~\eqref{eq:direct-locestim} as follows.
For all $\loc'\in\grid\domloc$, we identify a set of indexes $\cl I$ and a set of interpolation coefficients $\bs c$ such that 
\begin{equation}
    \label{eq:approx-scalprod}
    \sps{\matmeas_\obsindex [\chirpindex])} { \atom(\rfunsen_\obsindex(\loc')) } {}
    \simeq
    \bs c ^\top \casualvar_{\cl I},
\end{equation}
where $\casualvar_{\cl I}$ denotes the vector $\casualvar$ restricted to the set of indexes $\cl I$.
The quality of the approximation~\eqref{eq:approx-scalprod} is determined by both the interpolation method and the density of the frequency grid used in the \gls{fft} to obtain $\casualvar$.
In this paper, we use the \emph{polar} interpolation whose efficiency has been demonstrated in Fourier atoms interpolation~\cite{simoncelli2011, duarte2013, duarte2014}.
The computation of the interpolation coefficients for all $\loc'\in\grid\domloc$ is heavy but performed offline.

To apply grid hopping to the velocity estimation~\eqref{eq:direct-velestim}, we cannot compute the index sets and the coefficients offline; the speed sensing function $\ufunsen$ 
%which maps the velocity to the speed 
indeed depends on the location that must be first estimated.
%Fortunately, given a fixed value of $\loc$, the sensing function $\ufunsen$ is linear.
To avoid the online computation of interpolation coefficients, we use for the velocity estimation the simplest interpolation scheme where each set $\cl I$ contains a single index 
%and the interpolation coefficient is $1$ 
for all $\vel'\in\grid\domvel$.
%To put it differently, we select for all $\vel\in\grid\domvel$ the speed index such that

To summarize, grid hopping follows the same principle as the direct method.
Yet, the explicit scalar products required in~\eqref{eq:direct-locestim} and \eqref{eq:direct-velestim} are replaced by an interpolation procedure such as~\eqref{eq:approx-scalprod}. 
Indirect, direct and grid hopping strategies are compared hereafter.



% Figure environment removed


% Figure environment removed

\section{Experimental results}
%We assume targets that are at the same height as the antennas (2D)
%We assume point targets
To compare the three algorithms, % (indirect, direct and grid-hopping), 
we designed an active multistatic radar system and measured different scenarios with one to four cars moving along controlled patterns.
The radar system is built around the BGT24 family of radio frequency transceivers from Infineon. Figure \ref{fig:measurement-setup} shows a simplified schematic of the hardware. 
%The \emph{base station} includes a computer, the acquisition board, one transmitter and the 8 receivers. 
%The base station is equipped with a four antennas array. 
The \emph{base station} includes a computer, the acquisition board, one transmitter and a four antennas array.
Three \emph{remote stations} are connected to the base station through coaxial cables of length up to 24 m, and are equipped with respectively a two antennas array, twice a single antenna. % and again a single antenna. 
%The remote stations are connected to the base station through coaxial cables of length up to 24 m.
A general purpose data acquisition board (NI 6378 from National Instruments) outputs voltage ramps toward the voltage controlled oscillator (BGT24MTR11) to generate chirps of duration $\tchirp = 128\mu$s around $\carrier= 24 GHz$, with $\band = 250MHz$. 
This signal feeds the \gls{tx} antenna and is split to also feed all the \gls{rx}. The receivers are IQ demodulators (BGT24MR2) followed by baseband amplification and filtering circuits. The remote stations have a local low noise amplifier to compensate the losses in the cables. The baseband signals are sampled by the acquisition board such that $\nsamp = \nchirp = 128$ and sent to the computer for processing. 
%The linear chirp modulation used is characterized $\band = 250MHz$, $\nsamp = \nchirp = 128$ and $\tchirp = 128\mu$s. 

This paper presents a first glimpse of the results. 
We focus on a scenario represented in Fig.~\ref{fig:setup} where a single car is moving along the path depicted in the figure. 
For simplicity, one RX antenna in each station was used for the processing presented in this paper.
The radar signals were acquired for 60 seconds at a rate of one processed frame of $\nchirp \nsamp$ samples every 60ms on average. 
After discarding the frames where the car was outside the sight of all \gls{rx}, $433$ processed frames are left to assess the performances of the different algorithms. 
%Each frame is processed independently from the others using the three methods.

The performance and the computation time of the three methods are compared in Fig.~\ref{fig:results}. % were evaluated as follow. 
Each location estimate resulting from each algorithm was compared to the approximated
%\footnote{
%The ground truth was approximately determined from the line along which the car was known to be moving combined with a few recorded timestamps.} 
ground truth drawn in Fig.~\ref{fig:setup}. 
As the car is not a single point in space, such comparison can only make sense with a tolerance threshold. 
When the error between an estimate and the reference is smaller than the ``hit threshold", we counted it as a hit (see Fig.~\ref{fig:results}).
%and defined the hit ratio as $n_\mathrm{hit}/433$.
%and defined the hit ratio as the amount of hit devided by the total number of frames.
The performance was tested for multiple densities of $\grid\domloc$.
%A grid density of $n$ means that the spacing between consecutive bins of $\grid\domloc$ is $1/n$ times the radar resolution which is $\frac{2\band}{\cel} = .6$ meters where $\cel$ is the speed of light.
We observe that the direct method performs better than the indirect one with a higher computational time.
The hopping strategy provides an alternative with a computation time closer to the indirect method while exhibiting performance similar to the direct method. 

\section{Conclusion}
In this paper, we presented a novel methodology, called grid hopping, as a trade-off between the direct and the indirect methods.
The efficiency of grid hopping was evaluated on actual radar measurements. 
The comparison was restricted to a single interpolation scheme and to our most simple scenario with a single car moving. 
In the future, we will study the grid hopping to iterative direct algorithms 
%such as the \gls{bomp}~\cite{eldar2010} 
enabling the detection of multiple targets.
We will also compare multiple interpolation schemes.





\bibliographystyle{IEEEtran}
\bibliography{refs/ref-radar,refs/ref-sl,refs/ref-sparse}
\end{document}
