several decades, radar systems have been providing efficient low-cost solution for monitoring and surveillance systems in civil, military and imaging applications. 
More recently, radar systems called \emph{multistatic} or \emph{distributed MIMO} have taken the front of the scene to push radar capabilities to the next level.
A multistatic radar is composed of one or more \gls{tx} and at least two \gls{rx} with widely spread locations across the area to be observed.
This geometry enables the system to estimate both location and velocity vectors corresponding multiple targets while increasing the diversity of the measurement by viewing the targets from different angles.
This increased diversity with respect to co-located MIMO radars theoretically provides multistatic geometry with stronger robustness against nuisances such as noise, clutter and interferences~\cite{}. 
\GM{Pour profiter de cette diversité, il faut processing conjoint, c'est ce qu'on fait.
Ou bien : Dans ce papier : processing conjoint et rapide. Puis on développe.}

%Those nuisance being almost independent from one receiver to another (the receivers being widely spread), ...
The most common procedure to estimate a target's parameters from a multistatic radar is the \emph{indirect} method where ranges are separately estimated from each RX's measurement and then combined using multilateration~\cite{}.
% We call this method~\emph{indirect} throughout this paper
Yet, the method is hardly extendable to multiple targets estimation because of the difficulty to associate estimators from one RX to the others \GM{Evoquer le fait que cela est généralement géré avec du tracking dans le cas indirect ?}.
Moreover, for a multistatic radar to benefit from the diversity mentioned above, a \emph{direct} processing of all signals 
%jointly 
is required.
%
Indeed, a diversity gain relies on redundancy by enabling a correct estimation even if one or more signals is dominated by nuisance.
More specifically, the multilateration is strongly impacted by a completely wrong distance estimation while a direct procedure will not suffer from this issue \GM{Etre plus précis la dessus.}.
Although this advantage of direct procedures is often mentionned in contributions which tackle this topic, to the best of our knowledge, no actual simulation or measurement has yet been proposed to extensively compare the two methods. 
This is a first gap which we intend to fill with this paper.

As a second contribution of this paper, we propose a simple signal processing methodology which exhibits both the stronger diversity gain provided by the direct approach and a lower computational cost comparable to the indirect one. 
As a first step, we will consider location estimation only. 
In the indirect procedure, the range of a target with respect to a \gls{tx}-\gls{rx} pair is estimated from a measurement $\meas$ by the maximization of a correlation with a reference signal.
The correlations are commonly evaluated for values of ranges taken from a grid which results from the discretization of a continuous range domain, leading to as much scalar product computation as the number of grid bins. 
This is exactly the computation performed by the Fast Fourier Transform (FFT) for the processing of linear chirp \gls{fmcw} radars~\cite{}
On the other hand, the direct methods capitalizes on a location grid which results from the discretization of 2D (or 3D) location domain~\cite{}. 
While the physical meaning of the range of a target only makes sense with respect to a TX and RX locations, a target location is an absolute, intrinsic parameter of a target, independently of the properties of the system observing it.
This is why working on the location domain (for instance through the location grid) is essential for a direct processing.
%A location grid has the same geometrical meaning no matter
Yet, as will be shown in this paper, given a location grid resolution, such a location grid requires a much higher number of grid bins than a range grid would require in the indirect method, leading to more scalar product to be computed and hence, a higher computational cost. 
\GM{Paragraphe confu}

The proposed methodology, which we call \emph{Grid Hopping} suggests to approximate the correlations that need to be computed for each location grid bin with the correlation computed from the range grid.
In a nutshell, a mapping operator which links each location bin to a set of range bins (one for each TX-RX pair) is computed \emph{offline}.
Then, an approximation of the global location correlation function is built from the range correlation evaluations and the application of this mapping operator.
%Because the correlations remain the dominating operations regarding the computation time, a decent
In this paper, we show the ability of grid hopping to asymptotically reach the performance of direct methods with scalable computation time closer to those of indirect method using extensive Monte-Carlo simulation of an FMCW multistatic radar system.

As shown later in this paper\GM{redondant}, the main limitation of the direct method, regardless of the grid-hopping, is related to the density of the location grid which is used to estimate the targets' locations. 
That is because actual targets' locations originate from a continuous domain and not an arbitrary grid.
Interpolation strategies inspired from the sparse signal processing literature~\cite{simoncelli2011} has showed to be relevant for monostatic radar signal processing~\cite{monnoyer2021}.
Among the different interpolation scheme, the polar interpolation~\cite{duarte2013} is often used for its ability to accurately approximate the continuous framework while providing close-form to compute its approximation~\cite{champagnat2020}.
Yet, this scheme extends poorly to multistatic radars processing with direct procedure because of the non-linearity of the operator which links the locations to ranges. 
As a third and last contribution of this paper, we propose an extension of grid hopping, called \emph{Interpolation Hopping} which enables an effective application of the polar interpolation to multistatic radars.

The remaining of this paper is structured as follow.
In Section~\ref{sec:model}, we define a generic mathematical representation of a multisensor system and show how it links to the multistatic radar particular case.
In Section~\ref{sec:background}, we use the notations defined in the previous section to expose theoretical concepts from the literature on estimation algorithms. 
In Section~\ref{sec:direct}, we validate the superiority of the direct method over the indirect one in terms of robustness against nuisance and compute an explicit comparison of computation time required by both methods.
This study shows how untractable the computation time of the direct method is and motivates the research on faster alternatives which still provides better performances than the indirect method.
In Section~\ref{sec:hopping}, we describe our \emph{Grid-Hopping} principle as well as its extension to the \ofg framework, the~\emph{interpolation hopping}.
The computational complexity of these techniques as well as their limitations are discussed.
Section~\ref{sec:application} Describes the explicit application of grid-hopping to the multistatic radar estimation problem. 
In particular, the case of separated location-velocity estimation, which is essential to provide tractable computation time~\cite{monnoyer2020} is discussed.
Extensive Monte Carlo simulation results of an FMCW multistatic radar with linear chirp modulation are analyzed to validate the effectiveness of grid-hopping. 
%Finally, Section~\ref{sec:measurement} propose

\paragraph*{Notations}