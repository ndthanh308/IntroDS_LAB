\section{Long covers and tube dimension}
\label{sec:box-dimension}

The notion of tube dimension was introduced for $\mathbb{R}$-actions in \cite[Definition 8.6]{hirshbergszabowinterwu2017}. Here we extend the notion to actions of arbitrary locally compact groups. Recall that the \emph{multiplicity} of a cover $\cU$ of $X$ is the least number $d$ such that the intersection of any $d+1$ elements of $\cU$ is empty.

\begin{definition}
\label{def:tube-dim}
The \emph{tube dimension} of an action $G \grpaction{} X$, denoted by $\tubedim(G \grpaction{} X)$, is the least natural number $d$ such that for all compact subsets $K \subseteq G$ and $Y \subseteq X$ there is a family $\cU$ of open sets of $X$ satisfying the following properties:
\begin{enumerate}
    \item for all $x \in Y$ there is $U \in \cU$ such that $K x \subseteq U$,
    \item every $U \in \cU$ is contained in a tube, and
    \item the multiplicity of $\cU$ is at most $d + 1$.
\end{enumerate}
If no such natural number $d$ exists, then we define $\tubedim(G \grpaction{} X) = \infty$.
\end{definition}

In the present section we will show that for certain group actions there are covers with controlled multiplicity and arbitrary length, as specified by the group action. This is the content of \cref{thm:covering}, which generalizes \cite[Theorem 5.2]{kasprowskiruping17} and follows the same proof outline. As a consequence, we will obtain explicit bounds on the tube dimension of these actions. Recall the number $\rmd(G)$ associated with a group of polynomial growth $G$ from Section~\ref{sec:polynomial-growth}.

\begin{theorem}
  \label{thm:covering}
  Let $G \grpaction{} X$ be an action of a locally compact group of polynomial growth on a locally compact, second-countable Hausdorff space. Suppose that $G \curvearrowright X$ admits slices as in \Cref{def:admits_slices}. Then for every compact identity neighbourhood $K \subseteq G$ there exists an open cover $\cU$ of $X$ with the following properties.
  \begin{enumerate}
  \item The cover consists of open tubes.
  \item The multiplicity of the cover is at most $11^{\rmd(G)} \cdot (\dim X + 1)$.
  \item For every $x \in X$ there exists $U \in \cU$ such that $K \cdot x \subseteq U$.
  \end{enumerate}
In particular
\[ \tubedim(G \curvearrowright X) \leq 11^{\rmd(G)} \cdot (\dim X + 1) - 1 . \]
\end{theorem}
\begin{proof}
  First, note that once we have proved the theorem for $K$, it automatically holds for any compact identity neighbourhood $K' \subseteq K$, hence we may enlarge $K$ to a compact symmetric generating set for $G$. Using \cref{prop:cover_by_translates} with $a=10$, we can find $N \in \bN$ such that $K^{10N}$ can be covered by $11^{\rmd(G)}$ translates of $K^{2N}$. Replacing $K$ with the larger set $K^{2N}$, we now have that $K^5$ can be covered by $11^{\rmd(G)}$ translates of $K$. The first step will be to show the following claim, which is an adaption of \cite[Lemma 4.6]{kasprowskiruping17}.

  \begin{claim}
    \label{claim:locally_finite_cover}
    There is a countable collection $\cS$ of $K^5$-slices such that
    \begin{enumerate}
    \item $\{ (KS)^{\circ} \mid S \in \cS \}$ is a locally finite cover of $X$, and
    \item for every pair $S,S' \in \cS$ there exists $h \in G$ such that $\{ g \in K^3 \mid gS \cap S' \neq \emptyset \} \subseteq h \mathring{K}$.
    \end{enumerate}
  \end{claim}

  \begin{proof}[Proof of Claim~\ref{claim:locally_finite_cover}]
    For every $x \in X$ we can by \ref{cor:K-slices-exist-polynomial-growth} find a $K^5$-slice $S_x'$ such that $x \in S_x' \cap (K^5 S_x')^{\circ}$.  Since $K \subseteq K^5$, each $S_x'$ is also a $K$-slice, hence $x \in (K S_x')^{\circ}$ by \cref{lem:box-interior-independence}. Now consider the cover $\{ (K S_x')^{\circ} \mid x \in X \}$ of $X$. Since $G$ is a locally compact, second countable Hausdorff space, it is paracompact and Lindel{\"o}f.  We can therefore find a countable, locally finite, open refinement for the above cover, i.e. an open cover $\cV = \{ V_i \mid i \in \NN \}$ of $X$ such that every $x \in X$ belongs to only finitely many sets from $\cV$ and such that for every $i \in \NN$ there exists $x(i) \in X$ with $V_i \subseteq (K S_{x(i)}')^{\circ}$.  Since $X$ is locally compact and second countable, every compact subset of $X$ intersects only finitely many elements of $\cV$.  We may hence replace each $V_i$ by $\ol{V_i}^{\circ}$ and assume that it is regular open.  By definition of slices, $K S_{x(i)}'$ is homeomorphic to $K \times S_{x(i)}'$, so we can project the compact set $\overline{V_i} \subseteq K S_{x(i)}'$ to the second coordinate. The resulting  compact set $D_i \subseteq S_{x(i)}'$ is then also a $K$-slice satisfying $V_i \subseteq K D_i$.

    We claim that the cover $\{ (K D_i)^{\circ} \mid n \in \NN \}$ of $X$ is also locally finite.  Indeed, let $i \in \NN$ and suppose $x \in (K D_i)^{\circ}$.  Then there exists an open, precompact neighbourhood $U$ of $x$ such that $U \subseteq KD_i$.  Since $x \in KD_i$ we can write $x = ky$ for some $k \in K$ and $y \in D_i$.  By definition of $D_i$ it follows that there exists $x' \in \ol{V_i}$ such that $x' = k'y$ for some $k' \in K$.  Hence $x' = k'k^{-1}x \in K^2U \cap \ol{V_i}$, in particular $K^2U$ intersects $\ol{V_i}$.  Since $K^2U$ is open, it must intersect the regular open set $V_i$.  Since $K^2\ol{U}$ is compact and $\cV$ is a locally finite cover of $X$, this can happen for at only finitely many $i \in \NN$.  Hence there are only finitely many $i \in \NN$ such that $x \in (KD_i)^{\circ}$.

    For every pair $i,j \in \NN$ such that $D_i \cap K^3 D_j \neq \emptyset$, define the function
    \begin{gather*}
      f_{i,j} \colon D_i \cap K^3 D_j \to K^3
    \end{gather*}
    to be the restriction of the projection $K^3 D_j \to K^3$ to the set $D_i \cap K^3 D_j$.  Then $f_{i,j}$ is continuous by definition of slices, hence we can for every $x \in D_i \cap K^3D_j$ pick an open neighbourhood $U_{i,j,x} \subseteq D_i \cap K^3D_j$ of $x$ such that $f_{i,j}(U_{i,j,x}) \subseteq f_{i,j}(x)K^{\circ}$.  Since $X$ and hence $D_i$ is second countable, we can find an open set $U_{i,j,x}'$ in $D_i$ containing $x$ such that $U_{i,j,x'}' \cap K^3D_j = U_{i,j,x}$.  For every $i \in \NN$ the set $J_i = \{ j \in J \mid D_i \cap K^3D_j \neq \emptyset \}$ is finite, so $U_{i,x} = \bigcap_{j \in J_i} U_{i,j,x}'$ is a finite intersection of open sets, hence open.  Since $D_i$ is compact and $D_i = \bigcup_{x \in D_i} U_{i,x}$, we can find $x_1, \ldots, x_{m_i} \in D_i$ such that $D_i = \bigcup_{k=1}^{m_i} U_{i,x_k}$.  Pick open sets $V_{i,k}$ in $D_i$ with $D_i = \bigcup_{k=1}^{m_i} V_{i,k}$ and $\ol{V_{i,k}} \subseteq U_{i,x_k}$.  Let $S_{i,k}$ denote the closure of $V_{i,k}$ and consider the set
    \begin{gather*}
      \cS = \{ S_{i,k} \mid i \in \NN, 1 \leq k \leq m_i \}
      \eqstop
    \end{gather*}
    Then every element of $\cS$ is a closed subset of $D_i$ for some $i \in \NN$, hence a $K$-slice.
    
    We now argue that $\{ (KS)^{\circ} \mid S \in \cS \}$ is a cover of $X$.  To this end it suffices to show that $(KD_i)^{\circ} = \bigcup_{k = 1}^{m_i} (K V_{i,k})^\circ$ for all $i$.  The inclusion from the right to the left is clear.  To see the other inclusion, let $x \in (K D_i)^{\circ}$ and write $x = ky$ for $k \in K$ and $y \in D_i$.  Since $D_i = \bigcup_{k=1}^{m_i} V_{i,k}$ there is $1 \leq k \leq m_i$ such that $y \in V_{i,k}$.  Further, \cref{lem:box_interior} says that $(KD_i)^{\circ} = \mathring{K} \boxint{D_i}$, so that $k \in \mathring{K}$ follows.  So we find that $x \in \mathring{K}V_{i,k}$, which by the definition of a $K$-slice is open in $KD_i$. We conclude by summarising that $x \in \mathring{K}V_{i,k} \cap (KD_i)^{\circ} \subseteq (KV_{i,k})^{\circ}$.

    We also see that $\{ (KS)^{\circ} \mid S \in \cS \}$ is locally finite since each $(K D_i)^{\circ}$ intersects finitely many sets $KS$ for $S \in \cS$ and the cover $\{ (KD_i)^{\circ} \mid i \in \NN \}$ is locally finite.


    For some $i,j \in \NN$ and $1 \leq r \leq m_i$, $1 \leq s \leq m_j$ put $h = f_{i,j}(x_r)$.  We show that 
    \begin{gather*}
      \{ g \in K^3 \mid gS_{i,r} \cap S_{j,s} \neq \emptyset \} \subseteq h \mathring{K}
    \end{gather*}
    If $g$ is a member of the left-hand side, we can find some $x \in S_{i,r} \cap g S_{j,s} \subseteq U_{i,x_k} \cap gU_{j,x_s} \subseteq D_i \cap K^3 D_{j}$, so $j \in J_i$.  Hence $x \in U_{i,j,x_r}'$.  Since $x \in K^3D_j$ as well, we have that $x \in U_{i,j,x_r}$.  This implies that $g = f_{i,j}(x) \in f_{i,j}(x_r)\mathring{K}$.
  \end{proof}
  
  Let now $\cS = (S_i)_{i \in \NN}$ be as in Claim~\ref{claim:locally_finite_cover}.  By the shrinking lemma \cite[Lemma 41.6]{munkres00} we can find a cover $\{ V_i \mid i \in \NN \}$ of $X$ where each $V_i$ is open and $\ol{V_i} \subseteq (KS_i)^{\circ}$.  Then each $\ol{V_i}$ is a closed subset of $KS_i$, so we can project $\ol{V_i}$ onto $S_i$ to get a new $K$-slice $A_i^0 \subseteq S_i$ satisfying $V_i \subseteq K A_i^0$.  So $\{ (K A_i^0)^{\circ} \mid i \in \NN \}$ is a cover of $X$. Also $A_i^0 \subseteq \ol{V_i} \subseteq (KS_i)^{\circ}$, so that $A_i^0 \subseteq \boxint{S_i}$. Consider the following claim, which is an adaption of \cite[Lemma 5.1]{kasprowskiruping17}.

  \begin{claim}
    \label{claim:dimension_reduction}
    Let $k \in \NN$. If $(A_i)_{i \in \NN}$ is a sequence of compact sets where $A_i \subseteq \boxint{S_i}$ and $\dim A_i \leq k$ for all $i \in \NN$, then there exist regular open sets $B_i \subseteq \boxint{S_i}$ in $S_i$ for each $i \in \NN$ such that
    \begin{enumerate}[label=(\alph*)]
    \item \label{it:dim-red:multiplicity}
      the set $\{ \mathring{K}^5 B_i \mid i \in \NN \}$ has multiplicity at most $11^{\rmd(G)}$, and
    \item \label{it:dim-red:dimension}
      for every $i \in \NN$ the compact set
      \begin{gather*}
        A_i \setminus \bigcup_{j \in \NN} \mathring{K}^3B_j
      \end{gather*}
      has dimension at most $k-1$.
    \end{enumerate}
  \end{claim}
  Before proving the claim we show how it can be used to finish the proof of \cref{thm:covering}.  By \cref{prop:dim_properties} we have that $\dim(A_i^0) \leq \dim(X)$ for each $i \in \NN$.  Applying the claim to $(A_i^0)_{i \in \NN}$, we obtain sets $(B_i^0)_{i \in \NN}$ satisfying the conclusion of Claim~\ref{claim:dimension_reduction} with $k = \dim X$. Set
  \begin{gather*}
    A_i^1 = A_i \setminus \bigcup_{j \in \NN} \mathring{K}^3 B_j
    \eqstop
  \end{gather*}
  Then apply Claim~\ref{claim:dimension_reduction} again to the sequence $(A_i^1)_{i \in \NN}$, obtaining a new sequence of sets $(B_i^1)_{i \in \NN}$.  Continuing like this, we obtain sets $(A_i^k)_{i \in \NN}$ and $(B_i^k)_{i \in \NN}$ for every $k \in \NN$. We will now show that
  \begin{gather*}
    \cU = \{ \mathring{K}^5 B_i^k \mid i \in \NN, 0 \leq k \leq \dim X \}
  \end{gather*}
  satisfies the properties of \cref{thm:covering}.  First, note that since $B_i^k$ is a regular open subset of the $K^5$-slice $S_i$ with $B_i^k \subseteq \boxint{S_i}$, it follows from equations \eqref{eq:box_interior} and \eqref{eq:boxint_open} of \cref{lem:box_interior} that $\mathring{K}^5 B_i^k = (K^5\ol{B_i^k})^{\circ}$ is an open tube. Hence the first assertion of \cref{thm:covering} is established.

  We now claim that
  \begin{gather*}
  \label{eq:covering-family}
    \cU' = \{ \mathring{K}^4 B_i^k \mid i \in \NN, 0 \leq k \leq \dim X \}
  \end{gather*}
  is a cover of $X$.  Indeed, let $x \in X$, so that $x \in K A_i^0$ for some $i \in \NN$, say $x = x' y$ with $x' \in K$ and $y \in A_i^0$.  If $y \in \mathring{K}^3 B_j^0$ for some $j \in \NN$ then $x \in K \mathring{K}^3 B_j^0 = \mathring{K}^4 B_j^0$, so $x$ belongs to an element of $\cU'$.  If no such $j$ exists, then by definition we have $y \in A_i^1$.  Continuing like this, if $y \notin \mathring{K}^3 B_j^k$ for any $j \in \NN$ and $0 \leq k \leq \dim X$, we reach the conclusion that $y \in A_i^{\dim X + 1}$.  However $\dim A_i^{\dim X + 1} \leq -1$ so $A_i^{\dim X + 1} = \emptyset$ which is a contradiction. This shows that $\cU'$ is a cover of $X$.

  Now if $x \in X$, say $x \in \mathring{K}^4 B_i^k$ for some $i \in \NN$ and $0 \leq k \leq \dim X$, then $K x \subseteq K (\mathring{K}^4 B_i^k) = \mathring{K}^5 B_i^k$, which shows that $\cU$ satisfies the second assertion of \cref{thm:covering}. Also, by Claim~\ref{claim:dimension_reduction} the multiplicity of $\cU$, being a union of $\dim X + 1$ sets all of multiplicity at most $11^{\rmd(G)}$, cannot exceed $11^{\rmd(G)} \cdot (\dim X + 1)$.  Hence the last assertion of \cref{thm:covering} is also established.
\end{proof}


As we have just shown, it remains to prove Claim~\ref{claim:dimension_reduction}.
\begin{proof}[Proof of Claim~\ref{claim:dimension_reduction}]
  Let $(A_i)_{i \in \NN}$ be as in the statement of the claim.  We will divide the proof into five steps.

  \vspace{0.5em}

  \noindent \textbf{Step 1}.
  In the first step we will construct the sets $(B_i)_{i \in \NN}$.  For this, fix $i \in \NN$.  We apply the \cref{def:small_inductive_dimension} of small inductive dimension to the set $A_i$ viewed as a subset of the ambient space $S_i$ to obtain for every $x \in A_i$ an open neighbourhood $U_x$ of $x$ in $S_i$ which satisfies $\dim(\partial_{S_i}U_x) \leq k-1$.  As explained in \cref{rmk:inductive_dimension_open_regular}, we may assume that the sets $U_x$ are regular open.

  Since $A_i$ is compact we can find a finite subset $F_i\subseteq A_i$ such that $U_i = \bigcup_{x\in F_i}U_x$ contains $A_i$.  Note that $U_i$ is regular open in $S_i$.  Since $X$ is separable and metrizable, every subset is also separable and metrizable, in particular $\partial_{S_i}U_x$ is so for each $x \in F_i$.  Hence we can apply \cref{prop:dim_properties} to obtain
  \begin{equation}
    \label{eq:dimension_red_def}
    \dim(\partial_{S_i}U_i)
    \leq
    \dim \Big( \bigcup_{x \in F_i} \partial_{S_i}U_x \Big)
    \leq
    k-1. 
  \end{equation}
  We define subsets $(I_i)_{i \in \NN}$ of $\NN$ and sets $(B_i)_{i \in \NN}$ in $X$ recursively.  Set $I_1 = \emptyset$ and $B_1 = U_1$.  Further, for $i \geq 2$, set $I_{i} = \{ j \in \NN \mid j < i, K^2 \ol{B_j} \cap U_i \neq \emptyset \}$ and
  \begin{gather*}
    B_i = U_i \setminus \bigcup_{j \in I_i}K^3\ol{B_j}
    \eqstop
  \end{gather*}
  It is then clear from the construction that $B_i$ is an open subset of $S_i$ for each $i \in \NN$.  We will prove that each $B_i$ is regular open by induction.  For $i = 1$ we have that $B_1 = U_1$ which is regular open by construction.  Next, assume that $B_j$ is regular open for all $j < i$. Note first that
  \begin{gather*}
    \ol{B_i}
    \subseteq
    \ol{U_i} \setminus \Big( \bigcup_{j\in I_i} K^3\ol{B_j} \Big)^{\circ}
    \subseteq
    \ol{U_i} \setminus \bigcup_{j\in I_i} (K^3\ol{B_j})^{\circ}
    \eqstop
  \end{gather*}
  Now by the induction assumption combined with equations \eqref{eq:box_interior} and \eqref{eq:boxint_open} of \cref{lem:box_interior} we have that $(K^3\ol{B_j})^{\circ} = \mathring{K}^3\boxint{\ol{B_j}} = \mathring{K}^3 B_j$ for all $j < i$.  Since $S_i$ is a $K^3$-slice, we have $\ol{WB} = \ol{W}\, \ol{B}$ for all subsets $W \subseteq K^3$ and $B \subseteq S_i$.  So we get that $\ol{(K^3\ol{B_j})^{\circ}} = K^3\ol{B_j}$.  Using this and the fact that $U_i$ is regular open, we find
  \begin{gather*}
    \mathring{\ol{B_i}}
    =
    \Big( \ol{U_i}\setminus\bigcup_{j\in I_i} (K^3\ol{B_j} )^{\circ} \Big)^{\circ}
    \subseteq
    \mathring{\ol{U_i}} \setminus \ol{\bigcup_{j\in I_i}  (K^3\ol{B_j})^{\circ}}
    =
    U_i \setminus \bigcup_{j\in I_i} \ol{ (K^3\ol{B_j})^{\circ} }
    =
    U_i \setminus \bigcup_{j\in I_i} \mathring{K}^3B_j
    =
    B_i
    \eqstop
  \end{gather*}
  Since $B_i \subseteq \mathring{\ol{B_i}}$ is obvious, this proves that $B_i$ is regular open, finishing the induction argument.

  \vspace{0.5em}

  \noindent \textbf{Step 2}.
  In this step we show assertion \ref{it:dim-red:multiplicity} of Claim~\ref{claim:dimension_reduction}.  Note that by construction of $B_i$, the elements of the set $\{KB_i\}_{i\in\NN}$ are pairwise disjoint.  Indeed, suppose for a contradiction that $x \in KB_i \cap KB_j$ with $j < i$, say $x = g_iy_i = g_jy_j$ with $g_i,g_j \in K$, $y_i \in B_i$ and $y_j \in B_j$.  Then $y_i = g_i^{-1}x \in U_i \cap K(KB_j)$, hence $j \in I_i$.  It then follows from the definition of $B_i$ that $y_i \notin K^3 \ol{B_j}$ which contradicts $y_j = g_j^{-1}g_iy_i \in K^2B_j$.

  Consider now the set $\{K^5 B_j\}_{j\in\NN}$ from assertion \ref{it:dim-red:multiplicity} of the claim.  By choice of $K$, its power $K^5$ can be covered by $\ell = 11^{\rmd(G)}$ translates of $K$, say $K^5 \subseteq \bigcup_{n=1}^\ell g_n K$ for some $g_1, \dotsc, g_\ell \in G$.  Suppose for a contradiction that there exist indices $i_1 < \cdots < i_{\ell+1}$ such that there is $x \in K^5B_{i_1} \cap \cdots \cap K^5B_{i_{\ell+1}}$.  Then for each $1 \leq r \leq \ell + 1$ there exists $1 \leq n_r \leq \ell$ such that $g_{n_r}^{-1}x \in KB_{i_r}$.  Since $\{ n_1, \dotsc, n_{\ell+1} \} \subseteq \{ 1 , \dotsc, \ell \}$, there exist $1 \leq r,r' \leq \ell + 1$ such that $n_r = n_{r'}$.  This gives $g_{n_r}^{-1}x = g_{n_{r'}}^{-1}x \in KB_{i_r} \cap KB_{i_{r'}}$, which is a contradiction.  Hence the set $\{ K^5 B_j \}_{j \in \NN}$ has multiplicity at most $\ell$.
  
  \vspace{0.5em}

  \noindent \textbf{Step 3}.
  In this step we will show that the sets $S_j\cap (\partial K^3) \ol{B}_k$ are empty when $j \in \NN$ and $k \in I_j$.

  Suppose for a contradiction that there is some $x \in S_j \cap (\partial K^3) \ol{B}_k$, say $x = gy$ for $g \in \partial K^3$ and $y \in \ol{B_k}$.  Since $k \in I_j$, there exists some $x' \in K^2\ol{B_k} \cap U_j \subseteq K^2\ol{B_k}\cap S_j$, say $x' = g'y'$ with $g' \in K^2$ and $y' \in \ol{B_k}$.  We now have that $g,g' \in K^3$, $S_j \cap g S_k \neq \emptyset$ and $S_j \cap g' S_k \neq \emptyset$, so by Claim~\ref{claim:locally_finite_cover} we obtain that $g'^{-1}g \in \mathring{K}$.  Hence
  \begin{gather*}
    g= g'(g'^{-1}g) \in K^2\mathring{K} = \mathring{K}^3
    \eqcomma
  \end{gather*}
  which contradicts $g \in \partial K^3$.

  \vspace{0.5em}

  \noindent \textbf{Step 4}.
  In this step we prove that showing \ref{it:dim-red:dimension} of the present claim can be reduced to showing that $\partial_{S_j} B_j$ has dimension at most $k-1$ for all $j \in \NN$.  Fix $i \in \NN$ and note that
  \begin{gather*}
    \bigcup_{j \in \NN} \mathring{K}^3B_j
    \supseteq
    \bigcup_{j \leq i} \mathring{K}^3B_j
    =
    \mathring{K}^3\Big( U_i \setminus \bigcup_{j \in I_i} K^3 \ol{B_j} \Big) \cup \bigcup_{j < i} \mathring{K}^3B_j
    \supseteq
    U_i \setminus \bigcup_{j \in I_i} K^3 \ol{B_j} \cup \bigcup_{j < i} \mathring{K}^3B_j
    \eqstop
  \end{gather*}
  Hence, taking complements in $A_i$ and using the fact that $A_i \subseteq U_i$, we obtain
  \begin{align*}
    A_i \setminus \bigcup_{j\in\NN}\mathring{K}^3 B_j
    & \subseteq
      \Big( A_i \setminus  \Big( U_i \setminus \bigcup_{j \in I_i} K^3 \ol{B_j} \Big) \Big) \setminus \bigcup_{j \in I_i} \mathring{K}^3B_j \\
    & =
      \Big( A_i \setminus U_i \cup A_i \cap \bigcup_{j<i} K^3 \ol{B_j} \Big) \setminus \bigcup_{j \in I_i} \mathring{K}^3B_j \\
    & =
      A_i \cap \Big( \bigcup_{j \in I_i} K^3 \ol{B_j} \setminus \bigcup_{j \in I_i} \mathring{K}^3 B_j \Big) \\
    & \subseteq
      \bigcup_{j \in I_i} A_i \cap (K^3\ol{B_j} \setminus \mathring{K}^3 B_j)
  \end{align*}
  Thus, in order to show that $A_i \setminus \bigcup_{j \in \NN} \mathring{K}^3 B_j$ has dimension at most $k-1$, it suffices by \cref{prop:dim_properties} to show that the sets
  \begin{gather*}
    % \label{eq:sets_dimension_red}
    A_i \cap (K^3\ol{B_j} \setminus \mathring{K}^3B_j)\eqcomma \qquad j \in I_i\eqcomma
  \end{gather*}
  have dimension at most $k-1$.  By Step 3, the set $A_i \cap (\partial K^3)\ol{B_j}$ is empty.  Therefore, using \eqref{eq:several-boundary-expressions} of \cref{lem:box_interior}, we have that
  \begin{gather*}
    A_i \cap (K^3\ol{B_j} \setminus \mathring{K}^3B_j)
    =
    (A_i \cap (\partial K^3)\ol{B_j} ) \cup (A_i \cap K^3(\partial_{S_j}B_j) )
    =
    A_i \cap K^3(\partial_{S_j}B_j)
    \eqstop
  \end{gather*}
  For each $i,j \in \NN$, we consider the projection $K^3S_j \to S_j$, which maps $K^3(\partial_{S_j}B_j)$ to $\partial_{S_j}B_j$. The restriction of this projection to $A_i \cap K^3(\partial_{S_j}B_j)$ is a homeomorphism onto its image by \cref{lem:slice_restriction_injective}, so it suffices to show that the dimension of $\partial_{S_j}B_j$ is at most $k-1$.

  \vspace{0.5em}

  \noindent \textbf{Step 5}.
  In this step we finish the proof of Claim~\ref{claim:dimension_reduction} by showing that $\partial_{S_i}B_i$ has dimension at most $k-1$ for all $i \in \NN$. We establish this by induction. Consider first the base case $i=1$. Then $\partial_{S_1}B_1 = \partial_{S_1}U_1$ which has dimension at most $k-1$ by inequality \eqref{eq:dimension_red_def}.

  For the induction step, let $i \in \NN$ and assume that $\dim(\partial_{S_j}B_j) \leq k-1$ for all $j < i$.  First we estimate $\partial_{S_i}B_i$ using \Cref{lem:box_interior} to obtain
  \begin{align*}
    \partial_{S_i}B_i
    & \subseteq
      \partial_{S_i}U_i\cup\bigcup_{j \in I_i}\partial_{S_i}(S_i\cap\mathring{K}^3B_i) \\
    & =
      \partial_{S_i}U_i\cup\bigcup_{j \in I_i} S_i \cap\partial(\mathring{K}^3B_j)\\
    & =
      \partial_{S_i}U_i\cup\bigcup_{j \in I_i} (S_i \cap (\partial K^3)\ol{B_j} ) \cup (S_i \cap K^3(\partial_{S_j}B_j))
      \eqstop
  \end{align*}
  From \eqref{eq:dimension_red_def} we know that $\partial_{S_i}U_i$ has dimension at most $k-1$.  Further, by Step 3, the sets $S_i \cap (\partial K^3)\ol{B_j}$ for $j \in I_i$ are empty.  Applying \cref{prop:dim_properties}, we need only show that $S_i \cap K^3(\partial_{S_j}B_j)$.  But for this we can again consider the projection of the tube $K^3(\partial_{S_j}B_j) \to \partial_{S_j}B_j$ which, once restricted to $S_i \cap K^3(\partial_{S_j}B_j)$, becomes a homeomorphism onto its image by \cref{lem:slice_restriction_injective}.  Since the image is a subset of $\partial_{S_j}B_j$ which has dimension at most $k-1$ by the induction hypothesis, we can appeal to \cref{prop:dim_properties} to finish the proof.
\end{proof}


%%% Local Variables:
%%% mode: latex
%%% TeX-master: "classifiability"
%%% End:
