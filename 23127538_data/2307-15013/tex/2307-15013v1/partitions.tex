\section{Partitions of unity}
\label{sec:partitions}


Our next aim is to provide a suitable adaption and generalisation of \cite[Proposition 8.23]{hirshbergszabowinterwu2017}.  The notion of Lipschitz functions, which was used in the context of $\mathbb{R}$-actions, is not suitable for general amenable locally compact groups and we resort to the following notion of F{\o}lner functions instead.  As it turns out, this concept is still suitable to characterise finite tube dimension and can be used for nuclear dimension estimates.
\begin{definition}
  \label{def:folner-function}
  Let $G \curvearrowright X$ be an action and $(Y, \mathrm{d})$ a metric space. Let $K \subseteq G$ be a compact identity neighborhood and let $\epsilon > 0$. Then a function $\Phi: X \to Y$ is called $(K, \varepsilon)$-F{\o}lner if for every $g \in K$ and every $x \in X$ we have
  \begin{gather*}
    \mathrm{d}(\Phi(gx), \Phi(x)) < \varepsilon
    \eqstop
  \end{gather*}
  A function $\varphi: X \to \mathbb{C}$ is called $(K, \varepsilon)$-F{\o}lner if it is $(K, \varepsilon)$-F{\o}lner for the Euclidean metric on $\bC$.
\end{definition}



\begin{remark}
  This notion is not to be confused with the F{\o}lner function of a group.
\end{remark}



\begin{lemma}
  \label{lem:creating-folner-functions}
  Let $G$ be an amenable locally compact group and let $G \grpaction{} X$ be a continuous action by homeomorphisms.  Let $\varphi\colon X \to \CC$ be a continuous bounded function, let $K \subseteq G$ be compact and $\veps > 0$.  If $B \subseteq G$ is a $(K, \varepsilon/\|\phi\|_\infty)$-F{\o}lner set, then
  \begin{gather*}
    \psi(x) = \frac{1}{\haar(B)}\int_B \vphi(g^{-1}x) \rmd \haar(g)
  \end{gather*}
  is $(K, \veps)$-F{\o}lner and continuous.
\end{lemma}
\begin{proof}
  It is clear that $\psi$ is continuous, since $\varphi$ is continuous.  We have to show that it is $(K, \varepsilon)$-F{\o}lner.  For $g \in K$ we find that
  \begin{align*}
    \|g\psi - \psi\|_\infty
    & =
      \|g (\frac{1}{\mathrm{m}(B)} \mathbb{1}_B * \varphi) -  \frac{1}{\mathrm{m}(B)} \mathbb{1}_B * \varphi\|_\infty \\
    & =
      \|(g \frac{1}{\mathrm{m}(B)} \mathbb{1}_B) * \varphi -  \frac{1}{\mathrm{m}(B)} \mathbb{1}_B * \varphi\|_\infty \\
    & \leq 
      \|(g \frac{1}{\mathrm{m}(B)} \mathbb{1}_B) -  \frac{1}{\mathrm{m}(B)} \mathbb{1}_B\|_1 * \|\varphi\|_\infty    \\
    & \leq
      \frac{\varepsilon}{\|\varphi\|_\infty} \| \varphi\|_\infty\\
    & =
      \varepsilon
      \eqstop
  \end{align*}
\end{proof}




The next proposition is an adaption of \cite{hirshbergszabowinterwu2017}, which was formulated for $\mathbb{R}$-actions.  In the general setup of amenable groups, it is not possible any longer to show the existence of Lipschitz partitions of unity, however a slight adaption of the proof makes it possible to replace these by suitable F{\o}lner partitions of unity, which suffice for the purpose of proving nuclear dimension estimates.  There is only a conceptual innovation, but no substantial technical change over the proof of \cite[Proposition 8.23]{hirshbergszabowinterwu2017}.

In order to formulate our proposition, we briefly recall combinatorial simplicial complexes and their $\ell^1$-metric realisation.
\begin{itemize}
\item A (combinatorial) simplicial complex is a set $Z$ of finite sets closed under taking subsets.  An $l$-simplex in $Z$ is a element $\sigma \in Z$ with cardinality $|\sigma| = l + 1$ and we denote by $Z^{(l)}$ the set of all $l$-simplices of $Z$. The elements of $Z^{(0)}$ are called the \emph{vertices} of $Z$. The \emph{dimension} of $Z$ is the highest number $d$ such that $Z^{(d-1)} \neq \emptyset$ (if no such number exists, the dimension of $Z$ is infinite).
\item Given a simplicial complex $Z$, we denote by $\mathrm{C} Z = \{\sigma \in Z \sqcup \{\infty\} \mid \sigma \cap Z^{(0)} \in Z\}$ the cone over $Z$.
\item The $\ell^1$-metric realisation of a simplicial complex $Z$ is
  \begin{gather*}
    |Z| = \bigcup_{\sigma \in Z} \{(z_v)_v \in [0,1]^{Z^{(0)}} \mid \sum_{v \in \sigma} z_v = 1 \text{ and } z_v = 0 \text{ for any } v \in Z_0 \setminus \sigma \}
  \end{gather*}
  endowed with the restriction of the $\ell^1$-metric on finitely supported functions on $Z^{(0)}$.
\item For any vertex $v_0\in Z^{(0)}$, the \emph{simplicial star} around $v$ is the set of simplices that contain $v_0$, and the \emph{open star} around $v_0$ is the subset $\{ (z_v)_v \in |Z| : z_{v_0} > 0 \}$ of the geometric realization of $Z$.
\item If $\mathcal{U}$ is a finite open cover of $X$, its \emph{nerve complex} is the simplicial complex
\[ \mathcal{N}(\mathcal{U}) = \{ \mathcal{U}' \subseteq \mathcal{U} : \bigcap_{U \in \mathcal{U}'} U \neq \emptyset \} . \]
The dimension of $\mathcal{N}(\mathcal{U})$ is equal to $\mathrm{mult}(\mathcal{U}) - 1$.
\item If $\mathcal{U}$ is a finite open cover of a closed subset $C$ of $X$, we denote by $\mathcal{U}^+$ the open cover of $X$ given by $\mathcal{U}^+ = \mathcal{U} \cup \{ X \setminus C \}$. One can then identify $\mathcal{N}(\mathcal{U}^+)$ as a subset of $\mathrm{C}\mathcal{N}(\mathcal{U})$ via sending $X \setminus C$ to $\infty$.
\end{itemize}

We are now ready to formulate our adaption and generalisation of \cite[Proposition 8.23]{hirshbergszabowinterwu2017}.
\begin{proposition}
  \label{prop:characterisation-tube-dimension}
  Let $G$ be an amenable locally compact second countable group and let $G \curvearrowright X$ be a continuous action by homeomorphisms.  Then the following statements are equivalent.
  \begin{enumerate}
  \item \label{it:tube-dim}
    The tube dimension of $G \curvearrowright X$ is at most $d$.

  \item \label{it:map-into-cone}
    For every compact subset $K \subseteq G$, every compact subset $A \subseteq X$ and every $\varepsilon > 0$ there is a finite simplicial complex $Z$ of dimension at most $d$ and a continous map $\Phi: X \to |\mathrm{C} Z|$ such that
    \begin{enumerate}
    \item $\Phi$ is $(K, \varepsilon)$-F{\o}lner,
    \item for every vertex $v \in Z_0$ the preimage of the open star around $v$ under $\Phi$ is contained in a tube, and
    \item $\Phi(A) \subseteq |Z|$.
    \end{enumerate}
    
  \item \label{it:partition-plain}
    For every compact subset $K \subseteq G$, every $\varepsilon > 0$ and every compact subset $A \subseteq X$ there is a finite partition of unity $(\varphi_i)_{i \in I}$ for $A \subseteq X$ such that
    \begin{enumerate}
    \item $\varphi_i$ is $(K,\varepsilon)$-F{\o}lner for all $i$,
    \item $\varphi_i$ is supported in the interior of a tube, and
    \item there is a partition $I = I^{(0)} \sqcup \hdots \sqcup I^{(d)}$ such that for all $l \in \{0, \dotsc, d\}$ and all $i,j \in I^{(l)}$ we have
      \begin{gather*}
        \supp \varphi_i \cap \supp \varphi_j = \emptyset
        \eqstop
      \end{gather*}
    \end{enumerate}

  \item \label{it:partition-fattened}
    For every pair of compact subsets $K, L \subseteq G$, every $\varepsilon > 0$ and every compact subset $A \subseteq X$ there is a finite partition of unity $(\varphi_i)_{i \in I}$ for $A \subseteq X$ such that
    \begin{enumerate}
    \item $\varphi_i$ is $(K,\varepsilon)$-F{\o}lner for all $i$,
    \item $L\supp \varphi_i$ is contained in the interior of a tube, and
    \item there is a partition $I = I^{(0)} \sqcup \hdots \sqcup I^{(d)}$ such that for all $l \in \{0, \dotsc, d\}$ and all $i,j \in I^{(l)}$ we have
      \begin{gather*}
        L\supp \varphi_i \cap L \supp \varphi_j = \emptyset
        \eqstop
      \end{gather*}
    \end{enumerate}

  \item \label{it:cover-fattened}
    For every compact subset $L \subseteq G$ and every compact subset $A \subseteq X$ there is a finite collection $\mathcal{U}$ of open subsets of $X$ that cover $A$ such that
    \begin{enumerate}
    \item each $U \in \mathcal{U}$ is contained in a tube of shape larger than $L$, and
    \item there is a partition $\mathcal{U} = \mathcal{U}^{(0)} \sqcup \dotsm \sqcup \mathcal{U}^{(d)}$ such that for any $l \in \{0, \dotsc, d\}$ and any distinct elements $U_1, U_2 \in \mathcal{U}^{(l)}$ we have
      \begin{gather*}
        L\overline{U_1} \cap L\overline{U_2} = \emptyset
        \eqstop
      \end{gather*}
    \end{enumerate}
  \end{enumerate}
\end{proposition}
\begin{proof}
  We start by proving \ref{it:tube-dim} $\Rightarrow$ \ref{it:map-into-cone}.  Fix the notation of \ref{it:map-into-cone}, let $0 < C$ be such that $3(d+1)C < \varepsilon$ and let $L \subseteq G$ be an $(K,C)$-F{\o}lner set.  Consider $B = LA$.  Then by the definition of tube dimension there is a family of open subsets $\mathcal{U}$ of $X$, each contained in a tube, having multiplicity at most $d+1$ such that for every $x \in LB$ there is $U \in \mathcal{U}$ such that $Lx \subseteq U$. Consider the sets
  \begin{gather*}
    V_U = \{x \in X \mid Lx \subseteq U\} \quad U \in \mathcal{U}
    \text{,}
  \end{gather*}
  and the collection $\mathcal{V} = \{V_U \mid U \in \mathcal{U}\}$. Then $\mathcal{V}$ covers $B$ and we can find a finite subset $\mathcal{U}_0 \subseteq \mathcal{U}$ such that $\mathcal{V}_0 = \{V_U \mid U \in \mathcal{U}_0\}$ remains a cover of $B$.  Let $(\varphi_U)_{U \in \, \mathcal{U}_0}$ be a partition of unity for $B \subseteq X$ subordinate to $\mathcal{V}_0$.  By \cref{lem:creating-folner-functions}, the functions $\psi_U = \frac{1}{\mathrm{m}(L)} \mathbb{1}_L * \varphi_U$ for $U \in \mathcal{U}_0$ are $(K, C)$-F{\o}lner. Note also that $\supp(\psi_U) \subseteq \supp(\mathbb{1}_L)\supp(\phi_U) \subseteq LV_U \subseteq U$ for each $U \in \mathcal{U}_0$. Since $\mathcal{U}_0$ has multiplicity at most $(d + 1)$, at most $2(d+1)$ functions from the set $\{g\psi_U \mid U \in \mathcal{U}_0\} \cup \{\psi_U \mid U \in \mathcal{U}_0\}$ can be non-zero at a time.  So for $g \in K$, we find that
  \begin{gather*}
    \|g(\mathbb{1}_X - \sum_{U \in \mathcal{U}_0} \psi_U) - (\mathbb{1}_X - \sum_{U \in \mathcal{U}_0} \psi_U)\|_\infty
    =
    \|\sum_{U \in \mathcal{U}_0} g\psi_U - \sum_{U \in \mathcal{U}_0} \psi_U \|_\infty
    \leq
    2(d + 1)C
    \text{,}
  \end{gather*} 
which shows that $\mathbb{1}_X - \sum_{U \in \mathcal{U}_0} \psi_U$ is $(K, 2(d+1)C)$-F{\o}lner.

Let $Z = \mathcal{N}(\mathcal{U}_0)$ be the nerve of $\mathcal{U}_0$, which is a simplicial complex of dimension at most $\mathrm{mult}(\mathcal{U}_0) = (d+1)-1 = d$.  Consider the continuous map
\begin{gather*}
  \Phi\colon X \to |\cN(\cU_0^+)| \subseteq |\rC Z|
\end{gather*}
given by $\Phi = (\mathbb{1}_X - \sum_{U \in \, \cU_0} \psi_U) \oplus \bigoplus_{U \in \, \cU_0} \psi_U$ and observe that $\Phi$ is $(K, 3(d+1)C)$-F{\o}lner.  By the choice of $C$, this shows that $\Phi$ is $(K, \varepsilon)$-F{\o}lner.  It maps $A$ into $|\mathcal{N}(\mathcal{U}_0)|$, since $\mathbb{1}_X - \sum_{U \in \mathcal{U}_0} \psi_U$ vanishes on $A$.  Also the preimage of every open star is contained in some $U \in \mathcal{U}_0$, which in turn is contained in a tube.

Next we prove \ref{it:map-into-cone} $\Rightarrow$ \ref{it:partition-plain}. Let $K \subseteq G$ be a compact subset, let $A \subseteq X$ be compact and let $\varepsilon > 0$.  Let $\Phi: X \to |CZ|$ be chosen as \ref{it:map-into-cone} for $K \subseteq G$ and the constant $\frac{\varepsilon}{2 (d+1)(d+2)(2d+3)}$.  Let $I^{(l)}$ be the collection of $l$-simplices of $Z$ for $l \in \{0,\dotsc, d\}$, and let $I = I^{(0)} \cup \cdots \cup I^{(d)}$, which is a disjoint union. Consider the functions $\nu_\sigma: |\mathrm{C} Z| \to [0,1]$ for $\sigma \in Z$ as in \cite[Lemma 8.18]{hirshbergszabowinterwu2017}: They are $2(d+1)(d+2)(2d+3)$-Lipschitz and form a partition of unity for $|Z| \subseteq |\mathrm{C} Z|$ which is subordinate to the cover $\{ V_\sigma : \sigma \in Z \}$ where
\begin{gather*}
  V_\sigma
  = \left \{ (z_v)_v \in |Z| \mid z_v > z_{v'} \text{ for all } v \in \sigma \text{ and } v' \in Z_0\setminus \sigma \right \}
  \eqstop
\end{gather*}
Put
\begin{gather*}
  \varphi_\sigma = \nu_\sigma \circ \Phi: X \to [0,1]
  \eqstop
\end{gather*}
We get that $\sum_{\sigma \in I} \varphi_\sigma(x) = 1$ for $x \in A$ since $\Phi(A) \subseteq |Z|$ and $\sum_{\sigma} \nu_\sigma (z) = 1$ for $z \in |Z|$.  Since $\Phi$ is $(K, \frac{\varepsilon}{2 (d+1)(d+2)(2d+3)})$-F{\o}lner and $\nu_\sigma$ is $2(d+1)(d+2)(2d+3)$-Lipschitz, we find that for $g \in K$ we have
\begin{align*}
  |\nu_\sigma \circ \Phi(g x) - \nu_\sigma \circ \Phi(x)|
  & \leq
    2(d+1)(d+2)(2d+3) \mathrm{d}(\Phi(gx), \Phi(x)) \\
  & \leq
    2(d+1)(d+2)(2d+3) \frac{\varepsilon}{2(d+1)(d+2)(2d+3)} \\
  & =
    \varepsilon
    \text{,}
\end{align*}
so that $\varphi_\sigma$ is an $(K, \varepsilon)$-F{\o}lner function.  The remaining properties of $(\varphi_\sigma)_\sigma$ follow as in \cite{hirshbergszabowinterwu2017}: since the support of $\nu_\sigma$ is contained in an open star, the support of $\varphi_\sigma$ is contained in the interior of a tube, and the orthogonality condition on $\nu_\sigma$ implies the one on $\varphi_\sigma$.

We proceed to prove \ref{it:partition-plain} $\Rightarrow$ \ref{it:partition-fattened}. Let $K, L \subseteq G$ be compact, let $\varepsilon > 0$ and $A \subseteq X$ be compact.  Without loss of generality, we may enlarge $K$ and assume that $L \subseteq K$.

Choose constants $\delta_1, \delta_2 > 0$ such that
\begin{gather*}
  \delta_2 < \delta_1\text{,} \quad
  \frac{2(d+1)\delta_1}{(1 - (d+1)\delta_1)^2} < \frac{\varepsilon}{2} \quad \text{ and } \quad
  \frac{\delta_1 + \delta_2}{1 - (d+1)\delta_1} < \frac{\varepsilon}{2}
  \eqstop
\end{gather*}
By \ref{it:partition-plain} there is a partition of unity $(\psi_i)_{i \in I}$ for $A \subseteq X$ that consist of $(K, \delta_2)$-F{\o}lner functions, which are supported in the interior of a tube and satisfy the disjointness condition on their supports as in \ref{it:partition-plain}.  Put $\psi'_i = (\psi_i - \delta_1)_+: X \to [0, 1 - \delta_1]$.  Then $\psi'_i$ is $(K, \delta_1 + \delta_2)$-F{\o}lner. Further, the F{\o}lner condition for $\psi_i$ and the fact that $L \subseteq K$ imply that
\begin{gather*}
  L\supp \psi'_i
  \subseteq
  \{x \in X \mid \max_{g \in L} \psi_i(g^{-1}x) \geq \delta_1\}
  \subseteq
  \{x \in X \mid \psi_i(x) \geq \delta_1 - \delta_2\}
  \subseteq
  \supp \psi_i
  \eqstop
\end{gather*}
So $L \supp \psi'_i$ is contained in the interior of a tube.

Put 
\begin{gather*}
  \psi' = \left (\sum_{i \in I} \psi'_i \right ) + \left (\mathbb{1}_X - \sum_{i \in I} \psi_i \right )
  \eqstop
\end{gather*}
Then $\mathbb{1}_X - \psi' = \sum_{i \in I} \psi_i - \psi'_i$ and the image of the latter is contained in the interval $[0, (d+1)\delta_1]$, owing to the fact that $(\psi_i)_{i \in I}$ is the union of $d+1$ orthogonal families of functions. We note for later use that $\psi'$ is bounded from below by $1 - (d+1)\delta_1$. Put
\begin{gather*}
  \varphi_i = \frac{\psi'_i}{\psi'}
  \eqstop
\end{gather*}
Then $(\varphi_i)_{i \in I}$ is a partition of unity for $A \subseteq X$, since $(\sum_i \psi_i)(x) = 1$ for all $x\in A$. Further, $L\supp \varphi_i = L\supp \psi'_i$ is contained in the interior of a tube.  In order see that each $\varphi_i$ is a $(K, \varepsilon)$-F{\o}lner function, we calculate for $g \in K$ and $x \in A$ that
\begin{align*}
  |\varphi_i(gx) - \varphi_i(x)|
  & =
    \left |\frac{\psi'_i(gx)}{\psi'(gx)} - \frac{\psi_i(x)}{\psi'(x)} \right | \\
  & \leq
    \left |\frac{\psi'_i(gx) - \psi'_i(x)}{\psi'(gx)} \right | + \left | \frac{\psi'_i(x)}{\psi'(gx)} - \frac{\psi'_i(x)}{\psi'(x)} \right | \\
  & \leq
    \left |\frac{\psi'_i(gx) - \psi'_i(x)}{\psi'(gx)} \right | + |\psi'_i(x)| \left | \frac{\psi'(x) - \psi'(gx)}{\psi'(gx) \psi'(x)} \right | \\
  & \leq
    \frac{\delta_1 + \delta_2}{1 - (d+1)\delta_1} + 1 \frac{2(d+1)\delta_1}{(1 - (d+1)\delta_1)^2} \\
  & <
    \varepsilon
    \eqstop
\end{align*}

Let us next prove \ref{it:partition-fattened} $\Rightarrow$ \ref{it:cover-fattened}.  Fix $L \subseteq G$ be compact and $A \subseteq X$ be compact as given by \ref{it:cover-fattened}, let $K \subseteq G$ be some compact subset and let $(\varphi_i)_{i \in I}$ be a partition of unity for $A \subseteq X$ as provided by \ref{it:partition-fattened}.  Then $U_i = (\supp \varphi_i)^\circ$, $i \in I$ defines the desired open cover of $A$.


The implication \ref{it:cover-fattened} $\Rightarrow$ \ref{it:tube-dim} is immediate from the definition of the tube dimension.
\end{proof}

%%% Local Variables:
%%% mode: latex
%%% TeX-master: "classifiability"
%%% End:
