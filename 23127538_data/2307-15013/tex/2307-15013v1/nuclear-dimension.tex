\section{Nuclear dimension estimates from tube dimension}
\label{sec:nuclear-dimension}

In this section we obtain upper bounds for the nuclear dimension of crossed products by connected groups in terms of the tube dimension of a dynamical system.  We proceed by first establishing results on transformation groupoids, generalising results from \cite[Section 7]{hirshbergwu2021} and then employ these in a technically refined approach compared to the similar proof strategy of \cite[Theorem 8.1]{hirshbergwu2021}.



\subsection{Restriction to open tubes}
\label{subsec:restriction-tube}

We start by describing how restrictions to open tubes affect transformation groupoids and crossed product \Cstar-algebras.  For the present subsection, we fix the following notation.  Let $G \grpaction{} X$ be an action and set $\cG = G \ltimes X$. Let $K$ be a compact identity neighbourhood of $G$ and let $S$ be a $K$-slice in $X$. We are interested in the relationship between the restriction $\mathcal{G}|_{KS}$ to the corresponding tube $KS$ and the product groupoid $\pairtube = \pair(K) \times S$, where $\pair(K)$ is the pair groupoid associated with $K$ and $S$ is considered as a topological space. Note that by definition of a $K$-slice these groupoids have homeomorphic unit spaces since $(\mathcal{G}|_{KS})^{(0)} = KS$ while $(\pairtube)^{(0)} = K \times S$. We also set $\pairtubeopen = \pair(K^{\circ}) \times \boxint{S}$, which is an open subgroupoid of $\pairtube$.

\begin{proposition}
  \label{prop:pair-groupoid}
  Consider the map
  \begin{gather*}
    \tau \colon
    \pairtube \to \mathcal{G}|_{KS} \colon
    (g,h,x) \mapsto (gh^{-1},hx)
    \eqstop
  \end{gather*}
  Then the following statements hold true.
  \begin{enumerate}
  \item \label{it:pair-groupoid:cont-inj}
    $\tau$ is a continuous, injective groupoid homomorphism.
  \item \label{it:pair-groupoid:image}
    The image of $\tau$ is equal to
    \begin{gather*}
      \{ (g,x) \in \cG|_{KS} : \mathrm{proj}(x) = \mathrm{proj}(gx) \}
      \eqcomma
    \end{gather*}
    where $\mathrm{proj} \colon KS \to S$ denotes the tube projection.  Similarly, we have
    \begin{gather*}
      \tau(\pairtubeopen) = \{ (g,x) \in \cG|_{(KS)^\circ} : \mathrm{proj}(x) = \mathrm{proj}(gx) \}
    \end{gather*}
  \item \label{it:pair-groupoid:clopen-image}
    The set $\tau(\pairtubeopen)$ is clopen in $\cG|_{(KS)^\circ}$.
  \end{enumerate}
\end{proposition}
\begin{proof}
  First observe that $\tau$ is well-defined, as for $g,h \in K$ and $x \in S$ the computation
  \begin{align*}
    hx & = s(gh^{-1},hx) \eqcomma \\
    gx & = gh^{-1}hx = r(gh^{-1},hx)
    \eqcomma
  \end{align*}
  shows that $(gh^{-1},hx) \in \mathcal{G}|_{KS}$.

  We start by showing \ref{it:pair-groupoid:cont-inj}.  In order to show that $\tau$ is a groupoid homomorphism, we compute for $(g,h,x),(h,k,x) \in \pairtube$ that
  \begin{gather*}
    \tau((g,h,x),(h,k,x))
    =
    \tau(g,k,x)
    =
    (gk^{-1},kx)
    =
    (gh^{-1},hx)(hk^{-1},kx)
    =
    \tau(g,h,x)\tau(h,k,x)
    \eqstop
\end{gather*}
  Its defining formula shows that $\tau$ is continuous. To show injectivity, suppose that for a pair of elements $(g,h,x),(g',h',x') \in \pairtube$ we have $\tau(g,h,x) = \tau(g',h',x')$.  In different terms, we suppose $(gh^{-1}, hx) = g'(h')^{-1}, h'x')$.  Then $hx=h'x'$ implies $h=h'$ and $x=x'$ by the definition of a slice, so that subsequently $gh^{-1} = g'h'^{-1} = g'h^{-1}$ forces $g=g'$. This shows that $\tau$ is injective.

  We next prove \ref{it:pair-groupoid:image}.  If $(g,h,x) \in \pair(K) \times S$, its image under $\tau$ belongs to the claimed set, since $\mathrm{proj}(hx) = x = \mathrm{proj}(gx) = \mathrm{proj}(gh^{-1}hx)$.  Conversely, given an element $(g',x') \in \cG|_{KS}$ that satisfies $\mathrm{proj}(x') = \mathrm{proj}(g'x')$, we write $x' = hx$ and $g'x' = gx$ for certain $h,g \in K$ and $x \in S$. Then it follows that $g'hx = gx$ so that freeness of the action $G \grpaction{} X$ implies that $g' = gh^{-1}$. Hence we obtain that $(g',x') = (gh^{-1},hx) = \tau(g,h,x)$ lies in the image of $\tau$.  Using equation \eqref{eq:box_interior} of \Cref{lem:box_interior}, saying that $(KS)^\circ = K^{\circ}\boxint{S}$, we infer that also $\tau(\pairtubeopen) = \{ (g,x) \in \cG|_{(KS)^\circ} : \mathrm{proj}(x) = \mathrm{proj}(gx) \}$ holds.

  Let us finish by proving \ref{it:pair-groupoid:clopen-image}.  Since $\mathrm{proj}$ is continuous, the description of $\tau(\pairtubeopen)$ obtained in \ref{it:pair-groupoid:image} shows that $\tau(\pairtubeopen)$ is closed in $\cG|_{(KS)^{\circ}}$. Further, since $\tau$ is a continous injection from a compact space to a Hausdorff space, it is a homeomorphism onto its image. In particular $\tau(\pairtubeopen)$ is open in $\cG|_{KS}$ since $\pairtubeopen$ is open in $\pairtube$. To conclude, we use the fact that $\cG|_{(KS)^{\circ}}$ is an open subset of $\cG|_{KS}$, so that it follows that $\tau(\pairtubeopen)$ is open in $\cG|_{(KS)^{\circ}}$.
\end{proof}
%
%
%
Combining the previous result with \cite[Lemma 6.1 and Theorem 6.2]{hirshbergwu2021}, we obtain the following maps.
\begin{corollary}
  \label{cor:conditional-expectation}
  Consider the inclusion $\pairtubeopen \subseteq \cG|_{(KS)^{\circ}}$ given by the map $\tau$ of \cref{prop:pair-groupoid}. The following statements hold true.
  \begin{enumerate}
  \item The inclusion $\contc(\pairtubeopen) \to \contc(\cG|_{(KS)^{\circ}})$ extends to an injective *-homomorphism $\Cstarred(\cH) \to \Cstarred(\cG|_{(KS)^{\circ}})$.
  \item The restriction $\contc(\cG|_{(KS)^{\circ}}) \to \contc(\pairtubeopen)$ extends to a conditional expectation $\rE \colon \Cstarred(\cG|_{(KS)^{\circ}}) \to \Cstarred(\pairtubeopen)$.
  \end{enumerate}
\end{corollary}
%
%
%
The next lemma strongly localises subsets of tubes in the presence of connectedness.  Its special case for $\RR$ was used without explicit reference in \cite[p.36, line after (8.1.4)]{hirshbergwu2021}.
\begin{lemma}
  \label{lem:concentration-in-slices}
  Let $G \grpaction{} X$ be an action.  Let $L \subseteq G$ be a connected open subset and let $S$ be a $K$-slice.  If $Lx \subseteq (KS)^\circ$ for some $x \in X$, then there is $s \in S$ such that $Lx \subset Ks$.
\end{lemma}
\begin{proof}
  Write $B = KS$, denote by $\pi \colon B \to S$ the projection on the slice $S$ and consider the map $\psi \colon L \to S$ given by $l \mapsto \pi(lx)$.  We have to show that the image of $\psi$ is a singleton.  Since $\psi$ is continuous and $L$ is connected its image $\psi(L)$ is connected too.  Assume that $s \in \psi(L)$.  Since $Lx \subseteq B^\circ$, there is $g \in L$ and $k \in K^\circ$ such that $gx = ks$. Let $U$ be a symmetric identity neighbourhood in $G$ such that $Uk \subseteq K$ and $Ug \subseteq L$.  Then $ugx = uks \in Ks$ for all $u \in U$, so that $\psi^{-1}(\{s\})$ has non-empty interior.  So $L = \bigsqcup_{s \in S} \psi^{-1}(\{s\})$ is a partition of $L$ into sets with non-empty interior.  Since $G$ is second countable, this implies that the image of $\psi$ is countable.  So $\psi(L)$ is a countable, connected Hausdorff space, and hence a singleton.
\end{proof}



\subsection{The estimate}
\label{sec:estimate}

The aim of this section is to establish bounds on the nuclear dimension of crossed products associated with actions connected amenable groups in terms of the tube dimension of the action.  We will apply this to Lie groups of polynomial growth, for which estimates on the tube dimension can be established.



The next results shows that F{\o}lner functions, as introduced in \cref{sec:partitions}, give rise to quasi-central elements in the $\Lone$-crossed products.  This replaces arguments based on Lipschitz functions used in \cite{hirshbergszabowinterwu2017,hirshbergwu2021} that are only adapted to $\RR$-flows.
\begin{lemma}
  \label{lem:folner-quasi-central-for-I-norm}
  Let $G \curvearrowright X$ be an action and consider the groupoid $G \ltimes X$. Let $K \subseteq G$ be a symmetric compact subset and let $A \subseteq X$ be a compact subset.  Assume that $(\vphi_i)_{i \in I}$ are $(K, \veps)$-F{\o}lner functions in $\contb(X)$ such that $(\vphi_i^2)_i$ is a partition of unity for $A$.  Further assume that $I = I^{(0)} \sqcup \dotsm \sqcup I^{(d)}$ such that $\supp \vphi_i \cap \supp \vphi_j = \emptyset$ for all $0 \leq l \leq d$ and $i,j \in I^{(l)}$.  Then for all $a \in \contc(K \times A) \subseteq \contc(G \ltimes X)$, we have
  \begin{gather*}
    \|\sum_i \vphi_i a \vphi_i - a\|_I \leq 2 (d + 1) \veps^2 \|a\|_I
    \eqstop
  \end{gather*}
\end{lemma}
\begin{proof}
  Let $a \in \contc(G \times X)$ and $f \in \contb(X)$.  Then we have
  \begin{gather*}
    (af)(g,x) = a(g,x) f(x) \quad \text{ and } \quad (fa)(g,x) = a(g,x)f(gx)
    \eqstop
  \end{gather*}
  So we calculate for $a \in \contc(K \times A)$, for $g \in G$ and $x \in A$ that
  \begin{align*}
    |\sum_i \vphi_i(x)\vphi_i(gx) - 1|
    & =
      |\sum_i \vphi_i(x)\vphi_i(gx) - \vphi_i^2(x)| \\
    & =
      |\sum_i \vphi_i(x)(\vphi_i(gx) - \vphi_i(x))| \\
    & \leq
      \sum_i \vphi_i^2(x) \cdot \sum_i (\vphi_i(gx) - \vphi_i(x))^2 \\
    & \leq
      1 \cdot 2 (d + 1) \veps^2
      \eqstop
  \end{align*}
  For the last inequality we used the fact that $\vphi_i$ is a $(K, \veps)$-F{\o}lner function for each $i$ as well as the fact that there are at most $d + 1$ indices $i$ satisfying $\vphi_i(x) \neq 0$ and likewise there are at most $d + 1$ indices $i$ satisfying $\vphi_i(gx) \neq 0$. We can now estimate the following integrals for fixed $x$:
  \begin{align*}
    \int_G | \sum_i \vphi_i a \vphi_i -  a|(g,x) \rmd g
    & =
      \int_G |a(g,x)|  \cdot |\sum_i \vphi_i(gx) \vphi_i(x) -  1| \rmd g \\
    & \leq
      2 (d + 1) \veps^2 \int_G |a(g,x)|   \rmd g \\
    & \leq
      2 (d + 1) \veps^2 \|a\|_I
  \end{align*}
  and similarly
  \begin{align*}
    \int_G | \sum_i \vphi_i a \vphi_i -  a|(g^{-1},gx) \rmd g
    & =
      \int_G |a(g^{-1},gx)| \cdot  |\sum_i \vphi_i(x) \vphi_i(g^{-1}x) -  1| \rmd g \\
    & \leq
      2 (d + 1) \veps^2 \|a\|_I
    \eqstop
  \end{align*}
Taking the supremum of both terms over $x \in X$, we conclude the proof.
\end{proof}
%
% 
%
We are now ready to state and prove our main result, establishing nuclear dimension bounds for free actions in terms of the tube dimension and the dimension of the base space.
\begin{theorem}
  \label{thm:finite-nuclear-dimension}
  Let $G \grpaction{} X$ be a free action of a connected amenable group on a second-countable, locally compact Hausdorff space. Then
  \begin{gather*}
    \dimnuc(\conto(X) \rtimes G) 
    \leq
    (\dim X + 1) (\tubedim(G \grpaction{} X) + 1) - 1
  \end{gather*}
\end{theorem}
\begin{proof}
  Set $\mathcal{G} = G \ltimes X$. Let $B_r \subseteq \contc(G,\contc(X))$ be the $r$-Ball with respect to the $I$-norm. Then $\bigcup_{r \in \NN} \ol{B_r}^{\| \cdot \|} = \conto(X) \rtimes G$, so that a Baire category argument shows that there is some $R > 0$ such that the closure of $B_R$ contains the unit ball of $\conto(X) \rtimes G$. Let $F$ be a finite subset of $B_R$ and let $\veps > 0$. Let $L \subseteq G$ and $A \subseteq X$ be compact subsets such that $F \subseteq \contc(L, \contc(A))$.  Write $d = \tubedim(G \grpaction{} X)$ and pick $\delta < ( \frac{\veps}{2(d+1)R} )^{1/2}$.  Then \cref{prop:characterisation-tube-dimension} provides a finite family of functions $(\vphi_i)_{i \in I}$ such that
  \begin{itemize}
  \item $(\vphi_i^2)_{i \in I}$ is a partition of unity for $A \subseteq X$
  \item each function $\vphi_i$ is $(L, \delta)$-F{\o}lner
  \item $L \supp \vphi_i$ is contained in a the interior of a tube $K_i S_i$, and
  \item there is a partition $I = I^{(0)} \sqcup \hdots \sqcup I^{(d)}$ such that for all $l \in \{0, \dotsc, d\}$ and all $i,j \in I^{(l)}$ we have
    \begin{gather*}
      L\supp \varphi_i \cap L\supp \varphi_j = \emptyset
      \text{.}
    \end{gather*}
  \end{itemize}
  For each $i \in I$ denote by $\pairtubeopeni$ the product groupoid $\pair(K_i^{\circ}) \times \boxint{S_i}$ as in \cref{subsec:restriction-tube}. Let $\Psi_i \colon \Cstar(\pairtubeopeni) \to \Cstar(\cG)$ be the inclusion as in \cref{cor:conditional-expectation} and let $\rE_i \colon \Cstar(\cG|_{(K_iS_i)^{\circ}}) \to \Cstar(\pairtubeopeni)$ be the conditional expectation from \cref{cor:conditional-expectation}.  We observe that compression with $\varphi_i$ is a completely positive map on $\Cstar(\cG)$, which is contractive since $\| \varphi_i \|_\infty \leq 1$ and has its image in $\Cstar(\cG|_{(K_i S_i)^{\circ}})$ since $\supp(\varphi_i) \subseteq (K_iS_i)^{\circ}$.  Thus, we obtain a completely positive contractive map $\Phi_i \colon \Cstar(\cG) \to \Cstar(\pairtubeopeni)$ by putting
  \begin{gather*}
    \Phi_i(a)
    =
    \rE_i(\vphi_i a \vphi_i)
    \eqcomma \quad \text{for all } a \in \Cstar(\cG)
    \eqstop
  \end{gather*}

  For $l \in \{0, \dotsc, d\}$ put $A^{(l)} = \bigoplus_{i \in I^{(l)}} \Cstar(\pairtubeopeni)$ and $A = \bigoplus_{l=0}^d A^{(l)}$. Let $\Phi^{(l)} = \bigoplus_{i \in I^{(l)}} \Phi_i\colon \Cstar(\cG) \to A^{(l)}$ and let $\Psi^{(l)} = \bigoplus_{i \in I^{(l)}} \Psi_i\colon A^{(l)} \to \Cstar(\cG)$. Then each $\Phi^{(l)}$ is completely positive and contractive, so $\Phi = \bigoplus_{l \in \{0, \dotsc, d\}} \Phi^{(l)} \colon \Cstar(\cG) \to A$ is also completely positive and contractive.  Moreover, each $\Psi^{(l)}$ is a *-homomorphism, in particular an order zero contraction, so the number $d^{(l)}$ from \Cref{lem:nuclear_lemma} is zero. It follows that $\Psi = \sum_{l \in \{0, \dotsc, d\}} \Psi^{(l)} \colon A \to \Cstar(\cG)$ is completely positive.

  Since $L \supp \vphi_i$ is contained in the interior of $K_i S_i$ and $F \subseteq \contc(L, \contc(A))$ we find that $\vphi_i a \vphi_i \in \Cstar(\pair(K_i^\circ) \times S_i^\circ)$ when $a \in F$.  Indeed,
  \begin{gather*}
    0 \neq \vphi_i a \vphi_i(g,x)
    = \vphi_i(gx)\vphi_i(x) a(g,x)
  \end{gather*}
  implies that $g \in L$ and $gx, x \in \supp \vphi_i$, which in combination with \cref{lem:concentration-in-slices} shows that there is $s \in S_i$ and $k_1, k_2 \in K_i$ such that such that $x = k_1 s$ and $gx = k_2s$.  By freeness of the $G$-action, this implies $g = k_2k_1^{-1}$.  It follows by \cref{prop:pair-groupoid} that $(g, x) = (k_2k_1^{-1}, k_1 s)$ lies in $\tilde{\mathcal{H}}_i = \pair(K_i^\circ) \times \boxint{S_i}$ when identified with its image in $\mathcal{G}$ under $\tau$.  Combined with \cref{lem:folner-quasi-central-for-I-norm} this shows that for all $a \in F$ we have
  \begin{align*}
    \|\Psi \circ \Phi(a) - a\|
    & =
      \| \sum_{i \in I} \vphi_i a \vphi_i - a\| \\
    & \leq
      \| \sum_{i \in I} \vphi_i a \vphi_i - a\|_I \\
    & \leq
      2(d+1)\delta^2 R
    < \veps
      \eqstop
  \end{align*}

  To finish the proof, we observe that by \Cref{ex:pair_groupoid} and \Cref{ex:top_space_groupoid} the $\Cstar$-algebra $\Cstar(\pairtubeopeni)$ is isomorphic to $\Cstar(\pair(K_i^{\circ}) \times \boxint{S_i}) \cong {\cK \otimes \conto(\boxint{S_i})}$ where $\mathcal{K}$ denotes the compact operators on a separable, infinite-dimensional Hilbert space. This has nuclear dimension at most $\dim (\boxint{S_i}) \leq \dim X$, so that \cref{lem:nuclear_lemma} gives
  \begin{gather*}
    \dimnuc(\conto(X) \rtimes G)
    \leq 
    \sum_{l = 0}^d (\dim(X) + 1)(0 + 1) - 1
    =
    (d + 1) (\dim(X) + 1) - 1
    \eqstop
  \end{gather*}
\end{proof}
%
%
%
The previous theorem was formulated in a general form, and we next specialise to concrete nuclear dimension estimates in the context of actions of connected Lie groups of polynomial growth, thanks to our results on the tube dimension established in \cref{sec:box-dimension}.  Recall from \cref{sec:polynomial-growth} Breuillard's numbers $\rmd(G)$ describing the growth asymptotics of balls in a group of polynomial growth such as a nilpotent Lie group.
\begin{corollary}
  \label{cor:finite-nuclear-dimension}
Let $G \grpaction{} X$ be a free action of a connected Lie group of polynomial growth on a second-countable, locally compact Hausdorff space. Then
\begin{gather*}
  \dimnuc(\conto(X) \rtimes G) + 1 \leq
  (\dim X + 1)^2 11^{\rmd(G)}
\end{gather*}
\end{corollary}
\begin{proof}
  By \cref{thm:covering} combined with \cref{cor:K-slices-exist-polynomial-growth} the tube dimension of $G \grpaction{} X$ is bounded by $11^{\rmd(G)} \cdot (\dim X + 1) - 1$.  We hence obtain the estimate from \cref{thm:finite-nuclear-dimension}.
\end{proof}



\begin{remark}\label{rem:nuclear_dimension_bound_difference}
We remark on the difference between the nuclear dimension bound obtained in \Cref{cor:finite-nuclear-dimension} and \cite[Corollary 10.2]{hirshbergszabowinterwu2017} or \cite[Theorem 8.1]{hirshbergwu2021}. These differences are a result of the following two observations about \cite{kasprowskiruping17}:

Firstly, on p.\ 1214 in \cite{kasprowskiruping17} in the final line of the proof of Lemma 5.1, it is written that if $x \in \Phi_{[-4\alpha,4\alpha]}(gB_i)$ then there is $\beta \in \{ -4\alpha, -2\alpha, 0, 2\alpha, 4\alpha \}$ such that $\Phi_\beta(x) \in \Phi_{[-\alpha,\alpha]}(gB_i)$. While this is true, there is a more optimal choice here: $\beta$ can be chosen from the set $\{ -3\alpha,-\alpha,\alpha,3\alpha \}$ which has one less element. This affects the multiplicity in the statement of Kasprowski-R{\"u}ping's Lemma 5.1 (2) and consequently also Theorem 5.2, which is cited in \cite[Theorem 8.8]{hirshbergszabowinterwu2017} and \cite[Theorem 3.11]{hirshbergwu2021}.

Secondly, in the proof of \cite[Theorem 5.2]{kasprowskiruping17} it is written that by Lemma 5.1 (1) every element in $X_{>\gamma}$ is contained in an open set of the form $\Phi_{(-3\alpha,3\alpha)}(gB_i^k)$ for some $g \in G$, $i \in \NN$, $k=0,\ldots, \mathrm{ind} X$. Upon inspection of this claim it seems however that Lemma 5.1 (1) only guarantees that the sets $\Phi_{(-4,\alpha,4\alpha)}(gB_i^k)$ cover $X_{>\gamma}$. Thus, one will in the end need to control the multiplicity of the collection $\mathcal{B} = \{ \Phi_{(-5\alpha,5\alpha)}(B_i^k) \mid i \in \bN, k=0, 1 \ldots, \operatorname{ind} X \}$ rather than $\mathcal{B} = \{ \Phi_{(-4\alpha,4\alpha)}(B_i^k) \mid i \in \bN, k=0,1\ldots,\operatorname{ind}X \}$. This is exactly the situation we end up with in the present paper in the proof of \Cref{thm:covering} after Claim~\ref{claim:dimension_reduction}, and we account for this with the choice of $a=10$ in the very beginning of the proof of \Cref{thm:covering}.
\end{remark}


%%% Local Variables:
%%% mode: latex
%%% TeX-master: "classifiability"
%%% End:
