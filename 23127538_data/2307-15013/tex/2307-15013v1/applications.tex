\section{Classifiable C*-algebras associated with FLC point sets}
\label{sec:classifiable-point-sets}

In this section we describe examples of classifiable \Cstar-algebras arising from Theorem~\ref{thm:finite-nuclear-dimension}.  In greatest generality, abstract conditions on a point set in a connected Lie group of polynomial growth, that ensure that its \Cstar-algebra introduced in \cite{enstadraum2022} is classifiable.  Every nilpotent Lie group is of polynomial growth. Subsequently specialising these results, our main source of concrete examples arise as approximate lattices constructed from arithmetic lattices in products of nilpotent Lie groups.

Let us start by introducing some notation.  We recall that $G$ denotes a locally compact second countable group and we denote by $\cC(G)$ the Chabauty space of all closed subsets of $G$ described whose topology has a subbasis
\begin{align*}
  \cO_K & = \{A \in \cC(X) \mid A \cap K = \emptyset\} \\
  \cO^U & = \{A \in \cC(X) \mid A \cap U \neq \emptyset\}
          \eqcomma
\end{align*}
where $K \subseteq X$ is compact and $U \subseteq X$ is open.  For a closed subset $\Lambda \subseteq G$, we denote its hull and its discrete hull, respectively, by 
\begin{gather*}
  \Omega(\Lambda)
  =
  \ol{ \{g\Lambda\mid g\in G\} }
  \subseteq
  \cC(G)
  \quad
  \text{and}
  \quad
  \Omega_0(\Lambda) = \{P \in \Omega \mid e \in P\}
  \eqstop
\end{gather*}

Using the discrete hull, a Delone set of finite local complexity (FLC) in $G$ can be defined as a closed subset $\Lambda \subseteq G$ such that
\begin{itemize}
\item for every compact subset $K$ there is a finite set $\cF \subseteq \cC(G)$ such that for every $P \in \Omega_0(\Lambda)$ there is $F \in \cF$ such that $P \cap K = F$, and
\item $\emptyset \notin \Omega(\Lambda)$.
\end{itemize}


In order to apply \cref{cor:finite-nuclear-dimension} to dynamical systems arising from point sets, we need the following result, which is well-known for point sets in $\bR^d$, see e.g.\ \cite[Corollary 2.1]{bellissardherrman00}.

\begin{proposition}
  \label{prop:FLC-implies-tdlc}
  Let $\Lambda \subseteq G$ be an FLC Delone set.  Then $\Omega_0(\Lambda)$ is totally disconnected.
\end{proposition}
\begin{proof}
  Let $U$ be a non-empty open subset of $\Omega_0(\Lambda)$.  We have to find a non-empty compact open subset of $U$.  There are relatively compact open subsets $V_1, \dotsc, V_n \subseteq G$ and a compact subset $K \subseteq G$ such that $\emptyset \neq \cO_K\cap\cO^{V_1}\cap \dotsc \cap\cO^{V_n} \subseteq U$.  Let $C = K \cup \ol{V_1} \cup \dotsm \cup \ol{V_n}$.  Since $\Lambda$ is FLC, the set $\cF=\{Q \cap C \mid Q \in \Omega_0(\Lambda)\}$ is finite.  Since $\cF$ consists of finitely many finite subsets of $G$, there is an open identity neighbourhood $V \subseteq G$ satisfying the following condition: for all $F \in \cF$, we have $Q \cap C = F$ if and only if $Q \cap gV \neq \emptyset$ for all $g \in F$ and $Q \cap (C \setminus \bigcup_{g \in F} gV) = \emptyset$.  Writing
  \begin{gather*}
    U_F = \cO_{C \setminus \bigcup_{g \in F} gV} \cap \bigcap_{g \in F} \cO^{gV}
  \end{gather*}
  this can be restated as $Q \cap C = F$ if and only if $Q \in U_F$.  So the open sets $U_F$ for $F \in \cF$ are pairwise disjoint and cover $\Omega_0(\Lambda)$.  They are hence clopen.  Since $U_F \subseteq \cO_K\cap \cO^{V_1} \cap \dotsc \cap \cO^{V_n} \subseteq U$ for some $F \in \cF$, and since $U_F \neq \emptyset$ for all $F \in \cF$ this finishes the proof.  
\end{proof}
%
%
%
Since total disconnectedness is equivalent to zero-dimensionality, we can now derive dimension bounds for the hull of an FLC point set.
\begin{corollary}
  \label{cor:dimension-punctured-hull}
  Let $\Lambda \subseteq G$ be an FLC Delone set in a Lie group. Then $\dim(\Omega(\Lambda)) \leq \dim G$.
\end{corollary}
\begin{proof}
  Since $\Lambda$ is FLC, it is also uniformly discrete, that is there exists an open identity neighbourhood $U \subseteq G$ such that $|U \cap P| \leq 1$ for all $P \in \Omega(\Lambda)$.  

  Put $V = \{P \in \Omega(\Lambda) \mid P \cap U \neq \emptyset\}$, which is an open subset of $\Omega(\Lambda)$ containing $\Omega_0(\Lambda)$.  The map $\vphi_0 \colon V \to U$ satisfying $\{\vphi_0(P)\} = P \cap U$ for all $P \in V$ is continuous by definition of the Fell topology.  So the map $\vphi \colon V \to U \times \Omega_0(\Lambda)$ defined by $\vphi(P) = (\vphi_0(P), \vphi_0(P)^{-1}P)$ is also continuous.  A continuous inverse for $\vphi$ is obtained by restricting the action map for $G \grpaction{} \Omega(\Lambda)$ to obtain a map $\psi \colon U \times \Omega_0(\Lambda) \to V$ satisfying $\psi(g, P) = gP$.  So we infer that $V \cong U \times \Omega_0(\Lambda)$.
  
  In order to conclude the proof, we calculate
  \begin{gather*}
    \dim (\Omega(\Lambda)) = \dim(V) = \dim(U \times \Omega_0(\Lambda)) = \dim(U) + \dim(\Omega_0(\Lambda)) =  \dim(G)
    \eqcomma
  \end{gather*}
  where we used the equalities $\dim (\Omega(\Lambda)) = \sup_{g \in G} \dim (gV) = \dim V$ as well as $\dim U = \sup_{g \in G} \dim gU = \dim G$ which are justified by \cref{prop:dim_properties} together with the fact that $\dim(\Omega_0(\Lambda)) = 0$, which follows from \cref{prop:FLC-implies-tdlc}.
\end{proof}
%
%
%
In order to formulate the first main result of this section, we recall some terminology.  A point set $\Lambda$ is repetitive if for every compact set $K \subseteq G$ there exists a compact set $L \subseteq G$ such that the following statement holds: for all $h \in G$ and all $P \in \Omega(\Lambda)$ there is $g \in G$ such that $g(K \cap P) \subseteq hL \cap \Lambda$.  It follows from \cite[Propositions 2.2 and 2.4]{beckushartnickpogorzelski2020} that an FLC Delone set $\Lambda$ is repetitive if and only if the action $G \grpaction{} \Omega(\Lambda)$ is minimal.  We further recall that, $\Lambda$ is called aperiodic if the action $G \grpaction{} \Omega(\Lambda)$ is free.
%
%
%
In \cite{enstadraum2022} groupoids associated to point sets were introduced.  In particular, if $\Lambda \subseteq G$ is an FLC Delone set, then 
\begin{gather*}
  \cG(\Lambda) = G \ltimes \Omega(\Lambda)|_{\Omega_0(\Lambda)}
  \eqstop
\end{gather*}
is an {\'e}tale groupoid.  The reduced and universal \Cstar-algebra of the point set $\Lambda$ are then by definition
\begin{gather*}
  \Cstarred(\Lambda) = \Cstarred(\cG(\Lambda))
  \quad
  \text{and}
  \quad
  \Cstar(\Lambda) = \Cstar(\cG(\Lambda))
  \eqstop
\end{gather*}
If $G$ is amenable, e.g. a nilpotent Lie group, these \Cstar-algebras are isomorphic.  We will need the following result to analyse $\Cstar(\Lambda)$.
\begin{lemma}
  \label{lem:stable-isomorphism}
  The \Cstar-algebras $C_0(\Omega(\Lambda)) \rtimes_{\mathrm{red}} G$ and $\Cstarred(\Lambda)$ are stably isomorphic.
\end{lemma}
\begin{proof}
  It follows from \cite[Proposition 3.8]{enstadraum2022} and \cite[Example 2.40]{williams2019-toolkit} that the groupoids $G \ltimes \Omega(\Lambda)$ and $\cG(\Lambda)$ are equivalent.  So \cite[Theorem 4.9]{williams2019-toolkit} says that their \Cstar-algebras are Morita equivalent.  Since both are separable, as $\cG(\Lambda)$ is second countable, it follows that $\Cstar(\Lambda) = \Cstarred(\cG(\Lambda))$ is stably isomorphic with $\cont(\Omega(\Lambda)) \rtimes_{\mathrm{red}} G$. 
\end{proof}



\begin{theorem}
  \label{thm:point-set-nuclear-dimension}
  Let $\Lambda$ be an aperiodic, FLC Delone set in a connected Lie group of polynomial growth. Then $\Cstar(\Lambda)$ has finite nuclear dimension.  If additionally, $\Lambda$ is repetitive then the \Cstar-algebra $\Cstar(\Lambda)$ is classifiable.
\end{theorem}
\begin{proof}
  First assume that $\Lambda$ is an aperiodic, FLC Delone set in a connected, simply connected, nilpotent Lie group $G$.  By \Cref{lem:stable-isomorphism}, the \Cstar-algebras $\Cstar(\Lambda)$ and $\Cstarred(\cG(\Lambda))$ are stably isomorphic.  It hence suffices to show that $\cont(\Omega(\Lambda)) \rtimes G$ has finite nuclear dimension.  This follows from \Cref{thm:finite-nuclear-dimension}, making use of the assumptions of aperiodicity, which ensures that $G \grpaction{} \Omega(\Lambda)$ is free.

  Let us now additionally assume that $\Lambda$ is repetitive.  We already know that $\Cstar(\Lambda)$ is separable and has finite nuclear dimension and we will show that $\Cstar(\Lambda)$ is simple, unital, non-elementary and in the UCT class.  It is unital, since $\cG(\Lambda)$ is {\'e}tale and $\cG(\Lambda)\nought = \Omega_0(\Lambda)$ is compact.  Further, since $\cG(\Lambda)$ is infinite, it follows that $\Cstar(\Lambda)$ is not isomorphic to a matrix algebra, and hence not a (unital) elementary \Cstar-algebra. As $G$ is an amenable group, also $G \ltimes \Omega(\Lambda)$ and hence $\cG(\Lambda)$ are amenable groupoids.  By the work of Tu \cite[Proposition 10.7]{tu99} it follows that $\Cstar(\Lambda)$ is in the UCT class.

  It remains to argue that $\Cstar(\Lambda)$ is simple.  Since $\Lambda$ is repetitive the groupoid $G \ltimes \Omega(\Lambda)$ is minimal.  By \cite[Lemma 2.41]{williams2019-toolkit} the orbit spaces of $G \ltimes \Omega(\Lambda)$ and $\cG(\Lambda)$ are homeomorphic so that also the latter groupoid is minimal.  Since $\Lambda$ is aperiodic, $G \ltimes \Omega(\Lambda)$ is principal and so is its subgroupoid $\cG(\Lambda)$.  Finally, observe that $\Cstar(\cG(\Lambda)) = \Cstarred(\cG(\Lambda))$ by amenability of $\cG(\Lambda)$, so that \cite[Theorem 5.1]{brownclarkfarthingsims2014} applies and proves simplicity of $\Cstar(\Lambda)$. 
\end{proof}
%
%
%
We are next going to provide examples to which the previous theorem applies.  It is necessary to verify among others the condition of aperiodicity.  While for repetitive point sets in abelian groups, \cite[Proposition 5.5]{baakegrimm13} shows that aperiodicity is equivalent to the formally weaker requirement that the point stabiliser $G_\Lambda = \{g \in G \mid g\Lambda = \Lambda\}$ is trivial, this is not known in general. The known proof in the abelian case does not generalise to nilpotent groups.  However, in the specific context of model sets introduced below, aperiodicity can be easily verified as we will show.



A \textit{cut-and-project scheme} is a triple $(G,H,\Gamma)$, where $G$ and $H$ are locally compact groups, and $\Gamma$ is a lattice in $G\times H$ such that the projection $\pi_{G}:G\times H\to G$ is injective when restricted to $\Gamma$ and the image of $\Gamma$ under the projection $\pi_H:G\times H\to H$ is dense in $H$.  In view of results obtained in \cite{bjorklundhartnickpgorzelski2018}, we are interested in \emph{regular model sets}, which are obtained as
\begin{gather*}
  \Lambda = \pi_G(\Lambda\cap (G\times W))
\end{gather*}
for windows $W \subseteq H$ satisfying the following conditions: $W$ is a regular compact subset, its boundary is Haar negligible, it satisfies $\partial W \cap \pi_H(\Gamma) = \emptyset$ and $\mathrm{stab}_H(W)=\{e\}$.  If $\Lambda$ is a (regular) model set, then it has finite local complexity by \cite[Proposition 2.13 (i)]{bjorklundhartnick2018}.  As lattices in nilpotent Lie groups are automatically uniform, it follows that (regular) model sets constructed from cut-and-project schemes of nilpotent Lie groups are Delone sets.  Not every lattice in a product of nilpotent Lie groups however gives rise to a cut-and-project scheme and we need to require slightly more: a lattice in a product of groups $\Gamma \leq G \times H$ is \emph{irreducible} if 
the projections onto $G$ and $H$, respectively, are injective and have dense image when restricted to $\Gamma$.



\begin{corollary}
  \label{cor:irreducible-lattice-classifiable}
  Let $G$ be a connected, simply connected nilpotent Lie group and let $\Lambda \subseteq G$ be a regular model set arising from a cut-and-project scheme $(G, H, \Gamma)$ where $H$ is another nilpotent Lie group and $\Gamma$ is an irreducible lattice.  Then $\Cstar(\Lambda)$ is classifiable.
\end{corollary}
\begin{proof}
  By the previous discussion $\Lambda \subseteq G$ is an FLC Delone set.  So \Cref{thm:point-set-nuclear-dimension} applies if we can show that $\Lambda$ is repetitive and aperiodic.  The former follows form \cite[Proposition 3.3]{bjorklundhartnickpgorzelski2018} combined with \cite{beckushartnickpogorzelski2020}.  We have to show that $\Lambda$ is aperiodic.

  Consider the $G$-space $Y = (G \times H) / \Gamma$ where $G$ acts by multiplication on the left in the first component.  By \cite[Theorem 3.1]{bjorklundhartnickpgorzelski2018} there exists a $G$-equivariant map $\beta \colon \Omega(\Lambda) \to Y$.  Since $G_P \subseteq G_{\beta(P)}$ for every $P \in \Omega(\Lambda)$, it suffices to show that the action of $Y$ is a free $G$-space.  If $g' (g,h)\Gamma = (g,h)\Gamma$ for some $g,g' \in G$ and $h \in H$, then $(g^{-1}g'g, e)\Gamma = \Gamma$.  Since $\Gamma \subseteq G \times H$ is irreducible, the projection $G \times H \to H$ is injective when restricted to $\Gamma$, so that we can conclude that $g^{-1}g' g$ and hence also $g'$ is trivial.
\end{proof}
%
%
%
Concrete examples of irreducible lattices in products of connected, simply connected, nilpotent Lie groups can be obtained from arithmetic constructions.  We provide a class of examples based on the Heisenberg group and mention the work of Machado \cite{machado2020-approximate-lattices-nilpotent}, which shows that all approximate lattices in connected, simply connected, nilpotent Lie groups are of arithmetic origin.
\begin{example}
  \label{ex:arithmetic-lattice-classifiable}
  Denote by $\rH_n$ the Heisenberg group of dimension $2n + 1$.  Let $d \in \ZZ$ be a square-free integer and consider the embedding
  \begin{gather*}
    \ZZ(\sqrt{d}) \to \RR \times \RR \colon
    a+b\sqrt{d} \mapsto (a+b\sqrt{d} , a-b\sqrt{d})
    \eqstop
  \end{gather*}
  It induces an embedding of the Heisenberg groups
  \begin{gather*}
    \rH_n(\ZZ(\sqrt{d})) \to \rH_n(\RR) \times \rH_n(\RR)
  \end{gather*}
  whose image is an irreducible lattice.  We identify $\rH_n(\ZZ(\sqrt{d}))$ with this lattice.  Taking for $W$ any metric ball in $\rH_n(\RR)$, we obtain the regular model set $\Lambda = \pi_1(\rH_n(\ZZ(\sqrt{d})) \cap (\rH_n(\RR) \times W))$, where $\pi_1$ is the first factor projection.  \Cref{cor:irreducible-lattice-classifiable} applies and gives rise to a countable family of classifiable \Cstar-algebras parametrised by the pairs $(n,d)$ for $n \geq 1$ and $d$ a square-free integer.  We note also that instead of $\ZZ[\sqrt{d}]$ one could consider the ring of algebraic integers in $\QQ(\sqrt{d})$, which is a finite index extension if $d \equiv 1 (\mathrm{mod} 4)$.

  More examples can be obtained by following the construction of \cite[Proof of Theorem 1.5, necessity]{machado2020-approximate-lattices-nilpotent} and considering Lie algebras over number fields.  Their classification up to dimension 6 is given in \cite{degraaf2007}.
\end{example}
%
%
%
As we obtain classifiable \Cstar-algebras, it is intriguing to calculate their Elliott invariant and thus determine their isomorphism class.  We are able to achieve the following partial results in this direction. We remark that, descriptions of $\rK$-theory for groupoids associated with cut-and-project schemes in Euclidean groups have been obtained in \cite[Theorem 4.2]{forresthuntonkellendonk2002}. See also \cite[Chapters III, IV and V]{forresthuntonkellendonk2002} for concrete calculations.
\begin{proposition}
  \label{prop:elliott-invariant-model-set}
  Let $\Lambda$ be a regular model set in a conneceted, simply connected nilpotent Lie group arising from an irreducible lattice as in \cref{cor:irreducible-lattice-classifiable}.  Then
  \begin{itemize}
  \item $\rK_*(\Cstar(\Lambda)) \cong \rK^*(\Omega(\Lambda))$ as abelian groups, and
  \item $\Cstar(\Lambda)$ has a unique trace.
  \end{itemize}
\end{proposition}
\begin{proof}
  Denote by $\Lambda \subseteq G$ the inclusion in the assumptions.  By \cref{lem:stable-isomorphism}, the \Cstar-algebras $\Cstar(\Lambda)$ and $\Cstarred(\cG(\Lambda))$ are stably isomorphic and hence have isomorphic $\rK$-theory.  As connected simply connected nilpotent Lie groups are amenable and as such satisfy the Baum-Connes conjecture \cite{higsonkasparov01} and $G$ is a model for its classifying space for proper actions, it follows that there is a $\mathrm{KK}^G$-equivalence $\cont(\Omega(\Lambda)) \sim_{\mathrm{KK}^G} \conto(G \times \Omega(\Lambda))$, where $G$ acts diagonally on $G \times \Omega(\Lambda)$.  Considering the self-homeomorphism of $G \times \Omega(\Lambda)$ defined by $(g,P) \mapsto (g, gP)$, we find that $G \grpaction{} G \times \Omega(\Lambda)$ is conjugate to the action in its first factor.  By descent in $\mathrm{KK}$-theory, we infer that
  \begin{gather*}
    \cont(\Omega(\Lambda)) \rtimes G
    \sim_{\mathrm{KK}}
    (\conto(G) \rtimes G ) \otimes \cont(\Omega(\Lambda))
    \cong
    \cK(\Ltwo(G)) \otimes \cont(\Omega(\Lambda))
    \eqstop
\end{gather*}
This shows that
\begin{gather*}
  \rK_*(\Cstar(\Lambda))
  \cong
  \rK_*(\cont(\Omega(\Lambda)) \rtimes G)
  \cong
  \rK_*(\cont(\Omega(\Lambda)))
  \cong
  \rK^*(\Omega(\Lambda))
  \eqstop
\end{gather*}

Let us now show that $\Cstar(\Lambda)$ has a unique trace.  Since $\cG(\Lambda)$ is principal, by \cite[Theorem 1.1]{neshveyev2013-kms-states-groupoids} (see also \cite{renault87-produits-croises}), traces on $\Cstar(\Lambda)$ are precisely of the form $\int \rmd \mu \circ \rE$ for $\cG(\Lambda)$ invariant probability measures $\mu \in \cP(\Omega_0(\Lambda))$, where $\rE: \Cstar(\Lambda) \to \cont(\Omega_0(\Lambda))$ is the natural conditional expectation.  So it suffices to prove that $\Omega_0(\Lambda)$ caries a unique $\cG(\Lambda)$-invariant probability measures.  Let $\mu$ be such measure and consider the finite measure $\mu_G$ on $\Omega(\Lambda)$ associated with it by transverse measure theory through the correspondence $\cG(\Lambda) \sim G \ltimes \Omega(\Lambda)$. If $\nu$ denotes an admissible probability measure on $G$, and $\tilde \mu_G$ is a probability measure equivalent to $\mu_G$, then $\frac{1}{n} \sum_{i = 1}^n \nu^{*n} * \tilde \mu_G$ is a $\nu$-stationary probability measure that is equivalent to $\mu_G$.  Consider the continuous map $\beta \colon \Omega(\Lambda) \to G \times H / \Gamma$ and the subset $\Omega(\Lambda)^{\mathrm{ns}}$ of non-singular points introduced in \cite{bjorklundhartnickpgorzelski2018}.  Then \cite[Theorem 3.4]{bjorklundhartnickpgorzelski2018} applied to $\frac{1}{n} \sum_{i = 1}^n \nu^{*n} * \tilde \mu_G$ shows that $\Omega(\Lambda)^{\mathrm{ns}}$ is $\mu_G$-negligible.  Hence $\beta_* \mu_G$ is a well-defined $G$-invariant $\sigma$-finite measure on $G \times H / \Gamma$.  So the argument of \cite[Lemma 3.7]{bjorklundhartnickpgorzelski2018} shows that $\mu_G$ is a scalar multiple of the unique $G \times H$-invariant probability measure on $G \times H/\Gamma$.  This proves uniqueness of $\mu$.
\end{proof}



\begin{remark}
  \label{rem:transverse-measure-theory}
  In \cite[Proposition 4.1]{enstadraum2022} we described how transverse measure theory associates finite measures on $\Omega_0(\Lambda)$ to $G$-invariant probability measures on $\Omega(\Lambda)$.  The proof of uniqueness of the trace on $\Cstar(\Lambda)$ in \cref{prop:elliott-invariant-model-set} needs the reverse construction though, which is why we presented a proof that the measure $\mu_G$ in there is finite.
\end{remark}


\begin{remark}
  \label{rem:ordered-k-theory}  
  In view of \cref{prop:elliott-invariant-model-set} and the continuous map $\beta\colon \Omega(\Lambda) \to G \times H / \Gamma$ from \cite{bjorklundhartnickpgorzelski2018}, which is bijective on the set of non-singular points, it would seem natural to expect that $\rK_\bullet(\Cstar(\Lambda)) \cong \rK^\bullet(G \times H / \Gamma)$.  While this might hold as abstract groups, such isomorphism cannot respect the order of $\rK$-theory.  Indeed, $\Omega_0(\Lambda) = \cG(\Lambda)\nought$ is totally disconnected by \cref{prop:FLC-implies-tdlc} and carries an invariant probability measure so that there are infinite strictly descending chains in the positive cone of $\rK_\bullet(\Cstar(\Lambda))$.  As $G \times H / \Gamma$ is a closed manifold, its $\rK$-theory is finitely generated.
\end{remark}



\begin{problem}
  \label{prob:elliot-invariant}
  Calculate the Elliott invariant of $\Cstar(\Lambda)$ for a regular model set $\Lambda$ in a conneceted, simply connected nilpotent Lie group arising from an irreducible lattice as in \cref{cor:irreducible-lattice-classifiable}.
\end{problem}



%%% Local Variables:
%%% mode: latex
%%% TeX-master: "classifiability"
%%% End:
