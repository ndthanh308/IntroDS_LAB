\section{Slices}
\label{sec:slices}

Slices are used to understand in how far actions of locally compact groups are locally trivial.  In the relevant context of non-compact groups, the concept was introduced by Palais' for proper actions \cite{palais1961-slices}.  In this article we are concerned with free actions, for which we can show the existence of slices for suitable classes of groups in \cref{sec:slices-existence}.  The following definition of a slice is suitable for this setup, while for non-free actions a definition closer to Palais' work is needed.
\begin{definition}
  \label{def:slice}
  Let $G \grpaction{} X$ be an action.  Let $K \subseteq G$ be some compact identity neighbourhood. A \emph{$K$-slice} is a subset $S \subseteq X$ such that the map $K \times S \to X$ given by $(g,x) \mapsto gx$ is injective. The resulting image $KS$ in  $X$ is called a \emph{tube}, and its interior $(KS)^{\circ}$ is called an \emph{open tube}.
\end{definition}



\begin{remark}
  \label{rem:slice-homeomorphic}
  The injectivity of the map $K \times S \to X$ in \Cref{def:slice} implies that it is a homeomorphism onto its image since $K \times S$ is compact and $X$ is Hausdorff. Therefore the tube $KS$ is a compact set in $X$ homeomorphic to $K \times S$.
\end{remark}



\begin{remark}
  \label{rem:slice-disjoint-translates}
  We will frequently use the following facts.  Any closed subset of a $K$-slice is also a $K$-slice, and if $L \subseteq K$ is a compact identity neighbourhood, then every $K$-slice is also an $L$-slice.  Furthermore, $S$ is a $K$-slice if and only if $gS \cap S = \emptyset$ for all $g \in K^{-1}K \setminus \{ e \}$.
\end{remark}



\subsection{Properties of slices}
\label{sec:slices-properties}

In this section we collect some basic properties of slices that will be used in the remainder of the article.  We start by considering a suitable notion of the interior of a slice.
\begin{lemma}
  \label{lem:box-interior-independence}
  Let $K$ and $K'$ be compact identity neighborhoods in $G$ such that the set $S \subseteq X$ is both a $K$-slice and a $K'$-slice. Then
  \begin{gather*}
    S \cap (KS)^{\circ} = S \cap (K'S)^{\circ}
    \eqstop
  \end{gather*}
\end{lemma}
\begin{proof}
  Passing to the intersection $K \cap K'$, we may assume that $K \subseteq K'$ and prove that $S \cap (K'S)^{\circ} \subseteq S \cap (KS)^{\circ}$.  Let $x \in S \cap (K'S)^{\circ}$.  Since $(K'S)^{\circ} \subset K'S$ is open and products of open subsets form a basis of the topology of $K' \times S$, there are open subsets $U \subseteq K'$ and $V \subseteq S$ such that $x \in UV \subseteq (K'S)^\circ$.  Since $x \in S$, we find that $e \in U$ and may hence assume without loss of generality that $U \subseteq K$.  Then $x \in UV \subseteq KS$ on the one hand and on the other hand we obtain a inclusions where each term is open in the next one $UV \subseteq (K'S)^\circ \subseteq X$.  So $UV \subseteq X$ is open, which shows that $x \in (KS)^\circ$. 
\end{proof}




\begin{definition}\label{def:box-interior}
  Given a $K$-slice $S$, we define its \emph{tube interior} and its \emph{tube boundary} to be the sets
  \begin{align*}
    \boxint{S} &= S \cap (KS)^{\circ}, \\
    \boxboundary{S} &= S \setminus \boxint{S} ,
  \end{align*}
  respectively.
\end{definition}
By \cref{lem:box-interior-independence} we infer that $\boxint{S}$ and $\boxboundary{S}$ are independent of the compact identity neighborhood $K$.



We will next obtain some basic properties of the tube interior.  For this, the following result is instrumental, which allows to enlarge a tube in the direction of the group.
\begin{lemma}
  \label{lem:extend_slice}
  Assume that $G$ is second-countable. Let $S$ be a $K$-slice for a compact identity neighborhood $K$ in $G$. Then there exists a compact identity neighborhood $L$ with $K \subseteq \mathring{L}$ such that $S$ is an $L$-slice.
\end{lemma}
\begin{proof}
  Since $G$ is second-countable we can multiply $K$ with elements from a descending sequence of compact neighbourhoods of the identity to find a descending sequence $(L_n)_{n \in \NN}$ of compact sets such that $\mathring{L_n} \supseteq K$ for all $n \in \NN$ and $\bigcap_{n \in \NN} L_n = K$.  Suppose for a contradiction that $S$ is not an $L_n$-slice for any $n \in \NN$. By \Cref{rem:slice-disjoint-translates} we then find $g_n \in L_n^{-1}L_n \setminus K^{-1}K$ and $x_n,x_n' \in S$ such that $g_nx_n = x_n'$ for all $n \in \NN$. Passing to convergent subsequences, we may assume that $x_n \to x$, $x_n' \to x'$ with $x,x' \in S$ and
  \begin{gather*}
    g_n \to g \in \Big( \bigcap_{n \in \NN} L_n^{-1}L_n \Big) \setminus (K^{-1}K)^{\circ} = \partial(K^{-1}K) .
  \end{gather*}
  In particular, $g \neq e$. Continuity of the action then implies $gx' = x$, so $g S \cap S \neq \emptyset$. By \Cref{rem:slice-disjoint-translates} this is a contradiction to the fact that $S$ is a $K$-slice.
\end{proof}
%
%
%
\begin{lemma}
  \label{lem:box_interior}
  Let $K$ be a compact identity neighborhood and let $S$ be a $K$-slice. Then the following identities hold.
  \begin{align}
    (KS)^{\circ} & = \mathring{K} (\boxint{S}) \eqcomma
                   \label{eq:box_interior} \\
    \partial_X(KS) & = (\partial_G K)S \cup K(\boxboundary{S}) \eqstop
                     \label{eq:box_boundary}
  \end{align}
  Furthermore, if $B$ is regular open in $S$ and contained in $\boxint{S}$, then
  \begin{align}
    \boxint{\overline{B}} & = B \eqcomma
                            \label{eq:boxint_open} \\
    \boxboundary{\overline{B}} & = \partial_S B \eqcomma
                                 \label{eq:boxboundary_open} \\
    \partial(K\overline{B}) & = (K\overline{B}) \setminus \mathring{K}B = (\partial_G K)\overline{B} \cup K(\partial_S B) \eqstop
                              \label{eq:several-boundary-expressions}
  \end{align}
\end{lemma}
\begin{proof}
  We start by proving \eqref{eq:box_interior} and first show that $\mathring{K}\boxint{S} \subseteq (KS)^{\circ}$.  Let $x \in \mathring{K}\boxint{S}$, say $x=gy$ where $g \in \mathring{K}$ and $y \in \boxint{S}$.  Set $L = K \cap g^{-1}K$ and consider the open set $U = g(LS)^{\circ}$ in $X$.  Then $U \subseteq gLS \subseteq gg^{-1}KS = KS$, so that $U \subseteq (KS)^{\circ}$ follows.  Furthermore, since $L \subseteq K$, we know that $S$ is also an $L$-slice and \cref{lem:box_interior} shows that $y \in \boxint{S} = S \cap (LS)^{\circ}$. Thus $x = gy \in g(LS)^{\circ} = U \subseteq (KS)^{\circ}$.

  Now we show that $(KS)^{\circ} \subseteq \mathring{K}\boxint{S}$.  Let $x \in (KS)^{\circ}$, say $x = gy$ with $g \in K$ and $y \in S$.  Then $g^{-1}K$ is a compact identity neighbourhood and $S$ is a $g^{-1}K$-slice, so by \cref{lem:box_interior} we obtain
  \begin{gather*}
    y
    =
    g^{-1}x \in S \cap g^{-1}(KS)^{\circ}
    =
    S \cap (g^{-1}KS)^{\circ}
    =
    \boxint{S}
    \eqstop
  \end{gather*}
  It remains to argue that $g \in \mathring{K}$.  For this, we apply \cref{lem:extend_slice} to find a compact identity neighborhood $L$ in $G$ with $K \subseteq \mathring{L}$ such that $S$ is an $L$-slice.  By continuity of the group action, the set
  \begin{gather*}
    W = \mathring{L} \cap \{ h \in G \mid hy \in (KS)^{\circ} \}
  \end{gather*}
  is open in $G$ and it is clear that $g \in W$.  We show that $W \subseteq K$.  Indeed, if $h \in W$ then $hy \in (KS)^{\circ}$ so we can write $hy = h'y'$ with $h' \in K$ and $y' \in S$.  Since $h \in L$ and $S$ is an $L$-slice, this forces $h = h'$ and $y = y'$. In particular, we infer that $h \in K$.
  
  We next prove \eqref{eq:box_boundary}.  First observe that
  \begin{gather*}
    \mathring{K}\boxint{S} = \mathring{K}S \cap K\boxint{S}
    \eqstop
  \end{gather*}
  Indeed, the inclusion form left to right is obvious. Conversely, if $x = gy \in \mathring{K}S \cap K\boxint{S}$ for uniquely determined elements $g \in K$ and $y \in S$, it follows from $x \in \mathring{K}S$ that $g \in \mathring{K}$ and from $x \in K\boxint{S}$ that $y \in \boxint{S}$.  Hence $x \in \mathring{K}\boxint{S}$.  

  Applying the equality above and using \eqref{eq:box_interior}, we get
  \begin{align*}
    (\partial_G K)S \cup K(\boxboundary{S})
    & =
      (K\setminus \mathring{K})S \cup K(S\setminus \boxint{S}) \\
    & =
      (KS\setminus \mathring{K}S) \cup (KS\setminus  K\boxint{S})) \\
    & =
      KS\setminus (\mathring{K}S \cap K\boxint{S})) \\
    & =
      KS \setminus \mathring{K}\boxint{S} \\
    & =
      KS \setminus (KS)^{\circ} \\
    & =
      \partial_X (KS)
      \eqstop
  \end{align*}

  Let us next show \eqref{eq:boxint_open}. First we show that $B \subseteq \boxint{\ol{B}}$. Using that $B \subseteq \boxint{S}$ and \eqref{eq:box_interior} we obtain $\mathring{K}B \subseteq \mathring{K}\boxint{S} = (KS)^{\circ}$.  Since $\mathring{K}B$ is open in $KS$ and hence in $(KS)^{\circ}$, it is also open in $X$.  But then $B \subseteq \ol{B} \cap (K\ol{B})^{\circ} = \boxint{\ol{B}}$.

  Now we show that $\boxint{\ol{B}}\subseteq B$. Let $x \in \boxint{\ol{B}}$. Since $(K\ol{B})^{\circ}$ is open in $KS$ and $x \in (K\ol{B})^{\circ}$, we can by definition of tubes find open sets $W \subseteq K$ and $V \subseteq S$ such that $x \in WV \subseteq (K\ol{B})^{\circ}$. The inclusion $WV \subseteq K\ol{B}$ then forces $V \subseteq \ol{B}$, and since $x \in \ol{B} \subseteq S$ we get $x \in V$. Since $x \in V \subseteq \ol{B}$ and $V$ is open in $S$, this means that $x$ lies in the interior of $\ol{B}$ inside $S$.  By assumption the latter equals $B$, so $x \in B$.

  To show \eqref{eq:boxboundary_open}, we use \eqref{eq:boxint_open} and obtain
  \begin{gather*}
    \boxboundary{\ol{B}}
    =
    \ol{B}\setminus\boxint{\ol{B}}
    =
    \ol{B}\setminus B
    =
    \partial_S{B}
    \eqstop
  \end{gather*}

  Let us now show \eqref{eq:several-boundary-expressions}.  Since $\ol{B}$ is a compact subset of $S$, it is a $K$-slice. Hence \eqref{eq:box_boundary} together with \eqref{eq:boxboundary_open} gives that
  \begin{gather*}
    \partial(K\ol{B})
    =
    (\partial_G K)\ol{B} \cup K(\boxboundary{B})
    =
    (\partial_G K)\ol{B} \cup K(\partial_S B)
    \eqstop
  \end{gather*}

  Moreover, the definition of boundary together with \eqref{eq:box_interior} and \eqref{eq:boxboundary_open} gives that
  \begin{gather*}
    \partial(K\ol{B})
    =
    K\ol{B} \setminus (K\ol{B})^{\circ}
    =
    K\ol{B} \setminus K\boxint{\ol{B}}
    =
    K\ol{B} \setminus KB
    \eqstop
  \end{gather*}
  This finishes the proof.
\end{proof}
%
%
%
The next lemma will be frequently used to extract information about the intersection of tubes and slices.
\begin{lemma}
  \label{lem:slice_restriction_injective}
  If $S$ and $S'$ are $K$-slices, then the restriction of the second factor projection $KS \cong K \times S \to S$ to the set $KS \cap S'$ is a homeomorphism onto its image.
\end{lemma}
\begin{proof}
  Since $KS \cap S'$ is compact and $S$ is Hausdorff, it suffices to show that the map is injective. Let $g_1,g_2 \in K$ and $x_1,x_2 \in S$ such that $g_1x_1 , g_2x_2 \in S'$. Suppose that the images of $g_1x_1$ and $g_2x_2$ under the projection $KS \to S$ are equal, that is, $x_1 = x_2$.  We then have that $g_1^{-1}(g_1x_1) = g_2^{-1}(g_2x_2) \in KS'$ and since $S'$ is a $K$-slice, it follows that $g_1^{-1} = g_2^{-1}$. This implies injectivity.
\end{proof}



\subsection{Existence of slices}
\label{sec:slices-existence}

The existence of $G$-slices for actions of $G$ is a classical topic closely related to the early development of the theory of principal bundles and the question about their local triviality.  The ultimate existence result in this direction was obtained by Palais in \cite{palais1961-slices}.  The notion of $K$-slices and thus tubes as used in the present work goes back to the work of Bartels--L{\"u}ck--Reich in which an equivariant version for $\RR$-actions equipped with an additional commuting action of a discrete group was obtained \cite[Lemma 2.11]{bartelsluckreich2008-covers} based on ideas from Palais' work.  For our needs, this approach needs further refinement in order to treat actions of Lie groups and simplifies at the same time since we target free actions and there is no additional action of a discrete group present. The goal of the this subsection will be to show that actions of matrix Lie groups and of connected Lie groups of polynomial growth admit slices in the following sense.
\begin{definition}
  \label{def:admits_slices}
  We say that an action $G \curvearrowright X$ \emph{admits slices} if for every compact identity neighbourhood $K \subseteq G$ and every $x \in X$ there exists a $K$-slice $S \subseteq X$ such that $x \in \boxint{S}$.
\end{definition}



\begin{remark}
Notice that if an action $G \curvearrowright X$ admits slices then it must be free. Indeed, if $gx =x$ for $g \in G$ and $x \in X$, pick some compact identity neighborhood $K \subseteq G$ that contains $g$. If $S$ is a $K$-slice that contains $x$, then the equation $gx = x = ex$ forces $g=e$ by the definition of a $K$-slice.
\end{remark}
%
%
%
The next lemmas provide us with a sufficiently equivariant map from a free $G$-space into a finite dimensional representation, in analogy with Palais' result \cite[Theorem 1.2.7]{palais1961-slices} in the context of proper $G$-spaces.
\begin{lemma}
  \label{lem:large-identity-neighbourhoods}
  Let $K \subseteq G$ be a compact subset.  Then there is a symmetric relatively compact identity neighbourhood $U \subseteq G$ such that $\bigcap_{k \in K} U k$ is an identity neighbourhood.
\end{lemma}
\begin{proof}
  Replacing $K$ by $K \cup K^{-1}$, we may assume that $K$ is symmetric.  Take any symmetric, relatively compact identity neighbourhood $V \subseteq G$ and define $U = VK \cup KV$.  Then $U$ is symmetric and since $K$ is symmetric is follows that
  \begin{gather*}
    \bigcap_{k \in K} U k
    \supseteq
    \bigcap_{k \in K} \left ( \bigcup_{k' \in K} V k' \right) k
    \supseteq V
    \eqstop
  \end{gather*}
  Since $V$ was relatively compact, also $U$ is relatively compact.
\end{proof}



\begin{lemma}
  \label{lem:map-to-vectorspace}
  Let $G \grpaction{} X$ be a free action, $V$ a finite dimensional $G$-representation and $v \in V$ a vector with trivial stabiliser in $G$.  For every $x \in X$, every neighbourhood $A$ of $x$ and every relatively compact identity neighbourhood $U \subseteq G$ there is a function $f \in \contc(A, \RR)$ such that $\int_U f(g^{-1}x) gv \, \rmd g= v$.
\end{lemma}
\begin{proof}
  Consider the linear map $T \colon \contc(A, \RR) \to V$ given by $Tf = \int_U f(g^{-1}x) g v \, \rmd g$.  Since $V$ is finite dimensional and $\im T$ is a vector subspace of $V$, it suffices to show that for all convex neighbourhoods $v \in C$ we have $C \cap \im T \neq \emptyset$.

  Fix a convex neighbourhood $C$ of $v$ as above and let $U_0 \subseteq G$ be an identity neighbourhood satisfying $U_0 v \subseteq C$.  Since $\ol{U}x \subseteq X$ is closed, there is a neighbourhood $B \subseteq A$ of $x$ satisfying $B \cap \ol{U}x \subseteq U_0x$.  By freeness of $G \grpaction{} X$, this implies that for every $g \in U$, the statement $g x \in B$ implies $g \in U_0$.  Let $f \in \contc(A, \RR)$ satisfy $0 \leq f \leq 1$, $f(x) \neq 0$, $\int_U f(g^{-1}x) \, \rmd g = 1$ and $\supp f \subseteq B$. Then by choice of $B$, the map $g \mapsto \mathbb{1}_U(g) f( g^{-1} x)$ is a probability density on $G$ supported in $U_0$.  Since $U_0v \subseteq C$, this implies that $\int_U f(g^{-1}x) gv \, \rmd g \in C$.
\end{proof}
%
%
%
We are now prepared to constructed sufficiently equivariant maps into finite dimensional representations.
\begin{lemma}
  \label{lem:map-to-vectorspace-equivariant}
  Let $G \grpaction{} X$ be a free action  and assume that $X$ is second-countable. Let $V$ be a finite dimensional $G$-representation and $v \in V$ a vector with trivial stabiliser in $G$.  For every relatively compact identity neighbourhood $K \subseteq G$ and for every $x_0 \in X$ there is a neighbourhood $A$ of $x_0$ and a continuous map $F\colon X \to V$ such that $F(x_0) = v$ and $F(k x) = k \vphi(x)$ for all $k \in K$ and all $x \in A$.
\end{lemma}
\begin{proof}
  By \cref{lem:large-identity-neighbourhoods} there is a symmetric, relatively compact identity neighbourhood $U$ in $G$ such that $U_0 = \bigcap_{k \in K} Uk$ is an identity neighbourhood.  Since $G$ acts freely, we have $x_0 \notin (K \ol{U} \setminus U_0)x_0$.  Hence,
  \begin{gather*}
    \bigcap_{\substack{A \text{ compact}\\ \text{neighbourhood of }x_0}} A \cap (K \ol{U} \setminus U_0)A = \emptyset
    \eqstop
  \end{gather*}
  By compactness, there is a neighbourhood $A$ of $x_0$ such that for any $g \in K \ol{U} \setminus U_0$ and every $x \in A$ we have $gx \notin A$.  Fix such $A$.

  By \cref{lem:map-to-vectorspace}, there is $f \in \contc(A, \RR)$ supported in $A$ such that $\int_U f(g^{-1}x_0) gv \, \rmd g = v$.  Put
  \begin{gather*}
    F(x) = \int_U f(g^{-1}x) gv \, \rmd g
    \eqstop
  \end{gather*}
  For $k \in G$, $x \in X$, left-invariance of the Haar measures shows that
  \begin{align*}
    F(kx)
    & = \int_U f(g^{-1}k^{-1}x) gv \, \rmd g \\
    & = \int_{k^{-1}U} f(g^{-1}x) kgv \, \rmd g \\
    & = k F(x) + \int_{k^{-1}U \setminus U} f(g^{-1}x) kgv \, \rmd g - \int_{U \setminus k^{-1}U} f(g^{-1}x) kgv \, \rmd g
    \eqstop
  \end{align*}
  Let us simplify the last expression.  Recall that $\supp f \subseteq A$ and that the width of $A$ is at most $U_0 = \bigcap_{k \in K} Uk$.  We find that for $x \in A$ and $g \in k^{-1}U \setminus U$, we have in particular $g^{-1} \in KU \setminus U_0$ and hence $g^{-1}x \notin A$.  Similarly, for $g \in U \setminus k^{-1}U$ we have $g^{-1} \notin Uk$, so that $g^{-1}x \notin A$ follows as above.  We infer that $F(kx) = kF(x)$ for all $k \in K$ and all $x \in A$.  
\end{proof}
%
%
%
We are now able to prove the existence of slices in the sense of \Cref{def:admits_slices} for free actions of matrix Lie groups, that is, subgroups of $\GL(n,\RR)$ for some $n \in \bN$.
\begin{theorem}
  \label{thm:K-slices-exist-matrix-groups}
  Let $G$ be a matrix Lie group and $G \grpaction{} X$ be a free action and assume that $X$ is second-countable. Then $G \curvearrowright X$ admits slices.
\end{theorem}
\begin{proof}
  Fix an embedding with closed image $G \subseteq \GL(n, \RR)$ and consider the vector $v = 1 \in \rM_n(\RR) = V$.  By \cref{lem:map-to-vectorspace-equivariant}, there is a neighbourhood $A \subseteq X$ of $x$ and a continuous map $F\colon X \to V$ such that $F(x) = v$ and $F(ky) = kF(y)$ for all $k \in K$ and $y \in A$.  Since $G \subseteq \GL(n, \RR)$ is closed, its action on $V$ is proper, so that \cite[Theorem 2.3.2]{palais1961-slices} shows the existence of a $G$-slice $S_0$ at $v$.  Put $S = F^{-1}(S_0) \cap A$.  Then $x \in S$, since $F(x) = v$.  We have to show that $S$ is a $K$-slice and then that that $x \in \boxint{S}$.
  
  Let $(k_1,x_1), (k_2,x_2) \in K \times S$.  If $k_1x_1 = k_2x_2$, then
  \begin{gather*}
    k_1F(x_1) = F(k_1x_1) = F(k_2x_2) = k_2F(x_2)
    \eqcomma
  \end{gather*}
  implying that $k_1 = k_2$, since $S_0$ is a $G$-slice.  So $x_1 = x_2$ follows, proving injectivity of the map $K \times S \to KS$.  Since it is also continuous and $KS$ is compact, it is a homeomorphism, and we can conclude that $S$ is a $K$-slice.

  Now fix a neighbourhood $B \subseteq A$ of $x$ and a symmetric identity neighbourhood $L$ such that $L^2 \subseteq K$ and $L \cdot B \subseteq A$.  We claim that the neighbourhood $F^{-1}(L^\circ S_0) \cap B$ of $x$ is contained in $KS$, which will prove that $x \in \boxint{S}$.  Take $y \in F^{-1}(L^\circ S_0) \cap B$.  There is $k \in L$ satisfying $kF(y) \in S_0$.  Since $k \in K$ and $y \in A$, we find that $F(ky) = k F(y) \in S_0$, that is $ky \in F^{-1}(S_0)$.  As we also have $ky \in LB \subseteq A$, we infer that $ky \in S$.   Using the fact that $L$ is symmetric, we now conclude that $y \in L^{-1}S \subseteq KS$, which finishes the proof.
\end{proof}



\begin{corollary}
  \label{cor:K-slices-exist-polynomial-growth}
  Let $G \curvearrowright X$ be a free action of a connected Lie group of polynomial growth and assume that $X$ is second-countable. Then $G \curvearrowright X$ admits slices.
\end{corollary}
\begin{proof}
  Let $K$ be an identity neighborhood in $G$ and let $x \in X$. By \cite[Theorem 2]{losert2001} $G$ contains a maximal compact subgroup $H$. The quotient group $G/H$ is then a Lie group of polynomial growth containing no nontrivial compact normal subgroups, hence by \cite[Corollary 3.6]{losert2020} $G/H$ is a matrix Lie group. Denote by $\pi \colon G \to G/H$ and $p \colon X \to X/H$ the quotient maps. Since $G/H$ acts freely on $X/H$, we can apply \cref{thm:K-slices-exist-matrix-groups} to get a $\pi(K)$-slice $S' \subseteq G/H$ such that $[x] \in \boxint{S'}$.  Set $T = p^{-1}(S') \subseteq X$.  Then $T$ is $H$-invariant, so we can consider the action $H \curvearrowright T$.  Since $H$ is compact there exists by \cite[Theorem 2.1]{mostow57} an $H$-slice $S \subseteq T$ such that $x \in \boxint{S}$, where the interior is taken in the $H$-space $T$.

  We claim that $S$ is a $K$-slice. To see this, let $g,g' \in K$ and $x,x' \in S$ satisfy $gx=g'x'$. Then $(gH)[x] = (g'H)[x']$, so since $gH,g'H \in \pi(K)$ and $[x],[x'] \in \pi(S) \subseteq \pi(\pi^{-1}(S')) = S'$, we get that $gH = g'H$ and $[x]=[x']$. Let $h_1,h_2 \in H$ be such that $g' = gh_1$ and $x = h_2x'$. Then $gx = gh_1h_2x$, so by freeness of the action $h_1h_2 = e$. But $x = h_2x'$ also implies that $h_2=e$ since $h_2 \in H$ and $x,x' \in S$, so we arrive also at $h_1 = e$. We conclude that $g=g'$. It remains to argue that $x \in \boxint{S}$. Since $S$ is an $H$-slice at $x$ in $T$, there is a subset $U \subseteq HS$ which is open in $T$ and contains $x$. Let $W = U \cap \boxint{T}$.  Then $\mathring{K}W \subset \mathring{K}\boxint{T}$ is relatively open and the latter set is open in $X$. So $\mathring{K}W$ is an open subset of $X$ which contains $x$.
\end{proof}



%%% Local Variables:
%%% mode: latex
%%% TeX-master: "classifiability"
%%% End:
