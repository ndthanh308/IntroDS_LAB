\section{Introduction}
\label{sec:introduction}

An important step in Elliott's classification programme for C*-algebras has been the insight that besides long anticipated assumptions such as simplicity and nuclearity, it is necessary to assume some additional regularity of the C*-algebras under consideration. One such notion of regularity is that of finite nuclear dimension, which was introduced by Winter and Zacharias in \cite{winterzacharias2010} as a noncommutative generalisation of finite Lebesgue covering dimension. A C*-algebra is called classifiable if it is simple, separable, non-elementary, unital, satisfies the UCT and has finite nuclear dimension.  Work of many hands has shown that such C*-algebras are classified up to isomorphism by $\rK$-theory and tracial data; see \cite{winter18-icm} and the references therein.

After Elliott's classification programme saw its breakthrough, the investigation of natural examples arising for instance from topological dynamics moved into the focus of substantial parts of the community.  One bridge between these topics is created by the crossed product C*-algebra $\conto(X) \rtimes G$ which can be associated with any action $G \grpaction{} X$ of a locally compact group on a locally compact Hausdorff space.  This line of research can be roughly divided into two directions: One investigating actions of discrete groups and the other one actions of connected groups.  We are not concerned with discrete group actions in this article, but point the reader to the work in \cite{hirshbergwinterzacharias2015, szabowuzacharias2019, naryshkin2022, gardellageffenkranznaryshkin2023}, where classifiability of crossed products with various discrete groups has been established.  In particular, \cite{szabowuzacharias2019} shows finite nuclear dimension of crossed product \mbox{C*-algebras} associated with free actions of finitely generated groups of polynomial growth acting on compact spaces of finite covering dimension.

For actions of connected locally compact groups, classifiability results and nuclear dimension estimates so far have been restricted to $\RR$-actions.  In \cite{hirshbergszabowinterwu2017}, Hirshberg, Szabo, Winter and Wu developed a notion of Rokhlin dimension for $\RR$-actions which was used to show that the crossed product of a free action of $\RR$ on a finite-dimensional space has finite nuclear dimension.  We note that Hirshberg and Wu were later able to remove the assumption that the action is free in \cite{hirshbergwu2021}.

The main theorem of the present paper extends the main result of \cite{hirshbergszabowinterwu2017} to free actions of connected Lie groups of polynomial growth.  By Breuillard's work \cite{breuillard14} the growth rate of any compactly generated group of polynomial growth $G$ behaves asymptotically like $n^{\rmd(G)}$ for some uniquely determined integer $\rmd(G)$.  We prove the following explicit nuclear dimension bound in terms of the dimension $\dim X$ and $\rmd(G)$.
\begin{theorem}
  \label{thm:intro-nuclear-dimension-estimate}
  Let $G \grpaction{} X$ be a free action of a connected Lie group of polynomial growth on a locally compact space $X$. Then
  \begin{gather*}
    \dimnuc(\conto(X) \rtimes G)  \leq 11^{\rmd(G)} \cdot (\dim X + 1)^2 - 1
    \eqstop
  \end{gather*}
\end{theorem}

An important motivation for the present work is the application to groupoid C*-algebras associated with point sets in groups of polynomial growth. In Euclidean spaces, dynamical properties of point sets and tilings are studied in the field of aperiodic order \cite{baakegrimm13} which has been motivated by the discovery of quasicrystals in the 1980s. Bellissard and Kellendonk \cite{bellissard86,bellissard92,kellendonk95} constructed groupoids from such point sets and tilings, which have been studied from the point of view of operator algebras, noncommutative geometry and mathematical physics. Of particular interest to us are model sets (or cut-and-project sets) which were introduced by Meyer \cite{meyer1972}. More recently model sets and other point sets have been introduced and studied in the setting of non-abelian locally compact groups in \cite{bjorklundhartnickpgorzelski2018, bjorklundhartnickpogorzelski2021, bjorklundhartnickpgorzelski2022,hrushovski2012-stable-group-theory, bjorklundhartnick2018, machado2023-higher-rank}. In light of this development, the first and third named author recently extended the definition of the previously studied transversal groupoid of a point set in a Euclidean space to the locally compact group setting in \cite{enstadraum2022}. In particular, to a Delone set $\Lambda$ in a locally compact group $G$ one can associate an {\'e}tale groupoid $\cG(\Lambda)$ which arises as the restriction of the transformation groupoid of the so-called hull dynamical system $G \curvearrowright \Omega(\Lambda)$ to a canonical transversal $\Omega_0(\Lambda)$.  When $\Lambda$ is a discrete subgroup $\Omega_0(\Lambda)$ is a one-point space and the groupoid $\cG(\Lambda)$ agrees with $\Lambda$ as an abstract group.  Hence the groupoid $\Cstar$-algebra $\Cstar(\Lambda) := \Cstar(\cG(\Lambda))$ generalises the group $\Cstar$-algebra of a discrete group.

As $\Cstar(\Lambda)$ is Morita equivalent to a crossed product C*-algebra, the nuclear dimension estimates obtained in \Cref{thm:intro-nuclear-dimension-estimate} can be used to prove that it is classifiable. For an explanation of the terminology we refer the reader to \Cref{sec:classifiable-point-sets}.
\begin{theorem}
  \label{thm:intro-classifiable-point-sets}
  Let $\Lambda$ be a repetitive, aperiodic, FLC Delone set in a connected Lie group of polynomial growth. Then its C*-algebra $\Cstar(\Lambda)$ is classifiable by the Elliott invariant.
\end{theorem}
This theorem generalises the work of Ito-Whittaker-Zacharias on tiling C*-algebras associated with repetitive, aperiodic, FLC tilings \cite{itowhittakerzacharias2019-arxiv-v2}, whose approach established almost finiteness of the associated {\'e}tale groupoids and thus Jiang-Su stability of the tiling C*-algebra.  In contrast, our approach establishes a nuclear dimension bound for the C*-algebra $\Cstar(\Lambda)$, only depending on the Lie group under consideration.



The hypotheses of \Cref{thm:intro-classifiable-point-sets} can be verified in particular for model sets arising from irreducible lattices in products of connected nilpotent Lie groups. We refer the reader to Section~\ref{sec:classifiable-point-sets} for natural examples arising from nilpotent Lie algebras over algebraic number fields.
\begin{corollary}
  \label{cor:intro:irreducible-lattice-classifiable}
  Let $G$ be a connected, simply connected nilpotent Lie group and let $\Lambda \subseteq G$ be a regular model set arising from a cut-and-project scheme $(G, H, \Gamma)$ where $H$ is another nilpotent Lie group and $\Gamma$ is an irreducible lattice.  Then $\Cstar(\Lambda)$ is classifiable.
\end{corollary}



Let us now describe the techniques we use to prove \cref{thm:intro-nuclear-dimension-estimate}.  The results of \cite{hirshbergszabowinterwu2017,hirshbergwu2021} are based on the use of slices. For the purposes of the present paper, a slice for a group action $G \grpaction{} X$ is a compact subset $S$ of $X$ which has the property that for some compact identity neighborhood $K \subseteq G$, the group action $(g,x) \mapsto gx$ is a homeomorphism onto its image when restricted to $K \times S$. The image of this map is called a tube (or a box). This is a local version of slices introduced by Bartels, L{\"u}ck and Reich in \cite{bartelsluckreich2008-covers} as a modification of the global slices introduced in the context of principal bundles, for example by Palais \cite{palais1961-slices}.  For $\RR$-actions on metric spaces, the existence of so-called long, thin covers of tubes was established by Kasprowski and R{\"u}ping in \cite{kasprowskiruping17}.  In \cite{hirshbergwu2021,hirshbergszabowinterwu2017}, these covers are used to construct partitions of unity consisting of Lipschitz functions which are instrumental for the nuclear dimension estimates of the corresponding crossed products.

Our proof follows the strategy to locally trivialise actions by means of tubes as already done in \cite{hirshbergszabowinterwu2017,hirshbergwu2021}. It is however not possible anymore to directly refer to \cite{kasprowskiruping17,bartelsluckreich2008-covers}, where only $\RR$-actions are considered.  Instead we have to develop suitable extensions to connected groups of polynomial growth. Further, the notion of Lipschitz functions is not adapted to non-abelian groups, so that a suitable replacement has to be developed.  In particular we establish the following technical innovations. 
\begin{itemize}
\item We prove in \cref{thm:K-slices-exist-matrix-groups} that every free action of a matrix Lie group $G \grpaction{} X$ admits slices, that is, every point $x \in X$ belongs to the interior of some tube.  This result is extended to connected Lie groups of polynomial growth -- which are not necessarily matrix Lie groups -- in \cref{cor:K-slices-exist-polynomial-growth}.  This step extends the case of $\RR$-actions considered in \cite[Lemma 2.11]{bartelsluckreich2008-covers}.

\item We show in \cref{thm:covering} that if a free action $G \grpaction{} X$ admits slices and $G$ has polynomial growth, then $X$ admits so-called long covers.  That is, for every compact set $K \subseteq G$ there is a cover $\cU$ of $X$ of finite multiplicity (bounded in terms of $\rmd(G)$ and $\dim X$) by interiors of open tubes such that for every $x \in X$, $Kx \subseteq U$ for some $U \in \mathcal{U}$. This implies in particular a bound on the \emph{tube dimension} of $G \grpaction{} X$, a notion we extend from the case of $\RR$-actions in \cite{hirshbergszabowinterwu2017}.
  
\item For $G$ amenable, we prove that a bound on the tube dimension of $G \grpaction{} X$ is equivalent to a number of other statements, cf.\ \Cref{prop:characterisation-tube-dimension}, in particular the existence of partitions of unity consisting of so-called F{\o}lner functions.  This is a modification of the techniques used in \cite{hirshbergszabowinterwu2017,hirshbergwu2021} where Lipschitz partitions of unity are used.
\end{itemize}
Finally, we establish an explicit bound of the nuclear dimension of $\conto(X) \rtimes G$ in terms of the tube dimension of $G \grpaction{} X$ and the dimension of $X$, simplifying some arguments of \cite{hirshbergwu2021} along the way.


%%% Local Variables:
%%% mode: latex
%%% TeX-master: "classifiability"
%%% End:
