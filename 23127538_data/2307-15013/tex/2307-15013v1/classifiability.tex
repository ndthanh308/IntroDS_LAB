\documentclass[a4paper,leqno,12pt]{article}
% add 'draft' in order to mark overfull boxes

%%%%%%%%%
%% Packages
%%%%%%%%%

% Font encoding
\usepackage[utf8]{inputenc}
\usepackage[T1]{fontenc}
\usepackage[english]{babel}


%% Chose one of the following fonts
% Font Gyre Heros
%\usepackage{tgheros}

% Font Crimson
\usepackage{crimson}

% Font Computer Modern Bright
% \usepackage{cmbright}

% Font Carlito
% \usepackage[sfdefault,lf]{carlito}
% %% The 'lf' option for lining figures
% %% The 'sfdefault' option to make the base font sans serif
% \renewcommand*\oldstylenums[1]{\carlitoOsF #1}

% page layout
\usepackage{geometry}

% modify headers and footers
\usepackage{fancyhdr}

% references
\usepackage{hyperref}
\hypersetup{
  colorlinks   = true, %Colours links instead of ugly boxes
  urlcolor     = blue, %Colour for external hyperlinks
  linkcolor    = blue, %Colour of internal links
  citecolor   = black %Colour of citations
}

% More options for arrays and multi arrays
%\usepackage{array}
%\usepackage{multicol}

% More options for enumerate
%\usepackage{enumerate}
\usepackage[inline]{enumitem}

% Including Graphics
\usepackage{graphicx}

% Tables
%\usepackage{float}
% Extended table layout
%\usepackage{booktable}

% Text color
\usepackage{color}




%% Maths packages

% AMS Math Packages.
% The option intlimits sets limits of integrals in displayed equations above and below the integral symbol. A guide can be found at http://www.ams.org/tex/amslatex.html
\usepackage[intlimits]{amsmath}
\usepackage{amssymb, amsfonts}
% amsthm is incompatible with the ntheorem package
%\usepackage{amsthm}

% ntheorem is incompatible with the amsthm package

\usepackage[thmmarks, amsmath, amsthm]{ntheorem}

% More math symbols
\usepackage{MnSymbol}
%\usepackage{textcomp}
\usepackage{mathbbol}

% Extended blackboard characters (\mathbb)
\usepackage{bbm}

% Script characters (\mathscr)
\usepackage{mathrsfs}

% Physical units
%\usepackage{units}

% Other fractions (slash)
%\usepackage{nicefrac}

% Package for \xymatrix
\usepackage[all]{xy}

% Tikz
% \usepackage{tikz}
% \usetikzlibrary{}
% \usepackage{tikz-cd}




%%%%%%
%% Layout
%%%%%%

\geometry{left=2.5cm,right=2.5cm,top=2cm,bottom=2cm,includeheadfoot}

%% Page layout without the geometry package
% \setlength{\oddsidemargin}{0pt}
% \setlength{\topmargin}{5pt}
% \setlength{\textheight}{620pt}
% \setlength{\textwidth}{470pt}
% \setlength{\headsep}{10pt}
% \setlength{\parindent}{0pt}
% \setlength{\parskip}{1ex plus 0.5ex minus 0.2ex}

% Pagestyle fancy, plain or empty
% For fancy: \lhead, \chead, \rhead and \lfoot, \cfoot, \rfoot
\pagestyle{plain}

% Avoid line breaks
\sloppy

% Remove automatic indentation
% \setlength{\parindent}{0pt}

% Space after full stop
\nonfrenchspacing



%%%%%%
%% Theorem like environments
%%%%%%

% Number equations and theorems within sections
\numberwithin{equation}{section}
\newtheorem{theoremcounter}{theoremcounter}[section]

% Places the numbers behind `Theorem', `Lemma', ...
%\swapnumbers


%% Theorem environments

% theorem style plain
\theoremstyle{plain}
\newtheorem{conjecture}[theoremcounter]{Conjecture}
\newtheorem{corollary}[theoremcounter]{Corollary}
\newtheorem{lemma}[theoremcounter]{Lemma}
\newtheorem{proposition}[theoremcounter]{Proposition}
\newtheorem{solution}[theoremcounter]{Solution}
\newtheorem{theorem}[theoremcounter]{Theorem}
\newtheorem{claim}{Claim}

% Theorem in introduction
\theoremnumbering{Alph}
\newtheorem{introtheorem}{Theorem}
\newtheorem{introcorollary}[introtheorem]{Corollary}
%   \renewcommand{\thethmstarcounter}{\Alph{thmstarcounter}}
%\newtheorem{thmstar}[thmstarcounter]{Theorem}




% theorem style definition
\theoremstyle{definition}
\newtheorem{assumption}[theoremcounter]{Assumption}
\newtheorem{definition}[theoremcounter]{Definition}


% theoremstyle remark
\theoremstyle{remark}

\newtheorem{example}[theoremcounter]{Example}
\newtheorem{exercise}[theoremcounter]{Exercise}
\newtheorem{motivation}[theoremcounter]{Motivation}
\newtheorem{notation}[theoremcounter]{Notation}
\newtheorem{problem}[theoremcounter]{Problem}
\newtheorem{question}[theoremcounter]{Question}
\newtheorem{recall}[theoremcounter]{Recall}
\newtheorem{remark}[theoremcounter]{Remark}
\newtheorem{observation}[theoremcounter]{Observation}
% Remarks without numbering
%\theoremseparator{.}
%\newtheorem{remark}{Remark}
\newtheorem{summary}[theoremcounter]{Summary}


% Alternative proof environment

% \theorembodyfont{\normalfont}
% \theoremsymbol{\ensuremath{\Box}}
% \newtheorem*{proof}{Proof}




%%%%%%
%% Enumeration
%%%%%%

% Use small roman numbers on the first level of the 'enumeration' environment
\renewcommand{\theenumi}{(\roman{enumi})}
\renewcommand{\labelenumi}{\theenumi}



%%%%%%
%% Comments
%%%%%%

\usepackage{xargs}                      % Use more than one optional parameter in a new commands
\usepackage[pdftex,dvipsnames]{xcolor}  % Coloured text etc. 
\usepackage[colorinlistoftodos,prependcaption,textsize=tiny]{todonotes}
\newcommandx{\unsure}[2][1=]{\todo[linecolor=red,backgroundcolor=red!25,bordercolor=red,#1]{#2}}
\newcommandx{\change}[2][1=]{\todo[linecolor=blue,backgroundcolor=blue!25,bordercolor=blue,#1]{#2}}
\newcommandx{\info}[2][1=]{\todo[linecolor=OliveGreen,backgroundcolor=OliveGreen!25,bordercolor=OliveGreen,#1]{#2}}
\newcommandx{\improvement}[2][1=]{\todo[linecolor=Plum,backgroundcolor=Plum!25,bordercolor=Plum,#1]{#2}}



%%%%%%%
%% References
%%%%%%%

%% BibTex
%\bibliographystyle{plain}
%\bibliography{operatoralgebras.bib}
%\bibliography{mybibliography}{}

% Cleveref.  Needs to be loaded after amsmath
\usepackage[nameinlink, capitalize, noabbrev]{cleveref}

%% BibLaTeX
\usepackage[style=alphabetic]{biblatex}
\addbibresource{mybibliography.bib}

%% Ensure that quoted texts are typeset according to the rules of the main language
\usepackage{csquotes}

% parameters for BibLaTeX
\ExecuteBibliographyOptions{
bibencoding=utf8, 
maxbibnames=99,    % no et. al.
maxcitenames=3,
sorting=nyt,       % name n, title t, year y
giveninits=true,   % only initials of first names
backref=false,     % reference to citation places
isbn=true,
doi=true,
eprint=true,
url=false,
}

% Create links to preprints automaticsally
\DeclareFieldFormat{eprint}{\href{https://arxiv.org/abs/#1}{\texttt{#1}}}

% Remove the "In:" for jounrals
\renewbibmacro*{in:}{}

% Remove quotes from titles
\DeclareFieldFormat
[article,inbook,incollection,inproceedings,patent,thesis,unpublished,misc]
{title}{#1}

% Add quotes to misc
% \DeclareFieldFormat
  % [misc, book]
  % {title}{\mkbibquote{#1\isdot}}

  % small size bibliography
\renewcommand*{\bibfont}{\footnotesize}



%%%%%%%
%% Short cuts
%%%%%%%

\usepackage{times}
\usepackage{epsfig}
\usepackage{graphicx}
\usepackage{float}
\usepackage{wrapfig}
\usepackage{amsmath,amssymb,amsthm}
\usepackage{algorithm,algorithmicx,algpseudocode}
\usepackage{bm,xspace}
\usepackage{comment}
\usepackage{verbatim}
\usepackage{multirow}
\usepackage{balance}
\usepackage{url}
\usepackage{booktabs}
\usepackage{etoolbox}
\usepackage{siunitx}
\usepackage{calc}
\usepackage{pifont,hologo}
\usepackage{nicefrac}
%\usepackage[usenames, dvipsnames]{xcolor}
\usepackage[normalem]{ulem}

\setlength\heavyrulewidth{0.10em}
\setlength\lightrulewidth{0.05em}
\setlength\cmidrulewidth{0.03em}
\newcommand{\ra}[1]{\renewcommand{\arraystretch}{#1}}

\usepackage[super]{nth}
\usepackage{nicefrac}
\sisetup{detect-weight=true,detect-inline-weight=math}
\sisetup{quotient-mode = fraction}
\sisetup{fraction-function = \nicefrac}
\robustify\bfseries
\robustify\uline
\def\myuline#1{#1\llap{\uline{\phantom{#1}}}}

\def\aa{\mathbf{a}}
\def\bb{\mathbf{b}}
\def\cc{\mathbf{c}}
\def\dd{\mathbf{d}}
\def\ee{\mathbf{e}}
\def\ff{\mathbf{f}}
\def\gg{\mathbf{g}}
\def\hh{\mathbf{h}}
\def\ii{\mathbf{i}}
\def\jj{\mathbf{j}}
\def\kk{\mathbf{k}}
\def\ll{\mathbf{l}}
\def\mm{\mathbf{m}}
\def\nn{\mathbf{n}}
\def\oo{\mathbf{o}}
\def\pp{\mathbf{p}}
\def\qq{\mathbf{q}}
\def\rr{\mathbf{r}}
\def\sss{\mathbf{s}}
\def\ttt{\mathbf{t}}
\def\uu{\mathbf{u}}
\def\vv{\mathbf{v}}
\def\ww{\mathbf{w}}
\def\xx{\mathbf{x}}
\def\yy{\mathbf{y}}
\def\zz{\mathbf{z}}

\def\AA{\mathbf{A}}
\def\BB{\mathbf{B}}
\def\CC{\mathbf{C}}
\def\DD{\mathbf{D}}
\def\EE{\mathbf{E}}
\def\FF{\mathbf{F}}
\def\GG{\mathbf{G}}
\def\HH{\mathbf{H}}
\def\II{\mathbf{I}}
\def\JJ{\mathbf{J}}
\def\KK{\mathbf{K}}
\def\LL{\mathbf{L}}
\def\MM{\mathbf{M}}
\def\NN{\mathbf{N}}
\def\OO{\mathbf{O}}
\def\PP{\mathbf{P}}
\def\QQ{\mathbf{Q}}
\def\RR{\mathbf{R}}
\def\SS{\mathbf{S}}
\def\TT{\mathbf{T}}
\def\UU{\mathbf{U}}
\def\VV{\mathbf{V}}
\def\WW{\mathbf{W}}
\def\XX{\mathbf{X}}
\def\YY{\mathbf{Y}}
\def\ZZ{\mathbf{Z}}

\def\aA{\mathcal{A}}
\def\bB{\mathcal{B}}
\def\cC{\mathcal{C}}
\def\dD{\mathcal{D}}
\def\eE{\mathcal{E}}
\def\fF{\mathcal{F}}
\def\gG{\mathcal{G}}
\def\hH{\mathcal{H}}
\def\iI{\mathcal{I}}
\def\jJ{\mathcal{J}}
\def\kK{\mathcal{K}}
\def\lL{\mathcal{L}}
\def\mM{\mathcal{M}}
\def\nN{\mathcal{N}}
\def\oO{\mathcal{O}}
\def\pP{\mathcal{P}}
\def\qQ{\mathcal{Q}}
\def\rR{\mathcal{R}}
\def\sS{\mathcal{S}}
\def\tT{\mathcal{T}}
\def\uU{\mathcal{U}}
\def\vV{\mathcal{V}}
\def\wW{\mathcal{W}}
\def\xX{\mathcal{X}}
\def\yY{\mathcal{Y}}
\def\zZ{\mathcal{Z}}

\def\Ae{\mathbb{A}}
\def\Be{\mathbb{B}}
\def\Ce{\mathbb{C}}
\def\Le{\mathbb{L}}
\def\Ne{\mathbb{N}}
\def\Pe{\mathbb{P}}
\def\Qe{\mathbb{Q}}
\def\Re{\mathbb{R}}
\def\Se{\mathbb{S}}
\def\Te{\mathbb{T}}
\def\Xe{\mathbb{X}}
\def\Ye{\mathbb{Y}}
\def\Ze{\mathbb{Z}}

\def\btheta{{\bm\theta}}
\def\bTheta{{\bm\Theta}}
\def\bzeta{{\bm\zeta}}
\def\bmu{{\bm\mu}}
\def\bZero{\mathbf{0}}
\def\bOne{\mathbf{1}}

\DeclareMathOperator*{\argmax}{arg\,max}
\DeclareMathOperator*{\argmin}{arg\,min}
\DeclareMathOperator*{\minimize}{minimize}
\DeclareMathOperator*{\maximize}{minimize}

\DeclareMathOperator{\sign}{sign}
\DeclareMathOperator{\rank}{rank}
\DeclareMathOperator{\trace}{tr}
\DeclareMathOperator{\diag}{diag}
\DeclareMathOperator{\diam}{diam}

\def\eps{\varepsilon}
\def\vphi{\varphi}
\def\vsigma{\varsigma}

\def\trans{^{\top}}
\def\deg{^{\circ}}

\def\th{^{\textnormal{th}}}
\def\st{^{\textnormal{st}}}
\def\nd{^{\textnormal{nd}}}

\newcommand\ggreater{\mathbin{>\!\!\!>}}
\newcommand\llesser{\mathbin{<\!\!\!<}}

\newcommand{\YesV}{\ding{51}}%
\newcommand{\NoX}{\ding{55}}%

\DeclareMathSymbol{@}{\mathord}{letters}{"3B}

\newcommand\paren[1]{\left(#1\right)}
\newcommand\abs[1]{\left\lvert#1\right\rvert}
\newcommand\norm[1]{\left\lVert#1\right\rVert}
\newcommand\tuple[1]{\left\langle#1\right\rangle}
\newcommand\inner[2]{\langle #1, #2 \rangle}

\newcommand\timess{\mathbin{\!\times\!}}

\newcommand\nicebar[1]{\mkern 4mu\overline{\mkern-4mu#1}}

\newcommand\mypara[1]{\vspace{1mm}\noindent\textbf{#1}}

\def\rot#1{\rotatebox{90}{#1}}

\def\latex/{\LaTeX}
\def\bibtex/{\hologo{BibTeX}}

\def\mydown{\makebox[6pt]{$\downarrow$}}
\newcommand{\fg}[1]{\textcolor{blue}{$(#1\%)$}}
\def\slant#1{\rotatebox[origin=c]{90}{#1}}

\newcommand{\tubedim}{\operatorname{dim}_{\mathrm{tube}}}
\newcommand{\dimnuc}{\operatorname{dim}_{\mathrm{nuc}}}
\DeclareMathOperator{\interior}{int}
\DeclareMathOperator{\boxen}{tube}
\def\boxint#1{\interior_{\boxen}(#1)}
\def\boxboundary#1{\partial_{\boxen}(#1)}
\DeclareMathOperator{\vol}{vol}
\DeclareMathOperator{\cl}{cl}
\DeclareMathOperator{\pair}{Pair}
\DeclareMathOperator{\Per}{Per}
\newcommand{\pairtube}{\mathcal{H}}
\newcommand{\pairtubeopen}{\mathcal{H}^{\mathrm{open}}}
\newcommand{\pairtubei}{\mathcal{H}_i}
\newcommand{\pairtubeopeni}{\tilde{\mathcal{H}}_i}
\newcommand{\haar}{m}
\newcommand{\lreg}{\lambda}
\newcommand{\nought}{^{(0)}}

%%%%%% 
%% Authors
%%%%%%

\newcommand{\authors}{Ulrik Enstad $\bullet$ Gabriel Favre $\bullet$ Sven Raum}
\renewcommand{\title}{Free actions of polynomial growth Lie groups and classifiable C*-algebras}
\newcommand{\shorttitle}{}



%%%%%%%
%% Header
%%%%%%%

% \lhead{\shorttitle}
% \chead{}
% \rhead{\small by \authors}

% \cfoot{Page {\thepage} of \pageref{endpage}}



\begin{document}

%%%%%%%%
%% Front page
%%%%%%%%


\thispagestyle{empty}

\noindent
\begin{minipage}{\linewidth}
  \begin{center}
    \textbf{\Large \title} \\
    \authors    
  \end{center}
\end{minipage}

% \renewcommand{\thefootnote}{}
%  \footnotetext{last modified on \today}
% \footnotetext{
%   \textit{MSC classification: }
% }
% \footnotetext{
%   \textit{Keywords: }
% }

\vspace{2em}
\noindent
\begin{minipage}{\linewidth}
  \textbf{Abstract}.
  We show that any free action of a connected Lie group of polynomial growth on a finite dimensional locally compact space has finite tube dimension.  This is shown to imply that the associated crossed product C*-algebra has finite nuclear dimension.  As an application we show that C*-algebras associated with certain aperiodic point sets in connected Lie groups of polynomial growth are classifiable.  Examples include cut-and-project sets constructed from irreducible lattices in products of connected nilpotent Lie groups.
\end{minipage}



%%%%%%%%% 
%% Document body
%%%%%%%%%


\section{Introduction}
Current quantum hardware is unable to carry out universal quantum computations due to the buildup of errors that occur during the computation. 
The magnitude of the individual error is currently above the value that the Threshold Theorem requires in order to kick-start quantum error correction and fault-tolerant quantum computation~\cite[Section 10.6]{nielsen_chuang_2010}. 
Although the experimentally achieved fidelity rates are promising and the error bounds are inching closer to the required threshold, we will have to work for the foreseeable future with quantum hardware with errors that build-up during the computation.  This implies that we can only do a limited number of steps before the output of the computation has become completely uncorrelated with the intended one.

For fault-tolerant quantum computing, we repeat four steps: 
1) We apply a number of single and two-qubit quantum gates, in parallel whenever possible; 
2) We perform a syndrome measurement on a subset of the qubits; 
3) We perform fast classical computations to determine which errors have occurred and how to correct them; 
and, 4) We apply correction terms based on the classical computations.
We then repeat these four steps with a next sequence of gates. 
These four steps are essential to fault-tolerant quantum computing. 


The starting point of this work is to use the four steps outlined above, not to carry out error correction and fault-tolerant computation, but to enhance short, constant-depth, {\em uncorrected} quantum circuits that perform single qubit gates and {\em nearest-neighbor} two qubit gates. 
Since in the long run we will have to implement error-correction and fault-tolerant computation anyhow, and this is done by such a four-step process, why not make other use of this architecture? Moreover, on some of the quantum hardware platforms, these operations are already in place.
Embracing this idea we naturally arrive at the question: what is the computational power of \textit{low-depth} quantum-classical circuits organized as in the four steps outlined above? 
We thus investigate circuits that execute a small, ideally constant, number of stages, where at each stage we may apply, in parallel, single qubit gates and {\em nearest-neighbor} two qubit gates, followed by measurements, followed by low-depth classical computations of which the outcome can control quantum gates in later stages. 
It is not clear, at first, whether such circuits, especially with constant depth, can do anything remotely useful. 
But we will see that this is indeed the case: many quantum computations can be done by such circuits in constant depth. 
By parallelizing quantum computations in this way, we improve the overall computational capabilities of these circuits, as we do not incur errors on qubits that are idle, simply because qubits are not idle for a very long time. 
Furthermore, reducing the depth of quantum circuits, at the cost of increasing width, allows the circuit to be run faster even if errors occur.

The first usage of such a four-step layout, not to do error correction, but to perform computations, can be found in the paradigm of measurement-based quantum computing~\cite{gottesman1999demonstrating,raussendorf2001one,jozsa2006introduction,clark2007generalised}: 
A universal form of quantum computing where a quantum state is prepared and operations are performed by measuring qubits in different bases, depending on previous measurements and intermediate measurements.

\citeauthor{PhamSvore2013} were the first to formalize the four-step protocol for performing computations~\cite{PhamSvore2013}. They included specific hardware topologies by considering two-dimensional graphs for imposing constraints on qubit interactions. In their model, they develop circuits for particularly useful multi-qubit gates, including specifying costs in the width, number of qubits, depth, number of concurrent time steps, size, and total number of non-Identity operations.
As a result, they find an algorithm that factors integers in polylogarithmic depth.
\citeauthor{Browne:2011} showed that the main tool in the work by \citeauthor{PhamSvore2013}, the fan-out gate, can also be replaced by additional log-depth classical computations in the measurement-based quantum computing setting~\cite{Browne:2011}.

More recently, \citeauthor{Cirac:2021} introduced a scheme to implement unitary operations involving quantum circuits combined with Local Operations and Classical Communication ($\mathsf{LOCC}$) channels: $\mathsf{LOCC}$-assisted quantum circuits~\cite{Cirac:2021}. Similarly to the four-step scheme we just described, they allow for a short depth circuit to be run on the qubits, followed by one round of $\mathsf{LOCC}$, in which ancilla qubits are measured and local unitaries are applied based on the measurement outcomes. They show that in this model any 1D transitionally invariant matrix-product state (MPS) with fixed bond dimension is in the same phase of matter as the trivial state. Similar ideas can be found in~\cite{TVV_NonAbelianTopologicalOrder_2022, tantivasadakarn2021long}.

In this work, we introduce a new model, called \textit{Local Alternating Quantum-Classical Computations} ($\LAQCC$). In this model we alternate between running quantum circuits (constrained by locality), ending in the measurement of a subset of qubits, and fast classical computations based on the measurement results. The outcome of the classical computations are then used to control future quantum circuits. We allow for flexibility in this model, by giving different constraints to the power of both the quantum circuits and the classical circuits as well as the number of alternations between them. 
Most attention will be given to $\LAQCC$ containing quantum circuits of constant depth, classical circuits of logarithmic depth and at most a constant number of alternations between them. 
Any circuit constructed in this model is considered to be of constant depth. 
We restrict ourselves to logarithmic depth classical computations, as this is the first natural and non-trivial extension beyond constant-depth classical computations. 
Constant-depth classical computations do however also have an equivalent constant-depth quantum implementation.

The definition of $\LAQCC$ sharpens the original definition of \citeauthor{PhamSvore2013} by adding constraints to the intermediate classical computations. This allows us to bound the power of $\LAQCC$ from above. 

The main result of \citeauthor{Cirac:2021}, that 1D translational invariant MPS with fixed bond dimension can be prepared by $\mathsf{LOCC}$-assisted circuits, relies on local symmetries of the MPS. These symmetries allow them to prepare local states (on a constant number of qubits) and glue them together by doing one round of the appropriate entangling measurement and corrections, after which they run a round of local unitaries to get the desired result. This general scheme for preparing states that exhibit an MPS description with the appropriate local symmetries requires only geometrically local unitaries and one round of measurement and corrections an therefore is accessible in $\LAQCC$. Studying different local symmetries, known as Symmetry Protected Topological (SPT) phases of matter, to find measurement-based constant depth circuits for states is a broad ongoing field of research~\cite{TVV_NonAbelianTopologicalOrder_2022, tantivasadakarn2021long, smith2023deterministic}. 
All these schemes have a $\LAQCC$ implementation.

%$\LAQCC$-circuits also exist for general schemes of preparing local states, based on the local tensors, and gluing them together using one round of entangled measurement and corrections, based on the local symmetry. 
%The main result of \citeauthor{Cirac:2021}, that 1D translational invariant MPS with fixed bond dimension can be prepared by $\mathsf{LOCC}$-assisted circuits, relies heavily on local symmetries of the MPS and as a result also has an equivalent $\LAQCC$ implementation. 
%The corrections applied after the measurement round are local unitaries depending on the local symmetries of the MPS. 

 

%This general scheme of preparing local states, based on the local tensors, and gluing it together by doing one round of entangled measurement and corrections, based on the local symmetry, is accessible in $\LAQCC$.
Note however that \citeauthor{Cirac:2021} also suggest a circuit for the $W$-state.
This circuit uses sequentially and dependent measurement-based corrections of the ancilla qubits. 
These dependent measurements translate to sequential alternations between the quantum and classical circuits and therefore increase the total depth to linear depth, exceeding the constant-depth constraints imposed by $\LAQCC$-circuits. 

We study the power of the $\LAQCC$ model with respect to state preparation, showing that even with only constant quantum-depth and logarithmic classical depth it remains possible to prepare states with long-range entanglement.
Another surprising result is that it is unlikely that $\LAQCC$ circuits are classically simulatable. We show that any instantaneous quantum polynomial-time (IQP) circuit~\cite{Bremner2010,Shepherd2009} has an $\LAQCC$ implementation.
Classical simulation of IQP circuits implies the collapse of the polynomial hierarchy to the third level, which is not believed to be true~\cite{Bremner2017}. Therefore, we expect that $\LAQCC$ circuits are unlikely to be classically simulatable. We bound the power of $\LAQCC$ by showing that it is contained in $\QNC^1$, the class of polynomial-size, log-depth circuits.

Next, we also study the power that intermediate classical calculations can add to quantum computations, by considering a new model that alternates between polynomially many polynomial-depth quantum circuits and unbounded classical computations
We study this model by doing a complexity theoretical analysis, where we draw inspiration from the notions of complexity given by \citeauthor{RosenthalYuen:2022}, \citeauthor{MetgerYuen:2023}, and \citeauthor{Aaronson:2004}.
All three complexity notions are based on the notion of state preparation, instead of more traditional definition of complexity such as the decidability of a computational problem. 
The first two consider classes based on sequences of quantum states preparable by a polynomial-sized quantum circuit, where the circuits are uniformly generated by a computational class, for instance, the class $\mathsf{PSPACE}$, which results in the complexity class $\mathsf{StatePSPACE}$~\cite{RosenthalYuen:2022,MetgerYuen:2023}.
The third notion considers a relative complexity, where the complexity is measured between two given states, and is measured by the number of gates, from a given gate-set, required to transform one state in another state~\cite{Aaronson:2004}. 
For our definition of state preparation complexity, we drop the uniformity constraint from~\cite{RosenthalYuen:2022,MetgerYuen:2023} and define a class as $\mathsf{StateX}$, which refers to states preparable by circuits of type $\mathsf{X}$. 
As an example, if $\mathsf{X} = \QNC^0$, this results in the class $\mathsf{StateQNC^0}$, which is the set of states preparable from the $\ket{0}^n$ state by poly-size constant-depth circuits. 
This notion is similar to the relative complexity from~\cite{Aaronson:2004}, where one state is the  $\ket{0}^n$ state and instead of counting the number of gates we consider the set of states preparable by a fixed number of gates. Using this notion of complexity we show that any state preparable by an $\LAQCC^*$ circuit is also preparable by a $\mathsf{PostQPoly}$ circuit, the class of circuits of polynomial depth with an additional post-selection gate. 

All Clifford circuits have a constant-depth $\LAQCC$ implementation, implying that any stabilizer state can be implemented by a constant-depth $\LAQCC$ circuit, see Section~\ref{sec:clifford_circuits} for a proof of this statement. 
Efficient circuits for stabilizer states have been known already through measurement-based quantum computing. Therefore this paper focuses on the preparation of non-stabilizer states, and as a surprising result we find novel constant-depth protocols for four very natural classes of non-stabilizer states.
Despite the extensive research into these four classes of non-stabilizer states and the many applications of them, no efficient constant- or low-depth state preparation protocols are known yet. We specifically consider these four classes as they are all often used as initial states in other algorithms.

The first state is a uniform superposition over an arbitrary number of states. 
This state finds applications in many quantum algorithms, as they often start with a uniform superposition over multiple states. 
This superposition is often achieved by applying Hadamard gates to every qubit due to its simplicity to prepare. 
Yet, the analysis of many algorithms, such as Shor's algorithm~\cite{Shor:1997}, would benefit from a different initial superposition. 
The circuit to prepare the uniform superposition over an arbitrary number of states uses an exact version of Grover search as a subroutine, that turns a probabilistic circuit, with a known constant probability of success, into a deterministic circuit. 
We use the circuit for preparing a uniform superposition over an arbitrary number of states as a subroutine in the next two quantum state preparation protocols. 

The second state is the $W$-state, the uniform superposition over all computational basis states of Hamming-weight~$1$, a natural long-ranged entangled state that displays a fundamentally nonequivalent type of entanglement from the Greenberger–Horne–Zeilinger state~\cite{WState:2000}, for which $\LAQCC$-type constant-depth circuits were previously known~\cite{PhamSvore2013, Cirac:2021}. 
The $W$-state is often used as benchmark for new quantum hardware~\cite{Haffner2005,Neeley2010,GarciaPerez:2021}. 
A novel way to prepare the $W$-state therefore gives a new way to benchmark different quantum devices with each other. 
A circuit for preparing the $W$-state was given in~\cite{Cirac:2021}, but this implementation requires sequentially alternating measurements followed by local unitaries, which in the $\LAQCC$ model is not considered to be of constant depth. 
We improve this protocol by giving an $\LAQCC$ implementation of the $W$-state, based on a compress-uncompress method that links the one-hot and binary encoding of integers.

The third state considered is the Dicke state, a generalization of the $W$-state, a superposition over all computational basis states with Hamming-weight $k$~\cite{Dicke:1954}. 
Dicke states have relevance in various practical settings.
For instance, for quantum game theory~\cite{zdemir2007}, quantum storage~\cite{Bacon_Compress:2006,Plesch:2010}, quantum error correction~\cite{ouyang2014permutation}, quantum metrology~\cite{toth2012multipartite}, and quantum networking~\cite{prevedel2009experimental}. 
Dicke states have been used as a starting state for variational optimization algorithms, most notably Quantum Alternating Operator Ansatz (QAOA)~\cite{Hadfield2019}, to find solutions to problems such as Maximum k-vertex Cover~\cite{Brandhofer2022,cook2020quantum}.
The ground states of physical Hamiltonians describing one-dimensional chains tend to show a resemblance to Dicke states such as states resulting from the Bethe ansatz, making them an ideal starting state when investigating the ground state behavior of these Hamiltonians~\cite{TDL_BetheAnsatzDerivation:2010,B_ExcitedStateQuantumPhaseTransitions:2013,DickeTransitions:2021}. 
For instance, the algorithm by \citeauthor{van2021preparing}, who give an algorithm to prepare the Bethe ansatz eigenstates of the spin-1/2 XXZ spin chain, starts by first preparing a Dicke state~\cite{van2021preparing}. 
A Dicke-state preparation protocol based on the compress-uncompress methodology used in the $W$-state furthermore finds applications in entanglement distillation, where the entanglement of a large state is concentrated on only a few qubits. 
Efficient deterministic circuits for preparing Dicke states have been proposed by \citeauthor{bartschi2019deterministic}~\cite{bartschi2019deterministic, bartschi2022deterministic_short_depth}. 
They provide a quantum circuit of depth $\mathO(k \log(\frac{n}{k}))$, allowing arbitrary connectivity, to prepare a Dicke state, which they conjecture to be optimal when $k$ is constant. 
In this work, we provide a constant-depth $\LAQCC$ circuit below their conjectured bound already for constant $k$. 
However, this does not directly disprove their conjecture, as we allow for intermediate measurements and classical computations. 
More significantly, we even construct constant-depth $\LAQCC$ circuits for $k = \mathO(\sqrt{n})$ greatly improving their bound.
This construction extends the compress-uncompress method for the $W$-state combined with additional subroutines. 

We continue with a log-depth state preparation protocol for the Dicke-state for arbitrary $k$. 
This protocol implements an efficient transformation between the factoradic number representation and the combinatorial number representation of a positive integer. 
The combinatorial number representation relates directly to the Dicke state. 
The provided efficient transformation between number representation systems might be of independent interest. 

We conclude by modifying our protocol for preparing a Dicke-state to a protocol that prepares quantum many-body scar states in constant-depth. 
These states have low entanglement and longer coherence times than states with similar energy density.
These characteristics make many-body scar states interesting to analyze and relevant within physics.
Many-body scar states appear for instance in the AKLT model~\cite{AKLT:1987,MRBAR:2018,MRB:2018} and different spin models~\cite{SI:2019,MOBFR:2020}.
Known methods for preparing these states have polynomial-depth~\cite{Gustafson:2023}, whereas our circuit has constant depth. 

% We conclude by studying the power that intermediate classical calculations can add to quantum computations. 
% In this study, we define a new model that relaxes constant-depth quantum circuits to polynomial depth quantum circuits, log-depth classical calculations to unbounded classical computations and a constant number of alternations to a polynomial number of alternations. 
% We call this model $\LAQCC^*$. 
% We study this model by doing a complexity theoretical analysis, where we draw inspiration from the notions of complexity given by \citeauthor{RosenthalYuen:2022}, \citeauthor{MetgerYuen:2023}, and \citeauthor{Aaronson:2004}.
% All three complexity notions are based on the notion of state preparation, instead of more traditional definition of complexity such as the decidability of a computational problem. 
% The first two consider classes based on sequences of quantum states preparable by a polynomial-sized quantum circuit, where the circuits are uniformly generated by a computational class, for instance, the class $\mathsf{PSPACE}$, which results in the complexity class $\mathsf{StatePSPACE}$~\cite{RosenthalYuen:2022,MetgerYuen:2023}.
% The third notion considers a relative complexity, where the complexity is measured between two given states, and is measured by the number of gates, from a given gate-set, required to transform one state in another state~\cite{Aaronson:2004}. 
% For our definition of state preparation complexity, we drop the uniformity constraint from~\cite{RosenthalYuen:2022,MetgerYuen:2023} and define a class as $\mathsf{StateX}$, which refers to states preparable by circuits of type $\mathsf{X}$. 
% As an example, if $\mathsf{X} = \QNC^0$, this results in the class $\mathsf{StateQNC^0}$, which is the set of states preparable from the $\ket{0}^n$ state by poly-size constant-depth circuits. 
% This notion is similar to the relative complexity from~\cite{Aaronson:2004}, where one state is the  $\ket{0}^n$ state and instead of counting the number of gates we consider the set of states preparable by a fixed number of gates. Using this notion of complexity we show that any state preparable by an $\LAQCC^*$ circuit is also preparable by a $\mathsf{PostQPoly}$ circuit, the class of circuits of polynomial depth with an additional post-selection gate. 

\paragraph{Summary of results}
\begin{itemize}
    \item We give a new definition of a computational model that captures the power of the four step process: applying a constant number of layers of one- and two-qubit gates; performing a syndrome measurement; perform a fast classical computation determining corrections; apply corrections. We call this model \emph{Local Alternating Quantum Classical Computations}, or $\LAQCC$ for short. In this model we bound the allowed quantum operations, intermediate classical calculations, and number of rounds separately. In Section~\ref{sec:LAQCC_model} we define this model and give a list of operations based on results from literature contained in this computational model. In some of these operations we explicitly use that we allow for multiple, but at most constant, rounds  of corrections.
    \item  We show show that there exist $\LAQCC$ circuits that can not be weakly simulated in Section~\ref{sec:IQP_in_LAQCC}. We further show that for every $\LAQCC$ circuit there exists a $\QNC^1$ circuit simulating it perfectly, in Section~\ref{sec:LAQCC_in_QNC1}.
    \item We introduce a new type computational complexity for preparing states and show that the extension of $\LAQCC$ where we allow a polynomial number of rounds and unbounded classical computation, is contained in $\mathsf{PostQPoly}$, the class of polynomial circuits with post-selection, in Section~\ref{sec:Complexity results}.
    \item We show a protocol to prepare the uniform superposition state of size $q$ in $\LAQCC$ using $\mathO(\ceil{\log_2(q)}^2)$ qubits in Section~\ref{sec:superposition_modulo_q}. 
    \item We show a protocol to prepare the $W_n$ state in $\LAQCC$ using $\mathO(n\log(n))$ qubits in Section~\ref{sec:W_state_in_LAQCC}.
    \item We show two ways of preparing the Dicke-$(n,k)$ state. The first method is in $\LAQCC$, works up to $k = \mathO(\sqrt{n})$, uses $\mathO(n^2\log(n))$ qubits, and is found in Section~\ref{sec:dicke:small_k}. The second method is in $\LAQCC\text{-}\mathsf{LOG}$ (an extension of $\LAQCC$ allowing for logarithmic number of alterations instead of constant), works for any $k$, uses $\mathO(\text{poly}(n))$ qubits, and is found in Section~\ref{sec:Dicke_in_LAQCC_LOG}. 
    \item We extend on our $\LAQCC$ method of generating Dicke-$(n,k)$ states for $k = \mathO(\sqrt{n})$ and show a protocol to generate many-body scar states for a particular Hamiltonian in $\LAQCC$ (Section~\ref{sec:many_body_scar}). 
\end{itemize}
Summarized in a table, we provide the following state generation protocols:
\begin{table}[htb]
\centering
\begin{tabular}{l|l|l|l}
\textbf{State description} & \textbf{Width} & \textbf{Depth} & \textbf{Implementation}\\
\hline 
Uniform superposition mod $q$: $\frac{1}{\sqrt{q}} \sum_{i = 0}^{q-1}\ket{i}$ & $\mathO(\ceil{\log^2 q})$ & $\mathO(1)$ & Section~\ref{sec:superposition_modulo_q}\\

$W$-state: $\frac{1}{\sqrt{n}}\sum_{i = 0}^{n-1}\ket{e_i}$ & $\mathO(n \log n)$ & $\mathO(1)$ & Section~\ref{sec:W_state_in_LAQCC}\\

Dicke-$(n,k)$, $k = \mathO(\sqrt{n})$: $\binom{n}{k}^{-1/2}\sum_{x \in \{0,1\}^n: |x| = k} \ket{x}$ &  $\mathO(n^2\log n)$ & $\mathO(1)$ 
&Section~\ref{sec:dicke:small_k}\\

Dicke-$(n,k)$: $\binom{n}{k}^{-1/2}\sum_{x \in \{0,1\}^n: |x| = k} \ket{x}$ & $\mathO(\text{poly}(n))$ & $\mathO(\log n)$ &Section~\ref{sec:Dicke_in_LAQCC_LOG}\\

QMBS: $\ket{S_k} = \frac{1}{k! \sqrt{\mathcal N(n,k)}}(Q^\dagger)^k \ket{\Omega}$ &  $\mathO(n^2\log n)$ & $\mathO(1)$  &  Section~\ref{sec:many_body_scar}
\end{tabular}
\caption{Summary of state preparation protocols given in this paper.}
\label{tab:sate_prep}
\end{table}
In the entry for the quantum many-body scar state $Q$ denotes the raising operator and $\mathcal N(n,k)=\binom{n-k-1}{k}$. 
Section~\ref{sec:many_body_scar} will provide more details on the variables and the implementation. 

\paragraph{Organization of the paper}
\noindent We first introduce relevant preliminaries in Section~\ref{sec:preliminaries}. 
In Section~\ref{sec:LAQCC_model} we formally define the class of Local Alternating Quantum-Classical Computations ($\LAQCC$). We also show that any Clifford circuit can be implemented in constant depth $\LAQCC$ (a result based on a result from measurement-based quantum computing~\cite{jozsa2006introduction}). 
This result allows us to give many useful multi-qubit gates and routines in Section~\ref{sec:gates_created_in_LAQCC}. 
Beyond that we show that constant depth $\LAQCC$ circuits are contained in $\QNC^1$ and that any $\mathsf{IQP}$ circuit has an $\LAQCC$ implementation.
We conclude this section with an analysis of a more powerful instantiation of $\LAQCC$ and show an inclusion with respect to the class $\mathsf{PostQPoly}$, which is the class of circuits of polynomial depth with one additional post-selection gate. 
In Section~\ref{sec:state_prep_in_LAQCC} we give $\LAQCC$ circuit implementations for preparing the uniform superposition over an arbitrary number of states, the $W$-state and the Dicke state up to $k = \mathO(\sqrt{n})$. We furthermore give a log-depth circuit implementation for preparing the Dicke state for any $k$. We conclude by showing a $\LAQCC$ circuit for generating many body scar states of a particular type of Hamiltonian.



\subsection*{Acknowledgements}

The first author acknowledges support from The Research Council of Norway through project 314048. He is also indebted to Stockholm University and the University of Potsdam for their hospitality during extended research stays in 2021--2023. The second and third authors were supported by the Swedish Research Council through grant number 2018-04243.

The third author would like to thank Nigel Higson and Bram Mesland for useful discussions on the K-theory of C*-algebras considered in this work.

\section{Preliminaries}
In this section, we describe the necessary background for automated planning and the significance of the International Planning Competition. 

% \subsection{Ontology}
% A formal ontology is typically represented as a set of concepts, relations, and axioms. A concept represents a set of objects or entities that share common properties, while a relation represents a connection or association between two or more concepts. Axioms are statements that define the relationships between concepts and relations. It is a formal representation of knowledge that is designed to facilitate automated reasoning and information processing. It acts as a structured vocabulary that describes a domain and promotes interoperability, data integration, and communication between humans and machines. Formally, an ontology $O$ can be represented as a tuple $(C, R, A)$, where $C$ is the set of concepts, $R$ is the set of relations, and $A$ is the set of axioms. Each concept \textit{c} $\in$ $C$ can be represented as a set of attributes, denoted as $Att(c)$. Similarly, each relation \textit{r} $\in$ $R$ can be represented as a set of attributes, denoted as $Att(r)$.

% Ontology is a branch of philosophy that deals with the nature of existence and being. In the field of computer science, however, ontology refers to a formal representation of knowledge that is designed to facilitate automated reasoning and information processing. It is a structured vocabulary that describes a domain and promotes interoperability, data integration, and communication between humans and machines. Various tools and methodologies, including Protege and ontology editors, are available for ontology creation. Ontologies are increasingly important in artificial intelligence, knowledge engineering, and the semantic web, and researchers are exploring their potential in diverse domains and applications.

% Figure environment removed

\subsection{Automated Planning}

Automated planning, also known as AI planning, is the process of finding a sequence of actions that will transform an initial state of the world into a desired goal state \cite{ghallab2004automated}. It involves constructing a plan or a sequence of actions that will achieve a specified objective while respecting any constraints or limitations that may be present. Formally, automated planning can be defined as a tuple $(S, A, T, I, G)$, where:
\begin{itemize}
    \item $S$ is the set of possible states of the world
    \item $A$ is the set of possible actions that can be taken
    \item $T$ is the transition function that describes the effects of taking an action on the current state of the world
    \item $I$ is the initial state of the world
    \item $G$ is the desired goal state
\end{itemize}
Using this notation, the problem of automated planning can be framed as finding a sequence of actions $\prec a_1, a_2, ..., a_k\succ$ that will transform the initial state $I$ into the goal state $G$, while respecting any constraints or limitations on the actions. 
 % In automated planning, 
 A problem is defined in terms of a domain and a problem instance. The domain defines the possible actions that can be taken and the effects of each action, while the problem instance specifies the initial state of the world and the desired goal state. 
Various techniques can be used to solve the planning problem, such as search algorithms, constraint-based reasoning, and optimization methods. These techniques involve exploring the space of possible plans and selecting the one that satisfies the objective and any constraints. Figure \ref{fig:planning_bw} illustrates an automated planning scenario for the blocksworld domain, where an initial state can be transformed into a goal state by executing a sequence of actions.

% \noindent \textbf{Attributes modeled about a domain.}
%   %\noindent \textbf{Attributes modeled in a domain file}
%  \begin{enumerate}
%      \item \textbf{Requirements:} A list of requirements that the planner must satisfy in order to solve the domain. Requirements include durative actions, conditional effects, or negative preconditions. For example, in blocksworld domain with types involved, one of the requirements is \emph{typing}.
%     \item \textbf{Predicates:} Predicates are fundamental elements in the planning domain that define the properties of the world. They are used to describe the initial and goal states, as well as the preconditions and effects of actions. Predicates are usually defined as logical expressions over a set of variables, where each variable can take on a finite number of values. In the context of planning, predicates are typically used to represent facts about the world that can be true or false, such as the location of an object or the status of a machine. For example, in blocksworld domain, the predicate \verb|(on b1 b2)| could indicate that block 'b2' is on top of block 'b1'.
%      \item \textbf{Actions:} Actions are the basic units of change in the planning domain. They represent atomic operations that can be performed to transform the world from one state to another. Each action has a name, a set of parameters, preconditions that must be satisfied before the action can be executed, and effects that describe the changes that the action makes to the world. Actions can be used to model a wide variety of operations, ranging from simple movements or transformations to complex processes such as planning or decision-making. For example, in blocksworld domain, the action \verb|unstack b2 b1| can be used to unstack block 'b2' from block 'b1'. 
     
%      \item \textbf{Preconditions:} Preconditions are the conditions that must be true before an action can be executed. They are usually defined using predicates and can involve multiple variables. Preconditions can also be negative, which means that a certain condition must not be true for an action to be executed. In planning, preconditions ensure that actions are only executed when the necessary conditions have been met, such as ensuring that a machine is turned off before it is serviced. For example, in blocksworld domain, the action \verb|unstack b2 b1| has a precondition of \verb|(on b1 b2)|, meaning that for the action to be valid, the block 'b2' should be on top of block 'b1'.
     
%      \item \textbf{Effects:} Effects describe the changes that an action makes to the world. They are usually defined using predicates and can involve multiple variables. Effects can be positive, which means that a certain condition becomes true after the action is executed, or negative, which means that a certain condition becomes false after the action is executed. In the context of planning, effects are used to model the changes that result from executing an action, such as moving an object from one location to another or turning a machine on. For example, in blocksworld domain, when the action \verb|unstack b2 b1| is executed, one of its effect is \verb|(not (on b1 b2))|, indicating that block 'b2' is no longer on top of block 'b1'.
     
%      \item \textbf{Constants:} Constants are values that are fixed and do not change during the execution of the planning problem. They are used to represent objects or entities in the world that have a fixed value, such as the speed limit on a road. Constants can be used to simplify the planning problem by reducing the number of variables that need to be considered and by providing a fixed set of values that can be used in predicates and actions. For example, in blocksworld domain, the constant \emph{table} could represent the surface on which the blocks are initially placed.
     
%      \item \textbf{Types:} Types are used to classify objects or entities in the world based on their attributes or properties. They are used to define the domain of values that a variable can take on and can be used to constrain the values that are assigned to variables. In the context of planning, types are typically used to group related objects or entities together, such as cars or bicycles, and to specify the properties that are common to all members of a type, such as their color or size. For example, in blocksworld domain with types involved, one can represent the predicate as \verb|(on ?x - block ?y - block)| stating that the parameters in the predicate are of type \emph{block}.

%  \end{enumerate}


% ######### Shorter version for AI Planning preliminaries
% \subsection{Automated Planning}

% Automated planning, also known as AI planning, finds actions transforming an initial world state into a goal state \cite{ghallab2004automated}. It involves creating a plan, respecting constraints, defined as $(S, A, T, I, G)$ where $S$ is the world states set, $A$ is the actions set, $T$ is the state transition function, $I$ is the initial state, and $G$ is the goal state. The challenge is to find actions $\prec a_1, a_2, ..., a_k\succ$ converting $I$ to $G$ under constraints. 

% A problem has a domain (defining actions and effects) and an instance (specifying initial and goal states). Various techniques can be used to solve the planning problem, such as search algorithms, constraint-based reasoning, and optimization methods. These techniques involve exploring the space of possible plans and selecting the one that satisfies the objective and any constraints. Figure \ref{fig:planning_bw} illustrates an automated planning scenario for the blocksworld domain, where an initial state can be transformed into a goal state by executing a sequence of actions.

\noindent \textbf{Attributes modeled about a domain.}
 \begin{enumerate}
     \item \textbf{Requirements:} A list of requirements that the planner must satisfy to solve the given domain, e.g., \emph{typing} in blocksworld with types.
     \item \textbf{Predicates:} Define world properties, e.g., \verb|(on b1 b2)| in blocksworld.
     \item \textbf{Actions:} Units of change with preconditions and effects, e.g., \verb|unstack b2 b1| in blocksworld.
     \item \textbf{Preconditions:} Conditions for action execution, e.g., \verb|(on b1 b2)| for \\ \verb|unstack b2 b1|.
     \item \textbf{Effects:} Post-action world changes, e.g., \verb|(not (on b1 b2))| after \\ \verb|unstack b2 b1|.
     \item \textbf{Constants:} Fixed values, e.g., \emph{table} in blocksworld.
     \item \textbf{Types:} Classifications based on attributes, e.g., \\ \verb|(on ?x - block ?y - block)| in typed blocksworld.
 \end{enumerate}

\noindent \textbf{Attributes modeled about a problem instance from a domain.}
\begin{enumerate}
    \item \textbf{Name:} The name of the planning problem.
    \item \textbf{Domain:} The name of the planning domain that the problem belongs to.
    \item \textbf{Objects:} A list of objects that are present in the planning problem. Objects are typically defined in terms of their type and name. In the example shown in Figure \ref{fig:planning_bw}, objects are b1, b2, and b3.
    \item \textbf{Initial State:} A description of the initial state of the world, including the values of all relevant predicates. Figure \ref{fig:planning_bw} represents an example initial state.
    \item \textbf{Goal State:} A description of the desired goal state of the world, including the values of all relevant predicates. Figure \ref{fig:planning_bw} represents an example goal state.
\end{enumerate}

% \vspace{2cm}
\subsection{International Planning Competition (IPC)}

% IPC serves as a significant means of assessing and comparing various planning systems. By presenting new planners and benchmark problems each year, the competitions aim to stimulate the advancement of new planning methodologies and reflect current trends and challenges in the field. The competition comprises multiple tracks, each covering various planning problems such as classical, temporal, and probabilistic planning. These tracks include benchmark problems that evaluate the performance of planners concerning parameters such as plan quality, plan length, and run time. The results of these competitions provide insights into the current state-of-the-art in planning and help identify the strengths and weaknesses of different planning systems. IPC can serve as an excellent starting point for building a planning-related ontology as the benchmark problems used in these competitions can provide a comprehensive overview of the domain and the types of problems that planners need to solve. 

IPC is pivotal for evaluating and contrasting planning systems. Introducing new planners and benchmarks, it promotes innovative planning methodologies and reflects the field's evolving challenges. The competition has multiple tracks, such as classical and probabilistic planning, with benchmarks assessing plan quality, length, and run time. IPC results offer a glimpse into the latest in planning, highlighting system pros and cons. The benchmarks from IPC are ideal for crafting a planning-related ontology, encapsulating the domain's breadth and planners' challenges.

\section{Slices}
\label{sec:slices}

Slices are used to understand in how far actions of locally compact groups are locally trivial.  In the relevant context of non-compact groups, the concept was introduced by Palais' for proper actions \cite{palais1961-slices}.  In this article we are concerned with free actions, for which we can show the existence of slices for suitable classes of groups in \cref{sec:slices-existence}.  The following definition of a slice is suitable for this setup, while for non-free actions a definition closer to Palais' work is needed.
\begin{definition}
  \label{def:slice}
  Let $G \grpaction{} X$ be an action.  Let $K \subseteq G$ be some compact identity neighbourhood. A \emph{$K$-slice} is a subset $S \subseteq X$ such that the map $K \times S \to X$ given by $(g,x) \mapsto gx$ is injective. The resulting image $KS$ in  $X$ is called a \emph{tube}, and its interior $(KS)^{\circ}$ is called an \emph{open tube}.
\end{definition}



\begin{remark}
  \label{rem:slice-homeomorphic}
  The injectivity of the map $K \times S \to X$ in \Cref{def:slice} implies that it is a homeomorphism onto its image since $K \times S$ is compact and $X$ is Hausdorff. Therefore the tube $KS$ is a compact set in $X$ homeomorphic to $K \times S$.
\end{remark}



\begin{remark}
  \label{rem:slice-disjoint-translates}
  We will frequently use the following facts.  Any closed subset of a $K$-slice is also a $K$-slice, and if $L \subseteq K$ is a compact identity neighbourhood, then every $K$-slice is also an $L$-slice.  Furthermore, $S$ is a $K$-slice if and only if $gS \cap S = \emptyset$ for all $g \in K^{-1}K \setminus \{ e \}$.
\end{remark}



\subsection{Properties of slices}
\label{sec:slices-properties}

In this section we collect some basic properties of slices that will be used in the remainder of the article.  We start by considering a suitable notion of the interior of a slice.
\begin{lemma}
  \label{lem:box-interior-independence}
  Let $K$ and $K'$ be compact identity neighborhoods in $G$ such that the set $S \subseteq X$ is both a $K$-slice and a $K'$-slice. Then
  \begin{gather*}
    S \cap (KS)^{\circ} = S \cap (K'S)^{\circ}
    \eqstop
  \end{gather*}
\end{lemma}
\begin{proof}
  Passing to the intersection $K \cap K'$, we may assume that $K \subseteq K'$ and prove that $S \cap (K'S)^{\circ} \subseteq S \cap (KS)^{\circ}$.  Let $x \in S \cap (K'S)^{\circ}$.  Since $(K'S)^{\circ} \subset K'S$ is open and products of open subsets form a basis of the topology of $K' \times S$, there are open subsets $U \subseteq K'$ and $V \subseteq S$ such that $x \in UV \subseteq (K'S)^\circ$.  Since $x \in S$, we find that $e \in U$ and may hence assume without loss of generality that $U \subseteq K$.  Then $x \in UV \subseteq KS$ on the one hand and on the other hand we obtain a inclusions where each term is open in the next one $UV \subseteq (K'S)^\circ \subseteq X$.  So $UV \subseteq X$ is open, which shows that $x \in (KS)^\circ$. 
\end{proof}




\begin{definition}\label{def:box-interior}
  Given a $K$-slice $S$, we define its \emph{tube interior} and its \emph{tube boundary} to be the sets
  \begin{align*}
    \boxint{S} &= S \cap (KS)^{\circ}, \\
    \boxboundary{S} &= S \setminus \boxint{S} ,
  \end{align*}
  respectively.
\end{definition}
By \cref{lem:box-interior-independence} we infer that $\boxint{S}$ and $\boxboundary{S}$ are independent of the compact identity neighborhood $K$.



We will next obtain some basic properties of the tube interior.  For this, the following result is instrumental, which allows to enlarge a tube in the direction of the group.
\begin{lemma}
  \label{lem:extend_slice}
  Assume that $G$ is second-countable. Let $S$ be a $K$-slice for a compact identity neighborhood $K$ in $G$. Then there exists a compact identity neighborhood $L$ with $K \subseteq \mathring{L}$ such that $S$ is an $L$-slice.
\end{lemma}
\begin{proof}
  Since $G$ is second-countable we can multiply $K$ with elements from a descending sequence of compact neighbourhoods of the identity to find a descending sequence $(L_n)_{n \in \NN}$ of compact sets such that $\mathring{L_n} \supseteq K$ for all $n \in \NN$ and $\bigcap_{n \in \NN} L_n = K$.  Suppose for a contradiction that $S$ is not an $L_n$-slice for any $n \in \NN$. By \Cref{rem:slice-disjoint-translates} we then find $g_n \in L_n^{-1}L_n \setminus K^{-1}K$ and $x_n,x_n' \in S$ such that $g_nx_n = x_n'$ for all $n \in \NN$. Passing to convergent subsequences, we may assume that $x_n \to x$, $x_n' \to x'$ with $x,x' \in S$ and
  \begin{gather*}
    g_n \to g \in \Big( \bigcap_{n \in \NN} L_n^{-1}L_n \Big) \setminus (K^{-1}K)^{\circ} = \partial(K^{-1}K) .
  \end{gather*}
  In particular, $g \neq e$. Continuity of the action then implies $gx' = x$, so $g S \cap S \neq \emptyset$. By \Cref{rem:slice-disjoint-translates} this is a contradiction to the fact that $S$ is a $K$-slice.
\end{proof}
%
%
%
\begin{lemma}
  \label{lem:box_interior}
  Let $K$ be a compact identity neighborhood and let $S$ be a $K$-slice. Then the following identities hold.
  \begin{align}
    (KS)^{\circ} & = \mathring{K} (\boxint{S}) \eqcomma
                   \label{eq:box_interior} \\
    \partial_X(KS) & = (\partial_G K)S \cup K(\boxboundary{S}) \eqstop
                     \label{eq:box_boundary}
  \end{align}
  Furthermore, if $B$ is regular open in $S$ and contained in $\boxint{S}$, then
  \begin{align}
    \boxint{\overline{B}} & = B \eqcomma
                            \label{eq:boxint_open} \\
    \boxboundary{\overline{B}} & = \partial_S B \eqcomma
                                 \label{eq:boxboundary_open} \\
    \partial(K\overline{B}) & = (K\overline{B}) \setminus \mathring{K}B = (\partial_G K)\overline{B} \cup K(\partial_S B) \eqstop
                              \label{eq:several-boundary-expressions}
  \end{align}
\end{lemma}
\begin{proof}
  We start by proving \eqref{eq:box_interior} and first show that $\mathring{K}\boxint{S} \subseteq (KS)^{\circ}$.  Let $x \in \mathring{K}\boxint{S}$, say $x=gy$ where $g \in \mathring{K}$ and $y \in \boxint{S}$.  Set $L = K \cap g^{-1}K$ and consider the open set $U = g(LS)^{\circ}$ in $X$.  Then $U \subseteq gLS \subseteq gg^{-1}KS = KS$, so that $U \subseteq (KS)^{\circ}$ follows.  Furthermore, since $L \subseteq K$, we know that $S$ is also an $L$-slice and \cref{lem:box_interior} shows that $y \in \boxint{S} = S \cap (LS)^{\circ}$. Thus $x = gy \in g(LS)^{\circ} = U \subseteq (KS)^{\circ}$.

  Now we show that $(KS)^{\circ} \subseteq \mathring{K}\boxint{S}$.  Let $x \in (KS)^{\circ}$, say $x = gy$ with $g \in K$ and $y \in S$.  Then $g^{-1}K$ is a compact identity neighbourhood and $S$ is a $g^{-1}K$-slice, so by \cref{lem:box_interior} we obtain
  \begin{gather*}
    y
    =
    g^{-1}x \in S \cap g^{-1}(KS)^{\circ}
    =
    S \cap (g^{-1}KS)^{\circ}
    =
    \boxint{S}
    \eqstop
  \end{gather*}
  It remains to argue that $g \in \mathring{K}$.  For this, we apply \cref{lem:extend_slice} to find a compact identity neighborhood $L$ in $G$ with $K \subseteq \mathring{L}$ such that $S$ is an $L$-slice.  By continuity of the group action, the set
  \begin{gather*}
    W = \mathring{L} \cap \{ h \in G \mid hy \in (KS)^{\circ} \}
  \end{gather*}
  is open in $G$ and it is clear that $g \in W$.  We show that $W \subseteq K$.  Indeed, if $h \in W$ then $hy \in (KS)^{\circ}$ so we can write $hy = h'y'$ with $h' \in K$ and $y' \in S$.  Since $h \in L$ and $S$ is an $L$-slice, this forces $h = h'$ and $y = y'$. In particular, we infer that $h \in K$.
  
  We next prove \eqref{eq:box_boundary}.  First observe that
  \begin{gather*}
    \mathring{K}\boxint{S} = \mathring{K}S \cap K\boxint{S}
    \eqstop
  \end{gather*}
  Indeed, the inclusion form left to right is obvious. Conversely, if $x = gy \in \mathring{K}S \cap K\boxint{S}$ for uniquely determined elements $g \in K$ and $y \in S$, it follows from $x \in \mathring{K}S$ that $g \in \mathring{K}$ and from $x \in K\boxint{S}$ that $y \in \boxint{S}$.  Hence $x \in \mathring{K}\boxint{S}$.  

  Applying the equality above and using \eqref{eq:box_interior}, we get
  \begin{align*}
    (\partial_G K)S \cup K(\boxboundary{S})
    & =
      (K\setminus \mathring{K})S \cup K(S\setminus \boxint{S}) \\
    & =
      (KS\setminus \mathring{K}S) \cup (KS\setminus  K\boxint{S})) \\
    & =
      KS\setminus (\mathring{K}S \cap K\boxint{S})) \\
    & =
      KS \setminus \mathring{K}\boxint{S} \\
    & =
      KS \setminus (KS)^{\circ} \\
    & =
      \partial_X (KS)
      \eqstop
  \end{align*}

  Let us next show \eqref{eq:boxint_open}. First we show that $B \subseteq \boxint{\ol{B}}$. Using that $B \subseteq \boxint{S}$ and \eqref{eq:box_interior} we obtain $\mathring{K}B \subseteq \mathring{K}\boxint{S} = (KS)^{\circ}$.  Since $\mathring{K}B$ is open in $KS$ and hence in $(KS)^{\circ}$, it is also open in $X$.  But then $B \subseteq \ol{B} \cap (K\ol{B})^{\circ} = \boxint{\ol{B}}$.

  Now we show that $\boxint{\ol{B}}\subseteq B$. Let $x \in \boxint{\ol{B}}$. Since $(K\ol{B})^{\circ}$ is open in $KS$ and $x \in (K\ol{B})^{\circ}$, we can by definition of tubes find open sets $W \subseteq K$ and $V \subseteq S$ such that $x \in WV \subseteq (K\ol{B})^{\circ}$. The inclusion $WV \subseteq K\ol{B}$ then forces $V \subseteq \ol{B}$, and since $x \in \ol{B} \subseteq S$ we get $x \in V$. Since $x \in V \subseteq \ol{B}$ and $V$ is open in $S$, this means that $x$ lies in the interior of $\ol{B}$ inside $S$.  By assumption the latter equals $B$, so $x \in B$.

  To show \eqref{eq:boxboundary_open}, we use \eqref{eq:boxint_open} and obtain
  \begin{gather*}
    \boxboundary{\ol{B}}
    =
    \ol{B}\setminus\boxint{\ol{B}}
    =
    \ol{B}\setminus B
    =
    \partial_S{B}
    \eqstop
  \end{gather*}

  Let us now show \eqref{eq:several-boundary-expressions}.  Since $\ol{B}$ is a compact subset of $S$, it is a $K$-slice. Hence \eqref{eq:box_boundary} together with \eqref{eq:boxboundary_open} gives that
  \begin{gather*}
    \partial(K\ol{B})
    =
    (\partial_G K)\ol{B} \cup K(\boxboundary{B})
    =
    (\partial_G K)\ol{B} \cup K(\partial_S B)
    \eqstop
  \end{gather*}

  Moreover, the definition of boundary together with \eqref{eq:box_interior} and \eqref{eq:boxboundary_open} gives that
  \begin{gather*}
    \partial(K\ol{B})
    =
    K\ol{B} \setminus (K\ol{B})^{\circ}
    =
    K\ol{B} \setminus K\boxint{\ol{B}}
    =
    K\ol{B} \setminus KB
    \eqstop
  \end{gather*}
  This finishes the proof.
\end{proof}
%
%
%
The next lemma will be frequently used to extract information about the intersection of tubes and slices.
\begin{lemma}
  \label{lem:slice_restriction_injective}
  If $S$ and $S'$ are $K$-slices, then the restriction of the second factor projection $KS \cong K \times S \to S$ to the set $KS \cap S'$ is a homeomorphism onto its image.
\end{lemma}
\begin{proof}
  Since $KS \cap S'$ is compact and $S$ is Hausdorff, it suffices to show that the map is injective. Let $g_1,g_2 \in K$ and $x_1,x_2 \in S$ such that $g_1x_1 , g_2x_2 \in S'$. Suppose that the images of $g_1x_1$ and $g_2x_2$ under the projection $KS \to S$ are equal, that is, $x_1 = x_2$.  We then have that $g_1^{-1}(g_1x_1) = g_2^{-1}(g_2x_2) \in KS'$ and since $S'$ is a $K$-slice, it follows that $g_1^{-1} = g_2^{-1}$. This implies injectivity.
\end{proof}



\subsection{Existence of slices}
\label{sec:slices-existence}

The existence of $G$-slices for actions of $G$ is a classical topic closely related to the early development of the theory of principal bundles and the question about their local triviality.  The ultimate existence result in this direction was obtained by Palais in \cite{palais1961-slices}.  The notion of $K$-slices and thus tubes as used in the present work goes back to the work of Bartels--L{\"u}ck--Reich in which an equivariant version for $\RR$-actions equipped with an additional commuting action of a discrete group was obtained \cite[Lemma 2.11]{bartelsluckreich2008-covers} based on ideas from Palais' work.  For our needs, this approach needs further refinement in order to treat actions of Lie groups and simplifies at the same time since we target free actions and there is no additional action of a discrete group present. The goal of the this subsection will be to show that actions of matrix Lie groups and of connected Lie groups of polynomial growth admit slices in the following sense.
\begin{definition}
  \label{def:admits_slices}
  We say that an action $G \curvearrowright X$ \emph{admits slices} if for every compact identity neighbourhood $K \subseteq G$ and every $x \in X$ there exists a $K$-slice $S \subseteq X$ such that $x \in \boxint{S}$.
\end{definition}



\begin{remark}
Notice that if an action $G \curvearrowright X$ admits slices then it must be free. Indeed, if $gx =x$ for $g \in G$ and $x \in X$, pick some compact identity neighborhood $K \subseteq G$ that contains $g$. If $S$ is a $K$-slice that contains $x$, then the equation $gx = x = ex$ forces $g=e$ by the definition of a $K$-slice.
\end{remark}
%
%
%
The next lemmas provide us with a sufficiently equivariant map from a free $G$-space into a finite dimensional representation, in analogy with Palais' result \cite[Theorem 1.2.7]{palais1961-slices} in the context of proper $G$-spaces.
\begin{lemma}
  \label{lem:large-identity-neighbourhoods}
  Let $K \subseteq G$ be a compact subset.  Then there is a symmetric relatively compact identity neighbourhood $U \subseteq G$ such that $\bigcap_{k \in K} U k$ is an identity neighbourhood.
\end{lemma}
\begin{proof}
  Replacing $K$ by $K \cup K^{-1}$, we may assume that $K$ is symmetric.  Take any symmetric, relatively compact identity neighbourhood $V \subseteq G$ and define $U = VK \cup KV$.  Then $U$ is symmetric and since $K$ is symmetric is follows that
  \begin{gather*}
    \bigcap_{k \in K} U k
    \supseteq
    \bigcap_{k \in K} \left ( \bigcup_{k' \in K} V k' \right) k
    \supseteq V
    \eqstop
  \end{gather*}
  Since $V$ was relatively compact, also $U$ is relatively compact.
\end{proof}



\begin{lemma}
  \label{lem:map-to-vectorspace}
  Let $G \grpaction{} X$ be a free action, $V$ a finite dimensional $G$-representation and $v \in V$ a vector with trivial stabiliser in $G$.  For every $x \in X$, every neighbourhood $A$ of $x$ and every relatively compact identity neighbourhood $U \subseteq G$ there is a function $f \in \contc(A, \RR)$ such that $\int_U f(g^{-1}x) gv \, \rmd g= v$.
\end{lemma}
\begin{proof}
  Consider the linear map $T \colon \contc(A, \RR) \to V$ given by $Tf = \int_U f(g^{-1}x) g v \, \rmd g$.  Since $V$ is finite dimensional and $\im T$ is a vector subspace of $V$, it suffices to show that for all convex neighbourhoods $v \in C$ we have $C \cap \im T \neq \emptyset$.

  Fix a convex neighbourhood $C$ of $v$ as above and let $U_0 \subseteq G$ be an identity neighbourhood satisfying $U_0 v \subseteq C$.  Since $\ol{U}x \subseteq X$ is closed, there is a neighbourhood $B \subseteq A$ of $x$ satisfying $B \cap \ol{U}x \subseteq U_0x$.  By freeness of $G \grpaction{} X$, this implies that for every $g \in U$, the statement $g x \in B$ implies $g \in U_0$.  Let $f \in \contc(A, \RR)$ satisfy $0 \leq f \leq 1$, $f(x) \neq 0$, $\int_U f(g^{-1}x) \, \rmd g = 1$ and $\supp f \subseteq B$. Then by choice of $B$, the map $g \mapsto \mathbb{1}_U(g) f( g^{-1} x)$ is a probability density on $G$ supported in $U_0$.  Since $U_0v \subseteq C$, this implies that $\int_U f(g^{-1}x) gv \, \rmd g \in C$.
\end{proof}
%
%
%
We are now prepared to constructed sufficiently equivariant maps into finite dimensional representations.
\begin{lemma}
  \label{lem:map-to-vectorspace-equivariant}
  Let $G \grpaction{} X$ be a free action  and assume that $X$ is second-countable. Let $V$ be a finite dimensional $G$-representation and $v \in V$ a vector with trivial stabiliser in $G$.  For every relatively compact identity neighbourhood $K \subseteq G$ and for every $x_0 \in X$ there is a neighbourhood $A$ of $x_0$ and a continuous map $F\colon X \to V$ such that $F(x_0) = v$ and $F(k x) = k \vphi(x)$ for all $k \in K$ and all $x \in A$.
\end{lemma}
\begin{proof}
  By \cref{lem:large-identity-neighbourhoods} there is a symmetric, relatively compact identity neighbourhood $U$ in $G$ such that $U_0 = \bigcap_{k \in K} Uk$ is an identity neighbourhood.  Since $G$ acts freely, we have $x_0 \notin (K \ol{U} \setminus U_0)x_0$.  Hence,
  \begin{gather*}
    \bigcap_{\substack{A \text{ compact}\\ \text{neighbourhood of }x_0}} A \cap (K \ol{U} \setminus U_0)A = \emptyset
    \eqstop
  \end{gather*}
  By compactness, there is a neighbourhood $A$ of $x_0$ such that for any $g \in K \ol{U} \setminus U_0$ and every $x \in A$ we have $gx \notin A$.  Fix such $A$.

  By \cref{lem:map-to-vectorspace}, there is $f \in \contc(A, \RR)$ supported in $A$ such that $\int_U f(g^{-1}x_0) gv \, \rmd g = v$.  Put
  \begin{gather*}
    F(x) = \int_U f(g^{-1}x) gv \, \rmd g
    \eqstop
  \end{gather*}
  For $k \in G$, $x \in X$, left-invariance of the Haar measures shows that
  \begin{align*}
    F(kx)
    & = \int_U f(g^{-1}k^{-1}x) gv \, \rmd g \\
    & = \int_{k^{-1}U} f(g^{-1}x) kgv \, \rmd g \\
    & = k F(x) + \int_{k^{-1}U \setminus U} f(g^{-1}x) kgv \, \rmd g - \int_{U \setminus k^{-1}U} f(g^{-1}x) kgv \, \rmd g
    \eqstop
  \end{align*}
  Let us simplify the last expression.  Recall that $\supp f \subseteq A$ and that the width of $A$ is at most $U_0 = \bigcap_{k \in K} Uk$.  We find that for $x \in A$ and $g \in k^{-1}U \setminus U$, we have in particular $g^{-1} \in KU \setminus U_0$ and hence $g^{-1}x \notin A$.  Similarly, for $g \in U \setminus k^{-1}U$ we have $g^{-1} \notin Uk$, so that $g^{-1}x \notin A$ follows as above.  We infer that $F(kx) = kF(x)$ for all $k \in K$ and all $x \in A$.  
\end{proof}
%
%
%
We are now able to prove the existence of slices in the sense of \Cref{def:admits_slices} for free actions of matrix Lie groups, that is, subgroups of $\GL(n,\RR)$ for some $n \in \bN$.
\begin{theorem}
  \label{thm:K-slices-exist-matrix-groups}
  Let $G$ be a matrix Lie group and $G \grpaction{} X$ be a free action and assume that $X$ is second-countable. Then $G \curvearrowright X$ admits slices.
\end{theorem}
\begin{proof}
  Fix an embedding with closed image $G \subseteq \GL(n, \RR)$ and consider the vector $v = 1 \in \rM_n(\RR) = V$.  By \cref{lem:map-to-vectorspace-equivariant}, there is a neighbourhood $A \subseteq X$ of $x$ and a continuous map $F\colon X \to V$ such that $F(x) = v$ and $F(ky) = kF(y)$ for all $k \in K$ and $y \in A$.  Since $G \subseteq \GL(n, \RR)$ is closed, its action on $V$ is proper, so that \cite[Theorem 2.3.2]{palais1961-slices} shows the existence of a $G$-slice $S_0$ at $v$.  Put $S = F^{-1}(S_0) \cap A$.  Then $x \in S$, since $F(x) = v$.  We have to show that $S$ is a $K$-slice and then that that $x \in \boxint{S}$.
  
  Let $(k_1,x_1), (k_2,x_2) \in K \times S$.  If $k_1x_1 = k_2x_2$, then
  \begin{gather*}
    k_1F(x_1) = F(k_1x_1) = F(k_2x_2) = k_2F(x_2)
    \eqcomma
  \end{gather*}
  implying that $k_1 = k_2$, since $S_0$ is a $G$-slice.  So $x_1 = x_2$ follows, proving injectivity of the map $K \times S \to KS$.  Since it is also continuous and $KS$ is compact, it is a homeomorphism, and we can conclude that $S$ is a $K$-slice.

  Now fix a neighbourhood $B \subseteq A$ of $x$ and a symmetric identity neighbourhood $L$ such that $L^2 \subseteq K$ and $L \cdot B \subseteq A$.  We claim that the neighbourhood $F^{-1}(L^\circ S_0) \cap B$ of $x$ is contained in $KS$, which will prove that $x \in \boxint{S}$.  Take $y \in F^{-1}(L^\circ S_0) \cap B$.  There is $k \in L$ satisfying $kF(y) \in S_0$.  Since $k \in K$ and $y \in A$, we find that $F(ky) = k F(y) \in S_0$, that is $ky \in F^{-1}(S_0)$.  As we also have $ky \in LB \subseteq A$, we infer that $ky \in S$.   Using the fact that $L$ is symmetric, we now conclude that $y \in L^{-1}S \subseteq KS$, which finishes the proof.
\end{proof}



\begin{corollary}
  \label{cor:K-slices-exist-polynomial-growth}
  Let $G \curvearrowright X$ be a free action of a connected Lie group of polynomial growth and assume that $X$ is second-countable. Then $G \curvearrowright X$ admits slices.
\end{corollary}
\begin{proof}
  Let $K$ be an identity neighborhood in $G$ and let $x \in X$. By \cite[Theorem 2]{losert2001} $G$ contains a maximal compact subgroup $H$. The quotient group $G/H$ is then a Lie group of polynomial growth containing no nontrivial compact normal subgroups, hence by \cite[Corollary 3.6]{losert2020} $G/H$ is a matrix Lie group. Denote by $\pi \colon G \to G/H$ and $p \colon X \to X/H$ the quotient maps. Since $G/H$ acts freely on $X/H$, we can apply \cref{thm:K-slices-exist-matrix-groups} to get a $\pi(K)$-slice $S' \subseteq G/H$ such that $[x] \in \boxint{S'}$.  Set $T = p^{-1}(S') \subseteq X$.  Then $T$ is $H$-invariant, so we can consider the action $H \curvearrowright T$.  Since $H$ is compact there exists by \cite[Theorem 2.1]{mostow57} an $H$-slice $S \subseteq T$ such that $x \in \boxint{S}$, where the interior is taken in the $H$-space $T$.

  We claim that $S$ is a $K$-slice. To see this, let $g,g' \in K$ and $x,x' \in S$ satisfy $gx=g'x'$. Then $(gH)[x] = (g'H)[x']$, so since $gH,g'H \in \pi(K)$ and $[x],[x'] \in \pi(S) \subseteq \pi(\pi^{-1}(S')) = S'$, we get that $gH = g'H$ and $[x]=[x']$. Let $h_1,h_2 \in H$ be such that $g' = gh_1$ and $x = h_2x'$. Then $gx = gh_1h_2x$, so by freeness of the action $h_1h_2 = e$. But $x = h_2x'$ also implies that $h_2=e$ since $h_2 \in H$ and $x,x' \in S$, so we arrive also at $h_1 = e$. We conclude that $g=g'$. It remains to argue that $x \in \boxint{S}$. Since $S$ is an $H$-slice at $x$ in $T$, there is a subset $U \subseteq HS$ which is open in $T$ and contains $x$. Let $W = U \cap \boxint{T}$.  Then $\mathring{K}W \subset \mathring{K}\boxint{T}$ is relatively open and the latter set is open in $X$. So $\mathring{K}W$ is an open subset of $X$ which contains $x$.
\end{proof}



%%% Local Variables:
%%% mode: latex
%%% TeX-master: "classifiability"
%%% End:

\section{Long covers and tube dimension}
\label{sec:box-dimension}

The notion of tube dimension was introduced for $\mathbb{R}$-actions in \cite[Definition 8.6]{hirshbergszabowinterwu2017}. Here we extend the notion to actions of arbitrary locally compact groups. Recall that the \emph{multiplicity} of a cover $\cU$ of $X$ is the least number $d$ such that the intersection of any $d+1$ elements of $\cU$ is empty.

\begin{definition}
\label{def:tube-dim}
The \emph{tube dimension} of an action $G \grpaction{} X$, denoted by $\tubedim(G \grpaction{} X)$, is the least natural number $d$ such that for all compact subsets $K \subseteq G$ and $Y \subseteq X$ there is a family $\cU$ of open sets of $X$ satisfying the following properties:
\begin{enumerate}
    \item for all $x \in Y$ there is $U \in \cU$ such that $K x \subseteq U$,
    \item every $U \in \cU$ is contained in a tube, and
    \item the multiplicity of $\cU$ is at most $d + 1$.
\end{enumerate}
If no such natural number $d$ exists, then we define $\tubedim(G \grpaction{} X) = \infty$.
\end{definition}

In the present section we will show that for certain group actions there are covers with controlled multiplicity and arbitrary length, as specified by the group action. This is the content of \cref{thm:covering}, which generalizes \cite[Theorem 5.2]{kasprowskiruping17} and follows the same proof outline. As a consequence, we will obtain explicit bounds on the tube dimension of these actions. Recall the number $\rmd(G)$ associated with a group of polynomial growth $G$ from Section~\ref{sec:polynomial-growth}.

\begin{theorem}
  \label{thm:covering}
  Let $G \grpaction{} X$ be an action of a locally compact group of polynomial growth on a locally compact, second-countable Hausdorff space. Suppose that $G \curvearrowright X$ admits slices as in \Cref{def:admits_slices}. Then for every compact identity neighbourhood $K \subseteq G$ there exists an open cover $\cU$ of $X$ with the following properties.
  \begin{enumerate}
  \item The cover consists of open tubes.
  \item The multiplicity of the cover is at most $11^{\rmd(G)} \cdot (\dim X + 1)$.
  \item For every $x \in X$ there exists $U \in \cU$ such that $K \cdot x \subseteq U$.
  \end{enumerate}
In particular
\[ \tubedim(G \curvearrowright X) \leq 11^{\rmd(G)} \cdot (\dim X + 1) - 1 . \]
\end{theorem}
\begin{proof}
  First, note that once we have proved the theorem for $K$, it automatically holds for any compact identity neighbourhood $K' \subseteq K$, hence we may enlarge $K$ to a compact symmetric generating set for $G$. Using \cref{prop:cover_by_translates} with $a=10$, we can find $N \in \bN$ such that $K^{10N}$ can be covered by $11^{\rmd(G)}$ translates of $K^{2N}$. Replacing $K$ with the larger set $K^{2N}$, we now have that $K^5$ can be covered by $11^{\rmd(G)}$ translates of $K$. The first step will be to show the following claim, which is an adaption of \cite[Lemma 4.6]{kasprowskiruping17}.

  \begin{claim}
    \label{claim:locally_finite_cover}
    There is a countable collection $\cS$ of $K^5$-slices such that
    \begin{enumerate}
    \item $\{ (KS)^{\circ} \mid S \in \cS \}$ is a locally finite cover of $X$, and
    \item for every pair $S,S' \in \cS$ there exists $h \in G$ such that $\{ g \in K^3 \mid gS \cap S' \neq \emptyset \} \subseteq h \mathring{K}$.
    \end{enumerate}
  \end{claim}

  \begin{proof}[Proof of Claim~\ref{claim:locally_finite_cover}]
    For every $x \in X$ we can by \ref{cor:K-slices-exist-polynomial-growth} find a $K^5$-slice $S_x'$ such that $x \in S_x' \cap (K^5 S_x')^{\circ}$.  Since $K \subseteq K^5$, each $S_x'$ is also a $K$-slice, hence $x \in (K S_x')^{\circ}$ by \cref{lem:box-interior-independence}. Now consider the cover $\{ (K S_x')^{\circ} \mid x \in X \}$ of $X$. Since $G$ is a locally compact, second countable Hausdorff space, it is paracompact and Lindel{\"o}f.  We can therefore find a countable, locally finite, open refinement for the above cover, i.e. an open cover $\cV = \{ V_i \mid i \in \NN \}$ of $X$ such that every $x \in X$ belongs to only finitely many sets from $\cV$ and such that for every $i \in \NN$ there exists $x(i) \in X$ with $V_i \subseteq (K S_{x(i)}')^{\circ}$.  Since $X$ is locally compact and second countable, every compact subset of $X$ intersects only finitely many elements of $\cV$.  We may hence replace each $V_i$ by $\ol{V_i}^{\circ}$ and assume that it is regular open.  By definition of slices, $K S_{x(i)}'$ is homeomorphic to $K \times S_{x(i)}'$, so we can project the compact set $\overline{V_i} \subseteq K S_{x(i)}'$ to the second coordinate. The resulting  compact set $D_i \subseteq S_{x(i)}'$ is then also a $K$-slice satisfying $V_i \subseteq K D_i$.

    We claim that the cover $\{ (K D_i)^{\circ} \mid n \in \NN \}$ of $X$ is also locally finite.  Indeed, let $i \in \NN$ and suppose $x \in (K D_i)^{\circ}$.  Then there exists an open, precompact neighbourhood $U$ of $x$ such that $U \subseteq KD_i$.  Since $x \in KD_i$ we can write $x = ky$ for some $k \in K$ and $y \in D_i$.  By definition of $D_i$ it follows that there exists $x' \in \ol{V_i}$ such that $x' = k'y$ for some $k' \in K$.  Hence $x' = k'k^{-1}x \in K^2U \cap \ol{V_i}$, in particular $K^2U$ intersects $\ol{V_i}$.  Since $K^2U$ is open, it must intersect the regular open set $V_i$.  Since $K^2\ol{U}$ is compact and $\cV$ is a locally finite cover of $X$, this can happen for at only finitely many $i \in \NN$.  Hence there are only finitely many $i \in \NN$ such that $x \in (KD_i)^{\circ}$.

    For every pair $i,j \in \NN$ such that $D_i \cap K^3 D_j \neq \emptyset$, define the function
    \begin{gather*}
      f_{i,j} \colon D_i \cap K^3 D_j \to K^3
    \end{gather*}
    to be the restriction of the projection $K^3 D_j \to K^3$ to the set $D_i \cap K^3 D_j$.  Then $f_{i,j}$ is continuous by definition of slices, hence we can for every $x \in D_i \cap K^3D_j$ pick an open neighbourhood $U_{i,j,x} \subseteq D_i \cap K^3D_j$ of $x$ such that $f_{i,j}(U_{i,j,x}) \subseteq f_{i,j}(x)K^{\circ}$.  Since $X$ and hence $D_i$ is second countable, we can find an open set $U_{i,j,x}'$ in $D_i$ containing $x$ such that $U_{i,j,x'}' \cap K^3D_j = U_{i,j,x}$.  For every $i \in \NN$ the set $J_i = \{ j \in J \mid D_i \cap K^3D_j \neq \emptyset \}$ is finite, so $U_{i,x} = \bigcap_{j \in J_i} U_{i,j,x}'$ is a finite intersection of open sets, hence open.  Since $D_i$ is compact and $D_i = \bigcup_{x \in D_i} U_{i,x}$, we can find $x_1, \ldots, x_{m_i} \in D_i$ such that $D_i = \bigcup_{k=1}^{m_i} U_{i,x_k}$.  Pick open sets $V_{i,k}$ in $D_i$ with $D_i = \bigcup_{k=1}^{m_i} V_{i,k}$ and $\ol{V_{i,k}} \subseteq U_{i,x_k}$.  Let $S_{i,k}$ denote the closure of $V_{i,k}$ and consider the set
    \begin{gather*}
      \cS = \{ S_{i,k} \mid i \in \NN, 1 \leq k \leq m_i \}
      \eqstop
    \end{gather*}
    Then every element of $\cS$ is a closed subset of $D_i$ for some $i \in \NN$, hence a $K$-slice.
    
    We now argue that $\{ (KS)^{\circ} \mid S \in \cS \}$ is a cover of $X$.  To this end it suffices to show that $(KD_i)^{\circ} = \bigcup_{k = 1}^{m_i} (K V_{i,k})^\circ$ for all $i$.  The inclusion from the right to the left is clear.  To see the other inclusion, let $x \in (K D_i)^{\circ}$ and write $x = ky$ for $k \in K$ and $y \in D_i$.  Since $D_i = \bigcup_{k=1}^{m_i} V_{i,k}$ there is $1 \leq k \leq m_i$ such that $y \in V_{i,k}$.  Further, \cref{lem:box_interior} says that $(KD_i)^{\circ} = \mathring{K} \boxint{D_i}$, so that $k \in \mathring{K}$ follows.  So we find that $x \in \mathring{K}V_{i,k}$, which by the definition of a $K$-slice is open in $KD_i$. We conclude by summarising that $x \in \mathring{K}V_{i,k} \cap (KD_i)^{\circ} \subseteq (KV_{i,k})^{\circ}$.

    We also see that $\{ (KS)^{\circ} \mid S \in \cS \}$ is locally finite since each $(K D_i)^{\circ}$ intersects finitely many sets $KS$ for $S \in \cS$ and the cover $\{ (KD_i)^{\circ} \mid i \in \NN \}$ is locally finite.


    For some $i,j \in \NN$ and $1 \leq r \leq m_i$, $1 \leq s \leq m_j$ put $h = f_{i,j}(x_r)$.  We show that 
    \begin{gather*}
      \{ g \in K^3 \mid gS_{i,r} \cap S_{j,s} \neq \emptyset \} \subseteq h \mathring{K}
    \end{gather*}
    If $g$ is a member of the left-hand side, we can find some $x \in S_{i,r} \cap g S_{j,s} \subseteq U_{i,x_k} \cap gU_{j,x_s} \subseteq D_i \cap K^3 D_{j}$, so $j \in J_i$.  Hence $x \in U_{i,j,x_r}'$.  Since $x \in K^3D_j$ as well, we have that $x \in U_{i,j,x_r}$.  This implies that $g = f_{i,j}(x) \in f_{i,j}(x_r)\mathring{K}$.
  \end{proof}
  
  Let now $\cS = (S_i)_{i \in \NN}$ be as in Claim~\ref{claim:locally_finite_cover}.  By the shrinking lemma \cite[Lemma 41.6]{munkres00} we can find a cover $\{ V_i \mid i \in \NN \}$ of $X$ where each $V_i$ is open and $\ol{V_i} \subseteq (KS_i)^{\circ}$.  Then each $\ol{V_i}$ is a closed subset of $KS_i$, so we can project $\ol{V_i}$ onto $S_i$ to get a new $K$-slice $A_i^0 \subseteq S_i$ satisfying $V_i \subseteq K A_i^0$.  So $\{ (K A_i^0)^{\circ} \mid i \in \NN \}$ is a cover of $X$. Also $A_i^0 \subseteq \ol{V_i} \subseteq (KS_i)^{\circ}$, so that $A_i^0 \subseteq \boxint{S_i}$. Consider the following claim, which is an adaption of \cite[Lemma 5.1]{kasprowskiruping17}.

  \begin{claim}
    \label{claim:dimension_reduction}
    Let $k \in \NN$. If $(A_i)_{i \in \NN}$ is a sequence of compact sets where $A_i \subseteq \boxint{S_i}$ and $\dim A_i \leq k$ for all $i \in \NN$, then there exist regular open sets $B_i \subseteq \boxint{S_i}$ in $S_i$ for each $i \in \NN$ such that
    \begin{enumerate}[label=(\alph*)]
    \item \label{it:dim-red:multiplicity}
      the set $\{ \mathring{K}^5 B_i \mid i \in \NN \}$ has multiplicity at most $11^{\rmd(G)}$, and
    \item \label{it:dim-red:dimension}
      for every $i \in \NN$ the compact set
      \begin{gather*}
        A_i \setminus \bigcup_{j \in \NN} \mathring{K}^3B_j
      \end{gather*}
      has dimension at most $k-1$.
    \end{enumerate}
  \end{claim}
  Before proving the claim we show how it can be used to finish the proof of \cref{thm:covering}.  By \cref{prop:dim_properties} we have that $\dim(A_i^0) \leq \dim(X)$ for each $i \in \NN$.  Applying the claim to $(A_i^0)_{i \in \NN}$, we obtain sets $(B_i^0)_{i \in \NN}$ satisfying the conclusion of Claim~\ref{claim:dimension_reduction} with $k = \dim X$. Set
  \begin{gather*}
    A_i^1 = A_i \setminus \bigcup_{j \in \NN} \mathring{K}^3 B_j
    \eqstop
  \end{gather*}
  Then apply Claim~\ref{claim:dimension_reduction} again to the sequence $(A_i^1)_{i \in \NN}$, obtaining a new sequence of sets $(B_i^1)_{i \in \NN}$.  Continuing like this, we obtain sets $(A_i^k)_{i \in \NN}$ and $(B_i^k)_{i \in \NN}$ for every $k \in \NN$. We will now show that
  \begin{gather*}
    \cU = \{ \mathring{K}^5 B_i^k \mid i \in \NN, 0 \leq k \leq \dim X \}
  \end{gather*}
  satisfies the properties of \cref{thm:covering}.  First, note that since $B_i^k$ is a regular open subset of the $K^5$-slice $S_i$ with $B_i^k \subseteq \boxint{S_i}$, it follows from equations \eqref{eq:box_interior} and \eqref{eq:boxint_open} of \cref{lem:box_interior} that $\mathring{K}^5 B_i^k = (K^5\ol{B_i^k})^{\circ}$ is an open tube. Hence the first assertion of \cref{thm:covering} is established.

  We now claim that
  \begin{gather*}
  \label{eq:covering-family}
    \cU' = \{ \mathring{K}^4 B_i^k \mid i \in \NN, 0 \leq k \leq \dim X \}
  \end{gather*}
  is a cover of $X$.  Indeed, let $x \in X$, so that $x \in K A_i^0$ for some $i \in \NN$, say $x = x' y$ with $x' \in K$ and $y \in A_i^0$.  If $y \in \mathring{K}^3 B_j^0$ for some $j \in \NN$ then $x \in K \mathring{K}^3 B_j^0 = \mathring{K}^4 B_j^0$, so $x$ belongs to an element of $\cU'$.  If no such $j$ exists, then by definition we have $y \in A_i^1$.  Continuing like this, if $y \notin \mathring{K}^3 B_j^k$ for any $j \in \NN$ and $0 \leq k \leq \dim X$, we reach the conclusion that $y \in A_i^{\dim X + 1}$.  However $\dim A_i^{\dim X + 1} \leq -1$ so $A_i^{\dim X + 1} = \emptyset$ which is a contradiction. This shows that $\cU'$ is a cover of $X$.

  Now if $x \in X$, say $x \in \mathring{K}^4 B_i^k$ for some $i \in \NN$ and $0 \leq k \leq \dim X$, then $K x \subseteq K (\mathring{K}^4 B_i^k) = \mathring{K}^5 B_i^k$, which shows that $\cU$ satisfies the second assertion of \cref{thm:covering}. Also, by Claim~\ref{claim:dimension_reduction} the multiplicity of $\cU$, being a union of $\dim X + 1$ sets all of multiplicity at most $11^{\rmd(G)}$, cannot exceed $11^{\rmd(G)} \cdot (\dim X + 1)$.  Hence the last assertion of \cref{thm:covering} is also established.
\end{proof}


As we have just shown, it remains to prove Claim~\ref{claim:dimension_reduction}.
\begin{proof}[Proof of Claim~\ref{claim:dimension_reduction}]
  Let $(A_i)_{i \in \NN}$ be as in the statement of the claim.  We will divide the proof into five steps.

  \vspace{0.5em}

  \noindent \textbf{Step 1}.
  In the first step we will construct the sets $(B_i)_{i \in \NN}$.  For this, fix $i \in \NN$.  We apply the \cref{def:small_inductive_dimension} of small inductive dimension to the set $A_i$ viewed as a subset of the ambient space $S_i$ to obtain for every $x \in A_i$ an open neighbourhood $U_x$ of $x$ in $S_i$ which satisfies $\dim(\partial_{S_i}U_x) \leq k-1$.  As explained in \cref{rmk:inductive_dimension_open_regular}, we may assume that the sets $U_x$ are regular open.

  Since $A_i$ is compact we can find a finite subset $F_i\subseteq A_i$ such that $U_i = \bigcup_{x\in F_i}U_x$ contains $A_i$.  Note that $U_i$ is regular open in $S_i$.  Since $X$ is separable and metrizable, every subset is also separable and metrizable, in particular $\partial_{S_i}U_x$ is so for each $x \in F_i$.  Hence we can apply \cref{prop:dim_properties} to obtain
  \begin{equation}
    \label{eq:dimension_red_def}
    \dim(\partial_{S_i}U_i)
    \leq
    \dim \Big( \bigcup_{x \in F_i} \partial_{S_i}U_x \Big)
    \leq
    k-1. 
  \end{equation}
  We define subsets $(I_i)_{i \in \NN}$ of $\NN$ and sets $(B_i)_{i \in \NN}$ in $X$ recursively.  Set $I_1 = \emptyset$ and $B_1 = U_1$.  Further, for $i \geq 2$, set $I_{i} = \{ j \in \NN \mid j < i, K^2 \ol{B_j} \cap U_i \neq \emptyset \}$ and
  \begin{gather*}
    B_i = U_i \setminus \bigcup_{j \in I_i}K^3\ol{B_j}
    \eqstop
  \end{gather*}
  It is then clear from the construction that $B_i$ is an open subset of $S_i$ for each $i \in \NN$.  We will prove that each $B_i$ is regular open by induction.  For $i = 1$ we have that $B_1 = U_1$ which is regular open by construction.  Next, assume that $B_j$ is regular open for all $j < i$. Note first that
  \begin{gather*}
    \ol{B_i}
    \subseteq
    \ol{U_i} \setminus \Big( \bigcup_{j\in I_i} K^3\ol{B_j} \Big)^{\circ}
    \subseteq
    \ol{U_i} \setminus \bigcup_{j\in I_i} (K^3\ol{B_j})^{\circ}
    \eqstop
  \end{gather*}
  Now by the induction assumption combined with equations \eqref{eq:box_interior} and \eqref{eq:boxint_open} of \cref{lem:box_interior} we have that $(K^3\ol{B_j})^{\circ} = \mathring{K}^3\boxint{\ol{B_j}} = \mathring{K}^3 B_j$ for all $j < i$.  Since $S_i$ is a $K^3$-slice, we have $\ol{WB} = \ol{W}\, \ol{B}$ for all subsets $W \subseteq K^3$ and $B \subseteq S_i$.  So we get that $\ol{(K^3\ol{B_j})^{\circ}} = K^3\ol{B_j}$.  Using this and the fact that $U_i$ is regular open, we find
  \begin{gather*}
    \mathring{\ol{B_i}}
    =
    \Big( \ol{U_i}\setminus\bigcup_{j\in I_i} (K^3\ol{B_j} )^{\circ} \Big)^{\circ}
    \subseteq
    \mathring{\ol{U_i}} \setminus \ol{\bigcup_{j\in I_i}  (K^3\ol{B_j})^{\circ}}
    =
    U_i \setminus \bigcup_{j\in I_i} \ol{ (K^3\ol{B_j})^{\circ} }
    =
    U_i \setminus \bigcup_{j\in I_i} \mathring{K}^3B_j
    =
    B_i
    \eqstop
  \end{gather*}
  Since $B_i \subseteq \mathring{\ol{B_i}}$ is obvious, this proves that $B_i$ is regular open, finishing the induction argument.

  \vspace{0.5em}

  \noindent \textbf{Step 2}.
  In this step we show assertion \ref{it:dim-red:multiplicity} of Claim~\ref{claim:dimension_reduction}.  Note that by construction of $B_i$, the elements of the set $\{KB_i\}_{i\in\NN}$ are pairwise disjoint.  Indeed, suppose for a contradiction that $x \in KB_i \cap KB_j$ with $j < i$, say $x = g_iy_i = g_jy_j$ with $g_i,g_j \in K$, $y_i \in B_i$ and $y_j \in B_j$.  Then $y_i = g_i^{-1}x \in U_i \cap K(KB_j)$, hence $j \in I_i$.  It then follows from the definition of $B_i$ that $y_i \notin K^3 \ol{B_j}$ which contradicts $y_j = g_j^{-1}g_iy_i \in K^2B_j$.

  Consider now the set $\{K^5 B_j\}_{j\in\NN}$ from assertion \ref{it:dim-red:multiplicity} of the claim.  By choice of $K$, its power $K^5$ can be covered by $\ell = 11^{\rmd(G)}$ translates of $K$, say $K^5 \subseteq \bigcup_{n=1}^\ell g_n K$ for some $g_1, \dotsc, g_\ell \in G$.  Suppose for a contradiction that there exist indices $i_1 < \cdots < i_{\ell+1}$ such that there is $x \in K^5B_{i_1} \cap \cdots \cap K^5B_{i_{\ell+1}}$.  Then for each $1 \leq r \leq \ell + 1$ there exists $1 \leq n_r \leq \ell$ such that $g_{n_r}^{-1}x \in KB_{i_r}$.  Since $\{ n_1, \dotsc, n_{\ell+1} \} \subseteq \{ 1 , \dotsc, \ell \}$, there exist $1 \leq r,r' \leq \ell + 1$ such that $n_r = n_{r'}$.  This gives $g_{n_r}^{-1}x = g_{n_{r'}}^{-1}x \in KB_{i_r} \cap KB_{i_{r'}}$, which is a contradiction.  Hence the set $\{ K^5 B_j \}_{j \in \NN}$ has multiplicity at most $\ell$.
  
  \vspace{0.5em}

  \noindent \textbf{Step 3}.
  In this step we will show that the sets $S_j\cap (\partial K^3) \ol{B}_k$ are empty when $j \in \NN$ and $k \in I_j$.

  Suppose for a contradiction that there is some $x \in S_j \cap (\partial K^3) \ol{B}_k$, say $x = gy$ for $g \in \partial K^3$ and $y \in \ol{B_k}$.  Since $k \in I_j$, there exists some $x' \in K^2\ol{B_k} \cap U_j \subseteq K^2\ol{B_k}\cap S_j$, say $x' = g'y'$ with $g' \in K^2$ and $y' \in \ol{B_k}$.  We now have that $g,g' \in K^3$, $S_j \cap g S_k \neq \emptyset$ and $S_j \cap g' S_k \neq \emptyset$, so by Claim~\ref{claim:locally_finite_cover} we obtain that $g'^{-1}g \in \mathring{K}$.  Hence
  \begin{gather*}
    g= g'(g'^{-1}g) \in K^2\mathring{K} = \mathring{K}^3
    \eqcomma
  \end{gather*}
  which contradicts $g \in \partial K^3$.

  \vspace{0.5em}

  \noindent \textbf{Step 4}.
  In this step we prove that showing \ref{it:dim-red:dimension} of the present claim can be reduced to showing that $\partial_{S_j} B_j$ has dimension at most $k-1$ for all $j \in \NN$.  Fix $i \in \NN$ and note that
  \begin{gather*}
    \bigcup_{j \in \NN} \mathring{K}^3B_j
    \supseteq
    \bigcup_{j \leq i} \mathring{K}^3B_j
    =
    \mathring{K}^3\Big( U_i \setminus \bigcup_{j \in I_i} K^3 \ol{B_j} \Big) \cup \bigcup_{j < i} \mathring{K}^3B_j
    \supseteq
    U_i \setminus \bigcup_{j \in I_i} K^3 \ol{B_j} \cup \bigcup_{j < i} \mathring{K}^3B_j
    \eqstop
  \end{gather*}
  Hence, taking complements in $A_i$ and using the fact that $A_i \subseteq U_i$, we obtain
  \begin{align*}
    A_i \setminus \bigcup_{j\in\NN}\mathring{K}^3 B_j
    & \subseteq
      \Big( A_i \setminus  \Big( U_i \setminus \bigcup_{j \in I_i} K^3 \ol{B_j} \Big) \Big) \setminus \bigcup_{j \in I_i} \mathring{K}^3B_j \\
    & =
      \Big( A_i \setminus U_i \cup A_i \cap \bigcup_{j<i} K^3 \ol{B_j} \Big) \setminus \bigcup_{j \in I_i} \mathring{K}^3B_j \\
    & =
      A_i \cap \Big( \bigcup_{j \in I_i} K^3 \ol{B_j} \setminus \bigcup_{j \in I_i} \mathring{K}^3 B_j \Big) \\
    & \subseteq
      \bigcup_{j \in I_i} A_i \cap (K^3\ol{B_j} \setminus \mathring{K}^3 B_j)
  \end{align*}
  Thus, in order to show that $A_i \setminus \bigcup_{j \in \NN} \mathring{K}^3 B_j$ has dimension at most $k-1$, it suffices by \cref{prop:dim_properties} to show that the sets
  \begin{gather*}
    % \label{eq:sets_dimension_red}
    A_i \cap (K^3\ol{B_j} \setminus \mathring{K}^3B_j)\eqcomma \qquad j \in I_i\eqcomma
  \end{gather*}
  have dimension at most $k-1$.  By Step 3, the set $A_i \cap (\partial K^3)\ol{B_j}$ is empty.  Therefore, using \eqref{eq:several-boundary-expressions} of \cref{lem:box_interior}, we have that
  \begin{gather*}
    A_i \cap (K^3\ol{B_j} \setminus \mathring{K}^3B_j)
    =
    (A_i \cap (\partial K^3)\ol{B_j} ) \cup (A_i \cap K^3(\partial_{S_j}B_j) )
    =
    A_i \cap K^3(\partial_{S_j}B_j)
    \eqstop
  \end{gather*}
  For each $i,j \in \NN$, we consider the projection $K^3S_j \to S_j$, which maps $K^3(\partial_{S_j}B_j)$ to $\partial_{S_j}B_j$. The restriction of this projection to $A_i \cap K^3(\partial_{S_j}B_j)$ is a homeomorphism onto its image by \cref{lem:slice_restriction_injective}, so it suffices to show that the dimension of $\partial_{S_j}B_j$ is at most $k-1$.

  \vspace{0.5em}

  \noindent \textbf{Step 5}.
  In this step we finish the proof of Claim~\ref{claim:dimension_reduction} by showing that $\partial_{S_i}B_i$ has dimension at most $k-1$ for all $i \in \NN$. We establish this by induction. Consider first the base case $i=1$. Then $\partial_{S_1}B_1 = \partial_{S_1}U_1$ which has dimension at most $k-1$ by inequality \eqref{eq:dimension_red_def}.

  For the induction step, let $i \in \NN$ and assume that $\dim(\partial_{S_j}B_j) \leq k-1$ for all $j < i$.  First we estimate $\partial_{S_i}B_i$ using \Cref{lem:box_interior} to obtain
  \begin{align*}
    \partial_{S_i}B_i
    & \subseteq
      \partial_{S_i}U_i\cup\bigcup_{j \in I_i}\partial_{S_i}(S_i\cap\mathring{K}^3B_i) \\
    & =
      \partial_{S_i}U_i\cup\bigcup_{j \in I_i} S_i \cap\partial(\mathring{K}^3B_j)\\
    & =
      \partial_{S_i}U_i\cup\bigcup_{j \in I_i} (S_i \cap (\partial K^3)\ol{B_j} ) \cup (S_i \cap K^3(\partial_{S_j}B_j))
      \eqstop
  \end{align*}
  From \eqref{eq:dimension_red_def} we know that $\partial_{S_i}U_i$ has dimension at most $k-1$.  Further, by Step 3, the sets $S_i \cap (\partial K^3)\ol{B_j}$ for $j \in I_i$ are empty.  Applying \cref{prop:dim_properties}, we need only show that $S_i \cap K^3(\partial_{S_j}B_j)$.  But for this we can again consider the projection of the tube $K^3(\partial_{S_j}B_j) \to \partial_{S_j}B_j$ which, once restricted to $S_i \cap K^3(\partial_{S_j}B_j)$, becomes a homeomorphism onto its image by \cref{lem:slice_restriction_injective}.  Since the image is a subset of $\partial_{S_j}B_j$ which has dimension at most $k-1$ by the induction hypothesis, we can appeal to \cref{prop:dim_properties} to finish the proof.
\end{proof}


%%% Local Variables:
%%% mode: latex
%%% TeX-master: "classifiability"
%%% End:

\section{Partitions of unity}
\label{sec:partitions}


Our next aim is to provide a suitable adaption and generalisation of \cite[Proposition 8.23]{hirshbergszabowinterwu2017}.  The notion of Lipschitz functions, which was used in the context of $\mathbb{R}$-actions, is not suitable for general amenable locally compact groups and we resort to the following notion of F{\o}lner functions instead.  As it turns out, this concept is still suitable to characterise finite tube dimension and can be used for nuclear dimension estimates.
\begin{definition}
  \label{def:folner-function}
  Let $G \curvearrowright X$ be an action and $(Y, \mathrm{d})$ a metric space. Let $K \subseteq G$ be a compact identity neighborhood and let $\epsilon > 0$. Then a function $\Phi: X \to Y$ is called $(K, \varepsilon)$-F{\o}lner if for every $g \in K$ and every $x \in X$ we have
  \begin{gather*}
    \mathrm{d}(\Phi(gx), \Phi(x)) < \varepsilon
    \eqstop
  \end{gather*}
  A function $\varphi: X \to \mathbb{C}$ is called $(K, \varepsilon)$-F{\o}lner if it is $(K, \varepsilon)$-F{\o}lner for the Euclidean metric on $\bC$.
\end{definition}



\begin{remark}
  This notion is not to be confused with the F{\o}lner function of a group.
\end{remark}



\begin{lemma}
  \label{lem:creating-folner-functions}
  Let $G$ be an amenable locally compact group and let $G \grpaction{} X$ be a continuous action by homeomorphisms.  Let $\varphi\colon X \to \CC$ be a continuous bounded function, let $K \subseteq G$ be compact and $\veps > 0$.  If $B \subseteq G$ is a $(K, \varepsilon/\|\phi\|_\infty)$-F{\o}lner set, then
  \begin{gather*}
    \psi(x) = \frac{1}{\haar(B)}\int_B \vphi(g^{-1}x) \rmd \haar(g)
  \end{gather*}
  is $(K, \veps)$-F{\o}lner and continuous.
\end{lemma}
\begin{proof}
  It is clear that $\psi$ is continuous, since $\varphi$ is continuous.  We have to show that it is $(K, \varepsilon)$-F{\o}lner.  For $g \in K$ we find that
  \begin{align*}
    \|g\psi - \psi\|_\infty
    & =
      \|g (\frac{1}{\mathrm{m}(B)} \mathbb{1}_B * \varphi) -  \frac{1}{\mathrm{m}(B)} \mathbb{1}_B * \varphi\|_\infty \\
    & =
      \|(g \frac{1}{\mathrm{m}(B)} \mathbb{1}_B) * \varphi -  \frac{1}{\mathrm{m}(B)} \mathbb{1}_B * \varphi\|_\infty \\
    & \leq 
      \|(g \frac{1}{\mathrm{m}(B)} \mathbb{1}_B) -  \frac{1}{\mathrm{m}(B)} \mathbb{1}_B\|_1 * \|\varphi\|_\infty    \\
    & \leq
      \frac{\varepsilon}{\|\varphi\|_\infty} \| \varphi\|_\infty\\
    & =
      \varepsilon
      \eqstop
  \end{align*}
\end{proof}




The next proposition is an adaption of \cite{hirshbergszabowinterwu2017}, which was formulated for $\mathbb{R}$-actions.  In the general setup of amenable groups, it is not possible any longer to show the existence of Lipschitz partitions of unity, however a slight adaption of the proof makes it possible to replace these by suitable F{\o}lner partitions of unity, which suffice for the purpose of proving nuclear dimension estimates.  There is only a conceptual innovation, but no substantial technical change over the proof of \cite[Proposition 8.23]{hirshbergszabowinterwu2017}.

In order to formulate our proposition, we briefly recall combinatorial simplicial complexes and their $\ell^1$-metric realisation.
\begin{itemize}
\item A (combinatorial) simplicial complex is a set $Z$ of finite sets closed under taking subsets.  An $l$-simplex in $Z$ is a element $\sigma \in Z$ with cardinality $|\sigma| = l + 1$ and we denote by $Z^{(l)}$ the set of all $l$-simplices of $Z$. The elements of $Z^{(0)}$ are called the \emph{vertices} of $Z$. The \emph{dimension} of $Z$ is the highest number $d$ such that $Z^{(d-1)} \neq \emptyset$ (if no such number exists, the dimension of $Z$ is infinite).
\item Given a simplicial complex $Z$, we denote by $\mathrm{C} Z = \{\sigma \in Z \sqcup \{\infty\} \mid \sigma \cap Z^{(0)} \in Z\}$ the cone over $Z$.
\item The $\ell^1$-metric realisation of a simplicial complex $Z$ is
  \begin{gather*}
    |Z| = \bigcup_{\sigma \in Z} \{(z_v)_v \in [0,1]^{Z^{(0)}} \mid \sum_{v \in \sigma} z_v = 1 \text{ and } z_v = 0 \text{ for any } v \in Z_0 \setminus \sigma \}
  \end{gather*}
  endowed with the restriction of the $\ell^1$-metric on finitely supported functions on $Z^{(0)}$.
\item For any vertex $v_0\in Z^{(0)}$, the \emph{simplicial star} around $v$ is the set of simplices that contain $v_0$, and the \emph{open star} around $v_0$ is the subset $\{ (z_v)_v \in |Z| : z_{v_0} > 0 \}$ of the geometric realization of $Z$.
\item If $\mathcal{U}$ is a finite open cover of $X$, its \emph{nerve complex} is the simplicial complex
\[ \mathcal{N}(\mathcal{U}) = \{ \mathcal{U}' \subseteq \mathcal{U} : \bigcap_{U \in \mathcal{U}'} U \neq \emptyset \} . \]
The dimension of $\mathcal{N}(\mathcal{U})$ is equal to $\mathrm{mult}(\mathcal{U}) - 1$.
\item If $\mathcal{U}$ is a finite open cover of a closed subset $C$ of $X$, we denote by $\mathcal{U}^+$ the open cover of $X$ given by $\mathcal{U}^+ = \mathcal{U} \cup \{ X \setminus C \}$. One can then identify $\mathcal{N}(\mathcal{U}^+)$ as a subset of $\mathrm{C}\mathcal{N}(\mathcal{U})$ via sending $X \setminus C$ to $\infty$.
\end{itemize}

We are now ready to formulate our adaption and generalisation of \cite[Proposition 8.23]{hirshbergszabowinterwu2017}.
\begin{proposition}
  \label{prop:characterisation-tube-dimension}
  Let $G$ be an amenable locally compact second countable group and let $G \curvearrowright X$ be a continuous action by homeomorphisms.  Then the following statements are equivalent.
  \begin{enumerate}
  \item \label{it:tube-dim}
    The tube dimension of $G \curvearrowright X$ is at most $d$.

  \item \label{it:map-into-cone}
    For every compact subset $K \subseteq G$, every compact subset $A \subseteq X$ and every $\varepsilon > 0$ there is a finite simplicial complex $Z$ of dimension at most $d$ and a continous map $\Phi: X \to |\mathrm{C} Z|$ such that
    \begin{enumerate}
    \item $\Phi$ is $(K, \varepsilon)$-F{\o}lner,
    \item for every vertex $v \in Z_0$ the preimage of the open star around $v$ under $\Phi$ is contained in a tube, and
    \item $\Phi(A) \subseteq |Z|$.
    \end{enumerate}
    
  \item \label{it:partition-plain}
    For every compact subset $K \subseteq G$, every $\varepsilon > 0$ and every compact subset $A \subseteq X$ there is a finite partition of unity $(\varphi_i)_{i \in I}$ for $A \subseteq X$ such that
    \begin{enumerate}
    \item $\varphi_i$ is $(K,\varepsilon)$-F{\o}lner for all $i$,
    \item $\varphi_i$ is supported in the interior of a tube, and
    \item there is a partition $I = I^{(0)} \sqcup \hdots \sqcup I^{(d)}$ such that for all $l \in \{0, \dotsc, d\}$ and all $i,j \in I^{(l)}$ we have
      \begin{gather*}
        \supp \varphi_i \cap \supp \varphi_j = \emptyset
        \eqstop
      \end{gather*}
    \end{enumerate}

  \item \label{it:partition-fattened}
    For every pair of compact subsets $K, L \subseteq G$, every $\varepsilon > 0$ and every compact subset $A \subseteq X$ there is a finite partition of unity $(\varphi_i)_{i \in I}$ for $A \subseteq X$ such that
    \begin{enumerate}
    \item $\varphi_i$ is $(K,\varepsilon)$-F{\o}lner for all $i$,
    \item $L\supp \varphi_i$ is contained in the interior of a tube, and
    \item there is a partition $I = I^{(0)} \sqcup \hdots \sqcup I^{(d)}$ such that for all $l \in \{0, \dotsc, d\}$ and all $i,j \in I^{(l)}$ we have
      \begin{gather*}
        L\supp \varphi_i \cap L \supp \varphi_j = \emptyset
        \eqstop
      \end{gather*}
    \end{enumerate}

  \item \label{it:cover-fattened}
    For every compact subset $L \subseteq G$ and every compact subset $A \subseteq X$ there is a finite collection $\mathcal{U}$ of open subsets of $X$ that cover $A$ such that
    \begin{enumerate}
    \item each $U \in \mathcal{U}$ is contained in a tube of shape larger than $L$, and
    \item there is a partition $\mathcal{U} = \mathcal{U}^{(0)} \sqcup \dotsm \sqcup \mathcal{U}^{(d)}$ such that for any $l \in \{0, \dotsc, d\}$ and any distinct elements $U_1, U_2 \in \mathcal{U}^{(l)}$ we have
      \begin{gather*}
        L\overline{U_1} \cap L\overline{U_2} = \emptyset
        \eqstop
      \end{gather*}
    \end{enumerate}
  \end{enumerate}
\end{proposition}
\begin{proof}
  We start by proving \ref{it:tube-dim} $\Rightarrow$ \ref{it:map-into-cone}.  Fix the notation of \ref{it:map-into-cone}, let $0 < C$ be such that $3(d+1)C < \varepsilon$ and let $L \subseteq G$ be an $(K,C)$-F{\o}lner set.  Consider $B = LA$.  Then by the definition of tube dimension there is a family of open subsets $\mathcal{U}$ of $X$, each contained in a tube, having multiplicity at most $d+1$ such that for every $x \in LB$ there is $U \in \mathcal{U}$ such that $Lx \subseteq U$. Consider the sets
  \begin{gather*}
    V_U = \{x \in X \mid Lx \subseteq U\} \quad U \in \mathcal{U}
    \text{,}
  \end{gather*}
  and the collection $\mathcal{V} = \{V_U \mid U \in \mathcal{U}\}$. Then $\mathcal{V}$ covers $B$ and we can find a finite subset $\mathcal{U}_0 \subseteq \mathcal{U}$ such that $\mathcal{V}_0 = \{V_U \mid U \in \mathcal{U}_0\}$ remains a cover of $B$.  Let $(\varphi_U)_{U \in \, \mathcal{U}_0}$ be a partition of unity for $B \subseteq X$ subordinate to $\mathcal{V}_0$.  By \cref{lem:creating-folner-functions}, the functions $\psi_U = \frac{1}{\mathrm{m}(L)} \mathbb{1}_L * \varphi_U$ for $U \in \mathcal{U}_0$ are $(K, C)$-F{\o}lner. Note also that $\supp(\psi_U) \subseteq \supp(\mathbb{1}_L)\supp(\phi_U) \subseteq LV_U \subseteq U$ for each $U \in \mathcal{U}_0$. Since $\mathcal{U}_0$ has multiplicity at most $(d + 1)$, at most $2(d+1)$ functions from the set $\{g\psi_U \mid U \in \mathcal{U}_0\} \cup \{\psi_U \mid U \in \mathcal{U}_0\}$ can be non-zero at a time.  So for $g \in K$, we find that
  \begin{gather*}
    \|g(\mathbb{1}_X - \sum_{U \in \mathcal{U}_0} \psi_U) - (\mathbb{1}_X - \sum_{U \in \mathcal{U}_0} \psi_U)\|_\infty
    =
    \|\sum_{U \in \mathcal{U}_0} g\psi_U - \sum_{U \in \mathcal{U}_0} \psi_U \|_\infty
    \leq
    2(d + 1)C
    \text{,}
  \end{gather*} 
which shows that $\mathbb{1}_X - \sum_{U \in \mathcal{U}_0} \psi_U$ is $(K, 2(d+1)C)$-F{\o}lner.

Let $Z = \mathcal{N}(\mathcal{U}_0)$ be the nerve of $\mathcal{U}_0$, which is a simplicial complex of dimension at most $\mathrm{mult}(\mathcal{U}_0) = (d+1)-1 = d$.  Consider the continuous map
\begin{gather*}
  \Phi\colon X \to |\cN(\cU_0^+)| \subseteq |\rC Z|
\end{gather*}
given by $\Phi = (\mathbb{1}_X - \sum_{U \in \, \cU_0} \psi_U) \oplus \bigoplus_{U \in \, \cU_0} \psi_U$ and observe that $\Phi$ is $(K, 3(d+1)C)$-F{\o}lner.  By the choice of $C$, this shows that $\Phi$ is $(K, \varepsilon)$-F{\o}lner.  It maps $A$ into $|\mathcal{N}(\mathcal{U}_0)|$, since $\mathbb{1}_X - \sum_{U \in \mathcal{U}_0} \psi_U$ vanishes on $A$.  Also the preimage of every open star is contained in some $U \in \mathcal{U}_0$, which in turn is contained in a tube.

Next we prove \ref{it:map-into-cone} $\Rightarrow$ \ref{it:partition-plain}. Let $K \subseteq G$ be a compact subset, let $A \subseteq X$ be compact and let $\varepsilon > 0$.  Let $\Phi: X \to |CZ|$ be chosen as \ref{it:map-into-cone} for $K \subseteq G$ and the constant $\frac{\varepsilon}{2 (d+1)(d+2)(2d+3)}$.  Let $I^{(l)}$ be the collection of $l$-simplices of $Z$ for $l \in \{0,\dotsc, d\}$, and let $I = I^{(0)} \cup \cdots \cup I^{(d)}$, which is a disjoint union. Consider the functions $\nu_\sigma: |\mathrm{C} Z| \to [0,1]$ for $\sigma \in Z$ as in \cite[Lemma 8.18]{hirshbergszabowinterwu2017}: They are $2(d+1)(d+2)(2d+3)$-Lipschitz and form a partition of unity for $|Z| \subseteq |\mathrm{C} Z|$ which is subordinate to the cover $\{ V_\sigma : \sigma \in Z \}$ where
\begin{gather*}
  V_\sigma
  = \left \{ (z_v)_v \in |Z| \mid z_v > z_{v'} \text{ for all } v \in \sigma \text{ and } v' \in Z_0\setminus \sigma \right \}
  \eqstop
\end{gather*}
Put
\begin{gather*}
  \varphi_\sigma = \nu_\sigma \circ \Phi: X \to [0,1]
  \eqstop
\end{gather*}
We get that $\sum_{\sigma \in I} \varphi_\sigma(x) = 1$ for $x \in A$ since $\Phi(A) \subseteq |Z|$ and $\sum_{\sigma} \nu_\sigma (z) = 1$ for $z \in |Z|$.  Since $\Phi$ is $(K, \frac{\varepsilon}{2 (d+1)(d+2)(2d+3)})$-F{\o}lner and $\nu_\sigma$ is $2(d+1)(d+2)(2d+3)$-Lipschitz, we find that for $g \in K$ we have
\begin{align*}
  |\nu_\sigma \circ \Phi(g x) - \nu_\sigma \circ \Phi(x)|
  & \leq
    2(d+1)(d+2)(2d+3) \mathrm{d}(\Phi(gx), \Phi(x)) \\
  & \leq
    2(d+1)(d+2)(2d+3) \frac{\varepsilon}{2(d+1)(d+2)(2d+3)} \\
  & =
    \varepsilon
    \text{,}
\end{align*}
so that $\varphi_\sigma$ is an $(K, \varepsilon)$-F{\o}lner function.  The remaining properties of $(\varphi_\sigma)_\sigma$ follow as in \cite{hirshbergszabowinterwu2017}: since the support of $\nu_\sigma$ is contained in an open star, the support of $\varphi_\sigma$ is contained in the interior of a tube, and the orthogonality condition on $\nu_\sigma$ implies the one on $\varphi_\sigma$.

We proceed to prove \ref{it:partition-plain} $\Rightarrow$ \ref{it:partition-fattened}. Let $K, L \subseteq G$ be compact, let $\varepsilon > 0$ and $A \subseteq X$ be compact.  Without loss of generality, we may enlarge $K$ and assume that $L \subseteq K$.

Choose constants $\delta_1, \delta_2 > 0$ such that
\begin{gather*}
  \delta_2 < \delta_1\text{,} \quad
  \frac{2(d+1)\delta_1}{(1 - (d+1)\delta_1)^2} < \frac{\varepsilon}{2} \quad \text{ and } \quad
  \frac{\delta_1 + \delta_2}{1 - (d+1)\delta_1} < \frac{\varepsilon}{2}
  \eqstop
\end{gather*}
By \ref{it:partition-plain} there is a partition of unity $(\psi_i)_{i \in I}$ for $A \subseteq X$ that consist of $(K, \delta_2)$-F{\o}lner functions, which are supported in the interior of a tube and satisfy the disjointness condition on their supports as in \ref{it:partition-plain}.  Put $\psi'_i = (\psi_i - \delta_1)_+: X \to [0, 1 - \delta_1]$.  Then $\psi'_i$ is $(K, \delta_1 + \delta_2)$-F{\o}lner. Further, the F{\o}lner condition for $\psi_i$ and the fact that $L \subseteq K$ imply that
\begin{gather*}
  L\supp \psi'_i
  \subseteq
  \{x \in X \mid \max_{g \in L} \psi_i(g^{-1}x) \geq \delta_1\}
  \subseteq
  \{x \in X \mid \psi_i(x) \geq \delta_1 - \delta_2\}
  \subseteq
  \supp \psi_i
  \eqstop
\end{gather*}
So $L \supp \psi'_i$ is contained in the interior of a tube.

Put 
\begin{gather*}
  \psi' = \left (\sum_{i \in I} \psi'_i \right ) + \left (\mathbb{1}_X - \sum_{i \in I} \psi_i \right )
  \eqstop
\end{gather*}
Then $\mathbb{1}_X - \psi' = \sum_{i \in I} \psi_i - \psi'_i$ and the image of the latter is contained in the interval $[0, (d+1)\delta_1]$, owing to the fact that $(\psi_i)_{i \in I}$ is the union of $d+1$ orthogonal families of functions. We note for later use that $\psi'$ is bounded from below by $1 - (d+1)\delta_1$. Put
\begin{gather*}
  \varphi_i = \frac{\psi'_i}{\psi'}
  \eqstop
\end{gather*}
Then $(\varphi_i)_{i \in I}$ is a partition of unity for $A \subseteq X$, since $(\sum_i \psi_i)(x) = 1$ for all $x\in A$. Further, $L\supp \varphi_i = L\supp \psi'_i$ is contained in the interior of a tube.  In order see that each $\varphi_i$ is a $(K, \varepsilon)$-F{\o}lner function, we calculate for $g \in K$ and $x \in A$ that
\begin{align*}
  |\varphi_i(gx) - \varphi_i(x)|
  & =
    \left |\frac{\psi'_i(gx)}{\psi'(gx)} - \frac{\psi_i(x)}{\psi'(x)} \right | \\
  & \leq
    \left |\frac{\psi'_i(gx) - \psi'_i(x)}{\psi'(gx)} \right | + \left | \frac{\psi'_i(x)}{\psi'(gx)} - \frac{\psi'_i(x)}{\psi'(x)} \right | \\
  & \leq
    \left |\frac{\psi'_i(gx) - \psi'_i(x)}{\psi'(gx)} \right | + |\psi'_i(x)| \left | \frac{\psi'(x) - \psi'(gx)}{\psi'(gx) \psi'(x)} \right | \\
  & \leq
    \frac{\delta_1 + \delta_2}{1 - (d+1)\delta_1} + 1 \frac{2(d+1)\delta_1}{(1 - (d+1)\delta_1)^2} \\
  & <
    \varepsilon
    \eqstop
\end{align*}

Let us next prove \ref{it:partition-fattened} $\Rightarrow$ \ref{it:cover-fattened}.  Fix $L \subseteq G$ be compact and $A \subseteq X$ be compact as given by \ref{it:cover-fattened}, let $K \subseteq G$ be some compact subset and let $(\varphi_i)_{i \in I}$ be a partition of unity for $A \subseteq X$ as provided by \ref{it:partition-fattened}.  Then $U_i = (\supp \varphi_i)^\circ$, $i \in I$ defines the desired open cover of $A$.


The implication \ref{it:cover-fattened} $\Rightarrow$ \ref{it:tube-dim} is immediate from the definition of the tube dimension.
\end{proof}

%%% Local Variables:
%%% mode: latex
%%% TeX-master: "classifiability"
%%% End:

\section{Nuclear dimension estimates from tube dimension}
\label{sec:nuclear-dimension}

In this section we obtain upper bounds for the nuclear dimension of crossed products by connected groups in terms of the tube dimension of a dynamical system.  We proceed by first establishing results on transformation groupoids, generalising results from \cite[Section 7]{hirshbergwu2021} and then employ these in a technically refined approach compared to the similar proof strategy of \cite[Theorem 8.1]{hirshbergwu2021}.



\subsection{Restriction to open tubes}
\label{subsec:restriction-tube}

We start by describing how restrictions to open tubes affect transformation groupoids and crossed product \Cstar-algebras.  For the present subsection, we fix the following notation.  Let $G \grpaction{} X$ be an action and set $\cG = G \ltimes X$. Let $K$ be a compact identity neighbourhood of $G$ and let $S$ be a $K$-slice in $X$. We are interested in the relationship between the restriction $\mathcal{G}|_{KS}$ to the corresponding tube $KS$ and the product groupoid $\pairtube = \pair(K) \times S$, where $\pair(K)$ is the pair groupoid associated with $K$ and $S$ is considered as a topological space. Note that by definition of a $K$-slice these groupoids have homeomorphic unit spaces since $(\mathcal{G}|_{KS})^{(0)} = KS$ while $(\pairtube)^{(0)} = K \times S$. We also set $\pairtubeopen = \pair(K^{\circ}) \times \boxint{S}$, which is an open subgroupoid of $\pairtube$.

\begin{proposition}
  \label{prop:pair-groupoid}
  Consider the map
  \begin{gather*}
    \tau \colon
    \pairtube \to \mathcal{G}|_{KS} \colon
    (g,h,x) \mapsto (gh^{-1},hx)
    \eqstop
  \end{gather*}
  Then the following statements hold true.
  \begin{enumerate}
  \item \label{it:pair-groupoid:cont-inj}
    $\tau$ is a continuous, injective groupoid homomorphism.
  \item \label{it:pair-groupoid:image}
    The image of $\tau$ is equal to
    \begin{gather*}
      \{ (g,x) \in \cG|_{KS} : \mathrm{proj}(x) = \mathrm{proj}(gx) \}
      \eqcomma
    \end{gather*}
    where $\mathrm{proj} \colon KS \to S$ denotes the tube projection.  Similarly, we have
    \begin{gather*}
      \tau(\pairtubeopen) = \{ (g,x) \in \cG|_{(KS)^\circ} : \mathrm{proj}(x) = \mathrm{proj}(gx) \}
    \end{gather*}
  \item \label{it:pair-groupoid:clopen-image}
    The set $\tau(\pairtubeopen)$ is clopen in $\cG|_{(KS)^\circ}$.
  \end{enumerate}
\end{proposition}
\begin{proof}
  First observe that $\tau$ is well-defined, as for $g,h \in K$ and $x \in S$ the computation
  \begin{align*}
    hx & = s(gh^{-1},hx) \eqcomma \\
    gx & = gh^{-1}hx = r(gh^{-1},hx)
    \eqcomma
  \end{align*}
  shows that $(gh^{-1},hx) \in \mathcal{G}|_{KS}$.

  We start by showing \ref{it:pair-groupoid:cont-inj}.  In order to show that $\tau$ is a groupoid homomorphism, we compute for $(g,h,x),(h,k,x) \in \pairtube$ that
  \begin{gather*}
    \tau((g,h,x),(h,k,x))
    =
    \tau(g,k,x)
    =
    (gk^{-1},kx)
    =
    (gh^{-1},hx)(hk^{-1},kx)
    =
    \tau(g,h,x)\tau(h,k,x)
    \eqstop
\end{gather*}
  Its defining formula shows that $\tau$ is continuous. To show injectivity, suppose that for a pair of elements $(g,h,x),(g',h',x') \in \pairtube$ we have $\tau(g,h,x) = \tau(g',h',x')$.  In different terms, we suppose $(gh^{-1}, hx) = g'(h')^{-1}, h'x')$.  Then $hx=h'x'$ implies $h=h'$ and $x=x'$ by the definition of a slice, so that subsequently $gh^{-1} = g'h'^{-1} = g'h^{-1}$ forces $g=g'$. This shows that $\tau$ is injective.

  We next prove \ref{it:pair-groupoid:image}.  If $(g,h,x) \in \pair(K) \times S$, its image under $\tau$ belongs to the claimed set, since $\mathrm{proj}(hx) = x = \mathrm{proj}(gx) = \mathrm{proj}(gh^{-1}hx)$.  Conversely, given an element $(g',x') \in \cG|_{KS}$ that satisfies $\mathrm{proj}(x') = \mathrm{proj}(g'x')$, we write $x' = hx$ and $g'x' = gx$ for certain $h,g \in K$ and $x \in S$. Then it follows that $g'hx = gx$ so that freeness of the action $G \grpaction{} X$ implies that $g' = gh^{-1}$. Hence we obtain that $(g',x') = (gh^{-1},hx) = \tau(g,h,x)$ lies in the image of $\tau$.  Using equation \eqref{eq:box_interior} of \Cref{lem:box_interior}, saying that $(KS)^\circ = K^{\circ}\boxint{S}$, we infer that also $\tau(\pairtubeopen) = \{ (g,x) \in \cG|_{(KS)^\circ} : \mathrm{proj}(x) = \mathrm{proj}(gx) \}$ holds.

  Let us finish by proving \ref{it:pair-groupoid:clopen-image}.  Since $\mathrm{proj}$ is continuous, the description of $\tau(\pairtubeopen)$ obtained in \ref{it:pair-groupoid:image} shows that $\tau(\pairtubeopen)$ is closed in $\cG|_{(KS)^{\circ}}$. Further, since $\tau$ is a continous injection from a compact space to a Hausdorff space, it is a homeomorphism onto its image. In particular $\tau(\pairtubeopen)$ is open in $\cG|_{KS}$ since $\pairtubeopen$ is open in $\pairtube$. To conclude, we use the fact that $\cG|_{(KS)^{\circ}}$ is an open subset of $\cG|_{KS}$, so that it follows that $\tau(\pairtubeopen)$ is open in $\cG|_{(KS)^{\circ}}$.
\end{proof}
%
%
%
Combining the previous result with \cite[Lemma 6.1 and Theorem 6.2]{hirshbergwu2021}, we obtain the following maps.
\begin{corollary}
  \label{cor:conditional-expectation}
  Consider the inclusion $\pairtubeopen \subseteq \cG|_{(KS)^{\circ}}$ given by the map $\tau$ of \cref{prop:pair-groupoid}. The following statements hold true.
  \begin{enumerate}
  \item The inclusion $\contc(\pairtubeopen) \to \contc(\cG|_{(KS)^{\circ}})$ extends to an injective *-homomorphism $\Cstarred(\cH) \to \Cstarred(\cG|_{(KS)^{\circ}})$.
  \item The restriction $\contc(\cG|_{(KS)^{\circ}}) \to \contc(\pairtubeopen)$ extends to a conditional expectation $\rE \colon \Cstarred(\cG|_{(KS)^{\circ}}) \to \Cstarred(\pairtubeopen)$.
  \end{enumerate}
\end{corollary}
%
%
%
The next lemma strongly localises subsets of tubes in the presence of connectedness.  Its special case for $\RR$ was used without explicit reference in \cite[p.36, line after (8.1.4)]{hirshbergwu2021}.
\begin{lemma}
  \label{lem:concentration-in-slices}
  Let $G \grpaction{} X$ be an action.  Let $L \subseteq G$ be a connected open subset and let $S$ be a $K$-slice.  If $Lx \subseteq (KS)^\circ$ for some $x \in X$, then there is $s \in S$ such that $Lx \subset Ks$.
\end{lemma}
\begin{proof}
  Write $B = KS$, denote by $\pi \colon B \to S$ the projection on the slice $S$ and consider the map $\psi \colon L \to S$ given by $l \mapsto \pi(lx)$.  We have to show that the image of $\psi$ is a singleton.  Since $\psi$ is continuous and $L$ is connected its image $\psi(L)$ is connected too.  Assume that $s \in \psi(L)$.  Since $Lx \subseteq B^\circ$, there is $g \in L$ and $k \in K^\circ$ such that $gx = ks$. Let $U$ be a symmetric identity neighbourhood in $G$ such that $Uk \subseteq K$ and $Ug \subseteq L$.  Then $ugx = uks \in Ks$ for all $u \in U$, so that $\psi^{-1}(\{s\})$ has non-empty interior.  So $L = \bigsqcup_{s \in S} \psi^{-1}(\{s\})$ is a partition of $L$ into sets with non-empty interior.  Since $G$ is second countable, this implies that the image of $\psi$ is countable.  So $\psi(L)$ is a countable, connected Hausdorff space, and hence a singleton.
\end{proof}



\subsection{The estimate}
\label{sec:estimate}

The aim of this section is to establish bounds on the nuclear dimension of crossed products associated with actions connected amenable groups in terms of the tube dimension of the action.  We will apply this to Lie groups of polynomial growth, for which estimates on the tube dimension can be established.



The next results shows that F{\o}lner functions, as introduced in \cref{sec:partitions}, give rise to quasi-central elements in the $\Lone$-crossed products.  This replaces arguments based on Lipschitz functions used in \cite{hirshbergszabowinterwu2017,hirshbergwu2021} that are only adapted to $\RR$-flows.
\begin{lemma}
  \label{lem:folner-quasi-central-for-I-norm}
  Let $G \curvearrowright X$ be an action and consider the groupoid $G \ltimes X$. Let $K \subseteq G$ be a symmetric compact subset and let $A \subseteq X$ be a compact subset.  Assume that $(\vphi_i)_{i \in I}$ are $(K, \veps)$-F{\o}lner functions in $\contb(X)$ such that $(\vphi_i^2)_i$ is a partition of unity for $A$.  Further assume that $I = I^{(0)} \sqcup \dotsm \sqcup I^{(d)}$ such that $\supp \vphi_i \cap \supp \vphi_j = \emptyset$ for all $0 \leq l \leq d$ and $i,j \in I^{(l)}$.  Then for all $a \in \contc(K \times A) \subseteq \contc(G \ltimes X)$, we have
  \begin{gather*}
    \|\sum_i \vphi_i a \vphi_i - a\|_I \leq 2 (d + 1) \veps^2 \|a\|_I
    \eqstop
  \end{gather*}
\end{lemma}
\begin{proof}
  Let $a \in \contc(G \times X)$ and $f \in \contb(X)$.  Then we have
  \begin{gather*}
    (af)(g,x) = a(g,x) f(x) \quad \text{ and } \quad (fa)(g,x) = a(g,x)f(gx)
    \eqstop
  \end{gather*}
  So we calculate for $a \in \contc(K \times A)$, for $g \in G$ and $x \in A$ that
  \begin{align*}
    |\sum_i \vphi_i(x)\vphi_i(gx) - 1|
    & =
      |\sum_i \vphi_i(x)\vphi_i(gx) - \vphi_i^2(x)| \\
    & =
      |\sum_i \vphi_i(x)(\vphi_i(gx) - \vphi_i(x))| \\
    & \leq
      \sum_i \vphi_i^2(x) \cdot \sum_i (\vphi_i(gx) - \vphi_i(x))^2 \\
    & \leq
      1 \cdot 2 (d + 1) \veps^2
      \eqstop
  \end{align*}
  For the last inequality we used the fact that $\vphi_i$ is a $(K, \veps)$-F{\o}lner function for each $i$ as well as the fact that there are at most $d + 1$ indices $i$ satisfying $\vphi_i(x) \neq 0$ and likewise there are at most $d + 1$ indices $i$ satisfying $\vphi_i(gx) \neq 0$. We can now estimate the following integrals for fixed $x$:
  \begin{align*}
    \int_G | \sum_i \vphi_i a \vphi_i -  a|(g,x) \rmd g
    & =
      \int_G |a(g,x)|  \cdot |\sum_i \vphi_i(gx) \vphi_i(x) -  1| \rmd g \\
    & \leq
      2 (d + 1) \veps^2 \int_G |a(g,x)|   \rmd g \\
    & \leq
      2 (d + 1) \veps^2 \|a\|_I
  \end{align*}
  and similarly
  \begin{align*}
    \int_G | \sum_i \vphi_i a \vphi_i -  a|(g^{-1},gx) \rmd g
    & =
      \int_G |a(g^{-1},gx)| \cdot  |\sum_i \vphi_i(x) \vphi_i(g^{-1}x) -  1| \rmd g \\
    & \leq
      2 (d + 1) \veps^2 \|a\|_I
    \eqstop
  \end{align*}
Taking the supremum of both terms over $x \in X$, we conclude the proof.
\end{proof}
%
% 
%
We are now ready to state and prove our main result, establishing nuclear dimension bounds for free actions in terms of the tube dimension and the dimension of the base space.
\begin{theorem}
  \label{thm:finite-nuclear-dimension}
  Let $G \grpaction{} X$ be a free action of a connected amenable group on a second-countable, locally compact Hausdorff space. Then
  \begin{gather*}
    \dimnuc(\conto(X) \rtimes G) 
    \leq
    (\dim X + 1) (\tubedim(G \grpaction{} X) + 1) - 1
  \end{gather*}
\end{theorem}
\begin{proof}
  Set $\mathcal{G} = G \ltimes X$. Let $B_r \subseteq \contc(G,\contc(X))$ be the $r$-Ball with respect to the $I$-norm. Then $\bigcup_{r \in \NN} \ol{B_r}^{\| \cdot \|} = \conto(X) \rtimes G$, so that a Baire category argument shows that there is some $R > 0$ such that the closure of $B_R$ contains the unit ball of $\conto(X) \rtimes G$. Let $F$ be a finite subset of $B_R$ and let $\veps > 0$. Let $L \subseteq G$ and $A \subseteq X$ be compact subsets such that $F \subseteq \contc(L, \contc(A))$.  Write $d = \tubedim(G \grpaction{} X)$ and pick $\delta < ( \frac{\veps}{2(d+1)R} )^{1/2}$.  Then \cref{prop:characterisation-tube-dimension} provides a finite family of functions $(\vphi_i)_{i \in I}$ such that
  \begin{itemize}
  \item $(\vphi_i^2)_{i \in I}$ is a partition of unity for $A \subseteq X$
  \item each function $\vphi_i$ is $(L, \delta)$-F{\o}lner
  \item $L \supp \vphi_i$ is contained in a the interior of a tube $K_i S_i$, and
  \item there is a partition $I = I^{(0)} \sqcup \hdots \sqcup I^{(d)}$ such that for all $l \in \{0, \dotsc, d\}$ and all $i,j \in I^{(l)}$ we have
    \begin{gather*}
      L\supp \varphi_i \cap L\supp \varphi_j = \emptyset
      \text{.}
    \end{gather*}
  \end{itemize}
  For each $i \in I$ denote by $\pairtubeopeni$ the product groupoid $\pair(K_i^{\circ}) \times \boxint{S_i}$ as in \cref{subsec:restriction-tube}. Let $\Psi_i \colon \Cstar(\pairtubeopeni) \to \Cstar(\cG)$ be the inclusion as in \cref{cor:conditional-expectation} and let $\rE_i \colon \Cstar(\cG|_{(K_iS_i)^{\circ}}) \to \Cstar(\pairtubeopeni)$ be the conditional expectation from \cref{cor:conditional-expectation}.  We observe that compression with $\varphi_i$ is a completely positive map on $\Cstar(\cG)$, which is contractive since $\| \varphi_i \|_\infty \leq 1$ and has its image in $\Cstar(\cG|_{(K_i S_i)^{\circ}})$ since $\supp(\varphi_i) \subseteq (K_iS_i)^{\circ}$.  Thus, we obtain a completely positive contractive map $\Phi_i \colon \Cstar(\cG) \to \Cstar(\pairtubeopeni)$ by putting
  \begin{gather*}
    \Phi_i(a)
    =
    \rE_i(\vphi_i a \vphi_i)
    \eqcomma \quad \text{for all } a \in \Cstar(\cG)
    \eqstop
  \end{gather*}

  For $l \in \{0, \dotsc, d\}$ put $A^{(l)} = \bigoplus_{i \in I^{(l)}} \Cstar(\pairtubeopeni)$ and $A = \bigoplus_{l=0}^d A^{(l)}$. Let $\Phi^{(l)} = \bigoplus_{i \in I^{(l)}} \Phi_i\colon \Cstar(\cG) \to A^{(l)}$ and let $\Psi^{(l)} = \bigoplus_{i \in I^{(l)}} \Psi_i\colon A^{(l)} \to \Cstar(\cG)$. Then each $\Phi^{(l)}$ is completely positive and contractive, so $\Phi = \bigoplus_{l \in \{0, \dotsc, d\}} \Phi^{(l)} \colon \Cstar(\cG) \to A$ is also completely positive and contractive.  Moreover, each $\Psi^{(l)}$ is a *-homomorphism, in particular an order zero contraction, so the number $d^{(l)}$ from \Cref{lem:nuclear_lemma} is zero. It follows that $\Psi = \sum_{l \in \{0, \dotsc, d\}} \Psi^{(l)} \colon A \to \Cstar(\cG)$ is completely positive.

  Since $L \supp \vphi_i$ is contained in the interior of $K_i S_i$ and $F \subseteq \contc(L, \contc(A))$ we find that $\vphi_i a \vphi_i \in \Cstar(\pair(K_i^\circ) \times S_i^\circ)$ when $a \in F$.  Indeed,
  \begin{gather*}
    0 \neq \vphi_i a \vphi_i(g,x)
    = \vphi_i(gx)\vphi_i(x) a(g,x)
  \end{gather*}
  implies that $g \in L$ and $gx, x \in \supp \vphi_i$, which in combination with \cref{lem:concentration-in-slices} shows that there is $s \in S_i$ and $k_1, k_2 \in K_i$ such that such that $x = k_1 s$ and $gx = k_2s$.  By freeness of the $G$-action, this implies $g = k_2k_1^{-1}$.  It follows by \cref{prop:pair-groupoid} that $(g, x) = (k_2k_1^{-1}, k_1 s)$ lies in $\tilde{\mathcal{H}}_i = \pair(K_i^\circ) \times \boxint{S_i}$ when identified with its image in $\mathcal{G}$ under $\tau$.  Combined with \cref{lem:folner-quasi-central-for-I-norm} this shows that for all $a \in F$ we have
  \begin{align*}
    \|\Psi \circ \Phi(a) - a\|
    & =
      \| \sum_{i \in I} \vphi_i a \vphi_i - a\| \\
    & \leq
      \| \sum_{i \in I} \vphi_i a \vphi_i - a\|_I \\
    & \leq
      2(d+1)\delta^2 R
    < \veps
      \eqstop
  \end{align*}

  To finish the proof, we observe that by \Cref{ex:pair_groupoid} and \Cref{ex:top_space_groupoid} the $\Cstar$-algebra $\Cstar(\pairtubeopeni)$ is isomorphic to $\Cstar(\pair(K_i^{\circ}) \times \boxint{S_i}) \cong {\cK \otimes \conto(\boxint{S_i})}$ where $\mathcal{K}$ denotes the compact operators on a separable, infinite-dimensional Hilbert space. This has nuclear dimension at most $\dim (\boxint{S_i}) \leq \dim X$, so that \cref{lem:nuclear_lemma} gives
  \begin{gather*}
    \dimnuc(\conto(X) \rtimes G)
    \leq 
    \sum_{l = 0}^d (\dim(X) + 1)(0 + 1) - 1
    =
    (d + 1) (\dim(X) + 1) - 1
    \eqstop
  \end{gather*}
\end{proof}
%
%
%
The previous theorem was formulated in a general form, and we next specialise to concrete nuclear dimension estimates in the context of actions of connected Lie groups of polynomial growth, thanks to our results on the tube dimension established in \cref{sec:box-dimension}.  Recall from \cref{sec:polynomial-growth} Breuillard's numbers $\rmd(G)$ describing the growth asymptotics of balls in a group of polynomial growth such as a nilpotent Lie group.
\begin{corollary}
  \label{cor:finite-nuclear-dimension}
Let $G \grpaction{} X$ be a free action of a connected Lie group of polynomial growth on a second-countable, locally compact Hausdorff space. Then
\begin{gather*}
  \dimnuc(\conto(X) \rtimes G) + 1 \leq
  (\dim X + 1)^2 11^{\rmd(G)}
\end{gather*}
\end{corollary}
\begin{proof}
  By \cref{thm:covering} combined with \cref{cor:K-slices-exist-polynomial-growth} the tube dimension of $G \grpaction{} X$ is bounded by $11^{\rmd(G)} \cdot (\dim X + 1) - 1$.  We hence obtain the estimate from \cref{thm:finite-nuclear-dimension}.
\end{proof}



\begin{remark}\label{rem:nuclear_dimension_bound_difference}
We remark on the difference between the nuclear dimension bound obtained in \Cref{cor:finite-nuclear-dimension} and \cite[Corollary 10.2]{hirshbergszabowinterwu2017} or \cite[Theorem 8.1]{hirshbergwu2021}. These differences are a result of the following two observations about \cite{kasprowskiruping17}:

Firstly, on p.\ 1214 in \cite{kasprowskiruping17} in the final line of the proof of Lemma 5.1, it is written that if $x \in \Phi_{[-4\alpha,4\alpha]}(gB_i)$ then there is $\beta \in \{ -4\alpha, -2\alpha, 0, 2\alpha, 4\alpha \}$ such that $\Phi_\beta(x) \in \Phi_{[-\alpha,\alpha]}(gB_i)$. While this is true, there is a more optimal choice here: $\beta$ can be chosen from the set $\{ -3\alpha,-\alpha,\alpha,3\alpha \}$ which has one less element. This affects the multiplicity in the statement of Kasprowski-R{\"u}ping's Lemma 5.1 (2) and consequently also Theorem 5.2, which is cited in \cite[Theorem 8.8]{hirshbergszabowinterwu2017} and \cite[Theorem 3.11]{hirshbergwu2021}.

Secondly, in the proof of \cite[Theorem 5.2]{kasprowskiruping17} it is written that by Lemma 5.1 (1) every element in $X_{>\gamma}$ is contained in an open set of the form $\Phi_{(-3\alpha,3\alpha)}(gB_i^k)$ for some $g \in G$, $i \in \NN$, $k=0,\ldots, \mathrm{ind} X$. Upon inspection of this claim it seems however that Lemma 5.1 (1) only guarantees that the sets $\Phi_{(-4,\alpha,4\alpha)}(gB_i^k)$ cover $X_{>\gamma}$. Thus, one will in the end need to control the multiplicity of the collection $\mathcal{B} = \{ \Phi_{(-5\alpha,5\alpha)}(B_i^k) \mid i \in \bN, k=0, 1 \ldots, \operatorname{ind} X \}$ rather than $\mathcal{B} = \{ \Phi_{(-4\alpha,4\alpha)}(B_i^k) \mid i \in \bN, k=0,1\ldots,\operatorname{ind}X \}$. This is exactly the situation we end up with in the present paper in the proof of \Cref{thm:covering} after Claim~\ref{claim:dimension_reduction}, and we account for this with the choice of $a=10$ in the very beginning of the proof of \Cref{thm:covering}.
\end{remark}


%%% Local Variables:
%%% mode: latex
%%% TeX-master: "classifiability"
%%% End:


In this section we explore a few applications of the techniques
introduced in section~\ref{sec:mth}. First we consider the application
of the reweighting technique to an optimization problem. 
Second, we consider the application in Bayesian inference to obtain
the dependence of predictions on the parameters that characterize the
prior distribution.

\subsection{Applications in optimization}
\label{sec:opt}

As an example application of an optimization process we will consider
the probability density function
\begin{equation}
  p_\theta(x) = \frac{1}{\mathcal Z}\exp \left\{ -S(x;\theta) \right\}\,, \qquad \left(  \mathcal Z = \int {\rm d} x\, e^{-S(x:\theta)} \right) \,.
\end{equation}
with
\begin{equation}
  S(x; \theta) = \frac{1}{\theta_1^2+1} \left( x_1^2 + x_1^4 \right) + \frac{1}{2}x_2^2 + \theta_2 x_1x_2\,.
\end{equation}

The shape of $S(x;\theta)$ is inspired in the action of a quantum
field theory in zero dimensions, where $x_1$ and $x_2$ are two fields
with coupling $\theta_2$, while $\theta_1$ is related to the mass of
the field $x_1$.
Expectation values with respect to $p_\theta(x)$ are functions of
the parameters $\theta$.  

% Figure environment removed

As an example we consider the problem of minimizing $\mathbb
E_\theta[x_1^2 + x_2^2]$ (i.e. 
finding the values for $\theta$ that make $\mathbb
E_\theta[x_1^2+x_2^2]$ minimum). 
We have implemented two flavours of Stochastic Gradient Descent (SGD): the first -basic- one, having a constant learning rate, and the second one being the well-known
%both a basic stochastic gradient descent (with constant learning rate) and the
ADAM algorithm \cite{kingma2017adam}. It is worth noting at this
point that as a general concept, SGD implies a stochastic (but
unbiased) evaluation of the gradients of the objective function at
every iteration. While in typical applications in the ML community,
where the task is to fit some dataset, this is done by evaluating the
gradients at different random batches of the data, the present example
is different in that no data is involved. In this case, every
iteration of the SGD evaluates the gradients on the different Monte
Carlo samples used to approximate the objective function  $\mathbb
E_\theta[x_1^2 + x_2^2]$.  

%These algorithms require stochastic evaluations both of the functionand its gradient at arbitrary values of the parameters $\theta$. 
%Here we perform these evaluations via Monte Carlo sampling: we use a
Here we consider a simple implementation of the Metropolis Hastings algorithm in order to
first produce the samples $\{x^{\alpha}\}_{\alpha=1}^N \sim p_\theta(x)$. 
Second, we determine the reweighted expectation value truncated at
first order
\begin{equation}
  \frac{\sum w(x^{\alpha};\tilde\theta) \left[ (x^{\alpha}_1)^2+ (x_2^{\alpha})^2 \right]}{\sum w(x^{\alpha};\tilde \theta)} \approx \bar O + \bar O_i \epsilon_i\,,  
  \qquad \left( w(x^{\alpha};\theta) = e^{S(x^{\alpha};\theta) - S(x^{\alpha};\tilde \theta)}  \right)\,,
\end{equation}
where $\tilde \theta_i =  \theta_i + \epsilon_i$. 
This quantity gives an stochastic estimate of the function value
\begin{equation}
  \bar O = \frac{1}{N}\sum_{i=1}^N [x_1^{\alpha}]^2 + [x_2^{\alpha}]^2\,,
\end{equation}
and its derivatives
\begin{equation}
 \bar O_i \approx \frac{\partial \mathbb E_\theta[x_1^2+x_2^2]}{\partial \theta_i}\,.
\end{equation}

Figure~\ref{fig:sgd} shows the result of the optimization process. 
As the iteration count increases the function is driven to its minima,
while the values of the parameters approach the optimal values
$\theta_1^{\rm opt} = \theta_2^{\rm opt} = 0$. 

It is worth mentioning that in this particular example only $1000$ samples
were used at each step to estimate the loss function and its
derivatives. 
If one decides to use a larger number of samples (say $10^5$), the
value of the parameter $\theta_2$ is determined with a much better precision. 
Note that the direction associated
with $\theta_2$ is much flatter, and therefore its value affects much
less value of the loss function.

\subsection{An application in Bayesian inference}
\label{sec:bayesian}
The purpose of statistical inference is to determine properties of the
underlying statistical distribution of a dataset
$D=\{x_{i},y_{i}\}_{i=1}^{N}$. In many
  cases, the independent variables $x_i$ are fixed, and all the
  stochasticity is captured by the dependent variables $y_i$. As
  such,  
the data is assumed to be sampled from a certain model, specified by
the \textit{likelihood}, 
$p(y|x,\phi)$, which depends on a set of parameters $\phi$.
The Bayesian paradigm attributes a level of confidence to the model by
introducing the \textit{prior} 
$p_{\theta}(\phi)$, \textit{i.e.} an a priori distribution of the
models parameters, where in this context $\theta$ play the role of the
hyper-parameters specifying the prior. Following Bayes' rule, the
\textit{posterior} distribution $p_{\theta}(\phi|D)$ is computed
as\footnote{The normalization factor, 
  $p_{\theta}(D)$, called the evidence, or marginal likelihood, is $\phi$-independent
  and represents the probability distribution of the observed data, given the model.}:
\begin{equation}
  \label{eq:bayes}
  p_{\theta}(\phi|D) \propto p(D|\phi) p_{\theta}(\phi)~.
%  ~~~~
%  p(D|\phi) = \prod_{i=1}^N p(y_i|x_i,\phi)~
\end{equation}
The likelihood of the whole dataset, $p(D|\phi)$, is computed assuming independent data points following a Gaussian distribution:
\begin{equation}
  p(D|\phi) = \prod_{i=1}^{N}\mathcal N(y_{i}|f(x_{i};\phi),\sigma_{i})\,,
  \label{eq:likelihood}
\end{equation}
where $\sigma_i$ are the  uncertainties of the corresponding observations $y_i$ (and assumed here to be given), while the mean of the Gaussian is given by $f(x_i;\phi)$. 
% The posterior in expr.(\ref{eq:bayes}) is the distribution of $\phi$ given the observed data and assumptions.
From a practical standpoint, in addition to the normalization being,
in general, unknown, the usual complexity of the posterior
distribution makes this possibly highly dimensional integral difficult
to compute. The use of Monte Carlo techniques, in particular of the
HMC, is typical in this context.
We focus below on two types of predictions: 1) The variance of the
model parameters $\delta\phi_j^2 = \mathbb{E}_{p_\theta}[\phi_j^2] -
(\mathbb{E}_{p_\theta}[\phi_j])^2$, where $j=1,...,d$, being $d$ the
dimension of $\phi$, and 2) the variance of the output mean $\delta
f_t^2 = \mathbb{E}_{p_\theta}[f_t^2] -
(\mathbb{E}_{p_\theta}[f_t])^2$, where $f_t$ is a shorthand notation
for the output mean $f(x_t;\phi)$, evaluated at a new ``test''
datapoint $x_t$ \footnote{Note that $E_{p_\theta}[f_t]$ is analogous
  to the so-called ``predictive distribution'' of Bayesian inference,
  however here we focus on the expected value of the prediction mean,
  instead of the expected value of the likelihood of $y(x_t)$
  itself.}. 


We are interested in studying the dependence of these quantities on the choice of
hyperparameters $\theta$ that characterize the prior distributions. In
particular we will consider the case of Gaussian priors, and determine
the dependence of our predictions with the width of this Gaussian.

\subsubsection{Model and data set}

We generate a synthetic dataset (cf. Figure \ref{fig:dataset}) by defining the points on an irregular grid in the range $x_i\in[-1.0;1.0]$, such that
\begin{equation}
  y_i = f(x_i;\phi_{\rm true}) + \sigma_i\epsilon~,
\end{equation}
where the mean is a 3rd degree polynomial, $f(x;\phi)=\phi_0+\phi_1x + \phi_2x^2 + \phi_3x^3$, with $\phi_{\rm true} = (1,1,1,1)$; $\epsilon\sim{\cal N}(0,1)$ is sampled from a standard Gaussian, and we consider a heteroscedastic dataset by defining a noise $\sigma_i$ dependent on $x_i$. We adopt the same model in order to  make inference on the parameters $\phi$. 

% Figure environment removed

%The likelihood reads
%\begin{equation}
%  p(D|\phi) = \prod_{i=1}^{N}\mathcal N(y_{i}|f(x_{i},\phi),\sigma_{i})\,,
%\end{equation}
%where $\mathcal N(\mu|\sigma)$ is the usual Gaussian distribution of
%mean $\mu$ and variance $\sigma^2$. As a model we choose a third
%degree polynomial $f(x,\phi) = \phi_{0} + \phi_{1}x + \phi_{2}x^{2}+\phi_{3}x^{3}$. 
%Note that this is the model that was also used to obtain the dataset.

The prior distribution is also chosen as a Gaussian, $\phi\sim {\cal N}(\mu_p,\sigma_p)$. 
For simplicity we choose the priors centered on the ``correct'' values
of the model (i.e. $\mu_p =\phi_{\rm true}$), while we keep the width 
$\sigma_p$ as a hyperparameter to study the dependence on\footnote{This is a simplified setup for the sake of illustration, given the methodological scope of this work. Nonetheless, it is straightforward to apply the method to the situation where we are interested in studying the dependence on both parameters $\mu_p$ and $\sigma_p$ simultaneously, or in general on the joint set of hyperparameters of the model. }.

For any choice of the prior width $\sigma_p$ we can obtain a prediction by
generating $N$ samples $\{\phi^{(\alpha)}\}^N_{\alpha=1}$ according to the distribution
$p_{\theta}(\phi|D)$ computed from \cref{eq:bayes}.  
  
\subsubsection{Reweighting approach}
\label{sec:bayesianhmc}

The reweighting method takes $N$ samples
$\{\phi_{i}^{({\alpha})}\}_{\alpha=1}^{N}$ obtained at
$\sigma_{p}=\sigma_{p}^{*}$ and computes the reweighted average using
$\tilde\sigma_{p}=\sigma_{p}^{*}+\epsilon$ in \cref{eq:rw}.  

For each sample $\phi^{(\alpha)}$, the reweighting factor becomes a polynomial expansion in
$(\sigma-\sigma_{p}^{*})$  
\begin{equation}
  \label{eq:rw bi}
  \tilde w_{\alpha}(\epsilon) = \frac{p_{\mu,\sigma_{p}^{*}+\epsilon}(\phi_{\alpha}|D)}{p_{\mu,\sigma_{p}^{*}}(\phi_{\alpha}|D)}.
\end{equation}
Notice that the zeroth order of \cref{eq:rw bi} is one, such that the zeroth order result corresponds to the usual Monte Carlo point estimate for $\delta\phi_{0}(\sigma_{p}^{*})$.

In order to generate these samples, we used the standard HMC algorithm. 
The equations of motion are
\begin{align}
  &H_{\theta}(\phi,\pi) = \frac{\pi^{2}}{2} - \log(p_{\theta}(\phi|D)),\\
	&\dot\phi_{j} = \pi_{j},\\
	&\dot \pi_{j} = - \frac{1}{\sigma_{p}^{2}}(\phi_{j} - (\mu_p)_j) + \sum_{i=0}^{N}\frac{1}{\sigma_{i}^{2}}\left( y_{i} - f(x_{i},\phi) \right)(x_{i})^{j},
\end{align}
where $\pi=\{\pi_{0},\pi_{1},\pi_{2},\pi_{3}\}$ are the momenta conjugated to $\phi$.
Note that all $\phi$-independent terms can be dropped from the
equations of motion, namely the normalization of $p_{\theta}(\phi|D)$
is not needed.
The eom were solved numerically using a fourth-order symplectic
integrator \cite{OMELYAN2003272} providing a high acceptance rate in
the Metropolis-Hastings step even with a coarse integration.  

The chosen integration step-size was $\varepsilon = 0.001$, while the
trajectory length was uniformly sampled in the interval $[0,100]\times
\varepsilon$\footnote{Due to the quadratic form of the Hamiltonian,
  the phase space of this system is cyclic. The algorithm is ergodic
  only if the trajectory length is
  randomized \cite{RHMC2017}.}.
%Taking into account the conclusions from \cref{sec:nspt}, the average trajectory length is approximately tuned such that the variance is minimized.  

% In the following, all of the Monte Carlo chains correspond to half million thermalized trajectories.

%All the predictions are a function of the hyperparameter $\sigma^*$ and
%we would, generically, be interested in this dependence.
%As for the quantity to study we focus on the uncertainty of the
%average value for $\phi_{i}$, $\delta\phi_{i} =
%\mathbb{E}_{p_{\theta}}[\phi_{i}^{2}]-\mathbb{E}_{p_{\theta}}[\phi_{i}]^{2}$,
%and analogously the uncertainty for the prediction of a new point
%$x_{n}$.  

\subsubsection{Hamiltonian perturbative expansion}

Following the procedure in \cref{sec:nspt}, the Monte Carlo samples
$\{(\tilde\phi_{j})^{\alpha}\}_{\alpha=1}^{N},~j=0,1,2,3$ were
obtained with the modified HMC algorithm for some values of
$\sigma_{p}^{*}$. 
We used the same parameters for the HMC as described in the previous
section. In particular our acceptances were so close to 100\% that any
bias due to the missing accept/reject step is negligible. 
We checked this hypothesis by further performing another simulation
with a coarser value of the integration step and finding completely
compatible results.


\subsubsection{Results}

\begin{table}[t]
  \centering
  \caption{Results for the expansion coefficients of the variance,
    ${\delta\phi^{2}_{j,n}}$ for $\sigma_{p}^{*}=0.3$
    from the reweighting and hamiltonian expansion.}
\scalebox{0.9}{
  \begin{tabular}{cccccccc}
	% \toprule
     & & \multicolumn{6}{c}{$n$} \\\cmidrule{3-8}
     & & 0 & 1 & 2 & 3 & 4 & 5  \\
    \midrule
\multirow{2}{*}{$\delta\phi^{2}_{0,n}$} & RW &    0.00014705(86) &    0.0001384(63) &    -0.000248(29) &     0.000367(62) &     -0.00071(51) &      -0.0003(12)  \\
                  & HAD &   0.00014705(86) &    0.0001365(34) &   -0.0002850(60) &     0.000311(20) &     0.000178(77) &     -0.00115(26)  \\

    \midrule
\multirow{2}{*}{$\delta\phi^{2}_{1,n}$} & RW &     0.01099(15) &       0.0285(12) &      -0.0450(58) &        0.032(13) &         0.04(10) &        -0.61(25)  \\
                  & HAD &     0.01099(15) &      0.02787(69) &      -0.0518(11) &       0.0248(38) &        0.189(16) &       -0.700(46)  \\
    \midrule
\multirow{2}{*}{$\delta\phi^{2}_{2,n}$} & RW &      0.008938(74) &      0.00830(28) &      -0.0283(10) &       0.0850(39) &       -0.234(18) &        0.603(78) \\
                  & HAD &     0.008938(74) &      0.00817(15) &     -0.02789(42) &       0.0849(13) &      -0.2505(44) &        0.726(15)  \\
    \midrule
\multirow{2}{*}{$\delta\phi^{2}_{3,n}$} & RW &     0.03617(59) &      0.1205(51) &       -0.182(24) &        0.050(61) &         0.63(42) &         -4.0(12)  \\
                  & HAD &     0.03617(59) &       0.1177(30) &      -0.2052(42) &        0.020(16) &        1.132(66) &        -4.02(19)  \\
    \bottomrule
    \label{tab:variance phi0}
  \end{tabular}
  }
\end{table}

Here we compare the predictions for the average model parameters
$\phi$ and their dependence on the prior width $\sigma$. In particular
we focus on the variance of the model parameters $\delta\phi^{2}_j$,
since these are the quantities most sentitive to the prior width (i.e. 
very thin priors result in small variance for the model
parameters). We have fixed $\sigma^* = 0.3$, but similar conclusions
are obtained for other values.  

The results of the Monte Carlo average for $\delta\tilde\phi^{2}_i$ and
its derivatives with respect to $\sigma$ are
shown in \cref{tab:variance phi0}. 
Results labeled ``RW'' use the reweighting method, while results
labeled ``HAD'' use the Hamiltonian approach. 

It is obvious that results using the Hamiltonian approach are more
precise:
the uncertainties in the derivatives, $\delta\phi^{2}_{i,n},n\neq 0$, are
smaller for the Hamiltonian approach, despite the statistics being the
same. The difference is larger for higher order derivatives: the
approach based on reweighting struggles to get a signal for the fourth
and fifth derivatives, while the Hamiltonian approach is able to
obtain even the fifth derivative with a few percent precision. 
This fits our expectations (see section~\ref{sec:hamilt-appr-repar}). 
\noindent\newline\newline
On the other hand, for our second quantity of analysis $\delta f_t^2$ (i.e. the variance of the prediction mean), Figure~\ref{fig:ypred} shows the results of the dependence on $\sigma_p$, where we have fixed $x_t=0.5$.

%dependence of the variance of the
%parameter prediction
%\begin{equation}
%  y_{\text{pred}}(x_{n})=\mathbb{E}_{p_{\theta}}[f(x_{n},\phi)] = \int d\phi p_{\theta}(\phi|D) f(x,\phi).
%\end{equation}
%at $x = 0.5$ with respect to $\sigma_P$.
The Hamiltonian approach gives visually results with a reduced variance,
similar to the results presented in table~\ref{tab:variance phi0}.

% Figure environment removed



%%% Local Variables:
%%% mode: latex
%%% TeX-master: "paper"
%%% End:





%%%%%%%%%%
%% Bibliography
%%%%%%%%%%

% Alternative to Biber. Possible styles plain, mybibtexstyle
%\bibliographystyle{mybibtexstyle}
% \bibliography{mybibliography}{}
% \bibliographystyle{plain}

% BibLaTex
{\small
  \printbibliography
}

%%%%%%%%%%
%% Address
%%%%%%%%%%

\vspace{2em}


\begin{center}
\begin{minipage}[t]{0.33\linewidth}
  \small
  Ulrik Enstad \\
  Department of Mathematics \\
  University of Oslo \\
   Moltke Moes vei 35 \\
  0851 Oslo \\
  Norway \\[0.5em]
  ubenstad@math.uio.no
\end{minipage}
\begin{minipage}[t]{0.33\linewidth}
  \small
  Gabriel Favre \\
  Department of Mathematics \\
  Stockholm University \\
  SE-106 91 Stockholm \\
  Sweden \\[0.5em]
  favre@math.su.se
\end{minipage}
\begin{minipage}[t]{0.33\linewidth}
  \small
  Sven Raum \\
  Institut für Mathematik \\
  Universit{\"a}t Potsdam \\
  Campus Golm, Haus 9 \\
  Karl-Liebknecht-Str. 24-25 \\
  D-14476 Potsdam OT Golm \\
  Germany \\[0.5em]
  sven.raum@uni-potsdam.de
\end{minipage}
\end{center}
\end{document}