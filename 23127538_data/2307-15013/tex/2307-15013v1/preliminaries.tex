\section{Preliminaries}
\label{sec:preliminaries}

In this section we recall some of the objects used throughout the article.  This is also an opportunity to fix notations and conventions. As a general convention, all groups $G$ and spaces $X$ are locally compact Hausdorff unless otherwise specified.

\subsection{Dimension theory}
\label{sec:dimension-theory}

We will need some basic dimension theory for topological spaces. Topological dimension will only be applied to normal, metrisable spaces, and for such spaces the most commonly used notions of dimension coincide. In particular, the frequently encountered Lebesgue covering dimension equals the small inductive dimension, which is well suited to the arguments in the present work. We denote the dimension of such a topological space $X$ unambiguously by $\dim X$.
\begin{definition}
  \label{def:small_inductive_dimension}
  The \emph{small inductive dimension} of a topological space $X$ is defined recursively by the conditions that
  \begin{enumerate}
  \item $\dim(\emptyset) = -1$,
  \item
    \label{it:small-ind-dim:bound}
    $\dim X \leq k$ if for every $x \in X$ and every open neighbourhood $U$ of $x$ there exists an open neighbourhood $U' \subseteq U$ of $x$ such that $\dim(\partial U') \leq k-1$, and
  \item $\dim X = k$ if $\dim X \leq k$ but it is not the case that $\dim X \leq k-1$.
  \end{enumerate}
\end{definition}

\begin{remark}
  \label{rmk:inductive_dimension_open_regular}
  Note that the open neighbourhood $U'$ in \ref{it:small-ind-dim:bound} of \cref{def:small_inductive_dimension} can be chosen to be regular open, that is, $\overline{U'}^{\circ} = U'$, whenever $U$ is regular open. Indeed, if $U$ is regular open then $x \in U'' := \overline{U'}^{\circ} \subseteq \overline{U}^{\circ} = U$, and since $\partial U'' \subseteq \partial U'$ we obtain $\dim(\partial U'') \leq \dim(\partial U') \leq k-1$. Hence $U'$ may be replaced with the regular open set $U''$.
\end{remark}
%
%
%
We will frequently apply the following properties of dimension.
\begin{proposition}
  \label{prop:dim_properties}
  Let $M$ and $N$ be second countable, normal spaces.
  \begin{enumerate}
  \item
    \label{it:dim:monotone}
    Every subspace $M'$ of $M$ satisfies $\dim(M') \leq \dim(M)$.
  \item
    \label{it:dim:union}
    If $M$ is the countable union of a sequence of closed sets $(F_n)_{n \in \bN}$, each with $\dim(F_n) \leq k$, then $\dim(M) \leq k$.
  \item
    \label{it:dim:products}
    $\dim(M \times N) \leq \dim M + \dim N$.
  \item
    \label{it:dim:covers}
    If $\mathcal{U}$ is an open cover of $M$, then $\dim M = \sup_{U \in \mathcal{U}} \dim U$.
  \end{enumerate}
\end{proposition}

\begin{proof}
Parts \ref{it:dim:monotone}, \ref{it:dim:union} and \ref{it:dim:products} of the proposition are respectively 1.1.2, 1.5.4 and 1.5.16 of \cite{engelking78}. To see \ref{it:dim:covers}, note first that $M$ is metrisable and hence paracompact, so we may choose a locally finite refinement $\cV$ of $\cU$. Since open sets are $F_\sigma$ in a metrisable space, Theorem 4.1.10 of \cite{engelking78} gives that $\dim M = \sup_{V \in \cV} \dim V$. Since each $V \in \cV$ is a subset of some $U \in \cU$, \ref{it:dim:monotone} gives $\dim V \leq \dim U$, so $\dim M = \sup_{U \in \cU} \dim U$.
\end{proof}

\subsection{Nuclear dimension}

\begin{definition}
    The \emph{nuclear dimension} of a $\Cstar$-algebra $A$, denoted by $\dimnuc{A}$, is the least number $n$ such that the following hold: There exists a net $(F_\lambda, \Psi_\lambda, \Phi_\lambda)_{\lambda \in \Lambda}$ where $F_\lambda$ are finite-dimensional $\Cstar$-algebras and $\Psi_\lambda \colon A \to F_\lambda$, $\Phi_\lambda \colon F_\lambda \to A$ with the following properties:
    \begin{enumerate}
        \item $(\Phi_\lambda \circ \Psi_\lambda)(a) \to a$ uniformly on finite subsets of $A$,
        \item $\| \Psi_\lambda \| \leq 1$ for all $\lambda$, and
        \item for each $\lambda \in \Lambda$, $F_\lambda$ decomposes into $n+1$ ideals $F_\lambda = F_\lambda^{(0)} \oplus \cdots \oplus F_\lambda^{(n)}$ such that $\Phi_\lambda |_{F_\lambda^{(i)}}$ is a completely positive order zero map for each $0 \leq i \leq n$.
    \end{enumerate}
\end{definition}
The nuclear dimension generalises the dimension of a topological space in as fas as when $X$ is a metrisable locally compact space, $\dimnuc (\mathrm{C}_0(X))$ agrees with the unambiguous topological dimension of $X$ from \Cref{sec:dimension-theory}.



In the proof of our main result we will employ the following lemma which has already been used in \cite{hirshbergwu2017,hirshbergwu2021} to facilitate estimates of nuclear dimension. 
\begin{lemma}
  [See \mbox{\cite[Lemma 1.2]{hirshbergwu2017}}]
  \label{lem:nuclear_lemma}
  Let $B$ be a separable, nuclear $\Cstar$-algebra and let $B_0$ be a dense subset of the unit ball of $B$. Then $\dimnuc(B) \leq n$ if and only if for every finite subset $F \subseteq B_0$ and every $\veps > 0$ there exist a $\Cstar$-algebra $A = \bigoplus_{l=0}^d A^{(l)}$ and completely positive maps $\Phi = \bigoplus_{l=0}^d \Phi^{(l)} \colon B \to A$, $\Psi = \sum_{l=0}^d \Psi^{(l)} \colon A \to B$ such that
  \begin{enumerate}
  \item $\Phi$ is contractive,
  \item $\Psi^{(l)} = \sum_{k=0}^{d^{(l)}} \Psi^{(l,k)}$ where each $\Psi^{(l,k)} \colon A \to B$ is an order zero contraction,
  \item $\| \Psi(\Phi(x)) - x \| < \veps$ for all $x \in F$, and
  \item $\sum_{l=0}^d (\dimnuc(A^{(l)})+1)(d^{(l)} + 1) - 1 \leq n$.
  \end{enumerate}
\end{lemma}



\subsection{Locally compact groupoids and their C*-algebras}
\label{sec:groupoids}

We recall the definition of a topologial groupoid and its C*-algebras here and refer to \cite{renault80} for a comprehensive introduction to the subject.  A \emph{groupoid} is a set $\cG$ together with a distinguished subset $\cG^{(2)} \subseteq \cG \times \cG$, a multiplication map $\cG^{(2)} \to \cG$, $(\alpha,\beta) \mapsto \alpha \beta$, and an inversion map $\cG \to \cG$, $\alpha \mapsto \alpha^{-1}$, such that the following properties are satisfied.
\begin{enumerate}
\item If $(\alpha,\beta),(\beta,\gamma) \in \cG^{(2)}$, then $(\alpha\beta,\gamma),(\alpha,\beta\gamma) \in \cG^{(2)}$, and $(\alpha\beta)\gamma = \alpha(\beta\gamma)$.
\item $(\alpha,\alpha^{-1}) \in \cG^{(2)}$ for all $\alpha \in \cG$, and for all $(\alpha,\beta) \in \cG^{(2)}$ we have that $\alpha^{-1}(\alpha \beta) = \beta$ and $(\alpha\beta)\beta^{-1} = \alpha$.
\item For all $\alpha \in \cG$ we have that $(\alpha^{-1})^{-1} = \alpha$.
\end{enumerate}
The \emph{unit space} of the groupoid is the set $\cG^{(0)} = \{ \alpha^{-1}\alpha \mid \alpha \in \cG \} = \{ \alpha \alpha^{-1} \mid \alpha \in \cG \}$. The \emph{source} and \emph{range} maps $\rms,\rmr \colon \cG \to \cG^{(0)}$ are given by $\rms(\alpha) = \alpha^{-1}\alpha$ and $\rmr(\alpha) = \alpha \alpha^{-1}$ for $\alpha \in \cG$. We have that
\begin{gather*}
  \cG^{(2)} = \{ (\alpha,\beta) \in \cG \mid \rms(\alpha) = \rmr(\beta) \}
  \eqstop
\end{gather*}
For each $x \in \cG^{(0)}$ we set
\begin{gather*}
  \cG_x = \{ \alpha \in \cG \mid \rms(\alpha) = x \}\eqcomma
  \qquad \text{and} \quad
  \cG^x = \{ \alpha \in \cG \mid \rmr(\alpha) = x \}
  \eqstop
\end{gather*}
A \emph{locally compact groupoid} is a groupoid $\cG$ equipped with a locally compact Hausdorff topology under which the multiplication, inversion, source and range maps are continuous with respect to the relative topologies on $\cG^{(2)}$ and $\cG^{(0)}$. We will assume that all locally compact groupoids are Hausdorff and that range and source maps are open. We say that $\cG$ is \emph{{\'e}tale} if the source map (equivalently range map) is a local homeomorphism when viewed as a map $\cG \to \cG$.

Given a subset $X$ of the unit space $\cG^{(0)}$, the \emph{restriction of $\cG$ to $X$}, denoted by $\cG|_X$, is the subset $\{ \alpha \in \cG \mid \rms(\alpha),\rmr(\alpha) \in X \}$ of $\cG$. This is a groupoid with unit space $X$ where the operations are restricted from $\cG$.

A \emph{Haar system} for a locally compact groupoid $\cG$ is a collection $(\haar^x)_{x \in \cG^{(0)}}$ where $\haar^x$ is a positive regular Borel mesure on $\cG^x$ for each $x \in \cG^{(0)}$, such that the following requirements are satisfied:
\begin{enumerate}
\item The support of $\haar^x$ is equal to $\cG^x$ for each $x \in \cG^{(0)}$.
\item For every $g \in \contc(\cG)$ the function $x \mapsto \int_{\cG^{x}} g(\alpha) \, \rmd{\haar^x(\alpha)}$ belongs to $\contc(\cG^{(0)})$.
\item For every $\alpha \in \cG$ and $g \in \contc(\cG)$ we have that
  \begin{gather*}
    \int_{\cG^{\rms(\alpha)}} f(\alpha\beta) \, \rmd{\haar^{\rms(\alpha)}}(\beta)
    =
    \int_{\cG^{\rmr(\alpha)}} f(\beta) \, \rmd{\haar^{\rmr(\alpha)}(\beta)}
    \eqstop
  \end{gather*}
\end{enumerate}
Given a Haar system $(\haar^x)_{x \in \cG^{(0)}}$ on $\cG$, one defines the convolution of two functions $f,g \in \contc(\cG)$ by the formula
\begin{gather*}
  (f*g)(\alpha)
  =
  \int_{\cG^x} \sigma(\beta,\beta^{-1}\alpha) f(\beta)g(\beta^{-1}\alpha) \, \rmd \haar^{\rmr(\alpha)}(\beta)
  \eqcomma \quad \text{ for all } \alpha \in \cG
  \eqstop
\end{gather*}
One frequently writes the convolution product implicitly.

With respect to the convolution product and the involution given by $f^*(\alpha) = f(\alpha^{-1})$, the space $\contc(G)$ becomes a $*$-algebra. The $I$-norm on $\contc(\cG)$ is given by
\begin{align*}
  \| f \|_I
  =
  \max \Big\{
  \sup_{x \in \cG^{(0)}} \int_{\cG^x}|f(\alpha)| \, \rmd \haar^x(\alpha)\eqcomma
  \sup_{x \in \cG^{(0)}} \int_{\cG^x} |f(\alpha^{-1})| \, \rmd \haar^x(\alpha)
  \Big\}\eqcomma \quad f \in \contc(\cG)\eqstop
\end{align*}
Any $x \in \cG^{(0)}$ determines a corresponding left regular representation $\lreg_x$ of $\contc(\cG)$ on $\Ltwo(\cG^x,\haar^x)$ given by
\begin{gather*}
  \lreg_x(f)\xi = f * \xi, \quad \text{ for all } f \in \contc(\cG), \xi \in \Ltwo(\cG^x,\haar^x)
  \eqstop
\end{gather*}

The \emph{full C*-algebra} of $\cG$, denoted by $\Cstar(\cG)$ (with respect to the given Haar system) is the \mbox{C*-envelope} of $\contc(G)$, while the \emph{reduced C*-algebra}, denoted by $\Cstarred(\cG)$, is the completion of $\contc(G)$ with respect to the norm
\begin{gather*}
  \| f \|_{\mathrm{red}} = \sup_{x \in \cG^{(0)}}\| \lreg_x(f) \|, \qquad f \in \contc(\cG)
  \eqstop
\end{gather*}
All groupoids considered in this work are amenable, so that their full and reduced groupoid \mbox{C*-algebras} are naturally isomorphic.  See for example \cite[Theorem 6.1.4]{anantharamandelarocherenault2000}.

We have the norm inequality
\begin{equation}
  \| f \|_{\mathrm{red}} \leq \| f \|_I, \qquad \text{for all } f \in \contc(\cG)
  \eqstop
\end{equation}
There is an embedding of $\conto(\cG^{(0)})$ into the multiplier algebra $\rM(\Cstarred(\cG))$ of $\Cstarred(\cG)$ such that if $f \in \conto(\cG^{(0)})$ and $a \in \contc(\cG)$ then $fa, af \in \contc(\cG)$, with
\begin{align}
  (fa)(\alpha) &= f(\rmr(\alpha))a(\alpha)\eqcomma \text{ and}\\
  (af)(\alpha) &= a(\alpha)f(\rms(\alpha))\eqstop
\end{align}
For $a \in \Cstarred(\cG)$, the element $faf$ is called the \emph{compression} of $a$ by $f$.  It follows in particular from the above equations that if $f \in \contc(\cG^{(0)})$ is supported in an open subspace $X \subseteq \cG^{(0)}$ and $a \in \Cstarred(\cG)$, then $faf \in \Cstarred(\cG|_X)$.

The following examples will be important later.
\begin{example}[Transformation groupoids]
  \label{ex:transformation_groupoid}
  Let $G$ be a locally compact (Hausdorff) group acting (on the left) on a locally compact Hausdorff space $X$.  We can then associate to this data a corresponding groupoid $\cG = G \ltimes X$, called a \emph{transformation groupoid}.  As a topological space we have $\cG = G \times X$, and the unit space is given by $\{ (e,x) \mid x \in X \} \cong X$. The source and range maps are given by $\rms(g,x) = x$ and $\rmr(g,x) = gx$, the multiplication is given by $(h,gx)(g,x) = (hg,x)$ and the inversion is given by $(g,x)^{-1} = (g^{-1},gx)$.  A natural Haar system on this groupoid comes from the Haar measure $\haar$ on $G$.  Since $\cG^x = \{ (g,g^{-1}x) \mid g \in G \} \cong G$, we can define $\haar^x$ via
  \begin{gather*}
    \int_{\cG^x} f(\alpha) \, \rmd \haar^x(\alpha)
    =
    \int_G f(g,g^{-1}x) \, \rmd \haar(g)\eqcomma \qquad \text{for all } f \in \contc(\cG)
    \eqstop
  \end{gather*}

The full and reduced groupoid C*-algebras of $G \ltimes X$ are isomorphic to the full and reduced crossed products $\conto(X) \rtimes G$ and $\conto(X) \rtimes_{\mathrm{red}} G$ associated to $G \grpaction{} X$, respectively.  For amenable groups $G$, which will be exclusively considered in this work, the transformation groupoid $G \ltimes X$ is amenable by \cite[Example 2.2.14 (1) in connection with Corollary 2.2.10]{anantharamandelarocherenault2000}. Hence also the reduced and the full crossed products are isomorphic.
  
\end{example}

\begin{example}[Pair groupoids]
  \label{ex:pair_groupoid}
  Let $X$ be a locally compact Hausdorff space. The \emph{pair groupoid} $\pair(X)$ of $X$ is defined as a set $\pair(X) = X \times X$ with unit space $X \cong \{ (x,x) \mid x \in X \} \subseteq X \times X$ and with source and range maps given by $\rms(x,y) = y$ and $\rmr(x,y) = x$, respectively. Its multiplication is given by
  \begin{gather*}
    (x,y)(y,z) = (x,z)\eqcomma \quad \text{for all } x,y,z \in X
    \eqstop
  \end{gather*}
  Haar systems on $\cG$ are in one-to-one correspondence with regular Borel measures $\mu$ on $X$ with full support, and the reduced groupoid C*-algebra of $\pair(X)$ is isomorphic to $\mathcal{K}(\Ltwo(X,\mu))$, the compact operators on $\Ltwo(X,\mu)$, see e.g.\  \cite[Theorem 3.1.2]{paterson1999}. When $\Ltwo(X,\mu)$ is separable (e.g.\ when $\mu$ is $\sigma$-finite) then $\mathcal{K}(\Ltwo(X,\mu))$ has nuclear dimension equal to zero.
\end{example}



\begin{example}[Topological spaces]
  \label{ex:top_space_groupoid}
  A degenerate example of a locally groupoid occurs when $\cG = \cG^{(0)}$. In this case $\cG^{(2)} = \cG^{(0)} \times \cG^{(0)}$, with multiplication and inversion trivially given by $x \cdot x = x$ and $x^{-1} = x$ for $x \in \cG$. Conversely, any locally compact space $X$ can be equipped with this groupoid structure. For each $x \in \cG^{(0)}$ we have that $\cG^x = \{ x \}$, so a natural Haar system is given by assigning unit mass to $x$. The corresponding convolution and involution collapse to pointwise operations, and the associated C*-algebras are isomorphic to $\conto(X)$, the continuous functions vanishing at infinity on $X$.
\end{example}

\begin{example}[Product groupoids]
    \label{ex:product_groupoids}
    If $\cG$ and $\cH$ are locally compact groupoids, the Cartesian product $\cG \times \cH$ may be given a locally compact groupoid structure in the evident way. A pair of Haar systems for $\cG$ and $\cH$ gives rise to a Haar system for $\cG \times \cH$ via the corresponding product measures. The full (reduced) groupoid $\Cstar$-algebra of $\cG \times \cH$ is isomorphic to the maximal (spatial) tensor product of the full (reduced) $\Cstar$-algebras of $\cG$ and $\cH$.
\end{example}


\subsection{Polynomial growth}
\label{sec:polynomial-growth}

A locally compact group $G$ is said to be \emph{compactly generated} if it admits a compact generating set, that is, a compact subset $\Omega \subseteq G$ such that $G = \bigcup_{n=0}^\infty (\Omega \cup \Omega^{-1})^n$. We will usually assume $\Omega$ to be symmetric, that is, $\Omega^{-1} = \Omega$. Furthermore, $G$ is said to be of \emph{polynomial growth} if there exists $d \in \NN$ and $C > 0$ such that $m(\Omega^n) \leq C n^d$ for all $n \in \bN$. The definition can be shown to be independent of the chosen generating set. By a fundamental result of Breuillard \cite{breuillard14} there exists an integer $\rmd(G) \geq 0$ depending only on $G$ and a constant $c(\Omega) > 0$ (depending on the normalisation of a Haar measure $\haar$) such that
\begin{equation}
  \label{eq:breuillard}
  \lim_{n \to \infty} \frac{\vol_G(\Omega^n)}{n^{\rmd(G)}} = c(\Omega) \eqstop
\end{equation}

We will need the following covering property for generating sets in polynomial growth groups. The proof is a standard maximal packing argument that we include here for completeness.
\begin{proposition}
  \label{prop:cover_by_translates}
    Suppose $G$ is compactly generated of polynomial growth, say with compact symmetric generating set $\Omega \subseteq G$, and let $a \in \NN$. Then there exists an $N \in \NN$ such that for all $n \geq N$, any translate of $\Omega^{an}$ can be covered by $(a+1)^{\rmd(G)}$ translates of $\Omega^{2n}$.
\end{proposition}
\begin{proof}
  Let $\veps > 1$ be such that $\veps(a + 1)^{\rmd(G)} < (a + 1)^{\rmd(G)} + 1$, fix a Haar measure $\haar$ on $G$ with the associated constant $c(\Omega)$ and set $\delta = c(\Omega)\frac{\veps -1}{\veps+1} > 0$.  By formula \eqref{eq:breuillard} there exists $N \in \NN$ such that
  \begin{equation}
     \label{eq:inequalities_breuillard}
     (c(\Omega) - \delta) n^{\rmd(G)} \leq \haar(\Omega^n) \leq (c(\Omega) + \delta) n^{\rmd(G)}
    \eqcomma \qquad \text{ for all } n \geq N \eqstop
  \end{equation}
  Let $n \geq N$ and let $\cS$ be the set of subsets $S \subseteq \Omega^{an}$ such that $g^{-1}g' \notin \Omega^{2n}$ for all $g,g' \in S$ with $g \neq g'$. Note that for every $S \in \cS$ the sets $\{ g \Omega^{n} \mid g \in S \}$ are pairwise disjoint.  Indeed, if $h \in g \Omega^{n} \cap g' \Omega^{n}$ for distinct $g,g' \in S$ then $g^{-1}g' = (h^{-1}g)^{-1}(h^{-1}g') \in \Omega^{n}\Omega^{n} = \Omega^{2n}$, which is a contradiction.

  We order $\cS$ with respect to inclusion. Then every totally ordered subset of $\cS$ has an upper bound (namely the union), hence $\cS$ admits a maximal element $S$ by Zorn's lemma. We claim that $\Omega^{an} \subseteq \bigcup_{g \in S} g \Omega^{2n}$ holds.  Otherwise we can find $h \in \Omega^{an}$ with $h \notin g\Omega^{2n}$ for any $g \in S$, which implies that $S \cup \{ h \} \in \cS$. This contradicts the maximality of $S$, so we conclude that $\Omega^{an}$ can be covered by translates of $\Omega^{2n}$ by elements from $S$.

  It remains to bound the size of $S$. Using the pairwise disjointness of $\{ g\Omega^n \mid g \in S \}$ and the inclusion $\bigcup_{g \in S} g \Omega^{n} \subseteq \Omega^{an}\Omega^n = \Omega^{(a+1)n}$ we have that
  \begin{gather*}
    |S| \haar(\Omega^{n}) = \haar \Big( \bigcup_{g \in S} g \Omega^{n} \Big) \leq \haar(\Omega^{(a+1)n})
    \eqstop
  \end{gather*}

  Applying both bounds in \eqref{eq:inequalities_breuillard} gives
  \begin{gather*}
    |S|
    \leq
    \frac{\haar(\Omega^{(a+1)n})}{\haar(\Omega^{n})}
    \leq
    \frac{(c(\Omega) + \delta)((a+1)n)^{\rmd(G)}}{(c(\Omega) - \delta)n^{\rmd(G)}}
    =
    \veps \cdot (a+1)^{\rmd(G)} < (a+1)^{\rmd(G)} + 1
    \eqcomma
  \end{gather*}
  by the choice of $\veps$.  This implies that $|S| \leq (a+1)^{\rmd(G)}$.
\end{proof}



%%% Local Variables:
%%% mode: latex
%%% TeX-master: "classifiability"
%%% End:
