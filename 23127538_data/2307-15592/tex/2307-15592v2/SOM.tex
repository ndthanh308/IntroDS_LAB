\appendix
\begin{center}
	\textbf{\large Supplementary Online Material}
\end{center}
Here we provide the details concerning the truncation of the infinite dimensional MPS matrices appearing in the main text 
%(\textcolor{blue}{44})
and prove the corresponding bounds for the bond dimension.\\
\twocolumngrid
\section{Discretization of the integral}
In this section we prove that the function
\begin{equation}
\eta^0(t)=\alpha\int\limits_{0}^{\infty}\omega e^{-\frac{\omega}{\omega_c}} e^{i \omega t} d\omega,
\end{equation}
defined on an interval $[0,T]$, may be represented as a sum of $\sim \log^2(\frac{\alpha \omega_c T}{\epsilon})$  exponential terms, with $L_1$ error bounded by $\epsilon_1=\frac{\epsilon}{T}$. To do this, we mostly follow the ideas of ~\cite{beylkin2010approximation}.

It is instructive to rotate the contour, introducing a phase $\omega \to i\omega e^{-i\phi}$
\begin{equation}
\int\limits_{0}^{\infty}\omega e^{-\frac{\omega}{\omega_c}} e^{i \omega t} d\omega=-\int\limits_{0}^{\infty}e^{-2 i \phi}\omega e^{-\frac{\omega}{\omega_c}(\omega_c t+i)e^{-i\phi}}d\omega.
\end{equation}
Let us then introduce the change of variables: $\omega=\omega_c e^{x}$:
\begin{equation}\label{eta_int}
  \eta^0(t)=-\alpha \omega_c^2\int\limits_{-\infty}^{\infty}e^{-e^{x-i\phi}(\omega_ct +i)}e^{2(x-i\phi)}dx.
\end{equation}
It is useful to denote the integrand of \eqref{eta_int} as $f_t(x)$:
\begin{equation}
    \eta^0(t) \overset{\text{def}}{=}\int\limits_{-\infty}^{\infty}f_t(x) dx.
\end{equation}
The idea is to discretize the integral
\begin{equation}
    \int\limits_{-\infty}^{\infty} f_t(x) dx\to\chi\sum\limits_{k=-\infty}^{\infty}f_t(k\chi).
\end{equation}
Which implies for the $\eta^0$:
\begin{equation}\label{ft}
\eta^0(t)=\sum\limits_k f_t(k\chi)+\delta\eta(t)    
\end{equation}
To estimate the discrepancy $\delta\eta(t)$ we first estimate the difference between the integral and the discrete sum with the help of Poisson re-summation formula:
\begin{equation}
    \chi\sum\limits_{k=-\infty}^{\infty}f_t(x+\chi k)=    \sum\limits_{n=-\infty}^{\infty}\hat{f}_t(\frac{k}{\chi})e^{-\frac{2\pi i k x}{\chi}},
\end{equation}
where $\hat{f}_t(p)$ is the Fourier image of $f_t(x)$:
\begin{equation}
    \hat{f}_t(p)=\int\limits_{-\infty}^{\infty}f_t(x)e^{-2\pi i p x}dx.
\end{equation}
Such that we have:
\begin{equation}
 \Big|\chi\sum\limits_{k=-\infty}^{\infty}f(x+ \chi k)-\int\limits_{-\infty}^{\infty}f_t(x)dx\Big|\le \sum\limits_{k\ne 0}|\hat{f}_t(\frac{k}{\chi})|.
\end{equation}
Fourier transform of our function of interest is equal to:
\begin{equation}
    \hat{f}_t(p)=\alpha \omega_c^2\Gamma(2-2\pi i p) \left(\tau\omega_c\right)^{2\pi i p-2},
\end{equation}
with $\tau=\tau'+i \tau''=(t+\frac{i}{\omega_c})e^{-i\phi}$. We then have:
\begin{multline}
    |\hat{f}_t(p)|\le  \alpha |\Gamma(2-2\pi i p)||\tau|^{-2}e^{-2\pi p \ \text{arg}(\omega_c\tau) }\le \\ \le \alpha |\Gamma(2-2\pi i p)||\tau|^{-2}e^{\frac{\pi^2}{2} p},
\end{multline}
where in the last inequality we put $\phi =\frac{\pi}{4}$, which leads to $|\text{arg}(\omega_c\tau)|<\frac{\pi}{4}$.\\
The gamma function may be estimated as follows:
\begin{equation}
    \Gamma(z)=e^{i\theta z}\int\limits_{0}^{\infty}e^{-\omega e^{i\theta}}\omega^{z-1}d\omega.
    \end{equation}
Consequently, we have:
\begin{equation}
|\Gamma(2-i y)|\le e^{-y \theta}\int\limits_{0}^{\infty}e^{-\omega \cos(\theta)}\omega d\omega=\cos^{-2}(\theta)e^{-y \theta},
\end{equation}
for $\theta<\frac{\pi}{2}$.\\
If we put $\theta=\frac{\pi}{2}-\delta$, we will get:
\begin{equation}
    \Gamma(2-2\pi i p)\le e^{-2\pi (\frac{\pi}{2}-\delta)p}\sin^{-2}(\delta) , \quad \text{for any}\  \delta<\frac{\pi}{2}.
\end{equation}
We thus may restrict the sum $\sum\limits_{k\ne 0}|\hat{f}_t(\frac{k}{\chi})|$ as:
\begin{equation}
\sum\limits_{k\ne 0}|\hat{f}_t(\frac{k}{\chi})|\le 2(|\tau| \sin(\delta))^{-2}\frac{\alpha}{e^{\frac{2\pi}{\chi}(\frac{\pi}{4}-\delta)}-1}.
\end{equation}
Choosing $\chi=\frac{\frac{\pi^2}{2}-2\pi \delta}{\log(\frac{\alpha \omega_c}{\epsilon_1 \sin^2(\delta)})}$, we finally restrict the difference:
\begin{equation}
\Big|\chi\sum\limits_{k=-\infty}^{\infty}f_t(x+ \chi k)-\int\limits_{-\infty}^{\infty}f_t(x)dx\Big|\le \frac{\epsilon_1 \omega_c}{\left(\omega_ct\right)^2+1},
\end{equation}
with {$\epsilon_1=\frac{\epsilon}{T}$}.


In principle it is possible to optimise over $\delta$, but it is not sufficient for our estimate, we simply put $\delta=\frac{\pi}{16}$. This fixes the discretization step $\chi$:
\begin{equation}\label{chi}
\chi=\frac{\pi^2/{8}}{\log\left(\frac{4\alpha\omega_c T}{\epsilon \sin^2(\frac{\pi}{16})}\right)}.
\end{equation}
Now, one can note that the tails in the infinite sum are also decaying, such that one may restrict the summation as follows:
\begin{equation}\label{Beylikin-Monson}
    \Big|\chi\sum\limits_{k=-\infty}^{\infty}f_t(x+\chi k)-\sum\limits_{k=-N_{\epsilon}}^{M_{\epsilon}}f_t(x+ \chi k)\Big|_{L_1}\le \epsilon_1,
\end{equation}
with
\begin{gather}\label{N}
    N_{\epsilon}= \frac{\log\left(\frac{\alpha\omega_c}{\epsilon_1}\right)}{\chi}\sim\log^2\left(\frac{\alpha\omega_c}{\epsilon_1}\right) ,\\ \label{M}
    M_{\epsilon}=\frac{\log\left(\sqrt{2}\log(\frac{\alpha\omega_c }{\epsilon_1})\right)}{\chi}\sim \log\left(\frac{\alpha \omega_c}{\epsilon_1}\right)  \log\left(\log(\frac{\alpha \omega_c}{\epsilon_1})\right).
\end{gather}
Indeed, first we may majorate the absolute value of the sum, by the sum of the absolute values:
\begin{equation}
\Big|\sum\limits_{k=-\infty}^{-N_{\epsilon}}\chi f_t(x+\chi k)\Big|<\sum\limits_{k=-\infty}^{-N_{\epsilon}}\alpha \omega_c^2\chi e^{-e^{k\chi}\tau^{\prime}\omega_c+2k\chi},
\end{equation}
\begin{equation}
\Big|\sum\limits_{k=M_{\epsilon}}^{\infty}\chi f_t(x+\chi k)\Big|<\sum\limits^{\infty}_{k=M_{\epsilon}}\alpha\omega_c^2\chi e^{-e^{k\chi}\tau^{\prime}\omega_c+2k\chi}.
\end{equation}
If both $N_{\epsilon}$ and $M_{\epsilon}$ are large enough, we may further majorate the sum by the integral of the decreasing function:
\begin{multline}
    \alpha \omega_c^2\sum\limits_{k=-\infty}^{-N_{\epsilon}}\chi e^{-e^{k\chi}\tau^{\prime}\omega_c+2k\chi}<\\<\alpha \omega_c^2\int\limits^{\infty}_{N_{\epsilon}\chi} e^{-e^{-x}\tau^{\prime}\omega_c-2x}dx\overset{\text{def}}{=}\Delta_N(t),
\end{multline}
\begin{multline}
\alpha \omega_c^2\sum\limits^{\infty}_{k=M_{\epsilon}}\chi e^{-e^{k\chi}\tau^{\prime}\omega_c+2k\chi}<\\<\alpha \omega_c^2\int\limits^{\infty}_{M_{\epsilon}\chi}e^{-e^{x}\tau^{\prime}\omega_c+2x}dx\overset{\text{def}}{=}\Delta_M(t).
\end{multline}
In order to estimate the $L_1$ norm of the error, let us first integrate over $t$:
\begin{equation}
    \int\limits_{t=0}^T \Delta_N(t)\le \alpha \omega_c\int\limits_{N_{\epsilon}\chi}^{\infty}e^{-x}e^{-\frac{1}{\sqrt{2}}e^{-x}}\le
   \alpha \omega_c e^{-N_{\epsilon}\chi}
\end{equation}
\begin{equation}
     \int\limits_{t=0}^T \Delta_M(t)\le \alpha \omega_c\int\limits_{M_{\epsilon}\chi}^{\infty}e^{x}e^{-\frac{1}{\sqrt{2}}e^{x}}\le \alpha \omega_c e^{-\frac{1}{\sqrt{2}}e^{M_{\epsilon}\chi}}
\end{equation}
The estimates for $N_{\epsilon},M_{\epsilon}$ then follows. As we have seen, $N_{\epsilon}$ scales as $\log^2(\frac{\alpha \omega_c }{\epsilon_1})$ whereas $M_{\epsilon}$ scales slower, as $\log(\frac{\alpha \omega_c}{\epsilon_1})\log\left(\sqrt{2}\log(\frac{\alpha \omega_c}{\epsilon_1})\right)$. This is not surprising as $N_{\epsilon}-$ terms cover the long range behavior of the IF which is more important in this limit and carry most of the physics.

\section{Truncation of the boson}\label{Truncation of bosonic modes}
The aim of this section is to prove that the MPS matrices %(\textcolor{blue}{44})
defined in the main text can be truncated to the subspace of $n_{\star}$ bosonic excitations, preserving the fixed error $\epsilon$ on the observables. 

We do it in two steps, first we prove that the excitations with large number of bosons are exponentially suppressed. Then we will estimate the effect of neglecting the excitations with $n>n_{\star}\sim \log(\frac{(\omega_cT)^2}{\epsilon})$.

%In order to do the first estimate consider the operator:
%\begin{multline}
%M_{s_i,\bar{s}_i}=e^{-\Omega \Delta t a^{\dagger}a-\Omega^{\star}\Delta t\bar{a}^{\dagger}\bar{a}}e^{-\lambda \Delta t s_i a^{\dagger}-\lambda^{\star} \Delta t\bar{s}_i\bar{a}^{\dagger}} \times \\ \times e^{\Delta t(s_i-\bar{s}_i)(\lambda a-\lambda^{\star} \bar{a})}.
%\end{multline}
%First, we may assume the continuous time limit, because the relevant parameters , $\lambda,\gamma$ are no larger than $\log(T)$ such that $\lambda\Delta t,\gamma \Delta t\sim \log(T)/T$.
%Consider the state $\sum\limits_{n,\bar{n}=0}^{\infty}c_{n,\bar{n}}|n,\bar{n}\rangle$, the time evolution in the bosonic space looks as follows:
%\dot{c}_{n,\bar{n}}=-\left[\gamma(n+\bar{n})-i\omega(n-\bar{n})\right]c_{n,\bar{n}}-\\-\lambda\sqrt{2} s(t) \sqrt{n}c_{n-1,\bar{n}}-\sqrt{2}\lambda^{\star}\bar{s}(t)\sqrt{\bar{n}}c_{n,\bar{n}-1}+\\+
%\frac{s(t)-\bar{s}(t)}{\sqrt{2}}\Big(\sqrt{n+1}\lambda c_{n+1,\bar{n}}-\sqrt{\bar{n}+1}\lambda^{\star}c_{n,\bar{n}+1}\Big).
%\end{multline}
%Now, if coefficients $c_{n,\bar{n}}$ enjoy the inequality 
%\begin{equation} \label{ineq1}
%|c_{n,\bar{n}}|\le\Big|\frac{\sqrt{2}\lambda}{\gamma}\Big|^{n+\bar{n}}\frac{1}{\sqrt{n!\bar{n}!}},
%\end{equation}
%this inequality remains true at later times. Indeed, the equation \eqref{Evolution} results in the estimate:
%\frac{d}{dt}|c_{n,\bar{n}}|^2<-\gamma(n+\bar{n})|c_{n,\bar{n}}|^2+\\+\sqrt{2n}|\lambda|\Big(|c_{n-1,\bar{n}}c_{n,\bar{n}}|+\sqrt{\frac{n+1}{n}}|c_{n,\bar{n}+1}c_{n,\bar{n}}|\Big).
%\end{multline}
%It is then clear that whenever the $c_{n,\bar{n}}$ saturate the bound \eqref{ineq1}, its absolute value starts to decrease.

%The same estimate also applicable to the discrete time dynamics. Let us first rewrite the operator $M_{s_i,\bar{s}_i}$ as a single exponent:
%\begin{equation}
%M_{s_i,\bar{s}_i}=e^{\mathcal{M}_{s_i,\bar{s}_i}}.
%\end{equation}
%The matrix $\mathcal{M}_{s_i,\bar{s_i}}$ is given explicitly by:
%\begin{multline}
%\mathcal{M}_{s_i,\bar{s}_i}=-[\Omega \Delta t a^{\dagger}a+\Omega^{\star}\Delta t\bar{a}^{\dagger}\bar{a}]+\\+\Delta t(s_i-\bar{s}_i)(\tilde{\lambda}_- a-\tilde{\lambda}_-^{\star} \bar{a})-\tilde{\lambda}_+ \Delta t s_i a^{\dagger}-\tilde{\lambda}_+^{\star} \Delta t\bar{s}_i\bar{a}^{\dagger}-\\
% +\frac{1}{2}(s-\bar{s})s\tilde{\lambda}^2-\frac{1}{2}(s-\bar{s})\bar{s}(\tilde{\lambda}^{\star})^2,
%\end{multline}
%with redefined parameters 
%\begin{gather}
%\tilde{\lambda}_{\pm}={\pm\lambda}\frac{\Omega \Delta t}{1-e^{\pm \Omega \Delta t}}.
%\end{gather}
%The result of application of operator $M_{s_i,\bar{s}_i}$ may be then viewed as a continuous evolution from time zero to 1. Thus, we reproduce the estimate analogical to \eqref{ineq}
%\begin{equation}
%|c_{n,\bar{n}}|\le\Big|\frac{\sqrt{2}\lambda \Delta t}{1-e^{-\gamma \Delta t}}\Big|^{n+\bar{n}}\frac{1}{\sqrt{n!\bar{n}!}} , \ \text{for} \ n \ \text{large}.
%\end{equation}
Let us first consider the continuous limit $\Delta t \to 0$ of the quantum evolution%(\textcolor{blue}{47})
:
\begin{equation}
\mathcal{U}^{(i)}_{s_{i+1},\bar{s}_{i+1}|s_i,\bar{s}_i}= M_{s_i,\bar{s}_i} \otimes \left( U^{(i)}_{s_{i+1},s_{i}}\otimes(U^{(i)})^{\star}_{\bar{s}_{i+1},\bar{s}_{i}}\right).
\end{equation}
The wave function $\Psi_{\boldsymbol{n},\bar{\boldsymbol{n}}|s,\bar{s}}$ is labeled by a two integer valued vectors: $\boldsymbol{n}$,$\bar{\boldsymbol{n}}$ and also spin label $s,\bar{s}=\pm 1$. The equations of motion are written as:
\begin{multline}\label{continuous_eq}
    \dot{\Psi}_{\boldsymbol{n},\boldsymbol{\bar{n}}|s,\bar{s}}=-\Big({\sum\limits_{k=-N_{\epsilon}}^{M_{\epsilon}}}\Omega_k n_k+\Omega^{\star}_k\bar{n}_k\Big){\Psi}_{\boldsymbol{n},\boldsymbol{\bar{n}}|s,\bar{s}}-\\-{\sum\limits_{k=-N_{\epsilon}}^{M_{\epsilon}}}\Big[\sqrt{2} \lambda_k s\sqrt{n_k}\Psi_{\boldsymbol{n}-1_k,\bar{\boldsymbol{n}}|s,\bar{s}}-\sqrt{2}\lambda_k\bar{s}\sqrt{{\bar{n}_k}}\Psi_{\boldsymbol{n},\bar{\boldsymbol{n}}-1_k|s,\bar{s}}\Big]+\\+{\sum\limits_{k=-N_{\epsilon}}^{M_{\epsilon}}}
\lambda_k\frac{s-\bar{s}}{\sqrt{2}}\Big[\sqrt{n_k+1} \Psi_{\boldsymbol{n}+1_k,\bar{\boldsymbol{n}}|s,\bar{s}}-\sqrt{\bar{n}_k+1}\Psi_{\boldsymbol{n},\bar{\boldsymbol{n}}+1_k|s,\bar{s}}\Big]+\\+
i\sum\limits_{s'=\pm 1} h^{(i)}_{s,s'}{\Psi}_{\boldsymbol{n},\boldsymbol{\bar{n}}|s',\bar{s}}-i\sum\limits_{\bar{s}'=\pm 1} h^{(i)}_{\bar{s},\bar{s}'}{\Psi}_{\boldsymbol{n},\boldsymbol{\bar{n}}|s,\bar{s}'}.
\end{multline}
Here $h^{(i)}$ is a logarithm of a unitary $U^{(i)}=e^{ih^{(i)}\Delta t}$, acting on the spin. We note that the above equation is almost identical to an equation of motion appearing in the HEOM approach, see Ref.~\cite{xu2022taming}. This underlines the connection between the two approaches.

We will use these equations to estimate the norm squared of a wave function 
\begin{equation}
|\Psi_{\boldsymbol{n},\boldsymbol{\bar{n}}}|^2\overset{\text{def}}{=}\sum\limits_{s,\bar{s}=\pm 1} |\Psi_{\boldsymbol{n},\boldsymbol{\bar{n}}|s,\bar{s}}|^2, 
\end{equation}
we have:
\begin{multline}\label{Norm estimate}
\frac{d}{dt}|\Psi_{\boldsymbol{n},\boldsymbol{\bar{n}}}|<-\Big({\sum\limits_{k=-N_{\epsilon}}^{M_{\epsilon}}}\gamma_k (n_k+\bar{n}_k)\Big)|\Psi_{\boldsymbol{n},\boldsymbol{\bar{n}}}|+\\+4{\sum\limits_{k=-N_{\epsilon}}^{M_{\epsilon}}}\sqrt{2} |\lambda_k|\Big[\sqrt{n_k}|\Psi_{\boldsymbol{n}-1_k,\boldsymbol{\bar{n}}}|+\sqrt{\bar{n}_k}|\Psi_{\boldsymbol{n},\bar{\boldsymbol{n}}-1_k}|\Big]+
\\+4{\sum\limits_{k=-N_{\epsilon}}^{M_{\epsilon}}}
\sqrt{2}|\lambda_k|\Big[ \sqrt{n_k+1}|\Psi_{\boldsymbol{n}+1_k,\boldsymbol{\bar{n}}}|+\sqrt{\bar{n}_k+1}|\Psi_{\boldsymbol{n}+1_k,\boldsymbol{\bar{n}}}|\Big].
\end{multline}
We claim that the amplitudes fulfill the inequality:
\begin{equation}\label{bound}
|\Psi_{n,\bar{n}}|\le f_{\boldsymbol{n},\bar{\boldsymbol{n}}}\overset{\text{def}}{=}4
\prod\limits_{k=-N_{\epsilon}}^{M_{\epsilon}}\Big|\frac{4\sqrt{2}\kappa\lambda_k }{\gamma_k}\Big|^{n_k+\bar{n}_k}\frac{1}{\sqrt{n_k!\bar{n}_k!}},
\end{equation}
with $\kappa>1$ to be defined later. 

Indeed, the inequality holds for the initial wave function which is nonzero only for $\boldsymbol{n}=\bar{\boldsymbol{n}}=\boldsymbol{0}$. We will use an induction, assume that the inequality \eqref{bound} holds up to the moment $t$ and substitute it to the equation \eqref{Norm estimate}. Note that the function $f_{\boldsymbol{n},\bar{\boldsymbol{n}}}$ fulfill the following important relation:
\begin{equation}
f_{\boldsymbol{n}-1_k,\bar{\boldsymbol{n}}}=\frac{|\gamma_k|\sqrt{n_k!}}{4\sqrt{2}\kappa|\lambda_k|}f_{\boldsymbol{n},\bar{\boldsymbol{n}}},
\end{equation}
\begin{equation}
f_{\boldsymbol{n},\bar{\boldsymbol{n}}-1_k}=\frac{|\gamma_k|\sqrt{\bar{n}_k!}}{4\sqrt{2}\kappa|\lambda_k|}f_{\boldsymbol{n},\bar{\boldsymbol{n}}_k}.
\end{equation}
Using it, we may simplify \eqref{Norm estimate}:
\begin{multline} \label{Norm estimate 2}
\frac{d}{dt}|\Psi_{\boldsymbol{n},\boldsymbol{\bar{n}}}|<-\Big(\sum\limits_{k=-N_{\epsilon}}^{M_{\epsilon}}\gamma_k (n_k+\bar{n}_k)\Big)|\Psi_{\boldsymbol{n},\boldsymbol{\bar{n}}}|+\\+\frac{1}{\kappa}\Big({\sum\limits_{k=-N_{\epsilon}}^{M_{\epsilon}}}\gamma_k (n_k+\bar{n}_k)\Big)f_{\boldsymbol{n},\boldsymbol{\bar{n}}}+
\\+\sum\limits_{k=-N_{\epsilon}}^{M_{\epsilon}}
\frac{32 \kappa|\lambda_k|^2}{\gamma_k}\Big[ \sqrt{\frac{n_k+1}{n_k}}+\sqrt{\frac{\bar{n}_k+1}{\bar{n}_k}}\Big]f_{\boldsymbol{n},\boldsymbol{\bar{n}}}.
\end{multline}
Let us, for the moment, ignore the last term (we will see later that it is indeed suppressed). It is clear that $\Psi_{\boldsymbol{n},\bar{\boldsymbol{n}}}$ will always remain less than $f_{\boldsymbol{n},\bar{\boldsymbol{n}}}$, if it was true at the initial time $t$. Indeed, with the last term dropped, equation \eqref{Norm estimate 2} turns to:
\begin{multline}
\frac{d}{dt}|\Psi_{\boldsymbol{n},\boldsymbol{\bar{n}}}|<-\Big(\sum\limits_{k=-N_{\epsilon}}^{M_{\epsilon}}\gamma_k (n_k+\bar{n}_k)\Big)|\Psi_{\boldsymbol{n},\boldsymbol{\bar{n}}}|+\\+\frac{1}{\kappa}\Big(\sum\limits_{k=-N_{\epsilon}}^{M_{\epsilon}}\gamma_k (n_k+\bar{n}_k)\Big)f_{\boldsymbol{n},\boldsymbol{\bar{n}}}.
\end{multline}
Which means that the norm $|\Psi_{\boldsymbol{n},\boldsymbol{\bar{n}}}|$ always decrease, whenever it gets close to the $f_{\boldsymbol{n},\boldsymbol{\bar{n}}}$. It is useful to introduce the notation $\nu_k=\frac{32|\lambda_k^2|}{|\gamma_k^2|}$. The parameter $\nu_k$ controls the average excitation of $k$-th mode. It is important that the parameters $\lambda_k,\gamma_k$ %(\textcolor{blue}{40}),(\textcolor{blue}{41}) 
are such, that $\nu_k$ may be bounded as: 
\begin{equation}
\label{Nu_bound} 
\nu_k<\nu_{\star}=64\alpha\chi\sim\frac{\alpha}{\log(\frac{\alpha \omega_c T}{\epsilon })}.
\end{equation} 
Thus, for a small enough error $\epsilon$, there is always exists $\kappa\sim 1+\frac{\alpha}{\log(\frac{\alpha \omega_c T}{\epsilon })}$ such that the last term in \eqref{Norm estimate 2} may be estimated to be smaller than the second term. Namely, we choose $\kappa$ to fulfill the inequality:
\begin{equation}
    \frac{1-\kappa}{\kappa}
\le \frac{32\sqrt{2} \kappa|\lambda_k|^2}{\gamma^2_k}
\end{equation}
for any $k$. This implies that the norm $|\Psi_{\boldsymbol{n},\boldsymbol{\bar{n}}}|$ obeys the equation:
\begin{multline} \label{Final estimate}
\frac{d}{dt}|\Psi_{\boldsymbol{n},\boldsymbol{\bar{n}}}|<-\Big(\sum\limits_{k=-N_{\epsilon}}^{M_{\epsilon}}\gamma_k (n_k+\bar{n}_k)\Big)|\Psi_{\boldsymbol{n},\boldsymbol{\bar{n}}}|+\\+\Big(\sum\limits_{k=-N_{\epsilon}}^{M_{\epsilon}}\gamma_k (n_k+\bar{n}_k)\Big)f_{\boldsymbol{n},\boldsymbol{\bar{n}}}.
\end{multline}
This finally proves the bound \eqref{bound}. The same reasoning works in case of discrete dynamics. The operator of quantum evolution $U_{\text{tot}}$ is the consecutive application of unitary $U^{(i)}$ acting on the spin and operator $M_{s_i,\bar{s}_i}$ acting on both bosonic and spin degrees of freedom. The first operator $U^{(i)}$, obviously doesn't change the norm $|\Psi_{\boldsymbol{n},\bar{\boldsymbol{n}}}|$, and the MPS operator $M_{s,\bar{s}}$ preserves the bound \eqref{bound}. Indeed, we have:
\begin{equation}
M_{s_i,\bar{s}_i}=e^{-\Lambda(s_i-\bar{s}_i)^2}e^{\Delta t\mathcal{M}_{s_i,\bar{s}_i}}.
\end{equation}
%The matrix $\mathcal{M}_{s_i,\bar{s_i}}$ is given explicitly by:
%\begin{multline}
%\mathcal{M}_{s_i,\bar{s}_i}=-[\Omega \Delta t a^{\dagger}a+\Omega^{\star}\Delta t\bar{a}^{\dagger}\bar{a}]+\\+\Delta t(s_i-\bar{s}_i)(\tilde{\lambda}_- a-\tilde{\lambda}_-^{\star} \bar{a})-\tilde{\lambda}_+ \Delta t s_i a^{\dagger}-\tilde{\lambda}_+^{\star} \Delta t\bar{s}_i\bar{a}^{\dagger}-\\
%+\frac{1}{2}(s-\bar{s})s\tilde{\lambda}^2-\frac{1}{2}(s-\bar{s})\bar{s}(\tilde{\lambda}^{\star})^2,
%\end{multline}
%with redefined parameters 
%\begin{gather}
%\tilde{\lambda}_{\pm}={\lambda}\frac{2\Omega \Delta t e^{\pm \Omega \Delta t/2}}{\sinh( \Omega \Delta t)}.
%\end{gather}
The first operator $e^{-\Lambda(s_i-\bar{s}_i)^2}$ is clearly a quantum channel, and doesn't increase the norm $|c_{\boldsymbol{n},\bar{\boldsymbol{n}}}|$.
The result of application of operator $e^{\mathcal{M}_{s_i,\bar{s}_i}\Delta t}$ may be then viewed as a continuous evolution from time $t$ to $t+\Delta t$, which also preserves the bound \eqref{bound} by the same argument as in the continuous case, this finishes the prove of the bound \eqref{bound}.

This bound dictates that the wave function decays with the growth of the number of bosons. The natural idea is to truncate the states of bosons with more than $n_{\star}$ excitations, for some fixed $n_{\star}(\epsilon)$. In order to prove the bound on the error, let us first massage the function $f_{\boldsymbol{n},\bar{\boldsymbol{n}}}$. Using the bound for $\nu_k$ \eqref{Nu_bound}, we can assume that for $\epsilon$ small enough, all $\nu_k$ are less than $1$. Thus we may use a very rough estimate:
\begin{equation}\label{f_estimate}
f_{\boldsymbol{n},\bar{\boldsymbol{n}}} \le 4\nu_{\star}^{\sum\limits_{k=-N_{\epsilon}}^{M_{\epsilon}}\frac{n_k+\bar{n}_k}{2}}.
\end{equation}

To estimate the corresponding error, let us split the Hilbert space in two spaces: $\mathbb{H}_l$ consisting of excitations with $n_{\star}$ bosons or less, and $\mathbb{H}_e$ with more than $n_{\star}$ bosons. By projecting out the $\mathbb{H}_e$ subspace - we exclude the processes when particle enter this subspace, spend some time $\tau=\Delta t m$ inside and then relax back to $\mathbb{H}_l$. The amplitude of this process is given by: 
\begin{equation}
A_m=\langle \Psi_l|\mathbb{P}_l \left(\mathcal{U}\ \mathbb{P}_e\right)^{m-1}\mathcal{U}\mathbb{P}_l|\Psi_r\rangle,
\end{equation}
where $\mathbb{P}_l,\mathbb{P}_e$ are the projectors to the low/excited spaces respectievely.

As we just have proved, all the intermediate states fulfill the inequality \eqref{bound}. Because of the suppression of the excited states, we conclude that each amplitude may be estimated by the transition rate:
\begin{equation}
    |A_m|\le \Gamma\overset{\text{def}}{=}\max\limits_{\Psi_l,\Psi_r}|\langle\Psi_l|\mathbb{P}_l \mathcal{U}\ \mathbb{P}_e\mathcal{U}\mathbb{P}_l|\Psi_r\rangle|.
\end{equation}
In order to estimate $\Gamma$ we again use the equation \eqref{Final estimate}. One obtains:
\begin{multline}
\Gamma\le 16\nu_{\star}^{\sum\limits_{k=-N_{\epsilon}}^{M_{\epsilon}}(n_k+\bar{n}_k)}\left(1-e^{-\sum\limits_{k=-N_{\epsilon}}^{M_{\epsilon}} \gamma_k\Delta t(n_k+\bar{n}_k)}\right)^2\le \\ 16\nu_{\star}^{n_{\star}}\left(\sum\limits_{k=-N_{\epsilon}}^{M_{\epsilon}} \gamma_k\Delta t(n_k+\bar{n}_k)\right)^2.
\end{multline}
In the last inequality we used the fact that (as we will see below) the sum $\sum\limits_{k=-N_{\epsilon}}^{M_{\epsilon}} \gamma_k (n_k+\bar{n}_k)$ scales polylogarithmically in time, while $\Delta t$ scales as $\frac{\epsilon}{T}$, and so the argument in the exponent is less than $1$.
As $\{\gamma_n\}$ is an increasing sequence%, see (\textcolor{blue}{41})
, we can also bound
\begin{equation}
\gamma_k<\frac{\omega_c}{\sqrt{2}}\log(\frac{\alpha \omega_c}{\epsilon_1}).    
\end{equation}
Which imply a bound for a $\Gamma$:
\begin{equation}
\Gamma\le 8\nu_{\star}^{n_{\star}}\left(\Delta t \omega_c\log\left(\frac{\alpha \omega_c}{\epsilon_1}\right)n_{\star}\right)^2.
\end{equation}
The total error scales as:
\begin{equation}
\epsilon=N^2 \Gamma=8\nu_{\star}^{n_{\star}}\left(T \omega_c\log\left(\frac{\alpha \omega_c}{\epsilon_1}\right)n_{\star}\right)^2,
\end{equation} 
such that we can choose
\begin{equation}\label{n_star}
n_{\star}\sim \frac{\log(\frac{\left(\omega_cT\right)^2}{\epsilon})}{\log(\nu_{\star}^{-1})}.    
\end{equation}
If we denote the total number of modes $N_{\epsilon}+M_{\epsilon}=K\sim \log^2\left(\frac{\omega_c T}{\alpha}\right)$. The the total number of states scales as a sum of binomial coefficients
\begin{equation}
    D=\sum\limits_{n=1}^{n_{\star}}C_{n+K}^K=\sum\limits_{n=1}^{n_{\star}}\frac{(n+K)!}{n!K!}.
\end{equation}
For the estimate we can assume that $K>n_{\star}$, such that:
\begin{equation}
    \frac{(n+K)!}{K!}<(2K)^{n}<(2K)^{n_{\star}}.
\end{equation}
Putting all together, we obtain: 
\begin{equation}\label{D_final}
D<e(2K)^{n_{\star}}\sim e^{\# \frac{\log\left(\frac{(\omega_cT)^2}{\epsilon}\right)}{\log(\nu_{\star}^{-1})}\log(\log\left(\frac{\alpha \omega_cT}{\epsilon}\right))}.
\end{equation}
Asymptotically, when we increase the accuracy $\epsilon\to 0$, for fixed coupling strength $\alpha$, the fraction $\frac{\log(\log\left(\frac{\alpha \omega_c T}{\epsilon}\right))}{\log(\nu_{\star}^{-1})}$ limits to one. And we arrive to a simple estimate%~(\textcolor{blue}{1})
:
\begin{equation}
D\to\frac{(\omega_cT)^4}{\epsilon^2}
\end{equation}
Which is the main statement of the current paper.
\section{From quantum channel to an auxiliary zero temperature boson}
The way we introduced bosons in the main text was rather technical. In this short section we provide a connection with a physical picture. Namely, for the positive $\lambda$, the very same IF generated by MPS:
\begin{equation}
I_{\{\boldsymbol{s},\boldsymbol{\bar{s}}\}}=\langle  0 |M_{s_N,\bar{s}_N}\dots M_{s_1,\bar{s}_1}|\rho_b\rangle.
\end{equation}
May be generated by a quantum channel:
\begin{multline}
    \mathcal{E}(\rho_b\otimes |s_i\rangle \langle \bar{s}_i|)=\sum\limits_{n=0}^{\infty}E_n(\rho_b\otimes |s_i\rangle \langle \bar{s}_i|)E^{\dagger}_n=\\=\sum\limits_{n=0}^{\infty}\Big[\frac{\kappa^n}{n!} a^n e^{-\Omega \Delta t a^{\dagger}a}e^{-\lambda\Delta t s_i a^{\dagger}}e^{\lambda \Delta t s_i a}\left(\rho_b\otimes |s_i\rangle \langle \bar{s}_i|\right)\times \\ \times e^{-\lambda\Delta t \bar{s}_i a^{\dagger}} e^{-\lambda\Delta t \bar{s}_i a}e^{-\Omega^{\star}\Delta t a^{\dagger}a}\left(a^{\dagger}\right)^n\Big].
\end{multline}
The condition for $\mathcal{E}$ to be a quantum channel is
\begin{equation}
\sum\limits_{n=0}^{\infty}E_n^{\dagger}E_n=1,  
\end{equation}
which is equivalent to $\kappa=(e^{2\gamma \Delta t}-1)$.

In order to see the equivalence, we introduce the density matrix/state duality: $\rho_b\to  \sum\limits_{n=0}^{\infty}\rho_b|n\rangle\otimes |n\rangle$, we then may rewrite the quantum channel as:
\begin{multline}
\mathcal{E}(\rho_b\otimes |s_i\rangle \langle \bar{s}_i|)=\mathcal{E}_{s_i,\bar{s}_i}|\rho_b\rangle |s_i\rangle|\bar{s}_i\rangle =\\=e^{\kappa a \bar{a}}e^{-\Omega\Delta t a^{\dagger}a-\Omega^{\star}\Delta t\bar{a}^{\dagger}\bar{a}}\times \\ \times e^{-\lambda\Delta t s_ia^{\dagger}-\lambda\Delta t \bar{s}_i\bar{a}^{\dagger}}e^{s_i\lambda \Delta ta+\bar{s}_i\lambda\Delta t\bar{a}}|\rho_b\rangle |s_i\rangle|\bar{s}_i\rangle
\end{multline}
Direct calculation shows that the multiple application of quantum channel $\mathcal{E}_{s_i,\bar{s}_i}$ and trace over bosonic space is equivalent to the vacuum expectation value of multiple application of ZMKMP operator $M_{s_i,\bar{s}_i}$.
\begin{multline}\label{qChannel_to_ZMKMP}
I_{\{\boldsymbol{s},\boldsymbol{\bar{s}}\}}=\langle  0 |e^{a \bar{a}}\mathcal{E}_{s_N,\bar{s}_N}\dots\mathcal{E}_{s_1,\bar{s}_1}|\rho_b\rangle=\\=\langle  0 |M_{s_N,\bar{s}_N}\dots M_{s_1,\bar{s}_1}|\rho_b\rangle.
\end{multline}
