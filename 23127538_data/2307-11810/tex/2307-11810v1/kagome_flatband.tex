\documentclass[aps,prl,twocolumn,showpacs,superscriptaddress,amsmath,amssymb]{revtex4-2}
\usepackage{graphicx}
\usepackage{latexsym}
\usepackage{bm} 
\usepackage{color}
\usepackage{epsfig}
\usepackage{multirow}
\usepackage{xcolor}
\usepackage{colortbl}
\usepackage{hhline}
\usepackage{simplewick}

\usepackage{graphicx}
\usepackage[colorlinks=true,linkcolor=blue,citecolor=blue,urlcolor=blue]{hyperref}
\usepackage{bbold}
\usepackage{gensymb}

	
\AtBeginDocument{%
    \newwrite\bibnotes
    \def\bibnotesext{Notes.bib}
    \immediate\openout\bibnotes=\jobname\bibnotesext
    \immediate\write\bibnotes{@CONTROL{REVTEX42Control}}
    \immediate\write\bibnotes{@CONTROL{%
    apsrev42Control,author="08",editor="1",pages="0",title="0",year="1"}}
     \if@filesw
     \immediate\write\@auxout{\string\citation{apsrev42Control}}%
    \fi
}%


\newcommand{\dff}[2]{\frac{d {#1}}{d {#2}}}
\newcommand{\pd}[2]{\frac{\partial {#1}}{\partial {#2}}}
\def \mbf {\mathbf}
\def \subk {_{\mbf k}}
\def \del {\nabla}
\def \div {\nabla\cdot}
\def \curl {\nabla\times}
\def \delk {\nabla_{\mbf k}}
\def \divk {\nabla_{\mbf k}\cdot}
\def \curlk {\nabla_{\mbf k}\times}
\def \iint {\int_{-\infty}^{\infty}}
\def \pint {\int_{0}^{\infty}}
\def \nint {\int_{-\infty}^{0}}
\def \cint {\int_{0}^{2\pi}}
\def \hcint {\int_{0}^{\pi}}
\def \nphcint {\int_{-\pi/2}^{\pi/2}}
\newcommand{\intv}[1]{\int_{\mathbf #1}}
\newcommand{\intvs}[2]{\int_{\mathbf #1_{#2}}}
\newcommand{\intvd}[3]{\int_{\mathbf #1_{#2}\mathbf #1_{#3}}}
\newcommand{\intvdp}[1]{\int_{\mathbf #1\mathbf #1'}}
\newcommand{\inttauv}[1]{\int_{\tau\mathbf #1}}
\newcommand{\sumv}[1]{\sum_{\mathbf #1}}
\newcommand{\sumvs}[2]{\sum_{\mathbf #1_{#2}}}
\newcommand{\sumvd}[3]{\sum_{\mathbf #1_{#2}\mathbf #1_{#3}}}
\newcommand{\sumvdp}[1]{\sum_{\mathbf #1\mathbf #1'}}
\newcommand{\bra}[1]{\langle#1|}
\newcommand{\ket}[1]{|#1\rangle}
\newcommand{\braket}[3]{\langle#1|#2|#3\rangle}
\newcommand{\innp}[2]{\langle#1|#2\rangle}
\newcommand{\Bra}[1]{\left\langle#1\rightght|}
\newcommand{\Ket}[1]{\left|#1\rightght\rangle}
\newcommand{\Braket}[3]{\left\langle#1\left|#2\rightght|#3\rightght\rangle}
\newcommand{\Innp}[2]{\left\langle#1|#2\right\rangle}
\newcommand{\epvl}[1]{\langle#1\rangle}
\newcommand{\Epvl}[1]{\left\langle#1\right\rangle}
\def \Tr {\mathrm{Tr}}
\newcommand{\comment}[1]{}
\def \nfy {{\color{red} \LARGE{Not finished yet!!}}}

\newcommand{\red}[1]{{\color{red}#1}}
\newcommand{\blue}[1]{{\color{blue}#1}}
\definecolor{mygreen}{rgb}{0, 0.7, 0}
\newcommand{\green}[1]{{\color{mygreen}#1}}
\newcommand{\purple}[1]{{\color{purple}#1}}
\newcommand{\joel}[1]{{\color{blue}#1}}

\def \nfy {{\color{red} \LARGE{Not finished yet!!}}}

%% Figure environment removed



\begin{document}

\title{Complex magnetic and spatial symmetry breaking from correlations in kagome flat bands}

\author{Yu-Ping Lin}
\affiliation{Department of Physics, University of California, Berkeley, California 94720, USA}
\author{Chunxiao Liu}
\affiliation{Department of Physics, University of California, Berkeley, California 94720, USA}
\author{Joel E. Moore}
\affiliation{Department of Physics, University of California, Berkeley, California 94720, USA}
\affiliation{Materials Sciences Division, Lawrence Berkeley National Laboratory, Berkeley, California 94720, USA}


\date{\today}

\begin{abstract}
We present the mean-field phase diagram of electrons in a kagome flat band with repulsive interactions. In addition to flat-band ferromagnetism, the Hartree-Fock analysis yields cascades of unconventional magnetic orders driven by onsite repulsion as filling changes. These include a series of antiferromagnetic (AFM) spin-charge stripe orders, as well as an evolution from $120^\circ$AFM to intriguing noncoplanar spin orders with tetrahedral structures. We also map out the phase diagram under extended repulsion at half and empty fillings of the flat band. To examine the possibilities beyond mean-field level, we conduct a projective symmetry group analysis and identify the feasible $\mathbb Z_2$ spin liquids and the magnetic orders derivable from them. The theoretical phase diagrams are compared with recent experiments on FeSn and FeGe, enabling determination of the most likely magnetic instabilities in these and similar kagome flat-band materials.
\end{abstract}

\maketitle




\textit{Introduction.---}The study of flat bands has become a major focus of condensed matter research in the past decade. Among various flat-band systems, there is a family in which the flat bands are inherent to geometric frustration of the lattice \cite{mielke92jpa,bergman08prb,leykam18apx,maimaiti21prb,calugaru22np,regnault22n,graf21prb,hwang21prb,neves23ax}. The members of this family support compact localized states (CLS), which fail to spread under the destructive interference of hoppings. Combinations of these CLS become the dispersionless eigenstates of the flat bands. The search among frustrated materials has identified many candidates over the past few years. In particular, recent experiments observe (approximately) flat bands in the quasi-two-dimensional (2D) kagome metals FeSn \cite{kang20nm} and CoSn \cite{kang20nc} and their relevance to FeGe \cite{teng22n,yin22prl,teng23np}.

The kagome lattice is known to be a fruitful ground for correlated phases in the Mott-insulating limit. Even with weaker interactions, the massive density of states carried by the flat band can further enhance correlation effects, giving rise to a rich phase diagram. The famous Stoner criterion for ferromagnetism (FM) \cite{stoner38prsa} can be easily satisfied, supposedly accounting for the observed intralayer FM below 300-400 K in FeSn \cite{kang20nm,xie21cp} and FeGe \cite{teng22n,yin22prl,teng23np}. Interestingly, FeGe further shows a charge density wave (CDW) below 100 K \cite{teng22n,yin22prl,teng23np}, similar to the nonmagnetic kagome metals $A$V$_3$Sb$_5$ with $A=$ K, Rb, and Cs \cite{jiang21nm,zhao21n}. FM is proposed to be the dominant ground state under onsite repulsion in the kagome flat band, especially at and above half filling \cite{mielke92jpa,mielke93cmp,hanisch95annp,pollmann08prl}. Meanwhile, the large number of kagome materials suggests an opportunity for many other correlated phases to be explored.
Various unconventional correlated phases, such as the quantum anomalous Hall FM (QAHFM) and the combinations of spin and charge orders, have been proposed at half filling \cite{ren21prl} and empty filling with spinless \cite{nishimoto10prl,obrien10prb,wen10prb,zhu16prl,ren18prb} or spinful \cite{liu10prb,wen10prb,pollmann14prb} structures.

Given the rich range of correlated phases in both experiment and theory, it is natural to seek a comprehensive phase diagram for kagome flat bands. In this \textit{Letter}, we start with a Hartree-Fock analysis to map out the mean-field phase diagram (Fig.~\ref{fig:pd}). Our Hartree-Fock analysis goes beyond previous works \cite{hanisch95annp,liu10prb,wen10prb} in that we do not impose any symmetry constraints or particular ans\"atze for the ground state. This reveals many unconventional ordered states that were overlooked in previous studies. Interestingly, we find doping cascades of spin-charge (S-C) stripe and $120^\circ$AFM-to-noncoplanar spin (NCPS) orders, which fill the weak-to-moderate-coupling regime below half filling \cite{hanisch95annp}. Under extended repulsion, our study maps out a phase diagram extending the large-onsite-repulsion limit \cite{ren21prl} at half filling. A phase diagram is also presented at empty filling \cite{liu10prb,wen10prb}. We complement the Hartree-Fock study by a projective symmetry group analysis of possibilities beyond the mean-field level, identifying a $\mathbb Z_2$ spin liquid whose derived magnetic orders are compatible with the Hartree-Fock results. Furthermore, by comparing our theory with existing experiments on FeSn and FeGe, we guide the search for unconventional magnetism in future studies of flat-band kagome metals.

% Figure environment removed




\textit{Kagome lattice and flat band.---}The kagome lattice has a triangular Bravais lattice of $3$-site unit cells [Fig.~\ref{fig:kagome}(a)]. We start from the repulsive fermionic Hubbard model \cite{arovas22arcmp,qin22arcmp} on the kagome lattice
\begin{equation}
\label{eq:hubbard}
\begin{aligned}
H&=-\sum_{ii'\tau\tau'}\sum_\sigma t_{ii'\tau\tau'}c_{i\tau\sigma}^\dagger c_{i'\tau'\sigma}\\&\quad+\frac{1}{2}\sum_{ii'\tau\tau'}\sum_{\sigma\sigma'}U_{ii'\tau\tau'}c_{i\tau\sigma}^\dagger c_{i'\tau'\sigma'}^\dagger c_{i'\tau'\sigma'}c_{i\tau\sigma},
\end{aligned}
\end{equation}
where $c_{i\tau\sigma}^{(\dagger)}$ annihilates (creates) a fermion at the Bravais lattice site $i$ and sublattice $\tau=0,1,2$ with spin $\sigma=\pm1$. The first and second terms describe the tight-binding hoppings and density-density repulsions, respectively. These parameters are defined according to their ranges of action as onsite $t_0$ and $U_0$, nearest-neighbor $t_1$ and $U_1$, second-neighbor $t_2$ and $U_2$, etc.

% Figure environment removed

In the absence of repulsion $U=0$, the tight-binding Hamiltonian hosts three bands in the Brillouin zone [Figs.~\ref{fig:kagome}(b) and (c)]. The topmost band is completely flat under nearest-neighbor hopping $t_1>0$ \cite{mielke92jpa,bergman08prb}. This flat band crosses with the middle dispersive band, forming a quadratic band-crossing point (QBCP) at $\boldsymbol\Gamma$. Focusing on the flat band, we define the flat-band (FB) filling $n_\text{FB}\in[-1,1]$ for later convenience. Half filling is defined as $n_\text{FB}=0$, and $n_\text{FB}=\pm1$ correspond to full and empty fillings of the flat band, respectively.

\textit{Mean-field phase diagram.---}We employ the Hartree-Fock approximation to obtain the repulsion-driven ground states at the mean-field level \cite{supp}. Here we adopt a spatially unrestricted formalism, which is unbiased and is especially helpful for finding ground states with large unit cells. The ground states fall into distinct electronic phases distinguished by their symmetry-breaking S-C orders. Our starting point is the pure Hubbard model with nearest-neighbor hopping $t_1=1$ and onsite repulsion $U_0>0$. Sweeping across all flat-band fillings, we map out the $n_\text{FB}$-$U_0$ phase diagram [Fig.~\ref{fig:pd}(a)]. One immediate observation is the wide FM phase [Fig.~\ref{fig:pd}(b)] across all fillings \cite{hanisch95annp}, especially in the whole regime at and above half filling $n_\text{FB}\geq0$. Meanwhile, the other competing orders arise below half filling $n_\text{FB}<0$ and push the onset of FM to nonzero repulsion $U_0>0$.

\textit{Ferromagnetism.---} The onset of FM at various fillings can be understood from a Stoner-type analysis.  When FM develops, the bands split into majority- and minority-spin branches. The density imbalance $\delta n_\text{S}=n_\text{major}-n_\text{minor}$ 
determines the energy of this Stoner splitting
\begin{equation}
\label{eq:stoner}
\Delta E_\text{S}=U_0\delta n_\text{S}.
\end{equation}
To achieve maximal magnetization $m=\delta n_\text{S}/2$, the Stoner splitting should fully fill the majority-spin bands. This requires a large enough splitting to push all majority-spin bands below the minority-spin Fermi level. For half filling $n_\text{FB}=0$, the saturated minority-spin Fermi level lies at the QBCP $\boldsymbol\Gamma$ [Fig.~\ref{fig:kagome}(d)]. The saturation occurs under arbitrarily small repulsion, indicating a strong weak-coupling instability towards FM. This scenario also explains the dominance of FM above half filling $n_\text{FB}>0$. On the other hand, the saturation splitting increases below half filling $n_\text{FB}<0$, with the maximum at empty filling $n_\text{FB}=-1$ [Fig.~\ref{fig:kagome}(e)]. This implies a gradual suppression of FM, giving way to other energetically competitive orders to occur in the weak-to-moderate-coupling regime. Note that the FM in the phase diagram [Fig.~\ref{fig:pd}(a)] is always saturated, thereby acting as a half (semi)metal under the Stoner splitting.

\textit{Competing orders.---}Next we explore the plethora of phases below half filling $n_\text{FB}<0$. We begin with the moderate-coupling regime at empty filling $n_\text{FB}=-1$. An $\mbf M$-point order arises when the repulsion reaches $U_0=2$. The ground state is a S-C stripe order with an AFM-stripe-charge-stripe-period-1 (AFMSt-CStP1) pattern [Fig.~\ref{fig:pd}(c)], where AFM stripes develop on the high-density sublattices in the nematic charge stripes. As the repulsion exceeds $U_0=3$, the ground state is taken over by a $\mbf K$-point order. This order shows a $\sqrt3\times\sqrt3$ $120^\circ$ triangle AFM ($120^\circ$TAFM) [Fig.~\ref{fig:pd}(d)], where the coplanar $120^\circ$ order \cite{huse88prl} forms among three nearest-neighbor triangles. The band structure indicates that the $120^\circ$TAFM is an insulator \cite{supp}, but with trivial band topology as the coplanar spin order has zero scalar spin chirality \cite{ohgushi00prb,taguchi01sc,martin08prl}.




\textit{Cascade of S-C stripe orders.---} Moving away from empty filling $n_\text{FB}=-1$, many orders appear.  In the S-C-stripe regime, the AFMSt-CStP1 order deforms into the FM as $n_\text{FB}$ goes from $-1$ to $0$. Intuitively, the FM is an AFMSt-CSt order with infinite period (P$\infty$). There is an elegant cascade of P$n$ orders at the reciprocal integer fillings $n_\text{FB}=-1/n$ [Fig.~\ref{fig:pd}(c)], corresponding to the filling of low-density stripes, where one out of $n$ stripes are unfilled at $n_\text{FB}=-1/n$. Note that the cascade is also visible away from the reciprocal integer fillings $n_\text{FB}=-1/n$ \cite{supp}. However, noncollinearity may appear due to intertwinement with the $120^\circ$TAFM.

\textit{Cascade of noncoplanar spin orders.---} More intriguing magnetic orders appear when the $120^\circ$TAFM is doped. We first notice a $2\sqrt3\times2\sqrt3$ $120^\circ$ star-of-David (SoD) AFM at $n_\text{FB}=-1/2$ \cite{supp} slightly above the ground state, and the essential question is how the ground state deforms between the two $120^\circ$AFMs.  Remarkably, the energetically favorable path breaks coplanarity and travels through noncoplanar spin orders. At commensurate fillings $n_\text{FB}=-5/6$ and $-2/3$, which are the $1/3$ and $2/3$ points between $n_\text{FB}=-1$ and $-1/2$, we observe enriched tetrahedral spin orders [Fig.~\ref{fig:pd}(d)]. At both fillings, the S-C unit cell is $2\sqrt{3}\times2\sqrt{3}$. The charge unit cell is $\sqrt{3}\times\sqrt{3}$, containing both high- and low-density sites. The tetrahedral spin orders further enlarge the period by $2\times2$. At $n_\text{FB}=-5/6$, each high-density triangle hosts $\text{C}_3$-symmetric slightly canted spins about a principal order. The principal orders are tetrahedral, making the ground state a tetrahedral canted triangle-spin (TCTS) order. Note that the low-density sites also host a TCTS structure, which couples to the high-density TCTS order and induces the canting. At $n_\text{FB}=-2/3$, we observe a coplanar $60^\circ$ spin winding on each high-density hexagon, which we term the hexagon spin vortex (HSV). Amazingly, the HSVs are again tetrahedral, and the ground state is a tetrahedral HSV (THSV) order. Notably, it is actually a canted tripling of the cuboc order in the classical $J_1$-$J_2$ model \cite{domenge05prb,domenge08prb,messio11prb}. The low-density kagome superlattice hosts the $12$-site cuboc order. Meanwhile, the high-density sites show a $24$-site canted doubling, where each spin is the middle of its two low-density neighbors.

In the noncoplanar spin orders, the spin patterns can act as sources of fluxes and induce a nonzero Berry curvature \cite{ohgushi00prb,taguchi01sc,martin08prl}. When a full gap is opened, the ground state can become a Chern insulator.

\textit{Completion of phase diagram.---}The phase diagram [Fig.~\ref{fig:pd}(a)] exhibits a few more features apart from the uniform orders \cite{supp}. First, phase separation \cite{emery90prl,hanisch95annp,qin22jpcm} can occur between different phases in the doping phase diagram, where domains of both phases coexist. Second, coplanar spiral spin (SS) orders \cite{hanisch95annp,qin22jpcm} generally occur along the FM phase boundary. Last, we observe a wide regime of dilute CLS (DCLS) at weak coupling. Increasing filling raises the CLS density, until it turns into the AFMSt-CStP2 order at $n_\text{FB}=-1/2$.

Furthermore, we go beyond the pure Hubbard model by adding extended repulsions $U_1=2U_2>0$. The extended phase diagrams and the various S-C orders appearing in them are shown in Figs.~\ref{fig:nfb0} and \ref{fig:nfbn1}. The key observation is that the extended repulsion drives the ground state through a number of S-C orders \cite{supp}. Detecting a certain type of charge or magnetic order in the flat-band materials thus serves as a way to characterize the dominant interactions.

% Figure environment removed


% Figure environment removed



\textit{Possible spin liquids.---}The rich structure in the phase diagrams reveals an intricate relation among the energetically competitive magnetic orders. An interesting physical scenario may happen when they arise as symmetry-breaking instabilities of a \emph{parent} quantum spin liquid phase \cite{explanation}. 
The ground states of certain kagome spin models have been claimed to be $\mathbb{Z}_2$ spin liquids \cite{PhysRevB.65.224412,yan2011spin,nishimoto2013controlling,jiang2012identifying}.
Here, from a symmetry point of view, we explore this possibility by selecting the most likely $\mathbb{Z}_2$ kagome spin liquids and analyzing the various magnetic orders derived from them \cite{messio12prl,supp}.

\begin{table}[t]
\centering
\caption{Summary of magnetic orders derived from the projective symmetry group classes $(0,1)$ and $(1,0)$. Ordering structure in the enlarged magnetic unit cell: $\dag$: non-uniform, umbrella order. $\ddag$: Coplanar $120^\circ$. The cases 1, 2, and 3-5 of class (1,0) correspond to the FM, $120^\circ$TAFM, and spiral-spin orders in the Hartree-Fock analysis, respectively.}\label{Tab:mag_order_psg}
\begin{tabular}{cccccc}
\hline
Class&No.&Spinon $k_c$&  Mag. unit cell & Sublattice structure \\
\hline
\multirow{5}{*}{(0,1)}&1&$\Gamma$&Not enlarged&Collinear\\
&2&$\Gamma$&Not enlarged&Collinear\\
&3&$\pm\mathrm{K}$&$\sqrt{3}\times \sqrt{3}^\dag$&Collinear\\
&4&$\pm\mathrm{K}$&$\sqrt{3}\times \sqrt{3}^\dag$&Coplanar $120^\circ$\\
&5&$\mathrm{M}_{1,2,3}$&$2\times 2^\dag$& $\tau$--$\mathrm{M}_\tau$-locked stripe\\
\hline
\multirow{4}{*}{(1,0)}&1&$\Gamma$&Not enlarged&Collinear\\
&2&$\pm\mathrm{K}$&$\sqrt{3}\times \sqrt{3}^\ddag$&Collinear\\
&\multirow{2}{*}{3-5}&$\Gamma\leftrightarrow\pm\mathrm{K}$&\multirow{2}{*}{Incommensurate}&\multirow{2}{*}{$-$}\\
&&$\leftrightarrow \mathrm{M}_{1,2,3}$&&\\
\hline
\end{tabular}
\end{table}


Our starting point is the two $\mathbb{Z}_2$-spin-liquid Hamiltonians obtained from the projective symmetry group classification \cite{PhysRevB.74.174423} for bosonic spinons. This conveniently allows us to consider the leading ordering instability upon condensing the spinons. The resulting orders are given in Table \ref{Tab:mag_order_psg}. Starting from the $(1,0)$ class of $\mathbb Z_2$ parent spin liquid, we reproduce the $120^\circ$TAFM and FM orders in the Hartree-Fock analysis. Crucially, the spiral spin orders also appear as weaker possible instabilities \cite{note1}, suggesting the possibility of finding the $\mathbb Z_2$ parent spin liquid in the corresponding phase region [Fig.~\ref{fig:pd}(a)]. On the other hand, our analysis uncovers a new interesting sublattice-momentum-locked magnetic order from the $(0,1)$ class. This magnetic order may share similar structure to the tetrahedral spin order from chiral spin density wave \cite{martin08prl}. 




\textit{Experimental realization.---}We examine the flat-band kagome metals FeSn \cite{kang20nm} and FeGe \cite{teng22n,yin22prl,teng23np} in light of our phase diagrams. Starting from the Fe-$d_{xz/yz}$-orbital band structures of FeSn \cite{xie21cp} and FeGe (assuming the Van Hove singularity, VHS, at the Fermi level as was experimentally observed) \cite{teng23np} in the first-principle computations, we consider the single-orbital pure Hubbard model on the kagome lattice. The flat-band filling $n_\text{FB}$ and the onsite repulsion $U_0$ are estimated from the minority-spin filling and Stoner splitting energy (\ref{eq:stoner}), respectively. This estimation locates FeSn and FeGe in our phase diagram as shown in Fig.~\ref{fig:pd}(a): FeSn falls in the FM phase, consistent with the experiments. Although FeGe sits in the spiral-spin-order phase, the interlayer and multi-orbital effects may stabilize the FM. We further include the nearest-neighbor repulsion $U_1/t_1\approx1.07$ for FeGe by fitting the CDW splitting \cite{teng23np}. This induces the secondary tri-hexagonal bond order (FM-TH) [Fig.~\ref{fig:pd}(b)], consistent with the experimental $2\times2$ CDW \cite{teng22n,yin22prl,teng23np} and the theoretical charge order at VHS \cite{tan21prl,lin21prb,park21prb,feng21prb,christensen21prb}.

It is worth noting the possible implications of the results in this work. The majority of existing compounds (such as FeSn and FeGe) is located in the large-$U_0$ FM phase. Meanwhile, our analysis finds intriguing magnetic orders at smaller repulsion $U_0$. Based on these results, we propose the reduction of electronic repulsion as a feasible route toward unconventional magnetism. Due to the massive density of states on the flat bands, the magnetic orders may be observable at relatively high temperatures.

\textit{Discussion.---}Our analysis includes a detailed study of kagome flat bands at the mean-field level, but there remains considerable scope for further investigation. First, the Hartree-Fock approximation can overestimate the symmetry-breaking orders and miss essential intertwinement between different orders. While the former may be less serious in the flat band, the latter may be enhanced due to strong correlation effects. Second, there are important correlated phases that are not captured. These include superconductivity, which does not appear at the mean-field level under electronic repulsion. It will be interesting to search for flat-band superconductivity by doping the unconventional magnetic orders \cite{watanabe05jpsj,song21prb,peng21njp,zhu22prb,huang23prl,zhu23prb} in our phase diagram. Meanwhile, spin-triplet superconductivity may occur in the FM half metal, especially close to the CDW at effectively spinless VHS \cite{cheng10prb,gneist22epjb,he23prr} such as in FeGe. On the other hand, the ground states away from the fermion-bilinear condensates, such as the spin liquids \cite{savary16rpp} and fractional Chern insulators \cite{liu23bk}, are also invisible. While we suggest candidate $\mathbb Z_2$ spin liquids with a projective symmetry group analysis, a confirmation beyond the mean-field level is still necessary. Future studies with strong-correlation numerical methods will be a useful extension of our results.

\begin{acknowledgments}
The authors thank Shubhayu Chatterjee, Yi-Ping Huang, Haining Pan, and Linda Ye for fruitful discussions. This work was primarily supported by the Air Force Office of Scientific Research under Grant No. FA9550-22-1-0270. Y.P.L. and C.L. acknowledge fellowship support from the Gordon and Betty Moore Foundation through the Emergent Phenomena in Quantum Systems (EPiQS) program.  J.E.M. acknowledges a Simons Investigatorship. Parts of the numerical computations were performed on the Lawrencium cluster at Lawrence Berkeley National Laboratory.
\end{acknowledgments}




%%%%%%%%%%




%\appendix




%\section{}
%\label{app:}




\bibliography{reference}



\clearpage
\onecolumngrid




\section{Supplemental Material for ``Complex magnetic and spatial symmetry breaking from correlations in kagome flat bands"}




\setcounter{secnumdepth}{3}
\setcounter{equation}{0}
\setcounter{figure}{0}
\renewcommand{\theequation}{S\arabic{equation}}
\renewcommand{\thefigure}{S\arabic{figure}}
\newcommand\Scite[1]{[S\citealp{#1}]}
\makeatletter \renewcommand\@biblabel[1]{[S#1]} \makeatother




\section{Hartree-Fock approximation}


In this section, we introduce the Hartree-Fock (HF) approximation in our analysis. The central spirit is to approximate the interacting model by a mean-field noninteracting theory. In this mean-field theory, the fermions group into the environments of individual fermions.  The structures of these fermionic environments are set by the ground state, where various symmetry breaking orders may occur. Each fermion experiences the background potential from its environment, known as the Hartree potential. Meanwhile, the fermion can also immerse into the environment, followed by the ejection of a new fermion as a replacement. This process involves the Fermi exchange and leads to the Fock potential. The mean-field theory is described by the Hartree-Fock Hamiltonian, where the Hartree and Fock potentials replace the interaction. The ground state can be solved by diagonalizing and updating the Hartree-Fock Hamiltonian self-consistently under energy minimization.

Now we lay out the explicit formalism. For a given lattice, we consider the Hubbard model
\begin{equation}
H=-\sum_{ii'\tau\tau'}\sum_\sigma t_{ii'\tau\tau'}c_{i\tau\sigma}^\dagger c_{i'\tau'\sigma}+\frac{1}{2}\sum_{ii'\tau\tau'}\sum_{\sigma\sigma'}U_{ii'\tau\tau'}c_{i\tau\sigma}^\dagger c_{i'\tau'\sigma'}^\dagger c_{i'\tau'\sigma'}c_{i\tau\sigma}
\end{equation}
with Bravais lattice site $i$, sublattice $\tau$, and spin $\sigma$. Any wavefunction can be represented by its density matrix
\begin{equation}
P_{ii'\tau\tau'\sigma\sigma'}=\epvl{c_{i'\tau'\sigma'}^\dagger c_{i\tau\sigma}},
\end{equation}
which is a collection of the particle-hole condensates. The Hartree-Fock Hamiltonian is derived as
\begin{equation}
H^\text{HF}[P]=\sum_{ii'\tau\tau'\sigma\sigma'}c_{i\tau\sigma}^\dagger H^{HF}_{ii'\tau\tau'\sigma\sigma'}c_{i'\tau'\sigma'}=T+H^\text{H}[P]+H^\text{F}[P],
\end{equation}
where the three terms are
\begin{equation}
\begin{aligned}
&\text{Noninteracting potential: }T_{ii'\tau\tau'\sigma\sigma'}=-\delta_{\sigma\sigma'}t_{ii'\tau\tau'},\\
&\text{Hartree potential: }H^\text{H}_{ii'\tau\tau'\sigma\sigma'}[P]=\delta_{ii'}\delta_{\tau\tau'}\delta_{\sigma\sigma'}\sum_{i''\tau''\sigma''}U_{ii''\tau\tau''}P_{i''i''\tau''\tau''\sigma''\sigma''},\\
&\text{Fock potential: }H^\text{F}_{ii'\tau\tau'\sigma\sigma'}[P]=-U_{ii'\tau\tau'}P_{ii'\tau\tau'\sigma\sigma'}.
\end{aligned}
\end{equation}
With the density matrix and the Hartree-Fock Hamiltonian, the energy of the wavefunction can be computed
\begin{equation}
E[P]=\epvl{H}=\frac{1}{2}\Tr(P[T+H^\text{HF}]).
\end{equation}

We want to obtain the $N$-particle ground state, where the occupation number $N=n_fN_L$ is defined by the filling $n_f$ and the number of lattice sites $N_L$. This is achieved by variationally solving the density matrix $P$ under iterative energy minimization. The starting point is an initial density matrix $P_m=P_0$, from which the Hartree-Fock Hamiltonian $H^\text{HF}_m=H^\text{HF}[P_m]$ and energy $E_m=(1/2)\Tr(P_m[T+H^\text{HF}_m])$ are obtained. At the $m$-th iteration, the variational update is carried out as follows:
\begin{enumerate}
\item Diagonalize the Hartree-Fock Hamiltonian $H^\text{HF}_m=U EU^\dagger$, where $E=\text{diag}(E_1,E_2,\dots)$ with $E_1<E_2<\dots$ are the eigenvalues and $U=(u_1,u_2,\dots)$ are the eigenstates.
\item Assemble the density matrix $P_{m+1}=U D_NU^\dagger$, where $D_N=\text{diag}(1,1,\dots,1,0,0,\dots)$ selects the $N$ eigenstates with the lowest energies.
\item Compose $H^\text{HF}_{m+1}=H^\text{HF}[P_{m+1}]$.
\item Calculate the energy density $e_{m+1}=E_{m+1}/N_L$.
\item Check convergence: Stop if the error is small enough $|e_{m+1}-e_m|<\delta e$ and $|P_{m+1,ab}-P_{m,ab}|<\delta p$. We choose $\delta e,\delta p=10^{-15}$ in our computation, although the iteration may only achieve $\delta p=10^{-14}$ especially when the system size is large.
\item Adopt the optimal damping algorithm \cite{kudin02jcp} to accelerate the covergence.
\end{enumerate}
Our computation adopts a $12\times12\times3$ lattice with periodic boundary condition.  Larger lattices have also been examined as sanity check.

With the ground state obtained, the spin-charge (S-C) patterns can be studied by computing the densities
\begin{equation}
s^\nu_{ii'\tau\tau'}
=\sum_{\sigma\sigma'}\epvl{c_{i'\tau'\sigma'}^\dagger\tilde\sigma^\nu_{\sigma'\sigma}c_{i\tau\sigma}}
=\sum_{\sigma\sigma'}\epvl{c_{i'\tau'\sigma'}^\dagger c_{i\tau\sigma}}\tilde\sigma^\nu_{\sigma'\sigma}
=\sum_{\sigma\sigma'}P_{ii'\tau\tau'\sigma\sigma'}\tilde\sigma^\nu_{\sigma'\sigma}
=\Tr(P_{ii'\tau\tau'}\tilde\sigma^\nu).
\end{equation}
Here $P_{ii'\tau\tau'}=(P_{ii'\tau\tau'\sigma\sigma'})$ is a $2\times2$ density matrix in the spin space. The charge and spin densities $s^\nu=(s^0,\mbf s)$ are computed with the $2\times2$ identity and Pauli matrices $\tilde\sigma^\nu=(\sigma^0,\boldsymbol\sigma/2)$. The onsite and intersite densities correspond to the site and bond densities, respectively. While the site densities are real, the bond densities are generally complex. The real parts of the bond densities $\text{Re}(s^\nu_{ii'\tau\tau'})$ correspond to the ordinary charge and spin bond densities. Meanwhile, the imaginary parts $\text{Im}(s^\nu_{ii'\tau\tau'})$ are related to the charge and spin bond currents. To see this relation, we derive the charge bond current $j_{ab}$ from site $a$ to site $b$. The conservation law of site density at $b$ gives
\begin{equation}
\begin{aligned}
\sum_aj_{ab}
&=-\frac{d}{dt}\epvl{c_b^\dagger c_b}\\
&=-\frac{1}{i}\epvl{[c_b^\dagger c_b,H]}\\
&\approx-\frac{1}{i}\sum_{ac}\epvl{[c_b^\dagger c_b,-t_{ac}c_a^\dagger c_c]}\\
&=\frac{1}{i}\sum_{ac}t_{ac}\epvl{c_a^\dagger[c_b^\dagger c_b,c_c]+[c_b^\dagger c_b,c_a^\dagger]c_c}\\
&=\frac{1}{i}\sum_{ac}t_{ac}\epvl{c_a^\dagger(c_b^\dagger\{c_b,c_c\}-\{c_b^\dagger,c_c\}c_b)+(c_b^\dagger\{c_b,c_a^\dagger\}-\{c_b^\dagger,c_a^\dagger\}c_b)c_c}\\
&=\frac{1}{i}\sum_{ac}t_{ac}\epvl{-\delta_{bc}c_a^\dagger c_b+\delta_{ba}c_b^\dagger c_c}\\
&=\frac{1}{i}\sum_a(-t_{ab}\epvl{c_a^\dagger c_b}+t_{ba}\epvl{c_b^\dagger c_a})\\
&=\sum_a2\text{Im}(t_{ba}P_{ab}),
\end{aligned}
\end{equation}
the charge bond current is derived as
\begin{equation}
j_{ab}=2\text{Im}(t_{ba}s^0_{ab}).
\end{equation}
We have adopted a lowest-order approximation in line 3, where the higher-order term from the interaction is neglected. In fact, this turns out to be exact under the density-density interaction, since the higher-order term vanishes $\text{Im}(\epvl{c_a^\dagger c_b}\epvl{c_b^\dagger c_a})=\text{Im}|\epvl{c_a^\dagger c_b}|^2=0$. With the real hoppings $t_{ba}\in\mathbb R$, the charge bond currents are directly proportional to the imaginary part of the bond density $\text{Im}(s^0_{ab})$. The derivation of spin bond current is analogous. In the figures, we draw the site densities $s^\nu_{ii\tau\tau}$, real bond densities $\text{Re}(s^\nu_{ii'\tau\tau'})$, and imaginary bond densities $\text{Im}(s^\nu_{ii'\tau\tau'})$ to represent the S-C patterns.

% Figure environment removed

The mean-field renormalized band structure can also be obtained. This is simply done by diagonalizing the Hartree-Fock Hamlitonian in the momentum space, where a proper reduced Brillouin zone is chosen to match the periodicity. While the FM band structures have been presented in the main text, here we show the band structures under the AFMSt-CStP1 and $120^\circ$TAFM orders (Fig.~\ref{suppfig:hf}). Due to the enlarged periodicity, the number of bands is enlarged in the reduced Brillouin zone. The original flat bands are renormalized and acquire light dispersion under the symmetry-breaking orders.



\section{Miscellaneous ground states}


In the main text, we have presented the primary symmetry-breaking orders in the mean-field phase diagram in kagome flat bands. Here we list some more ground states that appear in the phase diagram (Fig.~\ref{suppfig:scp}). Note that the computation may obtain a nonuniform pattern if the ground state is metallic. This occurs especially when the Fermi level involves a flat segment, as is the case of the AFMSt-CStP1 order.

% Figure environment removed




\section{Effect of extended repulsion on symmetry-breaking orders}


In this section, we discuss further the effect of extended repulsion on the energetic favorability of ground states.


\subsection{Competition of charge orders}


% Figure environment removed

The primary effect of extended repulsion is the choice between three different types of charge orders (CO1,2,3) [Fig.~\ref{suppfig:cos}(a)]. Under onsite and nearest-neighbor repulsions, the three charge orders are degenerate. However, the degeneracy is lifted when further-neighbor repulsions are involved. For example, the second-, third-, fourth-neighbor repulsions favor CO1,2,3, respectively. This poses the question of how long we should allow the extended repulsion to reach in our computation. To see the interaction-range dependence of energetic favor among the three charge orders, we compute the interaction energy
\begin{equation}
E[n]=\sum_\tau\sum_{i'\tau'}U_{0i'\tau\tau'}n_{0\tau}n_{i'\tau'}
\end{equation}
in the three charge orders. Here the charge densities $n_{i\tau}$ are chosen as 0 and 1 for the lower- and higher-density sites, respectively. For the extended repulsion, we adopt the screening formula from the recent literature [Fig.~\ref{suppfig:cos}(b)] \cite{disante23prr}. The original Coulomb repulsion is inverse to the intersite distance $r$
\begin{equation}
U(r)\sim\frac{1}{r}.
\end{equation}
Under the local screening, the power-law scaling is screened by the typical extent of the Wannier functions $\delta$
\begin{equation}
\begin{aligned}
U_\text{Ohno}(r)\sim\frac{1}{\tilde r},\quad \tilde r=\sqrt{r^2+\delta^2}.
\end{aligned}
\end{equation}
The consideration of nonlocal screening further modifies the scaling to
\begin{equation}
U_\text{Ohno-Resta}(r)\sim
\begin{cases}
1/\frac{q_\text{TF}R}{\sinh[q_\text{TF}(R-\tilde r)]+q_\text{TF}\tilde r}\tilde r,&\tilde r<R,\\
1/\tilde r,&\tilde r\geq R.
\end{cases}
\end{equation}
Here $R$ is the ionic radius and $q_\text{TF}$ is the Thomas-Fermi wave vector. Note that the local-screening model only works for the large-gap insulators or semiconductors. Meanwhile, the nonlocal-screening model can describe the small-gap systems and metals by setting a large ionic radius. We compute the screened Coulomb potential in both scenarios and apply them to the interaction energy of the three charge orders [Fig.~\ref{suppfig:cos}(c)]. Here we adopt the parameters $\delta=0.75\text{\r{A}}$, $R=6\text{\r{A}}$, and $q_\text{TF}=0.75\text{\r{A}}^{-1}$, and the lattice constant $a=2.5\text{\r{A}}$. The results show that CO1 is the most frequent winner under the competition. Therefore, we admit the extended repulsion up to second neighbor in our computation, which selects CO1 among the three charge orders.

It is still interesting to examine what symmetry-breaking orders can appear under CO2 and CO3. The change of dominant charge order may affect the phase diagram significantly, especially at the half filling $n_\text{FB}=0$. As discussed in the main text, the dominant charge order determines the effective lower-density superlattice for the spin orders. We have explored the spin orders on the triangular superlattice from CO1. Meanwhile, CO2 and CO3 induce the square and kagome superlattices, respectively, which allow more possible symmetry-breaking orders to occur. Under the dominant CO2, the square superlattice (CSq) hosts the FM, NAFM, and AFMSt orders [Figs.~\ref{suppfig:cos}(d)]. On the other hand, the kagome superlattice in the dominant CO3 (CKa) can support the FM and $120^\circ$AFM [Figs.~\ref{suppfig:cos}(e)].


\subsection{Phase diagram}


To inspect the effects of longer-ranged Coulomb repulsion, we introduce the nearest- and second-neighbor repulsions $U_1=2U_2>0$ to the pure Hubbard model. These repulsions generally favor the charge orders, thereby driving the ground state through a series of phases with different S-C patterns.

At half filling $n_\text{FB}=0$, the FM ground state is a QBCP half semimetal. The large-$U_0$ regime \cite{ren21prl} is described by an effective spinless model \cite{zhu16prl,ren18prb}. Under the extended repulsion, the time-reversal-symmetry-breaking loop currents appear as a weak-coupling instability \cite{sun09prl}. This gaps out the QBCP and leads to the QAHFM \cite{ren21prl}, which expands a wide regime of small extended repulsion $U_{1,2}\ll U_0$. At a larger extended repulsion, the charge nematicity induces the FM-CStP1 order. Further increasing the extended repulsion leads to a series $90^\circ$AFM $\rightarrow$ $120^\circ$AFM $\rightarrow$ spiral $\rightarrow$ AFMSt away from the FM. Note that these magnetic orders occur on the lower-density triangular superlattice in the CStP1 order. The phase series may be related to the anisotropic Hubbard model \cite{szasz21prb} or certain equivalent spin models \cite{cookmeyer21prl} on the triangular lattice. On the other hand, there is an AFMSt-CDW order with a different stripe pattern, as well as a 5-higher-1-lower-density CDW matching the total fermion filling $n_f/2=5/6$.

The empty filling $n_\text{FB}=-1$ also presents a rich phase diagram, where additional phases occur under extended repulsion \cite{liu10prb,wen10prb}. In the large-$U_0$ regime, a large extended repulsion drives the AFM on a nematic CSt2P1 order. The AFM develops on the lower-density square superlattice, where both the N\'eel and stripe (NAFM and AFMSt) orders are observed. In the small-$U_0$ regime, we observe the degenerate quantum anomalous and spin Hall insulators (QAHI and QSHI). There also exists a CSt2P1 order, which transits to the CStP1 order at larger extended repulsion. Due to the occurrence of nonuniform patterns, it is difficult to accurately resolve the phase profile in the small-$U_0$ regime. On the other hand, we observe an exact transition line $U_0=3U_1=6U_2$ between the CStP1 and NAFM-CSt2P1 phases.




\section{Mean field ansatze from projective symmetry group (PSG)}

\subsection{Symmetry, convention and PSG}

% Figure environment removed

We set up the coordinate convention for the kagome lattice, see \ref{psg_lattice_setup}. We denote the length of the hexagon edge as $a$. We set up the coordinate system in the following way: the origin is at a hexagon center; the unit translations are $\bm{a}_1 = a\hat{x}-\sqrt{3} a \hat{y}$ and $\bm{a}_2 = a\hat{x}+\sqrt{3} a \hat{y}$. The reciprocal lattice vectors are then $\bm{b}_1 = \pi(1,1/\sqrt{3})$, $\bm{b}_2 = \pi(-1,1/\sqrt{3})$. The three sublattices $\tau=0,1,2$ have their origin at $(0,0,0)_0 = \bm{a}_1/2$, $(0,0,0)_1 = (\bm{a}_1+\bm{a}_2)/2$ and $(0,0,0)_2 = \bm{a}_2/2$. The symmetry group has generators $\{T_1,T_2,D,C_6\}$, where 
\begin{subequations}
\begin{align}
T_1\colon(r_1,r_2)_\tau&\rightarrow (r_1+1,r_2)_\tau\\
T_2\colon (r_1,r_2)_\tau&\rightarrow (r_1,r_2+1)_\tau\\
D \colon (r_1,r_2)_\tau&\rightarrow (r_2,r_1)_{2-\tau}\\
{C}_6\colon (r_1,r_2)_\tau&\rightarrow(
r1-r2-\delta_{\tau=2},r_1)_{\tau+1}
\end{align}
\end{subequations}
The group relations are  $T_1T_2T_1^{-1}T_2^{-1}=1$, ${C}^{-1}_6T_1{C}_6T_2=1$, ${C}^{-1}_6T_2{C}_6T_1^{-1}T_2^{-1}=1$, $DT_1DT_2^{-1}=1$, $DT_2DT_1^{-1}=1$, $({C}^{-1}_6D)^2=1$,  ${C}_6^6 =D^2=1$. 

The projective symmetry group (PSG) classification for $\mathbb{Z}_2$ spin liquids using Schwinger bosons on the kagome lattice has been done by Wang and Vishwanath in Ref.~\cite{PhysRevB.74.174423}. We summarize their results:
\begin{equation}\label{psgk}
\begin{aligned}
\phi_{T_1}(\vec{r}_\tau)&=0,\\
\phi_{T_2}(\vec{r}_\tau) &=n_1r_1\pi,\\
\phi_{{C}_6}(\vec{r}_\tau) &= \frac{n_{{C}_6}}{2}\pi
+n_1\left[r_1r_2+\frac{(r_1-1)r_1}{2}\right]\pi,\\
\phi_D(\vec{r}_\tau) &= \frac{n_D}{2}\pi + n_1 r_1r_2\pi,\\
\phi_{\mathcal{T}}(\vec{r}_\tau) & = 0,
\end{aligned}
\end{equation}
where the phase $\phi_{\mathcal{O}}(\vec{r}_\tau)$ for a symmetry operation $\mathcal{O}\in \{T_1,T_2,D,C_6\}$ appears in the projective symmetry transformation of Schwinger bosons:
\begin{equation}
\widetilde{\mathcal{O}}\colon
b_{\vec{r}_\tau}\rightarrow
e^{i\phi_{\mathcal{O}}(\mathcal{O}(\vec{r}_\tau))} U^\dag_{\mathcal{O}} b_{\mathcal{O}(\vec{r}_\tau)},
\end{equation}
here $b_{\vec{r}_\tau} = \Big(\begin{smallmatrix} b_{\vec{r}_\tau,\uparrow}\\b_{\vec{r}_\tau,\downarrow}\end{smallmatrix}\Big)$ is the two-component Schwinger boson annihilation operator,
$U_{\mathcal{O}}$ is the unitary (anti-unitary) matrix associated to a spatial (time-reversal) symmetry operation $\mathcal{O}$ that acts on the spin operator: $\mathcal{O}\colon \bm{S}\rightarrow U_{\mathcal{O}}\bm{S}U_{\mathcal{O}}^\dag$ or in the case of time-reversal operation $\mathcal{T}$, we have $\mathcal{T}\colon \bm{S}\rightarrow \mathcal{K}U_{\mathcal{T}}\bm{S}U_{\mathcal{T}}^\dag\mathcal{K}$ with $\mathcal{K}$ the complex conjugation operator.



\subsection{PSG to parton mean field ansatze}

The mean-field Hamiltonian symmetric under the projective operations $\widetilde{\mathcal{O}}$ is called a mean-field ansatz. Write the most general form of this Hamiltonian as
\begin{equation}
H = \sum_{i,j}
b^\dag_i u^h_{ij} b_j + b^\dag_i u^p_{ij} (b^\dag_j)^T +h.c.,
\end{equation}
where $i,j$ denote lattice sites that include both the unit cell and sublattice indices.
We assume the $(01)$ bond (see Fig.~\ref{psg_lattice_setup} for its definition) to be 
\begin{subequations}\label{uhup}
\begin{eqnarray}
u^h_{01} &=&  a\sigma^0 + i(b\sigma^1+c \sigma^2+d\sigma^3),\\
u^p_{01} &=& \left(a'\sigma^0 + i (b'\sigma^1+c'\sigma^2+d'\sigma^3)\right)\cdot i\sigma^2,
\end{eqnarray}
\end{subequations}
where $a,b,c,d$ and $a',b',c',d'$ are real due to time-reversal symmetry $\mathcal{T}$.

The order-two operation $D$ has well defined operation on the lattice sites, however its spin rotation has yet to be defined. To do this, we assume the kagome lattice  is a 2D reduction of a layered kagome lattice having space group P6/mmm and point group $D_{6h}\cong Dih_3\times \mathbb{Z}_2^2$. The P6/mmm is the symmetry for many kagome lattices, including the kagome compounds $A$V$_3$Sb$_5$ with $A=$ K, Rb, and Cs \cite{jiang21nm,zhao21n}.
The point group symmetry contains a six-fold axis $C_6$ along $z$ that goes through the hexagon center (the origin), a two-fold axis that is parallel to the origin and $(0,0,0)_{0,1,2}$, and also a mirror in the plane spanned by $\hat{z}$ and the origin and $(0,0,0)_{0,1,2}$. Since the V atom sits on the 3g Wyckoff position whose location is $(1/2,0,1/2)$, $(0,1/2,1/2)$ and $(1/2,1/2,1/2)$, the symmetry $D$ should be interpreted as a mirror reflection and not two-fold rotation (the two-fold rotation would map between layers at $z=1/2$ and $z=-1/2$ so it is not a symmetry of a single kagome layer). 

Now we write done the constraints on the bond parameters from the projective symmetry transformation. For nearest-neighbor (NN) bonds: using the fact that the $01$ bond is mapped back to itself under ${C}^{-1}_6D$, we get
\begin{subequations}
\begin{align}
\left(a,-\left(-\frac{b}{2}+\frac{\sqrt{3} c}{2}\right),-\left(\frac{\sqrt{3} b}{2}+\frac{c}{2}\right),-(-d)\right) &= (a,b,c,d),\\
(-1)^{n_{{C}_6}+n_D}\left(-a',-\frac{b'}{2}+\frac{\sqrt{3} c'}{2},\frac{\sqrt{3}b'}{2}+\frac{c'}{2},-d'\right)
&=(a',b',c',d'),
\end{align}
\end{subequations}

For next-nearest-neighbor (NNN) bonds: there are six independent bonds not related by translations. Setting the bond $(0,0)_2\leftrightarrow (0,0)_0$ to have the bond parameters $(A,B,C,D)$ and $(A',B',C',D')$, this bond is mapped back (to its hermitian conjugate) by $D$. We have the condition
\begin{subequations}
\begin{align}
(A,B,C,D) &= (A,B,-C,D),\\
(A',B',C',D') & = (-1)^{n_D}(-A',-B',C',-D').
\end{align}
\end{subequations}


These are obtained from the transformation rule for the bonds. The solutions are given in Table \ref{table:hoppair}.

\begin{table}[!thb]
\centering
\caption{Hopping and pairing parameters. The onsite bond is for the site $(0,0,0)_1$, and parameters shown are the \emph{allowed} ones; the NN bond is for the bond $(0,0,0)_1\leftarrow (0,0,0)_0$, and the relations shown are the \emph{constraints}; the NNN bond is for the bond $(0,0,0)_2\leftarrow (0,0,0)_0$ and the relations shown are the \emph{constraints}.}\label{table:hoppair}
\begin{tabular}{c|c|l|l}
\hline\hline
$n_1\pi$-$(n_{{C}_6},n_D)$ & Onsite & NN & NNN\\
\hline
0-(0,0)&$\mu$,$-$&$c=-\frac{b}{\sqrt{3}}$, $a'=0,c'=\sqrt{3}b',d'=0$&$C=0$, $A'=B'=D'=0$\\
0-(0,1)&$\mu$,$\delta'$&$c=-\frac{b}{\sqrt{3}}$, $c'=-\frac{b'}{\sqrt{3}}$&$C=0$, $C'=0$\\
0-(1,0)&$\mu$,$\gamma'$&$c=-\frac{b}{\sqrt{3}}$, $c'=-\frac{b'}{\sqrt{3}}$&$C=0$, $A'=B'=D'=0$\\
0-(1,1)&$\mu$,$\beta'$&$c=-\frac{b}{\sqrt{3}}$, $a'=0,c'=\sqrt{3}b',d'=0$&$C=0$, $C'=0$\\
$\pi$-(0,0)&$\mu$,$-$&$c=-\frac{b}{\sqrt{3}}$, $a'=0,c'=\sqrt{3}b',d'=0$&$C=0$, $A'=B'=D'=0$\\
$\pi$-(0,1)&$\mu$,$\delta'$&$c=-\frac{b}{\sqrt{3}}$, $c'=-\frac{b'}{\sqrt{3}}$&$C=0$, $C'=0$\\
$\pi$-(1,0)&$\mu$,$\gamma'$&$c=-\frac{b}{\sqrt{3}}$, $c'=-\frac{b'}{\sqrt{3}}$&$C=0$, $A'=B'=D'=0$\\
$\pi$-(1,1)&$\mu$,$\beta'$&$c=-\frac{b}{\sqrt{3}}$, $a'=0,c'= \sqrt{3}b',d'=0$&$C=0$, $C'=0$\\
\hline
\hline
\end{tabular}
\end{table}

Using this information one can write down the tight binding model for the partons symmetric under PSG. 
%At the onsite+NN level the parameter $n_D$ does not regulate the Hamitonian. 
In the following we restrict ourselves to the zero-flux states, i.e. $n_1=0$.

Recall that the bosonic Bogoliubov--de Gennes (BdG) Hamiltonian are of the generic form
\begin{equation}\label{Hh}
H = \sum_{\bm{k}} B^\dag_{\bm{k}}\mathcal{H}(\bm{k}) B_{\bm{k}},
\end{equation}
where we used the basis $B_{\bm{k}} = (b_{\bm{k},0},b_{\bm{k},1},b_{\bm{k},2},b^\dag_{-\bm{k},0},b^\dag_{-\bm{k},1},b^\dag_{-\bm{k},2})^T$, where each $b_{\bm{k},\tau}$ has two components labeled by spin $\sigma=\uparrow,\downarrow$. The Hamiltonian matrix $\mathcal{H}(\bm{k})$ reads

\begin{equation}\label{Hamiltonianden}
\mathcal{H}(\bm{k}) = \left(\begin{array}{cc} H_h(\bm{k}) & H_p(\bm{k})\\
H^\dag_p(\bm{k}) & H^T_h(-\bm{k})\end{array}\right),
\end{equation}
where $H_h(\bm{k}) = H^\dag_h(\bm{k})$ and $H_p(\bm{k}) = H^T_p(-\bm{k})$. 

We have, for $n_1=0$,
\begin{subequations}
\begin{align}
U^\dag_{{C}_6}\mathcal{H}(\bm{k}) U_{{C}_6}&=
\mathcal{H}({C}_6(\bm{k})),\\
U^\dag_{D}\mathcal{H}(\bm{k}) U_{D}&=
\mathcal{H}(D(\bm{k})),\\
U^\dag_{\mathcal{T}} \mathcal{H}(\bm{k}) U_{\mathcal{T}}&=\mathcal{H}^*(-\bm{k}),\\
U^\dag_{\mathcal{C}}\mathcal{H}(\bm{k}) U_{\mathcal{C}}&=\mathcal{H}^*(-\bm{k}),
\end{align}
\end{subequations}
where $C_6(\bm{k}) = ( \frac{1}{2}k_x -\frac{\sqrt{3}}{2}k_y,\frac{\sqrt{3}}{2}k_x + \frac{1}{2}k_y)$ and $D(\bm{k}) = (k_x,-k_y)$, and
\begin{equation}
U_{{C}_6} = \left(\begin{array}{cc} e^{-i\frac{\pi}{2} n_{{C}_6}} V_{{C}_6}&0\\0& e^{i\frac{\pi}{2} n_{{C}_6}} V^*_{{C}_6}\end{array}\right),\qquad
U_{D} = \left(\begin{array}{cc} e^{i\frac{\pi}{2} n_{D}} V_{D}&0\\0& e^{-i\frac{\pi}{2} n_{D}} V^*_{D}\end{array}\right),
\end{equation}
with
\begin{equation}
V_{{C}_6}= \left(\begin{array}{ccc}&1&\\&&1\\1&&\end{array}\right)
\otimes \left(e^{-i\frac{\pi}{6} \sigma^3}\right)^\dag,\qquad
V_{D}= \left(\begin{array}{ccc}&&1\\&1&\\1&&\end{array}\right)
\otimes \left(e^{-i\frac{\pi}{2} \sigma^2}\right)^\dag,
\end{equation}
and for the time reversal symmetry and bosonic particle-hole symmetry we have
\begin{equation}
U_{\mathcal{T}} = 1_{6\times 6}\otimes i \sigma^2,\qquad
U_{\mathcal{C}} = \left(\begin{array}{cc} & 1_{6\times 6}\\1_{6\times 6}\end{array}\right).
\end{equation}




Since there are too many parameters, let's restrict to only nonzero singlet terms and set all the triplet terms to be zero: $\beta=\gamma=\delta=\beta'=\gamma'=\delta'=b=c=d=b'=c'=d'=0$. Under this simplification, the four $0$-flux ansatze reduce to just two: $0-(0,1)$ and $0-(1,0)$ (the $0-(0,0)$ and $0-(1,1)$ classes are included in them). 

As an example, let us present the class $0-(0,0)$. This class reduces to a U(1) PSG class when restricting to a spin-singlet ansatz, i.e. $H_p(\bm{k}) = 0$, and we have the hopping part
\begin{equation}\label{hopping_00}
H_h(\bm{k})=
\begin{pmatrix}
 \mu  & a \cos \frac{k_x+\sqrt{3} k_y}{2}+ A \cos \frac{3k_x-\sqrt{3} k_y}{2} & a \cos k_x+A\cos \sqrt{3} k_y \\
 a \cos \frac{k_x+\sqrt{3} k_y}{2}+A \cos\frac{3 k_x - \sqrt{3}k_y}{2} & \mu  & a \cos\frac{k_x-\sqrt{3} k_y}{2}+A\cos \frac{3k_x+\sqrt{3}k_y}{2} \\
 a \cos k_x +A\cos \sqrt{3}k_y& a \cos \frac{k_x-\sqrt{3} k_y}{2}+A \cos \frac{3k_x+\sqrt{3}k_y}{2} & \mu  \\
\end{pmatrix}\otimes \sigma^0.
\end{equation}
Note that the NN hopping part of the Hamiltonians for the $0-(0,1)$ and $0-(1,0)$ classes, $H_h(\bm{k})$ is identical to the one in Eq.~\eqref{hopping_00}. Below we present the pairing part of the Hamiltonian. For the 0-(0,1) class,  the pairing part is
\begin{equation}
\begin{aligned}
&H_p(\bm{k})=\\
& \begin{pmatrix}
 0 & a' \cos \frac{k_x+\sqrt{3} k_y}{2} -A'  \cos \frac{3k_x-\sqrt{3}k_y}{2}& -a'\cos k_x+A'\cos \sqrt{3}k_y \\
 -a'\cos \frac{k_x+\sqrt{3}k_y}{2} + A' \cos \frac{3k_x-\sqrt{3}k_y}{2} & 0 & a' \cos \frac{k_x-\sqrt{3} k_y}{2}-A' \cos \frac{3k_x+\sqrt{3}k_y}{2} \\
 a' \cos k_x-A'\cos \sqrt{3}k_y & -a'\cos \frac{k_x-\sqrt{3}k_y}{2}+A'\cos \frac{3k_x+\sqrt{3}k_y}{2} & 0 \\
\end{pmatrix}\otimes i \sigma^2.
\end{aligned}
\end{equation}

For the class 0-(1,0),  the pairing part is
\begin{equation}
H_p(\bm{k}) = 
\begin{pmatrix}
 0 & ia' \sin \frac{k_x+\sqrt{3} k_y}{2} & -ia'\sin k_x \\
 ia'\sin \frac{k_x+\sqrt{3}k_y}{2} & 0 & ia' \sin \frac{k_x-\sqrt{3} k_y}{2} \\
 -ia' \sin k_x & ia'\sin \frac{k_x-\sqrt{3}k_y}{2} & 0 \\
\end{pmatrix}\otimes i \sigma^2.
\end{equation}

Below we will analyze the magnetic orders that border these spin liquids. To study them, we bring down the bosonic spinon band by tuning the chemical potential until the bosonic spinons at the band bottom start to condense. The condensation order parameter for the bosonic spinons will determine the type of magnetic order that borders the spin liquid phase. Our focus will be on the $0-(0,1)$ and $0-(1,0)$ classes. 
The phase diagrams 
are shown in Figs.~\ref{fig:phase_diagram_01_NNN} and \ref{fig:phase_diagram_10_NNN}. The magnetic orders that appear in the phase diagram are summarized in Table \ref{Tab:mag_order_psg} in the main text. The spinon dispersion for representative ansatze are given in Fig.~\ref{dispersionsPSG}.

% Figure environment removed

% Figure environment removed

% Figure environment removed



\section{Magnetic phase diagram from parton mean field ansatze}

Before diving into the detailed analysis of the magnetic orders, we first present two different interpretations and treatments possible for the parton mean field ansatze.

In the usual treatment of Schwinger boson mean field theory for spin liquids, one assumes a spin model (typically of Heisenberg type with full SO(3) spin symmetry) whose mean-field decomposition gives the Schwinger boson mean field Hamiltonian. This has been done for the kagome $J_1$-$J_2$ model, see Refs.~\cite{messio12prl,rossi23ax}. Here, instead of following this treatment, we interpret the Schwinger boson mean field Hamiltonian as an \emph{effective} Hamiltonian for spinons. We do not specify the underlying spin Hamiltonian, and do not require the spinon Hamiltonian to be self-consistently determined; rather, the single spin occupancy is to be enforced by applying Gutzwiller projection to the spinon states, and we make the crucial assumption that the magnetic order and spin liquid phases of the unprojected states persist after the projection. In this treatment, the spinon hoppings and pairings are taken as free parameters, and we focus on the magnetic phase diagram spanned by these parameters, which is obtained by bring down the spinon gap to zero at each point in this hopping and pairing parameter phase space. We note that this treatment allows us to discover many more magnetic orders from the Schwinger boson mean field theory than the self-consistent treatment used in previous works \cite{messio12prl,rossi23ax}, and further reveal a distinction between the two $\mathbb{Z}_2$ spin liquid classes --- the $0-(0,1)$ class and the $0-(1,0)$ class, through the magnetic orders that border the spin liquid phase.


\subsection{Class $0-(0,1)$}


We start from a large chemical potential $\mu$ such that the bosonic parton spectrum is gapped, and then lower $\mu$ until the spectrum becomes gapless and the bosonic parton begins to condense. 

If we only consider NN bonds, we find that $\Gamma=(0,0)$ in the Brillouin zone is always among the condensation momenta. At fine tuned parameter points (e.g. $(a,a')=(1,0)$ or $(a,a') = (0,1)$) there are other condensation momenta (the condensation manifold is the whole Brillouin zone for the former and the line $k_x=0$, $k_x = \pm \sqrt{3}k_y$ for the latter). The critical chemical potential is 
\begin{equation}\label{case1case2}
\mu = \max (a+\sqrt{3}|a'|,-2a) = \left\{\begin{array}{ll}-2a,&a <  -\frac{|a'|}{\sqrt{3}},\\a+\sqrt{3}|a'|,&a> -\frac{|a'|}{\sqrt{3}}\end{array}\right..
\end{equation}

If we include NNN bonds, we find the condensations can be $\Gamma$, $\pm\mathrm{K}$, and $\mathrm{M}_{1,2,3}$. The critical chemical potential is
\begin{equation}
\mu = \max\Big(\underbrace{-2(a+A),a+A+\sqrt{3}|a'-A'|}_{\Gamma},
\underbrace{a-2A,\frac{1}{2}(-a+2A+\sqrt{3}|a'+2A'|)}_{\pm\mathrm{K}},
\underbrace{\sqrt{(a-A)^2+(a'+A')^2}}_{\mathrm{M}_{1,2,3}}\Big),
\end{equation}
the corresponding parton condensation momenta are also given. We will call the above cases 1,2,3,4,5, where cases 1 and 2 have condensation momentum at $\Gamma$, cases 3 and 4 at $\pm \mathrm{K}$, and case 5 at $\mathrm{M}_{1,2,3}$. We will also use the parameterization 
\begin{equation}\label{class01params}
a=\sin\psi \sin \theta \cos \phi,\quad a'=\sin \psi\sin \theta \sin\phi, \quad A=\sin \psi\cos \theta,\quad A'=\cos\psi,
\end{equation}
and using this parameterization the condensation diagram is given in Fig.~\ref{fig:phase_diagram_01_NNN}.

If we keep the parameterization of the bond parameters as normalized in Eq.~\eqref{class01params}, the critical chemical potential for the whole phase diagram is shown in Fig.~\ref{fig:phase_diagram_01_NNN_critical_mu}. The critical chemical potential gives us a rough sense of which magnetic order in the condensation diagram Fig.~\ref{fig:phase_diagram_01_NNN} starts to form the first: keeping a global magnitude of the bond parameters  normalized to Eq.~\eqref{class01params}, the global maximum for the critical chemical potential determines the magnetic order. As we see, by comparing Fig.~\ref{fig:phase_diagram_01_NNN} and Fig.~\ref{fig:phase_diagram_01_NNN_critical_mu}, the largest critical chemical potential (the red spots on the right plots) always corresponds to the purple or blue phases, which are case 1 (collinear FM order) and case 2 ($120^\circ$ coplanar order). Then conceptually this means these two orders tend to form before the rest orders (yellow, red and green) do, and in this sense we can replace the yellow, red and green phases by the parent spin liquid class (0,1) to arrive at a phase diagram with both magnetic order and the parent spin liquid.

The above analysis should be treated with caveats. First, the normalization \eqref{class01params} is only a convenient way to plot the phase diagram and should not be taken for a realistic constraint. A global maximum critical chemical potential is meaningful only when this normalization constraint is retained even when varying the interaction strength. Second, the phase diagram itself is not a real phase diagram, as the bond parameters should in principle be determined self consistently.


% Figure environment removed

For bosonic BdG type Hamiltonian, it is known that the eigenvalues and eigenvectors are obtained by diagonalizing $J\mathcal{H}(\bm{k})$, where $J = \left(\begin{smallmatrix}
1_{6\times 6}&\\&-1_{6\times 6}\end{smallmatrix}\right)$. The zero energy levels are at least two-fold degenerate due to time reversal symmetry $\mathcal{T}$. The bosonic particle-hole $\mathcal{C}$ symmetry may or may not contribute an extra double degeneracy, and depending on these two scenarios the low energy field theory can be different. As we will see, case 1 and case 2 correspond to these two different scenarios.


\subsubsection{Case 1: $\Gamma$ condensation}

At $\Gamma$ the zero-energy eigenstates are four-fold degenerate due to time reversal symmetry $\mathcal{T}$ and particle-hole symmetry $\mathcal{C}$:
\begin{equation}
\mathbf{a},\quad U_{\mathcal{T}}\mathbf{a}^*,\quad U_{\mathcal{C}}\mathbf{a}^*,\quad U_{\mathcal{C}}U_{\mathcal{T}}\mathbf{a},
\end{equation}
where 
\begin{equation}\label{simplesta}
\mathbf{a} = (1,0,1,0,1,0,0,0,0,0,0,0),
\end{equation}
note that this belongs to the scenario where the bosonic particle-hole symmetry leads to extra double degeneracy.

Generally, the spin configuration is computed from the parton condensation as
\begin{equation}\label{spinConf}
S^i_{\tau} \propto \left(\begin{array}{cc} \langle b^\dag_{\tau\uparrow}\rangle & \langle b^\dag_{\tau\uparrow}\rangle \end{array}\right) \sigma^i \left(\begin{array}{c} \langle b_{\tau\uparrow}\rangle\\\langle b_{\tau\downarrow}\rangle\end{array}\right),
\end{equation}
where $i=x,y,z$ are the spin components, and $\tau=0,1,2$ denote the three sublattices. For each $\tau$, we have $ \langle b_\tau\rangle = (c_1\mathbf{a}+c_2U_{\mathcal{T}}\mathbf{a}^*)_{2\tau:2\tau+1}$ where $c_1$ and $c_2$ are arbitrary complex number and the notation $2\tau:2\tau+1$ denotes the extraction of the $2\tau$ and $2\tau+1$ components of the vector. We can then write the spin as (suppressing $\tau$)
\begin{equation}\label{spinconinspecialcase}
\bm{S} = \left(|c_1|^2-|c_2|^2\right) \bm{n}_1+2\mathrm{Re}(c^*_1c_2) \bm{n}_2-2\mathrm{Im}(c^*_1c_2) \bm{n}_3,
\end{equation}
where 
\begin{equation}\label{triad}
\begin{aligned}
\bm{n}_1 &= (2\mathrm{Re}(a^*b),2\mathrm{Im}(a^*b), |a|^2-|b|^2),\\
\bm{n}_2 &= (\mathrm{Re}(-a^{*2}+b^{*2}),-\mathrm{Im}(a^{*2}+b^{*2}),2\mathrm{Re}(a^*b^*)),\\
\bm{n}_3 &= (\mathrm{Im}(-a^{*2}+b^{*2}),\mathrm{Re}(a^{*2}+b^{*2}),2\mathrm{Im}(a^*b^*)).
\end{aligned}
\end{equation}
where $\bm{n}_1,\bm{n}_2$ and $\bm{n}_3$ are mutually orthogonal, and that $|\bm{n}_{1,2,3}|^2=(|a|^2+|b|^2)^2$, 
so we have $|\bm{S}|^2=(|a|^2+|b|^2)^2(|c_1|^2+|c_2|^2)^2$. Apply this to Case 1, we see that it gives the collinear ferromagnetic order.


\subsubsection{Case 2: $\Gamma$ condensation} %($a >  -\frac{|a'|}{\sqrt{3}}$)}

Let us first consider the case of $a'-A'>0$. Then at $\Gamma$ point there are again four eigenstates
\begin{equation}
\mathbf{a}_1,\quad U_{\mathcal{T}}\mathbf{a}_1^*,\quad \mathbf{a}_2,\quad U_{\mathcal{T}}\mathbf{a}_2^*,
\end{equation}
where
\begin{equation}\label{a1a2form}
\mathbf{a}_1 = \left(\frac{\sqrt{3}}{2},-\frac{1}{2},-\frac{\sqrt{3}}{2},-\frac{1}{2},0,1,\frac{\sqrt{3}}{2},-\frac{1}{2},-\frac{\sqrt{3}}{2},-\frac{1}{2},0,1\right),\quad
\mathbf{a}_2 = \left(\frac{i\sqrt{3}}{2},\frac{i}{2},-\frac{i\sqrt{3}}{2} ,\frac{i}{2},0,-i,-\frac{i\sqrt{3}}{2},-\frac{i}{2},\frac{i\sqrt{3}}{2},-\frac{i}{2},0,i\right).
\end{equation}
Note that now $U_{\mathcal{C}}\mathbf{a}^*_{1,2} = \mathbf{a}^*_{1,2}$, so this belongs to the ``nondiagonalizable'' case, whose low energy dispersion is linear ($z=1$). We have
\begin{equation}
\langle b \rangle = r_1 \mathbf{a}_1 + r_2 U_{\mathcal{T}}\mathbf{a}_1^* + r_3 \mathbf{a}_2 + r_4 U_{\mathcal{T}}\mathbf{a}_2^*,
\end{equation}
where $r_{1,2,3,4}$ are all real. Using Eq.~\eqref{spinConf} to get the spin configuration, we get
\begin{subequations}
\begin{align}
\bm{S}_1&=
\Big(\frac{\sqrt{3} (-r_1^2+r_2^2+r_3^2-r_4^2)-2 (r_1 r_2-r_3r_4)}{4},\frac{ \sqrt{3}(r_1r_3-r_2r_4)+r_1r_4+r_2r_3}{2},\frac{r_1^2-r_2^2+r_3^2-r_4^2-2 \sqrt{3}(r_1 r_2+ r_3 r_4)}{4}\Big),\\
\bm{S}_2&= \Big(\frac{\sqrt{3} (r_1^2-r_2^2-r_3^2+r_4^2)-2 (r_1 r_2-r_3r_4)}{4},\frac{-\sqrt{3} (r_1r_3-r_2r_4)+r_1 r_4+r_2r_3}{2},\frac{r_1^2-r_2^2+r_3^2-r_4^2 +2 \sqrt{3} (r_1 r_2+ r_3 r_4)}{4}\Big),\\
\bm{S}_3&= \Big(r_1 r_2-r_3 r_4,-r_1 r_4-r_2 r_3,\frac{-r_1^2+r_2^2-r_3^2+r_4^2}{2}\Big)
\end{align}
\end{subequations}
We have $|\bm{S}_{1,2,3}|^2 = \frac{1}{4}(r_1^2 + r_2^2 + r_3^2 + r_4^2)^2$, $\bm{S}_1\cdot \bm{S}_2 = \bm{S}_2 \cdot \bm{S}_3 = \bm{S}_1 \cdot \bm{S}_3 = -\frac{1}{8}(r_1^2 + r_2^2 + r_3^2 + r_4^2)^2$ and $\bm{S}_1\cdot (\bm{S}_2\times \bm{S}_3 ) =0$, so this describes a three sublattice coplanar $120^\circ$ order. 


We can similarly consider $a'<0$. In this case, instead of the eigenstates \eqref{a1a2form} we have
\begin{equation}
\mathbf{a}_1 = \Big(-\frac{\sqrt{3}}{2},\frac{1}{2},\frac{\sqrt{3}}{2},\frac{1}{2},0,-1,\frac{\sqrt{3}}{2},-\frac{1}{2},-\frac{\sqrt{3}}{2},-\frac{1}{2},0,1\Big),\quad
\mathbf{a}_2 = \Big(-\frac{i\sqrt{3}}{2},-\frac{i}{2},\frac{i\sqrt{3}}{2} ,-\frac{i}{2},0,i,-\frac{i\sqrt{3}}{2},-\frac{i}{2},\frac{i\sqrt{3}}{2},-\frac{i}{2},0,i\Big),
\end{equation}
i.e. the first six components of each eigenvector have flipped signs. It turns out that this will also give a three sublattice coplanar $120^\circ$ order, but with the \emph{opposite vector spin chirality} to the $a'>0$ case. 

\subsubsection{Case 3, K point}

Both the Hamiltonian $\mathcal{H}(\bm{k})$ and the nonhermitian one $J\mathcal{H}(\bm{k})$ have four zero eigenvalues at $\mathrm{K} = (\frac{2\pi}{3},0)$. This belongs to the ``diagonalizable'' case, whose low energy gapless dispersion is quadratic (having dynamical exponent $z=2$). This has been checked numerically. The null vectors at both momenta $\pm \mathrm{K}$ are
\begin{equation}
\mathbf{b},\quad U_{\mathcal{T}}\mathbf{b}^*,\quad U_{\mathcal{C}}\mathbf{b}^*,\quad U_{\mathcal{C}}U_{\mathcal{T}}\mathbf{b},
\end{equation}
where 
\begin{equation}\label{simplestbb}
\mathbf{b} = (1,0,-1,0,1,0,0,0,0,0,0,0).
\end{equation}
Note that the null vector $U_{\mathcal{T}}\mathbf{b}^*$ is not the time reversal partner of $\mathbf{b}$; $\mathbf{b}$ and $U_{\mathcal{T}}\mathbf{b}^*$ are just treated as two different null vectors. We have
\begin{equation}
\langle b_{\bm{r}}\rangle = e^{i \phi_{\bm{r}} }
(c_1\mathbf{b}_{1:6}+c_2 (U_{\mathcal{T}}\mathbf{b}^*)_{1:6})+e^{-i \phi_{\bm{r}} }
(c_3\mathbf{b}_{1:6}+c_4 (U_{\mathcal{T}}\mathbf{b}^*)_{1:6}),
\end{equation}
where we defined $\phi_{\bm{r}} = \left(\frac{2\pi}{3},0\right)\cdot \bm{r}$, and $c_1=x_1+ix_2,c_2=x_3+ix_4,c_3=x_5+ix_6,c_4=x_7+ix_8$ are arbitrary complex numbers. The spin order is then of the form
\begin{equation}
\bm{S}_{\bm{r}_\tau}
=\bm{n} + \bm{m}_1 \cos \phi_{\bm{r}} + \bm{m}_2 \sin \phi_{\bm{r}},
\end{equation}
where
\begin{equation}
\begin{aligned}
\bm{n} &= \Big(x_{1} x_{3}+x_{2} x_{4}+x_{5} x_{7}+x_{6} x_{8},x_{1} x_{4}-x_{2} x_{3}+x_{5} x_{8}-x_{6} x_{7},\frac{-x_{1}^2-x_{2}^2+x_{3}^2+x_{4}^2-x_{5}^2-x_{6}^2+x_{7}^2+x_{8}^2}{2}\Big),\\
\bm{m}_1 &= \left(x_{1} x_{7}+x_{2} x_{8}+x_{3} x_{5}+x_{4} x_{6},x_{1} x_{8}-x_{2} x_{7}-x_{3} x_{6}+x_{4} x_{5},-x_{1} x_{5}-x_{2} x_{6}+x_{3} x_{7}+x_{4} x_{8}\right),\\
\bm{m}_2 &= \left(x_{1} x_{8}-x_{2} x_{7}+x_{3} x_{6}-x_{4} x_{5},-x_{1} x_{7}-x_{2} x_{8}+x_{3} x_{5}+x_{4} x_{6},-x_{1} x_{6}+x_{2} x_{5}+x_{3} x_{8}-x_{4} x_{7}\right).
\end{aligned}
\end{equation}
The above notation means that for a fixed unit cell say $\bm{r} = (0,0)$, the three spins on the small triangle labeled by $\tau=0,1,2$ are collinear; however, the neighboring triangle belonging to the $\bm{r}=(1,0)$ cell has $\phi_{\bm{r}} = -\frac{2\pi}{3}$, and the three spins on this triangle ($\bm{r}=(1,0)$) are also collinear but the orientation and amplitude are different from those on the $\bm{r}=(0,0)$ triangle. The order contains both a FM part and a $120^\circ$ part. 


We call it the $\sqrt{3}\times \sqrt{3}$, nonuniform umbrella, sublattice-uniform order. 



\subsubsection{Case 4, K point}\label{supp:psg:case_4}

This belongs to the ``nondiagonalizable'' case with linear dispersion ($z=1$). To obtain the expectation value of the bosonic spinons we following the procedure as follows: for the Hamiltonian
\begin{equation}
H = \left(\begin{array}{cc} b^\dag_{\bm{k}} & b_{-\bm{k}}\end{array}\right)
\mathcal{H}(\bm{k}) \left(\begin{array}{c} b_{\bm{k}} \\ b^\dag_{-\bm{k}}\end{array}\right),
\end{equation}
whenever $z=1$, or equivalently, whenever there are $2m$ zero eigenvalues but the nullspace of $\mathcal{H}$ is $m$-dimensional, then, we use basis transformation to get into the position-momentum operator space of $(x_{\bm{k}},x_{-\bm{k}},p_{\bm{k}},p_{-\bm{k}})$:
\begin{equation}
\mathcal{H}_r = \frac{1}{\sqrt{2}} \left(\left( \begin{array}{cccc} 1 &0 & i &0\\ 0 & 1 & 0 & -i\end{array}\right)\otimes 1_{6\times 6}\right)^T \mathcal{H}
\frac{1}{\sqrt{2}}\left( \begin{array}{cccc} 1 &0 & i &0\\ 0 & 1 & 0 & -i\end{array}\right)\otimes 1_{6\times 6};
\end{equation}
The matrix $\mathcal{H}_r$ has a dimension twice that of $\mathcal{H}$. Then, we solve the nullspace of $\mathcal{H}_r$, which should give precisely $2m$ distinct eigenvectors $\mathbf{a}_{1,...,2m}$, and these are the gapless modes. In the $z=1$ case we expect the conjugate operators of these gapless modes are gapped, which we denote by $\mathbf{b}_{1,...,2m}$. And the conjugate operators can be obtained by $\mathcal{E} \mathcal{H}_r \mathbf{b}_i = \lambda_i \mathbf{a}_i$, $i = 1,2,...,2m$, but this is irrelevant to the expectation value of bosonic spinons. 

Since the $\mathbf{b}_i$'s are gapped, we assume they correspond to modes that having a zero expectation value in those operators; but the conjugate operators corresponding to $\mathbf{a}_i$'s are maximally fluctuating since they are gapless, Therefore they acquire some expectation value (indefinite), say $c_{1,...,2m}$. The expectation value for the $b$ operators are then
\begin{equation}
\left(\begin{array}{c}\langle b_{\bm{k}}\rangle  \\ \langle b^\dag_{-\bm{k}}\rangle \end{array}\right)
=\frac{1}{\sqrt{2}}\left(\begin{array}{cccc} 1 & 0 & i & 0\\ 0 & 1 & 0 & -i\end{array}\right)\otimes 1_{6\times 6} \left(\begin{array}{ccc} \mathbf{a}_1 & \cdot & \mathbf{a}_{2m}\end{array}\right)\left(\begin{array}{c} c_1 \\ \vdots \\ c_{2m}\end{array}\right),
\end{equation}
where $c_{1,...,2m}$ are real. 

Following the above procedure, we get
\begin{equation}
\left(\begin{array}{c}\langle b_{(\frac{2\pi}{3},0)}\rangle  \\ \langle b^\dag_{-(\frac{2\pi}{3},0)}\rangle \end{array}\right) = (x_1+ix_2)\mathbf{a}_1+(x_3+ix_4)\mathbf{a}_2+(x_5+ix_6)\mathbf{a}_3+(x_7+i x_8)\mathbf{a}_4,
\end{equation}
where $x_{1,2,...,8}$ are arbitrary real parameters denoting the real and imaginary parts of the condensation, and
\begin{equation}\label{simplestbb2}
\begin{aligned}
\mathbf{a}_1 &= \left(\frac{i}{\sqrt{3}},0,\frac{2 i}{\sqrt{3}},0,\frac{i}{\sqrt{3}},0,0,i,0,0,0,-i\right),\quad
\mathbf{a}_2 = \left(0,-\frac{i}{\sqrt{3}},0,-\frac{2 i}{\sqrt{3}},0,-\frac{i}{\sqrt{3}},i,0,0,0,-i,0\right),\\
\mathbf{a}_3 &=\left(\frac{i}{\sqrt{3}},0,-\frac{i}{\sqrt{3}},0,-\frac{2 i}{\sqrt{3}},0,0,-i,0,-i,0,0\right),\quad
\mathbf{a}_4 =\left(0,-\frac{i}{\sqrt{3}},0,\frac{i}{\sqrt{3}},0,\frac{2 i}{\sqrt{3}},-i,0,-i,0,0,0\right).
\end{aligned}
\end{equation}
Using the real space parton condensation $
\langle b_{\bm{r}}\rangle = e^{i \phi_{\bm{r}} } \langle b_{(\frac{2\pi}{3},0)}\rangle  + e^{-i \phi_{\bm{r}} } \langle b_{(-\frac{2\pi}{3},0)}\rangle$ to write the spin order, we obtain 
\begin{equation}
\bm{S}_{\bm{r}_\tau}
=\bm{n}_\tau + \bm{m}_{1,\tau} \cos 2\phi_{\bm{r}} + \bm{m}_{2,\tau} \sin 2\phi_{\bm{r}},
\end{equation}
where $\bm{n}_\tau$, $\bm{m}_{1,\tau}$ and $\bm{m}_{2,\tau}$ have the expression
\begingroup
\allowdisplaybreaks
\begin{align}
\bm{n}_0&= \left(\frac{1}{3} (x_{1} (x_{3}-2 x_{7})+x_{2} (x_{4}-2 x_{8})-2 x_{3} x_{5}-2 x_{4} x_{6}+x_{5} x_{7}+x_{6} x_{8}),\right.\notag\\
 &\qquad \left.\frac{1}{3} (x_{1} x_{4}-2 x_{1} x_{8}-x_{2} x_{3}+2 x_{2} x_{7}+2 x_{3} x_{6}-2 x_{4} x_{5}+x_{5} x_{8}-x_{6} x_{7}) ,\right. \notag\\
 &\qquad \left.\frac{1}{6} \left(-x_{1}^2+4 x_{1} x_{5}-x_{2}^2+4 x_{2} x_{6}+x_{3}^2-4 x_{3} x_{7}+x_{4}^2-4 x_{4} x_{8}-x_{5}^2-x_{6}^2+x_{7}^2+x_{8}^2\right) \right),\notag\\    
\bm{n}_1&=\left(
 \frac{1}{3} (x_{1} (x_{7}-2 x_{3})+x_{2} (x_{8}-2 x_{4})+x_{5} (x_{3}+x_{7})+x_{6} (x_{4}+x_{8})), \right.\notag\\
 &\qquad \left.\frac{1}{3} (x_{1} (x_{8}-2 x_{4})+x_{2} (2 x_{3}-x_{7})-x_{6} (x_{3}+x_{7})+x_{5} (x_{4}+x_{8})) , \right.\notag\\
 &\qquad \left.\frac{1}{6} \left(2 x_{1}^2-2 x_{1} x_{5}+2 x_{2}^2-2 x_{2} x_{6}-2 x_{3}^2+2 x_{3} x_{7}-2 x_{4}^2+2 x_{4} x_{8}-x_{5}^2-x_{6}^2+x_{7}^2+x_{8}^2\right) \right),\notag\\
\bm{n}_2&=\left(
 \frac{1}{3} (x_{1} (x_{3}+x_{7})+x_{2} (x_{4}+x_{8})+x_{3} x_{5}+x_{4} x_{6}-2 x_{5} x_{7}-2 x_{6} x_{8}),\right.\notag\\ 
 &\qquad \left.\frac{1}{3} (x_{1} (x_{4}+x_{8})-x_{2} (x_{3}+x_{7})-x_{3} x_{6}+x_{4} x_{5}-2 x_{5} x_{8}+2 x_{6} x_{7}) ,\right. \notag\\
 &\qquad \left.\frac{1}{6} \left(-x_{1}^2-2 x_{1} x_{5}-x_{2}^2-2 x_{2} x_{6}+x_{3}^2+2 x_{3} x_{7}+x_{4}^2+2 x_{4} x_{8}+2 x_{5}^2+2 x_{6}^2-2 \left(x_{7}^2+x_{8}^2\right)\right) \right),\notag\\    
\bm{m}_{1,0}&=\left(
 \frac{-x_{1}^2+x_{2}^2+x_{3}^2-x_{4}^2+x_{5}^2-x_{6}^2-x_{7}^2+x_{8}^2}{2 \sqrt{3}} , \frac{x_{1} x_{2}+x_{3} x_{4}-x_{5} x_{6}-x_{7} x_{8}}{\sqrt{3}} , \frac{-x_{1} x_{3}+x_{2} x_{4}+x_{5} x_{7}-x_{6} x_{8}}{\sqrt{3}} \right),\notag\\    
\bm{m}_{1,1}&=\left(
 \frac{2 x_{1} x_{5}-2 x_{2} x_{6}-2 x_{3} x_{7}+2 x_{4} x_{8}-x_{5}^2+x_{6}^2+x_{7}^2-x_{8}^2}{2 \sqrt{3}}, -\frac{x_{1} x_{6}+x_{2} x_{5}+x_{3} x_{8}+x_{4} x_{7}-x_{5} x_{6}-x_{7} x_{8}}{\sqrt{3}} ,\right.\notag\\
 &\qquad \left. \frac{x_{1} x_{7}-x_{2} x_{8}+x_{3} x_{5}-x_{4} x_{6}-x_{5} x_{7}+x_{6} x_{8}}{\sqrt{3}} \right),\notag\\    
\bm{m}_{1,2}&=\left(
 \frac{x_{1}^2-2 x_{1} x_{5}-x_{2}^2+2 x_{2} x_{6}-x_{3}^2+2 x_{3} x_{7}+x_{4}^2-2 x_{4} x_{8}}{2 \sqrt{3}},\frac{x_{1} (x_{6}-x_{2})+x_{2} x_{5}+x_{3} (x_{8}-x_{4})+x_{4} x_{7}}{\sqrt{3}} ,\right. \notag\\
 &\qquad \left. \frac{x_{1} (x_{3}-x_{7})+x_{2} (x_{8}-x_{4})-x_{3} x_{5}+x_{4} x_{6}}{\sqrt{3}} \right),\notag\\    
\bm{m}_{2,0}&=\left(
 \frac{x_{1} x_{2}-x_{3} x_{4}-x_{5} x_{6}+x_{7} x_{8}}{\sqrt{3}} , \frac{x_{1}^2-x_{2}^2+x_{3}^2-x_{4}^2-x_{5}^2+x_{6}^2-x_{7}^2+x_{8}^2}{2 \sqrt{3}} , \frac{x_{1} x_{4}+x_{2} x_{3}-x_{5} x_{8}-x_{6} x_{7}}{\sqrt{3}} \right),\notag\\    
\bm{m}_{2,1}&=\left(
 \frac{-x_{1} x_{6}-x_{2} x_{5}+x_{3} x_{8}+x_{4} x_{7}+x_{5} x_{6}-x_{7} x_{8}}{\sqrt{3}} , \frac{-2 x_{1} x_{5}+2 x_{2} x_{6}-2 x_{3} x_{7}+2 x_{4} x_{8}+x_{5}^2-x_{6}^2+x_{7}^2-x_{8}^2}{2 \sqrt{3}} ,\right.\notag\\
 &\qquad \left. -\frac{x_{1} x_{8}+x_{2} x_{7}+x_{3} x_{6}+x_{4} x_{5}-x_{5} x_{8}-x_{6} x_{7}}{\sqrt{3}} \right),\notag\\    
\bm{m}_{2,2}&=\left(
 \frac{x_{1} (x_{6}-x_{2})+x_{2} x_{5}+x_{3} (x_{4}-x_{8})-x_{4} x_{7}}{\sqrt{3}} ,\frac{-x_{1}^2+2 x_{1} x_{5}+x_{2}^2-2 x_{2} x_{6}-x_{3}^2+2 x_{3} x_{7}+x_{4}^2-2 x_{4} x_{8}}{2 \sqrt{3}} ,\right.\notag\\
  &\qquad \left. \frac{x_{1} (x_{8}-x_{4})+x_{2} (x_{7}-x_{3})+x_{3} x_{6}+x_{4} x_{5}}{\sqrt{3}} \right).
\end{align}
\endgroup
Despite the complicated form, we can verify that 
$|\bm{S}_{\bm{r}_0}|=|\bm{S}_{\bm{r}_1}|=|\bm{S}_{\bm{r}_2}|$, and $\langle \bm{S}_{\bm{r}_0}, \bm{S}_{\bm{r}_1}\rangle = \langle \bm{S}_{\bm{r}_0}, \bm{S}_{\bm{r}_2}\rangle=\langle \bm{S}_{\bm{r}_1}, \bm{S}_{\bm{r}_2}\rangle$ and $\bm{S}_{\bm{r}_0}\cdot (\bm{S}_{\bm{r}_1}\times \bm{S}_{\bm{r}_2})=0$, therefore for each $\bm{r}$, the three sites labeled by the sublattice index $\tau$ form a coplanar $120^\circ$ order. The order has both a $\bm{k}=0$ and a $\bm{k} = \left(\pm \frac{2\pi}{3},0\right)$ (i.e. $\pm \mathrm{K}$ points in the Brillouin zone) wave vector, i.e. it has an enlarged $\sqrt{3}\times \sqrt{3}$ unit cell, and the spins within this enlarged unit cell are of unequal sizes (although within the original sublattice of site $\bm{r}$ the three spins are equal). We have checked the three normal vectors of the coplanar spins and find their orientation are generally different.

We call this the $\sqrt{3}\times \sqrt{3}$ umbrella, nonuniform, sublattice-coplanar $120^\circ$ order. This order is illustrated in the left panel of Fig.~\ref{fig:order_0_1_case3_and_4}.

% Figure environment removed

\subsubsection{Case 5, M point}

Define $\mathrm{M}_1 = \left(0,\frac{\pi}{\sqrt{3}a}\right)$, $\mathrm{M}_2 = \left(\frac{\pi}{2a},\frac{\pi}{2\sqrt{3}a}\right)$, $\mathrm{M}_3 = \left(\frac{\pi}{2a},-\frac{\pi}{2\sqrt{3}a}\right)$. This belongs to the ``diagonalizable'' case with quadratic dispersion ($z=2$). The eigenvectors are
\begin{equation}
\text{For }\mathrm{M}_i,i=1,2,3\colon
\quad \mathbf{a}_i,\quad U_{\mathcal{T}}\mathbf{a}^*_i,\quad U_{\mathcal{C}}\mathbf{a}^*_i,\quad U_{\mathcal{C}}U_{\mathcal{T}} \mathbf{a}_i,
\end{equation}
with
\begin{subequations}
\begin{align}
\mathbf{a}_1 = \left(\Delta,0,0,0,A-a,0,0,0,0,0,0,a'+A'\right),\\
\mathbf{a}_2 = 
\left(-a+A,0,\Delta,0,0,0,0,a'+A',0,0,0,0\right),\\
\mathbf{a}_3 = 
\left(0,0,A-a,0,\Delta,0,0,0,0,a'+A',0,0\right).
\end{align}
\end{subequations}
where we defined $\Delta = \sqrt{(a-A)^2+(a'+A')^2}$. Note that $\mathbf{a}_3\propto U^\dag_{{C}_6} \mathbf{a}_1$, $\mathbf{a}_2 \propto U^\dag_{{C}_6}\mathbf{a}_2$, as required by symmetry.

We need to use the above eigenvectors to construct the particle-hole symmetric vectors: define $\mathbf{b}_i = c_1\mathbf{a}_i+c_2U_{\mathcal{T}}\mathbf{a}^*_i+c_3U_{\mathcal{C}}\mathbf{a}^*_i+c_4U_{\mathcal{C}}U_{\mathcal{T}} \mathbf{a}_i$, requiring $\mathbf{b}_i = U_{\mathcal{C}}\mathbf{b}_i^*$ gives
\begin{subequations}
\begin{align}
(\mathbf{b}_1)_{1:6}= \left( \Delta c_1,-\Delta c_2,0,0,(-a+A)c_1+(a'+A')c^*_2,(a-A)c_2+(a'+A')c_1^*\right),\\
(\mathbf{b}_2)_{1:6}= \left(0,0,(-a+A)c_3+(a'+A')c_4^*,(a-A)c_4+(a'+A')c_3^*,\Delta c_3,-\Delta c_4\right),\\
(\mathbf{b}_3)_{1:6}= \left((-a+A)c_5+(a'+A')c_6^*,(a-A)c_6+(a'+A')c_5^*,\Delta c_5,-\Delta c_6,0,0\right),
\end{align}
\end{subequations}
Using the above we obtain the real space parton condensation
\begin{equation}
\langle b_{\bm{r}}\rangle = \sum_{i=1,2,3} e^{i \bm{k}_i\cdot \bm{r}}
(\mathbf{b}_i)_{1:6}
\end{equation}
where $\bm{k}_i$ is the momentum for the $\mathrm{M}_i$ point in the Brillouin zone. We have $\bm{k}_{1,2,3}\cdot \bm{r} = \pi(-r_1+r_2),\pi r_2,\pi r_1$, and
\begin{equation}
\bm{S}_{\bm{r}_\tau}
=\bm{n}_{0,\tau} + \bm{n}_{1,\tau}(-1)^{r_1}+\bm{n}_{2,\tau}(-1)^{r_2}+\bm{n}_{3,\tau}(-1)^{r_1+r_2},
\end{equation}
where $\bm{n}_{0,\tau}$, $\bm{n}_{1,\tau}$, $\bm{n}_{2,\tau}$ and $\bm{n}_{3,\tau}$ are 12 vectors of $c_{1,2,...,6}$, which we omit here. The magnetic order has an enlarged unit cell that is $2\times 2$ times of the original lattice unit cell. We checked that while the order is generically nonplanar and complex, we find that the 12 spins in this $2\times 2$ magnetic unit cell satisfies
\begin{equation}
\begin{aligned}
\bm{S}_{(r_1,r_2)_0} = \bm{S}_{(r_1,r_2+1)_0}&,\quad \bm{S}_{(r_1+1,r_2)_0} = \bm{S}_{(r_1+1,r_2)_0},\quad
\bm{S}_{(r_1,r_2)_1} = \bm{S}_{(r_1+1,r_2)_1},\\ \bm{S}_{(r_1+1,r_2)_1} = \bm{S}_{(r_1+1,r_2+1)_1}&,\quad
\bm{S}_{(r_1,r_2)_2} = \bm{S}_{(r_1+1,r_2+1)_2},\quad \bm{S}_{(r_1,r_2+1)_2} = \bm{S}_{(r_1+1,r_2)_2},
\end{aligned}
\end{equation}
In other words, the three sublattice $\tau=0,1,2$ have different ordering momentum $\mathrm{M}_\tau$, i.e. the sublattice and the magnetic ordering wave vector are \emph{locked}.

We call this the $2\times 2$ sublattice-locked, sublattice-non-uniform strip order.  This order is illustrated in the right panel of Fig.~\ref{fig:order_0_1_case3_and_4}.




\subsection{Class $0-(1,0)$}


We consider both NN and NNN bonds and define 
\begin{equation}\label{class10params}
a=\sin \theta \cos \phi,\quad a' = \sin \theta \sin \phi,\quad A = \cos \theta.
\end{equation}

We checked numerically that the condensation momenta are one of the following: $\Gamma$, $\mathrm{K}$, $\mathrm{M}$, or along the line $\Gamma\mathrm{K}$, or along the line $\mathrm{KM}$. We see that all these points/paths can be viewed as lying on the $k_y=0$ line. The condensation diagram is given in Fig.~\ref{fig:phase_diagram_10_NNN}. As can be seen, when $\theta >\pi/2$ (i.e. $A<0$), we only have condensation at $\Gamma$ or $\mathrm{K}$.

At $\Gamma$, the critical chemical potential is $\mu = \max(a+A,-2(a+A))$; at $\mathrm{K}$, it is $\mu = \max(a-2A+\sqrt{3}|a'|,\frac{1}{2}(-a+2A+\sqrt{3}|a'|))$. As checked numerically, when $A<0$, the critical chemical potential are
\begin{equation}
\mu = \max(\underbrace{-2(a+A)}_{\Gamma},\underbrace{a-2A+\sqrt{3}|a'|}_{\mathrm{K}}).
\end{equation}
The phase boundary for $A<0$ is hence $\phi = \pi/3$ and $\phi = 2\pi/3$. We call them case 1 and case 2. 

Keeping the bond parameterization as in Eq.~\eqref{class10params}, the critical chemical potential is given in Fig.~\ref{fig:phase_diagram_10_NNN_critical_mu}. 

% Figure environment removed

By comparing Fig.~\ref{fig:phase_diagram_10_NNN} and Fig.~\ref{fig:phase_diagram_10_NNN_critical_mu}, we see that the largest critical chemical potential (the red regions in Fig.~\ref{fig:phase_diagram_10_NNN_critical_mu}) happen in case 1 (collinear FM order) and case 2 (sublattice-collinear, $\sqrt{3}\times \sqrt{3}$ $120^\circ$ order).

Another tractable case is for $0<\theta < \arctan3\approx 1.24905$, i.e. the part on the left of the red vertical line of the phase diagram in Fig.~\ref{fig:phase_diagram_10_NNN} are all incommensurate. These are due to the energy $E = \frac{\sqrt{\left(a^2+a'^2\right) \cos 2 k_x+ a^2+4 a (A-\mu ) \cos k_x+2 A^2-4 A\mu -a'^2+2 \mu^2}}{\sqrt{2}}$ becoming gapless
at critical momentum $\mu = A+ \sqrt{a^2+a'^2}$ and condensation momentum
\begin{equation}
\mathrm{L} = (\pm \arctan\frac{|a'|}{a},0).
\end{equation}
We call this case 3.

As one can observe from Fig.~\ref{fig:phase_diagram_10_NNN}, for $\arctan3<\theta <\frac{\pi}{2}$ there are also two incommensurate regions, which we call case 4 and 5. The detailed study of these incommensurate phases (cases 3,4,5) will be left to future work.

\subsubsection{Case 1, $\Gamma$ point}

We have four zero energy eigenstates defined by $\mathbf{a}$, $U_{\mathcal{T}}\mathbf{a}^*$, $U_{\mathcal{C}}\mathbf{a}^*$ and $U_{\mathcal{C}}U_{\mathcal{T}}\mathbf{a}$ with $\mathbf{a}$ given in Eq.~\eqref{simplesta}, so everything follows exactly the case 1 of the Class $0-(0,1)$, and we have a collinear ferromagnetic order. 

\subsubsection{Case 2, $\mathrm{K}$ point}


Again first consider $a'>0$. The Hamiltonian $\mathcal{H}(\bm{k})$ at $\mathrm{K} = (\frac{2\pi}{3},0)$ has two zero eigenvalues, whereas the nonhermitian one $J\mathcal{H}(\bm{k})$ has four zero eigenvalues. This belongs to the ``nondiagonalizable'' case, whose low energy gapless dispersion is linear with dynamical exponent $z=1$. Following the procedure outlined in previous sections, we obtain
\begin{equation}
\left(\begin{array}{c}\langle b_{(\frac{2\pi}{3},0)}\rangle  \\ \langle b^\dag_{-(\frac{2\pi}{3},0)}\rangle \end{array}\right) = (x_1+ix_2)\mathbf{a}_1+(x_3+ix_4)\mathbf{a}_2,
\end{equation}
where
\begin{equation}
\mathbf{a}_1 = \left(\frac{1}{\sqrt{2}},0,-\frac{1}{\sqrt{2}},0,\frac{1}{\sqrt{2}},0,0,-\frac{i}{\sqrt{2}},0,\frac{i}{\sqrt{2}},0,-\frac{i}{\sqrt{2}}\right),\quad
\mathbf{a}_2 = \left(0,-\frac{1}{\sqrt{2}},0,\frac{1}{\sqrt{2}},0,-\frac{1}{\sqrt{2}},-\frac{i}{\sqrt{2}},0,\frac{i}{\sqrt{2}},0,-\frac{i}{\sqrt{2}},0\right).
\end{equation}
From this we obtain the real space spin
\begin{equation}
\begin{aligned}
\bm{S}_{\bm{r}_\tau} = \Big(\sin (2\phi_{\bm{r}})\frac{x_1^2-x_2^2-x_3^2+x_4^2}{2}+\cos (2 \phi_{\bm{r}}) (x_1 x_2-x_3 x_4),& \cos (2 \phi_{\bm{r}}) \frac{x_1^2-x_2^2+x_3^2-x_4^2}{2}-\sin (2 \phi_{\bm{r}}) (x_1 x_2+x_3 x_4),\\
&\sin (2 \phi_{\bm{r}}) (x_1 x_3-x_2 x_4)+\cos (2 \phi_{\bm{r}}) (x_1 x_4+x_2 x_3)\Big),
\end{aligned}
\end{equation}
where $\phi_{\bm{r}} = \left(\frac{2\pi}{3},0\right)\cdot \bm{r}$. We check that this is a coplanar $120^\circ$ order. we call this the sublattice-uniform coplanar $120^\circ$ order. This order is illustrated in Fig.~\ref{fig:pd}(d) in the main text.

The magnetic order for the $a'<0$ can be obtained in the same way, which has the opposite vector spin chirality from the $a'>0$ case. 




\end{document}