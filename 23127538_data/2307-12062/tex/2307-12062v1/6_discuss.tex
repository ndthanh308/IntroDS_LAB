\section{Conclusion and Discussion}
\vspace{-0.5em}
% In this paper, we have introduced an attack model which follows a proposed temporally-coupled assumption. 
In this paper, we introduce a novel attack model to challenge deep RL models, based on a temporally-coupled constraint that can naturally arise in real life. 
Since existing robust RL methods usually focus on a traditional threat model that perturbs state observations or actions arbitrarily within an $L_p$ norm ball, they become too conservative and can fail to perform a good defense under the temporally-coupled attacks. 
In contrast, we propose a game-theoretical response approach \ours, which finds the best response against attacks with various constraints including temporally-coupled ones. 
Extensive experiments in continuous control tasks show that \ours significantly outperforms prior robust RL methods under both traditional attack models and the new temporally-coupled attacks in either state spaces or action spaces.

\ours is based on the PSRO paradigm, which is shown to be effective and theoretically grounded in finding Nash equilibrium in two-player zero-sum games. 
The current PSRO-based approach requires multiple iterations for convergence to the best response, which can be a limitation when applied to high-dimensional tasks with limited computation resources. 
Extensions to other game-theoretic RL approaches can possibly mitigate this issue and can be investigated in future work.

