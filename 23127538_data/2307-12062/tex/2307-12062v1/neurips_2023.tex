\pdfoutput=1
\documentclass{article}

% if you need to pass options to natbib, use, e.g.:
%     \PassOptionsToPackage{numbers, compress}{natbib}
% before loading neurips_2023


% ready for submission
\usepackage[preprint]{neurips_2023}


% to compile a preprint version, e.g., for submission to arXiv, add add the
% [preprint] option:
%     \usepackage[preprint]{neurips_2023}


% to compile a camera-ready version, add the [final] option, e.g.:
%     \usepackage[final]{neurips_2023}


% to avoid loading the natbib package, add option nonatbib:
%    \usepackage[nonatbib]{neurips_2023}

\setcitestyle{numbers,square}
\usepackage{xspace}
\usepackage{wrapfig}

% \usepackage[dvipsnames]{xcolor}
\PassOptionsToPackage{prologue,dvipsnames}{xcolor}
\newcommand{\ys}[1]{{\color{cyan}{YS:#1}}}
\newcommand{\yy}[1]{{\color{green}{YY:#1}}}
\newcommand{\fhc}[1]{{\scriptsize{\textcolor{Magenta}{[FH:#1]}}}}
\newcommand{\fh}[1]{\textcolor{Magenta}{#1}}
\usepackage{soul}
\newcommand{\fhst}[1]{{\textcolor{Magenta}{\st{#1}}}}
\newcommand{\ours}{{GRAD}\xspace}
\newcommand{\oursfull}{{Game-theoretic Response approach for Adversarial Defense}\xspace}

% to compile a preprint version, e.g., for submission to arXiv, add add the
% [preprint] option:
%     \usepackage[preprint]{neurips_2022}


% to compile a camera-ready version, add the [final] option, e.g.:
%     \usepackage[final]{neurips_2022}


% to avoid loading the natbib package, add option nonatbib:
% \usepackage[nonatbib]{neurips_2022}


\usepackage[utf8]{inputenc} % allow utf-8 input
\usepackage[T1]{fontenc}    % use 8-bit T1 fonts
\usepackage{hyperref}       % hyperlinks
\usepackage{url}            % simple URL typesetting
\usepackage{booktabs}       % professional-quality tables
\usepackage{amsfonts}       % blackboard math symbols
\usepackage{nicefrac}       % compact symbols for 1/2, etc.
\usepackage{microtype}      % microtypography
\usepackage{xcolor}         % colors

\usepackage{pgfplots}
\DeclareUnicodeCharacter{2212}{−}
\usepgfplotslibrary{groupplots,dateplot}
\usetikzlibrary{patterns,shapes.arrows}
\pgfplotsset{compat=newest}
\usepackage{colortbl}
\usepackage{multirow}
\usepackage{array}
\usepackage{makecell}
\usepackage{subcaption}
\usepackage{tikz}
\usepackage{siunitx}

%%%% Added by JB
% \bibliographystyle{abbrvnat}
%%%% Changed by Yanchao for saving space
\bibliographystyle{plain}


% \usepackage[noline]{algorithm2e}

\RequirePackage{algorithm}
\RequirePackage{algorithmic}
\RequirePackage{forloop}

\usepackage{epstopdf}
\usepackage{graphicx}
% \usepackage{subcaption}

\usepackage{tabulary}
\usepackage{colortbl}
\usepackage{amsmath}
\usepackage{amssymb}
\usepackage{mathtools}
\usepackage{amsthm}

\DeclareMathOperator{\E}{\mathbb{E}}

\DeclareMathOperator*{\argmax}{arg\,max}
\DeclareMathOperator*{\argmin}{arg\,min}

\DeclareMathOperator{\supp}{supp}

\newtheorem{theorem}{Theorem}
\newtheorem{proposition}[theorem]{Proposition}
\newtheorem{lemma}[theorem]{Lemma}
\newtheorem{corollary}[theorem]{Corollary}
\theoremstyle{definition}
\newtheorem{definition}[theorem]{Definition}
\newtheorem{assumption}[theorem]{Assumption}
\theoremstyle{remark}
\newtheorem{remark}[theorem]{Remark}


% \title{Breaking Rectangularity in Robust RL}
\title{
Game-Theoretic
Robust Reinforcement Learning Handles Temporally-Coupled Perturbations}


% The \author macro works with any number of authors. There are two commands
% used to separate the names and addresses of multiple authors: \And and \AND.
%
% Using \And between authors leaves it to LaTeX to determine where to break the
% lines. Using \AND forces a line break at that point. So, if LaTeX puts 3 of 4
% authors names on the first line, and the last on the second line, try using
% \AND instead of \And before the third author name.

\author{%
  Yongyuan Liang$^\dag$ \\
  \texttt{cheryllLiang@outlook.com} \\
  \And
  Yanchao Sun$^\dag$ \\
  \texttt{ycs@umd.edu} \\
  \And
  Ruijie Zheng$^\dag$ \\
  \texttt{rzheng12@umd.edu} \\
  \And
  Xiangyu Liu$^\dag$ \\
  \texttt{xyliu999@umd.edu} \\
  \And
  Tuomas Sandholm$^{\triangle \nabla}$\\
  % Carnegie Mellon University\\
  % Strategy Robot, Inc.\\
  % Optimized Markets, Inc.\\
  % Strategic Machine, Inc.\\
  \texttt{sandholm@cs.cmu.edu}\\
  \And
  Furong Huang$^\dag$\\
 %  University of Maryland \\
  \texttt{furongh@umd.edu}\\
  \And
  Stephen McAleer$^{\triangle}$\thanks{Corresponding Author,  $^\dag$ University of Maryland, $^\triangle$ Carnegie Mellon University,    $^\nabla$ Strategy Robot, Inc., Optimized Markets, Inc., Strategic Machine, Inc.}\\
 % \thanks{Corresponding author}\\  Carnegie Mellon University\\
  \texttt{smcaleer@cs.cmu.edu}
}

\begin{document}

\maketitle

\begin{abstract}
% Stephen's draft
Robust reinforcement learning (RL) seeks to train policies that can perform well under environment perturbations or adversarial attacks. Existing approaches typically assume that the space of possible perturbations remains the same across timesteps. However, in many settings, the space of possible perturbations at a given timestep depends on past perturbations.  
We formally introduce temporally-coupled perturbations, presenting a novel challenge for existing robust RL methods. To tackle this challenge, we propose \ours, a novel game-theoretic approach that treats the temporally-coupled robust RL problem as a partially-observable two-player zero-sum game. By finding an approximate equilibrium in this game, \ours ensures the agent's robustness against temporally-coupled perturbations. Empirical experiments on a variety of continuous control tasks demonstrate that our proposed approach exhibits significant robustness advantages compared to baselines against both standard and temporally-coupled attacks, in both state and action spaces.


% Recent years have witnessed the development of robust training to defend against the vulnerability of RL policies. Existing threat models impose static constraints on perturbations at each timestep and overlook the temporal influence of past perturbations on the current ones, despite its crucial consideration in many real-world scenarios. 
% We formally introduce temporally-coupled attacks to account for the temporal coupling between perturbations at consecutive time steps, presenting a novel challenge for existing robust RL methods. To tackle this challenge, we propose \ours, a novel game-theoretic response approach that treats the temporally-coupled robust RL problem as a partially-observable two-player game. By finding an approximate equilibrium in our approach, \ours ensures the agent's robustness against the learned adversary. Empirical experiments on a variety of continuous control tasks demonstrate that our proposed approach exhibits significant robustness advantages compared to baselines against both standard and temporally-coupled attacks, in both the state and action spaces.
\end{abstract}

\section{Introduction}

% Figure environment removed

Reinforcement Learning from Human Feedback (RLHF) has recently been used to great effect to align pretrained large language models (LLMs) to human preferences, optimizing for desirable qualities like harmlessness and helpfulness~\citep{bai2022training} and achieving state-of-the-art results across a variety of natural language tasks~\citep{openai2023gpt4}. %RLHF approaches fundamentally rely on collecting pairs of LLM outputs $(o_1, o_2)$ from a shared prompt $p$, with a human indicating which output in each pair is better on a specified attribute.
% A fundamental component of RLHF is a preference model derived from human labels, typically formatted as pairs of LLM outputs $(o_1, o_2)$ generated from a shared prompt $p$.

A standard RLHF procedure fine-tunes an initial unaligned LLM using an RL algorithm such as PPO~\citep{schulman2017proximal}, optimizing the LLM to align with human preferences. %\violet{not sure whether we need to provide this detail in the intro, especially this has nothing to do with our contribution.} % i feel like this context is useful later when e.g. explaining that context distillation is SFT
RLHF is thus critically dependent on a reward model derived from human-labeled preferences, typically \textit{pairwise preferences} on LLM outputs $(o_1, o_2)$ generated from a shared prompt $p$. % and labeled by humans. 

However, collecting human pairwise preference data, especially high-quality data, may be expensive and time consuming at scale. To address this problem, approaches have been proposed to obtain labels without human annotation, such as Reinforcement Learning from AI Feedback (RLAIF) and context distillation. 

\iffalse
raising the question of whether we can generate high-quality data for RLHF without using human labeling. %accurately-labeled preference pairs $(o_1, o_2)$
%, motivating model alignment approaches that aim to generate accurately-labeled preference pairs $(o_1, o_2)$ without human involvement. 
Two major categories of such approaches are . 
\fi

RLAIF approaches (e.g.,~\citet{bai2022constitutional}) simulate human pairwise preferences by scoring $o_1$ and $o_2$ with an LLM (Figure \ref{fig:rlcd_differences} center); the scoring LLM is often the same as the one used to generate the original pairs $(o_1, o_2)$. Of course, the resulting LLM pairwise preferences will be somewhat noisier compared to human labels. However, this problem is exacerbated by using the same prompt $p$ to generate both $o_1$ and $o_2$, causing $o_1$ and $o_2$ to often be of very similar quality and thus hard to differentiate (e.g., Table~\ref{tab:rlaif_bad_example}). Consequently, training signal can be overwhelmed by label noise, yielding lower-quality preference data. 

% While it avoids human labeling efforts, it has weakness. First, LLM preference labels will naturally be somewhat noisier compared to human labels. Furthermore, since the same prompt $p$ is used to generate both $o_1$ and $o_2$, their quality is often very similar and hard to differentiate (See Table~\ref{tab:rlaif_bad_example}). As a result, training signals can be overwhelmed by label noise, yielding lower-quality preference data. 

Meanwhile, context distillation methods (e.g., \citet{sun2023principle}) create more training signal by modifying the initial prompt $p$. 
%to create more significant training signal. 
The modified prompt $p_+$ typically contains additional context encouraging a \textit{directional attribute change} in the output $o_+$ (Figure \ref{fig:rlcd_differences} right). However, context distillation methods only generate a single output $o_+$ per prompt $p_+$, which is then used for supervised fine-tuning, losing the pairwise preferences which help RLHF-style approaches to 
%rather than using a RLHF-style preference model to 
derive signal from the contrast between outputs. 
Multiple works have observed that RL approaches using preference models for pairwise preferences can substantially improve over supervised fine-tuning by itself when aligning LLMs~\citep{ouyang2022training,dubois2023alpacafarm}. 

% conduct alignment by running supervised fine-tuning on model outputs $o_+$ generated from a modified prompt $p_+$. $p_+$ typically contains additional context encouraging desirable attributes (Figure \ref{fig:rlcd_differences} right), such as in \citet{sun2023principle}. However, multiple works have observed that RLHF-style approaches can substantially improve over supervised fine-tuning by itself when aligning LLMs~\citep{ouyang2022training,dubois2023alpacafarm}. 

Therefore, while both RLAIF and context distillation approaches have already been successfully applied in practice to align language models, we posit that it may be even more effective to combine the key advantages of both. That is, we will use RL with \textit{pairwise preferences}, while also using modified prompts to encourage \textit{directional attribute change} in outputs. %In particular, we will adapt the RLAIF data generation process with two different prompts rather than a single $p$, modifying both prompts similarly to context distillation. %\violet{this motivation is a little unexciting. I think we can more specifically discuss the potential benefits of our approach, like the benefits from RL: exploration/data generation; benefits from contrast. I don't think we get too much benefits from context distillation since we switched to the RL framework.} 

Concretely, we propose \oursfull{} (\ours{}). 
\ours{} generates preference data as follows. Rather than producing two i.i.d.\ model outputs $(o_1, o_2)$ from the same prompt $p$ as in RLAIF, \ours{} creates two variations of $p$: a \textit{positive prompt} $p_+$ similar to context distillation which encourages directional change toward a desired attribute, and a \textit{negative prompt} $p_-$ which encourages directional change \textit{against} it (Figure \ref{fig:rlcd_differences} left). We then generate model outputs $(o_+, o_-)$ respectively, and automatically label $o_+$ as preferred---that is, \ours{} automatically ``generates'' pairwise preference labels by construction. %, without further post hoc labeling.\violet{should make it clearer that our approach `generates' labels by construction} 
We then follow the standard RL pipeline of training a preference model followed by PPO. 

Compared to RLAIF-generated preference pairs $(o_1, o_2)$ from the same input prompt $p$, there is typically a clearer difference in the quality of $o_+$ and $o_-$ generated using \ours{}'s directional prompts $p_+$ and $p_-$, which may result in less label noise. %which may result in better training signal for the preference model. 
That is, intuitively, \ours{} exchanges having examples be \textit{closer to the classification boundary} for much more \textit{accurate labels} on average. Compared to standard context distillation methods, on top of leveraging pairwise preferences for RL training, \ours{} can derive signal not only from the positive prompt $p_+$ which improves output quality, but also from the negative prompt $p_-$ which degrades it. %\ours{} is not learning to imitate $o_+$, but to distill the \textit{contrast} between $o_+$ and $o_-$. 
Positive outputs $o_+$ don't need to be perfect; they only need to contrast with $o_-$ on the desired attribute while otherwise following a similar style.

% \todo{discuss our method and why intuitively it may be better.}

We evaluate the practical effectiveness of \ours{} through both human and automatic evaluations on three tasks, aiming to improve the ability of LLaMA-7B~\citep{touvron2023llama} to generate harmless outputs, helpful outputs, and high-quality story outlines. %\ours{} outperforms both RLAIF and context distillation baselines in pairwise comparisons on 
As shown in Sec. \ref{sec:experiments}, \ours{} substantially outperforms both RLAIF and context distillation baselines in pairwise comparisons when simulating preference data with LLaMA-7B, while still performing equal or better when simulating with LLaMA-30B. 
%On all three tasks, \ours{} substantially outperforms both RLAIF and context distillation baselines in pairwise comparisons---by a margin of at least 9\% and often more than 30\%---validating our method's efficacy. 
We will release all code at a later date, although in any case \ours{} is fairly easy to implement by modifying any reference RLAIF codebase. %We release all code at \todo{github link}.
\vspace{-0.5em}
\section{Related Work}
\vspace{-0.5em}
\label{sec:relate}
% There are a number of related deep reinforcement learning methods for two-player zero-sum games. CFR-based techniques such as Deep CFR~\citep{brown2019deep}, DREAM~\citep{steinberger2020dream}, and ESCHER~\citep{mcaleer2022escher}, use deep reinforcement learning to approximate CFR. Policy-gradient techniques such as RPG~\citep{srinivasan2018actor}, NeuRD~\citep{hennes2020neural}, Friction-FoReL~\citep{perolat2021poincare, perolat2022mastering}, and MMD~\citep{sokota2022unified}, approximate Nash equilibrium via modified actor-critic algorithms. In this section we mainly focus on double oracle techniques such as PSRO~\citep{psro}. PSRO-based algorithms have been shown to outperform the previously-mentioned algorithms in certain games~\citep{mcaleer2021xdo}.   
% \subsection{Robust Reinforcement Learning(RL)}

\textbf{Robust RL against perturbations of state observations.}
% There are many existing adversarial defense approaches developed for RL agents against adversarial perturbations on state observations. 
\textit{Regularization-based methods}~\citep{zhang2020robust, shen2020deep,oikarinen2020robust} enforce the policy to have similar outputs under similar inputs, which achieves certifiable performance for DQN in some Atari games. But in continuous control tasks, these methods may not reliably improve the worst-case performance. A recent work by Korkmaz~\citep{korkmaz2021investigating} points out that these adversarially trained models may still be sensible to new perturbations.
\textit{Attack-driven methods} train DRL agents with adversarial examples. Some early works~\citep{kos2017delving, behzadan2017whatever, mandlekar2017adversarially, pattanaik2017robust, franzmeyer2022illusionary, vinitsky2020robust} apply weak or strong gradient-based attacks on state observations to train RL agents against adversarial perturbations. 
Zhang et al.~\citep{zhang2021robust} and Sun et al.~\citep{sun2021strongest} propose to alternately train an RL agent and a strong RL adversary, namely ATLA, which significantly improves the policy robustness against rectangle state perturbations. 
A recent work by Liang et al.~\citep{liang2022efficient} introduce a more principled adversarial training framework which does not explicitly learn the adversary, and both the efficiency and robustness of RL agents are boosted. 
There is also a line of work studying \textit{theoretical guarantees} of adversarial defenses in RL~\citep{lutjens2020certified, oikarinen2020robust,fischer2019online,kumar2021policy, wu2021crop,sun2023certifiably} in various settings. 

%action robust
\textbf{Robust RL against action perturbations.} 
Besides observation perturbations, attacks can happen in many other scenarios. For example, the agent's executed actions can be perturbed~\citep{xiao2019characterizing,tan2020robustifying, tessler2019action,lee2020query, lanier2022feasible}. Moreover, in a multi-agent game, an agent's behavior can create adversarial perturbations to a victim agent~\citep{gleave2019adversarial}.
Pinto et al.~\citep{pinto2017robust} model the competition between the agent and the attacker as a zero-sum two-player game, and train the agent under a learned attacker to tolerate both environment shifts and adversarial disturbances.

\textbf{Robust Markov decision process and safe RL.} 
There are several lines of work that study RL under safety/risk constraints~\citep{Heger1994ConsiderationOR,gaskett2003reinforcement,garcia2015comprehensive,bechtle2020curious,thomas2021safe} or under intrinsic uncertainty of environment dynamics~\citep{lim2013reinforcement,Mankowitz2020Robust}.
In particular, there are several works discussing coupled or non-rectangular uncertainty sets, which allow less conservative and more efficient robust policy learning by incorporating realistic conditions that naturally arise in practice. 
Mannor et al.~\citep{mannor2012lightning} propose to model coupled uncertain parameters based on the intuition that the total number of states with deviated parameters will be small.
Mannor et al.~\citep{mannor2016robust} identify ``k-rectangular'' uncertainty sets defined by the cardinality of possible conditional projections of uncertainty sets, which can lead to more tractable solutions. 
Another recent work by Goyal et al.~\citep{goyal2023robust} propose to model the environment uncertainty with factor matrix uncertainty sets, which can efficiently compute an optimal robust policy. 

\textbf{Two-player zero-sum games.}
There are a number of related deep reinforcement learning methods for two-player zero-sum games. CFR-based techniques such as Deep CFR~\citep{brown2019deep}, DREAM~\citep{steinberger2020dream}, and ESCHER~\citep{mcaleer2022escher}, use deep reinforcement learning to approximate CFR. Policy-gradient techniques such as RPG~\citep{srinivasan2018actor}, NeuRD~\citep{hennes2020neural}, Friction-FoReL~\citep{perolat2021poincare, perolat2022mastering}, and MMD~\citep{sokota2022unified}, approximate Nash equilibrium via modified actor-critic algorithms. Our robust RL approach takes the double oracle techniques such as PSRO~\citep{psro} as the backbone. PSRO-based algorithms have been shown to outperform the previously-mentioned algorithms in certain games~\citep{mcaleer2021xdo}. 
More related work on game-theoretic RL is discussed in Appendix~\ref{sec:add_related}.



\vspace{-0.5em}
\section{Preliminaries}
\vspace{-0.5em}
\label{sec:pre}

% \begin{algorithm}
% \SetAlgoLined

% \DontPrintSemicolon
% \KwResult{Nash Equilibrium}
%  Input: initial population $\Pi^0$\;
%  \While{Not terminated}{
%   Solve game restricted to policies in $\Pi^t$ to get meta-distribution $\pi^r$\;
%   \For{$i \in \{1,2\}$}{
%       Find a best response $\mathbb{BR}_i(\pi^r_{-i})$\;
%       $\Pi^{t+1}_i = \Pi^t_i \cup \mathbb{BR}_i(\pi^r_{-i})$\;
%   } 
%   \If{No novel best response exists for both players}{
% %   \If{$gv(G(\Pi_0, all possible p1 strategies)) = gv(G(all possible p0 strategies, \Pi_1))$}
%     Return $\pi^r$\;
%   }
%  }
%  \caption{Double Oracle}
% \end{algorithm}

\textbf{Notations and Background.} 
A Markov decision process (MDP) can be defined as a tuple $\langle \mathcal{S}, \mathcal{A}, \mathcal{P}, \mathcal{R}, \gamma \rangle$, where $\mathcal{S}$ and $\mathcal{A}$ represent the state space and the action space, $\mathcal{R}$ is the reward function: $\mathcal{R}:\mathcal{S}\times\mathcal{A}\to\mathbb{R}$, $\mathcal{P}:\mathcal{S}\times\mathcal{A}\to\Delta(\mathcal{S})$ represents the set of probability distributions over the state space $\mathcal{S}$ and $\gamma\in(0,1)$ is the discount factor. The agent selects actions based on its policy, $\pi:\mathcal{S} \to \Delta(\mathcal{A})$, which is represented by a function approximator (e.g. a neural network) that is updated during training and fixed during testing. The value function is denoted by $V^{\pi}(s):=\mathbb{E}_{P,\pi}[\sum_{t=0}^{\infty}\gamma^{t}R(s_{t},a_{t})\mid s_{0}=s]$, which measures the expected cumulative discounted reward that an agent can obtain from state $s\in\mathcal{S}$ by following policy $\pi$.

\textbf{State Adversaries.} State adversary is a type of test-time attacker that perturbs the agent's state observation returned by the environment at each time step and aims to reduce the expected episode reward gained by the agent. While the input to the agent's policy is perturbed, the underlying state in the environment remains unchanged. State adversaries, such as those presented in ~\citep{zhang2020robust, zhang2021robust, sun2021strongest}, typically consider perturbations on a continuous state space under a certain attack budget $\epsilon$. The attacker perturbs a state $s$ into $\tilde{s} \in \mathcal{B}_{\epsilon}(s)$, where $\mathcal{B}_{\epsilon}(s)$ is a $\ell_p$ norm ball centered at $s$ with radius $\epsilon$. 

\textbf{Action Adversaries.}
Action adversaries' goal is to manipulate the behavior of the agent by directly perturbing the action $a$ executed by the agent to $\tilde{a}$ before the environment receives it (altering the output of the agent's policy), causing it to deviate from the optimal policy. In addition to directly perturbing actions, recent work ~\citep{tessler2019action} has also considered the setting where the action adversary selects a different, adversarial action with the probability $\alpha$ as an uncertainty constraint. In this paper, we focus solely on continuous-space perturbations and employ an admissible action perturbation budget as a commonly used $\ell_p$ threat model, similar to the state perturbation. 

\textbf{Zero-sum Game.}
We model the game between the agent and the adversary as a two-player zero-sum game that is a tuple $\langle \mathcal{S}, \Pi_a, \Pi_v, \mathcal{P}, \mathcal{R}, \gamma \rangle$, where $\Pi_a$ and $\Pi_v$ denote the sets of policies for the agent and the adversary, respectively. In this framework, both the transition kernels $\mathcal{P}$ and the reward function $\mathcal{R}$ of the victim agent depend on not only its own policy $\pi_a \in \Pi_a$, but also the adversary's policy $\pi_v\in \Pi_v$. The adversary's reward $R(s_t, \bar{a}_t)$ is defined as the negative of the victim agent's reward $R(s_t, a_t)$, reflecting the zero-sum nature of the game. 

\begin{algorithm}[tb]
  \caption{Policy Space Response Oracles \cite{psro}}
  \label{psro}
\begin{algorithmic}
 \STATE {\bfseries Result:} Nash Equilibrium
  \STATE {\bfseries Input:} Initial population $\Pi^0$
  \REPEAT[for $t=0,1,\ldots$]
  \STATE $\pi^r \gets$ NE in game restricted to strategies in $\Pi^t$
  \FOR{$i \in \{1,2\}$}
  \STATE Find a best response $\beta_i \gets \mathbb{BR}_i(\pi^r_{-i})$
  \STATE $\Pi^{t+1}_i \gets \Pi^t_i \cup \{\beta_i\}$
  \ENDFOR
  \UNTIL{Approximate exploitability is less than or equal to zero}
  \STATE {\bfseries Return:} $\pi^r$
\end{algorithmic}
\end{algorithm}
% \textcolor{red}{Connection between zero-sum game and PSRO? I added this in the related work, should we instead move it here?}

% \textbf{Policy-space response oracles (PSRO)} PSRO~\cite{psro} approximates the DO algorithm in extensive-form games. The restricted-game NE is computed on the empirical game matrix $U^\Pi$, generated by having each policy in the population $\Pi$ play each opponent policy and tracking average utility in a $\Pi_1 \times \Pi_2$ payoff matrix \citep{wellman2006methods}. In each iteration, an approximate best response to the current restricted NE over the policies is computed via any RL algorithm. 
\textbf{Double Oracle Algorithm (DO) and Policy Space Response Oracles (PSRO).}
Double oracle~\citep{double_oracle} is an algorithm for finding a NE in normal-form games. The algorithm operates by keeping a population of strategies $\Pi^t$ at time $t$. Each iteration, a NE $\pi^{*,t}$ is computed for the game restricted to strategies in $\Pi^t$. Then, a best response $\mathbb{BR}_i(\pi^{*,t}_{-i})$ to this NE is computed for each player $i$ and added to the population, $\Pi_i^{t+1} = \Pi_i^t \cup \{\mathbb{BR}_i(\pi^{*,t}_{-i}) \}$ for $i \in \{1, 2\}$. 
% The DO algorithm is described in Algorithm \ref{double_oracle}. 
Although in the worst case DO must expand all pure strategies before $\pi^{*,t}$ converges to a NE in the original game, in many games DO terminates early and outperforms alternative methods. An interesting open problem is characterizing games where DO will outperform other methods.

Policy Space Response Oracles (PSRO)~\citep{psro, muller2019generalized, feng2021discovering, mcaleer2022anytime, mcaleer2022self}, shown in Algorithm~\ref{psro} are a method for approximately solving very large games. PSRO maintains a population of reinforcement learning policies and iteratively trains a best response to a mixture of the opponent's population. PSRO is a fundamentally different method than the previously described methods in that in certain games it can be much faster but in other games it can take exponentially long in the worst case. Neural Extensive Form Double Oracle (NXDO)~\citep{mcaleer2021xdo} combines PSRO with extensive-form game solvers and can be used to converge faster that PSRO. 
\vspace{-3pt}
\section{Real Image Interpolation}
% Let $x^0, x^N$ be two real images that we want to interpolate with $N-1$ intermediate images.


\input{figs/2_pipeline}

\subsection{Latent interpolation}\label{sec:latent_interp}
Our general strategy for generating sequences of interpolations is to iteratively interpolate pairs of images, starting with the two given input images. 
For each pair of parent images, we add shared noise to their latent vectors, interpolate them, then denoise the result to generate an intermediate image. The amount of noise to add to the parent latent vectors should be small if the parents are close to each other in the sequence, to encourage smooth interpolations. If the parents are far apart, the amount of noise should be larger to allow the LDM to explore nearby trajectories in latent space that have higher probability and better match other conditioning information.

Concretely, we specify a sequence of increasing timesteps $\mathcal{T}=(t_1,\dots,t_K)$, and assign parent images using the following branching structure: images $0$ and $N$ (the input images) are diffused to timestep $t_K$ and averaged to generate image $\frac{N}{2}$, images $0$ and $\frac{N}{2}$ are diffused to timestep $t_{K-1}$ generate image $\frac{N}{4}$, images $\frac{N}{2}$ and $N$ are also diffused to timestep $t_{K-1}$ to generate image $\frac{3N}{4}$, and so on. By adding noise separately to each pair of parent images, this scheme encourages images to be close to their parents, but disentangles sibling images.  %images $kN/2^j$ and $(k+2)N/2^j$ generate image $(2k+1)N/2^{j+1}$ for all $j=$

\paragraph{Interpolation type}
We use spherical linear interpolations (\textit{slerp}) for latent space and text embedding interpolations, and linear interpolations for pose interpolations. Empirically, the difference between \textit{slerp} and linear interpolation appears to be fairly mild.

\paragraph{Noise schedule}
We perform DDIM sampling \cite{ddim}, and find that the LDM's quality is more consistent when the diffusion process is partitioned into at least 200 timesteps, and noticeably degrades at coarser schedules. Empirically, latent vectors denoised with less than 25\% of the schedule often resemble an alpha composite of their parent images, while images generated with more than 65\% of the schedule can deviate significantly from their parent images. For each interpolation we choose a linear noise schedule within this range, depending on the amount of variation desired in the output. Our approach is compatible with various stochastic samplers \cite{karras2022elucidating} which seem to yield comparable results.
% Written out explicitly, we create sequences of corresponding noisy latents $\{z_t^0\}_{t \in \mathcal{T}}, \{z_t^N\}_{t \in \mathcal{T}}$, such that:
% \begin{gather}
% z_t^i = \alpha_t z_{t-1}^i + \beta_t \eps_t,
% \end{gather}
% where $\eps_t \sim \normal(0,I)$ is shared for both images.% and $z_0^0, z_0^N$ are obtained as before.
% Each intermediate image is assigned a particular timestep $t := \texttt{image_schedule}(i)$ to generate its interpolated latent code:
% $z_t^i := \texttt{slerp}(z_t^0, z_t^N, i/N)$
% We then perform denoising with the LDM: $z_0^i := \mu_\theta(z_t^i, t)$ and use the decoder to produce the image.

% $z_0^0 := \gE(x^0)$, $z_0^N := \gE(x^N)$, and all images are generated $z_0^i = \texttt{slerp}(z_0^0, z_0^N, i/N)$, $x^i := \mathcal{D}(z_0^i)$
% We examine three different strategies for latent interpolation, which differ in how they combine diffusion with interpolation to create interpolated images.

% \paragraph{Denoise-renoise-interpolate}
% Rather than partially denoise each latent, we can fully denoise the latent, then add new noise back to the appropriate level before interpolating it. This strategy permits a much wider range of latent space to be traversed, by decoupling images $N/4$ from $3N/4$, etc., while still forcing adjacent images to be similar.
% \footnote{The interpolation of two latent vectors at a particular noise level may not remain at the same noise level due to correlations introduced during the denoising process. However, we observe empirically that the independent noise assumption.}

\subsection{Textual inversion}\label{sec:text_inversion}
Pre-trained latent diffusion models are heavily dependent on text conditioning to yield high quality outputs of a particular style. Given an initial text prompt describing the overall content and/or style of each image, we can adapt its embedding more specifically to the image by applying textual inversion. In particular, we encode the text prompt as usual, then fine-tune the prompt embedding to minimize the error of the LDM on denoising the latent vector at random noise levels when conditioned on this embedding. Specifically, we perform 100-500 iterations of gradient descent with the loss $\loss(c_{\rm{text}}) = \norm{\hat{\eps}_\theta(\alpha_t z_0 + \sigma_t \eps; t, c_{\rm{text}}) - \eps}$ and a learning rate of $10^{-4}$. The number of iterations can be increased for images with complicated layouts or styles which are harder to represent with a text prompt.

In this paper we specify the same initial prompt for both input images, although one can also substitute a captioning model for a fully automated approach. Both positive and negative text prompts are used and optimized, and we share the negative prompt for each pair of images. Since our task does not require a custom token, we choose to optimize the entire text embedding.
% We also want to interpolate the prompt between the input images so that the style and content can transition smoothly. We can either specify prompts for each of the images, or perform single-image textual inversion on the images. In our experience, the best approach is to choose an initial shared positive and negative prompt for the images, then .

\subsection{Pose guidance}\label{sec:pose_guidance}
\input{figs/4_pose_conditioning}
If the subject's pose differs significantly between the two images, image interpolation is challenging and often results in anatomical errors such as multiple limbs and faces. We obtain more plausible transitions between subjects in different poses by incorporating pose conditioning information in the LDM. We obtain poses of the input images using OpenPose \cite{openpose}, with the assistance of style transfer for cartoons or non-human subjects (see Fig. \ref{fig:openpose}). We then linearly interpolate all shared keypoint positions from the two images to obtain intermediate poses for each image. The resulting pose is provided to the LDM using ControlNet \cite{controlnet}, a powerful method for conditioning on arbitrary image-like inputs. Interestingly, we observe that \textit{even when the wrong pose is predicted} for input images, conditioning on pose still yields superior interpolations as it prevents abrupt pose changes (see Fig. \ref{fig:pose}). %Additionally, we find it helpful to increase the strength of the pose conditioning for images towards the middle of the sequence.
% If the modality of the input images is unsuitable for obtaining accurate pose information (e.g. stylized cartoons), we can first perform style transfer to a photorealistic image using the LDM, which will be more suitable as input to OpenPose even if the image quality is poor .
% Figure environment removed
%, so by choosing the LDM to use 10-20 timesteps, this step becomes very fast


\vspace{-3pt}
\subsection{CLIP ranking}\label{sec:clip_ranking}
\vspace{-3pt}
LDMs can yield outputs of widely varying quality and characteristics with different random seeds. This problem is compounded in real image interpolation since a single bad generated image compromises the quality of all other images derived from it.
Thus when quality is more important than speed, multiple candidates can be generated with different random seeds, then ranked with CLIP \cite{clip}. We repeat each forward diffusion step with different noise vectors, denoise each of the interpolated latent vectors, then measure the CLIP similarity of the decoded image with specified positive and negative prompts (e.g., positive: ``high quality, detailed, 2D'', negative: ``blurry, distorted, 3D render''). The image with the highest value of positive similarity minus negative similarity is kept. %We generate more candidates for images at higher noise levels since these images have more freedom to deviate from the desired style.
In applications requiring an even higher degree of control and quality, this pipeline can be changed into an interactive mode where users can manually select desired interpolations or even specify a new prompt or pose for a particular image.

We conduct various experiments on computer vision (CV) problems involving datasets such as CIFAR-10, CIFAR-100 and ImageNet. We choose SGD and Adam as the main baselines since it is widely used in these tasks and maintains the best training performance. 
We also compare with another quasi-Newton method, KFAC, in large-scale model training. 
We separately tune hyperparameters for each optimizer to ensure it achieves the best validation accuracy.

We use a single GPU server with 8 Nvidia Quadro RTX 5000 GPUs to simulate a distributed system, where each GPU is used as a worker to perform forward and backward passes, and model updates. Furthermore, each worker is also assigned with one Hessian block to compute the Hessian inverse and gradient conditioning. The current implementation is based on PyTorch. We set lower and upper thresholds of damping $\sigma_L, \sigma_H$ to be $0.01, 1.5$ in all experiments to smooth the Hessian approximation.

\subsection{Experiments on CIFAR-10/CIFAR-100}
We first evaluate \method{} on two small-scale problems: CIFAR-10 and CIFAR-100, and demonstrate the convergence advantage of \method{} compared to SGD and Adam. 
The models used are ResNet-18 and DeiT-Tiny \cite{DeiT}, where DeiT-Tiny is an efficient image transformer with 12 layers, 3 attention heads, and hidden and MLP dimension of 192.
% The ViT model is based on Vision Transformer model \citep{ViT}, with 6 layers, 8 attention heads, a patch size of 16, and both hidden and MLP dimension of 512 for a total of about 10M parameters.

For ResNet18, we divide it into 4 blocks such that each block consists of 2 \emph{resblocks} (\cite{2016_CVPR_ResNet}). The linear layer for classification is packed into the last block.  For DeiT-Tiny, due to the small model size, we choose to approximate the whole Hessian. 

Hyperparameters are tuned to achieve the best validation accuracy. Details are provided in Appendix \ref{appx:hparam:cifar}.

Figure \ref{fig:cifar} shows training loss. We observe that \method{} achieves a much faster convergence rate compared to SGD and ADAM. 
Table \ref{tab:acc:cifar} lists the validation accuracy on CIFAR-10 and CIFAR-100. 
We note that \method{} also achieves similar accuracy as SGD. On the other hand, as in Figure  \ref{fig:cifar}, although ADAM has a convergence rate close to \method{}, the validation accuracy are much lower in all experiments compared to \method{}.
Therefore, we obverse \method{} not only deliver faster convergence, but also achieves good generalization performance.
% Figure environment removed

\begin{table}[!htb]
\caption{\footnotesize Validation accuracy of ResNet-18 and Deit-Tiny on CIFAR-10/100 using SGD, ADAM, and \method{}. }
\label{tab:acc:cifar}
\centering
\small
\begin{tabular}{ccc|ccc|ccc}
\toprule
\multicolumn{3}{c|}{ResNet-18/CIFAR-10} & \multicolumn{3}{c|}{ResNet-18/CIFAR-100} & \multicolumn{3}{c}{DeiT-Tiny/CIFAR-100}\\
\midrule
 SGD & ADAM & \method{} & SGD & ADAM & \method{} & SGD & ADAM & \method{} \\
 $94.1\pm 0.1$ & $92.7\pm 0.1$ & $93.9\pm 0.1$ & $75\pm 0.15$ & $72.2\pm 0.16$ & $74.4 \pm 0.1$ & $80.5\pm 0.2$ & $75.3\pm 0.3$ & $79.9 \pm 0.2$ \\
\bottomrule
\end{tabular}
\vspace{-.4cm}
\end{table}

\subsection{Experiments on ImageNet}
ImageNet has been the gold standard for evaluating the performance of optimizers. It consists of $\sim$1.2M training and $\sim$50K test images, categorized into 1000 classes. 
We follow the standard data pre-processing procedure, where each image is first resized to $256\times 256$, and randomly cropped to $224\times 224$ and flipped horizontally. Each image is then normalized using pre-computed mean and variance.
%\subsubsection{ResNet-50}

\textbf{ResNet-50} --
When approximating the Hessian, we divide ResNet-50 into 8 blocks such that each block consists of 2 \emph{resblocks}. Similar to ResNet-18 in CIFAR-10, the linear layer is packed into the last block.
Figure~\ref{fig:resnet_imagenet} shows iteration-wise convergence on ResNet-50 using SGD, Adam, KFAC and \method{}. 
Detailed hyperparameter settings are provided in Appendix \ref{appx:hparam:imagenet}. Compared to Adam and SGD, \method{} enjoys much faster per-iteration convergence. Such fast convergence is also reflected in the validation dataset (Figure \ref{fig:resnet_imagenet_val}). Furthermore, it also generalizes well on the validation set, and finally reaches comparable validation accuracy to SGD.
% Figure environment removed

\begin{table}[!htb]
\caption{\footnotesize Validation accuracy of ResNet-50 on ImageNet using SGD, ADAM, KFAC, and \method{}. }
\label{tab:runtime}
\centering
\small
\begin{tabular}{cc|cc|cc|cc}
\toprule
\multicolumn{2}{c|}{SGD} & \multicolumn{2}{c|}{ADAM} & \multicolumn{2}{c|}{\method{}} & \multicolumn{2}{c}{KFAC}\\
\midrule
 Acc & Time/epoch & Acc & Time/epoch & Acc & Time/epoch & Acc & Time/epoch  \\
 $74.9\pm 0.11$ & 7.8min & $73.95\pm 0.1$ & 7.8min & $74.6\pm 0.13$ & 7.9min & $74.5\pm 0.1$ & 17min \\
\bottomrule
\end{tabular}
\end{table}

The benefit of \method{} is even more striking in terms of wall-clock time. 
As listed in Figure \ref{fig:resnet_imagenet_time} and Table \ref{tab:runtime}, due to light compute costs, the per-epoch runtime of \method{} is almost the same as SGD and Adam. 
On the other hand, for KFAC, while it delivers fast per-iteration convergence compared to SGD and ADAM, the wall-clock performance is significantly diminished by its additional compute costs. The per-epoch runtime is $>2\times$ more than \method{}.

\iffalse
\begin{table}[!htb]
\caption{Wall-clock time for each optimizer to reach the optimal accuracy. \method{} needs the least time to reach the optimal accuracy. }
\label{tab:runtime}
\centering
\begin{tabular}{c|ccc|cc}
\toprule
 & \multicolumn{3}{c|}{ResNet50} & \multicolumn{2}{c}{ViT}\\
\midrule
 Opt. & SGD & KFAC & \method{} & SGD & \method{} \\
 Time (h)& 31.2 & 50.5 & 22.7 & 12.5 & 10.4 \\
\bottomrule
\end{tabular}
\vspace{-.4cm}
\end{table}
\fi

\subsection{Ablation Study: The Effects of Momentum and Damping}\label{subsec:ablation}
In this section, we give more insight into the effects of momentum and damping used in \method{}. To this end, we ablate two critical components in \method{}: momentum and damping in the Hessian approximation, and then use the ablated version to train ResNet-18 on CIFAR-10. We focus on CIFAR-10 since we observed more convergence instability on this dataset compared to others. 

Figure~\ref{fig:ablation} shows convergence using the ablated \method{} with only momentum (black), with only damping (purple), and with no momentum or damping (red). Due to stochastic noise, the ablated version of \method{} without momentum/damping (vanilla L-BFGS) diverges easily in the early stages. 
With momentum (black), the whole optimization is significantly stabilized. However, it still fails to converge when there is a radical change in the loss landscape (for example, when learning rate decays). 
With damping (purple), the Hessian approximation is effectively restrained, especially when such sudden changes in the loss landscape happen. 
It is interesting to observe that while damping prevents divergence, the whole training is still largely affected by stochastic noise. Notable fluctuation in the loss is commonly observed during training. As a comparison, the complete \method{} (blue) effectively addresses these issues achieving much more stable convergence.
% Figure environment removed
%\vspace{-0.5em}
\section{Conclusion and Discussion}
% \vspace{-0.5em}
% In this paper, we have introduced an attack model which follows a proposed temporally-coupled assumption. 
Motivated by the perturbations that arise in real world scenarios, we introduce a new attack model for studying deep RL models.
Since existing robust RL methods usually focus on a traditional threat model that perturbs state observations or actions arbitrarily within an $L_p$ norm ball, they become too conservative and can fail to perform a good defense under the temporally-coupled attacks. 
In contrast, we propose a game-theoretical response approach \ours, which finds the best response against attacks with various constraints including temporally-coupled ones. 
% \ours is based on the PSRO paradigm, which is shown to be effective and theoretically grounded in finding Nash equilibrium in two-player zero-sum games. 
Experiments across a range of continuous control tasks underscore the good performance of our approach over previous robust RL methods for both non-temporally-coupled attacks and temporally-coupled attacks across diverse attack domains.
% , highlighting the generalized robustness of \ours.

\textbf{Limitations.}\quad
The current PSRO-based approach may require several iterations to converge to the best response, which can pose limitations when computational resources are constrained. We leverage distributed RL tools to expedite the training of RL agents within \ours, enabling efficient learning of the best response. Detailed computational cost analysis can be found in Appendix~\ref{app:efficient}.

Regarding scalability concerns, we have demonstrated the \ours in addressing robust RL problems on high-dimensional tasks. In principle, alternative game-theoretic algorithms~\citep{perolat2022mastering}, known for their practical efficiency, can be considered for defense in different game scenarios. As part of our future research directions, we plan to explore methods to further enhance the scalability of \ours. This exploration may involve harnessing parallel training techniques and drawing insights from other scalable PSRO approaches~\citep{mcaleer2020pipeline, psro}. Additionally, we aim to extend the applicability of our method to pixel-based RL scenarios and real-world situations with increased practicality and complexity.






\section*{Acknowledgements}
We thank Ben Eysenbach for helpful conversations. This material is based upon work supported by the National Science Foundation under Grant $2127309$ to the Computing Research Association for the CIFellows Project.
Liang, Sun, Zheng, Liu and Huang are supported by National Science Foundation NSF-IIS-FAI program, DOD-ONR-Office of Naval Research, DOD Air Force Office of Scientific Research, DOD-DARPA-Defense Advanced Research Projects Agency Guaranteeing AI Robustness against Deception (GARD), Adobe, Capital One and JP Morgan faculty fellowships. The work of Prof. Sandholm's research group is funded by the National Science Foundation under grants IIS1901403, CCF-1733556, and the ARO under award W911NF2210266.
\bibliography{main}

\newpage
\appendix
\section{Additional Related Work}
\label{sec:add_related}

\subsection{Game-Theoretic Reinforcement Learning}

Superhuman performance in two-player games usually involves two components: the first focuses on finding a model-free blueprint strategy, which is the setting we focus on in this paper. The second component improves this blueprint online via model-based subgame solving and search~\citep{burch2014solving, moravcik2016refining, brown2018depth, brown2020combining, brown2017safe, schmid2021player}. This combination of blueprint strategies with subgame solving has led to state-of the art performance in Go~\citep{silver2017mastering}, Poker~\citep{brown2017libratus, brown2018superhuman, moravvcik2017deepstack}, Diplomacy~\citep{gray2020human}, and The Resistance: Avalon~\citep{serrino2019finding}. Methods that only use a blueprint have achieved state-of-the-art performance on Starcraft~\citep{alphastar}, Gran Turismo~\citep{wurman2022outracing}, DouDizhu~\citep{zha2021douzero}, Mahjohng~\citep{li2020suphx}, and Stratego~\citep{mcaleer2020pipeline, perolat2022mastering}. In the rest of this section we focus on other model-free methods for finding blueprints.      

Deep CFR~\citep{deep_cfr, steinberger2019single} is a general method that trains a neural network on a buffer of counterfactual values. However, Deep CFR uses external sampling, which may be impractical for games with a large branching factor, such as Stratego and Barrage Stratego. DREAM~\citep{steinberger2020dream} and ARMAC~\citep{gruslys2020advantage} are model-free regret-based deep learning approaches. ReCFR~\citep{liu2022model} propose a bootstrap method for estimating cumulative regrets with neural networks. ESCHER~\citep{mcaleer2022escher} remove the importance sampling term of Deep CFR and show that doing so allows scaling to large games.  

Neural Fictitious Self-Play (NFSP)~\citep{nfsp} approximates fictitious play by progressively training a best response against an average of all past opponent policies using reinforcement learning. The average policy converges to an approximate Nash equilibrium in two-player zero-sum games.   
%but has slower convergence bounds than CFR.  

There is an emerging literature connecting reinforcement learning to game theory. QPG~\citep{srinivasan2018actor} shows that state-conditioned $Q$-values are related to counterfactual values by a reach weighted term summed over all histories in an infostate and proposes an actor-critic algorithm that empirically converges to a NE when the learning rate is annealed. NeuRD~\citep{hennes2020neural}, and F-FoReL~\citep{perolat2021poincare} approximate replicator dynamics and follow the regularized leader, respectively, with policy gradients. Actor Critic Hedge (ACH)~\citep{ach} is similar to NeuRD but uses an information set based value function. All of these policy-gradient methods do not have theory proving that they converge with high probability in extensive form games when sampling trajectories from the policy. In practice, they often perform worse than NFSP and DREAM on small games but remain promising approaches for scaling to large games \citep{perolat2022mastering}. 
\section{Experiment Details and Additional Results}
\label{app:exp}

\subsection{Implementation details}
\label{app:exp:imp}
We provide detailed implementation information for our proposed method (\ours) and baselines.

\textbf{Training Steps}\quad
For \ours, we specify the number of training steps required for different environments. In the Hopper, Walker2d, and Halfcheetah environments, we train for 10 million steps. In the Ant and Humanoid environments, we extend the training duration to 20 million steps. For the ATLA baselines, we train for 2 million steps and 10 million steps in environments of varying difficulty.

\textbf{Network Structure}\quad
Our algorithm (\ours) adopts the same PPO network structure as the ATLA baselines to maintain consistency. The network comprises a single-layer LSTM with 64 hidden neurons. Additionally, an input embedding layer is employed to project the state dimension to 64, and an output layer is used to project 64 to the output dimension. Both the agents and the adversaries use the same policy and value networks to facilitate training and evaluation. Furthermore, the network architecture for the best response and meta Nash remains consistent with the aforementioned configuration.

\textbf{Schedule of $\epsilon$ and $\bar{\epsilon}$}\quad
During the training process, we gradually increase the values of $\epsilon$ and $\bar{\epsilon}$ from 0 to their respective target maximum values. This incremental adjustment occurs over the first half of the training steps. We reference the attack budget $\epsilon$ used in other baselines for the corresponding environments. This ensures consistency and allows for a fair comparison with existing methods. The target value of $\bar{\epsilon}$ is determined based on the adversary's training results, which is set as $\epsilon/5$. In some smaller dimensional environments, $\bar{\epsilon}$ can be set to $\epsilon/10$. We have observed that the final performance of the trained robust models does not differ by more than 5\% when using these values for $\bar{\epsilon}$.

\textbf{Observation and Reward Normalization}\quad
To ensure consistency with PPO implementation and maintain comparability across different codebases, we apply observation and reward normalization. Normalization helps to standardize the input observations and rewards, enhancing the stability and convergence of the training process. We have verified the performance of vanilla PPO on different implementations, and the results align closely with our implementation of \ours based on Ray rllib.

\textbf{Hyperparameter Selection}\quad
Hyperparameters such as learning rate, entropy bonus coefficient, and other PPO-specific parameters are crucial for achieving optimal performance. Referring to the results obtained from vanilla PPO and the ATLA baselines as references, a small-scale grid search is conducted to fine-tune the hyperparameters specific to \ours. Because of the significant training time and cost associated with \ours, we initially perform a simplified parameter selection using the Inverted Pendulum as a test environment.

\subsection{Adversaries in experiments}
\textbf{State Adversaries}\quad
Aimed to introduce the attack methods utilized during training and testing in our experiments. When it comes to state adversaries, PA-AD as Alogrithm~\ref{alg:pa-ad} stands out as the strongest attack compared to other state attacks. Therefore, we report the best state attack rewards under PA-AD attacks.

\textbf{Action Adversaries}\quad
In terms of action adversaries, an RL-based action adversary as Alogrithm~\ref{alg:ac-ad} can inflict more severe damage on agents' rewards compared to OU noise and parameter noise in~\citep{tessler2019action}.

\begin{algorithm}[tb]
  \caption{Action Adversary (AC-AD)}
  \label{alg:ac-ad}
  \begin{algorithmic}
  \STATE \textbf{Input:} Initialization of action adversary policy $v$; victim policy $\pi$, initial state $s_0$
  \FOR {$t = 0, 1, 2, \ldots$}
  \STATE adversary $v$ samples an action perturbations $\widehat{a}_{t}\sim\nu(\cdot|s_{t})$, 
  \STATE victim policy $\pi$ outputs action $a_{t}\sim\pi(\cdot|s_{t})$
  \STATE the environment receives $\tilde{a}_{t} = a_{t} + \widehat{a}_{t} $, returns $s_{t+1}$ and $r_t$
  \STATE adversary saves $(s_t, \widehat{a}_t,-r_t,s_{t+1})$ to the adversary buffer
  \STATE adversary updates its policy $v$
  \ENDFOR
  \end{algorithmic}
\end{algorithm}
\vspace{-0.5em}

\textbf{Mixed Adversaries}\quad
When dealing with mixed adversaries capable of perturbing both state and action spaces, it becomes crucial to design the action space for the adversary. In Algorithm~\ref{alg:mixed-ad}, we extend the idea of PA-AD~\citep{sun2021strongest}, which learns a policy perturbation direction to generate perturbations. In our case, the mixed adversary director only needs to learn the policy perturbation direction $\hat{d}_t$. For various attack domains, the actor functions then translate the direction $\hat{d}_t$ into state or action perturbations. This design approach ensures that our mixed adversary doesn't increase the complexity of adversary training, as it deploys mixed perturbations using different actor functions as required by distinct attack domains.
\label{app:exp:alg}
\begin{algorithm}[!hb]
  \caption{Policy Adversarial Actor Director (PA-AD)}
  \label{alg:pa-ad}
  \begin{algorithmic}
  \STATE \textbf{Input:} Initialization of adversary director’s policy $v$; victim policy $\pi$, the actor function $g$ for the state space $\mathcal{S}$, initial state $s_0$
  \FOR {$t = 0, 1, 2, \ldots$}
  \STATE \textit{Director} $v$ samples a policy perturbing direction and perturbed choice, $\widehat{a}_{t}\sim\nu(\cdot|s_{t})$
  \STATE \textit{Actor} perturbs $s_t$ to $\tilde{s}_{t}=g(\widehat{a}_{t},s_{t})$
  \STATE Victim takes action $a_{t}\sim\pi(\cdot|\tilde{s}_{t})$, proceeds to $s_{t+1}$, receives $r_t$
  \STATE \textit{Director} saves $(s_t, \widehat{a}_t,-r_t,s_{t+1})$ to the adversary buffer
  \STATE \textit{Director} updates its policy $v$ using any RL algorithms
  \ENDFOR
  \end{algorithmic}
\end{algorithm}
\vspace{-0.5em}
\input{algorithms/mixed-AD}

\vspace{1em}
\textbf{Transition Adversaries. }
In addition to addressing adversarial perturbations, we extend the evaluation of \ours to consider transition uncertainty, mitigating the mismatch problem between the training simulator and the testing environment. Robustness under transition uncertainty is crucial for real-world applicability. To assess this aspect, experiments are conducted on perturbed MuJoCo environments (Hopper, Walker2d, and HalfCheetah) by modifying their physical parameters ('leg\_joint\_stiffness' value: 30, 'foot\_joint\_stiffness' value: 30, and bound on 'back\_actuator\_range': 0.5) following the protocol established by \cite{zhou2023natural}. Comparative evaluations are performed against robust natural actor-critic (RNAC)\citep{zhou2023natural} trained with Double-Sampling (DS) and Inaccurate Parameter Models (IPM) uncertainty. The results presented in Table~\ref{tab:transition} consistently demonstrate that \ours achieves competitive or superior performance compared to baseline methods in each perturbed environment, showcasing its effectiveness in robustly handling transition uncertainty.

\begin{table}[!t]
\vspace{-0.5em}
\centering
\renewcommand{\arraystretch}{1.3}
\resizebox{\textwidth}{!}{%
\setlength{\tabcolsep}{4pt}
  \centering
  \begin{tabular}{p{2.5cm}<{\centering} p{3.5cm}<{\centering} p{2.5cm}<{\centering} p{2.5cm}<{\centering} p{2.5cm}<{\centering}}
    \toprule
    Perturbed Environments & & RNAC-PPO (DS) & RNAC-PPO (IPM) & \textbf{GRAD} \\
    \midrule
    \multirow{2}{*}{Hopper} & Natural reward& \textbf{$3502 \pm 256$} & $3254 \pm 138$ & $3482 \pm 209$ \\
     & 'leg\_joint\_stiffness' & $2359 \pm 182$ & $2289 \pm 124$ & \textbf{$2692 \pm 236$} \\
    \midrule
    \multirow{2}{*}{Walker} & Natural reward & $4322 \pm 289$ & $4248 \pm 89$ & \textbf{$4359 \pm 141$} \\
    & 'foot\_joint\_stiffness' & $4078 \pm 297$ & $4129 \pm 78$ & \textbf{$4204 \pm 132$} \\
    \midrule
    \multirow{2}{*}{Halfcheetah} & Natural reward  & $5524 \pm 178$ & $5569 \pm 232$ & \textbf{$6047 \pm 241$} \\
    & 'back\_actuator\_range'& $768 \pm 102$ & $1143 \pm 45$ & \textbf{$1369 \pm 117$} \\
    \bottomrule
  \end{tabular}}
\caption{Comparison of cumulative reward in Perturbed Environments with changed physical parameters.}
\vspace{-0.5em}
\label{tab:transition}
\end{table}


\textbf{Short-term Memorized Temporall-coupled Attacks. }
While our temporally-coupled setting considering perturbation from the last time step aligns with the common practice of state adversaries, which typically perturb the current state without explicitly attacking short-term memory, we recognized the importance of exploring a more general scenario akin to a general partially observable MDP~\citep{efroni22provable}. We introduced a short-term memorized temporally-coupled attacker by calculating the mean of perturbations from the past 10 steps and applying the temporally-coupled constraint to this mean.
The results in Table~\ref{tab:memorized} from these additional experiments against short-term memorized temporally-coupled attacks underscore the efficacy of GRAD under this extended setting. GRAD consistently demonstrates heightened robustness compared to other robust baselines when confronted with a memorized temporally-coupled adversary. These findings provide valuable insights into the temporal scope of perturbations, contributing to a more comprehensive understanding of GRAD's capabilities in handling diverse adversarial scenarios.

\begin{table}[!t]
\vspace{-0.5em}
\centering
\renewcommand{\arraystretch}{1.4}
\resizebox{\textwidth}{!}{%
\setlength{\tabcolsep}{4pt}
\begin{tabular}{p{4cm}<{\centering} p{2.5cm}<{\centering} p{2.5cm}<{\centering} p{2.5cm}<{\centering} p{2.5cm}<{\centering} p{2.5cm}<{\centering}}
\toprule
Short-term Memorized Temporally-Coupled Attacks & Hopper & Walker2d & Halfcheetah & Ant & Humanoid \\
\midrule
PA-ATLA-PPO & 2334 $\pm$ 249 & 2137 $\pm$ 258 & 3669 $\pm$ 312 & 2689 $\pm$ 189 & 1573 $\pm$ 232 \\
WocaR-PPO & 2256 $\pm$ 332 & 2619 $\pm$ 198 & 4228 $\pm$ 283 & 3229 $\pm$ 178 & 2017 $\pm$ 213 \\
\textbf{GRAD} &  \textbf{2869 $\pm$ 228} & \textbf{3134 $\pm$ 251} & \textbf{4439 $\pm$ 287} & \textbf{3617 $\pm$ 188} & \textbf{2736 $\pm$ 269} \\
\bottomrule
\end{tabular}}
\caption{Performance Comparison under Memorized Temporally-Coupled Attacks}
\label{tab:memorized}
\vspace{-0.5em}

\end{table}







\subsection{Attack budgets}
\label{app:exp:eps}
In Figure~\ref{fig:eps}, we report the performance of baselines and \ours under different attack budget $\epsilon$. As the value of $\epsilon$ increases, the rewards of robust agents under different types of attacks decrease accordingly. However, our approach consistently demonstrates superior robustness as the attack budget changes.
\input{figures/fig_eps}

\subsection{Temporally-coupled constraints}
We also investigate the impact of temporally-coupled constraints $\bar{\epsilon}$ on attack performance, as we explained in our experiment section.
\input{figures/fig_bar_eps}

\subsection{Natural reward vs. Robustness}
We presents the natural performance comparison of \ours and action robust baselines in Figure~\ref{fig:action_natural}.
\label{app:natural}
% Figure environment removed

\subsection{Computational Cost}
\label{app:efficient}
The training time for \ours can vary depending on the specific environment and its associated difficulty. Typically, on a single V100 GPU, training \ours takes around 20 hours for environments like Hopper, Walker2d, and Halfcheetah. However, for more complex environments like Ant and Humanoid, the training duration extends to approximately 40 hours. It's worth noting that the training time required for defense against state adversaries or action adversaries is relatively similar.

\end{document}
