
\documentclass{article} % For LaTeX2e

\usepackage{iclr2024_conference,times}

\usepackage{pdfpages}
\usepackage{amsmath}
\usepackage{amsfonts}
% \usepackage{amssymb}
\usepackage{mathtools}
\usepackage{amsthm}
\usepackage{multirow}
\usepackage{microtype}
\usepackage{graphicx}
\usepackage{booktabs}
\usepackage{algorithm}
\usepackage{algorithmic}
\usepackage[backref=page,colorlinks=true,linkcolor=blue,citecolor=green,urlcolor=blue]{hyperref}
\usepackage{url}
\usepackage{bbm}
\usepackage{placeins}
\usepackage{adjustbox}
\usepackage{xcolor}
\usepackage{scalerel}
\setlength\parindent{0pt}

\usepackage{natbib}
\usepackage[outdir=./]{epstopdf}
\usepackage{graphicx,float,pgfplots,wrapfig,sidecap,lipsum}

\usepackage{colortbl}
\usepackage{diagbox} 
\usepackage{tablefootnote}
\usepackage[font=small,labelfont=bf]{caption}
\usepackage{subcaption}
\usepackage{subfloat}

\usepackage{tikz}
\usetikzlibrary{fit}
\usetikzlibrary{calc,shapes}
\usetikzlibrary{decorations.pathmorphing} % noisy shapes
\usetikzlibrary{fit}					% fitting shapes to coordinates
\usetikzlibrary{backgrounds}	
\usetikzlibrary{pgfplots.groupplots}
\usepackage{siunitx}

\usepackage[utf8]{inputenc}
\usepackage{pgfplots}
\usepgfplotslibrary{groupplots,dateplot}
\usetikzlibrary{patterns,shapes.arrows}
\pgfplotsset{compat=newest}

\usepackage{xspace}
\usepackage{soul}
\usepackage{booktabs}
\usepackage{bm}


\newcommand\calF{\mathcal{F}}
\newcommand\calG{\mathcal{G}}
\newcommand\calM{\mathcal{M}}
\newcommand\calV{\mathcal{V}}
\newcommand\calU{\mathcal{U}}
\newcommand\calW{\mathcal{W}}
\newcommand\calP{\mathcal{P}}
\newcommand\calD{\mathbb{D}}
%%%%%%%%%%%%%%%%%
%% macros introduced by Luke 
\newcommand\mydef[1]{{\bf\em #1}}
%%%%%%%%%%%%%%%%%

\newcommand{\numviparams}{{| \lambda |}}
\newcommand{\scoreaccvars}[1]{s_1^{#1}, \ldots, s_{\numviparams}^{#1}}
\newcommand{\scoreaccvar}[2]{s_{#1}^{#2}}
\newcommand{\isdeterm}[1]{\text{Deterministic}({#1})}


\newcommand{\expect}[1]{\mathbb{E}\left[{#1}\right]}
\newcommand{\var}[1]{\mathbb{V}\left[ {#1} \right]}
\newcommand{\expectdist}[2]{\mathbb{E}_{#1}\left[ {#2} \right]}
\newcommand{\vardist}[2]{\mathbb{V}_{#1}\left[ {#2} \right]}
\newcommand{\cov}[2]{\mathbb{C}\text{ov}[{#1}][{#2}]}
\newcommand{\covv}[1]{\mathbb{C}\text{ov}[{#1}]}
\newcommand{\corr}[1]{\mathbb{C}\text{orr}[{#1}]}

\newcommand{\fix}[1]{\mathit{fix}\left({#1}\right)}
\newcommand{\sbr}[1]{\left\llbracket {#1} \right\rrbracket}
\newcommand{\ctxtype}[3]{{#1} \cong_\text{ctx} {#2} : {#3}}
\newcommand{\bigstep}[3]{{#1} \Downarrow_{#2} {#3}}


% PCF types
\newcommand{\bool}{\mathit{bool}}
\newcommand{\nat}{\mathit{nat}}

\newcommand{\ctx}[1]{\mathcal{C}\left[ {#1}\right] }
\newcommand{\pcft}[1]{\text{PCF}_{#1}}

\newcommand{\nfl}{\mathbb{N}_\bot}
\newcommand{\bfl}{\mathbb{B}_\bot}

% PCF constructs
\newcommand{\succc}[1]{\mathbf{succ}({#1})}
\newcommand{\succcn}[2]{\mathbf{succ}^{#1}({#2})}
\newcommand{\zero}{\mathbf{0}}
\newcommand{\zerotest}[1]{\mathbf{zero}\left({#1}\right)}
\newcommand{\pred}[1]{\mathbf{pred}\left( {#1} \right)}
\newcommand{\predn}[2]{\mathbf{pred}^{#1}\left( {#2} \right)}
\def\solvable{\#}

\newcommand{\true}{\mathbf{true}}
\newcommand{\false}{\mathbf{false}}
\newcommand{\pcffix}[1]{\mathbf{fix}\left({#1}\right)}
\newcommand{\pcffn}[3]{\mathbf{fn}~{#1}:{#2}\mathpunct{.}{#3}}
\newcommand{\pairtype}[2]{{#1} * {#2}}
\newcommand{\pairexp}[2]{\mathbf{pair}({#1}, {#2})}
\newcommand{\leftexp}[1]{\mathbf{left}({#1})}
\newcommand{\rightexp}[1]{\mathbf{right}({#1})}

\newcommand{\RationalPos}{\mathbb{Q}^{+}}

\newcommand{\meas}[1]{\mathbb{M}\left( {#1} \right) }
\newcommand{\integ}[1]{\sbr{#1}_I}

\newcommand{\notbigstep}[2]{{#1}~\cancel{\Downarrow}_{#2}}
\newcommand{\subtrace}[3]{{#1}^{{#2} \ldots {#3}}}
\newcommand{\supp}[1]{\textsf{supp}\left({#1}\right)}
\newcommand{\dom}[1]{\textsf{Dom}\left({#1}\right)}
\newcommand{\suppk}[2]{\textsf{Supp}^{#1}\left({#2}\right)}
\newcommand{\tracespace}{\bigcup_{n \in \mathbb{N}}[0, 1]^n}
\newcommand{\generictracespace}{\mathbb{T}}
\newcommand{\nnreals}{\mathbb{R}_{\geq 0}}
\newcommand{\posreals}{\mathbb{R}_{> 0}}
\newcommand{\reals}{\mathbb{R}}

\newcommand{\unrollkM}[2]{\textsf{unroll}_{#1}\left({#2}\right)}
\newcommand{\nphmcint}[5]{\Psi_\textsf{NP}\left({#1}, {#2}, {#3}, {#4}, {#5}\right)}

%SPCF constructs
\newcommand{\spcfvalues}{\Lambda^0_v}

\newcommand{\prevalueM}[1]{\textsf{value}^{-1}_{#1}(\spcfvalues{})}
\newcommand{\num}[1]{\underline{#1}}

% \theoremstyle{definition}
% \newtheorem{thm}{Theorem}
% \newtheorem{lem}{Lemma}
% \newtheorem{defn}{Definition}
% \newtheorem{conj}{Conjecture}
% \newtheorem{prop}{Proposition}

%\theoremstyle{definition}
%\newtheorem{defn}{Definition}[section]
%\newtheorem{example}[defn]{Example}
%
%
%\theoremstyle{plain}
%\newtheorem{thm}{Theorem}[section]
%\newtheorem{lem}[thm]{Lemma}
%\newtheorem{cor}[thm]{Corollary}
%\newtheorem{conj}[thm]{Conjecture}
%\newtheorem{prop}[thm]{Proposition}
%\newtheorem{remark}[thm]{Remark}

%% Proofs
%\let\oldproof\proof
%\renewcommand{\proof}{\color{blue}\oldproof}


\definecolor{codegreen}{rgb}{0,0.6,0}
\definecolor{codegray}{rgb}{0.5,0.5,0.5}
\definecolor{codepurple}{rgb}{0.58,0,0.82}
\definecolor{backcolour}{rgb}{0.95,0.95,0.92}

\lstdefinestyle{myStyle}{
    belowcaptionskip=1\baselineskip,
    breaklines=true,
    frame=none,
    basicstyle=\footnotesize\ttfamily,
    keywordstyle=\bfseries\color{green!40!black},
    commentstyle=\itshape\color{purple!40!black},
    identifierstyle=\color{blue},
    backgroundcolor=\color{gray!10!white},
    %backgroundcolor=\color{backcolour}, 
    numberstyle=\tiny\color{codegray},
    stringstyle=\color{codepurple},
    breakatwhitespace=false,                          
    keepspaces=true,                 
    numbers=left,       
    numbersep=5pt,                  
    showspaces=false,                
    showstringspaces=false,
    showtabs=false,                  
    tabsize=2,
}

% argmin/argmax
\DeclareMathOperator*{\argmax}{arg\,max}
\DeclareMathOperator*{\argmin}{arg\,min}

% Concatenation of lists
\newcommand\doubleplus{+\kern-1.3ex+\kern0.8ex}

% Program configurations
\newcommand{\tuple}[1]{\ensuremath{\langle #1 \rangle}}
% Rule based definitions
\newcommand{\Rule}[4][]{\ensuremath{\inferrule*[lab={\hypertarget{#2}{(\TirName{#2})}},#1]{#3}{#4}}}

% Calligraphic symbols
\newcommand{\calI}{{\mathcal I}} 
\newcommand{\calT}{{\mathcal T}}

%  Macro for new Y operator.
\newcommand{\yBounded}[3]{\mu^{#1}_{#2}\rvert_{#3}}

%%%%%%%%%%%%%%%%%
 
%%%%%%%%%%%%%%%%%

\newcommand{\expv}{\mathbb{E}}

\newcommand{\combTr}[2]{\left[\begin{matrix}
		#1\\
		#2
	\end{matrix} \right]}

\newcommand{\exType}[2]{\left\{\begin{matrix}
		#1\\
		#2
	\end{matrix} \right\}}
\newcommand{\myint}[1]{ [#1]}
\newcommand{\Uniform}{\ensuremath{\mathrm{Uniform}}}
\newcommand{\Normal}{\ensuremath{\mathrm{normal}}}
\DeclareMathOperator{\abs}{abs}
\DeclareMathOperator{\pdf}{pdf}

\newcommand{\intConf}[1]{\lceil#1\rceil}
\newcommand{\tr}{\boldsymbol{t}}

\newcommand{\sample}{\tt{sample}}
%\newcommand{\fix}{\texttt{fix}}
%\newcommand{\num}[1]{\underline{#1}}
\newcommand{\myif}{\texttt{if}}
\newcommand{\mylet}{\texttt{let} \, }
\newcommand{\myin}{\, \texttt{in} \,}
\newcommand{\mythen}{\, \texttt{then} \,}
\newcommand{\myelse}{\, \texttt{else} \,}
\newcommand{\score}{\tt{score}}
\newcommand{\tick}{\tt{tick}}

\newcommand{\term}{\tt{term}}
\newcommand{\pv}{\mathbf{v}}
\newcommand{\rv}{\mathbf{r}}

\newcommand{\interval}{\mathfrak{I}}

\newcommand{\typeReal}{\textbf{\textsf{R}}}

\newcommand{\symbolInt}{\myint{\cdot}}

\newcommand{\LambdaInterval}{\Lambda_{\interval}}
\newcommand{\LambdaSymbolic}{\Lambda_{\text{sym}}}

\newcommand{\toIntervalTerm}[1]{#1^{2\interval}}

%Others
\newcommand{\Sset}{\mathbb{S}}
\newcommand{\Iset}{\mathbb{I}}
\newcommand{\Rset}{\mathbb{R}}
\newcommand{\Nset}{\mathbb{N}}
\newcommand{\Zset}{\mathbb{Z}}

\newcommand{\Term}{\mathbb{T}}
\newcommand{\prob}{\mathbb{P}}
\newcommand{\expt}{\mathbb{E}}


\newcommand{\Leb}{\tt{Leb}}
\newcommand{\Red}{\tt{Red}}
\newcommand{\cost}{\text{cost}}

%\newcommand{\intervalab}[2]{\underline{[#1,#2]}}
\newcommand{\intervalab}{\underline{[a,b]}}
\newcommand{\interI}{\mathcal{I}}
\newcommand{\trans}{\mathcal{T}}

\newcommand{\iv}{\mathbb{I}}

% Programming language constructs
\newcommand{\lit}[1]{\underline{#1}}
\newcommand{\letIn}[1]{\mathsf{let}\,{#1}\,\mathsf{in}\,}
\newcommand{\fixLam}[2]{\mu {#1} {#2}.}
\newcommand{\ifElse}[3]{\mathsf{if} (#1 \le \num{0}) \, {#2} \,\mathsf{else}\, {#3}}

%%Basic notions
\newcommand{\pspace}{(\Omega,\mathcal{F},\probm)}
\newcommand{\probm}{\mathbb{P}}
\newcommand{\condexpv}[2]{{\expt}{\left[{#1} \mid {#2}\right]}}

\newcommand{\stdConf}[1]{(#1)}
%\newcommand{\intConf}[1]{\lceil#1\rceil}
%\newcommand{\intConf}[1]{(#1)}
%\newcommand{\symConf}[1]{\langle\!\langle  #1 \rangle\!\rangle}
%\newcommand\symPath[1]{(#1)}
\newcommand{\symPath}[1]{\langle\!\langle  #1 \rangle\!\rangle}
\newcommand\symConf[1]{(#1)}

\newcommand{\ifSimple}[3]{\mathsf{if}(#1, #2, #3)}
%\newcommand{\ifElse}[3]{\mathsf{if} (#1 \le 0) \, \allowbreak {#2} \, \allowbreak \mathsf{else}\, {#3}}
%\newcommand{\ifElse}[3]{\ifSimple{#1}{#2}{#3}}

%\newcommand{\trace}{\mathsf{s}}
%
%\newcommand\defn[1]{{\bf \em #1}}
\newcommand{\traces}{\mathbb{T}}
%
%\newcommand{\stdConf}[1]{(#1)}
%%\newcommand{\intConf}[1]{\lceil#1\rceil}
%\newcommand{\intConf}[1]{(#1)}
%%\newcommand{\symConf}[1]{\langle\!\langle  #1 \rangle\!\rangle}
%%\newcommand\symPath[1]{(#1)}
%\newcommand{\symPath}[1]{\langle\!\langle  #1 \rangle\!\rangle}
%\newcommand\symConf[1]{(#1)}

\newcommand{\valueSem}[1]{\mathsf{val}_{#1}} % value (semantics)
\newcommand{\weightSem}[1]{\mathsf{wt}_{#1}} % weight (semantics)
\newcommand{\measureSem}[1]{\llbracket #1 \rrbracket}
\newcommand{\posterior}{\mathsf{posterior}}


%%%%%%%%%
% 
%%%%%%%%
\newcommand{\loc}{\ell}
\newcommand{\locs}{\mathit{L}}
\newcommand{\blocs}{\mathit{L}_{\mathrm{b}}}

\newcommand{\iflocs}{\mathit{L}_{\mathrm{if}}}
\newcommand{\looplocs}{\mathit{L}_{\mathrm{while}}}

\newcommand{\alocs}{\mathit{L}_{\mathrm{a}}}
\newcommand{\wlocs}{\mathit{L}_{\mathrm{w}}}
\newcommand{\rlocs}{\mathit{L}_{\mathrm{r}}}
\newcommand{\Alocs}[1]{\mathit{L}_{\mathrm{A}}^{\mathsf{#1}}}
\newcommand{\Dlocs}{\mathit{L}_{\mathrm{nd}}}
\newcommand{\transitions}{{\rightarrow}}

%%% 
\newcommand{\plocs}{\mathit{L}_{\mathrm{p}}}
\newcommand{\tlocs}{\mathit{L}_{\mathrm{t}}}

\newcommand{\lin}{\loc_\mathrm{init}}
\newcommand{\lout}{\loc_\mathrm{out}}
\newcommand{\val}[1]{\mbox{\sl Val}_{#1}}

\newcommand{\pvars}{V_\mathrm{p}}
\newcommand{\rvars}{V_{\mathrm{r}}}
\newcommand{\pre}{\mathrm{pre}}

\newcommand{\sle}{\sqsubseteq}
\newcommand{\sge}{\sqsupseteq}

\newcommand{\lfp}{\mathrm{lfp}}
\newcommand{\gfp}{\mathrm{gfp}}

\newcommand{\rdvarjdis}{\mathcal D}
\newcommand{\sampset}{\textit{supp}}

\newcommand{\upd}{\mbox{\sl upd}}
\newcommand{\wet}{\mbox{\sl wt}}
\newcommand{\transset}{\mathfrak T}
\newcommand{\valin}{\pv_{\mathrm{init}}}
\newcommand{\ret}{\mbox{\sl ret}}

\newcommand{\win}{w_{\mathrm{init}}}

\newcommand{\sampdpd}{\overline{\Upsilon}}

\newcommand{\outmap}{\text{O}}
\newcommand{\sat}[1]{\langle #1 \rangle}
\newcommand{\monoid}{\mbox{\sl Monoid}}
\newcommand{\handelmanformat}{(\dagger)}

\newcommand{\trunc}{\mathcal{B}}

\newcommand{\ewt}{\mbox{\sl ewt}}
\newcommand{\statemap}{\text{St}}

\newcommand{\valrd}{{\mathbf{r}}}
\newcommand{\frmloc}{\ell^{\mathrm{src}}}
\newcommand{\toloc}{\ell^{\mathrm{dst}}}

\newcommand{\monomials}{\mathbf{M}}

\newcommand{\yongyuan}[1]{\textcolor[rgb]{0.80, 0.10, 0.44}{{{#1}}}}

\title{	
Game-Theoretic Robust RL Handles Temporally-Coupled Perturbations}


\begin{document}
\author{%
  Yongyuan Liang${}^\dag$ \thanks{Corresponding author. \texttt{cheryunl@umd.edu}}
  \quad
  Yanchao Sun${}^\dag$
  \quad
  Ruijie Zheng${}^\dag$
  \quad
  Xiangyu Liu${}^\dag$
  \\
  \textbf{
  Benjamin Eysenbach${}^\S$
  \quad
  Tuomas Sandholm${}^\ddag$
  \quad
  Furong Huang${}^\dag$
  \quad
  Stephen McAleer${}^\ddag$}\\
  ${}^\dag$ University of Maryland, College Park \quad
 ${}^\ddag$ Carnegie Mellon University \quad
 ${}^\S$ Princeton University\\
}
%\iclrfinalcopy
\maketitle
\begin{abstract}
% Robust reinforcement learning (RL) seeks to train policies that can perform well under environment perturbations or adversarial attacks. Existing approaches typically assume that the space of possible perturbations remains the same across timesteps. However, in many settings, the space of possible perturbations at a given timestep depends on past perturbations.  
Deploying reinforcement learning (RL) systems requires robustness to uncertainty and model misspecification, yet prior robust RL methods typically only study noise introduced independently across time. However, practical sources of uncertainty are usually coupled across time.
We formally introduce temporally-coupled perturbations, presenting a novel challenge for existing robust RL methods. To tackle this challenge, we propose GRAD, a novel game-theoretic approach that treats the temporally-coupled robust RL problem as a partially-observable two-player zero-sum game. By finding an approximate equilibrium within this game, GRAD optimizes for general robustness against temporally-coupled perturbations. Experiments on continuous control tasks demonstrate that, compared with prior methods, our approach achieves a higher degree of robustness to various types of attacks on different attack domains, both in settings with temporally-coupled perturbations and decoupled perturbations.
% Empirical experiments on a variety of continuous control tasks demonstrate that our proposed approach exhibits significant robustness advantages compared to baselines against both standard and temporally-coupled attacks, in both state and action spaces.
\end{abstract}

\section{Introduction}

% Figure environment removed

Reinforcement Learning from Human Feedback (RLHF) has recently been used to great effect to align pretrained large language models (LLMs) to human preferences, optimizing for desirable qualities like harmlessness and helpfulness~\citep{bai2022training} and achieving state-of-the-art results across a variety of natural language tasks~\citep{openai2023gpt4}. %RLHF approaches fundamentally rely on collecting pairs of LLM outputs $(o_1, o_2)$ from a shared prompt $p$, with a human indicating which output in each pair is better on a specified attribute.
% A fundamental component of RLHF is a preference model derived from human labels, typically formatted as pairs of LLM outputs $(o_1, o_2)$ generated from a shared prompt $p$.

A standard RLHF procedure fine-tunes an initial unaligned LLM using an RL algorithm such as PPO~\citep{schulman2017proximal}, optimizing the LLM to align with human preferences. %\violet{not sure whether we need to provide this detail in the intro, especially this has nothing to do with our contribution.} % i feel like this context is useful later when e.g. explaining that context distillation is SFT
RLHF is thus critically dependent on a reward model derived from human-labeled preferences, typically \textit{pairwise preferences} on LLM outputs $(o_1, o_2)$ generated from a shared prompt $p$. % and labeled by humans. 

However, collecting human pairwise preference data, especially high-quality data, may be expensive and time consuming at scale. To address this problem, approaches have been proposed to obtain labels without human annotation, such as Reinforcement Learning from AI Feedback (RLAIF) and context distillation. 

\iffalse
raising the question of whether we can generate high-quality data for RLHF without using human labeling. %accurately-labeled preference pairs $(o_1, o_2)$
%, motivating model alignment approaches that aim to generate accurately-labeled preference pairs $(o_1, o_2)$ without human involvement. 
Two major categories of such approaches are . 
\fi

RLAIF approaches (e.g.,~\citet{bai2022constitutional}) simulate human pairwise preferences by scoring $o_1$ and $o_2$ with an LLM (Figure \ref{fig:rlcd_differences} center); the scoring LLM is often the same as the one used to generate the original pairs $(o_1, o_2)$. Of course, the resulting LLM pairwise preferences will be somewhat noisier compared to human labels. However, this problem is exacerbated by using the same prompt $p$ to generate both $o_1$ and $o_2$, causing $o_1$ and $o_2$ to often be of very similar quality and thus hard to differentiate (e.g., Table~\ref{tab:rlaif_bad_example}). Consequently, training signal can be overwhelmed by label noise, yielding lower-quality preference data. 

% While it avoids human labeling efforts, it has weakness. First, LLM preference labels will naturally be somewhat noisier compared to human labels. Furthermore, since the same prompt $p$ is used to generate both $o_1$ and $o_2$, their quality is often very similar and hard to differentiate (See Table~\ref{tab:rlaif_bad_example}). As a result, training signals can be overwhelmed by label noise, yielding lower-quality preference data. 

Meanwhile, context distillation methods (e.g., \citet{sun2023principle}) create more training signal by modifying the initial prompt $p$. 
%to create more significant training signal. 
The modified prompt $p_+$ typically contains additional context encouraging a \textit{directional attribute change} in the output $o_+$ (Figure \ref{fig:rlcd_differences} right). However, context distillation methods only generate a single output $o_+$ per prompt $p_+$, which is then used for supervised fine-tuning, losing the pairwise preferences which help RLHF-style approaches to 
%rather than using a RLHF-style preference model to 
derive signal from the contrast between outputs. 
Multiple works have observed that RL approaches using preference models for pairwise preferences can substantially improve over supervised fine-tuning by itself when aligning LLMs~\citep{ouyang2022training,dubois2023alpacafarm}. 

% conduct alignment by running supervised fine-tuning on model outputs $o_+$ generated from a modified prompt $p_+$. $p_+$ typically contains additional context encouraging desirable attributes (Figure \ref{fig:rlcd_differences} right), such as in \citet{sun2023principle}. However, multiple works have observed that RLHF-style approaches can substantially improve over supervised fine-tuning by itself when aligning LLMs~\citep{ouyang2022training,dubois2023alpacafarm}. 

Therefore, while both RLAIF and context distillation approaches have already been successfully applied in practice to align language models, we posit that it may be even more effective to combine the key advantages of both. That is, we will use RL with \textit{pairwise preferences}, while also using modified prompts to encourage \textit{directional attribute change} in outputs. %In particular, we will adapt the RLAIF data generation process with two different prompts rather than a single $p$, modifying both prompts similarly to context distillation. %\violet{this motivation is a little unexciting. I think we can more specifically discuss the potential benefits of our approach, like the benefits from RL: exploration/data generation; benefits from contrast. I don't think we get too much benefits from context distillation since we switched to the RL framework.} 

Concretely, we propose \oursfull{} (\ours{}). 
\ours{} generates preference data as follows. Rather than producing two i.i.d.\ model outputs $(o_1, o_2)$ from the same prompt $p$ as in RLAIF, \ours{} creates two variations of $p$: a \textit{positive prompt} $p_+$ similar to context distillation which encourages directional change toward a desired attribute, and a \textit{negative prompt} $p_-$ which encourages directional change \textit{against} it (Figure \ref{fig:rlcd_differences} left). We then generate model outputs $(o_+, o_-)$ respectively, and automatically label $o_+$ as preferred---that is, \ours{} automatically ``generates'' pairwise preference labels by construction. %, without further post hoc labeling.\violet{should make it clearer that our approach `generates' labels by construction} 
We then follow the standard RL pipeline of training a preference model followed by PPO. 

Compared to RLAIF-generated preference pairs $(o_1, o_2)$ from the same input prompt $p$, there is typically a clearer difference in the quality of $o_+$ and $o_-$ generated using \ours{}'s directional prompts $p_+$ and $p_-$, which may result in less label noise. %which may result in better training signal for the preference model. 
That is, intuitively, \ours{} exchanges having examples be \textit{closer to the classification boundary} for much more \textit{accurate labels} on average. Compared to standard context distillation methods, on top of leveraging pairwise preferences for RL training, \ours{} can derive signal not only from the positive prompt $p_+$ which improves output quality, but also from the negative prompt $p_-$ which degrades it. %\ours{} is not learning to imitate $o_+$, but to distill the \textit{contrast} between $o_+$ and $o_-$. 
Positive outputs $o_+$ don't need to be perfect; they only need to contrast with $o_-$ on the desired attribute while otherwise following a similar style.

% \todo{discuss our method and why intuitively it may be better.}

We evaluate the practical effectiveness of \ours{} through both human and automatic evaluations on three tasks, aiming to improve the ability of LLaMA-7B~\citep{touvron2023llama} to generate harmless outputs, helpful outputs, and high-quality story outlines. %\ours{} outperforms both RLAIF and context distillation baselines in pairwise comparisons on 
As shown in Sec. \ref{sec:experiments}, \ours{} substantially outperforms both RLAIF and context distillation baselines in pairwise comparisons when simulating preference data with LLaMA-7B, while still performing equal or better when simulating with LLaMA-30B. 
%On all three tasks, \ours{} substantially outperforms both RLAIF and context distillation baselines in pairwise comparisons---by a margin of at least 9\% and often more than 30\%---validating our method's efficacy. 
We will release all code at a later date, although in any case \ours{} is fairly easy to implement by modifying any reference RLAIF codebase. %We release all code at \todo{github link}.
\section{Mathematical preliminaries} \label{Sec:Prelim}
In this section, we introduce some useful definitions and results regarding the fractional derivative in the sense of Riemann--Liouville and recall the Aubin--Lions lemma, which is a key result featuring in proofs of existence of weak solutions to nonlinear PDEs based on compactness arguments. %

For a Hilbert space $H$ with inner product $(\cdot,\cdot)_H$ and norm $\|\cdot\|_H$, we shall denote the duality pairing between $H$ and its dual space $H'$ by $\langle \cdot,\cdot\rangle_H$. We shall denote the inner product on the Bochner space $L^2(0,T;H)$ by $(\cdot,\cdot)_{L^2H}$, and we shall write $(\cdot,\cdot)_{L^2_tH}$ when in this inner product the temporal interval of integration is $(0,t)$ for some $t \in (0,T)$ rather than the complete interval $(0,T)$, i.e., $$(u,v)_{L^2_tH}:=\int_0^t (u(s),v(s))_H \, \text{d}s \qquad \forall\, u,v \in L^2(0,T;H).$$
The norm induced by this inner product will be denoted by $\|\cdot\|_{L^2_t H}$.



\subsection{Riemann--Liouville kernels}
The Riemann--Liouville kernel function $g_\alpha$ of order $\alpha$ is defined by $g_\alpha(t):=t^{\alpha-1}/\Gamma(\alpha)$, $t \in (0,T)$, for $\alpha > 0$ and $g_0(t):=\delta_0(t)$ (the Dirac distribution concentrated at $0$) for $\alpha=0$. We observe that  $g_\alpha \in L^p(0,T)$ for any $\alpha\in (1-1/p,1)$ and $p \in [1,\infty)$, and the kernel function satisfies the following semigroup property; see \cite[Theorem 2.4]{diethelm2010analysis}:
\begin{equation} \label{Eq:Semigroup}
	\ga * g_\beta = g_{\alpha+\beta} \qquad \forall\, \alpha,\beta \geq 0.
\end{equation} 
%This can be proved as follows: One applies Fubini's theorem to interchange the order of integration 
%$$\begin{aligned}(\ga * g_\beta *u)(t) &=\frac{1}{\Gamma(\alpha)\Gamma(\beta)} \int_0^t (t-s)^{\alpha-1} \int_0^s (s-\tau)^{\beta-1} u(\tau) \dd \tau \dd s  \\
%	&=\frac{1}{\Gamma(\alpha)\Gamma(\beta)} \int_0^t u(\tau) \int_\tau^t (t-s)^{\alpha-1} (s-\tau)^{\beta-1}   \dd s \dd \tau,
%\end{aligned}$$
%and the substitution $s=\tau+\sigma(t-\tau)$ then yields
%$$\begin{aligned}(\ga * g_\beta *u)(t) 
%	&=\frac{1}{\Gamma(\alpha)\Gamma(\beta)} \int_0^t u(\tau) (t-\tau)^{\alpha+\beta-1} \int_0^1 (1-\sigma)^{\alpha-1} \sigma^{\beta}   \dd \sigma \dd \tau.
%\end{aligned}$$
%Lastly, we observe using the fundamental property of the Gamma function that the second integral is equal to $\Gamma(\alpha)\Gamma(\beta)/\Gamma(\alpha+\beta)$, see \cite[Theorem D.6]{diethelm2010analysis}, from which we deduce the desired semigroup property \cref{Eq:Semigroup} of $g_\alpha$.

We note that when $\alpha \in (0,1)$, one can bound the $L^p(0,t)$-norm of a function $u:(0,T) \to \R$ by its convolution with $\ga$ as follows: for any $t \in (0,T]$, we have that
\begin{equation}\begin{aligned} \|u\|_{L^p(0,t)}^p := \int_0^t |u(s)|^p \ds  &\leq t^{1-\alpha} \int_0^t (t-s)^{\alpha-1} |u(s)|^p \ds \\ &\leq T^{1-\alpha} \Gamma(\alpha) \big(\ga * |u|^p\big)(t).	
\end{aligned} 
\label{Eq:KernelNorm}
\end{equation}
This implies that the space $$L^p_\alpha(0,T):=\big\{u:(0,T) \to \R:\sup_{t \in (0,T)} (\ga*|u|^p)(t) < \infty \big\},$$ is indeed a subspace of $L^p(0,T)$.
%Further, this estimate can be generalized for a nonnegative function $u:(0,T) \to \R_{\geq 0}$ and for $0<\beta<\alpha<1$ in the following way:
%$$(\ga * u)(t)=\frac{1}{\Gamma(\alpha)} \int_0^t (t-s)^{\beta-1} \frac{(t-s)^{\alpha-1}}{(t-s)^{\beta-1}} u(s) \ds \leq \frac{T^{\alpha-\beta}\Gamma(\beta)}{\Gamma(\alpha)} (g_\beta * u)(t).$$
If the order $\alpha$ of the kernel function $g_\alpha$  is larger than 1, then one can exploit the semigroup property of the kernel and apply Young's convolution inequality (cf. Lemma 3.2 in \cite{Oparnica}) as follows:
$$(g_{1+\alpha}*u)(t)=(g_1*\ga*u)(t)=\int_0^t (\ga*u)(s) \ds \leq \|\ga\|_{L^1(0,t)} \|u\|_{L^1(0,t)},$$
for any $u \in L^1(0,T)$ and any $t \in (0,T]$.


\subsection{Time-fractional derivative} 
We can rewrite the definition of the Riemann--Liouville derivative stated in \cref{Eq:RL} in a compact form by using the convolution operator $*$ as $\pta w=\pt (\gb * w)$.
We refer to the classical textbooks \cite{diethelm2010analysis,baleanu2012fractional} and the newer monographs \cite{jin2021fractional,chen2022fractional} regarding fractional calculus and fractional differential equations.

 We define the fractional Riemann--Liouville--Bochner space   for $\alpha \in (0,1)$ and $p \in [1,\infty)$ on $(0,T)$ with values in $H$ by $$\W^{\alpha,p}(0,T;H):=\big\{u \in L^p(0,T;H) : \gb * u \in W^{1,p}(0,T;H)\big\}.$$
Here, the convolution $\ast$ is of course understood to be with respect to the temporal variable $t \in (0,T)$. In the limit, 
when $\alpha=1$, we have that $g_{1-\alpha} = g_0=\delta$, and then $$\W^{1,p}(0,T;H):=W^{1,p}(0,T;H):=\big\{u \in L^p(0,T;H) : \pt u \in L^p(0,T;H)\big\}.$$
However for $0 < \alpha < 1$,
the Riemann--Liouville space $\W^{\alpha,p}(0,T;H)$ differs from the fractional-order Sobolev--Bochner space
$$W^{\alpha,p}(0,T;H):=\Big\{u \in L^p(0,T;H) : (s,t) \mapsto \tfrac{\|u(t)-u(s)\|_H}{|t-s|^{\alpha+1/p}} \in L^{p}((0,T)\times(0,T))\Big\},$$
which can be confirmed by noting that the function $g_\alpha$ is an element of $\W^{\alpha,p}(0,T):=\W^{\alpha,p}(0,T;\R)$ for $\alpha \in (1-\tfrac{1}{p},1)$ but not of $W^{\alpha,p}(0,T)$; see \cite[Proposition 3.13]{carbotti2021note}. 
%Therefore, we are not able to apply classical results such as embedding theorems for Sobolev--Bochner spaces.

\begin{remark}  
Even though the space $\W^{\alpha,p}(0,T)$ is not a subspace of the Sobolev--Slobodecki\u{\i} space $W^{\alpha,p}(0,T)$, it is nevertheless continuously embedded into $C([0,T])$, the space of uniformly continuous functions defined on $[0,T]$, for $\alpha \in (1- \frac{1}{p},1]$ and $p \in [1,\infty)$; see, \cite[Remark 6.2]{carbotti2021note}. 
\end{remark}

\begin{comment}
Therefore, small values of $\alpha$ have to be studied carefully.
Further, we observe that $\ga$ does not belong to $L^p(0,T)$ for $\alpha \in (0,1-\tfrac{1}{p}]$ (e.g., $\ga \notin L^2(0,T)$ for $\alpha \in (0,\frac12]$) and therefore, we find that $$(\gb*\phi)(0)=0 \qquad \forall\, \phi \in \W^{\alpha,p}(0,T;H),~ \alpha \in (0,1-\tfrac{1}{p}],$$
by the inverse convolution property  \cref{Eq:InverseConvolution}. However, this might contradict a given nontrivial initial condition $\phi^0$. E.g., for $\phi \in \W^{\alpha,2}(0,T;H):=\W^{\alpha,2}(0,T;H)$ it has to hold that $(\gb*\phi)(0)=0$ for $\alpha\in (0,\tfrac12]$ and therefore, PDE solutions with this regularity are only well-posed for $\phi^0=0$. Such an issue can be avoided by studying PDEs of the form $\pta(\phi-\phi^0)=f(\phi)$ and considering instead the regularity of $\phi-\phi^0$, i.e., $\phi \in \W^{\alpha,2}_{\phi^0}(0,T;H)$. However, the time-fractional model \cref{Eq:System} in this work is not of this translated form and therefore, we cannot expect that this system is well-posed for non-zero initials. We note that $\psi_0=0$ is physically unreasonable anyway for probability density functions and, moreover, we will naturally observe in the existence's proof below that the restriction $\alpha \geq \frac12$ naturally appears in the energy estimates.
\end{comment}




We also introduce the following Riemann--Liouville space incorporating a homogeneous initial condition at $t=0$, albeit in a somewhat nonstandard manner:
$$\begin{aligned}
\W^{\alpha,p}_{0}(0,T;H)&:=\big\{u \in \W^{\alpha,p}(0,T;H) : (\gb*u)(0)=0 \big\}.
%\W^{\alpha,p}_{u^0}(0,T;H)&:=\big\{u \in L^p(0,T;H) : u-u^0 \in \W_0^{\alpha,p}(0,T;H) \big\}.
\end{aligned}$$ 
We note that the function $\gb*u:[0,T] \to H$ has a well-defined trace at $t=0$ (even when the function $u$ itself might not have one) thanks to the continuous embedding
$$\gb*u \in W^{1,p}(0,T;H) \hookrightarrow AC([0,T];H).$$
For a given element $z \in H$, the convolution $\gb*z$ should be understood to mean the function $t \mapsto (\gb*g_1)(t) z \in H$; recall that $g_1(t)\equiv 1$ for all $t\geq 0$. Thus, $z \in H$ is in this context now thought of as the mapping $t \mapsto g_1(t)z \in \mathcal{W}^{\alpha,p}(0,T;H)$,  for $\alpha \in (0,1)$, $p \in [1,\infty)$ and $0<\alpha p < 1$, or if $\alpha =1$ and $p \in [1,\infty)$. 
Thanks to the semigroup property \eqref{Eq:Semigroup} we then have that $$t\mapsto (\gb*z)(t)=z\,g_{2-\alpha}(t)=\frac{z}{\Gamma(2-\alpha)}  t^{1-\alpha} \in C([0,T];H)$$ for any $\alpha \in [0,1]$. Thus, for $\alpha \in (0,1]$, $p \in [1,\infty)$ and $z \in H$ we define the following `translated' Riemann--Liouville space:
\begin{equation} \label{Eq:RLSpaceU0}
    \W^{\alpha,p}_{z}(0,T;H):=\big\{u \in L^p(0,T;H) : u-z \in \W^{\alpha,p}(0,T;H), ~ (g_{1-\alpha}*u)(0)=0 \big\}.
\end{equation}
Note that if $\alpha \in (0,1)$, $p \in [1,\infty)$ and $0<\alpha p<1$, or if $\alpha=1$ and $p \in [1,\infty)$, then $u-z \in \W^{\alpha,p}(0,T;H)$ if, and only if $u \in \W^{\alpha,p}_0(0,T;H)$, and therefore, for such $\alpha$ and $p$ we have that $\W^{\alpha,p}_{z}(0,T;H) = \W^{\alpha,p}_0(0,T;H)$ irrespective of the choice of $z \in H$.

Next, we state an inverse convolution (or deconvolution) property. Its name stems from the fact that convolution with the kernel $\ga$ acts as an inverse mapping on the operator of taking $\alpha$-th fractional derivative, up to a term that involves the initial value at $t=0$.
%
\begin{lemma}[Inverse convolution] Let $\alpha \in (0,1]$ and $p\in [1,\infty)$. Suppose further that $H$ is a Hilbert space and $z \in H$. Then, for any $t \in (0,T)$, we have the following equalities:
\begin{align} 	\label{Eq:InverseConvolution1}
(\ga * \pta u)(t) &= u(t) - (\gb*u)(0)\ga(t) \quad &&\forall\, u \in \W^{\alpha,p}(0,T;H), \\ 	\label{Eq:InverseConvolution}
		(\ga* \pta u)(t)    &=u(t)  &&\forall\, u \in \W_{z}^{\alpha,p}(0,T;H). \end{align} 
\end{lemma}
\begin{proof}
	We start with the proof of the equality \cref{Eq:InverseConvolution1}.
Recall that for any function $u \in \W^{\alpha,p}(0,T;H)$ we have $\gb*u \in  AC([0,T];H)$, and the fundamental theorem of calculus for absolutely continuous functions therefore yields, for any $t\in [0,T]$,
$$(\gb *u)(t) - (\gb*u)(0) = \int_0^t \partial_s (\gb * u)(s) \ds=(g_1 * \pta u)(t).$$
We convolve this equality with the kernel $\ga$ and make use of the semigroup property \cref{Eq:Semigroup} to obtain
$$(g_1*u)(t) - (\gb*u)(0) g_{1+\alpha}(t)=g_{1+\alpha}*\pta u,$$
where we have used that  $g_\alpha*1=g_\alpha*g_1=g_{1+\alpha}$, because $\alpha \Gamma(\alpha) = \Gamma(1+\alpha)$.
Next, we differentiate this equality in $t$ and observe that $\pt (g_1*u)=u$, $\pt g_{1+\alpha}=g_\alpha$, and $\pt (g_{1+\alpha}*v)=g_\alpha*v$, which yields \cref{Eq:InverseConvolution1}.
We finally note that  \cref{Eq:InverseConvolution} follows trivially from \cref{Eq:InverseConvolution1} and \cref{Eq:RLSpaceU0}.
%We consider an element $u \in \W_{u^0}^{\alpha,p}(0,T;H)$, i.e., there exists an element $v \in \W_{0}^{\alpha,p}(0,T;H)$ with $u-u^0=v$ and using $v$ in \cref{Eq:InverseConvolution1}, we obtain $\ga*\pta v=v$, i.e.,
%\begin{equation*} \begin{aligned}\ga* \pta (u-u^0)   &= u-u^0 &&\forall\, u \in \W_{u^0}^{\alpha,p}(0,T;H).
	%\end{aligned} \end{equation*} 
%Moreover, we can split the left-hand side thanks to the linearity of the fractional derivative and obtain 
 %\begin{equation*} \begin{aligned}
	%	\ga* \pta u    &= u-u^0  + \ga * \pta u^0 =u  &&\forall\, u \in \W_{u^0}^{\alpha,p}(0,T;H),
	%\end{aligned} \end{equation*} 
	%where we have used that   $\ga * \pta 1 = \ga*\gb =1$ thanks to the semigroup property  \cref{Eq:Semigroup}.
 \end{proof}
 
The following result is a direct consequence of the interaction between fractional derivatives and kernel functions.
\begin{corollary} The following identities hold:
\begin{equation} \label{Eq:DerivativeofKernel}  \begin{aligned}
	\pta (\ga * u ) &=\pt ( \gb * \ga * u) = \pt (1*u) = u &&\forall\, u \in L^1(0,T;H), \\
	\ptb \pta u &=  \pt (\ga * \pta u) = \pt u &&\forall\, u \in  W_{0}^{1,1}(0,T;H).
\end{aligned}
\end{equation}
\end{corollary}




%However, in our setting of the time-fractional Navier--Stokes--Fokker--Planck system, we have already seen that the Riemann--Liouville derivative appears on the left-hand side without the translation of an initial value. This already explains intuitively the restriction on the values of $\alpha$ in the theorem of the system's well-posedness, see \cref{Thm:WellPosedness} below.


%As in the integer-order setting, there are continuous and compact embedding results for Riemann--Liouville spaces.
We shall require the following special case of the classical Aubin--Lions lemma; see \cite{simon1986compact}. Suppose that the Hilbert spaces $V,H,Z$ form a Gelfand triple $V \com H \con Z$. then, the following classical compact embeddings hold: 
\begin{equation} \begin{aligned} \label{Eq:aubin} 
W^{1,1}(0,T;Z) \cap L^p(0,T;V) &\com L^{p}(0,T;H), &&p \in [1,\infty), \\
W^{1,r}(0,T;Z) \cap L^\infty(0,T;V) &\com C([0,T];H), && r \in (1,\infty);
\end{aligned}\end{equation} 
see \cite{simon1986compact}. Several fractional counterparts of the Aubin--Lions lemma have been proposed; see \cite{ouedjedi2019galerkin,wittbold2020bounded,li2018some}. We make use of  the following result; see \cite[Corollary 3.2]{ouedjedi2019galerkin}:
\begin{equation*} \begin{aligned} %
\W^{\alpha,1}(0,T;Z) \cap L^p(0,T;V) &\com L^r(0,T;H), &&p \in (1,\infty), \quad r \in [1,p), \quad \alpha \in (0,1).
\end{aligned}\end{equation*} 
The proof can be easily adapted to the limit case $r=p$ if the $\alpha$-th fractional derivative is in a better space than $L^1(0,T;Z)$. This is done for Caputo derivatives in \cite{li2018some}. In fact, we obtain
%$r \in (\frac{p}{1+\alpha p},\infty) \cap [1,\infty)$.  In the spacial case when $p=2$ and $\alpha \in (\frac12,1]$ it yields
\begin{equation} \begin{aligned} \label{Eq:aubinfractional2} %
		\W^{\alpha,r}(0,T;Z) \cap L^p(0,T;V) &\com L^p(0,T;H), &&r\in (1,\infty), \quad \alpha \in (0,1).
\end{aligned}\end{equation} 

\begin{comment}
Next, we require a Gronwall-type inequality that allows an additional nonnegative factor $b \in L^1(0,T)$ in the integrand on the right-hand side of the inequality. Particularly, this function is only assumed to be integrable, and it is allowed to degenerate.
\begin{lemma}[Gronwall, cf. {\cite[Lemma II.4.10]{boyer2012mathematical}}] \label{Lem:Gron4}
    Let $C_1,C_2$ be  nonnegative constants and let $b\in L^1(0,T)$ be nonnegative. If the  function $u \in L^\infty(0,T)$  satisfies the inequality
    $$u(t) \leq C_1+C_2 \int_0^t b(s) u(s) \, \textup{d}s \qquad \text{for a.a. } t \in (0,T], $$
    then 
    $$u(t) \leq C_1 \textup{exp}\Big(C_2\int_0^t  b(s) \, \textup{d}s\Big) \qquad \text{for a.a. } t \in (0,T]. $$
\end{lemma}
\end{comment}


%\subsection{Fractional chain inequality}
The classical chain rule does not hold for fractional derivatives, but one can use the following inequality as a remedy; see \cite[Theorem 2.1]{vergara2008lyapunov}:
\begin{equation} \label{Eq:ChainOriginal}  \frac12 \pta \|u\|^2_H +\frac12 \gb(t) \|u\|_H^2 \leq (u,\pta u)_H \quad \forall\, u \in \W_{z}^{\alpha,2}(0,T;H),
\end{equation}
for $z \in H$ and almost all $t \in (0,T)$.
%Here, it has to be assumed that  $\big(\gb*(u-u^0)\big)(0)=0$. 

\begin{comment}
We will see that the initial condition $\psi^0$ to the time-fractional system considered later on does not satisfy $\big(\gb*(\phi-\psi^0)\big)(0)=0$. Instead, we say that the solution satisfies the initial condition if $(\gb*\phi)(0)=\psi^0$. One can transform this into the form from before by noting that
$$0=(\gb*\phi)(0)-\psi^0=(\gb*\phi)(0)-(\gb*\ga \psi^0)(0)=(\gb*(\phi-\ga \psi^0))(0).$$
Next, we derive a fractional chain inequality for such functions so that we can apply such a result later on in the existence proof.

We consider a fixed $u \in \H_{0}^\alpha(0,T;H)$ and we  introduce $v=u+\ga z$. Since in this case  $u^0=0$, this gives $(\gb*v)(0)=z$. 
%We note that $\pta v= \pta u$ and therefore we obtain
%$$\begin{aligned}
	%(v,\pta v)_H  &=(u,\pta %u)_H+(\ga z, \pta u)_H 
	%\\ &\geq \frac12 \pta %\|u\|_H^2 + \frac12 \gb \|u\|_H^2 + \ga  (z,\pta v)_H,
%\end{aligned}$$
Using the fractional chain inequality 
\cref{Eq:ChainOriginal} for $u=v-\ga z$, we trivially find
\begin{equation} \label{Eq:Chain} (v-\ga z,\pta v)_H \geq \frac12 \pta \|v-\ga z\|_H^2.\end{equation}

%We note that $u=\ga * \pta u$ thanks to the  inverse convolution property \cref{Eq:InverseConvolution} and inserting  $u=v-g_\alpha z$ yields $v-g_\alpha z=g_\alpha * \pta v$. Therefore, we can also write, instead of \cref{Eq:Chain},
%\begin{equation*}  (\ga * \pta v,\pta v)_H \geq \frac12 \pta \|\ga*\pta v\|_H^2.\end{equation*}
One might ask oneself if there is a generalized inequality of the form of $(\ga*u,u)_H \geq \frac12 \pta \|\ga * u\|_H^2$ for a sufficiently smooth function $u$, i.e., whether the convolution with $\ga$ is coercive in some sense. We partly answer this question in the next lemma.
\end{comment}
%
%We shall also require the inequalities stated in the next lemma. 

%\begin{lemma} Let $z\in H$ be given. For any $u \in \W_{0}^{\alpha,2}(0,T;H)$ and any $v \in \W^{\alpha,2}(0,T;H)$ with $(\gb * v)(0)=z \in H$ we have the following inequalities:
%\begin{align} \label{Eq:Coercive}   \int_0^t (u,\pta u)_H \ds &\geq \cos(\alpha \pi/2)\|\pt^{\alpha/2} u\|_{L^2_tH}^2, \\
%\int_0^t \! (v\!-\!\ga z,\pta v)_H \ds  &\geq \cos(\alpha \pi/2)  \Big( \tfrac12 \|\pt^{\alpha/2} v\|^2_{L^2_tH}-\tfrac{\Gamma(\alpha-1)}{\Gamma(\alpha/2)^2}g_{\alpha}(t) \|z\|_H^2 \Big).\label{Eq:Coercive2} 
%\end{align}
%\end{lemma}
%\begin{proof}
%By \cite[Lemma 3.1]{mustapha2014well} we have that
%\begin{equation} \label{Eq:Mustapha} \int_0^t (\ga*w,w)_H \ds \geq \cos(\alpha \pi/2) \|g_{\alpha/2} * w\|_{L^2_tH}^2 \qquad \forall\, w \in L^2_t(0,T;H).
%\end{equation}
%Hence, with $w=\pta u$ we obtain the inequality
%$$
%\int_0^t (u,\pta u)_H \ds \geq \cos(\alpha \pi/2) \|g_{\alpha/2} * \pta u \|_{L^2_tH}^2,$$
%where we have used the inverse convolution property $g_\alpha * \pta u=u$, see \cref{Eq:InverseConvolution} with $z=0$ to simplify the left-hand side. On the right-hand side, we can make use of the fact that $\pt^{\alpha}u =\pt^{\alpha/2}\pt^{\alpha/2} u$ and again the inverse convolution property \cref{Eq:InverseConvolution} to deduce that
%$$g_{\alpha/2}*\pt^{\alpha} u = g_{\alpha/2}*\pt^{\alpha/2} \pt^{\alpha/2} u = \pt^{\alpha/2} u.$$
%Therefore, we obtain the first of the inequalities stated in the lemma. We note that we can only split the $\alpha$-th derivative into the composition of two $\frac{\alpha}{2}$-th derivatives if the function $u$ satisfies an initial condition of the form $\big(\gb*u\big)(0)=0$. 
%
%
%For $v \in \W^{\alpha,2}(0,T;H)$ with $(g_{1-\alpha} \ast v)(0)=z \in H$, we define $u:=v-g_\alpha z$, and we find that $u \in \W_0^{\alpha,2}(0,T;H)$ and thus
%\begin{equation} \label{Eq:LemmaIneq} \begin{aligned} \int_0^t (v -\ga z,\pta v)_H \ds 
%%&\geq \cos(\alpha \pi/2) \|g_{\alpha/2} * \pta u \|_{L^2_tH}^2 
%%\\ &=\cos(\alpha \pi/2) \|g_{\alpha/2} * \pt^{\alpha/2} \pt^{\alpha/2} u\|_{L^2_tH}^2 \\ 
%\geq \cos(\alpha \pi/2)\|\pt^{\alpha/2} u\|_{L^2_tH}^2. \end{aligned} \end{equation}
%As, trivially, $(a/\sqrt{2} - \sqrt{2}b)^2 \geq 0$ for all $a, b \in \mathbb{R}$, it follows that  $|a-b|^2 \geq \frac{a^2}{2} - b^2$. Therefore, noting $(\pt^{\alpha/2} g_\alpha)^2= (g_{\alpha/2})^2=\tfrac{\Gamma(\alpha-1)}{\Gamma(\alpha/2)^2} g_{\alpha-1}$ and inserting $u=v-\ga z$ into the right-hand side of  \eqref{Eq:LemmaIneq} yields
%\begin{equation*} 
%\begin{aligned}
%\|\pt^{\alpha/2} u\|_{L^2_tH}^2  &=  \|\pt^{\alpha/2} (v-g_\alpha z)\|_{L^2_tH}^2 \\ 
%&\geq \frac12 \|\pt^{\alpha/2} v\|_{L^2_tH}^2 - \|\pt^{\alpha/2} \ga z\|_{L^2_tH}^2 \\
%&\geq    \frac12 \|\pt^{\alpha/2} v\|^2_{L^2_tH}-\tfrac{\Gamma(\alpha-1)}{\Gamma(\alpha/2)^2} \|z\|_H^2 \int_0^t g_{\alpha-1}(s)\ds
%%&\geq  \big|  \|\pt^{\alpha/2} v\|_{L^2_tH}-\big( \tfrac{\Gamma(\alpha-1)}{\Gamma(\alpha/2)^2} g_{\alpha}(t) \big)^{1/2} \|z\|_H \big|^2 
%\\
%&=    \frac12 \|\pt^{\alpha/2} v\|^2_{L^2_tH}-\tfrac{\Gamma(\alpha-1)}{\Gamma(\alpha/2)^2}g_{\alpha}(t) \|z\|_H^2.
%\end{aligned}
%\end{equation*}
%That completes the proof of the lemma.
%\end{proof}





\section{Model revisited} \label{Sec:Form}

Having summarised the required results from fractional calculus, we revisit the mathematical model that we have derived in \Cref{Sec:Derivation}.
Let us assume for the moment that the solution $\psi$ to the Fokker--Planck equation belongs to $\mathcal{W}^{1-\alpha,p}(0,T;H) \cap C([0,T];H)$ for some $\alpha \in (0,1)$ and a suitable Hilbert space $H$, to be chosen. As $\psi \in C([0,T];H)$, it follows that $\|(g_\alpha \ast \psi)(t)\|_H \leq \frac{t^\alpha}{\Gamma(1+\alpha)}\|\psi\|_{C([0,T];H)}$, and therefore  $(g_\alpha \ast \psi)(0)=0$. Hence, $\psi \in \mathcal{W}^{1-\alpha,p}_0(0,T;H)$. It then follows from \eqref{Eq:InverseConvolution1}, with $\alpha$ replaced by $1-\alpha$ and $u=\psi$ there, that $(g_{1-\alpha}\ast \partial_t^{1-\alpha} \psi)(t) = \psi(t)$ for $t \in (0,T)$. 
Motivated by these properties, we introduce the auxiliary function $\phi$ by
\begin{equation} \label{Eq:Substitute} \phi := \ptb \psi = \pt (\ga * \psi),
\end{equation} 
whereby $\psi=g_{1-\alpha}*\phi$. We then have that $\pt \psi = \pt(g_{1-\alpha} \ast \phi) = \pta \phi$; and, thanks to the assumed continuity of $\psi$ (i.e. $\psi \in C([0,T];H)$) it
makes sense to require attainment of the initial condition $\psi(0) = \psi^0$, i.e. $(g_{1-\alpha} \ast \phi)(0) = \psi^0$. We shall therefore introduce the substitution $\phi:=\partial_t^{1-\alpha} \psi$ in \eqref{Def:FP}, which results in the following system of equations:
\begin{equation} \begin{aligned}
	\pt u + (u \cdot \nablax) u - \nu \Delta_x u + \nablax p - \div_x \tau( \gb * \phi) &=0, \\
	\div_x u &=0, \\
	\pta \phi  + (u \cdot \nablax) \phi + \div_q (\omega(u) q \phi)-\tfrac{1}{2\lambda} \div_q(\nablaq \phi + U'q  \phi)   -\eps  \Delta_x  \phi&=0, \end{aligned} 
\label{Eq:System}
\end{equation}
subject to the initial conditions $u(0)=u^0$ and $(\gb*\phi)(0)=\psi^0$ for a given nonnegative $\psi^0$ that fulfils $\int_D \psi^0 \dq=1$. Furthermore, we equip the system with the following boundary conditions:
\begin{align}\label{eq:neumannbc}
\begin{aligned}
u &=0 \qquad\text{on } \partial \Omega \times (0,T), \\
\left(\tfrac{1}{2\lambda} (\nablaq \phi + U'q \phi)-\omega(u) q \phi \right) \cdot n_{\partial D} &=0 \qquad\text{on } \Omega \times \partial D \times (0,T), \\
\eps \nablax \phi \cdot n_{\partial \Omega} &=0 \qquad\text{on } \partial\Omega \times  D \times (0,T).
\end{aligned}
\end{align}




\subsection{The Maxwellian and Maxwellian-weighted function spaces} \label{Sec:Maxwell}
We introduce the normalized Maxwellian by 
\begin{equation}
	\label{Def:Max} M(q)=\frac{e^{-U(\tfrac12 |q|^2)}}{\int_D e^{-U(\tfrac12 |s|^2)} \dd s}.
\end{equation}
Moreover, we define the (Maxwellian-weighted) Hilbert spaces
$$\begin{alignedat}{5}
	&\mathcal{H}=\{h \in L^2(\Omega;\R^d): \div \, h =0\}, \qquad ~\mathcal{H}_0&&=\{h \in \mathcal{H} : h \cdot n_{\partial \Omega} = 0 \text{ on } \partial \Omega\}, 
	\\ &\mathcal{V}=\{v \in H^1(\Omega;\R^d) : \div \, v = 0\}, \qquad ~ \mathcal{V}_0&&=\{v \in \mathcal{V} : v|_{\partial \Omega} = 0 \text{ on } \partial \Omega\},
	\\  &\mathcal{Y}=L^2(\Omega \times D),  \qquad\quad\,\,\,   \widehat{\mathcal{Y}}= L^2_M(\Omega \times D) &&=\{y \in \mathcal{Y}: \|y\|_{\widehat{\mathcal{Y}}} := \|M^{1/2}y \|_\mathcal{Y}<\infty \}, 
	\\ &\mathcal{X} = H^1(\Omega \times D), \qquad\quad \widehat{\mathcal{X}}= H_M^1(\Omega \times D) &&= \{\phi \in \mathcal{X}: \|\phi\|_\hX <\infty  \},  \\
&\mathcal{Z}=H^1(D;H^1(\Omega)), ~ \widehat{\mathcal{Z}} =H_M^1(D; H^1(\Omega))&&=\{\zeta \in \mathcal{Z}: \|\zeta\|_\hZ <\infty  \}, 
\end{alignedat}$$
where the norms on $\hX$ and $\hZ$ are defined by $\|\phi\|_\hX^2:= \|\phi\|_\hY^2 + \|\nablaq \phi\|_\hY^2 + \|\nablax \phi\|_\hY^2$ and $\|\zeta\|_\hZ^2 :=\|\zeta\|_\hX^2 + \|\nablax \nablaq \zeta\|_\hY^2$. Obviously, $H_M^2(\Omega \times D) \subseteq \hZ$, where $H^2_M(\Omega \times D)$ is the subspace of $H^1_M(\Omega \times D)$ consisting of all functions defined on $\Omega \times D$ whose second (weak) partial derivatives belong to $\hY=L^2_M(\Omega \times D)$.
We refer to \cite{barrett2005existence} regarding theoretical results on these weighted Hilbert spaces. In particular, we have the Gelfand triples
$$\begin{aligned} &\HSV \com \HS \hookrightarrow \HSV', \quad &&\HSV_0 \com \HS_0 \hookrightarrow \HSV_0', \\
	&\mathcal{X} \com \mathcal{Y} \hookrightarrow \mathcal{X}', \quad &&\hX \com \hY \hookrightarrow \hX',
\end{aligned}$$
where $\mathcal{V}'$, $\mathcal{V}'_0$, $\mathcal{X}'$ and $\hX'$ denote the dual space of, respectively, 
$\mathcal{V}$, $\mathcal{V}_0$, $\mathcal{X}$ and $\hX$. 

Using the definition of the normalized Maxwellian $M$, see \cref{Def:Max}, we have that $$M(q)\nablaq M(q)^{-1}=-M(q)^{-1} \nablaq M(q) =\nablaq U(\tfrac12 |q|^2) = U'(\tfrac12 |q|^2) q.$$ We introduce the scaled variable $\hphi=\phi/M$ and  with the formula 
$$M \nablaq \hphi = \nablaq \phi + M \nablaq M^{-1} \phi  = \nablaq \phi + U'q \phi$$
we can rewrite the fractional Fokker--Planck equation in \cref{Eq:System} as
$$ \pta \phi + (u \cdot \nablax) \phi + \div_q \big(\omega(u) q \phi\big) = \tfrac{1}{2\lambda} \div_q(M \nablaq \phim) + \eps \Delta_x \phi.$$
As was indicated earlier, we shall confine ourselves here to considering the corotational model, i.e.,
$\omega(v)=-\omega(v)^{\mathrm{T}}$, $q^{\mathrm{T}} \omega(v) q = 0$; if $\div\, v = 0$ it then follows that 
\begin{equation}\label{Eq:SigmaZero} 
\big(M \hphi \,\omega(v)q,\nablaq \hphi\big)_{\mathcal{Y}} = \frac12 \big(M \omega(v)q, \nablaq \hphi^2\big)_{\mathcal{Y}}=-\frac12 \big(\div_q(M \omega(v) q),\hphi^2\big)_{\mathcal{Y}} = 0; \end{equation}
see \cite{barrett2009numerical,barrett2005existence}.
%
We note in passing that partial integration yields the following equalities: 
\begin{equation} \label{Eq:IntParts} \begin{aligned}
		-2\big(M \omega(u) q \hat\varphi,\nablaq \hphi\big)_{\mathcal{Y}} &=  \big(\nablax(M\hat\varphi \nablaq \hphi)q,u\big)_{\mathcal{Y}} + \big(u\cdot q, \div_x (M\hat\varphi\nablaq \hphi)\big)_{\mathcal{Y}}
		\\ &= \big(M \nablax \hat\varphi (\nablaq \hphi)^{\mathrm{T}} q,u\big)_{\mathcal{Y}} + \big(M \hat\varphi \nablax \nablaq \hphi\, q,u\big)_{\mathcal{Y}} \\ &\quad + \big(u \cdot q,M \nablax \hat\varphi \cdot \nablaq \hphi\big)_{\mathcal{Y}} + \big(u \cdot q,M \hat\varphi \,\div_x \nablaq \hphi\big)_{\mathcal{Y}}.
\end{aligned} \end{equation} 
%

We recall that the stress tensor $\tau(\psi) = \tau_1(\psi) + \tau_2(\psi)$ is of the form  given by \eqref{eq:tau1nd} and \eqref{eq:tau2nd}; i.e., 
%%%%%%%%%%%%%
%
\begin{align}\label{Eq:tau1tau2}\tau^1(\psi)=\gamma\, \C(\psi),\quad \tau^2(\psi)= \gamma \int_{D} \psi \d q  \ I_3,\quad  \C(\psi):= \int_{D} F(q) q^{\mathrm T} \psi \d q,
\end{align}
%
where $\gamma>0$ is a dimensionless constant.


%%%%%%%%%%%%
For $\C(M \hpsi)$, we are in a setting that allows us to deduce the following bound; see also \cite[Eq. (3.7)]{barrett2009numerical}:
\begin{equation} \label{Eq:C}  \begin{aligned}
		\int_\Omega |\C(M \hpsi)|^2 \d x & =
		\int_\Omega \left| \, \int_{D} F(q)  q^{\mathrm T} M \hpsi \d q \, \right|^2 \d x
		\\ &\leq \int_D M  | F(q)  q^{\mathrm T} |^2   \dq \, \int_{\Omega \times D} M|\hpsi|^2 \d(x,q)  \\ & \leq C  \| \hpsi\|_{\hY}^2 \quad \forall\, \hpsi \in \hY. \end{aligned}\end{equation}




\section{Methodology}
\label{sec:method}

\subsection{Overview}
\label{sec:method_fmwk}

As shown in~\cref{fig:method_fmwk}, the proposed unsupervised MOT framework is trained with the widely-used contrastive learning technique~\cite{chen2020simple,he2020momentum}. 
\lk{Specifically, for multi-object tracking}, objects within the tracklet ($\boldsymbol{k}_{+}$) should be pulled together and different tracklets ($\boldsymbol{k}_{-}$) should be separated. It can be mathematically formulated as:

\begin{equation}
% \begin{split}
    \mathcal{L}_{cl}( \boldsymbol{q}; \boldsymbol{k}_{+}; \boldsymbol{k}_{-} )= 
    - \log \frac{\exp(\boldsymbol{q} \cdot \boldsymbol{k}_{+} / \epsilon)}{\sum_{i}\exp(\boldsymbol{q} \cdot \boldsymbol{k}_{i} / \epsilon)}
  \label{eq:method_nce}
% \end{split}  
\end{equation}

\noindent where $\mathcal{L}_{cl}$ denotes the InfoNCE~\cite{oord2018representation} loss function, and $\epsilon$ is the temperature hyper-parameter~\cite{wu2018unsupervised}. 
Within a video, following the unsupervised tracking fashion~\cite{liu2022online,shuai2022id}, the positive and negative keys mainly come from two sources, \ie pseudo-labeled historical frame and self-augmented frame. 

\lk{However, two issues occur: (1) the uncertainty reduces the accuracy of pseudo-tracklets and (2) the randomly augmented samples fail to learn the inter-frame consistency.} 
We argue the above issues are not independent. 
\lk{By leveraging the uncertainty in turn,} the accurate pseudo-tracklets can guide the qualified positive and negative augmentations.

To address these two issues, we propose an uncertainty-aware pseudo-tracklet labeling strategy in \cref{sec:method_uoap}, which integrates a verification-and-rectification mechanism into the tracklet generation. Our method significantly improves the accuracy of pseudo-identities, especially in long-term interval. 
Then we propose a tracklet-guided augmentation strategy in \cref{sec:method_ada_aug}, which brings the temporary information into spatial augmentation. The augmented samples simulates the objects' motion. A hierarchical uncertainty-based sampling strategy is proposed for hard sample mining. More details are described in the following section.


\subsection{Uncertainty-aware Tracklet-Labeling}
\label{sec:method_uoap}

Accurate pseudo tracklet is critical in \liuk{learning feature consistency}. 
However, without manual annotation, \lk{the aggravated uncertainty makes} the tracklet-labeling a huge challenge due to various interference factors, including similar appearance among objects, frequent object cross and occlusions, \etc. 
\lk{In fact, the uncertainty can also be leveraged to improve the pseudo-accuracy in turn.}
In this section, we propose an \textbf{U}ncertainty-aware \textbf{T}racklet-\textbf{L}abeling (\textbf{UTL}) strategy for better pseudo-tracklets.

Given an input video sequence $V = \{I^{1}, I^{2}, \cdots, I^{N}\}$, each frame $I^{t}$ is annotated with the bounding boxes $B^{t} = \{b_{1}^{t}, b_{2}^{t}, \cdots, b_{M^{t}}^{t}\}$ of $M^{t}$ objects in $t_{th}$ frame, where $b_{i}^{t} = (cx_{i}^{t}, cy_{i}^{t}, w_{i}^{t}, h_{i}^{t})$ is the center coordinate and shape of the $i_{th}$ object $o_{i}^{t}$. As shown in~\cref{fig:method_fmwk}, \mywork~generates accurate pseudo-tracklets in four main steps:

1) \textbf{Association}. For a certain object $o_{i}^{t}$ in frame $I^{t}$, the $\ell_2$-normalized representation $\boldsymbol{f}_{i}^{t}$ can be expressed as $\boldsymbol{f}_{i}^{t} = {\phi}(I^{t}, b_{i}^{t})$, 
% \begin{equation}
%   \boldsymbol{f}_{i}^{t} = {\phi}(I^{t}, b_{i}^{t})
%   % / {\Vert {\phi}(I^{t}, b_{i}^{t}) \Vert}_{2}
%   \label{eq:method_feat}
% \end{equation}
where the embedding encoder is denoted as $\phi$.

To associate the objects in frame $I^{t}$ with the objects or trajectories in previous $I^{t \minus 1}$, a similarity matrix is constructed with their appearance embeddings:

\begin{equation}
  \boldsymbol{C} \in \mathbb{R}^{M^{t} \times M^{t \minus 1}}, \;
  c_{i,j} = {\boldsymbol{f}_{i}^{t}} \cdot  \boldsymbol{f}_{j}^{t \minus 1}
  \label{eq:method_matrix}
\end{equation}

\noindent where $c_{i,j}$ represents the cosine similarity between the $i_{th}$ object in frame $I^{t}$ and the $j_{th}$ object (or trajectory) in frame $I^{t \minus 1}$. Then the Hungarian algorithm~\cite{kuhn1955hungarian} is adopted to generate the identity association results.

2) \textbf{Verification}. However, the appearance representations are sometimes unreliable, especially in the unsupervised scenario. To solve this issue, an uncertainty metric is proposed to evaluate the association after the first stage.

% For an object $o_{i}^{t}$ in frame $I^{t}$, the similarities against the $M^{t \minus 1}$ objects in the previous frame can be expressed as:

% \begin{equation}
%   \boldsymbol{s}_{i} = \boldsymbol{C}_{i} = [c_{i,1}, c_{i,2}, \cdots, c_{i,M^{t \minus 1}}]^T
%   \label{eq:method_svec}
% \end{equation}

% Inspired by margin-based OOD detection~\cite{hendrycks2016baseline}, we assume that the assignment ($o_{i}^{t} \!\sim\! o_{j}^{t \minus 1}$) in the association stage is not convincing under the following circumstances:

% \begin{itemize}
%     \setlength{\itemsep}{0pt}
%     \item The assigned similarity between $o_{i}^{t}$ and $o_{j}^{t \minus 1}$ is relatively low (\ie, $c_{i,j} < m_1$).
%     \item The second-highest similarity with others ($c_{i,j_{2}}$) is close to the assigned $o_{j}^{t \minus 1}$ (\ie, $c_{i,j} - c_{i,j_{2}} < m_2$).
% \end{itemize}

% Based on these assumptions, we define an association-level uncertainty metric, which is formulated as:



Object association can be viewed as multi-category classification.
And confidence-score has been proved efficient and effective on detecting mis-classified examples~\cite{hendrycks2016baseline}.
Inspired by this, we propose to detect the mis-associated objects through the similarity-scores.


Given an object $o_{i}^{t}$ associated with $o_{j}^{t \minus 1}$ in the previous frame based on \cref{eq:method_matrix}, the association ($o_{i}^{t} \!\sim\! o_{j}^{t \minus 1}$) is unconvincing in two cases: 
1) the assigned similarity $c_{i,j}$ is relatively low (\eg, partial occlusion or motion blur) and 
2) there are other objects whose similarities are close to the assigned $c_{i,j}$ (\eg, similar appearance or indistinguishable embedding).
It can be formulated as:

\begin{equation}
  c_{i,j} < m_1; \quad c_{i,j_{2}} > c_{i,j} - m_2
  \label{eq:method_margin}
\end{equation}


\noindent 
where $m_1,m_2$ are constant margins. Only the second-highest similarity with others ($c_{i,j_{2}}$) is considered for simplicity.
In an ideal association, $c_{i,j}$ should be close to 1 and $c_{i,j_{2}}$ close to 0.
We thus proposed to estimate the association \lk{risk} as:

% \sigma_{i,j} = - \left( 
% \log c_{i,j} + \log \left( 1 - c_{i,j_{2}} \right)
% + \overline{\log \left( 1 - c_{i,l} \right) }
% \right)  
\begin{equation}
  \sigma_{i,j} = - \log c_{i,j} - \log \left( 1 - c_{i,j_{2}} \right)
  \label{eq:method_energy}
\end{equation}

Detailed derivation process refers to the supplementary materials.
Combining with \cref{eq:method_margin} and \cref{eq:method_energy} , an adaptive threshold is proposed:

\begin{equation}
  % \gamma_{i,j} = -\log \left( 1 + m_2 - c_{i,j} \right) -\log m_1 \left( 1 - m_3 \right)
  \gamma_{i,j} =  -\log m_1 - \log \left( 1 + m_2 - c_{i,j} \right)
  \label{eq:method_border}
\end{equation}

As shown in~\cref{fig:method_verify}, when the \lk{risk} $\sigma_{i,j}$ is higher than the threshold $\gamma_{i,j}$, the assignment ($o_{i}^{t} \!\sim\! o_{j}^{t \minus 1}$) should be re-considered. 
\lk{The \textbf{association uncertainty} is quantified as:}

\begin{equation}
  \delta_{i,j} = \sigma_{i,j} - \gamma_{i,j}
  \label{eq:method_uncertain}
\end{equation}

The results are not sensitive to the exact margins. We set $m_1 = 0.5$ and $m_2 = 0.05$ for a slightly better performance.
% More experimental details are shown in the supplementary materials.

The uncertain pairs after the verification stage and unmatched objects after the association stage are gathered as uncertain candidates for the rectification stage.


3) \textbf{Rectification}. 
The rectification stage is performed among the uncertain candidate. The similarities between two adjacent frames are no longer convincing.
% due to irregular motion, severe occlusion, and so on. 
More information should be taken into account, including motion \lk{estimation} and appearance \lk{variation} within a tracklet. 
% Specifically, intersection-over-union (IoU)~\cite{bewley2016simple} is the widely-used motion metric. At the same time, the tracklet embeddings can provide complementary appearance information.

For the uncertain candidates, \mywork~constructs another similarity matrix for the secondary rectification. 
First, \lk{the motion constraints should be relaxed}, so the association shares overlap \lk{higher than} $\beta$ 
% in adjacent frames 
\lk{are preserved}. 
Second, \lk{the appearance should not vary extremely fast}, so we adopt the averaged similarity between object $o_{i}^{t}$ and tracklet $trk_{j} = \{o_{j}^{t \minus K}, \cdots, o_{j}^{t \minus 1}\}$ within previous $K$ frames. 
In this stage, we solve the sub-problem of global identity assignments, which can be formulated as:

\begin{equation}
\begin{split}
  \boldsymbol{C}^\prime \in \mathbb{R}^{{M^{t}}^\prime \times {M^{t \minus 1}}^\prime} & \\
  c^\prime_{i,j} = \left( \frac{1}{K} \sum_{\hat{t} = t \minus K}^{t \minus 1} {\boldsymbol{f}_{i}^{t}} \cdot  \boldsymbol{f}_{j}^{\hat{t}} \right) 
            \times \mathbb{I} & \left( \text{IoU} \left( b_{i}^{t}, b_{j}^{t \minus 1} \right) > \beta \right) 
  \label{eq:method_recti}
\end{split}
\end{equation}

\noindent where $\mathbb{I}(*)$ is the indicator function. Then the match set is updated based on the Hungarian algorithm.

\lk{
\textit{Remark.} Our core contribution is the uncertainty-based verification mechanism, rather than the specific rectification, which shall be adjusted in practice. Empirically we set $\beta=0.1$ and $K=5$.
}

% Figure environment removed

4) \textbf{Propagation}. The pseudo-tracklets are propagated frame-by-frame. As shown in~\cref{fig:method_reidacc}, our strategy brings \lk{consistently} accurate pseudo-identities, \lk{\eg, reaching 97\% accuracy across 100 frames}.
% The pseudo-tracklets are progressively updated during the training process.
The long-term intra-tracklet consistency is successfully maintained.
% by the accurate pseudo-identities.

It is worth mentioning that the \lk{verification and rectification} stages can be naturally applied to the inference process to boost the performance, \lk{which does not conflict with existing association methods}.

\subsection{Tracklet-Guided Augmentation}
\label{sec:method_ada_aug}

The accurate pseudo-tracklets can guide the sample augmentation in the unsupervised MOT framework.
To learn the \liuk{inter-frame consistency}~\cite{chen2020simple,zhang2021fairmot}, good training samples should be diverse and \liuk{temporal-aware}. 
However, as illustrated in~\cref{fig:method_ada_aug}, existing methods usually treat augmentation and multi-object tracking as two isolated tasks, leading to ineffective augmentations. 
Instead, this paper utilizes the tracklet's spatial displacements to guide the augmentation process. 
According to a properly selected anchor pair, the proposed strategy makes the augmented frames aligned to the historical frames, simulating realistic tracklet movements. The proposed method concurrently focuses on the hard negative samples.
Details \lk{of the \textbf{T}racklet-\textbf{G}uided \textbf{A}ugmentation (TGA)} are given below.

% Figure environment removed

We introduce the temporal information into spatial transformation. 
First, given a current frame $I^{t}$ with $M^{t}$ objects, we select a source-anchor object $o_{a}^{t}$, whose bounding box is denoted as $b_{a}^{t} = (cx_{a}^{t}, cy_{a}^{t}, w_{a}^{t}, h_{a}^{t})$. Then, we choose a target-anchor $o_{a}^{t \minus \tau}$ in $(t \minus \tau)_{th}$  historical frame from the pseudo-tracklet $trk_{a} = \{o_{a}^{t_0}, o_{a}^{t_1}, \cdots, o_{a}^{t}\}$. 
Finally, to augment the current $I^{t}$ to align with historical $I^{t \minus \tau}$,  a tracklet-guided affine transformation can be expressed as:

\begin{equation}
  \begin{bmatrix}
      x^{t \minus \tau} \\ y^{t \minus \tau} \\ 1
  \end{bmatrix}
  =
  \boldsymbol{M}_{t}^{t \minus \tau}
  \begin{bmatrix}
      x^{t} \\ y^{t} \\ 1
  \end{bmatrix}
  =
  \begin{bmatrix}
      m_{11} & m_{12} & m_{13} \\
      m_{21} & m_{22} & m_{23} \\
      0      & 0      & 1
  \end{bmatrix}
  \begin{bmatrix}
      x^{t} \\ y^{t} \\ 1
  \end{bmatrix}
  \label{eq:method_affine}
\end{equation}

\noindent where $x^*,y^*$ are spatial coordinates, and $\boldsymbol{M}_{t}^{t \minus \tau}$ can be solved by direct linear transform (DLT) algorithm ~\cite{detone2016deep}. 
% with the corner locations of the anchor pair $(o_{a}^{t} \!\sim\! o_{a}^{t \minus \tau})$. 
Then an augmented frame $\tilde{I}^{t}$ is generated based on the tracklet-guided affine transformation with perspective jitter, which can be expressed as $\tilde{I}^{t} = \mathcal{T}\left(I^{t}, M_{t}^{t \minus \tau} \right)$.
% \begin{equation}
%   \tilde{I}^{t} = \mathcal{T}\left(I^{t}, M_{t}^{t \minus \tau} \right)
%   \label{eq:method_aug}
% \end{equation}

Intuitively, a proper anchor-selection is vitally important for our augmentation strategy. 

First, the identity accuracy of anchor pair $(o_{a}^{t} \!\sim\! o_{a}^{t \minus \tau})$ is important. In other words, the consistency of anchor tracklet $trk_{a}$ should be guaranteed. We thus design a tracklet-level uncertain metric based on the propagated association-level uncertainty defined in \cref{eq:method_uncertain}, which is formulated as:

\begin{equation}
  \Omega_{i} = \frac{1}{n} \sum_{s=t_0}^{t} \exp (\delta_{i}^{s})
  % \Omega_{i} = \sqrt[n]{\sigma_{i}^{t_0} \cdot \sigma_{i}^{t_1} \cdots \sigma_{i}^{t}}
  \label{eq:method_tenergy}
\end{equation}

\noindent where $\Omega_{i}$ represents the uncertainty of tracklet $trk_{i}$, \lk{and $n$ is the tracklet length}.
An uncertainty-based sampling strategy is designed to select the source anchor $o_{a}^{t}$ (along with the anchor $trk_{a}$) from the $M^{t}$ objects in frame $I^{t}$, which can be formulated as:

\begin{equation}
  p\left(a=i \mid t \right) 
  % = softmax\left(-\Omega_{i}\right)
  = \frac{\exp{\left(-\Omega_{i}\right)}}{\sum_{\hat{i}=1}^{M^{t}}\exp{\left(-\Omega_{\hat{i}}\right)}}
  \label{eq:method_sel_an_src}
\end{equation}

\noindent where $p\left(a=i \mid t \right)$ represents the probability to choose the $i_{th}$ tracklet $trk_{i}$ as the anchor $trk_{a}$.
The uncertain tracklet with high $\Omega$ is less likely to be selected, avoiding dramatic augmentations from erroneous pseudo-tracklets.

Second, hard negative samples matters in discriminablity learning. We tend to choose an indistinguishable (or, high uncertain) target anchor $o_{a}^{t \minus \tau}$ along the tracklet $trk_{i}$. The selection probability can be formulated as:

\begin{equation}
  p\left(\pi=t \minus \tau \mid a \right) 
  = \frac{\exp{\left(\delta_{a}^{t \minus \tau}\right)}}{\sum_{\hat{\tau}=t_0}^{t-1}\exp{\left(\delta_{a}^{t-\hat{\tau}}\right)}}
  \label{eq:method_sel_an_tgt}
\end{equation}

\lk{A visualization example are displayed in the supplementary material to illustrate the hierarchical sampling process.}

Compared with conventional random transformation, the proposed tracklet-guided augmentation is well-directed and tracking-related. 
\lk{Together with accurate pseudo-tracklets, \mywork~successfully improves the inter-frame consistency, as shown in \cref{fig:method_disc_vis}. }


% Figure environment removed

% \subsection{Momentum Memory Dictionary}
% \label{sec:method_md}


%To reuse the encoded samples from the intermediate mini-batches, we maintain a queue for each video in the memory dictionary by enqueueing the $M^{t}$ objects in the current frame and removing the oldest samples.
%In representation learning, high-quality negative samples play an essential role~\cite{chen2020simple,he2020momentum}. However, existing unsupervised trackers only take negative samples from adjacent frames, augmented frames, and the current frame itself. The lack of negative sample diversity prevents trackers from learning discriminative representations. \mywork~adopts a momentum dictionary mechanism to alleviate this problem.

%As shown in~\cref{fig:method_fmwk}, we build a memory dictionary for each \textit{independent} video input during training. Given an input image $I^{t}$ from video $V$, we randomly sample a number of negative object samples from other videos in the memory dictionary, so as to enrich the negative sample diversity. To reuse the encoded samples from the intermediate mini-batches, we maintain a queue for each video in the memory dictionary by enqueueing the $M^{t}$ objects in the current frame and removing the oldest samples.
% \vspace{-0.5em}
\section{Experiments}
\label{sec:exp}
% \vspace{-0.5em}
% This paper introduces a novel concept called temporally coupled attacks, which distinguishes itself from standard adversarial attacks by incorporating temporal coupling. Previous research has primarily focused on attackers with different functionalities, specifically targeting either the state space or the action space.
In our experiments, we investigate various types of attackers on different attack domains including state perturbations, action perturbations, model uncertainty and mixed perturbations. We will study a diverse set of attack and compare with state-of-the-art baselines.
% This evaluation sheds light on the effectiveness of \ours across a wide range of attack scenarios against different types of adversaries.

\noindent\textbf{Experiment setup.} \quad
Our experiments are conducted on five various and challenging MuJoCo environments: Hopper, Walker2d, Halfcheetah, Ant, and Humanoid, all using the v2 version of MuJoCo. We use the Proximal Policy Optimization (PPO) algorithm as the policy optimizer for \ours training. For attack constraint $\epsilon$, we use the commonly adopted values $\epsilon$ for each environment. We set the  temporally-coupled constraint $\bar{\epsilon} = \epsilon/5$ (with minor adjustments in some environments). Ablation experiments study the choice ofOther choices of $\bar{\epsilon}$ will be further discussed in the ablation studies.
Our experiments are conducted on five various and challenging MuJoCo environments: Hopper, Walker2d, Halfcheetah, Ant, and Humanoid, all using the v2 version of MuJoCo. We use the Proximal Policy Optimization (PPO) algorithm as the policy optimizer for \ours training. For attack constraint $\epsilon$, we use the commonly adopted values $\epsilon$ for each environment. We set the  temporally-coupled constraint $\bar{\epsilon} = \epsilon/5$ (with minor adjustments in some environments). Ablation experiments study the choice of $\bar{\epsilon}$.

We report the average test episodic rewards both under no attack and against the strongest adversarial attacks to reflect both the natural performance and robustness of trained agents, by training adversaries targeting the trained agents from scratch. For reproducibility, we train each agent configuration with 10 seeds and report the one with the median robust performance, rather than the best one. More implementation details are in Appendix~\ref{app:exp:imp}.
% Figure environment removed
% \vspace{-0.5em}
\paragraph{Case I: Robustness against state perturbations.}
In this experiment, our focus is on evaluating the robustness of our methods against state adversaries that perturb the states received by the agent. Among the alternating training~\citep{zhang2021robust, sun2021strongest} methods, PA-ATLA-PPO is the most robust, which trains with the standard strongest PA-AD attacker. As a modification, we train PA-ATLA-PPO* with a temporally-coupled PA-AD attacker. WocaR-PPO~\citep{liang2022efficient} is the state-of-the-art defense method against state adversaries. Our \ours method utilizes the temporally-coupled PA-AD attacker for training. Figure~\ref{fig:state_attacks} presents the performance of baseline and \ours under both non-temporally-coupled and temporally-coupled state perturbations. 

Despite being trained to handle temporally-coupled adversaries, our method also demonstrates strong performance in the non-robust (``natural'') setting, expecially on the high-dimensional Humanoid task.
% Even without training with a non-temporally-coupled state adversary, our method demonstrates better robustness under the non-temporally-coupled type of attack, particularly in the highest-dimensional and challenging environment, Humanoid, where it outperforms other methods by a large margin.
Under our temporally-coupled attacks, the average performance of \ours is 45\% higher than the strongest baseline.
% \ours shows the best robustness against all types of state adversarial attacks.
% Figure environment removed

% \vspace{-0.5em}
\paragraph{Case II: Robustness against action uncertainty.}
Beyond assessing the susceptibility of \ours to state attacks, we also investigate its robustness against action uncertainty, where the agent intends to execute an action but ultimately takes a different action than anticipated. We scrutinize two specific forms of action uncertainty, as outlined in prior work~\citep{tessler2019action}. The first one is action perturbations, introduced by an action adversary, which strategically adds noise to the agent's intended action. The second scenario revolves around model uncertainty, where, with a probability denoted as $\alpha$, an alternative action replaces the originally planned action output by the agent. These scenarios closely parallel real-world control situations, such as dealing with mass uncertainty (e.g., when a robot's weight changes) or facing sudden, substantial external forces (e.g., when an external force unexpectedly pushes a robot).

In our baseline comparisons, we include PR-MDP and NR-MDP~\citep{tessler2019action}, which are robust to action noise and model uncertainty. We also incorporate WocaR-PPO into our baseline evaluations. We train \ours using a temporally-coupled action adversary and evaluate its robustness in both action perturbation and model uncertainty scenarios.

\noindent\textbf{Action Perturbations.}\quad
To obtain a stronger evasion action perturbation rather than OU noise and parameter noise, we are the first to train an RL-based action adversary following the trajectory outlined in Algorithm~\ref{alg:ours}. This strategy aims to showcase the worst-case performance of our robust agents under action perturbations. For evaluation, we train both temporally-coupled and non-temporally-coupled action adversaries for each robust model. In Figure~\ref{fig:action_attacks}, we present the exceptional performance of \ours against standard and temporally-coupling action perturbations. \ours  demonstrates a high degree of robustness. For example, on the Humanoid task it outperforms the baselines by a 17\% margin for standard attacks and by a 40\% advantage against temporally-coupling action attacks. 
% Across other tasks, \ours  outperforms other methods, especially when confronted with temporally-coupling action attacks, where it exhibits a significant advantage.
These results provide evidence of \ours's defense mechanism against various types of adversarial attacks in the action space.

% Figure environment removed
\noindent\textbf{Model Uncertainty.}\quad
To evaluate robustness under model uncertainty, we consider a range of noise probabilities denoted as $\alpha$ in the range of [0, 0.05, 0.1, 0.15, 0.2]. These values represent the probability of a randomly generated noise replacing the action selected by the victim agent. As depicted in Figure~\ref{fig:uncertainty}, \ours exhibits superior robustness compared to action-robust baselines across a spectrum of $\alpha$ uncertainty value without explicit exposure to model uncertainty noises during training.
% Figure environment removed
\vspace{-0.5em}
\paragraph{Case III: Robustness against mixed adversaries.}
In prior works, adversarial attacks typically focused on perturbing either the agent's observations or introducing noise to the action space. However, in real-world scenarios, agents may encounter both types of attacks simultaneously. To address this challenge, we propose a mixed adversary, which allows the adversary to perturb the agent's state and action at each time step. We employ alternating training to create a baseline as Mixed-ATLA using this mixed adversary type. Our \ours model and Mixed-ATLA are trained with temporally-coupled mixed attackers. The detailed algorithm for the mixed adversary is provided in Appendix~\ref{alg:mixed-ad}.
% We believe that a well-trained mixed adversary not only holds practical significance but also provides a more comprehensive validation of the effectiveness of \ours in enhancing robustness.

Our results in Figure~\ref{fig:mixed_attacks} indicate that the combination of two different forms of attacks can  target robust agents in most scenarios, providing strong evidence of their robustness. \ours outperforms other methods in all five environments against non-temporally-coupled mixed adversaries, with a margin of over 20\% in the Humanoid environment. Moreover, when defending against temporally-coupled mixed attacks, \ours outperforms baselines by  30\% in multiple environments, with a minimum improvement of 10\%.
% These results clearly demonstrate the robustness of \ours against attackers that can target across domains.

\noindent\textbf{Natural Performance.}\quad
We also evaluate the natural performance of \ours and the baselines, as shown in Figure~\ref{fig:state_natural}, which compares natural rewards vs. rewards under the strongest temporally-coupled attacks. It is evident that while achieving robustness, \ours maintains a comparable natural performance with the baselines; the agent's performance does not degrade significantly in environments without adversaries. The natural performance comparing \ours with action-robust models can be found in Appendix~\ref{app:natural}.
% Figure environment removed

%\textbf{Summary.} We calculated the average normalized rewards for each evaluation metric and each robust agent in all the environments as in Figure~\ref{fig:exp}. This visualization vividly showcases that \ours demonstrates notably superior robustness under both standard and temporally-coupled attacks, in comparison to other approaches. Overall, these findings emphasize our empirical potential and contributions of \ours and provide intuitive insights into improving the robustness of agents through a novel and convincing evaluation framework for robust RL.

\begin{wrapfigure}{r}{0.35\textwidth}
% \vspace{-1em}
    \centering
    % Figure removed
    % \vspace{-0.5em}
    \caption{Ablated studies for $\bar{\epsilon}$.
    % $\pi_\theta$ is the acting policy being trained by $\loss_{\pi_\theta}$ defined in Equation~\eqref{loss:policy}, including the original loss of the base DRL algorithm $\lossrl$, a regularization term $\lossreg$, as well as $\lossworst$, a term for improving the \worstqname based on $\worstcritic$. Here $\worstcritic$ which estimates the \worstqname of $\pi_\theta$ is updated by $\loss_{\worstcritic}=\lossest$ depending on $\pi_\theta$.  
    }
    \label{fig:eps_}
% \vspace{-1em}
\end{wrapfigure}
\noindent\textbf{Ablation studies for temporally-coupled constraint $\bar{\epsilon}$.} \quad
As defined in our framework, the temporally-coupled constraint $\bar{\epsilon}$ limits the perturbations within a range that varies between timesteps. When $\bar{\epsilon}$ is set too large, the constraint becomes ineffective, resembling a standard attacker. 
Conversely, setting $\bar{\epsilon}$ close to zero overly restricts perturbations, leading to a decline in attack performance. An appropriate value for $\bar{\epsilon}$ is critical for effective temporally-coupled attacks. Figure~\ref{fig:eps_} illustrates the performance of robust models against temporally-coupled state attackers trained with different maximum $\bar{\epsilon}$. For WocaR-PPO, the temporally-coupled attacker achieves good performance when the values of $\bar{\epsilon}$ are set to 0.02. As the $\bar{\epsilon}$ values increase and the temporally-coupled constraint weakens, the agent's performance improves, indicating a decrease in the adversary's attack effectiveness. In the case of \ours agents, they consistently maintain robust performance as the $\bar{\epsilon}$ values become larger. This observation highlights the impact of temporal coupling on the vulnerability of robust baselines to such attacks. In contrast, \ours agents consistently demonstrate robustness against these attacks. \looseness=-1






% \vspace{-0.5em}
\section{Related Work}
% \vspace{-0.5em}
\label{sec:relate}
\textbf{Robust RL against adversarial perturbations.}
Existing defense approaches for RL agents are primarily designed to counter adversarial perturbations in state observations. These methods encompass a wide range of strategies, including regularization techniques~\citep{zhang2020robust, shen2020deep, oikarinen2020robust}, attack-driven approaches involving weak or strong gradient-based attacks~\citep{kos2017delving, behzadan2017whatever, mandlekar2017adversarially, pattanaik2017robust, franzmeyer2022illusionary, vinitsky2020robust}, RL-based alternating training methods~\citep{zhang2021robust, sun2021strongest}, and worst-case motivated methods~\citep{liang2022efficient}. Furthermore, there is a line of research that delves into providing \textit{theoretical guarantees} for adversarial defenses in RL~\citep{lutjens2020certified, oikarinen2020robust, fischer2019online, kumar2021policy, wu2021crop, sun2023certifiably}, exploring a variety of settings and scenarios where these defenses can be effectively applied.Aversarial attacks can take various forms. For instance, perturbations can affect the actions executed by the agent~\citep{xiao2019characterizing,  tessler2019action, lanier2022feasible, lee2020spatiotemporally}. Additionally, the study of adversarial multi-agent games has also received attention~\citep{gleave2019adversarial, pinto2017robust}. 

\textbf{Robust Markov decision process and safe RL.} 
There are several lines of work that study RL under safety/risk constraints~\citep{Heger1994ConsiderationOR,gaskett2003reinforcement,garcia2015comprehensive,bechtle2020curious,thomas2021safe} or under intrinsic uncertainty of environment dynamics~\citep{lim2013reinforcement,Mankowitz2020Robust}.
In particular, several works discuss coupled or non-rectangular uncertainty sets, which allow less conservative and more efficient robust policy learning by incorporating realistic conditions that naturally arise in practice. \citet{mannor2012lightning} propose to model coupled uncertain parameters based on the intuition that the total number of states with deviated parameters will be small. \citet{mannor2016robust} identify ``k-rectangular'' uncertainty sets defined by the cardinality of possible conditional projections of uncertainty sets, which can lead to more tractable solutions. Another recent work~\citep{goyal2023robust} proposes to model the environment uncertainty with factor matrix uncertainty sets, which can efficiently compute a robust policies. 

\textbf{Two-player zero-sum games.}
There are a number of related deep reinforcement learning methods for two-player zero-sum games. CFR-based techniques such as Deep CFR~\citep{deep_cfr}, DREAM~\citep{steinberger2020dream}, and ESCHER~\citep{mcaleer2022escher}, use deep reinforcement learning to approximate CFR. Policy-gradient techniques such as NeuRD~\citep{hennes2020neural}, Friction-FoReL~\citep{perolat2021poincare, perolat2022mastering}, and MMD~\citep{sokota2022unified}, approximate Nash equilibrium via modified actor-critic algorithms. Our robust RL approach takes the double oracle techniques such as PSRO~\citep{psro} as the backbone. PSRO-based algorithms have been shown to outperform the previously-mentioned algorithms in certain games~\citep{mcaleer2021xdo}. 

A more detailed discussion of related works in robust RL and game-theoretic RL are in Appendix~\ref{app:related}.



%\vspace{-0.5em}
\section{Conclusion and Discussion}
% \vspace{-0.5em}
% In this paper, we have introduced an attack model which follows a proposed temporally-coupled assumption. 
Motivated by the perturbations that arise in real world scenarios, we introduce a new attack model for studying deep RL models.
Since existing robust RL methods usually focus on a traditional threat model that perturbs state observations or actions arbitrarily within an $L_p$ norm ball, they become too conservative and can fail to perform a good defense under the temporally-coupled attacks. 
In contrast, we propose a game-theoretical response approach \ours, which finds the best response against attacks with various constraints including temporally-coupled ones. 
% \ours is based on the PSRO paradigm, which is shown to be effective and theoretically grounded in finding Nash equilibrium in two-player zero-sum games. 
Experiments across a range of continuous control tasks underscore the good performance of our approach over previous robust RL methods for both non-temporally-coupled attacks and temporally-coupled attacks across diverse attack domains.
% , highlighting the generalized robustness of \ours.

\textbf{Limitations.}\quad
The current PSRO-based approach may require several iterations to converge to the best response, which can pose limitations when computational resources are constrained. We leverage distributed RL tools to expedite the training of RL agents within \ours, enabling efficient learning of the best response. Detailed computational cost analysis can be found in Appendix~\ref{app:efficient}.

Regarding scalability concerns, we have demonstrated the \ours in addressing robust RL problems on high-dimensional tasks. In principle, alternative game-theoretic algorithms~\citep{perolat2022mastering}, known for their practical efficiency, can be considered for defense in different game scenarios. As part of our future research directions, we plan to explore methods to further enhance the scalability of \ours. This exploration may involve harnessing parallel training techniques and drawing insights from other scalable PSRO approaches~\citep{mcaleer2020pipeline, psro}. Additionally, we aim to extend the applicability of our method to pixel-based RL scenarios and real-world situations with increased practicality and complexity.





\newpage
\section*{Acknowledgements}%

The authors express their gratitude and a fond thought to Hassan \ak, who
with Gabriella Pasi set out to define a fuzzy version of OSF logic. This
paper originates from their work on the definition of similarity-based
unification for OSF terms, extending the approach of
\cite{AitKaciPasi2020}.

\bibliography{references}
\bibliographystyle{iclr2024_conference}
\newpage
\appendix
\section{Proof of Proposition~\ref{nash}}
\label{app:proof}

\begin{proof}
The \emph{exploitability} \( e(\pi) \) of a strategy profile \( \pi \) is defined as 
\[ e(\pi) = \sum_{i \in \mathcal{N}} \max_{\pi'_i}v_i(\pi'_i, \pi_{-i}). \]
A \emph{best response (BR)} strategy \( \mathbb{BR}_i(\pi_{-i}) \) for player \( i \) to a strategy \( \pi_{-i} \) is a strategy that maximally exploits \( \pi_{-i} \): 
\[ \mathbb{BR}_i(\pi_{-i}) = \arg\max_{\pi_i}v_i(\pi_i, \pi_{-i}). \]
% An \emph{\boldmath$\epsilon$\unboldmath-best response (\boldmath$\epsilon$\unboldmath-BR)} strategy \( \mathbb{BR}^\epsilon_i(\pi_{-i}) \) for player \( i \) to a strategy \( \pi_{-i} \) is a strategy that is at most \( \epsilon \) worse for player \( i \) than the best response:
% \[ v_i(\mathbb{BR}^\epsilon_i(\pi_{-i}), \pi_{-i}) \ge v_i(\mathbb{BR}_i(\pi_{-i}), \pi_{-i}) - \epsilon. \]
An \emph{\boldmath$\epsilon$\unboldmath-Nash equilibrium (\boldmath$\epsilon$\unboldmath-NE)} is a strategy profile \( \pi \) in which, for each player \( i \), \( \pi_i \) is an \( \epsilon \)-BR to \( \pi_{-i} \).

We can define the \emph{approximate exploitability} of the pair of meta-Nash equilibrium strategies for the agent and the adversary \( (\sigma_a, \sigma_v) \) as the sum of the expected reward their opponent approximate best responses achieve against them:
\[ \hat{e}(\sigma_a, \sigma_v) = v_a(\pi_a', \sigma_v) + v_v(\pi_v', \sigma_u), \]
where \( v_i(\pi_i', \sigma_{-i}) \) denotes the expected value of a player's approximate best response vs. the opponent meta-NE. 

Now assume that each approximate best response is within \( \frac{\epsilon}{4} \) of the optimal best response. Then 
\[ v_i(\pi_i', \sigma_{-i}) \ge v_i(\mathbb{BR}_i(\sigma_{-i}), \sigma_{-i}) - \frac{\epsilon}{4}. \]
As a result, upon convergence, when the approximate exploitability $\hat{e}(\sigma_a, \sigma_v)$ is less than $\frac{\epsilon}{2}$, 
then the exploitability of the pair of meta-Nash equilibrium strategies for the agent and the adversary \( (\sigma_a, \sigma_v) \) is less than $\epsilon$, and the pair of strategies are in an $\epsilon$-approximate Nash equilibrium. 

Every epoch where \ours does not converge to an approximate equilibrium, it must add a unique deterministic policy to the population for either the agent or the adversary because if both players added policies already included in their populations, those policies would not be approximate best responses. Given that the MDP has a finite horizon and operates in a discrete action space, there exists only a finite set of deterministic policies that can be added to the populations \( \Pi_a \) and \( \Pi_v \). Since the meta-Nash equilibrium over all possible deterministic policies is equivalent to the Nash equilibrium of the original game, in the worst case where all possible deterministic policies are added, the algorithm will terminate at an approximate Nash equilibrium. \qed
\end{proof}
\section{Additional Related Work}
\label{sec:add_related}

\subsection{Game-Theoretic Reinforcement Learning}

Superhuman performance in two-player games usually involves two components: the first focuses on finding a model-free blueprint strategy, which is the setting we focus on in this paper. The second component improves this blueprint online via model-based subgame solving and search~\citep{burch2014solving, moravcik2016refining, brown2018depth, brown2020combining, brown2017safe, schmid2021player}. This combination of blueprint strategies with subgame solving has led to state-of the art performance in Go~\citep{silver2017mastering}, Poker~\citep{brown2017libratus, brown2018superhuman, moravvcik2017deepstack}, Diplomacy~\citep{gray2020human}, and The Resistance: Avalon~\citep{serrino2019finding}. Methods that only use a blueprint have achieved state-of-the-art performance on Starcraft~\citep{alphastar}, Gran Turismo~\citep{wurman2022outracing}, DouDizhu~\citep{zha2021douzero}, Mahjohng~\citep{li2020suphx}, and Stratego~\citep{mcaleer2020pipeline, perolat2022mastering}. In the rest of this section we focus on other model-free methods for finding blueprints.      

Deep CFR~\citep{deep_cfr, steinberger2019single} is a general method that trains a neural network on a buffer of counterfactual values. However, Deep CFR uses external sampling, which may be impractical for games with a large branching factor, such as Stratego and Barrage Stratego. DREAM~\citep{steinberger2020dream} and ARMAC~\citep{gruslys2020advantage} are model-free regret-based deep learning approaches. ReCFR~\citep{liu2022model} propose a bootstrap method for estimating cumulative regrets with neural networks. ESCHER~\citep{mcaleer2022escher} remove the importance sampling term of Deep CFR and show that doing so allows scaling to large games.  

Neural Fictitious Self-Play (NFSP)~\citep{nfsp} approximates fictitious play by progressively training a best response against an average of all past opponent policies using reinforcement learning. The average policy converges to an approximate Nash equilibrium in two-player zero-sum games.   
%but has slower convergence bounds than CFR.  

There is an emerging literature connecting reinforcement learning to game theory. QPG~\citep{srinivasan2018actor} shows that state-conditioned $Q$-values are related to counterfactual values by a reach weighted term summed over all histories in an infostate and proposes an actor-critic algorithm that empirically converges to a NE when the learning rate is annealed. NeuRD~\citep{hennes2020neural}, and F-FoReL~\citep{perolat2021poincare} approximate replicator dynamics and follow the regularized leader, respectively, with policy gradients. Actor Critic Hedge (ACH)~\citep{ach} is similar to NeuRD but uses an information set based value function. All of these policy-gradient methods do not have theory proving that they converge with high probability in extensive form games when sampling trajectories from the policy. In practice, they often perform worse than NFSP and DREAM on small games but remain promising approaches for scaling to large games \citep{perolat2022mastering}. 
\section{Experiment Details and Additional Results}
\label{app:exp}

\subsection{Implementation details}
\label{app:exp:imp}
We provide detailed implementation information for our proposed method (\ours) and baselines.

\textbf{Training Steps}\quad
For \ours, we specify the number of training steps required for different environments. In the Hopper, Walker2d, and Halfcheetah environments, we train for 10 million steps. In the Ant and Humanoid environments, we extend the training duration to 20 million steps. For the ATLA baselines, we train for 2 million steps and 10 million steps in environments of varying difficulty.

\textbf{Network Structure}\quad
Our algorithm (\ours) adopts the same PPO network structure as the ATLA baselines to maintain consistency. The network comprises a single-layer LSTM with 64 hidden neurons. Additionally, an input embedding layer is employed to project the state dimension to 64, and an output layer is used to project 64 to the output dimension. Both the agents and the adversaries use the same policy and value networks to facilitate training and evaluation. Furthermore, the network architecture for the best response and meta Nash remains consistent with the aforementioned configuration.

\textbf{Schedule of $\epsilon$ and $\bar{\epsilon}$}\quad
During the training process, we gradually increase the values of $\epsilon$ and $\bar{\epsilon}$ from 0 to their respective target maximum values. This incremental adjustment occurs over the first half of the training steps. We reference the attack budget $\epsilon$ used in other baselines for the corresponding environments. This ensures consistency and allows for a fair comparison with existing methods. The target value of $\bar{\epsilon}$ is determined based on the adversary's training results, which is set as $\epsilon/5$. In some smaller dimensional environments, $\bar{\epsilon}$ can be set to $\epsilon/10$. We have observed that the final performance of the trained robust models does not differ by more than 5\% when using these values for $\bar{\epsilon}$.

\textbf{Observation and Reward Normalization}\quad
To ensure consistency with PPO implementation and maintain comparability across different codebases, we apply observation and reward normalization. Normalization helps to standardize the input observations and rewards, enhancing the stability and convergence of the training process. We have verified the performance of vanilla PPO on different implementations, and the results align closely with our implementation of \ours based on Ray rllib.

\textbf{Hyperparameter Selection}\quad
Hyperparameters such as learning rate, entropy bonus coefficient, and other PPO-specific parameters are crucial for achieving optimal performance. Referring to the results obtained from vanilla PPO and the ATLA baselines as references, a small-scale grid search is conducted to fine-tune the hyperparameters specific to \ours. Because of the significant training time and cost associated with \ours, we initially perform a simplified parameter selection using the Inverted Pendulum as a test environment.

\subsection{Adversaries in experiments}
\textbf{State Adversaries}\quad
Aimed to introduce the attack methods utilized during training and testing in our experiments. When it comes to state adversaries, PA-AD as Alogrithm~\ref{alg:pa-ad} stands out as the strongest attack compared to other state attacks. Therefore, we report the best state attack rewards under PA-AD attacks.

\textbf{Action Adversaries}\quad
In terms of action adversaries, an RL-based action adversary as Alogrithm~\ref{alg:ac-ad} can inflict more severe damage on agents' rewards compared to OU noise and parameter noise in~\citep{tessler2019action}.

\begin{algorithm}[tb]
  \caption{Action Adversary (AC-AD)}
  \label{alg:ac-ad}
  \begin{algorithmic}
  \STATE \textbf{Input:} Initialization of action adversary policy $v$; victim policy $\pi$, initial state $s_0$
  \FOR {$t = 0, 1, 2, \ldots$}
  \STATE adversary $v$ samples an action perturbations $\widehat{a}_{t}\sim\nu(\cdot|s_{t})$, 
  \STATE victim policy $\pi$ outputs action $a_{t}\sim\pi(\cdot|s_{t})$
  \STATE the environment receives $\tilde{a}_{t} = a_{t} + \widehat{a}_{t} $, returns $s_{t+1}$ and $r_t$
  \STATE adversary saves $(s_t, \widehat{a}_t,-r_t,s_{t+1})$ to the adversary buffer
  \STATE adversary updates its policy $v$
  \ENDFOR
  \end{algorithmic}
\end{algorithm}
\vspace{-0.5em}

\textbf{Mixed Adversaries}\quad
When dealing with mixed adversaries capable of perturbing both state and action spaces, it becomes crucial to design the action space for the adversary. In Algorithm~\ref{alg:mixed-ad}, we extend the idea of PA-AD~\citep{sun2021strongest}, which learns a policy perturbation direction to generate perturbations. In our case, the mixed adversary director only needs to learn the policy perturbation direction $\hat{d}_t$. For various attack domains, the actor functions then translate the direction $\hat{d}_t$ into state or action perturbations. This design approach ensures that our mixed adversary doesn't increase the complexity of adversary training, as it deploys mixed perturbations using different actor functions as required by distinct attack domains.
\label{app:exp:alg}
\begin{algorithm}[!hb]
  \caption{Policy Adversarial Actor Director (PA-AD)}
  \label{alg:pa-ad}
  \begin{algorithmic}
  \STATE \textbf{Input:} Initialization of adversary director’s policy $v$; victim policy $\pi$, the actor function $g$ for the state space $\mathcal{S}$, initial state $s_0$
  \FOR {$t = 0, 1, 2, \ldots$}
  \STATE \textit{Director} $v$ samples a policy perturbing direction and perturbed choice, $\widehat{a}_{t}\sim\nu(\cdot|s_{t})$
  \STATE \textit{Actor} perturbs $s_t$ to $\tilde{s}_{t}=g(\widehat{a}_{t},s_{t})$
  \STATE Victim takes action $a_{t}\sim\pi(\cdot|\tilde{s}_{t})$, proceeds to $s_{t+1}$, receives $r_t$
  \STATE \textit{Director} saves $(s_t, \widehat{a}_t,-r_t,s_{t+1})$ to the adversary buffer
  \STATE \textit{Director} updates its policy $v$ using any RL algorithms
  \ENDFOR
  \end{algorithmic}
\end{algorithm}
\vspace{-0.5em}
\input{algorithms/mixed-AD}

\vspace{1em}
\textbf{Transition Adversaries. }
In addition to addressing adversarial perturbations, we extend the evaluation of \ours to consider transition uncertainty, mitigating the mismatch problem between the training simulator and the testing environment. Robustness under transition uncertainty is crucial for real-world applicability. To assess this aspect, experiments are conducted on perturbed MuJoCo environments (Hopper, Walker2d, and HalfCheetah) by modifying their physical parameters ('leg\_joint\_stiffness' value: 30, 'foot\_joint\_stiffness' value: 30, and bound on 'back\_actuator\_range': 0.5) following the protocol established by \cite{zhou2023natural}. Comparative evaluations are performed against robust natural actor-critic (RNAC)\citep{zhou2023natural} trained with Double-Sampling (DS) and Inaccurate Parameter Models (IPM) uncertainty. The results presented in Table~\ref{tab:transition} consistently demonstrate that \ours achieves competitive or superior performance compared to baseline methods in each perturbed environment, showcasing its effectiveness in robustly handling transition uncertainty.

\begin{table}[!t]
\vspace{-0.5em}
\centering
\renewcommand{\arraystretch}{1.3}
\resizebox{\textwidth}{!}{%
\setlength{\tabcolsep}{4pt}
  \centering
  \begin{tabular}{p{2.5cm}<{\centering} p{3.5cm}<{\centering} p{2.5cm}<{\centering} p{2.5cm}<{\centering} p{2.5cm}<{\centering}}
    \toprule
    Perturbed Environments & & RNAC-PPO (DS) & RNAC-PPO (IPM) & \textbf{GRAD} \\
    \midrule
    \multirow{2}{*}{Hopper} & Natural reward& \textbf{$3502 \pm 256$} & $3254 \pm 138$ & $3482 \pm 209$ \\
     & 'leg\_joint\_stiffness' & $2359 \pm 182$ & $2289 \pm 124$ & \textbf{$2692 \pm 236$} \\
    \midrule
    \multirow{2}{*}{Walker} & Natural reward & $4322 \pm 289$ & $4248 \pm 89$ & \textbf{$4359 \pm 141$} \\
    & 'foot\_joint\_stiffness' & $4078 \pm 297$ & $4129 \pm 78$ & \textbf{$4204 \pm 132$} \\
    \midrule
    \multirow{2}{*}{Halfcheetah} & Natural reward  & $5524 \pm 178$ & $5569 \pm 232$ & \textbf{$6047 \pm 241$} \\
    & 'back\_actuator\_range'& $768 \pm 102$ & $1143 \pm 45$ & \textbf{$1369 \pm 117$} \\
    \bottomrule
  \end{tabular}}
\caption{Comparison of cumulative reward in Perturbed Environments with changed physical parameters.}
\vspace{-0.5em}
\label{tab:transition}
\end{table}


\textbf{Short-term Memorized Temporall-coupled Attacks. }
While our temporally-coupled setting considering perturbation from the last time step aligns with the common practice of state adversaries, which typically perturb the current state without explicitly attacking short-term memory, we recognized the importance of exploring a more general scenario akin to a general partially observable MDP~\citep{efroni22provable}. We introduced a short-term memorized temporally-coupled attacker by calculating the mean of perturbations from the past 10 steps and applying the temporally-coupled constraint to this mean.
The results in Table~\ref{tab:memorized} from these additional experiments against short-term memorized temporally-coupled attacks underscore the efficacy of GRAD under this extended setting. GRAD consistently demonstrates heightened robustness compared to other robust baselines when confronted with a memorized temporally-coupled adversary. These findings provide valuable insights into the temporal scope of perturbations, contributing to a more comprehensive understanding of GRAD's capabilities in handling diverse adversarial scenarios.

\begin{table}[!t]
\vspace{-0.5em}
\centering
\renewcommand{\arraystretch}{1.4}
\resizebox{\textwidth}{!}{%
\setlength{\tabcolsep}{4pt}
\begin{tabular}{p{4cm}<{\centering} p{2.5cm}<{\centering} p{2.5cm}<{\centering} p{2.5cm}<{\centering} p{2.5cm}<{\centering} p{2.5cm}<{\centering}}
\toprule
Short-term Memorized Temporally-Coupled Attacks & Hopper & Walker2d & Halfcheetah & Ant & Humanoid \\
\midrule
PA-ATLA-PPO & 2334 $\pm$ 249 & 2137 $\pm$ 258 & 3669 $\pm$ 312 & 2689 $\pm$ 189 & 1573 $\pm$ 232 \\
WocaR-PPO & 2256 $\pm$ 332 & 2619 $\pm$ 198 & 4228 $\pm$ 283 & 3229 $\pm$ 178 & 2017 $\pm$ 213 \\
\textbf{GRAD} &  \textbf{2869 $\pm$ 228} & \textbf{3134 $\pm$ 251} & \textbf{4439 $\pm$ 287} & \textbf{3617 $\pm$ 188} & \textbf{2736 $\pm$ 269} \\
\bottomrule
\end{tabular}}
\caption{Performance Comparison under Memorized Temporally-Coupled Attacks}
\label{tab:memorized}
\vspace{-0.5em}

\end{table}







\subsection{Attack budgets}
\label{app:exp:eps}
In Figure~\ref{fig:eps}, we report the performance of baselines and \ours under different attack budget $\epsilon$. As the value of $\epsilon$ increases, the rewards of robust agents under different types of attacks decrease accordingly. However, our approach consistently demonstrates superior robustness as the attack budget changes.
\input{figures/fig_eps}

\subsection{Temporally-coupled constraints}
We also investigate the impact of temporally-coupled constraints $\bar{\epsilon}$ on attack performance, as we explained in our experiment section.
\input{figures/fig_bar_eps}

\subsection{Natural reward vs. Robustness}
We presents the natural performance comparison of \ours and action robust baselines in Figure~\ref{fig:action_natural}.
\label{app:natural}
% Figure environment removed

\subsection{Computational Cost}
\label{app:efficient}
The training time for \ours can vary depending on the specific environment and its associated difficulty. Typically, on a single V100 GPU, training \ours takes around 20 hours for environments like Hopper, Walker2d, and Halfcheetah. However, for more complex environments like Ant and Humanoid, the training duration extends to approximately 40 hours. It's worth noting that the training time required for defense against state adversaries or action adversaries is relatively similar.
\end{document}
