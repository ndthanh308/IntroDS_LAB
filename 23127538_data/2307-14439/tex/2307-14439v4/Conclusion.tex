In this paper we introduced FINN, a method for computing the analytical integral of a learned neural network. Our approach defines several constraints and extensions crucial to potential applications, including 1) an integral constraint, which restricts $f$ to the set of functions which integrate to a given value, 2) a positivity constraint, which restricts $f$ to be positive, and 3) an extension to aribtrary domains, which allows integration over regions which cannot be represented by an $n$-dimensional box. To demonstrate the use cases of FINN, we also present four applications: distance metrics, trajectory optimisation, probability distributions, and bellman optimality operators.