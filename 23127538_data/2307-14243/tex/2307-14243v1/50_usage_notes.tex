The Fluorescent Neuronal Cells collection is available both as a comprehensive archive and as individual image collections for specific research requirements. This enables users to download the data efficiently and selectively, based on their specific needs.
The code provided is based on the Python and PyTorch frameworks, offering a robust foundation for analysis and modeling. However, thanks to the popularity of the annotation formats and the use of PNG images, users can easily employ their preferred deep learning framework.


\subsection*{Peculiar traits}

In all image collections, the visual representation is characterized by the prevalence of two distinct color tones, which result from the deliberate selection of a specific wavelength. One tone appears darker, indicating areas where light has been filtered out, while the other tone is brighter and more intense, emitted by the fluorophore corresponding to the color of each collection (see \Cref{fig:green_image,fig:red_image,fig:yellow_image}).
As a result, the images can generally be depicted using variations of a single color. Consequently, a 1-D representation may be sufficient, or an alternative color space other than RGB could provide more informative and less redundant data.

Notice, however, that the specific colors employed in our studies were dictated not by any inherent or functional property of the stained biological structures, but rather by their accessibility and practicality during the time of the experiments. Therefore, it would be a misinterpretation to associate specific colors to particular neuronal substructures. In fact, these colors serve only as contrasting elements to discern the stained foreground objects from the background. 
Consequently, the emphasis should lie primarily on learning this discrimination rather than matching specific colors with the neuronal structures. 
Thus, the particular colors should not be considered indicative of the type of neuronal cells or their functional attributes, but merely as a practical tool aiding in the overall visualization and interpretation.


% written by Jin Chen 0705


\subsection{Industrial Challenges} Personalization services, particularly with recommender systems, are complex industrial products that face numerous challenges when implemented in real-world scenarios. We will now summarize the key challenges as follows:

    \textbf{Scaling computational resources} Existing large language models, such as BERT and GPT, demand significant computational power for training and inference. This includes high memory usage and time consumption. Fine-tuning these models to align them with personalization systems, which has shown promising results for improved personalization performance, can be computationally intensive. Several efficient finetuning strategies, e.g., option tuning in M6-Rec~\cite{cui2022m6}, Lora~\cite{hu2021lora}, QLora~\cite{dettmers2023qlora}, have been developed to address this issue and pave the way for more efficient tuning.

    \textbf{Significant Response time} Achieving efficient response times is crucial for online serving and greatly impacts the personalized user experience. Response time includes both the inference phase of large language models and the concurrent user requests in large numbers. The introduction of large language models can result in considerable inference time, posing a challenge for real-world deployment. One approach is to pre-compute the embeddings of intermediate outputs from language models, storing and indexing them in a vector database, particularly for methods that utilize large language models as textual encoders. Other approaches, such as distillation and quantization, aim to strike a balance between performance and latency.


\subsection{Laborious Data Collection} Large language models are widely known to leverage extensive amounts of open-domain knowledge during their training and fine-tuning processes. These knowledge sources include well-known references such as Wikipedia, books, and various websites~\cite{brown2020language}. Similarly, when applied in recommender systems, these models often rely on representative open-domain datasets such as MovieLens and Amazon Books. While this type of open-domain knowledge contains a wealth of common-sense information, personalized tasks require access to more domain-specific data that is not easily shareable. Additionally, the nature of user feedback in personalized tasks can be complex and sparse, often accompanied by noisy feedback. Collecting and filtering this data, in contrast to acquiring common-sense knowledge, presents challenges. It incurs higher labor costs and introduces additional training redundancy due to the need for extensive data processing and filtering. Furthermore, designing appropriate prompts to instruct or fine-tune large language models is crucial for aligning them with the distribution of in-domain inputs in personalization tasks. By carefully tailoring the prompts, researchers and practitioners can guide the model to produce outputs that better cater to personalized applications, thereby maximizing performance and effectiveness.




\subsection{Long Text Modeling}
Large language models have a limitation on the maximum number of input tokens they can handle, typically constrained by the context window size, e.g., 4096 for ChatGPT. This poses challenges when dealing with long user behavior sequences, which are common in modern recommender systems. Careful design is necessary to generate effective and appropriate prompt inputs within this limited length. In the case of conversations with multiple rounds, accumulating several rounds of dialogue can easily exceed the token limit of models. The current approach in handling long conversations is to truncate the history, keeping only the most recent tokens. However, this truncation discards valuable historical information, potentially harming the model performance. To address these challenges, several techniques can be employed. One approach is to prioritize and select the most relevant parts of the user behavior sequence or conversation history to include in the prompt. This selection can be based on various criteria such as recency, importance, or relevance to the task at hand. Another technique involves summarizing or compressing the lengthy input while preserving essential information. This can be achieved through techniques like extractive summarization or representing the long sequence in a condensed form. Moreover, architectural modifications, such as hierarchical or memory-augmented models, can be explored to better handle long sequences by incorporating mechanisms to store and retrieve relevant information efficiently. 

In addition, collaborative modeling of long text data and recommendation tasks is an emerging and pressing challenge. In conventional personalization systems, item ID information along with other categorical information is commonly used for modeling feature interactions and user preferences. With the rise of large language models, there would be a growing trend toward leveraging textual information more extensively. Textual data provides unique insights about items or users, making it valuable for modeling purposes. From the perspective of modeling, dealing with long text data requires more attention and complexity compared to categorical data, not to mention the need to match the modeling of user interests. From the perspective of implementation, reforming the entire pipeline becomes necessary to accommodate the requirements of efficient latency. Efficiently processing and incorporating long text data into recommendation models and serving them in real-time present technical challenges. 







\subsection{Interpretability and Explainability}
While large language models provide good reasoning capabilities, they are notorious for the nature of the 'black box', which is highly complex and non-linear in their enormous size and layered architecture, making it challenging to comprehend the internal workings and understand the generation process of recommendations. Without a deep understanding of how the model operates, it becomes challenging to detect and address biases or ensure fair and ethical recommendations. Once transparency about the internal mechanisms is lacking, users struggle to trust and accept the decisions made by the system. Users often desire understandable explanations for recommended choices. Addressing the challenge of model interpretability and explainability requires research involving natural language processing, explainable AI, human-computer interaction, and recommendation systems. The development of techniques that unveil the inner workings of language models, facilitate the generation of meaningful and accurate interpretations, and enable robust evaluation methods is the main focus. By providing transparent and interpretable recommendations, users can establish trust, understand the reasoning behind the recommendations, and make informed decisions.




\subsection{Evaluation}
Conventional personalization systems typically rely on task-specific metrics such as ranking-oriented metrics, NDCG, AUC, and Recall to evaluate model performance. However, with the integration of large language models into recommender systems, the evaluation tools and metrics undergo significant changes. Traditional metrics may not sufficiently capture the performance of recommender systems powered by large language models, which introduce novel capabilities and generate recommendations in a different manner and require the development of new evaluation tools. 

One crucial aspect of evaluation is considering user preferences in large language model-powered systems, which requires a user-centric approach. Metrics such as user satisfaction, engagement, and overall experience become essential considerations. For example, Liu's work~\cite{liu2023chatgpt} proposes a crowdsourcing task to assess the quality of generated explanations and review summaries, providing a way to evaluate the effectiveness of the generated content. Additionally, user satisfaction surveys and feedback questionnaires can serve as valuable options.

Another perspective to consider is the health of the system, which involves evaluating novelty and assessing factors like diversity, novelty, serendipity, and user retention rates. These metrics help evaluate the freshness of recommendations and the long-term effects of large language models.

Furthermore, it is crucial to assess the interpretability and fairness of recommendations. The interpretability assessment focuses on measuring the clarity, understandability, and transparency of recommendations. Simultaneously, the fairness evaluation aims to address potential biases in personalized results. By prioritizing fairness, we strive to create personalized experiences that are equitable and inclusive for all users. Both of these evaluations are essential to enhance the overall user experience and build confidence in the personalized recommendations delivered by the system.




\subsection{Trade-off between Helpfulness, Honesty, Harmlessness}
When large language models are employed for personalization, some of their disadvantages would be magnified. Striving for a more honest and harmless system may come at the expense of system performance.

First of all, the accuracy and factuality of the system must be ensured. Although large language models can generate seemingly reasonable content, there is a risk of disseminating misleading or inaccurate information. This becomes even more critical when incorporating user feedback, as the model may mimic user behaviors in an attempt to appear honest. However, this imitation can result in biased guidance for users, offering no real benefits.

Secondly, in terms of harmlessness, concerns regarding privacy, discrimination, and ethics arise. While large language models have the potential to provide highly personalized recommendations by leveraging user data, privacy, and data security become paramount. Unlike open-domain datasets, the privacy of individual data used for training should be rigorously protected, with strict user permissions for sharing their personal information. For discrimination, large language models may inevitably reflect biases inherent in the training data, leading to discriminatory recommendations. Considering the biased user and item distribution, which is much more significant in recommender systems with the long-tail effect, where biased user and item distribution can lead to decisions that favor majority choices, resulting in discrimination against certain users. The final concern revolves around ethical considerations. Harmful messages, if clicked by users unconsciously, can guide large language models toward generating similar harmful content. However, when assisting in personalized decision-making, it is essential for large language models to have the capability to minimize exposure to harmful messages and guide users in a responsible manner. Approaches like constructing a Constitutional AI~\cite{bai2022constitutional}, where critiques, revisions, and supervised Learning are adopted for better training large language models, may offer valuable insights.

By addressing these concerns, safeguarding privacy, mitigating discrimination, and adhering to ethical guidelines, recommender systems can leverage the power of large language models while ensuring user trust, fairness, and responsible recommendations.


\section{Conclusion}
In conclusion, the emergence of large language models represents a significant breakthrough in the field of artificial intelligence. Their enhanced abilities in understanding, language analysis, and common-sense reasoning have opened up new possibilities for personalization. In this paper, we provide several perspectives on when large language models adapt to personalization systems. We have observed a progression from utilizing low-level capabilities of large language models to enhance performance, to leveraging their potential in complex interactions with external tools for end-to-end tasks. This evolution promises to revolutionize the way personalized services are delivered. We also acknowledge the open challenges that come with the integration of large language models into personalization systems.  



\subsection*{Challenges}


Some important insights for future studies can be drawn examining ground-truth masks at the pixel level, revealing significant characteristics that impact the training process.

The two classes, namely cells (1) and background (0), exhibit an extreme \textbf{class imbalance}, with background pixels being overwhelmingly predominant, typically exceeding cell pixels by over a factor of 100 (cf. \Cref{tab:summary-stats}, \textit{signal \%}).
These observations highlight the necessity for specialized training strategies to address this pronounced class imbalance and effectively learn the pixel classification.

Additional challenges are associated with the macroscopic content of the images. The Fluorescent Neuronal Cells data showcase a diverse collection of 11704 subnuclear neuronal structures, varying in shape, size, and extension (cf \Cref{tab:summary-stats}, \textit{area, Feret diameter} and \textit{equivalent diameter} columns). 
The distribution of these structures across the collections is uneven, with some images containing numerous cells while others are devoid of them. Consequently, the model needs to be flexible enough to handle both scenarios.

Furthermore, despite considerable efforts to stabilize the acquisition procedure, several technical challenges persist.
Firstly, there is a \textbf{high variability in terms of color, saturation, and contrast} from one image to another. For instance, there are instances where the tissues absorb some of the markers (see \Cref{fig:challenges:yellow_artifact,fig:challenges:yellow_stripe,fig:challenges:green,fig:challenges:green_artifact,fig:challenges:red_stripe}), causing irrelevant compounds to emit light which is then captured by the microscope. 
Consequently, the background's hue may shift towards values similar to those of faint neuronal cells (see \Cref{fig:challenges:yellow_artifact,fig:challenges:yellow_stripe,fig:challenges:green,fig:challenges:green_artifact,fig:challenges:red_filament}).
In such circumstances, relying solely on pixel intensity is insufficient to distinguish between signal and background, necessitating the consideration of additional characteristics such as saturation and contrast. However, even the analysis of these characteristics is not straightforward, as fluorescent emissions are naturally unstable, leading to fluctuations in the saturation levels exhibited by cell pixels (cf. \Cref{fig:challenges:yellow,fig:challenges:yellow_artifact,fig:challenges:yellow_stripe} or \Cref{fig:challenges:red_filament,fig:challenges:red_stripe}).


Moreover, the substructures of interest have a fluid nature. Also, the shot can capture different two-dimensional sections depending on how the cells are oriented within the tissues.
As a consequence, the \textbf{size and the shape of the stained cells can vary significantly} (cf. objects dimension in \Cref{fig:green_mask,fig:red_mask,fig:yellow_mask}), further complicating the discrimination between cells and the background.

Another challenge arises from the occasional presence of accumulations of fluorophore in narrow areas, resulting in emissions that closely resemble those of cells. 
These \textbf{artifacts} can manifest as small areas, such as point artifacts and filaments, or larger structures, like lateral stripes  (see \Cref{fig:challenges:yellow_artifact,fig:challenges:yellow_stripe,fig:challenges:green,fig:challenges:green_artifact,fig:challenges:red_filament,fig:challenges:red_stripe}).
Again, their presence hampers the detection task, making the recognition and the understanding of cells structure and size mandatory for the model.

A further source of complexity is represented by \textbf{overcrowding} (\Cref{fig:challenges:yellow,fig:challenges:green,fig:challenges:red_filament,fig:challenges:red_stripe}). When several cells are close-by, maybe partially overlapping, precisely localizing cell boundaries can be challenging, thus requiring adjustments to prevent the model from merging nearby cells into single agglomerations.

Last but not least, in some occasions the recognition of cells may be ambiguous even for human operators(cf. \textit{marked} and \textit{non-marked} instances in \Cref{fig:challenges:yellow,fig:challenges:green,fig:challenges:red_filament,fig:challenges:red_stripe}). Of course, this poses an issue of intrinsic \textbf{subjectivity} in the annotation process,
% which is then reflected on model performance.
which in turn affects both the training and assessment phases.

By and large, all of these factors make the recognition and counting tasks harder and complicate the learning process.
Likewise, borderline annotations hinder model evaluation as their subjectivity deprives the model of a reliable and indisputable testbed.

% % written by Jin Chen 0705


\subsection{Industrial Challenges} Personalization services, particularly with recommender systems, are complex industrial products that face numerous challenges when implemented in real-world scenarios. We will now summarize the key challenges as follows:

    \textbf{Scaling computational resources} Existing large language models, such as BERT and GPT, demand significant computational power for training and inference. This includes high memory usage and time consumption. Fine-tuning these models to align them with personalization systems, which has shown promising results for improved personalization performance, can be computationally intensive. Several efficient finetuning strategies, e.g., option tuning in M6-Rec~\cite{cui2022m6}, Lora~\cite{hu2021lora}, QLora~\cite{dettmers2023qlora}, have been developed to address this issue and pave the way for more efficient tuning.

    \textbf{Significant Response time} Achieving efficient response times is crucial for online serving and greatly impacts the personalized user experience. Response time includes both the inference phase of large language models and the concurrent user requests in large numbers. The introduction of large language models can result in considerable inference time, posing a challenge for real-world deployment. One approach is to pre-compute the embeddings of intermediate outputs from language models, storing and indexing them in a vector database, particularly for methods that utilize large language models as textual encoders. Other approaches, such as distillation and quantization, aim to strike a balance between performance and latency.


\subsection{Laborious Data Collection} Large language models are widely known to leverage extensive amounts of open-domain knowledge during their training and fine-tuning processes. These knowledge sources include well-known references such as Wikipedia, books, and various websites~\cite{brown2020language}. Similarly, when applied in recommender systems, these models often rely on representative open-domain datasets such as MovieLens and Amazon Books. While this type of open-domain knowledge contains a wealth of common-sense information, personalized tasks require access to more domain-specific data that is not easily shareable. Additionally, the nature of user feedback in personalized tasks can be complex and sparse, often accompanied by noisy feedback. Collecting and filtering this data, in contrast to acquiring common-sense knowledge, presents challenges. It incurs higher labor costs and introduces additional training redundancy due to the need for extensive data processing and filtering. Furthermore, designing appropriate prompts to instruct or fine-tune large language models is crucial for aligning them with the distribution of in-domain inputs in personalization tasks. By carefully tailoring the prompts, researchers and practitioners can guide the model to produce outputs that better cater to personalized applications, thereby maximizing performance and effectiveness.




\subsection{Long Text Modeling}
Large language models have a limitation on the maximum number of input tokens they can handle, typically constrained by the context window size, e.g., 4096 for ChatGPT. This poses challenges when dealing with long user behavior sequences, which are common in modern recommender systems. Careful design is necessary to generate effective and appropriate prompt inputs within this limited length. In the case of conversations with multiple rounds, accumulating several rounds of dialogue can easily exceed the token limit of models. The current approach in handling long conversations is to truncate the history, keeping only the most recent tokens. However, this truncation discards valuable historical information, potentially harming the model performance. To address these challenges, several techniques can be employed. One approach is to prioritize and select the most relevant parts of the user behavior sequence or conversation history to include in the prompt. This selection can be based on various criteria such as recency, importance, or relevance to the task at hand. Another technique involves summarizing or compressing the lengthy input while preserving essential information. This can be achieved through techniques like extractive summarization or representing the long sequence in a condensed form. Moreover, architectural modifications, such as hierarchical or memory-augmented models, can be explored to better handle long sequences by incorporating mechanisms to store and retrieve relevant information efficiently. 

In addition, collaborative modeling of long text data and recommendation tasks is an emerging and pressing challenge. In conventional personalization systems, item ID information along with other categorical information is commonly used for modeling feature interactions and user preferences. With the rise of large language models, there would be a growing trend toward leveraging textual information more extensively. Textual data provides unique insights about items or users, making it valuable for modeling purposes. From the perspective of modeling, dealing with long text data requires more attention and complexity compared to categorical data, not to mention the need to match the modeling of user interests. From the perspective of implementation, reforming the entire pipeline becomes necessary to accommodate the requirements of efficient latency. Efficiently processing and incorporating long text data into recommendation models and serving them in real-time present technical challenges. 







\subsection{Interpretability and Explainability}
While large language models provide good reasoning capabilities, they are notorious for the nature of the 'black box', which is highly complex and non-linear in their enormous size and layered architecture, making it challenging to comprehend the internal workings and understand the generation process of recommendations. Without a deep understanding of how the model operates, it becomes challenging to detect and address biases or ensure fair and ethical recommendations. Once transparency about the internal mechanisms is lacking, users struggle to trust and accept the decisions made by the system. Users often desire understandable explanations for recommended choices. Addressing the challenge of model interpretability and explainability requires research involving natural language processing, explainable AI, human-computer interaction, and recommendation systems. The development of techniques that unveil the inner workings of language models, facilitate the generation of meaningful and accurate interpretations, and enable robust evaluation methods is the main focus. By providing transparent and interpretable recommendations, users can establish trust, understand the reasoning behind the recommendations, and make informed decisions.




\subsection{Evaluation}
Conventional personalization systems typically rely on task-specific metrics such as ranking-oriented metrics, NDCG, AUC, and Recall to evaluate model performance. However, with the integration of large language models into recommender systems, the evaluation tools and metrics undergo significant changes. Traditional metrics may not sufficiently capture the performance of recommender systems powered by large language models, which introduce novel capabilities and generate recommendations in a different manner and require the development of new evaluation tools. 

One crucial aspect of evaluation is considering user preferences in large language model-powered systems, which requires a user-centric approach. Metrics such as user satisfaction, engagement, and overall experience become essential considerations. For example, Liu's work~\cite{liu2023chatgpt} proposes a crowdsourcing task to assess the quality of generated explanations and review summaries, providing a way to evaluate the effectiveness of the generated content. Additionally, user satisfaction surveys and feedback questionnaires can serve as valuable options.

Another perspective to consider is the health of the system, which involves evaluating novelty and assessing factors like diversity, novelty, serendipity, and user retention rates. These metrics help evaluate the freshness of recommendations and the long-term effects of large language models.

Furthermore, it is crucial to assess the interpretability and fairness of recommendations. The interpretability assessment focuses on measuring the clarity, understandability, and transparency of recommendations. Simultaneously, the fairness evaluation aims to address potential biases in personalized results. By prioritizing fairness, we strive to create personalized experiences that are equitable and inclusive for all users. Both of these evaluations are essential to enhance the overall user experience and build confidence in the personalized recommendations delivered by the system.




\subsection{Trade-off between Helpfulness, Honesty, Harmlessness}
When large language models are employed for personalization, some of their disadvantages would be magnified. Striving for a more honest and harmless system may come at the expense of system performance.

First of all, the accuracy and factuality of the system must be ensured. Although large language models can generate seemingly reasonable content, there is a risk of disseminating misleading or inaccurate information. This becomes even more critical when incorporating user feedback, as the model may mimic user behaviors in an attempt to appear honest. However, this imitation can result in biased guidance for users, offering no real benefits.

Secondly, in terms of harmlessness, concerns regarding privacy, discrimination, and ethics arise. While large language models have the potential to provide highly personalized recommendations by leveraging user data, privacy, and data security become paramount. Unlike open-domain datasets, the privacy of individual data used for training should be rigorously protected, with strict user permissions for sharing their personal information. For discrimination, large language models may inevitably reflect biases inherent in the training data, leading to discriminatory recommendations. Considering the biased user and item distribution, which is much more significant in recommender systems with the long-tail effect, where biased user and item distribution can lead to decisions that favor majority choices, resulting in discrimination against certain users. The final concern revolves around ethical considerations. Harmful messages, if clicked by users unconsciously, can guide large language models toward generating similar harmful content. However, when assisting in personalized decision-making, it is essential for large language models to have the capability to minimize exposure to harmful messages and guide users in a responsible manner. Approaches like constructing a Constitutional AI~\cite{bai2022constitutional}, where critiques, revisions, and supervised Learning are adopted for better training large language models, may offer valuable insights.

By addressing these concerns, safeguarding privacy, mitigating discrimination, and adhering to ethical guidelines, recommender systems can leverage the power of large language models while ensuring user trust, fairness, and responsible recommendations.


\section{Conclusion}
In conclusion, the emergence of large language models represents a significant breakthrough in the field of artificial intelligence. Their enhanced abilities in understanding, language analysis, and common-sense reasoning have opened up new possibilities for personalization. In this paper, we provide several perspectives on when large language models adapt to personalization systems. We have observed a progression from utilizing low-level capabilities of large language models to enhance performance, to leveraging their potential in complex interactions with external tools for end-to-end tasks. This evolution promises to revolutionize the way personalized services are delivered. We also acknowledge the open challenges that come with the integration of large language models into personalization systems.  





\subsection*{Research lines}

As long as potential applications, the FNC dataset offers rich opportunities for diverse research directions, including:

\begin{itemize}
\item \textit{Object Segmentation, Detection, and Counting}: The dataset's comprehensive annotations and diverse neuronal structures support studies focusing on accurate segmentation, detection, and counting of cells. Particularly, FNC may be a challenging benchmark for class imbalance, object overlapping/overcrowding, and uncertainty estimation 

\item \textit{Transfer Learning}: With the availability of multiple image collections within FNC, researchers can explore transfer learning techniques, leveraging knowledge from one collection to improve performance on another.

\item \textit{Unsupervised or Self-/Weakly-Supervised Learning}: The presence of both labeled and unlabeled data within the FNC dataset provides an ideal testbed for evaluating unsupervised or self-/weakly-supervised learning approaches.

\item \textit{Evaluation of Annotation Types}: Researchers can investigate the effectiveness of different annotation types for specific tasks, allowing for a comparative analysis and selection of the most suitable annotations considering the cost/performance requirements of a given use-case.

% \item \textit{Image Restoration or Generation}: given the high-resolution of FNC images, our data may presents an opportunity to explore image restoration or generation approaches specifically designed for microscopy applications.
\end{itemize}

\subsection*{Limitations}

Despite the Fluorescent Neuronal Cells collection presenting a variety of images in many aspects, it has limitations in terms of diversity across several parameters.

Firstly, all the images were collected by the same research laboratory in Bologna, utilizing fixed experimental conditions and acquisition settings. Furthermore, the images were captured using epifluorescence microscopy, which limits the range of techniques employed.
However, we believe that the adopted acquisition settings represent a more challenging scenario. Therefore, pre-training on FNC data should enable generalization to modern equipment such as confocal microscopy, which produces higher-quality images with sharper object boundaries and improved signal-to-noise ratio.

Another limitation lies in the lack of diversity in the cell types depicted and the animal species involved. Our dataset only focuses on subcellular components of rodent neurons. This might potentially impact the generalization of the models to different use cases and restrict their application to other cell types or animal species.
