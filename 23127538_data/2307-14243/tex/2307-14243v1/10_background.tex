Fluorescence microscopy is as a pivotal imaging technique in life-science experiments, allowing researchers to study biological structures or processes with remarkable precision. It employs fluorescent dyes or proteins that emit light at specific wavelengths depending on the illuminating wavelength they absorb. % when illuminated.
Exploiting this phenomenon, specific molecules can be tagged (\textit{stained}) with fluorescent markers, and visualized by filtering only their emitted light, thus providing valuable insights into their localization, activity, and interactions.

% Manual Recognition and Counting: A Bottleneck
Despite its widespread use, current practices in fluorescence microscopy analysis heavily rely on semi-automatic procedures, often necessitating manual recognition and/or counting of specific neuronal structures of interest \cite{luppi1, luppi3,chiocchetti2021phosphorylated}.
For instance, in the study of torpor mechanisms, researchers depend on laborious hand-crafted operations to identify neuronal networks associated with this process \cite{hitrec2019neural}. This manual aspect typically delays the analyses, also introducing potential errors due to limitations of human operators. Moreover, the similarity between structures of interest and the background often leads to challenges in distinguishing and accurately recognizing biological compounds, resulting in inherent arbitrariness and interpretation bias.

% Automated Analysis and Deep Learning: Potential and Challenges
For these reasons, there is a growing interest in automating the recognition and counting of tagged elements in fluorescence microscopy\cite{morelli2021cresunet,cao2020denseunet,Riccio2019,kumar2020multisegmentation}. 
Deep learning approaches have demonstrated great promise in various object recognition tasks. However, their performance can deteriorate when applied to data from domains significantly different from those adopted for pre-training (\textit{domain shift}\cite{medical_domain_shift,poon2023dataset}). Furthermore, the effectiveness of these approaches typically relies heavily on the availability of well-annotated data\cite{curse_dataset_annotation}, which is often scarce and limited in the fluorescence microscopy domain.

% The Fluorescent Neuronal Cells (FNC) Dataset: Overview and Significance
% In response to the need for comprehensive annotated data in fluorescence microscopy, 
To mitigate these issues,
we present the Fluorescent Neuronal Cells v2 (FNC) dataset. This archive features 3 data collections, for a total of 1874 high-resolution images of rodents brain slices capturing a diverse range of neuronal structures and staining patterns. To facilitate research in this field, we also provide 750 annotations in various formats, tailored to popular supervised learning tasks such as semantic segmentation, object detection, and counting. 
Apart from serving as an additional benchmark for testing model generalization in microscopy applications, the FNC dataset opens up several research opportunities.
Firstly, the heterogeneity of biological structures and their visual characteristics enable testing the generalization of trained models, and validating transfer learning and domain adaptation methods \cite{haq2020adversarial_domain_adaptation,brieu2019domain_adaptation}. 
Also, the availability of multiple annotation types allows the exploration of different learning paradigms, ranging from supervised and unsupervised approaches to self-/weakly-supervised techniques. 
Moreover, the specific challenges of our data well suit investigations into methodological advancements, e.g., assessing the effectiveness of different annotation formats and uncertainty estimation.

% Study Design: Image Acquisition and Data Annotation
The design of the data collection process involved two distinct stages. Firstly, data collection was conducted following standardized experimental protocols. Specifically, controlled experimental conditions were applied to the animals, whose brains were sliced and processed by a classical immunofluorescence protocol to stain various neuronal substructures.
Subsequently, a fluorescence microscopy was employed to capture high-resolution images of the areas of interest.
Secondly, domain experts performed data annotation providing ground-truth labels necessary for supervised learning.

% Figure environment removed

Despite the presence of open-source fluorescence microscopy datasets, several issues hinder their utilization for training deep learning models. 
Firstly, these collections typically lack accompanying ground-truth annotations, thus precluding the adoption of supervised learning techniques.
Secondly, labelled datasets often include just a few dozens of images\cite{RAZA2019160,SabineTaschner-Mandl2020}, that can be restrictive considering the data-intensive nature of deep learning models.
Also, the moderate resolution  of images in open datasets\cite{waithe_dominic_2019_2548493,stringer2021cellpose} hampers the effectiveness of resorting to crops as an alternative to whole images for augmenting sample size.
% A common approach to mitigate this issue is to use crops instead of whole images for training, thus enhancing data size. However the reduced resolution of open datasets\cite{waithe_dominic_2019_2548493,stringer2021cellpose} limits the impact of such strategies.
% Furthermore, the number of images alone is not the sole determinant, as the images are typically cropped during model training, thus making each crop the actual data sample size. In this context, image resolution plays a crucial role, as higher-resolution images generate a larger number of crops that serve as inputs for deep learning models. Thus, having a large set of low-resolution images\cite{waithe_dominic_2019_2548493,stringer2021cellpose} can be limiting.
% Another important consideration for achieving effective performance across diverse tasks is the type of markers used to generate the images. In fact, most existing datasets predominantly include individual cells in images that utilize nuclear and/or membrane markers\cite{RAZA2019160,SabineTaschner-Mandl2020}, leading to a lack of diversity that impedes robust model training. 
% Furthermore, from a technical standpoint, the quality and type of detection that can be performed are influenced by the manner in which the cells are annotated. Unfortunately, annotations in existing datasets are often provided in the form of dot annotations or bounding boxes\cite{RAZA2019160,SabineTaschner-Mandl2020,waithe_dominic_2019_2548493,stringer2021cellpose}, which do not capture cells shape accurately.
Thirdly, most existing datasets predominantly include a single marker type\cite{RAZA2019160,SabineTaschner-Mandl2020}, thus lacking diversity and limiting robust model training. 
Alongside these aspects, public datasets typically provide label types as dot-annotations or bounding boxes\cite{RAZA2019160,SabineTaschner-Mandl2020,waithe_dominic_2019_2548493,stringer2021cellpose}, which prevents their extension to fine-grained segmentation tasks.
Additionally, the data accessibility is sometimes restricted due to the use of domain-specific formats\cite{poon2023dataset}, which complicates integration into deep learning frameworks and wide dissemination.

In response to these challenges, we present a large archive comprising high-resolution fluorescent microscopy images, encompassing different markers and cell types. 
Furthermore, the data are shared as easily accessible \textit{PNG} files, and the corresponding annotations are provided in various types and formats, enabling the exploration of different learning approaches and tasks, thereby significantly expanding the scope of potential applications.
