The FNC dataset compiles images acquired from multiple studies and experimental conditions, while maintaining a consistent structure in the acquisition pipeline. Minor modifications were made to accommodate the specific requirements of each study and adapt to the current experimental circumstances and equipment.
The data collection process consisted of two distinct and independent stages: \textit{image acquisition} and \textit{data annotation}. This section provides a comprehensive description of the data acquisition design, including the dedicated measures implemented for each image collection (refer to Figure \ref{fig:design-diagram} for a visual summary).
\input{data_preview}

\subsection*{Image acquisition}

In the image acquisition phase, a total of 68 rodents were subjected to controlled experimental conditions to study torpor and thermoregulatory mechanisms. 
% The specific details are out of the scope of our study, please refer to hitrec.
% Subsequently, the animals underwent surgical injection with one or more distinct markers following standard protocols for immunofluorescence staining \cite{hitrec2019neural}.
At the end of the experimental session, the animals were deeply anaesthetized and transcardially perfused with 4\% formaldehyde\cite{hitrec2019neural}. % (see \cite{hitrec2019neural} for further details).
This process allowed for the tagging of several neuronal substructures located within the nucleus or cytoplasm of the neurons.
% In this way, several neuronal substructures located within the neurons nucleus or cytoplasm were tagged.

% After the injection, rodent brains were sectioned into 35 µm thick tissue slices, with sampling conducted at regular intervals of Y to avoid redundant data and ensure comprehensive coverage while maintaining manageable data size.
Rodents brains were then sectioned into 35 $\mu m$ thick tissue slices, with sampling conducted at regular intervals (105 $\mu m$ for mice and of 210 $\mu m$ for rats) to avoid redundant data and ensure comprehensive coverage while maintaining manageable data size. Brain slices were finally stained for distinct markers following a standard immunofluorescence protocol  \cite{hitrec2019neural}.
Only some areas of interest were observed, namely the Raphe Pallidus (RPa), Dorsomedial Hypothalamus (DMH), Lateral Hypotalamus (LH), and Ventrolateral Periaqueductal Gray (VLPAG).
These specific brain regions were chosen based on their relevance to the study of torpor mechanisms.
The resulting specimens were observed by means of a fluorescence microscope equipped with a high-resolution camera. 
A specific wavelength of excitation light was selected for each collection based on the excitation wavelength of the chosen marker, resulting in pictures acquired with the application of green, yellow/orange or red filters. For simplicity, the image collections are named according to their prevalent hue.
% Consequently, the acquired pictures present Based on their relative emission wavelengths, green, yellow/orange and red pictures were acquired; the three datasets were named according to their prevalent hue.
% For each dataset, a different wavelength of light was selected based on the injected marker and the available colorations. As a result, the acquired pictures present one prevalent tone, namely green, yellow/orangend red.
The original images were acquired as either \textit{TIF} or \textit{JPG} files depending on the camera default settings. 
To ensure traceability, a file naming convention was adopted to indicate their respective sample origins\footnote{\texttt{<animal\_id>\_S<sample\_id>C<column\_id>R<row\_id>\_<brain\_area>\_<zoom>\_<collection\_id>}}.
During the analysis phase, the raw data were converted to uncompressed \textit{PNG} format, taking care to preserve the extensive set of associated metadata. 
This conversion aimed to enhance accessibility and facilitate broader utilization of the data, allowing for inspection and manipulation without the need for specialized software.
Consequently, the FNC archive includes both these derived images and the original raw images, which are retained for data recovery and reproducibility purposes.
% \input{data_preview}

\subsubsection*{Green and Yellow collections}
The images within these collections were obtained during the same experiment \cite{hitrec2019neural}, in which brain sections from C57BL/6J mice were stained with two markers to highlight specific substructures present in the neurons' nucleus and cytoplasm. The resulting brain slices were then observed using a Nikon Eclipse 80i microscope, equipped with a Nikon Digital Sight DS-Vi1 color camera, at a magnification of 200x.

More specifically, the \textbf{green} collection corresponds to cFOS staining (cf. \Cref{fig:green_mask}). This staining method was employed to emphasize the nuclei of active neuronal cells\cite{kovacs2008measurement}, enabling the topographic analysis of brain areas that exhibit neuronal activity under specific experimental conditions. This approach is widely employed to identify neuronal cells responsible for regulating specific physiological phenomena.

In contrast, the \textbf{yellow} collection (cf. \Cref{fig:yellow_image}) utilized staining for the b-subunit of Cholera Toxin (CTb). This monosynaptic retrograde neuronal tracer migrates within the soma and axons of neuronal cells projecting to the brain area where CTb was previously injected during in vivo experiments\cite{lencer2003intracellular}. Consequently, this staining technique facilitates the identification of morphological connections between different brain regions.

\subsubsection*{Red collection}
The \textbf{red} collection comprises images obtained from multiple unpublished experiments, concerning specimens of both mice and rats (cf. \Cref{fig:red_image}). 
Despite sharing the same experimental setup as green and yellow collections, this time the brain tissues were stained for various elements to phenotypically characterize the cells involved in the neural circuits underlying the physiological phenomena of torpor and thermoregulation.
Specifically, slices were stained for orexin, tryptophan hydroxylase, and tyrosine hydroxylase.
% In this experiment, performed by using the same experimental set up employed for green and yellow datasets, brain tissue was stained for various elements, in order to phenotypically characterize the cells involved in the neural circuits underlying the physiological phenomena of torpor and thermoregulation. Namely, slices were stained for orexin, tryptophan hydroxylase, and tyrosine hydroxylase.% This staining technique allows for the phenotypical characterization of neurons and is extensively used to identify cells involved in neural circuits underlying physiological phenomena. 
In this case, image acquisition was conducted using both the aforementioned Nikon Eclipse 80i microscope and an ausJENA JENAVAL microscope, equipped with a Nikon Coolpix E4500 color camera, at a magnification of 250x. 
For further details, please refer to the accompanying metadata for each image. 


\subsection*{Data annotation}

The data annotation process was carried out by multiple proficient experimenters according to a fixed annotation protocol\footnote{check \textit{Annotations protocol.pdf} inside \textit{Annotations.zip} in the data archive}, with multiple revision rounds to ensure data quality and minimize operator bias.
We adopted the Visual Geometry Group Visual Image Annotator (VIA) annotation tool\cite{dutta2016via,dutta2019vgg}, which employs a web interface for image visualization and allows for the overlaying of annotations in different forms.
In our study, the tagging process involved creating polygon contours, and the resulting annotations were exported into \textit{CSV} format.
To generate the binary masks required for training, the polygon contours were transformed using programming libraries such as OpenCV and scikit-image. % (refer to the \hyperref[sec:usage]{Section \textit{Usage Notes}} for additional details).
For the yellow collection, we utilized the binary masks available from version 1\cite{clissa2021fluocells} as pre-annotations. Specifically, we employed erosions and dilations techniques to address fragmented contours resulting from semi-automatic labeling based on thresholding. 
Furthermore, we applied methods to fill small holes within segmented objects, and removed spurious objects that went overlooked in the previous annotations or were erroneously added by prior processing.
Subsequently, these pre-annotations were refined manually using VIA, enhancing their accuracy and ensuring better consistency across the annotations (see \Cref{fig:yellow_review}). In contrast, the green and red collections were annotated from scratch.

% Figure environment removed

Upon completion of the labeling process, the polygon contours exported from VIA were also converted into multiple annotation types and formats. This conversion aims at facilitating accessibility for a wide range of users and promoting the exploration of various learning problems related to our data. For a more comprehensive understanding of the available formats and annotation types, please refer to the \hyperref[sec:data]{Section \textit{Data Records}}.