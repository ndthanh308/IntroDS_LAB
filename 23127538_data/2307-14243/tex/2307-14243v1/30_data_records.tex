The FNC dataset is a collection of 1874 high-resolution fluorescent microscopy pictures, 750 of which also have their corresponding ground-truth segmentation masks, while the remaining 1124 are unlabelled.
It is hosted on \href{https://amsacta.unibo.it/id/eprint/7347}{AMS Acta}\footnote{available at: \href{https://doi.org/10.6092/unibo/amsacta/7347}{https://doi.org/10.6092/unibo/amsacta/7347} (link will be active after acceptance)}, the open access repository managed by the University of Bologna.
The data are organized into three standalone image collections, named for simplicity \textit{green}, \textit{yellow}, and \textit{red}, each available under the corresponding folder (see \Cref{fig:archive-composition}). 
The collections share a common layout to facilitate easy access and analysis (see \Cref{fig:dataset_structure_dirtree}).

To aid users in navigating the archive, the \texttt{metadata\_v2.xlsx} file provides a comprehensive overview of the FNC data collection.
It includes high-level metadata for each image, such as the corresponding animal, acquisition details, data partition, and annotation information.

% Figure environment removed

\subsection*{Image collection folder structure}

The \texttt{trainval} and \texttt{test} folders contain all labelled images for each collection. These data partitions were obtained through a random 75\%/25\% split and are recommended as a suggested configuration to ensure reproducibility and comparability in future studies.
The remaining images were collected under the \texttt{unlabelled} folder.

Inside each data partition folder, the \texttt{images} folder contains fluorescence microscopy images in \textit{PNG} format.
All the images are accompanied by a rich set of metadata, stored both in their \textit{EXIF} tags and as a separate \textit{TXT} file under the \texttt{metadata} folder.
The \texttt{ground\_truths} folder contains annotations in various formats commonly used within the machine learning community (see \Cref{fig:annotations_structure_dirtree}).

\subsection*{Annotation types and formats}

The FNC collection provides annotations of multiple \textit{types}, encoded in several standard \textit{formats}.
In the \texttt{masks} folder, we find the binary masks typically used for segmentation tasks (cf. \Cref{fig:green_mask,fig:red_mask,fig:yellow_mask}). The correspondence between the masks and the respective images can be established based on the filenames.
The other folders store a light-weight encoding of the binary masks, enriched with additional annotation types/formats.

The \texttt{rle} directory contains \textit{Running Length Encoding (RLE)} of the binary masks, stored as \textit{pickle} files. 
This encoding is a compressed representation that can effectively save disk space while preserving the complete segmentation information. It is particularly convenient for high-resolution images like those present in our dataset.

The other directories provide several annotation types, and they are named after their annotation format.
\textbf{Polygon} annotations are available in each of the \texttt{VIA}, \texttt{COCO} and \texttt{Pascal\_VOC} directories, in the form of \textit{json} or \textit{xml} files. 
\texttt{COCO}\cite{COCO} and \texttt{Pascal\_VOC}\cite{everingham2010pascal} formats also features \textbf{bounding boxes} and \textbf{dot} annotations for object detection tasks, and \textbf{count} labels for object counting.
