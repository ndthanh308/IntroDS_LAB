\section{\label{app2}Qualitative description of low-rank approximation filters}

This entire section is a verbatim quote from the Appendix B of the Ph.D. thesis \textit{"Development of new experimental and data processing methods at critical signal-to-noise conditions in nuclear magnetic resonance"} from page 101 to page 113.
\begin{displayquote}

\subsection{System parameters for Montecarlo simulation}
\begin{align}\label{eq:spectral_ap}
	\text{FID(t,T)}= \sum_k M_k(T) e^{i2\pi f_k t}e^{\frac{-t}{T_{2k}}} \qquad k=[\text{pyr, ala, lac, bic}].
\end{align}
Where FID(t,T) is the value of the \gls{fid} at the acquisition time $t$ after $T$ second from the injection, $M_k$ is the longitudinal polarization of the $k$ molecules, and $f_k$ is the molecules off-set frequency. The values of $M(T)_k$ is given by:
\begin{align}
	\label{eq:pyr_concentration_ap}
	M_{pyr}(T) = 
	\begin{dcases*}
		r_{inj}T_{1pyr} 	\left( 1- \exp{\frac{t_{arr}-T}{T_{1pyr}}}\right) & if  $t_{arr}\leq T < t_{end}  $\\
		M_{pyr}(t_{end})\exp{\frac{t_{end}-T}{T_{1pyr}}} & $T \geq t_{end}$,
	\end{dcases*}
\end{align}
where $r_{inj}$ is the injection rate, $t_{arr}$ and $t_{end}$ are respectively the starting time and the end time of injection.
While for the other metabolite we have:
\begin{align} 
	\label{eq:met_concentration_ap}
	M_{l}(T) &= 
	\begin{dcases*}
		A_l \left( T_{1l}\left( 1- e^{\hat{T}_l}\right) - T_{1pyr}\left( 1- e^{\hat{T}_{pyr}}\right)\right) & if  $t_{arr}\leq T < t_{end}  $\\
		B_l \left( e^{\hat{T}_l}-e^{\hat{T}_{pyr}} \right) + M_l (t_{end}) e^{\hat{T}_l} & $T \geq t_{end}$
	\end{dcases*}\\
	\hat{T}_l & =\frac{t_{arr}-T}{T_{1l}} \qquad \qquad \qquad \qquad \hat{T}_{pyr} =\frac{t_{arr}-T}{T_{1pyr}}\\
	A_l  &= \frac{r_{inj} k_{l}}{1/T_{1pyr}-1/T_{1l}} \qquad \qquad  \quad 	B_l   = \frac{M_l(t_{end}) k_{l}}{1/T_{1pyr}-1/T_{1l}}
\end{align}

Equation \ref{eq:spectral_ap}, \ref{eq:pyr_concentration_ap} and \ref{eq:met_concentration_ap} describe the model used for the Montecarlo simulation, the numerical values for the parameter used is reported in table \ref{tab:parms}.

\begin{table}
\begin{tabular}{lccccc}
\toprule
Metabolite   & Inj. Rate (s$^{-1}$) & T$_{1eff}$ (s) & r$_{eff}$ (s$^{-1}$) & T$_2$ (ms) & Freq (Hz)\\
\midrule
Pyruvate     & 2.354e-2  & 19.998 &       -   & 49.996 & 2.158e-06 \\
Lactate      & -         & 13.178 & 1.899e-2 & 50.026 & -392.000\\
Bicarbonate  & -         & 18.472 & 7.688e-3 & 50.061 & 321.309 \\
Alanine      & -         & 12.578 & 1.498e-2 & 50.031 & -270.000\\
\bottomrule
\end{tabular}
\caption{\label{tab:parms} Model parameters are used for the Montecarlo simulations, rounded to the third significant digit}
\end{table}

\subsection{Qualitative description of low-rank approximation filters}
The dataset under study has two dimensions\footnote{The data acquired during the experiment could be arranged into a matrix}: the first one describes how the signal evolves during the \gls{nmr} spectroscopy measurement and in particular how the transverse magnetization changes after the radio-frequency pulses; the second one describes how the longitudinal magnetization evolves between the \gls{nmr} spectroscopy experiments. The metabolite could be identified using the information on their Larmor frequency contained in the first dimension of the dataset. Once the metabolites have been identified, the second dimension could be studied to identify the apparent conversion rate for each metabolite. The first dimension could be called the spectral dimension, while the second one could be called the metabolic dimension. Each dimension could be decomposed into some base components.   
This section compares the data components naturally arising from the metabolic conversion of pyruvate model with the components obtained using the \gls{svd} in a data drive approach.
The first kind of components, here called natural components\footnote{Please note that this naming convention is informal and personal; this naming choice does not convey any deep insight on the properties of these components}, are identified considering the metabolic conversion model, each metabolite spectrum is a natural component for the spectral dimension, while the time evolution curves for the longitudinal magnetization for each metabolite is a natural component for the metabolic dimension.
\subsubsection{Natural components}
The natural components of this system were already described in the main body of this thesis, here we just report their shape in figure \ref{fig:conc_ap} to ease the comparison with the data driven components. Their values are obtained from equations \ref{eq:spectral_ap}, \ref{eq:pyr_concentration_ap}, and \ref{eq:met_concentration_ap}.

% Figure environment removed
\subsubsection{Data driven components: the noiseless case}
In the data driven approach the dataset components are obtained form the direct analysis of the acquired data and not from the study of the system's model.
The \gls{svd} is used to construct the components from the dataset producing three matrices such as:
\begin{align}
	\mathbf{M} = \mathbf{USV^{H}}
\end{align}
were $\mathbf{M}$ is a 59$\times$2048 matrix containing the simulated data, $\mathbf{U}$ is a 59$\times$59 unitary matrix which contains the information on the metabolic dimension of the dataset, $\mathbf{S}$ is 59$\times$59 a diagonal matrix that contains the information regarding the weights of the components and $\mathbf{V}$ is a  2048$\times$59 unitary matrix that contains the information on the spectral dimension of the dataset. The original data can be reconstructed from the sum and the product of the components. For example the \nth{10} spectrum is obtained by first multiplying the spectral components with their respective singular values (In this system we can consider just the first four components without a significant loss of accuracy). The singular values measure the contribution of the respective spectral components. Then, the weighted spectral components are multiplied with the elements of the \nth{10} metabolic component, now each spectral components is weighted for its general importance (the singular value) and its specific importance into the representation of the \nth{10} spectrum (the metabolic component elements).

The singular values intensity is reported in figure \ref{fig:svnoise}, only the first 4 singular values are different from zero considering the machine precision. Therefore, in the following analysis we will consider only the elements of $\mathbf{U}$ and $\mathbf{V}$ corresponding to these four singular values.

Figure \ref{fig:NNspectral} reports the first four components of the spectral dimension obtained by \gls{svd} analysis. We reported in blue the real values of the spectra and in orange the imaginary part. These components are quite different from the natural one, each component contains information on all the metabolites peaks mixed in different ways. Furthermore, the peaks "lineshape" is clearly non-Lorentzian and does not have the proper zero order phase factor between them. Indeed, these are not proper lineshapes originated form the physical nature of the system interaction, but just shapes needed to assure the linear independence of the components. Since the natural components are not orthogonal, the \gls{svd} derived components can not share their shape.

% Figure environment removed

Figure \ref{fig:NNmetabolic} reports the first four components of the metabolic dimension obtained by \gls{svd} analysis. We reported in blue the real values of the spectra and in orange the imaginary part. The metabolic dimension's components share the same characteristic of the spectral dimension one. The discrepancy between the natural and the data driven components is exacerbated by the higher degree of non-orthogonality in the metabolic natural components compared to the spectral natural components\footnote{The degree of non orthogonality could be better quantified taking the scalar product between the natural components after their normalization. A result of zero means that the components are orthogonal while a results of $\pm$1 means that the components are collinear}. Therefore, data driven metabolic components are even more different from natural components than spectral components. For example, the data-driven metabolic components are expressed as complex numbers while the natural components are real values as they describe the magnitude of polarization. 
% Figure environment removed
To sum up, both data drive components are clearly different from the natural components of the previous subsection. While the natural components has a clear physical interpretation, these components does not. For example, the metabolic components are complex numbers which is inconsistent with the physical interpretation of metabolic components as a direct description of the longitudinal magnetization magnitude associated with a single metabolite.
The discrepancy between the two sets of components derives from the mathematical properties of the \gls{svd}. Indeed, the matrices $\mathbf{U}$ and $\mathbf{V}$ should be unitary, therefore, their rows and columns (that constitute the data driven components) should be linear independent\footnote{A set of vectors are linearly independent when each vector can not be written as the weighted sum of the remaining vectors} while the natural components are linear dependent.
Nevertheless, the direct interpretation of the data driven components is not needed for the creation of a low-rank approximation filter. Indeed, the objective of the \gls{svd} is to separate the signal form the noise and not to decompose the signals into a model explaining components set. 

\subsubsection{Data driven components: the noisy case}

% Figure environment removed
The presence of noise into the dataset corrupts the components obtained using the \gls{svd}. In this context the components are corrupted if they represent not only the true signal but also a significant fraction of the noise. This intuitive definition tries to capture a quite complex mathematical phenomenon. The \gls{svd} decomposition is not aware of the signal and noise nature, the \gls{svd} is just an instrument to solve the low-rank approximation problem, i.e. to find the best approximation to the original dataset using a fixed number of linearly independent components. Since noise and signal are not linearly independent the \gls{svd} can not perfectly separate the noise from the signal and part of the signal gets corrupted by the noise because the addition of a noise portion reduces the overall distance between the approximated dataset and the original noisy dataset.



% Figure environment removed

Figure \ref{fig:spectral} represents the effect of four different noise levels on the first four components of the spectral dimension. The blue, orange and green curves are, respectively, obtained form a dataset with a \gls{snr} of 2048, 512 and 64. The red curves are obtained from a noiseless dataset. For ease of representation we reported just the real part of the components: it should be notate that part of the differences between the curves is due to the presence of a zero order phase factor due to the \gls{svd} properties.
Therefore, the smoothness of the curves is a better parameter to monitor the noise-related corruption for the data driven components compared to the curve magnitude. Indeed, the pure signal components are smooth because the \gls{nmr} spectrum is a continuous function of the frequency, while the noise is a Gaussian random process that creates a rough curve. 
Figure \ref{fig:metabolic}  represents the effect of four different noise levels on the first four components of the metabolic dimension using the same convention employed in figure \ref{fig:spectral}. Since both figures depict the components' corruption due to the noise, we can focus our attention just on  figure \ref{fig:metabolic}, which is easier to analyze. The same observations hold for figure \ref{fig:spectral}. 

We can summarize the results of our visual inspection of the curve smoothness as: 
\begin{itemize}
	\item The first component is smooth for all the \gls{snr} considered.
	\item The second component is rough just for the lowest \gls{snr} level considered, 64.
	\item Only the highest \gls{snr} level has an almost smooth third component. 
	\item The fourth component is rough for all the \gls{snr} levels considered.
\end{itemize}
All these results are in perfect accord with the analysis of the singular values intensity curve in figure \ref{fig:svnoise}. A singular value is usable in the signal reconstruction if it is before the knee in the curves. Therefore, there are 4 singular values usable for the dataset with \gls{snr} equal to 2048, and 3 for the other two datasets. 

% Figure environment removed

\subsection{Qualitative evaluation of low-rank filter performance}
The qualitative evaluation of the low-rank filter performance is carried out with a visual inspection of few spectra at different \gls{snr} levels. This is a qualitative analysis in the sense that it does not provide quantitative information on the underling physical system. The goal of hyperpolarized metabolic tracer experiment is to quantify and study the effective reaction rate. Therefore, the efficacy of a filter should be evaluated on its ability to improve the estimation of the effective reaction rate. The direct inspection of the spectra or the evaluation of the mean square errors between the filtered and the pure signals do not provide clear information on the ability of the filter to improve the estimation of the effective reaction rate, or other model parameters.

In figure \ref{fig:spec_noisy} the effect of the low-rank filter on the spectrum at different \gls{snr} levels 4,8,16, and 32 is reported. The \gls{snr} level is calculated on the highest intensity signal of the experiment, the signal acquired at the end of the injection, 14 second after the beginning of the injection. I chose this \gls{snr} levels because they provide a clear insight on the efficacy of the methods for retrieving clear spectra even in the presence of low \gls{snr}. While this is not a sufficient condition to improve the estimation of the parameter describing the system reaction it is still a notable result and can be useful for analysis based on peak integration. To further highlight the effect of the low-rank filter the spectra after 20 and 50 second are reported since both this times feature a lower \gls{snr} compared to the maximum of each experiment and also have lower differences in the peaks' intensity compared to the signal after 14 seconds.
The spectrum simulating the signal acquired after 20 and 50 seconds from injection are reported, respectively, on the left and the right column of the figure panel. The noisy signal is reported with light blue dots, the \gls{svd} filtered signal is reported with a orange solid line and the pure signal is reported with a red solid line. The \gls{svd} filter was constructed using the first 3 singular values based on the inspection of the singular values intensity in figure \ref{fig:svnoise}.
% Figure environment removed

The most interesting results are obtained for the 50 seconds after injection spectra, the \gls{svd} filter allows for a clear identification of the peaks and their area even when their intensity is lower than noise level, for example in the first two rows of the figure panel. This result is particularly interesting for the pyruvate metabolites, their lower intensity makes their identification harder and the estimation of the peaks area is more susceptible to the presence of noise compared with the stronger pyruvate peak. While these noise reduction performances do not directly translate into an increased quality of the metabolic parameters estimated from the filtered dataset, they are still useful for the qualitative inspection of the spectra.

\end{displayquote}
