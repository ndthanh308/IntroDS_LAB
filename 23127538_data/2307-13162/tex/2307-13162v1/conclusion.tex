\vspace{-2mm}
\section{Conclusion} \label{sec:conclusion}
In this paper, we study the problem of single-node PageRank computation on undirected graphs. We propose a novel method, \setpush, which achieves the $\tilde{O}\left(\min\left\{d_t, \sqrt{m}\right\}\right)$ expected time complexity for estimating the target node $t$'s PageRank with constant relative error and constant success probability. {\rev We prove that this is the best result among existing methods on undirected graphs. We also empirically demonstrate the effectiveness of \setpush on large-scale real-world datasets. For the future work, we note that the lower bound for the problem of single-node PageRank computation on undirected graphs is still unclear. Since we have already achieved the complexity bound $\tilde{O}\left(\min\left\{d_t, \sqrt{m}\right\}\right)$, a natural question is whether this complexity matches the lower bound for the problem. 

\begin{comment}
For the future work, 
%we note that the lower bound for the problem of single-node PageRank computation on undirected graphs is still unclear. Since we have already achieved the complexity bound $\tilde{O}\left(\min\left\{d_t, \sqrt{m}\right\}\right)$, a natural question is whether this complexity matches the lower bound for the problem. We leave open this problem for the future work. 
we leave open two non-trivial questions with respect to single-node PageRank computations on undirected and directed graphs, respectively.  
%In the future, we will continue to focus on the problem, and study an interesting question that 

\begin{itemize}
\item First, on undirected graphs, the lower bound for the problem is still unclear. Since we have already achieved the complexity of $\tilde{O}\left(\min\left\{d_t, \sqrt{m}\right\}\right)$, a natural question is whether this complexity matches the lower bound for the problem. 

\item Second, on directed graphs, the best complexity for estimating an arbitrary target node's PageRank within constant relative error is $\tilde{O}\left(\min\left\{\frac{m^{2/3}\cdot \dmax^{1/3}}{\davg^{2/3}}, \frac{m^{4/5}}{\davg^{3/5}} \right\}\right)$, achieved by the \sublinear method proposed by Bressan et al. in 2018~\cite{bressan2018sublinear}. Meanwhile, Bressan et al. proposed a lower bound $\tilde{O}\left(\min\left\{\frac{m^{1/2}\cdot \dmax^{1/2}}{\davg^{1/2}}, \frac{m^{2/3}}{\davg^{1/3}} \right\}\right)$ for the problem. As pointed out by~\cite{bressan2018sublinear}, there is a theoretical gap between the upper bound and the lower bound. We leave open the problem of whether the theoretical gap can be narrowed.  
\end{itemize}
\end{comment}



%is it possible to further reduce the time complexity of single-node PageRank computation to $O(\sqrt{n})$? 
}


%%% Local Variables:
%%% mode: latex
%%% TeX-master: "paper"
%%% End:
