\section{Results in real scenes}
\label{sec:real-captures}
In the following we illustrate and validate our methods in real  scenarios, imaging diffuse planar surfaces \new{inside the null-reconstruction space of} third-bounce methods in single-corner configurations, and then imaging scenes hidden behind two corners.

\paragraph{Hardware details}
%
% Figure environment removed
%
Our NLOS imaging system consists of a SPAD array sensor, a laser emitter and a two-mirror galvanometer (\fref{fig:hardware-setup}).
%
\new{The galvanometer guides the laser towards multiple points on the relay surface, while the detector is aimed at a fixed position on the relay surface.} 
%
A $\text{PM-1.03-25}^\text{TM}$ laser from Polar Laser Laboratories is used as an illumination source.
The laser is combined with a frequency doubler to emit $\unit{515}{\nano\metre}$ pulses with a maximum pulse width of $\unit{35}{\pico\second}$, an average power of $\unit{375}{\milli\watt}$, and at an average repetition rate of $\unit{5}{\mega\hertz}$. A two-mirror Thorlabs galvanometer (Thorlabs GVS012) is used to scan a relay surface at $\unit{1}{\centi\metre}$ spacing, with a total scan area \neww{around $\unit{1.9 \times 1.9}{\meter}$}.
%
Our detector is a 16x16 Single Photon Avalanche Diode (SPAD) array \cite{riccardoFastGated16162022} focused at a $\unit{7.1}{\centi\metre}$ by $\unit{4.7}{\centi\metre}$ area on the relay surface using a Canon EF $\unit{85}{\milli\metre}$ f/1.8 USM Lens. The temporal resolution of the array has a Full-Width at Half Maximum (FWHM) of around $\unit{60}{\pico\second}$ and a deadtime of less than $\unit{100}{\nano\second}$.
\new{All scene surfaces are diffuse expanded polystyrene foam and unfinished drywall, with no retroreflective properties. }


% Figure environment removed




\paragraph{\new{Addressing the missing cone}} We design a \new{scene similar to \fref{fig:goal2-overview}a to test our \newww{procedure to} address the \NEW{missing-cone problem} with our hardware setup. A photograph of the scene is displayed in  \fref{fig:goal2-real}a, with two \NEW{hidden} surfaces $\lM$ and $\lG$. Surface $\lG$ is in the null-reconstruction space of third-bounce methods since third-bounce illumination from $\lG$ falls outside our imaging aperture $\lS$ in the computational domain.}
%
\new{We aim to first infer the position and orientation of $\lG$, 
then to directly image $\lG$. We capture the impulse response function $H(\xl, \xs, t)$ for an illuminated point $\xl$ and points $\xs$ of the aperture $\lS$ at the relay surface. The surface $\lM$} produces fourth-bounce illumination at $\xs$ from our target diffuse surface $\lG$. We experiment with three different orientations of $\lG$ ($\unit{90}{\degree}$, $\unit{100}{\degree}$ and $\unit{80}{\degree}$) with respect to the relay surface, which is in all cases located at $\unit{50}{\centi\metre}$ from the illuminated point $\xl$ (in all three cases, $\lG$ cannot be imaged using existing NLOS algorithms). The surface $\lM$ is tilted at $\unit{30}{\degree}$ and separated $\unit{1.5}{\metre}$ from $\xl$ \NEW{at its center point}. \fref{fig:goal2-real}b and c show \NEW{photographs} and top-view schematics of the different orientations, respectively. 
%
\fref{fig:goal2-real}d shows how, for all three orientations, \neww{the position $\xg$ and orientation $\normalg$ of $\lG$} are accurately inferred from the illuminated point $\xl$ and its reflection $\xlG$ \new{\NEW{captured} with the transient camera from the aperture $\lM$ (\eref{eq:second_RSD}), in $\ftcM(\xw, t=0)$.}
%
For the $\unit{90}{\degree}$, $\unit{100}{\degree}$ and $\unit{80}{\degree}$ cases, our inferred $\xg$ is $\unit{6}{\centi\metre}$, $\unit{4}{\centi\metre}$ and $\unit{5}{\centi\metre}$ away from $\lG$, and $\normalg$ has an orientation error of $\unit{0.3}{\degree}$, $\unit{2.8}{\degree}$ and $\unit{0.1}{\degree}$, respectively. 
%
\neww{Additionally, to directly image plane $\lG$ we turn $\lM$ into a secondary aperture where we implement a} confocal camera  (\eref{eq:second_RSD_confocal}) to obtain the image $\fccM(\xw, t=0)$.
%at $t=0$} 
The results are shown in \fref{fig:goal2-real}e. \neww{Similar to \fref{fig:goal2-results}, the $\fccM$ model produces a bright region near $\xl$ due to \NEW{coupled} illumination in the impulse response $H(\xl, \xs, t)$.} 



% Figure environment removed


\paragraph{Looking around two corners}
In this experiment we use the scene shown in \fref{fig:goal1-real}a, where we image objects hidden behind two corners.
\new{The scene is made up of a diffuse surface} $\lM$ at a $\unit{45}{\degree}$ angle with respect to the relay surface, and several hidden geometries \NEW{$\lG$} (\fref{fig:goal1-real}b, top) oriented at $\unit{90}{\degree}$ with respect to the relay surface. Two occluders ensure that the \NEW{geometry} $\lG$ hidden around two corners is not directly visible from points $\xs \in \lS$, $\xl$, or the \new{capture hardware itself.}
%
For each geometry, we image \new{points $\xv$} in $\lV$ where the mirror images are produced by plane $\lM$, \new{using the confocal camera model \neww{(\eref{eq:RSD_freq_cc}) with the impulse response $H(\xl, \xs, t)$} to obtain $\fcc(\xv, t)$}.
%
\neww{\NEW{The frame at $t=0$ of} $\fcc(\xv, t)$} \NEW{captures the} objects hidden around two corners (\fref{fig:goal1-real}b, bottom) by leveraging five-bounce specular paths $\xl\rarr\lM\rarr\lG\rarr\lM\rarr\lS$ \new{in the computational domain}.
%
The resulting images appear blurrier than single-corner reconstructions since the mirror behavior at NLOS imaging frequencies is not perfectly specular (\sref{sec:discussion}). 


\paragraph{\neww{Capture noise}}
\neww{Previous NLOS imaging methods that relied on single-pixel SPAD sensors suffered from low signal-to-noise ratio, requiring long capture times.
%
The implementation of gated SPAD array sensors \cite{riccardoFastGated16162022}, which we use in our work, significantly mitigates this issue, and can enable imaging speeds of up to five frames per second \cite{nam2021low}.
%
While the signal degrades with the number of bounces,
%
in our experiments we observed only minor changes (noise) in our computed images over multiple measurements of the same scene, for the computational wavelengths previously specified. This suggests that our imaging procedures are mainly affected by other limiting factors (e.g., surface size and reflectance) than by capture noise; in fact, in our work we had to lower the power of the laser to prevent overexposing our SPAD array sensor.
%
Additionally, we compared the photon count of our two-corner experiments (\fref{fig:goal1-real}) based on the fifth bounce, and their third-bounce counterparts with the target object $\lG$ placed at the mirror location $\lGp$ in a single-corner configuration. Under the same exposure time, the total photons captured in our fifth-bounce setups is \NEW{an order of magnitude higher}
%
than their third-bounce counterparts (\NEW{around $10^9$ and $10^8$ photons, respectively}), which suggests third- and fifth-bounce setups are similar in terms of capture noise when imaging similar regions of the hidden scene.}
