Non-line-of-sight (NLOS) imaging methods are capable of reconstructing complex scenes that are not visible to an observer using indirect illumination. 
%
However, they assume only third-bounce illumination, so they are currently limited to single-corner configurations, and present limited visibility when imaging surfaces at certain orientations.
%
\new{To reason about and tackle these limitations, we make the key observation that planar diffuse surfaces behave specularly at wavelengths used in the computational wave-based NLOS imaging domain. We call such surfaces \emph{virtual mirrors}.
%
We leverage this observation to expand the capabilities of NLOS imaging using illumination beyond the third bounce, addressing two problems: imaging single-corner objects at limited visibility angles, and imaging objects hidden behind two corners.}
%
%
\new{To image objects at limited visibility angles,} we first analyze the reflections of the known illuminated point on surfaces of the scene as \NEWW{an} estimator of \NEWW{the position and orientation of objects with limited visibility}.
%
We then image those \new{limited} visibility objects by computationally building secondary apertures at other surfaces that observe the target object from a \new{direct} visibility perspective. 
%
Beyond single-corner NLOS imaging, \new{we exploit the specular behavior of {virtual mirrors}} to image objects hidden behind a second corner \new{by imaging the space behind such virtual mirrors}, where the mirror image of objects hidden around two corners is formed.  
%
No specular surfaces were involved in the making of this paper.  