\section{Introduction}
\label{sec:introduction}

% Figure environment removed

Non-line-of-sight (NLOS) imaging methods retrieve information of scenes that are hidden from the observer, including geometric reconstructions \cite{Velten2012nc,wu2021non}, position detection \cite{bouman2017turning,Yi2021SIGA}, or motion tracking \cite{gariepy2016detection}, with many applications in fields such as remote sensing, autonomous driving or biological imaging. 
%
Under this regime, methods that image scenes hidden around a corner have shown promising results thanks to ultra-fast imaging devices (e.g., \cite{velten2013femto,buttafava2015non,shin2016photon}). Such time-gated NLOS imaging methods provide detailed reconstructions of hidden scenes by triangulating geometric positions using the time of flight of round-trip third-bounce illumination paths between a visible relay surface and the hidden scene \cite{Velten2012nc,OToole2018confocal,Xin2019theory,Lindell2019wave}.

These methods operate under the assumption of third-bounce-only illumination, with higher-order illumination usually degrading the reconstructions due to ambiguities in the time of flight of light. Recent wave-based NLOS imaging methods have shown how such higher-order bounces can be isolated from third-bounce illumination and visualized, based on an imaging paradigm that interprets the time-resolved illumination captured at a relay surface as light arriving at a virtual aperture \cite{Liu2020phasor,Liu2019phasor,nam2021low,Marco2021NLOSvLTM}, effectively transforming \NEWW{the relay surface} into a virtual line-of-sight (LOS) imaging system.

In our work, we demonstrate how such higher-order bounces can be used to expand the capabilities of existing NLOS imaging systems, and overcome some of its current limitations.
In particular, we draw a parallelism between Huygens' principle and the recent wave-based phasor-field NLOS imaging formulation \cite{Liu2019phasor}.
%
We intuitively show how, due to well-known wave interference principles, \new{surfaces that are diffuse under visible light can} behave like mirrors during the \new{computational} NLOS \new{wave-based} imaging process; we call such surfaces \textit{virtual mirrors}. \new{It is thus important to understand that virtual mirrors only show specular behavior in the computational domain, since real NLOS capture systems still receive diffuse illumination.}

From this observation, we show how (in the computational domain) specular reflections in the form of \textit{fourth-} and \textit{fifth-bounce} illumination actually encode useful information about the scene, then leverage these higher-order bounces to address two longstanding problems in NLOS imaging: visibility issues due to the missing-cone problem, and looking around \textit{two} corners.

The missing-cone problem is inherent to all third-bounce NLOS imaging methods, and refers to the fact that \new{certain scene configurations and features cannot be accessed by NLOS measurements depending on their position and orientation with respect to the relay surface  \cite{Liu2019analysis}. }\new{The corresponding surfaces are said to be inside the null-reconstruction space of third-bounce methods.}

To image surfaces inside such null-reconstruction space, we first image the mirror reflections of a known illuminated point \neww{on the relay surface, produced by all surfaces of the scene, including those \new{surfaces inside the null-reconstruction space}. By analyzing these reflections, we} infer the position and orientation of the hidden surfaces \NEW{inside the null-reconstruction space} that produced such mirror reflections. 
%
\newww{To avoid ambiguities introduced by inference, we introduce a novel procedure} to directly image the hidden surfaces by \neww{creating} a \emph{second} virtual aperture at other scene surfaces.


In addition, we propose a second novel \neww{procedure to} image objects hidden behind \emph{two} corners (\fref{fig:teaser}). 
%
For this, we show how to use fifth-bounce illumination to image the space behind a diffuse hidden surface, which effectively acts as a mirror as seen from the relay surface, \newww{allowing us to observe a mirror image of the object hidden behind two corners}. 
%
\neww{Our key insight for this second procedure is selecting the location of the volume being imaged, which is in principle orthogonal to the particular imaging algorithm used.} Given our virtual mirror surfaces, we target the reflected space \emph{behind} such surfaces, where mirror images are formed (just like with real mirrors). \newww{We demonstrate our procedure} works even if the intermediate surface itself falls \new{inside the null-reconstruction space}. 

\neww{In summary, we demonstrate how to extend the capabilities of existing NLOS imaging methods, by i) imaging surfaces inside the null-reconstruction space by leveraging fourth-bounce illumination, and ii) imaging surfaces hidden around two corners using fifth-bounce illumination. }
\neww{We validate our findings both in simulation and using real captured data. Last, all of our experiment data, simulation and imaging software are publicly available\footnote{\href{https://graphics.unizar.es/projects/VirtualMirrors_2023}{https://graphics.unizar.es/projects/VirtualMirrors\_2023}}.}
