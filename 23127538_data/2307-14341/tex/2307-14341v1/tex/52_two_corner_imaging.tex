\section{Looking around two corners}
\label{sec:contribution-2:goal-1}
%
Here we leverage our virtual mirror reflections in the wave domain to image objects around \emph{two} corners.
%
\neww{The fundamental idea of our approach is exploiting our observation that diffuse planar surfaces behave like virtual mirrors (\sref{sec:light_interference}); we image the region where their mirror image is formed using only the confocal camera model (\eref{eq:RSD_freq_cc}). Note this imaging model is similar to those used by third-bounce NLOS imaging methods\newww{, analogous to the procedure detailed in \fref{fig:goal2-diagram}a}. However, unlike \newww{previous work}, we exploit fifth-bounce illumination in the impulse response function $H(\xl, \xs, t)$ by shifting the imaged region based on our observed behavior of virtual mirrors. 
%
In the following we detail this procedure using simulated data.}


% Figure environment removed


\new{We illustrate this with the simulated scene depicted in \fref{fig:goal1-results}b, composed by the relay surface with an illuminated point $\xl$ and the aperture $\lS$, a diffuse T-shaped object hidden behind two corners \NEW{denoted as} $\lG$ (which we aim to image), and a diffuse surface $\lM$ hidden behind a single corner, which behaves as a virtual mirror during the computational imaging process. We also place two black occluders to ensure that $\lM$ is not \NEW{directly} visible to the NLOS imaging device, and $\lG$ is not \NEW{directly} visible neither to the NLOS device nor to the imaging aperture $\lS$ \NEW{or illuminated point $\xl$}.}

\new{Our goal} is to image \new{the} object $\lG$.
\new{Due to the location and orientation of $\lM$ in our scene, the aperture $\lS$ and the object $\lG$ are at each other's specular direction with respect to $\lM$, \new{so that $\lM$ forms a mirror image of $\lG$ in the space behind $\lM$ \neww{that is captured from $\lS$}}. Specifically, we place the diffuse surface $\lM$ coplanar to the aperture $\lS$ with a lateral shift so it reflects light specularly towards $\lG$ and back to $\lS$ through another bounce in $\lM$. This creates fifth-bounce paths with the form $\xl \rarr \lM \rarr \lG \rarr \lM \rarr \lS$ (marked in red in the schematic of \fref{fig:goal1-results}b), which we leverage to image $\lG$. \newww{Also, note that 
in our setup $\lM$ is in the null-reconstruction space of third-bounce methods, as third-bounce specular paths do not reach $\lS$ in the computational domain. However, fifth-bounce specular paths actually reach $\lS$, and therefore we can image $\lG$ even when $\lM$ is in the null-reconstruction space.}}


\new{\neww{In our experiment setup, we first} obtain a simulated impulse response function $H(\xl,\xs,t)$ of the scene using transient rendering~\cite{royo2022non, jarabo2014framework} to mimic a real acquisition process of the relay surface.}
We then \new{implement a confocal camera model (\eref{eq:RSD_freq_cc}) and compute \NEW{the frame at $t=0$ of} $\fcc(\xv, t)$ \neww{at points $\xv$ in the imaged \NEW{plane on} $\lV$} (\fref{fig:goal1-results}b, green), where the mirror image} of $\lG$ \NEW{produced by} $\lM$ would be formed ($\lGp$ in the schematic). 
%
The computed images on \fref{fig:goal1-results}b show the result of this imaging process for two orientations of the T-shaped geometry, showing that the shape's structure is preserved on both, even after this second corner. For reference, we configure a single-corner scene (\fref{fig:goal1-results}a) by removing the surface $\lM$ and placing the object $\lG$ at the position \NEW{marked by} $\lGp$, where the mirror image should be formed for the two-corner case.
\neww{The images of the T-shaped object appear blurrier when imaged around two corners. Following our observations in \sref{sec:light_interference}, this effect is mainly caused by two factors. First, even if $\lM$ is perfectly diffuse, the mirror-like behavior of $\lM$ in the computational domain is not perfectly specular\newww{, and the resolution of the mirror images that $\lM$ produces is limited by diffraction.} Second, we use larger wavelengths on the two-corner case (i.e., values for $\lambda_c$ and $\sigma$ are higher in \fref{fig:goal1-results}b than \fref{fig:goal1-results}a). We discuss this mirror behavior and its effects further on \sref{sec:discussion}.}
