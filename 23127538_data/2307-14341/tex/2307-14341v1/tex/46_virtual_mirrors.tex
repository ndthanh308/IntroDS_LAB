\section{Diffuse surfaces as virtual mirrors}
\label{sec:light_interference}
A key observation in our work is that surfaces that are diffuse under visible light may still exhibit specular properties in \new{the computational NLOS wave imaging domain}. 
As we will show, leveraging this observation allows us to extend the current range of NLOS imaging capabilities, including imaging scenes that are hidden behind \textit{two} corners. 

In wave-based methods, time-resolved transport $\phasor(\xbf,t)$ at \new{any point $\xbf$} in the hidden scene becomes a phasor $\Pf(\xbf, \fq)$ in the frequency domain.
%
According to Huygens' principle, when reaching a surface $\lM$, this spherical wavefront will in turn generate multiple secondary spherical wavefronts. \new{As an example, consider a point light at $\xl$ \NEW{whose emission is} defined by a phasor $\Pf(\xl,\fq)$ that illuminates points $\xm$ on a planar surface $\lM$, resulting in phasors $\Pf(\xm,\fq)$. We can then} compute the resulting phasor at any point $\xv$ in a volume $\lV$ as a superposition of phasors \NEW{from $\xm$} by extending \eref{eq:phase_shift} as
%
\begin{align}
\begin{split}
    \Pf(\xv, \fq) & = \int_\lM \Pf(\xm, \fq) \frac{e^{i k \norm{\xv - \xm}}}{\norm{\xv - \xm}} \diff \xm \\
    & = \int_\lM \Pf(\xl, \fq) \frac{e^{i k \left(\norm{\xv - \xm} + \norm{\xm - \xl}\right)}}{\norm{\xv - \xm} \norm{\xm - \xl}} \diff \xm.
    \label{eq:phasor_single_scattering}
\end{split}
\end{align}
%
For diffuse surfaces which are planar with respect to the illumination wavelength $\wl = \Omega^{-1}$, the newly generated phasors \NEW{$\Pf(\xv, \fq)$} result in a specular reflection of the incoming wavefront \NEW{from $\Pf(\xl, \fq)$}. In practice, this means that while the surface \NEW{$\lM$} may reflect visible light in all directions, the transient modulations \neww{(components with frequency $\fq > 0$)} that we need to \NEW{image} the scene propagate in the \textit{specular} direction of the reflected \new{computational} wave.

This is shown in \fref{fig:hf_reflection}a; we illustrate the capture process \new{(real domain)} of incoming light from $\xl$ \neww{that reaches points $\xm$ on a diffuse surface $\lM$} which, as the simulation on the right shows for points $\xv$ in the volume $\lV$, reflects light isotropically in all directions. \neww{In \fref{fig:hf_reflection}b, we illustrate the wave-based computational light transport of the same scene}, ignoring unmodulated light (frequency $\fq=0$) and considering only the part of the signal containing transient-modulated intensity (i.e., choosing a Fourier component with frequency $\Omega>0$). \neww{Also in \fref{fig:hf_reflection}b, the first schematic shows a planar light wavefront (red) at the time that it reaches the surface. The second schematic shows the reflected wavefront (cyan) \NEW{at a later time instant,} resulting from the superposition of spherical wavefronts (grey) at points $\xm$ in $\lM$. This} produces a specular reflection, as predicted by Huygens' principle, shown in the simulation on the right. Both simulations on the right of \fref{fig:hf_reflection} have been generated using Monte Carlo integration, using standard ray optics (\fref{fig:hf_reflection}a) and wave optics as described by \eref{eq:phasor_single_scattering} (\fref{fig:hf_reflection}b).


% Figure environment removed

\paragraph{\new{Infinity mirror experiment}}
\label{sec:recursivemirror}
To further illustrate how these specular reflections take place in a NLOS scenario, we set up a simple \new{simulated} scene made up of a diffuse hidden surface $\lM$ in front of the relay surface with aperture $\lS$\NEW{, at a distance $d$} (see \fref{fig:infinity_mirror}). \new{We illuminate a point $\xl$ on the relay surface using a laser device, and obtain the impulse response $H(\xl, \xs, t)$ at \NEW{points $\xs$ in} $\lS$ on the relay surface using transient rendering simulations~\cite{royo2022non, jarabo2014framework}.}
%
\neww{The relay surface (with aperture $\lS$) and $\lM$ are planar diffuse surfaces, thus behave like virtual mirrors in the computational wave domain. Looking at $\lM$, \NEW{
\NEWW{light emitted from $\xl$ is reflected by $\lM$, forming a mirror image of $\xl$ \emph{behind} $\lM$ like any conventional mirror}.
%
To capture in-focus images of the mirror reflection of $\xl$, we place the focal plane of the virtual camera \emph{behind} $\lM$.}
%
\NEWW{In particular,} we use the impulse response $H(\xl, \xs, t)$ and implement a transient camera model (\eref{eq:RSD_freq_tc}) to obtain $\ftc(\xv, t)$. We \neww{compute the \newww{time-resolved} image $\ftc(\xv, t)$} on the specific locations $\xv$ where the mirror images are formed. In \fref{fig:infinity_mirror}a\NEW{, we place the focal plane of the camera at a distance $2d$, on a plane $\lSp$ (green) which denotes}
%
the mirror image of $\lS$ \textit{behind} $\lM$. \NEW{The plane $\lSp$ also contains} the mirror image \NEW{at} $\xlp$ of $\xl$ produced by $\lM$\NEW{, captured in our result on the right as a bright spot.}}
%
\neww{
\newww{
As stated in \sref{sec:time-resolved-models}, each frame \NEW{at $t$ of $\ftc(\xv, t)$} combines points $\xv$ at different times \NEW{in the hidden scene}: the transient camera model captures events in the frame \NEW{at} $t$ equal to the time of flight from $\xl$ and the actual scene element corresponding to the mirror image at $\xlp$, which is also $\xl$ in this case. Consequently, the mirror image at $\xlp$ is captured in the frame at $t=0$.}}


% Figure environment removed

We can image a higher-order mirror reflection pushing this effect even further, as shown in \neww{\fref{fig:infinity_mirror}b}. Since the diffuse relay surface that contains the aperture $\lS$ also behaves like a mirror in the computational wave domain, \neww{the specular interactions between $\lM$ and $\lS$ create additional mirror reflections of the \NEW{illuminated point} $\xl$ at locations further behind $\lM$}. 
%
\neww{This effect is analogous to the real situation where we observe multiple reflections of an object placed between two confronted, real mirrors. We showcase this effect in \fref{fig:infinity_mirror}b, using the same impulse response $H(\xl, \xs, t)$ and implementing the same transient camera model (\eref{eq:RSD_freq_tc}), but 
%
\newww{adjusting the focal \NEW{plane at a distance $4d$} to match}
%
points $\xv$ in the \NEW{plane} $\lSpp$. Looking at the computed image on the right, this yields a clear but dimmer spot that corresponds to the second mirror image at $\xlpp$ of the illuminated point $\xl$.}


\new{\paragraph{Computational wavelength}
The computational specular behavior of a real diffuse surface is explained by wave optics (illustrated in \fref{fig:hf_reflection}), and depends on the computational wavelengths $\lambda$ used through the imaging process ($\lambda$ is the inverse of the imaging frequencies $\fq$).
%
In practice, the computational wavelengths used through the imaging process depend on the frequency spectrum of the illumination function $\Pf(\xl,\fq)$. In our work, we determine the spectrum of frequencies $\fq$ of the imaging process through the central wavelength $\lambda_c$ and standard deviation $\sigma$ in \eref{eq:virtual_illumination}. 
The choice of these frequencies introduces a trade-off\neww{: lower values of $\lambda_c$ and $\sigma$ can properly image geometric features with more detail, but they may} also introduce unwanted high-frequency noise.
%
Following previous works, we use values for $\wl_c$ from $\unit{3}{\centi\meter}$ to $\unit{14}{\centi\meter}$, and values for $\sigma$ proportional to $\wl_c$. We specify the particular values used in each experiment. For reference, all experiments share the same aperture size of $\unit{2 \times 2}{\meter}$. Within this range of values, planar surfaces behave specularly during the NLOS imaging process, which we leverage to address different challenges of NLOS imaging methods.  We do not use wavelengths larger than this range, \NEW{as these may degrade the specular behavior of surfaces} due to the ratio between the surface size and the wavelength.}


% Figure environment removed

\new{\paragraph{Missing-cone problem through virtual mirrors} The frequency-space specular behavior of diffuse surfaces allows us to intuitively explain the missing-cone problem from \neww{our virtual mirrors} perspective:}
\neww{for a point $\xv$ in the hidden scene, if light from $\xl$ does not reach any point \NEW{$\xs$} in the aperture $\lS$ after a specular reflection in $\xv$, then the point $\xv$ is inside the null-reconstruction space of third-bounce imaging methods and cannot be reconstructed.}
%
\new{\fref{fig:missing-cone} \neww{illustrates this for third-bounce methods} \NEW{using} a scene with three diffuse surfaces $\lM_{1-3}$. However, only two \NEW{surfaces} are visible on the image shown in \fref{fig:missing-cone}d, computed using existing third-bounce NLOS imaging methods. Wave propagation in the computational NLOS imaging domain is illustrated in \fref{fig:missing-cone}a to \fref{fig:missing-cone}c for $\lM_1$ to $\lM_3$, respectively.} The reflected wavefronts from surfaces $\lM_1$ and $\lM_2$ reach the sensor $\lS$. However, given its particular position and orientation, this is not the case for $\lM_3$ and thus cannot be imaged. The surface $\lM_3$ is said to be \new{inside the null-reconstruction space}. 


\new{Even if a surface is inside the null-reconstruction space \neww{of third-bounce imaging methods} (such as $\lM_3$ in \fref{fig:missing-cone}), we observe that the combination of several surfaces may produce higher-order illumination bounces that actually do reach $\lS$.}
%
\new{In \sref{sec:contribution-2:goal-2}, we show how to leverage such higher-order illumination to infer and to directly image objects that \new{are inside} the null-reconstruction space of third-bounce methods. We achieve this by analyzing mirror images of other objects produced by the surface inside the null-reconstruction space to infer its position and orientation. Beyond inference, we provide a procedure to translate our imaging system to a secondary surface that directly observes the target object.}


Moreover, in \sref{sec:contribution-2:goal-1} we show another way to leverage virtual mirrors in NLOS imaging, and image objects hidden behind \emph{two} corners using existing imaging models. For this we rely on fifth-bounce specular reflections in the computational domain to image the space behind \NEW{virtual mirror} surfaces, where the mirror image of such objects would appear.