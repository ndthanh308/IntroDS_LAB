\section{Discussion and future work}
\label{sec:discussion}

We have established a connection between the surface reflectance defined by well-known wave propagation principles and a wave-based NLOS imaging formulation, showing how diffuse planar surfaces become virtual mirrors at NLOS imaging wavelengths. We have then introduced \new{\newww{a procedure} to address the missing-cone problem}: by analyzing mirror images produced by other \NEW{virtual mirrors}, and then showing how to directly \NEW{image} such surfaces using secondary apertures. \new{Moreover, our insights \neww{have allowed} us} to image objects hidden behind two corners by imaging the space behind \NEW{virtual mirrors}, where we \neww{have observed} mirror images of objects hidden around two corners.

\paragraph{\neww{Mirror reflections under existing imaging methods}}
\neww{
In our work, we have showed how to image mirror reflections of different scene elements to address current limitations of NLOS imaging methods. 
%
To image objects around two corners, our key idea is to reason about the location of the imaged volume based on our analysis of mirror-like behavior of planar surfaces in the computational domain.
%
The evaluation of the confocal camera model at $t=0$ (which we use to obtain the image of the object hidden behind two corners) is equivalent to third-bounce imaging used by existing single-corner NLOS imaging methods.  
%
We have described our procedure using the wave-based phasor-field formulation of such imaging model.
%
Nevertheless, our methodology and observations could, in principle, generalize to existing single-corner methods that use similar models in order to extend them to two-corner scenes, providing an interesting path for future research.
}

\paragraph{\new{Fourth-bounce assumptions and higher-order bounces}}
%
\new{To address the missing-cone problem, we have used fourth-bounce illumination paths. Looking at \fref{fig:goal2-overview}a, \neww{the scene} has to meet two conditions to be able to compute an image of $\lG$, which is inside the null-reconstruction space of third-bounce methods.
%
First, the hidden scene must contain another surface that is not in the null-reconstruction space for third-bounce methods. In our scene, the surface $\lM$ has this purpose, which then can be used as a secondary aperture.
%
Second, there must exist a \NEW{four}-bounce path that reaches both the \neww{surface $\lM$ where the secondary aperture is located,} and the target surface $\lG$. This fourth bounce must be able to reach $\lS$ when following the specular bounce direction in the computational NLOS imaging domain, else both surfaces would be in the null-reconstruction space of fourth-bounce methods too.
%
Note that imaging surfaces with third-bounce illumination already requires similar assumptions. This could in principle generalize to fifth- or even higher-order bounces, allowing to create additional \neww{higher-order} apertures to observe further into hidden scenes.
%
A more thorough exploration of the potential of these \neww{higher-order} apertures is thus an interesting avenue of future work.}

\paragraph{\new{Fifth-bounce assumptions and multiple virtual mirrors.}}
%
\new{\neww{We have demonstrated how to image a surface $\lG$ hidden around two corners from fifth-bounce illumination, using a diffuse surface $\lM$ as a virtual mirror. For this to work, light from the illuminated point $\xl$ has to follow specular paths in the computational domain} that must reach $\lG$ after one bounce on $\lM$, and must reflect back to the aperture $\lS$ after another bounce on $\lM$, yielding a five-bounce path $\xl \rarr \lM \rarr \lG \rarr \lM \rarr \lS$. Thus, imaging $\lG$ depends on the location and orientation of $\lM$ with respect to $\lG$ and $\lS$. Note that this is no different from classic third-bounce NLOS setups, where objects must be located and oriented in regions outside of the null-reconstruction space to be imaged.
%
Our method could in principle generalize to more cluttered scenarios, where specular reflections between different planar surfaces would increase the coverage of NLOS imaging. To explore this, an exhaustive analysis of the connection between imaging wavelength, surface size and features, and their reflectance properties at different imaging frequencies would be necessary.}


\paragraph{\new{Coverage of the missing-cone problem using higher-order bounces}}
%
\new{In our work we have demonstrated how to image surfaces inside the null-reconstruction space of third-bounce methods. However, as can be seen in \fref{fig:goal2-results} (simulated) and \fref{fig:goal2-real} (captured), only a part of the surface $\lG$ reflects fourth-bounce illumination towards $\lS$ in the computational domain. Thus, some parts of $\lG$ remain inside the null-reconstruction space of our fourth-bounce imaging procedure. The third-bounce analysis of the missing-cone problem by \citet{Liu2019analysis} concludes that the visibility of a point in the scene only depends on the position of points on the visible relay surface. This is not the case for the null-reconstruction space of higher-order imaging methods, where the visibility of a point in the scene also depends on other hidden scene elements. Thus, a thorough analysis of the coverage of the missing-cone is an open challenging contribution in NLOS imaging regarded as future work.}


\paragraph{Mirror behavior} While our experiments showed that diffuse planar surfaces produce specular reflections at NLOS imaging frequencies, these reflections do not follow exactly a delta function due to diffraction effects. This happens likely because the \new{surfaces} we use are not much larger than the wavelength of the computational wave. Image quality therefore depends significantly on the size and position of the mirror surface. In our experimental results\new{, we observed that reflections through such diffuse surfaces appear blurry, mainly due to diffraction artifacts.}
%
\new{
%
\neww{Finally, some higher-order paths may have the same time of flight as third-bounce paths, and thus are coupled in the impulse response $H(\xl, \xs, t)$ introducing undesired artifacts in the imaging process, which is the case already for all existing NLOS imaging methods.
%
To what extent diffraction, imperfect mirror behavior, noise, and coupling between bounces enter this problem is an interesting topic for future research.}}

\new{In conclusion, our virtual mirrors framework addresses two of the most limiting problems of current NLOS imaging algorithms, leveraging fourth- and fifth-bounce illumination to compute images of surfaces inside the null-reconstruction space of existing methods, and even hidden behind two corners. We hope that our work spurs further research in this direction to explore the full potential of the field.}



