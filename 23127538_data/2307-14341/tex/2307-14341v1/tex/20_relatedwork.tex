\section{Related work}
\label{sec:related-work}

\new{NLOS imaging methods \neww{analyze indirect illumination from paths that scatter one or multiple times in the target hidden scene to obtain information about it}. They can be divided into active and passive methods. Active methods \neww{use controlled light sources to illuminate} the hidden scene~\cite{velten2013femto, katz2014non, cao2022high, luesia2022non}, while passive methods rely on ambient illumination~\cite{bouman2017turning, krska2022double} \neww{or light emitted by hidden objects themselves}~\cite{saunders2019computational}. In our work, an active laser source emits light pulses to illuminate the hidden scene.}

\paragraph{Time-gated NLOS imaging}
The ability to capture time-gated illumination using time-of-flight detectors at picosecond resolution has nurtured a wide range of NLOS imaging methods ~\cite{Faccio2020non,maeda2019recent,Jarabo2017transient,pediredla2019snlos}. The first methods to demonstrate high-quality 3D reconstructions employed ellipsoidal backprojection \cite{Velten2012, buttafava2015non} acquired for non-confocal optical paths.
These methods operate on the time domain of the captured light transport and are computationally expensive. 
By restricting data acquisition to confocal optical light paths, NLOS reconstructions can be formulated as a closed-form, deconvolution-based linear inverse problem which can be solved efficiently in the frequency domain \cite{Lindell2019wave, OToole2018confocal}. Since they are based on frequency-space deconvolutions, they require the use of a regular sampling grid on a planar relay \neww{surface}. Furthermore, in contrast to our work, both ellipsoidal-based and deconvolution-based techniques are unable to account for light bounces beyond the third. 


\paragraph{Phasor-field formulation}
The phasor-field \neww{formulation \cite{Liu2019phasor} provides wave-based models to propagate} virtual waves into the hidden scene, which \neww{allows to turn} a visible relay surface into a virtual LOS camera. This allows to build fundamental wave-optics parallelisms between forward operators in LOS imaging and the backprojection operators that drive time-gated NLOS imaging approaches~\cite{rezaPhasorFieldWaves2019, rezaPhasorFieldWaves2019a, doveParaxialPhasorfieldPhysical2020, dove2020nonparaxial, dove2020speckled, teichmanPhasorFieldWaves2019, Guillen2020Effect, laurenzis2022time}. 
%
The phasor-field formalism allows to image hidden scenes using either confocal or non-confocal setups~\cite{Liu2019phasor}. Subsequent implementations have gained efficiency by working in the frequency domain \cite{Liu2020phasor}, which has led to interactive and real-time reconstructions of dynamic hidden scenes~\cite{nam2020real,liao2021fpga} at the cost of using a regular sampling grid. Moreover, it allows to use both planar and non-planar relay surfaces~\cite{manna_non-line--sight-imaging_2020}, and to leverage known occlusions in the reconstructions ~\cite{dove2019paraxial}. For memory-constrained applications, different propagation operators can be implemented, such as zone plates \cite{luesia2023zone}. 
%Fifth this formalism models light transport with multiple bounces in the hidden scene. 
Last, \citet{Marco2021NLOSvLTM} \NEW{leveraged the phasor-field formulation} to separate direct and indirect illumination of hidden scenes by combining exhaustive scans of both laser and sensor positions. In this work, we build on top of the virtual-wave LOS parallelisms and reason about the virtual reflectivity of hidden surfaces. We then leverage higher-order illumination to image geometry \neww{hidden around two corners}, as well as to \neww{directly} estimate hidden objects that \neww{have limited visibility to} classic third-bounce methods.


\paragraph{NLOS with specular reflections} 
Prior works utilized \new{actual} specular reflections by using centimeter-scale acoustic waves for NLOS reconstructions \cite{lindell2019acoustic}, or millimeter-scale radio waves for tracking hidden objects \cite{scheiner_seeing_2020}. Using specular reflections requires directly sampling the path from the hidden scene to the relay surface, needing strong assumptions about surface albedo and orientation~\cite{lindell2019acoustic}, or placing the scanning system far from the relay surface~\cite{scheiner_seeing_2020}. 
Specular reflections produced by infrared wavelenghts have also been utilized for \neww{passive} NLOS~\cite{maeda2019thermal, kaga_thermal_2019}.
These systems are also restricted to reconstructions of planar scenes or to object tracking inside the hidden scene. In our work we observe that, when computationally transforming the temporal profile of diffuse surfaces into the frequency domain (following the phasor-field model~\cite{Liu2019phasor}), they exhibit mirror-like reflectance properties. We leverage higher-order illumination bounces produced by these virtual mirrors to overcome classical NLOS visibility challenges and to look around an additional second corner.
