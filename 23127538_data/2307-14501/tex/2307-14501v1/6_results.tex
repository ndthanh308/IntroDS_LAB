\section{Experimental Results}
\label{sec:results} 
%\todo{I removed the word environment after each of them to save on space and hbox badness}
We conduct simulated experiments in three environments---our \emph{J-Intersection} (Sec.~\ref{sec:results:jint}), \emph{Parallel Hallway} (Sec.~\ref{sec:results:hallways}), and \emph{University Building} (Sec.~\ref{sec:results:office})---in which a robot must navigate to a point goal in unseen space.
For each trial, we evaluate performance of 4 planners:
\begin{LaTeXdescription}
\item[Non-Learned Baseline] Optimistically assumes the unseen space to be free and plans via grid-based A$^{\!*}$ search.
\item[LSP-Local (learned baseline)] Plans via Eq.~\eqref{eq:lsp-planning}, estimating subgoal properties via only local features, as in \cite{pmlr-v87-stein18a}.
\item[LSP-GNN (ours)] Plans via Eq.~\eqref{eq:lsp-planning}, yet uses our graph neural network learning backend to estimate subgoal properties using both local and non-local features.
\item[Fully-Known Planner] The robot uses the fully-known map to navigate; a lower bound on cost.
\end{LaTeXdescription}
For each planner, we compute average navigation cost across many (at least 100) random maps from each environment.

\begin{table}[t]
    \begin{center}
    \caption{Avg. Cost over 100 Trials in the J-Intersection Environment}
    \label{table:toy-stats}
        \begin{tabular}{cc}
        \toprule
            \textbf{Planner} & \textbf{Avg. Cost (grid cell units)} \\
            \hline     
            Non-Learned Baseline & $303.03$\\        
            LSP-Local (learned baseline) & $323.46$\\      
            LSP-GNN (ours) & \textbf{204.85}\\
            \hline
            Fully-Known Planner & $204.85$\\   
        \bottomrule
        \end{tabular}
    \end{center}
\end{table}

\subsection{J-Intersection Environment}\label{sec:results:jint}
% Figure environment removed

% To demonstrate the utility of our approach in a simple
We first show results in the J-Intersection environment, described in Sec.~\ref{sec:example-case} to motivate the importance of non-local information for good performance for navigation under uncertainty.
In this environment, the robot must choose where to travel at a fork in the road, yet non-locally observable information is needed to reliably make the correct choice---a blue-colored starting region indicates that the goal can be reached by turning towards the blue hallway at the intersection, and the same for the red-colored regions. We randomly mirror the environment so that the robot cannot learn a systematic policy that quickly reaches the goal without understanding.

We conduct 100 trials for each planner in this environment to evaluate their performance and show the average cost planning strategy in Table~\ref{table:toy-stats}.
Across all trials, our proposed LSP-GNN planner \emph{always} correctly decides where to go at the intersection and achieves near-perfect performance.
By contrast, both the LSP-Local and Non-Learned Baseline planners lack the knowledge to determine which is the correct way to go and perform poorly overall, resulting in poor performance in roughly half of the trials.
We highlight two example trials in Fig.~\ref{fig:example-case-results}.
We do not report the prediction accuracy empirically, because the prediction accuracy does not reflect the actual gain in performance for our work.

\subsection{The Parallel Hallway Environment}\label{sec:results:hallways}
% Figure environment removed

Our \emph{Parallel Hallway} environment (Fig.~\ref{fig:sample-hallway}) consists of parallel hallways connected by rooms.
We procedurally generate maps in this environment with three hallways and two room types: (i) \emph{dead-end} rooms and (ii) \emph{passage} rooms that provide connections between neighboring parallel hallways.
Only one passage room exists between a pair of hallways, and so the robot must identify this room if it is to travel to another hallway.
Environments are generated such that the dead-end rooms all have the same color (red or blue) distinct from the color of the passage rooms, which are thus blue or red, respectively.
We are making the environment such that the relational information, such as recognizing that if a room with certain color is explored as a dead-end, then the other colored room serves as a pass-through room can be learned. 
If the colors were entirely random, there would be no way to make predictions about the unseen space.
Both room types contain obstructions and are otherwise identical, so that it is not possible to tell whether or not a room will connect to a parallel hallway without trial-and-error or by utilizing semantic color information from elsewhere in the map.
Rooms are placed far enough apart that the robot cannot determine from the local observations if a room will lead to the next hallway or will be a dead end.
The start and goal locations are placed in separate hallways, so as to force the robot to understand its surroundings to reach the goal quickly.
Thus, to navigate well in this challenging procedurally-generated environment, the robot must first explore, trying nearby rooms to determine which color belongs to which room type, and then retain this information to inform navigation through the rest of the environment.


\begin{table}[t]
    \begin{center}  
    \caption{Avg. Cost over 500 Trials in the Parallel Hallway Environment}\label{table:hallway-stats}
        \begin{tabular}{cc}
            \toprule
            \textbf{Planner} & \textbf{Avg. Cost  (grid cell units)}\\
            \hline
            Non-Learned Baseline & $205.93$\\
            LSP-Local (learned baseline) & $236.47$\\
            LSP-GNN (ours) & \textbf{141.37}\\
            \hline
            Fully-Known Planner & $108.37$\\
            \bottomrule
        \end{tabular}
    \end{center} 
\end{table}


% Figure environment removed

We train the simulated robot on data from 2,000 distinct procedurally generated maps and evaluate in a separate set of 500 distinct procedurally generated maps. We show the average performance of each planning strategy in Table~\ref{table:hallway-stats} and include scatterplots of the relative performance of different planners for each trial in Fig.~\ref{fig:scatter-plot-hallway}.
The robot planning with our LSP-GNN approach is able to utilize non-local local information to improve its predictions about how best to reach the goal, achieving a 31.3\% improvement in average cost versus the optimistic Non-Learned Baseline planner and a 40.2\% improvement over the LSP-Local planner. In addition, our approach is \emph{reliable}: owing to the LSP planning abstraction, our robot is able to successfully reach the goal in all maps.

% Our approach, by retaining non-local information and using it to make predictions, improves average performance over the baselines.
% In addition, our approach is \emph{reliable}: owing to the LSP planning abstraction, our robot is able to successfully reach the goal in all maps.
% Both the non-learned baseline and the LSP-Local planner cannot take advantage of task-relevant non-local information and so perform poorly.

We highlight one trial in Fig.~\ref{fig:result-hall-path}, in which the robot is tasked to navigate from the top hallway to the bottom hallway, which contains the goal.
After a brief period of trial-and-error exploration in the first (top) hallway, the robot discovers the passage to the neighboring hall and uses the knowledge of the semantic color to quickly locate the passage to the next hallway and reach the goal.
% Our robot quickly discovers the passage to the neighboring hall, 
% Our robot explores very little in the first hallway (top) before discovering the passage to the neighboring hall and uses that knowledge to quickly locate and find the next passage room and reach the goal.
By contrast, the Non-Learned Baseline optimistically assumes unseen space to be free and enters every room in the direction of the goal.
The LSP-Local planner makes predictions using only local information and, unable to use important navigation-relevant information, cannot determine how to reach the goal; its poor predictions result in frequent turning-back behavior as it seeks alternate routes to the goal, reducing performance.

% Figure environment removed



\subsection{University Building Floorplans}\label{sec:results:office}
% Figure environment removed
Finally, we evaluate in large-scale maps generated from real-world floorplans of buildings from the Massachusetts Institute of Technology, including buildings of over 100 meters in extent along either side; see Fig.~\ref{fig:mit-train-example-map} for examples.
We generate data from 2,000 trials across 56 training floorplans and evaluate in 250 trials from 9 held-out test floorplans, each augmented by procedurally generated clutter to add furniture-like obstacles to rooms.
% The training and testing maps are sampled from distinct building.
In addition to occupancy information, \emph{rooms} in the map have a distinct semantic class from \emph{hallways} (and other large or accessible spaces); this semantic information is provided as input node features to the neural networks to inform their predictions.

% These floorplans have rooms
% The MIT building floorplans, which were employed to train our robot, encompass not only diverse shaped configurations, but also comprise hallways that are disconnected and accessible only through passthrough rooms.
% Additionally, there are rooms that can be reached solely via other rooms and not through hallways, and sometimes they can be accessed through both hallways and other rooms. 
% These features of the floorplans present distinct challenges to any robot, requiring it to navigate through complex environments with multiple possible paths to reach its destination. 
% The incorporation of these various elements in the training process enables our robot to learn how to operate efficiently and effectively in practical settings with diverse layouts and structures.
% We train our simulated robot in 2000 maps where we randomly sample the maps from a pool of 320 maps.
% Next, we evaluate the performance on 250 maps again randomly sampled from a different pool of 160 maps (that the robot has not seen during training).
% In both cases, we sample the maps with random initialisation of the robot start pose and goal pose.
% So, the agent always has a new trajectory to follow.

We show the average performance of each planning strategy in Table~\ref{table:office-stats} and include scatterplots of the relative performance of different planners for each trial in Fig.~\ref{fig:scatter-plot-office}.
The robot planning with our LSP-GNN approach achieves improvements in average cost of 9.3\% versus the optimistic Non-Learned Baseline planner and of 14.9\% improvement over the LSP-Local Learned Baseline planner. 
Unlike the LSP-Local planner, which does not have enough information to make good predictions about unseen space, our LSP-GNN approach can make use of non-local information to inform its predictions and thus performs well despite the complexity inherent in these large-scale testing environments. 

% The robot planning with our LSP-GNN approach is able to utilize non-local local information to improve its predictions about how best to reach the goal and achieves a 9.3\% improvement in average cost versus the optimistic Non-Learned Baseline planner and a 14.9\% improvement over the LSP-Local planner.

\begin{table}[t]
    \begin{center}  
    \caption{Avg. Cost over 250 Trials in the University Building Floorplans}\label{table:office-stats}
        \begin{tabular}{cc}
            \toprule
            \textbf{Planner} & \textbf{Avg. Cost (meter)}\\
            \hline
            Non-Learned Baseline & $44.98$\\
            LSP-Local (learned baseline) & $47.93$\\
            LSP-GNN (ours) & \textbf{40.80}\\
            \hline
            Fully-Known Planner & $31.77$\\
            \bottomrule
        \end{tabular}
    \end{center} 
\end{table}

% Figure environment removed

% Figure environment removed

% Figure environment removed

Fig.~\ref{fig:result-office-good} shows a typical navigation example in one of our test environments. 
In this scenario, the shortest possible trajectory involves knowing to follow hallways until near to the goal.
Both learned planners generally exhibit hallway-following behavior---often useful in building-like environments such as these---and improve upon the non-learned (optimistic) baseline. However, our LSP-GNN planner, able to make use of non-local information, can more reliably determine which is the more productive route and more quickly reaches the faraway goal.
Fig.~\ref{fig:result-office-bad} shows two additional examples that highlight the improvements of our LSP-GNN planner made possible by non-locally-available information. In Fig.~\ref{fig:result-office-bad}A, we highlight an example in which both learned planners cannot immediately find the correct path, yet LSP-GNN is able to improve its predictions about where is most likely to lead to the unseen goal and recover more quickly than does LSP-Local. Fig.~\ref{fig:result-office-bad}B shows a more extreme example, in which the LSP-Local planner fails to quickly turn back to seek a promising alternate route immediately identified by LSP-GNN.



% In Fig.~\ref{fig:result-office-bad} we illustrate two more maps, comparing LSP-GNN against LSP-Local.
% We show the ability for quick recovery of our learned planner from going off course. %against the non-learned baseline.
% Since the non-learned baseline uses only local information it struggles to make good predictions and keeps on going off course.
% Despite going off course the same way as LSP-Local in the beginning, LSP-GNN uses non-local information to get prediction that enables it to
%Whereas LSP-GNN utilizes the non-local information going off course and uses that to guide itself and 
% course correct quickly resulting in shorter trajectories.

% Fig.~\ref{fig:result-office-good} and~\ref{fig:result-office-bad} illustrate planning outcomes over two separate floorplans.
% LSP-GNN typically employs hallways as the primary means of reaching distant destinations. 
% It will only venture into rooms located far from its goal if it perceives them to be passthrough rooms.
% As it approaches the goal, it will enter rooms in an attempt to locate the objective. 
% In contrast, both the non-learned planner and the LSP-Local planner struggle to accurately recognize potential passthrough rooms and frequently deviate from the hallway. 
% This often results in suboptimal routes and increased difficulty in reaching the desired location.

% The LSP-GNN planner adheres to the assumption that hallways typically provide the most direct route to distant destinations. 
% Although this approach resulted in higher costs than the other planners in this particular case, the LSP-GNN planner acted in a rational manner based on its prior belief. 
% By sticking to the hallway, it employed a logical and systematic approach to navigating the environment, even though it did not lead to the desired outcome in this specific instance.