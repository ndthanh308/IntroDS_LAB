\section{Related Works}\label{sec:related-works}

\textbf{Planning under Uncertainty}
POMDPs~\cite{kaelbling1998, littman1997, thrun2005,parascandolo-DCMCTS} have been used to represent navigation and exploration tasks under uncertainty, yet direct solution of the model implicit in the POMDP is often computationally infeasible.
To mitigate this limitation, many approaches to planning rely on learning to inform behavior~\cite{pfeiffer2016, richter2014,ross2012}, yet only plan a few time steps into the future and so are not well-suited to long-horizon planning problems.
Some reinforcement learning approaches that deal with partially observed environments~\cite{DuanSCBSA16, yang2021,gupta2019cognitive, zhang2017deep, TaiPL17, MirowskiPVSBBDG16} are also limited to fairly small-scale environments.
The MERLIN agent~\cite{MERLIN2018_Greg_Wayne} uses a differentiable neural computer to recall information over much longer time horizons than is typically possible for end-to-end-trained model-free deep reinforcement learning systems.
However, the reinforcement learning approaches~\cite{MERLIN2018_Greg_Wayne,Kober2014, henderson2017} can be difficult to train and lacks plan completeness, making it somewhat brittle in practice.
Our proposed work improves long-horizon planning under uncertainty learning the relational properties from the non-local observation of the environment with the guarantee of completeness.

\textbf{Graph Neural Networks and Planning}
% [TODO: Section on GNNs (and maybe GNNs for planning).]
Battaglia et al.~\cite{peter2018} present a survey of GNN approaches, demonstrating how GNNs can be used for relational reasoning and exhibit combinatorial generalization, opening numerous opportunities for learning over structured and relational data.
% , and showing how to use it for relational reasoning and combinatorial generalization.
% Learning relation properties from different observations becomes easier using GNNs.
% Learning relation properties from different observations is something that was not possible before GNNs (hmmmm; no).
Zhou et al.~\cite{zhou2018} show how GNNs have been used in the field of modeling physics systems, learning molecular fingerprints, predicting protein interface, classifying diseases, and many others.
GNNs are fast to evaluate on sparse graphs and have shown capacity to generalize effectively in multiple domains \cite{peter2018,duvenaud2015gcnmolecule,monti2017geometric}.
Moreover, GNNs have recently been used to accelerate task and motion planning~\cite{kim2019guidedtamp,kim2021representation} and to inform other problems of interest to robotics: joint mapping and navigation \cite{chen2020gnnexplore}, object search in previously-seen environments~\cite{kurenkov2021semantic}, and modeling physical interaction~\cite{kossen2020structured}.
% \textbf{[You need to talk about GNNs in general a bit. Give some context and cite some foundational GNN papers (even if they are not in planning). Remember it is an insight of yours that GNNs are useful here, so it makes sense that there will not be many papers for your specific area.]}
%Kurenkov et al.~\cite{kurenkov2020} utilized GNN to learn relational properties in 3D Scene Graphs for Hierarchical Mechanical Search.
In particular, Chen et al.~\cite{fanfei2021} propose a framework that uses GNN in conjunction with deep reinforcement learning to address the problem of autonomous exploration under localization uncertainty for a mobile robot with 3D range sensing.
% Another approach~\citep{fanfei2021} presents a framework using GNN for self-learning a high-performance exploration policy under uncertainty in a single simulation environment that is transferable to either physical or virtual environment.
% TODO: http://ai.stanford.edu/mech-search/pdfs/2020_icra_semantic_search.pdf 