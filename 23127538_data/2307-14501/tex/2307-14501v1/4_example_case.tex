\section{Motivating Example: A Memory Test for Navigation}
\label{sec:example-case}
% Figure environment removed

Fig.~\ref{fig:example-case-j-shaped} shows an example scenario motivating the necessity of using non-locally observable information to make good predictions about the environment while trying to reach the goal under uncertainty.
Our \emph{J-Intersection} environment has either a red or blue square region inside of it and around the corner occluded from that square region far away at the intersection that colored region leads to the goal (bottom).

Maps in this environment are structured so that the color of the hallway the robot should follow matches the color of the center region of the map.
We randomize the color of the center map region and mirror the environment randomly so that no systematic policy (e.g., \emph{follow the blue hallway} or \emph{turn left at the fork}) will efficiently reach the goal.

Since the LSP approach is limited to making predictions for the subgoal using only locally observable information, it cannot to learn the (simple) defining structural characteristic of the environment: if the inside square region is red then the path to the goal is red and if the inside square is blue then blue is the path to the goal.
Instead, we will augment the LSP approach to rely on a \emph{graph neural network}~\cite{peter2018} to estimate the subgoal properties, allowing it to use both local and non-local information to make predictions about the goodness of actions that enter unseen space and thus perform well across a variety of complex environments.