\section{Introduction}
We focus on the task of goal-directed navigation in a partially-mapped environment, in which a robot is expected to reach an unseen goal in minimum expected time.
Often modeled as a Partially Observable Markov Decision Process (POMDP)~\cite{kaelbling1998}, long-horizon navigation under uncertainty is computationally demanding, and so many strategies turn to learning to make predictions about unseen space and thereby inform good behavior.
To perform well, a robot must understand how parts of the environment the robot cannot currently see (i.e., non-locally available information) inform where it should go next, a challenging problem for many existing planning strategies that rely on learning.

Consider the simple scenario from our \emph{J-Intersection} environment shown in Fig.~\ref{fig:intro-explain}: information at the center of the map (the color of that region) informs whether the robot should travel left or right; optimal behavior involves following the hallway whose color matches that of the center of the map.
As this color is not visible from the intersection, a robot must remember what the space looked like around the corner to perform well and learn how that information relates to its decision.
More generally, many real-world environments require such understanding, a particularly challenging task for building-scale environments. In this work, we aim to allow a robot to retain non-local knowledge and learn to use it to make predictions that inform where it should travel next.


% one location of the environment tells information about another part that is not readily observable from the previous location.
% The environment is such that the color that can be seen at the tip of J is the same as the color from the fork to the goal.
% So remembering what color the robot has seen helps to take accurate action at the fork.
% Where LSP takes only independent observation as input and predicts using local observation, we make predictions over the graph of observations, enabling access to non-local observation.
% We do so to learn the interdependent relationships of the environment.

%To perform well, a robot must understand where it should navigate next to best make progress towards the goal, a challenging problem requiring that the robot [understand how parts of the environment it cannot current see (i.e., non-locally available information) inform where it should go next]
%a challenging problem for many existing planning strategies that rely on learning.

Recently, learning-driven approaches---including many model-free approaches trained via deep reinforcement learning~\cite{MERLIN2018_Greg_Wayne, mirowski2018}---have demonstrated the capacity to perform well in this domain.
However, in the absence of an explicit map for the robot to use to keep track of where it has yet to go, many such approaches are unreliable, lacking guarantees that they will reach the goal~\cite{pfeiffer2016}.
Moreover, these approaches struggle to reason far enough into the future to understand the impact of their actions and thus perform poorly and can be brittle and unreliable for long-horizon planning.

% Learning-dominated approaches, despite the fact that they are capable of relating an entire history of sensor observations to the decisions they should make, struggle to do so reliably while reasoning far ahead into the future.

The recent Learning over Subgoals planning approach (LSP)~\cite{pmlr-v87-stein18a} introduces a high-level abstraction for planning in a partial map that allows for both state-of-the-art performance and reliability-by-design.
In LSP, actions correspond to exploration of a particular region of unseen space.
Learning (via a fully-connected neural network) is used to estimate the goodness of exploratory actions, including the likelihood an exploration will reveal the unseen goal.
These predictions inform model-based planning and are thus used to compute expected cost.
% and a learning-informed model-based planning abstraction
% unknown spaces and is a model-based approach for planning.
LSP overcomes two problems: (1) its state and action abstraction allows for learning-informed reasoning far into the future and (2) it is guaranteed to reach the goal if there exists a viable path.
However, LSP is limited: its ability to make predictions about unseen space only makes use of locally observable information, limiting its performance.
%on the J-Intersection environment (Fig.~\ref{fig:example-case-j-shaped}) and beyond.

% Figure environment removed

In this paper, we extend the Learning over Subgoals Planner (LSP-Local), replacing its learning backend with a Graph Neural Network (LSP-GNN), affording reliable learning-informed planning capable of using both local and non-local information to make predictions about unseen space and thus improve performance in complex navigation scenarios in building-scale environments.
% In this paper, we will use a Graph Neural Network to overcome this limitation of the Learning over Subgoals Planner (LSP), augmenting its learning backend to allow for reliable learning-informed planning capable of using both local and non-local information to make predictions about unseen space, and thus improve performance in complex navigation tasks.
Using a graph representation of the partial map---constructed via a map skeleton~\cite{zhang1984} so as to preserve topological structure---we demonstrate that our GNN allows for accurate predictions of unseen space using non-local information.
% We will extend the Learning over Subgoals Planner so as to leverage our GNN to make predictions about unseen space.
Additionally, we demonstrate that our LSP-GNN planner improves performance over the original LSP-Local planner while retaining guarantees on reliability: i.e., the robot always reaches the goal.
We show the effectiveness of our approach in our simulated \emph{J-Intersection}, \emph{Parallel Hallway}, and \emph{University Building} environments, in the latter yielding improvements of 9.3\% and 14.9\% (respectively) over non-learned and learned baselines. 
% \todo{update to mention the new floorplan environments.}
