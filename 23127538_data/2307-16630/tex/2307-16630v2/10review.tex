\section{Meta-Review}

\vspace{-2mm}
\subsection{Summary}
\vspace{-2mm}
This paper presents the first generalized framework for certifying the robustness of text classification models against textual adversarial attacks. The authors specifically tackle the vulnerability of language models to such attacks and demonstrate the effectiveness of their framework on a range of models and attack scenarios.

\vspace{-2mm}
\subsection{Scientific Contributions}
\vspace{-2mm}
\begin{itemize}
\item Addresses a Long-Known Issue
\item Provides a Valuable Step Forward in an Established Field
\end{itemize}

\vspace{-2mm}
\subsection{Reasons for Acceptance}
\vspace{-2mm}
\begin{enumerate}
\item The paper addresses a long-known issue: The authors address that previous approaches to certified robustness against word-level attacks have been constrained to synonym substitution, while widely utilized attacks rely on word reordering, insertion, or deletion.

\item The paper provides a valuable step forward in an established field. The paper highlights the limitations of currently certified robustness and emphasizes the importance of provable robustness guarantees. To address these concerns, the authors propose a generalized framework that offers guarantees against all four classes of word-level textual adversarial attacks.

\item The paper creates a new tool to enable future science. The authors propose a training toolkit designed to enhance the robustness of language models, which has the potential to inspire and facilitate future research.
\end{enumerate}

% \vspace{-2mm}
% \subsection{Noteworthy Concerns} % Exclude if your meta-review does not have noteworthy concerns
% \vspace{-2mm}
% It would be better if the authors give a clearer description of the threat model and a detailed setup of the training toolkit used in the evaluation.

% \begin{enumerate} % Enumerate environment is not necessary if there is only one
% \item 
% \end{enumerate}


% \section{Response to the Meta-Review} % Optional

% Less than 500 words response to the meta-review. The response to the meta-review is optional. Provide a response if you disagree with the meta-review. Shepherds will only deny responses to meta-reviews if they are too long or are abusive/inappropriate.