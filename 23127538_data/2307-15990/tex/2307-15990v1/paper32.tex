% This is samplepaper.tex, a sample chapter demonstrating the
% LLNCS macro package for Springer Computer Science proceedings;
% Version 2.20 of 2017/10/04
%
\documentclass[runningheads]{llncs}

%\usepackage[T1]{fontenc}
\usepackage{graphicx}
% If you use the hyperref package, please uncomment the following line
% to display URLs in blue roman font according to Springer's eBook style:
\usepackage{hyperref}
\renewcommand\UrlFont{\color{blue}\rmfamily}
\usepackage{amsmath} % Required for some math elements
\usepackage{amssymb}
\usepackage{multirow}
\usepackage{doi}
\usepackage{makecell}
\usepackage{subcaption}
\usepackage{graphicx}
\usepackage{epstopdf}
\usepackage{here}
\usepackage{bbding}
\usepackage{booktabs}
\usepackage{multirow}
\usepackage[table,xcdraw]{xcolor}

\input alphabet
\def\Diag{\mathrm{Diag}}
\def\svd{\mathop{\mathrm{svd}}}
\def\R{\mathbb{R}}



\usepackage{xcolor}
\definecolor{cadmiumgreen}{rgb}{0.0, 0.42, 0.24}
% Old %
\newcommand{\YZ}[1]{\textcolor{black}{#1}}
\newcommand{\CH}[1]{\textcolor{black}{#1}}
\newcommand{\DM}[1]{\textcolor{black}{#1}}
\newcommand{\JI}[1]{\textcolor{black}{#1}}
% New %
\newcommand{\yz}[1]{\textcolor{black}{#1}}
\newcommand{\ch}[1]{\textcolor{black}{#1}}
\newcommand{\dm}[1]{\textcolor{black}{#1}}
\newcommand{\ji}[1]{\textcolor{black}{#1}}

\usepackage{soul}
\def\suppJI#1{{\footnotesize \setstcolor{red}\st{#1}}}

\usepackage{comment}

\begin{document}

\title{Ultrasound Image Reconstruction with Denoising Diffusion Restoration Models}
\titlerunning{US Image Reconstruction with DDRM}

 \author{Yuxin Zhang \and Clément Huneau \and Jérôme Idier \and Diana Mateus} %index{Zhang, Yuxin} %index{Huneau, Clément} %index{Idier, Jérôme}  %index{Mateus, Diana}  
 \authorrunning{Y. Zhang et al.}
 \institute{Nantes Université, École Centrale Nantes, LS2N,\\ CNRS, UMR 6004, F-44000 Nantes, France \\ \email{yuxin.zhang@ls2n.fr}}

%\author{Anonymous}
%\authorrunning{*** et al.}
%\institute{Anonymous Organization \\ \email{******@***}}%

\maketitle            

\begin{abstract}
Ultrasound image reconstruction can be approximately cast as a linear inverse problem that has traditionally been solved with penalized optimization using the $l_1$ or $l_2$ norm, or wavelet-based terms. However, such regularization functions often struggle to balance the sparsity and the smoothness. A promising alternative is using learned priors to make the prior knowledge closer to reality. 
In this paper, we rely on learned priors under the framework of Denoising Diffusion Restoration Models (DDRM), initially conceived for restoration tasks with natural images. 
We propose and test two adaptions of DDRM to ultrasound inverse problem models, DRUS and WDRUS. Our experiments on synthetic and PICMUS data show that from a single plane wave our method can achieve image quality comparable to or better than DAS and state-of-the-art methods. 
The code is available at: \href{https://github.com/Yuxin-Zhang-Jasmine/DRUS-v1}{https://github.com/Yuxin-Zhang-Jasmine/DRUS-v1}.

\keywords{Ultrasound imaging  \and Inverse Problems \and Diffusion models.}
\end{abstract}


\section{Introduction}
%%%%%%%%%%%%%%%%%%%%%%%%%%%%%%%%%%%%%%%%%%%%%%%%%%%%%%%%%%%%%%%%%%%%%%%%%%%%%%%%
\section{Introduction}

Autonomous driving (AD) %with deep learning networks 
has shown promising achievements and is considered an important technological breakthrough that could revolutionize the future of transportation. Currently, ensuring the safety of autonomous driving systems has become a topic of extensive development.
% There has been much discussion on how to verify the safety of autonomous driving systems.
One traditional solution for safety tests is to exhaustively enumerate real scenarios for validation. Nevertheless, this process is not only labor-intensive and costly but also dangerous. Simulation has emerged as a robust, safe, and efficient alternative for training and evaluating AD software and algorithms~\cite{li2019aads, amini2020learning, amini2022vista}.

% Figure environment removed

Recently, neural radiance field (NeRF)~\cite{mildenhall2020nerf} has gained significant attention in AD simulation~\cite{drivesim}. This approach leverages multi-view images to construct a 3D scene and enable novel view synthesis for both indoor and outdoor applications. When it comes to constructing NeRF models in AD simulation, there are two options available: 1) collecting a large amount of data to cover as many viewpoints as possible, and constructing a fine-grained scene offline; 2) directly using log data from road tests to quickly create an environment and dynamically simulate driving scenarios. The first choice can deliver high-quality simulation~\cite{tancik2022block} by transforming the problem of view extrapolation into view interpolation through the use of large amounts of data. However, it is time- and cost-intensive, which makes it challenging to generalize. As for the second choice, the collected images from log data are usually similar to each other along the running trajectory, which may result in unsatisfactory outcomes, particularly when the camera pose is placed out-of-trajectory (see \figref{figSupportComp} as an example), semantic consistency cannot be guaranteed when synthesizing images from deviated views. We observe this problem under this data condition in all neural radiance approaches, and to the best of our knowledge, none of the existing work has solved this issue.
In our opinion, semantic consistency is crucial for AD simulation, and synthesizing on deviated views is unavoidable for scalability.

AD simulation usually involves map data for planning and control, which can be obtained from a prebuilt High-Definition Map (HD Map) or an online mapping module. While the map data may not be pixel-perfect, it can provide semantic-level information that is useful for enhancing the semantic consistency of the trained neural radiance field.
In this paper, we propose incorporating map priors into neural radiance fields to enhance the semantic consistency and rendering quality of deviated driving view synthesis. Firstly, we employ ground information from maps to supervise the density field of NeRF, providing a more reliable road base for semantic entities. Next, we propose sampling rays to simulate unseen views. Unlike most NeRF augmentation methods~\cite{zhang2022ray, chen2022geoaug}, we utilize ground and lane information in sampling computations to guide the radiance field. More importantly, we model the above two supervision methods as weak supervision by using an uncertainty parameter and propose an uncertainty tempering scheme to increase the uncertainty. This ensures that map priors only guide the training process rather than enforce it towards their absolute values. As a result, our proposed method not only improves the rendering quality of interpolated novel view synthesis quantitatively but also enhances the semantic consistency of deviated novel view synthesis. 
Our contributions can be summarized as follows:
% We summarize the contributions of this paper as follows.



% To overcome the limitations of the collected data, this paper proposes a novel approach that leverages map information to enhance the semantic consistency of the synthesized driving views. 

% Autonomous driving (AD) vehicles are being trained with the help of deep learning networks and have shown promising achievements. This technology is considered to be a breakthrough that could change the way of transportation in the near future. However, there are many discussions on how to verify or judge the safety of autonomous driving systems.
% A straightforward solution towards the safety tests is to exhaustively enumerate real scenarios for validation as many as possible. However, the process of implementing different real scenarios is not only labor-intensive and costly, but also dangerous. Simulation has been proved to be an alternative, which is robust, safe, efficient in training, and evaluating AD software and algorithms.
% Now, the emerging technology of neural radiance field (NeRF)~\cite{} leverages multi-view images to construct a 3D scene and enable novel view synthesis for many indoor and outdoor applications. For AD simulation, there are two choices for constructing NeRF models: 1) collect a large amount of data, such as LiDAR and camera data, similar to mapping, to construct a fine-grained scene offline; or 2) directly use the log file (typically in the format of ROS bag) to rapidly create an environment and then dynamically simulate the driving scenarios.
% The first choice can achieve high-quality simulation, but it is time-consuming and expensive, making it difficult to generalize to very large scales. On the other hand, the second option is fast but can lead to low-quality simulation due to the data being sparse and similar to each other in log data. This paper tackles the problem raised by choosing the latter option and attempts to improve the quality of out-of-trajectory driving view synthesis by incorporating map information. This approach is practical for many autonomous driving tests.
% In conclusion, the use of NeRF technology for AD simulation is a promising avenue for training and evaluating AD software and algorithms. While both options for constructing NeRF models have their pros and cons, this paper addresses the challenges of the second option and proposes a potential solution to improve the quality of simulation.

%There exist a few attempts to facilitate training a NeRF model for synthesizing out-of-trajectory (or called as extrapo trajectory) views.


\begin{itemize}
    \item We propose a novel method to incorporate commonly used map priors in AD scenes into neural radiance fields to improve the out-of-trajectory driving view synthesis.
    \item We explicitly model the uncertainty in map priors as a parameter and propose an uncertainty tempering scheme to guide the training process of the neural radiance field.
    \item Experiments demonstrated that the proposed method can improve the semantic consistency of out-of-trajectory views and the rendering quality of novel view trajectory interpolation.
\end{itemize}

Our proposed method is easy to implement, can be easily plugged into existing NeRF algorithms, and has the capability of extending to other formats of priors.
Next, we review DDRM and introduce our method in Section~\ref{sec:IPB}.

\section{Denoising Diffusion Restoration Models}
\label{sec:DDRM}
 \DM{
A DDPM is a parameterized Markov chain trained to generate \ji{synthetic} images from noise relying on variational inference~\cite{ho_denoising_2020,nichol_improved_2021,dhariwal_diffusion_2021}. 
The Markov chain consists }\YZ{of two processes: a forward fixed diffusion process and a backward learned generation process. The forward diffusion process gradually adds Gaussian noise with variance $\sigma_t^2$ ($t = 1,\ldots, T$) to the clean signal $\xv_0$ until it becomes random noise, while in the backward generation process (see Fig.\ref{fig: sub_DDPM}), the random noise $\xv_T$ undergoes a gradual denoising process until a clean $\xv_0$ is generated}.
%
% Figure environment removed

%One attractive direction in recent years is
\DM{An interesting question in model-based deep learning is how}
\YZ{ to use prior knowledge learned by generative models to solve inverse problems.} 
Denoising Diffusion Restoration Models (DDRM)~\cite{kawar_denoising_2022} were recently introduced for solving linear inverse problems, taking advantage of a pre-trained DDPM model as the learned prior. 
%\YZ{In other words, DDRM applies the influence of the measurements to each step of denoising in DDPM as in Fig.\ref{fig: sub_DDRM}}. 
\DM{Similar to a DDPM, a DDRM is also a Markov Chain but conditioned on measurements $\yv_d$ through a linear observation model $\Hv_d$  
%linking the generated images to their observations
\footnote{We use subscript $d$ to refer to the original equations of the DDRM model.}. The linear model serves as a link between an unconditioned image generator and any restoration task. In this way, DDRM makes it possible to exploit pre-trained DDPM models \yz{whose weights are assumed to generalize over tasks}. 
\dm{In this sense, DDRM is fundamentally different from previous task-specific learning paradigms requiring training with paired datasets.}
%which is fundamentally different from the previous task-specific paradigms requiring training individuals with paired datasets.} 
Relying on this principle, the original DDRM paper was shown to work on several}
natural image restoration tasks such as denoising, inpainting, and colorization.

\DM{Different from DDPMs, the Markov chain in DDRM is defined in the spectral space of the degradation operator $\Hv_d$. To this end,}
DDRM leverages the Singular Value Decomposition (SVD): %of the observation operator, i.e., 
$\Hv_d=\Uv_d \Sb_d \Vv_d^\tD$ with $\Sb_d=\Diag\left(s_1,\ldots,s_N\right)$,
%$\svd(\Hv_d)$
%to decouple 
\DM{which allows decoupling} the dependencies between the measurements. 
\DM{The original observation model 
%From
%and to apply a denoising process to the noisy data in $\yv_d$. 
%\begin{align*}
$
 \yv_d= \Hv_d \xv_d + \nv_d= \Uv_d \Sb_d \Vv^\tD _d\xv_d + \nv_d,
$
%\end{align*}
%we get 
can thus be cast as a denoising problem that can be addressed \yz{on the transformed measurements:}}
\begin{equation*}
\overline{\yv}_d= \overline{\xv}_d + \overline{\nv}_d
\end{equation*}
with $\overline{\yv}_d= \Sb_d^\dag\Uv_d^\tD\yv_d$, $ \overline{\xv}_d=\Vv_d^\tD\xv_d$, and $\overline{\nv}_d=\Sb_d^\dag\Uv_d^\tD \nv_d$, where $\Sb_d^\dag$ is the generalized inverse of $\Sb_d$. %The subscript $_d$ indicates DDRM. 
The additive noise $\nv_d$ being assumed \textit{i.i.d.} Gaussian: $\nv_d\sim \mathcal{N}\left(0, \sigma_d^2\Iv_N\right)$, with a known variance $\sigma_d^2$ and $\Iv_N$ the $N\times N$ identity matrix, we then have
$\overline{\nv}_d$ \yz{with} standard deviation $\sigma_d\Sb_d^\dag$.
%\left[\begin{array}{ccccc}
%\frac{\sigma_d^2}{s_1^2} & & &\\
%& .. & & & \\
%& & \frac{\sigma_d^2}{s_i^2} &\\
%& & & .. &\\
%& & & & \frac{\sigma_d^2}{s_N^2}\\
%\end{array}\right]
%, and the denoising process is realized as follows:

Each denoising step from $\overline{\xv}_t$ to $\overline{\xv}_{t-1}$ ($t=T,...,1$) is a linear combination of $\overline{\xv}_t$, the transformed measurements $\overline{\yv}_d$, the transformed prediction of 
$\xv_0$ at the current step $\overline{\xv}_{\theta,t}$, and random noise. To determine their coefficients which are denoted as $A$, $B$, $C$, and $D$ respectively, the condition on the noise, $(A\sigma_t)^2 + (B\sigma_d/s_i)^2 + D^2 = {\sigma_{t-1}}^2$, and on the signal, $A+B+C = 1$, are leveraged, and the two degrees of freedom are taken care of by two hyperparameters.% $\eta$ and $\eta_b$.  
\begin{comment}
In practice, a denoising step is computed by sampling from a probability distribution $p_\theta$ parameterized by a step-dependent DNN with parameters $\theta$. When $t=T$,
\begin{equation*}
p_\theta^{(T)}\left(\overline{\xv}_T^{(i)} \mid \yv_d\right)= \begin{cases}\mathcal{N}\left(\overline{\yv}_d^{(i)}, \sigma_T^2-\frac{\sigma_d^2}{s_i^2}\right) & \text{if } s_i>0 \\ \mathcal{N}\left(0, \sigma_T^2\right) & \text{if } s_i=0,\end{cases}
\end{equation*}
when $t<T$, 
\begin{equation*}
    p_\theta^{(t)}\left(\overline{\xv}_t^{(i)} \mid \xv_{t+1}, \yv_d\right)= \begin{cases}\mathcal{N}\left(\overline{\xv}_{\theta, t}^{(i)}+\sqrt{1-\eta^2} \sigma_t \frac{\overline{\xv}_{t+1}^{(i)}-\overline{\xv}_{\theta, t}^{(i)}}{\sigma_{t+1}}, \eta^2 \sigma_t^2\right) & \text{if } s_i=0 \\ \mathcal{N}\left(\overline{\xv}_{\theta, t}^{(i)}+\sqrt{1-\eta^2} \sigma_t \frac{\overline{\yv}_d^{(i)}-\overline{\xv}_{\theta, t}^{(i)}}{\sigma_{{d}} / s_i}, \eta^2 \sigma_t^2\right) & \text{if } \sigma_t<\frac{\sigma_{{d}}}{s_i} \\ \mathcal{N}\left(
    (1-\eta_b)\overline{\xv}_{\theta,t}^{(i)} + \eta_b\overline{\yv}_d^{(i)}, \sigma_t^2-\frac{\sigma_{{d}}^2}{s_i^2}\eta_b^2 \right) & \text{if } \sigma_t \geqslant  \frac{\sigma_{{d}}}{s_i},\end{cases}
\end{equation*}
where the notation $i$ indicates the element index over the singular values of $\Hv_d$.
\end{comment}

\yz{In this way, the iterative restoration is achieved by the iterative denoising, and the final restored image is $\xv_0 = \Vv_d\overline{\xv}_{0}$. %, where $\overline{\xv}_{0}$ is sampled from $p_\theta^{(0)}\left(\overline{\xv}_0^{(i)} \mid \xv_{1}, \yv_d\right)$. 
For speeding up this process, skip-sampling \ji{\cite{DDIM}} is applied in practice. We denote the number of iterations as \texttt{it}.}
\begin{comment}
The first denoising step (when $t=T$) is realized by sampling from $p_\theta^{(T)}\big(\overline{\xv}_T^{(i)} \mid \yv_d\big)$, with two cases:
%(i) $s_i>0$: since it is reasonable to assume that $\sigma_T$ is larger than any $\sigma_d / s_i$ as $\xv_T$ is a random noise, only the contribution of measurement $\overline{\yv}_d^{(i)}$ is considered; (ii) $s_i=0$: it samples from a random field.}

\begin{equation*}
p_\theta^{(T)}\left(\overline{\xv}_T^{(i)} \mid \yv_d\right)= \begin{cases}\mathcal{N}\left(\overline{\yv}_d^{(i)}, \sigma_T^2-\frac{\sigma_d^2}{s_i^2}\right) & \text{if } s_i>0 \\ \mathcal{N}\left(0, \sigma_T^2\right) & \text{if } s_i=0\end{cases}
\end{equation*}
%
\YZ{The denoising process in the next steps (when $t<T$) is achieved by sampling from $p_\theta^{(t)}\left(\overline{\xv}_t^{(i)} \mid \xv_{t+1}, \yv_d\right)$:
%, where there are three possible cases: $\sigma_d / s_i = +\infty,$ $\sigma_d / s_i > \sigma_t,$ and $\sigma_d / s_i \leqslant \sigma_t$. The synthetic output from the learned prior, $\overline{\xv}_{\theta, t}=\Vv_d^{t} \xv_{\theta, t}$, is used for the first two cases, the measurement information $\overline{\yv}_d^{(i)}$ is utilized for the last two cases, and the previous step's results, $\overline{\xv}_{t+1}^{(i)},$ are used when $\sigma_d / s_i = +\infty.$
% \begin{itemize}
%     \item if $\sigma_d / s_i = +\infty$, we assign weights to both the synthetic of the learned prior $\overline{\xv}_{\theta, t}^{(i)}$ and the results from the previous step $\overline{\xv}_{t+1}^{(i)}$;
%     \item if $\sigma_d / s_i > \sigma_t$, we assign weights to both the synthetic of the learned prior and the measurements;
%     \item if $\sigma_d / s_i \leqslant \sigma_t$, we totally believe the measurements.
% \end{itemize}
}
%
\begin{equation*}
    p_\theta^{(t)}\left(\overline{\xv}_t^{(i)} \mid \xv_{t+1}, \yv_d\right)= \begin{cases}\mathcal{N}\left(\overline{\xv}_{\theta, t}^{(i)}+\sqrt{1-\eta^2} \sigma_t \frac{\overline{\xv}_{t+1}^{(i)}-\overline{\xv}_{\theta, t}^{(i)}}{\sigma_{t+1}}, \eta^2 \sigma_t^2\right) & \text{if } s_i=0 \\ \mathcal{N}\left(\overline{\xv}_{\theta, t}^{(i)}+\sqrt{1-\eta^2} \sigma_t \frac{\overline{\yv}_d^{(i)}-\overline{\xv}_{\theta, t}^{(i)}}{\sigma_{{d}} / s_i}, \eta^2 \sigma_t^2\right) & \text{if } \sigma_t<\frac{\sigma_{{d}}}{s_i} \\ \mathcal{N}\left(\overline{\yv}_d^{(i)}, \sigma_t^2-\frac{\sigma_{{d}}^2}{s_i^2} \right) & \text{if } \sigma_t \geqslant  \frac{\sigma_{{d}}}{s_i},\end{cases}
\end{equation*}
where $\theta$ denotes the trainable parameters and ${\xv}_{\theta, t}$ represents the prediction of $\xv_0$ at step $t$ by a model, the notation $i$ indicates the element index, and $\eta$ is a hyperparameter.

In this way, the Markov chain $\mathbf{x}_T \rightarrow \mathbf{x}_{T-1} \rightarrow \ldots \rightarrow \mathbf{x}_1 \rightarrow \mathbf{x}_0$ is dependent on the inverse problem, and the final restored image is $\xv_0 = \Vv_d\overline{\xv}_{0}$, where $\overline{\xv}_{0}$ is sampled from $p_\theta^{(0)}\left(\overline{\xv}_0^{(i)} \mid \xv_{1}, \yv_d\right)$. \YZ{For speeding up this process, skip-sampling is always applied in practice. We denote the number of iterations as \texttt{it}.}
\end{comment}

\section{Method: Reconstructing US images with DDRM}
\label{sec:IPB}
We target the problem of reconstructing US images from raw data towards improving image quality. 
%dm To obtain a linear model, 
\dm{To model the reconstruction with a linear model,}
we consider the ultrasonic transmission-reception process under the first-order Born approximation.
\ji{We introduce the following notations: $\tau$, $k$, $x$, and $\rv$ respectively denote
the time delay, the time index, the reflectivity function, and the observation position in the field of view.}
%Denoting by $\tau$ the time delay, by $k$ the \yz{time index, by $x$ the reflectivity function, and by $\rv$ the observation position in the field of view.}
When the ultrasonic wave transmitted by the $i^{th}$ element passes through the scattering medium $\Omega$ and is received by the $j^{th}$ element, the received echo signal can be expressed as
\begin{equation}
    y_{i, j}(k) = \int_{\rv \in \Omega}a_i(\rv)a_j(\rv)h(k-\tau_{i,j}(\rv)) x(\rv) \mathrm{d} \rv + n_{j}(k),
\label{Equ: model_continuous}
\end{equation}
where $n_{j}(k)$ represents the noise for the $j^{th}$ receive element, function $h$ \yz{is the convolution of the emitted excitation pulse} and the two-way transducer impulse response, and $a$ represent\dm{s} the weights for apodization according to the transducer's limited directivity.
\begin{comment}
    \ch{I do noto agree. There is no question of computation tractability here, because we don't compute in continuous. The true reason is that we ignore the spatial impulse respone. What we write here as $a$ are not impulse response, if so you should write something like $a\delta$ and convolve. Here $a(r)$ are only weights related to the directivity, as you well said. Writing the model with SIR will complexify the eq. 1, and I think this is not necessary in this article.}
\end{comment}

The discretized linear physical model with $N$ observation points and $K$ time samples for all $L$ receivers can then be 
%dm formulated 
\dm{rewritten}
as $\yv=\Hv\xv + \nv$, \yz{where $\xv\in \R^{N\times 1}$, $\nv\in \R^{KL\times 1}$, and $\Hv\in \R^{KL\times N}$ is filled with the convolving and multiplying factors from $h$ and $a$ at the delays $\tau_{i,j}$}. 
Due to the \yz{Born approximation, the inaccuracy of $h$ and $a$,} and the discretization, the additive noise $\nv$ \yz{does} not only include the white Gaussian electronic noise but also the model error. However, for simplicity, we still assume $\nv$ as white Gaussian with standard deviation $\gamma$, which is reasonable for the plane wave transmission~\cite{iMAP}. 

While iterative methods exist for solving such \yz{linear inverse} problems~\cite{IPB_Ozkan,RED_USIPB}, our goal is to improve the quality of the reconstructed image by relying on recent advances in diffusion models and, notably, on DDRM.
%dm
\begin{comment}
Formally, we conceive a method to transform the pre-beamformed Radio-Frequency (RF) channel data $\yv \in \R^{KL\times 1}$ into a B-mode image $\xv\in \R^{N\times 1}$, where  $K$ stands for the number of time samples, $L$ the number of channels of a one-dimensional transducer array, and $N$ is the number of pixels in the image. We formulate the problem with a direct linear model $\yv=\Hv\xv + \nv$, where $\Hv\in \R^{KL\times N}$ is the model matrix containing the excitation waveform shifted regarding the pixels-transducer times of flight, and weighted (apodization) according to the transducer limited directivity; 
and $\nv\in \R^{KL\times 1}$ is Gaussian noise with standard deviation $\gamma$.
\end{comment}
Given the above linear model, we \yz{can now} rely on DDRM to iteratively guide the reconstruction of the US image from the measurements. However, since DDRM relies on \yz{the} SVD of $\Hv$ to go from a generic inverse problem \yz{to} a denoising/inpainting problem, and \yz{since this} SVD produces huge orthogonal matrices that cannot be implemented as operators, we \yz{propose to} transform the linear inverse problem model to:
\begin{comment}
this naive direct model leads to a huge model matrix $\Hv$. Since DDRM relies on SVD of $\Hv$, the solution becomes computationally impractical.
Instead, we used the beamformed data :    
\end{comment}
\begin{equation}
    \Bv\yv=\Bv\Hv\xv + \Bv\nv,
    \label{Equ: model_HtH}
\end{equation}
where $\Bv \in \R^{N \times KL}$ is a beamforming matrix that \yz{projects channel data to the image domain}. 
\begin{comment}
A classical beamformer in US imaging is Delay-And-Sum (DAS)\cite{Perrot_2021}, mainly based on pixel-to-transducer times of flight. \yz{In the case} we \yz{consider the pulse-echo response} $h$, another common beamformer is the matched filter performed by defining $\Bv=\Hv^\tD$.
\end{comment}
\yz{After this transformation, we then feed the new inverse problem (Eq.~\ref{Equ: model_HtH}) to DDRM to iteratively reconstruct $\xv$ from $\Bv\yv$ observations. In this way, the size of the SVD of $\Bv\Hv$ becomes more tractable.}
\begin{comment}
Considering the inverse problem in DDRM, we feed the 
$\yv_d=\Bv\yv$ as input and the matrix
$\Hv_d=\Bv\Hv$ for the SVD decomposition. 
\end{comment}
We call this first model DRUS for Diffusion Reconstruction in US.

However, the noise of the updated direct model $\Bv\nv$ is no longer white and thus, it does not meet the assumption of DDRM. For this reason, we introduce a whitening operator $\Cv \in \R^{M \times N}$, where $M \leqslant N$, and upgrade the inversion model to its final form:
\begin{equation}
    \Cv\Bv\yv = \Cv\Bv\Hv\xv + \Cv\Bv\nv,
    \label{Equ: model_CHtH}
\end{equation}
where \Cv is such that $\Cv\Bv\nv$ is a white noise sequence. In order to compute $\Cv$, we rely on the eigenvalue decomposition 
$
    \Bv\Bv^\tD = \Vv \Lambdab \Vv^\tD
$
where $\Lambdab \in \R ^{N \times N}$ is a diagonal matrix of the eigenvalues of $\Bv\Bv^\tD$, and $\Vv \in \R^{N \times N}$ is a matrix whose columns are the corresponding right eigenvectors. Then, the covariance matrix of the whitened additive noise $\Cv\Bv\nv$ can be written as
\begin{align*}
 \mathrm{Cov}(\Cv\Bv\nv)&= \ED[\Cv\Bv\nv \nv^\tD \Bv^\tD \Cv^\tD]
% &= \Cv\Bv \ED(\nv\nv^\tD)\Bv^\tD \Cv^\tD\\
 = \gamma^2 \Cv\Bv\Bv^\tD\Cv^\tD
 = \gamma^2 \Cv\Vv \Lambdab \Vv^\tD \Cv^\tD.
\end{align*}
Now, let $\Cv =\Pv\Lambdab^{-\frac{1}{2}} \Vv^\tD$ with $\Pv=[\Iv_M,\zerob_{M\times(N-M+1)}] \in \R^{M \times N}$. It can be easily checked that $\Cv\Vv \Lambdab \Vv^\tD \Cv^\tD=\Iv_M$, \yz{proving the noise $\Cv\Bv\nv$ is white}.

\yz{Besides, discarding the smallest eigenvalues by empirically choosing $M$, rather than strictly limiting ourselves to zero eigenvalues, can compress the size of the observation vector $\Cv\Bv\yv$ from $N \times 1$ to $M \times 1$ and make the size of the SVD of $\Cv\Bv\Hv$ more  tractable.} %which can help save some memory in some cases.}
\begin{comment}
In practice, we discard the smallest eigenvalues by empirically choosing $M$, rather than strictly limiting ourselves to zero eigenvalues. As a consequence, the size of the observation vector $\Cv\Bv\yv$ is compressed from $N \times 1$ to $M \times 1$. 
\end{comment}

In order to adapt DDRM to the final inverse model in Eq.~\ref{Equ: model_CHtH}, we consider $\yv_d=\Cv\Bv\yv$ and $\nv_d=\Cv\Bv\nv$ as input and \yz{compute the SVD of $\Hv_d=\Cv\Bv\Hv$.} We name this whitened version of the approach WDRUS. In summary:
\begin{itemize}
    \item \DM{DRUS model relies on} ~\eqref{Equ: model_HtH} \DM{with} $\yv_d=\Bv\yv$,  $\Hv_d=\Bv\Hv$ and $\nv_d=\Bv\nv$
    \item \DM{WDRUS model relies on}~\eqref{Equ: model_CHtH} \DM{with} $\yv_d=\Cv\Bv\yv$, $\Hv_d=\Cv\Bv\Hv$ and $\nv_d=\Cv\Bv\nv$.
\end{itemize}
\begin{comment}
In the special case where the DAS matrix $\Bv$ is chosen as the \emph{matched filtering} operator $\Bv= \Hv^\tD$, then
\begin{equation}
    \Bv\Hv=\Bv\Bv^\tD = \Vv \Lambdab \Vv^\tD.
    \label{Equ: B=Ht start}
\end{equation}
and
\begin{align}
\Cv\Bv\Hv &= \Cv\Bv\Bv^\tD\\
% &= \Pv\Lambdab^{-\frac{1}{2}} \Vv^\tD \ast  \Vv \Lambdab \Vv^\tD\\
 &= \Pv \Lambdab^{\frac{1}{2}} \Vv^\tD\\
 &= \Iv_M \Sigmab \Vv^\tD,
 \label{Equ: B=Ht end}
\end{align}
where $\Sigmab \in \R ^{M \times N}$ keeps the first M rows of the diagonal matrix $\Lambdab^{\frac{1}{2}}$.
%As a consequence, svd decompositions are simplified.
\end{comment}

\section{Experimental Validation}
\label{sec:results}
% Figure environment removed
%
\section{Results} \label{sec:results}
%
\subsection{Sanity Checks} \label{ssec:sanity_checks}
We first provide evidence that the TCAV method can be applied to explain EEG data and the LHB model. 
In Figure \ref{fig:results_sanity_check}, the high significance of class data as concepts (\textit{Left Fist Movement} with positive evidence and \textit{Right Fist Movement} with negative evidence) confirms this. Furthermore, concepts based on maximal activity in either the left or right hemisphere for the \emph{alpha} frequency band strongly indicate that lateralized cortical activity is detected by several layers in the model, as expected.

\quad Moreover, the negative alignment of a concept based on labeled artifacts with the model representation of motor task data implies that artifacts in EEG data significantly influence classification tasks. We find that \textit{eyem} has a negative impact on the classification of \textit{Left Fist Movement}. Note that this does \textit{not} mean that \textit{eyem} positively affects the opposite class, that is \textit{Right Fist Movement}, as the TCAV Score is specific to the "\textit{Left Fist Movement} dataset". Conversely, \textit{eyem} could negatively affect the classification of both \textit{Left Fist Movement} and \textit{Right Fist Movement}, due to the lower signal-to-noise ratio for classification when artifacts are present.

\subsection{Event-based concepts} \label{ssec:annotated_data_as_concepts}
We next investigate whether fine-tuning the LHB model for seizure classification on the TUSZ dataset and using explanatory concepts defined with labeled data from TUEV aligns with the model's internal representation for data labeled as containing seizures. The target of the investigation is the \textit{seizure} label and we test all bottlenecks in the LHB model. The results of this experiment are shown in Figure \ref{fig:results_seizure}.

\quad When compared to EEG data labeled as containing seizures, the epilepsy-related concepts \textit{pled}, which is present in certain brain areas, and \textit{gped}, which is present in most of the brain, exhibit high and positive evidence in nearly all bottlenecks. This observation aligns with existing literature that associates epileptiform discharges with seizures \cite{gajic2015detection}, and it is expected that the LHB model will use these properties for classification. The \textit{spsw} concept also demonstrates significant positive evidence in the \textit{encoder} bottleneck but not in the further downstream bottlenecks. Similarly, the \textit{bckg} concept shows negative evidence in the \textit{encoder} bottleneck but not in the further downstream bottlenecks. It is interesting that these concepts only come to be significant in the initial bottleneck.
 A possible explanation is that the technical artifacts \textit{artf} and \textit{bckg} are not significant for the classification, but BENDR effectively identifies seizure-related concepts and filters out noise. The results also suggest that the model's \textit{classifier} and \textit{extended classifier} can be further optimized, as \textit{artf} is near-significant level in these bottlenecks and, as a result, the noise has not been completely removed. In conclusion, these examples indicate that concept-based explainability can provide valuable model design information.

\subsection{Anatomy/Frequency-Based Concepts} \label{ssec:artefacts_as_concepts}
%
% Figure environment removed
%
We have demonstrated that labeled EEG data can generate human-aligned concepts, which are integrated into the LHB model for seizure classification. This comes quite naturally as labeled 
data is labeled by humans and tend to align with human-relatable concepts.
We then present evidence that defining explanatory concepts based on cortical activity in frequency bands may uncover patterns corresponding to the model's internal representations.

\quad In particular, for a motor classification task using the MMIDB EEG dataset and targeting the \textit{Left Fist Movement} class, we show that cortical activity in the \emph{alpha} band aligns with the model's internal representation. In Figure \ref{fig:results_lateralization}, we find that the CAV for \textit{Somatosensory and Motor Cortex} in the right hemisphere positively aligns with the activations of \textit{Left Fist Movement} class data across all bottlenecks in the model. The mean TCAV scores are also consistently positively significant. At the same time, the TCAV scores for the same cortical area in the \textit{Left Hemisphere} are either negatively significant or insignificant. These results strongly suggest that the model's internal representation incorporates lateralization, reflecting the fact that one hemisphere exhibits more electrical activity than the other. It is noteworthy that lateralization is most significant in the \textit{Encoding Augment} and \textit{Summarizer} bottlenecks, indicating that it is captured early in the network.

\quad Additionally, we observe that the \textit{Primary Visual Cortex (V1)} areas do not exhibit lateralization, and their TCAV scores are insignificant across all bottlenecks and for both hemispheres. This further supports the conclusion that the LHB model utilizes specific cortical areas in its classification rather than all areas indiscriminately.

\quad While no apparent lateralization is present in the \textit{Premotor Cortex}, this part of the cortex is negatively significant in the \textit{Encoder} and \textit{Summarizer} bottlenecks for both the left and right hemispheres. A possible explanation is that the instances we examine involve participants \textit{performing} movements; therefore, there may not necessarily be relevant activity in the \textit{Premotor Cortex}, which is primarily involved in movement planning \cite{gallego2022going}.

\quad Lastly, we observe significance in the \textit{Classifier} bottleneck for \textit{Early Visual Cortex} and \textit{Dorsal Stream Visual Cortex}. We note that the movement is activated by a visual cue; however, further experiments would be required to fully clarify the effect.

\section{Discussion and Conclusion}
\label{sec:conclusion}

%The proposed method is trained on a set of public datasets available in the ultrasound toolbox [10]. The proposed approach has been accepted for presentation during the Challenge on Ultrasound Beamforming with Deep Learning (CUBDL) at the 2020 IEEE International Ultrasonics Symposium (IUS) [11], [12]. synthetic / PICMUS difference

%ALSO DISCUSS ABOUT TYPE OF PRETRAINED VS TRAINED
Regarding the computing time, \yz{our approaches need 3-4 minutes to form one image, which is slower than DAS1, PCF~\cite{PCF} and MNV2~\cite{MNV2}, but faster \ji{than EMV ~\cite{asl_eigenspace-based_2010} and RED~\cite{RED_USIPB}, which need 8 and 20 minutes, respectively.} RED is slow because each iteration contains an inner iteration while \ji{EMV spends time on covariance matrix evaluation and decomposition.} Our iteration restoration approaches require multiple multiplication operations with the singular vector matrix, which currently hinders real-time imaging. }Accelerating this process is one of our key focuses for future work.

\begin{comment}
the computationally expensive SVD for DDRM actually does not affect imaging time since the SVD results can be precomputed, but the multiplication operation with the singular vector matrix during the image reconstruction process currently hinders real-time imaging. On our machine equipped with the GPU NVIDIA Quadro RTX 3000, each iteration takes approximately 4.5 seconds.
\end{comment}

\YZ{In conclusion, for the first time, we achieve the reconstruction of ultrasound images with} 
\DM{ two adapted diffusion models, DRUS and WDRUS. }
\yz{Different from previous model-based deep learning methods which are task-specific and require a large amount of data pairs for supervised training, our approach requires none or just a small fine-tuning dataset composed of high-quality (e.g., DAS101) images only (there is no need for paired data). Furthermore, the fine-tuned diffusion model can be used}
%applied to} 
\dm{for other US related inverse problems.}
%diverse inverse problems, e.g., DRUS and WDRUS, as long as the same prior knowledge.}
\YZ{Finally, our method demonstrated competitive performance compared to DAS75, and other state-of-the-art approaches on the PICMUS dataset.}



\begin{comment}
\YZ{Our approach has demonstrated superior performance compared to DAS, both in terms of visual quality and evaluation metrics, on both synthetic and PICMUS datasets}, \DM{even though we fine-tuned the \texttt{f-number} to fit DDRM while kept the default values from the open-source code for DAS, as 
%While we used slightly different parameters, e.g. \texttt{f-number} for DAS and our methods when testing on the PICMUS dataset, such as using default values from the PICMUS open-source code for DAS while fine-tuning these parameters for our methods to fit DDRM, 
fundamentally, the number of plane waves affects the image quality of DAS.} 
\YZ{Our method is able to compete with 75 plane waves, which is sufficient evidence of its effectiveness. As for the distortion observed in the WDRUS results, it may be due to the amplification of errors in the ultrasound model by the whitening operator, but the specific reason requires further investigation.
}

\YZ{
Our method provides an important insight for the medical imaging field by addressing the challenge of training} \DM{model-based deep learning methods
%neural networks 
}\YZ{when access to datasets is restricted due to privacy concerns. The model we used was trained on ImageNet only, without any ultrasound data.
}
\DM{However, it} 
\YZ{
%It 
should be noted that  %images in the 
ImageNet data %dataset 
are significantly different from ultrasound images. For example, pixel values in natural images are always positive, while} 
%in ultrasound image reconstruction, 
 \DM{the reconstructed ultrasound 
$\xv$ contains both positive and negative values, and 
%ultrasound images 
are typically displayed after log compression.}
\YZ{Therefore, fine-tuning existing models with a small amount of ultrasound data may lead to better results.
}

\YZ{
Although DDRM relies on the computationally expensive SVD, it does not affect imaging time since} 
%its results can be saved and repeatedly used. 
\DM{ the SVD results can be precomputed.}
\YZ{However, the multiplication operation} \DM{ with the singular vector matrix 
%between the singular matrix and other vectors 
}
\YZ{during the image reconstruction process currently hinders real-time imaging. On our machine equipped with the GPU NVIDIA Quadro RTX 3000, each iteration takes approximately 4.5 seconds. Accelerating this process is one of our key focuses for future work.}

\YZ{
In conclusion, for the first time, we achieve the reconstruction of ultrasound images with} 
%a diffusion model and test two ultrasound models, DRUS and WDRUS. 
\DM{ two adapted diffusion models, DRUS and WDRUS. }
\YZ{Our method with single plane wave is even comparable to DAS with 75 plane waves,}
%which are often used to produce target images, in the case where the generative model has never been trained on ultrasound data.
\DM{which is often used as reference to train generative models, whereas our diffusion model was never trained on ultrasound data}
\end{comment}


% \section{Reviews}
% \subsection{meta-review}
% \newpage
\section{Meta-Review}\label{sec:meta}
\subsection{Summary}
This paper presents a qualitative study to understand the industry's viewpoint on program analysis based security testing tools, to explore organizations' selection criteria of \sast{} tools, and to understand how practitioners understand and work around the limitations of these tools.

\subsection{Scientific Contributions}
\begin{itemize}
    \item Provides a Valuable Step Forward in an Established Field
    \item Independent Confirmation of Important Results with Limited Prior Research
    \item Addresses a Long-Known Issue
    \item Establishes a New Research Direction
\end{itemize}

\subsection{Reasons for Acceptance}
\begin{enumerate}
\item This work provides valuable qualitative understanding of the industry needs and use of SASTs and the organizational barriers that prevent widespread adoption.
\item The methodology of this paper is sound, using appropriate quantitative and qualitative techniques.
\end{enumerate}

\subsection{Noteworthy Concerns} %
None.

% \subsection{reviews}
% \input{reviews/reviews}
% \subsection{rebuttal}
% \documentclass[10pt,twocolumn,letterpaper]{article}
\usepackage{iccv_rebuttal}
\usepackage{times}
\usepackage{graphicx}
\usepackage{amsmath}
\usepackage{amssymb}
\usepackage{xcolor}
\usepackage{xspace}

\usepackage[pagebackref=true,breaklinks=true,letterpaper=true,colorlinks,bookmarks=false]{hyperref}

\definecolor{msgreen}{rgb}{0.19215686,  0.63921569,  0.32941176}
\definecolor{msred}{rgb}{0.87058824,  0.17647059,  0.14901961}
\definecolor{msblue}{rgb}{0.19215686,  0.50980392,  0.74117647}

\newcommand{\note}[1]{\textcolor{red}{#1}}
\newcommand{\issue}[3]{\noindent\textsf{[#1] \textbf{#2}} #3}
\newcommand{\Rone}{\textcolor{msred}{\textbf{R1}}}
\newcommand{\Rtwo}{\textcolor{msgreen}{\textbf{R2}}}
\newcommand{\Rthree}{\textcolor{msblue}{\textbf{R3}}}

\def\iccvPaperID{6874}
\def\httilde{\mbox{\tt\raisebox{-.5ex}{\symbol{126}}}}

\begin{document}
\title{Replay: Multi-modal Multi-view Acted Videos for Casual Holography}
\maketitle
\thispagestyle{empty}

We thank the reviewers for their helpful suggestions and questions, which we address below.
Unfortunately, \Rtwo~reviewed the paper we included in the sup.~mat. covering the audio part of the dataset instead of our submission. 

\issue{\Rone}{Only $<1$ min videos [...] released}
We will release raw 10+ minute videos containing calibrations and full acting stage alongside the processed crops used in the benchmarks.
We trained on 30–40 second crops for the benchmarks since SOTA methods are failing on longer sequences.

\issue{\Rone}{Lack of color in equalized images}
DSLR cameras capture a wider range of brightness and colour, with more details than GoPro footage in the bright and dark areas of the image.
Because they contain more information, they may look less vibrant after equalization for print.
``Normal''  contrast and vibrancy can easily be recovered in post-processing at the expense of losing details in bright or dark areas.
% However, these images can be transformed using simple post processing to achieve any desired look, such as increasing contrast and vibrancy at the expense of losing details in bright or dark areas.
This is demonstrated in our supplementary video (timestamp 1:15), where after the color equalization step we perform post processing to get more vibrant images that are still consistent between the two sensors.
The processed crops of videos will be released with corrected colours.
% seamlessly get the post processed images.

\issue{\Rone}{Difference from dome datasets}
While the background is not fully natural as it contains several cameras on tripods and, for rear cameras, GoPro operators, as opposed to dome datasets, we were able to fit in bulky props such as a sofa and TV, or a ping-pong table. Moreover, masked or not, lighting conditions are natural and vary noticeably with the time of day, which is not possible with a dome.

\issue{\Rone}{The ``acting'' benchmark is more simple than expected}
We agree that monocular reconstruction is the Holy Grail of AR/VR, however, the SOTA dynamic NeRFs fall short of even attempting to extrapolate beyond the convex hull of input viewpoints (reconstructing only humans could be possible due to the ability to learn strong priors but this is out of scope of our benchmarks).
Please note that we will release the data loaders which permit effortlessly slicing the data by sensor, so that researchers will be able to experiment with monocular reconstruction.

\issue{\Rthree}{Variety of data}
The data collection for this benchmark is an ongoing process. At submission time we reported results for 46 scenes, but we have since collected more and now plan to release 68 scenes.
You are right that they were filmed in the same room, however furniture changed significantly between the recordings.
Out of 42 actors participating in the scenes, 21 are white, 11 are Asian, 5 are black, and 5 are mixed race.

\issue{\Rthree}{Are any stereo cameras included in the dataset?}
We agree that stereo videos might become an important modality, depending on how consumer devices are going to evolve.
As part of our ongoing data collection, we replaced GoPros with head-mounted rigs of two GoPros each (not part of the data presented in the paper).
% for subsequent sequences, however, the scenes presented in the paper were not filmed with binocular sensors.

\issue{\Rthree}{Limited GoPro movement during the action}
It is true that during acting, GoPro operators stay in place, however they exhibit natural human movements (such as swaying from side to side or turning the head).
We acknowledge the limitation but we believe that is representative of many natural AR/VR recording / playback scenarios.

\issue{\Rthree}{Rolling-shutter artefacts}
We used the highest shutter speed as reasonably possible to not degrade the colours, and the cameras we use are either high-end (DSLR) or marketed for sports activities (GoPro), so we did not observe this problem.
We specify the models of all hardware we use in l.\,359+.

%%%%%%%%%%%%%%%
% minor issues

\issue{\Rone}{4K resolution images are probably overkill}
While they are a few years away, highly-detailed reconstructions are required in practical consumer applications of holography. This future-proofs our data.
% This is not the bottleneck as we downsample images in the data loader.
% The problem with longer scenes is limited model capacity and, for some implementations, caching the ray points in RAM.

\issue{\Rone}{What is ``foreground segmentation''?}
The definition is scene-specific and is guaranteed to be consistent within a scene.
The methods are expected to produce non-zero density only for the foreground objects.

\issue{\Rone}{Why can't NeRFies produce ``opacity masks''?}
We empirically found that learning NeRFies on masked videos produced worse results, in some cases degenerated to black renders.
Please note that we fairly evaluate all models on the foreground regions only.

\issue{\Rone}{The IOU metric is unclear}
We will describe the metric more formally.

\issue{\Rone}{Is there a bug in HexPlane for the ``flyaround'' (semi-static) scene?}
We don't think it is a bug.
We use the same code base with incremental differences between TensoRF and HexPlane.
We attempted to explain the phenomenon in l.\,477+.
Flyaround benchmark requires the methods to extrapolate more significantly, and HexPlane, having to model an additional degree of freedom (time), fails to generalise to distant viewpoints, while TensoRF has enough data assuming constancy in time.
The acting benchmark requires mostly to interpolate the viewpoints, so that methods are generally less prone to overfitting there.

\issue{\Rthree}{Dataset format}
Thank you for the suggestion.
We will provide highly configurable PyTorch data loaders, which will hopefully cover most use cases, and most users won't need to learn the underlying format.
We will distribute the metadata (including camera poses at 30 FPS) as an Sqlite database, and the undistorted images and masks re-encoded as videos (lossless or with high enough bitrate), as we found this the best balance between data-loading speed and dataset size. We will provide tutorials on using both data-loader API and low-level metadata access.


\end{document}



% ---- Bibliography ----
%
% BibTeX users should specify bibliography style 'splncs04'.
% References will then be sorted and formatted in the correct style.
\bibliographystyle{splncs04}
\bibliography{bibliography}
\newpage
\thispagestyle{empty}
\beginsupplement
\begin{refsection}

\section{Pseudocode Example of Cumulative Disruption Algorithm} \label{sec:psuedocode}

For readers seeking a succinct code-like description of our cumulative disruption curve algorithm, we have included \cref{lst:psuedocode}.

\begin{lstlisting}[label=lst:psuedocode, language=Python, caption=Pseudocode for disruption algorithm]
disruption = []
for c in communities:
    remaining = 0
    original = 0
    removeCommunity(c)
    for user in users:
        if degree(user) > 0:
            remaining += degree(user)
            original += originalDegree(user)
    disruption += [1 - (remaining / original)]
\end{lstlisting}

Note that when calculating disruption on large networks, it is much more efficient to cache the size of the smallest community that each user participates in. We can then sort all users by the order in which they will be removed, and avoid computationally expensive references to a graph or adjacency matrix for each removal-step in the algorithm.

\section{Applications to Unipartite Networks} \label{sec:unipartite}

Our influence metric is intended for settings with clearly defined communities. For example, participation in subreddits, membership on a Mastodon server, or committing to a software code repository, all discretely identify users as members of those explicitly-bounded groups. However, network data is often presented in a unipartite configuration such as users following other users. If it is still desirable to delineate communities and measure their influence in these settings, then they can be converted into compatible bipartite networks using the following procedure:

\begin{enumerate}
    \item Apply a context-appropriate community detection algorithm to label each user as belonging to one community

    \item Create a vertex for each community

    \item Replace all user-user edges with user-to-community edges, where the edge weight is equal to the number of unipartite edges each user had to other nodes in that community

    \item Apply our influence metric to the resulting bipartite graph
\end{enumerate}

An example of this procedure is illustrated in \cref{fig:unipartite}, using a unipartite Watts-Strogatz small-world network (100 nodes, 5 neighbors, rewiring probability of 5\%), and label-propagation for community detection. The unipartite graph is shown in the top-left with community labels visualized with color. It is converted to a bipartite representation shown in the upper-right, and the effect of removing each community is illustrated in the bottom frame.

% Figure environment removed






\section{Calculating the Area Under the Disruption Curve} \label{sec:auc_explanation}

For \cref{fig:real_networks_auc,fig:toy_networks_auc,fig:assortativity_auc} we use the area under the disruption curve as a single-variable summary of how centralized a network is around its largest communities. To calculate the AUC, we use a trapezoidal approximation in logarithmic space.

We chose a trapezoidal approximation to calculate the area even with limited sample points from real-world networks. Integration is possible for purely analytic disruption curve simulations as in \cref{sec:analytic_simulations}, but this is not feasible for our non-Erd\H{o}s-R\'{e}nyi networks, so we use a trapezoidal approximation for all synthetic networks for consistency.

We measure the AUC in logarithmic space, because measuring in linear space would heavily weight the influence of the smallest communities that are removed last, and our primary interest is in examining the influence of the largest communities on the broader population. 

\section{Synthetic Network Topology Details} \label{sec:toy_examples}

We measure centralization on a variety of synthetic networks introduced in \cref{sec:disruption_toy}. In this section, we include further description and visualization of the synthetic networks used.

Bipartite Near-Star networks are analogous to a unipartite star network with duplicate edges, but in a bipartite setting. Starting with a unipartite star, replace each edge from the hub to a leaf with a two-path from the hub community to a new ``user" vertex, to the leaf community. Duplicate edges from the unipartite hub to leaves are converted into multiple users that share a community, and serve to break ties when pruning communities for disruption curves. This is illustrated in \cref{fig:star}.

% Figure environment removed

For our ``Powerlaw" networks we follow a bipartite configuration model. We first create vertices representing the desired number of communities and users. We then draw from a powerlaw distribution with an assigned $\gamma$ exponent, and assign the drawn degree to each community. Then, we create a corresponding number of edges, wiring each community to users drawn uniformly at random without replacement. This yields networks where communities follow a powerlaw degree distribution, while users follow a normal degree distribution.

Bipartite community-user networks can be visualized in a flat plane, as in \cref{fig:centralization-pl}, or as a multi-layer graph, as in \cref{fig:pl-toy}. A multi-layer representation may be beneficial for representing inter-community relationships that are not explained by shared users, such as Mastodon federation agreements, or shared moderator staff in two subverses. However, these multiplex relationships were deemed out-of-scope for our current work.

% Figure environment removed




\begin{comment}
  #data structure for the dispersion metric
  D = np.zeros(nm)

  #calculate dispersion
  cumu_sum=0
  for n in np.arange(0,nm):
    cumu_sum += n*pn[n]
    #calculate U_n
    if pn[n]==0:
      continue
    Pnpm = Pnm[n,:]/np.sum(Pnm[n,:])
    U=0
    for m in np.arange(0,mm):
      if(np.sum(Pnm[:,m])>0):
        Pnmp = Pnm[:,m]/np.sum(Pnm[:,m])
        prob = np.sum(Pnmp[n:-1])
        U+=Pnpm[m]*prob**(m-1)
    D[n] = n*pn[n]*(1-U)/(cumu_sum-n*pn[n]*U)
\end{comment}

\section{Mathematical Analysis of Disruption in Random Networks} \label{sec:analytic_simulations}

We here calculate the disruption curves for random bipartite networks parameterized by their joint-degree distribution. This approach therefore fixes the distribution $\lbrace g_m \rbrace$ of communities $m$ per user, the distribution $\lbrace p_n \rbrace$ of community size $n$, and the joint-distribution $P_{n,m}$ for the degree of the node and community involved in a random bipartite link. Beyond these constraints, the networks are fully random but allow us to explore the role of heterogeneous connectivity at the user and community level as well as the impact of correlations between both levels.

We wish to calculate the disruption $D(n)$ involved when removing communities of size $n'<n$ in these random networks. By definition of the bipartite network, we know that $np_n$ edges are removed when removing communities of size $n$. Once again, we define disruption as the fraction of \textit{remaining} edges disrupted by communities of size $n$ during the pruning process. It is thus given by the number of edges that belong to communities of size $n$ minus the fraction $u_n$ of those that are the sole edge of the corresponding users (since these users are removed in the pruning) divided by the number of edges belonging to communities of size equal or smaller than $n$ minus the $u_nnp_n$ users removed. We write:

\vspace{2em}
\begin{equation}
    D(n) = \frac{
            \eqnmarkbox[NavyBlue]{bigedges}{np_n}
            -
            \eqnmarkbox[OliveGreen]{prunededges}{u_nnp_n}
        }{
            \eqnmarkbox[WildStrawberry]{remainingedges}{\sum_{n'\leq n}n'p_{n'}}
            -
            \eqnmarkbox[OliveGreen]{prunededges2}{u_nnp_n}
        } \; .
    % Here's Laurent's original expression
    %D(n) = \frac{np_n-u_nnp_n}{-u_nnp_n + \sum_{n'\leq n}n'p_{n'}} \; .
\end{equation}
\annotate[yshift=1em]{above,left}{bigedges}{Edges to comms. of size n}
\annotate[yshift=1em]{above,right}{prunededges}{Edges to removed users}
%\annotate[yshift=-1em]{below,right}{prunededges2}{Edges for removed users}
\annotate[yshift=-0.5em]{below}{remainingedges}{Edges to comms. n or smaller}
\vspace{2em}

The quantity $u_n$ can also be defined as the probability that a random user of a community of size $n$ has no community smaller than $n$. It can therefore be calculated like so:

\vspace{1em}
\begin{equation}
    u_n = \mathlarger{\sum}_m 
        \eqnmarkbox[NavyBlue]{users_in_n_with_m}{\frac{P_{n,m}}{\sum_{m'}P_{n,m'}}}
        \left(
            \eqnmarkbox[OliveGreen]{users_with_m_larger_than_n}{\frac{\sum_{n'\geq n} P_{n',m}}{\sum_{n'}P_{n',m}}}
        \right)^{m-1} \; .
    %u_n = \mathlarger{\sum}_m \frac{P_{n,m}}{\sum_{m'}P_{n,m}} \left(\frac{\sum_{n'\geq n} P_{n',m}}{\sum_{n'}P_{n',m}}\right)^{m-1} \; .
    \label{eq:un}
\end{equation}
\annotate[yshift=1em]{above,right}{users_in_n_with_m}{Fraction of users in comm. \\ \sffamily \footnotesize size n that have m edges}
\annotate[yshift=-0.5em]{below,left}{users_with_m_larger_than_n}{Fraction of users with m edges\\ \sffamily \footnotesize in comms. larger than size n}
\vspace{2.5em}

In the previous equation, we sum over every possible type of node in a community of size $n$, which will have a number of \textit{other} communities $m-1$ proportional to $P_{n,m}$, and ask for all of these communities to be larger or equal to $n$, which will be proportional to the sum of $P_{n',m}$ over all $n'$ larger or equal to $n$. Normalizing the probabilities appropriately yields Eq. (\ref{eq:un}) as written.

Note that these equations assume that edges are unweighted, and that there are no duplicate edges, which is what we expect from an infinite random simple graph. In our real-world data sets there are often duplicate edges (for example, one user following several different users on a Mastodon instance), which we compress to weighted edges for convenience.

Despite this difference between the analytical expression and real socio-technical networks, the analysis of random infinite graphs can be useful to test how disruption is impacted by simple network statistics such as degree distributions or correlations in the joint community-user degree matrix $P_{n,m}$. 

In a simple experiment, we create a random Erd\H{o}s-R\'{e}nyi-like bipartite network and correlated equivalent networks with the same degree distributions and variable community-user degree matrices $P_{n,m}$. The random network has a simple $P^{\textrm{rand}}_{n,m} \propto np_n mg_m$ (normalized) which we can modify manually. To do so, we calculate the maximally correlated $P^{\textrm{max}}_{n,m}$ by assigning users with highest degrees $m_{\textrm{max}}$ to the largest communities available before doing the same to users with the next higher degree and so on all the way down. We can do the same to calculate $P^{\textrm{min}}_{n,m}$ by assigning users with the lowest degree to the largest communities and working our way up in the user degree distribution. We can then create arbitrary community-user degree matrix $P_{n,m}$ by interpolating between linearly with $(1-\rho) P^{\textrm{rand}}_{n,m} + \rho P^{\textrm{max}}_{n,m}$ or $(1-\rho) P^{\textrm{rand}}_{n,m} + \rho P^{\textrm{min}}_{n,m}$.

Our results are shown in \cref{fig:assortivity_random_networks}. We find that positive user-community degree correlations increase disruption and therefore \textit{centralizes} the resulting socio-technical network. Conversely, negative correlations decreases correlations and \textit{decentralizes} the network. That being said, the relative effect of correlations is relatively small as the networks are still otherwise completely random.

% Figure environment removed

\section{Further Analysis of Assortativity} \label{sec:supplemental_assortativity}

There are multiple interpretations of degree assortativity in a bipartite setting. The linear correlation between user degrees and community degrees measures whether high-degree users are likely to be connected to high-degree communities. In our network definitions edges represent activity, like follow relationships or participation in conversations, so this measures whether active users are likely to be connected to communities with lots of activity. However, a second metric of interest is whether large communities are likely to be connected to other large communities, or in other words, the  assortativity of a unipartite-projected community-community graph. This can also be broken into two sub-cases: assortativity of community size (do communities with many users share users with other high-population communities), and assortativity of degree (do communities with lots of activity share users with other high-activity communities).

These three notions of assortativity are not independent; we might expect that users with lots of activity are active in communities with high populations, and may act as bridges between multiple communities with high activity and high population. However, the three metrics are not guaranteed to correlate and should be measured separately.

While rewiring to promote user-community degree assortativity, we also plotted the changes in community-community degree assortativity, shown in \cref{fig:assortivity_user_vs_community}. Strikingly, the community assortativity \textit{decreases} as we rewire to promote user assortativity. This is because as we rewire edges to focus user connections on the largest communities we implicitly decrease the number of edges between communities. This also matches the changes in disruption in \cref{fig:assortativity_auc}: increasing assortativity may reconnect large and insular communities with the rest of the network, briefly increasing their influence, but continued assortativity rewiring also cuts bridges to and between smaller communities, yielding a sparse network that is far less centralized.

% Figure environment removed

To further explore the relationship between these types of assortativity, we also rewired networks in the reverse direction: for randomly selected pairs of edges, we rewired those edges to \textit{decrease} user to community activity assortativity. We have plotted the change in disruption curves (\cref{fig:disassortative_auc}) and correlation between assortativity metrics (\cref{fig:disassortivity_user_vs_community}). In most networks, decreasing activity assortativity lowers centralization, although the effect diminishes as the network topology more closely approximates a random network. The one exception is the Penumbra; this network has such sparse inter-community connections that any perturbation of edges increases the cross-community links and therefore \textit{increases} centralization.

% Figure environment removed

% Figure environment removed

\section{Cumulative Impact on Giant Component Size} \label{sec:giant_components}

Some readers may be interested in how removing large communities influences the giant component size on each network. This is closely related to the cumulative population size in the top sub-plots of \cref{fig:real_networks_size_comparison} and \cref{fig:toy_networks_size_comparison}. Intuition suggests that the size of the giant component will be inversely proportional to the number of cumulative communities removed; as more large communities are pruned, the giant component should shrink. This relationship holds so long as the remaining communities are interlinked, but falters once a ``bridge" community is removed and the giant component splinters. Therefore, sparsely connected networks where bridges are more prominent will have a chaotic giant component size, while more densely connected networks will present a smooth curve until most communities are pruned. This relationship is illustrated in \cref{fig:real_giant_component}. Most curves are smooth until the tail of the distribution, with two notable exceptions: Voat's giant component changes once the largest insular communities are removed (see \cref{fig:voat_render}), and the Penumbra's curve is much ``spikier" as a result of its highly sparse structure.

% Figure environment removed

Measuring the change in giant component size captures some of the same features as our disruption metric. In particular, removing large insular communities may not change the giant component size if the community is completely isolated from the giant component, so this captures some aspect of both the size and topological role of a community. However, the impact of a community is boolean: if it touches the giant component, then removing the community will shrink the giant component by the size of that community. There is no distinction between a minimally integrated and tightly integrated community. Measuring the impact of a community in terms of fraction of edges severed, rather than component vertex size, offers finer insight into the interplay between size distribution and network structure.



\section{Comparison to Network Bottlenecking} \label{sec:cheeger}

The Cheeger number \cite{cheeger} is a single-valued metric representing how large of a ``bottleneck" inhibits conductance across a graph. It is typically written as:

\vspace{2em}
\begin{equation}
    h(G) = \min \left\{
        \frac{
                \eqnmarkbox[NavyBlue]{cheeger_crossedges}{|\partial A|}
            }{
                \eqnmarkbox[OliveGreen]{cheeger_alledges}{|A|}
            }
        : \eqnmarkbox[WildStrawberry]{cheeger_subset}{A \subseteq V(G)}, 
        \eqnmarkbox[Plum]{cheeger_bounds}{0 < |A| \leq \frac{1}{2} |V(G)|}
    \right\} 
\end{equation}
\annotate[yshift=1.2em]{above}{cheeger_crossedges}{Edges crossing the boundary of A}
\annotate[yshift=-0.2em]{below}{cheeger_alledges}{All edges in+across A}
\annotate[yshift=0.8em]{above}{cheeger_subset}{A is a subset of vertices of G}
\annotate[yshift=-2em]{below,left}{cheeger_bounds}{A contains at most half of all vertices}
\vspace{2em}

Our measurement of how much a community influences a larger population, and the Cheeger measurement of whether a community is a ``bottleneck" bear some conceptual similarities. Therefore, we compare our metric to the Cheeger number in two ways. First, we create a ``local Cheeger number," following an identical equation $\frac{|\partial A|}{|A|}$, but where $A$ is defined as the set of communities we are pruning, rather than via a global search. Second, we estimate bounds on the global Cheeger value of the graph. Since evaluating the graph conductance of all possible subsets of vertices is an NP-hard problem \cite{kaibel2004expansion}, it is impractical to directly measure the Cheeger constant on most large graphs. Fortunately, the Cheeger inequality offers upper and lower bounds on the Cheeger number based on the second eigenvalue of the normalized Laplacian of the adjacency matrix of G as follows:

$$\lambda_2/2 \leq h(G) \leq \sqrt{2\lambda_2}$$

Since they are sparse, these bounds can be calculated even on large real-world datasets. 
Unfortunately, in our tests the bounds are quite wide (see \cref{fig:cheeger}), limiting the utility of this approximation. We have plotted a comparison of the ``local" Cheeger number, bounds of the global Cheeger number, and our disruption metric, for a variety of simulated networks.

% Figure environment removed

\printbibliography[heading=subbibliography]
\end{refsection}

\end{document}



%--------------------table--------------------
% \begin{table}
% \caption{Table captions should be placed above the
% tables.}\label{tab1}
% \begin{tabular}{|l|l|l|}
% \hline
% Heading level &  Example & Font size and style\\
% \hline
% Title (centered) &  {\Large\bfseries Lecture Notes} & 14 point, bold\\
% 1st-level heading &  {\large\bfseries 1 Introduction} & 12 point, bold\\
% 2nd-level heading & {\bfseries 2.1 Printing Area} & 10 point, bold\\
% 3rd-level heading & {\bfseries Run-in Heading in Bold.} Text follows & 10 point, bold\\
% 4th-level heading & {\itshape Lowest Level Heading.} Text follows & 10 point, italic\\
% \hline
% \end{tabular}
% \end{table}

% \noindent Displayed equations are centered and set on a separate
% line.
%----------------------------------------------

%--------------------figure--------------------
% Please try to avoid rasterized images for line-art diagrams and
% schemas. Whenever possible, use vector graphics instead (see
% Fig.~\ref{fig1}).

% % Figure environment removed
%----------------------------------------------

%--------------------others--------------------
% the environments 'definition', 'lemma', 'proposition', 'corollary',
% 'remark', and 'example' are defined in the LLNCS documentclass as well.
%
% \begin{proof}
% Proofs, examples, and remarks have the initial word in italics,
% while the following text appears in normal font.
% \end{proof}
%----------------------------------------------



