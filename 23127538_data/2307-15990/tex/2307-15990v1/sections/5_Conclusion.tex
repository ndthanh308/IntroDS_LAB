
%The proposed method is trained on a set of public datasets available in the ultrasound toolbox [10]. The proposed approach has been accepted for presentation during the Challenge on Ultrasound Beamforming with Deep Learning (CUBDL) at the 2020 IEEE International Ultrasonics Symposium (IUS) [11], [12]. synthetic / PICMUS difference

%ALSO DISCUSS ABOUT TYPE OF PRETRAINED VS TRAINED
Regarding the computing time, \yz{our approaches need 3-4 minutes to form one image, which is slower than DAS1, PCF~\cite{PCF} and MNV2~\cite{MNV2}, but faster \ji{than EMV ~\cite{asl_eigenspace-based_2010} and RED~\cite{RED_USIPB}, which need 8 and 20 minutes, respectively.} RED is slow because each iteration contains an inner iteration while \ji{EMV spends time on covariance matrix evaluation and decomposition.} Our iteration restoration approaches require multiple multiplication operations with the singular vector matrix, which currently hinders real-time imaging. }Accelerating this process is one of our key focuses for future work.

\begin{comment}
the computationally expensive SVD for DDRM actually does not affect imaging time since the SVD results can be precomputed, but the multiplication operation with the singular vector matrix during the image reconstruction process currently hinders real-time imaging. On our machine equipped with the GPU NVIDIA Quadro RTX 3000, each iteration takes approximately 4.5 seconds.
\end{comment}

\YZ{In conclusion, for the first time, we achieve the reconstruction of ultrasound images with} 
\DM{ two adapted diffusion models, DRUS and WDRUS. }
\yz{Different from previous model-based deep learning methods which are task-specific and require a large amount of data pairs for supervised training, our approach requires none or just a small fine-tuning dataset composed of high-quality (e.g., DAS101) images only (there is no need for paired data). Furthermore, the fine-tuned diffusion model can be used}
%applied to} 
\dm{for other US related inverse problems.}
%diverse inverse problems, e.g., DRUS and WDRUS, as long as the same prior knowledge.}
\YZ{Finally, our method demonstrated competitive performance compared to DAS75, and other state-of-the-art approaches on the PICMUS dataset.}



\begin{comment}
\YZ{Our approach has demonstrated superior performance compared to DAS, both in terms of visual quality and evaluation metrics, on both synthetic and PICMUS datasets}, \DM{even though we fine-tuned the \texttt{f-number} to fit DDRM while kept the default values from the open-source code for DAS, as 
%While we used slightly different parameters, e.g. \texttt{f-number} for DAS and our methods when testing on the PICMUS dataset, such as using default values from the PICMUS open-source code for DAS while fine-tuning these parameters for our methods to fit DDRM, 
fundamentally, the number of plane waves affects the image quality of DAS.} 
\YZ{Our method is able to compete with 75 plane waves, which is sufficient evidence of its effectiveness. As for the distortion observed in the WDRUS results, it may be due to the amplification of errors in the ultrasound model by the whitening operator, but the specific reason requires further investigation.
}

\YZ{
Our method provides an important insight for the medical imaging field by addressing the challenge of training} \DM{model-based deep learning methods
%neural networks 
}\YZ{when access to datasets is restricted due to privacy concerns. The model we used was trained on ImageNet only, without any ultrasound data.
}
\DM{However, it} 
\YZ{
%It 
should be noted that  %images in the 
ImageNet data %dataset 
are significantly different from ultrasound images. For example, pixel values in natural images are always positive, while} 
%in ultrasound image reconstruction, 
 \DM{the reconstructed ultrasound 
$\xv$ contains both positive and negative values, and 
%ultrasound images 
are typically displayed after log compression.}
\YZ{Therefore, fine-tuning existing models with a small amount of ultrasound data may lead to better results.
}

\YZ{
Although DDRM relies on the computationally expensive SVD, it does not affect imaging time since} 
%its results can be saved and repeatedly used. 
\DM{ the SVD results can be precomputed.}
\YZ{However, the multiplication operation} \DM{ with the singular vector matrix 
%between the singular matrix and other vectors 
}
\YZ{during the image reconstruction process currently hinders real-time imaging. On our machine equipped with the GPU NVIDIA Quadro RTX 3000, each iteration takes approximately 4.5 seconds. Accelerating this process is one of our key focuses for future work.}

\YZ{
In conclusion, for the first time, we achieve the reconstruction of ultrasound images with} 
%a diffusion model and test two ultrasound models, DRUS and WDRUS. 
\DM{ two adapted diffusion models, DRUS and WDRUS. }
\YZ{Our method with single plane wave is even comparable to DAS with 75 plane waves,}
%which are often used to produce target images, in the case where the generative model has never been trained on ultrasound data.
\DM{which is often used as reference to train generative models, whereas our diffusion model was never trained on ultrasound data}
\end{comment}