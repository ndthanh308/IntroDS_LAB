Ultrasound (US) imaging is a popular non-invasive imaging modality, of
%that has been found
wide\-spread use in medical diagnostics due to its safety and cost-effectiveness tradeoff. Standard commercial scanners rely on simple beamforming algorithms,  e.g.\ Delay-and-Sum (DAS), to transform raw signals into B-mode images, trading spatial resolution for speed. Yet, many applications could benefit from improved resolution and contrast, enabling better organ and lesion boundary detection.

Recent techniques to improve US image quality include adaptive beamforming techniques, e.g.\ based on Minimum Variance (MV) estimation \cite{synnevag_adaptive_2007,asl_eigenspace-based_2010}, or Fourier-based reconstructions~\cite{Chernyakova-Eldar_2018}. Other methods focus on optimizing either pre-\cite{REFOCUS,khan_real-time_2021} or post-processing steps \cite{laroche_fast_2021}.
Today, there is an increasing interest in model-based approaches \cite{IPB_Ozkan,RED_USIPB} that better formalize the problem within an optimization framework. 
%unifying beamforming and inverse problems techniques
%
A second branch of methods for improving US image quality leverages the power of Deep Neural Networks (DNNs). Initial approaches in this direction have been trained to predict B-mode images directly \cite{Hyun19}, the
beamforming weights \cite{luijten_adaptive_2020,MNV2} or used as post-processing denoisers under supervised training schemes~\cite{perdios_cnn-based_2022,AUGAN_2021}. Despite their effectiveness, these methods require datasets of corresponding low-high quality image pairs and therefore do not generalize to other organs/tasks. 

Recent hybrid approaches have focused on improving generalizability by combining the best of the model-based and learning worlds. For instance, %Perdios et al. \cite{perdios_cnn-based_2022} use an NN as the denoiser step after a DAS reconstruction;
%Goudarzi IUS 2020 -> trains a mobilnet to mimic MVB , so , not hybrid
Chennakeshava et al.\ \cite{Chennakeshava:ius2020} propose an unfolding plane-wave compounding method, while Youn et al. \cite{youn:ius2020} combine deep beamforming with an unfolded algorithm for ultrasound localization microscopy. Our work falls within this hybrid model-based deep learning family of approaches \cite{van_sloun_deep_2020}.



% -> The work by Nair et al. (Nair et al., 2018, Nair et al., 2020) does this implicitly, by finding a direct mapping from the time domain to an output image, using DL.
%-> Kim et al. (Kim et al., 2021) adheres more strictly to a conventional beamforming structure and tackled this problem in two steps: first, the estimation of a local speed-of-sound map, and second, the calculation of the corresponding beamforming delays.
%An extension of this work is described in Khan et al.Khan et al., 2021, Khan et al., 2021), in which the neural network itself is replaced by a model-based network architecture.

%Chennakeshava et al. \cite{Chennakeshava:ius2020}  propose an unfolding plane-wave compounding method while Solomon et al. and van Sloun et. al \cite{Solomon:TMI2019,van_sloun_deep_2020} relying on unfolding for source separation.

%Arun Asokan Nair, Trac D Tran, Austin Reiter, and Muyinatu A Lediju Bell. A generative adversarial neural network for beamforming ultrasound images, 2019.
%Arun Asokan Nair, Trac D Tran, Austin Reiter, and Muyinatu A Lediju Bell. A deep learning based alternative to beamforming ultrasound images, 2018.
%
%Sanketh Vedula, Ortal Senouf, Grigoriy Zurakhov, Alex Bronstein, Oleg Michailovich, and Michael Zibulevsky. Learning beamforming in ultrasound imaging, 2018.
%
%Dimitris Perdios, Adrien Besson, Marcel Arditi, and Jean-Philippe Thiran. A deep learning approach to ultrasound image recovery, 2017.
%
%Walter Simson, Rudiger Göbl, Magdalini Paschali, Markus Krönke, Klemens Scheidhauer, Wolfgang Weber, and Nassir Navab. End-to-end learning-based ultrasound reconstruction, 2019
%
%Shujaat Khan, Jaeyoung Huh, and Jong Chul Ye. Universal deep beamformer for variable rate ultrasound imaging, 2019.
%
%Adam C Luchies and Brett C Byram. Deep neural networks for ultrasound beamforming, 2018.

%AA Nair, KN Washington, TD Tran, A Reiter, MAL Bell, Deep learning to obtain simultaneous image and segmentation outputs from a single input of raw ultrasound channel data, 2020

%Ultrasound Beamforming with Deep Learning (CUBDL) at the 2020 IEEE International Ultrasonics Symposium (IUS) [11], [12].

We propose the use of DNN image generators to determine and explore the available solution space for the US image reconstruction problem. In practice, we leverage the recent success of Denoising Diffusion Probabilistic Models (DDPMs) \cite{ho_denoising_2020,nichol_improved_2021,dhariwal_diffusion_2021},  which are the state-of-the-art in image synthesis in the domain of natural images. More specifically, we build on the Denoising Diffusion Restoration Models (DDRMs) framework proposed by Kawar et al.~\cite{kawar_denoising_2022}, which adapts DDPMs to various image restoration tasks modeled as linear inverse problems. The main advantage of DDRMs is exploiting the direct problem modeling to bypass the need to retrain DDPMs when addressing new tasks. While the combination of model-based and diffusion models has been explored in the context of CT/MRI imaging \cite{song_solving_2022}, this is, to the best of our knowledge, the first probabilistic diffusion model approach for ultrasound image reconstruction. 

Our methodological contributions are twofold. First, we adapt DDRMs from restoration tasks in the context of natural images (e.g.\ denoising, inpainting, superresolution), to the reconstruction of B-mode US images from raw radiofrequency RF channel data. Our approach can be applied to different acquisition types, e.g. sequential imaging, synthetic aperture, and plane-wave, as long as the acquisition can be approximately modeled as a linear inverse problem, i.e.\ with a model matrix depending only on the geometry and pulse-echo response (point spread function).  Our second contribution is introducing a whitening step to cope with the direct US imaging model breaking the \textit{i.i.d.} noise assumption implicit in diffusion models. In addition to the theoretical advances, we provide a qualitative and quantitative evaluation of the proposed approach on synthetic data under different noise levels, showing the feasibility of our approach. Finally, we also demonstrate results on the PICMUS dataset. 