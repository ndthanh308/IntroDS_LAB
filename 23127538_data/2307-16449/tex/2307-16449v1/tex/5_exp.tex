

\begin{table*}[t]
\tiny
\centering
\resizebox{0.99\linewidth}{!}{
\begin{tabular}{p{2cm}|p{3cm}|p{3cm}|p{3cm}|p{0.35cm}|p{0.44cm}|p{0.3cm}|p{0.3cm}|p{0.3cm}}
\toprule
prompt & real completion &  no watermark (NW) &  watermarked (W) &(NW) $z$   &(W) $z$ &(Real) PPL & (NW) PPL & (W) PPL \\
\midrule
DPR members Jim Ragsdale, Diane Kane, Angeles Liera and pro tem chair Mike Costello discuss condo conversion projects.\textbackslash n During the Jan  & .10 meeting of the La Jolla Development Permit Review committee (DPR), board members voted unanimously to form a research subcommittee that will look into the consequences of condo conversion in the neighborhoods south of Pearl Street.[...continues] &  . 27 meeting, Ragsdale and Kane discussed the need for a condo market and how to get there. Costello spoke about the need to have a condo market in the area but also said there is a need to be able to rent a condo and that the area is growing. [...continues]& . 11-12 meeting, the city announced that the development of condos to be built in the historic downtown has been approved by the city.\textbackslash n \"We feel it's important to be able to provide affordable housing for the people of the city in a way that the community feels they [...continues] &  0.71 &  10.5 & 5.42& 8.26 &  7.15 \\\midrule
In their first game since dropping out of the top five, the Irish delivered a redemption performance against Boston College, picking up a 50-point win over &    
 the Eagles while simultaneously moving one step closer to cementing Arike Ogunbowale’s legacy, as the senior guard passed current associate coach Beth Cunningham on the list of all-time scorers in the program.\textbackslash n No. 6 Notre Dame (23-2, 10-2 ACC) wasted [...continues] 
&    the Eagles. The Irish also defeated the Bulldogs in the final and will face the Bulldogs in the final.\textbackslash n The Irish also defeated the Eagles in the final and will face the Bulldogs in the final. Boston College:\textbackslash n The Eagles had a very good game against the Irish, [...continues] 
& South Carolina in the College Football Playoff Tournament to secure a berth in the NCAA Tournament. The Eagles will meet Notre Dame in the Big 12 Tournament on Sept. 14 in Austin, Texas.\textbackslash n  n\"We've got to win in the first round [...continues] &  1.13 &      11.3 & 4.78 & 6.78 &  9.15 \\\midrule
Two officers of a company that operates three Manhattan hotels were indicted yesterday in a scheme to help homeless people fraudulently obtain welfare checks and split the money  &   with the hotels.\textbackslash n District Attorney Robert M. Morgenthau of Manhattan, who announced the indictments, said they resulted from a study of the three hotels announced last December by the city. He said his office was delayed in moving more quickly on the case because of difficulty[...continues] &  they received from the government. \textbackslash n The indictment, which was released on Tuesday, said that police officers, who arrived at the hotel on a routine shift, met with a homeless man who asked for a check from his landlord. man told the officers about the scheme, which involved[...continues] &  between themselves and the homeless.\textbackslash n The scheme, alleged to be connected to the New York City Department of Health and Welfare, was uncovered in the wake of the 2012 Sandy Hook Elementary School shooting and the 2011 bombing of the Boston Marathon. Authorities say that the scheme  [...continues] &    -1.9 &  9.94 & 4.83 & 7.02 &  7.05 \\\midrule
Buddhadev had written a strong letter of protest to Manmohan Singh objecting to Mulford's behaviour. \textbackslash n Taking serious exception to  & US Ambassador David Mulford writing directly to West Bengal Chief Minister Buddhadev Bhattacharjee for his remarks against the American President, the CPI-M on Friday said the party[...continues] &  the comments made by Mulford, the BJP MP also called on the CM to resign immediately and the Centre to make a statement in the coming weeks.\textbackslash n In his letter to Manmohan Singh, the MP said he was not opposed to[...continues] & Mr Mulford's behaviour in the media and in the Parliament, the Union Minister has directed the Union Secretaries of Parliament and the Secretaries of the Supreme Court to take action against him in the matter[...continues] &  1.85 & 12.36 & 4.25 & 7.35 &      8.15 \\
\bottomrule
\end{tabular}}
\caption{Selected output examples from non-watermarked (NW) and watermarked (W) top-K sampling using $\gamma=0.5$, $\delta=2.0$ and $k=20$.  
}
\label{tab:demo-examples}
\end{table*}


\section{Experiments}
\begin{table}[t]
\centering
\Large
\setlength{\tabcolsep}{8pt}
\renewcommand{\arraystretch}{1.3}
\resizebox{\linewidth}{!}{
\begin{tabular}{l c c c c c c}
\toprule
\textbf{Method} & \multicolumn{2}{c}{\textbf{MSVD-QA}} & \multicolumn{2}{c}{\textbf{MSRVTT-QA}} & \multicolumn{2}{c}{\textbf{ActivityNet-QA}} \\
\cline{2-7}
 & \textbf{Accuracy} & \textbf{Score} & \textbf{Accuracy} & \textbf{Score} & \textbf{Accuracy} & \textbf{Score}\\
\midrule
\midrule
FrozenBiLM~\cite{yang2022zero} & 32.2 & -- & 16.8 & -- & 24.7 & -- \\
Video Chat~\cite{li2023videochat} & 56.3 & 2.8 & 45.0 & 2.5 & 26.5 & 2.2 \\
LLaMA Adapter~\cite{zhang2023llama} & 54.9 & 3.1 & 43.8 & 2.7 & 34.2 & 2.7 \\
Video LLaMA~\cite{zhang2023video} & 51.6 & 2.5 & 29.6 & 1.8 & 12.4 & 1.1 \\
Video-ChatGPT~\cite{maaz2023video} & \textbf{64.9} & \textbf{3.3} & \underline{49.3} & \underline{2.8} & \underline{35.2} & \underline{2.7} \\ 
\midrule
MovieChat~\textit{(Ours)} & \underline{61.0} & \underline{2.9} &\textbf{49.7} &\textbf{2.8} & \textbf{51.5} & \textbf{3.1}\\
\bottomrule
\end{tabular}
}
\caption{Quantitative evaluation on short video question answering. MovieChat achieves competitive results.}
\end{table}
\subsection{Quantitative Evaluation}
\label{exp:quantitative}

\paragraph{Short video question answering.} We conducted a comprehensive quantitative evaluation in this section. We use several widely used open-ended datasets: MSVD-QA~\cite{xu2017video}, MSRVTT-QA~\cite{xu2017video}, and ActivityNet-QA~\cite{yu2019activitynet} for short video question answering task. The evaluation process is under the assistant of LLM (details in Appendix~\ref{sec:gpt-eval}) under default hyper-parameter settings as shown in Appendix~\ref{sec:hyper-param}. The accuracy and relative score on a scale of $0$ to $5$ are reported. Compared to previous method~\cite{maaz2023video,li2023videochat,yang2022zero}, MovieChat achieves competitive result although there is no specific design for short video understanding.

\subsection{Case Study}
We perform an extensive case study of MovieChat on a variety of open-ended long video~(such as cartoon movie in and TV series) for long video question-answering and captioning task, including the \textcolor{global}{\textbf{global mode}} and the \textcolor{breakpoint}{\textbf{breakpoint mode}} as shown in Figure~\ref{fig:case}. For Q\#1 and Q\#2, we annotate timestamps in frames. For long videos over $10$K frames, MovieChat is still capable of providing excellent responses to questions regarding both the current moment and the entire video content.