\vspace{8pt}
\documentclass[prl,twocolumn,amsmath,amssymb,superscriptaddress]{revtex4-2}

\usepackage{bm}                     
\usepackage{appendix}
\usepackage[latin1]{inputenc}
\usepackage{dcolumn}      
\usepackage{bbm}
\usepackage{color}
\usepackage{xcolor}

\usepackage[mathscr]{eucal}
\usepackage{latexsym}
\usepackage{amsbsy}
\usepackage{float}
\usepackage{graphicx}
\usepackage{epsfig}
\usepackage{subfigure}
\usepackage{wrapfig}
\usepackage{epsf}
\usepackage{float}
\usepackage[normalem]{ulem}
\usepackage{qcircuit}

\definecolor{darkblue}{HTML}{004D6B}
\definecolor{darkred}{HTML}{8c1515}
\definecolor{darkgreen}{HTML}{006400}
\usepackage{hyperref}
\hypersetup{ colorlinks=true, urlcolor=darkblue, citecolor=darkred,
    linkcolor=darkblue, breaklinks }

\newcommand{\ba}{\begin{array}}
\newcommand{\ea}{\end{array}}
\newcommand{\be}{\begin{equation}}
\newcommand{\ee}{\end{equation}}
\newcommand{\bea}{\begin{eqnarray}}
\newcommand{\eea}{\end{eqnarray}}


%\newcommand{\new}[1]{\textcolor{blue}{#1}}
\begin{document}

\title{Approaching ideal rectification in superconducting diodes through multiple Andreev reflections}

\author{A. Zazunov}
\affiliation{Institut f\"ur Theoretische Physik, Heinrich-Heine-Universit\"at, D-40225  D\"usseldorf, Germany}
\author{J. Rech}
\affiliation{Aix Marseille Univ., Universit\'e de Toulon, CNRS, CPT, Marseille, France}
\author{T. Jonckheere}
\affiliation{Aix Marseille Univ., Universit\'e de Toulon, CNRS, CPT, Marseille, France}
\author{B. Gr{\'e}maud}
\affiliation{Aix Marseille Univ., Universit\'e de Toulon, CNRS, CPT, Marseille, France}
\author{T. Martin}
\affiliation{Aix Marseille Univ., Universit\'e de Toulon, CNRS, CPT, Marseille, France}
\author{R. Egger}
\affiliation{Institut f\"ur Theoretische Physik, Heinrich-Heine-Universit\"at, D-40225  D\"usseldorf, Germany}

\date{\today}

\begin{abstract}
We analyze the rectification properties of voltage-biased Josephson junctions exhibiting the superconducting diode effect. Taking into account multiple Andreev reflection (MAR) processes in our scattering theory, we consider a
short weak link of arbitrary transparency between two helical superconductors with finite
Cooper pair momentum $2q$.  In equilibrium, the diode efficiency is bounded from above in this 
model, with maximal efficiency $\eta_0\approx 0.4$. Out of equilibrium, we find a rich subharmonic structure in the current-voltage curve.  For high transparency and 
low bias voltage $V$, the rectification efficiency $\eta(V)$ approaches the ideal value $\eta=1$ for
$q\xi\to 1$ (with coherence length $\xi$). 
\end{abstract}
\maketitle

\emph{Introduction.---}Starting with the work by Ando \emph{et al.}~in 2020 \cite{Ando2020}, 
a surge of experiments reported evidence for the superconducting diode effect (SDE)  \cite{Lyu2021,Bauriedl2022,Baumgartner2022,Pal2022,Lin2022,Wu2022a,Jeon2022,Turini2022,Sundaresh2023,Mazur2023,Anwar2023,Banerjee2023,Ghosh2023,Hou2023a,Costa2023}, 
see Ref.~\cite{Nadeem2023} for a review.  Even though microscopic mechanisms behind the SDE are not yet 
fully understood, many aspects have been clarified by theoretical works   \cite{Edelstein1996,Hu2007,Reynoso2008,Zazunov2009,Misaki2021,Ilic2022,He2022,Zhang2022,Davydova2022,Kokkeler2022,Daido2022a,Daido2022b,Tanaka2022,Zinkl2022,Cheng2023,Ikeda2023,He2023a,Lu2023,Fu2023,Wang2023a,Yuan2023b,Legg2023,Nakamura2023,Picoli2023}, and the hope is that practically useful device applications will emerge soon. In essence, the SDE amounts to an asymmetry between the (absolute value of the) 
critical supercurrent flowing to the right ($I_{c+}>0$) and to the left ($I_{c-}<0$).
Assuming $|I_{c-}|<I_{c+}$, the
SDE efficiency is defined by $\eta_0 =(I_{c+}+I_{c-})/(I_{c+}-I_{c-})$,
where a dissipationless supercurrent $I$ can only flow to the right if $|I_{c-}|<|I|<I_{c+}$.  
We consider the intrinsic SDE in a single Josephson junction (the SDE is also possible in junction-free bulk 
superconductors, see, e.g., Refs.~\cite{Ando2020,Bauriedl2022,Nadeem2023}, and
in more complex multi-junction devices \cite{Souto2022,Fominov2022,Paolucci2023,Gupta2023,Ciaccia2023,Zhang2023expA}),
where the conditions for the anomalous Josephson effect 
\cite{Buzdin2008,Reynoso2008,Zazunov2009,Reynoso2012,Brunetti2013,Dolcini2015,Szombati2016,Qin2017}
have to be met. In particular, time-reversal and inversion symmetries must be broken.  In addition, 
the equilibrium current-phase relation must contain contributions from higher harmonics \cite{Reynoso2008,Zazunov2009,Bauriedl2022,Baumgartner2022}.

We here study the out-of-equilibrium behavior of
Josephson junctions exhibiting the SDE in equilibrium, 
in particular,  the DC current-voltage ($I$-$V$) curve of a voltage-biased
intrinsic Josephson diode.  At low temperatures,  
MAR processes \cite{Klapwijk1982,Bratus1995,Averin1995,Cuevas1996} 
can then provide the dominant transport mechanism, especially for subgap voltages $e|V|<2\Delta$
(with pairing gap $\Delta$).  We focus on junctions with a single (or a few uncoupled) 
channels, where the impedance is of order $h/e^2$ and thus much larger than the typical
impedance of the external circuit. We then do not have to account for
the self-consistent dynamics of the phase difference and voltage across the junction. 
Previous studies of nonequilibrium transport in Josephson diodes have considered weakly damped 
low-impedance junctions \cite{Misaki2021,Fominov2022,Trahms2023,Steiner2023} 
or externally driven junctions \cite{Paaske2023}, but MAR effects have not been addressed.
Our theory predicts a characteristic voltage-dependent 
rectification pattern, $I(-V)\ne -I(V)$, quantified by the efficiency parameter
\begin{equation}\label{efficiency}
    \eta(V) = \frac{I(V)+I(-V)}{I(V)-I(-V)},
\end{equation}
where $\eta(V)$ is especially large in the subgap regime. 
For the conventional case without SDE, MAR causes a subharmonic structure,
i.e., singular features in the nonlinear conductance for $eV=2\Delta/n$ with integer $n$ \cite{Klapwijk1982,Bratus1995,Averin1995}.   
For Josephson diodes, we predict an even richer subharmonic structure which determines
the rectification characteristics and might provide precious information about 
the microscopic mechanisms generating the SDE. 
Our central finding is that the efficiency $\eta(V)$ can approach the ideal limit of full rectification with $\eta=1$
at low voltages, even though $\eta_0\alt 0.4$ for the SDE efficiency in equilibrium for the model 
considered below.  Significant rectification efficiency persists also away from the ideal conditions discussed below, including the case of voltages well above the pairing gap.
 
It is well known that the SDE can arise from magnetochiral effects \cite{Rikken2001,Tokura2018,Morimoto2018,Legg2022} 
in noncentrosymmetric superconductors \cite{Edelstein1996}. 
An alternative mechanism arises from the finite Cooper pair
 momentum $2q$ in a helical superconductor \cite{Davydova2022,Yuan2022,Pal2022,Lin2022,Yuan2022,Banerjee2023}.
We here consider a weak link connecting two helical superconductors with identical
$q$, where the current-phase relation computed in Ref.~\cite{Davydova2022} implies the SDE. 
The simplicity of the corresponding model allows us to determine the full $I$-$V$ curve without approximations from  scattering theory, where
known results \cite{Bratus1995,Averin1995,Zazunov2006} are recovered for $q=0$.    
A detailed account  is given in Ref.~\cite{PRB}, where we also address the SDE efficiency $\eta_0$ in equilibrium and the $I$-$V$ curve for an NS
contact between a normal metal and a helical superconductor.  
In this Letter, we summarize the salient features of the theory and discuss the rectification efficiency $\eta(V)$ of the Josephson diode model with $q\ne 0$.
 
\emph{Model.---}For finite Cooper pair momentum $2q$,
the order parameter of an $s$-wave BCS superconductor oscillates in space, 
$\Delta(x)=\Delta e^{2iqx}$,
where $q\ne 0$ may originate from the interplay of the spin-orbit interaction with
a Zeeman field in superconducting films \cite{Daido2022a,He2022,Yuan2022,Levichev2023},
or from magnetic proximity and/or Meissner effects \cite{Davydova2022}. 
In either case, time-reversal and inversion symmetries are broken for $q\ne 0$.  
Following Ref.~\cite{Davydova2022}, we study a short single-channel weak link between two superconducting banks with the 
same pairing gap and Cooper pair momentum. The coherence length is $\xi=\hbar v_F/\Delta$ with Fermi velocity $v_F$.
For definiteness, we assume $0 \le q \xi < 1$ since the superconductor becomes 
gapless for $q\xi \ge 1$.  

Linearizing the band dispersion around the Fermi momentum points $\pm k_F$ with $k_F\xi\gg 1$, 
the Hamiltonian is expressed in terms of effectively one-dimensional quasiclassical Nambu spinor envelopes,
$\psi_{\pm}(x,t)=(\psi^{}_{\pm,\uparrow},\psi^\dagger_{\pm,\downarrow})^T$,
for right- and left-movers having momenta $\pm k_F+k$ with $|k|\ll k_F$,
resp., where $x<0$ ($x>0$) refers to the left (right) superconducting bank.
Gauging away the $e^{2iqx}$ factor from the order parameter, 
the Bogoliubov-de~Gennes (BdG) Hamiltonian for $x\ne 0$ follows as (we often put $\hbar= 1$) \cite{Davydova2022}
\begin{equation}\label{BdG}
H_{\rm BdG}  =  -iv_F \sigma_z\tau_z \partial_x + v_F q \sigma_z\tau_0 + \Delta\sigma_0\tau_x,
\end{equation} 
where we use Pauli matrices $\tau_{x,y,z}$ (and identity $\tau_0$) in Nambu space
and $\sigma_{x,y,z,0}$ in chiral (right-left mover) space.
Defining the bispinor $\Psi(x,t)=(\psi_+,\psi_-)^T$ in chiral space,
modeling the weak link as normal-conducting constriction with length much shorter than $\xi$ and 
transmission probability ${\cal T}$, and using the phase difference $\varphi(t)=2eVt$ 
across the junction, we arrive at a matching condition 
connecting the bispinors on the left ($x=0^-$) and right ($x=0^+$) side, see also Refs.~\cite{Zazunov2005,Nazarov2009,Zazunov2014,Ackermann2023},
\begin{equation}\label{BC}
\Psi(0^-,t) = \frac{1}{\sqrt{\cal T}} (\sigma_0+r\sigma_x) \,  e^{i\tau_z eV t} \, \Psi(0^+,t),
\end{equation}
with the reflection amplitude $r=\sqrt{1-{\cal T}}$.  

% Figure environment removed

\emph{Spectral properties.---}Consider first a ballistic junction (${\cal T}=1$) at zero voltage, where Eq.~\eqref{BC} is automatically fulfilled by continuous wave functions
and one recovers the ``bulk'' case. BdG eigenstates have conserved energy $E$ and chirality
$\alpha=\sigma_z=\pm$, where Eq.~\eqref{BdG} gives a quasiparticle dispersion with four branches in total, see Fig.~\ref{fig1},
\begin{equation}\label{bulkenergy}
  E(k) = \pm \sqrt{(v_F k)^2+ \Delta^2}  + \alpha v_F q.
\end{equation} 
From Eq.~\eqref{bulkenergy}, there are \emph{two} positive threshold energies for the quasiparticle continuum, $\Delta_\pm=\Delta\pm v_Fq$, which are Doppler shifted away from $\Delta$ due to the finite Cooper pair momentum.  

For transparency ${\cal T}<1$, Andreev bound state solutions localized near the junction at $x=0$ can then only exist for energies inside both spectral gaps, $|E|<\Delta_-$.
On the other hand, for $|E|>\Delta_+$, propagating continuum states are possible along both directions, while for $\Delta_-<|E| <\Delta_+$, we have mixed-character states which can 
freely propagate along one direction but are evanescent along the other. 
As an important technical step forward, we specify BdG solutions that apply 
in a unified manner to all three energy regions.  We choose a formulation that can be 
leveraged to describe scattering states for finite voltage.
We first observe that for $x\ne 0$, using the Doppler-shifted energy $E_\alpha= E-\alpha v_F q$ for an $\alpha$-mover ($\alpha=\pm$) with energy $E$, 
Eq.~\eqref{BdG} implies that  electron ($e$) and hole ($h$) type states have the Nambu spinor structure 
\begin{equation}\label{tildepsi}
\tilde \psi_{\alpha,e}(E) = \frac{1}{\sqrt2}\left(\begin{array}{c} 1 \\ \rho(E_\alpha) \end{array}\right),
\quad \tilde \psi_{\alpha,h}(E) =\tau_x \tilde\psi_{\alpha,e}(E), 
\end{equation}
with the Andreev reflection amplitude
\begin{equation}\label{tildegamma} 
\rho(E)=  \left\{\begin{array}{cc}  
{\rm sgn}(E)\, \frac{|E|- \sqrt{E^2-\Delta^2}}{\Delta}, & |E|\ge \Delta,\\ & \\
\frac{E-i \sqrt{\Delta^2-E^2}}{\Delta}, & |E| <\Delta.
\end{array} \right. 
\end{equation} 
The states \eqref{tildepsi} describe arbitrary energies and are very useful for the description of outgoing (scattered) states, even 
though they satisfy unconventional normalization conditions.  On the other hand,
incident states, $\psi_{\alpha,e/h}(E)$, should satisfy the standard normalization condition
$\psi_{\alpha,e/h}^\dagger(E)\cdot \psi_{\alpha,e/h}^{}(E)=1$.
Since incident states are only defined for $|E_\alpha|>\Delta$, we can simply obtain them
from Eq.~\eqref{tildepsi} by including a normalization factor,
$\psi_{\alpha,e/h}(E)=\sqrt{\frac{2}{1+\rho^2(E_\alpha)}} \,\tilde\psi_{\alpha,e/h}(E)$.

\emph{MAR scattering states.---}We next construct scattering states for the finite-voltage case taking into account MAR processes.  Typical MAR trajectories in energy space (``MAR ladder'') are shown in Fig.~\ref{fig1}.  We consider an incident $\alpha$-mover which is an electron or hole like quasiparticle with energy $E$ in the respective continuum, $|E_\alpha|>\Delta$. 
For each step of the MAR ladder sketched in Fig.~\ref{fig1}, 
the energy of an electron changes by $\pm eV$ for right- or left-movers 
when traversing the normal junction region, and similarly the energy shift for holes is $\mp eV$.
The energy $E_n$ of an outgoing (reflected or transmitted) state may therefore involve the 
emission or absorption of an integer number of  $eV$ quanta,  $E_n=E+neV$ with integer $n$.  
Noting that there are four possible types of incident states, labeled by $s\in \{1,2,3,4\}$ depending on whether an electron- or a hole-type state is injected from the left or from the right side, we obtain a general \emph{Ansatz}
for  MAR scattering states. For $x=0^\pm$, the corresponding bispinor states have the form 
\begin{eqnarray} \nonumber
\Psi_E(0^-,t) &=&e^{-iEt}  \left( \begin{array}{c} \delta_{s,1}\, \psi_{+,e}(E) \\ \delta_{s,2} \, \psi_{-,h}(E)\end{array} \right) \\
\nonumber &+& \sum_n e^{-iE_n t}\left( \begin{array}{c} a_n\tilde\psi_{+,h}(E_n) \\ b_n \tilde\psi_{-,e}(E_n)
\end{array} \right), \\ \label{MARAnsatz}
\Psi_E(0^+,t) &=&e^{-iEt}  \left( \begin{array}{c} \delta_{s,3} \,\psi_{+,h}(E) \\ \delta_{s,4} \,\psi_{-,e}(E)\end{array} \right)\\
&+&\nonumber \sum_n e^{-iE_n t}
\left( \begin{array}{c} c_n \tilde\psi_{+,e}(E_n) 
\\ d_n \tilde\psi_{-,h}(E_n)\end{array} \right).
\end{eqnarray}
Keeping the incident quasiparticle energy $E$ and scattering channel $s$ implicit,  
the complex-valued scattering amplitudes $(a_n,b_n,c_n,d_n)$ appearing in the outgoing spinor ($\tilde \psi_{\alpha,e/h})$ contributions 
in Eq.~\eqref{MARAnsatz} are determined from the matching conditions in Eq.~\eqref{BC}.
They are also indicated in Fig.~\ref{fig1}.  As a result, the scattering amplitudes
satisfy a set of recurrence relations encoding the MAR ladder, see Ref.~\cite{PRB} for their explicit form.  

Given a solution of the recurrence relations, using the Fermi function $n_F(E)=1/(e^{E/T}+1)$ and superconducting density of states factors ($\Theta$ is the Heaviside function),
$\nu_{\alpha=\pm} (E) =  \frac{|E_{\alpha}|}{\sqrt{E_{\alpha}^2-\Delta^2}}
\Theta(|E_{\alpha}|-\Delta),$
the $I$-$V$ characteristics follows as
\begin{equation}\label{MARcur}   
I(V) = \frac{e}{2h} \sum_{\alpha=\pm}\int dE\, n_F(E)\nu_\alpha(E) 
    I_\alpha(r,E)+ (r\to -r),
\end{equation}
with the reflection amplitude $r$  in Eq.~\eqref{BC} and the current matrix elements 
 \begin{eqnarray}\nonumber
I_\alpha (r,E) &=&  \sum_{{\rm odd}\,n} \Bigl [ 
|c^{}_{\alpha,n}|^2 \left(1+|\rho(E+neV-v_Fq)|^2\right) \\
&-&  |d_{\alpha,n}|^2 \left(1+|\rho(E+neV+v_Fq)|^2\right)
 \Bigr]. \label{iam}
 \end{eqnarray}
By taking advantage of symmetry relations connecting the solutions incident from the left side ($s=1,2$) to 
those incident from the right side ($s=3,4$), the latter solutions are contained in Eq.~\eqref{MARcur} through 
the term with $r\to -r$.  The index $\alpha$ in Eq.~\eqref{iam} then corresponds to $s=1$ (for $\alpha=+$) and  $s=2$ (for $\alpha=-$).  
The current expression in Eq.~\eqref{MARcur}  affords a transparent physical interpretation.  Summing over all scattering channels $s$ and integrating
over all incident energies $E$, the
weight of the corresponding incident state in the current is determined by the product of the Fermi function, the density of states, 
and a current matrix element.  The latter follows by summing over all orders $n$ of the MAR ladder, where current contributions
only arise  for odd $n$.  At given order $n$,  electrons $(\propto |c_{\alpha,n}|^2)$ and holes $(\propto |d_{\alpha,n}|^2)$ enter with 
opposite sign, where the corresponding Doppler-shifted energy $E_n\mp v_Fq$ appears in the Andreev reflection amplitude $\rho(E)$.
 
For arbitrary system parameters, which are represented by the four dimensionless quantities $q\xi$, ${\cal T}$, $k_BT/\Delta$, and $eV/\Delta$, 
the rectification efficiency $\eta(V)$ in Eq.~\eqref{efficiency} follows from Eq.~\eqref{MARcur} 
after a numerical solution of the recurrence relations.  The order at which the recurrence relations can be truncated is given by $n_{\rm max}\sim 2\Delta/e|V|$. 
It is thus numerically difficult to reach extremely low voltages for high transparency ${\cal T}\alt 1$, where the MAR ladder in Fig.~\ref{fig1} includes a very large number of round trips in the junction region.  Our code accurately reproduces the $q=0$ results for $I_{q=0}(V)$ reported in Ref.~\cite{Averin1995}.  Another check passed by our code
comes from the ballistic limit ${\cal T}=1$, where the recurrence relations can be solved analytically for arbitrary other parameter values.  We next describe the corresponding results in the zero-temperature limit.

\emph{Ballistic limit.---}For ${\cal T}=1$, the matching conditions \eqref{BC} 
as well as the BdG Hamiltonian conserve chirality, $\sigma_z=\alpha=\pm$, and the 
recurrence relations admit a closed solution  \cite{PRB}.
We find that $I(V)$ for $q\ne 0$ is related to the known $q=0$ curve $I_{q=0}(V)$ \cite{Averin1995}  by
a simple shift,
\begin{equation}\label{currdopplershift}
    I(V) = I_{q=0}(V)+ \frac{4e\Delta}{h}  q\xi.  
\end{equation}
This shift has a clear physical interpretation: it is the current carried by Cooper pairs with finite momentum $2q$ and charge $2e$.  The simple decomposition \eqref{currdopplershift} only 
applies in the ballistic limit where chirality is conserved.   
The rectification efficiency  then follows for arbitrary voltage from Eq.~\eqref{efficiency} as
\begin{equation}\label{rectball}
    \eta(V,q\xi,{\cal T}=1) = \frac{4e\Delta}{h} \frac{q\xi}{I_{q=0}(V)},
\end{equation}
where the dependence on $q\xi$ is a simple proportionality.  Since we consider the regime $0\le q\xi<1$,
maximal efficiency is reached for $q\xi \to 1$.  This is 
in contrast to the equilibrium SDE case, where the maximal SDE efficiency $\eta_0$ is found for 
$q\xi\approx 0.9$, with $\eta_0\approx 0.4$ \cite{Davydova2022,PRB}.
In both cases, however, the optimal conditions for rectification correspond to full junction
transparency with ${\cal T}=1$.

For $e|V|\gg \Delta$, the Ohmic result of the corresponding normal-normal contact is approached, $I_0(V)\approx (2e^2/h)V$, implying $\eta(V) \simeq  2q\xi \frac{\Delta}{eV}$.
On the other hand, for $V\to 0$, using $I_{q=0}(V\to 0)\approx (4e \Delta/h)\,{\rm sgn}(V)$ \cite{Averin1995}, Eq.~\eqref{rectball} implies $\eta(V)\simeq q\xi.$ 
For $q\xi\to 1$, one approaches the ideal rectification limit since the MAR-induced current $I_0$ now
precisely cancels the finite-momentum Cooper pair current for $V<0$, i.e, $I(V<0)=0$ in Eq.~\eqref{currdopplershift},
while both currents add for $V>0$ to give $I(V>0)=8e\Delta/h$.  As a result, we have $\eta(V)=1$.
We conclude that MAR processes can generate highly efficient superconducting diode 
behavior in the deep subgap regime $e|V|\ll \Delta$.   As we show next, this enhancement of $\eta(V)$ compared to the equilibrium value $\eta_0$ (for otherwise identical parameters) is also found  for non-ideal transparency.



% Figure environment removed
% Figure environment removed


\emph{Subharmonic structure.---}In Fig.~\ref{fig2}, we show numerical results for 
$\eta(V)$ for different values of $(q\xi,{\cal T})$, again assuming the zero-temperature limit.  
We observe an overall increase of $\eta(V)$ with increasing Cooper pair 
momentum $2q$ and/or junction transparency ${\cal T}$.
The efficiency is particularly large in the subgap regime $eV\alt 2\Delta$, where we also observe a subharmonic structure with peaks or dips.  These MAR features are more clearly visible 
in the derivative $d\eta(V)/dV$ (bottom panel in Fig.~\ref{fig2}).
Apart from the standard $q=0$ MAR features at $2\Delta/eV=n$ (integer $n$),
which are also observed for $q\ne 0$ and follow from MAR trajectories as 
drawn in the upper panel in Fig.~\ref{fig1}, we also find 
resonances or antiresonances corresponding to the Doppler-shifted pairing gaps $\Delta_\pm$ (indicated by arrows in Fig.~\ref{fig2}). 
The corresponding transitions are naturally explained from the MAR ladder picture shown in the lower panel of Fig.~\ref{fig1}, where the presence of normal reflection $r\ne 0$ enables
MAR trajectories between states near the same type of spectral gap ($\pm\Delta_+$ or $\pm\Delta_-$) where $\nu_\alpha(E)$ has sharp peaks.  Let us also note that for $eV\gg \Delta$, the rectification efficiency is given by 
\begin{equation}\label{Adef}
\eta(eV\gg\Delta,q\xi,{\cal T}) \simeq A(q\xi,{\cal T}) \frac{\Delta}{eV}.
\end{equation}
The dimensionless coefficient $A=A(q\xi,{\cal T})$ is illustrated
in Fig.~\ref{fig3}, with $A=2q\xi$ for ${\cal T}=1$ from the analytical solution. 
For $q\xi \alt 1$, our numerical results suggest $A(q\xi,{\cal T})\approx 2q\xi{\cal T}$.
  
\emph{Conclusions.---}We have studied a model for a voltage-biased Josephson diode with finite Cooper pair momentum.  For voltages in the subgap regime, we find that MAR processes allow for large rectification efficiencies, far beyond those found in equilibrium.  In particular,  the ideal one-way rectification limit could be 
reached in principle.  Similar low-bias efficiency enhancements may also be found for Josephson diodes based on other physical mechanisms. We hope that future experimental and theoretical work will shed light on this intriguing question.


\begin{acknowledgments}
We thank Liang Fu for discussions. 
We acknowledge funding by the Deutsche Forschungsgemeinschaft (DFG, German Research Foundation) under Grant No.~277101999 - TRR 183 (project C01), Grant No.~EG 96/13-1, 
and under Germany's Excellence Strategy - Cluster of Excellence Matter and Light for Quantum Computing (ML4Q) EXC 2004/1 - 390534769.
This work received support from the French government under the France 2030 investment plan, as part of the Initiative d'Excellence d'Aix-Marseille Universit\'e - A*MIDEX, through the institutes IPhU (AMX-19-IET-008) and AMUtech (AMX-19-IET-01X).
\end{acknowledgments}

%\bibliographystyle{aipnum4-1}
\bibliography{sup}
\end{document}
\begin{table}[t]
\centering
\setlength{\tabcolsep}{8pt}
\renewcommand{\arraystretch}{0.61}
\resizebox{\linewidth}{!}{
\begin{tabular}{@{} l c c c c c c @{}}
\toprule
\scriptsize \textbf{Method} & \scriptsize \textbf{CI} & \scriptsize \textbf{DO} & \scriptsize \textbf{CU} & \scriptsize \textbf{TU} & \scriptsize \textbf{CO}\\
\midrule
 \scriptsize Video Chat~\cite{li2023videochat} &  \scriptsize 2.23& \scriptsize 2.50& \scriptsize 2.53& \scriptsize 1.94& \scriptsize 2.24\\
 \scriptsize LLaMA Adapter~\cite{zhang2023llama}& \scriptsize 2.03& \scriptsize 2.32& \scriptsize 2.30& \scriptsize 1.98& \scriptsize 2.15\\
 \scriptsize Video LLaMA~\cite{zhang2023video} & \scriptsize 1.96& \scriptsize 2.18& \scriptsize 2.16& \scriptsize 1.82& \scriptsize 1.79\\
 \scriptsize Video-ChatGPT~\cite{maaz2023video}& \scriptsize \underline{2.40}& \scriptsize \underline{2.52}& \scriptsize \underline{2.62}& \scriptsize \underline{1.98}& \scriptsize \underline{2.37}\\
\midrule
\scriptsize MovieChat~\textit{(Ours)}& \scriptsize \textbf{2.76}& \scriptsize \textbf{2.93}& \scriptsize \textbf{3.01}& \scriptsize \textbf{2.24}& \scriptsize \textbf{2.42} \\
\bottomrule
\end{tabular}
}
\caption{Quantitative evaluation for short video generation performance with GPT-3.5~\cite{gpt3.5}. CI stands for correctness of information, DO stands for detail orientation, CU stands for contextual understanding, TU stands for temporal understanding, and CO stands for consistency. The best result is highlighted in bold, and the second best is underlined.}
\label{tab:short_varies}
\end{table}
\section{Experiments}

\vspace{-5pt}

We conduct quantitative and qualitative evaluations between MovieChat and previous methods. Additionally, we perform ablation studies to investigate MovieChat. Experimental settings and analyses can be found in appendix.

\subsection{Quantitative Evaluation}



\label{exp:quantitative}


\paragraph{Short video question-answering.} We use several widely used open-ended datasets: MSVD-QA~\cite{xu2017video}, MSRVTT-QA~\cite{xu2016msr-vtt}, and ActivityNet-QA~\cite{yu2019activitynet} for short video question-answering tasks. The evaluation process is under the assistance of LLM with the default hyper-parameter settings. The accuracy and relative scores on a scale of $0$ to $5$ are reported. Compared to previous methods~\cite{maaz2023video,li2023videochat,zhang2023llama, zhang2023video}, MovieChat achieves comparable performance even it is not specifically designed for short video question-answering tasks, as shown in Tab.~\ref{tab:short}.



\vspace{-12pt}

\paragraph{Short video generative performance.} Following ~\cite{maaz2023video}, we employ GPT-assisted evaluation to conduct a more comprehensive comparison of the text generation performance between MovieChat and previous methods~\cite{maaz2023video,li2023videochat,yang2022zero} on processed ActivityNet-QA~\cite{yu2019activitynet}. The evaluation pipeline covers crucial metrics (including \textit{Correctness of Information}, \textit{Detailed Orientation}, \textit{Contextual Understanding}, \textit{Temporal Understanding} and \textit{Consistency}) and assigns relative scores to the generated predictions on a scale of 1-5. We present the results of the generation performance evaluation in Tab.~\ref{tab:short_varies}. The results reveal its competitive performance across all key aspects compared to previous methods. 






\vspace{-12pt}

\paragraph{Long video question-answering.} We evaluate the long video question-answering performance of MovieChat with our proposed MovieChat-1K. We split 1,000 videos into training set~(800), test set~(100), validation set~(100) and only use test set for final performance evaluation. We select three recent LLM-based video understanding models~(\eg Video Chat~\cite{li2023videochat}, Video LLaMA~\cite{zhang2023video}, and Video-ChatGPT~\cite{maaz2023video}) as the baselines. Yet, none of those methods can support such long video~($\textgreater 10$K frames). Therefore, to accommodate their length limitations in global questions, we uniformly sample from the original video up to the maximum frame count which can be officially supported by each individual model. For breakpoint questions, we extend half of the maximum frame count before and after the breakpoint (\ie, placing the breakpoint at the center frame). 

% \begin{table}[t]
% \centering
% \setlength{\tabcolsep}{8pt}
% \renewcommand{\arraystretch}{1.3}
% \resizebox{\linewidth}{!}{
% \begin{tabular}{@{} l c c c c c @{}}
% \toprule
% \multirow{2}{*}{\textbf{Method}} & \multirow{2}{*}{\textbf{\# Frames}} & \multicolumn{2}{c}{\textbf{Global Mode}} & \multicolumn{2}{c}{\textbf{Breakpoint Mode}} \\
% \cline{3-6}
%  & & \textbf{Accuracy} & \textbf{Score} & \textbf{Accuracy} & \textbf{Score} \\
% \midrule
% Video Chat~\cite{li2023videochat}& 32 & \underline{61.0} & \underline{3.34} & 48.3 & 2.43 \\
% Video LLaMA~\cite{zhang2023video}& 32 & 51.4 & 3.10 & 38.2 & 2.31 \\
% Video-ChatGPT~\cite{maaz2023video}& 100 & 44.2 & 2.71 & \underline{49.8} & \underline{2.71} \\ 
% \midrule
% MovieChat~\textit{(ours)} & 2048 & \textbf{67.8} & \textbf{3.81} & \textbf{50.4} & \textbf{2.96}\\
% \bottomrule
% \end{tabular}
% }
% \caption{Quantitative evaluation for long video question answering on MovieChat-1K test set with GPT. The best result is highlighted in bold, and the second best is underlined.}
% \label{tab:long}
% \end{table}

% \begin{table}[t]
% \centering
% \setlength{\tabcolsep}{8pt}
% \renewcommand{\arraystretch}{1.3}
% \resizebox{\linewidth}{!}{
% \begin{tabular}{@{} l c c c c c @{}}
% \toprule
% \multirow{2}{*}{\textbf{Method}} & \multirow{2}{*}{\textbf{\# Frames}} & \multicolumn{2}{c}{\textbf{Global Mode}} & \multicolumn{2}{c}{\textbf{Breakpoint Mode}} \\
% \cline{3-6}
%  & & \textbf{Accuracy} & \textbf{Score} & \textbf{Accuracy} & \textbf{Score} \\
% \midrule
% Video Chat~\cite{li2023videochat}& 32 & \underline{61.0/52.1/60.2} & \underline{3.34/2.59/3.08} & 48.3/43.8/46.3 & 2.43/2.12/2.32 \\
% Video LLaMA~\cite{zhang2023video}& 32 & 51.4/47.3/56.3 & 3.10/2.19/2.72 & 38.2/33.2/45.8 & 2.31/1.69/2.11 \\
% Video-ChatGPT~\cite{maaz2023video}& 100 & 44.2/39.8/58.7 & 2.71/2.04/2.89 & \underline{49.8/46.4/47.8} & \underline{2.71/2.21/2.43} \\ 
% \midrule
% MovieChat~\textit{(ours)} & 2048 & \textbf{67.8/55.3/63.7} & \textbf{3.81/2.73/3.15} & \textbf{50.4/46.4/48.1} & \textbf{2.96/2.28/2.46}\\
% \bottomrule
% \end{tabular}
% }
% \caption{Quantitative evaluation for long video question answering on MovieChat-1K test set with GPT, Claude and human bling rating. The best result is highlighted in bold, and the second best is underlined.}
% \label{tab:long}
% \end{table}

\begin{table}[t]
\centering
\setlength{\tabcolsep}{8pt}
\renewcommand{\arraystretch}{1.3}
\resizebox{\linewidth}{!}{
\begin{tabular}{@{} l c c c c c @{}}
\toprule
\multirow{2}{*}{\textbf{Method}} & \multirow{2}{*}{\textbf{\# Frames}} & \multicolumn{2}{c}{\textbf{Global Mode}} & \multicolumn{2}{c}{\textbf{Breakpoint Mode}} \\
\cline{3-6}
 & & \textbf{Accuracy} & \textbf{Score} & \textbf{Accuracy} & \textbf{Score} \\
\midrule
Video Chat~\cite{li2023videochat}& 32 & \underline{57.8} & \underline{3.00} & 46.1 & 2.29 \\
Video LLaMA~\cite{zhang2023video}& 32 & 51.7 & 2.67 & 39.1 & 2.04 \\
Video-ChatGPT~\cite{maaz2023video}& 100 & 47.6 & 2.55 & \underline{48.0} & \underline{2.45} \\ 
\midrule
MovieChat~\textit{(ours)} & 2048 & \textbf{62.3} & \textbf{3.23} & \textbf{48.3} & \textbf{2.57}\\
\bottomrule
\end{tabular}
}
\caption{Quantitative evaluation for long video question answering on MovieChat-1K test set in global mode with the average of GPT-3.5~\cite{gpt3.5}, Claude~\cite{examplewebpage} and human bling rating. HBR stands for human blind rating. The best result is highlighted in bold, and the second best is underlined.}
\label{tab:long}
\end{table}
% \begin{table}[t]
% \centering
% \setlength{\tabcolsep}{8pt}
% \renewcommand{\arraystretch}{0.7}
% \resizebox{\linewidth}{!}{
% \begin{tabular}{@{} l c c c c c c @{}}
% \toprule
% \scriptsize \textbf{Method} & \scriptsize \textbf{CI} & \scriptsize \textbf{DO} & \scriptsize \textbf{CU} & \scriptsize \textbf{TU} & \scriptsize \textbf{CO}\\
% \midrule
% \scriptsize Video Chat~\cite{li2023videochat} & \scriptsize 3.26 & \scriptsize \underline{3.20} & \scriptsize \underline{3.38} & \scriptsize \underline{2.97}& \scriptsize  \underline{3.47}\\
% \scriptsize Video LLaMA~\cite{zhang2023video} & \scriptsize \underline{3.30} & \scriptsize 2.53& \scriptsize 3.28& \scriptsize 2.77& \scriptsize 3.42\\
% \scriptsize Video-ChatGPT~\cite{maaz2023video}& \scriptsize 2.48& \scriptsize 2.78& \scriptsize 3.03& \scriptsize 2.48& \scriptsize 2.99\\
% \midrule
% \scriptsize MovieChat~\textit{(Ours)}& \scriptsize \textbf{3.32}& \scriptsize \textbf{3.28} & \scriptsize \textbf{3.40}& \scriptsize \textbf{3.01}& \scriptsize \textbf{3.48} \\
% \bottomrule
% \end{tabular}
% }
% \caption{Quantitative evaluation for long video generation performance in global mode with GPT. CI stands for correctness of information, DO stands for detail orientation, CU stands for contextual understanding, TU stands for temporal understanding, and CO stands for consistency. The best result is highlighted in bold, and the second best is underlined.}
% \label{tab:long_varies}
% \end{table}

\begin{table}[t]
\centering
\setlength{\tabcolsep}{8pt}
\renewcommand{\arraystretch}{0.61}
\resizebox{\linewidth}{!}{
\begin{tabular}{@{} l c c c c c c @{}}
\toprule
\scriptsize \textbf{Method} & \scriptsize \textbf{CI} & \scriptsize \textbf{DO} & \scriptsize \textbf{CU} & \scriptsize \textbf{TU} & \scriptsize \textbf{CO}\\
\midrule
\scriptsize Video Chat~\cite{li2023videochat} & \scriptsize \underline{3.04} & \scriptsize \underline{2.75} & \scriptsize \underline{3.09} & \scriptsize \underline{3.00}& \scriptsize  \underline{3.21}\\
\scriptsize Video LLaMA~\cite{zhang2023video} & \scriptsize 2.75 & \scriptsize 2.24& \scriptsize 2.83& \scriptsize 2.62& \scriptsize 2.97\\
\scriptsize Video-ChatGPT~\cite{maaz2023video}& \scriptsize 2.37& \scriptsize 2.30& \scriptsize 2.58& \scriptsize 2.49& \scriptsize 2.69\\
\midrule
\scriptsize MovieChat~\textit{(Ours)}& \scriptsize \textbf{3.11}& \scriptsize \textbf{2.93} & \scriptsize \textbf{3.24}& \scriptsize \textbf{3.17}& \scriptsize \textbf{3.25} \\
\bottomrule
\end{tabular}
}
\caption{Quantitative evaluation for long video generation performance in global mode with the average of GPT-3.5~\cite{gpt3.5}, Claude~\cite{examplewebpage} and human blind rating. CI stands for correctness of information, DO stands for detail orientation, CU stands for contextual understanding, TU stands for temporal understanding, and CO stands for consistency. The best result is in bold, and the second best is underlined.}
\label{tab:long_varies}
\end{table}
\section{Ablation study on YCBV}
\label{sec:ablation_ycbv}

In Tab.~\ref{tab:ablation_ycbv} we report the results of our ablation study on YCBV~\cite{ycbv}.
We choose the Large Marker object and train a single model on it for each modification we applied.
Each model is trained for 20 epochs on the standard training set.
For the computation of the Feature Matching Recall (FMR), we set the distance threshold $\tau_1=10$ voxels and the inlier ratio threshold $\tau_2=5$\%, to account for the different density of the scene point cloud in YCBV.
All the other settings and parameters are the same as those in our ablation study on LMO~\cite{lmo} in the main paper.

We can observe that some changes do not increment performance, but instead cause a slight drop, in particular when adapting the safety threshold to the object dimension (third row, $-0.4$) and when colour augmentation is applied (sixth row, $-$0.3).
These additions do not benefit this particular object, but are instead advantageous when averaging all the object in the dataset.

We can note that, as in the ablation study on the LMO dataset in the main paper, the most significant improvements in ADD-S AUC result from applying the safety threshold ($+$1.5), adding RGB information ($+$5.5), and using the Adam optimiser ($+$12.3).
\renewcommand{\arraystretch}{0.9}
\begin{table}%[t!]
\centering
\tabcolsep 3pt
\caption{
Ablation study on the Large Marker object of YCBV.
Performance are compared in terms of RRE [radiants] and RTE [cm] errors (the lower the better), and FMR and ADD-S AUC (shortened to ADD) scores (the higher the better).
$\Delta$ shows the improvement of each contribution in terms of ADD-S AUC with respect to the previous row.
}
\vspace{-3mm}
\resizebox{\columnwidth}{!}{%
\begin{tabular}{clrrrrr}
\toprule
& Improvements &
RRE{\color{black!50}{$\,\downarrow$}} &
RTE{\color{black!50}{$\,\downarrow$}} & 
FMR{\color{black!50}{$\,\uparrow$}} & 
ADD{\color{black!50}{$\,\uparrow$}} & 
$\Delta$ \\ 
\toprule
& Baseline & 2.0 & 4.6 & 0.00 & 77.2 & -- \\
\midrule
\multirow{2}{*}{\rotatebox{90}{Loss}} & $+$ $\tau_{NS} = 0.1 D_S$ & 2.0 & 4.2 & 0.00 & 78.7 & $+$1.5 \\
& $+$ $\tau_{NS} = 0.1 D_O$ & 2.0 & 4.3 & 0.00 & 78.3 & $-$0.4 \\
\midrule
\multirow{2}{*}{\rotatebox{90}{Arch.}} & $+$ Independent weights & 2.0 & 4.1 & 0.00 & 79.4 & $+$1.1 \\
& $+$ Add RGB information & 1.2 & 3.2 & 49.1 & 84.9 & $+$5.5 \\
\midrule
\multirow{2}{*}{\rotatebox{90}{Aug.}} & $+$ Color augmentation & 1.2 & 3.3 & 50.0 & 84.6 & $-$0.3 \\
& $+$ Random erasing & 1.2 & 3.1 & 53.4 & 85.2 & $+$0.6 \\
\midrule
\multirow{2}{*}{\rotatebox{90}{Optim.}} & $+$ SGD $\to$ Adam & 0.0 & 0.4 & 100 & 97.5 & $+$12.3 \\
& $+$ Adam $\to$ AdamW  & 0.0 & 0.4 & 100 & 97.5 & 0 \\
& $+$ Exp $\to$ Cosine & 0.0 & 0.4 & 100 & 97.4 & $-$0.1 \\\bottomrule
\end{tabular}}
\label{tab:ablation_ycbv}
\end{table}
\renewcommand{\arraystretch}{1}

\section{Additional ablation study on LMO}

We include an ablation study on the $t_\text{scale}$ hyperparameter, which is used to set the radius of the ball volume in which negative mining around a certain point is not allowed. We train on the Can object of LMO using the standard setting, and varying only $t_\text{scale}$. The results are shown in Tab.~\ref{tab:ablation_ycbv}.
We can observe that our choice of $t_\text{scale} = 0.1$ leads to the best result. When $t_\text{scale}$ is increased, many candidate points are forbidden to be used as negatives, therefore decreasing the final performance. On the other hand, a lower $t_\text{scale}$ implies negative pairs composed by points which are near in the 3D space. This reduces the performance, as similar points are forced to have different descriptors. Notably, the worst results is obtained when $t_\text{scale} = 0.1$, i.e. when no negative candidates are excluded.

\begin{table}
\tabcolsep 3pt
\caption{
Ablation study on the Can object of LMO. Performance is shown in terms of ADD-0.1 (the higher the better) in function of the hyperparameter $t_\text{scale}$.}
\centering
\resizebox{.9\columnwidth}{!}{
\begin{tabular}{c|ccccc}
    \toprule
    $t_\text{scale}$ & 0.0 & 0.01 & 0.05 & \textbf{0.1} & 0.5 \\
    ADD-0.1d & 66.55 & 91.80 & 93.79 & \textbf{93.95} & 81.28 \\
    \bottomrule
\end{tabular}
\label{tab:tscale}
}
\end{table}

To enhance the robustness of the results, we simultaneously employ GPT-3.5~\cite{gpt3.5} and Claude~\cite{examplewebpage} as LLM assistants, with the additional support of human blind rating. We observe a discrepancy between the accuracy and relative score generated by the previously LLM-assisted evaluation method~\cite{maaz2023video} for video question-answering tasks. However, merely adjusting the prompt for the LLM cannot effectively address this issue. Therefore, after obtaining the accuracy and score from the LLM-assisted evaluation method, we implement manual filtering to remove results with inconsistent values, thus improving the reliability of our outcomes.


As shown in Tab.~\ref{tab:long}, compared to previous methods~\cite{maaz2023video,li2023videochat,zhang2023video}, MovieChat reads more video frames. In both global mode and breakpoint mode, our method maintains a performance gain in terms of the average accuracy and score provided by LLM assistants and human blind rating. We comprehensively evaluate MovieChat's question-answering performance across different question types compared to baselines. The results indicate that our approach outperforms the baselines in both open-ended and true-false questions.



\vspace{-8pt}

\paragraph{Long video generative performance.} 

We compare the quality of answers generated by MovieChat and previous methods~\cite{maaz2023video,li2023videochat,zhang2023video} in long video question-answering on MovieChat-1K. As shown in Tab.~\ref{tab:long_varies}, with the average score provided by GPT-3.5~\cite{gpt3.5}, Claude~\cite{examplewebpage} and human bling rating, our approach continues to generate higher-quality answers even as the video contents become more extensive.


\subsection{Ablation Study}



\begin{table}[t]
\centering
\setlength{\tabcolsep}{12pt}
\renewcommand{\arraystretch}{0.7}
\resizebox{\linewidth}{!}{
\begin{tabular}{@{} c c c c c @{}}
\toprule
\multirow{2}{*}{\textbf{\scriptsize Method}} & \multicolumn{2}{c}{\textbf{\scriptsize Global Mode}} & \multicolumn{2}{c}{\textbf{\scriptsize Breakpoint Mode}} \\
\cline{2-5}
 & \scriptsize {Accuracy} & \scriptsize {Score} & \scriptsize {Accuracy} &  \scriptsize {Score}\\
\midrule
 \rule{0pt}{5pt} \scriptsize w/o MM &   \scriptsize 51.4&  \scriptsize 3.10&  \scriptsize 38.2&  \scriptsize 2.31 \\
  \rule{0pt}{5pt} \scriptsize base&  \scriptsize \textbf{67.8}&  \scriptsize \textbf{3.81}&  \scriptsize \textbf{50.4}& \scriptsize \textbf{2.96} \\
\bottomrule
\end{tabular}
}
\caption{Ablation study on how memory mechanism (MM) affects the long video question answering. The best result is in bold.}
\label{tab:videollama_score}
\vspace{-5pt}
\end{table}



\vspace{-8pt}

\paragraph{Short-term and long-term memory buffers.} 
As MovieChat incorporates a memory mechanism including short-term memory and long-term memory, it is imperative to evaluate how the proposed memory mechanism influences the performance. Tab.~\ref{tab:videollama_score} and Tab.~\ref{tab:videollama_5} provide the memory-dependent performance of MovieChat for long video question-answering and generative tasks with the average results of GPT-3.5~\cite{gpt3.5}, Claude~\cite{examplewebpage}, and human blind rating. MovieChat with the memory mechanism significantly outperforms the memory-independent variant, which signifies the importance of memory mechanisms.

\vspace{-8pt}

\paragraph{Hyper-parameter ablations.} 
We perform a series of hyperparameter ablations based on the MovieChat-1K dataset to better understand MovieChat. Fig.~\ref{fig:ablation} shows the performance when ablating the length of memory buffers, consolidation length and short-term initialization with the average results of GPT-3.5~\cite{gpt3.5}, Claude~\cite{examplewebpage}, and human blind rating. The performance of MovieChat degrades when all four are significantly changed, showing the validity of our empirically chosen hyperparameyers. Fig.~\ref{fig:ablation} demonstrates that information obtained from the video expands with the growing length of memory buffers, while the loss of finer details intensifies with the fixed length of consolidation. Furthermore, using merged tokens for short-term initialization outperforms last few tokens and uniform sampling. Additionally, the length of merged tokens and the memory buffer size have a combined effect on MovieChat's performance.

\definecolor{global}{RGB}{21,96,130}
\definecolor{breakpoint}{RGB}{51,0,111}

\begin{table}[t]
\centering
\setlength{\tabcolsep}{8pt}
\renewcommand{\arraystretch}{1.3}
\resizebox{\linewidth}{!}{
\begin{tabular}{@{} c c c c c c c c c c c @{}}
\toprule
\multirow{2}{*}{\textbf{Method}} & \multicolumn{5}{c}{\textbf{Global Mode}} & \multicolumn{5}{c}{\textbf{Breakpoint Mode}} \\
\cline{2-11}
& \textbf{CI} & \textbf{DO} & \textbf{CU} & \textbf{TU} & \textbf{CO} & \textbf{CI} & \textbf{DO} & \textbf{CU} & \textbf{TU} & \textbf{CO}\\
\midrule
w/o MM &  3.30&  2.53&  3.28&  2.77& 3.42& 2.42& 2.85& 2.87& 2.00& 2.87 \\
base&  \textbf{3.32}&  \textbf{3.28}&  \textbf{3.40}&  \textbf{2.97}& \textbf{3.48}& \textbf{2.97}& \textbf{3.24}& \textbf{3.31}& \textbf{2.70}& \textbf{3.45}\\
\bottomrule
\end{tabular}
}
\caption{Ablation study on how memory mechanism (MM) affects the long video generative performance. CI stands for correctness of information, DO stands for detail orientation, CU stands for contextual understanding, TU stands for temporal understanding, and CO stands for consistency. The best result is in bold.}
\label{tab:videollama_5}
\vspace{-5pt}
\end{table}




\vspace{-8pt}

\subsection{Case Study}

\vspace{-7pt}

We perform an extensive case study of MovieChat on a variety of open-ended long video~(such as cartoon movie and TV series) for long video question-answering, including the \parbox[c][8pt][l]{8pt}{\colorbox{breakpoint}{}}breakpoint mode (Q\#1) and the \parbox[c][8pt][l]{8pt}{\colorbox{global}{}}global mode (Q\#2). The evaluation is conducted between MovieChat and previous methods~\cite{maaz2023video,li2023videochat,zhang2023llama} as shown in Fig.~\ref{fig:case} . For Q\#1 in breakpoint mode, we mark the timestamp when the question is asked. For long videos over $10$K frames, MovieChat is still capable of providing excellent responses to questions regarding both the current moment and the entire video content with less hallucination. More examples to show long video scene understanding and temporal understanding ability of MovieChat are available in appendix. 



\begin{table*}[t]
\tiny
\centering
\resizebox{0.99\linewidth}{!}{
\begin{tabular}{p{2cm}|p{3cm}|p{3cm}|p{3cm}|p{0.35cm}|p{0.44cm}|p{0.3cm}|p{0.3cm}|p{0.3cm}}
\toprule
prompt & real completion &  no watermark (NW) &  watermarked (W) &(NW) $z$   &(W) $z$ &(Real) PPL & (NW) PPL & (W) PPL \\
\midrule
DPR members Jim Ragsdale, Diane Kane, Angeles Liera and pro tem chair Mike Costello discuss condo conversion projects.\textbackslash n During the Jan  & .10 meeting of the La Jolla Development Permit Review committee (DPR), board members voted unanimously to form a research subcommittee that will look into the consequences of condo conversion in the neighborhoods south of Pearl Street.[...continues] &  . 27 meeting, Ragsdale and Kane discussed the need for a condo market and how to get there. Costello spoke about the need to have a condo market in the area but also said there is a need to be able to rent a condo and that the area is growing. [...continues]& . 11-12 meeting, the city announced that the development of condos to be built in the historic downtown has been approved by the city.\textbackslash n \"We feel it's important to be able to provide affordable housing for the people of the city in a way that the community feels they [...continues] &  0.71 &  10.5 & 5.42& 8.26 &  7.15 \\\midrule
In their first game since dropping out of the top five, the Irish delivered a redemption performance against Boston College, picking up a 50-point win over &    
 the Eagles while simultaneously moving one step closer to cementing Arike Ogunbowale’s legacy, as the senior guard passed current associate coach Beth Cunningham on the list of all-time scorers in the program.\textbackslash n No. 6 Notre Dame (23-2, 10-2 ACC) wasted [...continues] 
&    the Eagles. The Irish also defeated the Bulldogs in the final and will face the Bulldogs in the final.\textbackslash n The Irish also defeated the Eagles in the final and will face the Bulldogs in the final. Boston College:\textbackslash n The Eagles had a very good game against the Irish, [...continues] 
& South Carolina in the College Football Playoff Tournament to secure a berth in the NCAA Tournament. The Eagles will meet Notre Dame in the Big 12 Tournament on Sept. 14 in Austin, Texas.\textbackslash n  n\"We've got to win in the first round [...continues] &  1.13 &      11.3 & 4.78 & 6.78 &  9.15 \\\midrule
Two officers of a company that operates three Manhattan hotels were indicted yesterday in a scheme to help homeless people fraudulently obtain welfare checks and split the money  &   with the hotels.\textbackslash n District Attorney Robert M. Morgenthau of Manhattan, who announced the indictments, said they resulted from a study of the three hotels announced last December by the city. He said his office was delayed in moving more quickly on the case because of difficulty[...continues] &  they received from the government. \textbackslash n The indictment, which was released on Tuesday, said that police officers, who arrived at the hotel on a routine shift, met with a homeless man who asked for a check from his landlord. man told the officers about the scheme, which involved[...continues] &  between themselves and the homeless.\textbackslash n The scheme, alleged to be connected to the New York City Department of Health and Welfare, was uncovered in the wake of the 2012 Sandy Hook Elementary School shooting and the 2011 bombing of the Boston Marathon. Authorities say that the scheme  [...continues] &    -1.9 &  9.94 & 4.83 & 7.02 &  7.05 \\\midrule
Buddhadev had written a strong letter of protest to Manmohan Singh objecting to Mulford's behaviour. \textbackslash n Taking serious exception to  & US Ambassador David Mulford writing directly to West Bengal Chief Minister Buddhadev Bhattacharjee for his remarks against the American President, the CPI-M on Friday said the party[...continues] &  the comments made by Mulford, the BJP MP also called on the CM to resign immediately and the Centre to make a statement in the coming weeks.\textbackslash n In his letter to Manmohan Singh, the MP said he was not opposed to[...continues] & Mr Mulford's behaviour in the media and in the Parliament, the Union Minister has directed the Union Secretaries of Parliament and the Secretaries of the Supreme Court to take action against him in the matter[...continues] &  1.85 & 12.36 & 4.25 & 7.35 &      8.15 \\
\bottomrule
\end{tabular}}
\caption{Selected output examples from non-watermarked (NW) and watermarked (W) top-K sampling using $\gamma=0.5$, $\delta=2.0$ and $k=20$.  
}
\label{tab:demo-examples}
\end{table*}


