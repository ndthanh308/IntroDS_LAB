\section{Limitation}

Although MovieChat has demonstrated impressive abilities in long video understanding, it is still an early-stage prototype and has some limitations, including: 1) Limited perception capacities. MovieChat’s performance is hindered by the pretrained short video understanding model. 2) Inadequate Time Processing. MovieChat provides only rough estimates of the duration proportions of events within long videos, lacking precision in temporal details.

\section{Conclusion}

Conclusively, we presents an innovative video understanding system integrating video foundation models and large language models. By incorporating a memory mechanism represented by tokens in Transformers, our proposed system, MovieChat overcomes challenges associated with analyzing long videos.
MovieChat achieves state-of-the-art performance in long video understanding, surpassing existing systems limited to handling videos with few frames. 

% The study emphasizes the significance of memory mechanisms in video understanding, enabling the model to retain and retrieve relevant information over extended duration. MovieChat's success has practical implications in domains like video surveillance, content analysis, and video recommendation systems.
% Future research can explore further improvements to the memory mechanism and the integration of other modalities, such as audio, to enhance video understanding capabilities. This work opens up opportunities for applications requiring a comprehensive understanding of long-term visual information.