%% 
%% Copyright 2019-2020 Elsevier Ltd
%% 
%% This file is part of the 'CAS Bundle'.
%% --------------------------------------
%% 
%% It may be distributed under the conditions of the LaTeX Project Public
%% License, either version 1.2 of this license or (at your option) any
%% later version.  The latest version of this license is in
%%    http://www.latex-project.org/lppl.txt
%% and version 1.2 or later is part of all distributions of LaTeX
%% version 1999/12/01 or later.
%% 
%% The list of all files belonging to the 'CAS Bundle' is
%% given in the file `manifest.txt'.
%% 
%% Template article for cas-sc documentclass for 
%% double column output.

%\documentclass[a4paper,fleqn,longmktitle]{cas-sc}
\documentclass[a4paper,fleqn]{cas-sc}

% \usepackage[numbers]{natbib}
%\usepackage[authoryear]{natbib}
\usepackage[authoryear,longnamesfirst]{natbib}
\usepackage{mathtools}
\usepackage{wrapfig} % for the wrapfigure command

%%%Author definitions
\def\tsc#1{\csdef{#1}{\textsc{\lowercase{#1}}\xspace}}
\tsc{WGM}
\tsc{QE}
\tsc{EP}
\tsc{PMS}
\tsc{BEC}
\tsc{DE}
%%%

% Uncomment and use as if needed
\newdefinition{definition}{Definition}
\newtheorem{theorem}{Theorem}
\newtheorem{lemma}[theorem]{Lemma}
\newdefinition{rmk}{Remark}
\newproof{pf}{Proof}
\newproof{pot}{Proof of Theorem \ref{thm}}

\begin{document}
\let\WriteBookmarks\relax
\def\floatpagepagefraction{1}
\def\textpagefraction{.001}

% Short title
\shorttitle{Maxclique approximation}

% Short author
\shortauthors{D. Pfeifer}

% Main title of the paper
\title [mode = title]{Cliqueful graphs as a means of calculating the maximal number of maximum cliques of simple graphs}                      
% Title footnote mark
% eg: \tnotemark[1]
%\tnotemark[1,2]

% Title footnote 1.
% eg: \tnotetext[1]{Title footnote text}
% \tnotetext[<tnote number>]{<tnote text>} 
%\tnotetext[1]{This document is the results of the research
%   project funded by the National Science Foundation.}

%\tnotetext[2]{The second title footnote which is a longer text matter
%   to fill through the whole text width and overflow into
%   another line in the footnotes area of the first page.}


% First author
%
% Options: Use if required
% eg: \author[1,3]{Author Name}[type=editor,
%       style=chinese,
%       auid=000,
%       bioid=1,
%       prefix=Sir,
%       orcid=0000-0000-0000-0000,
%       facebook=<facebook id>,
%       twitter=<twitter id>,
%       linkedin=<linkedin id>,
%       gplus=<gplus id>]
\author[1]{Dániel Pfeifer}[type=editor,orcid=0000-0002-2434-6052]

% Corresponding author indication
%\cormark[1]

% Footnote of the first author
%\fnmark[1]

% Email id of the first author
\ead{pfeiferd@math.bme.hu}

% URL of the first author
%\ead[url]{www.cvr.cc, cvr@sayahna.org}

%  Credit authorship
%\credit{Conceptualization of this study, Methodology, Software}

% Address/affiliation
\affiliation[1]{organization={Budapest University of Technology and Economics},
    addressline={Műegyetem rkp. 3}, 
    city={Budapest},
    % citysep={}, % Uncomment if no comma needed between city and postcode
    postcode={1111}, 
    % state={},
    country={Hungary}}

% Here goes the abstract
\begin{abstract}
A simple graph on $n$ vertices may contain a lot of maximum cliques. But how many can it potentially contain? We will show that the maximum number of maximum cliques is taken over so-called cliqueful graphs, more specifically, later we will show that it is taken over saturated composite cliqueful graphs, if $n \ge 15$. Using this we will show that the graph that contains $3^{\lfloor n/3 \rfloor}c$ maxcliques has the most number of maxcliques on $n$ vertices, where $c\in\{1,\frac{4}{3},2\}$, depending on $n \text{ mod } 3$.
\end{abstract}

% Use if graphical abstract is present
% \begin{graphicalabstract}
% % Figure removed
% \end{graphicalabstract}

% Research highlights
\begin{highlights}
\item Proving that graphs that contain the maximum number of maximum cliques on $n$ vertices are saturated cliqueful.
\item Proving that graphs that contain the maximum number of maximum cliques on $n \ge 15$ vertices are saturated composite cliqueful.
\item Showing equivalence between cliqueful graphs' primeness and the connectedness of their complement.
\item Constructing two efficient polynomial-time graph factorization algorithms for cliqueful graphs.
\item Showing more properties of cliqueful graphs.
\item Proving that the maximum number of maximum cliques on $n \ge 15$ vertices is $M_n = 3^{\lfloor n/3 \rfloor}c$, where $c\in\{1,\frac{4}{3},2\}$, depending on $n \text{ mod } 3$.
\item Concluding that the maximum number of maximum cliques and the maximum number of maximal cliques is exactly the same.
\end{highlights}
% \asymp

% Keywords
% Each keyword is seperated by \sep
\begin{keywords}
Maximum number of maxcliques \sep Cliqueful graphs \sep  Saturated cliqueful graphs \sep Prime and composite graphs
\end{keywords}

\maketitle

\section{Introduction}

In this paper we will introduce the notion of cliqueful graphs, which are graphs that can be determined by their maximum cliques. We will show that graphs with the most number of \textbf{maximum cliques} (cliques of size $k$ such that there are no cliques of size $k+1$ in the graph) are cliqueful, moreover, later we will shrink this graph family to so-called composite saturated cliqueful graphs, which will be enough to show what the graphs with the most number of maximum cliques on $n$ vertices should look like.

We will introduce each notion rigorously, and by continuously extending the properties of cliqueful graphs, with each Theorem relying on previous ones, we will arrive at Theorem \ref{final}, which proves that the most number of maximum cliques for graphs on $n$ vertices, $n \ge 15$ is:

$$M_n = \begin{cases}
3^{\lfloor n/3 \rfloor} \text{ if }n \text{ mod } 3 = 0 \\
4 \cdot 3^{(\lfloor n/3 \rfloor-1)} \text{ if }n \text{ mod } 3 = 1 \\
2 \cdot 3^{\lfloor n/3 \rfloor} \text{ if }n \text{ mod } 3 = 2
\end{cases}$$

In 1965, \cite{moon1965cliques} have already calculated the maximum number of \textbf{maximal cliques} (cliques that are non-extendable), and they also obtained the exact same amount. From their result, it follows that the maximum number of maximum cliques is less than or equal to this amount. However in this paper, we can also conclude that the maximum number of maximal cliques and the maximum number of maximum cliques for graphs with $n \ge 15$ vertices is exactly the same. In \cite{moon1965cliques}'s proof, some crucial calculations and affirmations are not present, but in this study, we are careful to introduce and prove every statement rigorously.

Along with the main result, the introduced graph family, saturated cliqueful graphs, can be further used for other purposes; for example, finding graphs with a lot (but not necessarily maximum number of) maximum cliques. Cliqueful graphs have very special properties that make them easy to work with. For example, there exist $O(n^2)$ polynomial-time factorization algorithms for $G$ cliqueful graphs into $G= G_1 \times \dots \times G_m$ graphs, where all pairs of vertices between any two factors $G_i$ and $G_j$ are connected (see Theorem \ref{factor_algo}). Moveover, strict edge bounds can be given for all saturated cliqueful graphs (see Theorem \ref{edge_bound}).

The structure of the paper is the following: In Section \ref{prelim}, we will introduce all the graph theoretical definitions and notions that we will be using throughout the paper. We will also introduce shorthand notations that will be used commonly, so we ask the reader, even if they are familiar with these notions, please look through Section \ref{prelim} anyway.

Then in Sections \ref{cliqeful}-\ref{maxmaxcliques} we will proceed to build up our theory of cliqueful graphs, culminating in the main result we have stated above. In Section \ref{cliqeful}, the main definitions regarding cliqueful graphs will be introduced, along with crucial theorems that we will be using throughout the paper. In Seciton \ref{directproduct}, the notion of the direct product of graphs will be introduced, which will split graphs into prime and composite graphs. Special properties of prime and composite cliqueful graphs will also be proven. In Section \ref{saturatedcliqueful}, the notion of saturated cliqueful graphs will be introduced, along with important theorems regarding them. Finally, in Section \ref{maxmaxcliques} we will combine everything to calculate $M_n$.

\section{Preliminaries}\label{prelim}

\begin{definition}
Let $V$ be a set (vertices) and $E \subseteq V \times V$ (edges). Then $G=(V,E)$ is a \textbf{graph}. $G$ is a simple graph if $\forall e=(v,w) \in E: v \ne w$ and $(w,v) \notin E$. Let us denote the \textbf{set of all simple graphs} by $\mathbb{G}$. For convenience sake, for any edge $e=(v,w)$, let $(w,v)$ be the same edge, so $(v,w):=(w,v)$.
\end{definition}

\begin{definition}
Let $G = (V,E)$ be a simple graph. $\forall v \in V$, \textbf{the degree of vertex} $v$, $d(v)$ is the number of edges joining $v$ to other vertices, i.e. $d(v) = |\{e=(v,\cdot)|e\in E\}|$.
\end{definition}

\begin{definition}
Let $G=(V,E)$ be a simple graph. $H=(V',E')$ is a \textbf{subgraph} of $G$ if $V' \subseteq V$ and $E' \subseteq E$.
\end{definition}

\begin{definition}
Let $G = (V,E)$ be a simple graph, and let $H=(V',E')$ be a subgraph of $G$. Then $\forall v \in V$, \textbf{the subgraph degree of vertex} $v$, $d_H(v)$ is the number of edges joining $v$ to other vertices in $H$, i.e. $d_H(v) = |\{e=(v,\cdot)|e\in E'\}|$.
\end{definition}

\begin{definition}
Let $G = (V,E)$ be a simple graph with $|V|=n$. If all pairs of vertices in $G$ are connected, then $G$ is a \textbf{complete graph}. Complete graphs on $n$ vertices will be denoted as $G=K_n$.
\end{definition}

\begin{definition}
Let $G = (V,E)$ be a simple graph with $|V|=n$. If $E=\emptyset$, then $G$ is an \textbf{empty graph}. Empty graphs on $n$ vertices will be denoted as $G=E_n$.
\end{definition}

\begin{definition}
Let $G = (V,E)$ be a simple graph. Any complete subgraph $C$ of $G$  is called a \textbf{clique}.
\end{definition}

\begin{definition}
Let $C$ be a clique in $G$. It is a \textbf{maximum clique} or a \textbf{maxclique}, if no other cliques in $G$ have more vertices than $C$.
\end{definition}

\begin{definition}
Let $G=(V,E)$ be a simple graph. Then $\text{M}(G):=$ the \textbf{size of} $G$\textbf{'s maxcliqes}, i.e. if $H=(V',E')$ is a maxclique in $G$, then $\text{M}(G)=|V'|$.
\end{definition}

\begin{definition}
Let $G=(V,E)$ be a simple graph. Then $\text{MC}(G):=$ \textbf{the set of} $G$\textbf{'s maxcliqes}, i.e. $$MC(G)=|\{H = (V',E') | H\text{ is a maxclique in G }\}|$$
\end{definition}

\begin{definition}
Let $G=(V,E)$ be a simple graph. The graph $\overline{G}:=(V,V\times V \setminus E)$ is called the \textbf{complement graph} of $G$, i.e., in $\overline{G}$, all edges of $G$ become non-edges, and all non-edges become edges.
\end{definition}

\begin{rmk}
${}$\\
\vspace{-5mm}
\begin{itemize}
\item For a simple graph $G=(V,E)$, we will denote the edges of $\overline{G}$ by $\overline{E}$.
\item For any $\overline{e}\in\overline{E}$, we will denote $G \cup \overline{e}:=(V,E\cup\overline{e})$.
\item For any $e \in E$, we will denote $G \setminus e := (V,E\setminus e)$.
\end{itemize}
\end{rmk}

\begin{definition}
Let $G=(V,E)$ be a simple graph. The sequence of edges $(e_1,\dots,e_k)\in E^k$ is \textbf{path}, if $\forall i\in{1,\dots,k-1}$ $e_i$ shares a common vertex with $e_{i+1}$, i.e. if $e_i = (a,b)$ and $e_{i+1} = (c,d)$, then $b=c$. The path \textbf{goes between} $v$ and $w$ if $e_1 = (v,\cdot)$ and $e_k = (\cdot,w)$.
\end{definition}

\begin{definition}
Let $G=(V,E)$ be a simple graph. $G$ is \textbf{connected} if $\forall v_1,v_2\in V, (v_1 \ne v_2)$, there exists a path that goes between $v_1$ and $v_2$.
\end{definition}

\begin{lemma}
If $G=(V,E)$ is a connected simple graph with $|V|=n$, then $|E| \ge n-1$. 
\end{lemma}

\begin{definition}
$G=(V,E)$ is a \textbf{cycle} if $V=\{v_1,\dots,v_n\}$ and $E=\{(v_1,v_2),(v_2,v_3),\dots,(v_{n-1},v_{n})\}$ with $v_1 = v_n$. Cycles on $|V|=n$ vertices will be denoted as $C_n$.
\end{definition}

\begin{definition}
Let $G=(V,E)$ be a simple graph. Let us partition the vertex set into $S_1,S_2\subset V$, meaning that  $S_1 \cup S_2 = V, S_1 \cap S_2 = \emptyset$. Then
$$E_G(S_1,S_2) := \{e=(v_1,v_2)\in E | v_1 \in S_1, v_2 \in S_2\}$$
i.e., $E_G(S_1,S_2)$ is the \textbf{set of edges that go between} $S_1$ and $S_2$ in $G$.
\end{definition}

\begin{definition}
Let $G=(V,E)$ be a simple graph. The \textbf{cutting number} of $G$, denoted as $C(G)$, is the minimal number of edges that go between any partition of $V=S_1 \cup S_2$, i.e.
$$C(G) = \min \{\ \ |E_G(S_1,S_2)| \ \ | \ \ S_1,S_2 \subset V, S_1 \cup S_2 = V, S_1 \cap S_2 = \emptyset\}$$
\end{definition}

\begin{definition}
Let $G=(V,E)$ be a simple graph. $G$ is \textbf{bipartite} if $\exists A,B \subset V, A \cup B = V, A \cap B = \emptyset$ partition of $V$ such that $A$ and $B$ are  empty graphs. A bipartite graph is \textbf{complete} if $\forall v_1 \in A, v_2 \in B: e=(v_1,v_2) \in E$. Complete bipartite graphs on $|A|=n, |B|=m$ vertices will be denoted as $K_{n,m}$.
\end{definition}

\begin{definition}
Let $G=(V,E)$ be a simple graph. $G$ \textbf{contains an empty bipartite graph}, $E_{n,m} \subseteq G$, if $\exists A,B \subset V, A \cup B = V, A \cap B = \emptyset, |A|=n, |B|=m, \forall v_1 \in A, v_2 \in B: \neg \exists e=(v_1,v_2)\in E$. In other words, $G$ is \textbf{disconnected along} $A$ and $B$.
\end{definition}

\begin{definition}
Let $G=(V,E)$ be a simple graph. Then $\#G$ will denote the \textbf{number of maxcliques} in $G$, i.e. $$\#G = |MC(G)|$$
\end{definition}

In this study, our goal is to calculate how many maxcliques there could possibly be in a graph on $n$ vertices, disregarding the size of these maxcliques. So we are interested in the following amount:

\begin{definition}
The \textbf{maximum number of maxcliques on simple graphs on} $n$ \textbf{vertices} will be denoted as
$$M_n := \max \{ \#G | G=(V,E),|V|=n \}$$
\end{definition}

\begin{definition}
The \textbf{set of graphs on} $n$ \textbf{vertices with the most number of maxcliques} will be denoted as:
$$\text{MG}_n:= \text{argmax} \{\# G | G=(V,E), |V|=n\}$$
\end{definition}

\section{Cliqueful graphs}\label{cliqeful}

\begin{definition}
A graph $G^*=(V^*,E^*)$ is \textbf{cliqueful}, if it can be determined by its vertices and maxcliques, i.e. if
$$\{G=(V,E)|V=V^*\text{ and }MC(G)=MC(G^*)\}= \{G^*\}$$
\end{definition}

\begin{rmk}
For any simple graph $G=(V,E)$, if denoting the vertices with integers (i.e. $V \subseteq \mathbb{N})$, then for all $H$ maxcliques in $G$, $H \subseteq V \subseteq \mathbb{N}$, so $H \in \mathcal{P}(\mathbb{N})$, where $\mathcal{P}(A)$ denotes the powerset of $A$. Therefore, for the set of maxcliques of $G$, $\text{MC}(G)$:
$$\text{MC}(G) \subseteq \mathcal{P}(\mathcal{P}(\mathbb{N}))$$
\end{rmk}

\begin{theorem}
$\exists f: \mathcal{P}(\mathbb{N})\times \mathcal{P}(\mathcal{P}(\mathbb{N})) \to \mathbb{G} \cup \{0\}$ deterministic algorithm such that for all $G=(V,E)$ simple graphs
\begin{itemize}
\item $G\text{ is cliqueful } \Longrightarrow f(V,\text{MC}(G)) = G$
\item $G\text{ is not cliqueful } \Longrightarrow f(V,\text{MC}(G)) = 0$
\end{itemize}
\end{theorem}

\begin{pf}
We will construct $f$. For any input $(V,\text{MC}(G))$:
\begin{enumerate}
\item List all graphs on $|V|$ vertices.
\item Denote the ones that have maxclique set $\text{MC(G)}$ as $G_1,\dots,G_k$.
\item If $k=1$, return $G=G_1$.
\item Otherwise, return $0$.
\end{enumerate}

Now we will prove the implications:
\begin{itemize}
\item If $G$ is cliqueful $\Longrightarrow$ There are no other graphs with the same vertex set and the same maxclique set that $f$ found $\Longrightarrow$ Since $f$ went through all graphs on the same vertex set, there are no other graphs with the same maxclique set. $\Longrightarrow$ $f(V,\text{MC}(G)) = G$ is the only output.
\item If $G$ is not cliqueful $\Longrightarrow$ The algorithm found at least two graphs with the same vertex and maxclique set $\Longrightarrow$ In that case, $f$ will always output $0$. $\square$
\end{itemize}

\end{pf}

The following theorem characterizes cliqueful graphs, and the majority of our remaining theorems will rely on it:

\begin{theorem}\label{star}

For all simple graphs $G=(V^*,E^*)$,

$$G \text{ is cliqueful } \Longleftrightarrow
\begin{cases} (1)\ \forall e \in E^*: e \in \bigcup_{M(V,E)\in\text{MC}(G)} E \\
(2)\ \forall \overline{e} \in \overline{E}^*: \overline{e} \in \bigcup_{M(V,E)\in\text{MC}(G\cup\overline{e})} E
\end{cases}$$

Or in words, $G$ is cliqeful exactly if all edges $e$ are in maxcliques of $G$, and all non-edges $\overline{e}$ are in maxcliques of $G \cup \overline{e}$.

\end{theorem}

\begin{pf}
$(\Longrightarrow)$

\begin{enumerate}
\item Assume $\exists e\in E$ that is not part of any maxclique of $G$. Then $\text{MC}(G)=\text{MC}(G\cup e)$, so $G$ cannot be determined by its maxcliques, therefore $G$ is not cliqueful. $\Longrightarrow$ Our assumption was wrong, $\forall e \in E$ is part of a maxclique of $G$.
\item Assume $\exists \overline{e}\in \overline{E}$ that is not part of any maxclique of $G \cup \overline{e}$. Then $\text{MC}(G)=\text{MC}(G\cup\overline{e})$, therefore $G$ is not cliqueful. $\Longrightarrow$ Our assumption was wrong, $\forall \overline{e} \in \overline{E}$ is part of a maxclique of $G \cup \overline{e}$.
\end{enumerate}

$(\Longleftarrow)$

Let the following sets be given: an $V$ vertex set, and an $\text{MC}(G)$ maxclique set, such that $\exists$ a graph $G$ whose vertex set is $V$ and maxclique set is $\text{MC}(G)$, with the properties on the right side of the equivalence:

$$\begin{cases} (1)\ \forall e \in E^*: e \in \bigcup_{M(V,E)\in\text{MC}(G)} E \\
(2)\ \forall \overline{e} \in \overline{E}^*: \overline{e} \in \bigcup_{M(V,E)\in\text{MC}(G\cup\overline{e})} E
\end{cases}$$
The set of vertices on the right side of the equivalence can be obtained from $V=\bigcup_{e=(v_1,v_2)\in E} \left(v_1 \cup v_2\right)$.

We need to prove that $G$ is cliqueful. We can do that by proving that such a $G$, obtained from a vertex set and a maxcliqe set is unique.

Let us construct $G$ from $V$ and $\text{MC}(G)$:
\begin{enumerate}
\item Let $G$ be the empty graph $G=(V,\emptyset)$.
\item $\forall v_1,v_2 \in V$: Add $e=(v_1,v_2)$ exactly if $e \in \bigcup_{M(V,E)\in\text{MC}(G)} E$, i.e. $e$ appears in a maxclique of $G$.
\item Return $G$.
\end{enumerate}

This process outputs a unique $G$. If we were to add any more edges to $G$, obtaining $G \cup \overline{e}$, then $MC(G \cup \overline{e}) \ne MC(G)$, because of property (2). If we were to remove any edge from $G$, obtaining $G \setminus e$, then all maxcliques that contain $e$ wouldn't be part of $MC(G \setminus e)$ anymore, therefore $MC(G \setminus e)\ne M(G)$.

This means that $G$ is uniquely defined by its vertex set and maxclique set, therefore $G$ is cliqueful. $\square$
\end{pf}

\begin{theorem}\label{MG2}
Let $G=(V,E)$ be a simple graph with $M(G)\le 2$. Then $G$ is cliqueful. (i.e., All triangle-less graphs are cliqueful.)
\end{theorem}

\begin{pf}
If $M(G)=1$, then $E=\emptyset$, therefore $\text{MC}(G)=V$, i.e. all maxcliques of $G$ are just the vertices of $G$. There is only $1$ graph with vertex set $V$ ad maxclique set $V$, namely the empty graph on $V$. Therefore, $G$ is cliqueful.

If $M(G)=2$, then $\text{MC}(G)=E$, i.e. all maxcliques of $G$ are just the edges of $G$. There is only $1$ graph with vertex set $V$ and edge set $E$, namely $G=(V,E)$. Therefore $G$ is cliqueful. $\square$
\end{pf}

\begin{theorem}\label{conn}
Let $G=(V,E)$ be a cliqueful graph with $\text{M}(G)\ge 3$. Then $G$ is connected.
\end{theorem}

\begin{pf}
Assume $G$ is disconnected into components $S_1=(V_1,E_1),S_2=(V_2,E_2),\dots,S_m=(V_m,E_m)$.
Let $s_1\in V_1, s_2\in V_2$ and $e:=(s_1,s_2)$. Define the graph $G' := (V,E\cup e)$.

Then $e$ is not part of any triangle in $G$, because any triangle that would contain $e$ would also contain another edge between $S_1$ and $S_2$.

Then $MC(G)=MC(G')$, because $e=(s_1,s_2)$ is not part of any triangle in $G$, therefore not part of any maxclique in $G$ (since all maxcliques of $G$ are size $\ge 3$).

Which means that $G$ is not uniquely defined by its maxclique set, so $G$ is not cliqueful.

Therefore, our assumption that $G$ is disconnected was false, so $G$ has to be connected. $\square$
\end{pf}

\begin{theorem}\label{cut}
Let $G=(V,E)$ be a cliqueful graph with $\text{M}(G)\ge 3$. Then $\text{C}(G)\ge \text{M}(G)-1$.
\end{theorem}

\begin{pf}
Assume $\text{C}(G)\le \text{M}(G)-2$.

Then there exists a partition $S_1,S_2 \subset V$ with $S_1 \cup S_2 = V, S_1 \cap S_2 = \emptyset$, such that $1 \le E_G(S_1,S_2) \le \text{M}(G)-2$, where the first inequality comes from Theorem \ref{conn}, and the second inequality comes from the assumption.

The cutting number of a maxclique of size $\text{M}(G)$ is $C(K_{\text{M}(G)}) = \text{M}(G)-1$, namely the cut occurs between one vertex and the rest of the graph.

Therefore all edges in $E_G(S_1,S_2)$ cannot be part of a maxclique in $G$, because the cutting number of a maxclique is $M(G)-1$, and $E_G(S_1,S_2)$ contains at most $M(G)-2$ edges.

We have found at least $|E_G(S_1,S_2)|\ge 1$ edges that are not part of a maxclique in $G$, so by Theorem \ref{star}, $G$ cannot be cliqueful.

Therefore the assumption that $\text{C}(G)\le \text{M}(G)-2$ is incorrect, so it must be that $\text{C}(G)\ge \text{M}(G)-1$. $\square$
\end{pf}

\begin{theorem}
Let $G=(V,E)$ be a cliqueful graph with $\text{M}(G)\ge 3$. Then $\forall v \in V: v \in \bigcup_{M \in \text{MC}(G)} \bigcup_{e=(v,w) \in M} (v \cup w)$, i.e., all vertices are part of at least one maxclique of $G$. 
\end{theorem}

\begin{pf}
Assume $\exists v \in V$ that is not part of any maxcliqe in $G$. Since $\text{M}(G)\ge 3$, using Theorem \ref{conn}, $d(v)\ge 1$, otherwise the graph would be disconnected. So $\exists e \in E: e=(v,\cdot)$. $e$ cannot be in a maxclique, otherwise $v$ would also be in a maxclique.

Using Theorem \ref{star}, there exists an edge that is not part of a maxclique, so $G$ is not cliqueful. This is a contradiction, so the original assumption was wrong, meaning that $\forall v \in V$: $v$ has to be part of a maxclique. $\square$
\end{pf}

\begin{theorem}
Cliqueful graphs contain:
\begin{itemize}
\item cycles
\vspace{-2mm}
\item empty graphs
\vspace{-2mm}
\item complete graphs
\vspace{-2mm}
\item forests
\end{itemize}
\end{theorem}

\begin{pf}
Let $G=(V,E)$ be a cliqueful graph on $|V|=n$ vertices.
\begin{itemize}
\item If $G$ is a cycle, then $\text{M}(G)=2$. Using Theorem \ref{MG2}, $G$ is cliqueful.
\item If $G$ is the empty graph, then $\text{M}(G)=1$. Using Theorem \ref{MG2}, $G$ is cliqueful.
\item If $G$ is the complete graph, then $\text{MC}(G)=\{K_n\}$. There is only one graph with vertex set $\{1,\dots,n\}$ and maxclique set $K_n$, the complete graph on $n$ vertices, $K_n$ itself.
\item If $G$ is a forest, then $\text{M}(G)=2$. Using Theorem \ref{MG2}, $G$ is cliqueful. $\square$
\end{itemize}
\end{pf}

\begin{theorem}
Let $n,r\in\mathbb{N}^+$ with $n \ge 2r$, and $G:=K_n \setminus \{e_1,e_2,\dots,e_r\}$,  $e_1=(v_{11},v_{12}),e_2=(v_{21},v_{22}),\dots,e_r=(v_{r1},v_{r2})$ such that $\forall i,k \in \{1,\dots,r\}, j,l \in \{1,2\}: v_{ij} \ne v_{kl}$, i.e., the listed edges are all pairwise disjoint. Then $G$ is cliqueful.
\end{theorem}

\begin{pf}
Let $R$ be the set of endpoint of the listed edges, namely $R := \{v_{11},v_{12},v_{21},v_{22},\dots,v_{r1},v_{r2}\}$.

First we will prove that $\text{M}(G)=n-r$. Namely each maxclique in $G$ is
$$M := (V \setminus R) \cup \{v_{1\cdot},v_{2\cdot},\dots,v_{r\cdot}\}$$
where each $\cdot$ can be either $1$ or $2$.

$M$ clearly has $n-2r+r=n-r$ vertices. $M$ is a clique, because by construction, $(v_{i\cdot},v_{j\cdot})$ edges all connect, and the remaining vertices also all connect since we started from $K_n$.

And $M$ is maximal, because the only vertices we could add are all type $v_{i1}$ or $v_{i2}$, but if one of them is added, as the other one is already present, there would be a missing edge $(v_{i1},v_{i2})$.

Now we will set up the two conditions of Theorem \ref{star}, so first we will prove that $\forall e \in E: e$ is part of a maxclique $M$. Namely if $e \in V \setminus R$, then $e$ is part of all maxcliques, and if $e=(v_{ij},v_{kl}$ for some $i\ne k \in \{1,\dots,r\}$ and for some $j, l \in \{1,2\}$, then $e$ is part of any $M$ of form $(V \setminus R) \cup \{v_{1\cdot},\dots,v_{ij},v_{kl},\dots,v_{r\cdot}\}$.

Now we will also prove that $\forall \overline{e} \in \overline{E}: \overline{e}$ is part of a maxclique in $G \cup \overline{e}$. This is because $\overline{e}$ has to be one of $e_1,\dots,e_r$. Let $\overline{e} := e_i = (v_{i1},v_{i2})$. Then the maxcliques of $G \cup \overline{e}$ are of form
$$M := (V\setminus R) \cup \{v_{1\cdot},v_{2\cdot},\dots,v_{i1},v_{i2}\dots,v_{r\cdot}\}$$
any of which contains $e_i = (v_{i1},v_{i2})$.

By Theorem \ref{star}, $G$ is cliqueful. $\square$
\end{pf}

\begin{theorem}\label{Mn1}
The maximum number of maxcliques of graphs on $n$ vertices is taken on over cliqueful graphs, i.e.
$$M_n = \max \{\#G | G=(V,E), |V|=n, G\text{ is cliqueful}\}$$
\end{theorem}

\begin{pf}
Assume $\forall G=(V,E)$ simple graphs in $MG_n$, $G$ is not cliqueful.
Using Theorem \ref{star}, $G$ there are two possibilities for $G$:
\begin{enumerate}
\item $\forall e \in E: e$ is not part of a maxclique of $G$.
\item $\forall \overline{e} \in \overline{E}: \overline{e}$  is not part of a maxclique of $G \cup \overline{e}$.
\end{enumerate}
In Case $1$, let $e=(v,w)$. Then either $v$ or $w$ is not part of a maxclique in $G$, because if they both were, then $e$ would also be part of a maxclique. Without loss of generality, let us call the vertex not part of a maxclique $v$. Let us denote all edges that connect to $v$ by $E_v$, so $E_v = \{e \in E| e=(v_1,v_2), v_1 = v\text{ or } v_2 = v\}$.

Therefore the graph $G_v := (V \setminus \{v\}, E \setminus E_v)$ contains the same number of maxcliques as $G$, even though it has fewer vertices and fewer edges. Therefore, $\#G$ could not have been maximal, so $G$ could not have been in $MG_n$. This leads to a contradiction, so $M_n$ is indeed taken on over cliqueful graphs.

In Case $2$, $\exists \overline{e} \in \overline{E}$ that is not in a maxclique of $G \cup \overline{e}$. Let us add $\overline{e}$ to $G$, obtaining $G_1 := G \cup \overline{e}$. Since $\overline{e}$ was not in a maxclique if $G \cup \overline{e}$, $\text{MC}(G_1) = \text{MC}(G)$, so $\#G_1 = \#G$.

If $G_1$ is cliqueful, then the maximum $M_n$ is also taken on over cliqueful graphs, because for all non-cliqueful graphs $G$, we were able to find a cliqueful graphs with the same number of maxcliques as $G$.

If $G_1$ is not cliqueful, then we repeat the entire process. In Case $1$, we are done, and in Case $2$, we can obtain $G_2 := G_1 \cup \overline{e}$.

And so on. We keep adding edges until $G$ is cliqueful, which we can do, because the limit of this process in $K_n$, a cliqueful graph, so eventually we must reach a cliqueful graph $G_i$, such that $\#G_i = \#G$. So we always reach the contradiction of Case $1$, meaning that $M_n$ is indeed taken on over cliqueful graphs. $\square$
\end{pf}

\begin{rmk}
Note that the set $MG_n$ might contain non-cliqueful graphs, but $M_n$ can be calculated over cliqueful graphs, because for all $G$ non-cliqueful graphs we have found a cliqueful graph with the same number or more maxcliques.
\end{rmk}

\section{Direct product of graphs}\label{directproduct}

\begin{definition}\label{direct_product}
Given $G_1 = (V_1,E_1)$ and $G_2 = (V_2,E_2)$ simple graphs with $V_1 \cap V_2 = \emptyset$, $G_1 \times G_2 = (V,E)$ is the \textbf{direct product} of $G_1$ and $G_2$, if it has vertices $V = V_1 \cup V_2$ and edges
$$E = \{e=(v_1,v_2)|v_1 \in V_1,v_2\in V_2\}$$
i.e. Place $G_1$ and $G_2$ next to one another, and connect all vertices of $G_1$ to all vertices of $G_2$ to obtain $G_1 \times G_2$.
\end{definition}

\begin{theorem}\label{cartesian}
Given $G_1 = (V_1,E_1)$ and $G_2 = (V_2,E_2)$ simple graphs with $V_1 \cap V_2 = \emptyset$
$$\text{MC}(G_1 \times G_2) = \text{MC}(G_1)  \times \text{MC}(G_2)$$
where $\times$ on the right side means the Cartesian product, with each element evaulated using the direct product, i.e. if $A$ and $B$ are sets of graphs, then $A \times B = \{G_A \times G_B | G_A \in A, G_B \in B\}$.
\end{theorem}

\begin{pf}
We need to show what the maxcliques of $G_1 \times G_2$ look like. Since all vertices between $G_1$ and $G_2$ are connected, the maxcliques of $G_1 \times G_2$ are of form $M_1 \times M_2$, where $M_1 \in \text{MC}(G_1), M_2 \in \text{MC}(G_2)$. To see this, we need to show that $M_1 \times M_2$ is a clique, and that it is maximal in $G_1 \times G_2$.

\begin{itemize}
\item $M_1 \times M_2$ is a clique, because all vertices inside $M_1$ are connected (since $M_1$ is a clique in $G_1$), all vertices inside $M_2$ are connected (since $M_2$ is a clique in $G_2$), and all vertices between $M_1$ and $M_2$ are connected (since $V_1 \subseteq G_1$, $V_2 \subseteq G_2$). $M_1 \times M_2$
\item $M_1 \times M_2$ is maximal in $G_1 \times G_2$, because the only vertices we could add to $M_1 \times M_2$ are either in $G_1$ or $G_2$. However since $M_1$ is maximal in $G_1$, and $M_2$ is maximal in $G_2$, there are no other vertices we could add to $M_1 \times M_2$ to turn it into a larger clique.
\end{itemize}

Since all maxcliques of $G_1 \times G_2$ are the maxcliques of $G_1$ times the maxcliques of  $G_2$, $\text{MC}(G_1 \times G_2) = \text{MC}(G_1)  \times \text{MC}(G_2)$. $\square$
\end{pf}

\begin{theorem}\label{maxcliquetimes}
Given $G_1 = (V_1,E_1)$ and $G_2 = (V_2,E_2)$ simple graphs with $V_1 \cap V_2 = \emptyset$
$$\# (G_1 \times G_2) = \#(G_1) \#(G_2)$$
\end{theorem}

\begin{pf}
Using Theorem \ref{cartesian},
$$\# (G_1 \times G_2) = |\text{MC}(G_1 \times G_2)| = |\text{MC}(G_1) \times \text{MC}(G_2)| = |\text{MC}(G_1)|\ |\text{MC}(G_2)| = \#(G_1) \#(G_2)\ \square$$
\end{pf}


\begin{theorem}\label{times}
Given $G_1 = (V_1,E_1)$ and $G_2 = (V_2,E_2)$ simple graphs with $V_1 \cap V_2 = \emptyset$, if $G_1$ and $G_2$ are both cliqueful then $G_1 \times G_2$ is also cliqueful.
\end{theorem}

\begin{pf}
Let us denote $G_1 \times G_2=(V,E)$. We will prove the statement by checking the conditions of Theorem \ref{star}.

$\forall e \in G_1 \times G_2$, $e$ can either be in $E_1, E_2$, or $E_{G_1 \times G_2}(V_1,V_2)$. If $e \in E_1$, then, since $G_1$ is cliqueful, according to Theorem \ref{star}, $e_1$ is in a maxclique $M_1$ of $G_1$. Using Theorem \ref{cartesian}, this means that $e$ is also part of a maxclique in $G_1 \times G_2$, namely all maxcliques that contain $M_1$. Symmetrically, if $e \in E_2$, then it is also in a maxclique of $G_1 \times G_2$. And if $e \in E_{G_1 \times G_2}(V_1,V_2)$, then $e$ is in all maxcliques of $G_1 \times G_2$, because according to Theorem \ref{cartesian}, all maxcliques of $G_1 \times G_2$ contain all edges between $V_1$ and $V_2$. This means that the first condition of Theorem \ref{star} is satisfied.

For the second condition, let us investigate where $\overline{e} \in \overline{E}$. It can either be in $\overline{E_1}$ or $\overline{E_2}$, but not in $\overline{E_{G_1 \times G_2}(V_1,V_2)}$, since all edges in $E_{G_1 \times G_2}(V_1,V_2) \in E$, by the definition of the direct product. If $\overline{e} \in \overline{E_1}$, then, since $G_1$ is cliqueful, using Theorem \ref{star}, $\overline{e}$ is part of a maxclique $M_1$ in $G_1 \cup \overline{e}$. By Theorem \ref{cartesian}, the maxcliques of $(G_1 \cup \overline{e}) \times G_2$ are $\text{MC}(G_1 \cup \overline{e}) \times \text{MC}(G_2)$. Since $M_1$ is part of a maxclique of $G_1$, all edges of $M_1$ are part of a maxclique in $(G_1 \cup \overline{e}) \times G_2$. The argument can be done symmetrically for $\overline{e} \in \overline{E_2}$ as well. Therefore, the second condition of Theorem \ref{star} is also satisfied.

So according to Theorem \ref{star}, $G_1 \times G_2$ is also cliqueful.  $\square$
\end{pf}

\begin{definition}
$G = (V,E)$ cliqueful graph is \textbf{composite}, if $\exists G_1, G_2$ cliqueful graphs such that $G = G_1 \times G_2$
\end{definition}

\begin{definition}
$G = (V,E)$ cliqueful graph is \textbf{prime}, if it is not composite.
\end{definition}
\begin{rmk}
The decomposition of a $G$ composite cliqueful graph may not not unique. See Figure \ref{non_unique_factorization} for an example.
\end{rmk}

%\begin{wrapfigure}{r}{0.35\textwidth}
    %\vspace{-1cm}
%	\centering
%	% Figure removed
%	\caption{Example of a non-unique decomposition of a cliqueful graph into $2$ components}
%	\label{non_unique_factorization}
    %\vspace{-2cm}
%\end{wrapfigure}

% Figure environment removed

\begin{samepage}
\begin{theorem}\label{factors}
If $G$ is a composite cliqueful graph with $G=G_1 \times G_2$, then $G_1$ and $G_2$ are both cliqueful.
\end{theorem}

\begin{pf}
Let us denote $G=(V,E), G_1=(V_1,E_1), G_2=(V_2,E_2)$ . We will only prove the statement for $G_1$, the proof for $G_2$ can be done in the exact same way.

So we need to check if the conditions for Theorem \ref{star} still hold for $G_1$.

For the first condition, $\forall e \in E_1$, since $G_1$ is cliqueful, using Theorem \ref{star}, $e$ is contained in a maxclique of $G_1$, call it $M_e$. Then $e$ is indeed in a maxclique of $G_1$, namely all maxcliques of form $M_e \times M_2$, where $M_2$ is any maxclique in $G_2$.

For the second condition, $\forall \overline{e} \in \overline{E_1}$, since $G_1$ is cliqueful, using Theorem \ref{star}, $\overline{e}$ is contained in a maxclique of $G_1 \cup \overline{e}$, call it $M_{\overline{e}}$. Using Theorem \ref{cartesian}, all maxcliques of $(G_1 \cup \overline{e}) \cup G_2$ are of form $M_{\overline{e}} \times M_2$, where $M_2$ is any maxclique in $G_2$. So $\overline{e}$ is indeed in any of these maxcliques.

Using Theorem \ref{star}, $G_1$ is cliqueful. By swapping all $1$'s and $2$'s in the proof, $G_2$ can also be proved to be cliqueful. $\square$
\end{pf}
\end{samepage}

\begin{theorem}
The following algorithm gives an efficient, $O(n^3)$ factorization of an input $G=(V,E)$ cliqueful graph on $|V|=n$ vertices:
\begin{enumerate}
\item Let $S_1 = \{v\}$, where $v\in V$ is an arbitrary starting vertex.
\item Find a vertex $w \in V \setminus S_1$ that does not connect to any vertex in $S_1$.
\item If there exists one, add it to $S_1$, so let $S_1 := S_1 \cup \{w\}$, and go to $2$.
\item If there is none, then return $S_1, S_2 := V \setminus S_1$, $S_1 = V(G_1), S_2 = V(G_2), G = G_1 \times G_2$.
\item If we have added all vertices to $S_1$, so if $S_1 = V$, then $G$ is prime.
\end{enumerate}
\end{theorem}

\begin{pf}
We need to prove $3$ things:
\begin{itemize}
\item If the algorithm returned $S_1, S_2 := V \setminus G_2$, $S_1 = V(G_1), S_2 = V(G_2)$, then is it true that $G = G_1 \times G_2$?
\item If the algorithm returned that $S_1 = V$, then is $G$ truly prime?
\item Is the runtime of the algorithm $O(n^3)$?
\end{itemize}

If the algorithm returned $S_1, S_2 \subset V$, that could have only been if $\forall v_1 \in S_1, v_2 \in S_2: e=(v_1,v_2) \in E$, or in other words, $E(S_1,S_2) \subseteq E$. By definition, in this case, $G = G_1 \times G_2$. Using Theorem \ref{factors}, $G_1$ and $G_2$ are both cliqueful, therefore $G = G_1 \times G_2$ is a cliqueful factorization.

In the algorithm returned $S_1 = V$, then there is no $E(S_1,S_2)$ subset of the edges. Since no matter what $v\in V$ the algorithm starts from, that $v$ is eventually going to be in either $S_1$ or $S_2$. When building up either $S_1$ or $S_2$, if we cannot find an $E(S_1,S_2)$ subset, there there is none in $E$. Therefore, $G$ cannot be split into $G_1 \times G_2$, meaning that $G$ has to be prime.

Finally, the runtime of the algorithm in $O(n^3)$, since every time a vertex is added to $S_1$, in the worst case, the algorithm has to check whether or not any of $v_1 \in S_1$ connect to any of $v_2 \in V \setminus S_1$, which is $O(n^2)$ comparisons in the worst case. So $n$ times $O(n^2)$ comparisons gives $O(n^3)$ comparisons. $\square$
\end{pf}

\begin{theorem}\label{no_bipartite_subset}
If $G=(V,E)$ is a prime cliqueful graph with $|V|=n$, then $G$ does not contain a $K_{b,n-b}$ subgraph for any $b \in \{1,\dots,n-1\}$.
\end{theorem}

\begin{pf}
Assume $\exists b \in \{1,\dots,n-1\} \exists K_{b,n-b}$ subgraph in $G$ between the partition $S_1, S_2 \subset V$. Then by Definition \ref{direct_product}, $G = G_1 \times G_2$, where $G_1 \subset G$ the graph induced by vertices $V_1$, and $G_2 \subset G$ the graph induced by vertices $V_2$. Since $G$ is cliqueful, using Theorem \ref{factors}, both $G_1$ and $G_2$ are cliqueful. Therefore $G$ is composite cliqueful. However this is a contradiction, because $G$ was prime cliqueful. So our original assumption was false, meaning that $G$ does not contain a $K_{b,n-b}$ subgraph for any $b \in \{1,\dots,n-1\}$.
\end{pf}

\begin{theorem}\label{prime_equiv}
Let $G$ be a cliqueful graph. Then
$$G\text{ is prime }\Longleftrightarrow \overline{G}\text{ is connected}$$
\end{theorem}

\begin{pf}
$(\Longrightarrow)$

$G$ is prime cliqueful $\Longrightarrow$ Due to Theorem \ref{no_bipartite_subset}, $G$ contains no $K_{b,n-b}$ subgraph for any $b \in \{1,\dots,n-1\}$ $\Longrightarrow$ $\overline{G}$ contains no $E_{b,n-b}$ subgraph for any $b \in \{1,\dots,n-1\}$ $\Longrightarrow$ $\overline{G}$ is connected.

$(\Longleftarrow)$

In this case, we will instead prove the equivalent $G$ is composite cliqueful $\Longrightarrow$ $\overline{G}$ is disconnected:

$G$ is composite cliqueful $\Longrightarrow$ $G=G_1 \times G_2$ with $G_1=(V_1,E_1), G_2=(V_2,E_2)$ $\Longrightarrow$ $G$ contains a $K_{|V_1|,n-|V_1|}$ along the partition $(S_1,S_2)$. $\Longrightarrow$ $\overline{G}$ contains an $E_{|V_1|,n-|V_1|}$ along the partition $(S_1,S_2)$ $\Longrightarrow$ $\overline{G}$ is disconnected. $\square$
\end{pf}

\begin{theorem}\label{factor_algo}
The following algorithm gives an efficient, $O(n^2)$ factorization of an input $G=(V,E)$ cliqueful graph on $|V|=n$ vertices:

\begin{enumerate}
\item Calculate $\overline{G}$.
\item Starting from an arbitrary vertex $v \in V$, use breadth-first search on $\overline{G}$ to determine if it is connected.
\item If it connected, then $G$ is prime. If it is disconnected into subgraphs $\overline{G} = \overline{G_1} \cup \overline{G_2} \cup \dots \cup \overline{G_m}$, then $G = G_1 \times G_2 \times \dots \times G_m$. 
\end{enumerate}
\end{theorem}

\begin{pf}
Using Theorem \ref{prime_equiv}, the output of the  algorithm is clearly correct.

It is also $O(n^2)$, because calculating $\overline{G}$ takes $\frac{n(n-1)}{2}=O(n^2)$ steps, and walking through the edges of $\overline{G}$ takes at most $|\overline{E}| \le \frac{n(n-1)}{2}=O(n^2)$ steps, which is overall an $O(n^2)$ amount.
\end{pf}

\section{Saturated cliqueful graphs}\label{saturatedcliqueful}

\begin{definition}\label{saturated}
Let $G=(V,E)$ be a cliqueful graph. $G$ is \textbf{saturated}, if
$$\forall \overline{e} \in \overline{E}: \text{M}(G\cup \overline{e}) = \text{M}(G)+1$$
\end{definition}


\begin{theorem}\label{star2}
For all simple graphs $G=(V^*,E^*)$,

$$G \text{ is saturated cliqueful } \Longleftrightarrow
\begin{cases} (1)\ \forall e \in E^*: e \in \bigcup_{M(V,E)\in\text{MC}(G)} E \\
(2)\ \forall \overline{e} \in \overline{E}^*: \overline{e} \in \bigcup_{M(V,E)\in\text{MC}(G\cup\overline{e})} E\ \text{ and }\ \text{M}(G\cup\overline{e})= \text{M}(G)+1
\end{cases}$$

Or in words, $G$ is saturated cliqeful exactly if all edges $e$ are in maxcliques of $G$, and all non-edges $\overline{e}$ are in maxcliques of $G \cup \overline{e}$ that are of size $\text{M}(G)+1$.
\end{theorem}

\begin{pf}
Combining Theorem \ref{star} with Definition \ref{saturated} yields this Theorem.
\end{pf}

\begin{theorem}
If $G_1$ and $G_2$ are both saturated cliqueful, then $G = G_1 \times G_2$ is also saturated cliqueful.
\end{theorem}

\begin{pf}
If $G_1$ and $G_2$ are both saturated cliqueful, then they are both cliqueful, and by Theorem \ref{times}, $G$ is cliqueful. So we only need to prove that $G$ is saturated, namely that $\forall \overline{e} \in \overline{E}: \text{M}(G\cup \overline{e}) = \text{M}(G)+1$.

Let $G_1 = (V_1,E_1), G_2 = (V_2,E_2)$. By the definition of the direct product, $\overline{e}$ can only be in either $E_1$ or $E_2$. Let us first examine the case when $\overline{e} \in E_1$.

Using Theorem \ref{cartesian}, all maxcliques of $G \cup \overline{e}$ are of form $M_1 \times M_2$, where $M_1 \in \text{MC}(G_1 \cup \overline{e})$ and $M_2 \in \text{MC}(G_2)$. Since $G_1$ is saturated, $\text{M}(G_1 \cup \overline{e}) = \text{M}(G_1) + 1$. So

$$\text{M}(G \cup \overline{e}) = \text{M}((G_1 \cup \overline{e}) \times G_2) = \text{M}(G_1 \cup \overline{e}) + \text{M}(G_2) = \text{M}(G_1) + 1 + \text{M}(G_2) = \text{M}(G_1 \times G_2) + 1 = \text{M}(G)+1$$

So the condition of Definition \ref{saturated} is fulfilled. The proof can be done similarly for $\overline{e} \in E_2$, by swapping each $1$ and $2$ index. $\square$
\end{pf}

\begin{theorem}
If $G=(V,E)$ is composite saturated cliqueful with $G = G_1 \times G_2$, then both $G_1$ and $G_2$ are saturated cliqueful.
\end{theorem}

\begin{pf}
Let $G_1 = (V_1,E_1), G_2 = (V_2,E_2)$.

If $G$ is saturated cliqueful, then it is also cliqueful, so using Theorem \ref{factors}, $G_1$ and $G_2$ are both cliqueful. So the only thing we need to prove is that they are also saturated, namely that $\forall i \in \{1,2\} \forall \overline{e} in \overline{E}_i: \text{M}(G_i \cup \overline{e}) = \text{M}(G_i)+1$. We will only prove this for $i=1$, the proof for $i=2$ goes similarly.

Let $\overline{e} in \overline{E}_1$. Using Theorem \ref{cartesian}, all maxcliques of $G \cup \overline{e}$ are of form $M_1 \times M_2$, where $M_1 \in \text{MC}(G_1 \cup \overline{e})$ and $M_2 \in \text{MC}(G_2)$. Since $G$ is saturated, $\text{M}(G \cup \overline{e}) = \text{M}(G) + 1$. So

\begin{align*}
\text{M}(G \cup \overline{e}) &= \text{M}(G_1 \cup \overline{e}) + \text{M}(G_2) \\
\text{M}(G)+1 &= \text{M}(G_1 \cup \overline{e}) + \text{M}(G_2) \\ 
\text{M}(G_1) + \text{M}(G_2)+1 &= \text{M}(G_1 \cup \overline{e}) + \text{M}(G_2) \\ 
\text{M}(G_1)+1 &= \text{M}(G_1 \cup \overline{e})
\end{align*}

Which proves the condition of Definition \ref{saturated}. The proof for $G_2$ goes similarly, by swapping each $1$ and $2$ index. $\square$
\end{pf}

\begin{theorem}\label{Mn2}
$M_n = \max \{\# G | G=(V,E), |V|=n, G\text{ is saturated cliqueful}\}$
\end{theorem}

\begin{pf}
Using Theorem \ref{Mn1}, we already know that
$$M_n = \max \{\# G | G=(V,E), |V|=n, G\text{ is cliqueful}\}$$
So assume that $\forall G=(V,E) \in M_n: G$ is cliqueful but not saturated cliqueful. Which means that $\forall \overline{e} \in \overline{E}: \overline{e}$ is part of a maxclique if $G \cup \overline{e}$ of size $\text{M}(G)$, which we will call $M_{\overline{e}}\in \text{MC}(G \cup \overline{e})$.

Let us take such an $\overline{e} \in \overline{E}$. Since adding $\overline{e}$ to $G$ adds a clique of size $\text{M}(G)$, it does not merge any two $M_1, M_2 \in \text{MC}(G)$ into a clique of size $\text{M}(G)$. Therefore, $\#(G \cup \overline{e}) > \#G$. So $\#G$ could not have been maximal, therefore $\#G \ne \text{M}_n$, so all cliqueful elements of $\text{MG}_n$ are also saturated cliqueful. $\square$
\end{pf}


\begin{theorem}\label{edge_lower}
$G=(V,E), |V|=n$ is a saturated cliqueful graph with $G=G_1\times G_2 \times \dots \times G_m$ and  $\forall i \in \{1,\dots,m\} G_i := (V_i,E_i), |V_i|=n_i$. Then
$$|E| \ge \sum_{1\le i < j \le m} n_i n_j + \frac{1}{2}\sum_{i=1}^m \text{M}(G_i) (\text{M}(G_i)-1) + \frac{1}{2} \sum_{i=1}^m (\#G_i-1)(\text{M}(G_i)-1)$$
\end{theorem}

\begin{pf}
Using Theorem \ref{cartesian}, $\forall v_1 \in V_i, v_2 \in V_j: e=(v_1,v_2) \in E$. That means that there are a total of $\sum_{1\le i < j \le m} n_i n_j$
edges inbetween $G_1, G_2, \dots, G_m$, so

$$|E| \ge \sum_{1\le i < j \le m} n_i n_j$$

Furthermore, any simple graph $G_i$ with maxcliques of size $m$ contains at least one maxclique of size $\text{M}(G)=m$, therefore contains at least as many edges as $K_m$, which is $\frac{m(m-1)}{2} = \frac{\text{M}(G)(\text{M}(G)-1)}{2}$. These edges all go inside each $G_i$, therefore they are different from the previously listed edges, so
$$|E| \ge \sum_{1\le i < j \le m} n_i n_j + \frac{1}{2}\sum_{i=1}^m \text{M}(G_i) (\text{M}(G_i)-1)$$

However if each $G_i$ contains more than $1$ maxclique, we can further improve this bound. For each subsequent maxclique of $G_i$, $G_i$ contains at least $1$ extra vertex $v$, because each pair of maxcliques have to be different. Since $v$ is part of a maxclique of $G_i$, its degree $d_{G_i}(v) \ge \text{M}(G_i)-1$. Therefore, for each $G_i$, there are an additional $(\#G_i-1)(\text{M}(G_i)-1)$ vertex degrees that are still unaccounted for. Since the sum of the vertex degrees are the half of the number of edges, there are an additional $\frac{1}{2}(\#G_i-1)(\text{M}(G_i)-1)$ edges that are still unaccounted for, so
$$|E| \ge \sum_{1\le i < j \le m} n_i n_j + \frac{1}{2}\sum_{i=1}^m \text{M}(G_i) (\text{M}(G_i)-1) + \frac{1}{2} \sum_{i=1}^m (\#G_i-1)(\text{M}(G_i)-1) \quad \square$$
\end{pf}

\begin{theorem}\label{edge_upper}
$G=(V,E), |V|=n$ is a cliqueful graph with $G=G_1\times G_2 \times \dots \times G_m$ such that $\forall i \in \{1,\dots,m\}$ $G_i=(V_i,E_i), |V_i|=n$, and $G_i$ is prime. Then
$$|\overline{E}| \ge \sum_{i=1}^m (n_i-1)$$
\end{theorem}

\begin{pf}
Using Theorem \ref{prime_equiv}, each $\overline{G}_i, i \in \{1,\dots,m\}$ are connected, therefore each one has at least $n_i-1$ edges. So the total number of complement edges are
$$|\overline{E}| \ge \sum_{i=1}^m (n_i-1) \quad \square$$
\end{pf}

\begin{theorem}\label{edge_bound}
$G=(V,E), |V|=n$ is a saturated cliqueful graph with $G=G_1\times G_2 \times \dots \times G_m$ such that $\forall i \in \{1,\dots,m\}$ $G_i=(V_i,E_i), |V_i|=n$, and $G_i$ is prime. Then
$$\sum_{1\le i < j \le n} n_i n_j + \frac{1}{2}\sum_{i=1}^m \text{M}(G_i) (\text{M}(G_i)-1) + \frac{1}{2}\sum_{i=1}^m (\#G_i-1)(\text{M}(G_i)-1) \le |E| \le \frac{n(n-1)}{2} - \sum_{i=1}^m (n_i-1)$$
\end{theorem}

\begin{pf}
The lower bound on $|E|$ is the exact same as in Theorem \ref{edge_lower}. The upper bound is obtained from Theorem \ref{edge_upper}, as we are missing at least as many edges from $K_n$ as there are in $|\overline{E}|$, which is at least $\sum_{i=1}^m (n_i-1)$ edges. So $|E| \le \frac{n(n-1)}{2} - \sum_{i=1}^m (n_i-1)$. $\square$ 
\end{pf}

\begin{rmk}
Theorem \ref{edge_bound} gives a very strict upper and lower bound for the number of edges in a composite saturated cliqueful graph. As an example, consider a composite saturated cliqueful graph $G = (V,E)$ with the decomposition $G = G_1 \times G_2 \times G_3$, where $G_1 = (V_1,E_1),G_2 = (V_2,E_2),G_3 = (V_3,E_3)$, and 
\begin{itemize}
\item $|V_1|=5, |V_2|=6, |V_3|=7$
\item $\text{M}(G_1)=2,\text{M}(G_2)=3,\text{M}(G_3)=3$
\item $\#G_1 = 7, \#G_2 = 3, \#G_3 = 4$
\end{itemize}
Then the edge bound obtained from Theorem \ref{edge_bound} is
$$135 \le |E| \le 138$$
\end{rmk}

\begin{theorem}\label{primecliques}
If $G=(V,E), |V|=n$ is a saturated prime cliqueful graph, then
$$\#G \le n^2$$
\end{theorem}

\begin{pf}
Using Theorem \ref{edge_bound}, then the transitive property of inequalities, then subtracting some terms from the left side:
\begin{align*}
\sum_{1\le i < j \le n} n_i n_j + \frac{1}{2}\sum_{i=1}^m \text{M}(G_i) (\text{M}(G_i)-1) + \frac{1}{2}\sum_{i=1}^m (\#G_i-1)(\text{M}(G_i)-1) &\le |E| \le \frac{n(n-1)}{2} - \sum_{i=1}^m (n_i-1) \\
\sum_{1\le i < j \le n} n_i n_j + \frac{1}{2}\sum_{i=1}^m \text{M}(G_i) (\text{M}(G_i)-1) + \frac{1}{2}\sum_{i=1}^m (\#G_i-1)(\text{M}(G_i)-1) &\le \frac{n(n-1)}{2} - \sum_{i=1}^m (n_i-1) \\
\frac{1}{2}\sum_{i=1}^m (\#G_i-1)(\text{M}(G_i)-1) \le \frac{n(n-1)}{2} - \sum_{i=1}^m (n_i-1) - \sum_{1\le i < j \le n} n_i n_j &- \frac{1}{2}\sum_{i=1}^m \text{M}(G_i) (\text{M}(G_i)-1)
\end{align*}
Each subtracted term on the right side is $\ge 0$, so we can increase the right side by omitting them:
\begin{align*}
\frac{1}{2}\sum_{i=1}^m (\#G_i-1)(\text{M}(G_i)-1) &\le \frac{n(n-1)}{2}
\end{align*}
Since $G$ is prime, its prime factors are itself, so $m=1$:
\begin{align*}
\frac{1}{2}(\#G-1)(\text{M}(G)-1) &\le \frac{n(n-1)}{2}
\end{align*}
Multiplying by $2$ and dividing by $(\text{M}(G)-1)$ gives:
\begin{align*}
\#G-1 &\le \frac{n(n-1)}{\text{M}(G)-1} \\
\#G &\le \frac{n(n-1)}{\text{M}(G)-1}+1
\end{align*}
There are two cases:
\begin{enumerate}
\item $\text{M}(G)=1$
\item $\text{M}(G) \ge 2$
\end{enumerate}
In the first case, $G$ is the empty graph, so it has exactly $n$ maxcliques, which is $\le n^2$, since $n\ge 1$.
In the second case, $\text{M}(G)-1\ge 1$, so by omitting it, we further increase the right side:
\begin{align*}
\#G &\le n(n-1)+1
\end{align*}
Finally, since $n\ge 1$, $\#G \le n(n-1)+1 \le n^2$. So in all cases we have obtained that $\#G \le n^2$. $\square$
\end{pf}

\section{The maximum number of maxcliques}\label{maxmaxcliques}

\begin{definition}
Let $G$ be a simple graph and $m\in\mathbb{N}^+$. Then
$$G^m := \underbrace{G \times \dots \times G}_{m} $$
\end{definition}

\begin{theorem}\label{construction}
$\exists G=(V,E), |V|=n$ such that $\#G \ge 3^{\lfloor n/3 \rfloor}$
\end{theorem}

\begin{pf}
Let $G_0=(E_3)^{\lfloor n/3 \rfloor}$. $G_0$ has $3 \lfloor \frac{n}{3} \rfloor$ vertices, which is equal to $n$ if $n$ is divisible by $3$. Otherwise, add $1$ or $2$ more disconnected vertices to $G_0$ to obtain $G$.

Then using Theorem \ref{maxcliquetimes}:
$$\# G \ge \# G_0 = \#(E_3^{\lfloor n/3 \rfloor}) = (\#(E_3))^{\lfloor n/3 \rfloor} = 3^{\lfloor n/3 \rfloor}$$

So there exists a graph with this many maxcliques. $\square$
\end{pf}

\begin{theorem}
If $n \ge 15$, then
$$M_n = \max \{\# G | G=(V,E), |V|=n, G\text{ is composite saturated cliqueful}\}$$
\end{theorem}

\begin{pf}
The inequality $n^2 < 3^{\lfloor n/3 \rfloor}$ over the whole numbers is satisfied when $n \ge 15$.

Let $G=(V,E)$ be a prime cliqueful graph with $|V|=n$. Using Theorem \ref{construction}, if $n \ge 15$, then:
$$\#G \le n^2 < 3^{\lfloor n/3 \rfloor} \le M_n$$

So $G$ cannot be in $MG_n$. Since in Theorem \ref{Mn2} we have already proven that $G$ must be saturated cliqueful, $M_n$ is further reduced to composite saturated cliqueful graphs in the case when $n \ge 15$. $\square$
\end{pf}

\begin{theorem}\label{large_triangleless}
$\exists G=(V,E)$ triangle-less graph on $|V|=n$ vertices with
$$\#G \ge \frac{(n-1)(n+1)}{4}$$
\end{theorem}

\begin{pf}
If $G$ is triangle-less, then $\#G = |E|$.

If $n$ is even, then let $G := K_{n/2,n/2}$. This graph has $\frac{n^2}{4} > \frac{(n-1)(n+1)}{4}$ edges. It is triangle-less, so it also has at least this many maxcliques.

If $n$ is odd, then let $G := K_{\lfloor n/2 \rfloor, \lceil n/2 \rceil}$. This graph has $\lfloor\frac{n}{2}\rfloor \lceil\frac{n}{2}\rceil = \frac{(n-1)(n+1)}{4}$ edges. It is also triangle-less, so it also has at least this many maxcliques. $\square$
\end{pf}

\begin{theorem}\label{lessthan5}
Let $G=(V,E)$ be a composite saturated cliqueful graph $G=G_1 \times \dots \times G_m$ with $\text{M}(G_i) \ge 5$ for some $i \in \{1,\dots,m\}$. Then $G \notin \text{MG}_n$.
\end{theorem}

\begin{pf}
$\forall k \in \{1,\dots,m\}$, let us denote $G_k=(V_k,E_k)$, $|V_k|=n_k$, and the graph with maxcliques of size at least $\text{M}(G_i) \ge 5$ by $G_i$.

Let us use Theorem \ref{edge_bound} for $G_i$, so with $m=1$. Just like in the Proof of Theorem \ref{primecliques}, we can obtain a bound:

\begin{align*}
\frac{1}{2} \text{M}(G_i)(\text{M}(G_i)-1) + \frac{1}{2}(\#G_i-1)(\text{M}(G_i)-1) &\le |E| \le \frac{n_i(n_i-1)}{2}-(n_i-1) = \frac{(n_i-1)(n_i-2)}{2} \\
(\text{M}(G_i)-1)(\text{M}(G_i)+\#G_i-1) &\le (n_i-1)(n_i-2) \\
\text{M}(G_i)+\#G_i-1 &\le \frac{(n_i-1)(n_i-2)}{M(G_i)-1} \\
\#G_i &\le \frac{(n_i-1)(n_i-2)}{M(G_i)-1}-(\text{M}(G_i)-1) \\
\end{align*}

Since $\text{M}(G_i) \ge 5$, the right side can be estimated as
$$\#G_i \le \frac{(n_i-1)(n_i-2)}{M(G_i)-1}-(\text{M}(G_i)-1) \le \frac{(n_i-1)(n_i-2)}{4}-4 < \frac{(n_i-1)(n_i+1)}{4} \le \#\hat{G}$$

where $\hat{G}$ is a graph on $n_i$ vertices with at least $\frac{(n_i-1)(n_i+1)}{4}$ maxcliques, which was shown to exist in Theorem \ref{large_triangleless}.

Putting things together, using Theorem \ref{maxcliquetimes}:

$$\#G = \#G_1 \times \dots \times \#G_{i-1} \times \#G_i \times \#G_{i+1} \times \dots \times \#G_m < \#G_1 \times \dots \times \#G_{i-1} \times \#\hat{G} \times \#G_{i+1} \times \dots \times \#G_m := \#G^{+}$$

So the graph $G^+$, defined by swapping $G_i$ into $\hat{G}$ in the product, has the same number of vertices as $G$, but more maxcliques than $G$, therefore $G \notin \text{MG}_n$. $\square$
\end{pf}

\begin{theorem}
All saturated cliqueful graphs $G=(V,E)=G_1 \times \dots \times G_m$ in $\text{MG}_n$ on $|V|=n \ge 15$ vertices are made out of the following prime components:
$$\forall i \in \{1,\dots,m\}: G_i \in \{E_1,E_2,E_3,E_4,C_4\}$$
\end{theorem}

\begin{pf}
Let $G_i := (V_i,E_i), |V_i| = n_i$. Let us list all simple graphs on at most $4$ vertices, and see which ones have the highest amount of maxcliques:
\begin{itemize}
\item On $n_i=1$ vertex, there is only one graph, $E_1$ with $\#E_1=1$. Therefore, $E_1$ is optimal on $n_i=1$ vertex.
\item On $n_i=2$ vertices, there are two graphs: $E_2$ and $K_2$ with $\#E_2=2, \#K_2=1$. Therefore, $E_2$ is optimal on $n_i=2$ vertices.
\item On $n_i=3$ vertices, there are $4$ graphs: $E_3$, $E_3 \cup e$, $K_3 \setminus e$, and $K_3$, with $\#E_3=3, \#(E_3 \cup e)=1, \#(K_3 \setminus e)=2, \#K_3=1$. Therefore, $E_3$ is optimal on $n_i=3$ vertices.
\item On $n_i=4$ vertices, there are $11$ graphs. Figure \ref{all4s} shows all such graphs and their maxclique amounts. We can see that $E_4$ and $C_4$ have the most maxcliques, $4$.
\end{itemize}

% Figure environment removed

According to Theorem \ref{maxcliquetimes}, $\#G = \#G_1 \cdot \dots \cdot \#G_m$, and according to Theorem \ref{lessthan5}, all graphs in a prime decomposition must have $\text{M}(G_i) \le 4$, so $\forall i \in \{1,\dots,m\} G_i \in \{E_1,E_2,E_3,E_4,C_4\}$, otherwise $G$ would have fewer than maximal number of maxcliques. $\square$
\end{pf}

\begin{rmk}
Note that any two $E_2$'s, any one $E_4$ or any one $C_4$ can be freely swapped to one another in a prime decomposition of $G$, and $G$ will still retain the same number of maxcliques, because in the product $\#G = \#G_1 \times \dots \times \#G_m$, a factor of $4$ is being replaced by a different factor of $4$ with any of these swaps.
\end{rmk}

\begin{theorem}\label{final}
If $n \ge 15$, then
$$M_n = \begin{cases}
3^{\lfloor n/3 \rfloor} \text{ if }n \text{ mod } 3 = 0 \\
4 \cdot 3^{(\lfloor n/3 \rfloor-1)} \text{ if }n \text{ mod } 3 = 1 \\
2 \cdot 3^{\lfloor n/3 \rfloor} \text{ if }n \text{ mod } 3 = 2
\end{cases}$$
\end{theorem}

\begin{pf}
With the previous remark we can see that if there is an optimal graph $G_1 \in \text{MG}_n$ that has an $E_4$ or a $C_4$ component, then there exists at least one more $G_2 \in \text{MG}_n$ with the $E_4$'s replaced by two $E_2$'s and the $C_4$'s replaced by two $E_2$'s as well, so it is enough to search for the optimal graph with components $E_1$, $E_2$, or $E_3$.

Similarly, two $E_1$'s can be replaced by an $E_2$ to obtain more maxcliques in $G$, as two $E_1$'s would multiply the total number of maxcliques by $1\cdot 1=1$, while one $E_2$ would multiply it by $2$. So there can only be at most $1$ $E_1$ in a $G \in \text{MG}_n$.

And an optimal graph $G \in \text{MG}_n$ cannot both have an $E_1$ component and an $E_2$ component in it, because those could be replaced by an $E_3$ component to obtain $3$ maxcliques in $G$ istead of $2\cdot 1=2$. So any optimal graph may only have at most one $E_1$ and $E_3$'s in it, or some number of $E_2$'s and some number of $E_3$'s.

Let $G=(V,E) \in \text{MG}_n$ with $|V|=n$ and $G=G_1 \times \dots \times G_m$, and let us assume there are $t \le \lfloor \frac{n}{3} \rfloor$ number of $E_3$'s in its decomposition. Then there must be $\frac{n-3t}{2}$ $E_2$'s in its decomposition. So
$$G= (E_3)^t \times (E_2)^{\frac{n-3t}{2}}$$

Then the number of maxcliques in $G$ is
$$f(t):=\#G=3^t \cdot 2^{\frac{n-3t}{2}}$$

With some algebraic manipulation, we can rewrite $f(t)$ as
$$f(t)=3^t \cdot 2^{\frac{n-3t}{2}} = \left(2^{\log_2(3)}\right)^t 2^{\frac{n-3t}{2}} = 2^{t \log_2(3)+\frac{n-3t}{2}} = 2^{t\left(\log_2(3)-\frac{3}{2}\right)+\frac{n}{2}} \approx 2^{n/2} 2^{0.085t}$$
Which is an increasing function in $t$, meaning that its maximum is taken on when $t$ is maximal, namely when $t=\lfloor \frac{n}{3} \rfloor$.

Which means that if $n$ is divisible by $3$, then an optimal graph is
$$G = (E_3)^{n/3}$$
which has $\#G = 3^{n/3}$ maxcliques.

Otherwise, one $E_1$ or one $E_2$ can be appended.

If $n = 3k + 1, k \in \mathbb{N}$, then an optimal graph is
$$G = (E_3)^{(\lfloor n/3 \rfloor-1)} \times (E_2)^2$$
which has $\#G = 3^({\lfloor n/3 \rfloor}-1) \cdot 4$ maxcliques.

And if $n = 3k+2, k \in \mathbb{N}$, then an optimal graph is
$$G = (E_3)^{\lfloor n/3 \rfloor} \times E_2$$
which has $\#G = 3^{\lfloor n/3 \rfloor} \cdot 2$ maxcliques.

Overall, these three formulas can be expressed as
$$M_n = \begin{cases}
3^{\lfloor n/3 \rfloor} \text{ if }n \text{ mod } 3 = 0 \\
4 \cdot 3^{(\lfloor n/3 \rfloor-1)} \text{ if }n \text{ mod } 3 = 1 \\
2 \cdot 3^{\lfloor n/3 \rfloor} \text{ if }n \text{ mod } 3 = 2
\end{cases} \quad \square$$
\end{pf}

%\section{Maximum number of maxcliques}

%\subsection{The main result}

%The main result of the study an asymptotic approximation for $M_n$:

%$$\boxed{M_n \asymp e^{cn}}$$

%where $\frac{1}{e} \le c \le \ln(2)$, or approximately: $0.36 \le c \le 0.7$.

%\vspace{2mm}

%This is equivalent to: $\exists n_0 \in \mathbb{N}$, $\exists c_1,c_2 \in \mathbb{R}^+$ such that $\forall n > n_0$:

%$$\boxed{c_1 e^{\frac{n}{e}} \le M_n \le c_2 e^{\ln(2)n}}$$

%In the following subsections, our aim will be to prove this statement.

%\subsection{An upper bound}

%\begin{theorem}
%$\exists n_0 \in \mathbb{N}$ such that $\forall n > n_0$:

%$$M_n \le \left\lfloor\sqrt{\frac{2}{\pi n}}\ 2^n\right\rfloor$$
%\end{theorem}

%\begin{pf}
%Let a $G=(V,E)$ simple graph be given, and let us denote the size of its maxcliques by $k$.

%Each maxclqiue has $k$ vertices, and for the upper bound we will assume that no matter how we choose $k$ vertices, all of the will form a maxclique. The number of way to choose $k$ out of $n$ vertices is $n \choose k$.

%$M_n$ is certainly smaller than the maximum of the expression $n \choose k$ over all potential $k$'s.

%$$M_n \le \max_{k \in \{1,\dots,n\}} {n \choose k} = {n \choose \lfloor n/2 \rfloor}$$

%Where the last equality holds, because the highest value of each row of the Pascal triangle is the central (or central two) element(s).

%Since we are looking for an asymptotic approximation, will assume that $n$ is divisible by $2$, so $\lfloor n/2 \rfloor = \frac{n}{2}$.

%We will now use Stirling's approximation for $n!$ on the right hand side, and later argue that this will still give an upper bound.

%\begin{align*}
%M_n &\le {n \choose \lfloor n/2 \rfloor} = \frac{n!}{\left(\frac{n}{2}\right)!\left(\frac{n}{2}\right)!} = \frac{n!}{\left(\left(\frac{n}{2}\right)!\right)^2} = \frac{\sqrt{2\pi n}\left(\frac{n}{e}\right)^n \left(1+O\left(\frac{1}{n}\right)\right)}{\left(\sqrt{\pi n}\left(\frac{n}{2e}\right)^{\frac{n}{2}} \left(1+O\left(\frac{2}{n}\right)\right)\right)^2} = \\ &= \frac{\sqrt{2\pi n}\left(\frac{n}{e}\right)^n \left(1+O\left(\frac{1}{n}\right)\right)}{\pi n \left(\frac{n}{2e}\right)^n \left(1+O\left(\frac{1}{n}\right)\right)^2} = \frac{\sqrt{2}\ 2^n}{\sqrt{\pi n} \left(1+O\left(\frac{1}{n}\right)\right)} = \sqrt{\frac{2}{\pi n}} 2^n\ O\left(\frac{1}{1+\frac{1}{n}}\right)  = \\ &= \sqrt{\frac{2}{\pi n}}\ 2^n\ O\left(\frac{n}{n+1}\right) = \sqrt{\frac{2}{\pi n}}\ 2^n\ O(1)
%\end{align*}

%Since $\forall n \in \mathbb{N}^+: \frac{n}{n+1} < 1$, the $O(1)$ constant at the end will always be $<1$, and for $n\to\infty$ it will tend to $1$; meaning that $\sqrt{\frac{2}{\pi n}}\ 2^n$ is an actual upper bound for $M_n$.

%Whenever $\sqrt{\frac{2}{\pi n}}\ 2^n$ does not take on a whole value, its lower whole part is also an appropriate upper bound, since $M_n$ is always a whole number. $\square$
%\end{pf}

%\subsection{Lower bound constructions}

%\begin{theorem}
%Let $r \in \mathbb{N}^+, r \ge 2$, and let $n$ be an integer divisible by $r$. Let $G^C$ be the disjunct union of $\frac{n}{r}$ $K_r$'s. Then $G$ (which is the complement of $G^C$), has $2^{\frac{n}{r}}$ maxcliques of size $\frac{n}{r}$.
%\end{theorem}

%\begin{theorem}
%Let $r \in \mathbb{N}^+, r \ge 2$, and let $n$ be an integer divisible by $r$. Let $G^C$ be the disjunct union of $\frac{n}{r}$ $K_r$'s. Then $G$ has the most number of maxcliques if $r=3$.

%\begin{theorem}
%The function $f(x)=x^{\frac{1}{x}}$ takes on its maximum value at $x=e$.
%\end{theorem}
%\end{theorem}

%\begin{theorem}
%Let $t \in \mathbb{N}$. Let $G^C$ be the disjunct union of $t$ $K_2$'s and $\frac{n-2t}{3}$ $K_3$'s. (This way, $G^C$ has $n$ vertices.) Then $G$ has the most number of maxcliques if $t$ is the closest integer to $\left(\frac{\frac{3}{e}-\ln(3)}{2\ln(3)-3\ln(2)}\right)n$, or approximately $0.042672n$.
%\end{theorem}

%\begin{theorem}
%Let $n\in\mathbb{N}^+$ and $G^C=(V,E)$ with $|V|=n$ be the following graph: Let $t=\left(\frac{\frac{3}{e}-\ln(3)}{2\ln(3)-3\ln(2)}\right)n$, and let $G^C$ consist of $\lfloor t \rfloor$ $K_2$'s and $\frac{n-2\lfloor t \rfloor}{3}$ $K_3$'s. Let us denote the number of maxcliques in $G$ by $m_n$. Then
%$$\lim_{n \to \infty} m_n = e^{n/e}$$
%\end{theorem}

%\begin{theorem}
%$\exists n_0 \in \mathbb{N}^+$ such that $\forall n >n_0$:
%$$\lfloor e^{n/e} \rfloor \le M_n$$
%\end{theorem}

%\begin{theorem}
%$M_n \asymp e^{cn}$, where $\frac{1}{e} \le c \le \ln(2)$.
%\end{theorem}

%\begin{pf}

%\end{pf}

%\printcredits

%% Loading bibliography style file
%\bibliographystyle{model1-num-names}
\bibliographystyle{cas-model2-names}

% Loading bibliography database
%\bibliography{Daniel_Pfeifer_Cliqueful_graphs_as_a_means_of_calculating_the_maximal_number_of_maximum_cliques_of_simple_graphs}

\begin{thebibliography}{1}
\expandafter\ifx\csname natexlab\endcsname\relax\def\natexlab#1{#1}\fi
\providecommand{\url}[1]{\texttt{#1}}
\providecommand{\href}[2]{#2}
\providecommand{\path}[1]{#1}
\providecommand{\DOIprefix}{doi:}
\providecommand{\ArXivprefix}{arXiv:}
\providecommand{\URLprefix}{URL: }
\providecommand{\Pubmedprefix}{pmid:}
\providecommand{\doi}[1]{\href{http://dx.doi.org/#1}{\path{#1}}}
\providecommand{\Pubmed}[1]{\href{pmid:#1}{\path{#1}}}
\providecommand{\bibinfo}[2]{#2}
\ifx\xfnm\relax \def\xfnm[#1]{\unskip,\space#1}\fi
%Type = Article
\bibitem[{Moon and Moser(1965)}]{moon1965cliques}
\bibinfo{author}{Moon, J.W.}, \bibinfo{author}{Moser, L.},
  \bibinfo{year}{1965}.
\newblock \bibinfo{title}{On cliques in graphs}.
\newblock \bibinfo{journal}{Israel journal of Mathematics} \bibinfo{volume}{3},
  \bibinfo{pages}{23--28}.
\end{thebibliography}

\end{document}

