%% 
%% Copyright 2019-2020 Elsevier Ltd
%% 
%% This file is part of the 'CAS Bundle'.
%% --------------------------------------
%% 
%% It may be distributed under the conditions of the LaTeX Project Public
%% License, either version 1.2 of this license or (at your option) any
%% later version.  The latest version of this license is in
%%    http://www.latex-project.org/lppl.txt
%% and version 1.2 or later is part of all distributions of LaTeX
%% version 1999/12/01 or later.
%% 
%% The list of all files belonging to the 'CAS Bundle' is
%% given in the file `manifest.txt'.
%% 
%% Template article for cas-sc documentclass for 
%% double column output.

%\documentclass[a4paper,fleqn,longmktitle]{cas-sc}
\documentclass[a4paper,fleqn]{cas-sc}

% \usepackage[numbers]{natbib}
%\usepackage[authoryear]{natbib}
\usepackage[authoryear,longnamesfirst]{natbib}
\usepackage{mathtools}
\usepackage[font={small}]{caption} % for the captionof command

%%%Author definitions
\def\tsc#1{\csdef{#1}{\textsc{\lowercase{#1}}\xspace}}
\tsc{WGM}
\tsc{QE}
\tsc{EP}
\tsc{PMS}
\tsc{BEC}
\tsc{DE}
%%%

% Uncomment and use as if needed
\newdefinition{definition}{Definition}
\newtheorem{theorem}{Theorem}
\newtheorem{lemma}[theorem]{Lemma}
\newdefinition{rmk}{Remark}
\newproof{pf}{Proof}
\newproof{pot}{Proof of Theorem \ref{thm}}

\begin{document}
\let\WriteBookmarks\relax
\def\floatpagepagefraction{1}
\def\textpagefraction{.001}

% Short title
\shorttitle{Number of maxcliques}

% Short author
\shortauthors{D. Pfeifer}

% Main title of the paper
\title [mode = title]{Calculating the maximum number of maximum cliques for simple graphs}                     
% Title footnote mark
% eg: \tnotemark[1]
%\tnotemark[1,2]

% Title footnote 1.
% eg: \tnotetext[1]{Title footnote text}
% \tnotetext[<tnote number>]{<tnote text>} 
%\tnotetext[1]{This document is the results of the research
%   project funded by the National Science Foundation.}

%\tnotetext[2]{The second title footnote which is a longer text matter
%   to fill through the whole text width and overflow into
%   another line in the footnotes area of the first page.}


% First author
%
% Options: Use if required
% eg: \author[1,3]{Author Name}[type=editor,
%       style=chinese,
%       auid=000,
%       bioid=1,
%       prefix=Sir,
%       orcid=0000-0000-0000-0000,
%       facebook=<facebook id>,
%       twitter=<twitter id>,
%       linkedin=<linkedin id>,
%       gplus=<gplus id>]
\author[1]{Dániel Pfeifer}[type=editor,orcid=0000-0002-2434-6052]

% Corresponding author indication
%\cormark[1]

% Footnote of the first author
%\fnmark[1]

% Email id of the first author
\ead{pfeiferd@math.bme.hu}

% URL of the first author
%\ead[url]{www.cvr.cc, cvr@sayahna.org}

%  Credit authorship
%\credit{Conceptualization of this study, Methodology, Software}

% Address/affiliation
\affiliation[1]{organization={Budapest University of Technology and Economics},
    addressline={Műegyetem rkp. 3}, 
    city={Budapest},
    % citysep={}, % Uncomment if no comma needed between city and postcode
    postcode={1111}, 
    % state={},
    country={Hungary}}

% Here goes the abstract
\begin{abstract}
A simple graph on $n$ vertices may contain a lot of maximum cliques. But how many can it potentially contain? We will define prime and composite graphs, and we will show that if $n \ge 15$, then the grpahs with the maximum number of maximum cliques have to be composite. Moreover, we will show an edge bound from which we will prove that if any factor of a composite graph has $\omega(G_i) \ge 5$, then it cannot have the maximum number of maximum cliques. Using this we will show that the graph that contains $3^{\lfloor n/3 \rfloor}c$ maximum cliques has the most number of maximum cliques on $n$ vertices, where $c\in\{1,\frac{4}{3},2\}$, depending on $n \text{ mod } 3$.
\end{abstract}

% Use if graphical abstract is present
% \begin{graphicalabstract}
% % Figure removed
% \end{graphicalabstract}

% Research highlights
%\begin{highlights}
%\item Proving that graphs that contain the maximum number of maximum cliques on $n \ge 15$ vertices are composite.
%\item Showing equivalence between graphs' primeness and the connectedness of their complement.
%\item Constructing two efficient polynomial-time graph factorization algorithms.
%\item Proving that the maximum number of maximum cliques on $n \ge 15$ vertices is $M_n = 3^{\lfloor n/3 \rfloor}c$, where $c\in\{1,\frac{4}{3},2\}$, depending on $n \text{ mod } 3$.
%\item Concluding that the maximum number of maximum cliques and the maximum number of maximal cliques is exactly the same.
%\end{highlights}
% \asymp

% Keywords
% Each keyword is seperated by \sep
\begin{keywords}
Maximum number of maxcliques \sep Maximum cliques \sep Prime and composite graphs
\end{keywords}

\maketitle

\section{Introduction}
In this paper we will show that the maximum number of \textbf{maximum cliques} for simple graphs on $n$ vertices, $n \ge 15$ is:

$$M_n = \begin{cases}
3^{\lfloor n/3 \rfloor} \text{ if }n \text{ mod } 3 = 0 \\
4 \cdot 3^{(\lfloor n/3 \rfloor-1)} \text{ if }n \text{ mod } 3 = 1 \\
2 \cdot 3^{\lfloor n/3 \rfloor} \text{ if }n \text{ mod } 3 = 2
\end{cases}$$

In 1965, \cite{moon1965cliques} have already calculated the maximum number of \textbf{maximal cliques} (cliques that are non-extendable), and they also obtained the exact same amount. From their result, it follows that the maximum number of maximum cliques is less than or equal to this amount. However in this paper, we can also conclude that the maximum number of maximal cliques and the maximum number of maximum cliques for graphs with $n \ge 15$ vertices is exactly the same.

We will also show that there exist $O(n^2)$ polynomial-time factorization algorithms for $G$ graph into $G= G_1 \oplus \dots \oplus G_m$ graphs, where all pairs of vertices between any two factors $G_i$ and $G_j$ are connected (see Theorem \ref{factor_algo}). Moreover, strict edge bounds can be given for all graphs (see Theorem \ref{edge_bound}), using the size and number of maximum cliques of graphs in its factorization.

The structure of the paper is the following: In Section \ref{prelim}, we will introduce all the graph theoretical definitions and notions that we will be using throughout the paper. We will also introduce shorthand notations that will be used commonly, so we ask the reader, even if they are familiar with these notions, please look through Section \ref{prelim} anyway.

Then in Section \ref{directsum} we will define the notion of a direct sum of graphs, split graphs into prime and composite classes, and prove some properties of them. In Section \ref{edgebounds}, we will set up the most crucial theorem of the paper about edge bounds of composite graphs, from which all other Theorems will follow. This will be used to show that we indeed need to look for graphs with the maximum number of maximum cliques in the set of composite graphs. Finally, in Section \ref{maxmaxcliques} we will combine everything to calculate $M_n$.

\section{Preliminaries}\label{prelim}

\begin{definition}
Let $V$ be a set (vertices) and $E \subseteq V \times V$ (edges). Then $G=(V,E)$ is a \textbf{graph}. $G$ is a simple graph if $\forall e=(v,w) \in E: v \ne w$ and $(w,v) \notin E$. We will work with non-directed graphs, so for any edge $e=(v,w)$, let $(w,v)$ be the same edge: $(v,w):=(w,v)$.
\end{definition}

\begin{definition}
Let $G = (V,E)$ be a simple graph. $\forall v \in V$, \textbf{the degree of vertex} $v$, $d(v)$ is the number of edges joining $v$ to other vertices, i.e. $d(v) = |\{e=(v,\cdot)|e\in E\}|$.
\end{definition}

\begin{definition}
Let $G=(V,E)$ be a simple graph. $H=(V',E')$ is a \textbf{subgraph} of $G$ if $V' \subseteq V$ and $E' \subseteq E$.
\end{definition}

\begin{definition}
Let $G = (V,E)$ be a simple graph, and let $H=(V',E')$ be a subgraph of $G$. Then $\forall v \in V$, \textbf{the subgraph degree of vertex} $v$, $d_H(v)$ is the number of edges joining $v$ to other vertices in $H$, i.e. $d_H(v) = |\{e=(v,\cdot)|e\in E'\}|$.
\end{definition}

\begin{definition}
Let $G = (V,E)$ be a simple graph with $|V|=n$. If all pairs of vertices in $G$ are connected, then $G$ is a \textbf{complete graph}. The complete graph on $n$ vertices will be denoted as $G=K_n$.
\end{definition}

\begin{definition}
Let $G = (V,E)$ be a simple graph with $|V|=n$. If $E=\emptyset$, then $G$ is an \textbf{empty graph}. The empty graph on $n$ vertices will be denoted as $G=E_n$.
\end{definition}

\begin{definition}
Let $G = (V,E)$ be a simple graph. Any complete subgraph $C$ of $G$  is called a \textbf{clique}.
\end{definition}

\begin{definition}
Let $C$ be a clique in $G$. It is a \textbf{maximum clique} or a \textbf{maxclique}, if no other cliques in $G$ have more vertices than $C$, i.e. $C=(V,E)$ is a maxclique $\Longleftrightarrow$ $\forall C'=(V',E')$ cliques in $G$, $|V| \ge |V'|$.
\end{definition}

\begin{definition}
Let $G=(V,E)$ be a simple graph. Then $\omega(G):=$ the \textbf{size of} $G$\textbf{'s maxcliqes}, i.e. if $H=(V',E')$ is a maxclique in $G$, then $\omega(G)=|V'|$.
\end{definition}

\begin{definition}
Let $G=(V,E)$ be a simple graph. Then $\text{MC}(G):=$ \textbf{the set of} $G$\textbf{'s maxcliqes}, i.e. $$\text{MC}(G)=|\{H | H\text{ is a maxclique in G }\}|$$
\end{definition}

\begin{definition}
Let $G=(V,E)$ be a simple graph. The graph $\overline{G}:=(V,V\times V \setminus E)$ is called the \textbf{complement graph} of $G$, i.e., in $\overline{G}$, all edges of $G$ become non-edges, and all non-edges become edges.
\end{definition}

\begin{rmk}
${}$\\
\vspace{-5mm}
\begin{itemize}
\item For a simple graph $G=(V,E)$, we will denote the edges of $\overline{G}$ by $\overline{E}$.
\item For any $\overline{e}\in\overline{E}$, we will denote $G \cup \overline{e}:=(V,E\cup\overline{e})$.
\item For any $e \in E$, we will denote $G \setminus e := (V,E\setminus e)$.
\end{itemize}
\end{rmk}

\begin{definition}
Let $G=(V,E)$ be a simple graph. The sequence of edges $(e_1,\dots,e_k)\in E^k$ is \textbf{path}, if $\forall i\in\{1,\dots,k-1\}$ $e_i$ shares a common vertex with $e_{i+1}$, i.e. if $e_i = (a,b)$ and $e_{i+1} = (c,d)$, then $b=c$. The path \textbf{goes between} $v$ and $w$ if $e_1 = (v,\cdot)$ and $e_k = (\cdot,w)$.
\end{definition}

\begin{definition}
Let $G=(V,E)$ be a simple graph. $G$ is \textbf{connected} if $\forall v_1,v_2\in V, (v_1 \ne v_2)$, there exists a path that goes between $v_1$ and $v_2$.
\end{definition}

\begin{lemma}
If $G=(V,E)$ is a connected simple graph with $|V|=n$, then $|E| \ge n-1$. 
\end{lemma}

\begin{pf}
See \cite{west2001introduction}, Theorem 2.1.4.
\end{pf}

\begin{definition}
$G=(V,E)$ is a \textbf{cycle} if $V=\{v_1,\dots,v_n\}$ and $E=\{(v_1,v_2),(v_2,v_3),\dots,(v_{n-1},v_{n})\}$ with $v_1 = v_n$. The cycle on $|V|=n$ vertices will be denoted as $C_n$.
\end{definition}

\begin{definition}
Let $G=(V,E)$ be a simple graph. Let us partition the vertex set into $S_1,S_2\subset V$, meaning that  $S_1 \cup S_2 = V, S_1 \cap S_2 = \emptyset$. Then
$$E_G(S_1,S_2) := \{e=(v_1,v_2)\in E | v_1 \in S_1, v_2 \in S_2\}$$
i.e., $E_G(S_1,S_2)$ is the \textbf{set of edges that go between} $S_1$ and $S_2$ in $G$.
\end{definition}

\begin{definition}
Let $G=(V,E)$ be a simple graph. The \textbf{cutting number} of $G$, denoted as $C(G)$, is the minimal number of edges that go between any partition of $V=S_1 \cup S_2$, i.e.
$$C(G) = \min \{\ |E_G(S_1,S_2)| \ \ | \ \ S_1,S_2 \subset V, S_1 \cup S_2 = V, S_1 \cap S_2 = \emptyset\}$$
\end{definition}

\begin{definition}
Let $G=(V,E)$ be a simple graph. $G$ is \textbf{bipartite} if $\exists A,B \subset V, A \cup B = V, A \cap B = \emptyset$ partition of $V$ such that $A$ and $B$ are  empty graphs. A bipartite graph is \textbf{complete} if $\forall v_1 \in A, v_2 \in B: e=(v_1,v_2) \in E$. Complete bipartite graphs on $|A|=n, |B|=m$ vertices will be denoted as $K_{n,m}$.
\end{definition}

\begin{definition}
Let $G=(V,E)$ be a simple graph. $G$ \textbf{contains an empty bipartite graph}, $E_{n,m} \subseteq G$, if $\exists A,B \subset V, A \cup B = V, A \cap B = \emptyset, |A|=n, |B|=m, \forall v_1 \in A, v_2 \in B: \neg \exists e=(v_1,v_2)\in E$. In other words, $G$ is \textbf{disconnected along} $A$ and $B$.
\end{definition}

\begin{definition}
Let $G=(V,E)$ be a simple graph. Then $\#G$ will denote the \textbf{number of maxcliques} in $G$, i.e. $$\#G = |\text{MC}(G)|$$
\end{definition}

In this study, our goal is to calculate how many maxcliques there could possibly be in a graph on $n$ vertices, disregarding the size of these maxcliques. So we are interested in the following amount:

\begin{definition}
The \textbf{maximum number of maxcliques on simple graphs on} $n$ \textbf{vertices} will be denoted as
$$M_n := \max \{ \#G | G=(V,E),|V|=n \}$$
\end{definition}

\begin{definition}
The \textbf{set of graphs on} $n$ \textbf{vertices with the most number of maxcliques} will be denoted as:
$$\text{MG}_n:= \text{argmax} \{\# G | G=(V,E), |V|=n\}$$
\end{definition}

\section{Direct sum of graphs}\label{directsum}

\begin{definition}\label{direct_sum}
Given $G_1 = (V_1,E_1)$ and $G_2 = (V_2,E_2)$ simple graphs with $V_1 \cap V_2 = \emptyset$, $G_1 \oplus G_2 = (V,E)$ is the \textbf{direct sum} of $G_1$ and $G_2$, if it has vertices $V = V_1 \cup V_2$ and edges
$$E = \{e=(v_1,v_2)|v_1 \in V_1,v_2\in V_2\}$$
i.e. Place $G_1$ and $G_2$ next to one another, and connect all vertices of $G_1$ to all vertices of $G_2$ to obtain $G_1 \oplus G_2$.
\end{definition}

\begin{theorem}\label{cartesian}
Given $G_1 = (V_1,E_1)$ and $G_2 = (V_2,E_2)$ simple graphs with $V_1 \cap V_2 = \emptyset$:
$$\text{MC}(G_1 \oplus G_2) = \text{MC}(G_1) \times \text{MC}(G_2)$$
where $\times$ on the right side means the Cartesian product, with each element evaluated using the direct sum, i.e. if $A$ and $B$ are sets of graphs, then $A \times B = \{G_A \oplus G_B | G_A \in A, G_B \in B\}$.
\end{theorem}

\begin{pf}
We need to show what the maxcliques of $G_1 \oplus G_2$ look like. Since all vertices between $G_1$ and $G_2$ are connected, the maxcliques of $G_1 \oplus G_2$ are of form $M_1 \oplus M_2$, where $M_1 \in \text{MC}(G_1), M_2 \in \text{MC}(G_2)$. To see this, we need to show that $M_1 \oplus M_2$ is a clique, and that it is maximal in $G_1 \oplus G_2$.

\begin{itemize}
\item $M_1 \oplus M_2$ is a clique, because all vertices inside $M_1$ are connected (since $M_1$ is a clique in $G_1$), all vertices inside $M_2$ are connected (since $M_2$ is a clique in $G_2$), and all vertices between $M_1$ and $M_2$ are connected (by definition).
\item $M_1 \oplus M_2$ is maximal in $G_1 \oplus G_2$, because the only vertices we could add to $M_1 \oplus M_2$ are either in $G_1$ or $G_2$. However since $M_1$ is maximal in $G_1$, and $M_2$ is maximal in $G_2$, there are no other vertices we could add to $M_1 \oplus M_2$ to turn it into a larger clique.
\end{itemize}

Since all maxcliques of $G_1 \oplus G_2$ are the maxcliques of $G_1$ direct summed with the maxcliques of  $G_2$, $\text{MC}(G_1 \oplus G_2) = \text{MC}(G_1) \oplus \text{MC}(G_2)$. $\square$
\end{pf}

\begin{theorem}\label{maxcliquetimes}
Given $G_1 = (V_1,E_1)$ and $G_2 = (V_2,E_2)$ simple graphs with $V_1 \cap V_2 = \emptyset$
$$\# (G_1 \oplus G_2) = \#(G_1) \#(G_2)$$
\end{theorem}

\begin{pf}
Using Theorem \ref{cartesian},
$$\# (G_1 \oplus G_2) = |\text{MC}(G_1 \oplus G_2)| = |\text{MC}(G_1) \times \text{MC}(G_2)| = |\text{MC}(G_1)|\ |\text{MC}(G_2)| = \#(G_1) \#(G_2)\ \square$$
\end{pf}

\begin{samepage}
\begin{definition}
$G = (V,E)$ graph is \textbf{composite}, if $\exists G_1, G_2$ graphs such that $G = G_1 \oplus G_2$
\end{definition}

\begin{definition}
$G = (V,E)$ graph is \textbf{prime}, if it is not composite.
\end{definition}
\end{samepage}

\begin{rmk}
The decomposition of a $G$ composite graph into two composite graphs may not not unique. See Figure \ref{non_unique_factorization} for an example.
\end{rmk}

{\centering
% Figure removed
\captionof{figure}{Example of a non-unique decomposition of a graph into $2$ components}
\label{non_unique_factorization}}

\begin{rmk}
However the decomposition of $G$ into prime graphs is unique up to isomorphism. (Similarly to integers, e.g. $12 = 2 \cdot 6 = 3 \cdot 4 = 2 \cdot 2 \cdot 3$.) This will be proved in the following four Theorems.
\end{rmk}

{\centering
% Figure removed
\captionof{figure}{Decomposing the previous example further to obtain a unique factorization.}
\label{unique_factorization}}

\begin{theorem}
The following algorithm gives an efficient, $O(n^3)$ factorization of an input $G=(V,E)$ graph on $|V|=n$ vertices:
\begin{enumerate}
\item Let $S_1 = \{v\}$, where $v\in V$ is an arbitrary starting vertex.
\item Find a vertex $w \in V \setminus S_1$ that does not connect to any vertex in $S_1$.
\item If there exists one, add it to $S_1$, so let $S_1 := S_1 \cup \{w\}$, and go to $2$.
\item If there is none, then return $S_1, S_2 := V \setminus S_1$, $S_1 = V(G_1), S_2 = V(G_2), G = G_1 \oplus G_2$.
\item If we have added all vertices to $S_1$, so if $S_1 = V$, then $G$ is prime.
\end{enumerate}
\end{theorem}

\begin{pf}
We need to prove $3$ things:
\begin{itemize}
\item If the algorithm returned $S_1, S_2 := V \setminus G_2$, $S_1 = V(G_1), S_2 = V(G_2)$, then is it true that $G = G_1 \oplus G_2$?
\item If the algorithm returned that $S_1 = V$, then is $G$ truly prime?
\item Is the runtime of the algorithm $O(n^3)$?
\end{itemize}

If the algorithm returned $S_1, S_2 \subset V$, that could have only been if $\forall v_1 \in S_1, v_2 \in S_2: e=(v_1,v_2) \in E$, or in other words, all pairs of vertices between $S_1$ and $S_2$ are connected. By definition, in this case, $G = G_1 \oplus G_2$.

In the algorithm returned $S_1 = V$, then there is no bipartite partition of the edges into $S_1 \cup S_2 = V$. Since no matter what $v\in V$ the algorithm starts from, that $v$ is eventually going to be in either $S_1$ or $S_2$. When building up either $S_1$ or $S_2$, if we cannot find such a bipartite partition, there there is none in $E$. Therefore, $G$ cannot be split into $G_1 \oplus G_2$, meaning that $G$ has to be prime.

Finally, the runtime of the algorithm is $O(n^3)$, since every time a vertex is added to $S_1$, in the worst case, the algorithm has to check whether or not any of $v_1 \in S_1$ connect to any of $v_2 \in V \setminus S_1$, which is at most $O(n^2)$ comparisons. So $n$ times $O(n^2)$ comparisons gives $O(n^3)$ comparisons. $\square$
\end{pf}

\begin{theorem}\label{no_bipartite_subset}
If $G=(V,E)$ is a prime graph with $|V|=n$, then $G$ does not contain a $K_{b,n-b}$ subgraph for any $b \in \{1,\dots,n-1\}$.
\end{theorem}

\begin{pf}
Assume $\exists b \in \{1,\dots,n-1\} \exists K_{b,n-b}$ subgraph in $G$ between the partition $S_1, S_2 \subset V$. Then by Definition \ref{direct_sum}, $G = G_1 \oplus G_2$, where $G_1 \subset G$ the graph induced by vertices $V_1$, and $G_2 \subset G$ the graph induced by vertices $V_2$. Therefore $G$ is composite. However this is a contradiction, because $G$ was prime. So our original assumption was false, meaning that $G$ does not contain a $K_{b,n-b}$ subgraph for any $b \in \{1,\dots,n-1\}$.
\end{pf}

\begin{theorem}\label{prime_equiv}
Let $G$ be a graph.
$$G\text{ is prime }\Longleftrightarrow \overline{G}\text{ is connected}$$
\end{theorem}

\begin{pf}
$(\Longrightarrow)$

$G$ is prime $\Longrightarrow$ Due to Theorem \ref{no_bipartite_subset}, $G$ contains no $K_{b,n-b}$ subgraph for any $b \in \{1,\dots,n-1\}$ $\Longrightarrow$ $\overline{G}$ contains no $E_{b,n-b}$ subgraph for any $b \in \{1,\dots,n-1\}$ $\Longrightarrow$ $\overline{G}$ is connected.

$(\Longleftarrow)$

In this case, we will instead prove the equivalent statement: $G$ is composite $\Longrightarrow$ $\overline{G}$ is disconnected:

$G$ is composite $\Longrightarrow$ $G=G_1 \oplus G_2$ with $G_1=(V_1,E_1), G_2=(V_2,E_2)$ $\Longrightarrow$ $G$ contains a $K_{|V_1|,n-|V_1|}$ along the partition $(S_1,S_2)$. $\Longrightarrow$ $\overline{G}$ contains an $E_{|V_1|,n-|V_1|}$ along the partition $(S_1,S_2)$ $\Longrightarrow$ $\overline{G}$ is disconnected. $\square$
\end{pf}

\begin{theorem}\label{factor_algo}
The following algorithm gives an efficient, $O(n^2)$ factorization of an input $G=(V,E)$ graph on $|V|=n$ vertices:

\begin{enumerate}
\item Calculate $\overline{G}$.
\item Starting from an arbitrary vertex $v \in V$, use breadth-first search on $\overline{G}$ to determine if it is connected.
\item If it connected, then $G$ is prime. If it is disconnected into subgraphs $\overline{G} = \overline{G_1} \cup \overline{G_2} \cup \dots \cup \overline{G_m}$, then $G = G_1 \oplus G_2 \oplus \dots \oplus G_m$. 
\end{enumerate}
\end{theorem}

\begin{pf}
Using Theorem \ref{prime_equiv}, the output of the  algorithm is clearly correct.

It is also $O(n^2)$, because calculating $\overline{G}$ takes $\frac{n(n-1)}{2}=O(n^2)$ steps, and walking through the edges of $\overline{G}$ takes at most $|\overline{E}| \le \frac{n(n-1)}{2}=O(n^2)$ steps, which is overall an $O(n^2)$ amount.
\end{pf}

\section{Edge bound}\label{edgebounds}

\begin{theorem}\label{edge_lower}
$G=(V,E), |V|=n$ is a graph with $G=G_1 \oplus G_2 \oplus \dots \oplus G_m$ and  $\forall i \in \{1,\dots,m\}:  G_i := (V_i,E_i), |V_i|=n_i$. Then
$$|E| \ge \sum_{1\le i < j \le m} n_i n_j + \frac{1}{2}\sum_{i=1}^m \omega(G_i) (\omega(G_i)-1) + \frac{1}{2} \sum_{i=1}^m (\#G_i-1)(\omega(G_i)-1)$$
\end{theorem}

\begin{pf}
Using Theorem \ref{cartesian}, $\forall v_1 \in V_i, v_2 \in V_j: e=(v_1,v_2) \in E$. That means that there are a total of $\sum_{1\le i < j \le m} n_i n_j$
edges inbetween $G_1, G_2, \dots, G_m$, so

$$|E| \ge \sum_{1\le i < j \le m} n_i n_j$$

Furthermore, any simple graph $G_i$ with maxcliques of size $m$ contains at least one maxclique of size $\omega(G)=m$, therefore contains at least as many edges as $K_m$, which is $\frac{m(m-1)}{2} = \frac{\omega(G)(\omega(G)-1)}{2}$. These edges are all inside each $G_i$, therefore they are different from the previously listed edges, so
$$|E| \ge \sum_{1\le i < j \le m} n_i n_j + \frac{1}{2}\sum_{i=1}^m \omega(G_i) (\omega(G_i)-1)$$

However if each $G_i$ contains more than $1$ maxclique, we can further improve this bound. For each subsequent maxclique of $G_i$, $G_i$ contains at least $1$ extra vertex $v$, because each pair of maxcliques have to be different. Since $v$ is part of a maxclique of $G_i$, its degree $d_{G_i}(v) \ge \omega(G_i)-1$. Therefore, for each $G_i$, there are an additional $(\#G_i-1)(\omega(G_i)-1)$ vertex degrees that are still unaccounted for. Since the sum of the vertex degrees are the half of the number of edges (see \cite{west2001introduction}, Prop. 1.4.19), there are an additional $\frac{1}{2}(\#G_i-1)(\omega(G_i)-1)$ edges that are still unaccounted for, so
$$|E| \ge \sum_{1\le i < j \le m} n_i n_j + \frac{1}{2}\sum_{i=1}^m \omega(G_i) (\omega(G_i)-1) + \frac{1}{2} \sum_{i=1}^m (\#G_i-1)(\omega(G_i)-1) \quad \square$$
\end{pf}

\begin{theorem}\label{edge_upper}
$G=(V,E), |V|=n$ is a graph with $G=G_1 \oplus G_2 \oplus \dots \oplus G_m$ such that $\forall i \in \{1,\dots,m\}$ $G_i=(V_i,E_i), |V_i|=n$, and $G_i$ is prime. Then
$$|\overline{E}| \ge \sum_{i=1}^m (n_i-1)$$
\end{theorem}

\begin{pf}
Using Theorem \ref{prime_equiv}, each $\overline{G}_i, i \in \{1,\dots,m\}$ are connected, therefore each one has at least $n_i-1$ edges. So the total number of complement edges are
$$|\overline{E}| \ge \sum_{i=1}^m (n_i-1) \quad \square$$
\end{pf}

\begin{theorem}\label{edge_bound}
$G=(V,E), |V|=n$ is a graph with $G=G_1 \oplus G_2 \oplus \dots \oplus G_m$ such that $\forall i \in \{1,\dots,m\}$ $G_i=(V_i,E_i), |V_i|=n$, and $G_i$ is prime. Then
$$\sum_{1\le i < j \le n} n_i n_j + \frac{1}{2}\sum_{i=1}^m \omega(G_i) (\omega(G_i)-1) + \frac{1}{2}\sum_{i=1}^m (\#G_i-1)(\omega(G_i)-1) \le |E| \le \frac{n(n-1)}{2} - \sum_{i=1}^m (n_i-1)$$
\end{theorem}

\begin{pf}
The lower bound on $|E|$ is the exact same as in Theorem \ref{edge_lower}. The upper bound is obtained from Theorem \ref{edge_upper}, as we are missing at least as many edges from $K_n$ as there are in $|\overline{E}|$, which is at least $\sum_{i=1}^m (n_i-1)$ edges. So $|E| \le \frac{n(n-1)}{2} - \sum_{i=1}^m (n_i-1)$. $\square$ 
\end{pf}

\begin{rmk}
Theorem \ref{edge_bound} gives a strict upper and lower bound for the number of edges in a composite graph. As an example, consider a composite graph $G = (V,E)$ with the decomposition $G = G_1 \oplus G_2 \oplus G_3$, where $G_1 = (V_1,E_1),G_2 = (V_2,E_2),G_3 = (V_3,E_3)$, all of $G_1,G_2,G_3$ are prime, and
\begin{itemize}
\item $|V_1|=5, |V_2|=6, |V_3|=7$
\item $\omega(G_1)=2,\omega(G_2)=3,\omega(G_3)=3$
\item $\#G_1 = 5, \#G_2 = 4, \#G_3 = 7$
\end{itemize}
Then the edge bound obtained from Theorem \ref{edge_bound} is
$$125 \le |E| \le 138$$
And example for such a graph can be seen on Figure \ref{edge_bound_example}:
\vspace{2mm}

{\centering
% Figure removed
\captionof{figure}{An example for a graph with the specifications above. In reality, this specific graph has $|E|=133$ edges.}
\label{edge_bound_example}
\vspace{3mm}}

\end{rmk}

\begin{theorem}\label{primecliques}
If $G=(V,E), |V|=n$ is a prime graph, then
$$\#G \le n^2$$
\end{theorem}

\begin{pf}
Using Theorem \ref{edge_bound}, then the transitive property of inequalities, then subtracting some terms from the left side:
\begin{align*}
\sum_{1\le i < j \le n} n_i n_j + \frac{1}{2}\sum_{i=1}^m \omega(G_i) (\omega(G_i)-1) + \frac{1}{2}\sum_{i=1}^m (\#G_i-1)(\omega(G_i)-1) &\le |E| \le \frac{n(n-1)}{2} - \sum_{i=1}^m (n_i-1) \\
\sum_{1\le i < j \le n} n_i n_j + \frac{1}{2}\sum_{i=1}^m \omega(G_i) (\omega(G_i)-1) + \frac{1}{2}\sum_{i=1}^m (\#G_i-1)(\omega(G_i)-1) &\le \frac{n(n-1)}{2} - \sum_{i=1}^m (n_i-1) \\
\frac{1}{2}\sum_{i=1}^m (\#G_i-1)(\omega(G_i)-1) \le \frac{n(n-1)}{2} - \sum_{i=1}^m (n_i-1) - \sum_{1\le i < j \le n} n_i n_j &- \frac{1}{2}\sum_{i=1}^m \omega(G_i) (\omega(G_i)-1)
\end{align*}
Each subtracted term on the right side is $\ge 0$, so we can increase the right side by omitting them:
\begin{align*}
\frac{1}{2}\sum_{i=1}^m (\#G_i-1)(\omega(G_i)-1) &\le \frac{n(n-1)}{2}
\end{align*}
Since $G$ is prime, its prime factors are itself, so $m=1$:
\begin{align*}
\frac{1}{2}(\#G-1)(\omega(G)-1) &\le \frac{n(n-1)}{2}
\end{align*}
Multiplying by $2$ and dividing by $(\omega(G)-1)$ gives:
\begin{align*}
\#G-1 &\le \frac{n(n-1)}{\omega(G)-1} \\
\#G &\le \frac{n(n-1)}{\omega(G)-1}+1
\end{align*}
There are two cases:
\begin{enumerate}
\item $\omega(G)=1$
\item $\omega(G) \ge 2$
\end{enumerate}
In the first case, $G$ is the empty graph, so it has exactly $n$ maxcliques, which is $\le n^2$, since $n\ge 1$.
In the second case, $\omega(G)-1\ge 1$, so by omitting it, we further increase the right side:
\begin{align*}
\#G &\le n(n-1)+1
\end{align*}
Finally, since $n\ge 1$, $\#G \le n(n-1)+1 \le n^2$. So in all cases we have obtained that $\#G \le n^2$. $\square$
\end{pf}

\section{The maximum number of maxcliques}\label{maxmaxcliques}

\begin{definition}
Let $G$ be a simple graph and $m\in\mathbb{N}^+$. Then
$$G^m := \underbrace{G \oplus \dots \oplus G}_{m} $$
\end{definition}

\begin{theorem}\label{construction}
$\exists G=(V,E), |V|=n$ such that $\#G \ge 3^{\lfloor n/3 \rfloor}$
\end{theorem}

\begin{pf}
Let $G_0=(E_3)^{\lfloor n/3 \rfloor}$. $G_0$ has $3 \left\lfloor \frac{n}{3} \right\rfloor$ vertices, which is equal to $n$ if $n$ is divisible by $3$. Otherwise, add $1$ or $2$ more disconnected vertices to $G_0$ to obtain $G$.

Then using Theorem \ref{maxcliquetimes}:
$$\# G \ge \# G_0 = \#(E_3^{\lfloor n/3 \rfloor}) = (\#(E_3))^{\lfloor n/3 \rfloor} = 3^{\lfloor n/3 \rfloor}$$

So there exists a graph with this many maxcliques. $\square$
\end{pf}

\begin{theorem}
If $n \ge 15$, then
$$M_n = \max \{\# G | G=(V,E), |V|=n, G\text{ is composite}\}$$
\end{theorem}

\begin{pf}
The inequality $n^2 < 3^{\lfloor n/3 \rfloor}$ over the whole numbers is satisfied when $n \ge 15$.

Let $G=(V,E)$ be a prime graph with $|V|=n$. Using Theorem \ref{construction}, if $n \ge 15$, then:
$$\#G \le n^2 < 3^{\lfloor n/3 \rfloor} \le M_n$$

So $G$ cannot be in $MG_n$ in the case when $n \ge 15$. $\square$
\end{pf}

\begin{theorem}\label{large_triangleless}
$\exists G=(V,E)$ triangle-less graph on $|V|=n$ vertices with
$$\#G \ge \frac{(n-1)(n+1)}{4}$$
\end{theorem}

\begin{pf}
If $G$ is triangle-less, then $\#G = |E|$.

If $n$ is even, then let $G := K_{n/2,n/2}$. This graph has $\frac{n^2}{4} > \frac{(n-1)(n+1)}{4}$ edges. It is triangle-less, so it also has at least this many maxcliques.

If $n$ is odd, then let $G := K_{\left\lfloor n/2 \right\rfloor, \left\lceil n/2 \right\rceil}$. This graph has $\left\lfloor \frac{n}{2} \right\rfloor \left\lceil \frac{n}{2} \right\rceil = \frac{(n-1)(n+1)}{4}$ edges. It is also triangle-less, so it also has at least this many maxcliques. $\square$
\end{pf}

\begin{theorem}\label{lessthan5}
Let $G=(V,E)$ be a composite graph $G=G_1 \oplus \dots \oplus G_m$ with $\omega(G_i) \ge 5$ for some $i \in \{1,\dots,m\}$. Then $G \notin \text{MG}_n$.
\end{theorem}

\begin{pf}
$\forall k \in \{1,\dots,m\}$, let us denote $G_k=(V_k,E_k)$, $|V_k|=n_k$, and the graph with maxcliques of size at least $\omega(G_i) \ge 5$ by $G_i$.

Let us use Theorem \ref{edge_bound} for $G_i$, so with $m=1$. Just like in the Proof of Theorem \ref{primecliques}, we can obtain a bound:

\begin{align*}
\frac{1}{2} \omega(G_i)(\omega(G_i)-1) + \frac{1}{2}(\#G_i-1)(\omega(G_i)-1) &\le |E| \le \frac{n_i(n_i-1)}{2}-(n_i-1) = \frac{(n_i-1)(n_i-2)}{2} \\
(\omega(G_i)-1)(\omega(G_i)+\#G_i-1) &\le (n_i-1)(n_i-2) \\
\omega(G_i)+\#G_i-1 &\le \frac{(n_i-1)(n_i-2)}{\omega(G_i)-1} \\
\#G_i &\le \frac{(n_i-1)(n_i-2)}{\omega(G_i)-1}-(\omega(G_i)-1) \\
\end{align*}

Since $\omega(G_i) \ge 5$, the right side can be estimated as
$$\#G_i \le \frac{(n_i-1)(n_i-2)}{\omega(G_i)-1}-(\omega(G_i)-1) \le \frac{(n_i-1)(n_i-2)}{4}-4 < \frac{(n_i-1)(n_i+1)}{4} \le \#\hat{G}$$

where $\hat{G}$ is a graph on $n_i$ vertices with at least $\frac{(n_i-1)(n_i+1)}{4}$ maxcliques, which was shown to exist in Theorem \ref{large_triangleless}.

Putting things together, using Theorem \ref{maxcliquetimes}:
\vspace{-5mm}

$$\#G = \#G_1 \oplus \dots \oplus \#G_{i-1} \oplus \#G_i \oplus \#G_{i+1} \oplus \dots \oplus \#G_m < \#G_1 \oplus \dots \oplus \#G_{i-1} \oplus \#\hat{G} \oplus \#G_{i+1} \oplus \dots \oplus \#G_m := \#G^{+}$$

So the graph $G^+$, defined by swapping $G_i$ into $\hat{G}$ in the direct sum, has the same number of vertices as $G$, but more maxcliques than $G$, therefore $G \notin \text{MG}_n$. $\square$
\end{pf}

\begin{theorem}
All graphs $G=(V,E)=G_1 \oplus \dots \oplus G_m$ in $\text{MG}_n$ on $|V|=n \ge 15$ vertices are made out of the following prime components:
$$\forall i \in \{1,\dots,m\}: G_i \in \{E_1,E_2,E_3,E_4,C_4\}$$
\end{theorem}

\begin{pf}
Let $G_i := (V_i,E_i), |V_i| = n_i$. Let us list all simple graphs on at most $4$ vertices, and see which ones have the highest amount of maxcliques:

\begin{itemize}
\item On $n_i=1$ vertex, there is only one graph, $E_1$ with $\#E_1=1$. Therefore, $E_1$ is optimal on $n_i=1$ vertex.
\item On $n_i=2$ vertices, there are two graphs: $E_2$ and $K_2$ with $\#E_2=2, \#K_2=1$. Therefore, $E_2$ is optimal on $n_i=2$ vertices.
\item On $n_i=3$ vertices, there are $4$ graphs: $E_3$, $E_3 \cup e$, $K_3 \setminus e$, and $K_3$, with $\#E_3=3,\ \#(E_3 \cup e)=1,\ \#(K_3 \setminus e)=2,\ \#K_3=1$. Therefore, $E_3$ is optimal on $n_i=3$ vertices.
\item On $n_i=4$ vertices, there are $11$ graphs (see \cite{west2001introduction}, Example 1.1.31). Figure \ref{all4s} shows all such graphs and their maxclique amounts. We can see that $E_4$ and $C_4$ have the most maxcliques, $4$.
\end{itemize}

{\centering
% Figure removed
\captionof{figure}{All graphs on $4$ vertices, and their maxclique-amounts}
\label{all4s}
\vspace{3mm}}

According to Theorem \ref{maxcliquetimes}, $\#G = \#G_1 \oplus \dots \oplus \#G_m$, and according to Theorem \ref{lessthan5}, all graphs in a prime decomposition must have $\omega(G_i) \le 4$, so $\forall i \in \{1,\dots,m\}: G_i \in \{E_1,E_2,E_3,E_4,C_4\}$, otherwise $G$ would have fewer than maximal number of maxcliques. $\square$
\end{pf}

\begin{rmk}
Note that any two $E_2$'s, any one $E_4$ or any one $C_4$ can be freely swapped to one another in a prime decomposition of $G$, and $G$ will still retain the same number of maxcliques, because in the direct sum $\#G = \#G_1 \oplus \dots \oplus \#G_m$, a factor of $4$ is being replaced by a different factor of $4$ with any of these swaps.
\end{rmk}

\begin{theorem}\label{final}
If $n \ge 15$, then
$$M_n = \begin{cases}
3^{\lfloor n/3 \rfloor} \text{ if }n \text{ mod } 3 = 0 \\
4 \cdot 3^{(\lfloor n/3 \rfloor-1)} \text{ if }n \text{ mod } 3 = 1 \\
2 \cdot 3^{\lfloor n/3 \rfloor} \text{ if }n \text{ mod } 3 = 2
\end{cases}$$
\end{theorem}

\begin{pf}
With the previous remark we can see that if there is an optimal graph $G_1 \in \text{MG}_n$ that has an $E_4$ or a $C_4$ component, then there exists at least one more $G_2 \in \text{MG}_n$ with the $E_4$'s replaced by two $E_2$'s and the $C_4$'s replaced by two $E_2$'s as well, so it is enough to search for the optimal graph with components $E_1$, $E_2$, or $E_3$.

Similarly, two $E_1$'s can be replaced by an $E_2$ to obtain more maxcliques in $G$, as two $E_1$'s would multiply the total number of maxcliques by $1\cdot 1=1$, while one $E_2$ would multiply it by $2$. So there can only be at most $1$ $E_1$ in a $G \in \text{MG}_n$.

And an optimal graph $G \in \text{MG}_n$ cannot both have an $E_1$ component and an $E_2$ component in it, because those could be replaced by an $E_3$ component to obtain $3$ maxcliques in $G$ instead of $2\cdot 1=2$. So any optimal graph may only have at most one $E_1$ and $E_3$'s in it, or some number of $E_2$'s and some number of $E_3$'s.

Finally, if the optimal graph contains an $E_1$ and an $E_3$, then they can be replaced by two $E_2$'s to multiply the number of maxcliques in $G$ by $4$ instead of $3$. Therefore, in all cases, $E_1$ cannot be in an optimal graph. So for all $n \ge 15$, there exists a graph with the highest amount of maxcliques that is made out of only $E_2$ and $E_3$ components.

Let $G=(V,E) \in \text{MG}_n$ with $|V|=n$ and $G=G_1 \oplus \dots \oplus G_m$, and let us assume there are $t \le \left\lfloor \frac{n}{3} \right\rfloor$ number of $E_3$'s in its decomposition. Then there must be $\frac{n-3t}{2}$ $E_2$'s in its decomposition. So
$$G= (E_3)^t \oplus (E_2)^{\frac{n-3t}{2}}$$

Then the number of maxcliques in $G$ is
$$f(t):=\#G=3^t \cdot 2^{\frac{n-3t}{2}}$$

With some algebraic manipulation, we can rewrite $f(t)$ as
$$f(t)=3^t \cdot 2^{\frac{n-3t}{2}} = \left(2^{\log_2(3)}\right)^t 2^{\frac{n-3t}{2}} = 2^{t \log_2(3)+\frac{n-3t}{2}} = 2^{t\left(\log_2(3)-\frac{3}{2}\right)+\frac{n}{2}} \approx 2^{n/2} 2^{0.085t}$$
Which is an increasing function in $t$, meaning that its maximum is taken on when $t$ is maximal, namely when $t=\left\lfloor \frac{n}{3} \right\rfloor$.

Which means that if $n$ is divisible by $3$, then an optimal graph is
$$G = (E_3)^{n/3}$$
which has $\#G = 3^{n/3}$ maxcliques.

Otherwise, one $E_2$ can be appended or one $E_3$ and one extra vertex can be replaced by two $E_2$'s.

Meaning that if $n = 3k + 1, k \in \mathbb{N}$, then an optimal graph is
$$G = (E_3)^{(\lfloor n/3 \rfloor-1)} \oplus (E_2)^2$$
which has $\#G = 3^{(\lfloor n/3 \rfloor-1)} \cdot 4$ maxcliques.

And if $n = 3k+2, k \in \mathbb{N}$, then an optimal graph is
$$G = (E_3)^{\lfloor n/3 \rfloor} \oplus E_2$$
which has $\#G = 3^{\lfloor n/3 \rfloor} \cdot 2$ maxcliques.

Overall, these three formulas can be expressed as
$$M_n = \begin{cases}
3^{\lfloor n/3 \rfloor} \text{ if }n \text{ mod } 3 = 0 \\
4 \cdot 3^{(\lfloor n/3 \rfloor-1)} \text{ if }n \text{ mod } 3 = 1 \\
2 \cdot 3^{\lfloor n/3 \rfloor} \text{ if }n \text{ mod } 3 = 2
\end{cases} \quad \square$$
\end{pf}

\noindent {\large\bfseries Acknowledgements:}
\vspace{3mm}

The research reported in this paper is part of project no. BME-NVA-02, implemented with the support provided by the Ministry of Innovation and Technology of Hungary from the National Research, Development and Innovation Fund, financed under the TKP2021 funding scheme.

%\printcredits

%% Loading bibliography style file
%\bibliographystyle{model1-num-names}
\bibliographystyle{cas-model2-names}

% Loading bibliography database
%\bibliography{Pfeifer_Daniel_Maximum_number_of_maximum_cliques}

\begin{thebibliography}{2}
\expandafter\ifx\csname natexlab\endcsname\relax\def\natexlab#1{#1}\fi
\providecommand{\url}[1]{\texttt{#1}}
\providecommand{\href}[2]{#2}
\providecommand{\path}[1]{#1}
\providecommand{\DOIprefix}{doi:}
\providecommand{\ArXivprefix}{arXiv:}
\providecommand{\URLprefix}{URL: }
\providecommand{\Pubmedprefix}{pmid:}
\providecommand{\doi}[1]{\href{http://dx.doi.org/#1}{\path{#1}}}
\providecommand{\Pubmed}[1]{\href{pmid:#1}{\path{#1}}}
\providecommand{\bibinfo}[2]{#2}
\ifx\xfnm\relax \def\xfnm[#1]{\unskip,\space#1}\fi
%Type = Article
\bibitem[{Moon and Moser(1965)}]{moon1965cliques}
\bibinfo{author}{Moon, J.W.}, \bibinfo{author}{Moser, L.},
  \bibinfo{year}{1965}.
\newblock \bibinfo{title}{On cliques in graphs}.
\newblock \bibinfo{journal}{Israel journal of Mathematics} \bibinfo{volume}{3},
  \bibinfo{pages}{23--28}.
%Type = Book
\bibitem[{West et~al.(2001)}]{west2001introduction}
\bibinfo{author}{West, D.B.}, et~al., \bibinfo{year}{2001}.
\newblock \bibinfo{title}{Introduction to graph theory}.
  volume~\bibinfo{volume}{2}.
\newblock \bibinfo{publisher}{Prentice hall Upper Saddle River}.
\end{thebibliography}

\end{document}

