\chapter[Summary and outlook]{Summary and outlook}
\label{chapter9}
%\section{Chapter-5}
%PRC
The study of the critical point in the QCD phase diagram through collision of heavy ions at relativistic energies is highly contemporary. The location of the CEP is still not clear. Even if the first-principle lattice calculations are not applicable to the finite $\mu_{B}$ region, where the CEP is supposed to exist, there are several studies based on effective models to find the location (or at least the region of its existence) of the CEP. In this dissertation, we assume the existence of the CEP at some location ($T_{c},\,\mu_{c}$) and study its consequences on the dispersion relation, on the propagation of perturbation (both linear and nonlinear) and on the structure factor. The effects of the CEP goes into the calculation through an EoS (of 3D Ising universality class) and scaling behaviour of several thermodynamic coefficients. The EoS used here is found to agree with the available lattice results. %The evolution of the QGP is studied through the relativistic viscous hydrodynamics (MIS theory).



To observe the hydrodynamic response of the system, a space-time-dependent perturbation is placed into the medium. The response of the system when passing through the CEP (nearby to the CEP) or away from the CEP have been investigated to find the difference in the response of the fluid. The dispersion relations are calculated with the ambit of the linearized hydrodynamic equations. The real part and the imaginary part of the dispersion relations are separated out. The imaginary part of the solution provides us the information about the dissipation of the plane wave in the medium, whereas the real part of the solution is for the survival of the wave. Therefore, the interplay between the magnitude of the real and the imaginary part will decide the fate of the perturbation (sound wave). This motivates us to calculate a threshold wave number $k_{th}$, and consequently, a threshold wavelength for the wave, symbolized as $\lambda_{th}$. It is defined in such a way such that any wave with a wavelength smaller than $\lambda_{th}$ will eventually get dissipated in the medium for given values of transport coefficients and other thermodynamic quantities. Interestingly, near the CEP, we have found $\lambda_{th}$ to diverge implying that no wave is allowed to propagate if the system hits the CEP {\it i.e.} waves with all wavelength get dissipated at the CEP irrespective to the values of transport coefficients. The fluidity defined in Eq.\eqref{eq0530} of the system seem to diverge at the CEP.  



In conventional condensed matter system, it has been observed experimentally ~\cite{Schneider1951,Botch1965,Kadanoff1968,Kawasaki1970,Kemensky1973,Dengler1987} that the absorption of sound is maximum due to diffraction of sound wave from the critical region, which is similar to the scattering of light at the critical point resulting in the critical opalescence.  Near the CEP, the correlation length $(\xi)$ diverges, therefore, the hydrodynamic limit ($\xi<<\lambda_{s}$, where, $\lambda_{s}$ is the wavelength of the sound mode) is violated. As a result, the development of sound wave is prevented. The forbiddance of the sound wave will eventually lead to the vanishing of Mach cone  (Mach angle, $\alpha=${\it{sin}}$^{-1}(c_s/v)$, where, $c_{s}$ is the speed of sound and $v$ is the fluid velocity). Therefore, the vanishing of Mach cone may indicate the presence of critical point.

The presence of CEP makes the viscous horizon  scale, $R_v\sim 1/k_{th}$ to diverge. This scale is related to various flow harmonics of the azimuthal distribution of produced particles in relativistic heavy-ion collider experiments (RHIC-E), which are useful quantities to characterize the matter of its initial state of the formation. For example, the triangular flow helps to understand the initial fluctuations and elliptic flow can be used to comprehend  the EoS of the system. Since the highest order of surviving flow harmonic is defined as, $n_v\sim 2\pi R/R_v$, thus ideally the vanishing harmonics will indicate the presence of the CEP. However, in experiments, the measurable quantities are superpositions of different densities and temperature from the formation to the freeze-out stage, therefore, even if the system hits the critical point in the $T-\mu$ plane, the harmonics may not vanish, but the critical point may suppress them.
%%%%%%%
%%%%%%%
%EPJA

We also derived the expression for the dynamical spectral structure of the density fluctuation to study its behaviour near the QCD critical point using linear response theory and the MIS hydrodynamics, where we have kept all the relevant transport coefficients in the calculations. The change in the spectral structure of the system as it approaches the critical point has been studied. We have found that the Rayleigh and Brillouin peaks are distinctly visible when the system is away from the critical point but the peaks tend to merge near the critical point. The sensitivity of the structure of the spectral function on wave vector $(k)$ of the sound wave has been demonstrated. It has been shown that the Brillouin peaks get merged with the Rayleigh peak because of the absorption of sound waves in the vicinity of the critical point. As consequences, we have argued that Mach cone formation will be prevented and the flow harmonics are suppressed when the system passes near the CEP. We have also the shown mode dependent propagation of the perturbations which helps to find the speed of sound by looking at the position of the Brillouin peaks.



%The correlation of density fluctuation near the QCD critical point has been studied by using linearized perturbative equations obtained from IS hydrodynamics. The spectral structure, $\Snn$ of the density fluctuation has been derived rigorously by keeping all the relevant transport coefficients and response functions non-zero. The effects of the EoS and the critical behaviour of various transport coefficients on $\Snn$ have been investigated. While interpreting the suppression of Mach cone structure as a signature of the CEP, the mode dependence of  propagation speed should be taken into consideration. These findings suggest that beam energy dependence of  the correlation of number density fluctuation of produced particles in RHIC-E has the potential to carry the signature of the critical point.
%%%%%%
%%%%%
%%%%%%
%%PLB
%In summary, we have investigated the response of the QCD critical point to the nonlinear perturbations within the scope of second order IS hydrodynamics. The effects of CEP on the propagation of nonlinear waves have been taken into account through the EoS, critical behaviour of the transport coefficients and thermodynamic response functions.  We have derived relevant equations governing the propagation of nonlinear wave within the purview of second order causal hydrodynamics by taking into account the non-zero values of $\eta$, $\zeta$ and $\kappa$ in contrast to earlier works with nonlinear wave equations, where the effects of $\zeta$ and $\kappa$ were ignored. In the presence of CEP $\zeta$ and $\kappa$ play important roles as they diverge near the CEP and hence can not be ignored. It is found that similar to the linear perturbation, the nonlinear perturbations too get suppressed near the CEP. The diverging nature of thermal conductivity near the CEP plays dominant role for the suppression of the nonlinear wave. The presence of nonlinear effects makes the perturbation travel a bit faster than the linear one. One expects the vanishing of Mach cone effects (or away side double-peak structure) and broadening of the two and three particle correlation as a consequence of the presence of the CEP. The suppression or collapse of elliptic flow will also indicated the existence of CEP. This may lead to the large event-by-event fluctuation of flow harmonics. Therefore, vanishing Mach cone effect (or away side double-peak structure) on away side jet and the enhancement of fluctuation of flow harmonics in event-by-event analysis accompanied by suppressed flow harmonics could be considered as signals of the CEP.


Critical slowing down is an important phenomena, cause the system to behave differently, should be included in the study of critical dynamics. The effects of critical slowing down is put into the calculations by considering a scalar non-hydrodynamic mode in the system. Therefore, we have derived the relevant equation of motions for the slow modes when 
the extensivity condition of thermodynamics is unaltered. The role of extra out-of-equilibrium $\phi$ mode on the dynamic structure factor near the QCD critical point has been investigated. The $\Snn$ in presence of the $\phi$ modes shows four peaks of Lorentzian type, which are asymmetrically situated in frequency axis with uneven magnitudes. While the $\Snn$ without $\phi$ modes admits three Lorentzian peaks, symmetrically situated on frequency axis with even magnitudes. We find that the asymmetry in the peaks originates due to  coupling of the $\phi$ modes with the hydrodynamic modes. The presence of the out-of-equilibrium modes can generate the extra peak in $\Snn$ within the scope causal theory of hydrodynamics when the transverse modes (along with the longitudinal modes) are taken into account. This study may help in the experimental investigation of role of the out-of-equilibrium modes near any of the $\mathcal{O}(4)$ critical points. The out-of-equilibrium mode plays a crucial role in reducing the width of the Lorentzians representing the thermal fluctuation which indicate the enhancement of the relaxation time of the system. 


Along with the linear perturbations, we have also studied the nonlinear perturbations in the fluid. We have derived the equations for the propagation of nonlinear perturbations by the Reductive Perturbative method in MIS theory with considering all the relevant transport coefficients. We have derived relevant equations governing the propagation of nonlinear wave within the purview of second order causal hydrodynamics by taking into account the non-zero values of $\eta$, $\zeta$ and $\kappa$ in contrast to earlier works with nonlinear wave equations, where the effects of $\zeta$ and $\kappa$ were ignored~\cite{Fogaca1,Fogaca:2014gwa}. In the presence of the CEP, $\zeta$ and $\kappa$ play significant roles as they tend to diverge near the CEP and hence can not be ignored. We observe, similar to the linear perturbation, the nonlinear perturbations too get suppressed near the CEP. The diverging nature of $\kappa$ near the CEP plays dominant role in the suppression of the nonlinear wave. One expects the vanishing of Mach cone effects (or away side double-peak structure) and broadening of the two and three particle correlation as a consequence of the presence of the CEP. The suppression or collapse of elliptic flow will also indicated the existence of CEP. This may lead to the large event-by-event fluctuation of flow harmonics. Therefore, vanishing Mach cone effect (or away side double-peak structure) on away side jet and the enhancement of fluctuation of flow harmonics in event-by-event analysis accompanied by suppressed flow harmonics could be considered as signals of the CEP.


%%%%%
%%%%%
%%%%%
%General
The possibility of existence and detection of the CEP has been studied in~\cite{KapustaChi}.  The mode coupling theory has been used to calculate the thermal conductivity at the location near and away from the QCD critical point in the $T-\mu$ plane and shown that the $\kappa$ diverges at the critical point.  Authors in Ref.~\cite{KapustaChi} have also demonstrated that the sharp change in $\kappa$ at the critical point is eventually reflected in the two particle-correlation of fluctuations in rapidity space. Therefore, the study pave the way to confirm the existence of the CEP. The suppression of fluctuations in baryonic chemical potential ($\Delta \mu$) and temperature ($\Delta T$) due to the enhanced thermodynamic response functions at the CEP can signal the presence of the CEP~\cite{Stephanov:1998dy,Stephanov:1999zu}. The  suppression in flow harmonics and suppression in fluctuations in $\Delta \mu$ and $\Delta T$ will be reflected through the spectra of hadrons and proton to pion ratio respectively.


In a real scenario, the possibility of the trajectories passing through the CEP may be infrequent, which restricts the magnitude of the fluctuations near the critical point. These fluctuations will stay out of equilibrium because of the expansion of the system and critical slowing down~\cite{Berdnikov:1999ph,Stephanov:2017ghc}. This problem has been analysed~\cite{Akamatsu2019} in the evolution of hydrodynamic fluctuations in the system formed in RHIC-E. The  emergence of the Kibble-Zurek length scale~\cite{Kibble,Zurek} and its relation with short range spatial correlations has been also discussed~\cite{Akamatsu2019}. To note that the non-flow correlations get enhanced in presence of the CEP and such correlations may be measured as a function of $n/s$ (ratio of particle density and entropy density) for detecting the CEP~\cite{Akamatsu2019}.



The fluid dynamical description works in the region where the condition, $k<<q$ is maintained (where  $q$ is the inverse of correlation length, expressed as $q=q_o\,r^{\nu}$, $q_o$ is a constant and $\nu$ is the critical index with a numerical value $\nu=0.73\pm 0.02$~\cite{Rajagopal:1992qz}). The hydrodynamics becomes invalid when $\xi$ diverges at the CEP. However, there could be a region of validity in the neighbourhood of the CEP  where the predictions of fluid dynamical approach may be effective. Following procedure similar to the condensed matter physics~\cite{Stanley} we can write, $k<<q_o r^{\nu}$ for the region of validity of the hydrodynamics, which implies,
\begin{equation}
T>T_c\left[1+\left(\frac{k}{q_o}\right)^{1/\nu}\right]
\end{equation}
As hydrodynamics is an effective theory for soft physics (small wave vector or large wavelength), the fluid dynamical description can be applied in the neighbourhood of the CEP but becomes inappropriate at the CEP due to the divergence of the correlation length. The response of the trajectories in the neighbourhood of the CEP in the $T- \mu$ plane to the initial conditions away from equilibrium has been investigated in Ref.~\cite{Dore}. 


In a realistic scenario, the matter formed in RHIC-E 
evolves in space and time from the initial QGP phase to the final hadronic 
freeze-out state through an intermediary phase transition. 
The space time evolution of the locally equilibrated system is described
by relativistic viscous hydrodynamics. The experimentally  detected signals
is the superposition of the yields for all the possible values 
of temperatures and densities realised during the evolution of the system
from its initial to the freeze-out state.
The detection of CEP will require the disentanglement of contributions from the  
neighbourhood ($\mu_c,T_c$) from all other possible values of ($\mu,T$) which the system 
confronts during its evolution history.  
In the present work the expansion dynamics has not been taken into consideration, therefore,
the results obtained here can not be contrasted with experiments directly. The effects of the CEP with
(3+1) dimensional expansion within the scope of second order viscous hydrodynamics will be 
next step in this direction.


The electromagnetic (EM) probes of QGP (see
\cite{ALAM1996243} for review) 
{\it{i.e.}} real photons and lepton pairs can be used to study the evolution of the
system from the pristine partonic stage to the
final hadronic stage through an intermediary
phase transition or crossover.
The photons and lepton can bring the information of the thermodynamic 
state of their production points~\cite{ALAM1996243,ALAM2000159} efficiently as
their  mean free paths are larger than the size of the fireball created
in RHIC-E. This fact should be contrasted with hadrons which  are subjected to rescattering 
in the medium and consequently loose information of their production point. 
Therefore, in principle, the photons can bring the information
of the CEP very efficiently. In Ref.~\cite{Paquet2016} the photon spectra 
has been evaluated  
by using second-order dissipative hydrodynamics 
and the most updated rate of photon productions from QGP and 
hadronic phases. However, the effects of the CEP on EoS, various transport
coefficients and response functions are required to be included 
to get the imprints of CEP on the photon spectra which is beyond the
scope of the present work, although worth exploring and may be taken up in future endeavour.




