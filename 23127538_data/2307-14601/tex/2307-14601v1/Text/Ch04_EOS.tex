\chapter[Equation of State (EoS) with critical point]{Equation of State with critical point}
\label{chapter4}  
\section{Introduction}
\label{sec0401}
After the formation of the QGP in a nuclear collision at relativistic energies, the bulk evolution of the thermalized system is controlled by the hydrodynamic equations, equation of state (EoS) and initial condition. The EoS of the QGP characterizes the transition from the hadronic phase to the QGP phase, is understood by the rise in the number of degrees of freedom, in the  deconfined phase. One of the first EoS of the QGP was the one derived from the MIT bag model~\cite{MITBag}. Because of its simplicity, it has been widely used in the analysis in data from Relativistic Heavy-Ion Collider Experiments (RHIC-E) and in astrophysics and cosmology too.


The bag model was first suggested by the theoretical physicist Nikolay Bogoliubov in 1967. His idea was to consider the quarks with enormous mass making them unable to move. However, this seems to contradict the asymptotic freedom observed at a very close range. He solved this by confining the quarks within a spherical cavity of radius $R$ in which they feel an attractive field. This resulted in Bogoliubov’s bag model, where the quarks could move freely inside the bag but were completely confined within it. Although Bogoliubov’s model was a very simple one, it still gave some reasonable predictions~\cite{Thomas}.


The improvement in the MIT bag-model has an additional term in its Lagrangian density, called the bag constant $\B$. It corresponds to the outwards pressure exerted on the bag. This simple correction has shown to greatly improve agreement between predictions and tabulated values of the masses of different quarks and hadrons. In the Bag model, the energy density and the pressure are given by~\cite{MITBag,Fogaca1,FogacaEOS}:
\beqa
\epsilon=\frac{37\pi^{2}}{30}T^{4}+\B \,\,\,\,\,\,\,\text{and}\,\,\,\,\,\, p=\frac{37\pi^{2}}{90}T^{4}-\B~.
\label{eq0401}
\eeqa
The numerical value of $\B^{1/4}$ is taken as $220 MeV$. 


The lattice QCD provides the EoS in the strongly-coupled regime from first principles, but it is unable to serve at finite $\mu_{B}$ regime. The lattice EoS are till now the best fit to the data~\cite{Borsanyi:2016ksw,Borsanyi:2013bia,Borsanyi:2011sw}. Recently, the continuum extrapolated results for the EoS of QCD with the physical masses of light and strange quark have been incorporated~\cite{Borsanyi:2013bia,HotQCD:2014kol}. The bulk thermodynamic observables such as $P$ (pressure), $\epsilon$ (energy density) and $s$ (entropy density) have now been evaluated at $\mu_{B}=0$ for the three quark flavors $u, d$ and $s$. The thermodynamic observables behave smoothly in the transition region. At reasonably low temperature, the observables are found to be in quite good accordance with hadron resonance gas (HRG) model calculations. Although there are few systematic deviations observed, which may be attributed to the presence of additional resonances which are not accounted in the HRG model calculations in the standard particle data tables~\cite{HRG1,HRG2}. 


Due to the well-known sign problem for lattice QCD formulations at $\mu_{B}\ne 0$, an immediate evaluation of the EoS at non-zero $\mu_{B}$ is unfortunately not possible. Nevertheless, it is made possible for small values of the chemical potentials, by using a Taylor expansion of the thermodynamic potential~\cite{GavaiEoS1,AltonEoS1} and have been obtained on coarse lattices~\cite{AltonEoS1,GavaiEoS2,AltonEoS2}. Such type of expansion scheme have even been extended up to sixth order in the baryon chemical potential~\cite{AltonEoS3,EjiriEoS}. 


But as discussed earlier in Ch.\ref{chapter2}, Sec.\ref{sec0206}, that the CEP appears at $\mu_{B} \ne 0$ and $T\ne 0$ domain of the QCD phase diagram. Therefore, the lattice QCD can not be extended to provide the ultimate EoS near the CEP. In this chapter, we construct an EoS, which contains the CEP, based on Refs.~\cite{Nonaka,Parotto} (see Ref.~\cite{MITWathid} for pedagogical approach). 
%%
%%%
%%%
\section{Construction of the EoS}
\label{sec0402}
As far as the study of the critical point is concerned, the two most common (for better understanding) examples of such critical points are the liquid-gas coexistence curve and
the Curie point in ferromagnetic-paramagnetic transition. Although these two systems are way out on a fundamental, microscopic level, the physical behaviour near the critical point is remarkably similar on a qualitative and even on quantitative level. This is because the diverging nature of the correlation length happens for all the systems near the CEP, and this observation is based on the concept of universality of the second-order phase transitions. As per the previous discussion on the universality hypothesis, the CEP of the QCD belongs to the same universality class as that of the 3D Ising model. Therefore, calculations carried out in the Ising model can be mapped onto the phase diagram of the QCD. 


Within a small critical region around the critical point, the behaviour of thermodynamic quantities is quantitatively governed by the critical phenomena based on the universality class considerations. The order parameter, $M$ (magnetization of the system) in the Ising model is a function of the reduced temperature ($t=(T-T_{c})/T_{c}$) and the magnetic field strength ($h$), and the critical point is positioned at the origin $(t, h) = (0, 0)$ by the construction. For the Ising model, the EoS in the parametric representation is given by~\cite{Guida,Rajagopal:1992qz}
\beqa
M &=& M_0 R^{\beta} \theta~, \nonumber\\
h &=& h_0 R^{\beta \delta} \tilde{h}(\theta) ;\hspace{0.3cm} \tilde{h}(\theta)= (\theta - a\theta^3 + b\theta^5) ~,\nonumber\\
t &=& R(1-\theta^2) ~,
\label{eq0402}
\eeqa
where, $a, b$ are some numerical values given as $a=0.7620103, b=0.008040$, and $R \geq 0, -1.154 \leq \theta \leq +1.154$. The critical exponents are identified as $\beta = 0.326$ and $\delta =4.80$. The $M_0$ and $h_0$ are some normalization constant, can be found out by setting 
\beqa
\label{eq0403}
M(t=-1,h=0+) &=& 1~, \\
\label{eq0404}
M(t=0,h=1) &=&1~,\\
\label{eq0405}
M(t=0, h)&=& sgn(h) |h|^{1/\delta}~.
\eeqa
From Eq.\eqref{eq0403}
\beqa
 R(1-\theta^2)=-1 &\Longrightarrow& \theta^2 = 1+\frac{1}{R}~, \nonumber\\
\theta &=& \pm \sqrt{\frac{R+1}{R}}~.
\label{eq0406}
\eeqa
Thus, substituting in $M$, we have
\beqa
M &=& M_0R^{\beta}\theta~, \nonumber\\
1&=& M_0R^{\beta}\sqrt{\frac{R+1}{R}}~.
\label{0407}
\eeqa
\beq
M_0 =\frac{1}{R^{\beta} \sqrt{\frac{R+1}{R}}}~.
\label{0408}
\eeq
To evaluate $h_0$, we take Eq.\eqref{eq0404} and we calculate
\beqa
r &=& 0 = R(1-\theta^2) \nonumber\\
\theta &=& \pm 1 ~.\nonumber\\
\label{eq0409}
\eeqa
Considering $\theta =-1$, the magnetization defined in Eq.\eqref{eq0402}, $M$ becomes negative, which is unphysical because $M$ is always along the direction of the applied field $h$. Thus, $\theta=-1$ value is excluded and the physical value for $\theta$ will be $\theta=+1$. So
\beqa
h &=& h_0 R^{\beta \delta} \tilde{h}(\theta) \nonumber\\
&=& h_0 R^{\beta \delta} (\theta - a\theta^3 + b\theta^5) \nonumber\\
1 &=& h_0 R^{\beta \delta} (1-a+b)  
\label{eq0410}
\eeqa
Thus,
\beq
h_0 = \frac{1}{R^{\beta \delta}(1-a+b)}~.
\label{0411}
\eeq
Now to construct the EoS for Ising model, we will start with the deduction of the entropy density near the critical point, which is related to the Gibb's energy density 
\beq
G(h,t)=F(M,t)-Mh=E-Ts_c-Mh~.
\label{eq0412}
\eeq
where, $F (M, t)$ is the Helmholtz's free energy and $s_{c}$ is the entropy density at the CEP. Generally from thermodynamic relations, we have
\beqa
G =\epsilon-Ts_{c}+P+n\mu_B &\Longrightarrow& dG = dP-sdT+nd\mu_B~, \nn\\
s_{c} &=&- \Big(\frac{\partial G}{\partial T}\Big)_{\mu_B}~.
\label{eq0413}
\eeqa
But as $G \equiv G(h,t)$, we can have
\beqa
s_c =-\Big[\big(\frac{\partial G}{\partial h}\big)_t\big(\frac{\partial h}{\partial T}\big)_{\mu_B}+\big(\frac{\partial G}{\partial t}\big)_h\big(\frac{\partial t}{\partial T}\big)_{\mu_B}\Big]~.
\label{eq0414}
\eeqa
The four terms $(\frac{\partial G}{\partial h})_t, (\frac{\partial G}{\partial t})_h, \big(\frac{\partial h}{\partial T}\big)_{\mu_B}, \big(\frac{\partial t}{\partial T}\big)_{\mu_B}$ need to be evaluated as: 
\beqa
\Big(\frac{\partial G}{\partial h}\Big)_t &=& \Big(\frac{\partial F}{\partial h}\Big)_t-h\Big(\frac{\partial M}{\partial h}\Big)_t-M~.
\label{eq0415}
\eeqa
Decomposing the $\Big(\frac{\partial F}{\partial h}\Big)_t$ and substituting in the above equation, we get
\beqa
\Big(\frac{\partial G}{\partial h}\Big)_t &=& \Big(\frac{\partial F}{\partial M}\Big)_t \Big(\frac{\partial M}{\partial h}\Big)_t-h\Big(\frac{\partial M}{\partial h}\Big)_t-M~.
\label{eq0416}
\eeqa
From free energy, F, we have $h= \big(\frac{\partial F}{\partial M}\big)_t$ to impose into the previous equation to give 
\beqa
\Big(\frac{\partial G}{\partial h}\Big)_t=-M~.
\label{eq0417}
\eeqa
Now,
\beqa
\Big(\frac{\partial G}{\partial t}\Big)_h &=& \Big(\frac{\partial F}{\partial M}\Big)_t \Big(\frac{\partial M}{\partial t}\Big)_h+\Big(\frac{\partial F}{\partial t}\Big)_M-h\Big(\frac{\partial M}{\partial t}_{}\Big)_h-M\Big(\frac{\partial h}{\partial t}\Big)_{h}\nn\\
&=& \Big(\frac{\partial F}{\partial t}\Big)_M~.
\label{eq0418}
\eeqa
Therefore, the critical entropy density from Eq.\eqref{eq0413} becomes
 \beqa
 s_c(h,\,t)&=&-\Big[\big(-M\big)\Big(\frac{\partial h}{\partial T}\Big)_{\mu_B}+\Big(\frac{\partial F}{\partial t}\Big)_M\Big(\frac{\partial t}{\partial T}\Big)_{\mu_B}\Big]~,\nn\\
 &=&\Big[M\Big(\frac{\partial h}{\partial T}\Big)_{\mu_B}-\Big(\frac{\partial F}{\partial t}\Big)_M\Big(\frac{\partial t}{\partial T}\Big)_{\mu_B}\Big]~.
 \label{aaa}
 \eeqa

In the above equation, three unknown, $\big(\frac{\partial h}{\partial T}\big)_{\mu_B}, \big(\frac{\partial F}{\partial T}\big)_{M},\big(\frac{\partial t}{\partial T}\big)_{\mu_B}$ are to be evaluated.  And to evaluate the quantities, $\big(\frac{\partial h}{\partial T}\big)_{\mu_B}$, and $\big(\frac{\partial t}{\partial T}\big)_{\mu_B}$,  we need the mapping from $(t,h)\to (\mu_{B},T)$ in the QCD phase diagram.
 %%%
 
 %%%
\subsection{Mapping onto the QCD phase diagram}
The mapping between the two planes is essential for determining  $(\frac{\partial h}{\partial T})_{\mu_B}$ and $(\frac{\partial t}{\partial T})_{\mu_B}$. To map we consider the $t$-axis, tangential to the first-order phase transition line at the CEP of the QCD phase diagram. However, we can choose arbitrary axes for $h$-component. But for the simplicity of our calculations, we take $h$-axis to be perpendicular to the $t$-axis as shown in Fig.\ref{fig0401}.
%%%
%%
%%
% Figure environment removed
%%
%%
%%
\beq
(t,h)=(0,0) \,\,\,\,\,\,\,\text{is the position of the critical point}
\label{eq0419}
\eeq
Thus on QCD phase diagram ($T-\mu_B$) plane,
\beqa
t & > & 0~, \hspace{1cm} \text{is a crossover and} \nonumber\\
t & < & 0~, \hspace{1cm} \text{is the first-order phase transition}
\label{eq0420}
\eeqa
If we consider that the CEP is located very close to $T$-axis ($\mu_B \rightarrow 0$) in the QCD phase diagram, the $t$-axis is now parallel to $\mu_B$-axis and $h$-axis is parallel to $T$-axis. The most natural thing to do is to assume a finite (but small) critical region, where the relations between the scale of the $(\mu_{B}, T)$ and the scale of $(t, h)$ is linear. Therefore, it is necessary to choose the domain of the critical region, and for that the location of the CEP must be mentioned.
%%%%%
%%%%
%%%%%
\subsection{Position of the CEP and the choice of critical region} 
The exact location of the CEP in the QCD phase diagram has still remained questionable. Some of the QCD based models could also predict the location of the CEP, but the position of the CEP varies with the models of the parameters used. The location of the CEP having uncertainties ranging from 266-504 MeV in $\mu_{c}$ and 115-162 MeV in $T_c$~\cite{Luo2017}. The holographic model for five-dimensional black holes even indicated the critical coordinate as $(\mu_c,T_c)=(783$ MeV,143 MeV$)$~\cite{Holography1}. It is clear that the position of the CEP is sensitive to the parameters of the models used. It is argued in recent studies that it is unlikely for the CEP to be located at $\mu_{c}<2T_{c}$~\cite{BazavovCEP,Andronic:2017pug,Luo2017} or disfavours the position at $\mu_{c}/T_{c}\sim2-3$~\cite{VovchenkoCEP,Fodor2021PRL}. 

%There are fewer studies from Lattice QCD~\cite{Fodor:2004nz,Saumen1,Saumen2,KARSCH2017461,Karsch:2016yzt} as well as Dyson-Schwinger equation~\cite{Shi,Fischer,An} to predict about the location of the CEP. 


In this present dissertation, we chose ($\mu_{c},T_{c})= (367, 154)$MeV~\cite{Nonaka}, which certainly satisfies the criterion of $\mu_{c}/T_{c}>2$. However, the results and conclusion drawn from such a selection of $\mu_c$ and $T_c$  will remain valid for other values of $\mu_c$ and $T_c$ too.


The universality hypothesis tells that the critical exponents around second-order phase transitions are characterized by the dimensionality and symmetry of the system. The singular
part near the CEP (a function of two variables) to be mapped onto the variables characterizing the phase diagram of the 3D Ising model. In the
QCD phase diagram ($\mu_{B}-T$ plane), $t-$axis directs towards the direction of $T-axis$ but the direction of $h$ is not known~\cite{Berdnikov:1999ph,Hatta2003}. However, it is evident that the critical region is more elongated along the $t$ direction compared to the $h$ direction.  This is because of the fact that the critical exponent associated with $t$ is larger than that associated with $h$~\cite{Berdnikov:1999ph,PhysRevLett.101.122302}. Thermalized QGP follows a path of an isentropic trajectory and can be bent near the critical point (region), known as focussing effect~\cite{Nonaka,PhysRevLett.101.122302} is one of the fundamental properties of the CEP, can affect the final state of particle spectra.~\cite{PhysRevLett.101.122302,NPA2009}. Therefore, the choice of the elongation of the critical region plays an important role in the evolution of the QGP. Even model studies suggest that the size of the critical region is sensitive to affect the hydrodynamic evolution~\cite{PRD2007}. Here we simply assume that the elongation of the critical region is sufficiently large to induce a significant focusing effect and used the numerical values suggested in Ref.~\cite{Nonaka}. Finally, in the critical region, the linear mapping between $(t, h) \to (\mu_{B}, T)$ is depicted as,
\beqa
\label{eq0421}
t&=& \frac{\mu_B-\mu_{c}}{\Delta \mu_{c}}=\frac{\mu_B-0.367}{-0.2}~,\\
\label{eq0422}
h&=& \frac{T-T_{c}}{\Delta T_{c}}=\frac{T-0.154}{0.1}~.
\eeqa
It is clearly seen that the elongation of the critical region along $t$-axis than the $h$-axis. As $T$ and $\mu_{B}$ are two independent quantities in QCD phase diagram, we can calculate (using Eqs.\eqref{eq0421} and \eqref{eq0422})
\beqa
\Big(\frac{\partial h}{\partial T}\Big)_{\mu_B} = \frac{1}{\Delta T_{c}}~\,\,\,,\,\,\Big(\frac{\partial t}{\partial T}\Big)_{\mu_B} = 0~.
\label{eq0424}
\eeqa
At the end we plug into Eq.\eqref{aaa} to get
\beqa
s_c(T,\,\mu_{B}) =\frac{M(t, h)}{\Delta T_{c}}=M\Big(\frac{T-T_{c}}{\Delta T_{c}},\frac{\mu_B-\mu_{c}}{\Delta \mu_{c}}\Big)\frac{1}{\Delta T_{c}}~.
\label{eq0425}
\eeqa
%%%%%%
%%
%%
% Figure environment removed
%%
%%
%%
%%
%%%%%%
In order to construct the EoS, we first construct a  dimensionless entropy density  as
\begin{eqnarray}
{S_{c}}= A (\Delta T_c, \Delta\mu_{c})~s_c (T,\mu_{B})~.
\label{eq0426}
\end{eqnarray}
here, $A$ is defined as 
\begin{equation}
A(\Delta T_{c},\Delta \mu_{c}) = D \sqrt{\Delta T^2_{c}+\Delta \mu_{c}^2)}~.
\label{eq0427}
\end{equation}
%%
%%
% Figure environment removed
Here $D$ is also a dimensionless quantity which represents the extension of the critical region. In this dissertation, we use $(T_c, \mu_c)= (154 \text{MeV}, 367 \text{MeV})$ with $(\Delta T_{c}, \Delta \mu_{c}, D)=(0.1\,\text{GeV}, -0.2\, \text{GeV}, 2)$. The construction of the entropy density is done by using $S_{c}$ as a switching function and by appropriately connecting entropy densities of the QGP ($s_Q$) and the hadronic ($s_H$) gases. The final result reads as:  
	\begin{eqnarray}
	s(T, \mu_{B})&=&\frac{1}{2}\Big[1-{\rm tanh}{S_{c}}(T, \mu_{B})\Big]~ s_{Q}(T, \mu_{B}) \nn\\
	&+& \frac{1}{2}\Big[1+{\rm tanh}{S_{c}}(T, \mu_{B})\Big]~s_H(T, \mu_{B})~.
	\label{eq0428}
	\end{eqnarray}
%%%
A 3D plot of $s(T,\,\mu_{B})$ is shown in Fig.\ref{fig0402}, where we clearly see a discontinuity in the entropy density at $(T_{c},\, \mu_{c})$. We calculate $s_{Q}$ by using the following expression~\cite{Satarov2009}
	\begin{eqnarray}
	s_{Q}(T, \mu_{B})=\frac{32+21N_{f}}{45}\,\pi^{2}T^{3}+\frac{N_{f}}{9}\,\mu^{2}_{B}T ~.
	\label{eq0429}
	\end{eqnarray}
with $N_{f}$ to be the number of quark flavors. In Eq.~\eqref{eq0429}, massless quarks and gluons are considered. However, the effects of thermal masses can be taken into calculation through the effective degeneracy of quarks and gluons ~\cite{Jane1996} (see Ref.~\cite{Kolb1990}). To estimate the effective degeneracy factor with thermal quark mass we consider the energy density of quarks as,
%%%
% Figure environment removed
\begin{equation}
\epsilon_q(T,\mu_{B},m_q)=\frac{g_q^{eff}}{(2\pi)^3}\int\,d^3p \sqrt{p^2+m_q^{2}}\,f_{FD}(E_i)~.
\label{eq0430}
\end{equation}
Here, $m_{q}$ is the thermal mass of quark, $p$ is the momentum of the quarks and $f_{FD}$ is the Fermi-Dirac distribution function. The effective degeneracy of quark is given by
\begin{equation}
g_q^{eff}=\frac{\epsilon_q(T,\mu_{B},m_q)}{g_q}~,
\label{eq0431}
\end{equation}
where, $g_{q}$ is the quark's degeneracy factor. Similarly for gluons the effective degeneracy can be estimated by using the relation
\begin{equation}
g_g^{eff}=\frac{\epsilon_g(T,m_g)}{g_g}~.
\label{eq0432}
\end{equation}
where, $m_{g}$ is the thermal mass of the gluon and $g_{g}$ is the gluon's degeneracy factor.

%%%%
We have calculated $g_q^{eff}=5.76$ with thermal mass at $(\mu_{B},T)=(367,158)$ MeV, whereas for massless quark the degeneracy is found to be $g_{q}=6$. In case of gluon, with thermal mass $g_g^{eff}=14.2$ which is $g_{g}=16$ for massless gluon. For $(\mu_{B},\,T)=(367,\,158)$ MeV the thermal momenta of quarks and gluons are $1064$ MeV and $497$ MeV respectively where as the respective thermal masses are 102 MeV and 208 MeV respectively. This indicates that the effect of thermal mass at the critical point will not be
significant for the present study.
%%
%%
	%%%
% Figure environment removed
% Figure environment removed
%%


The hadronic entropy density
($s_H$) can be estimated from the following expression ~\cite{Andronic2012,Sarwar:2015irq}, 
\begin{eqnarray}
s_H(T, \mu_{B})=\pm \sum_{i}\frac{g_{i}}{2\pi^{2}}
\int^{\infty}_{0}&&dp^{\prime}{p^\prime}^{2}\Big[ln\Big(1\pm \{exp(E_{i}-\mu_{i})/T\}\Big)\nn\\
&&\pm \frac{E_{i}-\mu_{i}}{T\{exp(E_{i}-\mu_{i})/T\pm 1\}}\Big]~,
\label{eq0433}
\end{eqnarray}
where, the sum extends over all hadrons with mass up to 2.5 GeV~\cite{Sarwar:2015irq}, $g_i$  represents the statistical degeneracy factor of the $i^{th}$ hadron, and $E_{i}=\sqrt{{p^\prime}^{2}_{i}+m^{2}_{i}}$ is  the energy of the $i^{\text{th}}$ hadron of mass, $m_i$ and momentum, $p_i$. Once we know the entropy density, the thermodynamic quantities {\it e.g.} baryon number density ($n_{B}$), pressure ($P$) and energy density ($\epsilon$) can be evaluated as follows. The net baryon number density is given by 
	\begin{eqnarray}
	n_{B}(T, \mu_{B})= \int_{0}^{T} \frac{\partial s(T^{'}, \mu_{B})}{\partial \mu_{B}} dT^{'}
	+n_{B}(0, \mu_{B})~.
	\label{eq0434}
	\end{eqnarray}
	%%
%%
%%
	A plot of baryon number density is shown in Fig.\ref{fig0404}. From $n_{B}$, the pressure can be estimated as 
	\begin{eqnarray}
	P(T, \mu_{B})=  \int_{0}^{T} s (T^{'}, \mu_{B}) dT^{'} +P (0, \mu_{B})~,
	\label{eq0435}
	\end{eqnarray}
	where,  $n_{B}(0,T)$ and $p (0, \mu_{B})$ are the net baryon density and pressure respectively at  $T=0$. Finally, the energy density is given by, 
	\begin{eqnarray}
	\epsilon(T, \mu_{B})= Ts(T,\mu_{B})-P(T,\mu_{B})+\mu_{B} n_{B}~.
	\label{eq0436}
	\end{eqnarray}
	%where, $n(0, \mu)$ is the initial condition which is taken as zero for simplicity. 
	To get the first order phase boundary,  the discontinuity in the entropy density along the transition line also needs to be considered. We add the following term to Eq.\eqref{eq0434} to  take into account  this possibility (for $T>T_{c}$) 
	\begin{eqnarray}
	\Big|\frac{\partial T_{c}(\mu_{B})}{\partial \mu_{B}}
        \Big|\Big[s(T_c(\mu_{B})+\Delta,\mu_{B})-s(T_c(\mu_{B})-\Delta,\mu_{B})\Big]~,
	\label{eq0437}
	\end{eqnarray}
	where, $\Big|\frac{\partial T_{c}}{\partial \mu_{B}}\Big|=tan\theta_{c}$, is the tangent 
	at the $T_c$ and $\Delta (\rightarrow 0)$ is a small deviation in $T$   
        from $T_{c}$.
	The value of $T_{c}$  for the first order transition depends on $\mu_{B}$ 
	as indicated by Eq.~\eqref{eq0437}. 

In this context we contrast the entropy density, speed of sound $c_{s}^{2}=(\pd P/\pd \epsilon)_{s}$ and the baryon susceptibility, $\chi_2^B$ (Eq.\eqref{barsusc}) obtained in the present work with the corresponding lattice QCD results~\cite{Borsanyi:2013bia,Borsanyi:2011sw} in the $\mu\rightarrow 0$ limit, as shown in Figs.\ref{fig0405} and \ref{fig0406} respectively.


Recently, a variety of sophisticated EoS has been constructed~\cite{Parotto}, where authors have presented a procedure to construct a family of model EoS for QCD, each of which
contains a critical point in accordance of the universality class consideration of the 3D Ising model. They have reconstructed the EoS in a way such that it precisely features the critical behavior in the correct universality class, and as well as it matches lattice QCD results up to $\mathcal{O} (\mu_{B}^{4})$ exactly. They have also argued that the family of constructed model EoS can also be used directly to feature the divergence of quantities in the the critical region, in particular the baryon number cumulants: the second cumulant, which is related to the variance of the net-proton number distribution. For other variety of EoS at finite $\mu_{B}$, one can dig into the Refs.~\cite{Karthein2021,Fodor2021PRL,Monnai2019,Soloveva2021}.
%%%%%%
\section{Entropy density calculation for arbitrary chosen axis}
The transport coefficients show diverging nature, or at least enhance near the CEP~\cite{Monnai2017,KapustaChi,Antoniou2017,Minami_thesis,Hohenberg1977,Son2004,Moore2008,Onuki1997}. In Ref.\cite{Martinez2019}, Martinez {\textit{et al.}} showed that the critical bulk viscosity is very sensitive to the inclination of the Ising axes in the QCD phase diagram. In most of the construction of the critical EoS, it is assumed that the $t$-axis is aligned with the $\mu_{B}$-axis of the QCD phase diagram and that the $h$-axis is perpendicular to that. In that case, the bulk viscosity is suppressed by a factor of $(n/s)^{2}$ compared to when we include a possible tilt of the $t$-axis with the $\mu_{B}$-axis. This is also true in the sense that the CEP may be located at any arbitrary point, which may cause the $t$-axis to be tilted with respect to the $\mu_{B}$-axis. Therefore, the mapping procedure can be improved to evaluate $s_{c}$ (in Eq.\eqref{eq0425}) in a more general sense when the $t$-axis makes an angle $\theta_{c}$ with the $\mu_B$-axis. The mapping procedure have to include $cos(\theta_{c})$ and $sin(\theta_{c})$ terms. So, in Eq.\eqref{aaa}, the term $(\pd t/\pd T)_{\mu_{B}}$ must not vanish, and thus contribute in the form of $s_{c}(T, \mu_{B})$ as:
\beqa
s_c (T, \mu_{B})&=& M\Big(\frac{(T-T_{c})cos(\theta_{c})}{\Delta T_{c}},\frac{(\mu_B-\mu_{c})cos(\theta_{c})}{\Delta \mu_{c}}\Big)\frac{cos(\theta_{c})}{\Delta T_{c}} \nonumber\\
&-& \Big(\frac{\partial G}{\partial t}\Big)_h\Big(\frac{(T-T_{c})cos(\theta_{c})}{\Delta T_{c}},\frac{(\mu_B-\mu_{c})cos(\theta_{c})}{\Delta \mu_{c}}\Big)\frac{sin(\theta_{c})}{\Delta T_{c}}~.
\eeqa



%
%%
%%
\section{Hydrodynamic framework in space-time evolution of the QGP}
Fixing the initial condition and after construction of the EoS, we can now solve the hydrodynamic equations uniquely. In a collision of heavy ions moving relativistically, the QGP is likely to form, and it expands very rapidly after the formation. The relativistic hydrodynamics can be a powerful tool to describe the collective flow of QCD matter, created in heavy-ion collisions (HICs). Describing elliptic flow ($v_{2}$)~\cite{Snellings:2011sz} and other flow observables ($v_{n}$)~\cite{Ollitrault:2007du} on a quantitative level is one of the greatest successes of the fluid dynamical description~\cite{Rischke:1998fq,Shuryak:2003xe,Stoecker:1986ci}. Ideal fluid dynamics quantitatively can explain only in central collisions between large $(A \sim 200)$ nuclei at mid-rapidity at top RHIC energies, but gradually break down in a smaller systems, or in a system produced from peripheral collisions, {\it{i.e.}} away from the mid-rapidity region, and at lower collision energies~\cite{Heinz:2004ar}. Even with the highest achievable centre of mass energy $(\sqrt{s})$, the lowest limit of the shear viscosity to entropy ratio is found to be $\eta/s=1/4\pi$, has been proposed based on a correspondence with black-hole physics, known as KSS bound~\cite{Kovtun:2004de}. The viscous hydrodynamics was applied extensively ever since the estimation of surprisingly small value of $\eta/s$ from the analysis of the elliptic flow data~\cite{Romatschke:2007mq}. A study of elliptic flow suggests that the magnitude of viscous corrections is at least $30 \%$~\cite{Drescher:2007cd}. Therefore, the description of strongly coupled QGP produced in HICs will be in better agreement with the viscous effect. Furthermore, if QGP fluid is formed in heavy-ion collisions, it needs to be characterized by its relevant transport coefficients, {\it{e.g.}} bulk viscosity, shear viscosity and the thermal conductivity.
