\chapter[Dynamic structure factor near the QCD critical point]{Dynamic structure factor near the QCD critical point}
\label{chapter6}  
\section{Introduction}
\label{sec0601}
This chapter is based on the publication as shown in Ref.\cite{Hasan2}. Here, we calculate the dynamic structure factor, which is the time-dependent correlator of density fluctuations. This is evaluated from the linear analysis of the hydrodynamic equations, and we will see its behaviour near the QCD critical point. The importance of the spectral structure, more specifically the structure factor, denoted by, $\Snn$, is an accessible quantity in laboratory, in condensed matter physics by light scattering, X-ray diffraction, and neutron scattering. The spectrum of light scattered by a fluid is proportional to $\Snn$, where $k=|\vec{k}|$ is to be identified with the so-called scattering wave vector, which is related to the geometry of the scattering experiment and the wavelength of the incident light. The spectrum of scattered light contains separately identifiable Lorentzians, called Rayleigh-line (peaks) named after Lord Rayleigh~\cite{Rayleigh1881} and Brillouin-line (peaks), measured by Fleury and Boon~\cite{FlerryandBoon1969} in 1969. Rayleigh line (R-line), arises from entropy or, equivalently, temperature fluctuations at constant pressure, whereas B-lines arise from thermally excited propagating sound waves (modes) associated with adiabatic pressure fluctuations~\cite{Stanley,Linda}. The Rayleigh line and the Brillouin lines are usually investigated separately, both in theory and in experiments.



It is well-known that the correlation length ($\xi$) diverges at the critical point. The effects of the divergence on the particle correlations, baryon number fluctuations, and correlations of density  fluctuations will have a better chance to get detected provided the fluctuations to survives the evolution of the hadronic phase. It is important to understand the correlations of those fluctuations theoretically by identifying signatures of the CEP in hadronic spectra. The effects of CEP on the evolution of QGP go as input to the hydrodynamic equations through the equation of state (EoS) and via the criticality of the transport coefficients and the thermodynamic response functions of the medium~\cite{KapustaChi,Guida,Rajagopal:1992qz}. 

The dynamical structure factor, $\Snn$ has been estimated earlier in Ref.~\cite{Minami,Minami_thesis} without taking into account the effect of EoS. It is shown in this work that the EoS plays a vital role in determining the behaviour of $\Snn$, especially its strength at the R-peak, which is modified by several orders of magnitude when the effects of the CEP in EoS is incorporated. The EoS has a strong effects on B-peaks too because it determines $c_s$ (speed of sound) and hence the location of the B-peaks. It is also very crucial to understand whether all the hydrodynamic modes ($k$-modes) travel at the same speed or not. The R-peak and the B-peaks will be closer for slower modes even at points away from the CEP. Therefore, the structure of $\Snn$ will shed light on the speed of the perturbation propagating as a sound wave.
%%
%%
%%%
\section{Linearized Hydrodynamic equations}
\label{sec0602}
In this chapter, we use the Eckart's frame of reference as discussed in Sec.\ref{sec0303} in Ch.\ref{chapter3}, where it is considered that the heat flux is non-zero but the particle current is zero. Therefore, energy-momentum tensor (EMT) and the the particle current ($N^\mu$) are given by Eqs.\eqref{eq0129} and \eqref{eq0130}. The conservations of EMT and the net baryon number follows the Eq.\eqref{eq0349}.
%%%
%%%


The hydrodynamic Eqs.~\eqref{eq0349} are partial differential equations, which are non-linear, could not be solved analytically in general. Presently we aim to obtain the $\Snn$ of the dynamical density fluctuations in $\omega-k$ space. The hydrodynamical equations presented above can be linearized to express small perturbations (in magnitude) in the thermodynamical variables (small deviations from the equilibrium values of the variable). These linearized equations can be solved to obtain the dynamical density fluctuations. Let $Q_0$ ($Q$) represent a thermodynamic quantity in (away from) equilibrium. A small perturbation $\delta Q$ can be written as: $Q=Q_{0}+\delta Q$ where $Q$ can be any of the thermodynamic quantities such as $n, \epsilon, u^{\alpha}, q^{\alpha}, s, \Pi, \pi^{\alpha \beta}$ (baryon number density, energy density, fluid four-velocity, heat flow vector, entropy density, bulk pressure, shear stress tensor respectively), etc. The linearized hydrodynamic equations around the equilibrium~\cite{Minami,Romatschke2010,Hasan1,Sayantani2019,Grozdanov} thus become:
\beqa
0&=&\frac{\pd \delta n}{\pd t}+n_{0} \vec{\nabla}.\delta \vec{v}~,  \nn\\
%%%%%
0&=&h_{0}\frac{\pd \delta v}{\pd t}+\nabla(\delta P+ \delta \Pi)+\frac{\pd \delta q}{\pd t}+\vec{\nabla}.\delta\vec{\pi}~, \nn\\
 %%%%
0&=& \delta \Pi +\zeta[\vec{\nabla}.\delta \vec{v}+\beta_{0}\frac{\pd \delta \Pi}{\pd t}-\tilde{\alpha_{0}}\vec{\nabla}.\delta \vec{q}] ~,\nn\\
%%
0&=& \delta \pi^{ij}+\eta[\pd^{i}\delta v^{j}+\pd^{j}\delta v^{i}-\frac{2}{3}g^{ij}\vec{\nabla}.\delta \vec{v}+2\beta_{2}\frac{\pd \delta \pi^{ij}}{\pd t} \nn\\
&&-\tilde{\alpha_{1}}(\pd^{i}\delta q^{j}+\pd^{j}\delta q^{i}-\frac{2}{3}g^{ij}\vec{\nabla}.\delta \vec{q})~,\nn\\
%%%%%
\eeqa
%%%%%
\beqa
0&=& \delta q+\kappa T_{0}[\frac{\nabla \delta T}{T_{0}}+ \frac{\pd \delta v}{\pd t}+\tilde{\beta_{1}}\frac{\pd \delta q}{\pd t}-\alpha_{0}\nabla \delta \Pi-\tilde{\alpha_{1}}\vec{\nabla}.\delta\vec{\pi}] ~,\nn\\
%%%%
0&=& n_{0}\frac{\pd \delta s}{\pd t}+\frac{1}{T_{0}}\vec{\nabla}.\delta \vec{q}~,
\label{eq0605}
\eeqa
where, $h_{0}=\epsilon_{0}+P_{0}$ is the enthalpy density in equilibrium and $\vec{v}$ is the space dependent velocity of fluid four-velocity. Now we decompose the fluid velocity along the directions parallel and perpendicular to the direction of wavevector, $\vec{k}$ and denote them as $\delta \vec{v_{||}}$ and $\delta\vec{v_{\perp}}$ respectively. Here, we have considered the longitudinal components only by setting $\vec{k}.\delta \vec{v_{\perp}}=0$. The hydrodynamic equations 
can be solved for a given set of initial condition, $n(0),\, v_{||}(0),\, T(0),\, q(0),\, \Pi(0)$ and  $\pi(0)$, by using the Fourier-Laplace transformation as the following:
\beqa
\delta Q(\vec{k}, \omega)= \int^{\infty}_{-\infty} d^{3}{r}\,\, \int^{\infty}_{0}dt e^{-i(\vec{k}.\vec{r}-\omega t)} \delta \tilde{Q}(\vec{r}, t)~.
\label{eq0606}
\eeqa
And,
\beqa
\delta Q(\vec{k},\,0)=\delta Q(\vec{k}, t=0)= \int^{\infty}_{-\infty} d^{3}{r} \,\, e^{-i(\vec{k}.\vec{r})} \delta \tilde{Q}(\vec{r}, t=0)~.
\label{eq0606p}
\eeqa 
The $\delta P$ and $\delta s$ can be expressed in terms of the independent variables $n$ and $T$ as follows by using the thermodynamic relations:
\beqa
\delta P&=&\Big(\frac{\pd P}{\pd n}\Big)_{T}\delta n+ \Big(\frac{\pd P}{\pd T}\Big)_{n}\delta T~,\nn\\
\delta s&=&\Big(\frac{\pd s}{\pd n}\Big)_{T}\delta n+ \Big(\frac{\pd s}{\pd T}\Big)_{n}\delta T~.
\label{eq0607}
\eeqa
We use Eqs. \eqref{eq0606}, \eqref{eq0606p} and \eqref{eq0607} to write down the longitudinal linearized hydrodynamic equation as: 
\beqa
\delta Q(\vec{k}, \omega)=\mathcal{M}^{-1} \delta Q(\vec{k}, 0)~,
\label{eq0608}
\eeqa
where, 
%%
\beqa
\mathcal{M} =
\begin{bmatrix}
i\omega & ikn_{0} & 0 & 0 & 0  & 0   \\
%%%%%
\frac{ik}{h_{0}} \Big(\frac{\pd P}{\pd n}\Big)_{T}& i\omega & \frac{ik}{h_{0}} \Big(\frac{\pd P}{\pd T}\Big)_{n}& \frac{i\omega}{h_{0}} &\frac{ik}{h_{0}} & \frac{ik}{h_{0}}  \\
%%%%%
%%%%%
0 & ik\zeta & 0 & -ik\tilde{\alpha_{0}}\zeta & 1+i\omega \beta_{0} \zeta & 0 \\
%%%%%
0 & -i\frac{4}{3}k\eta & 0 & i\frac{4}{3}\tilde{\alpha_{1}}k \eta & 0 & 1+2i\omega \beta_{2}\eta\\
%%%%%
%%%%%%%
0 & i\omega \kappa T_{0} & ik\kappa & 1+i\omega \tilde{\beta_{1}} \kappa T_{0} & ik\alpha_{0} \kappa T_{0} & ik \tilde{\alpha_{1}} \kappa T_{0} \\
 %%%
-i\omega n_{0}\Big(\frac{\pd s}{\pd n}\Big)_{T} & 0 &   i\omega n_{0}\Big(\frac{\pd s}{\pd T}\Big)_{n}&\frac{ik}{T_{0}}  & 0  & 0  \\
\end{bmatrix}
\nn\\
,
\label{eq0609}
\eeqa
and 
\beqa
\delta Q(\vec{k},\omega)=
\begin{bmatrix}
\delta n(\vec{k},\omega) \\
\delta v_{||}(\vec{k},\omega)\\
\delta \Pi(\vec{k},\omega)\\
\delta \pi_{||}(\vec{k},\omega)\\
\delta q_{||}(\vec{k},\omega)\\
\delta T(\vec{k},\omega)
\end{bmatrix}
;~
\delta Q(\vec{k},\,0)=
\begin{bmatrix}
\delta n(\vec{k},0) \\
%%%%
\delta v_{||}(\vec{k},0)+\frac{1}{h_{0}}\delta q_{||}(\vec{k},0)\\
%%%
i\omega \beta_{0}\zeta \delta \Pi(\vec{k},0)\\
%%%%
%%%%%%
-2\beta_{2} \eta \delta \pi_{||}(\vec{k},0)\\
%%%%
-\kappa T_{0}\delta v_{||}(\vec{k},0)+\kappa T_{0}\tilde{\beta_{1}}\delta q_{||}(\vec{k},0)\\
%%%%%
-n_{0}\Big(\frac{\pd s}{\pd n}\Big)_{T}\delta n(\vec{k},0)+n_{0}\Big(\frac{\pd s}{\pd T}\Big)_{n}\delta T(\vec{k},0)\\
\end{bmatrix}
\label{eq0610}
\eeqa
We are concerned here 
about the density fluctuation which is given by, 
\beqa
\delta n(\vec{k},\omega)&=&\Big[ \mathcal{M}^{-1}_{11}-n_{0}\Big(\frac{\pd s}{\pd n}\Big)_{T}\mathcal{M}^{-1}_{16}\Big ] \delta n(\vec{k},0)
%%%
+\Big[\mathcal{M}^{-1}_{12} -\kappa T_{0}\mathcal{M}^{-1}_{15}\Big]\vec{\delta v_{||}}(\vec{k},0) \nn\\
%%
&&+\mathcal{M}^{-1}_{13}\Big[ i\omega \beta_{0} \zeta\Big] \delta \Pi(\vec{k},0) 
%%%%%
 - \mathcal{M}^{-1}_{14}\Big[ 2\beta_{2}\eta\Big]\delta\pi_{||}(\vec{k},0) \nn\\
%%%
&&\Big[\frac{1}{h_{0}}\mathcal{M}^{-1}_{12}+\kappa T_{0}\tilde{\beta_{1}}\mathcal{M}^{-1}_{15}\Big] \delta q_{||} (\vec{k},0)
%%%
+  \mathcal{M}^{-1}_{16} \Big[ n_{0} \Big(\frac{\pd s}{\pd T}\Big)_{n}\Big]\delta T(\vec{k},0)~.
\label{eq0611}
\eeqa
%%%
We define $\mathcal{S^\prime}_{nn}(\vec{k},\omega)$ by the 
following expression: 
\beqa
\mathcal{S^\prime}_{nn}(\vec{k},\omega)=\Big< \delta n(\vec{k},\omega)\delta n(\vec{k},0)\Big>~.
\label{eq0612}
\eeqa
The correlation between two independent thermodynamic variables, say, $Q_i$ and $Q_j$  vanishes {\it{i.e.}}
\beqa
\Big< \delta Q_{i}(\vec{k},\omega)\delta Q_{j}(\vec{k},0)\Big>=0, \,\,\,\, i\neq j~.
\label{eq0613}
\eeqa
Using Eq.\eqref{eq0613} into Eq.\eqref{eq0611}, the required correlator, $\mathcal{S^\prime}_{nn}(\vec{k},\omega)$ is obtained as:
\beqa
\mathcal{S^\prime}_{nn}(\vec{k},\omega)&=&\Big[ \mathcal{M}^{-1}_{11}-n_{0}\Big(\frac{\pd s}{\pd n}\Big)_{T}\mathcal{M}^{-1}_{16}\Big ] \Big< \delta n(\vec{k},0)\delta n(\vec{k},0)\Big>~.
\label{eq0614}
\eeqa
Finally, $\Snn$ is defined as: 
\begin{equation}
\mathcal{S}_{nn}(\vec{k},\omega)=\frac{\mathcal{S^\prime}_{nn}(\vec{k},\omega)}
   {\Big< \delta n(\vec{k},0)\delta n(\vec{k},0)\Big>}~.
\label{eq0615}
\end{equation}
The variation of $\Snn$ with $k$ and $\omega$ for given values of transport coefficients, thermodynamic variables and the response functions are studied below. In the small $k$ limit with $\tilde{\alpha_0}\rightarrow 0,\tilde{\alpha_1}\rightarrow 0,\beta_0\rightarrow 0,\tilde{\beta_1}\rightarrow 0, \beta_2\rightarrow 0$ the results for Navier-Stokes hydrodynamics can be retrieved from the full expression given in the Appendix~\ref{appendix06_A}. We will see below that the variation of $\Snn$ with $\omega$ possesses with three peaks positioned at $\omega =0$ and  $\omega=\pm\omega_B$. The $\omega_B$ is a function of $k$, $c_{s}$ and other thermodynamic variables. The peak at $\omega=0$ is called the Rayleigh-peak and the doublet symmetrically situated at $\pm\omega_{B}$ are called Brillouin-peaks. The quantities $\Big(\frac{\pd P}{\pd n}\Big)_{T} ,\Big(\frac{\pd P}{\pd T}\Big)_{n},
\Big(\frac{\pd s}{\pd n}\Big)_{T}, \Big(\frac{\pd s}{\pd T}\Big)_{n}$ appearing in $\Snn$ can be evaluated in terms of relevant thermodynamic variables 
(see Appendix~\ref{appendix06_A}). 
%%%%%\section{Relation between $\Snn$ and scattering experiment and velocity of sound}


The width of this R-line is proportional to $\kappa/nC_{P}$~\cite{Stanley}, where $\kappa$ is the thermal conductivity, $C_{P}$ is the isobaric specific heat. Therefore, if the order of divergence for $\kappa$ is less compared to $C_{P}$, then we see a narrow R-line. Two Brillouin lines are positioned at $\omega_{B} \pm c_{s}k$ with respect to the frequency of the incident light source, which may give rise to the speed of sound. The finite width of the B-line can provide information about the finite value of transport coefficients such as shear, and bulk viscosities. Also, the ratio of intensities of R-line and B-line is expressed as $I_{R}/2I_{B}=C_{P}/C_{V}-1=\kappa_{T}/\kappa_{S}-1$ (where, $\kappa_{T}$ and $\kappa_{S}$ are the isothermal and adiabatic compressibilities respectively), called the Landau-Placzek ratio. Therefore, studying the structure factor could be a useful theoretical as well as experimental aspects to investigate the behaviour of the medium near the CEP.


%It is well-known that the correlation length diverges at the critical point. The effects of the divergence on the particle correlations, baryon number fluctuations, and correlations of density  fluctuations will have a better chance to get detected provided the fluctuations to survives the evolution of the hadronic phase. It is important to understand the correlations of those fluctuations theoretically for identifying signatures of the CEP in hadronic spectra. The effects of CEP on the evolution of QGP go as input to the hydrodynamic equations through the EoS and via the criticality of the transport coefficients and the thermodynamic response functions of the medium. 


In  condensed matter physics, the  $\Snn$ is measured by light scattering, utilizing the relation between intensity of light with density-density correlation~\cite{Stanley}. However,
any such direct measurement of the corresponding critical opalescence in QCD is still not achieved. The possibility of such measurement of QCD opalescence has been proposed by measuring jet quenching~\cite{Csorgo:2009wc}.
 
%%%%%%%%%%%%%
\section{Behaviour of $\Snn$  near the CEP}
\label{sec0603}
The  dynamical structure factor, $\Snn$ defined in Eq.~\eqref{eq0615} directly depends on the transport coefficients, such as $\eta, \zeta$ and  $\kappa$ as well as on the four thermodynamic partial derivatives, $\Big(\frac{\pd P}{\pd n}\Big)_{T} ,\Big(\frac{\pd P}{\pd T}\Big)_{n}$, $\Big(\frac{\pd s}{\pd n}\Big)_{T}$ and $\Big(\frac{\pd s}{\pd T}\Big)_{n}$. 
It depends on the coupling coefficients $(\tilde{\alpha_{0}}, \tilde{\alpha_{1}})$ and also on the relaxation coefficients $(\beta_{0}, \tilde{\beta_{1}}, \beta_{2})$ which pervade into $\Snn$ through the M\"{u}ller-Israel-Stewart (MIS) hydrodynamics. In $\Snn$, the effects of the CEP (through the EoS) is carried by the thermodynamic derivatives. It is well known that the behaviour of the transport coefficients and the thermodynamic response functions near the CEP are characterized by the critical exponents. As the CEP is approached, some of those quantities start to diverge. A thermodynamic variable, $f(t)$ near the CEP can be written as~\cite{Linda}:
\beqa
f(t)=Ae^{\lambda}(1+Be^{y}+...)~,
\label{eq0616}
\eeqa
where, $y> 0$, $t=(T-T_{c})/T_{c}$ is the reduced temperature. The critical exponent $\lambda$ can be defined as:
\beqa
\lambda=\lim_{t \to 0} \frac{ln f(t)}{ln (t)}~.
\label{eq0617}
\eeqa
$\lambda$ can be either positive or negative correspondingly $f(t)$ will vanish or diverge at the CEP. 
%For $\lambda =0$, 
%several possibilities are there and one of the possibilities 
%is logarithmic divergence as,
%\beqa
%f(r)=A|ln(r)|+B
%\label{eq33}
%\eeqa
%or its dependence on $r$ could be,
%\beqa
%f(r)=A+Br^{1/2}
%\label{eq34}
%\eeqa

Now the partial derivatives appearing in the expression for $\Snn$ can be expressed 
as (see Appendix~\ref{appendix06_A}): 
\beqa
\Big(\frac{\pd P}{\pd n}\Big)_{T}&=&\frac{1}{n_{0}\kappa_{T}}, \,\,\,\,\Big(\frac{\pd P}{\pd T}\Big)_{n}= \mu n_{0}c^{2}_{s}\alpha_{p}
\frac{C_{V}}{C_{P}}~, \nn\\
\Big(\frac{\pd s}{\pd n}\Big)_{T}&=&\frac{h_{0}c^{2}_{s}\alpha_{p}}{n_{0}\gamma},\,\,\,\, \Big(\frac{\pd s}{\pd T}\Big)_{n}=\frac{C_{V}}{T}~,
\label{eq0618}
\eeqa
where, $\kappa_{T}=\frac{1}{n_{0}}\Big(\frac{\pd n}{\pd P}\Big)_{T}$, is the isothermal compressibility, $\alpha_{P}=-\frac{1}{n_{0}}\Big(\frac{\pd n}{\pd T}\Big)_{P}$ is the volume  expansivity coefficient. The symbol $C_P$ and $C_V$ are the specific heats at constant pressure and volume respectively. The critical behaviour of various transport coefficients and thermodynamic response functions plays important roles to determine the strength of the signal of the presence of the CEP. We take the following $t$ dependence near CEP for the present purpose~\cite{KapustaChi,Guida,Rajagopal:1992qz}.
\beqa
&&\kappa_{T}=\kappa_{T}^{0}|t|^{-\gamma^\prime}, C_{V}=C_{0}|t|^{-\alpha}, C_{P}=\frac{\kappa_{0}T_{0}}{n_{0}}\Big(\frac{\pd P}{\pd T}\Big)^{2}_{n}|t|^{-\gamma^\prime},\nn\\
&& c^{2}_{s}=\frac{T_{0}}{n_{0}h_{0}C_{0}}\Big(\frac{\pd P}{\pd T}\Big)^{2}_{n}|t|^{\alpha}, 
\alpha_{P}=\kappa_{0}\Big(\frac{\pd P}{\pd T}\Big)_{n}|t|^{-\gamma^\prime}~, \nn\\
&& \eta=\eta_{0}|t|^{1+a_{\kappa}/2-\gamma^\prime}, \zeta= \zeta_{0}|t|^{-\alpha_{\zeta}}, 
\kappa=\kappa_{0}|t|^{-a_{\kappa}}~,
\label{eq0619}
\eeqa 
where, $\alpha, \gamma^\prime, a_{\zeta}, a_{\kappa}$ are the critical exponents. Here $\alpha=0.11$ and $\gamma^\prime= 1.2$. For a liquid-gas critical point, $a_{\zeta}=\nu d-\alpha=1.78$ (here, $d=3, \nu=0.63$), and $a_{\kappa}=0.63$. For consistency the partial derivative, $\left(\frac{\pd P}{\pd T}\right)_n$ has been calculated by using the
the EoS. The critical behaviours of the second-order coupling as well as the relaxation coefficients enter into the calculation through the thermodynamic variables which contain the effects of the CEP (see Ch.\ref{chapter4}).
%%%%%%%
%%%%%%%
%%%%%%
\section{Results and discussion}
\label{sec0603}
%%%%%%%%%%%%%%%%
Now we discuss the effects of the CEP through the EoS and subsequently on the $\Snn$. We have evaluated the $\Snn$ by including the effects of the CEP through: (i) the EoS and
(ii) the critical behaviour of various transport coefficients as discussed above. As mentioned in Ch.\ref{chapter4}, we assume that the CEP is located at $(T_{c}, \,\mu_{c})=(154,\, 367)\, MeV$ in the QCD phase diagram.


% Figure environment removed
Fig.\ref{fig0601}(a) shows a comparison of $\Snn$ for first-order and second-order hydrodynamics. The results show a difference in the $\Snn$ estimated by using the Naiver-Stokes (NS) and the M\"{u}ller-Israel-Stewart (MIS) hydrodynamics. Fig.~\ref{fig0601}(b) depicts the variation of $\Snn$ with $\omega$ when the system is away from CEP, which is represented by $r=0.2$. To examine the effects of the EoS, we have kept the transport coefficients at their lower bounds, $\eta/s, \zeta/s, \kappa T/s=1/4\pi$. The coupling and relaxation coefficients are evaluated from Appendix~\ref{appendix01_A}. We notice that there are three distinct peaks (red line). The central one is larger in magnitude and is called the R-peak which originates from the thermal fluctuations. The symmetric doublet about the R-peak is identified as the B-peaks. The B-peaks originate from pressure fluctuation at constant entropy, which is related to the propagation of the sound waves.  We find that the B-peaks are smaller in magnitude than the R-peaks. We exclude the effects of the EoS to investigate the effects of transport coefficients and the thermodynamic response functions only near the CEP via their critical exponents in evaluating $\Snn$ (depicted by the green line).
%%%%%%%%%%%%%%%%
% Figure environment removed
%%%%%%%


Closer to the CEP, however, the scenario changes drastically. The $\Snn$ near the CEP ($r=0.01$) reveals only the R-peak with the magnitude enhanced by more than an order of magnitude. The B-peaks do not appear due to the absorption of sound waves near the CEP (Fig.\ref{fig0602}). The vanishing of the B-peaks can be understood from the fact that in the leading order of Brillouin frequencies, it depends as $\omega_B\sim \pm c_s k$. The effects of EoS are seen to reduce the magnitude of the R-peak (red line). The speed of sound is determined by EoS, therefore, the position, as well as the height of the B-peaks will depend on the EoS.
%%%%%%%%
% Figure environment removed
%%%%%%%%%%%%%%%%%%%%%%%

The $\Snn$ is plotted as a function of $\omega$ for a smaller $k$ ($k=0.02$ fm$^{-1}$) in Fig.\ref{fig0603} when the effects of the EoS only is considered. The red (blue) line corresponds to results closer to the CEP with $r=0.01$ (away from CEP with $r=0.2$). We observe that for smaller $k$ value, the R-peak gets bigger as well as sharper (in frequency axis).  A comparison with results shown in Fig.\ref{fig0604} reveals that near the CEP, the B-peaks vanish and the R-peak gets sharper and bigger. 
%%%%%%%%%%%%%%%%%%%%%%%%%%%%%%%%%
% Figure environment removed
%%%%%%%%%%%%%%%%%%%%%%%%%%%%%%%%%%%%%%%%%%%%%%%%
% Figure environment removed
%%%%%%%%%%%%%%%%%%%%%%%%%%%%%%%%%%%%%%%%%%%%



Motivated by the results shown in Figs.~\ref{fig0602} and \ref{fig0603} {\it i.e.} by looking into the sensitivity of the results on the $k$ values we investigate the behaviour of $\Snn$ for different hydrodynamic modes ($k-$modes). The variation of $\Snn$ with $k$ and $\omega$ near the CEP is displayed in Fig.\ref{fig0604}. We see that for all values of $k$, the B-peaks vanish and the R-peak remains. The magnitude of the R-peak is maximum in the neighbourhood of $k\rightarrow 0$. The $\Snn$ is plotted with $\omega$ and $k$ in Fig.\ref{fig0605} when the system is away from the CEP. We observe that the B-peaks shift away from R-peak with the increase in $k$. This is better manifested in Fig.\ref{fig0606}, where the Brillouin frequency $\omega_B$ is plotted against $k$. A linear variation is obtained for given values of $T$ and $\mu$. The slope ($c_s$) of the line is found to be $c_s=0.532$ ({\it i.e.} $c^{2}_{s}=0.283$) close to the speed of sound $c^{2}_s=0.25$ obtained from Eq.~\ref{cs2} at the same value of temperature and chemical potential. 
%%%%
% Figure environment removed
%%%%%%%%%%%%%%%%%%%%%%%%%%%%%%%%%%%%%%%%%%%%%%%%

%%%%%%%%%%%%
Therefore, we observe that the B-peaks move toward the R-peak when the system approaches the CEP and ultimately merges with R-peak at the CEP. This indicates that the speed of propagation of all hydrodynamic modes disappears at the CEP. This is quite consistent with the finding that the speed of sound reaches its minimum at the QCD critical point. The location of the B-peaks  depend on $k$ when the system is away from the CEP, indicating that the different k-modes travel with different speeds in the fluid.  


%%%%%%%%%%%%%%Mach Cone %%%%%%%%%%%%
%Partons produced in RHIC-E with relatively low $p_T$, on subsequent scattering, produce a locally thermalized hot and dense medium of QGP, whereas, the high $p_T$ partons do not contribute in the medium formation, however, they pass through the medium as jets with associated radiated partons. The supersonic partons can produce perturbations in the medium. The response of the medium to such perturbations, are reflected on the spectra of the produced hadrons at the freeze-out hyper-surface. Specifically, the appearance of two maxima at $\Delta \phi=\pi \pm 1.2$ radian in the quenched away side jet or the double-hump in the correlation function of the jet, the structure is explained as the effect of the Mach cone produced due to hydrodynamic response to the perturbation created by jets~\cite{Solana:2004fdk}. The  Mach cone appears as double hump in the two particle correlation in the low momentum domain of associated particles. The CEP can suppress the double hump in the two particle correlation contrary to the other mechanisms ({\it e.g.} (i) deflection of away side jets due to strong asymmetric flow, (ii) Cherenkov radiation, (iii) radiation of gluons  at large  angle) which can create the double hump. These mechanisms have the ability to obscure the suppression due to the CEP  by producing double humps which makes the detection of the CEP hard. Keeping these issues in mind, however, we note the following points. (i) The double hump of the away side jets may originate due to the deflection by the strong asymmetric flow  in non-central collisions and third flow harmonics ($v_{3}$) due to initial state fluctuations~\cite{Wang2013,Cao2020}. However, if the system passes through the CEP, the flow will be highly suppressed and hence, the deflection too will be strongly reduced and hence the ability to create double hump by deflection will also weaken. (ii) The Cherenkov radiation of gluons by the away side jet propagating through the medium~\cite{STAR2003,Koch2006} may also give rise to double hump. However, Cherenkov radiation is unlikely to be responsible for the double hump because of the lack of observed momentum dependence of the location of the double peaks of associated particles. (iii) Gluon radiation at large angle by the away side jet propagating through the medium and consequently its change of direction can be responsible for double hump creation. The quantitative prediction of the Mach cone positions studied through three particle correlation~\cite{Betzthesis,BetzPRL,LiPRL}, and the momentum independence of the location of the double hump indicate that the observed double humps originate from Mach cone effects. The vanishing of the Mach cone will, therefore, strongly indicate the existence of the CEP.


%This nature of the mode-dependent speed of sound also signifies that 
%the attenuation of the Mach-cone in linear channel can not be interpreted 
%as the unique effect of the critical point. Rather a mode dependent speed of 
%propagation at points away from the CEP and vanishing speed of sound at the 
%CEP should be taken into account to properly interpret the attenuation of 
%Mach-cone structure in the particle correlations.



%However, at CEP, modes  for which $k \zeta<<1$(i.e., mean free path 
%is than spatial size of mode) are hydrodynamic~\cite{Stephanov:2017ghc}, 
%where $\zeta$ is the bulk viscosity which diverges as relaxation rate 
%diverges at critical point as ~$correlation length^3$. So the smaller
% wavelength modes or large k modes, which do not satisfy the condition, 
%will not follow hydrodynamic propagation at the CEP. For these modes 
%local equilibrium will not be achieved in the evolution time scale. 
%However, to have the effect of CEP uniquely in terms of propagation of 
%modes larger k modes should be looked at. So there will  not be plenty 
%of large k- hydrodynamic modes for such study, rather there will be 
%upper limit of k, for which effect of CEP can be understood uniquely 
%in terms stopping of hydrodynamic propagation of those modes at CEP. 

In non-relativistic limit, it is evident that the width of the R-peak is determined by $\kappa$, the widths of the $B$-peaks can be calculated in terms of $\kappa$, $\eta$, $\zeta$ and the ratio $C_P/C_V$~\cite{Stanley}. The integrated intensity of the R and B-peaks are determined by the ratio $C_V/C_P$, is connected with the ratio of isothermal to adiabatic 
compressibilities. Therefore, the spectral function contains useful information about the thermodynamic state of the system. The spectral function in QGP is calculated as a function of $\omega$ and $k$ which is not directly measurable through external probe in experiments due to transient nature of the extremely small system. In such a situation, one should depend on the indirect measurement of $\Snn$. Since the width and the integrated intensities of the R and B peaks are related to the various transport coefficients and thermodynamic response functions, thus, the determination of these coefficients by some other experimental measurements will help in constructing the spectral functions.  



Some kind of mapping between $k$ and some observables is required to gain insight on $\Snn$. The spectral function can be measured by using the bin by bin density fluctuations. Then $k$'s can be associated to the inverse angular separation of the bins as $\delta\phi\sim\frac{1}{kR}$, where $R$ is the radius of the freeze-out surface (which analogous to the analysis of temperature fluctuation of CMB radiation~\cite{Durrer2008}). Therefore, the analysis of the bin by bin density correlation in $\delta\phi \sim (kR)^{-1}$ in the transverse plane for different beam energies will carry the effects of the CEP.  However, for $k \sim 0.1fm^{-1}$, $\delta\phi$ is large for typical freeze-out radius in relativistic heavy-ion collider experiments (RHIC-E), ($\sim 5$ fm). This implies that the correlation can be observed for large angular separation. That is the pattern corresponding to these $k$ values will show up over the angular size $\delta\phi$.

%The angular separation which corresponds to the highest $k$-mode 
%for which hydrodynamic propagation speed even away from the CEP 
%vanishes, covers all the ranges in such cases.  This is because 
%$k\sim 0.01fm^{-1}$ corresponds to $\lambda$ which is larger  
%than the size of the system formed in such collisions.


In the condensed matter physics, the effects of the critical point are usually investigated by measuring the intensity of the light scattering from the system. In contrast to this, such external probes are unavailable for detecting the CEP in QCD matter. However, numerous possibilities for the detection of the CEP in QCD have been discussed in the literature. The phenomenon of critical opalescence is considered as a signal of large density fluctuations in a condensed matter system~\cite{Stanley}. The possibility of detecting such phenomenon of QCD opalescence can in principle be done by measuring the suppression of hadronic spectra in heavy-ion collisions ($R_{AA}$) as demonstrated in Ref.~\cite{Csorgo:2009wc}. The $R_{AA}$ can guide to estimate the opacity factor, $\K$ as: 
\begin{equation}
\K=-\frac{ln(R_{AA})}{R_{HBT}}~,
\end{equation}
where, $R_{HBT}$ is the Hanbury-Brown-Twiss(HBT) radius of the system. 


The Fourier coefficients of the azimuthal distributions of hadrons can be used to understand various properties of the matter produced in RHIC-E. The coefficient of $cos3\phi$ (triangular flow, $v_{3}$) sheds light on the quantum fluctuations of the initial state. Similarly, the coefficients of $cos2\phi$ (elliptic flow, $v_{2}$) is concerned with the EoS of the system. Near the CEP, the highest order of the harmonics to survive vary as $\sim 1/\lambda_{th}$, where $\lambda_{th}$ is some wavelength of the pressure perturbation (sound) which blows up at the CEP and hence all the flow harmonics will vanish~\cite{Hasan1} (discussed in Ch.\ref{chapter5}). However, the hadronic spectra contain contribution from all the space-time points 
{\it i.e.} from creation of the fireball to the stage of freeze out with all possible values of $T$ and $\mu$. Therefore, instead of vanishing, the suppression of flow harmonics may be observed.


The physics of the hadronic matter under extreme conditions of temperatures or densities and the QCD phase transition has been dubbed as the condensed matter physics of elementary particles~\cite{Rajagopal:2000wf} where the microscopic interaction is controlled by the non-abelian gauge theory whereas, the condensed matter physics is governed 
by the abelian gauge theory. Identification of some appropriate probes analogous to light in the abelian system will go a long way to unfold the physics of QCD matter near the CEP. It is shown~\cite{Son2004} that the dynamical universality class of the QCD critical point belong to the model H ~\cite{Hohenberg1977} with a combination (linear) of the chiral condensate 
and the baryon density as hydrodynamic mode. The evaluation of baryon number fluctuation by using the correlation length prevailed in ~\cite{Son2004} will be an interesting problem to address, however, this is beyond the scope of the present work. 
%%%
%%%
%%%