% Introduction3 JA:
\chapter[Phenomenology of critical point]{Phenomenology of critical point}
\label{chapter2}  
\section{Historical Overview}
\label{sec0201}
The study of critical phenomena was started with a paper by Thomas Andrews in 1869~\cite{Andrews2}. He plotted the variation of pressure ($P$) with volume ($V$) for different temperature ($T$) for carbon-di-oxide ($CO_{2}$). Such diagrams are commonly known as $P-V$ diagram. He found that gases could only be liquified when the gas were below a specific temperature, which are different for each gas. He labelled this characteristic specific temperature as the critical temperature ($T_{c}$) of the gas. 
% Figure environment removed
Fig. \ref{fig0201} depicts the plot of $P$ on the vertical axis and $V$ on the horizontal axis, where individual curves are plotted different constant temperature ($T$). The curves are mentioned as the isotherms of $P$ against $V$. The state of the $CO_2$ depends on the values of $P, V$ and $T$. The liquid state of the $CO_{2}$ only exists at or below the critical temperature. This specific behaviour observed by Andrews was finally explained in 1873 by J. D. Van der Waals with his equation of state (EoS)~\cite{Vanderwaals} for gases as: 
\begin{equation}
(P+\frac{a}{V^2})(V-b)=RT~,
\end{equation}
where, $a$ and $b$ are some constants, which may vary numerically form one gas to another, and $R$ is known as molar gas constant. The above equation describes the behaviour of gases above the critical temperature as well as the two coexisting phases below the $T_{c}$. Critical phenomena are universal and was precisely described by the laws of corresponding state, and it expresses the behaviour of a gas solely in terms of ratios of thermodynamic variables, $P,\, V,\, T$ to their values at the critical point (a point where first-order phase transition line stops) {\it{i.e.}} $P_{c},\, V_{c},\, T_{c}$. As Van der Waals EoS is not independent on the type of gases due to the variations of $a$ and $b$, thus the corrected equation was suggested as law of corresponding states. According to mathematical description a critical point is said to be a point of inflection. Thus to calculate the critical values, $P_{c}, V_{c}, T_{c}$ from the Van der Waals EoS, we need to put
\begin{eqnarray}
\Big(\frac{\partial P}{\partial V}\Big) = 0,\,\,\,\, \Big(\frac{\partial^2 P}{\partial V^2}\Big)= 0,
\end{eqnarray}
and calculated the critical coefficient as 
\begin{eqnarray}
T_c = \frac{8a}{27Rb}; \hspace{1cm} P_c=\frac{a}{27b^2}; \hspace{1cm} V_c=3b~.
\end{eqnarray}
Now the reduced temperature, pressure, volume was defined as
\beqa
P_r=\frac{P}{P_c}; \hspace{1cm} T_r=\frac{T}{T_c}; \hspace{1cm} V_r=\frac{V}{V_c}~.
\eeqa
Substituting these reduced variables into the Van der Waals EoS we landed into the equation of corresponding states
\beq
(P_r+\frac{3}{V_r^2})(V_r-\frac{1}{3})=\frac{8}{3}T_r~.
\eeq
This approach works remarkably well at temperatures between the normal boiling point and the critical point for many compounds but tends to break down near and below the triple-point temperature. At these temperatures the liquid is influenced more by the behaviour of the solid, which has not been successfully correlated by corresponding states methods.
\section{Water-vapour and ferromagnetic-paramagnetic critical point}
\label{sec0202}
%%%

  % Figure environment removed
%%%%
For a simple and clear view of a critical point, the common specific examples are the vapour-liquid critical point and the Curie point of ferromagnetic-paramagnetic transition. These are still the best-known and most studied critical points. Fig. \ref{fig0202} shows the schematic $P-T$ diagram of a pure substance. The commonly known phases (solid, liquid, and vapour) are separated by the explicit phase boundaries (along which two phases can coexist). At the triple point, all the phase boundaries intersect thus all three phases can coexist in equilibrium. However, along the vapour-liquid phase boundary, it cease to terminate at the critical point. This termination point where the latent heat vanishes and first-order phase transition line stops, is identified as the critical end point or simply the critical point. For water-vapour phase transition, the critical point is observed at $(T_{c}, P_c) \sim (647 \,K, 218 \,atm)$. Up to now, no such endpoint is observed for ice-water transition.



A magnetic material behaves as a ferromagnet at low temperatures, whereas, it shows paramagnetic nature at higher temperatures. Even without application of any external magnetic field, a ferromagnet possesses non-vanishing magnetization ($M$), but a paramagnetic material only possesses magnetization under a magnetic field. Therefore, if there is no external field, a magnetic material at high temperature can not have any magnetization. But when it cools down below the Curie temperature, $T_{c}$, the material suddenly possesses the magnetization. This phenomena is an example of a phase transition, which is explained by the Ising model.
%%
%%
\section{Phase transition and characterization of its order}
\label{sec0203}
Phase transitions are usually studied by recording the response of the system to external perturbations (disturbances). For example, the liquid-gas phase transition may be understood by the response of the change in volume with a change in pressure, which is represented as the isothermal compressibility. A first-order phase transition is a transition form one phase to another at which the first-order derivative of the free energy to a thermodynamic variable shows discontinuity at $T = T_c$. An example of such kind of phase transition is the water-ice or water-vapour transition at normal pressure. The water entirely changes to ice (vapour) at the freezing point (boiling point). A second-order phase transition exhibits a discontinuity in the second-order derivatives of the free energy, but it is continuous in the first order derivative of the free energy. More clearly, first-order or second-order phase transitions are accompanied by discontinuity in the derivative of free energy. Transition from one phase to other can be accompanied by another type of transition, known as crossover, where it does not possess any discontinuity in the free energy or in the derivatives of the free energy. 

%As an example of this kind of the transition can be understood from the paramagnetic-ferromagnetic phase transition, where the magnetization (first derivative of the free energy with respect to the magnetic field) increases continuously from zero when the temperature is below the Curie temperature. However, the magnetic susceptibility which is the second order derivative of the free energy with respect to the magnetic field is discontinuous at $T_{c}$. Therefore the phase transition at the critical point is second-order.

In liquid-gas phase transition (first-order), the equilibrium pressure and temperature in $P-T$ diagram is governed by the Clausius-Clapeyron’s equation, where latent heat is associated with the phase transition. Whereas, in the second-order transition due to the fact of unchanged volume and entropy, the Clausius-Clapeyron's equation is not applicable, and this leads to Ehrenfest’s equation. Therefore, association of latent heat is missing in second-order phase transition.
%%
%%
%%
\section{Order parameters and critical behaviour}
\label{sec0204}
The phase transition for any thermodynamic system is usually accompanied by a change in the symmetry of the system. For a magnetic material, the two phases (the paramagnetic and the ferromagnetic) have different symmetries. Beyond the critical temperature, there is vanishing magnetization, which implies that the system is rotationally invariant. whereas, below the critical temperature, the non-vanishing magnetization infers that there is a preferred direction of the spin of the electrons. Therefore, the rotational symmetry for the system is explicitly broken, and since the symmetry is absent in one of the phases, we need a parameter in order to make distinction between the two phases. Such a parameter is called the order parameter. 


%By definition, there is more than one equilibrium phase on a coexistence line. As mentioned above, the order parameter is a thermodynamic function that is different in each phase, and hence can be used to distinguish two different phases. Usually order parameter is present in one of the phases and disappears at the critical point. 



Thus, order parameter plays a crucial role in phase transition. By looking into the behaviour of the order parameter, we can infer about the system. For paramagnetic-ferromagnetic system, the order parameter is the magnetization. For water-vapour transition, the order parameter is the density of the system, and for the QCD deconfinement phase transition, it is the Polyakov loop~\cite{Fukushima:2017csk}. 

%At the critical point, the order parameter fluctuates over a long wavelength, causing phenomena like Critical Opalescence. 
%%
%%
\subsection{Critical Opalescence}
Critical opalescence~\cite{Stanley} is a beautiful light scattering phenomenon, which was elegantly explained by Einstein in 1910. The phenomena was first observed by Thomas Andrews for $CO_2$ gas, and since then this phenomena has been observed in many other systems. When $CO_2$ passes through the critical point, the order parameter (here, density of $CO_2$) fluctuates over the whole dimension of the gas, which is comparable to the wavelength of light. Thus light can not pass through the system, causing the transparent liquid to appear cloudy as shown in Fig.\ref{fig0204}. 
% Figure environment removed

Now if we consider the magnetic system where the order parameter is the magnetization of the system. A ferromagnetic material possesses magnetization due to the alignment of the spins in some preferred direction. When the ferromagnetic material is being heated up and reaches the critical temperature, the magnetization fluctuates over the whole volume of the ferromagnet. This situation can be thought of as when $T=T_c$, a spin flips from its alignment and due to the fluctuation this flipping will cause the other spins to de-align and the magnetization disappears. 


The critical opalescence is also studied in case of the QCD critical point~\cite{Csorgo:2009wc,Antoniou:2006zb,Kunihiro:2009hv}, will be discussed in preceding chapters.


%A similar situation will occur in QCD critical temperature also. To know the critical opalescence for this case we need to study the Polyakov loop and $Z_3$ symmetry.
%%
%%
\section{Scaling laws, critical exponents and universality hypothesis}
\label{sec0205}
Renormalization group for scalar field theory is capable to predict the behaviour of correlations near the critical point of a thermodynamic system. It is predicted that the correlation length ($\xi$) will diverge as the critical point is approached. This follows a relation
\beqa
\xi \sim \big|T-T_{c}\big|^{-\nu},
\label{scaling_law}
\eeqa
where, $T_{c}$ is the critical temperature and $\nu$ is some critical exponent. This is known as scaling law. Statistical mechanics interprets that it is directly measurable in the realistic case of three dimensions. The value of $\nu$ is closed to $0.5$, suggested in Landau approximation~\cite{Landau_statistical_mechanics}, but may differ by some systematic corrections.


It is familiar that the thermodynamic properties at the critical point can be portrayed by critical exponents. Thus, critical exponents describe the behaviour of the singularity of the thermodynamic quantities at the critical point. To illustrate this, firstly we need to introduce the quantity known as reduced temperature
\beq
t=\frac{T-T_c}{T_c},
\eeq
such that the scaling law in Eq.\eqref{scaling_law}, can be written as
\beqa
\xi \sim |t|^{-\nu},
\eeqa


Near the critical point, the behaviour of the thermodynamic quantities is defined by a number of additional exponents. The specific heat at fixed external magnetic field ($h$) is defined as
\beqa
C_{h}\sim |t|^{-\alpha}\,.
\eeqa
Since, the ordering of spin sets in at $t=0$, the magnetization $M(t, h)$ at $h=0$ goes to zero at $t \to 0$ from below
\beqa
M\sim |t|^{\beta}\,.
\eeqa
Now even at $t=0$, finite magnetization remains at $h= 0$. Thus,
\beqa
M \sim h^{1/\delta}\,.
\eeqa
And as the magnetic susceptibility diverges at the critical point, it can be represented as
\beqa
\chi \propto |t|^{-\gamma} \,.
\eeqa
Here, $\alpha, \beta, \delta$ and $\gamma$ are the critical exponents, and are proficiently measured experimentally for a variety of thermodynamic systems.


One of the astounding attributes of critical behaviour is that it is universal, meaning that the properties of the thermodynamic variables can be understood without considering the microscopic details of the system. All the systems that behave in similar fashion belong to the same universality class. This is because of the divergence of the correlation lengths near the critical point, where the large wavelength feature dictates the dynamics of the system. It is experimentally discovered that a large variety of systems share the same critical exponents. The variety of systems that share the same sets of critical exponents are categorized into the same universality class. The property of any universality class is a rigorous prediction of the renormalization group theory of phase transitions. It is evidently found that the thermodynamic properties near the critical point depend on few features such as dimensionality and symmetry of the system.


Calculation shows that 3D Ising model and QCD belongs to the $\mathcal{O}(4)$ universality class~\cite{Halasz:1998qr,BERGES1999215}. Thus, the calculations of the 3D Ising model can be mapped onto the QCD phase diagram. This feature of universality will be used to construct the EoS which contains the effects of the critical point.
%%
%%
\section{Exploring the QCD critical point and its signatures}
\label{sec0206}
As discussed earlier in Ch.\ref{chapter1}, the first principle calculations {\textit{i.e.}} lattice simulations are not possible at finite $\mu_{B}$, the exact location of the critical point is not known so far. But the model calculations predicts the existence the CEP at finite $\mu_{B}$ region of the phase diagram, and thus raises the probability of discovering the CEP in such experiments~\cite{PhysRevLett.81.4816}. In Ref. \cite{PhysRevLett.81.4816}, the signatures were proposed on the basis of the fact that the CEP is an authentic thermodynamic singularity where different susceptibilities must diverge and the order parameter fluctuates over long wavelengths. All the resulting signatures due to the effect of the critical point share a usual property that they possess non-monotonic behaviour when passes through (near) the CEP.  The signatures get strengthen and then weaken again when the CEP is approached and then passed. After a collision, the evolution of the produced thermalized QGP follows a path of an isentropic trajectory that passes through (or near) the CEP is infrequent. Thus the imprint on the QGP due to the effects of the CEP through the critical fluctuations is less probable. Fortunately, long ago, it was shown that it is surprisingly easy to pass through (near) the critical region from a broad range of initial parameters~\cite{CSERNAI1986223}. In addition, let the QGP fireball pass through the critical region, the dynamics of critical fluctuations will be affected by an effect, known as `critical slowing-down'~\cite{Berdnikov:1999ph}, which allows the system to stay longer time near the CEP. This has both advantages and disadvantages to detect the CEP. If critical fluctuations induced by the CEP relax very quickly to thermodynamic equilibrium, the induced critical dynamics might have been washed away by the time it reaches the chemical freeze-out. However, the evolution of fluctuations is slowed down in the vicinity of the critical point, some signals of critical fluctuations may survive until freeze-out.


The existence of the CEP in the way of evolution process leads to a phenomena which is referred to as the `focusing' effect of trajectories towards the CEP~\cite{Nonaka}. This effect leniently permits the analysis of critical phenomena in the vicinity of the CEP that may not require a fine-tuned collision energy. Another feature of the focusing effect arises through the diverging nature of the isochoric specific heat capacity $\big(C_{V}=(\frac{\pd E}{\pd T})\big)$ at the CEP. As a result, the trajectories which pass near (through) the CEP will linger for a longer time near the vicinity of the CEP. This makes the system to freeze-out at a temperature fairly close to the critical temperature. So, while scanning in collision energies, we expect to find certain deviations in particle spectra if the isentropic trajectory passes by the vicinity of the critical point. 


From the above discussion, it is inferred that the final state of the heavy-ion collisions (HICs) will be affected by the critical phenomena. Therefore, isentropic trajectories of events passing through (away) the CEP can be distinguished from the analysis of the particle spectra. For that event-by-event analysis of suitable observables could be a standard practice to get the effects of the CEP~\cite{PhysRevD.60.114028}. Another important point to note is that the critical signatures which is directly reflected in thermodynamic properties of the system near the CEP are not very subtle to the finer details of the evolution. It is argued that the event-by-event fluctuations in both $T$ and $\mu_{B}$ should be anomalously small as the system passes near the critical point because event-by-event fluctuations of $T$ can be related to $C_{V}$ at freeze-out~\cite{Stodolsky_PhysRevLett.75.1044,Shuryak_1998} as
\beqa
\Big(\frac{\Delta T}{T}\Big)^{2}=\frac{1}{C_{V}}\,.
\eeqa
Now as the CEP is approached, $C_{V}$ diverges, thus the fluctuation in $T$ is suppressed. Also the fluctuation in $\mu_{B}$ is suppressed via the divergence of the other susceptibilities.



The event-by-event analysis is applied to the fluctuations of pion's multiplicity and momentum distributions. There are two promising reasons why the pions are the most sensitive to critical fluctuations. First, the pions are the lightest hadrons observed in a relativistic heavy-ion collisions, are the most abundant hadron. The second and very crucial reason is that pions couple strongly to the fluctuations of the sigma field (proportional to the magnitude of the chiral condensate). If the freeze-out occurs near the CEP, then it was predicted for the soft pions to follow a large non-thermal multiplicity as well as an enhanced event-by-event fluctuations~\cite{PhysRevD.60.114028}. Also due to the straggling of the system near the CEP, the lifetime of the QGP may get enhanced, causing the suppression of $J/\psi$~\cite{1998}.


So far the discussion on the signatures of detecting the CEP is based on the assumption that fluctuations are enhanced near the CEP. Unfortunately, there are several reasons to doubt the arguments~\cite{PhysRevLett.101.122302,NPA2009,Luo2009}. In Ref.~\cite{PhysRevLett.101.122302}, transverse velocity dependence $(\beta_{T})$ of the anti-proton to proton ratio ($\overline{p}/p$) is considered as a prominent signature of the CEP. The arguments are established on the fact that the CEP acts as an attractor to the hydrodynamic isentropic trajectories in the phase diagram~\cite{Nonaka}. It is shown that the evolution of the $\overline{p}/{p}$ ratio following a isentropic trajectory between the phase boundary in the QCD phase diagram to the point of chemical freeze-out is highly dependent on the existence of the CEP. With the presence of the CEP, the isentropic trajectory is deformed, and the ratio $\overline{p}/{p}$ is enhanced when approaching from the CEP to the chemical freeze-out point.
%%


Concerning the study of the critical point, the fluctuations and correlations have been considered to be sensitive observables to explore the phase structure of the QCD matter~\cite{AsakawaPhysRevLett.85.2072,Jeon2003eventbyevent,Koch2008hadronic}. They have an explicit physical interpretation for a system in thermal equilibrium and can provide essential information about the effective degrees of freedom. The crucial feature of a critical point driven by the divergence (in the ideal thermalized limit) of $\xi$ and the large scale magnitude of the fluctuations. The simplest and easiest measure of the fluctuations are the variances ($\sigma^{2}$) of the event-by-event observables namely the multiplicities or mean transverse momenta of particles. For that, a measurable and directly related observable, the net proton number cumulants can be obtained from protons and anti-protons multiplicities on an event-by-event basis, which is believed to be sensitive to the QCD critical point. The baryon susceptibilities, $\chi_{i}^{B}$, can be calculated from the EoS can be related to the cumulants, $C_{i}$ as~\cite{Sourendu2011,Nahrgang2016,Luo2017,Bzdak2020}:
\beqa
\chi_{i}^{B}=\frac{\pd^{i}(P/T^{4})}{\pd (\mu/T)^{i}}=\frac{1}{VT^{3}}C_{i}\,
\label{barsusc}
\eeqa
As the pressure, volume or temperature can fluctuate, one must cancel them out by taking ratios between the susceptibilities or the cumulants to obtain
\beqa
\frac{\sigma^{2}}{M}=\frac{\chi_{2}^{B}}{\chi_{1}^{B}}=\frac{C_{2}}{C_{1}}\,\,,\,\,\,\,\,\, S\sigma=\frac{\chi_{3}^{B}}{\chi_{2}^{B}}=\frac{C_{2}}{C_{2}}\,\,,\,\,\,\,\, \kappa\sigma^{2}=\frac{\chi_{4}^{B}}{\chi_{2}^{B}}=\frac{C_{4}}{C_{2}}\,,
\eeqa
where, $M, \sigma^{2}, S, \kappa$ are the moments of net baryon including mean, variance, skewness and kurtosis respectively, and can be defined in terms of baryon number $N$ as:
\beqa
M=<N>, \sigma^{2}={<N^{2}>-<N>^{2}}, S=<(\delta N)^{2}>,  \kappa=\frac{<(\delta N)^{4}>}{<(\delta N)^{2}>^{2}}-3\,.
\eeqa
The skewness and kurtosis describe how the shape of a probability distribution deviates from the Gaussian distribution, where the first one gives the asymmetry and the later
the `tailednes' of the distribution. They are also called the non-Gaussian fluctuations.


Now, the critical contribution to these variances diverges as $\xi^{2}$ and would demonstrate in a non-monotonic behaviour when the CEP is approached and passed in the system created in the beam energy scan program~\cite{PhysRevLett.81.4816,PhysRevD.60.114028}. But in a real situation of HICs, due to finite size effect and finite time effect (critical slowing down), the correlation length can grow up to 2-3 fm~\cite{Berdnikov:1999ph}, whereas, in absence of the CEP, the correlation length is of the order of $0.5-1$ fm, when the system is away from the CEP. Therefore, the contribution to the variance may not be a significant effect to detect the CEP. However, non-Gaussian, higher moments of the fluctuations are more sensitive on $\xi$~\cite{Stephanov:2008qz,Stephanov:2011pb}. For example, the third moment (skewness) depends as $\xi^{9/2}$, and the fourth moment (kurtosis) grows as $\xi^{7}$ near the CEP, making it a more reliable quantity to be reflected on the multiplicity fluctuations of pions and protons. Since the protons are much heavier in mass than the pions, the critical effects on their multiplicity fluctuations are expected to be more stronger in protons than for pions~\cite{Stephanov:2008qz}, and thus physicists are more interested in proton multiplicity fluctuations when searching for the CEP experimentally~\cite{Bzdak2020,Adam2021,Abdallah2021}.

It is also pointed out that the sign of the fourth moment may become negative as the critical point is approached from the side of the crossover region of the QCD phase transition. As a measurable and directly related observable, the net proton number cumulants can be obtained from protons and anti-protons multiplicities on an event-by-event basis.


%We, in this dissertation, are interested in searching of the various signatures as the system passes through the critical region, will be discussed in the preceding chapters.