% Introduction3 JA:
\chapter[Introduction]{Introduction}
\label{chapter1}  
\section{The elementary particles and interactions in Standard Model}
\label{sec0101}
Human beings are always fascinated to think what's next? The hunger to know even beyond our limitations is what drives us towards discovering new things. By looking up the stars, we could discover astronomy. Starting from our own Milky way galaxy (size, $d\sim 10^{16}$m) to stars in galaxy ($d\sim 10^{9}$m) to planets (earth with $d\sim 10^{6}$m) to human ($d\sim 2$m) to atoms ($d\sim 10^{-10}$m) to nucleus ($d\sim 10^{-14}$m) to nucleons ($d\sim 10^{-15}$m) to partons (constituents of proton, $d\sim 10^{-16}$m), we have come a long way to find out the basic ingredients of matter. Since 1930s, physicists have put efforts in discovering an exceptional insight into the basic structure of matter: everything in the universe is consisting of a few basic building blocks called fundamental particles, which are governed by four fundamental forces which are known to be the strong force, electromagnetic force, weak force and the gravitational force. So far with our best understanding, we know how the fundamental particles interact with three known forces (except the gravity) belongs to the Standard Model~\cite{Oerter2006} of particle physics. The theory of the Standard Model is described by the Quantum Field Theory (QFT). The Standard Model has potential to explain vast amount of experimental results and could precisely predict wide variety of phenomenon. It has become established as a well-tested physics theory over time. The Standard Model, as presently drawn up, has sixty one ($61$) elementary particles (including anti-particles), are able to form composite particles, attributing for the hundreds of different species of particles, have been discovered since the 1960s.
%%%%%
%%%%
\begin{table}[h]
\centering
% Figure removed	
\caption{Elementary particles, in total turns out to be 61.}
\label{table1}
\end{table}


Elementary particles~\cite{Griffiths2008} are arranged in two groups, called quarks and leptons. Each group are subdivided into three `generations'. The most stable and lightest particles are kept in the first generation, whereas the less stable and heavier particles belong to both the second and third generations. Therefore, the universe with stable elements are consisting of particles that belong to the first generation because the heavier ones rapidly decays into the lightest stable ones. Amongst six quarks (flavours), the `up quark' and the `down quark' make up the first generation. The `charm quark' and `strange quark, and then the `top quark' and `bottom (or beauty) quark are kept in the second and the third generations respectively. A quark may also come in three different `colours' and a colourless object may be created by mixing the quarks or a quark and an anti-quark. Three leptons are similarly organized in three generations: the `electron' in the first generation, the `muon' in the second generation, and the `tauon' in the third generation. Each leptons have their corresponding neutrinos, which are electrically neutral with very little (or zero) mass. 

%Among four basic interactions, we are interested in the strong interaction. There were two approaches for strongly interacting matter at that time. The first approach (model), called Bootstrap~\cite{Hagedorn1965,Rafelski2015} was based on the hypothesis that hadrons are composites of one another hadrons, infers, lighter hadron species together form heavier hadron species. This model also hypothesized that after a limiting value of temperature, known as the Hagedorn temperature ($T_{H}=170-180$ MeV), the system would lead to the creation of more and more massive hadron species on any subsequent heating. The second model approach, the quark model~\cite{Gell-Mann1964,Zweig1964}, speculated that the hadrons are composite particles of the quarks. Few people even predicted that there could be `leftover quarks' from the Big Bang roaming freely through the universe~\cite{Zeldovich1969}. Unfortunately, the best theories for the strong interaction were not able at that time to deal with temperature beyond $100$ MeV. 

Among the four basic interactions, we are concerned with the strong interaction here. The theory of strong interaction is formulated in Quantum Chromodynamics (QCD). It is kind of a quantum field theory (QFT) which belongs to non-abelian gauge theory, with symmetry group SU(3). It characterizes the interaction between the quarks with the help of the gauge boson: gluons. The dynamics of the quarks and gluons (together called as partons) are dictated by the QCD Lagrangian is written as
\beqa
\mathcal{L_{QCD}}= \sum_{f}\, \bar{\psi_{i,f}}\big(i\gamma^{\mu}D_{\mu}-m\big)\,\psi_{i,f}-\frac{1}{4}G^{\mu \nu}_{a}G^{a}_{\mu\nu}
\eeqa
where, $D_{\mu}=\partial_{\mu}-i\,g_{s}\,T_{a}\,A^{a}_{\mu}$ is the gauge covariant derivative and $g_{s}$ is the strong coupling constant. $A^{a}_{\mu}$ 's are the non-Abelian gauge fields with $a=1,\,2,\,...., \,N_{c}^{2}-1$ being the color representation. $T_{a}$'s are the generators of the SU($N_{c}$) group satisfying the group algebra $[T_{a},\,T_{b}]=i\,f_{abc}T_{c}$. $\psi_{i,f}$'s are the Dirac spinor for the quark fields. The index $i$ represents the quark color for $N_{c}=3$ (red, green, blue) and the index $f$ refer to the quark/anti-quark flavor, $N_{f}=6$ (up, down, strange, charm, bottom, top). For $N_{c}=3$, we have $N_{c}^{2}-1=8$ gluon fields. The $\gamma^{\mu}$ are the Dirac matrices connecting the spinor representation to the vector representation of the Lorentz group. The symbol $G^{\mu\nu}_{a}$ is gauge invariant gluon field tensor, like the term $F^{\mu\nu}$ (photon field tensor) in Quantum Electrodynamics (QED), is given by
\beqa
G^{a}_{\mu\nu}=\pd_{\mu}A^{a}_{\nu}-\pd_{\nu}A^{a}_{\mu}+g_{s}\,f_{abc}A^{b}_{\mu}A^{c}_{\nu}
\eeqa
The third term in the right hand side of the above equation represents the self interaction between gluons.


 The QCD exhibits two salient features:
 
 a) {\bf{Color confinement}}~\cite{PhysRevD.10.2445,Mandelstam,tHooft:1975krp}: It is the phenomenon that color-charged particles {\it{i.e.}} quarks and gluons cannot be isolated at low energies, therefore in normal condition direct observation of those partons are not possible. The partons stay together inside the hadrons, and can not be separated from their parent hadron without producing new hadrons.
 
 b) {\bf{Asymptotic freedom}}~\cite{PhysRev.96.191,Gross1973,Politzer1973,CollinsandPerry1975,Cabibbo1975}: This phenomena signifies that the interaction between quarks gets weaker as the distance between them gets smaller, and quarks are asymptotically free. This is called asymptotic freedom. This indicates that at high energies, the partons interact weakly.
 
 
% Therefore, at high energies when hadrons are brought together, the distance between the quarks are eventually decreased, and after a certain threshold distance, the quarks would no longer be confined inside hadrons. Authors in Ref.~\cite{CollinsandPerry1975} termed this state as superdense matter (which is characterized by density above the nuclear density), and finally concluded that matter in a superdense state would consist of a `quark soup'. Authors in Ref. ~\cite{Cabibbo1975} went on to reinterpret this prediction in terms of a phase transition. 
 
 %They suggested that the observed exponential spectrum (in Fig.\ref{fig0102}) is connected to the existence of a different phase of the vacuum in which quarks are not confined.
%%
%%


After the discovery of asymptotic freedom in QCD, the first successful applications were to deep inelastic scattering (DIS)~\cite{DIS} to probe the structure of hadrons (particularly the baryons, such as protons and neutrons), using beams of electrons, muons and neutrinos. It provided the first convincing evidence of the existence of quarks.

%, till then it had been considered to be a purely mathematical phenomenon.

From the idea of confinement and asymptotic freedom, it can be concluded that at very high energies, when the quarks gets very closer to each other, the interaction between them gets weaker. Just after the discovery of the asymptotic freedom, authors in Ref.\cite{CollinsandPerry1975} had shown that at very high density, the nuclear matter will dissolve to its basic constituents, {\it{i.e.}} quarks and gluons. Subsequently, studies~\cite{Shuryak1978,Shuryak1980} confirmed that at very high temperature and density the properties of nuclear matter is governed by quarks and gluons, not by the hadrons. A thermalized state of quarks and gluons is called quark gluon plasma (QGP), which may have existed after a few microsecond of the Big Bang~\cite{Yagi:2005yb}. But on further expansion of the universe, the partons get confined within the hadrons and till then no free quarks and gluons are observed in the nature. However, such a deconfined state as QGP is believed to exist in the core of the neutron stars~\cite{Shuryak1980,Nature2020}. 
 
 
 In laboratory, the QGP can be created by colliding nuclei at relativistic energies. The magnitude of density and temperature required to form a deconfined state of quarks and gluons can be achieved by colliding heavy ions at ultra-relativistic high energies. The typical value of temperature for transition from hadronic matter to QGP is $\sim 10^{12}\,K$, and density $\gtrsim 3-4$ times the normal nuclear matter density. At the high temperature and density, the hadrons overlap substantially, and as a result, the quarks and gluons confined within a hadronic volume get deconfined and roam throughout the nuclear volume in a system formed in heavy-ion collisions. The QGP state produced in nuclear collision is very transient in nature and having a small volume. Therefore, the detection of the QGP is extremely challenging. However, there are many studies which confirms that QGP can be created by colliding heavy ions at ultra-relativistic high energies~\cite{PhysRevD.31.545,SHURYAK1978150,SINHA198391,PhysRevLett.58.101,ALAM1996243,ALAM2000159,Niida:2021wut,SINGH1993147,KOCH1986167,Gazdzicki:1996pk,PhysRevD.51.3408,PhysRevC.51.1444,TIWARI1997225,Muller:1994rb,Bjorken:1982tu,Gyulassy1990,Gyulassy1992,Matsui:1986dk,Kluberg:2009wc,NA50:2004sgj,Wong:1997rm}.


%At this point, the quarks will no longer be able to reside within the hadrons and may be liberated to form a medium made up of quarks and gluons. This medium is known as the Quark Gluon Plasma (QGP). Model calculations indicate that beyond a critical energy density $\epsilon_c \sim 1GeV/fm^3$ or a critical temperature $T_c\sim 200 MeV$, matter can only exist as QGP~\cite{Shuryak1978}. Therefore, the liberations of those partons from the parent hadron can be thought of as a phase transition.




%The study of the QGP needs the understanding of the strong interaction. At the time, there were two approaches for strongly interacting matter. The first approach (model), called Bootstrap~\cite{Hagedorn1965,Rafelski2015} was based on the hypothesis that hadrons are composites of one another hadrons, infers, lighter hadron species together form heavier hadron species. This model also hypothesized that after a limiting value of temperature, known as the Hagedorn temperature ($T_{H}=170-180$ MeV), the system would lead to the creation of more and more massive hadron species on any subsequent heating. The second model approach, the quark model~\cite{Gell-Mann1964,Zweig1964}, speculated that the hadrons are composite particles of the quarks. Few people even predicted that there could be `leftover quarks' from the Big Bang roaming freely through the universe~\cite{Zeldovich1969}. Unfortunately, the best theories for the strong interaction were not able at that time to deal with temperature beyond $100$ MeV. 


%The first breakthrough was achieved from the work by David Gross and Frank Wilczek~\cite{Gross1973}, and Politzer~\cite{Politzer1973} with the discovery of asymptotic freedom in the theory of elementary particle interactions. Two groups, Collins-Perry~\cite{CollinsandPerry1975}, and Cabibbo-Parisi~\cite{Cabibbo1975} could explain the theory of the asymptotic freedom. In Ref.~\cite{CollinsandPerry1975}, they realized that, since the interaction between quarks gets weaker as the distance between them gets smaller, and after a certain threshold distance, the quarks would no longer be confined inside hadrons. They termed this state as superdense matter, which characterized any density above the nuclear density, and finally concluded that matter in a superdense state would consist of a `quark soup'. Cabibbo and Parisi went on to reinterpret this prediction in terms of a phase transition. They suggested that the observed exponential spectrum (in Fig.\ref{fig0102}) is connected to the existence of a different phase of the vacuum in which quarks are not confined.
%%



%%%%



\section{Signatures of QGP in Heavy-ion-collision}
\label{sec0103}
The official announcement of CERN on Feb 10, 2010, stated-``compelling evidence now exists for the formation of a new state of matter at energy densities about 20 times larger than that the centre of atomic nuclei and temperature about 100000 times higher than in the centre of the sun. This state exhibits characteristic properties which can not be understood with conventional hadronic dynamics but which are qualitatively consistent with expectations from the formation of a state of matter in which quarks and gluons no longer feel the constraint of color confinement.''

There are numerous signatures which confirms that a medium of QGP is created by a collision. In this dissertation, we are not going to discuss the details of the signatures confirming the formation of QGP. However, one may look into the details of such signatures, {\it{e.g}} 

i) Direct photons and dileptons~~\cite{PhysRevD.31.545,SHURYAK1978150,SINHA198391,PhysRevLett.58.101,ALAM1996243,ALAM2000159,Niida:2021wut,SINGH1993147}~,

ii) Strangeness enhancement~\cite{KOCH1986167,Gazdzicki:1996pk,PhysRevD.51.3408,PhysRevC.51.1444,TIWARI1997225,Muller:1994rb}~,

iii) Jet quenching~\cite{Bjorken:1982tu,Gyulassy1990,Gyulassy1992}~,

iv) $J/\psi$ suppression~\cite{Matsui:1986dk,Kluberg:2009wc,NA50:2004sgj,Wong:1997rm,SINGH1993147}~,

v) and many more.

%\subsection{Direct photons and dileptons}
%There are numerous works for the search of evidences of the formation of the QGP~\cite{Niida:2021wut,SINGH1993147}. External probes are often used to see the inside matter and study of its properties, e.g. the internal structure of hadrons by deep inelastic scattering. In heavy-ion collisions, the life time of the QGP is extremely short ($\sim 10-15$ fm/c) and thus external probe is literally impossible. Therefore, photons and di-leptons are considered as ``penetrating'' probes of the QGP because they are produced in the early thermal stage of the collision and they are not affected by the subsequent hadronization of the system~\cite{PhysRevD.31.545,SHURYAK1978150,SINHA198391,PhysRevLett.58.101,ALAM1996243,ALAM2000159,SINGH1993147}. They interact by electromagnetic interactions and their mean free paths are way larger than the size of the collisional volume created in any nuclear collision. As a result, high-energy photons and di-leptons produced in the interior of the plasma usually carry information directly from wherever they were formed and they leave the fireball without any change in their momentum distributions.


%\subsection{Strangeness enhancement}
%Another signature for QGP formation is the ``strangeness enhancement''~\cite{KOCH1986167,Gazdzicki:1996pk,PhysRevD.51.3408,PhysRevC.51.1444,TIWARI1997225,Muller:1994rb}. Unlike up and down quarks, heavier quark flavors like strange and charm predominantly approach towards chemical equilibrium in a dynamic evolution process. In the calculation for the strangeness abundance in the ultra-relativistic heavy-ion collision, one usually assumes that once the QGP has been produced, the resulting hadronic system does not get sufficient time to achieve thermal and chemical equilibrium. At a temperature $T\sim 150$ MeV which is expected to be achieved in the present day experiments, it is easy to produce $s\bar{s}$ pairs in QGP while strange hadron productions are usually suppressed. Consequently the approach to the chemical equilibrium in a hadron gas is rather slow. We thus conclude that the chemical composition existing during the early stage of the hadronization of the QGP suffers a negligible change and the strangeness enhancement achieved in the QGP still survives during the hadronization. Therefore, the hadronic spectra resulting from QGP differ much from those coming from an equilibrated hadron gas (HG) without any QGP formation. 

%\subsection{Jet quenching}
%In a heavy-ion collision, the survival of the system is extremely short ($\sim 10-15$ fm), limits the use of literally external probe. Therefore, effective way to characterise the medium is to use the internal probes. After the collision, the high energetic partons interact amongst themselves (Parton-parton scattering).  The scattered partons have very high transverse momentum and pass over the QGP by losing their energies. Since these partons are not able to exist on their own, they fragment into a spray of hadrons, which are identified as jet.


%In the medium, like the energy loss in QED (Quantum Electrodynamics), the energy loss of partons happens by collisional and radiative processes. The primary difference between the QED and the QCD is that the interacting gauge boson (here gluons) interact with themselves unlike whereas photons itself in QED do not interact. Collisional energy loss happens due to elastic scatterings between primary parton and a parton from the medium, while the radiative energy loss takes place due to gluon radiation.

%So, in presence of the medium, energy loss of the partons takes place which causes the dissipation of the jets in the medium, called the jet quenching~\cite{Bjorken:1982tu,Gyulassy1990,Gyulassy1992}. The jet quenching is measured by  so-called nuclear modification factor $R_{AA}$, defined as the ratio of normalized single particle yields in $p+p$ and $A+A$ collisions as
%\beqa
%R_{AA}=\frac{d^{2}N_{AA}/(dy\,dp_{T})}{<N_{coll}>d^{2}N_{pp}/(dy\,dp_{T})}
%\eeqa
%where, $<N_{coll}>$ is the average number of binary nucleon-nucleon collisions. One can say that jet quenching happened, when the factor $R_{AA}<1.$ The jets are always emitted back-to-back to preserve momentum conservation. The jet created near the edge of the medium has to travel a very short distance in the medium (known as trigger), whereas the other will traverse a long distance (away side jet) in the medium, and thus quenched more.

%\subsection{$J/\psi$ suppression}
 %Over the past three decades, substantial efforts have been put to search for unambiguous and experimentally viable probes to indicate the formation of the QGP. And, in this regard, $J/\Psi$ suppression~\cite{Matsui:1986dk,Kluberg:2009wc,NA50:2004sgj,Wong:1997rm,SINGH1993147} in heavy-ion collisions is also believed to be one of the prominent signature to indicate the formation of the partonic medium. It is argued that if nuclear collisions lead to QGP production, Debye screening of the huge number of partons that make up the QGP causes the binding between $c$ and $\bar{c}$ to become weaker, which ultimately leads to the disintegration of the pair $c\bar{c}$ and $J/\Psi$ disappears, {\textit{i.e..}}, it is suppressed.

\section{Experimental station}
\label{sec0104}
To recreate the quark matter in the laboratory, physicists need the help of particle colliders which can provide huge centre of mass (CM) energy. The CM energy ranges from $\sqrt{s}=2.4 \,GeV-5.04 \,TeV$ are available at the moment to conduct experiments with beam of atomic nuclei, with mass numbers ranging from proton $(A = 1)$ to Uranium $(A = 238)$. The Relativistic Heavy-Ion Collider (RHIC) at Brookhaven National Laboratory (BNL), and the Large Hadron Collider (LHC) at CERN, both of which collide either protons or heavy ions, or protons with ions. The ALICE (A Large Ion Collider Experiment) Collaboration, is one of eight detector experiments at the LHC, is optimized for studying heavy-ion collisions, and at RHIC there are two principal experiments in operation, STAR (for Solenoidal Tracker at RHIC) and PHENIX (for Pioneering High Energy Nuclear Interaction eXperiment) to conduct. There are other experiments, such as the Compressed Baryonic Matter Experiment (CBM) at the Facility for Antiproton and Ion Research (FAIR) and the Multi Purpose Detector (MPD) at NICA (Nuclotron-based Ion Collider fAcility) at the Joint Institute for Nuclear Research (Dubna, Russia) which are both in advanced stages of construction. A typical heavy-ion collision is depicted in Fig.\ref{fig0103}.
 
 
 %%%%
% Figure environment removed
 %, as well as the planned CEE (CSR External Target) at High Intensity Heavy-Ion Accelerator Facility (HIAF) in China, and a possible future heavy-ion program at J-PARC (Japan Proton Accelerator Research Complex) which presently is a high intensity proton accelerator facility. 
%%
%%%
%%%
%\section{QGP with various collision energies}
%In a collider, in a nucleon-nucleon, nucleon-nucleus or nucleus-nucleus collision, the colliding matter deposits a significant amount of energy. As this deposition of energy around the centre of mass is very high, the QGP is produced. Qualitatively, the collision are categorised into two different energy regions such as baryon-free QGP ($n_{B}\sim0$) which is produced with the centre of mass energy, $\sqrt{s}> 100\,GeV$ per nucleon, and the baryon-rich QGP ($n_{B}\ne 0$) with $\sqrt{s} < 100 \,GeV$ per nucleon. In the baryon-free QGP, two beams of baryons will pass through from the centre of mass with being stopped. In such a situation, QGP will be created with very small baryon content. Whereas, in baryon-rich QGP, as $\sqrt{s}$ is too small for the baryon to pass through but stop, leaving a large number of baryons as remnant near the centre of mass region. In such a scenario, a large amount of baryons will create a baryon-rich QGP. Different experiments are designed to create either baryon-less or baryon-rich QGP for their own purpose of study. As the critical point is supposed to exist at a higher baryon chemical potential region $(\mu_{B} \ne 0$ {\it{i.e.}} $n_{B}\ne 0)$, we are going to study the baryon-rich fluid to observe the effect of the QCD critical point.

\section{QCD phase diagram and its prospects}
\label{sec0105}
\label{phase_diagram}
QCD is a remarkably successful non-abelian quantum field theory to describe the strong interaction. Asymptotic freedom of QCD allows to use perturbative technique to calculate physical quantities with quarks and gluons as fundamental degrees of freedom. Full analytical treatment of QCD is not trivial because if we neglect the masses of the quarks, the theory has no numerically rudimentary parameters, and the confinement scale ($\Lambda_{QCD}\sim 200 MeV \sim 1 fm^{-1}$) becomes solely the independent intrinsic scale of this theory. But for large values of the thermodynamic parameters such as $T$ (temperature) and baryon chemical potential ($\mu_{B}$), the laws of thermodynamics is dominated by short-distance QCD dynamics, and due to the asymptotic freedom the theory can be studied purturbatively. 


Thermodynamic properties of a system are generally expressed in thermodynamic parameter space, simply known as a phase diagram. For the case of the QCD phase diagram, it is represented in $T-\mu_{B}$ parameter space. Any point on the phase diagram corresponds to an equilibrium state, generally characterized by different thermodynamic parameters, such as, $P$ (pressure), baryon density ($n_{B}$) as well as kinetic coefficients, like diffusion or viscous coefficients, or other characteristics of different correlation functions.


Lattice simulations has been applied to describe the phase diagram very accurately at $\mu_{B}=0.$ But the study of thermodynamics of the QCD at $\mu_{B}\ne 0$ is very difficult. The major hindrance in lattice simulations is the so called `sign problem'~\cite{Gavai:2014ela}, and so far no method has been devised to overcome this problem. However, the interesting part (especially the QCD critical point) in the QCD phase diagram lie at the region of non-zero $\mu_{B}$. But the study of critical point is not convergent due to lack of first-principle calculations, therefore, work in this direction will be very much needed to understand the QCD critical point at $\mu_{B} \ne 0$.



%In the chiral limit {\textit{i.e.}} when two lightest quarks, $u$ and $d$, are treated as massless particles, the QCD Lagrangian respects chiral symmetry. At sufficiently high enough temperature, $T>>\Lambda_{QCD}$, the asymptotic freedom of QCD permits to apply perturbative technique on the QGP where the chiral symmetry is restored. Thus, one would expect a transition from a broken chiral symmetric phase to a chirally symmetric phase at $T_{c} \gtrsim \Lambda_{QCD}$. Thermodynamic parameters must be singular at the transition point about the transition line. Thus, in the $T-\mu_{B}$ plane of the phase diagram, the region of broken chiral symmetry must be distinguished from the region of the restored chiral symmetry by a closed boundary as shown in Fig.\ref{fig0104}. The chiral symmetry argument alone is not enough to determine the order of transition. Pisarski and Wilczek in Ref.~\cite{PhysRevD.29.338} argued that the transition can not be second order for three massless quarks, and QCD with $N_{f}=3$ massless quarks must undergo a first order chiral restoration transition.


%For two massless quarks, the transition could be a first order or a second order phase transition. As lattice and model calculations show, both the transitions are possible depending on the numerical value of the mass of the strange quark ($m_{s}$). Along the chiral phase transition line, the transition changes its order at some point, is called tricritical point (see Fig.~\ref{fig0105}). The exact position of this point is still unknown with two massless quarks. As a matter of fact, at $\mu_{B}=0$, the order of the transition which many studies suggest is of the second order~\cite{PhysRevD.29.338} is still being questionable and neither claimed reliably that the transition changes to first order~\cite{Heller}.



%Now when the up and down quark masses are considered to their observed physical finite values, the diagram changes its shape as sketched in Fig.~\ref{fig0106}. A crossover takes the place of the second order phase transition~\cite{10.1143/PTPS.153.139}. 


%Lattice calculations also show, at vanishing $\mu_{B}$ the transition is a crossover~\cite{Fodor:2001pe,MASAYUKI1989668,Halasz:1998qr,deForcrand:2002hgr}. Therefore, in real QCD with non-vanishing quark masses, the second-order phase transition becomes a crossover transition and the tricritical point is replaced by a critical end point (second order). Universality arguments~\cite{Halasz:1998qr,PhysRevLett.81.4816} also predict that the end point in QCD with finite quark masses shifts towards larger $\mu_{B}$ in the phase diagram. There are numbers of model approach, show that at finite (large) $\mu_{B}$, the transition in the phase diagram is a first order phase transition~\cite{MASAYUKI1989668,PhysRevD.41.1610,PhysRevD.49.426,BERGES1999215,PhysRevD.62.105008,Halasz:1998qr,deForcrand:2002hgr,Endrodi:2011gv}. Thus Fig.\ref{fig0106}, is a compilation of results from model calculations, first principle lattice simulations as well as perturbative calculations in asymptotic regimes.


The lattice calculations predict that the transition governed by the temperature at $\mu_{B}=0$ is not a associated with a thermodynamic singularity, rather, it is a smooth crossover~\cite{Fodor_2002,MASAYUKI1989668,PhysRevD.58.096007,DEFORCRAND2002290,Aoki:2006we}  from the phase of  hadrons to the phase of the quarks and gluons. Whereas, the $\mu_{B}$ driven transition at $T=0$ is a first-order phase transition~\cite{DEFORCRAND2002290,Endrodi:2011gv}. This conclusion about the first-order phase transition is less robust due to the sign problem in this regime of $\mu_{B} \ne 0$. At $T=0$, the first-order phase transition line does not end at the vertical axis $\mu_{B}=0$, but stops somewhere in the middle of the phase diagram. This point is known as the critical end point (CEP) of the first-order phase transition line, as shown in Fig.\ref{fig0107}. In condensed matter physics, critical end points are observed where most liquids possess such a singularity, including water. 


The existence of the CEP of QCD phase diagram was suggested theoretically in Refs.~\cite{Halasz:1998qr,PhysRevD.49.426,PhysRevD.42.1757,BERGES1999215,PhysRevD.62.105008} and predicted later in lattice simulation ~\cite{FODOR200287,Fodor:2004nz}. Due to the scarcity of first principle calculations, the exact location of the CEP is still unknown. Some of the QCD based effective models such as NJL and PNJL predict the location of the CEP~\cite{Stephanov:2004wx} with uncertainties of 266-504 MeV in $\mu_{c}$ and 115-162 MeV in $T_{c}$. Therefore, locating the CEP in the phase diagram is still a big, and challenging task. It is one of the main aims of the Beam Energy Scan (BES) programme at RHIC~\cite{STAR:2010vob} to find the CEP by creating the QGP with different $\mu_{B}$ and $T$ by tuning the colliding energy ($\sqrt{s}$ ) of the heavy nuclei. So, the lack of conclusiveness on the position of the CEP pushes for more theoretical as well as experimental study. 

%%%%%%%%%%%%%%%%
% Figure environment removed


A well known feature of a critical point is that the correlation length diverges, and the system is characterised by large fluctuations~\cite{Stanley,SKMa,Minami,PhysRevLett.81.4816,PhysRevD.60.114028}. Therefore, if the system created in Relativistic Heavy-Ion Collider Experiments (RHIC-E) passes through (near) the CEP, the effects will be reflected on the particle spectra. One of the most promising signatures of the CEP is a non-monotonic behaviour of beam energy dependence of higher-order cumulants of the baryon fluctuations reflected through the net proton production. The net proton yield has been  calculated by using both QCD based models~\cite{Stephanov:1998dy,Stephanov:1999zu,Stephanov:2008qz,Stephanov:2011pb} and  gauge/gravity correspondence~\cite{Critelli}. When the CEP is approached, the fluctuations become very large, which is related to the diverging nature of the correlation length ($\xi$). However, due to the critical slowing down, correlation length does not grow as much and limits within 2-3 fm at most~\cite{Berdnikov:1999ph}. The effect of critical slowing down has been taken into account with the slow hydrodynamic modes, recently developed in Ref.~\cite{Stephanov:2017ghc}. In order to find the location of the CEP, we need to better understand the effect of the CEP on the evolution of the QGP, such that the effects could be translated into the particle spectra, and from the analysis of the spectra of the particles, we can have ideas about the QGP which was created with particular $\sqrt{s}$. Therefore, the BES program can help finding a range of $\sqrt{s}$ where we can see significant effects of the QCD critical point. In such scenario, we need more studies in this direction to understand the effects of the critical point.


\section{Organization of the thesis}
In the present dissertation, we are not searching for the location of the CEP, but we want to observe its effects on hydrodynamic evolution of the QGP, by assuming its existence in the phase diagram at some point $(T_{c}, \,\mu_{c})=(154,\,367) MeV$~\cite{Nonaka,Hasan1,Hasan2}, which certainly satisfies the constraints of $\mu_{c}> 2T_{c}$~\cite{BazavovCEP}. However, the choice of the position of the CEP is kept as parameter in the EoS. So, even if we change the position of the CEP in the phase diagram, the obtained result will not change significantly.



The theories as well as experimental results of critical point is well established in the condensed matter physics. However, I want to discuss the phenomenological consequences of the critical point in the context of condensed matter physics and also in the context of the domain of RHIC-E in the next chapter, {\textit{i.e.}} in Ch.~\ref{chapter2}. The evolution dynamics (space-time evolution) of the QGP is modelled by the relativistic viscous hydrodynamics. Therefore, the relevant equations of the relativistic viscous hydrodynamics will be discussed in Ch.~\ref{chapter3}. Hydrodynamic equations are closed by an important equation, which is known as the equation of state (EoS), and to observe the effects of the CEP, we have numerically constructed the EoS which contains the critical point. Thus, the construction of the EoS will be discussed in Ch.~\ref{chapter4}. The next chapters, {\textit{i.e.}} in Ch.~\ref{chapter5} and Ch.~\ref{chapter6}, we will discuss about the propagation of the perturbations (sound waves) based on the linear analysis and compare the propagation of sound near and away from the CEP. In Ch.~\ref{chapter5}, the propagation is studied in the context of dispersion relations, whereas in Ch.~\ref{chapter6}, the propagation of sound wave is studied in the context of the dynamic structure factor. Critical point is accompanied by an effect, called the critical slowing down, where the non-equilibrium modes will resist the system to come to equilibrium state. Therefore, inclusion of those out-of-equilibrium slow modes plays an important role in critical phenomena, and will be discussed in Ch.~\ref{chapter8} in the context of the dynamic structure factor. Now as the perturbations in QGP may not be always small in magnitudes, thus linear analyses are not enough to study their propagation. For that we have studied those perturbations in terms of nonlinear waves, will be discussed in Ch.~\ref{chapter7}.  We finally conclude the studies and provide some outlook in the context of heavy-ion collisions in Ch.~\ref{chapter9}.





