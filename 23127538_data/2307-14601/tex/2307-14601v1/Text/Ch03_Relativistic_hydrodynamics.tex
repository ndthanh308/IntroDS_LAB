\chapter[Relativistic Hydrodynamics]{Relativistic Hydrodynamics}
\label{chapter3}  
%The thermalised QGP evolves through the 
\section{Introduction}
\label{sec0301}
A fluid (liquid or gas) is defined as a substance that does not support a shear stress, and the motion of a fluid is a great concern of fluid dynamics~\cite{Landau_fluid_mechanics}. The fluid dynamical description is simple because the information of the system is encoded in its thermodynamic properties {\it{i.e.}} it provides the macroscopic description of the system. It deals with the collective behaviour, assuming the system attains the local thermal equilibrium {\it{i.e.}} the system has to be described as a complete entity like a liquid or gas rather than a collection of individual particles. If the mean free path of the fluid is defined by $\lambda$ and the characteristic system size is $R$, then a quantity, Knudsen number is defined by taking their ratio as $K_{n}=\lambda/R$. If the Knudsen number is very small ($K_{n}<<1$) then the local equilibrium is assumed to be established and the hydrodynamic description is allowed. There is no other assumption made concerning the nature of the particles and fields, and their interactions. Therefore, a fluid is a continuous medium, implies that even a small volume element of the fluid is an ensemble of a large number of particles. So, whenever we talk about a point of the fluid, it actually is not a mathematical point but it contains large number of particles moving together. A relativistic fluid is a classical fluid which follows the laws of special relativity and/or general relativity. The practical applications of hydrodynamics are extremely diverse. Hydrodynamics is used in designing ships, aircraft, pipelines, pumps, hydraulic turbines, and spillway dams and in studying sea currents, river drifts, and many more.


%Hydrodynamics is a qualitative and powerful tool to describe the collective flow of QCD matter, created in HICs. Describing elliptic flow ($v_{2}$) and other flow observables ($v_{n}$) on a quantitative level is one of the greatest successes of the fluid dynamical description~\cite{Rischke:1998fq,Shuryak:2003xe,Stoecker:1986ci}. Ideal fluid dynamics quantitatively can explain only in central collisions between large $(A \sim 200)$ nuclei at mid-rapidity at top RHIC energies, but gradually break down in a smaller systems, or in a system produced from peripheral collisions, {\it{i.e.}} away from the mid-rapidity region, and at lower collision energies~\cite{Heinz:2004ar}. Even with the highest achievable centre of mass energy $(\sqrt{s})$, the lowest limit of the shear viscosity to entropy ratio is found to be $\eta/s=1/4\pi$, has been proposed based on a correspondence with black-hole physics, known as KSS bound~\cite{Kovtun:2004de}. The viscous hydrodynamics was applied extensively ever since the estimation of surprisingly small value of $\eta/s$ from the analysis of the elliptic flow data~\cite{Romatschke:2007mq}. A study of elliptic flow suggests that the magnitude of viscous corrections is at least $30 \%$~\cite{Drescher:2007cd}. Therefore, the description of strongly coupled QGP produced in HICs will be in better agreement with the viscous effect. Furthermore, if QGP fluid is formed in heavy-ion collisions, it needs to be characterized by its transport coefficients, {\it{e.g.}} bulk viscosity, shear viscosity and the thermal conductivity, .


A fluid can be also categorized into two types: inviscid fluid (the ideal fluid) and viscid fluid (the viscous fluid). Relativistic ideal hydrodynamics deals with the inviscid fluid, considers local thermodynamic equilibrium, where each fluid element is homogeneous, infer vanishing spatial gradients (zeroth order in gradient expansion). Thus, the independent variables used here are $\epsilon$ (energy density), $P$ (pressure), and the fluid four-velocity ($u^{\mu}$). On the other hand, the dissipative hydrodynamical description does not depend on the assumption of local thermodynamic equilibrium. However, the fluid should not be far away from an equilibrium state. The theory of dissipative fluid was firstly formulated by Carl Eckart in 1940~\cite{Eckart:1940te} and then by Landau and Lifshitz in 1987 in a covariant manner~\cite{Landau_fluid_mechanics}. The dissipative hydrodynamics are established from an expansion of entropy four-current ($S^{\mu}$), concerning of dissipative fluxes, and $S^{\mu}$ contains terms linear in dissipative quantities ({\it{i.e.}} first-order gradient in hydrodynamic fields). This is why these are called a first-order theories and corresponding equation is known as Naiver-Stokes (NS) equation. Differential equations in first-order theories are parabolic in nature, thus violating causality: a signal can propagate with an arbitrary high speed. 


The speed of the propagation of signals can be finite even in a parabolic theory. Therefore, first-order theories cannot be excluded by referring to causality only~\cite{Van:2007pw}. However, recently there has been a lot of progress on overcoming the problem of causality and unstable solution. In Ref.~\cite{Bemfica_firstorder}, it has been shown that if hydrodynamic variables are chosen differently, distinct from the thermodynamic variables ($T, \mu, u^{\mu}$) adopted by Eckart and Landau-Lifshitz, the energy-momentum tensor (EMT) may provide a causal theory. The authors have also shown that linear perturbations of equilibrium states are stable. The first-order stable theory with non-vanishing baryon-chemical potential is found in Refs.~\cite{Kovtun:2019hdm,Bemfica:2020zjp}. Also, in Ref.~\cite{Bemfica:2020zjp}, authors have claimed that when the system coupled to Einstein's equations ($R^{\mn}-\frac{1}{2}Rg_{\mn}+\Lambda g_{\mn}=8\pi G T_{\mn}$, where, $R^{\mn}$ is the Ricci tensor, $R=g_{\mn}R^{\mn}$, $\Lambda$ is cosmological constant, and $G$ is the gravitational constant) it became causal, hyperbolic, and the solutions are well-posed with all dissipative contributions (shear viscosity, bulk viscosity, and heat flow) are included. There are also frame stabilized hydrodynamics in first-order theory and correspondence with the second-order theory can be found in Refs.~\cite{ArpanDas:2020fnr,ArpanDas:2020gtq,ArpanDas:2020grz}.


The second-order theory was developed by M\"{u}ller~\cite{Muller:1967zza}, and later by Israel and Stewart~\cite{Israel:1976tn,Stewart,Israel:1979wp}, generally known as M\"{u}ller-Israel-Stewart (MIS) theory. Differential equations concerning the first-order theories are in general parabolic, therefore they are considered as acausal. Whereas, the differential equations concerning the second-order theories, are generally hyperbolic, therefore they provide causal theories. The problem of acausality was remedied by introducing a time delay in the response of the dissipative currents. In second-order theories, the dissipative fluxes are treated as independent variables, which follow equations of motion, describing their relaxation towards their respective NS values. In Refs.~\cite{Hiscock:1983zz,Hiscock:1985zz}, the authors have studied the stability of the MIS theory by examining the behaviour of small perturbations about equilibrium and found that MIS theory is causal, stable, and well-posed. Recently~\cite{Bemfica:2019cop,Bemfica:2020xym}, significant progress was made to understand causality in the nonlinear regime of such theories. It is established in Ref.~\cite{Bemfica:2020xym} that a set of conditions is needed to hold the causality in the nonlinear regime of MIS-like theories at zero chemical potential. The new formalism of second-order theories was proposed, where covariance and causality are satisfied by incorporating the memory effect in dissipative currents~\cite{Denicol:2008hb,Denicol:2008ha}. Further reviews on second-order theory can be found in Refs.~\cite{Muronga:2006zw,Denicol:2008rj,Koide:2006ef}.


%In second-order relativistic theories, the entropy current is quadratic in terms the fluxes, contains terms like $q_{\mu}q^{\mu}u^{\alpha},\,\, \Pi^{2} u^{\alpha},\,\, \pi_{\mn}\pi^{\mn} u^{\alpha}, \,\,\Pi q^{\mu},\,\, \pi_{\mn}q^{\mu}$ etc., characterizing the deviation from the local equilibrium.

In this dissertation, without going into an argument on the choice of order of the theory, we are only considering the second-order theory, specifically, the MIS hydrodynamics to serve our purpose. Another issue of dissipative hydrodynamics is that of choice of frame. In general, there are number of ways of choosing the frames of reference, but the mostly used choices are Landau-Lifshitz (LL)~\cite{Landau_fluid_mechanics} and Eckart's~\cite{Eckart:1940te} frame of reference. As the physics in all the possible frames must be independent of the choice of the frame, in this dissertation, we have used both the frames of reference as mentioned specifically below. Definition of dissipative fluxes are also modified on the choice of the metric tensor, $g^{\mn}= (-1,\,1,\,1,\,1)$ or $g^{\mn}=(1,\,-1,\,-1,\,-1)$. However, in this dissertation, we are going to use the former definition of the metric. In this dissertation, we have also used units, $\hbar=K_{B}=c=1$.

%, along with the definition of the dissipative fluxes {\it{i.e.}} $\Pi, \pi^{\mn}, q^{\mu}$. Also, the relaxation effect in second-order theory appears in the dissipative fluxes in terms of few coefficients, $\alpha_{0}, \alpha_{1}, \beta_{0}, \beta_{1}, \beta_{2}$, are generally known as coupling and relaxation coefficients, which are evaluated through the thermodynamic integral from kinetic theory prescription. Those coefficients in two different frames are related by a constant translation~\cite{Israel:1979wp}. 
%%
%%
\section{Thermodynamics in covariant form}
\label{sec0302}
The covariant form of the thermodynamics was started with the Fourier's law of heat conduction where heat and temperature gradients are related in an un-relaxed manner. This infers that if two bodies at different temperatures are placed together, heat spontaneously flows from the warmer to the colder body without any time delay, depicted by an equation as:
\beqa
\vec{q}=-\kappa\vec{ \nabla} T\,,
\label{eq0301}
\eeqa
where, $q, \kappa, T$ are heat flux, thermal conductivity, and temperature respectively. Cattaneo figured a way out of this kind of situation by modifying the Fourier's law in which he introduced some characteristic time in the response for the material takes to react~\cite{Cattaneo2011,Lopez}, and finally got
\beqa
\vec{q}=-\kappa \vec{\nabla} T \,\,\,\,\rw \,\,\,\,\vec{q}+\tau \dot{\vec{q}}=-\kappa \vec{\nabla} T\,.
\label{eq0302}
\eeqa
Here, $\tau$ is the relaxation time of the medium. Such an adaptation leads to the famous telegrapher equation for the propagation of heat signals. The very first attempt of a relativistic extension of the heat equation was put forward by Charles Eckart in 1940~\cite{Eckart:1940te}. His proposed model has become a typecast of theories known as first-order relativistic viscous theories. In this type of theories, the entropy current, characterized by a vector field, $S^{\mu}$, depends on terms that are linear in deviations from equilibrium, specifically the heat flow or the shear viscosity. The modest way in such type of theories is to impose the second law, which takes the form as $\pd_{\mu} S^{\mu} \ge 0$, leads to the relativistic edition of the Fourier's law
\beqa
q^{\mu}=-\kappa \Delta^{\mn}\big[ \pd_{\nu}T+ T \dot{u}_{\nu} \big]\,,
\label{eq0303}
\eeqa
where, $q^{\mu}$ is the heat flux, $u^{\mu}=\gamma(1,\vec{v})$ is the fluid four velocity, and it follows
\beqa
u_{\mu}u^{\mu}=-1\,. 
\label{eq0118}
\eeqa
Here, $\gamma=1/(1-v^{2})$, is the Lorentz factor. 

The operator $ \Delta^{\mn}$ is a projection operator to $u^{\mu}$, and is defined as
\beqa
\Delta^{\mn}=g^{\mn}+u^{\mu}u^{\nu}\,.
\label{eq0119}
\eeqa
 It has the properties 
 \beqa
 \Delta^{\mu\nu\rho\lambda}=\frac{1}{2}\Big [\Delta^{\mu\rho}\Delta^{\nu\lambda}+\Delta^{\nu\rho}\Delta^{\mu\lambda}-\frac{2}{3}g^{\mu\nu}\Delta^{\rho\lambda}\Big ]
 \eeqa
 \beqa
 \Delta^{\mu\nu}u_{\mu}=\Delta^{\mu\nu}u_{\nu}=0, \,\,\,\,\,  \,\,\,\, \Delta^{\mu\nu}\Delta^{\alpha}_{\nu}=\Delta^{\mu\alpha}, \,\,\,\, \Delta^{\mn}\pd_{\nu}=\nabla^{\mu},\,\,\,\,\Delta^{\mu}_{\mu}=3\,,
 \label{eq0120}
 \eeqa


Before discussing the covariant thermodynamics, we recall the useful thermodynamic relations derived from first, second, and third law of thermodynamics as follows:
\beqa
Ts=\ep+P-\mu n\,.
\label{eq0304}
\eeqa
This relation is called the Euler's equation. Here, $T, \ep, P, \mu, n,$ and $s$ are the temperature, energy density, pressure, chemical potential, number density, and entropy density respectively.  
\beqa
dP=sdT+nd\mu\,.
\label{eq0305}
\eeqa
This is known as Gibbs-Duhem relation. Using Eqs.\eqref{eq0304} and \eqref{eq0305}, we find
\beqa
Tds=d\ep -\mu dn\,.
\label{eq0306}
\eeqa
To write the above three equation in covariant form, we define
\beqa
\beta=\frac{1}{T},\,\,\,\,\, \beta_{\mu}=\frac{u_{\mu}}{T},\,\,\,\,\, \text{and}\,\,\,\,\,\, \alpha=\frac{\mu}{T}\,.
\label{eq0307}
\eeqa
With this definition, Eqs.\eqref{eq0304}, \eqref{eq0305}, and \eqref{eq0306} are postulated in the following covariant form~\cite{Israel:1979wp}:
\beqa
S^{\mu}_{(0)}&=&P\beta^{\mu} +\beta_{\nu}T^{\mn}_{(0)} -\alpha N^{\mu}_{(0)}\,,
\label{eq0308}
\eeqa
\beqa
dS^{\mu}&=& -\alpha dN^{\mu}-\beta_{\nu} dT^{\mn}\,,
\label{eq0309}
\eeqa
\beqa
d(P\beta^{\mu})&=& N^{\mu}_{(0)} d\alpha + T^{\mn}_{(0)} d\beta_{\nu}\,.
\label{eq0310}
\eeqa
where, $T^{\mn}, N^{\mu}$ are the energy-momentum tensor and particle flow vector respectively. Also, $T^{\mn}_{(0)}$ and  $N^{\mu}_{(0)}$ are respectively the equilibrium states of energy-momentum tensor and the particle four current in equilibrium. The transition from the equilibrium to non-equilibrium state is affected by the assumption that Eq.\eqref{eq0310} holds for arbitrary virtual displacement (not just the neighbourhood of equilibrium state). Therefore, adding Eqs.\eqref{eq0308} and \eqref{eq0309}, we get an arbitrary non-equilibrium state
\beqa
S^{\mu}_{(0)}+dS^{\mu}&=& P\beta^{\mu} - \alpha \big[N^{\mu}_{(0)}+dN^{\mu}\big]-\beta_{\nu}\big[T^{\mn}_{(0)}+dT^{\mn}\big]\,.
\label{eq0311}
\eeqa
If the contribution of the higher order terms are present, the definition of the entropy density of any arbitrary non-equilibrium state can be written using Eq.\eqref{eq0311}
\beqa
S^{\mu}=P\beta^{\mu}-\alpha N^{\mu} -\beta_{\nu}T^{\mn}-\mathcal{Q^{\mu}}\,,
\label{eq0312}
\eeqa
Here, $\mathcal{Q}^{\mu}$ is function of $\delta T^{\mn}$ and $\delta N^{\mu}$, where $\delta T^{\mn}=T^{\mn}_{(0)}-dT^{\mn}$ and $\delta N^{\mu}=N^{\mu}_{(0)}-dN^{\mu}$ are just deviations from local equilibrium. Thus, $\mathcal{Q}^{\mu}$ can be expanded in Taylor's series in terms of dissipative currents up to any arbitrary order, and keeping up to second order gives rise to the second-order theory. Hence, the choice of $\mathcal{Q}^{\mu}$ fixes the structures of the dissipative fluxes.
%%
%%
\section{M\"{u}ller-Israel-Stewart Hydrodynamics}
\label{sec0303}
The set of hydrodynamic equations are partial differential equations which require a well defined set of initial conditions. However, existence of some hydrodynamic modes in the equations can travel backwards in time, thus the initial conditions cannot be put arbitrarily~\cite{Mario2000}. As a consequence solving the relativistic NS equation numerically becomes a challenging task. One feasible way to manage the theory was worked out in Maxwell-Cattaneo law. This law was thought to be a successful phenomenological extension of the NS equation. But it is unsatisfactory since it is not deduced from a first-principle framework, but rather introduced `by hand'. It is also found out that this law can not even correctly explain the causality of larger wave number of relativistic viscous hydrodynamic theory~\cite{Romatschke:2009im}. Therefore, it can not be generalized in order to formulate the theory of relativistic viscous hydrodynamics.


The covariant formulation of relativistic dissipative hydrodynamics was formulated by Eckart~\cite{Eckart:1940te} and Landau-Lifshitz~\cite{Landau_fluid_mechanics}. These are first-order theory, suffers from the problem of causality and stability~\cite{Hiscock:1983zz,Hiscock:1985zz}. Later second-order theory was established by firstly M\"{u}ller in 1967 and then Eckart's theory was generalized extensively by Israel and Stewart~\cite{Israel:1976tn,Stewart,Israel:1979wp} in 1976. The causality and stability of the Israel-Stewart (IS) hydrodynamics was tested by Hiscock and Lindblom~\cite{Hiscock:1983zz} in 1983 and found to be causal and stable in a far wider range of circumstances. 
%%
%%
\subsection{Equations of motion in general form}
Before formulating the equations in relativistic viscous fluid, one need to go through the equations of the ideal fluid. Therefore, to a relativistic ideal fluid, the general form of the EMT, $T^{\mn}_{(0)}$, (net) particle four-current, $N^{\mu}_{(0)}$, and the entropy four-current, $S^{\mu}_{(0)}$, have to be formulated from the fluid four velocity, $u^{\mu}$, and the metric tensor, $g^{\mu\nu}$. However, since, $T^{\mn}_{(0)}$ must be a symmetric tensor and are transformed by tensorial set of rules, and $N^{\mu}_{(0)}$ and $S^{\mu}_{(0)}$ transform by vector rules under the Lorentz transformations, the general form of these quantities can be obtained as
\beqa
T^{\mn}_{(0)}=\ep u^{\mu}u^{\nu}+P\Delta^{\mn}~,\,\,\,\,\, N^{\mu}_{(0)}=nu^{\mu}~,\,\,\,\,\,S^{\mu}_{(0)}=su^{\mu}~.
\label{eq0113}
\eeqa
The effective representation of an ideal fluid is derived by using the conservation laws of energy, momentum, and net particle number, usually by taking the four-divergences to those quantities as:
\beqa
\pd_{\mu}T^{\mn}_{(0)}=0~,\,\,\,\, \pd_{\mu}N^{\mu}_{(0)}=0
\label{eq0114}
\eeqa
The different dissipative effects lead to the dissipative currents such as $\tau^{\mn}$ and $\nu^{\mu}$ are to be added with the ideal currents $T^{\mn}_{(0)}$ and $N^{\mu}_{(0)}$ respectively. Here, $\tau^{\mn}$ must be symmetric tensor ($\tau^{\mn}=\tau^{\nu\mu}$) in order to satisfy the conservation of angular momentum. Now, the main concern is to find the suitable equations, which need to be satisfied by all of these dissipative currents. Therefore, the EMT appears as
\beqa
T^{\mn}&=&T^{\mn}_{(0)}+\tau^{\mn}\nn\\
&=& \ep u^{\mu}u^{\nu}+P\Delta^{\mn}+\tau^{\mn}\nn\\
&=& \epsilon u^{\mu}u^{\nu}+(P+\Pi)\Delta^{\mu \nu}+h^{\mu}u^{\nu}+h^{\nu}u^{\mu}+\pi^{\mu \nu} \,,
\label{eq0115}
\eeqa



where, $\Pi, \,\pi^{\mn}$ and $h^{\mu}$ are the bulk viscous pressure, shear stress tensor and energy dissipation respectively. 

Now, the particle four current with dissipation will appear as
\beqa
N^{\mu}&=&N^{\mu}_{(0)}+\nu^{\mu}\nn\\
&=&nu^{\mu}+\nu^{\mu} \,,
\label{eq0116}
\eeqa
%where, the energy-momentum tensor in ideal case is
%\beqa
%T^{\mn}_{(0)}= \ep u^{\mu} u^{\nu}+p \Delta^{\mn}
%\eeqa
where, $n$ is number density (baryon, charge, strangeness), and $\nu^{\mu}$ is the particle diffusion current, is related to heat flow vector ($q^{\mu}$) by the following relation
\beqa
q^{\mu}=h^{\mu}- \nu^{\mu}\,\,\frac{\ep+P}{n}\,,
\label{eq0117}
\eeqa
and it follows
\beqa
u_{\mu}h^{\mu}=0, \,\,\,\, u_{\mu}\nu^{\mu}=0\,.
\label{eq0121}
\eeqa
 The viscous fluxes follow the following relations:
\beqa
u_{\mu}q^{\mu}&=& q^{\mu}u_{\mu}=0,\nn\\
\pi^{\mn}&=&\pi^{\nu\mu},\,\,\,\,
u_{\mu}\pi^{\mu\nu}=0,\,\,\,\, \pi^{\mu}_{\mu}=0~.
\label{eq0122}
\eeqa
The relations between energy density, pressure and the dissipative fluxes and EMT are given by the following relations:
\beqa
q_{\alpha}&=&u_{\mu}\tau^{\mu\nu}\Delta _{\nu\alpha}, \,\,\,\, u_{\mu}\tau^{\mu\nu}=q^{\nu} \nn\\
P+\Pi&=&-\frac{1}{3}\Delta_{\mu\nu}T^{\mu\nu}, \,\,\,\, \Pi=-\frac{1}{3}\Delta_{\mu\nu}\tau^{\mu\nu}~.
\label{eq0123}
\eeqa
The equations of motion will be found out by the conservation equations as:
\beqa
\pd_{\mu}T^{\mn}=0~,\,\,\,\, \pd_{\mu}N^{\mu}=0~.
\label{eq0124}
\eeqa
In Eq.\eqref{eq0115}, $T^{\mn}$ is a second rank symmetric tensor and has $10$ independent components, and Eq.\eqref{eq0116}, $N^{\mu}$ is a four vector, has $4$
 independent components, {\it{i.e.}} a total of $14$ independent components. In tensor decomposition of $T^{\mn}$ and $N^{\mu}$, if we use Eq.\eqref{eq0121}, we have $3+3=6$ independent variables. The shear stress, $\pi^{\mn}$ is a symmetric traceless second rank tensor, and has $5$ independent variables. Taking into account the quantities, $\Pi, \ep, n, u^{\mu},$ and $P$, we have total 17 independent coordinates, which is extra in $3$ than expected, can cause ambiguity. But fluid four-velocity is defined as an arbitrary, normalized, time-like 4-vector, may need proper definition to reduce the number of independent coordinates.
 %%%%
 %%%%
\subsection{Fitting conditions, definition of fluid velocity, and the choice of a frame}
The flow in ideal hydrodynamics is uniquely calculated since the local energy fluxes and the charge densities are in the same direction, {\it{i.e.}}, the directions of the eigenvector of the energy-momentum tensor and the conserved current match as
\beqa
\ep=u_{\mu}u_{\nu}T^{\mn}_{(0)},\,\,\,\,\text{and}\,\,\,\,\,n=u_{\mu}N^{\mu}_{(0)}~.
\label{eq0124}
\eeqa
But the presence of the dissipative fluxes lead to the separation of the two local fluxes in the systems. Therefore, it is necessary to define a local rest frame (LRF) for the fluid. In such scenario, the fitting conditions also demands
\beqa
u_{\mu}u_{\nu}\tau^{\mn}=0,\,\,\,\, \text{and}\,\,\,\, u_{\mu}\nu^{\mu}=0~.
\label{eq0126}
\eeqa

So far $u^{\mu}$ is arbitrary. It attains a physical meaning by relating it to $T^{\mn}$ and $N^{\mu}$ only, and therefore it needs to be defined in some LRF. In general, there could be many choices for the frame of references (a frame can be attached to each conserved charge of the system). However, the most widely used choices are Eckart~\cite{Eckart:1940te} and Landau-Lifshitz (LL)~\cite{Landau_fluid_mechanics} frames of references. %The LL frame represents a local rest frame where the energy dissipation is zero but the net-number dissipation (diffusion) is non-zero, whereas the Eckart frame represents a local rest frame where the net-charge dissipation is vanishing but the energy dissipation is non-vanishing.
\section{Eckart frame of reference}
\label{sec0304}
Eckart frame represents a local rest frame where the net-charge dissipation is vanishing but the energy dissipation is non-vanishing {\it{i.e.}}
\beqa
\nu^{\mu}=0,\,\,\,\,\text{and}\,\,\,\,h^{\mu}\ne 0~,
\label{eq0127}
\eeqa
And, the velocity of fluid is defined as
\beqa
u^{\mu}_{E}=\frac{N^{\mu}}{\sqrt{-N^{\nu}N^{\nu}}},\,\,\,\,n=-u^{E}_{\mu}N^{\mu}~.
\label{eq0128}
\eeqa
 The EMT and the particle current will become
 \beqa
 T^{\mn}_{E}&=&\epsilon u^{\mu}u^{\nu}+(P+\Pi)\Delta^{\mu \nu}+q^{\mu}u^{\nu}+q^{\nu}u^{\mu}+\pi^{\mu \nu}~,
 \label{eq0129}
 \eeqa
%%
and
\beqa
 N^{\mu}_{E}&=&nu^{\mu}~.
 \label{eq0130}
 \eeqa
 The $14$ unknowns are $\ep (1), n (1), P (1), h^{\mu} (3), \pi^{\mn} (5), u^{\mu}_{E} (3)$.
 \section{Landau-Lifshitz frame of reference}
 \label{sec0305}
 The Landau-Lifshitz (LL) frame represents a local rest frame where the energy dissipation is zero but the net-number dissipation (diffusion) is non-zero implies
 \beqa
 h^{\mu}=0\,\,\,\,\text{and}\,\,\,\,\nu^{\mu}\ne0~.
 \label{eq0131}
 \eeqa
 The velocity of the fluid is defined as
 \beqa
 u^{\mu}_{L}=-\frac{u_{\nu}T^{\mn}}{\sqrt{u_{\alpha}T^{\alpha \beta}T_{\beta\gamma}u^{\gamma}}},\,\,\,\, u_{\mu}u_{\nu}T^{\mu\nu}=\ep~.
 \label{eq0132}
 \eeqa
 The EMT and the particle current will become
 \beqa
 T^{\mn}_{L}&=&\epsilon u^{\mu}u^{\nu}+(P+\Pi)\Delta^{\mu \nu}+\pi^{\mu \nu}~,
 \label{eq0133}
 \eeqa
 and
%%
\beqa
 N^{\mu}_{L}&=&nu^{\mu}+\nu^{\mu}=nu^{\mu}-\frac{nq^{\mu}}{\ep+P}~.
 \label{eq0134}
 \eeqa
 The $14$ unknowns are $\ep (1), n (1), P (1), \nu^{\mu} (3), \pi^{\mn} (5), u^{\mu}_{L} (3)$.
 
 

 The above definitions of $u^{\mu}$ on Eckart and LL frame impose some constraints on the dissipative currents. In Eckart frame of reference, the particle diffusion ($\nu^{\mu}$) is set to zero, whereas in LL frame of reference, the energy diffusion ($h^{\mu}$) is set to be zero. In simple words, the Eckart definition eliminates any diffusion of particles, whereas, the Landau-Lifshitz definition of the velocity field eliminates any diffusion of energy. 
 %%
 %%
 %%
\section{Choice of $\mathcal{Q^{\mu}}$ and the dissipative fluxes}
\label{sec0306}
 The equilibrium state of a thermodynamic system is interpreted as a stationary state, where the intensive and extensive thermodynamic variables of the system do not change with time. The second law of thermodynamics says that the entropy of an isolated thermodynamic system should either increase or remain constant. Hence, for a thermodynamic system in equilibrium, the entropy, being an extensive variable, settles to remain constant. On contrary, for an out of equilibrium system, the entropy must invariably increase. This is an exceptionally powerful concept that will be mainly used in this section to derive the equations of motion of a fluid with dissipation. Therefore, for our study we take space-time derivative of Eq.\eqref{eq0311} which satisfies the relation:
\beqa
\pd_{\mu}S^{\mu}\ge 0\,.
\label{eq0135}
\eeqa
 But before deriving the EOM and evaluating the dissipative fluxes, we need to fix the $\mathcal{Q^{\mu}}$ (entropy current due to heat flow). The choice of $\Q^{\mu}$ fixes the order of the theory. In Eckart's definition, a simple form of $\mathcal{Q^{\mu}}$ was considered as
 \beqa
 \mathcal{Q^{\mu}}=\frac{q^{\mu}}{T}\,\,\,\, \Rightarrow \,\,\,\,\,\,S^{\mu}=su^{\mu}+\frac{q^{\mu}}{T}\,.
 \label{eq0136}
 \eeqa 
 The first term in $S^{\mu}$ is the representation of the entropy carried along the motion of the fluid, whereas, the second term, represents the entropy flux due to the heat flow inside the fluid. Therefore using the second law {\textit{i.e.}}, Eq.\eqref{eq0135} we can derive
 \beqa
 T\pd_{\mu}S^{\mu}=-\Pi \pd_{\mu}u^{\mu}-q^{\mu}\Big[\frac{1}{T}\pd_{\mu}T+u^{\nu}\pd_{\nu}u_{\mu}\Big]-\pi^{\mu\nu}\Big<\pd_{\mu}u_{\nu}\Big>\,,
 \label{eq0137}
 \eeqa
 where, $\Big<\pd_{\mu}u_{\nu}\Big>$ is defined for any second rank tensor as
 \beqa
 \Big<\pd_{\mu}u_{\nu} \Big>=\frac{1}{2}\Delta^{\alpha}_{\mu}\Delta^{\beta}_{\nu}\Big[\pd_{\alpha}u_{\beta}+\pd_{\beta}u_{\alpha}-\frac{2}{3}\Delta_{\alpha\beta}\Delta^{\gamma\delta}\pd_{\gamma}u_{\delta}\Big]
 \label{eq0138}
 \eeqa
The simplest way to ensure that the Eq.\eqref{eq0135} holds, is to write
\beqa
\Pi&=&- \zeta \pd_{\mu}u^{\mu}\,,\nn\\
\pi^{\mu\nu}&=&-2\eta \Big<\pd_{\mu}u_{\nu}\Big>\,,\nn\\
q^{\mu}&=&-\kappa \Big[\frac{1}{T}\pd_{\mu}T+u^{\nu}\pd_{\nu}u_{\mu}\Big]\,.
\label{eq0139}
\eeqa
The three quantities $\zeta, \eta$, and  $\kappa$ are known as bulk viscosity, shear viscosity, and thermal conductivity respectively, and they must be positive. These three equations (in Eq.\eqref{eq0139}) are the dissipative fluxes correspond to three transport coefficients. With these, the second law becomes
\beqa
\frac{\Pi^{2}}{\zeta T}+\frac{q^{\mu}q_{\mu}}{\kappa T^{2}}+\frac{\pi^{\mu\nu}\pi_{\mu\nu}}{2\eta T}\ge 0\,.
\label{eq0140}
\eeqa
Therefore, the dissipative fluxes are decided with the choice of $\Q^{\mu}$, and this is coming from the NS theory, which is acausal. Now motivated by the above approach the theory, M\"{u}ller\cite{Muller:1967zza}, then Israel \& Stewart~\cite{Israel:1976tn,Stewart,Israel:1979wp} generalised the theory to overcome the difficulty of causality by proposing the following expression for the entropy current as
\beqa
S^{\mu}=~ su^{\mu}+\frac{q^{\mu}}{T} &&- \Big[\beta_0\Pi^2 - \beta_1 q_\nu q^\nu 
	+ \beta_2\pi_{\rho\sigma} \pi^{\rho\sigma}\Big] \frac{u^\mu}{2T} \nonumber\\
	&&+ \Big[\alpha_0\Pi\Delta^{\mu\nu} + \alpha_1\pi^{\mu\nu}\Big]\frac{q_\nu}{T}~,
	\label{eq0141}
\eeqa
The coefficients $\beta_{0}, \beta_{1}, \beta_{2}$ are called the relaxation coefficients, are causing the deviation from the physical entropy density, and $\alpha_{0}, \alpha_{1}$ are known as the coupling coefficients, and causes due to the thermo-viscous coupling. To hold the second law of thermodynamics, we must have~\cite{Israel:1979wp} in Eckart's frame of reference
\beqa
\label{eq0142}
\Pi&=& -\frac{1}{3}\zeta\Big[\pd_{\mu}u^{\mu}+\beta_{0}D\Pi-\tilde{\alpha_{0}}\pd_{\mu}q^{\mu}\Big]~, \\
%%%
\label{eq0143}
\pi^{\mu\nu}&=&-2\eta \Delta^{\mu\nu\alpha\beta}\Big[\partial_{\alpha}u_{\beta}+\beta_{2}D\pi_{\alpha\beta}-\tilde{\alpha_{1}}\partial_{\alpha}q_{\beta}\Big]~,\\
%%%
\label{eq0144}
q^{\mu}&=&-\kappa T\Delta^{\mu\nu} \Big[\frac{1}{T}\partial_\nu T +Du_{\nu}+\tilde{\beta_1} D{q_\nu}-\tilde{\alpha_0}\partial_\mu \Pi -\tilde{\alpha_1}\pd_{\lambda}\pi ^{\lambda}_{\nu}~ \Big] . 
\eeqa
whereas, the dissipative fluxes are modified in Landau-Lifshitz frame as
\begin{eqnarray}
\Pi &=&-\frac{1}{3}\zeta\Big[\pd_{\mu}u^\mu +\beta_0 D \Pi-\alpha_0  \pd_{\mu}q^\mu \Big]~,\nonumber\\   
\pi^{\mu\nu}&=&-2\eta \Delta^{\mu\nu\alpha\beta}\Big[\partial_{\alpha}u_{\beta}+\beta_{2}D\pi_{\alpha\beta}-\alpha_{1}\partial_{\alpha}q_{\beta}\Big]~,\nonumber\\
q^{\mu}&=&\kappa T\Delta^{\mu\nu} \Big[\frac{nT}{\epsilon+P}(\partial_\nu \alpha )-\beta_1 D{q_\nu}+\alpha_0\partial_\nu \Pi +\alpha_1\pd_{\lambda}\pi ^{\lambda}_{\nu} \Big]~. 
\label{eq0145}
\end{eqnarray}
where,  $D\equiv u^\mu\partial_\mu$, is known as co-moving derivative and in LRF, $D\Pi =\dot{\Pi }$ represents the time derivative. Also to note that the coupling and relaxation coefficients are different in two different frames and these coefficients in  the Eckart frame ($\tilde{\alpha_{0}}, \tilde{\alpha_{1}}, \tilde{\beta_{1}}$) are connected to the 
corresponding coupling and relaxation coefficients in the Landau frame ($\alpha_{0}$, $\alpha_{1}$, $\beta_{1}$) by the following relation~\cite{Israel:1979wp}
\begin{equation}
\tilde{\alpha_{0}}-\alpha_{0}=\tilde{\alpha_{1}}-\alpha_{1}=-(\tilde{\beta_{1}}-\beta_{1})=-[(\epsilon+p)]^{-1}~.
\label{eq0146}
\end{equation}
%%
The $\beta _0, {\beta_1}$ and $\beta_2$ are also related to the relaxation time scale as~\cite{Muronga:2003ta,Muronga:2001zk}:
\begin{equation}
\tau_{\Pi}=\zeta \beta_0, \,\,\,\,\tau_q=\kappa T\beta_1,\,\,\,\, \tau _{\pi}=2\eta \beta_2~.
\label{eq0147}
\end{equation}
The coupling coefficients are connected to the scale of relaxation lengths, which again couple to heat flux, and bulk pressure $(l_{q\Pi}, l_{\Pi q})$, the heat flux, and shear tensor $(l_{q\pi}, l_{\pi q})$ by the following relation
\begin{equation}
l_{\Pi q}=\zeta \alpha_0,\,\,\,\, l_{q\Pi}=\kappa T \alpha _0, \,\,\,\,l_{q\pi}=\kappa T\alpha_1,\,\,\,\, 
l_{\pi q}=2\eta \alpha_1~.
\label{eq0148}
\end{equation}  
%%%%%%%%%%%%%%%%%%%%%%%%%%%%%%%%%%%%%%%%%%%%%%%%
The  expressions for relaxation and coupling coefficients are given in Appendix-\ref{appendix01_A}, which are evaluated through the thermodynamic integral from kinetic theory prescription.~\cite{Israel:1979wp}. 
%%
%%

Once the dissipative fluxes are chosen by the choice of frame, the structure of $T^{\mn}$ and $N^{\mu}$ are determined. Therefore, the equation of motion of the fluid will be dictated by the following relations as:
\beqa
\pd_{\mu}T^{\mn}=0\,;\,\,\,\,\,\pd_{\mu}N^{\mu=0}\,.
\label{eq0349}
\eeqa
The hydrodynamic equations are not in general closed. Therefore, to close the equations we need an extra equation, known to be the equation of state (EoS). Also, the above equations are partial differential equations, which must be solved by the proper initial condition. However, in this dissertation, we will not discuss about the initial conditions (one can see for the references~\cite{Glauber1955,IPGlasma2012,Sandeep2016}). But we shall construct an EoS which will contain the critical point, shall be discussed in the next chapter to study the effects of critical point on the evolution of the QGP.


%\section{Region of validity of hydrodynamics}
%\label{sec0307}
%\section{Hydrodynamic framework in space-time evolution of the QGP}
%In a collision of heavy ions moving relativistically, the QGP is likely to form, and it expands very rapidly after the formation. The relativistic hydrodynamics can be a powerful tool to describe the collective flow of QCD matter, created in HICs. Describing elliptic flow ($v_{2}$)~\cite{Snellings:2011sz} and other flow observables ($v_{n}$)~\cite{Ollitrault:2007du} on a quantitative level is one of the greatest successes of the fluid dynamical description~\cite{Rischke:1998fq,Shuryak:2003xe,Stoecker:1986ci}. Ideal fluid dynamics quantitatively can explain only in central collisions between large $(A \sim 200)$ nuclei at mid-rapidity at top RHIC energies, but gradually break down in a smaller systems, or in a system produced from peripheral collisions, {\it{i.e.}} away from the mid-rapidity region, and at lower collision energies~\cite{Heinz:2004ar}. Even with the highest achievable centre of mass energy $(\sqrt{s})$, the lowest limit of the shear viscosity to entropy ratio is found to be $\eta/s=1/4\pi$, has been proposed based on a correspondence with black-hole physics, known as KSS bound~\cite{Kovtun:2004de}. The viscous hydrodynamics was applied extensively ever since the estimation of surprisingly small value of $\eta/s$ from the analysis of the elliptic flow data~\cite{Romatschke:2007mq}. A study of elliptic flow suggests that the magnitude of viscous corrections is at least $30 \%$~\cite{Drescher:2007cd}. Therefore, the description of strongly coupled QGP produced in HICs will be in better agreement with the viscous effect. Furthermore, if QGP fluid is formed in heavy-ion collisions, it needs to be characterized by its transport coefficients, {\it{e.g.}} bulk viscosity, shear viscosity and the thermal conductivity.

%The space-time evolution of the QGP will be studied by the theory relativistic viscous hydrodynamics. The hydrodynamic equations are not in general closed. Therefore to close the equations we need an extra equation, known to be the EoS. In this dissertation, however, we are going to study the effects of critical point on the evolution of the QGP. Thus, we shall construct an EoS which will contains the critical point, shall be discussed in the next chapter.
%%%
%%%
%%%
%\section{Transport coefficients near the critical point}
%\label{sec0307}
%A fluid (thermodynamic system) in equilibrium can fluctuate to a non-equilibrium state in a number of ways. Depending on the nature of fluctuation there is an emergence of the corresponding dissipative process to counterbalance this fluctuation to maintain the equilibrium. The transport coefficients are thus induced as a response of a system to the perturbations,  and this response determine the dynamics of the system towards the equilibrium state through dissipation. As the QGP is considerably away from equilibrium, therefore the role of the transport coefficients is very important. Within the scope of linear response theory, the medium response allows us to determine the transport coefficients like $\eta, \zeta$, etc. It is generally accepted that the hot and dense QGP behaves as an almost perfect fluid, where shear viscosity over entropy density ratio $\eta/s$ is very close to the lower bound anticipated by the string theory, known as KSS bound~\cite{Kovtun:2004de}. Relativistic viscous theory of hydrodynamics is proved to be exceptionally successful in expressing the space-time evolution of the QGP. At the final state of a typical Relativistic Heavy-Ion Collision Experiment (RHIC-E), we eventually extract physical observables such as the invariant yield $(d^{3}N/dp^{3})$ and the flow harmonics $(v_{n}, n=2, 3, 4,...)$ of charged hadron from viscous hydrodynamics simulation for a given EoS and proper initial conditions. Experimental results shows that extraction of $\eta/s$ and $\zeta/s$ are sensitive to the transverse momentum $(p_{T})$ dependence of flow harmonics, $v_{n}(p_{T})$~\cite{Luzum2008,Song2011,Schenke2011,Victor2011,Niemi2011,Ryu2015}.


%Near the QCD critical point, the transport coefficients of the QGP, are expected diverge. In Sec.\ref{sec0206} in Ch.\ref{chapter1}, we have discussed that the correlation length ($\xi$) grows near the CEP. The nature of $\xi$ near the CEP affects the bulk evolution of the thermalized medium, making the physical observables sensitive to such critical dynamics. The transport coefficients near the critical region are also affected by the criticality with universal exponents, which are decided by the universality class of the system. The QCD critical point belongs to the universality class of $\mathcal{O}(4)$,  provides~\cite{Hohenberg1977,Son2004,Moore2008,Onuki1997}:
%\beqa
%\eta \sim \xi^{\epsilon/19}~,\,\,\,\, \zeta \sim \xi^{3}~,\,\,\,\, \kappa \sim \xi~,
%\label{eq0349}
%\eeqa
%where, $\epsilon=4-d$ ($d$ is the spatial dimension) is some critical exponent. Due to the presence of the CEP, the transport coefficients get enhanced, and we can see that the bulk viscosity is more affected by the criticality. However, in this dissertation, we will use the results of recent studies~\cite{Monnai2017,KapustaChi,Antoniou2017,Minami_thesis}.

