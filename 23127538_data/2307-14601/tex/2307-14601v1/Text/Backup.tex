\chapter[Dynamic structure factor near the QCD critical point]{Dynamic structure factor near the QCD critical point}
\label{chapter6}  
This chapter is based on the publication as shown in Ref.\cite{Hasan2}. Here, we calculate the dynamic structure factor, which is nothing but the dynamic density-density correlator. This is evaluated from the linear analysis of the hydrodynamic equations, and we will see its behaviour near the QCD critical point.
\section{Introduction}
\label{sec0601}
Extensive efforts have been given to search for the CEP in the Beam Energy Scan (BES) program 
in Relativistic Heavy Ion Collider (RHIC) at Brookhaven National Laboratory (BNL) to create 
fireball at different $\mu$ and $T$ by varying the centre of mass energies $(\sqrt{s})$~\cite{STAR2010}. 
One of the most promising signatures of the CEP is a non-monotonic behaviour of beam energy dependence of 
higher-order cumulants of the baryon fluctuations reflected through the net proton production.
The net proton yield has been  calculated by using both QCD based models 
~\cite{Stephanov:1998dy,Stephanov:1999zu,Stephanov:2008qz,Stephanov:2011pb}
and  gauge/gravity correspondence~\cite{Critelli}. 
It is found in Ref.~\cite{Dore} that the path to the critical point is 
influenced by far from equilibrium initial conditions leading to
dramatically different $(T,\mu)$ due to viscous effects.
When the CEP is approached, the fluctuations become very large, 
which is related to the diverging nature of the correlation length ($\xi$). 
However, due to the critical slowing down, correlation length does not grow as much and limits within 2-3 fm at most~\cite{Berdnikov:1999ph}. 
The effect of critical slowing down has been taken into account with the slow hydrodynamic modes, recently developed by Stephanov {\it{et al}}~\cite{Stephanov:2017ghc}. 
Recently it has been shown~\cite{Hasan1} that in the presence of the CEP the QGP will suppress all the waves having 
finite wavelengths.

%propagate through it in presence of the CEP.
%the threshold wavelengves with the larger wavelength, $\lambda \ge \lambda_{th}$, 
%can propagate through the medium and the smaller ones gets suppressed~\cite{hasan}. 
%The calculations were carried out in a static 
%background. We found that near the CEP, the $\lambda_{th}$ is itself 
%diverging implies all the waves of different modes will be suppressed, 
%irrespective of the values of transport coefficients. Therefore, for the slow modes also, 
%we expect similar suppression in the static background and therefore do not contribute to our 
%present study.
 
%Different points of the QCD phase diagram in the $T- \mu$ plane can be reached experimentally by 
%varying the energy of the colliding nuclei. The
%deconfined, {\it i.e} the QGP systems with different $T$ and $\mu$ can be produced 
%by colliding heavy ions with different energies~\cite{Aggarwal:2010cw}. 
%These
%systems will follow different trajectories in $T- \mu$ plane while
%making a transition from QGP to hadrons due to cooling caused by 
%expansion. 
%Parallel to the theoretical endeavour, experimental efforts are on to explore these
%trajectories passing through/near the CEP by varying beam energy.

It is well-known that the density-density correlation length diverges at the critical point. The effects of the divergence 
on the baryon number fluctuations, particle correlations and correlations of density  
fluctuations will have a better chance to be detected provided the fluctuation survives the 
evolution of the hadronic phase. 
It is important to understand the correlations of these fluctuations theoretically for identifying signatures of the CEP in data. The fluid dynamical descriptions of the system are valid when the ratio, $\xi/\lambda<<1$. However, at the
CEP $\xi$ diverges resulting in the break down of the fluid dynamics. However, it is possible to identify a region near the CEP where
the fluid dynamics remain valid and can be used to understand the properties of the system~\cite{Stanley}. The effects of CEP on the evolution of matter go as input through the equation of state (EoS) and via the critical behaviour of the transport coefficients and the thermodynamic response functions of the fluid.



The correlation of density fluctuations can be investigated through
the spectral structure ($\Snn$) in Fourier space near the CEP containing
the Rayleigh (R) and Brillouin (B) peaks corresponding to thermal 
and mechanical fluctuations respectively~\cite{Stanley,Linda}.
The spectral function has been studied experimentally
in the condensed matter physics laboratories to estimate the speed of sound ($c_s$) by using scattering of light. 
The position of B-peaks in the $\Snn$ is
connected to $c_s$. The integrated intensities under  
the R and B peaks and their widths are connected to various transport coefficients like
thermal conductivity ($\kappa$), shear ($\eta$), and bulk ($\zeta$) 
viscosities and response functions like specific heats at constant volume 
($C_V$) and pressure ($C_P$). Sadly, such external probes are not available to examine the properties of QCD matter near the CEP. 



The dynamical spectral structure, $\Snn$ has been estimated in Ref.~\cite{Minami} 
but the effect of EoS containing the critical point was ignored there.  
It has been shown in this work that the EoS plays a vital role in determining the behaviour of $\Snn$, especially its strength at the R peak
changes by several orders of magnitude when the effects of the CEP in EoS is incorporated.
The EoS has strong effects on B-peaks too because it
determines $c_s$ and hence the location of the B peaks. 
It is also very important to understand whether all the hydrodynamic modes travel at the same speed or not. The R-peak and the B-peaks will be closer for slower modes even at points away from the CEP. Therefore, the structure of $\Snn$ will shed light on the speed of the perturbation propagating as a sound wave.



