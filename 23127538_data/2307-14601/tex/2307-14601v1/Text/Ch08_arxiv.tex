\chapter[Understanding the parametric slow mode through the dynamic structure factor]{Understanding the parametric slow mode through the dynamic structure factor}
\label{chapter8}  
\section{Introduction}
\label{sec0801}
This chapter is based on the arXiv preprint~\cite{Sarwar:2022iem}. The background fluid QGP is described by the hydrodynamic equations of second-order. Without considering the effect of critical point, hydrodynamic modelling of such collisions have been developed to a certain degree of sophistication~\cite{Lupeidu2021,Bleicher:2012qve,Aguiar2007,Nonaka}. The hydrodynamics is used to describe the slowly evolving
modes of the macroscopic system while the faster     
non-hydrodynamic modes are driven by the collision dynamics at the 
microscopic level. The time required to achieve local equilibrium
is much shorter than the time required to attain global 
equilibrium, and applicability of hydrodynamics depends on the separation of this time scale. 
The hydrodynamic modelling has been used to study the evolution 
of the fireball created in relativistic nuclear collisions with the 
inclusion of the CEP~\cite{Lupeidu2021,Bleicher:2012qve,Aguiar2007,Nonaka}, 
although strictly speaking it breaks down  near the  CEP
~\cite{Stephanov:2017ghc}



The underlying reason of the failure of validity of hydrodynamics near the CEP~\cite{Stephanov:2017ghc,Minami} happens due to long range correlations and enhanced fluctuations~\cite{AsakawaPhysRevLett.85.2072,Jeon2003eventbyevent,Koch2008hadronic}. In other words, if the system experience the CEP, the correlation 
length ($\xi$) diverges and the relaxation time
which evolves as $\sim \xi^3$, also diverges leading to critical slowing down
~\cite{Berdnikov:1999ph,Stanley,SKMa}. 
Consequently, for the system stays away from local thermal equilibrium, the  first-order or the second-order hydrodynamical model may become inapplicable.  
However, hydrodynamics can still be applied for systems far away
from equilibrium by including the higher order gradients
of hydrodynamic fields. This is a very active field of contemporary research
but beyond the scope of present work (we refer to ~\cite{Romatschke:2017ejr} 
and references therein for details).


It has been shown in Ref.~\cite{Stephanov:2017ghc} that the validity of 
the hydrodynamics can be extended near CEP by introducing a slowly evolving scalar 
non-hydrodynamic field or the slow out-of-equilibrium mode (OEM),  $\phi$, which is
subsequently incorporated in the definition
of the entropy along with other hydrodynamical variables.
The slow OEM are not treated 
separately~\cite{Stephanov:2017ghc,Rajagopal2020}
as they are coupled with hydrodynamic modes.  


It is important to point out here that the description of 
far-from-equilibrium conformal systems 
may converge to hydrodynamic evolution~\cite{Heller:2015dha, Romatschke:2017vte, Strickland:2017kux, Denicol:2020eij, Soloviev:2021lhs}.
However, this may not possible for  general  non-conformal systems ~\cite{Chattopadhyay:2021ive, Jaiswal:2021uvv} 
as the gradient expansion is not always convergent~\cite{ Heller:2021oxl}. 
The above mentioned gradient expansion 
of the hydrodynamic field may not be suitable for systems with large fluctuations  near the CEP  where the OEM
relax slowly.
Instead, a separate treatment of these modes will allow to avoid the problem of convergence of gradient 
expansion of hydrodynamic fields.
The presence of slow modes and its coupling with different hydrodynamic fields 
allow the study of the fluctuations of large magnitudes near the CEP. 
In the present work, the role of $\phi$ on $\Snn$ will be studied 
using relativistic causal dissipative hydrodynamics proposed by M\"uller-Israel-Stewart (MIS)~\cite{Israel:1976tn}. 
The dispersion relations for the hydrodynamic
modes are governed by the set of hydrodynamic equations and  
the extensivity of thermodynamics restricts the coupling of 
slow modes with gradients of other hydrodynamic fields. 

%This drives the system linger in its relaxation to its equilibrium, termed as the `critical slowing down'\cite{Berdnikov:1999ph,Stanley,SKMa}. They are out-of-equilibrium modes (OEM), also called slow modes, are couple with the hydrodynamic modes and may not be treated separately~\cite{Stephanov:2017ghc,Rajagopal2020,Lupeidu2021}. In Ref.~\cite{Stephanov:2017ghc}, it has been shown that the validity of the hydrodynamics may be extended near the CEP by introducing a scalar non-hydrodynamic field. 


% If the OEM with non-hydrodynamic nature in its spatial-scales that causes the hydrodynamic gradient expansion fail can be treated separately then a combined hydrodynamic modelling can be possible to include the effect of the critical point ???????????. The effect of such fluctuations in the hydrodynamic spatial scales may be well taken care of by considering the enhancement of the hydrodynamic fluctuations as dictated by the critical point behaviour of equation of state and transport coefficients, with contribution from those slow mode in the presence of couplings of modes. Presence of such slow modes are particularly very crucial for affecting the conserved charge fluctuations. 
 
% The hydrodynamics deals with the slowly evolving modes of the macroscopic system, whereas the fast non-hydrodynamic modes are driven by the collision dynamics at the microscopic level. Hydrodynamics can be used to systems where local equilibrium has set in. The time scale to reach the local equilibrium is way much shorter than the time necessary to attain the global equilibrium and there is the separation of time scale where hydrodynamics is applicable. At the CEP, the correlation length $(\xi)$ blows up which leads to the divergence of the time scale, and the hydrodynamics becomes inapplicable. In other words, the relaxation time which evolves as $\sim \xi^3$, also enhances leading to the critical slowing down. 
 
 
% In such a situation, a new variable, $\phi$ as a non-hydrodynamic mode is included in the definition of entropy along with the other hydrodynamical modes to increase the region of validity of the hydrodynamics near the CEP~\cite{Stephanov:2017ghc}. In this regards, it is crucial to understand the role of $\phi$ to study the critical point dynamics. In this chapter, we will discuss the effect of $\phi$ on the dynamic structure factor (discussed in the previous chapter) using M\"{u}ller-Israel-Stewart (MIS) theory~\cite{Israel:1976tn}. 
 
 In condensed matter system, the fluctuations has been extensively studied, specifically, the  spectral structure of the correlation of static as well as dynamical density fluctuations, also called the structure factor ($\Snn$). It has been shown both theoretically and experimentally that $\Snn$ depends on the transport coefficients of the system~\cite{Stanley, SKMa,Linda}. The thermal fluctuations are controlled by the the transport properties of the medium following the Onsager's hypothesis~\cite{OnsagarPhysRev.37.405}. The properties of the fluctuation are studied by the light scattering and neutron scatterings experiments. The scattered light spectrum contains separately identifiable Lorentzian peaks, called Rayleigh (R) peak~\cite{Rayleigh1881} situated at angular frequency, $\omega=0$, and two Brillouin (B) peaks symmetrically located about the R-peak, was first experimentally detected by Fleury and Boon~\cite{FlerryandBoon1969}. 
The R-line is originates from the temperature fluctuations at constant pressure, whereas the B-lines originate from the pressure fluctuations at constant entropy. The width and the integrated intensities of the R-peak and B-peaks provide information about various thermodynamic susceptibilities and transport coefficients as discussed in Ch.\ref{chapter6}. As some of the transport coefficients and response functions
change drastically, the study of the structure factor is very useful
to understand the behaviour of the fluid near the CEP where 
the B-peaks tend to vanish, giving rise to severe modification in $\Snn$. Since the width of a distribution represents the decay rate of the fluctuation a distribution of narrow width indicate the slower decay of the fluctuation.
 
 
 The CEP in QCD belongs to the same universality class~\cite{Halasz:1998qr,BERGES1999215}, $\mathcal{O}(4)$, as that of 3D Ising model and the liquid-gas critical point. The universality hypothesis is understood from the long ranged behaviour of the system and is independent on the microscopic characteristic. Therefore, understanding the slow modes in one system can be a useful guidance for the theoretical modelling relevant to the CEP of the QCD phase diagram. Near the CEP, the dynamic structure factor without any slow mode has been quite adequately studied earlier~\cite{Minami,Hasan2}. The static and dynamic  structure factors have also been 
studied for QCD matter near CEP with stochastic diffusion dynamics of net 
baryon number~\cite{Nahrgang:2018afz,Nahrgang:2020yxm,Pihan:2022xcl}. 
In this chapter, specifically, we will study the role of the slow modes on the Rayleigh and Brillouin peaks of $\Snn$.



By definition of the critical point the quantity,
$\left(\partial P/\partial V\right)_{\text{critical point}}$
approaches zero, leading to the divergence
of the isothermal compressibility ($\kappa_T\sim
\left(\partial V/\partial P\right)_T$)
and hence reulting in large fluctuation in density,
as $\overline{\left(\delta n\right)^2}\propto
\kappa_T$). This large fluctuation may reflect on
experimental observables.
Thermal fluctuations can cause entropy production  leading  to 
fluctuation in multiplicity, however, their ensemble average is 
unaffected~\cite{Nagai2016}. These fluctuations may not influence 
the lower flow harmonics but affect their correlation. The CMS 
collaborations have measured higher flow harmonics which again carry 
a strong signature of thermal fluctuations~\cite{CMS:2013jlh}, indicating 
that the measurement of fluctuations in multiplicity and higher harmonics 
as a function of collision energy at various rapidity bins may provide 
signal of the CEP (see Ref.~\cite{Yin:2018ejt} for a recent review). 
An enhancement of event-by-event fluctuation in net proton due
to the CEP has been observed in Ref.~\cite{Herold:2014zoa}.
The explicit dependence of the correlation of the multiplicity fluctuation 
on the hydrodynamic properties (speed of sound and         
shear viscosity) of the fluid in rapidity space has been shown~\cite{Kapusta:2011gt} 
within the scope of longitudinally expanding
system with boost invariance~\cite{Bjorken}. It will be
interesting to study the same correlation with the inclusion of the CEP
(as both the viscosity and speed of sound get affected by the CEP)
by using full (3+1) dimensional expansion within the ambit of
relativistic second-order viscous hydrodynamics.
The effects of out of equilibrium dynamics and
space-time inhomogeneity on the kurtosis of net baryon and $\sigma$ field
containing the signal of the CEP have been studied in Refs.~\cite{Herold:2016uvv}.

 
 
%Thermal fluctuations enhances the entropy fluctuation in the system, which leads to fluctuation in multiplicity, keeping their ensemble average unaffected~\cite{Nagai2016}. Thermal fluctuations may not affect lower flow harmonics  but affect their correlation. Recently, CMS collaborations have measured higher flow harmonics which again carry a strong signature of thermal  fluctuations~\cite{CMS:2013jlh} and hence that of CEP.
 
 % In condensed matter physics, the theory of fluctuations in fluids in thermodynamic equilibrium is well developed. Specifically, the intensity of density fluctuations is proportional to the isothermal compressibility ($\kappa_{T}$)~\cite{Stanley, SKMa,Linda}. Moreover, the dissipation of thermally excited fluctuations is governed by Onsager's hypothesis, says that the decay rates of the fluctuations is influenced by the transport coefficients of the medium~\cite{OnsagarPhysRev.37.405}. From the experimental point of view, such fluctuations can be investigated experimentally by light scattering and neutron scattering. It is well known that non-equilibrium fluids can equipped with large fluctuations. A microscopic picture of non-equilibrium phenomena in fluids was proposed by Bogoliubov~\cite{Bogoliubov}. It was based on a postulate that a fluid away from equilibrium would succeedingly proceed to a thermodynamic equilibrium in two distinct stages. Firstly, a microscopic kinetic stage, which is related to the collision rate of the system, after which local equilibrium is established. Secondly, a macroscopic hydrodynamic stage during which the dynamics of the fluid fluid is in accord with the hydrodynamic equations. The idea behind such postulates is that no long-ranged dynamic correlations would be present in a fluid of short-ranged forces, unless the system would be near hydrodynamic instability.
 
 
 
 %However, it turns out that the slow hydrodynamic modes (or OEM), associated with the conserved quantities and the fast modes associated with non-conserved quantities are not independent, but can interact to cause a coupling between modes resulting in long-ranged dynamic correlations~\cite{Sengers1965,Zarate,MCT_Bouchaud,MCT}. The classical picture of short-ranged correlations first became inadequate in fluids near the critical point due to the diverging nature of thermal conductivity,  and could not be explained by the Van Hove theory~\cite{VanHovePRL}, which finally led to develop of the mode-coupling theory (MCT) of critical dynamics~\cite{Fixman1967,Kadanoff1968,Kawasaki1976}. MCT also predicts the existence of long-ranged fluctuations in fluids due to the presence of stationary non-equilibrium states. Specifically, when a fluid is placed to a stationary temperature gradient, causes a coupling between the component of the velocity fluctuations and the temperature fluctuations, leading to divergence of the fluctuations in the limit of small wave numbers ($k$). The dependence of the non-equilibrium fluctuations as a function of the wave number is now believed to be a general feature of fluctuations in fluids in stationary non-equilibrium states. 
 
 
 
 %In this regards, it is very important to understand the role of OEM in the critical point dynamics. As discussed earlier that light scattering and neutron scattering are performed to investigate the properties of fluctuations experimentally. On the contrary of RHIC-E, such investigations are not possible. Fortunately, the universal behaviour of a large class of systems are independent of the dynamical details of the system is a well known feature in studying the critical phenomena. Since, the CEP in QCD belongs to same universality class, $\mathcal{O}(4)$~\cite{Halasz:1998qr,BERGES1999215}, as that of liquid-gas critical point, the role of such slow modes can be tested in liquid-gas system, and the knowledge can be very helpful to guide the theoretical modelling relevant to the CEP of QCD. For such investigations theoretical identifications for its qualitative and quantitative effects are required. The spectral structure of the fluctuations near the CEP is a suitable observable for such study~\cite{Minami,Hasan2}. The role of slow modes in the spectral structure will be good guide for such purpose. 
%%%%
%%%%%
%%%%%
\section{Hydrodynamic framework}
\label{sec0802}
 In this chapter, we use the Eckart's frame of reference as discussed in Sec.\ref{sec0303} in Ch.\ref{chapter3}, where it is considered that the heat flux is non-zero but the particle current is zero. Therefore, energy-momentum tensor (EMT) and the the particle current ($N^\mu$) are given by Eqs.\eqref{eq0129} and \eqref{eq0130}. The conservations of EMT and the net baryon number follows the Eq.\eqref{eq0349}.
 
 
 The equation for relaxation including the additional scalar soft mode $\phi$ is introduced as~\cite{Stephanov:2017ghc}:
\begin{equation}
D\phi=-F_{\phi}+A_{\phi}\theta\,,
\end{equation}
where, $\theta=\partial_{\mu}u^{\mu}$ and $D=u^{\mu}\partial_{\mu}$. Forms of $F_{\phi}$ and $A_{\phi}$ are obtained by the second law of thermodynamics: 
\beqa
\partial_{\mu}s^{\mu}\geq 0 \,,
\eeqa
where, $s^{\mu}$ is identified as the entropy four current and is defined as
\beqa
s^{\mu}=s u^{\mu}+\Delta s^{\mu}\,.
\eeqa
In the hydro+ formalism~\cite{Stephanov:2017ghc},  the partially equilibrated entropy gets an extra contribution from the soft mode $\phi$, and the change in entropy density can be written as
\begin{equation}
ds_{+}=\beta_{+}d\epsilon-\alpha_{+} dn-\pi d\phi\,,
\label{eq0805}
\end{equation}

where, $\epsilon$ is the energy density, $n$ is the baryon number density, and $\pi$ is the cost in energy for including $\phi$ modes in the system, called the corresponding chemical potential. Here,  $\beta_{+}=1/T$ and $\alpha_{+}=\mu_{+}/T$, where, $T$ is the temperature 
$\mu_{+}$ is the chemical potential. 

 
In the present work it is found that, imposition of the extensivity condition plays an important role to determine the form of $A_{\phi}$. From here onward, for simplicity we drop the subscript ``$+$" and write the following relations as obtained from the extensivity condition discussed in Appendix~\ref{extc}, in the presence of $\phi$ as:
\beqa
\beta\, dP&=&-(\epsilon+P) d\beta+n d\alpha+\phi d \pi\,,\\
%%
s&=&\beta (\epsilon+P-\mu n)-\pi \phi\,.
\label{eq0811}
\eeqa
The equation for the slow mode is a form of relaxation type, therefore, keeping up to first order of slowly evolving soft mode is adequate to maintain causality. Therefore,
\begin{equation}
\partial_\mu s^{\mu}=(F_{\phi}-b  \partial_{\mu} q^{\mu})\pi -q^{\mu}\partial_{\mu}(\beta+b \pi)-\beta (\partial_{\mu} u_{\nu})\Delta T^{\mu\nu}+\partial_{\mu}(\Delta s^{\mu}+\beta q^{\mu}+b \pi q^{\mu})+S_n \theta \,,
\end{equation}
where, $b$ represents the coupling strength and 
\begin{equation}
S_n=s-\beta(\epsilon+P)+\mu \beta n+\pi A_{\phi}\,.
\label{eq0813}
\end{equation}
The soft mode couples with the heat flux satisfying $\partial_{\mu}s^{\mu}\geq 0$. Now, with $S_n=0$ and redefining $\pi$, $q^{\mu}$, $\Delta T^{\mu\nu}$ and $s^{\mu}$, we have
\beqa
\Delta s^{\mu}=-\beta q^{\mu} -b \pi q^{\mu}\,,
\eeqa 
and
\beqa
q^{\mu}&=&-\kappa T[D u^{\mu}-\frac{1}{\beta}\Delta^{\mu\nu}\partial_{\nu}(\beta+b \pi)]\,,\\
%%
F_{\phi}&=&\gamma\pi -b \partial_{\mu}[ \kappa T D u^{\mu}-\frac{\kappa}{\beta}\Delta^{\mu\nu}\partial_{\nu}(\beta+b \pi)]\,,
\eeqa
where, the proportionality constants $\kappa\ge 0$, $\gamma\ge 0$ are respectively 
thermal conductivity and relaxation rate of slow modes. 
Along with the above expression of fluxes one also needs,  
$S_n=c \, \theta$ with $c \, \ge 0$ being a constant to
satisfy the relation $\partial_{\mu}s^{\mu}\geq 0$.  
However, since, $S_n$ contains local thermodynamic quantities at zeroth 
order in derivative, it can not be a function of the four divergence of 
fluid velocity, $\theta$. In other words, the relation 
$S_n=c\, \theta$ with non-zero $c$  will make the local equilibrium  
entropy density in fluid rest frame 
a function of the dissipative gradient of  the fluid velocity which is frame dependent. 
In that case, the local equilibrium entropy will be frame dependent which is contradictory 
to its definition as a frame independent local thermodynamic quantity, so $c$ should be zero, 
implies  $S_n=0$. Therefore, the dissipative fluxes (in Eqs.~\eqref{eq0142}-\eqref{eq0144}) get modified with the coupling to soft modes in the MIS theory as:


%where, the constants of proportionality $\kappa\ge 0$ and $\gamma\ge 0$ respectively are the thermal conductivity and the relaxation rate of slow modes. Comparing of Eqs.\eqref{eq0811} and \eqref{eq0813} infers $A_{\phi}=\phi$ for $S_n=0$. Therefore, the dissipative fluxes (in Eq.~\eqref{eq0603}) get modified with the coupling to soft modes in the MIS theory as:
\begin{subequations}
\begin{eqnarray}
\label{eq0812a}
&&\Pi=-\frac{1}{3}\zeta\Big[\partial_{\mu}u^{\mu}+\beta_0 D\Pi-\tilde{\alpha}_0\partial_{\mu}q^{\mu}\Big]\,,\\
\label{eq0812b}
&& q^{\mu}=-\kappa T \Delta^{\mu\nu}\Big[\beta\partial_{\nu} (T+b \pi)+Du_{\nu}+\beta_1 D q_{\nu}-\tilde{\alpha}_0\partial_{\nu}\Pi -\tilde{\alpha}_1\partial_{\lambda} \pi^{\lambda}_{\nu}\Big]\,,\\
\label{eq0812c}
&&\pi^{\mu\nu}=-2\eta\Big[\Delta^{\mu\nu\rho\lambda}\partial_{\rho}u_{\lambda}+\beta_2 D \pi^{\mu\nu}-\tilde{\alpha}_1\Delta^{\mu\nu\rho\lambda}\partial_{\rho}q_{\lambda}\Big]\,,
%\label{eq0813}
\end{eqnarray}
\end{subequations}
with proportionality constants $\eta\ge0$, $\zeta\ge0$, where $\eta$ and $\zeta$ are respectively the shear and bulk viscous coefficients. The coefficients $\tilde{\alpha_{0}}, \tilde{\alpha_{1}}, \tilde{\beta_{1}}$ are the relaxation and coupling coefficients, and the expressions for those coefficients are given in Appendix-\ref{appendix01_A}, which are evaluated through the thermodynamic integral from kinetic theory prescription.~\cite{Israel:1979wp}. 

\subsection{Linearized equations and the dynamic structure factor}
To linearize the hydrodynamic equations for small deviations from equilibrium field quantities, we assume $\mathcal{Q}=\mathcal{Q}_{0}+\mathcal{\delta Q}$, 
where $\mathcal{Q}$, $\mathcal{Q}_{0}$ and $\delta\mathcal{Q}$ 
represent general hydrodynamic variables, their average values and fluctuations respectively.  
$\mathcal{Q}_{0}=0$ is its average value for dissipative degrees 
of freedom and $u_0^{\mu}=(-1,\, 0,\, 0,\, 0)$ and $\delta u^{\mu}=(0,\,\delta \vec{u})$. 
%%%%%%%%%%%%%
The fluxes given in Eq.~\eqref{eq0813} are 
substited in Eq.~\eqref{eq0129} and solved in the linearized
domain in frequency-wave vector space to obtain $\Snn$. 
In the linearized domain, the equations of motion for different
space-time components of the energy-momentum tensor and baryonic 
charge current obtained from Eqs.~\eqref{eq0349} read as:  
\begin{subequations}
\begin{eqnarray}
%0&=& \frac{\partial \delta \epsilon}{\partial t}-(\epsilon_0+P_0)+\vec{\nabla}\cdot \delta \vec{u}-\vec{\nabla} \cdot \delta \vec{ q}, \\
0&=&-\frac{\partial \delta \epsilon}{\partial t}-(\epsilon_0+P_0)+\vec{\nabla}\cdot \delta \vec{u}-\vec{\nabla} \cdot \delta \vec{ q}\,, \\
   %%
 0&=&  -(\epsilon_0+P_0)\frac{\partial}{\partial t} \delta u^{i}-\partial^{i}(\delta P+\delta \Pi)+\frac{\partial}{\partial t} \delta q^{i}-\partial_{j}\Pi^{ij}\,,\\
   %%%
 0&=&  -\frac{\partial}{\partial t} \delta n-n_0\vec{\nabla}\cdot \delta \vec{u}\,,\\
   %%%%
 0&=&  \delta \Pi+\frac{1}{3}\zeta [\vec{\nabla}\cdot \delta \vec{u}+\beta_0\frac{\partial}{\partial t}\delta \Pi-\tilde{\alpha_0}\vec{\nabla}\cdot \delta \vec{q}]\,,\\
   %%
0&=&   \delta q^{i}+\kappa T_0\nabla^{i}\delta T+\kappa T_0 \frac{\partial}{\partial t} \delta u^{i}+\kappa T_0 \beta_1 
   \frac{\partial}{\partial t}\delta q^{i}-\kappa T_0 \tilde{\alpha}_0\nabla^{i}\delta \Pi -\kappa T_0\tilde{\alpha}_1\nabla_{j}\pi^{ij}\,,\\
   %%%%
0&=&   \delta \pi^{ij}+2\eta\delta^{ijlm}(\partial_{l}\delta u_{m}-\tilde{\alpha}_1\partial_{l}\delta q_{m})+2\eta \beta_2\frac{\partial}{\partial t} \delta \pi^{ij}\,,\\
   %%%
 0&=& - \frac{\partial}{\partial t}\delta \phi-(\gamma+T_0^2\frac{K_{q\pi}^2}{\kappa}\nabla^2)C_{\phi\pi}\delta \phi -
   \big[\gamma+(T_0^2\frac{K_{q\pi}^2}{\kappa}-\frac{K_{q\pi}}{C_{T \pi}})\nabla^2\big] C_{T \pi}\delta T\nn\\
   &&-(\gamma+T_0^2\frac{K_{q\pi}}{\kappa}\nabla^2)C_{\pi n}\delta n-T_0K_{q\pi}\frac{\partial}{\partial t}(\nabla_{i}\delta u^{i})+\tilde{\phi}\nabla_{i}\delta u^{i},
\end{eqnarray}
%nd $u_0^{\mu}=(-1, 0, 0, 0)$ and $\delta u^{\mu}=(0,\delta \vec{u})$. 
\end{subequations}
where $K_{q\pi}=b \kappa$ and
\beqa
C_{ A\pi}=\frac{\partial \pi}{\partial A}, \,\,\,\, \text{with}\,\,\, A\equiv(T,n,\phi)\,.
\eeqa
%$\displaystyle{\lim_{x \to \infty}} 
%\int d^3x\int_{0}^{\infty} \exp[{(z-\epsilon)t-i\vec{k}\cdot\vec{x}}]$ 
We get a set of linear equations  
in $\omega-k$ space (Appendix~\ref{appendixA})
by taking the Fourier-Laplace transformation
of the above equations. 
The set of equations given in Eqs. \eqref{eq23}-\eqref{eq33} 
can be written in matrix form as:
\beqa
\mathcal{M}\delta\mathcal{Q}=\mathcal{A}\,,
\eeqa
where $\mathcal{M}$, is an $11 \times 11$ matrix representing the
coefficients of the column vector comprinsing the
quantities, $\delta n,\delta T,\delta u_{||}, \delta u_{\perp},\delta \Pi, 
\delta q_{||}$, $\delta q_{\perp},\delta \pi_{||\,||},\delta \pi_{||\,\perp},
\delta \pi_{\perp\,\perp}$, and $\delta \phi$.
The solution of the set of linear equations can be written  as:
\begin{equation}
\delta\mathcal{Q}=\mathcal{M}^{-1}\mathcal{A}\,,
\label{delQ}
\end{equation}
In the present work we are interested in evaluating the two point 
correlation of density fluctuation. 
The solution of the set of equations represented by Eq.~\eqref{delQ} leads to the 
following expression for density fluctuation ($\delta n$),
\beqa
\delta n (\vec{k},\omega)&=& \big[\epsilon_n \mathcal{M}^{-1}_{12}+\mathcal{M}^{-1}_{11}\big]\delta {n}(\vec{k},0)-\epsilon_T \big[\mathcal{M}^{-1}_{12}\big]\delta {T}(\vec{k},0)\nn\\
%
&+&\big[\epsilon_{\phi } \mathcal{M}^{-1}_{12}+\mathcal{M}^{-1}_{13}\big]\delta {\phi }(\vec{k},0)\nonumber\\
&+&\big[\epsilon_0 \mathcal{M}^{-1}_{14}-i k T_0 \mathcal{M}^{-1}_{13} \kappa _{\text{q$\pi $}}+\mathcal{M}^{-1}_{14} P_0-T_0 \chi \mathcal{M}^{-1}_{15}\big] \delta {u_{||}}(\vec{k},0)\nonumber\\
%
&-&\big[\beta _1 T_0 \chi  \mathcal{M}^{-1}_{15}+\mathcal{M}^{-1}_{14}\big]\delta {q_{||}}(\vec{k},0) +\frac{1}{3} \zeta \mathcal{M}^{-1}_{16}\beta _0 \delta {\pi }(\vec{k},0)\nn\\
&-&2 \eta \mathcal{M}^{-1}_{17}\beta _2 \delta { \pi }_{||||}(\vec{k},0)-2\eta \mathcal{M}^{-1}_{18}\beta _2{\pi }_{\perp \perp}(\vec{k},0)\nn\\
&-&\big[\beta _1 T_0 \chi \mathcal{M}^{-1}_{110}+\mathcal{M}^{-1}_{19}\big]\delta {q}_{\perp}(\vec{k},0)-2\eta \mathcal{M}^{-1}_{111}\beta _2 \delta {\pi }_{||\perp}(\vec{k},0),\nn\\
&+&\big[\epsilon_0 \mathcal{M}^{-1}_{19}+\mathcal{M}^{-1}_{19} P_0-T_0 \chi \mathcal{M}^{-1}_{110}\big]\delta {u}_{\perp}(\vec{k},0)\,,
\label{spfn}
\eeqa
where, $\epsilon_{x}=\frac{\partial \epsilon}{\partial x}$ and $x=n$,$T$, $\phi$. 

%%\beqa
%\delta n(\vec{k},\omega)=\mathcal{M}^{-1}_{11}\big[-\big(\frac{\pd{\epsilon}}{\pd n}\big) \delta n (\vec{k}, 0)-\big(\frac{\pd{\epsilon}}{\pd T}\big) \delta T (\vec{k}, 0)-\big(\frac{\pd{\epsilon}}{\pd \phi}\big) \delta \phi (\vec{k}, 0)\Big]-\mathcal{M}^{-1}_{14}\big[-\delta n(\vec{k}, 0)\Big].
%\eeqa



Now we define the correlation of density fluctuations, 
as $\mathcal{S^\prime}_{nn}(\vec{k},\omega)$ by the following expression~\cite{Stanley}: 
\beqa
\mathcal{S^\prime}_{nn}(\vec{k},\omega)=\Big< \delta n(\vec{k},\omega)\delta {n}(\vec{k},\,0)\Big>\,.
\label{eq38}
\eeqa
Since the correlation between two independent thermodynamic variables, say, $\mathcal{Q}_i$ and $\mathcal{Q}_j$  vanishes, i.e. we have,
\beqa
\Big< \delta \mathcal{Q}_{i}(\vec{k},\omega)\delta {\mathcal{Q}}_{j}(\vec{k},\,0)\Big>=0, \,\,\,\, i\neq j \,.
\label{eq39}
\eeqa
Imposing Eq.\eqref{eq39} into Eq.\eqref{spfn}, the $\mathcal{S}^{'}_{nn}$ can only contain two elements of the matrix $\mathcal{M}^{-1}$ as: 
\beqa
\mathcal{S}^{'}_{nn}(\vec{k}, \omega)=-\Big[\big(\frac{\pd{\epsilon}}{\pd n}\big) \mathcal{M}^{-1}_{12}-\mathcal{M}^{-1}_{11}\Big] \Big<{ \delta {n}}(\vec{k},\, 0){\delta {n}}(\vec{k},\,0)\Big>\,,
\eeqa
The final expression for the $\Snn$ can be obtained as:
\beqa
\mathcal{S}_{nn}(\vec{k}, \omega)=\frac{\mathcal{S}^{'}_{nn}(\vec{k}, \omega)}{\Big<{ \delta {n}}(\vec{k},\,0){\delta {n}}(\vec{k},\,0)\Big>}=-\Big[\big(\frac{\pd{\epsilon}}{\pd n}\big) \mathcal{M}^{-1}_{12}-\mathcal{M}^{-1}_{11}\Big]\,.
\eeqa
This is the correlation  in density fluctuation 
or the power spectrum with the inclusion 
of the extra degree of freedom, $\phi$.
%%%
\section{Results and discussion}
\label{sec0803}
% Figure environment removed

We aim to study the dynamic structure factor near the CEP by introducing a slow scalar field. The effect of the CEP is taken through an EoS, containing the critical point. The various transport coefficients and the thermodynamic response functions near the CEP play important roles determining the structure factor are taken from the Eq.\eqref{eq0619}.
%%
%%
% Figure environment removed
%%
% Figure environment removed
%%%%%%%%%%%%%%%%%%%%%%%%%%
% Figure environment removed
%%%
 
%%%%
% Figure environment removed
%%%
The variation of $\Snn$ with $\omega$ is shown in Fig.\ref{fig0801}(a) for $k=0.1$ fm$^{-1}$, when the system is away from the CEP. Without the presence of $\phi$, the $\Snn$ has three peaks (see Fig.\ref{fig0601}(b)). These are identified as the  R-peak located at $\omega=0$ and the B-peaks symmetrically situated on either sides of the R-peak. While with the presence of the slow mode $\phi$, the $\Snn$  admits four peaks. The locations of the four peaks of $\Snn$ for different values of $\omega$ are obtained from the dispersion relation provided in the Appendix~\ref{appendix08_A}. The Brillouin peaks (B-peaks) are identified as the away side peaks of $\Snn$. The Stokes (left side) and the anti-Stokes (right side) components are located asymmetrically (unlike the figure shown in Fig.\ref{fig0601}) on either side of the origin with uneven magnitudes. The other two peaks (closer to $\omega=0$) 
are present even with vanishing speed of sound in the neighbourhood of CEP (as shown in Fig.\ref{fig0801}(b)), arising
due to the coupling of $\phi$ with $q^\mu$. The coupling of $\phi$ with the hydrodynamical fields 
results in one extra peak, and shifted the $R$-peak from $\omega=0$. The peak appears at  $\omega=0$ represents static thermal fluctuations 
(without time variation) on the magnitude of the fluctuations. But their appearance at non-zero $\omega$ 
represents that there is variation in the magnitude of the 
fluctuation with time. So the appearance of these peaks 
near $\omega\ne0$ is due to the 
time varying thermal fluctuations which is resulted from the coupling 
of the heat flux to slow OEM. The OEM have  much slower dynamic evolution rate, 
so that thermal fluctuations become time dependent instead of being 
time independent.
A non-relativistic fluid system 
placed in a stationary temperature gradient
showed  asymmetry 
between Stokes and anti-Stokes components,
the asymmetry  is  maximum when $k||\nabla T$ 
and vanishes with $k\perp\nabla T$. 
From Fig.\ref{fig0801}(a), it is seen that the asymmetry in Rayleigh component is exactly opposite to the Brillouin components. 
%%%%%%%%%%%%%%%%%%%
In the vicinity of the CEP (Fig.\ref{fig0801}(b)), the $\Snn$ 
shows two peaks closer to $\omega=0$ as the B-peaks disappear due to the absorption
of sound at the CEP. 



%They could be considered as R-peaks in the sense that they are appearing due to the thermal fluctuations in the system. For static thermal fluctuations {\it{i.e.}} with no-time variation of the magnitude of fluctuations, the corresponding R-peak appears at $\omega=0$. However, their appearance at non-zero $\omega$ represents that there could be variation of their magnitudes with time. So, the appearance of these peaks closer to $\omega=0$ (instead of at $\omega=0$) is due to the time varying thermal fluctuations. This time variation in thermal fluctuation happens due to the coupling of the heat flux to $\phi$ modes. %The slow non-equilibrium modes have much slower dynamic evolution rate, so that thermal fluctuations get dynamic nature in time, instead of being time independent.

 %In condensed matter physics, the asymmetry in magnitude of the B-peaks are understood by the fact that the sound modes with different $\omega$ values, $-c_sk$ and $+c_sk$ originate from different region of temperature zones~\cite{Rayeligh_Benard,Zarate}. We also observe that the asymmetry in Rayleigh component is exactly opposite to the Brillouin components. When the system is near the vicinity of  the CEP (right panel) the $\Snn$ shows only two peaks as the B-peaks merge with the R-peaks due to the absorption of sound at the CEP.
%In condensed matter physics in such situation 
%the fluid is placed in a stationary temperature gradient, 
%and showed that the asymmetry 
%between Stokes and anti-Stokes components will be maximum when $k||\nabla T$ 
%and vanishes with $k\perp\nabla T$. 
%%%%


In Fig.~\ref{fig0802} the $\Snn$ is plotted in $\omega-k$ plane when the system is away from CEP. The $\Snn$ with non-zero $\phi$ is quite similar (as shown in Fig.\ref{fig0604}) in a sense that the B-peaks are moving away from the R-peak for larger $k$-values, and their magnitudes gets smaller and smaller. However, in presence of $\phi$, these peaks may reappear at non-zero $k$ values too due to the coupling coupling of $\phi$ with the hydrodynamic fields. The symmetrically positioned B-peaks of $\Snn$ is reattained when the coupling $K_{q\pi}$ is set to zero (shown in Fig.~\ref{fig0803}). 
%%%%%%%%%%%%%5

%%%%
% Figure environment removed
%%%
% Figure environment removed
The asymmetry in the position in $\Snn$ of the B-peaks increases with  increase in $\big(\partial P/\partial\phi\big)$, represents the effect of slow modes 
on the local pressure, which accounts for the effect of slow modes in the speed of sound 
in the system (Appendix~\ref{appendix08_A}). The asymmetry in $\Snn$ with respect to $\omega$ 
increases with  increase in $\big(\partial P/\partial\phi\big)$. 
This is distinctly visible in Fig.~\ref{fig0804} in comparison
to results displayed in Fig.~\ref{fig0801}(a). 



%(which may be related to the speed of sound, $\big(\partial P/\partial\epsilon\big)$), is visible in Fig.~\ref{fig0804} in comparison to results displayed in Fig.~\ref{fig0801}(a).% The positionss of the four peaks in $\omega$ are gettable from the dispersion relation provided in the Appendix~\ref{appendixB}. 

Increasing $\phi$ modes in a system induces fluctuations in the system. In Fig.\ref{fig0805}, it is observed that the magnitudes of the peaks get enhanced due to
the enhancement in $\phi$ value. The introduction of slow modes is intended to account for the higher 
order gradients (required for system far-away from equilibrium)
which become relevant at hydrodynamic scales.


In Fig.~\ref{fig0806}, the structure factor for second-order hydrodynamics has been compared with first-order hydrodynamics. Interestingly, the structure factor for first-order hydrodynamics admits a R-peak at the origin and two symmetric B-peaks located on the opposite sides of the R-peak. The effects of $\phi$ seems to be inconsequential in-terms of the arising of extra peaks in the first-order theory because of the vanishing of various coupling and relaxation coefficients in first order approximation. However, the effect of slow modes in first-order theory is there in the asymmetry of B-peak magnitudes.

In Eq.\eqref{eq0812b} shows that the coupling of $\phi$ enters the system through $q^\mu$.
We know that first-order theory can be obtained
from second-order theory by setting the coupling coefficients ($\tilde{\alpha_0}$ and $\tilde{\alpha_1}$)
and relaxation coefficients ($\beta _0,\tilde{\beta_1}, \beta_2$) to zero in Eqs.\eqref{eq0812a}-\eqref{eq0812c}.
Therefore, the only term through which the coupling of $\phi$ enters
the first-order hydrodynamics is given by the first  term of $q^\mu$ in Eq.\eqref{eq0812c} that is,
$q^{\mu}=-\kappa T \Delta^{\mu\nu}\Big[\beta\partial_{\nu} (T+b \pi)\Big]$.
It is interesting to note that the incorporation of $\phi$ has just
shifted the temperature from $T$ to $T+b\pi$ through its chemical
potential $\pi$.  This term  does not introduce any  $\omega$ dependence,
therefore, the presence of $\phi$ in first-order hydrodynamics 
will not introduce any extra peak. 
%. Interestingly, the $\Snn$ for first-order hydrodynamics shows a single R-peak at the origin and two symmetric B-peaks positioned on the opposite sides of the R-peak. The effects of $\phi$ in the first-order theory seems to be insignificant because of the vanishing of various relaxation and coupling coefficients.


%%%%%%%%%%%%%%
Till now, it is evident that the extra-peaks are originating from coupling of heat flux with 
the scalar field in second-order theory. However, it is not clear whether, the coupling with 
the longitudinal or transverse modes or both  are responsible for
these peaks in second order theory. To verify, the $\Snn$ derived from the longitudinal (by neglecting the transverse modes) dispersion relation with and without $\phi$ has been depicted in Fig.~\ref{fig0807}. We find that the width of the peaks are narrowed down due to the inclusion of the $\phi$ field (blue dashed line). The decay rate of the thermal fluctuation is reflected through the width of the R-peak, which becomes smaller due to the introduction of slow mode $\phi$. Therefore, the decay of the fluctuation becomes relaxed in the presence of the $\phi$. In comparison of Figs.\ref{fig0801}(a), \ref{fig0803}, \ref{fig0806} and \ref{fig0807}, we can say that the extra peak appeared in Fig.~\ref{fig0801}(a) is due to the coupling of the transverse modes with $\phi$. It is also iimportant to mention that the presence of $\phi$ modes can result in the extra peak in the dynamic structure factor when the transverse modes in the causal theory of hydrodynamics is concerned.
 
 However, it may create confusion whether the absence of extra peaks at  $\phi\neq 0$ is due to the longitudinal modes only or because of first order hydrodynamics.
This can be resolved from the observation that the height asymmetry is different for these two scenarios.
    
    
In summary, we can argue that the dynamic structure factor with presence of a slow out-of-equilibrium mode changes significantly. We found four peaks of Lorentzian distribution, whereas the dynamic structure factor without any out-of-equilibrium mode shows three peaks. The peaks in $\Snn$ with $\phi$ are narrower in width compared to the peaks appeared in the $\Snn$ without any $\phi$. This narrower width of the distribution indicates the slow relaxation of the fluctuation in the system, causing critical slowing down. The extra peak appears in the $\Snn$ with $\phi$ is due to the coupling of the $\phi$ mode with the transverse modes when the second-order theory of hydrodynamics is concerned.
 
%The aim of this paper is to find out the spectral behaviour of density fluctuation near the CEP. The effect of the CEP is taken through an Equation of State (EoS), containing the critical point. The EoS is constructed from the arguments of universality hypothesis, and suggests that the CEP of the QCD belongs to same universality class $\mathcal{O}(4)$ to the critical point of 3d Ising model. For details of the construction of the EoS, see Refs.~\cite{Nonaka,Hasan1,Hasan2}. For contrast with the results near the CEP, we plot Fig.\ref{fig1}(a) to shows the variation of the dynamic structure factor with angular frequency, when the system is away from the CEP. We clearly see four distinct peaks of Lorentzian type. Two away side peaks are identified as the Brillouin peaks (B-peaks). The Stokes component (left side) and the anti-Stokes component (right side) are asymmetric in magnitudes, which is unlike the case we studied earlier~\cite{Hasan2}, where the Stokes and anti-Stokes components were symmetric without introducing any OEM. The asymmetry in the B-peaks may arise due to the local inhomogeneity present in the system. The B-peaks arise from thermally excited propagating sound waves (modes) associated with adiabatic pressure fluctuations. In condensed matter physics, the asymmetry of the B-peaks are understood from the fact that two different sound modes, $-\vec{k}, +\vec{k}$ are originating from two different regions of different temperature~\cite{Rayeligh_Benard,Zarate}. This kind of situation is rigorously tackled in Rayleigh-B\'{e}nard problem, where a fluid is placed in a stationary temperature gradient, and showed that the asymmetry between Stokes and anti-Stokes components will be maximum when $k || \Delta T$ and vanishes with $k \perp \Delta T$. Whereas, the asymmetry present in Rayleigh component (due to concentration fluctuation) is exactly opposite to the Brillouin components. 
%%






