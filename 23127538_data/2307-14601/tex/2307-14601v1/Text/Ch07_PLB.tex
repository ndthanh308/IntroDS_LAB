\chapter[Propagation of Nonlinear waves near the QCD critical point]{Propagation of Nonlinear waves near the QCD critical point}
\label{chapter7}  
\section{Introduction}
\label{sec0701}
In this chapter, we discuss about our publication, presented in Ref.~\cite{Sarwar:2020oux}. So far in Ch.\ref{chapter5}, Ch.\ref{chapter6} and Ch.\ref{chapter8}, we have discussed about the possible fate of small perturbations (under linear approximation) near the QCD critical point. But in this chapter, we discuss about the perturbations which are comparatively larger in magnitudes. In relativistic heavy-ion collider experiments (RHIC-E), the partons are produced with a wide range of transverse momentum ($p_T$). The partons with relatively smaller $p_T$, on subsequent scattering, contribute to produce a locally thermalized hot medium QGP, whereas the partons with high $p_T$ do not contribute in the medium formation, but they traverse the medium as jets with associated radiated partons, and produce disturbances inside the medium by depositing energy into it through interactions. The supersonic partons, can produce larger perturbations in the medium leading to a nonlinear wave. The quantum fluctuations may also lead to inhomogeneity in the medium, which may serve as perturbations on the hydrodynamically evolving medium. The hydrodynamic response of the medium to such perturbations, will be imprinted into the medium and those imprint are translated into the hadrons spectra at the freeze-out hyper-surface. This imprint will also have effects on other penetrating particles like photons and lepton pairs emitted throughout the evolution history. Specifically, the emergence of two maximas at $\Delta \phi=\pi \pm 1.2$ radian in the quenched away side jet~\cite{STAR:2005gfr,Wang2006}. This double-hump structure in the correlation function of jet is elucidated as the effect of the produced Mach cone in hydrodynamic response to the perturbation created by jets~\cite{Solana:2004fdk}. The flow harmonics (created due to the momentum anisotropy) of produced particle is attributed to the hydrodynamic response of the QGP to the initial geometry. During expansion and cooling of the QGP, on transition to hadronic phase, these anisotropies gets transmitted into hadronic momentum spectra for the conservation of momentum. 


%%%%%%%%%%%%%%%%%%%%%%%%%%%%%%%%%%%%%%%%%%%%%%%%%%%%%%%%%%%%%%%%%%%%%%%%%%%%%%%%
The damping of the Mach cone is related to the hydrodynamic response of the medium. Any change in the behaviour of QGP and in the nature of transition from QGP to hadron gas {\textit{e.g.}} presence of the CEP will affect the response. In general, the hydrodynamic response can also be treated as linear as well as nonlinear, depending on the magnitude of the perturbations. A small disturbance is treated as a linear perturbation~\cite{Shuryak2009,Staig:2010pn,Staig2011,Rafiei:2016zxk,Hasan1,Hasan2,Minami} 
whereas a relatively large perturbations (comparable to unperturbed value) should be treated as nonlinear perturbation~\cite{Raha1,Raha2,Raha3,Raha4,Fogaca1,Fogaca:2014gwa}. The momentum anisotropy can be accounted mostly as a linear response to the initial geometry (eccentricity) with a small contribution from nonlinear effects~\cite{Niemi2013,Giacalone2017}. Moreover, the produced Mach front has been found to travel as a shock front~\cite{AKC2006}. The propagation of this shock front is controlled by the hydrodynamic response which may be linear or nonlinear. Due to the large amplitude of this shock front one expects the perturbation to be nonlinear in nature~\cite{Osborne1946}. In Refs.~\cite{Betzthesis,BetzPRL,LiPRL} the presence of Mach cone effect on the away side jet in two particle and three particle correlations have been explained by the non-dissipative property of the nonlinear waves.


The fate of nonlinear waves in QGP has been studied by Foga\c{c}a et al.~\cite{Fogaca1,Fogaca:2014gwa} with the ambit of Naiver-Stokes (NS) as well as M\"uller-Israel-Stewart (MIS) hydrodynamics with the incorporation of the shear viscosity only. They have argued that the nonlinear waves survive, despite the presence of the shear viscosity. The propagation of nonlinear perturbation due to energy deposition by the jets into the medium have been argued to be responsible for broadening of away the side jet~\cite{Fogaca1}. The nonlinear radial spreading of localized waves (solitons) of large magnitude, which are created from the thermalization of the energy deposited by jets considered to be the cause of broadening of the peak. 


The hydrodynamic response will rely on the state of the fluid phase and also on the nature of the perturbations (linear or nonlinear). The effects of linear response has already been analysed and found to be suppressed near the CEP~\cite{Minami,Hasan1,Hasan2}. In this regard, however, it will be relevant to investigate the possible fate of the nonlinear perturbations in presence of the CEP. This is particularly crucial because physical process like the energy deposition by jets can be large enough to ignite nonlinear effects. In such cases, the nonlinear wave should play dominant role in the formation of Mach-cone. Therefore, the study of the  effects of nonlinear perturbations in QGP under the influence 
of the CEP is crucially important. 
%Moreover, the propagation of linear waves stops near 
%the CEP but propagate  when the system is away from the CEP.
%This front travels nonlinearly at the latter stage. 
%So, in that case also, the Mach cone effect will depend on the 
%survivability of the nonlinear propagation.

Near the CEP where fluctuations are expected to grow, the study of the nonlinear waves is important for understanding the broadening of the away side jet and the fate of double-hump structure in the correlation function. This could be a very good guide for the critical point search in QCD phase diagram. As a matter of fact to the best of the knowledge of the author of this dissertation, the fate of the nonlinear disturbances due to the presence of the CEP has not been reported till date. In this work, we address the propagation of nonlinear waves with the presence of the CEP, where we have considered the relevant dissipative coefficients. The inclusion of all those coefficients are crucial because some of these are known to diverge near the CEP~\cite{Kharzeev:2007wb,Karsch2008,Ryu2015,KapustaChi,Martinez2019}. However, the equations which govern the evolution of the nonlinear waves in the presence of all the relevant transport coefficients {\it i.e.} shear viscosity ($\eta$), bulk viscosity ($\zeta$) and thermal conductivity ($\kappa$) are not readily available within the scope of the MIS hydrodynamics, and hence we deduce those equations for the present work. 
%%%
\section{1-D flow equations}
\label{sec0702}
In this section, the relevant equations are derived to study the propagation of nonlinear perturbation in a fluid in $(1+1)$ dimension. We have used the MIS theory for the evolution of the QGP and adopted the Eckart's frame of reference (see Ch.\ref{chapter3}, Sec.\ref{sec0303}) to define fluid velocity, where it is considered that the heat flux is non-zero but the particle current is zero. Therefore, energy-momentum tensor (EMT) and the the particle current ($N^\mu$) are given by Eqs.\eqref{eq0129} and \eqref{eq0130}. The conservations of EMT and the net baryon number are written as follows
\beqa
\pd_{\mu}T^{\mu \nu}=0~,\,\,\,\,\,\pd_{\mu}N^{\mu}=0 ~.
\label{eq0701}
\eeqa
%%%
%%%%
The evolution equations of energy density and fluid velocity are obtained from  the Eqs.\eqref{eq0701} by taking projection along the directions parallel and perpendicular to $u^{\mu}$ as:
\beqa
\label{eq0707}
D\epsilon=-(\epsilon+P)\theta+(\partial_{\mu} u_{\nu})\Delta T^{\nu\mu}-(\partial_{\mu} q^{\mu})~,\\
\label{eq0708}
(\epsilon+P)D u^{\alpha}=\Delta^{\alpha\mu}\partial_{\mu}-\Delta^{\alpha}_{\nu}\partial_{\mu} \Delta T^{\nu\mu}~,
\eeqa
where, $u^{\mu}u_{\mu}=-1,$ and
\beqa
\theta&=&\partial_{\mu}u^{\mu}=v\gamma^3\frac{\partial v}{\partial t}+\gamma^3\frac{\partial v}{\partial x}~,\\
%%
\partial_{\mu}q^{\mu}&=&q^{x}\frac{\partial v}{\partial t}+v\frac{\partial q^{x}}{\partial t}+\frac{\partial q^{x}}{\partial x}~.
\eeqa
Here, 
\beqa
u_{\mu}q^{\nu}=0,\,\,\, u_{\mu}\pi^{\mu\nu}=0,\,\,\, g_{\mu\nu}\pi^{\mu\nu}=0,\,\,\, \pi^{xy}=\pi^{zx}=\pi^{yz}=0,\,\,\, \pi^{yy}=\pi^{zz}=\pi^{T}~,
\eeqa
which leads to, 
\beqa
\pi^{00}=v^2\pi^{xx},\,\,\, \pi^{xt}=\pi^{tx}=
v^2\pi^{xx},\,\,\, \pi^{T}=\pi^{tx}=\frac{(
v^2-1)}{2}\pi^{xx}~,
\eeqa
and 
\beqa
q^{t}=vq^{x}, \,\,\,q^{y}=q^{z}=
q^{T}~.
\eeqa
Now from the energy-momentum conservation equations {\it{i.e.} } from Eqs.~\eqref{eq0707} and~\eqref{eq0708}, we get,
\beqa
\label{eq0714}
\gamma v\Big(\frac{\partial \epsilon}{\partial t}+\frac{\partial \epsilon}{\partial x}\Big)=&& \Big[ (h+\Pi)v\gamma^3+v \gamma \pi^{xx}+q^{x}\}\frac{\partial v}{\partial t} \Big]\nn\\
&&+\Big[(h+\Pi)\gamma^3+\gamma \pi^{xx}\Big]
\frac{\partial v}{\partial x}+v\frac{\partial q^{x}}{\partial t}+\frac{\partial q^{x}}{\partial x},
\eeqa
\beqa
\label{eq0715}
&&\Big[ (h+\Pi)v\gamma^4+v \gamma^2(1-2v^2)v \pi^{xx}+\gamma^3(1+v^2)v
q^{x}\Big]\frac{\partial v}{\partial t}\nn\\
&+&\Big[ (h+\Pi)v\gamma^4-\gamma^2 v^2\pi^{xx}+\gamma v(1+2\gamma^2)q^{x}\Big]
\frac{\partial v}{\partial x}\nn\\
&=&-v^2\Big[\gamma^2\frac{\partial P_{\zeta}}{\partial t}+\frac{\gamma^2 }{v}\frac{\partial P_{\zeta}}{\partial x}
+\frac{\partial \pi^{xx}}{\partial t}+ \frac{1}{v}\frac{\partial \pi^{xx}}{\partial x}
+\frac{ \gamma}{v}\frac{\partial q^{x}}{\partial t}- \gamma\frac{\partial q^{x}}{\partial x}\Big]
\eeqa
We define $P+\Pi=P_{\zeta}$. The equation governing the net baryon number conservation reads,
\beqa
\label{eq0716}
\frac{\partial n}{\partial t}+ v\frac{\partial n}{\partial x}
= -n \gamma^2\Big[v\frac{\partial v}{\partial t}+\frac{\partial v}{\partial x}\Big]~,
\eeqa
The other three equations originating from dissipative fluxes are given by the Eqs.\eqref{eq0142}, \eqref{eq0143}, and \eqref{eq0144}, which 
can be written as:
\beqa
\label{eq0717}
\Pi+\zeta \beta_{0}\Big[\gamma \frac{\partial \Pi}{\partial t}+v\frac{\partial \Pi}{\partial x}\Big]&=&-\zeta\Big[(v\gamma^3 -\tilde{\alpha_{0}} q^{x})\frac{\partial v}{\partial t}+\gamma^3
\frac{\partial v}{\partial x}-\tilde{\alpha_{0}}(v\frac{\partial q^{x}}{\partial t}-\frac{\partial q^{x}}{\partial x})\Big]~,
\eeqa
%%%%
\beqa
\label{eq0718}
 &&q^{x}+\kappa T \tilde{\beta_{1}}\Big[\gamma \frac{\partial q^{x}}{\partial t}+v\frac{\partial q^{x}}{\partial x}\Big]= \kappa T\Big[
(-\gamma^4+ \tilde{\beta_{1}}\gamma^3 vq^{x}+\tilde{\alpha_{1}}(2-\gamma^2) \pi^{xx}
 )\frac{\partial v}{\partial t}\nn\\
 &&+(-v\gamma^4+ \tilde{\beta_{1}}\gamma^3 v^2 q^{x}- \tilde{\alpha_{1}}\gamma^2 v \pi^{xx})
\frac{\partial v}{\partial x}
+\tilde{\alpha_{1}}v \frac{\partial \pi^{xx}}{\partial t}+\tilde{\alpha_{1}} \frac{\partial \pi^{xx}}{\partial x}\nn\\
&&-\tilde{\alpha_{0}}v\gamma^2 \frac{\partial \Pi}{\partial t}-\tilde{\alpha_{0}}\gamma^2 \frac{\partial \Pi}{\partial x}
+\frac{1}{T}v\gamma^2 \frac{\partial T}{\partial t}+\frac{1}{T}\gamma^2 \frac{\partial T}{\partial x}\Big]~,\\ \nn\\
%%%
\label{eq0719}
&&\pi^{xx}+\frac{4}{3}\eta \beta_{2}\Big[\gamma \frac{\partial\pi^{xx}}{\partial t}+v\frac{\partial \pi^{xx}}{\partial x}\Big]= \frac{4}{3}\eta\Big[
(\gamma^5+\tilde{\alpha_{1}}\gamma^4 v q^{x}+ 2\beta_{2}\gamma^3 \pi^{xx})\frac{\partial v}{\partial t}\nn\\
 &&+(\gamma^5+ \tilde{\alpha_{1}}\gamma^4 v q^{x}+ 2\beta_{2}\gamma^3v^2 \pi^{xx})
\frac{\partial v}{\partial x}-\tilde{\alpha_{1}}\gamma^2 v\frac{\partial q^{x}}{\partial t}-\tilde{\alpha_{1}}\gamma^2 \frac{\partial q^{x}}{\partial x}\Big]~.
\eeqa

\subsection{Derivation of nonlinear wave equations}
%\label{sec4}
The above hydrodynamic equations from MIS theory is used to derive the equations governing the motion of the nonlinear perturbations in the fluid. We have adopted Reductive Perturbative Method (RPM) technique~\cite{Washimi1966,Davidson1974,Leblond2008,Kraenkel1995}, where we need to define `stretched co-ordinates' as,
\beqa
X=\frac{\sigma^{1/2}}{L}(x-c_s t), \text{and} \hspace{0.4cm} Y=\frac{\sigma^{3/2}}{L}(c_s t)~,
\label{eq0720}
\eeqa
where, $L$ is some characteristic length and $c_{s}$ is the speed of sound. Therefore, from the above equations we get
\beqa
\label{eq0721}
\frac{\partial}{\partial x}=\frac{\sigma^{1/2}}{L}\frac{\partial}{\partial X}, \text{and} \hspace{0.4cm} 
\frac{\partial}{\partial t}=-c_{s}\frac{\sigma^{1/2}}{L}\frac{\partial}{\partial X}+c_{s}\frac{\sigma^{3/2}}{L}\frac{\partial}{\partial Y}~,
\eeqa
where, $\sigma$ is the expansion parameter. The coordinate $X$ is measured from the frame of propagating sound waves, whereas the $Y$ represents fast moving coordinate. The RPM technique is designed to preserve the structural form of the parent equation. Now we perform the simultaneous series expansion of hydrodynamic quantities in  powers of
$\sigma$ to get,
\beqa
\hat{\epsilon}&=&\frac{\epsilon}{\epsilon_0}=1+\sigma \epsilon_1+\sigma^2 \epsilon_2+\sigma^3 \epsilon_3+...,\,
%%%%
\hat{p}=\frac{P}{P_0}=1+\sigma p_1+\sigma^2 p_2+\sigma^3 p_3+...,\nn\\
%%%%
\hat{v}&=&\frac{v}{c_s}=\sigma v_1+\sigma^2 v_2+\sigma^3 v_3+...,\,
%%%%
\,\,\,\,\,\hat{\Pi}=\frac{\Pi}{P_0}=\sigma \Pi_1+\sigma^2 \Pi_2+\sigma^3 \Pi_3+...,\nn\\
%%%%
\hat{q}^x&=&\frac{q^x}{\epsilon_0}=\sigma q^x_1+\sigma^2 q^x_2+\sigma^3 q^x_3+...,\,
%%%
\hat{\pi}^{xx}=\frac{\pi^{xx}}{P_0}=\sigma \pi^{xx}_1+\sigma^2 \pi^{xx}_2+\sigma^3 \pi^{xx}_3+...,
\label{eq0722}
\eeqa
where $\epsilon_0$ and $P_0$ are the background equilibrium energy density and pressure respectively on which the perturbations propagate. By collecting terms with different order of $\sigma$ from the above series, different types of equation like Breaking wave equation, Burgers equation or KdV (Korteweg-De Vries) like equation, etc can be
obtained~\cite{Fogaca1,Lick1990,Bhattacharyya:2020sua}. 

We rewrite the equations of motion {\textit{i.e.}} Eq.~\eqref{eq0714}-Eq.\eqref{eq0719}, using Eqs.~\eqref{eq0721} and~\eqref{eq0722} and collect terms with different orders in $\sigma$ 
and then finally revert from $(X,Y) \to (t,x)$ to obtain the evolution equations of different order of perturbations. It is found in Ref.~\cite{Fogaca:2014gwa} that for the conformal background, the evolution equation of the first-order perturbation does not include any relaxation and coupling coefficients arising from the MIS theory. Therefore, to get the effect of second-order, it is required to go to the second-order in perturbation ($\hat{\epsilon_{2}}$). To obtain the time evolution of second order perturbation, we need to collect terms up to third order in $\sigma$ in the expansion, because the equations in $n$-th order in $\sigma$ contains time derivatives of $\hat{\epsilon}_{n-1}$. Therefore, to get equations containing time 
derivative of $\hat{\epsilon}_{2}$ we  go up to 3rd order in $\sigma$. This leads to the following equations for the perturbation in the energy density ($\hat{\epsilon}$) as:
\beqa
\frac{\pd \hat{\epsilon_{1}}}{\pd t}
+\left[1+(1-c_{s}^2) \frac{\epsilon_{0}}{\epsilon_{0}+P_{0}}\hat{\epsilon_{1}}\right]c_{s}\frac{\pd \hat{\epsilon_{1}}}{\pd x}- \left[\frac{1}{2(\epsilon_{0}+P_{0})}(\zeta +\frac{4}{3}\eta)\right]
\frac{\pd^{2} \hat{\epsilon_{1}}}{\pd x^{2}}=0,
\label{eq0723}
\eeqa
%%%%
{\textrm and}
\beqa
\frac{\pd \hat{\epsilon_{2}}}{\pd t}+\mathcal{S}_{1}\frac{\pd \hat{\epsilon_{2}}}{\pd x}+\mathcal{S}_{2}\frac{\pd \hat{\epsilon_{1}}}{\pd x}+\mathcal{S}_{3} \frac{\pd^{2} \hat{\epsilon_{1}}}{\pd x^{2}}+\mathcal{S}_{4}\frac{\pd^{3} \hat{\epsilon_{1}}}{\pd x^{3}}+\mathcal{S}_{5}\frac{\pd^{2} \hat{\epsilon_{2}}}{\pd x^{2}}=0~.
\label{eq0724}
\eeqa
where $\hat{\epsilon_{i}}= \sigma^i \epsilon_i$ for $i=1,2,.....$.
The coefficients $\mathcal{S}_i$'s for $i=1$ to $5$ are:
\beqa
\mathcal{S}_{1}&=&\frac{1}{\epsilon_{0}+P_{0}}  \Big[c_s \{\epsilon _0 \left(1- \hat{\epsilon }_1\left(c_s^2-1\right)\right)+P_0\}\Big],\nn\\
 \mathcal{S}_{2}&=& \frac{1}{\epsilon_{0}+P_{0}}  \Big[\epsilon _0 c_s \{c_s^2-1\} \{\epsilon _0 \left(\left(2 c_s^2+1\right) \hat{\epsilon }_1{}^2-\hat{\epsilon }_2\right)-P_0 \hat{\epsilon }_2\}\Big],\nn\\
\mathcal{S}_{3}&=& \frac{1}{12(\epsilon_{0}+P_{0})^{2}}  \Big[\epsilon _0 \hat{\epsilon }_1 \{3 c_s^2 (7 \zeta +8 \eta )+3 \zeta +4 \eta \}\Big],\nn\\
 \mathcal{S}_{4}&=& \frac{1}{72 c_s c_V (\epsilon _0+P_{0})^2}
\Big[4 c_s^2 \{3 \kappa  T \epsilon _0(3 \alpha _0 \zeta +4 \alpha _1 \eta )\nn\\
&&+3 \zeta +4 \eta +P_0 (3 \alpha _0 \zeta +4 \alpha _1 \eta )\}+(\epsilon _0+P_{0}) (9 \beta _0 \zeta ^2+16 \beta _2 \eta ^2)-c_{V}(3 \zeta +4 \eta )^2\nn\\
&&+12 \kappa  \{\epsilon _0+P_{0}\} \{\epsilon _0 (3 \alpha _0 \zeta +4 \alpha _1 \eta )\}+3 \zeta +4 \eta +P_0 (3 \alpha _0 \zeta +4 \alpha _1 \eta )\Big],\nn\\
  \mathcal{S}_{5}  &=&-\frac{3 \zeta +4 \eta }{6 \left(\epsilon _0+P_{0}\right)}\,.
  \label{eq0725}
\eeqa
Eqs.~\eqref{eq0723} and \eqref{eq0724} have been solved to analyse the fate of the nonlinear waves propagating through a relativistic viscous fluid. The relevant EoS ({\textit{i.e.}}, relation between $\epsilon_0$ and $P_0$) and the initial conditions required to solve these equations.


We observe that the governing equation Eq.\eqref{eq0723} does not have any effect of second order theory. This equation can be derived from the NS theory as well. However, the second equation contains the second order effects via the coupling and relaxation coefficients of the MIS theory. Here, we have considered all the relevant transport coefficients to provide a general equations for the propagation of nonlinear waves. The dispersive term in Eq.\eqref{eq0724} (third order space derivative term) indicate that the combined effects of shear and bulk viscosities in the diffusive term act against the effect of thermal conductivity. This might weaken the diffusion of nonlinear waves in a dissimilar way.
%%%%%%%%%
%%%%%%%%%
%\section{Results and Discussions}
%\label{sec6}
\section{Results and discussion}
We present here the results on the fate of the nonlinear perturbations in the QGP when it passes near the region of the CEP. The equations we obtained look like KdV equation (ideal background) whose general analytic solution is a sec-hyperbolic function with a shifted argument. To better understand the deviation of propagation from that of the ideal background, the initial profile of perturbations of both the orders are taken as of the same form as the `{\it{sech}}' function. We solve the equations describing the evolution of $\hat{\epsilon_i}$ 
for the following initial profile of the perturbations:
\beqa
\hat{\epsilon}_1&=&A_{1}\big[\mathrm{sech} \frac{(x-x_{0})}{B_{1}}\big]^{2} \nn\\
%%
\hat{\epsilon}_2&=&A_{2}\big[\mathrm{sech} \frac{(x-x_{0})}{B_{2}}\big]^{2},
\label{eq0726}
\eeqa
where, $x_0$ is the initial position of the peak of the perturbations. Since, here we are dealing with 1D propagation in $+x$ direction in an extended and static background, any choice of value of $x_0$ should provide the same salient features of the propagation. Here we have taken $x_0=10$ fm to demonstrate the propagation of the perturbation however, one may choose any other value of $x_0$ as well. The solutions of the Eqs.\eqref{eq0723} and \eqref{eq0724} are solitonic in nature (with {\it{sech}}$^2$ dependence), which is similar to the solutions of KdV equations for small dissipations. This motivates us to choose initial profile given by Eq.~\eqref{eq0726}. Also to note that for small spatial gradient the solitonic behaviour of the solution has also been demonstrated in Ref.~\cite{Fogaca:2014gwa} for Gaussian initial profiles in a conformally invariant hydrodynamic background. In the initial profile, the  $A_{i}$ and $B_{i}$ respectively are determining the height and width of the initial profile. The  effects of the CEP has been taken into consideration through the EoS (Ch.\ref{chapter4}) and the scaling behaviour of  transport coefficients and thermodynamic response functions in Eq.\eqref{eq0619}. 
%%%%
% Figure environment removed
%{\color{red} Though there have been progress in constructing the EoS 
%to include the CEP, we use this EoS to capture the main aspect of the 
%CEP e.g. enhancement of transport coefficients, droping speed of sound. 
%It is seen in the linear analysis~\cite{hasan1,hasan2} that effect of 
%EoS is much more decisive in affecting the propagation of perturbation 
%near the CEP than the diverging transport coefficient. For the same purpose,
%%%%%%%%%%%%%%%%%%%%%%%%%
%\begin{widetext}
%\end{widetext}
%%%%%%%%%%%%%%%%


Fig.\ref{fig0701} depicts the propagation of a nonlinear wave in the medium when the system is passing away from the CEP. It is observed that, though the amplitude of the wave is reduced, the nonlinear wave survives the dissipation despite the presence of the non-zero values of transport coefficients of the medium. However, the nonlinear wave is substantially dissipated (almost died) near the CEP as evident from the results shown in Fig.~\ref{fig0702}. It has been observed that the dissipative feature remains unaltered with the variation of the location of the CEP along the transition line in the QCD phase diagram. This clearly indicates that the nonlinear perturbations will exhibit unique detectable effects of the CEP. It also shows that the speed of the attenuated wave is non-vanishing in contrast to linear waves~\cite{Hasan1}. This is due to the  amplitude dependent propagation speed of the nonlinear waves, shown in Fig.\ref{fig0703}. The attenuation of the perturbation is in comparison smaller for the second-order correction ($\hat{\epsilon_2}$). Furthermore, the second-order perturbation travels a bit faster than the first-order one. This is clear from the results displayed in Figs.\ref{fig0701} and Figs.\ref{fig0702}, which can also be understood from Eq.\eqref{eq0724}, where the second-order correction contains the third-order derivatives of the first-order perturbation. This is equivalent to the dispersive term in the KdV equation, which is responsible for height preserving solitonic behaviour~\cite{Lick1990}. Therefore, this dispersive terms compete with the diffusive terms,  results in weakening the damping effect. The relaxation terms in the dissipative fluxes incorporated in MIS theory (Eq.\eqref{eq0724}) makes the dissipation lesser in comparison to the NS theory (Eq.\eqref{eq0723}).
	%%%%%%%%%%%%%%%%
%\begin{widetext}
% Figure environment removed
%\end{widetext}
%%%%%%%%%%%%%%%%


Therefore, near the CEP, the disturbances with larger magnitudes, which are created in the medium due to energy deposition by energetic particles or jets, will be highly dissipated. In earlier studies~\cite{Hasan1}, it was shown that linear waves will be completely stopped and dissipated near the CEP. Similar suppression for nonlinear waves too are found in the presence of the CEP. Although the speed of propagation of the nonlinear waves are amplitude dependent, in presence of the CEP, irrespective of their amplitude all the waves are highly suppressed. This may have important consequences in detecting the CEP from the analysis of the hadron spectra.
 
 
In has been speculated earlier~\cite{Minami,Hasan1,Hasan2} that the formation of Mach cone is forbidden near the CEP for the propagation of linear perturbations. The remaining question to be reported if the Mach cone formation will  be there for nonlinear waves or not. This is important because nonlinear waves are found to maintain the solitonic nature in comparison to linear perturbations. Nevertheless, we find that even the nonlinear perturbations will not be able to preserve the Mach cone effects if it hits the CEP. So the propagation of both types of the perturbations (linear and nonlinear) are stopped due to the presence of the CEP. These results can be used to detect the CEP by looking into the suppression of the Mach cone effect in two particle correlation~\cite{BetzPRL,Betzthesis}. Due to the high dissipation of the nonlinear waves, the broadening effect of localized waves will also vanish ~\cite{Fogaca1} along with the Mach cone effect.
%%%%%%%%%%%%%%%%%%%%%%%%%%%%%%%%%%%%%%%%%%%%%%%%%%%%%%%%%%%%%%%%%%%%%%%%%%%%%%%%
% Figure environment removed
The flow harmonics play important role in characterizing the medium formed in RHIC-E. It was shown in Refs. ~\cite{Stocker:2007pd,Stoecker1976} that some of the flow harmonics will collapse at the CEP (along phase boundary of the QCD phase diagram). Based on linear analysis it has been argued in~\cite{Minami,Hasan1,Hasan2}  that $v_{2}$ or higher harmonics will be reduced near the CEP. The similar conclusion can be drawn for the nonlinear perturbations too near the CEP. If the initial spatial shape of the system formed in HICs is highly distorted azimuthally, the fluid dynamical response to the initial eccentricity can be nonlinear. The suppression of the nonlinear waves also suggests that $v_2$ and higher order flow harmonics will be suppressed near the CEP due to the absorption of sound wave.
%Since the internal 
%inhomogeneity is higher in magnitude, nonlinear effects may be triggered, and 
%will also be suppressed in the vicinity of CEP.

In a recent study~\cite{Dore}, it is argued that the path to the critical point is greatly influenced by far-from equilibrium initial conditions, where viscous effects may lead to dramatically different $(T,\mu)$ trajectories. The sound wave propagating through the system will be ceased near the CEP. Therefore, the pressure gradient produced (due to EoS) due to initial spatial anisotropy will not be converted effectively to momentum anisotropy through hydrodynamic expansion in an event of isentropic trajectory passing through the CEP. In such events the flow harmonics get suppressed. In another event, due to the different initial conditions, the system will follow a trajectory away from the CEP (see Ref.\cite{Nonaka}). In such events, sound will propagate accordingly and thus no suppression in flow harmonics should be observed. This difference in flow harmonics (between two events one nearby and the 
other away from the CEP) will lead to large event-by-event fluctuations in flow harmonics. Therefore,  large event-by-event fluctuation in flow harmonics could be a signature of the CEP.
%If the suppression is caused by effects other than the system passing thorough the CEP 
%then the event-by-event fluctuations will not be enhanced. 

So far it is evident that the nonlinear waves survive dissipation even in the presence of $\eta, \zeta$ and $\kappa$ without the critical effect, and the suppression of nonlinear wave only happens almost exclusively because of the presence of the CEP. This hints that the formation of Mach cone will be forbidden in the presence of the CEP. Therefore, the present theoretical investigation indicates that the Mach cones may disappear in the presence of the CEP. The Mach cone effect manifests as a double hump in the two-particle correlation in the low momentum domain of associated particles, and the CEP plays a unique role in suppressing the double hump in the two-particle correlation contrary to the other mechanisms: {\textit{e.g.}} (i) deflection of away side jets, (ii) Cherenkov radiation and (iii) radiation of gluons which produce the double hump. These mechanisms have the ability to obscure the suppression due to the CEP by creating double hump.



The dip in the azimuthal distribution of two particle-correlation at $\Delta \phi(=\phi-\phi_{trig}=\pi)$ accompanied by two local maxima on the both side of $\Delta \phi=\pi$ for the transverse momentum range $0.15<p_{assoc}^{T}<4$ GeV~\cite{Wang2006,Adams2005}, is attributed to the processes pointed above. Therefore, we contrast the effect of the CEP to the following mechanism. 

(i) Deflection of the away side jet by asymmetric flow and third flow harmonics (triangular flow) due to the initial state fluctuations~\cite{Betzthesis,Wang2013,Cao2020} lead to peaks on the away side jet on the both side of $\Delta \phi=\pi$. However, if the system passes near the CEP, the flow harmonics will get highly suppressed, and hence, the deflection also will be strongly diminished. 

(ii) Cherenkov radiation~\cite{Koch2006,STAR2003} is characterized by strong momentum dependence of the cone angle ($\sim 1/\mu_{r}$), where $\mu_{r}$ is the refractive index of the medium). This process is not likely to be responsible for the double hump due to the lack of momentum dependence of the location of the double maxima of associated particles.

(iii) The radiation of gluons by the
away side jet will deviate it from propagating at an angle $\pi$
with respect to the near side (trigger) jet. However, the quantitative prediction of the Mach cone positions studied through three particle correlation~\cite{Abelev2009}  and the momentum independence of
the location of the double hump indicate that the double hump
may originate from Mach cone effects. The vanishing of the Mach cone-like structure in particle correlation will therefore, indicate
the existence of the CEP.



In this study, the propagation of perturbation is studied in a static background. In presence of expansion, the system will cool 
and move toward the phase transition line with changing transport coefficients. However, during this cooling, if the trajectory goes away from the CEP, 
the response will allow the survival of perturbations. 
But if it passes near the CEP, the 
perturbative effects will be mostly washed out and no effect will survive at the latter stage. Therefore, it is expected that the results obtained in static situation may not
vary with the inclusion of expansion if the system passes near the CEP. We investigate the hydrodynamic propagation of perturbations in the system without considering the 
fluctuations~\cite{Stephanov:2017ghc,An2020} originating from the CEP itself, as they do not create any angular pattern as jets produce in correlations. Though the non-equilibrium fluctuations due to the CEP is not taken into account here for obtaining the non-linear wave equations, the enhancement of thermodynamic fluctuations near the critical point, which affects the hydrodynamic response, is inherently taken into account through the EoS and other thermodynamic quantities with the critical exponents.
%\section{Summary and Conclusion}
%\label{sec7}






