\chapter[Propagation of perturbation near the QCD critical point]{Propagation of perturbation near the QCD critical point}
\label{chapter5}  
\section{Introduction}
\label{sec0501}
This chapter is based on the paper of Ref.\cite{Hasan1}. Here we study the propagation of small disturbances if the evolution of the created QGP is passing through (near) the CEP (critical end point). As discussed earlier, fixing the location of the CEP is a challenging job. However, we are not searching for the location of CEP by direct means. Rather, we want to examine its effects on the possible fate of the sound wave (disturbance) propagating through the fluid in presence of the CEP. Here, the location of the CEP is taken at: $(T_c,\, \mu_{c})= (154 \text{MeV},\, 367 \text{MeV})$ (as mentioned in Ch.\ref{chapter4}). It is expected that a system with such $(T, \mu_{B})$ may be produced through nuclear collisions at GSI-FAIR, NICA, BES-RHIC. The QGP produced in such collisions will expand very rapidly along a trajectory with constant $s/n$ ($s$ and $n$ stand for the entropy density and baryon number density respectively) and cools down consequently. We have assumed that an isentropic trajectory in the $(T, \mu_{B})$ plane will pass through the critical region of the CEP.


The space-time evolution of the thermalized QGP can be modelled by the relativistic viscous hydrodynamics (Ch.\ref{chapter3}). The first-order theory of relativistic viscous hydrodynamics is governed by Navier-Stokes (NS) equations, is known to violate causality, and gives unstable numerical solutions \cite{Hiscock:1983zz,Hiscock:1985zz}. Therefore, it makes the theory unsuitable to study the space-time evolution of QGP. These problems were rectified by M\"{u}ller~\cite{Muller:1967zza} and Grad~\cite{Grad} after including quantities in second-order dissipative flux and therefore, these theories are called the `second-order hydrodynamics'. The relativistic generalization of this theory was formulated by Israel and Stewart \cite{Israel:1979wp}, and we have used it to describe the space-time evolution of QGP. The hydrodynamic response of the fluid to the perturbation is dictated by the relevant transport coefficients (shear and bulk viscosities, and thermal conductivity) of the fluid. The effects of thermal conductivity $(\kappa)$ and the shear viscosity $(\eta)$ have been considered in this work to investigate the propagation of acoustic waves when the system passes through the CEP. The effects of the CEP in the hydrodynamic evolution enter through the EoS, is constructed on basis of the universality hypothesis (discussed in Ch.\ref{chapter4}). The behaviour of the thermodynamic quantities near the CEP is governed by the critical exponents. Dispersion relations {\it i.e.} the functional dependence of the frequency ($\omega$) on the wave vector ($k$) will be set up to study the effects of the CEP on the propagation of the sound waves in the fluid.
%%%%%
%%%%%
\section{Propagation of the perturbations in viscous fluid} 
\label{sec0502}
M\"{u}ller-Israel-Stewart (MIS) hydrodynamics is appropriate to study the evolution of the QGP as the NS theory violates causality and introduces instability in the numerical solution. Therefore, in this section, we study the propagation of perturbations through the viscous fluid by using second-order causal hydrodynamics. One of the major differences between a relativistic and a non-relativistic fluid originates from the definition of chemical potential (coming through the fugacity factor). In the non-relativistic case, the chemical potential puts a constraint on the total number of particles in the system. But in the relativistic system, the total number of particles does not remain constant due creation and annihilation of particles within the fluid. However, through the creation and annihilation processes, the conservation of certain quantum numbers remains untouched. For example, in the strong interaction, net (baryon minus antibaryon) baryon number, net electric charge, or net strangeness remain conserved (although strangeness is not conserved in weak interaction). Accordingly, the net baryon number will remain conserved throughout the evolution process of the QGP. Therefore, in the discussion below the net charge density stands for net baryon number density. Also to note that from here onwards, we will drop the subscript `$B$' from $\mu_{B}$ and $n_{B}$ to define baryon chemical potential and baryon number density.

 
In this chapter, we use the Landau-Lifshitz frame (LL-frame) of reference as discussed in Sec.\ref{sec0305} in Ch.\ref{chapter3}, where it is considered that the heat flux is zero but the particle current is non-zero in the local rest frame (LRF). Therefore,
\beqa
 h^\mu=0,\,\,\, n^\mu =-n q^\mu /(\epsilon+P)\,,
 \label{eq0501}
 \eeqa
  and the different viscous fluxes are given by~\cite{Israel:1979wp}-
\begin{eqnarray}
\Pi &=&-\frac{1}{3}\zeta\Big[\pd_{\mu}u^\mu +\beta_0 D \Pi-\alpha_0  \pd_{\mu}q^\mu \Big]~,\nonumber\\   
\pi^{\mu\nu}&=&-2\eta \Delta^{\mu\nu\alpha\beta}\Big[\partial_{\alpha}u_{\beta}+\beta_{2}D\pi_{\alpha\beta}-\alpha_{1}\partial_{\alpha}q_{\beta}\Big]~,\nonumber\\
q^{\mu}&=&\kappa T\Delta^{\mu\nu} \Big[\frac{nT}{\epsilon+P}(\partial_\nu \alpha )-\beta_1 D{q_\nu}+\alpha_0\partial_\nu \Pi +\alpha_1\pd_{\lambda}\pi ^{\lambda}_{\nu} \Big]~. 
\label{eq0502}
\end{eqnarray}
where  $D\equiv u^\mu\partial_\mu$, is known as co-moving derivative and in local rest frame (LRF), $D\Pi =\dot{\Pi }$ represents the time derivative. The coefficients $\alpha_{0}, \alpha_{1}, \beta_{0}, \beta_{1}$ and $\beta_{2}$ can be calculated from thermodynamics integrals (see Appendix.\ref{appendix01_A}). But we have used the ultra-relativistic limit, $\beta=m/T\rightarrow 0$ where $m$ is the mass of the particle and we have the following relations ~\cite{Israel:1979wp},
\begin{eqnarray}
\alpha_0 \approx  6\beta^{-2}P^{-1}, \alpha_1 \approx -\frac{1}{4}P^{-1},
\beta_0 \approx 216 \beta^{-4} P^{-1},\beta_1 \approx \frac{5}{4}P^{-1}, 
\beta_2 \approx \frac{3}{4}P^{-1} \,.
\label{eq0503}
\end{eqnarray}

Since in energy frame,  $h^\mu=0$, then the energy-momentum tensor (EMT) reduces to 
\begin{equation}
T^{\mu\nu}=\epsilon u^\mu u^\nu+P\Delta^{\mu\nu} +\Pi\Delta^{\mu\nu}+\pi ^{\mu\nu} \,.
\label{eq0504}
\end{equation}
Putting the explicit forms of $\Pi, q^\mu$ and $\pi ^{\mu\nu}$ given by Eq.\eqref{eq0502} into Eq.\eqref{eq0504} and keeping only the terms up to second order in space time derivatives, the EMT becomes  \cite{Rahaman:2017ezf}
\begin{eqnarray}
 T^{\mu\nu}&= & \epsilon u^\mu u^\mu+P\Delta^{\mu \nu}- \frac{1}{3}\zeta \Delta^{\mu\nu}\partial_\alpha u  ^\alpha + \frac{1}{9} \zeta \beta_0\Delta^{\mu\nu} D(\zeta\partial_\alpha u  ^\alpha)   \nn\\
 &+&\frac{\zeta \alpha _0  }{3}\Delta^{\mu\nu}\partial_\alpha\Big [\frac{n\kappa T^2}{\epsilon+P}\nabla^\alpha(\alpha) \Big]
 -2\eta \Delta^{\lambda\mu\alpha\beta} \partial_\alpha u_\beta  
  + 4\eta \beta_2\Delta^{\lambda\mu\alpha\beta} D(\eta\Delta_{\alpha\beta}^{\rho\sigma} \partial_\rho u_\sigma)  \nn\\
  &+&2\alpha _1\eta\Delta^{\mu\nu\alpha\beta}\partial_{\alpha}\Big[\frac{ n\kappa T^2}{\epsilon+P}\nabla_\beta(\alpha)\Big]\,.
\label{eq0505}
\end{eqnarray}

The solution of the MIS hydrodynamic equations grant causality and stability, which is achieved by promoting the dissipative currents (Eq.~\eqref{eq0502}) as the independent dynamical variables and introducing relaxation time scales (time delay) for these currents. In NS theory, the dissipative currents promptly respond to the hydrodynamical gradients but in MIS theory, the response of the dissipative currents is controlled by the relaxation time scales (see Eq.~\eqref{eq0503}). The EMT given in Eq.~\eqref{eq0505} represents second-order dissipative hydrodynamics which is equivalent to MIS theory for small gradients. The general form of the EMT constrained by the conformal invariance can be found in ~\cite{Baier2008}.

The charge current (up to second-order in velocity gradient) can be written as, 
\begin{eqnarray}
 N^\mu &=&nu^\mu  -\frac{n\chi T\triangle^{\mu\nu }}{(\epsilon+P)}\Big[ \frac{nT}{(\epsilon+P)} (\partial _\nu \alpha )  -\beta_1D\ \Big\{ \frac{n\chi T^2}{(\epsilon+P)}   \triangle_{\nu \sigma }(\partial^\sigma \alpha) \Big\}\nn\\
 &&-\frac{\zeta\alpha_0}{3}\partial_\nu \partial _\sigma u^\sigma-2\eta \alpha_1 \partial^\rho u_{<\rho |\nu> }\Big]\,.
\label{eq0506}
\end{eqnarray}
\section{Perturbations in the relativistic fluid}
\label{sec0503}
The motion of perturbations in the relativistic viscous fluid with one conserved current (baryonic current for the present case) is governed by Eqs.\eqref {eq0505} and \eqref{eq0506}. Now we impart small perturbations $P_1,\epsilon_1, n_1, T_1, \mu_1$ and $ u^\alpha_1$  to $P, \epsilon, n, T,\mu$ and $u^\alpha $  respectively to study the propagation of acoustic wave in the fluid with $u^\alpha=(1,0,0,0)$  as outlined in Ref.\cite{Weinberg:1971mx}. We set $u^0_1=0$ to preserve the normalization condition $u^\alpha u_\alpha =-1$. 
 
 
A  space-time dependent perturbation, $\sim exp[-i(k  x -\omega t)]$, is placed into the fluid and its fate is being studied. The equation of motions that dictate the evolution of different components of the perturbations can be obtained from the conservations of the energy-momentum tensor ($T^{\mu\nu}$)  and net-baryon number flux ($N^\mu$) of the fluid:
\begin{equation}
\partial_\mu T^{\mu\nu }=0, \,\,\,\,\, \partial_\mu {N}^\mu=0\,.%\mathcal{J}^\mu=0
\label{eq0507}
\end{equation} 
The equations of motion (EoMs) have been linearized to find the dispersion relation for small perturbation. After that the various components of energy momentum tensor are written in $\omega-k$ (frequency-wave vector) space as: 
%%%%%%
\begin{eqnarray}
0=\omega T_1^{00}-k_iT_1^{i0}=\omega\epsilon_1-(\epsilon+P)(\vec{k}\cdot \vec{u_1})\,.
\label{eq0508}
\end{eqnarray}
%%%%%%%
and the other components of the EMT satisfies,
\begin{eqnarray}
0&=&\omega T_1^{i0}-k_jT_1^{ij}\nonumber\\
&=&\omega (\epsilon+P)u_1^i-k^iP_1+\frac{1}{3}\zeta k^i\Big[ i(\vec{k}\cdot \vec{u_1})+\frac{1}{3}\zeta\beta _0\omega(\vec{k}\cdot \vec{u_1})\Big] \nn\\
&&+i\eta\Big[ k^2u^i_1+\frac{1}{3}k^i (\vec{k}\cdot \vec{u_1})\Big]-2\eta^2 \beta_2 \omega\Big[ k^2u^i_1+\frac{1}{3}k^i (\vec{k}\cdot \vec{u_1})\Big]\nn\\
&&+\frac{nT\kappa}{(\epsilon+P)}(\mu _1-\alpha T_1)\Big[ \frac{\alpha _0\zeta k^2}{3}k^i+\frac{4}{3}\alpha_1\eta k^2 k^i\Big]\,.
\label{eq0509}
\end{eqnarray}
%%%%%%%
Using the linearization technique, the number conservation equation in the $\omega-k$ space becomes,
\begin{eqnarray}
 0&=&\omega n_1-n(\vec{k}\cdot\vec{u_1})-i \frac{n^2\kappa T k^2}{(\epsilon+P)^2}\Big[(\mu_1-\alpha T_1)(1+i\omega\kappa \beta_1)\Big]\nn\\
 &&+\frac{1}{3}\frac{n \kappa Tk^2}{(\epsilon+P)}\Big[\zeta \alpha_0+4 \eta \alpha_1\Big](\vec{k}\cdot\vec{u_1})\,.
\label{eq0510}
\end{eqnarray}
We have considered terms up to the first order in perturbation as
\beqa
n\,u^{\mu}\to~(n+n_{1})(u^{\mu}+u_{1}^{\mu})\approx~nu^{\mu}+n_{1}u^{\mu}+nu_{1}^{\mu}\,.
\label{eq0511}
\eeqa
In deriving the Eqs.\eqref{eq0508}, \eqref{eq0509}, \eqref{eq0510}, we have used
\beqa
\pd_{\mu}\alpha&=&\pd_{\mu}\Big[\frac{\mu+\mu_{1}}{T+T_{1}}\Big]~\approx \pd_{\mu}\Big[(\frac{\mu+\mu_{1}}{T})(1-\frac{T_{1}}{T})\Big]\nn\\
&=&\pd_{\mu}\Big[\frac{\mu+\mu_{1}}{T}-\frac{\mu}{T^{2}}T_{1}\Big]=~\frac{1}{T}\big[\pd_{\mu}\mu_{1}-\alpha \pd_{\mu}T_{1}\big]~,
\label{eq0512}
\eeqa
and,
\beqa
\nabla_{\mu}\alpha=\frac{1}{T}\big[\nabla_{\mu}\mu_{1}-\alpha \nabla_{\mu}T_{1}\big]\,.
\label{eq0513}
\eeqa
In Eqs.~\eqref{eq0508}, \eqref{eq0509} and \eqref{eq0510}, we have considered terms up to first order in perturbations and neglected the higher order terms. Also, we have not perturbed the different transport coefficients, as they are not hydrodynamical variables. In LRF, we take them as constant in space and time, hence their comoving derivative are zero.  As mentioned, we have considered, $\eta\neq 0$,  $\kappa\neq 0$ and $\zeta$=0. We decompose the fluid velocity into directions perpendicular and parallel to the direction of wave vector, $\vec{k}$ as:
\begin{equation}
\vec{u_1}=\vec{u_1}_\bot +\vec{k} (\vec{k}\cdot\vec{u_1})/k^2\,.
\label{eq0514}
\end{equation}
The modes propagating along the direction of $\vec{k}$ are called longitudinal 
and those perpendicular to $\vec{k}$ are called transverse modes. 

The quantities, $\epsilon_1$, $P_1$ and $\mu_1$ defined above can be 
expressed in terms of the derivatives of the thermodynamic quantities (which
are connected to response functions like specific heat, etc.) 
as follows:
\begin{eqnarray}
&&\epsilon_1=\Big(\frac{\partial \epsilon}{\partial T}\Big)_n T_1+\Big( \frac{\partial \epsilon}{\partial n}\Big)_T n_1\,,
\nn\\&&P_1=\Big( \frac{\partial P}{\partial T}\Big)_n T_1+\Big( \frac{\partial P}{\partial n}\Big)_T n_1\,,
\nn\\&&\mu_1=\Big[-\Big( \frac{\partial n}{\partial T}\Big)_n T_1+n_{1}\Big]\Big( \frac{\partial \mu}{\partial n}\Big)_T \,.
\label{eq0515}
\end{eqnarray}
%%%%%%
%%%%%%
%%%%%%
\section{Dispersion Relations}
\label{sec0503}
The dispersion relation ($\omega$ (frequency) as a function of $k$ (wave vector)) plays an important role in determining the fate of the perturbation in a fluid. The $\omega$ can be a real as well as an imaginary quantity. The real part of the frequency, $\omega_{\R e}$, dictates the possibility of the propagation of the wave in the medium. Whereas, the imaginary part of the frequency, $\omega_{Im}$, decides the decay of the waves along its propagation in the medium. 
%Therefore, we must calculate the dispersion relation to know the fortune of the wave.


To evaluate the dispersion relation, we would like to write down a equation that dictates the $k$ dependence of $\omega$. For this purpose we use Eqs.~\eqref{eq0508}, \eqref{eq0509} and \eqref{eq0510}, and collected the coefficients $(\vec{k}\cdot\vec{u_1})$, $T_1$ and $n_1$ to put row wise to form a determinant as
\beqa
\begin{vmatrix}
&&a_{11} && a_{12} && a_{13} ~~~~ \\ 
&&a_{21} && a_{22} && a_{23} ~~~~ \\
&&a_{31} && a_{32} && a_{33} ~~~~
\end{vmatrix}
\label{eq0516}
\eeqa
 and upon the expansion of the determinant above, yield
\begin{eqnarray}
a\omega^3+b\omega^2+c\omega=0\hspace{0.5cm}\rightarrow\hspace{0.5cm}\omega(a\omega^2+b\omega+c)&=&0 \,.
\label{eq0517}
\end{eqnarray}
The coefficients $a, b$ and $c$ are determined by solving 
Eqs.~\eqref{eq0508}, \eqref{eq0509} and \eqref{eq0510}, simultaneously. 
The solutions of this equation which provide a relation between $\omega$ and $k$
which is called the dispersion relation.
The equation, \eqref{eq0517} has  
three roots, one real which is 
$\omega=0$ and two complex roots with real ($\omega_{\R e}$) and imaginary ($\omega_{im}$) 
parts given below by Eqs.~\eqref{eq0518} and \eqref{eq0521} respectively.
%%%%%%%%%%%%%%%%%%%%%%%%%%%%%%%
The real part of $\omega$ can be expressed as:
\begin{equation}
\omega_{\R e}=\sqrt{\frac{a_0k^2-a_1k^3+a_2k^4}{b_0-b_1k^2}}\,.
\label{eq0518}
\end{equation}
where,
\beqa
&&a_0=9h\Big[\Big(\frac{\partial P}{\partial T}\Big)_n
+\alpha_1n
\Big\lbrace\Big(\frac{\partial\epsilon}{\partial n}\Big)_T
\Big(\frac{\partial P}{\partial T}\Big)_n
-\Big(\frac{\partial\epsilon}{\partial T}\Big)_n
\Big(\frac{\partial P}{\partial n}\Big)_T
\Big\rbrace\Big]\,,\nn\\
&&a_1=\frac{9\alpha\beta_1n^2T^2\kappa^2}{h}+
12\alpha_1\eta\kappa nT\Big[\alpha+\frac{\alpha n}{h}\Big(\frac{\partial\epsilon}{\partial n}\Big)_T
-\frac{T}{h}\Big(\frac{\partial P}{\partial T}\Big)_n
+\frac{T}{h}\Big(\frac{\partial\epsilon}{\partial T}\Big)_n\Big]\,,
\nn\\
&&a_2=\frac{9\beta_1\kappa^2n^2T}{h}\Big[
\Big(\frac{\partial n}{\partial T}\Big)_\mu\Big(\frac{\partial\mu }{\partial n}\Big)_T
+\Big(\frac{\partial P}{\partial T}\Big)_n\Big(\frac{\partial P}{\partial n}\Big)_T\Big]\nn\\
&&\,\,\,\,\,\,\,\,\,\,\,\,\,\,+\frac{12\alpha_1\eta\kappa nT}{h}\Big[\Big(\frac{\partial P}{\partial T}\Big)_n 
+n\Big(\frac{\partial n}{\partial T}\Big)_\mu\Big(\frac{\partial\mu}{\partial n}\Big)_T
+\frac{n}{h}\Big(\frac{\partial P}{\partial T}\Big)_n\Big(\frac{\partial\epsilon}{\partial n}\Big)_T\Big]\,,\nn\\
&&b_0=9h\Big(\frac{\partial\epsilon}{\partial T}\Big)_n\,,\nn\\
&&b_1=24\beta_2\eta^2\Big(\frac{\partial\epsilon}{\partial T}\Big)_n+\frac{9\beta_1\kappa^2 n^2}{h}\Big[
T\Big(\frac{\partial\mu}{\partial n}\Big)_T\Big(\frac{\partial n}{\partial T}\Big)_\mu \nn\\
&&\,\,\,\,\,\,\,\,\,\,\,\,\,\,-T^2\Big(\frac{\partial\epsilon}{\partial n}\Big)_T\Big(\frac{\partial\mu}{\partial n}\Big)_n
+\alpha\Big(\frac{\partial\epsilon}{\partial n}\Big)_T\Big]\,.
\label{eq0519}
\eeqa
and, $h=\epsilon+P$ is the enthalpy density.   
We have kept terms up to quadratic power of
transport coefficients in Eq.~\eqref{eq0518}.  
We have also neglected the higher order terms in $\alpha_{0},\alpha_{1},\beta_{0},\beta_{1},\beta_{2}$.
%i.e. terms like, $\alpha_{0}^{2},\alpha_{0},\alpha_{1},\alpha_{0},\beta_{1},\beta_{1}^{2}$. 
Expanding $\omega_{\R e}$ in powers of $k$ and keeping terms up to $\cal{O}$$(k^4)$ we obtain,
\begin{equation}
\omega_{\R e}=\sqrt{\frac{a_0}{b_0}}\left[k-\frac{1}{2}\frac{a_1}{a_0}k^2+(\frac{1}{2}\frac{a_2}{a_0}-\frac{1}{8}{a_1}{a_0^2}+\frac{b_1}{b_0})k^3
+(\frac{1}{4}\frac{a_1a_2}{a_0^2}+\frac{1}{16}{a_1^2}{a_0^3}-\frac{1}{2}\frac{a_1b_1}{a_0b_0})k^4\right]\,.
\label{eq0520}
\end{equation}

Similarly, the expression for the imaginary part of $\omega$ reads as:
\begin{equation}
\omega_{Im}=\frac{-c_0k^2+c_1k^3+c_2k^4}{d_0+d_1k^2}\,.
\label{eq0521}
\end{equation}
where,
\begin{eqnarray}
&& c_0=2\eta h^2\Big(\frac{\partial\epsilon}{\partial T}\Big)_n
-3hn^2\kappa T\beta_1\Big[\alpha\kappa\Big(\frac{\partial\epsilon}{\partial n}\Big)_T
+h\Big(\frac{\partial n}{\partial T}\Big)_\mu
\Big(\frac{\partial\mu }{\partial n}\Big)_T\nn\\
&&\,\,\,\,\,\,\,\,\,\,\,\,+\frac{\alpha_1}{\beta_1}\Big(\frac{\partial P}{\partial T}\Big)_n\Big(\frac{\partial\epsilon}{\partial n}\Big)_T\Big]\,,\nn\\
&&c_1=2\alpha\beta_1\eta n^2 T\Big(T\kappa^2+4 h\eta\frac{\beta-2}{\beta_1}\Big)\,,\nn\\
\label{eq06}
&&c_2=8\beta_2\eta\kappa n^2T\Big[\Big(\frac{\partial\epsilon}{\partial T}\Big)_n-\Big(\frac{\partial P}{\partial n}\Big)_{T}^2
\Big(\frac{\partial\epsilon}{\partial n}\Big)_T\Big]\,,\nn\\
&&d_0=3h^3\Big(\frac{\partial \epsilon}{\partial T}\Big)_n\,,\nn\\
&&d_1=3h\beta_1n^2\kappa\Big[\alpha\kappa\Big(\frac{\partial\epsilon}{\partial n}\Big)_T+4T^2\kappa\Big(\frac{\partial\epsilon}
{\partial T}\Big)_n\Big(\frac{\partial\mu}{\partial n}\Big)_T^2-
T\Big(\frac{\partial n}{\partial T}\Big)_n\Big(\frac{\partial\epsilon}{\partial n}\Big)_T\Big]\,.
\label{eq0522}
\end{eqnarray}
The imaginary part of $\omega$ up to $\cal{O}$$(k^4)$  is given by,
\begin{equation}
\omega_{Im}=-\frac{c_0}{d_0}\left[k^2-\frac{c_1}{c_0}k^3-(\frac{d_1}{d_0}+\frac{c_2}{c_0})k^4\right]\,.
\label{eq0523}
\end{equation}
%%%%%%%%%%%%%%%%%%%%%%%%%%%%%%%
The dispersion relation for first order hydrodynamics can be obtained by setting the relaxation coefficients ($\beta _0,\beta_1,\beta_2$) and  the coupling coefficients ($\alpha_0$, $\alpha_1$) to zero which allow only $a_0$, $b_0$, $c_0$ and $d_0$ to be non-zero. Therefore, keeping terms
up to $\cal{O}$$(k^2)$ in Eqs.~\eqref{eq0520} and \eqref{eq0523} we get
(see also~\cite{Grozdanov}),
\begin{equation}
\omega (k)=c_s k -\frac{i}{2}k^{2}\frac{\eta}{s}s\frac{4/3}{h}\,,
\label{eq0524}
\end{equation}
where, $c_{s}=\sqrt{\big(\frac{\pd P}{\pd \epsilon}\big)_{s/n}}$ is the speed of sound and $\eta/s$ is 
shear viscosity ($\eta$) to the entropy density ($s$) ratio. The Eq.\eqref{eq0524} is the dispersion relation for
NS hydrodynamics, which is obtained from the limiting case of MIS theory.

\subsection{Fluidity near the critical region}
The imaginary and real parts of $\omega (k)$ provide the information respectively on the attenuation and the propagation of the sound wave in the dissipative fluid. Thus, if magnitude of the imaginary part  is greater than the real part, the wave will dissipate quickly. Therefore, the dispersion relations in Eqs.\eqref{eq0518} and \eqref{eq0521} can be used to determine the upper limit of $k$ of the sound wave for its survivability in the medium. The threshold value of $k$, {\textit{i.e.}} $k_{th}$ can be evaluated by using the 
following condition~\cite{liao}
\begin{eqnarray}
\Big|\frac{\omega_{Im}(k)}{\omega_{\R e}(k)} \Big|_{k=k_{th}}=1\,.
\label{eq0525}
\end{eqnarray}
{\it i.e.} any wave with wave vector higher than $k_{th}$ will get
dissipated in the fluid.
Solving the above equation, we get, 
%%%%%%%%%%%%%%%%%%%%%%%%%%%%%%%%%%%%%%%%%%%%%%%
\begin{equation}
k_{th}=\sqrt{\frac{\mathcal{P}}{\mathcal{Q}}}~.
\label{eq0526}
\end{equation}
where,
\begin{eqnarray}
\label{eq0527}
\mathcal{P}&=&\frac{a_{0}}{c_{0}^{2}}-\frac{a_{0}d_{1}c^{2}_{2}}{b_{0}c_{0}^{2}d_{0}^{2}}-\frac{a_{1}d_{0}}{c_{0}^{3}}+\frac{a_{1}c_{2}^{2}d_{1}}{b_{0}c_{0}^{3}d_{0}}-\frac{b_{1}^{2}c_{2}^{3}d_{1}}{b_{0}c_{0}^{2}d_{0}^{3}}+\frac{a_{1}d_{0}}{a_{0}b_{0}c_{0}^{2}}~,\\
\label{eq0528}
\mathcal{Q}&=&\frac{b_{0}}{d_{0}^{2}}-\frac{b_{1}}{c_{0}^{2}}-\frac{b_{0}c_{2}}{c_{0}^{3}}+\frac{a_{1}}{a_{0}b_{0}d_{0}}-\frac{a_{1}b_{1}d_{0}}{a_{0}b_{0}^{2}c_{0}^{2}}+\frac{a_{1}c_{2}d_{0}}{b_{0}^{2}c_{0}^{3}}~.
\end{eqnarray}
Expanding $\mathcal{P}$ and $\mathcal{Q}$ and keeping the leading terms of the series we get, 
\begin{eqnarray}
k_{th}&=&\sqrt{\frac{a_{0}d_{0}^{2}}{b_{0}c_{0}^{2}}}\Bigg[1-\frac{1}{2}\Bigg(\frac{d_{1}c_{2}^{2}}{b_{0}d_{0}^{2}}+\frac{a_{1}d_{0}}{a_{0}c_{0}}-\frac{a_{1}c_{2}^{2}d_{1}}{a_{0}b_{0}c_{0}d_{0}}+\frac{b_{1}^{2}c_{2}^{3}d_{1}}{a_{0}b_{0}d_{0}^{3}}-\frac{a_{1}d_{0}}{a_{0}^{2}b_{0}}\Bigg)\Bigg]\nonumber\\
&&\Bigg[1+\frac{1}{2}\Bigg(\frac{b_{1}d_{0}^{2}}{b_{0}c_{0}^{2}}+\frac{c_{2}d_{0}^{2}}{c_{0}^{3}} -\frac{a_{1}d_{0}}{a_{0}b_{0}^{2}}+\frac{a_{1}b_{1}d_{0}^{3}}{a_{0}b_{0}^{3}c_{0}^{2}}-\frac{a_{1}c_{2}d_{0}^{3}}{b_{0}^{3}c_{0}^{3}}\Bigg)\Bigg]
\label{eq0529}
\end{eqnarray}
The first term of the above expression gives rise to the value of $k_{th}$ in the NS limit,  $k_{th}=\sqrt{\frac{a_{0}d_{0}^{2}}{b_{0}c_{0}^{2}}}=\frac{3}{2}c_s\frac{h}{s}\frac{1}{\eta/s}$~\cite{liao}. The subsequent terms are arising from the second-order hydrodynamical effects as indicated by the presence of coupling and relaxation coefficients appearing through the quantities defined in Eqs.~\eqref{eq0503}.


The threshold wavelength ($\lambda_{th}$) corresponding to $k_{th}$ is given by $\lambda_{th}= 2\pi/k_{th}$. Sound waves with  wavelength, $\lambda < \lambda_{th}$ will dissipate in the medium. However, sound waves with $\lambda \ge \lambda_{th}$ will survive to propagate in the fluid without much dissipative effects. The quantity $\lambda_{th}$ can also be used to define the fluidity ($\mathcal{F}$) of the system with widely varying particle density and temperature by selecting a length scale (inter-particle separation), $l\sim \rho^{-1/3}$~\cite{liao} of the system as: 
\begin{equation}
\mathcal{F}\sim \frac{\lambda_{th}}{l}.
\label{eq0530}
\end{equation}
where, $\rho$ is the particle number density of the fluid (for a relativistic fluid $l$ can be chosen as $l\sim s^{-1/3}$). The length scale, $R_v\sim 1/k_{th}$, called viscous horizon \cite{staig}, which fixes the limit for the sound wave with $\lambda$ smaller than $R_v$ will be dissipated due to highly viscous and thermal conduction effects. $R_v$ can also be used to estimate
the value of order of the highest harmonics $n_v=2\pi R/R_v$ ($n=2, 3, 4,...$)  which will survive the dissipation {\it i.e.} any harmonics of order higher than $n_v$ will not survive against dissipation. We find that $k_{th}$ is directly proportional to the $c_{s}$, which approaches  zero at the critical point, that is the sound wave loses its strength and get damped strongly near critical point. Therefore, we can argue that $k_{th}$ also vanishes or in other words $\lambda_{th}$ diverges at the critical point.
 %%%%%%%%%%%%%%%%%%%%%%%%%
%%%%%%%%%%%%%%%%%%
%%%%%%%%%%%%%%%%%%
\section{Results and discussion}
\label{sec0504}
% Figure environment removed
%%%%
%%%%
Now we discuss the dissipation of the perturbation in the fluid when the system hits the CEP in the QCD phase diagram. The damping caused by the imaginary part of the frequency of hydrodynamic modes of perturbation at the critical point $(T_c,\,\mu_{c})=(154,\,367)\,MeV$ is depicted in Fig.\ref{fig0501}. It is seen that the waves with larger (smaller) values of wavenumber ($k$) damp faster (slower). The waves in fluid damp highly for larger values of transport coefficients (right panel). The results displayed in Fig.\ref{fig0502} (left panel) show that when the system is away from the critical point then the waves damp slowly for fluid with higher temperatures and densities. The waves in a medium with higher viscosity and thermal conductivity damp faster (Fig.~\ref{fig0502}, right panel) for obvious reasons.
% Figure environment removed
  %%%%%%%%%%%%%%%%%%
% Figure environment removed
%%%%%%%%%%%%%%%%%%


The variation of $\lambda_{th}$ with temperature ($T/T_c$) is shown in Fig.\ref{fig0503}. It is evident that $\lambda_{th}(=2\pi/{k_{th}})$ depends on the transport coefficients ($\eta, \kappa$) of the system, as well as on the various response functions appearing through the thermodynamic derivatives, 
$\Big(\frac{\partial \epsilon}{\partial T}\Big)_n,\Big (\frac{\partial \epsilon}{\partial n}\Big)_T,$ $\Big(\frac{\partial P}{\partial T}\Big)_n, \Big(\frac{\partial P}{\partial n}\Big)_T, \Big(\frac{\partial n}{\partial \mu}\Big)_T, \Big(\frac{\partial n}{\partial T}\Big)_\mu$, coupling constant ($\alpha_1$), and the relaxation coefficients ($\beta_{1}, \beta_2$). To observe the behaviour of $k_{th}$ through the EoS only, we have not considered the scaling behaviour (near the CEP) of the transport coefficients but we have taken as $\eta/s=\kappa T/s=1/4\pi$. The values of $\Big(\frac{\partial \epsilon}{\partial T}\Big)_n, \Big (\frac{\partial \epsilon}{\partial n}\Big)_T, \Big(\frac{\partial P}{\partial T}\Big)_n, \Big(\frac{\partial P}{\partial n}\Big)_T, \Big(\frac{\partial n}{\partial \mu}\Big)_T, \Big(\frac{\partial n}{\partial T}\Big)_\mu$ are calculated in terms of different response function by using relevant thermodynamic relations (Appendix~\ref{appendix05_A}).  In the left panel of Fig.\ref{fig0503}, the variation of $\lambda_{th}$ with $T/T_c$ is presented. It is observed that at the CEP ($T_c= 154 \text{MeV}$, $\mu_{c}= 367 \text{MeV}$) the $\lambda_{th}$ diverges. As mentioned above, the $\lambda_{th}$ is defined as the wavelength such that waves with wavelengths, $\lambda\geq \lambda_{th}$ are only allowed to propagate and others get dissipated.  It is observed that when we consider $\mu= 347$ MeV and 387 MeV (away from the critical point) a finite value of $\lambda_{th}$ is obtained, that is wave with $\lambda\ge\lambda_{th}$ will propagate in the medium without substantial damping in such cases.  At the critical point, however, $\lambda_{th}$ diverges which infer that waves with all wavelength will dissipate in the fluid.  We also observe that for the lower value of $T$ the value of $\lambda_{th}$ is smaller. At lower temperature region, the magnitude of $\lambda_{th}$ for $\mu=387$ MeV is larger compared to $\mu = 347$ MeV.  This indicates that for higher values of temperature and chemical potential $\lambda_{th}$ is larger. The fluidity defined in Eq.\eqref{eq0530} is directly proportional to $\lambda_{th}$ which diverges at CEP, implies that fluidity also diverges at the CEP. Away from CEP,  the fluidity decreases. The fluidity is larger for $\mu= 387$ MeV compared to $\mu= 347$ MeV.
%%%%%%%%%%%%%%%%%%%%%%%


The viscous damping of a perturbation can be understood
from the relation: 
\beqa
T^{\mn}_{1}(t)= T^{\mu\nu}_{1}(0) exp({-\omega_{Im}t})~,
\label{eq0531}
\eeqa
 where, $T^{\mn}_{1}(0)$ is the perturbation in EMT at $t=0$ and $T^{\mu\nu}_{1}(t)$ is at some later time $t$ which gets dissipated as indicated by the exponential term. The spectrum of the initial $(t=0)$ perturbations can be associated with the harmonics of the shape deformations and density fluctuations \cite{Lacey:2011ug}. The dispersion relation for $\omega$ provides the value, $k_{th}$, which can
be used to define a length scale, $R_v\sim 1/k_{th}$. For system of size $R$, $R_v$ can be used to define $n_{v}= \frac{2\pi R}{R_{v}}$ which is linked to the value of the highest harmonic $n_v$ (eccentricity-driven) that will effectively survive damping. 
We have seen that the nature of the plot, $\lambda_{th}$ vs $T$ 
does not change considerably with the change in shear viscosity ($\eta/s$) but changes significantly with the variation of thermal conductivity ($\kappa T/s$). The right panel of Fig.\ref{fig0503} displays the variation of $\lambda_{th}$ with temperature for higher values of $\kappa T/s$ and $\eta/s$. As the magnitude of $\kappa T/s$ increases the gap between the divergences in the two phases gets narrower. It is well-known~\cite{KapustaChi} that the thermal conductivity diverges at the critical point. Therefore, the nature of the variation
of $\lambda_{th}$ with $T/T_c$ near the CEP will be essentially governed by the thermal conductivity. We have found that with increasing thermal conductivity the width of the divergence gets narrower.
%%%
%%%


