\documentclass[11pt,letterpaper,,usenames,dvipsnames]{article}
\renewcommand{\familydefault}{\rmdefault}
\usepackage[T1]{fontenc}
\usepackage[latin9]{inputenc}
\usepackage{longtable}
\usepackage[authoryear]{natbib}
\usepackage[caption=false]{subfig}
\usepackage{ae,aecompl,geometry,float,framed,rotfloat,amsthm,soul,amsmath,
amssymb,graphicx,todonotes,setspace,enumitem,
booktabs,rotating,microtype,soul,units,bbm,mathtools, comment,	palatino}
\usepackage{xcolor}
\usepackage[hyphens]{url}
\usepackage{dsfont}
\geometry{verbose,tmargin=1in,bmargin=1in,lmargin=1in,rmargin=1in,includefoot}

%\geometry{verbose,tmargin=1in,bmargin=1in,lmargin=1in,rmargin=1in,includefoot}
\usepackage{hyperref}
\hypersetup{colorlinks=true,citecolor=Bittersweet,linkcolor=Bittersweet,urlcolor=Bittersweet}

\setlength{\parskip}{1ex plus 0.5ex minus 0.2ex}
%\setlength{\parskip}{0.5ex plus 0.5ex minus 0.2ex}
%\setlength{\parindent}{7.5ex}
\makeatletter
\newcommand{\noun}[1]{\textsc{#1}}
\newcommand*\diff{\mathop{}\!\mathrm{d}}
\newcommand*\Diff[1]{\mathop{}\!\mathrm{d^#1}}
\newcommand{\mk}[1]{\sethlcolor{yellow} \hl{Mariana: #1}} % highlight parts of the text
\theoremstyle{plain}
\newtheorem{thm}{\protect\theoremname}
  \theoremstyle{plain}
  \newtheorem{lem}{\protect\lemmaname}
  \theoremstyle{definition}
  \newtheorem{defn}{\protect\definitionname}
  \theoremstyle{plain}
  \newtheorem{cor}{\protect\corollaryname}
  \theoremstyle{plain}
  \newtheorem{prop}{\protect\propositionname}
  \theoremstyle{plain}
  \newtheorem*{cor*}{\protect\corollaryname}
\newcounter{asscount}
\setcounter{asscount}{0}
\def\sgn{\mathop{\rm sgn}\nolimits}
\newtheorem{sublemma}{Sublemma}[section]
\newtheorem{subcor}{Corollary}[section]

\usepackage{amsthm}



\newcommand{\minus}{\scalebox{0.6}{$-$}}
\newcommand{\plus}{\scalebox{0.6}{$+$}}


\newtheoremstyle{assumption}
  {0.2cm}{0cm}%                                 margin top and bottom
  {\rmfamily}%                                  text layout
  {0cm}%                                        indention of header
  {\bfseries}{ }%                               header font and text after
  {0cm}%                                        space after header
  {\thmname{#1}\thmnumber{ #2}:\thmnote{ #3}}%  header
\theoremstyle{assumption}
\newtheorem{ass}{Assumption}

\newtheoremstyle{prediction}
  {0.2cm}{0cm}%                                 margin top and bottom
  {\rmfamily}%                                  text layout
  {0cm}%                                        indention of header
  {\bfseries}{ }%                               header font and text after
  {0cm}%                                        space after header
  {\thmname{#1}\thmnumber{ #2}:\thmnote{ #3}}%  header
\theoremstyle{prediction}
\newtheorem{pred}{Prediction}


\@ifundefined{showcaptionsetup}{}{%
 \PassOptionsToPackage{caption=false}{subfig}}
\usepackage{subfig}
\makeatother
  \providecommand{\corollaryname}{Corollary}
  \providecommand{\definitionname}{Definition}
  \providecommand{\lemmaname}{Lemma}
  \providecommand{\propositionname}{Proposition}
\providecommand{\theoremname}{Theorem}

\newcommand{\note}[1]{\hl{#1}} % note highlighted in text
%\renewcommand{\note}[1]{} % do not display notes
\newcommand{\hlt}[1]{\hl{#1}} % highlight parts of the text
%\renewcommand{\hlt}[1]{#1} % turn highlighting off


\newcommand{\minusL}{\scalebox{1.15}{$-$}}
\newcommand{\plusL}{\scalebox{1.15}{$+$}}
\newcommand{\T}{\textbf{T}}
\newcommand{\C}{\textbf{C}}

\usepackage{titling}


%njs
\newcommand{\citec}[1]{\citeauthor{#1} (\citeyear{#1})}
\newcommand{\citecomma}[1]{\citeauthor{#1} (\citeyear{#1})}
\newcommand{\mz}[1]{{\color{NavyBlue}#1}}
\newcommand{\vk}[1]{{\color{red}#1}}
\newcommand{\ak}[1]{\footnote{\color{blue}#1}}
\newcommand{\atk}[1]{{\color{blue}#1}}
\newcommand{\cp}[1]{{\color{violet}#1}}

% \usepackage{draftwatermark}
% \SetWatermarkText{For peer review only}
% \SetWatermarkScale{0.5}

\newcommand{\argmax}{\operatornamewithlimits{arg\,max}}
\newcommand{\myfootnote}[1]{\footnote{#1}}
\newcommand{\AndreasFN}[1]{\footnote{\bfseries \color{medium-blue} #1 \color{black}}}
\newcommand{\Andreas}[1]{{\color{medium-blue} #1 \color{black}}}
\newcommand{\smallthreehalf}{{\textstyle{\frac{3}{2}}}}
\newcommand{\smallhalf}{{\textstyle{\frac{1}{2}}}}
\newcommand{\SB}[1]{\left[ #1 \right]}
\newcommand{\RB}[1]{\left( #1 \right)}
\newcommand{\CB}[1]{\left\{ #1 \right\}}
\newcommand{\dt}{\, dt}
\renewcommand{\d}{d} %\, \text{\normalfont d}}
\newcommand{\E}[1]{\mathbbm{E} \SB{#1}}
\newcommand{\G}{\mathcal{G}}
\newcommand{\LL}{\mathcal{L}_L}
\newcommand{\LH}{\mathcal{L}_H}
\newcommand{\Et}[2]{\mathbbm{E}_{#2} \SB{#1}}
\newcommand{\EQTt}[3]{\mathbbm{E}_{#2}^{\mathbbm{Q}_{#3}} \SB{#1}}
\newcommand{\EQt}[2]{\mathbbm{E}_{#2}^{\mathbbm{Q}} \SB{#1}}
\newcommand{\VQt}[2]{\mathbbm{V}_{#2}^{\mathbbm{Q}} \SB{#1}}
\newcommand{\EQ}[1]{\mathbbm{E}^{\mathbbm{Q}} \SB{#1}}
\newcommand{\EG}[2]{\mathbbm{E} \SB{\left. #1 \right| #2 }}
\newcommand{\p}[2]{p \RB{\left. #1 \right| #2 }}
\newcommand{\JIX}{J \hspace{-0.08cm} I \hspace{-0.08cm} X }
\newcommand{\VSP}{V \hspace{-0.08cm} S \hspace{-0.08cm} P }
\newcommand{\RV}{R \hspace{-0.05cm} V }
\newcommand{\IV}{I \hspace{-0.05cm} V }
\newcommand{\RS}{R \hspace{-0.05cm} S }
\newcommand{\IS}{I \hspace{-0.05cm} S }
\newcommand{\ISC}{I \hspace{-0.05cm} S \hspace{-0.05cm} C}
\newcommand{\CF}{\Psi}
\newcommand{\defeq}{\overset{\mathrm{def}}{=\joinrel=}}
\newcommand{\CFG}{\Gamma}
\newcommand{\qmg}{q_\text{mg}}
\newcommand{\B}{\mathcal{B}}
\newcommand{\citetcomma}[1]{\citeauthor{#1}, \citeyear{#1}}
\newcommand{\TQ}{\Q}
\newcommand{\TP}{\P}
\newcommand{\LT}{\textbf{LT}}
\newcommand{\LP}{\textbf{LP}}
\newcommand{\A}{\textbf{A}}
\newcommand{\VIX}{\text{\normalfont VIX}}
\newcommand{\VIXF}{\text{\normalfont VF}}
\newcommand{\VC}{\text{\normalfont VC}}
\newcommand{\Q}{\mathbbm{Q}}
\renewcommand{\P}{\mathbbm{P}}
\renewcommand{\d}{\, \text{\normalfont d}}
% \newcommand{\cp}[2]{p\RB{ #1 \, | \, #2 }}
\newcommand{\pp}[1]{p\RB{ #1 }}
\renewcommand{\citet}[1]{\cite{#1}}
\newcommand{\citetc}[1]{\citeauthor{#1}, \citeyear{#1}}
\newcommand{\citetwo}[2]{\citeauthor{#1} (\citeyear{#1}, \citeyear{#2})}
\newcommand{\MM}{\textbf{MM}}
\usepackage[normalem]{ulem}


%%\usepackage[subtle]{savetrees}
%\usepackage[margin=2cm]{geometry}
\usepackage{tikz,amsmath, amssymb,bm,color, amsthm,amsfonts}
\usetikzlibrary{positioning, calc,chains,fit,shapes}
%\usetikzlibrary{circuits.logic.US,circuits.logic.IEC,fit}
\usepackage{enumerate}
\usepackage{comment}
\usepackage{tikz}
\usepackage{graphics}
%\usepackage[cm]{fullpage}
\usepackage{longtable}
\usepackage{mdframed}
\usepackage{caption}
\usepackage{subcaption}
\usepackage{slashbox}
\usepackage{url}
\usepackage{framed}
\usepackage{array}
\usepackage{tabu}
\usepackage{lscape}
\usepackage{multirow}
\usepackage{ulem}
\usepackage{multicol}
\usepackage{placeins}
\usepackage{cite}
\usepackage{enumitem}
\usepackage{mathtools}
%\usepackage[numbers]{natbib}
%\usepackage{mathtools}
%\usepackage{authblk}

\mdfsetup{skipabove=2pt,skipbelow=2pt}
%\setlenght {\marginparwidth }{2cm}
%\usepackage{todonotes}

%\usepackage{floatrow}
%\usepackage{adjustbox}
%\setlength{\extrarowheight}{.05ex}
%\renewcommand\thesubfigure{\roman{subfigure}}


%\newtheorem{theorem}{Theorem}[section]
%\newtheorem{lemma}[theorem]{Lemma}
%\newtheorem{observation}[theorem]{Observation}
%\newtheorem{corollary}[theorem]{Corollary}
%\newtheorem{proposition}[theorem]{Proposition}
%\newtheorem{definition}[theorem]{Definition}
\newtheorem{construction}{Construction}
%\newtheorem{conjecture}{Conjecture}
%\newtheorem{remark}[theorem]{Remark}

\newcommand{\pname}[1]{\textnormal{\textsc{#1}}}
\newcommand{\cclass}[1]{\textnormal{\textsf{#1}}}
\newcommand{\nog}{nine} % no of members in the gang!
\newcommand{\nogd}{nineteen} % no of members in the gang - for deletion/completion
\newcommand{\nogl}{eighteen} % no of members in the larger gang - for editing
\newcommand{\nogld}{thirty eight} % no of members in the larger gang - for deletion/completion
\newcommand{\diffnog}{ten} %
%\newcommand{\dominatedby}{dominated by} %
%\newcommand{\dominatingset}{dominating set} %
%\newcommand{\dominates}{dominates} %
\newcommand{\simulates}{simulates} %
\newcommand{\baseset}{base} %
\newcommand{\issimulatedby}{is simulated by} %

\newcommand{\StarSAT}{\pname{8-SAT$_{\geq 6}$}}
\newcommand{\FSAT}{\pname{4-SAT$_{\geq 2}$}}
\newcommand{\FISAT}{\pname{5-SAT$_{\geq 3}$}}
\newcommand{\SIXSAT}{\pname{6-SAT$_{\geq 4}$}}
\newcommand{\ESAT}{\pname{8-SAT$_{\geq 6}$}}
\newcommand{\KSAT}{\pname{$k$-SAT$_{\geq {k-2}}$}}
\newcommand{\KSATO}{\pname{$k$-SAT}}
\newcommand{\ESATO}{\pname{8-SAT}}
\newcommand{\FSATO}{\pname{4-SAT}}
\newcommand{\FISATO}{\pname{5-SAT}}
\newcommand{\TSAT}{\pname{3-SAT}}
\newcommand{\HED}{\pname{${H}$-free Edge Deletion}}
\newcommand{\AEE}{\pname{${A}$-free Edge Editing}}
\newcommand{\AED}{\pname{${A}$-free Edge Deletion}}
\newcommand{\TSED}{\pname{$t$-star-free Edge Deletion}}
\newcommand{\ATSED}{\pname{Annotated $t$-star-free Edge Deletion}}
\newcommand{\AFSED}{\pname{Annotated $4$-star-free Edge Deletion}}
\newcommand{\FSED}{\pname{$4$-star-free Edge Deletion}}
\newcommand{\FVSED}{\pname{$5$-star-free Edge Deletion}}
\newcommand{\HEE}{\pname{${H}$-free Edge Editing}}
\newcommand{\HEC}{\pname{${H}$-free Edge Completion}}
\newcommand{\HDEE}{\pname{${H'}$-free Edge Editing}}
\newcommand{\HDDEE}{\pname{${H''}$-free Edge Editing}}
\newcommand{\HDED}{\pname{${H'}$-free Edge Deletion}}
\newcommand{\HDEC}{\pname{${H'}$-free Edge Completion}}
\newcommand{\HBEE}{\pname{${\overline{H}}$-free Edge Editing}}
\newcommand{\HBED}{\pname{${\overline{H}}$-free Edge Deletion}}
\newcommand{\HBEC}{\pname{${\overline{H}}$-free Edge Completion}}
\newcommand{\HOEDCE}{\pname{${H_1}$-free Edge Deletion(Completion/Editing)}}
\newcommand{\HEDCE}{\pname{${H}$-free Edge Deletion(Completion/Editing)}}
\newcommand{\HEEDC}{\pname{${H}$-free Edge Editing(Deletion/Completion)}}
\newcommand{\HDEEDC}{\pname{${H'}$-free Edge Editing(Deletion/Completion)}}
\newcommand{\BFED}{\pname{Bow-free Edge Deletion}}
\newcommand{\ABFED}{\pname{Annotated Bow-free Edge Deletion}}
\newcommand{\DTIS}{\pname{Distance-3 Independent Set}}
\newcommand{\SVC}{\pname{Strong Vertex Cover}}
\newcommand{\CLIQUE}{\pname{Clique}}
\newcommand{\IS}{\pname{Independent Set}}
\newcommand{\PFS}{\pname{Propagational-$f$ Satisfiability}}
\newcommand{\RHED}{\pname{Restricted ${H}$-free Edge Deletion}}
\newcommand{\RHEC}{\pname{Restricted ${H}$-free Edge Completion}}
\newcommand{\RHDED}{\pname{Restricted ${H'}$-free Edge Deletion}}
\newcommand{\RHDEC}{\pname{Restricted ${H'}$-free Edge Completion}}
\newcommand{\RHEE}{\pname{Restricted ${H}$-free Edge Editing}}
\newcommand{\PH}{$\cclass{NP} \subseteq \cclass{coNP/poly}$}
\newcommand{\NOPH}{$\cclass{NP} \not\subseteq \cclass{coNP/poly}$}
\newcommand{\LG}{\mathcal{W}}
\newcommand{\LGD}{\mathcal{W}'}
\newcommand{\LGDD}{\mathcal{W}''}


%\let\oldvee\vee
\renewcommand\vee{\boxtimes}

\newcommand\addvmargin[1]{
  \node[fit=(current bounding box),inner ysep=#1,inner xsep=0]{};
}
\setlength{\fboxrule}{0pt}

\newcommand{\defstage}[2]{% PGD Version
  \hfill\\\smallskip\noindent%
  \begin{tabularx}{\textwidth}{|l X|}%
    \hline%
    \multicolumn{2}{|l|}{\textbf{#1}}\\%
    &#2\\\hline%
  \end{tabularx}%
%  \smallskip%
}
\setlength\extrarowheight{15pt}

\newcounter{rowcntr}[table]
\renewcommand{\therowcntr}{\thetable.\arabic{rowcntr}}

% A new columntype to apply automatic stepping
\newcolumntype{N}{>{\refstepcounter{rowcntr}\therowcntr}c}

% Reset the rowcntr counter at each new tabular
\AtBeginEnvironment{longtabu}{\setcounter{rowcntr}{0}}

\newcounter{rowcntra}[table]
\renewcommand{\therowcntra}{\arabic{rowcntra}}

% A new columntype to apply automatic stepping
\newcolumntype{M}{>{\refstepcounter{rowcntra}\therowcntra}c}

% Reset the rowcntr counter at each new tabular
\AtBeginEnvironment{tabular}{\setcounter{rowcntra}{0}}

\newcommand{\NPC}{NP-Complete}


\newcommand{\highlight}[1]{\textcolor{blue}{#1}}
\newcommand{\dhanya}[1]{\textcolor{blue}{dhanya: #1}}


%\newcommand{\XCD1}[1]{\pname{$\chi_{cd}$\ensuremath{(#1)}}}
\newcommand{\XCD}{\pname{$\chi_{cd}$}}
\newcommand{\SC}{\pname{$\omega_{s}$}}

\newcommand{\CDC}{\textsc{CD-coloring}}
\newcommand{\SCP}{\textsc{Separated-Cluster}}
\newcommand{\TD}{\textsc{Total Domination}}
\newcommand{\ISP}{\textsc{Independent Set}}
\newcommand{\CC}{\textsc{Clique Cover}}
\newcommand{\TETHS}{Further, the problem cannot be solved in time \ensuremath{2^{o(|V(G)|)}}, unless the ETH fails}
%\usetikzlibrary{positioning,chains,shapes,calc}
\usetikzlibrary{fit}
\thispagestyle{empty}
\usetikzlibrary{
  graphs,
  graphs.standard
}

\begin{document}


\newcommand{\titlepaper}{Liquidity fragmentation on \\ decentralized exchanges}


\doublespacing  %Mariana's editing

\title{\textbf {\huge \titlepaper}}

\newcommand{\keywords}{FinTech, decentralized exchanges (DEX), liquidity, fragmentation, retail trading}
\newcommand{\JELcodes}{G11, G12, G14}

\newcommand{\mazabstract}{\noindent 
% Liquidity providers (LPs) on decentralized exchanges pay a fixed transaction cost (gas price) whenever they update their positions. Different economies of scale across LPs lead in equilibrium to the fragmentation of liquidity supply between low- and high-fee pools. Using data on liquidity updates from Uniswap, we document that while high-fee pools attract 56\% of liquidity supply, they only execute 35\% of trading volume. Low-fee pools cater to large (institutional) LPs, who update positions frequently in response to large trading volume. In contrast, small (retail) LPs converge to high-fee pools, trading off lower execution probabilities against a smaller liquidity management cost.

\noindent We study economies of scale in liquidity provision on decentralized exchanges, focusing on the impact of fixed transaction costs such as gas prices on liquidity providers (LPs). Small LPs are disproportionately affected by the fixed cost, resulting in liquidity supply fragmentation between low- and high-fee pools. Analyzing Uniswap data, we find that high-fee pools attract 58\% of liquidity supply but execute only 21\% of trading volume. Large (institutional) LPs dominate low-fee pools, frequently adjusting positions in response to substantial trading volume. In contrast, small (retail) LPs converge to high-fee pools, accepting lower execution probabilities to mitigate smaller liquidity management costs.

\bigskip{}
}


\author{\Large Alfred Lehar, Christine Parlour, and Marius Zoican\thanks{Alfred Lehar (\href{alfred.lehar@haskayne.ucalgary.ca}{alfred.lehar@haskayne.ucalgary.ca}) is affiliated with Haskayne School of Business at University of Calgary. Christine Parlour (\href{parlour@berkeley.edu}{parlour@berkeley.edu}) is with Haas School of Business at UC Berkeley. Marius Zoican (\href{marius.zoican@rotman.utoronto.ca}{marius.zoican@rotman.utoronto.ca}) is affiliated with University of Toronto Mississauga and Rotman School of Management. We have greatly benefited from discussion on this research with Michael Brolley, Itay Goldstein, Olga Klein, Katya Malinova, Uday Rajan, Thomas Rivera, Gideon Saar, Andriy Shkilko (discussant), and Shihao Yu. 
% We thank Xiaopeng Wu for excellent research assistance. 
We are grateful to 
conference participants at the Financial Intermediation Research Society 2023, the Northern Finance Association 2023, the UNC Junior Faculty Finance Conference, as well as to 
seminar participants at the Microstructure Exchange. 
Marius Zoican gratefully acknowledges funding support from the Rotman School of Managements' FinHub Lab and the Canadian Social Sciences and Humanities Research Council (SSHRC) through an Insight Development research grant (430-2020-00014).
}}

%\date{}

\maketitle


\vspace{-10mm}

% \begin{center}
%     \emph{Preliminary and incomplete. Please do not cite.}
% \end{center}

\begin{abstract}
\mazabstract

\noindent \textbf{Keywords}: \keywords

\noindent \textbf{JEL Codes}: \JELcodes
% \bigskip{}



\thispagestyle{empty}

\newpage{}
\thispagestyle{empty}
\end{abstract}

\vfill{}

\vfill{}

\pagebreak{}

\vspace*{20mm}
\begin{center}
\huge \titlepaper
\par\end{center}{\Large \par}

\vspace{12mm}

\bigskip{}
\begin{abstract}
\mazabstract

\bigskip{}


\noindent \textbf{Keywords}: \keywords 

\noindent \textbf{JEL Codes}: \JELcodes
\bigskip{}

\newpage{}
\setcounter{page}{1}
\end{abstract}



\newpage
\setcounter{page}{1}

\section{Introduction \label{sec:introduction}}
\section{Introduction}
Current quantum hardware is unable to carry out universal quantum computations due to the buildup of errors that occur during the computation. 
The magnitude of the individual error is currently above the value that the Threshold Theorem requires in order to kick-start quantum error correction and fault-tolerant quantum computation~\cite[Section 10.6]{nielsen_chuang_2010}. 
Although the experimentally achieved fidelity rates are promising and the error bounds are inching closer to the required threshold, we will have to work for the foreseeable future with quantum hardware with errors that build-up during the computation.  This implies that we can only do a limited number of steps before the output of the computation has become completely uncorrelated with the intended one.

For fault-tolerant quantum computing, we repeat four steps: 
1) We apply a number of single and two-qubit quantum gates, in parallel whenever possible; 
2) We perform a syndrome measurement on a subset of the qubits; 
3) We perform fast classical computations to determine which errors have occurred and how to correct them; 
and, 4) We apply correction terms based on the classical computations.
We then repeat these four steps with a next sequence of gates. 
These four steps are essential to fault-tolerant quantum computing. 


The starting point of this work is to use the four steps outlined above, not to carry out error correction and fault-tolerant computation, but to enhance short, constant-depth, {\em uncorrected} quantum circuits that perform single qubit gates and {\em nearest-neighbor} two qubit gates. 
Since in the long run we will have to implement error-correction and fault-tolerant computation anyhow, and this is done by such a four-step process, why not make other use of this architecture? Moreover, on some of the quantum hardware platforms, these operations are already in place.
Embracing this idea we naturally arrive at the question: what is the computational power of \textit{low-depth} quantum-classical circuits organized as in the four steps outlined above? 
We thus investigate circuits that execute a small, ideally constant, number of stages, where at each stage we may apply, in parallel, single qubit gates and {\em nearest-neighbor} two qubit gates, followed by measurements, followed by low-depth classical computations of which the outcome can control quantum gates in later stages. 
It is not clear, at first, whether such circuits, especially with constant depth, can do anything remotely useful. 
But we will see that this is indeed the case: many quantum computations can be done by such circuits in constant depth. 
By parallelizing quantum computations in this way, we improve the overall computational capabilities of these circuits, as we do not incur errors on qubits that are idle, simply because qubits are not idle for a very long time. 
Furthermore, reducing the depth of quantum circuits, at the cost of increasing width, allows the circuit to be run faster even if errors occur.

The first usage of such a four-step layout, not to do error correction, but to perform computations, can be found in the paradigm of measurement-based quantum computing~\cite{gottesman1999demonstrating,raussendorf2001one,jozsa2006introduction,clark2007generalised}: 
A universal form of quantum computing where a quantum state is prepared and operations are performed by measuring qubits in different bases, depending on previous measurements and intermediate measurements.

\citeauthor{PhamSvore2013} were the first to formalize the four-step protocol for performing computations~\cite{PhamSvore2013}. They included specific hardware topologies by considering two-dimensional graphs for imposing constraints on qubit interactions. In their model, they develop circuits for particularly useful multi-qubit gates, including specifying costs in the width, number of qubits, depth, number of concurrent time steps, size, and total number of non-Identity operations.
As a result, they find an algorithm that factors integers in polylogarithmic depth.
\citeauthor{Browne:2011} showed that the main tool in the work by \citeauthor{PhamSvore2013}, the fan-out gate, can also be replaced by additional log-depth classical computations in the measurement-based quantum computing setting~\cite{Browne:2011}.

More recently, \citeauthor{Cirac:2021} introduced a scheme to implement unitary operations involving quantum circuits combined with Local Operations and Classical Communication ($\mathsf{LOCC}$) channels: $\mathsf{LOCC}$-assisted quantum circuits~\cite{Cirac:2021}. Similarly to the four-step scheme we just described, they allow for a short depth circuit to be run on the qubits, followed by one round of $\mathsf{LOCC}$, in which ancilla qubits are measured and local unitaries are applied based on the measurement outcomes. They show that in this model any 1D transitionally invariant matrix-product state (MPS) with fixed bond dimension is in the same phase of matter as the trivial state. Similar ideas can be found in~\cite{TVV_NonAbelianTopologicalOrder_2022, tantivasadakarn2021long}.

In this work, we introduce a new model, called \textit{Local Alternating Quantum-Classical Computations} ($\LAQCC$). In this model we alternate between running quantum circuits (constrained by locality), ending in the measurement of a subset of qubits, and fast classical computations based on the measurement results. The outcome of the classical computations are then used to control future quantum circuits. We allow for flexibility in this model, by giving different constraints to the power of both the quantum circuits and the classical circuits as well as the number of alternations between them. 
Most attention will be given to $\LAQCC$ containing quantum circuits of constant depth, classical circuits of logarithmic depth and at most a constant number of alternations between them. 
Any circuit constructed in this model is considered to be of constant depth. 
We restrict ourselves to logarithmic depth classical computations, as this is the first natural and non-trivial extension beyond constant-depth classical computations. 
Constant-depth classical computations do however also have an equivalent constant-depth quantum implementation.

The definition of $\LAQCC$ sharpens the original definition of \citeauthor{PhamSvore2013} by adding constraints to the intermediate classical computations. This allows us to bound the power of $\LAQCC$ from above. 

The main result of \citeauthor{Cirac:2021}, that 1D translational invariant MPS with fixed bond dimension can be prepared by $\mathsf{LOCC}$-assisted circuits, relies on local symmetries of the MPS. These symmetries allow them to prepare local states (on a constant number of qubits) and glue them together by doing one round of the appropriate entangling measurement and corrections, after which they run a round of local unitaries to get the desired result. This general scheme for preparing states that exhibit an MPS description with the appropriate local symmetries requires only geometrically local unitaries and one round of measurement and corrections an therefore is accessible in $\LAQCC$. Studying different local symmetries, known as Symmetry Protected Topological (SPT) phases of matter, to find measurement-based constant depth circuits for states is a broad ongoing field of research~\cite{TVV_NonAbelianTopologicalOrder_2022, tantivasadakarn2021long, smith2023deterministic}. 
All these schemes have a $\LAQCC$ implementation.

%$\LAQCC$-circuits also exist for general schemes of preparing local states, based on the local tensors, and gluing them together using one round of entangled measurement and corrections, based on the local symmetry. 
%The main result of \citeauthor{Cirac:2021}, that 1D translational invariant MPS with fixed bond dimension can be prepared by $\mathsf{LOCC}$-assisted circuits, relies heavily on local symmetries of the MPS and as a result also has an equivalent $\LAQCC$ implementation. 
%The corrections applied after the measurement round are local unitaries depending on the local symmetries of the MPS. 

 

%This general scheme of preparing local states, based on the local tensors, and gluing it together by doing one round of entangled measurement and corrections, based on the local symmetry, is accessible in $\LAQCC$.
Note however that \citeauthor{Cirac:2021} also suggest a circuit for the $W$-state.
This circuit uses sequentially and dependent measurement-based corrections of the ancilla qubits. 
These dependent measurements translate to sequential alternations between the quantum and classical circuits and therefore increase the total depth to linear depth, exceeding the constant-depth constraints imposed by $\LAQCC$-circuits. 

We study the power of the $\LAQCC$ model with respect to state preparation, showing that even with only constant quantum-depth and logarithmic classical depth it remains possible to prepare states with long-range entanglement.
Another surprising result is that it is unlikely that $\LAQCC$ circuits are classically simulatable. We show that any instantaneous quantum polynomial-time (IQP) circuit~\cite{Bremner2010,Shepherd2009} has an $\LAQCC$ implementation.
Classical simulation of IQP circuits implies the collapse of the polynomial hierarchy to the third level, which is not believed to be true~\cite{Bremner2017}. Therefore, we expect that $\LAQCC$ circuits are unlikely to be classically simulatable. We bound the power of $\LAQCC$ by showing that it is contained in $\QNC^1$, the class of polynomial-size, log-depth circuits.

Next, we also study the power that intermediate classical calculations can add to quantum computations, by considering a new model that alternates between polynomially many polynomial-depth quantum circuits and unbounded classical computations
We study this model by doing a complexity theoretical analysis, where we draw inspiration from the notions of complexity given by \citeauthor{RosenthalYuen:2022}, \citeauthor{MetgerYuen:2023}, and \citeauthor{Aaronson:2004}.
All three complexity notions are based on the notion of state preparation, instead of more traditional definition of complexity such as the decidability of a computational problem. 
The first two consider classes based on sequences of quantum states preparable by a polynomial-sized quantum circuit, where the circuits are uniformly generated by a computational class, for instance, the class $\mathsf{PSPACE}$, which results in the complexity class $\mathsf{StatePSPACE}$~\cite{RosenthalYuen:2022,MetgerYuen:2023}.
The third notion considers a relative complexity, where the complexity is measured between two given states, and is measured by the number of gates, from a given gate-set, required to transform one state in another state~\cite{Aaronson:2004}. 
For our definition of state preparation complexity, we drop the uniformity constraint from~\cite{RosenthalYuen:2022,MetgerYuen:2023} and define a class as $\mathsf{StateX}$, which refers to states preparable by circuits of type $\mathsf{X}$. 
As an example, if $\mathsf{X} = \QNC^0$, this results in the class $\mathsf{StateQNC^0}$, which is the set of states preparable from the $\ket{0}^n$ state by poly-size constant-depth circuits. 
This notion is similar to the relative complexity from~\cite{Aaronson:2004}, where one state is the  $\ket{0}^n$ state and instead of counting the number of gates we consider the set of states preparable by a fixed number of gates. Using this notion of complexity we show that any state preparable by an $\LAQCC^*$ circuit is also preparable by a $\mathsf{PostQPoly}$ circuit, the class of circuits of polynomial depth with an additional post-selection gate. 

All Clifford circuits have a constant-depth $\LAQCC$ implementation, implying that any stabilizer state can be implemented by a constant-depth $\LAQCC$ circuit, see Section~\ref{sec:clifford_circuits} for a proof of this statement. 
Efficient circuits for stabilizer states have been known already through measurement-based quantum computing. Therefore this paper focuses on the preparation of non-stabilizer states, and as a surprising result we find novel constant-depth protocols for four very natural classes of non-stabilizer states.
Despite the extensive research into these four classes of non-stabilizer states and the many applications of them, no efficient constant- or low-depth state preparation protocols are known yet. We specifically consider these four classes as they are all often used as initial states in other algorithms.

The first state is a uniform superposition over an arbitrary number of states. 
This state finds applications in many quantum algorithms, as they often start with a uniform superposition over multiple states. 
This superposition is often achieved by applying Hadamard gates to every qubit due to its simplicity to prepare. 
Yet, the analysis of many algorithms, such as Shor's algorithm~\cite{Shor:1997}, would benefit from a different initial superposition. 
The circuit to prepare the uniform superposition over an arbitrary number of states uses an exact version of Grover search as a subroutine, that turns a probabilistic circuit, with a known constant probability of success, into a deterministic circuit. 
We use the circuit for preparing a uniform superposition over an arbitrary number of states as a subroutine in the next two quantum state preparation protocols. 

The second state is the $W$-state, the uniform superposition over all computational basis states of Hamming-weight~$1$, a natural long-ranged entangled state that displays a fundamentally nonequivalent type of entanglement from the Greenberger–Horne–Zeilinger state~\cite{WState:2000}, for which $\LAQCC$-type constant-depth circuits were previously known~\cite{PhamSvore2013, Cirac:2021}. 
The $W$-state is often used as benchmark for new quantum hardware~\cite{Haffner2005,Neeley2010,GarciaPerez:2021}. 
A novel way to prepare the $W$-state therefore gives a new way to benchmark different quantum devices with each other. 
A circuit for preparing the $W$-state was given in~\cite{Cirac:2021}, but this implementation requires sequentially alternating measurements followed by local unitaries, which in the $\LAQCC$ model is not considered to be of constant depth. 
We improve this protocol by giving an $\LAQCC$ implementation of the $W$-state, based on a compress-uncompress method that links the one-hot and binary encoding of integers.

The third state considered is the Dicke state, a generalization of the $W$-state, a superposition over all computational basis states with Hamming-weight $k$~\cite{Dicke:1954}. 
Dicke states have relevance in various practical settings.
For instance, for quantum game theory~\cite{zdemir2007}, quantum storage~\cite{Bacon_Compress:2006,Plesch:2010}, quantum error correction~\cite{ouyang2014permutation}, quantum metrology~\cite{toth2012multipartite}, and quantum networking~\cite{prevedel2009experimental}. 
Dicke states have been used as a starting state for variational optimization algorithms, most notably Quantum Alternating Operator Ansatz (QAOA)~\cite{Hadfield2019}, to find solutions to problems such as Maximum k-vertex Cover~\cite{Brandhofer2022,cook2020quantum}.
The ground states of physical Hamiltonians describing one-dimensional chains tend to show a resemblance to Dicke states such as states resulting from the Bethe ansatz, making them an ideal starting state when investigating the ground state behavior of these Hamiltonians~\cite{TDL_BetheAnsatzDerivation:2010,B_ExcitedStateQuantumPhaseTransitions:2013,DickeTransitions:2021}. 
For instance, the algorithm by \citeauthor{van2021preparing}, who give an algorithm to prepare the Bethe ansatz eigenstates of the spin-1/2 XXZ spin chain, starts by first preparing a Dicke state~\cite{van2021preparing}. 
A Dicke-state preparation protocol based on the compress-uncompress methodology used in the $W$-state furthermore finds applications in entanglement distillation, where the entanglement of a large state is concentrated on only a few qubits. 
Efficient deterministic circuits for preparing Dicke states have been proposed by \citeauthor{bartschi2019deterministic}~\cite{bartschi2019deterministic, bartschi2022deterministic_short_depth}. 
They provide a quantum circuit of depth $\mathO(k \log(\frac{n}{k}))$, allowing arbitrary connectivity, to prepare a Dicke state, which they conjecture to be optimal when $k$ is constant. 
In this work, we provide a constant-depth $\LAQCC$ circuit below their conjectured bound already for constant $k$. 
However, this does not directly disprove their conjecture, as we allow for intermediate measurements and classical computations. 
More significantly, we even construct constant-depth $\LAQCC$ circuits for $k = \mathO(\sqrt{n})$ greatly improving their bound.
This construction extends the compress-uncompress method for the $W$-state combined with additional subroutines. 

We continue with a log-depth state preparation protocol for the Dicke-state for arbitrary $k$. 
This protocol implements an efficient transformation between the factoradic number representation and the combinatorial number representation of a positive integer. 
The combinatorial number representation relates directly to the Dicke state. 
The provided efficient transformation between number representation systems might be of independent interest. 

We conclude by modifying our protocol for preparing a Dicke-state to a protocol that prepares quantum many-body scar states in constant-depth. 
These states have low entanglement and longer coherence times than states with similar energy density.
These characteristics make many-body scar states interesting to analyze and relevant within physics.
Many-body scar states appear for instance in the AKLT model~\cite{AKLT:1987,MRBAR:2018,MRB:2018} and different spin models~\cite{SI:2019,MOBFR:2020}.
Known methods for preparing these states have polynomial-depth~\cite{Gustafson:2023}, whereas our circuit has constant depth. 

% We conclude by studying the power that intermediate classical calculations can add to quantum computations. 
% In this study, we define a new model that relaxes constant-depth quantum circuits to polynomial depth quantum circuits, log-depth classical calculations to unbounded classical computations and a constant number of alternations to a polynomial number of alternations. 
% We call this model $\LAQCC^*$. 
% We study this model by doing a complexity theoretical analysis, where we draw inspiration from the notions of complexity given by \citeauthor{RosenthalYuen:2022}, \citeauthor{MetgerYuen:2023}, and \citeauthor{Aaronson:2004}.
% All three complexity notions are based on the notion of state preparation, instead of more traditional definition of complexity such as the decidability of a computational problem. 
% The first two consider classes based on sequences of quantum states preparable by a polynomial-sized quantum circuit, where the circuits are uniformly generated by a computational class, for instance, the class $\mathsf{PSPACE}$, which results in the complexity class $\mathsf{StatePSPACE}$~\cite{RosenthalYuen:2022,MetgerYuen:2023}.
% The third notion considers a relative complexity, where the complexity is measured between two given states, and is measured by the number of gates, from a given gate-set, required to transform one state in another state~\cite{Aaronson:2004}. 
% For our definition of state preparation complexity, we drop the uniformity constraint from~\cite{RosenthalYuen:2022,MetgerYuen:2023} and define a class as $\mathsf{StateX}$, which refers to states preparable by circuits of type $\mathsf{X}$. 
% As an example, if $\mathsf{X} = \QNC^0$, this results in the class $\mathsf{StateQNC^0}$, which is the set of states preparable from the $\ket{0}^n$ state by poly-size constant-depth circuits. 
% This notion is similar to the relative complexity from~\cite{Aaronson:2004}, where one state is the  $\ket{0}^n$ state and instead of counting the number of gates we consider the set of states preparable by a fixed number of gates. Using this notion of complexity we show that any state preparable by an $\LAQCC^*$ circuit is also preparable by a $\mathsf{PostQPoly}$ circuit, the class of circuits of polynomial depth with an additional post-selection gate. 

\paragraph{Summary of results}
\begin{itemize}
    \item We give a new definition of a computational model that captures the power of the four step process: applying a constant number of layers of one- and two-qubit gates; performing a syndrome measurement; perform a fast classical computation determining corrections; apply corrections. We call this model \emph{Local Alternating Quantum Classical Computations}, or $\LAQCC$ for short. In this model we bound the allowed quantum operations, intermediate classical calculations, and number of rounds separately. In Section~\ref{sec:LAQCC_model} we define this model and give a list of operations based on results from literature contained in this computational model. In some of these operations we explicitly use that we allow for multiple, but at most constant, rounds  of corrections.
    \item  We show show that there exist $\LAQCC$ circuits that can not be weakly simulated in Section~\ref{sec:IQP_in_LAQCC}. We further show that for every $\LAQCC$ circuit there exists a $\QNC^1$ circuit simulating it perfectly, in Section~\ref{sec:LAQCC_in_QNC1}.
    \item We introduce a new type computational complexity for preparing states and show that the extension of $\LAQCC$ where we allow a polynomial number of rounds and unbounded classical computation, is contained in $\mathsf{PostQPoly}$, the class of polynomial circuits with post-selection, in Section~\ref{sec:Complexity results}.
    \item We show a protocol to prepare the uniform superposition state of size $q$ in $\LAQCC$ using $\mathO(\ceil{\log_2(q)}^2)$ qubits in Section~\ref{sec:superposition_modulo_q}. 
    \item We show a protocol to prepare the $W_n$ state in $\LAQCC$ using $\mathO(n\log(n))$ qubits in Section~\ref{sec:W_state_in_LAQCC}.
    \item We show two ways of preparing the Dicke-$(n,k)$ state. The first method is in $\LAQCC$, works up to $k = \mathO(\sqrt{n})$, uses $\mathO(n^2\log(n))$ qubits, and is found in Section~\ref{sec:dicke:small_k}. The second method is in $\LAQCC\text{-}\mathsf{LOG}$ (an extension of $\LAQCC$ allowing for logarithmic number of alterations instead of constant), works for any $k$, uses $\mathO(\text{poly}(n))$ qubits, and is found in Section~\ref{sec:Dicke_in_LAQCC_LOG}. 
    \item We extend on our $\LAQCC$ method of generating Dicke-$(n,k)$ states for $k = \mathO(\sqrt{n})$ and show a protocol to generate many-body scar states for a particular Hamiltonian in $\LAQCC$ (Section~\ref{sec:many_body_scar}). 
\end{itemize}
Summarized in a table, we provide the following state generation protocols:
\begin{table}[htb]
\centering
\begin{tabular}{l|l|l|l}
\textbf{State description} & \textbf{Width} & \textbf{Depth} & \textbf{Implementation}\\
\hline 
Uniform superposition mod $q$: $\frac{1}{\sqrt{q}} \sum_{i = 0}^{q-1}\ket{i}$ & $\mathO(\ceil{\log^2 q})$ & $\mathO(1)$ & Section~\ref{sec:superposition_modulo_q}\\

$W$-state: $\frac{1}{\sqrt{n}}\sum_{i = 0}^{n-1}\ket{e_i}$ & $\mathO(n \log n)$ & $\mathO(1)$ & Section~\ref{sec:W_state_in_LAQCC}\\

Dicke-$(n,k)$, $k = \mathO(\sqrt{n})$: $\binom{n}{k}^{-1/2}\sum_{x \in \{0,1\}^n: |x| = k} \ket{x}$ &  $\mathO(n^2\log n)$ & $\mathO(1)$ 
&Section~\ref{sec:dicke:small_k}\\

Dicke-$(n,k)$: $\binom{n}{k}^{-1/2}\sum_{x \in \{0,1\}^n: |x| = k} \ket{x}$ & $\mathO(\text{poly}(n))$ & $\mathO(\log n)$ &Section~\ref{sec:Dicke_in_LAQCC_LOG}\\

QMBS: $\ket{S_k} = \frac{1}{k! \sqrt{\mathcal N(n,k)}}(Q^\dagger)^k \ket{\Omega}$ &  $\mathO(n^2\log n)$ & $\mathO(1)$  &  Section~\ref{sec:many_body_scar}
\end{tabular}
\caption{Summary of state preparation protocols given in this paper.}
\label{tab:sate_prep}
\end{table}
In the entry for the quantum many-body scar state $Q$ denotes the raising operator and $\mathcal N(n,k)=\binom{n-k-1}{k}$. 
Section~\ref{sec:many_body_scar} will provide more details on the variables and the implementation. 

\paragraph{Organization of the paper}
\noindent We first introduce relevant preliminaries in Section~\ref{sec:preliminaries}. 
In Section~\ref{sec:LAQCC_model} we formally define the class of Local Alternating Quantum-Classical Computations ($\LAQCC$). We also show that any Clifford circuit can be implemented in constant depth $\LAQCC$ (a result based on a result from measurement-based quantum computing~\cite{jozsa2006introduction}). 
This result allows us to give many useful multi-qubit gates and routines in Section~\ref{sec:gates_created_in_LAQCC}. 
Beyond that we show that constant depth $\LAQCC$ circuits are contained in $\QNC^1$ and that any $\mathsf{IQP}$ circuit has an $\LAQCC$ implementation.
We conclude this section with an analysis of a more powerful instantiation of $\LAQCC$ and show an inclusion with respect to the class $\mathsf{PostQPoly}$, which is the class of circuits of polynomial depth with one additional post-selection gate. 
In Section~\ref{sec:state_prep_in_LAQCC} we give $\LAQCC$ circuit implementations for preparing the uniform superposition over an arbitrary number of states, the $W$-state and the Dicke state up to $k = \mathO(\sqrt{n})$. We furthermore give a log-depth circuit implementation for preparing the Dicke state for any $k$. We conclude by showing a $\LAQCC$ circuit for generating many body scar states of a particular type of Hamiltonian.



\section{A model of fragmented liquidity on decentralized exchanges \label{sec:model}}
Consider the following continuous time model of trade in a single token, \textbf{T}.  The expected value of the token, $v>0$, is common knowledge.  Two risk neutral trader types consummate trade in this market: a continuum of liquidity providers ($\LP$s) and a continuum of liquidity takers ($\LT$s). Trade occurs because 
  market participants have heterogeneous private values for the asset. In particular, liquidity providers have no private value for the token, while liquidity takers value the token at $v\left(1+{\cal I}\Delta\right)$.  Here  $\Delta>0$ are the gains from trade and ${\cal I}$ is an indicator that takes on the value of $1$ if the taker buys and $-1$ if the taker sells. In what follows for expositional simplicity, as in \citet{foucault2013liquidity}, we focus on a one sided market in which liquidity takers act as buyers. 
 
 Liquidity providers differ in their endowments of the token. Each provider $i$ can supply at most $q_i$  of the token, where $q_i$ follows a truncated Pareto distribution on $\left[1,Q\right]$.  So, 
\begin{equation}\label{eq:density}
    \varphi\left(q\right)=\Big(\frac{Q}{Q-1}\Big) \frac{1}{q^2} \; \; \text{ for } q\in\left[1,Q\right].
\end{equation}

\noindent The right skew of the Pareto distribution captures the idea that there are many  low-endowment liquidity providers such as retail traders, but few high-capital $\LP$s such as sophisticated quantitative funds.   Heterogeneity in $\LP$ size is captured by $Q$, where a larger $Q$ naturally corresponds to a larger dispersion of endowments. Given the endowment distribution, collectively $\LP$s  supply at most
\begin{equation}
   S  =   \int_1^Q q \varphi\left(q\right) \diff q = \frac{Q}{Q-1} \log Q
\end{equation}
tokens. 

\begin{comment}

Figure \ref{fig:distribution} illustrates the theoretical distribution of $\LP$s endowments for $Q\in\left\{3,4\right\}$. 

% Figure environment removed 

\end{comment}


There are two types of liquidity takers: small and large. Small $\LT$s  arrive at the market at constant rate $\theta \diff t$, and each demand one unit of the token. The large $\LT$ arrival time follows a Poisson process with rate $\lambda>0$. Conditional on arrival, a large $\LT$ demands $\Theta$ units of the token, where $\Theta>S$. That is,  the large $\LT$ liquidity demand exceeds the maximum liquidity supply.  Let $D_t$ denote aggregate liquidity demand.  Then,  
\begin{equation}
    \diff D_t = \theta \diff t + \Theta \diff J_t\left(\lambda\right),
\end{equation}
where $J_t\left(\lambda\right)$ is the Poisson arrival process.

Liquidity demanders and suppliers can interact in two liquidity pools in which token trade occurs against a num\'{e}raire asset (cash).   We assume that the terms of trade are fixed, so that all trades occur at the expected value of the token, $v$. We do this by assuming prices in both pools satisfy a linear bonding curve, so there is no price impact of trade.\footnote{In practice, most decentralized exchanges use convex bonding curves, for example constant product pricing.} So, for a pool with $T$ tokens and $N$ of the num\'{e}raire good, 
\begin{equation}\label{eq:bond_curve}
    vT + N = \text{constant}.
\end{equation}
%  It is well known that price impact costs can lead to order splitting and fragmentation. Fixing the terms of trade allows us to focus on how fees affect liquidity supply. 
%  \mz{I think this assumption is actually harmless. Small LTs have no mass, and therefore no price impact. Large LTs have to consume all available liquidity anyway. The one thing that the assumption buys us is that we do not need arbitrageurs to restore the price after a large trader or a sufficiently long sequence of small traders.}
% %\cp{Is it obvious that the equilibrium effect of price impact costs lead to OS and F?  We might need a cite or two here. }

We note in passing that allowing for a bonding curve for which trades have price impact (such as a constant product function) would not materially affect our results. This is because small LTs have no mass, and therefore no price impact, whereas large LTs need to consume all available liquidity. It is well known that price impact costs can lead to order splitting and fragmentation. Fixing the terms of trade allows us to focus on how fees affect liquidity supply. 

Fees are levied on liquidity takers  as a fraction of the value of the trade and distributed pro rata to liquidity providers.  The pools have different fees.  One pool charges a low fee, and one pool charges a high fee which we denote $\ell$ and $h$ respectively.  Specifically, to purchase $\tau$ units of the token on the low fee pool, the total cost to a taker is $\tau\left(v+\ell\right)$. The $\LP$s  in the pool receive $\tau\ell$ in fees. 
 

In addition, consistent with gas costs on Ethereum,  any interaction with a liquidity pool (for example, trading or managing liquidity)  incurs a fixed execution cost $\Gamma>0$. 
It is important to note that 
gas fees on decentralized exchanges differ from trading fees on traditional exchanges.  First, they are not set by individual exchanges to compete with each other, but are a common transaction cost \emph{across} trading venues. Second, they are levied on a per-order rather than per-share basis: this implies economies of scale for larger orders. Third, there is significant time variation in gas fees which will allow us to identify the impact of transaction costs on liquidity pool market shares.\footnote{Empirical properties of gas fees are exhibited in \citet{LeharParlour2021}, while \citet{CapponiJia2021}, consider traders who affect the gas price.} 

To ensure that both small and large liquidity traders participate in the market, we assume that the gains from trade are larger than the aggregate transaction cost, including the pool fee and the gas price. 
To rule out trivial cases, we further assume that $Q\ell-\Gamma>0$. The condition ensures that there are at least some liquidity providers who can earn positive expected profit on the low fee pool.

\begin{ass} The gains from trade are sufficiently large so that all liquidity takers participate in the market, and the fixed costs are sufficiently low so that both pools can attract liquidity. 

\begin{enumerate} 
\item [i.] Gains from trade are sufficiently large $\Delta > h + \frac{\Gamma}{v}$.
\item [ii.] The low fee pool can attract liquidity $Q\ell-\Gamma>0$
\end{enumerate}
\end{ass}


We follow \citet{foucault2013liquidity} and partition the continuous timeline into \emph{liquidity cycles}. A liquidity cycle starts with an empty pool (zero liquidity offered) which triggers $\LP$ token deposits and ends when incoming trades deplete the liquidity supply. The first liquidity cycle starts at $t=0$, and  the sequence of events within a cycle is as follows:  each liquidity provider deposits their token into the high fee pool or the low fee pool and pays the gas cost $\Gamma$, or withdraw from the market. The $\LP$s do not interact with the pool again until the their liquidity is consumed at some random time $\tilde{\tau}$ and the next cycle begins. Figure \ref{fig:timing} illustrates the model timing.


% Figure environment removed

\subsection{Equilibrium}\label{sec:liqprov}

First, consider the liquidity traders' decisions. Faced with pool sizes of $\mathcal{L}_\ell$ and $\mathcal{L}_h$ in the low and high pool respectively, they choose the pool which minimizes their trading costs.  Conditional on trading, the small liquidity takers choose the $\ell$ fee pool.  Thus, small traders arrive at the $\ell$ fee pool at rate $\theta$, and each trade one unit.  By assumption the large liquidity taker wants to trade more than the posted liquidity and exhausts both pools.  Thus, liquidity is consumed on the low fee pool at rate $\frac{\theta}{\mathcal{L}_\ell} \diff t$, while liquidity is only consumed on the high fee pool if a large trader arrives. Once each pool is empty, liquidity providers refill it and restart the cycle. Let $d_k, k=\ell,h$ denote the duration of a liquidity cycle on the low and high pool respectively. Then, the expected duration of a cycle on the low fee pool is 
\begin{align}
    d_\ell &=e^{-\lambda \frac{\LL}{\theta}} \frac{\LL}{\theta}+\int_{0}^{\frac{\LL}{\theta}} t\lambda e^{-\lambda t}\diff t =\frac{1}{\lambda}-\frac{1}{\lambda}e^{-\frac{\LL}{\theta}\lambda},
    \end{align}
\noindent while the expected duration on a high fee pool is $d_h=\frac{1}{\lambda}.$  Thus,


\begin{lem}\label{lem:duration}
 \setlength{\parskip}{0ex}
The expected duration of a liquidity cycle is shorter in the low fee pool than the high feel pool. Or, $d_\ell<d_h$.
\end{lem}

Lemma \ref{lem:duration} is intuitive.  A liquidity cycle on the high fee pool only ends with the arrival of a large trader.   Conversely, a liquidity cycle on the low fee pool ends either because a large liquidity taker arrived or the cumulative small liquidity trader orders exhaust the pool. 

Notice that the low fee pool provides liquidity for a large share of the order flow, as both small and large $\LT$s trade there. However, the liquidity providers on that pool earn a low fee per traded unit. Additionally, as the expected cycle duration is shorter, they need to manage their liquidity more often, which leads to larger gas costs per unit of time. Liquidity providers face a choice between the low and high fee pool or not participating in the market. Thus, an \LP of size $q_i$ chooses between:

\begin{equation}\label{eq:optimal_pool}
  \max \Big[\frac{q_i \ell - \Gamma}{d_\ell} , \; \frac{q_i h -\Gamma}{d_h}, \; 0\Big].
\end{equation}

\noindent First, consider the choice between pools.  Rearranging Equation \ref{eq:optimal_pool}, liquidity provider $i$ chooses the low-fee pool  if and only if
\begin{equation}\label{eq:cost_comparison}
    \left(d_h \ell - d_\ell h\right) q_i> \Gamma \left(d_h-d_\ell\right).
\end{equation}

Equation \ref{eq:cost_comparison} highlights the tradeoff between the expected fee revenue per unit time and the fixed cost of accessing the market. 
If $\frac{h}{d_h}>\frac{\ell}{d_\ell}$,  then  the expected liquidity fee per unit of time on the high fee pool is larger than that of the low fee pool, and  the left hand side of Equation \ref{eq:cost_comparison} is negative.  From Lemma \ref{lem:duration}, the expected duration is higher on the high fee pool and the right hand side is always positive.  In this case,  all liquidity providers choose the high fee pool.
The more natural case to consider is if $\frac{h}{d_h}<\frac{\ell}{d_\ell}$ so that 
 liquidity providers face a trade-off between a higher liquidity fee per unit of time on the low fee pool against lower gas costs on the high fee pool. Clearly, the larger a liquidity providers' endowment, the more important is the fee revenue.  We have

 \begin{lem}\label{lem:sort} For any liquidity pools, $\{\mathcal{L}_\ell, \mathcal{L}_h\}$, for which $\frac{h}{d_h}<\frac{\ell}{d_\ell}$, if a liquidity provider of size $q$ prefers the low fee pool, then any liquidity provider with a larger endowment,  $\widetilde{q}>q$, also prefers the low fee pool.
 \end{lem}

 Now consider the choice of participating in the market.  An agent  only provides liquidity if she is able to break even on the high-fee pool -- that is, if her endowment $q_i$ is large enough. The participation constraint follows from equation \eqref{eq:optimal_pool}:
\begin{equation}\label{eq:pc}
    q_i h - \Gamma \geq 0.
\end{equation}

Define $\underline{q}=\frac{\Gamma}{h}$.  Recall, that the lower bound of the Pareto distribution is 1.  Thus, if $\frac{\Gamma}{h}>1$,  all $\LP$s with $q_i>\underline{q}$ enter the market, and the marginal entrant earns zero expected profit. Conversely, if $\frac{\Gamma}{h}\leq 1$,  liquidity provision is not competitive; all $\LP$s enter the market and earn strictly positive profits.
 
 

Following Lemma \ref{lem:sort}, let $\qmg$  be the threshold endowment such that all $\LP$s with $q_i>\qmg$ post liquidity on the low-fee pool and all $\LP$s with $q_i\leq \qmg$ choose the high-fee pool. 
This allows us to characterize pool sizes
$\LL$ and $\LH$ as a function of the endowment for the marginal $\LP$:
\begin{align}\label{eq:liquidity_levels}
    \LL&=\int_{\qmg}^Q q_i \varphi\left(q_i\right) \diff i = \frac{Q}{Q-1}\left(\log Q - \log \qmg \right)  \text{ and }\nonumber \\
    \LH&=\int_{\underline{q}}^{\qmg}  q_i \varphi\left(q_i\right) \diff i = \frac{Q}{Q-1}\left(\log \qmg - \log \underline{q}\right)
\end{align}
From equations \eqref{eq:cost_comparison} through \eqref{eq:liquidity_levels} it follows that the expected profit difference between the low- and high-fee pools can be written as an increasing function of the marginal $\LP$'s endowment, that is
\begin{align}\label{eq:pi_diff}
    \pi_\ell-\pi_h &=\frac{1}{\lambda d_h d_\ell}\left[\exp\left(-\frac{\lambda}{\theta} \LL\right) \underbrace{\left(q_i h - \Gamma\right)}_{>0} - q_i\left(h-\ell\right)\right] \nonumber \\
        &=\frac{1}{\lambda d_h d_\ell}\left[\qmg^{\frac{\lambda}{\theta}\frac{Q}{Q-1}} \times Q^{-\frac{\lambda}{\theta}\frac{Q}{Q-1}} \underbrace{\left(q_i h - \Gamma\right)}_{>0}- q_i\left(h-\ell\right)\right]. 
\end{align}
 Proposition \ref{prop:equilibria} characterizes the equilibrium liquidity provision.

\begin{prop}\label{prop:equilibria}
%\begin{leftbar} \setlength{\parskip}{0ex}
%\emph{(Fragmentation)} 
\begin{enumerate}
    \item [i.]
If $\frac{h-l}{h} Q^{\frac{\lambda}{\theta}\frac{Q}{Q-1}}<1$ and $\frac{\Gamma}{h}<1-\frac{h-l}{h} Q^{\frac{\lambda}{\theta}\frac{Q}{Q-1}}$, then  all $\LP$s deposit liquidity on the low fee pool. 
\item[ii.]If $\Gamma>Q\ell$, then all $\LP$s deposit liquidity on the high fee pool. 
\item [iii.] Otherwise, there exists a unique fragmented equilibrium characterized by marginal trader $\qmg^\star$ which solves
\begin{equation}\label{eq:mg_eq}
    \qmg^\star = \Gamma \frac{\qmg^{\frac{\lambda}{\theta}\frac{Q}{Q-1}} \times Q^{-\frac{\lambda}{\theta}\frac{Q}{Q-1}}}{h\left[\qmg^{\frac{\lambda}{\theta}\frac{Q}{Q-1}} \times Q^{-\frac{\lambda}{\theta}\frac{Q}{Q-1}}\right]-\left(h-\ell\right)} \in \left[\underline{q},Q\right]
\end{equation}
such that all $\LP$s with $q_i\leq \qmg^\star$ deposit liquidity in the high fee pool and all $\LP$s with $q_i>\qmg^\star$ choose the low fee pool.
\end{enumerate}
%\end{leftbar}    
\end{prop}

Figure \ref{fig:region_equilibrium} illustrates the equilibrium regions in Proposition \ref{prop:equilibria}. If the gas price is low and $\LP$s are homogeneous (low $\Gamma$ and $Q$), then all liquidity providers choose the low fee pool because  managing liquidity is relatively cheap. If more high-endowment $\LP$s enter the market (i.e., there is an increase in $Q$) the more liquidity is posted the low fee pool.  Keeping the $\LT$ arrival rate fixed, a larger pool depletes more slowly and thus the liquidity cycle on the pool becomes longer. As a consequence, the liquidity fee per unit of time on pool $L$ drops and smaller $\LP$s switch to the high-fee pool. As a result, liquidity becomes fragmented across the two pools. 

% Figure environment removed

An equilibrium in which all liquidity consolidates on the high fee pool is only sustainable for very high gas costs $\Gamma>Q\ell$. In this case,  none of the $\LP$s breaks even on pool $L$. For intermediate values of gas price, both pools co-exist  with positive market share.

Proposition \ref{cor:comp_stat_ms} establishes comparative statics for the two pools' liquidity market shares. From equation \eqref{eq:liquidity_levels}, we can compute the liquidity market share of the low-fee pool at the beginning of each cycle as
\begin{equation}
w_\ell=\frac{\LL}{\LL+\LH}=\frac{\log Q - \log \qmg^\star}{\log Q - \log \underline{q}}.
\end{equation}

\begin{prop}\label{cor:comp_stat_ms}
%\begin{leftbar} \setlength{\parskip}{0ex}
 In equilibrium, the market share of the low fee pool $w_\ell$
\begin{itemize}
    \item[i.] decreases in the gas cost ($\Gamma$), the arrival rate of large trades ($\lambda$), and the fee on pool $H$ (h).
    \item[ii.] increases in the fee on pool $L$ ($\ell$) and the arrival rate of small trades ($\theta$)
\end{itemize}
%\end{leftbar}    
\end{prop}

The results in Proposition \ref{cor:comp_stat_ms} are intuitive. The market share of the low fee pool increases if the fee gap $h-\ell$ is narrower, since this reduces $\LP$ incentives to switch to the high-fee pool. If the small $\LT$ arrival rate is large, then liquidity cycles in the low-fee pool are shorter, increasing the revenue per unit of time and consequently the market share of pool $L$. Conversely, if large trades arrive more often (high $\lambda$), then the high fee pool attracts a higher share of incoming order flow and becomes more appealing for liquidity providers.

Figure \ref{fig:liqshares} shows that the market share of the low fee pool (weakly) decreases in the gas cost $\Gamma$. A larger gas price increases the costs of active liquidity management, everything else equal, and incentivizes smaller $\LP$s to switch from the low fee pool to the high fee pool, since the latter has a lower turnover. For $\Gamma\leq h$, any increase in gas costs leads to a \emph{redistribution} of liquidity from one pool to another; the aggregate liquidity across both pools is constant since all $\LP$s participate in the market.

% Figure environment removed

If gas prices increase beyond a threshold ($\Gamma>h$), then the aggregate liquidity falls since $\LP$s with $q_i<\frac{\Gamma}{h}$ are shut out of the market. Both the low and high fee pool experience a decrease in liquidity deposits. However, the liquidity drop is sharper for the low fee pool which further depresses its market share.

\subsection{Pool fragmentation and market quality}


In our model, liquidity takers incur two main costs beyond gas prices: pool fees, which represent the cost of taking liquidity, as well as foregone gains from trade if liquidity providers do not fully participate in the market. To integrate these costs into a single measure of market quality, we use an \emph{implementation shortfall} metric, following \citet{Perold_1988}. If the asset is traded on a sequence of pools, where $f_k$ and $\mathcal{L}_k$ represent the fees and liquidity deposits on pool $k$, respectively, the implementation shortfall (IS) for a large $\LT$ is defined as
\begin{equation}\label{eq:IS_general}
    \text{IS}\left(\left\{f_k\right\}_k\right)=\underbrace{\sum_k f_k \mathcal{L}_k}_\text{trading fees} + \underbrace{\Delta \left(\Theta - \sum_k \mathcal{L}_k\right)}_\text{unrealized gains},
\end{equation}
where $\Theta - \sum_k \mathcal{L}_k$ is the difference between the $\LT$ trading demand and the cumulative liquidity available across all pools, and  $\Delta$ is  the per-unit gains from trade.

Suppose an asset is traded on a single pool that imposes a liquidity fee $f$. From equation \eqref{eq:IS_general} it follows that the implementation shortfall on this pool is:
\begin{align}
    \text{IS}\left(\left\{f\right\}\right)&= f \int_{\frac{\Gamma}{f}}^{Q}  q_i \varphi\left(q_i\right) \diff i + \Delta \left(\Theta - \int_{\frac{\Gamma}{f}}^{Q}  q_i \varphi\left(q_i\right) \diff i\right).
\end{align}
Here, the magnitude of the liquidity fee drives the trade-off between the participation of liquidity providers (\(\LP\)) and trading costs. A lower fee \( f \) results in fewer \(\LP\)s offering liquidity, leading to increased unrealized gains. In contrast, a higher fee increases trading costs, potentially outweighing the benefits of increased \(\LP\) participation. 

The trade-off is illustrated in the left panel of Figure \ref{fig:is}. The optimal fee \( f^\star \) that minimizes the single-pool implementation shortfall is equal to \( f^\star = g W^{-1}\left({\rm e}\frac{g Q}{\Gamma }\right) \), where \( W(\cdot) \) represents the Lambert function.

\begin{prop}\label{prop:optimality}
For any single-pool fee $f\geq 0$, there exists a set of fees \( \{h, \ell\} \) for a two-pool fragmented market, where \( h = f \) and \( h > \ell \), that guarantees an equal or reduced implementation shortfall in a fragmented market compared to the single-pool market. \\ Furthermore, when $f>\frac{\Gamma}{Q}$, the fee structure \( \{h, \ell\} \) can be chosen to ensure a strictly lower implementation shortfall in the fragmented market.
\end{prop}


Proposition \ref{prop:optimality} suggests that fragmentation with multiple fee levels improves market quality. Specifically, it is always possible to devise a fee structure in a fragmented market that yields a (weakly) lower implementation shortfall than a single-fee market. The logic is as follows: First, the highest fee in the fragmented market is set equal to the single pool fee, ensuring that the marginal $\LP$ participating the market is the same across both scenarios (i.e., the $\LP$ with endowment $\underline{q}=\frac{\Gamma}{f}$). This condition guarantees the same level of realized gains from trade in fragmented and non-fragmented markets. Second, a lower fee is then chosen for another pool to attract liquidity providers with higher token endowments, resulting in lower trading costs. This combination of reduced trading costs and unchanged gains from trade leads to a lower implementation shortfall in a fragmented market. 

% Figure environment removed

\begin{cor}\label{cor:optimality}
There exists a fee menu in a two-pool fragmented market structure that achieves an equal or lower implementation shortfall than on any single-fee pool configuration. Further, the fee menu can be chosen to achieve strictly lower implementation shortfall in fragmented markets if $f^\star=g W^{-1}\left({\rm e}\frac{g Q}{\Gamma }\right)>\frac{\Gamma}{Q}$.
\end{cor}

Corollary \ref{cor:optimality} emerges as a particular case of Proposition \ref{prop:optimality}, with the assumption that the single-pool operates at its optimal fee level. In essence, if a fragmented fee structure can be designed to achieve a lower implementation shortfall compared to an arbitrary single-pool fee, then a fee structure that dominates the optimally set single-pool fee achieves a lower implementation shortfall than any single-fee pool. 

The right panel of Figure \ref{fig:is} illustrates the scenario where the lower fee in a fragmented market is set to half the optimal fee of a single pool, denoted as \( \ell = \frac{1}{2}f^\star \). When gas prices are sufficiently low, $\LP$s with large endowments choose to provide liquidity on the low-fee pool, leading to reduced trading costs and, consequently, a lower implementation shortfall. As gas prices increase, the implementation shortfall rises in both single-pool and fragmented markets, primarily because more $\LP$s are priced out which results in higher unrealized gains. Additionally, the shortfall increases more rapidly with rising gas prices in the fragmented market as $\LP$s switch over from the low- to the high-fee pool --- that is, the result in Proposition \ref{cor:comp_stat_ms}. At very high gas prices, all $\LP$s converge in the high-fee pool, effectively collapsing the fragmented market into a single pool with the optimal fee level.








\subsection{Model implications and empirical predictions}

\begin{pred}\label{pred:comp_stat_Gamma}
The liquidity market share of the low-fee pool  decreases in the gas fee $\Gamma$.
\end{pred}

Prediction \ref{pred:comp_stat_Gamma} follows directly from Proposition \ref{cor:comp_stat_ms} and Figure \ref{fig:liqshares}. A higher gas price increases the fixed cost of active liquidity management, particularly so for smaller liquidity providers. In response, $\LP$s with lower endowments migrate to the high-fee pool  where they trade less often. 

\begin{pred}\label{pred:clienteles}
$\LP$s on the low-fee pool make larger liquidity deposits than $\LP$s on the high-fee pool.
\end{pred}

Prediction \ref{pred:clienteles} follows from the equilibrium discussion in Section \ref{sec:liqprov}. Liquidity providers with large token endowments ($q_i>\qmg$) deposit them in the low-fee pool since they are better positioned to actively manage liquidity due to economies of scale. $\LP$s with lower endowments ($q_i\leq\qmg$) either stay out of the market or choose pool $H$ which allows them to offer liquidity in a more passive manner.

% Figure environment removed

Figure \ref{fig:theory_liqsupply} illustrates this prediction through a Monte Carlo simulation. We plot the equilibrium liquidity supply decisions of 100,000 $\LP$s with endowments drawn from density \eqref{eq:density} and $Q=3$. The top panel highlights three groups of liquidity providers: low-endowment $\LP$s (in green) that are being rationed out of the market due to high gas cost, medium-endowment $\LP$s (blue) that deposit liquidity on pool $H$, and high-endowment $\LP$s (orange) that choose the low-fee pool $L$. 



\begin{pred}\label{pred:trade_size_volume}
The average trade size is higher on pool $H$ than on pool $L$. At the same time, trading volume is higher on pool $L$ than on pool $H$.
\end{pred}

Next, Prediction \ref{pred:trade_size_volume} deals with differences between incoming trades on the two liquidity pools. For a wide range of model parameters, incoming order flow on the low fee pool consists of a large number of small trades, and occasional large trades. In contrast, there are few trades on the high fee pool, but they are all relatively large. The model can therefore reconcile two apparently conflicting patterns: one liquidity pool captures most of the trading volume, while the largest trades are executed on the competitor. Figure \ref{fig:theory_trade} illustrates the prediction through a Monte Carlo simulation of the model.


% Figure environment removed


\begin{pred}\label{pred:clienteles_cs}
The average liquidity deposit on both the low- and- high fee pool increases with gas costs.
\end{pred}

An increase in the gas cost $\Gamma$ has two effects: first, the $\LP$s with the lowest endowments on pool $L$ switch to pool $H$. As a result, the average deposit on pool $L$ increases. Second, the $\LP$s with low endowments on pool $H$ may leave the market. Both channels translate to a higher average deposit on pool $H$, which experiences an inflow (outflow) of relatively high (low) endowment $LP$ following an increase in gas costs.



The bottom left panel highlights the clientele effect: that is, the average deposit is higher on pool $L$. Due to the skew of the Pareto distribution, however, there are more $\LP$ accounts active on pool $H$ than on pool $L$ for a wide range of parameter values.



\begin{pred}\label{pred:updates}
$\LP$s update liquidity more frequently on the low-fee than on the high-fee pool.
\end{pred}

Prediction \ref{pred:updates} is a consequence of Lemma \ref{lem:duration}. Liquidity cycles on pool $L$ are shorter than on pool $H$, since $\LP$s on the low-fee pool trade against both small and large orders, rather than only against large orders on the high-fee pool. Consequently, we expect $\LP$s on pool $L$ to actively manage their liquidity positions.


\begin{pred}\label{pred:updates_gas}
A larger gas price leads to more frequent liquidity updates on the low-fee pool.
\end{pred}
An increase in gas price leads to some $\LP$s switching from the low- to the high-fee pool. As a result, liquidity supply on the low-fee pool drops, leading to a shorter cycle as incoming order flow depletes the pool at a faster pace.



\section{Data and descriptive statistics \label{sec:data}}
\lstMakeShortInline[columns=fixed]@
% Figure environment removed
\lstDeleteShortInline@

In this section, we describe how we collect examples for learning repair strategies without any version-controlled data. Specifically, we first detect \safeprogs and corresponding witnesses using \sawitnessfull (witnesses are sanitizers and guards that protect from vulnerabilities)  in Section~\ref{subsec:sa-witness}. Using these witness annotations, we generate unsafe programs and \textit{edits} from the \safeprog using a \textbf{witness-removal} step (Section ~\ref{subsec:witness-removal}). In the following, we define terminology for the \astree  data-structure we operate on. 


\astree refers to the abstract syntax tree representation of programs, augmented with data flow edges and annotations for sources, sinks, sanitizers, guards, witnesses etc. 
An \astree is a five-tuple 
$\langle \mathcal{N},\mathcal{V},\mathcal{T},\mathcal{E}, \mathcal{A} \rangle$, where:
\begin{enumerate}
\item
$\mathcal{N}=\{\mathit{id}_0,\ldots\mathit{id}_n\}$  is a set of nodes, where  $\mathit{id_i}\in\mathbb{N}$ for 
$ 0 \leq i \leq n$.
\item
$\mathcal{V}$ is a map from nodes to program snippets
represented as strings. For a node $n$, we have that $\mathcal{V}(n)$ is a string representing the code snippet associated with $n$
\item
$\mathcal{T}$ is a map from nodes to their types defined by 
 \sa~\cite{codeqlast}. For example, \callexpr is the type of a node representing a function call, \indexexpr is the type of a node representing an array index, and \blockstmt is the type of a node representing a basic block of statements.
\item
$\mathcal{E}$ is a set of directed edges.
Each edge is of the form $(n_1,n_2,\edgetype,z)$, where
$n_1$ is a source node, $n_2$ is a target node, 
$\edgetype \in \{\T{SynParent}, \T{SynChild}, \T{SemParent},
\T{SemChild} \}$ denotes the relationship from 
$n_1$ to $n_2$, as one of syntactic parent, syntactic child, semantic parent or semantic child,
and $z\in\mathbb{Z}$ is the index of $n_2$ among $n_1's$ children if this edge is a child edge, and $-1$ if the edge is a parent edge. 
\item
$\mathcal{A}$ is a set of annotations associated with each node. The annotations are from the set $\{\T{source},
\T{sink},\T{sanitizer},\T{guard}$,\T{witness}\}. We also refer to annotations using predicates or relations. For instance, for a node $n$, if an annotation  $\T{source}$ is present, we say that
the predicate $\T{source}(n)$ is true.
\end{enumerate}

%\setlength{\grammarindent}{5em} % increase separation between LHS/RHS

% Figure environment removed



A {\em traversal} or a {\em path} in an \astree is a sequence of edges $e_0,\ldots,e_{i-1},e_i,\ldots ,e_k$ such that the target node of $e_{i-1}$ is also the source node of $e_i$, for all $i\in\{1,\ldots,k\}$. That is, $e_{i-1}$ is of the form $(\_,n,\_,\_)$ and $e_i$ is of the form $(n,\_,\_,\_,\_)$. The source node of $e_0$ is the source of this path and the target node of $e_k$ is the target of the path.


\lstMakeShortInline[columns=fixed]@
%Note that these additional edges can capture long-range dependencies in programs. E.g. edge 4 in Figure ~\ref{fig:unsafememberex} links two nodes across the function boundaries. 
Figure~\ref{fig:example1-pdg} depicts a partial \pdg corresponding to the unsafe program in Figure~\ref{fig:unsafememberex}. Each oval corresponds to an \astree-node containing a type $\tau$ and an associated value. The dark edges denote the syntactic child edges. For example, the oval with value @foo(data)@ is an \astree-node with type \callexpr and has two children -- @foo@ and @data@, both with the type \varexpr. 
%Similarly, the \blockstmt node on the top refers to the function body between Line~\ref{lst:line:handlers-run} and Line~\ref{lst:line:handlers-run-end} in Figure ~\ref{fig:unsafememberex}. As the body of a function block can contain a variable number of children, we link to @handlers[callerId](data);@ as the k-th child of the \blockstmt. 
The semantic child edges are at the bottom in cyan. These edges correspond to the ones depicted in cyan in Figure ~\ref{fig:unsafememberex}. 
\lstDeleteShortInline@

%TODO:FIX THIS

%With this simplification, 
If $\prog$ is an \pdg then
we use  $\prog.\mathtt{source}$ to denote the source node, $\prog.\mathtt{sink}$ to denote the sink node, and $\prog.\mathtt{witness}$ to denote the witness node.
If the program has several sources, sinks and sanitizers then we generate a separate \pdg for each $(\mathtt{source},\mathtt{witness},\mathtt{sink})$ triple.
For a node $n$, its syntactic parent is $n.\mathtt{parent}$, syntactic children are $n.\mathtt{children}$, semantic parent is $n.\mathtt{semparent}$, and semantic children are $n.\mathtt{semchildren}$.

%\input{ql.tex}

\subsection{Static Analysis Witnessing}
\label{subsec:sa-witness}

\newcommand{\DMethodjudge}[1]{\texttt{#1(}\checknextarga}

% Figure environment removed

%\naman{TODO - sell this more as technique to work with any \sa tool ; our master query is a general framework implemented in \codeql that can work for any vulnerability -- easily extendable to other languages }
In this section, we show how to repurpose \sa tools to generate witnesses.
\sa tools perform dataflow analysis to check for rule-violations in programs. They use pattern matching to identify known sources, sinks, sanitizers, and guards. For commercial tools, these patterns are implemented (and continuously updated) manually by developers and encode this domain knowledge. Next, 
%these patterns are used to detect sources, sinks, sanitizers, and guards in programs and
\sa checks if there exists a flow between a source and a sink that does not cross a sanitizer or guard. We capture this formally in Figure~\ref{fig:judgements} (top two rules), and explain the notation used in it below.

\sa tools encode domain knowledge about the vulnerability by annotating nodes as \T{Source}, \T{Sink}, \T{Sanitizer}, and \T{Guard}. %These relations operate on the set of dataflow nodes in the programs.
So \DMethod{Source}{\I{n}}\ is true iff the node \I{n} is a \textit{source} node for a vulnerability. Next, \sa tools perform dataflow analysis by defining the relation \DMethod{SemChild}{$n_1$}{$n_2$}\ which is true iff there is a \taintpropedge between $n_1$ and $n_2$. Then the \DMethod{Vulnerability}{$n_1$}{$n_2$}\ relation can be defined as:
\begin{enumerate}
    \item $n_1$ and $n_2$ are source and sink nodes (\DMethod{Source}{$n_1$}\ and \DMethod{Sink}{$n_2$}\ are true)
    \item There exists a \textit{path} between $n_1$ and $n_2$ which is free of sanitizers or guards (\DMethod{SanGuardFree*}{$n_1$}{$n_2$}\ is true). A path is free of sanitizers and guards iff every \textit{edge} in the \textit{path} is free of sanitizers and guards. An edge between $n_1$ and $n_2$ is considered free of sanitizers and guards (\DMethod{SanGuardFree}{$n_1$}{$n_2$}\ is true) iff $(n_1, n_2, \_, \T{SemChild}) \in \mathcal{E}$ and neither of $n_1$ or $n_2$ is a sanitizer or a guard
\end{enumerate}

Here, we make the following observation - \emph{this domain knowledge present in these annotations and relations is helpful beyond just detecting vulnerabilities}. For instance, simply using the sanitizer relation allows us to query the different kinds of sanitizers domain experts have specified. We use this observation to discover \emph{\safeprogs} i.e., programs having a source, sink, and a sanitizer or guard that \textit{blocks} the \taintprop or, in simpler terms, make the program safe. In addition, we also detect the corresponding sanitizers or guards in the programs and refer to them as \textit{witnesses} because they serve as the evidence of making the program safe. We call this procedure \sawitnessfull (abbreviated as \sawitness). 
We define this as the \T{Witness} relation in Figure~\ref{fig:judgements} (bottom two rules). Specifically, \DMethod{Witness}{$n_1$}{$n_3$}{$n_2$}\ is defined as:
\begin{enumerate}
    \item $n_1$ and $n_2$ are source and sink nodes (\DMethod{Source}{$n_1$}\ and \DMethod{Sink}{$n_2$}\ are true)
    \item There exists a node $n_3$ such that it satisfies \DMethod{SanGuardInMid}{$n_1$}{$n_3$}{$n_2$}. \DMethod{SanGuardInMid}{$n_1$}{$n_3$}{$n_2$}\ is true iff there exists a \T{SemChild}
    %\naga{notation for flow inconsistent with (2) above} 
    path between $n_1$, $n_3$, between $n_3$ and $n_2$, with the additional constraint of $n_3$ being a sanitizer or guard. 
\end{enumerate}

The difference between the \T{Vulnerability} relation (which \sa populates) and \T{Witness} relations (which we want to find) is highlighted in {\color{red} red} and {\color{ForestGreen} green}. Notice that while defining the \T{Witness} relation, we simply use the existing relations that define the \T{Vulnerability} relation. Thus, we argue that \sawitness can be implemented on top of \sa by using the intermediate relations that \sa is computing.
%for every pair of source and sink, they track taint through a taint-flow analysis. If there is a flow from a source to a sink that does not go through a sanitizer or guard, then the source-sink pair is reported as vulnerable.

%We make the following observation - \emph{the patterns defined by experts encodes domain knowledge which can be used for use cases beyond just detecting vulnerabilities}. For instance, we can use the sanitizer patterns to search for all sanitizers in source-code. In this work, we use this idea to detect \safeprogs, which we define as programs having a source, sink, and a sanitizer or guard that blocks the \unsure{flow} or in other words, makes the program safe.  \naman{highlighted part of Figure somethings shows the difference between semantics of witnessing vs traditional semantics}

%We realize the following -- the set of patterns of sources, sinks, and sanitizers are useful beyond detecting vulnerabilities. We override the existing static analysis query that detects unsafe programs and use these encoded sanitizers for detecting sanitizers and guards in programs. Specifically, in the existing query that detects unsafe programs, we modify the taint-propagation steps to propagate taints through sanitizers and guards and then use static analysis to then find these dataflows containing sanitizers and guards. Thus, we can directly find the safe programs containing these \textit{witnesses} of safety. 
%Once such a dataset is collected, we use these witnesses to convert safe  to unsafe  and thus obtain paired examples for learning repair strategies (Section~\ref{subsec:witness-removal}). 

\lstMakeShortInline[columns=fixed]@
%We instantiate our \sawitness technique using \codeql~\cite{a}. It is an open-source \sa tool that allows implementing custom static analysis as queries in a high-level object-oriented extension of datalog. These queries usually contain a \Verb|select from where| statement that allows querying the program database. \codeql maintains these patterns of sources, sinks, sanitizers, and guards using \Verb"Configuration" classes. Consider an example of a simplified \Verb"Configuration" for \xss vulnerability in Figure~\ref{fig:configuration}. It defines a set of predicates @isSource@, @isSink@, @isSanitizer@, and @isGuard@. These predicates are written manually by \codeql authors and improved through rich community support\footnote{\url{https://github.com/github/codeql}}. With this configuration, vulnerabilities are reported by selecting source-sink pairs such that the @cfg.hasFlow@ predicate is true for the source, and the sink. This predicate is internally defined by \codeql and uses the patterns defined in the configuration to check for the presence of vulnerability-causing dataflows. %\spsays{Showing corresponding programs will be useful}

%Now, we demonstrate the static-analysis-witnessing approach for collecting examples of \safeprog and witnesses in Figure~\ref{fig:safe-configuration}. Specifically, we inherit from the existing configuration, using the same @isSource@ and @isSink@ predicates while overriding the @isSanitizer@ and @isGuard@ predicates to @none()@. This ensures that all the source and sink pairs are detected independent of the presence of sanitizers/guards between them. Finally, to detect our witnesses, we define the @isWitness@ predicate which uses the @isSanitizer@ and @isGuard@ predicates from the original configuration. Specifically, witnesses are defined as sanitizers/guards that lie between a source-sink pair. Finally, to report \safeprog and witnesses, the @cfg.hasFlow@ predicate is used to select all valid source-sink pairs and the corresponding witnesses are detected via the @isWitness@ predicate. Note that Figure~\ref{fig:configuration-vs-safe-configuration} depicts the key idea behind our approach in a simplified view. In practice, additional measures need to block the taint propagation internally and we share the actual \codeql queries used as part of the Appendix~\ref{app:codeql-queries}.


\subsection{Witness Removal}
\label{subsec:witness-removal}

We obtain \safeprogs and witnesses by applying \sawitness to a snapshot of a codebase. Recall that the witnesses block the flow between a source and a sink and thus help make programs  \textit{safe}. Hence, removing these witnesses will make the programs unsafe. Recall also that the witnesses are either sanitizing functions of the form @sanitize(taintedVar)@ or guards of the form @if checkSafe(taintedVar) {executeSink(taintedVar)}@. %Usually, they are used only for ensuring the safety of programs and are not critical to the functionality of programs. Therefore, 
We implement witness-removal perturbations  that precisely remove the guard-checks and sanitizer-functions. Note that our goal here is to generate unsafe programs and corresponding edits that enable learning repair strategies that insert such witnesses. So, while we generate the unsafe programs by perturbation, they should look structurally similar to natural unsafe programs written by the developers, otherwise the repair strategies learned on this artificially generated data through perturbations would not generalize to code in the wild. 
%At the same time, minor syntactic-semantic issues in parts of unsafe programs not directly relevant to the vulnerability or repair do not impact learning.
\lstDeleteShortInline@

% Figure environment removed

\lstMakeShortInline[columns=fixed]@

\input{witnessremoval.tex}

We use \rmSan and \rmGuard functions to programmatically remove the witnesses. A high-level sketch of these functions is illustrated in Figure~\ref{fig:remove-functions}. The functions use the structure of the corresponding \astree (node types $\tau$) to decide how to remove witnesses. Consider the \rmGuard function. It first computes the parent (\witnesspar) and grand-parent (\witnessparpar) of the witness guard condition. Then if the type of \witnesspar is \ifstmt (i.e., program is of the form @if (witness) body@ then we modify the \astree edge from \witnessparpar and \witnesspar to instead point to the body of the \ifstmt (index 1 child is body of \ifstmt). Similarly, if the type of \witnesspar is \binaryexpr with operator @&&@ (i.e. of the form @if (otherCond && guard)@ or @if (guard && otherCond)@) then we again modify the edge from \witnessparpar and \witnesspar to instead point to the non-guard child of \binaryexpr (@otherCond@ in the example). Note that since \binaryexpr has 3 children, the index of non-guard child is index of guard-child subtracted from 2. 
Figure~\ref{fig:witness-removal} depicts this removal on the \astree level, where the syntactic edges in red are removed and the syntactic edges in green are inserted.
In the end, the functions returns a tuple of the \pdg of the unsafe program ($\prog_{unsafe}$), \pdg of the safe program ($\prog_{safe}$)
and an edit object (\edit) which stores


\begin{enumerate}
    \item \astree for the removed witness (referred to as \editprog)
    \item location in the \pdg where the witness is removed (referred to as editloc
    %\naga{shouldn't it be editloc to be consistent with (1)?} 
    or \editloc)
    %\item an enum (\insertsc or \replace) depending on whether \concedit is inserted or replaced 
\end{enumerate}

Since $\prog_{unsafe}$ and edit-object can generate the safe program, we only propagate the unsafe programs and edits as the output of this step. Applying \rmGuard function to the safe program in Figure~\ref{fig:safememberex} removes the \ifstmt on Line~\ref{lst:line:fix-start} while preserving the @handlers[callerId](data);@ statement and in fact produces the unsafe program in Figure~\ref{fig:unsafememberex}. Additionally, it  returns the removed witness guard  @if handlers.hasOwnProperty(data.id){ ... }@ as the \editprog and \blockstmt (blue oval in Figure~\ref{fig:example1-pdg}) as the edit location \edit.editloc. Figure~\ref{fig:example1-editprog} shows the \astree for the \editprog containing the \ifstmt. 
The dashed line and dark circle correspond to the \textit{removed} \astree edge between the \blockstmt and the \expr @handlers[callerId](data)@. 

Note that Figure~\ref{fig:remove-functions} provides a high-level sketch of witness-removal and elides over implementation details that are required to make it work for real \js programs. We discuss these issues in the implementation section (Section~\ref{subsec:impl:witness-removal}).% and include the full implementation as part of supplementing source code\naga{we should make sure we are doing these, else remove this sentence}. 
%. In practice, we need implement such decisions more carefully to cover other traditional cases in which guards occur and we document them in the supplementing source code.
\lstDeleteShortInline@

%\naman{add examples $\dots$ } \spsays{do we re-run codeql on this generated bad program? -- NO (naman)}



\section{Empirical results}
\subsection{Liquidity supply on high- and low-fee pools}

To formally test the model predictions and quantify the differences in liquidity supply across fragmented pools, we build a panel data set for the 37 fragmented pairs in our sample where the unit of observation is pool-day. We estimate linear regressions of liquidity and volume measures on liquidity fees and gas costs:
\begin{equation}
    y_{ijt}=\alpha + \beta_0 d_\emph{low-fee, ij} + \beta_1 \text{GasPrice}_{jt} + \beta_2 \text{GasPrice}_{jt} \times d_\emph{low-fee, ij} + \sum \beta_k \emph{Controls}_{ijt} + \theta_j + \delta_w + \varepsilon_{ijt},
\end{equation}
where $y$ is a variable of interest, $i$ indexes liquidity pools, $j$ runs over asset pairs, and $t$ and $w$ indicates days and weeks, respectively. The dummy $d_\emph{low-fee, ij}$ takes the value one for the pool with the lowest fee in pair $j$ and zero else. 



Further, our set of controls includes pair and week fixed effects, the log aggregate trading volume and log liquidity supply (i.e., total value locked) for day $t$ across all pools $i$. Volume and liquidity are measured in US dollars. We also control for return volatility, computed as the 20-day rolling standard deviation of daily pair returns. Daily returns are obtained from the Kaiko Cross-Price data set, which measures the daily volume-weighted average price across all exchanges where the pair is traded. Therefore, our measure of returns and volatility is less likely to be affected by attempts to manipulate prices on any particular exchange.

Consistent with Figure \ref{fig:stat_liq}, most of the capital deployed to provide liquidity for a given pair is locked in high-fee pools. At the same time, low-fee pools attract much larger trading volume. Models (1) and (5) show that low-fee pools attract 44.2\% of liquidity supply for the average pair (that is, equal to $\nicefrac{(100-11.57)}{2}$) while they execute 64.5\% (i.e., $\nicefrac{(100+29.01)}{2}$) of the total trading volume. At a first glance, it would seem that a majority of capital on decentralized exchanges is inefficiently deployed in pools with low execution probability. We will show that, in line with our model, the difference is driven by heterogeneous liquidity cycles across pools, leading to the formation of $\LP$ clienteles. 

The regression results in Table \ref{tab:markeshare} support Prediction \ref{pred:comp_stat_Gamma}, stating that market share differences between pools are linked to variation in fixed transaction costs on the blockchain. A one-standard deviation increase in gas prices leads to a 2.29 percentage point increase in the high-fee liquidity share. The results suggests that blockchain transaction costs have an economically meaningful and statistically significant impact on liquidity fragmentation. In line with the theoretical model in Section \ref{sec:model}, a jump in gas prices leads to a reshuffling of liquidity supply from low- to high-fee pools. 

Evidence suggests that a higher gas price leads to a larger volume share for the high-fee pool, but the correlation is noisier, and the effect is not statistically significant at conventional levels. The weak link between gas prices and volume is consistent with our model assumption that incoming order flow is inelastic with respect to transaction costs. 


% Table created by stargazer v.5.2.3 by Marek Hlavac, Social Policy Institute. E-mail: marek.hlavac at gmail.com
% Date and time: Tue, Oct 25, 2022 - 8:34:54 PM

\begin{table}[H] 
  \caption{Liquidity pool market shares and gas prices}   \label{tab:markeshare}
\begin{minipage}[t]{1\columnwidth}%
\footnotesize
			This table reports the coefficients of the following regression:
	\begin{align*}
    \text{MarketShare}_{ijt}=\alpha + \beta_0 d_\emph{low-fee, ij} + \beta_1 \text{GasPrice}_{jt} + \beta_2 \text{GasPrice}_{jt} \times d_\emph{low-fee, ij} + \sum \beta_k \emph{Controls}_{ijt} + \theta_j + \varepsilon_{ijt}
    \end{align*}
	where the dependent variable is the liquidity or trading volume market share for pool $i$ in asset pair $j$ on day $t$. $d_\emph{low-fee, ij}$ is a dummy that takes the value one for the pool with the lowest fee in pair $j$ and zero else. $\emph{GasPrice}_{jt}$ is the average of the lowest 100 bids on liquidity provision events across all pairs on day $t$, standardized to have a zero mean and unit variance. \emph{Volume} is the natural logarithm of the sum of all swap amounts on day $t$, expressed in thousands of US dollars. \emph{Total value locked} is the natural logarithm of the total value locked on Uniswap v3 pools on day $t$, expressed in millions of dollars. \emph{Volatility} is 20-day rolling standard deviation of daily returns, using the volume-weighted average price for each day across all exchanges where the pair is traded. All regressions include pair and week fixed-effects. Robust standard errors in parenthesis are clustered by week and ***, **, and * denote the statistical significance at the 1, 5, and 10\% level, respectively.  The sample period is from May 4, 2021 to September 15, 2022. 
\end{minipage}
\small
\begin{center}
\resizebox{\textwidth}{!}{
\begin{tabular}{lcccc@{\hskip 0.3in}cccc}
\toprule
 & \multicolumn{4}{c}{Liquidity market share (\%)} & \multicolumn{4}{c}{Volume market share (\%)} \\ 
 & (1) & (2) & (3) & (4) & (5) & (6) & (7) & (8) \\
 \cmidrule{1-9}

$ d_\text{low-fee}$ & -11.57*** & -11.82*** & -11.57*** & -11.57*** & 29.01*** & 28.49*** & 29.01*** & 29.01*** \\
 & (-16.82) & (-16.18) & (-16.82) & (-16.82) & (25.95) & (23.43) & (25.92) & (25.92) \\
Gas price $\times$ $ d_\text{low-fee}$ & -2.30*** & -2.02** & -2.30*** & -2.30*** & -2.30* & -1.37 & -2.30* & -2.30* \\
 & (-3.19) & (-2.64) & (-3.19) & (-3.19) & (-1.77) & (-0.99) & (-1.77) & (-1.77) \\
Gas price & 1.26*** & 1.09*** & 1.30*** & 1.30*** & 1.28* & 0.80 & 1.50** & 1.50** \\
 & (3.32) & (2.68) & (3.43) & (3.43) & (1.95) & (1.14) & (2.23) & (2.23) \\
Volume & -0.28** & -0.31** & -0.34** & -0.34** & -0.54*** & -0.58*** & -0.91* & -0.91* \\
Total value locked & -0.77* & -1.10** &  &  & -4.56*** & -5.11*** &  &  \\
 & (-1.74) & (-2.45) &  &  & (-3.54) & (-4.10) &  &  \\
Volatility & 0.06** &  & 0.06** & 0.06** & 0.02 &  & 0.02 & 0.02 \\
 & (2.54) &  & (2.53) & (2.53) & (0.87) &  & (0.78) & (0.78) \\
Constant & 65.84*** & 68.95*** & 60.58*** & 60.58*** & 79.44*** & 84.64*** & 48.53*** & 48.53*** \\
 & (15.85) & (15.87) & (24.95) & (24.95) & (7.21) & (7.72) & (6.44) & (6.44) \\
Pair FE & Yes & Yes & Yes & Yes & Yes & Yes & Yes & Yes \\
Week FE & Yes & Yes & Yes & Yes & Yes & Yes & Yes & Yes \\
Observations & 20,454 & 21,097 & 20,454 & 20,454 & 20,454 & 21,097 & 20,454 & 20,454 \\
 R-squared & 0.03 & 0.03 & 0.03 & 0.03 & 0.13 & 0.13 & 0.13 & 0.13 \\ \hline
\bottomrule
\multicolumn{9}{l}{Robust t-statistics in parentheses. Standard errors are clustered at week level.  *** p$<$0.01, ** p$<$0.05, * p$<$0.1} \\
\end{tabular}
}
\end{center}
\end{table} 

What drives the market share gap across fragmented pools? In Table \ref{tab:ordersize} we document stark differences between the characteristics of individual orders supplying or demanding liquidity on pools with low and high fees. On the liquidity supply side, model (1) in Table \ref{tab:ordersize} shows that the average liquidity mint is 91.5\% larger on low-fee pools, which supports Prediction \ref{pred:clienteles} of the model.\footnote{Since all dependent variables are measured in natural logs, the marginal impact of a dummy coefficient $\beta$ is computed $\left(e^\beta-1\right) \times 100$ percent.} At the same time, there are 18\% fewer $\LP$ wallets (Model 5) and 16\% fewer liquidity events on the low-fee pool (Model 6).

\begin{table}[H] 
\caption{Fragmentation and order flow characteristics}  \label{tab:ordersize}
\begin{minipage}[t]{1\columnwidth}%
\footnotesize
			This table reports the coefficients of the following regression:
	\begin{align*}
    y_{ijt}=\alpha + \beta_0 d_\emph{low-fee, ij} + \beta_1 \text{GasPrice}_{jt} d_\emph{low-fee, ij} + \beta_2 \text{GasPrice}_{jt} \times d_\emph{high-fee, ij} + \sum \beta_k \emph{Controls}_{ijt} + \theta_j + \varepsilon_{ijt}
    \end{align*}
	where the dependent variable $y_{ijt}$ can be (i) the log median mint size, (ii) the log median trade size, (iii) the log trading volume, (iv) the log trade count $\log(1+\# trades)$, (v) the log count of unique $\LP$ wallets interacting with a pool in a given day, or (vi) the log count of interactions with a liquidity pool, for pool $i$ in asset $j$ on day $t$for pool $i$ in asset $j$ on day $t$. $d_\emph{low-fee, ij}$ is a dummy that takes the value one for the pool with the lowest fee in pair $j$ and zero else. $d_\emph{high-fee, ij}$ is defined as $1-d_\emph{low-fee, ij}$. $\emph{GasPrice}_{jt}$ is the average of the lowest 100 bids on liquidity provision events across all pairs on day $t$, standardized to have a zero mean and unit variance. \emph{Volume} is the natural logarithm of the sum of all swap amounts on day $t$, expressed in thousands of US dollars. \emph{Total value locked} is the natural logarithm of the total value locked on Uniswap v3 pools on day $t$, expressed in millions of dollars. \emph{Volatility} is 20-day rolling standard deviation of daily returns, using the volume-weighted average price for each day across all exchanges where the pair is traded. All regressions include pair and week fixed-effects. Robust standard errors in parenthesis are clustered by week and ***, **, and * denote the statistical significance at the 1, 5, and 10\% level, respectively.  The sample period is from May 4, 2021 to September 15, 2022. 
\end{minipage}
\begin{center}
\resizebox{1\textwidth}{!}{  
\begin{tabular}{lccccccc}
\toprule
 & \multicolumn{1}{c}{Mint size} &  \multicolumn{1}{c}{Trade size} & \multicolumn{1}{c}{Volume} & \multicolumn{1}{c}{\# Trades}  & \multicolumn{1}{c}{\# Wallets} & \multicolumn{1}{c}{$\LP$ interactions} & \multicolumn{1}{c}{Price range} \\
 & (1) & (2) & (3) & (4) & (5) & (6) & (7) \\
\cmidrule{1-8}
$ d_\text{low-fee}$ & 0.65*** & -0.32*** & 1.08*** & 1.19*** & -0.20*** & -0.18*** & -0.14*** \\
 & (9.76) & (-14.48) & (19.85) & (47.69) & (-7.71) & (-5.91) & (-24.80) \\
Gas price $\times$ $ d_\text{low-fee}$ & 0.38*** & 0.12*** & -0.03 & -0.15*** & -0.21*** & -0.24*** & -0.00 \\
 & (4.03) & (4.21) & (-0.80) & (-5.64) & (-10.07) & (-9.76) & (-0.57) \\
Gas price $\times$ $ d_\text{high-fee}$ & 0.54*** & 0.16*** & 0.21*** & -0.01 & -0.12*** & -0.11*** & -0.03*** \\
 & (5.85) & (6.75) & (4.06) & (-0.28) & (-4.19) & (-3.38) & (-4.21) \\
Volume & 0.69*** & 0.30*** & 0.73*** & 0.38*** & 0.06*** & 0.11*** & -0.04*** \\
 & (8.13) & (14.84) & (15.88) & (12.89) & (3.44) & (5.09) & (-4.14) \\
Total value locked & -0.58* & -0.03 & -0.07 & -0.28*** & -0.08 & -0.15** & 0.08** \\
 & (-1.97) & (-0.16) & (-0.34) & (-3.38) & (-1.33) & (-2.23) & (2.28) \\
Volatility & -0.00 & -0.03*** & 0.05 & 0.11** & 0.01 & 0.02 & 0.01 \\
 & (-0.03) & (-3.39) & (0.62) & (2.58) & (0.65) & (1.38) & (1.59) \\
Constant & -2.68 & -2.01 & -4.08*** & 0.54 & 1.15** & 1.24*** & 0.53** \\
 & (-1.08) & (-1.50) & (-2.72) & (1.01) & (2.62) & (2.69) & (2.15) \\
 &  &  &  &  &  &  \\
Pair FE & Yes & Yes & Yes & Yes & Yes & Yes & Yes  \\
Week FE & Yes & Yes & Yes & Yes & Yes & Yes  & Yes \\
Observations & 11,695 & 20,454 & 20,454 & 20,454 & 20,454 & 20,454 & 13,920 \\
R-squared & 0.26 & 0.55 & 0.59 & 0.67 & 0.62 & 0.60 & 0.48 \\ 
\bottomrule
\multicolumn{7}{l}{Robust t-statistics in parentheses. Standard errors are clustered at week level.} \\
\multicolumn{7}{l}{*** p$<$0.01, ** p$<$0.05, * p$<$0.1} \\
\end{tabular}
}
\end{center}
\end{table}



On the liquidity supply side, trades on the low-fee pool are 27.3\% smaller (Model 2), consistent with Prediction \ref{pred:trade_size_volume}. However, the low-fee pool executes more than three times the number of trades (i.e., trade count is 228\% higher from Model 4) and has 194\% higher volume than the high-fee pool (Model 3).

Our findings (Model 7) indicate that liquidity providers on low-fee pools select price ranges that are 26\% (=0.14/0.53) narrower when minting liquidity compared to those on high-fee pools. This pattern aligns with the capability of large LPs to adjust their liquidity positions frequently, enabling more efficient capital concentration. Similarly, \citet{caparros2023blockchain} report a higher concentration of liquidity in pools on alternative blockchains like Polygon, known for lower transaction costs than Ethereum.

The results point to an asymmetric match between liquidity supply and demand across pools. On low-fee pools, a few $\LP$s provide large chunks of liquidity for the vast majority of incoming small trades. Conversely, on high-fee pools there is a sizeable mass of small liquidity providers that mostly trade against a few large incoming trades. 

How does variation in fixed transaction costs impact the gap between individual order size across pools? We find that increasing the gas price by one standard deviation leads to higher liquidity deposits on both the low- and the high-fee pools (54\% and 67\% higher, respectively). The result supports Prediction \ref{pred:clienteles_cs} of the model. Our theoretical framework implies that a larger gas price leads to some (marginal) $\LP$s switching from the low- to the high-fee pool. The switching $\LP$s have low capital endowments relative to their low-fee pool peers, but higher than $\LP$s on the high-fee pool. Therefore, the gas-driven reshuffle of liquidity leads to a higher average endowment on both high- and low-fee pools. Consistent with the model, a higher gas price leads to fewer active liquidity providers, particularly on low-fee pools. A one-standard increase in gas costs is associated with an 18\% and 11\% drop in the number of $\LP$ wallets interacting daily with the low- and high-fee pools, respectively (Model 5).

While a higher gas price is correlated with a shift in liquidity supply, it has a muted impact on liquidity demand on low-fee pools. A higher gas cost is associated with 12.7\% larger trades (Model 2), likely as traders aim to achieve better economies of scale. At the same time, the number of trades on the low-fee pool drops by 13.9\% (Model 4) -- since small traders might be driven out of the market. The net of gas prices effect on aggregate volume on the low-fee pool is small and not statistically significant (Model 3). The result matches our model assumption that the aggregate order flow on low-fee pool is not sensitive to gas prices.\footnote{Formally, one could extend the model to assume that small $\LT$s arrive at the market at rate $\tilde{\theta}\left(\Gamma\right) \diff t$ and demand $f\left(\Gamma\right)$ units each, where $\tilde{\theta}\left(\Gamma\right)$ decreases in $\Gamma$ and $f\left(\Gamma\right)$ increases in $\Gamma$ such that $\tilde{\theta}\left(\Gamma\right)f\left(\Gamma\right) =\theta$.} 

\begin{table}[H] 
\caption{Liquidity flows and gas costs on fragmented pools} \label{tab:flows}
\begin{minipage}[t]{1\columnwidth}%
\footnotesize
			This table reports the coefficients of the following regression:
	\begin{align*}
    y_{ijt}=\alpha + \beta_0 d_\emph{low-fee, ij} + \beta_1 \text{GasPrice}_{jt} d_\emph{low-fee, ij} + \beta_2 \text{GasPrice}_{jt} \times d_\emph{high-fee, ij} + \sum \beta_k \emph{Controls}_{ijt} + \theta_j + \varepsilon_{ijt}
    \end{align*}
	where the dependent variable $y_{ijt}$ can be (i) the aggregate dollar value of mints (in logs), or (vi) a dummy variable taking value one hundred if there is at least one mint on liquidity pool $i$ in asset $j$ on day $t$. $d_\emph{low-fee, ij}$ is a dummy that takes the value one for the pool with the lowest fee in pair $j$ and zero else. $d_\emph{high-fee, ij}$ is defined as $1-d_\emph{low-fee, ij}$. $\emph{GasPrice}_{jt}$ is the average of the lowest 100 bids on liquidity provision events across all pairs on day $t$, standardized to have a zero mean and unit variance. \emph{Volume} is the natural logarithm of the sum of all swap amounts on day $t$, expressed in thousands of US dollars. \emph{Total value locked} is the natural logarithm of the total value locked on Uniswap v3 pools on day $t$, expressed in millions of dollars. \emph{Volatility} is 20-day rolling standard deviation of daily returns, using the volume-weighted average price for each day across all exchanges where the pair is traded. All regressions include pair and week fixed-effects. Robust standard errors in parenthesis are clustered by week and ***, **, and * denote the statistical significance at the 1, 5, and 10\% level, respectively.  The sample period is from May 4, 2021 to September 15, 2022. 
\end{minipage}
\begin{center}
\begin{tabular}{lcccccc}
\toprule
 & \multicolumn{3}{c}{Daily mints (log US\$)} &  \multicolumn{3}{c}{$\text{Prob}\left(\text{at least one mint}\right)$} \\
 & (1) & (2) & (3) & (4) & (5) & (6) \\
\cmidrule{1-7}
$ d_\text{low-fee}$ & 0.15* & 0.16** & 0.15* & 1.33* & 1.30* & 1.33* \\
 & (1.94) & (2.03) & (1.94) & (1.82) & (1.85) & (1.82) \\
Gas price $\times$ $ d_\text{low-fee}$ & -0.36*** & -0.36*** & -0.39*** & -7.60*** & -7.63*** & -5.68*** \\
 & (-6.66) & (-6.43) & (-5.22) & (-9.36) & (-9.09) & (-8.22) \\
Gas price $\times$ $ d_\text{high-fee}$ & 0.03 & 0.00 &  & -1.92*** & -2.14*** &  \\
 & (0.33) & (0.00) &  & (-2.74) & (-2.85) &  \\
Volume & 0.45*** & 0.44*** & 0.45*** & 1.19 & 1.17 & 1.19 \\
 & (7.16) & (7.04) & (7.16) & (1.33) & (1.25) & (1.33) \\
Total value locked & -0.45*** & -0.52*** & -0.45*** & -5.31** & -5.56** & -5.31** \\
 & (-2.75) & (-3.34) & (-2.75) & (-2.43) & (-2.52) & (-2.43) \\
Volatility & 0.02 &  & 0.02 & 1.50* &  & 1.50* \\
 & (0.73) &  & (0.73) & (1.80) &  & (1.80) \\
Gas price &  &  & 0.03 &  &  & -1.92*** \\
 &  &  & (0.33) &  &  & (-2.74) \\
Constant & 0.55 & 1.14 & 0.55 & 81.06*** & 82.73*** & 81.06*** \\
 & (0.60) & (1.36) & (0.60) & (6.12) & (5.72) & (6.12) \\
 &  &  &  &  &  &  \\
Pair FE & Yes & Yes & Yes & Yes & Yes & Yes  \\
Week FE & Yes & Yes & Yes & Yes & Yes & Yes  \\
Observations & 20,454 & 21,097 & 20,454 & 21,097 & 20,454 & 20,454 \\
 R-squared & 0.51 & 0.51 & 0.51 & 0.61 & 0.62 & 0.62 \\ \hline
\bottomrule
\multicolumn{7}{l}{Robust t-statistics in parentheses. Standard errors are clustered at week level.} \\
\multicolumn{7}{l}{*** p$<$0.01, ** p$<$0.05, * p$<$0.1} \\
\end{tabular}
\end{center}
\end{table}


On the high-fee pool, a higher gas price is also associated with a higher trade size, but does not lead to a change in the number of trades. The implication is that large-size traders who typically route orders to the high-fee pool are unlikely to be deterred by an increase in fixed costs, while they do adjust quantities to achieve better economies of scale. In the context of our model, the large $\LT$ arrival rate $\lambda$ does not depend on the gas price $\Gamma$.




In Table \ref{tab:flows}, we shift the analysis from individual orders to aggregate daily liquidity flows to Uniswap pools. We find that higher gas prices lead to a decrease in liquidity inflows, but only on the low fee pools. A one standard deviation increase in gas prices leads to a 30\% drop in new liquidity deposits by volume (Model 1) and an 7.60\% drop in probability of having at least one mint (Model 4) on the low-fee pool. However, the slow-down in liquidity inflows is less evident in high fee pools. While an increase in gas prices reduce the probability of liquidity inflows by 1.92\%, it has a negligible impact on the daily dollar inflow to the pool. Together with the result in Table \ref{tab:ordersize} that the size of individual mints increases with gas prices, our evidence is consistent with the model implication that higher fixed transaction costs change the composition of liquidity supply on the high-fee pool, with small $\LP$ being substituted by larger $\LP$s switching over from the low-fee pool.

\subsection{Liquidity cycles on high- and low-fee pools}

Next, we test Predictions \ref{pred:updates} and \ref{pred:updates_gas} on the duration of liquidity cycles on fragmented pools. Since the descriptive statistics in Table \ref{tab:sumstat} suggest that $\LP$s manage their positions over multiple days, we cannot accurately measure liquidity cycles in a pool-day panel. Instead, we use intraday data on liquidity events (either mints or burns) to measure the duration between two consecutive opposite-sign interactions by the same Ethereum wallet with a liquidity pool: either a mint followed by a burn, or vice-versa. 


\begin{table}[H] 
\caption{Liquidity cycles on fragmented pools} \label{tab:cycles}
\begin{minipage}[t]{1\columnwidth}%
\footnotesize
			This table reports the coefficients of the following regression:
	\begin{align*}
    y_{ijtk}=\alpha + \beta_0 d_\emph{low-fee, ij} + \beta_1 \text{GasPrice}_{jt} d_\emph{low-fee, ij} + \beta_2 \text{GasPrice}_{jt} \times d_\emph{high-fee, ij} + \sum \beta_k \emph{Controls}_{ijt} + \theta_j + \varepsilon_{ijt}
    \end{align*}
	where the dependent variable $y_{ijt}$ can be (i) the mint-to-burn time, (ii) the burn-to-mint time, measured in hours, for a transaction initiated by wallet $k$ on day $t$ and pool $i$ trading asset $j$. The mint-to-burn and burn-to-mint times are computed for consecutive interactions of the same wallet address with the liquidity pool. $d_\emph{low-fee, ij}$ is a dummy that takes the value one for the pool with the lowest fee in pair $j$ and zero else. $d_\emph{high-fee, ij}$ is defined as $1-d_\emph{low-fee, ij}$. $\emph{GasPrice}_{jt}$ is the average of the lowest 100 bids on liquidity provision events across all pairs on day $t$, standardized to have a zero mean and unit variance. \emph{Volume} is the natural logarithm of the sum of all swap amounts on day $t$, expressed in thousands of US dollars. \emph{Total value locked} is the natural logarithm of the total value locked on Uniswap v3 pools on day $t$, expressed in millions of dollars. \emph{Volatility} is 20-day rolling standard deviation of daily returns, using the volume-weighted average price for each day across all exchanges where the pair is traded. $d_\text{out-range}$ is a dummy taking value one if the position being burned or minted is out of range, that is if the price range selected by the $\LP$ does not straddle the current pool price. All variables are measured as of the time of the second leg of the cycle (i.e., the burn of a mint-burn cycle). All regressions include pair, week, and trader wallet fixed-effects. Robust standard errors in parenthesis are clustered by day and ***, **, and * denote the statistical significance at the 1, 5, and 10\% level, respectively.  The sample period is from May 4, 2021 to September 15, 2022. 
\end{minipage}
\begin{center}
\begin{tabular}{lcccccc}
\toprule
 & \multicolumn{3}{c}{Mint-burn time} & \multicolumn{3}{c}{Burn-mint time} \\
 & (1) & (2) & (3) & (4) & (5) & (6)  \\
\cmidrule{1-7}
$ d_\text{low-fee}$ & -77.26*** & -95.74*** & -99.63*** & -117.18*** & -132.22*** & -132.65*** \\
 & (-8.53) & (-10.46) & (-11.01) & (-9.76) & (-10.49) & (-10.53) \\
Gas price $\times$ $ d_\text{low-fee}$ & -30.43*** & -33.90*** & -33.62*** & -10.03 & -13.01* & -12.93* \\
 & (-3.76) & (-4.04) & (-4.02) & (-1.61) & (-1.88) & (-1.86) \\
Gas price $\times$ $ d_\text{high-fee}$ & -16.84*** & -9.75* & -9.13 & -1.08 & 0.45 & 0.53 \\
 & (-2.99) & (-1.77) & (-1.67) & (-0.20) & (0.07) & (0.08) \\
Volume &  & 1.46 & -1.04 &  & -6.54 & -7.01 \\
 &  & (0.18) & (-0.13) &  & (-0.76) & (-0.82) \\
Total value locked &  & 73.87 & 80.68 &  & -74.14* & -73.89* \\
 &  & (1.05) & (1.18) &  & (-1.72) & (-1.71) \\
Volatility &  & 3.37 & 2.70 &  & -50.19*** & -50.23*** \\
 &  & (0.17) & (0.14) &  & (-5.57) & (-5.59) \\
Position out-of-range &  &  & 46.80*** &  &  & 14.39** \\
 &  &  & (8.60) &  &  & (2.27) \\
Constant & 389.08*** & -174.79 & -222.32 & 150.80*** & 831.58** & 833.67** \\
 & (110.18) & (-0.30) & (-0.39) & (29.01) & (2.39) & (2.40) \\
 &  &  &  &  &  &  \\
 Pair FE & Yes & Yes & Yes & Yes & Yes & Yes  \\
 Week FE & Yes & Yes & Yes & Yes & Yes & Yes  \\
 Trader wallet FE & Yes & Yes & Yes & Yes & Yes & Yes \\
Observations & 287,505 & 265,182 & 265,182 & 196,145 & 182,581 & 182,581 \\
 R-squared & 0.82 & 0.82 & 0.82 & 0.37 & 0.38 & 0.38 \\ \hline
\bottomrule
\multicolumn{7}{l}{Robust t-statistics in parentheses. Standard errors are clustered at week level.} \\
\multicolumn{7}{l}{*** p$<$0.01, ** p$<$0.05, * p$<$0.1} \\
\end{tabular}

\end{center}
\end{table}

To complement our previous analysis, we additionally control for whether each liquidity position is out of range (i.e., the price range set by the $\LP$ does not straddle the current price and therefore the $\LP$ does not earn fees). We further introduce wallet fixed effects to soak up variation in reaction times across traders.

Table \ref{tab:cycles} presents the results. Liquidity updates on decentralized exchanges are very infrequent, as times elapsed between consecutive interactions are measured in days or even weeks. In line with Prediction \ref{pred:updates}, we find evidence for shorter liquidity cycles on low-fee pools. The average time between consecutive mint and burn orders is 19.85\% shorter on the low-fee pool (from Model 1, the relative difference is 77.26 hours/389.08 hours).

Liquidity cycles are in part driven by fixed Blockchain transaction costs. A one standard deviation increase in gas prices speeds up the liquidity cycle on low-fee pools by a further 33.6 hours (Model 3), without significantly changing the average mint-to-burn time on high-fee pools. The result supports the intuition behind Prediction \ref{pred:updates_gas}. When the gas price spikes, the liquidity supply on the low-fee pool decreases at a higher rate than the liquidity demand. As a result, liquidity deposits deplete faster (i.e., they move outside the fee-earning range), triggering the need for more frequent updates. We do not see a similar impact on the high-fee pool, for which the duration of liquidity cycles is not sensitive to gas prices (Models 2 and 3). The result is intuitive, since neither aggregate liquidity supply (measured as mint inflows in Table \ref{tab:flows}) or liquidity demand (i.e., corresponding to trade volume in Table \ref{tab:ordersize}) on the high-fee pool change significantly with gas prices.

The reaction time of a trader may depend on capital constraints, network congestion, and other confounding factors that can correlate with gas prices. As a robustness check, we repeat the analysis above with burn-to-mint times as the dependent variables. The burn-to-mint time measures the speed at which $\LP$s deposit liquidity at updated prices after removing (out-of-range) positions, and should not depend on the rate at which liquidity is consumed. Consistent with the theory, we find no significant relationship between gas price and the burn-to-mint duration (Models 4 through 6).

\subsection{Adverse selection costs across low- and high-fee pools}

Our findings indicate that larger aggressive orders tend to execute on high-fee liquidity pools. Given that, at least in equity markets, larger incoming orders tend to be better informed \citep{Hasbrouck1991}, a natural question is whether liquidity providers on high-fee pools face higher adverse selection. 

A widely-used measure for gauging informational costs for liquidity providers on decentralized exchanges employing automated market makers (AMMs) is the ``impermanent loss,'' the equivalent of adverse selection measures in traditional limit order markets  \citep[see, for example,][]{aoyagi2020,BarbonRanaldo2021}. The impermanent loss (IL) is defined as the negative return from providing liquidity as opposed to holding the assets outside the exchange and marking them to market as the price evolves. Intuitively, if the fundamental value of the asset changes, an arbitrageur trades at the stale price in the direction of the price change, thus minimizing the value of the liquidity pool \citep{zhang2023amm}. 

To illustrate, let's consider a straightforward example. At some initial time $t_0$, a liquidity provider deposits 1 ETH (i.e., a token quantity of $T_0=1$) and 3000 USDC (i.e., a numeraire $N_0=3000$) into a liquidity pool, establishing a token price of $p_0=3000$.  If at the next time interval ($t=1$), the intrinsic value surges to $v_1=3500$, an arbitrager has the incentive to remove ETH from the pool and deposit USDC until the price reflects the intrinsic value, or $\frac{N_1}{T_1}=3500$.

Given that the AMM requires the product of token and numeraire quantities to be constant ($T_0 N_0 = T_1 N_1$), this leads to $T_1 = 0.926$ and $N_1=3240.37$. In terms of the numeraire equivalent, the liquidity provider's position is valued at $V_\text{pos}=\$3500 \times 0.926 + \$3240.37 = \$6480.74$.  Had the liquidity provider not deposited their tokens on the exchange and instead held it in its own account, the value would be $V_\text{hold}=\$3500 \times 1 + \$3000 = \$6500$ (that is, equal to $T_0 \times p_1 + N_0$).

Hence, disregarding gas cost and liquidity fees, the impermanent loss can be calculated as:
\begin{equation}
    \text{ImpermanentLoss}=\frac{V_\text{hold}-V_\text{pos}}{V_\text{hold}}=29 \text{ bps}.
\end{equation}

However, liquidity providers on Uniswap V3 pools can set a price range for their orders. While this feature caps the potential adverse selection cost, it simultaneously introduces a new layer of computational complexity. In Appendix \ref{app:IL}, we present the exact formulas for calculating impermanent loss on Uniswap V3, based on the methodology described by \citet{Heimbach2023}. 

We obtain block-by-block liquidity snapshots from Kaiko, which allow us to calculate the impermanent loss for symmetric liquidity positions within a price range of $\left[\frac{1}{\alpha} p, \alpha p\right]$ centered around the current pool price $p$. We consider different ranges of $\alpha$, specifically $\alpha\in\left\{1.01, 1.05, 1.1, 1.25\right\}$. The liquidity management horizon, which represents the time delay between measuring $p_0$ and $p_1$, is set to 200 blocks or approximately 45 minutes.

Table \ref{tab:il} presents the empirical results. In Models 1, 3, 5, and 7, where we exclude control variables, the impermanent loss appears to be higher in low-fee pools. However, the link is mechanical: since high-fee pools exhibit lower trading activity, the price updates less frequently leading to a lower measured impermanent loss. Once we account for trading activity in our analysis (Models 2, 4, 6, and 8), we find no significant difference in the impermanent loss between high-fee and low-fee pools. This finding suggests that liquidity providers in same-asset pools with different fees do not encounter substantially different informational costs, which aligns with the assumptions of our theoretical framework.

\begin{table}[H] 
\scriptsize
\caption{Impermanent loss across high- and low-fee pools}\label{tab:il}
\begin{minipage}[t]{1\columnwidth}%
\footnotesize
This table reports the coefficients of the following regression:
	\begin{align*}
    \text{Impermanent Loss}_{ijt}=\alpha + \beta_0 d_\emph{low-fee, ij}  + \sum \beta_k \emph{Controls}_{ijt} + \theta_j + \varepsilon_{ijt}
    \end{align*}
	where the dependent variable is the daily average impermanent loss for a liquidity position with price range $\left[\frac{p}{\alpha}, \alpha p\right]$ around the current pool price $p$, for $\alpha\in\left\{1.01, 1.05, 1.1., 1.25\right\}$. The average impermanent loss is computed across all Ethereum blocks mined within the day. To compute the impermanent loss, we use a liquidity provider horizon of 200 blocks: that is, we compare the current pool price with the pool price 200 blocks later. $d_\emph{low-fee, ij}$ is a dummy that takes the value one for the pool with the lowest fee in pair $j$ and zero else. $\emph{GasPrice}_{jt}$ is the average of the lowest 100 bids on liquidity provision events across all pairs on day $t$, standardized to have a zero mean and unit variance. $\emph{TradeCount}_{jt}$ is number of trades on pool $j$ and day $t$, standardized to have a zero mean and unit variance. \emph{Volume} is the natural logarithm of the sum of all swap amounts on day $t$, expressed in thousands of US dollars. \emph{Total value locked} is the natural logarithm of the total value locked on Uniswap v3 pools on day $t$, expressed in millions of dollars. \emph{Volatility} is the 20-day rolling standard deviation of daily returns, using the volume-weighted average price for each day across all exchanges where the pair is traded. All regressions include pair and week fixed-effects. Robust standard errors in parenthesis are clustered by week and ***, **, and * denote the statistical significance at the 1, 5, and 10\% level, respectively.  The sample period is from May 4, 2021 to September 15, 2022. 
\end{minipage}

\begin{center}
\resizebox{1\textwidth}{!}{  
\begin{tabular}{lcccccccc}
\toprule
& \multicolumn{8}{c}{Impermanent loss for a liquidity position with range $\left[\frac{p}{\alpha}, \alpha p\right]$ around price $p$} \\
\cmidrule{1-9} 
& \multicolumn{2}{c}{$\alpha=1.01$} & \multicolumn{2}{c}{$\alpha=1.05$} & \multicolumn{2}{c}{$\alpha=1.1$} & \multicolumn{2}{c}{$\alpha=1.25$} \\
 & (1) & (2) & (3) & (4) & (5) & (6) & (7) & (8) \\
\cmidrule{1-9}
$ d_\text{low-fee}$ & 3.41*** & 0.72 & 1.23*** & -0.61 & 0.64*** & -0.66 & 0.20 & -0.49 \\
 & (9.56) & (0.87) & (4.52) & (-0.84) & (2.65) & (-1.10) & (0.90) & (-1.25) \\
Gas price &  & 3.04*** &  & 2.27*** &  & 1.68*** &  & 1.02*** \\
 &  & (3.09) &  & (3.07) &  & (3.03) &  & (2.77) \\
Trade count &  & 2.39*** &  & 1.64** &  & 1.21** &  & 0.68* \\
 &  & (2.96) &  & (2.31) &  & (2.07) &  & (1.94) \\
Volume &  & 7.85*** &  & 3.63*** &  & 2.17*** &  & 0.95*** \\
 &  & (7.37) &  & (5.04) &  & (4.13) &  & (2.74) \\
Total value locked &  & -15.46*** &  & -6.46* &  & -3.18 &  & -1.09 \\
 &  & (-3.41) &  & (-1.87) &  & (-1.18) &  & (-0.64) \\
Volatility &  & 6.39*** &  & 3.79*** &  & 2.45*** &  & 1.25*** \\
 &  & (5.21) &  & (4.92) &  & (4.59) &  & (4.22) \\
Constant & 12.62*** & 26.06 & 5.97*** & 6.42 & 3.84*** & -1.30 & 2.13*** & -2.65 \\
 & (73.61) & (0.88) & (45.69) & (0.27) & (32.98) & (-0.07) & (19.91) & (-0.21) \\
 &  &  &  &  &  &  \\
Pair FE & Yes & Yes & Yes & Yes & Yes & Yes & Yes  \\
Week FE & Yes & Yes & Yes & Yes & Yes & Yes  & Yes \\
Observations & 24,937 & 20,449 & 24,937 & 20,449 & 24,937 & 20,449 & 24,937 & 20,449 \\
 R-squared & 0.14 & 0.28 & 0.07 & 0.17 & 0.04 & 0.12 & 0.02 & 0.07 \\ \hline
\bottomrule
\multicolumn{9}{l}{Robust t-statistics in parentheses. Standard errors are clustered at week level.} \\
\multicolumn{9}{l}{*** p$<$0.01, ** p$<$0.05, * p$<$0.1} \\
\end{tabular}
}
\end{center}
\end{table}








 



\section{Conclusion}
\section{Conclusion and Future Work}
In this work, I design corruption-robust algorithms for the Lipschitz contextual search problem. I present the \emph{agnostic checking} technique and demonstrate its effectiveness in designing corruption-robust algorithms. There are several open problems for future research. First, in the algorithm I propose for pricing loss, the schedule for agnostic checks is fixed upfront. Can the learner design an adaptive checking schedule for the pricing loss? Second, this work assumes the learner has knowledge of the Lipschitz constant $L$. Can the learner design efficient no-regret algorithms without knowledge of $L$? 

\bibliographystyle{jf}
\bibliography{references}

\newpage

\appendix
%\singlespacing

\numberwithin{equation}{section}
\numberwithin{prop}{section}
\numberwithin{lem}{section}
\numberwithin{defn}{section}
\numberwithin{cor}{section}
\numberwithin{figure}{section}
\numberwithin{table}{section}

\let\normalsize\small

\appendix


\section{Liquidity provision mechanism on Uniswap v3 \label{sec:app-dex}}
In this appendix, we walk through a numerical example to illustrate the mechanism of liquidity provision and trading on Uniswap V3 liquidity pools. To facilitate understanding, we highlight the similarities and differences between the Uniswap mechanism and the familiar economics of a traditional limit order book.

Let $p_c=1500.62$ be the current price of the ETH/USDT pair. Traders can provide liquidity on Uniswap V3 pools at prices on a log-linear tick space. In particular, consecutive prices are always $\theta$ basis point apart: $p_i=1.0001^{\theta i}$, where $\theta$ is the tick spacing. For the purpose of the example, we take $\theta=60$. Consequently, the current price of $1500.62$ corresponds to a tick index of $c=73140$. Figure \ref{fig:app_grid} illustrates three ticks on grid below and above the current price of ETH/USDT $1500.62$.

% Figure environment removed

\paragraph{Two-sided liquidity provision.} Trader \textbf{A} starts out with a capital of USDT 20,000 and wants to provide liquidity over the price range $\left[1491.64, 1527.87\right]$, a range which spans four ticks. Liquidity provision over a range that includes the current price corresponds to posting quotes on both the bid and ask side of a traditional limit order book, where the current price of the pool corresponds to the mid-point of the book.
\begin{enumerate}
    \item \emph{Bid quotes:} trader \textbf{A} deposits USDT over the price range $\left[1491.64,1500.62\right)$. This action is equivalent to submitting a buy limit order with a bid price of $1491.64$. An incoming Ether seller can swap their ETH for the USDT deposited by \textbf{A}, generating price impact until the limit price of $1491.64$ is reached.
    \item \emph{Ask quotes:} at the same time, trader \textbf{A} deposits ETH over three ticks: $\left[1500.62,1509.65\right)$, $\left[1509.65,1518.73\right)$, and $\left[1518.73,1527.87\right)$. The action corresponds to submitting \emph{three} sell limit orders with ask prices $1509.65$, $1518.73$, and $1527.87$, respectively. Incoming Ether buyers can swap USDT for trader \textbf{A}'s ETH.
\end{enumerate}

In the Uniswap V3 protocol, deposit amounts over each tick $\left[p_i,p_{i+1}\right)$ must satisfy
\begin{align}\label{eq:app_deposits_tick}
    \text{ETH deposit over $\left[p_{i},p_{i+1}\right)$: } x_i&=L\left(\frac{1}{\sqrt{p_i}}-\frac{1}{\sqrt{p_{i+1}}}\right) \\
    \text{USDT deposit over $\left[p_{i},p_{i+1}\right)$: } y_i &= L\left(\sqrt{p_{i+1}}-\sqrt{p_i}\right), 
\end{align}
where $L$ (``liquidity units'') is a scaling factor proportional to the capital committed to the liquidity position. The scaling factor $L$ is pinned down by setting the total committed capital equal to the sum of the positions (in USDT), that is  $p_c\sum_{i} x_i + \sum_{i} y_i$. In our example,
\begin{equation}
    1500.62 \times L_A \times \left(\frac{1}{\sqrt{1500.62}}-\frac{1}{\sqrt{1527.87}}\right) + L_A \times \left(\sqrt{1500.62}-\sqrt{1491.64}\right) = 20000,
\end{equation}
leading to $L_A=43188.6$. We the value of $L_A$ into \eqref{eq:app_deposits_tick} and conclude that trader \textbf{A} deposits 5,013.38 USDT over $\left[1491.64, 1500.62\right)$ and ETH 9.99 over  $\left[1500.62, 1527.87\right)$ (approximately ETH 3.33 over each tick size covered).

\paragraph{One-sided liquidity provision.} Trader \textbf{B} has USDT 20,000 and wants to post liquidity over the range $\left[1509.65,1527.87\right)$, which does not include the current price.  This action corresponds to posting ask quotes to sell ETH deep in the book, at price levels $1518.73$ and $1527.87$. Liquidity is not ``active'' -- that is, the quotes are not filled -- until the existing depth at $1509.65$ is consumed by incoming trades. 

We use equation \eqref{eq:app_deposits_tick} to solve for the amount of liquidity units provided by \textbf{B}:
\begin{equation}
    1500.62 \times L_B \times \left(\frac{1}{\sqrt{1509.65}}-\frac{1}{\sqrt{1527.87}}\right) = 20000,
\end{equation}
which leads to $L_B=86589.4$. Trader \textbf{B} deposits 6.67 ETH on each of the two ticks covered by the chosen range. 

% Figure environment removed


Figure \ref{fig:liquidity_pool} illustrates market depth after \textbf{A} and \textbf{B} deposit liquidity in the pool. The current price of the pool is equivalent to a midpoint in traditional limit order markets. The ``ask side'' of the pool is deeper, consistent with both liquidity providers choosing ranges skewed towards prices above the current midpoint. Liquidity is uniformly provided over ticks -- that is, each trader deposits an equal share of their capital at each price tick covered by their price range. 


\paragraph{Trading, fees, and price impact.} Suppose now that a trader \textbf{C} wants to buy 10 ETH from the pool. For each tick interval $\left[p_i,p_{i+1}\right)$, price impact is computed using a constant product function over virtual reserves:
\begin{equation}
    \underbrace{\left(x+\frac{L}{\sqrt{p_{i+1}}}\right)}_\text{Virtual ETH reserves}\underbrace{\left(y+L\sqrt{p_i}\right)}_\text{Virtual USDT reserves}=L^2,
\end{equation}
where $x$ and $y$ are the actual ETH and USDT deposits in that tick range, respectively. Virtual reserves are just a mathematical artifact: they extend the physical (real) deposits as if liquidity would be uniformly distributed over all possible prices on the real line. Working with constant product functions over real reserves is not feasible: in our example, the product of real reserves is zero throughout the order book (since only one asset is deposited in each tick range).

Let $\tau=1\%$ denote the pool fee that serves as an additional compensation for liquidity providers. That is, if the buyer pays to pay $\Delta y$ USDT to purchase a quantity $\Delta x$ ETH, he needs to effectively pay $\Delta y\left(1+\tau\right)$. As per the Uniswap V3 white paper, liquidity fees are not automatically deposited back into the pool.

\begin{enumerate}
    \item \textbf{Tick 1: $\left[1500.62,1509.65\right)$}. Trader \textbf{C} first purchases $3.33$ ETH at the first available tick above the current price (equivalent to the ``best ask''). To remove the ETH, he needs to deposit $\Delta y_1$ USDT, where $\Delta y_1$ solves:
    \begin{equation}
        \left(3.33-3.33+\frac{L_A}{\sqrt{1509.65}}\right)\left(0+\Delta y_1+L_A\sqrt{1500.62}\right)=L_A^2,
    \end{equation}

which leads to $\Delta y_1=5026.19$ USDT. Trader \textbf{C} pays an average price of 50216.19/3.33=1507.86 USDT for each unit of ETH purchased. Further, he pays a fee of 50.26 USDT to liquidity provider \textbf{A} (the only liquidity provider at this tick).

The new current price is given by the ratio of virtual reserves,
\begin{equation}
    p^\prime=\frac{\Delta y_1 +L_A\sqrt{1500.62}}{3.33-3.33+\frac{L_A}{\sqrt{1509.65}}}=1509.65,
\end{equation}
that is the next price on the tick grid since \textbf{C} exhausts the entire liquidity on $\left[1500.62, 1509.65\right)$.

\item \textbf{Tick 2: $\left[1509.65,1518.73\right)$}. Trader \textbf{C} still needs to purchase 6.67 ETH at the next tick level (where the depth is 10 ETH). The liquidity level at this tick is $L_A+L_B$, that is the sum of liquidity provided by \textbf{A} and \textbf{B}. To remove the 6.67 ETH from the pool, he needs to deposit $\Delta y_2$, where
\begin{equation}
    \left(10-6.67+\frac{L_A+L_B}{\sqrt{1518.73}}\right)\left(0+\Delta y_2 + \left(L_A+L_B\right)\sqrt{1509.65}\right)=\left(L_A+L_B\right)^2.
\end{equation}
It follows that trader \textbf{C} purchases 6.67 ETH by depositing $\Delta y_2=10089.12$ USDT, at an average price of 1512.61. The pool price is updated as the ratio of virtual reserves:
\begin{equation}
    p^{\prime\prime}=\frac{\Delta y_2 +\left(L_A+L_B\right)\sqrt{1509.65}}{10-6.67+\frac{L_A+L_B}{\sqrt{1518.73}}}=1515.7.
\end{equation}
The updated price is in between the two liquidity ticks, since not all depth on this tick level was exhausted in the trade. Following the swap, liquidity on the tick range $\left[1509.65,1518.73\right)$ is composed of both assets: that is 10089.12 USDT and 10-6.67=3.33 ETH. 

Finally, trader $C$ pays 100.89 USDT as liquidity fees (1\% of the trade size), which are distributed to \textbf{A} and \textbf{B} proportionally to their liquidity share. That is, \textbf{A} receives a fraction $\frac{L_A}{L_A+L_B}$ of the total fee (33.57 USDT), whereas \textbf{B} receives 67.32 USDT.

\end{enumerate}

Figure \ref{fig:swap_pool} illustrates the impact of the swap. Within tick $\left[1500.62,1509.65\right)$, \textbf{A} sells 3.33 ETH and buys 5026 USDT. Unlike on limit order books, the execution does not remove liquidity from the book. Rather, $\textbf{A}$'s capital is converted from one token to another and remains available to trade. This feature underscores the passive nature of liquidity supply on decentralized exchanges. Mapping the concepts to traditional limit order book, this mechanism would imply that every time a market maker's sell order is executed at the ask, a buy order would automatically be placed on the bid side of the market.

The final price of $1517.70$ lies within the tick $\left[1509.65,1518.73\right)$, rather than on its boundary. Trader \textbf{C} only purchases 6.66 ETH out of 10 ETH available within this price interval. The implication is that liquidity on $\left[1509.65,1518.73\right)$ contains both tokens: 3.33 ETH (the amount that was not swapped by \textbf{C}) as well as 10089.12 USDT that \textbf{C} deposited in the pool. 

% Figure environment removed

The bottom panel of Figure \ref{fig:swap_pool} shows the price impact of the swap. From equation (6.15) in the Uniswap V3 white paper, we can solve for the price within tick $\left[p_\text{min},p_\text{max}\right)$ with liquidity $L$, following the execution of a buy order of size $x$:
\begin{equation}
    p\left(x\right)=\frac{p_\text{min} L^2}{\left(L-\sqrt{p_\text{min}}x\right)^2}.
\end{equation}
As expected, the price impact of a swap decreases in the liquidity available $L$ -- each ETH unit purchased by \textbf{C} has a smaller impact on the price once tick 1509.65 is crossed and the market becomes deeper.



\newpage

\section{Notation summary\label{sec:Variable-Definitions}}
%\input{notation_table}

\onehalfspacing
\noindent
\begin{center}
\begin{tabular}{@{}ll@{}}
\toprule
\cmidrule{1-2}

\multicolumn{2}{c}{\textbf{Variable Subscripts}}\\
\cmidrule{1-2}
Subscript & Definition \\
\cmidrule{1-2}
\textbf{T} and \textbf{N} & Pertaining to the token and numeraire assets, respectively. \\
\textbf{L} and \textbf{H} & Pertaining to the low- and high-fee pool, respectively. \\
\textbf{LP} & Pertaining to liquidity providers. \\
\textbf{LT} & Pertaining to liquidity traders. \\
\cmidrule{1-2}




\multicolumn{2}{c}{\textbf{Exogenous Parameters}}\\
\cmidrule{1-2}
Parameters & Definition \\
\cmidrule{1-2}
$v$, $\Delta$ & Common and private values of the token. \\
$\ell$, $h$ & Liquidity fee on the low- and high-fee pool. \\
$q_i$ & Token endowment of liquidity provider $i$. \\
$\varphi\left(q,Q\right)$ & Distribution density of $\textbf{LP}$ endowment $q$. \\
$Q$ & Maximum token endowment for $\textbf{LP}$s. \\
$\Gamma$ & Gas price on the blockchain. \\
$\theta$ & Linear trading rate of small $\LT$s. \\
$\lambda$ & Poisson arrival rate of large $\LT$s. \\
\cmidrule{1-2}



\multicolumn{2}{c}{\textbf{Endogenous Quantities}}\\
\cmidrule{1-2}
Variable & Definition \\
\cmidrule{1-2}
$\mathcal{L}_k$ & Equilibrium liquidity supply on exchange $k\in\left\{\textbf{L},\textbf{H}\right\}$. \\
$\qmg^\star$ & Token endowment of the $\LP$ who is indifferent between pools. \\
$\underline{q}$ &  Lowest token endowment deposited on the market (from break-even condition). \\
$d_k$ & Duration of a liquidity cycle on exchange $k$. \\
$\pi_k$ & Expected liquidity provider profit on exchange $k$. \\

\bottomrule
\end{tabular}
\end{center}

\newpage


\section{Proofs \label{sec:proofs}}
%\singlespacing
\input{Appendix/proofs_rebalance}



\newpage
\section{Just-in-time liquidity \label{sec:app-jit}}
Just-in-time (JIT) liquidity is a strategy that leverages the transparency of orders on the public blockchains. If a liquidity provider observes an incoming large order that has not been processed by miners and it deems uninformed in the public mempool, it can conveniently re-arrange transactions and propose a sequence of actions to sandwich this trade as follows:
\begin{enumerate}
    \item Add a large liquidity deposit at block position $k$, at the smallest tick around the current pool price. 
    \item Let the trade at block position $k+1$ execute and receive liquidity fees.
    \item Remove or burn any residual un-executed liquidity at block position $k+2$.
\end{enumerate}
The mint size is optimally very large (i.e., of the order of hundred of millions USD for liquid pairs), such that the JIT liquidity provider effectively crowds out the existing liquidity supply and collects most fees for the trade. That is, the strategy is made possible by pro-rata matching on decentralized exchanges because with time priority, the JIT provider cannot queue-jump existing liquidity providers. Since the JIT liquidity provider does not want to passively provide capital, it removes any residual deposit immediately after the trade.

We identify JIT liquidity events by the following algorithm as in \citet{WanAdams2022}:
\begin{enumerate}
\item Search for mints and burns in the same block, liquidity pool, and initiated by the same wallet address. The mint needs to occur exactly two block positions ahead of the burn (at positions $k$ and $k+2$).
\item Classify the mint and the burn as a JIT event if the transaction in between (at position $k+1$) is a trade in the same liquidity pool.
\end{enumerate}
 
JIT events are rare in our sample, and account for less than 1\% of the traded volume on Uniswap v3. Further, more than half of them occur in a single pair - USDC-WETH, and in low-fee pools. The Uniswap Labs provides further discussions on the aggregate impact of JIT liquidity provision \href{https://uniswap.org/blog/jit-liquidity}{here}. Regarding the economic effects, JIT liquidity reduces price impact for incoming trades, but dilutes existing liquidity providers in the pro-rata markets, and can discourage liquidity supply in the long run. 



\newpage
\section{Impermanent loss measure \label{app:IL}}
We build our measure of impermanent loss in line with the definition of token reserves within a price range in the Uniswap V3 white paper \citep{Uniswapv3Core2021} and Section 4.1 in \citet{Heimbach2023}.

Consider a liquidity provider who supplies $L$ units of liquidity into a pool trading a token $x$ for a token $y$. The chosen price range is $\left[p_\ell, p_u\right]$ with $p_\ell<p_u$. Further, the current price of the pool is $p_0$. We are interested in computing the impermanent loss at a future point in time, when the price updates to $p_1$.

From \citet{Uniswapv3Core2021}, the actual amount of tokens $x$ and $y$ (``real reserves'') deposited on a Uniswap v3 liquidity pool with a price range $\left[p_\ell, p_u\right]$ to yield liquidity $L$ are functions of the current pool price $p$:
\begin{equation}\label{eq:reserves_t0}
    x\left(p\right)=\begin{cases}
        L \times \left(\frac{1}{\sqrt{p_\ell}}-\frac{1}{\sqrt{p_u}}\right) & \text { if } p\leq p_\ell \\
        L \times \left(\frac{1}{\sqrt{p}}-\frac{1}{\sqrt{p_u}}\right) & \text { if } p_\ell<p\leq p_u \\
        0 & \text { if } p > p_u
    \end{cases} \text{ and }     y\left(p\right)=\begin{cases}
        0 & \text { if } p\leq p_\ell \\
        L \times  \left(\sqrt{p}-\sqrt{p_\ell}\right) & \text { if } p_\ell<p\leq p_u \\
        L \times  \left(\sqrt{p_u}-\sqrt{p_\ell}\right) & \text { if } p > p_u.
    \end{cases}
\end{equation}

From equation \eqref{eq:reserves_t0}, the value of the liquidity position at $t=1$ is therefore
\begin{equation}
    V_\text{position}=p_1 x\left(p_1\right) + y\left(p_1\right)=\begin{cases}
        L p_1 \times \left(\frac{1}{\sqrt{p_\ell}}-\frac{1}{\sqrt{p_u}}\right) & \text { if } p_1\leq p_\ell \\
        L \times \left(2\sqrt{p_1}-\frac{p_1}{\sqrt{p_u}}-\sqrt{p_\ell}\right) & \text { if } p_\ell<p_1\leq p_u \\
         L \times  \left(\sqrt{p_u}-\sqrt{p_\ell}\right) & \text { if } p_1 > p_u.
    \end{cases} 
\end{equation}

Conversely, the value of a strategy where the liquidity provider holds the original token quantities and marks them to market at the updated price is
\begin{equation}
    V_\text{hold}=p_1 x\left(p_0\right) + y\left(p_0\right)=\begin{cases}
        L p_1 \times \left(\frac{1}{\sqrt{p_\ell}}-\frac{1}{\sqrt{p_u}}\right) & \text { if } p_0\leq p_\ell \\
        L \times \left(\frac{p_1+p_0}{\sqrt{p_0}}-\frac{p_1}{\sqrt{p_u}}-\sqrt{p_\ell}\right) & \text { if } p_\ell<p_0\leq p_u \\
         L \times  \left(\sqrt{p_u}-\sqrt{p_\ell}\right) & \text { if } p_0 > p_u.
    \end{cases} 
\end{equation}

The impermanent loss is then defined as the excess return from holding the assets versus providing liquidity on the decentralized exchange:
\begin{equation}
    \text{ImpermanentLoss}=\frac{V_\text{hold}-V_\text{position}}{V_\text{hold}}.
\end{equation}

Empirically, we follow \citet{Heimbach2023} and compute impermanent loss for ``symmetric'' positions around the current pool price, that is $p_\ell=p_0 \alpha^{-1}$ and $p_u= p_0\alpha$, with $\alpha>1$. We allow for a time lag of one hour between $p_0$ and $p_1$.

% \newpage
% \section{New model}
% Consider the following continuous time model of trade in a single token, \textbf{T}.  The expected value of the token, $v>0$, is common knowledge.  Two risk neutral trader types consummate trade in this market: a continuum of liquidity providers ($\LP$s) and a continuum of liquidity takers ($\LT$s). Trade occurs because 
  market participants have heterogeneous private values for the asset. In particular, liquidity providers have no private value for the token, while liquidity takers value the token at $v\left(1+{\cal I}\Delta\right)$.  Here  $\Delta>0$ are the gains from trade and ${\cal I}$ is an indicator that takes on the value of $1$ if the taker buys and $-1$ if the taker sells. In what follows for expositional simplicity, as in \citet{foucault2013liquidity}, we focus on a one sided market in which liquidity takers act as buyers. 
 
 Liquidity providers differ in their endowments of the token. Each provider $i$ can supply at most $q_i$  of the token, where $q_i$ follows a truncated Pareto distribution on $\left[1,Q\right]$.  So, 
\begin{equation}\label{eq:density}
    \varphi\left(q\right)=\Big(\frac{Q}{Q-1}\Big) \frac{1}{q^2} \; \; \text{ for } q\in\left[1,Q\right].
\end{equation}

\noindent The right skew of the Pareto distribution captures the idea that there are many  low-endowment liquidity providers such as retail traders, but few high-capital $\LP$s such as sophisticated quantitative funds.   Heterogeneity in $\LP$ size is captured by $Q$, where a larger $Q$ naturally corresponds to a larger dispersion of endowments. Given the endowment distribution, collectively $\LP$s  supply at most
\begin{equation}
   S  =   \int_1^Q q \varphi\left(q\right) \diff q = \frac{Q}{Q-1} \log Q
\end{equation}
tokens. 

\begin{comment}

Figure \ref{fig:distribution} illustrates the theoretical distribution of $\LP$s endowments for $Q\in\left\{3,4\right\}$. 

% Figure environment removed 

\end{comment}


There are two types of liquidity takers: small and large. Small $\LT$s  arrive at the market at constant rate $\theta \diff t$, and each demand one unit of the token. The large $\LT$ arrival time follows a Poisson process with rate $\lambda>0$. Conditional on arrival, a large $\LT$ demands $\Theta$ units of the token, where $\Theta>S$. That is,  the large $\LT$ liquidity demand exceeds the maximum liquidity supply.  Let $D_t$ denote aggregate liquidity demand.  Then,  
\begin{equation}
    \diff D_t = \theta \diff t + \Theta \diff J_t\left(\lambda\right),
\end{equation}
where $J_t\left(\lambda\right)$ is the Poisson arrival process.

Liquidity demanders and suppliers can interact in two liquidity pools in which token trade occurs against a num\'{e}raire asset (cash).   We assume that the terms of trade are fixed, so that all trades occur at the expected value of the token, $v$. We do this by assuming prices in both pools satisfy a linear bonding curve, so there is no price impact of trade.\footnote{In practice, most decentralized exchanges use convex bonding curves, for example constant product pricing.} So, for a pool with $T$ tokens and $N$ of the num\'{e}raire good, 
\begin{equation}\label{eq:bond_curve}
    vT + N = \text{constant}.
\end{equation}
%  It is well known that price impact costs can lead to order splitting and fragmentation. Fixing the terms of trade allows us to focus on how fees affect liquidity supply. 
%  \mz{I think this assumption is actually harmless. Small LTs have no mass, and therefore no price impact. Large LTs have to consume all available liquidity anyway. The one thing that the assumption buys us is that we do not need arbitrageurs to restore the price after a large trader or a sufficiently long sequence of small traders.}
% %\cp{Is it obvious that the equilibrium effect of price impact costs lead to OS and F?  We might need a cite or two here. }

We note in passing that allowing for a bonding curve for which trades have price impact (such as a constant product function) would not materially affect our results. This is because small LTs have no mass, and therefore no price impact, whereas large LTs need to consume all available liquidity. It is well known that price impact costs can lead to order splitting and fragmentation. Fixing the terms of trade allows us to focus on how fees affect liquidity supply. 

Fees are levied on liquidity takers  as a fraction of the value of the trade and distributed pro rata to liquidity providers.  The pools have different fees.  One pool charges a low fee, and one pool charges a high fee which we denote $\ell$ and $h$ respectively.  Specifically, to purchase $\tau$ units of the token on the low fee pool, the total cost to a taker is $\tau\left(v+\ell\right)$. The $\LP$s  in the pool receive $\tau\ell$ in fees. 
 

In addition, consistent with gas costs on Ethereum,  any interaction with a liquidity pool (for example, trading or managing liquidity)  incurs a fixed execution cost $\Gamma>0$. 
It is important to note that 
gas fees on decentralized exchanges differ from trading fees on traditional exchanges.  First, they are not set by individual exchanges to compete with each other, but are a common transaction cost \emph{across} trading venues. Second, they are levied on a per-order rather than per-share basis: this implies economies of scale for larger orders. Third, there is significant time variation in gas fees which will allow us to identify the impact of transaction costs on liquidity pool market shares.\footnote{Empirical properties of gas fees are exhibited in \citet{LeharParlour2021}, while \citet{CapponiJia2021}, consider traders who affect the gas price.} 

To ensure that both small and large liquidity traders participate in the market, we assume that the gains from trade are larger than the aggregate transaction cost, including the pool fee and the gas price. 
To rule out trivial cases, we further assume that $Q\ell-\Gamma>0$. The condition ensures that there are at least some liquidity providers who can earn positive expected profit on the low fee pool.

\begin{ass} The gains from trade are sufficiently large so that all liquidity takers participate in the market, and the fixed costs are sufficiently low so that both pools can attract liquidity. 

\begin{enumerate} 
\item [i.] Gains from trade are sufficiently large $\Delta > h + \frac{\Gamma}{v}$.
\item [ii.] The low fee pool can attract liquidity $Q\ell-\Gamma>0$
\end{enumerate}
\end{ass}


We follow \citet{foucault2013liquidity} and partition the continuous timeline into \emph{liquidity cycles}. A liquidity cycle starts with an empty pool (zero liquidity offered) which triggers $\LP$ token deposits and ends when incoming trades deplete the liquidity supply. The first liquidity cycle starts at $t=0$, and  the sequence of events within a cycle is as follows:  each liquidity provider deposits their token into the high fee pool or the low fee pool and pays the gas cost $\Gamma$, or withdraw from the market. The $\LP$s do not interact with the pool again until the their liquidity is consumed at some random time $\tilde{\tau}$ and the next cycle begins. Figure \ref{fig:timing} illustrates the model timing.


% Figure environment removed

\subsection{Equilibrium}\label{sec:liqprov}

First, consider the liquidity traders' decisions. Faced with pool sizes of $\mathcal{L}_\ell$ and $\mathcal{L}_h$ in the low and high pool respectively, they choose the pool which minimizes their trading costs.  Conditional on trading, the small liquidity takers choose the $\ell$ fee pool.  Thus, small traders arrive at the $\ell$ fee pool at rate $\theta$, and each trade one unit.  By assumption the large liquidity taker wants to trade more than the posted liquidity and exhausts both pools.  Thus, liquidity is consumed on the low fee pool at rate $\frac{\theta}{\mathcal{L}_\ell} \diff t$, while liquidity is only consumed on the high fee pool if a large trader arrives. Once each pool is empty, liquidity providers refill it and restart the cycle. Let $d_k, k=\ell,h$ denote the duration of a liquidity cycle on the low and high pool respectively. Then, the expected duration of a cycle on the low fee pool is 
\begin{align}
    d_\ell &=e^{-\lambda \frac{\LL}{\theta}} \frac{\LL}{\theta}+\int_{0}^{\frac{\LL}{\theta}} t\lambda e^{-\lambda t}\diff t =\frac{1}{\lambda}-\frac{1}{\lambda}e^{-\frac{\LL}{\theta}\lambda},
    \end{align}
\noindent while the expected duration on a high fee pool is $d_h=\frac{1}{\lambda}.$  Thus,


\begin{lem}\label{lem:duration}
 \setlength{\parskip}{0ex}
The expected duration of a liquidity cycle is shorter in the low fee pool than the high feel pool. Or, $d_\ell<d_h$.
\end{lem}

Lemma \ref{lem:duration} is intuitive.  A liquidity cycle on the high fee pool only ends with the arrival of a large trader.   Conversely, a liquidity cycle on the low fee pool ends either because a large liquidity taker arrived or the cumulative small liquidity trader orders exhaust the pool. 

Notice that the low fee pool provides liquidity for a large share of the order flow, as both small and large $\LT$s trade there. However, the liquidity providers on that pool earn a low fee per traded unit. Additionally, as the expected cycle duration is shorter, they need to manage their liquidity more often, which leads to larger gas costs per unit of time. Liquidity providers face a choice between the low and high fee pool or not participating in the market. Thus, an \LP of size $q_i$ chooses between:

\begin{equation}\label{eq:optimal_pool}
  \max \Big[\frac{q_i \ell - \Gamma}{d_\ell} , \; \frac{q_i h -\Gamma}{d_h}, \; 0\Big].
\end{equation}

\noindent First, consider the choice between pools.  Rearranging Equation \ref{eq:optimal_pool}, liquidity provider $i$ chooses the low-fee pool  if and only if
\begin{equation}\label{eq:cost_comparison}
    \left(d_h \ell - d_\ell h\right) q_i> \Gamma \left(d_h-d_\ell\right).
\end{equation}

Equation \ref{eq:cost_comparison} highlights the tradeoff between the expected fee revenue per unit time and the fixed cost of accessing the market. 
If $\frac{h}{d_h}>\frac{\ell}{d_\ell}$,  then  the expected liquidity fee per unit of time on the high fee pool is larger than that of the low fee pool, and  the left hand side of Equation \ref{eq:cost_comparison} is negative.  From Lemma \ref{lem:duration}, the expected duration is higher on the high fee pool and the right hand side is always positive.  In this case,  all liquidity providers choose the high fee pool.
The more natural case to consider is if $\frac{h}{d_h}<\frac{\ell}{d_\ell}$ so that 
 liquidity providers face a trade-off between a higher liquidity fee per unit of time on the low fee pool against lower gas costs on the high fee pool. Clearly, the larger a liquidity providers' endowment, the more important is the fee revenue.  We have

 \begin{lem}\label{lem:sort} For any liquidity pools, $\{\mathcal{L}_\ell, \mathcal{L}_h\}$, for which $\frac{h}{d_h}<\frac{\ell}{d_\ell}$, if a liquidity provider of size $q$ prefers the low fee pool, then any liquidity provider with a larger endowment,  $\widetilde{q}>q$, also prefers the low fee pool.
 \end{lem}

 Now consider the choice of participating in the market.  An agent  only provides liquidity if she is able to break even on the high-fee pool -- that is, if her endowment $q_i$ is large enough. The participation constraint follows from equation \eqref{eq:optimal_pool}:
\begin{equation}\label{eq:pc}
    q_i h - \Gamma \geq 0.
\end{equation}

Define $\underline{q}=\frac{\Gamma}{h}$.  Recall, that the lower bound of the Pareto distribution is 1.  Thus, if $\frac{\Gamma}{h}>1$,  all $\LP$s with $q_i>\underline{q}$ enter the market, and the marginal entrant earns zero expected profit. Conversely, if $\frac{\Gamma}{h}\leq 1$,  liquidity provision is not competitive; all $\LP$s enter the market and earn strictly positive profits.
 
 

Following Lemma \ref{lem:sort}, let $\qmg$  be the threshold endowment such that all $\LP$s with $q_i>\qmg$ post liquidity on the low-fee pool and all $\LP$s with $q_i\leq \qmg$ choose the high-fee pool. 
This allows us to characterize pool sizes
$\LL$ and $\LH$ as a function of the endowment for the marginal $\LP$:
\begin{align}\label{eq:liquidity_levels}
    \LL&=\int_{\qmg}^Q q_i \varphi\left(q_i\right) \diff i = \frac{Q}{Q-1}\left(\log Q - \log \qmg \right)  \text{ and }\nonumber \\
    \LH&=\int_{\underline{q}}^{\qmg}  q_i \varphi\left(q_i\right) \diff i = \frac{Q}{Q-1}\left(\log \qmg - \log \underline{q}\right)
\end{align}
From equations \eqref{eq:cost_comparison} through \eqref{eq:liquidity_levels} it follows that the expected profit difference between the low- and high-fee pools can be written as an increasing function of the marginal $\LP$'s endowment, that is
\begin{align}\label{eq:pi_diff}
    \pi_\ell-\pi_h &=\frac{1}{\lambda d_h d_\ell}\left[\exp\left(-\frac{\lambda}{\theta} \LL\right) \underbrace{\left(q_i h - \Gamma\right)}_{>0} - q_i\left(h-\ell\right)\right] \nonumber \\
        &=\frac{1}{\lambda d_h d_\ell}\left[\qmg^{\frac{\lambda}{\theta}\frac{Q}{Q-1}} \times Q^{-\frac{\lambda}{\theta}\frac{Q}{Q-1}} \underbrace{\left(q_i h - \Gamma\right)}_{>0}- q_i\left(h-\ell\right)\right]. 
\end{align}
 Proposition \ref{prop:equilibria} characterizes the equilibrium liquidity provision.

\begin{prop}\label{prop:equilibria}
%\begin{leftbar} \setlength{\parskip}{0ex}
%\emph{(Fragmentation)} 
\begin{enumerate}
    \item [i.]
If $\frac{h-l}{h} Q^{\frac{\lambda}{\theta}\frac{Q}{Q-1}}<1$ and $\frac{\Gamma}{h}<1-\frac{h-l}{h} Q^{\frac{\lambda}{\theta}\frac{Q}{Q-1}}$, then  all $\LP$s deposit liquidity on the low fee pool. 
\item[ii.]If $\Gamma>Q\ell$, then all $\LP$s deposit liquidity on the high fee pool. 
\item [iii.] Otherwise, there exists a unique fragmented equilibrium characterized by marginal trader $\qmg^\star$ which solves
\begin{equation}\label{eq:mg_eq}
    \qmg^\star = \Gamma \frac{\qmg^{\frac{\lambda}{\theta}\frac{Q}{Q-1}} \times Q^{-\frac{\lambda}{\theta}\frac{Q}{Q-1}}}{h\left[\qmg^{\frac{\lambda}{\theta}\frac{Q}{Q-1}} \times Q^{-\frac{\lambda}{\theta}\frac{Q}{Q-1}}\right]-\left(h-\ell\right)} \in \left[\underline{q},Q\right]
\end{equation}
such that all $\LP$s with $q_i\leq \qmg^\star$ deposit liquidity in the high fee pool and all $\LP$s with $q_i>\qmg^\star$ choose the low fee pool.
\end{enumerate}
%\end{leftbar}    
\end{prop}

Figure \ref{fig:region_equilibrium} illustrates the equilibrium regions in Proposition \ref{prop:equilibria}. If the gas price is low and $\LP$s are homogeneous (low $\Gamma$ and $Q$), then all liquidity providers choose the low fee pool because  managing liquidity is relatively cheap. If more high-endowment $\LP$s enter the market (i.e., there is an increase in $Q$) the more liquidity is posted the low fee pool.  Keeping the $\LT$ arrival rate fixed, a larger pool depletes more slowly and thus the liquidity cycle on the pool becomes longer. As a consequence, the liquidity fee per unit of time on pool $L$ drops and smaller $\LP$s switch to the high-fee pool. As a result, liquidity becomes fragmented across the two pools. 

% Figure environment removed

An equilibrium in which all liquidity consolidates on the high fee pool is only sustainable for very high gas costs $\Gamma>Q\ell$. In this case,  none of the $\LP$s breaks even on pool $L$. For intermediate values of gas price, both pools co-exist  with positive market share.

Proposition \ref{cor:comp_stat_ms} establishes comparative statics for the two pools' liquidity market shares. From equation \eqref{eq:liquidity_levels}, we can compute the liquidity market share of the low-fee pool at the beginning of each cycle as
\begin{equation}
w_\ell=\frac{\LL}{\LL+\LH}=\frac{\log Q - \log \qmg^\star}{\log Q - \log \underline{q}}.
\end{equation}

\begin{prop}\label{cor:comp_stat_ms}
%\begin{leftbar} \setlength{\parskip}{0ex}
 In equilibrium, the market share of the low fee pool $w_\ell$
\begin{itemize}
    \item[i.] decreases in the gas cost ($\Gamma$), the arrival rate of large trades ($\lambda$), and the fee on pool $H$ (h).
    \item[ii.] increases in the fee on pool $L$ ($\ell$) and the arrival rate of small trades ($\theta$)
\end{itemize}
%\end{leftbar}    
\end{prop}

The results in Proposition \ref{cor:comp_stat_ms} are intuitive. The market share of the low fee pool increases if the fee gap $h-\ell$ is narrower, since this reduces $\LP$ incentives to switch to the high-fee pool. If the small $\LT$ arrival rate is large, then liquidity cycles in the low-fee pool are shorter, increasing the revenue per unit of time and consequently the market share of pool $L$. Conversely, if large trades arrive more often (high $\lambda$), then the high fee pool attracts a higher share of incoming order flow and becomes more appealing for liquidity providers.

Figure \ref{fig:liqshares} shows that the market share of the low fee pool (weakly) decreases in the gas cost $\Gamma$. A larger gas price increases the costs of active liquidity management, everything else equal, and incentivizes smaller $\LP$s to switch from the low fee pool to the high fee pool, since the latter has a lower turnover. For $\Gamma\leq h$, any increase in gas costs leads to a \emph{redistribution} of liquidity from one pool to another; the aggregate liquidity across both pools is constant since all $\LP$s participate in the market.

% Figure environment removed

If gas prices increase beyond a threshold ($\Gamma>h$), then the aggregate liquidity falls since $\LP$s with $q_i<\frac{\Gamma}{h}$ are shut out of the market. Both the low and high fee pool experience a decrease in liquidity deposits. However, the liquidity drop is sharper for the low fee pool which further depresses its market share.

\subsection{Pool fragmentation and market quality}


In our model, liquidity takers incur two main costs beyond gas prices: pool fees, which represent the cost of taking liquidity, as well as foregone gains from trade if liquidity providers do not fully participate in the market. To integrate these costs into a single measure of market quality, we use an \emph{implementation shortfall} metric, following \citet{Perold_1988}. If the asset is traded on a sequence of pools, where $f_k$ and $\mathcal{L}_k$ represent the fees and liquidity deposits on pool $k$, respectively, the implementation shortfall (IS) for a large $\LT$ is defined as
\begin{equation}\label{eq:IS_general}
    \text{IS}\left(\left\{f_k\right\}_k\right)=\underbrace{\sum_k f_k \mathcal{L}_k}_\text{trading fees} + \underbrace{\Delta \left(\Theta - \sum_k \mathcal{L}_k\right)}_\text{unrealized gains},
\end{equation}
where $\Theta - \sum_k \mathcal{L}_k$ is the difference between the $\LT$ trading demand and the cumulative liquidity available across all pools, and  $\Delta$ is  the per-unit gains from trade.

Suppose an asset is traded on a single pool that imposes a liquidity fee $f$. From equation \eqref{eq:IS_general} it follows that the implementation shortfall on this pool is:
\begin{align}
    \text{IS}\left(\left\{f\right\}\right)&= f \int_{\frac{\Gamma}{f}}^{Q}  q_i \varphi\left(q_i\right) \diff i + \Delta \left(\Theta - \int_{\frac{\Gamma}{f}}^{Q}  q_i \varphi\left(q_i\right) \diff i\right).
\end{align}
Here, the magnitude of the liquidity fee drives the trade-off between the participation of liquidity providers (\(\LP\)) and trading costs. A lower fee \( f \) results in fewer \(\LP\)s offering liquidity, leading to increased unrealized gains. In contrast, a higher fee increases trading costs, potentially outweighing the benefits of increased \(\LP\) participation. 

The trade-off is illustrated in the left panel of Figure \ref{fig:is}. The optimal fee \( f^\star \) that minimizes the single-pool implementation shortfall is equal to \( f^\star = g W^{-1}\left({\rm e}\frac{g Q}{\Gamma }\right) \), where \( W(\cdot) \) represents the Lambert function.

\begin{prop}\label{prop:optimality}
For any single-pool fee $f\geq 0$, there exists a set of fees \( \{h, \ell\} \) for a two-pool fragmented market, where \( h = f \) and \( h > \ell \), that guarantees an equal or reduced implementation shortfall in a fragmented market compared to the single-pool market. \\ Furthermore, when $f>\frac{\Gamma}{Q}$, the fee structure \( \{h, \ell\} \) can be chosen to ensure a strictly lower implementation shortfall in the fragmented market.
\end{prop}


Proposition \ref{prop:optimality} suggests that fragmentation with multiple fee levels improves market quality. Specifically, it is always possible to devise a fee structure in a fragmented market that yields a (weakly) lower implementation shortfall than a single-fee market. The logic is as follows: First, the highest fee in the fragmented market is set equal to the single pool fee, ensuring that the marginal $\LP$ participating the market is the same across both scenarios (i.e., the $\LP$ with endowment $\underline{q}=\frac{\Gamma}{f}$). This condition guarantees the same level of realized gains from trade in fragmented and non-fragmented markets. Second, a lower fee is then chosen for another pool to attract liquidity providers with higher token endowments, resulting in lower trading costs. This combination of reduced trading costs and unchanged gains from trade leads to a lower implementation shortfall in a fragmented market. 

% Figure environment removed

\begin{cor}\label{cor:optimality}
There exists a fee menu in a two-pool fragmented market structure that achieves an equal or lower implementation shortfall than on any single-fee pool configuration. Further, the fee menu can be chosen to achieve strictly lower implementation shortfall in fragmented markets if $f^\star=g W^{-1}\left({\rm e}\frac{g Q}{\Gamma }\right)>\frac{\Gamma}{Q}$.
\end{cor}

Corollary \ref{cor:optimality} emerges as a particular case of Proposition \ref{prop:optimality}, with the assumption that the single-pool operates at its optimal fee level. In essence, if a fragmented fee structure can be designed to achieve a lower implementation shortfall compared to an arbitrary single-pool fee, then a fee structure that dominates the optimally set single-pool fee achieves a lower implementation shortfall than any single-fee pool. 

The right panel of Figure \ref{fig:is} illustrates the scenario where the lower fee in a fragmented market is set to half the optimal fee of a single pool, denoted as \( \ell = \frac{1}{2}f^\star \). When gas prices are sufficiently low, $\LP$s with large endowments choose to provide liquidity on the low-fee pool, leading to reduced trading costs and, consequently, a lower implementation shortfall. As gas prices increase, the implementation shortfall rises in both single-pool and fragmented markets, primarily because more $\LP$s are priced out which results in higher unrealized gains. Additionally, the shortfall increases more rapidly with rising gas prices in the fragmented market as $\LP$s switch over from the low- to the high-fee pool --- that is, the result in Proposition \ref{cor:comp_stat_ms}. At very high gas prices, all $\LP$s converge in the high-fee pool, effectively collapsing the fragmented market into a single pool with the optimal fee level.








\subsection{Model implications and empirical predictions}

\begin{pred}\label{pred:comp_stat_Gamma}
The liquidity market share of the low-fee pool  decreases in the gas fee $\Gamma$.
\end{pred}

Prediction \ref{pred:comp_stat_Gamma} follows directly from Proposition \ref{cor:comp_stat_ms} and Figure \ref{fig:liqshares}. A higher gas price increases the fixed cost of active liquidity management, particularly so for smaller liquidity providers. In response, $\LP$s with lower endowments migrate to the high-fee pool  where they trade less often. 

\begin{pred}\label{pred:clienteles}
$\LP$s on the low-fee pool make larger liquidity deposits than $\LP$s on the high-fee pool.
\end{pred}

Prediction \ref{pred:clienteles} follows from the equilibrium discussion in Section \ref{sec:liqprov}. Liquidity providers with large token endowments ($q_i>\qmg$) deposit them in the low-fee pool since they are better positioned to actively manage liquidity due to economies of scale. $\LP$s with lower endowments ($q_i\leq\qmg$) either stay out of the market or choose pool $H$ which allows them to offer liquidity in a more passive manner.

% Figure environment removed

Figure \ref{fig:theory_liqsupply} illustrates this prediction through a Monte Carlo simulation. We plot the equilibrium liquidity supply decisions of 100,000 $\LP$s with endowments drawn from density \eqref{eq:density} and $Q=3$. The top panel highlights three groups of liquidity providers: low-endowment $\LP$s (in green) that are being rationed out of the market due to high gas cost, medium-endowment $\LP$s (blue) that deposit liquidity on pool $H$, and high-endowment $\LP$s (orange) that choose the low-fee pool $L$. 



\begin{pred}\label{pred:trade_size_volume}
The average trade size is higher on pool $H$ than on pool $L$. At the same time, trading volume is higher on pool $L$ than on pool $H$.
\end{pred}

Next, Prediction \ref{pred:trade_size_volume} deals with differences between incoming trades on the two liquidity pools. For a wide range of model parameters, incoming order flow on the low fee pool consists of a large number of small trades, and occasional large trades. In contrast, there are few trades on the high fee pool, but they are all relatively large. The model can therefore reconcile two apparently conflicting patterns: one liquidity pool captures most of the trading volume, while the largest trades are executed on the competitor. Figure \ref{fig:theory_trade} illustrates the prediction through a Monte Carlo simulation of the model.


% Figure environment removed


\begin{pred}\label{pred:clienteles_cs}
The average liquidity deposit on both the low- and- high fee pool increases with gas costs.
\end{pred}

An increase in the gas cost $\Gamma$ has two effects: first, the $\LP$s with the lowest endowments on pool $L$ switch to pool $H$. As a result, the average deposit on pool $L$ increases. Second, the $\LP$s with low endowments on pool $H$ may leave the market. Both channels translate to a higher average deposit on pool $H$, which experiences an inflow (outflow) of relatively high (low) endowment $LP$ following an increase in gas costs.



The bottom left panel highlights the clientele effect: that is, the average deposit is higher on pool $L$. Due to the skew of the Pareto distribution, however, there are more $\LP$ accounts active on pool $H$ than on pool $L$ for a wide range of parameter values.



\begin{pred}\label{pred:updates}
$\LP$s update liquidity more frequently on the low-fee than on the high-fee pool.
\end{pred}

Prediction \ref{pred:updates} is a consequence of Lemma \ref{lem:duration}. Liquidity cycles on pool $L$ are shorter than on pool $H$, since $\LP$s on the low-fee pool trade against both small and large orders, rather than only against large orders on the high-fee pool. Consequently, we expect $\LP$s on pool $L$ to actively manage their liquidity positions.


\begin{pred}\label{pred:updates_gas}
A larger gas price leads to more frequent liquidity updates on the low-fee pool.
\end{pred}
An increase in gas price leads to some $\LP$s switching from the low- to the high-fee pool. As a result, liquidity supply on the low-fee pool drops, leading to a shorter cycle as incoming order flow depletes the pool at a faster pace.



\end{document}