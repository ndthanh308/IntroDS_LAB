\subsection{Sample construction \label{sec:sample}}

We obtain data from the Uniswap V3 Subgraph, covering all trades, liquidity deposits (referred to as ``mints''), and liquidity withdrawals (referred to as ``burns'') on 4,069 Uniswap v3 pools. The data spans from the protocol's launch on May 4, 2021, up until July 15, 2023. Each entry in our data includes a transaction hash that uniquely identifies each trade and liquidity update on the Ethereum blockchain. Additionally, it provides details such as trade price, direction, and quantity, along with quantities and price ranges for each liquidity update. Moreover, the data also includes wallet addresses associated with initiating each transaction, akin to anonymous trader IDs. The Subgraph data we obtained also provides USD-denominated values for each trade and liquidity mint. We further collect daily pool snapshots from the Uniswap V3 Subgraph, including the end-of-day pool size in Ether and US Dollars, and summary price information (e.g., open, high, low, and closing prices for each pool).

To enhance our dataset, we combine the Subgraph data with public Ethereum data available on \href{https://cloud.google.com/blog/products/data-analytics/ethereum-bigquery-public-dataset-smart-contract-analytics}{Google Big Query} to obtain the position of each transaction in its block, as well as the gas price limit set by the trader and the amount of gas used. 

%Finally, to enrich our analysis, we use the Kaiko ``Cross-Price'' API to collect token prices in US dollars. This API provides us with volume-weighted average prices at minute and day frequencies for each token. The data includes prices from all major centralized and decentralized exchanges. In cases where a specific token did not have a direct trading pair against US dollars, Kaiko's methodology involved computing the benchmark price across the most liquid trading path. Typically, this involved converting the token to BTC or ETH first and then determining the price in US dollars.

% We obtain data from Kaiko on all trades and liquidity updates on 1,361 Uniswap v3 pools in 881 asset pairs from May 4, 2021 to September 15, 2022 -- that is, from the launch of the platform until Ethereum's transition to a Proof-of-Stake protocol.\footnote{There are virtually no restrictions to create a liquidity pool on Uniswap for any pair of ERC-20 tokens. Throughout our sample, a total of 8,167 Uniswap v3 pools were created for 6,535 asset pairs. Kaiko filters out most of these pools in their trade data set -- either because the trading volume is zero, or because tokens are labelled as untrustworthy (including scams or clones).}

There are no restrictions to list a token pair on Uniswap. Some pools might therefore be used for experiments, or they might include untrustworthy tokens. Following \citet{LeharParlour2021}, we remove pools that are either very small or that are not attracting an economically meaningful trading volume. We retain liquidity pools that fulfill the following four criteria: (i) have at least one interaction in more than 100 days in the sample, (ii) have more than 500 liquidity interactions throughout the sample, (iii) have an average daily liquidity balance in excess of US\$100,000, and (iv) capture more than 1\% of trading volume for a particular asset pair. 
We exclude burn events with zero liquidity withdrawal in both base and quote assets, as traders use them solely to collect fees without altering their liquidity position.


These basic screens give us a baseline sample of 274 liquidity pools covering 242 asset pairs, with combined daily dollar volume of \$1.12 billion and total value locked (i.e., aggregate liquidity supply) of \$2.53 billion as of July 15, 2023. We capture 24,202,803 interactions with liquidity pool smart contracts (accounting for 86.04\% of the entire universe of trades and liquidity updates). Trading and liquidity provision on Uniswap is heavily concentrated: the five largest pairs (USDC-WETH, WETH-USDT, USDC-USDT, WBTC-WETH, and DAI-USDC) account on average for 86\% of trading volume and 63\% of supplied liquidity.\footnote{WETH and WBTC stand for ``wrapped'' Bitcoin and Ether. Plain vanilla Bitcon and Ether are not compliant with the ERC-20 standard for tokens, and therefore cannot be directly used on decentralized exchanges' smart contracts. USDC (USD Coin), USDT (Tether), and DAI are stablecoins meant to closely track the US dollar.}


\subsection{Liquidity fragmentation patterns}

For 32 out of the 242 asset pairs in our baseline sample, liquidity supply is fragmented across two pools with different fees -- either with 1 and 5 bps fees (5 pairs), 5 and 30 bps fees (6 pairs), or 30 and 100 bps fees (21 pairs).\footnote{In some cases, more than two pools are created for a pair -- e.g., for USDC-WETH there are four pools with 1, 5, 30, and 100 bps liquidity fees. In all cases however, two pools heavily dominate the others: As described in Section \ref{sec:sample} we filter out small pools with less than 1\% volume share or less than \$100,000 liquidity deposits.} Despite being fewer in number, fragmented pairs are economically important: they account on average for 95\% of the capital committed to Uniswap v3 and for 93\% of its dollar trading volume. All major token pairs such as WETH-USDC, WETH-USDT, or WBTC-WETH trade on fragmented pools.

For each fragmented liquidity pair, we label the \emph{low} and the \emph{high} fee liquidity pool to facilitate analysis across assets. For example, the low and high liquidity fees for USDC-WETH are 5 and 30 bps, respectively, but only 1 and 5 bps for a lower volatility pair such as USDC-USDT. We refer to non-fragmented pools as \emph{single} (i.e., the unique pool for an asset pair).

We aggregate all interactions with Uniswap smart contracts into a panel across days and liquidity pools. To compute the end-of-day pool size, we account for all changes in token balances, across all price ranges. There are three possible interactions: A deposit or ``mint'' adds tokens to the pool, a withdrawal or ``burn'' removes tokens, whereas a trade or ``swap'' adds one token and removes the other. We track these changes across to obtain daily variation in the quantity of tokens on each pool. We obtain dollar values for the end-of-day liquidity pool sizes, intraday trade volumes, and liquidity events from the Uniswap V3 Subgraph. To determine a token's price in dollars, the Subgraph  searches for the most liquid path on Uniswap pools to establish the token's price in Ether and subsequently converts the Ether price to US dollars.

Table \ref{tab:sumstat} reports summary statistics across pools with different fee levels. High-fee pools attract on average 58\% of total liquidity supply, significantly more than their low-fee counterparts (\$46.50 million and \$33.78 million, respectively), but only capture 20.74\% percent of the trading volume (computed as $\nicefrac{8,071.24}{(8,071.24+30,848.79)}$ from the first column of Table \ref{tab:sumstat}). Consistent with our theoretical predictions, low-fee pools attract five times as many trades as high-fee competitors (610 versus 110 average trade count per day).  At the same time, the average trade on a high-fee pool is twice as large (\$14,490) than on a low-fee pool (\$6,340). 

The distribution of mint sizes is heavily skewed to the right, with 6.6\% of deposits exceeding \$1 million. There are large differences across pools -- the median $\LP$ deposit on the low-fee pool is \$15,680, twice as much as the median deposit on the high-fee pool (\$7,430). At the same time, the number of liquidity providers on high-fee pools is 51\% higher than on low-fee pools (10.08 unique addresses per day on high-fee pools versus only 6.68 unique address on high-fee pools).




% Table created by stargazer v.5.2.3 by Marek Hlavac, Social Policy Institute. E-mail: marek.hlavac at gmail.com
% Date and time: Fri, Oct 21, 2022 - 4:01:40 PM
\begin{table} \centering 
\caption{Descriptive statistics} 
  \label{tab:sumstat} 

\begin{minipage}[t]{1\columnwidth}%
\footnotesize
This table reports descriptive statistics for variables used in the empirical analysis. \emph{Pool size} is defined as the total value locked in the pool's smart contract at the end of each day. We compute the balance on day $t$ as follows: we take the balance at day $t-1$ and add (subtract) liquidity deposits (withdrawals) on day $t$, as well as accounting for token balance changes due to trades. The liquidity balance on the first day of the pool is taken to be zero. End of day balances are finally converted to US dollars. \emph{Daily volume} is computed as the sum of US dollar volume for all trades in a given pool and day. \emph{Liquidity share} (\emph{Volume share}) is computed as the ratio between a pool size (trading volume) for a given fee level and the aggregate size of all pools (trading volumes) for the same pair in a given day. \emph{Trade size} and \emph{Mint size} are the median trade and liquidity deposit size on a given pool and day, denominated in US dollars. \emph{Trade count} represents the number of trades in a given pool and day. \emph{\textbf{LP} wallets} counts the unique number of wallet addresses interacting with a given pool in a day.  The \emph{liquidity yield} is computed as the ratio between the daily trading volume and end-of-day TVL, multiplied by the fee tier. The \emph{price range} for every mint is computed as the difference between the top and bottom of the range, normalized by the range midpoint -- a measure that naturally lies between zero and two. The \emph{impermanent loss} is computed as in \citet{Heimbach2023} for a position in the range of 95\% to 105\% of the current pool price, with a forward-looking horizon of one hour. Finally, \emph{mint-to-burn} and \emph{burn-to-mint} times are defined as the time between a mint (burn) and a subsequent burn (mint) by the same address in the same pool, measured in hours. \emph{Mint-to-burn} and \emph{burn-to-mint} are recorded on the day of the final interaction with the pool. 
\end{minipage}

\vspace{0.05in}
\resizebox{0.95\textwidth}{!}{

\small
\begin{tabular}{@{\extracolsep{5pt}}llrrrrrr} 
\\[-1.8ex]\hline 
\hline \\[-1.8ex] 
Statistic & Pool fee & \multicolumn{1}{c}{Mean} & \multicolumn{1}{c}{Median} & \multicolumn{1}{c}{St. Dev.} & \multicolumn{1}{c}{Pctl(25)} & \multicolumn{1}{c}{Pctl(75)} & \multicolumn{1}{c}{N} \\ 
\hline \\[-1.8ex] 

Pool size (\$M) & Low & 33.78 & 2.05 & 96.91 & 0.30 & 14.12 & 20,151 \\   
& High & 46.50 & 3.85 & 95.73 & 1.43 & 27.51 & 20,151 \\ 
& Single & 3.89 & 0.84 & 13.56 & 0.26 & 2.62 & 130,767 \\ 

Liquidity share (\%) & Low & 39.52 & 35.52 & 32.53 & 7.37 & 72.16 & 20,151 \\ 
 & High & 60.48 & 64.48 & 32.53 & 27.84 & 92.63 & 20,151 \\ 

 
Daily volume (\$000) & Low & 30,848.79 & 619.77 & 118,908.80 & 6.18 & 5,697.30 & 20,151 \\  
& High & 8,071.24 & 114.96 & 36,777.38 & 7.83 & 1,882.12 & 20,151 \\  
& Single  & 915.73 & 36.07 & 6,059.78 & 1.93 & 277.00 & 130,767 \\    

Volume share & Low & 66.51 & 88.38 & 38.50 & 29.43 & 98.48 & 18,001 \\  
& High &  42.20 & 23.83 & 41.18 & 3.19 & 95.03 & 18,058 \\ 

Trade size (\$000) & Low & 6.34 & 2.20 & 13.36 & 0.61 & 6.03 & 18,001 \\ 
& High & 14.49 & 2.76 & 33.19 & 0.82 & 10.48 & 18,060 \\    
& Single    & 4.12 & 1.32 & 11.03 & 0.45 & 3.79 & 113,362 \\ 

Mint size (\$000) & Low & 820.84 & 15.68 & 13,114.83 & 3.78 & 58.98 & 10,640 \\  
& High & 1,001.10 & 7.43 & 13,807.10 & 1.55 & 30.52 & 10,370 \\ 
& Single & 96.97 & 6.93 & 622.12 & 1.42 & 30.39 & 45,300 \\ 

Trade count & Low & 610.61 & 95 & 1,518.52 & 12 & 414 & 20,151 \\   
& High & 110.59 & 26 & 490.29 & 8 & 89 & 20,151 \\  
& Single & 63.94 & 19 & 194.03 & 4 & 55 & 130,767 \\  

$\LP$ wallets & Low & 6.68 & 1 & 16.01 & 0 & 6 & 20,151 \\    
& High & 10.08 & 1 & 37.79 & 0 & 5 & 20,151 \\    
& Single  & 1.57 & 1.17 & 1.19 & 1.00 & 1.85 & 55,580 \\   

Liquidity yield (bps) & Low & 11.72 & 2.58 & 56.31 & 0.16 & 9.08 & 20,122 \\ 
& High & 9.69 & 1.65 & 51.44 & 0.15 & 6.40 & 20,130 \\
& Single & 17.90 & 1.94 & 90.18 & 0.18 & 8.58 & 130,433 \\

Price range & Low &  0.39 & 0.30 & 0.37 & 0.13 & 0.56 & 11,866 \\ 
& High & 0.61 & 0.54 & 0.44 & 0.32 & 0.84 & 12,195 \\    
& Single  & 0.68 & 0.58 & 0.52 & 0.27 & 1.02 & 55,580 \\ 

Impermanent loss (bps) & Low  & 8.46 & 1.84 & 27.88 & 0.06 & 7.23 & 20,118 \\  
& High & 7.37 & 1.33 & 27.21 & 0.05 & 5.93 & 20,132 \\  
& Single  & 17.20 & 2.44 & 71.34 & 0.17 & 11.37 & 130,340 \\  

Mint-to-burn (hrs) & Low & 450.40 & 59.82 & 1,341.67 & 19.70 & 243.83 & 10,186 \\  
& High & 952.14 & 165.61 & 2,076.42 & 39.66 & 711.67 & 9,979 \\   
& Single & 760.26 & 126.64 & 1,778.62 & 27.01 & 563.50 & 39,735 \\   

Burn-to-mint (hrs) & Low & 105.29 & 0.20 & 521.62 & 0.08 & 5.63 & 8,279 \\ 
& High & 224.26 & 0.32 & 941.31 & 0.10 & 27.78 & 7,289 \\ 
& Single & 177.74 & 0.23 & 803.40 & 0.07 & 20.64 & 27,477 \\  

\hline \\[-1.8ex] 

\end{tabular} 
}
\end{table} 

One concern with measuring average mint size is just-in-time liquidity provision (JIT). JIT liquidity providers submit very large and short-lived deposits to the pool to dilute competitors on an incoming large trade; they immediately withdraw the balance in the same block after executing the trade. In our sample, JIT liquidity provision is not economically significant, accounting for less than 1\% of aggregate trading volume. However, it has the potential to skew mint sizes to the right, particularly in low-fee pools, without providing liquidity to the market at large. We address this issue by (i) filtering out JIT mints using the algorithm in Appendix \ref{sec:app-jit} and (ii) taking the median liquidity mint size at day-pool level rather than the mean. 

Further, we follow \citet{augustin2022reaching} to compute the daily liquidity fee yield as the product between pool's fee tier and the ratio between trading volume and the lagged total value locked (TVL). That is,
\begin{equation}\label{eq:liq_yield}
    \text{Liquidity yield}=\text{liquidity fee}_i \times \frac{\text{Volume}_{i,t}}{\text{TVL}_{i,t-1}},
\end{equation}
for pool $i$ and day $t$. The average daily yield is slightly higher on low-fee pools, at 11.72 basis points, compared to 9.69 basis points on high-fee pools.

A salient observation in Table \ref{tab:sumstat} is that non-fragmented pairs (``single'' pools) are significantly smaller -- on average less than 10\% of the pool size and trading volume of fragmented pairs. Average trade and mint sizes are correspondingly lower as well. The evidence suggests that pairs for which there is significant trading interest, and therefore potentially a broader cross-section of potential liquidity providers, are more likely to become fragmented.

Figure \ref{fig:stat_liq} plots the distributions of our empirical measures across low- and high-fee liquidity pools. It suggest a sharp segmentation of liquidity supply and trading across pools. High-fee pools attract smaller liquidity providers by mint size, and end up with a larger \emph{aggregate} size than their low-fee counterparts. Trading volume is similarly segmented: most small value trades are executed on the cheaper low-fee pools, making up the majority of daily volume for a given pair. High-value trades, of which there are fewer, are more likely to (also) execute on high-fee pools.

% Figure environment removed




Our theoretical framework in Section \ref{sec:model} implies that liquidity suppliers manage their positions more actively in the low- than the high-fee pool. Figure \ref{fig:liq_cycles} provides suggestive evidence for liquidity cycles of different lengths in the cross-section of pools. Liquidity on decentralized exchanges is significantly more passive than on traditional equity markets. That is, liquidity providers do not often manage their positions at high frequencies. The median time from a mint (deposit) to a subsequent burn (withdrawal) from the same wallet on the same pool ranges from 59.82 hours, or 2.49 days, on low-fee pools to 165.61 hours, or 6.9 days on high-fee pools. 

When do $\LP$s re-balance their positions? In 53\% of cases, liquidity providers only withdraw tokens from the pool when their position exits the price range that allows them to collect fees. Concretely, $\LP$s specified price range for liquidity provision does not straddle the most recent reference price of the pool. The scenario mirrors a limit order market where a liquidity provider's outstanding limit orders are deep in the book, such that she doesn't stand to earn the spread on the marginal incoming trade. In this case, a rational market maker might want to cancel their outstanding order and place a new one at the top of the book. This is exactly the pattern we observe on Uniswap: the subsequent mint following a burn straddles the new price 77\% of the time -- $\LP$s reposition
their liquidity around the current prices to keep earning fees on incoming trades. Moreover, re-balancing is swift -- the median time between a burn and a subsequent mint is just 12 minutes (0.20 hours).

The empirical pattern in Figure \ref{fig:liq_cycles} echoes liquidity cycles as described in Section \ref{sec:model}. Liquidity providers deposit tokens in Uniswap pools and then patiently wait for days until incoming order flow exhausts their position (i.e., posted liquidity no longer earns fees). Once this happens, $\LP$s quickly re-balance their position in a matter of minutes -- by removing stale liquidity and adding a new position around the current price. The cycle is longer on high-fee pools for which trading volume is lower and liquidity takes longer to deplete.

Importantly, $\LP$s do not seem to ``race'' to update liquidity upon information arrival as in \citet{Budish2015TheResponse}. First, they very rarely manage their position intraday. Second, $\LP$s on Uniswap typically do not remove in-range liquidity that stands to trade first against incoming order flow and therefore bears the highest adverse selection risk. Our results are consistent with \citet{CapponiJia2021} who theoretically argue that $\LP$s have low incentives to manage liquidity on news arrival, as well as with \citet{CapponiJiaYu2022} who find no evidence of traders racing to trade on information on Uniswap v2.


% Figure environment removed

\paragraph{Measuring gas prices.} Each interaction with smart contracts on the Ethereum blockchain requires computational resources, measured in units of ``gas.'' Upon submitting a mint or burn transaction to the decentralized exchange, each liquidity provider specifies their willingness to pay per unit of gas, that is they bid a  ``gas price.'' Traders are likely to bid higher prices for more complex transactions or if they require a faster execution. To generate a conservative daily benchmark for the gas price, we compute the average of the lowest 1000 user gas bids for mint and burn interactions on day $t$, across all liquidity pools in the benchmark sample. %These transactions are more likely to be plain vanilla deposits or withdrawals of liquidity, capturing the cost of a simple interaction with the decentralized exchange smart contract.\al{Clarify: do we use gas prices or gas paid? Are there 1000 transactions per day? The gas price is independent of tx complexity. What about zero gas price tx due to MEV?}

Figure \ref{fig:gascosts} showcases the significant fluctuation in gas costs for Uniswap liquidity transactions over time. 
Gas costs denominated in USD are influenced by two primary factors: network congestion, which leads to variations in gas prices measured in Ether, and the fluctuation of Ether's value relative to the US dollar. On a monthly average, gas costs peaked at above US\$100 in November 2021 and have since plummeted to around US\$6 from the second half of 2022, albeit with occasional spikes.

% Figure environment removed