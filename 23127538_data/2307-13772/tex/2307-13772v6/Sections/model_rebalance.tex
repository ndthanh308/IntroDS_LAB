Consider the following continuous time model of trade in a single token, \textbf{T}.  The expected value of the token, $v>0$, is common knowledge.  Two risk neutral trader types consummate trade in this market: a continuum of liquidity providers ($\LP$s) and a continuum of liquidity takers ($\LT$s). Trade occurs because 
  market participants have heterogeneous private values for the asset. In particular, liquidity providers have no private value for the token, while liquidity takers value the token at $v\left(1+{\cal I}\Delta\right)$.  Here  $\Delta>0$ are the gains from trade and ${\cal I}$ is an indicator that takes on the value of $1$ if the taker buys and $-1$ if the taker sells. In what follows for expositional simplicity, as in \citet{foucault2013liquidity}, we focus on a one sided market in which liquidity takers act as buyers. 
 
 Liquidity providers differ in their endowments of the token. Each provider $i$ can supply at most $q_i$  of the token, where $q_i$ follows a truncated Pareto distribution on $\left[1,Q\right]$.  So, 
\begin{equation}\label{eq:density}
    \varphi\left(q\right)=\Big(\frac{Q}{Q-1}\Big) \frac{1}{q^2} \; \; \text{ for } q\in\left[1,Q\right].
\end{equation}

\noindent The right skew of the Pareto distribution captures the idea that there are many  low-endowment liquidity providers such as retail traders, but few high-capital $\LP$s such as sophisticated quantitative funds.   Heterogeneity in $\LP$ size is captured by $Q$, where a larger $Q$ naturally corresponds to a larger dispersion of endowments. Given the endowment distribution, collectively $\LP$s  supply at most
\begin{equation}
   S  =   \int_1^Q q \varphi\left(q\right) \diff q = \frac{Q}{Q-1} \log Q
\end{equation}
tokens. 

\begin{comment}

Figure \ref{fig:distribution} illustrates the theoretical distribution of $\LP$s endowments for $Q\in\left\{3,4\right\}$. 

% Figure environment removed 

\end{comment}


There are two types of liquidity takers: small and large. Small $\LT$s  arrive at the market at constant rate $\theta \diff t$, and each demand one unit of the token. The large $\LT$ arrival time follows a Poisson process with rate $\lambda>0$. Conditional on arrival, a large $\LT$ demands $\Theta$ units of the token, where $\Theta>S$. That is,  the large $\LT$ liquidity demand exceeds the maximum liquidity supply.  Let $D_t$ denote aggregate liquidity demand.  Then,  
\begin{equation}
    \diff D_t = \theta \diff t + \Theta \diff J_t\left(\lambda\right),
\end{equation}
where $J_t\left(\lambda\right)$ is the Poisson arrival process.

Liquidity demanders and suppliers can interact in two liquidity pools in which token trade occurs against a num\'{e}raire asset (cash).   We assume that the terms of trade are fixed, so that all trades occur at the expected value of the token, $v$. We do this by assuming prices in both pools satisfy a linear bonding curve, so there is no price impact of trade.\footnote{In practice, most decentralized exchanges use convex bonding curves, for example constant product pricing.} So, for a pool with $T$ tokens and $N$ of the num\'{e}raire good, 
\begin{equation}\label{eq:bond_curve}
    vT + N = \text{constant}.
\end{equation}
%  It is well known that price impact costs can lead to order splitting and fragmentation. Fixing the terms of trade allows us to focus on how fees affect liquidity supply. 
%  \mz{I think this assumption is actually harmless. Small LTs have no mass, and therefore no price impact. Large LTs have to consume all available liquidity anyway. The one thing that the assumption buys us is that we do not need arbitrageurs to restore the price after a large trader or a sufficiently long sequence of small traders.}
% %\cp{Is it obvious that the equilibrium effect of price impact costs lead to OS and F?  We might need a cite or two here. }

We note in passing that allowing for a bonding curve for which trades have price impact (such as a constant product function) would not materially affect our results. This is because small LTs have no mass, and therefore no price impact, whereas large LTs need to consume all available liquidity. It is well known that price impact costs can lead to order splitting and fragmentation. Fixing the terms of trade allows us to focus on how fees affect liquidity supply. 

Fees are levied on liquidity takers  as a fraction of the value of the trade and distributed pro rata to liquidity providers.  The pools have different fees.  One pool charges a low fee, and one pool charges a high fee which we denote $\ell$ and $h$ respectively.  Specifically, to purchase $\tau$ units of the token on the low fee pool, the total cost to a taker is $\tau\left(v+\ell\right)$. The $\LP$s  in the pool receive $\tau\ell$ in fees. 
 

In addition, consistent with gas costs on Ethereum,  any interaction with a liquidity pool (for example, trading or managing liquidity)  incurs a fixed execution cost $\Gamma>0$. 
It is important to note that 
gas fees on decentralized exchanges differ from trading fees on traditional exchanges.  First, they are not set by individual exchanges to compete with each other, but are a common transaction cost \emph{across} trading venues. Second, they are levied on a per-order rather than per-share basis: this implies economies of scale for larger orders. Third, there is significant time variation in gas fees which will allow us to identify the impact of transaction costs on liquidity pool market shares.\footnote{Empirical properties of gas fees are exhibited in \citet{LeharParlour2021}, while \citet{CapponiJia2021}, consider traders who affect the gas price.} 

To ensure that both small and large liquidity traders participate in the market, we assume that the gains from trade are larger than the aggregate transaction cost, including the pool fee and the gas price. 
To rule out trivial cases, we further assume that $Q\ell-\Gamma>0$. The condition ensures that there are at least some liquidity providers who can earn positive expected profit on the low fee pool.

\begin{ass} The gains from trade are sufficiently large so that all liquidity takers participate in the market, and the fixed costs are sufficiently low so that both pools can attract liquidity. 

\begin{enumerate} 
\item [i.] Gains from trade are sufficiently large $\Delta > h + \frac{\Gamma}{v}$.
\item [ii.] The low fee pool can attract liquidity $Q\ell-\Gamma>0$
\end{enumerate}
\end{ass}


We follow \citet{foucault2013liquidity} and partition the continuous timeline into \emph{liquidity cycles}. A liquidity cycle starts with an empty pool (zero liquidity offered) which triggers $\LP$ token deposits and ends when incoming trades deplete the liquidity supply. The first liquidity cycle starts at $t=0$, and  the sequence of events within a cycle is as follows:  each liquidity provider deposits their token into the high fee pool or the low fee pool and pays the gas cost $\Gamma$, or withdraw from the market. The $\LP$s do not interact with the pool again until the their liquidity is consumed at some random time $\tilde{\tau}$ and the next cycle begins. Figure \ref{fig:timing} illustrates the model timing.


% Figure environment removed

\subsection{Equilibrium}\label{sec:liqprov}

First, consider the liquidity traders' decisions. Faced with pool sizes of $\mathcal{L}_\ell$ and $\mathcal{L}_h$ in the low and high pool respectively, they choose the pool which minimizes their trading costs.  Conditional on trading, the small liquidity takers choose the $\ell$ fee pool.  Thus, small traders arrive at the $\ell$ fee pool at rate $\theta$, and each trade one unit.  By assumption the large liquidity taker wants to trade more than the posted liquidity and exhausts both pools.  Thus, liquidity is consumed on the low fee pool at rate $\frac{\theta}{\mathcal{L}_\ell} \diff t$, while liquidity is only consumed on the high fee pool if a large trader arrives. Once each pool is empty, liquidity providers refill it and restart the cycle. Let $d_k, k=\ell,h$ denote the duration of a liquidity cycle on the low and high pool respectively. Then, the expected duration of a cycle on the low fee pool is 
\begin{align}
    d_\ell &=e^{-\lambda \frac{\LL}{\theta}} \frac{\LL}{\theta}+\int_{0}^{\frac{\LL}{\theta}} t\lambda e^{-\lambda t}\diff t =\frac{1}{\lambda}-\frac{1}{\lambda}e^{-\frac{\LL}{\theta}\lambda},
    \end{align}
\noindent while the expected duration on a high fee pool is $d_h=\frac{1}{\lambda}.$  Thus,


\begin{lem}\label{lem:duration}
 \setlength{\parskip}{0ex}
The expected duration of a liquidity cycle is shorter in the low fee pool than the high feel pool. Or, $d_\ell<d_h$.
\end{lem}

Lemma \ref{lem:duration} is intuitive.  A liquidity cycle on the high fee pool only ends with the arrival of a large trader.   Conversely, a liquidity cycle on the low fee pool ends either because a large liquidity taker arrived or the cumulative small liquidity trader orders exhaust the pool. 

Notice that the low fee pool provides liquidity for a large share of the order flow, as both small and large $\LT$s trade there. However, the liquidity providers on that pool earn a low fee per traded unit. Additionally, as the expected cycle duration is shorter, they need to manage their liquidity more often, which leads to larger gas costs per unit of time. Liquidity providers face a choice between the low and high fee pool or not participating in the market. Thus, an \LP of size $q_i$ chooses between:

\begin{equation}\label{eq:optimal_pool}
  \max \Big[\frac{q_i \ell - \Gamma}{d_\ell} , \; \frac{q_i h -\Gamma}{d_h}, \; 0\Big].
\end{equation}

\noindent First, consider the choice between pools.  Rearranging Equation \ref{eq:optimal_pool}, liquidity provider $i$ chooses the low-fee pool  if and only if
\begin{equation}\label{eq:cost_comparison}
    \left(d_h \ell - d_\ell h\right) q_i> \Gamma \left(d_h-d_\ell\right).
\end{equation}

Equation \ref{eq:cost_comparison} highlights the tradeoff between the expected fee revenue per unit time and the fixed cost of accessing the market. 
If $\frac{h}{d_h}>\frac{\ell}{d_\ell}$,  then  the expected liquidity fee per unit of time on the high fee pool is larger than that of the low fee pool, and  the left hand side of Equation \ref{eq:cost_comparison} is negative.  From Lemma \ref{lem:duration}, the expected duration is higher on the high fee pool and the right hand side is always positive.  In this case,  all liquidity providers choose the high fee pool.
The more natural case to consider is if $\frac{h}{d_h}<\frac{\ell}{d_\ell}$ so that 
 liquidity providers face a trade-off between a higher liquidity fee per unit of time on the low fee pool against lower gas costs on the high fee pool. Clearly, the larger a liquidity providers' endowment, the more important is the fee revenue.  We have

 \begin{lem}\label{lem:sort} For any liquidity pools, $\{\mathcal{L}_\ell, \mathcal{L}_h\}$, for which $\frac{h}{d_h}<\frac{\ell}{d_\ell}$, if a liquidity provider of size $q$ prefers the low fee pool, then any liquidity provider with a larger endowment,  $\widetilde{q}>q$, also prefers the low fee pool.
 \end{lem}

 Now consider the choice of participating in the market.  An agent  only provides liquidity if she is able to break even on the high-fee pool -- that is, if her endowment $q_i$ is large enough. The participation constraint follows from equation \eqref{eq:optimal_pool}:
\begin{equation}\label{eq:pc}
    q_i h - \Gamma \geq 0.
\end{equation}

Define $\underline{q}=\frac{\Gamma}{h}$.  Recall, that the lower bound of the Pareto distribution is 1.  Thus, if $\frac{\Gamma}{h}>1$,  all $\LP$s with $q_i>\underline{q}$ enter the market, and the marginal entrant earns zero expected profit. Conversely, if $\frac{\Gamma}{h}\leq 1$,  liquidity provision is not competitive; all $\LP$s enter the market and earn strictly positive profits.
 
 

Following Lemma \ref{lem:sort}, let $\qmg$  be the threshold endowment such that all $\LP$s with $q_i>\qmg$ post liquidity on the low-fee pool and all $\LP$s with $q_i\leq \qmg$ choose the high-fee pool. 
This allows us to characterize pool sizes
$\LL$ and $\LH$ as a function of the endowment for the marginal $\LP$:
\begin{align}\label{eq:liquidity_levels}
    \LL&=\int_{\qmg}^Q q_i \varphi\left(q_i\right) \diff i = \frac{Q}{Q-1}\left(\log Q - \log \qmg \right)  \text{ and }\nonumber \\
    \LH&=\int_{\underline{q}}^{\qmg}  q_i \varphi\left(q_i\right) \diff i = \frac{Q}{Q-1}\left(\log \qmg - \log \underline{q}\right)
\end{align}
From equations \eqref{eq:cost_comparison} through \eqref{eq:liquidity_levels} it follows that the expected profit difference between the low- and high-fee pools can be written as an increasing function of the marginal $\LP$'s endowment, that is
\begin{align}\label{eq:pi_diff}
    \pi_\ell-\pi_h &=\frac{1}{\lambda d_h d_\ell}\left[\exp\left(-\frac{\lambda}{\theta} \LL\right) \underbrace{\left(q_i h - \Gamma\right)}_{>0} - q_i\left(h-\ell\right)\right] \nonumber \\
        &=\frac{1}{\lambda d_h d_\ell}\left[\qmg^{\frac{\lambda}{\theta}\frac{Q}{Q-1}} \times Q^{-\frac{\lambda}{\theta}\frac{Q}{Q-1}} \underbrace{\left(q_i h - \Gamma\right)}_{>0}- q_i\left(h-\ell\right)\right]. 
\end{align}
 Proposition \ref{prop:equilibria} characterizes the equilibrium liquidity provision.

\begin{prop}\label{prop:equilibria}
%\begin{leftbar} \setlength{\parskip}{0ex}
%\emph{(Fragmentation)} 
\begin{enumerate}
    \item [i.]
If $\frac{h-l}{h} Q^{\frac{\lambda}{\theta}\frac{Q}{Q-1}}<1$ and $\frac{\Gamma}{h}<1-\frac{h-l}{h} Q^{\frac{\lambda}{\theta}\frac{Q}{Q-1}}$, then  all $\LP$s deposit liquidity on the low fee pool. 
\item[ii.]If $\Gamma>Q\ell$, then all $\LP$s deposit liquidity on the high fee pool. 
\item [iii.] Otherwise, there exists a unique fragmented equilibrium characterized by marginal trader $\qmg^\star$ which solves
\begin{equation}\label{eq:mg_eq}
    \qmg^\star = \Gamma \frac{\qmg^{\frac{\lambda}{\theta}\frac{Q}{Q-1}} \times Q^{-\frac{\lambda}{\theta}\frac{Q}{Q-1}}}{h\left[\qmg^{\frac{\lambda}{\theta}\frac{Q}{Q-1}} \times Q^{-\frac{\lambda}{\theta}\frac{Q}{Q-1}}\right]-\left(h-\ell\right)} \in \left[\underline{q},Q\right]
\end{equation}
such that all $\LP$s with $q_i\leq \qmg^\star$ deposit liquidity in the high fee pool and all $\LP$s with $q_i>\qmg^\star$ choose the low fee pool.
\end{enumerate}
%\end{leftbar}    
\end{prop}

Figure \ref{fig:region_equilibrium} illustrates the equilibrium regions in Proposition \ref{prop:equilibria}. If the gas price is low and $\LP$s are homogeneous (low $\Gamma$ and $Q$), then all liquidity providers choose the low fee pool because  managing liquidity is relatively cheap. If more high-endowment $\LP$s enter the market (i.e., there is an increase in $Q$) the more liquidity is posted the low fee pool.  Keeping the $\LT$ arrival rate fixed, a larger pool depletes more slowly and thus the liquidity cycle on the pool becomes longer. As a consequence, the liquidity fee per unit of time on pool $L$ drops and smaller $\LP$s switch to the high-fee pool. As a result, liquidity becomes fragmented across the two pools. 

% Figure environment removed

An equilibrium in which all liquidity consolidates on the high fee pool is only sustainable for very high gas costs $\Gamma>Q\ell$. In this case,  none of the $\LP$s breaks even on pool $L$. For intermediate values of gas price, both pools co-exist  with positive market share.

Proposition \ref{cor:comp_stat_ms} establishes comparative statics for the two pools' liquidity market shares. From equation \eqref{eq:liquidity_levels}, we can compute the liquidity market share of the low-fee pool at the beginning of each cycle as
\begin{equation}
w_\ell=\frac{\LL}{\LL+\LH}=\frac{\log Q - \log \qmg^\star}{\log Q - \log \underline{q}}.
\end{equation}

\begin{prop}\label{cor:comp_stat_ms}
%\begin{leftbar} \setlength{\parskip}{0ex}
 In equilibrium, the market share of the low fee pool $w_\ell$
\begin{itemize}
    \item[i.] decreases in the gas cost ($\Gamma$), the arrival rate of large trades ($\lambda$), and the fee on pool $H$ (h).
    \item[ii.] increases in the fee on pool $L$ ($\ell$) and the arrival rate of small trades ($\theta$)
\end{itemize}
%\end{leftbar}    
\end{prop}

The results in Proposition \ref{cor:comp_stat_ms} are intuitive. The market share of the low fee pool increases if the fee gap $h-\ell$ is narrower, since this reduces $\LP$ incentives to switch to the high-fee pool. If the small $\LT$ arrival rate is large, then liquidity cycles in the low-fee pool are shorter, increasing the revenue per unit of time and consequently the market share of pool $L$. Conversely, if large trades arrive more often (high $\lambda$), then the high fee pool attracts a higher share of incoming order flow and becomes more appealing for liquidity providers.

Figure \ref{fig:liqshares} shows that the market share of the low fee pool (weakly) decreases in the gas cost $\Gamma$. A larger gas price increases the costs of active liquidity management, everything else equal, and incentivizes smaller $\LP$s to switch from the low fee pool to the high fee pool, since the latter has a lower turnover. For $\Gamma\leq h$, any increase in gas costs leads to a \emph{redistribution} of liquidity from one pool to another; the aggregate liquidity across both pools is constant since all $\LP$s participate in the market.

% Figure environment removed

If gas prices increase beyond a threshold ($\Gamma>h$), then the aggregate liquidity falls since $\LP$s with $q_i<\frac{\Gamma}{h}$ are shut out of the market. Both the low and high fee pool experience a decrease in liquidity deposits. However, the liquidity drop is sharper for the low fee pool which further depresses its market share.

\subsection{Pool fragmentation and market quality}


In our model, liquidity takers incur two main costs beyond gas prices: pool fees, which represent the cost of taking liquidity, as well as foregone gains from trade if liquidity providers do not fully participate in the market. To integrate these costs into a single measure of market quality, we use an \emph{implementation shortfall} metric, following \citet{Perold_1988}. If the asset is traded on a sequence of pools, where $f_k$ and $\mathcal{L}_k$ represent the fees and liquidity deposits on pool $k$, respectively, the implementation shortfall (IS) for a large $\LT$ is defined as
\begin{equation}\label{eq:IS_general}
    \text{IS}\left(\left\{f_k\right\}_k\right)=\underbrace{\sum_k f_k \mathcal{L}_k}_\text{trading fees} + \underbrace{\Delta \left(\Theta - \sum_k \mathcal{L}_k\right)}_\text{unrealized gains},
\end{equation}
where $\Theta - \sum_k \mathcal{L}_k$ is the difference between the $\LT$ trading demand and the cumulative liquidity available across all pools, and  $\Delta$ is  the per-unit gains from trade.

Suppose an asset is traded on a single pool that imposes a liquidity fee $f$. From equation \eqref{eq:IS_general} it follows that the implementation shortfall on this pool is:
\begin{align}
    \text{IS}\left(\left\{f\right\}\right)&= f \int_{\frac{\Gamma}{f}}^{Q}  q_i \varphi\left(q_i\right) \diff i + \Delta \left(\Theta - \int_{\frac{\Gamma}{f}}^{Q}  q_i \varphi\left(q_i\right) \diff i\right).
\end{align}
Here, the magnitude of the liquidity fee drives the trade-off between the participation of liquidity providers (\(\LP\)) and trading costs. A lower fee \( f \) results in fewer \(\LP\)s offering liquidity, leading to increased unrealized gains. In contrast, a higher fee increases trading costs, potentially outweighing the benefits of increased \(\LP\) participation. 

The trade-off is illustrated in the left panel of Figure \ref{fig:is}. The optimal fee \( f^\star \) that minimizes the single-pool implementation shortfall is equal to \( f^\star = g W^{-1}\left({\rm e}\frac{g Q}{\Gamma }\right) \), where \( W(\cdot) \) represents the Lambert function.

\begin{prop}\label{prop:optimality}
For any single-pool fee $f\geq 0$, there exists a set of fees \( \{h, \ell\} \) for a two-pool fragmented market, where \( h = f \) and \( h > \ell \), that guarantees an equal or reduced implementation shortfall in a fragmented market compared to the single-pool market. \\ Furthermore, when $f>\frac{\Gamma}{Q}$, the fee structure \( \{h, \ell\} \) can be chosen to ensure a strictly lower implementation shortfall in the fragmented market.
\end{prop}


Proposition \ref{prop:optimality} suggests that fragmentation with multiple fee levels improves market quality. Specifically, it is always possible to devise a fee structure in a fragmented market that yields a (weakly) lower implementation shortfall than a single-fee market. The logic is as follows: First, the highest fee in the fragmented market is set equal to the single pool fee, ensuring that the marginal $\LP$ participating the market is the same across both scenarios (i.e., the $\LP$ with endowment $\underline{q}=\frac{\Gamma}{f}$). This condition guarantees the same level of realized gains from trade in fragmented and non-fragmented markets. Second, a lower fee is then chosen for another pool to attract liquidity providers with higher token endowments, resulting in lower trading costs. This combination of reduced trading costs and unchanged gains from trade leads to a lower implementation shortfall in a fragmented market. 

% Figure environment removed

\begin{cor}\label{cor:optimality}
There exists a fee menu in a two-pool fragmented market structure that achieves an equal or lower implementation shortfall than on any single-fee pool configuration. Further, the fee menu can be chosen to achieve strictly lower implementation shortfall in fragmented markets if $f^\star=g W^{-1}\left({\rm e}\frac{g Q}{\Gamma }\right)>\frac{\Gamma}{Q}$.
\end{cor}

Corollary \ref{cor:optimality} emerges as a particular case of Proposition \ref{prop:optimality}, with the assumption that the single-pool operates at its optimal fee level. In essence, if a fragmented fee structure can be designed to achieve a lower implementation shortfall compared to an arbitrary single-pool fee, then a fee structure that dominates the optimally set single-pool fee achieves a lower implementation shortfall than any single-fee pool. 

The right panel of Figure \ref{fig:is} illustrates the scenario where the lower fee in a fragmented market is set to half the optimal fee of a single pool, denoted as \( \ell = \frac{1}{2}f^\star \). When gas prices are sufficiently low, $\LP$s with large endowments choose to provide liquidity on the low-fee pool, leading to reduced trading costs and, consequently, a lower implementation shortfall. As gas prices increase, the implementation shortfall rises in both single-pool and fragmented markets, primarily because more $\LP$s are priced out which results in higher unrealized gains. Additionally, the shortfall increases more rapidly with rising gas prices in the fragmented market as $\LP$s switch over from the low- to the high-fee pool --- that is, the result in Proposition \ref{cor:comp_stat_ms}. At very high gas prices, all $\LP$s converge in the high-fee pool, effectively collapsing the fragmented market into a single pool with the optimal fee level.








\subsection{Model implications and empirical predictions}

\begin{pred}\label{pred:comp_stat_Gamma}
The liquidity market share of the low-fee pool  decreases in the gas fee $\Gamma$.
\end{pred}

Prediction \ref{pred:comp_stat_Gamma} follows directly from Proposition \ref{cor:comp_stat_ms} and Figure \ref{fig:liqshares}. A higher gas price increases the fixed cost of active liquidity management, particularly so for smaller liquidity providers. In response, $\LP$s with lower endowments migrate to the high-fee pool  where they trade less often. 

\begin{pred}\label{pred:clienteles}
$\LP$s on the low-fee pool make larger liquidity deposits than $\LP$s on the high-fee pool.
\end{pred}

Prediction \ref{pred:clienteles} follows from the equilibrium discussion in Section \ref{sec:liqprov}. Liquidity providers with large token endowments ($q_i>\qmg$) deposit them in the low-fee pool since they are better positioned to actively manage liquidity due to economies of scale. $\LP$s with lower endowments ($q_i\leq\qmg$) either stay out of the market or choose pool $H$ which allows them to offer liquidity in a more passive manner.

% Figure environment removed

Figure \ref{fig:theory_liqsupply} illustrates this prediction through a Monte Carlo simulation. We plot the equilibrium liquidity supply decisions of 100,000 $\LP$s with endowments drawn from density \eqref{eq:density} and $Q=3$. The top panel highlights three groups of liquidity providers: low-endowment $\LP$s (in green) that are being rationed out of the market due to high gas cost, medium-endowment $\LP$s (blue) that deposit liquidity on pool $H$, and high-endowment $\LP$s (orange) that choose the low-fee pool $L$. 



\begin{pred}\label{pred:trade_size_volume}
The average trade size is higher on pool $H$ than on pool $L$. At the same time, trading volume is higher on pool $L$ than on pool $H$.
\end{pred}

Next, Prediction \ref{pred:trade_size_volume} deals with differences between incoming trades on the two liquidity pools. For a wide range of model parameters, incoming order flow on the low fee pool consists of a large number of small trades, and occasional large trades. In contrast, there are few trades on the high fee pool, but they are all relatively large. The model can therefore reconcile two apparently conflicting patterns: one liquidity pool captures most of the trading volume, while the largest trades are executed on the competitor. Figure \ref{fig:theory_trade} illustrates the prediction through a Monte Carlo simulation of the model.


% Figure environment removed


\begin{pred}\label{pred:clienteles_cs}
The average liquidity deposit on both the low- and- high fee pool increases with gas costs.
\end{pred}

An increase in the gas cost $\Gamma$ has two effects: first, the $\LP$s with the lowest endowments on pool $L$ switch to pool $H$. As a result, the average deposit on pool $L$ increases. Second, the $\LP$s with low endowments on pool $H$ may leave the market. Both channels translate to a higher average deposit on pool $H$, which experiences an inflow (outflow) of relatively high (low) endowment $LP$ following an increase in gas costs.



The bottom left panel highlights the clientele effect: that is, the average deposit is higher on pool $L$. Due to the skew of the Pareto distribution, however, there are more $\LP$ accounts active on pool $H$ than on pool $L$ for a wide range of parameter values.



\begin{pred}\label{pred:updates}
$\LP$s update liquidity more frequently on the low-fee than on the high-fee pool.
\end{pred}

Prediction \ref{pred:updates} is a consequence of Lemma \ref{lem:duration}. Liquidity cycles on pool $L$ are shorter than on pool $H$, since $\LP$s on the low-fee pool trade against both small and large orders, rather than only against large orders on the high-fee pool. Consequently, we expect $\LP$s on pool $L$ to actively manage their liquidity positions.


\begin{pred}\label{pred:updates_gas}
A larger gas price leads to more frequent liquidity updates on the low-fee pool.
\end{pred}
An increase in gas price leads to some $\LP$s switching from the low- to the high-fee pool. As a result, liquidity supply on the low-fee pool drops, leading to a shorter cycle as incoming order flow depletes the pool at a faster pace.

