We build our measure of impermanent loss in line with the definition of token reserves within a price range in the Uniswap V3 white paper \citep{Uniswapv3Core2021} and Section 4.1 in \citet{Heimbach2023}.

Consider a liquidity provider who supplies $L$ units of liquidity into a pool trading a token $x$ for a token $y$. The chosen price range is $\left[p_\ell, p_u\right]$ with $p_\ell<p_u$. Further, the current price of the pool is $p_0$. We are interested in computing the impermanent loss at a future point in time, when the price updates to $p_1$.

From \citet{Uniswapv3Core2021}, the actual amount of tokens $x$ and $y$ (``real reserves'') deposited on a Uniswap v3 liquidity pool with a price range $\left[p_\ell, p_u\right]$ to yield liquidity $L$ are functions of the current pool price $p$:
\begin{equation}\label{eq:reserves_t0}
    x\left(p\right)=\begin{cases}
        L \times \left(\frac{1}{\sqrt{p_\ell}}-\frac{1}{\sqrt{p_u}}\right) & \text { if } p\leq p_\ell \\
        L \times \left(\frac{1}{\sqrt{p}}-\frac{1}{\sqrt{p_u}}\right) & \text { if } p_\ell<p\leq p_u \\
        0 & \text { if } p > p_u
    \end{cases} \text{ and }     y\left(p\right)=\begin{cases}
        0 & \text { if } p\leq p_\ell \\
        L \times  \left(\sqrt{p}-\sqrt{p_\ell}\right) & \text { if } p_\ell<p\leq p_u \\
        L \times  \left(\sqrt{p_u}-\sqrt{p_\ell}\right) & \text { if } p > p_u.
    \end{cases}
\end{equation}

From equation \eqref{eq:reserves_t0}, the value of the liquidity position at $t=1$ is therefore
\begin{equation}
    V_\text{position}=p_1 x\left(p_1\right) + y\left(p_1\right)=\begin{cases}
        L p_1 \times \left(\frac{1}{\sqrt{p_\ell}}-\frac{1}{\sqrt{p_u}}\right) & \text { if } p_1\leq p_\ell \\
        L \times \left(2\sqrt{p_1}-\frac{p_1}{\sqrt{p_u}}-\sqrt{p_\ell}\right) & \text { if } p_\ell<p_1\leq p_u \\
         L \times  \left(\sqrt{p_u}-\sqrt{p_\ell}\right) & \text { if } p_1 > p_u.
    \end{cases} 
\end{equation}

Conversely, the value of a strategy where the liquidity provider holds the original token quantities and marks them to market at the updated price is
\begin{equation}
    V_\text{hold}=p_1 x\left(p_0\right) + y\left(p_0\right)=\begin{cases}
        L p_1 \times \left(\frac{1}{\sqrt{p_\ell}}-\frac{1}{\sqrt{p_u}}\right) & \text { if } p_0\leq p_\ell \\
        L \times \left(\frac{p_1+p_0}{\sqrt{p_0}}-\frac{p_1}{\sqrt{p_u}}-\sqrt{p_\ell}\right) & \text { if } p_\ell<p_0\leq p_u \\
         L \times  \left(\sqrt{p_u}-\sqrt{p_\ell}\right) & \text { if } p_0 > p_u.
    \end{cases} 
\end{equation}

The impermanent loss is then defined as the excess return from holding the assets versus providing liquidity on the decentralized exchange:
\begin{equation}
    \text{ImpermanentLoss}=\frac{V_\text{hold}-V_\text{position}}{V_\text{hold}}.
\end{equation}

Empirically, we follow \citet{Heimbach2023} and compute impermanent loss for ``symmetric'' positions around the current pool price, that is $p_\ell=p_0 \alpha^{-1}$ and $p_u= p_0\alpha$, with $\alpha>1$. We allow for a time lag of one hour between $p_0$ and $p_1$.