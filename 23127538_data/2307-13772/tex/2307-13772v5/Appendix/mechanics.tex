In this appendix, we walk through a numerical example to illustrate the mechanism of liquidity provision and trading on Uniswap V3 liquidity pools. To facilitate understanding, we highlight the similarities and differences between the Uniswap mechanism and the familiar economics of a traditional limit order book.

Let $p_c=1500.62$ be the current price of the ETH/USDT pair. Traders can provide liquidity on Uniswap V3 pools at prices on a log-linear tick space. In particular, consecutive prices are always $\theta$ basis point apart: $p_i=1.0001^{\theta i}$, where $\theta$ is the tick spacing. For the purpose of the example, we take $\theta=60$. Consequently, the current price of $1500.62$ corresponds to a tick index of $c=73140$. Figure \ref{fig:app_grid} illustrates three ticks on grid below and above the current price of ETH/USDT $1500.62$.

% Figure environment removed

\paragraph{Two-sided liquidity provision.} Trader \textbf{A} starts out with a capital of USDT 20,000 and wants to provide liquidity over the price range $\left[1491.64, 1527.87\right]$, a range which spans four ticks. Liquidity provision over a range that includes the current price corresponds to posting quotes on both the bid and ask side of a traditional limit order book, where the current price of the pool corresponds to the mid-point of the book.
\begin{enumerate}
    \item \emph{Bid quotes:} trader \textbf{A} deposits USDT over the price range $\left[1491.64,1500.62\right)$. This action is equivalent to submitting a buy limit order with a bid price of $1491.64$. An incoming Ether seller can swap their ETH for the USDT deposited by \textbf{A}, generating price impact until the limit price of $1491.64$ is reached.
    \item \emph{Ask quotes:} at the same time, trader \textbf{A} deposits ETH over three ticks: $\left[1500.62,1509.65\right)$, $\left[1509.65,1518.73\right)$, and $\left[1518.73,1527.87\right)$. The action corresponds to submitting \emph{three} sell limit orders with ask prices $1509.65$, $1518.73$, and $1527.87$, respectively. Incoming Ether buyers can swap USDT for trader \textbf{A}'s ETH.
\end{enumerate}

In the Uniswap V3 protocol, deposit amounts over each tick $\left[p_i,p_{i+1}\right)$ must satisfy
\begin{align}\label{eq:app_deposits_tick}
    \text{ETH deposit over $\left[p_{i},p_{i+1}\right)$: } x_i&=L\left(\frac{1}{\sqrt{p_i}}-\frac{1}{\sqrt{p_{i+1}}}\right) \\
    \text{USDT deposit over $\left[p_{i},p_{i+1}\right)$: } y_i &= L\left(\sqrt{p_{i+1}}-\sqrt{p_i}\right), 
\end{align}
where $L$ (``liquidity units'') is a scaling factor proportional to the capital committed to the liquidity position. The scaling factor $L$ is pinned down by setting the total committed capital equal to the sum of the positions (in USDT), that is  $p_c\sum_{i} x_i + \sum_{i} y_i$. In our example,
\begin{equation}
    1500.62 \times L_A \times \left(\frac{1}{\sqrt{1500.62}}-\frac{1}{\sqrt{1527.87}}\right) + L_A \times \left(\sqrt{1500.62}-\sqrt{1491.64}\right) = 20000,
\end{equation}
leading to $L_A=43188.6$. We the value of $L_A$ into \eqref{eq:app_deposits_tick} and conclude that trader \textbf{A} deposits 5,013.38 USDT over $\left[1491.64, 1500.62\right)$ and ETH 9.99 over  $\left[1500.62, 1527.87\right)$ (approximately ETH 3.33 over each tick size covered).

\paragraph{One-sided liquidity provision.} Trader \textbf{B} has USDT 20,000 and wants to post liquidity over the range $\left[1509.65,1527.87\right)$, which does not include the current price.  This action corresponds to posting ask quotes to sell ETH deep in the book, at price levels $1518.73$ and $1527.87$. Liquidity is not ``active'' -- that is, the quotes are not filled -- until the existing depth at $1509.65$ is consumed by incoming trades. 

We use equation \eqref{eq:app_deposits_tick} to solve for the amount of liquidity units provided by \textbf{B}:
\begin{equation}
    1500.62 \times L_B \times \left(\frac{1}{\sqrt{1509.65}}-\frac{1}{\sqrt{1527.87}}\right) = 20000,
\end{equation}
which leads to $L_B=86589.4$. Trader \textbf{B} deposits 6.67 ETH on each of the two ticks covered by the chosen range. 

% Figure environment removed


Figure \ref{fig:liquidity_pool} illustrates market depth after \textbf{A} and \textbf{B} deposit liquidity in the pool. The current price of the pool is equivalent to a midpoint in traditional limit order markets. The ``ask side'' of the pool is deeper, consistent with both liquidity providers choosing ranges skewed towards prices above the current midpoint. Liquidity is uniformly provided over ticks -- that is, each trader deposits an equal share of their capital at each price tick covered by their price range. 


\paragraph{Trading, fees, and price impact.} Suppose now that a trader \textbf{C} wants to buy 10 ETH from the pool. For each tick interval $\left[p_i,p_{i+1}\right)$, price impact is computed using a constant product function over virtual reserves:
\begin{equation}
    \underbrace{\left(x+\frac{L}{\sqrt{p_{i+1}}}\right)}_\text{Virtual ETH reserves}\underbrace{\left(y+L\sqrt{p_i}\right)}_\text{Virtual USDT reserves}=L^2,
\end{equation}
where $x$ and $y$ are the actual ETH and USDT deposits in that tick range, respectively. Virtual reserves are just a mathematical artifact: they extend the physical (real) deposits as if liquidity would be uniformly distributed over all possible prices on the real line. Working with constant product functions over real reserves is not feasible: in our example, the product of real reserves is zero throughout the order book (since only one asset is deposited in each tick range).

Let $\tau=1\%$ denote the pool fee that serves as an additional compensation for liquidity providers. That is, if the buyer pays to pay $\Delta y$ USDT to purchase a quantity $\Delta x$ ETH, he needs to effectively pay $\Delta y\left(1+\tau\right)$. As per the Uniswap V3 white paper, liquidity fees are not automatically deposited back into the pool.

\begin{enumerate}
    \item \textbf{Tick 1: $\left[1500.62,1509.65\right)$}. Trader \textbf{C} first purchases $3.33$ ETH at the first available tick above the current price (equivalent to the ``best ask''). To remove the ETH, he needs to deposit $\Delta y_1$ USDT, where $\Delta y_1$ solves:
    \begin{equation}
        \left(3.33-3.33+\frac{L_A}{\sqrt{1509.65}}\right)\left(0+\Delta y_1+L_A\sqrt{1500.62}\right)=L_A^2,
    \end{equation}

which leads to $\Delta y_1=5026.19$ USDT. Trader \textbf{C} pays an average price of 50216.19/3.33=1507.86 USDT for each unit of ETH purchased. Further, he pays a fee of 50.26 USDT to liquidity provider \textbf{A} (the only liquidity provider at this tick).

The new current price is given by the ratio of virtual reserves,
\begin{equation}
    p^\prime=\frac{\Delta y_1 +L_A\sqrt{1500.62}}{3.33-3.33+\frac{L_A}{\sqrt{1509.65}}}=1509.65,
\end{equation}
that is the next price on the tick grid since \textbf{C} exhausts the entire liquidity on $\left[1500.62, 1509.65\right)$.

\item \textbf{Tick 2: $\left[1509.65,1518.73\right)$}. Trader \textbf{C} still needs to purchase 6.67 ETH at the next tick level (where the depth is 10 ETH). The liquidity level at this tick is $L_A+L_B$, that is the sum of liquidity provided by \textbf{A} and \textbf{B}. To remove the 6.67 ETH from the pool, he needs to deposit $\Delta y_2$, where
\begin{equation}
    \left(10-6.67+\frac{L_A+L_B}{\sqrt{1518.73}}\right)\left(0+\Delta y_2 + \left(L_A+L_B\right)\sqrt{1509.65}\right)=\left(L_A+L_B\right)^2.
\end{equation}
It follows that trader \textbf{C} purchases 6.67 ETH by depositing $\Delta y_2=10089.12$ USDT, at an average price of 1512.61. The pool price is updated as the ratio of virtual reserves:
\begin{equation}
    p^{\prime\prime}=\frac{\Delta y_2 +\left(L_A+L_B\right)\sqrt{1509.65}}{10-6.67+\frac{L_A+L_B}{\sqrt{1518.73}}}=1515.7.
\end{equation}
The updated price is in between the two liquidity ticks, since not all depth on this tick level was exhausted in the trade. Following the swap, liquidity on the tick range $\left[1509.65,1518.73\right)$ is composed of both assets: that is 10089.12 USDT and 10-6.67=3.33 ETH. 

Finally, trader $C$ pays 100.89 USDT as liquidity fees (1\% of the trade size), which are distributed to \textbf{A} and \textbf{B} proportionally to their liquidity share. That is, \textbf{A} receives a fraction $\frac{L_A}{L_A+L_B}$ of the total fee (33.57 USDT), whereas \textbf{B} receives 67.32 USDT.

\end{enumerate}

Figure \ref{fig:swap_pool} illustrates the impact of the swap. Within tick $\left[1500.62,1509.65\right)$, \textbf{A} sells 3.33 ETH and buys 5026 USDT. Unlike on limit order books, the execution does not remove liquidity from the book. Rather, $\textbf{A}$'s capital is converted from one token to another and remains available to trade. This feature underscores the passive nature of liquidity supply on decentralized exchanges. Mapping the concepts to traditional limit order book, this mechanism would imply that every time a market maker's sell order is executed at the ask, a buy order would automatically be placed on the bid side of the market.

The final price of $1517.70$ lies within the tick $\left[1509.65,1518.73\right)$, rather than on its boundary. Trader \textbf{C} only purchases 6.66 ETH out of 10 ETH available within this price interval. The implication is that liquidity on $\left[1509.65,1518.73\right)$ contains both tokens: 3.33 ETH (the amount that was not swapped by \textbf{C}) as well as 10089.12 USDT that \textbf{C} deposited in the pool. 

% Figure environment removed

The bottom panel of Figure \ref{fig:swap_pool} shows the price impact of the swap. From equation (6.15) in the Uniswap V3 white paper, we can solve for the price within tick $\left[p_\text{min},p_\text{max}\right)$ with liquidity $L$, following the execution of a buy order of size $x$:
\begin{equation}
    p\left(x\right)=\frac{p_\text{min} L^2}{\left(L-\sqrt{p_\text{min}}x\right)^2}.
\end{equation}
As expected, the price impact of a swap decreases in the liquidity available $L$ -- each ETH unit purchased by \textbf{C} has a smaller impact on the price once tick 1509.65 is crossed and the market becomes deeper.

