\subsection{Sample construction \label{sec:sample}}
We obtain data from Kaiko on all trades, liquidity deposits (``mints''), and liquidity withdrawals (``burns'') on 1,361 Uniswap v3 pools from the launch of the protocol on May 4, 2021 until September 15, 2022 -- that is, the date when the Ethereum blockchain replaced its proof-of-work protocol with a proof-of-stake one. Our data includes a transaction hash that identifies each trade and liquidity update on the Ethereum blockchain; trade price, direction, and quantity; as well as quantities and price range for each liquidity update.

Further, we collect data from Blockchair on the wallet addresses that initiate each transaction (i.e., akin to an anonymous trader id), the position of each transaction in its block, as well as the gas price limit set by the trader and the amount of gas used -- in Ether as well as US dollars. Finally, we use the Kaiko ``Cross-Price'' API to collect token prices in US dollars. For each token, we obtain the volume-weighted average price at minute and day frequencies, using data from all major centralized and decentralized exchanges. If a particular token does not directly trade against U.S. dollars, Kaiko computes the benchmark price across the most liquid trading path (usually first converting the token to BTC or ETH, and then to U.S. dollars). 

% We obtain data from Kaiko on all trades and liquidity updates on 1,361 Uniswap v3 pools in 881 asset pairs from May 4, 2021 to September 15, 2022 -- that is, from the launch of the platform until Ethereum's transition to a Proof-of-Stake protocol.\footnote{There are virtually no restrictions to create a liquidity pool on Uniswap for any pair of ERC-20 tokens. Throughout our sample, a total of 8,167 Uniswap v3 pools were created for 6,535 asset pairs. Kaiko filters out most of these pools in their trade data set -- either because the trading volume is zero, or because tokens are labelled as untrustworthy (including scams or clones).}

There are no restrictions to list a token pair on Uniswap. Some pools might therefore be used for experiments, or they might include untrustworthy tokens. Following \citet{LeharParlour2021}, we remove pools that are either very small or that are not attracting an economically meaningful trading volume. We retain liquidity pools that fulfill the following four criteria: (i) have at least one interaction in more than 30 days in the sample, (ii) have more than 100 liquidity interactions throughout the sample, (iii) have an average daily liquidity balance in excess of US\$100,000, and (iv) capture more than 1\% of trading volume for a particular asset pair.

These basic screens give us a baseline sample of 262 liquidity pools covering 224 asset pairs, with combined daily dollar volume of \$1.32 billion and total value locked (i.e., aggregate liquidity supply) of \$3.07 billion as of September 15, 2022. We capture 13,006,390 interactions with liquidity pool smart contracts (accounting for 87.56\% of the entire universe of trades and liquidity updates). Trading and liquidity provision on Uniswap is heavily concentrated: the five largest pairs (USDC-WETH, WETH-USDT, WBTC-WETH, USDC-USDT, and DAI-WETH) account on average for 81\% of trading volume and 52\% of supplied liquidity.\footnote{WETH and WBTC stand for ``wrapped'' Bitcoin and Ether. Plain vanilla Bitcon and Ether are not compliant with the ERC-20 standard for tokens, and therefore cannot be directly used on decentralized exchanges' smart contracts. USDC (USD Coin), USDT (Tether), and DAI are stablecoins meant to closely track the US dollar.}


\subsection{Liquidity fragmentation patterns}

For 37 out of the 224 asset pairs in our baseline sample, liquidity supply is fragmented across two pools with different fees -- either with 1 and 5 bps fees (7 pairs), 5 and 30 bps fees (8 pairs), or 30 and 100 bps fees (22 pairs).\footnote{In some cases, more than two pools are created for a pair -- e.g., for USDC-WETH there are four pools with 1, 5, 30, and 100 bps liquidity fees. In all but one case however, two pools heavily dominate the others. As described in Section \ref{sec:sample} we filter out small pools with less than 1\% volume share or less than \$100,000 liquidity deposits. The exception is FEI-USDC, for which three pools attract significant volume throughout the sample. However, on any given day two pools remain dominant: either the 5 and 30 bps fee pools before the launch of the 1 bps pool, and the 1 and 5 bps fee pool afterwards.} Despite being fewer in number, fragmented pairs are economically important: they account on average for 84\% of the capital committed to Uniswap v3 and for 94\% of its dollar trading volume. All major token pairs such as WETH-USDC, WETH-USDT, or WBTC-WETH trade on fragmented pools.

For each fragmented liquidity pair, we label the \emph{low} and the \emph{high} fee liquidity pool to facilitate analysis across assets. For example, the low and high liquidity fees for USDC-WETH are 5 and 30 bps, respectively, but only 1 and 5 bps for a lower volatility pair such as USDC-USDT. We refer to non-fragmented pools as \emph{single} (i.e., the unique pool for an asset pair).

We aggregate all interactions with Uniswap smart contracts into a panel across days and liquidity pools. To compute the end-of-day pool size, we account for all changes in token balances, across all price ranges. There are three possible interactions: A deposit or ``mint'' adds tokens to the pool, a withdrawal or ``burn'' removes tokens, whereas a trade or ``swap'' adds one token and removes the other. We track these changes across to obtain daily variation in the quantity of tokens on each pool. We compute the end-of-day pool size as the cumulative sum of daily changes in token balance, marked to market in US dollars at end-of-day prices.\footnote{Since we track all pools since inception, the initial balance is set to zero.} Intra-day trades and liquidity mint volumes are converted in US dollars using the price of the corresponding minute. 

Table \ref{tab:sumstat} reports summary statistics across pools with different fee levels. High-fee pools attract on average more liquidity than their low-fee counterparts (\$43.42 million and \$31.35 million, respectively), but only capture 36.66\% percent of the trading volume. Consistent with our theoretical predictions, low-fee pools attract five times as many trades as high-fee competitors (577 versus 111 average trade count per day).  At the same time, the average trade on a high-fee pool is twice as large (\$22,460) than on a low-fee pool (\$9,890). 

The distribution of mint sizes is heavily skewed to the right, with 6.5\% of deposits exceeding \$1 million. There are large differences across pools -- the median $\LP$ deposit on the low-fee pool is \$22,780, twice as much as the median deposit on the high-fee pool (\$10,590). At the same time, the number of liquidity providers on high-fee pools is 91\% higher than on low-fee pools (12.48 unique addresses per day on high-fee pools versus only 6.52 unique address on high-fee pools).




% Table created by stargazer v.5.2.3 by Marek Hlavac, Social Policy Institute. E-mail: marek.hlavac at gmail.com
% Date and time: Fri, Oct 21, 2022 - 4:01:40 PM
\begin{table}[H] \centering 
\caption{Descriptive statistics} 
  \label{tab:sumstat} 

\begin{minipage}[t]{1\columnwidth}%
\footnotesize
This table reports descriptive statistics for variables used in the empirical analysis. \emph{Pool size} is defined as the total value locked in the pool's smart contract at the end of each day. We compute the balance on day $t$ as follows: we take the balance at day $t-1$ and add (subtract) liquidity deposits (withdrawals) on day $t$, as well as accounting for token balance changes due to trades. The liquidity balance on the first day of the pool is taken to be zero. End of day balances are finally converted to US dollars. \emph{Daily volume} is computed as the sum of US dollar volume for all trades in a given pool and day. \emph{Liquidity share} (\emph{Volume share}) is computed as the ratio between a pool size (trading volume) for a given fee level and the aggregate size of all pools (trading volumes) for the same pair in a given day. \emph{Trade size} and \emph{Mint size} are the median trade and liquidity deposit size on a given pool and day, denominated in US dollars. \emph{Trade count} represents the number of trades in a given pool and day. \emph{\textbf{LP} wallets} counts the unique number of wallet addresses interacting with a given pool in a day.  The \emph{liquidity yield} is computed as the ratio between the daily trading volume and end-of-day TVL, multiplied by the fee tier. The \emph{price range} for every mint is computed as the difference between the top and bottom of the range, normalized by the range midpoint -- a measure that naturally lies between zero and two. The \emph{impermanent loss} is computed as in \citet{Heimbach2023} for a position in the range of 99\% to 101\% of the current pool price, with a forward-looking horizon of 200 blocks. Finally, \emph{mint-to-burn} and \emph{burn-to-mint} times are defined as the time between a mint (burn) and a subsequent burn (mint) by the same address in the same pool, measured in hours. \emph{Mint-to-burn} and \emph{burn-to-mint} are recorded on the day of the final interaction with the pool. 
\end{minipage}

\vspace{0.05in}
\resizebox{0.95\textwidth}{!}{

\small
\begin{tabular}{@{\extracolsep{5pt}}llrrrrrr} 
\\[-1.8ex]\hline 
\hline \\[-1.8ex] 
Statistic & Pool fee & \multicolumn{1}{c}{Mean} & \multicolumn{1}{c}{Median} & \multicolumn{1}{c}{St. Dev.} & \multicolumn{1}{c}{Pctl(25)} & \multicolumn{1}{c}{Pctl(75)} & \multicolumn{1}{c}{N} \\ 
\hline \\[-1.8ex] 

Pool size (\$M) & Low & 31.35 & 1.87 & 96.74 & 0.21 & 14.12 & 12,334 \\   
& High & 43.42 & 2.78 & 96.92 & 0.32 & 23.54 & 13,252 \\ 
& Single & 4.56 & 0.60 & 16.91 & 0.18 & 2.15 & 88,269 \\ 

Liquidity share (\%) & Low & 38.82 & 23.95 & 34.82 & 6.94 & 75.37 & 12,334 \\ 
 & High & 48.29 & 49.66 & 36.30 & 10.81 & 82.87 & 13,252 \\ 

 
Daily volume (\$000) & Low & 36,405.67 & 586.66 & 130,441.30 & 6.29 & 8,752.90 & 12,334 \\ 
& High & 10,317.39 & 72.41 & 36,160.20 & 0.62 & 2,605.98 & 13,252 \\  
& Single & 867.11 & 32.68 & 4,628.82 & 0.89 & 267.78 & 88,405 \\   

Volume share & Low & 64.64 & 80.11 & 37.35 & 32.64 & 97.76 & 10,546 \\ 
& High & 36.66 & 21.54 & 37.96 & 2.43 & 72.72 & 10,660 \\ 

Trade size (\$000) & Low & 9.89 & 3.68 & 18.03 & 0.94 & 10.00 & 10,546 \\
& High & 22.46 & 4.49 & 45.05 & 0.90 & 21.34 & 10,660 \\  
& Single   & 6.79 & 2.30 & 23.30 & 0.84 & 6.00 & 72,404 \\ 

Mint size (\$000) & Low & 1,134.84 & 22.78 & 15,770.57 & 5.20 & 88.24 & 6,076 \\ 
& High & 1,086.86 & 10.59 & 14,313.54 & 2.53 & 40.87 & 5,911 \\
& Single & 193.34 & 13.97 & 1,077.82 & 3.05 & 61.64 & 21,189 \\ 

Trade count & Low & 577.91 & 76 & 1,404.21 & 10 & 410 & 12,334 \\ 
& High & 111.26 & 20 & 478.30 & 2 & 84 & 13,252 \\ 
& Single & 42.45 & 12 & 125.46 & 2 & 36 & 88,405 \\  

$\LP$ wallets & Low & 6.52 & 1 & 16.19 & 0 & 5 & 12,334 \\   
& High & 12.48 & 1 & 45.63 & 0 & 6 & 13,252 \\  
& Single  & 1.28 & 0 & 4.05 & 0 & 1 & 88,405 \\ 

Liquidity yield (bps) & Low & 12.84 & 3.39 & 63.36 & 0.20 & 10.68 & 12,308 \\ 
& High & 11.79 & 2.18 & 62.94 & 0.05 & 7.99 & 13,075 \\ 
& Single & 11.76 & 1.76 & 57.93 & 0.11 & 7.50 & 87,852 \\

Price range & Low &  0.46 & 0.36 & 0.42 & 0.17 & 0.66 & 7,394 \\ 
& High & 0.60 & 0.54 & 0.43 & 0.32 & 0.82 & 7,209 \\  
& Single & 0.68 & 0.58 & 0.55 & 0.23 & 1.06 & 32,833 \\ 

Impermanent loss (bps) & Low  & 16.06 & 7.74 & 29.68 & 0.51 & 19.25 & 11,981 \\  
& High & 12.52 & 4.17 & 35.06 & 0.0005 & 13.72 & 13,112 \\  
& Single & 17.51 & 6.24 & 55.70 & 0.01 & 19.72 & 87,149 \\ 

Mint-to-burn (hrs) & Low & 458.20 & 60.35 & 1,253.01 & 18.72 & 256.80 & 6,233 \\
& High & 693.84 & 119.98 & 1,526.30 & 30.94 & 468.60 & 6,110 \\  
& Single & 629.24 & 135.39 & 1,343.18 & 31.99 & 508.04 & 22,900 \\ 

Burn-to-mint (hrs) & Low & 75.26 & 0.17 & 430.05 & 0.07 & 1.11 & 4,973 \\
& High & 107.81 & 0.18 & 589.12 & 0.06 & 1.58 & 4,985 \\  
& Single & 105.48 & 0.17 & 507.14 & 0.06 & 4.07 & 15,897 \\ 

\hline \\[-1.8ex] 

\end{tabular} 
}
\end{table} 

One concern with measuring average mint size is just-in-time liquidity provision (JIT). JIT liquidity providers submit very large and short-lived deposits to the pool to dilute competitors on an incoming large trade; they immediately withdraw the balance in the same block after executing the trade. In our sample, JIT liquidity provision is not economically significant, accounting for less than 1\% of aggregate trading volume. However, it has the potential to skew mint sizes to the right, particularly in low-fee pools, without providing liquidity to the market at large. We address this issue by (i) filtering out JIT mints using the algorithm in Appendix \ref{sec:app-jit} and (ii) taking the median liquidity mint size at day-pool level rather than the mean. 

Further, we follow \citet{augustin2022reaching} to compute the daily liquidity fee yield as the product between pool's fee tier and the ratio between trading volume and the lagged total value locked (TVL). That is,
\begin{equation}
    \text{Liquidity yield}=\text{liquidity fee}_i \times \frac{\text{Volume}_{i,t}}{\text{TVL}_{i,t-1}},
\end{equation}
for pool $i$ and day $t$. The average daily yield is about 12 basis points, with slightly more elevated figures for low-fee pools.


A salient observation in Table \ref{tab:sumstat} is that non-fragmented pairs (``single'' pools) are significantly smaller -- on average less than 5\% of the pool size and trading volume of fragmented pairs. Average trade and mint sizes are correspondingly lower as well. The evidence suggests that pairs for which there is significant trading interest, and therefore potentially a broader cross-section of potential liquidity providers, are more likely to become fragmented.

Figure \ref{fig:stat_liq} plots the distributions of our empirical measures across low- and high-fee liquidity pools. It suggest a sharp segmentation of liquidity supply and trading across pools. High-fee pools attract a large number of small liquidity providers, and end up with a larger \emph{aggregate} size than their low-fee counterparts. Trading volume is similarly segmented: most small value trades are executed on the cheaper low-fee pools, making up the majority of daily volume for a given pair. High-value trades, of which there are fewer, are more likely to (also) execute on high-fee pools.

% Figure environment removed




Our theoretical framework in Section \ref{sec:model} implies that liquidity suppliers manage their positions more actively in the low- than the high-fee pool. Figure \ref{fig:liq_cycles} provides suggestive evidence for liquidity cycles of different lengths in the cross-section of pools. Liquidity on decentralized exchanges is significantly more passive than on traditional equity markets. That is, liquidity providers do not often manage their positions at high frequencies. The median time from a mint (deposit) to a subsequent burn (withdrawal) from the same wallet on the same pool ranges from 2.48 days on low-fee pools to 4.97 days on high-fee pools. 

When do $\LP$s re-balance their positions? In 67\% of cases, liquidity providers only withdraw tokens from the pool when their position exits the price range that allows them to collect fees. Concretely, $\LP$s specified price range for liquidity provision does not straddle the most recent reference price of the pool. The scenario mirrors a limit order market where a liquidity provider's outstanding limit orders are deep in the book, such that she doesn't stand to earn the spread on the marginal incoming trade. In this case, a rational market maker might want to cancel their outstanding order and place a new one at the top of the book. This is exactly the pattern we observe on Uniswap: the subsequent mint following a burn straddles the new price 81\% of the time -- $\LP$s reposition
their liquidity around the current prices to keep earning fees on incoming trades. Moreover, re-balancing is swift -- the median time between a burn and a subsequent mint is just 10 minutes (0.17 hours).

The empirical pattern in Figure \ref{fig:liq_cycles} echoes liquidity cycles as described in Section \ref{sec:model}. Liquidity deposit tokens in Uniswap pools and then patiently wait for days until incoming order flow exhausts their position (i.e., posted liquidity no longer earns fees). Once this happens, $\LP$s quickly re-balance their position in a matter of minutes -- by removing stale liquidity and adding a new position around the current price. The cycle is longer on high-fee pools for which trading volume is lower and liquidity takes longer to deplete.

Importantly, $\LP$s do not seem to ``race'' to update liquidity upon information arrival as in \citet{Budish2015TheResponse}. First, they very rarely manage their position intraday. Second, $\LP$s on Uniswap typically do not remove in-range liquidity that stands to trade first against incoming order flow and therefore bears the highest adverse selection risk. Our results are consistent with \citet{CapponiJia2021} who theoretically argue that $\LP$s have low incentives to manage liquidity on news arrival, as well as with \citet{CapponiJiaYu2022} who find no evidence of traders racing to trade on information on Uniswap v2.


% Figure environment removed

\paragraph{Measuring gas prices.} Each interaction with smart contracts on the Ethereum blockchain requires computational resources, measured in units of ``gas.'' Upon submitting a mint or burn transaction to the decentralized exchange, each liquidity provider specifies their willingness to pay per unit of gas, that is they bid a  ``gas price.'' Traders are likely to bid higher prices for more complex transactions or if they require a faster execution. To generate a conservative daily benchmark for the gas price, we compute the average of the lowest 100 user gas bids for mint and burn interactions on day $t$, across all liquidity pools in the benchmark sample. These transactions are more likely to be plain vanilla deposits or withdrawals of liquidity, capturing the cost of a simple interaction with the decentralized exchange smart contract.

Figure \ref{fig:gascosts} showcases the significant fluctuation in gas costs for Uniswap liquidity transactions over time. 
Gas costs denominated in USD are influenced by two primary factors: network congestion, which leads to variations in gas prices measured in Ether, and the fluctuation of Ether's value relative to the US dollar. On a monthly average, gas costs peaked at US\$195 in November 2021 and have since plummeted to US\$6 in August and September 2022.

% Figure environment removed
