Just-in-time (JIT) liquidity is a strategy that leverages the transparency of orders on the public blockchains. If a liquidity provider observes an incoming large order that has not been processed by miners and it deems uninformed in the public mempool, it can conveniently re-arrange transactions and propose a sequence of actions to sandwich this trade as follows:
\begin{enumerate}
    \item Add a large liquidity deposit at block position $k$, at the smallest tick around the current pool price. 
    \item Let the trade at block position $k+1$ execute and receive liquidity fees.
    \item Remove or burn any residual un-executed liquidity at block position $k+2$.
\end{enumerate}
The mint size is optimally very large (i.e., of the order of hundred of millions USD for liquid pairs), such that the JIT liquidity provider effectively crowds out the existing liquidity supply and collects most fees for the trade. That is, the strategy is made possible by pro-rata matching on decentralized exchanges because with time priority, the JIT provider cannot queue-jump existing liquidity providers. Since the JIT liquidity provider does not want to passively provide capital, it removes any residual deposit immediately after the trade.

We identify JIT liquidity events by the following algorithm as in \citet{WanAdams2022}:
\begin{enumerate}
\item Search for mints and burns in the same block, liquidity pool, and initiated by the same wallet address. The mint needs to occur exactly two block positions ahead of the burn (at positions $k$ and $k+2$).
\item Classify the mint and the burn as a JIT event if the transaction in between (at position $k+1$) is a trade in the same liquidity pool.
\end{enumerate}
 
JIT events are rare in our sample, and account for less than 1\% of the traded volume on Uniswap v3. Further, more than half of them occur in a single pair - USDC-WETH, and in low-fee pools. The Uniswap Labs provides further discussions on the aggregate impact of JIT liquidity provision \href{https://uniswap.org/blog/jit-liquidity}{here}. Regarding the economic effects, JIT liquidity reduces price impact for incoming trades, but dilutes existing liquidity providers in the pro-rata markets, and can discourage liquidity supply in the long run. 

