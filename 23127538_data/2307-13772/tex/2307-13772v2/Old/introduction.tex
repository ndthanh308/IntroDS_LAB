Uniswap V3 offers liquidity providers two instruments to manage adverse selection risk:
\begin{itemize}
    \item A menu of liquidity fees (currently set to either 0.01\%, 0.05\%, 0.3\%, or 1\% of the incoming trade value), analogous to a tick size that pins down a fixed spread \citep{LeharParlour2021}. 
    \item ``Concentrated liquidity:'' the possibility to set up a (narrow) price band for providing liquidity, akin to a limit order in traditional markets.
\end{itemize}

\mz{(To confirm with model!)} To an extent, the two instruments are substitutes. Adjusting the liquidity provision price band is a more flexible and precise tool to manage adverse selection risk. At the same time, it is more costly since it requires frequent adjustment of the price band as the reference price moves. Managing liquidity on Uniswap is costly since every interaction with the decentralized exchange requires paying a gas fee. Importantly, the gas fee is (to a first approximation) a fixed cost as does not scale with the dollar value of the order.

In an environment with heterogeneous liquidity provider (LP) capital, liquidity becomes segmented across pools with different fees. High-capital LPs commit large chunks of liquidity to the market and manage it frequently: gas fees are small relative to order size. The ability to frequently adjust their liquidity position allows high-capital LPs to participate in markets with a smaller fixed liquidity fee. On the other hand, low-capital LPs cannot manage liquidity at high frequency and rely on a large liquidity fee to protect them against adverse selection risk. In equilibrium, the low-fee market captures most of the liquidity and trading volume. Liquidity is provided in large chunks by high-capital accounts, is concentrated in a smaller price range, and actively managed. The high-fee market captures liquidity provided in smaller amounts by low-capital accounts and is less actively managed.

\paragraph{Implications/hypotheses}
\begin{enumerate}
    \item Time variation in gas fees can be used to test the model implications. A lower gas fee reduces constraints for low-capital investors, yielding higher liquidity share for the low fee exchange. 
    \item What drives the emergence of multiple active pools for the same share? The volatility of the pair is a factor. For pairs like USDC/USDT, where information asymmetries are low, everyone converges to the low-fee 0.01\% pair.
    \item Informational content on trades on the two markets? Low-capital LPs might be exposed to large informed trades that have already consumed the liquidity on low-fee pools. Is the liquidity fee sufficient compensation? Is there an optimal liquidity fee menu? Or, is it better to have a single fee as on Uniswap V2?
\end{enumerate}

% Figure environment removed