\subsection{Liquidity supply on high- and low-fee pools}

To formally test the model predictions and quantify the differences in liquidity supply across fragmented pools, we build a panel data set for the 32 fragmented pairs in our sample where the unit of observation is pool-day. We estimate linear regressions of liquidity and volume measures on liquidity fees and gas costs:
\begin{equation}
    y_{ijt}=\alpha + \beta_0 d_\emph{low-fee, ij} + \beta_1 \text{GasPrice}_{jt} + \beta_2 \text{GasPrice}_{jt} \times d_\emph{low-fee, ij} + \sum \beta_k \emph{Controls}_{ijt} + \theta_j + \delta_w + \varepsilon_{ijt},
\end{equation}
where $y$ is a variable of interest, $i$ indexes liquidity pools, $j$ runs over asset pairs, and $t$ and $w$ indicates days and weeks, respectively. The dummy $d_\emph{low-fee, ij}$ takes the value one for the pool with the lowest fee in pair $j$ and zero else. 

Further, our set of controls includes pair and week fixed effects, the log aggregate trading volume and log liquidity supply (i.e., total value locked) for day $t$ across all pools $i$. Volume and liquidity are measured in US dollars. We also control for daily return volatility, computed as the range between the daily high and low prices for a given pair $j$ \citep[following][]{Alizadeh2002}: 
\begin{equation}
    \text{Volatility}_{jt}=\frac{1}{2\sqrt{\log 2}} \log\left(\frac{\text{High}_{jt}}{\text{Low}_{jt}}\right).
\end{equation}
To measure volatility for fragmented pairs that actively trade in multiple pools, we select the pool with the highest trading volume for a given day.

Consistent with Figure \ref{fig:stat_liq}, most of the capital deployed to provide liquidity for a given pair is locked in high-fee pools. At the same time, low-fee pools attract much larger trading volume. Models (1) and (5) show that the average low-fee pool attracts 39.5\% of liquidity supply for the average pair (that is, equal to $\nicefrac{(100-20.92)}{2}$) while it executes 62\% (i.e., $\nicefrac{(100+24.62)}{2}$) of the total trading volume. At a first glance, it would seem that a majority of capital on decentralized exchanges is inefficiently deployed in pools with low execution probability. We will show that, in line with our model, the difference is driven by heterogeneous liquidity cycles across pools, leading to the formation of $\LP$ clienteles. 

The regression results in Table \ref{tab:markeshare} support Prediction \ref{pred:comp_stat_Gamma}, stating that market share differences between pools are linked to variation in fixed transaction costs on the blockchain. A one-standard deviation increase in gas prices leads to a 4.52 percentage point increase in the high-fee liquidity share. The results suggests that blockchain transaction costs have an economically meaningful and statistically significant impact on liquidity fragmentation. In line with the theoretical model in Section \ref{sec:model}, a jump in gas prices leads to a reshuffling of liquidity supply from low- to high-fee pools. 

Evidence suggests that a higher gas price leads to a 6.39\% lower volume share for the low-fee pool. This outcome is natural, as the incoming order flow is optimally routed to the high-fee pool, following the liquidity providers.


% Table created by stargazer v.5.2.3 by Marek Hlavac, Social Policy Institute. E-mail: marek.hlavac at gmail.com
% Date and time: Tue, Oct 25, 2022 - 8:34:54 PM

\begin{table}[H] 
  \caption{Liquidity pool market shares and gas prices}   \label{tab:markeshare}
\begin{minipage}[t]{1\columnwidth}%
\footnotesize
			This table reports the coefficients of the following regression:
	\begin{align*}
    \text{MarketShare}_{ijt}=\alpha + \beta_0 d_\emph{low-fee, ij} + \beta_1 \text{GasPrice}_{jt} + \beta_2 \text{GasPrice}_{jt} \times d_\emph{low-fee, ij} + \sum \beta_k \emph{Controls}_{ijt} + \theta_j + \varepsilon_{ijt}
    \end{align*}
 	where the dependent variable is the liquidity or trading volume market share for pool $i$ in asset pair $j$ on day $t$. $d_\emph{low-fee, ij}$ is a dummy that takes the value one for the pool with the lowest fee in pair $j$ and zero else. $\emph{GasPrice}_{jt}$ is the average of the lowest 100 bids on liquidity provision events across all pairs on day $t$, standardized to have a zero mean and unit variance. \emph{Volume} is the natural logarithm of the sum of all swap amounts on day $t$, expressed in thousands of US dollars. \emph{Total value locked} is the natural logarithm of the total value locked on Uniswap v3 pools on day $t$, expressed in millions of dollars.\emph{Volatility} is computed as the daily range between high and low prices on the most active pool for a given pair.All regressions include pair and week fixed-effects. Robust standard errors in parenthesis are clustered by week and ***, **, and * denote the statistical significance at the 1, 5, and 10\% level, respectively.  The sample period is from May 4, 2021 to July 15, 2023. 
\end{minipage}
\small
\begin{center}
\resizebox{\textwidth}{!}{
\begin{tabular}{lcccc@{\hskip 0.3in}cccc}
\toprule
 & \multicolumn{4}{c}{Liquidity market share (\%)} & \multicolumn{4}{c}{Volume market share (\%)} \\ 
 & (1) & (2) & (3) & (4) & (5) & (6) & (7) & (8) \\
 \cmidrule{1-9}

$ d_\text{low-fee}$ & -20.92*** & -20.92*** & -20.92*** & -20.92*** & 24.62*** & 24.63*** & 24.62*** & 24.62*** \\
 & (-27.19) & (-27.17) & (-27.19) & (-27.19) & (20.44) & (20.45) & (20.44) & (20.44) \\
Gas price $\times$ $ d_\text{low-fee}$ & -4.52*** & -4.52*** & -4.52*** & -4.52*** & -6.39*** & -6.39*** & -6.39*** & -6.39*** \\
 & (-7.04) & (-7.04) & (-7.04) & (-7.04) & (-5.77) & (-5.76) & (-5.77) & (-5.77) \\
Gas price & 2.26*** & 2.26*** & 2.26*** & 2.26*** & 3.55*** & 3.53*** & 3.53*** & 3.53*** \\
 & (7.05) & (7.04) & (7.05) & (7.05) & (7.16) & (7.12) & (7.09) & (7.09) \\
Volume & 0.00 & 0.00 & 0.00 & 0.00 & -0.19** & -0.19** & -0.18** & -0.18** \\
 & (0.64) & (1.34) & (0.64) & (0.64) & (-2.46) & (-2.53) & (-2.42) & (-2.42) \\
Total value locked & -0.00 & -0.00 &  &  & 0.58 & 0.58 &  &  \\
 & (-0.56) & (-0.05) &  &  & (1.44) & (1.44) &  &  \\
Volatility & -0.29 &  & -0.29 & -0.29 & -1.16*** &  & -1.16*** & -1.16*** \\
 & (-0.90) &  & (-0.90) & (-0.90) & (-2.74) &  & (-2.74) & (-2.74) \\
Constant & 60.45*** & 60.46*** & 60.45*** & 60.45*** & 41.96*** & 41.99*** & 41.96*** & 41.96*** \\
 & (156.62) & (157.09) & (156.62) & (156.62) & (69.62) & (69.85) & (69.65) & (69.65) \\
Pair FE & Yes & Yes & Yes & Yes & Yes & Yes & Yes & Yes \\
Week FE & Yes & Yes & Yes & Yes & Yes & Yes & Yes & Yes \\
Observations & 40,288 & 40,288 & 40,288 & 40,288 & 36,059 & 36,059 & 36,059 & 36,059 \\
 R-squared & 0.10 & 0.10 & 0.10 & 0.10 & 0.13 & 0.13 & 0.13 & 0.13 \\ \hline
\bottomrule
\multicolumn{9}{l}{Robust t-statistics in parentheses. Standard errors are clustered at week level.  *** p$<$0.01, ** p$<$0.05, * p$<$0.1} \\
\end{tabular}
}
\end{center}
\end{table} 

What drives the market share gap across fragmented pools? In Table \ref{tab:ordersize} we document stark differences between the characteristics of individual orders supplying or demanding liquidity on pools with low and high fees. On the liquidity supply side, model (1) in Table \ref{tab:ordersize} shows that the average liquidity mint is 107.5\% larger on low-fee pools, which supports Prediction \ref{pred:clienteles} of the model.\footnote{Since all dependent variables are measured in natural logs, the marginal impact of a dummy coefficient $\beta$ is computed $\left(e^\beta-1\right) \times 100$ percent.} At the same time, there are 7.68\% fewer unique wallets (Model 5) providing liquidity on the low-fee pool.

\begin{table}[H] 
\caption{Fragmentation and order flow characteristics}  \label{tab:ordersize}
\begin{minipage}[t]{1\columnwidth}%
\footnotesize
			This table reports the coefficients of the following regression:
	\begin{align*}
    y_{ijt}=\alpha + \beta_0 d_\emph{low-fee, ij} + \beta_1 \text{GasPrice}_{jt} d_\emph{low-fee, ij} + \beta_2 \text{GasPrice}_{jt} \times d_\emph{high-fee, ij} + \sum \beta_k \emph{Controls}_{ijt} + \theta_j + \varepsilon_{ijt}
    \end{align*}
	where the dependent variable $y_{ijt}$ can be (i) the log median mint size, (ii) the log median trade size, (iii) the log trading volume, (iv) the log trade count $\log(1+\# trades)$, (v) the log count of unique $\LP$ wallets interacting with a pool in a given day, (vi) the liquidity yield in bps for pool $i$ in asset $j$ on day $t$, computed as in equation \eqref{eq:liq_yield}, and (vii) the average liquidity mint price range for pool $i$ in asset $j$ on day $t$. Price range is computed as the difference between the top and bottom of the range, normalized by the range midpoint -- a measure that naturally lies between zero and two. $d_\emph{low-fee, ij}$ is a dummy that takes the value one for the pool with the lowest fee in pair $j$ and zero else. $d_\emph{high-fee, ij}$ is defined as $1-d_\emph{low-fee, ij}$. $\emph{GasPrice}_{jt}$ is the average of the lowest 100 bids on liquidity provision events across all pairs on day $t$, standardized to have a zero mean and unit variance. \emph{Volume} is the natural logarithm of the sum of all swap amounts on day $t$, expressed in thousands of US dollars. \emph{Total value locked} is the natural logarithm of the total value locked on Uniswap v3 pools on day $t$, expressed in millions of dollars. \emph{Volatility} is computed as the daily range between high and low prices on the most active pool for a given pair. All regressions include pair and week fixed-effects. Robust standard errors in parenthesis are clustered by week and ***, **, and * denote the statistical significance at the 1, 5, and 10\% level, respectively.  The sample period is from May 4, 2021 to July 15, 2023. 
\end{minipage}
\begin{center}
\resizebox{1\textwidth}{!}{  
\begin{tabular}{lccccccc}
\toprule
 & \multicolumn{1}{c}{Mint size} &  \multicolumn{1}{c}{Trade size} & \multicolumn{1}{c}{Volume} & \multicolumn{1}{c}{\# Trades}  & \multicolumn{1}{c}{\# Wallets} & \multicolumn{1}{c}{Liquidity yield} & \multicolumn{1}{c}{Price range} \\
 & (1) & (2) & (3) & (4) & (5) & (6) & (7) \\
\cmidrule{1-8}
$ d_\text{low-fee}$ & 0.73*** & -0.30*** & 0.89*** & 1.02*** & -0.08*** & 2.03*** & -0.18*** \\
 & (12.25) & (-10.01) & (14.15) & (32.65) & (-2.85) & (3.60) & (-42.99) \\
Gas price $\times$ $ d_\text{low-fee}$ & 0.36*** & 0.08*** & -0.04 & -0.22*** & -0.21*** & 3.12** & -0.01 \\
 & (4.89) & (3.66) & (-0.98) & (-6.88) & (-9.83) & (2.04) & (-1.07) \\
Gas price $\times$ $ d_\text{high-fee}$ & 0.57*** & 0.16*** & 0.23*** & 0.07** & -0.07*** & 5.13** & -0.03*** \\
 & (7.31) & (8.83) & (5.64) & (2.37) & (-3.13) & (2.56) & (-4.79) \\
Volume & 0.38*** & 0.16*** & 0.43*** & 0.20*** & 0.05*** & 1.15 & -0.01*** \\
 & (8.81) & (21.29) & (15.46) & (14.03) & (5.43) & (0.94) & (-2.82) \\
Total value locked & -0.16 & 0.11*** & 0.23** & -0.01 & 0.01 & -13.51 & -0.01 \\
 & (-1.27) & (3.56) & (1.99) & (-0.18) & (0.29) & (-1.10) & (-0.42) \\
Volatility & -0.04 & -0.01 & -0.07 & 0.01 & -0.01 & 1.18** & 0.02*** \\
 & (-1.10) & (-1.35) & (-1.38) & (0.88) & (-1.38) & (2.21) & (3.98) \\
Constant & 1.89*** & 1.64*** & 5.27*** & 3.26*** & 1.26*** & 10.01*** & 0.61*** \\
 & (58.34) & (111.09) & (167.66) & (207.96) & (91.28) & (26.08) & (202.85) \\
 &  &  &  &  &  &  \\
Pair FE & Yes & Yes & Yes & Yes & Yes & Yes & Yes  \\
Week FE & Yes & Yes & Yes & Yes & Yes & Yes  & Yes \\
Observations & 21,000 & 36,059 & 36,059 & 40,288 & 40,288 & 40,252 & 25,135 \\
 R-squared & 0.26 & 0.53 & 0.55 & 0.52 & 0.55 & 0.09 & 0.44 \\ \hline
\bottomrule
\multicolumn{7}{l}{Robust t-statistics in parentheses. Standard errors are clustered at week level.} \\
\multicolumn{7}{l}{*** p$<$0.01, ** p$<$0.05, * p$<$0.1} \\
\end{tabular}
}
\end{center}
\end{table}


On the liquidity demand side, trades on the low-fee pool are 25.91\% smaller (Model 2), consistent with Prediction \ref{pred:trade_size_volume}. However, the low-fee pool executes almost three times the number of trades (i.e., trade count is 177\% higher from Model 4) and has 143\% higher volume than the high-fee pool (Model 3). On average, liquidity providers on low-fee pools earn 2.03 basis points higher revenue than their counterparts on high-fee pools (Model 6), indicating significant positive returns resulting from economies of scale.

Our findings (Model 7) indicate that liquidity providers on low-fee pools select price ranges that are 29\% (=0.18/0.61) narrower when minting liquidity compared to those on high-fee pools. This pattern aligns with the capability of large LPs to adjust their liquidity positions frequently, enabling more efficient capital concentration. Similarly, \citet{caparros2023blockchain} report a higher concentration of liquidity in pools on alternative blockchains like Polygon, known for lower transaction costs than Ethereum.

The results point to an asymmetric match between liquidity supply and demand across pools. On low-fee pools, a few $\LP$s provide large chunks of liquidity for the vast majority of incoming small trades. Conversely, on high-fee pools there is a sizeable mass of small liquidity providers that mostly trade against a few large incoming trades. 

How does variation in fixed transaction costs impact the gap between individual order size across pools? We find that increasing the gas price by one standard deviation leads to higher liquidity deposits on both the low- and the high-fee pools (13.7\% and 30.1\% higher, respectively).\footnote{The relative effects are computed as $\nicefrac{0.36}{(1.89+0.73)}=13.8\%$ for low pools and $\nicefrac{0.57}{1.89}=30.15\%$ for high-fee pools, respectively.} The result supports Prediction \ref{pred:clienteles_cs} of the model. Our theoretical framework implies that a larger gas price leads to some (marginal) $\LP$s switching from the low- to the high-fee pool. The switching $\LP$s have low capital endowments relative to their low-fee pool peers, but higher than $\LP$s on the high-fee pool. Therefore, the gas-driven reshuffle of liquidity leads to a higher average endowment on both high- and low-fee pools. Consistent with the model, a higher gas price leads to fewer active liquidity providers, particularly on low-fee pools. Specifically, a one-standard increase in gas costs leads to a substantial 17.8\% and 5.5\% decrease in the number of $\LP$ wallets interacting daily with low- and high-fee pools, respectively (Model 5). 

While a higher gas price is correlated with a shift in liquidity supply, it has a muted impact on liquidity demand on low-fee pools. A higher gas cost is associated with 6\% larger trades (Model 2), likely as traders aim to achieve better economies of scale. At the same time, the number of trades on the low-fee pool drops by 5.1\% (Model 4) -- since small traders might be driven out of the market. The net of gas prices effect on aggregate volume on the low-fee pool is small and not statistically significant (Model 3). The result matches our model assumption that the aggregate order flow on low-fee pool is not sensitive to gas prices.\footnote{Formally, one could extend the model to assume that small $\LT$s arrive at the market at rate $\tilde{\theta}\left(\Gamma\right) \diff t$ and demand $f\left(\Gamma\right)$ units each, where $\tilde{\theta}\left(\Gamma\right)$ decreases in $\Gamma$ and $f\left(\Gamma\right)$ increases in $\Gamma$ such that $\tilde{\theta}\left(\Gamma\right)f\left(\Gamma\right) =\theta$.} 

On the high-fee pool, a higher gas price is also associated with a higher trade size, but a much smaller relative increase in the number of trades. The implication is that large-size traders who typically route orders to the high-fee pool are unlikely to be deterred by an increase in fixed costs, while they do adjust quantities to achieve better economies of scale. In the context of our model, the large $\LT$ arrival rate $\lambda$ does not depend on the gas price $\Gamma$.

In Table \ref{tab:flows}, we shift the analysis from individual orders to aggregate daily liquidity flows to Uniswap pools. We find that higher gas prices lead to a decrease in liquidity inflows, but only on the low fee pools. A one standard deviation increase in gas prices leads to a 28.8\% drop in new liquidity deposits by volume (Model 1) and an 5.91\% drop in probability of having at least one mint (Model 4) on the low-fee pool. However, the slow-down in liquidity inflows is less evident in high fee pools. While an increase in gas prices reduce the probability of liquidity inflows by 1.50\%, it actually leads to a 11.6\% increase in the daily dollar inflow to the pool. Together with the result in Table \ref{tab:ordersize} that the size of individual mints increases with gas prices, our evidence is consistent with the model implication that higher fixed transaction costs change the composition of liquidity supply on the high-fee pool, with small $\LP$ being substituted by larger $\LP$s switching over from the low-fee pool.


\begin{table}[H] 
\caption{Liquidity flows and gas costs on fragmented pools} \label{tab:flows}
\begin{minipage}[t]{1\columnwidth}%
\footnotesize
			This table reports the coefficients of the following regression:
	\begin{align*}
    y_{ijt}=\alpha + \beta_0 d_\emph{low-fee, ij} + \beta_1 \text{GasPrice}_{jt} d_\emph{low-fee, ij} + \beta_2 \text{GasPrice}_{jt} \times d_\emph{high-fee, ij} + \sum \beta_k \emph{Controls}_{ijt} + \theta_j + \varepsilon_{ijt}
    \end{align*}
	where the dependent variable $y_{ijt}$ can be (i) the aggregate dollar value of mints (in logs), or (vi) a dummy variable taking value one hundred if there is at least one mint on liquidity pool $i$ in asset $j$ on day $t$. $d_\emph{low-fee, ij}$ is a dummy that takes the value one for the pool with the lowest fee in pair $j$ and zero else. $d_\emph{high-fee, ij}$ is defined as $1-d_\emph{low-fee, ij}$. $\emph{GasPrice}_{jt}$ is the average of the lowest 100 bids on liquidity provision events across all pairs on day $t$, standardized to have a zero mean and unit variance. \emph{Volume} is the natural logarithm of the sum of all swap amounts on day $t$, expressed in thousands of US dollars. \emph{Total value locked} is the natural logarithm of the total value locked on Uniswap v3 pools on day $t$, expressed in millions of dollars.\emph{Volatility} is computed as the daily range between high and low prices on the most active pool for a given pair.All regressions include pair and week fixed-effects. Robust standard errors in parenthesis are clustered by week and ***, **, and * denote the statistical significance at the 1, 5, and 10\% level, respectively.  The sample period is from May 4, 2021 to July 15, 2023. 
\end{minipage}
\begin{center}
\begin{tabular}{lcccccc}
\toprule
 & \multicolumn{3}{c}{Daily mints (log US\$)} &  \multicolumn{3}{c}{$\text{Prob}\left(\text{at least one mint}\right)$} \\
 & (1) & (2) & (3) & (4) & (5) & (6) \\
\cmidrule{1-7}
$ d_\text{low-fee}$ & 0.43*** & 0.43*** & 0.43*** & 1.37* & 1.37* & 1.37* \\
 & (6.01) & (6.01) & (6.01) & (1.69) & (1.69) & (1.69) \\
Gas price $\times$ $ d_\text{low-fee}$ & -0.34*** & -0.34*** & -0.45*** & -5.91*** & -5.91*** & -4.41*** \\
 & (-8.22) & (-8.22) & (-6.78) & (-8.64) & (-8.64) & (-6.30) \\
Gas price $\times$ $ d_\text{high-fee}$ & 0.11** & 0.11** &  & -1.50** & -1.50** &  \\
 & (2.02) & (2.02) &  & (-2.57) & (-2.56) &  \\
Volume & 0.26*** & 0.26*** & 0.26*** & 0.90*** & 0.91*** & 0.90*** \\
 & (14.74) & (14.73) & (14.74) & (3.73) & (3.73) & (3.73) \\
Total value locked & -0.07 & -0.07 & -0.07 & 1.44 & 1.44 & 1.44 \\
 & (-0.78) & (-0.78) & (-0.78) & (0.99) & (0.99) & (0.99) \\
Volatility & -0.01 &  & -0.01 & 0.26 &  & 0.26 \\
 & (-0.68) &  & (-0.68) & (0.59) &  & (0.59) \\
Gas price &  &  & 0.11** &  &  & -1.50** \\
 &  &  & (2.02) &  &  & (-2.57) \\
Constant & 2.61*** & 2.61*** & 2.61*** & 51.44*** & 51.43*** & 51.44*** \\
 & (72.72) & (72.75) & (72.72) & (125.29) & (125.89) & (125.29) \\
 &  &  &  &  &  &  \\
Pair FE & Yes & Yes & Yes & Yes & Yes & Yes  \\
Week FE & Yes & Yes & Yes & Yes & Yes & Yes  \\
Observations & 40,288 & 40,288 & 40,288 & 40,288 & 40,288 & 40,288 \\
 R-squared & 0.47 & 0.47 & 0.47 & 0.28 & 0.28 & 0.28 \\ \hline
\bottomrule
\multicolumn{7}{l}{Robust t-statistics in parentheses. Standard errors are clustered at week level.} \\
\multicolumn{7}{l}{*** p$<$0.01, ** p$<$0.05, * p$<$0.1} \\
\end{tabular}
\end{center}
\end{table}



\subsection{Liquidity cycles on high- and low-fee pools}

Next, we test Predictions \ref{pred:updates} and \ref{pred:updates_gas} on the duration of liquidity cycles on fragmented pools. Since the descriptive statistics in Table \ref{tab:sumstat} suggest that $\LP$s manage their positions over multiple days, we cannot accurately measure liquidity cycles in a pool-day panel. Instead, we use intraday data on liquidity events (either mints or burns) to measure the duration between two consecutive opposite-sign interactions by the same Ethereum wallet with a liquidity pool: either a mint followed by a burn, or vice-versa. 

To complement our previous analysis, we additionally control for whether each liquidity position is out of range (i.e., the price range set by the $\LP$ does not straddle the current price and therefore the $\LP$ does not earn fees). We further introduce wallet fixed effects to soak up variation in reaction times across traders.

Table \ref{tab:cycles} presents the results. Liquidity updates on decentralized exchanges are very infrequent, as times elapsed between consecutive interactions are measured in days or even weeks. In line with Prediction \ref{pred:updates}, we find evidence for shorter liquidity cycles on low-fee pools. The average time between consecutive mint and burn orders is 21.3\% shorter on the low-fee pool (from Model 1, the relative difference is 86.59 hours/405.78 hours).

Liquidity cycles are in part driven by fixed Blockchain transaction costs. A one standard deviation increase in gas prices speeds up the liquidity cycle on low-fee pools by a further 15.25 hours, a result which supports the intuition behind Prediction \ref{pred:updates_gas}. When the gas price spikes, the liquidity supply on the low-fee pool decreases at a higher rate than the liquidity demand. As a result, liquidity deposits deplete faster (i.e., they move outside the fee-earning range), triggering the need for more frequent updates. At the same time, a higher gas fee speeds up the liquidity cycle on the high-fee pool as well. The result is consistent with small liquidity providers on the high-fee pool being crowded out by the high fixed costs, leading to a lower liquidity supply.



\begin{table}[H] 
\caption{Liquidity cycles on fragmented pools} \label{tab:cycles}
\begin{minipage}[t]{1\columnwidth}%
\footnotesize
			This table reports the coefficients of the following regression:
	\begin{align*}
    y_{ijtk}=\alpha + \beta_0 d_\emph{low-fee, ij} + \beta_1 \text{GasPrice}_{jt} d_\emph{low-fee, ij} + \beta_2 \text{GasPrice}_{jt} \times d_\emph{high-fee, ij} + \sum \beta_k \emph{Controls}_{ijt} + \theta_j + \varepsilon_{ijt}
    \end{align*}
	where the dependent variable $y_{ijt}$ can be (i) the mint-to-burn time, (ii) the burn-to-mint time, measured in hours, for a transaction initiated by wallet $k$ on day $t$ and pool $i$ trading asset $j$. The mint-to-burn and burn-to-mint times are computed for consecutive interactions of the same wallet address with the liquidity pool. $d_\emph{low-fee, ij}$ is a dummy that takes the value one for the pool with the lowest fee in pair $j$ and zero else. $d_\emph{high-fee, ij}$ is defined as $1-d_\emph{low-fee, ij}$. $\emph{GasPrice}_{jt}$ is the average of the lowest 100 bids on liquidity provision events across all pairs on day $t$, standardized to have a zero mean and unit variance. \emph{Volume} is the natural logarithm of the sum of all swap amounts on day $t$, expressed in thousands of US dollars. \emph{Total value locked} is the natural logarithm of the total value locked on Uniswap v3 pools on day $t$, expressed in millions of dollars. \emph{Volatility} is computed as the daily range between high and low prices on the most active pool for a given pair. \emph{Position out-of-range} is a dummy taking value one if the position being burned or minted is out of range, that is if the price range selected by the $\LP$ does not straddle the current pool price. All variables are measured as of the time of the second leg of the cycle (i.e., the burn of a mint-burn cycle). All regressions include pair, week, and trader wallet fixed-effects. Robust standard errors in parenthesis are clustered by day and ***, **, and * denote the statistical significance at the 1, 5, and 10\% level, respectively.  The sample period is from May 4, 2021 to July 15, 2023. 
\end{minipage}
\begin{center}
\begin{tabular}{lcccccc}
\toprule
 & \multicolumn{3}{c}{Mint-burn time} & \multicolumn{3}{c}{Burn-mint time} \\
 & (1) & (2) & (3) & (4) & (5) & (6)  \\
\cmidrule{1-7}
$ d_\text{low-fee}$ & -80.51*** & -80.92*** & -86.59*** & -130.65*** & -131.87*** & -132.04*** \\
 & (-8.43) & (-8.53) & (-9.19) & (-10.46) & (-10.64) & (-10.64) \\
Gas price $\times$ $ d_\text{low-fee}$ & -18.83*** & -15.33** & -15.25** & -4.06 & 4.38 & 4.46 \\
 & (-2.70) & (-2.21) & (-2.20) & (-0.75) & (0.72) & (0.73) \\
Gas price $\times$ $ d_\text{high-fee}$ & -19.32*** & -16.23*** & -16.17*** & -2.17 & 5.15 & 5.20 \\
 & (-3.28) & (-2.63) & (-2.65) & (-0.49) & (1.09) & (1.10) \\
Volume &  & -8.91** & -10.33*** &  & -16.70*** & -16.90*** \\
 &  & (-2.62) & (-3.02) &  & (-3.60) & (-3.65) \\
Total value locked &  & -27.62 & -26.21 &  & -19.00 & -19.03 \\
 &  & (-1.02) & (-0.98) &  & (-0.92) & (-0.92) \\
Volatility &  & -2.01*** & -2.22*** &  & -2.87*** & -2.89*** \\
 &  & (-3.17) & (-3.40) &  & (-3.38) & (-3.41) \\
Position out-of-range &  &  & 66.69*** &  &  & 16.51** \\
 &  &  & (12.92) &  &  & (2.51) \\
Constant & 447.50*** & 447.43*** & 405.78*** & 210.59*** & 211.80*** & 208.58*** \\
 & (98.21) & (97.32) & (70.68) & (31.40) & (31.72) & (31.27) \\
 &  &  &  &  &  &  \\
 Pair FE & Yes & Yes & Yes & Yes & Yes & Yes  \\
 Week FE & Yes & Yes & Yes & Yes & Yes & Yes  \\
 Trader wallet FE & Yes & Yes & Yes & Yes & Yes & Yes \\
Observations & 459,112 & 459,110 & 459,110 & 317,894 & 317,894 & 317,894 \\
 R-squared & 0.81 & 0.81 & 0.81 & 0.34 & 0.34 & 0.34 \\ \hline
\bottomrule
\multicolumn{7}{l}{Robust t-statistics in parentheses. Standard errors are clustered at week level.} \\
\multicolumn{7}{l}{*** p$<$0.01, ** p$<$0.05, * p$<$0.1} \\
\end{tabular}

\end{center}
\end{table}


The reaction time of a trader may depend on capital constraints, network congestion, and other confounding factors that can correlate with gas prices. As a robustness check, we repeat the analysis above with burn-to-mint times as the dependent variables. The burn-to-mint time measures the speed at which $\LP$s deposit liquidity at updated prices after removing (out-of-range) positions, and should not depend on the rate at which liquidity is consumed. Consistent with the theory, we find no significant relationship between gas price and the burn-to-mint duration (Models 4 through 6).



\subsection{Adverse selection costs across low- and high-fee pools}

Our findings indicate that larger aggressive orders tend to execute on high-fee liquidity pools. Given that, at least in equity markets, larger incoming orders tend to be better informed \citep{Hasbrouck1991}, a natural question is whether liquidity providers on high-fee pools face higher adverse selection. 

A widely-used measure for gauging informational costs for liquidity providers on decentralized exchanges employing automated market makers (AMMs) is the ``impermanent loss,'' the equivalent of adverse selection measures in traditional limit order markets  \citep[see, for example,][]{aoyagi2020,BarbonRanaldo2021}. The impermanent loss (IL) is defined as the negative return from providing liquidity as opposed to holding the assets outside the exchange and marking them to market as the price evolves. Intuitively, if the fundamental value of the asset changes, an arbitrageur trades at the stale price in the direction of the price change, thus minimizing the value of the liquidity pool \citep{zhang2023amm}. 

To illustrate, let's consider a straightforward example. At some initial time $t_0$, a liquidity provider deposits 1 ETH (i.e., a token quantity of $T_0=1$) and 3000 USDC (i.e., a numeraire $N_0=3000$) into a liquidity pool, establishing a token price of $p_0=3000$.  If at the next time interval ($t=1$), the intrinsic value surges to $v_1=3500$, an arbitrager has the incentive to remove ETH from the pool and deposit USDC until the price reflects the intrinsic value, or $\frac{N_1}{T_1}=3500$.

Given that the AMM requires the product of token and numeraire quantities to be constant ($T_0 N_0 = T_1 N_1$), this leads to $T_1 = 0.926$ and $N_1=3240.37$. In terms of the numeraire equivalent, the liquidity provider's position is valued at $V_\text{pos}=\$3500 \times 0.926 + \$3240.37 = \$6480.74$.  Had the liquidity provider not deposited their tokens on the exchange and instead held it in its own account, the value would be $V_\text{hold}=\$3500 \times 1 + \$3000 = \$6500$ (that is, equal to $T_0 \times p_1 + N_0$).

Hence, disregarding gas cost and liquidity fees, the impermanent loss can be calculated as:
\begin{equation}
    \text{ImpermanentLoss}=\frac{V_\text{hold}-V_\text{pos}}{V_\text{hold}}=29 \text{ bps}.
\end{equation}

However, liquidity providers on Uniswap V3 pools can set a price range for their orders. While this feature caps the potential adverse selection cost, it simultaneously introduces a new layer of computational complexity. In Appendix \ref{app:IL}, we present the exact formulas for calculating impermanent loss on Uniswap V3, based on the methodology described by \citet{Heimbach2023}. 

We obtain hourly liquidity snapshots from the Uniswap V3 Subgraph, which allow us to calculate the impermanent loss for symmetric liquidity positions within a price range of $\left[\frac{1}{\alpha} p, \alpha p\right]$ centered around the current pool price $p$. We consider different ranges of $\alpha$, specifically $\alpha\in\left\{1.01, 1.05, 1.1, 1.25\right\}$. The liquidity management horizon, which represents the time delay between measuring $p_0$ and $p_1$, is set to one hour.

Table \ref{tab:il} presents the empirical results. In Models 1, 3, 5, and 7, where we exclude control variables, the impermanent loss appears to be higher in low-fee pools. However, the link is mechanical: since high-fee pools exhibit lower trading activity, the price updates less frequently leading to a lower measured impermanent loss. Once we account for trading activity in our analysis (Models 2, 4, 6, and 8), we find that impermanent loss is between 1.09 and 1.85 basis points lower in low-fee pools, although the effect is not statistically significant in all specifications. This finding aligns with the intuition that larger incoming orders on high-fee pools tend to impose higher adverse selection costs on liquidity providers.

\begin{table}[H] 
\scriptsize
\caption{Impermanent loss across high- and low-fee pools}\label{tab:il}
\begin{minipage}[t]{1\columnwidth}%
\footnotesize
This table reports the coefficients of the following regression:
	\begin{align*}
    \text{Impermanent Loss}_{ijt}=\alpha + \beta_0 d_\emph{low-fee, ij}  + \sum \beta_k \emph{Controls}_{ijt} + \theta_j + \varepsilon_{ijt}
    \end{align*}
	where the dependent variable is the daily average impermanent loss for a liquidity position with price range $\left[\frac{p}{\alpha}, \alpha p\right]$ around the current pool price $p$, for $\alpha\in\left\{1.01, 1.05, 1.1., 1.25\right\}$. The average impermanent loss is computed across all Ethereum blocks mined within the day. To compute the impermanent loss, we use a liquidity provider horizon of one hour: that is, we compare the current pool price with the pool price one hour later. $d_\emph{low-fee, ij}$ is a dummy that takes the value one for the pool with the lowest fee in pair $j$ and zero else. $\emph{GasPrice}_{jt}$ is the average of the lowest 100 bids on liquidity provision events across all pairs on day $t$, standardized to have a zero mean and unit variance. $\emph{TradeCount}_{jt}$ is number of trades on pool $j$ and day $t$, standardized to have a zero mean and unit variance. \emph{Volume} is the natural logarithm of the sum of all swap amounts on day $t$, expressed in thousands of US dollars. \emph{Total value locked} is the natural logarithm of the total value locked on Uniswap v3 pools on day $t$, expressed in millions of dollars. \emph{Volatility} is computed as the daily range between high and low prices on the most active pool for a given pair. All regressions include pair and week fixed-effects. Robust standard errors in parenthesis are clustered by week and ***, **, and * denote the statistical significance at the 1, 5, and 10\% level, respectively.  The sample period is from May 4, 2021 to July 15, 2023. 
\end{minipage}

\begin{center}
\resizebox{1\textwidth}{!}{  
\begin{tabular}{lcccccccc}
\toprule
& \multicolumn{8}{c}{Impermanent loss for a liquidity position with range $\left[\frac{p}{\alpha}, \alpha p\right]$ around price $p$} \\
\cmidrule{1-9} 
& \multicolumn{2}{c}{$\alpha=1.01$} & \multicolumn{2}{c}{$\alpha=1.05$} & \multicolumn{2}{c}{$\alpha=1.10$} & \multicolumn{2}{c}{$\alpha=1.25$} \\
 & (1) & (2) & (3) & (4) & (5) & (6) & (7) & (8) \\
\cmidrule{1-9}
$ d_\text{low-fee}$ & 2.59*** & -1.38 & 1.08*** & -1.85** & 0.71*** & -1.56** & 0.37** & -1.09* \\
 & (11.26) & (-1.57) & (5.72) & (-2.28) & (4.28) & (-2.18) & (2.58) & (-1.98) \\
Gas price &  & 4.43*** &  & 3.43*** &  & 2.53*** &  & 1.41*** \\
 &  & (3.70) &  & (3.67) &  & (3.52) &  & (3.04) \\
Trade count &  & 4.82*** &  & 3.56*** &  & 2.78*** &  & 1.79*** \\
 &  & (4.59) &  & (3.72) &  & (3.30) &  & (2.83) \\
Volume &  & 3.15*** &  & 1.28*** &  & 0.67*** &  & 0.26 \\
 &  & (7.10) &  & (4.04) &  & (2.66) &  & (1.47) \\
Total value locked &  & 0.38 &  & 1.75 &  & 1.99 &  & 1.80 \\
 &  & (0.14) &  & (0.77) &  & (1.04) &  & (1.36) \\
Volatility &  & 6.98*** &  & 6.65** &  & 6.39** &  & 6.06** \\
 &  & (2.69) &  & (2.59) &  & (2.51) &  & (2.40) \\
Constant & 15.52*** & 15.87*** & 7.37*** & 7.65*** & 4.63*** & 4.88*** & 2.45*** & 2.66*** \\
 & (134.72) & (102.93) & (77.84) & (60.06) & (55.47) & (43.72) & (34.58) & (29.31) \\
 &  &  &  &  &  &  \\
Pair FE & Yes & Yes & Yes & Yes & Yes & Yes & Yes  & Yes \\
Week FE & Yes & Yes & Yes & Yes & Yes & Yes  & Yes & Yes \\
Observations & 40,250 & 40,248 & 40,250 & 40,248 & 40,250 & 40,248 & 40,250 & 40,248 \\
 R-squared & 0.17 & 0.23 & 0.09 & 0.15 & 0.06 & 0.11 & 0.03 & 0.08 \\ \hline
\bottomrule
\multicolumn{9}{l}{Robust t-statistics in parentheses. Standard errors are clustered at week level.} \\
\multicolumn{9}{l}{*** p$<$0.01, ** p$<$0.05, * p$<$0.1} \\
\end{tabular}
}
\end{center}
\end{table}