This paper argues that fixed costs associated with liquidity management drive a wedge between large (institutional) and small (retail) market makers. In the context of blockchain-based decentralized exchanges, the most evident fixed cost is represented by gas fees, where market makers compensate miners and validators for transaction processing in proof-of-work, respectively in proof-of-stake blockchains. Innovative solutions such as Proof of Stake (PoS) consensus algorithms and Layer 2 scaling aim to address the concern of network costs. However, even if gas fees were eliminated entirely, individual retail traders still encounter disproportionate fixed costs in managing their liquidity, such as the expenditure of time and effort.

Our paper highlights a trade-off between capital efficiency and the fixed costs of active management. During the initial phase of decentralized exchanges, such as Uniswap V2, liquidity providers were not able to set price limits, resulting in an even more passive liquidity supply and fewer incentives for active position management. However, the mechanism implied that incoming trades incurred significant price impact. To enhance the return on liquidity provision and reduce price impact on incoming trades, modern decentralized exchanges (DEXs) have evolved to enable market makers to fine-tune their liquidity positions, albeit at the expense of more active management.

We show, both theoretically and empirically, that fixed costs of liquidity management promote market fragmentation across decentralized pools and generate clienteles of liquidity providers. Large market makers, likely institutions and funds, have stronger economies of scale and can afford to frequently manage their positions on very active low fee markets. On the other hand, smaller retail liquidity providers become confined to high fee markets with scant activity, trading off a lower execution probability against lower gas costs to update their positions. Since large liquidity providers can churn their position at a faster pace, two thirds of the trading volume interacts with less than half the capital locked on Uniswap V3.

Our findings indicate that substantial fixed costs can hinder the participation of small market makers in the forefront of liquidity provision, where active order management is crucial. Instead, smaller liquidity providers tend to operate on the market maker ``fringe,'' opting for a lower execution probability in exchange for better prices. The results are particularly relevant the context of a resurgence in retail trading activity and the ongoing evolution of technology that fosters market structures aimed at enhancing broader access to financial markets.



