\subsection{Primitives}

\paragraph{Token.} A single token \textbf{T} is traded in a continuous-time economy. There are two trader types in the economy: a continuum of liquidity providers ($\LP$s) and liquidity takers ($\LT$s). Both market participant types are risk neutral.  

The expected value of the token equals $v>0$, which is common knowledge. However, market participants have heterogeneous private values for the asset. In particular, liquidity takers value the token at $v\left(1+\Delta\right)$ with $\Delta>0$ whereas liquidity providers have no private value for the token. Therefore, there are gains from trade if $\LT$s buy the token from $\LP$s \citep[we follow, e.g.,][in that we focus on a one-sided market where liquidity takers only act as buyers]{foucault2013liquidity}.

\paragraph{Liquidity providers.} Liquidity providers have heterogeneous token endowments, which can be thought of as heterogeneous capital constraints: $\LP$ $i$ can supply at most $q_i$ units of the token, where $q_i$ has a bounded Pareto distribution on $\left[1,Q\right]$ with $Q>1$:
\begin{equation}\label{eq:density}
    \varphi\left(q\right)=\frac{Q}{Q-1} \frac{1}{q^2} \text{ for } q\in\left[1,Q\right].
\end{equation}

Figure \ref{fig:distribution} illustrates the theoretical distribution of $\LP$s endowments for $Q\in\left\{3,4\right\}$. A Pareto distribution translates to a right skew in liquidity provider endowments and captures an environment with a large number of low-endowment liquidity providers such as retail traders and a handful of high-capital $\LP$s -- for example, sophisticated quantitative funds. A larger $Q$ corresponds to a higher degree of $\LP$ heterogeneity, i.e., to a higher dispersion of endowments. On aggregate, $\LP$s can provide at most
\begin{equation}
    \int_1^Q q \varphi\left(q\right) \diff q = \frac{Q}{Q-1} \log Q
\end{equation}
units of liquidity. 

% Figure environment removed


\paragraph{Liquidity takers.} There are two types of liquidity takers: small and large. A flow of small $\LT$s arrive at the market at constant rate $\theta \diff t$ and demand one unit of the token each. The large $\LT$ arrival time follows a Poisson process with rate $\lambda>0$. Conditional on arrival, a large $\LT$ demands $\Theta$ units of the token, where $\Theta>\frac{Q\log Q}{Q-1}$. That is, we assume that the large $\LT$ liquidity demand exceeds the maximum liquidity supply.  The aggregate liquidity demand from small and large $\LT$s evolves over time as
\begin{equation}
    \diff \emph{LiquidityDemand}_t = \theta \diff t + \Theta \diff J_t\left(\lambda\right),
\end{equation}
where $J_t\left(\lambda\right)$ stands in for a Poisson process with rate $\lambda$.

\paragraph{Trading environment.} The token is traded on two liquidity pools against a numeraire asset (cash). The token and numeraire quantities ($T$ and $N$, respectively) in each pool must satisfy at all times the following linear bonding curve:
\begin{equation}\label{eq:bond_curve}
    v \times T_k + N_k = \text{constant}, \forall k\in\left\{L,K\right\}
\end{equation}
The bonding curve \eqref{eq:bond_curve} implies that the token price is constant and equal to $v$ units of the num\'{e}raire, since $-\frac{\diff T_k}{\diff N_k}=v$.\footnote{In practice, most decentralized exchanges use convex bonding curve, for example constant product pricing. While the choice of bonding curve is of first order importance for liquidity takers who need to split trades to optimize price impact, we argue it is less important for liquidity suppliers. For tractability, we assume a linear bonding curve which implies that liquidity traders only care about pool fees -- the focus of our model. In a richer model with convex pricing, we would expect to see relatively more fragmentation since minimizing price impact might require further splitting trades.} 

Further, each pool charges a liquidity fee as a fraction of the value of the trade. The low-fee pool $L$ charges a fee of $\ell$ and the high-fee pool $H$ charges a fee of $h$ per token traded, where $\ell<h$. Fees are levied on liquidity takers and distributed in full to liquidity providers on a pro-rata basis. For example, to purchase $\tau$ units of the token on the $L$ pool, the total cost of an investor is $\tau\left(v+\ell\right)$, whereas $\LP$s receive $\tau\ell$ in fees. 

\paragraph{Gas costs.} Any interaction with a liquidity pool (for example, trading or managing liquidity)  incurs a fixed execution cost $\Gamma>0$. The execution cost can be interpreted as the gas price on the Ethereum blockchain as in \citet{CapponiJia2021}. In contrast to \citet{CapponiJia2021}, we assume traders on the decentralized exchanges are price takers on the gas market and their orders do not impact the gas cost. A natural interpretation is that message volume at any time in a particular asset pair traded on decentralized exchanges is small relative to the aggregate transaction flow on the entire blockchain.

Gas fees on decentralized exchanges differ from trading fees on traditional exchanges in a number of ways. First, they are not set by individual exchanges to compete with each other, but are a common transaction cost \emph{across} exchanges. Second, they are levied on a per-order rather than per-share basis: this implies economies of scale for larger orders. Third, there is significant time variation in gas fees which allows us to better empirically identify the impact of transaction costs on liquidity pool market shares.

To ensure that both small and large liquidity traders participate in the market, we assume that the gains from trade are larger than the aggregate transaction cost, including the pool fee and the gas price:
\begin{equation}
    v\left(1+\Delta\right)-v > v\left(1+h\right) + \Gamma \Rightarrow \Delta > h + \frac{\Gamma}{v}.
\end{equation}

To rule out trivial cases, we further assume that $Q\ell-\Gamma>0$. The condition ensures that there are at least some liquidity providers who can earn positive expected profit on pool $L$; otherwise, pool $L$ never attracts any deposits.

\paragraph{Timing.} We follow \citet{foucault2013liquidity} and partition the continuous timeline in \emph{liquidity cycles}. A liquidity cycle starts with an empty pool (zero liquidity offered) which triggers $\LP$ token deposits and ends when incoming trades deplete the liquidity supply, returning the pool to an empty state. The first cycle starts at $t=0$: Each liquidity provider $i$ selects a pool $k\in\left\{L,H\right\}$, pays the gas cost $\Gamma$, and deposits their token endowment $q_i$. Liquidity providers also have the option to not provide any liquidity, that is $k=\emptyset$. Following the initial liquidity deposit, $\LP$s do not further interact with the pool until the their liquidity is consumed at some random time $\tilde{\tau}$. The liquidity cycle restarts at $\tilde{\tau}$: $\LP$s use the sales cash proceeds to borrow or purchase $q_i$ tokens and refill liquidity on decentralized exchange pools. Figure \ref{fig:timing} illustrates the model timing.


% Figure environment removed

\subsection{Equilibrium analysis}\label{sec:liqprov}

An equilibrium in our trading game consists of (i) a liquidity pool choice for small and large $\LT$s upon arrival, as well as (ii) a mapping from each $\LP$'s endowment $q_i$ to a liquidity pool $k\in\left\{L,H, \emptyset \right\}$, such that no $\LP$ or $\LT$ is better off by deviating and choosing a different action. 

Conditional on making a deposit, liquidity providers choose the pool $k^\star$ that maximizes their expected profit per unit of time, that is
\begin{equation}\label{eq:optimal_pool}
    k^\star\left(q_i\right) \coloneqq \arg\max_{k} \left(q_i f_k - \Gamma\right) \frac{1}{d_k}, 
\end{equation}
where $f_k$ is the liquidity fee on pool $k$, and $d_k$ is the duration of a liquidity cycle on pool $k$. The revenue per liquidity cycle is equal to the pool liquidity fee times the deposit $q_i$, net of the gas cost to process the order.

Let $\Omega_k$ denote the set of $\LP$s who choose pool $k$ in equilibrium, that is $\Omega_k\coloneqq\left\{i \mid k^\star\left(q_i\right)=k\right\}$. We define the total liquidity on pool $k$ at the start of a cycle as the aggregate deposit from all liquidity providers in $\Omega_k$:
\begin{equation}
    \mathcal{L}_k=\int_{i\in\Omega_k} q_i \varphi(q_i) \diff i.
\end{equation}

Before solving for the equilibrium $\LP$ pool choice, we first analyze the liquidity traders' decisions. A large $\LT$'s liquidity demand exceeds the aggregate $\LP$ token endowment: upon arrival, the large $\LT$ therefore consumes the entire liquidity on both pools and triggers the end of the liquidity cycle. In contrast, small $\LT$s trade smoothly over time and consume liquidity from the cheaper pool $L$ at rate $\frac{\theta}{\mathcal{L}_L} \diff t$. Once the low-fee pool is empty, liquidity providers immediately refill it and restart the cycle.

\begin{lem}\label{lem:duration}
\begin{leftbar} \setlength{\parskip}{0ex}
The duration of a liquidity cycle is lower in pool $L$ than in pool $H$. Formally, $d_L<d_H$.
\end{leftbar}
\end{lem}

Lemma \ref{lem:duration} states that liquidity cycles in pool $L$ are shorter than in pool $H$. The result is intuitive: From the discussion above, a liquidity cycle on pool $H$ only ends with the arrival of a large liquidity trader: from the properties of Poisson processes, the expected cycle duration $d_H$ equals $\frac{1}{\lambda}$. Conversely, a liquidity cycle on pool $L$ ends either upon the large $\LT$ arrival or when small $\LT$s exhaust the pool liquidity. The expected duration of a cycle is
\begin{align}
    d_L &=\exp\left(-\lambda \frac{\LL}{\theta}\right) \frac{\LL}{\theta}+\int_{0}^{\frac{\LL}{\theta}} t\lambda\exp\left(-\lambda t\right)\diff t \nonumber \\
    &=\frac{1}{\lambda}-\frac{1}{\lambda}\exp\left(-\frac{\LL}{\theta}\lambda\right)<\frac{1}{\lambda}=d_H.
\end{align}

Liquidity providers on pool $L$ absorb a larger share of the order flow, as they trade with both small and large $\LT$s. However, they earn a lower fee per traded volume than they would on pool $H$. Furthermore, they need to manage their liquidity more often, which implies larger gas costs per unit of time. From equation \eqref{eq:optimal_pool}, liquidity provider $i$ chooses the low-fee pool $L$ if and only if
\begin{equation}\label{eq:cost_comparison}
    \pi_L-\pi_H=\frac{1}{d_H d_L}\left[\left(d_H \ell - d_L h\right) q_i - \Gamma \left(d_H-d_L\right)\right]>0
\end{equation}
If $\frac{h}{d_H}>\frac{\ell}{d_L}$, that is if the expected liquidity fee per unit of time on pool $H$ is higher than on pool $L$, all liquidity providers choose pool $H$. Otherwise, $\LP$s face a trade-off between a higher liquidity fee per unit of time on pool $L$ versus lower gas costs to manage liquidity on pool $H$. A salient implication of equation \eqref{eq:cost_comparison} is that $\LP$s with larger endowments are more likely to choose the low-fee pool, keeping liquidity shares constant. The rationale is that gas fees are a fixed cost, and larger $\LP$s are more likely to achieve economies of scale. The expected fee revenue scales linearly with a liquidity provider's share in the liquidity pool (proportional to $q_i$), better compensating $\LP$s for the gas fee paid at the start of the liquidity cycle. 

We follow \citet{KatzShapiro1985} and conjecture that there exists a marginal liquidity provider $\qmg$ such that all $\LP$s with $q_i>\qmg$ post liquidity on the low-fee pool and all $\LP$s with $q_i\leq \qmg$ choose the high-fee pool.\footnote{We note that there cannot be a ``reversed-sign'' equilibrium, where $\LP$s with $q_i\leq\qmg$ post liquidity on the low-fee pool and all $\LP$s with $q_i>\qmg$ choose the high-fee pool. To understand the rationale, we point out that for an $\LP$ with endowment $\tilde{q}<\qmg$, submitting liquidity to pool $L$ yields a lower expected revenue since
\begin{equation*}
    \pi_L-\pi_H=\frac{1}{d_H d_L}\left[\left(d_H \ell - d_L h\right) \tilde{q} - \Gamma \left(d_H-d_L\right)\right] < 0.
\end{equation*}
Therefore, it is optimal for the $\LP$ with endowment $\tilde{q}$ to post liquidity on the high-fee pool, which contradicts the conjectured equilibrium.  Importantly, since each $\LP$ is ``small'' relative to the aggregate mass of liquidity providers, individual deviations do not impact aggregate pool sizes $\LL$ and $\LH$.} Further, some liquidity providers might stay out of the market altogether. An $\LP$ only provides liquidity if she is able to break even on the high-fee pool -- that is, if her endowment $q_i$ is large enough. The participation constraint follows from equation \eqref{eq:optimal_pool}:
\begin{equation}\label{eq:pc}
    q_i h - \Gamma =\geq 0 \Rightarrow q_i>\underline{q}\coloneqq \max\left\{\frac{\Gamma}{h},1\right\}.
\end{equation}
If $\frac{\Gamma}{h}>1$, liquidity provision is competitive: all $\LP$s with $q_i>\underline{q}$ enter the market, and the marginal entrant earns zero expected profit. Conversely, if $\frac{\Gamma}{h}\leq 1$, capital scarcity in the economy implies that liquidity provision is not competitive; all $\LP$s enter the market and earn strictly positive profits. 

The conjectured equilibrium pins down the pool sizes $\LL$ and $\LH$ as a function of the endowment for the marginal $\LP$:
\begin{align}\label{eq:liquidity_levels}
    \LL&=\int_{\qmg}^Q q_i \varphi\left(q_i\right) \diff i = \frac{Q}{Q-1}\left(\log Q - \log \qmg \right)  \text{ and }\nonumber \\
    \LH&=\int_{\underline{q}}^{\qmg}  q_i \varphi\left(q_i\right) \diff i = \frac{Q}{Q-1}\left(\log \qmg - \log \underline{q}\right)
\end{align}
From equations \eqref{eq:cost_comparison} through \eqref{eq:liquidity_levels} it follows that the expected profit difference between the low- and high-fee pools can be written as an increasing function of the marginal $\LP$'s endowment, that is
\begin{align}\label{eq:pi_diff}
    \pi_L-\pi_H &=\frac{1}{\lambda d_H d_L}\left[\exp\left(-\frac{\lambda}{\theta} \LL\right) \underbrace{\left(q_i h - \Gamma\right)}_{>0} - q_i\left(h-\ell\right)\right] \nonumber \\
        &=\frac{1}{\lambda d_H d_L}\left[\qmg^{\frac{\lambda}{\theta}\frac{Q}{Q-1}} \times Q^{-\frac{\lambda}{\theta}\frac{Q}{Q-1}} \underbrace{\left(q_i h - \Gamma\right)}_{>0}- q_i\left(h-\ell\right)\right]. 
\end{align}
We search for fulfilled expectation equilibria \citep[following the definition of][]{KatzShapiro1985}, where the beliefs of $\LP$s at $t=0$ about the size of liquidity pools are consistent with the outcome of their choices. Proposition \ref{prop:equilibria} characterizes the equilibrium liquidity provision.

\begin{prop}\label{prop:equilibria}
\begin{leftbar} \setlength{\parskip}{0ex}
\emph{(Fragmentation)} If $\frac{h-l}{h} Q^{\frac{\lambda}{\theta}\frac{Q}{Q-1}}<1$ and $\frac{\Gamma}{h}<1-\frac{h-l}{h} Q^{\frac{\lambda}{\theta}\frac{Q}{Q-1}}$, then there exists a unique corner equilibrium where all $\LP$s deposit liquidity on pool $L$. If $\Gamma>Q\ell$, then all $\LP$s deposit liquidity on pool $H$. Otherwise, there exists a unique fragmented equilibrium characterized by marginal trader $\qmg^\star$ which solves
\begin{equation}\label{eq:mg_eq}
    \qmg = \Gamma \frac{\qmg^{\frac{\lambda}{\theta}\frac{Q}{Q-1}} \times Q^{-\frac{\lambda}{\theta}\frac{Q}{Q-1}}}{h\left[\qmg^{\frac{\lambda}{\theta}\frac{Q}{Q-1}} \times Q^{-\frac{\lambda}{\theta}\frac{Q}{Q-1}}\right]-\left(h-\ell\right)} \in \left[\underline{q},Q\right]
\end{equation}
such that all $\LP$s with $q_i\leq \qmg^\star$ deposit liquidity in pool $H$ and all $\LP$s with $q_i>\qmg^\star$ choose pool $L$.
\end{leftbar}    
\end{prop}

Figure \ref{fig:region_equilibrium} illustrates the equilibrium regions in Proposition \ref{prop:equilibria}. If the gas price is low and $\LP$s are homogeneous (low $\Gamma$ and $Q$), then all liquidity providers choose pool $L$ since managing liquidity is relatively cheap. However, if more high-endowment $\LP$s enter the market (i.e., an increase in $Q$), the liquidity cycle on pool $L$ becomes longer. The rationale is that,, keeping the $\LT$ arrival rate fixed, a larger pool depletes more slowly. The liquidity fee per unit of time on pool $L$ drops and smaller $\LP$s switch to the high-fee pool. As a result, liquidity becomes fragmented across the two pools. 

% Figure environment removed

An equilibrium where all liquidity consolidates on pool $H$ is only sustainable for very high gas costs $\Gamma>Q\ell$, such that none of the $\LP$s breaks even on pool $L$. For intermediate values of gas price, both pools co-exist in equilibrium with positive market share.

Proposition \ref{cor:comp_stat_ms} establishes comparative statics for the two pools' liquidity market shares. From equation \eqref{eq:liquidity_levels}, we can compute the liquidity market share of the low-fee pool at the beginning of each cycle as
\begin{equation}
w_L=\frac{\LL}{\LL+\LH}=\frac{\log Q - \log \qmg^\star}{\log Q - \log \underline{q}}.
\end{equation}

\begin{prop}\label{cor:comp_stat_ms}
\begin{leftbar} \setlength{\parskip}{0ex}
\emph{(Comparative statics)} In equilibrium, the market share of the low fee pool $w_L$
\begin{itemize}
    \item[(i)] decreases in the gas cost ($\Gamma$), the arrival rate of large trades ($\lambda$), and the fee on pool $H$ (h).
    \item[(ii)] increases in the fee on pool $L$ ($\ell$) and the arrival rate of small trades ($\theta$)
\end{itemize}
\end{leftbar}    
\end{prop}

The results in Proposition \ref{cor:comp_stat_ms} are intuitive. The market share of pool $L$ increases if the fee gap $h-\ell$ is narrower, since this reduces $\LP$ incentives to switch to the high-fee pool. If the small $\LT$ arrival rate is large, then liquidity cycles in the low-fee pool are shorter, increasing the revenue per unit of time and consequently the market share of pool $L$. Conversely, if large trades arrive more often (high $\lambda$), then pool $H$ attracts a higher share of incoming order flow and becomes more appealing for liquidity providers.

Figure \ref{fig:liqshares} shows that the market share of pool $L$ (weakly) decreases in the gas cost $\Gamma$. A larger gas price increases the costs of active liquidity management, everything else equal, and incentivizes smaller $\LP$s to switch from pool $L$ to pool $H$, since the latter has a lower turnover. For $\Gamma\leq h$, any increase in gas costs leads to a \emph{redistribution} of liquidity from one pool to another; the aggregate liquidity across both pools is constant since all $\LP$s participate in the market.

% Figure environment removed

If gas prices increase beyond a threshold ($\Gamma>h$), then the aggregate liquidity falls since $\LP$s with $q_i<\frac{\Gamma}{h}$ are shut out of the market. Both the $L$ and the $H$ pool experience a decrease in liquidity deposits. However, the liquidity drop is sharper for pool $L$, which further depresses its market share.


\subsection{Model implications and empirical predictions}

\begin{pred}\label{pred:comp_stat_Gamma}
\emph{The liquidity market share of the low-fee pool $L$ decreases in the gas fee $\Gamma$.}
\end{pred}

Prediction \ref{pred:comp_stat_Gamma} follows directly from Proposition \ref{cor:comp_stat_ms} and Figure \ref{fig:liqshares}. A higher gas price increases the fixed cost of active liquidity management, particularly so for smaller liquidity providers. In response, $\LP$s with lower endowments migrate to the high-fee pool $H$ where they trade less often. 

\begin{pred}\label{pred:clienteles}
\emph{$\LP$s on the low-fee pool make larger liquidity deposits than $\LP$s on the high-fee pool.}
\end{pred}

Prediction \ref{pred:clienteles} follows from the equilibrium discussion in Section \ref{sec:liqprov}. Liquidity providers with large token endowments ($q_i>\qmg$) deposit them in the low-fee pool since they are better positioned to actively manage liquidity due to economies of scale. $\LP$s with lower endowments ($q_i\leq\qmg$) either stay out of the market or choose pool $H$ which allows them to offer liquidity in a more passive manner.

Figure \ref{fig:theory_liqsupply} illustrates this prediction through a Monte Carlo simulation. We plot the equilibrium liquidity supply decisions of 100,000 $\LP$s with endowments drawn from density \eqref{eq:density} and $Q=3$. The top panel highlights three groups of liquidity providers: low-endowment $\LP$s (in green) that are being rationed out of the market due to high gas cost, medium-endowment $\LP$s (blue) that deposit liquidity on pool $H$, and high-endowment $\LP$s (orange) that choose the low-fee pool $L$. 

% Figure environment removed

\begin{pred}\label{pred:trade_size_volume}
\emph{The average trade size is higher on pool $H$ than on pool $L$. At the same time, trading volume is higher on pool $L$ than on pool $H$.}
\end{pred}

Next, Prediction \ref{pred:trade_size_volume} deals with differences between incoming trades on the two liquidity pools. For a wide range of model parameters, incoming order flow on pool $L$ consists of a large number of small trades, and occasional large trades. In contrast, there are few trades on pool $H$, but they are all relatively large. The model can therefore reconcile two apparently conflicting patterns: one liquidity pool captures most of the trading volume, while the largest trades are executed on the competitor. Figure \ref{fig:theory_trade} illustrates the prediction through a Monte Carlo simulation of the model.


% Figure environment removed


\begin{pred}\label{pred:clienteles_cs}
\emph{The average liquidity deposit on both the low- and- high fee pool increases with gas costs.}
\end{pred}

An increase in the gas cost $\Gamma$ has two effects: first, the $\LP$s with the lowest endowments on pool $L$ switch to pool $H$. As a result, the average deposit on pool $L$ increases. Second, the $\LP$s with low endowments on pool $H$ may leave the market. Both channels translate to a higher average deposit on pool $H$, which experiences an inflow (outflow) of relatively high (low) endowment $LP$ following an increase in gas costs.



The bottom left panel highlights the clientele effect: that is, the average deposit is higher on pool $L$. Due to the skew of the Pareto distribution, however, there are more $\LP$ accounts active on pool $H$ than on pool $L$ for a wide range of parameter values.



\begin{pred}\label{pred:updates}
\emph{$\LP$s update liquidity more frequently on the low-fee than on the high-fee pool.}
\end{pred}

Prediction \ref{pred:updates} is a consequence of Lemma \ref{lem:duration}. Liquidity cycles on pool $L$ are shorter than on pool $H$, since $\LP$s on the low-fee pool trade against both small and large orders, rather than only against large orders on the high-fee pool. Consequently, we expect $\LP$s on pool $L$ to actively manage their liquidity positions.


\begin{pred}\label{pred:updates_gas}
\emph{A larger gas price leads to more frequent liquidity updates on the low-fee pool.}
\end{pred}
An increase in gas price leads to some $\LP$s switching from the low- to the high-fee pool. As a result, liquidity supply on the low-fee pool drops, leading to a shorter cycle as incoming order flow depletes the pool at a faster pace.




