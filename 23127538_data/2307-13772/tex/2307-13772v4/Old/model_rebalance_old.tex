\subsection{Primitives}

\paragraph{Token and traders.} A single token is traded in a continuous-time economy. There are two trader types in the economy: a continuum of liquidity providers ($\LP$s) and liquidity takers ($\LT$s). Both market participant types are risk neutral.  

The expected value of the token equals $v>0$, which is common knowledge. However, market participants have heterogeneous private values for the asset. In particular, liquidity takers value the token at $v\left(1+\Delta\right)$ with $\Delta>0$ whereas $\LP$s have no private value for the token. It follows that there are gains from trade if $\LT$s buy the token from $\LP$s \citep[we follow, e.g.,][in that we focus on a one-sided market where liquidity providers only act as sellers]{foucault2013liquidity}.

Liquidity providers have heterogeneous capital constraints: at any time $\LP$ $i$ can supply liquidity for at most $q_i$ units of the token, where $q_i$ has a log-uniform distribution on $\left[1,Q\right]$:
\begin{equation}
    \varphi\left(q\right)=\frac{1}{\log Q} \frac{1}{q} \text{ for } q\in\left[1,Q\right].
\end{equation}
Our choice of density implies that there are many low-endowment liquidity providers (e.g., retail traders) and a long tail of high-capital investors who provide liquidity to decentralized exchange -- for example, quantitative institutional investors. On aggregate, $\LP$s can provide at most
\begin{equation}
    \int_1^Q q \varphi\left(q\right) \diff q = \frac{Q-1}{\log Q}
\end{equation}
units of liquidity.

There are two types of liquidity takers: small and large. Small $\LT$s arrive at the market at rate $\theta$ per unit of time and demand one unit of the token. Further, a large $\LT$ demands $\Theta$ units of the token, where $\Theta>\frac{Q-1}{\log Q}$. That is, we assume that the large $\LT$ liquidity demand exceeds the maximum liquidity supply. The large $\LT$ arrival time follows a Poisson process with rate $\lambda>0$. The aggregate liquidity demand from small and large $\LT$s evolves over time as
\begin{equation}
    \diff \emph{LiquidityDemand}_t = \theta \diff t + \Theta \diff J_t\left(\lambda\right),
\end{equation}
where $J_t\left(\lambda\right)$ stands in for a Poisson process with rate $\lambda$.

\paragraph{Trading environment.} The token is traded on two liquidity pools that charge liquidity fees proportional to the value of the trade. The token and numeraire quantities ($T$ and $N$, respecitvely) in each pool must satisfy at all times the following linear bonding curve:
\begin{equation}\label{eq:bond_curve}
    v \times T_k + N_k = \text{constant}, \forall k\in\left\{L,K\right\}
\end{equation}
The bonding curve \eqref{eq:bond_curve} implies that the token price is constant and equal to $v$ units of the num\'{e}raire, since $-\frac{\diff T_k}{\diff N_k}=v$.\footnote{In practice, most decentralized exchanges use convex bonding curve, for example constant product pricing. While the choice of bonding curve is of first order importance for liquidity takers who need to split trades to optimize price impact, we argue it is less important for liquidity suppliers. For tractability, we assume a linear bonding curve which implies that liquidity traders only care about pool fees -- the focus of our model. In a richer model with convex pricing, we would expect to see relatively more fragmentation since minimizing price impact might require further splitting trades.} The low-fee pool $L$ charges a fee of $\ell$ and the high-fee pool $H$ charges a fee of $h$ per token traded, where $\ell<h$. For example, to purchase $t$ units of the token on the $L$ pool, the total cost of an investor is $t\left(v+\ell\right)$. Incoming trades are matched with $\LP$s on a pro rata basis, that is all liquidity providers receive fees proportional to their outstanding deposits.

\paragraph{Gas costs.} Any interaction with a liquidity pool (for example, to add or remove liquidity)  incurs a fixed execution cost $\Gamma>0$. The execution cost can be interpreted as the gas price on the Ethereum blockchain as in \citet{CapponiJia2021}. In contrast to \citet{CapponiJia2021}, we assume traders on the decentralized exchanges are price takers on the gas market and their orders do not impact the gas cost. A natural interpretation is that message volume at any time in a particular asset pair traded on decentralized exchanges is small relative to the aggregate transaction flow on the entire blockchain.

Gas fees on decentralized exchanges differ from trading fees on traditional exchanges in a number of ways. First, they are not set by individual exchanges to compete with each other, but are a common transaction cost \emph{across} exchanges. Second, they are levied on a per-order rather than per-share basis: this implies economies of scale for larger orders. Third, there is significant time variation in gas fees which allows us to better empirically identify the impact of transaction costs on liquidity pool market shares.

To ensure that both small and large liquidity traders participate in the market, we assume that the gains from trade are larger than the aggregate transaction cost, including the pool fee and the gas price:
\begin{equation}
    v\left(1+\Delta\right)-v > v\left(1+h\right) + \Gamma \Rightarrow \Delta > h + \frac{\Gamma}{v}.
\end{equation}

To rule out trivial cases, we further assume that $Q\ell-\Gamma>0$. The condition ensures that there are at least some liquidity providers who can earn positive expected profit on pool $L$; otherwise, pool $L$ never attracts any deposits.

\paragraph{Timing.} We follow \citet{foucault2013liquidity} and partition time between liquidity cycles. A liquidity cycle starts with an empty pool and $\LP$ deposits and ends when incoming trades deplete the liquidity supply, returning the pool to an empty state. The first cycle starts at $t=0$ when each liquidity provider $i$ selects a pool $k\in\left\{L,H\right\}$, pays the gas cost $\Gamma$, and deposits their token endowment $q_i$. Liquidity providers also have the option to not provide any liquidity, that is $k=\emptyset$. Following the initial liquidity deposit, $\LP$s do not further interact with the pool until the cycle ends at some random time $\tilde{\tau}$. The liquidity cycle restarts at $\tilde{\tau}$: $\LP$s use the sales cash proceeds to borrow or purchase $q_i$ tokens and refill liquidity on decentralized exchange pools.

\subsection{Equilibrium analysis}\label{sec:liqprov}

An equilibrium in our trading game is a mapping from the $\LP$ endowment $q_i$ to a liquidity pool $k\in\left\{L,H, \emptyset \right\}$ such that no liquidity provider is better off by deviating and choosing either to deposit liquidity in another pool or to not provide liquidity at all. Conditional on making a deposit, liquidity providers choose the pool $k^\star$ that maximizes their expected profit per unit of time, that is
\begin{equation}\label{eq:optimal_pool}
    k^\star\left(q_i\right) \coloneqq \arg\max_{k} \left(q_i f_k - \Gamma\right) \frac{1}{d_k}, 
\end{equation}
where $f_k$ is the liquidity fee on pool $k$, and $d_k$ is the duration of a liquidity cycle on pool $k$. The revenue per liquidity cycle is equal to the pool liquidity fee times the deposit $q_i$, net of the gas cost to submit the order.

Let $\Omega_k$ denote the set of $\LP$s who choose pool $k$ in equilibrium, that is $\Omega_k\coloneqq\left\{i \mid k^\star\left(q_i\right)=k\right\}$. We can define total liquidity on pool $k$ at $t=0$ as the aggregate deposit from all liquidity providers in $\Omega_k$:
\begin{equation}
    \mathcal{L}_k=\int_{i\in\Omega_k} q_i \varphi(q_i) \diff i.
\end{equation}

Before solving for the equilibrium $\LP$ pool choice, we first analyze the liquidity traders' decisions. A large $\LT$'s liquidity demand exceeds the aggregate $\LP$ token endowment: upon arrival, the large $\LT$ therefore consumes the entire liquidity on both pools and triggers the end of the liquidity cycle. In contrast, small $\LT$s trade smoothly over time and consume liquidity from the cheaper pool $L$ at rate $\frac{\theta}{\mathcal{L}_L} \diff t$. Once the low-fee pool is empty, liquidity providers immediately refill it and restart the cycle.

\begin{lem}\label{lem:duration}
\begin{leftbar} \setlength{\parskip}{0ex}
The duration of a liquidity cycle is lower in pool $L$ than in pool $H$. Formally, $d_L<d_H$.
\end{leftbar}
\end{lem}

Lemma \ref{lem:duration} states that liquidity cycles in pool $L$ are shorter than in pool $H$. The result is intuitive: From the discussion above, a liquidity cycle on pool $H$ only ends with the arrival of a large liquidity trader: from the properties of Poisson processes, the expected cycle duration $d_H$ equals $\frac{1}{\lambda}$. Conversely, a liquidity cycle on pool $L$ ends either upon the large $\LT$ arrival or when small $\LT$s exhaust the pool liquidity. Formally,
\begin{align}
    d_L &=\exp\left(-\lambda \frac{\LL}{\theta}\right) \frac{\LL}{\theta}+\int_{0}^{\frac{\LL}{\theta}} t\lambda\exp\left(-\lambda t\right)\diff t \nonumber \\
    &=\frac{1}{\lambda}-\frac{1}{\lambda}\exp\left(-\frac{\LL}{\theta}\lambda\right)<\frac{1}{\lambda}=d_H.
\end{align}

Liquidity providers on pool $L$ absorb a larger share of the order flow, as they trade with both small and large $\LT$s. However, they earn a lower fee per traded volume than they would on pool $H$. Furthermore, they need to manage their liquidity more often, which implies larger gas costs per unit of time. From equation \eqref{eq:optimal_pool}, liquidity provider $i$ chooses the low-fee pool $L$ if and only if
\begin{equation}\label{eq:cost_comparison}
    \pi_\ell-\pi_h=\frac{1}{d_h d_\ell}\left[\left(d_h \ell - d_\ell h\right) q_i - \Gamma \left(d_h-d_\ell\right)\right]>0
\end{equation}
A salient implication of equation \eqref{eq:cost_comparison} is that $\LP$s with larger endowments are more likely to choose the low-fee pool, keeping liquidity shares constant. The rationale is that gas fees are a fixed cost, and larger $\LP$s are more likely to achieve economies of scale. The expected fee revenue scales linearly with a liquidity provider's share in the liquidity pool (proportional to $q_i$), better compensating $\LP$s for the gas fee paid at the start of the liquidity cycle. 

We follow \citet{KatzShapiro1985} and conjecture that there exists a marginal liquidity provider $\qmg$ such that all $\LP$s with $q_i>\qmg$ post liquidity on the low-fee pool and all $\LP$s with $q_i\leq \qmg$ choose the high-fee pool.\footnote{We note that there cannot be a ``reversed-sign'' equilibrium, where $\LP$s with $q_i\leq\qmg$ post liquidity on the low-fee pool and all $\LP$s with $q_i>\qmg$ choose the high-fee pool. To understand the rationale, we point out that for an $\LP$ with endowment $\tilde{q}<\qmg$, submitting liquidity to pool $L$ yields a lower expected revenue since
\begin{equation*}
    \pi_\ell-\pi_h=\frac{1}{d_h d_\ell}\left[\left(d_h \ell - d_\ell h\right) \tilde{q} - \Gamma \left(d_h-d_\ell\right)\right] < 0.
\end{equation*}
Therefore, it is optimal for the $\LP$ with endowment $\tilde{q}$ to post liquidity on the high-fee pool, which contradicts the conjectured equilibrium.  Importantly, since each $\LP$ is ``small'' relative to the aggregate mass of liquidity providers, individual deviations do not impact aggregate pool sizes $\LL$ and $\LH$.} To provide liquidity, $\LP$s need to earn positive expected profits per unit of time, at least in the high-fee pool. From equation \eqref{eq:optimal_pool}, the $\LP$ participation constraint becomes
\begin{equation}\label{eq:pc}
    q_i h - \Gamma =\geq 0 \Rightarrow q_i>\underline{q}\coloneqq \max\left\{\frac{\Gamma}{h},1\right\}.
\end{equation}
If $\frac{\Gamma}{h}>1$, then liquidity providers are competitive since the marginal $\LP$ earns zero profit in expectation. 

The conjectured equilibrium pins down the pool sizes $\LL$ and $\LH$ as a function of the endowment for the marginal $\LP$:
\begin{align}\label{eq:liquidity_levels}
    \LL&=\int_{\qmg}^Q q_i \varphi\left(q_i\right) \diff i = \frac{1}{\log Q}\left(Q - \qmg \right)  \text{ and }\nonumber \\
    \LH&=\int_{\underline{q}}^{\qmg}  q_i \varphi\left(q_i\right) \diff i = \frac{1}{\log Q}\left(\qmg -  \underline{q}\right)
\end{align}
From equations \eqref{eq:cost_comparison} through \eqref{eq:liquidity_levels} it follows that the expected profit difference between the low- and high-fee pools can be written as an increasing function of the marginal $\LP$'s endowment, that is
\begin{align}\label{eq:pi_diff}
    \pi_\ell-\pi_h &=\frac{1}{\lambda d_h d_\ell}\left[\exp\left(-\frac{\lambda}{\theta} \LL\right) \underbrace{\left(q_i h - \Gamma\right)}_{>0} - q_i\left(h-\ell\right)\right] \nonumber \\
        &=\frac{1}{\lambda d_h d_\ell}\left[\exp\left(-\frac{\lambda}{\theta \log Q}\left(Q-\qmg\right)\right) \underbrace{\left(q_i h - \Gamma\right)}_{>0}- q_i\left(h-\ell\right)\right]. 
\end{align}
We search for fulfilled expectation equilibria \citep[following the definition of][]{KatzShapiro1985}, where the beliefs of $\LP$s at $t=0$ about the size of liquidity pools are consistent with the outcome of their choices. Proposition \ref{prop:equilibria} characterizes the equilibrium liquidity provision.

\begin{prop}\label{prop:equilibria}
\begin{leftbar} \setlength{\parskip}{0ex}
If $\frac{h-l}{h} \exp\left(-\frac{\lambda}{\theta \log Q}\left(Q-1\right)\right)<1$ and $\frac{\Gamma}{h}<1-\exp\left(-\frac{\lambda}{\theta \log Q}\left(Q-1\right)\right)$, then there exists a unique corner equilibrium where all $\LP$s deposit liquidity on pool $L$. Otherwise, there exists a unique fragmented equilibrium characterized by marginal trader $\qmg^\star$ which solves
\begin{equation}\label{eq:mg_eq}
    \qmg = \Gamma \frac{\exp\left(-\frac{\lambda}{\theta \log Q}\left(Q-\qmg\right)\right)}{h\exp\left(-\frac{\lambda}{\theta \log Q}\left(Q-\qmg\right)\right)-\left(h-\ell\right)} \in \left[\underline{q},Q\right]
\end{equation}
such that all $\LP$s with $q_i\leq \qmg^\star$ deposit liquidity in pool $H$ and all $\LP$s with $q_i>\qmg^\star$ choose pool $L$.
\end{leftbar}    
\end{prop}

\begin{cor}\label{cor:comp_stat}
\begin{leftbar} \setlength{\parskip}{0ex}
In equilibrium, the endowment marginal trader $\qmg^\star$:
\begin{itemize}
    \item[(i)] increases in the gas cost ($\Gamma$), the fee on pool $H$ ($h$), the arrival rate of large trades ($\lambda$), and the liquidity distribution parameter $Q$.
    \item[(ii)] decreases in the fee on pool $L$ ($\ell$) and the arrival rate of small trades ($\theta$).
\end{itemize}
\end{leftbar}    
\end{cor}

From equation \eqref{eq:liquidity_levels}, we can compute the market share of the low-fee pool as
\begin{equation}
w_L=\frac{\LL}{\LL+\LH}=\frac{Q - \qmg^\star}{Q - \underline{q}}.
\end{equation}

\begin{cor}\label{cor:comp_stat_ms}
\begin{leftbar} \setlength{\parskip}{0ex}
In equilibrium, the market share of the low fee pool $w_L$
\begin{itemize}
    \item[(i)] increases in the fee on pool $L$ ($\ell$) and the arrival rate of small trades ($\theta$).
    \item[(ii)] decreases in the arrival rate of large trades ($\lambda$) and the fee on pool $H$ (h).
\end{itemize}
\end{leftbar}    
\end{cor}


\subsection{Empirical predictions}

\begin{pred}\label{pred:clienteles}
\emph{$\LP$s on the low-fee pool make larger liquidity deposits than $\LP$s on the high-fee pool.}
\end{pred}

Prediction \ref{pred:clienteles} follows from the discussion in Section \ref{sec:liqprov}. Liquidity providers with large token endowments ($q_i>\qmg$) deposit them in the low-fee pool since they have stronger incentives to pay the gas fee and actively manage the liquidity. Conversely, $\LP$s with lower endowments ($q_i\leq\qmg$) gravitate towards the high-fee pool, accepting a lower trade probability in exchange for reducing adverse selection costs without active liquidity management and paying gas fees.

\begin{pred}\label{pred:updates}
\emph{$\LP$s on the low-fee pool update liquidity more often than $\LP$s on the high-fee pool.}
\end{pred}

\begin{pred}\label{pred:trade_size}
\emph{The average trade on the high-fee pool is larger than the average trade on the low-fee pool.}
\end{pred}


\begin{pred}\label{pred:trade_volume}
\emph{The trading volume on the low-fee pool is larger than the trading volume on the high-fee pool.}
\end{pred}

\begin{pred}\label{pred:dispersion_trades}
\emph{The liquidity share of the low-fee pool $L$ decreases in the trade size heterogeneity for the pair.}
\end{pred}
We proxy trade size heterogeneity by $\frac{\lambda}{\theta}$, i.e., the ratio between the likelihood of a large trade and the size of small trades.

\begin{pred}\label{pred:comp_stat_Gamma}
\emph{The liquidity share of the low-fee pool $L$ decreases in the gas fee $\Gamma$.}
\end{pred}

\begin{pred}\label{pred:frag_exist}
\emph{A fragmented market equilibrium is more likely if the average liquidity deposit, the trade size heterogeneity, and gas fees are larger.}
\end{pred}

