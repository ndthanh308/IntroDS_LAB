\noindent \textbf{\large Proposition \ref{prop:equilibria}}
\begin{proof}

To account for kinks in the reaction function \eqref{eq:reaction_main} -- depending on the fraction of $\LP$s that optimally manage liquidity on platform $L$ -- we separately discuss three parameter regions: $\Gamma\leq 1-f$, $\Gamma\in\left(1-f, 2\left(1-f\right)\right]$, and $\Gamma>2\left(1-f\right)$, respectively.

\subsubsection*{Region \#1: $\Gamma\leq 1-f$}

We start by considering the case that $\Gamma\leq 1-f$: that is, it is optimal for any $\LP$ to react upon news arrival if they post liquidity on pool $L$.First, let $\bar{\Gamma}_1=f\frac{1-\eta}{2\eta}$. If $\Gamma<\bar{\Gamma}_1$, then from equation \eqref{eq:reaction_main} it follows that $\tilde{q}\left(1\right)=1$ and therefore $\qmg=1$ is a fixed point of the reaction function. That is, for $\Gamma<\bar{\Gamma}_1$ there is an equilibrium for which all $\LP$s choose to provide liquidity on the $L$ platform. 

Second, let $\bar{\Gamma}_2=\frac{\left(2f-1\right)\left(1-\eta\right)}{\eta}$. From \eqref{eq:reaction_main}, if $\Gamma\geq \bar{\Gamma}_2$ then $\tilde{q}\left(2\right)=2$, which implies that $\qmg=2$ is a fixed point and there is an equilibrium for which all $\LP$s choose to provide liquidity on the $H$ platform.

Third, we note that among possible interior equilibria, $q_1^\star$ increases in $\Gamma$ and $q_2^\star$ decreases in $\Gamma$. If $f\leq\frac{1}{2}$, then $q_1^\star<1$ and is not a feasible root. However, if $f>\frac{1}{2}$, then $q_1^\star\geq 1$ for $\Gamma\geq\bar{\Gamma}_1$. and $q_1^\star<2$ for $\Gamma<\bar{\Gamma}_2$. Similarly, if $f\leq\frac{1}{2}$, then $q_2^\star<2$ and $q_2^\star<1$ for $\Gamma>\bar{\Gamma}_1$. For $f>\frac{1}{2}$, $q_2^\star\geq 2$ for $\Gamma\leq\bar{\Gamma}_2$. However, for $f>\frac{3}{4}$, $q_2>2$ for all $\Gamma$ so it stops being a feasible equilibrium. Finally, the roots $q_1^\star$ and $q_2^\star$ are real if and only if $\Gamma\leq\bar{\Gamma}_3$ where $\bar{\Gamma}_3=\frac{1-\eta}{8\left(1-f\right)\eta}$. For any $f$ and $\eta$ in $\left[0,1\right] \times \left[0,1\right]$, $\bar{\Gamma}_3\geq \bar{\Gamma}_1$ (with equality for $f=\frac{1}{2}$) and $\bar{\Gamma}_3\geq \bar{\Gamma}_2$ (with equality for $f=\frac{3}{4}$). We analyze all possible cases sequentially.

\paragraph{Case \#1: Low fee on $L$ ($f\leq \frac{1}{2}$).} If $f\leq \frac{1}{2}$, then $\bar{\Gamma}_2\leq 0 < \bar{\Gamma}_1 < \bar{\Gamma}_3$. We distinguish between two potential regions for the gas fee $\Gamma$:
\begin{enumerate}
    \item[(i)] $\Gamma\leq\bar{\Gamma}_1$. For a low gas fee, $\qmg^\star=1$ is a fixed point of the reaction function and therefore an equilibrium. Since $\bar{\Gamma}_2<0$, then immediately $\Gamma>\bar{\Gamma}_2$ and therefore $\qmg^\star=2$ is also an equilibrium. From the intermediate value theorem, and monotonicity of $\tilde{q}\left(\cdot\right)$, there is another fixed point (i.e., interior equilibrium) at $q^\star_2$. However, this interior equilibrium if unstable. If we perturb the $\LP$s' beliefs such the expectation for the marginal $\LP$ endowment is $\qmg+\epsilon$, there is pressure to switch to the high-fee pool since $\tilde{q}\left(q^\prime\right)>q^\prime$ for any $q^\prime>q^\star_2$. The system spirals towards an equilibrium in which all $\LP$s provide liquidity on the high-fee pool. Conversely, if we perturb beliefs in the opposite direction ($\qmg-\epsilon$) there is pressure to switch to the low-fee pool.
    \item[(ii)] $\Gamma>\bar{\Gamma}_1$. For a larger gas fee, the only equilibrium is $\qmg^\star=2$, that is all $\LP$s provide liquidity on the high-fee pool. This is because $q=1$ is not a fixed point of the reaction function, and the interior roots $q^\star_1$ and $q^\star_2$ are either outside the $\left[1,2\right]$ interval or they are irrational. 
\end{enumerate}

\paragraph{Case \#2: Medium fee on $L$ (i.e., $f\in\left(\frac{1}{2},\frac{2}{3}\right]$).} If $f\in\left(\frac{1}{2},\frac{2}{3}\right]$, then $0<\bar{\Gamma}_2 < \bar{\Gamma}_1 < \bar{\Gamma}_3$. We distinguish between four potential regions for the gas fee $\Gamma$:
\begin{enumerate}
    \item[(i)] $\Gamma\leq\bar{\Gamma}_2$. From the discussion above, $\qmg^\star=1$ is an equilibrium since $\bar{\Gamma}_2\leq \bar{\Gamma}_1$ and $\qmg^\star=2$ is not. Further, $q^\star_1<1$ and $q^\star_2>2$: the only equilibrium is the one where all $\LP$s provide liquidity on the low-fee pool.
    \item[(ii)] $\Gamma\in\left(\bar{\Gamma}_2,\bar{\Gamma}_1\right]$. Both $\qmg^\star=1$ and $\qmg^\star=2$ are equilibria in this case. There is also third, unstable equilibrium at $q^\star_2<2$ since  $\Gamma>\bar{\Gamma}_2$. The remaining interior root is not a fixed point since $q^\star_1<1$ for $\Gamma<\bar{\Gamma}_1$.
    \item[(iii)] $\Gamma\in\left(\bar{\Gamma}_1,\bar{\Gamma}_3\right]$. There are two stable equilibria, an interior one at $q^\star_1>1$ (since $\Gamma>\bar{\Gamma}_1$) and at $\qmg^\star=2$. There is a third, unstable equilibrium at $q^\star_2$.
    \item[(iv)] $\Gamma>\bar{\Gamma}_3$. The unique equilibrium is for $\qmg^\star=2$. The interior roots $q^\star_{1,2}$ are not real numbers, and $\tilde{q}(1)>1$ such that $\qmg^\star=1$ is not an equilibrium. 
\end{enumerate}

\paragraph{Case \#3: Large fee on $L$ (i.e., $f\in\left(\frac{2}{3},\frac{3}{4}\right]$).} If  $f\in\left(\frac{2}{3},\frac{3}{4}\right]$, then $0<\bar{\Gamma}_1 < \bar{\Gamma}_2 < \bar{\Gamma}_3$. We distinguish again between four potential regions for the gas fee $\Gamma$:
\begin{enumerate}
    \item[(i)] $\Gamma\leq\bar{\Gamma}_1$. As in the low gas fee case for $f\in\left(\frac{1}{2},\frac{2}{3}\right]$, $\qmg^\star=1$ is an equilibrium since $\bar{\Gamma}_2\leq \bar{\Gamma}_1$ and $\qmg^\star=2$ is not. Further, $q^\star_1<1$ and $q^\star_2>2$: the only equilibrium is the one where all $\LP$s provide liquidity on the low-fee pool.
    \item[(ii)] $\Gamma\in\left(\bar{\Gamma}_1,\bar{\Gamma}_2\right]$. Neither $\qmg^\star=1$ and $\qmg^\star=2$ are equilibria in this case. There is an unique interior equilibrium at $q^\star_1<1$ since $\Gamma>\bar{\Gamma}_1$. The second root is out of bounds: that is,  $q^\star_2>2$ since $\Gamma<\bar{\Gamma}_2$.
    \item[(iii)] $\Gamma\in\left(\bar{\Gamma}_2,\bar{\Gamma}_3\right]$. There are two stable equilibria, an interior one at $q^\star_1>1$ (since $\Gamma>\bar{\Gamma}_1$) and at $\qmg^\star=2$. There is a third, unstable equilibrium at $q^\star_2<2$.
    \item[(iv)] $\Gamma>\bar{\Gamma}_3$. The unique equilibrium is for $\qmg^\star=2$. The interior roots $q^\star_{1,2}$ are not real numbers, and $\tilde{q}(1)>1$ such that $\qmg^\star=1$ is not an equilibrium. 
\end{enumerate}

\paragraph{Case \#4: Very large fee on $L$ (i.e., $f>\frac{3}{4}$).} The scenario where $f>\frac{3}{4}$ is similar to the one above, the  but $q_2>2$ for $\Gamma\in\left(\bar{\Gamma}_2,\bar{\Gamma}_3\right]$ so one possible equilibrium is no longer feasible. We distinguish between three potential regions for the gas fee $\Gamma$:
\begin{enumerate}
    \item[(i)] $\Gamma\leq\bar{\Gamma}_1$. As in the low gas fee case for $f\in\left(\frac{1}{2},\frac{2}{3}\right]$, $\qmg^\star=1$ is an equilibrium since $\bar{\Gamma}_2\leq \bar{\Gamma}_1$ and $\qmg^\star=2$ is not. Further, $q^\star_1<1$ and $q^\star_2>2$: the only equilibrium is the one where all $\LP$s provide liquidity on the low-fee pool.
    \item[(ii)] $\Gamma\in\left(\bar{\Gamma}_1,\bar{\Gamma}_2\right]$. Neither $\qmg^\star=1$ and $\qmg^\star=2$ are equilibria in this case. There is an unique interior equilibrium at $q^\star_1<1$ since $\Gamma>\bar{\Gamma}_1$. The second root is out of bounds: that is,  $q^\star_2>2$ since $\Gamma<\bar{\Gamma}_2$.
    \item[(iii)] $\Gamma>\bar{\Gamma}_2$. The unique equilibrium is for $\qmg^\star=2$. The interior roots $q^\star_{1,2}$ are not real numbers, and $\tilde{q}(1)>1$ such that $\qmg^\star=1$ is not an equilibrium. 
\end{enumerate}

\subsubsection*{Region \#2: $\Gamma\in\left(1-f, 2\left(1-f\right)\right]$}

For $\Gamma>1-f$, the reaction function \eqref{eq:reaction_main} becomes
\begin{equation}
    \tilde{q}\left(\qmg\right)=\begin{cases}
    1, & \text{ for } \qmg \leq \frac{1-\eta\left(1+2(1-f)\right)}{\left(1-f\right)\left(1-\eta\right)}\\
    \frac{2\eta\Gamma}{ \left(1-\eta\right) \left(1-\left(1-f\right)\qmg\right)}, & \text{ for } \qmg \in \left(\frac{1-\eta\left(1+2(1-f)\right)}{\left(1-f\right)\left(1-\eta\right)}, \frac{1-\eta\left(1+\Gamma\right)}{\left(1-f\right)\left(1-\eta\right)}\right] \\
    2, & \text{ for } \qmg > \frac{1-\eta\left(1+\Gamma\right)}{\left(1-f\right)\left(1-\eta\right)}.
    \end{cases}
\end{equation}
First, $\qmg=1$ is a fixed point and therefore an equilibrium of the liquidity provision game if and only if
\begin{equation}
    \frac{1-\eta\left(1+2(1-f)\right)}{\left(1-f\right)\left(1-\eta\right)}>1 \Longleftrightarrow f>\frac{2\eta}{1+\eta}.
\end{equation}
Second, as for the first gas fee region, $\qmg=2$ is an equilibrium if $\Gamma>\bar{\Gamma}_2$.  The existence condition for a stable equilibrium is:
\begin{equation}
1\leq\frac{1}{2\left(1-f\right)} - \frac{1}{2\left(1-f\right)}\sqrt{1-\eta\frac{1+8\left(1-f\right)\Gamma}{1-\eta}}<\frac{1-\eta\left(1+\Gamma\right)}{\left(1-f\right)\left(1-\eta\right)},
\end{equation}
which translates to $\Gamma\in\left[f\frac{1-\eta}{2\eta},\frac{(2f-1)(1-\eta)}{\eta}\right)$. However, the interior equilibrium is only unique for $f\leq\frac{2\eta}{1+\eta}$.

\subsubsection*{Region \#3: $\Gamma>2\left(1-f\right)$}
If $\Gamma>2\left(1-f\right)$, then no liquidity provider reacts upon news to remove liquidity since the gas fee is too high. The expected profit difference from submitting quotes on the low- and high- fee pool is therefore proportional to $q_i$:
\begin{equation}\label{eq:pi_diff_highG}
    \pi^\LP_L-\pi^\LP_H = q_i \left[\frac{1}{2} \left(1-\eta\right) \left(1-\left(1-f\right)\qmg\right) - \eta \left(1-f\right)\right],
\end{equation}
which is always positive if $\qmg\leq\frac{1-\eta\left(3-2f\right)}{\left(1-f\right)\left(1-\eta\right)}$ and negative otherwise. Therefore, either all liquidity providers choose the low-fee pool or all liquidity providers choose the high-fee pool. Depending on whether $\frac{1-\eta\left(3-2f\right)}{\left(1-f\right)\left(1-\eta\right)}\in\left[1,2\right]$ or not, the coordination game might have one or two equilibria, but with a single liquidity pool dominating the market.

\end{proof}

\noindent \textbf{\large Corollary \ref{cor:monitoring}}
\begin{proof}
    Generally, from \eqref{eq:pi_diff} and \eqref{eq:pi_diff_highG}, the expected profit difference from submitting quotes on the low- and high- fee pool is
\begin{equation}\label{eq:pi_diff_general}
    \pi^\LP_L-\pi^\LP_H = q_i \left[\frac{1}{2} \left(1-\eta\right) \left(1-\left(1-f\right)\qmg\right)\right] - \eta \min\left\{q_i\left(1-f\right),\Gamma\right\},
\end{equation}
where the latter term reflects whether is is optimal for an LP with endowment $q_i$ to monitor on pool $L$ or not.

\paragraph{Case 1: $\qmg<\frac{1-\eta\left(3-2f\right)}{\left(1-f\right)\left(1-\eta\right)}$.} In this case, it follows that
\begin{equation}
   \pi^\LP_L-\pi^\LP_H > q_i\left[\frac{1}{2} \left(1-\eta\right) \left(1-\left(1-f\right)\qmg\right) - \eta \left(1-f\right)\right]>0,
\end{equation}
and therefore it is optimal for all liquidity providers to choose pool $L$, leading to $\qmg^\star=1$ -- a corner equilibrium where all liquidity providers choose pool $L$.

\paragraph{Case 2: $\qmg>\frac{1-\eta\left(3-2f\right)}{\left(1-f\right)\left(1-\eta\right)}$.} In this case,  $\frac{1}{2} \left(1-\eta\right) \left(1-\left(1-f\right)\qmg\right) - \eta \left(1-f\right)<0$ and therefore for any $q_i\leq\frac{\Gamma}{1-f}$
\begin{equation}
   \pi^\LP_L-\pi^\LP_H > q_i\left[\frac{1}{2} \left(1-\eta\right) \left(1-\left(1-f\right)\qmg\right) - \eta \left(1-f\right)\right]<0,
\end{equation}
That is, all liquidity providers who do not optimally monitor their quotes choose to provide liquidity on the high-fee pool $H$. The marginal LP, if it exists, is the one for whom the expected profit on the high-fee and low-fee pool are equal, conditional on paying the gas fee to monitor their quotes:
\begin{equation}\label{eq:pi_diff_general}
    \pi^\LP_L-\pi^\LP_H = q_i \left[\frac{1}{2} \left(1-\eta\right) \left(1-\left(1-f\right)\qmg\right)\right] - \eta \Gamma  =0,
\end{equation}
which leads to the same fixed points $q_i=\qmg$ as in Proposition \ref{prop:equilibria}. 

% \subsection{A more general condition for interior equilibrium existence}

% Formally, the LP reaction function can be augmented as:
% \begin{equation}\label{eq:reaction}
%     \tilde{q}\left(\qmg\right)=\begin{cases}
%     1, & \text{ for } \qmg \leq \frac{1-\eta\left(1+2\min\left\{\Gamma,1-f\right\}\right)}{\left(1-f\right)\left(1-\eta\right)}\\
%     \frac{2\eta\Gamma}{ \left(1-\eta\right) \left(1-\left(1-f\right)\qmg\right)}, & \text{ for } \qmg \in \left(\frac{1-\eta\left(1+2\min\left\{\Gamma,1-f\right\}\right)}{\left(1-f\right)\left(1-\eta\right)}, \frac{1-\eta\left(1+\Gamma\right)}{\left(1-f\right)\left(1-\eta\right)}\right] \\
%     2, & \text{ for } \qmg > \frac{1-\eta\left(1+\Gamma\right)}{\left(1-f\right)\left(1-\eta\right)}.
%     \end{cases}
% \end{equation}
% The existence condition for a stable equilibrium is:
% \begin{equation}
%  \frac{1-\eta\left(1+2\min\left\{\Gamma,1-f\right\}\right)}{\left(1-f\right)\left(1-\eta\right)}<\frac{1}{2\left(1-f\right)} - \frac{1}{2\left(1-f\right)}\sqrt{1-\eta\frac{1+8\left(1-f\right)\Gamma}{1-\eta}}<\frac{1-\eta\left(1+\Gamma\right)}{\left(1-f\right)\left(1-\eta\right)},
% \end{equation}
% which translates to $\Gamma\in\left[\underline{\Gamma},\bar{\Gamma}\right)$, where
% \begin{align}
%     \underline{\Gamma}&=\max\left\{f\frac{1-\eta}{2\eta},\frac{1+\eta(2f-3)}{1-\eta}\right\} \text{ and } \nonumber \\
%     \bar{\Gamma}&=\begin{cases} \frac{1-\eta}{8\eta(1-f)} & f \leq 0.75 \\
%                            \frac{(2f-1)(1-\eta)}{\eta} & f>0.75 \end{cases}
% \end{align}


\end{proof}





