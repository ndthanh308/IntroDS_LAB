\paragraph{Asset and agents.} Consider a continuous time model of trade in a single token $\T$ with expected value $v_t>0$. Three risk neutral trader types consummate trade in this market: a continuum of liquidity providers ($\LP$s), liquidity traders ($\LT$s), and arbitrageurs ($\A$). Trade occurs either because public news  triggers a change in the common value of the asset, or because market participants have heterogeneous private values for the asset.

Arrival times of news and private value shocks follow independent Poisson processes with rates $\eta\in\left(0,1\right)$ and $1-\eta$, respectively.\footnote{This is without loss of generality, as what matters in the model is the \emph{relative} arrival rate of news relative to liquidity traders.} For notational compactness, we first characterize the generic shock distribution and then describe its effects on arbitrageurs or liquidity traders. Conditional on an event at time $t$, the asset value changes to $v_t\left(1+{\cal I}\tilde{\delta}\right)$ for all traders in the case of a common value shock, or for an arriving  ($\LT$) in the case of the private value shock.  Here,  ${\cal I}$ is an indicator that takes on the value of $1$ if the taker buys and $-1$ if the taker sells. The value innovation $\tilde{\delta}$ has a probability density
\begin{equation}
    \phi\left(\delta\right)=\frac{1}{2\Delta \sqrt{1+\delta}} \; \; \text{ for } \delta\in\left[0,\Delta^2-1\right],
\end{equation}
thus $\sqrt{1+\tilde{\delta}}$ is uniformly distributed between $\left[1,\Delta\right]$. This assumption is innocuous and made for tractability purposes. 



When news arrives, the innovation is to the common value of the token. (As we are agnostic as to the source of value of cryptocurrencies, this common value shock could include the possibility of resale on another exchange.) After such a shock, an arbitrageur $\textbf{A}$ trades with the liquidity providers whenever profitable, and $\LP$s face an adverse selection loss. Conversely, when a liquidity trader enters the market, they experience a private value shock --- and liquidity providers continue to value the token at $v_t$. In what follows for expositional simplicity, as in \citet{foucault2013liquidity}, we focus on a one sided market in which liquidity takers act as buyers, and news lead to an increase in token value.

Liquidity providers ($\LP$) differ in their endowments of the token. Each provider $i$ can supply at most $q_i \diff i$  of the token, where $q_i$ follows an exponential distribution with scale parameter $\lambda$. The right skew of the distribution captures the idea that there are many  low-endowment liquidity providers such as retail traders, but few high-capital $\LP$s such as sophisticated quantitative funds.   Heterogeneity in $\LP$ size is captured by $\lambda$, where a larger $\lambda$ naturally corresponds to a larger dispersion of endowments and larger aggregate liquidity supply. Given the endowment distribution, collectively $\LP$s  supply at most
\begin{equation}
   S  =   \int_0^\infty q_i \frac{1}{\lambda} e^{-\frac{q_i}{\lambda}} \diff i = \lambda
\end{equation}
tokens. 


\paragraph{Trading environment.} Traders can interact in two liquidity pools in which token trade occurs against a num\'{e}raire asset (cash). At the start of the trading game, each liquidity provider (LP) deposits liquidity to a single pool within a symmetric price band around the current asset value $\left[\frac{v}{\left(1+r\right)^2}, v(1+r)^2\right]$, where $r \geq 0$. Here, we make use of the fact that V3 features ``price bands,''  and thus liquidity can be consumed with a bounded price impact. Within this range, prices in both pools satisfy a constant product bonding curve as in \citet{Uniswapv3Core2021}. In particular, for pool $k$,
\begin{equation}
    \underbrace{\left(T_k+\frac{L_k}{\sqrt{v}\left(1+r\right)}\right)}_\text{virtual token reserves}\underbrace{\left(T_k v + L_k\frac{\sqrt{v}}{1+r}\right)}_\text{virtual numeraire reserves}=L_k^2,
\end{equation}
where $T_k$ is the amount of tokens deposited on pool $k$ and $L_k$ is the liquidity level of pool $k$, defined as
\begin{equation}
    L_k=\frac{T_k}{\frac{1}{\sqrt{v}}-\frac{1}{\sqrt{v}}\left(1+r\right)}.
\end{equation}
To purchase $\tau$ tokens, a trader needs to deposit an amount $n\left(\tau\right)=\tau T_k \frac{v\left(1+r\right)}{\tau+\left(1+r\right)\left(T_k-\tau\right)}$ of num\'{e}raire into the pool, where $n\left(\tau\right)$ is the solution to the invariance condition
\begin{equation}
    \underbrace{\left(T_k-\tau+\frac{L_k}{\sqrt{v}\left(1+r\right)}\right)}_\text{virtual token reserves}\underbrace{\left(T_k v + n\left(\tau\right) +  L_k\frac{\sqrt{v}}{1+r}\right)}_\text{virtual numeraire reserves}=L_k^2.
\end{equation}

Fees are levied on liquidity takers  as a fraction of the value of the trade and distributed pro rata to liquidity providers.  Crucially, the pools have different fees.  One pool charges a low fee, and one pool charges a high fee which we denote $\ell$ and $h$ respectively.  Specifically, to purchase $\tau$ units of the token on the low fee pool, the total cost to a taker is $\left(1+\ell\right) n\left(\tau, T_\ell\right)$. The $\LP$s  in the pool receive $\ell n\left(\tau, T_\ell\right)$ in fees. In addition, consistent with gas costs on Ethereum, all traders incur a fixed execution cost $\Gamma \diff i >0$ to interact with the market.

Figure \ref{fig:timing_revised} illustrates the timing of the model.

% Figure environment removed

To ensure the possibility of liquidity re-balancing in both pools, we assume that innovations are large enough to ensure that $\LP$s may need to rebalance their position on the high fee pool or:

\begin{ass}\label{ass:Delta} The size of innovations are sufficiently large so that there is a positive probability that liquidity providers need to re-balance on the high-fee pool. That is, $\Delta>\left(1+r\right)\sqrt{1+h}$.
\end{ass}

\subsection{Equilibrium}

\subsubsection{Optimal trade size}

First, consider the decisions of arbitrageurs and liquidity traders holding a value $v\left(1+\delta\right)$ for the asset. Faced with pool sizes of $T_\ell$ and $T_h$ in the low and high pool respectively, their optimal trade on pool $k$ maximizes their expected profit, net of fees and price impact:
\begin{equation}
    \max_{\tau} \text{Profit $\LT$}\left(\tau,\delta\right) \equiv \tau v \left(1+\delta\right) - \left(1+f_k\right) \tau T_k \frac{v\left(1+r\right)}{\tau+\left(1+r\right)\left(T_k-\tau\right)},
\end{equation}
which yields the optimal trade quantity:
\begin{equation}\label{eq:optimal_trade_size}
    \tau^\star\left(\delta\right)=T_k\min\left\{1,\frac{1+r}{r}\max\left\{0,1-\sqrt{\frac{1+f_k}{1+\delta}}\right\}\right\}.
\end{equation}
From equation \eqref{eq:optimal_trade_size}, a trader with valuation $v\left(1+\delta\right)$ only trades on pool $k$ if the gains from trade are larger than the liquidity fee, i.e., $\delta>f_k$. Further, if $\delta>\left(1+f_k\right)\left(1+r\right)^2-1$ so that the gains from trade are larger than the maximum price impact, then the trader consumes all available liquidity in the pool.

\subsubsection{Fee revenue for liquidity providers from private value trades}

The revenue for liquidity providers can be expressed as the product of the pool fee and the num\'{e}raire deposit required from liquidity traders to purchase \(\tau^\star\) token units, denoted by \(n(\tau^\star, T_k)\). That is, 
\begin{equation}\label{eq:revenue_raw}
    f_k n\left(\tau^\star\left(\delta\right), T_k\right) = f_k v T_k \min\left\{1+r,\frac{1+r}{r}\max\left\{0,\sqrt{\frac{1+\delta}{1+f_k}}-1\right\}\right\}.
\end{equation}

If the innovation $\delta$ corresponds to a private rather than common value shock, then an arbitrageur optimally steps in to reverse the liquidity trade as described in \citet{LeharParlour2021}. In this case, liquidity providers effectively earn double the fee revenue in \eqref{eq:revenue_raw} without affecting the capital structure of the pool; there is neither a capital gain nor a loss for the $\LP$s. 

The fee revenue in \eqref{eq:revenue_raw} scales linearly with the size of the pool $T_k$. Since fee proceeds are distributed pro-rata among liquidity providers based on their share $\frac{q_i}{T_k}$, it follows that fee revenue an $\LP$ with endowment $q_i$ providing liquidity on pool $k$  increases linearly in their endowment:
\begin{align}\label{eq:fee_revenue_delta}
    \text{FeeRevenue}_{i,k}\left(\delta\right)&=2\frac{q_i}{T_k}f_k n\left(\tau^\star, T_k\right) \nonumber \\
    &= 2q_i v f_k \min\left\{1+r,\frac{1+r}{r}\max\left\{0,\sqrt{\frac{1+\delta}{1+f_k}}-1\right\}\right\}.
\end{align}

The expression in \eqref{eq:fee_revenue_delta} denotes the fee revenue conditional on the private value $\delta$ of the incoming trade. To compute the expected fee revenue, we integrate this expression across all posible value shocks:
\begin{align}
\mathbb{E}\text{FeeRevenue}_{i,k}&=\int_{\delta=1}^{\Delta^2-1} \text{FeeRevenue}_{i,k}\left(\delta\right) \phi\left(\delta\right)\diff \delta \nonumber \\
&=q_i \underbrace{v \frac{f_k (r+1) \left(2 \Delta -r\sqrt{f_k+1} -2 \sqrt{f_k+1}\right)}{\Delta }}_{\equiv \mathcal{L}\left(f_k\right)},
\end{align}
where we define $\mathcal{L}\left(f_k\right)$ as the \emph{liquidity yield}: that is, the per-unit $\LP$ fee revenue from supplying liquidity to $\LT$s in pool $k$.

\begin{lem}\label{lem:liq_revenue} There exists a threshold fee level $\overline{f}>0$ such that the liquidity revenue $\mathcal{L}\left(f_k\right)$ first increases in the pool fee $f_k$ for $f\leq\overline{f}$, then decreases in the pool fee for $f>\overline{f}$.
\end{lem}

Lemma \ref{lem:liq_revenue} points out to a non-linear relationship between fee levels and liquidity yield. Initially, as fees increase, the enhanced revenue from higher fees outweighs the decrease in trading volume due to increased transaction costs, resulting in a net gain in revenue. However, beyond a certain fee threshold, the drop in trading volume dominates the larger fee, leading to a decrease in overall revenue. A salient implication is that if pool fees are large enough, the liquidity yield on the high fee pool may exceed the yield on the low-fee pool.

\subsubsection{Adverse selection cost for liquidity providers}

If news occurs (i.e., if $\delta$ represents a common value shock), liquidity providers trade against arbitrageurs rather than liquidity traders. In this case, there is no subsequent price reversal following the initial trade. The capital structure of the liquidity pool changes, as arbitrageurs remove the more valuable asset: i.e., buy tokens upon a positive common value shock. While $\LP$s earn fee revenues on arbitrage trades, they also incur adverse selection losses by trading against the direction of the news. Moreover, if the magnitude of news is large enough that arbitrageurs remove all tokens supplied in the price range, then $\LP$s face additional costs, that is a gas fee $\Gamma \diff i$ to re-balance liquidity around the new asset value.

Table \ref{tab:fee_rev} delineates the $\LP$ fee revenue from selling tokens to arbitrageurs, as well as the marked-to-market value of the tokens sold. If the size of news ($\delta$) does not exceed the pool fee, then arbitrageurs do not trade since  the potential profit does not justify the transaction cost. Conversely, if the news size is larger than pool fee, then arbitrageurs execute a trade proportional to the size of the pool, and they exhaust the available liquidity on the price range if the news is large enough: specifically, if $\delta>\left(1+f_k\right)\left(1+r\right)^2-1$. The profit for liquidity providers in each scenario is the difference between the revenue and the marked-to-market value. Notably, the profit is consistently negative, since $\LP$s are trading against the direction of news.

\begin{table}[]
    \caption{Fee revenue and capital losses on arbitrage trades}
    \label{tab:fee_rev}
\begin{center}
    \begin{tabular}{lll}
\toprule
News size & Revenue (numeraire) & Marked-to-market token value \\ 
\cmidrule{1-3}
$\delta\leq f_k$     & 0 & 0  \\
$\delta \in \left(f_k, \left(1+f_k\right)\left(1+r\right)^2-1\right]$     &  $v q_i \frac{1+r}{r}\left(1+f_k\right)\left(\sqrt{\frac{1+\delta}{1+f_k}}-1\right)$ & $v q_i \frac{1+r}{r}\left(1+\delta\right)\left(1-\sqrt{\frac{1+f_k}{1+\delta}}\right)$ \\
$\delta>\left(1+f_k\right)\left(1+r\right)^2-1$ & $vq_i\left(1+f_k\right)\left(1+r\right)$ & $v q_i \left(1+\delta\right)$ \\
\bottomrule
\end{tabular}
\end{center}

\end{table}


The expected $\LP$ profit from trading with arbitrageurs equals $-q_i v \times \mathcal{A}\left(f_k\right)$,
where $\mathcal{A}\left(f_k\right)$ is the per-unit adverse selection cost from liquidity provision in pool $k$:
\begin{align}\label{eq:AScost}
\mathcal{A}\left(f_k\right)&=\mathbb{P}\left(f_k<\delta \leq (1+f_k)(1+r)^2-1\right) \times \frac{1+r}{r}\mathbb{E} \left[\left(1+f_k\right)+\left(1+\delta\right)-2\sqrt{\left(1+\delta\right)\left(1+f_k\right)}\right] + \nonumber \\
    &+ \mathbb{P}\left(\delta>(1+f_k)(1+r)^2-1\right) \times \left[\mathbb{E}\left(1+\delta\right)\right]-\left(1+f_k\right)\left(1+r\right) \Big\}.
\end{align}
 
\begin{lem}\label{lem:advsel_fees} The adverse selection cost $\mathcal{A}\left(f_k\right)$ decreases in the pool fee $f_k$. In particular, the high-fee pool has a lower adverse selection cost than the low-fee pool. 
\end{lem}

Lemma \ref{lem:advsel_fees} indicates that higher pool fees lower adverse selection costs through two mechanisms: First, they increase compensation per unit traded for liquidity providers (\(\LP\)s), enhancing their returns on trades with arbitrageurs. Second, higher fees discourage arbitrageur activity, effectively reducing the volume of informed trades. Figure \ref{fig:liqrev_as} showcases the results in Lemmas \ref{lem:liq_revenue} and \ref{lem:advsel_fees} and illustrates the comparative statics of liquidity yield and adverse selection cost with respect to the pool fee.

% Figure environment removed

Liquidity rebalancing costs arise only when news events are large enough to deplete all available liquidity within a given price range, pushing liquidity providers' (\(\LP\)s) positions ``out of range.'' Rebalancing only occurs post-news, since equally large liquidity trades would be reversed by arbitrageurs. Conditional on news arrival, the expected cost of rebalancing is 
\begin{equation}
    \mathcal{C}\left(k\right)= \mathbb{P}\left(\delta>(1+f_k)(1+r)^2-1\right) \Gamma = \Gamma\left(1-\frac{\sqrt{1+f_k}\left(1+r\right)}{\Delta}\right),
\end{equation}
which is decreasing in the pool fee $f_k$. This result is straightforward: smaller news events can cause arbitrageurs to deplete liquidity in low-fee pools, whereas it takes larger news to do the same in high-fee pools. Consequently, \(\LP\)s in lower fee pools face more frequent rebalancing and incur higher fixed costs.


\subsubsection{Liquidity provider pool choice}

Liquidity providers face a choice between the low and high fee pool or not participating in the market. An $\LP$ of size $q_i$ earns expected profit
\begin{align}\label{eq:profitLP}
    \pi_L &= q_i \left[\left(1-\eta\right) \mathcal{L}\left(\ell\right)-\eta \mathcal{A}\left(\ell\right)\right]-\eta \Gamma\left(1-\frac{\sqrt{1+\ell}\left(1+r\right)}{\Delta}\right) \text{ and } \\
    \pi_H &= q_i \left[\left(1-\eta\right) \mathcal{L}\left(h\right)-\eta \mathcal{A}\left(h\right)\right]-\eta \Gamma\left(1-\frac{\sqrt{1+h}\left(1+r\right)}{\Delta}\right) \nonumber,
\end{align}
from choosing pool $L$ or $H$, respectively. Equation \eqref{eq:profitLP} underscores the trade-off faced by liquidity providers (\(\LP\)s): balancing the liquidity yield from trades with liquidity traders (\(\LT\)s) against the adverse selection costs and the fixed gas costs associated with re-balancing their position.

First, consider the choice of participating in the market. An agent  only provides liquidity on pool $k$ if she is able to break even -- that is, if her endowment $q_i$ is large enough. We define $\underline{q}_L$ and $\underline{q}_H$ as the thresholds at which the participation constraints $\pi_L\left(q\right)=0$ and $\pi_H\left(q\right)=0$ are satisfied, respectively. If \(\underline{q}_k \geq 0\) for a pool \(k\), it indicates that any \(\LP\) with an endowment \(q_i\) at least equal to \(\underline{q}_k\) can join pool \(k\) and expect to earn a positive profit, with the marginal entrant breaking even.  Conversely, if \(\underline{q}_k < 0\), it suggests that pool \(k\) is not economically viable as the participation constraint is breached for all \(\LP\)s.

\begin{ass}\label{ass:eta}
    To avoid trivial cases, we focus on the case that both markets are potentially viable, or equivalently the intensity of news is low enough:
    \begin{equation}
        \eta \leq \min_k \frac{\mathcal{L}\left(k\right)}{\mathcal{L}\left(k\right)+\mathcal{A}\left(k\right)},
    \end{equation}
    such that $\underline{q}_k\geq 0$.
\end{ass}

Next, consider the choice between pools. Liquidity provider $i$ chooses the low-fee pool if and only if
\begin{equation}\label{eq:pi_diff}
    \pi_L-\pi_H = q_i \left[\left(1-\eta\right)\left(\mathcal{L}\left(\ell\right)-\mathcal{L}\left(h\right)\right)+\eta \underbrace{\left(\mathcal{A}\left(h\right)-\mathcal{A}\left(\ell\right)\right)}_{<0} \right] - \underbrace{\Gamma \frac{\eta(1+r)}{\Delta}\left(\sqrt{1+h}-\sqrt{1+\ell}\right)}_{>0}>0.
\end{equation}
Liquidity providers in the high-fee pool face both lower adverse selection and rebalancing costs. Therefore, the low-fee pool can only be chosen in equilibrium if it offers a higher liquidity yield, specifically if \(\mathcal{L}(\ell) - \mathcal{L}(h) > 0\), and if the intensity of news \(\eta\) is sufficiently low. Otherwise, all liquidity providers prefer to supply tokens to the high-fee pool if the participation constraint is satisfied. Further, equation \eqref{eq:pi_diff} highlights the economies of scale embedded in liquidity provision with fixed rebalancing costs. That is, if a liquidity provider of size $q$ prefers the low fee pool, then any liquidity provider with a larger endowment,  $\widetilde{q}>q$, also prefers the low fee pool.

% If there exists a fragmented equilibrium, then there exists a threshold $\qmg$ such that all $\LP$s with $q_i>\qmg$ choose the low-fee pool and all $\LP$s with $q_i\leq\qmg$ choose the high-fee pool, where
% \begin{equation}
%     \qmg=\Gamma \frac{\eta(1+r\left(\sqrt{1+h}-\sqrt{1+\ell}\right))}{\Delta \left[\left(1-\eta\right)\left(\mathcal{L}\left(\ell\right)-\mathcal{L}\left(h\right)\right)+\eta \left(\mathcal{A}\left(h\right)-\mathcal{A}\left(\ell\right)\right) \right]}
% \end{equation}

Proposition \ref{prop:equilibria} characterizes the equilibrium liquidity provision.

\begin{prop}\label{prop:equilibria}
%\begin{leftbar} \setlength{\parskip}{0ex}
%\emph{(Fragmentation)} 
\begin{enumerate}
    \item [i.]
If $\eta>\frac{\mathcal{L}\left(l\right)-\mathcal{L}\left(h\right)}{\mathcal{L}\left(l\right)-\mathcal{L}\left(h\right)+\mathcal{A}\left(l\right)-\mathcal{A}\left(h\right)}$, then  all $\LP$s with $q_i>\underline{q}_h$ deposit liquidity on the high fee pool. 
% \item[ii.]If $\eta\leq \frac{\mathcal{L}\left(l\right)-\mathcal{L}\left(h\right)}{\mathcal{L}\left(l\right)-\mathcal{L}\left(h\right)+\mathcal{A}\left(l\right)-\mathcal{A}\left(h\right)}$ and $\frac{\mathcal{C}\left(h\right)}{\mathcal{C}\left(\ell\right)}>\frac{\left(1-\eta\right)\mathcal{L}\left(h\right)-\eta\mathcal{A}\left(h\right)}{\left(1-\eta\right)\mathcal{L}\left(\ell\right)-\eta\mathcal{A}\left(\ell\right)}$, then all $\LP$s with $q_i>\underline{q}_\ell$ deposit liquidity on the low fee pool. 
\item [ii.] Otherwise, there exists a unique fragmented equilibrium characterized by marginal trader $\qmg^\star>\underline{q}_h$ which solves
\begin{equation}\label{eq:mg_eq}
    \qmg^\star = \Gamma \frac{\eta(1+r\left(\sqrt{1+h}-\sqrt{1+\ell}\right))}{\Delta \left[\left(1-\eta\right)\left(\mathcal{L}\left(\ell\right)-\mathcal{L}\left(h\right)\right)+\eta \left(\mathcal{A}\left(h\right)-\mathcal{A}\left(\ell\right)\right) \right]}
\end{equation}
such that all $\LP$s with $q_i\in\left(\underline{q}_h,\qmg^\star\right]$ deposit liquidity in the high fee pool and all $\LP$s with $q_i>\qmg^\star$ choose the low fee pool.
\end{enumerate}
%\end{leftbar}    
\end{prop}

Figure \ref{fig:region_equilibrium} illustrates the equilibrium regions in Proposition \ref{prop:equilibria}. When news intensity $\eta$ is high, or pool $H$ offers a substantially higher fee than pool $L$, liquidity suppliers gravitate towards pool $H$, resulting in a single-maker equilibrium. Conversely, a lower $\eta$ translates to lower adverse selection costs. If this is the case, or if the fee differential between the two pools is low, liquidity providers with large endowments $q$ migrate to the lower-fee pool to compete for order flow from small traders, causing liquidity to fragment between the two pools.

% Figure environment removed

Proposition \ref{cor:comp_stat_ms} establishes the impact of gas prices on the two pools' liquidity market shares. We can compute the liquidity market share of the low-fee pool in a fragmented equilibrium as
\begin{equation}
w_\ell=\frac{\exp\left(-\frac{\qmg-\underline{q}_h}{\lambda}\right)\left(\qmg+\lambda\right)}{\underline{q}_h+\lambda} \leq 1,
\end{equation}
with equality for $\Gamma=0$. That is, as fixed costs drop to zero, the low fee pool asymptotically captures the full market share.

\begin{prop}\label{cor:comp_stat_ms}
%\begin{leftbar} \setlength{\parskip}{0ex}
In equilibrium, the market share of the low fee pool $w_\ell$ decreases in the gas cost ($\Gamma$). 

%\end{leftbar}    
\end{prop}

We stress the critical role of fixed gas costs in driving market fragmentation. Since liquidity fee revenues and adverse selection costs are distributed pro-rata, in the absence of gas fees, all liquidity providers (\(\LP\)s) would converge on a single pool --- the one offering the optimal balance between fee yield and informational costs. For instance, if \(\Gamma=0\), all \(\LP\)s would select pool \(L\) if the news arrival rate is sufficiently low, as defined by \(\eta\leq\frac{\mathcal{L}(l)-\mathcal{L}(h)}{\mathcal{L}(l)-\mathcal{L}(h)+\mathcal{A}(l)-\mathcal{A}(h)}\), or choose pool \(H\) otherwise. It is the introduction of fixed costs that drives \(\LP\)s to segregate into different pools based on their size.



Figure \ref{fig:liqshares} shows that the market share of the low fee pool decreases in the gas cost $\Gamma$. A larger gas price increases the costs of re-balancing upon the arrival of large enough news, everything else equal, and incentivizes smaller $\LP$s to switch from the low fee pool to the high fee pool, since the arbitrageurs are less likely to fully consume liquidity there. Further, the right panel illustrates the extensive margin effect of gas prices: any increase in gas costs leads to a decrease in aggregate liquidity supply as some $\LP$ with low endowments are driven out of the market (that is, the threshold $\underline{q}_h$ increases in $\Gamma$).

% Figure environment removed





\subsection{Pool fragmentation and market quality}

We measure market quality by the realized gains from trade of liquidity traders. If the asset is traded on a sequence of pools, where $f_k$ and $T_k$ represent the fees and liquidity deposits on pool $k$, respectively, the expected gains from trade for liquidity traders are
\begin{equation}\label{eq:gains}
    \text{GainsFromTrade}\left(\left\{f_k\right\}_k\right)= v \mathbb{E}\left[ \sum_k  \tau^\star\left(f_k,\delta\right) \times \delta\right],
\end{equation}
where $\tau^\star\left(\delta\right)=T_k\min\left\{1,\frac{1+r}{r}\max\left\{0,1-\sqrt{\frac{1+f_k}{1+\delta}}\right\}\right\}$ is the optimal \textbf{LT} trade size, as defined in equation \eqref{eq:optimal_trade_size}.

Suppose an asset is traded on a single pool that imposes a liquidity fee $f$. From equation \eqref{eq:gains} it follows that the gains from trade for an $\LT$ with private value $1+\delta$ are equal to
\begin{equation}
        \text{GainsFromTrade}\left(f\mid \delta\right)=v \delta T_f\min\left\{1,\frac{1+r}{r}\max\left\{0,1-\sqrt{\frac{1+f}{1+\delta}}\right\}\right\}.
\end{equation}
The total token supply on the single pool equals $T_f=e^{-\underline{q}_f \lambda}\left(\underline{q}_f+\lambda\right)$, where $\underline{q}_f$ is the marginal liquidity provider such that all $\LP$s with endowment $q_i>\underline{q}_f$ join the pool. Here, the magnitude of the liquidity fee drives the trade-off between the participation of liquidity providers (\(\LP\)) and trading costs. A lower fee \( f \) results in fewer \(\LP\)s offering liquidity, a lower token supply $T_f$, which limits gains from trade for liquidity traders. In contrast, a higher fee increases trading costs, potentially outweighing the benefits of increased \(\LP\) participation.

\begin{prop}\label{prop:optimality}
For any single-pool fee $f\geq 0$, there exists a set of fees \( \{h, \ell\} \) for a two-pool fragmented market, where \( h = f \) and \( h > \ell \), that guarantees equal or higher gains from trade in a fragmented market compared to the single-pool market.
\end{prop}

Proposition \ref{prop:optimality} suggests that fragmentation with multiple fee levels improves market quality. Specifically, it is always possible to devise a fee structure in a fragmented market that yields (weakly) higher gains from trade than a single-fee market. The logic is as follows: First, the highest fee in the fragmented market is set equal to the single pool fee, ensuring that the marginal $\LP$ participating the market is the same across both scenarios (i.e., the $\LP$ with endowment $\underline{q}_h$). This condition guarantees the same aggregate liquidity supply in fragmented and non-fragmented markets. Second, a lower fee is then chosen for another pool to attract liquidity providers with higher token endowments, resulting in larger trade sizes per unit of supplied liquidity. This combination of larger liquidity trades and unchanged aggregate liquidity supply leads to higher gains from trade in a fragmented market. 



% Figure environment removed

Figure \ref{fig:gft} illustrates the result. As gas price increase, the gains from trade drop in both single-pool and fragmented markets, primarily because more $\LP$s are priced out which results in lower liquidity supply and higher price impact. Nevertheless, irrespective of the level of gas costs, the gains from trade are higher in the fragmented market.

We note that the argument discussed in this section is valid for \emph{any} single-fee pool, including an optimally designed one. In essence, if a fragmented fee structure can be designed to achieve higher gains from trade compared to an arbitrary single-pool fee, then a fee structure that dominates the optimally set single-pool fee achieves higher gains from trade than any single-fee pool.

\subsection{Model implications and empirical predictions}

\begin{pred}\label{pred:comp_stat_Gamma}
The liquidity market share of the low-fee pool  decreases in the gas fee $\Gamma$.
\end{pred}

Prediction \ref{pred:comp_stat_Gamma} follows directly from Proposition \ref{cor:comp_stat_ms} and Figure \ref{fig:liqshares}. A higher gas price increases the cost of liquidity re-balancing. Given that re-balancing is more frequently required in the low-fee pool due to more intense arbitrage activity, liquidity providers, particularly those with smaller endowments, optimally migrate to the high-fee pool in response to a gas cost increase.

\begin{pred}\label{pred:clienteles}
$\LP$s on the low-fee pool make larger liquidity deposits than $\LP$s on the high-fee pool.
\end{pred}

Prediction \ref{pred:clienteles} follows from the equilibrium discussion in Proposition \ref{prop:equilibria}. Liquidity providers with large token endowments ($q_i>\qmg$) deposit them in the low-fee pool since they are better positioned to actively manage liquidity due to economies of scale. $\LP$s with lower endowments ($q_i\leq\qmg$) either stay out of the market or choose pool $H$ which allows them to offer liquidity in a more passive manner. Figure \ref{fig:theory_liqsupply} illustrates this prediction by overlaying optimal pool choices on the distribution of $\LP$ endowments. Low-endowment $\LP$s (in blue) that are being rationed out of the market due to high gas cost, medium-endowment $\LP$s (gray) that deposit liquidity on pool $H$, and high-endowment $\LP$s (red) that choose the low-fee pool $L$. 

% Figure environment removed




\begin{pred}\label{pred:trade_size_volume}
The average trade size is higher on pool $H$ than on pool $L$. At the same time, trading volume is higher on pool $L$ than on pool $H$.
\end{pred}

Next, Prediction \ref{pred:trade_size_volume} deals with differences between incoming trades on the two liquidity pools. If liquidity traders and arbitrageurs find it optimal to trade on pool $H$ since $\delta>h$, then they would also trade on pool $L$ since $h>\ell$ and therefore $\delta>\ell$. However, the opposite is not true: $\LT$s and arbitrageurs with $\delta\in\left[\ell,h\right)$ only trade on pool $L$. In equilibrium, only a fraction of traders with sufficiently high private values are drawn to pool $H$.

\begin{pred}\label{pred:ly}
    In a fragmented market equilibrium, the liquidity yield is higher on the low fee pool than on the high fee pool.
\end{pred}

Prediction \ref{pred:ly} is a consequence of Proposition \ref{prop:equilibria}. The high fee pool offers better protection against adverse selection and re-balancing costs. If the low fee pool attracts a positive market share, then it necessarily compensates with a higher liquidity yield.

\begin{pred}\label{pred:clienteles_cs}
The average liquidity deposit on both the low- and- high fee pool increases with gas costs.
\end{pred}

An increase in the gas cost $\Gamma$ has two effects: first, the $\LP$s with the lowest endowments on pool $L$ switch to pool $H$. As a result, the average deposit on pool $L$ increases. Second, the $\LP$s with low endowments on pool $H$ may leave the market. Both channels translate to a higher average deposit on pool $H$, which experiences an inflow (outflow) of relatively high (low) endowment $LP$ following an increase in gas costs.


\begin{pred}\label{pred:updates}
$\LP$s re-balance liquidity more frequently on the low-fee than on the high-fee pool.
\end{pred}

Liquidity providers re-balance their positions in a pool charging a fee $f$ only when the magnitude of news exceeds a threshold, specifically if $\delta > (1+f)(1+r)^2 - 1$. The likelihood of re-balancing given the news is $(1-\frac{\sqrt{1+f}(1+r)}{\Delta})$. Consequently, the duration of a liquidity cycle, expressed as $\frac{1}{\eta(1-\frac{\sqrt{1+f}(1+r)}{\Delta})}$, increases in the pool fee level.

\begin{pred}\label{pred:as}
    Adverse selection cost is higher on the low fee pool than on the high fee pool.
\end{pred}

This prediction follows directly from Lemma \ref{lem:advsel_fees}: a higher pool fee serves as a deterrent to arbitrageurs, particularly if the size of news remains below a threshold.





%\insert{comments.tex}

