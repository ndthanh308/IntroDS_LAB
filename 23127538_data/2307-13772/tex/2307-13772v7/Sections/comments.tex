What does this model give us?

\begin{enumerate}
  \item \textbf{Old prediction.} The liquidity market share of the low-fee pool  decreases in the gas fee $\Gamma$. \\
   \mz{Confirmed, since the market share of the low fee pool is $e^{-\frac{\max\left\{0,\qmg-q_h\right\}}{\lambda}}$ and $\qmg-q_h$ increases in $\Gamma$ in any fragmented equilibrium. }
  \item \textbf{Old prediction.} $\LP$s on the low-fee pool make larger liquidity deposits than $\LP$s on the high-fee pool.
    \mz{Confirmed.}
    \item \textbf{Old prediction.} The average trade size is higher on pool $H$ than on pool $L$. At the same time, trading volume is higher on pool $L$ than on pool $H$. \\
    \mz{Confirmed. Trades with low enough private values or news size only go to low-fee pool, if private value/news is high enough the trade goes on both pools.}
    \item \textbf{Old prediction.} The average liquidity deposit on both the low- and- high fee pool increases with gas costs.
    \mz{Likely confirmed, mechanism is very similar. First, the $\LP$s with the lowest endowments on pool $L$ switch to pool $H$. As a result, the average deposit on pool $L$ increases. Second, the $\LP$s with low endowments on pool $H$ may leave the market. }
    \item \textbf{Old prediction.} $\LP$s update liquidity more frequently on the low-fee than on the high-fee pool.\\
    \mz{Confirmed. Liquidity more likely to hit the bound on low fee pool.}
    \item \textbf{Old prediction.} A larger gas price leads to more frequent liquidity updates on the low-fee pool.\\
    \mz{\textbf{Not true} in the current model, the frequency of re-balancing does not depend on gas costs.}
    \item \textbf{New prediction.} Adverse selection cost is higher on the low fee pool than on the high fee pool. \\
    \mz{True in the data, we have the analysis.}
    \item \textbf{New prediction.} In a fragmented equilibrium, the liquidity yield must be higher on the low fee pool than on the high fee pool. \\
    \mz{True in the data, we have the analysis on liquidity yield.}
    \item Model has new, more realistic ingredients:
        \begin{itemize}
            \item Price impact on high- and low-fee pools.
            \item Adverse selection costs.
            \item A price range which is Uniswap v3-specific (but exogenous).
            \item A clear rationale for re-balancing (i.e., hit the price limit).
        \end{itemize}
    
\end{enumerate}}

