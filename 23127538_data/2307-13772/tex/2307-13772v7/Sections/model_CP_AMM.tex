\paragraph{Asset and trading enviroment.} Consider a continuous time model of trade in a single token $\T$ with expected value $v_t>0$. The asset is traded in two liquidity pools in which token trade occurs against a num\'{e}raire asset (cash). Prices in both pools satisfy a constant product bonding curve, as common in practice. So, for a pool with $T$ tokens and $N$ of the num\'{e}raire good, 
\begin{equation}\label{eq:bond_curve}
    T \times N = \text{constant}.
\end{equation}
To purchase $\Delta T$ tokens, a traders needs to add 
\begin{align}
    \Delta N & = \frac{T \times N}{T - \Delta T} - N \\
    & = \frac{v T^2}{T-\Delta T}-v T \nonumber
\end{align}
units of cash to the pool, where the last equality follows from the pool trading at the fair price, that is $N=v \times T$.

Fees are levied on liquidity takers  as a fraction of the value of the trade and distributed pro rata to liquidity providers.  The pools have different fees.  One pool charges a low fee, and one pool charges a high fee which we denote $\ell$ and $h$ respectively. Specifically, to purchase $\Delta T$ units of the token on the low fee pool, the total cost to a taker is
\begin{equation}
    \Delta N \left(1+\ell\right) = v T \left(1+\ell\right)\left(\frac{T}{T-\Delta T}-1\right),
\end{equation}
while the liquidity providers receive a fee revenue equal to $\Delta N \ell$.

In addition, consistent with gas costs on Ethereum,  liquidity providers incur a fixed execution cost $\Gamma>0$ to interact with the market.  

\paragraph{Liquidity providers.}  Liquidity providers ($\LP$) differ in their endowments of the token. Each provider $i$ can supply at most $q_i$  of the token, where $q_i$ follows a distribution $\varphi\left(q\right)$ between $\left[\underline{q},\overline{q}\right]$ tokens. \mz{This can be defined later, or left general?}

\paragraph{Traders.} Traders arrive at the market as a Poisson process with unit intensity. Upon arrival, with probability $\eta$ there is news: i.e., the common asset value increases to $v\left(1+\tilde{\theta}_\text{news}\right)$. In this case, the incoming trader is an arbitrageur ($\A$). With probability $1-\eta$, a liquidity trader ($\LT$) receives a positive shock to their private value $v\left(1+\tilde{\theta}_\text{private}\right)$. 

The two shocks are independently and identically distributed: with probability $\alpha$ the shock is $\theta_b$ and with probability $1-\alpha$ the shock is $\theta_s$, with
\begin{equation}
    \theta_b > h > \theta_s > l.
\end{equation}
Private value trades are immediately reversed by arbitrageurs, whereas common value trades are not and trigger costly re-balancing by liquidity providers.

Figure \ref{fig:timing_revised} illustrates the timing of the model.

% Figure environment removed 

\subsection{Equilibrium}

Consider the general case of a trader with valuation $v\left(1+\tilde{\theta}\right)$ -- that can be either a private or common value shock. The optimal trade quantity on a pool with liquidity supply $T_k$ and fee $f_k$ solves:
\begin{equation}
    \max_\tau v\left(1+\tilde{\theta}\right) \tau - \left(1+\ell\right)\left(\frac{v T_k^2}{T_k-\tau}-v T_k\right),
\end{equation}
which leads to the optimal trade quantity
\begin{equation}
    \tau^\star= v T_k \max\left\{0,\left(1-\sqrt{\frac{1+f_k}{1+\tilde{\theta}}}\right)\right\}.
\end{equation}
The total revenue for liquidity providers is
\begin{equation}
    f_k\left(\frac{v T_k^2}{T_k-\tau^\star}-v T_k\right)=f_k T_k v \left(\sqrt{\frac{1+\tilde{\theta}}{1+f_k}}-1\right).
\end{equation}
Since the revenue is prorated across all $\LP$ at rates $\frac{q_i}{T_k}$, the revenue for liquidity provider $i$ is proportional to their endoment:
\begin{equation}
    q_i\text{Revenue}\left(f_k,\tilde{\theta}\right)=q_i f_k v \left(\sqrt{\frac{1+\tilde{\theta}}{1+f_k}}-1\right)
\end{equation}