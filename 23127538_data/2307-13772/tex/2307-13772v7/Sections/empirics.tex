\subsection{Liquidity supply on high- and low-fee pools}

To formally test the model predictions and quantify the differences in liquidity supply across fragmented pools, we build a panel data set for the 32 fragmented pairs in our sample where the unit of observation is pool-day. We estimate linear regressions of liquidity and volume measures on liquidity fees and gas costs:
\begin{equation}
    y_{ijt}=\alpha + \beta_0 d_\emph{low-fee, ij} + \beta_1 \text{GasPrice}_{jt} + \beta_2 \text{GasPrice}_{jt} \times d_\emph{low-fee, ij} + \sum \beta_k \emph{Controls}_{ijt} + \theta_j + \delta_w + \varepsilon_{ijt},
\end{equation}
where $y$ is a variable of interest, $i$ indexes liquidity pools, $j$ runs over asset pairs, and $t$ and $w$ indicates days and weeks, respectively. The dummy $d_\emph{low-fee, ij}$ takes the value one for the pool with the lowest fee in pair $j$ and zero else. 

Further, our set of controls includes pair and week fixed effects, the log aggregate trading volume and log liquidity supply (i.e., total value locked) for day $t$ across all pools $i$. Volume and liquidity are measured in US dollars. We also control for daily return volatility, computed as the range between the daily high and low prices for a given pair $j$ \citep[following][]{Alizadeh2002}: 
\begin{equation}
    \text{Volatility}_{jt}=\frac{1}{2\sqrt{\log 2}} \log\left(\frac{\text{High}_{jt}}{\text{Low}_{jt}}\right).
\end{equation}
To measure volatility for fragmented pairs that actively trade in multiple pools, we select the pool with the highest trading volume for a given day.

Consistent with Figure \ref{fig:stat_liq}, we show in Table \ref{tab:markeshare} that most of the capital deployed to provide liquidity for a given pair is locked in high-fee pools. At the same time, low-fee pools attract much larger trading volume. Models (1) and (5) show that the average low-fee pool attracts 39.5\% of liquidity supply for the average pair (that is, equal to $\nicefrac{(100-20.92)}{2}$) while it executes 62\% (i.e., $\nicefrac{(100+24.62)}{2}$) of the total trading volume. At a first glance, it would seem that a majority of capital on decentralized exchanges is inefficiently deployed in pools with low execution probability. We will show that, in line with our model, the difference is driven by heterogeneous rebalancing costs across pools, leading to the formation of $\LP$ clienteles. 

The regression results in Table \ref{tab:markeshare} support Prediction \ref{pred:comp_stat_Gamma}, stating that market share differences between pools are linked to variation in fixed transaction costs on the blockchain. A one-standard deviation increase in gas prices leads to a 4.63 percentage point increase in the high-fee liquidity share. The results suggests that blockchain transaction costs have an economically meaningful and statistically significant impact on liquidity fragmentation. In line with the theoretical model in Section \ref{sec:model}, a jump in gas prices leads to a reshuffling of liquidity supply from low- to high-fee pools. 

Evidence suggests that a higher gas price leads to a 6.52\% lower volume share for the low-fee pool. This outcome is natural, as the incoming order flow is optimally routed to the high-fee pool, following the liquidity providers.


% Table created by stargazer v.5.2.3 by Marek Hlavac, Social Policy Institute. E-mail: marek.hlavac at gmail.com
% Date and time: Tue, Oct 25, 2022 - 8:34:54 PM

\begin{table}
  \caption{Liquidity pool market shares and gas prices}   \label{tab:markeshare}
\begin{minipage}[t]{1\columnwidth}%
\footnotesize
			This table reports the coefficients of the following regression:
	\begin{align*}
    \text{MarketShare}_{ijt}=\alpha + \beta_0 d_\emph{low-fee, ij} + \beta_1 \text{GasPrice}_{jt} + \beta_2 \text{GasPrice}_{jt} \times d_\emph{low-fee, ij} + \sum \beta_k \emph{Controls}_{ijt} + \theta_j + \varepsilon_{ijt}
    \end{align*}
 	where the dependent variable is the liquidity or trading volume market share for pool $i$ in asset pair $j$ on day $t$. $d_\emph{low-fee, ij}$ is a dummy that takes the value one for the pool with the lowest fee in pair $j$ and zero else. $\emph{GasPrice}_{jt}$ is the average of the lowest 100 bids on liquidity provision events across all pairs on day $t$, standardized to have a zero mean and unit variance. \emph{Volume} is the natural logarithm of the sum of all swap amounts on day $t$, expressed in thousands of US dollars. \emph{Total value locked} is the natural logarithm of the total value locked on Uniswap v3 pools on day $t$, expressed in millions of dollars.\emph{Volatility} is computed as the daily range between high and low prices on the most active pool for a given pair.All regressions include pair and week fixed-effects. Robust standard errors in parenthesis are clustered by week and ***, **, and * denote the statistical significance at the 1, 5, and 10\% level, respectively.  The sample period is from May 4, 2021 to July 15, 2023. 
\end{minipage}
\small
\begin{center}
\resizebox{\textwidth}{!}{
\begin{tabular}{lcccc@{\hskip 0.3in}cccc}
\toprule
 & \multicolumn{4}{c}{Liquidity market share (\%)} & \multicolumn{4}{c}{Volume market share (\%)} \\ 
 & (1) & (2) & (3) & (4) & (5) & (6) & (7) & (8) \\
 \cmidrule{1-9}

$ d_\text{low-fee}$ & -20.92*** & -20.92*** & -20.92*** & -20.94*** & 24.62*** & 24.63*** & 24.62*** & 24.71*** \\
 & (-27.42) & (-27.41) & (-27.42) & (-23.95) & (20.55) & (20.56) & (20.55) & (18.54) \\
Gas price $\times$ $ d_\text{low-fee}$ & -4.63*** & -4.62*** & -4.63*** &  & -6.52*** & -6.52*** & -6.52*** &  \\
 & (-7.32) & (-7.32) & (-7.32) &  & (-5.92) & (-5.92) & (-5.92) &  \\
Gas price & 2.31*** & 2.31*** & 2.31*** &  & 3.63*** & 3.61*** & 3.61*** &  \\
 & (7.32) & (7.32) & (7.32) &  & (7.33) & (7.30) & (7.26) &  \\
Volume & 0.00 & 0.00 & 0.00 & 0.00 & -0.19** & -0.20** & -0.19** & -0.12 \\
 & (0.65) & (1.33) & (0.65) & (0.66) & (-2.54) & (-2.61) & (-2.50) & (-1.56) \\
Total value locked & -0.00 & -0.00 &  & -0.00 & 0.58 & 0.58 &  & 0.44 \\
 & (-0.58) & (-0.06) &  & (-0.64) & (1.44) & (1.44) &  & (1.10) \\
Volatility & -0.29 &  & -0.29 & -0.28 & -1.15*** &  & -1.15*** & -1.13** \\
 & (-0.90) &  & (-0.90) & (-0.82) & (-2.74) &  & (-2.74) & (-2.56) \\
Constant & 60.45*** & 60.46*** & 60.45*** & 60.46*** & 41.96*** & 41.99*** & 41.96*** & 41.96*** \\
 & (158.00) & (158.46) & (158.00) & (137.54) & (69.99) & (70.22) & (70.02) & (62.81) \\
Pair FE & Yes & Yes & Yes & Yes & Yes & Yes & Yes & Yes \\
Week FE & Yes & Yes & Yes & Yes & Yes & Yes & Yes & Yes \\
Observations & 40,288 & 40,288 & 40,288 & 40,288 & 36,059 & 36,059 & 36,059 & 36,059 \\
 R-squared & 0.10 & 0.10 & 0.10 & 0.09 & 0.13 & 0.13 & 0.13 & 0.12 \\ \hline
\bottomrule
\multicolumn{9}{l}{Robust t-statistics in parentheses. Standard errors are clustered at week level.  *** p$<$0.01, ** p$<$0.05, * p$<$0.1} \\
\end{tabular}
}
\end{center}
\end{table} 

What drives the market share gap across fragmented pools? In Table \ref{tab:ordersize} we document stark differences between the characteristics of individual orders supplying or demanding liquidity on pools with low and high fees. On the liquidity supply side, model (1) in Table \ref{tab:ordersize} shows that the average liquidity mint is 107.5\% larger on low-fee pools, which supports Prediction \ref{pred:clienteles} of the model.\footnote{Since all dependent variables are measured in natural logs, the marginal impact of a dummy coefficient $\beta$ is computed $\left(e^\beta-1\right) \times 100$ percent.} At the same time, there are 3.40 fewer unique wallets (Model 5) providing liquidity on the low-fee pool -- that is, a 34\% relative difference between high- and low-fee pools.

\begin{table}
\caption{Fragmentation and order flow characteristics}  \label{tab:ordersize}
\begin{minipage}[t]{1\columnwidth}%
\footnotesize
			This table reports the coefficients of the following regression:
	\begin{align*}
    y_{ijt}=\alpha + \beta_0 d_\emph{low-fee, ij} + \beta_1 \text{GasPrice}_{jt} d_\emph{low-fee, ij} + \beta_2 \text{GasPrice}_{jt} \times d_\emph{high-fee, ij} + \sum \beta_k \emph{Controls}_{ijt} + \theta_j + \varepsilon_{ijt}
    \end{align*}
	where the dependent variable $y_{ijt}$ can be (i) the log median mint size, (ii) the log median trade size, (iii) the log trading volume, (iv) the log trade count $\log(1+\# trades)$, (v) count of unique $\LP$ wallets interacting with a pool in a given day, (vi) the liquidity yield in bps for pool $i$ in asset $j$ on day $t$, computed as in equation \eqref{eq:liq_yield}, and (vii) the average liquidity mint price range for pool $i$ in asset $j$ on day $t$. Price range is computed as the difference between the top and bottom of the range, normalized by the range midpoint -- a measure that naturally lies between zero and two. $d_\emph{low-fee, ij}$ is a dummy that takes the value one for the pool with the lowest fee in pair $j$ and zero else. $d_\emph{high-fee, ij}$ is defined as $1-d_\emph{low-fee, ij}$. $\emph{GasPrice}_{jt}$ is the average of the lowest 100 bids on liquidity provision events across all pairs on day $t$, standardized to have a zero mean and unit variance. \emph{Volume} is the natural logarithm of the sum of all swap amounts on day $t$, expressed in thousands of US dollars. \emph{Total value locked} is the natural logarithm of the total value locked on Uniswap v3 pools on day $t$, expressed in millions of dollars. \emph{Volatility} is computed as the daily range between high and low prices on the most active pool for a given pair. All regressions include pair and week fixed-effects. Robust standard errors in parenthesis are clustered by week and ***, **, and * denote the statistical significance at the 1, 5, and 10\% level, respectively.  The sample period is from May 4, 2021 to July 15, 2023. 
\end{minipage}
\begin{center}
\resizebox{1\textwidth}{!}{  
\begin{tabular}{lccccccc}
\toprule
 & \multicolumn{1}{c}{Mint size} &  \multicolumn{1}{c}{Trade size} & \multicolumn{1}{c}{Volume} & \multicolumn{1}{c}{\# Trades}  & \multicolumn{1}{c}{\# Wallets} & \multicolumn{1}{c}{Liquidity yield} & \multicolumn{1}{c}{Price range} \\
 & (1) & (2) & (3) & (4) & (5) & (6) & (7) \\
\cmidrule{1-8}
$ d_\text{low-fee}$ & 0.73*** & -0.30*** & 0.89*** & 1.02*** & -3.40*** & 2.03*** & -0.18*** \\
 & (12.27) & (-10.05) & (14.23) & (32.95) & (-5.00) & (3.60) & (-41.84) \\
Gas price $\times$ $ d_\text{low-fee}$ & 0.37*** & 0.08*** & -0.03 & -0.22*** & -3.00*** & 3.57** & -0.00 \\
 & (4.96) & (3.75) & (-0.95) & (-7.29) & (-3.43) & (2.30) & (-0.47) \\
Gas price $\times$ $ d_\text{high-fee}$ & 0.58*** & 0.17*** & 0.24*** & 0.07** & -2.89*** & 5.57*** & -0.03*** \\
 & (7.52) & (8.81) & (5.95) & (2.46) & (-3.15) & (2.83) & (-4.65) \\
Volume & 0.37*** & 0.16*** & 0.43*** & 0.20*** & 1.22*** & 1.01 & -0.01** \\
 & (8.68) & (21.38) & (15.27) & (13.85) & (6.56) & (0.81) & (-2.56) \\
Total value locked & -0.16 & 0.11*** & 0.23** & -0.01 & -1.86 & -13.42 & -0.02 \\
 & (-1.30) & (3.54) & (1.99) & (-0.18) & (-0.99) & (-1.09) & (-0.99) \\
Volatility & -0.04 & -0.01 & -0.07 & 0.01 & -0.09 & 1.18** & 0.02*** \\
 & (-1.11) & (-1.34) & (-1.38) & (0.88) & (-1.03) & (2.21) & (3.98) \\
Constant & 1.88*** & 1.64*** & 5.27*** & 3.26*** & 10.12*** & 10.01*** & 0.59*** \\
 & (58.27) & (111.47) & (168.58) & (209.84) & (28.65) & (26.04) & (184.91) \\
 &  &  &  &  &  &  \\
Pair FE & Yes & Yes & Yes & Yes & Yes & Yes & Yes  \\
Week FE & Yes & Yes & Yes & Yes & Yes & Yes  & Yes \\
Observations & 21,000 & 36,059 & 36,059 & 40,288 & 40,288 & 40,252 & 24,058 \\
 R-squared & 0.26 & 0.53 & 0.55 & 0.52 & 0.37 & 0.09 & 0.42 \\ \hline
\bottomrule
\multicolumn{7}{l}{Robust t-statistics in parentheses. Standard errors are clustered at week level.} \\
\multicolumn{7}{l}{*** p$<$0.01, ** p$<$0.05, * p$<$0.1} \\
\end{tabular}
}
\end{center}
\end{table}


On the liquidity demand side, trades on the low-fee pool are 25.91\% smaller (Model 2), consistent with Prediction \ref{pred:trade_size_volume}. However, the low-fee pool executes almost three times the number of trades (i.e., trade count is 177\% higher from Model 4) and has 143\% higher volume than the high-fee pool (Model 3). 

Next, in line with Prediction \ref{pred:ly}, low-fee pools generate a higher liquidity yield. On average, liquidity providers on low-fee pools earn 2.03 basis points higher revenue than their counterparts on high-fee pools (Model 6), indicating significant positive returns resulting from economies of scale.

Our findings (Model 7) indicate that liquidity providers on low-fee pools select price ranges that are 30\% (=0.18/0.59) narrower when minting liquidity compared to those on high-fee pools. This pattern aligns with the capability of large LPs to adjust their liquidity positions frequently, enabling more efficient capital concentration. Similarly, \citet{caparros2023blockchain} report a higher concentration of liquidity in pools on alternative blockchains like Polygon, known for lower transaction costs than Ethereum.

The results point to an asymmetric match between liquidity supply and demand across pools. On low-fee pools, a few $\LP$s provide large chunks of liquidity for the vast majority of incoming small trades. Conversely, on high-fee pools there is a sizeable mass of small liquidity providers that mostly trade against a few large incoming trades. 

How does variation in fixed transaction costs impact the gap between individual order size across pools? We find that increasing the gas price by one standard deviation leads to higher liquidity deposits on both the low- and the high-fee pools (14.2\% and 30.1\% higher, respectively).\footnote{The relative effects are computed as $\nicefrac{0.37}{(1.88+0.73)}=13.8\%$ for low pools and $\nicefrac{0.58}{1.88}=30.85\%$ for high-fee pools, respectively.} The result supports Prediction \ref{pred:clienteles_cs} of the model. Our theoretical framework implies that a larger gas price leads to some (marginal) $\LP$s switching from the low- to the high-fee pool. The switching $\LP$s have low capital endowments relative to their low-fee pool peers, but higher than $\LP$s on the high-fee pool. Therefore, the gas-driven reshuffle of liquidity leads to a higher average endowment on both high- and low-fee pools. Consistent with the model, a higher gas price leads to fewer active liquidity providers, particularly on low-fee pools. Specifically, a one-standard increase in gas costs leads to a significant decrease in the number of $\LP$ wallets interacting daily with low- and high-fee pools, respectively (Model 5). 

While a higher gas price is correlated with a shift in liquidity supply, it has a muted impact on liquidity demand on low-fee pools. A higher gas cost is associated with 7.6\% larger trades (Model 2), likely as traders aim to achieve better economies of scale. At the same time, the number of trades on the low-fee pool drops by 19.7\% (Model 4) -- since small traders might be driven out of the market. The net of gas prices effect on aggregate volume on the low-fee pool is small and not statistically significant (Model 3). The result matches our model assumption that the aggregate order flow on low-fee pool is not sensitive to gas prices.
% \footnote{Formally, one could extend the model to assume that small $\LT$s arrive at the market at rate $\tilde{\theta}\left(\Gamma\right) \diff t$ and demand $f\left(\Gamma\right)$ units each, where $\tilde{\theta}\left(\Gamma\right)$ decreases in $\Gamma$ and $f\left(\Gamma\right)$ increases in $\Gamma$ such that $\tilde{\theta}\left(\Gamma\right)f\left(\Gamma\right) =\theta$.} 

On the high-fee pool, a higher gas price is also associated with a higher trade size, but also an increase in traded volume. As gas prices rise, liquidity providers switch from low- to high-fee pools. The outcome is greater depth and reduced price impact for liquidity demanders on high fee pools, which leads to higher trading volume.



\begin{table}[H]
\caption{Liquidity flows and gas costs on fragmented pools} \label{tab:flows}
\begin{minipage}[t]{1\columnwidth}%
\footnotesize
			This table reports the coefficients of the following regression:
	\begin{align*}
    y_{ijt}=\alpha + \beta_0 d_\emph{low-fee, ij} + \beta_1 \text{GasPrice}_{jt} d_\emph{low-fee, ij} + \beta_2 \text{GasPrice}_{jt} \times d_\emph{high-fee, ij} + \sum \beta_k \emph{Controls}_{ijt} + \theta_j + \varepsilon_{ijt}
    \end{align*}
	where the dependent variable $y_{ijt}$ can be (i) the aggregate dollar value of mints (in logs), or (vi) a dummy variable taking value one hundred if there is at least one mint on liquidity pool $i$ in asset $j$ on day $t$. $d_\emph{low-fee, ij}$ is a dummy that takes the value one for the pool with the lowest fee in pair $j$ and zero else. $d_\emph{high-fee, ij}$ is defined as $1-d_\emph{low-fee, ij}$. $\emph{GasPrice}_{jt}$ is the average of the lowest 100 bids on liquidity provision events across all pairs on day $t$, standardized to have a zero mean and unit variance. \emph{Volume} is the natural logarithm of the sum of all swap amounts on day $t$, expressed in thousands of US dollars. \emph{Total value locked} is the natural logarithm of the total value locked on Uniswap v3 pools on day $t$, expressed in millions of dollars.\emph{Volatility} is computed as the daily range between high and low prices on the most active pool for a given pair.All regressions include pair and week fixed-effects. Robust standard errors in parenthesis are clustered by week and ***, **, and * denote the statistical significance at the 1, 5, and 10\% level, respectively.  The sample period is from May 4, 2021 to July 15, 2023. 
\end{minipage}
\begin{center}
\begin{tabular}{lcccccc}
\toprule
 & \multicolumn{3}{c}{Daily mints (log US\$)} &  \multicolumn{3}{c}{$\text{Prob}\left(\text{at least one mint}\right)$} \\
 & (1) & (2) & (3) & (4) & (5) & (6) \\
\cmidrule{1-7}
$ d_\text{low-fee}$ & 0.43*** & 0.43*** & 0.43*** & 1.38* & 1.37* & 1.38* \\
 & (6.07) & (6.07) & (6.07) & (1.71) & (1.71) & (1.71) \\
Gas price $\times$ $ d_\text{low-fee}$ & -0.35*** & -0.35*** & -0.46*** & -6.02*** & -6.01*** & -4.58*** \\
 & (-8.50) & (-8.50) & (-7.14) & (-9.13) & (-9.13) & (-6.76) \\
Gas price $\times$ $ d_\text{high-fee}$ & 0.11** & 0.11** &  & -1.43** & -1.43** &  \\
 & (2.15) & (2.15) &  & (-2.57) & (-2.57) &  \\
Volume & 0.26*** & 0.26*** & 0.26*** & 0.96*** & 0.96*** & 0.96*** \\
 & (14.78) & (14.77) & (14.78) & (3.93) & (3.93) & (3.93) \\
Total value locked & -0.07 & -0.07 & -0.07 & 1.47 & 1.47 & 1.47 \\
 & (-0.78) & (-0.78) & (-0.78) & (1.01) & (1.00) & (1.01) \\
Volatility & -0.01 &  & -0.01 & 0.26 &  & 0.26 \\
 & (-0.68) &  & (-0.68) & (0.59) &  & (0.59) \\
Gas price &  &  & 0.11** &  &  & -1.43** \\
 &  &  & (2.15) &  &  & (-2.57) \\
Constant & 2.61*** & 2.61*** & 2.61*** & 51.44*** & 51.43*** & 51.44*** \\
 & (73.46) & (73.49) & (73.46) & (126.67) & (127.28) & (126.67) \\
 &  &  &  &  &  &  \\
Pair FE & Yes & Yes & Yes & Yes & Yes & Yes  \\
Week FE & Yes & Yes & Yes & Yes & Yes & Yes  \\
Observations & 40,288 & 40,288 & 40,288 & 40,288 & 40,288 & 40,288 \\
 R-squared & 0.47 & 0.47 & 0.47 & 0.28 & 0.28 & 0.28 \\ \hline
\bottomrule
\multicolumn{7}{l}{Robust t-statistics in parentheses. Standard errors are clustered at week level.} \\
\multicolumn{7}{l}{*** p$<$0.01, ** p$<$0.05, * p$<$0.1} \\
\end{tabular}
\end{center}
\end{table}

In Table \ref{tab:flows}, we shift the analysis from individual orders to aggregate daily liquidity flows to Uniswap pools. We find that higher gas prices lead to a decrease in liquidity inflows, but only on the low fee pools. A one standard deviation increase in gas prices leads to a 29.5\% drop in new liquidity deposits by volume (Model 1) and an 6.02\% drop in probability of having at least one mint (Model 4) on the low-fee pool. However, the slow-down in liquidity inflows is less evident in high fee pools. While an increase in gas prices reduce the probability of liquidity inflows by 1.43\%, it actually leads to a 11.6\% increase in the daily dollar inflow to the pool. Together with the result in Table \ref{tab:ordersize} that the size of individual mints increases with gas prices, our evidence is consistent with the model implication that higher fixed transaction costs change the composition of liquidity supply on the high-fee pool, with small $\LP$ being substituted by larger $\LP$s switching over from the low-fee pool.


\subsection{Re-balancing activity on high- and low-fee pools}
Next, we test Prediction \ref{pred:updates} on the duration of liquidity re-balancing cycles on fragmented pools. Since the descriptive statistics in Table \ref{tab:sumstat} suggest that $\LP$s manage their positions over multiple days, we cannot accurately measure liquidity cycles in a pool-day panel. Instead, we use intraday data on liquidity events (either mints or burns) to measure the duration between two consecutive opposite-sign interactions by the same Ethereum wallet with a liquidity pool: either a mint followed by a burn, or vice-versa. 

To ensure consistency with the model described in Section \ref{sec:model}, we conduct our analysis on the entire sample as well as on a sub-sample focused solely on re-balancing events where the liquidity position falls out of range (i.e., the price range set by the $\LP$ does not straddle the current price and therefore the $\LP$ does not earn fees). We further introduce wallet fixed effects to soak up variation in reaction times across traders, and winsorize the liquidity cycle duration at the 1\% level to mitigate the influence of extreme values.

Table \ref{tab:cycles} presents the results. Liquidity updates on decentralized exchanges are very infrequent, as times elapsed between consecutive interactions are measured in days or even weeks. In line with Prediction \ref{pred:updates}, we find evidence for shorter liquidity cycles on low-fee pools. The average time between consecutive mint and burn orders is 22.05\% shorter on the low-fee pool (from Model 2, the relative difference is 112.42 hours/509.19 hours).

% Liquidity cycles are in part driven by fixed Blockchain transaction costs. A one standard deviation increase in gas prices speeds up the liquidity cycle on low-fee pools by a further 15.80 hours, a result which supports the intuition behind Prediction \ref{pred:updates_gas}. When the gas price spikes, the liquidity supply on the low-fee pool decreases at a higher rate than the liquidity demand. As a result, liquidity deposits deplete faster (i.e., they move outside the fee-earning range), triggering the need for more frequent updates. At the same time, a higher gas fee speeds up the liquidity cycle on the high-fee pool as well. The result is consistent with small liquidity providers on the high-fee pool being crowded out by the high fixed costs, leading to a lower liquidity supply.

We repeat the analysis above with burn-to-mint times as the dependent variables. The burn-to-mint time measures the speed at which $\LP$s deposit liquidity at updated prices after removing (out-of-range) positions. Our findings reveal that \(\LP\)s in low-fee pools replenish liquidity 63\% faster than those in high-fee pools. This supports the notion that \(\LP\)s in low-fee environments are larger, more sophisticated market participants.  

\begin{table}[H]
\caption{Liquidity cycles on fragmented pools} \label{tab:cycles}
\begin{minipage}[t]{1\columnwidth}%
\footnotesize
			This table reports the coefficients of the following regression:
	\begin{align*}
    y_{ijtk}=\alpha + \beta_0 d_\emph{low-fee, ij} + \beta_1 \text{GasPrice}_{jt} d_\emph{low-fee, ij} + \beta_2 \text{GasPrice}_{jt} \times d_\emph{high-fee, ij} + \sum \beta_k \emph{Controls}_{ijt} + \theta_j + \varepsilon_{ijt}
    \end{align*}
	where the dependent variable $y_{ijt}$ can be (i) the mint-to-burn time, (ii) the burn-to-mint time, measured in hours, for a transaction initiated by wallet $k$ on day $t$ and pool $i$ trading asset $j$. The mint-to-burn and burn-to-mint times are computed for consecutive interactions of the same wallet address with the liquidity pool. $d_\emph{low-fee, ij}$ is a dummy that takes the value one for the pool with the lowest fee in pair $j$ and zero else. $d_\emph{high-fee, ij}$ is defined as $1-d_\emph{low-fee, ij}$. $\emph{GasPrice}_{jt}$ is the average of the lowest 100 bids on liquidity provision events across all pairs on day $t$, standardized to have a zero mean and unit variance. \emph{Volume} is the natural logarithm of the sum of all swap amounts on day $t$, expressed in thousands of US dollars. \emph{Total value locked} is the natural logarithm of the total value locked on Uniswap v3 pools on day $t$, expressed in millions of dollars. \emph{Volatility} is computed as the daily range between high and low prices on the most active pool for a given pair. \emph{Position out-of-range} is a dummy taking value one if the position being burned or minted is out of range, that is if the price range selected by the $\LP$ does not straddle the current pool price. All variables are measured as of the time of the second leg of the cycle (i.e., the burn of a mint-burn cycle). All regressions include pair, week, and trader wallet fixed-effects. Robust standard errors in parenthesis are clustered by day and ***, **, and * denote the statistical significance at the 1, 5, and 10\% level, respectively.  The sample period is from May 4, 2021 to July 15, 2023. 
\end{minipage}
\begin{center}
\begin{tabular}{lcccccc}
\toprule
 & \multicolumn{4}{c}{Mint-burn time (hours)} & \multicolumn{2}{c}{Burn-mint time (hours)}  \\
 & \multicolumn{2}{c}{Out-of-range positions} & \multicolumn{2}{c}{Full sample} \\
 & (1) & (2) & (3) & (4) & (5) & (6)  \\
\cmidrule{1-7}
$ d_\text{low-fee}$ & -110.94*** & -112.42*** & -99.74*** & -100.17*** & -157.95*** & -159.71*** \\
 & (-7.49) & (-7.69) & (-8.86) & (-8.94) & (-10.59) & (-10.81) \\
Gas price $\times$ $ d_\text{low-fee}$ & -14.27 & -6.54 & -16.65** & -15.41* & -11.29 & 2.95 \\
 & (-1.49) & (-0.68) & (-2.13) & (-1.98) & (-1.65) & (0.40) \\
Gas price $\times$ $ d_\text{high-fee}$ & -19.57** & -12.83 & -14.44** & -13.42* & -10.52* & 1.96 \\
 & (-2.34) & (-1.57) & (-2.04) & (-1.89) & (-1.69) & (0.32) \\
Volume &  & -16.71*** &  & -5.87 &  & -24.84*** \\
 &  & (-3.24) &  & (-1.15) &  & (-4.10) \\
Total value locked &  & -35.14 &  & -53.17* &  & -12.71 \\
 &  & (-1.05) &  & (-1.70) &  & (-0.52) \\
Volatility &  & -3.48** &  & -2.11*** &  & -2.99*** \\
 &  & (-2.49) &  & (-2.75) &  & (-3.36) \\
Constant & 509.19*** & 509.66*** & 497.18*** & 497.00*** & 248.00*** & 250.13*** \\
 & (61.93) & (58.34) & (91.65) & (90.60) & (29.91) & (30.27) \\
 &  &  &  &  &  &  \\
 Pair FE & Yes & Yes & Yes & Yes & Yes & Yes  \\
 Week FE & Yes & Yes & Yes & Yes & Yes & Yes  \\
 Trader wallet FE & Yes & Yes & Yes & Yes & Yes & Yes \\
Observations & 215,454 & 215,454 & 405,586 & 405,584 & 265,848 & 265,848 \\
 R-squared & 0.87 & 0.87 & 0.82 & 0.82 & 0.37 & 0.37 \\ \hline
\bottomrule
\multicolumn{7}{l}{Robust t-statistics in parentheses. Standard errors are clustered at week level.} \\
\multicolumn{7}{l}{*** p$<$0.01, ** p$<$0.05, * p$<$0.1} \\
\end{tabular}

\end{center}
\end{table}

\subsection{Adverse selection costs across low- and high-fee pools}

Finally, we test Prediction \ref{pred:as} of our model, which states that $\LP$ on the low-fee pool face higher adverse selection costs. Our main metric for informational costs is the \emph{loss-versus-rebalancing} (LVR), as defined in \citet{zhang2023amm}. The measure is equivalent to the adverse selection component of the bid-ask spread in equity markets. To calculate it, for each swap \(j\) exchanging \(\Delta x_j\) for \(\Delta y_j\) in a pool with assets \(x\) and \(y\), we use:
\begin{equation}
      \text{LVR}_{j}=d_j \times \Delta x_j (p_{\text{swap},j}-p^\prime_j),
\end{equation}
where \(d_j\) is one for a ``buy'' trade (\(\Delta x_j<0\)) and minus one for a ``sell'' trade (\(\Delta x_j>0\)). The effective swap price is \(p_{\text{swap},j}=-\frac{\Delta y_j}{\Delta x_j}\), and \(p^\prime_j\) represents a benchmark price.

% In our model, the first-order friction driving liquidity fragmentation is the fixed gas cost, which leads to heterogeneous rebalancing costs for small and large market makers respectively. Thus, pool fees in our setup simply offset liquidity providers' costs to manage their position. An alternative channel driving fragmentation might be adverse selection: The rationale is that liquidity providers gravitate towards high-fee pools as they provide superior protection against informed traders \citep[following the arguments in][]{milionis2023automated}. 

% For example, consider a token pair traded in the 5 (low) and 30 (high) basis point fee pools. If an arbitrage opportunity is worth 50 basis points and the relative gas cost is 15 basis points, arbitrageurs would optimally execute swaps on both pools since the trade value surpasses total cost, but obtain lower profits on the high-fee pool. However, if gas cost rises to 25 basis points, arbitrageurs would only trade in the low fee pool. Thus, large gas costs might make the high-fee pool more appealing to liquidity providers, as it offers added protection against adverse selection. The mechanism could provide an alternative explanation for the result in Table \ref{tab:markeshare} showing that market makers shift to high-fee pools when gas prices rise. In this section, we investigate differences in adverse selection across low- and high-fee pools, as well as the role of asymmetric information costs in driving liquidity fragmentation.

% Our main metric for informational costs is the \emph{loss-versus-rebalancing} (LVR), as defined in \citet{zhang2023amm}. The measure is equivalent to the adverse selection component of the bid-ask spread in equity markets. To calculate it, for each swap \(j\) exchanging \(\Delta x_j\) for \(\Delta y_j\) in a pool with assets \(x\) and \(y\), we use:
% \begin{equation}
%       \text{LVR}_{j}=d_j \times \Delta x_j (p_{\text{swap},j}-p^\prime_j),
% \end{equation}
% where \(d_j\) is one for a ``buy'' trade (\(\Delta x_j<0\)) and minus one for a ``sell'' trade (\(\Delta x_j>0\)). The effective swap price is \(p_{\text{swap},j}=-\frac{\Delta y_j}{\Delta x_j}\), and \(p^\prime_j\) represents a benchmark price.

We use two benchmark prices \(p^\prime_j\) in our analysis. The first, \(p^\prime_j=p_{j}^{\Delta t=0}\), is the pool's equilibrium price immediately after a swap. The resulting LVR metric captures both temporary and permanent price impact, driven by uninformed and informed trades, respectively, and represents an upper bound for $\LP$'s adverse selection cost. 

The second benchmark is the liquidity-weighted average price across Uniswap v3 pools, measured with a one-hour delay after the swap (\(p^\prime_j=p_{j}^{\Delta t=1h}\)). This approach assumes that any price deviations caused by uninformed liquidity trades are corrected by arbitrageurs within an hour, as supported by \citet{LeharParlour2021}. Thus, the LVR metric derived using this benchmark captures only the permanent price impact, a more precise measure of adverse selection cost for liquidity providers.\footnote{Our methodology is equivalent to the one in \citet{zhang2023amm} under two assumptions. First, liquidity providers can re-balance their position following each swap. Second, we assume that our two benchmarks for $p^\prime_j$, derived from decentralized exchange data, closely track the fundamental value of the token. This perspective aligns with \citet{han2022trust}, who also note that centralized exchange prices are subject to manipulative practices such as wash trading. Further, our selection of benchmarks reflects the fact that our sample includes several token pairs not traded on major centralized exchanges such as Binance.}

To compute LVR for each day $t$ and liquidity pool $i$, we aggregate the loss-versus-balancing for each swap within a day. We subsequently winsorize our measures at the 0.5\% and 99.5\% quantiles to remove extreme outliers. The resulting sum is normalized by dividing it by the total value locked (TVL) in the pool at day's end:
\begin{equation}\label{eq:LVR}
      \text{LVR}_{i,t}=\frac{\sum_j \text{LVR}_{j, i,t}}{\text{TVL}_{i,t}},  
\end{equation}
which ensures that the LVR metric is comparable across pools trading different token pairs.

We complement our analysis with the calculation of \emph{impermanent loss} (IL), an additional metric for assessing adverse selection costs. Impermanent loss is defined as the negative return from providing liquidity compared to simply holding the assets outside the exchange and marking them to market as prices change \citep[see, for example,][]{aoyagi2020,BarbonRanaldo2021}.

The key distinction between IL and loss-versus-rebalancing (LVR) measures lies in their assumptions about liquidity providers' strategies \citep{zhang2023amm}. While LVR assumes that providers actively re-balance their holdings by mirroring decentralized exchange trades on centralized exchanges at the fundamental value to hedge market risk, IL is based on a more passive approach where providers maintain their positions without active re-balancing. Loss-versus-rebalancing is a function of the entire price path, reflecting constant rebalancing by liquidity providers. In contrast, impermanent loss is determined solely by the initial and final prices of the assets.

We obtain hourly liquidity snapshots from the Uniswap V3 Subgraph to calculate impermanent loss for a theoretical symmetric liquidity position. This position is set within a price range of \(\left[\frac{1}{\alpha} p, \alpha p\right]\), centered around the current pool price \(p\), with \(\alpha\) set to 1.05. We set a one-hour horizon to measure changes in position value, aligning with the time horizon used for the LVR metric. In Appendix \ref{app:IL}, we present the exact formulas for calculating impermanent loss on Uniswap V3, based on the methodology described by \citet{Heimbach2023}. 



\begin{table}
\scriptsize
\caption{Adverse selection costs on high- and low-fee pools}\label{tab:lvr}
\begin{minipage}[t]{1\columnwidth}%
\footnotesize
This table presents regression results that analyze adverse selection costs in fragmented Uniswap v3 pools. For columns (1) through (4), the dependent variable is loss-versus-rebalancing (LVR), as defined in equation \eqref{eq:LVR}. We use the one-hour horizon benchmark (\(p_j^{\Delta t=1h}\)) in models (1) and (2) to measure permanent price impact, and the immediate, same-block price benchmark (\(p_j^{\Delta t=0}\)) in models (3) and (4) to measure total price impact. For columns (5) and (6), the dependent variable is the impermanent loss for a symmetric liquidity position at $\pm 5\%$ centered around the current pool price. The average impermanent loss is calculated for each day, based on Ethereum blocks mined within that day. The impermanent loss computation uses a one-hour liquidity provider horizon, comparing current pool prices with those one hour later. For columns (7) and (8), the dependent variables are the liquidity (TVL) and volume share of the pool, measured in percent. Finally, in columns (9) and (10) the dependent variable is the absolute deviation of the Uniswap pool price from Binance prices, sampled hourly, and measured in percent.
$d_\emph{low-fee, ij}$ is a dummy that takes the value one for the pool with the lowest fee in pair $j$ and zero else. $\emph{GasPrice}_{jt}$ is the average of the lowest 100 bids on liquidity provision events across all pairs on day $t$, standardized to have a zero mean and unit variance.  \emph{Volume} is the natural logarithm of the sum of all swap amounts on day $t$, expressed in thousands of US dollars. \emph{Total value locked} is the natural logarithm of the total value locked on Uniswap v3 pools on day $t$, expressed in millions of dollars.  When LVR is an explanatory variable, it is calculated using the one-hour ahead benchmark price.  \emph{Volatility} is computed as the daily range between high and low prices on the most active pool for a given pair. All regressions include pair and week fixed-effects. Robust standard errors in parenthesis are clustered by week, and ***, **, and * denote the statistical significance at the 1, 5, and 10\% level, respectively.  The sample period is from May 4, 2021 to July 15, 2023. 
\end{minipage}

\begin{center}
\resizebox{1\textwidth}{!}{  
\begin{tabular}{lcccccccc}
\toprule
& \multicolumn{2}{c}{LVR (1h horizon)} & \multicolumn{2}{c}{LVR (after swap)} & \multicolumn{2}{c}{Impermanent loss} & \multicolumn{2}{c}{CEX price deviation} \\
& \multicolumn{2}{c}{Permanent price impact} & \multicolumn{2}{c}{Total price impact} \\
\cmidrule{1-9} 
 & (1) & (2) & (3) & (4) & (5) & (6) & (7) & (8)  \\
\cmidrule{1-9}
$ d_\text{low-fee}$ & 6.39*** & 6.39*** & 29.78*** & 29.67*** & 1.08*** & 1.13*** & 0.06 & 0.04 \\
 & (16.57) & (17.05) & (14.86) & (14.95) & (5.72) & (6.18) & (1.51) & (1.33) \\
Gas price $\times$ $ d_\text{low-fee}$ &  & -0.75** &  & 3.51** &  & -0.01 &  & 0.08 \\
 &  & (-2.05) &  & (2.10) &  & (-0.05) &  & (1.09) \\
Gas price &  & 2.61*** &  & 6.16** &  & 3.71*** &  & -0.03 \\
 &  & (2.74) &  & (2.53) &  & (3.76) &  & (-0.28) \\
Volume &  & 3.22*** &  & 8.67*** &  & 1.81*** &  & 0.22*** \\
 &  & (8.15) &  & (6.61) &  & (6.22) &  & (4.74) \\
Total value locked &  & 0.53 &  & -2.12 &  & 1.93 &  & -0.39*** \\
 &  & (0.14) &  & (-0.34) &  & (0.74) &  & (-4.05) \\
Volatility &  & 1.85*** &  & 4.23*** &  & 6.69** &  & 1.04*** \\
 &  & (2.87) &  & (3.17) &  & (2.61) &  & (3.51) \\
Constant & 7.85*** & 7.86*** & 8.88*** & 8.97*** & 7.37*** & 7.51*** & 0.60*** & 0.67*** \\
 & (40.71) & (36.89) & (8.87) & (8.88) & (77.84) & (61.32) & (30.71) & (35.75) \\
 &  &  &  &  &  &  \\
Pair FE & Yes & Yes & Yes & Yes & Yes & Yes & Yes & Yes \\
Week FE & Yes & Yes & Yes & Yes & Yes & Yes & Yes & Yes \\
Observations & 40,302 & 40,288 & 40,302 & 40,288 & 40,250 & 40,248 & 5,207 & 5,207 \\
 R-squared & 0.14 & 0.15 & 0.09 & 0.10 & 0.09 & 0.11 & 0.10 & 0.11 \\ \hline
\bottomrule
\multicolumn{9}{l}{Robust t-statistics in parentheses. Standard errors are clustered at week level.} \\
\multicolumn{9}{l}{*** p$<$0.01, ** p$<$0.05, * p$<$0.1} \\
\end{tabular}
}
\end{center}
\end{table}

Table \ref{tab:lvr} presents our empirical results. In line with \citet{milionis2023automated}, all price impact measures --- that is, the immediate and one-hour horizon LVR and the impermanent loss --- are significantly larger in low-fee pools. This indicates that a higher liquidity fee indeed acts as barrier to arbitrageurs. Specifically, the permanent price impact, measured by the one-hour horizon LVR, is 6.39 basis points or 81\% larger in low-fee pools. The total price impact, represented by the after-swap LVR metric, is 3.5 times larger in low-fee pools. The wide gap between permanent and total price impact highlights the substantially higher volume of uninformed trading in low-fee pools. Our secondary measure of adverse selection, the impermanent loss at 5\% around the current price, is also 15\% higher on low- than high-fee pools.  

We note that an increase in gas prices leads to a 3.51 wider gap in total price impact but a 0.75 bps narrower gap in permanent price impact between the high- and low-fee pools. The result suggests that a higher gas price primarily discourages uninformed traders, rather than arbitrageurs, from trading on high-fee pools. 

% The findings in Table \ref{tab:markeshare} reveal that a higher gas cost translates to a lower liquidity and volume market share for low-fee pools. We attribute this effect primarily to the fixed costs associated with re-balancing. To examine whether the impact of gas prices on market share is driven by adverse selection, in Models (7) and (8) we add an interaction between our permanent price impact measure (one-hour horizon LVR), a dummy variable for low-fee pools, and the standardized gas price. The analysis reveals that adverse selection contributes only 8\% (computed as $\nicefrac{0.38}{4.41+0.38}$) to the effect of gas prices on liquidity market shares. Moreover, this coefficient is only marginally statistically significant. A similar pattern is observed for trading volume shares in Model 8.

Finally, in Models (7) and (8) of Table \ref{tab:lvr} we explore whether various arbitrage frictions lead to price discrepancies between high- and low-fee pools. For this purpose, we collect hourly price data from Binance for the largest four pairs by trading volume: WBTC-WETH, USDC-WETH, WETH-USDT, and USDT-USDC. We subsequently compute daily averages of hourly price deviations between centralized and decentralized exchanges. The analysis reveals that the average hourly price deviation across centralized and decentralized exchanges is 0.60\%. Notably, there is no significant difference in price deviations between low- and high-fee pools. The result suggests that arbitrage activities remain efficient despite the differences in trading costs between these pools.

% Figure environment removed

Figure \ref{fig:lvr} graphically illustrates the result, contrasting permanent price impact measures against the liquidity yield, as calculated in equation \eqref{eq:liq_yield}. Notably, before accounting for gas costs, we observe that liquidity providers in low-fee pools experience losses on average: the average daily permanent price impact in these pools is 14.21 basis points, which exceeds the fee revenue of 11.71 bps.  In contrast, liquidity providers in high-fee pools approximately break even before considering gas costs: the fee revenue amounts to 9.69 bps, which is slightly higher than the permanent price impact of 7.85 bps. One should keep in mind, however, that the magnitude of losses from adverse selection depends on the horizon at which we measure the loss-versus-rebalancing.

