\paragraph{Asset.} Consider a continuous time model of trade in a single token $\T$ with expected value $v_t>0$. Common value innovations (``news'') arrive as Poisson process with rate $\eta>0$. News is equally likely to be positive or negative. The absolute size of news is uniformly distributed between $\left[0,\sigma\right]$.\footnote{\mz{I added a distribution of news sizes so there is more arbitrage activity on low-fee pools than on high-fee pools. I.e., arbitrageurs go to the low-fee pool only if $\ell\leq\text{news size}<h$.}} Three risk-neutral trader types trade in this market: a continuum of liquidity providers ($\LP$s), a continuum of liquidity takers ($\LT$s), and arbitrageurs ($\A$). 

\paragraph{Trading environment.} Traders can interact in two liquidity pools in which token trade occurs against a num\'{e}raire asset (cash).   We assume that the terms of trade are fixed, so that all trades occur at the expected value of the token, $v$. We do this by assuming prices in both pools satisfy a linear bonding curve, so there is no price impact of trade.\footnote{In practice, most decentralized exchanges use convex bonding curves, for example constant product pricing.} So, for a pool with $T$ tokens and $N$ of the num\'{e}raire good, 
\begin{equation}\label{eq:bond_curve}
    vT + N = \text{constant}.
\end{equation}
Fees are levied on liquidity takers  as a fraction of the value of the trade and distributed pro rata to liquidity providers.  The pools have different fees.  One pool charges a low fee, and one pool charges a high fee which we denote $\ell$ and $h$ respectively.  Specifically, to purchase $\tau$ units of the token on the low fee pool, the total cost to a taker is $\tau\left(v+\ell\right)$. The $\LP$s  in the pool receive $\tau\ell$ in fees. In addition, consistent with gas costs on Ethereum,  liquidity providers incur a fixed execution cost $\Gamma>0$ to interact with the market.

\paragraph{Liquidity providers.}  Liquidity providers ($\LP$) differ in their endowments of the token. Each provider $i$ can supply at most $q_i$  of the token, where $q_i$ follows a log-uniform distribution on $\left[1,Q\right]$.  So, 
\begin{equation}\label{eq:density}
    \varphi\left(q\right)=\Big(\frac{1}{\log{Q}}\Big) \frac{1}{q} \; \; \text{ for } q\in\left[1,Q\right].
\end{equation}

\noindent The right skew of the log-uniform distribution captures the idea that there are many  low-endowment liquidity providers such as retail traders, but few high-capital $\LP$s such as sophisticated quantitative funds.   Heterogeneity in $\LP$ size is captured by $Q$, where a larger $Q$ naturally corresponds to a larger dispersion of endowments. Given the endowment distribution, collectively $\LP$s  supply at most
\begin{equation}
   S  =   \int_1^Q q \varphi\left(q\right) \diff q = \frac{Q-1}{\log Q} 
\end{equation}
tokens. 

\paragraph{Liquidity takers.} Liquidity takers ($\LT$) arrive at the market following a Poisson process with unit rate.\footnote{\mz{This is WLOG, since what matters is the ratio between news and $\LT$ arrival rates, which can be controlled through $\eta$.}} \mz{The $\LT$s arrive in pairs, that is a buyer followed by an symmetrical seller.} The assumption captures ``reversal arbitrage'' as in \citet{LeharParlour2021} without the need to explicitly model price impact: that is, the pool remains balanced upon the arrival of uninformed trade flow and no rebalancing is required.

There are two types of liquidity takers: small and large. Small $\LT$s represent a fraction $1-\lambda$ of liquidity traders. Small $\LT$s have a private value $\delta\in\left[\ell,h\right)$ such that they trade on low-fee pools but not high-fee pools. Conditional on arrival, the demand from small $\LT$s is uniformly distributed between $\left[0,\theta\right]$. Large $\LT$s represent a fraction $\lambda$ of liquidity traders and have a private value $\Delta>h$ for the asset. Conditional on arrival, a large $\LT$ demands $\Theta$ units of the token, where $\Theta>S$.


\paragraph{Arbitrageurs.} Arbitrageurs trade on common value innovations by depleting the liquidity pools as long as the size of news is larger than the pool fee. Unlike liquidity trades, arbitrageur trades are not reversed and trigger costly re-balancing from liquidity providers.

Figure \ref{fig:timing_revised} illustrates the timing of the model.

% Figure environment removed

\subsection{Equilibrium}\label{sec:liqprov}

First, consider the liquidity traders' decisions. Faced with pool sizes of $\mathcal{L}_\ell$ and $\mathcal{L}_h$ in the low and high pool respectively, they choose the pool which minimizes their trading costs. Since small traders' private value is between $\ell$ and $h$, they only trade on the low-fee pool. If their demand $\tilde{\theta}$ is lower than the available liquidity in the pool, their trade is executed in full. Otherwise, they exhaust the low fee pool and the trade size is rationed. The fraction of the low-fee pool that is consumed after the arrival of a small $\LT$ is:
\begin{equation}
    \mathbb{E}\left[\min\left\{\frac{\tilde{\theta}}{\LL},1\right\}\right]=1-\frac{\LL}{2\theta}.
\end{equation}
By assumption the large liquidity taker wants to trade more than the posted liquidity and exhausts both pools. \mz{Once a liquidity trader executes the trade, a reversal arbitrage trade of same size and opposite direction restores the original state of the pool.}

Second, consider the trading decisions of arbitrageurs. If the size of news $\tilde{\sigma}\geq h$, then arbitrageurs obtain positive profit on both pools since $\tilde{\sigma}-h>0$ and $\tilde{\sigma}-\ell>0$. They exhaust liquidity on both pools in the direction of the trade. Since the trade is motivated by information, there is no reversal arbitrage trade and liquidity providers need to re-balance at cost $\Gamma$. However, if $\tilde{\sigma}\in\left[\ell,h\right)$, then arbitrageurs trade only on the low-fee pool. Finally, if $\tilde{\sigma}<\ell$, then arbitrageurs do not trade upon the realization of news.

Let $d_k, k=\ell,h$ denote the duration of a re-balancing cycle on the low and high pool respectively. Then, the expected duration of a cycle on the low fee pool is 
\begin{align}
    d_\ell &= \eta \left(1-\frac{\ell}{\sigma}\right),
    \end{align}
\noindent while the expected duration on a high fee pool is $d_h=\eta \left(1-\frac{h}{\sigma}\right).$  Thus,


\begin{lem}\label{lem:duration}
 \setlength{\parskip}{0ex}
The expected duration of a liquidity cycle is shorter in the low fee pool than the high fee pool. Or, $d_\ell<d_h$.
\end{lem}