\section{Experimental Evaluations}\label{sec:experiment}

\textbf{Implementation.}
We implement \puma\ on top of SecretFlow-SPU~\citep{spu} in \textrm{C++} and Python. %SecretFlow compiles a high-level Flax code to secure computation protocols, which are then executed by our designed cryptographic backends, and 
We encode the data in a fixed-point form under ring $\mathbb{Z}_{2^{64}}$ with $18$-bit fractional part. 
Our experiments are run on 3 Alibaba Cloud ecs.g7.8xlarge servers with 32 vCPU and 128GB RAM each. The CPU model is Intel Xeon(Ice Lake) Platinum 8369B CPU @ 2.70GHz. We evaluate \puma\ on Ubuntu 20.04.6 LTS with Linux kernel 5.4.0-144-generic. Our bandwidth is about 5Gbps and round trip time is about 1ms. %\cheng{Describe fixed point parameters: scale, share bits.}

\textbf{Models \& Datasets.}
We evaluate \puma\ on seven NLP models: Bert-Base, Roberta-Base, and Bert-Large~\citep{bert}; GPT2-Base, GPT2-Medium, and GPT2-Large~\citep{gpt}; and LLaMA-7B~\citep{touvron2023llama}. We measure the Bert performance for three NLP tasks over the datasets of Corpus of Linguistic Acceptability (CoLA), Recognizing Textual Entailment (RTE), Stanford Question Answering Dataset (QNLI) from GLUE benchmarks~\citep{wang2018glue}, and GPT2 performance on Wikitext-103 V1~\citep{merity2016pointer}.

\textbf{Baseline.}
We compare \puma\ to the most similar prior work \mpcformer~\citep{li2023mpcformer}. But for fair comparison, we have the following considerations:
\romannumeral1) As \mpcformer\ neither supports loading pretrained transformer models nor implements LayerNorm faithfully\footnote{ As \mpcformer~does not support loading pre-trained Transformer models, we did an experiment in plaintext Bert-Base that replaced LayerNorm with BatchNorm  as \mpcformer~did. This resulted in a significant drop in the MCC score for CoLA task from $0.616$ to $-0.020$. On the contrary, \puma~achieves an MCC score of $0.613$. }, we cannot achieve meaningful secure inference results using their framework.
Therefore, we compare our performance to that of plaintext (floating-point) to show our precision guarantee.
\romannumeral2) \mpcformer\ with \textit{Quad} approximations requires retraining the  modified models. As \puma\ does not require retraining, we compare our cost to that of \mpcformer\ without \textit{Quad} approximations. Also, we re-run \mpcformer~in our environment.



\subsection{Precision}\label{sec:accuracy}
\iffalse
% Figure environment removed
\fi 


\begin{table}
\centering
\caption{Performance on GLUE benchmark of Bert-Base, Roberta-Base, and Bert-Large on CoLA, RTE, and QNLI, Matthews correlation is reported for CoLA. Accuracy is reported for other datasets.}\label{table:bertacc}
\begin{tabular}{c|ccc|ccc|ccc}
\hline \hline
 Model & \multicolumn{3}{c|}{Bert-Base} & \multicolumn{3}{c|}{Roberta-Base} & \multicolumn{3}{c}{Bert-Large} \\ \hline
 TASK & CoLA & RTE & QNLI & CoLA & RTE & QNLI & CoLA & RTE & QNLI \\ \hline
CPU & $0.616$     & $0.700$      & $0.916$     & $0.629$ & $0.805$ & $0.920$  & $0.686$   & $0.755$ & $0.922$ \\
\puma   & $0.613$     & $0.700$     & $0.916$     & $0.618$ & $0.805$ & $0.918$ & $0.690$ & $0.747$ & $0.918$ \\ \hline \hline
\end{tabular}
\end{table}

\begin{table}[]
    \centering
    \caption{Perplexity of GPT2-Base, GPT2-Medium, and GPT2-Large on Wikitext-103 V1.}
    \label{tab:gpot2ppl}
    \begin{tabular}{c|c|c|c}
    \hline \hline
      Model & GPT2-Base & GPT2-Medium & GPT2-Large \\ \hline
      CPU & $16.284$ & $12.536$ & $10.142$ \\
      \puma & $16.284$ & $12.540$ & $10.161$ \\
      \hline \hline
    \end{tabular}
    
\end{table}

We compare our secure model 
inference performance to that of plaintext (floating-point) in Table~\ref{table:bertacc} and~\ref{tab:gpot2ppl} to show our precision guarantee.

In Table~\ref{table:bertacc}, we show the Matthews correlation/accuracy of plaintext and \puma\ on the Bert-Base, Roberta-base, and Bert-Large. We observe that the accuracy achieved by \puma~ matches the accuracy of the plaintext Flax code. Specifically, the accuracy difference does
not exceed $0.011$ over all datasets. 
Moreover, in Table~\ref{tab:gpot2ppl}, we also compare our perplexity on dataset Wikitext-103 V1 with the plaintext baseline on GPT2 models. The results are similar and the perplexity differences do not exceed $0.02$ over all models.

The above accuracy and perplexity advantages experimentally validate that our protocols are numerically precise. 


\begin{table}[h]
    \centering
    \caption{Costs of Bert-Base, Roberta-Base, and Bert-Large for one sentence of length $128$. Time is in seconds and Communication (Comm. for short) is in GB, which is the same for the following tables.}\label{tab:costbert}
    \begin{tabular}{c|cc|cc|cc}
    \hline \hline
       Model & \multicolumn{2}{c|}{Bert-Base} & \multicolumn{2}{c|}{Roberta-Base} & \multicolumn{2}{c}{Bert-Large} \\ \hline
       Costs & Time & Comm. & Time & Comm. & Time & Comm. \\ \hline
       \mpcformer & $55.320$ & $12.089$ & $57.256$ & $12.373$ & $141.222$ & $32.577$ \\
       \puma & $33.913$ & $10.773$ & $41.641$ & $11.463$ & $73.720$ & $27.246$ \\
       \cellcolor{mygray} Improv. & \cellcolor{mygray} $1.631\times$ & \cellcolor{mygray} $1.122\times$ & \cellcolor{mygray} $1.375\times$ & \cellcolor{mygray} $1.079\times$ & \cellcolor{mygray} $1.916\times$ & \cellcolor{mygray} $1.195\times$ \\
       \hline \hline
    \end{tabular}
    \vspace{-0.2cm}
\end{table}

\begin{table}[]
    \centering
    \caption{Costs of GPT2-Base, GPT2-Medium, and GPT2-Large. The input sentence is of length $32$, all of the costs are for generating $1$ token.}\label{tab:costgpt2}
    \begin{tabular}{c|cc|cc|cc}
    \hline \hline
       Model & \multicolumn{2}{c|}{GPT2-Base} & \multicolumn{2}{c|}{GPT2-Medium} & \multicolumn{2}{c}{GPT2-Large} \\ \hline
       Costs & Time & Comm. & Time & Comm. & Time & Comm. \\ \hline
       \mpcformer & $34.889$ & $4.999$ & $73.078$ & $11.766$ & $129.095$ & $22.522$  \\
       \puma & $15.506$ & $3.774$ & $30.272$ & $7.059$ & $54.154$ & $11.952$ \\
       \cellcolor{mygray} Improv. & \cellcolor{mygray} $2.250\times$ & \cellcolor{mygray} $1.325\times$ & \cellcolor{mygray} $2.414\times$ & \cellcolor{mygray} $1.667\times$ & \cellcolor{mygray} $2.383\times$ & \cellcolor{mygray} $1.884\times$ \\
       \hline \hline
    \end{tabular}
    \vspace{-0.2cm}
\end{table}

\subsection{Inference Costs}\label{sec:efficiency}


We compare \puma's inference cost to that of \mpcformer. 
%We evaluate three Bert models (Bert-Base, Roberta-Base, and Bert-Large) and three GPT2 models (GPT2-Base, GPT2-Medium, and GPT2-Large).
The costs are for processing one input sentence: \romannumeral1) For Bert models the input sentence is of length $128$. \romannumeral2) For GPT2 models the input  length is 32 and generate $1$ new word. 

On the 3 Bert models in Table~\ref{tab:costbert}, \puma\ is  $1.375\sim 1.916\times$ faster than  \mpcformer, and is $1.079\sim 1.195\times$ more communication-efficient. For the GPT2 models in Table~\ref{tab:costgpt2}, \puma\ is $2.250\sim 2.414\times$ faster than \mpcformer, and is $1.325\sim 1.884\times$ more communication-efficient. 
    
We observe that \puma's improvements increase as the model size grows, particularly for the GPT2 models. This trend is because our specialized optimizations are more effective when processing large-scale evaluations.



\subsection{Scalability}\label{sec:scala}

In this subsection, we measure the costs of evaluating \puma\ on Bert-Base and GPT2-Base models for batched inputs, varying-length inputs, and varying-length outputs (only for GPT2-Base). We also compare our costs to those of \mpcformer~to demonstrate our improvements.

\iffalse

\begin{table}[]
    \centering
    \caption{Costs of Bert-Base and GPT2-Base for a batch of $\{2, 4, 8 ,16\}$ sentences. The input lengths for Bert-Base and GPT2-Base are respectively set as $128$ and $32$, and the costs of GPT2 are for generating $1$ token.}\label{tab:costbertbatch}
    \begin{tabular}{cc|cc|cc|cc|cc}
    \hline \hline
       \multicolumn{2}{c|}{\#Batch} & \multicolumn{2}{c|}{2} & \multicolumn{2}{c|}{4} & \multicolumn{2}{c|}{8} & \multicolumn{2}{c}{16}  \\ \hline
       \multicolumn{2}{c|}{Costs} & Time & Comm. & Time & Comm. & Time & Comm. & Time & Comm. \\ \hline
       \multirow{3}{*}{Bert}& \mpcformer & $90.915$ & $22.579$ & $166.582$ & $43.560$ & $320.621$ & $85.521$ & $633.474$ & $169.433$ \\
       & \puma & $56.756$ & $21.306$ & $104.629$ & $43.089$ & $201.899$ & $85.758$ & $403.510$ & $172.371$ \\
       &\cellcolor{mygray} Improv. & \cellcolor{mygray} $1.602\times$ &\cellcolor{mygray} $1.060\times$ &\cellcolor{mygray} $1.592\times$ &\cellcolor{mygray} $1.011\times$ & \cellcolor{mygray} $1.588\times$ &\cellcolor{mygray} $0.997\times$ & \cellcolor{mygray} $1.570\times$ & \cellcolor{mygray} $0.983\times$ \\
       \hline
       \multirow{3}{*}{GPT2}& \mpcformer &  $42.205$ & $7.005$ & $58.495$ & $11.018$ & $90.260$ & $19.044$ & $161.637$ & $35.096$ \\
       & \puma & $24.397$ & $7.530$ & $43.098$ & $15.114$ & $79.605$ & $30.001$ & $153.852$ & $59.645$\\
       &\cellcolor{mygray} Improv. & \cellcolor{mygray} $1.730\times$ & \cellcolor{mygray} $0.930\times$ & \cellcolor{mygray} $1.357\times$ & \cellcolor{mygray} $0.729\times$ & \cellcolor{mygray} $1.134\times$ & \cellcolor{mygray} $0.633\times$ & \cellcolor{mygray} $1.051\times$ & \cellcolor{mygray} $0.588\times$\\
       \hline \hline
    \end{tabular}
\end{table}
\textbf{Batch Inputs Evaluation.}
Table~\ref{tab:costbertbatch} presents our costs on batched inputs. For Bert-Base, \puma\ is $1.570\sim 1.602\times$ faster. For GPT2-Base, our improvements in runtime are in the range of $1.051\sim 1.730\times$, respectively. 
And our communication costs are comparable to \mpcformer.
Unlike the observations in Section~\ref{sec:efficiency}, our efficiency gains decrease with increasing batch sizes, and \puma~requires more communication when processing large batch of inputs. This phenomenon is attributed to the interesting fact: To directly support pre-trained plaintext models, \puma\ strictly follows the plaintext model format that only accept token ids as input, so \puma\ has to compute the one-hot vectors from token ids in an MPC way. On the other hand, \mpcformer\ uses modified models that accept one-hot vectors as input, so the one-hot function could be computed at the client side in plaintext. Nevertheless, \puma\ remains faster than \mpcformer.
\fi 


\begin{table}[]
    \centering
    \caption{Costs of Bert-Base and GPT2-Base for different input length (denoted as \#Input). The input lengths for Bert-Base and GPT2-Base are respective $\{64, 128, 256, 512\}$ and $\{16, 32, 64, 128\}$. GPT2-Base generates $1$ token.}\label{tab:costbertinput}
    \begin{tabular}{cc|cc|cc|cc|cc}
    \hline \hline
       \multicolumn{2}{c|}{\#Input} & \multicolumn{2}{c|}{$64 / 16$} & \multicolumn{2}{c|}{$128 / 32$} & \multicolumn{2}{c|}{$256 / 64$} & \multicolumn{2}{c}{$512 / 128$}  \\ \hline
       \multicolumn{2}{c|}{Costs} & Time & Comm. & Time & Comm. & Time & Comm. & Time & Comm. \\ \hline
       \multirow{3}{*}{Bert}& \mpcformer & $46.428$ & $4.750$ & $85.887$ & $9.673$ & $196.372$ & $23.443$ & $582.787$ & $68.069$ \\
       & \puma & $24.345$ & $1.627$ & $42.525$ & $3.591$ & $87.561$ & $8.668$ & $212.600$ & $23.439$\\
       & \cellcolor{mygray} Improv. & \cellcolor{mygray} $1.907\times$ & \cellcolor{mygray} $2.919\times$ & \cellcolor{mygray} $2.020\times$ & \cellcolor{mygray} $2.694\times$ & \cellcolor{mygray} $2.243\times$ & \cellcolor{mygray} $2.705\times$ & \cellcolor{mygray} $2.741\times$ & \cellcolor{mygray} $2.904$ \\
       \hline
       \multirow{3}{*}{GPT2}& \mpcformer & $34.522$ & $3.767$ & $42.615$ & $4.516$ & $60.451$ & $6.281$ & $105.028$ & $11.225$  \\
       & \puma & $20.692$ & $0.625$ & $29.248$ & $1.258$ & $40.968$ & $2.607$ & $74.529$ & $5.611$\\
       &\cellcolor{mygray} Improv. & \cellcolor{mygray} $1.668\times$ & \cellcolor{mygray} $6.027\times$ & \cellcolor{mygray} $1.457\times$ & \cellcolor{mygray} $3.590\times$ & \cellcolor{mygray} $1.476\times$ & \cellcolor{mygray} $2.409\times$ & \cellcolor{mygray} $1.409\times$ & \cellcolor{mygray} $2.001\times$\\
       \hline \hline
    \end{tabular}
\end{table}

\textbf{Input Length Evaluation.}
Table~\ref{tab:costbertinput} shows our costs on varying-length inputs, we evaluate Bert-Base on inputs of length $\{64, 128, 256\}$, and GPT2-Base on inputs of length $\{16, 32, 64\}$.
For Bert-Base, \puma\ is $1.631\sim 1.837\times$ faster, and for GPT2-Base, \puma\ is $1.744\sim 2.686\times$ faster. 

%\begin{table}[]
    \centering
    \caption{Costs of GPT2-small for generating different output tokens (denoted as \#Output), the input length is set as $32$.}\label{tab:costgpt2tokens}
    \begin{tabular}{c|cc|cc|cc|cc}
    \hline \hline
       \#Output & \multicolumn{2}{c|}{2} & \multicolumn{2}{c|}{4} & \multicolumn{2}{c|}{8} & \multicolumn{2}{c}{16}  \\ \hline
       Costs & Time & Comm. & Time & Comm. & Time & Comm. & Time & Comm. \\ \hline
       \mpcformer & $72.833$ & $7.676$ & $132.644$ & $13.998$ & $252.796$ & $26.648$ & $494.509$ & $51.972$ \\
       \puma & $53.191$ & $2.549$ & $111.457$ & $5.167$ & $215.352$ & $11.115$ & $457.994$ & $24.917$ \\
       Improv. & $1.369\times$ & $3.011\times$ & $1.190\times$ & $2.709\times$ & $1.174\times$ & $2.397\times$ & $1.080\times$ & $2.086\times$ \\
       \hline \hline
    \end{tabular}
\end{table}



\textbf{Output Length Evaluation.}
Fig~\ref{fig:gptwoutcosts} presents our costs on varying-length outputs for GPT2-Base. Our improvements against \mpcformer\  range from $1.279\sim 2.700\times$.

We observe in Table~\ref{tab:costbertinput} and Fig~\ref{fig:gptwoutcosts} that for GPT-2, our efficiency gains decrease  with more input/output tokens. This is because \puma\ introduces extra one-hot embedding costs (as described in~\ref{sec:embed}). We should emphasize
again that  \puma\ is compatible with plaintext models, and could achieve a similar accuracy as plaintext models while \mpcformer\ could not.

 % Figure environment removed



\subsection{Evaluating LLaMA-7B in Five Minutes.}\label{sec:llama}
\begin{table}[]
    \centering
    \caption{Costs of the secure inference of LLaMA-7B, \#Input denotes the length of input sentence and \#Output denotes the number of generated tokens.}\label{tab:llama7b}
    \begin{tabular}{c|cc|cc|cc}
    \hline \hline
       (\#Input, \#Output) & \multicolumn{2}{c|}{$(4,1)$} & \multicolumn{2}{c|}{$(8,1)$} & \multicolumn{2}{c}{$(8,2)$} \\ \hline
       Costs & Time & Comm. & Time & Comm. & Time & Comm. \\ \hline
       \puma & $122.004$ & $0.907$ & $200.473$ & $1.794$ & $364.527$ & $3.857$ \\
       \hline \hline
    \end{tabular}
    \vspace{-0.2cm}
\end{table}

Our protocols are already complete for evaluating any Transformer-based models including LLaMA-7B. Unfortunately, existing serialization libraries such as Protobuf~\citep{protobuf} and FlatBuffers~\citep{van2014flatbuffers} only support data trunks with size up to 2GB, which is not sufficient for large MPC tasks. To address this problem, we propose an optimization to SecretFlow-SPU. Concretely, the system could automatically divide and serialize overly large secret-shared structures into smaller chunks when communicating or performing I/O operations.

We evaluated the large language model LLaMA-7B using \puma\ under 3 Alibaba Cloud
ecs.r7.32xlarge servers, each has 128 threads and 1TB RAM, with 20GB bandwidth, 0.1ms round-trip-time. 
As shown in Table~\ref{tab:llama7b}, \puma\ can support secure inference of LLaMA-7B with reasonable costs. For example, given an input sentence of 8 tokens, \puma\ can output one token in around $200$ seconds with communication costs of $1.794$ GB. To our knowledge, this is the first time that LLaMA-7B has been evaluated using MPC. Moreover, \puma\ can generate the same tokens exactly as plaintext LLaMA-7B, see Appendix~for an example.


%Llama-7B, LAN=(20GB, 0.06ms), 128 threads, input length=8, output=1 token, costs: 346.126s, 2002213760 bytes