\begin{appendix}
\section{Used software and packages}

\begin{minipage}{\textwidth}
In this work, we used \texttt{PyTorch} \citep{pytorch} as the fundamental deep learning framework. It was chosen because of its flexibility and the ability to directly develop algorithms in the programming language \texttt{Python} \citep{python}. In addition to \texttt{PyTorch}, we used the deep learning library \texttt{fast.ai} \citep{fastai} which supplies high-level components to build customized deep learning algorithms in a quick and efficient way.
For simulations and data analysis, the Python packages \texttt{NumPy} \citep{numpy}, \texttt{Astropy} \citep{astropy1, astropy2}, \texttt{Cartopy} \citep{cartopy}, \texttt{scikit-image} \citep{scikit-image}, and \texttt{Pandas} \citep{pandas} were used.
The illustration of the results was done using the plotting package \texttt{Matplotlib} \citep{matplotlib}.
A full list of the used packages and our developed open-source \texttt{radionets} framework can be found on github\footnote{\url{https://github.com/radionets-project/radionets}}.
More information about our developed open-source simulation framework \texttt{pyvisgen} can be found in the git repository\footnote{\url{https://github.com/radionets-project/pyvisgen}}.
\end{minipage}

%%%%%%

% Figure environment removed

%%%%%%

% Figure environment removed

%%%%%
\FloatBarrier

\begin{table}
    \centering
\begin{minipage}{\textwidth}
\section{Computer setup}
    \centering
    \caption{Computer specifications for the setup used in the training process}
    \begin{tabular}{c|c|c}
    \toprule
    Part & Specification & Value \\
    \midrule
    GPU &  Nvidia DGX A100 & \SI{8}{\gigabyte}\\
    CPU & AMD Rome & 64 Cores @ \SI{3.35}{\giga\hertz}\\
    Hard drive & SSD & \SI{512}{\gigabyte} \\
    \bottomrule
    \end{tabular}
    \label{tab:computer-specifications}
\end{minipage}
\end{table}

%%%%%
% \FloatBarrier

\begin{table}[!htbp]
    \centering
\begin{minipage}{\textwidth}
\section{Antenna positions VLA B-configuration}
    \centering
    \caption{Overview of the antenna positions of the VLA used for the RIME in the Earth-centered coordinate system. The given positions correspond to the B-configuration of the VLA used for the FIRST observations.}
    \begin{tabular}{
    l
    S[table-format=-4.2]
    S[table-format=-4.2]
    S[table-format=-4.2]
    }
    \toprule
    Station Name &  {X / m} &  {Y / m} &  {Z / m} \\
    \midrule
    W32&1640.03 &   -4329.93    &   -2416.71\\
    N24&-1660.49&   -259.40     &   2454.42 \\
    W20&733.35  &   -1932.98    &   -1078.11\\
    E24&765.21  &   2889.45     &   -1108.88\\
    N32&-2629.09&   -410.65     &   3885.60\\
    E36&1534.56 &   5793.91     &   -2223.48\\
    N16&-801.40 &   -124.97     &   1182.12\\
    W8&152.76   &   -401.27     &   -223.40\\
    W28&1316.45 &   -3443.31    &   -1913.53\\
    W12&306.17  &   -804.58     &   -448.08\\
    W36&2000.07 &   -5299.80    &   -2962.89\\
    N28&-2091.44&   -326.59     &   3089.41\\
    E32&1253.25 &   4733.62     &   -1816.91\\
    N36&-3217.58&   -502.67     &   4756.12\\
    E28&998.67  &   3764.31     &   -1443.47\\
    N4&-74.82   &   -11.76      &   111.63\\
    E4&35.60    &   133.65      &   -51.10\\
    W4&46.92    &   -122.02     &   -67.61\\
    W16&499.84  &   -1318.00    &   -735.21\\
    N8&-243.60  &   -38.05      &   360.04\\
    E16&376.99  &   1440.99     &   -556.13\\
    E8&114.43   &   438.69      &   -169.49\\
    E20&560.10  &   2113.27     &   -810.70\\
    E12&229.47  &   879.60      &   -339.87\\
    N20&-1174.34&   -183.30     &   1734.24\\
    N12&-489.31 &   -76.31      &   721.52\\
    W24&1005.43 &   -2642.99    &   -1472.20\\
    \bottomrule
    \end{tabular}
    \label{tab:antennas-vla}
\end{minipage}
\end{table}

\FloatBarrier


% Figure environment removed

% Figure environment removed

% \FloatBarrier

\begin{table*}[!htbp]
    \centering
    \begin{minipage}{\textwidth}
    \section{Summary of evaluation methods results}
    \centering
    \caption{Results of the evaluation methods for data sets with different SNRs reconstructed with \texttt{radionets}}
    \begin{tabular}{c|c|c|c|c}
    \toprule
     SNR &  Area & Peak fluxes & Integrated fluxes & MS-SSIM \\
     \midrule
     0-10 & \num{0.945(146)} & \num{1.002(106)} & \num{0.936(107)} & \num{0.998(5)} \\
     10-20 & \num{0.950(95)} & \num{0.994(68)} & \num{0.958(57)} & \num{0.999(2)} \\
     20-30 & \num{0.974(85)} & \num{0.997(65)} & \num{0.978(40)} & \num{0.999(1)} \\
     30-40 & \num{0.980(72)} & \num{1.000(66)} & \num{0.988(29)} & \num{1.000(1)} \\
     40-50 & \num{0.981(70)} & \num{1.001(69)} & \num{0.990(30)} & \num{1.000(1)} \\
     50-60 & \num{0.987(75)} & \num{0.997(72)} & \num{0.990(28)} & \num{1.000(0)} \\
     $>60$ & \num{0.988(54)} & \num{1.000(68)} & \num{0.993(26)} & \num{1.000(0)} \\
     \bottomrule
    \end{tabular}
    \label{tab:snr}
    \end{minipage}
\end{table*}

\begin{table*}[!htbp]
    \centering
    \caption{Results of the evaluation methods for data sets with different SNRs reconstructed with \texttt{WSCLEAN}}
    \begin{tabular}{c|c|c|c|c}
    \toprule
     SNR &  Area & Peak fluxes & Integrated fluxes & MS-SSIM \\
     \midrule
     0-10 & \num{1.378(413)} & \num{0.479(68)} & \num{0.907(105)} & \num{0.822(84)} \\
     10-20 & \num{1.242(345)} & \num{0.468(57)} & \num{0.952(82)} & \num{0.881(54)} \\
     20-30 & \num{1.229(329)} & \num{0.481(58)} & \num{0.981(57)} & \num{0.919(41)} \\
     30-40 & \num{1.189(285)} & \num{0.488(58)} & \num{0.988(45)} & \num{0.938(33)} \\
     40-50 & \num{1.169(267)} & \num{0.499(57)} & \num{0.990(38)} & \num{0.949(30)} \\
     50-60 & \num{1.169(281)} & \num{0.493(55)} & \num{0.992(35)} & \num{0.955(27)} \\
     $>60$ & \num{1.063(196)} & \num{0.483(58)} & \num{1.008(38)} & \num{0.964(21)} \\
     \bottomrule
    \end{tabular}
    \label{tab:snr-wsclean}
\end{table*}

%%%%%
\FloatBarrier

% Figure environment removed

% Figure environment removed

\end{appendix}

