\documentclass[twocolumn]{aastex631}

\newcommand{\vdag}{(v)^\dagger}
\newcommand\aastex{AAS\TeX}
\newcommand\latex{La\TeX}
\newcommand{\ac}[1]{\textcolor{brown}{#1}}
\newcommand{\yy}[1]{\textcolor{magenta}{#1}}
\usepackage{amsmath}
\usepackage[utf8]{inputenc}

\begin{document}

\title{The impact of electron anisotropy on the polarization of the X-ray emission from black hole accretion disks and implications for the black hole X-ray binary 4U\,1630--47}

\author[0000-0002-1084-6507]{Henric Krawczynski}
\affiliation{Physics Department, McDonnell Center for the Space Sciences, and Center for Quantum Leaps, Washington University in St. Louis, St. Louis, MO 63130, USA}
\email{Corresponding authors:
H. Krawczynski (krawcz@wustl.edu), Yajie Yuan
(yajiey@wustl.edu), Alexander Y. Chen (cyuran@wustl.edu).}
\author[0000-0002-0108-4774]{Yajie Yuan}
\affiliation{Physics Department and McDonnell Center for the Space Sciences, Washington University in St. Louis, St. Louis, MO 63130, USA}
\author[0000-0002-4738-1168]{Alexander Y. Chen}
\affiliation{Physics Department and McDonnell Center for the Space Sciences, Washington University in St. Louis, St. Louis, MO 63130, USA}
\author[0000-0001-5256-0278]{Nicole Rodriguez Cavero}
\affiliation{Physics Department, McDonnell Center for the Space Sciences, and Center for Quantum Leaps, Washington University in St. Louis, St. Louis, MO 63130, USA}
\author[0000-0002-9705-7948]{Kun Hu}
\affiliation{Physics Department, McDonnell Center for the Space Sciences, and Center for Quantum Leaps, Washington University in St. Louis, St. Louis, MO 63130, USA}
\author[0000-0002-5250-2710]{Ephraim Gau}
\affiliation{Physics Department, McDonnell Center for the Space Sciences, and Center for Quantum Leaps, Washington University in St. Louis, St. Louis, MO 63130, USA}
\author[0000-0002-5872-6061]{James F. Steiner}
\affiliation{Harvard-Smithsonian Center for Astrophysics, 60 Garden Street, Cambridge, MA 02138, USA}
\author[0000-0003-0079-1239]{Michal Dovčiak}
\affiliation{Astronomical Institute of the Czech Academy of Sciences, Boční II 1401/1, 
14100 Praha 4, Czech Republic}

\begin{abstract}
The {\it Imaging X-ray Polarimetry Explorer (IXPE)}  observations of the
X-ray binary 4U\,1630--47 in the high soft state revealed linear polarization degrees (PDs) rising from 6\% at 2\,keV to 10\% at 8\,keV. Explaining the results in the framework of the standard 
optically thick, geometrically thin accretion disk scenario requires careful 
fine-tuning of the relevant model parameters.
We argue here that the emission of polarized Bremsstrahlung by anisotropic electrons 
in the accretion disk atmosphere can account for the overall high PDs and the increase of the PDs with energy. We discuss 
plasma and accretion effects that can 
generate electron anisotropies at a level 
required by the 4U\,1630--47 results.
We conclude by emphasizing that 
X-ray polarimetry affords us the 
opportunity to  obtain information 
about the magnetization  of the accretion 
disk atmosphere.
%TC:endignore
\end{abstract}

%% Keywords should appear after the \end{abstract} command.
%% The AAS Journals now uses Unified Astronomy Thesaurus concepts:
%% https://astrothesaurus.org
%% You will be asked to selected these concepts during the submission process
%% but this old "keyword" functionality is maintained in case authors want
%% to include these concepts in their preprints.
\keywords{Polarimetry (1278) --- X-ray astronomy (1810) --- Stellar mass black holes (1611)}

\section{Introduction}
\label{s:intro}
The \textit{Imaging X-ray Polarimetry Explorer (IXPE)} \citep[\textit{IXPE},][]{ixpe} launched on Dec. 9, 2021
measured or constrained the polarization 
of the X-rays from several Black Hole X-ray 
Binaries (BHXRBs), including Cyg X-1 \citep{2022Sci...378..650K,2023ATel16084....1D}, LMC X-1 \citep{2023arXiv230312034P}, and
4U\,1630--47 \citep{2023arXiv230412752R,2023arXiv230510630R}. 
In this paper, we will focus on the latter source, 4U 1630--47, a low-mass X-ray binary (LMXB) system showing recurrent outbursts every 2--3 years \citep{1998ApJ...494..753K, 2015MNRAS.450.3840C}.
{\it IXPE} observed 4U 1630--47 in the 
High Soft State (HSS) \citep{2023arXiv230412752R}
and in the Steep Power Law (SPL) state \citep{2023arXiv230510630R}.
The HSS observations are particularly interesting, as the HSS emission is 
dominated by the thermal multi-temperature emission from the optically thick, 
geometrically thin accretion disk \citep{1973A&A....24..337S}.
X-ray polarimetric observations of BHXRBs 
thus give us a new way of testing the 
thin disk model which has been extremely 
successful in explaining the spectral observations of BHXRBs.
The {\it IXPE} observations of
4U 1630--47 in the HSS revealed PDs 
increasing from $\sim$6\% at 
2~keV to $\sim$10\% at 8~keV
in the HSS \citep{2023arXiv230412752R}.
Intriguingly, similar PDs were found 
in the SPL rising from $\sim5\%$ at 2~keV 
to $\sim$8\% at 8~keV.
Although the PDs were overall higher in the HSS than in the SPL state, the PDs varied
as much during the HSS observations as their average between the HSS and the SPL observations. For both observations, the polarization angles (PAs) did not exhibit statistically significant variations with energy or in time.

Remarkably, the PDs measured in the HSS exceed those 
predicted by  current state-of-the-art models \citep{2009ApJ...691..847L,2009ApJ...701.1175S,2022ApJ...934....4K,2019ApJ...875..148Z,2020MNRAS.493.4960T,2023arXiv230412752R}.
%
According to Chandrasekhar's classical 
treatment of radiation transport in 
pure electron scattering atmospheres, 
scattering can create PDs reaching 11.71\% 
for an observer at $90^{\circ}$ inclination 
from the surface normal of the atmosphere \citep{Chandra:60}.
For the BHXRB 4U\,1630--47, the observer is 
believed to view the accretion disk 
at $\sim$65$^{\circ}$ from the accretion 
disk normal and the axis of the binary. 
This inclination angle of the binary is 
consistent with the absence of eclipses 
and the presence of dips in the X-ray light curves \citep{1998ApJ...494..747T,1998ApJ...494..753K}.
For this inclination, Chandrasekhar's theory
predicts a much lower PD of 2.8\%.
\citet{2023arXiv230412752R} show that General Relativity (GR) effects 
close to the black hole (i.e., the parallel transport of the polarization direction of 
X-rays propagating through the black hole's 
curved spacetime) and emission 
returning to the accretion disk and reflecting off the accretion disk reduce the predicted PDs by---depending on black hole spin
and inclination---as much as 50\%, 
rendering the maximal locally emitted PDs of 11.71\% insufficient to explain the detected
6-10\% PDs. Although the absorption of X-rays in the accretion 
disk can lead to slightly higher PDs and to a positive PD-energy
correlation, the predicted PDs are still not high enough to explain the observations.
\citet{2023arXiv230412752R} manage to fit the 
{\it IXPE} results by adopting a model with a 
low black hole spin value, extremely high locally
emitted PDs from models with strong photospheric absorption, and the ad-hoc assumption of the emitting plasma outflowing with a velocity of 
50\% of the 
speed of light away from the accretion disk. 
The relativistic motion of the emitting plasma
increases the PDs observed at moderate inclinations  as X-rays 
emitted by the plasma at high inclinations reach  the observer 
at lower inclinations owing to effect of relativistic aberration.
The same authors discuss two other scenarios but find that they cannot explain the {\it IXPE} results: scattering off a wind tends to produce 
a constant polarization degree, contrary to the observational results \citep{2023arXiv230301174V}; although slim disk models can generate increased 
PDs \citep{West23},
they are again not high enough to account 
for the observed PDs.

We argue here that the previous studies neglected the possibility that the electron distribution function
may exhibit anisotropies in the accretion disk atmosphere. {As opposed to deep within the accretion flow, the accretion disk atmosphere can be weakly collisional, with scale height only a few times the electron collision mean free path. In such an environment, adiabatic invariance will adjust the parallel and perpendicular velocities with respect to the local magnetic field, leading to pressure anisotropy~\citep[e.g.][]{1958PhRv..109.1874P}. A common cause leading to such an anisotropy is large scale shearing motion which is prevalent in accretion disks. In moderate to high $\beta$ plasmas, the pressure anisotropy can trigger fast-growing micro-scale mirror and firehose instabilities that regulate the anisotropy to marginally stable levels, which is observed in the solar wind~\citep{2014PhRvL.112t5003K}.} 
%

This scenario is different from the one mentioned above of an outflowing plasma
\citep{2023arXiv230412752R}. 
Both scenarios predict an anisotropic electron distribution for an observer 
co-rotating with the accretion disk.
The scenario described here predicts substantially higher polarization degrees as the electrons are anisotropic in the rest frame of the emitting plasma (i.e., as seen by the much slower ions).  

Motivated by the {\it IXPE} results,  we ran Monte Carlo radiation transport calculations to evaluate 
the combined effects of Bremsstrahlung emission and 
scattering by anisotropic electrons for different ratios of the 
scattering to absorption cross sections. 
Our calculations show that electron anisotropies can boost the 
polarization by large amounts, easily reaching the $\sim$20\% PDs of the locally emitted X-rays required to explain the 4U\,1630--47 results.
The rest of this letter is organized as follows. After describing 
the Monte Carlo radiation transport code in Sect.\ \ref{s:mc}, 
we present the results from the simulations in Sect.\,\ref{s:res}.
We conclude with a discussion of the plasma and accretion disk 
effects which might generate the required strong electron 
anisotropies in Sect.\,\ref{s:dis}.

Note that solar flares are believed to emit strongly polarized Bremsstrahlung X-rays as well.
We refer the reader to \citep[][and references therein]{2018A&A...612A..64S} for a discussion 
of the observational evidence and 
related theoretical treatments.
%
\section{Methods: radiation transport in an atmosphere with anisotropic electrons}
\label{s:mc}
We demonstrate the effect of electron anisotropies on the polarization 
of the emergent X-rays based on Monte Carlo radiation 
transport simulations similar to those of 
\citet{1978ApJ...219..705B,2011A&A...536A..93J}. 
Our code uses the scattering engine described in \citep{2017ApJ...850...14B}.
Recent publications describing  
Monte Carlo simulations of Comptonizing
plasmas include  \citep{2019ApJ...875..148Z,2022ApJ...934....4K,2023arXiv230708023K}.

In this letter, we model the electron anisotropy 
as a purely directional anisotropy without an
associated temperature anisotropy.
Future work might explore the observational 
signatures of different electron distributions 
in the three dimensional momentum space.
We use two different parameterizations 
of the angular distribution of the electrons.
The first parameterization assumes that the electrons are distributed with azimuthal symmetry
around the surface normal of the atmosphere.
The cosines $\mu$ of the polar angles $\theta$ 
of the electrons are assumed to follow 
the distribution:
\begin{equation}
p(\mu)\propto (\mu^2)^n {\rm\,for\,n \ge 0,\,or}\label{e1}
\end{equation}
\begin{equation}
p(\mu)\propto (1-\mu^2)^{-n} {\rm\,for\,n<0.}\label{e2}
\end{equation}
Values $n>0$ ($n<0$) correspond to electron distributions  with electrons moving preferentially perpendicular (parallel) to the  atmosphere. 

% Figure environment removed
The second parameterization mimics gyrotropic electrons for magnetic field lines
parallel to the surface of the atmosphere, 
for which the electron velocity distribution is  isotropic perpendicular to the magnetic field, 
but the velocity distribution perpendicular 
to and parallel to the magnetic field are different. 
Such conditions are expected to develop
in accretion disks as shearing motions are likely 
to generate a dominant toroidal magnetic field. 
In this case, we also use the probability distributions of Equations (\ref{e1}) 
and (\ref{e2}), but with $\mu$ being 
the cosine of the angle between 
the electron velocity and the $x$-axis 
parallel to the atmosphere (corresponding to the
direction of the magnetic field).
We call the index $m$ instead of $n$ to clearly
distinguish this setup from the one above. 
We collect photons leaving the atmosphere 
within  $\pm10^{\circ}$ azimuthal angles 
from the $x$-axis ($y$-axis) to get the 
polarization for observers viewing the 
atmosphere along the direction of the $x$-axis
($y$-axis).

Our code runs in two configurations. Configuration \#1 is used to study the polarization of X-rays
emitted by pure scattering atmospheres 
with anisotropic electrons. The photons are emitted  at the bottom of the atmosphere extending
from $z=-5\,l_{\rm sc}$ to $z=0$ with $l_{\rm sc}$ being the scattering mean free path. 
We set the polarization of the emitted 
photons to 0 to clearly see the effect of the scatterings on the photon polarization.

Photons scatter off electrons drawn from a 
Maxwell-J\"uttner distribution of temperature $T$. 
The code uses four wave-vectors $k^{\mu}=(E,E\vec{n})$ to keep track of the energy 
$E$ and the direction $\vec{n}$ of a photon. 
The code keeps track of the photon's linear polarization with the help of a parameter tracking the PD, 
and the polarization vector $f^{\mu}=(0,\vec{f})$ with $|\vec{f}|=1$ encoding 
the electric field polarization direction 
\citep{2018grav.book.....M}.
The Compton scattering is effected by Lorentz transforming $k^{\mu}$ and $f^{\mu}$ into the 
scattering electron's rest frame. 
After drawing a random direction of the scattered 
photon, we construct the Stokes vector of the incoming photon referenced to the plane 
spanned by the wave vectors of the 
incoming and outgoing photon. 
The Stokes vector of the outgoing photon is 
calculated by multiplying the Stokes vector 
of the incoming photon with Fano's 
fully relativistic scattering matrix \citep{1957RvMP...29...74F,1961RvMP...33....8M,2017ApJ...850...14B}.
% Figure environment removed
A rejection algorithm uses the Stokes-$I$ parameter of the scattered photon to account for the 
energy and scattering angle dependence of the 
Klein-Nishina cross section. 
The scattering changes the photon energy in 
the electron rest frame according to 
Compton's equation.
The Stokes vector is subsequently used to 
infer the PD and polarization direction 
$\vec{f}$ of the scattered photon. 
In the last step, the wave and polarization vectors 
are transformed back into the plasma frame.
We switched off all relativistic effects 
\citep[change of electron energy, 
Klein-Nishina cross section, scattering probability as function 
of the angle between electron velocity and photon wave vector, see][]{2017ApJ...850...14B} to verify that the code reproduces Chandrasekhar's results.
The Comptonization code was furthermore cross-checked in the deep
Klein-Nishina regime against the MONK code 
(W. Zhang, private communication, 2022).

% Figure environment removed
Configuration \#2 is used to study the impact of the polarized 
Bremsstrahlung emission, photon absorption, and Compton scattering
on the polarization of the emergent emission. 
We simulate a 5-absorption-length-deep atmosphere 
at (electron) temperature $T$. 
The relative importance of emission and scattering is 
parameterized by the ratio $r_{\rm sc/a}$ of the absorption to scattering cross sections, 
and we use the parameters $n$ and $m$ to 
characterize the electron anisotropy.
Bremsstrahlung photons are emitted uniformly throughout the atmosphere. 
The PD and polarization angle are generated making use of the relativistic cross sections $\sigma_{II}$ and $\sigma_{III}$ for the emission of photons polarized parallel and perpendicular 
to the plane defined by the electron and photon velocity vectors derived by \citet{1953PhRv...90.1030G}
in the first Born approximation. 
Note that Equations (4.2) and (4.3) of \citep{1953PhRv...90.1030G} 
are correct up to a typo (multiplication instead of 
subtraction at the beginning of the last line their Equation (4.3)) 
that Gluckstern acknowledged in a private communication mentioned 
in \citep{1978ApJ...219..705B}.
The reproductions of these equations in \citep{1978ApJ...219..705B}
include errors as do those in \citep{2016MNRAS.461.2162K}.
The latter authors give the Bremsstrahlung 
cross sections in convenient form, but 
their Equation (16) for $L$ 
includes a factor of 2 that should be dropped.
In our code, photons propagate until they are absorbed, 
scatter, or escape the atmosphere. The Compton scatterings 
are simulated as explained for configuration \#1.
% Figure environment removed

Photons escaping the atmosphere are sorted into 6 bins in the cosine 
of the inclination of the observer $\mu_{\rm obs}$ and in 5 bins in energy, and the Stokes parameters within each bin are summed. Owing to the symmetry of the 
plane parallel atmosphere, Stokes-$U$ vanishes. We denote electric 
field polarizations perpendicular to the atmosphere as 
positive polarization (PD=$Q/I>$0), and polarizations parallel 
to the atmosphere as negative polarization (PD=$Q/I<$0).
%
\section{Results: polarization from anisotropic electrons}
\label{s:res}
%
We present here the results from simulations performed for a plasma temperature of $1\,keV$, which roughly equals the temperature of the 
upper layers of the photosphere of the inner portion of the accretion 
disk of 4U\,1630--47 during the 
{\it IXPE} observations
(accounting for a spectral hardening factor of 1.8).

The black solid line in Figure \ref{f:chandra} shows the polarization from 
configuration \#1 (pure electron scattering atmosphere) for an isotropic electron distribution ($n=0$). Interestingly, the strength of the polarization decreases with energy, as on average more scatterings are needed to scatter thermally 
emitted lower-energy photons to higher energies.
The red lines show that electrons moving preferentially perpendicular to the atmosphere ($n>0$) make the polarization parallel to the atmosphere stronger. The blue lines give the 
results for electron moving preferentially 
parallel to the atmosphere ($n<0$) creating 
polarization perpendicular to the atmosphere which
can win over the parallel polarization at higher energies. 

% Figure environment removed
The results above show that anisotropic 
Compton scattering can increase the PDs. 
In the next step, we will show that the inclusion 
of absorption and Bremsstrahlung 
can give  rise to much higher PDs. 
Figure\,\ref{f:bs} shows the polarization 
of the Bremsstrahlung from mono-directional 1\,keV electrons. At the lowest energies ($^<_{\sim}0.1$\,keV) the emission is polarized 
perpendicular to the direction of the electrons, 
as the emission comes predominantly from 
small deflections of the electrons. 
At higher energies, the emission is polarized parallel to the  electron beam, as the 
emission comes mostly from the longitudinal acceleration of the electrons. 
The parallel polarization reaches 100\% in the limit that the photon energy equals the electron energy.
A similar switch from perpendicular to parallel polarization occurs for the Bremsstrahlung emission from  monodirectional electrons at a temperature of $k_{\rm B}T=1$\,keV, see Fig.\,\ref{f:bs}, dashed red line.


Figures \ref{f:updown} and \ref{f:leftright} show the results from the radiative transport simulations of configuration \#2, where the emergent PD depends on the combined effect of polarized Bremsstrahlung emission, photon absorption, and photon scattering. Figure\,\ref{f:updown} shows that electrons moving  preferentially perpendicular to the disk with 
$n=1$ can produce 65$^{\circ}$-inclination 
PDs increasing from 2 to 8 keV from 
25\% to 40\% if absorption dominates over scattering 
($r_{\rm sc/a}=0.2$). 
Although the polarization is weaker when the electrons move preferentially parallel to the disk 
with $n=-1$ (Fig.\,\ref{f:leftright}), the PDs can still be sufficiently high to explain the 
4U\,1630--47 results.

Figures\,\ref{f:m+1} and Figures\,\ref{f:m-1} present the results for electron direction
distributions symmetric around the $x$-axis parallel
to the surface of the atmosphere. 
For both types of distributions ($m>0$ and $m<0$), the polarization direction 
depends on the viewing direction.
The forward-backward anisotropies with $m>0$ 
tend to produce stronger polarizations than the 
toroidal anisotropies with $m<0$.
% xxx
%
% Figure environment removed
\section{Discussion}
\label{s:dis}
Our calculations demonstrate that polarized Bremsstrahlung can produce large net-PDs as long 
as Compton scatterings do not suppress 
the polarization towards Chandrasekhar's classical results.  For electron anisotropies of order unity, the net PD can easily reach the $\sim$20\% levels required for explaining the observed 6\%-10\% PDs  of the 2-8 keV emission from  4U\,1630--47. 
Work is in progress to embed the results presented above into the {\tt kerrC} code that simulates the emission from all parts of an accretion disk of a spinning Kerr black hole and ray traces the emission to the observer
\citep{2022ApJ...934....4K}.
Simulations for a wide range of black hole spin parameters and accretion rates would allow us 
to fit the {\it IXPE} observations of 4U\,1630--47 and  other sources observed in the HSS.

In the thermal state, the deeper layers of the accretion disk are highly collisional and emit blackbody emission. The radiation then passes through the accretion disk atmosphere where it is reprocessed owing to bound-bound, bound-free, and free-free absorption processes and Compton scatterings. The electrons 
cool by Bremsstrahlung emission 
and inverse Compton processes. 
The net outcome of the absorption,
scattering, and emission processes
is a reprocessed quasi-thermal emission 
at a higher temperature than 
in the mid-disk, but with a lower 
surface brightness than blackbody 
emission at this temperature
\citep[e.g., ][]{1993ApJ...419...78S,1995ApJ...445..780S,2021MNRAS.501.3393T}.  
We argue in this letter that the reprocessing of the emission in the atmosphere can generate the strong polarization of the emerging X-rays.

If the density in the accretion disk atmosphere is low enough such that the scale height is only a few times the electron collisional mean free path, the plasma is weakly collisional. Since the electron-electron collision rate is similar to the electron-ion collision rate which determines the efficiency of Bremsstrahlung emission \citep[e.g.,][]{kulsrud_plasma_2005,thorne_modern_2017}, the electrons in the atmosphere are weakly collisional while the ions can be collisionless.
In this regime, the electron and ion distributions can develop anisotropy: their temperature/pressure parallel and perpendicular to the magnetic field can be different. The anisotropy is typically driven by the change of the magnetic field. For example, the differential rotation of the gas in the disk atmosphere leads to the amplification of the toroidal magnetic field. In the absence of collisions, particles will conserve their adiabatic invariants, therefore, the pressure perpendicular to the magnetic field $P_{\perp}$ and parallel to the magnetic field $P_{\parallel}$ evolve separately according to:
\begin{equation}
    \frac{d}{dt}\left(\frac{P_{\perp}}{\rho B}\right)=0,\ 
    \frac{d}{dt}\left(\frac{P_{\parallel}B^2}{\rho^3}\right)=0,
\end{equation}
where $\rho$ is the plasma density, $B$ is the magnetic field \citep{1956RSPSA.236..112C}. The shear motion increasing $B$ without changing $\rho$ will lead to $P_{\perp}>P_{\parallel}$. On the other hand, if $B$ decreases, $P_{\parallel}>P_{\perp}$ develops instead.

The anisotropy can be limited by a few processes. Firstly, collisions will tend to isotropize the particle distribution. As a rough estimation, 
\begin{equation}
    \frac{P_{\perp}-P_{\parallel}}{P}\sim\frac{1}{\nu}\frac{1}{B}\frac{dB}{dt} \sim \frac{u}{v_{\rm th}}\frac{\lambda}{L_u}, 
\end{equation}
where $\nu$ is the collision rate, $\lambda$ is the collisional mean free path, $v_{\rm th}$ is the thermal velocity of plasma particles, $u$ is the (shear) flow velocity and $L_u$ is the length scale over which the velocity varies \citep{2016MNRAS.461.2162K}. In the disk atmosphere, the orbital velocity $u$ is close to Keplerian, $u\gg v_{\rm th}$, and $L_u$ is a few times of the collisional mean free path $\lambda$ in the weakly collisional regime, therefore it is possible to get order unity anisotropy. Other processes that can limit the anisotropy include kinetic instabilities caused by the pressure anisotropy itself: 
the fire hose instability can develop when $P_{\parallel}/P_{\perp}>(1-2/\beta_{\parallel})^{-1}$, and a mirror instability can develop in the opposite regime $P_{\perp}/P_{\parallel}>1+1/\beta_{\perp}$, where $\beta_{\parallel}=P_{\parallel}/(B^2/8\pi)$ and $\beta_{\perp}=P_{\perp}/(B^2/8\pi)$ \citep[e.g.,][]{2014PhRvL.112t5003K}. These instabilities lead to microscopic fluctuations that 
%scatter particles around, 
can scatter particles, leading to collisional effects that
limit the pressure anisotropy at the threshold values. The amount of anisotropy will be small if $\beta_{\perp}, \beta_{\parallel}\gg1$, e.g. inside the accretion disk. However, in the atmosphere, the beta parameters can be relatively small, $\beta\lesssim$ a few. When $\beta_{\parallel}<2$, the fire hose instability can no longer develop and $P_{\parallel}/P_{\perp}$ is unlimited; for the mirror boundary, when $\beta_{\perp}<1$, the limiting value of $P_{\perp}/P_{\parallel}$ becomes larger than 2. 
The discussion shows that electron anisotropies of order unity are not guaranteed but can develop. 

Since the electron anisotropy evolves with the local coherent magnetic field, it is closely linked to the geometry and evolution of the global magnetic field. 
Most General Relativistic Magnetohydrodynamic (GRMHD) simulations of 
black hole accretion find that any 
initially weak magnetic field in the 
accretion disk is sheared and amplified to be predominantly toroidal \citep[e.g.,][]{2004ApJ...606.1083H}. This appears to be the case for both the standard and normal evolution (SANE) and magnetically arrested disks (MAD) \citep[e.g.][]{2022MNRAS.511.2040B}, except for the region close to the event horizon and the jet base, where significant amount of poloidal field may exist. The second parameterization described in Section~\ref{s:mc} (denoted by index $m$) corresponds to electron anisotropy with respect to a predominantly toroidal magnetic field. 
If shear motion amplifies the magnetic field in the accretion disk and in the accretion disk atmosphere, the outcome $P_{\perp}>P_{\parallel}$ 
corresponds to the polarization signature of Fig.\,\ref{f:m-1}. The $m=-1$ case corresponds to $P_{\perp}/P_{\parallel}=2$ and $m=-5$ case has $P_{\perp}/P_{\parallel}=6$, which may be realized in a weakly collisional, low beta plasma. The polarization perpendicular to
the atmosphere would be stronger than the polarization parallel to the atmosphere and would likely dominate the overall polarization of the signal. 
The $P_{\perp}<P_{\parallel}$ case would produce the opposite
outcome of Fig.\,\ref{f:m+1}. In this scenario, $m=1$ gives $P_{\parallel}/P_{\perp}=3$ and $m=5$ would mean $P_{\parallel}/P_{\perp}=11$.
%
Unfortunately, for the specific case of 
4U~1630$-$47 for which the strong HSS 
polarization  indicates strong electron anisotropies,
we do not have independent information about the orientation of the binary system or the accretion disk in the sky. Thus, even though the electric field polarization gives us the preferred direction of the electron motion, we cannot relate this to the orientation of the accretion disk. 
In contrast, for the source Cyg X-1, the detected 
radio jet \citep{2001MNRAS.327.1273S} constrains 
the accretion disk orientation in the sky 
to be perpendicular to the jet. 
In this case, detailed modeling of HSS data should enable us to observationally determine if
$P_{\perp}>P_{\parallel}$ or 
$P_{\perp}<P_{\parallel}$.

As mentioned above, the {\it IXPE} observations of 4U\,1630--47 in the HSS and SPL states revealed 
similar polarization properties in both states  \citep{2023arXiv230510630R}.  These results can possibly be explained in the framework of the emission model discussed here. The transition  from the HSS to the SPL state would merely 
require the emergence of a non-thermal tail of the
electron distribution. 
A somewhat stronger 
Comptonization of the Bremsstrahlung emission 
could account for the somewhat smaller PDs but identical polarization direction of the emission. 
The relative constancy of the polarization degree would then argue for a rather similar overall disk configuration in the HSS and SPL states, i.e., similar disk truncation radii.  
Given the high PDs that Bremsstrahlung can produce, it might be worthwhile to evaluate if it may impact the polarization of the hard state emission of the sources Cyg X-1 \citep{2022Sci...378..650K} 
and Cyg X-3 \citep{Sasha:23}. 
% The examples of anisotropic distributions we used for calculations in \S\ref{s:res} have 
\\[4ex]
{\it Acknowledgments:}
H.K. thanks Banafsheh Beheshtipour for the code of the
original Compton scattering engine.  
H.K., N.R.C., and K.H. acknowledge extremely pleasant and fruitful discussions with Michal Dov$\check{\rm c}$iak, Jack Steiner, and Javier Garc\'{i}a, as well as with the entire {\it IXPE} stellar mass black hole 
science working group. H.K., N.R.C., and K.H.\ acknowledge NASA support through the grants NNX16AC42G, 80NSSC20K0329, 80NSSC20K0540, NAS8- 03060, 80NSSC21K1817, 80NSSC22K1291, and 80NSSC22K1883. A.C.\ and Y.Y.\ acknowledge support from NSF grants DMS-2235457 and AST-2308111. A.C.\ also acknowledges NASA support from grant 80NSSC21K2027. Y.Y.\ also acknowledges support by the Multimessenger Plasma Physics Center (MPPC), NSF grant PHY-2206608. The Washington University authors acknowledge support from the McDonnell Center for the Space Sciences.
%\bibliographystyle{mnras}
\bibliography{apjl}
\end{document}
