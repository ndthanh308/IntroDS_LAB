\subsection{Finding the extrema of $z(x)$}

As has been done in prior research, we shall make use of a graph to help with the analysis. Here, though, we make the observation that the analysis of critical points of $z$ is made dramatically simpler by plotting the coarse approximation function in pseudolog-pseudolog space since, as a consequence of \eqref{eq:line-XY}, it will always be a straight line. That is, instead of plotting $y$ against $x$, we plot $Y=L(y)$ against $X=L(x)$. We shall find that this reduces much of the problem at hand to the classification of certain crossing points.

Consider a point $P(X,Y)$ lying on the line \eqref{eq:line-XY}. The value of $z$ at $P$, which we shall write as $z|_P$ is given by equation \eqref{eq:z(X,Y)}.

Now, clearly the point $P'(X+b,Y-a)$ must also lie on the line. If the exponents corresponding to $X+b$ and $Y-a$ are labelled respectively $E_x'$ and $E_y'$, and the corresponding mantissas $m_x'$ and $m_y'$, by \eqref{eq:Em(X)} we have
\begin{align*}
  E_x'&=\floor{X+b}=E_x+b\,, \quad\quad m_x'=X+b-E_x'=m_x\,, \\
  E_y'&=\floor{Y-a}=E_y-a\,, \quad\quad m_y'=Y-a-E_y'=m_y\,.
\end{align*}
Thus only the exponents change, and the value of $z$ at $P'$ is
\begin{equation} \label{eq:periodicity}
  z|_{P'} = 2^{a (E_x+b) + b (E_y-a)} (1+m_x)^a (1+m_y)^b = z|_P\,.
\end{equation}
We conclude that the function $z$ is periodic on the line \eqref{eq:line-XY}, with period at most $b$ along the $X$-axis and, equivalently, period at most $a$ along the $Y$-axis. We can thus completely characterise the behaviour of $z$ by examining it over a representative interval, say $X\in[0,b]$.

From Algorithm \ref{alg:FRGR} and equation \eqref{eq:L_inv(X)}, we see that $z$ is a continuous function of $X$, and by the extreme value theorem attains a minimum and a maximum on $X \in [0,b]$. Due to the periodic nature of $z$, these also serve as the minimum and maximum of $z$ over all values of $X$.

We can also see that $z$ must have continuous derivative with respect to $X$ except wherever $X$ or $Y$ is an integer, since these are the points where the exponent of $x$ or $y$ changes. Thus, the minimum and maximum of $z$ must occur either where the line crosses a boundary of an integer $(X,Y)$-grid square, or where $z$ is stationary on the line.

In order to locate the points at which $z$ is stationary, consider again the point $P(X,Y)$. Since $E_x=\floor{X}$ and $E_y=\floor{Y}$, we can say that $P$  lies in the square $[E_x,E_x+1)\times[E_y,E_y+1)$. On this region, the exponents $E_x$ and $E_y$ can be treated as constants - only the mantissas $m_x$ and $m_y$ vary. To find the derivative of $z$ with respect to $X$, we make use of \eqref{eq:line-XY} and \eqref{eq:Em(X)} to establish the following:
\[  \frac{dm_x}{dX}=\frac{dm_y}{dY}=1\,, \quad \frac{dY}{dX}=-\frac{a}{b}\,,  \]
and then apply the chain rule to \eqref{eq:z(X,Y)} to obtain
\[  \frac{dz}{dX} =  2^{aE_x+bE_y} a (1+m_x)^{a-1} (1+m_y)^{b-1} (m_y-m_x) \,.  \]
Since $2^{aE_x+bE_y}>0$, $a>0$, and $m_x,m_y \geqslant 0$, $\frac{dz}{dX}$
is zero if and only if $m_x=m_y$. Thus there is exactly one stationary point of $z$ on the grid square in question, and it is located where the line \eqref{eq:line-XY} crosses the diagonal which joins the corners with $(X,Y)$-coordinates $(E_x,E_y)$ and $(E_x+1,E_y+1)$.

We can summarise these findings by stating that for a given value of $c$, the extrema $z_\text{min}(c)$ and $z_\text{max}(c)$ of $z$ must be contained in the union of the following finite sets:
\begin{align} \label{eq:HVD1}
  H &= \{z: \enspace X \in \mathbb{Z}, \enspace 0 \leqslant X < b \} \,, \nonumber \\
  V &= \{z: \enspace Y \in \mathbb{Z}, \enspace 0 \leqslant X < b \} \,, \nonumber \\
  D &= \{z: \enspace X-Y \in \mathbb{Z}, \enspace 0 \leqslant X < b \}\,,
\end{align}
the notation suggesting points whose horizontal, vertical and diagonal coordinates, respectively, are integers.

We have thus characterised a set of candidates for the extrema of $z$, namely the set $H \cup V \cup D$ in \eqref{eq:HVD1}. We proceed to find expressions for the members of $H$. The members of $V$ can be treated analogously, and the members of $D$ by a slight modification of the same method.

In the relationship defined in \eqref{eq:line-XY}, let $s$ and $t$ be the floor and fractional parts, respectively, of $c$,
\begin{equation} \label{eq:s-t}
  s = \floor{c} \,, \quad  t = c - s \,,
\end{equation}
so that we have
\begin{equation} \label{eq:line-XY-st}
  aX+bY=s+t\,, \quad s \in \mathbb{Z}\,, \quad 0 \leqslant t < 1\,.
\end{equation}
Now let $X$ be an integer, as will be the case for a member of the set $H$. Considering that $s-aX$ and $b$ are both integers, let $q_b$ and $r_b$ be respectively the quotient and remainder upon dividing $s-aX$ by $b$, so that
\[  s-aX = q_b b + r_b\,, \quad q_b,r_b\in\mathbb{Z}\,, \quad 0\leqslant r_b < b \,.  \]
By \eqref{eq:line-XY-st}, therefore,
\[  Y = q_b + \frac{r_b+t}{b}\,,  \]
and, since $0 \leqslant r_b+t < b$, the fractional part of $Y$ (i.e. $Y-\floor{Y}$) is $\frac{r_b+t}{b}$.

We are now in a position to evaluate $z$ for the chosen integer $X$-coordinate. From \eqref{eq:Em(X)} and \eqref{eq:L_inv(X)}, since $X \in \mathbb{Z}$ we have $x=2^X$. We have also broken $Y$ into integer and fraction parts, allowing us to write $y=2^{q_b} (1+\frac{r_b+t}{b})$. Substituting into \eqref{eq:z(X,Y)} we obtain
\begin{equation} \label{eq:z-intX}
  z|_{X \in \mathbb{Z}} = (2^X)^a (2^{q_b})^b \left(1+\frac{r_b+t}{b} \right)^b = 2^{s-r_b} \left(1+\frac{r_b+t}{b} \right)^b\,.
\end{equation}
Note that as $X$ ranges over the representative set $\{0,...,b-1\}$, since $a$ and $b$ are coprime, the remainder $r_b$ varies over the full set of residues modulo $b$. However, $q_b$ has disappeared from the expression for $z$, as a consequence of the periodicity observed in \eqref{eq:periodicity}.

With $s$ and $t$ as in \eqref{eq:s-t}, let us now introduce a function $\zeta_{r,k}(c)$ defined by
\begin{equation} \label{eq:zeta}
  \zeta_{r,k}(c) = 2^{s-r} \left( 1+\frac{r+t}{k} \right)^k\,, \quad c \in \mathbb{R}\,, k \in \mathbb{Z^+},  r \in \mathbb{Z}_k \,.
\end{equation}
We are now able to express \eqref{eq:z-intX} more compactly as
\[  z|_{X \in \mathbb{Z}} = \zeta_{r_b,b}(c)\,,  \]
where $r_b$ is some integer in $\{0,...,b-1 \}$. By completely analogous reasoning, the value of $z$ at a point on the line $\eqref{eq:line-XY}$ such that $Y$ is an integer is found to be
\[  z|_{Y \in \mathbb{Z}} = \zeta_{r_a,a}(c)\,,  \]
with $r_a$ some integer in $\{0,...,a-1 \}$.

A slight variant of this reasoning allows us to evaluate $z$ at points where $X-Y$ is an integer. We first rewrite equation \eqref{eq:line-XY-st} as
\[  aW + \gamma Y=s+t \,,  \]
with $W=X-Y$ and $\gamma = a+b$. Now we let $q_{\gamma}$ and $r_{\gamma}$ be respectively the quotient and remainder on dividing $s-aW$ by $\gamma$, so that
\[  s-aW=q_{\gamma} \gamma +r_{\gamma}\,, \quad q_{\gamma},r_{\gamma} \in \mathbb{Z}\,, \quad 0 \leqslant r_{\gamma} < \gamma \,,  \]
from which we derive
\begin{equation} \label{eq:Y-diag}
  Y=q_{\gamma}+\frac{r_{\gamma}+t}{\gamma}\,,
\end{equation}
and since $0 \leqslant r_{\gamma}+t < \gamma$, this expression decomposes $Y$ into integer and fraction parts. We now note that if $X-Y$ is an integer then we must also have
\begin{equation} \label{eq:X-diag}
  X=q'+\frac{r_{\gamma}+t}{\gamma} \,,
\end{equation}
for some $q' \in \mathbb{Z}$. Since $X$ and $Y$ must satisfy \eqref{eq:line-XY-st},
\[  a \left(q'+\frac{r_{\gamma}+t}{\gamma} \right) + b \left(q_{\gamma}+\frac{r_{\gamma}+t}{\gamma} \right) = s+t\,.  \]
Solving for $q'$ we find
\begin{equation} \label{eq:q-bar}
  q'=\frac{s-bq_{\gamma}-r_{\gamma}}{a}\,.
\end{equation}
Having decomposed both $X$ and $Y$ we may proceed to evaluate $z$ at the point $(X,Y)$. From \eqref{eq:Y-diag} and \eqref{eq:X-diag} we have
\[  x = 2^{q'} \left(1+\frac{r_{\gamma}+t}{\gamma} \right)\,, \quad y = 2^{q_{\gamma}} \left(1+\frac{r_{\gamma}+t}{\gamma} \right)\,,  \]
and therefore
\[  z = 2^{aq'+bq_{\gamma}} \left(1+\frac{r_{\gamma}+t}{\gamma} \right)^{a+b}\,.  \]
Now using \eqref{eq:q-bar} and simplifying, we find
\[  z|_{X-Y \in \mathbb{Z}} = \zeta_{r_{\gamma},\gamma}(c)\,,  \]
where $r_{\gamma} \in \{0,...,\gamma-1 \}$.

This allows us to list the values of the members of sets $H$, $V$ and $D$ explicitly:
\begin{align} \label{eq:HVD2}
  H & = \{ \zeta_{r_b,b}(c):0 \leqslant r_b < b \} \,, \nonumber \\
  V & = \{ \zeta_{r_a,a}(c):0 \leqslant r_a < a \} \,, \nonumber \\
  D & = \{ \zeta_{r_{\gamma},\gamma}(c):0 \leqslant r_{\gamma} < \gamma \}\,.
\end{align}
It remains to determine which of the elements in these sets are in fact the extrema of $z$. We break this task into two parts: first, we identify which set contains $z_\text{min}$, and which contains $z_\text{max}$. We then pick out the extrema from among the members of each set so identified.

It can be shown (see Appendix \ref{appendix:zeta}) that $\zeta_{r,k}(c)$ is increasing with respect to $k$,
\[  r < k_1 <  k_2 \implies \zeta_{r,k_1}(c) \leqslant \zeta_{r,k_2}(c)\,.  \]
If we now set
\begin{equation} \label{eq:alpha-beta-gamma}
  \alpha = \min(a,b)\,, \quad \beta = \max(a,b)\,, \quad \gamma = a + b \,,
\end{equation}
then we have
\[  0 < \alpha \leqslant \beta < \gamma \,,  \]
and hence for any $r < \alpha$,
\[  \zeta_{r,\alpha}(c) \leqslant \zeta_{r,\beta}(c)\,,  \]
and for any $r < \beta$,
\[  \zeta_{r,\beta}(c) \leqslant \zeta_{r,\gamma}(c)\,.  \]
Comparing with \eqref{eq:HVD2} we conclude that $z_\text{min}$ belongs to $V$ when $a \leqslant b$ and to $H$ when $a \geqslant b$, and that $z_\text{max}$ belongs to the set $D$:
\begin{align}
  z_\text{min}(c) & \in \{ \zeta_{r_{\alpha},\alpha}(c) : 0 \leqslant r_{\alpha} < \alpha \} \,, \label{eq:zmin-candidates} \\
  z_\text{max}(c) & \in \{ \zeta_{r_{\gamma},\gamma}(c) : 0 \leqslant r_{\gamma} < \gamma \} \,. \label{eq:zmax-candidates}
\end{align}
These two sets of candidate points are illustrated for the FRSR case in Figure \ref{fig:candidates} using an arbitrarily chosen value of $0.7$ for $c$. We have shaded the representative interval $[0,b]$, and shown the periodic repeats of the candidate points outside this interval.

% Figure environment removed

The final stage in finding expressions for the extrema is to identify from among the candidates in \eqref{eq:zmin-candidates} which one is in fact $z_\text{min}$, and similarly for \eqref{eq:zmax-candidates} and $z_\text{max}$.

Let $k \geqslant 1$ be a fixed integer, and consider the set of functions
\[  S_k = \left \{ \zeta_{r,k}(c) \right \}_{0 \leqslant r < k}\,.  \]
Clearly, for the simple case $k=1$, since the set contains only the single function $\zeta_{0,1}(c)$, it must serve as both the minimum and maximum member of the set, for all values of $c$. So let us assume that $k \geqslant 2$ in the remainder of this section.

To analyse the general case, consider first the continuous function
\[  \eta(\theta) = 2^{-\theta} \left( 1+\theta \right), \quad \theta \in [0,1]\,,  \]
which is plotted in Figure \ref{fig:eta}.

% Figure environment removed

Elementary calculus enables us to determine that $\eta(\theta)$ is minimum at the endpoints of the interval, is maximum at $\theta = \frac{1}{\ln 2}-1=\bar{\theta}$, say, and that it is strictly increasing for $\theta < \bar{\theta}$ and strictly decreasing for $\theta > \bar{\theta}$. By restricting $\theta$ to the values $\frac{r}{k}$ with $0 \leqslant r < k$ it becomes clear that $\eta(\frac{r}{k})$ is minimum only when $r=0$.

As for the maximum, we note that $\frac{r}{k}$ cannot equal $\bar{\theta}$, the latter being irrational, hence the maximum must occur for one of the two points straddling $\bar{\theta}$ (and there must be two, since $0 < \bar{\theta} < \frac{1}{2} \leqslant \frac{k-1}{k}$). It cannot be both; for if $\eta(\frac{r}{k}) = \eta(\frac{r+1}{k})$, then by expanding and rearranging we have
\[  2^{1/k} = \frac{r+k+1}{r+k}\,,  \]
and clearly if $k \geqslant 2$ then the left side is irrational while the right side is rational, which is impossible.

Hence we can say that there exists a unique $\bar{r} \in \mathbb{Z}_k$ for which $\eta(\frac{\bar{r}}{k})$ is maximal, whence
\begin{align}
  0 \leqslant r < r' \leqslant \bar{r} & \implies \eta \left( \frac{r}{k} \right) < \eta \left( \frac{r'}{k} \right) \,, \label{eq:eta-ineq0} \\
  \bar{r} \leqslant r < r' < k & \implies \eta \left( \frac{r}{k} \right) > \eta \left( \frac{r'}{k} \right) \,.   \label{eq:eta-ineq1}
\end{align}
Now define a new function $\widehat{\zeta}_{r,k}(t)$ closely related to $\zeta_{r,k}(c)$ as follows:
\[  \widehat{\zeta}_{r,k}(t) = 2^{-r/k} \left( 1+\frac{r+t}{k} \right), \quad 0 \leqslant t \leqslant 1, \quad 0 \leqslant r < k\,, \]
and define a corresponding set of functions
\begin{equation} \label{eq:set-S-hat}
  \widehat{S}_k = \left \{ \widehat{\zeta}_{r,k}(t) \right \}_{0 \leqslant r < k}\,.
\end{equation}
It is easily verified that
\begin{equation} \label{eq:zeta_hat(0)}
  \widehat{\zeta}_{r,k}(0) = \eta \left( \frac{r}{k} \right)\,,
\end{equation}
and that
\begin{equation} \label{eq:zeta_hat(1)}
  \widehat{\zeta}_{r,k}(1) = 2^{\frac{1}{k}} \eta \left( \frac{r+1}{k} \right)\,.
\end{equation}
Putting $r'=r+1$ into \eqref{eq:eta-ineq0} and using \eqref{eq:zeta_hat(0)}, we find
\begin{align}
  0 \leqslant r < \bar{r} & \implies \widehat{\zeta}_{r+1,k}(0) - \widehat{\zeta}_{r,k}(0) > 0 \,, \nonumber \\
  \bar{r} \leqslant r < k-1 & \implies \widehat{\zeta}_{r+1,k}(0) - \widehat{\zeta}_{r,k}(0) < 0 \,.  \label{eq:zeta_hat(0)_monotonic}
\end{align}
By similar reasoning using \eqref{eq:eta-ineq1} and \eqref{eq:zeta_hat(1)},
\begin{align}
  0 \leqslant r < \bar{r}-1 & \implies \widehat{\zeta}_{r+1,k}(1) - \widehat{\zeta}_{r,k}(1) > 0 \,, \nonumber \\
  \bar{r}-1 \leqslant r < k-1 & \implies \widehat{\zeta}_{r+1,k}(1) - \widehat{\zeta}_{r,k}(1) < 0 \,.  \label{eq:zeta_hat(1)_monotonic}
\end{align}
Noting that $\widehat{\zeta}$ is linear in $t$, i.e.,
\[  \widehat{\zeta}_{r,k}(t) = (1-t) \widehat{\zeta}_{r,k}(0) + t \widehat{\zeta}_{r,k}(1)\,,  \]
for $t \in [0,1]$ we can say that $\widehat{\zeta}_{r,k}(t)$ is a convex combination of $\widehat{\zeta}_{r,k}(0)$ and $\widehat{\zeta}_{r,k}(1)$, so that \eqref{eq:zeta_hat(0)_monotonic} and \eqref{eq:zeta_hat(1)_monotonic} combine to yield
\begin{align}
  0 \leqslant r < \bar{r}-1 & \implies \widehat{\zeta}_{r+1,k}(t) > \widehat{\zeta}_{r,k}(t) \,, \nonumber \\
  \bar{r} \leqslant r < k-1 & \implies \widehat{\zeta}_{r+1,k}(t) < \widehat{\zeta}_{r,k}(t) \,, \label{eq:zeta-hat-partition}
\end{align}
and this partitions the set $\widehat{S}_k$ of \eqref{eq:set-S-hat} into two subsets, each of which contains non-intersecting functions.

At this point, an example will serve to clarify. Let us consider the case $k=7$, and plot the set of functions $\widehat{S}_7$ for $t \in [0,1]$. The result is shown in Figure \ref{fig:zeta_hat}; the meaning of the symbols $t_0$ and $t_1$ will be explained presently. In this case we have $\bar{r}=3$, and as hinted at by the two different line styles, $\widehat{S}_7$ can be partitioned into two subsets,
\[  \left \{ \widehat{\zeta}_{r,7}(t) \right \}_{0 \leqslant r < 3 } \cup \left \{ \widehat{\zeta}_{r,7}(t) \right \}_{3 \leqslant r < 7 }  \]
each of which contains disjoint line segments and is thus totally ordered.

% Figure environment removed

In the general case, \eqref{eq:zeta-hat-partition} partitions $\widehat{S}_k$ into the subsets
\[  \left \{ \widehat{\zeta}_{r,k}(t) \right \}_{0 \leqslant r < \bar{r} } \cup \left \{ \widehat{\zeta}_{r,k}(t) \right \}_{\bar{r} \leqslant r < k } \,.  \]
Using the total order which \eqref{eq:zeta-hat-partition} induces on each subset, the minimum and maximum members of the first subset are $\widehat{\zeta}_{0,k}(t)$ and $\widehat{\zeta}_{\bar{r}-1,k}(t)$, respectively, and those of the second subset are  $\widehat{\zeta}_{k-1,k}(t)$ and $\widehat{\zeta}_{\bar{r},k}(t)$, respectively. 

The minimum for the full set $\widehat{S}_k$ is obtained by choosing appropriately, depending on the value of $t$, between the pair of subset minima $\widehat{\zeta}_{0,k}(t)$ and $\widehat{\zeta}_{k-1,k}(t)$. Suppose these two functions intersect at $t=t_0$. Setting the two functions equal and solving for $t_0$ yields
\begin{equation} \label{eq:t0}
  t_0(k) = \frac{k-1}{2^{1-\frac{1}{k}}-1}-k\,.
\end{equation}
Although we have stipulated that $k$ be a positive integer, it is instructive to extend the domain of $t_0$ to include all non-negative real numbers. A graph of $t_0(k)$ is shown in Figure \ref{fig:t0}.

% Figure environment removed

Note that this function is undefined when $k$ takes either of the values $0$ or $1$. However, the limit exists in both cases, and we find that
\[  \lim_{k \to 0} t_0(k) = 1\,, \quad \lim_{k \to 1} t_0(k) = \frac{1}{\ln{2}} -1 \,, \]
the first limit being trivial to show, and the second being easily evaluated using l'H$\hat{\text{o}}$pital's rule. For completeness, then, we define $t_0(0)=1$ and $t_0(1)=\frac{1}{\ln{2}}-1$.

In Appendix \ref{appendix:t0}, we demonstrate bounds on $t_0(k)$ (see equation \eqref{eq:t0-bounds}); for present purposes it suffices to note that they imply $t_0(k)$ lies strictly between $0$ and $1$ under our assumption that $k \geqslant 2$.

From \eqref{eq:zeta_hat(0)}, $\widehat{\zeta}_{r,k}(0)$ is minimised by setting $r=0$, and we deduce that the minimum member of $\widehat{S}_k$ is $\widehat{\zeta}_{0,k}(t)$ for values of $t$ less than $t_0(k)$, and is $\widehat{\zeta}_{k-1,k}(t)$ all greater values of $t$:
\begin{equation} \label{eq:min-zeta-hat}
  \min_{0 \leqslant r < k} \widehat{\zeta}_{r,k}(t) =
  \begin{cases}
    \widehat{\zeta}_{0,k}(t)	& \mbox{if } t < t_0(k) \,, \\ 
    \widehat{\zeta}_{k-1,k}(t)	& \mbox{if } t \geqslant t_0(k)\,.
  \end{cases}
\end{equation}
Similarly, for the maximum, we select between the two subset maxima $\widehat{\zeta}_{\bar{r}-1,k}(t)$ and $\widehat{\zeta}_{\bar{r},k}(t)$. Supposing these functions to intersect at $t=t_1$, we shall show that $t_1$ lies strictly between $0$ and $1$.

Since the unique minimum of $\eta(\frac{r}{k})$ occurs for $r=0$, then the maximum must occur for some other value of $r \in \mathbb{Z}_k$, since $k \geqslant 2$. Hence,
\begin{equation} \label{eq:r_bar-bounds}
  1 \leqslant \bar{r} \leqslant k-1 \,.
\end{equation}
We can therefore substitute $r = \bar{r}-1$ into both \eqref{eq:zeta_hat(0)_monotonic} and \eqref{eq:zeta_hat(1)_monotonic} to obtain
\begin{align*}
  \widehat{\zeta}_{\bar{r}-1,k}(0) - \widehat{\zeta}_{\bar{r},k}(0) < 0 \,, \\
  \widehat{\zeta}_{\bar{r}-1,k}(1) - \widehat{\zeta}_{\bar{r},k}(1) > 0 \,.
\end{align*}
By the intermediate value theorem, therefore, the functions $\widehat{\zeta}_{\bar{r}-1,k}(t)$ and $\widehat{\zeta}_{\bar{r},k}(t)$ are equal for some $t$ in the open interval $(0,1)$. By definition, this value is $t_1$; hence $0 < t_1 < 1$.

By equating the two functions we obtain
\[  2^{-\frac{\bar{r}-1}{k}} \left( 1+\frac{\bar{r}-1+t_1}{k} \right) = 2^{-\frac{\bar{r}}{k}} \left( 1+\frac{\bar{r}+t_1}{k} \right)\,,  \]
which can be rearranged to give
\[  \bar{r}(k) + t_1(k) = \phi(k)\,,  \]
where
\begin{equation} \label{eq:phi}
  \phi(k) = \frac{1}{2^{\frac{1}{k}}-1} - k+1\,.
\end{equation}
A graph of the function $\phi(k)$ is shown in Figure \ref{fig:phi}, where again we have extended the domain to include all non-negative real $k$ (and have defined $\phi(0) = \underset{k \to 0}{\lim} \phi(k) = 1$).

% Figure environment removed

Since $\bar{r} \in \mathbb{Z}$ and $0 < t_1 <1$, clearly $\bar{r}$ and $t_1$ must be the integer and fraction parts, respectively, of $\phi(k)$:
\begin{equation} \label{eq:r_bar-t1}
  \bar{r}(k) = \left\lfloor \phi(k) \right\rfloor\,, \quad  t_1(k) = \phi(k) - \bar{r}(k) \,.
\end{equation}  
Furthermore, since $\widehat{\zeta}_{r,k}(0)$ is maximised by setting $r=\bar{r}$ (by definition of $\bar{r}$), we can deduce that $\widehat{\zeta}_{\bar{r},k}(t)$ is the maximum for values of $t$ less than $t_1$, and that $\widehat{\zeta}_{\bar{r}-1,k}(t)$ is the maximum for all greater values:
\begin{equation} \label{eq:max-zeta-hat}
  \max_{0 \leqslant r < k} \widehat{\zeta}_{r,k}(t) =
  \begin{cases}
    \widehat{\zeta}_{\bar{r},k}(t)	& \mbox{if } t < t_1(k) \,, \\ 
    \widehat{\zeta}_{\bar{r}-1,k}(t)	& \mbox{if } t \geqslant t_1(k)\,.
  \end{cases}
\end{equation}
The intersection points corresponding to $t_0(7)$ and $t_1(7)$ have been highlighted in Figure \ref{fig:zeta_hat}.

At this point, we note that $\zeta$ and $\widehat{\zeta}$ are related via
\[  \zeta_{r,k}(c) = 2^s \left( \widehat{\zeta}_{r,k}(t) \right) ^k \,.  \]
Since $\widehat{\zeta}_{r,k}(t) > 0$, raising it to the positive power $k$ and multiplying by the positive number $2^s$ both preserve monotonicity with respect to $r$ . Hence we can extend results $\eqref{eq:min-zeta-hat}$ and $\eqref{eq:max-zeta-hat}$ from $\widehat{\zeta}$ to $\zeta$ to obtain
\begin{equation} \label{eq:min-zeta}
  \min_{0 \leqslant r < k} \zeta_{r,k}(c) =
  \begin{cases}
    \zeta_{0,k}(c)	& \mbox{if } t < t_0(k) \,, \\ 
    \zeta_{k-1,k}(c) 	& \mbox{if } t \geqslant t_0(k) \,,
  \end{cases}
\end{equation}
\begin{equation} \label{eq:max-zeta}
  \max_{0 \leqslant r < k} \zeta_{r,k}(c) =
  \begin{cases}
    \zeta_{\bar{r},k}(c)	& \mbox{if } t < t_1(k) \,, \\ 
    \zeta_{\bar{r}-1,k}(c)	& \mbox{if } t \geqslant t_1(k) \,.
  \end{cases}
\end{equation}
As the final step, we combine \eqref{eq:min-zeta} with \eqref{eq:zmin-candidates}, and \eqref{eq:max-zeta} with \eqref{eq:zmax-candidates}, to find that
\begin{align}
  z_\text{min}(c) &= \zeta_{r_{\alpha},\alpha}(c) \,,      \label{eq:z_min}  \\
  z_\text{max}(c) &= \zeta_{r_{\gamma},\gamma}(c)\,, \label{eq:z_max}
\end{align}
where the indices $r_{\alpha}$ and $r_{\gamma}$ are given by the piecewise-constant expressions
\begin{equation} \label{eq:r_alpha}
  r_{\alpha} =
  \begin{cases}
    0		& \mbox{if } t < t_0(\alpha) \,, \\
    \alpha-1	& \mbox{if } t \geqslant t_0(\alpha) \,,
  \end{cases}
\end{equation}
\begin{equation} \label{eq:r_gamma}
  r_{\gamma} =
  \begin{cases}
    \bar{r}(\gamma)		& \mbox{if } t < t_1(\gamma) \,, \\
    \bar{r}(\gamma)-1	& \mbox{if } t \geqslant t_1(\gamma)\,.
  \end{cases}
\end{equation}
We now have the desired expressions for $z_\text{min}$ and $z_\text{max}$ in terms of $c$.

For notational compactness, we will henceforth use $t_0, t_1$ and $\bar{r}$ to mean exclusively $t_0(\alpha), t_1(\gamma)$ and $\bar{r}(\gamma)$, respectively.
