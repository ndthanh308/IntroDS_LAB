
\section{Related Work}

An investigation by \citet{sommefeldt2006} into the history of the algorithm concluded that it was developed by Greg Walsh and Cleve Moler, who had learned about the bit-manipulation technique from an unpublished 1986 paper by William Kahan and K. C. Ng, a copy of which can be seen in the source for \texttt{fdlibm} from \citet{sun1993}. The algorithm gained widespread attention in 2005 when id Software released the source code for their game \textit{Quake III: Arena} \citep{id1998}. Their code contained a version with the ``magic constant" \texttt{0x5F3759DF}, achieving a peak relative error of $1.752339 \times 10^{-3}$. \citet{lomont2003} used analysis coupled with a numerical search to show that the optimal choice for the constant is \texttt{0x5F375A86}, lowering the error bound slightly to $1.751302 \times 10^{-3}$. \citet{moroz2016} used a purely analytical method to find the same constant Lomont had found, without requiring a numerical search. Far more significant improvements became possible using the observation that the Newtonian iteration step itself contained two further constants, which could be tuned together with the magic constant. \citet{pizer2008} thus found a trio of constants which lowered the error bound to $6.531342 \times 10^{-4}$, a $2.7$-fold improvement. Using a similar approach, \citet{kadlec2010} lowered the error bound a little further to $6.501967 \times 10^{-4}$, but still required a lengthy numerical search for the values of the constants, and no demonstration of their optimality. \citet{walczyk2021} provided an analytical method of finding a theoretically optimal set of constants, but did not perform any fine tuning to reduce evaluation error, so that Kadlec's remains the most accurate hitherto published version\footnote{When comparing results, care must be taken to ensure that the testing conditions match, since the hardware platform, the function implementation, and the instructions emitted by the compiler can all affect the outcome, even when IEEE 754 compliance is guaranteed. Our basis for comparison is described in the results section.}. The work by Walzcyk et al. also showed how to extend the optimisation technique to additional iterations, providing a dramatic improvement in accuracy over the multi-iteration version of the Quake code. A reciprocal cube root counterpart to the original algorithm was given by \citet{levin2012}. \citet{moroz2021} provided the hitherto best published constants for this algorithm, and showed how to replace the Newtonian iteration step with a quadratic Householder iteration for much greater accuracy. They proceeded to find superior coefficients for this method, but we show that their approach did not produce an optimal value for the magic constant. \citet{blinn1997} showed a corresponding coarse approximation for a general power of $x$, but did not consider optimisation of the relevant constant, or refinement of the coarse value.
