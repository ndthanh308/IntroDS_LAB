\subsection{Method for choosing an optimal value for $c$} \label{sec:optimal-c}

With $n$ the degree of the approximating polynomial $p(z)$ in \eqref{eq:y-tilde}, our optimisation problem now requires the determination of $n+2$ values: the $n+1$ coefficients of $p$, and the number $c$. It may appear as though these two problems are inextricably linked. Fortunately, however, there is a method of optimising $c$ independently of $p$.

Given some value of $c$, consider a linear remapping of $z$ given by
\[  \widehat{z}(z)=\frac{z}{z_\text{min}(c)}\,,  \]
and note that as $z$ varies over $[z_\text{min},z_\text{max}]$, so $\widehat{z}$ varies over $[1,\rho]$ with
\begin{equation} \label{eq:rho}
  \rho(c)=\frac{z_\text{max}}{z_\text{min}}\,.
\end{equation}
Note also that the error function in equation \eqref{eq:e-tilde} can be expressed as
\[  \widetilde{e} = \frac{(z_\text{min} \widehat{z})^{-\frac{1}{b}} - p(z_\text{min} \widehat{z})} {(z_\text{min} \widehat{z})^{-\frac{1}{b}}}
                = \frac{\widehat{z}^{-\frac{1}{b}} - q(\widehat{z})} {\widehat{z}^{-\frac{1}{b}}}\,,  \]
where $q(\widehat{z})$ is a polynomial of degree $n$. In other words, the error is the same as that incurred by approximating the function $\widehat{z}^{-\frac{1}{b}}$ by a polynomial of degree $n$ over the interval $[1,\rho]$. Furthermore, the derivative of $\widetilde{e}$ with respect to $\widehat{z}$ is
\[  \frac{d \widetilde{e}}{d \widehat{z}} = -\frac{1}{b} \widehat{z}^{\frac{1}{b}-1} (q(\widehat{z}) + b \widehat{z} q'(\widehat{z}))\,,  \]
which, since $\widehat{z}>0$, is zero only when the polynomial $q(\widehat{z}) + b \widehat{z} q'(\widehat{z})$ is zero. Since the degree of this polynomial is at most $n$, the error function has at most $n$ stationary points.

The Chebyshev Alternation Theorem (see, for example, \citet{fike1968}\footnote{The theorem is usually stated and proved for minimax absolute error approximations. The cited reference also states the theorem in the relative error context suitable for our purposes. The conditions this latter version requires are fulfilled by our scenario.}) tells us that for optimal $q$, the error curve $\widetilde{e}$ must exhibit at least $n+2$ greatest deviations from zero, of equal magnitudes and alternating signs. Since $\widetilde{e}$ has continuous derivative with respect to $\widehat{z}$ on $(0,\infty)$, each point of greatest deviation must either be a stationary point or an endpoint of the interval. And since the stationary points account for at most $n$ of them, the endpoints must constitute the remaining two, and there must be exactly $n+2$ in total.

Now consider the effect of reducing the interval from $[1,\rho]$ to $[1,\rho_1]$, with $0<\rho_1<\rho$, and suppose that a polynomial $q_1(\widehat{z})$ of degree $n$ serves as a minimax approximation on the reduced interval. Let $\epsilon$ and $\epsilon_1$ denote the resulting minimax errors on the original and reduced intervals, respectively.

Since $q(\widehat{z})$ incurs peak error $\epsilon$ on $[1,\rho]$, it incurs no greater an error on a subset of that interval, and hence $\epsilon_1 \leqslant \epsilon$. On the other hand, if $\epsilon_1 = \epsilon$ then both $q$ and $q_1$ are minimax polynomials on $[1,\rho_1]$, and by the uniqueness of the minimax polynomial (see, for example, \citet{fike1968}) we must have $q_1 \equiv q$.

However, the reduced interval excludes the endpoint $\widehat{z}=\rho$ at which one of the original $n+2$ points of greatest deviation lies, leaving at most $n+1$ such points, so that $q$ fails to satisfy the Chebyshev Alternation Theorem on the reduced interval, which is impossible. Hence we must have $\epsilon_1 < \epsilon$.

We conclude that an optimal interval is one which minimises the ratio $\rho = z_\text{max} / z_\text{min}$. Since $z_\text{min}$ and $z_\text{max}$ are dependent on the value of $c$, this minimisation is performed by an appropriate choice of $c$. \footnote{Strictly, we have not addressed the question of the existence of an optimal interval, but our method will demonstrate its existence by construction.}
