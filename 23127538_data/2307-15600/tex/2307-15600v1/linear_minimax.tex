\section{Analytic solution of the linear case}

In general, minimax polynomial coefficients do not have closed form expressions, and we must resort to numerical methods to determine them. However, the case of a linear polynomial approximation to $z^{-\frac{1}{b}}$, which occurs in Algorithm \ref{alg:FRGR} for the case $n=1$, is a sufficiently simple one for a closed form solution to exist.

\subsection{\mbox{General linear minimax approximation for $z^{-\frac{1}{b}}$}}

Suppose $c_0 + c_1 z$ is the linear minimax approximation for $z^{-\frac{1}{b}}$ on an interval $[z_\text{min}, z_\text{max}]$. The relative error function has three peaks of equal magnitude and alternating signs, two at the endpoints of the interval and a third, interior point where the error function is stationary. The error can be expressed as
\[  e(z) = 1 - z^{\frac{1}{b}} (c_0 + c_1 z)\,,  \]
with derivative
\[  \frac{de}{dz} = -\frac{1}{b} z^{\frac{1}{b}-1} (c_0 + (b+1) c_1 z)\,.  \]
This can only be zero when $z$ takes the value $z_\text{mid}$, where
\[  z_\text{mid} = -\frac{c_0}{(b+1) c_1}\,,  \]
with equioscillation implying
\begin{equation} \label{eq:linear-equioscillation}
  e(z_\text{min}) = -e(z_\text{mid}) = e(z_\text{max}) = \epsilon\,,
\end{equation}
$\epsilon$ being the minimax error. Solving \eqref{eq:linear-equioscillation} for $c_0, c_1$ and $\epsilon$ leads to
\begin{equation} \label{eq:linear-minimax-solution}
  c_0 = \frac{2T}{U+V}\,, \quad c_1 = \frac{-2}{U+V}\,, \quad \epsilon = \frac{U-V}{U+V}\,,
\end{equation}
with the quantities $T, U, V$ defined as
\begin{align*}
  T & = \frac {z_\text{max}{}^{1+\frac{1}{b}} - z_\text{min}{}^{1+\frac{1}{b}}} {z_\text{max}{}^{\frac{1}{b}} - z_\text{min}{}^{\frac{1}{b}}} \,, \\[5pt]
  U & = b \left( \frac{T}{b+1} \right) ^ {1+\frac{1}{b}} \,, \\[5pt]
  V & = \frac {(z_\text{min} z_\text{max})^{\frac{1}{b}} (z_\text{max} - z_\text{min})} {z_\text{max}{}^{\frac{1}{b}} - z_\text{min}{}^{\frac{1}{b}}} \,.
\end{align*}



\subsection{Application to the FRSR case} \label{sec:app-FRSR}

\citet{walczyk2021} presented an analytic derivation of an optimal set of theoretical constants for the FRSR case. Now that we have developed methods for tackling the more general case, we can use them to quickly derive the FRSR result and confirm their conclusion.

\renewcommand{\thefootnote}{\fnsymbol{footnote}}
The FRSR case corresponds to setting the values $a=1$, $b=2$, $n=1$ in Algorithm \ref{alg:FRGR}, for which the optimal values of $c\footnote[2]{The derivation of the ``magic constant" from the value $c$ will be given in Section \ref{sec:implementation}}, c_0, c_1$ can be obtained using Algorithm \ref{alg:FRGR-constants}. The latter algorithm contains one degree of freedom, the integer $s$; here we choose the value $-1$ so as to correspond with the prior literature. (Choosing any other value would yield an identical value for $\epsilon$, but different values for $c_0$ and $c_1$.)
\renewcommand{\thefootnote}{\arabic{footnote}}

Inserting these values into Algorithm \ref{alg:FRGR-constants}, we obtain
\[  c = -\frac{1}{2}, \quad z_\text{min} = \frac{3}{4}, \quad z_\text{max} = \frac{27}{32} \,,  \]
and the polynomial $p(z) = c_0 + c_1 z$ is then the minimax polynomial of degree $1$ approximating $z^{-\frac{1}{2}}$ on $z \in \left[ \frac{3}{4}, \frac{27}{32} \right]$. Using \eqref{eq:linear-minimax-solution} for the special case $b=2$, we obtain
\begin{align} \label{eq:c0-c1-eps-FRSR}
  c_0 &= \frac { 12(27\sqrt{2}-32) } { \sqrt{10729 - 7242\sqrt{2}} + 9\sqrt{6} } \approx 1.68191391\,, \nonumber \nonumber \\[5pt]
  c_1 &= \frac { 128(4 - 3\sqrt{2}) } { \sqrt{10729 - 7242\sqrt{2}} + 9\sqrt{6} } \approx -0.703952009\,, \nonumber \nonumber \\[5pt]
  \epsilon &=  \frac { \sqrt{10729 - 7242\sqrt{2}} - 9\sqrt{6} } { \sqrt{10729 - 7242\sqrt{2}} + 9\sqrt{6} } \approx 6.50070298 \times 10^{-4} \,.
\end{align}
Our results agree identically with those found by Walczyk et al.



\subsection{Application to the reciprocal function}

It is worth noting in passing that the case of the reciprocal function with a linear minimax polynomial, where $a = b = n = 1$, yields a particularly simple solution under our analysis. For this case, Algorithm \ref{alg:FRGR-constants} together with equation \eqref{eq:linear-minimax-solution} yields the following values (once again choosing the value $-1$ for $s$):
\[  c = \sqrt{2}-2, \quad z_\text{min} = \frac{\sqrt{2}}{2}, \quad z_\text{max} = \frac{1}{8} (3 + 2\sqrt{2}) \,,  \]
\begin{align*}
  c_0 &= \frac {192}{6913} (206\sqrt{2} - 191) \approx 2.78648558\,, \\[3pt]
  c_1 &= \frac {512}{6913} (84\sqrt{2} - 145) \approx -1.94090888 \,, \\[3pt]
  \epsilon &=  \frac {1}{6913} (4481 - 3168\sqrt{2}) \approx 1.11591842 \times 10^{-4}\,.
\end{align*}
We can observe that, even with one fewer multiply instruction than the FRSR algorithm, it can achieve almost $6$ times better accuracy.
