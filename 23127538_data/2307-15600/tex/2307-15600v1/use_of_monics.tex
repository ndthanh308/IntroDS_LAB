\section{The use of monic polynomials} \label{sec:monics}

It is common practice to evaluate the polynomial part of the approximation using Horner's rule to reduce the number of floating point operations. For example, in the case $n=3$,
\[  c_3z^3+c_2z^2+c_1z+c_0 = ((c_3z+c_2)z+c_1)z+c_0\,.  \]
In architectures lacking a fused multiply-add (FMA) instruction, this evaluation scheme requires 6 arithmetic operations: 3 multiplies, and 3 adds. In general, a polynomial $p(z)$ of degree $n$ requires $2n$ operations to evaluate in this manner, so that only even numbers of operations arise. It is natural to wonder whether there might be expressions using an odd number of operations which constitute intermediates, in terms of both cost and accuracy. \textit{Monic polynomials}, i.e. polynomials where the leading coefficient is $1$, provide such intermediates.

A monic cubic polynomial, for example, can be evaluated using just 5 operations:
\[  z^3+c_2z^2+c_1z+c_0 =  ((z+c_2)z+c_1)z+c_0\,.  \]
We may also include polynomials with leading coefficient $-1$, which can be evaluated using the same number of operations, by replacing an add with a subtract:
\[  -z^3+c_2z+c_1z+c_0 = ((c_2-z)z+c_1)z+c_0\,.  \]
Suppose in Algorithm \ref{alg:FRGR} that the constant $c$, and hence the approximation interval $[z_\text{min},z_\text{max}]$ has been chosen, and consider the effect of enforcing that the degree-$n$ polynomial $p(z)$ be a signed monic of the form
\[  p(z) = (-z)^n + q(z) \,,  \]
for some polynomial $q$ of degree at most $n-1$, the sign $(-1)^n$ of the leading term of $p$ having been chosen to match that of the $n^\text{th}$-degree general minimax polynomial arising from Algorithm \ref{alg:FRGR-constants}.

If, as in Section \ref{sec:refined-approximation}, $p(z)$ is an approximation for $z^{-\frac{1}{b}}$, the relative error is
\[  \widetilde{e} = \frac{z^{-\frac{1}{b}} - (-z)^n - q(z)}{z^{-\frac{1}{b}}}\,,  \]
the optimisation of which is a \textit{weighted} minimax problem (see, for example, \citet{green2002}): find the polynomial $q$ of degree at most $n-1$ which best approximates $z^{-\frac{1}{b}} - (-z)^n$ using a weight function $z^{\frac{1}{b}}$. Once again, the problem yields to standard minimax theory and the coefficients can be found numerically using a minimax solver.

As before, we can only proceed to find the coefficients once we know the approximation interval. However, in this case, there may not exist a technique that corresponds to the one in Section \ref{sec:optimal-c} which isolates the optimisation of the value $c$ independently from the polynomial coefficients.

On the other hand, it is possible to develop iterative numerical software that converges on the best constant $c$ and simultaneously the best coefficients $c_i$. Empirically, it has been found that for many small values of the input variables $a, b, n$, the optimal value of $c$ is the unique one for which the minimax (general) polynomial happens to be a signed monic. (This is not true in general, a counterexample being $a=b=1, n=3$.)

In the results section, we will use a monic polynomial to demonstrate a reciprocal square root algorithm which is 1 operation faster than the original Quake code while at the same time being roughly twice as accurate. We also exhibit a monic quadratic polynomial for FRSR, showing how, with the addition of a single floating point add instruction, the accuracy can be improved more than 86-fold over the original Quake FRSR.

It may also be observed that, since the monic polynomial of degree $0$ is the constant $1$, a function which simply returns the value of the coarse approximation now fits into our framework, corresponding to the case $p(z) \equiv 1$.
