\subsection{Coarse Approximation}

As has been well-established by prior work, given a suitable choice for the constant $c$, the value of $y$ in line \ref{alg:FRSR:line:y} of Algorithm \ref{alg:FRSR} gives a coarse approximation to $1/ \sqrt{x}$. That is, multiplication of $X$ by $-\frac{1}{2}$ in pseudolog space approximates raising $x$ to the power $-\frac{1}{2}$. By the same token, we can approximate other fractional powers of $x$ by multiplying $X$ by the appropriate fraction.\footnote{On many platforms this fractional multiply can be achieved using a small number of low-cost integer operations - for example, when using \texttt{gcc} to compile for a 64-bit x86 target, the integer expression \texttt{X/3} typically generates an integer multiply followed by a shift operation.}

In the present work, for reasons which will become clear, only rational powers of $x$ will be considered, and so the target function to be approximated can be written as
\begin{equation} \label{eq:requirements}
  f(x)=x^{-\frac{a}{b}}\,,  \quad  x \in \mathbb{R^+}\,, \enskip a,b \in \mathbb{Z}^+\,, \enskip gcd(a,b)=1\,.
\end{equation}
Note that we are assuming the fraction $\frac{a}{b}$ has been reduced to its lowest terms, and that only negative powers of $x$ are being considered. (Positive rational powers can then be obtained by multiplying by a suitable integer power of $x$, which can be achieved using only multiply instructions.) Note also that we are considering the domain of $f$ to include all positive real numbers, and that all calculations will be considered exact. (This assumption obviously does not hold in a practical implementation, where only finite precision is available, and where the optimal choices for the values of the constants may deviate from the analytically derived ones. We will return to this problem in section \ref{sec:implementation}.)

To generalise the coarse approximation in Algorithm \ref{alg:FRSR}, we replace the fractions $\frac{c}{2}$ and $\frac{1}{2}$ in line \ref{alg:FRSR:line:Y} with $\frac{c}{b}$ and $\frac{a}{b}$, respectively, so that the following linear relationship holds between $X$ and $Y$:
\begin{equation} \label{eq:line-XY}
  aX+bY=c\,.
\end{equation}
In justification of this, since $X \approx \log_2{x}$ and $Y \approx \log_2{y}$, then
\[  y \approx 2^Y = 2^{\frac{c-aX}{b}} \approx 2^{\frac{c}{b}} 2^{-\frac{a}{b} \log_2 x} = 2^{\frac{c}{b}} x^{-\frac{a}{b}}\,,  \]
so that our coarse approximation $y$ is roughly the target function $x^{-\frac{a}{b}}$ scaled by the constant value $2^{\frac{c}{b}}$. Setting $c$ to zero would provide a simple way to have $y \approx x^{-\frac{a}{b}}$, though as the FRSR case suggests, this is not generally optimal. We show later how to choose an optimal value for $c$.
