\section{Discussion \& Conclusion}
\label{sec:conclusion}
In this paper, we assessed the potential of using GPUs to simulate quantum computers using the state vector approach that simulates the evolution of the quantum state complex array after applying a number of gate transformations. We evaluated the performance of different quantum applications with the IBM Qiskit Aer simulator that provides two backends for GPUs: one based on Nvidia Thrust library and one based on the cuQuantum SDK. 

We are now in the position of answering the initial research question: \emph{are GPUs suitable and a key enabling technology for the acceleration of quantum computer simulations?} Brief answer: yes they are. Overall, we found that GPUs can provide a large improvement of performance with cuQuantum providing a major computational boost both in memory and compute throughput. In particular, for simulations with large number of qubits, GPUs can provide up to $14\times$ performance boost with Nvidia cuQuantum being $1.5-3\times$ faster than the original Thrust backend for several of the presented benchmarks and reaching the roofline in our experiments. While the usage of cuQuantum's compute capability are outstanding, we note that Nvidia tensor core units~\cite{markidis2018nvidia} are not used. Their usage could in principle provide an extra performance gain at the cost of reduced accuracy of the calculations (tensor cores work in mixed precision).

The major obstacle of state-vector simulations is that they quickly hit the memory and computational wall due to the exponential growth of the computational and memory requirements with the increase of the number of qubits we want to simulate. One possibility is to use multiple computational nodes or disaggregated systems and memory pooling~\cite{wahlgren2022evaluating} for additional memory. However, we have shown that at least on-node, the integration of MPI and GPU backends is not optimal. The usage of MPI is a necessary technology for high efficiency. Yet, the challenge of exponential memory will limit the state vector simulation below 50 qubits.

Two main approaches could address these limitations. The first strategy is to adopt hybrid approaches, combining for instance state vector and Feynman path simulator~\cite{markov2018quantum}, that trade memory usage for increased computation. For instance, this simulation approach is used in Google \texttt{qsimh} simulator~\cite{isakov2021simulations}. Because these hybrid approaches use the state vector simulator technology they are capable of exploiting GPU acceleration, as shown in this work. A second strategy is to exploit the sparsity of state vector and use compression. We note that the Qiskit Aer and many other state-of-the-art simulators use dense state vector. Important considerations for future work should be sparse representations, for instance like in bitwise-representation quantum computer simulators~\cite{da2020qsystem}, and usage of sparse linear algebra libraries for GPUs.


