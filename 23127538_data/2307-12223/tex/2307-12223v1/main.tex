\documentclass[aps,prb,twocolumn,superscriptaddress, longbibliography]{revtex4-2}
\usepackage{graphicx}
%\usepackage[draft]{graphicx}
\usepackage{latexsym}
\usepackage{amssymb}
\usepackage{amsmath}
\usepackage{amsfonts}
\usepackage[vcentermath]{youngtab}
\usepackage{upgreek}
\usepackage{bm,dsfont}
\usepackage{multirow}
\usepackage{enumitem}
\usepackage[usenames, dvipsnames]{color}
\usepackage[svgnames]{xcolor}
\usepackage{hyperref}
\hypersetup{
colorlinks = true,
linkcolor = [rgb]{0.70,0.13,0.13}, %{0.0, 0.14, 0.4}
citecolor = [rgb]{0.13,0.55,0.13},
urlcolor = [rgb]{0.25, 0.41, 0.88}}
\usepackage{makecell}
\renewcommand\theadalign{bc}
\renewcommand\theadfont{\bfseries}
\renewcommand\theadgape{\Gape[0pt]}
\renewcommand\cellgape{\Gape[2pt]}
\newcommand{\bra}[1]{\langle #1|}
\newcommand{\ket}[1]{|#1 \rangle}
\newcommand{\braket}[2]{\langle #1|#2 \rangle}
\newcommand{\dd}{\mathrm{d}}
\newcommand{\DD}{\mathcal{D}}
\newcommand{\ii}{\mathrm{i}}
\newcommand{\e}{\mathrm{e}}
\newcommand{\rep}{\mathbf{r}}
\newcommand{\ybox}{\tiny\yng}
\newcommand{\id}{\mathds{1}}
\newcommand{\E}{\mathop{\mathbb{E}}}
\newcommand{\LU}{\mathrm{LU}}
\newcommand{\U}{\mathrm{U}}
\newcommand{\SU}{\mathrm{SU}}
\renewcommand{\O}{\mathrm{O}}
\newcommand{\SO}{\mathrm{SO}}
\newcommand{\Sp}{\mathrm{Sp}}
\newcommand{\Spin}{\mathrm{Spin}}
\renewcommand{\u}{\mathfrak{u}}
\newcommand{\su}{\mathfrak{su}}
\renewcommand{\o}{\mathfrak{o}}
\newcommand{\so}{\mathfrak{so}}
\newcommand{\spin}{\mathfrak{spin}}
\newcommand{\Cl}{{\mathcal{C}\ell}}
\newcommand{\dsA}{\mathbb{A}}
\newcommand{\dsB}{\mathbb{B}}
\newcommand{\dsC}{\mathbb{C}}
\newcommand{\dsD}{\mathbb{D}}
\newcommand{\dsE}{\mathbb{E}}
\newcommand{\dsF}{\mathbb{F}}
\newcommand{\dsG}{\mathbb{G}}
\newcommand{\dsH}{\mathbb{H}}
\newcommand{\dsI}{\mathbb{I}}
\newcommand{\dsJ}{\mathbb{J}}
\newcommand{\dsK}{\mathbb{K}}
\newcommand{\dsL}{\mathbb{L}}
\newcommand{\dsM}{\mathbb{M}}
\newcommand{\dsN}{\mathbb{N}}
\newcommand{\dsO}{\mathbb{O}}
\newcommand{\dsP}{\mathbb{P}}
\newcommand{\dsQ}{\mathbb{Q}}
\newcommand{\dsR}{\mathbb{R}}
\newcommand{\dsS}{\mathbb{S}}
\newcommand{\dsT}{\mathbb{T}}
\newcommand{\dsU}{\mathbb{U}}
\newcommand{\dsV}{\mathbb{V}}
\newcommand{\dsW}{\mathbb{W}}
\newcommand{\dsX}{\mathbb{X}}
\newcommand{\dsY}{\mathbb{Y}}
\newcommand{\dsZ}{\mathbb{Z}}
\newcommand{\scA}{\mathcal{A}}
\newcommand{\scB}{\mathcal{B}}
\newcommand{\scC}{\mathcal{C}}
\newcommand{\scD}{\mathcal{D}}
\newcommand{\scE}{\mathcal{E}}
\newcommand{\scF}{\mathcal{F}}
\newcommand{\scG}{\mathcal{G}}
\newcommand{\scH}{\mathcal{H}}
\newcommand{\scI}{\mathcal{I}}
\newcommand{\scJ}{\mathcal{J}}
\newcommand{\scK}{\mathcal{K}}
\newcommand{\scL}{\mathcal{L}}
\newcommand{\scM}{\mathcal{M}}
\newcommand{\scN}{\mathcal{N}}
\newcommand{\scO}{\mathcal{O}}
\newcommand{\scP}{\mathcal{P}}
\newcommand{\scQ}{\mathcal{Q}}
\newcommand{\scR}{\mathcal{R}}
\newcommand{\scS}{\mathcal{S}}
\newcommand{\scT}{\mathcal{T}}
\newcommand{\scU}{\mathcal{U}}
\newcommand{\scV}{\mathcal{V}}
\newcommand{\scW}{\mathcal{W}}
\newcommand{\scX}{\mathcal{X}}
\newcommand{\scY}{\mathcal{Y}}
\newcommand{\scZ}{\mathcal{Z}}
\newcommand{\sfh}{\mathsf{h}}
\newcommand{\Tr}{\operatorname{Tr}}
\newcommand{\sgn}{\operatorname{sgn}}
\newcommand{\vol}{\operatorname{vol}}
\renewcommand{\Re}{\operatorname{Re}}
\renewcommand{\Im}{\operatorname{Im}}
\newcommand{\vect}[1]{{\bm{#1}}}
\newcommand{\norm}[1]{{\lVert #1\rVert}}
\newcommand{\mat}[1]{\left[\begin{matrix}#1\end{matrix}\right]}
\newcommand{\smat}[1]{\left[\begin{smallmatrix}#1\end{smallmatrix}\right]}
\newcommand{\eq}[1]{\begin{equation}#1\end{equation}}
\newcommand{\eqs}[1]{\begin{equation}\begin{split}#1\end{split}\end{equation}}
\newcommand{\eqnref}[1]{Eq.\,\eqref{#1}}
\newcommand{\figref}[1]{Fig.\,\ref{#1}}
\newcommand{\tabref}[1]{Tab.\,\ref{#1}}
\newcommand{\secref}[1]{Sec.\,\ref{#1}}
\newcommand{\appref}[1]{Appendix\,\ref{#1}}
\newcommand{\refcite}[1]{Ref.\,\onlinecite{#1}}
\newcommand{\tocite}[1]{[{\color[rgb]{0.13,0.55,0.13}{#1}}]}

\def \Z{\mathbb{Z}}

\newcommand{\bea}{\begin{eqnarray}}
\newcommand{\eea}{\end{eqnarray}}
\def\be{\begin{equation}}
\def\ee{\end{equation}}
\newcommand{\beq}{\begin{equation}}
\newcommand{\eeq}{\end{equation}}
\newcommand{\beqn}{\begin{eqnarray}}
\newcommand{\eeqn}{\end{eqnarray}}
\newcommand{\cred}[1]{\textcolor{red}{#1}}
\newcommand{\cblue}[1]{\textcolor{blue}{#1}}
\newcommand{\yzy}[1]{{\color[rgb]{0.70,0.13,0.13}{#1}}}

\newcommand{\dcl}[1]{{\color[rgb]{1,0.5,0.0}{\textsf{DCL}: #1}}}
\newcommand{\mz}[1]{{\color[rgb]{0,0.5,1}{\textsf{MZ}: #1}}}
%%RG notation
\newcommand{\hda}{H_\text{int, AA}}
\newcommand{\hdb}{H_\text{int, AB}}
\newcommand{\hbb}{H_\text{int, BB}}
\newcommand{\hcf}{H_\text{int, CF}}

\newcommand{\abs}[1]{|#1|}

\newcommand{\jw}[1]{\textcolor{brown}{#1}} 

% Green function notations
\newcommand{\bk}{{\vect{k}}}
\newcommand{\br}{{\vect{r}}}
\newcommand{\bR}{{\vect{R}}}
\newcommand{\bq}{\vect{q}}
\newcommand{\bQ}{\vect{Q}}
%    \newcommand{\bk}{\vect{k}}
\newcommand{\bK}{\vect{K}}
\newcommand{\ssb}{\text{SSB}}
\newcommand{\bcs}{\text{BCS}}
\newcommand{\smg}{\text{SMG}}
\newcommand{\LZ}[1]{\mathrm{L}\mathbb{Z}_{#1}}
\newcommand{\zero}{\text{zero}}
\newcommand{\qp}{\text{qp}}
\newcommand{\ttx}{\tilde{x}}
\newcommand{\ttk}{\Tilde{k}}

\usepackage{ulem}
\normalem
 
\begin{document}

\title{Green's Function Zeros in Fermi Surface Symmetric Mass Generation}

\author{Da-Chuan Lu}
\affiliation{Department of Physics, University of California, San Diego, CA 92093, USA}
\author{Meng Zeng}
\affiliation{Department of Physics, University of California, San Diego, CA 92093, USA}
\author{Yi-Zhuang You}
\affiliation{Department of Physics, University of California, San Diego, CA 92093, USA}


%\date{\today} 

\begin{abstract}
The Fermi surface symmetric mass generation (SMG) is an intrinsically interaction-driven mechanism that opens an excitation gap on the Fermi surface without invoking symmetry-breaking or topological order. We explore this phenomenon within a bilayer square lattice model of spin-1/2 fermions, where the system can be tuned from a metallic Fermi liquid phase to a strongly-interacting SMG insulator phase by an inter-layer spin-spin interaction. The SMG insulator preserves all symmetries and has no mean-field interpretation at the single-particle level. It is characterized by zeros in the fermion Green's function, which encapsulate the same Fermi volume in momentum space as the original Fermi surface, a feature mandated by the Luttinger theorem. Utilizing both numerical and field-theoretical methods, we provide compelling evidence for these Green's function zeros across both strong and weak coupling regimes of the SMG phase. Our findings highlight the robustness of the zero Fermi surface, which offers promising avenues for experimental identification of SMG insulators through spectroscopy experiments despite potential spectral broadening from noise or dissipation. 
\end{abstract}
\maketitle

%\tableofcontents

\section{Introduction}

Symmetric mass generation (SMG) \cite{Fid2010ISPT, Fid2011ISPT, Wang1307.7480, Slagle1409.7401, Ayyar1410.6474, Catterall1510.04153, Tong2022SMG, YZY2022SMGreview} is an interaction-driven mechanism that creates many-body excitation gaps in anomaly-free fermion systems \emph{without} condensing any fermion bilinear operator or developing topological orders. It has emerged as a alternative symmetry-preserving approach for mass generation in relativistic fermion systems, which is distinct from the traditional symmetry-breaking Higgs mechanism \cite{Nambu1960higgs,Nambu1961higgs,Goldstone1962higgs,Anderson1963higgs,Englert1964higgs,Higgs1964higgs}. The prospect of SMG offering a potential solution to the long-standing fermion doubling problem \cite{Nielsen1981b,Nielsen1981a,Nielsen1981NoGo,Swift1984,Eichten1986Chiral,Smit1986,KaplanRLB1992} has sparked significant interest in the lattice gauge theory community \cite{Wen1305.1045,You1402.4151,You1412.4784,BenTov1412.0154,Ayyar1511.09071,Ayyar1606.06312,Ayyar1611.00280,DeMarco1706.04648,Ayyar1709.06048,Schaich1710.08137,Butt1811.01015,Butt1810.06117,Kikukawa1710.11618,Kikukawa1710.11101,Wang1807.05998,Wang1809.11171,Catterall2002.00034,Razamat2009.05037,Catterall2021SMG,Butt2101.01026,Butt2111.01001,ZM2022SMG,Catterall2201.00750,Catterall2023KDfermion,Guo2023S2306.17420}. In condensed matter physics, SMG was initially explored within the framework of the interaction-reduced classification of fermionic symmetry protected topological (SPT) states \cite{Fid2010ISPT,Fid2011ISPT,Turner1008.4346,Ryu1202.4484,Qi1202.3983,Yao1202.5805,Gu1304.4569,Wang1401.1142,Metlitski1406.3032,Kapustin2015IFSPT,YZY2014,Cheng1501.01313,Yoshida1505.06598,Gu1512.04919,Song1609.07469,Queiroz1601.01596,Witten1605.02391,Wang1703.10937,Kapustin1701.08264,Wang2018Tunneling1801.05416,Wang1811.00536,Guo1812.11959,Aasen2109.10911,Barkeshli2109.11039}, and has been recently extended to systems with Fermi surfaces \cite{Zou2004.14391,yahui2020-1,Zou2002.02972,yahui2020-2,yahui2021,YZY2022FSSMG}, given the growing understanding that Fermi liquids can be perceived as fermionic SPT states within the phase space \cite{Bulmash1410.4202,yzy2023FSanomaly}.

One important feature of the SMG gapped state lies in the zeros of fermion Green's function \cite{Gurarie1011.2273, yzy2014greenfunctionzero, Catterall1609.08541, YZY2018SMGDQCP-1, Catterall1708.06715, Xu2103.15865} at low-energy. Investigations reveal that the poles of the fermion Green's function in the pristine gapless fermion state will be replaced by zeros in the gapped SMG state as the fermion system goes across the SMG transition upon increasing the interaction strength. This pole-to-zero transition was postulated \cite{yzy2014greenfunctionzero} as a direct indicator of the SMG transition \cite{YZY2018SMGDQCP-1, YZY2018SMGDQCP-2} that can be probed by spectroscopy experiments. However, the presence of similar zeros in the Green's function within Fermi surface SMG states has not been investigated yet, and it is the focus of our present research.


Fermi surface SMG \cite{YZY2022FSSMG} refers to the occurrence of SMG phenomena on Fermi surfaces with non-zero Fermi volumes. It describes scenarios where the fermion interaction transforms a gapless Fermi liquid state (metal) into a non-degenerate, gapped, direct product state (trivial insulator), without breaking any symmetry (for example, without invoking Cooper pairing or density wave orders). Such a metal-insulator transition is viable when Fermi surfaces collaboratively cancel the Fermi surface anomaly \cite{Chong2021FSanomaly, XiaoGang2021FSanomaly, YZY2022FSSMG}. This anomaly can be perceived as a mixed anomaly between the translation symmetry and the charge conservation $\U(1)$ symmetry on the lattice \cite{Meng2016Topo, Shinsei2017FSLSM, Bultinck1808.00324, Else2018FSLSMcrystalline, XiaoGang2021FSanomaly, Chong2021FSanomaly, Seiberg2022emanant}, or as an anomaly of an emergent loop $\LU(1)$ symmetry \cite{Senthil2007.07896,Else2010.10523,Darius-Shi2204.07585} in the infrared theory.


In this work, we present evidence of robust Green's function zeros in Fermi surface SMG states. Let $t$ be the energy scale of band dispersion and $J$ be the energy scale of SMG gapping interaction, we investigate the problem from two parameter regimes:
\begin{itemize}
\item Deep in the SMG phase ($J/t\gg 1$), we start with an exact-solvable SMG product state in a lattice model and calculate the fermion Green's function by treating the fermion hopping as perturbation \cite{CPT2000}. We find that the Green's function $G_{\smg}(\omega,\bk)$ deep in the SMG phase takes the following form 
\begin{equation}\label{eq: GSMG strong}
    G_{\smg}(\omega,\bk) = \frac{\omega+\alpha \epsilon_{\bk}/J^2}{(\omega-\epsilon_\bk/2)^2-J^2},
\end{equation}
where $(\omega,\bk)$ labels the frequency-momentum of the the fermion. $\epsilon_\bk$ is the energy dispersion of the original band structure in the free-fermion limit,  and $\alpha$ is an order-one number depending on other details of the system. One salient feature of $G_{\smg}$ is that it has a series of zeros at $\omega=-\alpha \epsilon_\bk/J^2$ in the frequency-momentum space. At $\omega=0$, the Green's function zeros form a zero Fermi surface that replaces the original Fermi surface.

\item If the SMG phase is adjacent to a spontaneous symmetry breaking (SSB) phase, we use perturbative field theory to argue that the Green's function in the SMG phase near the symmetry-breaking transition ($J/t\gtrsim 1$) should take the form of
\eq{\label{eq: GSMG weak}
G'_\smg(\omega,\bk)=\frac{\omega+\epsilon_\bk}{\omega^2-\epsilon_\bk^2-\Delta_0^2}}
where we assume that the SSB order parameter retains a finite amplitude $\Delta_0$ in the SMG phase, but its phase is randomly fluctuating \cite{Wang2023P2212.05737}. Again, $G'_\smg$ features a series of zeros at $\omega=-\epsilon_\bk$, with the same zero Fermi surface.

%\item Assuming the SMG energy scale is much smaller than the Fermi energy, we provide a topological argument that the shape of the Green's function zeros at $\omega=0$ in the Brillouin zone must match that of the Fermi surface before the SMG transition. The result turns out to be consistent with \eqnref{eq: GSMG strong} and \eqnref{eq: GSMG weak} in both the strong and weak coupling limits, which suggests that the Green's function zero should be a robust feature in the entire SMG phase.
\end{itemize}

%\dcl{1. Motivation - pseudo-gap physics. 2. difference between spontaneously symmetry breaking (Higgs mechanism) and SMG}

%Traditionally, the gapless fermion can acquire mass by the fermion bilinear condensation \cite{Nambu1960higgs,Nambu1961higgs,Goldstone1962higgs,Anderson1963higgs,Englert1964higgs,Higgs1964higgs}, for example, the superconducting gap is obtained by condensing the Cooper pairs. However, this Higgs mechanism spontaneously breaks the symmetry, since the condensation of the fermion bilinears is charged under the global symmetry. On the other hand, recent works develop a novel mechanism that can give gapless fermion mass without spontaneously breaking symmetry but often involves multi-fermion condensation, and it is dubbed as symmetric mass generation (SMG) \cite{Kikukawa2019SMG,Catterall2021SMG,Tong2022SMG,YZY2022SMGreview}.

%The SMG mechanism has been studied in many systems with relativistic fermion, including the system with chiral symmetry \cite{Kikukawa2019SMG,Catterall2021SMG,Tong2022SMG,ZM2022SMG}. In the condensed matter systems, SMG transition is proposed in the honeycomb lattice model, where the Dirac semi metal can acquire mass without spontaneously breaking the symmetry , this model is further studied by the variational Monte Carlo method \cite{YZY2022nuSMG}. The conditions for the SMG to happen necessarily require the system free of anomaly. And this relates to the study of interaction reduced classification of symmetry protected topological phases, for example, see \cite{Fid2010ISPT,Fid2011ISPT,YZY2014,Kapustin2015IFSPT}.

%It is natural to ask the gapping condition for the system with finite density of fermions. The SMG mechanism is recently applied to the system with Fermi surface \cite{YZY2022FSSMG}. In the system with Fermi surface, the translation symmetry and the charge conservation $\U(1)$ symmetry emanates the loop $\LU(1)$ symmetry in the infrared theory \tocite{loopU1,emanate sym}\cite{Senthil2007.07896,Chong2021FSanomaly,Seiberg2022emanant}. The Fermi surface anomaly is characterized by the anomaly of loop $\LU(1)$ symmetry with generalized Lieb-Schultz-Mattis theorem \tocite{FS-LSM}\cite{Shinsei2017FSLSM,Else2018FSLSMcrystalline,Senthil2007.07896}. Fermi surface SMG states it is possible to gap out the Fermi surface without breaking the $\LU(1)$ symmetry, provided the Fermi surface anomaly vanishes \cite{Chong2021FSanomaly,XiaoGang2021FSanomaly,YZY2022FSSMG}. This symmetry preserving gapping mechanism could explain the exotic metal-insulator transition with no symmetry breaking \dcl{I don't know if there is an example}, and pseudo-gap phase in the high-temperature superconductors \cite{yahui2020-1,yahui2020-2,yahui2021}.

%As mentioned earlier, the SMG mechanism involves the condensation of multi-fermion operators, which essentially needs large strength of the fermion interaction to achieve the symmetrically gapped phase. The large interaction seems to render the problem hard to understand perturbatively. However, the strong interaction limit is exactly solvable, and there is a unique tensor product ground state \cite{YZY2022FSSMG,YZY2022nuSMG}. 

%Once the SMG phase is achievable, a realistic question to ask is what is the signature of the SMG and the difference from the Higgs mechanism. In this paper, we show there are unique features of the two-point correlation functions in the SMG phase compared to the Higgs mechanism, and the features can be reconstructed from the experiment measurable spectral functions. 

%In particular, based on our numerical calculations, we find that the strong interaction pushes the interaction bands to around $\omega=\pm U$, where $U$ is the interaction strength, and there are zeros of the Green's function follows the free fermion dispersion, but the bandwidth scales with $U$ as $1/U^2$. To summarize our finding, the Green's function in the SMG phase can be written as,
%\begin{equation}
%    G_{\smg}(\omega,\bk) = \frac{\omega+\alpha \epsilon_{\bk}/U^2}{(\omega-\epsilon_\bk/2)^2-U^2}
%\end{equation}
%where $\epsilon_\bk$ is the energy dispersion of the free fermion and $\alpha$ is an order 1 number depending on the other details of the model. Note that this Green's function is quite different from the Green's function in the usual symmetry breaking phase $G_{\ssb}=(\omega - \epsilon_\bk -\Delta)^{-1}$, where $\Delta$ is the symmetry breaking gap. One salient feature of $G_\smg$ is that it has zeros at $-\alpha \epsilon_\bk/U^2 $, while $G_\ssb$ doesn't, and we argue this is the essential feature of the SMG phase. 

Many previous works \cite{Altshuler1998Lcond-mat/9703120, Oshikawacond-mat/0002392, 2003LuttingerSurface, Gnezdilov2022S2111.09906} suggest that the Luttinger theorem \cite{LuttingerRP1960} will not be violated in the presence of the interaction that preserves the translation and charge conservation symmetry. However, quasi-particles (poles of Green's function) may not exist in the strongly correlated systems, the Fermi surface is instead defined by the surface of Green's function zeros at zero frequency, i.e.,~$G(0,\bk)=0$, and the Green's function changes sign on the two sides of the \emph{zero Fermi surface}, or the so-called Luttinger surface \cite{Stanescu2007Tcond-mat/0602280, 2003LuttingerSurface, Senthil2007.07896, Sachdev1711.09925, xiang2022d}. This can be regarded as the remnant of the conventional Fermi surface in the strongly interacting gapped phase. Our analysis shows that the volume enclosed by the zeros of the Green's function in the SMG phase is the same as the Fermi volume in the Fermi liquid phase, which agrees with the Luttinger theorem.

The paper will be structured as follows. We start by introducing a concrete lattice model for Fermi surface SMG in \secref{sec: model} and briefly discussing its phase diagram. We give theoretical arguments for Green's function zeros in the SMG phase from the Luttinger theorem in \secref{sec: Luttinger} (general), and the particle-hole symmetry in \secref{sec: symmetry} (specific). We provide numerical and field theoretical evidence of Green's function zeros from both the strong coupling \secref{sec: strong} and the weak coupling \secref{sec: weak} perspectives. We comment on the robustness of probing the zero structure in spectroscopy experiments in \secref{sec: probe}. We conclude in \secref{sec: summary} with a discussion of the relevance of our model to the nickelate superconductor La$_3$Ni$_2$O$_7$.

%% Figure environment removed


\section{Argument For Green's Function Zeros}

\subsection{Lattice Model and Phase Diagram}\label{sec: model}

As a specific example of Fermi surface SMG, we consider a bilayer square lattice \cite{Zhai2009A0905.1711, Ruger2014T1311.6504, Rhim2019C1808.05926} model of spin-1/2 fermions, as illustrated in \figref{fig: lattice}(a). Let $c_{il\sigma}$ be the fermion annihilation operator on site-$i$ layer-$l$ ($l=1,2$) and spin-$\sigma$ ($\sigma=\uparrow,\downarrow$). The model is described by the following Hamiltonian
\begin{equation}\label{eq: H}
H = -t\sum_{\langle ij \rangle,l,\sigma} (c_{il\sigma}^\dagger c_{jl\sigma}+ \text{h.c.})+ J \sum_{i} \vect{S}_{i1}\cdot \vect{S}_{i2},
\end{equation}
where $\vect{S}_{il}:=\tfrac{1}{2}c_{il\sigma}^\dagger\vect{\sigma}_{\sigma\sigma'}c_{il\sigma'}$ denotes the spin operator with $\vect{\sigma}:=(\sigma^1,\sigma^2,\sigma^3)$ being the Pauli matrices. The Hamiltonian $H$ contains a nearest-neighbor hopping $t$ of the fermions within each layer and an inter-layer Heisenberg spin-spin interaction with antiferromagnetic coupling $J>0$. The Heisenberg interaction should be understood as a four-fermion interaction, that there is no explicitly formed local moment degrees of freedom. Unlike the standard $t$-$J$ model \cite{Chao1978tJ}, we do \emph{not} impose any on-site single-occupancy constraint \cite{Gutzwiller1964E} here. We assume that the fermions are half-filled in each layer. 

% Figure environment removed

In the non-interacting limit ($J/t\to 0$), the ground state of the tight-binding Hamiltonian in \eqnref{eq: H} is a Fermi liquid with a four-fold degenerated (two layers and two spins) square-shaped Fermi surface in the Brillouin zone, as shown in \figref{fig: lattice}(b). The fermion system is gapless in this limit. However, given that the fermion carries one unit charge under the $\U(1)$ symmetry, the Fermi surface anomaly vanishes due to \cite{Shinsei2017FSLSM,yzy2023FSanomaly} 
\begin{equation}
    \sum_{a=1}^{4} q_a \nu_a = 4\times 1 \times \frac{1}{2}=0 \mod 1,
\end{equation}
where $a$ indexes the four-fold degenerated Fermi surface with $q_a=1$ being the $\U(1)$ charge carried by the fermion and $\nu_a=1/2$ being the filling fraction. This implies there must be a way to gap out the Fermi surface into a trivial insulator while preserving both the translation and the $\U(1)$ charge conservation symmetries. Nevertheless, these symmetry requirements are restrictive enough to rule out all possible fermion bilinear gapping mechanisms, leaving Fermi surface SMG the only available option.

One possible SMG gapping interaction is the interlayer Heisenberg spin-spin interaction $J$ in \eqnref{eq: H}. In the strong interaction limit ($J/t\to\infty$), the system has a unique ground state, given by
\eq{\label{eq: SMG GS}\ket{0}=\bigotimes_i (c_{i1\uparrow}^\dagger c_{i2\downarrow}^\dagger - c_{i1\downarrow}^\dagger c_{i2\uparrow}^\dagger)\ket{\text{vac}},}
which is a direct product of the inter-layer spin-singlet state on every site. $\ket{\text{vac}}$ stands for the vacuum state of fermions (i.e.~$c_{il\sigma}\ket{\text{vac}}=0$). The SMG ground state $\ket{0}$ does not break any symmetry and does not have topological order. All excitations are gapped by an energy of the order $J$ from the ground state. Any local perturbation far below the energy scale $J$ can not close this excitation gap, so the SMG phase is expected to be stable in a large parameter regime as long as $J\gg t$.



Given the distinct ground states in the two limits of $J/t$, we anticipate at least one quantum phase transition separating the Fermi liquid and the SMG insulator. However, due to the perfect nesting of the Fermi surface, the Fermi liquid state is unstable towards spontaneous symmetry breaking (SSB) upon infinitesimal interaction, so a more plausible phase diagram should look like \figref{fig: lattice}(c), where an intermediate SSB phase sets in. A mean-field analysis based on the Fermi liquid fixed point shows that there are two degenerated leading instabilities: (i) the inter-layer exciton condensation (EC) and (ii) the inter-layer superconductivity (SC). They are respectively described by the following order parameters
\eq{\label{eq: def phi}
\phi_\text{EC}=\sum_{i,\sigma}(-)^{i} c_{i1\sigma}^\dagger c_{i2\sigma},\quad \phi_\text{SC}=\sum_{i,\sigma}(-)^{\sigma} c_{i1\sigma}^\dagger c_{i2\bar{\sigma}}^\dagger.}
Here, $(-)^i$ denotes the stagger sign on the square lattice of lattice momentum $(\pi,\pi)$. $(-)^\sigma=+1$ for $\sigma=\uparrow$ and $-1$ for $\sigma=\downarrow$. $\bar{\sigma}$ stands for the opposite spin of $\sigma$. 

The energetic degeneracy of these two SSB orders can be explained by the fact that their order parameters $\phi_\text{EC}$ and $\phi_\text{SC}$ are related by a particle-hole transformation $c_{i2\sigma}\to(-)^{i}(-)^{\sigma}c_{i2\bar{\sigma}}^\dagger$ in the second layer only, which is a symmetry of the model Hamiltonian in \eqnref{eq: H}. The EC $\langle\phi_\text{EC}\rangle\neq 0$ spontaneously breaks the translation and interlayer $\U(1)$ symmetry, and the SC $\langle\phi_\text{SC}\rangle\neq 0$ spontaneously breaks the total $\U(1)$ symmetry. Both of them gap out the Fermi surfaces fully, leading to an SSB insulator (or superconductor). The SSB and SMG phases are likely separated by an XY transition, at which the symmetry gets restored. We will leave the numerical verification of the proposed phase diagram \figref{fig: lattice}(c) for future study, as the main focus of this research is to investigate the structure of fermion Green's function in the SMG insulating phase.

We note that the model \eqnref{eq: H} was also introduced as the ``coupled ancilla qubit'' model to describe the pseudo-gap physics in the recent literature \cite{yahui2020-1,yahui2020-2,yahui2021}. Its honeycomb lattice version has been investigated in recent numerical simulations \cite{YZY2022nuSMG}, where a direct quantum phase transition between semimetal and insulator phases was observed.

\subsection{Luttinger Theorem and Green's Function Zeros}
\label{sec: Luttinger}

The Luttinger theorem \cite{Luttinger1960G, LuttingerRP1960} asserts that in a fermion many-body system with lattice translation and charge $\U(1)$ symmetries, the ground state charge density $\langle N\rangle/V$ (i.e.,~the $\U(1)$ charge per unit cell) is tied to the momentum space volume in which the real part of the zero-frequency fermion Green's function is positive $\Re G(0,\bk)>0$. This can be formally expressed as
\eq{\label{eq: Luttinger}
\frac{\langle N\rangle}{V}=N_f\int_{\Re G(0,\bk)>0}\frac{\dd^2\bk}{(2\pi)^2}.}
Here, the $\U(1)$ symmetry generator $N=\sum_{i,l,\sigma}c_{il\sigma}^\dagger c_{il\sigma}$ measures the total charge, and the volume $V=\sum_{i}1$ is defined as the number of unit cells in the lattice system. $N_f=4$ counts the fermion flavor number (or the Fermi surface degeneracy), including two layers and two spins. The Green's function $G(\omega,\bk)$ in \eqnref{eq: Luttinger} is defined by the fermion two-point correlation as 
\eq{\label{eq: def G}
\langle c_{l\sigma}(\omega,\bk)c_{l'\sigma'}(\omega,\bk)^\dagger\rangle=G(\omega,\bk)\delta_{ll'}\delta_{\sigma\sigma'}.}
The correlation function is proportional to an identity matrix in the flavor (layer-spin) space because of the layer $\U(1):c_{l\sigma}\to \e^{(-)^l\ii\theta}c_{l\sigma}$, the layer interchange $\dsZ_2:c_{1\sigma}\leftrightarrow c_{2\sigma}$, and the spin $\SU(2):c_{l\sigma}\to (\e^{\ii\vect{\theta}\cdot\vect{\sigma}/2})_{\sigma\sigma'}c_{l\sigma'}$ symmetries.


The Luttinger theorem applies to the Fermi liquid and SMG states in the bilayer square lattice model \eqnref{eq: H}, as both states preserve the translation and charge $\U(1)$ symmetries. Given that the fermions are half-filled ($\nu=1/2$) in the system, the Fermi volume should be
\eq{\int_{\Re G(0,\bk)>0}\frac{\dd^2\bk}{(2\pi)^2}=\frac{\langle N \rangle}{V N_f}=\nu=\frac{1}{2}.} 
The Fermi volume is enclosed by the Fermi surface, across which $\Re G(0,\bk)$ changes sign. The sign change can be achieved either by poles or zeros in the Green's function.

In the Fermi liquid state, the required Fermi volume is satisfied via Green's function poles along the Fermi surface, as pictured in \figref{fig: lattice}(b). However, the SMG insulator is a fully gapped state of fermions that has no low-energy quasi-particles (below the energy scale $J$). Consequently, the Green's function $G(\omega,\bk)$ cannot develop poles at $\omega=0$, meaning the required Fermi volume can only be satisfied by Green's function zeros. Therefore, the Lutinger theorem implies that there must be robust Green's function zeros at low energy in the SMG phase, and the zero Fermi surface must enclose half of the Brillouin zone volume in place of the original pole Fermi surface.

It is known that the Luttinger theorem can be violated in the presence of topological order \cite{Senthil2003Topo, Senthilcond-mat/0305193, Paramekanticond-mat/0406619, Senthil2006Topo, Senthil2012Topo, Meng2016Topo, Sachdev1711.09925, Sachdev2019T1801.01125, Bultinck1808.00324, Skolimowski2022L2202.00426}. However, this concern does not affect our discussion in the SMG phase, because the SMG insulator is a trivial insulator without topological order.

\subsection{Particle-Hole Symmetry and Zero Fermi Surface}
\label{sec: symmetry}

The Luttinger theorem only constrains the Fermi volume but does not impose requirements on the shape of the Fermi surface. However, in this particular example of the bilayer square lattice model \eqnref{eq: H}, the system has sufficient symmetries to determine even the shape of the Fermi surface. 

The key symmetry here is a particle-hole symmetry $\dsZ_2^C$, which acts as
\eq{\label{eq: Z2C}
c_{il\sigma}\to(-)^i(-)^\sigma c_{il\bar{\sigma}}^\dagger.}
The Hamiltonian $H$ in \eqnref{eq: H} is invariant under this transformation. Since the Green's function is an identity matrix in the flavor space \eqnref{eq: def G} which is invariant under any flavor basis transformation, we can omit the flavor indices and focus on the frequency-momentum dependence of the Green's function, written as
\eq{G(\omega,\vect{k})=\sum_{t,\vect{x},t',\vect{x}'}\langle c(t,\vect{x})c(t',\vect{x}')^\dagger\rangle \e^{\ii(\omega (t-t')-\vect{k}\cdot(\vect{x}-\vect{x}'))}.}
Given \eqnref{eq: Z2C}, the fermion field $c(t,\vect{x})$  transforms under the $\dsZ_2^C$ symmetry as 
\eq{c(t,\vect{x})\to c(t,\vect{x})^\dagger \e^{\ii \vect{Q}\cdot\vect{x}},\quad c(t,\vect{x})^\dagger \to c(t,\vect{x}) \e^{-\ii \vect{Q}\cdot\vect{x}},} 
where $\vect{Q}=(\pi,\pi)$ is the momentum associated with the stagger sign factor $(-)^i$ on the square lattice. As a consequence, the Green's function transforms as
\eq{G(\omega,\vect{k})\to -G(-\omega,\vect{Q}-\vect{k}).
}
Furthermore, there are also two diagonal reflection symmetries on the square lattice, which maps $\bk=(k_x,k_y)$ to $(k_y,k_x)$ or $(-k_y,-k_x)$ in the momentum space.

Both the Fermi liquid and the SMG states preserve the particle-hole symmetry $\dsZ_2^C$ and the lattice reflection symmetry, which requires the Green's function to be invariant under the combined symmetry transformations. So the zero-frequency Green's function must satisfy
\eq{G(0,k_x,k_y)=-G(0,\pi\pm k_y,\pi\pm k_x),}
meaning that the sign change of $G(0,\bk)$ should happen along $k_x\pm k_y=\pi\mod 2\pi$, which precisely describes the shape of the Fermi surface. The Fermi surface is pole-like in the Fermi liquid state and becomes zero-like in the SMG state, but its shape and volume remain the same.

However, it should be noted that the precise overlap of the zero Fermi surface in the SMG insulator and the pole Fermi surface in the Fermi liquid is a fine-tuned feature of the bilayer square lattice model \eqnref{eq: H}. In more general cases, such as including further neighbor hopping in the model, the particle-hole symmetry would cease to exist, thus the invariance in the shape of the Fermi surface is no longer guaranteed. Nevertheless, the Luttinger theorem can still ensure the invariance in the Fermi volume, thereby providing the SMG insulator with robust Green's function zeros.

To verify this proposition, we will analyze the behavior of the Green's function in the SMG phase from both strong and weak coupling perspectives in \secref{sec: evidence}. Our calculations suggest that, for this specific model, the SMG state indeed possesses a Fermi surface (of Green's function zeros) that is identical in shape to that in the Fermi liquid state.


%Each complex fermion has charge 1 under the charge $\U(1)$ symmetry, and the Fermi surface anomaly vanishes due to \cite{Shinsei2017FSLSM,yzy2023FSanomaly} 
%\begin{equation}
%    \sum_{i=1}^{4} q_i \nu_i = 4\times 1 \times \frac{1}{2}=0 \mod 1.
%\end{equation}
%We can further consider the interaction that preserve $\LU(1)$ symmetry, 
%\begin{equation}
%    H_J = J \sum_{i} \vect{S}_{i1}\cdot \vect{S}_{i2}
%\end{equation}
%where $S_{il}^a = c_{il\sigma_1}^\dagger \sigma^a_{\sigma_1,\sigma_2} c_{il\sigma_2}$ and $\sigma^a$ are the Pauli matrices. In the strong interaction limit $J/t \gg 1$, the system $H_0+H_J$ has an unique ground state $\bigotimes_i (c_{i1\uparrow}^\dagger c_{i2\downarrow}^\dagger - c_{i1\downarrow}^\dagger c_{i2\uparrow}^\dagger)\ket{0} $. Similar model on the honeycomb lattice has been studied in \cite{YZY2018SMGDQCP-1,YZY2018SMGDQCP-2,YZY2022nuSMG}, the fermion density is zero in that case.


%If only preserving the $\LZ{4}$ symmetry, the system can also be symmetrically gapped by the interaction,
%\begin{equation}
%    H_{4e} = g \sum_{i} c_{i1\uparrow}c_{i1\downarrow}c_{i2\uparrow}c_{i2\downarrow} +h.c.
%\end{equation}
%In the $g/t \gg 1$ limit, the unique ground state of $H_0+H_{4e}$ is $\bigotimes_i (1- c_{i1\uparrow}^\dagger c_{i1\downarrow}^\dagger c_{i2\uparrow}^\dagger c_{i2\downarrow}^\dagger )\ket{0}$.


\section{Evidence of Green's Function Zeros}\label{sec: evidence}

\subsection{Strong Coupling Analysis}\label{sec: strong}

We will first focus on the strong interaction limit $(J/t\to\infty)$, where the system is deep in the SMG phase and the exact ground state is known (see \eqnref{eq: SMG GS}). We start from this limit and turn on the hopping term as a perturbation. We employ exact diagonalization and cluster perturbation theory (CPT) \cite{CPT2000,senechal2002cluster} to compute the Green's function in the SMG phase. The details of our method are described in Appendix~\ref{append:cpt}. It is valid to use a small cluster to reconstruct the Green's function in the SMG phase since the ground state is close to a product state that does not have long-range correlation or long-range quantum entanglement. This is quite different from the Hubbard model, where the Fermi surface anomaly is non-vanishing, and the infrared phase must be either SSB order or topological order \cite{Senthil2003Topo, Senthil2006Topo, Senthil2012Topo, Meng2016Topo}. In either case, the ground state wave functions cannot be reconstructed from the small clusters due to the long-range correlation/entanglement. This argument has been noted in the original paper on the CPT method \cite{CPT2000}.


% Figure environment removed

To be specific, we first partition the square lattice (including both layers) into $2\times 2$ square clusters as shown in \figref{fig: cluster}. Let us first ignore the inter-cluster hopping. Within each cluster, we represent the Hamiltonian in the many-body Hilbert space and use the Lanczos method to obtain the lowest $\sim 2000$ eigenvalues and eigenvectors. The Green's function in the cluster can then be obtained by the K\"all\'en–Lehmann representation
\begin{equation}
    G_0(\omega)_{ij} = \sum_{m>0} \frac{\bra{0}c_{i}\ket{m}\bra{m} c_{j}^\dagger\ket{0}}{\omega -(E_m-E_0)}+  \frac{\bra{m}c_{i}\ket{0}\bra{0} c_{j}^\dagger\ket{m}}{\omega +(E_m-E_0)},
\end{equation}
where $\ket{m}$ is the $m$th excited state with energy $E_m$, and $\ket{0}$ is the ground state with energy $E_0$, whose wave function was previously given in \eqnref{eq: SMG GS}. Since the four fermion flavors (two spins and two layers) are identical under the internal flavor symmetry, we can drop the flavor index in the Green's function and only focus on one particular flavor with the site indices $i,j$, where $i,j=1,2,3,4$ as indicated in \figref{fig: cluster}. %We verified the convergence of the Green's function with more eigen pairs.
The convergence of the Green's function can be verified by including more eigenstates from the Lanczos method. We checked that increasing the number of eigenpairs to $\sim 8000$ will not change the result significantly, indicating that the result with $\sim 2000$ eigenpairs has already converged.

%% Figure environment removed



Now we restore the inter-cluster hoping to extend the Green's function from small clusters to the infinite lattice. The Green's function of super-lattice momentum $\bk$ can be obtained from the random phase approximation (RPA) approach \cite{CPT2000},
\eq{G(\omega,\bk)_{ij} = \left(\frac{G_0(\omega)}{1-T(\bk)G_0(\omega)}\right)_{ij},}
where the $T(\bk)$ matrix
\begin{equation}
\label{eq-V}
T(\bk) =-t\begin{pmatrix}
0 &  \e^{-\ii 2 k_x} & 0 & \e^{\ii 2 k_y}\\
\e^{\ii 2 k_x} &0 & \e^{\ii 2 k_y} & 0\\
0 &  \e^{-\ii 2 k_y} & 0&  \e^{\ii 2 k_x}\\
 \e^{-\ii 2 k_y} & 0 &  \e^{-\ii 2 k_x} & 0
\end{pmatrix}
\end{equation}
describes the inter-cluster fermion hopping. The resulting Green's function $G(\omega,\bk)_{ij}$ is defined in the folded Brillouin zone $\bk\in(-\pi/2,\pi/2]^{\times 2}$ with sub-lattice indices $i,j$. To unfold the Green's function to the original Brillouin zone $\bk\in(-\pi,\pi]^{\times 2}$, we perform the following (partial) Fourier transform
\begin{equation}\label{eq: GSMG numerics}
    G(\omega,\bk) = \frac{1}{L}\sum_{i,j}\e^{-\ii \bk\cdot (\br_i-\br_j)}G(\omega,\bk)_{ij}.
\end{equation}

%\yzy{YZY: This paragraph is repetitive ---}
%We argue that this method is suitable to the SMG problem, because the unique ground state of the SMG phase is a tensor product state, which can be easily obtained by exact diagonalizing the small clusters and extending the cluster wave function to the whole space. This is not true, for example, for 2d Hubbard model at half-filling, where the ground state of a small cluster is not able to faithfully represent the anti-ferromagnetic order of the true ground state in the thermodynamic limit.%the true ground state of the large system,
%Therefore, it is no longer valid to use the cluster wave function to construct the wave function for the whole space.


% Figure environment removed

%\dcl{figures of numerical results, continuum?}
We numerically calculated the unfolded Green's function $G(\omega,\bk)$ using the above-mentioned cluster perturbation method. We take a large interaction strength $J/t=8$ deep in the SMG phase and present the resulting Green's function in \figref{fig: Gdeep}. From \figref{fig: Gdeep}(a), the poles of the Green's function form two dispersing bands around $\omega=\pm J$, which resembles the upper and lower Hubbard bands in the Hubbard model. This indicates the quasi-particles are fully gapped in the SMG phase. Meanwhile, from \figref{fig: Gdeep}(b,c), the zeros of the Green's function appear around $\omega= -\alpha \epsilon_\bk/J^2$ with some non-universal but positive coefficient $\alpha>0$. We find that the ``dispersion'' of zeros is reversed compared to the original band dispersion $\epsilon_\bk$. In \figref{fig: wzero}, we also numerically confirmed that the ``bandwidth'' $w_\text{zero}$ of zeros is suppressed by the interaction $J$ as $w_\text{zero}\sim J^{-2}$ as $J\to\infty$.

% Figure environment removed

Building upon the above observation of the poles and zeros of the Green's function, we put forth the following empirical formula:
\begin{equation}\label{eq: GSMG deep}
G_{\smg}(\omega,\bk) = \frac{\omega+\alpha \epsilon_{\bk}/J^2}{(\omega-\epsilon_\bk/2)^2-J^2},
\end{equation}
as an approximate description of our numerical result \eqnref{eq: GSMG numerics}. An important aspect of this formula is the positioning of the Green's function zeros precisely around the initial Fermi surface (where $\epsilon_\bk=0$) at $\omega=0$. This is indicated by the small arrows in \figref{fig: Gdeep}(c). 

Assuming $\Re G_\smg(0,\bk)=0$ as the definition of the zero Fermi surface in the SMG phase, it would encompass the same Fermi volume as the pole Fermi surface in the Fermi liquid phase. As both translation and charge conservation symmetries remain unbroken in the SMG phase, the Luttinger theorem mandates the preservation of the Fermi volume. Given that the SMG state is a fully gapped trivial insulator, there is no pole (no quasi-particle) at low energy, thus the Green's function can only rely on zeros to fulfill the Fermi volume required by the Luttinger theorem, which is explicitly demonstrated by \eqnref{eq: GSMG deep}.


%One can understand the $G_{\smg}$ via the self-energy correction to the free fermion Green's function $G_{\smg} = (\omega - \epsilon_\bk -\Sigma(\omega,\bk) )^{-1}$. To the leading order in $1/U$, the self-energy correction is $\Sigma(\omega,\bk) = \frac{U^2}{\omega}-\frac{\alpha \epsilon_\bk}{\omega^2}+\mathcal{O}(\frac{1}{U})$. The first term in the self-energy correction comes from the interaction effect as shown in \figref{fig:bubblediag} (a). However, the second term is more mysterious, since it does not depend on the interaction strength $U$. The dependence on the $\epsilon_\bk$ and $\omega^{-2}$ suggests the correction comes from the interaction with massive boson as shown in \figref{fig:bubblediag} (b), $\sum_{\nu_n=\frac{2\pi n}{\beta}}\frac{1}{\ii \omega_n +\ii \nu_n -\epsilon_\bk}\frac{1}{(\ii \nu_n)^2-m^2}\simeq -\frac{\epsilon_\bk}{\omega^2 m^2} +\mathcal{O}(m)$, where $m$ is the mass of the boson. \dcl{why massive boson.}
%% Figure environment removed


\subsection{Weak Coupling Analysis}\label{sec: weak}

Nevertheless, SMG is not the sole mechanism for gapping out the Fermi surface. SSB might also open a full gap on the Fermi surface, which corresponds to the Higgs mechanism for fermion mass generation. Specifically, in the bilayer square lattice model \eqnref{eq: H}, due to the perfect nesting of the Fermi surface, the Fermi liquid exhibits strong instability toward SSB orders. Without loss of generality, we will focus on the inter-layer exciton condensation in the weak coupling limit. The corresponding order parameter $\phi_\text{EC}$ was introduced in \eqnref{eq: def phi}, which carries momentum $\vect{Q}=(\pi,\pi)$. The exciton condensation leads to an SSB insulating phase, as noted in the phase diagram \figref{fig: lattice}(c). However, there are significant differences between the SMG insulator and the SSB insulator, especially in terms of the structure of Green's function zeros.

% Figure environment removed


In the SSB insulator phase, the Brillouin zone folds by the nesting vector $\vect{Q}=(\pi,\pi)$. The fermion Green's function can be written in the $(c_{\bk},c_{\bk+\vect{Q}})^\intercal$ basis (omitting layers and spins freedom) as
\eq{\label{eq: GSSB}
G_\ssb(\omega, \bk)=\frac{\omega \sigma^0+\epsilon_\bk \sigma^3+ \Re\Delta\sigma^1+\Im\Delta\sigma^2}{\omega^2-\epsilon_\bk^2-|\Delta|^2},}
where $\Delta= J \langle \phi_\text{EC}\rangle$ denotes the exciton gap induced by the exciton condensation $\langle \phi_\text{EC} \rangle\neq 0$. The properties of $G_\ssb$ are illustrated in \figref{fig: Gssb}. The spectral function in \figref{fig: Gssb}(a) depicts the quasi-particle peak along the band dispersion, reflecting a gapped (insulating) band structure. 

Since $G_\ssb$ is a matrix, its zero structure should be defined by its determinant being zero, i.e.,~$\det G_\ssb(\omega,\bk)=0$, which is the only way to define the zero structure in a basis independent manner. \figref{fig: Gssb}(b) indicates the determinant of $G_\ssb$ remains the same sign within the band gap induced by the exciton condensation. Since $G_\ssb$ does not preserve the translation symmetry (as $\Delta\to-\Delta$ is translation-odd), and $\Delta$ is non-zero, $\det G_\ssb$ does not have zeros crossing $\omega=0$ at the original Fermi surface. These two observations are linked: the absence of translation symmetry makes the Luttinger theorem ineffective, hence there is no expectation for the zero Fermi surface in the SSB insulator.

As the interaction $J$ intensifies, the SSB insulator ultimately transitions into the SMG insulator, as depicted in the phase diagram \figref{fig: lattice}(c). During this transition, the broken symmetry is restored, yet the fermion excitation gap remains intact, similar to the pseudo-gap phenomenon seen in correlated materials \cite{Lee2004Dcond-mat/0410445, Keimer2014H1409.4673}. In the context of modeling fermion spectral functions, the pseudo-gap phenomenon can be interpreted as a consequence of the phase (or orientation) fluctuations of fermion bilinear order parameters \cite{Franz1998Pcond-mat/9805401, Kwon1999Econd-mat/9809225, Kwon2001Ocond-mat/0006290, Franz2001Acond-mat/0012445, Curty2003Tcond-mat/0401124, Li2018P1805.05530, Li2018W1803.08226, Ye2019H1905.11412}. In this picture, the order parameter $\Delta=\Delta_0\e^{\ii\theta}$ maintains a finite amplitude $\Delta_0$ as we enter the SMG phase from the adjacent SSB phase, but its phase $\theta$ is disordered by long-wavelength random fluctuations. Consequently, on the large scale, $\Delta$ cannot condense to form long-range order; but on a smaller scale, $\Delta_0$ still provides a local excitation gap everywhere for fermions.

Based on this picture of the SMG state, the simplest treatment is to focus on the long wavelength fluctuation of $\Delta$ and estimate its self-energy correction for the fermion by
\eq{
\Sigma(\omega,\bk)=
\raisebox{-3pt}{% Figure removed}=\E_\Delta\hat{\Delta}^\dagger G_0(\omega,\bk)\hat{\Delta}=\frac{\Delta_0^2}{\omega\sigma^0+\epsilon_\bk\sigma^3},}
where the vertex operator is $\hat{\Delta}:=\Re \Delta\sigma^1+\Im\Delta\sigma^2$ and the bare Green's function is $G_0(\omega,\bk)=(\omega\sigma^0-\epsilon_\bk\sigma^3)^{-1}$. Here we have assumed that the correlation length $\xi$ of the bosonic field $\Delta$ is long enough that its momentum is negligible for fermions. This assumption is valid near the transition to the SSB phase, as the correlation length diverges ($\xi\to\infty$) at the transition. 

Using this self-energy to correct the bare Green's function, we obtain
\eqs{G(\omega,\bk)&=(G_0(\omega,\bk)^{-1}-\Sigma(\omega,\bk))^{-1}\\
&=\frac{\omega\sigma^0+\epsilon_\bk\sigma^3}{\omega^2-\epsilon_\bk^2-\Delta_0^2}.}
Since the translation symmetry has been restored in the SMG phase, we can unfold the Green's function back to the original Brillouin zone (by taking the $G(\omega,\bk)_{11}$ component), which leads to a weak coupling description of the Green's function in the shallow SMG phase near the transition to the SSB phase
\eq{\label{eq: GSMG shallow}
G'_\smg(\omega,\bk)=\frac{\omega+\epsilon_\bk}{\omega^2-\epsilon_\bk^2-\Delta_0^2}.}
A more rigorous treatment of a similar problem can be found in \refcite{Wang2023P2212.05737}, which includes finite momentum fluctuations of $\Delta$. The major effect of these fluctuations is to introduce a spectral broadening for the fermion Green's function as if replacing $\omega\to\omega+\ii\delta$ in \eqnref{eq: GSMG shallow}. It was also found that the broadening $\delta\sim\xi^{-1}$  scales inversely with the correlation length $\xi$ of the order parameter, which justifies our simple treatment in the large-$\xi$ regime. Similar Green's functions as \eqnref{eq: GSMG shallow} was previously constructed to describe non-Fermi liquid  \cite{2003LuttingerSurface} statisfying the Luttinger theorem. However, its physical meaning is now clarified as Green's function in the SMG phase.

% Figure environment removed


The features of $G'_\smg$ in \eqnref{eq: GSMG shallow} are presented in \figref{fig: Gshallow}. When comparing \figref{fig: Gshallow}(a) and \figref{fig: Gssb}(a), we can observe that the pole structure of $G'_\smg$ is identical to that of $G_\ssb$ (in the diagonal component), both showcasing a gapped spectrum. However, they significantly differ in their zero structures, as seen by comparing \figref{fig: Gshallow}(b) and \figref{fig: Gssb}(b). Due to the restoration of symmetry, the low-energy zeros reemerge in the Green's function in the SMG phase. Additionally, its zero Fermi surface perfectly aligns with the original pole Fermi surface, fulfilling the Luttinger theorem's requirement for the Fermi volume.


Comparing the Green's function in the SMG phase derived from the strong coupling analysis \eqnref{eq: GSMG deep} and the weak coupling analysis \eqnref{eq: GSMG shallow} (see also \figref{fig: Gdeep} and \figref{fig: Gshallow}), we find that despite the apparent difference in high-energy spectral features, the zero Fermi surface defined by $G(0,\bk)=0$ remains a resilient low-energy feature. The persistent zero Fermi surface in the SMG phase is a consequence of the Luttinger theorem. 

Nonetheless, besides the low-energy zero structure, it is also intriguing to understand how the high-energy spectral feature deforms from the weak coupling case to the strong coupling case. However, this problem requires non-perturbative numerical simulations. Fortunately, the bilayer square lattice model \eqnref{eq: H} admits a sign-problem-free \cite{Troyer2005Ccond-mat/0408370} quantum Monte Carlo \cite{Fucito1981A, Scalapino1981M, Blankenbecler1981M, Hirsch1981E, Hirsch1985T} simulation. We will leave this interesting direction for future research.

\section{Probing Green's Function Zeros}\label{sec: probe}

While Green's function zeros are an important feature of the SMG insulator, they are not directly observable in experiments. Spectroscopy experiments, such as angle-resolved photoemission spectroscopy (ARPES), can directly probe the fermion's spectral function $A(\omega,\bk)=-2\Im G(\omega+\ii 0_+,\bk)$, which is the imaginary part of Green's function. By employing the Kramers-Kronig (KK) relation to recover the real part of Green's function from the spectral function, 
\eq{\Re G(\omega,\bk) = \frac{1}{2\pi}\scP\int d\omega' \frac{A(\omega',\bk)}{\omega'-\omega},}
we can indirectly study the zero structure of the Green's function.

% Figure environment removed

However, the spectral function might be broadened in experimental data due to noise or dissipation. We are interested in studying how sensitive the reconstructed Green's function zero is to these disturbances, in order to understand the stability of the method. Following \secref{sec: strong}, we start from the strong coupling limit and use the CPT approach to calculate Green's function. To account for the spectral broadening effect, we replace $\omega$ with $\omega +\ii \delta$, where $\delta$ is relatively large, say, about the order of the hopping $t$. Based on the broadened spectral function in \figref{fig: Gbroad}(a), we use the KK relation to reconstruct the real part, as shown in \figref{fig: Gbroad}(b). We find that the zero Fermi surface maintains the same shape, but the zero ``dispersion'' bandwidth gets larger.

The increase in bandwidth can be understood by taking the SMG Green's function $G_\text{SMG}(\omega,\bk)$ in \eqnref{eq: GSMG deep}, and solving for its zeros $\Re G(\omega+\ii\delta,\bk)=0$. To the leading order of $1/J$ and $\delta$, the solution is given by
\eq{\omega(\bk)=-\Big(1+\frac{\delta^2}{\alpha}\Big)\frac{\alpha\epsilon_\bk}{J^2}+\cdots,}
meaning that the bandwidth of Green's function zero dispersion will increase by $\delta^2/\alpha$, but the corresponding Luttinger surface remains unchanged. Therefore, the Green's function zero in the SMG phase is a robust feature that can be potentially identified from spectroscopy measurements, even in the presence of noises or dissipations. 

%\dcl{1. Luttinger theorem, 2. SMG by disorder bosons, 3. robustness of zero. }
%% Figure environment removed
%
%\subsection{Green's function in SMG phase and spontaneously symmetry breaking phase}
%
%The SMG Green's function in \eqnref{eq:gfsmg} can be zero when $\omega=-\alpha \epsilon_\bk /U^2$, this gives the dispersion of the Green's function zero, or zero dispersion, denote as $E^\zero(\bk)=-\alpha \epsilon_\bk /U^2$. In particular, $E^{\zero}(\bk)=0$ or $\Re G_\smg(0,\bk)=0$ defines the zero Fermi surface in the SMG phase. The zero Fermi surface encloses the same Fermi volume as that in the Fermi liquid phase, with the dispersion of the quasiparticle being $E^{\qp}(\bk)=\epsilon_\bk$. This agrees with the Luttinger theorem, since the translation and charge conservation symmetry are unbroken in the SMG phase, the Fermi surface anomaly constrains the Fermi volume to be invariant in all the low energy phases, otherwise, the symmetry will be spontaneously broken.
%
%On the contrary, in the usual scenario where the fermions acquire mass due to the Higgs mechanism with spontaneously symmetry breaking, the Green's function $G_\ssb$ won't have zeros at the Fermi surface in the Fermi liquid phase. Since the symmetry breaking order parameter necessarily appears in the numerator of the $G_\ssb$ and the Green's function $G_\ssb$ is no longer invariant under the symmetry transformation. For example, free spinful fermion on the square lattice with the antiferromagnetic order parameter $\Delta$, the Brillouin zone shrinks, the SSB Green's function in the basis $(c_\bk,c_{\bk+(\pi,\pi)})^\intercal$ is $G_\ssb(\omega,\bk)=\frac{\omega \sigma^0 +\epsilon_\bk \sigma^3 -\Delta \sigma^1}{\omega^2-\epsilon_\bk^2-\Delta^2}$, whose determinant cannot be $0$ (see Fig.~\ref{fig:Gssb} in Appendix~\ref{append:4e-and-ssb} for details).
%
%Previous attempt to understand the Luttinger theorem in the non-Fermi liquid took an artificial Green's function $\tilde{G}(\omega,\bk)=\frac{\omega \sigma^0 +\epsilon_\bk \sigma^3 }{\omega^2-\epsilon_\bk^2-\Delta^2}$ \cite{2003LuttingerSurface}. This Green's function $\Tilde{G}$ is similar to $G_\ssb$ but with $\Delta=0,\Delta^2\ne 0$. The $\Tilde{G}$ certainly has zeros at the Fermi surface of Fermi liquid phase. However, the physical meaning of this $\Tilde{G}$ can now be understood as a Green's function in the SMG phase, and the detail goes as follows, the fermions couple to the bosons in the high energy scale, and the bosons open a large gap $\Delta$ in the low energy, this also gives the mass to fermions. To the end, the massive bosons enter the disordered phase with vanishing expectation value $\Delta = 0$, but the massive fluctuation $\Delta^2$ can still be seen by the fermions \cite{YZY2018SMGDQCP-1,YZY2018SMGDQCP-2,YZY2022nuSMG}. 
%
%\subsection{Green's function zero from the Kramers–Kronig relation}
%The real part of the Green's function can be obtained only by the knowledge of spectral function $A(\omega,\bk)$ via the Kramers–Kronig relation, $\Re G(\omega,\bk) = \frac{1}{\pi}\scP\int d\omega' \frac{A(\omega',\bk)}{\omega'-\omega} $, where $\scP$ denotes the Cauchy principal value. The spectral function in the SMG phase can be measured in the experiment, for example, by the angle-resolved photoemission spectroscopy (ARPES). The Green's function zeros can be reconstructed from the spectral weight of the upper and lower interaction band around $\omega =\pm U$. 
%
%Due to noise, disorder and other factors, the spectral function measured by ARPES is usually broadened. To account for this broadening effect of the spectral function, we take the $\omega \rightarrow \omega +\ii \delta$, with relative large $\delta \sim 1$. We find that though the interaction bands are broadened, the Green's function zero that is constructed from the Kramers–Kronig relation only changes its bandwidth, the Luttinger surface remains unchanged. 
%
%This can be understood as follows. The SMG Green's function is written analytically as in \eqnref{eq:gfsmg}. With the broadening effect, the spectral function is given by the imaginary part of the Green's function $A(\omega,\bk)=-\Im G_\smg(\omega+\ii \delta,\bk)$. And by the Kramers–Kronig relation, one can reconstruct the real part of the Green's function $\Re G_\smg(\omega+\ii \delta,\bk)$. Solving the equation $\Re G_\smg(\omega+\ii \delta,\bk)=0$ gives $\omega(\bk)= -\frac{\alpha  \epsilon_\bk}{U^2}-\frac{\delta ^2 \epsilon_\bk}{U^2}+\scO(\frac{1}{U},\delta)$. This shows the bandwidth of the Green's function zero dispersion is increasing by $\delta^2/\alpha$, but the corresponding Luttinger surface remains unchanged. Therefore, one can in principle identify the Green's function zero or the Luttinger surface in the SMG phase from the ARPES spectral function data.



%as the free fermion with self-energy correction, $G_{\smg} = (\omega - \epsilon_\bk -\Sigma(\omega,\bk) )^{-1}$. With the broadening effect, the spectral function is given by the imaginary part of the Green's function, $A_\smg(\omega,\bk) =-\frac{1}{\pi}\Im G_\smg(\omega,\bk)=\frac{\delta}{\pi(\delta^2+(\omega-\epsilon_\bk-\Sigma(\omega,\bk))^2)}$. Using the Kramers–Kronig relation, one obtains $\Re G_\smg = \frac{\omega -\epsilon_\bk-\Sigma(\omega,\bk) }{\delta ^2+(\omega -\epsilon_\bk -\Sigma(\omega,\bk) )^2}$. The broadening factor $\delta$ only appears in the denominator, thus, broadening effect won't change the zero of the Green's function. Therefore, the Green's function zeros can in principle reconstructed by the spectral function measured by the ARPES experiment, and this is a salient feature of the SMG phase.


\section{Summary and Discussions}\label{sec: summary}

In this paper, we investigated the Fermi surface SMG in a bilayer square lattice model. A crucial finding of this study lies in the robust Green's function zero in the SMG phase. Traditionally, a Fermi liquid state is characterized by poles in the Green's function along the Fermi surface. However, as the fermion system is driven into the SMG state by interaction effects, these poles are replaced by zeros. This is a robust phenomenon underlined by the constraints of the Luttinger theorem.

Our exploration is not limited to theoretical assertions. We also offer a tangible demonstration of this occurrence in the bilayer square lattice model. By applying both strong and weak coupling analyses, we provide a comprehensive portrayal of the fermion Green's function across different interaction regimes. We highlight that the emergence of the zero Fermi surface is not an ephemeral or fine-tuned phenomenon, but rather a robust and enduring feature of the SMG phase. We show that even when the system is subjected to spectral broadening, the zero Fermi surface persists, retaining the Fermi volume.

The results of this study confirm the robustness of the zero Fermi surface and underscore the possibility of observing it in experimental setups, such as through ARPES. Despite not being directly observable, the zero structure of the Green's function could be inferred indirectly via the KK relation.

The bilayer square lattice model may be relevant to the nickelate superconductor recently discovered in pressurized La$_3$Ni$_2$O$_7$ \cite{Sun2023S, Hou2023E2307.09865}, which is a layered two-dimensional material where each layer consists of nickel atoms arranged in a bilayer square lattice. The Fermi surface is dominated by $d_{z^2}$ and $d_{x^2-y^2}$ electrons of Ni. The $d_{z^2}$ electron has a relatively small intra-layer hopping $t$ due to the rather localized $d_{z^2}$ orbital wave function in the $xy$-plane but enjoys a large interlayer antiferromagnetic Heisenberg interaction $J$ due to the super-exchange mechanism mediated by the apical oxygen. This likely puts the $d_{z^2}$ electrons in an SMG insulator phase in the bilayer square lattice model and opens up opportunities to investigate the proposed Green's function zeros in real materials. The potential implication of SMG physics on the nickelate high-$T_c$ superconductor still requires further theoretical research in the future.



%\yzy{Maybe comment 4e-SC model later in the discussion} Results for $H_{4e}$ are shown in Appendix~\ref{append:4e-and-ssb}. 

\begin{acknowledgments}
We acknowledge the helpful discussions with Liujun Zou, Zi-Xiang Li, Fan Yang, Yang Qi, Subir Sachdev, and Ya-Hui Zhang. All authors are supported by the NSF Grant DMR-2238360.

\end{acknowledgments}

\bibliography{ref}



\clearpage
\appendix


\section{Cluster Perturbation Theory}
\label{append:cpt}

Here we review the details of cluster perturbation theory (CPT) originally developed in \cite{CPT2000}. 
Denote the superlattice lattice points by $\bR$, then the position of any original lattice point would be given by $\bR+\br$, where $\br$ is the relative position of the lattice point to the location $\bR$ of the cluster containing that particular lattice point. For clusters of size $L$, the generic Green's function in real space can be denoted by $G_{i,j}^{\bR,\bR'}$, with $i,j=1,...,L$, where the time-dependence is implicitly assumed and same goes for the frequency-dependence in Fourier space. Due to the translation invariance of the clusters on the \textit{superlattice}, the real space Green's function can be firstly partially Fourier-transformed to give,
\begin{equation}
   G_{i,j}^{\bR,\bR'}=\frac{1}{N}\sum_{\bq} G(\bq)_{ij}\e^{i\bq\cdot (\bR-\bR')},
   \label{eq-G-real-space}
\end{equation}
where the $\bq$-summation is over the Brillouin zone (BZ) of the superlattice and $N$ is the number of clusters on the superlattice, which goes to infinity in the thermodynamic limit. In contrast to the translation invariance of the $(\bR,\bR')$-part of $G_{i,j}^{\bR,\bR'}$, or equivalently it only depends on the difference $\bR-\bR'$ as can be seen in Eq.~(\ref{eq-G-real-space}),  the $(i,j)$-part of the Green's function loses translation invariance due to the introduction of clusters. This is so because correlation between two points within the same cluster is not manifestly the same with the correlation between another pair of equally separated points \textit{across} clusters. Therefore, it takes two lattice momenta to fully characterize $G_{i,j}^{\bR,\bR'}$ in Fourier space. More precisely, we have, 
\begin{equation}
    G(\bk,\bk')=\frac{1}{NL}\sum_{\bR,\bR'}\sum_{i,j}G_{i,j}^{\bR,\bR'}\e^{i\bk\cdot(\bR+\br_i)-i\bk'\cdot(\bR'+\br_j)}.
    \label{eq-G-k-space}
\end{equation}
Then we can plug Eq.~(\ref{eq-G-real-space}) into Eq.~(\ref{eq-G-k-space}) and integrate out the superlattice lattice vectors $\bR,\bR'$ to obtain the following,
\begin{equation}
\label{eq-G-k-space-2}
    G(\bk,\bk')=\frac{1}{L}\sum_{i,j}\sum_{\bq}G(\bq)_{ij}\tilde{\delta}_{\bk,\bq}\tilde{\delta}_{\bk',\bq}\e^{i(\bk\cdot\br_i-\bk'\cdot\br_j)},
\end{equation}
where the $\tilde{\delta}$-function denotes the fact that the two wavevectors are equivalent only up to a superlattice reciprocal lattice vector $\bQ$ because $\bQ\cdot\bR=2\pi\dsZ$ in the phase factor. More precisely, we have
\begin{equation}
    \tilde{\delta}_{\bk,\bq}=\sum_{s=1}^{L}\delta_{\bk,\bq+\bQ_s},
\end{equation}
where $\bQ_s$ with $s=1,...,L$ are the $L$ inequivalent wave vectors in the reciprocal lattice of the original lattice (see the 1d case shown in Fig.~\ref{fig:reciprocal-lattice}). Then we can perform the $\bq$-summation in Eq.~(\ref{eq-G-k-space-2}) to have, %\mz{the $1/L$ factor}
\begin{equation}
    \begin{split}
     G(\bk,\bk')&=\frac{1}{L}\sum_{i,j}\sum_{s,s'}G(\bk-\bQ_s)_{ij}\delta_{\bk'-\bk,\bQ_s-\bQ_{s'}}\e^{i(\bk\cdot\br_i-\bk'\cdot\br_j)}\\
        &=\sum_{i,j}\sum_{\Delta\bQ}G(\bk)_{ij}\delta_{\bk'-\bk,\Delta\bQ}\e^{i(\bk\cdot\br_i-\bk'\cdot\br_j)},
    \end{split}
\end{equation}
where we have used the fact that $G(\bq)_{ij}$ is invariant under the shift by a superlattice reciprocal lattice vector $\bQ_s$.  

% Figure environment removed

The translation invariant approximation for the Green's function on the original lattice is obtained when $\Delta\bQ=0$, i.e. $\bk=\bk'$.  Therefore, the Green's function becomes,
\begin{equation} 
\label{eq-Green-final}
     G(\bk) =\sum_{i,j}G(\bk)_{ij}\e^{i\bk\cdot(\br_i-\br_j)}.
\end{equation}

% Figure environment removed

Now we just need to calculate $G_{i,j}(\bk)$ using cluster perturbation. The idea is to treat hopping between clusters as perturbation when consider strong on-site interactions. In particular,
\begin{equation}
    \hat{H}=\hat{H}_0+\hat{V},
\end{equation}
where $\hat{H}_0$ contains intra-cluster terms and $\hat{V}$ contains inter-cluster hopping. Considering nearest-neighbor hopping between the square clusters used in the main text. The cluster construction is reproduced in Fig.~\ref{fig:cluster} with the 4 sites in each cluster labeled by $1,2,3,4$. The hopping matrix is given by (setting lattice constant $a=1$)
\begin{equation}
\begin{split}
V_{i,j}^{\bR,\bR'}=&-t\delta_{\bR,\bR'-2\hat{x}}(\delta_{i,2}\delta_{j,1}+\delta_{i,3}\delta_{j,4})\\
&-t\delta_{\bR,\bR'+2\hat{x}}(\delta_{i,1}\delta_{j,2}+\delta_{i,4}\delta_{j,3})\\
&-t\delta_{\bR,\bR'-2\hat{y}}(\delta_{i,1}\delta_{j,4}+\delta_{i,2}\delta_{j,3})\\
&-t\delta_{\bR,\bR'+2\hat{y}}(\delta_{i,3}\delta_{j,2}+\delta_{i,4}\delta_{j,1})
\end{split}
\end{equation}
Fourier transforming $V_{i,j}^{\bR,\bR'}$ into the superlattice reciprocal space, we have 
\begin{equation}
\begin{split}
V_{i,j}(\bq)&=-t\e^{i2q_x}(\delta_{i,2}\delta_{j,1}+\delta_{i,3}\delta_{j,4})\\
&\quad -t\e^{-i2q_x}(\delta_{i,1}\delta_{j,2}+\delta_{i,4}\delta_{j,3})\\
&\quad -t\e^{i2q_y}(\delta_{i,1}\delta_{j,4}+\delta_{i,2}\delta_{j,3})\\
&\quad -t\e^{-i2q_y}(\delta_{i,3}\delta_{j,2}+\delta_{i,4}\delta_{j,1})\\
&=-t\begin{pmatrix}
0 &  e^{-i2\bq_x} & 0 & e^{i2\bq_y}\\
e^{i2\bq_x} &0 & e^{i2\bq_y} & 0\\
0 &  e^{-i2\bq_y} & 0&  e^{i2\bq_x}\\
 e^{-i2\bq_y} & 0 &  e^{-i2\bq_x} & 0
\end{pmatrix}_{i,j},
\end{split}
\end{equation}
which is the form presented in Eq.~(\ref{eq-V}) in the main text. Then the interacting Green's function is given by
\begin{equation}
\begin{split}
    \hat{G}(\bq)&=\frac{1}{\omega-\hat{H}}=\frac{1}{\omega-\hat{H}_0-\hat{V}(\bq)}\\
    &=\frac{\hat{G}_0}{1-\hat{V}(\bq)\hat{G}_0},
    \end{split}
\end{equation}
where $\hat{G}_0\equiv (\omega-\hat{H}_0)^{-1}$ is the intra-cluster Green's function that can be easily obtained by exact diagonalization as long as the cluster size is not too big. The obtained $G(\bq)_{ij}$ can now be plugged into Eq.~(\ref{eq-Green-final}) to calculate the CPT Green's function for the interacting system. 
%%%%%%%%%%%%%%%%%%%%%%%%%%%%%%%% good how are you, I'm good too.
%\section{Results for Charge-4e SMG and SSB}
%
%If only preserving the $\dsZ_{4}$ symmetry, the system can also be symmetrically gapped by the interaction,
%\begin{equation}
%    H_{4e} = g \sum_{i} c_{i1\uparrow}c_{i1\downarrow}c_{i2\uparrow}c_{i2\downarrow} +h.c.
%\end{equation}
%In the $g/t \gg 1$ limit, the unique ground state of $H_0+H_{4e}$ is $\bigotimes_i (1- c_{i1\uparrow}^\dagger c_{i1\downarrow}^\dagger c_{i2\uparrow}^\dagger c_{i2\downarrow}^\dagger )\ket{0}$.
%
%\label{append:4e-and-ssb}
%% Figure environment removed
%
%% Figure environment removed
%
%
%\section{Topological Number for the Fermi Surface}
%As discussed in \cite{yzy2023FSanomaly}, the bulk of Fermi surface is the Fermi sea, which can be alternatively viewed as the phase space Chern insulator. The phase space is parameterized by real space coordinates $x_i$ and momentum space coordinates $k_i$. The phase space Chern insulator is given by,
%\begin{equation} \label{eq:phasespaceCI}   H=\psi^\dagger(\ii\partial_{\vect{x}}\cdot\vect{\Gamma}_x+\ii\partial_{\vect{k}}\cdot\vect{\Gamma}_k+m(\vect{k})\Gamma^0)\psi,
%\end{equation}
%where $\Gamma^i$ are the Gamma matrices from the complex Clifford algebra $\dsC \ell_{2d+1}$. The mass term only depends on the momentum and negative inside the Fermi sea $\Omega$ while positive outside the Fermi sea,
%\begin{equation}\label{eq:massprofile}
%    m(\vect{k})\left\{
%\begin{array}{cc}
%\leq 0 & \text{if }\vect{k}\in \Omega,\\
%>0 & \text{if }\vect{k}\notin \Omega.
%\end{array}\right.
%\end{equation}
%One can imagine putting a lattice in the phase space, with the lattice spacing $a_{x_i}$ and $a_{k_i}$. Under the Fourier transformation, each of the coordinates has its dual and with spacing $2\pi /a_{x_i}$, $2\pi /a_{k_i}$. One should take the spacing of $(x_i,k_i)$ as the larger spacing compared to their dual $(\ttk_i,\ttx_i)$. Put in other words, the fields with coordinate $(x_i,k_i)$ describe the slow-varying modes in the phase space, while fields of $(\ttk_i,\ttx_i)$ describe the fast-varying modes in the phase space. Certainly, $(x_i,k_i)$ and $(\ttk_i,\ttx_i)$ are equivalent and both describe the same system, but they have different modulation behaviours.
%\begin{equation*}
%    % Figure removed
%\end{equation*}
%
%Using this phase space description, the fast-varying modes and the slow-varying modes are naturally separated, and they are related by the Fourier transformation. This allows us to define the topological index in the slow-varying momentum space. Under the Fourier transformation, the Hamiltonian of phase space Chern insulator becomes,
%\begin{equation} \label{eq:phasespaceCIdual}   H=\psi^\dagger(\vect{\ttk}\cdot\vect{\Gamma}_x+\vect{\ttx}\cdot\vect{\Gamma}_k+m(\vect{k})\Gamma^0)\psi \equiv \psi^\dagger \sfh \psi.
%\end{equation}
%Note that the kinetic terms support on the fast-varying coordinates $(\ttx_i,\ttk_i)$, while the mass term still depend on the slow-varying coordinate $k_i$. The ordinary UV completion of the above Hamiltonian is given by,
%\begin{align}
%    \sfh=&\sin\ttk_i\Gamma_x^i+\sin\ttx_i\Gamma_k^i+ \nonumber\\
%&(m(\vect{k})+\sum_i (\cos \ttk_i+\cos \ttx_i ))\Gamma^0.
%\end{align}
%The topological index for $d$-dimensional real space is given by $C_d(\vect{k})=\scN_{d}\int (G^{-1}dG)^{\wedge 2d+1}$, where $G(\omega,\vect{\ttx},\vect{\ttk})=(\omega +\ii \delta -\sfh)^{-1}$ and $\scN_d$ is the normalization factor.
%
%For 1d Fermi surface in space dimension $d=2$, due to the UV completion, we redefine $m(\vect{k})$ as, 
%\begin{equation}\label{eq:massprofile2}
%    \left\{
%\begin{array}{cc}
%-4< m(\vect{k}) < -2 & \text{if }\vect{k}\in \Omega,\\
%m(\vect{k})\leq -4 & \text{if }\vect{k}\notin \Omega.
%\end{array}\right.
%\end{equation}
%The topological index is given by \cite{XLQ2008TI},
%\begin{equation}
%\begin{split}
%C_2(\vect{k})=&-\frac{\pi^2}{ 15}\epsilon^{\mu\nu\rho\sigma\tau
%}\int\frac{d^2\vect{\ttk}d^2\vect{\ttx}d\omega}{\left(2\pi\right)^5}{\rm
%Tr}\bigg[\left(G\frac{\partial G^{-1}}{\partial
%q^\mu}\right)\\
%&\left(G\frac{\partial G^{-1}}{\partial
%q^\nu}\right)\left(G\frac{\partial G^{-1}}{\partial
%q^\rho}\right)\left(G\frac{\partial G^{-1}}{\partial
%q^\sigma}\right)\left(G\frac{\partial G^{-1}}{\partial
%q^\tau}\right)\bigg]
%\end{split}
%\end{equation}
%where $q^\mu=(\omega,\ttk_1,\ttk_2,\ttx_1,\ttx_2)$. One can show,
%\begin{equation}
%    C_2(\vect{k}) = \left\{
%\begin{array}{cc}
%1 & \text{if }\vect{k}\in \Omega,\\
%0 & \text{if }\vect{k}\notin \Omega.
%\end{array}\right.
%\end{equation}
%This means the topological index $C_2(\vect{k})$ has a integer jump across the Fermi surface $\partial \Omega$. This integer jump cannot be changed continuously, therefore the topological index is stable against symmetric perturbations. Moreover, the integer jump of topological index also happens when across the zeros of the Green's function instead of the poles of the Green's function \tocite{topoindexzero}\cite{yzy2014greenfunctionzero}. This shows the topological index is valid in the strongly interacting regime, and the position of the zeros should match with the poles of the Green's function. 


%\onecolumngrid
%\vspace{24pt}
%\twocolumngrid

\end{document}

