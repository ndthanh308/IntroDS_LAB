\documentclass[11pt]{article}

\usepackage[utf8]{inputenc}
\usepackage[english]{babel}
\usepackage{color}

\usepackage{xr}
%\usepackage{ulem}
\usepackage{soul}

\usepackage{CJKutf8}
\usepackage{titling}

\usepackage{xpatch}
\makeatletter
\newenvironment{mytitlepage}%
  {\begin{titlepage}\def\@thanks{}}%
  {\end{titlepage}}
\xpatchcmd\titlepage{\setcounter{page}\@ne}{}{}{}
\xpatchcmd\endtitlepage{\setcounter{page}\@ne}{}{}{}
\makeatother

% % underline
% \usepackage[normalem]{ulem}
% \useunder{\uline}{\ul}{}

%math
\usepackage{amsfonts,amsmath,amssymb,amsbsy}
\usepackage{mathpazo}
\usepackage{mathptmx}

\newtheorem{theorem}{Theorem}
\newtheorem{corollary}[theorem]{Corollary}
\newtheorem{proposition}{Proposition}
\newenvironment{proof}[1][Proof]{\noindent\textbf{#1.}}{\ \rule{0.5em}{0.5em}}

%figure
\usepackage{graphicx}
%\usepackage{lscape}
\usepackage{pdflscape}

%table
\usepackage{tabularx}
\usepackage{dcolumn} 
\usepackage{booktabs}
\usepackage{threeparttable}
\usepackage{comment}
% \usepackage{multirow}

%caption
\usepackage[justification=justified,singlelinecheck=false]{caption}
\usepackage{subcaption}
\usepackage{float}
%\floatstyle{plaintop}
\restylefloat{table}
\restylefloat{figure}
%\renewcommand{\thefigure}{\Roman{figure}}
%\renewcommand{\thetable}{\Roman{table}}

% % hyperlink
% \PassOptionsToPackage{hyphens}{url}\usepackage{hyperref}

%reference
\usepackage{xcolor}
\usepackage{footnotebackref}
\usepackage{hyperref}
\hypersetup{
    colorlinks=true,
    %filecolor=violet,
    citecolor=black,
    linkcolor=black,
    urlcolor=black
}
\usepackage{enumitem}
% \usepackage{xr-hyper}

%bibtex
%\usepackage[citestyle=authoryear, natbib]{biblatex}
% \usepackage[
% backend=biber,
% style=science,
% ]{biblatex}
% \addbibresource{art.bib}

\usepackage[square,numbers]{natbib}

% \makeatletter
% \newcommand{\bianca}{\renewcommand\NAT@open{[}\renewcommand\NAT@close{]}}
% \makeatother
\makeatletter
\newcommand*{\addFileDependency}[1]{% argument=file name and extension
  \typeout{(#1)}
  \@addtofilelist{#1}
  \IfFileExists{#1}{}{\typeout{No file #1.}}
}
\makeatother

\newcommand*{\myexternaldocument}[1]{%
    \externaldocument{#1}%
    \addFileDependency{#1.tex}%
    \addFileDependency{#1.aux}%
}

\myexternaldocument{si_pnas} % change here to relevant file

%\listfiles

\newcommand*\sqcitet[1]{{\bianca\citet{#1}}}
\newcommand*\sqcitep[1]{{\bianca\citep{#1}}}

%spacing
\usepackage{setspace}

%margin
\usepackage{geometry}
\usepackage{afterpage}

\oddsidemargin 0in
\evensidemargin\oddsidemargin
\setlength{\topmargin}{-0.5in}
\setlength{\textheight}{9.2in}
\setlength{\textwidth}{6.5in}

\renewcommand{\rmdefault}{cmr}

\renewcommand{\baselinestretch}{1.5}

%Climate, Psychology, and Human Culture: A Quantitative Analysis of Art in Global History
%\title{Quantifying the Influence of Climate on Human Mind and Culture: Evidence from Visual Art}\author{Shuhei Kitamura\thanks{%
%Center for Infectious Disease Education and Research (CiDER), Osaka University, D74-1 Office for Industry-University Co-Creation (Bldg. D), 2-8 Yamadaoka, Suita, Osaka, 565-0871 Japan. Email: kitamura@cider.osaka-u.ac.jp.}}
%\date{}

\begin{document}

\begin{mytitlepage}
\title{Quantifying the Influence of Climate on Human Mind and Culture: Evidence from Visual Art}\author{Shuhei Kitamura\thanks{%
Center for Infectious Disease Education and Research (CiDER), Osaka University, D74-1 Office for Industry-University Co-Creation (Bldg. D), 2-8 Yamadaoka, Suita, Osaka, 565-0871 Japan. Email: kitamura@cider.osaka-u.ac.jp.}}
\date{}
\maketitle

\begin{abstract}
This paper examines the influence of climate change on the human mind and culture from the 13th century to the 21st century. By quantitatively analyzing 100,000 paintings and the biological data of over 2,000 artists, an interesting U-shaped pattern in the lightness of paintings was found, which correlated with trends in global temperature. Event study analysis revealed that when an artist is subjected to a high-temperature shock, their paintings become brighter in later periods. Moreover, the effects are more pronounced in art genres that rely less on real things and more on the artist's imagination, indicating the influence of artists' minds. Overall, this study demonstrates the significant and enduring influence of climate on the human mind and culture over centuries. \\
%{\bf Classification}: Psychological and Cognitive Sciences \\
{\bf Keywords}: Climate, art, culture
\end{abstract}

\end{mytitlepage}

\clearpage

\noindent {\large {\bf Significance Statement.}} 
Climate change profoundly impacts human society, influencing agriculture, migration, and warfare, amongst others. Unlike previous studies, this study examines climate changes’ effects on the human mind and culture. Analyzing 100,000 paintings and data on 2,000 artists from the 13th to 21st centuries, intriguing insights emerged. Notably, the lightness of the paintings exhibited a U-shaped pattern mirroring global temperature trends, with a significant correlation between the two. Event study analysis further revealed that high-temperature shocks led to brighter paintings in later periods, which is remarkably more pronounced in art genres that rely on artists' imaginations. This underscores the influence of artists’ minds. These findings demonstrate the significant and enduring influence of climate on the human mind and culture throughout history. \\

\section{Introduction\label{sec:intro}}

To understand major social changes like political transitions and economic development, comprehending the role of people's minds and cultures is essential. This includes exploring the impact of emotions during specific social changes, such as the overthrow of reigns, and identifying triggering factors for psychological responses. Previous studies have discussed the impact of climate on political transitions and the demise of ancient empires and dynasties (e.g., the Akkadian Empire \cite{cullen2000}, Classic Maya civilization \cite{haug2003, kennett2012}, Chinese dynasties \cite{zhang2006, zhang2014}, Angkor \cite{buckley2010}, the Western Roman Empire \cite{butgen2011}, and contemporary sub-Saharan Africa \cite{bruckner2011}), suggesting that social instability caused by climate-driven economic difficulties is a plausible driving force (see \cite{hsiang2013} for a review). Climatic shocks may also influence people's psychology, evoking emotions such as anger and fear. Findings from previous studies indicate a direct link between climate and ``aggressive" behaviors such as horn honking \cite{kenrick1986}, sports violence \cite{larrick2010}, and swearing on social media \cite{baylist2018}. Laboratory experiments have also shown that high temperatures can lead to heightened destructive behavior \cite{almas2020}. Additionally, an indirect link may exist where emotional responses arise owing to inadequate responses from ruling authorities during climate change crises. Quantitative measures are needed to assess whether the climate has influenced people's minds and cultures.

Previous studies have used text data to measure people's minds and cultures. Analyzing the language in millions of books shows that words associated with emotion have declined in the last century \cite{acerbi2013, morin2017, scheffer2021}. The bias that people use more positive than negative words has also declined \cite{iliev2016}. Other studies using text data have found changes in morality \cite{kesebir2012}, individualism \cite{greenfield2013, grossmann2014}, gender and ethnic stereotypes \cite{garg2018}, subjective well-being \cite{hills2019}, prosociality \cite{martins2020}, and cultural tightness-looseness \cite{jackson2019, choi2022} (see also \cite{atari2023} for a review). While text data are valuable sources for measuring the human mind and culture, studies often focus on relatively recent periods or selected countries, possibly owing to current data availability. Furthermore, some aspects of the human mind and culture may not be fully captured by text data.

To address these challenges, this study proposes the use of art, particularly paintings, as an alternative tool for measuring people's minds and cultures. Art data serve as a valuable complement to existing text data. The idea of using art to infer people's minds and cultures dates back to Georg Wilhelm Friedrich Hegel, who used art to infer the {\it zeitgeists} of its time of creation \cite{hegel1998}. According to Tolstoy, art has a unique ability to convey artists' feelings \cite{tolstoy1930, winner1982}. Art historian and sociologist Arnold Hauser attempted to infer cultural values and norms in society from an analysis of artworks \cite{hauser1999}.

This study compiles digital images of 100,000 paintings from the late 13th century to the early 21st century, as well as biographical data of over 2,000 artists. Combining artist’s biographical data with their artworks enables us to assign information on when and where each artwork was created. This allows measuring characteristics of the artwork in certain periods and geographical units (e.g., the average brightness of paintings in the 19th century in France). This study used this unique dataset to analyze the influence of climate on the human mind and culture.

Although art can potentially be used to measure the human mind and culture in various ways, this study focused on analyzing its color, especially brightness, as it is a fundamental element of art associated with emotional valence \cite{valdez1994, wilms2018, jonauskaite2019}. The first analysis investigated how the lightness of paintings changed over time, revealing an interesting U-shape, corresponding to global temperature patterns. The pattern roughly corresponds to the pattern of each period approximately coinciding with the Medieval Warm Period, the Little Ice Age, and the period of recent global warming. There is also a strong correlation between these factors.

Although the above analysis revealed an interesting relationship, the underlying reasons for these patterns remain unclear. Factors such as the availability of particular art materials and methods, common macroeconomic shocks, and political shocks can be potential confounders. Thus, the second analysis employs an econometric technique to investigate the causal relationship between temperature and the lightness of paintings, separating the influence of temperature from confounding factors.

Specifically, the analysis applies an event study design with artist-, country-, and year-fixed effects and examines the effect of hot temperature shocks on the lightness of paintings. With these fixed effects, the regression examines what would happen to the lightness of paintings if artists were unintentionally exposed to historically hot temperatures during their lifetimes. The analysis reveals that although there were no systematic changes in the lightness of paintings before the shocks, the paintings became {\it brighter} after the artists had been exposed to these shocks, especially in paintings that strongly rely on the artists’ imaginations, such as religious and mythological paintings. By contrast, no similar effects were found for paintings that rely more on real objects, such as landscapes and portraits, indicating that the influence of temperature shocks comes through people’s minds, such as emotions, sentiments, and mood. By contrast, cold-temperature shocks tend to create {\it darker} paintings.

Finally, heterogeneous responses based on the characteristics of certain artists were examined. The analysis found that self-trained artists are more likely to respond to temperature shocks, while academically trained artists are less likely. However, no significant heterogeneous responses were found with respect to the artists’ age, sex, or income sources.

Overall, these results indicate a non-negligible impact of climate on the human mind and culture.

\section{Results\label{sec:results}}

\subsection{Global Patterns}

The top panel of Figure \ref{fig:main}A shows the lightness of the painting (HSV-V) between 1270-2006, revealing an interesting U-shaped pattern: the paintings were initially brighter, then darker, and eventually brighter again. The middle panel in Figure \ref{fig:main}A displays the mean surface temperature anomalies, defined relative to the 1961-1990 reference period mean. The figure includes a period of coldest temperatures known as the Little Ice Age, describing the period between the 14th and 19th centuries \cite{white2013}. Figure \ref{fig:main}B shows the correlation between the lightness of paintings and the temperature anomaly, with each dot representing the average value of paintings per country and year. The figure illustrates a clear positive relationship ($r$=0.356, $p<$0.001). The patterns are also similar across different countries (Figure \ref{fig:main}C), which is somewhat surprising considering the spatial segregation of art materials and methods, especially in earlier periods before the advancement of transportation and communication technology.

Although the co-movement of temperature and lightness is interesting, the underlying reasons for these patterns remain unclear. One possible conjecture is that paintings become darker owing to changes in painting styles that might have occurred alongside global temperature changes. An example of the painting style used during the dark period of paintings is {\it chiaroscuro}, a technique that was developed during the Renaissance period and involves a dramatic contrast between light and shadow. A related, but more dramatic, form of chiaroscuro is {\it tenebrism}, which was introduced and developed during the Baroque period. Artists like Caravaggio (1571-1610), Georges de La Tour (1593-1652), and Rembrandt (1606-1669) are known for using this style \cite{chilvers2009}. Figure \ref{fig:main}D shows examples of paintings from each period, which are colored gray in Figure \ref{fig:main}A (1400-1499, 1600-1699, and 1800-1899). In this figure, the paintings at the top are brighter paintings (90th percentile), those in the middle are paintings with average lightness, and those at the bottom are darker paintings (10th percentile). As shown in the figure, one of the darkest paintings in the 17th century, Georges de la Tour's {\it The Adoration of the Shepherds} (c.1644), employs this style. This suggests that the darkness in paintings is partially explained by a change in painting style, especially at the bottom tail. However, the figure also shows that the upper tails of paintings, not necessarily using this style, tend to be darker. This simple exploration suggests that although the change in the lightness of paintings is partly explained by the change in painting style, it is not the sole explanation for this phenomenon. Similarly, the number of paintings in the dataset does not seem to explain this pattern (bottom panel in Figure \ref{fig:main}A).

Alternative definitions of lightness show very similar patterns ({\it SI} Appendix, Figure \ref{fig:lightness_alt}A), whereas the other elements of color in the HSV color space, saturation and hue, do not exhibit similar patterns ({\it SI} Appendix, Figure \ref{fig:lightness_alt}B).

\subsection{Causal Analysis}
Although the above analysis revealed an interesting relationship between climate and the lightness of paintings, it remains unclear whether this relationship is causal. Factors such as the availability of particular art materials and methods (e.g., specific colors and paint tubes), common macroeconomic shocks (e.g., business cycles), and political shocks (e.g., wars) could potentially function as confounders. Thus, the next analysis employs an econometric technique to directly assess whether there is a causal relationship between temperature and lightness of paintings. Specifically, it employs an event study design in which artist-, country-, and year-fixed effects are included as control variables. These fixed effects account for common factors within an artist, (such as personality and educational background), within a country, (such as geography and nationality), and within a year, (such as the availability of particular art materials and methods, common macroeconomic shocks, and political shocks), respectively. Controlling for such potential confounders, the regression analysis examines what would happen to the lightness of paintings if artists had been unintentionally exposed to historically hot temperatures during their lifetimes. The focus on hot temperatures is owing to data availability. The Supplementary Material includes a discussion and analysis of cold temperatures ({\it SI} Appendix, Section \ref{sec:cold}).

Figure \ref{fig:event_main}A displays the coefficients from the regression and the associated 95\% confidence intervals, where zero on the x-axis represents the year when artists were first exposed to historically hot temperatures in their lifetimes. Shocks were defined as the top 1\% of global temperature anomalies (see {\it Materials and Methods}). The figure illustrates that before experiencing the temperature shock, the lightness of the paintings was very similar between the treatment and control groups. After the shocks, however, the treated artists (those who experienced hot temperature shocks for the first time in their lives) tend to create brighter paintings. The effect is clear immediately after the shock, and it becomes statistically significant after several years. The magnitude of the effect was 0.28 standard deviations ($p$=0.000) after 20 years from the shock.

Although the current method using a panel controls for common factors within an artist, country, and year, and uses granular spatiotemporal variation to estimate the causal effect, there might still be omitted factors correlated with temperature shocks. The Supplementary Material includes regression results that additionally control for the covariates of economic development, population changes, and political disturbances ({\it SI Appendix}, top-left panel in Figure \ref{fig:event_study_robust}). Furthermore, the Supplementary Material includes regression results that take into account correlations in the error term ({\it SI Appendix}, top-right panel in Figure \ref{fig:event_study_robust}), the results using a different estimation method \cite{sun2021} ({\it SI Appendix}, middle-left panel in Figure \ref{fig:event_study_robust}), and the results using a subset of the sample ({\it SI Appendix}, middle-right panel in Figure \ref{fig:event_study_robust}. See also Section \ref{sec:dist_shocks} in the Supplementary Material), and the results after removing very bright pixels ({\it SI Appendix}, bottom-left figure in Figure \ref{fig:event_study_robust}. See also {\it Materials and Methods}). Overall, the analyses yield similar results.

The Supplementary Material also includes an analysis on artists' ``productivity,” measured by the number of paintings for each artist and year ({\it SI Appendix}, Section \ref{sec:productivity}). The results provide no clear evidence that temperature shocks affect productivity.

\subsection{Mechanism}

To further understand the underlying mechanism behind the above findings, the next analysis examines whether changes in the lightness of paintings come through either the internal channel, that is, the effect through artists’ minds, or the external (or mechanical) channel, that is, artists simply drawing bright views outside, brightly dressed figures, or both. For example, Dutch paintings from the Little Ice Age often depict scenes from winter landscapes \cite{behringer2010, degroot2018}. The following analysis uses particular art genres to narrow down the possible channels.

In paintings, some references are made to things that exist in reality, whereas others depend on the artist’s imagination. Therefore, if the effect appears in art genres that rely more on real objects, such as landscapes and portraits, it would indicate an external channel. However, if the effect appears in art genres that rely more on the artist’s imagination, such as religious and mythological paintings, it indicates an internal channel.

The top-left panel of Figure \ref{fig:event_main}B includes only religious and mythological paintings. Similar to the previous results, the figure shows that artists tended to draw brighter paintings after hot temperature shocks. They responded relatively immediately, and the size of the effect was greater than that shown early (0.93 standard deviations ($p$=0.002) after 20 years from the shocks). However, this does not imply that previous results were driven only by these particular art genres. The top-right panel in Figure \ref{fig:event_main}B shows that excluding these paintings does not significantly change the main findings. In contrast, no clear pattern was observed for either portraits (bottom-left panel in Figure \ref{fig:event_main}B) or landscapes (bottom-right panel in Figure \ref{fig:event_main}B). These results indicate the existence of a channel that cannot be solely explained by an external channel.

\subsection{Influence of Personal Attributes}

Artists may respond differently to temperature shocks depending on their age, sex, source of income, and how they learned to paint. The following analysis examines these heterogeneous responses using econometric techniques. Specifically, for each individual attribute (e.g., age), the lightness of paintings is regressed on the same temperature shocks interacting with the attribute, along with other variables, using a difference-in-differences (DID) estimation (see {\it Materials and Methods}). 

Figure \ref{fig:attribute} shows the estimated coefficients and standard errors of the interaction terms for each attribute. The figure reveals that the effect of hot temperature shocks does not systematically differ by artists’ age ($b$=-0.002, $p$=0.084), gender ($b$=0.066, $p$=0.206), or income source (remittances: $b$=0.033, $p$=0.702, patrons: $b$=-0.095, $p$=0.173, employed: $b$=0.065, $p$=0.067). In contrast, there are differences depending on how artists learned to paint: self-taught artists ($b$=0.377, $p$=0.000) and those who learned to paint through apprenticeship ($b$=0.105, $p$=0.006) are more likely to respond to shocks, whereas those who learned to paint at an academy ($b$=-0.303, $p$=0.000) are less likely to be affected. One plausible interpretation is that academy-trained artists are influenced by what has been taught at an academy and find it difficult to deviate from it. In contrast, other types of artists, particularly those who learned to paint on their own, could relatively freely adjust their painting styles when affected by temperature shocks.

\section{Discussion\label{sec:discussion}}

Climate change has had far-reaching impacts on human societies, shaping various aspects of human life such as agriculture, migration, and warfare (see \cite{ljungqvist2021} for a review). In contrast to previous studies, this study examines how climate has impacted the human mind and culture using unique data from paintings and artists. The first analysis demonstrated that the lightness of paintings shows an interesting U-shaped pattern associated with the pattern of temperature anomalies in the same periods. The second analysis applied the event study design with artist-, country-, and year-fixed effects and examined the effects of temperature shocks on the brightness of paintings. The results indicate that the lightness of paintings is influenced by temperature shocks, especially paintings that rely more on artists’ imaginations than on real objects. These findings suggest that climate has a non-negligible impact on the human mind and culture.

To gain a deeper understanding of the mechanisms behind major social changes such as political transitions and economic development, it is crucial to comprehend the influence of people’s minds and cultures that enable such transformations. While indicators of economic development, such as The Maddison Project \cite{bolt2020}, and indicators of political regimes, such as The Polity Project \cite{polity}, have been widely used, there has been no common measure of people’s minds and cultures over an extended period. Although text data have been used in previous studies \cite{acerbi2013, morin2017, scheffer2021, iliev2016, kesebir2012, greenfield2013, grossmann2014, garg2018, hills2019, martins2020, jackson2019, choi2022}, there remain challenges in terms of coverage (both time and space) and content. This study demonstrates that art can serve as a valuable tool for analyzing people’s minds and cultures, thus complementing the existing data but providing unique insights.

However, a limitation of using paintings as data is that older paintings are more likely to be available in Western countries, although the spatial coverage becomes wider if paintings in more recent years are considered (see Figure \ref{fig:map_coverage} in the Supplementary Material). The future studies may also use different types of art, such as sculptures and music, to improve the analysis. Furthermore, although this study focuses on the lightness of paintings, future research can also explore other features of paintings. Finally, while this analysis uses past data, it raises an intriguing question regarding potential future trends for paintings. This aspect remains a topic for future research.

\section{Materials and Methods\label{sec:methods}}

\subsection{Data sources and construction procedures}

The images of the paintings were obtained from Wikiart \cite{wikiart}. A total of 153,178 images and 2,836 artists were downloaded initially. Then, artworks from the Proto-Renaissance to modern art were retained, and the remainder were dropped. The following art genres were excluded: artists’ books, architecture, calligraphy, design, graffiti, installation, jewelry, mobile, mosaic, mural, ornament, performance, photo, quadratura, sculpture, sketch and study, stabile, tapestry, and video. Artists whose artworks were unavailable owing to copyright infringement issues were removed. This process yielded 126,310 images and 2,227 artists. Of these, approximately 21\% (26,351) of the artworks with missing information on the year of creation were removed. The final sample consisted of 99,959 images and 2,174 artists. For the remaining images with a range of years (e.g., 1752--1754) provided (7,366 images), the mid-year period (e.g., 1753) was used as the year of creation. The location of creation for most artworks lacking information was estimated based on the artist’s residential country at each point in time using the migration history of artists.

The biographies of artists, including their migration histories, were obtained from online sources, such as artists’ websites and online encyclopedias. Migration in this study is defined as the change in residential places between countries, but not within a country, as migration history is only available at the country level in many cases. Short trips between countries were not considered migration owing to limited availability of such details. The country boundaries were based on GADM Version 3.6 \cite{gadm}. Figure \ref{fig:map_coverage} in the Supplementary Material shows the spatial coverage of the data.
 
The lightness of the images was computed using Python OpenCV. The main analysis used the hue, saturation, value (HSV) color space, based on the Munsell color system. After importing the images into the BGR, they were converted to HSV. However, 12 images collapsed during this process, resulting in missing values. Therefore, the color information was obtained from 99,947 images. Missing observations in color information for the panel of artists were filled in using linear interpolation. 

For the main analysis, the removal of any range of pixels from images was avoided to prevent arbitrary truncation of certain brightness values. The Supplementary Material shows that the results remain very similar even after removing very bright pixels ({\it SI Appendix}, bottom-left figure in Figure \ref{fig:event_study_robust}).

Other data sources and construction procedures were as follows: Global temperature data were taken from Mann et al. (2009) \cite{mann2009}, and values were resampled for smaller cells using the natural neighbor interpolation method. The average annual temperature was then calculated for each polygon. The geocoded battle data were taken from Kitamura (2022) \cite{kitamura2022}, matched with country polygons, and the number of battles per annum was counted within each country polygon. The gridded population density was obtained from Goldewijk et al. (2017) \cite{goldewijk2017}, and values were resampled for smaller cells using the natural neighbor interpolation method, with average values computed for each country polygon. Annual data were linearly interpolated for earlier periods where data were unavailable. Data on GDP per capita were taken from Bolt and van Zanden (2020) \cite{bolt2020} and also linearly interpolated for missing observations. These country-level data were matched with each artist by year, based on their place of residence. All GIS processes were conducted using ArcMap Version 10.8. Table \ref{fig:summary_table} in the Supplementary Material presents the summary statistics for the variables used in this study.

\subsection{Statistical Analysis}

\subsubsection{Correlation}

For the correlational analysis between temperature anomalies and the lightness of paintings, Pearson’s correlation coefficient and the {\it p}-value were reported. The statistical analysis was performed using R 4.3.1. For all statistical tests, the significance level was set at 0.05.

\subsubsection{Event Study}
The regression equation for the event study analysis is as follows:
\begin{equation}\label{eq:event_main}
Y_{art} = \sum_{l = -10, \neq -1}^{30} \beta_l I[t - k_{ar} = l] + \rho_a + \gamma_r + \tau_t + \epsilon_{art},
\end{equation}
where $Y_{art}$ is the dependent variable for artist $a$ living in country $r$ and year $t$; $I[t - k_{ar} = l]$ is an indicator that takes the value of 1 if the equation in brackets holds and 0 otherwise; $k_{ar}$ is the year that artist $a$ experienced a global hot temperature shock; $\rho_a$, $\gamma_r$ and $\tau_t$ are fixed effects; and $\epsilon_{art}$ is the error term. The Supplementary Material includes the results with additional controls: GDP per capita, population density, and the presence of battles in the country where artists were living ({\it SI Appendix}, top-left panel in Figure \ref{fig:event_study_robust}). The main analysis uses robust standard errors, and the Supplementary Material includes the results using three-way clustered standard errors ({\it SI Appendix}, top-right panel in Figure \ref{fig:event_study_robust}). Global hot-temperature shock was defined as the top 1\% of the temperature anomalies in the data. The first shock was administered to those who experienced two or more shocks. The distribution of these shocks is shown and discussed in the Supplementary Material ({\it SI Appendix}, Section \ref{sec:dist_shocks}).

The year before the shock was set as the baseline, and the coefficients $\beta_l$ capture the differences in outcome values between the treatment and control groups compared to the baseline year. To increase the interpretability of the results, the estimated coefficients were reported by dividing each coefficient by the standard deviation of the dependent variable. Finally, the original Wikiart data contained information on art genres, which was used to select paintings from particular art genres. All regression analyses were performed using STATA 15.

\subsubsection{DID estimation}
The regression equation for the analysis to examine the influence of personal attributes is as follows:
\begin{equation}\label{eq:event_hetero}
Y_{art} = \alpha_0 D_{art} + \alpha_1(D_{art} \times Z_{a}) + \phi_a + \psi_r + \xi_t + u_{art},
\end{equation}
Where $Y_{art}$ is the dependent variable for artist $a$ living in country $r$ and year $t$, $D_{art}$ is an indicator that takes the value of one after the shock and zero otherwise, $Z_{a}$ is a personal attribute, $\phi_a$, $\psi_r$ and $\xi_t$ are fixed effects, and $u_{art}$ is the error term. The shock was defined the same way as in the main analysis. Personal attributes included age at the time of shock, sex (female = 1), source of income, and how the artist learned to paint. The attributes are selected in accordance with previous discussions with art historians, but are limited by data availability. The underlying hypotheses were such that the influence of hot temperatures might be greater for those exposed during their early childhood, for female artists, for those who had an unstable income, and for those who had learned to paint by themselves. The coefficient $\alpha_1$ captures the heterogeneous responses to shocks based on each personal attribute (e.g., female artists’ responses to the shock compared to male artists’ responses to the same shock). \\

\noindent {\bf Acknowledgments.} This research was financially supported by the JSPS (18K12768, 21K132840). I would like to express my gratitude to Shuji Asaka, Hoshimi Hoashi, Shibao Mayo, Aiko Takehana, and Elizaveta Kugaevskaya for their valuable assistance with this research. Additionally, I extend my thanks to Takuma Kamada, Tsukasa Kodera, and Nils-Petter Lagerlöf for their valuable and insightful discussions, as well as the seminar participants at Osaka and York for their helpful feedback. \\

\noindent {\bf Competing Interest Statement.} The author declares no conflict of interest.

% Bibliography
\bibliographystyle{unsrtnat}
\bibliography{art}

%\printbibliography

\section*{Tables and Figures}

\vspace{3cm}

% Figure environment removed

\clearpage

% Figure environment removed

\clearpage

% Figure environment removed

\clearpage

\appendix


\renewcommand{\thesection}{A.\arabic{section}} \setcounter{section}{0} %
\renewcommand{\thefigure}{A.\arabic{figure}} \setcounter{figure}{0} %
\renewcommand{\thetable}{A.\arabic{table}} \setcounter{table}{0}


\begin{mytitlepage}
\title{Supplementary Information for {\it Quantifying the Influence of Climate on Human Mind and Culture: Evidence from Visual Art}}\author{Shuhei Kitamura\thanks{%
CiDER, Osaka University. Email: kitamura@cider.osaka-u.ac.jp.}}
\date{}
\maketitle
\end{mytitlepage}

\section{Supplementary Text}

\subsection{Distribution of shocks}\label{sec:dist_shocks}

Figure \ref{fig:dist_shocks} illustrates the artists’ treatment and control status, where a brighter blue color indicates that artists have been treated, whereas a darker blue color indicates that artists have not been treated (yet). As global temperatures increase, temperature shocks tend to occur in the 20th and 21st centuries (between 1900-2005). This figure also shows that a relatively substantial number of artists were exposed to the 1938 shock. To ensure that the main results were not driven by this shock, I ran a regression using equation (\ref{eq:event_main}) by excluding all artists whose treatment year was 1938. The middle-right panel of Figure \ref{fig:event_study_robust} shows that excluding relatively large samples does not systematically invalidate the main results. Thus, the results are unlikely to have been influenced by this particular shock.

\subsection{Effects of cold temperatures}\label{sec:cold}

In contrast to examining the effect of hot temperatures, examining the effect of cold temperatures is difficult owing to data availability. As the bottom panel of Figure \ref{fig:main}A in the main text indicates, more paintings have become available in recent years. Therefore, if the cold temperature events are defined first and then matched with artist data, only a few events (actually only two years) could be successfully matched with the artists’ data. This makes it almost impossible to conduct an econometric analysis.

One possible way to avoid this issue is to redefine the events of cold temperatures {\it after} merging the painting data. Therefore, the interpretation of the following regression estimates is the impact of cold temperature shocks {\it in the final sample}. The bottom-right panel in Figure \ref{fig:event_study_robust} displays the coefficient from the regression, where zero indicates the year in which an artist was exposed to global cold temperatures. A global cold-temperature shock was defined as the bottom 1\% of temperature anomalies in the final sample. 

Similar to hot temperature shocks, it takes several years for the effect to be statistically significant. The overall pattern was that artists tended to draw darker paintings after cold-temperature shocks. The magnitude of the change was 0.29 standard deviations ($p$=0.006) after 20 years after the shock. In contrast to the monotonic increase in the estimates in the case of hot-temperature shocks, the effects of cold-temperature shocks tend to diminish after a peak. An interpretation of this result is that temperature shocks are not symmetrical; the effects of hot temperature shocks last relatively long, whereas cold temperature shocks have relatively short-lived effects.

\subsection{Effects on productivity}\label{sec:productivity}

Temperature may also affect the number of paintings that an artist produces per year. To check this possibility, Figure \ref{fig:count_art} illustrates the estimates of regressing the equation (\ref{eq:event_main}) in the main text, by replacing the dependent variable by a measure of ``productivity," that is, the total number of paintings for each artist and year. Missing observations in the number of paintings for the panel of artists were filled in using linear interpolation. The figure shows that there is no clear evidence that temperature shocks affect artists’ productivity.

% Bibliography
%\bibliographystyle{unsrtnat}
%\bibliography{art2}

\section{Supplementary Tables and Figures}

\clearpage

\begin{table}[p] \centering 
\begin{tabular}{@{\extracolsep{5pt}}lccccc} 
\\[-1.8ex]\hline 
\hline \\[-1.8ex] 
 & \multicolumn{1}{c}{N} & \multicolumn{1}{c}{Mean} & \multicolumn{1}{c}{Std. Dev.} & \multicolumn{1}{c}{Min} & \multicolumn{1}{c}{Max} \\ 
\hline \\[-1.8ex] 
Age of the first shock & 50,979 & 50.74 & 24.25 & 0 & 96 \\ 
Female & 141,762 & 0.11 & 0.31 & 0 & 1 \\ 
Income source: Remittance & 141,837 & 0.03 & 0.17 & 0 & 1 \\ 
Income source: Patron & 141,837 & 0.10 & 0.31 & 0 & 1 \\ 
Income source: Employed & 141,837 & 0.55 & 0.50 & 0 & 1 \\ 
Art education: Self & 136,670 & 0.07 & 0.25 & 0 & 1 \\ 
Art education: Apprentice & 136,670 & 0.44 & 0.50 & 0 & 1 \\ 
Art education: Academy & 136,670 & 0.74 & 0.44 & 0 & 1 \\ 
Number of artworks & 66,634 & 2.24 & 3.75 & 1.00 & 191.00 \\ 
HSV-V & 66,634 & 137.13 & 39.14 & 5.96 & 254.32 \\ 
HSV-S & 66,417 & 91.29 & 36.56 & 0.0003 & 254.16 \\ 
HSV-H & 66,417 & 49.16 & 22.59 & 0.01 & 176.34 \\ 
HLS-L & 66,634 & 115.03 & 37.20 & 4.00 & 248.18 \\ 
CIELUV-L & 66,634 & 123.50 & 38.65 & 3.04 & 247.80 \\ 
Average global temperature anomalies & 141,837 & $-$0.15 & 0.31 & $-$1.41 & 1.52 \\ 
GDP per capita & 129,122 & 8,999.82 & 9,102.32 & 295.00 & 51,862.96 \\ 
Battles & 141,802 & 0.53 & 4.86 & 0 & 140 \\ 
Average population density & 141,837 & 72.07 & 67.76 & 0.01 & 2,247.82 \\ 
\hline \\[-1.8ex] 
\end{tabular} 
\caption{Summary statistics.\label{fig:summary_table}}
\end{table} 

\clearpage

% Figure environment removed

\clearpage

% Figure environment removed

\clearpage

% Figure environment removed

\clearpage

% Figure environment removed

\clearpage

% Figure environment removed


\end{document}


