%%%%%%%%%%%%%%%%%%%%%%%%%%%%%%%%%%%%%%%%%%%%%%%%%%%%%%%%%%%%%%%%%%%%%%%%%%%%%%%%
%2345678901234567890123456789012345678901234567890123456789012345678901234567890
%        1         2         3         4         5         6         7         8

%\documentclass[letterpaper, 10 pt, conference]{mhs}  % Comment this line out if you need a4paper

\documentclass[a4paper, 10pt, conference]{mhs}      % Use this line for a4 paper

\IEEEoverridecommandlockouts                              % This command is only needed if 
% you want to use the \thanks command

\overrideIEEEmargins                                      % Needed to meet printer requirements.

%In case you encounter the following error:
%Error 1010 The PDF file may be corrupt (unable to open PDF file) OR
%Error 1000 An error occurred while parsing a contents stream. Unable to analyze the PDF file.
%This is a known problem with pdfLaTeX conversion filter. The file cannot be opened with acrobat reader
%Please use one of the alternatives below to circumvent this error by uncommenting one or the other
%\pdfobjcompresslevel=0
%\pdfminorversion=4

% See the \addtolength command later in the file to balance the column lengths
% on the last page of the document

%\renewcommand{\thesection}{\arabic{section}}
%\renewcommand{\thesubsection}{\thesection.\arabic{subsection}}
%\renewcommand{\thesubsubsection}{\arabic{subsubsection}}

% The following packages can be found on http:\\www.ctan.org
\usepackage{graphics} % for pdf, bitmapped graphics files
\usepackage{epsfig} % for postscript graphics files
\usepackage{mathptmx} % assumes new font selection scheme installed
\usepackage{times} % assumes new font selection scheme installed
\usepackage{amsmath} % assumes amsmath package installed
\usepackage{amssymb}  % assumes amsmath package installed
\usepackage{here}
\usepackage{threeparttable}
\usepackage{caption}

\captionsetup[table]{labelsep=period, labelfont=bf, justification=raggedright, singlelinecheck=off}
%\usepackage{titlesec}

%\titleformat*{\section}{\sc \bfseries \centering}
%\titleformat*{\subsection}{\sc \bfseries}

\title{\Large \bf
Human Preferences and Robot Constraints Aware Shared Control \\ for Smooth Follower Motion Execution
}
% Human Preferences and Robot Constraints Aware Shared Control for Smooth Follower Motion Execution
% Human Intervention-Possible Alternating Shared Control \\
% with Transitioned Grasping Pose in Teleoperation

\author{Qibin Chen$^1$, Yaonan Zhu$^1$$^{\ast}$, Kay Hansel$^2$, Tadayoshi Aoyama$^1$, and Yasuhisa Hasegawa$^1$\\% <-this % stops a space
1. Department of Micro-Nano Mechanical Science and Engineering, Nagoya University,\\
Nagoya, Aichi, 464-8603, Japan\\
2. Department of Computer Science, Technical University of Darmstadt,\\
% FG IAS, Technical University of Darmstadt,\\
Hochschulstr. 10, 64289 Darmstadt, Germany\\\\
% TU Darmstadt, FG IAS,
% Hochschulstr. 10, 64289 Darmstadt
% Department of Computer Science, Technical University of Darmstadt, Hesse, Germany\\\\
\thanks{This work has been submitted to the IEEE for possible publication. Copyright may be transferred without notice, after which this version may no longer be accessible.}
\thanks{$^{\ast}$Corresponding author email: zhu@robo.mein.nagoya-u.ac.jp}
\thanks{This work was supported in part by JST Trilateral AI Research, Japan, under Grant JPMJCR20G8; and in part by JSPS KAKENHI under
Grant JP22K14222.}
\vspace{-14mm}
}


\begin{document}
	
	\maketitle
	\thispagestyle{empty}
	\pagestyle{empty}
	
	
	%%%%%%%%%%%%%%%%%%%%%%%%%%%%%%%%%%%%%%%%%%%%%%%%%%%%%%%%%%%%%%%%%%%%%%%%%%%%%%%%
	\begin{abstract}
    With the continuous advancement of robot teleoperation technology, shared control is used to reduce the physical and mental load of the operator in teleoperation system. 
    This paper proposes an alternating shared control framework for object grasping that considers both operator's preferences through their manual manipulation and the constraints of the follower robot. 
    The switching between manual mode and automatic mode enables the operator to intervene the task according to their wishes. 
    % The alternating switching between manual mode and automatic mode enables that the operator can intervene in the teleoperation system according to their wishes. 
    The generation of the grasping pose takes into account the current state of the operator's hand pose, as well as the manipulability of the robot. 
    % Hence, it respects the results caused by user willingness, and ensures both the ease of understanding of the system guidance on the user side and the feasibility of the motion on the robot side. 
    % The object grasping experiment indicates that by using this proposed grasping pose selection strategy can make the follower motion smoother and coherent when switching modes from manual to automation.
    The object grasping experiment indicates that the use of the proposed grasping pose selection strategy leads to smoother follower movements when switching from manual mode to automatic mode.

	\end{abstract}
	
	
	%%%%%%%%%%%%%%%%%%%%%%%%%%%%%%%%%%%%%%%%%%%%%%%%%%%%%%%%%%%%%%%%%%%%%%%%%%%%%%%%
	
	\section{Introduction}
Shared control, which employs automation to support human operation, has been widely used in teleoperation systems, since it effectively improves performance while reducing the physical and mental strain on users.
% Robotic teleoperation has been widely employed in a lot of areas, including surgical robotics, assistive robotics, and space exploration. 
% The pure teleoperation system that directly follows operator's manipulation has several problems such as high operational difficulty, heavy mental and physical burden on operators, and limited task performance. 
% The pure teleoperation system that directly follows operator's manipulation has several problems such as high operational difficulty, heavy burden on operators and limited task performance, caused by network delays, degraded sensing information from the follower and different configurations between humans and robots.
% These problems are caused by network delays, insufficient or inaccurate sensing information from the follower, inadequate reproduction for telepresence at the master side, and different configurations between humans and robots. 
% To address these shortcomings, the shared control, which combines operator movements with automation decisions, is a feasible solution.
Object grasping is one of the most important teleoperation tasks and is the subject of this paper.
% Towards this task, our previous work \cite{zhu2023shared} has developed a shared control system and proves that the system improves the grasping performance and reduces the fatigue of the operator.
To address this task, previous work \cite{zhu2023shared} has developed a shared control system and demonstrated that the system improved grasping performance and reduced operator fatigue.
% Facing this task, Zhu et al. \cite{zhu2023shared} have developed a shared control system and proved that the system improves the grasping performance and reduces the fatigue of the operator.
% This shared control framework can guide the operator to the appropriate grasping pose towards the user desired object with the best manipulability of the robot, by dynamically and continuously blending user intention and automation suggestion.
This shared control framework can guide the operator to an appropriate grasping pose with the best manipulability of the robot, by dynamically and continuously blending user intention and automation assistance.
% This kind of shared control system often faces three important issues that need to be addressed: (1) The first problem is how to ensure that operators can intervene in the teleoperation system at any time, including changing automation decision and exiting automation assistance. (2) The second one is that how to decrease the operator's unease feeling caused by the unpredictability of the automation results. (3) For the third point, it is how we can ensure that the robot's actions remain coherent, stable, and reliable in finishing the grasping during manual intervention and automation assistance switching.
% This paper aims to address problems of shared control system not considering human choices, proposing an alternating shared control system that separates automation and manual part, allowing human intervention at any time. In addition, the target grasping pose selected from all generated candidates, is generated by not only taking into account the robot's manipulability, but also the operator's preferences.  
% This paper aims to address the problem of shared control system which does not consider human choices, by proposing an alternating shared control system that separates the automation and manual parts, allowing human intervention at any time. 
This paper aims to address the problem of shared control system which does not consider human choices.
An alternating shared control system is proposed, which segregates the automation and manual aspects, empowering users to intervene at any time.
% The alternating shared control in teleoperation exploits the full manual manipulation when user inputs. 
On the other hand, conventional continuous shared control maintains the combination of manual input and automation throughout the entire process.
In Maeda's work \cite{maeda2022blending}, the advantages of alternating shared control were demonstrated that the alternating way won a higher score for ease of use in subjective ratings while keeping the performance as the continuous method.
In addition, to achieve a smooth follower motion execution, the target grasping pose is selected from all candidates by taking into account not only the manipulability of the robot, but also the preferences of the operator.  
%Thus, it ensures the robot's motion is sufficiently coherent during the transition from manual mode to automatic mode.

%This paper aims to address the weaknesses in the previous work. Firstly, an alternating shared control system that separates automation and manual is proposed, allowing manual intervention at any time, like adjusting the robot movements, and exiting automation assistance. It also avoids the unpredictable and insecurity feeling of users towards the system response caused by the continuous fusion of dynamic weights of the automation part. Secondly, the generated grasping pose not only takes into account the robot's manipulability, but also the operator's preferences, thus ensuring the robot's motion is sufficiently coherent during the transition from manual mode to automatic mode.
	
	\section{ Shared Control System with Human Preferences }
% \subsection{Advantage of the Alternating Shared Control}
% \subsection{Alternating Shared Control}
% The alternating shared control is a practicable way to simultaneously solve both question 1 and question 2 raised in the introduction.
% The alternating shared control in teleoperation exploits the full manual manipulation when user inputs. On the other hand, there is always a degree of automation in continuous shared control.
% In Maeda's work \cite{maeda2022blending}, the advantages of alternating shared control were demonstrated that the alternating way won a higher score for ease of use in subjective ratings while keeping the performance as the continuous method.
% Maeda \cite{maeda2022blending} proposed a blending primitive policy that implicitly blended human and robot policies without weight-based arbitration.
% The differences were experimentally evaluated between alternating way and continuous way, showing that the former one has the same objective performance as the continuous method on the task, but in subjective ratings the participants believe that the former one improves both the ease of use and the sense of autonomy.
% The experimental evaluation compared the alternating way and continuous way, demonstrating that the former achieved equivalent objective performance as the continuous method in completing the task. However, in subjective ratings, participants perceived that the former method significantly improved both ease of use and the sense of autonomy.
% In Maeda's work \cite{maeda2022blending}, which proposed the blending primitive policy that it implicitly blended human and robot policies without weight-based arbitration, the differences were evaluated between alternating shared control, which became the full manual teleoperation when user inputs, and continuous blending, which combined autonomy prediction and user input together. 
% The experimental results showed that the alternating method had the same objective performance as the continuous method on the task, but in subjective ratings the participants believed that the former one had a greater ease of use and a higher sense of control intuition.
% Maeda \cite{maeda2022blending} has proposed the blending primitive policies that by hiring the Dynamical Movement Primitives (DMPs), it implicitly blends human and robot policies without weighting-based arbitration. In his paper, based on his primitive policies, the differences are evaluated between alternating shared control that it becomes the full manual teleoperation when user inputs, and continuous blending that autonomy prediction and user input are combined together. The experimental results show that the alternating method has the same objective performance as the continuous one in the task, but in subjective ratings, the participants believe that the former one had higher ease of use and a higher sense of control intuition.

\subsection{Pose Remapping in Shared Control System}
The alternating shared control is realized by setting the status switching trigger on the VR controller, and processing the positional and directional gaps between the two states.
% The alternating shared control is realized by selecting the target grasping pose from candidates, setting the status switching trigger on the VR controller, and processing the positional and directional gaps between the two states.
% The alternating shared control is realized by generating multi-directional grasping poses as target candidates, selecting the best grasping pose based on some laws (presented in section 2.3), setting the status switching trigger on the VR controller, and processing the positional and directional gaps between the two states.
% No matter from manual control to automation or from automation to manual mode, the pose remapping approaches are in the same, divided into the position remapping and the spheical linear interpolation (SLERP) for rotations.
In our system, the pure teleoperation part establishes a transformation from the VR coordinate system ($\boldsymbol{p}_{hand,0}^{htc}$, $\boldsymbol{p}_{hand,k}^{htc}$, $0$ means the initial step, $k$ is for the k'th sampling step) to the robot base coordinate system ($\boldsymbol{p}_{e,k}^b$, $\boldsymbol{p}_{e,0}^b$), by calculating the relative position: $\boldsymbol{p}_{e,k}^b = \boldsymbol{p}_{e,0}^b + (\boldsymbol{p}_{hand,k}^{htc}-\boldsymbol{p}_{hand,0}^{htc})$.
% \begin{equation}
%     \boldsymbol{p}_{e,d,k}^b = \boldsymbol{p}_{e,0}^b + (\boldsymbol{p}_{hand,k}^{htc}-\boldsymbol{p}_{hand,0}^{htc})
% \end{equation}
For the rotational motion mapping, the absolute orientation related to the robot base frame is used as $\boldsymbol{R}_{e,k}^b = \boldsymbol{R}_{htc}^b \cdot \boldsymbol{R}_{hand,k}^{htc}$.
% \begin{equation}
%     \boldsymbol{R}_{e,d,k}^b = \boldsymbol{R}_{htc}^b \cdot \boldsymbol{R}_{hand,k}^{htc}
% \end{equation}
% where a rotation matrix $\boldsymbol{R}_{htc}^b$ turns the rotation in the VR coordinate system into the robot base coordinate system. 
% No matter from manual control to automation or from automation to manual mode, the pose remapping ways are the same, divided into the position remapping and the spheical linear interpolation (SLERP) for rotations.
When the status change is triggered, the position remapping is as follows.
\begin{equation}
    \boldsymbol{p}_{e,0}^b=\boldsymbol{p}_{e,k}^b
\end{equation}
\begin{equation}
    \boldsymbol{p}_{hand,0}^{htc}=\boldsymbol{p}_{hand,k}^{htc}
\end{equation}
Meanwhile, the new absolute orientation is filtered by the spheical linear interpolation (SLERP). 
\begin{equation}
    \boldsymbol{R}_{e,k}^b=SLERP(\boldsymbol{R}_{e,k-1}^b,\boldsymbol{R}_{htc}^b \cdot \boldsymbol{R}_{hand,k}^{htc},\alpha)
    \label{eq:orientation_blending}
\end{equation}
% where, $\boldsymbol{\alpha}$ is the interpolation parameter between $0$ and $1$, representing linear placement position from $\boldsymbol{R_{e,k}^b}$ to $\boldsymbol{R_{e,d,k}^b}$. These processes can not only solve problems (1) and (2) by realizing the alternating shared control, but also directly affect problem (3), making the actions between state transitions more coherent.
where, $\boldsymbol{\alpha}$ is the interpolation parameter between $0$ and $1$, representing linear placement position from $\boldsymbol{R_{e,k}^b}$ to $\boldsymbol{R}_{htc}^b \cdot \boldsymbol{R}_{hand,k}^{htc}$. 
% These processes not only enable human intervention and enhance the user's sense of autonomy through alternating shared control but also ensure smoother actions during state transitions.
% These processes mentioned above can not only allow human intervention and improve user's sense of autonomy by realizing the alternating shared control, but also directly makes the actions between state transitions more coherent.
% 需要说明一下如何使用基于速度的自动化接近。简单用一句话说一下。

\subsection{Grasping Pose Selection Based on Human Preferences and Robot Constraints}
% Figure environment removed

% % Figure environment removed

% In \cite{zhu2023shared}, the target grasping poses had been detected from multiple directions, with template matching based object point cloud compensation. However, the one with the best robot manipulability which was selected from the library of generated grasping poses, leading to the fact that in the alternating shared control, user's preferences are not considered and the robot motion has a jump when switching from manual mode to automatic mode. 
In \cite{zhu2023shared}, the target grasping poses were detected from multiple directions using template matching based object point cloud compensation. 
However, the final selection of the grasping pose was solely based on the best robot manipulability from the library of generated poses. 
As a result, when utilized in the alternating shared control, the user's preferences were not taken into account. 
It led to a noticeable jump in robot motion when transitioning from manual mode to automatic mode.

% It happens since the user manually manipulate the robot end-effector to a specific pose, possibly closed to the target in distance, while the selected pose has a large gap in position and direction between the human operation.

% In order to enhance user's sense of participation and make the system performing automatic assistance based on the results of human manipulation, this paper gives out a feasible solution using optimal target grasping pose selection strategy that filters out the candidate with the closest positional and directional distance in order, and then selects the most robot-operable target grasping pose among them.
% To enhance the user's sense of participation and enable the system to provide automatic assistance based on human manipulation outcomes, this paper proposes a feasible solution. The solution follows these steps: 
To enable the system to provide automatic assistance based on human manipulation outcomes, this paper proposes a feasible solution. The solution follows these steps: 
Firstly, for each object, 150 reliable generated grasping poses are stored when the object is not visually occluded by the manipulator. 
Secondly, it filters out candidates with the closest positional and directional distances in sequence. 
Thirdly, it selects the most robot-operable target grasping pose among them.
% In the pre-processing part, for each object, 150 reliable generated grasping poses are stored when the object is not visually occluded by the manipulator. 
The reason is to avoid wrong grasping poses being included in the grasping pose library. 
Given the current quaternions of the end-effector $\boldsymbol{q}_{ee}$ and candidates in the grasping pose list $\boldsymbol{q}_l=[\boldsymbol{q}_1,\boldsymbol{q}_2,…,\boldsymbol{q}_i]$, the system calculates the angular absolute distance for each candidate.
% \begin{equation}
%     d_{a_i}=min(\lVert \boldsymbol{q}_{ee}\pm \boldsymbol{q}_{i} \rVert)
% \end{equation}
\begin{equation}
    d_{a,i}=min(\lVert \boldsymbol{q}_{ee} + \boldsymbol{q}_{i} \rVert , 
    \lVert \boldsymbol{q}_{ee} - \boldsymbol{q}_{i} \rVert)
\end{equation}
The distance is the chord length of the shortest path that connects the two quaternions. 
The top 30 grasping poses closest in the orientation are updated in the candidate list.
Sequentially, the linear distance is calculated between the gripper position $\boldsymbol{p}_{ee}$ and the candidate $\boldsymbol{p}_i$ from the updated list.
\begin{equation}
    d_{l,i}=\lVert \boldsymbol{p}_{ee}-\boldsymbol{p}_i \rVert
\end{equation}
% The %$n_l=6$ 
The top 6 grasping poses with the shortest distances are collected into a new list.
% Knowing the current position of the end-effector $\boldsymbol{p}_{ee}$, considering candidates in the grasping pose list $\boldsymbol{p}_l=[\boldsymbol{p}_1,\boldsymbol{p}_2,…,\boldsymbol{p}_i]$, then it calculates the linear distance for each candidate: $ d_{l,i}=\lVert \boldsymbol{p}_{ee}-\boldsymbol{p}_{l,i} \rVert$. Select $n_l$ linearly closest grasping poses and update the candidate list.
% In order to enhance user's sense of participation and make the system prepare automatic assistance based on the results of manual operations, this paper gives out a feasible solution using optimal target grasping pose selection strategy that filters out the candidate with the closest position and direction distance in order, and then selects the most robot-operable target grasping pose among them. In the pre-processing part, the reliable generated grasping poses are stored when the object is not occluded by the manipulator in the view of camera to avoid wrong grasping poses being included in the depository. Knowing the current position of the end-effector $p_{ee}$, considering candidates in the grasping pose list $p_l=[p_1,p_2,p_3,…]$, then it calculates the linear distance list $d_{l,l}=[d_{l,1},d_{l,2},d_{l,3},…]$. Select $n_l$ linearly closest grasping poses and obtain list $p_{l,l}$.
% \begin{equation}
%     d_{g,i}=\lVert p_{ee}-p_{l,i} \rVert
% \end{equation}
% \begin{equation}
%     p_{l,l}=argmin(d_{l,l}) [:n]
% \end{equation}
% Sequentially, the rotation distance is then considered. The directions are in the forms of quaternions.
% To calculate the angular distances list $d_{a,l}=[d_{a,1},d_{a,2},d_{a,3},…]$ between gripper quaternion $q_{ee}$ and target candidate quaternion list $q_a=[q_1,q_2,q_3,…]$, which corresponds to the list $p_{l,l}$. Select $n_a$ angularly closest grasping poses and obtain list $p_{a,l}$.
% The angular absolute distance is calculated, finding the norm for the difference between two quaternions. Instead of calculating in the hypersphere space, the absolute distance is the chord length of the shortest path that connects the two quaternions. It follows the fact that $q$ and $-q$ represent the same rotation.
% \begin{equation}
%     d_{a,i}^+,d_{a,i}^-=\lVert q_{ee}\pm q_{i} \rVert
% \end{equation}
% \begin{equation}
%    d_{a,i}=min(d_{a,i}^-,d_{a,i}^+)
% \end{equation}
% \begin{equation}
%    d_{a,l}=argmin(d_{a,l})[:n_a] 
% \end{equation}
% Todo: specify na nl number
% Because the  in the form of absolute distance that is the norm. 
Then, the robot manipulabilities are calculated and the highest result is choosed as the target grasping pose. The penalized manipulability, which considers singular configurations $ S(\boldsymbol{\theta})$ and joint limits $L(\boldsymbol\theta)$ by $ M(\boldsymbol\theta) = S(\boldsymbol\theta)L(\boldsymbol\theta)$, is presented in \cite{zhu2023shared}.
Automatic approach towards the target is achieved through interpolation from the current pose to the target pose.
Hence, the grasping pose changes according to the user manipulation and the robot constraints (Fig. 1(a)(b)).
% Hence, the grasping pose selection strategy obtains a proper balance among human preferences, sense of autonomy and follower robot constraints.
% By multiplying the $M(\boldsymbol\theta)$ with $P(\boldsymbol\theta)$, a penalized manipulability can be obtained as follows:
% \begin{equation}
%     C(\boldsymbol\theta) = P(\boldsymbol\theta)M(\boldsymbol\theta)
%     \label{c_theta}
% \end{equation}
	\section {Experiment and Result}
As the improvement of manipulability on the follower side has been evaluated in our previous work \cite{zhu2023shared}, here we check the level of motion smoothness when the user preferences are carried out in the automatic mode.
% here we check the degree of respect for human preferences when switching control modes.
% As the improvement of manipulability in the follower side has been evaluated in our previous work \cite{zhu2023shared}, here we checked the degree of respect for human preferences when switching systems, shown as the robot joint changes coherence.
The object grasping experiment is conducted in two modes: one considering only the robot constraints, and the proposed method which considers both the robot manipulability and the robot current pose. 
In each mode, the system first automatically moves to a fixed preparation pose to simulate user operation, and then completes the grasping task in automatic mode.
The Fig. 1(c) demonstrates the transformation of the robot's end-effector pose from the ready pose to the selected grasping pose.
% It shows that the proposed grasping pose selection strategy makes the gripper moves less and smoother with a shorter completion time as the execution speed is the same ($0.1m/s$).
It shows that the proposed grasping pose selection strategy reduces gripper movements, making them smoother, and achieves a shorter completion time as the execution speed is the same ($0.1m/s$).

% experiment and picture
% The figure indicates that the joint changes decrease for , showing that the follower robot continues the human motions, and the motion becomes more coherent.

% % Figure environment removed

\section{Conclusions}
In this paper, we have applied an alternating shared control that enables human intervention, enhances the comprehensibility of the robot's motion, and smoothes the gap between control mode switching. 
Positional remapping and directional spherical linear interpolation are employed to realize the intervention-possible alternating shared control.
% The grasping pose selection, taking into account both manual operation according to human preferences and manipulability due to the robot constrains, is also uitilized to achieve an appropriate balance between user feelings in the master side and robot constraints in the follower side.
The proposed grasping pose selection takes into account both manual operation and robot manipulability. The two factors are affected by human preferences and robot constraints, respectively.
The result of an object grasping experiment shows that the shared control system extends the operators' wishes and makes the follower's motion smooth when switching control modes.
Future work includes making different degrees of freedom have different weights in the cost function of grasping pose selection, and enabling people to fine-tune the final grasping pose.
	
	
	% \section*{ACKNOWLEDGMENT}
	% This work is supported by JST AIP Grant Number JPMJCR20G8, Japan, and JSPS KAKENHI Grant Number JP22K14222.

\bibliographystyle{IEEEtran}
% argument is your BibTeX string definitions and bibliography database(s)
\bibliography{arxive}


	
	
	
\end{document}