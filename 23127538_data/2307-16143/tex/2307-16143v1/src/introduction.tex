\section{Introduction}
Magnetic resonance imaging (MRI) and computed tomography (CT) are two commonly used cross-sectional medical imaging techniques. %MRI and CT use different physical principles for producing tissue contrast, and they are often used in tandem to provide complementary information. 
MRI and CT produce different tissue contrast and are often used in tandem to provide complementary information. 
While MRI is useful for visualizing soft tissues (e.g. muscle, fat), %and neural elements), 
CT is superior for visualizing bony structures. 
Some medical procedures, such as radiotherapy for brain tumors, craniosynostosis, and spinal surgery, typically require both MRI and CT for planning. 
%However, obtaining and subsequently co-registering both scans add extra cost.
Unfortunately, CT imaging exposes patients to ionizing radiation, which can damage DNA and increase cancer risk~\cite{richardson2015risk}, especially in children and adolescents.
Given these issues, there are clear advantages for synthesizing anatomically accurate CT data from MRI.

Most synthesis methods adopt supervised learning paradigms and train generative models to synthesize CT~\cite{emami2021sa,liu2021ct,dalmaz2022resvit,zhang2022map,armanious2020medgan}.
Despite the superior performance, supervised methods require a large amount of paired data, which is prohibitively expensive to acquire. 
Several unsupervised MRI-to-CT synthesis methods~\cite{yang2020unsupervised,liu2021ct,ge2019unpaired}, leverage CycleGAN with cycle consistency supervision to %\son{(CycleGAN is not the only unsupervised method)}
eliminate the need for paired data. Unfortunately, the performance of unsupervised CT synthesis methods~\cite{yang2020unsupervised,yang2018unpaired,ge2019unpaired} is inferior to supervised counterparts.  %Specifically, they apply CycleGAN~\cite{zhu2017unpaired}, which is a state-of-the-art model for unpaired image translation. 
%While CycleGAN generates realistic images, 
%Several methods apply CycleGAN with cycle consistency loss to leverage unpaired data. However, 
Due to the lack of direct constraints on the synthetic outputs, CycleGAN~\cite{zhu2017unpaired} struggles to preserve the anatomical structure when synthesizing CT images, as shown in Fig.~\ref{fig:teaser}(b). \new{The structural distortion in synthetic results exacerbates when data from the two modalities are heavily misaligned, which usually occurs in pediatric scanning due to the rapid growth in children.}   %shows that without CycleGAN wrongfully preserves the low-intensity region corresponding to fluid at the top of the head when synthesizing CT. 
%In CycleGAN, the cycle consistency loss indirectly imposes the structural similarity between the input and the synthesized images. This often generates a mismatch in anatomical structures in the synthesized results. 
% Figure environment removed

Recent unsupervised methods impose structural constraints on the synthesized CT through pixel-wise or shape-wise consistency. Pixel-wise consistency methods \cite{yang2020unsupervised,yang2018unpaired,matsuo2022unsupervised} capture and align pixel-wise correlations between MRI and synthesized CT. %capture the image structures by computing the correlations between pixels, encouraging the synthesized CT to have a similar structure as the input MRI. 
However, enforcing pixel-wise consistency may introduce undesirable artifacts in the synthetic results.
% as the real CT and MRI can have different spatial and noise statistics.
This problem is particularly relevant in brain scanning, where both the pixel-wise correlation and noise statistics in MR and CT images are different, as a direct consequence of the signal acquisition technique. 
%, \son{and noise statistics as a direct consequence of the signal acquisition technique},. %However, pixel-wise structures in the real CT differ from the MRI in certain areas, such as the brain tissue. Thus, this approach may create undesired artifacts in the synthetic CT.  %However, pixel-wise structures in the real CT can be different from the MRI in certain areas. For example, brain structures are smooth in the CT, while having complex textures in MRI. Thus, imposing MRI-like pixel-wise consistency may create undesired artifacts in the synthetic CT. %Instead of constraining pixel-wise correlations, 
The alternative shape-wise consistency methods~\cite{emami2021sa,ge2019unpaired,zhou2021anatomy}  
aim to preserve the shapes of major body parts in the synthetic image. Notably, shape-CycleGAN~\cite{ge2019unpaired} segments synthesized CT and enforces consistency with the ground-truth MRI segmentation. %enforces the segmentation to be consistent with the ground-truth MRI segmentation. %Then they minimize the discrepancy between the extracted segmentation and the ground-truth segmentation of the input MRI. 
However, these methods rely on segmentation annotations, which are time-consuming, labor-intensive, and require expert radiological annotators. %They also train additional segmentation extractor networks, which leads to computational overhead. 
A recent natural image synthesis approach, called AttentionGAN~\cite{tang2021attentiongan}, learns attention masks to identify discriminative structures. %It then synthesizes only the foreground structures. 
AttentionGAN implicitly learns prominent structures in the image without using the ground-truth shape. %In contrast to previous shape-aware methods, AttentionGAN only uses the coarse foreground masks as a hint to roughly locate the foreground structures. 
%AttentionGAN indirectly learns the foreground masks by minimizing the standard adversarial loss on the final synthetic outputs.  %AttentionGAN implicitly enforces the shape consistency when synthesizing images. 
%Despite not requiring ground-truth masks, 
Unfortunately, the lack of explicit mask supervision can lead to imprecise attention masks and, in turn, produce inaccurate mappings of the anatomy, as shown in Fig.~\ref{fig:teaser}(c).


 % \vspace{-0.5em}
\begin{table}[!h]
\centering
\renewcommand{\arraystretch}{1.05}
\setlength\tabcolsep{10.00pt}
\caption{Comparisons of different shape-aware image synthesis.}
\label{tab:compare}
\begin{tabular}{cccc}\toprule
\textbf{Method} &  \begin{tabular}{@{}c@{}}Mask \\ Supervision\end{tabular} & \begin{tabular}{@{}c@{}}Human\\ Annotation\end{tabular} & \begin{tabular}{@{}c@{}}Structural\\ Consistency\end{tabular}  \\ \midrule
Shape-cycleGAN~\cite{ge2019unpaired} & Precise mask & Yes & Yes \\
AttentionGAN~\cite{tang2021attentiongan} & Not required & No & No \\
\textbf{MaskGAN (Ours)} & \textbf{Coarse mask} & \textbf{No} & \textbf{Yes} \\ \bottomrule
\end{tabular}% 
%\vspace{-1em}
\end{table}


% % Figure environment removed
%\vspace{-1em}

In this paper, we propose \textbf{MaskGAN}, a novel unsupervised MRI-to-CT synthesis method, that preserves the anatomy under the explicit supervision of coarse masks without using costly manual annotations. %and a new cycle shape consistency loss. 
Unlike segmentation-based methods~\cite{ge2019unpaired,zhang2018translating},
MaskGAN bypasses the need for precise annotations, replacing them with standard (unsupervised) image processing techniques, which can produce coarse anatomical masks. 
Such masks, although imperfect, provide sufficient cues for MaskGAN to capture anatomical outlines and produce structurally consistent images. Table~\ref{tab:compare} highlights our differences compared with previous shape-aware methods~\cite{ge2019unpaired,tang2021attentiongan}.
%This contrasts with previous methods, as shown in  Table~\ref{tab:compare}.
%our approach bypasses the need for precise human annotations by using automatically extracted coarse masks. Table~\ref{tab:compare} gives an overview of our method in contrast to other shape-aware methods. Standing out from recent works, MaskGAN preserves the anatomy without relying on human annotation. 
%we automatically extract the coarse ground-truth masks using image processing algorithms without relying on expensive human-annotated segmentation masks.  %We introduce a robust image processing technique based on the connected component analysis algorithm to extract the background masks from CT and MR scans. 
%Additionally, we propose a cycle shape consistency (CSC) loss to encourage the image to have a consistent foreground mask during translation. 
Our major contributions are summarized as follows. \textbf{1)} We introduce \textbf{MaskGAN}, a novel unsupervised MRI-to-CT synthesis method. MaskGAN is the first framework that maintains shape consistency without relying on human-annotated segmentation.  \textbf{2)} We present two new structural supervisions to enforce consistent extraction of anatomical structures across MRI and CT domains. \textbf{3)} Extensive experiments show that our method outperforms state-of-the-art methods by using automatically extracted coarse masks to effectively enhance structural consistency.



















