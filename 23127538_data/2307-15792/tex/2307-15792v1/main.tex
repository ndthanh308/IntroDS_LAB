%% ****** Start of file template.aps ****** %
%%
%%
%%   This file is part of the APS files in the REVTeX 4 distribution.
%%   Version 4.0 of REVTeX, August 2001
%%
%%
%%   Copyright (c) 2001 The American Physical Society.
%%
%%   See the REVTeX 4 README file for restrictions and more informa≠=tion.
%%
%
% This is a template for producing manuscripts for use with REVTEX 4.0
% Copy this file to another name and then work on that file.
% That way, you always have this original template file to use.
%
% Group addresses by affiliation; use superscriptaddress for long
% author lists, or if there are many overlapping affiliations.
% For Phys. Rev. appearance, change preprint to twocolumn.
% Choose pra, prb, prc, prd, pre, prl, prstab, or rmp for journal
%  Add 'draft' option to mark overfull boxes with black boxes
%  Add 'showpacs' option to make PACS codes appear

\documentclass[aps,prl,twocolumn,showpacs,superscriptaddress]{revtex4-2}  

\usepackage{float}
\usepackage{bbm}
\usepackage{epsfig}
\usepackage{epstopdf}
\usepackage{graphicx}
\usepackage{amsmath,amssymb}
\usepackage{amsmath,bm}
\usepackage{physics}
\usepackage{color}
\usepackage{hyperref}
\usepackage{lineno,blindtext}
\setlength{\tabcolsep}{9pt}
\usepackage{siunitx, booktabs}
\usepackage{diagbox, eqparbox, hhline}
\usepackage{soul}
\usepackage{xcolor}
\usepackage[normalem]{ulem}
\setlength{\doublerulesep}{2.5pt}

\newcolumntype{P}[1]{>{\centering\arraybackslash}p{#1}}
\newcommand{\new}[1]{\textcolor{blue}{#1}}          % for markup
\newcommand{\cut}[1]{\textcolor{red}{\sout{#1}}}          % for markup
\newcommand{\comment}[1]{\textcolor{green}{#1}}          % for markup
\newcommand{\ans}[1]{\textcolor{orange}{#1}}




\begin{document}

\title{Reciprocal Reservoir Induced Non-Hermitian Skin Effect}
\begin{abstract}
%The non-Hermitian skin effect (NHSE), which describes a state accumulation at a specific boundary, is inherently tied to nonreciprocity.
 The non-Hermitian skin effect (NHSE), which describes the localization of macroscopic fraction of eigenstates at a specific boundary, is inherently tied to nonreciprocity. 
Here, we show that the NHSE can be engineered in an open magnetic system interacting with a reciprocal reservoir, through the interplay between the reservoir-induced coherent and dissipative couplings.  Based on a Lindbladian time evolution, we investigate the transient nonreciprocal dynamics along a spin chain and its unidirectional limit, which allow us to reveal both consistency with and limitation of the non-Hermitian Hamiltonian approach. 
We comment on the connection to the semiclassical dissipative magnetization dynamics, and identify the key ingredients underlying the NHSE in magnetic systems as Dzyaloshinskii-Moriya interaction (DMI) and dissipative coupling, analogous to those in multilayered magnet-metal heterostructures. 
Our work suggests the generality of nonreciprocal dynamics in  magnetic systems and may inspire new schemes for engineering nonreciprocity in other quantum platforms.
%Our findings illuminate the correlation between the non-Hermitian skin effect and classical magnetization dynamics, indicating that our predictions can be experimentally verified in multilayered magnetic structures with interlayer Dzyaloshinskii-Moriya interactions.   
%Based on a Lindbladian time evolution, we investigate the transient nonreciprocal dynamics along a spin chain, revealing both consistency with and limitation of non-Hermitian Hamiltonian approach.
% associated with the complex-spectrum topology.
% in magnetic systems can be induced by a reciprocal reservoir.  taking the example of a spin chain interacting with a shared reservoir, 
% exhibit localized behaviors drastically different from the extended Bloch waves of Hermitian systems, 
% is among the most scrutinized dissipative phenomena. 
% The localization of the eigenstates at a system's edge hints at nonreciprocal transport towards the latter. 
% However, non-Hermitian Hamiltonians are obtained as an approximation of more complex dynamics, which might wash out critical information on the timescales and conditions required for the nonreciprocal dynamics to manifest. 
% Using a symmetry-based approach, we derive a master equation that reduces, by neglecting quantum jumps, to the non-Hermitian Hamiltonian under investigation, and we also address the role played by quantum jumps in the single-quasiparticle dynamics. 
% Our analyses uncover the limitation of non-Hermitian approaches and
%We comment on the connection to the semiclassical dissipative magnetization dynamics, and identify the key ingredients underlying the NHSE in magnetic systems as Dzyaloshinskii-Moriya interaction (DMI) and dissipative coupling, analogous to those in multilayered magnet-metal heterostructures. 

\end{abstract}
\author{Xin Li}
\affiliation{Department of Physics, Boston College, 140 Commonwealth Avenue Chestnut Hill, MA 02467, USA}
\author{Mohamed Al Begaowe}
\affiliation{Department of Physics, Boston College, 140 Commonwealth Avenue Chestnut Hill, MA 02467, USA}
\author{Shu Zhang}
\affiliation{Max Planck Institute for the Physics of Complex Systems, 01187 Dresden, Germany}
\author{Benedetta Flebus}
\affiliation{Department of Physics, Boston College, 140 Commonwealth Avenue Chestnut Hill, MA 02467, USA}


\date{\today}% It is always \today, today,
             %  but any date may be explicitly specified
\maketitle

% Explore the common ground between quantum optics and spintronics~\cite{chelpanova2021}.

% we examine the unidirectionality 



% Countless physical phenomena and applications rely on the dynamics of open quantum systems whose excitations experience a finite lifetime due to dissipative interactions with the surrounding environment. In macroscopic settings, the dynamics of open systems are often well-described by semiclassical approaches in which dissipation is encapsulated in a non-Hermitian Hamiltonian~\cite{ashida2020non,bender2007making}.


The advent of non-Hermitian notions has drastically extended our understanding of dynamical open systems~\cite{ashida2020non,bender2007making,el2018non,bergholtz2021exceptional}. 
Non-Hermitian approaches have shown that dissipation, perceived for a long time solely as a foe, can induce far richer and more significant phenomena than finite quasiparticle lifetimes.
One of particular interest is the NHSE|a nonreciprocal accumulation of the eigenmodes at open boundaries, which contradicts Bloch band theory and the conventional bulk-boundary correspondence~\cite{bergholtz2021exceptional,zhang2022review}.  
Heretofore, the NSHE has been proposed and studied in various platforms, including electrical~\cite{liu2021non} and topoelectrical circuits~\cite{hofmann2020reciprocal}, mechanical metamaterials~\cite{ghatak2020observation}, cold-atom~\cite{liang2022dynamic},  photonic~\cite{xiao2020non,weidemann2020topological}, acoustic~\cite{zhang2021acoustic}, and magnonic~\cite{yu2023non} systems, with promising applications in quantum sensing and signal amplification~\cite{Bao2022,Budich2020,Koch2022,el2015optical,Koutserimpas2018,Wang2022}.


The essence of the NHSE lies in nonreciprocity. On a model level, it is intuitive that a directionally biased particle hopping rate can lead to state accumulation at the preferred boundary. 
In practice, nonreciprocity typically needs to be engineered in an open system. Taking, e.g., cavity-based photonic systems, this can be achieved using synthetic gauge fields, structured external drives, or chiral damping~\cite{fang2012realizing,ranzani2015graph,song2019non}. 
The growing interest in non-Hermitian physics in magnetic systems makes it highly desirable to develop a clear framework for reservoir engineering of the magnetic NHSE that would shed light on its generality and feasibility~\cite{li2022multitude,PhysRevB.105.104433,mcclarty2019non,flebus2020non,PhysRevB.105.L180401,yuan2022master, deng2022non,hurst2022non,deng2023exceptional,lu2021magnetic,zhang2020dynamic}. The questions are twofold:  Further clarification is needed on the minimal physical ingredients required for a magnetic system to exhibit the NHSE and whether 
% the latter survives 
it persists in
the full Liouvillian dynamics including quantum jumps~\cite{song2019non,longhi2020unraveling}. 
Since dissipative magnetization dynamics are often discussed in spintronic platforms, it is worth exploring the connection between the nonreciprocal dynamics therein and the NHSE.


% The skin effect is commonly reported in systems with directional coherent or incoherent interactions, e.g., models with an asymmetry in the coherent hopping from left-to-right, versus right-to-left, or with chiral damping \cite{song2019non,yi2020non,liu2020helical}. In our model~(\ref{eq2}), the coherent, $\propto J \pm iD$, and the nonlocal dissipative, $\propto \Gamma$,  couplings  are both reciprocal, i.e., they allow hopping in both directions between two nearest neighbors. It is the balancing between them that yields nonreciprocity, as it can be easily visualized by redefining $J + i D= \tilde{J}e^{i\phi}$ and rewriting the nonlocal components of Eq.~(\ref{eq2}) for $\alpha=1,2$ as


% This mechanism underlying nonreciprocity is intimately connected with the reservoir engineering approaches proposed by Refs.~\cite{metelmann2015nonreciprocal,fang2017generalized} for constructing nonreciprocal photonic devices. In these setups, synthetic gauge fields are introduced by using nonlinearities and external drives. 

In this work, we show that the magnetic NHSE stems from the interplay between coherent chiral interactions and  dissipative couplings, which can be induced by a reciprocal reservoir~\cite{metelmann2015nonreciprocal,fang2017generalized,clerk2022introduction}. 
The mechanism is then examined in the Lindbladian description of magnon dynamics, where the quantum jumps are included to correctly describe the time evolution of two-point correlation functions. 
The nonreciprocal dynamics stay valid, though we show that the non-Hermitian approach hinders the timescales reflecting causality and locality~\cite{bergholtz2021exceptional,borgnia2020non,okuma2020topological,lee2022anomalously}. 
We also investigate the connection to the generalized Landau–Lifshitz–Gilbert (LLG) equation, a phenomenological classical equation widely used to study dissipative magnetization dynamics, for example, in a magnetic$|$metallic heterostructure, and discuss the absence of the unidirectional limit of nonreciprocity in this system. 




In an interacting spin system, the coherent chiral interaction can be an effective DMI resulting from inversion-symmetry breaking, and the dissipative coupling naturally arises between spins coupled to a shared reservoir. This suggests that nonreciprocity and the associated NHSE can be statically engineered and commonly exist in open spin systems.  Our study also sheds light on the correspondence between quantum and classical modeling of dissipative magnetic dynamics.


% An open system is typically treated as a subsystem of a larger, closed and unitarily evolving system. However, since the subsystem interacts with other parts, its dynamics can result in non-norm-preserving behaviors. In the weak coupling limit, the subsystem can be approximated by an effective non-Hermitian Hamiltonian, but this raises the question of whether higher order contributions to the dynamics will cause non-Hermitian effects to disappear~\cite{longhi2020unraveling,lee2022anomalously}. Here we address this question by taking as a case study of the magnetic non-Hermitian skin effect, recently predicted in a broad class of systems displaying both non-local reciprocal dissipation and Hermitian chiral interactions, such as Dzyaloshinskii-Moriya interactions (DMI)~\cite{deng2022non,hurst2022non}. 
% As for several other works, the starting point adopted in Ref.~\cite{deng2022non} relies on a phenomenological non-Hermitian Hamiltonian derivation and does not offer a comprehensive insight on how and under which conditions quasi-excitations generated in the system’s bulk would accumulate at a boundary, leading to a detectable experimental signal. For instance, one might expect that, if the local damping is much stronger than nonlocal interactions, spin waves generated in the middle of the chain might not be able to reach its ends. However, we show that a non-Hermitian analyis based on the winding number hinders the timescales reflecting casuality and locality~\cite{bergholtz2021exceptional,borgnia2020non,okuma2020topological}. To go beyond the non-Hermitian phenomenology,  we derive a master equation that reduces, by neglecting quantum jump terms, to a non-Hermitian Hamiltonian displaying the skin effect by virtue of the same ingredients of Ref.~\cite{deng2022non}.  While our conclusions, which highlight the limitations of a non-Hermitian description, are based on detailed studies of a specific magnetic system, we believe them to be generic to any models.



% We also uncover the origin of the magnetic skin effect and the ensuing nonreciprocity, which  lies in the balancing of coherent interactions arising from nontrivial gauge fields, i.e., DMI, with nonlocal dissipative couplings. We show that the key ingredients for the emergence of the magnetic skin effect are non-Hermiticity and time-reversal symmetry (TRS)-breaking of the magnon Hamiltonian.  This result is intimately connected with recent reservoir engineering approaches in which gauge synthetic fields are introduced via external drives ~\cite{metelmann2015nonreciprocal,fang2017generalized}. Our work suggests that nonreciprocity can be induced as well by static symmetry-breaking mechanisms, opening up new pathways for engineering cascade quantum systems. 


 % In our model and in Ref.~\cite{deng2022non}, effective DM interactions emerge as a result of a spatial symmetry breaking, suggesting that nonreciprocity can be statically engineered.



% Figure environment removed




\textit{Model.} 
The model under consideration is a one-dimensional (1$d$) array of $N$ spins weakly coupled with a common reservoir, as sketched in Fig.~\ref{fig:master}(a).
 The reservoir induces a local dissipation for the dynamics of each spin, and it mediates coherent and dissipative couplings between the spins~\cite{zou2022bell}.
 % We assume axial symmetry around the $\hat{\mathbf{z}}$ axis in spin space and  retain only  nearest-neighbor (NN) bath-mediated interactions between the macrospins.  
 % Neglecting thermal effects, which do not play a relevant role in our discussion, 
 We capture these effects using a Lindblad master equation for the density matrix $\rho$ of the spin array, which results from perturbatively treating the exchange interactions between the spin system and the reservoir under the Born-Markov approximation, i.e., 
\begin{eqnarray}
\frac{d\rho}{dt}= -i \left[ \mathcal{H}, \rho \right] - \mathcal{L}[\rho] \, . \label{eq1}
\end{eqnarray}
Here the Hamiltonian $\mathcal{H}$ and the Lindblad dissipator $\mathcal{L}[\rho]$ respectively describe the coherent and dissipative time evolution, i.e.,
\begin{eqnarray}
&&\mathcal{H}=\sum_{\alpha=1}^N\omega_\alpha \hat{s}^z_\alpha+\frac{1}{4}\sum_{\substack{\alpha\neq\beta:\\ \alpha,\beta=1}}^N 
\left( \hat{s}_{\alpha}^+ 
J_{\alpha\beta} \, \hat{s}_{\beta}^- \, \label{hamiltonian} +\text{h.c.} \right), \\
&&\mathcal{L}[\rho] =\frac{1}{2}\sum_{\alpha,\beta=1}^N\Gamma_{\alpha \beta}\left(\{\hat{s}_\alpha^-\hat{s}_\beta^+,\rho\}-2\hat{s}_\beta^+\rho \hat{s}_\alpha^-\right) \, .\label{eqlin}
\end{eqnarray}
% with $\{\hat{s}_\alpha^-\hat{s}_\beta^+,\rho\}=\hat{s}_\alpha^-\hat{s}_\beta^+\rho+\rho\hat{s}_\alpha^-\hat{s}_\beta^+$,
The $N$ spins are originally isolated, with a Zeeman splitting of frequency $\omega_\alpha$ at site $\alpha$. The dimensionless spin operators are defined as $\hat{s}^{\pm}_\alpha\equiv \hat{s}^{x}_\alpha\pm i\hat{s}  ^{y}_\alpha$.  
Individually, each spin interacting with the reservoir causes an energy shift, which is absorbed into $\omega_\alpha$,
and a local spin relaxation rate $\Gamma_{\alpha \alpha}$. 
Collectively, for a pair of spins at sites $\alpha \neq \beta$, the shared reservoir mediates a coherent coupling  $J_{\alpha\beta}$ and a cooperative decay of rate $\Gamma_{\alpha \beta}$. These parameters are determined by the correlation functions of the degrees of freedom in the reservoir. Since the induced Ising-type interactions $\propto s_\alpha^zs_\beta^z$,  giving rising to pure dephasing effects, are dominated by reservoir fluctuations at much lower frequencies~\cite{zou2022bell}, we have neglected them assuming a gapped reservoir. 
In this study, we focus on the zero-temperature limit, where all thermal excitation processes in the reservoir are fully suppressed. However, our results are generally applicable to the finite-temperature regime with the parameters modified accordingly~\cite{supplementary}.

To clarify the essential features in later discussions, we make the following simplifications.
We set the Zeeman frequencies and local relaxation rates to be uniform, $\omega_\alpha = \omega_0$ and $\Gamma_{\alpha \alpha} = \Gamma_0$, and consider the induced couplings between nearest-neighbor sites only: for $\beta = \alpha + 1$, $\Gamma_{\alpha \beta} = \Gamma$ and $J_{\alpha\beta} = J+iD$; for $\beta = \alpha - 1$, $\Gamma_{\alpha \beta} = \Gamma^*$ and $J_{\alpha\beta} = J-iD$. Here, $J,\; D,\; \omega_0\;$ and $\Gamma_0$ are purely real and $\Gamma$ is assumed complex.
It is natural to identify $J$ and $D$, respectively, as an isotropic $XY$ interaction, assuming an axial symmetry around the $z$ axis in the spin space, and a DMI  allowed by a broken inversion symmetry.


% Here, the first term on the right-hand-side of Eq.~(\ref{hamiltonian})  is the Hamiltonian of an array of $N$ isolated macropsins, with frequencies  $\omega_\alpha$ at the $\alpha$th site.  $\hat{\mathbf{s}}_{\alpha}=(\hat{s}_{\alpha}^-,  \hat{s}_{\alpha}^+,  \hat{s}_{\alpha}^z )^T$ represents the dimensionless operator vector,  where the superscript $T$ means transpose, and the operators $\hat{s}_\alpha^{\pm}$ are defined as $\hat{s}^{\pm}_\alpha\equiv \hat{s}^{x}_\alpha\pm i\hat{s}  ^{y}_\alpha$ . 
% $\mathcal{J}_{\alpha\beta}$ is a diagonal matrix,
% $\mathcal{J}_{\alpha\beta}\equiv\text{diag}\{J-iD,J+iD,2J^z\}$, where $\text{diag}\{J,J, 2J^z\}$ and $\text{diag}\{iD,-iD, 0\}$  parameterize, respectively, the coherent XXZ Heisenberg exchange and DM interactions mediated by the bath. 
% For simplicity, we set  $\omega_0 \equiv \omega_{\alpha}$ and $J=J^{z}$.  In the same spirit, we set the incoherent couplings between spins to $\Gamma_{\alpha \beta}\equiv \Gamma$ for $\beta = \alpha\pm1$, while, for $\alpha=\beta$, $\Gamma_{\alpha \alpha}\equiv \Gamma_{0}$  reduces to the relaxation rate of an isolated macrospin which independently interacts with the magnetic bath. 

Generally, the dissipative coupling $\Gamma$ is complex. 
To ensure the stability of the system, the evolution generated by Eq.~(\ref{eqlin}) should be positive semidefinite, implying  $\Gamma_0\geq2|\Gamma|$~\cite{supplementary,breuer2002theory}. Physically, this means that the local decay of a spin is sufficiently rapid  to dissipate the energy transferred from its neighboring spins~\cite{supplementary}.
Assuming reciprocity of the correlation functions in the reservoir would impose a vanishing $\Im\Gamma$~\cite{seif2022prl,zou2022bell}. 
Interestingly, as we will demonstrate below, a real-valued $\Gamma$ alone is sufficient to induce a non-Hermitian skin effect. 

% In the limit $N \rightarrow \infty$,  the evolution generated by Eq.~(\ref{eqlin}) is  positive semidefinite~\cite{breuer2002theory} only if $\Gamma \leq 2 \Gamma_{0}$ \cite{supplementary}. Physically, this means that the stability of the system is guaranteed when the on-site loss $\Gamma_{0}$ of a macrospin is strong enough to dissipate the energy pumped by its two neighbors. 
{\it Effective non-Hermitian Hamiltonian.}
We first examine the spin-chain dynamics under an approximated non-Hermitian quadratic Hamiltonian.
It is convenient to take a boson (magnon) description by performing the Holstein-Primakoff transformation and approximate the master equation (\ref{eq1}) into a quadratic form using $\hat{s}^{+}_{\alpha} \approx \sqrt{2s} \hat{a}^\dag_{\alpha} $ and $\hat{s}^{z}_{\alpha}= -s + \hat{a}^{\dagger}_{\alpha} \hat{a}_{\alpha}$, which applies to macrospins with small excitation numbers, i.e. $s \gg 1$ and $n_{\alpha}\equiv \langle \hat{a}^\dag_{\alpha}\hat{a}_{\alpha}\rangle \ll s$. A further approximation of neglecting the quantum jumps ($\sim\hat{a}_\beta^+\rho \hat{a}_\alpha^-$)  leads to dynamics governed by a non-Hermitian Hamiltonian of a tight-binding form~\cite{supplementary}
% Neglecting the quantum jump terms $\propto \hat{s}_\beta^+\rho \hat{s}_\alpha^-$ in Eq.~(\ref{eqlin}), the right-hand side of  Eq.~(\ref{eq1}) can be recast as an effective Hamiltonian $\mathcal{H}_{nh}$,  
\begin{align}
\mathcal{H}_{\text{nh}}
% =\mathcal{H}_{nh}^0
=\mathcal{H}_{nh}^0
+ \sum_{\alpha=1}^{N-1} \left(\gamma_L\hat{a}^\dag_\alpha\hat{a}_{\alpha+1}  +\gamma_R\hat{a}^\dag_{\alpha+1}\hat{a}_\alpha\right),
\label{eq2}
\end{align}
where $\mathcal{H}_{nh}^0=\sum_{\alpha=1}^{N} (\omega_0- is\Gamma_0)\hat{a}^\dag_\alpha\hat{a}_\alpha$ captures the effective on-site potential, $\gamma_L\equiv s\left(J+iD-i\Gamma \right)$  and $\gamma_R\equiv s\left( J-iD-i\Gamma^*\right)$ are the left- and  right-hopping amplitudes, respectively. 
% $|\gamma_L|\neq|\gamma_R|$.
% For an open system, the Hermiticity condition $\gamma_L=\gamma^*_R$ is relaxed, and, thus, there can be an asymmetry in the amplitude for hopping from left-to-right, versus right-to-left, i.e., 
% ~\cite{hatano1997vortex}. 
% Such an asymmetry is known to result in an accumulation of a macroscopic number of modes on one side of the system, i.e., the  skin effect.
The non-Hermitian Hamiltonian~(\ref{eq2}) takes a tridiagonal Toeplitz matrix form and the eigenvectors can be solved analytically~\cite{supplementary}. For an  array of $N$ macrospins with open boundary conditions, the probability density distribution of the right and left eigenvectors is, respectively,
\begin{align}
|\psi^R_{n,\alpha}|^2&=\left\lvert \frac{\gamma_{R}}{\gamma_{L}}\right\rvert^{\alpha}\left(\sin\frac{n\alpha\pi}{N+1}\right)^2,\label{modeprobleft}\\
|\psi^L_{n,\alpha}|^2&=\left\lvert \frac{\gamma_{L}}{\gamma_{R}}\right \rvert^{\alpha}\left(\sin\frac{n\alpha\pi}{N+1}\right)^2,
\label{modeprobright}
\end{align} 
where $n$ labels the eigenmodes. 
Clearly,
% from Eqs.~(\ref{modeprobleft}) and~(\ref{modeprobright}) 
when $|\gamma_R/\gamma_L| \neq1$, the eigenmodes tend to accumulate towards one of the two edges of the array, which is known as the NHSE. 
The effect is maximized at $J = \pm \Im\Gamma$ and $D = \mp \Re\Gamma$, where the hopping becomes unidirectional and the wavefunctions show a perfect localization at the edge.
For $\Re\Gamma = 0$, we have the typical Hatano-Nelson model (without randomness), where the $\Im\Gamma$ provides the nonreciprocal hopping component. In this work, we highlight the (nonreciprocal) skin effect induced by a reciprocal reservoir, and, from now on, impose a real-valued $\Gamma = \Gamma^*$.



% As also shown in Fig.~\ref{fig:master}(b), nonreciprocal accumulation requires both finite DMI and nonlocal dissipation, which is consistent with the results of Ref.~\cite{deng2022non}. 
% The localization at the edge of the bulk states is maximized when $J=0$ and $D=\pm\Gamma$, while,  when the strength of symmetric-exchange-interactions greatly exceeds the antisymmetric ones, i.e., $J \gg |D|$,  the skin effect tends to disappear as $|\psi^{R(L)}_{n,\alpha}|^2\rightarrow\left(\sin\frac{n\alpha\pi}{N+1}\right)^2$.

The NHSE can be associated with a topological winding number of the complex energy dispersion under the periodic boundary condition~\cite{borgnia2020non,okuma2020topological,zhang2020correspondence}.
Diagonalizing Eq.~(\ref{eq2}) in the Fourier space yields  $\mathcal{H}_\text{nh}(k)=\sum_k \epsilon_k \hat{a}^\dag_{k}\hat{a}_{k}$~\cite{supplementary}, with
\begin{eqnarray}
   \epsilon_k= 2s\left[ (J - i \Gamma)\cos k-D\sin k\right]\,,\label{eq3}
\end{eqnarray}
where we have set the equidistance between two nearest spins as $a=1$ and the on-site potential term as the reference point.
Here, we observe a direct correlation between  the degree of the edge localization of the bulk states and the circularity of the winding numbber loop, as shown in Figs.~\ref{fig:master}(b) and (c), as both are controlled by $\Gamma$. The unidirectional limit corresponds to a circular loop  (orange curve). 

% Figure environment removed



\textit{Origin of the NHSE.} The NHSE is commonly reported in systems with directional coherent or incoherent interactions, e.g., models with an asymmetry in the left and right coherent hoppings, or with chiral damping \cite{song2019non,longhi2020unraveling, yi2020non,liu2020helical}. In our model~(\ref{eq2}), the coherent, $\propto J \pm iD$, and the nonlocal dissipative, $\propto \Gamma$ (real-valued),  couplings   allow hopping in both directions between two nearest neighbors. It is the balancing between them that yields nonreciprocity, as it can be easily visualized by redefining $J + i D= \tilde{J}e^{i\phi}$ and rewriting  Eq.~(\ref{eq2}) for $\alpha=1,2$ as
\begin{align}
\mathcal{H}_\text{eff}=\begin{pmatrix} \epsilon_{0} & \tilde{J} e^{i\phi} - i\Gamma \\ \tilde{J} e^{-i\phi} - i\Gamma & \epsilon_{0}\end{pmatrix}.
\end{align}
For $\phi\neq 0$, i.e., $D\neq 0$, the propagation between the two sites is non-reciprocal, while for $\phi=\pm\pi/2$ (i.e., $J=0$) and $\Gamma=\pm\tilde{J}$ the hopping becomes purely unidirectional. 

Our results show that the key ingredients for the emergence of the NHSE in magnetic systems are the nonlocal dissipation and the complex coherent hoppings. At the level of the magnon Hamiltonian that is expanded with respect to a time-reversal-broken ground state, the DMI  breaks time-reversal symmetry (TRS), leading to a nontrivial phase in the coherent hopping that cannot be gauged out~\cite{kim2016realization,koch2010time}.
This mechanism underlying nonreciprocity is intimately connected with the reservoir engineering approaches proposed by Refs.~\cite{metelmann2015nonreciprocal,fang2017generalized} for constructing nonreciprocal photonic devices. In these setups, synthetic gauge fields are introduced by using nonlinearities and external drives. 
In our model and in Ref.~\cite{deng2022non}, effective DMI emerges as a result of an inversion symmetry breaking, suggesting that nonreciprocity can be statically engineered. Furthermore, our finding has the important implication that the NHSE can exist very generally in open magnonic systems: Firstly, DMI commonly exists as part of the exchange interactions either in noncentrosymmetric magnetic systems or at magnetic surfaces. Second, the reservoir can be reciprocal and essentially featureless. Therefore, our discussion here generally applies to an inversion-breaking dissipative magnetic system.

















%For magnetic skin effect to arise in our dissipative spin system, the interplay between the complex coherent hopping (nonzero DMI) and the nonlocal dissipation $\Gamma$ is essential, and a real-valued $\Gamma$ from coupling to a reciprocal reservoir suffices.
%To gain a further intuition, we can rewrite the hopping Hamiltonian into an Su-Schrieffer-Heeger form, though no sublattice unbalance is present in our model. For two neighboring sites,
%\begin{align}
%\mathcal{H}_\text{12}=\begin{pmatrix} \epsilon_{0} & \tilde{J} e^{i\phi} - i\Gamma \\ \tilde{J} e^{-i\phi} - i\Gamma & \epsilon_{0}\end{pmatrix},
%\end{align}
%where $\tilde{J}e^{i\phi} = J + i D$.
%As a result of a nonzero $\Gamma$, the phase $\phi$ cannot be gauged away even at the local level. 

%An asymmetric density accumulation can arise from an unbalanced imaginary part in the left- and right- hopping rates, which is as effective in inducing a non-Hermitian skin effect as the intuitive picture of having unbalanced real hopping rates as in the Hatono-Nelson model.
%This has the important implication that skin effects can exist very generally in open magnonic systems: Firstly, DMI commonly exists as part of the exchange interactions either in noncentrosymmetric magnetic systems or at magnetic surfaces. Second, the reservoir can be reciprocal and essentially featureless. 
%Therefore, our discussion here generally applies to an inversion-breaking dissipative magnetic system. 





% For $\phi\neq 0$, i.e., $D\neq 0$, the propagation between the two sites is non-reciprocal, while for $\phi=\pm\pi/2$ (i.e., $J=0$) and $\Gamma=\tilde{J}$ the hopping becomes purely unidirectional. 

% Our results show that the key ingredients for the emergence of the skin effect in magnetic systems are the nonlocal dissipation and the complex coherent hoppings that can not be gauged away. 




\textit{Magnon dynamics.}  In order to investigate the magnon dynamics through the spin array beyond the non-Hermitian Hamiltonian approach, 
we derive the equations of motion for magnon operators directly from the master equation (\ref{eq1}), retaining the  quantum jumps~\cite{supplementary}. 
% that describe the dynamical evolution of the expectation values of the one- and two-body magnon operators. Here, to analyze the full dynamics, we retain the complete quantum jump terms, 
% as opposed to the prior non-Hermitian Hamiltonian approach~\cite{supplementary}. 
Under the quadratic approximation of the master equation, the following set of equations are closed, including those of the magnon number operator $n_{\alpha}\equiv \langle \hat{a}^\dag_{\alpha}\hat{a}_{\alpha}\rangle$:
\begin{eqnarray}
    \frac{d}{dt}\langle \hat{a}^\dag_{\alpha}\hat{a}_{\beta}\rangle
    &=&\gamma_0\langle \hat{a}^\dag_{\alpha}\hat{a}_{\beta}\rangle-i\gamma_L\langle \hat{a}^\dag_{\alpha}\hat{a}_{{\beta}+1}\rangle-i\gamma_R\langle \hat{a}^\dag_{\alpha}\hat{a}_{{\beta}-1}\rangle\nonumber\\
    &&+i\gamma_L^*\langle \hat{a}^\dag_{\alpha+1}\hat{a}_{\beta}\rangle+i\gamma_R^*\langle \hat{a}^\dag_{\alpha-1}\hat{a}_{\beta}\rangle\,,\label{eq:dynamic}
    \end{eqnarray}
with $\gamma_0=-2s\Gamma_0$. 
We simulate the time evolution of the magnon number distribution by solving Eq.~(\ref{eq:dynamic}) numerically with an initial state of one magnon excitation at the center of a $N=9$ spin array and $\langle \hat{a}^\dag_{\alpha}\hat{a}_{\beta}\rangle=0$ for any $\alpha, \beta \neq 5$. 
% Physically, this corresponds to an initial state with a magnon excitation in the middle of an otherwise frozen macrospin array. 
The overall dynamics, as shown in Fig.~\ref{fig:dynamics}(a)-(c) for
% show that the evolution of the magnon number $n_{\alpha}\equiv \langle \hat{a}^\dag_{\alpha}\hat{a}_{\alpha}\rangle$ for $J=0$ and $\;D,\Gamma=1$ (a), $J,D,\Gamma=1$ (b), and for $\Gamma,J=1$ and $\; D=0$ (c), is, 
unidirectional, nonreciprocal, and reciprocal propagation, respectively, is largely consistent with the understanding from the non-Hermitian description. 
% , as suggested by the eigenmode density and  winding number results reported in Fig.~\ref{fig:master}. 
In particular, the unidirectional limit remains, as can be seen directly from the Eq.~\eqref{eq:dynamic}: When $\gamma_R (\gamma_L)= 0$, two-point correlators can only build up to the left (right)  during the entire time evolution.

However, an important caveat is revealed in the transient dynamics: a large local dissipation can fully suppress the  state accumulation at the system boundary, as depicted by Fig.~\ref{fig:dynamics}(d). The time and length scales on which the NHSE is observable cannot be captured by the topological properties of the non-Hermitian Hamiltonian, but rather strongly depends on the local dissipation, which can be adjusted by pumping. Moreover, the quantum jumps are necessary for the time-evolution equations of the two-point correlators~(\ref{eq:dynamic})  to be closed~\cite{supplementary}, which maintains the inner consistency of the single-quasiparticle dynamics.
% However, whether the state accumulation appears at the system boundary strongly depends on the time and length scales of the observation. 
% there is an important caveat: The local dissipation suppresses the non-Hermitian skin effect, and the accumulation at system boundary predicted by the winding numbers and eigenmodes will never take place, as shown by Fig.~\ref{fig:dynamics}(d). The time and length scales
% for experimental observables,
% which can, on the other hand, be adjusted by pumping.
% While the winding number loop in Fig.~\ref{fig:master}(c) does not depend on the value of the local dissipation $\Gamma_{0}$, the spin-wave propagation does.
These results highlight the limitation of an analysis based on non-Hermitian eigenenergies and eigenmodes.~\cite{lee2022anomalously}.







%We simulate the time evolution of the magnon number distribution by solving Eqs.~(\ref{eq:dynamic}) numerically with an initial state of one magnon excitation at the center of the spin array and $\langle \hat{a}^\dag_{\alpha}\hat{a}_{\beta}\rangle=0$ for any $\alpha, \beta \neq 5$. 
% Physically, this corresponds to an initial state with a magnon excitation in the middle of an otherwise frozen macrospin array. 

%The overall dynamics, as shown in Fig.~\ref{fig:dynamics}(a)-(c) for
% show that the evolution of the magnon number $n_{\alpha}\equiv \langle \hat{a}^\dag_{\alpha}\hat{a}_{\alpha}\rangle$ for $J=0$ and $\;D,\Gamma=1$ (a), $J,D,\Gamma=1$ (b), and for $\Gamma,J=1$ and $\; D=0$ (c), is, 
%unidirectional, nonreciprocal, and reciprocal propagation, respectively, is largely consistent with the understanding from the non-Hermitian description. 
% , as suggested by the eigenmode density and  winding number results reported in Fig.~\ref{fig:master}. 
%It is worth to clarify that the nonreciprocity here refers to that related to the skin effect, rather than the group velocity. Especially, the unidirectional limit remains, as can be seen directly from the Eq.~\ref{eq:dynamic}: When $\gamma_R (\gamma_L)= 0$, two-point correlators can only build up to the left (right)  during the entire time evolution.

%In addition, since the dynamics is damped, all magnon numbers eventually decay to zero. The skin effect is suppressed by the local dissipation $\Gamma_{0}$. Fig.~\ref{fig:dynamics}(d) shows an extreme case where the a large $\Gamma_{0}$ fully suppresses the spread of the excitation. This thus sets the conditions of the parameter regime as well as time and length scales for experimental observation of the state accumulation, which can, on the other hand, be adjusted by pumping. 
% there is an important caveat: while the topology of the loop in Fig.~\ref{fig:master}(c) does not depend on the value of the local dissipation $\Gamma_{0}$, the spin-wave propagation does. As shown by Fig.~\ref{fig:dynamics}(d), if the  local dissipation is strong enough, the accumulation at one boundary predicted by the winding numbers and eigenmodes will never take place. 
%This result highlights the limitation of an analysis of time evolution or transient behavior  based 
%\textcolor{red}{what is the idea of ``infinitely temporally-resolved''?} 
%on non-Hermitian eigenenergies and eigenmodes~\cite{lee2022anomalously}.
% in predicting experimental observables. 





\textit{Connection to the classical magnetization dynamics.} An analogy can be drawn between our model~(\ref{eq1}) and the magnetic multilayer sketched in Fig.~\ref{fig:LLG}(a), where each magnetic layer interacts with its nearest neighbors via a metallic spacer. In addition to the intrinsic Gilbert damping $\alpha_{l}$ of the magnetic dynamics, the metallic spacer mediates a nonlocal spin pumping $\alpha_{nl}$ between the long-wavelength magnetization dynamics of adjacent layers~\cite{arxivSklenar,PhysRevB.66.224403,RevModPhys.77.1375,PhysRevLett.88.117601}. 
 The electric Fermi surface in metallic spacer can also mediates an effective coherent RKKY coupling $J$, with a DMI component $D$ due to interfacial inversion-symmetry  breaking~\cite{huang2022growth,di2015direct,ma2020longitudinal,han2019long,fernandez2019symmetry,avci2021chiral}. 

A minimal model for the magnetic Hamiltonian of the  multilayer can be written as~\cite{yuan2023unidirectional}
\begin{eqnarray}
   \tilde{\mathcal{H}}&=& -\sum_{\langle \alpha\beta\rangle}
   \left[ J\mathbf{m}_\alpha\cdot\mathbf{m}_\beta 
   +D \hat{\mathbf{z}}\cdot(\mathbf{m}_\alpha\times\mathbf{m}_\beta) \right] \nonumber \\
   &&-\sum_{\alpha} \mu_0M_s\mathbf{H}\cdot \mathbf{m}_\alpha\, ,
\end{eqnarray}
where $\mathbf{H}=H\hat{\mathbf{z}}$ is an externally applied magnetic field oriented along the $\hat{\mathbf{z}}$ direction, $M_s$ the saturation magnetization and $\mu_{0}$ the vacuum permeability. 
% Here, $\langle \alpha\beta\rangle$ denotes the nearest neighbors.
The macrospin dynamics of the magnetization direction $\mathbf{m}_{\alpha}$ of the $\alpha$th layer read as
 \begin{align}
 \frac{\partial \mathbf{m}_\alpha}{\partial t} &= -\frac{\gamma}{M_s}\mathbf{m}_{\alpha}\times\mathbf{H}_{\text{eff},\alpha}+\alpha_l \mathbf{m}_\alpha\times\frac{\partial\mathbf{m}_\alpha}{\partial t}\nonumber \\
 &+ \alpha_{nl} \mathbf{m}_\alpha\times \left(\frac{\partial\mathbf{m}_{\alpha-1}}{\partial t}+\frac{\partial\mathbf{m}_{\alpha+1}}{\partial t} \right)\,,
 \label{eq7}
 \end{align}
with $\mathbf{H}_{\text{eff},\alpha}=-\partial \tilde{\mathcal{H}}/\partial\mathbf{m}_\alpha $ and $\gamma$ the gyromagnetic ratio.
Transformation into a bosonic hopping model reveals an energy spectrum under periodic boundary condition that also forms a closed loop in the complex plane, as shown in Fig.~\ref{fig:LLG}(b). The nonzero winding number of the loop reveals a similar nonreciprocity as the NHSE~\cite{supplementary}.


% Figure environment removed


For a bilayer, a balance between DMI and nonlocal damping, i.e., $D=\pm \alpha_{nl}\mu_0M_sH$~\cite{supplementary} can yield unidirectional transport~\cite{yuan2023unidirectional}. 
% This observation is in agreement with our findings, suggesting that the non-Hermitian skin effect can emerge already at the level of an LLG description of the magnetization dynamics.
% Figure~\ref{fig:dynamics}(f) confirms that the energy of spectrum of our model for periodic boundary conditions describes a closed  loop in the complex energy plane, characterized by a nonzero winding number~\cite{supplementary}.    
However, a key difference between the non-Hermitian model~(\ref{eq2}) and Eq.~(\ref{eq7}) arises for more than two layers: the latter cannot exhibit exact unidirectionality~\cite{supplementary}. 
% This discrepancy stems from the form of nonlocal damping term: throughout this work, we assume it, for simplicity, to be a constant, 
The nonlocal damping in Eq.~(\ref{eq7}) is determined for a given layer by the instantaneous dynamic state of the adjacent layers. This
% In the LLG dynamics~(\ref{eq7}) the magnetization dynamics of a given layer experience  dissipation that depends on the instantaneous state of the adjacent magnetic layers. 
effectively establishes a next nearest neighbor and even further interactions~\cite{supplementary},
which stay active as the nearest-neighbor hopping in one direction can be turned off. 
Therefore, the dynamics can be nonreciprocal but not unidirectional, as reflected by the nonvanishing ellipticity of the eigenenergy loop.
% which can not be compensated exactly and, thus, yields a winding-number-loop in the  complex plane that can not be circular for any choice of parameters. 


%For a bilayer, exact balance between DMI and nonlocal damping, i.e., $D=\pm \alpha_{nl}\mu_0M_sH$~\cite{supplementary} can yield unidirectional transport~\cite{yuan2023unidirectional}. 
% This observation is in agreement with our findings, suggesting that the non-Hermitian skin effect can emerge already at the level of an LLG description of the magnetization dynamics.
% Figure~\ref{fig:dynamics}(f) confirms that the energy of spectrum of our model for periodic boundary conditions describes a closed  loop in the complex energy plane, characterized by a nonzero winding number~\cite{supplementary}.    
%However, a key difference between the non-Hermitian model~(\ref{eq2}) and Eq.~(\ref{eq7}) arises for more than two layers: the latter cannot exhibit exact unidirectionality~\cite{supplementary}. 
% This discrepancy stems from the form of nonlocal damping term: throughout this work, we assume it, for simplicity, to be a constant, 
%The nonlocal damping in Eq.~(\ref{eq7}) is determined for a given layer by the instantaneous dynamic state of the adjacent layers. This
% In the LLG dynamics~(\ref{eq7}) the magnetization dynamics of a given layer experience  dissipation that depends on the instantaneous state of the adjacent magnetic layers. 
%effectively establishes a next-nearest-neighbor and even further interactions~\cite{supplementary},
%which stay active as the nearest-neighbor hopping in one direction can be turned off. 
%Therefore, the dynamics can be nonreciprocal but not unidirectional, as reflected by the nonvanishing ellipticity of the eigenenergy loop.
% which can not be compensated exactly and, thus, yields a winding-number-loop in the  complex plane that can not be circular for any choice of parameters. 
 



\textit{Discussion.} In this work, we uncover the origin of the NHSE in magnetic systems, which lies in the interplay between nonlocal dissipation and any coherent interaction that breaks the TRS of the magnon Hamiltonian. We develop a master equation framework for coupled magnetization dynamics induced by the reservoir. Finite-temperature and spin pumping effects can be easily integrated into our model by adjusting the relevant parameters accordingly~\cite{chelpanova2021,zou2022bell}. Our approach highlights the limitations of the winding number in predicting whether a macroscopic accumulation of bulk states at a boundary can physically take place~\cite{borgnia2020non,okuma2020topological,zhang2020correspondence}.
The mechanism that we uncover is intimately connected with approaches to quantum nonreciprocity at light-matter interfaces~\cite{metelmann2015nonreciprocal,fang2017generalized},
for which engineering a nontrivial phase in the nonlocal quantum many-body dynamics requires a combination of external drives. Our  results suggests that quantum nonreciprocity in spin ensembles can be engineered at equilibrium via their mutual interactions with a magnetic bath lacking, e.g., inversion or mirror symmetries. 

We also establish a link between the NHSE and the classical magnetization dynamics, showing that non-reciprocity (\textit{albeit} not unidirectionality) can be realized by virtue of the same ingredients underlying the NHSE. Our findings, together with the growing interest in non-Hermitian engineering of magnetic systems, call for the development of a more precise connection between the dissipative coupling in the Lindbladian and the nonlocal damping in the classical equations of dissipative magnetic dynamics needs to be established. Finally, non-Markovian effects may also arise in dynamical magnetic systems~\cite{kim2018}, which go beyond the Lindblad formalism, making this an important topic for further exploration.   



%Here, the magnon description and quadratic approximations restrict our discussion to the dynamics of large with small excitation numbers, where the nonreciprocal spreading of magnon density is consistent with that predicted by the non-Hermitian skin effect, albeit with small caveats. Similar phenomena allow us to make a connection to the classical description of dissipative magnetic dynamics, which reveals the fragility of unidirectional transport. The nonlinear nature of spin dynamics is also likely to lead to the absence of exact unidiretionality. This can be seen by keeping $\mathcal{O}(1/s)$ or higher orders in the Holstein-Primakoff expansion of the master equation, which generates a cumulant hierarchy in the equations of motion of the correlators.
%%Meanwhile, the nonreciprocity generally remains, as a result of the interplay between the complex coherent interaction and the dissipative coupling. 
%Based on our Linidbladian framework and the understanding of the origin of the magnetic skin effect, it would be interesting for future work to investigate the many-body quantum dynamics of a dissipative spin-$1/2$ array, or explore nonlinear effects in macrospins with high excitation numbers using cumulant expansion.
%Furthermore, a more precise connection between the dissipative coupling in the Lindbladian and the nonlocal damping in the classical equations of dissipative magnetic dynamics needs to be established. Finally, non-Markovian effects may also arise in dynamical magnetic systems~\cite{kim2018}, which go beyond the Lindblad formalism, making this an important topic for further exploration.   

% In this work, we uncover the origin of the skin effect in magnetic systems, showing that the interplay between nonlocal dissipation and a complex coherent hopping can lead to nonreciprocity. Through a master equation approach, we highlight the limitations of the winding number in predicting whether a macroscopic accumulation of bulk states at a boundary can physically take place.  Our framework can be readily used to address the role of higher-order corrections to the spin-wave dynamics. 
% Finally, we establish a connection between the non-Hermitian skin effect and the classical magnetization dynamics, showing that nonreciprocity (\textit{albeit} not unidirectionality) can be realized by virtue of the same ingredients underlying the non-Hermitian skin effect. 

% Our results, together with the growing interest in non-Hermitian engineering of magnetic systems \cite{li2022multitude,PhysRevB.105.104433,mcclarty2019non,flebus2020non,PhysRevB.105.L180401,yuan2022master, deng2022non,hurst2022non,deng2023exceptional,lu2021magnetic,zhang2020dynamic}, call for the development of  a rigorous link between microscopic models for dissipative nonlocal interactions in the Markovian and non-Markovian spin dynamics and the phenomenological terms routinely used in the Landau-Lifshitz-Gilbert formalism at zero and finite temperatures~\cite{PhysRevB.66.224403,RevModPhys.77.1375,PhysRevLett.88.117601,PhysRevB.94.214428,PhysRevB.100.064410}. 

% \&pumping; nonlinearity; no cutoff at quadratic order, large excitation number; spin-1/2;



\textit{Acknowledgments.}
The authors thank R. A. Duine and J. Marino for helpful discussions. 
This work was supported by the National Science Foundation under Grant No. NSF DMR-2144086.

\bibliographystyle{apsrev4-2}

\bibliography{library}



%%%%%%%%%% Merge with supplemental materials %%%%%%%%%%
\widetext
\clearpage

\begin{center}
	\large \textbf{Supplemental material:  ``Reciprocal Reservoir Induced Non-Hermitian Skin Effect"} \\ \vspace{0.3cm} 
	X. Li, M. Al Begaowe, S. Zhang and B. Flebus
\end{center}
%%%%%%%%%% Merge with supplemental materials %%%%%%%%%%
%%%%%%%%%% Prefix a "S" to all equations, figures, tables and reset the counter %%%%%%%%%%
\setcounter{section}{0}
\setcounter{equation}{0}
\setcounter{figure}{0}
\setcounter{table}{0}
\setcounter{page}{1}
\makeatletter
\renewcommand{\theequation}{S\arabic{equation}}
\renewcommand{\thefigure}{S\arabic{figure}}
\renewcommand{\bibnumfmt}[1]{[S#1]}
\renewcommand{\citenumfont}[1]{S#1}
%%%%%%%%%% Prefix a "S" to all equations, figures, tables and reset the counter %%%%%%%%%%

\section*{S1. Master equation framework}


\subsection{The complete Lindbladian}
\label{s1.a.complete.master}
In the main text, we have neglected the influence of thermal  and dephasing effects in the Lindblad dissipator (\ref{eqlin}). Here, we present the complete Lindblad dissipator as follows:

\begin{eqnarray}
   \mathcal{L}[\rho] =\frac{1}{2}\sum_{\alpha,\beta=1}^N\Gamma_{\alpha \beta} \bigg[ \left(\{\hat{s}_\alpha^-\hat{s}_\beta^+,\rho\}-2\hat{s}_\beta^+\rho \hat{s}_\alpha^-\right) +\tilde{\Gamma}_{\alpha \beta}\left(\{\hat{s}_\alpha^+\hat{s}_\beta^-,\rho\}-2\hat{s}_\beta^-\rho \hat{s}_\alpha^+\right)+\Gamma^z_{\alpha \beta}\left(\{\hat{s}_\alpha^z\hat{s}_\beta^z,\rho\}-2\hat{s}_\beta^z\rho \hat{s}_\alpha^z\right) \bigg]
,\label{fulllin}
\end{eqnarray}
where the Hermiticity of the density matrix $\rho$ implies  $\Gamma_{\alpha\beta}=\Gamma^*_{\beta\alpha}$, $\tilde{\Gamma}_{\alpha\beta}=\tilde{\Gamma}^*_{\beta\alpha}$ and $\Gamma^z_{\alpha\beta}=\Gamma^z_{\alpha\beta}$. While generally the coefficients $\Gamma_{\alpha\beta},\  \tilde{\Gamma}_{\alpha\beta}$ and $\Gamma^z_{\alpha\beta}$ in (\ref{fulllin}) can be complex, for $\alpha=\beta$ they are real. At thermal equilibrium, the fluctuation-dissipation theorem dictates $\tilde{\Gamma}_{\alpha\beta}=e^{-\beta\hbar\omega_0}\Gamma_{\alpha\beta}$ and, thus, one has $\tilde{\Gamma}_{\alpha\beta}\rightarrow0$ for $T\rightarrow0$. The terms $\propto \Gamma^z_{\alpha \beta}$ capture  dephasing effects, which are dominated by low-frequency (dc-like) response of the reservoir and can be neglected for a gapped reservoir.




\subsection{Positive semidefiniteness of the decoherence matrix}
\label{s1.a.positive.semidefinite}

The positive semidefinite decoherence matrix $\boldsymbol{\Gamma}$, whose entries correspond to the coefficients $\Gamma_{\alpha\beta}$ in Eq.~(\ref{eqlin}), encodes the effect of the reservoir on the spin system. It takes the form of a tridiagonal Toeplitz matrix, i.e., 

\begin{eqnarray}
    \boldsymbol{\Gamma}=\left(
    \begin{array}{ccccccccc}
       \Gamma_0  & \Gamma&&&& \\
        \Gamma^* & \Gamma_0&\Gamma&&&\\
        & \Gamma^*&\ddots&\ddots&&\\
        &&\ddots & \ddots&\Gamma&\\
        &&& \Gamma^*& \Gamma_0&\Gamma\\
        &&&&\Gamma^* & \Gamma_0
    \end{array}
    \right)_{N\times N},
\end{eqnarray}



which can be diagonalized to obtain the eigenvalues:
 
 \begin{eqnarray}
\gamma_n=\Gamma_0+2|\Gamma|\cos\frac{n\pi}{N+1}\, ,
\end{eqnarray}
with $n=1,\cdots,N$. For an infinitely-long spin chain, i.e., $N\rightarrow\infty$, the requirement of positive semidefiniteness $\gamma_n \geq0$ implies $\Gamma_0\geq2|\Gamma|$.

\subsection{Derivation of the effective non-Hermitian Hamiltonian}


For macrospins with small excitation numbers, i.e., $s=|\mathbf{s}|/\hbar\gg1$ and $n_{\alpha}\equiv \langle \hat{a}^\dag_{\alpha}\hat{a}_{\alpha}\rangle \ll s$, we can invoke the Holstein Primakoff (HP) transformation, i.e., 

\begin{eqnarray}
\hat{s}_\alpha^z=-s+\hat{a}^\dag_\alpha \hat{a}_\alpha\,,\quad
\hat{s}_\alpha^+\approx\sqrt{2s}\hat{a}^\dag_\alpha\,,\quad
\hat{s}_\alpha^-\approx\sqrt{2s}\hat{a}_\alpha\,,
\end{eqnarray}
where $\hat{a}^\dag_\alpha$ and $\hat{a}_\alpha$ are, respectively, the magnon creation and annihilation operators satisfying the following commuting relations:

\begin{align}
[\hat{a}_\alpha,\hat{a}_\beta]=\hat{a}_\alpha \hat{a}_\beta-\hat{a}_\beta \hat{a}_\alpha=0\,,\quad
[\hat{a}^\dag_\alpha ,\hat{a}^\dag_\beta]=\hat{a}^\dag_\alpha \hat{a}^\dag_\beta-\hat{a}^\dag_\beta \hat{a}^\dag_\alpha=0\,,\quad
[\hat{a}_\alpha,\hat{a}^\dag_\beta]=\hat{a}_\alpha \hat{a}^\dag_\beta-\hat{a}^\dag_\beta \hat{a}_\alpha=\delta_{\alpha \beta}\ .
\end{align}
Focusing only on  nearest-neighbor (NN) interactions, we can rewrite the coherent coupling Hamiltonian~(\ref{hamiltonian})  in terms of magnon operators as


\begin{eqnarray}
 \mathcal{H}  
&=&\sum_{\alpha=1}^N\omega_0\hat{a}^\dag_\alpha \hat{a}_\alpha+ \sum_{\alpha =1}^{N-1}\Big[s(J-iD) \hat{a}^\dag_{\alpha+1}\hat{a}_\alpha+h.c.\Big]\,,
\end{eqnarray}
where $h.c.$ stands for the corresponding Hermitian conjugate terms. Similarly, the Lindblad dissipator $\mathcal{L}[\rho]$ given in Eq.~(\ref{eqlin}) can be rewritten as

\begin{eqnarray}
\mathcal{L}[\rho] 
= i[\mathcal{H}', \rho] -\sum_{\alpha=1}^Ns\Gamma_0\hat{a}^\dag_\alpha \rho \hat{a}_\alpha-\sum_{\alpha=1}^{N-1} (2s\Gamma\hat{a}_{\alpha+1}^\dag \rho\hat{a}_\alpha +h.c.), \label{nonHermitian}
\end{eqnarray}
where 


\begin{eqnarray}
\mathcal{H}'=-\sum_{\alpha=1}^Nis\Gamma_0\hat{a}^\dag_\alpha \hat{a}_\alpha+\sum_{\alpha=1}^{N-1} \left(-is\Gamma\hat{a}_{\alpha+1}^\dag\hat{a}_\alpha  -h.c.\right),  
\end{eqnarray}
 and  $[\mathcal{H}', \rho]=\mathcal{H}' \rho-\rho\mathcal{H}'^\dag$. The first term on the right-hand-side of Eq.~(\ref {nonHermitian}) introduces an effective non-Hermitian Hamiltonian $\mathcal{H}'$ and the remaining terms are the quantum jump operators. 
Plugging Eq.~(\ref{nonHermitian}) into
 Eq.~(\ref{eq1}) while neglecting the quantum jump terms, we obtain 
\begin{eqnarray}
    \frac{d\rho}{dt}=-i \left[ \mathcal{H}_{nh}, \rho \right] \,,\label{mastereff}
\end{eqnarray}
where $\mathcal{H}_{nh}=\mathcal{H}+\mathcal{H}'$ is the effective non-Hermitian Hamiltonian~(\ref{eq2}) in the main text.


\subsection{Diagonalization of the effective non-Hermitian Hamiltonian}




In the  non-unidirectional case, $\gamma_L\gamma_R\neq0$,   the non-Hermitian Hamiltonian (\ref{eq2}) is a tridiagonal Toeplitz form matrix,




\begin{eqnarray}
    \mathcal{H}_{nh}=\left(
    \begin{array}{ccccccccc}
       \epsilon_0  & \gamma_L&&&& \\
        \gamma_R & \epsilon_0&\gamma_L&&&\\
        & \gamma_R&\ddots&\ddots&&\\
        &&\ddots & \ddots&\gamma_L&\\
        &&& \gamma_R& \epsilon_0&\gamma_L\\
        &&&&\gamma_R & \epsilon_0
    \end{array}
    \right)_{N\times N}.\label{nhhamilmat}
\end{eqnarray}
Diagonalizing Eq.~(\ref{nhhamilmat}), we obtain a series of eigenvalues:


 \begin{eqnarray}
    \lambda_n=\epsilon_0+2\sqrt{\gamma_L\gamma_R}\cos\frac{n\pi}{N+1},\quad n=1, \cdots, N\, ,
\end{eqnarray}
and the corresponding right and left eigenvectors 
\begin{eqnarray}
    |\Psi^R_n\rangle=(\psi^R_{n,1},\cdots,\psi^R_{n,\alpha},\cdots,\psi^R_{n,N})^T,\qquad
    |\Psi^L_n\rangle=(\psi^L_{n,1},\cdots,\psi^L_{n,\alpha},\cdots,\psi^L_{n,N})^T\,,
\end{eqnarray}
with
\begin{eqnarray}
    \psi^R_{n,\alpha}=\left(\frac{\gamma_R}{\gamma_L}\right)^{\alpha/2}\sin\frac{n\alpha\pi}{N+1}\,,\qquad   \psi^L_{n,\alpha}=\left(\frac{\gamma^*_L}{\gamma^*_R}\right)^{\alpha/2}\sin\frac{n\alpha\pi}{N+1}\,,\qquad\qquad n,\alpha =1,\cdots, N\,,
\end{eqnarray}
from which it is straightforward to derive  Eqs.~(\ref{modeprobleft}) and  (\ref{modeprobright}).
On the other hand, if the system satisfies the unidirectional condition, i.e.,  $\gamma_L=0$ or $\gamma_R=0$, the Hamiltonian 
(\ref{eq2}) reduces to a Jordan  block of size $N$,


\begin{eqnarray}
\mathcal{H}^L_{nh}=\left(
    \begin{array}{ccccccccc}
       \epsilon_0  & \gamma_L&&&& \\
        0 & \epsilon_0&\gamma_L&&&\\
        & 0&\ddots&\ddots&&\\
        &&\ddots & \ddots&\gamma_L&\\
        &&& 0& \epsilon_0&\gamma_L\\
        &&&&0 & \epsilon_0
    \end{array}
    \right)_{N\times N}
,\qquad\qquad
\mathcal{H}^R_{nh}=\left(
    \begin{array}{ccccccccc}
       \epsilon_0  & 0&&&& \\
        \gamma_R & \epsilon_0&0&&&\\
        & \gamma_R&\ddots&\ddots&&\\
        &&\ddots & \ddots&0&\\
        &&& \gamma_R& \epsilon_0&0\\
        &&&&\gamma_R & \epsilon_0
    \end{array}
    \right)_{N\times N}.\label{Jordanblock}
\end{eqnarray}
The Jordan  block  form Hamiltonians (\ref{Jordanblock}) are non-diagonalizable, but we can directly read off the only eigenvalue with $N$ multiplicity for both $\mathcal{H}^L_{nh}$ and $\mathcal{H}^R_{nh}$, i.e., $\lambda=\epsilon_0$, and the corresponding right and left eigenvectors as

\begin{eqnarray}
    |\Psi^{R}_L\rangle=(1,0,\cdots,0)^T,\qquad |\Psi^{L}_L\rangle=(0,\cdots,0,1)^T,\label{jordanupp}
\end{eqnarray}
for $\mathcal{H}^L_{nh}$,  and


\begin{eqnarray}
    |\Psi^{R}_R\rangle=(0,\cdots,0,1)^T,\qquad |\Psi^{L}_R\rangle=(1,0,\cdots,0)^T,\label{jordandown}
\end{eqnarray}
for $\mathcal{H}^R_{nh}$. Equations~(\ref{jordanupp}) and (\ref{jordandown}) show that, under the unidirectional condition, the eigenmodes will exactly only reside at one of boundaries. 



\subsection{Winding number loops for the macrospin array}


In  Eq.~(\ref{eq3}), we have obtained the spectrum $\epsilon_k=\text{Re}[\epsilon_k]+i\text{Im}[\epsilon_k]$, where $\text{Re}[\epsilon_k]=2s(J\cos k-D\sin k)$ and $\text{Im}[\epsilon_k]=-2s\Gamma\cos k$ denote the real and imaginary parts, respectively. For vanishing DMI and finite symmetric exchange interactions and nonlocal dissipation, i.e.,   $D=0$ and $ J,\ \Gamma\neq0$, the real and imaginary energies satisfy a linear relationship, i.e.,

\begin{eqnarray}
    \text{Re}[\epsilon_k]=-\frac{J}{\Gamma}\text{Im}[\epsilon_k]\,,
\end{eqnarray}
pointing to the absence of the non-Hermitian skin effect.  In the general case, i.e., $D, J, \Gamma\neq0$,  the winding number loop obeys the equation of an elliptical curve:

\begin{eqnarray}
    \frac{1}{4D^2}(\text{Re}[\epsilon_k])^2+\frac{J}{2D^2\Gamma}\text{Re}[\epsilon_k]\text{Im}[\epsilon_k]+\frac{J^2+D^2}{4D^2\Gamma^2}(\text{Im}[\epsilon_k])^2=1.\label{ellipse}
\end{eqnarray}
When the system satisfies the unidirectionality condition, i.e., $J=0$ and $D=\pm\Gamma$, the ellipse~(\ref{ellipse}) simplifies to a circle, i.e.,

\begin{eqnarray}
    \left(\text{Re}[\epsilon_k]\right)^2+ \left(\text{Im}[\epsilon_k]\right)^2=4D^2.
\end{eqnarray}





\subsection{Derivation of the dynamical equations}


In the Schrödinger picture, the dynamical behaviours of the expectation value of a general operator $\langle\hat{\mathcal{O}}\rangle\equiv \text{Tr}\left(\hat{\mathcal{O}}\rho\right)$ are governed by the following equation:

\begin{eqnarray}
    \frac{d}{dt}\langle \mathcal{O}\rangle
  = \text{Tr}\left(\hat{\mathcal{O}}\frac{d}{dt}\rho\right)=-i\langle [\hat{\mathcal{O}},\mathcal{H}]\rangle -\text{Tr}\left(\hat{\mathcal{O}}\mathcal{L}[\rho]  \right).\label{dyneq1}
\end{eqnarray}
We first investigate the dynamical evolution  of the one-body magnon operators by replacing $\hat{\mathcal{O}}$ with  $\hat{a}_\alpha$ in Eq.~(\ref{dyneq1}), i.e.,

\begin{eqnarray}
-i\langle [\hat{a}_{\alpha},\mathcal{H}]\rangle&=&-i\sum_{\alpha'} \omega_0 \langle \hat{a}_{\alpha}\hat{a}^\dag_{\alpha'} \hat{a}_{\alpha'}\rangle-i s\sum_{{\alpha'} }(J+iD)\langle \hat{a}_{\alpha} \hat{a}^\dag_{\alpha'} \hat{a}_{\alpha'+1}\rangle-i s\sum_{{\alpha'}}(J-iD)\langle \hat{a}_{\alpha} \hat{a}^\dag_{\alpha'+1} \hat{a}_{\alpha'}\rangle\nonumber\\
&&+i\sum_{\alpha'} \omega_0 \langle \hat{a}^\dag_{\alpha'} \hat{a}_{\alpha'}\hat{a}_{\alpha}\rangle+i s\sum_{{\alpha'}}(J+iD)\langle  \hat{a}^\dag_{\alpha'} \hat{a}_{\alpha'+1}\hat{a}_{\alpha}\rangle+is\sum_{{\alpha'}}(J-iD)\langle  \hat{a}^\dag_{\alpha'+1} \hat{a}_{\alpha'}\hat{a}_{\alpha}\rangle,\nonumber\\
&=&-i \omega_0 \langle \hat{a}_{\alpha}\rangle-i s(J+iD)\langle \hat{a}_{\alpha+1}\rangle-i s(J-iD)\langle  \hat{a}_{\alpha-1}\rangle,\label{linear:ham}
\end{eqnarray}

and


\begin{eqnarray}
-\text{Tr}(\hat{a}_{\alpha}\mathcal{L}[\rho]  )
&=&-s\sum_{{\alpha'}}\Gamma_0\Big(\langle \hat{a}_{\alpha}\hat{a}^\dag_{\alpha'} \hat{a}_{\alpha'} \rangle+\langle \hat{a}^\dag_{\alpha'} \hat{a}_{\alpha'}\hat{a}_{\alpha}\rangle-2 \langle \hat{a}^\dag_{\alpha'}\hat{a}_{\alpha}\hat{a}_{\alpha'} \rangle  \Big)\nonumber\\
&&-s\sum_{{\alpha'}}\Gamma\Big(\langle \hat{a}_{\alpha}\hat{a}^\dag_{\alpha'} \hat{a}_{\alpha'+1} \rangle+\langle \hat{a}^\dag_{\alpha'} \hat{a}_{\alpha'+1}\hat{a}_{\alpha}
\rangle-2 \langle \hat{a}^\dag_{\alpha'}\hat{a}_{\alpha}\hat{a}_{\alpha'+1} \rangle  \Big)\nonumber\\
&&-s\sum_{{\alpha'}}\Gamma\Big(\langle \hat{a}_{\alpha}\hat{a}^\dag_{\alpha'+1} \hat{a}_{\alpha'} \rangle +\langle  \hat{a}^\dag_{\alpha'+1} \hat{a}_{\alpha'}\hat{a}_{\alpha}\rangle-2 \langle \hat{a}^\dag_{\alpha'+1}\hat{a}_{\alpha}\hat{a}_{\alpha'} \rangle  \Big),\nonumber\\
&=&-s\Gamma_0\langle  \hat{a}_{\alpha} \rangle -s\Gamma\langle \hat{a}_{\alpha+1} \rangle-s\Gamma\langle  \hat{a}_{\alpha-1} \rangle.   \label{linear:dis}
\end{eqnarray}


Combing   Eqs.~(\ref{linear:ham}) and (\ref{linear:dis}), we find

\begin{eqnarray}
    \frac{d}{dt}\langle \hat{a}_{\alpha}\rangle
   =-(i\omega_0+s\Gamma_0)\langle    \hat{a}_{\alpha}\rangle-i \gamma_L  \langle \hat{a}_{\alpha+1} \rangle-i \gamma_R  \langle \hat{a}_{\alpha-1} \rangle,\label{onebody}
\end{eqnarray}
with $\gamma_L\equiv s\left( J+iD-i\Gamma\right)$ and $\gamma_R\equiv s\left(J-iD-i\Gamma^* \right)$. Here, we focus on the scenario where, initially, only one magnon at the center of the macrospin array is excited, and all the expectation values of the single magnon operators are vanishing, i.e., 
$\langle \hat{a}_\alpha\rangle_{t=0}=0$. Accordingly, the linear Eqs.~(\ref{onebody}) do not possess nontrivial solutions.

We can derive the equation governing the dynamical evolution of the expectation values of the two-body magnon operators, i.e., $\langle \hat{a}^\dag_\alpha\hat{a}_\beta\rangle$, as 

\begin{eqnarray}
    \frac{d}{dt}\langle \hat{a}^\dag_{\alpha}\hat{a}_{\beta}\rangle
   =-i\langle[\hat{a}^\dag_{\alpha}\hat{a}_{\beta},\mathcal{H} ]\rangle -\text{Tr}\left(\hat{a}^\dag_{\alpha}\hat{a}_{\beta}\mathcal{L}[\rho]  \right)\ .\label{dymaeq}
\end{eqnarray}
The first term on the right-hand-side of
  Eq.~(\ref{dymaeq}) leads to

\begin{eqnarray}
-i\langle [\hat{a}^\dag_\alpha\hat{a}_\beta,\mathcal{H}]\rangle&=&i\sum_{\alpha'} \omega_0 \langle \hat{a}^\dag_{\alpha}\hat{a}_{\beta}\hat{a}^\dag_{\alpha'} \hat{a}_{\alpha'}\rangle-i s\sum_{{\alpha'} }(J+iD)\langle \hat{a}^\dag_{\alpha}\hat{a}_{\beta} \hat{a}^\dag_{\alpha'} \hat{a}_{\alpha'+1}\rangle-i s\sum_{{\alpha'}}(J-iD)\langle \hat{a}^\dag_{\alpha}\hat{a}_{\beta} \hat{a}^\dag_{\alpha'+1} \hat{a}_{\alpha'}\rangle\nonumber\\
&&-i\sum_{\alpha'} \omega_0 \langle \hat{a}^\dag_{\alpha'} \hat{a}_{\alpha'}\hat{a}^\dag_{\alpha}\hat{a}_{\beta}\rangle+i s\sum_{{\alpha'} }(J+iD)\langle  \hat{a}^\dag_{\alpha'} \hat{a}_{\alpha'+1}\hat{a}^\dag_{\alpha}\hat{a}_{\beta}\rangle+i s\sum_{{\alpha'} }(J-iD)\langle  \hat{a}^\dag_{\alpha'+1} \hat{a}_{\alpha'}\hat{a}^\dag_{\alpha}\hat{a}_{\beta}\rangle,\nonumber\\
&=&i s(J-iD) \left[ \langle  \hat{a}^\dag_{\alpha+1} \hat{a}_{\beta}\rangle - \langle \hat{a}^\dag_{\alpha}  \hat{a}_{\beta-1}\rangle \right] +i s(J+iD) \left[ \langle  \hat{a}^\dag_{\alpha-1} \hat{a}_{\beta}\rangle - \langle \hat{a}^\dag_{\alpha}\hat{a}_{\beta+1}\rangle \right],\label{qudratic:ham}
\end{eqnarray}
In order to determine how the quantum jump terms, $\propto \hat{a}_\beta^\dag\rho \hat{a}_\alpha^-$, affect the dynamical evolution of the two-point correlations $\langle \hat{a}^\dag_\alpha\hat{a}_\beta\rangle$, we expand the  second term on the right-hand-side of Eq.~(\ref{dymaeq}) as
\begin{eqnarray}
-\text{Tr}(\hat{a}^\dag_{\alpha}\hat{a}_{\beta}\mathcal{L}[\rho]  )
&=&-s\sum_{{\alpha'}}\Gamma_0\Big(\langle \hat{a}^\dag_{\alpha}\hat{a}_{\beta}\hat{a}^\dag_{\alpha'} \hat{a}_{\alpha'} \rangle+\langle \hat{a}^\dag_{\alpha'} \hat{a}_{\alpha'}\hat{a}^\dag_{\alpha}\hat{a}_{\beta}\rangle -2 \langle \hat{a}^\dag_{\alpha'}\hat{a}^\dag_{\alpha}\hat{a}_{\beta}\hat{a}_{\alpha'}\rangle \Big) 
\nonumber\\    
&&-s\sum_{{\alpha'}}\Gamma\Big(\langle \hat{a}^\dag_{\alpha}\hat{a}_{\beta}\hat{a}^\dag_{\alpha'} \hat{a}_{{\alpha'}+1} \rangle+\langle \hat{a}^\dag_{\alpha'} \hat{a}_{{\alpha'}+1}\hat{a}^\dag_{\alpha}\hat{a}_{\beta}\rangle -2 \langle \hat{a}^\dag_{\alpha'}\hat{a}^\dag_{\alpha}\hat{a}_{\beta}\hat{a}_{{\alpha'}+1}\rangle \Big)\nonumber\\
&&-s\sum_{{\alpha'}}\Gamma\Big(\langle \hat{a}^\dag_{\alpha}\hat{a}_{\beta}\hat{a}^\dag_{\alpha'} \hat{a}_{{\alpha'}-1} \rangle+\langle \hat{a}^\dag_{\alpha'} \hat{a}_{{\alpha'}-1}\hat{a}^\dag_{\alpha}\hat{a}_{\beta}\rangle -2 \langle \hat{a}^\dag_{\alpha'}\hat{a}^\dag_{\alpha}\hat{a}_{\beta}\hat{a}_{{\alpha'}-1}\rangle \Big)\nonumber\\
&=&-2s\Gamma_0\langle \hat{a}^\dag_{\alpha} \hat{a}_{\beta}\rangle 
-s\Gamma\langle \hat{a}^\dag_{\alpha} \hat{a}_{{\beta}+1} \rangle-s\Gamma^*\langle \hat{a}^\dag_{\alpha}\hat{a}_{{\beta}-1} \rangle-s\Gamma^*\langle \hat{a}^\dag_{\alpha+1} \hat{a}_{\beta}\rangle-s\Gamma\langle \hat{a}^\dag_{\alpha-1} \hat{a}_{\beta}\rangle  .\label{qudratic:lin}
\end{eqnarray}


Combining Eq.~(\ref{qudratic:ham}) and Eq.~(\ref{qudratic:lin}) we obtain the dynamical equation given in Eq.~(\ref{eq:dynamic}), i.e.,

\begin{eqnarray}
    \frac{d}{dt}\langle \hat{a}^\dag_{\alpha}\hat{a}_{\beta}\rangle
    =\gamma_0\langle \hat{a}^\dag_{\alpha}\hat{a}_{\beta}\rangle-i\gamma_L\langle \hat{a}^\dag_{\alpha}\hat{a}_{{\beta}+1}\rangle-i\gamma_R\langle \hat{a}^\dag_{\alpha}\hat{a}_{{\beta}-1}\rangle+i\gamma_L^*\langle \hat{a}^\dag_{\alpha+1}\hat{a}_{\beta}\rangle+i\gamma_R^*\langle \hat{a}^\dag_{\alpha-1}\hat{a}_{\beta}\rangle\,.
    \end{eqnarray}

Note that the quantum jump terms are necessary to cancel the four-point  correlators and obtain a closed set of equations of motion on the quadratic level. To go beyond single-quasiparticle dynamics and study nonlinear effects, one should keep the $\mathcal{O}(1/s)$ or higher orders in the Holstein-Primakoff expansion of the master equation, which generates a cumulant hierarchy in the equations of motion of the correlators.


 
 
 
 
 
 
 


 

\section{S2. Classical magnetization dynamics}









Equation $(\ref{eq7})$ can be rewritten  as

\begin{eqnarray}
    \frac{\partial \mathbf{m}_\alpha}{\partial t} 
    &=& -\frac{\gamma J}{M_s}\mathbf{m}_{\alpha}\times(\mathbf{m}_{\alpha-1}+\mathbf{m}_{\alpha+1})+\frac{\gamma}{M_s}\mathbf{m}_{\alpha}\times[(\mathbf{m}_{\alpha+1}\times\mathbf{D})-(\mathbf{m}_{\alpha-1}\times\mathbf{D})] -\gamma\mu_0 \mathbf{m}_{\alpha}\times \mathbf{H}\nonumber\\
&&+\alpha_l\mathbf{m}_\alpha\times\frac{\partial\mathbf{m}_\alpha}{\partial t}+\alpha_{nl}\mathbf{m}_\alpha\times\frac{\partial\mathbf{m}_{\alpha-1}}{\partial t}+\alpha_{nl}\mathbf{m}_\alpha\times\frac{\partial\mathbf{m}_{\alpha+1}}{\partial t}\ . \label{CoupledLLG}
\end{eqnarray}
For small fluctuations of the magnetic order parameter around the equilibrium direction, i.e., $\mathbf{m}_\alpha=(m_{\alpha}^{x},m_{\alpha}^{y},1)$ with $|m_{\alpha}^{x(y)}|\ll1$, we can simplify Eq.~(\ref{CoupledLLG}) by only retaining terms linear in $m_{\alpha}^x$ and $m_{\alpha}^y$. Similarly, for  $\alpha_l,\alpha_{nl}\ll1$,   terms of second or higher order in these parameters can be neglected. Then, invoking the Holstein-Primakoff transformation,  i.e., $\langle\hat{a}_{\alpha}\rangle\equiv m_{\alpha}^x+ im_{\alpha}^y$, Eq.~(\ref{CoupledLLG}) can be written as
\begin{eqnarray}
\frac{\partial \hat{a}_\alpha}{\partial t} 
\approx &&+ \frac{\gamma}{M_s}\left\{\Big[(i-\alpha_l)\left(2 J +\mu_0 M_sH
\right)
+2 J\alpha_{nl}\right]\hat{a}_{\alpha}
+\Big[(-i+\alpha_l)\left( J -i D\right)-\alpha_{nl}\left(2 J 
+\mu_0 M_sH
\right)\Big]\hat{a}_{\alpha-1}\nonumber\\
&&+\Big[(-i+\alpha_l)\left( J+i D\right)-\alpha_{nl}\left(2 J+\mu_0M_sH\right)\Big]\hat{a}_{\alpha+1} +\alpha_{nl}\left( J-i D\right)\hat{a}_{\alpha-2} +\alpha_{nl}\left( J+i D\right)\hat{a}_{\alpha+2}\Big\} . \label{LLGlinear}
\end{eqnarray}
As shown by the last two terms on the right-hand-side of Eq.~(\ref{LLGlinear}), the nonlocal dissipation terms, i.e., $\alpha_{nl}\mathbf{m}_\alpha\times \left(\frac{\partial\mathbf{m}{\alpha-1}}{\partial t}+\frac{\partial\mathbf{m}{\alpha+1}}{\partial t} \right)$,  gives rise to effective next nearest neighbor (NNN) interactions. For a bilayer, i.e., $N=2$,  Eq.~(\ref{LLGlinear}) simplifies to

\begin{eqnarray}
\frac{\partial \hat{a}_1}{\partial t} 
&=&\frac{\gamma}{M_s}\Big\{\left[(i-\alpha_l)\left(2 J +\mu_0 M_sH
\right)
+2 J\alpha_{nl}\right]\hat{a}_{1}
+\Big[(-i+\alpha_l)\left( J+i D\right)-\alpha_{nl}\left(2 J+\mu_0M_sH\right)\Big]\hat{a}_{2}\Big\}\, ,\label{pairlone}\\
\frac{\partial \hat{a}_2}{\partial t} 
&=&\frac{\gamma}{M_s}\Big\{\left[(i-\alpha_l)\left(2 J +\mu_0 M_sH
\right)
+2 J\alpha_{nl}\right]\hat{a}_{2}
+\Big[(-i+\alpha_l)\left( J -i D\right)-\alpha_{nl}\left(2 J 
+\mu_0 M_sH
\right)\Big]\hat{a}_{1}\Big\}\ .  \label{pairltwo}
\end{eqnarray}
To achieve unidirectional transport, the second terms on the right-hand-side of Eqs.~(\ref{pairlone}) or (\ref{pairltwo}) must vanish, i.e.,

\begin{eqnarray}
    J=\pm\alpha_lD,\qquad D=\pm\alpha_{nl}\mu_0M_sH(1+\alpha^2_l-2\alpha_{nl}\alpha_l)^{-1}\, ,\label{relationfortwo}
\end{eqnarray}
where the positive and negative signs correspond to the unidirectionality of Eqs.~(\ref{pairlone}) and (\ref{pairltwo}), respectively.
Recalling the condition $\alpha_{nl}, \alpha_{l}\ll 1$,  Eqs.~(\ref{relationfortwo}) can be simplified to

\begin{eqnarray}
    J=0,\qquad D=\pm\alpha_{nl}\mu_0M_sH .\label{relationfortwosim}
\end{eqnarray}
However, for magnetic multilayers with $N\geq3$,  unidirectionality cannot be achieved due to the emergence of effective NNN couplings. This can be shown explicitly by considering Eq.~\eqref{LLGlinear} in the limit of an infinitely-long spin chain, i.e., $N \rightarrow \infty$.
Invoking the Heisenberg equation of motion, i.e., $d\hat{a}/dt=-i[\hat{a},\mathcal{H}]$, and performing a Fourier transform, one can derive an effective Hamiltonian in momentum space, whose spectrum $\tilde{\varepsilon}_k$ reads as


\begin{eqnarray}
    \tilde{\varepsilon}_k=-\frac{\gamma}{M_s}\Big[1+i(\alpha_0+2\alpha_{nl}\cos k)\Big]\Big[ J(1-\cos k)+\mu_0 H M_s+ D\sin k\Big]\, .\label{mometumham}
\end{eqnarray}
Pluggling the unidirectionality condition~(\ref{relationfortwosim}) into Eq.~(\ref{mometumham}) yields


\begin{eqnarray}
    \tilde{\varepsilon}_k=-\gamma\mu_0H\Big[(1\pm\alpha_{nl}\sin k)+i(\alpha_l+2\alpha_{nl}\cos k)\Big]\, .\label{llgmomentum}
\end{eqnarray}
For simplicity, we normalize Eq.~(\ref{llgmomentum}) by dividing $-\gamma\mu_0H$ on both sides, i.e., $\varepsilon=\tilde{\varepsilon}_k/(-\gamma\mu_0H)$ . It is then straightforward to determine the relationship between the real and imaginary parts as

\begin{eqnarray}
    \frac{(\text{Re}[\varepsilon_k]-1)^2}{\alpha_{nl}^2}+\frac{(\text{Im}[\varepsilon_k]-\alpha_{l})^2}{4\alpha_{nl}^2}=1\, .\label{llgellipse}
\end{eqnarray}
One can readily recognize Eq.~(\ref{llgellipse}) as describing an ellipse rather than a circle, indicating the absence of unidirectionality but the presence of nonreciprocity.  







\end{document}
