\documentclass[aps,prl,onecolumn,showpacs,superscriptaddress]{revtex4-2}  

\usepackage{float}
\usepackage{bbm}
\usepackage{epsfig}
\usepackage{epstopdf}
\usepackage{graphicx}
\usepackage{amsmath,amssymb}
\usepackage{amsmath,bm}
\usepackage{physics}
\usepackage{color}
\usepackage{chngcntr}
\renewcommand{\theequation}{S\arabic{equation}}
\usepackage{hyperref}
\usepackage{lineno,blindtext}
\setlength{\tabcolsep}{9pt}
\usepackage{siunitx, booktabs}
\usepackage{diagbox, eqparbox, hhline}
\usepackage{soul}
\usepackage{xcolor}
\usepackage[normalem]{ulem}
\setlength{\doublerulesep}{2.5pt}

\newcolumntype{P}[1]{>{\centering\arraybackslash}p{#1}}
\newcommand{\new}[1]{\textcolor{blue}{#1}}          % for markup
\newcommand{\cut}[1]{\textcolor{red}{\sout{#1}}}          % for markup
\newcommand{\comment}[1]{\textcolor{green}{#1}}          % for markup
\newcommand{\ans}[1]{\textcolor{orange}{#1}}




\begin{document}

\title{Supplemental material:  ``Non-Hermitian  and Liouvillian skin effects in magnetic systems"}

\author{Xin Li}
\email{licqp@bc.edu}
\affiliation{Department of Physics, Boston College, 140 Commonwealth Avenue Chestnut Hill, MA 02467, USA}
\author{Mohamed Al Begaowe}
\affiliation{Department of Physics, Boston College, 140 Commonwealth Avenue Chestnut Hill, MA 02467, USA}
\author{Shu Zhang}
\affiliation{Collective Dynamics and Quantum Transport Unit, Okinawa Institute of Science and Technology Graduate University, Okinawa 904-0495, Japan}
\affiliation{Max Planck Institute for the Physics of Complex Systems, 01187 Dresden, Germany}
\author{Benedetta Flebus}

\affiliation{Department of Physics, Boston College, 140 Commonwealth Avenue Chestnut Hill, MA 02467, USA}


\date{\today}% It is always \today, today,
             %  but any date may be explicitly specified
\maketitle



\section*{S1. Master equation framework}




\subsection{Conditional non-Hermitian Hamiltonian for spin waves }


In this section, we derive the conditional non-Hermitian Hamiltonian (5) presented in the main text. We begin by rewriting the master equation (1) as
\begin{align}
    \frac{d}{dt}\rho
    &=-i[\mathcal{H},\rho]+\sum_{\alpha,\beta=1}^N  \Gamma_{\alpha \beta}\left(\hat{s}_\beta^+\rho \hat{s}_\alpha^--\frac{1}{2}\{\hat{s}_\alpha^-\hat{s}_\beta^+,\rho\}\right)+\sum_{\alpha,\beta=1}^N\tilde{\Gamma}_{\alpha \beta}\left(\hat{s}_\beta^-\rho \hat{s}_\alpha^+- \frac{1}{2}\{\hat{s}_\alpha^+\hat{s}_\beta^-,\rho\}\right),\nonumber\\
    &=-i\left(\mathcal{H}_{\text{nh}}\rho-\rho{\mathcal{H}}_{\text{nh}}^{\dag}\right)+\sum_{\alpha,\beta=1}^N  \Gamma_{\alpha \beta}\hat{s}_\beta^+\rho \hat{s}_\alpha^-+\sum_{\alpha,\beta=1}^N\tilde{\Gamma}_{\alpha\beta}\hat{s}_\beta^-\rho\hat{s}_\alpha^+.\label{effhamil}
\end{align}
By neglecting the last two terms in the second row of Eq.~(\ref{effhamil})—i.e., the quantum jump terms—we obtain a conditional evolution equation,

\begin{align}
    \frac{d}{dt}\rho
    &=-i\left(\mathcal{H}_{\text{nh}}\rho-\rho{\mathcal{H}}_{\text{nh}}^\dag\right),\label{trunceffhamil}
\end{align}
where the conditional non-Hermitian Hamiltonian $\mathcal{H}_{\text{nh}}$ is given as follows:
\begin{align}
\mathcal{H}_{\text{nh}}&=-\sum_{\alpha=1}^N\omega_\alpha \hat{s}^z_\alpha+\frac{1}{2}\sum_{\substack{\alpha,\beta=1\\ \alpha\neq\beta }}^N 
 \mathcal{J}_{\alpha\beta}\hat{s}_{\alpha}^+ 
 \, \hat{s}_{\beta}^--\frac{i}{2}\sum_{\alpha,\beta=1}^N\Gamma_{\alpha\beta}\hat{s}_\alpha^-\hat{s}_\beta^+-\frac{i}{2}\sum_{\alpha,\beta=1}^N\tilde{\Gamma}_{\alpha\beta}\hat{s}_\alpha^+\hat{s}_\beta^-,\nonumber\\
&=-\sum_{\alpha=1}^N\left(\omega_\alpha+i\tilde{\Gamma}_{\alpha\alpha}\right) \hat{s}^z_\alpha-\frac{i}{2}\sum_{\alpha,\beta=1}^N\left(i\mathcal{J}_{\beta\alpha}+\Gamma_{\alpha\beta}+\tilde{\Gamma}_{\beta\alpha}\right)\hat{s}_\alpha^-\hat{s}_\beta^+,\nonumber\\
&=-\text{Tr}[(\boldsymbol{\Omega}+i\boldsymbol{\tilde{\Gamma}}_0)\boldsymbol{\mathcal{S}}^z]-\frac{i}{2}\mathbf{s}^{-}\boldsymbol{\mathcal{K}}_{\text{nh}}(\mathbf{s}^{+})^T.\label{nonherhamspinwave}
\end{align}
Here we have used the commutation relation $[\hat{s}^+_{\alpha}, \hat{s}^-_{\beta}] = 2\delta_{\alpha\beta} \hat{s}^z_{\alpha}$ and the Hermiticity of the coherent interaction matrix $\boldsymbol{\mathcal{J}}$ and the decoherence matrix $\boldsymbol{\Gamma}$, i.e., $\mathcal{J}_{\beta\alpha} = \mathcal{J}^*_{\alpha\beta}$ and $\Gamma_{\beta\alpha} = \Gamma^*_{\alpha\beta}$. In the final row, we expressed $\mathcal{H}_{\text{nh}}$ in matrix form by introducing the vectors $\mathbf{s}^{-} = (\hat{s}_1^{-}, \cdots, \hat{s}_N^{-})$ and $\mathbf{s}^{+} = (\hat{s}_1^{+}, \cdots, \hat{s}_N^{+})^T$. Here, the matrices $\mathbf{\Omega}$, $\boldsymbol{\mathcal{S}}^z$, and $\boldsymbol{\mathcal{K}}_{\text{nh}}$ are defined as $\mathbf{\Omega} = \text{diag}(\omega_1, \cdots, \omega_N)$, $\boldsymbol{\mathcal{S}}^z = \text{diag}(s_1^z, \cdots, s_N^z)$, and $\boldsymbol{\mathcal{K}}_{\text{nh}} = i\boldsymbol{\mathcal{J}}^* + \boldsymbol{\Gamma} + \boldsymbol{\tilde{\Gamma}}^*$.




\subsection{Dynamical equations for two-point correlation functions}

We now  derive Eq.~(7) in the main text. We start by considering a general Lindblad master equation:
\begin{align}
    \frac{d}{dt}\rho=-i[\mathcal{H},\rho]+\mathcal{D}[\rho],
\end{align}
where the Lindbladian superoperator $\mathcal{D}[\rho]$ takes the form
\begin{align}
    \mathcal{D}[\rho]=\sum_{\alpha,\beta=1}^N\gamma_{\alpha\beta}\left(L_\beta\rho L_\alpha^\dag-\frac{1}{2}\{L_\alpha^\dag L_\beta,\rho\}\right),
\end{align}
with $L_{\alpha}$ and $L_{\beta}$ denoting the Lindblad operators. One can show that an operator $\mathcal{O}$ evolves according to the following equation: 

\begin{align}
    \frac{d}{dt} \mathcal{O}
    = i[\mathcal{H},\mathcal{O}]+\mathcal{D}^\dag[\mathcal{O}],\label{oopdjoint}
\end{align}
where the adjoint Lindbladian $\mathcal{D}^\dag[\mathcal{O}]$ is defined as
\begin{align}
    \mathcal{D}^\dag[\mathcal{O}]=\sum_{\alpha,\beta=1}^N\gamma_{\alpha\beta}\left(L_\alpha^\dag\mathcal{O}L_{\beta}-\frac{1}{2}\{L_{\alpha}^\dag L_{\beta},\mathcal{O}\}\right).
\end{align}
Taking $\mathcal{O}=\hat{s}^-_l\hat{s}^+_k$ and $L_\alpha=\hat{s}^-_\alpha$ in \eqref{oopdjoint}, we obtain the evolution equation for the two-point correlation operator $\hat{s}^-_l\hat{s}^+_k$ as follows:

\begin{align}
    \frac{d}{dt} \hat{s}^-_l\hat{s}^+_k =i [\mathcal{H},\hat{s}^-_l\hat{s}^+_k] + \mathcal{D}^\dag[\hat{s}^-_l\hat{s}^+_k].\label{twopointeq}
\end{align}
For the first term on the right hand side of Eq.~(\ref{twopointeq}), we have:

\begin{align}
    i[\mathcal{H},\hat{s}^-_l\hat{s}^+_k]&=i[-\sum_{\alpha=1}^N\omega_\alpha \hat{s}^z_\alpha+\frac{1}{2}\sum_{\substack{\alpha,\beta=1\\ \alpha\neq\beta }}^N 
 \mathcal{J}_{\alpha\beta}\hat{s}_{\alpha}^+ 
 \, \hat{s}_{\beta}^-,\hat{s}^-_l\hat{s}^+_k],\nonumber\\
 &=-i \hat{s}^-_l\hat{s}_{k}^+\omega_k-i\sum_{\substack{\alpha=1\\ \alpha\neq\,k}}^N\hat{s}^-_l\hat{s}_{\alpha}^+\mathcal{J}_{\alpha\,k}\hat{s}^z_k+i\omega_l\hat{s}^-_l\hat{s}^+_k+i\sum_{\substack{\alpha=1\\ \alpha\neq\,l}}^N\hat{s}^z_l\mathcal{J}_{l\alpha}\hat{s}_{\alpha}^-\hat{s}^+_k.\label{spinwaveham}
\end{align} 

For the second term on the right hand side of Eq.~(\ref{twopointeq}), we have:

\begin{align}
\mathcal{D}^\dag[\hat{s}^-_l\hat{s}^+_k] 
 &=\sum_{\alpha,\beta=1}^N  \Gamma_{\alpha \beta}\left(\hat{s}_\alpha^-\hat{s}^-_l\hat{s}^+_k \hat{s}_\beta^+-\frac{1}{2}\{\hat{s}_\alpha^-\hat{s}_\beta^+,\hat{s}^-_l\hat{s}^+_k\}\right) +\sum_{\alpha,\beta=1}^N\tilde{\Gamma}_{\alpha \beta}\left(\hat{s}_\alpha^+\hat{s}^-_l\hat{s}^+_k \hat{s}_\beta^-- \frac{1}{2}\{\hat{s}_\alpha^+\hat{s}_\beta^-,\hat{s}^-_l\hat{s}^+_k\}\right),\nonumber\\
  &=-\sum_{\alpha=1}^N  \Gamma_{\alpha \,l}\hat{s}_\alpha^-\hat{s}_l^z\hat{s}^+_k-\sum_{\alpha=1}^N  \Gamma_{k \alpha}\hat{s}^-_l\hat{s}^z_k\hat{s}_\alpha^+
 +\sum_{\alpha=1}^N\tilde{\Gamma}_{\alpha \,k}\hat{s}_\alpha^+\hat{s}^-_l\hat{s}^z_k+\sum_{\alpha=1}^N\tilde{\Gamma}_{l \alpha}\hat{s}^z_l\hat{s}^+_k\hat{s}_\alpha^-,\nonumber\\
 &=-\sum_{\alpha=1}^N  \hat{s}_l^z\Gamma^T_{l\alpha }\hat{s}_\alpha^-\hat{s}^+_k+\sum_{\alpha=1}^N\hat{s}^z_l\tilde{\Gamma}_{l \alpha}\hat{s}_\alpha^-\hat{s}^+_k-\sum_{\alpha=1}^N  \hat{s}^-_l\hat{s}_\alpha^+\Gamma^T_{\alpha\,k }\hat{s}^z_k
 +\sum_{\alpha=1}^N\hat{s}^-_l\hat{s}_\alpha^+\tilde{\Gamma}_{\alpha \,k}\hat{s}^z_k,\nonumber\\
 &\quad-  \Gamma_{l \,l}\hat{s}_l^-\hat{s}^+_k- \hat{s}^-_l\hat{s}^+_k\Gamma_{k k}+4\hat{s}^z_l\tilde{\Gamma}_{lk}\hat{s}^z_k,\label{spinwavelinb}
\end{align}
where we used the commutation relations, $[\hat{s}^+_\alpha,\hat{s}^-_\beta]=2\delta_{\alpha\beta}\hat{s}^z_\alpha$ and $[\hat{s}^z_\alpha,\hat{s}^{\pm}_\beta]=\pm\delta_{\alpha\beta}\hat{s}^{\pm}_\alpha$.

Combining~\eqref{spinwaveham} and \eqref{spinwavelinb}, we obtain
\begin{align}
    \frac{d}{dt}\hat{s}^-_l\hat{s}^+_k&=i\omega_ls^-_l\hat{s}^+_k-  \Gamma_{l \,l}\hat{s}_l^-\hat{s}^+_k+i\sum_{\substack{\alpha=1\\ \alpha\neq\,l}}^Ns^z_l\mathcal{J}_{l\alpha}\hat{s}_{\alpha}^-\hat{s}^+_k-\sum_{\alpha=1}^N  \hat{s}_l^z\Gamma^T_{l\alpha }\hat{s}_\alpha^-\hat{s}^+_k+\sum_{\alpha=1}^N\hat{s}^z_l\tilde{\Gamma}_{l \alpha}\hat{s}_\alpha^-\hat{s}^+_k\nonumber\\
    &\quad-i \hat{s}^-_l\hat{s}_{k}^+\omega_k- \hat{s}^-_l\hat{s}^+_k\Gamma_{k k}-i\sum_{\substack{\alpha=1\\ \alpha\neq\,k}}^N\hat{s}^-_l\hat{s}_{\alpha}^+\mathcal{J}_{\alpha\,k}\hat{s}^z_k-\sum_{\alpha=1}^N  \hat{s}^-_l\hat{s}_\alpha^+\Gamma^T_{\alpha\,k }\hat{s}^z_k
 +\sum_{\alpha=1}^N\hat{s}^-_l\hat{s}_\alpha^+\tilde{\Gamma}_{\alpha \,k}\hat{s}^z_k+4\hat{s}^z_l\tilde{\Gamma}_{lk}\hat{s}^z_k.
\end{align}
In terms of matrices, with $\mathcal{A}_{\alpha\beta}=\hat{s}^-_\alpha\hat{s}^+_\beta$, we finally obtain Eq.~(7) in the main text,
\begin{align}
    \frac{d}{dt}\langle\boldsymbol{\mathcal{A}}\rangle
    =-\langle\boldsymbol{\mathcal{N}}\boldsymbol{\mathcal{A}}\rangle-\langle\boldsymbol{\mathcal{A}}\boldsymbol{\mathcal{N}}^\dag\rangle+4\langle\boldsymbol{\mathcal{S}}^z\boldsymbol{\tilde{\Gamma}}\boldsymbol{\mathcal{S}}^z\rangle,
\end{align}
where the dynamical matrix $\boldsymbol{\mathcal{N}}$ is defined as
\begin{align}
    \boldsymbol{\mathcal{N}}=-i\mathbf{\Omega}+\boldsymbol{\Gamma}_0+\boldsymbol{\mathcal{S}}^z(-i\boldsymbol{\mathcal{J}}+\boldsymbol{\Gamma}^*-\boldsymbol{\tilde{\Gamma}})=-i\mathbf{\Omega}+\boldsymbol{\Gamma}_0+\boldsymbol{\mathcal{S}}^z\boldsymbol{\mathcal{K}}_{\mathcal{L}},
\end{align}
with 
\begin{align}
    \boldsymbol{\mathcal{K}}_{\mathcal{L}}=-i\boldsymbol{\mathcal{J}}+\boldsymbol{\Gamma}^*-\boldsymbol{\tilde{\Gamma}}.
\end{align}

\subsection{Magnon picture}

In the dilute magnon regime at zero temperature, we apply the Holstein–Primakoff transformation 
$\hat{s}^+ \approx \sqrt{2s}\,\hat{a}$, 
$\hat{s}^- \approx \sqrt{2s}\,\hat{a}^\dagger$, 
and $\hat{s}^z = s - \hat{a}^\dagger \hat{a}$ 
to express the conditional non-Hermitian spin Hamiltonian [Eq.~(8) of the main text] in terms of bosonic operators. 
Neglecting the constant energy shift $-s\sum_\alpha \omega_\alpha$, the resulting magnon Hamiltonian reads:

\begin{align}
    \mathcal{H}_{\text{m}}
    &=-\sum_{\alpha=1}^N\omega_{\alpha}(s-\hat{a}_\alpha^\dag\hat{a}_\alpha)+s\sum_{\substack{\alpha,\beta=1\\\alpha\neq\beta}}^N\mathcal{J}_{\alpha\beta}\hat{a}^\dag_{\beta}\hat{a}_{\alpha}-is\sum_{\alpha,\beta=1}^N\Gamma_{\alpha\beta}\hat{a}^\dag_{\alpha}\hat{a}_{\beta},\nonumber\\
&=\boldsymbol{\hat{a}}^\dag(\mathbf{\Omega}+s\boldsymbol{\mathcal{J}}^*-is\boldsymbol{\Gamma})\boldsymbol{\hat{a}}^T,\nonumber\\
&=\boldsymbol{\hat{a}}^\dag(-i\boldsymbol{\mathcal{H}}_{\text{m}})\boldsymbol{\hat{a}}^T,
\end{align}
where $\boldsymbol{\mathcal{H}}_{\text{m}} = i\mathbf{\Omega} + s(i\boldsymbol{\mathcal{J}}^* + \boldsymbol{\Gamma})$.

Next, we derive Eq.~(11) in the main text using Eq.~(\ref{oopdjoint}), i.e.,
\begin{align}
    \frac{d}{dt}\hat{a}_l^\dag\hat{a}_k&=i [\mathcal{H},\hat{a}_l^\dag\hat{a}_k] + \mathcal{D}^\dag[\hat{a}_l^\dag\hat{a}_k].\label{magnondynamical}
\end{align}

For the first term on the right-hand side of Eq.~\eqref{magnondynamical}, we obtain:
\begin{align}
    i [\mathcal{H},\,\hat{a}_l^\dag\hat{a}_k]&=i [-s\sum_{\alpha=1}^N\omega_\alpha+\sum_{\alpha=1}^N\,\omega_\alpha\hat{a}_\alpha^\dag\hat{a}_\alpha+s\sum_{\substack{\alpha,\beta=1\\\alpha\neq\beta}}^N 
 \mathcal{J}_{\beta\alpha}\hat{a}_{\alpha}^\dag\hat{a}_{\beta},\,\hat{a}_l^\dag\hat{a}_k],\nonumber\\
 &=-i \sum_{\alpha=1}^N \hat{a}_l^\dag\hat{a}_\alpha\omega_{\alpha}\delta_{\alpha\,k}+i \sum_{\alpha=1}^N\delta_{l\alpha}\omega_\alpha\,\hat{a}_\alpha^\dag\hat{a}_k-i s\sum_{\substack{\alpha=1\\\alpha\neq\,k}}^N 
 \mathcal{J}_{\alpha\,k}\hat{a}_l^\dag\,\hat{a}_{\alpha}+i s\sum_{\substack{\alpha=1\\\alpha\neq\,l}}^N 
 \mathcal{J}_{l\alpha}\hat{a}_{\alpha}^\dag\,\hat{a}_k.\label{cohemagon}
\end{align}

For the second term on the right-hand side of Eq.~\eqref{magnondynamical}, we find:

\begin{align}
    \mathcal{D}^\dag[\hat{a}_l^\dag\,\hat{a}_k]&=\sum_{\alpha,\beta=1}^N  2s\Gamma_{\alpha \beta}\left(\hat{a}_\alpha^\dag\,\hat{a}_l^\dag\,\hat{a}_k\,\hat{a}_{\beta} -\frac{1}{2}\{\hat{a}_\alpha^\dag\,\hat{a}_\beta,\hat{a}_l^\dag\,\hat{a}_k\}\right),\nonumber\\
    &= -s\sum_{\alpha=1}^N \Gamma^*_{l\alpha}\hat{a}_\alpha^\dag\,\hat{a}_k-s\sum_{\alpha=1}^N \Gamma_{k \alpha}\hat{a}_l^\dag\,\hat{a}_\alpha.
    \label{Lindmagon}
\end{align}

Combining Eqs. (\ref{cohemagon}) and (\ref{Lindmagon}), we obtain

\begin{align}
    \frac{d}{dt}\langle\hat{a}_l^\dag\hat{a}_k\rangle&=i\sum_{\alpha=1}^N\delta_{l\alpha}\omega_\alpha\langle\hat{a}_\alpha^\dag\hat{a}_k\rangle+i s\sum_{\substack{\alpha=1\\\alpha\neq\,l}}^N 
 \mathcal{J}_{l\alpha}\langle\hat{a}_\alpha^\dag\hat{a}_k\rangle-s\sum_{\alpha=1}^N \Gamma^*_{l\alpha}\langle\hat{a}_\alpha^\dag\,\hat{a}_k\rangle\nonumber\\
 &-i \sum_{\alpha=1}^N \langle\hat{a}_l^\dag\hat{a}_\alpha\rangle\omega_{\alpha}\delta_{\alpha\,k}-i s\sum_{\substack{\alpha=1\\\alpha\neq\,k}}^N 
 \langle\hat{a}_l^\dag\hat{a}_{\alpha}\rangle\mathcal{J}_{\alpha\,k}  -s\sum_{\alpha=1}^N \langle\hat{a}_l^\dag\hat{a}_\alpha\rangle\Gamma^*_{ \alpha\,k}.\label{magnoncorre}
\end{align}
In terms of matrix, Eq.~(\ref{magnoncorre}) can be rewritten as

\begin{align}
    \frac{d}{dt}\boldsymbol{\mathcal{C}}
    =-\boldsymbol{\mathcal{H}}_{\text{m}}^*\boldsymbol{\mathcal{C}}-\boldsymbol{\mathcal{C}}(\boldsymbol{\mathcal{H}}_{\text{m}}^*)^\dag,
\end{align}
which is Eq.~(11) in the main text.

\subsection{Third quantization method}

In this section, we apply the third quantization method~\cite{prosen2008third,prosen2010quantization} to derive Eq.~(9) of the main text. We start from the Lindbladian master equation expressed in terms of magnon operators:


\begin{align}
    \frac{d}{dt}\rho
    &=-i[\mathcal{H},\rho]+\mathcal{D}[\rho],
\end{align}
with
\begin{align}
    \mathcal{H}&=\,-s\sum_{\alpha=1}^N\omega_\alpha +\sum_{\alpha=1}^N\omega_\alpha \,\hat{a}_\alpha^\dag\,\hat{a}_\alpha+s\sum_{\substack{\alpha,\beta=1\\\alpha\neq\beta}}^N 
\mathcal{J}^T_{\beta\alpha}\hat{a}_{\beta}^\dag\,\hat{a}_{\alpha},\\
\mathcal{D}[\rho]
    &=\sum_{\alpha,\beta=1}^N  2s\Gamma_{\alpha \beta}\left(\hat{a}_{\beta}\rho \hat{a}_\alpha^\dag-\frac{1}{2}\{\hat{a}_\alpha^\dag\hat{a}_\beta,\rho\}\right).
\end{align}
The quantum jump terms,  such as \( \hat{a}_\alpha \rho \hat{a}_\beta^\dag \) and \( \hat{a}_\beta^\dag \rho \hat{a}_\alpha \),  represent left and right actions of operators on \( \rho \), with the first operator acting from the left and the second from the right. To formalize this structure, we introduce the left and right multiplication superoperators \( \hat{a}^L \) and \( \hat{a}^R \), defined by
\begin{align}
\hat{a}^L \ket{\rho} = \ket{\hat{a} \rho}, \qquad \hat{a}^R \ket{\rho} = \ket{\rho \hat{a}},
\label{super}
\end{align}
where \( \ket{\rho} \) denotes the vectorized form of the density matrix.
The superoperators~\ref{super} obey
the following relations
\begin{align}
    \hat{a}_\alpha^L\hat{a}_\beta^R\ket{\rho}=\ket{\hat{a}_\alpha\rho\hat{a}_\beta}=\hat{a}_\beta^R\hat{a}_\alpha^L\ket{\rho},\qquad [\hat{a}_\alpha^L,\hat{a}_\beta^R]=0.
\end{align}

Then the Hamiltonian part can be written as
\begin{align}
    -i[\mathcal{H},\rho]
    &\,\rightarrow\,(-i\mathcal{H}^L+i\mathcal{H}^R)\ket{\rho},\nonumber\\
&=\Big[-i\sum_{\alpha=1}^N\omega_\alpha (\hat{a}_\alpha^{\dag\,L}\hat{a}_\alpha^L- \hat{a}_\alpha^R\hat{a}_\alpha^{\dag\,R})+i\sum_{\substack{\alpha,\beta=1\\\alpha\neq\beta}}^N 
s\mathcal{J}^T_{\beta\alpha}(\hat{a}_{\alpha}^R\hat{a}_{\beta}^{\dag\,R}-\hat{a}_{\beta}^{\dag\,L}\hat{a}_{\alpha}^L)\Big]\ket{\rho}.
\end{align}


Note that the two operators $\hat{a}_\alpha^{\dag\,L}-\hat{a}_\alpha^{\dag\,R}$ and $\hat{a}_{\alpha}^R-\hat{a}_{\alpha}^L$ annihilate the identity operator $\hat{I}$,
\begin{align}
        (\hat{a}_\alpha^{\dag\,L}-\hat{a}_\alpha^{\dag\,R})\ket{\hat{I}}=\ket{(\hat{a}_\alpha^{\dag}-\hat{a}_\alpha^{\dag})}=0,\qquad (\hat{a}_\alpha^{R}-\hat{a}_\alpha^{L})\ket{\hat{I}}=\ket{(\hat{a}_\alpha-\hat{a}_\alpha)}=0,\qquad 
    \hat{a}^{L}\ket{\rho_0}=0,\qquad \hat{a}^{\dag\,R}\ket{\rho_0}=0,
\end{align}

which motivates the introduction of four new particle operators,

\begin{align}
    \hat{b}_{1,\alpha}=\hat{a}_{\alpha}^L,\quad \hat{b}_{2,\alpha}=\hat{a}_\alpha^{\dag\,R},\qquad  \hat{\tilde{b}}_{1,\alpha}=\hat{b}_{1,\alpha}^\dag-\hat{b}_{2,\alpha}=\hat{a}_\alpha^{\dag\,L}-\hat{a}_\alpha^{\dag\,R},\quad \hat{\tilde{b}}_{2,\alpha}=\hat{b}_{2,\alpha}^\dag-\hat{b}_{1,\alpha}=\hat{a}_{\alpha}^R-\hat{a}_\alpha^L,
\end{align}
satisfying the following commuting relation
\begin{align}
    \,[\hat{b}_{\mu,\alpha},\hat{b}_{\nu,\beta}]=0,\qquad [\hat{\tilde{b}}_{\mu,\alpha},\hat{\tilde{b}}_{\nu,\beta}]=0,\qquad [\hat{b}_{\mu,\alpha},\hat{\tilde{b}}_{\nu,\beta}]=\delta_{\mu\nu}\delta_{\alpha\beta}.
\end{align}



In terms of the new operators, the Hamiltonian part, can be written as


\begin{align}
    -i[\mathcal{H},\rho]
    &\,\rightarrow\,(-i\mathcal{H}^L+i\mathcal{H}^R)\ket{\rho},\nonumber\\
&=\left[-i\sum_{\alpha=1}^N\omega_\alpha (\hat{\tilde{b}}_{1,\alpha}\,\hat{b}_{1,\alpha}- \hat{\tilde{b}}_{2,\alpha}\,\hat{b}_{2,\alpha})+i\sum_{\substack{\alpha,\beta=1\\\alpha\neq\beta}}^N 
s\mathcal{J}^T_{\beta\alpha}(\hat{\tilde{b}}_{2,\alpha}\hat{b}_{2,\beta}-\hat{\tilde{b}}_{1,\beta}\,\hat{b}_{1,\alpha})\right]\ket{\rho},\nonumber\\
&=[-i(\hat{\boldsymbol{\tilde{b}}}_{1}\boldsymbol{\Omega}\,\hat{\boldsymbol{b}}_{1}^T- \hat{\boldsymbol{\tilde{b}}}_{2}\boldsymbol{\Omega}\,\hat{\boldsymbol{b}}_{2}^T)+i 
s(\hat{\boldsymbol{\tilde{b}}}_{2}\boldsymbol{\mathcal{J}}{\boldsymbol{b}}^T_{2}-\hat{\boldsymbol{\tilde{b}}}_{1}\boldsymbol{\mathcal{J}}^T\,\hat{\boldsymbol{b}}^T_{1})]\ket{\rho},\nonumber\\
&=[ {\hat{\boldsymbol{\tilde{b}}}_{1}}(-i\boldsymbol{\Omega}-i 
s\boldsymbol{\mathcal{J}}^*)\hat{\boldsymbol{b}}^T_{1}+\hat{\boldsymbol{\tilde{b}}}_{2}( i\boldsymbol{\Omega}+i 
s\boldsymbol{\mathcal{J}})\hat{\boldsymbol{b}}^T_{2}]\ket{\rho},\label{thirdcoherent}
\end{align}
with
\begin{align}
    \hat{\mathbf{b}}_{\mu}=(\hat{b}_{\mu,1},\cdots,\hat{b}_{\mu,N}),\qquad \hat{\boldsymbol{\tilde{b}}}_{\mu}=(\hat{\tilde{b}}_{\mu,1},\cdots,\hat{\tilde{b}}_{\mu,N}),\qquad \mu=1,2.
\end{align}

Similarly, we can rewrite the Lindbladian as 

\begin{align}
    \mathcal{D}[\rho]
    &\rightarrow\sum_{\alpha,\beta=1}^N 2s\Gamma_{\alpha \beta} \left(\hat{a}^L_{\beta} \hat{a}_{\alpha}^{\dag\,R}-\frac{1}{2}[(\hat{a}_\alpha^\dag\,\hat{a}_\beta)^L+(\hat{a}_\alpha^\dag\,\hat{a}_\beta)^R]\right)\ket{\rho},\nonumber\\
    &=\sum_{\alpha,\beta=1}^N -s\Gamma_{\alpha \beta} \left[(\hat{a}_\alpha^{\dag\,L}-\hat{a}_\alpha^{\dag\,R})\hat{a}^L_\beta+(\hat{a}_\beta^R-\hat{a}_\beta^L)\hat{a}_\alpha^{\dag\,R}\right]\ket{\rho},\nonumber\\
    &= -s\left(\hat{\boldsymbol{\tilde{b}}}_{1}\boldsymbol{\Gamma} \hat{\boldsymbol{b}}^T_{1}+\hat{\boldsymbol{\tilde{b}}}_{2}\boldsymbol{\Gamma}^T\hat{\boldsymbol{b}}^T_{2}\right)\ket{\rho}.\label{thirdlindbla}
\end{align}




Combining Eq.(\ref{thirdcoherent}) and (\ref{thirdlindbla}), we obtain 

\begin{align}
    \mathcal{L}_{\text{m}}
    &=- \hat{\boldsymbol{\tilde{b}}}_{1}[i\boldsymbol{\Omega}+s(i 
\boldsymbol{\mathcal{J}}^*+\boldsymbol{\Gamma})]\hat{\boldsymbol{b}}^T_{1}-\hat{\boldsymbol{\tilde{b}}}_{2}[-i\boldsymbol{\Omega}+s(-i 
\boldsymbol{\mathcal{J}}+\boldsymbol{\Gamma}^*)]\hat{\boldsymbol{b}}^T_{2},\nonumber\\
&=\hat{\boldsymbol{\tilde{b}}}_{1}(-\boldsymbol{\mathcal{H}}_{\text{m}})\hat{\boldsymbol{b}}^T_{1}+\hat{\boldsymbol{\tilde{b}}}_{2}(-\boldsymbol{\mathcal{H}}^*_{\text{m}})\hat{\boldsymbol{b}}^T_{2},\nonumber\\
&=\hat{\boldsymbol{\tilde{b}}}_{1}(-\frac{1}{2}\boldsymbol{\mathcal{H}}_{\text{m}})\hat{\boldsymbol{b}}^T_{1}+\hat{\boldsymbol{\tilde{b}}}_{2}(-\frac{1}{2}\boldsymbol{\mathcal{H}}^*_{\text{m}})\hat{\boldsymbol{b}}^T_{2}+\hat{\boldsymbol{\tilde{b}}}_{1}(-\frac{1}{2}\boldsymbol{\mathcal{H}}_{\text{m}})\hat{\boldsymbol{b}}^T_{1}+\hat{\boldsymbol{\tilde{b}}}_{2}(-\frac{1}{2}\boldsymbol{\mathcal{H}}_{\text{m}}^*)\hat{\boldsymbol{b}}^T_{2}\nonumber\\
&=\hat{\boldsymbol{\tilde{b}}}_{1}(-\frac{1}{2}\boldsymbol{\mathcal{H}}_{\text{m}})\hat{\boldsymbol{b}}^T_{1}+\hat{\boldsymbol{\tilde{b}}}_{2}(-\frac{1}{2}\boldsymbol{\mathcal{H}}_{\text{m}}^*)\boldsymbol{b}^T_{2}+\hat{\boldsymbol{b}}_{1}(-\frac{1}{2}\boldsymbol{\mathcal{H}}_{\text{m}})^T\hat{\boldsymbol{\tilde{b}}}^T_{1}+\hat{\boldsymbol{b}}_{2}(-\frac{1}{2}\boldsymbol{\mathcal{H}}_{\text{m}}^*)^T\hat{\boldsymbol{\tilde{b}}}^T_{2}+\frac{1}{2}\text{Tr}\boldsymbol{\mathcal{H}}_{\text{m}}+\frac{1}{2}\text{Tr}\boldsymbol{\mathcal{H}}^*_{\text{m}}.\label{linbdthirdfin}
\end{align}





Eq.~(\ref{linbdthirdfin}) can be rewritten in matrix form, 
\begin{align}  
\mathcal{L}_{\text{m}}
     &=(\hat{\boldsymbol{b}}_1,\hat{\boldsymbol{b}}_2,\hat{\tilde{\boldsymbol{b}}}_1,\hat{\tilde{\boldsymbol{b}}}_2)
     \left(
     \begin{array}{cc}
         0 & -\mathbf{X}^T \\
       - \mathbf{X}  & 0
     \end{array}
     \right)\left(
     \begin{array}{cccc}
     \hat{\boldsymbol{b}}_1\\
     \hat{\boldsymbol{b}}_2\\
     \hat{\boldsymbol{\tilde{b}}}_1\\
     \hat{\boldsymbol{\tilde{b}}}_2
     \end{array}
     \right)+s\text{Tr}\boldsymbol{\Gamma}\label{thirdlimcontex}
\end{align}
with
\begin{align}
    \mathbf{X}=\frac{1}{2}\left(
    \begin{array}{cc}
       \boldsymbol{\mathcal{H}}_{\text{m}}  & 0 \\
       0  & \boldsymbol{\mathcal{H}}_{\text{m}}^*
    \end{array}
    \right).\label{thirdmatcontex}
\end{align}
Eqs.~(\ref{thirdlimcontex}) and (\ref{thirdmatcontex}) are respectively Eqs.~(9) and (10) in the main text.






\subsection{Diagonalization of the effective non-Hermitian Hamiltonian for nearest neighbor interactions}




In the  non-unidirectional case, $\gamma_L\gamma_R\neq0$,   the non-Hermitian Hamiltonian in Eq.~(12) takes the form of a tridiagonal Toeplitz matrix:




\begin{eqnarray}
    \mathcal{H}_{\text{m}}=\left(
    \begin{array}{ccccccccc}
       \epsilon_0  & \gamma_L&&&& \\
        \gamma_R & \epsilon_0&\gamma_L&&&\\
        & \gamma_R&\ddots&\ddots&&\\
        &&\ddots & \ddots&\gamma_L&\\
        &&& \gamma_R& \epsilon_0&\gamma_L\\
        &&&&\gamma_R & \epsilon_0
    \end{array}
    \right)_{N\times N}.\label{nhhamilmat}
\end{eqnarray}
Diagonalizing Eq.~(\ref{nhhamilmat}), we obtain a series of eigenvalues:


 \begin{eqnarray}
    \lambda_n=\epsilon_0+2\sqrt{\gamma_L\gamma_R}\cos\frac{n\pi}{N+1},\quad n=1, \cdots, N\, ,
\end{eqnarray}
and the corresponding right and left eigenvectors 
\begin{eqnarray}
    |\Psi^R_n\rangle=(\psi^R_{n,1},\cdots,\psi^R_{n,\alpha},\cdots,\psi^R_{n,N})^T,\qquad
    |\Psi^L_n\rangle=(\psi^L_{n,1},\cdots,\psi^L_{n,\alpha},\cdots,\psi^L_{n,N})^T\,,
\end{eqnarray}
with
\begin{eqnarray}
    \psi^R_{n,\alpha}=\left(\frac{\gamma_R}{\gamma_L}\right)^{\alpha/2}\sin\frac{n\alpha\pi}{N+1}\,,\qquad   \psi^L_{n,\alpha}=\left(\frac{\gamma^*_L}{\gamma^*_R}\right)^{\alpha/2}\sin\frac{n\alpha\pi}{N+1}\,,\qquad\qquad n,\alpha =1,\cdots, N\,,
\end{eqnarray}
from which Eqs.~(13) of the main text follow directly.
On the other hand, if the system satisfies the unidirectional condition, i.e.,  $\gamma_L=0$ or $\gamma_R=0$, the Hamiltonian 
~(12) reduces to a Jordan  block of size $N$,


\begin{eqnarray}
\mathcal{H}^L_{\text{m}}=\left(
    \begin{array}{ccccccccc}
       \epsilon_0  & \gamma_L&&&& \\
        0 & \epsilon_0&\gamma_L&&&\\
        & 0&\ddots&\ddots&&\\
        &&\ddots & \ddots&\gamma_L&\\
        &&& 0& \epsilon_0&\gamma_L\\
        &&&&0 & \epsilon_0
    \end{array}
    \right)_{N\times N}
,\qquad\qquad
\mathcal{H}^R_{\text{m}}=\left(
    \begin{array}{ccccccccc}
       \epsilon_0  & 0&&&& \\
        \gamma_R & \epsilon_0&0&&&\\
        & \gamma_R&\ddots&\ddots&&\\
        &&\ddots & \ddots&0&\\
        &&& \gamma_R& \epsilon_0&0\\
        &&&&\gamma_R & \epsilon_0
    \end{array}
    \right)_{N\times N}.\label{Jordanblock}
\end{eqnarray}
The Jordan block Hamiltonians in Eq.~(\ref{Jordanblock}) are non-diagonalizable. However, one can immediately identify the single eigenvalue $\lambda=\epsilon_0$ with algebraic multiplicity $N$ and the corresponding right and left eigenvectors

\begin{eqnarray}
    |\Psi^{R}_L\rangle=(1,0,\cdots,0)^T,\qquad |\Psi^{L}_L\rangle=(0,\cdots,0,1)^T,\label{jordanupp}
\end{eqnarray}
for $\mathcal{H}^L_{\text{m}}$,  and


\begin{eqnarray}
    |\Psi^{R}_R\rangle=(0,\cdots,0,1)^T,\qquad |\Psi^{L}_R\rangle=(1,0,\cdots,0)^T,\label{jordandown}
\end{eqnarray}
for $\mathcal{H}^R_{\text{m}}$. Equations~(\ref{jordanupp}) and (\ref{jordandown}) demonstrate that, under unidirectional conditions, the eigenmodes become fully localized at one boundary, exemplifying the non-Hermitian skin effect.





\section{S2. Classical magnetization dynamics}









Equation (14) can be rewritten  as

\begin{eqnarray}
    \frac{\partial \mathbf{m}_\alpha}{\partial t} 
    &=& -\frac{\gamma J}{M_s}\mathbf{m}_{\alpha}\times(\mathbf{m}_{\alpha-1}+\mathbf{m}_{\alpha+1})+\frac{\gamma}{M_s}\mathbf{m}_{\alpha}\times[(\mathbf{m}_{\alpha+1}\times\mathbf{D})-(\mathbf{m}_{\alpha-1}\times\mathbf{D})] -\gamma\mu_0 \mathbf{m}_{\alpha}\times \mathbf{H}\nonumber\\
&&+\alpha_l\mathbf{m}_\alpha\times\frac{\partial\mathbf{m}_\alpha}{\partial t}+\alpha_{nl}\mathbf{m}_\alpha\times\frac{\partial\mathbf{m}_{\alpha-1}}{\partial t}+\alpha_{nl}\mathbf{m}_\alpha\times\frac{\partial\mathbf{m}_{\alpha+1}}{\partial t}\ . \label{CoupledLLG}
\end{eqnarray}
For small fluctuations of the magnetic order parameter around the equilibrium direction, i.e., $\mathbf{m}_\alpha=(m_{\alpha}^{x},m_{\alpha}^{y},1)$ with $|m_{\alpha}^{x(y)}|\ll1$, we can simplify Eq.~(\ref{CoupledLLG}) by only retaining terms linear in $m_{\alpha}^x$ and $m_{\alpha}^y$. Similarly, for  $\alpha_l,\alpha_{nl}\ll1$,   terms of second or higher order in these parameters can be neglected. Then, invoking the Holstein-Primakoff transformation,  i.e., $\langle\hat{a}_{\alpha}\rangle\equiv m_{\alpha}^x+ im_{\alpha}^y$, Eq.~(\ref{CoupledLLG}) can be written as
\begin{eqnarray}
\frac{\partial \hat{a}_\alpha}{\partial t} 
\approx &&+ \frac{\gamma}{M_s}\left\{\Big[(i-\alpha_l)\left(2 J +\mu_0 M_sH
\right)
+2 J\alpha_{nl}\right]\hat{a}_{\alpha}
+\Big[(-i+\alpha_l)\left( J -i D\right)-\alpha_{nl}\left(2 J 
+\mu_0 M_sH
\right)\Big]\hat{a}_{\alpha-1}\nonumber\\
&&+\Big[(-i+\alpha_l)\left( J+i D\right)-\alpha_{nl}\left(2 J+\mu_0M_sH\right)\Big]\hat{a}_{\alpha+1} +\alpha_{nl}\left( J-i D\right)\hat{a}_{\alpha-2} +\alpha_{nl}\left( J+i D\right)\hat{a}_{\alpha+2}\Big\} . \label{LLGlinear}
\end{eqnarray}
As shown by the last two terms on the right-hand-side of Eq.~(\ref{LLGlinear}), the nonlocal dissipation terms, i.e., $\alpha_{nl}\mathbf{m}_\alpha\times \left(\frac{\partial\mathbf{m}{\alpha-1}}{\partial t}+\frac{\partial\mathbf{m}{\alpha+1}}{\partial t} \right)$,  gives rise to effective next nearest neighbor  interactions. For a bilayer, i.e., $N=2$,  Eq.~(\ref{LLGlinear}) simplifies to

\begin{eqnarray}
\frac{\partial \hat{a}_1}{\partial t} 
&=&\frac{\gamma}{M_s}\Big\{\left[(i-\alpha_l)\left(2 J +\mu_0 M_sH
\right)
+2 J\alpha_{nl}\right]\hat{a}_{1}
+\Big[(-i+\alpha_l)\left( J+i D\right)-\alpha_{nl}\left(2 J+\mu_0M_sH\right)\Big]\hat{a}_{2}\Big\}\, ,\label{pairlone}\\
\frac{\partial \hat{a}_2}{\partial t} 
&=&\frac{\gamma}{M_s}\Big\{\left[(i-\alpha_l)\left(2 J +\mu_0 M_sH
\right)
+2 J\alpha_{nl}\right]\hat{a}_{2}
+\Big[(-i+\alpha_l)\left( J -i D\right)-\alpha_{nl}\left(2 J 
+\mu_0 M_sH
\right)\Big]\hat{a}_{1}\Big\}\ .  \label{pairltwo}
\end{eqnarray}
To achieve unidirectional transport, the second terms on the right-hand-side of Eqs.~(\ref{pairlone}) or (\ref{pairltwo}) must vanish, i.e.,

\begin{eqnarray}
    J=\pm\alpha_lD,\qquad D=\pm\alpha_{nl}\mu_0M_sH(1+\alpha^2_l-2\alpha_{nl}\alpha_l)^{-1}\, ,\label{relationfortwo}
\end{eqnarray}
where the positive and negative signs correspond to the unidirectionality of Eqs.~(\ref{pairlone}) and (\ref{pairltwo}), respectively.
Recalling the condition $\alpha_{nl}, \alpha_{l}\ll 1$,  Eqs.~(\ref{relationfortwo}) can be simplified to

\begin{eqnarray}
    J=0,\qquad D=\pm\alpha_{nl}\mu_0M_sH .\label{relationfortwosim}
\end{eqnarray}
However, for magnetic multilayers with $N\geq3$,  unidirectionality cannot be achieved due to the emergence of effective next nearest neighbor couplings. This can be shown explicitly by considering Eq.~\eqref{LLGlinear} in the limit of an infinitely-long spin chain, i.e., $N \rightarrow \infty$.
Invoking the Heisenberg equation of motion, i.e., $d\hat{a}/dt=-i[\hat{a},\mathcal{H}]$, and performing a Fourier transform, one can derive an effective Hamiltonian in momentum space, whose spectrum $\tilde{\varepsilon}_k$ reads as


\begin{eqnarray}
    \tilde{\varepsilon}_k=-\frac{\gamma}{M_s}\Big[1+i(\alpha_0+2\alpha_{nl}\cos k)\Big]\Big[ J(1-\cos k)+\mu_0 H M_s+ D\sin k\Big]\, .\label{mometumham}
\end{eqnarray}
Pluggling the unidirectionality condition~(\ref{relationfortwosim}) into Eq.~(\ref{mometumham}) yields


\begin{eqnarray}
    \tilde{\varepsilon}_k=-\gamma\mu_0H\Big[(1\pm\alpha_{nl}\sin k)+i(\alpha_l+2\alpha_{nl}\cos k)\Big]\, .\label{llgmomentum}
\end{eqnarray}
For simplicity, we normalize Eq.~(\ref{llgmomentum}) by dividing $-\gamma\mu_0H$ on both sides, i.e., $\varepsilon=\tilde{\varepsilon}_k/(-\gamma\mu_0H)$ . It is then straightforward to determine the relationship between the real and imaginary parts as

\begin{eqnarray}
    \frac{(\text{Re}\varepsilon_k-1)^2}{\alpha_{nl}^2}+\frac{(\text{Im}\varepsilon_k-\alpha_{l})^2}{4\alpha_{nl}^2}=1\, .\label{llgellipse}
\end{eqnarray}
One can readily recognize Eq.~(\ref{llgellipse}) as describing an ellipse rather than a circle, indicating the absence of unidirectionality but the presence of nonreciprocity.  



\bibliographystyle{apsrev4-2}

\bibliography{suplibrary}



\end{document}



