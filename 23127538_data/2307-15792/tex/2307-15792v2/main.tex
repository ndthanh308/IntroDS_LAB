%% ****** Start of file template.aps ****** %
%%
%%
%%   This file is part of the APS files in the REVTeX 4 distribution.
%%   Version 4.0 of REVTeX, August 2001
%%
%%
%%   Copyright (c) 2001 The American Physical Society.
%%
%%   See the REVTeX 4 README file for restrictions and more informa≠=tion.
%%
%
% This is a template for producing manuscripts for use with REVTEX 4.0
% Copy this file to another name and then work on that file.
% That way, you always have this original template file to use.
%
% Group addresses by affiliation; use superscriptaddress for long
% author lists, or if there are many overlapping affiliations.
% For Phys. Rev. appearance, change preprint to twocolumn.
% Choose pra, prb, prc, prd, pre, prl, prstab, or rmp for journal
%  Add 'draft' option to mark overfull boxes with black boxes
%  Add 'showpacs' option to make PACS codes appear

\documentclass[aps,prl,twocolumn,showpacs,superscriptaddress]{revtex4-2}  

\usepackage{float}
\usepackage{bbm}
\usepackage{epsfig}
\usepackage{epstopdf}
\usepackage{graphicx}
\usepackage{amsmath,amssymb}
\usepackage{amsmath,bm}
\usepackage{physics}
\usepackage{color}
\usepackage{hyperref}
\usepackage{lineno,blindtext}
\setlength{\tabcolsep}{9pt}
\usepackage{siunitx, booktabs}
\usepackage{diagbox, eqparbox, hhline}
\usepackage{soul}
\usepackage{xcolor}
\usepackage{eufrak}
\usepackage{dsfont}
\usepackage[normalem]{ulem}
\setlength{\doublerulesep}{2.5pt}

\newcolumntype{P}[1]{>{\centering\arraybackslash}p{#1}}
\newcommand{\new}[1]{\textcolor{blue}{#1}}          % for markup
\newcommand{\cut}[1]{\textcolor{red}{\sout{#1}}}          % for markup
\newcommand{\comment}[1]{\textcolor{green}{#1}}          % for markup
\newcommand{\ans}[1]{\textcolor{orange}{#1}}




\begin{document}

\title{Non-Hermitian skin effects in open spin systems }
\begin{abstract}

The non-Hermitian skin effect (NHSE), in which eigenstates exhibit localized behaviors at boundaries drastically different from the extended Bloch waves of Hermitian systems, is among the most scrutinized dissipative phenomena. The localization of the eigenstates at a system's edge hints at nonreciprocal transport towards the latter. In several open systems, non-Hermitian Hamiltonians are obtained as an approximation of Liouvillian dynamics by neglecting the quantum jump operators. However this approximation might lose critical information about timescales and conditions required for the nonreciprocal dynamics to manifest. We compare the dynamics governed by non-Hermitian Hamiltonians with those described by the corresponding Liouvillians, identifying the conditions under which both approaches predict the skin effect at the same level. Our analyses uncover the consistency and limitation of non-Hermitian approaches and identify the key ingredients underlying the skin effect in open magnetic systems. Also, our results highlight the connection between the NHSE and the classical magnetization dynamics, suggesting that our predictions can be tested in multilayered magnetic structures with interlayer Dzyaloshinskii-Moriya interactions (DMI).




\end{abstract}
\author{Xin Li}
\email{licqp@bc.edu}
\affiliation{Department of Physics, Boston College, 140 Commonwealth Avenue Chestnut Hill, MA 02467, USA}
\author{Mohamed Al Begaowe}
\affiliation{Department of Physics, Boston College, 140 Commonwealth Avenue Chestnut Hill, MA 02467, USA}
\author{Shu Zhang}
\affiliation{Max Planck Institute for the Physics of Complex Systems, 01187 Dresden, Germany}
\author{Benedetta Flebus}
\email{flebus@bc.edu}
\affiliation{Department of Physics, Boston College, 140 Commonwealth Avenue Chestnut Hill, MA 02467, USA}


\date{\today}% It is always \today, today,
             %  but any date may be explicitly specified
\maketitle




The advent of non-Hermitian notions has drastically extended our understanding of dynamical open systems~\cite{ashida2020non,bender2007making,el2018non,bergholtz2021exceptional}. 
One particular interest is the NHSE, a nonreciprocal accumulation~\footnote{The $\mathbb{Z}_2$ skin effect can be viewed as two copies of non-Hermitian skin effect} of the eigenmodes at open boundaries, which contradicts Bloch band theory and the conventional bulk-boundary correspondence~\cite{bergholtz2021exceptional,zhang2022review}.  
Heretofore, the NHSE has attracted attention in various platforms, including electrical~\cite{liu2021non} and topoelectrical circuits~\cite{hofmann2020reciprocal}, mechanical metamaterials~\cite{ghatak2020observation}, cold-atom~\cite{liang2022dynamic},  photonic~\cite{xiao2020non,weidemann2020topological}, acoustic~\cite{zhang2021acoustic}, and magnonic~\cite{yu2023non} systems, with promising applications in quantum sensing and signal amplification~\cite{Bao2022,Budich2020,Koch2022,el2015optical,Koutserimpas2018,Wang2022}.


Traditionally, the non-Hermitian Hamiltonian forms the basis for analyzing the NHSE. However, in many open quantum systems, the full dynamics is governed by the Liouvillian master equation, rather than solely by the non-Hermitian Hamiltonian. 
In numerous studies, researchers derive a non-Hermitian Hamiltonian by disregarding quantum jumps, thereby effectively transforming the Liouvillian into a non-Hermitian Hamiltonian framework. A question naturally arises: whether this non-Hermitian Hamiltonian is sufficient as an approximation to the dynamics described by the Liouvillian master equation? In particular, can the NHSE derived from the non-Hermitian Hamiltonian survive in the full Liouvillian dynamics? We address these questions by considering a spin chain coupling to a magnetic reservoir. We demonstrate that the non-Hermitian Hamiltonian and the full Liouvillian can exhibit the skin effect at the same level when the spins are identical and the thermal fluctuations in the reservoir are completely suppressed. 


The questions are twofold:  Further clarification is needed on the minimal physical ingredients required for a magnetic system to exhibit the NHSE and whether 
% the latter survives 
it persists in
the full Liouvillian dynamics including quantum jumps~\cite{song2019non,longhi2020unraveling}. 
We show that the magnetic NHSE stems from the interplay between coherent  interactions and  dissipative couplings induced by a common magnetic reservoir~\cite{metelmann2015nonreciprocal,fang2017generalized,clerk2022introduction}. 
The mechanism is then examined in the Liouvillian description of dynamics, where the quantum jumps are included to correctly describe the time evolution of observables. 
The nonreciprocal dynamics stay valid, though we show that the non-Hermitian approach hinders the timescales reflecting causality and locality~\cite{bergholtz2021exceptional,borgnia2020non,okuma2020topological,lee2022anomalously}. 
We also investigate the connection to the generalized Landau–Lifshitz–Gilbert (LLG) equation, a phenomenological classical equation widely used to study dissipative magnetization dynamics, for example, in a magnetic$|$metallic heterostructure, and discuss the absence of the unidirectional limit of nonreciprocity in this system. 





In a noncentrosymmetric magnetic reservoir with broken inversion symmetry, the DMI can give rise to nonreciprocity, which imprints onto the spin system and induces the NHSE in the spin array. This suggests that nonreciprocity and the associated NHSE can be statically engineered and commonly exist in open spin systems.  Our study also sheds light on the correspondence between quantum and classical modeling of dissipative magnetic dynamics.






% Figure environment removed




\textit{Model.} 
The model under consideration is a one-dimensional chain of spins weakly coupled with a common magnetic reservoir, as sketched in Fig.~\ref{fig:master}(a). The spins are originally isolated and interact locally with the reservoir via the Zeemann coupling to the stray field generated by the reservoir. After tracing out the degrees of freedom in the reservoir with the Born-Markov approximation, we obtain a quantum Liouvillian master equation of the density operator $\rho$ of the spin chain,
\begin{eqnarray}
\frac{d}{dt}\rho= \hat{\mathcal{L}}[\rho]=-i \left[ \mathcal{H}, \rho \right] + \mathcal{D}[\rho] \, . \label{eq1}
\end{eqnarray}
Here the Hermitian Hamiltonian $\mathcal{H}$ and the Lindblad dissipator $\mathcal{D}[\rho]$ respectively describe the coherent and dissipative time evolution, i.e.,
\begin{align}
\mathcal{H}=&\sum_{\alpha}\omega_\alpha \hat{s}^z_\alpha+\frac{1}{2}\sum_{\alpha\neq\beta} 
 \mathcal{J}_{\alpha\beta}\hat{s}_{\alpha}^+ 
 \, \hat{s}_{\beta}^- \, \label{hamiltonian} , \\
\mathcal{D}[\rho] =&\sum_{\alpha,\beta}\Gamma_{\alpha \beta}\left(\hat{s}_\beta^+\rho \hat{s}_\alpha^--\frac{1}{2}\{\hat{s}_\alpha^-\hat{s}_\beta^+,\rho\}\right) \nonumber\\
+&\sum_{\alpha,\beta}\tilde{\Gamma}_{\alpha \beta}\left(\hat{s}_\beta^-\rho \hat{s}_\alpha^+-\frac{1}{2}\{\hat{s}_\alpha^+\hat{s}_\beta^-,\rho\}\right) 
,\label{eqlin}
\end{align}
where the dimensionless spin operators are defined as $\hat{s}^{\pm}_\alpha\equiv \hat{s}^{x}_\alpha\pm i\hat{s}^{y}_\alpha$.
The coupling to the reservoir induces several effects: i. a shift in the Zeeman frequency of individual spins, which is absorbed into $\omega_\alpha$, ii. coherent spin interactions $\mathcal{J}_{\alpha\beta}$, 
and iii. the local ($\alpha=\beta$) and cooperative ($\alpha\neq\beta$) spin loss and pump with respective rates $\Gamma_{\alpha\beta}$  and $\tilde{\Gamma}_{\alpha\beta}$. 
These parameter matrices ($\boldsymbol{\mathcal{J}}$, $\boldsymbol{\Gamma}$, and $\boldsymbol{\tilde{\Gamma}}$) are given by the dynamic correlation functions of spin degrees of freedom in the reservoir. 
The rate matrices $\Gamma$ and $\tilde{\Gamma}$ must be positive semidefinite to describe the time evolution of a physical system~\cite{supplementary,breuer2002theory}.  
In the zero-temperature limit, the pump rates vanish as the thermal fluctuations in the reservoir are completely suppressed. 
We have neglected the induced Ising-type interactions and pure dephasing effects by assuming a gapped reservoir, since they originate from reservoir fluctuations at low frequencies~\cite{zou2022bell}.






{\it Non-Hermitian Hamiltonian and NHSE.}
Supposing one can disregard the quantum jump terms [those proportional to $\hat{s}^+_\beta\rho\hat{s}_\alpha^-$ and $\hat{s}^-_\beta\rho\hat{s}_\alpha^+$ in Eq.~(\ref{eqlin})], the Liouvillian master equation equation~(\ref{eq1}) reduces to a form of coherent evolution following a non-Hermitian Hamiltonian $\mathcal{H}_\text{nh}$, 

\begin{align}
 d\rho/dt=-i(\mathcal{H}_\text{nh}\rho-\rho\mathcal{H}_\text{nh}^\dag),  
\end{align}
with
\begin{align}
    \mathcal{H}_\text{nh}
 &=\sum_{\alpha}\Omega^{\mathcal{H}}_{\alpha\alpha}\hat{s}^z_\alpha+\frac{1}{2}\sum_{ \alpha\neq\beta}
 \hat{s}_{\alpha}^+ \mathcal{K}^{\mathcal{H}}_{\alpha\beta}
 \hat{s}_{\beta}^-,
\label{effhamilexp}
\end{align}
where $\boldsymbol{\Omega}^{\mathcal{H}}=\text{diag}\{\cdots, \omega_\alpha+i\Gamma_{\alpha \alpha},\cdots\}$ can be interpreted as a matrix of complex effective magnetic fields locally coupled to each spin, and $\boldsymbol{\mathcal{K}}^{\mathcal{H}}=\boldsymbol{\mathcal{J}}-i\boldsymbol{\Gamma}^T-i\boldsymbol{\tilde{\Gamma}}$ is the spin interaction kernel. 


The non-Hermitian Hamiltonian (\ref{effhamilexp}) takes the form of a Hatano-Nelson model.  
It is now intuitive to see that NHSE in our spin model can arise from nonreciprocal hoppings, as the excitations have a directional preference in transport and tend to accumulate at the corresponding boundary. Interestingly, the nonreciprocity of the non-Hermitian spin interaction kernel $|\mathcal{K}^\mathcal{H}_{\alpha\beta}| \neq |\mathcal{K}^\mathcal{H}_{\beta\alpha}|$ is generic, although the coherent and dissipative interaction matrices $\boldsymbol{\mathcal{J}}$, $\boldsymbol{\Gamma}$, and $\boldsymbol{\tilde{\Gamma}}$ are all hermitian hence reciprocal ($|\mathcal{J}_{\alpha\beta}| = |\mathcal{J}_{\beta\alpha}^*| =  |\mathcal{J}_{\beta\alpha}|$ and so on). 

Recall that the non-Hermitian Hamiltonian formalism does not account for quantum jumps, which assumes that the time evolution is conditioned to a subset of the Hilbert space, for example, through postselection. 
On the other hand, the complete dynamics of an open quantum system may also exhibit state localization to the boundaries, termed Liouvillian skin effect.
To what extend does the NHSE discussed above provide a consistent description for the complete spin-chain dynamics (\ref{eq1})?

To address this question, 
we examine the equations of motion of the two-point operators $\mathcal{C}_{\alpha \beta}= \hat{s}^+_\alpha \hat{s}^-_\beta$ under the full master equation (\ref{eq1}), and obtain, in the matrix form,
\begin{align}
\frac{d}{dt}\boldsymbol{\mathcal{C}}
&=- \boldsymbol{\mathcal{M}}\boldsymbol{\mathcal{C}} - \boldsymbol{\mathcal{C}} \boldsymbol{\mathcal{M}}^\dag +4 \boldsymbol{\mathcal{S}} \,\boldsymbol{\Gamma}\boldsymbol{\mathcal{S}}.\label{mastereq}
\end{align}
Here we have introduced the damping matrix $\boldsymbol{\mathcal{M}} = -i \boldsymbol{\Omega}^{\mathcal{L}} + i \boldsymbol{\mathcal{K}}^{\mathcal{L}}\boldsymbol{\mathcal{S}}$, with the frequency matrix $\boldsymbol{\Omega}^{\mathcal{L}} = \text{diag}\{ \cdots, \omega_\alpha +  i \tilde{\Gamma}_{\alpha\alpha},\cdots \}$, the dynamical kernel $\boldsymbol{\mathcal{K}}^{\mathcal{L}} = \boldsymbol{\mathcal{J}} - i \boldsymbol{\Gamma}^T + i \boldsymbol{\tilde{\Gamma}}$, and the spin density matrix $\boldsymbol{\mathcal{S}} = \text{diag}\{\cdots, \hat{s}^z_\alpha,\cdots\}$. 
% This set of equations serve as the basis for constructing multiple-point operators. 
This set of equations explicitly tracks the dynamics of the spin excitations. The transfer of excitations between lattice sites is governed by the $\boldsymbol{\mathcal{K}}^\mathcal{L}$. Similar to the previous discussion, the interplay between coherent and dissipative interactions yields nonreciprocal transport. The skin effect at the Liouvillian level will  manifests when the transport $|\mathcal{K}^{\mathcal{L}}_{\alpha\beta}|\neq|\mathcal{K}^{\mathcal{L}}_{\beta\alpha}|$.
  



When all the spins are identical, with the same transition frequency, local dissipation, and absorption ($\omega_0\equiv\omega_\alpha$,  $\Gamma_0\equiv\Gamma_{\alpha\alpha}$, and $\tilde{\Gamma}_0\equiv\tilde{\Gamma}_{\alpha\alpha}$), the first terms in the  non-Hermitian Hamiltonian $\mathcal{H}_{nh}$ and the damping matrix $\boldsymbol{\mathcal{M}}$ respectively reduce to identical matrices $\boldsymbol{\mathcal{W}}^{\mathcal{H}=
}(\omega_0+i\Gamma_0)\mathbb{I}$ and $\boldsymbol{\mathcal{W}}^{\mathcal{L}
}=(\omega_0+i\tilde{\Gamma}_0)\mathbb{I}$, where $\mathbb{I}$ is an $N\times N$ identical matrix. In this case, the spectrum of $\mathcal{H}_{nh}$ and  $\boldsymbol{\mathcal{M}}$ will be determined, respectively, by $\boldsymbol{\mathcal{M}}^{\mathcal{H}}=\boldsymbol{\mathcal{J}}-i\left(\boldsymbol{\Gamma}^*+\boldsymbol{\tilde{\Gamma}}\right)$ and  $\boldsymbol{\mathcal{M}}^{\mathcal{L}}=-i\left[\boldsymbol{\mathcal{J}}^*+i\left(\boldsymbol{\Gamma}-\boldsymbol{\tilde{\Gamma}}^*\right)\right]$. At the zero temperature limit, the thermal pumping to the spin system (absorption of the spin system) will vanish, i.e., $\boldsymbol{\tilde{\Gamma}}\rightarrow0$, and $\boldsymbol{\mathcal{M}}^{\mathcal{H}}$ and $\boldsymbol{\mathcal{M}}^{\mathcal{L}}$ become $\boldsymbol{\mathcal{M}}^{\mathcal{H}}=\boldsymbol{\mathcal{J}}-i\boldsymbol{\Gamma}^*=i\left(\boldsymbol{\mathcal{M}}^{\mathcal{L}}\right)^*$. It indicates that, for identical spins at zero temperature, the non-Hermitian Hamiltonian $\mathcal{H}_{nh}$ exhibits the same skin effect as the full Liouvillian master equation. Under this condition, if the non-Hermitian Hamiltonian exhibits the NHSE, then the full Liouvillian master equation will also display the skin effect. 


{\it The non-Hermitian skin effect }
 Let us now discuss the physical properties based on the non-Hermitian Hamiltonian kernel $\boldsymbol{\mathcal{M}}^{\mathcal{H}}=\boldsymbol{\mathcal{J}}-i\boldsymbol{\Gamma}^*$ of identical spins at zero temperature. In general, both $\boldsymbol{\mathcal{J}}$ and $\boldsymbol{\Gamma}$ can be complex. The real and imaginary parts of $\mathcal{J}_{\alpha\beta}$ and  $\Gamma_{\alpha\beta}$ respectively correspond to inversion symmetric and anti-symmetric interactions: $\text{Re}\mathcal{J}_{\alpha\beta}=\text{Re}\mathcal{J}_{\beta\alpha}$, $\text{Re}\Gamma_{\alpha\beta}=\text{Re}\Gamma_{\beta\alpha}$, $\text{Im}\mathcal{J}_{\alpha\beta}=-\text{Im}\mathcal{J}_{\beta\alpha}$ and $\text{Im}\Gamma_{\alpha\beta}=-\text{Im}\Gamma_{\beta\alpha}$  . For an inversion symmetric reservoir, the imaginary parts of $\mathcal{J}_{\alpha\beta}$ and $\Gamma_{\alpha\beta}$ will disappear. Then, we have $|\mathcal{M}^{\mathcal{H}}_{\alpha\beta}|=|\mathcal{M}^{\mathcal{H}}_{\beta\alpha}|=|\text{Re}\mathcal{J}_{\alpha\beta}-i\text{Re}\Gamma_{\alpha\beta}|$, which indicates that this symmetric (reciprocal) non-Hermitian system will not display the NHSE. On the contrary, breaking the inversion symmetry, we usually have $|\mathcal{M}^{\mathcal{H}}_{\alpha\beta}|\neq|\mathcal{M}^{\mathcal{H}}_{\beta\alpha}|$, nonreciprocal interactions, signaling the existence of the NHSE. The inversion symmetry in the reservoir can be broken both intrinsically due to crystalline structure and externally by applying a field or current. 


As a concrete example, we will discuss a simple and practical scenario is that $\text{Im}\Gamma_{\alpha\beta}$ is negligible compared to other quantities the interplay between the antisymmetric $\text{Im}\mathcal{J}_{\alpha\beta}$ and the symmetric (real) $\Gamma_{\alpha\beta}$ can induce the NHSE.  Then we have $\mathcal{M}^{\mathcal{H}}_{\alpha\beta}=\text{Re}\mathcal{J}_{\alpha\beta}+i(\text{Im}\mathcal{J}_{\alpha\beta}-\text{Re}\Gamma_{\alpha\beta})$ and $\mathcal{M}^{\mathcal{H}}_{\beta\alpha}=\text{Re}\mathcal{J}_{\alpha\beta}-i(\text{Im}\mathcal{J}_{\alpha\beta}+\text{Re}\Gamma_{\alpha\beta})$. It is the interplay between the antisymmetric coherent interaction $\text{Im}\mathcal{J}_{\alpha\beta}$ and the symmetric dissipation $\text{Re}\Gamma_{\alpha\beta}$ that makes $|\mathcal{M}^{\mathcal{H}}_{\alpha\beta}|\neq |\mathcal{M}^{\mathcal{H}}_{\beta\alpha}|$ and hence the occurrence of the NHSE. For the long-range interaction, $\boldsymbol{\mathcal{M}}^{\mathcal{H}}$ is a Toeplitz matrix, which can be numerically diagonalized. Howver for a system dominated by the nearest neighbor interaction, the Toeplitz matrix can be solved analytically. Based on these solutions, we can gain physical insight into the system.

{\it A nearest-neighbor interaction dominated model }For notation simplicity, let's denote the nearest neighbor interaction $\text{Re}\mathcal{J}_{\alpha,\alpha+1}=\text{Re}\mathcal{J}_{\alpha+1,\alpha}=J$, $\text{Im}\mathcal{J}_{\alpha,\alpha+1}=-\text{Im}\mathcal{J}_{\alpha+1,\alpha}=D$, and $\text{Re}\Gamma_{\alpha,\alpha+1}=\text{Re}\Gamma_{\alpha+1,\alpha}=\Gamma$ ($\text{Im}\Gamma_{\alpha,\alpha+1}=0$). Then the non-Hermitian Hamiltonian becomes $\mathcal{H}_{nh}=\omega_0\sum_{\alpha=1}^N\hat{s}^z_\alpha+\frac{1}{2}\sum_{ \alpha=1}^N 
 \left(\gamma_R\hat{s}_{\alpha+1}^+ 
 \hat{s}_{\alpha}^-+\gamma_L\hat{s}_{\alpha}^+ 
 \hat{s}_{\alpha+1}^-\right)$, where the coefficients for right and left hopping are defined as $\gamma_{R}=J+i(D-\Gamma)$ and $\gamma_{L}=J-i(D+\Gamma)$. From a topological viewpoint, a finite non-Hermitian system with a point gap will display the NHSE instead of the conventional bulk-boundary correspondence~\cite{okuma2020topological}. To understand this let's transform the Hamiltonian  to momentum space under the periodic boundary condition, $\mathcal{H}_{nh}(k)=\sqrt{N}\omega_0\hat{s}^z_{k=0}+\sum_{k}\epsilon_k\hat{s}^+_k\hat{s}^-_k$ with $\epsilon_k=\left(J\cos ka -D\sin ka\right)-i\Gamma\cos ka$, where $a$ is the lattice constant as shown in Fig.~\ref{fig:master}(a). It is obvious that it has a point gap (0,0) in the complex energy space, as $\text{Re}\epsilon_k$ and  $\text{Im}\epsilon_k$ can't simultaneously be zero, see Fig.~\ref{fig:master}(c).
 
 The point gap indicates that this system will display the NHSE under the open boundary condition due to  the extreme sensitivity  of  the eigenvalues of the non-Hermitian Hamiltonian to the boundary. We can confirm this by directly diagonalzing the Hamiltonian kernel $\boldsymbol{\mathcal{M}}^{\mathcal{H}}$ under the open boundary condition, and the left and right eigenvectors are $|\psi^{R/L}_{n,\alpha}|^=\left\lvert \frac{\gamma_{R}}{\gamma_{L}}\right\rvert^{\pm\alpha}\left(\sin\frac{n\alpha\pi}{N+1}\right)^2$. The factor $\left\lvert\gamma_{R}/\gamma_{L}\right\rvert^{\pm\alpha}$ clearly shows the localization of the eigenstates at the boundaries as shown in Fig.~\ref{fig:master}(b). There is  a positive correlation between  the degree of the boundary localization of the bulk states in Fig.~\ref{fig:master}(b) and the circularity of the winding numbber loop,  in Figs.~\ref{fig:master}(c). An extreme case is the unidirectional limit, $\gamma_L(\gamma_R)=0$, where all the egienstates are localized on the right- (left-) most site, corresponding to a circular loop  (orange curve) in Figs.~\ref{fig:master}(c). 









% Figure environment removed



    \textit{Origin of the NHSE in open magnetic systems} The NHSE is commonly reported in the system with  asymmetric  interactions, or with chiral damping \cite{song2019non,longhi2020unraveling, yi2020non,liu2020helical}. In our model, the coherent interaction, $\propto J \pm iD$, and the nonlocal dissipative, $\propto \Gamma$ (real-valued),  couplings   allow hopping in both directions between two nearest neighbors. It is the balancing between them that yields nonreciprocity (asymmetry), as it can be easily visualized by rewriting $\gamma_{R,L}= \sqrt{J^2+(D\mp\Gamma)^2}e^{i\phi_{R,L}}$ with the phases defined as $\phi_{R,L}=\arctan(\pm D-\Gamma)/J$. 
For  $D\neq 0$, the propagation is non-reciprocal, $|\gamma_L|\neq|\gamma_R|$, while for  $J=0$ and $D=\pm\Gamma$ the hopping becomes purely unidirectional. 



At the excitation level, the DMI in the magnetic reservoir breaks both global inversion and effective time-reversal symmetry (TRS), leading to a finite $D$ and hence a nontrivial phase $\phi$, ($J\pm iD=\sqrt{J^2+D^2}e^{i\phi}$) in the coherent hopping that cannot be gauged out ,~\cite{kim2016realization,koch2010time}.
This mechanism underlying nonreciprocity is intimately connected with the reservoir engineering approaches proposed by Refs.~\cite{metelmann2015nonreciprocal,fang2017generalized} for constructing nonreciprocal photonic devices. In these setups, synthetic gauge fields are introduced by using nonlinearities and external drives. 
In our model and in Ref.~\cite{deng2022non}, the antisymmetric coherent interaction is induced by DMI in the magnetic reservoir, suggesting that nonreciprocity can be statically engineered. Furthermore, our finding has the important implication that the NHSE can exist very generally in open magnetic systems:  DMI commonly exists as part of the exchange interactions either in noncentrosymmetric magnetic systems or at magnetic surfaces. Therefore, our discussion here generally applies to an inversion-breaking dissipative magnetic system.





\textit{Liouvillian skin effect}  In this section we will turn to discuss the  skin effect based on the full Liouvillian master equation. The spectrum is determined by the damping matrix $\boldsymbol{\mathcal{M}}=\boldsymbol{\mathcal{W}}^{\mathcal{L}} + \boldsymbol{\mathcal{M}}^{\mathcal{L}}\boldsymbol{\mathcal{M}}^z$ in Eq.~(\ref{mastereq}). In general, one can directly diagonalize  $\boldsymbol{\mathcal{M}}$ to get its eigenvalues. Without loss of generality, let us consider an ensemble of identical spins where $\boldsymbol{\mathcal{W}}^{\mathcal{L}}$ reduces to an identical matrix $\boldsymbol{\mathcal{W}}^{\mathcal{L}}=(\omega_0+i\tilde{\Gamma}_0)\mathbb{I}$, which commutes with $\boldsymbol{\mathcal{M}}^{\mathcal{L}}\boldsymbol{\mathcal{M}}^z$. In this case, the spectrum will be correspondingly dependent on $\boldsymbol{\mathcal{M}}^{\mathcal{L}}\boldsymbol{\mathcal{M}}^z$, i.e., on the matrix $\boldsymbol{\mathcal{M}}^{\mathcal{L}}=-i\left[\boldsymbol{\mathcal{J}}^*+i\left(\boldsymbol{\Gamma}-\boldsymbol{\tilde{\Gamma}}^*\right)\right]$. Similar to the NHSE analysis the relation $|\mathcal{M}^{\mathcal{L}}_{\alpha\beta}|\neq|\mathcal{M}^{\mathcal{L}}_{\beta\alpha}|$ also indicates the skin effect at the Liouvillian level and we will not repeat the derivations.

A deserving noting point is that the non-Hermitian Hamiltonian only predicts the existence of the NHSE, but it does not specify whether these localized bulk states will be finally occupied and hence one can observe the localization on the boundary. In order to investigate the excitation dynamics through the spin chain one needs to resort to the evolution equation of the two-point correlation Eq.~(\ref{mastereq}).  
For simplicity, we turn off the temperature, i.e., $\tilde{\Gamma}\rightarrow0$ and only consider the nearest neighbor interaction. 
We simulate the time evolution of the excitation density distribution by solving Eq.~(\ref{mastereq}) numerically with an initial state of one spin excitation at the center of a $N=9$ spin array and $ \hat{s}^+_{\alpha}\hat{s}_{\beta}=0$ for any $\alpha, \beta \neq 5$. 
The evolution of the excitation at the $\alpha$th site, $n_\alpha(t)=\langle\,s^+_\alpha\,s^-_\alpha\rangle$, is shown in Fig.~\ref{fig:dynamics}(a)-(c) for
unidirectional, nonreciprocal, and reciprocal propagation, respectively, is largely consistent with the understanding from the non-Hermitian description. 
In particular, the unidirectional limit remains, when $\gamma_R (\gamma_L)= 0$, two-point correlations can only build up to the left (right)  during the entire time evolution.

However, an important caveat is revealed in the transient dynamics: a large local dissipation can fully suppress the  state accumulation at the system boundary, as depicted by Fig.~\ref{fig:dynamics}(d). The time and length scales on which the NHSE is observable cannot be captured by the topological properties of the non-Hermitian Hamiltonian, but rather strongly depends on the local dissipation, which can be adjusted by pumping, balancing $\Gamma$ vs. $\tilde{\Gamma}$.
These results highlight the limitation of an analysis based on non-Hermitian eigenenergies and eigenmodes.~\cite{lee2022anomalously}.




\textit{Connection to the classical magnetization dynamics.} An analogy can be drawn between our model~(\ref{eq1}) and the magnetic multilayer sketched in Fig.~\ref{fig:LLG}(a), where each magnetic layer interacts with its nearest neighbors via a metallic spacer. In addition to the intrinsic Gilbert damping $\alpha_{l}$ of the magnetic dynamics, the metallic spacer mediates a nonlocal spin pumping $\alpha_{nl}$ between the long-wavelength magnetization dynamics of adjacent layers~\cite{arxivSklenar,PhysRevB.66.224403,RevModPhys.77.1375,PhysRevLett.88.117601}. 
 The electric Fermi surface in metallic spacer can also mediates an effective coherent RKKY coupling $J$, with a DMI component $D$ due to interfacial inversion-symmetry  breaking~\cite{huang2022growth,di2015direct,ma2020longitudinal,han2019long,fernandez2019symmetry,avci2021chiral}. 

A minimal model for the magnetic Hamiltonian of the  multilayer can be written as~\cite{yuan2023unidirectional}
$\tilde{\mathcal{H}}= -\sum_{\langle \alpha\beta\rangle}
   \left[ J\mathbf{m}_\alpha\cdot\mathbf{m}_\beta 
   +D \hat{\mathbf{z}}\cdot(\mathbf{m}_\alpha\times\mathbf{m}_\beta) \right] 
   -\sum_{\alpha} \mu_0M_s\mathbf{H}\cdot \mathbf{m}_\alpha$
where $\mathbf{H}=H\hat{\mathbf{z}}$ is an externally applied magnetic field oriented along the $\hat{\mathbf{z}}$ direction, $M_s$ the saturation magnetization and $\mu_{0}$ the vacuum permeability. 
% Here, $\langle \alpha\beta\rangle$ denotes the nearest neighbors.
The macrospin dynamics of the magnetization  $\mathbf{m}_{\alpha}$ of the $\alpha$th layer is determined by the a modified LLG equation
 \begin{align}
 \frac{\partial \mathbf{m}_\alpha}{\partial t} &= -\frac{\gamma}{M_s}\mathbf{m}_{\alpha}\times\mathbf{H}_{\text{eff},\alpha}+\alpha_l \mathbf{m}_\alpha\times\frac{\partial\mathbf{m}_\alpha}{\partial t}\nonumber \\
 &+ \alpha_{nl} \mathbf{m}_\alpha\times \left(\frac{\partial\mathbf{m}_{\alpha-1}}{\partial t}+\frac{\partial\mathbf{m}_{\alpha+1}}{\partial t} \right)\,,
 \label{eq7}
 \end{align}
with $\mathbf{H}_{\text{eff},\alpha}=-\partial \tilde{\mathcal{H}}/\partial\mathbf{m}_\alpha $ and $\gamma$ the gyromagnetic ratio.
We linearize the LLG equation and derive its  spectrum under periodic boundary condition. As shown in Fig.~\ref{fig:LLG}(b), it has a point gap, singling the existence of the NHSE~\cite{supplementary}.


% Figure environment removed


For a bilayer, a balance between DMI and nonlocal damping, i.e., $D=\pm \alpha_{nl}\mu_0M_sH$~\cite{supplementary} can yield unidirectional transport~\cite{yuan2023unidirectional}.    
However, a key difference between the non-Hermitian model~(\ref{effhamilexp}) and Eq.~(\ref{eq7}) arises for more than two layers: the latter cannot exhibit exact unidirectionality~\cite{supplementary}.  
The nonlocal damping in Eq.~(\ref{eq7}) is determined for a given layer by the instantaneous dynamic state of the adjacent layers. This
effectively establishes a next nearest neighbor and even further interactions~\cite{supplementary},
which stay active as the nearest-neighbor hopping in one direction can be turned off. 
Therefore, the dynamics can be nonreciprocal but not unidirectional, as reflected by the nonvanishing ellipticity of the eigenenergy loop.




\textit{Discussion.} In this work, through  a spin chain coupling to a magnetic reservoir, we demonstrate the condition under which the non-Hermitian Hamiltonian  displays the skin effect at the same level with the full master equation. We reveal that the inversion symmetry broken in the magnetic reservoir causing asymmetric interactions resulting into the skin effect. In particular, the anti-symmetric coherent interaction can be attributed to the DMI,   the interplay between which and the nonlocal dissipation gives rise to the skin effect. Our approach highlights the limitations of the topological explanation in predicting whether a macroscopic accumulation of bulk states at a boundary can physically take place~\cite{borgnia2020non,okuma2020topological,zhang2020correspondence}.





The mechanism that we uncover is intimately connected with approaches to quantum nonreciprocity at light-matter interfaces~\cite{metelmann2015nonreciprocal,fang2017generalized},
for which engineering a nontrivial phase in the nonlocal quantum many-body dynamics requires a combination of external drives. Our  results suggests that quantum nonreciprocity in spin ensembles can be engineered at equilibrium via their mutual interactions with a magnetic bath lacking, e.g., inversion or mirror symmetries. 

We also establish a link between the NHSE and the classical magnetization dynamics, showing that non-reciprocity (\textit{albeit} not unidirectionality) can be realized by virtue of the same ingredients underlying the NHSE. Our findings, together with the growing interest in non-Hermitian engineering of magnetic systems, call for the development of a more precise connection between the dissipative coupling in the Lindbladian and the nonlocal damping in the classical equations of dissipative magnetic dynamics needs to be established. Finally, non-Markovian effects may also arise in dynamical magnetic systems~\cite{kim2018}, which go beyond the Lindblad formalism, making this an important topic for further exploration.   






\textit{Acknowledgments.}
The authors thank R. A. Duine and J. Marino for helpful discussions. 
This work was supported by the National Science Foundation under Grant No. NSF DMR-2144086.

\bibliographystyle{apsrev4-2}

\bibliography{library}

\end{document}

