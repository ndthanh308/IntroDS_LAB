\documentclass[11pt,reqno, a4paper]{amsart}

\usepackage{amsmath,amsthm}
\usepackage{amssymb}
\usepackage{amsfonts}
\usepackage{amscd}
\usepackage{enumerate}
\usepackage{color}
\usepackage[all]{xy}
%
% \usepackage[notcite,notref]{showkeys}
%
\setlength{\topmargin}{-15mm}
\setlength{\textheight}{240mm}
\setlength{\textwidth}{150mm}
\setlength{\oddsidemargin}{0.46cm}
\setlength{\evensidemargin}{0.46cm}

\font \manual=manfnt at 7pt
\font \sc=cmcsc10 at 9truept

\newtheorem{thm}{Theorem}[section]
\newtheorem{theorem}[thm]{Theorem}
\newtheorem{prop}[thm]{Proposition}
\newtheorem{claim}[thm]{Claim}
\newtheorem{corollary}[thm]{Corollary}
\newtheorem{lemma}[thm]{Lemma}
\newtheorem{sublemma}[thm]{Sublemma}
\newtheorem{proposition}[thm]{Proposition}
\newtheorem{definition}[thm]{Definition}
%

\theoremstyle{definition}
% \newtheorem{definition}[thm]{Definition}
\newtheorem{example}[thm]{Example}
\newtheorem{examples}[thm]{Examples}
\newtheorem{remark}[thm]{Remark}
\newtheorem{remarks}[thm]{Remarks}
\newtheorem{observation}[thm]{Observation}
\newtheorem{notation}[thm]{Notation}
\newtheorem{free text}[thm]{}

\newtheorem{rmvarthm}[thm]{\hspace{-1.7mm}}


\numberwithin{equation}{section}
\renewcommand{\theequation}{\thesection.\arabic{equation}}

\renewcommand{\theenumi}{\roman{enumi}}
\renewcommand{\theenumii}{\alph{enumii}}
\renewcommand{\labelenumi}{{\rm (}{\it \theenumi}\,{\rm )}}


\newcommand{\dbend} {$ {}^{\hbox{\manual \char127}} $}  % "dangerous bend" sign

\newcommand{\N} {\mathbb{N}}
\newcommand{\Z} {\mathbb{Z}}
\newcommand{\R} {\mathbb{R}}
\newcommand{\C} {\mathbb{C}}
\newcommand{\KK} {\mathbb{K}}
\newcommand{\bk} {\Bbbk}
\newcommand{\J} {{\mathfrak{J}}}
\newcommand{\F}{{\mathcal F}}
\newcommand{\e} {\epsilon}

%%%%% Niels' macros

\newcommand{\rmref}[1]{{\rm (}\ref{#1}{\rm )}}

\newcommand{\Aut}{\operatorname{Aut}}
\newcommand{\End}{\operatorname{End}}
\newcommand{\Hom}{\operatorname{Hom}}
\newcommand{\Der}{\operatorname{Der}}
\newcommand{\Rep}{\operatorname{Rep}}


\newcommand{\Tor}{{\rm Tor}}
\newcommand{\Ext}{{\rm Ext}}

\newcommand{\id}{{\rm id}}
\newcommand{ \strad} {\rm strad}



%
%  greek letters
%

\newcommand{\ga}{\alpha} 
\newcommand{\gb}{\beta}  
%\newcommand{\gg}{\gamma} is already defined
\newcommand{\gG}{\Gamma}
\newcommand{\gd}{\delta} 
\newcommand{\gD}{\Delta} 
%\newcommand{\ge{\epsilon} is already defined
\newcommand{\gve}{\varepsilon} 
\newcommand{\gf}{\phi}  
\newcommand{\gvf}{\varphi}  
\newcommand{\gF}{\Phi}
\newcommand{\gl}{\lambda} 
\newcommand{\gL}{\Lambda}
\newcommand{\go}{\omega} 
\newcommand{\gO}{\Omega}
\newcommand{\gr}{\rho} 
\newcommand{\gvr}{\varrho} 
\newcommand{\gR}{\Rho}
\newcommand{\gs}{\sigma} 
\newcommand{\gS}{\Sigma}
\newcommand{\gt}{\theta} 
\newcommand{\gvt}{\vartheta} 
\newcommand{\h}{{\mathfrak h}}
\newcommand{\g}{{\mathfrak g}}
\newcommand{\Co}{\rm Coind}
\newcommand{\Ind}{\rm Ind}

\newcommand{{\Uop}}{{U^{\rm op}}}
\newcommand{{\Ae}}{{A^{\rm e}}}
\newcommand{{\Aop}}{{A^{\rm op}}}
\newcommand{{\Bop}}{{B^{\rm op}}}

\newcommand{\ahha}{{\scriptscriptstyle{A}}}
\newcommand{\Aopp}{{\scriptscriptstyle{\Aop}}}
\newcommand{\Aee}{{\scriptscriptstyle{\Ae}}}

\newcommand{{\op}}{{{\rm op}}}
\newcommand{{\coop}}{{{\rm coop}}}
\newcommand{{\sop}}{{*^{\rm op}}}
\newcommand{{\Ber}}{{{\rm Ber}}}

\newcommand{\lact}{{\,\raise1pt\hbox{$\scriptscriptstyle{\rhd}$}\,}}                  %
\newcommand{\ract}{{\,\raise1pt\hbox{$\scriptscriptstyle{\lhd}$}\,}}                  %  A-actions
\newcommand{\blact}{{\,\raise1pt\hbox{$\scriptscriptstyle{\blacktriangleright}$}\,}}  %
\newcommand{\bract}{{\,\raise1pt\hbox{$\scriptscriptstyle{\blacktriangleleft}$}\,}}   %

\newcommand{\due}[3]{{}_{{#2 }} {#1}_{{ #3}}\,}    % double index

\sloppy

\begin{document}

\title{Duality properties for induced and coinduced representations in positive characteristic}
\author{Sophie Chemla}

\maketitle



\begin{abstract}
Let $k$ be a field of positive characteristic $p>2$. We prove a duality property concerning the kernel of coinduced representations of Lie superalgebras. This property was already proved by M. Duflo (\cite{Du})
for Lie algebras in any characteristic under more restrictive finiteness conditions.  It was then generalized to Lie superalgebras in characteristic 0 in  previous works (\cite{C0}, \cite{C4}). 
In a second part of the article, we study the links between  coinduced representations and induced representations in the case of restricted Lie superalgebras. 
\end{abstract}
\section{Introduction}

Assume that  $\g=\g_{\overline{0}}\oplus \g_{\overline{1}}$ is a Lie superalgebra over a field $k$ 
of characteristic $p>2$ and $\h \subset \g$ is a Lie subsuperalgebra of $\g$. From a 
representation $(\pi, V)$ of $\h$, one can construct a representation of $\g$ in two ways : 
\begin{itemize}
\item Induction : $\Ind_\h^\g (V)=U(\g)\otimes_{U(\h)} V$ with left  $U(\g)$-module structure given by left multiplication;
\item  Coinduction : $\Co_\h^\g (\pi )=Hom_{U(\h)}(U(\g),V)$ with left $U(\g)$-module structure given by the transpose of right multiplication.
\end{itemize}

It is easy to see that the contragredient representation of $\Ind_\h^\g(V)$ is isomorphic to the coinduced representation from the contragredient representation $\pi^*$ of $\pi$ (\cite{Di}).

M. Duflo (\cite{Du}) proved that, in any characteristic, for a finite dimensional Lie algebra $\g$, the kernel $I_\pi\subset U(\g)$ of 
$\Co_\h^\g (\pi)$ satisfies a the duality property 
$$\check{I}_\pi=I_{\pi^* \otimes k_{trad_{\g/\h}}}$$
where $\check{(-)}$ is the antipode of $U(\g)$ and $trad_{\g/\h}$ is the character $tr\circ ad_{\g/\h}$ of $\h$. 
In characteristic $0$, this duality property was extended to a Lie superalgebra $\g$ such that only $\g/\h$ is finite dimensional. In this article, we treat the case of a Lie superalgebra with 
$\g/\h$ finite dimensional when $k$ is of  positive characteristic case $p>2$. We make use of the 
following ascending filtration of $U(\g)$
$$\begin{array}{l}
{\mathcal F}_{-1}U(\g)=k\\
{\mathcal F}_0U(\g)=U(\h) \\
{\mathcal F}_rU(\g)=U(\h) Vect  \left (X^{p^j}, \quad X \in \g, \; j\in [0,r-1] \right )\; {\rm if}\; r>0.
\end{array}$$

From now on, we assume that $\g/\h$ is finite dimensional and $k$ of positive characteristic $p>2$.  
In the case where $\g$ is a restricted superalgebra with restricted enveloping superalgebra 
$U^\prime (\g)$ and $\h$ is a  restricted subsuperalgebra of $\g$, we get stronger results. 
For a restricted 
representation $(\pi, V)$ of $\h$, we introduce the restricted induced representation from $\pi$ and the restricted 
coinduced representation from $\pi$ : 
\begin{itemize}
\item Restricted induction : $\Ind_{U^\prime (\h)}^{U^\prime (\g)} (V)=U^\prime (\g)\otimes_{U^\prime (\h)} V$ 
with left  $U^\prime(\g)$-module structure given by left multiplication;
\item  Restricted coinduction : $\Co_{U^\prime(\h)}^{U^\prime(\g)} (\pi )=Hom_{U^\prime(\h)}(U^\prime(\g),V)$ with left 
$U^\prime (\g)$-module structure given by the transpose of right multiplication.
\end{itemize}


Generalizing a result of Borho-Brylinski (\cite{B-B}), it was proved in \cite{C4} that, in characteristic 0, the induced representation of a Lie superalgebra could be realized in terms of Grothendieck local cohomology. We prove an analog result for restricted Lie superalgebras ~: \\\\
{\bf Theorem} \ref{Ind and Coind} {\it Let $k$ be a field of characteristic $p>0$. Assume that the $k$-superspace 
 ${\g}/{\h}$ is finite dimensional. Set $m=dim \left (\g/\h\right )_{\overline{1}}$. Let $(\pi, V)$ be a representation of $U^\prime (\h)$. 
The restricted induced representation $\Ind_{U^\prime (\h)}^{U^\prime (\g)}(\pi \otimes \Pi^m k_{strad_{\g/\h}})$ is isomorphic to $\Co_{ U^\prime (\h)}^{U^\prime (\g)}(\pi)$.}\\

In \cite{C0}, it was noticed that (for any characteristic) Berezin integral provides  a $\g$-invariant duality between $\Co_{ \h}^\g(\pi^*)$ and  
$\Co_{ \h }^\g (\pi \otimes \Pi^m k_{-strad_{\g /\h}}) $ in case the Lie superalgebra $\g/\h$ is totally odd. We extend this result to any restricted Lie superalgebra $\g$ in the positive characteristic case   : \\ 

{\bf Theorem} \ref{definition of  Psi} {\it Let $k$ be a field of characteristic $p>2$. Assume that the $k$-superspace 
 ${\g}/{\h}$ is finite dimensional with odd dimension $m$. There exists   a  $\g$-invariant duality $\Phi$ between 
 $\Co_{U^\prime( \h) }^{U^\prime (\g)} (\pi) $ and 
 $\Co_{U^\prime( \h)}^{U^\prime(\g)}(\pi^* \otimes  \Pi^m k_{-strad_{\g /\h}})$.} \\
 
 Theorem \ref{Ind and Coind} and Theorem \ref{definition of  Psi} are linked by the $U^\prime (\g)$-module isomorphism  
 
$$\begin{array}{rcl}
\Theta : Coind_{U^\prime (\h)}^{U^\prime (\g)} (\pi^*)&\to &
 {\mathcal I}nd_{U^\prime (\h)}^{U^\prime (\g)} (\pi)^*\\
\lambda & \mapsto & \left [ u \otimes v\mapsto <\lambda(\check{u}),v>\right ].\\
\end{array}$$
 





 

\section{Notation and preliminaries}

In this article, $k$ will be a commutative field of characteristic $p$. For most definitions about supermathematics, we refer the reader to \cite{Leites}. 
We will denote by $\overline{0}$ and $\overline{1}$ the elements of $\Z/2\Z$. We will call superspace a $k$-vector space graded over $\Z/2\Z$, 
 $V=V_{\overline{0}}\oplus V_{\overline{1}}$. If $v\in V_{i}$,  its degree will be denoted $\mid v \mid =i$. 
 As usual, formulas are meant for homogeneous elements and extended to any element by linearity. Let $V$ and $W$ be two superspaces. If $f$ is a morphism of degree $i$ from $V$ to $W$and if $v$ is in $V_j$, the element $f(v)$ will be also denoted $<f,v>$ and 
 we will use the notation
 $$<v,f>=(-1)^{ij}f(v)$$
 when it avoids the apparition of signs. If $V$ is a superspace, one defines the superspace $\Pi V$ which,  as a vector space, is equal to $V$ but the grading of which is 
 $(\Pi V)_{\overline{0}}=V_{\overline{1}}$ and $(\Pi V)_{\overline{1}}=V_{\overline{0}}$ . Let us introduce the map $\pi :V \to \Pi V$ which, as a morphism of vector spaces,  equals identity. It is of degree $\overline{1}$.  The functor $\Pi$ is called functor "change of parity". The symmetric superalgebra of $V$ will be denoted $S(V)$. 
 
 Let $A$ be an associative supercommutative superalgebra with unity and $M$ be an $A$-module. A basis of $M$ is a family 
 $(m_i)_{i \in I \amalg J} \in M_{\overline{0}}^I \times M_{\overline{1}}^J$ such that each element of $M$ can be expressed in a unique way 
 as a linear combination of $(m_i)_{i \in I \amalg J}$. If $I$ and $J$ are finite, their cardinality is independent of the basis of the $A$-module $M$. Then, the dimension of $M$ is 
 the dual number $(\mid I \mid +{\mathfrak e} \mid J \mid ) \in \Z[{\mathfrak e}]/({\mathfrak e}^2-1)$. If $(e_1, \dots , e_n )$ is a basis of the $A$-module $M$, then the family $(e^1, \dots ,e^n)$ where 
 $<e_i, e^j>=\delta_{i,j}$ is a basis of $Hom_A (M,A)$ called the dual basis of $(e_1, \dots , e_n)$. Moreover, if $M$ is an $A$-module, then $\Pi M$ has a natural 
 $A$-module structure defined by :
 $$\forall m \in M, \; \forall a \in A, \quad a\cdot \pi m=(-1)^{\mid a \mid} \pi (a\cdot m).$$
 The following proposition is proved in \cite{Manin} p.172.
 \begin{proposition}\label{Berezinian}
 Assume that $M$ is a free $A$-module with finite dimension  $dim M=(n,m)$, denote by 
 $d$  left multiplication by 
 ${\displaystyle \sum_{i=1}^{n+m}(-1)^{\mid e_i \mid +1}\pi e_i \otimes e^i}$ in the superalgebra 
 $S_A(\Pi M\oplus M^*)$. The endomorphism $d$ does not depend on the choice of a basis. 
 The complex 
 $$J(M)^\bullet=\left (S_A(\Pi M\oplus M^*)=\oplus_{n\in \N}S^n _A(\Pi M)\otimes_A S_A(M^*), d\right )$$
 has no cohomology except in degree $n$. The $A$-module $H^{n}(J(M))$ is free of dimension (1,0) or (0,1). More precisely the element 
 $\pi e_1 \dots \pi e_{n} \otimes e^{n+1} \dots e^{n+m}$ is a cycle the class of which is a basis of $H^n(J(M))$. 
  \end{proposition}
  The module $H^n(J(M))$ is called the Berezinian module of $M$ and is denoted $Ber (M)$. The Berezinian module 
  generalizes the maximal wedge, which does not exist if $M_{\overline{1}}\neq \{0\}$.\\
  
  Denote by ${\mathfrak g}l(M)$ the Lie superalgebra of endomorphims  of $M$. It acts on $S_A(\Pi M\oplus M^*)$ and its action commute with the differential $d$. Thus, it acts on 
  $Ber(M)$ by a character called "supertrace" and denoted $str$.\\
  
  If $\g$ is a $k$-Lie superalgebra, we will write $U(\g)$ for  its enveloping superalgebra and $\Delta$ for the coproduct in $U(\g)$. If $V$ is a left $U(\g)$-module, then $V^*$ will be the contragredient module. Let us now describe the primitive elements of the Hopf superalgebra $U(\g)$. The following result is well known but, as  we did not find any reference, we give a proof of it. 
  
  \begin{proposition}\label{primitive elements}
  Let $\g$ be a Lie  superalgebra. 
 Let $(e_1, \dots , e_n)$ be a basis of $\g_{\bar{0}}$ and $( \epsilon_1, \dots  , \epsilon_m)$ be a basis of $\g_{\bar{1}}$.  The vector space of primitive elements of  $U(\g)$ is generated by  $\{e_i^{p^j}, \epsilon_s,  \quad (i,j,s) \in [1,n] \times \N \times \{1,m\} \}$.
 \end{proposition}
 
 \begin{notation} Let us introduce the following notation $\N^{n,m}:= \N^n \times \{0,1\}^m$.  If $(\underline{a}, \underline{\alpha} )\in \N^n \times \{0,1\}^m$, we set 
 $$e^{\underline{a}}\epsilon^{\underline{\alpha}}=e_1^{a_1}\dots e_n^{a_n}\epsilon_1^{\alpha_1}\dots \epsilon_n^{\alpha_n} .$$
 \end{notation}
 
 
 {\it Proof of proposition \ref{primitive elements}}
 
 
 $\left ( e^{\underline{a}}\epsilon^{\underline{\alpha}}\right )_{(\underline{a}, \underline{\alpha})\in \N^{n,m}}$ is a basis of $ U({\mathfrak g})$ and 
 $\left ( e^{\underline{a^\prime}}\epsilon^{\underline{\alpha}^\prime}
 \otimes e^{\underline{a^{\prime \prime}}}\epsilon^{\underline{\alpha^{\prime \prime}}}\right ) 
 _{(\underline{a^\prime},\underline{\alpha^\prime}), ( \underline{a^{\prime \prime}}, \underline{\alpha^{\prime \prime}})\in \N^{n,m}}$ is a basis of
  $U({\mathfrak g})\otimes U({\mathfrak g})$. 
  
  Let $x=\sum x_{\underline{a}, \underline{\alpha}} e^{\underline{a}}\epsilon^{\underline{\alpha}}$ be a primitive element of $U(\g)$ 
  One has 
  $$\Delta(x)-x\otimes 1-1 \otimes x = \sum_{
  \begin{array}{l}
 (\underline{a^\prime}, \underline{\alpha^\prime}) , (\underline{a^{\prime \prime}}, \underline{\alpha^{\prime \prime}})\neq  (\underline{0}, \underline{0}),\\
  (\underline{a}^\prime, \underline{\alpha^{\prime}}) +(\underline{a}^{\prime \prime}, \underline{\alpha^{\prime \prime}})=(\underline{a}, \underline{\alpha}),\\
  
  \end{array}}
  x_{\underline{a}, \underline{\alpha}}
  e^{\underline{a^\prime}}\epsilon^{\underline{\alpha^\prime}}
  \otimes e^{\underline{a^{\prime \prime}}}\epsilon^{\underline{\alpha^{\prime\prime}}}=0$$
  If the term $e^{\underline{a}}\epsilon^{\underline{\alpha}}$ involves more than one $e_i$ or $\epsilon_i$,  then  
  $\Delta(e^{\underline{a}}\epsilon^{\alpha})-e^{\underline{a}}\epsilon^{\alpha}\otimes 1-1 \otimes e^{\underline{a}}\epsilon^{\alpha}\neq 0$.  
  
  
 Thus 
  $x$ can be written $x=\sum_{i\in [1,n]}x_{a_i} e_i^{a_i}+ \sum_{k\in [1,m]}x_{k} \epsilon_k$. 
  Let us now show that all the $a_i$'s are a power of $p$. Assume, it is not the case for $a_i$. 
  Let $p^{t_i} \in \N$ such that $p^{t_i}<a_i<p^{t_i+1}$. Set $a_i=p^t+b_i$ with $b_i\in [1,p-1]$.
  One has 
  $$\begin{array}{rcl}
  \Delta(e^{a_i})&= &\Delta(e^{p^t})\Delta(e^{b_i})\\
  &=& (e^{p^t}\otimes1+1 \otimes e^{p^t})(e^{b_i}\otimes 1+b_ie^{b_i-1}\otimes e_i+\cdots )
  \end{array}$$
  and the term $b_ie^{b_i-1}\otimes e_i$ is non zero so that 
  $\Delta(x)-x\otimes 1-1 \otimes x =0$ is not zero. $\Box$
  
  
 % In $\Delta(x)-x\otimes 1-1 \otimes x =0$, the term 
  %$\begin{pmatrix}a_i \cr p^{t_i} \end{pmatrix} e_i^{p^{t_i}}\otimes e_i^{a_i-p^{t_i}}$ occurs. But 
 % $\begin{pmatrix} a_i\cr p^{t_i} \end{pmatrix}$ is not a multiple of $p$. Impossible. 
 % $\Box$.\\

%Introduce the  filtration $\left ( \widetilde {\mathcal F}_rU(\g)\right )_{r-1 \in \N}$ on $U(\g)$ given by :
%$$\begin{array}{l}
%\widetilde{\mathcal F}_{-1}U(\g)=k, \\
%{\rm If\; r \in \N, \;}\widetilde{\mathcal F}_rU(\g)=
%Vect \{ \prod_{i=1}^n \prod _{j=0}^r e_i^{p^j a_{i,j}} \prod_{s=1}^m \epsilon_s^{\alpha_s}, 
%a_{i,j}\in [0, p-1], \quad \alpha_s \in \{0,1\}\}.
%\end{array}$$

%  \begin{proposition}\label{stability under check}
%  Denote by $\check{}$ the antipode of $U(\g)$. The vector space $\widetilde{\mathcal F}_rU(\g)$ is stable under $\; \check{}$.
 % \end{proposition}
  
%  {\it Proof of the proposition \ref{stability under check} :}
%  Let $X$ and $Y$ in $\g$. Using proposition \ref{primitive elements}, we know that $[X^{p^r}, Y^{p^s}]$ belongs to 
 % $\widetilde{\mathcal F}_{max(r,s)}U(\g) .  \Box$\\
 
 
 




\section{Generalities on Lie-Rinehart superalgebras}

\begin{definition}
Let $A$ be a  $k$-superalgebra. A $k-A$-Lie Rinehart (\cite{Rinehart})
superalgebra (with anchor $\rho$) is a triple 
$(L, [\;,\; ], \rho )$ such that 
\begin{enumerate}
\item $(L, [\; , \; ])$ is a $k$-Lie superalgebra;
\item $L$ is an $A$-module;
\item The anchor $\rho : L \to \Der (A)$ is a $k$-Lie  superalgebra morphism and an $A$-module morphism such that : For all $(D,\Delta ) \in L$ and all $a\in A$,
$$[D,a\Delta ]=\rho(D)(a)\Delta +(-1)^{\mid a \mid \mid D \mid} a [D, \Delta ].$$
\end{enumerate}
\end{definition}
\begin{examples}
\begin{enumerate}
\item If $A=k$, a Lie Rinehart superalgebra is a $k$-Lie superalgebra.
\item The $A$-module $\Der (A)$ is a $k$-$A$-Lie-Rinehart superalgebra with anchor equal to $id$.
\item Assume that $\g$ is a $k$-Lie superalgebra given with a Lie superalgebra morphism $\sigma :\g \to \Der (A)$. Then the $A$-module $A \otimes \g$ is  endowed with a unique $k-A$ Lie Rinehart superalgebra such that 
\begin{itemize}
\item The anchor $\rho : A \otimes \g$ is defined by $\rho (a D)=a\sigma (D)$.
\item The Lie bracket on $A \otimes \g$ extends that of $\g$.
\end{itemize}
The Lie Rinehart superalgebra constructed that way is called the crossed product of $A$ with $\g$ and is denoted 
$(A\sharp \g , \sigma )$ or just $A\sharp \g$ when there is no ambiguity.
\item Poisson superalgebras also gives rise to Lie Rinehart superalgebras but we won't use them in this article 
(\cite{Hu},\cite{C2})
\end{enumerate}
\end{examples}

Rinehart (\cite{Rinehart}) has associated an enveloping algebra to a Lie Rinehart algebra. This notion generalizes the enveloping algebra of a Lie algebra.

\begin{definition}  \label{def_U_A(L)}
 Let  $ (L,\rho) $  be a $k$-$A$-Lie-Rinehart superalgebra.  The  {\sl universal enveloping superalgebra}  of  $ L $  is the  $ k $--algebra
  $$  U_A(L)  \; := \;  T_k^+(A \oplus L) \Big/ I  $$
%
where  $ T_k(A \oplus L) $  is the tensor  $ k $-superalgebra  over  $ \, A \oplus L \, $  and  $ I $  is the two sided ideal in  $ T_k^+(A \oplus L) $  generated by the elements
  $$  a \otimes b - (-1)^{\mid a \mid \mid b \mid}a \, b \;\; ,  \qquad  a \otimes \xi - a \, \xi \;\; ,
 \qquad  \xi \otimes \eta -(-1)^{\mid \eta \mid \mid \xi \mid}\eta \otimes \xi - [\xi,\eta] \;\; ,
 \qquad  \xi \otimes a -(-1)^{\mid a \mid \mid \xi \mid}a \otimes \xi -\rho (\xi)(a)  $$
%
for all  $ \, a, b \in A \, $,  $ \, \xi , \eta \in L \, $.
\end{definition}

\begin{remark}
The anchor endows $A$ with a left $U_A(L)$-module structure. 
\end{remark}

In this article, we will mostly be in the case of the  crossed product $A \sharp \g$ given by a coinduced superalgebra.\\

 Let $\h$ be a subLie super algebra of $\g$ and let $(\pi, V)$ be a representation of ${\mathfrak h}$ . We set 
 $$ \Co_\h^\g (\pi )=\{ \mu : U(\g) \to V, \; \forall u\in U(\g), \; \forall H \in \h, \; 
 <Hu, \mu >=\pi (H) <u,\mu> \}.$$
 The coinduced representation from $\pi$ is a representation of $\g$ over the space 
 $\Co_\h^\g (\pi)$ defined by 
 $$\forall (u,v) \in U(\g)^2, \forall \mu \in \Co_\h^\g (\pi ), \quad 
 <v, u\cdot \mu ,>=<vu, \mu >.$$
 The action of $X\in \g$  on $\Co_\h^\g (\pi)$ will be denoted $\delta_X^\pi$. 
 
 If $(\pi , V)$ is the trivial representation (just denoted $k$), the coproduct of $U(\g)$ allows to endow $\Co_\h^\g (k)$ with a 
 $k$-superalgebra structure : If $(\lambda , \mu) \in \Co_{\h}^\g (k)^2$,
 $$\forall u \in U(\g), \quad <u, \lambda \mu >=
\sum  (-1)^{\mid \lambda \mid \mid u_{(2)}\mid }<u_{(1)}, \lambda><u_{(2)}, \mu>\; {\rm with}\;
 \Delta (u)=\sum u_{(1)}\otimes u_{(2)}.$$
 The superalgebra $A:=\Co_\h^\g(k)$ is local with maximal ideal
 $${\mathfrak a}=\{\lambda \in A,\quad <\lambda ,1>=0\}.$$

 The action of $X \in \g$ on $A:=\Co_\h^\g (k )$ is given by a derivation denoted 
 $\delta_X  \in \Der \left (\Co_\h^\g (k) \right )$ (instead of $\delta_X^k$).
% $$\forall \mu \in \Co_\h^\g (k), \quad \sigma (X)( \mu )= \delta_X (\mu).$$

 We can thus perform the crossed product construction $(\Co_\h^\g (k)\sharp \g , \delta)$. 
  From now on, we will write $A \sharp \g$ for $(A \sharp \g , \delta )$.\\
 The coproduct on $U(\g)$ allows to endow $\Co_h^\g (\pi)$ with a left $\Co_h^\g (k)$-module structure and (adding the coinduced representation of $\g$) 
 $\Co_\h^\g (\pi )$ become a left $U_A(\Co_\h^\g (k) \sharp \g)$-module.


If $(\pi , V)$ is a representation of $\h$, then the induced representation from $(\pi, V)$ is the $U(\g)$-module structure given by left multiplication on the superspace 
$$Ind_\h^\g(V)=U(\g)\otimes_{U(\h)}V.$$
 As in \cite{C4}, let us notice that the $U(\g)\otimes U(\h)^{op}$-module $U(\g)$ is isomorphic to 
$\dfrac{U_A(A\sharp \g)}{U_A(A\sharp \g){\mathfrak a}}$. Thus,   $Ind^\g_\h (V)$ is endowed  with a natural left $U_A(A\sharp \g)$-module structure.


\section{Algebraic structure on the $\Co_\h^\g (k)$}

\subsection{The symmetric algebra} 

Let $V=V_{\overline{0}}\oplus V_{\overline{1}}$ be a $k$-vector superspace of finite dimension.  
Let  $(e_1, \dots, e_n)$ be a basis of $V_{\overline{0}}$ and $(\epsilon_{1}, \dots , \epsilon_m )$ be a basis of $V_{\overline{1}}$.
If $\underline{a}=(a_1, \dots , a_n, \alpha_1, \dots , \alpha_m) \in \N^n\times \{0,1\}^m$, we set 
$$e^{\underline{a}}\epsilon^{\underline{\alpha}}=e_{1}^{a_1}\dots e_{n}^{a_n}\e_{1}^{\alpha_1}\dots \e_{m}^{\alpha_m}.$$
 The monomials  $e^{\underline{a}}\epsilon^{\underline{\alpha}}$ form a basis of the $k$ superalgebra $S(V)$. \\

\begin{notation}
We will write $e^{\underline{a}}$ for  $e^{\underline{a}}\epsilon^{\underline{\alpha}}$  when there is no ambiguity.
\end{notation}
 
 As $S(V)$ is a cocommutative Hopf algebra, its dual is a supercommutative $k$-superalgebra. Denote by ${S(V)^*}_f$ its final dual consisting in linear forms with finite codimensional kernel. The dual basis   denoted 
 $\left (\mu_{e^{\underline{a}}} \nu_{\epsilon^{\underline{\alpha}}} \right )$  is defined by 
 $$
\begin{array}{l} 
< \underline{e}^{\underline{b}}\underline{\epsilon}^\beta , \mu_{\underline{e}^{\underline{a}}}>=
 \delta _{\underline{a}, \underline{b}} \delta_{\underline{0}, \beta}.\\
 < \underline{e}^{\underline{b}}\underline{\epsilon}^\beta , \nu_{\epsilon^{\underline{\alpha}}}>=
 \delta _{\underline{0}, \underline{b}} \delta_{\underline{\alpha}, \underline{\beta}}.\\
 \end{array}$$
 
It  is a basis of ${S(V)^*}_f$ (the elements of $S(V)^*$ with finite rank) and satisfies the following properties :
\begin{proposition} If $a_i, b_i \in \N$ and $\alpha_i, \beta_i \in \{0,1\}$, then 
 $$\begin{array}{l}
\mu_{e_i^{a_i}} \times \mu_{e_i^{b_i}}=
\begin{pmatrix}a_i+b_i\cr a_i \end{pmatrix} \mu_{e_i^{a_i+b_i}}\\
(\mu_{\e^i})^2=0\\
 \mu_{e_i^{a_i}} \times \mu _{e_j^{a_j}}=\mu _{e_i^{a_i}e_j^{a_j}}\; {\rm if}\; i<j\\
 % \mu_{e_i^{a_i}} \times \nu _{\e_j^{\alpha_j}}=\mu _{e_i^{a_i}\e_j^{\alpha_j}}\\
   \nu_{\e_i^{\alpha_i}} \times \nu _{\e_j^{\alpha_j}}=
   (-1)^{\mid \alpha_i \mid \mid \alpha_j \mid }\nu _{\e_i^{\alpha_i}\e_j^{\alpha_j}}\\
 \end{array}$$
 \end{proposition}
 
 {\bf Notation :}

For simplicity, the elements $\mu_{e_i^{p^j}} \in S(V)^*$ will be denoted $\mu_{i,j}$ and the elements $\mu_{\epsilon_s} \in S(V)^*$ will be denoted $\nu_s$
 
 $S(V)^*$ is a super commutative local superalgebra algebra with maximal ideal 
 $${\mathfrak m}=\{ f\in S(V)^*, \quad <f,1>=0\}.$$
   \begin{proposition}
   \begin{enumerate}
  \item $(\mu_{i,j})^p=0$ and $(\nu_s)^2=0$.
  \item $(S(V)^*)_f=k\left [\mu_{i,j}, \nu_s ,\quad  i\in [1,n], j\in \N, s \in [1,m] \right ]$.
 \item  $({\mathfrak m}_{\overline{0}})^p=\{0\}$. 
  \end{enumerate}
 \end{proposition}
 
 {\it Proof :}
 
 Let us prove that $\mu_{e_{i}^{a_i}}^p=0$. From the previous proposition, we get 
  $$\mu_{e_{i}^{a_i}}^p=\dfrac{(pa_i)!}{(a_i!)^p}\mu_{e_i^{pa_i}}.$$
But  $\dfrac{(pa_i)!}{(a_i!)^p}$ is divisible by $p$. Indeed 
$$\begin{array}{rcl}
(pa_i)!&=&[pa_1\times (pa_1-1) \times (pa_1-2)\dots] \times \\
&&[(p-1)a_1\times ((p-1)a_1-1) \times ((p-1)a_1-2)\dots] \times\\
&&[(p-2)a_1\times ((p-2)a_1-1) \times ((p-2)a_1-2)\dots] \times \dots 
\end{array}$$

As $\nu_i$ is an odd element, we have $(\nu_i)^2=0$. $\Box$\\

 {\bf Notation :}

For simplicity,  the derivation 
$\dfrac{\partial}{\partial \mu_{i,j}} \in Der_k({S(V)^*})$  will be denoted $\partial_{i,j}$ 
and the derivations 
$\dfrac{\partial}{\partial \nu_{s}}\in Der_k({S(V)^*})$  will be denoted $\overline{\partial_{s}}$. \\




 
 \subsection{The algebra $\Co_{\mathfrak h}^{\mathfrak g}(k)$}
 
 
 Let $\g$ be a Lie $k$-superalgebra and $\h$ be a Lie subsuperalgebra of $\g$. 
 Given a supplement  ${\mathfrak p}$ of $\mathfrak h$ in $\mathfrak g$ and a basis 
 $\underline{e}=(e_1, \dots e_n; \epsilon_1, \dots, \epsilon_m)$ of it, we define a map
 $J_0 :  \Co_{\mathfrak h}^{\mathfrak g}(k) \to Hom_k \left (S({\mathfrak p}), k \right )$
by for any $\lambda \in \Co_{\mathfrak h}^{\mathfrak g}(k)$, 
 $$<\underline{e}^{\underline{a}}, J_0(\lambda)>= <\underline{e}^{\underline{a}}, \lambda>.$$
 From now on, we set 
 $$A:=\Co_\h^\g(k).$$
The following proposition is proved in \cite{C1}
 \begin{proposition} 
 The map $J_0$ is an isomorphism of superalgebras.
 
   \end{proposition}
   
   We also define  an isomorphism 
 $J_\pi : \Co_{\mathfrak h}^{\mathfrak g}(\pi) \to Hom \left (S({\mathfrak p} \right ), V)$ by : For all $\alpha \in \N^{n,m}$, 
 $$<e^{\underline{a}}\epsilon^{\alpha}, J_\pi (\mu)>= 
 <e^{\underline{a}}\epsilon^{\alpha},\mu >.$$
 %$J_\pi$ sends $\Co_{ {\mathfrak h},f}^{\mathfrak g}(\pi) $ onto ${S(V)^*}_f$. 
 One has for any $a \in A$ and $\mu \in \Co_{\mathfrak h}^{\mathfrak g}(\pi)$,
 $$J_\pi (a\mu)=J_0(a)J_\pi (\mu ).$$

  
  
  


  To study $A=\Co_\h^\g (k)$, we will  make use of a filtration $\F$ on $U(\g)$  slightly different from $\widetilde{\F}$. 
  
 \begin{notation} Let us introduce the following notation :
 $$\begin{array}{l}
 e_{i,j}={e_i}^{p^j}, \\ 
 \left [0,p-1\right ]^{k,m}=\left [0,p-1 \right ]^{k} \times \{0,1\}^m,\\
 {\rm  For} \; (\underline {a}, \underline{\alpha}) \in [0,p-1]^{n(r+1),m}, \;
 e^{\underline{a}}\epsilon^{\underline{\alpha}}=
\left [ \prod_{  j \in [0,r]} \prod_{i \in [1,n]}\e_{i,j}^{a_{i,j}} \right ]\prod_{s=1}^m \epsilon_s^{\alpha_s} 
 \end{array}.$$
 
 \end{notation}
 $$\begin{array}{l}
 \F_{-1}U(\g)=U(\h)\\
 {\rm If\;} r \in \N, \; {\F}_rU(\g)= 
  U(\h)Vect < e^{\underline{a}}\epsilon^{\underline{\alpha}}; 
  \quad (\underline{a}, \underline{\alpha})\in [0,p-1]^{n(r+1),m}>.
  \end{array}$$
  
  \begin{proposition}\label{properties of mathcal F}
  \begin{enumerate}
  \item If $H\in \h$, one has Proposition $[H,{\mathcal F}_rU(\g) ]\subset {\mathcal F}_rU(\g)$. As a consequence, ${\mathcal F}_rU(\g) H\subset {\mathcal F}_rU(\g)$.
  \item ${\mathcal F }_rU(\g)=
 {\displaystyle  \oplus}_{(\underline{a}, \underline{\alpha})\in [0,p-1]^{n(r+1),m}}
  e^{\underline{a}}\epsilon^{\underline{\alpha}}U(\h)$.
%  \item Denote by $\check{}$ the antipode of $U(\g)$. The vector space ${\mathcal F}_rU(\g)$ is stable under $\; \check{}$
  \end{enumerate}
  \end{proposition}
  
  {\it Proof of Proposition \ref{properties of mathcal F} :} 
  \begin{enumerate}
  \item It follows from Proposition \ref{primitive elements} that $[H,e^{\underline{a}}\epsilon^{\underline{\alpha}}]\in{\mathcal F}_rU(\g)$ if $e^{\underline{a}}\epsilon^{\underline{\alpha}}\in {\mathcal F_r}U(\g).$
  \item The inclusion $ {\displaystyle  \oplus}_{(\underline{a}, \underline{\alpha})\in [0,p-1]^{n(r+1),m}}
  e^{\underline{a}}\epsilon^{\underline{\alpha}}U(\h) \subset 
   {\displaystyle  \oplus}_{(\underline{a}, \underline{\alpha})\in [0,p-1]^{n(r+1),m}}
  U(\h)e^{\underline{a}}\epsilon^{\underline{\alpha}}$ can be shown by induction on the length of an element of 
$U(\h)$. So can the inverse inclusion.
% \item   It follows easily from the proof of Proposition \ref{stability under check}.
  $\Box$\\
  \end{enumerate}
  
  The filtration ${\F}$ we have introduced on $U(\g)$ induces a filtration  on the $k$-algebra $A$  as follows :  
 $$ \dots \F_r A \subset \F_{r-1}A \subset \dots \subset \F_1 A  \subset \F_0 A \subset A$$
  where 
  $\F_rA=\{\lambda \in A, \; <\lambda, u>=0 \; if \, u \in {\F}_{r-1}U(\g)\}$.
  As ${\displaystyle \cap_{r \in \N} \F_rA=\{0\}}$, this filtration defines a separated topology on $A$. Moreover, $A$ is complete for this topology. \\
  
  Let us denote 
  ${\mathfrak a}=J_0({\mathfrak m})=\{\lambda \in A, \; <1,\lambda>=0\}$. 
  
  \begin{remarks}
  Denote by $F_kU(\g)$ the usual ascending filtration of $U(\g)$ and $F_kA$ the decresing filtration it induces on $A$.
  
  \begin{enumerate}
  
  \item  ${\mathfrak M}^q$ is included in $F_qA$ but is not equal if $q\geq p$. Indeed, 
  ${\mathfrak M}^p$ equals $k[\nu_1, \dots , \nu_m]$ but not $F_pA$.
  
  \item $J_0(F_q A)=Vect (\mu_{\underline{e}^{\underline{a}}}, \; \mid a \mid \geq q-1)$.
  \end{enumerate}
  \end{remarks}
 
  {\bf Notation :} Associated to the choice of a basis of a supplement ${\mathfrak p}$ of 
  $\mathfrak h$ in $\mathfrak g$, the element 
  $\mu_{i,j} \in S({\mathfrak p})^*$ and 
  $\nu_s \in  S({\mathfrak p})^*$ were defined earlier for $i\in [1,n]$, $j \in \N$, $s \in [1,m]$. 
  
  The element 
  ${J_{0}}^{-1}\ \left ( \mu_{i,j}\right ) \in A$ will be  denoted 
  $ \eta_{e_{i}^{p^j}}$ or $\eta_{i,j}$. The element ${J_{0}}^{-1}\ \left ( \nu_{s}\right ) \in A$ will be  denoted 
  $ \zeta_s$.
  
 
  
  
  

  The derivation ${J_0}^{-1} \circ \partial _{i,j} \circ J_0$ 
  (respectively ${J_0}^{-1} \circ \overline{\partial _{s}} \circ J_0$)
   of $A$ will also be denoted $\partial_{i,j}$ (respectively $\overline{\partial_s}$). \\
  
  
The study of differential operators of $Hom \left (S({\mathfrak p}), V \right )$ is transferred to 
 $\Co_{\mathfrak h}^{\mathfrak g} (\pi)$ by $J_\pi$. 
The derivation 
 ${J_\pi }^{-1} \circ \partial_{i,j}\circ J_\pi$ (respectively ${J_\pi }^{-1} \circ \overline{\partial_{s}}\circ J_\pi$)
 will be denoted $\partial^\pi_{i,j}$ (respectively $\overline{\partial_s^\pi}$) or more simply 
 $\partial_{i,j}$ (respectively $\overline{\partial_s}$) when there is no ambiguity.\\
 
 


The $k$-superalgebra $A$ is the projective limit of $\dfrac{A}{{\mathcal F}_r A}$ defined by the transpose of the natural injection 
${\mathcal F}_rU(\g) \hookrightarrow {\mathcal F}_{r^\prime}U(\g)$ if $r\leq r^\prime$.
Introduce the following notation : 
$$A_{\leq r}:=k[\{\eta_{i,j}, \zeta_s, \quad i \in [1,n],  \; j \in [0, r], s\in [1,m]\}]. $$
Let $a \in A$.  The class of $a$ in $\dfrac{A}{{\mathcal F}_{r}A}$ has  a unique representant, denoted by $a^{\leq r}$,
 that is a polynomial in the $(\eta_{i,j})_{ j \leq r}$ and $(\zeta)_{s \in [1,m]}$. The map
 $$\begin{array}{rcl}
 A_{\leq r} & \to & \dfrac{A}{{\mathcal F}_r A}\\
 x & \mapsto & x+ \F_r A.
 \end{array}$$
 is a superalgebra isomorphism.
 
 
 
 For $X\in \g$, we define the derivation $\delta_X^{\leq r}$ of $A_{\leq r}$ as follows :
 $$\forall a \in A_{\leq r}, \quad  \delta_X^{\leq r}(a):= \delta_X(a)^{\leq r}.$$
 \begin{proposition}
 If $a \in A$ and $a^{\leq r}$ the representant of the class of $a$ in  $\dfrac{A}{{\mathcal F}_{r+1}A}$ that is a polynomial in the 
$(\eta_{i,j})_{ j \leq r}$'s and $\zeta_s$'s, one has 
$$\delta _X (a)= \varprojlim\delta_{X}^{\leq r} (a^{\leq r})=
 \varprojlim \left [ \delta_X (a^{\leq r})\right ]^{\leq r}.$$
 One has $\delta_X= \varprojlim \delta_X^{ \leq r}.$
\end{proposition}
We write 
$$\delta_{X}^{ \leq r}=\sum_{i=1}^n \sum_{j=0}^{r}f_{i,j}(X)^{\leq r}\partial_{i,j}+
\sum_{s=1}^m g_s(X)^{\leq r}\overline{\partial}_s.$$

More generally, for $r \in \N$, we define 
$$\Co_\h^\g (\pi)_{\leq r}:=\{\lambda \in \Co_\h^\g (\pi), \; \lambda_{\mid {\mathcal F}_{r-1}U(\g)}=0\}=\dfrac{\Co_\h^\g(\pi)}{{\mathcal F}_rA\, \Co_\h^\g(\pi)}.$$
$\Co_\h^\g (\pi)$ is the projective limit of the $\Co_\h^\g (\pi)_{\leq r}$. 
The element $X \in \g$ defines a derivation $\delta_X^{\pi, \leq r}$ of $\Co^\g_{ \h}{(\pi)}^{\leq r}$. We will write 
$$\delta_X^{\pi, \leq r}=F_X^{\pi, \leq r} + \sum_{i=1}^n \sum_{j=0}^r f_{i,j}^{\leq r}(X)\partial_{i,j}
+\sum_{s=1}^m g_s (X)^{\leq r}\overline{\partial_s}$$
where $F_X^{\pi, \leq r}$ is the element of $End_A[\Co_\h^\g (\pi)_{\leq r}]$ such that 
$$\forall v \in V, \quad F_X^{\pi, \leq r} (v)=(X\cdot v)^{\leq r}.$$
One has $\delta_X^\pi=\varprojlim \delta_X^{\pi, \leq r}$.\\



  
  We will now study  a duality property for  the kernel of  coinduced representations. 
  
 
  
  
\section{A duality property on kernels of coinduced representations}


This section is devoted to the proof of the following theorem :
 
 \begin{theorem} \label{annihilators duality}
 Let $k$ be a field of characteristic $p>2$. Denote by $I_\pi \subset U({\mathfrak g})$ the kernel of the representation $\Co_\h^\g (\pi)$. Assume that the $k$-vector space 
 ${\g}/{\h}$ is finite dimensional with odd dimenion $m$.  Then 
 $$I_{\pi}=\check{I}_{\pi^*\otimes \Pi^{m}k_{-strad_{\g /\h}}}.$$
 \end{theorem}
\begin{remarks}
Theorem \ref{annihilators duality} is proved in \cite{Du} for Lie algebras in any characteristic but with the assumption $\g$ finite dimensional. 
It is proved in \cite {C1} in the setting of Lie superalgebras for $\g/\h$ finite dimensional but only in characteristic $0$.
%Our proof is an adjustment of the proof of \cite{C1} to the positive characteristic case. 

The statement of theorem \ref{annihilators duality} will be improved for restricted enveloping superalgebras later. 
\end{remarks}


{\it Proof:}


Let $\g$ be a $k$-Lie superalgebra and $\h \subset \g$ be a Lie subsuperalgebra of $\g$. Let $\mathfrak{p}$ be such that $\g=\h \oplus {\mathfrak p}$ and let $(e_1, \dots , e_n)$ be a basis of ${\mathfrak p}_{\overline{0}}$ and $(\epsilon_1, \dots , \epsilon_m)$ be a basis of ${\mathfrak p}_{\overline{1}}$ . Then 
$U(\g)$ is a free $U(\h)$-module with basis $e^{\underline{a}}\epsilon^{\underline{\alpha}}$ where $\underline{a}$ is a $n$-uple of elements of $[1,p-1]$ and 
$\underline{\alpha} \in \{0,1\}^m$. For $i \in [1,n]$ and $j \in \N$, we define 
$\eta_{i,j} \in \Co_\h^\g(k)$ by 
$$<e^{\underline{a}}\epsilon^{\underline{\alpha}}, \eta_{i,j}>=
\left \{
\begin{array}{l}
0\; if \;   e^{\underline{a}}\epsilon^{\underline{\alpha}} \neq {e_i}^{p^j}\\
1 \; if \; e^{\underline{a}}\epsilon^{\underline{\alpha}}={e_i}^{p^j}
\end{array}\right . $$
and for $s\in [1,m]$, we define $\zeta_{s} \in \Co_\h^\g(k)$ by 
$$<e^{\underline{a}}\epsilon^{\underline{\alpha}}, \zeta_s>=
\left \{
\begin{array}{l}
0\; if \;   e^{\underline{a}}\epsilon^{\underline{\alpha}} \neq \e_s\\
1 \; if \; e^{\underline{a}}\epsilon^{\underline{\alpha}}=\e_s
\end{array}\right . $$






There is a natural truncation functor 
$\tau_{\leq r} : \Co_\h^\g (\pi) \to \Co_{ \h}^{\g}(\pi)^{\leq r}$  and if $f \in \Co_\h^\g (\pi)$, then $\tau_r(f)$ will be denoted $f^{\leq r}$. 
Let us fix $r \in \N$. For simplicity, we set 
$$\begin{array}{l}
\Lambda_{\leq r, \underline{e}}=\left [ \prod_{j \leq r} \eta_{1,j}^{p-1}\dots \eta_{n,j}^{p-1}\right ] \prod_{s \in [1,m]}\zeta_s\\
Ind_{\leq r, \h}^\g(\pi)=
Vect <\left [{\displaystyle 
\prod_{j=0}^r\prod_{i \in [1,n]}}e_{i}^{p^j}\right ]\prod_{s=1}^m \epsilon_s^{\alpha_s}\otimes v, 
\quad v \in V >
\end{array}$$ 

The representation $\pi \otimes \Pi^m k_{strad_{\g/\h}}$ will be denoted $\tilde{\pi}$. 


   
   \begin{lemma}\label{h-module structure on Lambda_r}
   
   If $H\in \h$, one has: $$\delta_H^{\leq r}(\Lambda_{\leq r})=strad_{\g/\h} (H)\Lambda_{\leq r}.$$
  

\end{lemma}

{\it Proof of Lemma \ref{h-module structure on Lambda_r}}: Using Proposition \ref{primitive elements}, we know that  if $w \in {\mathcal F}_rU(\g)$, then $[H,w]\in {\mathcal F}_rU(\g)$ and 
$<wH, \Lambda_{\leq r}>=<[w,H], \Lambda_{\leq r}>.$ 
If $H\in \h$, let us set 
$$[H,e_i]= \sum_{i=1}^n ad(H)_{k,i}e_k +  \sum_{s=1}^m ad(H)_{s,i}\epsilon_s \quad mod\; \h .$$ 
If $H \in \h_{\overline{1}}$, then $ad(H)_{i,i}=0$. 
Let us write 
$$ e^{\underline{a}}\epsilon^{\underline{\alpha}}H=\sum_{(\underline{b},\underline{\beta}) \in [0,p-1]^{n,m}}f_{\underline{b}, \underline{\beta}}(H)
e^{\underline{b}}\epsilon^{ \underline{\beta}}\quad 
mod \;  \h{\mathcal F}_r(U(\g))$$

Let us denote by $\underline{p-1}$ (respectively $\underline{1}$) the element of $[0,p-1]^{n(r+1), m}$ whose components are all equal to $p-1$ (respectively $1$).
If $H \in \h_{\overline{0}}$, the coefficient $f_{\underline{p-1},\underline{1}}(H)$ is 
$$\begin{array}{rcl}
f_{\underline{p-1},\underline{1}}(H)&=&-\sum_{i=1}^n\sum_{j=0}^rp^j(p-1)ad(H)_{i,i}- 
\sum_{s=1}^m ad(H)_{s,s}\\
&=&\sum_{i=1}^n ad(H)_{i,i}- \sum_{s=1}^m ad(H)_{s,s}.\\
\end{array}$$
If $H \in \h_{\overline{1}}$, the coefficient $f_{\underline{p-1},\underline{1}}(H)$ is zero.



Thus 
$$\delta_H^{\leq r}(\Lambda_{\leq r})=strad_{\g/\h}(H)\Lambda_{\leq r}.\Box$$

We will make use of partially defined elements of $\Co_h^\g(\pi)$. If $u\in U(\g)$, denote by $\mu_u$ right multiplication by $u$. We define  
  $$\begin{array}{l}
    \Co_{\h,  \mu_u^{-1}\left ( {\mathcal F}_rU(\g)\right )}^{\g} (\pi):=\\
    \{ \lambda : \mu_u^{-1}\left ( {\mathcal F}_rU(\g)\right ) \to V, \quad \forall H\in \h, \; 
    \forall w \in \mu^{-1}_u \left ( {\mathcal F}_rU(\g)\right ); \; 
    <Hw, \lambda >=\pi (H)\left ( <w, \lambda>\right )
   \}.
   \end{array}$$
    \begin{equation}\label{iota_r,r+1}
   \begin{array}{rcl}
   \iota_{r,r+1} : 
   \Co_{\h, \mu_u^{-1}\left ({\mathcal F}_rU(\g)\right )}^{\g} (\pi) &\to& 
   \Co_{\h, \mu_u^{-1} \left ( {\mathcal F}_{r+1}U(\g)\right )}^{\g} (\pi)\\
   f& \mapsto&f\eta_{1,r+1}^{p-1}\dots \eta_{n,r+1}^{p-1}
   \end{array}
   \end{equation}
    \begin{equation}\label{iota_r,s}
   \iota_{r,s}=\iota_{r,r+1}\dots \iota_{s-1,s} : 
   \Co_{\h, \mu_u^{-1}\left ( {\mathcal F}_rU(\g) \right ) }^{\g} (\pi) \to 
   \Co_{\h, \mu_u^{-1}\left ({\mathcal F}_sU(\g)\right )}^{\g} (\pi )
   \end{equation}
   
   and we set 
   $$\widehat{\Co}^\g_{\h,u}(\pi )=
   \lim\limits_{\substack{\to\\ \iota}}\Co_{\h , \mu_u^{-1}\left ({\mathcal F}_rU(\g)\right ) }^{\g}(\pi).$$
   
 
To any $u \in {\mathcal F}_rU(\g)$ and $v \in V$, we associate an element 
$\Phi^r_{u\otimes (\Lambda_{ \leq r}\otimes  v)} \in \Co^\g_{\h, \mu_u^{-1}\left ( {\mathcal F}_r U(\g)\right )}(\pi )$. 
$$\begin{array}{rcl}
\Phi^r _{u\otimes (\Lambda_{\leq r}\otimes v) }: 
\quad\mu_u^{-1}\left ( {\mathcal F}_rU(\g) \right ) & \to & V\\
w& \mapsto & <wu, \Lambda_{\leq r} \widehat{v}_\pi >
\end{array}$$
where $\widehat{v}_\pi$ is the element of $\Co_h^\g (\pi)^{\leq r}$ determined by : for all $(\underline{a}, \underline{\alpha})\in [0,p-1]^{n(r+1),m}$, 
$$\begin{array}{l}
<e^{\underline{a}}\epsilon^{\underline{\alpha}}, \widehat{v}_\pi>=0 \; {\rm}\; (\underline{a}, \underline{\alpha})\neq 0\\
<1, \widehat{v}_\pi>=0
\end{array}.$$




\begin{lemma} \label{definition of Phi} 
We endow $k\Lambda_{\leq r}\otimes V$ with the following $\h$-module structure :
$$\forall v\in V, \quad \forall H\in \h, \quad H\cdot (\Lambda_{\leq r}\otimes v)=strad_{\g/\h}(H)\Lambda_{\leq r}\otimes v+\Lambda_{\leq r}\otimes \pi(H)(v).$$
Let 
$u\otimes_{U(\h)}(\Lambda_{\leq r} \otimes v) \in Ind_{ \h}^\g(\tilde{\pi})^{\leq r} $. The map 

$$\begin{array}{rcl}
\Phi^r _{u\otimes_{U(\h)} (\Lambda_{\leq r}\otimes v) }: 
\quad\mu_u^{-1}\left ( {\mathcal F}_rU(\g) \right ) & \to & V\\
w& \mapsto & <wu, \Lambda_{\leq r} \widehat{v}_\pi >
\end{array}$$

is well defined.

\end{lemma}

{\it Proof of Lemma \ref{definition of Phi} :}

Let $v\in V$, $H \in \h$, $u \in U(\g)$ such that  $wuH \in {\mathcal F}_r U(\g)$ . 
The equality $$(-1)^{\mid H \mid (\mid u \mid+\mid w\mid )}wuH=Hwu-[H,wu]$$ 
and Proposition \ref{primitive elements} show that  if 
$wu \in {\mathcal F}_r U(\g)$ then $wuH \in {\mathcal F}_r U(\g)$ 



On one hand, one has :
$$\begin{array}{rcl}
<w, \Phi^r_{uH \otimes (\Lambda_{\leq r} \otimes v)}>&=& <wuH, \Lambda_{\leq r} \hat{v}_\pi >\\
&=&
<wu, \delta_H^{\pi, \leq r}(\Lambda _{\leq r} \hat{v}_\pi)>\\
&=& <wu, \delta_H^{\leq r}(\Lambda _{\leq r}) \hat{v}_\pi+ (-1)^{m\mid H\mid }\Lambda_{\leq r}\delta_H^{\pi, \leq r}(\hat{v}_\pi)>\\
&=& <wu, \strad_{\g/\h}(H)\Lambda _{\leq r} \hat{v}_\pi +(-1)^{m\mid H\mid } \Lambda_{\leq r}\delta_H^{\pi, \leq r}(\hat{v}_\pi)>\\
\end{array}$$ 
On the other hand :
$$\begin{array}{rcl}
<w, \Phi^r_{u \otimes H \cdot (\Lambda_{\leq r} \otimes v)}>
&=& <wu,\strad_{\g/\h}(H)\Lambda _{\leq r} \hat{v}_\pi+ (-1)^{m\mid H\mid }\Lambda_{\leq r}(\widehat{\pi(H)v})_\pi>\\
%&=& <wu, \delta_H^{\leq r}(\Lambda _{\leq r}) \hat{v}_\pi>+ <wu,\Lambda_{\leq r}><1,\delta^{\pi, \leq r}_H(\widehat{v}_\pi )>\\


\end{array}$$ 

To finish the proof of the lemma \label{definition of Phi^r}, we need to prove that 
$$<wu, \Lambda_{\leq r}\delta_H^{\leq r}(\hat{v}_\pi)>= <wu,\Lambda_{\leq r}(\widehat{\pi(H)v})_\pi>.$$
But as $wu \in {\mathcal F}_r U(\g)$, one has 
$<wu, \Lambda_{\leq r>}=0$ except if 
$wu=\left [\prod_{j=0}^r e_{1}^{p^j}\dots e_n^{p^j}\right ]\epsilon_1 \dots \epsilon_m.$ Thus, 
$$\begin{array}{rcl}
<wu, \Lambda_{\leq r}\delta_H^{\pi, \leq r}(\hat{v}_\pi)>&=&
<wu, \Lambda_{\leq r}><1,\delta_H^{\leq r}(\hat{v}_\pi)>\\
&=&<wu, \Lambda_{\leq r}><H, \hat{v}_\pi>\\
&=&<wu, \Lambda_{\leq r}>\pi(H)(<1, \hat{v}_\pi>)\\
&=&<wu, \Lambda_{\leq r}>\pi(H)(v)\\
&=& <wu,\Lambda_{\leq r}><1,(\widehat{\pi(H)v})_\pi>\\
 &=&<wu,\Lambda_{\leq r}(\widehat{\pi(H)v})_\pi>.\Box
 \end{array}$$



%The following sublemma allows to express $\Phi^r_{u\otimes (\Lambda_r\otimes  v)}$ in another way :

% \begin{sublemma}
 %$wX_1\dots X_t \in {\mathcal F}_rU(\g)$. 
%$$ <w,  \delta_{X_1}^{\pi, \leq r} \dots \delta_{X_t}^{\pi, \leq r}f>=<wX_1\dots X_t, f>.$$
% \end{sublemma}
 
% {\it Proof of the sublemma :}
 
% We will prove the lemma by induction on $t$. 
 
 %For $t=1$, it is obvious by the definition  of 
 %$\Co^\g_\h (\pi)^{\leq r}$. 
 
 %Assume now that $t>1$ and that the lemma is true for $t-1$. Then, by using the case $t=1$ and the induction hypothesis, we have the following computation 
%<w,\delta_{X_1}^{\pi, \leq r} \dots \delta_{X_t}^{\pi, \leq r}f>&=&
%<w, \delta_{X_1}^{\pi, \leq r}\left [\delta_{X_2}^{\pi , \leq r}\dots \delta_{X_t}^{\pi, \leq r}f\right ]>\\
% &=& <wX_1, \left [\delta_{X_2}^{\pi, \leq r} \dots \delta_{X_t}^{\pi ,\leq r} f\right ]>\\
% &=& <wX_1\dots X_t, f>.\Box
 %\end{array}$$

The following lemma ensures that 
$\Phi ^r_{u\otimes_{U(\h)} (\Lambda_{\leq r}\otimes  v)}\in \Co^\g_{\h, \mu_u^{-1} 
\left ( {\mathcal F_r}U(\g)\right )}(\pi)$  defines an element 
$\Phi _{u\otimes_{U(\h)} (\Lambda_{\leq r} \otimes  v)} \in \widehat{\Co}^\g_{\h, u}(\pi )$ by taking inductive limit.




\begin{lemma}\label{relation iota and Phi}
For any $r\in \N$,  for any $u \in {\mathcal F}_{r} U(\g)$ and  any $w \in \mu_u^{-1}\left ({\mathcal F}_{r+1} U(\g)\right )$, 
the following relation is satisfied 
$\iota_{r,r+1}\left [  \Phi^r_{u\otimes_{U(\h)} (\Lambda_{\leq r}\otimes  v)}\right ](w)=
\Phi^{r+1}_{u\otimes_{U(\h)} (\Lambda_{\leq r+1}\otimes  v)}( w)$. 
 \end{lemma}
 
 {\it Proof :}
 

 
 Write $w$ in the basis $e_{i,j}$ : 
 $w=e_{1,r+1}^{a_{1,r+1}}\dots e_{n,r+1}^{a_{n,r+1}}w^\prime $ with 
 $w^\prime \in \mu_u^{-1}\left [{\mathcal F}_rU(\g)\right ]$. 
 One has 
 $<w,\eta_{1,r+1}^{p-1}\dots \eta_{n,r+1}^{p-1}\Phi^r_{u\otimes_{U(\h)} (\Lambda_{\leq r}\otimes  v)}>\neq 0$ 
 implies $a_{1,r+1}=\dots=a_{n,r+1}=p-1$. 
 Then 
 $$<w, \iota_{r,r+1}\Phi^r_{u\otimes (\Lambda_{\leq r}\otimes  v)}>=
 <w^\prime, \Phi^r_{u\otimes (\Lambda_{\leq r}\otimes  v)}>=
(p-1)! ^n<w^\prime u,\Lambda_{\leq r}v>.$$
 On the other hand,
 $$<w,\Phi^{r+1}_{u\otimes (\Lambda_{\leq r+1} \otimes  v)}>
 =<wu, \Lambda_{\leq r+1}\widehat{v}_\pi>$$ 
 
 Write 
$w=e_{1,r+1}^{a_{1,r+1}}\dots e_{n,r+1}^{a_{n,r+1}}w^\prime $ with $w^\prime \in {\mathcal F}_rU(\g)$. 
If $a_{i,r+1}<p-1$, then the maximal power of $e_i$ in $wu$ 
is $a_{i,r+1}p^{r+1}+2(p^{r+1}-1)\leq p^{r+1}(p-2)+2(p^{r+1}-1)=p^{r+2}-2$. Here, we have used the fact that the maximal power of $e_i$ in 
${\mathcal F}_rU(\g)$ is $p^{r+1}-1$. 
Thus $<wu, \Lambda_{\leq r+1}v>=0$.

Consequently  $<wu, \Lambda_{\leq r+1}v>\neq 0$ implies $w=e_{1,r+1}^{p-1}\dots e_{n,r+1}^{p-1}w^\prime $ with $w^\prime \in {\mathcal F}_rU(\g)$ and then 
$$<w,\Phi^{r+1}_{u\otimes_{U(\h)} (\Lambda_{\leq r+1}\otimes  v)}>
 =<wu, \Lambda_{\leq r+1}\widehat{v}_\pi>=(p-1)!^n<w^\prime u, \Lambda_{\leq r}\widehat{v}_\pi>.$$ 
 This finishes the proof of Lemma \ref{relation  iota and Phi}.$\Box$
 
 \vspace{2em}
 
 Lemma \ref{relation iota and Phi} also shows that the map $\Phi^r_{u\otimes (\Lambda_r\otimes  v)}$ does not depend on the ${\mathcal F}_rU(\g)$ to which $u$ belongs. This allows to define a map $\Phi : U(\g)\otimes_{U(\h)}(\Pi^m k_{strad_{\g/\h}} \otimes \pi)  \to \widehat{\Co_h^\g(\pi )}$ as the inductive limit of the $\Phi_r$'s. 
 It does not seem to be true that $\Phi$ is a morphism of $\g$-modules. 

 \begin{lemma}\label{Phi is injective}
Let $u$ be an element of $\mu_u^{-1}\left [{\mathcal F}_rU(\g)\right ]$.  
If $\Phi^r_{u\otimes_{U(\h)} (\Lambda_{\leq r} \otimes v)}=0$ then 
 $u\otimes_{U(\h)} (\Lambda_{\leq r} \otimes v)=0$. 
 \end{lemma}
 
 {\it Proof of Lemma \ref{Phi is injective}}
 
 
 Let $u\otimes_{U(\h)}( \Lambda_{\leq r }\otimes v)=
 \sum e^{\underline{a}}\epsilon^{ \underline{\alpha}}\otimes \Lambda_{\leq r }v _{\underline{a}, \underline{\alpha}}$ 
 such that $\Phi^{r} \left (u\otimes_{U(\h)}( \Lambda_{\leq r}\otimes v)\right )=0$. We set
 $$e^{\check{\underline{a}}}\epsilon^{\check{\underline{\alpha}}}=
e^{\underline{p-1}-\underline{a}}\epsilon^{\underline{1}-{\underline{\alpha}}} $$
where $\underline{p-1}=(p-1, \dots, p-1)\in \N^{n(r+1)}$ and $\underline{1}=(1,\dots,1)\in \{0,1\}^m$. This is possible because all the 
$a_{i,j}\leq p-1$. One has 
$$0=\Phi^{r}_{u\otimes (\Lambda_{\leq r} \otimes v)}
\left ( e^{\check{\underline{a}}_0}\epsilon^{\check{\underline{\alpha}}_0}\right )=
<\Lambda_{\leq r}\hat{v}_\pi, ue^{\check{\underline{a}}_0}\epsilon^{\check{\underline{\alpha}}_0}>=C^{ste}
v_{\underline{a_0}, \underline{\alpha_0}}.\Box$$\\

 
 {\it Proof of Theorem \ref{annihilators duality}}

Let $J_{\Pi^{m}k_{-strad_{\g/\h}} \otimes \pi}$  be the kernel of the induced representation 
$Ind_\h^\g(\Pi^m k_{-strad_{\g/\h}}\otimes V)$. We will show the inclusion $I_\pi \subset J_{\Pi^mk_{strad_{\g/\h}} \otimes \pi}$ 

Let $u \in I_\pi$, $u^\prime, w \in U(\g)$. Let $r$ such that ${wuu^\prime  \in {\mathcal F}_rU(\g)}$.  One has 
$$<w, \Phi^r_{uu^\prime \otimes (\Lambda_{\leq r}\otimes v)} >=
<wuu^\prime , \Lambda_{\leq r}\hat{v}_\pi >=<w, \delta_u \left [ \delta_{u^\prime} (\Lambda_{\leq r} \hat{v}_\pi) \right ]>=0.$$
The last equality holds because $u \in I_\pi$. Thus $\Phi^r_{uu^\prime \otimes (\Lambda_{\leq r}\otimes v)}=0$ and 
$uu^\prime \otimes (\Lambda_{\leq r}\otimes v)$=0. This proves that  $u \in J_{\Pi^m k_{strad_{\g/\h}} \otimes \pi}$. 

But we know that  $J_{\Pi^m k_{strad_{\g/\h}} \otimes \pi}=\check{I}_{\Pi ^m k_{{-strad}_{\g/\h}} \otimes \pi^*}$ because 
$\left [ Ind^{U(\g)}_{U(\h)}(\Pi^m k_{strad_{\g/\h}} \otimes \pi) \right ]^*\simeq \Co^\g_\h (\Pi ^m k_{-strad_{\g/\h}} \otimes \pi^*)$. Thus, we have proved the inclusion 
$$I_\pi \subset \check{I}_{\Pi ^m k_{-strad_{\g/\h}}\otimes \pi ^*}.$$
Replacing $\pi$ by $\Pi^{m}k_{strad_{\g/\h}}\otimes \pi ^*$, we get the inverse inclusion.$\Box$\\
 

 For restricted enveloping algebras, we will have  a real duality between $Coind_{U^\prime (\h)}^{U^\prime (\g)}(\pi )$ and  
$Coind_{U^\prime (\h)}^{U^\prime (\g)}(\pi \otimes \Pi^mk_{-strad_{\g/ \h}})$. 



 
 
 



\section{Restricted Lie superalgebra} 
Let $k$ be a field of characteristic $p>2$. A restricted Lie algebra $\g$ is a Lie algebra endowed with a  $p$ operation $(-)^{[p]} :\g \to \g$ , $X \mapsto X^{[p]}$ satisfying some special conditions (\cite{J}). A morphism of restricted Lie algebras is a map of Lie algebras preserving the $p$-operation.

\begin{definition}(\cite{P})
Let $k$ be a field of positive characteristic $p>2$.
 A Lie superalgebra $\g=\g_{\overline{0}}\oplus \g_{\overline{1}}$ is called restricted if 

\begin{itemize}
\item the Lie algebra $\g_{\overline{0}}$ is restricted;
\item the action of $\g_{\overline{0}}$ on $\g_{\overline{1}}$ defines a restricted morphism from 
$\g_{\overline{0}}$ to $\g l(\g_{\overline{1}})$.
\end{itemize}
A linear map   $f :\g \to \g^\prime$ is a morphism of restricted Lie superalgebras if 
$$\forall X \in \g_{\overline{0}}, \quad  f(X^{[p]}) =  f(X)^{[p]}.$$


\end{definition}



\begin{definition} Let $\mathfrak g$ be a restricted Lie superalgebra. A Lie subsuperalgebra $\mathfrak h$ of 
$\mathfrak g$ is a restricted subsuperalgebra of $\mathfrak g$ if 

(i) $\mathfrak h$ is a restricted superalgebra.

(ii) The inclusion  map ${\mathfrak h} \to {\mathfrak g}$ is a morphism of restricted superalgebras. 

\end{definition}

The restricted enveloping super algebra of a restricted Lie superalgebra is defined as follows :
$$U^\prime({\mathfrak g})=\dfrac{U({\mathfrak g})}{<X^p-X^{[p]}, \quad X \in \g_{\overline{0}}>}.$$
 

%We now introduce the definition of a restricted Lie Rinehart superalgebra (\cite{Rumynin}):

% \begin{definition} Let $k$ be a field of characteristic $p$ and $A$ be  commutative $k$-superalgebra with unity. A restricted Lie-Rinehart superalgebra $(A, L, (-)^{[p]}, \omega )$ over $A$ is a Lie-Rinehart over $A$ 
 %(\cite{Rinehart}) such that 
 %\begin{enumerate}
% \item  $(L, (-)^{[p]})$ is a restricted Lie superalgebra over $k$; 
 %\item the anchor map $\omega : L \to \Der (A)$ is a  restricted Lie superalgebra morphism; 
 %\item For all $a \in A$ and all $X\in L$, the following relation holds 
 %$$(aX)^{[p]}=a^p X^{[p]}+ \omega ((aX)^{p-1})(a)X.$$
 %\end{enumerate}
 %A morphism of restricted Lie Rinehart superalgebras is a morphism of $A$-modules and of restricted Lie super algebras. 
% \end{definition}
 
 % The restricted universal enveloping algebra of a restricted Lie-Rinehart superalgebra is defined (\cite{Rumynin}):
 %\begin{definition}
 %Let $A$ be a commutative $k$-superalgebra and 
% let $(A,L,(-)^{[-]}, \omega )$ be a restricted Lie-Rinehart superalgebra. The restricted universal enveloping superalgebra is a universal triple $(U^\prime_A(L), \iota_A , \iota_L )$ with an associative superalgebra map $\iota_A :A \to U^\prime_A(L)$ and a restricted Lie  superalgebra map 
 %$\iota_L : L \to U^\prime_A(L)$ such that for all $D \in L$ and $a\in A$
%$$ \begin{array}{l}
%\iota_A(a)\iota_L(D)=\iota_L(aD)\\
%\iota_L(D)\iota_A(a)-\iota_A(a)\iota_L(D)=\iota_A\left (\omega(D)(a)) \right )
%\end{array}$$
 %\end{definition}
% One has $U^\prime _A(L)=\dfrac{U_A(L)}{<D^p-D^{[p]}, \quad D \in L >}$. 
% \begin{examples}
% \begin{enumerate}
% \item If $A=k$, a restricted Lie Rinehart superalgebra over $k$ is a $k$-restricted Lie superalgebra and its enveloping superalgebra is 
 %$$U^\prime({\mathfrak g})=U({\mathfrak g})/<X^p-X^{[p]}, \quad X \in \g_{\overline{0}}>.$$
%If the restricted Lie superalgebra is abelian, its enveloping superalgebra will be denoted 
%$$S^\prime({\mathfrak g})=S({\mathfrak g})/<X^p, \quad X \in \g_{\overline{0}}>.$$
% \item If $A$ is a commutative $k$-superalgebra, then $(A, \Der (A), (-)^p, id)$ is a restricted Lie-Rinehart superalgebra (\cite{Hochschild} lemma 1).
% In characteristic $p$, if $D$ is a derivation of $A$, so is $D^p$. The restricted enveloping superalgebra of $(A, \Der (A), (-)^p, id)$ is the superalgebra of differential operators on 
 %$A$. 
% \item  In \cite{BYZ}, it is shown that weakly restricted Poisson algebras give rise to restricted Lie-Rinehart algebras.
%\item The restricted crossed product: Assume that ${\mathfrak g}$ is a restricted Lie superalgebra and that there exists a morphism of restricted Lie-superalgebras $\sigma : {\mathfrak g} \to Der (A)$. Then, there exists a unique structure of restricted Lie-Rinehart superalgebra on $A \otimes {\mathfrak g}$  (extending that of ${\mathfrak g}$)
% with anchor $\omega : A\otimes {\mathfrak g} \to Der (A)$, $\omega (a\otimes X)=a\sigma (X)$
%and  such that: 
% For all $X, Y \in {\mathfrak g}$ and all $a,b \in A$
 %$$\begin{array}{l}
 %[a\otimes X, b\otimes Y]=a\sigma (X)(b) \otimes Y -b\sigma (Y)(a) \otimes X +ab \otimes [X,Y]\\
% \end{array}$$

 %\end{enumerate}
% \end{examples}

  

 
 
 The Poincaré-Birkhow-Witt theorem holds for restricted Lie  superalgebras (\cite{P})
 
 
 
\begin{theorem}

Let $\g$ be a restricted Lie superalgebra. 
Suppose that $(e_i)_{i \in I}$ is an ordered  basis of ${\mathfrak g}_{\overline{0}}$ and $(\epsilon_j)_{j\in J}$ is an ordered basis of $\g _{\overline{1}}$. 

The monomials $e_{i_1}^{a_1}\dots e_{i_n}^{a_n}\epsilon_{j_1}...\epsilon_{j_t}$ with 
$i_1<\dots <i_n$, $ j_1<\dots <j_t$, $a_i \in [0,p-1]$  form a basis of the restricted enveloping algebra.
\end{theorem}

Let $\g$ be a restricted $k$-Lie superalgebra and $\h$ a restricted Lie subalgebra of $\g$. 
The Lie superalgebra $\h$ acts on $\g/\h$ by $ad_{\g/\h}$. Thus, the superalgebra $U(\h)$ acts on 
$\g/\h$ and, for $H\in \h_{\overline{0}}$, 
$$ad_{\g/\h}(H^p)=ad_{\g/\h}(H)^p.$$
As  $(ad_\g(H))^p=ad_\g(H^{[p]})$, one has 
$(ad_{\g/\h}(H))^p=ad_{\g/\h}(H^{[p]})$. As a consequence $ad_{\g/\h}$ is a representation of $U^\prime (\h)$ over $\h$. 

The character $strad_{\g/\h}$ is well defined as a character of  $U^\prime(\h)$. \\


 
 
 
 \begin{definition}\label{Lie derivative}
 (\cite{C2})  Let $A$ be a $k$-superalgebra such that $Der(A)$ is a  finitely generated free  $A$-module.   The adjoint action of the Lie super algebra $Der (A)$ 
on the complex $J(L^*)^\bullet$ induces an action of $L$ on  $Ber (L^*)$ 
(with the notation of Proposition \ref{Berezinian}). If $D \in Der (A)$, the Lie derivative $L_D$ of $D$ is defined by 
$$\forall \omega \in Ber (L^*), \quad L_D(\omega)=D \cdot \omega.$$
\end{definition}



\begin{example}
If $A=k[X_1, \dots ,X_{m+n}]$ is a polynomial superalgebra with 
even variables $X_1, \dots , X_n$ and odd variables $X_{n+1}, \dots , X_{n+m}$. Denote by 
$\omega \in H^m\left ( J(Der (A)^*)\right )$ the class of 
$\pi \left ( \frac{\partial }{\partial X_1}\right )^*\dots \pi \left ( \frac{\partial }{\partial X_n}\right )^*
\frac{\partial }{\partial X_{n+1}}\dots \frac{\partial }{\partial X_{m+n}}$ in $J\left ( Der (A)^*\right )$.
 If $D=\sum_{i=1}^{n+m}f_i\frac{\partial }{\partial X_i}$, we define the divergence of $D$ by 
 $Div_D=\sum_{i=1}^{n+m}(-1)^{\mid f_i\mid \mid X_i\mid }\dfrac{\partial f_i}{\partial X_i}$
The following assertion is easy to check :
$$\omega \cdot D=-Div (D)\omega.$$
\end{example}

 
 We will now concentrate on the restricted crossed product defined by the restricted coinduced representation. 
 
 \begin{definition}

If $V$ is a $U^\prime ({\mathfrak h})$-module. We define its coinduced representation as 
$U^\prime ({\mathfrak g})$ acting on $Hom_{U^\prime({\mathfrak h})}(U^\prime ({\mathfrak g}),V)$ 
by the transpose of the right multiplication. 

\end{definition}

{\bf Notation:} From now on, we set 
$$\begin{array}{l}{\mathcal A}= Hom_{U^\prime({\mathfrak h})}(U^\prime ({\mathfrak g}),k), \\
\Co_{U^\prime (\mathfrak h)}^{U^\prime (\mathfrak g )}(V)=
Hom_{U^\prime({\mathfrak h})}(U^\prime ({\mathfrak g}),V).
\end{array}$$

${\mathcal A}$ is a $k$-superalgebra and $\Co_{U^\prime (\mathfrak h)}^{U^\prime (\mathfrak g )}(V)$ is a 
${\mathcal A}$-module and a $U^\prime (\g)$-module. 
It is a restricted crossed product (\cite{C4} for example). 
\begin{notation}\label{coordinates on {mathcal A}}
Let $\mathfrak h$ is a subrestricted Lie superalgebra of $\mathfrak g$. Given a supplement  ${\mathfrak p}$ of $\mathfrak h$ in $\mathfrak g$ and a basis $(e_1, \dots e_n, \epsilon_1, \dots , \epsilon_m)$ of it.
Let $(\mu_1, \dots , \mu_n, \zeta_1, \dots , \zeta_m)$ the element of ${\mathcal A}$ defined by $\forall (a_1, \dots ,a_n, \alpha_1, \dots , \alpha_m)\in [0,p-1]^{n(r+1),m}$, 
$$\begin{array}{l}
< e_1^{a_1} \dots e_n^{a_n}\epsilon_1^{\alpha_1}\dots \epsilon_m^{\alpha_m}, \mu_i>=\delta_{0,a_1}\dots 
\delta_{0,a_{i-1}}\delta_{1,a_i}\delta_{0,a_{i+1}}\dots\delta_{a_n,0}\delta_{0,\alpha_{1}}\dots \delta_{0,\alpha_m};\\
< e_1^{a_1} \dots e_n^{a_n}\epsilon_1^{\alpha_1}\dots \epsilon_m^{\alpha_m}, \zeta_s>=
\delta_{0,a_{1}}\dots\delta_{a_n,0}\delta_{0,\alpha_{1}}\dots \delta_{0,\alpha_{s-1}}\delta_{1,\alpha_{s}}
\delta_{0,\alpha_{s+1}}\dots  \delta_{0,\alpha_m};\\
\end{array}$$
If $(V,\pi)$ is a $U^\prime(\h)$-module and $v\in V$, we denote $\hat{v}_\pi$ or  $\hat{v}$ the element of $Hom_{U^\prime (\h)}(U^\prime (\g), V)$ determined by 
$\forall (a_1, \dots ,a_n, \alpha_1, \dots , \alpha_m)\in [0,p-1]^{n,m}$, 
$$< e_1^{a_1} \dots e_n^{a_n}\epsilon_1^{\alpha_1}\dots \epsilon_m^{\alpha_m}, \widehat{v}>=\delta_{0,a_1}\dots 
\delta_{0,a_n}\delta_{0,\alpha_{1}}\dots \delta_{0,\alpha_m}.$$
\end{notation}

We define a map
 ${\mathcal J}_0 :  \Co_{U^\prime (\mathfrak h )}^{U^\prime (\mathfrak g)}(k) \to Hom_k \left ({\mathcal S}^\prime({\mathfrak p}), k \right )$
by for any $\lambda \in \Co_{U^\prime (\mathfrak h )}^{U^\prime (\mathfrak g )}(k)$, 
 $$<X_1^{\alpha_1}\dots X_r^{\alpha_r}, {\mathcal J}_0(\lambda)>= 
 <X_1^{\alpha_1}\dots X_r^{\alpha_r}, \lambda>.$$
 
 
The following proposition is proved in \cite{C1}
 \begin{proposition} Set ${\mathcal A}= \Co_{U^\prime (\mathfrak h)}^{U^\prime (\mathfrak g )}(k)$
 \begin{enumerate}
\item The map ${\mathcal J}_0$ is an isomorphism of superalgebras.
\item  If $X\in {\mathfrak g}$, denote by $\delta_X$ the transpose of the right multiplication on 
 $U^\prime({\mathfrak g})$. Then $\delta_X$ is a derivation of $\mathcal{A}$ and 
 $\delta: {\mathfrak g} \to Der (\mathcal{A})$ ($\delta(X)=\delta_X$) is a morphism of restricted Lie superalgebras. 
 \item The derivations $(\partial_1, \dots , \partial_n, \overline {\partial}_1, \dots , \overline{\partial}_s)$ defined by 
 $$\partial_i(\mu_j)=\delta_{i,j},\quad \partial_i(\zeta_t)=0, \quad  \overline{\partial}_s(\mu_j)=0, 
 \quad \overline{\partial}_s(\zeta_t)=\delta_{s,t}$$ 
 form a basis of the ${\mathcal A}$-module 
 $Der ({\mathcal A})$. 
 \end{enumerate}
\end{proposition}

 \begin{notation}\label{notation delta_X}
 Let $X \in {\mathfrak g}$, it defines a derivation of ${\mathcal A}$ that will be written 
$$\delta_X=\sum_{i=1}^n f_{i}(X)\partial_{i}+ \sum_{k=1}^m g_s(X) \overline{\partial}_s.$$
%The derivation $D$ sends $F_rA$ into itself if and only if for all $j>r$, $a_{i,j}\in  {\mathcal F}_rA$. 


The element $X \in \g$ defines a derivation $\delta_X^\pi$ of $\Co^{U^\prime (\g)}_{ U^\prime (\h)}{(\pi)}$. If 
 $F^\pi (X)$ denotes  the element of $End_A[\Co_{U^\prime (\h)}^{U^\prime (\g)} (\pi)]$ defined by 
$$\forall v \in V, \quad F_X^\pi (v)=X\cdot v.$$
then 
$$\delta_X^\pi=F^\pi (X) + \sum_{i=1}^n  f_{i}(X)\partial_{i}
+\sum_{s=1}^m g_s (X)\overline{\partial_s}.$$

\end{notation}






\begin{proposition}\label {Omega and coinduced}
\begin{enumerate}
\item $\Ber(Der \left ( {\mathcal A})^*\right )$ is endowed with a left ${\mathcal A}$-module and a 
$U^\prime(\g)$-module by the operations :
$$\begin{array}{l}
\forall a \in {\mathcal A}, \quad \forall X \in \g,  \quad \forall \omega \in \Ber(Der \left ( {\mathcal A})^*\right )\\
a\cdot \omega=a\omega\\
X\cdot \omega =L_{\sigma (X)}(\omega).
\end{array}$$
\item The ${\mathcal A}-U^\prime (\g)$-module $\Ber(Der \left ( {\mathcal A})^*\right )$ is isomorphic to 
${\Co}_{U^\prime (\h)}^{U^\prime (\g)} (\Pi^mk_{-strad_{\g/\h}})$. 
\end{enumerate}
\end{proposition}

{ \it Proof of Proposition:} 

For simplicity, we set $\Omega:= Ber\left (Der ({\mathcal A})^* \right )$.
The ${\mathcal A}$-module $\Omega $ is a free  ${\mathcal A}$-module of rank one with basis 
$\omega_{\underline{e}}=\partial_{1}^*\dots \partial_n^*\overline{\partial_{1}}\dots \overline{\partial_s}$. 
Denote by ${\mathfrak a}=\{\lambda \in {\mathcal A}, \quad <1, \lambda>=1\}$ the maximal ideal of ${\mathcal A}$. Consider the map 
\begin{equation}\label{chi_r}
\begin{array}{rcl} 
\chi : \Omega  & \to & \Co_{U^\prime (\h)}^{U^\prime(\g)} (\dfrac{\Omega}{{\mathfrak a}\Omega})
\\
\omega & \mapsto & [X_{1}\dots X_{t} \in U^\prime (\g)\mapsto 
cl(\delta_{X_1}\dots \delta_{X_t} \cdot \omega)]
\end{array}
\end{equation}
It is easy to check that $\chi$ is a morphism of $U^\prime ( \g )$-modules and of ${\mathcal A}$-modules.  

Let us now show that it is an isomorphism. 
It is  an isomorphism modulo ${\mathfrak a}$ as  
$\chi (\omega_{\underline{e}})(1)=\omega_{ \underline{e}} $ mod ${\mathfrak a}$. 
%Thus $\chi (\omega_{\leq r ,\underline{e}}) =a1_{-trad_{\g/\h}}$. 
%As $a(1)=1$, the element $a$ is an invertible element of $A_{\leq r}$. 
Moreover, $\Omega$ and 
 $\Co_{U^\prime(\h)}^{U^\prime ( \g)}(\Omega/{\mathfrak a}\Omega)$ 
 are free ${\mathcal A}$-modules of dimension 1 and $\chi $ sends a basis of $\Omega$ to a basis of 
 $\Co_{U^\prime (\h)}^{U^\prime (\g)}(\Omega/{\mathfrak a}\Omega)$ . 
Let us now compute the $\h$-action on  $\Omega/{\mathfrak a}\Omega$. If $H$ is in $\h$, we set 
$$[H,e_j]=\sum_{i=1}^n ad(H)_{i,j} e_i +\sum_{s=1}^m ad(H)^{s,j}\epsilon_s+ \; mod\; \h.$$
$$\begin{array}{rcl}
Div_{\underline{e}}\delta_{H}\;  mod \; {\mathfrak a}
&=&\sum_{i=1}^n  
<1, \dfrac{\partial}{\partial \eta_{i} }
\delta_{H} (\eta_{i})  + 
(-1)^{\mid H \mid +1}\sum_{s=1}^m \dfrac{\partial}{\partial \zeta_s}\delta_{H}(\zeta_s) >\\
%&= & \sum_{i=1}^n \sum_{j=0}^r 
%<\dfrac{\partial}{\partial \eta_{i,j}} \delta_{H, \leq r } (\eta_{i,j }) ,1>
%+(-1)^{\mid H \mid +1}\sum_{s=1}^m <\dfrac{\partial}{\partial \zeta_s}\delta_{H, \leq r}(\zeta_s) ,1>\\
&= &\sum_{i=1}^n <1,\dfrac{\partial}{\partial \eta_{i}} \delta_{H } (\eta_{i }) >
+(-1)^{\mid H \mid +1}\sum_{s=1}^m <1, \dfrac{\partial}{\partial \zeta_s}\delta_{H}(\zeta_s) >\\
&= &\sum_{i=1}^n 
<e_i, \delta_{H } (\eta_{i }) >- (-1)^{\mid H \mid }\sum_{s=1}^m< \epsilon_s, \delta_{H } (\zeta_s) >\\
&=& \sum_{i=1}^n <e_iH, \eta_{i}>
-(-1)^{\mid H \mid} \sum_{s=1}^m< \epsilon_s H, \zeta_s>\\
&=&-\sum_{i=1}^n ad(H)_{i,i}+\sum_{s=1}^m ad(H)_{s,s}\\
&=&-strad_{{\mathfrak g }/{\mathfrak h}}(H).\Box\\
\end{array}$$





The duality properties obtained above for coinduced representations  hold in the restricted context and can be improved. \\




%By definition of $Trad$, for $H\in \h$, one has 
%$trad_{\g/\h}(H^p)=\left [trad_{\g/\h}(H)\right ]^p$. 
%Moreover, $ad_{\g}(H)^p=ad_{\g}(H^{[p]})$ and $ad_{\h}(H)^p=ad_{\h}(H^{[p]})$ so that 
%$ad_{\g/\h}(H)^p=ad_{\g/\h}(H^{[p]})$.Thus 
%$$ad_{\g/\h}(H^p)=ad_{\g/\h}(H^{[p]}).$$




We set $\Omega=Ber\left ( Der ({\mathcal A})^*\right )$. The choice of basis of a supplement of $\h$ in $\g$ defines coordinates on ${\mathcal A}$ (as in Notation \ref{coordinates on {mathcal A}}) and a basis 
$\omega_{\underline{e}}=
\left ( \dfrac{\partial}{\partial \eta_1}\right )^*\dots \left ( \dfrac{\partial}{\partial \eta_n}\right )^*\dfrac{\partial}{\partial \zeta_1} \dots \dfrac{\partial} {\partial \zeta_m}$ of $\Omega$.
Using Proposition \ref {Omega and coinduced}, we define a map 
$$\Psi : {\Co}_{ U^\prime (\h)}^{U^\prime (\g)}(\pi) \otimes 
{\Co}_{ U^\prime (\h ) }^{U^\prime (\g)} (\pi^* \otimes  Ber ( (\g /\h)^*)\simeq 
{\Co}_{ U^\prime (\h)}^{U^\prime (\g)}(\pi) \otimes\left ({\Co}_{ U^\prime (\h ) }^{U^\prime (\g)} (\pi^*) \otimes_{\mathcal A}  \Omega\right ) \to k$$
$$\Psi \left [  \eta^{\underline{a}} \zeta^{\underline{\alpha}}
\widehat {v}_{\underline{a}, \underline{\alpha}}, 
  \eta^{\underline{b}}\zeta^{\underline{\beta}}
 \widehat{v}^*_{\underline{b}, \underline{\beta}}\otimes \omega_{\underline{e}}\right ]=
 \sum_{a_{i}+b_{i}=p-1, \; \alpha_s+\beta_s=1} (-1)^{Inv(\underline{\alpha}, \underline{\beta})}
 (-1)^{\mid v_{\underline{a}, \underline{\alpha}}\mid \mid \beta \mid }
 <v_{\underline{a}, \underline{\alpha}}, v^*_{\underline{b}, \underline{\beta}}>
$$
where $\underline{a}, \underline{b}\in [0,p-1]^n$, $\underline{\alpha}, \underline{\beta }\in \{0,1\}^m$ and $\zeta^{\underline{\alpha}}\zeta^{\underline{\beta}}=(-1)^{Inv(\underline{\alpha}, \underline{\beta})}\zeta_1 \dots \zeta_m$.

If $\lambda \in \Co_{U^\prime (\h)}^{U^\prime (\g)}(\pi)$ and $\lambda^* \in \Co_{U^\prime (\h)}^{U^\prime (\g)}(\pi^*)$, $\Phi$ can be written as follows :
$$\Phi \left (\lambda, \lambda^*\omega_{\underline{e}}\right )=
(-1)^{\frac{n(n-1)}{2}}\dfrac{1}{(p-1)!^n}
<e_1^{p-1}\dots e_n^{p-1}\epsilon_1\dots \epsilon_m, <\lambda,\lambda^*>>
$$
where $<\lambda,\lambda^*>$ is the element of ${\mathcal A}$ defined by 
$$\forall u \in U(\g), \quad <u, <\lambda, \lambda^*>>=\sum < <u_{(1)},\lambda>, u_{(2)},\lambda^*>>(-1)^{\mid u_{(2)}\mid \mid \lambda \mid}$$.
 
 
 \begin{proposition}\label{definition of Psi}
 The map $\Psi$ defines a non degenerate ${\mathfrak g}$-invariant duality between 
 ${\Co}_{U^\prime ( \h)}^{U^\prime (\g)}(\pi)$ and  
${\Co}_{U^\prime ( \h) }^{U^\prime (\g)} (\pi^* \otimes Ber [(\g /\h)^*] )$. 

\end{proposition}

{\it Proof}

We are in the situation where $Der({\mathcal A})$ is a finite dimensional free 
${\mathcal A}$-module of dimension $n+{\mathfrak e}m$ with basis 
$(\partial _1 , \dots , \partial_n, \overline{\partial_1}, \dots ,\overline{\partial_m})$. 
Thus the Lie derivative is defined on $\Ber(Der \left ( {\mathcal A})^*\right )$.




We need to show that for all $X\in {\mathfrak g}$, 
$L_{\delta_X} \left [ \eta_1^{p-1}\dots \eta_n^{p-1} \zeta_1 \dots \zeta_m \omega_{\underline{e}}\right ]=0$. 
Write 
$$\delta_X=\sum_{i=1}^n f_i(X)\partial_i +\sum_{s=1}^m g_i(X)\overline{\partial_s} .$$

First 

$$X\cdot \omega_{\underline{e}}=L_{\delta_X}(\omega_{\underline{e}})\\
= \left [\sum_{i=1}^n \partial_i (f_i(X)) -(-1)^{\mid X \mid} \sum_{i=1}^n \overline{\partial}_s(f_s(X))\right ]  \omega_{\underline{e}}.$$
 Then 
 
 $$\begin{array}{l}
 X\cdot (\left [ \eta_1^{p-1}\dots \eta_r^{p-1} \omega_{\underline{e}}\right ]\\
 =
\left (  \sum_{i=1}^n (p-1)<e_i, \delta_X(\eta_i)>+\sum_{s=1}^m <\epsilon_s, \delta_X(\zeta_s)>\right )
 \left [ \eta_1^{p-1}\dots \eta_n^{p-1} \zeta_1\dots \zeta_m\omega_{\underline{e}}\right ]\\
+ <1, Div_{\underline{e}} (\delta_X)>
\left [ \eta_1^{p-1}\dots \eta_n^{p-1}\zeta_1\dots \zeta_m \omega_{\underline{e}}\right ]\\
=\left (  \sum_{i=1}^n (p-1)<e_iX, \eta_i>+ \sum_{s=1}^m (<\epsilon_sX, \zeta_s>\right )
\left [ \eta_1^{p-1}\dots \eta_n^{p-1} \zeta_1\dots \zeta_m\omega_{\underline{e}}\right ]\\
+ \left ( \sum_{i=1^n}<1,\partial_i \left (f_i(X)\right )>-
(-1)^{\mid X \mid}\sum_{s=1}^m<1,\overline{\partial}_s \left (g_i(X)\right )>\right )
\left [ \eta_1^{p-1}\dots \eta_n^{p-1} \zeta_1\dots \zeta_m\omega_{\underline{e}}\right ]\\
=\left ( \sum_{i=1}^n (p-1)<e_iX, \eta_i>+< \epsilon_s X, \zeta_s>\right )
\left [ \eta_1^{p-1}\dots \eta_n^{p-1} \zeta_1\dots \zeta_m\omega_{\underline{e}}\right ]\\
+ \left ( \sum_{i=1}^n <1, \partial_i \left (\delta_X(\eta_i )\right )>
-(-1)^{\mid X \mid}\sum_{s=1}^m <1, \overline{\partial}_s \left (\delta_X(\zeta_s )\right )>\right )
\left [ \eta_1^{p-1}\dots \eta_n^{p-1} \zeta_1\dots \zeta_m\omega_{\underline{e}}\right ]\\
= \sum_{i=1}^n \left ( (p-1)<e_iX, \eta_i> + <e_i, \delta_X(\eta_i )>\right )
\left [ \eta_1^{p-1}\dots \eta_n^{p-1} \zeta_1\dots \zeta_m \omega_{\underline{e}}\right ]\\
+ \sum_{s=1}^m (-1)^{\mid X \mid (s-1)}\left ( <\epsilon_s X , \zeta_s>-
(-1)^{\mid X \mid}\sum_{s=1}^m < \epsilon_s, \delta_X(\zeta_s )>\right )
\left [ \eta_1^{p-1}\dots \eta_n^{p-1} \zeta_1\dots \zeta_m \omega_{\underline{e}}\right ]\\
\end{array}$$
Here, we  have used the fact that if $\lambda \in {\mathcal A}$, then 
$$<1, \partial_{i}\lambda >=<e_i , \lambda > \quad \rm{and} \quad 
 <1, \overline{\partial}_{s}\lambda >=<\epsilon_s , \lambda >.$$
 Notice that if $\mid X \mid =1$, a parity argument shows that $< \epsilon_s X, \zeta_s>=0.$ Eventually, we find 
 $$X\cdot (\left [ \eta_1^{p-1}\dots \eta_r^{p-1} \omega_{\underline{e}}\right ]=0.\Box$$
 

In the following theorem, we will see that the restricted induced representation $Ind_{U^\prime(\h)}^{U^\prime(\g)}(\pi \otimes k_{strad_{\g/\h}} )$ is isomorphic to 
$\Co_{U^\prime(\h)}^{U^\prime(\g)} (\pi )$. 


\begin{theorem}\label{Ind and  Coind}
\begin{enumerate}
\item Set $\Lambda=\eta_1\dots \eta_n \zeta_1 \dots \zeta_m$. 
The one dimensional space $k\Lambda$  is endowed with the following $\h$-module structure 
(see Lemma \ref{h-module structure on Lambda_r})  
$$\forall H\in \h, \quad \delta_H(\Lambda )=strad_{\g/\h}(H) \Lambda.$$
\item Let $(\pi, V)$ be a restricted representation of $\h$. 
The map $$\begin{array}{rcl}
\Phi : Ind_{U^\prime (\h)}^{U^\prime (\g)}( \Pi^m k_{stad_{\g/\h}} \otimes \pi)  & \to & \Co^{U^\prime (\g)}_{U^\prime (\h)}(\pi )\\
u\otimes_{U(\h)}(\Lambda  \otimes v) & \mapsto & \delta_{u} (\Lambda \hat{v}_\pi) 
\end{array}$$ is a $U^\prime (\g)$-isomorphism.
\end{enumerate}
\end{theorem}

{\it Proof :} 


The map $$\begin{array}{rcl}
\Phi : Ind_{U^\prime (\h)}^{U^\prime (\g)}(\Pi^mk_{strad_{\g/\h}}\otimes \pi)  & \to & \Co^{U^\prime (\g)}_{U^\prime (\h)}(\pi )\\
u\otimes_{U(\h)}(\Lambda  \otimes v) & \mapsto & \delta_{u}(\Lambda  \hat{v}_\pi) 
\end{array}$$
is well defined (see Lemma \ref{definition of Phi}).  Proceeding as in the proof of lemma \ref{Phi is injective}, one shows that $\Phi$ is injective. 
To show that $\Phi$ is onto, we will make use of the following lemma :

\begin{lemma}
The derivation $\delta^\pi _{e_i}-\dfrac{\partial}{\partial \eta_i}$ can be written 
$$\delta^\pi_{e_i}-\dfrac{\partial}{\partial \eta_i} =\sum_{j=1}^nf_j \dfrac{\partial}{\partial \eta_j}+  
\sum_{s=1}^mg_s \dfrac{\partial}{\partial \zeta_s}+F^\pi(e_i)$$ 
with $f_j$,  $g_s$  belonging to ${\mathfrak a}$  and $F^\pi(e_i)(\widehat{v})\in {\mathfrak a}\Co_{U^\prime(\h)}^{U^\prime (\g)}(V)$. A similar statement holds for 
$\delta_{\epsilon_s}$.
\end{lemma}

{\it Proof of the lemma : }

We know that 
$$\delta_{e_i}=\sum_{j=1}^n\delta_{e_i}(\eta_j) \dfrac{\partial}{\partial \eta_j}+
\sum_{s=1}^m \delta_{e_i}(\zeta_s) \dfrac{\partial}{\partial \zeta_s}.$$ 
But 
$$\begin{array}{l}
<\delta_{e_i}(\eta_j),1>=<\eta_j, e_i>=\delta_{i,j}, \\
<\delta_{e_i}(\zeta_s),1>=<\zeta_s, e_i>=0, \\
<\delta_{\epsilon_s}(\zeta_t),1>=<\zeta_t, \epsilon_s>=\delta_{s,t}, \\
<\delta_{\epsilon_s}(\eta_j),1>=<\eta_j, \epsilon_s>=0, \\
<F^\pi(e_i)(\widehat{v}),1>=<\delta^\pi_{e_i}\widehat{v},1>=<\widehat{v},e_i>=0.\Box
\end{array}$$

Let $f=\eta_1^{a_1}\dots \eta_n^{a_n}\zeta_1^{\alpha_1}\dots \zeta_m^{\alpha_m} \in {\mathcal A}$ be a monomial. Let us show by decreasing induction on $\sum a_i+\sum \alpha_s$ that, for any $v\in V$, 
$f\widehat{v}$ belongs to $Im \Phi$. The maximal degree of $f$ is $n(p-1)+m$ and 
$\eta_1^{p-1}\dots \eta_n^{p-1}\zeta_1\dots \zeta_n \widehat{v}= \Phi (1\otimes (\Lambda \otimes v))$. 

Assume that $f$ is of degree $d-1$ and that the property has been shown for any degree $\geq d$. Two cases can happen:

{\it First case : There exists $i\in [1,n]$} such that $\eta_ifv$ is of degree $d$. Then \linebreak 
$\delta_{e_i}(\eta_i f)v-(a_i+1)f\hat{v}$ has a degree superior or equal to $d$. Thus, par induction hypothesis, 
$\delta_{e_i}(\eta_i f\hat{v})-(a_i+1)f\hat{v} \in Im \Phi$ . But $\eta_if\hat{v} \in Im \Phi$, so does 
$\delta_{e_i}(\eta_i f\hat{v})$ because $\Phi$ is a $U^\prime(\g)$-module morphism. Eventually $f\hat{v}$ is in $Im \Phi$.

{\it Second case : There exists $s\in [1,m]$} such that $\zeta_s f\hat{v}$ is of degree $d$. The argument is analog to the previous case.$\Box$\\


The following proposition establishes a link between  Propositions \ref{definition of Psi} and \ref{Ind and Coind}.
It is well known that the map 
$$\begin{array}{rcl}
\Theta : Coind_{U^\prime (\h)}^{U^\prime (\g)} (\pi^*)&\to &
 {\mathcal I}nd_{U^\prime (\h)}^{U^\prime (\g)} (\pi)^*\\
\lambda & \mapsto & \left [ u \otimes v\mapsto <\lambda(\check{u}),v>\right ]\\
\end{array}$$ is an isomorphism of $U^\prime (\g)$-modules. 



\begin{proposition}\label{comparison of Phi and Psi}
Let $\Phi : Ind_{U^\prime(\h)}^{U^\prime(\g)}(\pi \otimes \Pi^mk_{strad_{\g /\h}}) \to \Co_{U^\prime(\h)}^{U^\prime (\g)}(\pi )$ 
the isomorphism constructed in  Theorem \ref{Ind and Coind} and 
$^{t}\Phi : \Co_{U^\prime(\h)}^{U^\prime(\g)}(\pi )^* \to 
Ind_{U^\prime(\h)}^{U^\prime(\g)}(\pi \otimes \Pi^mk_{strad_{\g /\h}})^*$ its transpose. 
Denote by $\Psi^\natural : \Co_{U^\prime (\h)}^{U^\prime(\g)}(\pi ^*\otimes \Pi^m k_{-strad}) \to \left [ \Co_{U^\prime (\h)}^{U^\prime(\g)}(\pi)\right ]^*$ 
the monomorphism determined by $\Psi
$. We will show that  the equality holds 
$$^t \Phi  \circ  \Psi ^\natural =\Theta.$$ 

\end{proposition}

{\it Proof of Proposition \ref{comparison of Phi and Psi}}:

Let $u \in U(\g)$, $v \in V$, $f\in {\mathcal A}$,

$$\begin{array}{rcl}
\left [ ^t \Phi  \circ  \Psi^\natural (f\widehat{v^*}\widehat{\omega_{\underline e}}) \right ] 
\left (u\otimes_{U^\prime(\h)}( \Lambda \otimes v) \right )
&=&
\Psi \left [f\widehat{v^*}\widehat{\omega_{\underline{e}}}, \delta_{u} (\Lambda  \widehat{v_\pi} ) \right ]\\
&=& \Psi \left [\delta_{\check{u}}(f\widehat{v^*}\widehat{\omega_{\underline{e}}}), \Lambda \widehat{v}_\pi   \right ]
(-1)^{\mid u \mid (\mid f \mid + \mid v^*\mid +m)}\\
%<e_1^{p-1}\dots e_n^{p-1}\epsilon_1\dots \epsilon_m, <
&=&<<1,\delta_{\check{u}}(f\widehat{v^*}\widehat{\omega_{\underline{e}}})>,v>
(-1)^{\mid u \mid (\mid f \mid + \mid v^*\mid +m)}\\
&=&<(f\widehat{v^*}\widehat{\omega_{\underline{e}}})(\check{u}),v>\\
&=&\Theta (f  \widehat{v^*} \widehat{\omega_{\underline{e}}})\left (u\otimes_{U^\prime(\h)} (v\otimes  \Lambda)\right ).
\Box
\end{array}$$





%\section{Hom dual}

%Let ${\mathfrak g}$ be a restricted algebra. Its  restricted enveloping algebra 
%$U^\prime ({\mathfrak g})$ is a Frobenius algebra (\cite{B}). 
%Consequently, $Ext^i_{U^\prime ({\mathfrak g})}(k, U^\prime({\mathfrak g}))$ is zero if $i\neq 0$.
%Let us now compute the right $U^\prime (\g)$-module 
%$Hom_{U^\prime ({\mathfrak g})}(k, U^\prime({\mathfrak g}))$.

%\begin{proposition}
%The right $U^\prime({\mathfrak g})$-module 
%$Hom_{U^\prime ({\mathfrak g})}(k, U^\prime({\mathfrak g}))$ is isomorphic to 
%$k_{strad_{\mathfrak g}}$.
%\end{proposition}

%{\it Proof :} Let $(e_1, \dots ,e_n)$ be a basis of $\g_{\overline{0}}$ and $(\epsilon_1, \dots ,\epsilon_m)$ be a basis of $\g_{\overline{1}}$
%The restricted enveloping algebra is filtered and its graded associated algebra is 
%$\dfrac {S({\mathfrak g})}{<e_1^{p-1}, \dots e_n^{p-1}>}$ Choosing a convenient filtration, one has  ({\color{red} check})
%$$\begin{array}{rcl}
%GrHom_{U^\prime({\mathfrak g})}(k,U^\prime({\mathfrak g}))&=&
%Hom_{GrU^\prime({\mathfrak g})}(k,GrU^\prime({\mathfrak g}))\\
%&=&
%\{ cl(P), P\in S({\mathfrak g})\quad and \quad \forall i\in [1,r],\; cl(e_iP)=0\}\\
%&=&e_1^{p}\dots e_n^{p}\epsilon_1\dots \epsilon_m
% \dfrac{S(\g)}{ <e_1^{p-1}, \dots e_n^{p-1}>}.
% \end{array}$$
 
% Consequently, the $k$-vector space 
 %$GrHom_{U^\prime({\mathfrak g})}\left (k,U^\prime({\mathfrak g})\right )$ is one dimensional and so is 
 %$Hom_{U^\prime({\mathfrak g})}\left (k,U^\prime({\mathfrak g})\right )$. 
 %Right multiplication endows $Hom_{U^\prime({\mathfrak g})}(k,U^\prime({\mathfrak g}))$ with a right $U^\prime ({\mathfrak g})$-structure. This right $U^\prime ({\mathfrak g})$ 
 %structure is given by a character, which we would like to compute. Let us call it $\theta$.
 
% Remark that the map 
% $$\begin{array}{rcl}
% Hom_{U^\prime({\mathfrak g})}(k, U^\prime({\mathfrak g})) & \to & 
 %\{u \in U^\prime({\mathfrak g}), \quad \forall X \in { \mathfrak g}, \;  Xu=0 \}\\
 %\phi & \mapsto&\phi (1)
% \end{array}$$
% is an isomorphism of right $U^\prime(\g)$-modules. 
% Let $\phi \in Hom_{U^\prime({\mathfrak g})}(k, U^\prime({\mathfrak g}))$, set 
 %$\omega=\phi (1)$. Let 
 %$X \in {\mathfrak g}$. One has 
 %$$\theta(X) \omega=\omega X.$$
 %Hence if $\overline{\omega}=\alpha cl(e_1^{p-1}\dots e_n^{p-1})\epsilon_1\dots \epsilon_m$ is the class of 
 %$\omega$ in 
 %$e_1^{p-1}\dots e_n^{p-1}\dfrac{S(\g)}{ <e_1^{p}, \dots e_n^{p}>}$.
 %$$\theta(X) \overline{\omega}=\overline{\omega}X= -[X, \overline{\omega}]=
 %-\left ( (p-1)\sum_{i=1}^n ad(X)_{i,i}+\sum_{j=1}^m ad(X)_{j,j}\right ] e_1^{p-1}\dots e_n^{p-1}\epsilon_1 \dots \epsilon_m$$
 %which proves that $\theta=strad _{\mathfrak g}$
 %$\Box$\\
 
 %\begin{remark}
% $Hom_{U^\prime}(k, U^\prime)$ is the space of left integrals $\int_{U^\prime}^l$.
% \end{remark}
 
% \begin{corollary}
% Let $\pi$ be a restricted representation of $U^\prime({\mathfrak h})$.
 %$$Hom_{U^\prime({\mathfrak g})}\left (Ind_{U^\prime({\mathfrak h})}^{U^\prime({\mathfrak g})}(\pi ), U^\prime({\mathfrak g})\right )=
% Ind_{U^\prime({\mathfrak h})}^{U^\prime({\mathfrak g})}(\pi ^*\otimes k_{-strad_{\mathfrak g}})$$
% \end{corollary}
 
 %Such a corollary was obtained for enveloping algebras  (\cite{C3}). The proof  of the corollary is the same as in the enveloping algebra case. 

\begin{thebibliography}{foo}

%\bibitem{B}
%A. Berkson,{\it The u-algebra of a restricted Lie  algebra is Frobenius}, Proc. Amer. Math. Soc. {\bf 15} (1964)

\bibitem{C0}S. Chemla, {\it Propriétés de dualité dans les représentations coïnduites de superalgèbres de Lie},  Thèse de doctorat Paris 7, (1990).
\bibitem{C1}S. Chemla, {\it Propriétés de dualité dans les représentations coïnduites de superalgèbres de Lie},  Annales de l'Institut Fourier, {\bf 44}, (1994), 1067-1090.
\bibitem{C4} S. Chemla {\it Cohomology locale de Grothendieck et représentations induites de superalgèbres de Lie}, Mathematische Annalen, {\bf 297}, p.371-382 (1994).
\bibitem{C2} S. Chemla, {\it Poincare duality for k-A Lie superalgebras}, Bulletin de la Soci\'et\'e Math\'ematique de France, {\bf 122}, p 371-397 (1994). 
%\bibitem{C3} S. Chemla , {\it Operations for modules on Lie Rinehart superalgebras}, Manuscripta Mathematica, {\bf 87}, p 199-223 (1995)
\bibitem{C4} S. Chemla {\it Frobenius and quasi-Frobenius  left Hopf algebroids}, Preprint.


%\bibitem{Bo} Borel, {\it Algebraic D-modules}

\bibitem{B-B} W.Borho and J-L. Brylinski, {\it Differential operators on homogeneous spaces I}, Invent. Maths {\bf 69} (1982), 437-476.

%\bibitem{BYZ}
%Y.-H Bao, Y. Ye, J.J Zhang, \emph{Restricted Poisson algebras}, Pacific Journal of Maths., 
%{\bf 289} (2017), 1-34. 

\bibitem{Di} J. Dixmier, {\it Enveloping algebras}, Graduate studies in mathematics, {\bf Vol 11}, American mathematical Society. 

\bibitem{Du} M. Duflo, {\it Sur les idéaux induits dans les algèbres enveloppantes}, Invent. Math. {\bf 67} (1982), 385-393.

\bibitem{Hartshorne} R. Hartshorne, {\it Local cohomology}, Lecture notes in mathematics {\bf 41} (1967)

\bibitem{Hochschild} G. Hochschild, \emph{Simple 	algebras with purely inseparable 
splitting fields of exponent 1}, Transactions of the American Mathematical Society {\bf 79} (1955), 
477-489. 

%\bibitem{Huneke} C. Huneke, {\it Lectures on local cohomology (with an appendix by Amelia Taylor)}, Cont. Math. 436 (2007), 51-100.

\bibitem{Hu}
J. Huebschmann, \emph{Poisson cohomology and the quantization}, J. reine angew. Math. (1990), 57-113.

\bibitem{J} N. Jacobson, {\it Lie algebras}, Wiley-Interscience, New York, 1962.

\bibitem{Leites} D. Leites, {\it Introduction to the theory of supermanifolds}, Uspeki. Mat. Nauk, {\bf 35}, 1980.

\bibitem{L} T. Levasseur, {\it Critères d'induction et de coinduction pour certains anneaux d'opérateurs différentiels}, J. of Algebra, {\bf 110}, Issue 2, 530-562.

\bibitem{Manin} Y. Manin, {\it Gauge field theory and complex geometry}, A series of comprehensive studies in mathematics, 1988.

\bibitem{P} V. Petrodradski, {\it Identities in the enveloping algebras for modular Lie superalgebras }, 
Journal of algebra, 145 (1) (1992).

\bibitem{Rinehart} G.S Rinehart, {\it Differential forms on general commutative algebra}, Trans. Amer. Math. Soc. {\bf 115}, No2,  261-277 (1962).

 \bibitem{Rumynin}
D. Rumynin, \emph{Duality for Hopf algebroids},
  Journal of Algebra \textbf{223 (1)} (2000), 237--255.
\end{thebibliography}

\vspace{1em}
Sophie Chemla

Sorbonne Université and  Université  Paris Cité, CNRS, IMJ-PRG, F-75005 Paris

Email address: sophie.chemla@sorbonne-universite.fr




\end{document}