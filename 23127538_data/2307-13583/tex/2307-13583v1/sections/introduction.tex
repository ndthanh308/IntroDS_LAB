\section{Introduction} \label{sec:introduction}
% https://mathworld.wolfram.com/InhomogeneousLinearOrdinaryDifferentialEquationwithConstantCoefficients.html
In physics, many systems are described using linear inhomogeneous differential equations. In the 1820s George Green developed tools to deal with such problems \cite{green1889essay, duffy2015green}.
Unfortunately, his work remained largely undiscovered during his lifetime, but was luckily rediscovered by William Thomson (later Lord Kelvin) \cite{challis2003green}. The mathematical methods Green proposed are used throughout physics now and were a major component in the discovery of quantum field theory. Julian Schwinger, who shared the 1965 Nobel prize in physics with Sin-Itiro Tomonaga and Richard P. Feynman \cite{Nobel1965}, actually acknowledged Green's contribution to his work in \cite{schwinger1993greening}. %The function Green introduced was coined ``Green's funciton'' by Bernhard Riemann \cite{kline1990mathematical}.

Since then, the Green's function technique has become one of the most important tools in many-body theories \cite{szabo2012modern}.
The method allows one to calculate many properties of a system, for example: excitation and ionization energies, ground-state energies, transition matrix elements, absorption coefficients, and dynamical polarizabilities, as well as elastic and inelastic electron cross sections \cite{onida2002electronic}. In fact, self-consistent perturbation theories can be formulated in terms of the Green’s function \cite{onida2002electronic}. More details on Green's function theory may be found in standard textbooks on many-body theory \cite{farid1999electron}.

In this paper, we restrict our work to describing how to obtain the single-particle many-body Green's function using a quantum computer. There have been several different proposals utilising quantum devices to calculate the matrix elements of the Green's function. Some focus on variational approaches \cite{rungger2019dynamical, cai2020quantum, endo2020calculation, chen2021variational, jamet2021krylov, sakurai2022hybrid, zhu2022calculating} and other near-term methods \cite{steckmann2021simulating}. Alternatively, some proposals utilize quantum phase estimation \cite{bauer2016hybrid, kosugi2020construction}. We follow a similar approach to Tong \textit{et al.} in \cite{tong2021fast}, where the quantum singular value transform (QSVT) algorithm is used \cite{gilyen2019quantum}. However, we do not apply their fast inversion strategy, that can reduce the query complexity to the block encoding by preconditioning the linear problem. At a high level, this method assumes the matrix to block encode $H$ can be split as $H = A + B$, under the assumption the spectral norm of $A$ is much greater than $B$: $\| A\| >> \| B\|$. By preconditioning the problem using $A^{-1}$, the new query complexity depends on $\| B\|$ rather than $\| H\|$, thereby reducing the overall cost \cite{tong2021fast}. 

% d ˆA can be efficiently unitarily diagonalized as in
% This was because we did not perform circuit sampling and instead did a full classical unitary simulation of the quantum circuits and classically used projectors on the resulting matrices to deterministically project the state into the required subspace. % As this algorithm will require a fault tolerant quantum computer, amplitude amplification will likely be used to boost the probability of success. % \cite{rungger2019dynamical, cai2020quantum, chen2021variational, jamet2021krylov,  kosugi2020construction, sakurai2022hybrid, zhu2022calculating, steckmann2021simulating}. 


We tested our approach on the two-site single-impurity Anderson model (SIAM) \cite{anderson1961localized}, which is a four-qubit problem. Despite its simplicity, it captures some very interesting physics, for example it can be used to approximate the Mott insulator phase transition \cite{potthoff2001two}. We calculate the single-particle Green's function for this system, via the QSVT, and use the results to plot this phase transition. We show how approximating the inverse function can lead to singular values not being inverted properly, which causes errors in a given calculation - due to the matrix not being inverted properly. We then discuss possible ways to mitigate against this.

The outline of this paper is as follows. Section \ref{sec:background} introduces all the necessary background material for the paper. The single particle Green's function, the QSVT algorithm and the single-impurity Anderson model are reviewed. Our numerical study is then presented in Section  \ref{sec:discussion}. Finally, in Section \ref{sec:LCU_circuits} we compare the presented block encoding strategy to prior works. 
