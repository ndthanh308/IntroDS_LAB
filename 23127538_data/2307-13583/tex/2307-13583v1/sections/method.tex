\section{Method} \label{sec:method}
We numerically investigated the performance of calculating the Green's fucntion for the two-site Anderson model via the QSVT algorithm. To build the qubit Hamiltonian,  Quantinuum's InQuanto package was utilized \cite{inquanto, inquanto_prod}. The circuits required to perform QSVT were then constructed using PyTket \cite{sivarajah2020t}. Importantly, we only built the QSVT circuit to perform matrix inversion.  First, we generated the QSP phase angles in the open-source python library QSPPACK \cite{QSPPACK}. The phases $\vec{\phi}$ obtained (for the different polynomial approximations of the inverse function -
$k=10$ and $k=50$) are supplied in the Supporting Material. Next, for each $z$, we built two quantum circuits that performed $T \approx \big( z - [H-E_{0}] \big)^{-1}$ and $W \approx \big(z + [H-E_{0}] \big)^{-1}$ via the quantum singular value transform algorithm - see Figure \ref{fig:QSVT}. This required the block-encoding circuits for $(\|B_{(e)}\|_{1} ,3,0)$ and $(\|C_{(h)}\|_{1} ,3,0)$, except for the SIAM Hamiltonian defined for $U=8$ and $V=0$ where a $(\|B_{(e)}\|_{1} ,1,0)$ and $(\|C_{(h)}\|_{1} ,1,0)$ was used, due to certain Pauli operators having a coefficient of zero. Each block encoding was constructed according to the template in Figure \ref{fig:block_encode_circ}. We note here, that QSP only approximates the true inverse function via a polynomial, hence the approximately equal use. In all instances, the complex part of $z= \omega + i\delta$ was fixed to be $\delta = 0.1$. This was chosen to ensure all the singular values of each block encoded matrix were above $0.02$.

After each quantum circuit was built, a noise-free classical simulation was performed giving the unitary of the whole QSVT cirucit for each $z$ value. We post-select into the correct block of the unitary (see equation \ref{eq:QSVT_eq}) to obtain the transformed matrix. For each pair of quantum circuits, we denote these post-selected matrices $T$ and $W$.  We then classically determined $G_{ij}(z)$   by evaluating $\bra{\Psi_{0}} a_{i} T a_{j}^{\dagger} \ket{\Psi_{0}}$ and $\bra{\Psi_{0}} a_{j}^{\dagger} W  a_{i}\ket{\Psi_{0}}$ (equation \ref{eq:G_plus} and \ref{eq:G_minus}) for all $i,j$, where $i$ and $j$ run over all qubit indices using the standard linear algebra python libraries \cite{Numpy, SciPy}. The ground state $\ket{\Psi_{0}}$ used in each calculation was obtained by diagonalizing $H$ on a classical computer for particular $(U,V)$ parameterizations. For each QSVT simulation, we also calculated the exact classical solution, where the Green's function was calculated via matrix inversion performed on classical hardware.
