% Please make sure you insert your data according to the instructions in PoSauthmanual.pdf
\documentclass[a4paper,11pt]{article}

\usepackage{pos}
\usepackage{lipsum} %package to generate placeholder text in the following

%package added by the author
\usepackage{tabto}
\usepackage{wrapfig}
\usepackage{graphicx}
\usepackage{amsmath}
\usepackage[noabbrev,capitalize]{cleveref}
\usepackage[tableposition=top]{caption}
\usepackage{subcaption}
\usepackage{adjustbox}
\usepackage{float}
\floatstyle{plaintop}
\restylefloat{table}



%\usepackage{lineno}
%\linenumbers


\title{Performance studies on new 4" photomultiplier types intended for IceCube-Gen2 optical modules}

\ShortTitle{IceCube-Gen2 PMT performance}

% Don't change:
%\author{The IceCube Collaboration \\{\normalsize 
\author{The IceCube-Gen2 Collaboration \\{\normalsize 
\normalfont(a complete list of authors can be found at the end of the proceedings)}\\}

% Your emails:
\emailAdd{markus.dittmer@uni-muenster.de}
\emailAdd{alexander.kappes@uni-muenster.de}

\abstract{

% start of abstract
In the upcoming IceCube-Gen2 extension, the newly developed optical modules will include \linebreak[3]4--inch PMTs. For this purpose, the manufacturers Hamamatsu and North Night Vision Technology have developed new PMT models to meet the requirements of the IceCube-Gen2 science case. The specifications include strict requirements on temporal resolution, detection efficiency, and dark noise. We summarize the efforts to measure these performance characteristics and show that both PMT models meet the performance specifications set by IceCube-Gen2. Prototype optical modules based on both PMT models will be deployed with the IceCube Upgrade in order to test them in situ and help decide on a vendor for the Gen2 extension.
% end of abstract

\vspace{4mm}
{\bfseries Corresponding authors:}
Markus Dittmer$^{1*}$, Alexander Kappes$^{1}$\\
{$^{1}$ \itshape University of Muenster}\\[4mm]
$^*$ Presenter

\ConferenceLogo{PoS_ICRC2023_logo.pdf}

\FullConference{The 38th International Cosmic Ray Conference (ICRC2023)\\ 26 July -- 3 August, 2023\\ Nagoya, Japan}
}



\begin{document}

\maketitle






\section{Introduction}\label{sec:intro}
\noindent
Photomultiplier-Tubes (PMTs) are the central component in Cherenkov telescopes such as BAIKAL-GVD \cite{BAIKAL}, KM3NeT \cite{KM3Net} and IceCube \cite{Aartsen2017}. In the planned extension of IceCube-Gen2 \cite{Aartsen2021}, 4" PMTs will be integrated into the optical modules to be embedded in the Antarctic ice. Specifically for this purposes, the proposed PMT models R16293-01-Y001 (hereafter referred to as BB) and N2041 (hereafter referred to as PO) were developed by Hamamatsu Photonics K.K (HPK) and North Night Vision Technology (NNVT), respectively. This study focuses on the third and final iteration of the PMT development process, excluding the results from earlier prototyping stages.
The performance of the utilized PMTs is a major contributor to the efficiency of the entire detector. Hence, strict requirements for various PMT characteristics have been set by the IceCube collaboration. The aim of the work described here is to verify that these specifications are met by conducting comparable measurements for both PMT models.
This work is structured into temperature dependent measurements (\cref{sec:temp}), the PMT quantum efficiency (\cref{sec:qe}) and homogeneity studies (\cref{sec:scan}).


\section{Temperature dependence}\label{sec:temp} %Performance with respect to temperature dependence

\begin{wrapfigure}{r}{0.45\textwidth}
\vspace{-20pt}
  \begin{center}
    % Figure removed
  \end{center}
  \vspace{-20pt}
  \caption{Experimental setup for the measurement of PMT pulse parameters. Taken from \cite{MartinDoctorialThesis}.}
  \label{fig:TempSetup}
\end{wrapfigure}

\noindent
%Since the optical modules of IceCube are embedded in ice with a temperature ranging from -20\,°C to \linebreak -35\,°C \cite{Price2002}, it is crucial to examine the low temperature performance of the PMTs and its temperature dependency. 
%since the performance of a PMT can vary depending on the environmental temperature.
Since the optical modules of IceCube-Gen2 will be embedded in ice at different depths, with ambient temperatures ranging from $-8$\,°C to \linebreak $-40$\,°C \cite{Price2002}, it is crucial to examine the low temperature performance of the PMTs and its temperature dependency.

The measurements of this section were conducted using the setup depicted in \cref{fig:TempSetup}. It consists of a PMT %illuminated by a fibre equipped with a diffuser at 385\,nm \cite{Rongen2018}
illuminated by a LED at 385nm \cite{Rongen2018} through a fiber equipped with a diffuser
which is placed inside a light-tight enclosure. The enclosure is further housed inside a climate chamber which is ramped in 10\,°C steps from $-50$\,°C -- 20\,°C with the measurements being performed for several hours at each temperature. %The PMTs are driven by a passive base and operated with supply voltages of 700\,V -- 1000\,V, depending on the PMT gain. 
The PMT response is captured by an oscilloscope and the maximum amplitude, time of this amplitude and charge within a specified window are recorded for each waveform. For a more comprehensive understanding of the data acquisition, the reader is referred to \cite[pp.~50-52]{MartinDoctorialThesis}. With this setup, three different measurements were conducted for three Hamamatsu (BB9780, BB9786, BB9789) and three NNVT PMTs (PO4049, PO4052 PO4068) PMTs. % whereas two PMTs are measured simultaneously to save time.



\subsection{Gain calibration}
\noindent
The first measurement with this setup aims at determining the supply voltage for which an amplification of $5 \cdot 10^6$ is achieved. This amplification level is referred to as \textit{nominal gain} and the corresponding supply voltage (dynode voltage ratio as recommended by the manufacturer) is known as \textit{nominal voltage}. %for which the dynode voltage ratio recommended by the manufacturer is applied.
%referred to as, pulses around the main response time of the PMT are measured % whereas the waveform sampling length is 150\,ns in respect to the LED pulse trigger. 
For this, the supply voltage is varied at each temperature. The charge spectrum measured for a given voltage is fit by the Single Photoelectron (SPE) formula as described \linebreak[3]in \cite[p.~45]{MartinDoctorialThesis}. An example of the SPE fit is shown in \cref{fig:Gain} (left).

% Figure environment removed





\begin{wrapfigure}{r}{0.45\textwidth}
\vspace{-20pt}
  \begin{center}
    % Figure removed
  \end{center}
  \vspace{-20pt}
  \caption{Nominal voltages plotted against operation temperature. NNVT PMTs are plotted in red and Hamamatsu PMTs in blue colors.}
  \label{fig:NominalVoltage}
  \vspace{-10pt}
\end{wrapfigure}

\noindent
From this fit, the gain can be obtained from the charge of a single photoelectron ($Q_1/e$) amongst other parameters. The nominal voltage for each temperature of each PMT is obtained by fitting a power law of the form $c V^\beta$ \cite[p.~206]{Wright2017} which is shown in \cref{fig:Gain} (right) whereas the corresponding nominal voltages for all PMTs are presented in \cref{fig:NominalVoltage}. 


Decreasing the temperature by $\Delta\text{T}=70\,$°C results in an average gain increase of $(29.1 \pm 0.4)\,\%$ for Hamamatsu and $(13.6 \pm 0.3)\,\%$ for NNVT PMTs. %reason for difference?
This temperature dependence of the gain is most likely attributed to an increase of the secondary emission coefficient of the dynodes. The obtained nominal voltages are set in the subsequent measurements.  
Furthermore, it is observed that the Hamamatsu PMTs exhibit a smaller spread in nominal voltage compared to the NNVT PMTs which on the other hand need a lower supply voltage to achieve the same gain. Although the sample size within this study is limited, manufacturer data of large production batches support this claim.% this claim is supported by data provided by the manufacturer. 


%(standard deviation 1.07\,V vs 45.72\,V) %This is however still expected since NNVT is still in the process of optimizing their manufacturing process as they started developing PMTs only quite recently.

%From this measurement, the rise time, fall time, pulse width, peak to valley ratio, SPE resolution, and light intensity can be obtained as well.




\subsection{Dark rate}
\noindent
The dark rate refers to the intrinsic background noise characterized by counting pulse rates in the absence of any illumination. Since background pulses appear randomly in time, long waveforms of 1\,ms length were measured. From the number of pulses present in a given waveform one can calculate the dark rate which is done for more than $10^7$ waveforms for each temperature step. To avoid triggering on baseline noise, a threshold of 3\,mV ($\sim 0.47$\,PE) was set%meaning that this is the lowest value that a PMT pulse may be recorded with
. Consequently, low amplitude pulses are rejected. Since the charge is recorded and the nominal gain is known one can express this threshold in units of PE (charge$ / Q_1$) which is shown in \cref{fig:DR} (left). 

% Figure environment removed

\noindent
In addition to this amplitude threshold, a charge cut of 0.2\,PE is set to further reject baseline noise. Note that a threshold--free measurement will produce higher dark rates than shown here. %However, the aim is the have a consistent comparison between all PMT within the specifications of the collaboration.





\begin{wrapfigure}{r}{0.45\textwidth}
\vspace{-20pt}
  \begin{center}
    % Figure removed
  \end{center}
  \vspace{-20pt}
  \caption{Mean dark rates of both vendors plotted against temperature.}
  \label{fig:Darkrate}
  %\vspace{-20pt}
\end{wrapfigure}


A histogram based on the time difference between two successive pulses is shown in \cref{fig:DR} (right) where one can distinguish between correlated pulses caused by scintillation induced by radioactive decays in the glass bulb, and uncorrelated pulses, random pulses fitted by an exponential decay. For a more in--depth description, the reader is referred \linebreak[3]to \cite[pp.~108-112]{MartinDoctorialThesis}.

\cref{fig:Darkrate} displays the mean total dark rate of all PMTs from both vendors. Hamamatsu PMTs exhibit somewhat lower dark rates than NNVT PMTs at temperatures below 0\,°C. This difference is attributed to the reduced presence of correlated noise in Hamamatsu PMTs, indicating less scintillation induced by radioactive decays from the enveloping glass bulb. However, at temperatures above 0\,°C, the dark rates of Hamamatsu PMT increase significantly due to thermionic emission.
%Std from 3 PMTs below 0°C
%Hamamatsu: 4.478811848320826 NNVT: 20.20176726887706
%-30 and 20: 2.0687667437730703  15.727930657351164
%at -30 1.7988237455684983 13.519609664043763



\subsection{Timing resolution}
\noindent
To determine the timing characteristics, low light illumination (pulse occupancy less than 10\%; <5\,\% multi-PE pulses) is used such that mostly SPE pulses contribute -- otherwise the characteristics would yield different values in case multiple photons are detected simultaneously.  The timing resolution, known as Transit Time Spread (TTS), is calculated from 10-\textmu s-long waveforms. 


% Figure environment removed


For the TTS analysis, the relative pulse times with respect to the light source trigger are plotted in a histogram as shown in \cref{fig:TTS} (left). The distribution of the histogram deviates from a Gaussian shape, particularly in the right \textit{tail} region which is a consequence of full photocathode illumination in conjunction with the asymmetric curvature of the first dynode. While this is observed for all PMTs it is more pronounced in the models studied in this work. Hence, instead of fitting a Gaussian distribution to the peak, the numerical standard deviation of the entire main peak distribution (-8\,ns, 17\,ns) is used (\cref{fig:TTS} (right)). %The results for the TTS are presented in \cref{fig:TTS} (right). 
Overall, NNVT PMTs feature on average about ($18.6 \pm 0.1$)\,\% better time resolution compared to Hamamatsu PMTs.
%Overall, NNVT PMTs feature a better time resolution than Hamamatsu PMTs with an average TTS decrease of ($18.62 \pm 0.09$)\,\%. 
PMT BB9780 shows an increase of 3.5\,\% in TTS towards lower temperatures whilst PO4052 demonstrates an 8\,\% decrease. The other PMTs perform essentially constant (<1.5\,\% relative deviations).  %NNVT 0.0602 HPK 0.1327
Though statistics is low Hamamatsu PMTs tend to have stronger variations in TTS, with comparatively twice the overall standard deviation.


\section{Quantum efficiency}\label{sec:qe}


\begin{wrapfigure}{r}{0.4\textwidth}
\vspace{-20pt}
  \begin{center}
    % Figure removed
  \end{center}
  \vspace{-20pt}
  \caption{Experimental setup to measure the quantum efficiency. Taken from \cite{MartinDoctorialThesis}.}
  \label{fig:QEsetup}
  %\vspace{-20pt}
\end{wrapfigure}


\noindent
The quantum efficiency (QE) is a crucial PMT characteristic. It represents the probability that a photoelectron is emitted by a photon striking the photocathode, and is wavelength dependent. For this, the setup shown in \cref{fig:QEsetup} is used.
The wavelength $\lambda$ of high intensity light (emitted from a Xenon lamp) is selected by a monochromator in the range of 250\,nm -- 700\,nm in increments of 10\,nm. %which is guided into a light-tight enclosure. 
The beam is directed either to a calibrated photodiode (PHD in \cref{fig:QEsetup}) or a PMT. The PMT is attached to a base that shortcuts all dynodes, allowing collection of each photoelectron striking the multiplier system without multiplication. 
%Since the beam is divergent, either the entire photocathode ($\sim$ 80\,mm) or a small spotlight ($\sim$ 3\,mm) can be illuminated depending on the distance. 
Since the beam is divergent, its spot size on the PMT can be adjusted to illuminate the entire photocathode by positioning the PMT further from the iris.
%For more information, the reader is directed to \cite[p.~30]{MarkusMasterThesis}.
The respective current $I$ as well as the dark current $DC$ are measured using a picoamperemeter. The quantum efficiency of the PMT is calculated using the equation
\begin{equation*}
    QE(\lambda) = \frac{I_{\text{PMT}}(\lambda)-DC_{\text{PMT}}}{I_{\text{Diode}}(\lambda)-DC_{\text{Diode}}} \cdot QE_{\text{Diode}}(\lambda).
\end{equation*}



% Figure environment removed

\noindent
The results are shown in \cref{fig:QE}. 
%NNVT PMTs feature a higher maximum quantum efficiency at wavelengths around 360\,nm (($31.1 \pm 0.1$)\% vs. ($25.8 \pm 0.1$)\%) whereas Hamamatsu PMTs exhibit a slightly broader spectrum. %and can feature \textit{high QE} as well for extra costs. 
NNVT PMTs feature a higher maximum quantum efficiency at wavelengths around 360\,nm (($31.1 \pm 0.1$)\%) than Hamamatsu PMTs (($25.8 \pm 0.1$)\%) which exhibit a slightly broader spectrum.
Both PMT models feature a similar cutoff at around 270\,nm since both PMT corpora consist of borosilicate glass. The same holds true for the pressure vessel of IceCube-Gen2s optical module as indicated by the black line in the right plot where a comparable PMT quantum efficiency is desired. Hamamatsu PMTs tend to show slightly higher quantum efficiencies near this cutoff which may be attributed to thinner glass in front of the photocathode area.




\section{Homogeneity scan}\label{sec:scan} %Performance with respect to photocathode position
\noindent
Depending on the incident position of the photon on the photocathode, the aforementioned characteristics may change due to different path lengths of photoelectrons inside of a PMT.% due to the complex shaped electric potential. 


\begin{wrapfigure}{r}{0.45\textwidth}
\vspace{-20pt}
  \begin{center}
    % Figure removed
  \end{center}
  \vspace{-20pt}
  \caption{Experimental setup to scan the photocathode. Taken from \cite{MartinScan}.}
  \label{fig:Scansetup}
  \vspace{-20pt}
\end{wrapfigure}

Instead of using diffuse illumination, light from a collimated fiber is used to scan the photocathode surface (\cref{fig:Scansetup}). %This provides valuable information on the spatial dependence of these parameters and helps to assess the homogeneity of the PMT’s performance.
The PMT is located inside a Helmholtz-cube which is set to compensate Earth's magnetic field, nullifying its influence on the measurements, whereas the data acquisition is analogous to \cref{sec:temp}. For additional information, the reader is referred to \cite{MartinScan}.

In the following, the center region is defined for the area with $r<30$\,mm and the edge region for the area with $r>45$\,mm.
The resulting scan of the transit time relative to the center region and the absolute gain is shown in \cref{fig:ScanMaps}. It is apparent that the performance deteriorates towards the edges which is averaged over in the diffuse light measurements as presented in \cref{sec:temp}. The asymmetric distribution of the transit time values arises from the first dynode being curved in the positive y-direction in this representation.





% Figure environment removed

While the gain values demonstrate relatively homogeneous behavior across the radial distance,  the transit time values exhibit a noticeable slope along the y-direction.
%Glass smudges on NNVT...surface not as smooth, more reflections...
In order to compare both PMT models, instead of heatmaps as presented in \cref{fig:ScanMaps}, the results in \cref{fig:ScanCompare} are shown as scatter plots as a function of radial distance to the photocathode center.


% Figure environment removed
\noindent Comparing both data sets, the mean and standard deviations for the overall transit time are \linebreak ($1.75 \pm 2.48$)\,ns for NNVT and ($2.10 \pm 2.66$)\,ns for Hamamatsu PMTs whereas the edge regions exhibit values of ($5.37 \pm 3.46$)\,ns and ($5.93 \pm 3.07$)\,ns respectively, while NNVT PMTs show slightly higher maximum deviations but a more uniform distribution in the center. %Even though Hamamatsu PMTs are operated at a higher nominal voltage, 
Thus, NNVT PMTs demonstrate a more uniform timing response in the center but worse towards the edges. 
Contrarily, regarding the gain, the standard deviations relative to the mean in the center are 3.7\,\% (NNVT) and 3.1\,\% (Hamamatsu), whereas at the edges, they are 20.5\,\% (NNVT) and 19.0\,\% (Hamamatsu). %Consequently, Hamamatsu PMTs exhibit approximately 6\,\% better homogeneity in the center region and an 18\,\% lower deviation near the edges. 
Smoothed curves reveal a drop of 20\% in gain at approximately $0.93 \cdot r$ for both models and a slight increase in gain is observed around 40\,mm. %which may stem from internal reflections.
In both regards the investigated 4-inch PMTs exhibit similar performance and are within the expectations. However, it is important to note that only a single PMT from each vendor was measured.



\section{Conclusion}\label{sec:final}
\noindent
Considering the importance of the PMTs' outer dimensions for module construction (refer to \cite{Yuya:2023icrc}), the performance requirements have a certain degree of flexibility.
The metrics presented in this work are listed in \cref{tab:final}, and both manufacturers fulfill the specifications set by the collaboration. In summary, NNVT PMTs operate at lower nominal voltages and exhibit higher dark rates due to correlated noise, with larger PMT-to-PMT fluctuations compared to Hamamatsu PMTs. NNVT PMTs offer better timing resolution and higher quantum efficiency, though with a slightly higher cutoff. Whilst NNVT PMTs are slightly more homogeneous in the PMT center, the PMT response exhibits more variation towards the edges. Although this study had a limited sample size and more PMTs need to be measured, the results suggest that the tested 4-inch NNVT PMTs performed slightly better compared to Hamamatsu PMTs.
To conclude, both vendors are viable options for the optical modules of IceCube-Gen2, which require production of over 160,000 PMTs. %Hamamatsu dark rates high with mass production melting furnace. Both struggle with the enorm order volume.



%I had to kick out SPE resolution, Peak-To-Valley ratio, fall time, rise time, pulse type probabilities (all measure temperature dependant) since it would be too crowded
\begin{table}[!htb]
\centering
\adjustbox{max width=\textwidth}{%
\begin{tabular}{l|ccccccc}
& Req.  & BB9780 & BB9786 & BB9789 & PO4049 & PO4052 & PO4052 \\
\hline
Nominal  Voltage (V)        & <\,1500    &  994  &  996  &  995  &  741  &  728  &  828  \\
%TTS (ns)       &  3.4  &  3.37  &  3.49  &  3.18  &  2.84  &  2.78 & 2.85 \\ %numerical ... requirement stated in FWHM
TTS (ns)       &  <\,3.4  &  2.51  &  2.58  &  2.38  &  1.75  &  1.86 & 1.75 \\ %From gaussian peak fit
%Rise Time ns)        &  <\,5  &    &    &    &    &    \\
%Peak to Valley ratio    &  >\,2   &    &    &    &    &    \\
Max QE (\%)       &  >\,25  &  25.95  &   26.24 &  25.29  &  29.85  &  30.15 &  33.24 \\
%Pre/late/after (\%) & <\,1/5/15 &    &    &    &    &    &   
\end{tabular}}
\caption{PMT requirements considered in this work. The TTS is defined as 8\,ns (FWHM), and the values stated here represent the standard deviations of a Gaussian fit to the peak of the main peak timing histogram.}
\label{tab:final}
\end{table}














% Bibtex references:
\bibliographystyle{ICRC}
\bibliography{references}



\clearpage



%The following list of authors, affiliations and funding agencies will be updated at the day of submission. The following template is a placeholder generated via https://authorlist.icecube.wisc.edu/icecube on March 25, 2023 and will be updated.
%\section*{Full Author List: IceCube Collaboration}

\scriptsize
\noindent
R. Abbasi$^{17}$,
M. Ackermann$^{63}$,
J. Adams$^{18}$,
S. K. Agarwalla$^{40,\: 64}$,
J. A. Aguilar$^{12}$,
M. Ahlers$^{22}$,
J.M. Alameddine$^{23}$,
N. M. Amin$^{44}$,
K. Andeen$^{42}$,
G. Anton$^{26}$,
C. Arg{\"u}elles$^{14}$,
Y. Ashida$^{53}$,
S. Athanasiadou$^{63}$,
S. N. Axani$^{44}$,
X. Bai$^{50}$,
A. Balagopal V.$^{40}$,
M. Baricevic$^{40}$,
S. W. Barwick$^{30}$,
V. Basu$^{40}$,
R. Bay$^{8}$,
J. J. Beatty$^{20,\: 21}$,
J. Becker Tjus$^{11,\: 65}$,
J. Beise$^{61}$,
C. Bellenghi$^{27}$,
C. Benning$^{1}$,
S. BenZvi$^{52}$,
D. Berley$^{19}$,
E. Bernardini$^{48}$,
D. Z. Besson$^{36}$,
E. Blaufuss$^{19}$,
S. Blot$^{63}$,
F. Bontempo$^{31}$,
J. Y. Book$^{14}$,
C. Boscolo Meneguolo$^{48}$,
S. B{\"o}ser$^{41}$,
O. Botner$^{61}$,
J. B{\"o}ttcher$^{1}$,
E. Bourbeau$^{22}$,
J. Braun$^{40}$,
B. Brinson$^{6}$,
J. Brostean-Kaiser$^{63}$,
R. T. Burley$^{2}$,
R. S. Busse$^{43}$,
D. Butterfield$^{40}$,
M. A. Campana$^{49}$,
K. Carloni$^{14}$,
E. G. Carnie-Bronca$^{2}$,
S. Chattopadhyay$^{40,\: 64}$,
N. Chau$^{12}$,
C. Chen$^{6}$,
Z. Chen$^{55}$,
D. Chirkin$^{40}$,
S. Choi$^{56}$,
B. A. Clark$^{19}$,
L. Classen$^{43}$,
A. Coleman$^{61}$,
G. H. Collin$^{15}$,
A. Connolly$^{20,\: 21}$,
J. M. Conrad$^{15}$,
P. Coppin$^{13}$,
P. Correa$^{13}$,
D. F. Cowen$^{59,\: 60}$,
P. Dave$^{6}$,
C. De Clercq$^{13}$,
J. J. DeLaunay$^{58}$,
D. Delgado$^{14}$,
S. Deng$^{1}$,
K. Deoskar$^{54}$,
A. Desai$^{40}$,
P. Desiati$^{40}$,
K. D. de Vries$^{13}$,
G. de Wasseige$^{37}$,
T. DeYoung$^{24}$,
A. Diaz$^{15}$,
J. C. D{\'\i}az-V{\'e}lez$^{40}$,
M. Dittmer$^{43}$,
A. Domi$^{26}$,
H. Dujmovic$^{40}$,
M. A. DuVernois$^{40}$,
T. Ehrhardt$^{41}$,
P. Eller$^{27}$,
E. Ellinger$^{62}$,
S. El Mentawi$^{1}$,
D. Els{\"a}sser$^{23}$,
R. Engel$^{31,\: 32}$,
H. Erpenbeck$^{40}$,
J. Evans$^{19}$,
P. A. Evenson$^{44}$,
K. L. Fan$^{19}$,
K. Fang$^{40}$,
K. Farrag$^{16}$,
A. R. Fazely$^{7}$,
A. Fedynitch$^{57}$,
N. Feigl$^{10}$,
S. Fiedlschuster$^{26}$,
C. Finley$^{54}$,
L. Fischer$^{63}$,
D. Fox$^{59}$,
A. Franckowiak$^{11}$,
A. Fritz$^{41}$,
P. F{\"u}rst$^{1}$,
J. Gallagher$^{39}$,
E. Ganster$^{1}$,
A. Garcia$^{14}$,
L. Gerhardt$^{9}$,
A. Ghadimi$^{58}$,
C. Glaser$^{61}$,
T. Glauch$^{27}$,
T. Gl{\"u}senkamp$^{26,\: 61}$,
N. Goehlke$^{32}$,
J. G. Gonzalez$^{44}$,
S. Goswami$^{58}$,
D. Grant$^{24}$,
S. J. Gray$^{19}$,
O. Gries$^{1}$,
S. Griffin$^{40}$,
S. Griswold$^{52}$,
K. M. Groth$^{22}$,
C. G{\"u}nther$^{1}$,
P. Gutjahr$^{23}$,
C. Haack$^{26}$,
A. Hallgren$^{61}$,
R. Halliday$^{24}$,
L. Halve$^{1}$,
F. Halzen$^{40}$,
H. Hamdaoui$^{55}$,
M. Ha Minh$^{27}$,
K. Hanson$^{40}$,
J. Hardin$^{15}$,
A. A. Harnisch$^{24}$,
P. Hatch$^{33}$,
A. Haungs$^{31}$,
K. Helbing$^{62}$,
J. Hellrung$^{11}$,
F. Henningsen$^{27}$,
L. Heuermann$^{1}$,
N. Heyer$^{61}$,
S. Hickford$^{62}$,
A. Hidvegi$^{54}$,
C. Hill$^{16}$,
G. C. Hill$^{2}$,
K. D. Hoffman$^{19}$,
S. Hori$^{40}$,
K. Hoshina$^{40,\: 66}$,
W. Hou$^{31}$,
T. Huber$^{31}$,
K. Hultqvist$^{54}$,
M. H{\"u}nnefeld$^{23}$,
R. Hussain$^{40}$,
K. Hymon$^{23}$,
S. In$^{56}$,
A. Ishihara$^{16}$,
M. Jacquart$^{40}$,
O. Janik$^{1}$,
M. Jansson$^{54}$,
G. S. Japaridze$^{5}$,
M. Jeong$^{56}$,
M. Jin$^{14}$,
B. J. P. Jones$^{4}$,
D. Kang$^{31}$,
W. Kang$^{56}$,
X. Kang$^{49}$,
A. Kappes$^{43}$,
D. Kappesser$^{41}$,
L. Kardum$^{23}$,
T. Karg$^{63}$,
M. Karl$^{27}$,
A. Karle$^{40}$,
U. Katz$^{26}$,
M. Kauer$^{40}$,
J. L. Kelley$^{40}$,
A. Khatee Zathul$^{40}$,
A. Kheirandish$^{34,\: 35}$,
J. Kiryluk$^{55}$,
S. R. Klein$^{8,\: 9}$,
A. Kochocki$^{24}$,
R. Koirala$^{44}$,
H. Kolanoski$^{10}$,
T. Kontrimas$^{27}$,
L. K{\"o}pke$^{41}$,
C. Kopper$^{26}$,
D. J. Koskinen$^{22}$,
P. Koundal$^{31}$,
M. Kovacevich$^{49}$,
M. Kowalski$^{10,\: 63}$,
T. Kozynets$^{22}$,
J. Krishnamoorthi$^{40,\: 64}$,
K. Kruiswijk$^{37}$,
E. Krupczak$^{24}$,
A. Kumar$^{63}$,
E. Kun$^{11}$,
N. Kurahashi$^{49}$,
N. Lad$^{63}$,
C. Lagunas Gualda$^{63}$,
M. Lamoureux$^{37}$,
M. J. Larson$^{19}$,
S. Latseva$^{1}$,
F. Lauber$^{62}$,
J. P. Lazar$^{14,\: 40}$,
J. W. Lee$^{56}$,
K. Leonard DeHolton$^{60}$,
A. Leszczy{\'n}ska$^{44}$,
M. Lincetto$^{11}$,
Q. R. Liu$^{40}$,
M. Liubarska$^{25}$,
E. Lohfink$^{41}$,
C. Love$^{49}$,
C. J. Lozano Mariscal$^{43}$,
L. Lu$^{40}$,
F. Lucarelli$^{28}$,
W. Luszczak$^{20,\: 21}$,
Y. Lyu$^{8,\: 9}$,
J. Madsen$^{40}$,
K. B. M. Mahn$^{24}$,
Y. Makino$^{40}$,
E. Manao$^{27}$,
S. Mancina$^{40,\: 48}$,
W. Marie Sainte$^{40}$,
I. C. Mari{\c{s}}$^{12}$,
S. Marka$^{46}$,
Z. Marka$^{46}$,
M. Marsee$^{58}$,
I. Martinez-Soler$^{14}$,
R. Maruyama$^{45}$,
F. Mayhew$^{24}$,
T. McElroy$^{25}$,
F. McNally$^{38}$,
J. V. Mead$^{22}$,
K. Meagher$^{40}$,
S. Mechbal$^{63}$,
A. Medina$^{21}$,
M. Meier$^{16}$,
Y. Merckx$^{13}$,
L. Merten$^{11}$,
J. Micallef$^{24}$,
J. Mitchell$^{7}$,
T. Montaruli$^{28}$,
R. W. Moore$^{25}$,
Y. Morii$^{16}$,
R. Morse$^{40}$,
M. Moulai$^{40}$,
T. Mukherjee$^{31}$,
R. Naab$^{63}$,
R. Nagai$^{16}$,
M. Nakos$^{40}$,
U. Naumann$^{62}$,
J. Necker$^{63}$,
A. Negi$^{4}$,
M. Neumann$^{43}$,
H. Niederhausen$^{24}$,
M. U. Nisa$^{24}$,
A. Noell$^{1}$,
A. Novikov$^{44}$,
S. C. Nowicki$^{24}$,
A. Obertacke Pollmann$^{16}$,
V. O'Dell$^{40}$,
M. Oehler$^{31}$,
B. Oeyen$^{29}$,
A. Olivas$^{19}$,
R. {\O}rs{\o}e$^{27}$,
J. Osborn$^{40}$,
E. O'Sullivan$^{61}$,
H. Pandya$^{44}$,
N. Park$^{33}$,
G. K. Parker$^{4}$,
E. N. Paudel$^{44}$,
L. Paul$^{42,\: 50}$,
C. P{\'e}rez de los Heros$^{61}$,
J. Peterson$^{40}$,
S. Philippen$^{1}$,
A. Pizzuto$^{40}$,
M. Plum$^{50}$,
A. Pont{\'e}n$^{61}$,
Y. Popovych$^{41}$,
M. Prado Rodriguez$^{40}$,
B. Pries$^{24}$,
R. Procter-Murphy$^{19}$,
G. T. Przybylski$^{9}$,
C. Raab$^{37}$,
J. Rack-Helleis$^{41}$,
K. Rawlins$^{3}$,
Z. Rechav$^{40}$,
A. Rehman$^{44}$,
P. Reichherzer$^{11}$,
G. Renzi$^{12}$,
E. Resconi$^{27}$,
S. Reusch$^{63}$,
W. Rhode$^{23}$,
B. Riedel$^{40}$,
A. Rifaie$^{1}$,
E. J. Roberts$^{2}$,
S. Robertson$^{8,\: 9}$,
S. Rodan$^{56}$,
G. Roellinghoff$^{56}$,
M. Rongen$^{26}$,
C. Rott$^{53,\: 56}$,
T. Ruhe$^{23}$,
L. Ruohan$^{27}$,
D. Ryckbosch$^{29}$,
I. Safa$^{14,\: 40}$,
J. Saffer$^{32}$,
D. Salazar-Gallegos$^{24}$,
P. Sampathkumar$^{31}$,
S. E. Sanchez Herrera$^{24}$,
A. Sandrock$^{62}$,
M. Santander$^{58}$,
S. Sarkar$^{25}$,
S. Sarkar$^{47}$,
J. Savelberg$^{1}$,
P. Savina$^{40}$,
M. Schaufel$^{1}$,
H. Schieler$^{31}$,
S. Schindler$^{26}$,
L. Schlickmann$^{1}$,
B. Schl{\"u}ter$^{43}$,
F. Schl{\"u}ter$^{12}$,
N. Schmeisser$^{62}$,
T. Schmidt$^{19}$,
J. Schneider$^{26}$,
F. G. Schr{\"o}der$^{31,\: 44}$,
L. Schumacher$^{26}$,
G. Schwefer$^{1}$,
S. Sclafani$^{19}$,
D. Seckel$^{44}$,
M. Seikh$^{36}$,
S. Seunarine$^{51}$,
R. Shah$^{49}$,
A. Sharma$^{61}$,
S. Shefali$^{32}$,
N. Shimizu$^{16}$,
M. Silva$^{40}$,
B. Skrzypek$^{14}$,
B. Smithers$^{4}$,
R. Snihur$^{40}$,
J. Soedingrekso$^{23}$,
A. S{\o}gaard$^{22}$,
D. Soldin$^{32}$,
P. Soldin$^{1}$,
G. Sommani$^{11}$,
C. Spannfellner$^{27}$,
G. M. Spiczak$^{51}$,
C. Spiering$^{63}$,
M. Stamatikos$^{21}$,
T. Stanev$^{44}$,
T. Stezelberger$^{9}$,
T. St{\"u}rwald$^{62}$,
T. Stuttard$^{22}$,
G. W. Sullivan$^{19}$,
I. Taboada$^{6}$,
S. Ter-Antonyan$^{7}$,
M. Thiesmeyer$^{1}$,
W. G. Thompson$^{14}$,
J. Thwaites$^{40}$,
S. Tilav$^{44}$,
K. Tollefson$^{24}$,
C. T{\"o}nnis$^{56}$,
S. Toscano$^{12}$,
D. Tosi$^{40}$,
A. Trettin$^{63}$,
C. F. Tung$^{6}$,
R. Turcotte$^{31}$,
J. P. Twagirayezu$^{24}$,
B. Ty$^{40}$,
M. A. Unland Elorrieta$^{43}$,
A. K. Upadhyay$^{40,\: 64}$,
K. Upshaw$^{7}$,
N. Valtonen-Mattila$^{61}$,
J. Vandenbroucke$^{40}$,
N. van Eijndhoven$^{13}$,
D. Vannerom$^{15}$,
J. van Santen$^{63}$,
J. Vara$^{43}$,
J. Veitch-Michaelis$^{40}$,
M. Venugopal$^{31}$,
M. Vereecken$^{37}$,
S. Verpoest$^{44}$,
D. Veske$^{46}$,
A. Vijai$^{19}$,
C. Walck$^{54}$,
C. Weaver$^{24}$,
P. Weigel$^{15}$,
A. Weindl$^{31}$,
J. Weldert$^{60}$,
C. Wendt$^{40}$,
J. Werthebach$^{23}$,
M. Weyrauch$^{31}$,
N. Whitehorn$^{24}$,
C. H. Wiebusch$^{1}$,
N. Willey$^{24}$,
D. R. Williams$^{58}$,
L. Witthaus$^{23}$,
A. Wolf$^{1}$,
M. Wolf$^{27}$,
G. Wrede$^{26}$,
X. W. Xu$^{7}$,
J. P. Yanez$^{25}$,
E. Yildizci$^{40}$,
S. Yoshida$^{16}$,
R. Young$^{36}$,
F. Yu$^{14}$,
S. Yu$^{24}$,
T. Yuan$^{40}$,
Z. Zhang$^{55}$,
P. Zhelnin$^{14}$,
M. Zimmerman$^{40}$\\
\\
$^{1}$ III. Physikalisches Institut, RWTH Aachen University, D-52056 Aachen, Germany \\
$^{2}$ Department of Physics, University of Adelaide, Adelaide, 5005, Australia \\
$^{3}$ Dept. of Physics and Astronomy, University of Alaska Anchorage, 3211 Providence Dr., Anchorage, AK 99508, USA \\
$^{4}$ Dept. of Physics, University of Texas at Arlington, 502 Yates St., Science Hall Rm 108, Box 19059, Arlington, TX 76019, USA \\
$^{5}$ CTSPS, Clark-Atlanta University, Atlanta, GA 30314, USA \\
$^{6}$ School of Physics and Center for Relativistic Astrophysics, Georgia Institute of Technology, Atlanta, GA 30332, USA \\
$^{7}$ Dept. of Physics, Southern University, Baton Rouge, LA 70813, USA \\
$^{8}$ Dept. of Physics, University of California, Berkeley, CA 94720, USA \\
$^{9}$ Lawrence Berkeley National Laboratory, Berkeley, CA 94720, USA \\
$^{10}$ Institut f{\"u}r Physik, Humboldt-Universit{\"a}t zu Berlin, D-12489 Berlin, Germany \\
$^{11}$ Fakult{\"a}t f{\"u}r Physik {\&} Astronomie, Ruhr-Universit{\"a}t Bochum, D-44780 Bochum, Germany \\
$^{12}$ Universit{\'e} Libre de Bruxelles, Science Faculty CP230, B-1050 Brussels, Belgium \\
$^{13}$ Vrije Universiteit Brussel (VUB), Dienst ELEM, B-1050 Brussels, Belgium \\
$^{14}$ Department of Physics and Laboratory for Particle Physics and Cosmology, Harvard University, Cambridge, MA 02138, USA \\
$^{15}$ Dept. of Physics, Massachusetts Institute of Technology, Cambridge, MA 02139, USA \\
$^{16}$ Dept. of Physics and The International Center for Hadron Astrophysics, Chiba University, Chiba 263-8522, Japan \\
$^{17}$ Department of Physics, Loyola University Chicago, Chicago, IL 60660, USA \\
$^{18}$ Dept. of Physics and Astronomy, University of Canterbury, Private Bag 4800, Christchurch, New Zealand \\
$^{19}$ Dept. of Physics, University of Maryland, College Park, MD 20742, USA \\
$^{20}$ Dept. of Astronomy, Ohio State University, Columbus, OH 43210, USA \\
$^{21}$ Dept. of Physics and Center for Cosmology and Astro-Particle Physics, Ohio State University, Columbus, OH 43210, USA \\
$^{22}$ Niels Bohr Institute, University of Copenhagen, DK-2100 Copenhagen, Denmark \\
$^{23}$ Dept. of Physics, TU Dortmund University, D-44221 Dortmund, Germany \\
$^{24}$ Dept. of Physics and Astronomy, Michigan State University, East Lansing, MI 48824, USA \\
$^{25}$ Dept. of Physics, University of Alberta, Edmonton, Alberta, Canada T6G 2E1 \\
$^{26}$ Erlangen Centre for Astroparticle Physics, Friedrich-Alexander-Universit{\"a}t Erlangen-N{\"u}rnberg, D-91058 Erlangen, Germany \\
$^{27}$ Technical University of Munich, TUM School of Natural Sciences, Department of Physics, D-85748 Garching bei M{\"u}nchen, Germany \\
$^{28}$ D{\'e}partement de physique nucl{\'e}aire et corpusculaire, Universit{\'e} de Gen{\`e}ve, CH-1211 Gen{\`e}ve, Switzerland \\
$^{29}$ Dept. of Physics and Astronomy, University of Gent, B-9000 Gent, Belgium \\
$^{30}$ Dept. of Physics and Astronomy, University of California, Irvine, CA 92697, USA \\
$^{31}$ Karlsruhe Institute of Technology, Institute for Astroparticle Physics, D-76021 Karlsruhe, Germany  \\
$^{32}$ Karlsruhe Institute of Technology, Institute of Experimental Particle Physics, D-76021 Karlsruhe, Germany  \\
$^{33}$ Dept. of Physics, Engineering Physics, and Astronomy, Queen's University, Kingston, ON K7L 3N6, Canada \\
$^{34}$ Department of Physics {\&} Astronomy, University of Nevada, Las Vegas, NV, 89154, USA \\
$^{35}$ Nevada Center for Astrophysics, University of Nevada, Las Vegas, NV 89154, USA \\
$^{36}$ Dept. of Physics and Astronomy, University of Kansas, Lawrence, KS 66045, USA \\
$^{37}$ Centre for Cosmology, Particle Physics and Phenomenology - CP3, Universit{\'e} catholique de Louvain, Louvain-la-Neuve, Belgium \\
$^{38}$ Department of Physics, Mercer University, Macon, GA 31207-0001, USA \\
$^{39}$ Dept. of Astronomy, University of Wisconsin{\textendash}Madison, Madison, WI 53706, USA \\
$^{40}$ Dept. of Physics and Wisconsin IceCube Particle Astrophysics Center, University of Wisconsin{\textendash}Madison, Madison, WI 53706, USA \\
$^{41}$ Institute of Physics, University of Mainz, Staudinger Weg 7, D-55099 Mainz, Germany \\
$^{42}$ Department of Physics, Marquette University, Milwaukee, WI, 53201, USA \\
$^{43}$ Institut f{\"u}r Kernphysik, Westf{\"a}lische Wilhelms-Universit{\"a}t M{\"u}nster, D-48149 M{\"u}nster, Germany \\
$^{44}$ Bartol Research Institute and Dept. of Physics and Astronomy, University of Delaware, Newark, DE 19716, USA \\
$^{45}$ Dept. of Physics, Yale University, New Haven, CT 06520, USA \\
$^{46}$ Columbia Astrophysics and Nevis Laboratories, Columbia University, New York, NY 10027, USA \\
$^{47}$ Dept. of Physics, University of Oxford, Parks Road, Oxford OX1 3PU, United Kingdom\\
$^{48}$ Dipartimento di Fisica e Astronomia Galileo Galilei, Universit{\`a} Degli Studi di Padova, 35122 Padova PD, Italy \\
$^{49}$ Dept. of Physics, Drexel University, 3141 Chestnut Street, Philadelphia, PA 19104, USA \\
$^{50}$ Physics Department, South Dakota School of Mines and Technology, Rapid City, SD 57701, USA \\
$^{51}$ Dept. of Physics, University of Wisconsin, River Falls, WI 54022, USA \\
$^{52}$ Dept. of Physics and Astronomy, University of Rochester, Rochester, NY 14627, USA \\
$^{53}$ Department of Physics and Astronomy, University of Utah, Salt Lake City, UT 84112, USA \\
$^{54}$ Oskar Klein Centre and Dept. of Physics, Stockholm University, SE-10691 Stockholm, Sweden \\
$^{55}$ Dept. of Physics and Astronomy, Stony Brook University, Stony Brook, NY 11794-3800, USA \\
$^{56}$ Dept. of Physics, Sungkyunkwan University, Suwon 16419, Korea \\
$^{57}$ Institute of Physics, Academia Sinica, Taipei, 11529, Taiwan \\
$^{58}$ Dept. of Physics and Astronomy, University of Alabama, Tuscaloosa, AL 35487, USA \\
$^{59}$ Dept. of Astronomy and Astrophysics, Pennsylvania State University, University Park, PA 16802, USA \\
$^{60}$ Dept. of Physics, Pennsylvania State University, University Park, PA 16802, USA \\
$^{61}$ Dept. of Physics and Astronomy, Uppsala University, Box 516, S-75120 Uppsala, Sweden \\
$^{62}$ Dept. of Physics, University of Wuppertal, D-42119 Wuppertal, Germany \\
$^{63}$ Deutsches Elektronen-Synchrotron DESY, Platanenallee 6, 15738 Zeuthen, Germany  \\
$^{64}$ Institute of Physics, Sachivalaya Marg, Sainik School Post, Bhubaneswar 751005, India \\
$^{65}$ Department of Space, Earth and Environment, Chalmers University of Technology, 412 96 Gothenburg, Sweden \\
$^{66}$ Earthquake Research Institute, University of Tokyo, Bunkyo, Tokyo 113-0032, Japan \\

\subsection*{Acknowledgements}

\noindent
The authors gratefully acknowledge the support from the following agencies and institutions:
USA {\textendash} U.S. National Science Foundation-Office of Polar Programs,
U.S. National Science Foundation-Physics Division,
U.S. National Science Foundation-EPSCoR,
Wisconsin Alumni Research Foundation,
Center for High Throughput Computing (CHTC) at the University of Wisconsin{\textendash}Madison,
Open Science Grid (OSG),
Advanced Cyberinfrastructure Coordination Ecosystem: Services {\&} Support (ACCESS),
Frontera computing project at the Texas Advanced Computing Center,
U.S. Department of Energy-National Energy Research Scientific Computing Center,
Particle astrophysics research computing center at the University of Maryland,
Institute for Cyber-Enabled Research at Michigan State University,
and Astroparticle physics computational facility at Marquette University;
Belgium {\textendash} Funds for Scientific Research (FRS-FNRS and FWO),
FWO Odysseus and Big Science programmes,
and Belgian Federal Science Policy Office (Belspo);
Germany {\textendash} Bundesministerium f{\"u}r Bildung und Forschung (BMBF),
Deutsche Forschungsgemeinschaft (DFG),
Helmholtz Alliance for Astroparticle Physics (HAP),
Initiative and Networking Fund of the Helmholtz Association,
Deutsches Elektronen Synchrotron (DESY),
and High Performance Computing cluster of the RWTH Aachen;
Sweden {\textendash} Swedish Research Council,
Swedish Polar Research Secretariat,
Swedish National Infrastructure for Computing (SNIC),
and Knut and Alice Wallenberg Foundation;
European Union {\textendash} EGI Advanced Computing for research;
Australia {\textendash} Australian Research Council;
Canada {\textendash} Natural Sciences and Engineering Research Council of Canada,
Calcul Qu{\'e}bec, Compute Ontario, Canada Foundation for Innovation, WestGrid, and Compute Canada;
Denmark {\textendash} Villum Fonden, Carlsberg Foundation, and European Commission;
New Zealand {\textendash} Marsden Fund;
Japan {\textendash} Japan Society for Promotion of Science (JSPS)
and Institute for Global Prominent Research (IGPR) of Chiba University;
Korea {\textendash} National Research Foundation of Korea (NRF);
Switzerland {\textendash} Swiss National Science Foundation (SNSF);
United Kingdom {\textendash} Department of Physics, University of Oxford.
\section*{Full Author List: IceCube-Gen2 Collaboration (March 25, 2023)}

\scriptsize
\noindent
R. Abbasi$^{17}$,
M. Ackermann$^{77}$,
J. Adams$^{22}$,
S. K. Agarwalla$^{48,\: 78}$,
J. A. Aguilar$^{12}$,
M. Ahlers$^{26}$,
J.M. Alameddine$^{27}$,
N. M. Amin$^{54}$,
K. Andeen$^{51}$,
G. Anton$^{30}$,
C. Arg{\"u}elles$^{14}$,
Y. Ashida$^{48}$,
S. Athanasiadou$^{77}$,
J. Audehm$^{1}$,
S. N. Axani$^{54}$,
X. Bai$^{62}$,
A. Balagopal V.$^{48}$,
M. Baricevic$^{48}$,
S. W. Barwick$^{34}$,
V. Basu$^{48}$,
R. Bay$^{8}$,
K.-H. Becker$^{76}$,
J. Becker Tjus$^{11,\: 79}$,
J. Beise$^{75}$,
C. Bellenghi$^{31}$,
C. Benning$^{1}$,
S. BenZvi$^{64}$,
D. Berley$^{23}$,
E. Bernardini$^{60}$,
D. Z. Besson$^{40}$,
G. Binder$^{8,\: 9}$,
D. Bindig$^{76}$,
A. Bishop$^{48}$,
E. Blaufuss$^{23}$,
S. Blot$^{77}$,
M. Bohmer$^{31}$,
F. Bontempo$^{35}$,
J. Y. Book$^{14}$,
J. Borowka$^{1}$,
C. Boscolo Meneguolo$^{60}$,
S. B{\"o}ser$^{49}$,
O. Botner$^{75}$,
J. B{\"o}ttcher$^{1}$,
S. Bouma$^{30}$,
E. Bourbeau$^{26}$,
J. Braun$^{48}$,
B. Brinson$^{6}$,
J. Brostean-Kaiser$^{77}$,
R. T. Burley$^{2}$,
R. S. Busse$^{53}$,
D. Butterfield$^{48}$,
M. A. Campana$^{61}$,
K. Carloni$^{14}$,
E. G. Carnie-Bronca$^{2}$,
M. Cataldo$^{30}$,
S. Chattopadhyay$^{48,\: 78}$,
N. Chau$^{12}$,
C. Chen$^{6}$,
Z. Chen$^{67}$,
D. Chirkin$^{48}$,
S. Choi$^{68}$,
B. A. Clark$^{23}$,
R. Clark$^{42}$,
L. Classen$^{53}$,
A. Coleman$^{75}$,
G. H. Collin$^{15}$,
J. M. Conrad$^{15}$,
D. F. Cowen$^{72,\: 73}$,
P. Dave$^{6}$,
C. Deaconu$^{20,\: 21}$,
C. De Clercq$^{13}$,
S. De Kockere$^{13}$,
J. J. DeLaunay$^{71}$,
D. Delgado$^{14}$,
H. Dembinski$^{54}$,
S. Deng$^{1}$,
K. Deoskar$^{66}$,
A. Desai$^{48}$,
P. Desiati$^{48}$,
K. D. de Vries$^{13}$,
G. de Wasseige$^{45}$,
T. DeYoung$^{28}$,
A. Diaz$^{15}$,
J. C. D{\'\i}az-V{\'e}lez$^{48}$,
M. Dittmer$^{53}$,
A. Domi$^{30}$,
H. Dujmovic$^{48}$,
M. A. DuVernois$^{48}$,
T. Ehrhardt$^{49}$,
P. Eller$^{31}$,
S. El Mentawi$^{1}$,
R. Engel$^{35,\: 36}$,
H. Erpenbeck$^{48}$,
J. Evans$^{23}$,
J. J. Evans$^{50}$,
P. A. Evenson$^{54}$,
K. L. Fan$^{23}$,
K. Fang$^{48}$,
K. Farrag$^{43}$,
K. Farrag$^{16}$,
A. R. Fazely$^{7}$,
A. Fedynitch$^{69}$,
N. Feigl$^{10}$,
S. Fiedlschuster$^{30}$,
C. Finley$^{66}$,
L. Fischer$^{77}$,
D. Fox$^{72}$,
A. Franckowiak$^{11}$,
E. Friedman$^{23}$,
A. Fritz$^{49}$,
T. Fujii$^{58}$,
P. F{\"u}rst$^{1}$,
T. K. Gaisser$^{54}$,
J. Gallagher$^{47}$,
E. Ganster$^{1}$,
A. Garcia$^{14}$,
L. Gerhardt$^{9}$,
R. Gernhaeuser$^{31}$,
A. Ghadimi$^{71}$,
P. Giri$^{41}$,
C. Glaser$^{75}$,
T. Glauch$^{31}$,
T. Gl{\"u}senkamp$^{30,\: 75}$,
N. Goehlke$^{36}$,
J. G. Gonzalez$^{54}$,
S. Goswami$^{71}$,
D. Grant$^{28}$,
S. J. Gray$^{23}$,
O. Gries$^{1}$,
S. Griffin$^{48}$,
S. Griswold$^{64}$,
C. G{\"u}nther$^{1}$,
P. Gutjahr$^{27}$,
C. Haack$^{31}$,
T. Haji Azim$^{1}$,
A. Hallgren$^{75}$,
R. Halliday$^{28}$,
S. Hallmann$^{77}$,
L. Halve$^{1}$,
F. Halzen$^{48}$,
H. Hamdaoui$^{67}$,
M. Ha Minh$^{31}$,
K. Hanson$^{48}$,
J. Hardin$^{15}$,
A. A. Harnisch$^{28}$,
P. Hatch$^{37}$,
J. Haugen$^{48}$,
A. Haungs$^{35}$,
D. Heinen$^{1}$,
K. Helbing$^{76}$,
J. Hellrung$^{11}$,
B. Hendricks$^{73,\: 74}$,
F. Henningsen$^{31}$,
J. Henrichs$^{77}$,
L. Heuermann$^{1}$,
N. Heyer$^{75}$,
S. Hickford$^{76}$,
A. Hidvegi$^{66}$,
J. Hignight$^{29}$,
C. Hill$^{16}$,
G. C. Hill$^{2}$,
K. D. Hoffman$^{23}$,
B. Hoffmann$^{35}$,
K. Holzapfel$^{31}$,
K. Hoshina$^{48,\: 80}$,
W. Hou$^{35}$,
T. Huber$^{35}$,
T. Huege$^{35}$,
K. Hughes$^{19,\: 21}$,
K. Hultqvist$^{66}$,
M. H{\"u}nnefeld$^{27}$,
R. Hussain$^{48}$,
K. Hymon$^{27}$,
S. In$^{68}$,
A. Ishihara$^{16}$,
M. Jacquart$^{48}$,
O. Janik$^{1}$,
M. Jansson$^{66}$,
G. S. Japaridze$^{5}$,
K. Jayakumar$^{48,\: 78}$,
M. Jeong$^{68}$,
M. Jin$^{14}$,
B. J. P. Jones$^{4}$,
O. Kalekin$^{30}$,
D. Kang$^{35}$,
W. Kang$^{68}$,
X. Kang$^{61}$,
A. Kappes$^{53}$,
D. Kappesser$^{49}$,
L. Kardum$^{27}$,
T. Karg$^{77}$,
M. Karl$^{31}$,
A. Karle$^{48}$,
T. Katori$^{42}$,
U. Katz$^{30}$,
M. Kauer$^{48}$,
J. L. Kelley$^{48}$,
A. Khatee Zathul$^{48}$,
A. Kheirandish$^{38,\: 39}$,
J. Kiryluk$^{67}$,
S. R. Klein$^{8,\: 9}$,
T. Kobayashi$^{58}$,
A. Kochocki$^{28}$,
R. Koirala$^{54}$,
H. Kolanoski$^{10}$,
T. Kontrimas$^{31}$,
L. K{\"o}pke$^{49}$,
C. Kopper$^{28}$,
D. J. Koskinen$^{26}$,
P. Koundal$^{35}$,
M. Kovacevich$^{61}$,
M. Kowalski$^{10,\: 77}$,
T. Kozynets$^{26}$,
C. B. Krauss$^{29}$,
I. Kravchenko$^{41}$,
E. Krupczak$^{28}$,
A. Kumar$^{77}$,
E. Kun$^{11}$,
N. Kurahashi$^{61}$,
N. Lad$^{77}$,
C. Lagunas Gualda$^{77}$,
M. J. Larson$^{23}$,
F. Lauber$^{76}$,
J. P. Lazar$^{14,\: 48}$,
J. W. Lee$^{68}$,
A. Leszczy{\'n}ska$^{54}$,
M. Lincetto$^{11}$,
Q. R. Liu$^{48}$,
M. Liubarska$^{29}$,
M. Lohan$^{52}$,
E. Lohfink$^{49}$,
J. LoSecco$^{57}$,
C. Love$^{61}$,
C. J. Lozano Mariscal$^{53}$,
L. Lu$^{48}$,
F. Lucarelli$^{32}$,
A. Ludwig$^{44}$,
Y. Lyu$^{8,\: 9}$,
J. Madsen$^{48}$,
K. B. M. Mahn$^{28}$,
Y. Makino$^{48}$,
S. Mancina$^{48,\: 60}$,
S. Mandalia$^{43}$,
W. Marie Sainte$^{48}$,
I. C. Mari{\c{s}}$^{12}$,
S. Marka$^{56}$,
Z. Marka$^{56}$,
M. Marsee$^{71}$,
I. Martinez-Soler$^{14}$,
R. Maruyama$^{55}$,
T. McElroy$^{29}$,
F. McNally$^{46}$,
J. V. Mead$^{26}$,
K. Meagher$^{48}$,
S. Mechbal$^{77}$,
A. Medina$^{25}$,
M. Meier$^{16}$,
Y. Merckx$^{13}$,
L. Merten$^{11}$,
Z. Meyers$^{77}$,
J. Micallef$^{28}$,
T. Montaruli$^{32}$,
R. W. Moore$^{29}$,
Y. Morii$^{16}$,
R. Morse$^{48}$,
M. Moulai$^{48}$,
T. Mukherjee$^{35}$,
R. Naab$^{77}$,
R. Nagai$^{16}$,
M. Nakos$^{48}$,
A. Narayan$^{52}$,
U. Naumann$^{76}$,
J. Necker$^{77}$,
A. Nelles$^{30,\: 77}$,
M. Neumann$^{53}$,
H. Niederhausen$^{28}$,
M. U. Nisa$^{28}$,
A. Noell$^{1}$,
S. C. Nowicki$^{28}$,
E. Oberla$^{20,\: 21}$,
A. Obertacke Pollmann$^{16}$,
V. O'Dell$^{48}$,
M. Oehler$^{35}$,
B. Oeyen$^{33}$,
A. Olivas$^{23}$,
R. Orsoe$^{31}$,
J. Osborn$^{48}$,
E. O'Sullivan$^{75}$,
H. Pandya$^{54}$,
L. Papp$^{31}$,
N. Park$^{37}$,
G. K. Parker$^{4}$,
E. N. Paudel$^{54}$,
L. Paul$^{51}$,
C. P{\'e}rez de los Heros$^{75}$,
T. C. Petersen$^{26}$,
J. Peterson$^{48}$,
S. Philippen$^{1}$,
S. Pieper$^{76}$,
J. L. Pinfold$^{29}$,
A. Pizzuto$^{48}$,
I. Plaisier$^{77}$,
M. Plum$^{62}$,
A. Pont{\'e}n$^{75}$,
Y. Popovych$^{49}$,
M. Prado Rodriguez$^{48}$,
B. Pries$^{28}$,
R. Procter-Murphy$^{23}$,
G. T. Przybylski$^{9}$,
L. Pyras$^{77}$,
J. Rack-Helleis$^{49}$,
M. Rameez$^{52}$,
K. Rawlins$^{3}$,
Z. Rechav$^{48}$,
A. Rehman$^{54}$,
P. Reichherzer$^{11}$,
G. Renzi$^{12}$,
E. Resconi$^{31}$,
S. Reusch$^{77}$,
W. Rhode$^{27}$,
B. Riedel$^{48}$,
M. Riegel$^{35}$,
A. Rifaie$^{1}$,
E. J. Roberts$^{2}$,
S. Robertson$^{8,\: 9}$,
S. Rodan$^{68}$,
G. Roellinghoff$^{68}$,
M. Rongen$^{49}$,
C. Rott$^{65,\: 68}$,
T. Ruhe$^{27}$,
D. Ryckbosch$^{33}$,
I. Safa$^{14,\: 48}$,
J. Saffer$^{36}$,
D. Salazar-Gallegos$^{28}$,
P. Sampathkumar$^{35}$,
S. E. Sanchez Herrera$^{28}$,
A. Sandrock$^{27}$,
P. Sandstrom$^{48}$,
M. Santander$^{71}$,
S. Sarkar$^{29}$,
S. Sarkar$^{59}$,
J. Savelberg$^{1}$,
P. Savina$^{48}$,
M. Schaufel$^{1}$,
H. Schieler$^{35}$,
S. Schindler$^{30}$,
L. Schlickmann$^{1}$,
B. Schl{\"u}ter$^{53}$,
F. Schl{\"u}ter$^{12}$,
T. Schmidt$^{23}$,
J. Schneider$^{30}$,
F. G. Schr{\"o}der$^{35,\: 54}$,
L. Schumacher$^{31}$,
G. Schwefer$^{1}$,
S. Sclafani$^{61}$,
D. Seckel$^{54}$,
S. Seunarine$^{63}$,
M. H. Shaevitz$^{56}$,
R. Shah$^{61}$,
A. Sharma$^{75}$,
S. Shefali$^{36}$,
N. Shimizu$^{16}$,
M. Silva$^{48}$,
B. Skrzypek$^{14}$,
D. Smith$^{19,\: 21}$,
B. Smithers$^{4}$,
R. Snihur$^{48}$,
J. Soedingrekso$^{27}$,
A. S{\o}gaard$^{26}$,
G. Sommani$^{11}$,
D. Southall$^{19,\: 21}$,
C. Spannfellner$^{31}$,
G. M. Spiczak$^{63}$,
C. Spiering$^{77}$,
M. Stamatikos$^{25}$,
T. Stanev$^{54}$,
T. Stezelberger$^{9}$,
J. Stoffels$^{13}$,
T. St{\"u}rwald$^{76}$,
T. Stuttard$^{26}$,
G. W. Sullivan$^{23}$,
I. Taboada$^{6}$,
A. Taketa$^{70}$,
H. K. M. Tanaka$^{70}$,
S. Ter-Antonyan$^{7}$,
M. Thiesmeyer$^{1}$,
W. G. Thompson$^{14}$,
J. Thwaites$^{48}$,
S. Tilav$^{54}$,
K. Tollefson$^{28}$,
C. T{\"o}nnis$^{68}$,
J. Torres$^{24,\: 25}$,
S. Toscano$^{12}$,
D. Tosi$^{48}$,
A. Trettin$^{77}$,
Y. Tsunesada$^{58}$,
C. F. Tung$^{6}$,
R. Turcotte$^{35}$,
J. P. Twagirayezu$^{28}$,
B. Ty$^{48}$,
M. A. Unland Elorrieta$^{53}$,
A. K. Upadhyay$^{48,\: 78}$,
N. Valtonen-Mattila$^{75}$,
J. Vandenbroucke$^{48}$,
N. van Eijndhoven$^{13}$,
D. Vannerom$^{15}$,
J. van Santen$^{77}$,
J. Vara$^{53}$,
D. Veberic$^{35}$,
J. Veitch-Michaelis$^{48}$,
M. Venugopal$^{35}$,
S. Verpoest$^{33}$,
A. Vieregg$^{18,\: 19,\: 20,\: 21}$,
C. Walck$^{66}$,
T. B. Watson$^{4}$,
C. Weaver$^{28}$,
P. Weigel$^{15}$,
A. Weindl$^{35}$,
J. Weldert$^{72,\: 73}$,
C. Welling$^{77}$,
C. Wendt$^{48}$,
J. Werthebach$^{27}$,
M. Weyrauch$^{35}$,
N. Whitehorn$^{28,\: 44}$,
C. H. Wiebusch$^{1}$,
N. Willey$^{28}$,
D. R. Williams$^{71}$,
S. Wissel$^{72,\: 73,\: 74}$,
A. Wolf$^{1}$,
M. Wolf$^{31}$,
G. W{\"o}rner$^{35}$,
G. Wrede$^{30}$,
S. Wren$^{50}$,
X. W. Xu$^{7}$,
J. P. Yanez$^{29}$,
E. Yildizci$^{48}$,
S. Yoshida$^{16}$,
F. Yu$^{14}$,
S. Yu$^{28}$,
T. Yuan$^{48}$,
Z. Zhang$^{67}$,
P. Zhelnin$^{14}$,
S. Zierke$^{1}$
\\
\\
$^{1}$ III. Physikalisches Institut, RWTH Aachen University, D-52056 Aachen, Germany \\
$^{2}$ Department of Physics, University of Adelaide, Adelaide, 5005, Australia \\
$^{3}$ Dept. of Physics and Astronomy, University of Alaska Anchorage, 3211 Providence Dr., Anchorage, AK 99508, USA \\
$^{4}$ Dept. of Physics, University of Texas at Arlington, 502 Yates St., Science Hall Rm 108, Box 19059, Arlington, TX 76019, USA \\
$^{5}$ CTSPS, Clark-Atlanta University, Atlanta, GA 30314, USA \\
$^{6}$ School of Physics and Center for Relativistic Astrophysics, Georgia Institute of Technology, Atlanta, GA 30332, USA \\
$^{7}$ Dept. of Physics, Southern University, Baton Rouge, LA 70813, USA \\
$^{8}$ Dept. of Physics, University of California, Berkeley, CA 94720, USA \\
$^{9}$ Lawrence Berkeley National Laboratory, Berkeley, CA 94720, USA \\
$^{10}$ Institut f{\"u}r Physik, Humboldt-Universit{\"a}t zu Berlin, D-12489 Berlin, Germany \\
$^{11}$ Fakult{\"a}t f{\"u}r Physik {\&} Astronomie, Ruhr-Universit{\"a}t Bochum, D-44780 Bochum, Germany \\
$^{12}$ Universit{\'e} Libre de Bruxelles, Science Faculty CP230, B-1050 Brussels, Belgium \\
$^{13}$ Vrije Universiteit Brussel (VUB), Dienst ELEM, B-1050 Brussels, Belgium \\
$^{14}$ Department of Physics and Laboratory for Particle Physics and Cosmology, Harvard University, Cambridge, MA 02138, USA \\
$^{15}$ Dept. of Physics, Massachusetts Institute of Technology, Cambridge, MA 02139, USA \\
$^{16}$ Dept. of Physics and The International Center for Hadron Astrophysics, Chiba University, Chiba 263-8522, Japan \\
$^{17}$ Department of Physics, Loyola University Chicago, Chicago, IL 60660, USA \\
$^{18}$ Dept. of Astronomy and Astrophysics, University of Chicago, Chicago, IL 60637, USA \\
$^{19}$ Dept. of Physics, University of Chicago, Chicago, IL 60637, USA \\
$^{20}$ Enrico Fermi Institute, University of Chicago, Chicago, IL 60637, USA \\
$^{21}$ Kavli Institute for Cosmological Physics, University of Chicago, Chicago, IL 60637, USA \\
$^{22}$ Dept. of Physics and Astronomy, University of Canterbury, Private Bag 4800, Christchurch, New Zealand \\
$^{23}$ Dept. of Physics, University of Maryland, College Park, MD 20742, USA \\
$^{24}$ Dept. of Astronomy, Ohio State University, Columbus, OH 43210, USA \\
$^{25}$ Dept. of Physics and Center for Cosmology and Astro-Particle Physics, Ohio State University, Columbus, OH 43210, USA \\
$^{26}$ Niels Bohr Institute, University of Copenhagen, DK-2100 Copenhagen, Denmark \\
$^{27}$ Dept. of Physics, TU Dortmund University, D-44221 Dortmund, Germany \\
$^{28}$ Dept. of Physics and Astronomy, Michigan State University, East Lansing, MI 48824, USA \\
$^{29}$ Dept. of Physics, University of Alberta, Edmonton, Alberta, Canada T6G 2E1 \\
$^{30}$ Erlangen Centre for Astroparticle Physics, Friedrich-Alexander-Universit{\"a}t Erlangen-N{\"u}rnberg, D-91058 Erlangen, Germany \\
$^{31}$ Physik-department, Technische Universit{\"a}t M{\"u}nchen, D-85748 Garching, Germany \\
$^{32}$ D{\'e}partement de physique nucl{\'e}aire et corpusculaire, Universit{\'e} de Gen{\`e}ve, CH-1211 Gen{\`e}ve, Switzerland \\
$^{33}$ Dept. of Physics and Astronomy, University of Gent, B-9000 Gent, Belgium \\
$^{34}$ Dept. of Physics and Astronomy, University of California, Irvine, CA 92697, USA \\
$^{35}$ Karlsruhe Institute of Technology, Institute for Astroparticle Physics, D-76021 Karlsruhe, Germany  \\
$^{36}$ Karlsruhe Institute of Technology, Institute of Experimental Particle Physics, D-76021 Karlsruhe, Germany  \\
$^{37}$ Dept. of Physics, Engineering Physics, and Astronomy, Queen's University, Kingston, ON K7L 3N6, Canada \\
$^{38}$ Department of Physics {\&} Astronomy, University of Nevada, Las Vegas, NV, 89154, USA \\
$^{39}$ Nevada Center for Astrophysics, University of Nevada, Las Vegas, NV 89154, USA \\
$^{40}$ Dept. of Physics and Astronomy, University of Kansas, Lawrence, KS 66045, USA \\
$^{41}$ Dept. of Physics and Astronomy, University of Nebraska{\textendash}Lincoln, Lincoln, Nebraska 68588, USA \\
$^{42}$ Dept. of Physics, King's College London, London WC2R 2LS, United Kingdom \\
$^{43}$ School of Physics and Astronomy, Queen Mary University of London, London E1 4NS, United Kingdom \\
$^{44}$ Department of Physics and Astronomy, UCLA, Los Angeles, CA 90095, USA \\
$^{45}$ Centre for Cosmology, Particle Physics and Phenomenology - CP3, Universit{\'e} catholique de Louvain, Louvain-la-Neuve, Belgium \\
$^{46}$ Department of Physics, Mercer University, Macon, GA 31207-0001, USA \\
$^{47}$ Dept. of Astronomy, University of Wisconsin{\textendash}Madison, Madison, WI 53706, USA \\
$^{48}$ Dept. of Physics and Wisconsin IceCube Particle Astrophysics Center, University of Wisconsin{\textendash}Madison, Madison, WI 53706, USA \\
$^{49}$ Institute of Physics, University of Mainz, Staudinger Weg 7, D-55099 Mainz, Germany \\
$^{50}$ School of Physics and Astronomy, The University of Manchester, Oxford Road, Manchester, M13 9PL, United Kingdom \\
$^{51}$ Department of Physics, Marquette University, Milwaukee, WI, 53201, USA \\
$^{52}$ Dept. of High Energy Physics, Tata Institute of Fundamental Research, Colaba, Mumbai 400 005, India \\
$^{53}$ Institut f{\"u}r Kernphysik, Westf{\"a}lische Wilhelms-Universit{\"a}t M{\"u}nster, D-48149 M{\"u}nster, Germany \\
$^{54}$ Bartol Research Institute and Dept. of Physics and Astronomy, University of Delaware, Newark, DE 19716, USA \\
$^{55}$ Dept. of Physics, Yale University, New Haven, CT 06520, USA \\
$^{56}$ Columbia Astrophysics and Nevis Laboratories, Columbia University, New York, NY 10027, USA \\
$^{57}$ Dept. of Physics, University of Notre Dame du Lac, 225 Nieuwland Science Hall, Notre Dame, IN 46556-5670, USA \\
$^{58}$ Graduate School of Science and NITEP, Osaka Metropolitan University, Osaka 558-8585, Japan \\
$^{59}$ Dept. of Physics, University of Oxford, Parks Road, Oxford OX1 3PU, UK \\
$^{60}$ Dipartimento di Fisica e Astronomia Galileo Galilei, Universit{\`a} Degli Studi di Padova, 35122 Padova PD, Italy \\
$^{61}$ Dept. of Physics, Drexel University, 3141 Chestnut Street, Philadelphia, PA 19104, USA \\
$^{62}$ Physics Department, South Dakota School of Mines and Technology, Rapid City, SD 57701, USA \\
$^{63}$ Dept. of Physics, University of Wisconsin, River Falls, WI 54022, USA \\
$^{64}$ Dept. of Physics and Astronomy, University of Rochester, Rochester, NY 14627, USA \\
$^{65}$ Department of Physics and Astronomy, University of Utah, Salt Lake City, UT 84112, USA \\
$^{66}$ Oskar Klein Centre and Dept. of Physics, Stockholm University, SE-10691 Stockholm, Sweden \\
$^{67}$ Dept. of Physics and Astronomy, Stony Brook University, Stony Brook, NY 11794-3800, USA \\
$^{68}$ Dept. of Physics, Sungkyunkwan University, Suwon 16419, Korea \\
$^{69}$ Institute of Physics, Academia Sinica, Taipei, 11529, Taiwan \\
$^{70}$ Earthquake Research Institute, University of Tokyo, Bunkyo, Tokyo 113-0032, Japan \\
$^{71}$ Dept. of Physics and Astronomy, University of Alabama, Tuscaloosa, AL 35487, USA \\
$^{72}$ Dept. of Astronomy and Astrophysics, Pennsylvania State University, University Park, PA 16802, USA \\
$^{73}$ Dept. of Physics, Pennsylvania State University, University Park, PA 16802, USA \\
$^{74}$ Institute of Gravitation and the Cosmos, Center for Multi-Messenger Astrophysics, Pennsylvania State University, University Park, PA 16802, USA \\
$^{75}$ Dept. of Physics and Astronomy, Uppsala University, Box 516, S-75120 Uppsala, Sweden \\
$^{76}$ Dept. of Physics, University of Wuppertal, D-42119 Wuppertal, Germany \\
$^{77}$ Deutsches Elektronen-Synchrotron DESY, Platanenallee 6, 15738 Zeuthen, Germany  \\
$^{78}$ Institute of Physics, Sachivalaya Marg, Sainik School Post, Bhubaneswar 751005, India \\
$^{79}$ Department of Space, Earth and Environment, Chalmers University of Technology, 412 96 Gothenburg, Sweden \\
$^{80}$ Earthquake Research Institute, University of Tokyo, Bunkyo, Tokyo 113-0032, Japan

\subsection*{Acknowledgments}

\noindent
The authors gratefully acknowledge the support from the following agencies and institutions:
USA {\textendash} U.S. National Science Foundation-Office of Polar Programs,
U.S. National Science Foundation-Physics Division,
U.S. National Science Foundation-EPSCoR,
Wisconsin Alumni Research Foundation,
Center for High Throughput Computing (CHTC) at the University of Wisconsin{\textendash}Madison,
Open Science Grid (OSG),
Advanced Cyberinfrastructure Coordination Ecosystem: Services {\&} Support (ACCESS),
Frontera computing project at the Texas Advanced Computing Center,
U.S. Department of Energy-National Energy Research Scientific Computing Center,
Particle astrophysics research computing center at the University of Maryland,
Institute for Cyber-Enabled Research at Michigan State University,
and Astroparticle physics computational facility at Marquette University;
Belgium {\textendash} Funds for Scientific Research (FRS-FNRS and FWO),
FWO Odysseus and Big Science programmes,
and Belgian Federal Science Policy Office (Belspo);
Germany {\textendash} Bundesministerium f{\"u}r Bildung und Forschung (BMBF),
Deutsche Forschungsgemeinschaft (DFG),
Helmholtz Alliance for Astroparticle Physics (HAP),
Initiative and Networking Fund of the Helmholtz Association,
Deutsches Elektronen Synchrotron (DESY),
and High Performance Computing cluster of the RWTH Aachen;
Sweden {\textendash} Swedish Research Council,
Swedish Polar Research Secretariat,
Swedish National Infrastructure for Computing (SNIC),
and Knut and Alice Wallenberg Foundation;
European Union {\textendash} EGI Advanced Computing for research;
Australia {\textendash} Australian Research Council;
Canada {\textendash} Natural Sciences and Engineering Research Council of Canada,
Calcul Qu{\'e}bec, Compute Ontario, Canada Foundation for Innovation, WestGrid, and Compute Canada;
Denmark {\textendash} Villum Fonden, Carlsberg Foundation, and European Commission;
New Zealand {\textendash} Marsden Fund;
Japan {\textendash} Japan Society for Promotion of Science (JSPS)
and Institute for Global Prominent Research (IGPR) of Chiba University;
Korea {\textendash} National Research Foundation of Korea (NRF);
Switzerland {\textendash} Swiss National Science Foundation (SNSF);
United Kingdom {\textendash} Department of Physics, University of Oxford.

\end{document}
