\section{Numerical implementation}
\label{sec:3}

The collagen density evolution is computed by implicitly integrating Eq.\ (\ref{eq:2-10}). A standard Backward-Euler scheme is implemented to compute the collagen density ${\rho}^{0}_{\mathrm{co}}$. The collagen density ${\rho}^{0}_{\mathrm{co}}$ is multiplied by the mass-specific Helmholtz free energy $\mathit{\psi}_{\mathrm{co, m}}$ to obtain the corresponding energy for the collagen fibers per unit reference volume $\mathit{\psi}_{\mathrm{co}}$ as described by Eq.\ (\ref{eq:2-20}). 

\subsection{Time integration}
To apply the time-discretization of the collagen evolution equation, we define the time step as $\Delta t = (t_{\mathrm{n+1}} - t_{\mathrm{n}})$ where $t_{\mathrm{n+1}}$ refers to the current time-step and $t_{\mathrm{n}}$ refers to the previous time-step. Variables from time-step $t_{\mathrm{n}}$ are denoted by the subscript $\mathrm{n}$, and variables from the current time $t_{\mathrm{n+1}}$ are without a subscript. Consequently, the implicit Backward-Euler integration scheme can be expressed as following \begin{equation}
\label{eq:3-1}
{\rho}^{0}_{\mathrm{co}} =  ({\rho}^{0}_{\mathrm{co}})_{\mathrm{n}} + {\Delta t} \, \dot{\rho}^{0}_{\mathrm{co}},
\end{equation}
and the corresponding residual equation is 
\begin{equation}
\label{eq:3-2}
r_{\mathrm{\rho}} = {\rho}^{0}_{\mathrm{co}} - ({\rho}^{0}_{\mathrm{co}})_{\mathrm{n}} - {\Delta t} \, \dot{\rho}^{0}_{\mathrm{co}}  = 0.
\end{equation}

Solving the residual equation using the Newton-Raphson scheme gives us the densification rate $\dot{\rho}^{0}_{\mathrm{co}}$ at the current time-step $t_{\mathrm{n+1}}$. By substituting with $\dot{\rho}^{0}_{\mathrm{co}}$ in Eq.\ (\ref{eq:3-1}), we can find the corresponding collagen density ${\rho}^{0}_{\mathrm{co}}$. Then, ${\rho}^{0}_{\mathrm{co}}$ is inserted in Eq.\ (\ref{eq:2-20}) to compute the corresponding strain energy $\mathit{\psi}_{\mathrm{co}}$. The time-integration steps are explained in Algorithm \ref{densification_algorithm}.

\vspace{0.5cm}
\setlength\belowcaptionskip{1ex}
\begin{algorithm}[H]
	\begin{algorithmic}[1]
		\State Initialize Backward-Euler integration scheme
		\State Input $a_{1}$, $a_{2}$, $c_{\mathrm{cell}}$,  $h$, $\tau$, $t$, ${\psi}_{\mathrm{crit}}$
		\State Output $\rho^{0}_{\mathrm{co}}$ 
		%\For{$t \leq t_{end}$} 
		\State Compute $\dot{\alpha}_{\mathrm{bio}} \gets  \frac{h}{\tau}\, e^{-(t / \tau)^{h}}  \, (\frac{t}{\tau})^{h - 1}$  
		\State Compute $\dot{\rho}^{0}_{\mathrm{bio}} \gets a_{1} \, c_{\mathrm{cell}} \, \dot{\alpha}_{\mathrm{bio}}$
		\State Compute ${\psi}_{\mathrm{co, m}}$
		\If{${\psi}_{\mathrm{co, m}}$ $\geq {\psi}_{\mathrm{crit}} $}
		\State Compute $\dot{\alpha}_{\mathrm{mech}} \gets  e^{-({\rho}^{0}_{\mathrm{co}} / \rho_{\mathrm{th}})}$ 
		\State Compute $\dot{\rho}^{0}_{\mathrm{mech}} \gets a_{2} \, c_{\mathrm{cell}} \, \dot{\alpha}_{\mathrm{mech}} \, {\rho}^{0}_{\mathrm{co}} \, \frac{({\psi}_{\mathrm{co, m}} - {\psi}_{\mathrm{crit}})}{{\psi}_{\mathrm{crit}}} $
		\State $r_{\mathrm{\rho}} \gets {\rho}^{0}_{\mathrm{co}} - ({\rho}^{0}_{\mathrm{co}})_{\mathrm{n}} - {\Delta t} \, \dot{\rho}^{0}_{\mathrm{co}}  = 0 $
		\While  {$|r_{\mathrm{\rho}}| \leq tolerance$}
		\State Compute $\rho^{0}_{\mathrm{co}}$ using Newton-Raphson method
		\EndWhile
		\EndIf
	\end{algorithmic} 
	\caption{Computing collagen fiber density}
	\label{densification_algorithm}
\end{algorithm}


\subsection{Computing stresses and material tangents}
To construct a global finite element system, first we compute the second Piola-Kirchhoff stresses for each constituent and then sum up their individual contribution to get the total second Piola-Kirchhoff stress $\mathbf{S}$ as described by the Eq.\ \ref{eq:2-27}. The computation of second Piola-Kirchhoff stress of the textile scaffold $\mathbf{S}_{\mathrm{tex}}$ and the isotropic matrix $\mathbf{S}_{\mathrm{matrix}}$ can be performed in a straightforward way as already described by the Eq.\ \ref{eq:2-28} and Eq.\ \ref{eq:2-29} respectively. However, the computation of the second Piola-Kirchhoff stress for the collagen part $\mathbf{S}_{\mathrm{co}}$ requires taking into consideration the influence of the collagen evolution equations. The collagen density ${\rho}^{0}_{\mathrm{co}}$ depends on the mass-specific energy ${\psi}_{\mathrm{co, m}}$ and consequently on the Cauchy-Green tensor $\mathbf{C}$, which
introduces additional terms to compute the derivatives as shown in Eq.\ \ref{eq:2-9}. By summing up the contributions from each constituent, we end with an expression for the second Piola-Kirchhoff stress that reads  
\begin{equation}
\label{eq:3-3}
\mathbf{S} = 2 \, \left(\frac{\partial {\psi}_{\mathrm{tex}}}{\partial \mathbf{C}} + \frac{\partial {\psi}_{\mathrm{matrix}}}{\partial \mathbf{C}} + \frac{\partial {\rho}^{0}_{\mathrm{co}}}{\partial \mathbf{C}} \, {\psi}_{\mathrm{co, m}} + {\rho}^{0}_{\mathrm{co}} \, \frac{\partial {\psi}_{\mathrm{co, m}}}{\partial \mathbf{C}} \right).
\end{equation}

 The next step is to compute the material tangent operator ${\bf \mathbb{C}}$. By applying the same approach used to compute $\mathbf{S}$, ${\bf \mathbb{C}}$ is computed by summing up the tangents of each constituent as described by the following expression
\begin{equation}
\label{eq:3-4}
{ \mathbb{C}} = {\mathbb{C}_{\mathrm{tex}}  + { \mathbb{C}_{\mathrm{matrix}}} + { \mathbb{C}_{\mathrm{co}}}}.
\end{equation}

The tangent tensor for each constituent is computed by taking the partial derivative of the second Piola-Kirchhoff stress with respect to the right Cauchy-Green tensor. This gives us the following term for the textile part
\begin{equation}
\label{eq:3-5}
{ \mathbb{C}_{\mathrm{tex}}} = 2 \, \frac{\partial \mathbf{S}_{\mathrm{tex}}}{\partial \mathbf{C}} = 4 \, \frac{\partial^{2} {\psi}_{\mathrm{tex}}}{\partial \mathbf{C}^{2}}.
\end{equation}

Analogously, for the matrix part, we compute the following term
\begin{equation}
\label{eq:3-6}
{ \mathbb{C}_{\mathrm{matrix}}} = 4 \, \frac{\partial^{2} {\psi}_{\mathrm{matrix}}}{\partial \mathbf{C}^{2}}.
\end{equation}

For the collagen part, computing the tangent operator is more complex, because the Helmholtz free energy ${\psi}_{\mathrm{co}}$ depends on the collagen density ${\rho}^{0}_{\mathrm{co}}$. Similar to the other constituents, we describe the tangent operator by the following expression
\begin{equation}
\label{eq:3-7}
{ \mathbb{C}_{\mathrm{co}}} = 4 \, \frac{\partial^{2} {\psi}_{\mathrm{co}}}{\partial \mathbf{C}^{2}}.
\end{equation}

The derivatives in Eq.\ \ref{eq:3-3} and Eqs.\ \ref{eq:3-5}-\ref{eq:3-7} are computed using the code generated by the automatic differentiation software package AceGen \cite{Korelc_2002, Korelc_2009}.

\subsection{Finite element implementation}
In our structural computations, we used a special finite element technology with reduced integration, namely the solid-shell element Q1STs \cite{Reese_2007, Barfusz_2021b}. Q1STs is a low-order isoparametric element with eight nodes. Due to the application of the reduced integration concept, the element contains only one Gauss point within the shell plane. It is especially beneficial for modeling thin shell structures such as heart valves because we can use an arbitrary number of Gauss points through the element's thickness. Furthermore, Q1STs element formulation offers a remedy to volumetric and shear locking. Locking treatment and hourglass stabilization in Q1STs are achieved by applying the concepts of enhanced assumed strain and assumed natural strain. Using elements capable of treating such locking phenomena is essential for us to compute the examples presented in Section \ref{sec:5} because: (i) soft biological tissues are almost incompressible materials which makes them susceptible to volumetric locking, and (b) the structure presented in our work are under severe bending which would cause shear locking in case of using standard low-order element formulation. An additional advantage of using a solid-shell formulation is its ability to model the non-linear material behavior along the thickness direction using only one element. Consequently, we can drastically reduce the number of elements needed in our computations compared to using a standard continuum solid element.
