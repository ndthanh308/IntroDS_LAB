\section{Introduction}
\label{sec:1}

In modern industrialized societies, a significant portion of the population suffers from valvular heart disease \cite{Lung_Vahanian_2011}. The number in the US is estimated by Nkoma et al.\ \cite{Nkomo_EtAl_2006} to be $2.5 \%$ of the population. Valvular heart defects are not restricted to a specific age group. They affect young and old people since many are born with congenital heart defects \cite{Hoffman_Kaplan_2002}. A typical treatment approach is to replace the defective native valve with a biomedical heart valve implant. Finding the optimal type and design of heart valve implants has been investigated for decades. Heart valve implants can be categorized into three groups: (i) mechanical valves, (ii) bioprosthetic valves and (iii) tissue-engineered heart valves \cite{Mela_2019}.   

The essential structural requirement of a heart valve implant is to withstand the cyclic loading conditions ($\approx$ 90000 loading cycles/day) for decades. In this regard, mechanical valves offer the most reliable solution. However, it is necessary for patients using mechanical valves to take anti-coagulation medications for the rest of their life, adversely affecting their lifestyle. This severe drawback leads to the increasing adaptation of bioprosthetic valves produced out of decellularized bovine or porcine heart valves. Bioprosthetic valves improve the patient's lifestyle since they do not need to take anti-coagulation medicines \cite{Mol_2009}. However, the lifetime of bioprosthetic valves is often shorter than the patient's expected, lifespan making it more suited for older patients. The short lifetime is caused by damage due to calcification and the inability of the implant to grow or remodel. 

Studies have shown that the heart valve's ability to adapt to hemodynamic conditions through tissue growth and remodeling is essential for its long-term durability and integrity \cite{Chester_etal_2014}. Adaptation is crucial in the case of children born with congenital valvular defects since hemodynamic conditions are altered significantly during their lifespan (e.g.\ change in blood pressure and heart rate) \cite{Oomen_etal_2016}. The drawbacks of mechanical and bioprosthetic valves were the main impetus for many researchers to investigate the implantation of living biological valves. One of the earliest developments in this field was the Ross procedure performed in 1967 \cite{Ross_1967}, which transplanted damaged aortic and mitral heart valves with pulmonary autografts. Tissue engineering aims to avoid the transplantation procedure through the fabrication and implantation of functional living valves \cite{Yacoub_Takkenberg_2005}.   

One of the main deterrents to using tissue-engineered heart valves (TEHVs) is their low mechanical strength. Improving the mechanical properties can be achieved by impeding a reinforcement material with the desired mechanical properties. These reinforced valves are called biohybrid heart valves. Biohybrid valve constituents can be decomposed into mechanobiologically active and passive materials. The reinforcement materials (e.g.\ scaffold) act as a passive constituent, while the extracellular matrix (ECM) actively reacts to mechanobiological stimulation. The scaffold is a porous material that supports the ECM during maturation. Among the widely used scaffold types are textile and 3D printed scaffolds \cite{Vukicevic_etal_2017}. Textile scaffolds are produced from biocompatible polymers using special techniques such as electrospinning \cite{Hinderer_etal_2017} or knitting \cite{Liberski_etal_2016}. Biomimetic non-woven textile scaffolds have been used by Moreira et al.\ \cite{Moreira_etal_2016} to fabricate TEHVs. A macroporous textile scaffold provides structural support to the implant and guides the ECM growth during the maturation process. The long-term goal is to produce valves in which the ECM develops enough mechanical strength to sustain physiological loads while the scaffold degrades after the implantation procedure \cite{Mendelson_2006}.

An essential aspect of developing TEHVs is tailoring their mechanical properties to resemble the behavior of native human heart valves. From a structural mechanics point of view, a native heart valve is a soft, thin and highly anisotropic shell structure. The density and orientation of collagen fibers highly influence the valve's mechanical properties. In the case of a biohybrid heart valve we are considering in this work, the textile scaffold is the main load-carrying constituent in the implant. Consequently, it is necessary to tailor the anisotropic behavior of the scaffold to mimic collagenous native valves. A native-like mechanical behavior can be achieved through (i) producing scaffolds with anisotropic microstructure and (ii) embedding fiber reinforcements to induce anisotropic mechanical behavior \cite{Moreira_etal_2016}.

% literature review
There is extensive literature on the mechanical modeling of heart valves. Among them is the work of Driessen et al.\ \cite{Driessen_etal_2007, Driessen_etal_2008} on modeling tissue-engineered heart valves and remodeling of angular collagen fiber distribution. With the help of these finite element models, Loerakker et al.\ \cite{Loerakker_etal_2013} showed that tissue anisotropy could significantly influence the hemodynamics of the valves. On another front, Driessen et al.\ \cite{Driessen_etal_2005} developed an approach to consider the angular collagen fiber distribution in the constitutive model. Later, Gasser et al.\ \cite{Gasser_etal_2006} introduced a new approach based on generalized structural tensors to address a similar issue in modeling arteries.

One of the main challenges in manufacturing heart valves is their complex geometry. Native valves have a semi-lunar shape with curved cusps. Fabricating and suturing an artificial valve with a semilunar shape is a complex and challenging process which suffers from many unpredictable conditions. Reliability can be improved by using simpler designs that mimic the hemodynamics of native valves. Therefore, Weber et al.\ \cite{Weber_etal_2014} proposed a tubular valve design. Mechanical properties can be enhanced by attaching fiber reinforcement. Stapleton et al.\ \cite{Stapleton_etal_2015} studied the influence of fiber reinforcement density and orientation on the mechanical behavior of the valves. Furthermore, Sodhani et al.\ \cite{Sodhani_etal_2017, Sodhani_etal_2018a} developed a multiscale modeling approach to simulate valves made of knitted textile scaffolds. The model was then extended to consider a coupled fluid-structural interaction \cite{Sodhani_etal_2018b}. Although these models are highly accurate in predicting the valve's hemodynamics, their computational cost is enormous. This makes them ill-suited for exploring and optimizing the design of TEHVs. Furthermore, the heart valve models from Stapleton et al.\ \cite{Stapleton_etal_2015} and Sodhani et al.\ \cite{Sodhani_etal_2017, Sodhani_etal_2018a, Sodhani_etal_2018b} only considered the passive mechanical behavior of the implant without considering mechanobiologically induced tissue growth and remodeling during the maturation process.

ECM is secreted during the maturation process. It surrounds the cells and provides structural support to the tissue through networks of protein fibers that provide mechanical strength. Biological tissues undergo various types of growth \cite{Kuhl_2014, Eskandari_2015}, such as volumetric growth, mass growth or cross-sectional area growth. Among the pioneering work on modeling finite volumetric growth is the article from Rodriguez et al.\ \cite{Rodriguez_1994}, in which they applied the multiplicative split of the deformation gradient to describe the material's inelastic response during the growth process. Later, Humphrey \& Rajagopal \cite{Humphrey_2002} introduced the constrained mixture model which considers the kinetics of the production and degradation processes of the tissue's constituents. Volumetric changes in soft biological tissue's can be either isotropic or anisotropic. Anisotropic growth can be implemented by defining the growth direction \cite{Ambrosi_2019}. However, this approach is only valid for simple geometries. More generalized approach based on the homogenized constrained mixture models was introduced by Braeu et al.\ \cite{Braeu_2017, Braeu_2019}. Based on the multiplicative split concept from Rodriguez et al.\ \cite{Rodriguez_1994}, Lamm et al.\ \cite{Lamm_2021, Lamm_2022} introduced a general approach to model volumetric growth without pre-defining a growth tensor. This approach is based on defining a homeostatic stress surface. The model was extended by Holthusen et al.\ \cite{Holthusen_2023} to consider collagen fiber reorientation using a novel framework based on a co-rotated intermediate configuration. Another interesting application for growth models is in-stent restenosis which was investigated by Manjunatha et al.\ \cite{Manjunatha_2022}.


During the in-vitro maturation process of soft tissues, volumetric contraction and mass growth due to the production of collagen fibers are observed. Volumetric contraction is undesirable during the fabrication of heart valves as it makes them unsuitable for clinical use.  On the other hand, high collagen content improves the mechanical properties of the tissue. Moreira et al.\ \cite{Moreira_etal_2015} showed that volumetric contraction can be prevented by using textile reinforcement and choosing a tubular instead of semi-lunar heart valve design. Since modeling textile-reinforced tubular valves is the topic of this paper, we will focus on the evolution of collagen density (collagen mass growth) during the maturation process.


Collagen fiber bundles are responsible for most of the load-carrying capacity in soft biological tissues. Therefore, changes in collagen density and orientation significantly alter the mechanical properties of the tissue. Growth drivers in living tissues differ from one tissue type to another depending on the environment and the tissue microstructure \cite{Kuhl_2014, Eskandari_2015}. Finding the relevant growth driving factors is essential to optimize the design of tissue-engineering processes. Various studies found that the enzymatic degradation rate is influenced by the strain applied to the tissue \cite{Huang_Yannas_1977, Wyatt_etal_2009, Siadat_Ruberti_2023}. This makes it possible to control the collagen density in the tissue \cite{Ruberti_Hallab_2005}. With regards to heart valves, Ku et al.\ \cite{Ku_etal_2009} found that collagen synthesis by valvular interstitial cells (VICs) from the aortic valve can be controlled by the level and duration of the applied tissue stretching. Furthermore, Oomen et al.\ \cite{Oomen_etal_2016} studied the mechanical properties of an extensive data set of post-mortem healthy human heart valves for various age groups. The study showed that native human heart valves maintain the same level of stretch along the circumferential direction for different age groups. However, the circumferential stress varies considerably from one age group to another. They concluded that human heart valves maintain stretch homeostasis. These results were later applied to study the influence of growth and remodeling on the evolution of heart valves \cite{Oomen_etal_2018}. With regards to textile-reinforced biohybrid implants, Hermans et al.\ \cite{Hermans_etal_2022} investigated the influence of the scaffold microstructure on the collagen evolution during the maturation process.  

Modeling the evolution of collagen fibers in soft and hard biological tissues has been widely investigated. Baek et al.\ \cite{Baek_etal_2006} introduced a model for collagen fiber deposition during intracranial fusiform aneurysms, where fiber stretching drives collagen evolution. Martufi \& Gasser \cite{Martufi_Gasser_2012} applied a similar approach to model aortic aneurysms. Furthermore, Hadi et al.\ \cite{Hadi_etal_2012} modeled collagen fiber's enzymatic degradation and growth as a function of collagen fiber strain. A similar assumption was applied to study wound healing by Gierig et al.\ \cite{Gierig_etal_2021}. Also, Topol et al.\ \cite{Topol_etal_2021} introduced a model that examines the influence of tissue stretch on collagen fiber density. However, modeling the evolution of collagen fibers for cardiovascular tissue-engineering applications needs to be better investigated. Among the limited work in this area, Szafron et al.\ \cite{Szafron_etal_2019} applied a constrained mixture model to optimize the micro-structural design of tissue-engineered vascular grafts with 3d printed scaffold. In their model, collagen growth is driven by the graft's wall-shear stress. However, in the case of heart valves, experimental investigations from \cite{Oomen_etal_2016, Oomen_etal_2018} showed that growth is driven by the strain of the collagen fibers. Furthermore, in \cite{Szafron_etal_2019}, the collagen evolution depends explicitly on time, making the model not general enough to consider a different process setup.

In this paper, we introduce a simple approach to model the in-vitro maturation of textile-reinforced biohybrid implants. In Section \ref{sec:2}, we introduce an energy-based approach to model the evolution of collagen content during tissue maturation. Although energy-based models were investigated for other biomechanics applications, to our knowledge, a similar approach has not been yet developed to model the maturation of regenerative tissues. A special challenge in applying an energy-based approach to tissue maturation is that at the beginning of the cultivation process, the initial collagen density is zero and hence the corresponding Helmholtz free energy is zero. Also, our experimental investigation showed that collagen growth occurs for unloaded samples. We find the idea of decomposing the collagen growth into a mechanically-driven and biologically-driven part by Szafron et al.\ \cite{Szafron_etal_2019} to be very beneficial in this regard. Then, we apply the concept of structural tensors to model the anisotropic mechanical behavior of the scaffold and collagen fibers. Then Section \ref{sec:3} discusses the finite element implementation where we embed the material model into a special solid-shell finite element formulation with reduced integration. We also show the necessary steps to compute the collagen density. In Section \ref{sec:4}, we experimentally measure the stress-stretch behavior of collagenous regenerative tissues and their corresponding collagen fiber density. The results are used to validate our choice for the Helmholtz free energy function of the collagen fibers and identify the material parameters concerning this part. Furthermore, we perform biaxial tensile testing experiments on the textile scaffold and use the results to identify the values of the material parameters for the textile scaffold. Finally in Section \ref{sec:5}, we use our finite element framework to compute two structural problems. The first problem concerns a pressurized shell construct. In this example, we explore the validity of our model and study sensitivity of the results to the newly introduced modeling parameters. The second problem, we compute the collagen evolution and stress distribution for a tubular-shaped heart valve. 

