\section{Identification of material parameters}
\label{sec:4}

To accurately model the mechanical behavior of the material, it is necessary to identify the material parameters of our constitutive equations. First, mechanical tests are performed on the textile scaffold and tissue-engineered material. Then, we chemically measure the collagen content of the ECM during the cultivation process. In this way, it is possible to characterize the strain energy function in Eq.\ (\ref{eq:2-20}), which introduces a linear correlation between collagen density and the collagen fibers strain energy ${\psi}_{\mathrm{co}}$. Later on, we identify the parameters of the Weibull cumulative distribution function introduced in Eq.\ (\ref{eq:2-11}), which describes the biologically-driven part of the collagen evolution.

\subsection{Electrospun textile scaffold}

The first material investigated here is the textile scaffold. Biaxial tensile testing experiments are performed to measure the mechanical behavior of the scaffold.  The choice for this specific experimental setup is motivated by the deformation behavior in heart valves, where the valve wall is under biaxial tension during the valve closure. In large segments of the heart valve, the strain along the radial direction is significantly higher than the strain along the circumferential direction. The stress-strain behavior of the scaffold is measured for two different setups. In the first setup, equal deformation was applied along the radial direction $u_{1}$ and circumferential directions $u_{2}$ as shown in Fig.\ \ref{fig:3_1}. In the second setup, we apply different displacement boundary conditions along the two directions, with $u_{1} = 3 \, u_{2}$.

%\textcolor{red}{[Add a photo for scaffold material]}

% Figure environment removed

\pgfplotsset{%
	width=0.46\textwidth,
	height=0.45\textwidth
}
% Figure environment removed

We set up a FEM (Finite Element Method) model for the structure. Displacement boundary conditions are defined as shown by the schematic in Fig.\ \ref{fig:3_1b}, and reaction forces are computed. To identify the material parameters, we implemented an optimization process in MATLAB \cite{MATLAB_2019b} that uses the derivative-free optimization function {\it fminsearch}. The Matlab code simultaneously fits the FEM results to the experimental data for the two experimental setups. The parameter values identified by the optimization routine are listed in Tab.\ \ref{table:3_1}. In Fig. \ref{fig:3_2}, we show the mean and standard deviation of the experimental results and the corresponding FEM results.

\begin{table}[H]
	\centering
	\setlength{\tabcolsep}{30pt}
	\begin{tabular}{c c c}
		\toprule
		Parameter & Value & Units \\
		\midrule
		$K^\mathrm{tex,1}_\mathrm{1} $  & $38.51$ & $\mathrm{[kPa]}$ \\
		$K^\mathrm{tex,1}_\mathrm{2} $ & $1.48$ & $\mathrm{[kPa]}$ \\
		$\beta_1 $ & $3$ &  $[-]$ \\
		$\beta_2 $ & $2$ & $[-]$ \\
		$K^\mathrm{tex,2}_\mathrm{1}$ & $214.39$ & $\mathrm{[kPa]}$ \\
		$K^\mathrm{tex,2}_\mathrm{2}$ & $0.0001$  & $\mathrm{[kPa]}$  \\
		$\gamma_1 $   & $4$ & $[-]$ \\
		$\gamma_2 $  & $2$ & $[-]$ \\
		$ K^\mathrm{tex}_\mathrm{coup,1}$  & $183.72$ & $\mathrm{[kPa]}$ \\
		$\delta_1$  & $2$ &$[-]$ \\
		$ K^\mathrm{tex}_\mathrm{coup,2}$ &  $58.71$ &  $\mathrm{[kPa]}$ \\
		$\delta_2$ & $3$ & $[-]$ \\
		$K^\mathrm{tex}_\mathrm{coup,ani}$  & $571.83$ & $\mathrm{[kPa]}$ \\
		 $\xi $  & $12$ & $[-]$ \\
		\bottomrule
	\end{tabular}
	\caption{Identified material parameters of the textile scaffold constitutive model.}
\label{table:3_1}
\end{table}

\subsection{Tissue-engineered construct}

The second constituent investigated here is the regenerative collagenous tissue. The tissue is produced by an in-vitro maturation process in a controlled bio-reactor. We fabricated rod-shaped samples where deformation is constrained from both ends of the rod, as shown in Fig.\ \ref{fig:3_3a}. Mechanical loads were not applied during the entire cultivation period. Images taken by 2-photon microscopy have shown that this experimental setup leads to fabricating a tissue with highly oriented collagen fibers. 
% Figure environment removed
There are several approaches to measuring the change in collagen content during the maturation process. Among these approaches is to take microscopy images during the maturation process and then quantify the relative change in collagen content using image analysis techniques. Another approach is to chemically quantify the collagen content using a method called {\it hydroxyproline assay}. Chemical quantification is especially valuable since we can accurately measure the absolute collagen density in the tissue. Furthermore, to measure the changes in the tissue's mechanical behavior during the maturation process, we perform tensile test experiments after 14, 21 and 28 days of maturation. By quantifying the collagen density and measuring corresponding stress-stretch curves, we can assess the validity of our choice for the energy function $\mathit{\psi}_{\mathrm{co}}$ introduced in Eq.\ (\ref{eq:2-20}), where we assume that  $\mathit{\psi}_{\mathrm{co}}$ is linearly dependent on the relative collagen density $ \rho^{0}_{\mathrm{co}} /\rho_{\mathrm{co, f}}$. This requires finding a set of material parameters that accurately describes the tensile behavior of the tissue after 14, 21 and 28 days of maturation. 

\pgfplotsset{%
	width=0.45\textwidth,
	height=0.42\textwidth
}
% Figure environment removed

\begin{table}[H]
	\centering
	\setlength{\tabcolsep}{6pt}
	\begin{tabular}{c c c}
		\toprule
		Duration $\mathrm{[days]}$ & Average collagen density $\mathrm{[\SI{}{\micro\gram} / \SI{}{\micro l}]}$ & Relative collagen density $[-]$ \\
		\midrule
		$ 14 $  & $23.46$ & $0.6060$ \\
		$ 21 $ & $32.35$ & $0.8357$ \\
		$ 28 $ & $38.71$ &  $1.0$ \\
		\bottomrule
	\end{tabular}
	\caption{Collagen density values measured after different periods of tissue maturation using hydroxyproline assay.}
	\label{table:3_2}
\end{table}

We set up a FEM simulation of a soft collagenous rod to identify the material parameters. In our simulation, we use an idealized geometry with equal rod width and thickness. The rod width chosen in the simulation is equal to the average width of experimentally tested samples. The model is then fitted to the experimental data provided by tensile test specimens for the tissue after 14, 21 and 28 days of maturation. In the simulation, we only vary the relative collagen density  $\rho^{0}_{\mathrm{co}} /\rho_{\mathrm{co, f}}$ where the values are listed in Tab.\ \ref{table:3_2}; other material parameters listed in Tab.\ \ref{table:3_3} remain the same in all the simulations. The value of  $\kappa$ was chosen to be $\kappa = 0$ since the microscopy images show that collagen fibers are uni-axially oriented for rods produced in this specific setups.

 It is important to emphasize that these mechanical tests were performed for six different samples cultivated under the same conditions, with two samples tested for each time points. As a result the collagen content varies from one sample to another, leading to slightly different stress-strain behavior. Furthermore, due to the destructive nature of these tests, the collagen densities listed in Tab.\ \ref{table:3_2} were measured for a different set of samples cultivated under the same conditions. Then, we used the mean value of the collagen density in our FEM computations. Despite this uncertainty, we can see in Fig.\ \ref{fig:3_4} that the constitutive model and the material parameters listed in Tab.\ \ref{table:3_3} can reasonably describe the mechanical behavior of tissue-engineered soft material at different time points during the maturation process.

\begin{table}[H]
	\centering
	\setlength{\tabcolsep}{30pt}
	\begin{tabular}{c c c}
		\toprule
		Parameter & Value & Units \\
		\midrule
		$ \lambda $  & $10.0$ & $\mathrm{[MPa]}$ \\
		$ \mu  $ & $0.05$ & $\mathrm{[MPa]}$ \\
		$ k_1 $ & $0.825$ &  $\mathrm{[MPa]}$ \\
		$ k_2 $ & $4.0$ & $[-]$ \\
		$ \kappa $ & $0.0$ & $[-]$ \\
		\bottomrule
	\end{tabular}
	\caption{Identified material parameters of the collagenous tissue.}
	\label{table:3_3}
\end{table}

\pgfplotsset{%
	width=0.6\textwidth,
	height=0.5\textwidth
}
% Figure environment removed

The increase in the relative collagen density during the maturation process is plotted in Fig.\ \ref{fig:3_5} where the black dot represents the average collagen density, and the black lines represent the ranges of density values measured experimentally by hydroxyproline assay. The cumulative increase in collagen density deposition can be accurately modeled by the Weibull cumulative distribution function introduced in Eq.\ (\ref{eq:2-11}). The parameters for the plotted Weibull cumulative distribution are listed in Tab.\ \ref{table:3_4}. 

\begin{table}[H]
	\centering
	\setlength{\tabcolsep}{30pt}
	\begin{tabular}{c c c}
		\toprule
		Parameter & Value & Units \\
		\midrule
		$ \tau $  & $14.21$ & $\mathrm{[days]}$ \\
		$ h  $ & $1.65$ & $[-]$ \\
		\bottomrule
	\end{tabular}
	\caption{Parameters identified for the Weibull distribution curve.}
	\label{table:3_4}
\end{table}