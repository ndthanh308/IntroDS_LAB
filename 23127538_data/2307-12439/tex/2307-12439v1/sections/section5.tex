\section{Results and discussion}
\label{sec:5}

Our next step is to apply the material model and finite element implementation explained in Sections \ref{sec:2} and \ref{sec:3} to compute structural mechanics examples. To better understand our model's characteristics, we evaluate the maturation of a pressurized shell construct. Collagen growth in soft cardiovascular tissues under pressure loads, such as the formation of atheromatous fibrous caps or collagen accumulation due to aortic aneurysm, are well investigated phenomena. This makes it possible to qualitatively assess the validity of our model. In the next step, we compute the collagen evolution in a tubular-shaped biohybrid heart valve. As explained in Section \ref{sec:1}, the tubular design was introduced by Weber et al.\ \cite{Weber_etal_2014} to improve the reliability of the fabrication and suturing process of the valve. Since this special design is not widely investigated in the literature, applying our model can give us a better insight on the growth behavior during the tissue maturation process. Furthermore, from a numerical point of view, the tubular valve example is highly challenging, since the structure shows a snap-through behavior under quasi-static loading conditions. In addition to that, it is necessary to consider the contact conditions.

We perform the finite element computations using the software package FEAP \cite{Taylor_2020}. To visualize our results, we use ParaView \cite{Ahrens_2005} to generate the contour plots and Matplotlib \cite{Hunter_2007} for curve plotting. 

\subsection{Pressurized shell construct}

% Figure environment removed
\vspace{-0.8cm}

In this problem, we consider a tissue-engineered biohybrid shell construct. During the in-vitro maturation process a constant pressure load is applied, as illustrated in Fig.\ \ref{fig:5_1}. The construct length is $l = 20 \; \mathrm{mm}$ and width $w = 6 \; \mathrm{mm}$. The thickness is $t = 0.3 \; \mathrm{mm}$. The tissue is fixed at both ends, and a constant pressure load of $P = 2 \; \mathrm{kPa}$ is applied on the bottom surface.


Initially, the mechanical characteristics of the construct are dominated by the mechanical behavior of the scaffold. However, during the maturation process, the density of collagen increases which affects the overall behavior of the structure. We investigated the mechanical behavior of the scaffold and soft collagenous tissue in Section \ref{sec:3}. The identified material parameters are listed in Tab.\ \ref{table:3_1} and \ref{table:3_3}. These parameters are used to compute this example. However, to consider that collagen fibers are not fully uni-axially aligned within the scaffold, we change the value of the parameter $\kappa$ to $\kappa = 0.15$, with the mean orientation of the collagen fibers defined to be along the x-axis. 

The last set of parameters needed for the computations are parameters that define the collagen density evolution during the maturation process. For the parameters $\tau$ and $h$, which describe the Weibull function, we use the values listed in Tab.\ \ref{table:3_4}. The values used for the remaining parameters and the corresponding references are listed in Tab.\ \ref{table:5_1}.
\begin{table}[H]
	\centering
	\setlength{\tabcolsep}{30pt}
	\begin{tabular}{c c c c}
		\toprule
		Parameter & Value & Units & Reference \\
		\midrule
		$ a_{1} $  & $5 \times 10^{-4}$ & $\mathrm{[\SI{}{\micro\gram} / cells]}$ & own \\
		$ a_{2} $  & $5 \times 10^{-7}$ & $\mathrm{[mm^{3}/cells/day]}$ & own \\
		$ {\psi}_{\mathrm{crit}} $  & $2 \times 10^{-5}$ & $\mathrm{[J/\SI{}{\micro\gram}]}$ & own \\
		$ {\rho}_{\mathrm{th}} $ & $10.0$ & $\mathrm{[\SI{}{\micro\gram} / mm^{3}]}$ & own \\
		$ {\rho}_{\mathrm{co, f}} $ & $38.71$ & $\mathrm{[\SI{}{\micro\gram} / mm^{3}]}$ & Table \ref{table:3_2} \\
		$ c_{\mathrm{cell}}  $ & $15 \times 10^{3}$ & $\mathrm{[cells/mm^{3}]}$ & \cite{Hermans_etal_2022} \\
		\bottomrule
	\end{tabular}
	\caption{Values for additional modeling parameters.}
	\label{table:5_1}
\end{table}

% Figure environment removed

The computations are performed using the solid-shell element with reduced integration Q1STs \cite{Reese_2007, Barfusz_2021b}. A mesh convergence study is performed using three mesh refinement levels with 240, 480 and 960 elements, as shown in Fig.\ \ref{fig:5_11}. At each refinement level; the boundary value problem was computed using three and five Gauss points defined along the thickness of each element. By plotting the maximum deflection along the z-direction against the time, as shown in Fig.\ \ref{fig:5_3}, we observe an excellent mesh convergence behavior, even with a coarse mesh with only 240 elements, where each element contains three Gauss points. Therefore, we use for our simulations the mesh in Fig.\ \ref{fig:5_11a}, which consists 240 elements, with three Gauss points in each element.

\pgfplotsset{%
	width=0.6\textwidth,
	height=0.45\textwidth
}
% Figure environment removed

The contour plots in Fig.\ \ref{fig:5_2} show the collagen density distribution at different time points during the maturation process. We can observe that collagen density gradually increases during the maturation process. Collagen fibers accumulate in regions which are highly strained along the fiber direction. The results here resemble the accumulation of collagen in abdominal aortic aneurysm \cite{Takahashi_2023}. The accumulation of collagen fibers increases the stiffness of the construct. As a result, the maximum deflection of the construct along the z-direction decreases over time, as shown in Fig.\ \ref{fig:5_3}.  


To understand the influence of the parameters introduced in the collagen evolution model, we study the influence of various parameters on the collagen density evolution in the element highlighted in Fig.\ \ref{fig:5_4}. We choose this specific element because it shows the highest collagen concentration in structure. There, we output the average collagen density in the highlighted element. The parameters of interest are $a_{1}$, $a_{2}$, ${\psi}_{\mathrm{crit}}$ and ${\rho}_{\mathrm{th}}$, which are listed in Tab.\ \ref{table:5_1}. We vary the value of each parameter one at a time and plot the results for three different values. The results are plotted in Fig.\ \ref{fig:5_5}.  

% Figure environment removed
%\vspace{-1cm}



% Figure environment removed

\pgfplotsset{%
	width=0.495\textwidth,
	height=0.45\textwidth
}

% Figure environment removed

The parameters $a_{1}$ and $a_{2}$ influence the biologically and mechanically-driven part of the collagen growth, respectively. We can see in Fig.\ \ref{fig:5_5a} that a higher value for $a_{1}$ leads to a steeper increase in collagen density at the initial period of the process and, consequently to a higher collagen density at the end of the process. Similarly, in Fig.\ \ref{fig:5_5b} we see a similar pattern with higher $a_{2}$ values leading to a higher collagen densification rate. However, it is evident from both figures that in highly strained regions, our results are more sensitive to $a_{2}$ than $a_{1}$ since mechanical stimulation is the dominant factor for collagen growth in these regions. In Fig.\ \ref{fig:5_5c} we see the influence of ${\rho}_{\mathrm{th}}$ on the S-shaped curve and and the collagen density's saturation level. As expected, a lower value for ${\rho}_{\mathrm{th}}$ shifts the steep part of the curve to the left and leads to reaching the saturation level at a lower collagen density. We also looked at the influence of ${\psi}_{\mathrm{crit}}$ on the results in Fig.\ \ref{fig:5_5d}. There, we can also see that a lower value of ${\psi}_{\mathrm{crit}}$ leads to higher collagen density. Such an outcome is reasonable, as we can see in Eq.\ (\ref{eq:2-14}) that the densification rate is a function of the difference between collagen fiber strain energy ${\psi}_{\mathrm{co, m}}$ and the threshold value ${\psi}_{\mathrm{crit}}$. Furthermore, ${\psi}_{\mathrm{crit}}$ shows up in the denominator of Eq.\ (\ref{eq:2-14}), making our results highly sensitive to its value.

The material parameters of the constitutive model can also influence the collagen evolution behavior. Here we only look at the influence of collagen fiber alignment on the results. As we can see in Eq.\ \ref{eq:2-19}, the collagen fiber stretching $\lambda_{\mathrm{co}}$ is a function of the structural tensor $\mathbf{H}$. A lower value of $\kappa$ means collagen fibers are highly oriented along the mean orientation vector ${\mathbf a}$, which leads to a higher value for the fiber stretch $\lambda_{\mathrm{co}}$ and the fiber strain $E_{\mathrm{co}}$. This leads to a higher rate of collagen accumulation due to mechanical stimulation, as shown in Fig.\ \ref{fig:5_6}.

\pgfplotsset{%
	width=0.55\textwidth,
	height=0.45\textwidth
}
% Figure environment removed

\subsection{Tubular-shaped heart valve}

The tubular design was proposed to avoid the complications of synthesizing and suturing semilunar-shaped heart valves. Similar to the previous example, we study the in-vitro cultivation of textile reinforced heart valves. 

The valve's shape is shown in Fig.\ \ref{fig:5_7a} and its dimensions are indicated in Fig.\ \ref{fig:5_7b}, with diameter $d = 23 \; \, \mathrm{mm}$, length $l = 18 \; \mathrm{mm}$ and thickness $t = 0.3 \; \mathrm{mm}$. The valve has three  leaflets which are tied using a commissural suture as illustrated in Fig.\ \ref{fig:5_8a}. Due to the symmetry of the leaflets, it is sufficient to compute only one-half leaflet (one-sixth of the structure) as indicated in Fig.\ \ref{fig:5_8a}. We consider suture points in our boundary value problem by defining two fixed nodes, as indicated in Fig.\ \ref{fig:5_8b}. To avoid excessive distortion of the elements close to the commissural suture, we do not apply the symmetry boundary conditions on the nodes at the top $3 \; \mathrm{mm}$ of the left side. Therefore, symmetry boundary conditions are applied only on $15 \; \mathrm{mm}$ of the left side, as illustrated in Fig.\ \ref{fig:5_8b} \cite{Sesa_2023}. A similar approach was applied in the work of Stapleton et al.\ \cite{Stapleton_etal_2015} and Sodhani et al.\ \cite{Sodhani_etal_2017,Sodhani_etal_2018a, Sodhani_etal_2018b}. We discretize the structure using 1944 solid-shell elements Q1STs, with 54 elements along the longitudinal direction and 36 elements along the circumferential direction. Each element uses three Gauss points through the thickness. 

% Figure environment removed

% Figure environment removed

The leaflet contact is modeled in the same way as in \cite{Stapleton_etal_2015}, where we defined a rigid wall along the symmetry surface of the cylinder. Then, we define the contact surface to be between the valve leaflet's inner-surface and the rigid wall. We use the standard penalty method in FEAP \cite{Taylor_2020}, where the contact surface of the rigid wall is defined as the master surface and the valve's inner-surface is the slave.

% Figure environment removed

The mechanical deformation of the heart valve is a highly complex fluid-structure interaction problem. However, in this example, we simplify it by modeling the blood pressure as a quasi-static pressure load. The computation is subdivided into two phases. In the first phase, we model the closure of the valve, and then in the second phase, we apply a constant pressure load. The heart valve closure is a physical instability problem, where the structure shows a snap-through behavior. Therefore, we perform the finite element computations using the arc-length method. In the second phase, we perform the computations using a classical force-controlled Newton-Raphson method, where we apply a constant pressure load of $P = 2 \; \, \mathrm{kPa}$. The modeling parameters are the same as in the previous example. However, to avoid numerical instability, we increase the shear modulus value to $\mu = 0.15 \, \mathrm{MPa}$, and use $\lambda = 1.35 \, \mathrm{MPa}$. In addition, we define the mean orientation of the collagen fibers to be along the circumferential direction and the fiber dispersion parameter as $\kappa = 0.15$. 

% Figure environment removed

In the contour plots for the collagen distribution in Fig.\ \ref{fig:5_9}, we can observe that collagen density is higher along the inner edge of the leaflets and regions surrounding the suture points. This can be explained by the fact that this structure experiences severe bending along the circumferential direction in these regions. In these results, we find a higher collagen density along the boundaries between the leaflets which is similar to the architecture of collagen fibers in native aortic heart valve leaflets shown by Peskin \& McQueen \cite{Peskin_1994}. However, our results does not show high concentration of collagen fibers along the lower edge of the leaflets. It is also important to point out that the bending behavior in tubular valves differs significantly from the standard semilunar-shaped valves. Furthermore, in biohybrid valves, the textile scaffold contributes considerably to the material's stiffness, while in native heart valves, collagen fibers are the main structural constituent of the tissue. We can also see in Fig.\ \ref{fig:5_10} the contours of the Cauchy stresses along the circumferential direction $\sigma_{\mathrm{\theta}}$ and the longitudinal direction $\sigma_{\mathrm{zz}}$. The stress contour shows higher circumferential stresses on the boundaries between the leaflets, due to the severe bending. We also see very high longitudinal stresses close to the commissural suture and the on bottom edge of the valve.


\section{Conclusion and outlook}
\label{sec:6}

 In this work, we developed a finite element framework that can predict the evolution of collagen density during the maturation process of textile-reinforced biohybrid implants. The model successfully applies an energy based approach to tissue-engineered materials, where the collagen density evolution is driven by the mass specific Helmholtz free energy of the collagen fibers. We also show that this energy based approach satisfies the second law of thermodynamics. During the cultivation of tissue-engineered materials, the initial collagen density is zero and accordingly the volume specific Helmholtz free energy is also zero. This characteristic poses a unique challenge to apply collagen densification models previously developed for other applications such as bone or cardiovascular tissue remodeling. This was solved by using the assumption that collagen evolution can be split into a part driven by biological factors and a second part driven by mechanobiological stimulation. In this way, our approach satisfies experimental results which showed collagen evolution for unloaded tissues and it allows us to model collagen evolution in a thermodynamically consistent manner. Later we used experimental data to identify most of the material parameters in our constitutive model. On the other hand, due to the lack of experimental data on collagen evolution caused by mechanical stimulation, the corresponding parameters are chosen to give physiologically reasonable results. The results produced with such a simple material model are qualitatively in accordance with collagen distribution observed in similarly loaded native tissues. 

To fully validate the model, we plan to perform a set of experiments to quantify collagen evolution under mechanical loading under in-vitro and in-situ experimental settings. Furthermore, the coupling effect between the ECM and the textile scaffold shall be quantified in the future. Another aspect that needs to be investigated is the influence of hemodynamics on tissue growth. This is especially challenging because it requires bridging two different time scales: the short-term scale for cardiac cycles (seconds) and the long-term time scale for tissue maturation (days to weeks). Work on fluid-solid-growth models can be a good starting point \cite{Figueroa_2009}.


The model can be extended by introducing additional material invariants to consider fiber-reinforced scaffolds. The extended model would allow us to study how the fiber reinforcement architecture influences collagen growth. With such a model we can optimize the fiber layout to achieve a mechanical behavior comparable to native human tissues. Other design variables such as the implant's geometry or the loading conditions during the maturation process can also be interesting to study. In our framework, we implemented the material and element routines using automatic differentiation techniques. This can be beneficial to perform sensitivity analysis studies since automatic differentiation can potentially improve the efficiency and reduce the computational errors that arise from the standard semi-analytic approaches often used in structural mechanics \cite{Korelc_2016}.

In this paper, the focus is on mass growth since inelastic volumetric deformation in textile-reinforced tissues is negligible. However, unreinforced biological tissues experience mass, volumetric growth, and fiber reorientation. The previous work on volumetric growth from Lamm et al.\ \cite{Lamm_2021, Lamm_2022} and fiber reorientation from Holthusen et al.\ \cite{Holthusen_2023} considered biological tissues with constant collagen content. A future goal would be to formulate an overarching model capable of describing the simultaneous processes of collagen deposition, fiber reorientation, and volumetric growth based on the work on these publications and the model proposed here.

\section{Acknowledgment}
S.\ Jockenhövel, S.\ Reese, and T.\ Brepols gratefully acknowledge the financial support provided by the German Research Foundation (DFG) for the subproject "Experimental investigations and modeling of biohybrid heart valves including tissue maturation – from in vitro to in situ tissue engineering" (Project number 403471716, RE 1057/45-1 and RE 1057/45-2) of DFG PAK-961 consortium "Towards a model based control of biohybrid implant maturation". Furthermore, S.\ Reese acknowledges the support granted by the DFG for the project "In-stent restenosis in coronary arteries - in silico investigations based on patient-specific data and meta modeling" (Project number 465213526, RE 1057/53-1). S.\ Reese and T.\ Brepols acknowledge the financial support for the project "Experimental and numerical investigations of laminated, fibre reininforced plastics under crash loading" (Project number 404502442, RE 1057/46-1).
