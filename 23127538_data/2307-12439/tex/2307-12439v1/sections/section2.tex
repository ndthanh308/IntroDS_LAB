\section{Continuum mechanical model}
\label{sec:2}

During the maturation process of tissue-engineered implants, collagen fibers are secreted by VICs, which infuse the textile scaffold. In this section, we introduce a continuum mechanical approach to model the material during the maturation process. The model is described by defining the main constituents of the material and their corresponding constitutive equations. Then we sum up each part's contribution to describe the total response of the material. The main constituents of biohybrid implants are the regenerative tissue and the textile scaffold. The regenerative tissue contains VICs and the ECM. VICs lack mechanical strength; therefore, only the ECM is considered in our constitutive model. The ECM mechanical response is highly influenced by the mass growth of collagen fibers. 

The section is organized in the following way. We start by explaining the kinematic relations. Then, we explain the thermodynamic basis of our model. Afterwards, the collagen density evolution equations are introduced. Finally, define show the constitutive equations used for each constituent.  

\subsection{Kinematics}
To define the kinematic relations of our model, we consider a three-dimensional body that occupies the domain $\Omega_{0}$ with the boundary $\partial \Omega_{0}$ in the reference configuration at a reference time $t = 0$. A material particle within the domain is defined in the reference configuration by the position vector $\mathbf{X}$. The current position vector is denoted by $\mathbf{x} = \phi(\mathbf{X}, t) $. This gives us the deformation gradient tensor
\begin{equation}
\label{eq:2-1}
\mathbf{F} = \frac{\partial \mathbf{x}}{\partial \mathbf{X}}, 
\end{equation}
and the right Cauchy-Green tensor
\begin{equation}
\label{eq:2-2}
\mathbf{C} = \mathbf{F}^{\mathrm{T}} \, \mathbf{F} .
\end{equation}

\subsection{Thermodynamic basis}

An essential aspect of fabricating functional tissue-engineered implants is the production of collagen fibers. At the beginning of the maturation process, the tissue does not contain collagen fibers; then, the valvular interstitial cells (VIC) produce collagen during the maturation. This process leads to a change in the mass of collagen fibers or, in other words, mass growth. The mass growth can be expressed in a continuum model as a change in the material density. The in-vitro maturation process spans several weeks, while the loading and unloading of cardiovascular tissues, such as heart valves, takes less than a second. As a result, mass growth occurs over a significantly longer time scale than the time scale for loading the material. This makes it possible to simplify the balance equations by applying the slow-growth assumption \cite{Goriely_2017}.

From a thermodynamic perspective, mass growth turns our problem into an open system. In our model, we consider mass change due to the evolution of the material density in the reference configuration $\rho^{0}$. To formulate the balance of mass equation for an open system, we introduce the mass flux vector $\mathbf R$ and a local mass source term $R_{0}$. The mass balance equation becomes
\begin{equation}
\label{}
\dot{\rho}^{0} = \mathrm{Div}({\mathbf R} ) + R_{\mathrm{0}},
\end{equation} 
with the shorthand notation $({\bullet})$ denotes the material time derivative. From the slow-growth assumption \cite{Goriely_2017}, the mass flux term becomes $\mathrm{Div}({\mathbf R} ) = 0$. Consequently, the mass balance equation is reduced to $\dot{\rho}^{0} = R_{\mathrm{0}}$, where the density change rate equals the mass source $R_{\mathrm{0}}$. Furthermore, the application of the slow-growth assumption lead to the standard balance of linear momentum equation
\begin{equation}
\label{eq:2-3}
\mathrm{Div}({\mathbf F} \, {\mathbf S}) + {\mathbf b}_{0} = {\mathbf 0},
\end{equation}
where ${\mathbf S}$ is the second Piola-Kirchhoff stress and ${\mathbf b}_{0}$ is the referential body force vector.

To ensure that the second law of thermodynamics is satisfied, an entropy term $S_{0}$ is introduced by Kuhl \& Steinmann \cite{Kuhl_Steinmann_2003}. The Clausius-Duhem inequality becomes 
\begin{equation}
\label{eq:2-4}
\frac{1}{2} \, {\mathbf S} \cdot \dot{{\mathbf C}} - \dot{\psi} + S_{\mathrm{0}} \geqslant 0,
\end{equation}
where $S_{0}$ accounts for both the local entropy production and entropy fluxes through the boundaries of our system. The concept was successfully used in various problems, such as bone remodeling \cite{Kuhl_2003} or volumetric growth of soft tissue \cite{Lamm_2021, Lamm_2022}. Then, we can formulate the Clausius-Duhem inequality as 
\begin{equation}
\label{eq:2-8}
\left( {\mathbf S} - 2 \, \frac{\partial \psi}{\partial \mathbf{C}} \right) \cdot \, \frac{1}{2} \, \dot{{\mathbf C}} + S_{\mathrm{0}} \geqslant 0 .
\end{equation}
By applying the Coleman-Noll procedure \cite{Coleman_Noll_1963}, we get the following term for the second Piola-Kirchhoff stress
\begin{equation}
\label{}
{\mathbf S} = 2 \, \frac{\partial {\psi}}{\partial \mathbf{C}},
\end{equation}

where the total Helmholtz free energy function ${\psi}$ is the sum of the energy contribution from the textile reinforcement ${\psi}_{\mathrm{tex}}$ and the ECM ${\psi}_{\mathrm{ECM}}$ as described by the equation
\begin{equation}
\label{}
{\psi} =  {\psi}_{\mathrm{tex}} + {\psi}_{\mathrm{ECM}} .
\end{equation}

The extracellular matrix contains protein fibers such as collagen and elastin fibers as well as proteoglycans and glycosaminoglycans. Collagen bundles are stiff and highly anisotropic, while other constituents show isotropic mechanical behavior. Collagen bundles are responsible for most cardiovascular tissues' mechanical strength, as in the case of heart valve tissue. Therefore, the extracellular matrix is modeled as an isotropic ground matrix with anisotropic collagen reinforcement. The energetic part ${\psi}_{\mathrm{ECM}}$ is defined as the sum of the contribution from the isotropic matrix ${\psi}_{\mathrm{matrix}}$ and anisotropically oriented collagen bundles ${\psi}_{\mathrm{co}}$. This gives us the relation
\begin{equation}
\label{}
{\psi}_{\mathrm{ECM}} = {\psi}_{\mathrm{matrix}} + {\psi}_{\mathrm{co}},
\end{equation}
for the Helmholtz free energy function ${\psi}_{\mathrm{ECM}}$. Therefore, the total Helmholtz free energy can be described as the sum of the contributions from the textile reinforcement, matrix, and collagen parts as described by the equation
\begin{equation}
\label{eq:2-30}
{\psi} =  {\psi}_{\mathrm{matrix}}(\mathbf{C}) + {\psi}_{\mathrm{co}}(\mathbf{C}, \mathbf{H}, {\rho}^{0}_{\mathrm{co}}) + {\psi}_{\mathrm{tex}}(\mathbf{C}, \mathbf{M}_{1}, \mathbf{M}_{2}) 
\end{equation}
where ${\rho}^{0}_{\mathrm{co}}$ is the collagen density in the reference configuration. It is important to mention that in this model only the density of the collagen changes while the density of other constituents remain constant. The evolution equations for ${\rho}^{0}_{\mathrm{co}}$ are introduced in Section \ref{subsec:2-3}. Furthermore, we introduce the generalized structural tensor $\mathbf{H}$, and the structural tensors $\mathbf{M}_{1}$ and $\mathbf{M}_{2}$, which are defined in the reference configuration. 

The anisotropic behavior of the collagen fibers is introduced into the constitutive model using the structural tensor $\mathbf{H}$. One unique aspect to consider while modeling the anisotropic behavior of collagen fibers is that the fibers are not uni-oriented, but they show dispersed orientation over a range of angles. The level of fiber dispersity needs to be considered in the constitutive model. Here, we apply the concept of generalized structural tensors introduced by Gasser et al.\ \cite{Gasser_etal_2006} to model collagen fibers in arteries. To account for fiber dispersity, an additional material parameter $\kappa$ is introduced. We define the mean orientation of the collagen fibers in the reference configuration by the vector ${\mathbf a}$. This leads to the generalized structural tensor
\begin{equation}
\label{eq:2-18}
\mathbf{H} = \kappa{\bf I} + (1 - 3 \kappa)({\mathbf a} \otimes {\mathbf a}),
\end{equation}
where the value of $\kappa$ varies within the range $ 0 \le \kappa \le \frac{1}{3} $, with $\kappa = 0$ in the case of uni-oriented collagen fibers, and $\kappa = \frac{1}{3}$ in the case of isotropically dispersed fibers.

From the structural tensor $\mathbf{H}$ and the right Cauchy-Green tensor $\mathbf{C}$, the square of collagen stretching $\lambda_{\mathrm{co}}$ is computed by
\begin{equation}
\label{eq:2-19}
\lambda_{\mathrm{co}}^{2} = \mathrm{tr}(\mathbf{C} \, \mathbf{H}),
\end{equation}
and the strain along the fiber direction reads
\begin{equation}
\label{ }
E_{\mathrm{co}} = \lambda_{\mathrm{co}}^{2} - 1.
\end{equation}

The scaffold is an orthotropic material made out of non-woven textiles. The orthotropic behavior is modeled using structural tensors defined in the reference configuration. In the case of orthotropic materials, there are two anisotropic material orientations to consider. These orientations are defined by the vectors ${\bf n}_{1}$ and ${\bf n}_{2}$. Then, we use the dyadic product of the directional vectors to compute the corresponding structural tensors
\begin{equation}
\label{eq:2-24}
\mathbf{M}_{1} = {\bf n}_{1} \otimes {\bf n}_{1} \; \; \text{and} \; \; \mathbf{M}_{2} = {\bf n}_{2} \otimes {\bf n}_{2} .
\end{equation}

The strain energy function ${\psi}_{\mathrm{tex}}$ is defined in terms of material invariants \cite{Reese_etal_2001, Reese_etal_2021}. In this model, ${\psi}_{\mathrm{tex}}$ is a function of the following invariants
\begin{equation}
\label{eq:2-25}
\begin{gathered}
\mathit{I}_{\mathrm{1}} = \mathrm{tr}(\mathbf{C}),  \\
\mathit{I}^{\mathrm{tex}}_{\mathrm{2}} = \mathrm{tr}(\mathbf{C} \, \mathbf{M}_{1}), \;  \mathit{I}^{\mathrm{tex}}_{\mathrm{3}} = \mathrm{tr}(\mathbf{C}^{2} \, \mathbf{M}_{1}), \\
\mathit{I}^{\mathrm{tex}}_{\mathrm{4}} = \mathrm{tr}(\mathbf{C} \, \mathbf{M}_{2}) \; \text{and} \; 
\mathit{I}^{\mathrm{tex}}_{\mathrm{5}} = \mathrm{tr}(\mathbf{C}^{2} \, \mathbf{M}_{2}) .
\end{gathered}
\end{equation}

The next step is to compute the second Piola-Kirchhoff stresses. Similar to the energy term in Eq.\ \ref{eq:2-30}, the total second Piola-Kirchhoff stress $\mathbf{S}$ can be computed by summing up the stresses of each constituent which leads to the relation
\begin{equation}
\label{eq:2-27}
\mathbf{S} =  \mathbf{S}_{\mathrm{matrix}} + \mathbf{S}_{\mathrm{co}} + \mathbf{S}_{\mathrm{tex}}.
\end{equation}

The second Piola-Kirchhoff stress of the textile part $\mathbf{S}_{\mathrm{tex}}$ is computed by taking the partial derivative of the Helmholtz free energy ${\psi}_{\mathrm{tex}}$ with respect to the right Cauchy-Green tensor $\mathbf{C}$ which can be written as 
\begin{equation}
\label{eq:2-28}
\mathbf{S}_{\mathrm{tex}} = 2 \, \frac{\partial {\psi}_{\mathrm{tex}}}{\partial \mathbf{C}},
\end{equation}
whereas the second Piola-Kirchhoff stress for the matrix part reads
\begin{equation}
\label{eq:2-29}
\mathbf{S}_{\mathrm{matrix}} = 2 \, \frac{\partial {\psi}_{\mathrm{matrix}}}{\partial \mathbf{C}}.
\end{equation}

The evaluation of the second Piola-Kirchhoff stress of the collagen part ${\mathbf S}_{\mathrm{co}}$ requires considering the evolution of the referential collagen density ${\rho}^{0}_{\mathrm{co}}$. The Helmholtz free energy of the collagen fibers ${\psi}_{\mathrm{co}}$ per unit volume can be expressed in terms of the collagen density in the reference configuration ${\rho}^{0}_{\mathrm{co}}$ and the corresponding free energy per unit mass ${\psi}_{\mathrm{co, m}}$, i.e.
\begin{equation}
\label{eq:2-5}
{\psi}_{\mathrm{co}} =  {\rho}^{0}_{\mathrm{co}} \, {\psi}_{\mathrm{co, m}} .
\end{equation}

By taking the time derivative of ${\psi}_{\mathrm{co}}$ in Eq.\ (\ref{eq:2-5}), we obtain the expression for the energy rate
\begin{equation}
\label{eq:2-6}
\dot{\psi}_{\mathrm{co}} =  \dot{\rho}^{0}_{\mathrm{co}} \, {\psi}_{\mathrm{co, m}} \, + \, {\rho}^{0}_{\mathrm{co}} \, \dot{\psi}_{\mathrm{co, m}}.
\end{equation}

Then with applying the chain rule of differentiation, the energy rate $\dot{\psi}_{\mathrm{co}}$ can be rewritten in the following way:
\begin{equation}
\label{eq:2-7}
\dot{\psi}_{\mathrm{co}} =  \frac{\partial {\rho}^{0}_{\mathrm{co}}}{\partial \mathbf{C}} \cdot \dot{{\mathbf C}} \, {\psi}_{\mathrm{co, m}} + {\rho}^{0}_{\mathrm{co}} \, \frac{\partial {\psi}_{\mathrm{co, m}}}{\partial \mathbf{C}} \cdot \dot{{\mathbf C}}.
\end{equation}
Later, we insert Eq.\ \ref{eq:2-7} into Eq.\ \ref{eq:2-8} to get the following expression for the second Piola-Kirchhoff stress of the collagen part
\begin{equation}
\label{eq:2-9}
{\mathbf S}_{\mathrm{co}} = 2 \, \left(\frac{\partial {\rho}^{0}_{\mathrm{co}}}{\partial \mathbf{C}} \, {\psi}_{\mathrm{co, m}} + {\rho}^{0}_{\mathrm{co}} \, \frac{\partial {\psi}_{\mathrm{co, m}}}{\partial \mathbf{C}} \right).
\end{equation}

The second Piola-Kirchhoff stress does not have a clear physical interpretation. This makes it necessary to transform it into physically meaningful stress measures. The stress measures that we will use in this paper are the first Piola-Kirchhoff stress $\mathbf{P} = \mathbf{F} \, \mathbf{S}$ and the Cauchy stress $\boldsymbol{\sigma} = J^{-1} \, \mathbf{P} \, \mathbf{F}^{T}$, where $J = \mathrm{det}(\mathbf{F})$, and it represents the volumetric change.

\subsection{Collagen evolution equations}
\label{subsec:2-3}

In our experiments, we observed that during the maturation, collagen fibers are secreted even for unloaded tissues. Furthermore, work in the literature over decades shows that mechanobiological stimulation plays a major role in collagen growth. A reasonable way to consider these factors is to decompose the total rate of collagen density change $\dot{\rho}^{0}_{\mathrm{co}}$ into two parts. The first part considers collagen growth due to bio-chemical factors $\dot{\rho}^{0}_{\mathrm{bio}}$, and the second part considers the growth due to mechanobiological stimulation $\dot{\rho}^{0}_{\mathrm{mech}}$ \cite{Szafron_etal_2019}. This gives us the following expression for the collagen density change rate
\begin{equation}
\label{eq:2-10}
\dot{\rho}^{0}_{\mathrm{co}} = \dot{\rho^{0}}_{\mathrm{bio}} + \dot{\rho}^{0}_{\mathrm{mech}}, 
\end{equation}
where the quantities $\dot{\rho}^{0}_{\mathrm{co}}$, $\dot{\rho}^{0}_{\mathrm{bio}}$ and $\dot{\rho}^{0}_{\mathrm{mech}}$ are defined in the reference configuration.

Experimental measurements of collagen density for unloaded samples showed that a collagen density increase with respect to maturation time follows an S-shaped curve. This can be described by the following Weibull cumulative distribution function
\begin{equation}
\label{eq:2-11}
{\alpha}_{\mathrm{bio}}(t) = 1 - e^{-(t / \tau)^{h}} ,
\end{equation}
where $\tau$ is the half-time and $h$ is a parameter that controls the curve's steepness. Compared to other types of S-shaped curves, Weibull cumulative distribution function is more suited to our model because: (i) it satisfies the condition that ${\alpha}_{\mathrm{bio}} = 0$ when $t = 0$ and (ii) it can describe curves with positively skewed first derivative. By taking the time derivative of ${\alpha}_{\mathrm{bio}}$, we get the following expression
\begin{equation}
\label{eq:2-12}
\dot{\alpha}_{\mathrm{bio}}  = \frac{h}{\tau}\, e^{-(t / \tau)^{h}}  \, \left(\frac{t}{\tau}\right)^{h - 1}.
\end{equation}

From Eq.\ \ref{eq:2-12} we can describe the collagen density evolution for the biologically-driven part by the following equation
\begin{equation}
\label{eq:2-13}
\dot{\rho}^{0}_{\mathrm{bio}} = a_{1} \, c_{\mathrm{cell}} \, \dot{\alpha}_{\mathrm{bio}} .
\end{equation}
where $c_{\mathrm{cell}}$ is the valvular interstitial cell density and $a_{1}$ ($\mathrm{\SI{}{\micro\gram} / cells}$) is an additional material parameter.

In the second part, we consider collagen growth rate is driven by the stretching of the collagen fibers. To formulate this in a thermodynamically consistent  manner, we choose the collagen fiber strain energy per unit mass ${\psi}_{\mathrm{co, m}}$ to be the driving factor for the density change.  In several works, the strain energy was assumed to be the driving factor for collagen mass growth. See \cite{Waffenschmidt_etal_2012} for hard tissue such as bone and \cite{Saez_etal_2013} for hypertension in soft tissue. However, the strain energy of the collagen fibers is a function of the fiber density, which means that in cases where the initial collagen density is zero, the growth rate will remain zero. Therefore splitting the growth rate as we described in Eq.\ (\ref{eq:2-10}) provides a straightforward and physically motivated remedy for this particular problem. 

Here we assume that mechanical stimulation occurs when the energy per mass for collagen fibers ${\psi}_{\mathrm{co, m}}$ exceeds the threshold value ${\psi}_{\mathrm{crit}}$, leading to the case-dependent evolution equation
\begin{equation}
\label{eq:2-14}
\dot{\rho}^{0}_{\mathrm{mech}}  = 
\begin{cases}
a_{2} \, c_{\mathrm{cell}} \, f_{\mathrm{mech}}  \, {\rho}^{0}_{\mathrm{co}} \, \frac{({\psi}_{\mathrm{co, m}} - {\psi}_{\mathrm{crit}})}{{\psi}_{\mathrm{crit}}} ,
&        {\psi}_{\mathrm{co, m}} \ge  {\psi}_{\mathrm{crit}},\\
0  , &       {\psi}_{\mathrm{co, m}} <  {\psi}_{\mathrm{crit}},
\end{cases} 
\end{equation}
where we introduce the material parameter $a_{2}$ ($\mathrm{mm^{3}/cells/day}$).

The collagen density evolution depends on the local density of collagen fibers. To describe an S-shaped evolution of the collagen density, we introduce the exponential decay function 
\begin{equation}
\label{eq:2-15}
f_{\mathrm{mech}} =  e^{-({\rho}^{0}_{\mathrm{co}} / \rho_{\mathrm{th}})},
\end{equation}
where $\rho_{\mathrm{th}}$ is the parameter that controls the saturation level of the collagen density. The value of $f_{\mathrm{mech}}$ is higher when collagen density is significantly lower than $\rho_{\mathrm{th}}$, while for collagen densities significantly higher than $\rho_{\mathrm{th}}$, the value of $f_{\mathrm{mech}}$ is close to zero. 


\subsection{Particular choice of the Helmholtz free energy}
The next step is to define the Helmholtz energy per unit volume for each part in the Eq.\ \ref{eq:2-30}. The first part is for the isotropic matrix is described by a compressible Neo-Hookean material law, which is expressed by
\begin{equation}
\label{eq:2-16}
{\psi}_{\mathrm{matrix}} = \frac{\mu}{2}  (\mathrm{tr}({\mathbf C})  -3) - \mu \mathrm{ln}(J) + \frac{\lambda}{4}(J^{2} - 1 - 2 \mathrm{ln}(J)) .
\end{equation}

The second part in the Eq.\ \ref{eq:2-30} is ${\psi}_{\mathrm{co}}$. Collagen bundles in soft biological tissues show exponential stress-strain behavior. This hyperelastic material behavior is prevalent in cardiovascular tissues such as arteries and heart valves. We model this behavior using a Fung-type material model with an exponential strain function \cite{Fung_1990, Holzapfel_etal_2000}. Here, we choose the following strain energy function introduced by Holzapfel et al.\ \cite{Holzapfel_etal_2000}  which is defined in terms of the fiber strain $E$, i.e. 
\begin{equation}
\label{eq:2-20}
{\psi}_{\mathrm{co}} = \frac{\rho^{0}_{\mathrm{co}}}{\rho_{\mathrm{co, f}}} \begin{cases} \frac{k_{1}}{2 \, k_{2}} \, [e^{\mathrm{k_{2}} \, E_{\mathrm{co}}^2} - 1], \: \; \; \lambda_{\mathrm{co}} \ge 1,  \\ 0, \; \; \; \; \; \; \; \; \; \; \; \; \; \; \: \; \; \: \; \; \: \:\; \; \lambda_{\mathrm{co}} < 1, \end{cases}
\end{equation}
where $\rho^{0}_{\mathrm{co}}$ is the density of the collagen fibers during the maturation process and $\rho_{\mathrm{co, f}}$ is the final density of the collagen fiber at the end of the process. The ratio $\frac{\rho^{0}_{\mathrm{co}}}{\rho_{\mathrm{co, f}}}$ is the relative density of the collagen fibers during maturation. The densities $\rho^{0}_{\mathrm{co}}$ and $\rho_{\mathrm{co, f}}$ are defined in the reference configuration. Furthermore, $k_{1}$ is a stiffness-like parameter and $k_{2}$ is a dimensionless quantity.

The last energy in Eq.\ \ref{eq:2-30} is ${\psi}_{\mathrm{tex}}$. Here we use the constitutive equation introduced by Reese et al.\ \cite{Reese_etal_2001} which reads
\begin{equation}
\label{eq:2-26}
\begin{gathered}
{\psi}_{\mathrm{tex}} = K^\mathrm{tex,1}_\mathrm{1} (\mathit{I}^{\mathrm{tex}}_{\mathrm{2}} - 1)^{\beta_1} + K^\mathrm{tex,1}_\mathrm{2} (\mathit{I}^{\mathrm{tex}}_{\mathrm{3}} - 1)^{\beta_2} \\ + K^\mathrm{tex,2}_\mathrm{1} (\mathit{I}^{\mathrm{tex}}_{\mathrm{4}} - 1)^{\gamma_1}  + 
K^\mathrm{tex,2}_\mathrm{2} (\mathit{I}^{\mathrm{tex}}_{\mathrm{5}} - 1)^{\gamma_2}   
+  K^\mathrm{tex}_\mathrm{coup,1} (\mathit{I}_{\mathrm{1}} - 3)^{\delta_1}  (\mathit{I}^{\mathrm{tex}}_{\mathrm{2}} - 1)^{\delta_1} 
\\  +  K^\mathrm{tex}_\mathrm{coup,2} (\mathit{I}_{\mathrm{1}} - 3)^{\delta_2}  (\mathit{I}^{\mathrm{tex}}_{\mathrm{4}} - 1)^{\delta_2} 
+ K^\mathrm{tex}_\mathrm{coup,ani} (\mathit{I}^{\mathrm{tex}}_{\mathrm{2}} - 1)^{\xi}  (\mathit{I}^{\mathrm{tex}}_{\mathrm{4}} - 1)^{\xi} ,
\end{gathered}
\end{equation} 
where $K^\mathrm{tex,1}_\mathrm{1}$, $K^\mathrm{tex,1}_\mathrm{2}$, $K^\mathrm{tex,2}_\mathrm{1}$, $K^\mathrm{tex,2}_\mathrm{2}$, $K^\mathrm{tex}_\mathrm{coup,1}$, $K^\mathrm{tex}_\mathrm{coup,2}$ and $ K^\mathrm{tex}_\mathrm{coup,ani}$ are stiffness-like material parameters. Furthermore, the exponents ${\beta_1}$, ${\beta_2}$, ${\gamma_1}$, ${\gamma_2}$, ${\delta_1}$, ${\delta_2}$ and ${\xi}$ are dimensionless quantities. These parameters will be later identified in Section \ref{sec:4} using experimental data. 
