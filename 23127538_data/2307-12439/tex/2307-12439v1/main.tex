\documentclass[12pt,a4paper,notitlepage]{article}

\usepackage{amsfonts}
\usepackage{amsmath}
\usepackage{amssymb}
\usepackage{amsthm}
\usepackage{color}
\usepackage[T1]{fontenc}
\usepackage{graphicx}
\usepackage{hyperref}
\usepackage{mathrsfs}
\usepackage{multirow}
\usepackage[numbers]{natbib}
\usepackage[format=hang]{subcaption}
\usepackage{times}
\usepackage{upgreek}
\usepackage[utf8]{inputenc}
\usepackage{pgfplots}
\usepackage{longtable}
\usepackage{booktabs}
\usepackage{algorithm, algpseudocode}
\usepackage{adjustbox}
\usepackage{siunitx}
\usepackage{svg}
\usepackage{tikz}
\usepackage{import}
\usetikzlibrary{patterns,shapes.arrows}
\pgfplotsset{compat=newest}
\usepackage{float}
\pgfplotsset{compat=newest}
%% the following commands are needed for some matlab2tikz features
\usetikzlibrary{plotmarks}
\usetikzlibrary{arrows.meta}
\usetikzlibrary{calc}
\usepackage{grffile}
\AtBeginDocument{\renewcommand{\harvardand}{and}}

\setlength{\textwidth}{160mm}
\setlength{\textheight}{240mm}
\setlength{\topmargin}{-21mm}
\setlength{\oddsidemargin}{-2.5mm}
\topmargin -12mm
\linespread{1.2}
\parindent0mm
\parskip 3.0mm

\date{}

\graphicspath{{./images/}}

\newcounter{remark}
\newtheorem*{remark}{Remark}

\begin{document}

\author{ \large {M. Sesa${}^{ *, \,\dag}$, H. Holthusen${}^{\,\dag}$, L. Lamm${}^{\,\dag}$, C. B\"ohm${}^{\,\ddag}$,} \\ {T. Brepols${}^{\,\dag}$, S. Jockenh\"ovel${}^{\,\ddag}$, S. Reese${}^{\,\dag}$}\\[0.5cm]
\hspace*{-0.1cm}
\normalsize{\em ${}^{\dag}$ Institute of Applied Mechanics, RWTH Aachen
  University,}\\
\normalsize{\em Mies-van-der-Rohe-Str.\ 1, 52074 Aachen, Germany}\\[0.25cm]\
\normalsize{\em ${}^{\ddag}$ Biohybrid \& Medical Textiles, Institute of Applied Medical Engineering,} \\ \normalsize{\em RWTH Aachen University, Forckenbeckstr. \ 55, 52074 Aachen, Germany}\\
%\normalsize{\em  }
\normalsize{\em ${}^{*}$ Corresponding author: mahmoud.sesa@ifam.rwth-aachen.de }
%\normalsize{\em swu@tf.uni-kiel.de}
}
\title{\LARGE Mechanical modeling of the maturation process for tissue-engineered implants: application to biohybrid heart valves}
\maketitle

\small
{\bf Abstract.}
The development of tissue-engineered cardiovascular implants can improve the lives of large segments of our society who suffer from cardiovascular diseases. Regenerative tissues are fabricated using a process called tissue maturation. Furthermore, it is highly challenging to produce cardiovascular regenerative implants with sufficient mechanical strength to withstand the loading conditions within the human body. Therefore, biohybrid implants for which the regenerative tissue is reinforced by standard reinforcement material (e.g.\ textile or 3d printed scaffold) can be an interesting solution. In silico models can significantly contribute to characterizing, designing, and optimizing biohybrid implants. The first step towards this goal is to develop a computational model for the maturation process of tissue-engineered implants. This paper focuses on the mechanical modeling of textile-reinforced tissue-engineered cardiovascular implants. First, we propose an energy-based approach to compute the collagen evolution during the maturation process. Then, we apply the concept of structural tensors to model the anisotropic behavior of the extracellular matrix and the textile scaffold. Next, the newly developed material model is embedded into a special solid-shell finite element formulation with reduced integration. Finally, we use our framework to compute two structural problems: a pressurized shell construct and a tubular-shaped heart valve. The results show the ability of the model to predict collagen growth in response to the boundary conditions applied during the maturation process. Consequently, we can predict the implant's mechanical response, such as the deformation and stresses of the implant.


\vspace*{0.3cm}
{\bf Keywords:} {Regenerative medicine, tissue-engineered heart valves (TEHVs), finite element method, growth modeling, anisotropy}

\normalsize

%%%%%%%%%%%%%%%%%%%%%%%%%%%%%%%%%%%%%%%%%%%%%%%%%%%%%%%%%%%%%%%%%%%%%%%%%%%%%%%%%%%%%%%%%%%%%%%%%%%%%%%%%

\section{Introduction}

Automatic Speech Recognition (ASR) is a key component in processing audio materials such as audio translation and voice assistant \cite{federmann-lewis-2016-microsoft}, and speech information extraction \cite{cho-etal-2021-streamhover}. Typical ASR systems produce chunks of transcription without any text structures such as sentence and phrase boundaries \cite{jones2003measuring}. As a result, it lowers the readability of the generated ASR texts \cite{jones2003measuring} and severely affects the performance of systems for downstream tasks over this type of text, e.g., information extraction \cite{alam2015comparing}. To address this issue, the Punctuation Restoration (PR) task has been added to the ASR systems \cite{tilk2015lstm} to improve the text readability and the performance of downstream tasks for ASR-generated texts such as question answering \cite{pouran-ben-veyseh-etal-2022-behanceqa}, chitchat detection \cite{lai-etal-2022-behancecc}, and tutorial recommendation \cite{Veyseh2022TutorialRF}. The most recent successful work for PR was all built on top of transformer-based PLMs such as BERT \cite{devlin-etal-2019-bert} and ELECTRA \cite{clark2020electra}.

%together with other post-processing tasks such as true-casing \cite{lita-etal-2003-truecasing}.

% Modern PR systems model the PR task as a word-level sequence labeling task, in which each word is labeled with either a punctuation mark or a NULL label. These models depend on  

% Due to its importance, many studies have been conducted for PR in recent years. 
%Early work employed various combinations of both text-feature and audio features \cite{}. More recent studies mainly develop advanced neural network architectures \cite{}, and integrate external knowledge \cite{}. 
Despite such progress, lacking domain-specific training data is still a major obstacle that hinders the research and development of PR systems for real-world applications \cite{lai-etal-2022-behancepr}. We identify two factors accounting for this issue. First, speech topics involve a unique set of keywords as well as slang in spoken languages. The ASR system and PR system without topic knowledge can be severely affected by the shift of topics in the source audio. Second, unlike other tasks where the unlabeled data is created by humans, the input of PR is generated by an ASR system. This creates a unique dependency that must be addressed by the PR model. Consequently, creating cost-effective datasets for a wide range of domains for PR is highly challenging.

Moreover, naive adoption of available punctuated data is problematic. While large-scale punctuated texts corpora are available, they are mostly written texts (REF texts), which are substantially well-punctuated. In contrast, ASR-generated texts (ASR texts) inherit a substantial amount of noise from both spoken language (e.g., verbal pauses) and the transcription process (e.g., word errors). Accordingly, prior studies have shown that a PR model that was trained on REF texts performed poorly on real-world ASR texts \cite{alam-etal-2020-punctuation}. In other words, directly using readily available written texts does not help to improve the PR model.


To overcome these issues, we introduce a novel data generation method to automatically generate large-scale, high-quality labeled data for PR. In particular, instead of manual annotation, we employ a pre-trained language model, namely GPT2 \cite{radford2019language}, to create synthetic labeled data for PR because generative models like GPT2 can generate punctuated texts that can be converted to labeled data for PR easily. Since the GPT2 model was trained on written texts across diverse topics, this leads to two issues that need to be addressed. 

First, the topics in the generated texts are unconstrained, which is suboptimal for some specific applications, such as gaming livestreaming. As such, we propose a method to control the topic of the generated texts. Instead of unconditional text generation, we feed the GPT2 model with an in-topic seed text, which was sampled from an in-topic unsupervised source. Hence, we encourage the GPT2 model to generate more texts within the initial topic. As a result, we can leverage GPT2's knowledge to obtain unlimited in-topic labeled texts for PR.

Second, the disconnection of the GPT2 model and the target PR model might cause a discrepancy between GPT2-generated texts and the target PR text. Therefore, to improve the quality of the  GPT2-generated data for PR, we propose to further finetune the GPT2 model in parallel with the training of the PR model to generate optimal customized texts for PR. Particularly, we propose a meta-learning framework to consider the GPT2 model as a meta-parameter for the training of the PR model, in which the GPT2 model will be fine-tuned based on the performance of the PR model on the development set. A trivial solution is reinforcement learning, where the reward can be calculated directly from the evaluation metrics of the PR model on the development set, e.g., the F1-score. However, obtaining a reliable, fast reward is challenging due to either the small scale of the evaluation or the computational cost of the evaluation that has to be done at every single iteration. To alleviate this issue, we propose a novel reward function that relies on the gradients of the PR model obtained from the generated texts and the development set. Intuitively, a generated sample should have a higher reward if the PR model's gradients derived from the sample follows the PR expected gradients derived from the development set. Toward this end, in each iteration, after generating synthetic PR data, we compute an average gradient of the PR model over the generated data for each training example. Then, we compute another average gradient of the PR model over a sampled subset of the development set. Finally, the reward for each generated sample is computed using the cosine similarity score between the two gradients. We evaluate the effectiveness of the proposed methods on two benchmark datasets for PR. The experiments show that our model outperforms the strongest baseline on both datasets.






% The GPT2 model is fed with a small chunk of in-topic text to generate more in-topic texts for PR.

% To overcome these issues, instead of annotating more data for some particular domains, we introduce a novel domain-agnostic data generation method to generate labeled PR data automatically. The generated data can inherit a wide range of keywords from the PLM, while the PLM-generated text is also customized to the domain of interest. In particular, we propose to employ a generative PLM named GPT2 to create synthetic data for PR. The GPT2 model is trained on the augmented in-topic data, and then the fine-tuned GPT2 is asked to generate a large amount of labeled semi-in-domain data. This data is then combined with the real in-domain PR data to train a PR model. Moreover, this method allows us to control the seed text that is fed to the GPT2 model, hence controlling the topic in the PLM-generated texts. Hence, this method brings the topic gap between the PLM-generated data and the target PR data.

%  As such, it is necessary that the GPT2 and the PR models can interact with each other so that the PR model can guide the GPT2 model to generate spoken texts instead of written texts. In particular, we proposed a reinforcement learning method to allow the PR model to give feedback to the GPT2 model. So, the GPT2 model can generate texts to optimize the PR model's performance.

\section{Related Work}

Early PR studies employed syntactic features and prosodic features\cite{szaszak2019leveraging} to train graphical models such as HMM and CRF \cite{tilk2015lstm}. Recent models for PR employed artificial neural networks to model the PR problem as a sequence-to-sequence problem using various network architectures such as convolutional neural network \cite{che-etal-2016-punctuation}, recurrent neural network \cite{tilk2015lstm,kim2019deep}, and transformer \cite{alam-etal-2020-punctuation}. Pretrained language models stand at the core of the recent PR models. There have been variants of pre-trained language models used for PR such as BERT \cite{fu-etal-2021-improving}, RoBERTa \cite{alam-etal-2020-punctuation,courtland-etal-2020-efficient}, ELECTRA \cite{hentschel2021making,chen2021discriminative}, XLM-RoBERTa \cite{chordia-2021-punktuator}, and funnel-transformer \cite{shi2021incorporating}. Recent advance in training and preprocessing leads to many training techniques such as data augmentation \cite{alam-etal-2020-punctuation}, adversarial training \cite{yi2020adversarial}, multitask learning \cite{lin2020joint,hentschel2021making}, self-training \cite{chen2021discriminative}, two-stage training \cite{fu-etal-2021-improving}, and contrastive learning \cite{huang2021token}. External knowledge was also incorporated into the PR model including external punctuated data \cite{fu-etal-2021-improving}, syntactic features \cite{shi2021incorporating} and acoustic features \cite{zhu2022unified}.
\section{Numerical implementation}
\label{sec:3}

The collagen density evolution is computed by implicitly integrating Eq.\ (\ref{eq:2-10}). A standard Backward-Euler scheme is implemented to compute the collagen density ${\rho}^{0}_{\mathrm{co}}$. The collagen density ${\rho}^{0}_{\mathrm{co}}$ is multiplied by the mass-specific Helmholtz free energy $\mathit{\psi}_{\mathrm{co, m}}$ to obtain the corresponding energy for the collagen fibers per unit reference volume $\mathit{\psi}_{\mathrm{co}}$ as described by Eq.\ (\ref{eq:2-20}). 

\subsection{Time integration}
To apply the time-discretization of the collagen evolution equation, we define the time step as $\Delta t = (t_{\mathrm{n+1}} - t_{\mathrm{n}})$ where $t_{\mathrm{n+1}}$ refers to the current time-step and $t_{\mathrm{n}}$ refers to the previous time-step. Variables from time-step $t_{\mathrm{n}}$ are denoted by the subscript $\mathrm{n}$, and variables from the current time $t_{\mathrm{n+1}}$ are without a subscript. Consequently, the implicit Backward-Euler integration scheme can be expressed as following \begin{equation}
\label{eq:3-1}
{\rho}^{0}_{\mathrm{co}} =  ({\rho}^{0}_{\mathrm{co}})_{\mathrm{n}} + {\Delta t} \, \dot{\rho}^{0}_{\mathrm{co}},
\end{equation}
and the corresponding residual equation is 
\begin{equation}
\label{eq:3-2}
r_{\mathrm{\rho}} = {\rho}^{0}_{\mathrm{co}} - ({\rho}^{0}_{\mathrm{co}})_{\mathrm{n}} - {\Delta t} \, \dot{\rho}^{0}_{\mathrm{co}}  = 0.
\end{equation}

Solving the residual equation using the Newton-Raphson scheme gives us the densification rate $\dot{\rho}^{0}_{\mathrm{co}}$ at the current time-step $t_{\mathrm{n+1}}$. By substituting with $\dot{\rho}^{0}_{\mathrm{co}}$ in Eq.\ (\ref{eq:3-1}), we can find the corresponding collagen density ${\rho}^{0}_{\mathrm{co}}$. Then, ${\rho}^{0}_{\mathrm{co}}$ is inserted in Eq.\ (\ref{eq:2-20}) to compute the corresponding strain energy $\mathit{\psi}_{\mathrm{co}}$. The time-integration steps are explained in Algorithm \ref{densification_algorithm}.

\vspace{0.5cm}
\setlength\belowcaptionskip{1ex}
\begin{algorithm}[H]
	\begin{algorithmic}[1]
		\State Initialize Backward-Euler integration scheme
		\State Input $a_{1}$, $a_{2}$, $c_{\mathrm{cell}}$,  $h$, $\tau$, $t$, ${\psi}_{\mathrm{crit}}$
		\State Output $\rho^{0}_{\mathrm{co}}$ 
		%\For{$t \leq t_{end}$} 
		\State Compute $\dot{\alpha}_{\mathrm{bio}} \gets  \frac{h}{\tau}\, e^{-(t / \tau)^{h}}  \, (\frac{t}{\tau})^{h - 1}$  
		\State Compute $\dot{\rho}^{0}_{\mathrm{bio}} \gets a_{1} \, c_{\mathrm{cell}} \, \dot{\alpha}_{\mathrm{bio}}$
		\State Compute ${\psi}_{\mathrm{co, m}}$
		\If{${\psi}_{\mathrm{co, m}}$ $\geq {\psi}_{\mathrm{crit}} $}
		\State Compute $\dot{\alpha}_{\mathrm{mech}} \gets  e^{-({\rho}^{0}_{\mathrm{co}} / \rho_{\mathrm{th}})}$ 
		\State Compute $\dot{\rho}^{0}_{\mathrm{mech}} \gets a_{2} \, c_{\mathrm{cell}} \, \dot{\alpha}_{\mathrm{mech}} \, {\rho}^{0}_{\mathrm{co}} \, \frac{({\psi}_{\mathrm{co, m}} - {\psi}_{\mathrm{crit}})}{{\psi}_{\mathrm{crit}}} $
		\State $r_{\mathrm{\rho}} \gets {\rho}^{0}_{\mathrm{co}} - ({\rho}^{0}_{\mathrm{co}})_{\mathrm{n}} - {\Delta t} \, \dot{\rho}^{0}_{\mathrm{co}}  = 0 $
		\While  {$|r_{\mathrm{\rho}}| \leq tolerance$}
		\State Compute $\rho^{0}_{\mathrm{co}}$ using Newton-Raphson method
		\EndWhile
		\EndIf
	\end{algorithmic} 
	\caption{Computing collagen fiber density}
	\label{densification_algorithm}
\end{algorithm}


\subsection{Computing stresses and material tangents}
To construct a global finite element system, first we compute the second Piola-Kirchhoff stresses for each constituent and then sum up their individual contribution to get the total second Piola-Kirchhoff stress $\mathbf{S}$ as described by the Eq.\ \ref{eq:2-27}. The computation of second Piola-Kirchhoff stress of the textile scaffold $\mathbf{S}_{\mathrm{tex}}$ and the isotropic matrix $\mathbf{S}_{\mathrm{matrix}}$ can be performed in a straightforward way as already described by the Eq.\ \ref{eq:2-28} and Eq.\ \ref{eq:2-29} respectively. However, the computation of the second Piola-Kirchhoff stress for the collagen part $\mathbf{S}_{\mathrm{co}}$ requires taking into consideration the influence of the collagen evolution equations. The collagen density ${\rho}^{0}_{\mathrm{co}}$ depends on the mass-specific energy ${\psi}_{\mathrm{co, m}}$ and consequently on the Cauchy-Green tensor $\mathbf{C}$, which
introduces additional terms to compute the derivatives as shown in Eq.\ \ref{eq:2-9}. By summing up the contributions from each constituent, we end with an expression for the second Piola-Kirchhoff stress that reads  
\begin{equation}
\label{eq:3-3}
\mathbf{S} = 2 \, \left(\frac{\partial {\psi}_{\mathrm{tex}}}{\partial \mathbf{C}} + \frac{\partial {\psi}_{\mathrm{matrix}}}{\partial \mathbf{C}} + \frac{\partial {\rho}^{0}_{\mathrm{co}}}{\partial \mathbf{C}} \, {\psi}_{\mathrm{co, m}} + {\rho}^{0}_{\mathrm{co}} \, \frac{\partial {\psi}_{\mathrm{co, m}}}{\partial \mathbf{C}} \right).
\end{equation}

 The next step is to compute the material tangent operator ${\bf \mathbb{C}}$. By applying the same approach used to compute $\mathbf{S}$, ${\bf \mathbb{C}}$ is computed by summing up the tangents of each constituent as described by the following expression
\begin{equation}
\label{eq:3-4}
{ \mathbb{C}} = {\mathbb{C}_{\mathrm{tex}}  + { \mathbb{C}_{\mathrm{matrix}}} + { \mathbb{C}_{\mathrm{co}}}}.
\end{equation}

The tangent tensor for each constituent is computed by taking the partial derivative of the second Piola-Kirchhoff stress with respect to the right Cauchy-Green tensor. This gives us the following term for the textile part
\begin{equation}
\label{eq:3-5}
{ \mathbb{C}_{\mathrm{tex}}} = 2 \, \frac{\partial \mathbf{S}_{\mathrm{tex}}}{\partial \mathbf{C}} = 4 \, \frac{\partial^{2} {\psi}_{\mathrm{tex}}}{\partial \mathbf{C}^{2}}.
\end{equation}

Analogously, for the matrix part, we compute the following term
\begin{equation}
\label{eq:3-6}
{ \mathbb{C}_{\mathrm{matrix}}} = 4 \, \frac{\partial^{2} {\psi}_{\mathrm{matrix}}}{\partial \mathbf{C}^{2}}.
\end{equation}

For the collagen part, computing the tangent operator is more complex, because the Helmholtz free energy ${\psi}_{\mathrm{co}}$ depends on the collagen density ${\rho}^{0}_{\mathrm{co}}$. Similar to the other constituents, we describe the tangent operator by the following expression
\begin{equation}
\label{eq:3-7}
{ \mathbb{C}_{\mathrm{co}}} = 4 \, \frac{\partial^{2} {\psi}_{\mathrm{co}}}{\partial \mathbf{C}^{2}}.
\end{equation}

The derivatives in Eq.\ \ref{eq:3-3} and Eqs.\ \ref{eq:3-5}-\ref{eq:3-7} are computed using the code generated by the automatic differentiation software package AceGen \cite{Korelc_2002, Korelc_2009}.

\subsection{Finite element implementation}
In our structural computations, we used a special finite element technology with reduced integration, namely the solid-shell element Q1STs \cite{Reese_2007, Barfusz_2021b}. Q1STs is a low-order isoparametric element with eight nodes. Due to the application of the reduced integration concept, the element contains only one Gauss point within the shell plane. It is especially beneficial for modeling thin shell structures such as heart valves because we can use an arbitrary number of Gauss points through the element's thickness. Furthermore, Q1STs element formulation offers a remedy to volumetric and shear locking. Locking treatment and hourglass stabilization in Q1STs are achieved by applying the concepts of enhanced assumed strain and assumed natural strain. Using elements capable of treating such locking phenomena is essential for us to compute the examples presented in Section \ref{sec:5} because: (i) soft biological tissues are almost incompressible materials which makes them susceptible to volumetric locking, and (b) the structure presented in our work are under severe bending which would cause shear locking in case of using standard low-order element formulation. An additional advantage of using a solid-shell formulation is its ability to model the non-linear material behavior along the thickness direction using only one element. Consequently, we can drastically reduce the number of elements needed in our computations compared to using a standard continuum solid element.

\section{Experiments}

\newcommand{\tablett}[1]{\multicolumn{1}{|c|}{\textbf{#1}}}

\begin{table*}[t]
\centering
\begin{tabular}{|l|ccc|ccc|ccc|ccc|}
\hline
    \mtrb{2}{Model} & 
    \mytitle{3}{Comma} & 
    \mytitle{3}{Period} & 
    \mytitle{3}{Question} & 
    \mytitle{3}{Overall} \\
    \cline{2-13}
    & P & R & F & P & R & F & P & R & F  & P & R & F  \\
    \hline 
    \hline
    % \myrotate{6}{\textbf{ASR}}
    RoBERTA-large \cite{lai-etal-2022-behancepr} & - & - & - & - & - & - & - & - & - &  62.0 & 61.4 & 61.7 \\
    + Augmentation & - & - & - & - & - & - & - & - & - &  63.8 & 60.7 & 62.2 \\
    + CRF & - & - & - & - & - & - & - & - & - &62.2 & 63.5 & 62.9 \\
    + CRF + Augmentation & - & - & - & - & - & - & - & - & - & 61.1 & 62.8 & 62.0 \\
    % ELECTRA-base & 63.3 & 62.4 & 62.9 \\
    \hline
    DeBERTa-large & 61.8 & 58.3 & 60.0 & 65.1 & \textbf{74.6} & \textbf{69.5} & 72.1 & \textbf{56.7} & \textbf{63.5} & 63.7 & 64.8 & 64.2 \\
    % DeBERTa-large + Augmentation &  \\
    \textbf{+ RL (Ours)} &  \textbf{62.1} &\textbf{63.0} & \textbf{62.5} & \textbf{65.9} & 72.4 & 69.0 & \textbf{73.0} & 53.1 & 61.4 & \textbf{64.1} & \textbf{66.2} & \textbf{65.2} \\
    \hline
\end{tabular}
\caption{Performances on the BehancePR test set. Note that \cite{lai-etal-2022-behancepr} did not report the breakdown performance for each type.}
\label{table-result-behance}
\end{table*}



%\newcommand{\tablett}[1]{\multicolumn{1}{|c|}{\textbf{#1}}}

% \begin{table}[t]
% \centering
% % \resizebox{0.4\textwidth}{!}{
% \begin{tabular}{|l|r|r|r|}
%     \hline
%     \tablett{Model} & \tablett{P} & \tablett{R}& \tablett{F1} \\
%     \hline
%     % \myrotate{6}{\textbf{ASR}}
%     RoBERTA-large &  62.0 & 61.4 & 61.7 \\
%     + Augmentation &  63.8 & 60.7 & 62.2 \\
%     + CRF & 62.2 & 63.5 & 62.9 \\
%     + CRF + Augmentation & 61.1 & 62.8 & 62.0 \\
%     % ELECTRA-base & 63.3 & 62.4 & 62.9 \\
%     \hline
%     DeBERTa-large  & 63.7 & 64.8 & 64.2 \\
%     % DeBERTa-large + Augmentation &  \\
%     \textbf{+ RL (Ours)} &  \textbf{64.1} & \textbf{66.2} & \textbf{65.2} \\
%     \hline
% \end{tabular}
% % }
% \caption{Performances on the BehancePR test set. Note that \cite{lai-etal-2022-behancepr} did not report the breakdown performance for each type.}
% \label{table-result-behance}
% \end{table}



\textbf{Settings}: In this paper, we evaluate our proposed model on two available English datasets that have been used in previous studies. 
\textbf{IWSLT} is the benchmark dataset for the PR task in English. It annotates three prominent punctuation marks: \textit{PERIOD, COMMA, QUESTION}. The IWSLT corpus contains texts derived from TED Talks, which are mainly monologues. The testing set of this corpus contains both reference text (REF), which is well-written text, and transcribed text (ASR) with manually inserted punctuation. Whereas the training set consists of only REF text. The training, development, and test sets contain approximately 2.1M, 300K, and 12K words, respectively. 
\textbf{BehancePR} is a human-annotated dataset for livestreaming videos. It features multiple speakers as well as interaction with a large number of audiences. BehancePR corpus contains only ASR text. The training/development/testing sets contain approximately 1.2M, 34K, and 44K words, respectively. 
The models are evaluated using the standard precision, recall, and F1-score (micro).


\textbf{Hyperparameters}: In this paper, each input word is tokenized using the word-piece tokenizer provided in the PLM. The representation of the first word-piece is collected as the input of the classifier head, which is a fully connected layer, to predict the punctuation.
We employed the DeBERTa-large PLM \cite{he2021deberta} as the encoder of the PR model. The hidden states of the top 8 layers are used as the representation of a token, searched from a pool of \{1,4,8,12\} layers. The GPT2-medium is used to generate the text. The seed texts for the GPT2 model contain 64 consecutive words randomly sampled from these pools. Both models are trained using the Adam optimizer with a learning rate in \{2e-5, 5e-5\}. The augmentation ratios $\alpha_1,\alpha_2,\alpha_3$ are set to 5\%, similar to \cite{alam-etal-2020-punctuation}. We concatenate $C=20$ context words to the head and tail of each chunk. Due to the high cost of evaluating the PR model on the whole development set, in each iteration, we only sample a subset $|B_j|=16$ chunks from $\mathcal{D}_{dev}$ to compute the reward.



\subsection{IWSLT corpus}

\textbf{Baselines}: We compared our model with the state-of-the-art PR models: \textbf{RoBERTa-large+Augmentation} model employs a RoBERTa-large PLM \cite{alam-etal-2020-punctuation}. The input data is augmented using three augmentation strategies: insertion, substitution, and deletion. \textbf{ELECTRA-base+Multitask} \cite{hentschel2021making} is finetuned using additional augmentation detection loss and knowledge distillation loss. \textbf{ELECTRA-large+Discriminative Self-Training} \cite{chen2021discriminative} was self-trained with a discriminator to detect human-annotated data and pseudo-machine-labeled data. \textbf{Funnel-transformer-xlarge+POSFusion} \cite{shi2021incorporating} incorporates additional part-of-speech features from an external neural-network based POS tagger.

\textbf{Results}: Table \ref{table-result-ted} compares the examined models' performance on both the REF test set and the ASR test set. The performance on the REF test set shows us the performance in case the ASR text is close to the written text, while the ASR test shows the actual performance on ASR text.

On the REF test set, ELECTRA-large is the best model among the five examined PLMs with an F1 score of 84.4\%, and it is closely followed by DeBERTa-large (0.3\%  lower). These models leave a large margin to the smaller models such as ELECTRA-base (approx. 3\% lower). Comparing the full models, our DeBERTA-large + RL model gains 1\% over the DeBERTa-large model, achieving 85.1\%. This performance is on par with the ELECTRA-large + Discriminative Self-Training model with a mere margin of 0.1\%. 


% On the REF test set, ELECTRA-large is the best model among the five examined PLMs with an F1 score of 84.4\%, and it is closely followed by DeBERTa-large (0.3\%  lower) and funnel-transformer-xlarge (0.7\% lower). This is reasonable given their similar sizes and architectures \cite{clark2020electra,he2021deberta}. These models leave a large margin to the smaller models: RoBERTa-large (approx. 2\% lower) and ELECTRA-base (approx. 3\% lower). Second, comparing the full models, our DeBERTA-large + RL model gains 1\% over the DeBERTa-large model, achieving 85.1\%. This performance is on par with the ELECTRA-large + Discriminative Self-Training model with a mere margin of 0.1\%. Third, comparing the performances for each punctuation, even though the DeBERTa-large + RL loses on QUESTION with substantially lower performance compared to ELECTRA-base and ELECTRA-large models, it yields an identical F1 score to the F1 score of the ELECTRA-large+Discriminative Self-Training model on both COMMA and PERIOD, which account for more than 90\% of the dataset. This experiment shows the effectiveness of the RL training process in providing helpful examples for training the PR model on reference data.


For ASR text, comparing the full models, our DeBERTa-large + RL model (77\% in terms of overall F1) outperforms all the other models at a large margin of 3\% to the highest competitor, RoBERTa-large + Augmentation,  with $p<0.01$. Moreover, without additional training signals or external features, the DeBERTa-large model yields similar performance to other PLMs (e.g., RoBERTa-large and funnel-transformer-xlarge). Furthermore, our proposed model outperforms the other models on all three punctuation marks with a consistently large margin ranging from 1.8\% to 5.1\%, compared to the next highest. These results clearly show the robustness of our proposed RL method to boost the performance of real-world ASR data significantly. The improvement suggests that the RL method has provided helpful training examples to help the model bridge the gap between the REF text and the ASR text in the training and testing data, respectively.


\subsection{BehancePR corpus}

\textbf{Baselines}: We compare our models with the state-of-the-art models that have been evaluated on this corpus. These models include the \textbf{RoBERTa-large} model and its variants with \textbf{Data Augmentation} and \textbf{Conditional Random Field} \cite{alam-etal-2020-punctuation}.

\textbf{Results}: First, we found that data augmentation does not improve the performance of the model trained on the BehancePR dataset. The reason is that the BehancePR dataset's training and testing data are all ASR texts, which is different from the IWSLT corpus in which the training texts are REF texts, and the testing texts are ASR texts. As such, introducing data augmentation skewed the distribution of training and testing data in the BehancePR corpus. Hence, hurting the model's performance. Table \ref{table-result-behance} presents the overall performance of our proposed models on the BehancePR corpus. The DeBERTa-large outperforms the current state-of-the-art  RoBERTa-large+CRF model (62.2\% versus 62.9\%). Furthermore, the DeBERTa-large + RL improves the F1 score from 64.2\% to 65.2\% (+1.0) (statistically significant with $p<0.01$). This again shows the effectiveness of the proposed reinforcement learning methods.



\begin{table}[t]
\centering
% \resizebox{0.83\linewidth}{!}{
\begin{tabular}{|l|l|r|r|r|}
    \hline
    \tablett{Model} & \tablett{P} & \tablett{R}& \tablett{F1} \\
    \hline
    % \myrotate{7}{\textbf{ASR}} 
    RoBERTa-large                   & 66.5 & 76.7 & 71.3  \\
    \hspace{0.2cm}+ Augmentation    & 72.0 & 76.2 & 74.0  \\
    \hspace{0.4cm} + GPT + RL             & 73.3 & 76.7 & 75.0 \\
    \hline
    DeBERTa-large                       & 66.1 & 77.6 & 71.4  \\
    \hspace{0.2cm}+ Augmentation        & 73.0 & 77.1 & 75.0  \\    
    \hspace{0.4cm} + GPT & 74.9 & 76.3 & 75.6 \\
    \hspace{0.6cm} + RL (Full model)  & \textbf{74.6} & \textbf{79.4} & \textbf{77.0} \\
    \hline
    DeBERTa-large + GPT + RL & & & \\
    + PR pretraining (1 epoch)                & 74.5 & 78.5 & 76.4   \\
    + PR pretraining (2 epochs)                & 74.2 & 78.3 & 76.2   \\
    + GPT2 pretraining (1 epoch)               & 75.7 & 77.0 & 76.3   \\
    + GPT2 pretraining (2 epochs)              & 74.5 & 77.2 & 75.8 \\
    \hline
\end{tabular}
% }


\caption{Performances on the IWSLT ASR test set.}
\label{table-ablation}
\end{table}



\subsection{Ablation study}

We perform an ablation study to examine the contribution of each component of the model on the IWSLT ASR test set as shown in Table \ref{table-ablation} (Rows 1-7). Adding the augmentation to the DeBERTa-large model boosts the performance from 71.4\% to 75.0\% (\textbf{+3.6\%}), while \textit{GPT} improves the F1 score from 75.0\% to 75.6\% (\textbf{+0.6\%}). Finally, when we add \textit{RL}, the F1 score jumps from 75.6\% to 77.0\%. These demonstrate that all the proposed components contribute to the improvement. However, data augmentation and RL contribute largely to the performance gain on the IWSLT ASR test set. Finally, to further show the effectiveness of the \textit{RL}, we add it to the RoBERTa-large+Augmentation, resulting in an increase of 1\% in the F1 score. This experiment shows that our RL method is model-agnostic that can be applied to any PR model.

The PR model and the GPT2 model could be finetuned/pre-trained with different strategies. To examine whether finetuned or pre-trained model before the reinforcement learning could further improve the performance of the model. We used the configuration of the full model with GPT2 and RL. However, for the PR model, we trained the PR alone with the same training data for 1 and 2 epochs. Similarly, we pre-trained the GPT2 model on the unsupervised text derived from the training set for the same epochs. Table \ref{table-ablation} (Rows 8-12) reports the performance of these runs. As can be seen from the performance, training/finetuning the model using only PR or GPT2 data significantly hurts the performance of the model. In particular, pretraining a single epoch on PR or GPT2  reduced the performance by 0.4\% to 0.7\%, respectively. Further training the model for one more epoch decreased the performance by 0.4\% to 0.5\%, respectively.


% \input{tab:pretraining}


\section{Conclusion}

This paper focuses on generating helpful training data for the punctuation restoration task, especially for real-world ASR texts.
We devise a reinforcement learning method to use the GPT2 model to generate additional data to train the punctuation restoration model. This method allows the GPT2 model to learn from real-world ASR text to generate more helpful training examples based on gradient feedback from the PR model. Our model improves PR performance on real-world ASR tests on IWSLT and BehancePR  (+3\% and +2.3\%, respectively). In the future, we would like to extend this research with more advanced gradient feedback to improve the generated data.

\section{Acknowledgement}: This research has been supported by the Army Research Office (ARO) grant W911NF-21-1-0112, the NSF grant CNS-1747798 to the IUCRC Center for Big Learning, and the NSF grant \# 2239570. This research is also supported in part by the Office of the Director of National Intelligence (ODNI), Intelligence Advanced Research Projects Activity (IARPA), via the HIATUS Program contract 2022-22072200003. The views and conclusions contained herein are those of the authors and should not be interpreted as necessarily representing the official policies, either expressed or implied, of ODNI, IARPA, or the U.S. Government. The U.S. Government is authorized to reproduce and distribute reprints for governmental purposes notwithstanding any copyright annotation therein.
%\begin{comment}
\section{System Architecture}
\label{appendix:architecture}
\system has a novel modularized system architecture with three key components: 
\emph{StreamManager}, 
\emph{TxnManager} and \emph{TxnScheduler}. 
These components are instantiated in each thread locally.
The execution outline of \system is presented in Algorithm~\ref{alg:algo}.
Transactional stream processing is continuous and potentially never ends (Line 1$\sim$8).
The dependency resolution and execution of state transactions are separated into two non-overlapping phases by punctuations~\cite{Tucker:2003:EPS:776752.776780} (Line 2 and 5), which guarantees that no subsequent input event will have a smaller timestamp. 
Effectively, a batch of state transactions is collected during the first phase, and processed during the second phase.

In the first phase (i.e., stream processing phase), 
the \emph{StreamManager} conducts preprocessing for every input event ($e$). Similar to some prior works~\cite{tstream}, state transactions may be issued but not immediately processed during preprocessing (Line 3).
The \emph{pre\_processing} and \emph{post\_processing} functions are exposed as APIs to users.
The \emph{TxnManager} handles dependency resolution (Line 4) among state transactions and insert decomposed operations to construct a \tpg. We discuss the detailed two-phase \tpg construction process in Section~\ref{subsec:construction}.

In the second phase  (i.e., transaction processing phase), 
the \emph{TxnManager} is first involved again to refine (Line 6) the constructed \tpg with further dependency resolution.
The \emph{TxnScheduler} 
schedules operations for concurrent execution based on the constructed \tpg according to the three dimensions of scheduling decisions (Line 7). 
In particular, a scheduling decision model $M$ is instantiated based on the constructed \tpg (Line 14).
\textbf{\circled{1}} Guided by $M$, execution threads adopt an exploration strategy (Section~\ref{subsec:explore}) to explore the constructed \tpg for operations available to be scheduled constrained by dependencies. 
\textbf{\circled{2}} 
During exploration, one or multiple operations may be treated as the 
% basic 
unit of scheduling (Section~\ref{subsec:granularity}). 
Subsequently, \textbf{\circled{3}} every thread executes operation(s) in the unit of scheduling with various abort handling mechanisms (Section~\ref{subsec:abort_handling}).
Only when state transactions are processed (i.e., committed or aborted) can the associated input events be postprocessed (Line 8) by the \emph{StreamManager} based on transaction processing results.
\end{comment}

\begin{comment}
\begin{algorithm}
\footnotesize
    \KwData{$e$ \tcp{Input event}}
    \KwData{$txn_{ts}$ \tcp{State transaction}}
    \KwData{$G$ \tcp{The currently constructed TPG}}
    \While{!finish processing of input streams}{
        \eIf(\tcp*[h]{Phase 1}){\text{$e$ is not a $punctuation$}}{
                $txn_{ts}$ $\gets$ PRE\_Processing($e$)\;
                \textbf{TPG\_Construction}($G$, $txn_{ts}$)\; 
          }(\tcp*[h]{Phase 2}){
                \textbf{TPG\_Refinement}($G$)\; 
                \textbf{TXN\_Scheduling}($G$)\; 
                POST\_Processing()\;
          }
    }
    
    \SetKwFunction{FMain}{TPG\_Construction}
    \SetKwProg{Fn}{Function}{:}{}
    \Fn{\FMain{$G$, $txn_{ts}$}}{
        $O_{1..k}$ $\gets$ \textbf{Partition} $txn_{ts}$\;
        \ForEach{\text{operation $O_{i}$ $\in$ $O_{1..k}$}}{
            \textbf{Identify} its \ld\;
            $G$ $\gets$ $G$ + $O_{i}$ \;
        }
    }
    \SetKwFunction{FMain}{TPG\_Refinement}
    \SetKwProg{Fn}{Function}{:}{}
    \Fn{\FMain{$G$}}{
        \ForEach{\text{vertex $e_{i}$ $\in$ $G$}}{
            \textbf{Identify} its \td, \pd\;
        }
    }
    
    \SetKwFunction{FMain}{TXN\_Scheduling}
    \SetKwProg{Fn}{Function}{:}{}
    \Fn{\FMain{$G$}}{
        $M$ $\gets$ Instantiated with $G$;\tcp{A decision model}
        \While{!finish scheduling of $G$
        }{
          \textbf{\circled{2}} $Scheduling Unit$ $\gets$ \textbf{\circled{1}} \emph{Explore}($G$, $M$)\; 
            \textbf{\circled{3}} \emph{Execute with Abort Handling} ($Scheduling Unit$)\; 
        }
    }
  \caption{Execution Outline of \system}
  \label{alg:algo}
\end{algorithm}
\end{comment}

%%%%%%%%%%%%%%%%%%%%%%%%%%%%%%%%%%%%%%%%%%%%%%%%%%%%%%%%%%%%%%%%%%%%%%%%%%%%%%%%%%%%%%%%%%%%%%%%%%%%%%%%%
\bibliographystyle{IEEEtran.bst}
\bibliography{literature}

\end{document}

