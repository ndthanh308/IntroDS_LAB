	\section{Introduction}
	
%This paper is concerned with the existence of weak solutions to a partial differential equation that arises in the dynamics of anamolous diffusion \cite{henry2010introduction}. In this class of models, we focus on subdiffusion-type behavior. Specifically, we assume power law waiting times in the derivation of the equation, leading to the appearance of time-fractional derivatives.
%This gives rise to new problems about the study of well-posedness, and we present rigorous proofs for the existence of weak solutions. 

Mathematicians and engineers have given time-fractional differential equations significant consideration in recent years.
Such equations are nonlocal in time and possess an inherent history effect.
We refer the interested reader to the multi-volume work “Handbook of Fractional Calculus with Applications” \cite{baleanu2019handbook,petras2019handbook,tarasov2019handbook} and its references therein for more information and typical real-world applications such as physics, control theory, engineering, life and social sciences.

In this work, we are concerned with the time-fractional Fokker--Planck equation, which permits subdiffusive behavior and its derivation and application have been investigated earlier in literature. We distinguish the model by its exterior force, which can be time-dependent \cite{sokolov2006field,magdziarz2009stochastic}, space-dependent \cite{barkai2000continuous,barkai2001fractional,chechkin2003fractional,fu2021continuous,metzler1999anomalous,metzler1999deriving,metzler2002space,sandev2015diffusion,sokolov2001dynamics},  or space-time dependent \cite{angstmann2015generalized,heinsalu2007use,magdziarz2008equivalence,weron2008modeling}. We focus on the latter, most general, case and mention the publications \cite{huang2020new,le2016numerical,le2018semidiscrete,le2021alpha,mclean2021uniform,mustapha2022second,pinto2017numerical,yan2019finite} that explored numerical methods for the time-fractional Fokker--Planck equation with space-time dependent forces. 

We emphasize that several articles have investigated a “time-fractional Fokker--Planck”-type equation, where the time-fractional derivative in the sense of Caputo appears on the left-hand side of the PDE. This is correct in the case of a time-independent force. However, for time dependent forces this model is not correct and according to  \cite{heinsalu2007use}, it is “physically defeasible” and its solution “does not correspond to a physical stochastic process”. In this work, we provide some mathematical and numerical insights on this reformulation and the differences of both models.

We present an analytical treatment of weak solutions to the time-fractional Fokker--Planck equation with space-time dependent forces. Specifically, we follow the Galerkin ansatz by spatially discretizing the system and deriving appropriate energy constraints, allowing us to reach the limit in the discretized system.
We mention that weak solutions to other nonlinear time-fractional PDEs have been previously studied using the Galerkin method in the published works \cite{fritz2021subdiffusive,fritz2023equivalence,fritz2022time}.
In addition, preliminary steps have been taken in the optimal control \cite{camilli2020approximation} and analysis \cite{mclean2020regularity,le2019existence,le2021alpha,mclean2021uniform,peng2022existence,mclean2019well} of the time-fractional Fokker–Planck system.
Nonetheless, mild, strong, and classical solutions have been investigated. The difficulty lies in the low regularity of  weak solutions and the appearance of time-dependent forces, which do not allow us to transform the system to a more accessible system regarding analysis. A coupled system of a time-fractional Fokker--Planck equation with the Navier--Stokes equations was investigated in the work \cite{fritz2023analysis} but because of the complex coupling between the equations only the case of $\alpha \in (\frac12,1)$ was considered.



In Section \ref{Sec:Derivation}, we discuss the mathematical model with its initial and boundary data.
%We observe that time-fractional derivatives in the Riemann--Liouville sense appear on the right-hand side of the Fokker--Planck equation. 
In Section \ref{Sec:Prelim}, we present various function spaces and recall important conclusions from the theory of fractional derivatives, including chain inequalities, embedding theorems, and Gronwall-type inequalities.
In Section \ref{Sec:Analysis}, we finally present and verify the theorem declaring the well-posedness of weak solutions.
Here, the system is discretized, and appropriate energy bounds are derived to pass the limit in the discretized system. 
In Section \ref{Sec:Numerics}, we propose a numerical discretization of the time-fractional equation based on the nonuniform L1 scheme in time and finite elements in space. We show simulation results and focus on the influence of the fractional derivative. Moreover, we compare the model that is studied here to the physical defeasible model as mentioned above.

%Textbooks on FP \cite{carmichael2000statistical,coffey2012langevin,bogachev2015f}
%Modeling of space-time force in time-fractional FP \cite{angstmann2015generalized,heinsalu2007use,magdziarz2008equivalence,weron2008modeling}
%Fractional diffusion \cite{henry2010introduction}
%\cite{heinsalu2007use}
%\cite{henry2010fractional}: dichotomously alter-
%nating force fields $F(x,t)=\xi(t)F(x)$ with $\xi(t)=\pm 1$, also in \cite{heinsalu2007use}
%Modeling time-force in time-frac FP \cite{sokolov2006field,magdziarz2009stochastic}
%Modeling space-force in time-fractional FP \cite{barkai2000continuous,barkai2001fractional,chechkin2003fractional,fu2021continuous,metzler1999anomalous,metzler1999deriving,metzler2002space,sandev2015diffusion,sokolov2001dynamics}
%Analytical solution with space-force \cite{sutradhar2014analytical}
%Correct numerics of time-space force in time-frac FP \cite{huang2017new,le2016numerical,le2018semidiscrete,le2021alpha,mclean2021uniform,mustapha2022second,pinto2017numerical,yan2019finite}
%Analysis of time-space force in time-frac FP \cite{le2019existence,peng2022existence}

%Numerics of space-force in time-frac FP \cite{chen2009finite,deng2007numerical,javidi2021predictor,jiang2015new,jiang2019monotone,ren2019high,vong2015high,zheng2015novel}

%Numerics of space-force in time-space-frac FP \cite{deng2008finite,deng2012finite,yang2009computationally,yang2018numerical,zhang2018time,zhao2013error}

%Optimal control \cite{camilli2020approximation}

%Wrong space-time formulation \cite{dehestani2020fractional,eshaghi2017numerical,firoozjaee2018numerical,freihet2019computational,habenom2020numerical,kachhia2020comparative,kumar2013numerical,mahdy2021numerical,nuntadilok2021solution,odibat2007numerical,prakash2017numerical,ray2014two} \cite{singh2022new,yang2019finite} (numerical)

%eshaghi2017numerical,firoozjaee2018numerical,freihet2019computational,habenom2020numerical,kachhia2020comparative,kumar2013numerical,mahdy2021numerical,nuntadilok2021solution,odibat2007numerical,prakash2017numerical,ray2014two

%advection-diffusion-reaction weak analysis \cite{mclean2020regularity}