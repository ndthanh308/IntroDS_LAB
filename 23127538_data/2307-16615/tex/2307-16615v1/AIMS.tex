\documentclass{aims} % Use the aims.cls file to compile your paper
\usepackage{amsmath}
\usepackage{paralist}
\usepackage[misc]{ifsym}
\usepackage{epsfig} 
\usepackage{epstopdf} 
\usepackage[colorlinks=true]{hyperref}
\hypersetup{urlcolor=blue, citecolor=red}
\allowdisplaybreaks

\textheight=8.2 true in
 \textwidth=5.0 true in
  \topmargin 30pt
   \setcounter{page}{1}

\def\currentvolume{X}
 \def\currentissue{X}
  \def\currentyear{200X}
   \def\currentmonth{XX}
    \def\ppages{X-XX}
     \def\DOI{10.3934/xx.xxxxxxx}

%%%%%%%%%%%%%%%%%%%%%%%%%%%%%%%%%%%%%%%%%%%%%%%%%%%%%%%%
%          2. CUSTOM COMMANDS
%%%%%%%%%%%%%%%%%%%%%%%%%%%%%%%%%%%%%%%%%%%%%%%%%%%%%%%%
% PLEASE NOTE: The AIMS cls file is updated regularly with the standard AMS usepackages - you do NOT need to insert custom commands for AMS symbols,

% Insert your custom commands in this section.
% Please minimize the use of "newtheorem", "newcommand", and use equation numbers only in situations where they provide essential convenience.
% Use \usepackage{amssymb}, \usepackage{amsthm}, etc, but please do not define individual mathematical symbols unless it is completely necessary.
% Try to avoid defining your own macros.

% Do not change or remove the commands below.
\newtheorem{theorem}{Theorem}[section]
\newtheorem{corollary}[theorem]{Corollary}
\newtheorem*{main}{Main Theorem}
\newtheorem{lemma}[theorem]{Lemma}
\newtheorem{proposition}[theorem]{Proposition}
\newtheorem{conjecture}[theorem]{Conjecture}
\newtheorem*{problem}{Problem}
\theoremstyle{definition}
\newtheorem{definition}[theorem]{Definition}
\newtheorem{remark}[theorem]{Remark}
\newtheorem{assumption}[theorem]{Assumption}
\newtheorem*{notation}{Notation}
\newcommand{\ep}{\varepsilon}
\newcommand{\eps}[1]{{#1}_{\varepsilon}}


%%%%%%%%%%%%%%%%%%%%%%%%%%%%%%%%%%%%%%%%%%%%%%%%%%%%%%%%
%         3. HEADER AND FOOTER SECTION
%%%%%%%%%%%%%%%%%%%%%%%%%%%%%%%%%%%%%%%%%%%%%%%%%%%%%%%%


% Place the running head in [], and the full title of the article in {}.
\title[A Cahn--Hilliard model coupled to fractional viscoelasticity]
% Running head is the full title or shortened version of the full title. This will appear at the top of odd pages. It should be no more than 40 characters to fit within the width limit.
{Analytical and numerical treatment of a Cahn--Hilliard model coupled to fractional viscoelasticity} % Only the first word and proper nouns should be capitalized.

% Place all authors' names in []. This will be shown as the running head on even pages. Leave {} empty.
% Please use `and' to connect the last two authors' names if applicable.
% List full names if possible. If all authors' full names will not fit, use FirstNameInitial. MiddleNameInitial. LastName, only last names, or full names of first few authors, et al.
\author[Marvin Fritz]{}

% 2020 MSC numbers are required.
\subjclass{Primary: 58F15, 58F17; Secondary: 53C35.}
% Please provide a minimum of 5 keywords or phrases.
\keywords{Cahn--Hilliard model; viscoelasticity.}

% Put your short thanks below. For thanks/acknowledgments over 30 words, please place them in \section*{Acknowledgments} located above the reference section.
% Remove \thanks{The first author is supported by NSF grant xx-xxxx} if nothing is added here.
\thanks{The author is supported by the state of Upper Austria.}


% Add corresponding author in the footnote of the first page if necessary.
% Add $^*$ adjacent to the corresponding author's name on the first page if necessary.
% The corresponding author in your article should match the corresponding author listed for your article in EditFlow (if applicable).
% In the example shown below, the first author is the corresponding author. Please move or remove $^*$ as needed for your article.
\thanks{$^*$Corresponding author: Marvin Fritz}

%%%%%%%%%%%%%%%%%%%%%%%%%%%%%%%%%%%%%%%%%%%%%%%%%%%%%%%%%%
%      4. AUTHOR NAMES/ADDRESSES/AFFILIATIONS SECTION
%%%%%%%%%%%%%%%%%%%%%%%%%%%%%%%%%%%%%%%%%%%%%%%%%%%%%%%%%%

\usepackage[T1]{fontenc}

\usepackage{xcolor,bm, amssymb, mathtools}
%\usepackage[colorlinks=true,linkcolor=blue,citecolor=green!65!black]{hyperref}
\usepackage[capitalise,nameinlink]{cleveref}

%\newtheorem{theorem}{Theorem}[section]
%\newtheorem{corollary}{Corollary}[theorem]
%\newtheorem{lemma}[theorem]{Lemma}


\makeatother
\newcommand{\ECM}{{E\hspace{-.05em}C\hspace{-.15em}M}}
\newcommand{\MDE}{{M\hspace{-.15em}D\hspace{-.1em}E}}
\newcommand{\TAF}{{T\hspace{-.15em}A\hspace{-.1em}F}}
\newcommand{\ecm}{\phi_{\ECM}}
\newcommand{\taf}{\phi_{\TAF}}
\newcommand{\phim}{\hat\phi}
\newcommand{\hzeta}{\hat\zeta}
\newcommand{\psim}{\hat\psi}
\newcommand{\mde}{\phi_{\MDE}}
\newcommand{\phib}{\bm{\phi}}
\newcommand{\pt}{\partial_t}
\renewcommand{\eps}{\varepsilon}
%\renewcommand{\phi}{\varphi}
\newcommand{\pta}{\p_t^\alpha}
\newcommand{\ptb}{\p_t^{1-\alpha}}
\newcommand{\ga}{g_{\alpha}}
\newcommand{\gb}{g_{1-\alpha}}
\newcommand{\X}{\vec{X}}
\newcommand{\x}{\vec{x}}
\newcommand{\y}{\vec{y}}
\newcommand{\z}{\vec{z}}
\newcommand{\trace}{\textup{tr}}
\newcommand{\ds}{\,\textup{ds}}
\newcommand{\J}{\mathcal{J}}
\newcommand{\T}{\mathcal{T}}
\newcommand{\dev}[1]{\text{dev}\left[#1\right]}
\newcommand{\lambdap}{\lambda^{\!\textup{pro}}}
\newcommand{\lambdad}{\lambda^{\!\textup{deg}}}
\newcommand{\dx}{\Delta x}
\newcommand{\sgn}[1]{\ \text{sgn}\left(#1\right) }
\newcommand{\frc}[2]{{\textstyle\frac{#1}{#2}}}
\newcommand{\minus}{\scalebox{0.7}[1.0]{\( - \)}}
\newcommand{\inv}{^{\scalebox{0.7}[1.0]{\( - \)}1}}%{^{-1}}
\newcommand{\tp}{^\intercal}
\renewcommand{\exp}[1]{e^{#1}}
\renewcommand{\i}{\mathrm{i}}
\renewcommand{\Re}{\mathfrak{Re}}
\renewcommand{\Im}{\mathfrak{Im}}
\newcommand{\B}{\lbrace -1, 1\rbrace}
\newcommand{\Ind}{\mathbbm{1}}
\renewcommand{\O}{\mathcal{O}}
\newcommand{\diag}{\operatorname{diag}}
\DeclareMathOperator*{\esssup}{ess\,sup}
\newcommand{\Id}{\operatorname*{Id}}
\newcommand{\sign}{\operatorname*{sign}}
\newcommand{\op}[1]{\operatorname*{#1}}
\newcommand{\Exp}[1]{\operatorname*{Exp}\left(#1\right)}
\newcommand{\tr}{\operatorname{tr}}
\newcommand{\doublehookrightarrow}{%
	\mathrel{\mathrlap{{\mspace{4mu}\lhook}}{\hookrightarrow}}}
\newcommand{\sca}[2]{\langle #1, \; #2 \rangle}
\newcommand{\norm}[1]{\lVert #1 \rVert}
\newcommand{\abs}[1]{\left| #1 \right|}
\newcommand{\mat}[1]{\begin{bmatrix} #1 \end{bmatrix}}
\renewcommand{\vec}[1]{\pmb{#1}}
\newcommand{\vct}[1]{\vec{#1}}
\newcommand{\ten}[1]{\pmb{#1}}
\newcommand{\tns}[1]{\ten{#1}}
\newcommand{\tnsfour}[1]{\pmb{#1}}
\newcommand{\bbc}[1]{\left( #1 \right)}
\newcommand{\rbc}[1]{\left[ #1 \right]}
\newcommand{\drv}[2]{\frac{\mathrm{d} #1}{\mathrm{d} #2}} % {\partial_{#1} #2} %
\newcommand{\D}[2]{\partial_{#1} #2}
\renewcommand{\div}{\textup{div}}
\newcommand{\grad}{\nabla}
\newcommand{\gradS}{\nabla\!_s}
\newcommand{\Jac}[2][]{ \operatorname*{D_{#1}}\! #2  }
\newcommand{\Hess}[1]{ D^2#1  }
\newcommand{\dd}{\mathop{}\!\mathrm{d}}
\renewcommand{\d}{\mathop{}\!\mathrm{d}}
\newcommand{\ddt}{\frac{\dd}{\dd\mathrm{t}}}
\newcommand{\dt}{\,\textup{d}t}
\newcommand{\p}{\partial}
\newcommand{\symgrad}{\bm{\varepsilon}}
\newcommand{\CD}{\mspace{0mu}^C\!D}
\newcommand{\RA}[1]{r_h\!\left(#1\right)}
\newcommand{\Ea}{\op{E}\!} % Mittag-Leffler function
\newcommand{\La}[1]{\mathfrak{L}\lbrace #1 \rbrace }
\newcommand{\Lai}[1]{\mathfrak{L}\inv\left\lbrace #1 \right\rbrace }
\newcommand{\F}{\mathcal{F}} % Fourrier transform
\newcommand{\Fi}{\mathcal{F}^{-1}} % inverse Fourrier transform
\newcommand{\FFT}[1]{\operatorname{FFT}\left\lbrace #1 \right\rbrace} % Discrete (Fast) Fourrier Transform
\newcommand{\IFFT}[1]{\operatorname{IFFT}\left\lbrace #1 \right\rbrace} % inverse Discrete (Fast) Fourrier Transform
\newcommand{\Z}{\mathbb{Z}} % integer
\newcommand{\E}{\mathcal{E}} % integer
\newcommand{\R}{\mathbb{R}} % real
\newcommand{\I}{\mathcal{I}}
\newcommand{\red}{\textcolor{red}}
\newcommand{\blue}{\textcolor{blue}}
\newcommand{\orange}{\textcolor{orange}}
\newcommand{\cyan}{\textcolor{cyan}}
\newcommand{\green}{\textcolor{green!70!black}}
\newcommand{\CX}{\mathbb{C}} % complex
\newcommand{\N}{\mathbb{N}} % natural
\newcommand{\PY}{\mathbb{Y}} %projection to H^1(\Omega)
\newcommand{\PZ}{\mathbb{Z}} %projection to H^1_{0,\Sigma}(\Omega)
\newcommand{\PW}{\mathbb{W}} %projection to H^1_{0,\Sigma}(\Omega,\R^d)
\newcommand{\Cphi}{\vartheta}
\newcommand{\Cmu}{\varrho}
\renewcommand{\rho}{\varrho}
\newcommand{\Cu}{\varsigma}
\newcommand{\con}{\hookrightarrow}
\newcommand{\com}{\mathrel{\mathrlap{{\mspace{4mu}\lhook}}{\hookrightarrow}}}
\newcommand{\Cpsi}{\varkappa}
\newcommand{\Cchi}{\varpi}
\newcommand{\Func}{H}
\renewcommand{\L}{\mathcal{L}} %Lebesque space
\newcommand{\W}{\mathcal{W}} %sobolev space
\renewcommand{\H}{\mathcal{H}} %Sobolev hilbert
\newcommand{\C}{\mathcal{C}} %continuous space
\newcommand{\Ba}{\mathcal{U}} %ball in infinite dimensional space with metric
\newcommand{\Diru}{\Sigma_{1}} % Dirichlet boundary for u
\newcommand{\Dirpsi}{\Sigma_{2}} % Dirichlet boundary for psi and chi
\newcommand{\Banach}{\mathcal{B}} % general Banach spaces
\newcommand{\HS}{\mathcal{H}} % general Hilbert spaces
\newcommand{\HSV}{\mathcal{V}} % general Hilbert spaces
\newcommand{\HSVD}{\mathcal{V}'} % general Hilbert spaces
%\newcommand{\E}[1]{\operatorname{\mathbb{E}}\rbc{#1}}
\newcommand{\Prob}[1]{\operatorname{\mathbb{P}}\rbc{#1}}
\newcommand{\Var}[1]{\operatorname{Var}\rbc{#1}}
\newcommand{\Gaussian}{\mathcal{N}}
\newcommand{\Cov}{\tns{C}}
\newcommand{\WN}{\mathcal{W}} % White noise
\newcommand{\WNhat}{\hat{\mathcal{W}}} % Fourier of White noise
\newcommand{\BesselJ}[1]{\mathcal{J}_{#1}}
\newcommand{\BesselK}[1]{\mathcal{K}_{#1}}
\newcommand{\erf}{\operatorname{erf}}
\newcommand{\erfc}{\operatorname{erfc}}
\newcommand{\hypTwoOne}{{\;}_{2}F_{1}}
\newcommand{\longweak}{\relbar\joinrel\rightharpoonup}
\newcommand{\MatLab}{MatLab R2016b}
\newcommand{\rownumber}{\stepcounter{rownumber}\therownumber}
\newcommand{\RHS}{\mathcal{F}}
\newcommand{\opL}{\operatorname{L}}
\newcommand{\FDphiM}{\p_t^{\alpha}(\phi^m-\phi^m_0)}
\newcommand{\FDmuM}{\p_t^{\alpha}(\mu^m-\mu^m_0)}
\newcommand{\FDdivuM}{\p_t^{\alpha}(\nabla\cdot \vec{u}^m-\nabla\cdot\vec{u}^m_0)}
\newcommand{\FDsymuM}{\p_t^{\alpha}\left(\symgrad(\vec{u}^m)-\symgrad(\vec{u}^m_0)\right)}

\newcommand{\dq}{\,\text{d}q}
\newcommand{\hpsi}{{\widehat{\psi}}}
\newcommand{\hphi}{{\widehat{\phi}}}
\newcommand{\hY}{{\widehat{Y}}}
\newcommand{\hX}{{\widehat{X}}}

\newcommand{\displacement}{{\bm u}}
\newcommand{\energy}{\mathcal{E}}
\newcommand{\pf}{\varphi}
%\newcommand{\D}{\mathcal{D}}
%\newcommand{\T}{\mathcal{T}}
%\newcommand{\R}{\mathcal{R}}
\newcommand{\strain}[1][u]{{\bm \varepsilon}(\bm #1)}
\renewcommand{\div}{\nabla\cdot }
%\newcommand{\F}{\mathcal{F}}
\DeclareMathOperator*{\argmin}{arg\,min}

\begin{document}
\maketitle

\centerline{\scshape
Marvin Fritz$^{{\href{mailto:marvin.fritz@ricam.oeaw.ac.at}{\textrm{\Letter}}}*1}$}

\medskip

{\footnotesize
% Enter the full affiliation and country name:
% Do not insert commas or periods at the end of lines.
 \centerline{$^1$Johann Radon Institute for Computational and Applied Mathematics, Linz, Austria}
} % Do not forget to end {\footnotesize with the sign }

\bigskip

% The name of the handling editor will be entered by AIMS production staff.
% "Communicated by Handling Editor" is not needed for special issue.
 \centerline{(Communicated by Handling Editor)}

%%%%%%%%%%%%%%%%%%%%%%%%%%%%%%%%%%%%%%%%%%%%%%%%%%%%%%%
%             5. ABSTRACT
%%%%%%%%%%%%%%%%%%%%%%%%%%%%%%%%%%%%%%%%%%%%%%%%%%%%%%%

\begin{abstract}
In this work, we study a viscoelastic Cahn--Hilliard model. In contrast to other models, we propose a fractional viscoelastic equation based on the fractional laws accounting for memory effects. The laws are fractional counterparts of the well-known Kelvin--Voigt, Zener, and Maxwell models. This approach introduces a time-fractional derivative into the system, and we study the new system with respect to its well-posedness. Further, we investigate the equations numerically and explore the influence of the fractional derivative in the phase-field's evolution. 
\end{abstract}

%%%%%%%%%%%%%%%%%%%%%%%%%%%%%%%%%%%%%%%%%%%%%%%%%%%%%%
%                   6. BODY
%%%%%%%%%%%%%%%%%%%%%%%%%%%%%%%%%%%%%%%%%%%%%%%%%%%%%%


\section{Introduction}
Deep learning models have been widely used in many applications.
For example, BERT~\citep{devlin_bert_2019}, GPT-3~\citep{brown_language_2020}, and T5~\citep{raffel_exploring_2020} achieved state-of-the-art~(SOTA) results on different natural language processing~(NLP) tasks. 
For computer vision~(CV), Transformer-like models such as ViT~\citep{dosovitskiy_image_2021} and Swin Transformer~\citep{liu_swin_2021} deliver excellent accuracy performance upon multiple tasks. 


At the same time, training deep learning models has been a critical problem troubling the community due to the long training time, especially for those large models with billions of parameters~\citep{brown_language_2020}. 
In order to enhance the training efficiency, researchers propose some manually designed parallel training strategies~\citep{narayanan_efficient_2021,shazeer_mesh-tensorflow_2018,xu_gspmd_2021}. 
However, selecting, tuning, and combining these strategies require extensive domain knowledge in deep learning models and hardware environments. With the increasing diversity of modern hardware architectures~\cite{flynn_very_1966,flynn_computer_1972} and the rapid development of deep learning models, these manually designed approaches are bringing heavier burdens to developers. 
Hence, \emph{automatic parallelism} is introduced to automate the parallel strategy searching for training models.


There are two main categories of parallelism in deep learning models: inter-layer parallelism~\citep{huang_gpipe_2019,narayanan_pipedream_2019,narayanan_memory-efficient_2021,fan_dapple_2021,li_chimera_2021,lepikhin_gshard_2021,du_glam_2022,fedus_switch_2022} and intra-layer parallelism~\citep{li_pytorch_2020,narayanan_efficient_2021,rasley_deepspeed_2020,fairscale_authors_fairscale_2021}. 
Inter-layer parallelism partitions the model into disjoint sets on different devices without slicing tensors. 
Alternatively, intra-layer parallelism partitions tensors in a layer along one or more axes and distributes them across different devices.


Current automatic parallelism techniques focus on optimizing strategies within these two categories. However, they treat these two categories separately. 
Some methods~\citep{zhao_vpipe_2022,jia_exploring_2018,cai_tensoropt_2022,wang_supporting_2019,jia_beyond_2019,schaarschmidt_automap_2021,liu_colossal-auto_2023} overlook potential opportunities for inter- or intra-layer parallelism, the others optimize inter- and intra-layer parallelism hierarchically and sequentially~\citep{narayanan_pipedream_2019,fan_dapple_2021,he_pipetransformer_2021,tarnawski_efficient_2020,tarnawski_piper_2021,zheng_alpa_2022}. 
As a result, current automatic parallelism techniques often fail to achieve the global optima and instead become trapped in local optima. 
Therefore, a unified inter- and intra-layer approach is needed to enhance the effectiveness of automatic parallelism.


This paper aims to find the optimal parallelism strategy while simultaneously considering inter- and intra-layer parallelism. 
It enables us to search in a more extensive strategy space where the globally optimal solution lurk. 
However, unifying inter- and intra-layer parallelism in automatic parallelism brings us two challenges. 
Firstly, to adopt a unified perspective on the inter- and intra-layer automatic parallelism, we should not formalize them with separate formulations as prior works. Therefore, how can we express these parallelism strategies in a unified formulation? 
Secondly, previous methods take a long time to obtain the solution with a limited strategy space. Therefore, how can we ensure that the best solution can be obtained in a reasonable time while expanding the strategy space?


To solve the above challenges, we propose UniAP. For the first challenge, UniAP adopts the mixed integer quadratic programming~(MIQP)~\citep{lazimy_mixed_1982} to search for the globally optimal parallel strategy automatically. 
It unifies the inter- and intra-layer automatic parallelism in a single MIQP formulation. 
For the second challenge, our complexity analysis and experimental results show that UniAP can obtain the globally optimal solution in a significantly shorter time.


The contributions of this paper are summarized as follows: 
\begin{itemize}
    \item We propose UniAP, the first framework to unify inter- and intra-layer automatic parallelism in model training.
    \item The optimal parallel strategies discovered by UniAP exhibit scalability on training throughput and strategy searching time.
    \item The experimental results show that UniAP speeds up model training on four Transformer-like models by up to 1.70$\times$ and reduces the strategy searching time by up to 16$\times$, compared with the SOTA method.
\end{itemize}

\input{2_modeling}
\input{3_analysis}
\section{Numerical simulations} \label{Sec:Numerics}



%One might also investigate the mixed system
%\begin{equation} \begin{aligned}
%    \phi &= \ptb \psi \\
%	0 &= \pt \psi -D \Delta \phi + \div (F  \phi)  , \end{aligned} 
%\label{Eq:System5}
%\end{equation}
%with the initial $\psi(0)=\psi^0$. Theoretically, it should hold $\phi^0 = \psi^0 g_\alpha(0)=\infty$. 

Various numerical methods for time-fractional PDEs are summarized in the review article \cite{diethelm2020good} and in the monographs \cite{baleanu2012fractional,owolabi2019numerical,jin2023numerical}. %Since the Caputo derivative $\capb \psi=\ptb (\psi-\psi_0)$ is usually treated in a discrete manner, we follow this approach by adding and subtracting the initial condition $\psi^0$ from the original system as follows
%$$\begin{aligned}\pt \psi - \div(D \nabla \ptb (\psi-\psi_0)) +\div(F \ptb (\psi-\psi_0)) &=f+\div(D\nabla \ptb \psi_0) - \div(F  \ptb \psi_0).% \\
%&=\ga (D \Delta \psi_0 -  F \cdot \nabla \psi_0).
%\end{aligned}$$
%Consequently, we can write the system in terms of the Caputo derivative 
%$$\begin{aligned}\pt \psi - \div(D\nabla \ptb \psi) +\div(F  \ptb \psi) = f+ \ga  \div(D\nabla \psi_0) - \ga  \div(F  \psi_0),
%\end{aligned}$$
%where we further used that $\ptb 1 = \ga$.
%We note that the right-hand side of the PDE is an element in $L^{1/(1-\alpha)-\eps}(0,T;H^{-1}(\Omega))$ for $D,F \in L^\infty(\Omega_T)$, $f\in L^2(0,T;H^{-1}(\Omega))$ and $\psi_0 \in H^1(\Omega)$. We observe that $\alpha> 1/2$ yields a right-hand side in $L^2(0,T;H^{-1}(\Omega))$.
%ga in Lp for a>1-1/p i.e 1/p>1-a i.e 1>p(1-a) i.e. p<1/(1-a)

We assume a discretization $0=t_0<t_1<\dots<t_N=T$ of the time interval $[0,T]$. We do not utilize an equispaced time mesh, but a nonuniform one by discretizing the early times in finer steps. In particular, we assume that the $n$-th time step is of the form $t_n=(n/N)^\gamma T$ for $\gamma\geq 1$. If it holds $\gamma=1$, then we are again in a setting of a uniform mesh, see also Fig. \ref{Fig:Time} for a depiction of some time meshes for various values of $\gamma$.

% Figure environment removed

We discretize the Caputo derivative by the nonuniform L1 scheme \cite[Section 3.2]{diethelm2020good}, i.e., it reads $$\ptb \psi \approx \frac{1}{\Gamma(1+\alpha)} \sum_{j=0}^{n-1} \omega_{n-j-1,n} (\psi_{n-j}-\psi_{n-j-1}),
%=\frac{1}{\Gamma(1+\alpha)} \bigg( \psi_n-\psi_{n-1} + \sum_{j=1}^{n-1} \omega_{n-j-1,n} (\psi_{n-j}-\psi_{n-j-1}) \bigg)
$$ where $\psi_{n-j} \approx \psi(t_{n-j})$. The quadrature weights $\{\omega_{k,n}\}_{k=0}^{n-1}$ are given by the formula
$$\omega_{k,n}=\frac{(t_n-t_k)^{\alpha}-(t_n-t_{k+1})^{\alpha}}{\Delta t_{n-k}},$$
where we introduced the notation $\Delta t_{n-k}=t_{n-k}-t_{n-k-1}$. We use the finite element space $P_1$ for the space discretization and consequently, the fully discrete system reads
\begin{equation} \label{Eq:FP_Discretized} \begin{aligned}& \Big(\frac{\psi^n-\psi^{n-1}}{\Delta t_n},\zeta\Big)_H +  \sum_{j=0}^{n-1} \frac{\omega_{n-j-1,n}}{\Gamma(1+\alpha)}  (D\nabla(\psi_{n-j}-\psi_{n-j-1}),\nabla \zeta)_H \\&\quad - \sum_{j=0}^{n-1}   \frac{\omega_{n-j-1,n}}{\Gamma(1+\alpha)} (\psi_{n-j}-\psi_{n-j-1},F(t_n)\cdot \nabla \zeta)_H   \\ &= (f(t_n),\zeta)_H -  g_\alpha(t_n) \cdot (D(t_n)\nabla \psi_0,\nabla \zeta)_H + g_\alpha(t_n) \cdot  (\psi_0, F(t_n) \cdot\nabla \zeta)_H
\end{aligned} 
\end{equation}
%or written differently by multiplying by $\Delta t$ and  bringing the $\psi_n$-terms to the left-hand side and the remaining terms to the right-hand side
%$$\begin{aligned}& (\psi^n,u)_H + (\Delta t)^{\alpha}   D (\nabla\psi_{n},\nabla u)_H -(\Delta t)^{\alpha}  (\psi_{n},F\cdot \nabla u)_H   \\ &=(\psi^{n-1},u)_H + (\Delta t)^{\alpha}   D (\nabla\psi_0,\nabla u)_H-(\Delta t)^{\alpha}  (\psi_0,F\cdot \nabla u)_H \\ &\quad -\Delta t g_\alpha(t) D(\nabla \psi_0,\nabla u)_H +\Delta t g_\alpha(t) (\psi_0, F \cdot\nabla u)_H \\
%&\quad -(\Delta t)^{\alpha} \sum_{j=1}^{n-1}  D (\nabla(\psi_{n-j}-\psi_0),\nabla u)_H +(\Delta t)^{\alpha} \sum_{j=1}^{n-1}  (\psi_{n-j}-\psi_0,F\cdot \nabla u)_H \Big]
%\end{aligned} $$
for any test function $\zeta$. In particular, taking $\zeta=1$ gives
$$\begin{aligned}& \int_\Omega \psi^n \dx =\int_\Omega \psi^{n-1} \dx    + \Delta t_n \int_\Omega f(t_n) \dx,  
\end{aligned} $$
i.e., the Fokker--Planck setting with $f\equiv 0$ yields discrete mass conservation. We implement the discrete system in open-source computing platform \linebreak FEniCS, see \cite{alnaes2015fenics}.

%$$\begin{aligned}& (\psi^n,u)_H + (\Delta t)^{\alpha} D (\nabla\psi_{n},\nabla u)_H - (\Delta t)^{\alpha} (\psi_{n},F\cdot \nabla u + u\div F)_H   \\ &=(\psi^{n-1},u)_H -\Delta t \cdot g_\alpha(t) \cdot \big(D(\nabla \psi_0,\nabla u)_H + (F \cdot \nabla \psi_0,u)_H\big)
%\\&\quad + (\Delta t)^{\alpha} D (\nabla\psi_0,\nabla u)_H - (\Delta t)^{\alpha} (\psi_0,F\cdot \nabla u + u\div F)_H 
%\\&\quad  - (\Delta t)^{\alpha} \sum_{j=1}^{n-1} \Big[ D (\nabla(\psi_{n-j}-\psi_0),\nabla u)_H -  (\psi_{n-j}-\psi_0,F\cdot \nabla u + u\div F)_H \Big] 
%\end{aligned} $$
 We consider the space interval $\Omega=(-5,15)$ with $\Delta x=1/1024$  and the time interval $[0,T]$ with $T=5$ where the $n$-th time step is given by $t_n=5(n/100)^2$. Moreover, we select as the initial data the Gaussian
$$\psi(0,x)=\psi_0(x)=\frac{1}{\sigma \sqrt{2\pi}} \text{exp}\Big(-\frac12 \Big( \frac{x-\mu}{\sigma} \Big)^2 \Big)$$
for $\sigma=0.1$ and $\mu=2$. 

Regarding model parameters, we choose $D=1$ and $f\equiv 0$. We take the space-time dependent force $F(t,x)=\sin(t)+x$ in Sec. \ref{Sec:Ex2} similar to \cite{angstmann2015generalized,mustapha2022second,le2016numerical,pinto2017numerical}. However, we first consider the case of an absent force $F \equiv 0$ in Sec. \ref{Sec:Ex1}, i.e., we are in the setting of a classical subdiffusion equation. In Sec. \ref{Sec:Ex3}, we consider the physically defeasible time-fractional Fokker--Planck equation with the Caputo derivative on the left-hand side, see again Sec. \ref{Sec:Derivation}, and compare this model numerically to the physically meaningful model that we have analyzed in this work.

\subsection{Example 1: Subdiffusion equation} \label{Sec:Ex1}

As we consider $F\equiv 0$ in this example, we essentially study the time-fractional heat equation
$$\pta \psi(x,t)=\Delta \psi(x,t),$$
which is also referred to as subdiffusion equation.

We observe the typical behavior of a subdiffusive equation in the numerical simulations. At early times, the time-fractional model evolves faster stand the integer-order model. In Fig. \ref{Fig:F0_AlphaVary} (a), we see that the solution is more damped for $\alpha<1$ than for $\alpha=1$ at $t=0.02$. Moreover, the damping is larger for smaller values for $\alpha$. However, this behavior is exactly flipped if one considers a point further in time, e.g. $t=0.5$ as depicted in Fig. \ref{Fig:F0_AlphaVary} (b). After the initial fast evolution of the subdiffusion equation, the process is slower, and we observe that the smallest maximal value is represented by $\alpha=1$ at $t=0.5$. We can also observe that for $\alpha=1$ the typical round shape is present, whereas for $\alpha<1$ the tip at $x=2$ is less round.





% Figure environment removed

% Figure environment removed

% Figure environment removed


We consider the time evolution for $\alpha=1$ in Fig. \ref{Fig:F0_TimeVary} (a) and for $\alpha=\frac12$ in Fig. \ref{Fig:F0_TimeVary} (b). The typical diffusion process can be observed and again, we notice the spikier tip for $\alpha=\frac12$. Moreover, the support of the function is larger for smaller $\alpha$.



Lastly, we try to fit the solution $\psi$ for different values of $\alpha$. The goal is to analyze whether it is necessary to consider the more complicated (analytically and numerically) time-fractional model, or whether this model's behavior can be replicated by an integer-order model.  This is done in Fig. \ref{Fig:F0_Fitting}, and we observe that the subdiffusive behavior cannot be imitated directly by the standard Fokker--Planck equation. Again, we observe the different support for each curve and the difference in the tip at $x=2$.





\subsection{Example 2: Space-time dependent force} \label{Sec:Ex2}
This time, we consider the space-time dependent force $F(x,t)=\sin(x)+t$ and therefore, we study the time-fractional Fokker--Planck equation
$$\begin{aligned}
&\pt \psi(x,t)-\Delta \ptb\psi(x,t)+ \div(F(x,t)\ptb \psi(x,t)) = \ga D\Delta \psi_0 - \ga \div(F\psi_0).
\end{aligned}$$

Again, we observe the typical initial behavior of a subdiffusive equation. At the start, the time-fractional model evolves much faster stand the integer-order model. In Fig. \ref{Fig:F1_AlphaVary} (a), we see that the solution is more damped for $\alpha<1$ than for $\alpha=1$ at $t=0.02$. However, this time, we observe that the symmetry of the probability density functional $\psi$ is lost for $\alpha<1$. In the case of $\alpha=\frac12$ and $\alpha=\frac14$, the solution admits a large support up to the right end of the domain. that  In Fig. \ref{Fig:F1_AlphaVary} (b), we have plotted $\psi$ at a later time.  We observe that $\alpha=1$ is vastly different from the case of $\alpha<1$. This is also pronounced by the fact that $\ga(t) \to 0$ as $t \to \infty$ for $\alpha<1$, but in the case of $\alpha=1$ it holds $\ga(t) \equiv 1$, i.e., the right-hand side is just as large for all times.





% Figure environment removed

% Figure environment removed


We consider the time evolution for $\alpha=1$ in Fig. \ref{Fig:F0_TimeVary} (a) and for $\alpha=\frac12$ in Fig. \ref{Fig:F0_TimeVary} (b). The typical diffusion process can be observed and again, we notice the edgier tip for $\alpha=\frac12$. Moreover, the support of the function is larger for smaller $\alpha$.

Lastly, we try to fit the solution $\psi$ for different values of $\alpha$. This is done in Fig. \ref{Fig:F0_Fitting}, and we observe that the subdiffusive behavior cannot be imitated by an integer-order model. Again, we observe the different support for each curve and the difference in the tip at $x=2$.



% Figure environment removed

\subsection{Example 3: Model comparison} \label{Sec:Ex3}

We consider the model as introduced in \eqref{Eq:ModelWrong} with no right-hand side, i.e.,
\begin{equation} \label{Eq:ModelWrong1} \begin{aligned}
&\pta \psi(x,t)-D \Delta \psi(x,t) +\div \big(F(t,x) \psi(x,t) \big) =0,
\end{aligned}\end{equation}
and we discretize it in the same manner as done for the time-fractional Fokker--Planck equation in \eqref{Eq:FP_Discretized}. Since this model has been studied in literature, we want to give it some attention by comparing it to the physically meaningful model. Again, we consider $F(x,t)=\sin(x)+t$.

We compare it for $\alpha=\frac14$ in Fig. \ref{Fig:Wrong} (a) and for $\alpha=\frac34$ in Fig. \ref{Fig:Wrong} (b) for several time steps. We notice that the error gets larger for increasing time, and it is also more pronounced for smaller values for $\alpha$. We argue that this results from the fact that these models coincide for $\alpha=1$ and by continuity of the fractional parameter, the difference only gets larger the further one is from $\alpha=1$. Moreover, it holds $\ga(t) \to 0$ as $t \to \infty$ for $\alpha<1$ and therefore, it makes sense that asymptotically the right-hand side is negligible.

% Figure environment removed

%\section*{Some sources}

%Good \cite{angstmann2015generalized}: $F=-x+\eps \sin(5\pi t)$. In the case $\eps = 0$, where the external force does not vary in time, the results from the numerical simulations for the first moment and the variance are indistinguishable for the nondelayed forcing and the trap-time delayed forcing, in agreement with the algebraic analysis. The further discussion below is based on the case $\eps = 1$, i.e., the external force varies periodically in time.

%Mustapha A second-order accurate numerical scheme for a time-fractional Fokker-Planck equation: $F=\sin(t)-x$

%PENG The existence of mild and classical solutions for time fractional Fokker–Planck equations

%Pinto Numerical solution of a time-space fractional Fokker Planck equation with variable force field and diffusion: $F(x)=\sin(t)+x$, also in \cite{le2016numerical}

%\cite{le2018a}: A SEMIDISCRETE FINITE ELEMENT APPROXIMATION OF A: $F=-V'$ with $V=x^4/4 -x^2/2 - x \cos t$.

%Deng Numerical algorithm for the time fractional Fokker–Planck equation  $U=\cos x-6x$ and $F=-U'$

%A high-order compact difference method for time fractional: $F=e^{(x-1/2)^2}$

%A Space-Time Petrov-Galerkin Spectral Method for Time Fractional Fokker-Planck Equation with Nonsmooth Solution: $F=-x-1$, also have rhs-force $f$, $F=-1$, assume nonpositive and decreasing, have inhom Dirichlet

%Interval Shannon Wavelet Collocation Method for Fractional Fokker-Planck Equation: $F=-1$, $\psi_0=x(1-x)$, inhom Dir

%Numerical algorithms for the time-space tempered fractional Fokker-Planck equation: $F(x)=x^2$, zero initial, zero Dir and forcing fct and Gaussian initial in other ex and $F(x)=x$



%The acknowledgments section should not be numbered.
\section*{Acknowledgments}
Supported by the state of Upper Austria.

%%%%%%%%%%%%%%%%%%%%%%%%%%%%%%%%%%%%%%%%%%%%%%%%%%%%%%
%          7. REFERENCES SECTION
%%%%%%%%%%%%%%%%%%%%%%%%%%%%%%%%%%%%%%%%%%%%%%%%%%%%%%

%       READ THIS SECTION CAREFULLY

{\small	
	\bibliography{literature.bib}
	\bibliographystyle{AIMS} }

\medskip
% The information below will be filled in by AIMS production staff.
Received xxxx 20xx; revised xxxx 20xx; early access xxxx 20xx.
\medskip

\end{document}

