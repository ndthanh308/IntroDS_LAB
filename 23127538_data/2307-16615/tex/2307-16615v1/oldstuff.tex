\newpage
$$g_{1+\alpha}*u=\ga * 1 * u \leq u_T \ga *1 = u_T$$
$$\gb*\int_\Omega |\ga * \pt \psi_k|^2 \dx $$

%and exploit the inverse convolution property $\ga*\pta u=u$ to obtain
%$$\begin{aligned} &\|\ptb \psi_k(t)\|_{L^2}^2 + D \big(\ga* \|\nabla \ptb \psi_k\|_{L^2}^2\big)(t) \\  &\leq  (\ga* (\|\div F\|_{L^\infty} \|\ptb \psi_k\|^2_{L^2})(t)+C \|f\|_{L^2_\alpha V'}^2  \\ &\quad + C (\ga * (g_{2\alpha-1}  \|F\|_{L^4}^2))(t) \|\psi_0\|_{L^4}^2   + C g_{3\alpha-1}(t) D_\infty^2 \|\nabla \psi_0\|_{L^2}^2.
%\end{aligned}$$
%No initials come because there would $\|\ptb \psi_k(0)\|_{L^2}^2$ but the Caputo of a constant is zero. Further, we used $\ga^2=Cg_{2\alpha-1}$ and convolving gives $g_{3\alpha-1}$ with is continuous for $3\alpha-1>1$ i.e $\alpha>2/3$ or L1 for $\alpha>1/3$

We integrate this inequality over the time span $(0,t)$, $t \leq T_k$, which yields 
$$\begin{aligned}
&\frac{1}{2\Gamma(1-\alpha)t^\alpha} \|\ptb \psi_k\|_{L^2_t L^2}^2 + D \|\nabla \ptb \psi_k\|_{L^2_t L^2}^2 \\  &\leq   C\|f\|_{L^2H^{-1}}^2 + \int_0^t \|\div F\|_{L^\infty} \|\ptb \psi_k\|_{L^2}^2  \dd s    + C g_{2\alpha}(t)  \| \psi_0\|^2_{H^1_0} \Big(  1   +  \|F\|_{L^\infty L^4}^2  \Big).
\end{aligned}$$
can absorb it for
$$\frac{1}{T^\alpha} > \|\div F\|_{L^\infty L^\infty} \Leftrightarrow t < \frac{1}{\|\div F\|_{L^\infty L^\infty}}$$
So the existence is restrained for short time only hm.

we could not have put $\ga$ to $\ptb \psi_k$ but do
$$Cg_{2\alpha-1} + \eps\|\ptb \psi_k\|_{L^2}^2$$
and then integrate
$$Cg_{2\alpha} + \eps \|\ptb \psi_k\|_{L^2_tL^2}^2$$
we could absorb the right term with Dirichlet bdy. The first term is integrable. So need L1 gronwall version, I mean $u(t) \leq a(t) + \int_0^t u(s) ds$ for $a \in L^1$. but then we also get an energy inequality where $g_{2\alpha}(t)$ is on the rhs. holds for a.e. t in (0,T) but cannot take supremum since would explode. so, is it really bdd then to extract subseq? get
$$\int_0^t \|\nabla \ptb \psi_k\|_{L^2}^2 \ds \leq \ga(t).$$




%On the other hand, we could have done
%$$(\pt \psi_k,\ga*\pt\psi_k)_{L^2} \geq C\|\pt^{1-\frac{\alpha}{2}}\psi_k\|_{L^2}^2$$
FROM HERE OLD

We note again that it holds $\phi_k - \init \ga = \ga * \pta \phi_k$ thanks to the inverse convolution property, see \eqref{Eq:InverseConvolution}.
We take this function as the test function in the variational Fokker--Planck equation \eqref{Eq:FP}, i.e., $\zeta= \phi_k - \init \ga = \ga * \pta \phi_k$, which gives  
\begin{equation} \label{Eq:TestingRHS}\begin{aligned} &(\pta \phi_k,\ga * \pta \phi_k)_{L^2} +D  \|\nabla \phi_k\|_{L^2}^2 \\ %
&=(F\phi_k,\nabla \phi_k)_{L^2} -g_\alpha(t) \cdot  (F \phi_k,\nabla \psi_k^0)_{L^2} + \ga(t) \cdot   D \big(\nabla \phi_k,\nabla \init\big)_{L^2}  =:R.
\end{aligned}\end{equation}

$$(F\phi,\nabla \phi) = -(\div F, \phi^2)-(F\nabla \phi,\phi)$$
i.e.
$$(F\phi,\nabla \phi) = -\frac12 (\div F, \phi^2)$$

$$\begin{aligned}
&\|\ptb \psi_k\|_{L^2_tL^2}^2 + D (g_{1+\alpha}*\|\nabla \ptb \psi_k\|_{L^2}^2)(t) \\  &\leq   C\|f\|_{L^2_T(H^1)'}^2 g_{1+\alpha}(T) +C \int_0^t g_{1+\alpha}(t-s) \|\ptb \psi_k(s)\|_{L^2}^2  \dd s  \\ &\quad   + C g_{3\alpha}(t)  \| \psi_0\|^2_{H^1} \Big(  \|D\|_{L^\infty L^\infty}^2   +  \|F\|_{L^\infty L^4}^2  \Big).
\end{aligned}$$
Then we can apply the Henry--Gronwall lemma, see \eqref{Lem:Gron4}, to conclude
$$\|\ptb \psi_k\|_{L_t^2L^2}^2 \leq C\cdot(1+g_{3\alpha}(t)\|\psi_0\|^2_{H^1}) + C (E*(1+g_{3\alpha})(t)$$

We estimate the left-hand side of \eqref{Eq:TestingRHS} by the fractional chain inequality \eqref{Eq:Chain}, which yields
$$
\frac12 \pta \|\phi_k-\ga \init\|^2_{L^2}  \leq (\pta \phi_k,\phi_k-\ga \init)_{L^2} = (\pta \phi_k,\ga * \pta \phi_k)_{L^2}. 
$$
%Regarding the right-hand side of the inequality, we integrate the last term containing $\omega(u_k)$ by parts, see
%\eqref{Eq:IntParts}, and get
%$$\begin{aligned} -  \big(M\omega(u_k)  q \phi, \nabla \init\big)_Y &=  \big(M \nabla \phi_k (\nabla \init)^\top q,u_k\big)_Y + \big(M \phi_k \nabla \nabla \init q,u_k\big)_Y \\ &\quad + \big(u_k \cdot q,M \nabla \phi_k \cdot \nabla \init\big)_Y + \big(u_k \cdot q,M \phi_k \div \nabla \init\big)_Y. 
%\end{aligned} $$
We apply the H\"older inequality to obtain the following bound for the right-hand side $R$, see \eqref{Eq:TestingRHS}, 
$$\begin{aligned} R \leq{}& \|F\|_\infty  \|\phi_k\|_{L^2} \|\nabla \phi_k\|_{L^2} +  \ga(t) \cdot\! \Big(  \|F\|_\infty \|\phi_k\|_{L^2}  \|\nabla \init\|_{L^2}+ D \|\nabla \phi_k\|_{L^2}  \|\nabla \init\|_{L^2}   \Big).
\end{aligned}$$
Then,  we have thanks to Young's inequality
$$\begin{aligned} R \leq&~{} \frac{1}{D} \|F\|_\infty^2  \|\phi_k\|_{L^2}^2 + \frac{D}{4} \|\nabla \phi_k\|_{L^2}^2  + \frac{1}{4} \|F\|_\infty^2\|\phi_k\|_{L^2}^2   + \ga(t)^2 \|\nabla \init\|_{L^2}^2  \\ &+\frac{D}{4} \|\nabla \phi_k\|_{L^2}^2   + D\ga(t)^2 \|\nabla \init\|_{L^2}^2  
\end{aligned}$$
After putting the left-hand and right-hand sides together, it holds
$$\begin{aligned} &\frac12 \pta \|\phi_k-\ga \init\|^2_{L^2} +D  \|\nabla \phi_k\|_{L^2}^2 \\ &\leq \frac{1}{D} \|F\|_\infty^2  \|\phi_k\|_{L^2}^2 + \frac{D}{4} \|\nabla \phi_k\|_{L^2}^2  + \frac{1}{4} \|F\|_\infty^2\|\phi_k\|_{L^2}^2   + \ga(t)^2 \|\nabla \init\|_{L^2}^2  \\ &\quad +\frac{D}{4} \|\nabla \phi_k\|_{L^2}^2   + D\ga(t)^2 \|\nabla \init\|_{L^2}^2  \end{aligned}$$
and absorbing the terms from the right-hand side gives
$$\begin{aligned} &\frac12 \pta \|\phi_k-\ga \init\|^2_{L^2} +\frac{D}{2}  \|\nabla \phi_k\|_{L^2}^2 \\ &\leq C\|\phi_k\|_{L^2}^2 + C\ga(t)^2 \|\nabla \init\|_{L^2}^2.
\end{aligned}$$
If we convolve, then we get
$$\begin{aligned} &\frac12 \|\phi_k(t)-\ga(t) \init\|^2_{L^2} +\frac{D}{2}  \|\nabla \phi_k\|_{L^2_\alpha H}^2 \\ &\leq (\gb*\|\phi_k(t)-\ga(t) \init\|^2_{L^2})(0) \ga(t) + C\|\phi_k\|_{L^2_\alpha H}^2 + Cg_{3\alpha-1} \|\nabla \init\|_{L^2}^2.
\end{aligned}$$

We note that $\ga^2=\frac{\Gamma(2\alpha-1)}{\Gamma(\alpha)^2}g_{2\alpha-1}$ is integrable for $\alpha\in (\tfrac12,1)$, and it holds $g_1*g_{2\alpha-1}=g_{2\alpha}$ which is continuous, bounded, and monotonically increasing on $[0,T]$ for $\alpha\in(\tfrac12,1)$.  We integrate the energy inequality on $(0,t)$ and exploit the representation $\pta v = \pt (g_{1-\alpha} * v)$ of the Riemann--Liouville derivative which gives
\begin{equation} \label{Eq:EnergyFP}\begin{aligned} & \frac12 \big(\gb* \|\phi_k-\init \ga\|_{L^2}^2\big)(t)  +\frac{1}{4\lambda} \|\nabla \phi_k\|_{L^2_tH}^2 + \frac{\eps}{2} \| \nabla \phi_k\|_{L^2_t H}^2   \\
&\leq C \|\phi_k\|_{L^2_tH} +C\|\nabla \init\|_{L^2}^2  g_{2\alpha}(T).
\end{aligned}\end{equation}
Further, we derive a lower bound for the first term of the left-hand side by noting that $(g_1*v)(t) \leq T^{\alpha} 1(1-\alpha) (\gb*v)(t)$, see \eqref{Eq:KernelNorm}, and therefore, it yields
$$\begin{aligned} &\frac12 \big(\gb* \|\phi_k-\init \ga\|_{L^2}^2\big)(t) \\ &\geq \frac{T^{-\alpha}}{21(1-\alpha)} \int_0^t  \|\phi_k(s)-\init \ga(s)\|_{L^2}^2 \ds \\ &\geq \frac{T^{-\alpha}}{21(1-\alpha)} \int_0^t \big|\,\|\phi_k(s)\|_{L^2}-\ga(s)\|\init \|_{L^2} \big|^2\ds \\
&=\frac{T^{-\alpha}}{21(1-\alpha)} \int_0^t \|\phi_k(s)\|_{L^2}^2 - 2\ga(s)\|\phi_k(s)\|_{L^2}\|\init \|_{L^2} +g_{\alpha}^2(s)\|\init \|_{L^2}^2\ds,
\end{aligned} $$
where we applied the reverse triangle inequality in the second estimate.
The function $g_{\alpha}$ is in $L^2(0,t)$ for any $\alpha\in (\tfrac12,1)$ and the integral of $g_\alpha^2$ is positive. We apply the H\"older inequality in the second term of the integral and note that the $L^2(0,t)$-norm of $g_\alpha$ has the upper bound $C(\alpha)T^{\alpha-1/2}$. Thus, it yields
$$\begin{aligned}   \frac12 \big(\gb* \|\phi_k-\init \ga\|_{L^2}^2\big)(t)   &\geq \frac12 \|\phi_k\|_{L^2_tH}^2 - 2 C(\alpha) T^{\alpha-1/2} \|\init \|_{L^2}  \|\phi_k\|_{L^2_tH} \\ &\geq \frac14 \|\phi_k\|_{L^2_tH}^2 - C(T,\alpha) \|\init \|_{L^2}^2,
\end{aligned} $$
where we applied the Young inequality in the last step. We obtain for $\delta=\frac{1}{16}$ the desired estimated 
\begin{equation} \label{Est:SolFP} \begin{aligned} & \frac{1}{2} \big(\gb* \|\phi_k-\init \ga\|_{L^2}^2\big)(t) + \frac{1}{8} \|\phi_k\|_{L^2_tH}^2  +\frac{D}{2} \|\nabla \phi_k\|_{L^2_tH}^2  \\
	&\leq C \|\phi_k\|_{L^2_tH}
	+ C\| \nabla \init\|_{L^2}^2.
\end{aligned}\end{equation}
The combined energy inequality becomes
\begin{equation} \label{Eq:EnergyBack}\begin{aligned} & \big(\gb* \|\phi_k-\init \ga\|_{L^2}^2\big)(t) +  \|\phi_k\|_{L^2_tV}^2    %+ \|u_k(t)\|_{L^2}^2 +  \|\nabla u_k\|_{L^2_tH}^2 
\\
&\leq C(\alpha,T) \|M^{1/2} \init\|_{H^1(D;W^{1,\infty}(\Omega)) }^2  \int_0^t g_{2\alpha-1}(s) \|u_k(s)\|_{L^2}^2 \ds
\\ &\quad + C(\alpha,T) \big( \| \init\|_{H^1}^2 + \|u_k^0\|_{L^2}^2\big),
\end{aligned}\end{equation}
where we took the minimum of each prefactor of the norms on the left-hand side and divided the inequality by this value.
We want to apply the Gronwall lemma to this estimate and therefore,
we focus on the relevant terms and estimate the superfluous nonnegative terms on the left-hand side by their trivial lower bound zero. We obtain the inequality
$$\begin{aligned} \|u_k(t)\|_{L^2}^2 
&\leq C(\alpha,T) \|M^{1/2} \init\|_{H^1(D;W^{1,\infty}(\Omega)) }^2  \int_0^t g_{2\alpha-1}(s) \|u_k(s)\|_{L^2}^2 \ds
\\ &\quad + C(\alpha,T) \big( \| \init\|_{H^1}^2 + \|u_k^0\|_{L^2}^2\big),
\end{aligned}$$
and for $w(t)=\|u_k(t)\|_{L^2}^2$ we have an inequality of the form
$$\begin{aligned} w(t) &\leq C_1 + C_2  (b w)(t),\end{aligned}$$ %
where $b$ is a positive and integrable function for any $\alpha\in(\tfrac12,1)$. Therefore, the Gronwall lemma, see \eqref{Lem:Gron4}, yields
$$ \|u_k(t)\|_{L^2}^2 \leq C(\alpha,T) \cdot \big(  \| \init\|_{H^1}^2  + \|u_k^0\|_{L^2}^2  \big) \cdot \textup{exp}\bigg(\frac{T^{2\alpha-1}}{2\alpha-1} \|M^{1/2} \init\|^2_{H^1(D;W^{1,\infty}(\Omega)) }\bigg). $$
We note that the initial conditions of the Galerkin system are defined by $u_k^0=\Pi_{H_k} u^0$ and $\init=\Pi_{H_k} \psi^0$. Therefore, it holds $\|u_k^0\|_{L^2}^2 \leq \|u^0\|_{L^2}^2$ and $$\|M^{1/2}\init\|_{H^1(D;W^{1,\infty}(\Omega)) }^2 \leq \|M^{1/2}\psi^0\|_{H^1(D;W^{1,\infty}(\Omega)) }^2,$$
see \cite[Theorem 8.1.11]{brenner2008mathematical} with regards to the stability of the Ritz projection in $W^{1,\infty}$.
We plug this result back into the estimate \eqref{Eq:EnergyBack} to achieve the $k$-uniform energy estimate \eqref{Eq:EnergyIneq}.


\subsection*{Convergence of subsequences} 
In the second step of the proof, we obtained the $k$-uniform energy estimate \eqref{Eq:EnergyIneq} in \eqref{Lem:EstComb}. Next, we extract weakly/ weakly-$*$ converging subsequences of the Galerkin solution $(u_k,\phi_k)$. Moreover, we want to  prove strong convergence of $u_{k_j}$ in $L^2(0,T;H_0)$ in order to pass to the limit $j\to \infty$ in  the nonlinear terms of the variational Fokker--Planck equation.   \medskip


In \ref{Lem:EstComb}, we have derived a $k$-uniform bound on $u_k$ and $\phi_k$. Thanks to the Banach--Alaoglu and Eberlein--\v{S}mulian theorems, see \cite[Theorem 8.10
]{alt2016linear}, there is a weakly/weakly-$*$ converging subsequences $\phi_{k_j}$. In particular, we obtain the convergences
\begin{equation} \label{Eq:Weak1} \begin{aligned}	
%u_{k_j} &\longweak u &&\text{weakly-$*$ in } L^\infty(0,T;H_0), \\
%u_{k_j} &\longweak u &&\text{weakly\phantom{-*} in } L^2(0,T;V_0), \\
\phi_{k_j} &\longweak 		 \phi &&\text{weakly\phantom{-*} in }
L^2(0,T;V). 
\end{aligned}\end{equation}


We obtain the strong convergence of $\phi_{k_j}$ in $L^2(0,T;H_0)$ by applying the Aubin--Lions compactness lemma, see \eqref{Eq:aubin}. For this, we need to bound the time derivative of $\phi_k$ in the dual space of $V_0$. By the typical inequalities and the energy inequality \eqref{Eq:EnergyIneq}, we consider an arbitrary element $v \in L^{8/(4-d)}(0,T;V_0)$ and estimate with the typical inequalities
$$\begin{aligned}\int_0^T | \langle \pta \phi_k,v \rangle_{V_0}| \dt   =& \int_0^T  \Big| -\frac12 ((u_k \cdot \nabla)u_k,\Pi_{H_k} v)_{L^2} + \frac12 ((u_k \cdot \nabla)\Pi_{H_k} v,u_k)_{L^2}   \\ &{}  - \nu (\nabla u_k,\nabla \Pi_{H_k} v)_{L^2} - k_B \mu_T \big(\C(M\gb*\phi_k),\nabla \Pi_{H_k} v \big)_{L^2} \Big| \dt
	\\ 
\leq& C \int_0^T \Big( \|u_k\|_{L^4(\Omega)} \|u_k\|_{H^1} \|\Pi_{H_k} v\|_{L^4(\Omega)} + \|u_k\|_{L^4(\Omega)}^2 \|\Pi_{H_k} v\|_{H^1}  \\ &{}  + \|u_k\|_{H^1} \|\Pi_{H_k} v\|_{H^1} +  \|\C(M\gb*\phi_k)\|_{L^2}  \|\Pi_{H_k} v\|_{H^1} \Big) \dt 
 %\\
 %\leq& C \int_0^T \Big( \|u_k\|_{L^2}^{1-d/4} \|u_k\|_{H^1}^{1+d/4} \|\Pi_{H_k} v\|_{L^4(\Omega)} + \|u_k\|_{L^2}^{2-d/2} \|u_k\|_{H^1}^{d/2} \|\Pi_{H_k} v\|_{H^1}  \\ &{}  + \|u_k\|_{H^1} \|\Pi_{H_k} v\|_{H^1} +  \|\phi_k\|_{L^2}  \|\Pi_{H_k} v\|_{H^1} \Big) \dt 
 %\\
%\leq& C \Big( \|u_k\|_{L^\infty H}^{1-d/4} \|u_k\|_{L^2H^1}^{1+d/4} \|v\|_{L^{8/(4-d)}V} + \|u_k\|_{L^\infty H}^{2-d/2} \|u_k\|_{L^2H^1}^{d/2} \| v\|_{L^{4/(4-d)}V} 
%\\ &{} + \|u_k\|_{L^2H^1} \|v\|_{L^2H^1} +  \|\phi_k\|_{L^2L^2} \|v\|_{L^2H^1} \Big)
	\\
	\leq& C\|v\|_{L^{8/(4-d)} V}.
\end{aligned}$$
 It yields that $\pta \phi_k$ is bounded in $L^{8/(4+d)}(0,T;V_0
 ')$. 
 Moreover, thanks to the Aubin--Lions lemma, see \eqref{Eq:aubin}, we conclude the convergences  %
\begin{equation} \label{Eq:Weak2} \begin{aligned}	
\pta \phi_{k_j} &\longweak \pta \phi &&\text{weakly\phantom{-*} in } L^{8/(4+d)}(0,T;V_0'), \\
\phi_{k_j} &\longrightarrow  \phi &&\text{strongly\hspace{1mm} in } L^2\big(0,T;L^{r}(\Omega;\R^d)\big), 
\end{aligned}\end{equation}
where it holds $r \in [1,\infty)$ for $d=2$ and $r \in [1,6)$ for $d=3$.

We note that we can test the weak form of $\phi_k$ again by $\hzeta=g_\alpha * \pta \phi_k$, i.e., we consider the tested weak form \eqref{Eq:TestingRHS}. However, this time we do not apply the fractional chain inequality, but we exploit the coercivity of the kernel function in the integrated weak form, see
 \eqref{Eq:Coercive}. We insert $t=T$ and use the bounds from before and obtain the boundedness of $\pt^{\alpha/2} \phi_k$ in $L^2(0,T;H)$. Therefore,  we can apply the fractional Aubin--Lions lemma, see \eqref{Eq:aubinfractional}, and obtain the strong convergence of $\phi_{k_j}$ in $L^2(0,T;H)$, i.e., it holds
\begin{equation} \label{Eq:Weak3} \begin{aligned}	
\pt^{\alpha/2} \phi_{k_j} &\longweak \pt^{\alpha/2} \phi &&\text{weakly\phantom{-*} in } L^{2}(0,T;H), \\
\phi_{k_j} &\longrightarrow  \phi &&\text{strongly\hspace{1mm} in } L^2(0,T;H).
\end{aligned} \end{equation}


