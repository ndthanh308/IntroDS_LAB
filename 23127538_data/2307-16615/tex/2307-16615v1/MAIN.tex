\documentclass{aims} % Use the aims.cls file to compile your paper
\usepackage{amsmath,float}
\usepackage{paralist}
\usepackage[misc]{ifsym}
\usepackage{epsfig} 
\usepackage{epstopdf} 
\usepackage[colorlinks=true]{hyperref}
\hypersetup{urlcolor=blue, citecolor=red}
\allowdisplaybreaks
\def\theequation{\arabic{section}.\arabic{equation}}
\textheight=8.2 true in
 \textwidth=5.0 true in
  \topmargin 30pt
   \setcounter{page}{1}

\def\currentvolume{X}
 \def\currentissue{X}
  \def\currentyear{200X}
   \def\currentmonth{XX}
    \def\ppages{X-XX}
     \def\DOI{ }
%\usepackage[pagewise]{lineno}\linenumbers
\pdfpagesattr
{ /CropBox [55 40 546 730] }
%%%%%%%%%%%%%%%%%%%%%%%%%%%%%%%%%%%%%%%%%%%%%%%%%%%%%%%%
%          2. CUSTOM COMMANDS
%%%%%%%%%%%%%%%%%%%%%%%%%%%%%%%%%%%%%%%%%%%%%%%%%%%%%%%%
% PLEASE NOTE: The AIMS cls file is updated regularly with the standard AMS usepackages - you do NOT need to insert custom commands for AMS symbols,

% Insert your custom commands in this section.
% Please minimize the use of "newtheorem", "newcommand", and use equation numbers only in situations where they provide essential convenience.
% Use \usepackage{amssymb}, \usepackage{amsthm}, etc, but please do not define individual mathematical symbols unless it is completely necessary.
% Try to avoid defining your own macros.

% Do not change or remove the commands below.
\newtheorem{theorem}{Theorem}[section]
\newtheorem{corollary}[theorem]{Corollary}
\newtheorem*{main}{Main Theorem}
\newtheorem{lemma}[theorem]{Lemma}
\newtheorem{proposition}[theorem]{Proposition}
\newtheorem{conjecture}[theorem]{Conjecture}
\newtheorem*{problem}{Problem}
\theoremstyle{definition}
\newtheorem{definition}[theorem]{Definition}
\newtheorem{remark}[theorem]{Remark}
\newtheorem{assumption}[theorem]{Assumption}
\newtheorem*{notation}{Notation}
\newcommand{\ep}{\varepsilon}
\newcommand{\eps}[1]{{#1}_{\varepsilon}}


%%%%%%%%%%%%%%%%%%%%%%%%%%%%%%%%%%%%%%%%%%%%%%%%%%%%%%%%
%         3. HEADER AND FOOTER SECTION
%%%%%%%%%%%%%%%%%%%%%%%%%%%%%%%%%%%%%%%%%%%%%%%%%%%%%%%%


% Place the running head in [], and the full title of the article in {}.
\title[Weak solutions to  the time-fractional Fokker--Planck equation]
% Running head is the full title or shortened version of the full title. This will appear at the top of odd pages. It should be no more than 40 characters to fit within the width limit.
{Well-posedness and simulation of weak solutions to  the time-fractional Fokker--Planck equation with general forcing} % Only the first word and proper nouns should be capitalized.

% Place all authors' names in []. This will be shown as the running head on even pages. Leave {} empty.
% Please use `and' to connect the last two authors' names if applicable.
% List full names if possible. If all authors' full names will not fit, use FirstNameInitial. MiddleNameInitial. LastName, only last names, or full names of first few authors, et al.
\author[Marvin Fritz]{}

% 2020 MSC numbers are required.
\subjclass{Primary: 35R11; 35D30; 35A01; 65M60}
% Please provide a minimum of 5 keywords or phrases.
\keywords{Time-fractional Fokker--Planck equation; well-posedness of weak solutions; Galerkin approximation; nonuniform L1 scheme.}

% Put your short thanks below. For thanks/acknowledgments over 30 words, please place them in \section*{Acknowledgments} located above the reference section.
% Remove \thanks{The first author is supported by NSF grant xx-xxxx} if nothing is added here.
\thanks{The author is supported by the state of Upper Austria.}


% Add corresponding author in the footnote of the first page if necessary.
% Add $^*$ adjacent to the corresponding author's name on the first page if necessary.
% The corresponding author in your article should match the corresponding author listed for your article in EditFlow (if applicable).
% In the example shown below, the first author is the corresponding author. Please move or remove $^*$ as needed for your article.
\thanks{$^*$Corresponding author: Marvin Fritz}

%%%%%%%%%%%%%%%%%%%%%%%%%%%%%%%%%%%%%%%%%%%%%%%%%%%%%%%%%%
%      4. AUTHOR NAMES/ADDRESSES/AFFILIATIONS SECTION
%%%%%%%%%%%%%%%%%%%%%%%%%%%%%%%%%%%%%%%%%%%%%%%%%%%%%%%%%%

\usepackage[T1]{fontenc}

\usepackage{xcolor,bm, amssymb, mathtools,subfigure}
%\usepackage[colorlinks=true,linkcolor=blue,citecolor=green!65!black]{hyperref}
\usepackage[capitalise,nameinlink]{cleveref}

\usepackage{tikz,bm}
\usepackage[sort,compress]{cite}


\newcommand{\ECM}{{E\hspace{-.05em}C\hspace{-.15em}M}}
\newcommand{\MDE}{{M\hspace{-.15em}D\hspace{-.1em}E}}
\newcommand{\TAF}{{T\hspace{-.15em}A\hspace{-.1em}F}}
\newcommand{\ecm}{\phi_{\ECM}}
\newcommand{\taf}{\phi_{\TAF}}
\newcommand{\phim}{\hat\phi}
\newcommand{\hzeta}{\hat\zeta}
\newcommand{\nablax}{\nabla_{\!x}}
\newcommand{\nablaq}{\nabla_{\!q}}
\newcommand{\hx}{\hat{x}}
\newcommand{\hq}{\hat{q}}
\newcommand{\init}{\psi_k^0}
\newcommand{\psim}{\hat\psi}
\newcommand{\mde}{\phi_{\MDE}}
\newcommand{\phib}{\bm{\phi}}
\newcommand{\pt}{\partial_t}
\renewcommand{\eps}{\varepsilon}
\newcommand{\Pta}{D_t^\alpha}
\newcommand{\pta}{\p_t^\alpha}
\newcommand{\capa}{{}^C\!\p_t^\alpha}
\newcommand{\ptb}{\p_t^{1-\alpha}}
\newcommand{\Ptb}{D_t^{1-\alpha}}
\newcommand{\capb}{{}^C\!\p_t^{1-\alpha}}
\newcommand{\ga}{g_{\alpha}}
\newcommand{\gb}{g_{1-\alpha}}
\newcommand{\X}{\vec{X}}

\newcommand{\y}{\vec{y}}
\newcommand{\z}{\vec{z}}
\newcommand{\trace}{\textup{tr}}
\newcommand{\ds}{\,\textup{ds}}
\newcommand{\J}{\mathcal{J}}
\newcommand{\T}{\mathcal{T}}
\newcommand{\dev}[1]{\text{dev}\left[#1\right]}
\newcommand{\lambdap}{\lambda^{\!\textup{pro}}}
\newcommand{\lambdad}{\lambda^{\!\textup{deg}}}
\newcommand{\dx}{\dd x}
\newcommand{\sgn}[1]{\ \text{sgn}\left(#1\right) }
\newcommand{\frc}[2]{{\textstyle\frac{#1}{#2}}}
\newcommand{\minus}{\scalebox{0.7}[1.0]{\( - \)}}
\newcommand{\inv}{^{\scalebox{0.7}[1.0]{\( - \)}1}}%
\newcommand{\loc}{\textup{loc}}

\newcommand{\tp}{^\intercal}
\renewcommand{\exp}[1]{e^{#1}}
\renewcommand{\i}{\mathrm{i}}
\renewcommand{\Re}{\mathfrak{Re}}
\renewcommand{\Im}{\mathfrak{Im}}
\newcommand{\B}{\lbrace -1, 1\rbrace}
\newcommand{\Ind}{\mathbbm{1}}
\renewcommand{\O}{\mathcal{O}}
\newcommand{\diag}{\operatorname{diag}}
\DeclareMathOperator*{\esssup}{ess\,sup}
\newcommand{\Id}{\operatorname*{Id}}
\newcommand{\sign}{\operatorname*{sign}}
\newcommand{\op}[1]{\operatorname*{#1}}
\newcommand{\Exp}[1]{\operatorname*{Exp}\left(#1\right)}
\newcommand{\tr}{\operatorname{tr}}
\newcommand{\doublehookrightarrow}{%
	\mathrel{\mathrlap{{\mspace{4mu}\lhook}}{\hookrightarrow}}}
\newcommand{\sca}[2]{\langle #1, \; #2 \rangle}
\newcommand{\norm}[1]{\lVert #1 \rVert}
\newcommand{\abs}[1]{\left| #1 \right|}
\newcommand{\mat}[1]{\begin{bmatrix} #1 \end{bmatrix}}
\renewcommand{\vec}[1]{\pmb{#1}}
\newcommand{\vct}[1]{\vec{#1}}
\newcommand{\ten}[1]{\pmb{#1}}
\newcommand{\tns}[1]{\ten{#1}}
\newcommand{\tnsfour}[1]{\pmb{#1}}
\newcommand{\bbc}[1]{\left( #1 \right)}
\newcommand{\rbc}[1]{\left[ #1 \right]}
\newcommand{\drv}[2]{\frac{\mathrm{d} #1}{\mathrm{d} #2}} %
\newcommand{\D}[2]{\partial_{#1} #2}
\renewcommand{\div}{\textup{div}}
\newcommand{\grad}{\nabla}
\newcommand{\gradS}{\nabla\!_s}
\newcommand{\Jac}[2][]{ \operatorname*{D_{#1}}\! #2  }
\newcommand{\Hess}[1]{ D^2#1  }
\newcommand{\dd}{\mathop{}\!\mathrm{d}}
\renewcommand{\d}{\mathop{}\!\mathrm{d}}
\newcommand{\ddt}{\frac{\dd}{\dd\mathrm{t}}}
\newcommand{\dt}{\,\textup{d}t}
\newcommand{\p}{\partial}
\newcommand{\symgrad}{\bm{\varepsilon}}
\newcommand{\CD}{\mspace{0mu}^C\!D}
\newcommand{\RA}[1]{r_h\!\left(#1\right)}
\newcommand{\Ea}{\op{E}\!} %
\newcommand{\La}[1]{\mathfrak{L}\lbrace #1 \rbrace }
\newcommand{\Lai}[1]{\mathfrak{L}\inv\left\lbrace #1 \right\rbrace }
\newcommand{\F}{\mathcal{F}} %
\newcommand{\Fi}{\mathcal{F}^{-1}} %
\newcommand{\FFT}[1]{\operatorname{FFT}\left\lbrace #1 \right\rbrace} %
\newcommand{\IFFT}[1]{\operatorname{IFFT}\left\lbrace #1 \right\rbrace} %
\newcommand{\Z}{\mathbb{Z}} %
\newcommand{\E}{\mathcal{E}} %
\newcommand{\R}{\mathbb{R}} %
\newcommand{\I}{\mathcal{I}}
\newcommand{\red}{\textcolor{red}}
\newcommand{\blue}{\textcolor{blue}}
\newcommand{\orange}{\textcolor{orange}}
\newcommand{\cyan}{\textcolor{cyan}}
\newcommand{\green}{\textcolor{green!70!black}}
\newcommand{\CX}{\mathbb{C}} %
\newcommand{\N}{\mathbb{N}} %
\newcommand{\PY}{\mathbb{Y}} %
\newcommand{\PZ}{\mathbb{Z}} %
\newcommand{\PW}{\mathbb{W}} %
\newcommand{\Cphi}{\vartheta}
\newcommand{\Cmu}{\varrho}
\renewcommand{\rho}{\varrho}
\newcommand{\Cu}{\varsigma}
\newcommand{\con}{\hookrightarrow}
\newcommand{\com}{\mathrel{\mathrlap{{\mspace{4mu}\lhook}}{\hookrightarrow}}}
\newcommand{\Cpsi}{\varkappa}
\newcommand{\Cchi}{\varpi}
\newcommand{\Func}{H}
\renewcommand{\L}{\mathcal{L}} %
\newcommand{\W}{W} %
\renewcommand{\H}{H} %
\newcommand{\C}{\mathcal{C}} %
\newcommand{\OT}{\Omega_T} %
\newcommand{\Ba}{\mathcal{U}} %
\newcommand{\Diru}{\Sigma_{1}} %
\newcommand{\Dirpsi}{\Sigma_{2}} %
\newcommand{\Banach}{\mathcal{B}} %
\newcommand{\HS}{\mathcal{H}} %
\newcommand{\HSV}{\mathcal{V}} %
\newcommand{\HSVD}{\mathcal{V}'} %
\newcommand{\Prob}[1]{\operatorname{\mathbb{P}}\rbc{#1}}
\newcommand{\Var}[1]{\operatorname{Var}\rbc{#1}}
\newcommand{\Gaussian}{\mathcal{N}}
\newcommand{\Cov}{\tns{C}}
\newcommand{\WN}{\mathcal{W}} %
\newcommand{\WNhat}{\hat{\mathcal{W}}} %
\newcommand{\BesselJ}[1]{\mathcal{J}_{#1}}
\newcommand{\BesselK}[1]{\mathcal{K}_{#1}}
\newcommand{\erf}{\operatorname{erf}}
\newcommand{\erfc}{\operatorname{erfc}}
\newcommand{\hypTwoOne}{{\;}_{2}F_{1}}
\newcommand{\longweak}{\relbar\joinrel\rightharpoonup}
\newcommand{\MatLab}{MatLab R2016b}
\newcommand{\rownumber}{\stepcounter{rownumber}\therownumber}
\newcommand{\RHS}{\mathcal{F}}
\newcommand{\opL}{\operatorname{L}}
\newcommand{\FDphiM}{\p_t^{\alpha}(\phi^m-\phi^m_0)}
\newcommand{\FDmuM}{\p_t^{\alpha}(\mu^m-\mu^m_0)}
\newcommand{\FDdivuM}{\p_t^{\alpha}(\nabla\cdot \vec{u}^m-\nabla\cdot\vec{u}^m_0)}
\newcommand{\FDsymuM}{\p_t^{\alpha}\left(\symgrad(\vec{u}^m)-\symgrad(\vec{u}^m_0)\right)}

\newcommand{\dq}{\,\text{d}q}
\newcommand{\hpsi}{{\widehat{\psi}}}
\newcommand{\hphi}{{\widehat{\phi}}}
\newcommand{\hY}{{\widehat{Y}}}
\newcommand{\hX}{{\widehat{X}}}

\newcommand{\ext}{{\textup{ext}}}

\newcommand{\drag}{{\textup{drag}}}

\newcommand{\ran}{{\textup{ran}}}

\newcommand{\rms}{{\textup{rms}}}

\begin{document}
\maketitle

\centerline{\scshape
Marvin Fritz$^{{\href{mailto:marvin.fritz@ricam.oeaw.ac.at}{\textrm{\Letter}}}*1}$}

\medskip

{\footnotesize
% Enter the full affiliation and country name:
% Do not insert commas or periods at the end of lines.
 \centerline{$^1$Computational Methods for PDEs, Johann Radon Institute for Computational and}
 \centerline{Applied Mathematics, Linz, Austria}
} % Do not forget to end {\footnotesize with the sign }

\bigskip

% The name of the handling editor will be entered by AIMS production staff.
% "Communicated by Handling Editor" is not needed for special issue.
 %\centerline{(Communicated by Handling Editor)}

%%%%%%%%%%%%%%%%%%%%%%%%%%%%%%%%%%%%%%%%%%%%%%%%%%%%%%%
%             5. ABSTRACT
%%%%%%%%%%%%%%%%%%%%%%%%%%%%%%%%%%%%%%%%%%%%%%%%%%%%%%%

\begin{abstract}
In this paper, we investigate the well-posedness of weak solutions to the time-fractio\-nal Fokker--Planck equation. Its dynamics is governed by anomalous diffusion, and we consider the most general case of space-time dependent forces. Consequently, the fractional derivatives appear on the right-hand side of the equation, and they cannot be brought to the left-hand side, which would have been preferable from an analytical perspective.  
For showing the model's well-posedness, we derive an energy inequality by considering nonstandard and novel testing methods that involve a series of convolutions and integrations. We close the estimate by a Henry--Gronwall-type inequality. Lastly, we  propose a numerical algorithm based on a nonuniform L1 scheme and present some simulation results for various forces.
\end{abstract}

%%%%%%%%%%%%%%%%%%%%%%%%%%%%%%%%%%%%%%%%%%%%%%%%%%%%%%
%                   6. BODY
%%%%%%%%%%%%%%%%%%%%%%%%%%%%%%%%%%%%%%%%%%%%%%%%%%%%%%

	\section{Introduction}
	
This paper is concerned with the existence of weak solutions to a system of nonlinear partial differential equations that arises in the kinetic theory of dilute solutions of polymeric fluids. Within this class of models we focus on finitely-extensible nonlinear elastic, FENE-type, dumbbell models with a corotational drag term. In contrast to previous literature on the analysis of these models we assume power law waiting times in the derivation of the system, which results in the appearance of a time-fractional derivative in the Fokker--Planck equation describing the evolution of the probability density function. This raises new questions about the study of well-posedness, and we provide rigorous results concerning the existence of global-in-time weak solutions to the system of partial differential equations featuring in the model.

Dilute polymer models are derived and extensively described in the monograph \cite{bird1987dynamics2} and in the book by \"Ottinger \cite{ottinger2012stochastic}; see also \cite{suli2018mckeanvlasov} for a mathematically rigorous derivation of the Hookean bead-spring-chain model from Brownian dynamics. We also refer to the papers \cite{lemou2002viscoelastic,herrchen1997a} for a comparison of several FENE-type dumbbell models. Such systems are of microscopic-macroscopic type since they involve a coupling of the (macroscopic) Navier--Stokes equations for the description of incompressible fluid flow and the Fokker--Planck equation for the microscopic processes associated with 
the statistical properties of polymer molecules immersed in the fluid. 
Concerning the weak and strong well-posedness of FENE-type models, we refer to the works \cite{jourdain2004existence,kreml2010on,masmoudi2013global,zhang2006local,renardy1991an}. More general dilute polymer models are analyzed in \cite{barrett2005existence,barrett2007existence,barrett2008existence,barrett2010existence,barrett2010existence2}. Further, we mention the papers \cite{lions2000global,lions2007global,schonbek2009existence,masmoudi2008well,barrett2005existence,barrett2009numerical,debiec2023corotational,lin2008global,busuioc2014fene}, which, similarly to the discussion herein, are concerned with dumbbell models that assume a corotational drag term in the Fokker--Planck equation.  In such models it is supposed that polymer molecules are not stretched by the surrounding solvent, although they are allowed to rotate without stretching; see, for example, \cite{la2020diffusive}.


Time-fractional differential equations have been the focus of considerable attention in the mathematical and engineering literature in recent years. Such equations are nonlocal in time and have an innate history effect. They are of relevance in applications where memory effects are present and hereditary properties of materials are studied; see, for example, the textbooks on viscoelasticity \cite{mainardi2022fractional,yang2020general}, hydrology \cite{su2020fractional}, financial economics \cite{fallahgoul2016fractional}, and mechanical processes \cite{atanackovic2014fractional,pilipovic2014fractional}. The time-fractional Fokker--Planck system, in particular, allows subdiffusive behaviour and has been previously studied in \cite{metzler1999anomalous,metzler1999deriving,metzler2000random,barkai2000continuous,barkai2001fractional,henry2006anomalous,henry2010fractional,henry2010introduction,langlands2008anomalous} with regards to its derivation and applicability. The articles \cite{pinto2017numerical,le2016numerical,le2018a,le2019existence,le2021alpha} have investigated the numerical analysis and the simulation of solutions to the time-fractional Fokker--Planck equation. The time-fractional model considered herein has been explored computationally in \cite{beddrich2023numerical}, albeit in the simpler setting of a linear (Hookean) elastic spring force instead of the FENE spring model that we study here.
 
We employ a spatial Galerkin approximation in conjunction with a compactness argument to prove the existence of weak solutions to the time-fractional Navier--Stokes--Fokker--Planck system under consideration. More specifically, we discretize the system in space and derive appropriate energy bounds, which then enable us to pass to the limit in the discretized system.  Spatial discretizations of dilute polymer models were previously considered  in  \cite{barrett2009numerical,barrett2011finite,barrett2012finite}. In addition, weak solutions to time-fractional PDEs have been investigated using the Galerkin approach in the publications \cite{fritz2021sub,fritz2020time,fritz2021equivalence}. There have also been initial steps in the analysis of a decoupled time-fractional Fokker--Planck equation with time-dependent forces; see the papers \cite{fritz2023well,mclean2020regularity,le2019existence,le2021alpha,mclean2021uniform}. However, the coupling of the time-fractional Fokker--Planck equation to the Navier--Stokes system gives rise to new technical complications, which have not been addressed previously.


In \Cref{Sec:Derivation} we derive the model from the Langevin equation assuming power-law waiting time. In this way time-fractional derivatives in the sense of Riemann--Liouville appear in the associated  Fokker--Planck equation. By mimicking the technique for the derivation of the standard dumbbell model, a time-fractional Navier--Stokes--Fokker--Planck system is obtained.  In \Cref{Sec:Prelim} we introduce several function spaces of Sobolev-type and recall some important results from the theory of fractional derivatives, including chain inequalities and embedding theorems. In \Cref{Sec:Form} we transform the model in order to make it amenable to the subsequent analysis.  We then equip the model with suitable initial and boundary conditions and we make use of the associated Maxwellian to rescale the Navier--Stokes--Fokker--Planck system. In \Cref{Sec:Analysis} we finally state and prove a theorem asserting the existence of large-data global-in-time weak solutions to the model with a time-fractional derivative of order $\alpha \in (\tfrac12,1)$. 


	\section{Modeling of the time-fractional Fokker--Planck equation} \label{Sec:Derivation}
	


%The time-fractional Fokker--Planck model can be derived by utilizing Langevin equations, we refer to Appendix \ref{App:Derivation} at the end of the article for the details on the equation's derivation.
%Historically, three years after Einstein's publication on Brownian motion, the French scientist Paul Langevin published his study \cite{langevin1908theory} on Brownian motion, which obtained the identical results utilizing an entirely new mathematical framework. In fact, Langevin's method is far closer to the conventional physics approach: To account for the unpredictable “kicks” that the Brownian particle receives from smaller fluid particles, one alters Newton's equation of motion by adding a randomly fluctuating force. The resulting equation is what we would identify as a stochastic differential equation in the present day. In fact, it turns out that its moment is governed by the Fokker--Planck equation.


 Let $\Omega \subset \R^d$, $d\in \mathbb{N}$, be a Lipschitz domain and $T<\infty$ a fixed final time. Shortly, we denote the time-space domain by $\Omega_T=\Omega \times (0,T)$. Let $\psi:\Omega_T \to \R$ denote a probability density function that represents the probability at a time $t$ of finding the center of mass of a particle in the volume element $x+\d x$. 


The time-fractional Fokker--Planck model with space-time dependent force can be derived by utilizing the Langevin equations, see \cite{magdziarz2008equivalence,magdziarz2009stochastic}, and the model reads
\begin{equation} \label{Eq:DerivFP}
\pt \psi(x,t)- D \Delta \Ptb \psi(x,t)+\div\!\left(F(x,t) \Ptb \psi(x,t) \right)  = 0.\end{equation}
Here, $F:\Omega_T  \to \R^d$ denotes the space-time dependent external force and $D$ the diffusion coefficient. In contrast to the typical model of integer-order, the fractional derivative in the sense of Riemann--Liouville is introduced, which is defined by
$$\Ptb u(t) =\frac{1}{\Gamma(\alpha)} \ddt \int_0^t \frac{u(s)}{(t-s)^{1-\alpha}} \ds, $$
where $\Gamma$ denotes Euler's Gamma function. We introduce the singular kernel function $g_\alpha(t)=t^{\alpha-1}/\Gamma(\alpha)$ and therefore, we can rewrite the fractional derivative with the convolution operator as $$\Ptb u =\pt (\ga*u).$$ In the limit case of $\alpha=1$, the model is reduced to the standard Fokker--Planck equation.
This time-fractional model has been studied in the previous works \cite{huang2020new,le2016numerical,le2018semidiscrete,le2021alpha,mclean2021uniform,mustapha2022second,pinto2017numerical,yan2019finite} with regard to numerical methods and in \cite{le2019existence} for the existence of mild and classical solutions. 

We note that the fractional derivative in the sense of Riemann--Liouville appears naturally in the equation's derivation, see \cite{magdziarz2008equivalence}. However, the fractional derivative in the sense of Caputo would be preferable considering our variational approach to time-fractional partial differential equations and the involved analytical machinery. The Caputo derivative of order $\alpha$ is denoted by $\pta$ and it reads 
\begin{equation} \label{Eq:Caputo} \pta u= \Pta (u-u_0).
\end{equation} 
Here, $u_0$ is the initial of the underlying system, which shall fulfill $$\big(\gb*(u-u_0)\big)(0)=0$$ in the case that $u$ is not continuous. 

If the force is time-independent, we could simply convolve the time-fractional Fokker--Planck equation \eqref{Eq:DerivFP} with the singular kernel function $\gb$ and exploit the properties $\gb*\Ptb u=u$ and $\pta u=\gb*\pt u$, see below in Section \ref{Sec:Prelim}, to obtain the time-fractional equation
\begin{equation} \label{Eq:DerivFP3}
\pta \psi(x,t)-D \Delta \psi(x,t) +\div \big(F(x) \psi(x,t) \big)  =0,\end{equation}
which would be more accessible for analytical and numerical methods. However, we cannot simply exclude the relevant cases of time-dependent forces. In such cases, one would require a product rule for fractional derivatives to write $F\Ptb \psi$ as $\Ptb(F\psi)-\Ptb F \psi$.  However, this is not correct for fractional derivatives, as it can be already seen from the example $\psi=F=1$. Then it holds 
$$F\Ptb \psi = \ga \neq 0 = \ga-\ga =\Ptb(F\psi)-\Ptb F \psi.$$
There is a fractional version of the Leibniz rule that requires two smooth functions $f,g$ and reads \cite[Theorem 2.18]{diethelm2010analysis}
$$\Pta(fg)=f \Pta g + \sum_{k=1}^\infty \binom{\alpha}{k} \pt^k f \cdot (g_{1-k+\alpha}*g).$$
We can already see the issue of this formula. It requires smooth functions, and it turns out that there is an infinite sum on the right-hand side. Let us assume that $F$ and $\psi$ are smooth. Then we want to bring the fractional derivative in front of $F\psi$ by the formula
$$F \Ptb \psi = \Ptb(F\psi) - \sum_{k=1}^\infty \binom{1-\alpha}{k} \pt^k F \cdot (g_{2-k-\alpha}*\psi).$$
Afterward, we convolve the system with $\gb$ and obtain the system
\begin{equation} \label{Eq:ModelWrong} \begin{aligned}
&\pta \psi(x,t)-D \Delta \psi(x,t) +\div \big(F(t,x) \psi(x,t) \big)  \\&=\sum_{k=1}^\infty \binom{1-\alpha}{k} \gb*\big(\pt^k F \cdot (g_{2-k-\alpha}*\psi)\big),
\end{aligned}\end{equation}
There have been several published articles that studied this model but neglecting the complete right-hand side. This is also the reason it is claimed in \cite{heinsalu2007use} that such a model (with neglecting the right-hand side) is “physically defeasible” and its solution “does not correspond to a physical stochastic process”.  In the case that $F$ is affine linear in $t$, i.e. $F(t,x)=a(x)+b(x)t$,  it yields 
\begin{equation*} \begin{aligned}
&\pta \psi(x,t)-D \Delta \psi(x,t) +\div \big(F(t,x) \psi(x,t) \big)  \\&=(1-\alpha) \cdot b(x) \cdot (g_{2-2\alpha}* \psi)(t)
\end{aligned}\end{equation*}
 We would rather not consider infinitely many terms on the right-hand side of the PDE for a general $F$ and therefore, we instead 
exploit the definition \eqref{Eq:Caputo} of the Caputo derivative to obtain
$$\Ptb u(t) = \ptb u(t) + \Ptb u_0=\ptb u(t)+u_0g_{\alpha}(t),$$
and rewrite the time-fractional Fokker--Planck equation \eqref{Eq:DerivFP} as follows:
\begin{equation} \label{Eq:DerivFP2}
\begin{aligned}
&\pt \psi(x,t)-D \Delta \ptb \psi(x,t) +\div\big(F(x,t) \ptb \psi(x,t) \big) \\ &=\ga D\Delta  \psi_0 - \ga \div(F\psi_0).
\end{aligned} \end{equation}
We consider an initial condition $\psi_0 \in H_0^1(\Omega)$ and therefore, it holds that the right-hand side has the regularity $L^p(0,T;H^{-1}(\Omega))$ with $p<1/(1-\alpha).$
%We follow this approach since, in our opinion, it is more accessible and easier to understand in which way the fractional derivative enters than the CTRW approach
%We translate the $(x,q)$-coordinates to the center of mass by defining the new function $\psi(x,q,t):=\tilde\psi(x-\tfrac12 q,x+\tfrac12 q,t)$. Together with the definition of $b$, see \cref{Def:Vectorb}, we obtain
%\begin{equation} \label{Eq:Langevin3} \begin{aligned}&\pt \psi(x,q,t) + \div \bigg(\frac{u(x-\tfrac12q,t)+u(x+\tfrac12q,t)}{2} \ptb\psi(x,q,t)\bigg)\\ &\quad+ \div\bigg(\big(u(x+\tfrac12q,t)-u(x-\tfrac12q,t)\big) \ptb\psi(x,q,t) - \frac{2F(q)}{\zeta }  \ptb\psi(x,q,t)\bigg) \\ &= \frac{k_B\mu_T}{2\zeta } \Delta_x \ptb\psi(x,q,t)+ \frac{2k_B\mu_T}{\zeta } \Delta_x \ptb\psi(x,q,t). \end{aligned} \end{equation}
%We make an assumption of local homogeneity, i.e., the velocity's variation over the microscopic length scale of a single dumbbell is assumed to be small. Consequently, the arithmetic mean $\big(u(x-\frac12q,t)+u(x+\frac12q,t)\big)/2$ can be approximated by $u(x,t)$, which is relevant for the second term on the left-hand side of \cref{Eq:Langevin3}.  In case of the third term, we use Taylor's expansion to obtain 
%\begin{equation} \begin{aligned} u(x+\tfrac12 q,t)-u(x-\tfrac12 q,t) &= \nabla u(x,t) q + \mathcal{O}(|q|^3) \\ &=\Big(\sigma\big(u(x,t)\big) + \omega\big(u(x,t)\big)  \Big)q + \mathcal{O}(|q|^3),
	%%	\end{aligned}
	%\label{Eq:ApproxTaylor}
	%\end{equation}
%where we have further split the gradient of $u$ into its symmetric and antisymmetic part, which are defined as follows:
% \begin{equation} \label{Eq:Omega} \sigma(u)=\frac{\nabla u + (\nabla u)^\top}{2}, \qquad \omega(u)=\frac{\nabla u - (\nabla u)^\top}{2}.
 %	\end{equation}
%In the approximation \cref{Eq:ApproxTaylor}, we omit the $\mathcal{O}(|q|^3)$ term and further, we also omit the symmetric part of $u$ for analytical simplicity, i.e., we consider the corotational case. Thus, we obtain the time-fractional PDE
%\begin{equation} \label{Eq:FokkerDim}\begin{aligned}\pt \psi + \div(u \ptb\psi) + \div\Big(\omega( u) q \ptb\psi - \frac{2F(q)}{\zeta }  \ptb\psi\Big) \\ = \frac{k_B \mu_T}{2\zeta } \Delta_x \ptb\psi + \frac{2k_B\mu_T}{\zeta } \Delta_x \ptb\psi.\end{aligned} \end{equation}
%Lastly, we put the equation into its nondimensionalized form. Therefore, we define quantities
%$$q=l_0 \hat q$$
%with microscopic length scale %$l_0=\sqrt{k_B \mu_T/H}$. Further, we introduce the nondimensional Weissenberg number $\lambda=\zeta  U_0/(4HL_0)$ that reflects the ratio of the microscopic to the macroscopic time-scale.
%We multiply \cref{Eq:FokkerDim} by $L_0/U_0$ and obtain (omitting the hats over the symbols for the ease of readability)
%
	%\begin{aligned} 
 %&\pt \psi =\tfrac{1}{2\lambda} %\div(\nabla\ptb \psi + U'q \ptb\psi),
%\end{aligned}\end{equation} 
%where we have further defined the parameter $\eps:=l_0^2/(8\lambda L_0^2)$. %
%Finally, the Navier--Stokes equations governing the velocity $u$ are derived in a standard manner from the conservation of momentum and mass equations, see \cite{temam2001navier}. We note that polymeric fluids are non-Newtonian fluids and the polymer molecules contribute
%a symmetric extra stress tensor $\tau(x,t)$ to the total stress tensor while the equations for the conservation
%of linear momentum and mass remain the same. %
%Because the continuum mechanical “macroscopic” equations of incompressible fluid flow are coupled to a “microscopic” model, we call the polymer models under consideration microscopic–macroscopic-type models. Here, the microscropic equations are governed by the time-fractional Fokker--Planck equation describing the statistical properties of particles in the continuum. We begin by presenting these equations and collecting the relevant assumptions on the various parameters featuring in the model.
%We generalize the time-fractional Fokker--Planck equation   by assuming a space-time varying diffusion function $D:\Omega \times (0,T) \to \R^d$ and an additional force $f:\Omega \times (0,T) \to \R$ and, therefore, we study in this work the following time-fractional advection-diffusion equation:
%\begin{equation} \label{Def:FP} \begin{aligned}
%	&\pt \psi -  \div(D(x,t)\nabla  \ptb  \psi) + \div (F(x,t) \ptb  \psi)  \\  &=f(x,t) + \ga \div(D\nabla \psi_0) - \ga \div(F\psi_0).\end{aligned} \end{equation}
We equip this equation with the homogeneous Dirichlet boundary condition $\psi=0$ on $\p\Omega$. However, our analytical results also hold for no-flux boundary conditions (i.e. homogeneous Neumann). Moreover, the system is equipped
with the initial condition $\psi(0) = \psi^0 \geq 0$ in  $\Omega$. %Mathematically, it is enough to demand $\psi_0 \in H_0^1(\Omega)$ to show our desired results on the well-posedness of weak solutions. 
Physically, $\psi^0$ is a given probability density function, i.e.,  it is nonnegative function and satisfies $\int_\Omega \psi^0(x) \dx =1$ (however, we do not need to assume such properties in our well-posedness theorem below).   
Integrating the time-fractional Fokker--Planck equation  in $\Omega$  and employing integration by parts, we find
$\ddt \int_{\Omega} \psi(x,t) \dx =0.$ This implies then $\int_{\Omega} \psi(x,t) \dx = 1$  for almost all $t$.






\section{Mathematical preliminaries} \label{Sec:Prelim}
In this section, we introduce some useful definitions and results regarding the fractional derivative in the sense of Riemann--Liouville and recall the Aubin--Lions lemma, which is a key result featuring in proofs of existence of weak solutions to nonlinear PDEs based on compactness arguments. %

For a Hilbert space $H$ with inner product $(\cdot,\cdot)_H$ and norm $\|\cdot\|_H$, we shall denote the duality pairing between $H$ and its dual space $H'$ by $\langle \cdot,\cdot\rangle_H$. We shall denote the inner product on the Bochner space $L^2(0,T;H)$ by $(\cdot,\cdot)_{L^2H}$, and we shall write $(\cdot,\cdot)_{L^2_tH}$ when in this inner product the temporal interval of integration is $(0,t)$ for some $t \in (0,T)$ rather than the complete interval $(0,T)$, i.e., $$(u,v)_{L^2_tH}:=\int_0^t (u(s),v(s))_H \, \text{d}s \qquad \forall\, u,v \in L^2(0,T;H).$$
The norm induced by this inner product will be denoted by $\|\cdot\|_{L^2_t H}$.



\subsection{Riemann--Liouville kernels}
The Riemann--Liouville kernel function $g_\alpha$ of order $\alpha$ is defined by $g_\alpha(t):=t^{\alpha-1}/\Gamma(\alpha)$, $t \in (0,T)$, for $\alpha > 0$ and $g_0(t):=\delta_0(t)$ (the Dirac distribution concentrated at $0$) for $\alpha=0$. We observe that  $g_\alpha \in L^p(0,T)$ for any $\alpha\in (1-1/p,1)$ and $p \in [1,\infty)$, and the kernel function satisfies the following semigroup property; see \cite[Theorem 2.4]{diethelm2010analysis}:
\begin{equation} \label{Eq:Semigroup}
	\ga * g_\beta = g_{\alpha+\beta} \qquad \forall\, \alpha,\beta \geq 0.
\end{equation} 
%This can be proved as follows: One applies Fubini's theorem to interchange the order of integration 
%$$\begin{aligned}(\ga * g_\beta *u)(t) &=\frac{1}{\Gamma(\alpha)\Gamma(\beta)} \int_0^t (t-s)^{\alpha-1} \int_0^s (s-\tau)^{\beta-1} u(\tau) \dd \tau \dd s  \\
%	&=\frac{1}{\Gamma(\alpha)\Gamma(\beta)} \int_0^t u(\tau) \int_\tau^t (t-s)^{\alpha-1} (s-\tau)^{\beta-1}   \dd s \dd \tau,
%\end{aligned}$$
%and the substitution $s=\tau+\sigma(t-\tau)$ then yields
%$$\begin{aligned}(\ga * g_\beta *u)(t) 
%	&=\frac{1}{\Gamma(\alpha)\Gamma(\beta)} \int_0^t u(\tau) (t-\tau)^{\alpha+\beta-1} \int_0^1 (1-\sigma)^{\alpha-1} \sigma^{\beta}   \dd \sigma \dd \tau.
%\end{aligned}$$
%Lastly, we observe using the fundamental property of the Gamma function that the second integral is equal to $\Gamma(\alpha)\Gamma(\beta)/\Gamma(\alpha+\beta)$, see \cite[Theorem D.6]{diethelm2010analysis}, from which we deduce the desired semigroup property \cref{Eq:Semigroup} of $g_\alpha$.

We note that when $\alpha \in (0,1)$, one can bound the $L^p(0,t)$-norm of a function $u:(0,T) \to \R$ by its convolution with $\ga$ as follows: for any $t \in (0,T]$, we have that
\begin{equation}\begin{aligned} \|u\|_{L^p(0,t)}^p := \int_0^t |u(s)|^p \ds  &\leq t^{1-\alpha} \int_0^t (t-s)^{\alpha-1} |u(s)|^p \ds \\ &\leq T^{1-\alpha} \Gamma(\alpha) \big(\ga * |u|^p\big)(t).	
\end{aligned} 
\label{Eq:KernelNorm}
\end{equation}
This implies that the space $$L^p_\alpha(0,T):=\big\{u:(0,T) \to \R:\sup_{t \in (0,T)} (\ga*|u|^p)(t) < \infty \big\},$$ is indeed a subspace of $L^p(0,T)$.
%Further, this estimate can be generalized for a nonnegative function $u:(0,T) \to \R_{\geq 0}$ and for $0<\beta<\alpha<1$ in the following way:
%$$(\ga * u)(t)=\frac{1}{\Gamma(\alpha)} \int_0^t (t-s)^{\beta-1} \frac{(t-s)^{\alpha-1}}{(t-s)^{\beta-1}} u(s) \ds \leq \frac{T^{\alpha-\beta}\Gamma(\beta)}{\Gamma(\alpha)} (g_\beta * u)(t).$$
If the order $\alpha$ of the kernel function $g_\alpha$  is larger than 1, then one can exploit the semigroup property of the kernel and apply Young's convolution inequality (cf. Lemma 3.2 in \cite{Oparnica}) as follows:
$$(g_{1+\alpha}*u)(t)=(g_1*\ga*u)(t)=\int_0^t (\ga*u)(s) \ds \leq \|\ga\|_{L^1(0,t)} \|u\|_{L^1(0,t)},$$
for any $u \in L^1(0,T)$ and any $t \in (0,T]$.


\subsection{Time-fractional derivative} 
We can rewrite the definition of the Riemann--Liouville derivative stated in \cref{Eq:RL} in a compact form by using the convolution operator $*$ as $\pta w=\pt (\gb * w)$.
We refer to the classical textbooks \cite{diethelm2010analysis,baleanu2012fractional} and the newer monographs \cite{jin2021fractional,chen2022fractional} regarding fractional calculus and fractional differential equations.

 We define the fractional Riemann--Liouville--Bochner space   for $\alpha \in (0,1)$ and $p \in [1,\infty)$ on $(0,T)$ with values in $H$ by $$\W^{\alpha,p}(0,T;H):=\big\{u \in L^p(0,T;H) : \gb * u \in W^{1,p}(0,T;H)\big\}.$$
Here, the convolution $\ast$ is of course understood to be with respect to the temporal variable $t \in (0,T)$. In the limit, 
when $\alpha=1$, we have that $g_{1-\alpha} = g_0=\delta$, and then $$\W^{1,p}(0,T;H):=W^{1,p}(0,T;H):=\big\{u \in L^p(0,T;H) : \pt u \in L^p(0,T;H)\big\}.$$
However for $0 < \alpha < 1$,
the Riemann--Liouville space $\W^{\alpha,p}(0,T;H)$ differs from the fractional-order Sobolev--Bochner space
$$W^{\alpha,p}(0,T;H):=\Big\{u \in L^p(0,T;H) : (s,t) \mapsto \tfrac{\|u(t)-u(s)\|_H}{|t-s|^{\alpha+1/p}} \in L^{p}((0,T)\times(0,T))\Big\},$$
which can be confirmed by noting that the function $g_\alpha$ is an element of $\W^{\alpha,p}(0,T):=\W^{\alpha,p}(0,T;\R)$ for $\alpha \in (1-\tfrac{1}{p},1)$ but not of $W^{\alpha,p}(0,T)$; see \cite[Proposition 3.13]{carbotti2021note}. 
%Therefore, we are not able to apply classical results such as embedding theorems for Sobolev--Bochner spaces.

\begin{remark}  
Even though the space $\W^{\alpha,p}(0,T)$ is not a subspace of the Sobolev--Slobodecki\u{\i} space $W^{\alpha,p}(0,T)$, it is nevertheless continuously embedded into $C([0,T])$, the space of uniformly continuous functions defined on $[0,T]$, for $\alpha \in (1- \frac{1}{p},1]$ and $p \in [1,\infty)$; see, \cite[Remark 6.2]{carbotti2021note}. 
\end{remark}

\begin{comment}
Therefore, small values of $\alpha$ have to be studied carefully.
Further, we observe that $\ga$ does not belong to $L^p(0,T)$ for $\alpha \in (0,1-\tfrac{1}{p}]$ (e.g., $\ga \notin L^2(0,T)$ for $\alpha \in (0,\frac12]$) and therefore, we find that $$(\gb*\phi)(0)=0 \qquad \forall\, \phi \in \W^{\alpha,p}(0,T;H),~ \alpha \in (0,1-\tfrac{1}{p}],$$
by the inverse convolution property  \cref{Eq:InverseConvolution}. However, this might contradict a given nontrivial initial condition $\phi^0$. E.g., for $\phi \in \W^{\alpha,2}(0,T;H):=\W^{\alpha,2}(0,T;H)$ it has to hold that $(\gb*\phi)(0)=0$ for $\alpha\in (0,\tfrac12]$ and therefore, PDE solutions with this regularity are only well-posed for $\phi^0=0$. Such an issue can be avoided by studying PDEs of the form $\pta(\phi-\phi^0)=f(\phi)$ and considering instead the regularity of $\phi-\phi^0$, i.e., $\phi \in \W^{\alpha,2}_{\phi^0}(0,T;H)$. However, the time-fractional model \cref{Eq:System} in this work is not of this translated form and therefore, we cannot expect that this system is well-posed for non-zero initials. We note that $\psi_0=0$ is physically unreasonable anyway for probability density functions and, moreover, we will naturally observe in the existence's proof below that the restriction $\alpha \geq \frac12$ naturally appears in the energy estimates.
\end{comment}




We also introduce the following Riemann--Liouville space incorporating a homogeneous initial condition at $t=0$, albeit in a somewhat nonstandard manner:
$$\begin{aligned}
\W^{\alpha,p}_{0}(0,T;H)&:=\big\{u \in \W^{\alpha,p}(0,T;H) : (\gb*u)(0)=0 \big\}.
%\W^{\alpha,p}_{u^0}(0,T;H)&:=\big\{u \in L^p(0,T;H) : u-u^0 \in \W_0^{\alpha,p}(0,T;H) \big\}.
\end{aligned}$$ 
We note that the function $\gb*u:[0,T] \to H$ has a well-defined trace at $t=0$ (even when the function $u$ itself might not have one) thanks to the continuous embedding
$$\gb*u \in W^{1,p}(0,T;H) \hookrightarrow AC([0,T];H).$$
For a given element $z \in H$, the convolution $\gb*z$ should be understood to mean the function $t \mapsto (\gb*g_1)(t) z \in H$; recall that $g_1(t)\equiv 1$ for all $t\geq 0$. Thus, $z \in H$ is in this context now thought of as the mapping $t \mapsto g_1(t)z \in \mathcal{W}^{\alpha,p}(0,T;H)$,  for $\alpha \in (0,1)$, $p \in [1,\infty)$ and $0<\alpha p < 1$, or if $\alpha =1$ and $p \in [1,\infty)$. 
Thanks to the semigroup property \eqref{Eq:Semigroup} we then have that $$t\mapsto (\gb*z)(t)=z\,g_{2-\alpha}(t)=\frac{z}{\Gamma(2-\alpha)}  t^{1-\alpha} \in C([0,T];H)$$ for any $\alpha \in [0,1]$. Thus, for $\alpha \in (0,1]$, $p \in [1,\infty)$ and $z \in H$ we define the following `translated' Riemann--Liouville space:
\begin{equation} \label{Eq:RLSpaceU0}
    \W^{\alpha,p}_{z}(0,T;H):=\big\{u \in L^p(0,T;H) : u-z \in \W^{\alpha,p}(0,T;H), ~ (g_{1-\alpha}*u)(0)=0 \big\}.
\end{equation}
Note that if $\alpha \in (0,1)$, $p \in [1,\infty)$ and $0<\alpha p<1$, or if $\alpha=1$ and $p \in [1,\infty)$, then $u-z \in \W^{\alpha,p}(0,T;H)$ if, and only if $u \in \W^{\alpha,p}_0(0,T;H)$, and therefore, for such $\alpha$ and $p$ we have that $\W^{\alpha,p}_{z}(0,T;H) = \W^{\alpha,p}_0(0,T;H)$ irrespective of the choice of $z \in H$.

Next, we state an inverse convolution (or deconvolution) property. Its name stems from the fact that convolution with the kernel $\ga$ acts as an inverse mapping on the operator of taking $\alpha$-th fractional derivative, up to a term that involves the initial value at $t=0$.
%
\begin{lemma}[Inverse convolution] Let $\alpha \in (0,1]$ and $p\in [1,\infty)$. Suppose further that $H$ is a Hilbert space and $z \in H$. Then, for any $t \in (0,T)$, we have the following equalities:
\begin{align} 	\label{Eq:InverseConvolution1}
(\ga * \pta u)(t) &= u(t) - (\gb*u)(0)\ga(t) \quad &&\forall\, u \in \W^{\alpha,p}(0,T;H), \\ 	\label{Eq:InverseConvolution}
		(\ga* \pta u)(t)    &=u(t)  &&\forall\, u \in \W_{z}^{\alpha,p}(0,T;H). \end{align} 
\end{lemma}
\begin{proof}
	We start with the proof of the equality \cref{Eq:InverseConvolution1}.
Recall that for any function $u \in \W^{\alpha,p}(0,T;H)$ we have $\gb*u \in  AC([0,T];H)$, and the fundamental theorem of calculus for absolutely continuous functions therefore yields, for any $t\in [0,T]$,
$$(\gb *u)(t) - (\gb*u)(0) = \int_0^t \partial_s (\gb * u)(s) \ds=(g_1 * \pta u)(t).$$
We convolve this equality with the kernel $\ga$ and make use of the semigroup property \cref{Eq:Semigroup} to obtain
$$(g_1*u)(t) - (\gb*u)(0) g_{1+\alpha}(t)=g_{1+\alpha}*\pta u,$$
where we have used that  $g_\alpha*1=g_\alpha*g_1=g_{1+\alpha}$, because $\alpha \Gamma(\alpha) = \Gamma(1+\alpha)$.
Next, we differentiate this equality in $t$ and observe that $\pt (g_1*u)=u$, $\pt g_{1+\alpha}=g_\alpha$, and $\pt (g_{1+\alpha}*v)=g_\alpha*v$, which yields \cref{Eq:InverseConvolution1}.
We finally note that  \cref{Eq:InverseConvolution} follows trivially from \cref{Eq:InverseConvolution1} and \cref{Eq:RLSpaceU0}.
%We consider an element $u \in \W_{u^0}^{\alpha,p}(0,T;H)$, i.e., there exists an element $v \in \W_{0}^{\alpha,p}(0,T;H)$ with $u-u^0=v$ and using $v$ in \cref{Eq:InverseConvolution1}, we obtain $\ga*\pta v=v$, i.e.,
%\begin{equation*} \begin{aligned}\ga* \pta (u-u^0)   &= u-u^0 &&\forall\, u \in \W_{u^0}^{\alpha,p}(0,T;H).
	%\end{aligned} \end{equation*} 
%Moreover, we can split the left-hand side thanks to the linearity of the fractional derivative and obtain 
 %\begin{equation*} \begin{aligned}
	%	\ga* \pta u    &= u-u^0  + \ga * \pta u^0 =u  &&\forall\, u \in \W_{u^0}^{\alpha,p}(0,T;H),
	%\end{aligned} \end{equation*} 
	%where we have used that   $\ga * \pta 1 = \ga*\gb =1$ thanks to the semigroup property  \cref{Eq:Semigroup}.
 \end{proof}
 
The following result is a direct consequence of the interaction between fractional derivatives and kernel functions.
\begin{corollary} The following identities hold:
\begin{equation} \label{Eq:DerivativeofKernel}  \begin{aligned}
	\pta (\ga * u ) &=\pt ( \gb * \ga * u) = \pt (1*u) = u &&\forall\, u \in L^1(0,T;H), \\
	\ptb \pta u &=  \pt (\ga * \pta u) = \pt u &&\forall\, u \in  W_{0}^{1,1}(0,T;H).
\end{aligned}
\end{equation}
\end{corollary}




%However, in our setting of the time-fractional Navier--Stokes--Fokker--Planck system, we have already seen that the Riemann--Liouville derivative appears on the left-hand side without the translation of an initial value. This already explains intuitively the restriction on the values of $\alpha$ in the theorem of the system's well-posedness, see \cref{Thm:WellPosedness} below.


%As in the integer-order setting, there are continuous and compact embedding results for Riemann--Liouville spaces.
We shall require the following special case of the classical Aubin--Lions lemma; see \cite{simon1986compact}. Suppose that the Hilbert spaces $V,H,Z$ form a Gelfand triple $V \com H \con Z$. then, the following classical compact embeddings hold: 
\begin{equation} \begin{aligned} \label{Eq:aubin} 
W^{1,1}(0,T;Z) \cap L^p(0,T;V) &\com L^{p}(0,T;H), &&p \in [1,\infty), \\
W^{1,r}(0,T;Z) \cap L^\infty(0,T;V) &\com C([0,T];H), && r \in (1,\infty);
\end{aligned}\end{equation} 
see \cite{simon1986compact}. Several fractional counterparts of the Aubin--Lions lemma have been proposed; see \cite{ouedjedi2019galerkin,wittbold2020bounded,li2018some}. We make use of  the following result; see \cite[Corollary 3.2]{ouedjedi2019galerkin}:
\begin{equation*} \begin{aligned} %
\W^{\alpha,1}(0,T;Z) \cap L^p(0,T;V) &\com L^r(0,T;H), &&p \in (1,\infty), \quad r \in [1,p), \quad \alpha \in (0,1).
\end{aligned}\end{equation*} 
The proof can be easily adapted to the limit case $r=p$ if the $\alpha$-th fractional derivative is in a better space than $L^1(0,T;Z)$. This is done for Caputo derivatives in \cite{li2018some}. In fact, we obtain
%$r \in (\frac{p}{1+\alpha p},\infty) \cap [1,\infty)$.  In the spacial case when $p=2$ and $\alpha \in (\frac12,1]$ it yields
\begin{equation} \begin{aligned} \label{Eq:aubinfractional2} %
		\W^{\alpha,r}(0,T;Z) \cap L^p(0,T;V) &\com L^p(0,T;H), &&r\in (1,\infty), \quad \alpha \in (0,1).
\end{aligned}\end{equation} 

\begin{comment}
Next, we require a Gronwall-type inequality that allows an additional nonnegative factor $b \in L^1(0,T)$ in the integrand on the right-hand side of the inequality. Particularly, this function is only assumed to be integrable, and it is allowed to degenerate.
\begin{lemma}[Gronwall, cf. {\cite[Lemma II.4.10]{boyer2012mathematical}}] \label{Lem:Gron4}
    Let $C_1,C_2$ be  nonnegative constants and let $b\in L^1(0,T)$ be nonnegative. If the  function $u \in L^\infty(0,T)$  satisfies the inequality
    $$u(t) \leq C_1+C_2 \int_0^t b(s) u(s) \, \textup{d}s \qquad \text{for a.a. } t \in (0,T], $$
    then 
    $$u(t) \leq C_1 \textup{exp}\Big(C_2\int_0^t  b(s) \, \textup{d}s\Big) \qquad \text{for a.a. } t \in (0,T]. $$
\end{lemma}
\end{comment}


%\subsection{Fractional chain inequality}
The classical chain rule does not hold for fractional derivatives, but one can use the following inequality as a remedy; see \cite[Theorem 2.1]{vergara2008lyapunov}:
\begin{equation} \label{Eq:ChainOriginal}  \frac12 \pta \|u\|^2_H +\frac12 \gb(t) \|u\|_H^2 \leq (u,\pta u)_H \quad \forall\, u \in \W_{z}^{\alpha,2}(0,T;H),
\end{equation}
for $z \in H$ and almost all $t \in (0,T)$.
%Here, it has to be assumed that  $\big(\gb*(u-u^0)\big)(0)=0$. 

\begin{comment}
We will see that the initial condition $\psi^0$ to the time-fractional system considered later on does not satisfy $\big(\gb*(\phi-\psi^0)\big)(0)=0$. Instead, we say that the solution satisfies the initial condition if $(\gb*\phi)(0)=\psi^0$. One can transform this into the form from before by noting that
$$0=(\gb*\phi)(0)-\psi^0=(\gb*\phi)(0)-(\gb*\ga \psi^0)(0)=(\gb*(\phi-\ga \psi^0))(0).$$
Next, we derive a fractional chain inequality for such functions so that we can apply such a result later on in the existence proof.

We consider a fixed $u \in \H_{0}^\alpha(0,T;H)$ and we  introduce $v=u+\ga z$. Since in this case  $u^0=0$, this gives $(\gb*v)(0)=z$. 
%We note that $\pta v= \pta u$ and therefore we obtain
%$$\begin{aligned}
	%(v,\pta v)_H  &=(u,\pta %u)_H+(\ga z, \pta u)_H 
	%\\ &\geq \frac12 \pta %\|u\|_H^2 + \frac12 \gb \|u\|_H^2 + \ga  (z,\pta v)_H,
%\end{aligned}$$
Using the fractional chain inequality 
\cref{Eq:ChainOriginal} for $u=v-\ga z$, we trivially find
\begin{equation} \label{Eq:Chain} (v-\ga z,\pta v)_H \geq \frac12 \pta \|v-\ga z\|_H^2.\end{equation}

%We note that $u=\ga * \pta u$ thanks to the  inverse convolution property \cref{Eq:InverseConvolution} and inserting  $u=v-g_\alpha z$ yields $v-g_\alpha z=g_\alpha * \pta v$. Therefore, we can also write, instead of \cref{Eq:Chain},
%\begin{equation*}  (\ga * \pta v,\pta v)_H \geq \frac12 \pta \|\ga*\pta v\|_H^2.\end{equation*}
One might ask oneself if there is a generalized inequality of the form of $(\ga*u,u)_H \geq \frac12 \pta \|\ga * u\|_H^2$ for a sufficiently smooth function $u$, i.e., whether the convolution with $\ga$ is coercive in some sense. We partly answer this question in the next lemma.
\end{comment}
%
%We shall also require the inequalities stated in the next lemma. 

%\begin{lemma} Let $z\in H$ be given. For any $u \in \W_{0}^{\alpha,2}(0,T;H)$ and any $v \in \W^{\alpha,2}(0,T;H)$ with $(\gb * v)(0)=z \in H$ we have the following inequalities:
%\begin{align} \label{Eq:Coercive}   \int_0^t (u,\pta u)_H \ds &\geq \cos(\alpha \pi/2)\|\pt^{\alpha/2} u\|_{L^2_tH}^2, \\
%\int_0^t \! (v\!-\!\ga z,\pta v)_H \ds  &\geq \cos(\alpha \pi/2)  \Big( \tfrac12 \|\pt^{\alpha/2} v\|^2_{L^2_tH}-\tfrac{\Gamma(\alpha-1)}{\Gamma(\alpha/2)^2}g_{\alpha}(t) \|z\|_H^2 \Big).\label{Eq:Coercive2} 
%\end{align}
%\end{lemma}
%\begin{proof}
%By \cite[Lemma 3.1]{mustapha2014well} we have that
%\begin{equation} \label{Eq:Mustapha} \int_0^t (\ga*w,w)_H \ds \geq \cos(\alpha \pi/2) \|g_{\alpha/2} * w\|_{L^2_tH}^2 \qquad \forall\, w \in L^2_t(0,T;H).
%\end{equation}
%Hence, with $w=\pta u$ we obtain the inequality
%$$
%\int_0^t (u,\pta u)_H \ds \geq \cos(\alpha \pi/2) \|g_{\alpha/2} * \pta u \|_{L^2_tH}^2,$$
%where we have used the inverse convolution property $g_\alpha * \pta u=u$, see \cref{Eq:InverseConvolution} with $z=0$ to simplify the left-hand side. On the right-hand side, we can make use of the fact that $\pt^{\alpha}u =\pt^{\alpha/2}\pt^{\alpha/2} u$ and again the inverse convolution property \cref{Eq:InverseConvolution} to deduce that
%$$g_{\alpha/2}*\pt^{\alpha} u = g_{\alpha/2}*\pt^{\alpha/2} \pt^{\alpha/2} u = \pt^{\alpha/2} u.$$
%Therefore, we obtain the first of the inequalities stated in the lemma. We note that we can only split the $\alpha$-th derivative into the composition of two $\frac{\alpha}{2}$-th derivatives if the function $u$ satisfies an initial condition of the form $\big(\gb*u\big)(0)=0$. 
%
%
%For $v \in \W^{\alpha,2}(0,T;H)$ with $(g_{1-\alpha} \ast v)(0)=z \in H$, we define $u:=v-g_\alpha z$, and we find that $u \in \W_0^{\alpha,2}(0,T;H)$ and thus
%\begin{equation} \label{Eq:LemmaIneq} \begin{aligned} \int_0^t (v -\ga z,\pta v)_H \ds 
%%&\geq \cos(\alpha \pi/2) \|g_{\alpha/2} * \pta u \|_{L^2_tH}^2 
%%\\ &=\cos(\alpha \pi/2) \|g_{\alpha/2} * \pt^{\alpha/2} \pt^{\alpha/2} u\|_{L^2_tH}^2 \\ 
%\geq \cos(\alpha \pi/2)\|\pt^{\alpha/2} u\|_{L^2_tH}^2. \end{aligned} \end{equation}
%As, trivially, $(a/\sqrt{2} - \sqrt{2}b)^2 \geq 0$ for all $a, b \in \mathbb{R}$, it follows that  $|a-b|^2 \geq \frac{a^2}{2} - b^2$. Therefore, noting $(\pt^{\alpha/2} g_\alpha)^2= (g_{\alpha/2})^2=\tfrac{\Gamma(\alpha-1)}{\Gamma(\alpha/2)^2} g_{\alpha-1}$ and inserting $u=v-\ga z$ into the right-hand side of  \eqref{Eq:LemmaIneq} yields
%\begin{equation*} 
%\begin{aligned}
%\|\pt^{\alpha/2} u\|_{L^2_tH}^2  &=  \|\pt^{\alpha/2} (v-g_\alpha z)\|_{L^2_tH}^2 \\ 
%&\geq \frac12 \|\pt^{\alpha/2} v\|_{L^2_tH}^2 - \|\pt^{\alpha/2} \ga z\|_{L^2_tH}^2 \\
%&\geq    \frac12 \|\pt^{\alpha/2} v\|^2_{L^2_tH}-\tfrac{\Gamma(\alpha-1)}{\Gamma(\alpha/2)^2} \|z\|_H^2 \int_0^t g_{\alpha-1}(s)\ds
%%&\geq  \big|  \|\pt^{\alpha/2} v\|_{L^2_tH}-\big( \tfrac{\Gamma(\alpha-1)}{\Gamma(\alpha/2)^2} g_{\alpha}(t) \big)^{1/2} \|z\|_H \big|^2 
%\\
%&=    \frac12 \|\pt^{\alpha/2} v\|^2_{L^2_tH}-\tfrac{\Gamma(\alpha-1)}{\Gamma(\alpha/2)^2}g_{\alpha}(t) \|z\|_H^2.
%\end{aligned}
%\end{equation*}
%That completes the proof of the lemma.
%\end{proof}





\section{Model revisited} \label{Sec:Form}

Having summarised the required results from fractional calculus, we revisit the mathematical model that we have derived in \Cref{Sec:Derivation}.
Let us assume for the moment that the solution $\psi$ to the Fokker--Planck equation belongs to $\mathcal{W}^{1-\alpha,p}(0,T;H) \cap C([0,T];H)$ for some $\alpha \in (0,1)$ and a suitable Hilbert space $H$, to be chosen. As $\psi \in C([0,T];H)$, it follows that $\|(g_\alpha \ast \psi)(t)\|_H \leq \frac{t^\alpha}{\Gamma(1+\alpha)}\|\psi\|_{C([0,T];H)}$, and therefore  $(g_\alpha \ast \psi)(0)=0$. Hence, $\psi \in \mathcal{W}^{1-\alpha,p}_0(0,T;H)$. It then follows from \eqref{Eq:InverseConvolution1}, with $\alpha$ replaced by $1-\alpha$ and $u=\psi$ there, that $(g_{1-\alpha}\ast \partial_t^{1-\alpha} \psi)(t) = \psi(t)$ for $t \in (0,T)$. 
Motivated by these properties, we introduce the auxiliary function $\phi$ by
\begin{equation} \label{Eq:Substitute} \phi := \ptb \psi = \pt (\ga * \psi),
\end{equation} 
whereby $\psi=g_{1-\alpha}*\phi$. We then have that $\pt \psi = \pt(g_{1-\alpha} \ast \phi) = \pta \phi$; and, thanks to the assumed continuity of $\psi$ (i.e. $\psi \in C([0,T];H)$) it
makes sense to require attainment of the initial condition $\psi(0) = \psi^0$, i.e. $(g_{1-\alpha} \ast \phi)(0) = \psi^0$. We shall therefore introduce the substitution $\phi:=\partial_t^{1-\alpha} \psi$ in \eqref{Def:FP}, which results in the following system of equations:
\begin{equation} \begin{aligned}
	\pt u + (u \cdot \nablax) u - \nu \Delta_x u + \nablax p - \div_x \tau( \gb * \phi) &=0, \\
	\div_x u &=0, \\
	\pta \phi  + (u \cdot \nablax) \phi + \div_q (\omega(u) q \phi)-\tfrac{1}{2\lambda} \div_q(\nablaq \phi + U'q  \phi)   -\eps  \Delta_x  \phi&=0, \end{aligned} 
\label{Eq:System}
\end{equation}
subject to the initial conditions $u(0)=u^0$ and $(\gb*\phi)(0)=\psi^0$ for a given nonnegative $\psi^0$ that fulfils $\int_D \psi^0 \dq=1$. Furthermore, we equip the system with the following boundary conditions:
\begin{align}\label{eq:neumannbc}
\begin{aligned}
u &=0 \qquad\text{on } \partial \Omega \times (0,T), \\
\left(\tfrac{1}{2\lambda} (\nablaq \phi + U'q \phi)-\omega(u) q \phi \right) \cdot n_{\partial D} &=0 \qquad\text{on } \Omega \times \partial D \times (0,T), \\
\eps \nablax \phi \cdot n_{\partial \Omega} &=0 \qquad\text{on } \partial\Omega \times  D \times (0,T).
\end{aligned}
\end{align}




\subsection{The Maxwellian and Maxwellian-weighted function spaces} \label{Sec:Maxwell}
We introduce the normalized Maxwellian by 
\begin{equation}
	\label{Def:Max} M(q)=\frac{e^{-U(\tfrac12 |q|^2)}}{\int_D e^{-U(\tfrac12 |s|^2)} \dd s}.
\end{equation}
Moreover, we define the (Maxwellian-weighted) Hilbert spaces
$$\begin{alignedat}{5}
	&\mathcal{H}=\{h \in L^2(\Omega;\R^d): \div \, h =0\}, \qquad ~\mathcal{H}_0&&=\{h \in \mathcal{H} : h \cdot n_{\partial \Omega} = 0 \text{ on } \partial \Omega\}, 
	\\ &\mathcal{V}=\{v \in H^1(\Omega;\R^d) : \div \, v = 0\}, \qquad ~ \mathcal{V}_0&&=\{v \in \mathcal{V} : v|_{\partial \Omega} = 0 \text{ on } \partial \Omega\},
	\\  &\mathcal{Y}=L^2(\Omega \times D),  \qquad\quad\,\,\,   \widehat{\mathcal{Y}}= L^2_M(\Omega \times D) &&=\{y \in \mathcal{Y}: \|y\|_{\widehat{\mathcal{Y}}} := \|M^{1/2}y \|_\mathcal{Y}<\infty \}, 
	\\ &\mathcal{X} = H^1(\Omega \times D), \qquad\quad \widehat{\mathcal{X}}= H_M^1(\Omega \times D) &&= \{\phi \in \mathcal{X}: \|\phi\|_\hX <\infty  \},  \\
&\mathcal{Z}=H^1(D;H^1(\Omega)), ~ \widehat{\mathcal{Z}} =H_M^1(D; H^1(\Omega))&&=\{\zeta \in \mathcal{Z}: \|\zeta\|_\hZ <\infty  \}, 
\end{alignedat}$$
where the norms on $\hX$ and $\hZ$ are defined by $\|\phi\|_\hX^2:= \|\phi\|_\hY^2 + \|\nablaq \phi\|_\hY^2 + \|\nablax \phi\|_\hY^2$ and $\|\zeta\|_\hZ^2 :=\|\zeta\|_\hX^2 + \|\nablax \nablaq \zeta\|_\hY^2$. Obviously, $H_M^2(\Omega \times D) \subseteq \hZ$, where $H^2_M(\Omega \times D)$ is the subspace of $H^1_M(\Omega \times D)$ consisting of all functions defined on $\Omega \times D$ whose second (weak) partial derivatives belong to $\hY=L^2_M(\Omega \times D)$.
We refer to \cite{barrett2005existence} regarding theoretical results on these weighted Hilbert spaces. In particular, we have the Gelfand triples
$$\begin{aligned} &\HSV \com \HS \hookrightarrow \HSV', \quad &&\HSV_0 \com \HS_0 \hookrightarrow \HSV_0', \\
	&\mathcal{X} \com \mathcal{Y} \hookrightarrow \mathcal{X}', \quad &&\hX \com \hY \hookrightarrow \hX',
\end{aligned}$$
where $\mathcal{V}'$, $\mathcal{V}'_0$, $\mathcal{X}'$ and $\hX'$ denote the dual space of, respectively, 
$\mathcal{V}$, $\mathcal{V}_0$, $\mathcal{X}$ and $\hX$. 

Using the definition of the normalized Maxwellian $M$, see \cref{Def:Max}, we have that $$M(q)\nablaq M(q)^{-1}=-M(q)^{-1} \nablaq M(q) =\nablaq U(\tfrac12 |q|^2) = U'(\tfrac12 |q|^2) q.$$ We introduce the scaled variable $\hphi=\phi/M$ and  with the formula 
$$M \nablaq \hphi = \nablaq \phi + M \nablaq M^{-1} \phi  = \nablaq \phi + U'q \phi$$
we can rewrite the fractional Fokker--Planck equation in \cref{Eq:System} as
$$ \pta \phi + (u \cdot \nablax) \phi + \div_q \big(\omega(u) q \phi\big) = \tfrac{1}{2\lambda} \div_q(M \nablaq \phim) + \eps \Delta_x \phi.$$
As was indicated earlier, we shall confine ourselves here to considering the corotational model, i.e.,
$\omega(v)=-\omega(v)^{\mathrm{T}}$, $q^{\mathrm{T}} \omega(v) q = 0$; if $\div\, v = 0$ it then follows that 
\begin{equation}\label{Eq:SigmaZero} 
\big(M \hphi \,\omega(v)q,\nablaq \hphi\big)_{\mathcal{Y}} = \frac12 \big(M \omega(v)q, \nablaq \hphi^2\big)_{\mathcal{Y}}=-\frac12 \big(\div_q(M \omega(v) q),\hphi^2\big)_{\mathcal{Y}} = 0; \end{equation}
see \cite{barrett2009numerical,barrett2005existence}.
%
We note in passing that partial integration yields the following equalities: 
\begin{equation} \label{Eq:IntParts} \begin{aligned}
		-2\big(M \omega(u) q \hat\varphi,\nablaq \hphi\big)_{\mathcal{Y}} &=  \big(\nablax(M\hat\varphi \nablaq \hphi)q,u\big)_{\mathcal{Y}} + \big(u\cdot q, \div_x (M\hat\varphi\nablaq \hphi)\big)_{\mathcal{Y}}
		\\ &= \big(M \nablax \hat\varphi (\nablaq \hphi)^{\mathrm{T}} q,u\big)_{\mathcal{Y}} + \big(M \hat\varphi \nablax \nablaq \hphi\, q,u\big)_{\mathcal{Y}} \\ &\quad + \big(u \cdot q,M \nablax \hat\varphi \cdot \nablaq \hphi\big)_{\mathcal{Y}} + \big(u \cdot q,M \hat\varphi \,\div_x \nablaq \hphi\big)_{\mathcal{Y}}.
\end{aligned} \end{equation} 
%

We recall that the stress tensor $\tau(\psi) = \tau_1(\psi) + \tau_2(\psi)$ is of the form  given by \eqref{eq:tau1nd} and \eqref{eq:tau2nd}; i.e., 
%%%%%%%%%%%%%
%
\begin{align}\label{Eq:tau1tau2}\tau^1(\psi)=\gamma\, \C(\psi),\quad \tau^2(\psi)= \gamma \int_{D} \psi \d q  \ I_3,\quad  \C(\psi):= \int_{D} F(q) q^{\mathrm T} \psi \d q,
\end{align}
%
where $\gamma>0$ is a dimensionless constant.


%%%%%%%%%%%%
For $\C(M \hpsi)$, we are in a setting that allows us to deduce the following bound; see also \cite[Eq. (3.7)]{barrett2009numerical}:
\begin{equation} \label{Eq:C}  \begin{aligned}
		\int_\Omega |\C(M \hpsi)|^2 \d x & =
		\int_\Omega \left| \, \int_{D} F(q)  q^{\mathrm T} M \hpsi \d q \, \right|^2 \d x
		\\ &\leq \int_D M  | F(q)  q^{\mathrm T} |^2   \dq \, \int_{\Omega \times D} M|\hpsi|^2 \d(x,q)  \\ & \leq C  \| \hpsi\|_{\hY}^2 \quad \forall\, \hpsi \in \hY. \end{aligned}\end{equation}




\section{Existence of weak solutions} \label{Sec:Analysis}
In this section, we prove the existence of a weak solution to the time-fractional Navier--Stokes--Fokker--Planck system. We use a Galerkin procedure and discretize the partial differential equations in space and derive suitable energy estimates. We will emphasize the places where the time-fractional derivative comes into play. We shall then pass to the limit in the sequence of Galerkin approximations to deduce the existence of a weak solution. We shall proceed step by step and prove this result through several lemmas. While, for physical reasons, the derivation of the system was in the previous sections discussed in the case of $d=3$ space dimensions, the analysis below applies to both two and three space dimensions. We begin by introducing the concept of a weak solution to the time-fractional corotational Navier--Stokes--Fokker--Planck system under consideration. %\medskip

\begin{definition} \label{Eq:DefWeak} Suppose that $d \in \{2,3\}$.
We call the pair $(u,\hphi)$ a weak solution to the system \cref{Eq:System}, \cref{eq:neumannbc} provided that   
\begin{align*}
u &\in L^\infty(0,T;\mathcal{H}_0) \cap L^2(0,T;\mathcal{V}_0) \cap W^{1,\tfrac{8}{4+d}}(0,T;\mathcal{V}_0'), \\ 
%L^{
\hphi &\in L^2(0,T;\hX), \quad \pta \hphi \in  L^{\tfrac{8}{4+d}}\big(0,T; \hZ' \big),
\end{align*}
satisfies the initial conditions $u(0)=u^0$, $(\gb*\hphi)(0)=\hpsi^0 := \psi^0/M$ and  the variational problems
\begin{align}
		\langle \pt u,v \rangle_{L^{8/(4-d)}\mathcal{V}}  + \big((u \cdot \nablax)u,v\big)_{L^2\mathcal{H}} + \nu (\nablax u,\nablax v)_{L^2\mathcal{H}} && \label{Eq:NS} \\[-.1cm] + k_B \mu_T \big(\C(M \gb *\hphi),\nablax v\big)_{L^2\mathcal{H}} =0, && \notag  \\[.1cm]
		\langle  \pta \hphi,\hat\zeta\rangle_{L^{8/(4-d)} \hZ}  - (u \hphi,\nablax \hat\zeta)_{L^2\hY}+ \frac{1}{2\lambda} (\nablaq \hphi,\nablaq \hat\zeta)_{L^2\hY}  + \eps (\nablax \hphi,\nablax \hat\zeta)_{L^2\hY} && \label{Eq:FP} \\[-.2cm]   + \frac12 (\nablax(\hphi\nablaq \hzeta)q,u)_{L^2\hY} - \frac12 \big(u\cdot q,\div_x (\hphi\nablaq \hzeta)\big)_{L^2\hY} = 0. &&  \notag
\end{align}
for all $v \in L^{8/(4-d)}(0,T;\mathcal{V}_0)$ and $\hat\zeta \in L^{8/(4-d)}(0,T;\hZ)$. In these variational problems and hereafter, for a Hilbert space $H$ and $p \in [1,\infty]$, subscripts of the form $L^pH$ and $L^p_tH$ appearing in the various inner products, norms, and duality pairings, signify $L^p(0,T;H)$ and $L^p(0,t;H)$, respectively. %Similarly, for a reflexive Banach space $B$, $\langle \cdot, \cdot \rangle_{C B}$ denotes the duality pairing between $C([0,T];B)$ and its dual space, 
%$\mathcal{M}(0,T;B')$ the space of all $B'$-valued finitely additive finite signed measures defined on $(0,T)$, which are absolutely continuous with respect to the Lebesgues measure, equipped with the total variation norm over $(0,T)$.
\end{definition} %\medskip


We summarize the assumptions that we require for proving the existence of a weak solution in the sense of \cref{Eq:DefWeak} in Assumption \ref{Ass:WellPosedness} below. %\medskip

\begin{assumption} \label{Ass:WellPosedness}
	Let the following assumptions hold:
	\begin{itemize}
		\item $D \subset \R^d$,  $d\in\{2,3\}$, is a bounded open ball centered at the origin, $\Omega \subseteq \R^d$ is a Lipschitz domain (i.e., bounded, open, connected set in $\R^d$, with a Lipschitz-continuous boundary $\partial\Omega$), and $T<\infty$ is a fixed final time;
  \item $u^0 \in \mathcal{H}_0$, $\hpsi^0 \in \hX$ with $\hpsi^0 \in H^1_M\big(D; H^{1+d/2+\delta}(\Omega)\big)$ for $\delta>0$ arbitrarily small;
  \item $\tau(\psi) =\tau_1(\psi) + \tau_2(\psi)$ is given by \eqref{Eq:tau1tau2}, with the identity matrix $I_3 \in \mathbb{R}^{3 \times 3}$ replaced by the identity matrix $I_d \in \mathbb{R}^{d \times d}$ in the definition of $\tau_2(\psi)$,  and $\C$ satisfies \cref{Eq:C};
    \item $\alpha \in (1/2,1)$;
  \item $k_B,\mu_T, \nu,\lambda,\eps >0$.
	\end{itemize}
\end{assumption} %\medskip

The main result of this paper is the following theorem asserting the existence of global-in-time large-data weak solutions to the time-fractional Navier--Stokes--Fokker--Planck system under consideration. 

%\medskip

\begin{theorem} \label{Thm:WellPosedness}
	Let \cref{Ass:WellPosedness} hold; 
	then, there exists a weak solution $(u,\hphi)$ to the system \cref{Eq:System}, \cref{eq:neumannbc} in the sense of \cref{Eq:DefWeak}.
\end{theorem}%\medskip

In order to prove this theorem, we state several lemmas, which will eventually imply \cref{Thm:WellPosedness}. We begin by constructing a sequence of Galerkin approximations $\{(u_k,\hphi_k)\}_{k=1}^\infty$ to the system of partial differential equations under consideration, resulting in a system of fractional-order ordinary differential equations, which admits a local-in-time solution $(u_k,\hphi_k)$
for each $k \geq 1$ thanks to  standard theory. We then derive an energy estimate for the sequence of Galerkin approximations, which is uniform with respect to $k$; this then implies that, for each $k \geq 1$, the local-in-time solution of the fractional-order system of ordinary differential equations can be extended to the entire time-interval $[0,T]$; it also implies the existence of a weakly/weakly-$*$ convergent subsequence $(u_{k_j},\hphi_{k_j})$. Finally, we pass to the limit $j \to \infty$ and apply a compactness argument to deduce that the limiting pair of functions 
$(u,\hphi)$ is in fact a weak solution to the system of partial differential equations in the sense of \cref{Eq:DefWeak}. The Galerkin method has been applied to various time-fractional PDEs; see, e.g., \cite{fritz2020time,fritz2021sub,fritz2021equivalence,vergara2015optimal}; it has also been applied to Navier--Stokes--Fokker--Planck systems in \cite{knezevic2009heterogeneous,barrett2009numerical,barrett2011finite,barrett2012finite}, with an integer-order Fokker--Planck equation. 

\subsection{Galerkin discretization}
We follow the construction of \cite[Section 2.1]{bulicek2013existence} and conclude by the Hilbert--Schmidt theorem \cite[Lemma A.4]{bulicek2013existence} the existence of a countable set $\{h_j\}_{j=1}^\infty$ of eigenfunctions in $\mathcal{V}_0 \cap H^{1+\frac{d}{2}+\delta}(\Omega)^d$, with $\delta>0$ arbitrarily small, whose linear span is dense in $\mathcal{H}_0$ such that the $h_j$, $j\in \{1,2,\dots\}$, are orthonormal in $\mathcal{H}$ and orthogonal in $H^{1+\frac{d}{2}+\delta}(\Omega)^d$ in the sense that $(h_j,h_i)_{H^{1+\frac{d}{2}+\delta}(\Omega)}=\lambda_j \delta_{i,j}$ for any $i,j \in \{1,2,\ldots\}$ and $\lambda_j>0$ for all $j=1,2,\ldots$. Similarly, we fix a countable set $\{y_j\}_{j=1}^\infty$ in $H^2_M(\Omega \times D)$ that forms an orthogonal system in $\hX$ and an orthonormal system in $\hY$. 
%Consider the high-order elliptic problem of finding a solution tuple $(u,\lambda) \in H^{1+d}(\Omega)^d \times \R$ of
%$$(u,v)_{H^{1+d}(\Omega)} + (\nabla u,\nabla v)_{L^2(\Omega)}  = \lambda (u,v)_{\mathcal{H}} \qquad \forall  v \in W^{1+d}(\Omega)^d.$$
%By a version of the Hilbert--Schmidt theorem, see \cite[Lemma A.4]{bulicek2013existence}, there is a countable set of eigenfunctions $\{h_j\}_{j=1}^\infty$, which are orthogonal in the inner product of $H^{1+d}(\Omega)^d$ and orthonormal in the inner product of $L^2(\Omega)^d$. We note that $H^{1+d}(\Omega)^d$ is continuously embedded into $W^{1,\infty}(\Omega)$
We then define the $k$-dimensional linear spaces
\begin{align*}
	\mathcal{H}_k  :=\text{span}\{ h_1,\dots,h_k\}, \quad 
	\widehat{\mathcal{Y}}_k  :=\text{span}\{ y_1,\dots,y_k\},
	%	Z_K &=\text{span}\{ z_1,\dots,z_k\},
\end{align*}
%where $h_j: \Omega \to \R$ and $y_j : \Lambda \to \R$, $j \in \{1,\dots,k\}$, are the eigenfunctions corresponding to the eigenvalues $\lambda_{j}, \mu_{j} \in \R$ of the following respective problems
%$$\begin{aligned}
	%(h_j,v)_{H^{1+d}(\Omega)} + (\nabla h_j,\nabla v)_{L^2(\Omega)}  &= \lambda_{j} (h_j,v)_{\mathcal{H}} &&\forall  v \in W^{1+d}(\Omega;\R^d),  \\
	%(\nabla y_j,\nabla y)_\hY &= \mu_{j} (y_j,y)_\hY &&\forall  y \in \hY. 
%\end{aligned}$$
%\begin{alignat*}{1}
%	 & \begin{cases} \begin{aligned}
		%			-\Delta h_j          & = \lambda_{h,j} h_j &  & \text{in } \Omega,         \\
		%			\nabla h_j \cdot n & = 0               &  & \text{on } \partial\Omega,
		%		\end{aligned} \end{cases} 
%			  \begin{cases} \begin{aligned}
		%		-\Delta h_j^0          & = \lambda_{h^0,j} h_j^0 &  & \text{in } \Omega,         \\
		%		\nabla h_j^0 \cdot n & = 0             &  & \text{on } \partial\Omega \backslash \p \Omega_D, \\
		%				h_j^0 & = 0               &  & \text{on } \partial\Omega_D, 
		%		\end{aligned} \end{cases} 
%	  \begin{cases} \begin{aligned}
		%			-\Deltala y_j                  & = \lambda_{y,j} y_j &  & \text{in } \Lambda,         \\
		%			\nablala y_j \cdot n_\Lambda & = 0               &  & \text{on } \partial\Lambda_N, \\
		%			y_j & = 0               &  & \text{on } \partial\Lambda_D.
		%		\end{aligned} \end{cases}
%\end{alignat*}
%As stated in \cite{bulicek2013existence}, there is a countable 
%Since both the inverse Stokes and the inverse Neumann--Laplace operator are compact, self-adjoint, injective, positive operators on $\mathcal{H}_0$ and $\hY$, respectively, we conclude by the spectral theorem, see e.g. \cite[12.12 and 12.13]{alt2016linear}, that
%\begin{alignat*}{3}
	%& \{h_j\}_{j \in \mathbb{N}} &  & \text{ is an orthonormal basis in } \mathcal{H}_0 &  & \text{ and orthogonal in } \mathcal{V}_0, \\
	%	& \{y_j\}_{j \in \mathbb{N}} &  & \text{ is an orthonormal basis in } \hY &  & \text{ and orthogonal in } \hX.
%\end{alignat*}
%By the orthonormality of the eigenfunctions, we conclude that ${\cup_{k\in\mathbb{N}}} \mathcal{H}_k$ and ${\cup_{k\in\mathbb{N}}} \hY_k$ are dense in $\mathcal{V}_0$ in $\hX$, respectively. 
and we consider the Galerkin approximations
\begin{equation}\begin{gathered}
		u_k (t) = \sum_{j=1}^k u_k^j(t) h_j,
		\quad \hphi_k (t) = \sum_{j=1}^k \hphi^j_k(t) y_j,
	\end{gathered}
	\label{Eq:GalerkinAnsatzFunctions}
\end{equation}
where 
$u^{j}_k$ and $\hphi^j_k$ are real-valued coefficient functions for all $j \in \{1,\dots,k\}$.
 %Let $h>0$ denote a discretization parameter tending to zero. As in \cite{barrett2009numerical}, we choose finite-dimensional spaces $\hY_k^x \subset W^{1,\infty}(\Omega)$ and $\hY_k^q \subset W^{1,\infty}(D)$ such that it holds
%$$\text{dist}_{W^{1,\infty}(\Omega)} (\eta,\hY_k^x) \to 0, \qquad \text{dist}_{W^{1,\infty}(D)} (\xi,\hY_k^q) \to 0,$$
%as $h\to 0$ for all $\eta \in C^\infty(\overline\Omega)$ and $\xi \in C^\infty(\overline D)$. Moreover, we define the tensor space $\hY_k=\hY_k^x \otimes \hY_k^q \subset W^{1,\infty}(\Omega \times D)$ and note that $\widehat{X}_h \subset X \subset \hX$. 
%Further, we define the finite-dimensional spaces $W_h$, $R_h$ and $\mathcal{H}_k$ such that
%$$\begin{aligned}W_h &\subset \mathcal{H}_0^1(\Omega;\R^d) \cap W^{1,\infty}(\Omega;\R^d), \quad R_h \subset L_0^2(\Omega), \\ \mathcal{H}_k&=\{w_h\in W_h: (\div_x w_h,r_h)_{\mathcal{H}} \,\forall r_h \in R_h \},\end{aligned}$$
%where $\cup_{h>0} W_h$ and $\cup_{h>0} R_h$ are supposed to be dense in $\mathcal{H}_0^1(\Omega;\R^d)$ and $L_0^2(\Omega;\R^d)$, respectively. Further, we assume that for all $v \in V$ there exists a sequence $v_k \in \mathcal{H}_k$ such that $v_k \to v$ in $H^1(\Omega)$ for $h \to 0$. This holds for the typical Galerkin approximation with the availability of an uniform inf-sup condition.
The canonical orthogonal projection onto the finite-dimensional space $\mathcal{H}_k$ is defined by $\Pi_{\mathcal{H}_k}: \mathcal{H} \to \mathcal{H}_k$, $h \mapsto \sum_{j=1}^k (h,h_j)_{\mathcal{H}} h_j$,  and in the same way for $\Pi_{\hY_k}:\hY \to \hY_k$. 
For $h=\sum_{j=1}^\infty (h,h_j)_{\mathcal{H}}h_j$ we have that $$\|h\|^2_{H^{1+\frac{d}{2}+\delta}(\Omega)} = \sum_{j=1}^\infty \lambda_j |(h,h_j)_{\mathcal{H}}|^2, \quad \|\Pi_{\mathcal{H}_k} h\|^2_{H^{1+\frac{d}{2}+\delta}(\Omega)} = \sum_{j=1}^k \lambda_j |(h,h_j)_{\mathcal{H}}|^2,$$
from which we conclude via the Sobolev embedding theorem that, for each $k \geq 1$, 
$$\|\Pi_{\mathcal{H}_k} h\|_{W^{1,\infty}(\Omega)} \leq C\|\Pi_{\mathcal{H}_k} h\|_{H^{1+\frac{d}{2}+\delta}(\Omega)} \leq C\|h\|_{H^{1+\frac{d}{2}+\delta}(\Omega)}.$$


%Thanks to \cite[Theorem 8.1.11]{brenner2008mathematical} and \cite{guzman2009holder}, we have that $\Pi_{\mathcal{H}_k}$ is uniformly $H^1$-stable and $\Pi_{\hY_k}$ is $W^{1,\infty}$-stable.
%Given the initial data $u^0$ and $\psi^0$ from the continuous system, we choose $u_k^0 \in \mathcal{H}_k$ and $\hphi_k^0 \in \hY_k$ such that $u_k^0 = \Pi_{\mathcal{H}_k} u^0$ and $\hpsi_k^0=\Pi_{\hY_k} \hpsi^0$. 

The Galerkin equations read as follows: we wish to find a tuple $(u_k,\hphi_k) \in \mathcal{H}_k \times \hY_k$ for each $k \geq 1$ such that $u_k(0)=u_k^0:= \Pi_{\mathcal{H}_k}u^0$, $(\gb*\hphi_k)(0)=\hpsi_k^0:=\Pi_{\hY_k} \hpsi^0$, and
\begin{align}(\pt u_k,v_k )_{\mathcal{H}}  &+ ((u_k \cdot \nablax)u_k,v_k)_{\mathcal{H}} + \nu (\nablax u_k,\nablax v_k)_{\mathcal{H}}  \label{Eq:NS_dis} \\ 
& + \,k_B \mu_T (\C(M \gb *\hphi_k),\nablax v_k)_{\mathcal{H}} =0, \notag \\
	(\pt (\gb*\hphi_k),\hat\zeta_k)_\hY  &+  
	((u_k \cdot \nablax) \hphi_k, \hzeta_k)_\hY + \frac{1}{2\lambda} (\nablaq \hphi_k,\nablaq \hat\zeta_k)_\hY  \label{Eq:FP_dis} \\ &+ \eps (\nablax \hphi_k,\nabla_x \hat\zeta_k)_\hY- (\omega(u_k) q \hphi_k, \nablaq \hat\zeta_k)_\hY=0, \notag \end{align} for all $v_k \in \mathcal{H}_k$ and $\hat\zeta_k \in \hY_k$.  %\medskip

\begin{lemma} \label{Eq:LemmaExistenceODE}
	Suppose that \cref{Ass:WellPosedness} holds; then, for each $k \geq 1$,
	there exists a local-in-time solution $(u_k,\hphi_k)$ to the Galerkin system \cref{Eq:NS_dis}, \cref{Eq:FP_dis}.
	\end{lemma} %\medskip

\begin{proof}
Let $U_k(t):=(u_k^1(t),\ldots,u_k^k(t))^{\mathrm{T}}$ and $\widehat\Phi_k(t)=(\widehat{\phi}_1^k,\ldots, \widehat{\phi}_k^k)^{\rm T}$. With this notation the Galerkin subsystem \eqref{Eq:NS_dis} becomes an initial-value problem for a system of ordinary differential equations of the form $\ddt U_k  = F(t, U_k, \widehat\Phi_k)$, while, by noting that $\ddt (\gb*\hphi_k)=\pta \hphi_k$, the Galerkin subsystem \eqref{Eq:FP_dis} takes the form of an initial-value problem for a system of fractional-order ordinary differential equations $(\ddt)^\alpha \widehat\Phi_k  = G(t, U_k, \widehat\Phi_k)$. As the functions $F$ and $G$ are continuous with respect to their arguments and locally Lipschitz continuous with respect to their second and third arguments, we can appeal to the generalization of the Cauchy--Lipschitz theorem stated in Theorem 5.1 of \cite{diethelm2010analysis} to deduce the existence of a unique continuous solution, defined on a time interval $[0, T_k]$ where $0<T_k \leq T$, where $U_k$ is, in fact, a continuously differentiable function of $t$ by the classical Cauchy--Lipschitz theorem. 
\end{proof}

\subsubsection{Energy estimates} Next, we derive a $k$-uniform energy estimate, which will allow us to extend, for each $k \geq 1$, the corresponding local-in-time Galerkin solution, whose existence is guaranteed by Lemma \ref{Eq:LemmaExistenceODE}, to the entire time interval $[0,T]$; it will also enable us to extract weakly converging subsequences of Galerkin approximations.  We begin by deriving a bound on the solution to the Galerkin approximation of the Navier--Stokes equation; we shall then derive a bound on the solution to the Galerkin approximation of the Fokker--Planck equation. At the end, we will add the two bounds and apply Gronwall's lemma to obtain a $k$-uniform energy estimate. %\medskip

\begin{lemma} \label{Lem:EstU} Let \cref{Ass:WellPosedness} hold; then the following bound on the Galerkin solution $u_k$, in terms of $u_k^0$ and $\hphi_k$, holds for all $t \in (0,T_k)$:
	\begin{equation} \label{Eq:EnergyU2}
		\frac12 \|u_k(t)\|_{\mathcal{H}}^2  + \frac{\nu}{2} \|\nablax u_k\|_{L^2_t{\mathcal{H}}}^2  \leq \frac12 \|u^0_h\|_{\mathcal{H}}^2 +\frac{ T^{2-2\alpha} k_B^2\mu_T^2}{2\nu (1-\alpha)^2}  \|\hphi_k\|_{L^2_t\hY}^2.
	\end{equation}
\end{lemma} %\medskip

\begin{proof}
We take the test function $v_k=u_k(t)$ in the equation \cref{Eq:NS}, which gives
$$\begin{aligned}
	\frac12 \frac{\dd}{\dd t} \|u_k\|_{\mathcal{H}}^2  + \nu \|\nablax u_k\|_{\mathcal{H}}^2  &= -k_B \mu_T \big(\C(M\gb*\hphi),\nablax u_k\big)_{\mathcal{H}}, 
	\end{aligned}$$
and we can further bound the right-hand side from above to deduce that
$$\frac12 \frac{\dd}{\dd t} \|u_k\|_{\mathcal{H}}^2  + \nu \|\nablax u_k\|_{\mathcal{H}}^2  \leq \frac{\nu}{2} \|\nablax u_k\|_{\mathcal{H}}^2 + \frac{k_B^2\mu_T^2}{2\nu} \|\C(M\gb*\hphi_k)\|_{\mathcal{H}}^2. $$
By bounding the term  $\C(M\gb * \hphi_k)$ as in \cref{Eq:C} we arrive at the inequality
\begin{equation} \begin{aligned}\frac12 \frac{\dd}{\dd t} \|u_k\|_{\mathcal{H}}^2  + \frac{\nu}{2} \|\nablax u_k\|_{\mathcal{H}}^2  &\leq  \frac{k_B^2\mu_T^2}{2\nu} \|\gb*\hphi_k\|_\hY^2. %
\end{aligned} \label{Eq:EnergyU}
\end{equation}
Next, we note that, for any $t \in (0,T_k)$,
%
\begin{align}\label{Eq:EnergyU1} 
\int_0^t \|(\gb*\hphi_k)(s)\|_\hY^2 \ds \leq \int_0^t  \big(\gb * \|\hphi_k\|_\hY\big)^2(s) \ds\leq \|\gb\|_{L^1_t}^2 \|\hphi_k\|_{L^2_t\hY}^2.
\end{align}
This follows by observing that $g_{1-\alpha}$ only depends on the scalar variable $s$, which permits pulling the $\hY$-norm inside of the convolution, followed by applying Young's convolution inequality (cf. Lemma 3.2 in \cite{Oparnica}) in the resulting integrand. 

We then integrate \eqref{Eq:EnergyU} over the interval $[0,t]$ where $t \in (0,T_k)$ and use \eqref{Eq:EnergyU1} to bound the right-hand side of the resulting inequality. Finally we
note that $g_{1-\alpha}$ is integrable on $(0,t)$ and its integral is bounded by $T^{1-\alpha}/(1-\alpha)$. 
This gives \cref{Eq:EnergyU2}.
\end{proof} %%\medskip

Having derived a bound on $u_k$, we move on to the derivation of a bound on $\hphi^k$ by testing the Galerkin system \eqref{Eq:FP_dis}.  %\medskip

\begin{lemma} \label{Lem:EstPhi} Let \cref{Ass:WellPosedness} hold and let $\gamma>0$ be arbitrary but fixed; then, the following bound on the sequence of Galerkin solutions $\{(u_k,\hphi_k)\}_{k=1}^\infty$ holds:
\begin{equation} \label{Est:SolFP} \begin{aligned} & \frac{\gamma}{2} \big(\gb* \|\hphi_k-\init \ga\|_\hY^2\big)(t) + \frac{\gamma \,T^{-\alpha}}{16\, \Gamma(1-\alpha)} \|\hphi_k\|_{L^2_t\hY}^2  +\frac{\gamma}{2\lambda} \|\nablaq \hphi_k\|_{L^2_t\hY}^2 + \gamma \eps \| \nablax \hphi_k\|_{L^2_t \hY}^2  \\
	&\quad \leq C(\alpha,\gamma) \|M^{1/2} \init\|_{H^1(D;W^{1,\infty}(\Omega)) }^2  \int_0^t g_{2\alpha-1}(s) \|u_k(s)\|_{\mathcal{H}}^2 \ds
	+ C(\alpha,\gamma,T)\| \init\|_\hX^2.
\end{aligned}\end{equation}
\end{lemma} %\medskip

\begin{proof}
We note again that,  thanks to the inverse convolution property, see \cref{Eq:InverseConvolution}, $\hphi_k - \init \ga = \ga * \pta \hphi_k$.
We take this function as the test function in the variational Fokker--Planck equation \cref{Eq:FP}, i.e., $\hat\zeta= \hphi_k - \init \ga = \ga * \pta \hphi_k$, which gives  
\begin{equation} \label{Eq:TestingRHS}\begin{aligned} &(\pta \hphi_k,\ga * \pta \hphi_k)_\hY +\frac{1}{2\lambda}  \|\nablaq \hphi_k\|_\hY^2 + \eps \| \nablax \hphi_k\|_\hY^2\\ %
&\quad =\ga(t) \cdot \Big( \big((u_k \cdot \nablax) \hphi_k, \init\big)_\hY + \frac{1}{2\lambda} \big(\nablaq \hphi_k,\nablaq \init\big)_\hY  \\[0cm] &\qquad + \eps \big(\nablax \hphi_k,\nabla \init\big)_\hY- \big(\omega(u_k) q \hphi_k, \nablaq \init\big)_\hY \Big) =:R.
\end{aligned}\end{equation}
We then use the fractional chain inequality \cref{Eq:ChainOriginal} to bound the left-hand side of \cref{Eq:TestingRHS} from below, which yields
$$
\frac12 \pta \|\hphi_k-\ga \init\|^2_\hY  \leq (\pta \hphi_k,\hphi_k-\ga \init)_\hY = (\pta \hphi_k,\ga * \pta \hphi_k)_\hY. 
$$
Regarding the right-hand side of \cref{Eq:TestingRHS}, we integrate the last term containing $\omega(u_k)$ by parts, see
\cref{Eq:IntParts}, and get
$$\begin{aligned} -  \big(M\omega(u_k)  q \hphi, \nablaq \init\big)_{\mathcal{Y}} &=  \big(M \nablax \hphi_k (\nablaq \init)^{\mathrm{T}} q,u_k\big)_{\mathcal{Y}} + \big(M \hphi_k \nablax \nablaq \init q,u_k\big)_{\mathcal{Y}} \\ &\quad + \big(u_k \cdot q,M \nablax \hphi_k \cdot \nablaq \init\big)_{\mathcal{Y}} + \big(u_k \cdot q,M \hphi_k \div_x \nablaq \init\big)_{\mathcal{Y}}. 
\end{aligned} $$
We apply H\"older's inequality to obtain the following bound on the right-hand side, $R$, of the equality \cref{Eq:TestingRHS}:
$$\begin{aligned} R \leq{}& \ga(t) \cdot\! \Big(  \|u_k\|_{\mathcal{H}}  \|\nablax\hphi_k\|_\hY \|M^{1/2}\init\|_{L^2(D;L^\infty(\Omega)) }\\
%
%[-.2cm]  &+\|u_k\|_{\mathcal{H}}  \|\hphi_k\|_\hY \|M^{1/2}
%\init\|_{L^2(D;W^{1,\infty}(\Omega)) }  \\[0cm]  
%
&+ \frac{1}{2\lambda} \|\nablaq \hphi_k\|_\hY  \|\nablaq \init\|_\hY  + \eps \|\nablax \hphi_k\|_\hY \|\nablax \init\|_\hY   \\
%[-.1cm]  
&+ C\|u_k\|_{\mathcal{H}} \|q\|_{L^\infty(D)}  \|M^{1/2}\nablaq \init\|_{L^2(D;W^{1,\infty}(\Omega)) } \big(\|\hphi_k\|_\hY + \|\nablax \hphi_k\|_\hY\big) \Big).
\end{aligned}$$
Hence, thanks to Young's inequality, we arrive at the following bound on $R$: 
$$\begin{aligned} R \leq&~{} \frac{1}{\eps} \ga(t)^2 \|M^{1/2} \init\|_{L^2(D;W^{1,\infty}(\Omega)) }^2  \|u_k\|_{\mathcal{H}}^2  + \frac{\eps}{4} \|\nablax \hphi_k\|_\hY^2  \\ &{} + \frac{1}{4\lambda} \|\nablaq \hphi_k\|_\hY^2   + \frac{\ga(t)^2}{4\lambda} \|\nablaq \init\|_\hY^2  +\frac{\eps}{4} \|\nablax \hphi_k\|_\hY^2 + \eps \ga(t)^2 \|\nablax \init\|_\hY^2 \\
&{}+\delta  \|\hphi_k\|_\hY + \frac{\eps}{4} \|\nablax \hphi_k\|_\hY^2 + C(\eps,\delta) \ga(t)^2 \|M^{1/2}\nablaq \init\|_{L^2(D;W^{1,\infty}(\Omega)) }^2 \|u_k\|_{\mathcal{H}}^2,
\end{aligned}$$
where $\delta>0$ is sufficiently small, to be chosen appropriately later on. After combining the lower bound on the left-hand side of \cref{Eq:TestingRHS} and the upper bound on the right-hand side we have that
$$\begin{aligned} &\frac12 \pta \|\hphi_k-\ga \init\|^2_\hY +\frac{1}{2\lambda}  \|\nablaq \hphi_k\|_\hY^2 + \eps \| \nablax \hphi_k\|_\hY^2\\ &\quad \leq \frac{1}{\eps}\ga(t)^2 \|M^{1/2} \init\|_{L^2(D;W^{1,\infty}(\Omega)) }^2  \|u_k\|_{\mathcal{H}}^2  + \frac{\eps}{4} \|\nablax \hphi_k\|_\hY^2  \\ &{} \qquad + \frac{1}{4\lambda} \|\nablaq \hphi_k\|_\hY^2   + \frac{\ga(t)^2}{4\lambda} \|\nablaq \init\|_\hY^2  +\frac{\eps}{4} \|\nablax \hphi_k\|_\hY^2 + \eps \ga(t)^2 \|\nablax \init\|_\hY^2 \\
&{}\qquad +\delta  \|\hphi_k\|_\hY^2 + \frac{\eps}{4} \|\nablax \hphi_k\|_\hY^2 + C(\eps,\delta) \ga(t)^2 \|M^{1/2}\nablaq \init\|_{L^2(D;W^{1,\infty}(\Omega)) }^2 \|u_k\|_{\mathcal{H}}^2,\end{aligned}$$
and absorbing terms on the right-hand side into the left-hand side gives
$$\begin{aligned} &\frac12 \pta \|\hphi_k-\ga \init\|^2_\hY +\frac{1}{4\lambda}  \|\nablaq \hphi_k\|_\hY^2 + \frac{\eps}{2} \| \nablax \hphi_k\|_\hY^2\\ &\quad \leq C(\eps,\delta) \ga(t)^2  \|M^{1/2} \init\|_{H^1(D;W^{1,\infty}(\Omega))}   \|u_k\|_{\mathcal{H}}^2  + \delta \|\hphi_k\|_\hY^2 + C(\eps,\delta)\ga(t)^2 \|\init\|_\hX^2.
\end{aligned}$$

We note that $\ga^2=\frac{\Gamma(2\alpha-1)}{\Gamma(\alpha)^2}g_{2\alpha-1}$ is integrable for $\alpha\in (\tfrac12,1)$ and $g_1*g_{2\alpha-1}=g_{2\alpha}$, which is continuous, bounded, and monotonically increasing on $[0,T]$ for $\alpha\in(\tfrac12,1)$.  We integrate the inequality over $(0,t)$ and exploit the representation $\pta v = \pt (g_{1-\alpha} * v)$ of the Riemann--Liouville derivative, which gives
\begin{equation} \label{Eq:EnergyFP}\begin{aligned} & \frac12 \big(\gb* \|\hphi_k-\init \ga\|_\hY^2\big)(t)  +\frac{1}{4\lambda} \|\nablaq \hphi_k\|_{L^2_t\hY}^2 + \frac{\eps}{2} \| \nablax \hphi_k\|_{L^2_t \hY}^2 - \delta \|\hphi_k\|^2_{L^2_t\hY}  \\
&\quad \leq C(\delta,\alpha) \|M^{1/2} \init\|_{H^1(D;W^{1,\infty}(\Omega)) }^2  \int_0^t g_{2\alpha-1}(s) \|u_k(s)\|_{\mathcal{H}}^2 \ds
+ C(\delta)\| \init\|_\hX^2  g_{2\alpha}(T).
\end{aligned}\end{equation}
Further, we derive a lower bound on the first term of the left-hand side by noting that $(g_1*v)(t) \leq T^{\alpha} \Gamma(1-\alpha) (\gb*v)(t)$, see \cref{Eq:KernelNorm}, and therefore we have that
$$\begin{aligned} &\frac12 \big(\gb* \|\hphi_k-\init \ga\|_\hY^2\big)(t) \\ &\quad \geq \frac{T^{-\alpha}}{2\Gamma(1-\alpha)} \int_0^t  \|\hphi_k(s)-\init \ga(s)\|_{\hY}^2 \ds \\ &\quad \geq \frac{T^{-\alpha}}{2\Gamma(1-\alpha)} \int_0^t \big|\,\|\hphi_k(s)\|_\hY-\ga(s)\|\init \|_{\hY} \big|^2\ds \\
&\quad =\frac{T^{-\alpha}}{2\Gamma(1-\alpha)} \int_0^t \|\hphi_k(s)\|_\hY^2 - 2\ga(s)\|\hphi_k(s)\|_\hY\|\init \|_{\hY} +g_{\alpha}^2(s)\|\init \|_{\hY}^2\ds,
\end{aligned} $$
where we applied the reverse triangle inequality in the second estimate.
The function $g_{\alpha}$ belongs to $L^2(0,t)$ for any $\alpha\in (\tfrac12,1)$ and the integral of $g_\alpha^2$ is positive. We apply H\"older's inequality in the second term of the integrand and note that the $L^2(0,t)$-norm of $g_\alpha$ has the upper bound $C(\alpha)T^{\alpha-1/2}$. We thus have that
$$\begin{aligned}   \frac12 \big(\gb* \|\hphi_k-\init \ga\|_\hY^2\big)(t)   &\geq 
\frac{T^{-\alpha}}{2\Gamma(1-\alpha)} \left[
\frac12 \|\hphi_k\|_{L^2_t\hY}^2 - 2 C(\alpha) T^{\alpha-1/2} \|\init \|_{\hY}  \|\hphi_k\|_{L^2_t\hY} \right]\\ &\geq \frac{T^{-\alpha}}{2\Gamma(1-\alpha)}\left[\frac14 \|\hphi_k\|_{L^2_t\hY}^2 - C(T,\alpha) \|\init \|_{\hY}^2\right],
\end{aligned} $$
where we have applied Young's inequality in the last step. We multiply the energy estimate \cref{Eq:EnergyFP} by $\gamma>0$ and obtain for $\delta=\frac{T^{-\alpha}}{16 \Gamma(1-\alpha)}$ the estimate \cref{Est:SolFP}.
\end{proof} %\medskip

Next, we combine the estimates on $u_k$ and $\hphi_k$, see \cref{Lem:EstU} and \cref{Lem:EstPhi}, to obtain a $k$-uniform bound. %\medskip

\begin{lemma} \label{Lem:EstComb}
	Let \cref{Ass:WellPosedness} hold; then, the following $k$-uniform estimate on the Galerkin solution $(u_k,\hphi_k)$ holds:
	\begin{equation} \label{Eq:EnergyIneq}
		\begin{aligned} &  \big(\gb* \|\hphi_k-\init \ga\|_\hY^2\big)(t) +  \|\hphi_k\|_{L^2_t\hX}^2    + \|u_k(t)\|_{\mathcal{H}}^2 +  \|\nabla u_k\|_{L^2_t{\mathcal{H}}}^2 \\
			&\quad \leq C\big(\alpha,T, \|u^0\|_{\mathcal{H}}^2,\|M^{1/2} \hpsi^0\|^2_{H^1(D;H^{1+d/2+\delta}(\Omega)) }\big).
	\end{aligned}\end{equation}
\end{lemma} 

\begin{proof}
We add the integrated velocity inequality \cref{Eq:EnergyU2} to the bound \cref{Est:SolFP} and obtain the following combined bound:
$$\begin{aligned} &\frac{\gamma}{2} \big(\gb* \|\hphi_k-\init \ga\|_\hY^2\big)(t) + \frac{\gamma\, T^{-\alpha}}{16\,\Gamma(1-\alpha)} \|\hphi_k\|_{L^2_t\hY}^2  +\frac{\gamma}{2\lambda} \|\nablaq \hphi_k\|_{L^2_t\hY}^2 + \gamma \eps \| \nablax \hphi_k\|_{L^2_t \hY}^2  \\ &\quad + \frac12 \|u_k(t)\|_{\mathcal{H}}^2 + \frac{\nu}{2} \|\nabla u_k\|_{L^2_t{\mathcal{H}}}^2 \\
&\leq C(\alpha,\gamma) \|M^{1/2} \init\|_{H^1(D;W^{1,\infty}(\Omega)) }^2  \int_0^t g_{2\alpha-1}(s) \|u_k(s)\|_{\mathcal{H}}^2 \ds
+ C(\alpha,\gamma,T)\| \init\|_\hX^2 \\&\quad + \frac12 \|u_k^0\|_{\mathcal{H}}^2 + %
\frac{ T^{2-2\alpha} k_B^2\mu_T^2}{2\nu (1-\alpha)^2}  \|\hphi_k\|_{L^2_t\hY}^2.
\end{aligned}$$
We now choose $\gamma$ such that 
$$\frac{\gamma\, T^{-\alpha}}{16\, \Gamma(1-\alpha)} \geq \frac{ T^{2-2\alpha} k_B^2\mu_T^2}{\nu (1-\alpha)^2}.$$
Hence, we can absorb the last term on the right-hand side into the second term on the left-hand side, and 
 the combined energy inequality thus becomes
\begin{equation} \label{Eq:EnergyBack}\begin{aligned} & \big(\gb* \|\hphi_k-\init \ga\|_\hY^2\big)(t) +  \|\hphi_k\|_{L^2_t\hX}^2    + \|u_k(t)\|_{\mathcal{H}}^2 +  \|\nabla u_k\|_{L^2_t{\mathcal{H}}}^2 \\
&\quad\leq C(\alpha,T) \|M^{1/2} \init\|_{H^1(D;W^{1,\infty}(\Omega)) }^2  \int_0^t g_{2\alpha-1}(s) \|u_k(s)\|_{\mathcal{H}}^2 \ds
\\ &\qquad + C(\alpha,T) \big( \| \init\|_\hX^2 + \|u_k^0\|_{\mathcal{H}}^2\big),
\end{aligned}\end{equation}
where we took the minimum of each prefactor of the norms on the left-hand side and divided the inequality by this value. Gronwall's lemma then implies that
\begin{equation} 
\begin{aligned} \label{Eq:EnergyBack1}
& \big(\gb* \|\hphi_k-\init \ga\|_\hY^2\big)(t) +  \|\hphi_k\|_{L^2_t\hX}^2    + \|u_k(t)\|_{\mathcal{H}}^2 +  \|\nabla u_k\|_{L^2_t{\mathcal{H}}}^2 \\
&\quad \leq C(\alpha,T) \cdot \big(  \| \init\|_\hX^2  + \|u_k^0\|_{\mathcal{H}}^2  \big) \cdot \textup{exp}\bigg(\frac{T^{2\alpha-1}}{2\alpha-1} \|M^{1/2} \init\|^2_{H^1(D;W^{1,\infty}(\Omega)) }\bigg).
\end{aligned}
\end{equation}


We note that the initial conditions of the Galerkin system are defined by $u_k^0=\Pi_{\mathcal{H}_k} u^0$ and $\init=\Pi_{\hY_k} \hpsi^0$. Therefore, we have that $\|u_k^0\|_{\mathcal{H}}^2 \leq \|u^0\|_{\mathcal{H}}^2$ and $$\|M^{1/2}\init\|_{H^1(D;W^{1,\infty}(\Omega)) }^2 \leq C \|M^{1/2}\hpsi^0\|_{H^1(D;H^{1+d/2+\delta}(\Omega)) }^2.$$
%see \cite[Theorem 8.1.11]{brenner2008mathematical} with regards to the stability of the Ritz projection in $W^{1,\infty}$.
We insert these bounds into the right-hand side of the inequality \cref{Eq:EnergyBack1} and we thus arrive at the desired $k$-uniform energy estimate \cref{Eq:EnergyIneq}.
\end{proof}





\subsection{Convergence of subsequences} 
Having derived the $k$-uniform energy estimate \cref{Eq:EnergyIneq} stated in \cref{Lem:EstComb},
we shall extract weakly/weakly-$*$ converging subsequences of Galerkin solutions $(u_k,\hphi_k)$. We shall also prove strong convergence of a subsequence $u_{k_j}$ in $L^2(0,T;\mathcal{H}_0)$ in order to pass to the limit $j\to \infty$ in the nonlinear terms in the variational Navier--Stokes--Fokker--Planck system.   %\medskip

\begin{lemma} Let \cref{Ass:WellPosedness} hold and assume that $r \in [1,\infty)$ for $d=2$ and $r \in [1,6)$ for $d=3$; then, the sequence of Galerkin solutions $(u_k,\hphi_k)$ from \cref{Eq:LemmaExistenceODE} contains a subsequence $(u_{k_j},\hphi_{k_j})$ that admits the following convergences as $j \to \infty$:
	%\footnote{\color{red}~I don't see where the 7th weak convergence, with spatial function space $\hY$, is coming from. Surely this is incorrect. -- This should be corrected now} 
	\begin{equation} \label{Eq:Weak} \begin{aligned}	
			u_{k_j} &\longweak u &&\text{weakly-$*$ in } L^\infty(0,T;\mathcal{H}_0), \\
			u_{k_j} &\longweak u &&\text{weakly\phantom{-*} in } L^2(0,T;\mathcal{V}_0) \cap L^{8/d}(0,T;L^4(\Omega)^d), \\
			\hphi_{k_j} &\longweak 		 \hphi &&\text{weakly\phantom{-*} in }
			L^2(0,T;\hX), \\
		\pt u_{k_j} &\longweak \pt u &&\text{weakly\phantom{-*} in } L^{8/(4+d)}(0,T;\mathcal{V}_0'), \\
		u_{k_j} &\longrightarrow  u &&\text{strongly\hspace{1mm} in } L^2\big(0,T;L^{r}(\Omega;\R^d)\big), \\
		u_{k_j} &\longrightarrow  u &&\text{strongly\hspace{1mm} in } C([0,T];\mathcal{V}_0'),\\
		\pt^{\alpha} \hphi_{k_j} &\longweak \pt^{\alpha} \hphi &&\text{weakly\phantom{-*} in } L^{8/(4+d)}(0,T;\hZ'),\\
		\hphi_{k_j} &\longrightarrow  \hphi &&\text{strongly\hspace{1mm} in } L^2(0,T;\hY),
\\ 
\C(M \gb * \hphi_{k_j}) &\longrightarrow \C(M\gb *\hphi) &&\text{strongly\hspace{1mm} in }
		L^2\big(0,T;L^2(\Omega;\R^{d\times d})\big), \\
		\gb*\hphi_{k_j} &\longweak \gb *\hphi &&\text{weakly-$*$ in } L^\infty(0,T;\hY) \cap L^2(0,T;\hX), \\
		\gb*\hphi_{k_j} &\longrightarrow \gb *\hphi &&\text{strongly\hspace{1mm} in }
		C([0,T];\hX') \cap L^2(0,T;\hY).
		\end{aligned}
	\end{equation} 
	\end{lemma}
\begin{proof}
In Lemma \ref{Lem:EstComb} we stated various $k$-uniform bounds on $u_k$ and $\hphi_k$. Thanks to the Banach--Alaoglu and Eberlein--\v{S}mulian theorems, see \cite[Theorem 8.10]{alt2016linear}, there are weakly/weakly-$*$ converging subsequences $u_{k_j}$ and $\hphi_{k_j}$. In particular, we obtain the convergences
\begin{equation} \label{Eq:Weak1} \begin{aligned}	
u_{k_j} &\longweak u &&\text{weakly-$*$ in } L^\infty(0,T;\mathcal{H}_0), \\
u_{k_j} &\longweak u &&\text{weakly\phantom{-*} in } L^2(0,T;\mathcal{V}_0), \\
\hphi_{k_j} &\longweak 		 \hphi &&\text{weakly\phantom{-*} in }
L^2(0,T;\hX). 
\end{aligned}\end{equation}


We shall establish the strong convergence of $u_{k_j}$ in $L^2(0,T;\mathcal{H}_0)$ by applying the Aubin--Lions compactness lemma; see \cref{Eq:aubin}. To this end, we need to bound the time derivative of $u_k$ in a suitable dual space. Let us therefore consider an arbitrary element $v \in L^{8/(4-d)}(0,T;\mathcal{V}_0)$ and bound each of the terms appearing on the right-hand side of \cref{Eq:NS_dis} below by means of H\"older's inequality:
$$\begin{aligned}\int_0^T | \langle \pt u_k,v \rangle_{\mathcal{V}_0}| \dt   &= \int_0^T  \Big| -((u_k \cdot \nablax)u_k,\Pi_{\mathcal{H}_k} v)_{\mathcal{H}}  \\ 
&\quad  - \nu (\nablax u_k,\nablax \Pi_{\mathcal{H}_k} v)_{\mathcal{H}} - k_B \mu_T \big(\C(M\gb*\hphi_k),\nablax \Pi_{\mathcal{H}_k} v \big)_{\mathcal{H}} \Big| \dt
	\\ 
& \leq C \int_0^T \Big( \|u_k\|_{L^4(\Omega)} \|u_k\|_{\mathcal{V}} \|\Pi_{\mathcal{H}_k} v\|_{L^4(\Omega)}  \\ &\quad  + \|u_k\|_{\mathcal{V}} \|\Pi_{\mathcal{H}_k} v\|_{\mathcal{V}} +  \|\C(M\gb*\hphi_k)\|_{\mathcal{H}}  \|\Pi_{\mathcal{H}_k} v\|_{\mathcal{V}} \Big) \dt. 
\end{aligned}$$
Hence, using Ladyzhenskaya's inequality, we have that
$$\begin{aligned}
&\int_0^T | \langle \pt u_k,v \rangle_{\mathcal{V}_0}| \dt\\  
 &\leq C \int_0^T \Big( \|u_k\|_{\mathcal H}^{1-d/4} \|u_k\|_{\mathcal{V}}^{1+d/4} \|\Pi_{\mathcal{H}_k} v\|_{L^4(\Omega)} + \|u_k\|_{\mathcal{V}} \|\Pi_{\mathcal{H}_k} v\|_{\mathcal{V}} +  \|\hphi_k\|_\hY  \|\Pi_{\mathcal{H}_k} v\|_{\mathcal{V}} \Big) \dt 
 \\
&\leq C \Big( \|u_k\|_{L^\infty {\mathcal{H}}}^{1-d/4} \|u_k\|_{L^2{\mathcal{V}}}^{1+d/4} \|v\|_{L^{8/(4-d)} {\mathcal{V}}} + \|u_k\|_{L^2{\mathcal{V}}} \|v\|_{L^2{\mathcal{V}}} +  \|\hphi_k\|_{L^2\hY} \|v\|_{L^2{\mathcal{V}}} \Big)
	\\
	&\leq C\|v\|_{L^{8/(4-d)} {\mathcal{V}}}.
\end{aligned}$$
This then implies that $\pt u_k$ is bounded in $L^{8/(4+d)}(0,T;\mathcal{V}_0
 ')$. 
It follows by the Aubin--Lions lemma \cref{Eq:aubin} that  %
\begin{equation} \label{Eq:Weak2} \begin{aligned}	
\pt u_{k_j} &\longweak \pt u &&\text{weakly\phantom{-*} in } L^{8/(4+d)}(0,T;\mathcal{V}_0'), \\
u_{k_j} &\longrightarrow  u &&\text{strongly\hspace{1mm} in } L^2\big(0,T;L^{r}(\Omega;\R^d)\big), \\
u_{k_j} &\longrightarrow  u &&\text{strongly\hspace{1mm} in } C([0,T];\mathcal{V}_0'),
\end{aligned}\end{equation}
where $r \in [1,\infty)$ for $d=2$ and $r \in [1,6)$ for $d=3$.


Similarly, we consider an arbitrary element $\hzeta \in L^\frac{8}{4-d}(0,T;\hZ)$ and we recall that $\hZ$ was defined in the beginning of \cref{Sec:Maxwell}. We test the Galerkin equation of $\hphi_k$ with $\Pi_{\mathcal H_k} \hzeta$ giving
\begin{equation} \label{Eq:BoundDerivative} \begin{aligned}\int_0^T \!\! |\langle \pta \hphi_k,\Pi_{\mathcal H_k} \hzeta \rangle_{\hX}| \dt  &=\!
\int_0^T\!\! \Big| -((u_k \cdot \nablax) \hphi_k, \Pi_{\mathcal H_k} \hzeta)_\hY - \frac{1}{2\lambda} (\nablaq \hphi_k,\nablaq \Pi_{\mathcal H_k} \hzeta)_\hY   \\ &\quad - \eps (\nablax \hphi_k,\nabla_x \Pi_{\mathcal H_k} \hzeta)_\hY+ (\omega(u_k) q \hphi_k, \nablaq \Pi_{\mathcal H_k} \hzeta)_\hY \Big|\d t.\end{aligned} \end{equation}
We note that $u_k$ is bounded in $L^{8/d}(0,T;L^4(\Omega)^d)$ by the following interpolation result
$$\int_0^T \|u_k\|_{L^4}^{8/d} \dt \leq \int_0^T  \|u_k\|_{\mathcal H}^{8/d-2} \|u_k\|^{2}_{\mathcal V} \dt \leq \|u_k\|_{L^\infty \mathcal{H}}^{8/d-2} \|u_k\|_{L^{2} \mathcal{V}}^{2}.$$
Regarding the last term in \eqref{Eq:BoundDerivative}, we integrate by parts and estimate by H\"older's inequality
$$\begin{aligned} -  &\big(\omega(u_k)  q \hphi, \nablaq \hzeta\big)_\hY \\ &=  \big(\nablax \hphi_k (\nablaq \hzeta)^{\mathrm{T}} q,u_k\big)_\hY + \big( \hphi_k \nablax \nablaq \hzeta q,u_k\big)_\hY + \big(u_k \cdot q, \nablax \hphi_k \cdot \nablaq \hzeta\big)_\hY \\ &\quad  + \big(u_k \cdot q, \hphi_k \div_x \nablaq \hzeta\big)_\hY \\
	&\leq C  \| \nablax \hphi_k\|_{L^2\hY}
	\big( \|\nablaq \hzeta\|_{L^{8/(4-d)} \hY}  +  \|\nablax \nablaq \hzeta\|_{L^{8/(4-d)} \hY}  \big) \|q\|_{L^\infty} \|u_k\|_{L^{8/d} L^4}  \\ &\quad+ \| \hphi_k\|_{L^2\hX} \|\nablax \nablaq \hzeta\|_{{L^{8/(4-d)} \hY}} \|q\|_{L^\infty} \|u_k\|_{L^{8/d} L^4}  \\ &\quad + C \|u_k\|_{L^{8/d} L^4} \|q\|_{L^\infty} \|\nablax \hphi_k\|_{L^2 \hY} \big(\|\nablaq \hzeta\|_{L^{8/(4-d)}\hY}  + \|\nablax \nablaq \hzeta\|_{L^{8/(4-d)}\hY} \big) \\ &\quad +\|u_k\|_{L^{8/d} L^4} \|q\|_{L^\infty}  \|\hphi_k\|_{L^2 \hX} \|\div_x \nablaq \hzeta\|_{L^{8/(4-d)}\hY}  \\
	&\leq C \|u_k\|_{L^{8/d} L^4} \|q\|_{L^\infty}  \|\hphi_k\|_{L^2 \hX} \| \hzeta\|_{L^{8/(4-d)}(0,T;\hZ) } .
\end{aligned} $$
Using this estimate, we can bound \eqref{Eq:BoundDerivative} as follows:
\begin{equation} \label{Eq:BoundDerivativePhi} \begin{aligned}& \int_0^T |\langle \pta \hphi_k,\Pi_{\mathcal H_k} \hzeta \rangle_{\hX}| \dt  \\ &\leq C \Big( \|u_k \|_{L^\infty\mathcal{H}_0 }  \|\hphi_k \|_{L^2 \hX}   \|\hzeta \|_{L^2\hX} +  \|\nablaq \hphi_k\|_{L^2\hY}   \|\nablaq \hzeta \|_{L^2\hY}    \\ &\quad +  \| \nablax \hphi_k\|_{L^2\hY}   \|\nabla_x\hzeta \|_{L^2\hY} + \|u_k\|_{L^{8/d} L^4} \|q\|_{L^\infty}  \|\hphi_k\|_{L^2 \hX} \| \hzeta\|_{L^{8/(4-d)}(0,T;\hZ) } \Big) \\
		&\leq  C \|\hzeta\|_{L^{8/(4-d)}(0,T;\hZ) } .
\end{aligned} \end{equation}
Hence, we obtain the $k$-uniform boundedness of $\pt(\gb*\hphi_{k})=\pt^{\alpha} \hphi_k$ in the space $L^{8/(4+d)}(0,T;\hZ')$, which is continuously embedded  in $L^{8/(4+d)}(0,T;(H^2_M(\Omega \times D))')$. 
Therefore, we are in the setting of the Gelfand triple
$$\hX \com \hY\hookrightarrow \big( H_M^2(\Omega \times D)\big)'.$$ 
%and $\hphi_{k}$ is bounded in the space 
%$$L^2(0,T;\hX) \cap W^{\alpha,8/(4+d)}(0,T;(H^2(\Omega \times D;M))').$$
We thus obtain from the fractional Aubin--Lions lemma, see \eqref{Eq:aubinfractional2}, that
\begin{equation} \label{Eq:Weak3} \begin{aligned}	
		\pt^{\alpha} \hphi_{k_j} &\longweak \pt^{\alpha} \hphi &&\text{weakly\phantom{-*} in } L^{8/(4+d)}(0,T;\hZ'), \\
		\hphi_{k_j} &\longrightarrow  \hphi &&\text{strongly\hspace{1mm} in } L^2(0,T;\hY).
\end{aligned} \end{equation}

%We note that we can test the weak form of $\hphi_k$ again by $\hzeta=g_\alpha * \pta \hphi_k$, i.e., we consider the tested weak form \cref{Eq:TestingRHS}. However, this time we do not apply the fractional chain inequality on the first term on the left-hand side of \cref{Eq:TestingRHS} but we exploit the coercivity of the kernel function in the integrated weak form, see
% \cref{Eq:Coercive2}, to obtain
% $$\int_0^t (\pta \hphi_k,\hphi_k - \ga \hpsi_k^0)_\hY \ds  \geq  \cos(\alpha \pi/2)  \Big( \tfrac12 \|\pt^{\alpha/2} \hphi_k\|^2_{L^2_t \hY}-\tfrac{\Gamma(\alpha-1)}{\Gamma(\alpha/2)^2}g_{\alpha}(t) \|\hpsi_k^0\|_\hY^2 \Big). $$
  %We take $t=T$ and use the derived energy estimates to bound the right-hand side of \cref{Eq:TestingRHS} similiarly to before. 
 % We can apply the fractional Aubin--Lions lemma, see \eqref{Eq:aubinfractional2}, to obtain the strong convergence of $\hphi_{k_j}$ in $L^2(0,T;\hY)$; in other words, 
The convolution $\gb*\hphi_{k}$ is bounded in $L^2(0,T;\hX)$ thanks to Young's convolution inequality 
$$\|\gb * \hphi_k\|_{L^2_t\hX} \leq \|g_{1-\alpha}\|_{L^1_t} \|\hphi_{k}\|_{L^2_t\hX} \leq C T^{1-\alpha} \|\hphi_{k}\|_{L^2_t\hX}.$$
Moreover,  $\gb*\hphi_{k}$ is bounded in $L^\infty(0,T;\hY)$ by the following chain of estimates:
$$\begin{aligned} &\|\gb*\hphi_{k}\|_{L^\infty \hY} \\&\quad \leq \sup_{t \in (0,T)} \int_0^t \gb(t-s) \|\hphi_{k}(s)\|_\hY \ds \\ 
	&\quad \leq \sup_{t \in (0,T)} \int_0^t \gb(t-s) \|\hphi_{k}(s)-\hpsi_{k}^0 \ga(s)\|_\hY \ds + (\gb*\ga)(t) \|\hpsi_{k}^0\|_\hY   \\ 
	&\quad \leq   \sup_{t \in (0,T)} \int_0^t \gb(t-s) \|\hphi_{k}(s)-\hpsi_{k}^0 \ga(s)\|^2_\hY \ds + \frac14 \int_0^t \gb(t-s) \ds + \|\hpsi_{k}^0\|_\hY  
	\\ &\quad = \sup_{t \in (0,T)} (\gb*\|\hphi_{k}-\hpsi_{k}^0 \ga\|_\hY^2)(t) + \frac14 g_{2-\alpha}(T) + \|\hpsi_{k}^0\|_\hY,
\end{aligned}$$
and the first term on the right-hand side is bounded by \cref{Lem:EstComb}. Since we have already proved a bound on $\pt(\gb*\hphi_{k})=\pta \hphi_{k}$, see \eqref{Eq:BoundDerivativePhi}, we may use the Aubin--Lions lemma, see \cref{Eq:aubin}, to obtain the following strong convergence results: 
\begin{equation} \label{Eq:Weak5} \begin{aligned}	
		\gb*\hphi_{k_j} &\longrightarrow  \gb*\hphi &&\text{strongly\hspace{1mm} in } L^2(0,T;\hY), \\
		\gb*\hphi_{k_j} &\longrightarrow  \gb*\hphi &&\text{strongly\hspace{1mm} in } C([0,T];\hX').
\end{aligned} \end{equation}
Lastly, we note that the mapping $M\gb*\varphi \mapsto \C(M\gb*\varphi)$ is linear and continuous thanks to \eqref{Eq:C},  %\leq C \|g_{1-\alpha}\|_{L^1_t} \|\varphi\|_{L^2_t\hY} \leq C T^{1-\alpha} \|\varphi\|_{L^2_t\hY}, $$
and therefore we have from \eqref{Eq:Weak5}$_1$ that
\begin{equation} \label{Eq:Weak4} \C(M \gb * \hphi_{k_j}) \longrightarrow \C(M\gb *\hphi) \quad \text{ strongly in }
L^2\big(0,T;L^2(\Omega;\R^{d\times d})\big).
\end{equation} 
\end{proof}

\subsection{Passage to the limit}
Next, we pass to the limit $j \to \infty$ in the  time-integrated $k_j$-th Galerkin system \cref{Eq:NS_dis}, \cref{Eq:FP_dis}. Specifically, we shall use the convergence results stated in the preceding lemma to show that the weak limits, $u$ and $\hphi$, satisfy the variational Navier--Stokes--Fokker--Planck system in the sense of \cref{Eq:DefWeak}. 
%
\begin{proof}[Proof of \cref{Thm:WellPosedness}]
We consider the time-integrated Galerkin system
\begin{align} 
&\int_0^T \Big(\langle\pt u_{k_j},v \rangle_V  + ((u_{k_j} \cdot \nablax)u_{k_j},v)_{\mathcal{H}} \label{Eq:NS_dis_time} \\ 
& \quad + \nu (\nablax u_{k_j},\nablax v)_{\mathcal{H}}+ k \mu (C(M \gb *\hphi_{k_j}),\nablax v)_{\mathcal{H}} \Big) \eta(t) \dt =0  \notag\\
&\int_0^T  -(\gb*\hphi_{k_j},\hzeta)_\hY \eta'(t) + \Big(    ((u_{k_j}\cdot \nablax) \hphi_{k_j}, \hzeta)_\hY+ \frac{1}{2\lambda} ( \nablaq \hphi_{k_j},\nablaq \hat\zeta)_\hY \label{Eq:FP_dis_time} \\
&\quad + \eps (\nablax \hphi_{k_j},\nabla \hat\zeta)_\hY- (\omega(u_{k_j}) q \hphi_{k_j}, \nablaq \hat\zeta)_\hY \Big) \eta(t) \dt=0, \notag \end{align} for all $v \in \H_{k_j}$, $\eta \in C_0^\infty(0,T)$ and $\hat\zeta \in \hY_{k_j}$.  
Passing to the limit $j \rightarrow \infty$ in \eqref{Eq:NS_dis_time} using \cref{Eq:Weak1}--\cref{Eq:Weak4}
is standard, and results in \eqref{Eq:NS}. It therefore remains to pass to the limit $j \rightarrow \infty$ in \eqref{Eq:FP_dis_time}. In particular, the convergence of the linear terms follow immediately by weak convergence, and we only consider the two nonlinear terms. 
 We note that $\hphi_{k_j} \to \hphi$ strongly in $L^2\big(0,T;\hY)$ and $\omega(u_{k_j}) \to \omega(u)$ weakly in $L^2(0,T;\H_0)$, from which we deduce that
$$\int_0^T (\omega(u_{k_j}) q \hphi_{k_j}, \nablaq \hat\zeta)_\hY  \eta(t) \dt \longrightarrow  \int_0^T (\omega(u) q \hphi, \nablaq \hat\zeta)_\hY  \eta(t) \dt,$$
as $j \to \infty$. With the same reasoning, we are able to show that
$$\int_0^T  ((u_{k_j}\cdot \nablax) \hphi_{k_j}, \hat\zeta)_{\hY} \eta(t) \dt \longrightarrow \int_0^T  ( (u \cdot\nablax) \hphi, \hat\zeta)_{\hY} \eta(t) \dt\quad \mbox{as $j \to \infty$}.$$

%To this end we apply integration by parts in time in \eqref{Eq:FP_dis_time}, which yields
%
%$$\begin{aligned} 0=&- \int_0^T ( M\gb * \hphi_{k_j},\hat\zeta)_{\mathcal{Y}} \eta'(t) \dt - ( M\hpsi_0,\hat\zeta)_{\mathcal{Y}} \eta(0) && \\ &+  
%\int_0^T \Big( ((u_{k_j}\cdot \nablax) \hphi_{k_j}, \hzeta)_\hY+ \frac{1}{2\lambda} ( \nablaq \hphi_{k_j},\nablaq \hat\zeta)_\hY \\
%&\quad + \eps (\nablax \hphi_{k_j},\nabla \hat\zeta)_\hY- (\omega(u_{k_j}) q \hphi_{k_j}, \nablaq \hat\zeta)_\hY \Big) \eta(t) \dt=0, \notag 
%\end{aligned}$$
%for all $\hat\zeta \in \hX_{k_j}$ and $\eta \in C_0^\infty(-T,T)$. We apply Young's convolution inequality to the first term on the right-hand side to obtain
%$$\begin{aligned} - \int_0^T (M \gb * \hphi_{k_j},\hat\zeta)_{\mathcal{Y}} \eta'(t) \dt &\leq \|\gb * \hphi_{k_j}\|_{L^1\hY} \|\hat\zeta\|_\hY \|\eta'\|_{L^\infty_T}
%\\&\leq C \|\gb\|_{L^1_T} \|\hphi_{k_j}\|_{L^1 \hY} \|\hat\zeta\|_\hY  \|\eta'\|_{L^\infty_T},
%\end{aligned} $$
%from which we conclude the convergence of the first term.
%We exploit the corotationality of $\omega$ in the form of
%$$(M\omega(u)q,\eta)_{\mathcal{Y}} = \frac12 (M v \cdot q, \div_x \eta)_{\mathcal{Y}} - \frac12 (M\nablax \eta q,v)_{\mathcal{Y}},$$
%which allows us to rewrite \cref{Eq:FP_dis_time} as
%$$\begin{aligned} 0=&- \int_0^T ( M\gb * \hphi_{k_j},\hat\zeta)_{\mathcal{Y}} \eta'(t) \dt - ( M\hpsi_0,\hat\zeta)_{\mathcal{Y}} \eta(0) && \\  &+ \frac12 \int_0^T \Big(  2 (M (u_{k_j} \cdot \nablax) \hphi_{k_j}, \hat\zeta)_{\mathcal{Y}}  + \frac{1}{\lambda} (M \nablaq \hphi_{k_j},\nablaq \hat\zeta)_{\mathcal{Y}} + 2\eps (M\nablax \hphi_{k_j},\nablax \hat\zeta)_{\mathcal{Y}} && \\   
%&\qquad \qquad - \big(M u_{k_j} \cdot q, \div_x(\hphi_{k_j} \nablaq \hat\zeta)\big)_{\mathcal{Y}} +  (M\nablax (\hphi_{k_j} \nablaq \hat\zeta) q,u_{k_j})_{\mathcal{Y}} \Big) \eta(t) \dt,
%\end{aligned}$$
%for all $\hat\zeta \in \hX_{k_j}$ and $\eta \in C_0^\infty(-T,T)$.

%{\color{red}~ I don't understand why it was necessary in the 7 lines, immediately above, to rewrite $(M\omega(u)q,\eta)$ as $\frac12 (M v \cdot q, \div_x \eta)_{\mathcal{Y}} - \frac12 (M\nablax \eta q,v)_{\mathcal{Y}}$. I also don't understand the argument below. What space does $\zeta$ belong to? }

%\medskip

%{\color{blue} Since $u_{k_j} \to u$ strongly in $L^2\big(0,T;L^4(\Omega)\big)$, $u_{k_j} \zeta \to u \zeta$ strongly in $L^2\big(0,T;L^2(\Omega \times D;\R^d)\big)$  and $\nablax \hphi_{k_j} \rightharpoonup \nablax \hphi$ weakly in $L^2(0,T;\hY)$, it follows that
%$$\int_0^T  (M (u_{k_j}\cdot \nablax) \hphi_{k_j}, \hat\zeta)_{\mathcal{Y}} \eta(t) \dt \longrightarrow \int_0^T  (M (u \cdot\nablax) \hphi, \hat\zeta)_{\mathcal{Y}} \eta(t) \dt.$$
%Passing to the limit $k \rightarrow \infty$ in the terms with prefactors $1/\lambda$ and $2\eps$ in the second integral on the right-hand side is straightforward using weak convergence 
%of $\hphi_k$ to $\hphi$ in $\widehat{\mathcal{X}}=L^2(0,T; H^1(\Omega \times D;M))$.


%It remains to deal with the last two terms. For the nonlinearities that couple the Fokker--Planck equation to the Navier--Stokes equation, we note again
%$$\int_0^T  ((u_k \cdot \nablax)\xi,\varphi)_{\mathcal{H}} + ((u_k \cdot \nablax)\varphi,\xi)_{\mathcal{H}} \dt = \int_0^T \int_\Omega \div_x((\xi \cdot \varphi)u_k) \, \text{d}x \dt=0,$$
%for any $\xi,\varphi \in L^2(0,T;\hX)$.}


We use the density of $\cup_{k=1}^\infty \H_{k}$ in $\mathcal{V}$ and of $\cup_{k=1}^\infty \hY_{k}$ in $H^2_M(\Omega \times D)$, which completes the proof by observing that the tuple $(u,\hphi)$ satisfies the variational form of the time-fractional Navier--Stokes--Fokker--Planck system as stated in \cref{Eq:DefWeak}. 

It remains to check that the initial conditions are satisfied. First, we obtain the convergence $u_{k_j}(0) \to u(0)$ in $\mathcal{V}_0'$  as $j \rightarrow \infty$; see again \cref{Eq:Weak}. However, by definition, $u_{k_j}(0)=\Pi_{\H_{k_j}} u^0$, which converges to $u^0$ in $\mathcal{H}_0$
as $j \rightarrow \infty$. By the uniqueness of the limit it follows that $u(0)=u^0$. Regarding the solution of the Fokker--Planck equation, we use again the strong convergence \cref{Eq:Weak} to conclude $(\gb*\hphi)(0) = \hpsi^0$.
\end{proof}


 










\section{Numerical simulations} \label{Sec:Numerics}



%One might also investigate the mixed system
%\begin{equation} \begin{aligned}
%    \phi &= \ptb \psi \\
%	0 &= \pt \psi -D \Delta \phi + \div (F  \phi)  , \end{aligned} 
%\label{Eq:System5}
%\end{equation}
%with the initial $\psi(0)=\psi^0$. Theoretically, it should hold $\phi^0 = \psi^0 g_\alpha(0)=\infty$. 

Various numerical methods for time-fractional PDEs are summarized in the review article \cite{diethelm2020good} and in the monographs \cite{baleanu2012fractional,owolabi2019numerical,jin2023numerical}. %Since the Caputo derivative $\capb \psi=\ptb (\psi-\psi_0)$ is usually treated in a discrete manner, we follow this approach by adding and subtracting the initial condition $\psi^0$ from the original system as follows
%$$\begin{aligned}\pt \psi - \div(D \nabla \ptb (\psi-\psi_0)) +\div(F \ptb (\psi-\psi_0)) &=f+\div(D\nabla \ptb \psi_0) - \div(F  \ptb \psi_0).% \\
%&=\ga (D \Delta \psi_0 -  F \cdot \nabla \psi_0).
%\end{aligned}$$
%Consequently, we can write the system in terms of the Caputo derivative 
%$$\begin{aligned}\pt \psi - \div(D\nabla \ptb \psi) +\div(F  \ptb \psi) = f+ \ga  \div(D\nabla \psi_0) - \ga  \div(F  \psi_0),
%\end{aligned}$$
%where we further used that $\ptb 1 = \ga$.
%We note that the right-hand side of the PDE is an element in $L^{1/(1-\alpha)-\eps}(0,T;H^{-1}(\Omega))$ for $D,F \in L^\infty(\Omega_T)$, $f\in L^2(0,T;H^{-1}(\Omega))$ and $\psi_0 \in H^1(\Omega)$. We observe that $\alpha> 1/2$ yields a right-hand side in $L^2(0,T;H^{-1}(\Omega))$.
%ga in Lp for a>1-1/p i.e 1/p>1-a i.e 1>p(1-a) i.e. p<1/(1-a)

We assume a discretization $0=t_0<t_1<\dots<t_N=T$ of the time interval $[0,T]$. We do not utilize an equispaced time mesh, but a nonuniform one by discretizing the early times in finer steps. In particular, we assume that the $n$-th time step is of the form $t_n=(n/N)^\gamma T$ for $\gamma\geq 1$. If it holds $\gamma=1$, then we are again in a setting of a uniform mesh, see also Fig. \ref{Fig:Time} for a depiction of some time meshes for various values of $\gamma$.

% Figure environment removed

We discretize the Caputo derivative by the nonuniform L1 scheme \cite[Section 3.2]{diethelm2020good}, i.e., it reads $$\ptb \psi \approx \frac{1}{\Gamma(1+\alpha)} \sum_{j=0}^{n-1} \omega_{n-j-1,n} (\psi_{n-j}-\psi_{n-j-1}),
%=\frac{1}{\Gamma(1+\alpha)} \bigg( \psi_n-\psi_{n-1} + \sum_{j=1}^{n-1} \omega_{n-j-1,n} (\psi_{n-j}-\psi_{n-j-1}) \bigg)
$$ where $\psi_{n-j} \approx \psi(t_{n-j})$. The quadrature weights $\{\omega_{k,n}\}_{k=0}^{n-1}$ are given by the formula
$$\omega_{k,n}=\frac{(t_n-t_k)^{\alpha}-(t_n-t_{k+1})^{\alpha}}{\Delta t_{n-k}},$$
where we introduced the notation $\Delta t_{n-k}=t_{n-k}-t_{n-k-1}$. We use the finite element space $P_1$ for the space discretization and consequently, the fully discrete system reads
\begin{equation} \label{Eq:FP_Discretized} \begin{aligned}& \Big(\frac{\psi^n-\psi^{n-1}}{\Delta t_n},\zeta\Big)_H +  \sum_{j=0}^{n-1} \frac{\omega_{n-j-1,n}}{\Gamma(1+\alpha)}  (D\nabla(\psi_{n-j}-\psi_{n-j-1}),\nabla \zeta)_H \\&\quad - \sum_{j=0}^{n-1}   \frac{\omega_{n-j-1,n}}{\Gamma(1+\alpha)} (\psi_{n-j}-\psi_{n-j-1},F(t_n)\cdot \nabla \zeta)_H   \\ &= (f(t_n),\zeta)_H -  g_\alpha(t_n) \cdot (D(t_n)\nabla \psi_0,\nabla \zeta)_H + g_\alpha(t_n) \cdot  (\psi_0, F(t_n) \cdot\nabla \zeta)_H
\end{aligned} 
\end{equation}
%or written differently by multiplying by $\Delta t$ and  bringing the $\psi_n$-terms to the left-hand side and the remaining terms to the right-hand side
%$$\begin{aligned}& (\psi^n,u)_H + (\Delta t)^{\alpha}   D (\nabla\psi_{n},\nabla u)_H -(\Delta t)^{\alpha}  (\psi_{n},F\cdot \nabla u)_H   \\ &=(\psi^{n-1},u)_H + (\Delta t)^{\alpha}   D (\nabla\psi_0,\nabla u)_H-(\Delta t)^{\alpha}  (\psi_0,F\cdot \nabla u)_H \\ &\quad -\Delta t g_\alpha(t) D(\nabla \psi_0,\nabla u)_H +\Delta t g_\alpha(t) (\psi_0, F \cdot\nabla u)_H \\
%&\quad -(\Delta t)^{\alpha} \sum_{j=1}^{n-1}  D (\nabla(\psi_{n-j}-\psi_0),\nabla u)_H +(\Delta t)^{\alpha} \sum_{j=1}^{n-1}  (\psi_{n-j}-\psi_0,F\cdot \nabla u)_H \Big]
%\end{aligned} $$
for any test function $\zeta$. In particular, taking $\zeta=1$ gives
$$\begin{aligned}& \int_\Omega \psi^n \dx =\int_\Omega \psi^{n-1} \dx    + \Delta t_n \int_\Omega f(t_n) \dx,  
\end{aligned} $$
i.e., the Fokker--Planck setting with $f\equiv 0$ yields discrete mass conservation. We implement the discrete system in open-source computing platform \linebreak FEniCS, see \cite{alnaes2015fenics}.

%$$\begin{aligned}& (\psi^n,u)_H + (\Delta t)^{\alpha} D (\nabla\psi_{n},\nabla u)_H - (\Delta t)^{\alpha} (\psi_{n},F\cdot \nabla u + u\div F)_H   \\ &=(\psi^{n-1},u)_H -\Delta t \cdot g_\alpha(t) \cdot \big(D(\nabla \psi_0,\nabla u)_H + (F \cdot \nabla \psi_0,u)_H\big)
%\\&\quad + (\Delta t)^{\alpha} D (\nabla\psi_0,\nabla u)_H - (\Delta t)^{\alpha} (\psi_0,F\cdot \nabla u + u\div F)_H 
%\\&\quad  - (\Delta t)^{\alpha} \sum_{j=1}^{n-1} \Big[ D (\nabla(\psi_{n-j}-\psi_0),\nabla u)_H -  (\psi_{n-j}-\psi_0,F\cdot \nabla u + u\div F)_H \Big] 
%\end{aligned} $$
 We consider the space interval $\Omega=(-5,15)$ with $\Delta x=1/1024$  and the time interval $[0,T]$ with $T=5$ where the $n$-th time step is given by $t_n=5(n/100)^2$. Moreover, we select as the initial data the Gaussian
$$\psi(0,x)=\psi_0(x)=\frac{1}{\sigma \sqrt{2\pi}} \text{exp}\Big(-\frac12 \Big( \frac{x-\mu}{\sigma} \Big)^2 \Big)$$
for $\sigma=0.1$ and $\mu=2$. 

Regarding model parameters, we choose $D=1$ and $f\equiv 0$. We take the space-time dependent force $F(t,x)=\sin(t)+x$ in Sec. \ref{Sec:Ex2} similar to \cite{angstmann2015generalized,mustapha2022second,le2016numerical,pinto2017numerical}. However, we first consider the case of an absent force $F \equiv 0$ in Sec. \ref{Sec:Ex1}, i.e., we are in the setting of a classical subdiffusion equation. In Sec. \ref{Sec:Ex3}, we consider the physically defeasible time-fractional Fokker--Planck equation with the Caputo derivative on the left-hand side, see again Sec. \ref{Sec:Derivation}, and compare this model numerically to the physically meaningful model that we have analyzed in this work.

\subsection{Example 1: Subdiffusion equation} \label{Sec:Ex1}

As we consider $F\equiv 0$ in this example, we essentially study the time-fractional heat equation
$$\pta \psi(x,t)=\Delta \psi(x,t),$$
which is also referred to as subdiffusion equation.

We observe the typical behavior of a subdiffusive equation in the numerical simulations. At early times, the time-fractional model evolves faster stand the integer-order model. In Fig. \ref{Fig:F0_AlphaVary} (a), we see that the solution is more damped for $\alpha<1$ than for $\alpha=1$ at $t=0.02$. Moreover, the damping is larger for smaller values for $\alpha$. However, this behavior is exactly flipped if one considers a point further in time, e.g. $t=0.5$ as depicted in Fig. \ref{Fig:F0_AlphaVary} (b). After the initial fast evolution of the subdiffusion equation, the process is slower, and we observe that the smallest maximal value is represented by $\alpha=1$ at $t=0.5$. We can also observe that for $\alpha=1$ the typical round shape is present, whereas for $\alpha<1$ the tip at $x=2$ is less round.





% Figure environment removed

% Figure environment removed

% Figure environment removed


We consider the time evolution for $\alpha=1$ in Fig. \ref{Fig:F0_TimeVary} (a) and for $\alpha=\frac12$ in Fig. \ref{Fig:F0_TimeVary} (b). The typical diffusion process can be observed and again, we notice the spikier tip for $\alpha=\frac12$. Moreover, the support of the function is larger for smaller $\alpha$.



Lastly, we try to fit the solution $\psi$ for different values of $\alpha$. The goal is to analyze whether it is necessary to consider the more complicated (analytically and numerically) time-fractional model, or whether this model's behavior can be replicated by an integer-order model.  This is done in Fig. \ref{Fig:F0_Fitting}, and we observe that the subdiffusive behavior cannot be imitated directly by the standard Fokker--Planck equation. Again, we observe the different support for each curve and the difference in the tip at $x=2$.





\subsection{Example 2: Space-time dependent force} \label{Sec:Ex2}
This time, we consider the space-time dependent force $F(x,t)=\sin(x)+t$ and therefore, we study the time-fractional Fokker--Planck equation
$$\begin{aligned}
&\pt \psi(x,t)-\Delta \ptb\psi(x,t)+ \div(F(x,t)\ptb \psi(x,t)) = \ga D\Delta \psi_0 - \ga \div(F\psi_0).
\end{aligned}$$

Again, we observe the typical initial behavior of a subdiffusive equation. At the start, the time-fractional model evolves much faster stand the integer-order model. In Fig. \ref{Fig:F1_AlphaVary} (a), we see that the solution is more damped for $\alpha<1$ than for $\alpha=1$ at $t=0.02$. However, this time, we observe that the symmetry of the probability density functional $\psi$ is lost for $\alpha<1$. In the case of $\alpha=\frac12$ and $\alpha=\frac14$, the solution admits a large support up to the right end of the domain. that  In Fig. \ref{Fig:F1_AlphaVary} (b), we have plotted $\psi$ at a later time.  We observe that $\alpha=1$ is vastly different from the case of $\alpha<1$. This is also pronounced by the fact that $\ga(t) \to 0$ as $t \to \infty$ for $\alpha<1$, but in the case of $\alpha=1$ it holds $\ga(t) \equiv 1$, i.e., the right-hand side is just as large for all times.





% Figure environment removed

% Figure environment removed


We consider the time evolution for $\alpha=1$ in Fig. \ref{Fig:F0_TimeVary} (a) and for $\alpha=\frac12$ in Fig. \ref{Fig:F0_TimeVary} (b). The typical diffusion process can be observed and again, we notice the edgier tip for $\alpha=\frac12$. Moreover, the support of the function is larger for smaller $\alpha$.

Lastly, we try to fit the solution $\psi$ for different values of $\alpha$. This is done in Fig. \ref{Fig:F0_Fitting}, and we observe that the subdiffusive behavior cannot be imitated by an integer-order model. Again, we observe the different support for each curve and the difference in the tip at $x=2$.



% Figure environment removed

\subsection{Example 3: Model comparison} \label{Sec:Ex3}

We consider the model as introduced in \eqref{Eq:ModelWrong} with no right-hand side, i.e.,
\begin{equation} \label{Eq:ModelWrong1} \begin{aligned}
&\pta \psi(x,t)-D \Delta \psi(x,t) +\div \big(F(t,x) \psi(x,t) \big) =0,
\end{aligned}\end{equation}
and we discretize it in the same manner as done for the time-fractional Fokker--Planck equation in \eqref{Eq:FP_Discretized}. Since this model has been studied in literature, we want to give it some attention by comparing it to the physically meaningful model. Again, we consider $F(x,t)=\sin(x)+t$.

We compare it for $\alpha=\frac14$ in Fig. \ref{Fig:Wrong} (a) and for $\alpha=\frac34$ in Fig. \ref{Fig:Wrong} (b) for several time steps. We notice that the error gets larger for increasing time, and it is also more pronounced for smaller values for $\alpha$. We argue that this results from the fact that these models coincide for $\alpha=1$ and by continuity of the fractional parameter, the difference only gets larger the further one is from $\alpha=1$. Moreover, it holds $\ga(t) \to 0$ as $t \to \infty$ for $\alpha<1$ and therefore, it makes sense that asymptotically the right-hand side is negligible.

% Figure environment removed

%\section*{Some sources}

%Good \cite{angstmann2015generalized}: $F=-x+\eps \sin(5\pi t)$. In the case $\eps = 0$, where the external force does not vary in time, the results from the numerical simulations for the first moment and the variance are indistinguishable for the nondelayed forcing and the trap-time delayed forcing, in agreement with the algebraic analysis. The further discussion below is based on the case $\eps = 1$, i.e., the external force varies periodically in time.

%Mustapha A second-order accurate numerical scheme for a time-fractional Fokker-Planck equation: $F=\sin(t)-x$

%PENG The existence of mild and classical solutions for time fractional Fokker–Planck equations

%Pinto Numerical solution of a time-space fractional Fokker Planck equation with variable force field and diffusion: $F(x)=\sin(t)+x$, also in \cite{le2016numerical}

%\cite{le2018a}: A SEMIDISCRETE FINITE ELEMENT APPROXIMATION OF A: $F=-V'$ with $V=x^4/4 -x^2/2 - x \cos t$.

%Deng Numerical algorithm for the time fractional Fokker–Planck equation  $U=\cos x-6x$ and $F=-U'$

%A high-order compact difference method for time fractional: $F=e^{(x-1/2)^2}$

%A Space-Time Petrov-Galerkin Spectral Method for Time Fractional Fokker-Planck Equation with Nonsmooth Solution: $F=-x-1$, also have rhs-force $f$, $F=-1$, assume nonpositive and decreasing, have inhom Dirichlet

%Interval Shannon Wavelet Collocation Method for Fractional Fokker-Planck Equation: $F=-1$, $\psi_0=x(1-x)$, inhom Dir

%Numerical algorithms for the time-space tempered fractional Fokker-Planck equation: $F(x)=x^2$, zero initial, zero Dir and forcing fct and Gaussian initial in other ex and $F(x)=x$



%The acknowledgments section should not be numbered.
\section*{Acknowledgments}
Supported by the state of Upper Austria.

%%%%%%%%%%%%%%%%%%%%%%%%%%%%%%%%%%%%%%%%%%%%%%%%%%%%%%
%          7. REFERENCES SECTION
%%%%%%%%%%%%%%%%%%%%%%%%%%%%%%%%%%%%%%%%%%%%%%%%%%%%%%

%       READ THIS SECTION CAREFULLY

{\small	
	\bibliography{literature.bib}
	\bibliographystyle{AIMS} }

%\medskip
% The information below will be filled in by AIMS production staff.
%Received xxxx 20xx; revised xxxx 20xx; early access xxxx 20xx.
%\medskip

\end{document}


 
 \RequirePackage{fix-cm}
    %\documentclass{svjour3}                 % onecolumn (standard format)
    %\documentclass[smallcondensed]{svjour3} % onecolumn (ditto)

 \documentclass[smallextended]{svjour3} %% onecolumn (second format) %% use this one for FCAA ! %%
 \smartqed  % flush right qed marks, e.g. at end of proof %

%%% publisher's and journal definitions come here: %%%%%%%%%%%%%%%%%%%%%%%
\def\theequation{\arabic{section}.\arabic{equation}}%%% for FCAA ! %%%%%%%
   % \numberwithin{equation}{section}
\usepackage{amssymb,amsmath,amsfonts,latexsym,cite,hyperref,comment}
   \newtheorem{hypothesis}{Hypothesis}
   %%% similarly you can define other missing statements as Conjecture, etc.
   %%% how to make proofs so square to appear at ends: %%%%%%%%%%%%%%%%%%%%%
 \def\proofname{\smallskip \noindent\it Proof} %%
 \def\proofend{\hfill$\Box$} % same as: \def\qed{\hfill$\Box$} % for \square
 %% may add \smallskip afterwards...

%%%%%%%%%%%%%% additional, if necessary %%%%%%%%%%%%%%%%%%%%%%%%%%%%%%%%%%%%
\usepackage{mathtools}
\usepackage{multirow,array}
\usepackage{graphicx}
\usepackage{subfigure}
\usepackage{epstopdf}
\usepackage{float} %%% to use [H], [ht] etc. to fix places of figures
\usepackage{color, xcolor}
\allowdisplaybreaks

%%%%%% Specific Authors Defs come here: %%%%%%%%%%

%%% ........... %%%%%%%%%%%%%%%%%%%%%%%%%%%%%%%%%%

%%%%%%%%%%%%%%%%%%%%%%%%%%%%%%%%%%%%%%%%%%%%%%%%%%

%%%%%%%%%%%%%%% begin make title %%%%%%%%%%%%%%%%%

\journalname{Fract. Calc. Appl. Anal.} %%%%% and FCAA-logo will appear on the right %%

\input{preamble_fcaa}

    \allowdisplaybreaks

\begin{document}

 %%% Pls. specify the kind of the article:
 %%% This is: %% ORIGINAL PAPER %%%% 
 %%% or: REVIEW PAPER / SHORT PAPER, etc. %%%%%%%%

%%%% Title of article for FCAA %%%%%%%%%%%%%%%%%%%%%%%%
\title{Well-posedness and simulation of weak solutions to  the time-fractional Fokker--Planck equation with general forcing}

\titlerunning{Well-posedness and simulation of the time-fractional Fokker--Planck equation}
%% if too long for running head - use the text from 1st line !%

%%%%% authors:
\author{Marvin Fritz$^{1}$ 
 }

\authorrunning{M. Fritz} %%  Short form of author list % if too long for running head

%%% affiliations:
\institute{Marvin Fritz
\at
Computational Methods for PDEs, Johann Radon Institute for Computational and Applied Mathematics, Austrian Academy of Sciences, Linz, Austria\\
\email{marvin.fritz@ricam.oeaw.ac.at}
}

%%% Dates:
\date{Received: ... / Revised: ... / Accepted: ...}
% These dates will be entered by the Editor (EiC, V. Kiryakova) or authors/ system %

%%%%% For Production Dept.: variants of COPYRIGHT notice to appear: %%%%%%
% if Open Access option chosen, it is as:
   %% "\copyright The Author(s) 2022" % (or next year..) %
% if No Open Access, the TCA form signed to FCAA journal is used, and it appears:
   %% "\copyright Diogenes Co. Ltd. 2022" %% or 2021, or next year %%%%%
%%%%%%%%%%%%%%%%%%%%%%%%%%%%%%%%%%%%%%%%%%%%%%%%%%%%%%%%%%%%%%%%%%%%%%%%%%

\maketitle

%%%%%%%%%%%%%%%%%%%%%%%%%%%%%%%%%%%%%%%%%%%%%%%%%%%%%%
\begin{abstract}
In this paper, we investigate the well-posedness of the time-fractional Fokker--Planck equation.
Its dynamics are governed by anomalous diffusion, and
we describe the model's derivation from subordinated Langevin equations. 
As our primary result, we prove the existence of weak solutions to a class of time-fractional advection-diffusion PDEs including the time-fractional Fokker--Planck equation with space-time dependent force. In this regard, we derive an energy inequality by considering nonstandard and novel testing methods. Lastly, we  propose a numerical algorithm based on a nonuniform L1 scheme and present some simulation results for various forces.


%%%%%%%%%% Enter suitable key words and phrases %%% examples:
\keywords{time-fractional Fokker--Planck equation \and existence of weak solutions \and  anomalous diffusion \and energy inequality  \and nonuniform L1 scheme}

\subclass{26A33 \and  33E12 \and 34A08 \and 34K37 \and 35R11 \and 60G22}
%%% These are examples only. Pls. use MSC 2020 for suitable topic numbers %%%%%%

\end{abstract} %%%%%%%%%%%%%

%%%%%%%% begin papers' body %%%%%%%%%%%%%%%%%%%%%%%%%%%%%
		\section{Introduction}
	
This paper is concerned with the existence of weak solutions to a system of nonlinear partial differential equations that arises in the kinetic theory of dilute solutions of polymeric fluids. Within this class of models we focus on finitely-extensible nonlinear elastic, FENE-type, dumbbell models with a corotational drag term. In contrast to previous literature on the analysis of these models we assume power law waiting times in the derivation of the system, which results in the appearance of a time-fractional derivative in the Fokker--Planck equation describing the evolution of the probability density function. This raises new questions about the study of well-posedness, and we provide rigorous results concerning the existence of global-in-time weak solutions to the system of partial differential equations featuring in the model.

Dilute polymer models are derived and extensively described in the monograph \cite{bird1987dynamics2} and in the book by \"Ottinger \cite{ottinger2012stochastic}; see also \cite{suli2018mckeanvlasov} for a mathematically rigorous derivation of the Hookean bead-spring-chain model from Brownian dynamics. We also refer to the papers \cite{lemou2002viscoelastic,herrchen1997a} for a comparison of several FENE-type dumbbell models. Such systems are of microscopic-macroscopic type since they involve a coupling of the (macroscopic) Navier--Stokes equations for the description of incompressible fluid flow and the Fokker--Planck equation for the microscopic processes associated with 
the statistical properties of polymer molecules immersed in the fluid. 
Concerning the weak and strong well-posedness of FENE-type models, we refer to the works \cite{jourdain2004existence,kreml2010on,masmoudi2013global,zhang2006local,renardy1991an}. More general dilute polymer models are analyzed in \cite{barrett2005existence,barrett2007existence,barrett2008existence,barrett2010existence,barrett2010existence2}. Further, we mention the papers \cite{lions2000global,lions2007global,schonbek2009existence,masmoudi2008well,barrett2005existence,barrett2009numerical,debiec2023corotational,lin2008global,busuioc2014fene}, which, similarly to the discussion herein, are concerned with dumbbell models that assume a corotational drag term in the Fokker--Planck equation.  In such models it is supposed that polymer molecules are not stretched by the surrounding solvent, although they are allowed to rotate without stretching; see, for example, \cite{la2020diffusive}.


Time-fractional differential equations have been the focus of considerable attention in the mathematical and engineering literature in recent years. Such equations are nonlocal in time and have an innate history effect. They are of relevance in applications where memory effects are present and hereditary properties of materials are studied; see, for example, the textbooks on viscoelasticity \cite{mainardi2022fractional,yang2020general}, hydrology \cite{su2020fractional}, financial economics \cite{fallahgoul2016fractional}, and mechanical processes \cite{atanackovic2014fractional,pilipovic2014fractional}. The time-fractional Fokker--Planck system, in particular, allows subdiffusive behaviour and has been previously studied in \cite{metzler1999anomalous,metzler1999deriving,metzler2000random,barkai2000continuous,barkai2001fractional,henry2006anomalous,henry2010fractional,henry2010introduction,langlands2008anomalous} with regards to its derivation and applicability. The articles \cite{pinto2017numerical,le2016numerical,le2018a,le2019existence,le2021alpha} have investigated the numerical analysis and the simulation of solutions to the time-fractional Fokker--Planck equation. The time-fractional model considered herein has been explored computationally in \cite{beddrich2023numerical}, albeit in the simpler setting of a linear (Hookean) elastic spring force instead of the FENE spring model that we study here.
 
We employ a spatial Galerkin approximation in conjunction with a compactness argument to prove the existence of weak solutions to the time-fractional Navier--Stokes--Fokker--Planck system under consideration. More specifically, we discretize the system in space and derive appropriate energy bounds, which then enable us to pass to the limit in the discretized system.  Spatial discretizations of dilute polymer models were previously considered  in  \cite{barrett2009numerical,barrett2011finite,barrett2012finite}. In addition, weak solutions to time-fractional PDEs have been investigated using the Galerkin approach in the publications \cite{fritz2021sub,fritz2020time,fritz2021equivalence}. There have also been initial steps in the analysis of a decoupled time-fractional Fokker--Planck equation with time-dependent forces; see the papers \cite{fritz2023well,mclean2020regularity,le2019existence,le2021alpha,mclean2021uniform}. However, the coupling of the time-fractional Fokker--Planck equation to the Navier--Stokes system gives rise to new technical complications, which have not been addressed previously.


In \Cref{Sec:Derivation} we derive the model from the Langevin equation assuming power-law waiting time. In this way time-fractional derivatives in the sense of Riemann--Liouville appear in the associated  Fokker--Planck equation. By mimicking the technique for the derivation of the standard dumbbell model, a time-fractional Navier--Stokes--Fokker--Planck system is obtained.  In \Cref{Sec:Prelim} we introduce several function spaces of Sobolev-type and recall some important results from the theory of fractional derivatives, including chain inequalities and embedding theorems. In \Cref{Sec:Form} we transform the model in order to make it amenable to the subsequent analysis.  We then equip the model with suitable initial and boundary conditions and we make use of the associated Maxwellian to rescale the Navier--Stokes--Fokker--Planck system. In \Cref{Sec:Analysis} we finally state and prove a theorem asserting the existence of large-data global-in-time weak solutions to the model with a time-fractional derivative of order $\alpha \in (\tfrac12,1)$. 


		\section{Modeling of the time-fractional Fokker--Planck equation} \label{Sec:Derivation}
	


%The time-fractional Fokker--Planck model can be derived by utilizing Langevin equations, we refer to Appendix \ref{App:Derivation} at the end of the article for the details on the equation's derivation.
%Historically, three years after Einstein's publication on Brownian motion, the French scientist Paul Langevin published his study \cite{langevin1908theory} on Brownian motion, which obtained the identical results utilizing an entirely new mathematical framework. In fact, Langevin's method is far closer to the conventional physics approach: To account for the unpredictable “kicks” that the Brownian particle receives from smaller fluid particles, one alters Newton's equation of motion by adding a randomly fluctuating force. The resulting equation is what we would identify as a stochastic differential equation in the present day. In fact, it turns out that its moment is governed by the Fokker--Planck equation.


 Let $\Omega \subset \R^d$, $d\in \mathbb{N}$, be a Lipschitz domain and $T<\infty$ a fixed final time. Shortly, we denote the time-space domain by $\Omega_T=\Omega \times (0,T)$. Let $\psi:\Omega_T \to \R$ denote a probability density function that represents the probability at a time $t$ of finding the center of mass of a particle in the volume element $x+\d x$. 


The time-fractional Fokker--Planck model with space-time dependent force can be derived by utilizing the Langevin equations, see \cite{magdziarz2008equivalence,magdziarz2009stochastic}, and the model reads
\begin{equation} \label{Eq:DerivFP}
\pt \psi(x,t)- D \Delta \Ptb \psi(x,t)+\div\!\left(F(x,t) \Ptb \psi(x,t) \right)  = 0.\end{equation}
Here, $F:\Omega_T  \to \R^d$ denotes the space-time dependent external force and $D$ the diffusion coefficient. In contrast to the typical model of integer-order, the fractional derivative in the sense of Riemann--Liouville is introduced, which is defined by
$$\Ptb u(t) =\frac{1}{\Gamma(\alpha)} \ddt \int_0^t \frac{u(s)}{(t-s)^{1-\alpha}} \ds, $$
where $\Gamma$ denotes Euler's Gamma function. We introduce the singular kernel function $g_\alpha(t)=t^{\alpha-1}/\Gamma(\alpha)$ and therefore, we can rewrite the fractional derivative with the convolution operator as $$\Ptb u =\pt (\ga*u).$$ In the limit case of $\alpha=1$, the model is reduced to the standard Fokker--Planck equation.
This time-fractional model has been studied in the previous works \cite{huang2020new,le2016numerical,le2018semidiscrete,le2021alpha,mclean2021uniform,mustapha2022second,pinto2017numerical,yan2019finite} with regard to numerical methods and in \cite{le2019existence} for the existence of mild and classical solutions. 

We note that the fractional derivative in the sense of Riemann--Liouville appears naturally in the equation's derivation, see \cite{magdziarz2008equivalence}. However, the fractional derivative in the sense of Caputo would be preferable considering our variational approach to time-fractional partial differential equations and the involved analytical machinery. The Caputo derivative of order $\alpha$ is denoted by $\pta$ and it reads 
\begin{equation} \label{Eq:Caputo} \pta u= \Pta (u-u_0).
\end{equation} 
Here, $u_0$ is the initial of the underlying system, which shall fulfill $$\big(\gb*(u-u_0)\big)(0)=0$$ in the case that $u$ is not continuous. 

If the force is time-independent, we could simply convolve the time-fractional Fokker--Planck equation \eqref{Eq:DerivFP} with the singular kernel function $\gb$ and exploit the properties $\gb*\Ptb u=u$ and $\pta u=\gb*\pt u$, see below in Section \ref{Sec:Prelim}, to obtain the time-fractional equation
\begin{equation} \label{Eq:DerivFP3}
\pta \psi(x,t)-D \Delta \psi(x,t) +\div \big(F(x) \psi(x,t) \big)  =0,\end{equation}
which would be more accessible for analytical and numerical methods. However, we cannot simply exclude the relevant cases of time-dependent forces. In such cases, one would require a product rule for fractional derivatives to write $F\Ptb \psi$ as $\Ptb(F\psi)-\Ptb F \psi$.  However, this is not correct for fractional derivatives, as it can be already seen from the example $\psi=F=1$. Then it holds 
$$F\Ptb \psi = \ga \neq 0 = \ga-\ga =\Ptb(F\psi)-\Ptb F \psi.$$
There is a fractional version of the Leibniz rule that requires two smooth functions $f,g$ and reads \cite[Theorem 2.18]{diethelm2010analysis}
$$\Pta(fg)=f \Pta g + \sum_{k=1}^\infty \binom{\alpha}{k} \pt^k f \cdot (g_{1-k+\alpha}*g).$$
We can already see the issue of this formula. It requires smooth functions, and it turns out that there is an infinite sum on the right-hand side. Let us assume that $F$ and $\psi$ are smooth. Then we want to bring the fractional derivative in front of $F\psi$ by the formula
$$F \Ptb \psi = \Ptb(F\psi) - \sum_{k=1}^\infty \binom{1-\alpha}{k} \pt^k F \cdot (g_{2-k-\alpha}*\psi).$$
Afterward, we convolve the system with $\gb$ and obtain the system
\begin{equation} \label{Eq:ModelWrong} \begin{aligned}
&\pta \psi(x,t)-D \Delta \psi(x,t) +\div \big(F(t,x) \psi(x,t) \big)  \\&=\sum_{k=1}^\infty \binom{1-\alpha}{k} \gb*\big(\pt^k F \cdot (g_{2-k-\alpha}*\psi)\big),
\end{aligned}\end{equation}
There have been several published articles that studied this model but neglecting the complete right-hand side. This is also the reason it is claimed in \cite{heinsalu2007use} that such a model (with neglecting the right-hand side) is “physically defeasible” and its solution “does not correspond to a physical stochastic process”.  In the case that $F$ is affine linear in $t$, i.e. $F(t,x)=a(x)+b(x)t$,  it yields 
\begin{equation*} \begin{aligned}
&\pta \psi(x,t)-D \Delta \psi(x,t) +\div \big(F(t,x) \psi(x,t) \big)  \\&=(1-\alpha) \cdot b(x) \cdot (g_{2-2\alpha}* \psi)(t)
\end{aligned}\end{equation*}
 We would rather not consider infinitely many terms on the right-hand side of the PDE for a general $F$ and therefore, we instead 
exploit the definition \eqref{Eq:Caputo} of the Caputo derivative to obtain
$$\Ptb u(t) = \ptb u(t) + \Ptb u_0=\ptb u(t)+u_0g_{\alpha}(t),$$
and rewrite the time-fractional Fokker--Planck equation \eqref{Eq:DerivFP} as follows:
\begin{equation} \label{Eq:DerivFP2}
\begin{aligned}
&\pt \psi(x,t)-D \Delta \ptb \psi(x,t) +\div\big(F(x,t) \ptb \psi(x,t) \big) \\ &=\ga D\Delta  \psi_0 - \ga \div(F\psi_0).
\end{aligned} \end{equation}
We consider an initial condition $\psi_0 \in H_0^1(\Omega)$ and therefore, it holds that the right-hand side has the regularity $L^p(0,T;H^{-1}(\Omega))$ with $p<1/(1-\alpha).$
%We follow this approach since, in our opinion, it is more accessible and easier to understand in which way the fractional derivative enters than the CTRW approach
%We translate the $(x,q)$-coordinates to the center of mass by defining the new function $\psi(x,q,t):=\tilde\psi(x-\tfrac12 q,x+\tfrac12 q,t)$. Together with the definition of $b$, see \cref{Def:Vectorb}, we obtain
%\begin{equation} \label{Eq:Langevin3} \begin{aligned}&\pt \psi(x,q,t) + \div \bigg(\frac{u(x-\tfrac12q,t)+u(x+\tfrac12q,t)}{2} \ptb\psi(x,q,t)\bigg)\\ &\quad+ \div\bigg(\big(u(x+\tfrac12q,t)-u(x-\tfrac12q,t)\big) \ptb\psi(x,q,t) - \frac{2F(q)}{\zeta }  \ptb\psi(x,q,t)\bigg) \\ &= \frac{k_B\mu_T}{2\zeta } \Delta_x \ptb\psi(x,q,t)+ \frac{2k_B\mu_T}{\zeta } \Delta_x \ptb\psi(x,q,t). \end{aligned} \end{equation}
%We make an assumption of local homogeneity, i.e., the velocity's variation over the microscopic length scale of a single dumbbell is assumed to be small. Consequently, the arithmetic mean $\big(u(x-\frac12q,t)+u(x+\frac12q,t)\big)/2$ can be approximated by $u(x,t)$, which is relevant for the second term on the left-hand side of \cref{Eq:Langevin3}.  In case of the third term, we use Taylor's expansion to obtain 
%\begin{equation} \begin{aligned} u(x+\tfrac12 q,t)-u(x-\tfrac12 q,t) &= \nabla u(x,t) q + \mathcal{O}(|q|^3) \\ &=\Big(\sigma\big(u(x,t)\big) + \omega\big(u(x,t)\big)  \Big)q + \mathcal{O}(|q|^3),
	%%	\end{aligned}
	%\label{Eq:ApproxTaylor}
	%\end{equation}
%where we have further split the gradient of $u$ into its symmetric and antisymmetic part, which are defined as follows:
% \begin{equation} \label{Eq:Omega} \sigma(u)=\frac{\nabla u + (\nabla u)^\top}{2}, \qquad \omega(u)=\frac{\nabla u - (\nabla u)^\top}{2}.
 %	\end{equation}
%In the approximation \cref{Eq:ApproxTaylor}, we omit the $\mathcal{O}(|q|^3)$ term and further, we also omit the symmetric part of $u$ for analytical simplicity, i.e., we consider the corotational case. Thus, we obtain the time-fractional PDE
%\begin{equation} \label{Eq:FokkerDim}\begin{aligned}\pt \psi + \div(u \ptb\psi) + \div\Big(\omega( u) q \ptb\psi - \frac{2F(q)}{\zeta }  \ptb\psi\Big) \\ = \frac{k_B \mu_T}{2\zeta } \Delta_x \ptb\psi + \frac{2k_B\mu_T}{\zeta } \Delta_x \ptb\psi.\end{aligned} \end{equation}
%Lastly, we put the equation into its nondimensionalized form. Therefore, we define quantities
%$$q=l_0 \hat q$$
%with microscopic length scale %$l_0=\sqrt{k_B \mu_T/H}$. Further, we introduce the nondimensional Weissenberg number $\lambda=\zeta  U_0/(4HL_0)$ that reflects the ratio of the microscopic to the macroscopic time-scale.
%We multiply \cref{Eq:FokkerDim} by $L_0/U_0$ and obtain (omitting the hats over the symbols for the ease of readability)
%
	%\begin{aligned} 
 %&\pt \psi =\tfrac{1}{2\lambda} %\div(\nabla\ptb \psi + U'q \ptb\psi),
%\end{aligned}\end{equation} 
%where we have further defined the parameter $\eps:=l_0^2/(8\lambda L_0^2)$. %
%Finally, the Navier--Stokes equations governing the velocity $u$ are derived in a standard manner from the conservation of momentum and mass equations, see \cite{temam2001navier}. We note that polymeric fluids are non-Newtonian fluids and the polymer molecules contribute
%a symmetric extra stress tensor $\tau(x,t)$ to the total stress tensor while the equations for the conservation
%of linear momentum and mass remain the same. %
%Because the continuum mechanical “macroscopic” equations of incompressible fluid flow are coupled to a “microscopic” model, we call the polymer models under consideration microscopic–macroscopic-type models. Here, the microscropic equations are governed by the time-fractional Fokker--Planck equation describing the statistical properties of particles in the continuum. We begin by presenting these equations and collecting the relevant assumptions on the various parameters featuring in the model.
%We generalize the time-fractional Fokker--Planck equation   by assuming a space-time varying diffusion function $D:\Omega \times (0,T) \to \R^d$ and an additional force $f:\Omega \times (0,T) \to \R$ and, therefore, we study in this work the following time-fractional advection-diffusion equation:
%\begin{equation} \label{Def:FP} \begin{aligned}
%	&\pt \psi -  \div(D(x,t)\nabla  \ptb  \psi) + \div (F(x,t) \ptb  \psi)  \\  &=f(x,t) + \ga \div(D\nabla \psi_0) - \ga \div(F\psi_0).\end{aligned} \end{equation}
We equip this equation with the homogeneous Dirichlet boundary condition $\psi=0$ on $\p\Omega$. However, our analytical results also hold for no-flux boundary conditions (i.e. homogeneous Neumann). Moreover, the system is equipped
with the initial condition $\psi(0) = \psi^0 \geq 0$ in  $\Omega$. %Mathematically, it is enough to demand $\psi_0 \in H_0^1(\Omega)$ to show our desired results on the well-posedness of weak solutions. 
Physically, $\psi^0$ is a given probability density function, i.e.,  it is nonnegative function and satisfies $\int_\Omega \psi^0(x) \dx =1$ (however, we do not need to assume such properties in our well-posedness theorem below).   
Integrating the time-fractional Fokker--Planck equation  in $\Omega$  and employing integration by parts, we find
$\ddt \int_{\Omega} \psi(x,t) \dx =0.$ This implies then $\int_{\Omega} \psi(x,t) \dx = 1$  for almost all $t$.






	\section{Mathematical preliminaries} \label{Sec:Prelim}
In this section, we introduce some useful definitions and results regarding the fractional derivative in the sense of Riemann--Liouville and recall the Aubin--Lions lemma, which is a key result featuring in proofs of existence of weak solutions to nonlinear PDEs based on compactness arguments. %

For a Hilbert space $H$ with inner product $(\cdot,\cdot)_H$ and norm $\|\cdot\|_H$, we shall denote the duality pairing between $H$ and its dual space $H'$ by $\langle \cdot,\cdot\rangle_H$. We shall denote the inner product on the Bochner space $L^2(0,T;H)$ by $(\cdot,\cdot)_{L^2H}$, and we shall write $(\cdot,\cdot)_{L^2_tH}$ when in this inner product the temporal interval of integration is $(0,t)$ for some $t \in (0,T)$ rather than the complete interval $(0,T)$, i.e., $$(u,v)_{L^2_tH}:=\int_0^t (u(s),v(s))_H \, \text{d}s \qquad \forall\, u,v \in L^2(0,T;H).$$
The norm induced by this inner product will be denoted by $\|\cdot\|_{L^2_t H}$.



\subsection{Riemann--Liouville kernels}
The Riemann--Liouville kernel function $g_\alpha$ of order $\alpha$ is defined by $g_\alpha(t):=t^{\alpha-1}/\Gamma(\alpha)$, $t \in (0,T)$, for $\alpha > 0$ and $g_0(t):=\delta_0(t)$ (the Dirac distribution concentrated at $0$) for $\alpha=0$. We observe that  $g_\alpha \in L^p(0,T)$ for any $\alpha\in (1-1/p,1)$ and $p \in [1,\infty)$, and the kernel function satisfies the following semigroup property; see \cite[Theorem 2.4]{diethelm2010analysis}:
\begin{equation} \label{Eq:Semigroup}
	\ga * g_\beta = g_{\alpha+\beta} \qquad \forall\, \alpha,\beta \geq 0.
\end{equation} 
%This can be proved as follows: One applies Fubini's theorem to interchange the order of integration 
%$$\begin{aligned}(\ga * g_\beta *u)(t) &=\frac{1}{\Gamma(\alpha)\Gamma(\beta)} \int_0^t (t-s)^{\alpha-1} \int_0^s (s-\tau)^{\beta-1} u(\tau) \dd \tau \dd s  \\
%	&=\frac{1}{\Gamma(\alpha)\Gamma(\beta)} \int_0^t u(\tau) \int_\tau^t (t-s)^{\alpha-1} (s-\tau)^{\beta-1}   \dd s \dd \tau,
%\end{aligned}$$
%and the substitution $s=\tau+\sigma(t-\tau)$ then yields
%$$\begin{aligned}(\ga * g_\beta *u)(t) 
%	&=\frac{1}{\Gamma(\alpha)\Gamma(\beta)} \int_0^t u(\tau) (t-\tau)^{\alpha+\beta-1} \int_0^1 (1-\sigma)^{\alpha-1} \sigma^{\beta}   \dd \sigma \dd \tau.
%\end{aligned}$$
%Lastly, we observe using the fundamental property of the Gamma function that the second integral is equal to $\Gamma(\alpha)\Gamma(\beta)/\Gamma(\alpha+\beta)$, see \cite[Theorem D.6]{diethelm2010analysis}, from which we deduce the desired semigroup property \cref{Eq:Semigroup} of $g_\alpha$.

We note that when $\alpha \in (0,1)$, one can bound the $L^p(0,t)$-norm of a function $u:(0,T) \to \R$ by its convolution with $\ga$ as follows: for any $t \in (0,T]$, we have that
\begin{equation}\begin{aligned} \|u\|_{L^p(0,t)}^p := \int_0^t |u(s)|^p \ds  &\leq t^{1-\alpha} \int_0^t (t-s)^{\alpha-1} |u(s)|^p \ds \\ &\leq T^{1-\alpha} \Gamma(\alpha) \big(\ga * |u|^p\big)(t).	
\end{aligned} 
\label{Eq:KernelNorm}
\end{equation}
This implies that the space $$L^p_\alpha(0,T):=\big\{u:(0,T) \to \R:\sup_{t \in (0,T)} (\ga*|u|^p)(t) < \infty \big\},$$ is indeed a subspace of $L^p(0,T)$.
%Further, this estimate can be generalized for a nonnegative function $u:(0,T) \to \R_{\geq 0}$ and for $0<\beta<\alpha<1$ in the following way:
%$$(\ga * u)(t)=\frac{1}{\Gamma(\alpha)} \int_0^t (t-s)^{\beta-1} \frac{(t-s)^{\alpha-1}}{(t-s)^{\beta-1}} u(s) \ds \leq \frac{T^{\alpha-\beta}\Gamma(\beta)}{\Gamma(\alpha)} (g_\beta * u)(t).$$
If the order $\alpha$ of the kernel function $g_\alpha$  is larger than 1, then one can exploit the semigroup property of the kernel and apply Young's convolution inequality (cf. Lemma 3.2 in \cite{Oparnica}) as follows:
$$(g_{1+\alpha}*u)(t)=(g_1*\ga*u)(t)=\int_0^t (\ga*u)(s) \ds \leq \|\ga\|_{L^1(0,t)} \|u\|_{L^1(0,t)},$$
for any $u \in L^1(0,T)$ and any $t \in (0,T]$.


\subsection{Time-fractional derivative} 
We can rewrite the definition of the Riemann--Liouville derivative stated in \cref{Eq:RL} in a compact form by using the convolution operator $*$ as $\pta w=\pt (\gb * w)$.
We refer to the classical textbooks \cite{diethelm2010analysis,baleanu2012fractional} and the newer monographs \cite{jin2021fractional,chen2022fractional} regarding fractional calculus and fractional differential equations.

 We define the fractional Riemann--Liouville--Bochner space   for $\alpha \in (0,1)$ and $p \in [1,\infty)$ on $(0,T)$ with values in $H$ by $$\W^{\alpha,p}(0,T;H):=\big\{u \in L^p(0,T;H) : \gb * u \in W^{1,p}(0,T;H)\big\}.$$
Here, the convolution $\ast$ is of course understood to be with respect to the temporal variable $t \in (0,T)$. In the limit, 
when $\alpha=1$, we have that $g_{1-\alpha} = g_0=\delta$, and then $$\W^{1,p}(0,T;H):=W^{1,p}(0,T;H):=\big\{u \in L^p(0,T;H) : \pt u \in L^p(0,T;H)\big\}.$$
However for $0 < \alpha < 1$,
the Riemann--Liouville space $\W^{\alpha,p}(0,T;H)$ differs from the fractional-order Sobolev--Bochner space
$$W^{\alpha,p}(0,T;H):=\Big\{u \in L^p(0,T;H) : (s,t) \mapsto \tfrac{\|u(t)-u(s)\|_H}{|t-s|^{\alpha+1/p}} \in L^{p}((0,T)\times(0,T))\Big\},$$
which can be confirmed by noting that the function $g_\alpha$ is an element of $\W^{\alpha,p}(0,T):=\W^{\alpha,p}(0,T;\R)$ for $\alpha \in (1-\tfrac{1}{p},1)$ but not of $W^{\alpha,p}(0,T)$; see \cite[Proposition 3.13]{carbotti2021note}. 
%Therefore, we are not able to apply classical results such as embedding theorems for Sobolev--Bochner spaces.

\begin{remark}  
Even though the space $\W^{\alpha,p}(0,T)$ is not a subspace of the Sobolev--Slobodecki\u{\i} space $W^{\alpha,p}(0,T)$, it is nevertheless continuously embedded into $C([0,T])$, the space of uniformly continuous functions defined on $[0,T]$, for $\alpha \in (1- \frac{1}{p},1]$ and $p \in [1,\infty)$; see, \cite[Remark 6.2]{carbotti2021note}. 
\end{remark}

\begin{comment}
Therefore, small values of $\alpha$ have to be studied carefully.
Further, we observe that $\ga$ does not belong to $L^p(0,T)$ for $\alpha \in (0,1-\tfrac{1}{p}]$ (e.g., $\ga \notin L^2(0,T)$ for $\alpha \in (0,\frac12]$) and therefore, we find that $$(\gb*\phi)(0)=0 \qquad \forall\, \phi \in \W^{\alpha,p}(0,T;H),~ \alpha \in (0,1-\tfrac{1}{p}],$$
by the inverse convolution property  \cref{Eq:InverseConvolution}. However, this might contradict a given nontrivial initial condition $\phi^0$. E.g., for $\phi \in \W^{\alpha,2}(0,T;H):=\W^{\alpha,2}(0,T;H)$ it has to hold that $(\gb*\phi)(0)=0$ for $\alpha\in (0,\tfrac12]$ and therefore, PDE solutions with this regularity are only well-posed for $\phi^0=0$. Such an issue can be avoided by studying PDEs of the form $\pta(\phi-\phi^0)=f(\phi)$ and considering instead the regularity of $\phi-\phi^0$, i.e., $\phi \in \W^{\alpha,2}_{\phi^0}(0,T;H)$. However, the time-fractional model \cref{Eq:System} in this work is not of this translated form and therefore, we cannot expect that this system is well-posed for non-zero initials. We note that $\psi_0=0$ is physically unreasonable anyway for probability density functions and, moreover, we will naturally observe in the existence's proof below that the restriction $\alpha \geq \frac12$ naturally appears in the energy estimates.
\end{comment}




We also introduce the following Riemann--Liouville space incorporating a homogeneous initial condition at $t=0$, albeit in a somewhat nonstandard manner:
$$\begin{aligned}
\W^{\alpha,p}_{0}(0,T;H)&:=\big\{u \in \W^{\alpha,p}(0,T;H) : (\gb*u)(0)=0 \big\}.
%\W^{\alpha,p}_{u^0}(0,T;H)&:=\big\{u \in L^p(0,T;H) : u-u^0 \in \W_0^{\alpha,p}(0,T;H) \big\}.
\end{aligned}$$ 
We note that the function $\gb*u:[0,T] \to H$ has a well-defined trace at $t=0$ (even when the function $u$ itself might not have one) thanks to the continuous embedding
$$\gb*u \in W^{1,p}(0,T;H) \hookrightarrow AC([0,T];H).$$
For a given element $z \in H$, the convolution $\gb*z$ should be understood to mean the function $t \mapsto (\gb*g_1)(t) z \in H$; recall that $g_1(t)\equiv 1$ for all $t\geq 0$. Thus, $z \in H$ is in this context now thought of as the mapping $t \mapsto g_1(t)z \in \mathcal{W}^{\alpha,p}(0,T;H)$,  for $\alpha \in (0,1)$, $p \in [1,\infty)$ and $0<\alpha p < 1$, or if $\alpha =1$ and $p \in [1,\infty)$. 
Thanks to the semigroup property \eqref{Eq:Semigroup} we then have that $$t\mapsto (\gb*z)(t)=z\,g_{2-\alpha}(t)=\frac{z}{\Gamma(2-\alpha)}  t^{1-\alpha} \in C([0,T];H)$$ for any $\alpha \in [0,1]$. Thus, for $\alpha \in (0,1]$, $p \in [1,\infty)$ and $z \in H$ we define the following `translated' Riemann--Liouville space:
\begin{equation} \label{Eq:RLSpaceU0}
    \W^{\alpha,p}_{z}(0,T;H):=\big\{u \in L^p(0,T;H) : u-z \in \W^{\alpha,p}(0,T;H), ~ (g_{1-\alpha}*u)(0)=0 \big\}.
\end{equation}
Note that if $\alpha \in (0,1)$, $p \in [1,\infty)$ and $0<\alpha p<1$, or if $\alpha=1$ and $p \in [1,\infty)$, then $u-z \in \W^{\alpha,p}(0,T;H)$ if, and only if $u \in \W^{\alpha,p}_0(0,T;H)$, and therefore, for such $\alpha$ and $p$ we have that $\W^{\alpha,p}_{z}(0,T;H) = \W^{\alpha,p}_0(0,T;H)$ irrespective of the choice of $z \in H$.

Next, we state an inverse convolution (or deconvolution) property. Its name stems from the fact that convolution with the kernel $\ga$ acts as an inverse mapping on the operator of taking $\alpha$-th fractional derivative, up to a term that involves the initial value at $t=0$.
%
\begin{lemma}[Inverse convolution] Let $\alpha \in (0,1]$ and $p\in [1,\infty)$. Suppose further that $H$ is a Hilbert space and $z \in H$. Then, for any $t \in (0,T)$, we have the following equalities:
\begin{align} 	\label{Eq:InverseConvolution1}
(\ga * \pta u)(t) &= u(t) - (\gb*u)(0)\ga(t) \quad &&\forall\, u \in \W^{\alpha,p}(0,T;H), \\ 	\label{Eq:InverseConvolution}
		(\ga* \pta u)(t)    &=u(t)  &&\forall\, u \in \W_{z}^{\alpha,p}(0,T;H). \end{align} 
\end{lemma}
\begin{proof}
	We start with the proof of the equality \cref{Eq:InverseConvolution1}.
Recall that for any function $u \in \W^{\alpha,p}(0,T;H)$ we have $\gb*u \in  AC([0,T];H)$, and the fundamental theorem of calculus for absolutely continuous functions therefore yields, for any $t\in [0,T]$,
$$(\gb *u)(t) - (\gb*u)(0) = \int_0^t \partial_s (\gb * u)(s) \ds=(g_1 * \pta u)(t).$$
We convolve this equality with the kernel $\ga$ and make use of the semigroup property \cref{Eq:Semigroup} to obtain
$$(g_1*u)(t) - (\gb*u)(0) g_{1+\alpha}(t)=g_{1+\alpha}*\pta u,$$
where we have used that  $g_\alpha*1=g_\alpha*g_1=g_{1+\alpha}$, because $\alpha \Gamma(\alpha) = \Gamma(1+\alpha)$.
Next, we differentiate this equality in $t$ and observe that $\pt (g_1*u)=u$, $\pt g_{1+\alpha}=g_\alpha$, and $\pt (g_{1+\alpha}*v)=g_\alpha*v$, which yields \cref{Eq:InverseConvolution1}.
We finally note that  \cref{Eq:InverseConvolution} follows trivially from \cref{Eq:InverseConvolution1} and \cref{Eq:RLSpaceU0}.
%We consider an element $u \in \W_{u^0}^{\alpha,p}(0,T;H)$, i.e., there exists an element $v \in \W_{0}^{\alpha,p}(0,T;H)$ with $u-u^0=v$ and using $v$ in \cref{Eq:InverseConvolution1}, we obtain $\ga*\pta v=v$, i.e.,
%\begin{equation*} \begin{aligned}\ga* \pta (u-u^0)   &= u-u^0 &&\forall\, u \in \W_{u^0}^{\alpha,p}(0,T;H).
	%\end{aligned} \end{equation*} 
%Moreover, we can split the left-hand side thanks to the linearity of the fractional derivative and obtain 
 %\begin{equation*} \begin{aligned}
	%	\ga* \pta u    &= u-u^0  + \ga * \pta u^0 =u  &&\forall\, u \in \W_{u^0}^{\alpha,p}(0,T;H),
	%\end{aligned} \end{equation*} 
	%where we have used that   $\ga * \pta 1 = \ga*\gb =1$ thanks to the semigroup property  \cref{Eq:Semigroup}.
 \end{proof}
 
The following result is a direct consequence of the interaction between fractional derivatives and kernel functions.
\begin{corollary} The following identities hold:
\begin{equation} \label{Eq:DerivativeofKernel}  \begin{aligned}
	\pta (\ga * u ) &=\pt ( \gb * \ga * u) = \pt (1*u) = u &&\forall\, u \in L^1(0,T;H), \\
	\ptb \pta u &=  \pt (\ga * \pta u) = \pt u &&\forall\, u \in  W_{0}^{1,1}(0,T;H).
\end{aligned}
\end{equation}
\end{corollary}




%However, in our setting of the time-fractional Navier--Stokes--Fokker--Planck system, we have already seen that the Riemann--Liouville derivative appears on the left-hand side without the translation of an initial value. This already explains intuitively the restriction on the values of $\alpha$ in the theorem of the system's well-posedness, see \cref{Thm:WellPosedness} below.


%As in the integer-order setting, there are continuous and compact embedding results for Riemann--Liouville spaces.
We shall require the following special case of the classical Aubin--Lions lemma; see \cite{simon1986compact}. Suppose that the Hilbert spaces $V,H,Z$ form a Gelfand triple $V \com H \con Z$. then, the following classical compact embeddings hold: 
\begin{equation} \begin{aligned} \label{Eq:aubin} 
W^{1,1}(0,T;Z) \cap L^p(0,T;V) &\com L^{p}(0,T;H), &&p \in [1,\infty), \\
W^{1,r}(0,T;Z) \cap L^\infty(0,T;V) &\com C([0,T];H), && r \in (1,\infty);
\end{aligned}\end{equation} 
see \cite{simon1986compact}. Several fractional counterparts of the Aubin--Lions lemma have been proposed; see \cite{ouedjedi2019galerkin,wittbold2020bounded,li2018some}. We make use of  the following result; see \cite[Corollary 3.2]{ouedjedi2019galerkin}:
\begin{equation*} \begin{aligned} %
\W^{\alpha,1}(0,T;Z) \cap L^p(0,T;V) &\com L^r(0,T;H), &&p \in (1,\infty), \quad r \in [1,p), \quad \alpha \in (0,1).
\end{aligned}\end{equation*} 
The proof can be easily adapted to the limit case $r=p$ if the $\alpha$-th fractional derivative is in a better space than $L^1(0,T;Z)$. This is done for Caputo derivatives in \cite{li2018some}. In fact, we obtain
%$r \in (\frac{p}{1+\alpha p},\infty) \cap [1,\infty)$.  In the spacial case when $p=2$ and $\alpha \in (\frac12,1]$ it yields
\begin{equation} \begin{aligned} \label{Eq:aubinfractional2} %
		\W^{\alpha,r}(0,T;Z) \cap L^p(0,T;V) &\com L^p(0,T;H), &&r\in (1,\infty), \quad \alpha \in (0,1).
\end{aligned}\end{equation} 

\begin{comment}
Next, we require a Gronwall-type inequality that allows an additional nonnegative factor $b \in L^1(0,T)$ in the integrand on the right-hand side of the inequality. Particularly, this function is only assumed to be integrable, and it is allowed to degenerate.
\begin{lemma}[Gronwall, cf. {\cite[Lemma II.4.10]{boyer2012mathematical}}] \label{Lem:Gron4}
    Let $C_1,C_2$ be  nonnegative constants and let $b\in L^1(0,T)$ be nonnegative. If the  function $u \in L^\infty(0,T)$  satisfies the inequality
    $$u(t) \leq C_1+C_2 \int_0^t b(s) u(s) \, \textup{d}s \qquad \text{for a.a. } t \in (0,T], $$
    then 
    $$u(t) \leq C_1 \textup{exp}\Big(C_2\int_0^t  b(s) \, \textup{d}s\Big) \qquad \text{for a.a. } t \in (0,T]. $$
\end{lemma}
\end{comment}


%\subsection{Fractional chain inequality}
The classical chain rule does not hold for fractional derivatives, but one can use the following inequality as a remedy; see \cite[Theorem 2.1]{vergara2008lyapunov}:
\begin{equation} \label{Eq:ChainOriginal}  \frac12 \pta \|u\|^2_H +\frac12 \gb(t) \|u\|_H^2 \leq (u,\pta u)_H \quad \forall\, u \in \W_{z}^{\alpha,2}(0,T;H),
\end{equation}
for $z \in H$ and almost all $t \in (0,T)$.
%Here, it has to be assumed that  $\big(\gb*(u-u^0)\big)(0)=0$. 

\begin{comment}
We will see that the initial condition $\psi^0$ to the time-fractional system considered later on does not satisfy $\big(\gb*(\phi-\psi^0)\big)(0)=0$. Instead, we say that the solution satisfies the initial condition if $(\gb*\phi)(0)=\psi^0$. One can transform this into the form from before by noting that
$$0=(\gb*\phi)(0)-\psi^0=(\gb*\phi)(0)-(\gb*\ga \psi^0)(0)=(\gb*(\phi-\ga \psi^0))(0).$$
Next, we derive a fractional chain inequality for such functions so that we can apply such a result later on in the existence proof.

We consider a fixed $u \in \H_{0}^\alpha(0,T;H)$ and we  introduce $v=u+\ga z$. Since in this case  $u^0=0$, this gives $(\gb*v)(0)=z$. 
%We note that $\pta v= \pta u$ and therefore we obtain
%$$\begin{aligned}
	%(v,\pta v)_H  &=(u,\pta %u)_H+(\ga z, \pta u)_H 
	%\\ &\geq \frac12 \pta %\|u\|_H^2 + \frac12 \gb \|u\|_H^2 + \ga  (z,\pta v)_H,
%\end{aligned}$$
Using the fractional chain inequality 
\cref{Eq:ChainOriginal} for $u=v-\ga z$, we trivially find
\begin{equation} \label{Eq:Chain} (v-\ga z,\pta v)_H \geq \frac12 \pta \|v-\ga z\|_H^2.\end{equation}

%We note that $u=\ga * \pta u$ thanks to the  inverse convolution property \cref{Eq:InverseConvolution} and inserting  $u=v-g_\alpha z$ yields $v-g_\alpha z=g_\alpha * \pta v$. Therefore, we can also write, instead of \cref{Eq:Chain},
%\begin{equation*}  (\ga * \pta v,\pta v)_H \geq \frac12 \pta \|\ga*\pta v\|_H^2.\end{equation*}
One might ask oneself if there is a generalized inequality of the form of $(\ga*u,u)_H \geq \frac12 \pta \|\ga * u\|_H^2$ for a sufficiently smooth function $u$, i.e., whether the convolution with $\ga$ is coercive in some sense. We partly answer this question in the next lemma.
\end{comment}
%
%We shall also require the inequalities stated in the next lemma. 

%\begin{lemma} Let $z\in H$ be given. For any $u \in \W_{0}^{\alpha,2}(0,T;H)$ and any $v \in \W^{\alpha,2}(0,T;H)$ with $(\gb * v)(0)=z \in H$ we have the following inequalities:
%\begin{align} \label{Eq:Coercive}   \int_0^t (u,\pta u)_H \ds &\geq \cos(\alpha \pi/2)\|\pt^{\alpha/2} u\|_{L^2_tH}^2, \\
%\int_0^t \! (v\!-\!\ga z,\pta v)_H \ds  &\geq \cos(\alpha \pi/2)  \Big( \tfrac12 \|\pt^{\alpha/2} v\|^2_{L^2_tH}-\tfrac{\Gamma(\alpha-1)}{\Gamma(\alpha/2)^2}g_{\alpha}(t) \|z\|_H^2 \Big).\label{Eq:Coercive2} 
%\end{align}
%\end{lemma}
%\begin{proof}
%By \cite[Lemma 3.1]{mustapha2014well} we have that
%\begin{equation} \label{Eq:Mustapha} \int_0^t (\ga*w,w)_H \ds \geq \cos(\alpha \pi/2) \|g_{\alpha/2} * w\|_{L^2_tH}^2 \qquad \forall\, w \in L^2_t(0,T;H).
%\end{equation}
%Hence, with $w=\pta u$ we obtain the inequality
%$$
%\int_0^t (u,\pta u)_H \ds \geq \cos(\alpha \pi/2) \|g_{\alpha/2} * \pta u \|_{L^2_tH}^2,$$
%where we have used the inverse convolution property $g_\alpha * \pta u=u$, see \cref{Eq:InverseConvolution} with $z=0$ to simplify the left-hand side. On the right-hand side, we can make use of the fact that $\pt^{\alpha}u =\pt^{\alpha/2}\pt^{\alpha/2} u$ and again the inverse convolution property \cref{Eq:InverseConvolution} to deduce that
%$$g_{\alpha/2}*\pt^{\alpha} u = g_{\alpha/2}*\pt^{\alpha/2} \pt^{\alpha/2} u = \pt^{\alpha/2} u.$$
%Therefore, we obtain the first of the inequalities stated in the lemma. We note that we can only split the $\alpha$-th derivative into the composition of two $\frac{\alpha}{2}$-th derivatives if the function $u$ satisfies an initial condition of the form $\big(\gb*u\big)(0)=0$. 
%
%
%For $v \in \W^{\alpha,2}(0,T;H)$ with $(g_{1-\alpha} \ast v)(0)=z \in H$, we define $u:=v-g_\alpha z$, and we find that $u \in \W_0^{\alpha,2}(0,T;H)$ and thus
%\begin{equation} \label{Eq:LemmaIneq} \begin{aligned} \int_0^t (v -\ga z,\pta v)_H \ds 
%%&\geq \cos(\alpha \pi/2) \|g_{\alpha/2} * \pta u \|_{L^2_tH}^2 
%%\\ &=\cos(\alpha \pi/2) \|g_{\alpha/2} * \pt^{\alpha/2} \pt^{\alpha/2} u\|_{L^2_tH}^2 \\ 
%\geq \cos(\alpha \pi/2)\|\pt^{\alpha/2} u\|_{L^2_tH}^2. \end{aligned} \end{equation}
%As, trivially, $(a/\sqrt{2} - \sqrt{2}b)^2 \geq 0$ for all $a, b \in \mathbb{R}$, it follows that  $|a-b|^2 \geq \frac{a^2}{2} - b^2$. Therefore, noting $(\pt^{\alpha/2} g_\alpha)^2= (g_{\alpha/2})^2=\tfrac{\Gamma(\alpha-1)}{\Gamma(\alpha/2)^2} g_{\alpha-1}$ and inserting $u=v-\ga z$ into the right-hand side of  \eqref{Eq:LemmaIneq} yields
%\begin{equation*} 
%\begin{aligned}
%\|\pt^{\alpha/2} u\|_{L^2_tH}^2  &=  \|\pt^{\alpha/2} (v-g_\alpha z)\|_{L^2_tH}^2 \\ 
%&\geq \frac12 \|\pt^{\alpha/2} v\|_{L^2_tH}^2 - \|\pt^{\alpha/2} \ga z\|_{L^2_tH}^2 \\
%&\geq    \frac12 \|\pt^{\alpha/2} v\|^2_{L^2_tH}-\tfrac{\Gamma(\alpha-1)}{\Gamma(\alpha/2)^2} \|z\|_H^2 \int_0^t g_{\alpha-1}(s)\ds
%%&\geq  \big|  \|\pt^{\alpha/2} v\|_{L^2_tH}-\big( \tfrac{\Gamma(\alpha-1)}{\Gamma(\alpha/2)^2} g_{\alpha}(t) \big)^{1/2} \|z\|_H \big|^2 
%\\
%&=    \frac12 \|\pt^{\alpha/2} v\|^2_{L^2_tH}-\tfrac{\Gamma(\alpha-1)}{\Gamma(\alpha/2)^2}g_{\alpha}(t) \|z\|_H^2.
%\end{aligned}
%\end{equation*}
%That completes the proof of the lemma.
%\end{proof}





\section{Model revisited} \label{Sec:Form}

Having summarised the required results from fractional calculus, we revisit the mathematical model that we have derived in \Cref{Sec:Derivation}.
Let us assume for the moment that the solution $\psi$ to the Fokker--Planck equation belongs to $\mathcal{W}^{1-\alpha,p}(0,T;H) \cap C([0,T];H)$ for some $\alpha \in (0,1)$ and a suitable Hilbert space $H$, to be chosen. As $\psi \in C([0,T];H)$, it follows that $\|(g_\alpha \ast \psi)(t)\|_H \leq \frac{t^\alpha}{\Gamma(1+\alpha)}\|\psi\|_{C([0,T];H)}$, and therefore  $(g_\alpha \ast \psi)(0)=0$. Hence, $\psi \in \mathcal{W}^{1-\alpha,p}_0(0,T;H)$. It then follows from \eqref{Eq:InverseConvolution1}, with $\alpha$ replaced by $1-\alpha$ and $u=\psi$ there, that $(g_{1-\alpha}\ast \partial_t^{1-\alpha} \psi)(t) = \psi(t)$ for $t \in (0,T)$. 
Motivated by these properties, we introduce the auxiliary function $\phi$ by
\begin{equation} \label{Eq:Substitute} \phi := \ptb \psi = \pt (\ga * \psi),
\end{equation} 
whereby $\psi=g_{1-\alpha}*\phi$. We then have that $\pt \psi = \pt(g_{1-\alpha} \ast \phi) = \pta \phi$; and, thanks to the assumed continuity of $\psi$ (i.e. $\psi \in C([0,T];H)$) it
makes sense to require attainment of the initial condition $\psi(0) = \psi^0$, i.e. $(g_{1-\alpha} \ast \phi)(0) = \psi^0$. We shall therefore introduce the substitution $\phi:=\partial_t^{1-\alpha} \psi$ in \eqref{Def:FP}, which results in the following system of equations:
\begin{equation} \begin{aligned}
	\pt u + (u \cdot \nablax) u - \nu \Delta_x u + \nablax p - \div_x \tau( \gb * \phi) &=0, \\
	\div_x u &=0, \\
	\pta \phi  + (u \cdot \nablax) \phi + \div_q (\omega(u) q \phi)-\tfrac{1}{2\lambda} \div_q(\nablaq \phi + U'q  \phi)   -\eps  \Delta_x  \phi&=0, \end{aligned} 
\label{Eq:System}
\end{equation}
subject to the initial conditions $u(0)=u^0$ and $(\gb*\phi)(0)=\psi^0$ for a given nonnegative $\psi^0$ that fulfils $\int_D \psi^0 \dq=1$. Furthermore, we equip the system with the following boundary conditions:
\begin{align}\label{eq:neumannbc}
\begin{aligned}
u &=0 \qquad\text{on } \partial \Omega \times (0,T), \\
\left(\tfrac{1}{2\lambda} (\nablaq \phi + U'q \phi)-\omega(u) q \phi \right) \cdot n_{\partial D} &=0 \qquad\text{on } \Omega \times \partial D \times (0,T), \\
\eps \nablax \phi \cdot n_{\partial \Omega} &=0 \qquad\text{on } \partial\Omega \times  D \times (0,T).
\end{aligned}
\end{align}




\subsection{The Maxwellian and Maxwellian-weighted function spaces} \label{Sec:Maxwell}
We introduce the normalized Maxwellian by 
\begin{equation}
	\label{Def:Max} M(q)=\frac{e^{-U(\tfrac12 |q|^2)}}{\int_D e^{-U(\tfrac12 |s|^2)} \dd s}.
\end{equation}
Moreover, we define the (Maxwellian-weighted) Hilbert spaces
$$\begin{alignedat}{5}
	&\mathcal{H}=\{h \in L^2(\Omega;\R^d): \div \, h =0\}, \qquad ~\mathcal{H}_0&&=\{h \in \mathcal{H} : h \cdot n_{\partial \Omega} = 0 \text{ on } \partial \Omega\}, 
	\\ &\mathcal{V}=\{v \in H^1(\Omega;\R^d) : \div \, v = 0\}, \qquad ~ \mathcal{V}_0&&=\{v \in \mathcal{V} : v|_{\partial \Omega} = 0 \text{ on } \partial \Omega\},
	\\  &\mathcal{Y}=L^2(\Omega \times D),  \qquad\quad\,\,\,   \widehat{\mathcal{Y}}= L^2_M(\Omega \times D) &&=\{y \in \mathcal{Y}: \|y\|_{\widehat{\mathcal{Y}}} := \|M^{1/2}y \|_\mathcal{Y}<\infty \}, 
	\\ &\mathcal{X} = H^1(\Omega \times D), \qquad\quad \widehat{\mathcal{X}}= H_M^1(\Omega \times D) &&= \{\phi \in \mathcal{X}: \|\phi\|_\hX <\infty  \},  \\
&\mathcal{Z}=H^1(D;H^1(\Omega)), ~ \widehat{\mathcal{Z}} =H_M^1(D; H^1(\Omega))&&=\{\zeta \in \mathcal{Z}: \|\zeta\|_\hZ <\infty  \}, 
\end{alignedat}$$
where the norms on $\hX$ and $\hZ$ are defined by $\|\phi\|_\hX^2:= \|\phi\|_\hY^2 + \|\nablaq \phi\|_\hY^2 + \|\nablax \phi\|_\hY^2$ and $\|\zeta\|_\hZ^2 :=\|\zeta\|_\hX^2 + \|\nablax \nablaq \zeta\|_\hY^2$. Obviously, $H_M^2(\Omega \times D) \subseteq \hZ$, where $H^2_M(\Omega \times D)$ is the subspace of $H^1_M(\Omega \times D)$ consisting of all functions defined on $\Omega \times D$ whose second (weak) partial derivatives belong to $\hY=L^2_M(\Omega \times D)$.
We refer to \cite{barrett2005existence} regarding theoretical results on these weighted Hilbert spaces. In particular, we have the Gelfand triples
$$\begin{aligned} &\HSV \com \HS \hookrightarrow \HSV', \quad &&\HSV_0 \com \HS_0 \hookrightarrow \HSV_0', \\
	&\mathcal{X} \com \mathcal{Y} \hookrightarrow \mathcal{X}', \quad &&\hX \com \hY \hookrightarrow \hX',
\end{aligned}$$
where $\mathcal{V}'$, $\mathcal{V}'_0$, $\mathcal{X}'$ and $\hX'$ denote the dual space of, respectively, 
$\mathcal{V}$, $\mathcal{V}_0$, $\mathcal{X}$ and $\hX$. 

Using the definition of the normalized Maxwellian $M$, see \cref{Def:Max}, we have that $$M(q)\nablaq M(q)^{-1}=-M(q)^{-1} \nablaq M(q) =\nablaq U(\tfrac12 |q|^2) = U'(\tfrac12 |q|^2) q.$$ We introduce the scaled variable $\hphi=\phi/M$ and  with the formula 
$$M \nablaq \hphi = \nablaq \phi + M \nablaq M^{-1} \phi  = \nablaq \phi + U'q \phi$$
we can rewrite the fractional Fokker--Planck equation in \cref{Eq:System} as
$$ \pta \phi + (u \cdot \nablax) \phi + \div_q \big(\omega(u) q \phi\big) = \tfrac{1}{2\lambda} \div_q(M \nablaq \phim) + \eps \Delta_x \phi.$$
As was indicated earlier, we shall confine ourselves here to considering the corotational model, i.e.,
$\omega(v)=-\omega(v)^{\mathrm{T}}$, $q^{\mathrm{T}} \omega(v) q = 0$; if $\div\, v = 0$ it then follows that 
\begin{equation}\label{Eq:SigmaZero} 
\big(M \hphi \,\omega(v)q,\nablaq \hphi\big)_{\mathcal{Y}} = \frac12 \big(M \omega(v)q, \nablaq \hphi^2\big)_{\mathcal{Y}}=-\frac12 \big(\div_q(M \omega(v) q),\hphi^2\big)_{\mathcal{Y}} = 0; \end{equation}
see \cite{barrett2009numerical,barrett2005existence}.
%
We note in passing that partial integration yields the following equalities: 
\begin{equation} \label{Eq:IntParts} \begin{aligned}
		-2\big(M \omega(u) q \hat\varphi,\nablaq \hphi\big)_{\mathcal{Y}} &=  \big(\nablax(M\hat\varphi \nablaq \hphi)q,u\big)_{\mathcal{Y}} + \big(u\cdot q, \div_x (M\hat\varphi\nablaq \hphi)\big)_{\mathcal{Y}}
		\\ &= \big(M \nablax \hat\varphi (\nablaq \hphi)^{\mathrm{T}} q,u\big)_{\mathcal{Y}} + \big(M \hat\varphi \nablax \nablaq \hphi\, q,u\big)_{\mathcal{Y}} \\ &\quad + \big(u \cdot q,M \nablax \hat\varphi \cdot \nablaq \hphi\big)_{\mathcal{Y}} + \big(u \cdot q,M \hat\varphi \,\div_x \nablaq \hphi\big)_{\mathcal{Y}}.
\end{aligned} \end{equation} 
%

We recall that the stress tensor $\tau(\psi) = \tau_1(\psi) + \tau_2(\psi)$ is of the form  given by \eqref{eq:tau1nd} and \eqref{eq:tau2nd}; i.e., 
%%%%%%%%%%%%%
%
\begin{align}\label{Eq:tau1tau2}\tau^1(\psi)=\gamma\, \C(\psi),\quad \tau^2(\psi)= \gamma \int_{D} \psi \d q  \ I_3,\quad  \C(\psi):= \int_{D} F(q) q^{\mathrm T} \psi \d q,
\end{align}
%
where $\gamma>0$ is a dimensionless constant.


%%%%%%%%%%%%
For $\C(M \hpsi)$, we are in a setting that allows us to deduce the following bound; see also \cite[Eq. (3.7)]{barrett2009numerical}:
\begin{equation} \label{Eq:C}  \begin{aligned}
		\int_\Omega |\C(M \hpsi)|^2 \d x & =
		\int_\Omega \left| \, \int_{D} F(q)  q^{\mathrm T} M \hpsi \d q \, \right|^2 \d x
		\\ &\leq \int_D M  | F(q)  q^{\mathrm T} |^2   \dq \, \int_{\Omega \times D} M|\hpsi|^2 \d(x,q)  \\ & \leq C  \| \hpsi\|_{\hY}^2 \quad \forall\, \hpsi \in \hY. \end{aligned}\end{equation}




	\section{Existence of weak solutions} \label{Sec:Analysis}
In this section, we prove the existence of a weak solution to the time-fractional Navier--Stokes--Fokker--Planck system. We use a Galerkin procedure and discretize the partial differential equations in space and derive suitable energy estimates. We will emphasize the places where the time-fractional derivative comes into play. We shall then pass to the limit in the sequence of Galerkin approximations to deduce the existence of a weak solution. We shall proceed step by step and prove this result through several lemmas. While, for physical reasons, the derivation of the system was in the previous sections discussed in the case of $d=3$ space dimensions, the analysis below applies to both two and three space dimensions. We begin by introducing the concept of a weak solution to the time-fractional corotational Navier--Stokes--Fokker--Planck system under consideration. %\medskip

\begin{definition} \label{Eq:DefWeak} Suppose that $d \in \{2,3\}$.
We call the pair $(u,\hphi)$ a weak solution to the system \cref{Eq:System}, \cref{eq:neumannbc} provided that   
\begin{align*}
u &\in L^\infty(0,T;\mathcal{H}_0) \cap L^2(0,T;\mathcal{V}_0) \cap W^{1,\tfrac{8}{4+d}}(0,T;\mathcal{V}_0'), \\ 
%L^{
\hphi &\in L^2(0,T;\hX), \quad \pta \hphi \in  L^{\tfrac{8}{4+d}}\big(0,T; \hZ' \big),
\end{align*}
satisfies the initial conditions $u(0)=u^0$, $(\gb*\hphi)(0)=\hpsi^0 := \psi^0/M$ and  the variational problems
\begin{align}
		\langle \pt u,v \rangle_{L^{8/(4-d)}\mathcal{V}}  + \big((u \cdot \nablax)u,v\big)_{L^2\mathcal{H}} + \nu (\nablax u,\nablax v)_{L^2\mathcal{H}} && \label{Eq:NS} \\[-.1cm] + k_B \mu_T \big(\C(M \gb *\hphi),\nablax v\big)_{L^2\mathcal{H}} =0, && \notag  \\[.1cm]
		\langle  \pta \hphi,\hat\zeta\rangle_{L^{8/(4-d)} \hZ}  - (u \hphi,\nablax \hat\zeta)_{L^2\hY}+ \frac{1}{2\lambda} (\nablaq \hphi,\nablaq \hat\zeta)_{L^2\hY}  + \eps (\nablax \hphi,\nablax \hat\zeta)_{L^2\hY} && \label{Eq:FP} \\[-.2cm]   + \frac12 (\nablax(\hphi\nablaq \hzeta)q,u)_{L^2\hY} - \frac12 \big(u\cdot q,\div_x (\hphi\nablaq \hzeta)\big)_{L^2\hY} = 0. &&  \notag
\end{align}
for all $v \in L^{8/(4-d)}(0,T;\mathcal{V}_0)$ and $\hat\zeta \in L^{8/(4-d)}(0,T;\hZ)$. In these variational problems and hereafter, for a Hilbert space $H$ and $p \in [1,\infty]$, subscripts of the form $L^pH$ and $L^p_tH$ appearing in the various inner products, norms, and duality pairings, signify $L^p(0,T;H)$ and $L^p(0,t;H)$, respectively. %Similarly, for a reflexive Banach space $B$, $\langle \cdot, \cdot \rangle_{C B}$ denotes the duality pairing between $C([0,T];B)$ and its dual space, 
%$\mathcal{M}(0,T;B')$ the space of all $B'$-valued finitely additive finite signed measures defined on $(0,T)$, which are absolutely continuous with respect to the Lebesgues measure, equipped with the total variation norm over $(0,T)$.
\end{definition} %\medskip


We summarize the assumptions that we require for proving the existence of a weak solution in the sense of \cref{Eq:DefWeak} in Assumption \ref{Ass:WellPosedness} below. %\medskip

\begin{assumption} \label{Ass:WellPosedness}
	Let the following assumptions hold:
	\begin{itemize}
		\item $D \subset \R^d$,  $d\in\{2,3\}$, is a bounded open ball centered at the origin, $\Omega \subseteq \R^d$ is a Lipschitz domain (i.e., bounded, open, connected set in $\R^d$, with a Lipschitz-continuous boundary $\partial\Omega$), and $T<\infty$ is a fixed final time;
  \item $u^0 \in \mathcal{H}_0$, $\hpsi^0 \in \hX$ with $\hpsi^0 \in H^1_M\big(D; H^{1+d/2+\delta}(\Omega)\big)$ for $\delta>0$ arbitrarily small;
  \item $\tau(\psi) =\tau_1(\psi) + \tau_2(\psi)$ is given by \eqref{Eq:tau1tau2}, with the identity matrix $I_3 \in \mathbb{R}^{3 \times 3}$ replaced by the identity matrix $I_d \in \mathbb{R}^{d \times d}$ in the definition of $\tau_2(\psi)$,  and $\C$ satisfies \cref{Eq:C};
    \item $\alpha \in (1/2,1)$;
  \item $k_B,\mu_T, \nu,\lambda,\eps >0$.
	\end{itemize}
\end{assumption} %\medskip

The main result of this paper is the following theorem asserting the existence of global-in-time large-data weak solutions to the time-fractional Navier--Stokes--Fokker--Planck system under consideration. 

%\medskip

\begin{theorem} \label{Thm:WellPosedness}
	Let \cref{Ass:WellPosedness} hold; 
	then, there exists a weak solution $(u,\hphi)$ to the system \cref{Eq:System}, \cref{eq:neumannbc} in the sense of \cref{Eq:DefWeak}.
\end{theorem}%\medskip

In order to prove this theorem, we state several lemmas, which will eventually imply \cref{Thm:WellPosedness}. We begin by constructing a sequence of Galerkin approximations $\{(u_k,\hphi_k)\}_{k=1}^\infty$ to the system of partial differential equations under consideration, resulting in a system of fractional-order ordinary differential equations, which admits a local-in-time solution $(u_k,\hphi_k)$
for each $k \geq 1$ thanks to  standard theory. We then derive an energy estimate for the sequence of Galerkin approximations, which is uniform with respect to $k$; this then implies that, for each $k \geq 1$, the local-in-time solution of the fractional-order system of ordinary differential equations can be extended to the entire time-interval $[0,T]$; it also implies the existence of a weakly/weakly-$*$ convergent subsequence $(u_{k_j},\hphi_{k_j})$. Finally, we pass to the limit $j \to \infty$ and apply a compactness argument to deduce that the limiting pair of functions 
$(u,\hphi)$ is in fact a weak solution to the system of partial differential equations in the sense of \cref{Eq:DefWeak}. The Galerkin method has been applied to various time-fractional PDEs; see, e.g., \cite{fritz2020time,fritz2021sub,fritz2021equivalence,vergara2015optimal}; it has also been applied to Navier--Stokes--Fokker--Planck systems in \cite{knezevic2009heterogeneous,barrett2009numerical,barrett2011finite,barrett2012finite}, with an integer-order Fokker--Planck equation. 

\subsection{Galerkin discretization}
We follow the construction of \cite[Section 2.1]{bulicek2013existence} and conclude by the Hilbert--Schmidt theorem \cite[Lemma A.4]{bulicek2013existence} the existence of a countable set $\{h_j\}_{j=1}^\infty$ of eigenfunctions in $\mathcal{V}_0 \cap H^{1+\frac{d}{2}+\delta}(\Omega)^d$, with $\delta>0$ arbitrarily small, whose linear span is dense in $\mathcal{H}_0$ such that the $h_j$, $j\in \{1,2,\dots\}$, are orthonormal in $\mathcal{H}$ and orthogonal in $H^{1+\frac{d}{2}+\delta}(\Omega)^d$ in the sense that $(h_j,h_i)_{H^{1+\frac{d}{2}+\delta}(\Omega)}=\lambda_j \delta_{i,j}$ for any $i,j \in \{1,2,\ldots\}$ and $\lambda_j>0$ for all $j=1,2,\ldots$. Similarly, we fix a countable set $\{y_j\}_{j=1}^\infty$ in $H^2_M(\Omega \times D)$ that forms an orthogonal system in $\hX$ and an orthonormal system in $\hY$. 
%Consider the high-order elliptic problem of finding a solution tuple $(u,\lambda) \in H^{1+d}(\Omega)^d \times \R$ of
%$$(u,v)_{H^{1+d}(\Omega)} + (\nabla u,\nabla v)_{L^2(\Omega)}  = \lambda (u,v)_{\mathcal{H}} \qquad \forall  v \in W^{1+d}(\Omega)^d.$$
%By a version of the Hilbert--Schmidt theorem, see \cite[Lemma A.4]{bulicek2013existence}, there is a countable set of eigenfunctions $\{h_j\}_{j=1}^\infty$, which are orthogonal in the inner product of $H^{1+d}(\Omega)^d$ and orthonormal in the inner product of $L^2(\Omega)^d$. We note that $H^{1+d}(\Omega)^d$ is continuously embedded into $W^{1,\infty}(\Omega)$
We then define the $k$-dimensional linear spaces
\begin{align*}
	\mathcal{H}_k  :=\text{span}\{ h_1,\dots,h_k\}, \quad 
	\widehat{\mathcal{Y}}_k  :=\text{span}\{ y_1,\dots,y_k\},
	%	Z_K &=\text{span}\{ z_1,\dots,z_k\},
\end{align*}
%where $h_j: \Omega \to \R$ and $y_j : \Lambda \to \R$, $j \in \{1,\dots,k\}$, are the eigenfunctions corresponding to the eigenvalues $\lambda_{j}, \mu_{j} \in \R$ of the following respective problems
%$$\begin{aligned}
	%(h_j,v)_{H^{1+d}(\Omega)} + (\nabla h_j,\nabla v)_{L^2(\Omega)}  &= \lambda_{j} (h_j,v)_{\mathcal{H}} &&\forall  v \in W^{1+d}(\Omega;\R^d),  \\
	%(\nabla y_j,\nabla y)_\hY &= \mu_{j} (y_j,y)_\hY &&\forall  y \in \hY. 
%\end{aligned}$$
%\begin{alignat*}{1}
%	 & \begin{cases} \begin{aligned}
		%			-\Delta h_j          & = \lambda_{h,j} h_j &  & \text{in } \Omega,         \\
		%			\nabla h_j \cdot n & = 0               &  & \text{on } \partial\Omega,
		%		\end{aligned} \end{cases} 
%			  \begin{cases} \begin{aligned}
		%		-\Delta h_j^0          & = \lambda_{h^0,j} h_j^0 &  & \text{in } \Omega,         \\
		%		\nabla h_j^0 \cdot n & = 0             &  & \text{on } \partial\Omega \backslash \p \Omega_D, \\
		%				h_j^0 & = 0               &  & \text{on } \partial\Omega_D, 
		%		\end{aligned} \end{cases} 
%	  \begin{cases} \begin{aligned}
		%			-\Deltala y_j                  & = \lambda_{y,j} y_j &  & \text{in } \Lambda,         \\
		%			\nablala y_j \cdot n_\Lambda & = 0               &  & \text{on } \partial\Lambda_N, \\
		%			y_j & = 0               &  & \text{on } \partial\Lambda_D.
		%		\end{aligned} \end{cases}
%\end{alignat*}
%As stated in \cite{bulicek2013existence}, there is a countable 
%Since both the inverse Stokes and the inverse Neumann--Laplace operator are compact, self-adjoint, injective, positive operators on $\mathcal{H}_0$ and $\hY$, respectively, we conclude by the spectral theorem, see e.g. \cite[12.12 and 12.13]{alt2016linear}, that
%\begin{alignat*}{3}
	%& \{h_j\}_{j \in \mathbb{N}} &  & \text{ is an orthonormal basis in } \mathcal{H}_0 &  & \text{ and orthogonal in } \mathcal{V}_0, \\
	%	& \{y_j\}_{j \in \mathbb{N}} &  & \text{ is an orthonormal basis in } \hY &  & \text{ and orthogonal in } \hX.
%\end{alignat*}
%By the orthonormality of the eigenfunctions, we conclude that ${\cup_{k\in\mathbb{N}}} \mathcal{H}_k$ and ${\cup_{k\in\mathbb{N}}} \hY_k$ are dense in $\mathcal{V}_0$ in $\hX$, respectively. 
and we consider the Galerkin approximations
\begin{equation}\begin{gathered}
		u_k (t) = \sum_{j=1}^k u_k^j(t) h_j,
		\quad \hphi_k (t) = \sum_{j=1}^k \hphi^j_k(t) y_j,
	\end{gathered}
	\label{Eq:GalerkinAnsatzFunctions}
\end{equation}
where 
$u^{j}_k$ and $\hphi^j_k$ are real-valued coefficient functions for all $j \in \{1,\dots,k\}$.
 %Let $h>0$ denote a discretization parameter tending to zero. As in \cite{barrett2009numerical}, we choose finite-dimensional spaces $\hY_k^x \subset W^{1,\infty}(\Omega)$ and $\hY_k^q \subset W^{1,\infty}(D)$ such that it holds
%$$\text{dist}_{W^{1,\infty}(\Omega)} (\eta,\hY_k^x) \to 0, \qquad \text{dist}_{W^{1,\infty}(D)} (\xi,\hY_k^q) \to 0,$$
%as $h\to 0$ for all $\eta \in C^\infty(\overline\Omega)$ and $\xi \in C^\infty(\overline D)$. Moreover, we define the tensor space $\hY_k=\hY_k^x \otimes \hY_k^q \subset W^{1,\infty}(\Omega \times D)$ and note that $\widehat{X}_h \subset X \subset \hX$. 
%Further, we define the finite-dimensional spaces $W_h$, $R_h$ and $\mathcal{H}_k$ such that
%$$\begin{aligned}W_h &\subset \mathcal{H}_0^1(\Omega;\R^d) \cap W^{1,\infty}(\Omega;\R^d), \quad R_h \subset L_0^2(\Omega), \\ \mathcal{H}_k&=\{w_h\in W_h: (\div_x w_h,r_h)_{\mathcal{H}} \,\forall r_h \in R_h \},\end{aligned}$$
%where $\cup_{h>0} W_h$ and $\cup_{h>0} R_h$ are supposed to be dense in $\mathcal{H}_0^1(\Omega;\R^d)$ and $L_0^2(\Omega;\R^d)$, respectively. Further, we assume that for all $v \in V$ there exists a sequence $v_k \in \mathcal{H}_k$ such that $v_k \to v$ in $H^1(\Omega)$ for $h \to 0$. This holds for the typical Galerkin approximation with the availability of an uniform inf-sup condition.
The canonical orthogonal projection onto the finite-dimensional space $\mathcal{H}_k$ is defined by $\Pi_{\mathcal{H}_k}: \mathcal{H} \to \mathcal{H}_k$, $h \mapsto \sum_{j=1}^k (h,h_j)_{\mathcal{H}} h_j$,  and in the same way for $\Pi_{\hY_k}:\hY \to \hY_k$. 
For $h=\sum_{j=1}^\infty (h,h_j)_{\mathcal{H}}h_j$ we have that $$\|h\|^2_{H^{1+\frac{d}{2}+\delta}(\Omega)} = \sum_{j=1}^\infty \lambda_j |(h,h_j)_{\mathcal{H}}|^2, \quad \|\Pi_{\mathcal{H}_k} h\|^2_{H^{1+\frac{d}{2}+\delta}(\Omega)} = \sum_{j=1}^k \lambda_j |(h,h_j)_{\mathcal{H}}|^2,$$
from which we conclude via the Sobolev embedding theorem that, for each $k \geq 1$, 
$$\|\Pi_{\mathcal{H}_k} h\|_{W^{1,\infty}(\Omega)} \leq C\|\Pi_{\mathcal{H}_k} h\|_{H^{1+\frac{d}{2}+\delta}(\Omega)} \leq C\|h\|_{H^{1+\frac{d}{2}+\delta}(\Omega)}.$$


%Thanks to \cite[Theorem 8.1.11]{brenner2008mathematical} and \cite{guzman2009holder}, we have that $\Pi_{\mathcal{H}_k}$ is uniformly $H^1$-stable and $\Pi_{\hY_k}$ is $W^{1,\infty}$-stable.
%Given the initial data $u^0$ and $\psi^0$ from the continuous system, we choose $u_k^0 \in \mathcal{H}_k$ and $\hphi_k^0 \in \hY_k$ such that $u_k^0 = \Pi_{\mathcal{H}_k} u^0$ and $\hpsi_k^0=\Pi_{\hY_k} \hpsi^0$. 

The Galerkin equations read as follows: we wish to find a tuple $(u_k,\hphi_k) \in \mathcal{H}_k \times \hY_k$ for each $k \geq 1$ such that $u_k(0)=u_k^0:= \Pi_{\mathcal{H}_k}u^0$, $(\gb*\hphi_k)(0)=\hpsi_k^0:=\Pi_{\hY_k} \hpsi^0$, and
\begin{align}(\pt u_k,v_k )_{\mathcal{H}}  &+ ((u_k \cdot \nablax)u_k,v_k)_{\mathcal{H}} + \nu (\nablax u_k,\nablax v_k)_{\mathcal{H}}  \label{Eq:NS_dis} \\ 
& + \,k_B \mu_T (\C(M \gb *\hphi_k),\nablax v_k)_{\mathcal{H}} =0, \notag \\
	(\pt (\gb*\hphi_k),\hat\zeta_k)_\hY  &+  
	((u_k \cdot \nablax) \hphi_k, \hzeta_k)_\hY + \frac{1}{2\lambda} (\nablaq \hphi_k,\nablaq \hat\zeta_k)_\hY  \label{Eq:FP_dis} \\ &+ \eps (\nablax \hphi_k,\nabla_x \hat\zeta_k)_\hY- (\omega(u_k) q \hphi_k, \nablaq \hat\zeta_k)_\hY=0, \notag \end{align} for all $v_k \in \mathcal{H}_k$ and $\hat\zeta_k \in \hY_k$.  %\medskip

\begin{lemma} \label{Eq:LemmaExistenceODE}
	Suppose that \cref{Ass:WellPosedness} holds; then, for each $k \geq 1$,
	there exists a local-in-time solution $(u_k,\hphi_k)$ to the Galerkin system \cref{Eq:NS_dis}, \cref{Eq:FP_dis}.
	\end{lemma} %\medskip

\begin{proof}
Let $U_k(t):=(u_k^1(t),\ldots,u_k^k(t))^{\mathrm{T}}$ and $\widehat\Phi_k(t)=(\widehat{\phi}_1^k,\ldots, \widehat{\phi}_k^k)^{\rm T}$. With this notation the Galerkin subsystem \eqref{Eq:NS_dis} becomes an initial-value problem for a system of ordinary differential equations of the form $\ddt U_k  = F(t, U_k, \widehat\Phi_k)$, while, by noting that $\ddt (\gb*\hphi_k)=\pta \hphi_k$, the Galerkin subsystem \eqref{Eq:FP_dis} takes the form of an initial-value problem for a system of fractional-order ordinary differential equations $(\ddt)^\alpha \widehat\Phi_k  = G(t, U_k, \widehat\Phi_k)$. As the functions $F$ and $G$ are continuous with respect to their arguments and locally Lipschitz continuous with respect to their second and third arguments, we can appeal to the generalization of the Cauchy--Lipschitz theorem stated in Theorem 5.1 of \cite{diethelm2010analysis} to deduce the existence of a unique continuous solution, defined on a time interval $[0, T_k]$ where $0<T_k \leq T$, where $U_k$ is, in fact, a continuously differentiable function of $t$ by the classical Cauchy--Lipschitz theorem. 
\end{proof}

\subsubsection{Energy estimates} Next, we derive a $k$-uniform energy estimate, which will allow us to extend, for each $k \geq 1$, the corresponding local-in-time Galerkin solution, whose existence is guaranteed by Lemma \ref{Eq:LemmaExistenceODE}, to the entire time interval $[0,T]$; it will also enable us to extract weakly converging subsequences of Galerkin approximations.  We begin by deriving a bound on the solution to the Galerkin approximation of the Navier--Stokes equation; we shall then derive a bound on the solution to the Galerkin approximation of the Fokker--Planck equation. At the end, we will add the two bounds and apply Gronwall's lemma to obtain a $k$-uniform energy estimate. %\medskip

\begin{lemma} \label{Lem:EstU} Let \cref{Ass:WellPosedness} hold; then the following bound on the Galerkin solution $u_k$, in terms of $u_k^0$ and $\hphi_k$, holds for all $t \in (0,T_k)$:
	\begin{equation} \label{Eq:EnergyU2}
		\frac12 \|u_k(t)\|_{\mathcal{H}}^2  + \frac{\nu}{2} \|\nablax u_k\|_{L^2_t{\mathcal{H}}}^2  \leq \frac12 \|u^0_h\|_{\mathcal{H}}^2 +\frac{ T^{2-2\alpha} k_B^2\mu_T^2}{2\nu (1-\alpha)^2}  \|\hphi_k\|_{L^2_t\hY}^2.
	\end{equation}
\end{lemma} %\medskip

\begin{proof}
We take the test function $v_k=u_k(t)$ in the equation \cref{Eq:NS}, which gives
$$\begin{aligned}
	\frac12 \frac{\dd}{\dd t} \|u_k\|_{\mathcal{H}}^2  + \nu \|\nablax u_k\|_{\mathcal{H}}^2  &= -k_B \mu_T \big(\C(M\gb*\hphi),\nablax u_k\big)_{\mathcal{H}}, 
	\end{aligned}$$
and we can further bound the right-hand side from above to deduce that
$$\frac12 \frac{\dd}{\dd t} \|u_k\|_{\mathcal{H}}^2  + \nu \|\nablax u_k\|_{\mathcal{H}}^2  \leq \frac{\nu}{2} \|\nablax u_k\|_{\mathcal{H}}^2 + \frac{k_B^2\mu_T^2}{2\nu} \|\C(M\gb*\hphi_k)\|_{\mathcal{H}}^2. $$
By bounding the term  $\C(M\gb * \hphi_k)$ as in \cref{Eq:C} we arrive at the inequality
\begin{equation} \begin{aligned}\frac12 \frac{\dd}{\dd t} \|u_k\|_{\mathcal{H}}^2  + \frac{\nu}{2} \|\nablax u_k\|_{\mathcal{H}}^2  &\leq  \frac{k_B^2\mu_T^2}{2\nu} \|\gb*\hphi_k\|_\hY^2. %
\end{aligned} \label{Eq:EnergyU}
\end{equation}
Next, we note that, for any $t \in (0,T_k)$,
%
\begin{align}\label{Eq:EnergyU1} 
\int_0^t \|(\gb*\hphi_k)(s)\|_\hY^2 \ds \leq \int_0^t  \big(\gb * \|\hphi_k\|_\hY\big)^2(s) \ds\leq \|\gb\|_{L^1_t}^2 \|\hphi_k\|_{L^2_t\hY}^2.
\end{align}
This follows by observing that $g_{1-\alpha}$ only depends on the scalar variable $s$, which permits pulling the $\hY$-norm inside of the convolution, followed by applying Young's convolution inequality (cf. Lemma 3.2 in \cite{Oparnica}) in the resulting integrand. 

We then integrate \eqref{Eq:EnergyU} over the interval $[0,t]$ where $t \in (0,T_k)$ and use \eqref{Eq:EnergyU1} to bound the right-hand side of the resulting inequality. Finally we
note that $g_{1-\alpha}$ is integrable on $(0,t)$ and its integral is bounded by $T^{1-\alpha}/(1-\alpha)$. 
This gives \cref{Eq:EnergyU2}.
\end{proof} %%\medskip

Having derived a bound on $u_k$, we move on to the derivation of a bound on $\hphi^k$ by testing the Galerkin system \eqref{Eq:FP_dis}.  %\medskip

\begin{lemma} \label{Lem:EstPhi} Let \cref{Ass:WellPosedness} hold and let $\gamma>0$ be arbitrary but fixed; then, the following bound on the sequence of Galerkin solutions $\{(u_k,\hphi_k)\}_{k=1}^\infty$ holds:
\begin{equation} \label{Est:SolFP} \begin{aligned} & \frac{\gamma}{2} \big(\gb* \|\hphi_k-\init \ga\|_\hY^2\big)(t) + \frac{\gamma \,T^{-\alpha}}{16\, \Gamma(1-\alpha)} \|\hphi_k\|_{L^2_t\hY}^2  +\frac{\gamma}{2\lambda} \|\nablaq \hphi_k\|_{L^2_t\hY}^2 + \gamma \eps \| \nablax \hphi_k\|_{L^2_t \hY}^2  \\
	&\quad \leq C(\alpha,\gamma) \|M^{1/2} \init\|_{H^1(D;W^{1,\infty}(\Omega)) }^2  \int_0^t g_{2\alpha-1}(s) \|u_k(s)\|_{\mathcal{H}}^2 \ds
	+ C(\alpha,\gamma,T)\| \init\|_\hX^2.
\end{aligned}\end{equation}
\end{lemma} %\medskip

\begin{proof}
We note again that,  thanks to the inverse convolution property, see \cref{Eq:InverseConvolution}, $\hphi_k - \init \ga = \ga * \pta \hphi_k$.
We take this function as the test function in the variational Fokker--Planck equation \cref{Eq:FP}, i.e., $\hat\zeta= \hphi_k - \init \ga = \ga * \pta \hphi_k$, which gives  
\begin{equation} \label{Eq:TestingRHS}\begin{aligned} &(\pta \hphi_k,\ga * \pta \hphi_k)_\hY +\frac{1}{2\lambda}  \|\nablaq \hphi_k\|_\hY^2 + \eps \| \nablax \hphi_k\|_\hY^2\\ %
&\quad =\ga(t) \cdot \Big( \big((u_k \cdot \nablax) \hphi_k, \init\big)_\hY + \frac{1}{2\lambda} \big(\nablaq \hphi_k,\nablaq \init\big)_\hY  \\[0cm] &\qquad + \eps \big(\nablax \hphi_k,\nabla \init\big)_\hY- \big(\omega(u_k) q \hphi_k, \nablaq \init\big)_\hY \Big) =:R.
\end{aligned}\end{equation}
We then use the fractional chain inequality \cref{Eq:ChainOriginal} to bound the left-hand side of \cref{Eq:TestingRHS} from below, which yields
$$
\frac12 \pta \|\hphi_k-\ga \init\|^2_\hY  \leq (\pta \hphi_k,\hphi_k-\ga \init)_\hY = (\pta \hphi_k,\ga * \pta \hphi_k)_\hY. 
$$
Regarding the right-hand side of \cref{Eq:TestingRHS}, we integrate the last term containing $\omega(u_k)$ by parts, see
\cref{Eq:IntParts}, and get
$$\begin{aligned} -  \big(M\omega(u_k)  q \hphi, \nablaq \init\big)_{\mathcal{Y}} &=  \big(M \nablax \hphi_k (\nablaq \init)^{\mathrm{T}} q,u_k\big)_{\mathcal{Y}} + \big(M \hphi_k \nablax \nablaq \init q,u_k\big)_{\mathcal{Y}} \\ &\quad + \big(u_k \cdot q,M \nablax \hphi_k \cdot \nablaq \init\big)_{\mathcal{Y}} + \big(u_k \cdot q,M \hphi_k \div_x \nablaq \init\big)_{\mathcal{Y}}. 
\end{aligned} $$
We apply H\"older's inequality to obtain the following bound on the right-hand side, $R$, of the equality \cref{Eq:TestingRHS}:
$$\begin{aligned} R \leq{}& \ga(t) \cdot\! \Big(  \|u_k\|_{\mathcal{H}}  \|\nablax\hphi_k\|_\hY \|M^{1/2}\init\|_{L^2(D;L^\infty(\Omega)) }\\
%
%[-.2cm]  &+\|u_k\|_{\mathcal{H}}  \|\hphi_k\|_\hY \|M^{1/2}
%\init\|_{L^2(D;W^{1,\infty}(\Omega)) }  \\[0cm]  
%
&+ \frac{1}{2\lambda} \|\nablaq \hphi_k\|_\hY  \|\nablaq \init\|_\hY  + \eps \|\nablax \hphi_k\|_\hY \|\nablax \init\|_\hY   \\
%[-.1cm]  
&+ C\|u_k\|_{\mathcal{H}} \|q\|_{L^\infty(D)}  \|M^{1/2}\nablaq \init\|_{L^2(D;W^{1,\infty}(\Omega)) } \big(\|\hphi_k\|_\hY + \|\nablax \hphi_k\|_\hY\big) \Big).
\end{aligned}$$
Hence, thanks to Young's inequality, we arrive at the following bound on $R$: 
$$\begin{aligned} R \leq&~{} \frac{1}{\eps} \ga(t)^2 \|M^{1/2} \init\|_{L^2(D;W^{1,\infty}(\Omega)) }^2  \|u_k\|_{\mathcal{H}}^2  + \frac{\eps}{4} \|\nablax \hphi_k\|_\hY^2  \\ &{} + \frac{1}{4\lambda} \|\nablaq \hphi_k\|_\hY^2   + \frac{\ga(t)^2}{4\lambda} \|\nablaq \init\|_\hY^2  +\frac{\eps}{4} \|\nablax \hphi_k\|_\hY^2 + \eps \ga(t)^2 \|\nablax \init\|_\hY^2 \\
&{}+\delta  \|\hphi_k\|_\hY + \frac{\eps}{4} \|\nablax \hphi_k\|_\hY^2 + C(\eps,\delta) \ga(t)^2 \|M^{1/2}\nablaq \init\|_{L^2(D;W^{1,\infty}(\Omega)) }^2 \|u_k\|_{\mathcal{H}}^2,
\end{aligned}$$
where $\delta>0$ is sufficiently small, to be chosen appropriately later on. After combining the lower bound on the left-hand side of \cref{Eq:TestingRHS} and the upper bound on the right-hand side we have that
$$\begin{aligned} &\frac12 \pta \|\hphi_k-\ga \init\|^2_\hY +\frac{1}{2\lambda}  \|\nablaq \hphi_k\|_\hY^2 + \eps \| \nablax \hphi_k\|_\hY^2\\ &\quad \leq \frac{1}{\eps}\ga(t)^2 \|M^{1/2} \init\|_{L^2(D;W^{1,\infty}(\Omega)) }^2  \|u_k\|_{\mathcal{H}}^2  + \frac{\eps}{4} \|\nablax \hphi_k\|_\hY^2  \\ &{} \qquad + \frac{1}{4\lambda} \|\nablaq \hphi_k\|_\hY^2   + \frac{\ga(t)^2}{4\lambda} \|\nablaq \init\|_\hY^2  +\frac{\eps}{4} \|\nablax \hphi_k\|_\hY^2 + \eps \ga(t)^2 \|\nablax \init\|_\hY^2 \\
&{}\qquad +\delta  \|\hphi_k\|_\hY^2 + \frac{\eps}{4} \|\nablax \hphi_k\|_\hY^2 + C(\eps,\delta) \ga(t)^2 \|M^{1/2}\nablaq \init\|_{L^2(D;W^{1,\infty}(\Omega)) }^2 \|u_k\|_{\mathcal{H}}^2,\end{aligned}$$
and absorbing terms on the right-hand side into the left-hand side gives
$$\begin{aligned} &\frac12 \pta \|\hphi_k-\ga \init\|^2_\hY +\frac{1}{4\lambda}  \|\nablaq \hphi_k\|_\hY^2 + \frac{\eps}{2} \| \nablax \hphi_k\|_\hY^2\\ &\quad \leq C(\eps,\delta) \ga(t)^2  \|M^{1/2} \init\|_{H^1(D;W^{1,\infty}(\Omega))}   \|u_k\|_{\mathcal{H}}^2  + \delta \|\hphi_k\|_\hY^2 + C(\eps,\delta)\ga(t)^2 \|\init\|_\hX^2.
\end{aligned}$$

We note that $\ga^2=\frac{\Gamma(2\alpha-1)}{\Gamma(\alpha)^2}g_{2\alpha-1}$ is integrable for $\alpha\in (\tfrac12,1)$ and $g_1*g_{2\alpha-1}=g_{2\alpha}$, which is continuous, bounded, and monotonically increasing on $[0,T]$ for $\alpha\in(\tfrac12,1)$.  We integrate the inequality over $(0,t)$ and exploit the representation $\pta v = \pt (g_{1-\alpha} * v)$ of the Riemann--Liouville derivative, which gives
\begin{equation} \label{Eq:EnergyFP}\begin{aligned} & \frac12 \big(\gb* \|\hphi_k-\init \ga\|_\hY^2\big)(t)  +\frac{1}{4\lambda} \|\nablaq \hphi_k\|_{L^2_t\hY}^2 + \frac{\eps}{2} \| \nablax \hphi_k\|_{L^2_t \hY}^2 - \delta \|\hphi_k\|^2_{L^2_t\hY}  \\
&\quad \leq C(\delta,\alpha) \|M^{1/2} \init\|_{H^1(D;W^{1,\infty}(\Omega)) }^2  \int_0^t g_{2\alpha-1}(s) \|u_k(s)\|_{\mathcal{H}}^2 \ds
+ C(\delta)\| \init\|_\hX^2  g_{2\alpha}(T).
\end{aligned}\end{equation}
Further, we derive a lower bound on the first term of the left-hand side by noting that $(g_1*v)(t) \leq T^{\alpha} \Gamma(1-\alpha) (\gb*v)(t)$, see \cref{Eq:KernelNorm}, and therefore we have that
$$\begin{aligned} &\frac12 \big(\gb* \|\hphi_k-\init \ga\|_\hY^2\big)(t) \\ &\quad \geq \frac{T^{-\alpha}}{2\Gamma(1-\alpha)} \int_0^t  \|\hphi_k(s)-\init \ga(s)\|_{\hY}^2 \ds \\ &\quad \geq \frac{T^{-\alpha}}{2\Gamma(1-\alpha)} \int_0^t \big|\,\|\hphi_k(s)\|_\hY-\ga(s)\|\init \|_{\hY} \big|^2\ds \\
&\quad =\frac{T^{-\alpha}}{2\Gamma(1-\alpha)} \int_0^t \|\hphi_k(s)\|_\hY^2 - 2\ga(s)\|\hphi_k(s)\|_\hY\|\init \|_{\hY} +g_{\alpha}^2(s)\|\init \|_{\hY}^2\ds,
\end{aligned} $$
where we applied the reverse triangle inequality in the second estimate.
The function $g_{\alpha}$ belongs to $L^2(0,t)$ for any $\alpha\in (\tfrac12,1)$ and the integral of $g_\alpha^2$ is positive. We apply H\"older's inequality in the second term of the integrand and note that the $L^2(0,t)$-norm of $g_\alpha$ has the upper bound $C(\alpha)T^{\alpha-1/2}$. We thus have that
$$\begin{aligned}   \frac12 \big(\gb* \|\hphi_k-\init \ga\|_\hY^2\big)(t)   &\geq 
\frac{T^{-\alpha}}{2\Gamma(1-\alpha)} \left[
\frac12 \|\hphi_k\|_{L^2_t\hY}^2 - 2 C(\alpha) T^{\alpha-1/2} \|\init \|_{\hY}  \|\hphi_k\|_{L^2_t\hY} \right]\\ &\geq \frac{T^{-\alpha}}{2\Gamma(1-\alpha)}\left[\frac14 \|\hphi_k\|_{L^2_t\hY}^2 - C(T,\alpha) \|\init \|_{\hY}^2\right],
\end{aligned} $$
where we have applied Young's inequality in the last step. We multiply the energy estimate \cref{Eq:EnergyFP} by $\gamma>0$ and obtain for $\delta=\frac{T^{-\alpha}}{16 \Gamma(1-\alpha)}$ the estimate \cref{Est:SolFP}.
\end{proof} %\medskip

Next, we combine the estimates on $u_k$ and $\hphi_k$, see \cref{Lem:EstU} and \cref{Lem:EstPhi}, to obtain a $k$-uniform bound. %\medskip

\begin{lemma} \label{Lem:EstComb}
	Let \cref{Ass:WellPosedness} hold; then, the following $k$-uniform estimate on the Galerkin solution $(u_k,\hphi_k)$ holds:
	\begin{equation} \label{Eq:EnergyIneq}
		\begin{aligned} &  \big(\gb* \|\hphi_k-\init \ga\|_\hY^2\big)(t) +  \|\hphi_k\|_{L^2_t\hX}^2    + \|u_k(t)\|_{\mathcal{H}}^2 +  \|\nabla u_k\|_{L^2_t{\mathcal{H}}}^2 \\
			&\quad \leq C\big(\alpha,T, \|u^0\|_{\mathcal{H}}^2,\|M^{1/2} \hpsi^0\|^2_{H^1(D;H^{1+d/2+\delta}(\Omega)) }\big).
	\end{aligned}\end{equation}
\end{lemma} 

\begin{proof}
We add the integrated velocity inequality \cref{Eq:EnergyU2} to the bound \cref{Est:SolFP} and obtain the following combined bound:
$$\begin{aligned} &\frac{\gamma}{2} \big(\gb* \|\hphi_k-\init \ga\|_\hY^2\big)(t) + \frac{\gamma\, T^{-\alpha}}{16\,\Gamma(1-\alpha)} \|\hphi_k\|_{L^2_t\hY}^2  +\frac{\gamma}{2\lambda} \|\nablaq \hphi_k\|_{L^2_t\hY}^2 + \gamma \eps \| \nablax \hphi_k\|_{L^2_t \hY}^2  \\ &\quad + \frac12 \|u_k(t)\|_{\mathcal{H}}^2 + \frac{\nu}{2} \|\nabla u_k\|_{L^2_t{\mathcal{H}}}^2 \\
&\leq C(\alpha,\gamma) \|M^{1/2} \init\|_{H^1(D;W^{1,\infty}(\Omega)) }^2  \int_0^t g_{2\alpha-1}(s) \|u_k(s)\|_{\mathcal{H}}^2 \ds
+ C(\alpha,\gamma,T)\| \init\|_\hX^2 \\&\quad + \frac12 \|u_k^0\|_{\mathcal{H}}^2 + %
\frac{ T^{2-2\alpha} k_B^2\mu_T^2}{2\nu (1-\alpha)^2}  \|\hphi_k\|_{L^2_t\hY}^2.
\end{aligned}$$
We now choose $\gamma$ such that 
$$\frac{\gamma\, T^{-\alpha}}{16\, \Gamma(1-\alpha)} \geq \frac{ T^{2-2\alpha} k_B^2\mu_T^2}{\nu (1-\alpha)^2}.$$
Hence, we can absorb the last term on the right-hand side into the second term on the left-hand side, and 
 the combined energy inequality thus becomes
\begin{equation} \label{Eq:EnergyBack}\begin{aligned} & \big(\gb* \|\hphi_k-\init \ga\|_\hY^2\big)(t) +  \|\hphi_k\|_{L^2_t\hX}^2    + \|u_k(t)\|_{\mathcal{H}}^2 +  \|\nabla u_k\|_{L^2_t{\mathcal{H}}}^2 \\
&\quad\leq C(\alpha,T) \|M^{1/2} \init\|_{H^1(D;W^{1,\infty}(\Omega)) }^2  \int_0^t g_{2\alpha-1}(s) \|u_k(s)\|_{\mathcal{H}}^2 \ds
\\ &\qquad + C(\alpha,T) \big( \| \init\|_\hX^2 + \|u_k^0\|_{\mathcal{H}}^2\big),
\end{aligned}\end{equation}
where we took the minimum of each prefactor of the norms on the left-hand side and divided the inequality by this value. Gronwall's lemma then implies that
\begin{equation} 
\begin{aligned} \label{Eq:EnergyBack1}
& \big(\gb* \|\hphi_k-\init \ga\|_\hY^2\big)(t) +  \|\hphi_k\|_{L^2_t\hX}^2    + \|u_k(t)\|_{\mathcal{H}}^2 +  \|\nabla u_k\|_{L^2_t{\mathcal{H}}}^2 \\
&\quad \leq C(\alpha,T) \cdot \big(  \| \init\|_\hX^2  + \|u_k^0\|_{\mathcal{H}}^2  \big) \cdot \textup{exp}\bigg(\frac{T^{2\alpha-1}}{2\alpha-1} \|M^{1/2} \init\|^2_{H^1(D;W^{1,\infty}(\Omega)) }\bigg).
\end{aligned}
\end{equation}


We note that the initial conditions of the Galerkin system are defined by $u_k^0=\Pi_{\mathcal{H}_k} u^0$ and $\init=\Pi_{\hY_k} \hpsi^0$. Therefore, we have that $\|u_k^0\|_{\mathcal{H}}^2 \leq \|u^0\|_{\mathcal{H}}^2$ and $$\|M^{1/2}\init\|_{H^1(D;W^{1,\infty}(\Omega)) }^2 \leq C \|M^{1/2}\hpsi^0\|_{H^1(D;H^{1+d/2+\delta}(\Omega)) }^2.$$
%see \cite[Theorem 8.1.11]{brenner2008mathematical} with regards to the stability of the Ritz projection in $W^{1,\infty}$.
We insert these bounds into the right-hand side of the inequality \cref{Eq:EnergyBack1} and we thus arrive at the desired $k$-uniform energy estimate \cref{Eq:EnergyIneq}.
\end{proof}





\subsection{Convergence of subsequences} 
Having derived the $k$-uniform energy estimate \cref{Eq:EnergyIneq} stated in \cref{Lem:EstComb},
we shall extract weakly/weakly-$*$ converging subsequences of Galerkin solutions $(u_k,\hphi_k)$. We shall also prove strong convergence of a subsequence $u_{k_j}$ in $L^2(0,T;\mathcal{H}_0)$ in order to pass to the limit $j\to \infty$ in the nonlinear terms in the variational Navier--Stokes--Fokker--Planck system.   %\medskip

\begin{lemma} Let \cref{Ass:WellPosedness} hold and assume that $r \in [1,\infty)$ for $d=2$ and $r \in [1,6)$ for $d=3$; then, the sequence of Galerkin solutions $(u_k,\hphi_k)$ from \cref{Eq:LemmaExistenceODE} contains a subsequence $(u_{k_j},\hphi_{k_j})$ that admits the following convergences as $j \to \infty$:
	%\footnote{\color{red}~I don't see where the 7th weak convergence, with spatial function space $\hY$, is coming from. Surely this is incorrect. -- This should be corrected now} 
	\begin{equation} \label{Eq:Weak} \begin{aligned}	
			u_{k_j} &\longweak u &&\text{weakly-$*$ in } L^\infty(0,T;\mathcal{H}_0), \\
			u_{k_j} &\longweak u &&\text{weakly\phantom{-*} in } L^2(0,T;\mathcal{V}_0) \cap L^{8/d}(0,T;L^4(\Omega)^d), \\
			\hphi_{k_j} &\longweak 		 \hphi &&\text{weakly\phantom{-*} in }
			L^2(0,T;\hX), \\
		\pt u_{k_j} &\longweak \pt u &&\text{weakly\phantom{-*} in } L^{8/(4+d)}(0,T;\mathcal{V}_0'), \\
		u_{k_j} &\longrightarrow  u &&\text{strongly\hspace{1mm} in } L^2\big(0,T;L^{r}(\Omega;\R^d)\big), \\
		u_{k_j} &\longrightarrow  u &&\text{strongly\hspace{1mm} in } C([0,T];\mathcal{V}_0'),\\
		\pt^{\alpha} \hphi_{k_j} &\longweak \pt^{\alpha} \hphi &&\text{weakly\phantom{-*} in } L^{8/(4+d)}(0,T;\hZ'),\\
		\hphi_{k_j} &\longrightarrow  \hphi &&\text{strongly\hspace{1mm} in } L^2(0,T;\hY),
\\ 
\C(M \gb * \hphi_{k_j}) &\longrightarrow \C(M\gb *\hphi) &&\text{strongly\hspace{1mm} in }
		L^2\big(0,T;L^2(\Omega;\R^{d\times d})\big), \\
		\gb*\hphi_{k_j} &\longweak \gb *\hphi &&\text{weakly-$*$ in } L^\infty(0,T;\hY) \cap L^2(0,T;\hX), \\
		\gb*\hphi_{k_j} &\longrightarrow \gb *\hphi &&\text{strongly\hspace{1mm} in }
		C([0,T];\hX') \cap L^2(0,T;\hY).
		\end{aligned}
	\end{equation} 
	\end{lemma}
\begin{proof}
In Lemma \ref{Lem:EstComb} we stated various $k$-uniform bounds on $u_k$ and $\hphi_k$. Thanks to the Banach--Alaoglu and Eberlein--\v{S}mulian theorems, see \cite[Theorem 8.10]{alt2016linear}, there are weakly/weakly-$*$ converging subsequences $u_{k_j}$ and $\hphi_{k_j}$. In particular, we obtain the convergences
\begin{equation} \label{Eq:Weak1} \begin{aligned}	
u_{k_j} &\longweak u &&\text{weakly-$*$ in } L^\infty(0,T;\mathcal{H}_0), \\
u_{k_j} &\longweak u &&\text{weakly\phantom{-*} in } L^2(0,T;\mathcal{V}_0), \\
\hphi_{k_j} &\longweak 		 \hphi &&\text{weakly\phantom{-*} in }
L^2(0,T;\hX). 
\end{aligned}\end{equation}


We shall establish the strong convergence of $u_{k_j}$ in $L^2(0,T;\mathcal{H}_0)$ by applying the Aubin--Lions compactness lemma; see \cref{Eq:aubin}. To this end, we need to bound the time derivative of $u_k$ in a suitable dual space. Let us therefore consider an arbitrary element $v \in L^{8/(4-d)}(0,T;\mathcal{V}_0)$ and bound each of the terms appearing on the right-hand side of \cref{Eq:NS_dis} below by means of H\"older's inequality:
$$\begin{aligned}\int_0^T | \langle \pt u_k,v \rangle_{\mathcal{V}_0}| \dt   &= \int_0^T  \Big| -((u_k \cdot \nablax)u_k,\Pi_{\mathcal{H}_k} v)_{\mathcal{H}}  \\ 
&\quad  - \nu (\nablax u_k,\nablax \Pi_{\mathcal{H}_k} v)_{\mathcal{H}} - k_B \mu_T \big(\C(M\gb*\hphi_k),\nablax \Pi_{\mathcal{H}_k} v \big)_{\mathcal{H}} \Big| \dt
	\\ 
& \leq C \int_0^T \Big( \|u_k\|_{L^4(\Omega)} \|u_k\|_{\mathcal{V}} \|\Pi_{\mathcal{H}_k} v\|_{L^4(\Omega)}  \\ &\quad  + \|u_k\|_{\mathcal{V}} \|\Pi_{\mathcal{H}_k} v\|_{\mathcal{V}} +  \|\C(M\gb*\hphi_k)\|_{\mathcal{H}}  \|\Pi_{\mathcal{H}_k} v\|_{\mathcal{V}} \Big) \dt. 
\end{aligned}$$
Hence, using Ladyzhenskaya's inequality, we have that
$$\begin{aligned}
&\int_0^T | \langle \pt u_k,v \rangle_{\mathcal{V}_0}| \dt\\  
 &\leq C \int_0^T \Big( \|u_k\|_{\mathcal H}^{1-d/4} \|u_k\|_{\mathcal{V}}^{1+d/4} \|\Pi_{\mathcal{H}_k} v\|_{L^4(\Omega)} + \|u_k\|_{\mathcal{V}} \|\Pi_{\mathcal{H}_k} v\|_{\mathcal{V}} +  \|\hphi_k\|_\hY  \|\Pi_{\mathcal{H}_k} v\|_{\mathcal{V}} \Big) \dt 
 \\
&\leq C \Big( \|u_k\|_{L^\infty {\mathcal{H}}}^{1-d/4} \|u_k\|_{L^2{\mathcal{V}}}^{1+d/4} \|v\|_{L^{8/(4-d)} {\mathcal{V}}} + \|u_k\|_{L^2{\mathcal{V}}} \|v\|_{L^2{\mathcal{V}}} +  \|\hphi_k\|_{L^2\hY} \|v\|_{L^2{\mathcal{V}}} \Big)
	\\
	&\leq C\|v\|_{L^{8/(4-d)} {\mathcal{V}}}.
\end{aligned}$$
This then implies that $\pt u_k$ is bounded in $L^{8/(4+d)}(0,T;\mathcal{V}_0
 ')$. 
It follows by the Aubin--Lions lemma \cref{Eq:aubin} that  %
\begin{equation} \label{Eq:Weak2} \begin{aligned}	
\pt u_{k_j} &\longweak \pt u &&\text{weakly\phantom{-*} in } L^{8/(4+d)}(0,T;\mathcal{V}_0'), \\
u_{k_j} &\longrightarrow  u &&\text{strongly\hspace{1mm} in } L^2\big(0,T;L^{r}(\Omega;\R^d)\big), \\
u_{k_j} &\longrightarrow  u &&\text{strongly\hspace{1mm} in } C([0,T];\mathcal{V}_0'),
\end{aligned}\end{equation}
where $r \in [1,\infty)$ for $d=2$ and $r \in [1,6)$ for $d=3$.


Similarly, we consider an arbitrary element $\hzeta \in L^\frac{8}{4-d}(0,T;\hZ)$ and we recall that $\hZ$ was defined in the beginning of \cref{Sec:Maxwell}. We test the Galerkin equation of $\hphi_k$ with $\Pi_{\mathcal H_k} \hzeta$ giving
\begin{equation} \label{Eq:BoundDerivative} \begin{aligned}\int_0^T \!\! |\langle \pta \hphi_k,\Pi_{\mathcal H_k} \hzeta \rangle_{\hX}| \dt  &=\!
\int_0^T\!\! \Big| -((u_k \cdot \nablax) \hphi_k, \Pi_{\mathcal H_k} \hzeta)_\hY - \frac{1}{2\lambda} (\nablaq \hphi_k,\nablaq \Pi_{\mathcal H_k} \hzeta)_\hY   \\ &\quad - \eps (\nablax \hphi_k,\nabla_x \Pi_{\mathcal H_k} \hzeta)_\hY+ (\omega(u_k) q \hphi_k, \nablaq \Pi_{\mathcal H_k} \hzeta)_\hY \Big|\d t.\end{aligned} \end{equation}
We note that $u_k$ is bounded in $L^{8/d}(0,T;L^4(\Omega)^d)$ by the following interpolation result
$$\int_0^T \|u_k\|_{L^4}^{8/d} \dt \leq \int_0^T  \|u_k\|_{\mathcal H}^{8/d-2} \|u_k\|^{2}_{\mathcal V} \dt \leq \|u_k\|_{L^\infty \mathcal{H}}^{8/d-2} \|u_k\|_{L^{2} \mathcal{V}}^{2}.$$
Regarding the last term in \eqref{Eq:BoundDerivative}, we integrate by parts and estimate by H\"older's inequality
$$\begin{aligned} -  &\big(\omega(u_k)  q \hphi, \nablaq \hzeta\big)_\hY \\ &=  \big(\nablax \hphi_k (\nablaq \hzeta)^{\mathrm{T}} q,u_k\big)_\hY + \big( \hphi_k \nablax \nablaq \hzeta q,u_k\big)_\hY + \big(u_k \cdot q, \nablax \hphi_k \cdot \nablaq \hzeta\big)_\hY \\ &\quad  + \big(u_k \cdot q, \hphi_k \div_x \nablaq \hzeta\big)_\hY \\
	&\leq C  \| \nablax \hphi_k\|_{L^2\hY}
	\big( \|\nablaq \hzeta\|_{L^{8/(4-d)} \hY}  +  \|\nablax \nablaq \hzeta\|_{L^{8/(4-d)} \hY}  \big) \|q\|_{L^\infty} \|u_k\|_{L^{8/d} L^4}  \\ &\quad+ \| \hphi_k\|_{L^2\hX} \|\nablax \nablaq \hzeta\|_{{L^{8/(4-d)} \hY}} \|q\|_{L^\infty} \|u_k\|_{L^{8/d} L^4}  \\ &\quad + C \|u_k\|_{L^{8/d} L^4} \|q\|_{L^\infty} \|\nablax \hphi_k\|_{L^2 \hY} \big(\|\nablaq \hzeta\|_{L^{8/(4-d)}\hY}  + \|\nablax \nablaq \hzeta\|_{L^{8/(4-d)}\hY} \big) \\ &\quad +\|u_k\|_{L^{8/d} L^4} \|q\|_{L^\infty}  \|\hphi_k\|_{L^2 \hX} \|\div_x \nablaq \hzeta\|_{L^{8/(4-d)}\hY}  \\
	&\leq C \|u_k\|_{L^{8/d} L^4} \|q\|_{L^\infty}  \|\hphi_k\|_{L^2 \hX} \| \hzeta\|_{L^{8/(4-d)}(0,T;\hZ) } .
\end{aligned} $$
Using this estimate, we can bound \eqref{Eq:BoundDerivative} as follows:
\begin{equation} \label{Eq:BoundDerivativePhi} \begin{aligned}& \int_0^T |\langle \pta \hphi_k,\Pi_{\mathcal H_k} \hzeta \rangle_{\hX}| \dt  \\ &\leq C \Big( \|u_k \|_{L^\infty\mathcal{H}_0 }  \|\hphi_k \|_{L^2 \hX}   \|\hzeta \|_{L^2\hX} +  \|\nablaq \hphi_k\|_{L^2\hY}   \|\nablaq \hzeta \|_{L^2\hY}    \\ &\quad +  \| \nablax \hphi_k\|_{L^2\hY}   \|\nabla_x\hzeta \|_{L^2\hY} + \|u_k\|_{L^{8/d} L^4} \|q\|_{L^\infty}  \|\hphi_k\|_{L^2 \hX} \| \hzeta\|_{L^{8/(4-d)}(0,T;\hZ) } \Big) \\
		&\leq  C \|\hzeta\|_{L^{8/(4-d)}(0,T;\hZ) } .
\end{aligned} \end{equation}
Hence, we obtain the $k$-uniform boundedness of $\pt(\gb*\hphi_{k})=\pt^{\alpha} \hphi_k$ in the space $L^{8/(4+d)}(0,T;\hZ')$, which is continuously embedded  in $L^{8/(4+d)}(0,T;(H^2_M(\Omega \times D))')$. 
Therefore, we are in the setting of the Gelfand triple
$$\hX \com \hY\hookrightarrow \big( H_M^2(\Omega \times D)\big)'.$$ 
%and $\hphi_{k}$ is bounded in the space 
%$$L^2(0,T;\hX) \cap W^{\alpha,8/(4+d)}(0,T;(H^2(\Omega \times D;M))').$$
We thus obtain from the fractional Aubin--Lions lemma, see \eqref{Eq:aubinfractional2}, that
\begin{equation} \label{Eq:Weak3} \begin{aligned}	
		\pt^{\alpha} \hphi_{k_j} &\longweak \pt^{\alpha} \hphi &&\text{weakly\phantom{-*} in } L^{8/(4+d)}(0,T;\hZ'), \\
		\hphi_{k_j} &\longrightarrow  \hphi &&\text{strongly\hspace{1mm} in } L^2(0,T;\hY).
\end{aligned} \end{equation}

%We note that we can test the weak form of $\hphi_k$ again by $\hzeta=g_\alpha * \pta \hphi_k$, i.e., we consider the tested weak form \cref{Eq:TestingRHS}. However, this time we do not apply the fractional chain inequality on the first term on the left-hand side of \cref{Eq:TestingRHS} but we exploit the coercivity of the kernel function in the integrated weak form, see
% \cref{Eq:Coercive2}, to obtain
% $$\int_0^t (\pta \hphi_k,\hphi_k - \ga \hpsi_k^0)_\hY \ds  \geq  \cos(\alpha \pi/2)  \Big( \tfrac12 \|\pt^{\alpha/2} \hphi_k\|^2_{L^2_t \hY}-\tfrac{\Gamma(\alpha-1)}{\Gamma(\alpha/2)^2}g_{\alpha}(t) \|\hpsi_k^0\|_\hY^2 \Big). $$
  %We take $t=T$ and use the derived energy estimates to bound the right-hand side of \cref{Eq:TestingRHS} similiarly to before. 
 % We can apply the fractional Aubin--Lions lemma, see \eqref{Eq:aubinfractional2}, to obtain the strong convergence of $\hphi_{k_j}$ in $L^2(0,T;\hY)$; in other words, 
The convolution $\gb*\hphi_{k}$ is bounded in $L^2(0,T;\hX)$ thanks to Young's convolution inequality 
$$\|\gb * \hphi_k\|_{L^2_t\hX} \leq \|g_{1-\alpha}\|_{L^1_t} \|\hphi_{k}\|_{L^2_t\hX} \leq C T^{1-\alpha} \|\hphi_{k}\|_{L^2_t\hX}.$$
Moreover,  $\gb*\hphi_{k}$ is bounded in $L^\infty(0,T;\hY)$ by the following chain of estimates:
$$\begin{aligned} &\|\gb*\hphi_{k}\|_{L^\infty \hY} \\&\quad \leq \sup_{t \in (0,T)} \int_0^t \gb(t-s) \|\hphi_{k}(s)\|_\hY \ds \\ 
	&\quad \leq \sup_{t \in (0,T)} \int_0^t \gb(t-s) \|\hphi_{k}(s)-\hpsi_{k}^0 \ga(s)\|_\hY \ds + (\gb*\ga)(t) \|\hpsi_{k}^0\|_\hY   \\ 
	&\quad \leq   \sup_{t \in (0,T)} \int_0^t \gb(t-s) \|\hphi_{k}(s)-\hpsi_{k}^0 \ga(s)\|^2_\hY \ds + \frac14 \int_0^t \gb(t-s) \ds + \|\hpsi_{k}^0\|_\hY  
	\\ &\quad = \sup_{t \in (0,T)} (\gb*\|\hphi_{k}-\hpsi_{k}^0 \ga\|_\hY^2)(t) + \frac14 g_{2-\alpha}(T) + \|\hpsi_{k}^0\|_\hY,
\end{aligned}$$
and the first term on the right-hand side is bounded by \cref{Lem:EstComb}. Since we have already proved a bound on $\pt(\gb*\hphi_{k})=\pta \hphi_{k}$, see \eqref{Eq:BoundDerivativePhi}, we may use the Aubin--Lions lemma, see \cref{Eq:aubin}, to obtain the following strong convergence results: 
\begin{equation} \label{Eq:Weak5} \begin{aligned}	
		\gb*\hphi_{k_j} &\longrightarrow  \gb*\hphi &&\text{strongly\hspace{1mm} in } L^2(0,T;\hY), \\
		\gb*\hphi_{k_j} &\longrightarrow  \gb*\hphi &&\text{strongly\hspace{1mm} in } C([0,T];\hX').
\end{aligned} \end{equation}
Lastly, we note that the mapping $M\gb*\varphi \mapsto \C(M\gb*\varphi)$ is linear and continuous thanks to \eqref{Eq:C},  %\leq C \|g_{1-\alpha}\|_{L^1_t} \|\varphi\|_{L^2_t\hY} \leq C T^{1-\alpha} \|\varphi\|_{L^2_t\hY}, $$
and therefore we have from \eqref{Eq:Weak5}$_1$ that
\begin{equation} \label{Eq:Weak4} \C(M \gb * \hphi_{k_j}) \longrightarrow \C(M\gb *\hphi) \quad \text{ strongly in }
L^2\big(0,T;L^2(\Omega;\R^{d\times d})\big).
\end{equation} 
\end{proof}

\subsection{Passage to the limit}
Next, we pass to the limit $j \to \infty$ in the  time-integrated $k_j$-th Galerkin system \cref{Eq:NS_dis}, \cref{Eq:FP_dis}. Specifically, we shall use the convergence results stated in the preceding lemma to show that the weak limits, $u$ and $\hphi$, satisfy the variational Navier--Stokes--Fokker--Planck system in the sense of \cref{Eq:DefWeak}. 
%
\begin{proof}[Proof of \cref{Thm:WellPosedness}]
We consider the time-integrated Galerkin system
\begin{align} 
&\int_0^T \Big(\langle\pt u_{k_j},v \rangle_V  + ((u_{k_j} \cdot \nablax)u_{k_j},v)_{\mathcal{H}} \label{Eq:NS_dis_time} \\ 
& \quad + \nu (\nablax u_{k_j},\nablax v)_{\mathcal{H}}+ k \mu (C(M \gb *\hphi_{k_j}),\nablax v)_{\mathcal{H}} \Big) \eta(t) \dt =0  \notag\\
&\int_0^T  -(\gb*\hphi_{k_j},\hzeta)_\hY \eta'(t) + \Big(    ((u_{k_j}\cdot \nablax) \hphi_{k_j}, \hzeta)_\hY+ \frac{1}{2\lambda} ( \nablaq \hphi_{k_j},\nablaq \hat\zeta)_\hY \label{Eq:FP_dis_time} \\
&\quad + \eps (\nablax \hphi_{k_j},\nabla \hat\zeta)_\hY- (\omega(u_{k_j}) q \hphi_{k_j}, \nablaq \hat\zeta)_\hY \Big) \eta(t) \dt=0, \notag \end{align} for all $v \in \H_{k_j}$, $\eta \in C_0^\infty(0,T)$ and $\hat\zeta \in \hY_{k_j}$.  
Passing to the limit $j \rightarrow \infty$ in \eqref{Eq:NS_dis_time} using \cref{Eq:Weak1}--\cref{Eq:Weak4}
is standard, and results in \eqref{Eq:NS}. It therefore remains to pass to the limit $j \rightarrow \infty$ in \eqref{Eq:FP_dis_time}. In particular, the convergence of the linear terms follow immediately by weak convergence, and we only consider the two nonlinear terms. 
 We note that $\hphi_{k_j} \to \hphi$ strongly in $L^2\big(0,T;\hY)$ and $\omega(u_{k_j}) \to \omega(u)$ weakly in $L^2(0,T;\H_0)$, from which we deduce that
$$\int_0^T (\omega(u_{k_j}) q \hphi_{k_j}, \nablaq \hat\zeta)_\hY  \eta(t) \dt \longrightarrow  \int_0^T (\omega(u) q \hphi, \nablaq \hat\zeta)_\hY  \eta(t) \dt,$$
as $j \to \infty$. With the same reasoning, we are able to show that
$$\int_0^T  ((u_{k_j}\cdot \nablax) \hphi_{k_j}, \hat\zeta)_{\hY} \eta(t) \dt \longrightarrow \int_0^T  ( (u \cdot\nablax) \hphi, \hat\zeta)_{\hY} \eta(t) \dt\quad \mbox{as $j \to \infty$}.$$

%To this end we apply integration by parts in time in \eqref{Eq:FP_dis_time}, which yields
%
%$$\begin{aligned} 0=&- \int_0^T ( M\gb * \hphi_{k_j},\hat\zeta)_{\mathcal{Y}} \eta'(t) \dt - ( M\hpsi_0,\hat\zeta)_{\mathcal{Y}} \eta(0) && \\ &+  
%\int_0^T \Big( ((u_{k_j}\cdot \nablax) \hphi_{k_j}, \hzeta)_\hY+ \frac{1}{2\lambda} ( \nablaq \hphi_{k_j},\nablaq \hat\zeta)_\hY \\
%&\quad + \eps (\nablax \hphi_{k_j},\nabla \hat\zeta)_\hY- (\omega(u_{k_j}) q \hphi_{k_j}, \nablaq \hat\zeta)_\hY \Big) \eta(t) \dt=0, \notag 
%\end{aligned}$$
%for all $\hat\zeta \in \hX_{k_j}$ and $\eta \in C_0^\infty(-T,T)$. We apply Young's convolution inequality to the first term on the right-hand side to obtain
%$$\begin{aligned} - \int_0^T (M \gb * \hphi_{k_j},\hat\zeta)_{\mathcal{Y}} \eta'(t) \dt &\leq \|\gb * \hphi_{k_j}\|_{L^1\hY} \|\hat\zeta\|_\hY \|\eta'\|_{L^\infty_T}
%\\&\leq C \|\gb\|_{L^1_T} \|\hphi_{k_j}\|_{L^1 \hY} \|\hat\zeta\|_\hY  \|\eta'\|_{L^\infty_T},
%\end{aligned} $$
%from which we conclude the convergence of the first term.
%We exploit the corotationality of $\omega$ in the form of
%$$(M\omega(u)q,\eta)_{\mathcal{Y}} = \frac12 (M v \cdot q, \div_x \eta)_{\mathcal{Y}} - \frac12 (M\nablax \eta q,v)_{\mathcal{Y}},$$
%which allows us to rewrite \cref{Eq:FP_dis_time} as
%$$\begin{aligned} 0=&- \int_0^T ( M\gb * \hphi_{k_j},\hat\zeta)_{\mathcal{Y}} \eta'(t) \dt - ( M\hpsi_0,\hat\zeta)_{\mathcal{Y}} \eta(0) && \\  &+ \frac12 \int_0^T \Big(  2 (M (u_{k_j} \cdot \nablax) \hphi_{k_j}, \hat\zeta)_{\mathcal{Y}}  + \frac{1}{\lambda} (M \nablaq \hphi_{k_j},\nablaq \hat\zeta)_{\mathcal{Y}} + 2\eps (M\nablax \hphi_{k_j},\nablax \hat\zeta)_{\mathcal{Y}} && \\   
%&\qquad \qquad - \big(M u_{k_j} \cdot q, \div_x(\hphi_{k_j} \nablaq \hat\zeta)\big)_{\mathcal{Y}} +  (M\nablax (\hphi_{k_j} \nablaq \hat\zeta) q,u_{k_j})_{\mathcal{Y}} \Big) \eta(t) \dt,
%\end{aligned}$$
%for all $\hat\zeta \in \hX_{k_j}$ and $\eta \in C_0^\infty(-T,T)$.

%{\color{red}~ I don't understand why it was necessary in the 7 lines, immediately above, to rewrite $(M\omega(u)q,\eta)$ as $\frac12 (M v \cdot q, \div_x \eta)_{\mathcal{Y}} - \frac12 (M\nablax \eta q,v)_{\mathcal{Y}}$. I also don't understand the argument below. What space does $\zeta$ belong to? }

%\medskip

%{\color{blue} Since $u_{k_j} \to u$ strongly in $L^2\big(0,T;L^4(\Omega)\big)$, $u_{k_j} \zeta \to u \zeta$ strongly in $L^2\big(0,T;L^2(\Omega \times D;\R^d)\big)$  and $\nablax \hphi_{k_j} \rightharpoonup \nablax \hphi$ weakly in $L^2(0,T;\hY)$, it follows that
%$$\int_0^T  (M (u_{k_j}\cdot \nablax) \hphi_{k_j}, \hat\zeta)_{\mathcal{Y}} \eta(t) \dt \longrightarrow \int_0^T  (M (u \cdot\nablax) \hphi, \hat\zeta)_{\mathcal{Y}} \eta(t) \dt.$$
%Passing to the limit $k \rightarrow \infty$ in the terms with prefactors $1/\lambda$ and $2\eps$ in the second integral on the right-hand side is straightforward using weak convergence 
%of $\hphi_k$ to $\hphi$ in $\widehat{\mathcal{X}}=L^2(0,T; H^1(\Omega \times D;M))$.


%It remains to deal with the last two terms. For the nonlinearities that couple the Fokker--Planck equation to the Navier--Stokes equation, we note again
%$$\int_0^T  ((u_k \cdot \nablax)\xi,\varphi)_{\mathcal{H}} + ((u_k \cdot \nablax)\varphi,\xi)_{\mathcal{H}} \dt = \int_0^T \int_\Omega \div_x((\xi \cdot \varphi)u_k) \, \text{d}x \dt=0,$$
%for any $\xi,\varphi \in L^2(0,T;\hX)$.}


We use the density of $\cup_{k=1}^\infty \H_{k}$ in $\mathcal{V}$ and of $\cup_{k=1}^\infty \hY_{k}$ in $H^2_M(\Omega \times D)$, which completes the proof by observing that the tuple $(u,\hphi)$ satisfies the variational form of the time-fractional Navier--Stokes--Fokker--Planck system as stated in \cref{Eq:DefWeak}. 

It remains to check that the initial conditions are satisfied. First, we obtain the convergence $u_{k_j}(0) \to u(0)$ in $\mathcal{V}_0'$  as $j \rightarrow \infty$; see again \cref{Eq:Weak}. However, by definition, $u_{k_j}(0)=\Pi_{\H_{k_j}} u^0$, which converges to $u^0$ in $\mathcal{H}_0$
as $j \rightarrow \infty$. By the uniqueness of the limit it follows that $u(0)=u^0$. Regarding the solution of the Fokker--Planck equation, we use again the strong convergence \cref{Eq:Weak} to conclude $(\gb*\hphi)(0) = \hpsi^0$.
\end{proof}


 










 \section{Numerical simulations} \label{Sec:Numerics}



%One might also investigate the mixed system
%\begin{equation} \begin{aligned}
%    \phi &= \ptb \psi \\
%	0 &= \pt \psi -D \Delta \phi + \div (F  \phi)  , \end{aligned} 
%\label{Eq:System5}
%\end{equation}
%with the initial $\psi(0)=\psi^0$. Theoretically, it should hold $\phi^0 = \psi^0 g_\alpha(0)=\infty$. 

Various numerical methods for time-fractional PDEs are summarized in the review article \cite{diethelm2020good} and in the monographs \cite{baleanu2012fractional,owolabi2019numerical,jin2023numerical}. %Since the Caputo derivative $\capb \psi=\ptb (\psi-\psi_0)$ is usually treated in a discrete manner, we follow this approach by adding and subtracting the initial condition $\psi^0$ from the original system as follows
%$$\begin{aligned}\pt \psi - \div(D \nabla \ptb (\psi-\psi_0)) +\div(F \ptb (\psi-\psi_0)) &=f+\div(D\nabla \ptb \psi_0) - \div(F  \ptb \psi_0).% \\
%&=\ga (D \Delta \psi_0 -  F \cdot \nabla \psi_0).
%\end{aligned}$$
%Consequently, we can write the system in terms of the Caputo derivative 
%$$\begin{aligned}\pt \psi - \div(D\nabla \ptb \psi) +\div(F  \ptb \psi) = f+ \ga  \div(D\nabla \psi_0) - \ga  \div(F  \psi_0),
%\end{aligned}$$
%where we further used that $\ptb 1 = \ga$.
%We note that the right-hand side of the PDE is an element in $L^{1/(1-\alpha)-\eps}(0,T;H^{-1}(\Omega))$ for $D,F \in L^\infty(\Omega_T)$, $f\in L^2(0,T;H^{-1}(\Omega))$ and $\psi_0 \in H^1(\Omega)$. We observe that $\alpha> 1/2$ yields a right-hand side in $L^2(0,T;H^{-1}(\Omega))$.
%ga in Lp for a>1-1/p i.e 1/p>1-a i.e 1>p(1-a) i.e. p<1/(1-a)

We assume a discretization $0=t_0<t_1<\dots<t_N=T$ of the time interval $[0,T]$. We do not utilize an equispaced time mesh, but a nonuniform one by discretizing the early times in finer steps. In particular, we assume that the $n$-th time step is of the form $t_n=(n/N)^\gamma T$ for $\gamma\geq 1$. If it holds $\gamma=1$, then we are again in a setting of a uniform mesh, see also Fig. \ref{Fig:Time} for a depiction of some time meshes for various values of $\gamma$.

% Figure environment removed

We discretize the Caputo derivative by the nonuniform L1 scheme \cite[Section 3.2]{diethelm2020good}, i.e., it reads $$\ptb \psi \approx \frac{1}{\Gamma(1+\alpha)} \sum_{j=0}^{n-1} \omega_{n-j-1,n} (\psi_{n-j}-\psi_{n-j-1}),
%=\frac{1}{\Gamma(1+\alpha)} \bigg( \psi_n-\psi_{n-1} + \sum_{j=1}^{n-1} \omega_{n-j-1,n} (\psi_{n-j}-\psi_{n-j-1}) \bigg)
$$ where $\psi_{n-j} \approx \psi(t_{n-j})$. The quadrature weights $\{\omega_{k,n}\}_{k=0}^{n-1}$ are given by the formula
$$\omega_{k,n}=\frac{(t_n-t_k)^{\alpha}-(t_n-t_{k+1})^{\alpha}}{\Delta t_{n-k}},$$
where we introduced the notation $\Delta t_{n-k}=t_{n-k}-t_{n-k-1}$. We use the finite element space $P_1$ for the space discretization and consequently, the fully discrete system reads
\begin{equation} \label{Eq:FP_Discretized} \begin{aligned}& \Big(\frac{\psi^n-\psi^{n-1}}{\Delta t_n},\zeta\Big)_H +  \sum_{j=0}^{n-1} \frac{\omega_{n-j-1,n}}{\Gamma(1+\alpha)}  (D\nabla(\psi_{n-j}-\psi_{n-j-1}),\nabla \zeta)_H \\&\quad - \sum_{j=0}^{n-1}   \frac{\omega_{n-j-1,n}}{\Gamma(1+\alpha)} (\psi_{n-j}-\psi_{n-j-1},F(t_n)\cdot \nabla \zeta)_H   \\ &= (f(t_n),\zeta)_H -  g_\alpha(t_n) \cdot (D(t_n)\nabla \psi_0,\nabla \zeta)_H + g_\alpha(t_n) \cdot  (\psi_0, F(t_n) \cdot\nabla \zeta)_H
\end{aligned} 
\end{equation}
%or written differently by multiplying by $\Delta t$ and  bringing the $\psi_n$-terms to the left-hand side and the remaining terms to the right-hand side
%$$\begin{aligned}& (\psi^n,u)_H + (\Delta t)^{\alpha}   D (\nabla\psi_{n},\nabla u)_H -(\Delta t)^{\alpha}  (\psi_{n},F\cdot \nabla u)_H   \\ &=(\psi^{n-1},u)_H + (\Delta t)^{\alpha}   D (\nabla\psi_0,\nabla u)_H-(\Delta t)^{\alpha}  (\psi_0,F\cdot \nabla u)_H \\ &\quad -\Delta t g_\alpha(t) D(\nabla \psi_0,\nabla u)_H +\Delta t g_\alpha(t) (\psi_0, F \cdot\nabla u)_H \\
%&\quad -(\Delta t)^{\alpha} \sum_{j=1}^{n-1}  D (\nabla(\psi_{n-j}-\psi_0),\nabla u)_H +(\Delta t)^{\alpha} \sum_{j=1}^{n-1}  (\psi_{n-j}-\psi_0,F\cdot \nabla u)_H \Big]
%\end{aligned} $$
for any test function $\zeta$. In particular, taking $\zeta=1$ gives
$$\begin{aligned}& \int_\Omega \psi^n \dx =\int_\Omega \psi^{n-1} \dx    + \Delta t_n \int_\Omega f(t_n) \dx,  
\end{aligned} $$
i.e., the Fokker--Planck setting with $f\equiv 0$ yields discrete mass conservation. We implement the discrete system in open-source computing platform \linebreak FEniCS, see \cite{alnaes2015fenics}.

%$$\begin{aligned}& (\psi^n,u)_H + (\Delta t)^{\alpha} D (\nabla\psi_{n},\nabla u)_H - (\Delta t)^{\alpha} (\psi_{n},F\cdot \nabla u + u\div F)_H   \\ &=(\psi^{n-1},u)_H -\Delta t \cdot g_\alpha(t) \cdot \big(D(\nabla \psi_0,\nabla u)_H + (F \cdot \nabla \psi_0,u)_H\big)
%\\&\quad + (\Delta t)^{\alpha} D (\nabla\psi_0,\nabla u)_H - (\Delta t)^{\alpha} (\psi_0,F\cdot \nabla u + u\div F)_H 
%\\&\quad  - (\Delta t)^{\alpha} \sum_{j=1}^{n-1} \Big[ D (\nabla(\psi_{n-j}-\psi_0),\nabla u)_H -  (\psi_{n-j}-\psi_0,F\cdot \nabla u + u\div F)_H \Big] 
%\end{aligned} $$
 We consider the space interval $\Omega=(-5,15)$ with $\Delta x=1/1024$  and the time interval $[0,T]$ with $T=5$ where the $n$-th time step is given by $t_n=5(n/100)^2$. Moreover, we select as the initial data the Gaussian
$$\psi(0,x)=\psi_0(x)=\frac{1}{\sigma \sqrt{2\pi}} \text{exp}\Big(-\frac12 \Big( \frac{x-\mu}{\sigma} \Big)^2 \Big)$$
for $\sigma=0.1$ and $\mu=2$. 

Regarding model parameters, we choose $D=1$ and $f\equiv 0$. We take the space-time dependent force $F(t,x)=\sin(t)+x$ in Sec. \ref{Sec:Ex2} similar to \cite{angstmann2015generalized,mustapha2022second,le2016numerical,pinto2017numerical}. However, we first consider the case of an absent force $F \equiv 0$ in Sec. \ref{Sec:Ex1}, i.e., we are in the setting of a classical subdiffusion equation. In Sec. \ref{Sec:Ex3}, we consider the physically defeasible time-fractional Fokker--Planck equation with the Caputo derivative on the left-hand side, see again Sec. \ref{Sec:Derivation}, and compare this model numerically to the physically meaningful model that we have analyzed in this work.

\subsection{Example 1: Subdiffusion equation} \label{Sec:Ex1}

As we consider $F\equiv 0$ in this example, we essentially study the time-fractional heat equation
$$\pta \psi(x,t)=\Delta \psi(x,t),$$
which is also referred to as subdiffusion equation.

We observe the typical behavior of a subdiffusive equation in the numerical simulations. At early times, the time-fractional model evolves faster stand the integer-order model. In Fig. \ref{Fig:F0_AlphaVary} (a), we see that the solution is more damped for $\alpha<1$ than for $\alpha=1$ at $t=0.02$. Moreover, the damping is larger for smaller values for $\alpha$. However, this behavior is exactly flipped if one considers a point further in time, e.g. $t=0.5$ as depicted in Fig. \ref{Fig:F0_AlphaVary} (b). After the initial fast evolution of the subdiffusion equation, the process is slower, and we observe that the smallest maximal value is represented by $\alpha=1$ at $t=0.5$. We can also observe that for $\alpha=1$ the typical round shape is present, whereas for $\alpha<1$ the tip at $x=2$ is less round.





% Figure environment removed

% Figure environment removed

% Figure environment removed


We consider the time evolution for $\alpha=1$ in Fig. \ref{Fig:F0_TimeVary} (a) and for $\alpha=\frac12$ in Fig. \ref{Fig:F0_TimeVary} (b). The typical diffusion process can be observed and again, we notice the spikier tip for $\alpha=\frac12$. Moreover, the support of the function is larger for smaller $\alpha$.



Lastly, we try to fit the solution $\psi$ for different values of $\alpha$. The goal is to analyze whether it is necessary to consider the more complicated (analytically and numerically) time-fractional model, or whether this model's behavior can be replicated by an integer-order model.  This is done in Fig. \ref{Fig:F0_Fitting}, and we observe that the subdiffusive behavior cannot be imitated directly by the standard Fokker--Planck equation. Again, we observe the different support for each curve and the difference in the tip at $x=2$.





\subsection{Example 2: Space-time dependent force} \label{Sec:Ex2}
This time, we consider the space-time dependent force $F(x,t)=\sin(x)+t$ and therefore, we study the time-fractional Fokker--Planck equation
$$\begin{aligned}
&\pt \psi(x,t)-\Delta \ptb\psi(x,t)+ \div(F(x,t)\ptb \psi(x,t)) = \ga D\Delta \psi_0 - \ga \div(F\psi_0).
\end{aligned}$$

Again, we observe the typical initial behavior of a subdiffusive equation. At the start, the time-fractional model evolves much faster stand the integer-order model. In Fig. \ref{Fig:F1_AlphaVary} (a), we see that the solution is more damped for $\alpha<1$ than for $\alpha=1$ at $t=0.02$. However, this time, we observe that the symmetry of the probability density functional $\psi$ is lost for $\alpha<1$. In the case of $\alpha=\frac12$ and $\alpha=\frac14$, the solution admits a large support up to the right end of the domain. that  In Fig. \ref{Fig:F1_AlphaVary} (b), we have plotted $\psi$ at a later time.  We observe that $\alpha=1$ is vastly different from the case of $\alpha<1$. This is also pronounced by the fact that $\ga(t) \to 0$ as $t \to \infty$ for $\alpha<1$, but in the case of $\alpha=1$ it holds $\ga(t) \equiv 1$, i.e., the right-hand side is just as large for all times.





% Figure environment removed

% Figure environment removed


We consider the time evolution for $\alpha=1$ in Fig. \ref{Fig:F0_TimeVary} (a) and for $\alpha=\frac12$ in Fig. \ref{Fig:F0_TimeVary} (b). The typical diffusion process can be observed and again, we notice the edgier tip for $\alpha=\frac12$. Moreover, the support of the function is larger for smaller $\alpha$.

Lastly, we try to fit the solution $\psi$ for different values of $\alpha$. This is done in Fig. \ref{Fig:F0_Fitting}, and we observe that the subdiffusive behavior cannot be imitated by an integer-order model. Again, we observe the different support for each curve and the difference in the tip at $x=2$.



% Figure environment removed

\subsection{Example 3: Model comparison} \label{Sec:Ex3}

We consider the model as introduced in \eqref{Eq:ModelWrong} with no right-hand side, i.e.,
\begin{equation} \label{Eq:ModelWrong1} \begin{aligned}
&\pta \psi(x,t)-D \Delta \psi(x,t) +\div \big(F(t,x) \psi(x,t) \big) =0,
\end{aligned}\end{equation}
and we discretize it in the same manner as done for the time-fractional Fokker--Planck equation in \eqref{Eq:FP_Discretized}. Since this model has been studied in literature, we want to give it some attention by comparing it to the physically meaningful model. Again, we consider $F(x,t)=\sin(x)+t$.

We compare it for $\alpha=\frac14$ in Fig. \ref{Fig:Wrong} (a) and for $\alpha=\frac34$ in Fig. \ref{Fig:Wrong} (b) for several time steps. We notice that the error gets larger for increasing time, and it is also more pronounced for smaller values for $\alpha$. We argue that this results from the fact that these models coincide for $\alpha=1$ and by continuity of the fractional parameter, the difference only gets larger the further one is from $\alpha=1$. Moreover, it holds $\ga(t) \to 0$ as $t \to \infty$ for $\alpha<1$ and therefore, it makes sense that asymptotically the right-hand side is negligible.

% Figure environment removed

%\section*{Some sources}

%Good \cite{angstmann2015generalized}: $F=-x+\eps \sin(5\pi t)$. In the case $\eps = 0$, where the external force does not vary in time, the results from the numerical simulations for the first moment and the variance are indistinguishable for the nondelayed forcing and the trap-time delayed forcing, in agreement with the algebraic analysis. The further discussion below is based on the case $\eps = 1$, i.e., the external force varies periodically in time.

%Mustapha A second-order accurate numerical scheme for a time-fractional Fokker-Planck equation: $F=\sin(t)-x$

%PENG The existence of mild and classical solutions for time fractional Fokker–Planck equations

%Pinto Numerical solution of a time-space fractional Fokker Planck equation with variable force field and diffusion: $F(x)=\sin(t)+x$, also in \cite{le2016numerical}

%\cite{le2018a}: A SEMIDISCRETE FINITE ELEMENT APPROXIMATION OF A: $F=-V'$ with $V=x^4/4 -x^2/2 - x \cos t$.

%Deng Numerical algorithm for the time fractional Fokker–Planck equation  $U=\cos x-6x$ and $F=-U'$

%A high-order compact difference method for time fractional: $F=e^{(x-1/2)^2}$

%A Space-Time Petrov-Galerkin Spectral Method for Time Fractional Fokker-Planck Equation with Nonsmooth Solution: $F=-x-1$, also have rhs-force $f$, $F=-1$, assume nonpositive and decreasing, have inhom Dirichlet

%Interval Shannon Wavelet Collocation Method for Fractional Fokker-Planck Equation: $F=-1$, $\psi_0=x(1-x)$, inhom Dir

%Numerical algorithms for the time-space tempered fractional Fokker-Planck equation: $F(x)=x^2$, zero initial, zero Dir and forcing fct and Gaussian initial in other ex and $F(x)=x$


 %\appendix
\section{Derivation of the time-fractional Fokker--Planck equation} \label{App:Derivation}

Let $t$ denote the current time state, and let $x=x(t)$ be the space variable depending on $t$. We consider a particle being dragged through a fluid and are interested in the forces involved.
We exert on the particle an external force $F_\ext$ proportional to the gradient of the potential $U$ it encounters. Typically, it is dependent on space and time.
In addition, it is countered by the fluid's drag force $F_\drag$.
The drag or damping acts as resistance to the particle's motion, which is the outcome of the particle's attempt to shift the fluid out of the way.
A drag force requires motion, and hence it is proportional to the particle's velocity. The friction coefficient $\zeta$ is the proportionality constant that determines the damping magnitude, i.e., \begin{equation}
\label{Eq:Drag}
F_\drag=-\zeta \ddt x.
\end{equation}

Microscopically, we also recognize that the molecules of a fluid impose time-dependent random forces $F_\ran$, such as random collisions, on a molecule.
So that the specific molecular details of solute–solvent collisions can be averaged out, one considers a nanoscale solute in water (e.g., biological macromolecules) with dimensions large enough that its position is simultaneously influenced by multiple solvent molecules, but light enough that the constant interactions with the solvent leave an unbalanced force acting on the solute at any given moment.

According to Newton's second law, the acceleration of this particle is equal to the sum of these forces: 
\begin{equation} \label{Eq:Newton} m \Big(\ddt \Big)^2 x = F_\drag + F_\ext +F_\ran.
\end{equation}
%or simplified
%The drag force is present regardless of whether an external force is present, so in the absence of external forces the equation of motion governing the spontaneous fluctuations of this solute is determined from the forces due to drag and the random fluctuations:
%$$m \Big(\ddt \Big)^2 x + \partial_x U+ \zeta  \ddt  x -F_\ran =0.$$
This equation of motion is called the Langevin equation.
Moreover, since it incorporates a time-dependent random force, we  refer to it as a stochastic equation.
Incorporating a random process into a deterministic equation necessitates the employment of a statistical method to solve the problem.

We shall attempt to describe the particle's average and root-mean-squared position.
What can we say about the random force initially?
Even while there may be instant imbalances, the average perturbations from the solvent on a bigger particle will be zero at equilibrium, i.e., $\langle F_\ran \rangle=0$, notwithstanding the possibility of momentary imbalances.
It appears to imply that the drag force and the random force are independent, but in reality, they stem from the same molecular forces.
%If the molecule of interest is a protein that is exposed to the fluctuations of numerous quickly moving solvent molecules, then the averaged forces due to random fluctuations and the drag forces are connected.
The fluctuation-dissipation theorem \cite[Section 9.2]{pavliotis2014stochastic} is the general link between friction and the random force's correlation function, and in the Markovian limit it reads 
$$\langle  F_\ran(t)F_\ran(t')\rangle=2\zeta  k_B\mu_T\delta(t-t'),$$
where $\mu_T$ denotes the absolute temperature, $k_B$ the Boltzmann constant, and $\delta$ the Dirac-delta functional. Thanks to Einstein's formula, the diffusion constant reads $D=\frac{k_B\mu_T}{\zeta }$, see \cite[Example 9.8]{pavliotis2014stochastic}.
%or
%$$\zeta =\frac{\langle  F_\ran^2\rangle}{2k_B\mu_T}$$
%We note that Markovian indicates that no correlation exists between the random force for $|t-t'| > 0$. 
%More generally, we can recover the friction coefficient from the integral over the correlation function for the random force
%$$\zeta =\frac{1}{2k_B\mu_T}\int_\R\langle F_\ran(0)F_\ran(t)\rangle  \dt$$
%To describe the time evolution of the position of our protein molecule, we would like to obtain an expression for mean-square displacement $\langle x^2(t)\rangle $. The position of the molecule can be described by integrating over its time-dependent velocity: $x(t)=\int_0^t\ddt  x(t') \dt'$
%so we can express the mean-square displacement in terms of the velocity autocorrelation function
%$$\langle x^2(t)\rangle =\int^t_0 \int^t_0 \langle \ddt  x(t') \ddt  x(t'')\rangle\dt''\dt' $$
%Our approach to obtaining $\langle x^2(t)\rangle$  starts by multiplying by $x$ and then ensemble averaging.
%$$m\langle x\ddt \ddt  x\rangle +\zeta \langle x\ddt  x\rangle -\langle xF_\ran(t)\rangle =0$$
%The last term is zero, and from the chain rule we know
%$$\ddt (x\ddt  x)=x\ddt \ddt  x+\ddt x\ddt  x$$
%Therefore, we can write 
%$$m(\ddt \langle x\ddt  x\rangle -\langle \ddt  x\ddt  x\rangle )+\zeta \langle x\ddt  x\rangle =0$$
%Further, the equipartition theorem states that for each translational degree of freedom the kinetic energy is partitioned as
%$$\frac12m\langle (\ddt  x)^2\rangle =\frac{k_B\mu_T}{2}$$
%So,
%$$m\ddt \langle x\ddt  x\rangle +\zeta \langle x\ddt  x\rangle =k_B\mu_T$$
%Here we are describing motion in 1D, but when fluctuations and displacement are included for 3D motion, then we switch $x → r$ and $k_B\mu_T→3k_B\mu_T$. Integrating eq. (13.1.10) twice with respect to time, and using the initial condition $x(0) = 0$, we obtain
%$$\langle x2\rangle =2k_B\mu_T\zeta {t+m\zeta [exp(-\zeta mt)-1]}$$
%in 3D:
%$$\langle r2\rangle =6k_B\mu_T\zeta {t+m\zeta [exp(-\zeta mt)-1]}$$
%Let’s consider two limiting cases. We see that $m/\zeta$ has units of time, and so we define the relaxation time
%$$\tau_C=m/\zeta$$
%and investigate time scale short and long compared to $\tau_C$:
%1) For $t \ll \tau_C$, we can expand the exponential  and retain the first three terms, which leads to
%$$\langle x2\rangle \approx k_B\mu_Tmt2=\langle v2\rangle t2
%$$
%(short time: inertial)
%
%2) For $t\gg \tau_C$
%and the equation is dominated by the leading term:
%$$\langle x2\rangle =2k_B\mu_T\zeta t$$
%(long time: diffusive)
%In the diffusive limit the behavior of the molecule is governed entirely by the fluid, and its mass does not matter. The diffusive limit in a stochastic equation of motion is equivalent to setting $m\to 0$.
%We see that $\tau_C$ is a time-scale separating motion in the inertial and diffusive limits. It is a correlation time for the randomization of the velocity of the particle due to the random fluctuations of the environment.
%For very little friction or short time, the particle moves with traditional deterministic motion $x_\rms= v_\rms t$, where root-mean-square displacement $x_\rms = (\langle x2\rangle)^1/2$ and $v_\rms$ comes from the average translational kinetic energy of the particle. For high-friction or long times, we see diffusive behavior with $x_\rms\sim t^1/2$. Furthermore, by comparing to our earlier continuum result, $\langle x^2\rangle  = 2Dt$, 
%We see that the diffusion constant can be related to the friction coefficient by
%Einstein's formula that reads
%$D=\frac{k_B\mu_T}{\zeta}$. Therefore, the Langevin equation becomes
%$$m (\ddt)^2 x+\partial_x U+\zeta \ddt x-\sqrt{2\zeta k_B\mu_T} R(t)=0.$$

We introduce a Gaussian distributed sequence $R(t)$ of random numbers with $\langle R(t)\rangle=0$ and $\langle R(t)R(t')\rangle=\delta(t-t')$. We are in the setting of Brownian dynamics simulations and therefore, the equation of motion is diffusion-dominated, i.e., in the strong friction limit it holds $|m (\ddt)^2 x|\ll |\zeta \ddt x|$. Therefore, we can neglect inertial motion, and set the acceleration of the particle to zero. Hence, we can insert the representation of the drag force, see \Cref{Eq:Drag} into Newton's second law \Cref{Eq:Newton} to obtain an expression for the velocity of the particle
$$\ddt x=F_\ext(x(t),t)+\sqrt{2 D} R(t)$$
or in the form of a stochastic differential equation
$$\dd X_t=F_\ext(X_t,t) \dt +\sigma \dd W_t,$$
%$\tau_C=m\zeta =mDk_B\mu_T$
%We emphasize the differences in the derivation of the time-fractional NSFP model and examine the steps where the time-fractional derivative is introduced. 
%Later in this section, we formulate suitable assumptions about the quantities that appear in the upcoming PDE model and equip the system with initial and boundary data.
%\subsection{Derivation}
%We consider a bead-spring representation of a polymer molecule in a Newtonian solvent with velocity $u$ and two massless beads connected by a massless elastic spring. Let the position vectors of the centers of mass of the two beads in the dumbbell at time $t$ be $x_i(t) \in \R^d$ for $i \in \{1,2\}$. At time $t$, the center of mass of the system is $x_c(t)=\frac12\big(x_1(t)+x_2(t)\big)$ and the elongation vector from $x_1(t)$ to $x_2(t)$ reads $q_1(t)=x_2(t)-x_1(t)$. We denote the vector pointing in the opposite direction $q_2(t)=-q_1(t)$ and assume that $q_1(t)$ and $q_2(t)$ are contained at all times within a given convex open set $D \subset \R^d$ that fulfills $0 \in D$ and $-q \in D$ whenever $q\in D$.
%Three kinds of forces act on the $i$-th bead in the bead-spring chain in the fluid. Namely, the elastic spring force, the Brownian force thanks to random collisions, and the drag force for the movement of the $i$-th bead through the solvent. Recalling that the beads are assumed to be massless, Newton's second law reads:
%\begin{equation} \label{Eq:Newton} \text{Drag Force} + \text{Spring Force} + \text{Brownian Force}=0.\end{equation} 
%The elastic spring force $F:D \to \R^d$ of the spring connecting the two beads is defined as $$F(q):=HU'(\tfrac12 |q|^2)q,$$ where $U$ is a nonnegative continuously differentiable potential, and $H$ is the spring constant.
%In the case of Hookean dumbbells, the spring force is given by $F(q)=Hq$ with $q\in D=\R^d$ and the corresponding potential reads $U(s)=s$ for $s \in [0,\infty)$. In the case of a FENE-type model, one is in the setting $$D=B_{|q_\text{max}|}(0), \quad F(q)=\frac{Hq}{1-|q|^2/|q_\text{max}|^2}, \quad U(s)=-\frac{|q_\text{max}|^2}{2}\ln\!\bigg(1-\frac{2s}{|q_\text{max}|^2}\bigg)$$ for $q\in D$ and $s\in [0,\tfrac{|q_\text{max}|^2}{2})$ where $|q_\text{max}|>0$ is the maximal extension to which a dumbbell can be stretched.
where $W_t$ is the corresponding Wiener process, and we defined $\sigma:= \sqrt{2D}$. We note that $B_t$ is a Brownian force with $B_t\dt=\sigma \dd W_t$.

%Moreover, $\zeta $ denotes the drag coefficient and the drag force is governed by the Stokes law, which in a fluid moving with velocity $u(x,t)$ reads $\zeta \big(u(x,t)-\ddt  x(t)\big)$.
%Taking the assumptions of the forces into consideration, Newton's second law reads
%$$\zeta  (\dd x(t) - u(x(t),t) \dt) = F(q(t)) \dt + \sqrt{2k_B \mu_T \zeta } \dd W(t).$$
%We divide this equation by $\zeta $ and define
%We define
%\begin{equation} \label{Def:Vectorb} X(t):=x(t), \quad b(X(t),t):=F_\ext(x(t),t), \quad ,\end{equation}
%which yields the stochastic differential equation 
%\begin{equation*} %
%\dd X(t) = b(X(t),t) \dt + \sigma \dd W(t). \end{equation*}


At this point, one can derive the typical Fokker--Planck equation,
see \cite[Section 5.3.2]{pavliotis2014stochastic}. However, we consider the subordination of the Langevin equation, i.e., %
the inverse of the subordinator becomes the operational time of the system. Physically, this means that one introduces trapping events to the motion of the particles.
The parent process $X$ is assumed to satisfy the following subordinated Langevin equation: \begin{equation} \label{Eq:Langevin} \dd X_t = F_\ext\big(X_t,U^\alpha_t\big) \dt + \sigma \dd W_t. \end{equation} 
Here, $U^\alpha_t$ is defined as a L\'evy process with nonnegative increments, see \cite[Definition 21.4, Example 21.7]{sato1999levy}, and
the inverse of $U^\alpha_t$ reads $S^\alpha_t:=\inf\{\tau:U^\alpha_\tau > t\}$. Its Laplace transform reads, see \cite[Example 24.12]{sato1999levy},  %
$$\langle  e^{-nS^\alpha_t}\rangle=E_\alpha(-nt^\alpha)=\sum_{j=0}^\infty \frac{(-n^2t^\alpha)^j}{\Gamma(j\alpha+1)},$$
where $E_\alpha$ is called the Mittag--Leffler function, see \cite{diethelm2010analysis}.
This step is essential to the derivation of the time-fractional PDE and explains why the fractional derivative appears in the mathematical model.
The Mittag-Leffler function has been used as a phenomenological physical model for exponential relaxation at short times and power law relaxation at long times; for further information, consult the book \cite{west2003physics} on the physics of fractional operators. 

We integrate the subordinated Langevin equation \cref{Eq:Langevin} on the time interval $(0,t)$, and like this, the subordinated process $Y^\alpha_t:=X_{S^\alpha_t}$ satisfies the integral equation
$$Y^\alpha_t=\int_0^t F_\ext(X_\tau,\tau) \dd S^\alpha_\tau + \sigma W_{S^\alpha_t}.$$
%where the Lebesgue--Stieltjes integral of $b$ is meant in the sense that each component of $b$ is integrated, and the result is again a vector. 
As proved in \cite[Proof of Theorem 1]{magdziarz2009stochastic}, the Fourier transform of $Y^\alpha_t$ is holomorphic in a neighborhood of zero and therefore, it is well-known that the moments uniquely determine the distribution, e.g., see \cite[Section VII.3]{feller1971introduction}. The moments are computed in \cite[Equation (11)]{sokolov2006field} and \cite[Proof of Theorem 1]{magdziarz2009stochastic}. Finally, we can conclude that the moment $\psi(t):=\langle  Y^\alpha_t \rangle$  satisfies 
\begin{equation} \label{Eq:DerivFP}
\pt \psi(x,t)= -\partial_x \left(F_\ext(x,t) \ptb \psi(x,t) \right) + D \partial_{x}^2 \ptb \psi(x,t),\end{equation}
which we call the time-fractional Fokker--Planck equation.
%%%%%%%%%%%% Section 1 %%%%%%%%%%%%%%%%%%%%%%%%%%
\bibliography{literature.bib}
	\bibliographystyle{spmpsci.bst} 

\bigskip  %%%%%%%%%%%%%%%%%%%%%%%%%%%%%%%%%

\small %%%
\noindent
{\bf Publisher's Note}
Springer Nature remains neutral with regard to jurisdictional claims in published maps and institutional affiliations.


\end{document}