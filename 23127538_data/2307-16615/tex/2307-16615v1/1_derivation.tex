	\section{Modeling of the time-fractional Fokker--Planck equation} \label{Sec:Derivation}
	


%The time-fractional Fokker--Planck model can be derived by utilizing Langevin equations, we refer to Appendix \ref{App:Derivation} at the end of the article for the details on the equation's derivation.
%Historically, three years after Einstein's publication on Brownian motion, the French scientist Paul Langevin published his study \cite{langevin1908theory} on Brownian motion, which obtained the identical results utilizing an entirely new mathematical framework. In fact, Langevin's method is far closer to the conventional physics approach: To account for the unpredictable “kicks” that the Brownian particle receives from smaller fluid particles, one alters Newton's equation of motion by adding a randomly fluctuating force. The resulting equation is what we would identify as a stochastic differential equation in the present day. In fact, it turns out that its moment is governed by the Fokker--Planck equation.


 Let $\Omega \subset \R^d$, $d\in \mathbb{N}$, be a Lipschitz domain and $T<\infty$ a fixed final time. Shortly, we denote the time-space domain by $\Omega_T=\Omega \times (0,T)$. Let $\psi:\Omega_T \to \R$ denote a probability density function that represents the probability at a time $t$ of finding the center of mass of a particle in the volume element $x+\d x$. 


The time-fractional Fokker--Planck model with space-time dependent force can be derived by utilizing the Langevin equations, see \cite{magdziarz2008equivalence,magdziarz2009stochastic}, and the model reads
\begin{equation} \label{Eq:DerivFP}
\pt \psi(x,t)- D \Delta \Ptb \psi(x,t)+\div\!\left(F(x,t) \Ptb \psi(x,t) \right)  = 0.\end{equation}
Here, $F:\Omega_T  \to \R^d$ denotes the space-time dependent external force and $D$ the diffusion coefficient. In contrast to the typical model of integer-order, the fractional derivative in the sense of Riemann--Liouville is introduced, which is defined by
$$\Ptb u(t) =\frac{1}{\Gamma(\alpha)} \ddt \int_0^t \frac{u(s)}{(t-s)^{1-\alpha}} \ds, $$
where $\Gamma$ denotes Euler's Gamma function. We introduce the singular kernel function $g_\alpha(t)=t^{\alpha-1}/\Gamma(\alpha)$ and therefore, we can rewrite the fractional derivative with the convolution operator as $$\Ptb u =\pt (\ga*u).$$ In the limit case of $\alpha=1$, the model is reduced to the standard Fokker--Planck equation.
This time-fractional model has been studied in the previous works \cite{huang2020new,le2016numerical,le2018semidiscrete,le2021alpha,mclean2021uniform,mustapha2022second,pinto2017numerical,yan2019finite} with regard to numerical methods and in \cite{le2019existence} for the existence of mild and classical solutions. 

We note that the fractional derivative in the sense of Riemann--Liouville appears naturally in the equation's derivation, see \cite{magdziarz2008equivalence}. However, the fractional derivative in the sense of Caputo would be preferable considering our variational approach to time-fractional partial differential equations and the involved analytical machinery. The Caputo derivative of order $\alpha$ is denoted by $\pta$ and it reads 
\begin{equation} \label{Eq:Caputo} \pta u= \Pta (u-u_0).
\end{equation} 
Here, $u_0$ is the initial of the underlying system, which shall fulfill $$\big(\gb*(u-u_0)\big)(0)=0$$ in the case that $u$ is not continuous. 

If the force is time-independent, we could simply convolve the time-fractional Fokker--Planck equation \eqref{Eq:DerivFP} with the singular kernel function $\gb$ and exploit the properties $\gb*\Ptb u=u$ and $\pta u=\gb*\pt u$, see below in Section \ref{Sec:Prelim}, to obtain the time-fractional equation
\begin{equation} \label{Eq:DerivFP3}
\pta \psi(x,t)-D \Delta \psi(x,t) +\div \big(F(x) \psi(x,t) \big)  =0,\end{equation}
which would be more accessible for analytical and numerical methods. However, we cannot simply exclude the relevant cases of time-dependent forces. In such cases, one would require a product rule for fractional derivatives to write $F\Ptb \psi$ as $\Ptb(F\psi)-\Ptb F \psi$.  However, this is not correct for fractional derivatives, as it can be already seen from the example $\psi=F=1$. Then it holds 
$$F\Ptb \psi = \ga \neq 0 = \ga-\ga =\Ptb(F\psi)-\Ptb F \psi.$$
There is a fractional version of the Leibniz rule that requires two smooth functions $f,g$ and reads \cite[Theorem 2.18]{diethelm2010analysis}
$$\Pta(fg)=f \Pta g + \sum_{k=1}^\infty \binom{\alpha}{k} \pt^k f \cdot (g_{1-k+\alpha}*g).$$
We can already see the issue of this formula. It requires smooth functions, and it turns out that there is an infinite sum on the right-hand side. Let us assume that $F$ and $\psi$ are smooth. Then we want to bring the fractional derivative in front of $F\psi$ by the formula
$$F \Ptb \psi = \Ptb(F\psi) - \sum_{k=1}^\infty \binom{1-\alpha}{k} \pt^k F \cdot (g_{2-k-\alpha}*\psi).$$
Afterward, we convolve the system with $\gb$ and obtain the system
\begin{equation} \label{Eq:ModelWrong} \begin{aligned}
&\pta \psi(x,t)-D \Delta \psi(x,t) +\div \big(F(t,x) \psi(x,t) \big)  \\&=\sum_{k=1}^\infty \binom{1-\alpha}{k} \gb*\big(\pt^k F \cdot (g_{2-k-\alpha}*\psi)\big),
\end{aligned}\end{equation}
There have been several published articles that studied this model but neglecting the complete right-hand side. This is also the reason it is claimed in \cite{heinsalu2007use} that such a model (with neglecting the right-hand side) is “physically defeasible” and its solution “does not correspond to a physical stochastic process”.  In the case that $F$ is affine linear in $t$, i.e. $F(t,x)=a(x)+b(x)t$,  it yields 
\begin{equation*} \begin{aligned}
&\pta \psi(x,t)-D \Delta \psi(x,t) +\div \big(F(t,x) \psi(x,t) \big)  \\&=(1-\alpha) \cdot b(x) \cdot (g_{2-2\alpha}* \psi)(t)
\end{aligned}\end{equation*}
 We would rather not consider infinitely many terms on the right-hand side of the PDE for a general $F$ and therefore, we instead 
exploit the definition \eqref{Eq:Caputo} of the Caputo derivative to obtain
$$\Ptb u(t) = \ptb u(t) + \Ptb u_0=\ptb u(t)+u_0g_{\alpha}(t),$$
and rewrite the time-fractional Fokker--Planck equation \eqref{Eq:DerivFP} as follows:
\begin{equation} \label{Eq:DerivFP2}
\begin{aligned}
&\pt \psi(x,t)-D \Delta \ptb \psi(x,t) +\div\big(F(x,t) \ptb \psi(x,t) \big) \\ &=\ga D\Delta  \psi_0 - \ga \div(F\psi_0).
\end{aligned} \end{equation}
We consider an initial condition $\psi_0 \in H_0^1(\Omega)$ and therefore, it holds that the right-hand side has the regularity $L^p(0,T;H^{-1}(\Omega))$ with $p<1/(1-\alpha).$
%We follow this approach since, in our opinion, it is more accessible and easier to understand in which way the fractional derivative enters than the CTRW approach
%We translate the $(x,q)$-coordinates to the center of mass by defining the new function $\psi(x,q,t):=\tilde\psi(x-\tfrac12 q,x+\tfrac12 q,t)$. Together with the definition of $b$, see \cref{Def:Vectorb}, we obtain
%\begin{equation} \label{Eq:Langevin3} \begin{aligned}&\pt \psi(x,q,t) + \div \bigg(\frac{u(x-\tfrac12q,t)+u(x+\tfrac12q,t)}{2} \ptb\psi(x,q,t)\bigg)\\ &\quad+ \div\bigg(\big(u(x+\tfrac12q,t)-u(x-\tfrac12q,t)\big) \ptb\psi(x,q,t) - \frac{2F(q)}{\zeta }  \ptb\psi(x,q,t)\bigg) \\ &= \frac{k_B\mu_T}{2\zeta } \Delta_x \ptb\psi(x,q,t)+ \frac{2k_B\mu_T}{\zeta } \Delta_x \ptb\psi(x,q,t). \end{aligned} \end{equation}
%We make an assumption of local homogeneity, i.e., the velocity's variation over the microscopic length scale of a single dumbbell is assumed to be small. Consequently, the arithmetic mean $\big(u(x-\frac12q,t)+u(x+\frac12q,t)\big)/2$ can be approximated by $u(x,t)$, which is relevant for the second term on the left-hand side of \cref{Eq:Langevin3}.  In case of the third term, we use Taylor's expansion to obtain 
%\begin{equation} \begin{aligned} u(x+\tfrac12 q,t)-u(x-\tfrac12 q,t) &= \nabla u(x,t) q + \mathcal{O}(|q|^3) \\ &=\Big(\sigma\big(u(x,t)\big) + \omega\big(u(x,t)\big)  \Big)q + \mathcal{O}(|q|^3),
	%%	\end{aligned}
	%\label{Eq:ApproxTaylor}
	%\end{equation}
%where we have further split the gradient of $u$ into its symmetric and antisymmetic part, which are defined as follows:
% \begin{equation} \label{Eq:Omega} \sigma(u)=\frac{\nabla u + (\nabla u)^\top}{2}, \qquad \omega(u)=\frac{\nabla u - (\nabla u)^\top}{2}.
 %	\end{equation}
%In the approximation \cref{Eq:ApproxTaylor}, we omit the $\mathcal{O}(|q|^3)$ term and further, we also omit the symmetric part of $u$ for analytical simplicity, i.e., we consider the corotational case. Thus, we obtain the time-fractional PDE
%\begin{equation} \label{Eq:FokkerDim}\begin{aligned}\pt \psi + \div(u \ptb\psi) + \div\Big(\omega( u) q \ptb\psi - \frac{2F(q)}{\zeta }  \ptb\psi\Big) \\ = \frac{k_B \mu_T}{2\zeta } \Delta_x \ptb\psi + \frac{2k_B\mu_T}{\zeta } \Delta_x \ptb\psi.\end{aligned} \end{equation}
%Lastly, we put the equation into its nondimensionalized form. Therefore, we define quantities
%$$q=l_0 \hat q$$
%with microscopic length scale %$l_0=\sqrt{k_B \mu_T/H}$. Further, we introduce the nondimensional Weissenberg number $\lambda=\zeta  U_0/(4HL_0)$ that reflects the ratio of the microscopic to the macroscopic time-scale.
%We multiply \cref{Eq:FokkerDim} by $L_0/U_0$ and obtain (omitting the hats over the symbols for the ease of readability)
%
	%\begin{aligned} 
 %&\pt \psi =\tfrac{1}{2\lambda} %\div(\nabla\ptb \psi + U'q \ptb\psi),
%\end{aligned}\end{equation} 
%where we have further defined the parameter $\eps:=l_0^2/(8\lambda L_0^2)$. %
%Finally, the Navier--Stokes equations governing the velocity $u$ are derived in a standard manner from the conservation of momentum and mass equations, see \cite{temam2001navier}. We note that polymeric fluids are non-Newtonian fluids and the polymer molecules contribute
%a symmetric extra stress tensor $\tau(x,t)$ to the total stress tensor while the equations for the conservation
%of linear momentum and mass remain the same. %
%Because the continuum mechanical “macroscopic” equations of incompressible fluid flow are coupled to a “microscopic” model, we call the polymer models under consideration microscopic–macroscopic-type models. Here, the microscropic equations are governed by the time-fractional Fokker--Planck equation describing the statistical properties of particles in the continuum. We begin by presenting these equations and collecting the relevant assumptions on the various parameters featuring in the model.
%We generalize the time-fractional Fokker--Planck equation   by assuming a space-time varying diffusion function $D:\Omega \times (0,T) \to \R^d$ and an additional force $f:\Omega \times (0,T) \to \R$ and, therefore, we study in this work the following time-fractional advection-diffusion equation:
%\begin{equation} \label{Def:FP} \begin{aligned}
%	&\pt \psi -  \div(D(x,t)\nabla  \ptb  \psi) + \div (F(x,t) \ptb  \psi)  \\  &=f(x,t) + \ga \div(D\nabla \psi_0) - \ga \div(F\psi_0).\end{aligned} \end{equation}
We equip this equation with the homogeneous Dirichlet boundary condition $\psi=0$ on $\p\Omega$. However, our analytical results also hold for no-flux boundary conditions (i.e. homogeneous Neumann). Moreover, the system is equipped
with the initial condition $\psi(0) = \psi^0 \geq 0$ in  $\Omega$. %Mathematically, it is enough to demand $\psi_0 \in H_0^1(\Omega)$ to show our desired results on the well-posedness of weak solutions. 
Physically, $\psi^0$ is a given probability density function, i.e.,  it is nonnegative function and satisfies $\int_\Omega \psi^0(x) \dx =1$ (however, we do not need to assume such properties in our well-posedness theorem below).   
Integrating the time-fractional Fokker--Planck equation  in $\Omega$  and employing integration by parts, we find
$\ddt \int_{\Omega} \psi(x,t) \dx =0.$ This implies then $\int_{\Omega} \psi(x,t) \dx = 1$  for almost all $t$.





