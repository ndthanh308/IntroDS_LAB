\section{Numerical simulations} \label{Sec:Numerics}



%One might also investigate the mixed system
%\begin{equation} \begin{aligned}
%    \phi &= \ptb \psi \\
%	0 &= \pt \psi -D \Delta \phi + \div (F  \phi)  , \end{aligned} 
%\label{Eq:System5}
%\end{equation}
%with the initial $\psi(0)=\psi^0$. Theoretically, it should hold $\phi^0 = \psi^0 g_\alpha(0)=\infty$. 

Various numerical methods for time-fractional PDEs are summarized in the review article \cite{diethelm2020good} and in the monographs \cite{baleanu2012fractional,owolabi2019numerical,jin2023numerical}. %Since the Caputo derivative $\capb \psi=\ptb (\psi-\psi_0)$ is usually treated in a discrete manner, we follow this approach by adding and subtracting the initial condition $\psi^0$ from the original system as follows
%$$\begin{aligned}\pt \psi - \div(D \nabla \ptb (\psi-\psi_0)) +\div(F \ptb (\psi-\psi_0)) &=f+\div(D\nabla \ptb \psi_0) - \div(F  \ptb \psi_0).% \\
%&=\ga (D \Delta \psi_0 -  F \cdot \nabla \psi_0).
%\end{aligned}$$
%Consequently, we can write the system in terms of the Caputo derivative 
%$$\begin{aligned}\pt \psi - \div(D\nabla \ptb \psi) +\div(F  \ptb \psi) = f+ \ga  \div(D\nabla \psi_0) - \ga  \div(F  \psi_0),
%\end{aligned}$$
%where we further used that $\ptb 1 = \ga$.
%We note that the right-hand side of the PDE is an element in $L^{1/(1-\alpha)-\eps}(0,T;H^{-1}(\Omega))$ for $D,F \in L^\infty(\Omega_T)$, $f\in L^2(0,T;H^{-1}(\Omega))$ and $\psi_0 \in H^1(\Omega)$. We observe that $\alpha> 1/2$ yields a right-hand side in $L^2(0,T;H^{-1}(\Omega))$.
%ga in Lp for a>1-1/p i.e 1/p>1-a i.e 1>p(1-a) i.e. p<1/(1-a)

We assume a discretization $0=t_0<t_1<\dots<t_N=T$ of the time interval $[0,T]$. We do not utilize an equispaced time mesh, but a nonuniform one by discretizing the early times in finer steps. In particular, we assume that the $n$-th time step is of the form $t_n=(n/N)^\gamma T$ for $\gamma\geq 1$. If it holds $\gamma=1$, then we are again in a setting of a uniform mesh, see also Fig. \ref{Fig:Time} for a depiction of some time meshes for various values of $\gamma$.

% Figure environment removed

We discretize the Caputo derivative by the nonuniform L1 scheme \cite[Section 3.2]{diethelm2020good}, i.e., it reads $$\ptb \psi \approx \frac{1}{\Gamma(1+\alpha)} \sum_{j=0}^{n-1} \omega_{n-j-1,n} (\psi_{n-j}-\psi_{n-j-1}),
%=\frac{1}{\Gamma(1+\alpha)} \bigg( \psi_n-\psi_{n-1} + \sum_{j=1}^{n-1} \omega_{n-j-1,n} (\psi_{n-j}-\psi_{n-j-1}) \bigg)
$$ where $\psi_{n-j} \approx \psi(t_{n-j})$. The quadrature weights $\{\omega_{k,n}\}_{k=0}^{n-1}$ are given by the formula
$$\omega_{k,n}=\frac{(t_n-t_k)^{\alpha}-(t_n-t_{k+1})^{\alpha}}{\Delta t_{n-k}},$$
where we introduced the notation $\Delta t_{n-k}=t_{n-k}-t_{n-k-1}$. We use the finite element space $P_1$ for the space discretization and consequently, the fully discrete system reads
\begin{equation} \label{Eq:FP_Discretized} \begin{aligned}& \Big(\frac{\psi^n-\psi^{n-1}}{\Delta t_n},\zeta\Big)_H +  \sum_{j=0}^{n-1} \frac{\omega_{n-j-1,n}}{\Gamma(1+\alpha)}  (D\nabla(\psi_{n-j}-\psi_{n-j-1}),\nabla \zeta)_H \\&\quad - \sum_{j=0}^{n-1}   \frac{\omega_{n-j-1,n}}{\Gamma(1+\alpha)} (\psi_{n-j}-\psi_{n-j-1},F(t_n)\cdot \nabla \zeta)_H   \\ &= (f(t_n),\zeta)_H -  g_\alpha(t_n) \cdot (D(t_n)\nabla \psi_0,\nabla \zeta)_H + g_\alpha(t_n) \cdot  (\psi_0, F(t_n) \cdot\nabla \zeta)_H
\end{aligned} 
\end{equation}
%or written differently by multiplying by $\Delta t$ and  bringing the $\psi_n$-terms to the left-hand side and the remaining terms to the right-hand side
%$$\begin{aligned}& (\psi^n,u)_H + (\Delta t)^{\alpha}   D (\nabla\psi_{n},\nabla u)_H -(\Delta t)^{\alpha}  (\psi_{n},F\cdot \nabla u)_H   \\ &=(\psi^{n-1},u)_H + (\Delta t)^{\alpha}   D (\nabla\psi_0,\nabla u)_H-(\Delta t)^{\alpha}  (\psi_0,F\cdot \nabla u)_H \\ &\quad -\Delta t g_\alpha(t) D(\nabla \psi_0,\nabla u)_H +\Delta t g_\alpha(t) (\psi_0, F \cdot\nabla u)_H \\
%&\quad -(\Delta t)^{\alpha} \sum_{j=1}^{n-1}  D (\nabla(\psi_{n-j}-\psi_0),\nabla u)_H +(\Delta t)^{\alpha} \sum_{j=1}^{n-1}  (\psi_{n-j}-\psi_0,F\cdot \nabla u)_H \Big]
%\end{aligned} $$
for any test function $\zeta$. In particular, taking $\zeta=1$ gives
$$\begin{aligned}& \int_\Omega \psi^n \dx =\int_\Omega \psi^{n-1} \dx    + \Delta t_n \int_\Omega f(t_n) \dx,  
\end{aligned} $$
i.e., the Fokker--Planck setting with $f\equiv 0$ yields discrete mass conservation. We implement the discrete system in open-source computing platform \linebreak FEniCS, see \cite{alnaes2015fenics}.

%$$\begin{aligned}& (\psi^n,u)_H + (\Delta t)^{\alpha} D (\nabla\psi_{n},\nabla u)_H - (\Delta t)^{\alpha} (\psi_{n},F\cdot \nabla u + u\div F)_H   \\ &=(\psi^{n-1},u)_H -\Delta t \cdot g_\alpha(t) \cdot \big(D(\nabla \psi_0,\nabla u)_H + (F \cdot \nabla \psi_0,u)_H\big)
%\\&\quad + (\Delta t)^{\alpha} D (\nabla\psi_0,\nabla u)_H - (\Delta t)^{\alpha} (\psi_0,F\cdot \nabla u + u\div F)_H 
%\\&\quad  - (\Delta t)^{\alpha} \sum_{j=1}^{n-1} \Big[ D (\nabla(\psi_{n-j}-\psi_0),\nabla u)_H -  (\psi_{n-j}-\psi_0,F\cdot \nabla u + u\div F)_H \Big] 
%\end{aligned} $$
 We consider the space interval $\Omega=(-5,15)$ with $\Delta x=1/1024$  and the time interval $[0,T]$ with $T=5$ where the $n$-th time step is given by $t_n=5(n/100)^2$. Moreover, we select as the initial data the Gaussian
$$\psi(0,x)=\psi_0(x)=\frac{1}{\sigma \sqrt{2\pi}} \text{exp}\Big(-\frac12 \Big( \frac{x-\mu}{\sigma} \Big)^2 \Big)$$
for $\sigma=0.1$ and $\mu=2$. 

Regarding model parameters, we choose $D=1$ and $f\equiv 0$. We take the space-time dependent force $F(t,x)=\sin(t)+x$ in Sec. \ref{Sec:Ex2} similar to \cite{angstmann2015generalized,mustapha2022second,le2016numerical,pinto2017numerical}. However, we first consider the case of an absent force $F \equiv 0$ in Sec. \ref{Sec:Ex1}, i.e., we are in the setting of a classical subdiffusion equation. In Sec. \ref{Sec:Ex3}, we consider the physically defeasible time-fractional Fokker--Planck equation with the Caputo derivative on the left-hand side, see again Sec. \ref{Sec:Derivation}, and compare this model numerically to the physically meaningful model that we have analyzed in this work.

\subsection{Example 1: Subdiffusion equation} \label{Sec:Ex1}

As we consider $F\equiv 0$ in this example, we essentially study the time-fractional heat equation
$$\pta \psi(x,t)=\Delta \psi(x,t),$$
which is also referred to as subdiffusion equation.

We observe the typical behavior of a subdiffusive equation in the numerical simulations. At early times, the time-fractional model evolves faster stand the integer-order model. In Fig. \ref{Fig:F0_AlphaVary} (a), we see that the solution is more damped for $\alpha<1$ than for $\alpha=1$ at $t=0.02$. Moreover, the damping is larger for smaller values for $\alpha$. However, this behavior is exactly flipped if one considers a point further in time, e.g. $t=0.5$ as depicted in Fig. \ref{Fig:F0_AlphaVary} (b). After the initial fast evolution of the subdiffusion equation, the process is slower, and we observe that the smallest maximal value is represented by $\alpha=1$ at $t=0.5$. We can also observe that for $\alpha=1$ the typical round shape is present, whereas for $\alpha<1$ the tip at $x=2$ is less round.





% Figure environment removed

% Figure environment removed

% Figure environment removed


We consider the time evolution for $\alpha=1$ in Fig. \ref{Fig:F0_TimeVary} (a) and for $\alpha=\frac12$ in Fig. \ref{Fig:F0_TimeVary} (b). The typical diffusion process can be observed and again, we notice the spikier tip for $\alpha=\frac12$. Moreover, the support of the function is larger for smaller $\alpha$.



Lastly, we try to fit the solution $\psi$ for different values of $\alpha$. The goal is to analyze whether it is necessary to consider the more complicated (analytically and numerically) time-fractional model, or whether this model's behavior can be replicated by an integer-order model.  This is done in Fig. \ref{Fig:F0_Fitting}, and we observe that the subdiffusive behavior cannot be imitated directly by the standard Fokker--Planck equation. Again, we observe the different support for each curve and the difference in the tip at $x=2$.





\subsection{Example 2: Space-time dependent force} \label{Sec:Ex2}
This time, we consider the space-time dependent force $F(x,t)=\sin(x)+t$ and therefore, we study the time-fractional Fokker--Planck equation
$$\begin{aligned}
&\pt \psi(x,t)-\Delta \ptb\psi(x,t)+ \div(F(x,t)\ptb \psi(x,t)) = \ga D\Delta \psi_0 - \ga \div(F\psi_0).
\end{aligned}$$

Again, we observe the typical initial behavior of a subdiffusive equation. At the start, the time-fractional model evolves much faster stand the integer-order model. In Fig. \ref{Fig:F1_AlphaVary} (a), we see that the solution is more damped for $\alpha<1$ than for $\alpha=1$ at $t=0.02$. However, this time, we observe that the symmetry of the probability density functional $\psi$ is lost for $\alpha<1$. In the case of $\alpha=\frac12$ and $\alpha=\frac14$, the solution admits a large support up to the right end of the domain. that  In Fig. \ref{Fig:F1_AlphaVary} (b), we have plotted $\psi$ at a later time.  We observe that $\alpha=1$ is vastly different from the case of $\alpha<1$. This is also pronounced by the fact that $\ga(t) \to 0$ as $t \to \infty$ for $\alpha<1$, but in the case of $\alpha=1$ it holds $\ga(t) \equiv 1$, i.e., the right-hand side is just as large for all times.





% Figure environment removed

% Figure environment removed


We consider the time evolution for $\alpha=1$ in Fig. \ref{Fig:F0_TimeVary} (a) and for $\alpha=\frac12$ in Fig. \ref{Fig:F0_TimeVary} (b). The typical diffusion process can be observed and again, we notice the edgier tip for $\alpha=\frac12$. Moreover, the support of the function is larger for smaller $\alpha$.

Lastly, we try to fit the solution $\psi$ for different values of $\alpha$. This is done in Fig. \ref{Fig:F0_Fitting}, and we observe that the subdiffusive behavior cannot be imitated by an integer-order model. Again, we observe the different support for each curve and the difference in the tip at $x=2$.



% Figure environment removed

\subsection{Example 3: Model comparison} \label{Sec:Ex3}

We consider the model as introduced in \eqref{Eq:ModelWrong} with no right-hand side, i.e.,
\begin{equation} \label{Eq:ModelWrong1} \begin{aligned}
&\pta \psi(x,t)-D \Delta \psi(x,t) +\div \big(F(t,x) \psi(x,t) \big) =0,
\end{aligned}\end{equation}
and we discretize it in the same manner as done for the time-fractional Fokker--Planck equation in \eqref{Eq:FP_Discretized}. Since this model has been studied in literature, we want to give it some attention by comparing it to the physically meaningful model. Again, we consider $F(x,t)=\sin(x)+t$.

We compare it for $\alpha=\frac14$ in Fig. \ref{Fig:Wrong} (a) and for $\alpha=\frac34$ in Fig. \ref{Fig:Wrong} (b) for several time steps. We notice that the error gets larger for increasing time, and it is also more pronounced for smaller values for $\alpha$. We argue that this results from the fact that these models coincide for $\alpha=1$ and by continuity of the fractional parameter, the difference only gets larger the further one is from $\alpha=1$. Moreover, it holds $\ga(t) \to 0$ as $t \to \infty$ for $\alpha<1$ and therefore, it makes sense that asymptotically the right-hand side is negligible.

% Figure environment removed

%\section*{Some sources}

%Good \cite{angstmann2015generalized}: $F=-x+\eps \sin(5\pi t)$. In the case $\eps = 0$, where the external force does not vary in time, the results from the numerical simulations for the first moment and the variance are indistinguishable for the nondelayed forcing and the trap-time delayed forcing, in agreement with the algebraic analysis. The further discussion below is based on the case $\eps = 1$, i.e., the external force varies periodically in time.

%Mustapha A second-order accurate numerical scheme for a time-fractional Fokker-Planck equation: $F=\sin(t)-x$

%PENG The existence of mild and classical solutions for time fractional Fokker–Planck equations

%Pinto Numerical solution of a time-space fractional Fokker Planck equation with variable force field and diffusion: $F(x)=\sin(t)+x$, also in \cite{le2016numerical}

%\cite{le2018a}: A SEMIDISCRETE FINITE ELEMENT APPROXIMATION OF A: $F=-V'$ with $V=x^4/4 -x^2/2 - x \cos t$.

%Deng Numerical algorithm for the time fractional Fokker–Planck equation  $U=\cos x-6x$ and $F=-U'$

%A high-order compact difference method for time fractional: $F=e^{(x-1/2)^2}$

%A Space-Time Petrov-Galerkin Spectral Method for Time Fractional Fokker-Planck Equation with Nonsmooth Solution: $F=-x-1$, also have rhs-force $f$, $F=-1$, assume nonpositive and decreasing, have inhom Dirichlet

%Interval Shannon Wavelet Collocation Method for Fractional Fokker-Planck Equation: $F=-1$, $\psi_0=x(1-x)$, inhom Dir

%Numerical algorithms for the time-space tempered fractional Fokker-Planck equation: $F(x)=x^2$, zero initial, zero Dir and forcing fct and Gaussian initial in other ex and $F(x)=x$

