\section{Mathematical preliminaries} \label{Sec:Prelim}
In this part, we present some important concepts and conclusions addressing fractional derivatives.
For instance, we provide a fractional version of the Aubin--Lions lemma and a suitable Gronwall lemma. These are important results used in Galerkin-based proofs for showing the existence of weak solutions to partial differential equations.

%p(a-1)>-1 i.e
%int t^{-1} = t^0

%\subsection{Fractional derivatives}
Let $T<\infty$ be a fixed final time. We have already defined the singular kernel function in the previous section by $g_\alpha(t)=t^{\alpha-1}/\Gamma(\alpha)$, $t \in (0,T)$, $\alpha >0$. We can extend the definition to the limit case of $\alpha=0$ by $g_0=\delta$. We observe that it holds  $g_\alpha \in L^p(0,T)$ for any $\alpha>1-\frac{1}{p}$, i.e., 
\begin{equation}\label{Eq:GaLp}
\ga \in L^{\frac{1}{1-\alpha}-\eps}(0,T) \quad \forall \eps \in \big(0,\tfrac{\alpha}{1-\alpha}\big].
\end{equation} 
Alternatively, using the concept of locally integrable functions, it naturally holds $\ga \in L_\loc^{1/(1-\alpha)}(0,T)$.
E.g., it holds $\ga \in L^2(0,T)$ for any $\alpha>\frac12$ and $\ga \in L_\loc^2(0,T)$ for any $\alpha \geq \frac12$. Moreover, the kernel function satisfies the following semigroup property, see \cite[Theorem 2.4]{diethelm2010analysis},
\begin{equation} \label{Eq:Semigroup}
	\ga * g_\beta = g_{\alpha+\beta} \qquad \forall \alpha,\beta \in (0,1).
\end{equation} 
%This can be proved as follows: One applies Fubini's theorem to interchange the order of integration 
%$$\begin{aligned}(\ga * g_\beta *u)(t) &=\frac{1}{\Gamma(\alpha)\Gamma(\beta)} \int_0^t (t-s)^{\alpha-1} \int_0^s (s-\tau)^{\beta-1} u(\tau) \dd \tau \dd s  \\
%	&=\frac{1}{\Gamma(\alpha)\Gamma(\beta)} \int_0^t u(\tau) \int_\tau^t (t-s)^{\alpha-1} (s-\tau)^{\beta-1}   \dd s \dd \tau,
%\end{aligned}$$
%and the substitution $s=\tau+\sigma(t-\tau)$ then yields
%$$\begin{aligned}(\ga * g_\beta *u)(t) 
%	&=\frac{1}{\Gamma(\alpha)\Gamma(\beta)} \int_0^t u(\tau) (t-\tau)^{\alpha+\beta-1} \int_0^1 (1-\sigma)^{\alpha-1} \sigma^{\beta}   \dd \sigma \dd \tau.
%\end{aligned}$$
%Lastly, we observe using the fundamental property of the Gamma function that the second integral is equal to $\Gamma(\alpha)\Gamma(\beta)/\Gamma(\alpha+\beta)$, see \cite[Theorem D.6]{diethelm2010analysis}, from which we deduce the desired semigroup property \cref{Eq:Semigroup} of $g_\alpha$.
We note that one can bound the $L^p(0,t)$-norm of a function $u:(0,T) \to \R$ by its convolution with $\ga$ as follows:
\begin{equation}\begin{aligned} \|u\|_{L^p_t}^p := \int_0^t |u(s)|^p \ds  &\leq t^{1-\alpha} \int_0^t (t-s)^{\alpha-1} |u(s)|^p \ds \\ &\leq T^{1-\alpha} \Gamma(\alpha) \big(\ga * |u|^p\big)(t).	
\end{aligned} 
\label{Eq:KernelNorm}
\end{equation}
In other words, the space
\begin{equation} \label{Eq:LpAlpha} L^p_\alpha(0,T)=\Big\{u:(0,T) \to \R: \|u\|_{L^p_\alpha}^p:=\sup_{t \in (0,T)}(\ga*|u|^p)(t) < \infty\Big\},
\end{equation}
is indeed continuously embedded in the space $L^p(0,T)$.
We can relate the estimate  \eqref{Eq:KernelNorm} to $\ga$ by noting that
\begin{equation*} \begin{aligned} (\ga* |u|^p)(t) \geq \frac{t^{\alpha-1}}{\Gamma(\alpha)} \|u\|^p_{L^p_t} = \ga(t) \|u\|^p_{L^p_t} \geq \ga(T) \|u\|^p_{L^p_t}.
\end{aligned}
\end{equation*}
%$$\|u\|_{L^p}^p = \int_0^T 1\cdot |u|^p dt \leq \|1\|_{L^r}^q \|u\|_{L^q}^q, \qquad \frac{1}{p} = \frac{1}{r}+\frac{1}{q}=$$
In particular, this yields for any $s \leq t$
\begin{equation} \label{Eq:EstimateST} (\ga* |u|^p)(t) \geq (\ga* |u|^p)(s) \geq  \ga(s) \|u\|^p_{L^p_s}.
\end{equation}
Therefore, we can integrate this inequality on the time interval $(0,t)$ to obtain
$$t \cdot (\ga* |u|^p)(t)  \geq  \int_0^t  \ga(s) \|u\|^p_{L^p_s} \ds,$$
which implies the following useful bound
\begin{equation}\label{Eq:IneqGaG1}\begin{aligned}(\ga* |u|^p)(t)  &\geq \frac{1}{t} \int_0^t  \ga(s) \|u\|^p_{L^p_s} \ds \\ &\geq  \frac{1}{T} \int_0^t  \ga(s) \|u\|^p_{L^p_s} \ds .\end{aligned}\end{equation}
Similarly, if we take the convoluton instead of the integration of the inequality \eqref{Eq:EstimateST}, we obtain
\begin{equation}\label{Eq:IneqGaG1Conv}\begin{aligned} g_{\alpha+1}(T) (\ga* |u|^p)(t) &\geq g_{\alpha+1}(t) (\ga* |u|^p)(t) \\  &\geq  \big(\ga * (\ga\cdot \|u\|^p_{L^p_t})\big)(t).
\end{aligned}
\end{equation}


%Further, this estimate can be generalized for a nonnegative function $u:(0,T) \to \R_{\geq 0}$ for $0<\beta<\alpha<1$ in the following way:
%$$(\ga * u)(t)=\frac{1}{\Gamma(\alpha)} \int_0^t (t-s)^{\beta-1} \frac{(t-s)^{\alpha-1}}{(t-s)^{\beta-1}} u(s) \ds \leq \frac{T^{\alpha-\beta}\Gamma(\beta)}{\Gamma(\alpha)} (g_\beta * u)(t).$$
%If the order of the kernel function  is larger than one, then one can exploit the semigroup property of the kernel and apply Young's convolution inequality as follows:
%$$(g_{1+\alpha}*u)(t)=(g_1*\ga*u)(t)=\int_0^t (\ga*u)(s) \ds \leq \|\ga\|_{L^1(0,t)} \|u\|_{L^1(0,t)}.$$

In the previous section, have also introduced the fractional derivatives in the sense of Riemann--Liouville $\Pta u = \pt(\gb*u)$ and Caputo $\pta u=\Pta(u-u_0)$. It is well-known that the Caputo derivative can also be written as $\pta u=\gb*\pt u$ if $u$ is absolutely continuous, see \cite[Lemma 3.5]{diethelm2010analysis}. We note that it does not hold $\pta \pt^\beta u = \pt^{\alpha+\beta} u$ in general for the Caputo derivative. However, it holds, see \cite[Theorem 3.14]{diethelm2010analysis}, 
\begin{equation} \label{Eq:PtbPtaPt}
\pta\ptb u = \pt u.
\end{equation}
%Further, this estimate can be generalized for a nonnegative function $u:(0,T) \to \R_{\geq 0}$ for $0<\beta<\alpha<1$ in the following way
%$$(\ga * u)(t)=\frac{1}{\Gamma(\alpha)} \int_0^t (t-s)^{\beta-1} \frac{(t-s)^{\alpha-1}}{(t-s)^{\beta-1}} u(s) \ds \leq \frac{T^{\alpha-\beta}\Gamma(\beta)}{\Gamma(\alpha)} (g_\beta * u)(t).$$
%If the order of the kernel function  is larger than one, then one can exploit the semigroup property of the kernel and apply the Young convolution inequality as follows:
%$$(g_{1+\alpha}*u)(t)=(g_1*\ga*u)(t) \leq \|\ga\|_{L^1(0,t)} \|u\|_{L^1(0,t)}.$$
%\subsection{} %In this work, the time-fractional derivative of order $\alpha \in (0,1)$ is understood in the sense of Riemann--Liouville and is given by 
%\begin{equation} \label{Eq:RL} 
%	\Pta w(t) := \pt \int_0^t \frac{(t-s)^{-\alpha}}{\Gamma(1-\alpha)} w(s) \ds,\end{equation} 
%where $\Gamma(\alpha):=\int_0^\infty t^{\alpha-1}e^{-t}\dt$ is called Euler's Gamma function.
%We can rewrite the definition of the Riemann--Liouville derivative in a compact form by introducing the convolution operator $*$ on the positive half-line. We then have that $\Pta w=\pt (\gb * w)$.
%We refer to the classical textbooks \cite{diethelm2010analysis,baleanu2012fractional} and the newer monographs \cite{jin2021fractional,chen2022fractional} regarding fractional calculus and fractional differential equations.
We define the fractional Sobolev--Bochner space   for $\alpha \in (0,1)$ on $(0,T)$ with values in a given Hilbert space $H$ by $$\W^{\alpha,p}(0,T;H)=\big\{u \in L^p(0,T;H) : \pta u \in L^{p}(0,T;H)\big\}.$$
%In the case $\alpha=1$, we define $$W^{1,p}(0,T;H)=\big\{u \in L^p(0,T;H) : \pt u \in L^p(0,T;H)\big\}.$$
%We note that the Riemann--Liouville space $\W^{\alpha,p}(0,T;H)$ differs from the Sobolev--Slobodeckij space of order $\alpha$ and $p$
%as it can be seen from the function $g_\alpha$ that is an element of $\W^{\alpha,p}(0,T)$ for $\alpha \in (1-\tfrac{1}{p},1)$ but not in the Sobolev--Slobodeckij space, see \cite[Proposition 3.13]{carbotti2021note}. Therefore, we are not able to apply classical results such as embedding theorems for Sobolev--Slobodeckij spaces.
%We also introduce the Riemann--Liouville space combined with the zero trace condition at $t=0$, which reads for a given $u^0 \in H$
%$$\begin{aligned}
%\W^{\alpha,p}_{0}(0,T;H)&=\big\{u \in \W^{\alpha,p}(0,T;H) : (\gb*u)(0)=0 \big\}, \\
%\W^{\alpha,p}_{u^0}(0,T;H)&=\big\{u \in L^p(0,T;H) : u-u^0 \in \W_0^{\alpha,p}(0,T;H) \big\}.
%\end{aligned}$$ 
%We note that we can evaluate $\gb*(u-u^0)$ at zero since it is a continuous functional thanks to the continuous embedding
%$$\gb*u \in W^{1,p}(0,T;H) \hookrightarrow AC([0,T];H),$$
%and the observation $t\mapsto (\gb*u^0)(t)=u^0g_{2-\alpha}(t)=Cu^0 t^{1-\alpha} \in C([0,T];H)$.
Next, we state the inverse convolution property. Its name origins from the fact that the convolution with the kernel $\ga$ acts as an inverse operation on the $\alpha$-th fractional derivative up to the initial condition. In fact,  it holds
\begin{align} \label{Eq:InverseConvolution}
		(\ga* \pta u)(t)    &=u(t)-u_0  \qquad \forall u \in \W^{\alpha,p}(0,T;H). \end{align} 
  This can be seen from the computation
  $$(\ga*\pta u)(t)=(\ga*\gb *\pt u)(t)=(1*\pt u)(t)=\int_0^t \pt u(s) \ds = u(t)-u_0,$$
  where we used \eqref{Eq:Semigroup} to conclude $\ga*\gb=g_1=1$.
Furthermore, we mention the following consequences of the interaction between fractional derivatives and kernel functions:
\begin{equation} \label{Eq:DerivativeofKernel}  \begin{aligned}
	\pta (\ga * u ) &=\Pta (\ga * u)=\pt ( \gb * \ga * u) = \pt (1*u) = u,
 \end{aligned}
\end{equation}
which holds for any $u \in L^1(0,T;H)$. %\\
	%\ptb \pta u &=  \ga * \pt^{\alpha} \pt u = \pt u - (\pt u)(0) &&\forall u \in  \W_{u^0}^{1,1}(0,T;H).

%$$\pta(\ga*u)=\gb*g_{\alpha-1}*u=g_0*u=u$$


%However, in our setting of the time-fractional Navier--Stokes--Fokker--Planck system, we have already seen that the Riemann--Liouville derivative appears on the left-hand side without the translation of an initial value. This already explains intuitively the restriction on the values of $\alpha$ in the theorem of the system's well-posedness, see \cref{Thm:WellPosedness} below.


%\subsection{Compactness results}


As in the integer-order setting, there are continuous and compact embedding results for fractional Sobolev spaces; see \cite[Theorem 3.2]{wittbold2021bounded}. For a given Gelfand triple $V \con\con H \con V'$, the classical Aubin--Lions lemma \cite{simon1986compact} reads
\begin{equation}\label{Eq:aubin}  \begin{aligned}
\W^{1,1}(0,T;V') \cap L^\infty(0,T;V) &\con\con C([0,T];H),\\ %
\W^{1,1}(0,T;V') \cap L^p(0,T;V) &\con\con L^p(0,T;H), \quad p \in [1,\infty),
\end{aligned}\end{equation} 
and the fractional counterparts is as follows:
\begin{equation}\label{Eq:aubinfractional}  \begin{aligned}
\W^{\alpha,p}(0,T;V') \cap L^{p'}(0,T;V) &\con C([0,T];H), &&p \in [1,\infty),\\%
\W^{\alpha,p}(0,T;V') \cap L^p(0,T;V) &\con\con L^p(0,T;H), &&p \in [1,\infty).
\end{aligned}\end{equation} 
We observe that there is a give-and-take involved: The fractional derivative is of order $\alpha<1$ i.e. it is less than the full derivative in the classical Aubin--Lions lemma. However, we require that the derivative is in the better space $L^p(0,T;V')$ instead of only $L^1(0,T;V')$ to achieve the same target space $L^p(0,T;H)$ in the compactness result.














The classical chain rule does not hold for fractional derivatives, but one can use the following inequality, see \cite[Theorem 2.1]{vergara2008lyapunov}, as a remedy:
\begin{equation} \label{Eq:ChainOriginal}  \frac12 \pta \|u\|^2_H  \leq (u,\pta u)_H \quad \forall u \in W^{\alpha,p}(0,T;H),
\end{equation}
for almost all $t \in (0,T)$, which is also known as Alikhanov's inequality, see the original work \cite{alikhanov2010priori}. Moreover, we conclude from \eqref{Eq:PtbPtaPt} and Alikhanov's inequality the following:
\begin{equation} \label{Eq:ChainExtended} (\pt u, \pta u)_H = (\ptb \pta u,\pta u)_H \geq \frac12 \ptb \|\pta u\|_H^2,
\end{equation}
which gives after integrating it over the time interval $(0,t)$
$$\int_0^t (\pt u, \pta u)_H \ds \geq \frac12 (\ga * \|\pta u\|_H^2)(t) \geq \frac{1}{2\Gamma(\alpha)T^{1-\alpha}} \|\pta u\|_{L^2_tH}^2,
$$
where we applied \eqref{Eq:KernelNorm} in the last step.
%Here, it has to be assumed  $\big(\gb*(u-u^0)\big)(0)=0$. 
%Moreover, by \cite{mustapha2014well} it holds the following inequality
%$$\int_0^t (\ga * u,u)_H \ds \geq \cos(\frac{\alpha\pi}{2}) \|g_{\alpha/2}*u\|_{L^2_tH}^2 \qquad \forall u \in C([0,T];H).$$
%From here, we can conclude
%$$\int_0^t (\ptb u,\pt u)_H \ds=\int_0^t (\ga * \pt u,\pt u)_H \ds \geq C(\alpha) \|\pt^{1-\alpha/2} u\|_{L^2_tH}^2.$$
%Alternatively,
%$$(\pt^{1-\alpha} u,\pt^\alpha \pt^{1-\alpha} u)_H \geq \frac12 \pta \|\pt^{1-\alpha} u\|^2_H.$$
%$$\pta\pt^b u=\gb*(\pt^b \pt u)=g_{2-a-b}*\pt^2 u=\pt^{a+b} u$$



%\subsection{Gronwall inequality}

Next, we require a Gronwall-type inequality that allows convolutions on the right-hand side of the inequality. Moreover, we want to have an additional function on the right-hand side that is only  locally integrable. Such inequalities are known as Henry--Gronwall inequalities.

\begin{lemma}[{Henry--Gronwall, cf. \cite[Lemma 7.1.1]{henry1981geometric}}]
    Let $b\geq 0$, $\beta>0$, $a \in L^1_\loc(0,T;\R_{\geq 0})$. If $u \in L^1_\loc(0,T;\R_{\geq 0})$ satisfies
    $$u(t) \leq a(t)+b (g_\beta * u)(t), \quad \text{ for a.e. } t \in (0,T),$$
    then it yields
    $$u(t) \leq C(\alpha,b,T) \cdot \big((g_0+E)*a\big)(t) , \quad \text{ for a.e. } t \in (0,T),$$
    where $E$ is related to the Mittag--Leffler function.
\end{lemma}

We prove the following extension of the Henry--Gronwall inequality that allows an additional term on the left-hand side.

\begin{lemma}\label{Lem:GronFrac}
    Let $b\geq 0$, $b>0$, $a \in L^1_\loc(0,T;\R_{\geq 0})$. If the  functions $u,v \in L_\loc^1(0,T;\R_{\geq 0})$  satisfy the inequality
    $$u(t) + (\ga * v)(t) \leq a(t)+b (\ga* u)(t) \qquad \text{for a.a. } t \in (0,T], $$
    then it yields
    $$u(t) + \int_0^t v(s) \ds \leq C(\alpha,b,T) \cdot \big( (g_0+E)*a\big)(t) \qquad \text{for a.a. } t \in (0,T]. $$
\end{lemma}
\begin{proof}
We define the function $w=u+\ga*v$.
    Since $\ga*v$ is again nonnegative, we obtain
    $$w(t) \leq a(t)+b (\ga* u)(t) \leq a(t) + (\ga*w)(t),$$
    and by the Henry--Gronwall it yields
    $$w(t) \leq C(\alpha,b,T) \cdot \big( (g_0+E)*a\big)(t).$$
    Moreover, we can use \eqref{Eq:KernelNorm} to estimate $\ga*v$ by the integral of $v$, and we obtain the lemma's desired bound.
\end{proof}



