\section{Well-posedness of weak solutions} \label{Sec:Analysis}
In this section, we state and prove the well-posedness of weak solutions to the time-fractional Fokker--Planck equation \eqref{Eq:FP}. As we already mentioned, we equip the equation with a homogeneous Dirichlet boundary condition. As noted before, our analysis holds for no-flux boundary conditions as well. 
We analyze the PDE in the Hilbert triple 
$$H_0^1(\Omega) \hookrightarrow \hookrightarrow L^2(\Omega) \hookrightarrow H^{-1}(\Omega).$$
We equip $H_0^1(\Omega)$ with the norm $\|\cdot\|_{H_0^1}=\|\nabla \cdot \|_{L^2(\Omega)}$ with is equivalent to the natural norm on $H^1(\Omega)$ due to Poincar\'e's inequality \cite[6.7]{alt2016linear}.
We use the Galerkin method and discretize the partial differential equations in space. Further, we derive suitable energy estimates, and we emphasize the places where the time-fractional derivative comes into play. We shall then pass to the limit to deduce the existence of a weak solution. The uniqueness is obtained as usual.

First off, however, we introduce the concept of a  weak solution to the time-fractional Fokker--Planck equation in the following definition.\medskip

\begin{definition} \label{Def:Weak}
We call a function $\psi:\Omega_T \to \R$ a weak solution to the time-fractional Fokker--Planck equation \eqref{Eq:FP} if it is of the regularity
$$\psi \in W^{1,1}(0,T;H^{-1}(\Omega)) \cap H^{1-\alpha}(0,T;H_0^1(\Omega)),$$
fulfills the initial data  $\psi(0)=\psi_0$ in $H^{-1}(\Omega)$, and the following variational form: 
\begin{equation}\label{Eq:FP} \begin{aligned}
  &\langle \pt \psi,\zeta \rangle_{H_0^1}
  +  D(\ptb \nabla \psi,\nabla \zeta)_{L^2} - (F \ptb \psi,\nabla \zeta)_{L^2} \\ &=\langle f,\zeta\rangle_{H_0^1} - \ga (D \nabla \psi_0,\nabla \zeta)_{L^2} + \ga \cdot (F \psi_0,\nabla \zeta)_{L^2} \qquad \forall \zeta \in H_0^1(\Omega).
\end{aligned} \end{equation}
\end{definition} \medskip
%$$\begin{aligned}\pt \psi - \div(D\nabla \ptb \psi) +\div(F  \ptb \psi) = f+ \ga  \div(D\nabla \psi_0) - \ga  \div(F  \psi_0),
%\end{aligned}$$
%the RHS gets time regularity from $\ga$ and space regularity from $\psi_0$, i.e., should be in $L^p(0,T;L^2)$ for $\alpha \in (1-1/p,1)$ i.e. $1-\alpha<1/p$ i.e. $p<1/(1-\alpha)$. E.g. for $\alpha=1/2$ we get $p<2$. Maybe only need $L^1$ in time. doable with generalized Gronwall. First do this estimate by testing with $\ptb \psi$.
%$$\begin{aligned} &\frac12 \pt^{1-\alpha/2} \|\psi\|^2 + D\|\nabla \ptb \psi\|^2 +\frac12 (\div F,|\ptb \psi|^2) \\  &\leq  (f,\ptb \psi) + \ga  (F\psi_0-D\nabla \psi_0, \nabla \ptb \psi)
%\end{aligned}$$
%should work for $H^2$ initials. then rhs is better.
%This is kinda better, but for $F$ term need estimate on $\ptb \psi$, only get from first
%$$\begin{aligned} &\frac12 \pt \|\psi\|^2 + \frac{D}{2} \pt^{(1-\alpha)/2} \|\nabla \psi\|^2 - (F\ptb \psi,\nabla \psi) \\  &\leq  (f,\psi) + \ga  (F\psi_0-D\nabla \psi_0, \nabla \psi)
%\end{aligned}$$
%We have now some terms that require convolution and some integration. Probably first integrate and then leave out positive term and afterwards, insert back. However, we need maybe both terms for Gronwall.
%We estimate the product term as follows
%$$(F\ptb \psi,\nabla \psi)_{L^2} \leq \|F\|_\infty \|\ptb \psi\|_{L^2} \|\nabla \psi\|_{L^2}$$

%As written in \cite{le2019existence}, one can also do
%$$\int_0^t (\ptb \phi,\phi)_{L^2} \ds \geq C t^{1-\alpha} \|\phi\|_{L^2_tH}^2$$
%Then one can estimate the second term differently and get directly $L^2$ in time norm. However, this is not so favorable actually, since no Gronwall. maybe better to keep $(\ga*\|\nabla \phi\|)(t).$ Maybe also better to convolve. Then one obtains nice $L^\infty$ terms. However, then the first term becomes
%$$\ga * \pt \|\psi\|^2 = \pta \|\psi\|^2$$



%$$\pta \phi = \pta(\phi-\phi_0) + \pta \phi_0 = D_t^\alpha \phi + g_{1-\alpha}(t) \phi_0$$

As we see, we expect a solution that is continuous in time with values in the Hilbert space $H^{-1}(\Omega)$. Therefore, it is well-defined for the initial to fulfill $\psi(0)=\psi_0$ in $H^{-1}(\Omega)$. Moreover, it holds
$$H^{1-\alpha}(0,T;H_0^1(\Omega)) \hookrightarrow C([0,T];H_0^1(\Omega)),$$
if $1-\alpha>1/2$ i.e. $\alpha<\frac12$. In this case, the initial is even satisfied in $H_0^1(\Omega)$. 

Next, we state the main result of this work on the well-posedness of weak solutions to the time-fractional Fokker--Planck equation \eqref{Eq:FP}.

\begin{theorem}[Well-posedness of weak solutions] \label{Thm:WellPosedness} Let us assume: 
\begin{itemize}
	\item $\Omega \subseteq \R^d$, $d \in \mathbb{N}$, bounded Lipschitz domain,  $T<\infty$ fixed final time,
 \item $\alpha \in (0,1)$,
       \item $\psi_0 \in H_0^1(\Omega)$,
 \item $f \in L^2_\alpha(0,T;H^{-1}(\Omega))$,
  %\item $\C,\tau$ are of the form \eqref{Def:tau}--\eqref{Def:C}  and $\C$ fulfills \eqref{Eq:C},
% \item $D \in W^{1,\infty}(\Omega_T;\R^{d\times d})$ with $u^\top D^\top(x,t) u \geq D u \cdot u$ for any $u \in \R^d$ and a.e. $(x,t) \in \Omega_T$ 
\item $F \in L^\infty(\Omega_T;\R^d)$ with $\|F\|_{L^\infty(\Omega_T)} \leq F_\infty<\infty$.
      %\item $\psi_0 \in H^1(\Omega)$ and $\alpha  \in (\frac12,1)$.
  \end{itemize}
	Then there exists a unique weak solution $\psi$ to the time-fractional Fokker--Planck equation \eqref{Eq:FP} in the sense of Definition \ref{Def:Weak}. Further, it has the additional regularity
 $$\psi \in W^{1,r'}(0,T;H^{-1}(\Omega)) \cap W^{1-\alpha,p}(0,T;L^2(\Omega)) \cap H^{1-\alpha}(0,T;H_0^1(\Omega)),$$
 with $r'$ being the H\"older conjugate of $r=\max\{q',2\}$, $q'$ being the H\"older conjugate of $q=\frac{1}{1-\alpha}-\eps$ for $\eps\in (0,\frac{\alpha}{1-\alpha}]$, and $$\begin{cases}
	p=\infty, &\alpha>\frac12, \\
	p<\infty, &\alpha=\frac12,\\
	p<\frac{2}{1-2\alpha}, &\alpha<\frac12.
\end{cases} $$
\end{theorem}

We comment on the assumptions in this well-posedness result. We see it as an advantage that we can show the equation's well-posedness for all fractional values between $0$ and $1$, and any dimension $d \geq 1$. Further, we only require $F \in L^\infty(\Omega_T;\R^d)$ as opposed to \cite{le2019existence} that required $F \in W^{2,\infty}(\Omega_T)$ for showing results on mild solutions. Moreover, the work \cite{mclean2019well} studied a Volterra integral form of a class of time-fractional advection-diffusion-reaction equations, including the time-fractional Fokker--Planck equations. However, they required $F \in C^2([0,T];W^{1,\infty}(\Omega)^d)$ to show the well-posedness of the Volterra integral equation.

As the solution lies in the space $W^{1-\alpha,\infty}(0,T;L^2(\Omega))$ for $\alpha>\frac12$, we obtain $\psi \in C([0,T];L^2(\Omega))$ for $\alpha>\frac12$. It remains to consider $\alpha=\frac12$. In this case, we have $q=2-\eps$ and $r=\max\{1-1/(2-\eps),2\}=2$. Therefore, it holds $\psi \in H^1(0,T;H^{-1}(\Omega)) \cap L^2(0,T;H_0^1(\Omega))$, i.e., by an interpolation result $\psi \in C([0,T];L^2(\Omega))$. We summarize the continuity results as follows:
$$\psi \in \begin{cases}
    C([0,T];L^2(\Omega)), &\alpha \in (0,1), \\
    C([0,T];H_0^1(\Omega)), &\alpha \in (0,\frac12).
\end{cases}$$
We conclude that the initial is indeed at least satisfied in $L^2(\Omega)$. 

\bigskip

%Moreover, we can state a well-posedness result for the strengthened assumption of $\psi_0 \in H^2(\Omega)$. This results in improved regularity and continuity results, as seen in the next theorem.

%\begin{theorem}[Higher regularity] \label{Thm:HigherRegularity} Assume the hypotheses of Theorem \ref{Thm:WellPosedness}. If it additionally holds $\psi_0 \in H^2(\Omega)$, then it yields the higher regularity $$\psi \in W^{\alpha,\infty}(0,T;H).$$
%\end{theorem}

%In the cases of a force that is affine linear in $t$, i.e., it is of the form \begin{equation} \label{Eq:FQuadratic} F(t,x)=a(x)+b(x)t,
%\end{equation} which includes the time-independent forces for $b\equiv 0$. In this case, we can relax the initial condition to $\psi_0 \in L^2(\Omega)$ and we obtain the following result:
%\begin{theorem}[Well-posedness of weak solutions for special $\bm{F}$] \label{Thm:SpecialF}  Assume the hypotheses of Theorem \ref{Thm:WellPosedness} but let $\psi_0 \in L^2(\Omega)$ and let $F$ be of the form \eqref{Eq:FQuadratic} with $a,b \in L^\infty(\Omega)$. Then there exists a unique weak solution $\psi$ to \eqref{Eq:FP} in the sense that $\psi$ satisfies initial $\psi(0)=\psi_0$ and the weak form
%$$\begin{aligned} &\langle \pta\psi,\zeta \rangle_{H_0^1}
%  +  D(\nabla \psi,\nabla \zeta)_{L^2} - (F \psi,\nabla \zeta)_{L^2} \\ &=\langle f,\zeta\rangle_{H_0^1} + (1-\alpha) (b g_{2-2\alpha} * \psi,\zeta)_{L^2} \qquad \forall \zeta \in H_0^1(\Omega).\end{aligned}$$
%\end{theorem}

\noindent\textbf{Proof} In order to prove this theorem, we  employ the Galerkin method to discretize the variational form in space. This reduces the time-fractional PDE to a system of fractional ODEs, which admits a discretized solution $\psi_k$. We then derive $k$-uniform energy estimates, which imply the existence of weakly/weakly-$*$ convergent subsequence $\psi_{k_j}$. Finally, we pass to the limit $j \to \infty$ and apply compactness methods to return to the variational form of the continuous system. Recently, the Galerkin method has been applied to various time-fractional PDEs, see, e.g., \cite{fritz2021subdiffusive,fritz2022time,fritz2023equivalence,vergara2015optimal}. \medskip

\noindent \textbf{(1) Galerkin discretization.}
We introduce the discrete spaces
\begin{align*}
	H_k & =\text{span}\{ h_1,\dots,h_k\}, \\
	%H_k & =\text{span}\{ y_1,\dots,H_k\},
	%	Z_K &=\text{span}\{ z_1,\dots,z_k\},
\end{align*}
where $h_j: \Omega \to \R$, %and $y_j : \Lambda \to \R$, 
$j \in \{1,\dots,k\}$, are the eigenfunctions to the eigenvalues $\lambda_{j} \in \R$ of the following problems
$$\begin{aligned}
	(\nabla h_j,\nabla v)_{L^2} &= \lambda_{j} (h_j,v)_{L^2} &&\forall  v \in H_0^1(\Omega).  %\\
	%(\nabla y_j,\nabla y)_{L^2} &= \mu_{j} (y_j,y)_{L^2} &&\forall  y \in H. 
\end{aligned}$$

%\begin{alignat*}{1}
%	 & \begin{cases} \begin{aligned}
		%			-\Delta h_j          & = \lambda_{h,j} h_j &  & \text{in } \Omega,         \\
		%			\nabla h_j \cdot n & = 0               &  & \text{on } \partial\Omega,
		%		\end{aligned} \end{cases} 
%			  \begin{cases} \begin{aligned}
		%		-\Delta h_j^0          & = \lambda_{h^0,j} h_j^0 &  & \text{in } \Omega,         \\
		%		\nabla h_j^0 \cdot n & = 0             &  & \text{on } \partial\Omega \backslash \p \Omega_D, \\
		%				h_j^0 & = 0               &  & \text{on } \partial\Omega_D, 
		%		\end{aligned} \end{cases} 
%	  \begin{cases} \begin{aligned}
		%			-\Deltala y_j                  & = \lambda_{y,j} y_j &  & \text{in } \Lambda,         \\
		%			\nablala y_j \cdot n_\Lambda & = 0               &  & \text{on } \partial\Lambda_N, \\
		%			y_j & = 0               &  & \text{on } \partial\Lambda_D.
		%		\end{aligned} \end{cases}
%\end{alignat*}
Since the inverse Dirichlet--Laplace operator is compact, self-adjoint, injective, positive operators on $L^2(\Omega)$, we conclude by the spectral theorem, see e.g.,\cite[12.12 and 12.13]{alt2016linear}, that
\begin{alignat*}{3}
	& \{h_j\}_{j \in \mathbb{N}} &  & \text{ is an orthonormal basis in } L^2(\Omega) &  & \text{ and orthogonal in } H_0^1(\Omega), %\\
		%& \{y_j\}_{j \in \mathbb{N}} &  & \text{ is an orthonormal basis in } H &  & \text{ and orthogonal in } V.
\end{alignat*}
Therefore, ${\cup_{k\in\mathbb{N}}} H_k$ 
%and ${\cup_{k\in\mathbb{N}}} H_k$ 
is dense in $H_0^1(\Omega)$. We consider the Galerkin approximations
\begin{equation}\begin{gathered}
		\psi_k (t) = \sum_{j=1}^k \psi^j_k(t) y_j,
	\end{gathered}
	\label{Eq:GalerkinAnsatzFunctions}
\end{equation}
where 
$\psi^j_k: (0,T) \to \R$
are coefficient functions for all $j \in \{1,\dots,k\}$.
 %Let $h>0$ denote a discretization parameter tending to zero. As in \cite{barrett2009numerical}, we choose finite-dimensional spaces $H_k^x \subset W^{1,\infty}(\Omega)$ and $H_k^q \subset W^{1,\infty}(D)$ such that it holds
%$$\text{dist}_{W^{1,\infty}(\Omega)} (\eta,H_k^x) \to 0, \qquad \text{dist}_{W^{1,\infty}(D)} (\xi,H_k^q) \to 0,$$
%as $h\to 0$ for all $\eta \in C^\infty(\overline\Omega)$ and $\xi \in C^\infty(\overline D)$. Moreover, we define the tensor space $H_k=H_k^x \otimes H_k^q \subset W^{1,\infty}(\Omega \times D)$ and note that $\widehat{X}_{L^2} \subset X \subset V$. 
%Further, we define the finite-dimensional spaces $W_{L^2}$, $R_{L^2}$ and $H_k$ such that
%$$\begin{aligned}W_{L^2} &\subset H_0^1(\Omega;\R^d) \cap W^{1,\infty}(\Omega;\R^d), \quad R_{L^2} \subset L_0^2(\Omega), \\ H_k&=\{w_{L^2}\in W_{L^2}: (\div w_{L^2},r_{L^2})_{L^2} \,\forall r_{L^2} \in R_{L^2} \},\end{aligned}$$
%where $\cup_{h>0} W_{L^2}$ and $\cup_{h>0} R_{L^2}$ are supposed to be dense in $H_0^1(\Omega;\R^d)$ and $L_0^2(\Omega;\R^d)$, respectively. Further, we assume that for all $v \in V$ there exists a sequence $v_k \in H_k$ such that $v_k \to v$ in $H^1(\Omega)$ for $h \to 0$. This holds for the typical Galerkin approximation with the availability of an uniform inf-sup condition.
We denote the orthogonal projections onto the finite-dimensional space by $\Pi_{H_k}: L^2(\Omega) \to H_k$. %Thanks to \cite[Theorem 8.1.11]{brenner2008mathematical} and \cite{guzman2009holder}, we have that $\Pi_{H_k}$ is uniformly $H^1$-stable and $\Pi_{H_k}$ is $W^{1,\infty}$-stable.
Given the initial data $\psi_0$ from the continuous system, we choose $\psi_{0k} \in H_k$ such that  $\psi_{0k}=\Pi_{H_k} \psi_0$, i.e., there are coefficient $\{\psi_{0k}^{j}\}_{j=1}^k$ such that $\psi_{0k}=\sum_{j=1}^k \psi_{0k}^{j} y_j$. Moreover, due to well-known properties of the projection operator, see \cite[9.7]{alt2016linear}, it holds as $k \to \infty$
\begin{equation} \label{Eq:Projection}
   \|\psi_{0k}\|_X \leq \|\psi_0\|_X ~\text{ and }~ \psi_{0k} \to \psi_0 ~\text{ in }~ X \in \{H^{-1}(\Omega), L^2(\Omega), H_0^1(\Omega) \}.
\end{equation}

The Galerkin equations read as follows: We want to find  $\psi_k \in H_k$ such that $\psi_k(0)=\psi_{0k}$ and
\begin{equation}\label{Eq:FP_dis}\begin{aligned}
%(\pt u_k,v_k )_{L^2}  + \frac12 ((u_k \cdot \nabla)u_k,v_k)_{L^2} - \frac12 ((u_k \cdot \nabla)v_k,u_k)_{L^2}   && \label{Eq:NS_dis} \\[-.2cm] + \nu (\nabla u_k,\nabla v_k)_{L^2}  + k_B \mu_T (\C(M \gb *\phi_k),\nabla v_k)_{L^2} =0, \notag \\[.1cm]
	&(\pt\psi_k,\zeta )_{L^2}
  +  D(\ptb \nabla \psi_k,\nabla \zeta)_{L^2} - (F \ptb \psi_k,\nabla \zeta)_{L^2} \\ &=\langle f,\zeta\rangle_{H^1_0} - \ga(D\nabla \psi_{0k},\nabla \zeta)_{L^2} + \ga \cdot (F\psi_{0k},\nabla \zeta)_{L^2}. %\\[-.1cm] + \eps (\nabla \phi_k,\nabla \zeta_k)_{L^2}- (\omega(u_k) q \phi_k, \nabla \zeta_k)_{L^2}=0, \notag 
 \end{aligned}\end{equation} for all $\zeta \in H_k$. We want to apply an existence result on ODEs with Riemann--Liouville derivatives and therefore, we rewrite the Galerkin system as follows:
 \begin{equation*}\begin{aligned}
%(\pt u_k,v_k )_{L^2}  + \frac12 ((u_k \cdot \nabla)u_k,v_k)_{L^2} - \frac12 ((u_k \cdot \nabla)v_k,u_k)_{L^2}   && \label{Eq:NS_dis} \\[-.2cm] + \nu (\nabla u_k,\nabla v_k)_{L^2}  + k_B \mu_T (\C(M \gb *\phi_k),\nabla v_k)_{L^2} =0, \notag \\[.1cm]
	&(\pt\psi_k,\zeta )_{L^2}
  +  D(\Ptb \nabla \psi_k,\nabla \zeta)_{L^2} - (F \Ptb \psi_k,\nabla \zeta)_{L^2} =\langle f,\zeta\rangle_{H^1_0}, %\\[-.1cm] + \eps (\nabla \phi_k,\nabla \zeta_k)_{L^2}- (\omega(u_k) q \phi_k, \nabla \zeta_k)_{L^2}=0, \notag 
 \end{aligned}\end{equation*}
 for any $\zeta \in H_k$. We rewrite $\psi_k$ as the sum of the basis functions $\{\psi_k^j\}_{j=1}^k$ as introduced in \eqref{Eq:GalerkinAnsatzFunctions}, from which we obtain that the coefficients
are governed by the system
 \begin{equation}\label{Eq:FP_dis3}\begin{aligned}
%(\pt u_k,v_k )_{L^2}  + \frac12 ((u_k \cdot \nabla)u_k,v_k)_{L^2} - \frac12 ((u_k \cdot \nabla)v_k,u_k)_{L^2}   && \label{Eq:NS_dis} \\[-.2cm] + \nu (\nabla u_k,\nabla v_k)_{L^2}  + k_B \mu_T (\C(M \gb *\phi_k),\nabla v_k)_{L^2} =0, \notag \\[.1cm]
	&\pt\psi_k^i 
  +  \lambda_i D\Ptb\psi_k^i - \sum_{j=1}^k \Ptb \psi_k^j (F y_j,\nabla y_i)_{L^2}=\langle f,y_i\rangle_{H^1},  %\\[-.1cm] + \eps (\nabla \phi_k,\nabla \zeta_k)_{L^2}- (\omega(u_k) q \phi_k, \nabla \zeta_k)_{L^2}=0, \notag 
 \end{aligned}\end{equation}
 for any $i \in \{ 1, \dots, k\}$.
 Equivalently, we define the function $\phi_k^i=\Ptb\psi_k^i$ for any $i$, which is governed by the equation
 \begin{equation}\label{Eq:FP_dis2}\begin{aligned}
%(\pt u_k,v_k )_{L^2}  + \frac12 ((u_k \cdot \nabla)u_k,v_k)_{L^2} - \frac12 ((u_k \cdot \nabla)v_k,u_k)_{L^2}   && \label{Eq:NS_dis} \\[-.2cm] + \nu (\nabla u_k,\nabla v_k)_{L^2}  + k_B \mu_T (\C(M \gb *\phi_k),\nabla v_k)_{L^2} =0, \notag \\[.1cm]
	&\Pta\phi_k^i 
  +  \lambda_i D \phi_k^i - \sum_{j=1}^k \phi_k^j (F y_j,\nabla y_i)_{L^2}  =\langle f,y_i\rangle_{H^1}, %\\[-.1cm] + \eps (\nabla \phi_k,\nabla \zeta_k)_{L^2}- (\omega(u_k) q \phi_k, \nabla \zeta_k)_{L^2}=0, \notag 
 \end{aligned}\end{equation} for any $i \in \{ 1, \dots, k\}$. We notice that it holds $\gb*\phi_k^i=\psi_k^i$ due to the inverse convolution property \eqref{Eq:InverseConvolution} and we observe that \eqref{Eq:FP_dis2} is naturally equipped with the initial $(\gb*\phi_k^i)=\psi_{0k}^i$.

 We denote the vector of components $\big(\phi_k^j(t)\big)_{1\leq j\leq k}$
by $\Phi(t)$. Then the approximate problem can be written as a system of ordinary
differential equations for $\Phi(t)$ of the form
$\Pta \Phi  = h(t,\Phi)$,
where $h$ is continuous and locally Lipschitz continuous with respect to $\Phi$. Therefore, the  fractional variant of the Cauchy--Lipschitz theorem, see \cite[Theorem 5.1]{diethelm2010analysis},  yields the existence of a unique continuous solution, defined on a short-time interval $[0, T_k]$ with $0<T_k \leq T$. From here, we conclude $\phi_k + \ga \psi_{0k} = \ptb \psi_k \in C((0,T_k];H_k)$ and $\ga*\psi_k \in C^1((0,T_k];H_k)$.
%$\phi_k = \ptb \psi_k + \ga \psi_{0k}$
\medskip


\noindent \textbf{(2) Energy estimates: Part 1.} Next, we derive $k$-uniform estimates that will allow us to extract weakly converging subsequences. 
We test the Galerkin equation \eqref{Eq:FP_dis} by $\ga*\pt \psi_k=\ptb \psi_k \in H_k$ giving
\begin{equation} \label{Eq:Tested1}\begin{aligned} &(\pt \psi_k,\ptb \psi_k)_{L^2} + D\|\nabla \ptb \psi_k\|^2_{L^2} -  (F\cdot \nabla\ptb \psi_k,\ptb \psi_k)_{L^2} \\  &=  \langle f,\ptb \psi_k\rangle_{H_0^1} - \ga( D\nabla \psi_{0k} - F\psi_{0k},\nabla\ptb \psi_k)_{L^2}
\end{aligned}\end{equation}
For the first term on the left-hand side, we use $\pt \psi_k=\pta \ptb \psi_k$, see \eqref{Eq:PtbPtaPt}, to conclude with Alikhanov's inequality, see \eqref{Eq:ChainExtended},
\begin{equation} \label{Eq:Tested1a}
\begin{aligned} (\pt \psi_k,\ptb \psi_k)_{L^2} &=(\pta \ptb \psi_k,\ptb \psi_k)_{L^2} \\ &\geq \frac12 \pta \|\ptb \psi_k\|_{L^2}^2.% \\
%\int_0^t (\pt \psi_k,\ga * \pt \psi_k)_{L^2} \geq C_\alpha \|\pt^{1-\alpha/2} \psi_k\|_{L^2_tL^2}^2
\end{aligned}
\end{equation}
%Moreover, it holds $\ga*u \leq g_{\beta} * u$ for any $\alpha>\beta$. We have
%$$\int_0^t (\pt \psi_k,\ga * \pt \psi_k)_{L^2} \geq C_\alpha \|g_{\alpha/2}*\pt \psi_k\|_{L^2_tL^2}^2 \geq C \|g_{\beta}*\pt \psi_k\|_{L^2_tL^2}^2$$
%for any $\beta >\alpha/2$. We would want to have $\ptb \psi_k=\ga*\pt \psi_k$ on the rhs. And $\alpha>\alpha/2$. Let us check if we were really allowed to do this.
%For the term involving the space-time dependent force $F$, we obtain by integration by parts
%$$\begin{aligned} -&(F\cdot \nabla\ptb \psi_k,\ptb \psi_k)_{L^2}  \\&=(\div(\ptb \psi_k F),\ptb \psi_k)_{L^2} \\ &= (\div F,|\ptb \psi_k|^2)_{L^2} + (F\cdot \nabla\ptb \psi_k,\ptb \psi_k)_{L^2},
%\end{aligned}$$
%and therefore, we conclude
%$$-(F\cdot \nabla\ptb \psi_k,\ptb \psi_k)_{L^2} =\frac12 (\div F,|\ptb \psi_k|^2)_{L^2}.$$
%IF WE DO NOT DO THIS THEN ON RHS
We bring the term involving the force $F$ to the right-hand side of \eqref{Eq:Tested1} and apply the H\"older inequality to conclude
$$(F\nabla\ptb \psi_k,\ptb \psi_k)_{L^2} \leq F_\infty \|\nabla\ptb \psi_k\|_{L^2} \|\ptb \psi_k\|_{L^2},$$
where $F_\infty<\infty$ is the constant as introduced in the theorem's assumptions.
Further, we apply the Young inequality to give the norm of $\nabla \ptb \psi_k$ a prefactor that is smaller than $D$, i.e., we obtain
\begin{equation} \label{Eq:Tested1b}F_\infty \|\nabla\ptb \psi_k\|_{L^2} \|\ptb \psi_k\|_{L^2} \leq \frac{D}{4} \|\nabla\ptb \psi_k\|_{L^2}^2 +\frac{F_\infty^2}{D} \|\ptb \psi_k\|_{L^2}^2.
\end{equation}

Using again a combination of the H\"older and $\eps$-Young inequalities, we estimate the term on the right-hand side of the tested equation \eqref{Eq:Tested1} with the initials $\psi_{0k}$ by
$$\begin{aligned} &\ga(F\psi_{0k}-D\nabla \psi_{0k},\nabla \ptb \psi_k)_{L^2} \\ &\leq \frac{\ga}{2\eps} (F_\infty^2 \|\psi_{0k}\|_{L^2}^2+D^2\|\nabla \psi_{0k}\|_{L^2}^2)  + \eps_1\ga  \|\nabla \ptb \psi_k\|_{L^2}^2,\end{aligned}$$
where $\eps_1>0$ is a constant that we will determine accordingly below. We are not interested in tracking the constants $D$ and $F_\infty$ and therefore, we include them in a generic constant $C$ that may change from line to line. Moreover, we can estimate the norm of $\psi_{0k}$ by $\psi_0$ due to the projection property \eqref{Eq:Projection}. Consequently, we obtain the estimate
\begin{equation} \label{Eq:Tested1c}\begin{aligned} \ga(F\psi_0\!-\!D\nabla \psi_0,\nabla \ptb \psi_k)_{L^2} &\leq C\ga(t)  \|\psi_0\|_{H_0^1}^2 \!+\! \eps\ga \|\nabla \ptb \psi_k\|_{L^2}^2\end{aligned}\end{equation}
%MAYBE TRY THIS: WE NEED TO CONVOLVE TO ABSORB THE DIVF TERM. SO, WE KEEP GALPHA TO NABLA PTB AND MULTIPLY IT BY A SUPER SMALL EPSILON TERM DEPENDING ON T WHICH MAKES IT SMALLER THAN; STH LIKE:
%$$\ga \|\nabla \ptb \psi_k\| (INIT) \leq \eps(t) \ga(t) \|\nabla \ptb \psi_k\|^2 + \frac{\ga(t)}{4\eps(t)} INIT$$
%NOW GUARANTEE $\eps(t)\ga(t) \leq \frac{D}{2}$ and $\ga*(\ga/\eps) \in L^1$. It is most likely just $\eps(t)=D/(2\ga)$ which would not work, since $\ga/\eps=\ga^2$
%Since we did not assume that $\div F$ has a non-negative lower bound, we bring this term to the right-hand side of the tested inequality. 
%We obtain
%$$\begin{aligned} &\frac12 \pta \|\ptb \psi_k\|_{L^2}^2 + D \|\nabla \ptb \psi_k\|_{L^2}^2 \\  &\leq  - \frac12 (\div F,|\ptb \psi_k|^2)_{L^2}+\langle f,\ptb \psi_k\rangle_{H^1}%+ \ga  (F\psi_0-D\nabla \psi_0,  \ptb \nabla \psi_k)_{L^2}.
%- \ga(D\nabla \psi_0-F\psi_0 ,\nabla \ptb \psi_k)_{L^2}
%\end{aligned}$$
%Moreover, since we assumed an initial in $H^2(\Omega)$ and we want to avoid giving $\ptb \psi_k$ a gradient in the last term at this moment (see Eq... where we do this; however, we require $\alpha>\frac12$ in this case as we will see), we integrate by parts and obtain
%$$\begin{aligned} &\frac12 \pta \|\ptb \psi_k\|_{L^2}^2 + D \|\nabla \ptb \psi_k\|_{L^2}^2 \\  &\leq  - \frac12 (\div F,|\ptb \psi_k|^2)_{L^2}+(f,\ptb \psi_k)_{L^2} \\
%&\quad + \ga(\div D^\top \cdot \nabla \psi_0  + D : \nabla\nabla \psi_0 - \div F \psi_0 - F \cdot \nabla \psi_0,\ptb \psi_k)_{L^2}
%\end{aligned}$$
%We estimate the right-hand side by the H\"older inequality, which yields
%$$\begin{aligned} &\frac12 \pta \|\ptb \psi_k\|_{L^2}^2 + D \|\nabla \ptb \psi_k\|_{L^2}^2 \\  &\leq   \frac12 \|\div F\|_{L^\infty} \|\ptb \psi_k\|^2_{L^2}+\|f\|_{H^{-1}} \|\nabla \ptb \psi_k\|_{L^2} \\ &\quad + \ga \|\nabla\ptb \psi_k\|_{L^2} \Big(D\|\nabla \psi_0\|_{L^2}  + \|F\|_{L^4} \|\psi_0\|_{L^4} \Big)
%+ \ga  \|\ptb \psi_k\|_{L^2} \Big(\|\div D^\top\|_{L^4} \|\nabla \psi_0\|_{L^4} + \|D\|_{L^\infty} \|\nabla\nabla \psi_0\|_{L^2} \\&\quad + \|\div F\|_{L^\infty} \|\psi_0\|_{L^2} + \|F\|_{L^4} \|\nabla \psi_0\|_{L^4}
%\Big)
%\end{aligned}$$
%We estimate the $L^4(\Omega)$-norm of $\psi_0$ by the $L^2(\Omega)$-norm of $\nabla\psi_0$ due to the Sobolev embedding theorem. Moreover, we separate the terms on the right-hand side of the inequality using the $\eps$-Young inequality as follows
Lastly, we estimate the external force $f$ by
\begin{equation} \label{Eq:Tested1d}
\begin{aligned} \langle f,\ptb \psi_k\rangle_{H_0^1} &\leq \|f\|_{H^{-1}} \|\ptb \psi_k\|_{H_0^1} \\ &\leq C\|f\|_{H^{-1}}^2 + \frac{D}{4} \|\ptb \nabla \psi_k\|_{L^2}^2.
\end{aligned}
\end{equation}

Hence, we insert the estimates \eqref{Eq:Tested1a}--\eqref{Eq:Tested1d} in the tested equation \eqref{Eq:Tested1} to obtain the inequality
$$\begin{aligned} &\frac12 \pta \|\ptb \psi_k\|_{L^2}^2 + D \|\nabla \ptb \psi_k\|_{L^2}^2 \\  &\leq F_\infty \|\ptb \psi_k\|^2_{L^2}+C\|f\|_{H^{-1}}^2  + \frac{D}{2} \|\nabla \ptb \psi_k\|_{L^2}^2   \\&\quad + \eps_1 \ga(t) \|\nabla \ptb \psi_k\|_{L^2}^2  +Cg_{\alpha}(t)  \|\psi_0\|_{H_0^1}^2,
\end{aligned}$$
and we absorb the terms involving $D$ on the right-hand side by the respective term on the left-hand side, giving
$$\begin{aligned} &\frac12 \pta \|\ptb \psi_k\|_{L^2}^2 + \frac{D}{2} \|\nabla \ptb \psi_k\|_{L^2}^2 \\  &\leq F_\infty \|\ptb \psi_k\|^2_{L^2}+C\|f\|_{H^{-1}}^2    + \eps_1 \ga(t) \|\nabla \ptb \psi_k\|_{L^2}^2  +Cg_{\alpha}(t)  \|\psi_0\|_{H_0^1}^2.
\end{aligned}$$
We convolve this inequality with the kernel function $\ga$ to conclude
\begin{equation} \label{Eq:Galerkin2} \begin{aligned} &\frac12 \|\ptb \psi_k(t)\|_{L^2}^2 + \frac{D}{2} (\ga*\|\nabla \ptb \psi_k\|_{L^2}^2)(t) \\  &\leq F_\infty^2   (\ga *  \|\ptb \psi_k\|_{L^2}^2)(t)  + C(\ga*\|f\|_{H^{-1}}^2)(t)  \\ &\quad +\eps_1 (\ga*(\ga\cdot \|\nabla \ptb \psi_k\|_{L^2}^2))(t)  + C g_{2\alpha}(t)  \| \psi_0\|^2_{H^1_0}.
\end{aligned}
\end{equation}
where we used that $\ga*g_{\alpha}=g_{2\alpha}$, see \eqref{Eq:Semigroup}, and $$\begin{aligned}\big(\ga*(\pta \|\ptb \psi_k\|_{L^2}^2)\big)(t) &=\|\ptb \psi_k(t)\|_{L^2}^2 - \|(\ga*\pt \psi_k)(0)\|_{L^2}^2 \\ &= \|\ptb \psi_k(t)\|_{L^2}^2,
\end{aligned}$$ see \eqref{Eq:InverseConvolution}. Now, we observe that  the term $(\ga*\|f\|_{H^{-1}}^2)(t)$, $t \in (0,T_k)$,  can be bounded by
$$(\ga*\|f\|_{H^{-1}}^2)(t) \leq \sup_{t \in (0,T)} (\ga*\|f\|_{H^{-1}}^2)(t) =: \|f\|_{L^2_\alpha H^{-1}},$$
see again \eqref{Eq:LpAlpha} for the definition of the space $L^2_\alpha(0,T)$.

Further, we use \eqref{Eq:IneqGaG1Conv} to absorb the term involving $\eps_1$ on the right-hand side of the inequality \eqref{Eq:Galerkin2}. In fact, we absorb it by the term $\frac{D}{2} (\ga*\|\nabla \ptb \psi_k\|_{L^2}^2)(t)$ on the left-hand side  by noting that
$$\eps_1 (\ga * (\ga\|\nabla \psi_k\|_{L^2L^2}^2))(t)  \leq \eps_1 g_{\alpha+1}(T)  (\ga* \|\nabla \psi_k\|_{L^2}^2)(t).$$
We choose $\eps_1=\frac{D}{4g_{\alpha+1}(T)}$ to get 
$$\eps_1 (\ga * (\ga\|\nabla \psi_k\|_{L^2L^2}^2))(t)  \leq \frac{D}{4}  (\ga* \|\nabla \psi_k\|_{L^2}^2)(t),$$
and consequently, we obtain from \eqref{Eq:Galerkin2}  the inequality
$$\begin{aligned} &\frac12 \|\ptb \psi_k(t)\|_{L^2}^2 + \frac{D}{4} (\ga*\|\nabla \ptb \psi_k\|_{L^2}^2)(t) \\  &\leq F_\infty^2   (\ga *  \|\ptb \psi_k\|_{L^2}^2)(t)  + C\|f\|_{L^2_\alpha H^{-1}}^2   + C g_{2\alpha}(t)  \| \psi_0\|^2_{H^1_0}.
\end{aligned}$$

We notice that we are in the situation of the extended Henry--Gr\"onwall lemma, see Lemma \ref{Lem:GronFrac}, and we obtain the energy estimate
\begin{equation} \label{Eq:FinalEst1} \begin{aligned} &\frac12 \|\ptb \psi_k(t)\|_{L^2}^2 +  \frac{D}{4} \|\nabla \ptb \psi_k\|_{L^2_tL^2}^2 \\  &\leq C(F_\infty, \alpha, T) \cdot \Big((g_0+E)*\big( \|f\|_{L^2_\alpha H^{-1}}^2 +  g_{2\alpha}  \| \psi_0\|^2_{H^1_0} \big)\Big)(t) \\
		&=: \text{RHS}_{\eqref{Eq:FinalEst1}}(t)
\end{aligned}
\end{equation}
%WHAT IF WE TRY TO ABSORB WITH NABLA
%$$\ga*\|\nabla \ptb \psi_k\|-\eps g_1 * (\ga \cdot \|\nabla \ptb \psi_k\|) \geq \int_0^t (t^{\alpha-1}-\eps \ga(s)) \|\nabla \ptb \psi_k\|^2 \ds  $$
%and this is positive for 
%$$t^{\alpha-1} \geq \eps s^{\alpha-1}$$
%and $s \in (0,t)$ i.e. direction goes in other direction sadly ...
%on the other hand what about
%$$(t-s)^{\alpha-1} \geq \eps s^{\alpha-1}$$
%e.g. if $\alpha=1/2$ then $(t-s)^{-1/2} \geq \eps s^{-1/2}$ if $s^{1/2} \geq \eps (t-s)^{1/2}$ i.e. if $s \geq \eps(t-s)$ i.e. $(\eps+1)s \geq \eps t$ i.e. for $s \geq \frac{\eps}{\eps+1} t$
%if we integrate then
%$$\gb * \| \ptb \psi_k\|^2 - \|\div F\| g_1*\|\ptb \psi_k\|^2\geq (t^{-\alpha} -\|\div F\|) \|\ptb \psi_k\|_{L^2L^2}^2 $$
%if $t^{-\alpha}>\|\div F\|$ i.e. $t < \|\div F\|^{-1/\alpha}$
%THIS WORKS FOR DIVF IN LINFLINF AND IF ALPHA>1/3. PREFER TO ASSUME INITIAL IN H2 SO WE DO NOT HAVE TO SQUARE THE GALPHA TERM ON THE RHS. WE CAN ALSO STATE AN ARGUMENT IF D BIGGER THAN DIVF. WITH H2 WE GET
%$$\begin{aligned} &\|\ptb \psi_k\|_{L^2}^2 + D (\ga*\|\nabla \ptb \psi_k\|_{L^2}^2)(t) \\  &\leq    \Big(\ga * (\|\div F\|_{L^\infty}+\|\psi_0\|_{H^2} \ga) \|\ptb \psi_k\|_{L^2}^2\Big)(t)  + C(\ga*\|f\|_{H^{-1}}^2)(t) 
%\end{aligned}$$
%At this point, we would normally convolve the inequality by the singular kernel function $\ga$. However, on the right-hand side we have in front of the initial already the term $\ga$ and therefore, we get 
%$$\begin{aligned}
%&\|\ptb \psi_k\|_{L^2}^2 + D \big(\ga*\|\nabla \ptb \psi_k\|_{L^2}^2\big)(t) \\  &\leq   C\big( \ga*\|f\|_{(H^1)'}^2\big)(t) + \Big( \ga*\|\ptb \psi_k\|_{L^2}^2 \big( \|\div F\|_{L^\infty}+C  \big) \Big)(t)  \\ &\quad   + C g_{3\alpha-1}(t)  \| \psi_0\|^2_{H^1} \Big(  \|D\|_{L^\infty L^\infty}^2   +  \|F\|_{L^\infty L^4}^2  \Big).
%\end{aligned}$$
%but $g_{3\alpha-1}$ is only in $L^1$ for $\alpha<1/3$. 
%Now convolve (or integrate the convolved one) to get
The estimate on the right-hand side is independent of $T_k$ and we infer from the no-blow-up theorem that we can continue the maximal time to $T$. However, since the right-hand side is "only" continuous in $t$ on $(0,T]$ and not at $t=0$ because of the presence of the term $g_{2\alpha}$, we are not able to take the essential supremum of the inequality  \eqref{Eq:FinalEst1} over $t \in (0,T)$. Therefore, we can obtain no bound of $\ptb \psi_k$ in $L^\infty$-in-time. Nonetheless,  $\ptb \psi_k$ is bounded in $L^2(0,T;H_0^1(\Omega))$ by inserting $t=T$ into the inequality \eqref{Eq:FinalEst1}. Moreover, we notice that $\|\ptb \psi_k(t)\|_{L^2(\Omega)}$ is bounded by the leading term $\sqrt{g_{2\alpha}(t)}=t^{\alpha-1/2}=g_{\alpha+1/2}$, which is continuous at $t=0$ for $\alpha > 1/2$ and in $L^p(0,T)$ for $\alpha+\frac12 > 1-\frac{1}{p}$, which is equivalent to $p<\frac{2}{1-2\alpha}$. Therefore, $\ptb \psi_k$ is bounded in the space $L^p(0,T;L^2(\Omega))$ with
\begin{equation} \label{Eq:LimitP} \begin{cases}
			p<\frac{2}{1-2\alpha}, &\alpha<\frac12, \\
	p<\infty, &\alpha=\frac12,\\
	p=\infty, &\alpha>\frac12.
\end{cases} \end{equation}
%E.g., we obtain a bound in $L^6(0,T;L^2(\Omega))$ for $\alpha=\frac13$.

By the Eberlein--Smulian and Banach--Alaoglu theorems, see \cite[8.7]{alt2016linear}, these bounds yield the existence of a weakly converging subsequence $\ptb\psi_{k_j}$, i.e., it holds
\begin{equation}\label{Eq:Convergence1}\ptb \psi_{k_j} \longweak \zeta \quad \text{ in } L^p(0,T;L^2(\Omega)) \cap L^2(0,T;H_0^1(\Omega)),
\end{equation}
as $j \to \infty$.
%We may insert this above to also obtain that $\ptb \nabla \psi_k$ is bounded in $L^2(0,T;L^2(\Omega))$. 
We still need to figure out the representation of $\zeta$. If we are able to bound $\psi_k$ again in the Bochner space $L^2(0,T;H_0^1(\Omega))$, then we would obtain $\psi_k \rightharpoonup \psi$ for some limit function $\psi$, from which we can infer $\zeta=\ptb \psi$. We want to mention that we could also have obtained a bound of $\ptb \psi_k$ in the space $L^2_\alpha(0,T;H_0^1(\Omega))$. However, this space is not known to be reflexive and therefore, we cannot apply the Banach--Alaoglu theorem to infer a limit function in this space. \medskip

\noindent \textbf{(3) Energy estimates: Part 2.}
In order to obtain the desired bound of $\psi_k$, we test the Galerkin equation \eqref{Eq:FP_dis} with $\psi_k$, which yields
\begin{equation} \label{Eq:Galerkin3} \begin{aligned} &\frac12 \ddt \|\psi_k\|_{L^2}^2 + D (\ptb \nabla \psi_k,\nabla \psi_k)_{L^2}
\\ 
&= (F\cdot \nabla\ptb \psi_k,\psi_k)_{L^2}  +  \langle f,\psi_k \rangle_{H_0^1} - \ga( D\nabla \psi_{0k} - F\psi_{0k},\nabla \psi_k)_{L^2}.
\end{aligned}\end{equation}
For the term on the left-hand side involving the diffusion $D$, we apply Alikhanov's inequality \eqref{Eq:ChainOriginal} to infer
$$D (\ptb \nabla \psi_k,\nabla \psi_k)_{L^2} \geq \frac{D}{2} \ptb \|\nabla \psi_k\|_{L^2}^2.$$
Moreover, we apply again the H\"older inequality on the right-hand side of \eqref{Eq:Galerkin3} to obtain the energy estimate
$$\begin{aligned} &\frac12 \ddt \|\psi_k\|_{L^2}^2 + \frac{D}{2} \ptb \|\nabla \psi_k\|_{L^2}^2
\\ 
&\leq F_\infty  \|\nabla\ptb \psi_k\|_{L^2} \|\psi_k\|_{L^2}  +  \|f\|_{H^{-1}} \|\nabla \psi_k\|_{L^2}  \\ & \quad + \ga(t) \|\nabla \psi_k\|_{L^2} \big( D\|\nabla \psi_0\|_{L^2} + F_\infty \|\psi_0\|_{L^2} \big),
\end{aligned}$$
where we used the boundedness of the projection operator, see \eqref{Eq:Projection}.
Again, with the Young inequality, we obtain
$$\begin{aligned} &\frac12 \ddt \|\psi_k\|_{L^2}^2 + \frac{D}{2} \ptb \|\nabla \psi_k\|_{L^2}^2
\\ 
&\leq C F_\infty^2 \|\nabla \ptb \psi_k\|_{L^2}^2 +  \frac{\eps_2}{2}\|\nabla\psi_k\|_{L^2}  + C \|f\|_{H^{-1}}^2 + \frac{\eps_2}{2} \|\nabla \psi_k\|_{L^2}^2  \\ & \quad + \eps_3 \ga(t) \|\nabla \psi_k\|_{L^2}^2 + C\ga(t) \|\psi_0\|_{H_0^1}^2,
\end{aligned}$$
for some $\eps_2,\eps_3>0$ that we determine below.
After integrating this inequality over the time interval $(0,t)$, $t \leq T$, it yields
\begin{equation} \label{Eq:Estimate5} \begin{aligned} &\frac12 \|\psi_k(t)\|_{L^2}^2 + \frac{D}{2} (\ga* \|\nabla \psi_k\|_{L^2}^2)(t)
\\ 
&\leq \frac12 \|\psi_{0k}\|_{L^2}^2 + C F_\infty^2 \|\nabla \ptb \psi_k\|_{L^2_tL^2}^2   + C \|f\|_{L^2H^{-1}}^2 + \eps_2 \|\nabla \psi_k\|_{L^2_tL^2}^2   \\&\quad + \eps_3 \int_0^t \ga(s) \|\nabla \psi_k(s)\|_{L^2}^2 \ds + C g_{\alpha+1}(T) \|\psi_0\|_{H_0^1}^2 ,
\end{aligned}
\end{equation}
where we used that $\ga*1=g_{\alpha+1}$, see \eqref{Eq:Semigroup}, which is a continuous and bounded function on $[0,T]$ for any $\alpha>0$. 

We use the energy estimate \eqref{Eq:FinalEst1} from before to infer
\begin{equation} \label{Eq:Help1} 
\|\nabla \ptb \psi_k\|_{L^2_tL^2}^2 \leq \|\nabla \ptb \psi_k\|_{L^2L^2}^2 \leq \text{RHS}_{\eqref{Eq:FinalEst1}}(T).
\end{equation}
Furthermore, we use the auxiliary result \eqref{Eq:KernelNorm} to get
\begin{equation} \label{Eq:Help2} \begin{aligned} 
\eps_2 \|\nabla \psi_k\|_{L^2_tL^2}^2 &\leq \frac{\eps_2}{\ga(T)}  (\ga*\|\nabla \psi_k\|^2_{L^2})(t) \\ &\leq \frac{D}{8} (\ga*\|\nabla \psi_k\|^2_{L^2})(t),
\end{aligned}
\end{equation}
where we have chosen $\eps_2=\frac{D\ga(T)}{8}$.
Lastly, we use again \eqref{Eq:IneqGaG1} to infer
\begin{equation} \label{Eq:Help3} \begin{aligned} \eps_3 \int_0^t \ga(t) \|\nabla \psi_k\|_{L^2}^2 \ds \leq \eps_3 T (\ga* \|\nabla \psi_k\|_{L^2}^2)(t) \leq \frac{D}{8} (\ga* \|\nabla \psi_k\|_{L^2}^2)(t).
\end{aligned}
\end{equation}
Therefore, we set $\eps_3=\frac{D}{8T}$ and together with the auxiliary estimates \eqref{Eq:Help1}--\eqref{Eq:Help3} we obtain from \eqref{Eq:Estimate5}
\begin{equation} \label{Eq:FinalEst2} \begin{aligned} &\frac12 \|\psi_k(t)\|_{L^2}^2 + \frac{D}{4} (\ga* \|\nabla \psi_k\|_{L^2}^2)(t)
\\ 
&\leq  C \cdot \text{RHS}_{\eqref{Eq:FinalEst1}}(T)  + C \|f\|_{L^2H^{-1}}^2  + C g_{\alpha+1}(T) \|\psi_0\|_{H_0^1}^2 \\
&=:\text{RHS}_{\eqref{Eq:FinalEst2}}.
\end{aligned}\end{equation}
%where we used that the initial $\psi_k(0)$ is defined by $\Pi_{H_k} \psi_0$ and that the projection $\Pi_{H_k}:L^2(\Omega) \to L^2(\Omega)$ is a linear and bounded operator.

\noindent\textbf{(4) Weak and strong convergences.} From the estimate that we derived in \eqref{Eq:FinalEst2} we infer that that $\psi_k$ is bounded in the spaces $L^\infty(0,T;L^2(\Omega))$ and $L^2(0,T;H_0^1(\Omega))$, i.e, there is a limit function $\psi$ with
\begin{equation} \label{Eq:Convergence2a} \begin{aligned}
\psi_{k_j} &\longweak \psi \quad \text{ in } L^2(0,T;H_0^1(\Omega)), \\
\psi_{k_j} &\overset{*}{\longweak} \psi \quad \text{ in } L^\infty(0,T;L^2(\Omega)),
\end{aligned}\end{equation}
as $j \to \infty$.
By linearity of the differential operators, we obtain from \eqref{Eq:Convergence1}
\begin{equation} \label{Eq:Convergence2} \ptb \psi_{k_j} \longweak \ptb \psi \quad \text{ in } L^p(0,T;L^2(\Omega)) \cap L^2(0,T;H_0^1(\Omega)),\end{equation}
as $j \to \infty$, with $p$ as defined in \eqref{Eq:LimitP}. We note the compact embedding, see \eqref{Eq:aubinfractional},
 $$ L^2(0,T;H_0^1(\Omega)) \cap H^{1-\alpha}(0,T;H_0^1(\Omega)) \hookrightarrow\hookrightarrow L^2(0,T;L^2(\Omega)),$$
 from which we obtain the strong convergence 
   \begin{equation}\label{Eq:StrongConv1}\begin{aligned}
\psi_{k_j} &\longrightarrow \psi \quad &&\text{ in } L^2(0,T;L^2(\Omega).
  \end{aligned}\end{equation}

The derived convergences \eqref{Eq:Convergence2a}--\eqref{Eq:StrongConv1} are enough to pass to the limit in the Galerkin equation \eqref{Eq:FP_dis}. Nonetheless, we want to derive an additional estimate on $\pt \psi_k$ by testing the discretized Fokker--Planck equation \eqref{Eq:FP_dis} with $\Pi_{H_k} \zeta$ where $\zeta$ is an arbitrary element in $L^r(0,T;H_0^1(\Omega))$ with $r\geq 1$ depending on $\alpha$ (will be specified below). Then we obtain by the usual inequalities
$$\begin{aligned}(\pt\psi_k,\Pi_{H_k}\zeta )_{L^2L^2} 
  &=   (F \ptb \psi_k,\nabla \Pi_{H_k}\zeta)_{L^2L^2}-D(\ptb \nabla \psi_k,\nabla \Pi_{H_k}\zeta)_{L^2L^2}  \\ &\quad +\langle f,\Pi_{H_k}\zeta\rangle_{L^2H_0^1} - (  D \nabla \psi_{0k} - F \psi_{0k},\ga\nabla\Pi_{H_k}\zeta)_{L^2L^2}
  \\ 
  &\leq F_\infty \|\ptb \psi_k\|_{L^2L^2} \|\nabla \zeta\|_{L^2L^2} +D \|\ptb \nabla \psi_k\|_{L^2L^2}  \|\nabla \zeta\|_{L^2L^2}  \\ &\quad+\|f\|_{L^2H^{-1}}  \|\zeta\|_{L^2H_0^1}  + (F_\infty+D)\|\ga\|_{L^q}   \|\psi_0\|_{H^1}  \|\nabla\zeta\|_{L^{q'} L^2}   \\
  &\leq C \|\zeta\|_{L^r H_0^1},
  \end{aligned}$$ 
  where $r=\max\{q',2\}$ and $q'$ is the H\"older conjugate of $q=\frac{1}{1-\alpha}-\eps$ for $\eps\in (0,\frac{\alpha}{1-\alpha}]$.
  Therefore, $\pt \psi_k$ is bounded in $L^{r'}(0,T;H^{-1}(\Omega))$ where $r'$ is the H\"older conjugate of $r$. We note the compact embeddings, see \eqref{Eq:aubin}--\eqref{Eq:aubinfractional},
  $$\begin{aligned} 
  %L^2(0,T;H_0^1(\Omega)) \cap W^{1,r'}(0,T;H^{-1}(\Omega)) &\hookrightarrow\hookrightarrow L^2(0,T;L^2(\Omega)), \\
  L^\infty(0,T;L^2(\Omega)) \cap W^{1,r'}(0,T;H^{-1}(\Omega)) &\hookrightarrow\hookrightarrow C([0,T];H^{-1}(\Omega)), \\
    W^{1-\alpha,2}(0,T;H_0^1(\Omega)) \cap W^{1,r'}(0,T;H^{-1}(\Omega))  &\hookrightarrow\hookrightarrow W^{1-\alpha,2}(0,T;L^2(\Omega)),
  \end{aligned}$$
  which provides us with the strong convergences (as $j\to \infty$)
  \begin{equation}\label{Eq:StrongConv}\begin{aligned}
%\psi_{k_j} &\longrightarrow \psi \quad &&\text{ in } L^2(0,T;L^2(\Omega), \\
\psi_{k_j} &\longrightarrow \psi \quad &&\text{ in } C([0,T];H^{-1}(\Omega), \\
\ptb \psi_{k_j} &\longrightarrow \ptb \psi \quad &&\text{ in } L^2(0,T;L^2(\Omega)). 
  \end{aligned}\end{equation}
\medskip

\noindent\textbf{(5) Limit process.}
In this step, we pass to the limit $j \to \infty$ in the  time-integrated $k_j$-th Galerkin system \eqref{Eq:FP_dis}. We use the derived convergences from the preceding result and show that the weak limit function $\psi$ satisfies the variational form of the time-fractional Fokker--Planck equation, i.e., $\psi$ is a weak solution in the sense of Definition \ref{Def:Weak}.

We consider the time-integrated $k_j$-th Galerkin system
$$\begin{aligned} 
& \int_0^T \Big( \langle \psi_{k_j}',\zeta \rangle_{H_0^1} +D(\ptb \nabla \psi_{k_j},\nabla \zeta)_{L^2} - (F \ptb \psi_{k_j},\nabla \zeta)_{L^2} \Big) \eta(t) \dt \\ &=\int_0^T \Big( \langle f,\zeta\rangle_{H_0^1} - \ga( D \nabla \psi_0 - F \psi_0,\nabla\zeta)_{L^2} \Big) \eta(t) \dt  \end{aligned}$$ for all 
%$v \in V_{k_j}$, $w \in C_0^\infty(0,T)$, 
$\zeta \in H_{k_j}$ and $\eta \in C_c^\infty(0,T)$.  Obviously, we are able to pass to the limit in all the terms thanks to the derived weak  convergences. E.g., we have
$$\begin{aligned}\int_0^T  (F\ptb \psi_{k_j},\nabla \zeta)_{L^2} \eta(t) \dt &\leq F_\infty \|\ptb \psi_{k_j} \|_{L^2L^2} \|\nabla\zeta\|_{L^2} \|\eta\|_{L^2}  \\ &\leq C \|\ptb\psi_{k_j} \|_{L^2H^1_0},
\end{aligned}$$
for all $\zeta \in H_{k_j}$, $\eta \in C_c^\infty(0,T)$. Since it holds the weak convergence 
$$\ptb\psi_{k_j} \rightharpoonup \ptb\psi_{k_j} \text{ in } L^2(0,T;H_0^1(\Omega)),$$ see \eqref{Eq:Convergence2}, it yields (as $j \to \infty$)
$$\int_0^T  (F \ptb \psi_{k_j},\nabla \zeta)_{L^2} \eta(t) \dt \longrightarrow \int_0^T (F \ptb \psi,\nabla \zeta)_{L^2} \eta(t) \dt,$$
for all $\zeta \in \cup_{j} H_{k_j}$. We observe that $\cup_j H_{k_j}$ is dense in $H_0^1(\Omega)$, which implies that the limit function $\psi$ indeed solves the variational form of the time-fractional Fokker--Planck equation. %Therefore, we only need to investigate the nonlinear terms of the Galerkin system. 
%In the case of the nonlinear parts in the Navier--Stokes equations, it holds for any $\varphi \in L^2(0,T;V_0)$
%$$\int_0^T  ((u_k \cdot \nabla)u_k,\varphi)_{L^2} + ((u_k \cdot \nabla)v_k,\varphi)_{L^2} \dt = \int_0^T \int_\Omega \div((u_k \cdot \varphi)u_k) \, \text{d}x \dt=0,$$
%and together with the strong convergence of $u_{k_j}$ we conclude that the weak limit $u$ indeed satisfies the variational form of the Navier--Stokes equations. Moreover, the condition on the initial data in \eqref{Eq:DefWeak} is satisfied thanks to the strong convergence of $u_{k_j}$ to $u$ in $C([0,T];V_0')$. Therefore, it holds $u_{k_j}(0) \to u(0)$ in $V_0'$, i.e., $u(0)=u^0$ in $V_0'$ by the uniqueness of limits.


%In the case of the Fokker--Planck equation, we apply integration by parts and exploit the corotationality of $\omega$ in the form of
%$$(M\omega(u)q,\eta)_Y = \frac12 (M v \cdot q, \div \eta)_Y - \frac12 (M\nabla \eta q,v)_Y,$$
%which allows us to rewrite \eqref{Eq:FP_dis_time} as
%$$\begin{aligned} 0=&- \int_0^T ( M\gb * \phi_{k_j},\zeta)_Y \eta'(t) \dt - ( M\psi_0,\zeta)_{Y} \eta(0) && \\[-.2cm]  &+ \frac12 \int_0^T \Big(  (M u_{k_j},\nabla \phi_{k_j} \zeta)_Y  - (M u_{k_j},\nabla \zeta \phi_{k_j})_Y  + \frac{1}{\lambda} (M \nabla \phi_{k_j},\nabla \zeta)_Y  && \\[-.2cm]     &+ 2\eps (M\nabla \phi_{k_j},\nabla \zeta)_Y- \big(M u_{k_j} \cdot q, \div(\phi_{k_j} \nabla \zeta)\big)_Y +  (M\nabla (\phi_{k_j} \nabla \zeta) q,u_{k_j})_Y \Big) \eta(t) \dt.
%\end{aligned}$$ 
\medskip

\noindent\textbf{(6) Initial condition.}
  By the strong convergences, see \eqref{Eq:StrongConv}, we obtain at $t=0$ the convergence $\psi_{k_j}(0) \to \psi(0)$ in $H^{-1}(\Omega)$. However, it also holds $\psi_k(0) = \Pi_{H_k} \psi_0 \to \psi_0$ in $H_0^1(\Omega)$ as $j \to \infty$, from which we conclude $\psi(0) = \psi_0$ by the uniqueness of limits. Therefore, $\psi$ is a weak solution to the time-fractional Fokker--Planck equation in the sense of Definition \ref{Def:Weak}.  
  \medskip
  
 \noindent\textbf{(7) Uniqueness.} We consider two weak solutions $\psi_1$ and $\psi_2$ of the time-fractional Fokker--Planck equation in the sense of Definition \ref{Def:Weak}. Both solutions shall have the same initial data $\psi_0$ and outer force $f$. We subtract the variational forms of $\psi_1$ and $\psi_2$ from each other, and we define $\psi=\psi_1-\psi_2$, which satisfies 
 \begin{equation} \label{Eq:Uniqueness} \begin{aligned}
  &\langle \pt \psi,\zeta \rangle_{H_0^1}
  +  D(\ptb \nabla \psi,\nabla \zeta)_{L^2} - (F \ptb \psi,\nabla \zeta)_{L^2} = 0 \quad \forall \zeta \in H_0^1(\Omega).
\end{aligned} \end{equation}
%I HAVE TO CHECK THIS TEST FUNCTION IF IT IS REALLY IN H01
We consider the test function $\zeta = \ptb\psi(t) \in H_0^1(\Omega)$ for a.e. $t \in (0,T)$, which yields together with Alikhanov's  inequality, see \eqref{Eq:ChainOriginal},
\begin{equation*} \begin{aligned}
  &\frac12  \pta \|\ptb \psi\|_{L^2}^2
  +  D \|\ptb\nabla \psi\|_{L^2}^2 \leq F_\infty \|\ptb \psi\|_{L^2} \|\nabla \ptb \psi\|_{L^2}.
\end{aligned}\end{equation*}
Furthermore, we apply Young's inequality to obtain 
\begin{equation} \label{Eq:Unique} \begin{aligned}
  &\frac12  \pta \|\ptb \psi\|_{L^2}^2
  +  \frac{D}{2} \|\ptb\nabla \psi\|_{L^2}^2 \leq \frac{F_\infty^2}{D} \|\ptb \psi\|_{L^2}^2.
\end{aligned}\end{equation}
After convolving this inequality with $\ga$ and applying the extended Henry--Gronwall inequality with $a\equiv 0$, see Lemma \ref{Lem:GronFrac}, the estimate \eqref{Eq:Unique} becomes
\begin{equation*} \begin{aligned}
  &\frac12 \|\ptb \psi(t)\|_{L^2}^2
  +  \frac{D}{2} \|\ptb\nabla \psi\|_{L^2_tL^2}^2 \leq 0.
\end{aligned}\end{equation*}
At this point, we further test the variational form \eqref{Eq:Uniqueness} by $\zeta=\psi(t) \in H_0^1(\Omega)$, which yields
 \begin{equation*} \begin{aligned}
  &\frac12 \ddt \|\psi\|_{L^2}^2
  +  \frac{D}{2} \ptb \|\nabla \psi\|_{L^2}^2 \leq F_\infty \|\ptb \psi\|_{L^2} \|\nabla \psi\|_{L^2} = 0.
\end{aligned}\end{equation*}
We integrate this inequality and observe that it holds $\|\psi(t)\|_{L^2}=0$ for any $t \in (0,T)$ i.e. $\psi_1=\psi_2$.
  \qed

%\subsection{Proof of Theorem \ref{Thm:HigherRegularity}}

%Let $\psi$ be the unique weak solution that exists according to \eqref{Thm:WellPosedness}. It satisfies the variational form 
%\begin{equation}\label{Eq:FP2} \begin{aligned}
%		&\langle \pt \psi,\zeta \rangle_{H_0^1}		+  D(\ptb \nabla \psi,\nabla \zeta)_{L^2} - (F \ptb \psi,\nabla \zeta)_{L^2} \\ &=\langle f,\zeta\rangle_{H_0^1} - (\ga D \nabla \psi_0,\nabla \zeta)_{L^2} + \ga \cdot (F \psi_0,\nabla \zeta)_{L^2}.
%\end{aligned} \end{equation}
%In this theorem, we have assumed $\psi_0 \in H^2(\Omega)$. We derive the energy estimate formally since we would need to do this procedure on a discrete level but we skip this step at this point. 
%We formally consider $\zeta=(-\Delta) \ptb \psi$, which gives
%\begin{equation*} \begin{aligned}
%		&( \pt \nabla \psi,\ptb \nabla \psi)_{L^2}
%		+  D\|\ptb \Delta \psi\|^2_{L^2} \\
%		&= (\div F \ptb \psi,\Delta\ptb \psi)_{L^2}+ (F \cdot \ptb \nabla \psi,\Delta \ptb \psi)_{L^2} + (f,\Delta \ptb \psi)_{L^2}  \\ &\quad - (\ga D \Delta \psi_0,\Delta \ptb \psi)_{L^2}  + \ga (\div F \psi_0+F\cdot \nabla \psi_0,\Delta \ptb\psi)_{L^2}
%\end{aligned} \end{equation*}
%By applying the Alikhanov, H\"older and Young inequalities, we obtain
%\begin{equation*} \begin{aligned}
%		&\frac12 \pta \|\nabla \ptb \psi\|_{L^2}^2
%		+ D  \|\Delta \ptb \psi\|_{L^2}^2 \\
%		&\leq C \|\div F\|_{L^4}^2 \|\ptb \psi\|_{L^4}^2 +\frac{D}{8} \|\Delta \ptb \psi\|_{L^2}^2+  (F \cdot \ptb \nabla \psi,\Delta \ptb \psi)_{L^2}  \\ &\quad + (f,\Delta \ptb \psi)_{L^2}  - (\ga D \Delta \psi_0,\Delta \ptb \psi)_{L^2}  + \ga (\div F \psi_0+F\cdot \nabla \psi_0,\Delta \ptb\psi)_{L^2}
%\end{aligned} \end{equation*}

%Further, we formally consider $\zeta=(-\Delta) \psi$, which gives
%\begin{equation*} \begin{aligned}
%		&( \pt \nabla \psi,\nabla \psi)_{L^2}
%		+  D(\ptb \Delta \psi,\Delta \psi)_{L^2} \\
%		&= (\div F \ptb \psi,\Delta \psi)_{L^2}+ (F \cdot \ptb \nabla \psi,\Delta \psi)_{L^2} + (f,\Delta \psi)_{L^2}  \\ &\quad - (\ga D \Delta \psi_0,\Delta \psi)_{L^2}  + \ga \cdot (\div F \psi_0,\Delta \psi)_{L^2}+(F\cdot \nabla \psi_0,\Delta \psi)_{L^2}
%\end{aligned} \end{equation*}
%By applying the Alikhanov, H\"older and Young inequalities, we obtain
%\begin{equation*} \begin{aligned}
	%	&\frac12 \ddt \|\nabla \psi\|_{L^2}^2
	%	+  \frac{D}{2} \ptb  \|\Delta \psi\|_{L^2}^2 \\
	%	&\leq (\div F \ptb \psi,\Delta \psi)_{L^2}+ (F \cdot \ptb \nabla \psi,\Delta \psi)_{L^2} + (f,\Delta \psi)_{L^2}  \\ &\quad - (\ga D \Delta \psi_0,\Delta \psi)_{L^2}  + \ga \cdot (\div F \psi_0,\Delta \psi)_{L^2}+(F\cdot \nabla \psi_0,\Delta \psi)_{L^2}
%\end{aligned} \end{equation*}

%\subsection{Proof of Theorem \ref{Thm:SpecialF}}

%We only derive the energy estimates for the continuous equation with the affine linear $F$. The limit works accordingly to the proof of Theorem \ref{Thm:WellPosedness}. We consider the test function $\zeta=\psi$ yielding 
%$$\begin{aligned} &(\pta \psi,\psi)_{L^2}+  D\|\nabla \psi\|_{L^2}^2  \\ &\leq  (F\psi,\nabla \psi)_{L^2} + \langle f,\psi\rangle_{H_0^1} + (1-\alpha) (b g_{2-2\alpha} * \psi,\psi)_{L^2}.\end{aligned}$$
%We apply the Alikhanov, H\"older and Young inequalities to obtain 
%$$\begin{aligned} &\frac12 \pta \|\psi\|_{L^2}^2+  D\|\nabla \psi\|_{L^2}^2  \\ &\leq \frac{\|a\|_\infty + T \|b\|_\infty}{D} \|\psi\|_{L^2}^2 + \frac{D}{4} \|\nabla \psi\|_{L^2} + \frac{1}{D} \|f\|_{H^{-1}}^2 \\ &\quad +\frac{D}{4} \|\nabla \psi\|_{L^2}^2 + \eps \|g_{2-2\alpha} * \psi\|_{L^2}^2 + \frac{(1-\alpha)^2}{4\eps}\|b\|_{L^\infty}^2\|\psi\|^2_{L^2}.\end{aligned}$$
 % It holds after integration
%$$\begin{aligned} \int_0^t \|(g_{2-2\alpha}*\psi)(s)\|_{L^2}^2 \ds \leq \int_0^t  \big(g_{2-2\alpha} * \|\psi\|_{L^2}\big)^2(s) \ds\leq \|g_{2-2\alpha}\|_{L^1_t}^2 \|\psi\|_{L^2_tL^2}^2.
%\end{aligned}$$
%%But we need to convolve instead 
%$$\begin{aligned} \ga * \|g_{2-2\alpha}*\psi\|_{L^2}^2 \leq \\end{aligned}$$










