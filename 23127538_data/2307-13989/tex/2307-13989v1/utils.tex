% Packages
\usepackage{csquotes}
\usepackage{hyperref}
\usepackage{tikz}
\usepackage{collcell}
\usepackage{booktabs}
\usepackage{tabularray}
\usepackage{tabularx}
\usepackage{makecell}
\usepackage{longtable}
\usepackage{multirow}
\usepackage[table]{xcolor}
\usepackage{etoolbox}
\usepackage{calc}
\usepackage{pgf} % for calculating the values for gradient
%======================================
% Color set related!
\definecolor{lightblue}{HTML}{e8f6fc}
\definecolor{lightgreen}{HTML}{b6cfba}
\definecolor{darkgreen}{HTML}{228833}
\definecolor{contrast}{HTML}{AA3377}
\definecolor{orange}{HTML}{FF5733}
\definecolor{lightcontrast}{HTML}{e6c8d9}
\definecolor{light}{HTML}{f5f5f5}
\definecolor{white}{HTML}{ffffff}
\definecolor{high}{HTML}{683AEC}  % the color for the highest number in your data set
\definecolor{low}{HTML}{F2F2F2}  % the color for the lowest number in your data set
\newcommand*{\opacity}{90}% here you can change the opacity of the background color!
%======================================
% Data set related!
\newcommand*{\minval}{0.0}% define the minimum value on your data set
\newcommand*{\maxval}{1.0}% define the maximum value in your data set!
%======================================
% gradient function!
\newcommand{\gradient}[1]{
    % The values are calculated linearly between \minval and \maxval
    \ifdim #1pt > \maxval pt
        \cellcolor{high!\opacity} 
    \else
        \ifdim #1pt < \minval pt
            \cellcolor{low!\opacity} 
        \else
            \pgfmathparse{int(round(100*(#1/(\maxval-\minval))-(\minval*(100/(\maxval-\minval)))))}
            \xdef\tempa{\pgfmathresult}
            \cellcolor{high!\tempa!low!\opacity} 
        \fi
    \fi
    \ifdim #1pt > 0.75 pt
        \leavevmode\color{white} #1
    \else
        #1
    \fi
}
\newcommand*{\minvalrelative}{0.0}% define the minimum value on your data set
\newcommand*{\maxvalrelative}{1.0}% define the maximum value in your data set!
\newcommand{\gradientrelative}[2]{ %2 is reference value
    % The values are calculated linearly between \minval and \maxval
    \pgfmathsetmacro{\diff}{#2-#1}
    \ifdim \diff pt > 0 pt
        %\cellcolor{high!\opacity} 
        \pgfmathparse{int(round(100*(\diff/(\maxvalrelative-\minvalrelative))-(\minvalrelative*(100/(\maxvalrelative-\minvalrelative)))))}
        \xdef\tempa{\pgfmathresult}
        \cellcolor{orange!\tempa!white!\opacity} 
    \else
        \pgfmathparse{int(round(100*(-\diff/(\maxval-\minval))-(\minval*(100/(\maxval-\minval)))))}
        \xdef\tempa{\pgfmathresult}
        \cellcolor{darkgreen!\tempa!white!\opacity} 
    \fi
    \ifdim \diff pt > 0.75 pt
        \leavevmode\color{white} #1
    \else
        #1
    \fi
}
%======================================
% gradient function single cell! 
\newcommand{\gradientcell}[6]{
    % The values are calculated linearly between \minval and \maxval
    \ifdimcomp{#1pt}{>}{#3 pt}{#1}{
        \ifdimcomp{#1pt}{<}{#2 pt}{#1}{
            \pgfmathparse{int(round(100*(#1/(#3-#2))-(\minval*(100/(#3-#2)))))}
            \xdef\tempa{\pgfmathresult}
            \cellcolor{#5!\tempa!#4!#6} #1
    }}
}

\usepackage{tikz}
\usetikzlibrary{shapes.geometric, arrows, positioning, calc}

\newcommand{\ruleNegationTikz}[0]{
    \begin{tikzpicture}[align=center,node distance=2cm] 
    \tikzstyle{myellipse} = [ellipse, draw=black, fill=lightblue]
    \tikzstyle{process} = [rectangle, draw=black, fill=lightblue, align=center]
    \tikzstyle{decision} = [diamond, draw=black, fill=lightblue, aspect=3, align=center, inner sep=1pt]
    \node(inp) [myellipse] {\textbf{INPUT SENTENCE}};
    \node(init)[process] at (0, -1.2) {Run POS tagging and dependency parsing};
    \node(neg_found)[decision, below of = init,node distance=1.5cm] {Any token tagged\\ as \enquote{neg}?};
    
    \node(aux_do)[decision] at (-4,-4.5) {Negated token\\ is AUX \enquote{do}?};
    \node(del_do)[process, below of=aux_do, left of=aux_do] {Remove negated \\\enquote{do}};
    \node(del_part)[process, below of=aux_do, right of=aux_do] {Remove\\ negation particle\\ from the AUX};
    \node(conjugate)[process, below right of =del_do,node distance=2.25cm] {Conjugate verb in same tense as input};
    
    \node(root)[process] at (4.55,-3.75) {Get ROOT};
    \node(root_pos)[decision, below of =root,node distance=1.25cm] {ROOT POS tag?};
    \node(add_do)[process]  at (1.5,-6.75) {Add negated \enquote{do},\\ conjugated in the\\same tense\\ as the ROOT};
    \node(root_inf)[process, below of = add_do] {Change ROOT\\ to infinitive form};
    \node(root_child)[decision, inner sep=0pt] at (6.25,-6.75) {ROOT has another\\AUX child?};
    \node(neg_child)[process, below left of =root_child,node distance=2.25cm] {Negate first\\AUX child};
    \node(neg_root)[process, below right of =root_child,node distance=2.25cm] {Negate\\ ROOT AUX};
    
    \node(out)[myellipse] at (1.5, -10.25) {\textbf{NEGATED SENTENCE}};
    
    
    \draw[->] (inp) -- (init);
    \draw[->] (init) -- (neg_found);
    \draw[->]  (neg_found) -- node[anchor=east] {yes} (aux_do);
    
    \draw[->]  (aux_do) -- node[anchor=east] {yes} (del_do);
    \draw[->]  (aux_do) -- node[anchor=west] {no} (del_part);
    \draw[->] (del_do) --node[anchor=east] {\color{darkblue}(1)\hspace{4pt}} (conjugate);
    \draw[->] (del_part) --node[anchor=west] {\color{darkblue}\hspace{5pt}(2)} (conjugate);
    
    \draw[->] (neg_found) -- node[anchor= south west] {no} (root);
    \draw[->] (root) -- (root_pos);
    \draw[->] (root_pos) -- node[anchor=south east] {VERB} (add_do);
    \draw[->] (add_do) -- (root_inf);
    \draw[->] (root_pos) -- node[anchor=south west] {AUX} (root_child);
    \draw[->] (root_child) -- node[anchor=east] {yes} (neg_child);
    \draw[->] (root_child) -- node[anchor=west] {no} (neg_root);
    
    \draw[->] (conjugate) -- (out);
    \draw[->] (root_inf) --node[anchor=east] {\color{darkblue}(3)} (out);
    \draw[->] (neg_child) --node[anchor= west] {\hspace{4pt}\color{darkblue}(4)} (out);
    \draw[->] (neg_root) --node[anchor= north] {\color{darkblue}(5)} (out);
    
\end{tikzpicture}
}

\newcommand\urlsize{\fontsize{8.5pt}{10.2pt}\selectfont}
