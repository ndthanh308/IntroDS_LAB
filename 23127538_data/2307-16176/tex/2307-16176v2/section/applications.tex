\section{Applications}
Our InvVis is competent in various scenarios, such as embedding URL or copyright information in a visualization, 
which has been discussed in \cite{zhang2020viscode}. Compared to previous methods, our approach has the 
advantage of being applicable to scenarios that involve large amounts of data. For instance,
it can facilitate invertible visualization of data-intensive charts, large-scale source code embedding and 
scientific data embedding.


% Figure environment removed


\subsection{Invertible Visualization of Data-Intensive Chart}
Data-intensive charts are common in the field of information visualization. Due to the large data volume they
contain, such charts generally require some interactive functionalities to assist users in comprehending the 
data. However, there is a lack of a convenient way to disseminate this kind of visualization while preserving their 
interactivity. InvVis enables the reconstruction of interactive charts from a single image, greatly simplifying
their dissemination. \autoref{fig:ui} shows the application interface of our InvVis. In the data embedding process, 
the user can upload a chart image and its metadata, including the chart data and chart information, the application 
can embed the uploaded data into the chart image and output an encoded image. In the data restoring procedure, 
the application can decode an encoded image and reconstruct the interactive chart.

InvVis can also facilitate the reusing of chart data. Since it allows to hide all the original data into the 
chart image, users can have full access to all the metadata of the chart after decoding and restoring the data. 
The recovered data contains not only the chart information but also the underlying data, which means that users may 
further reuse the data. As shown in \autoref{fig:reuse}, apart from reconstructing the chart, users may redesign the
chart from various perspectives with the decoded metadata, such as changing the visual 
mapping of the chart. Also, the chart data can be reorganized and mapped to new visualizations. For example, users 
can compute the statistics of the chart data and map it to a doughnut chart.

% Figure environment removed


\subsection{Large-Scale Source Code Embedding}
Source code is an important part of visualizations. Some visualization charts may contain a large amount of source 
code due to their sophisticated interactive methods or large data volume. VisCode \cite{zhang2020viscode} proposed to hide the 
source code in a chart image, but due to the limited steganography capacity of this method, it cannot handle the cases in 
which the amount of source code or data is quite large. As will be discussed in \ref{eva: cap}, InvVis has a much larger 
steganography capacity than VisCode, even when only hiding QR Codes, 
which can be indicated by \autoref{tab:stegCap1}.
Therefore, our method can better handle this problem.

To better illustrate the practicality of large-scale source code embedding, we present an application scenario, 
which is the visualization dashboard. Visualization dashboard is a collection of various visualization charts 
that are displayed together on a single screen or interface. It can be used in business intelligence or data 
analysis to provide a quick overview of key metrics and statistical data. Since a visualization dashboard generally
contains several visualization charts, the amount of its metadata may be much larger than that of a single chart.
Additionally, if the chart data is also written directly in the source code, it will make the source code very long.
For methods like VisCode that encode all the source code into QR Codes, the number of QR Codes can be too large to be hidden in the 
carrier image. Meanwhile, if a relatively small image is used to display a visualization dashboard, the charts and data 
contained in it may be too small to be clearly seen. Therefore, if the source code, together with the chart data of a visualization 
dashboard can be effectively hidden in an image, it can greatly facilitate the dissemination of this kind of visualization.

InvVis can handle this problem in a novel way. Since InvVis can hide both data images and QR Code image simultaneously, we can separate 
the data with large volume from the source code and transfer them to data images. This effectively reduces the length of the source code, 
and meanwhile increases the efficiency of data embedding. Our method allows both the data and source code to be carried by a relatively 
smaller carrier image. \autoref{fig: dashboard} demonstrates an example, the source data of the heat map is separated from the 
source code and transferred into data images. The data images and the QR Code image that contains the source code are embedded into 
a relatively small image. By decoding this image, users can obtain a complete and interactive visualization dashboard.

% Figure environment removed


\subsection{Scientific Data Embedding}
Scientific data typically has large data volume, such as terrain data, ocean current data, volumetric data, etc.
These kinds of data generally require visualization methods to present the data and design some interactive ways to 
facilitate researchers or common users in understanding the data intuitively. Since our InvVis can effectively embed 
data into carrier images through data images, it has great potential for the embedding of scientific data. This 
allows users to spread scientific visualizations through a single image.

\autoref{fig:sci}(a)(b)(c) shows an example of scientific data embedding. We embed a set of wind field
data which contains a total of 35180 2-dimensional vectors into an image. Although the data volume is quite large, the
size of the carrier image is as small as 520 $\times$ 360. \autoref{fig:sci}(b)(c) shows the reconstructed visualization, which 
is interactive, allowing users to query data within a certain range or zoom in to view the distribution of data in 
a specific area. \autoref{fig:sci}(d)(e)(f) demonstrates another example, we embed the volumetric data of an engine
into a carrier image. The amount of this volumetric data is very large, reaching 256 $\times$ 256 $\times$ 128, while 
in contrast, the carrier image is only 2304 $\times$ 1280. Our method can effectively hide such a large amount of data 
into the carrier image and results \autoref{fig:sci}(d), whose distortion is almost unobservable. And the restored 
data is also very accurate, as shown in \autoref{fig:sci}(e)(f), the reconstructed visualization is indistinguishable 
from the original one and allows users to view the volume data visualization from different directions.

% Figure environment removed
    