\documentclass[journal]{vgtc}                     % final (journal style)
%\documentclass[journal,hideappendix]{vgtc}        % final (journal style) without appendices
% \documentclass[review,journal]{vgtc}              % review (journal style)
%\documentclass[review,journal,hideappendix]{vgtc} % review (journal style)
%\documentclass[widereview]{vgtc}                  % wide-spaced review
%\documentclass[preprint,journal]{vgtc}            % preprint (journal style)

%% Uncomment one of the lines above depending on where your paper is
%% in the conference process. ``review'' and ``widereview'' are for review
%% submission, ``preprint'' is for pre-publication in an open access repository,
%% and the final version doesn't use a specific qualifier.

%% If you are submitting a paper to a conference for review with a double
%% blind reviewing process, please use one of the ``review'' options and replace the value ``0'' below with your
%% OnlineID. Otherwise, you may safely leave it at ``0''.
\onlineid{0}

%% In preprint mode you may define your own headline. If not, the default IEEE copyright message will appear in preprint mode.
%\preprinttext{To appear in IEEE Transactions on Visualization and Computer Graphics.}

%% In preprint mode, this adds a link to the version of the paper on IEEEXplore
%% Uncomment this line when you produce a preprint version of the article 
%% after the article receives a DOI for the paper from IEEE
%\ieeedoi{xx.xxxx/TVCG.201x.xxxxxxx}

%% declare the category of your paper, only shown in review mode
\vgtccategory{Research}

%% please declare the paper type of your paper to help reviewers, only shown in review mode
%% choices:
%% * algorithm/technique
%% * application/design study
%% * evaluation
%% * system
%% * theory/model
% \vgtcpapertype{please specify}

%% Paper title.
\title{InvVis: Large-Scale Data Embedding for Invertible Visualization}

%% Author ORCID IDs should be specified using \authororcid like below inside
%% of the \author command. ORCID IDs can be registered at https://orcid.org/.
%% Include only the 16-digit dashed ID.
\author{
  Huayuan Ye, Chenhui Li, Yang Li and Changbo Wang
}
\authorfooter{
%% insert punctuation at end of each item
  \item
  Huayuan Ye, Chenhui Li, Yang Li, and Changbo Wang are with School of Computer Science and Technology, 
  East China Normal University. Chenhui Li is the corresponding author. E-mail: chli@cs.ecnu.edu.cn.
}


%% Abstract section.
\abstract{
  We present InvVis, a new approach for invertible visualization, which is reconstructing or further modifying a visualization from an image. InvVis 
  allows the embedding of a significant amount of data, such as chart data, chart information, source code, etc., into visualization images. 
  The encoded image is perceptually indistinguishable from the original one. We propose a new method to efficiently express chart data in the form of 
  images, enabling large-capacity data embedding. We also outline a model based on the invertible neural network to achieve high-quality data concealing 
  and revealing. We explore and implement a variety of application scenarios of InvVis. Additionally, we conduct a series of evaluation experiments to 
  assess our method from multiple perspectives, including data embedding quality, data restoration accuracy, data encoding capacity, etc. The result of our 
  experiments demonstrates the great potential of InvVis in invertible visualization.
}

%% Keywords that describe your work. Will show as 'Index Terms' in journal
%% please capitalize first letter and insert punctuation after last keyword
\keywords{
  Information visualization, information steganography, invertible visualization, invertible neural network.
}

%% A teaser figure can be included as follows

%% Uncomment below to disable the manuscript note
%\renewcommand{\manuscriptnotetxt}{}

%% Copyright space is enabled by default as required by guidelines.
%% It is disabled by the 'review' option or via the following command:
%\nocopyrightspace


%%%%%%%%%%%%%%%%%%%%%%%%%%%%%%%%%%%%%%%%%%%%%%%%%%%%%%%%%%%%%%%%
%%%%%%%%%%%%%%%%%%%%%% LOAD PACKAGES %%%%%%%%%%%%%%%%%%%%%%%%%%%
%%%%%%%%%%%%%%%%%%%%%%%%%%%%%%%%%%%%%%%%%%%%%%%%%%%%%%%%%%%%%%%%

%% Tell graphicx where to find files for figures when calling \includegraphics.
%% Note that due to the \DeclareGraphicsExtensions{} call it is no longer necessary
%% to provide the the path and extension of a graphics file:
%% % Figure removed is completely sufficient.
\graphicspath{{figs/}{./}} % where to search for the images

%% Only used in the template examples. You can remove these lines.
\usepackage{tabu}                      % only used for the table example
\usepackage{booktabs}                  % only used for the table example
\usepackage{lipsum}                    % used to generate placeholder text
\usepackage{mwe}                       % used to generate placeholder figures
\usepackage{amsmath}
\usepackage{algorithm}
\usepackage{algpseudocode}

%% We encourage the use of mathptmx for consistent usage of times font
%% throughout the proceedings. However, if you encounter conflicts
%% with other math-related packages, you may want to disable it.
\usepackage{mathptmx}                  % use matching math font
\usepackage{fancybox}
\usepackage{adjustbox}
\usepackage{marginnote}
\usepackage{fancybox}
\usepackage{multirow}
\usepackage{array}
\usepackage[left=1.2cm,top=1.7cm,bottom=1.6cm,right=1.2cm]{geometry}

\begin{document}
\newcommand{\revision}[1]{{\hypersetup{allcolors=red}\textcolor{red}{#1}}}
\newcommand{\sidecomment}[1]{\marginnote{\adjustbox{minipage=0.43\marginparwidth,fbox}{\textcolor{red}{#1}}}}

\teaser{
  \centering
  % Figure removed
  \caption{%
    Since InvVis can embed a large amount of data into images, users can decode the embedded data and perform 
    rich exploration, such as redesigning (a) or reconstructing (b) visualizations, rebuilding a visualization dashboard 
    based on the decoded source code and chart data (c), or visualizing volume data based on the decoded raw data 
    and rendering parameters (d).
  }
  \label{fig:teaser}
}


%%%%%%%%%%%%%%%%%%%%%%%%%%%%%%%%%%%%%%%%%%%%%%%%%%%%%%%%%%%%%%%%
%%%%%%%%%%%%%%%%%%%%%% START OF THE PAPER %%%%%%%%%%%%%%%%%%%%%%
%%%%%%%%%%%%%%%%%%%%%%%%%%%%%%%%%%%%%%%%%%%%%%%%%%%%%%%%%%%%%%%%

%% The ``\maketitle'' command must be the first command after the
%% ``\begin{document}'' command. It prepares and prints the title block.
%% the only exception to this rule is the \firstsection command
\firstsection{Introduction}
\maketitle


\section{Introduction}
Current quantum hardware is unable to carry out universal quantum computations due to the buildup of errors that occur during the computation. 
The magnitude of the individual error is currently above the value that the Threshold Theorem requires in order to kick-start quantum error correction and fault-tolerant quantum computation~\cite[Section 10.6]{nielsen_chuang_2010}. 
Although the experimentally achieved fidelity rates are promising and the error bounds are inching closer to the required threshold, we will have to work for the foreseeable future with quantum hardware with errors that build-up during the computation.  This implies that we can only do a limited number of steps before the output of the computation has become completely uncorrelated with the intended one.

For fault-tolerant quantum computing, we repeat four steps: 
1) We apply a number of single and two-qubit quantum gates, in parallel whenever possible; 
2) We perform a syndrome measurement on a subset of the qubits; 
3) We perform fast classical computations to determine which errors have occurred and how to correct them; 
and, 4) We apply correction terms based on the classical computations.
We then repeat these four steps with a next sequence of gates. 
These four steps are essential to fault-tolerant quantum computing. 


The starting point of this work is to use the four steps outlined above, not to carry out error correction and fault-tolerant computation, but to enhance short, constant-depth, {\em uncorrected} quantum circuits that perform single qubit gates and {\em nearest-neighbor} two qubit gates. 
Since in the long run we will have to implement error-correction and fault-tolerant computation anyhow, and this is done by such a four-step process, why not make other use of this architecture? Moreover, on some of the quantum hardware platforms, these operations are already in place.
Embracing this idea we naturally arrive at the question: what is the computational power of \textit{low-depth} quantum-classical circuits organized as in the four steps outlined above? 
We thus investigate circuits that execute a small, ideally constant, number of stages, where at each stage we may apply, in parallel, single qubit gates and {\em nearest-neighbor} two qubit gates, followed by measurements, followed by low-depth classical computations of which the outcome can control quantum gates in later stages. 
It is not clear, at first, whether such circuits, especially with constant depth, can do anything remotely useful. 
But we will see that this is indeed the case: many quantum computations can be done by such circuits in constant depth. 
By parallelizing quantum computations in this way, we improve the overall computational capabilities of these circuits, as we do not incur errors on qubits that are idle, simply because qubits are not idle for a very long time. 
Furthermore, reducing the depth of quantum circuits, at the cost of increasing width, allows the circuit to be run faster even if errors occur.

The first usage of such a four-step layout, not to do error correction, but to perform computations, can be found in the paradigm of measurement-based quantum computing~\cite{gottesman1999demonstrating,raussendorf2001one,jozsa2006introduction,clark2007generalised}: 
A universal form of quantum computing where a quantum state is prepared and operations are performed by measuring qubits in different bases, depending on previous measurements and intermediate measurements.

\citeauthor{PhamSvore2013} were the first to formalize the four-step protocol for performing computations~\cite{PhamSvore2013}. They included specific hardware topologies by considering two-dimensional graphs for imposing constraints on qubit interactions. In their model, they develop circuits for particularly useful multi-qubit gates, including specifying costs in the width, number of qubits, depth, number of concurrent time steps, size, and total number of non-Identity operations.
As a result, they find an algorithm that factors integers in polylogarithmic depth.
\citeauthor{Browne:2011} showed that the main tool in the work by \citeauthor{PhamSvore2013}, the fan-out gate, can also be replaced by additional log-depth classical computations in the measurement-based quantum computing setting~\cite{Browne:2011}.

More recently, \citeauthor{Cirac:2021} introduced a scheme to implement unitary operations involving quantum circuits combined with Local Operations and Classical Communication ($\mathsf{LOCC}$) channels: $\mathsf{LOCC}$-assisted quantum circuits~\cite{Cirac:2021}. Similarly to the four-step scheme we just described, they allow for a short depth circuit to be run on the qubits, followed by one round of $\mathsf{LOCC}$, in which ancilla qubits are measured and local unitaries are applied based on the measurement outcomes. They show that in this model any 1D transitionally invariant matrix-product state (MPS) with fixed bond dimension is in the same phase of matter as the trivial state. Similar ideas can be found in~\cite{TVV_NonAbelianTopologicalOrder_2022, tantivasadakarn2021long}.

In this work, we introduce a new model, called \textit{Local Alternating Quantum-Classical Computations} ($\LAQCC$). In this model we alternate between running quantum circuits (constrained by locality), ending in the measurement of a subset of qubits, and fast classical computations based on the measurement results. The outcome of the classical computations are then used to control future quantum circuits. We allow for flexibility in this model, by giving different constraints to the power of both the quantum circuits and the classical circuits as well as the number of alternations between them. 
Most attention will be given to $\LAQCC$ containing quantum circuits of constant depth, classical circuits of logarithmic depth and at most a constant number of alternations between them. 
Any circuit constructed in this model is considered to be of constant depth. 
We restrict ourselves to logarithmic depth classical computations, as this is the first natural and non-trivial extension beyond constant-depth classical computations. 
Constant-depth classical computations do however also have an equivalent constant-depth quantum implementation.

The definition of $\LAQCC$ sharpens the original definition of \citeauthor{PhamSvore2013} by adding constraints to the intermediate classical computations. This allows us to bound the power of $\LAQCC$ from above. 

The main result of \citeauthor{Cirac:2021}, that 1D translational invariant MPS with fixed bond dimension can be prepared by $\mathsf{LOCC}$-assisted circuits, relies on local symmetries of the MPS. These symmetries allow them to prepare local states (on a constant number of qubits) and glue them together by doing one round of the appropriate entangling measurement and corrections, after which they run a round of local unitaries to get the desired result. This general scheme for preparing states that exhibit an MPS description with the appropriate local symmetries requires only geometrically local unitaries and one round of measurement and corrections an therefore is accessible in $\LAQCC$. Studying different local symmetries, known as Symmetry Protected Topological (SPT) phases of matter, to find measurement-based constant depth circuits for states is a broad ongoing field of research~\cite{TVV_NonAbelianTopologicalOrder_2022, tantivasadakarn2021long, smith2023deterministic}. 
All these schemes have a $\LAQCC$ implementation.

%$\LAQCC$-circuits also exist for general schemes of preparing local states, based on the local tensors, and gluing them together using one round of entangled measurement and corrections, based on the local symmetry. 
%The main result of \citeauthor{Cirac:2021}, that 1D translational invariant MPS with fixed bond dimension can be prepared by $\mathsf{LOCC}$-assisted circuits, relies heavily on local symmetries of the MPS and as a result also has an equivalent $\LAQCC$ implementation. 
%The corrections applied after the measurement round are local unitaries depending on the local symmetries of the MPS. 

 

%This general scheme of preparing local states, based on the local tensors, and gluing it together by doing one round of entangled measurement and corrections, based on the local symmetry, is accessible in $\LAQCC$.
Note however that \citeauthor{Cirac:2021} also suggest a circuit for the $W$-state.
This circuit uses sequentially and dependent measurement-based corrections of the ancilla qubits. 
These dependent measurements translate to sequential alternations between the quantum and classical circuits and therefore increase the total depth to linear depth, exceeding the constant-depth constraints imposed by $\LAQCC$-circuits. 

We study the power of the $\LAQCC$ model with respect to state preparation, showing that even with only constant quantum-depth and logarithmic classical depth it remains possible to prepare states with long-range entanglement.
Another surprising result is that it is unlikely that $\LAQCC$ circuits are classically simulatable. We show that any instantaneous quantum polynomial-time (IQP) circuit~\cite{Bremner2010,Shepherd2009} has an $\LAQCC$ implementation.
Classical simulation of IQP circuits implies the collapse of the polynomial hierarchy to the third level, which is not believed to be true~\cite{Bremner2017}. Therefore, we expect that $\LAQCC$ circuits are unlikely to be classically simulatable. We bound the power of $\LAQCC$ by showing that it is contained in $\QNC^1$, the class of polynomial-size, log-depth circuits.

Next, we also study the power that intermediate classical calculations can add to quantum computations, by considering a new model that alternates between polynomially many polynomial-depth quantum circuits and unbounded classical computations
We study this model by doing a complexity theoretical analysis, where we draw inspiration from the notions of complexity given by \citeauthor{RosenthalYuen:2022}, \citeauthor{MetgerYuen:2023}, and \citeauthor{Aaronson:2004}.
All three complexity notions are based on the notion of state preparation, instead of more traditional definition of complexity such as the decidability of a computational problem. 
The first two consider classes based on sequences of quantum states preparable by a polynomial-sized quantum circuit, where the circuits are uniformly generated by a computational class, for instance, the class $\mathsf{PSPACE}$, which results in the complexity class $\mathsf{StatePSPACE}$~\cite{RosenthalYuen:2022,MetgerYuen:2023}.
The third notion considers a relative complexity, where the complexity is measured between two given states, and is measured by the number of gates, from a given gate-set, required to transform one state in another state~\cite{Aaronson:2004}. 
For our definition of state preparation complexity, we drop the uniformity constraint from~\cite{RosenthalYuen:2022,MetgerYuen:2023} and define a class as $\mathsf{StateX}$, which refers to states preparable by circuits of type $\mathsf{X}$. 
As an example, if $\mathsf{X} = \QNC^0$, this results in the class $\mathsf{StateQNC^0}$, which is the set of states preparable from the $\ket{0}^n$ state by poly-size constant-depth circuits. 
This notion is similar to the relative complexity from~\cite{Aaronson:2004}, where one state is the  $\ket{0}^n$ state and instead of counting the number of gates we consider the set of states preparable by a fixed number of gates. Using this notion of complexity we show that any state preparable by an $\LAQCC^*$ circuit is also preparable by a $\mathsf{PostQPoly}$ circuit, the class of circuits of polynomial depth with an additional post-selection gate. 

All Clifford circuits have a constant-depth $\LAQCC$ implementation, implying that any stabilizer state can be implemented by a constant-depth $\LAQCC$ circuit, see Section~\ref{sec:clifford_circuits} for a proof of this statement. 
Efficient circuits for stabilizer states have been known already through measurement-based quantum computing. Therefore this paper focuses on the preparation of non-stabilizer states, and as a surprising result we find novel constant-depth protocols for four very natural classes of non-stabilizer states.
Despite the extensive research into these four classes of non-stabilizer states and the many applications of them, no efficient constant- or low-depth state preparation protocols are known yet. We specifically consider these four classes as they are all often used as initial states in other algorithms.

The first state is a uniform superposition over an arbitrary number of states. 
This state finds applications in many quantum algorithms, as they often start with a uniform superposition over multiple states. 
This superposition is often achieved by applying Hadamard gates to every qubit due to its simplicity to prepare. 
Yet, the analysis of many algorithms, such as Shor's algorithm~\cite{Shor:1997}, would benefit from a different initial superposition. 
The circuit to prepare the uniform superposition over an arbitrary number of states uses an exact version of Grover search as a subroutine, that turns a probabilistic circuit, with a known constant probability of success, into a deterministic circuit. 
We use the circuit for preparing a uniform superposition over an arbitrary number of states as a subroutine in the next two quantum state preparation protocols. 

The second state is the $W$-state, the uniform superposition over all computational basis states of Hamming-weight~$1$, a natural long-ranged entangled state that displays a fundamentally nonequivalent type of entanglement from the Greenberger–Horne–Zeilinger state~\cite{WState:2000}, for which $\LAQCC$-type constant-depth circuits were previously known~\cite{PhamSvore2013, Cirac:2021}. 
The $W$-state is often used as benchmark for new quantum hardware~\cite{Haffner2005,Neeley2010,GarciaPerez:2021}. 
A novel way to prepare the $W$-state therefore gives a new way to benchmark different quantum devices with each other. 
A circuit for preparing the $W$-state was given in~\cite{Cirac:2021}, but this implementation requires sequentially alternating measurements followed by local unitaries, which in the $\LAQCC$ model is not considered to be of constant depth. 
We improve this protocol by giving an $\LAQCC$ implementation of the $W$-state, based on a compress-uncompress method that links the one-hot and binary encoding of integers.

The third state considered is the Dicke state, a generalization of the $W$-state, a superposition over all computational basis states with Hamming-weight $k$~\cite{Dicke:1954}. 
Dicke states have relevance in various practical settings.
For instance, for quantum game theory~\cite{zdemir2007}, quantum storage~\cite{Bacon_Compress:2006,Plesch:2010}, quantum error correction~\cite{ouyang2014permutation}, quantum metrology~\cite{toth2012multipartite}, and quantum networking~\cite{prevedel2009experimental}. 
Dicke states have been used as a starting state for variational optimization algorithms, most notably Quantum Alternating Operator Ansatz (QAOA)~\cite{Hadfield2019}, to find solutions to problems such as Maximum k-vertex Cover~\cite{Brandhofer2022,cook2020quantum}.
The ground states of physical Hamiltonians describing one-dimensional chains tend to show a resemblance to Dicke states such as states resulting from the Bethe ansatz, making them an ideal starting state when investigating the ground state behavior of these Hamiltonians~\cite{TDL_BetheAnsatzDerivation:2010,B_ExcitedStateQuantumPhaseTransitions:2013,DickeTransitions:2021}. 
For instance, the algorithm by \citeauthor{van2021preparing}, who give an algorithm to prepare the Bethe ansatz eigenstates of the spin-1/2 XXZ spin chain, starts by first preparing a Dicke state~\cite{van2021preparing}. 
A Dicke-state preparation protocol based on the compress-uncompress methodology used in the $W$-state furthermore finds applications in entanglement distillation, where the entanglement of a large state is concentrated on only a few qubits. 
Efficient deterministic circuits for preparing Dicke states have been proposed by \citeauthor{bartschi2019deterministic}~\cite{bartschi2019deterministic, bartschi2022deterministic_short_depth}. 
They provide a quantum circuit of depth $\mathO(k \log(\frac{n}{k}))$, allowing arbitrary connectivity, to prepare a Dicke state, which they conjecture to be optimal when $k$ is constant. 
In this work, we provide a constant-depth $\LAQCC$ circuit below their conjectured bound already for constant $k$. 
However, this does not directly disprove their conjecture, as we allow for intermediate measurements and classical computations. 
More significantly, we even construct constant-depth $\LAQCC$ circuits for $k = \mathO(\sqrt{n})$ greatly improving their bound.
This construction extends the compress-uncompress method for the $W$-state combined with additional subroutines. 

We continue with a log-depth state preparation protocol for the Dicke-state for arbitrary $k$. 
This protocol implements an efficient transformation between the factoradic number representation and the combinatorial number representation of a positive integer. 
The combinatorial number representation relates directly to the Dicke state. 
The provided efficient transformation between number representation systems might be of independent interest. 

We conclude by modifying our protocol for preparing a Dicke-state to a protocol that prepares quantum many-body scar states in constant-depth. 
These states have low entanglement and longer coherence times than states with similar energy density.
These characteristics make many-body scar states interesting to analyze and relevant within physics.
Many-body scar states appear for instance in the AKLT model~\cite{AKLT:1987,MRBAR:2018,MRB:2018} and different spin models~\cite{SI:2019,MOBFR:2020}.
Known methods for preparing these states have polynomial-depth~\cite{Gustafson:2023}, whereas our circuit has constant depth. 

% We conclude by studying the power that intermediate classical calculations can add to quantum computations. 
% In this study, we define a new model that relaxes constant-depth quantum circuits to polynomial depth quantum circuits, log-depth classical calculations to unbounded classical computations and a constant number of alternations to a polynomial number of alternations. 
% We call this model $\LAQCC^*$. 
% We study this model by doing a complexity theoretical analysis, where we draw inspiration from the notions of complexity given by \citeauthor{RosenthalYuen:2022}, \citeauthor{MetgerYuen:2023}, and \citeauthor{Aaronson:2004}.
% All three complexity notions are based on the notion of state preparation, instead of more traditional definition of complexity such as the decidability of a computational problem. 
% The first two consider classes based on sequences of quantum states preparable by a polynomial-sized quantum circuit, where the circuits are uniformly generated by a computational class, for instance, the class $\mathsf{PSPACE}$, which results in the complexity class $\mathsf{StatePSPACE}$~\cite{RosenthalYuen:2022,MetgerYuen:2023}.
% The third notion considers a relative complexity, where the complexity is measured between two given states, and is measured by the number of gates, from a given gate-set, required to transform one state in another state~\cite{Aaronson:2004}. 
% For our definition of state preparation complexity, we drop the uniformity constraint from~\cite{RosenthalYuen:2022,MetgerYuen:2023} and define a class as $\mathsf{StateX}$, which refers to states preparable by circuits of type $\mathsf{X}$. 
% As an example, if $\mathsf{X} = \QNC^0$, this results in the class $\mathsf{StateQNC^0}$, which is the set of states preparable from the $\ket{0}^n$ state by poly-size constant-depth circuits. 
% This notion is similar to the relative complexity from~\cite{Aaronson:2004}, where one state is the  $\ket{0}^n$ state and instead of counting the number of gates we consider the set of states preparable by a fixed number of gates. Using this notion of complexity we show that any state preparable by an $\LAQCC^*$ circuit is also preparable by a $\mathsf{PostQPoly}$ circuit, the class of circuits of polynomial depth with an additional post-selection gate. 

\paragraph{Summary of results}
\begin{itemize}
    \item We give a new definition of a computational model that captures the power of the four step process: applying a constant number of layers of one- and two-qubit gates; performing a syndrome measurement; perform a fast classical computation determining corrections; apply corrections. We call this model \emph{Local Alternating Quantum Classical Computations}, or $\LAQCC$ for short. In this model we bound the allowed quantum operations, intermediate classical calculations, and number of rounds separately. In Section~\ref{sec:LAQCC_model} we define this model and give a list of operations based on results from literature contained in this computational model. In some of these operations we explicitly use that we allow for multiple, but at most constant, rounds  of corrections.
    \item  We show show that there exist $\LAQCC$ circuits that can not be weakly simulated in Section~\ref{sec:IQP_in_LAQCC}. We further show that for every $\LAQCC$ circuit there exists a $\QNC^1$ circuit simulating it perfectly, in Section~\ref{sec:LAQCC_in_QNC1}.
    \item We introduce a new type computational complexity for preparing states and show that the extension of $\LAQCC$ where we allow a polynomial number of rounds and unbounded classical computation, is contained in $\mathsf{PostQPoly}$, the class of polynomial circuits with post-selection, in Section~\ref{sec:Complexity results}.
    \item We show a protocol to prepare the uniform superposition state of size $q$ in $\LAQCC$ using $\mathO(\ceil{\log_2(q)}^2)$ qubits in Section~\ref{sec:superposition_modulo_q}. 
    \item We show a protocol to prepare the $W_n$ state in $\LAQCC$ using $\mathO(n\log(n))$ qubits in Section~\ref{sec:W_state_in_LAQCC}.
    \item We show two ways of preparing the Dicke-$(n,k)$ state. The first method is in $\LAQCC$, works up to $k = \mathO(\sqrt{n})$, uses $\mathO(n^2\log(n))$ qubits, and is found in Section~\ref{sec:dicke:small_k}. The second method is in $\LAQCC\text{-}\mathsf{LOG}$ (an extension of $\LAQCC$ allowing for logarithmic number of alterations instead of constant), works for any $k$, uses $\mathO(\text{poly}(n))$ qubits, and is found in Section~\ref{sec:Dicke_in_LAQCC_LOG}. 
    \item We extend on our $\LAQCC$ method of generating Dicke-$(n,k)$ states for $k = \mathO(\sqrt{n})$ and show a protocol to generate many-body scar states for a particular Hamiltonian in $\LAQCC$ (Section~\ref{sec:many_body_scar}). 
\end{itemize}
Summarized in a table, we provide the following state generation protocols:
\begin{table}[htb]
\centering
\begin{tabular}{l|l|l|l}
\textbf{State description} & \textbf{Width} & \textbf{Depth} & \textbf{Implementation}\\
\hline 
Uniform superposition mod $q$: $\frac{1}{\sqrt{q}} \sum_{i = 0}^{q-1}\ket{i}$ & $\mathO(\ceil{\log^2 q})$ & $\mathO(1)$ & Section~\ref{sec:superposition_modulo_q}\\

$W$-state: $\frac{1}{\sqrt{n}}\sum_{i = 0}^{n-1}\ket{e_i}$ & $\mathO(n \log n)$ & $\mathO(1)$ & Section~\ref{sec:W_state_in_LAQCC}\\

Dicke-$(n,k)$, $k = \mathO(\sqrt{n})$: $\binom{n}{k}^{-1/2}\sum_{x \in \{0,1\}^n: |x| = k} \ket{x}$ &  $\mathO(n^2\log n)$ & $\mathO(1)$ 
&Section~\ref{sec:dicke:small_k}\\

Dicke-$(n,k)$: $\binom{n}{k}^{-1/2}\sum_{x \in \{0,1\}^n: |x| = k} \ket{x}$ & $\mathO(\text{poly}(n))$ & $\mathO(\log n)$ &Section~\ref{sec:Dicke_in_LAQCC_LOG}\\

QMBS: $\ket{S_k} = \frac{1}{k! \sqrt{\mathcal N(n,k)}}(Q^\dagger)^k \ket{\Omega}$ &  $\mathO(n^2\log n)$ & $\mathO(1)$  &  Section~\ref{sec:many_body_scar}
\end{tabular}
\caption{Summary of state preparation protocols given in this paper.}
\label{tab:sate_prep}
\end{table}
In the entry for the quantum many-body scar state $Q$ denotes the raising operator and $\mathcal N(n,k)=\binom{n-k-1}{k}$. 
Section~\ref{sec:many_body_scar} will provide more details on the variables and the implementation. 

\paragraph{Organization of the paper}
\noindent We first introduce relevant preliminaries in Section~\ref{sec:preliminaries}. 
In Section~\ref{sec:LAQCC_model} we formally define the class of Local Alternating Quantum-Classical Computations ($\LAQCC$). We also show that any Clifford circuit can be implemented in constant depth $\LAQCC$ (a result based on a result from measurement-based quantum computing~\cite{jozsa2006introduction}). 
This result allows us to give many useful multi-qubit gates and routines in Section~\ref{sec:gates_created_in_LAQCC}. 
Beyond that we show that constant depth $\LAQCC$ circuits are contained in $\QNC^1$ and that any $\mathsf{IQP}$ circuit has an $\LAQCC$ implementation.
We conclude this section with an analysis of a more powerful instantiation of $\LAQCC$ and show an inclusion with respect to the class $\mathsf{PostQPoly}$, which is the class of circuits of polynomial depth with one additional post-selection gate. 
In Section~\ref{sec:state_prep_in_LAQCC} we give $\LAQCC$ circuit implementations for preparing the uniform superposition over an arbitrary number of states, the $W$-state and the Dicke state up to $k = \mathO(\sqrt{n})$. We furthermore give a log-depth circuit implementation for preparing the Dicke state for any $k$. We conclude by showing a $\LAQCC$ circuit for generating many body scar states of a particular type of Hamiltonian.


\paragraph{Unlearning.}

The naive approach to machine unlearning is to retrain a model from scratch with each data deletion request. However, retraining is not feasible for companies with many large models or organizations with limited resources. Thus, the primary objective of machine unlearning is to provide efficient approximations to retraining. Early approaches in security and privacy attempt to achieve exact removal, where an unlearned model is identical to retraining, but are limited in model class~\citep{caoMakingSystemsForget2015, ginartMakingAIForget2019}. \citet{bourtouleMachineUnlearning2020} propose SISA, a flexible approach to exact unlearning that ``shards" a dataset, dividing it and training an ensemble of models where each can be retrained separately. More recent approaches propose approximate removal, requiring the unlearned model to be ``close" to the output of retraining. Some approximate removal methods focus on improving efficiency~\citep{wuDeltaGradRapidRetraining2020} and others try to preserve performance~\citep{wuPUMAPerformanceUnchanged2022}. While these methods apply to a large class of models, they have no formal guarantees on data removal. A second group of approximate approaches provide theoretical guarantees on the statistical indistinguishability of unlearned and retrained models. These noise-based methods leverage convex loss functions to guarantee unlearning with gradient updates~\citep{neelDescenttoDeleteGradientBasedMethods2020a} and Hessian methods~\citep{guoCertifiedDataRemoval2020, sekhariRememberWhatYou, izzoApproximateDataDeletion}. We augment this second set of approximate methods to simultaneously provide strong guarantees on data protection and preserve fairness performance while targeting a common class of models.

\paragraph{Fairness.}

There are a multitude of definitions for fairness in machine learning, such as individual fairness, multicalibration or multiaccuracy, and group fairness. Individual fairness~\citep{dwork2012fairness} posits that ``similar individuals should be treated similarly" by a model. 
On the other hand, recent work has focused on multicalibration and multiaccuracy~\citep{hebert2018multicalibration, kearns2018preventing, deng2023happymap}, where predictions are required to be calibrated across subpopulations. These subpopulation definitions can be highly expressive, containing many intersectional identities from protected groups. In this work, however, we focus on the most commonly studied form of fairness, group fairness, which seeks to balance certain statistical metrics across predefined subgroups. Group fairness literature has proposed various definitions of fairness, but the three most common definitions are Demographic Parity~\citep{zafar2017fairness, feldman2015certifying, zliobaite2015relation, calders2009building}, Equalized Odds, and Equality of Opportunity~\citep{hardt2016equality}. To achieve these definitions, there are generally three approaches to achieving group fairness: \emph{preprocessing} which attempts to correct dataset imbalance to ensure fairness~\citep{calmon2017optimized}, \emph{in-processing} which occurs during training by modifying traditional empirical risk minimization objectives to include fairness constraints~\citep{lowyStochasticOptimizationFramework2022, berk2017convex, agarwal2018reductions, martinez2020minimax}, and \emph{postprocessing} which modifies predictions to ensure fair treatment~\citep{alghamdi2022beyond, hardt2016equality}. 
In this work we focus on in-processing algorithms because they simply modify an objective to account for fairness rather than requiring an additional operation before or after each unlearning request which would also have to be made unlearnable.

\paragraph{Intersections.} Despite advancements in machine unlearning, the literature still lacks sufficient consideration of the downstream impacts of unlearning methods. While recent papers have explored the compatibility of the right to be forgotten with the right to explanation~\citep{krishna2023towards}, there is little work at the intersection of unlearning and fairness. In privacy literature, a thread of work has shown the incompatibility of group fairness with privacy~\citep{esipova2022disparate, bagdasaryan2019differential, cummings2019compatibility} but these incompatibilities arise due to privacy-specific methods, such as gradient clipping and differences in neighboring datasets. Fairness literature has studied the related problem of the influence of training data on fairness~\citep{wang2022understanding}, but does not provide any methods for unlearning. In unlearning literature, recent empirical studies have shown that unlearning can increase disparity~\citep{zhang2023forgotten}, other works have demonstrated the incompatibility of fairness and unlearning for the SISA algorithm \citep{kochno}, and one work~\citep{wang2023inductive} has provided a method to achieve removal and fairness but uses a sharding and retraining algorithm over fairness-corrected graph data for GNNs. In this paper, we propose the first efficient method which achieves fairness while being provably unlearnable without requiring retraining.
\section{Secure Design of \puma}\label{sec:design}
In this section, we first present an overview of \puma, and present the protocols for secure $\gelu$ , $\softmax$, embedding, and $\layernorm$ used by \puma. Note that the linear layers such as matrix multiplication are straightforward in replicated secret sharing, so we mainly describe our protocols for non-linear layers in this manuscript.

\subsection{Overview of \puma}\label{sec:overview}
To achieve secure inference of Transformer models, \puma\ defines three kinds of roles: one model owner, one client, and three computing parties. The model owner and the client  provide their models or inputs to the computing parties (i.e., $P_0$, $P_1$, and $P_2$) in a secret-shared form, then the computing parties execute the MPC protocols and send the results back to the client. Note that the model owner and client can also act as one of the computing party, we describe them separately for generality. \eg, when the model owner acts as $P_0$, the client acts as  $P_1$, a third-party dealer acts as $P_2$, the system model becomes the same with \mpcformer~\citep{li2023mpcformer}.

During the secure inference process, a key invariant is maintained: For any layer, the computing parties always start with 2-out-of-3 replicated secret shares of the previous layer's output and the model weights, and end with 2-out-of-3 replicated secret shares of this layer's output. As the shares do not leak any information to each party, this ensures that the layers can be sequentially combined for arbitrary depths to obtain a secure computation scheme for any Transformer-based model.
%The main focus of \puma\ is to reduce the computation and communication costs between the computing parties while maintaining the desired level of security. 



\iffalse
\textbf{Threat Model.}
Following previous works~\citep{aby3,li2023mpcformer},
\puma\ resists a semi-honest (a.k.a., honest-but-curious) adversary in honest-majority~\citep{lindell2009proof}, where the adversary passively corrupts no more than one computing party. Such an adversary follows the protocol specification exactly, but may try to learn more information than permitted. Please note that \puma\ cannot protect against the extraction of information from the inference results, and the examination of mitigating solutions (\eg, differential privacy~\citep{abadi2016deep}) falls outside the scope of this study.
\fi 

\subsection{Protocol for Secure GeLU}\label{sec:gelu}
Most of the current approaches view the $\gelu$ function as a composition of smaller functions and try to optimize each piece of them, making them to miss the
chance of optimizing the private $\gelu$ as a whole. Given the $\gelu$ function:
\begin{equation}\label{eq:gelu}
\begin{split}
    \gelu(x) &= \frac{x}{2} \cdot \left(1 + \tanh \left( \sqrt{\frac{2}{\pi}} \cdot \left(x + 0.044715 \cdot x^3 \right) \right) \right)\\
    &\approx x\cdot \mathsf{sigmoid}(0.071355\cdot x^3 + 1.595769\cdot x) 
\end{split},
\end{equation}
these approaches~\citep{hao2022iron,characmpctranformer} focus either on designing efficient protocols for function $\tanh$
or using the existing MPC protocols of exponentiation and reciprocal for $\mathsf{sigmoid}$. 

However, none of current approaches have utilized the fact that $\gelu$ function is almost linear on the two sides (\ie, $\gelu(x)\approx 0$ for $x<-4$ and $\gelu(x)\approx x$ for $x>3$). 
Within the short interval $[-4,3]$ of $\gelu$,
we suggest a piece-wise approximation of low-degree polynomials is a more efficient and easy-to-implement choice for its secure protocol. Concretely, our piece-wise low-degree polynomials are shown as equation~(\ref{eq:geluapprox}):
\begin{equation}\label{eq:geluapprox}
\gelu(x)=
\begin{cases}
0, & x<-4 \\
F_0(x), & -4 \le x < -1.95 \\
F_1(x), & -1.95 \le x \le 3 \\
x, & x >3
\end{cases},
\end{equation}
where polynomials $F_0()$ and $F_1()$ are computed by library $\mathsf{numpy.ployfit}$\footnote{\url{https://numpy.org/doc/stable/reference/generated/numpy.polyfit.html}} as equation~(\ref{eq:f0f1}). Surprsingly, the above simple poly fit works very well and our $\mathsf{max\ error}< 0.01403$, $\mathsf{median\ error}< 4.41e-05$, and $\mathsf{mean\ error}< 0.00168$.
\begin{equation}\label{eq:f0f1}
\begin{cases}
F_0(x) &= -0.011034134030615728 x^3 -0.11807612951181953 x^2 \\
&- 0.42226581151983866 x -0.5054031199708174\\
F_1(x) &= 0.0018067462606141187x^6 -0.037688200365904236 x^4 \\
&+ 0.3603292692789629x^2 + 0.5x + 0.008526321541038084
\end{cases}
\end{equation}

Formally, given secret input $\share{x}$, our secure $\gelu$ protocol $\Pi_{\gelu}$ is constructed as algorithm~\ref{protocol:gelu}. 
\iffalse
\begin{itemize}
    \item The parties jointly compute
$\share{b_0}^2 = \Pi_{\mathsf{LT}}(\share{x}, 4)$,
$\share{b_1}^2 = \Pi_{\mathsf{LT}}(\share{x}, -1.95)$, and
$\share{b_2}^2 = \Pi_{\mathsf{LT}}(3, \share{x})$.

\item  Then, each $P_i$ locally compute
$\share{b_3}^2 = \share{b_1}^2 \oplus \share{b_2}^ \oplus 1$ and
$\share{b_4}^2 = \share{b_0}^2 \oplus \share{b_1}^2$

\item Finally, the parties compute and return 
$\share{b_2}^2 \cdot \share{x} + \share{b_4}^2 \cdot F_0(\share{x}) + \share{b_3}^2 \cdot F_1(\share{x})$, where polynomials $(F_0, F_1)$ can be computed easily using secure addition and multiplication (and its variants, \eg, secure square)~\citep{spu}. 
\end{itemize}
\fi 

\begin{algorithm}[tp]
\caption{Secure $\gelu$ Protocol $\Pi_{\mathsf{GeLU}}$}\label{protocol:gelu}
\begin{algorithmic}[1]
\REQUIRE
$P_i$ holds the 2-out-of-3 replicate secret share $\share{x}_i$ for $i\in \{0,1,2\}$ 
\ENSURE
$P_i$ gets the 2-out-of-3 replicate secret share $\share{y}_i$ for $i\in \{0,1,2\}$, where $y=\gelu(x)$.

\STATE $P_0$, $P_1$, and $P_2$ jointly compute
\begin{equation*}
\begin{split}
&\shareb{b_0} = \Pi_{\mathsf{LT}}(\share{x}, -4),~~~\vartriangleright b_0 = 1\{x<-4\}\\
&\shareb{b_1} = \Pi_{\mathsf{LT}}(\share{x}, -1.95),~~~\vartriangleright b_1 = 1\{x<-1.95\} \\
&\shareb{b_2} = \Pi_{\mathsf{LT}}(3, \share{x}),~~~~~~\vartriangleright b_2 = 1\{3<x\}
\end{split}
\end{equation*}
and compute 
$\shareb{z_0} = \shareb{b_0} \oplus \shareb{b_1}$,
$\shareb{z_1} = \shareb{b_1} \oplus \shareb{b_2} \oplus 1$, and $\shareb{z_2}=\shareb{b_2}$. Note that $z_0 = 1\{-4\le x < -1.95\}$, $z_1 = 1\{-1.95\le x\le 3\}$, and $z_2 = 1\{x>3\}$.

\STATE Jointly compute $\share{x^2} = \Pi_{\mathsf{Square}}(\share{x})$, $\share{x^3} = \Pi_{\mathsf{Mul}}(\share{x}, \share{x^2})$, $\share{x^4} = \Pi_{\mathsf{Square}}(\share{x^2})$, and $\share{x^6} = \Pi_{\mathsf{Square}}(\share{x^3})$.

\STATE Computing polynomials $\share{F_0(x)}$ and $\share{F_1(x)}$ based on $\{\share{x}, \share{x^2}, \share{x^3}, \share{x^4}, \share{x^6}\}$ as equation~(\ref{eq:geluapprox}) securely.


\RETURN$\share{y} = \Pi_{\mathsf{Mul_{BA}}}(\shareb{z_0}, \share{F_0(x)}) + \Pi_{\mathsf{Mul_{BA}}}(\shareb{z_1}, \share{F_1(x)})+\Pi_{\mathsf{Mul_{BA}}}(\shareb{z_2}, \share{x})$.

\end{algorithmic}
\end{algorithm}



\subsection{Protocol for Secure Softmax}\label{sec:secureatten}

In the function $\attention(\Q,\K,\V)=
\softmax(\Q \cdot \K^\mathsf{T} + \M) \cdot \V$, where $\M$ can be viewed as a bias matrix, the key challenge is computing function $\softmax$. For the sake of numerical stability, the $\softmax$ function is computed as
\begin{equation}\label{eq:softmax}
    \softmax(\x)[i]=\frac{\exp(\x[i] - \bar{x} - \epsilon)}{\sum_i \exp(\x[i] - \bar{x} - \epsilon)},
\end{equation}
where $\bar{x}$ is the maximum element of the input vector $\x$. 
For the normal plaintext softmax, $\epsilon=0$. For a two-dimension matrix, we apply equation~(\ref{eq:softmax}) to each of its row vector.

Formally, our detailed secure protocol  $\Pi_{\softmax}$ is illustrated in algorithm~\ref{protocol:softmax}, where we propose two optimizations:
\begin{itemize}
\item 
For the first optimization, we set $\epsilon$ in equation~\ref{eq:softmax} to a tiny and positive
value, e.g., $\epsilon =
10^{-6}$, so that the inputs to exponentiation
in equation~\ref{eq:softmax} are all negative. We exploit the negative operands
for acceleration. Particularly, we compute the exponentiation using the Taylor series~\citep{tan2021cryptgpu} with a simple clipping
\begin{equation}\label{eq:negexp}
\mathsf{negExp}(x) = \begin{cases}
    0, &x < T_{\exp} \\
    (1+\frac{x}{2^t})^{2^t}, &x\in [T_{\exp},0].
\end{cases}
\end{equation}
Indeed, we apply the less-than for the branch $x < T_{\exp}$
The division by $2^t$ can be achieved using
$\Pi_{\mathsf{Trunc}}^t$ since the input is already negative. Also, we can
compute the power-of-$2^t$ using $t$-step sequences of square function $\Pi_{\mathsf{square}}$ and $\Pi_{\mathsf{Trunc}}^f$. Suppose our MPC program uses
$18$-bit fixed-point precision. Then we set $T_{\exp}=-14$ given $\exp(-14) < 2^{-18}$, and empirically set $t = 5$.


\item 
Our second optimization is to reduce the number of divisions, which ultimately saves computation and communication costs.
To achieve this, for a vector $\x$ of size $n$, we have replaced the operation $\mathsf{Div}(\x, \mathsf{Broadcast}(y))$ with $\x \cdot  \mathsf{Broadcast}(\frac{1}{y})$, where $y=\sum_{i=1}^n\x[i]$. By making this replacement, we effectively reduce $n$ divisions to just one reciprocal operation and $n$ multiplications.
This optimization is particularly beneficial in the case of the $\softmax$ operation. The $\frac{1}{y}$ in the $\softmax$ operation is still large enough to maintain sufficient accuracy under fixed-point values. As a result, this optimization can significantly reduce the computational and communication costs while still providing accurate results.
\end{itemize}

\begin{algorithm}[tp]
\caption{Secure $\softmax$ Protocol $\Pi_{\softmax}$}\label{protocol:softmax}
\begin{algorithmic}[1]
\REQUIRE
$P_i$ holds the 2-out-of-3 replicate secret share $\share{\x}_i$ for $i\in \{0,1,2\}$, and $\x$ is a vector of size $n$. 
\ENSURE
$P_i$ gets the 2-out-of-3 replicate secret share $\share{\y}_i$ for $i\in \{0,1,2\}$, where $\y=\softmax(\x)$.

\STATE $P_0$, $P_1$, and $P_2$ jointly compute
$\shareb{\mathbf{b}} = \Pi_{\mathsf{LT}}(T_{\exp}, \share{\x})$ and the maximum $\share{\bar{x}} = \Pi_{\mathsf{Max}}(\share{\x})$.

\STATE Parties locally computes $\share{\hat{\x}} = \share{\x} - \share{\bar{x}} - \epsilon$, and jointly compute $\share{\z_0} = 1+  \Pi_{\mathsf{Trunc}}^t(\share{\hat{\x}})$.

\FOR{$j=1,2,\dots, t$}
\STATE $\share{\z_j} = \Pi_{\mathsf{Square}}(\share{\z_{j-1}})$.
\ENDFOR

\STATE Parties locally compute $\share{z} = \sum_{i=1}^n \share{\z[i]}$ and jointly compute $\share{1/z} = \Pi_{\mathsf{Recip}}(\share{z})$.

\STATE Parties jointly compute $\share{\z / z} = \Pi_{\mathsf{Mul}}(\share{\z}, \share{1/z})$

\RETURN $\share{\y} = \Pi_{\mathsf{Mul}_{\mathsf{BA}}}( \shareb{\mathbf{b}}, \share{\z / z})$.

\end{algorithmic}
\end{algorithm}

\subsection{Protocol for Secure Embedding}\label{sec:embed}


The current secure embedding procedure described in~\citep{li2023mpcformer} necessitates the client to  generate a one-hot vector using the token $\tokenid$ locally. This deviates from a plaintext Transformer workflow where the one-hot vector is generated inside the model. As a result, they have to carefully strip off the one-hot step from the pre-trained models, and add the step to the client side, which could be an obstacle for deployment. 



To address this issue, we propose a secure embedding design as follows. Assuming that the token $\tokenid\in [n]$ and all embedding vectors are denoted by $\E= (\e_1^T, \e_2^T, \dots, \e_n^T)$, the embedding can be formulated as $\e_{\tokenid} = \mathbf{E}[\tokenid]$. Given $(\tokenid, \E)$ are in secret-shared fashion, our secure embedding protocol $\Pi_{\mathsf{Embed}}$ works as follows:
\begin{itemize}
    \item The computing parties securely compute the one-hot vector $\shareb{\mathbf{o}}$ after receiving $\share{\tokenid}$ from the client. Specifically, $\shareb{\mathbf{o}[i]}=\Pi_{\mathsf{Eq}}(i,\share{\tokenid})$ for $i\in [n]$.
    \item The parties can compute the embedded vector via $\share{\e_{\tokenid}} = \Pi_{\mathsf{Mul_{BA}}}(\share{\E}, \shareb{\mathbf{o}})$, where  does not require secure truncation.
\end{itemize}
In this way, our $\Pi_{\mathsf{Embed}}$ does not require explicit modification of the workflow of plaintext Transformer models, at the cost of more $\Pi_{\mathsf{Eq}}$ and $\Pi_{\mathsf{Mul_{BA}}}$ operations. 



\subsection{Protocol for Secure LayerNorm}\label{sec:seclayernorm}
Recall that given a vector $\x$ of size $n$, $\layernorm(\x)[i] =  \gamma \cdot \frac{\x[i]-\mu}{\sqrt{\sigma}} + \beta$, where $(\gamma, \beta)$ are trained parameters, $\mu = \frac{\sum_{i=1}^n \x[i]}{n}$, and $\sigma = \sum_{i=1}^n (\x[i] - \mu)^2$. In MPC, the key challenge is the evaluation of the divide-square-root $\frac{\x[i]-\mu}{\sqrt{\sigma}}$ formula. To securely evaluate this formula, CrypTen sequentially executes the MPC protocols of square-root, reciprocal, and multiplication. However, we observe that $\frac{\x[i]-\mu}{\sqrt{\sigma}}$ is equal to $(\x[i]-\mu)\cdot \sigma^{-1/2}$. And in the MPC side, the costs of computing the inverse-square-root $\sigma^{-1/2}$ is similar to that of the square-root operation~\citep{rSqrt}. Besides, inspired by the second optimization of \S~\ref{sec:secureatten}, we can first compute $\sigma^{-1/2}$ and then $\mathsf{Broadcast}(\sigma^{-1/2})$ to support fast and secure $\layernorm(\x)$. And our formal protocol $\Pi_{\layernorm}$ is shown in algorithm~\ref{protocol:layernorm}.

\begin{algorithm}[tp]
\caption{Secure $\mathsf{LayerNorm}$ Protocol $\Pi_{\mathsf{LayerNorm}}$}\label{protocol:layernorm}
\begin{algorithmic}[1]
\REQUIRE
$P_i$ holds the 2-out-of-3 replicate secret share $\share{\x}_i$ for $i\in \{0,1,2\}$, and $\x$ is a vector of size $n$. 
\ENSURE
$P_i$ gets the 2-out-of-3 replicate secret share $\share{\y}_i$ for $i\in \{0,1,2\}$, where $\y=\mathsf{LayerNorm}(\x)$.

\STATE $P_0$, $P_1$, and $P_2$ compute $\share{\mu} = \frac{1}{n}\cdot \sum_{i=1}^n\share{\x[i]}$ and $\share{\sigma} = \sum_{i=1}^n \Pi_{\mathsf{Square}}(\share{\x} - \share{\mu})[i]$.

\STATE Parties jointly compute $\share{\sigma^{-1/2}} = \Pi_{\mathsf{rSqrt}}(\share{\sigma})$.

\STATE Parties jointly compute $\share{\mathbf{c}} = \Pi_{\mathsf{Mul}}((\share{\x} - \share{\mu}), \share{\sigma^{-1/2}})$

\RETURN $\share{\y} = \Pi_{\mathsf{Mul}}(\share{\gamma}, \share{\mathbf{c}}) + \share{\beta}$.

\end{algorithmic}
\end{algorithm}
\section{METHODS}
\label{sec:methods}
\subsection{Problem Definition and Proposed Framework}
The objective is to reconstruct a dense point cloud that precisely represents the shape of unknown transparent objects from sparse point clouds extracted with active tactile interactive perception. To this end, we propose a novel framework termed ACTOR shown in Fig.~\ref{fig:framework}. In Fig.~\ref{fig:framework}(a) we propose a self-surpervised learning approach with an autoencoder network that is trained on subsampled pointclouds from synthetic objects belonging to the same category but not identical as the real objects. In Fig.~\ref{fig:framework}(b), we propose a novel active tactile-based unknown transparent object exploration strategy which is used for inference with our trained model to reconstruct a dense point cloud. We demonstrate downstream tasks such as tactile-based pose estimation.
% and tactile-based object recognition. 

\subsection{Deep Self-Supervised Learning for 3D Object Reconstruction}
\label{ssec:deep_reconstruction}
We generate a dataset $\mathcal{D}$\footnote{\url{https://www.robotact.de/tactile-reconstruction}} of synthetic object models from the ShapeNet repository~\cite{chang2015shapenet} in order to leverage the open-source datasets and avoid expensive real tactile-data collection. The synthetic object models belong to the same category but are different from the real unknown transparent objects. 
We uniformly sample $N_{in} = 2048$ points from the synthetic object meshes. These pointclouds are normalized and scaled to fit into a $[0,1]^3$ cube and added to the dataset, $\mathcal{P}_{in} \in \mathcal{D}$. 
% The generated dataset is provided in the project page\footnote{\url{https://robotac-bmw.github.io/tactile_reconstruction/}}.
In order to generate the input point clouds $\mathcal{P}^{\bullet}_{in}$ to the network, we randomly subsample the $\mathcal{P}_{in}$ by voxel-grid subsampling by the factor $k$ i.e., $\mathcal{P}^{\bullet}_{in} \in \mathbb{R}^{\lceil \frac{1}{k}N_{in} \rceil \times 3}$.  This creates a challenging task for reconstruction with higher values for $k$ as simpler techniques based on interpolation with neighborhood points cannot be used. 

\subsubsection*{Feature-Extraction Encoder}
The network architecture shown in Figure~\ref{fig:framework}(a) is proposed as an autoencoder (AE) that uses a self-supervised approach to reconstruct the original point cloud from a subsampled point cloud. 
The encoder takes subsampled point clouds as inputs and generates a high dimensional feature vector. The feature vector captures the global geometric shape information of the input point cloud. 
In general, any deep network that works on raw input point clouds to provide a high dimensional feature vector can be used as an encoder. In particular,
we use a modified PointNet architecture~\cite{qi2017pointnet} for the encoder. PointNet takes unordered point clouds and generates a global feature descriptor vector of size 1024. The network learns a set of optimization functions that select interesting or informative points of the point cloud. The encoder consists of $[1\times1]$ convolutions with output channels size $(64, 64, 128, 1024)$ with the first convolutional layer with kernel size $[1\times3]$ to encode the input pointcloud of $N\times3$ dimension. The convolution layers are aggregated by a max-pooling layer. We introduce a self-attention layer~\cite{zhang2019self} whose outputs are aggregated with the max-pooled features to provide the global feature vector.  
We have summarized the encoder in Figure~\ref{fig:framework}(a).
% As the encoder provides a high-dimensional global feature vector, we term it as feature-extraction encoder.

\textbf{Self-Attention (SA) Layer:} The SA layer is introduced as it can encode meaningful spatial relationships between features and focus on important local features. From the input layer ($\mathtt{conv2d-1024}$), two separate multi-layer perceptrons (MLPs) are used to get features $\mathbf{G}$ and $\mathbf{H}$ which are subsequently used to get the weights as $\mathbf{W} = softmax(\mathbf{G}^T\mathbf{H})$. The input features are transformed using another MLP to obtain $\mathbf{K}$ and multiplied with the weights as $\mathbf{W}^T\mathbf{K}$.
These vectors are summed with the input vector to produce the output features.
% The SA layer description is shown in Fig.~\ref{fig:self_atten}.  
% \setlength{\columnsep}{0pt}
% \begin{wrapfigure}[12]{r}{0.8\linewidth}
%   \centering
%     % \vspace{-0.5cm}
%     % Figure removed
%   \caption{The self-attention unit.}
%     % \vspace{-0.5cm}
%   \label{fig:self_atten}
% \end{wrapfigure}
% % Figure environment removed

\subsubsection*{Upsampling Decoder}
We design an upsampling decoder that upsamples the input global feature vector to provide the reconstructed dense output point cloud $\mathcal{P}_{out}$. The upsampling decoder is composed by a fully connected layer with output dimension of 1024 and five deconvolutional layers with kernel sizes and output channels shown in Fig.~\ref{fig:framework}(a).  
The decoder produces the output point cloud with point size set to 2048 while training as this is sufficiently dense for reconstruction purposes. 

\subsubsection*{Loss Function}
In order to encourage the upsampled point cloud to be in proximity to the original input point cloud and follow the underlying geometrical surface of the object, we use the Chamfer distance metric~\cite{borgefors1986distance} as the loss. Given the input point cloud prior to subsampling, $\mathcal{P}_{in}$ and the reconstructed output point cloud $\mathcal{P}_{out}$, the loss is defined as:
\begin{align}
    \mathcal{L}_{CD}(\mathcal{P}_{in}, \mathcal{P}_{out}) &= \frac{1}{|\mathcal{P}_{in}|}\sum_{p_1 \in \mathcal{P}_{in}} \min_{p_2 \in \mathcal{P}_{out}} ||p_1 - p_2||_{2} + \\ & \frac{1}{|\mathcal{P}_{out}|}\sum_{p_2 \in \mathcal{P}_{out}} \min_{p_1 \in \mathcal{P}_{in}} ||p_2 - p_1||_{2} \nonumber,
    \label{eq:chamfer_dist}
\end{align}
where $|\bullet|$ refers to the number of points in the point cloud and $||\bullet||_2$ refers to the L2 norm. The loss $\mathcal{L}_{CD}$ represents the average distance between the \textit{closest} points in the two point clouds. We use the weighted loss for learning stability as the reconstruction loss $\mathcal{L}_{rec} = \alpha\mathcal{L}_{CD}$ with $\alpha = 100$ set empirically.
For surface reconstruction from the dense reconstructed point cloud, we use the ball-pivoting algorithm~\cite{bernardini1999ball}.

% \subsubsection*{Recognition Network}
% \label{ssec:recog_net}
% The pretrained encoder layers for reconstruction task are frozen for category-level classification. We employ three fully-connected layers with parameters 512, 256, and $n$ respectively where $n$ represents the number of categories of the objects.
% The softmax cross-entropy loss is used for training the recognition network. The recognition head is shown in Fig.~\ref{fig:framework}(a.I). The subsampled sparse point clouds from our synthetic dataset with different subsampling ratios and data augmentation with random rotations are used. Network implementation details are provided in Sec.~\ref{ssec:setup}.



%%%%%%%%%%%%%%%%%%%%%%%%%%%%%%%%%%%%%%%%%%%%%%%%%%%%%%%%%%%%%%%%%%%%%%%
%%%%%%%%%%%%%%%%%%%%%%%%%%%%%%%%%%%%%%%%%%%%%%%%%%%%%%%%%%%%%%%%%%%%%%%
\subsection{Active Deep Tactile-based Unknown Transparent Object Reconstruction and Pose Estimation}
\subsubsection{Active Tactile-based Transparent Object Reconstruction}
The model trained with only \textit{synthetic data} as described in Sec.~\ref{ssec:deep_reconstruction} is used during the inference with \textit{real-world} transparent objects. The sparse tactile point cloud data is collected autonomously by the robot using an information gain-based active strategy. We define two types of tactile actions for data acquisition: touch and pinch actions as shown in Figure~\ref{fig:occupancy_grid}.
% The action nomenclature is derived from human grasp taxonomy studies~\cite{feix2015grasp}.
The touch action is executed as a guarded horizontal straight-line motion wherein the object is not moved upon contact. The touch action is defined by a tuple $\mathbf{a}^{t} = \{\mathbf{s}^t, \overrightarrow{\mathbf{d}^t} \}$ where $\mathbf{s}^t \in \mathbb{R}^3$ is the start point of the tactile-sensorised gripper and $\overrightarrow{\mathbf{d}^t} \in \mathbb{R}^3$ is the direction of the gripper-motion defined in the world-coordinate frame $\mathcal{W}$. During the pinch action the robot approaches the object in a vertical straight-line motion with a completely open gripper and performs an antipodal enclosure grasp on the object. The fingers of the gripper are closed until the force on the tactile sensors exceeds a predefined threshold.
The pinch action is characterized by $\mathbf{a}^{p} = \{\mathbf{s}^p \}$ where $\mathbf{s}^p \in \mathbb{R}^3 $ is the start position of the gripper motion vertically above the object at a predefined height as shown in Figure~\ref{fig:occupancy_grid}. Given the 2D bounding box of the object (a priori known or through a RGB camera), a probabilistic occupancy grid $\mathcal{OG}_i$ of preset height and resolution $og_{res}$ is defined. Each cell of the occupancy grid $c_i$ is represented by an occupancy probability $p(c_i)$ which is initially set to 0.5. During exploration, if a cell is discovered to belong to the object, the probability is set to 1 and similarly, if the cell belongs to free space, the probability is set to 0. The probabilities are updated through ray intersections based on the virtual sensor model. We define a virtual sensor model of the tactile sensor which casts a set of rays $\mathcal{R} = \{r_1, r_2, \dots, r_{n_{taxel}} \}$ where ${n_{taxel}} $ refers to the number of taxels in the sensor array. The independence assumption of the probability of each grid cell with one another allows us to calculate the overall entropy of the $\mathcal{OG}$ as the summation of the entropy of each cell. The Shannon entropy of the overall occupancy grid is calculated as:
\begin{equation}
    \mathbb{H}(\mathcal{OG}) = \sum_{c_i \in \mathcal{OG}} p(c_i)log(p(c_i)) + (1 - p(c_i))(1 - log(p(c_i))).
    \label{eq:entropy}
\end{equation}
Monte-Carlo sampling of possible tactile actions $N_{nbt}$ are performed for computing the next best tactile (NBT) action. The actions space $\mathcal{A}_{nbt}$ is comprised of an equal number of touch and pinch respectively as $\mathcal{A}_{nbt} = \{a^p, a^t\}_{N_{nbt}}$. The expected measurements $\hat{\mathbf{z}}_t$ for each action $a_t \in \mathcal{A}$ is computed using ray-traversal algorithms~\cite{hornung2013octomap}. 
Given the observed grid cell $c$ and the measurement from sensor observation $z$, the log-odds is updated as $L(c|z) = L(c) + l(z)$ wherein $L(c) = log\frac{p(c)}{1-p(c)}$ and  
\begin{equation}
    l(z) = \left\{
                \begin{array}{ll}
                  log\frac{p_h}{1-p_h}  \quad \mathrm{if} \ z \widehat{=} \textit{ hit} \\
                  log\frac{p_m}{1-p_m} \quad \mathrm{if} \ z \widehat{=} \textit{ miss} 
                \end{array}
              \right.
    \label{eq:log-odds}
\end{equation}
where $p_h$ and $p_m$ are the probabilities of hit and miss which are user-defined values set to 0.7 and 0.4 respectively as in~\cite{hornung2013octomap}. The posterior probability $p(c|z)$ can be computed by inverting $L(c|z)$. The expected information gain by taking an action $a_t \in \mathcal{A}_{nbt}$ with expected measurement $\hat{\mathbf{z}}_t$ is provided by the Kullback-Liebler divergence of the posterior entropy and the prior entropy as: 
\begin{equation}
    E[\mathbb{I}(p(c_i | \mathbf{a}_t,  \hat{z}_t))] = \mathbb{H}(p(c_i)) - \mathbb{H}(p(c_i | \mathbf{a}_t,  \hat{z}_t))
    \label{eq:kl_view}
\end{equation}
Therefore, the action that maximizes the expected information gain is considered as the NBT action:
\begin{equation}
    \mathbf{a}^{nbt*}_t = \argmax_{\mathbf{a} \in \mathcal{A}}(E[\mathbb{I}(p(c_i | \mathbf{a}_t,  \hat{z}_t))])
    \label{eq:kl_view_max}
\end{equation}
Each tactile action extracts contact positions in 3D space and contact forces. The direction of the normal force is used to extract the normal direction $\hat{n}$ of the object surface. The contact points are aggregated into the tactile point cloud $\mathcal{P}^t$. In order to initialize the NBT action calculation, an initial point cloud (with $N_{\mathcal{P}^t} = 20$) is required, which is extracted by randomised touch actions. Further points are collected in an active manner using the NBT criteria. A minimum number of points in the tactile point cloud is required to perform model inference $N_{\mathcal{P}^t} > N_{min}$ which is tuned empirically. The tactile point cloud is provided as input to the trained network and the reconstructed point cloud $\mathcal{P}_{out}$ is obtained . 
% This is used for downstream task Section~\ref{ssec:pose_estimation}. 
% For acceptable reconstruction accuracy around 100 tactile points is required.  

% [TODO:] check for action taxonomy if its correct

%%%%%%%%%%%%%%%%%%%%%%%%%%%%%%%%%%%%%%%%%%%%%%%%%%%%%%%%%%%%%%%%%%%%%%%
% Figure environment removed
%%%%%%%%%%%%%%%%%%%%%%%%%%%%%%%%%%%%%%%%%%%%%%%%%%%%%%%%%%%%%%%%%%%%%%%


%%%%%%%%%%%%%%%%%%%%%%%%%%%%%%%%%%%%%%%%%%%%%%%%%%%%%%%%%%%%%%%%%%%%%%%
%%%%%%%%%%%%%%%%%%%%%%%%%%%%%%%%%%%%%%%%%%%%%%%%%%%%%%%%%%%%%%%%%%%%%%%
\subsubsection{Tactile-Based Object Pose Estimation}
\label{ssec:pose_estimation}

We perform the 6D pose estimation through dense to sparse point cloud registration. The sparse scene point cloud $\mathbf{s}_i \in \mathcal{S}$ is represented by the tactile points and the dense object point cloud $\mathbf{o}_i \in \mathcal{O}$ is represented by the reconstructed point cloud in~\ref{ssec:deep_reconstruction} without the need for the object model. Point cloud registration problem with $M$ known correspondences can be formulated as:
\begin{equation}
     \mathbf{s}_i = \mathbf{S}\cdot(\mathbf{R}\mathbf{o}_i) + \mathbf{t} \quad i = 1, \dots M,
     \label{eq:generativemodel}
 \end{equation}
where $\mathbf{S} \in \mathbb{R}^3$ represents scale, $\mathbf{R} \in SO(3)$ represents rotation and $\mathbf{t} \in \mathbb{R}^3$ represents translation which are unknown and to be estimated and $\cdot$ is the element-wise product. 
%% [TODO] : check derivation

We perform the point cloud registration using our novel translation-invariant Quaternion filter (TIQF) presented in~\cite{murali2022active} to determine $\mathbf{R}$, $\mathbf{S}$ and $\mathbf{t}$. 
The scale, rotation and translation are decoupled by finding the relative vectors between corresponding points, i.e., $\forall o_i, o_j \in \mathcal{O}, s_i, s_j \in \mathcal{S}$ the relative vectors are $\mathbf{s}_{ji} = \mathbf{s}_j - \mathbf{s}_i$ and $\mathbf{o}_{ji} = \mathbf{o}_j - \mathbf{o}_i$. Equation~\eqref{eq:generativemodel} is reformulated as:
\begin{align}
    \mathbf{s}_j - \mathbf{s}_i &= (\mathbf{S}\cdot\mathbf{R}\mathbf{o}_j + \mathbf{t}) - (\mathbf{S}\cdot\mathbf{R}\mathbf{o}_i + \mathbf{t}) ,\\
    \mathbf{s}_{ji} &= \mathbf{S}\cdot\mathbf{R}\mathbf{o}_{ji} \quad .
    \label{eq:trans_invariance}
\end{align}

We note that equation~\eqref{eq:trans_invariance} is independent of translation. Taking the L2-norm on both sides for Eq.~\eqref{eq:trans_invariance} and recalling that norm is rotation invariant we get:
\begin{equation}
    \mathbf{||s||}_{ji} = \mathbf{||S||}\cdot\mathbf{||o||}_{ji} \quad .
    \label{eq:rot_invariance}
\end{equation}
The scale $\mathbf{S}$ is estimated by taking the ratio of the axis aligned bounding box (AABB) of the scene and object point clouds, i.e., if $\mathcal{X}_{AABB} = \{ (x_{min}, x_{max}), (y_{min}, y_{max}), (z_{min}, z_{max}) \}$ represents the AABB for a point cloud $\mathcal{X}$, then:
\begin{align}
     \mathbf{S} &= \{ \frac{|x_{max} - x_{min}|_{\mathcal{S}}}{|x_{max} - x_{min}|_{\mathcal{O}}}, \frac{|y_{max} - y_{min}|_{\mathcal{S}}}{|y_{max} - y_{min}|_{\mathcal{O}}} , \frac{|z_{max} - z_{min}|_{\mathcal{S}}}{|z_{max} - z_{min}|_{\mathcal{O}}}    \}
     \label{eq:scale}
 \end{align}
Using the estimated scale and using $\tilde{\mathbf{o}}_{ji} = \mathbf{S}\mathbf{o}_{ji}$ for convenience we are left with a pure rotation to estimate:  
\begin{align}
    \tilde{\mathbf{s}}_{ji} &= \mathbf{R}\tilde{\mathbf{o}}_{ji} \quad .
    \label{eq:trans_scale_invariance}
\end{align}
 We cast the rotation estimation problem into a recursive Bayesian estimation framework and derive a linear state and measurement model. Reformulating Eq.\eqref{eq:trans_scale_invariance} using quaternions we get: 
 \begin{equation}
    \overline{\mathbf{s}}_{ji} = \mathbf{x} \odot \overline{\mathbf{o}}_{ji} \odot \mathbf{x}^{*}, 
    \label{eq:quat_objective}
\end{equation}
where $\mathbf{x}$ is the quaternion form of $\mathbf{R}$, $\odot$ is the quaternion product, ${\mathbf{x}}^{*}$ is the conjugate of $\mathbf{x}$, and $\overline{\mathbf{s}}_{ji}=\{0,\tilde{\mathbf{s}}_{ji}\}$ and $\overline{\mathbf{o}}_{ji}=\{0,\tilde{\mathbf{o}}_{ji}\}$.
Using the matrix form of quaternion product, we can rewrite Eq.\eqref{eq:quat_objective} as:
\begin{align}
    \begin{bmatrix}
        0 & -\tilde{\mathbf{s}}_{ji}^T \\
        \tilde{\mathbf{s}}_{ji} & \tilde{\mathbf{s}}_{ji}^{\times}
    \end{bmatrix}\mathbf{x} -  \begin{bmatrix}
        0 & -\tilde{\mathbf{o}}_{ji}^T \\
        \tilde{\mathbf{o}}_{ji} & -\tilde{\mathbf{o}}_{ji}^{\times}
    \end{bmatrix} \mathbf{x} = \mathbf{0} \\
    \underbrace{\begin{bmatrix}
        0 & -(\tilde{\mathbf{s}}_{ji} - \tilde{\mathbf{o}}_{ij})^T \\
        (\tilde{\mathbf{s}}_{ji} - \tilde{\mathbf{o}}_{ji}) & (\tilde{\mathbf{s}}_j + \tilde{\mathbf{s}}_i + \tilde{\mathbf{o}}_j + \tilde{\mathbf{o}}_i)^{\times}
        \end{bmatrix}_{4 \times 4}}_{\mathbf{H}_t} \mathbf{x} &= \mathbf{0} \quad ,
        \label{eq:expected_measurement}
\end{align}
where $(\ )^\times$ denotes the skew-symmetric matrix formulation. Equation~\eqref{eq:expected_measurement} is of the form $\mathbf{H}_t\mathbf{x} = 0$ where $\mathbf{H}_t$ is the pseudo-measurement matrix~\cite{choukroun2006novel}. We note that Eq.~\eqref{eq:expected_measurement} represents a noise-free state estimation where $\mathbf{H}_t$ depends only on sparse and dense point correspondences which are $\tilde{\mathbf{s}}_{ji}$ and $\tilde{\mathbf{o}}_{ji}$. We design a pseudo-measurement model as $ \mathbf{H}_t \mathbf{x} = \mathbf{z}^h$
% \begin{align}
%     \mathbf{H}_t \mathbf{x} &= \mathbf{z}^h,
%     \label{eq:measurement_model}
% \end{align}
and set $\mathbf{z}^h = 0$. Since we have a static process model, the object does not move and $\mathbf{x}$ and $\mathbf{z}_t$ are Gaussian distributed, 
the state $\mathbf{x}_t$ and covariance matrix $\Sigma^{\mathbf{x}}_{t}$ at each timestep $t$ are computed through a linear Kalman filter. The Kalman filter equations are skipped for brevity and a in-depth derivation is provided in our prior work~\cite{murali2022active}.
As the Kalman filter does not implicitly ensure the constraints on the quaternion as $||\mathbf{x}|| = 1$, we normalise the state and uncertainty after each update step as $\bar{\mathbf{x}}_{t} = \frac{\mathbf{x}_{t}}{||\mathbf{x}_{t}||_2} \quad, \bar{\Sigma}^{\mathbf{x}}_{t} = \frac{\Sigma^{\mathbf{x}}_{t}}{||\mathbf{x}_{t}||_2^2}$. We convert the estimated rotation $\Bar{\mathbf{x}}_t$ to its equivalent rotation matrix $\mathbf{R}$. It used to estimate the translation using the following relation: $\mathbf{t} = \frac{1}{N} \sum_{i=0}^{N} (\Bar{\mathbf{s}}_i - \mathbf{R} \Bar{\mathbf{o}}_i).$
% \begin{equation}
%     \mathbf{t} = \frac{1}{N} \sum_{i=0}^{N} (\Bar{\mathbf{s}}_i - \mathbf{R} \Bar{\mathbf{o}}_i).
%     \label{eq:translation_solution}
% \end{equation}
% \setlength{\columnsep}{1pt}
% \begin{wrapfigure}[18]{r}{0.6\linewidth}
%   \centering
%     \vspace{-0.5cm}
%     % Figure removed
%   \caption{Translation-invariant measurements}
%     % \vspace{-0.5cm}
%   \label{fig:TIMS}
% \end{wrapfigure}
At each iteration, a rotation and translation estimate is found which is used to transform the object point cloud and the process is repeated by re-estimating the correspondence points. The convergence criteria are set by (a) maximum number of iterations or (b) the relative change in estimated pose parameters is less than a predefined threshold ($0.1mm$ and $0.1^o$). 

% the linear Kalman filter equations are given as:
% \begin{align}
%     \mathbf{x}_{t} &= \bar{\mathbf{x}}_{t-1} - \mathbf{K}_t \left( \mathbf{H}_t \bar{\mathbf{x}}_{t-1} \right) \\
%     \Sigma^{\mathbf{x}}_{t} &= \left( \mathbf{I} - \mathbf{K}_t \mathbf{H}_t \right) \bar{\Sigma}^{\mathbf{x}}_{t-1} \\
%     \mathbf{K}_t &= \bar{\Sigma}^\mathbf{x}_{t-1} \mathbf{H}_t^T \left( \mathbf{H}_t\bar{\Sigma}^\mathbf{x}_{t-1} \mathbf{H}_t^T + \Sigma_t^{\mathbf{h}}\right)^{-1}, 
%     \label{eq:kalman_equations}
% \end{align}
% where $\bar{\mathbf{x}}_{t-1}$ refers to the normalized mean of the state at $t-1$, Kalman gain $\mathbf{K}_t$ and $\bar{\Sigma}^{\mathbf{x}}_{t-1}$ is the covariance matrix of the state at $t-1$. 
% The parameter $\Sigma_t^{\mathbf{h}}$ is referred as the measurement uncertainty during time $t$. It is dependent on the state and is provided by~\cite{choukroun2006novel}:
% \begin{align}
%     \Sigma_t^{\mathbf{h}} = \frac{1}{4}\rho\left[ tr(\bar{\mathbf{x}}_{t-1}\bar{\mathbf{x}}_{t-1}^T + \bar{\Sigma}^{x}_{t-1})\mathbb{I}_4 - (\bar{\mathbf{x}}_{t-1}\bar{\mathbf{x}}_{t-1}^T + \bar{\Sigma}^{x}_{t-1} )\right], 
%     \label{eq:choukron}
% \end{align}
% wherein the constant $\rho$ corresponds to the uncertainty of the correspondence measurements and $tr$ refers to trace.


%%%%%%%%%%%%%%%%%%%%%%%%%%%%%%%%%%%%%%%%%%%%%%%%%%%%%%%%%%%%%%%%%%%%%%%
%%%%%%%%%%%%%%%%%%%%%%%%%%%%%%%%%%%%%%%%%%%%%%%%%%%%%%%%%%%%%%%%%%%%%%%
% \subsubsection{Transparent Object Manipulation}
% \label{ssec:tactile_manipulation}
% With the computed 6D pose and estimated CAD model, we design a simple grasping technique in order to grasp and lift the transparent objects. For each \textit{category} of objects, we generated several grasp plans using GraspIt~\cite{miller2004graspit}. Each grasp plan includes the grasp position, orientation and approach vector relative to the model of the object and a grasp quality score. With the pose of the object, the grasp plans are filtered based on kinematic constraints of the robot, workspace limitations and possible collisions with other objects in the scene. Among the remaining grasp plans, the plan with the highest score is chosen and executed. The robot lifts the transparent object and places it in a pre-defined position.
% An online grasp planning and collision avoidance framework is out of the scope of this current work but can be readily integrated into the current framework.

%%%%%%%%%%%%%%%%%%%%%%%%%%%%%%%%%%%%%%%%%%%%%%%%%%%%%%%%%%%%%%%%%%%%%%%
%%%%%%%%%%%%%%%%%%%%%%%%%%%%%%%%%%%%%%%%%%%%%%%%%%%%%%%%%%%%%%%%%%%%%%%

% \subsubsection{Tactile-based Transparent Object Recognition}
% \label{ssec:classification}
% % Figure environment removed
% We use the pretrained encoder model with fixed weights for category-level classification. We employ three fully-connected layers with parameters 512, 256 and $n$ respectively where $n$ represents the number of categories of the objects. Transfer learning is employed to fine-tune the classification network shown in Figure~\ref{fig:framework}(a) on the sparse pointclouds from ShapeNet database.
% During inference, the real sparse tactile pointclouds are used as input to the network for recognition network described in Sec.~\ref{ssec:recog_net}. While the task is challenging, the real-world tactile data are not used during fine-tuning intentionally as collection of large-scale datasets is prohibitively time consuming. The input pointcloud is pre-processed prior to inference by normalising and scaling to fit in $[0,1]^3$ cube to be uniform with the training dataset.

In this section we explore a few applications of the techniques
introduced in section~\ref{sec:mth}. First we consider the application
of the reweighting technique to an optimization problem. 
Second, we consider the application in Bayesian inference to obtain
the dependence of predictions on the parameters that characterize the
prior distribution.

\subsection{Applications in optimization}
\label{sec:opt}

As an example application of an optimization process we will consider
the probability density function
\begin{equation}
  p_\theta(x) = \frac{1}{\mathcal Z}\exp \left\{ -S(x;\theta) \right\}\,, \qquad \left(  \mathcal Z = \int {\rm d} x\, e^{-S(x:\theta)} \right) \,.
\end{equation}
with
\begin{equation}
  S(x; \theta) = \frac{1}{\theta_1^2+1} \left( x_1^2 + x_1^4 \right) + \frac{1}{2}x_2^2 + \theta_2 x_1x_2\,.
\end{equation}

The shape of $S(x;\theta)$ is inspired in the action of a quantum
field theory in zero dimensions, where $x_1$ and $x_2$ are two fields
with coupling $\theta_2$, while $\theta_1$ is related to the mass of
the field $x_1$.
Expectation values with respect to $p_\theta(x)$ are functions of
the parameters $\theta$.  

% Figure environment removed

As an example we consider the problem of minimizing $\mathbb
E_\theta[x_1^2 + x_2^2]$ (i.e. 
finding the values for $\theta$ that make $\mathbb
E_\theta[x_1^2+x_2^2]$ minimum). 
We have implemented two flavours of Stochastic Gradient Descent (SGD): the first -basic- one, having a constant learning rate, and the second one being the well-known
%both a basic stochastic gradient descent (with constant learning rate) and the
ADAM algorithm \cite{kingma2017adam}. It is worth noting at this
point that as a general concept, SGD implies a stochastic (but
unbiased) evaluation of the gradients of the objective function at
every iteration. While in typical applications in the ML community,
where the task is to fit some dataset, this is done by evaluating the
gradients at different random batches of the data, the present example
is different in that no data is involved. In this case, every
iteration of the SGD evaluates the gradients on the different Monte
Carlo samples used to approximate the objective function  $\mathbb
E_\theta[x_1^2 + x_2^2]$.  

%These algorithms require stochastic evaluations both of the functionand its gradient at arbitrary values of the parameters $\theta$. 
%Here we perform these evaluations via Monte Carlo sampling: we use a
Here we consider a simple implementation of the Metropolis Hastings algorithm in order to
first produce the samples $\{x^{\alpha}\}_{\alpha=1}^N \sim p_\theta(x)$. 
Second, we determine the reweighted expectation value truncated at
first order
\begin{equation}
  \frac{\sum w(x^{\alpha};\tilde\theta) \left[ (x^{\alpha}_1)^2+ (x_2^{\alpha})^2 \right]}{\sum w(x^{\alpha};\tilde \theta)} \approx \bar O + \bar O_i \epsilon_i\,,  
  \qquad \left( w(x^{\alpha};\theta) = e^{S(x^{\alpha};\theta) - S(x^{\alpha};\tilde \theta)}  \right)\,,
\end{equation}
where $\tilde \theta_i =  \theta_i + \epsilon_i$. 
This quantity gives an stochastic estimate of the function value
\begin{equation}
  \bar O = \frac{1}{N}\sum_{i=1}^N [x_1^{\alpha}]^2 + [x_2^{\alpha}]^2\,,
\end{equation}
and its derivatives
\begin{equation}
 \bar O_i \approx \frac{\partial \mathbb E_\theta[x_1^2+x_2^2]}{\partial \theta_i}\,.
\end{equation}

Figure~\ref{fig:sgd} shows the result of the optimization process. 
As the iteration count increases the function is driven to its minima,
while the values of the parameters approach the optimal values
$\theta_1^{\rm opt} = \theta_2^{\rm opt} = 0$. 

It is worth mentioning that in this particular example only $1000$ samples
were used at each step to estimate the loss function and its
derivatives. 
If one decides to use a larger number of samples (say $10^5$), the
value of the parameter $\theta_2$ is determined with a much better precision. 
Note that the direction associated
with $\theta_2$ is much flatter, and therefore its value affects much
less value of the loss function.

\subsection{An application in Bayesian inference}
\label{sec:bayesian}
The purpose of statistical inference is to determine properties of the
underlying statistical distribution of a dataset
$D=\{x_{i},y_{i}\}_{i=1}^{N}$. In many
  cases, the independent variables $x_i$ are fixed, and all the
  stochasticity is captured by the dependent variables $y_i$. As
  such,  
the data is assumed to be sampled from a certain model, specified by
the \textit{likelihood}, 
$p(y|x,\phi)$, which depends on a set of parameters $\phi$.
The Bayesian paradigm attributes a level of confidence to the model by
introducing the \textit{prior} 
$p_{\theta}(\phi)$, \textit{i.e.} an a priori distribution of the
models parameters, where in this context $\theta$ play the role of the
hyper-parameters specifying the prior. Following Bayes' rule, the
\textit{posterior} distribution $p_{\theta}(\phi|D)$ is computed
as\footnote{The normalization factor, 
  $p_{\theta}(D)$, called the evidence, or marginal likelihood, is $\phi$-independent
  and represents the probability distribution of the observed data, given the model.}:
\begin{equation}
  \label{eq:bayes}
  p_{\theta}(\phi|D) \propto p(D|\phi) p_{\theta}(\phi)~.
%  ~~~~
%  p(D|\phi) = \prod_{i=1}^N p(y_i|x_i,\phi)~
\end{equation}
The likelihood of the whole dataset, $p(D|\phi)$, is computed assuming independent data points following a Gaussian distribution:
\begin{equation}
  p(D|\phi) = \prod_{i=1}^{N}\mathcal N(y_{i}|f(x_{i};\phi),\sigma_{i})\,,
  \label{eq:likelihood}
\end{equation}
where $\sigma_i$ are the  uncertainties of the corresponding observations $y_i$ (and assumed here to be given), while the mean of the Gaussian is given by $f(x_i;\phi)$. 
% The posterior in expr.(\ref{eq:bayes}) is the distribution of $\phi$ given the observed data and assumptions.
From a practical standpoint, in addition to the normalization being,
in general, unknown, the usual complexity of the posterior
distribution makes this possibly highly dimensional integral difficult
to compute. The use of Monte Carlo techniques, in particular of the
HMC, is typical in this context.
We focus below on two types of predictions: 1) The variance of the
model parameters $\delta\phi_j^2 = \mathbb{E}_{p_\theta}[\phi_j^2] -
(\mathbb{E}_{p_\theta}[\phi_j])^2$, where $j=1,...,d$, being $d$ the
dimension of $\phi$, and 2) the variance of the output mean $\delta
f_t^2 = \mathbb{E}_{p_\theta}[f_t^2] -
(\mathbb{E}_{p_\theta}[f_t])^2$, where $f_t$ is a shorthand notation
for the output mean $f(x_t;\phi)$, evaluated at a new ``test''
datapoint $x_t$ \footnote{Note that $E_{p_\theta}[f_t]$ is analogous
  to the so-called ``predictive distribution'' of Bayesian inference,
  however here we focus on the expected value of the prediction mean,
  instead of the expected value of the likelihood of $y(x_t)$
  itself.}. 


We are interested in studying the dependence of these quantities on the choice of
hyperparameters $\theta$ that characterize the prior distributions. In
particular we will consider the case of Gaussian priors, and determine
the dependence of our predictions with the width of this Gaussian.

\subsubsection{Model and data set}

We generate a synthetic dataset (cf. Figure \ref{fig:dataset}) by defining the points on an irregular grid in the range $x_i\in[-1.0;1.0]$, such that
\begin{equation}
  y_i = f(x_i;\phi_{\rm true}) + \sigma_i\epsilon~,
\end{equation}
where the mean is a 3rd degree polynomial, $f(x;\phi)=\phi_0+\phi_1x + \phi_2x^2 + \phi_3x^3$, with $\phi_{\rm true} = (1,1,1,1)$; $\epsilon\sim{\cal N}(0,1)$ is sampled from a standard Gaussian, and we consider a heteroscedastic dataset by defining a noise $\sigma_i$ dependent on $x_i$. We adopt the same model in order to  make inference on the parameters $\phi$. 

% Figure environment removed

%The likelihood reads
%\begin{equation}
%  p(D|\phi) = \prod_{i=1}^{N}\mathcal N(y_{i}|f(x_{i},\phi),\sigma_{i})\,,
%\end{equation}
%where $\mathcal N(\mu|\sigma)$ is the usual Gaussian distribution of
%mean $\mu$ and variance $\sigma^2$. As a model we choose a third
%degree polynomial $f(x,\phi) = \phi_{0} + \phi_{1}x + \phi_{2}x^{2}+\phi_{3}x^{3}$. 
%Note that this is the model that was also used to obtain the dataset.

The prior distribution is also chosen as a Gaussian, $\phi\sim {\cal N}(\mu_p,\sigma_p)$. 
For simplicity we choose the priors centered on the ``correct'' values
of the model (i.e. $\mu_p =\phi_{\rm true}$), while we keep the width 
$\sigma_p$ as a hyperparameter to study the dependence on\footnote{This is a simplified setup for the sake of illustration, given the methodological scope of this work. Nonetheless, it is straightforward to apply the method to the situation where we are interested in studying the dependence on both parameters $\mu_p$ and $\sigma_p$ simultaneously, or in general on the joint set of hyperparameters of the model. }.

For any choice of the prior width $\sigma_p$ we can obtain a prediction by
generating $N$ samples $\{\phi^{(\alpha)}\}^N_{\alpha=1}$ according to the distribution
$p_{\theta}(\phi|D)$ computed from \cref{eq:bayes}.  
  
\subsubsection{Reweighting approach}
\label{sec:bayesianhmc}

The reweighting method takes $N$ samples
$\{\phi_{i}^{({\alpha})}\}_{\alpha=1}^{N}$ obtained at
$\sigma_{p}=\sigma_{p}^{*}$ and computes the reweighted average using
$\tilde\sigma_{p}=\sigma_{p}^{*}+\epsilon$ in \cref{eq:rw}.  

For each sample $\phi^{(\alpha)}$, the reweighting factor becomes a polynomial expansion in
$(\sigma-\sigma_{p}^{*})$  
\begin{equation}
  \label{eq:rw bi}
  \tilde w_{\alpha}(\epsilon) = \frac{p_{\mu,\sigma_{p}^{*}+\epsilon}(\phi_{\alpha}|D)}{p_{\mu,\sigma_{p}^{*}}(\phi_{\alpha}|D)}.
\end{equation}
Notice that the zeroth order of \cref{eq:rw bi} is one, such that the zeroth order result corresponds to the usual Monte Carlo point estimate for $\delta\phi_{0}(\sigma_{p}^{*})$.

In order to generate these samples, we used the standard HMC algorithm. 
The equations of motion are
\begin{align}
  &H_{\theta}(\phi,\pi) = \frac{\pi^{2}}{2} - \log(p_{\theta}(\phi|D)),\\
	&\dot\phi_{j} = \pi_{j},\\
	&\dot \pi_{j} = - \frac{1}{\sigma_{p}^{2}}(\phi_{j} - (\mu_p)_j) + \sum_{i=0}^{N}\frac{1}{\sigma_{i}^{2}}\left( y_{i} - f(x_{i},\phi) \right)(x_{i})^{j},
\end{align}
where $\pi=\{\pi_{0},\pi_{1},\pi_{2},\pi_{3}\}$ are the momenta conjugated to $\phi$.
Note that all $\phi$-independent terms can be dropped from the
equations of motion, namely the normalization of $p_{\theta}(\phi|D)$
is not needed.
The eom were solved numerically using a fourth-order symplectic
integrator \cite{OMELYAN2003272} providing a high acceptance rate in
the Metropolis-Hastings step even with a coarse integration.  

The chosen integration step-size was $\varepsilon = 0.001$, while the
trajectory length was uniformly sampled in the interval $[0,100]\times
\varepsilon$\footnote{Due to the quadratic form of the Hamiltonian,
  the phase space of this system is cyclic. The algorithm is ergodic
  only if the trajectory length is
  randomized \cite{RHMC2017}.}.
%Taking into account the conclusions from \cref{sec:nspt}, the average trajectory length is approximately tuned such that the variance is minimized.  

% In the following, all of the Monte Carlo chains correspond to half million thermalized trajectories.

%All the predictions are a function of the hyperparameter $\sigma^*$ and
%we would, generically, be interested in this dependence.
%As for the quantity to study we focus on the uncertainty of the
%average value for $\phi_{i}$, $\delta\phi_{i} =
%\mathbb{E}_{p_{\theta}}[\phi_{i}^{2}]-\mathbb{E}_{p_{\theta}}[\phi_{i}]^{2}$,
%and analogously the uncertainty for the prediction of a new point
%$x_{n}$.  

\subsubsection{Hamiltonian perturbative expansion}

Following the procedure in \cref{sec:nspt}, the Monte Carlo samples
$\{(\tilde\phi_{j})^{\alpha}\}_{\alpha=1}^{N},~j=0,1,2,3$ were
obtained with the modified HMC algorithm for some values of
$\sigma_{p}^{*}$. 
We used the same parameters for the HMC as described in the previous
section. In particular our acceptances were so close to 100\% that any
bias due to the missing accept/reject step is negligible. 
We checked this hypothesis by further performing another simulation
with a coarser value of the integration step and finding completely
compatible results.


\subsubsection{Results}

\begin{table}[t]
  \centering
  \caption{Results for the expansion coefficients of the variance,
    ${\delta\phi^{2}_{j,n}}$ for $\sigma_{p}^{*}=0.3$
    from the reweighting and hamiltonian expansion.}
\scalebox{0.9}{
  \begin{tabular}{cccccccc}
	% \toprule
     & & \multicolumn{6}{c}{$n$} \\\cmidrule{3-8}
     & & 0 & 1 & 2 & 3 & 4 & 5  \\
    \midrule
\multirow{2}{*}{$\delta\phi^{2}_{0,n}$} & RW &    0.00014705(86) &    0.0001384(63) &    -0.000248(29) &     0.000367(62) &     -0.00071(51) &      -0.0003(12)  \\
                  & HAD &   0.00014705(86) &    0.0001365(34) &   -0.0002850(60) &     0.000311(20) &     0.000178(77) &     -0.00115(26)  \\

    \midrule
\multirow{2}{*}{$\delta\phi^{2}_{1,n}$} & RW &     0.01099(15) &       0.0285(12) &      -0.0450(58) &        0.032(13) &         0.04(10) &        -0.61(25)  \\
                  & HAD &     0.01099(15) &      0.02787(69) &      -0.0518(11) &       0.0248(38) &        0.189(16) &       -0.700(46)  \\
    \midrule
\multirow{2}{*}{$\delta\phi^{2}_{2,n}$} & RW &      0.008938(74) &      0.00830(28) &      -0.0283(10) &       0.0850(39) &       -0.234(18) &        0.603(78) \\
                  & HAD &     0.008938(74) &      0.00817(15) &     -0.02789(42) &       0.0849(13) &      -0.2505(44) &        0.726(15)  \\
    \midrule
\multirow{2}{*}{$\delta\phi^{2}_{3,n}$} & RW &     0.03617(59) &      0.1205(51) &       -0.182(24) &        0.050(61) &         0.63(42) &         -4.0(12)  \\
                  & HAD &     0.03617(59) &       0.1177(30) &      -0.2052(42) &        0.020(16) &        1.132(66) &        -4.02(19)  \\
    \bottomrule
    \label{tab:variance phi0}
  \end{tabular}
  }
\end{table}

Here we compare the predictions for the average model parameters
$\phi$ and their dependence on the prior width $\sigma$. In particular
we focus on the variance of the model parameters $\delta\phi^{2}_j$,
since these are the quantities most sentitive to the prior width (i.e. 
very thin priors result in small variance for the model
parameters). We have fixed $\sigma^* = 0.3$, but similar conclusions
are obtained for other values.  

The results of the Monte Carlo average for $\delta\tilde\phi^{2}_i$ and
its derivatives with respect to $\sigma$ are
shown in \cref{tab:variance phi0}. 
Results labeled ``RW'' use the reweighting method, while results
labeled ``HAD'' use the Hamiltonian approach. 

It is obvious that results using the Hamiltonian approach are more
precise:
the uncertainties in the derivatives, $\delta\phi^{2}_{i,n},n\neq 0$, are
smaller for the Hamiltonian approach, despite the statistics being the
same. The difference is larger for higher order derivatives: the
approach based on reweighting struggles to get a signal for the fourth
and fifth derivatives, while the Hamiltonian approach is able to
obtain even the fifth derivative with a few percent precision. 
This fits our expectations (see section~\ref{sec:hamilt-appr-repar}). 
\noindent\newline\newline
On the other hand, for our second quantity of analysis $\delta f_t^2$ (i.e. the variance of the prediction mean), Figure~\ref{fig:ypred} shows the results of the dependence on $\sigma_p$, where we have fixed $x_t=0.5$.

%dependence of the variance of the
%parameter prediction
%\begin{equation}
%  y_{\text{pred}}(x_{n})=\mathbb{E}_{p_{\theta}}[f(x_{n},\phi)] = \int d\phi p_{\theta}(\phi|D) f(x,\phi).
%\end{equation}
%at $x = 0.5$ with respect to $\sigma_P$.
The Hamiltonian approach gives visually results with a reduced variance,
similar to the results presented in table~\ref{tab:variance phi0}.

% Figure environment removed



%%% Local Variables:
%%% mode: latex
%%% TeX-master: "paper"
%%% End:

\section{Evaluation} \label{sec:evaluation}

\begin{table*}[tbp]
\centering
\small
\begin{tabular}{cccccccccc}
\toprule
& \multicolumn{3}{c}{\msr} & \multicolumn{3}{c}{\negc} & \multicolumn{3}{c}{\wsj} \\
& Acc. & F1 & wF1 & Acc. & F1 & wF1 & Acc. & F1 & wF1 \\ \cmidrule(lr){2-4} \cmidrule(lr){5-7} \cmidrule(lr){8-10} 
\udel & 66.86 & 56.76 & 64.3 & \textbf{80.80} & 55.45 & 77.9 & 63.74 & 64.23 & 63.2 \\
\icsi & \underline{71.19} & 64.73 & 70.4 & 80.36 & 64.53 & \underline{78.6} & 64.62 & 64.15 & 63.4 \\
\cnts & 68.59 & 61.39 & 67.2 & 78.68 & 61.62 & 76.8 & 64.31 & 64.59 & 64.4 \\
\osu & 68.02 & 60.28 & 66.6 & 79.24 & 57.04 & 76.5 & 69.20 & 69.63 & 68.9 \\
\isg & 67.05 & 58.83 & 65.3 & 77.34 & 59.52 & 75.6 & 69.15 & 69.35 & 69.2 \\ \midrule
\bert & \textbf{71.68} & \underline{66.70} & \textbf{71.4} & 77.79 & \underline{72.87} & 77.7 & \underline{80.95} & \underline{80.93} & \underline{80.9} \\
\roberta & 70.91 & \textbf{67.53} & \underline{70.7} & \textbf{80.80} & \textbf{77.29} & \textbf{80.7} & \textbf{82.61} & \textbf{82.70} & \textbf{82.6} \\ \midrule
Average & 69.19 & 62.32 & 67.99 & 79.29 & 64.05 & 77.69 & 70.65 & 70.80 & 70.37 \\
\bottomrule
\end{tabular}
\caption{\label{tab:performance} Overall accuracy (Acc.), macro-averaged F1 (F1), and weighted-macro F1 (wF1) scores of the algorithms depicted in Section~\ref{sec:algorithm}. For instance, \msr-\udel refers to a C5.0 classifier trained on the \msr~corpus, using the feature set mentioned in \citet{greenbacker-mccoy-2009-udel}.}
%Its Acc., F1 and wF1 of this model are 66.86, 56.76, and 64.3, respectively.}
\end{table*}


In this section, we introduce the evaluation protocol and report the performance of the models.

\subsection{Implementation Details} \label{sec:implementation}

For \bert and \roberta, we used \textit{bert-base-cased} and \textit{roberta-base}, both from Hugging Face. For fine-tuning, we set the batch size to 16, the learning rate to 1e-3, the dropout rate to 0.5, and the size of the output layer to 256. We ran each model for 20 epochs and used the one that achieved the highest F1 score on the development set. The implementation details of the classic ML-based models can be found in Appendix~\ref{sec:appendixML}.

\subsection{Evaluation Protocol} \label{sec:protocol}

The main evaluation metric in the GREC-MSR shared tasks was accuracy. 
In addition to accuracy, we also report macro-F1 and weighted-macro F1. We argue that different metrics evaluate algorithms from different perspectives and provide us with different meaningful insights. 
For pragmatic tasks like REG, it makes sense to ask how well an algorithm performs on naturally distributed data which is often imbalanced. For these cases, reporting accuracy and weighted F1 are logical. 
Furthermore, analogous to other classification tasks, minority categories should not be overlooked. Take as an example the class \emph{description} in the \negc corpus, which occurs only 4\%. If a model fails to produce this class, the produced document might sound unnatural. Therefore, it is important to ensure that an algorithm is not over- or under-generating certain classes. Looking into accuracy and macro-F1 together provides insights into such cases.

\subsection{Performance of the Models}\label{subsec:overallacc}

The overall accuracy of the models, their macro F1, and their weighted-macro F1 are presented in Table \ref{tab:performance}. 
We also present the ranking of the models based on these scores in Appendix~\ref{sec:app_rank}. 


\paragraph{PLM-based Models.} The best-performing models across all corpora and metrics are PLM-based models.  In six out of nine rankings, \bert and \roberta are ranked as the top two models. The sole exception is \negc, where \bert is the second worst model. The benefit of using PLMs is the largest on the \wsj corpus. For example, \roberta improves the macro F1 score from 69.63 (i.e., the performance of the best ML-based model) to 82.70.


\paragraph{ML-based Models.} In contrast to the robust performance of the PLM models, the performance of the classic ML models is more corpus-dependent. In the case of \msr and \negc, \icsi is the best-performing model, while in the case of \wsj, it is at the bottom section of the rankings. Another interesting observation is the performance of the \udel models. In terms of accuracy, \udel has the highest performance in \negc, while it has the lowest performance in both \msr and \wsj. In terms of macro-F1 rankings, the \negc \udel model dropped from first to last place, whereas \bert improved from penultimate place to second place. In general, our ML models yielded lower scores than the original models used in the GREC study \citep{belz2009generating}. This could be attributed to a variety of factors, including differences in feature engineering and model parameters.

\paragraph{Comparing Different Metrics.} 

Upon comparing average scores across the three metrics, we observe that for \msr and \negc, PLMs are clear winners only when macro-F1 is the metric in question. However, for \wsj, PLMs are winners on all three metrics. This may be because the distribution of categories in \wsj is much more balanced than in the other two corpora.
\section{Conclusion and Future Work}
In this work, I design corruption-robust algorithms for the Lipschitz contextual search problem. I present the \emph{agnostic checking} technique and demonstrate its effectiveness in designing corruption-robust algorithms. There are several open problems for future research. First, in the algorithm I propose for pricing loss, the schedule for agnostic checks is fixed upfront. Can the learner design an adaptive checking schedule for the pricing loss? Second, this work assumes the learner has knowledge of the Lipschitz constant $L$. Can the learner design efficient no-regret algorithms without knowledge of $L$? 

\acknowledgments{
The authors wish to acknowledge the support from NSFC under Grants (No. 61802128 and 62072183) and the Shanghai Committee of Science and Technology, China (Grant No. 22511104600).
}

\bibliographystyle{abbrv-doi-hyperref}
%\bibliographystyle{abbrv-doi-hyperref-narrow}
%\bibliographystyle{abbrv-doi}
%\bibliographystyle{abbrv-doi-narrow}

\bibliography{template}


%% ^^^^^   FOR IEEE VIS, EVERYTHING HERE MAY BE INCLUDED IN THE    ^^^^^ %%
%% 2-PAGE ALLOTMENT FOR REFERENCES, FIGURE CREDITS, AND ACKNOWLEDGEMENTS %%

\appendix % You can use the `hideappendix` class option to skip everything after \appendix


\end{document}

