\section{Evaluation}
We evaluate our method from two aspects, steganography quality and steganography capacity,
which are two of the most important criteria for information steganography.
We conduct our experiments on a PC with an Intel Core i7 CPU, an NVIDIA GeForce 3090 
GPU, and 64 GB of memory. Our model is implemented with PyTorch \cite{paszke2019pytorch}.
We use 32 ACBs in our ISN and train our model using the Adam optimizer \cite{Kingma2014AdamAM}. 
The weight coefficients of the loss function (\autoref{eq:totloss}) are set as $\alpha = 0.5$ and $\beta = 1.6$
to balance the data embedding quality and data restoration accuracy.
In addition, as mentioned in \autoref{sec:dataset}, our experiments are conducted on our
test dataset.


\subsection{Steganography Quality}
\label{sec:steg_q}
Steganography quality refers to both the perceptual quality of the encoded image and the 
accuracy of restored information. These two criteria are inversely related to each
other, which means higher embedding quality generally leads to lower data restoration
capacity and vice versa. A good steganography method should achieve a satisfactory 
trade-off between them. 

We evaluate the quality of encoded image using three metrics, which are the peak 
signal-to-noise ratio (PSNR) \cite{almohammad2010stego}, the structural similarity 
index (SSIM) \cite{wang2004image} and the learned perceptual image patch similarity 
(LPIPS) \cite{zhang2018unreasonable}. The PSNR is often used to evaluate the
distortion of images, and the SSIM measures the structural similarity between two 
images. The LPIPS is based on the high-level features of images, which represent 
the perceptual similarity between two images. In our experiments, we use VGG 
\cite{DBLP:journals/corr/SimonyanZ14a} as the feature extraction network for the calculation of LPIPS.
Higher PSNR, SSIM and lower LPIPS is better.

For the data restoration accuracy, we use root mean squared error (RMSE) to measure 
the difference between the initial data images and the restored data images. In addition, we 
use the text recovery accuracy (TRA) \cite{zhang2020viscode}, which refers to the 
proportion of the restored characters among all input characters, to measure the 
performance of QR Code image recovery. Lower RMSE and higher TRA is better.

% Figure environment removed

% % Figure environment removed


We first conduct a series of ablation experiments to investigate the impact of FFB and 
wavelet domain steganography on the final perceptual quality of the encoded image. We hide 3 
channels of $512 \times 512$ data images and one channel of $360 \times 360$ QR Code image 
into each carrier image. Each QR Code image is encoded with $T$ characters and 
$T \in [1, 1200]$. As shown in \autoref{tab:ablation}, our full model achieves
the best balance between information embedding quality and data restoration accuracy.\vspace{-7pt}

\begin{table}[htb]
\caption{Ablation experiment results of different metrics}
\vspace{-8pt}
\newcolumntype{M}[1]{>{\centering\arraybackslash}m{#1}}
\renewcommand\arraystretch{1.2}
\centering
\small
\begin{tabular}{@{}M{4.0cm}M|M{0.9cm}M{0.9cm}M{0.9cm}|M{0.9cm}M{0.9cm}|@{}}
\bottomrule
\multicolumn{2}{|c|}{Method} & {PSNR$\uparrow$} & {SSIM$\uparrow$} & {LPIPS$\downarrow$} & {RMSE$\downarrow$} & {TRA$\uparrow$} \\ [0.5pt]
\hline
\multicolumn{2}{|c|}{Ours w/o wavlet} & {$39.1858$} & {$0.9755$} & {$0.0572$} & {$0.0073$} & {$0.4517$} \\
\multicolumn{2}{|c|}{Ours w/o FFB}    & {$41.6315$} & {$0.9681$} & {$0.0566$} & {$0.0066$} & {$0.9916$} \\
\hline
\multicolumn{2}{|c|}{\textbf{Ours}} & \textbf{44.6889} & \textbf{0.9830} & \textbf{0.0120} & \textbf{0.0062} & \textbf{0.9892} \\ [0.5pt]
\toprule
\end{tabular}
\label{tab:ablation}
\end{table}\vspace{-8pt}

In addition, we experiment on how the linear interpolation discussed in \ref{sec:dtoi_discrete} 
affects the steganography quality. We regenerate the scatter data images in our dataset with different 
values of $K$ and retrain the model, and then we evaluate it on the corresponding test set 
(contains scatter data images only). The result 
is shown in \autoref{tab:linear}. As we can see, larger $K$ can lead to better steganography quality 
at the cost of lower steganography capacity.


\begin{table}[htb]\vspace{-7pt}
\caption{Steganography quality under different values of $K$}\vspace{-9pt}
\newcolumntype{M}[1]{>{\centering\arraybackslash}m{#1}}
\renewcommand\arraystretch{1.2}
\centering
\small
\begin{tabular}{@{}M{4.0cm}M|M{1.0cm}M{1.0cm}M{1.0cm}|M{1.0cm}M{1.0cm}|@{}}
\bottomrule
\multicolumn{2}{|c|}{K value} & {PSNR$\uparrow$} & {SSIM$\uparrow$} & {LPIPS$\downarrow$} & {RMSE$\downarrow$} & {TRA$\uparrow$} \\ [0.5pt]
\hline
\multicolumn{2}{|c|}{0} & {$42.3589$} & {$0.9724$} & {$0.0379$} & {$0.0241$} & {$0.9875$} \\
\multicolumn{2}{|c|}{1} & {$43.2026$} & {$0.9776$} & {$0.0258$} & {$0.0202$} & {$0.9918$} \\
\multicolumn{2}{|c|}{3} & {$45.5204$} & {$0.9880$} & {$0.0061$} & {$0.0090$} & {$1.0000$} \\
\multicolumn{2}{|c|}{7} & {$46.3686$} & {$0.9908$} & {$0.0028$} & {$0.0050$} & {$1.0000$} \\ [0.5pt]
\toprule
\end{tabular}
\vspace{-7pt}
\label{tab:linear}
\end{table}

We also compare our model with other image steganography methods, i.e. HiNet 
\cite{jing2021hinet} and IICNet \cite{cheng2021iicnet}. We hide 2 channels of 
$512 \times 512$ data images and one channel of $360 \times 360$ QR Code image 
into each carrier image. \autoref{fig:encode_compare} shows a comparison between 
the encoded image generated by our method with Hinet and IICNet. The magnified area 
and its corresponding residual (enhanced by 4 times), along with the PSNR and SSIM 
of the encoded images are shown. As we can see, InvVis can generate encoded images 
that are more perceptually similar to the carrier image
and has less distortion. The results of the comparative experiments are shown in 
\autoref{tab:comparision}, in which we can see that our model achieves the best performance
on both steganography quality and data restoration accuracy.\vspace{-6pt}


\begin{table}[htb]
 \vspace{-1pt}
\caption{Comparision of our model with other methods}
\vspace{-6pt}
\newcolumntype{M}[1]{>{\centering\arraybackslash}m{#1}}
\renewcommand\arraystretch{1.2}
\centering
\small
\begin{tabular}{@{}M{4.0cm}M|M{1.0cm}M{1.0cm}M{1.0cm}|M{1.0cm}M{1.0cm}|@{}}
\bottomrule
\multicolumn{2}{|c|}{Method} & {PSNR$\uparrow$} & {SSIM$\uparrow$} & {LPIPS$\downarrow$} & {RMSE$\downarrow$} & {TRA$\uparrow$} \\ [0.5pt]
\hline
\multicolumn{2}{|c|}{HiNet \cite{jing2021hinet}} & {$38.9608$} & {$0.9377$} & {$0.0563$} & {$0.0188$} & {$0.9360$} \\
\multicolumn{2}{|c|}{IICNet \cite{cheng2021iicnet}} & {$38.7422$} & {$0.9276$} & {$0.0474$} & {$0.0105$} & {$0.6069$} \\
\hline
\multicolumn{2}{|c|}{\textbf{Ours}} & \textbf{45.5246} & \textbf{0.9866} & \textbf{0.0090} & \textbf{0.0062} & \textbf{0.9961} \\ [0.5pt]
\toprule
\end{tabular}
\vspace{-13pt}
\label{tab:comparision}
\end{table}

\subsection{Steganography Capacity}
\label{eva: cap}
The capacity of steganography concerns how much chart information can be concealed in the image, 
which is crucial for the invertible visualization of charts with large amounts of data. There
is also a trade-off between steganography capacity and quality, larger steganography capacity
can lead to lower quality of the encoded image.

To measure the steganography capacity of different methods, we use bits per pixel (BPP),
which indicates the average number of bits hidden in the encoded image per pixel, as the metric.
Formally, given a carrier image $I$ whose size is $(C, H, W)$, its BPP is calculated as:\vspace{-7pt}
\begin{equation}
    BPP(I) = \frac{L}{C \times H \times W}
    ,\vspace{-5pt}
\end{equation}
where $L$ is the total number of bits hidden in $I$.

We compare our method with VisCode \cite{zhang2020viscode}. VisCode uses a region proposal algorithm 
based on saliency detection to choose the places to hide QR Code images. Although this method can 
achieve better image encoding quality than using predetermined regions for data embedding, it cannot 
guarantee that the image is maximally utilized to hide data. Also, if the encoded data volume is large,
VisCode cannot achieve a satisfactory embedding quality. In contrast, our method concatenates the QR 
codes horizontally and vertically into QR Code images to maximize spatial utilization, and our method can achieve 
good steganography quality even if the data volume is
 large. 

We first compare the steganography capacity of our method with VisCode when only hiding QR Codes. Since 
our methods can not only hide QR Code image but also hide data images, we also compare the maximum 
steganography capacity of our method with VisCode. 
The result is shown in \autoref{tab:stegCap1}, as we can see, our method achieves a larger steganography
capacity than VisCode even when only hiding QR Code image, and our maximum steganography capacity far
exceeds that of VisCode. Moreover, the quality of the encoded image is also much better than that of VisCode
in both experiments. 
Besides, the average computational time of encoding and decoding, together with the
model size of VisCode and InvVis are also shown in \autoref{tab:stegCap1}. As we can see, InvVis has a smaller
model size and faster computational speed.

\begin{table}[htb]
\vspace{-4pt}
\caption{Steganography capacity compared with VisCode}
\vspace{-7pt}
\newcolumntype{M}[1]{>{\centering\arraybackslash}m{#1}}
\renewcommand\arraystretch{1.2}
\centering
\small
\begin{tabular}{|M{1.8cm}|M{0.75cm}|M{0.7cm}|M{0.65cm}M{0.65cm}M{0.65cm}|M{0.8cm}|@{}} 
\bottomrule
{Method} & {Time} & {Size} & {PSNR$\uparrow$} & {SSIM$\uparrow$} & {LPIPS$\downarrow$} & {BPP$\uparrow$} \\ [0.5pt]
\hline
VisCode \cite{zhang2020viscode} & {4.9/2.6} & {$335$mb} & {29.8098} & {0.9499} & {0.0682} & {0.0049} \\
\hline
Ours (QR only) & \multirow{2}{*}{\textbf{3.0/2.2}} & \multirow{2}{*}{\textbf{158mb}} & \textbf{45.1067} & \textbf{0.9851} & \textbf{0.0073} & \textbf{0.0218} \\
Ours (maximum) & & & \textbf{44.5082} & \textbf{0.9822} & \textbf{0.0131} & \textbf{5.5218} \\ [0.5pt]
\toprule
\end{tabular}
\vspace{-5pt}
\label{tab:stegCap1}
\end{table}
    
We also compare the steganography quality of our method under different BPP values.
We divide the BPP into three levels, which are low, medium and maximum. The 
BPP value of different levels and the corresponding steganography quality are shown in 
\autoref{tab:stegCap2}, as we can see, InvVis can maintain high steganography quality under
different levels of BPP.

\begin{table}[htb]
\vspace{-7pt}
\caption{Steganography quality under different BPP levels}
\vspace{-7pt}
\newcolumntype{M}[1]{>{\centering\arraybackslash}m{#1}}
\renewcommand\arraystretch{1.2}
\centering
\small
\begin{tabular}{@{}M{4.0cm}M|M{1.0cm}|M{1.2cm}M{1.2cm}M{1.2cm}|@{}}
\bottomrule
\multicolumn{2}{|c|}{BPP Level} & {BPP} & {PSNR$\uparrow$} & {SSIM$\uparrow$} & {LPIPS$\downarrow$} \\ [0.5pt]
\hline
\multicolumn{2}{|c|}{Low} & {0.5458} & {46.0429} & {0.9893} & {0.0051} \\
\multicolumn{2}{|c|}{Medium} & {2.7604} & {45.5163} & {0.9868} & {0.0104} \\
\multicolumn{2}{|c|}{Maximum} & {5.5218} & {44.5082} & {0.9822} & {0.0131} \\ [0.5pt]
\toprule
\end{tabular}\vspace{-8pt}
\label{tab:stegCap2}
\end{table}\vspace{-5pt}