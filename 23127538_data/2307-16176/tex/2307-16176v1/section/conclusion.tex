\section{Conclusion}
We propose InvVis, a new approach for embedding information into visualization images, which is suitable for cases 
with large amounts of data. Our InvVis can support various application scenarios like invertible visualization of
data-intensive charts, large-scale source code embedding, scientific data embedding, etc. We propose a new method 
to transfer chart data into images that can be more easily embedded in chart image with high quality. We also 
outline a deep learning-based pipeline to embed information into chart images. Our evaluation experiments show 
that InvVis can achieve high-quality information concealing and revealing with large steganography capacity.

The current version of InvVis has certain limitations in addressing invertible visualization problems with zero 
tolerance for errors. In addition, the steganography capacity of InvVis is related to the size of the carrier image,
it may not achieve a satisfactory embedding capacity if the image size is too small. Also, InvVis is mainly
designed to facilitate chart image transmission on the internet, so it lacks stability when facing real-world 
image interference such as printing and shooting.

In the future, we plan to propose a more widely applicable method that can handle more types of data and achieve 
better steganography performance and recovery accuracy while ensuring steganography capacity. Additionally, as 
shown in the work of Tancik et al. \cite{tancik2020stegastamp} and Fu et al. \cite{fu2022chartstamp}, the method 
can be more robust if it is stable against some steganography attacks, this is also a potential research direction.
