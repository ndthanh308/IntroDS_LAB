\section{Overview}
\label{sec:ove}

In the field of information visualization, chart plays a critical role in describing
and presenting data. Ideally, a well-designed chart should possess 
properties like user-friendly interactive methods and reasonable demonstration
of chart data. These features can improve the efficiency and accuracy for users to 
cognize and comprehend data, even when the data volume is large at times. Unfortunately,
in most cases, visualization charts are disseminated in the form of raster images,
forfeiting the aforementioned features. This greatly limits users' ability to 
explore and interact with the chart. To solve this problem, we aim to implement
invertible visualization by encoding the
metadata of the chart into the generated chart image using image steganography
techniques. 

In this paper, we propose InvVis, a new approach for invertible visualization.
Different from previous studies \cite{zhang2020viscode, fu2020chartem, fu2022chartstamp}, 
we focus not just on visualization charts with relatively limited amounts of metadata, 
but also on those with large data volume. \autoref{fig:pipline} shows the pipline
of our InvVis, which contains three main components:

\begin{itemize}

\item \emph{Data-to-Image Module}
To ensure that the chart data enters the concealing network in the form of images, we propose
a data-to-image (DTOI) module to implement the data-to-image transfer.
This module takes chart data as input and generates several data images representing the data
based on the data type. In addition, this module is invertible, which means the original data can be restored by performing the 
inverse operation of the DTOI module on these images. 

\item \emph{Concealing Network}
The concealing network embeds the metadata of the chart in the carrier image. At first,
data images and QR Code image are padded to the same size as the carrier image.
Then, these images, including the carrier image, are concatenated and fed into
the concealing network. The network produces an encoded image with the same 
resolution as the carrier image while introducing subtle perturbations.

\item \emph{Revealing Network}
The revealing network retrieves the metadata from an encoded image. The network
first decodes an encoded image to obtain the restored data images and QR Code image.
After performing corresponding operations on these images, the restored chart data 
and chart information are available to users. Users may use the restored data for 
various applications (e.g., redesigning the chart). 
    
\end{itemize}


% Figure environment removed


\subsection{Definition}
In our InvVis, the input comprises three parts, a carrier image $I_{c}$, 
a sequence of chart data $S$, and a plain text $T$ that represents the 
chart information. The DTOI module $Dtoi(\cdot)$ transfers the chart 
data to $K$ data images $I_{d}^{1:K}$. The chart information is encoded into
several QR Codes and stitched together into a QR Code image $I_{qr}$. These images
are fed into the concealing network $E(\cdot)$ and result in an encoded image $I_{e}$. 
In the revealing process, the revealing network $D(\cdot)$ takes an encoded image 
$I^{'}_{e}$ as input and outputs the restored data images $I_{d}^{'1:K}$ and QR Code 
image $I^{'}_{qr}$. After that, the restored chart data $S^{'}_{d}$ and chart information 
$T^{'}$ can be obtained by performing the reverse DTOI operation $Dtoi^{-1}(\cdot)$ and QR Code 
decoding operation separately. For our InvVis, we aim to make the encoded image perceptually 
similar to the original chart image, while ensuring that the decoded data is as accurate as 
possible. Formally, we aim to optimize the two networks $E(\cdot)$ and $D(\cdot)$ by minimizing:\vspace{-5pt}
\begin{equation}
    \left\| I_{c} - I_{e} \right\| + 
    \lambda _1 \sum_{k=1}^{K} \left\| I_{d}^{k} - I_{d}^{'k} \right\| +
    \lambda _2 \left\| I_{qr} - I_{qr}^{'} \right\|
    ,\vspace{-7pt}
\end{equation}
where $\lambda _1$ and $\lambda _2$ are weight coefficients.



