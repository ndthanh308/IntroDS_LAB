Information visualizations are designed to convey information effectively and efficiently. 
Nowadays, people usually face massive amounts of data. In such cases, the interactivity 
and operability of a well-designed visualization 
allow users to perceive information intuitively and quickly. However, these features are 
often based on certain applications or visualization tools. Typically, the dissemination 
of visualization charts is based on bitmap images. For example, some users may just take a 
screenshot of the visualization chart and share it with other people, this leads to the loss of these 
excellent properties. As a result, it would be meaningful and useful if the initial chart can be recovered
from a chart image. In this paper, we call this procedure as invertible visualization, which is concerned
with reconstructing or further modifying the visualization from an image. 

Some methods have been proposed to address this issue, some of them try to recover the data 
and visual elements using pattern recognition and object detection techniques \cite{poco2017extracting,song2022graphdecoder}. 
However, these methods generally cannot achieve enough recognition accuracy and have limited 
applicability. Some other studies have taken a different approach, which is leveraging 
information steganography, a technique of implicit information embedding. This kind of 
approach \cite{zhang2020viscode, fu2020chartem, fu2022chartstamp} hides
chart information in the image in an inconspicuous way and restores the chart by extracting 
the embedded data. These methods avoid the unstable procedure of identifying chart elements, 
but they cannot achieve a large steganography capacity while ensuring steganography quality, 
making it difficult to deal with charts with large amounts of data. 

Visualizations containing large amounts of data are ubiquitous, such as large-scale scatter 
plots, geographic data visualizations, etc. Actually, it is this kind of visualization charts 
that require more well-designed interactive methods and visual encoding to facilitate users' 
understanding and perception of data. However, previous methods cannot handle the invertible
visualization of this kind of charts due to their limitations mentioned above. Recently, the 
field of information steganography has made significant progress. Many studies have successfully 
hidden arbitrary data or natural images in images without causing significant visual 
differences. Especially, image steganography based on invertible neural network (INN) 
\cite{Dinh2014NICENI, kingma2018glow} has achieved impressive performance. Nevertheless, these methods focus on natural
images instead of chart images and chart data. In this paper, we attempt to address the invertible visualization 
problem of data-intensive charts.

We present a novel method for invertible visualization, called InvVis.
InvVis is an end-to-end pipeline that embeds chart information and chart data into chart images 
in the concealing process and the embedded data can be restored or further reused after the revealing 
process. InvVis can handle the invertible visualization of charts with large amounts of data, which 
is different from previous methods. We propose a data-to-image (DTOI) module which can transfer the
chart data into data images that can be hidden into images in a less perceivable way. We also 
outline a concealing and revealing network based on INN that handles the embedding and restoring of
information. The data images output by the DTOI module, together with the chart 
information that is encoded into QR Code \cite{qrcodeweb} image, are embedded into a chart image through the network. 
And these images can be recovered by the network and transferred back to chart data and chart information. 
The restored data can be used in various scenarios like redesigning the chart, reorganizing chart 
data, etc. 

We conduct a series of experiments and the results show that the encoded chart image generated by 
InvVis have better perceptual similarity compared to other methods, and meanwhile the recovered data 
has less difference. Also, our method can achieve a much larger steganography capacity than previous 
methods. In summary, our InvVis has great potential in addressing the problem of invertible 
visualization, especially in cases involving large amounts of data, where our method can handle the
problem that previous methods cannot. Our contributions include three aspects:

\begin{itemize}
\setlength{\itemsep}{0pt}
\setlength{\parsep}{0pt}
\setlength{\parskip}{0pt}
\item [(1)] {
    We define the problem of invertible visualization that involves large amounts of data. We present 
    various application scenarios of this problem and conduct corresponding explorations and implementations.
}
\item [(2)] {
    We propose a new method of transferring chart data into data images. The data images can effectively represent 
    various kinds of data.
}
\item [(3)] {
    We propose a deep learning-based pipeline to embed information into chart images. Our method can achieve 
    high-quality information concealing and revealing with large embedding capacity.
}
\end{itemize}


