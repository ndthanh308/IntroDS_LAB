\section{Related Work}
\label{sec:rel}

\subsection{Information Steganography}
Information steganography is a technique that conceals information in a carrier with
limited perceptual changes. The carrier can be various formats of data such as image, 
text, audio, video, etc. This technique has considerable applications, e.g., 
digital watermarking, copyright protection and secret communication. An information
steganography method is generally evaluated from three aspects: capacity, security 
and robustness. Capacity means the embedding payload. Security is concerned with the 
undetectability and imperceptibility of embedded data. Robustness refers to the ability 
against distortion. Delina et al. \cite{delina2008information} proposed a text steganography
method to generate steg-text dynamically according to content length and several user-decided 
options. Hota et al. \cite{hota2019embedding} outlined a pipeline to embed 
digital watermarks for scientific visualizations. Yang et al. \cite{yang20163d} embedded
watermarks in a 3D model by modifying its histogram and achieved a fine robustness/distortion 
trade-off. Delforouzi et al. \cite{delforouzi2008adaptive} hid encrypted data into 
the coefficients of audio in the integer wavelet domain while preserving high audio quality.
\cite{swanson1997multiresolution, zhu1999multiresolution} applied information steganography
to carrier videos.

In summary, although previous studies have explored various kinds of carriers, these
methods commonly have limited steganography capacity and are not specially designed for 
information visualization. In our study, a different approach is proposed to address
this problem.



\subsection{Image Steganography}
Image steganography hides data by performing imperceptible changes on a host image.
Traditional methods generally alter the image in spatial domain or transform domain. 
Spatial-domain embedding techniques like Least-Significant Bit (LSB) replacement 
\cite{mielikainen2006lsb} and Bit Plane Complexity Segmentation (BPCS) 
\cite{kawaguchi1999principles} modify pixel values subtly to carry information.
However, these algorithms fail to preserve the statistical properties of images and are
easily detected by steganalysis methods \cite{fridrich2001detecting, yu2004reliable}.
To address this problem, highly undetectable steganography (HUGO) \cite{pevny2010using} 
utilizes high-dimensional image features based on high-order Markov chains to 
improve algorithm security. Transform domain, e.g., discrete cosine transform 
(DCT) domain \cite{almohammad2008high} and discrete wavelet-transform (DWT) domain 
\cite{swanson1997multiresolution, zhu1999multiresolution}, can also be modfied to 
embed information. Although transform-domain embedding is generally more robust
than spatial-domain embedding, both of them can only hide limited bit-level information.

Recently, a variety of deep learning-based image steganography schemes have been 
proposed and have achieved impressive performance. 
\cite{zhu2018hidden,tancik2020stegastamp} adopted the autoencoder to embed 
binary messages in a cover image. \cite{zhang2019steganogan, qin2020coverless} 
used generative adversarial network (GAN) to optimize the distortion of encoded 
images. There were also various studies that attempted to hide one or more secret 
images into a carrier image. Baluja et al. \cite{baluja2017hiding} first utilized 
an end-to-end network to learn feature representations of color images and 
embed them into cover images. Wengrowski et al. \cite{wengrowski2019light} introduced
a photographic steganography algorithm based on the encoder-decoder network for light 
field messaging (LFM). Wu et al. \cite{wu2021embedding} proposed to embed multiplane 
images (MPIs) in a JPEG image and synthesize novel views by decoding it.
More recently, the invertible neural network (INN) \cite{kingma2018glow, Dinh2014NICENI} has been 
utilized to hide single \cite{jing2021hinet, lu2021large} or multiple 
\cite{cheng2021iicnet, guan2022deepmih} images into carrier images. 

Most of the aforementioned image steganography methods are designed to hide information 
in natural images. However, we focus on embedding data into chart images. Chart images 
typically comprise many homogeneous regions and have different features from natural images.
As a result, a different method is required to handle this problem.

% % Figure environment removed
% Figure environment removed

\subsection{Invertible Visualization}
Invertible visualization allows users to regain or further modify the visualization chart
from a raster image. The reconstructed chart preserves its initial information 
including chart type, chart data, interactive methods, visual mapping, etc.
This facilitates scenarios such as redesigning the chart, reorganizing 
chart data, helping visually impaired users to understand chart
\cite{choi2019visualizing, singh2021chartsight, poco2017reverse}, 
and enabling reproducibility
for scientific visualizations \cite{10.1145/1142473.1142574}.

Previous studies on invertible visualization mainly took two kinds of approaches.
First kind of methods aim to extract the information directly from the chart image.
Savva et al. \cite{savva2011revision} used machine learning techniques to 
infer chart type and underlying data. Poco et al. \cite{poco2017reverse} proposed a 
pipeline to automatically recover the visual encoding specification of a chart from 
a raster image. \cite{poco2017extracting} utilized a convolutional neural network (CNN) to
extract color mappings from visualization images. Later, some studies tried to
improve the extraction performance by introducing human interactions
\cite{mendez2016ivolver, flower2016validity} or focusing on certain types of charts 
\cite{al2017machine, Liu2019DataEF, song2022graphdecoder, chen2019towards}.
However, this kind of methods are still likely to suffer from insufficient 
accuracy or limited application scenarios due to the diversity and structural 
complexity of visualization charts. Another kind of methods embed the metadata of
charts in the generated visualization images and reconstruct the chart by
decoding the image. VisCode \cite{zhang2020viscode} trains an encoder-decoder 
network to embed and recover QR Codes which contain chart information. To optimize 
the encoding quality, VisCode leverages a visual importance network to guide the 
embedding process. Chartem \cite{fu2020chartem} detects the background of chart images and 
hides metadata by slightly modifying background pixel values. ChartStamp 
\cite{fu2022chartstamp} utilizes a distortion layer that simulates real-world image 
manipulations to improve the robustness. This kind of methods have addressed the 
abovementioned problem to some extent at a cost of the perceptual quality of chart images. 
However, to our best knowledge, few previous study has addressed the issue
of invertible visualization when the data volume is large. In this paper,
we mainly focus on embedding and restoring large amounts of data in the 
context of information visualization, and we present a new method to implement 
invertible visualization.
