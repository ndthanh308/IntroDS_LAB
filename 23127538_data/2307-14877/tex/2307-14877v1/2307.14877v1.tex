% !TeX spellcheck = en_GB
\documentclass[12pt]{amsart}
\linespread{1.20}
\usepackage[T1]{fontenc}
\usepackage{latexsym} 
\usepackage{amsfonts, amsthm, amsmath,amssymb}
\usepackage{BOONDOX-calo}
\usepackage{geometry}
\geometry{
	a4paper,
	total={160mm,247mm},
	left=25mm,
	top=25mm,
}
\usepackage{times}
\usepackage{xcolor}
\usepackage{slashed}
%%\usepackage{showkeys}
\definecolor{refkey}{rgb}{0,0,1}
\definecolor{labelkey}{rgb}{1,0,0}
\usepackage{graphicx}
\usepackage{float}
\usepackage{mdframed}
\usepackage{ulem}
\usepackage{comment}
%%%%%%%%%%%%%%%%%%%%%%%%%%%%%%%%%%%%%
\definecolor{darkblue}{rgb}{0.0, 0.0, 0.55}
\definecolor{darkcerulean}{rgb}{0.03, 0.27, 0.49}
\definecolor{darkpowderblue}{rgb}{0.0, 0.2, 0.6}
\definecolor{britishracinggreen}{rgb}{0.0, 0.26, 0.15}
\newenvironment{blu}{\color{darkpowderblue}}{}
\newenvironment{mgg}{\color{magenta}}{}
\newenvironment{brg}{\color{britishracinggreen}}{}
\newenvironment{red}{\color{red}}{}
\newcommand{\bre}{\begin{red}}
	\newcommand{\ere}{\end{red}}
\newcommand{\bas}{\begin{brg}}
	\newcommand{\eas}{\end{brg}}
\newcommand{\bblu}{\begin{blu}}
	\newcommand{\eblu}{\end{blu}}
\newcommand{\bmag}{\begin{mgg}}
	\newcommand{\emag}{\end{mgg}}
\providecommand{\dd}{\mathrm{d}}
%%%%%%%%%%%%%%%%%%%%%%%%%%%%%%%%%%%%%%%%%%%%%%%%%%%%%%%%%%%%%%%%
\newcommand{\cmt}[1]{\begin{mdframed}[backgroundcolor=orange!20] #1 \end{mdframed}}
\newcommand{\cmtd}[1]{\begin{mdframed}[backgroundcolor=red!20] \bf #1 \end{mdframed}}
\NewDocumentCommand{\colorrule}{O{.4pt}m}{{\color{#2}\hrule height#1}\vspace{4mm}}
%%%%%%%%%%%%%%%%%%%%%%%%%%%%%%%%%%%%%%%%%%%%%%%%%%%%%%%%%%%%%%%%
\newtheorem{thm}{Theorem}[section]
\newtheorem{prop}[thm]{Proposition}
\newtheorem{conj}[thm]{Conjecture}
\newtheorem{lem}[thm]{Lemma}
\newtheorem{coro}[thm]{Corollary}
\theoremstyle{definition}
\newtheorem{defn}[thm]{Definition}
\newtheorem{rem}[thm]{Remark}
\newtheorem{exa}[thm]{Example}
\numberwithin{equation}{section}
%%%%%%%%%%%%%%%%%%%%%%%%%%%%%%%%%%%%%%%%%%%%%%%%
\def\tens{\mathop{\otimes}} 
\def\proof{{\sl Proof.~~}} 
\def\eps{\varepsilon} 
\def\cH{{\mathcal H}} 
\def\cA{{\mathcal A}}
\def\cB{{\mathcal B}}
\def\beq{\begin{equation}} 
	\def\eeq{\end{equation}} 
\def\fp{{\mathfrak p}} 
\def\ot{{\otimes}} 
\def\wres{\mathcal{W\!}}
\def\wresd{\mathcal{w}}
\def\tr{\hbox{tr}}
\def\Tr{\hbox{Tr}}
%%%%%%%%%%%%%% --cal + math blackboard
\def\cL{{\mathcal L}}
\def\IC{\mathbb{C}}
\def\IR{\mathbb{R}}
\def\IN{\mathbb{N}}
\def\IZ{\mathbb{Z}}
\def\IT{\mathbb{T}}
\newcommand{\bl}[1]{{\color{blue}#1}}
\newcommand{\pp}{\partial}
\def\Dir{\slashed{D}}
%%%%%%%%%%%%%%%%%%%%%%%%%%%%%%%%%%%%%%%%%%%%%%%%
\title[Spectral Metric and Einstein Functionals  for Hodge-Dirac operator]{Spectral Metric and Einstein Functionals \\[2mm] for Hodge-Dirac operator} 
\author[L.\ D\k{a}browski]{Ludwik D\k{a}browski${}^{(1)}$}
\address{${}^{(1)}$ SISSA (Scuola Internazionale Superiore di Studi Avanzati), \newline\indent Via Bonomea 265, 34136 Trieste, Italy} 
\email{dabrow@sissa.it} 
%%%%%%%%%%%%%%%%%%%%%%%%%%%%%%%%%%%%%%%%%%%%%%%%
\author[A.\ Sitarz]{Andrzej Sitarz${}^{(2)}$}
\author[P.\ Zalecki]{Pawe\l{} Zalecki${}^{(2)}$}
\thanks{This work is supported by the Polish National Science Centre grant 2020/37/B/ST1/01540}
\address{${}^{(2)}$ Institute of Theoretical Physics, Jagiellonian University, \newline\indent
	prof.\ Stanis\l awa \L ojasiewicza 11, 30-348 Krak\'ow, Poland.}
\email{andrzej.sitarz@uj.edu.pl}  
\email{pawel.zalecki@doctoral.uj.edu.pl}
\date{}
%%%%%%%%%%%%%%%%%%%%%%%%%%%%%%%%%%%%%%%%%%%%%%%%
\begin{document}
\maketitle
\begin{abstract}
We examine the metric and Einstein bilinear functionals of differential forms introduced in \cite{DSZ23}, for Hodge-Dirac operator $d+\delta$ on an oriented even-dimensional Riemannian manifold. We show that they reproduce these functionals for the canonical Dirac operator on a spin manifold up to a numerical factor. Furthermore, we demonstrate that the associated spectral triple is spectrally closed, which implies that it is torsion-free. \\[4mm]
\end{abstract}
	
\noindent Keywords: {\it Noncommutative geometry, Einstein tensor, spectral geometry, Wodzicki residue. }
	
\section{Introduction}
Spectral geometry investigates relationships between geometric structures of manifolds and the spectra of certain differential operators. Its direct and inverse problems are inextricably linked to other areas of mathematics such as number theory, representation theory, and areas of mathematical physics such as quantum mechanics and general relativity. In this regard, starting with the Laplace-Beltrami operator on a closed Riemannian manifold, general Laplace-type operators have been extensively studied, and their spectra provide insights into the geometry and topology of the underlying space. The distribution of eigenvalues, for example, reveals information about the curvature or shape and global geometric properties such as diameter or volume, connectivity, or the presence of holes.
	In this vein, the Dirac-type operators have also been studied, beginning with the canonical Dirac operator on the spin manifold.
	When subsumed into Connes' concept of spectral triples \cite{Co80,Co94}, they "can hear the shape of a drum" \cite{Ka66} in the sense that their equivalence (a suitably strengthened isospectrality) implies the isometricity of manifolds in virtue of the reconstruction theorem \cite{Co13}.
	Furthermore, they allow for broad and captivating generalisations in noncomutative geometry.
	
	Various (interrelated) spectral schemes that generate geometric objects on manifolds such as volume, scalar curvature, and other scalar combinations of curvature tensors and their derivatives are
	the small-time asymptotic expansion of the (localised) trace of the heat kernel \cite{Gi84,Gi04}, certain values or residues of the (localised) zeta function of the Laplacian, the spectral action,
	and the Wodzicki residue $\wres$ (also known as noncommutative residue).
	In this paper, we focus on the latter one, which is the unique (up to multiplication by a constant)
	tracial state on the algebra of pseudo-differential operators
	($\Psi$DO) on a complex vector bundle $E$ over a compact manifold $M$ of dimension $n\geq 2$ \cite{Gu85,Wo87}.
	For the oriented manifold $M$ it is given by a simple integral formula,
	\begin{equation}
		\wres\,(P) :=
		\int_M \left( \int_{|\xi|=1} tr\, \sigma_{-n}(P)(x,\xi)~ {\mathcal V}_\xi\right)~ d^n x,
	\end{equation}
	where $tr$ is the trace over endomorphisms of the bundle $E$ at any given point of $M$, 
	$\sigma_{-n}(P)$ is the symbol of order $-n$ of a pseudodifferential operator $P$ and ${\mathcal V}_\xi$ denotes the volume form on the unit sphere.
	
	
	When applied to the (scalar) Laplacian $\Delta$ on a Riemannian manifold $M$
	of dimension $n=2m$ equipped with a metric tensor $g$ it yields, 
	in a {\it localized} form,  a functional of $f \in C^\infty(M)$,
	\beq
	\label{WresfL}
	{\mathcal v}(f) :=  \wres\,(f \Delta^{-m})= v_{n-1} \int_M f~vol_g,
	\eeq
	where 
	$$v_{n-1}:=vol(S^{n-1}) = \frac{2\pi^{m}}{\Gamma(m)},$$ 
	is the volume of the unit sphere $S^{n-1}$ in $\IR^n$.
	
	A startling result regarding a higher power of the Laplacian divulged by Connes 
	\cite{Co96} in the early 1990s (see \cite{Ka95} and \cite{KaWa95} for explicit confirmation for $n=4$) states that
	\beq\label{WresfLL}
	{\mathcal R}(f) := \wres\,(f \Delta^{-m+1})= \frac{n-2}{12} v_{n-1} \int_M f R(g) vol_g, 
	\eeq which for $n>2$ and $f=1$ is, up to a constant, a Riemannian analogue of the Einstein-Hilbert action functional of general relativity in vacuum.  Here $R=R(g)$ is the scalar curvature, that is the $g$-trace  
	$R\!=\!g^{jk}R_{jk}$ of the Ricci tensor  with components $R_{jk}$ in local coordinates, where $g^{jk}$ are the raised components of the metric $g$. 
	
	
	In the noncommutative realm, the spectral-theoretic approach to scalar curvature has been extended to quantum tori in the seminal work of Connes and Tretkoff  
	\cite{CoTr11} and extensively studied by many authors (see references in \cite{DSZ23}).
	
	In the recent paper \cite{DSZ23} we accomplished the task of extracting two other important tensorial geometrical objects through spectral methods. These were the metric tensor ${ g}$ itself, its dual, and the Einstein tensor
	$$ { G}:= \hbox{Ric} - \frac{1}{2} R(g)\, {g},$$
	which directly enters the Einstein field equations with matter and its dual. In fact, for this purpose, we employed the Wodzicki residue of a suitable power of the Laplace type operator or of the Dirac type operator, multiplied by a pair of other differential operators. Notably, we have recovered the tensors ${g}$ and ${G}$ as the density of certain bilinear functionals of vector fields on a manifold $M$, while their dual tensors are the density of bilinear functionals of differential one-forms on $M$. The latter functionals (up to a numerical factor) we have obtained also for the canonical Dirac operators (in case $M$ is a spin manifold).
	Then, using Connes' and Moscovici's \cite{CoMo95} generalisation of pseudodifferential calculus for noncommutative spectral triples, we introduced their conspicuous quantum analogue and probed it on 2 and 4-dimensional noncommutative tori.
	
	The aim of this paper is, employing methods of the Wodzicki residue, to analyse the metric and Einstein functionals for another natural Dirac-type operator, namely the Dirac-Hodge operator $d+\delta$ acting on (complex) differential forms $\Omega(M)$ of arbitrary order on a oriented even-dimensional Riemannian manifold $M$. It is worth mentioning that the associated Hodge-Dirac spectral triple is characterised \cite{DDS18} by the fact that the Hilbert space of square-integrable forms provides a Morita equivalence $Cl(M)-Cl(M)$ bimodule,
	where $Cl(M)$ is the $C^*$-algebra of continuous sections of the bundle of Clifford algebras on $M$.
	As is well known, the canonical spectral triple on a spin manifold is instead characterised by the fact that its Hilbert space of square-integrable Dirac spinors provides a Morita equivalence $Cl(M)-C(M)$ bimodule, where $C(M)$ is the algebra of continuous complex functions on $M$.
	As our first main result, we demonstrate that these two different pivotal cases yield in fact equal spectral metric and Einstein functionals (up to a numerical factor). Moreover, as our second main result, we prove that the associated spectral triple is spectrally closed, that is, for any operator $T$ of zero-order,
	$$ \wres (T D|D|^{-n}) \equiv 0. $$
	A forthcoming result \cite{DSZ23b} demonstrates that, as a consequence, the Hodge-Dirac operator has no torsion.
	%%%%%%%%%%%%%%%%%%%%%%%%%%%%%%%%%%%%%%%%%%%
	%%%%%%%%%%%%%%%%%%%%%%%%%%%%%%%%%%%%%%%%%%%
	\section{Preliminaries} %%%%%%%%%%%%%%%%%%%
	%%%%%%%%%%%%%%%%%%%%%%%%%%%%%%%%%%%%%%%%%%%
	%%%%%%%%%%%%%%%%%%%%%%%%%%%%%%%%%%%%%%%%%%%
	Let $n=2k$ be the dimension of an oriented, closed, smooth Riemannian manifold $M$.
	We will use capital letters to denote increasing sequences of numbers between $1$ and $n$, of fixed length $0 \leq \ell \leq n$.
	A differential $\ell$-form $\omega = \sum_J \omega_J dx^J$ is determined by its
	coefficients $\omega_J$, with respect to coordinates indicated by the multi-index $J$, where with a slight abuse of notation $0$-forms (i.e. functions) will correspond to $J=\emptyset$.
	
	We introduce the operators $\lambda^j_+$ and $\lambda^j_-$ which respectively raise/lower the degree of forms, with components given by 
	$$ (\lambda^p_+)^I_J = \epsilon^{I}_{pJ}, \qquad (\lambda^p_-)^I_J = \epsilon^{pI}_{J}, $$
	where $\epsilon^{I}_{pJ}=(-)^{|\pi|}$ if the juxtaposed index $pJ$ is a permutation $\pi$ of $I$ and $\epsilon^{I}_{pJ}=0$ otherwise, and similarly for $\epsilon^{pI}_{J}$. They satisfy
	\begin{equation}
		\begin{aligned}
			&\lambda^p_+ \lambda^r_+ + \lambda^r_+ \lambda^p_+ = 0,  \\	
			&\lambda^p_- \lambda^r_- + \lambda^r_- \lambda^p_- = 0,  \\
			&\lambda^p_+ \lambda^r_- +  \lambda^r_- \lambda^p_+= \delta_{pr} \,{\rm id},
		\end{aligned}
	\end{equation}
	which follow from the relations  (c.f. \cite{MMMT})
	\begin{equation}
		\begin{aligned}
			&	\sum_K   \epsilon^{I}_{pK}  \epsilon^{K}_{rJ} =  \epsilon^{I}_{prJ}, \qquad
			&	\sum_K   \epsilon^{pI}_{K}  \epsilon^{rK}_{J} =  \epsilon^{I}_{rpJ}, \\
			&	\sum_K   \epsilon^{I}_{pK}  \epsilon^{rK}_{J} = \delta_{pr} \epsilon^I_J - \epsilon_{pJ}^{rI}, \qquad 
			&	\sum_K   \epsilon^{rI}_{K}  \epsilon^{K}_{pJ} =  \epsilon_{pJ}^{rI}, \\
		\end{aligned}
	\end{equation}
	where the juxtaposed indices can be ordered using a signed permutation.
	We also introduce 
	$$\gamma^p  = -i(\lambda_+^p - \lambda_-^p),$$
	which satisfy the following Clifford algebra relation
	$$\{ \gamma^p, \gamma^r\} =2\delta_{pr}.$$ 
	
	In the rest of the paper, we employ normal coordinates $x$ centred around some fixed point on the manifold. Recall that then the components of the metric tensor $g$, its covariant (raised) components, and the square root of the determinant 
	of the matrix of the components of $g$ and the components of the Levi-Civit\'a connection have  the following Taylor expansion:
	\begin{equation}
		\begin{aligned}	
			&		g_{ab} = \delta_{ab} - \frac{1}{3} R_{acbd} x^c x^d + o({\bf x^2}), \\
			&		g^{ab} = \delta_{ab} + \frac{1}{3} R_{acbd} x^c x^d + o({\bf x^2}), \\
			&		\sqrt{\hbox{det}(g)}  = 1  - \frac{1}{6} \mathrm{Ric}_{ab} x^a x^b + o({\bf x^2}), 	 \\
			&		\Gamma^a_{bc}(x)= -\frac{1}{3}(R_{abcd}+R_{acbd } ) x^{d} +o({\bf x^2}).	
		\end{aligned}				
		\label{allNorm} 
	\end{equation}
	Here $R_{acbd}$ and $\mathrm{Ric}_{ab}$ are the components of the Riemann and Ricci tensors, respectively, at the point with ${\bf x}=0$ and we use the notation $o({\bf x^k})$ to denote that we expand a function up to the polynomial of order $k$ in the normal coordinates. 
	%%%%%%%%%%%%%%%%%%%%%%%%%%%%%%%%%%%%%%%%
	\subsection{Hodge-Dirac operator}
	We focus on the Hodge-Dirac operator $D=d+d^*$, where $d$ is the exterior derivative and $d^*$ is its  (formal) adjoint. Using our notation, we compute the symbol of $D$, 
	\begin{equation}
		\sigma(D) =  (i  \lambda^p_+ - i g^{pr}  \lambda^r_-) \xi_p
		+ \lambda_-^p \lambda_+^r \lambda_-^s  \Gamma^s_{rt} g^{pt},
		\label{HoDi}
	\end{equation}
	which in normal coordinates takes form 
	\begin{equation}
		\sigma(D) =  - \gamma^p \xi_p  - \frac{1}{3} i \lambda^p_- R_{sapb} x^a x^b \xi_s 
		- \frac{1}{3} \lambda_-^p \lambda_+^r \lambda_-^s 
		(R_{srpa}+R_{spra}) x^a + o({\bf x^2}).   
	\end{equation}	
	We compute then the symbols of the Hodge-Dirac Laplacian $D^2$
	in normal coordinates up to orders relevant for our purposes. 
	\begin{lem}
		The three homogeneous symbols of $D^2$ read
		$$ 
		\begin{aligned}
			{\mathfrak a}_2 =& \bigl( \delta_{ab} + \frac{1}{3} R_{acbd} x^c x^d \bigr) 
			\xi_a \xi_b + o({\bf x^2}), \\
			%%%%%%%%%%%%%%%%%%%%%%%%%%%%%%%%%%%%%%%%%%%%		
			{\mathfrak a}_1 =&+ \frac{2}{3} i Ric_{ab} \xi_a x^b 
			- \frac{2}{3} i \lambda_+^p \lambda_-^r   (R_{rpab} + R_{rapb} )  x^b   \xi_a + o({\bf x}), \\
			%%%%%%%%%%%%%%%%%%%%%%%%%%%%%%%%%%%%%%%%%%%%		
			{\mathfrak a}_0 = &  +\frac{2}{3}  \lambda_+^p   \lambda_-^r   Ric_{pr}
			+  \frac{1}{3}  \lambda_+^p   \lambda_+^r  \lambda_-^s \lambda_-^t   (R_{tsrp}+R_{trsp })  + o({\bf 1}).
		\end{aligned}
		$$
	\end{lem}
	\begin{proof}
		The computation of the principal symbol ${\mathfrak a}_2$ is obvious for the symbol of order 1:
		$$ 
  	\begin{aligned}
			{\mathfrak a}_1 =& -\frac{1}{3} i  \{\lambda_+^t , \lambda_-^p \lambda_+^r \lambda_-^s  \}
			(R_{srpa}+R_{spra}) x^a \xi_t  
			+ \frac{1}{3} i  \{\lambda_-^t , \lambda_-^p \lambda_+^r \lambda_-^s \}
			(R_{srpa}+R_{spra}) x^a \xi_t  \\ &
			\quad + \frac{1}{3}  \gamma^p \lambda_-^r  (R_{aprb}+R_{abrp}) x^b   \xi_a  \\ &
			%%%%%%%%%%%%%%%%%%%%%%%%%%%%%%%%%%%%%%%%%%%%%%%%%%
			= - \frac{1}{3} i \lambda_+^r \lambda_-^s (R_{srta}+R_{stra})  x^a \xi_t  
			- \frac{1}{3} i \lambda_-^p \lambda_+^r  (R_{trpa}+R_{tpra})  x^a \xi_t  \\
			& \quad -  \frac{1}{3} i   \lambda_-^p \lambda_-^s  (R_{stpa}+R_{spta}) x^d \xi_t  
			- \frac{1}{3} i \lambda_+^p \lambda_-^s  (R_{tpsa}+R_{tasp}) x^a   \xi_t  \\
			&  \quad + \frac{1}{3} i \lambda_-^p \lambda_-^s  (R_{tpsa}+R_{tasp}) x^a   \xi_t  \\
			%%%%%%%%%%%%%%%%%%%%%%%%%%%%%%%%%%%%%%%%%%%%%%%%%%%%%%%%%%%
			& =  \frac{2}{3} i Ric_{sa}  x^a \xi_s
			- \frac{1}{3} i\lambda_+^r \lambda_-^s 
			(R_{srta} + R_{stra} - R_{trsa} - R_{tsra} +R_{trsa} +R_{tasr})  x^a   \xi_t \\
			& \quad  - \frac{1}{3} i \lambda_-^r \lambda_-^s   
			(R_{stra}+R_{srta} - R_{trsa} -R_{tasr}  ) x^a \xi_t \\
			& =   \frac{2}{3} i Ric_{sa}  x^a \xi_s 
			- \frac{2}{3} i\lambda_+^r \lambda_-^s   (R_{srta} + R_{stra} )  x^a   \xi_t. 
			%%%%%%%%%%%%%%%%%%%%%%%%%%%%%%%%%%%%%%%%%%%%%%%%%%%%%%	
		\end{aligned}
		$$
		and the order $0$ symbol:
		$$ 
		\begin{aligned}
			%%%%%%%%%%%%%%%%%%%%%%%%%%%%%%%%%%%%%%%%%%%%		
			{\mathfrak a}_0 =
			& -\frac{1}{3} i  \gamma^p   \lambda_-^q \lambda_+^r \lambda_-^s   (R_{sqrp}+R_{srqp })   \\
			=&
			-\frac{1}{3}  \lambda^p_+   \lambda_-^q \lambda_+^r \lambda_-^s   (R_{sqrp}+R_{srqp})\\
			= &  \; \frac{2}{3}  \lambda_+^p   \lambda_-^s   Ric_{ps}
			+  \frac{1}{3}  \lambda_+^p   \lambda_+^r  \lambda_-^q \lambda_-^s   (R_{sqrp}+R_{srqp }) .
		\end{aligned}
		$$
	\end{proof}
%%%%%%%%%%%%%%%%%%%%%%%%%%%%%%%%%%%%%%%%%%%%%%%%
%%%%%%%%%%%%%%%%%%%%%%%%%%%%%%%%%%%%%%%%%%%%%%%%
\subsection{The inverse of $D^2$ and its powers}
In this section, we  present the results that can be applied to a more general situation than the Hodge-Dirac operator. Let us start with the following lemma:
	\begin{lem} \label{inv_laplace_type}
		Let $L$ be a Laplace-type operator with symbol 
		$$
		\sigma(L)= \mathfrak a_2 +\mathfrak a_1 +\mathfrak a_0
		$$
		expressed in normal coordinates as
		$$
		\begin{aligned}
			&\mathfrak a_2 = \bigl(\delta_{ab} +\frac13R_{acbd}x^cx^d\bigr) \xi_a\xi_b +o({\bf x^2}),\\
			&\mathfrak a_1 = i P_{ab}\xi_ax^b + o({\bf x}),\\
			&\mathfrak a_0 = Q + o({\bf 1}).
		\end{aligned}
		$$
		Then the theree leading symbols of $\sigma(L^{-k})= \mathfrak c_{2k} +\mathfrak c_{2k+1} +\mathfrak c_{2k+2}$ are:
		$$
		\begin{aligned}
			& \mathfrak c_{2k}= ||\xi||^{-2k-2} \left( \delta_{ab} - \frac{k}{3} R_{acbd} x^c x^d \right)  \xi_a \xi_b + o({\bf x^2}),\\
			&\mathfrak c_{2k+1}= -ik ||\xi||^{-2k-2} P_{ab}\xi_ax^b + o({\bf x}),\\
			& \mathfrak c_{2k+2}= -k||\xi||^{-2k-2} Q +k(k+1) ||\xi||^{-2k-4}\left(P_{ab}-\frac13 Ric_{ab}\right) \xi_a \xi_b + o({\bf 1}).
		\end{aligned}
		$$
	\end{lem}
	\begin{proof}
We start with computing leading symbols of the inverse of $L$, i.e. $\sigma(L^{-1})= \mathfrak b_2 +\mathfrak b_3 +\mathfrak b_4$ using the fact, that $\sigma(LL^{-1})=\sigma(1)=1$. We have:
$$
		\begin{aligned}
			\mathfrak b_2 = &\; (\mathfrak a_2)^{-1} = ||\xi||^{-4} \left(\delta_{ab} - \frac13R_{acbd}x^cx^d\right) \xi_a\xi_b +o({\bf x^2}),\\
			\mathfrak b_3 =& \;\mathfrak b_2(-\mathfrak a_1 \mathfrak b_2 +i \partial_\xi^a \mathfrak a_2 \partial_a \mathfrak b_2)= -i ||\xi||^{-4} P_{ab}\xi_ax^b + o({\bf x}),\\
			\mathfrak b_4 =& \;\mathfrak b_2\left(-\mathfrak a_0 \mathfrak b_2 - \mathfrak a_1 \mathfrak b_3 + i \partial_\xi^a \mathfrak a_2 \partial_a \mathfrak b_3+ i \partial_\xi^a \mathfrak a_1 \partial_a \mathfrak b_2 + \frac{1}{2} \partial_\xi^{ab} \mathfrak a_2 \partial_{ab} \mathfrak b_2\right)\\
			=& -||\xi||^{-4} Q +2 ||\xi||^{-6}\left(P_{ab}-\frac13 Ric_{ab}\right) \xi_a \xi_b + o({\bf 1}).
		\end{aligned}
$$
	To finish we apply Lemma\,A1 in \cite{DSZ23} and compute the three leading symbols of the powers of the pseudodifferential operator $L^{-k}$.
\end{proof}
%%%%%%%%%%%%%%%%%%%%%%%%%%%%%%%%%%%%%%%%%%%%%%%%
Using the above  lemma for $L=D^2$, with the Hodge-Dirac operator $D$ \eqref{HoDi} , we get the following result.
	\begin{prop}\label{prop23}
The leading symbols of  $D^{-2k}$ are, up to the appropriate order in {\bf x},
		\begin{equation} 
			\begin{aligned}
				&\mathfrak c_{2k} = ||\xi||^{-2k-2} \left( \delta_{ab} - \frac{k}{3} R_{acbc} x^c x^d \right)  \xi_a \xi_b + o(\bf x^2),\\
				&\mathfrak c_{2k+1}=-\frac{2}{3} ki||\xi||^{-2k-2}
				\mathrm{Ric}_{ab} x^b \xi_a 
				{
					+} \frac{2}{3} k i ||\xi||^{-2k-2}  \lambda_+^r \lambda_-^s   \bigl(  R_{srba} + R_{sbra}  \bigr) x^a   \xi_b  + o(\bf x)  \\
				&\mathfrak c_{2k+2}=\frac{k(k+1)}{3} ||\xi||^{-2k-4} \mathrm{Ric}_{ab}\xi_a\xi_b \\
				& \qquad \qquad  
				- \frac{2}{3} k (k+1) ||\xi||^{-2k-4}  \, 
				\lambda_+^r \lambda_-^s   (R_{srab} + R_{sarb} )  \xi_a   \xi_b   \\
				& \qquad \qquad 
				+\frac{1}{3} k ||\xi||^{-2k-2}\lambda_+^p  \lambda_-^q  \lambda_+^r \lambda_-^s   (R_{sqrp}+R_{srqp }) 
				+ o({\bf 1}).  
			\end{aligned}\label{LapTF3}
		\end{equation}	
	\end{prop}
\begin{proof}
	For $L=D^2$ we substitute in Lemma \ref{inv_laplace_type}
		$$
		\begin{aligned}
			&P_{ab} = \frac{2}{3} Ric_{ab} 
			- \frac{2}{3} \lambda_+^r \lambda_-^s   (R_{srab} + R_{sarb} ),  \\
			&Q= -  \frac{1}{3}  \lambda_+^p  \lambda_-^q \lambda_+^r \lambda_-^s   (R_{sqrp}+R_{srqp }).
		\end{aligned}
		$$
	\end{proof}
	%%%%%%%%%%%%%%%%%%%%%%%%%%%%%%%%%%%%%%%%%%%%%%%%%%%%%%%%%
	%%%%%%%%%%%%%%%%%%%%%%%%%%%%%%%%%%%%%%%%%%%%%%%%%%%%%%%%%
	\section{Spectral functionals}
In \cite{DSZ23}, we defined two spectral functionals for finitely summable spectral triples, which for the canonical spectral triple over the spin manifold $M$ allow to recover the metric and the Einstein tensors, viewed as bilinear functionals over a pair of one-forms. We recall the definition:
	\begin{defn}[cf. \cite{DSZ23}, Definition 5.4]\label{Dwres}
		If $(\cA, D, \cH)$ is a $n$-summable spectral triple, let $\Omega^1_D$ be the $\cA$ bimodule of one form generated by  $\cA$ and $[D, \cA]$. Moreover, assume there exists a generalised algebra of pseudodifferential operators which contains $\cA$, $D$,  $|D|^\ell$ for $\ell\in\IZ$ with
		a tracial state $\wres$ over this algebra (called a noncommutative residue), which identically 
		vanishes  on $T |D|^{-k}$ for any $k>n$ and a zero-order operator $T$ (an 
		operator in the algebra generated by $\cA$ and $\Omega^1(\cA)$). Then, 
		denoting by $\hat{u}, \hat{v}$ the image of $u,w \in \Omega^1_D(\cA)$ in the Clifford algebra, we call 
		$$ \mathcal{g}_D(u,w) = \wres (u w|D|^{-n}), $$	
		{\em metric functional}, and
		$$ \mathcal{G}_D(u,w) = \wres (u \{ D, w \} D |D|^{-n}). $$	
		{\em Einstein functional}.
	\end{defn}
		First, let us compute  the metric functional ${\mathcal g}$. For one-forms $u,w$  
		the suitable expansion (up to $o({\bf 1})$)  in normal coordinates) of Clifford  multiplication by 
		$u = u_a dx^a$ and  $w = w_b dx^b$ is
		$$  \hat u =  \gamma^p u_p + o({\bf 1}), \qquad  \hat w =  \gamma^r w_r+ o({\bf 1}).$$
		%%%%%%%%%%%%%%%%%%%%%%%%%%%%%%%%%%%%%%
		\begin{prop}
			The metric spectral functional reads
			\begin{equation}
				{\mathcal g}(u,w) = 2^n v_{n-1}\int\limits_M  \sqrt{g} \,u_p w_p,
			\end{equation}
			where $v_{n-1}$ is the volume of $(n-1)$-dimensional sphere.
		\end{prop}	
		%%%%%%%%%%%%%%%%%%%%%%%%%%%%%%%%%%%%%%
		\begin{proof}
			We compute explicitly:
			$$
			\begin{aligned}
				{\mathcal W} \left( (\gamma^p \gamma^r u_p w_r )  |D|^{-n}  \right)  = 
				\int\limits_M \sqrt{g} \int_{||\xi||=1} \hbox{Tr\ } (\gamma^p \gamma^r u_p w_r )  {\mathfrak c}_{2m}(D) \\
				\qquad \qquad  = v_{n-1}   \int\limits_M \sqrt{g} \hbox{Tr\ } (\gamma^p \gamma^r u_p w_r )  =  2^n v_{n-1}\int\limits_M \sqrt{g} \,u_p w_p .
			\end{aligned}
			$$		
			The last factor comes from the trace of $1$ over the space of differential forms. 
		\end{proof}
			Next, we have,	
			\begin{prop}\label{HDEin}
The Einstein functional is, similarly to the case of the canonical spectral triple,
				$$
				\mathcal{G}(u,w)=\frac{2^n}{6}v_{n-1}\int\limits_M \sqrt{g} \, G_{pq} u_p w_q.
				$$    
				where $G_{pq}$ is the Einstein tensor for $M$.
			\end{prop}
%%%%%%%%%%%%%%%%%%%%%%%%%%%%%%%%%%%%%%%%%%%%%%%%%%%%	
Before we begin with the proof  let us demonstrate some useful lemmas.
			The first computes $\mathcal W(ED^{-2m+2})$ for two specific cases of
			endomorphism $E$. 
			\begin{lem} \label{funkcjonaly}
				If,
				$$E=e^{(0)} +e^{(2)}_{pq}\gamma^p\gamma^q $$
				the functional 
				$$	\mathcal W(ED^{-2m+2})= \frac{n-2}{24}2^nv_{n-1}\int\limits_M \sqrt{g} R( -e^{(0)} -e^{(2)}_{pp} ).
				$$
				On the other hand, if, 
				$$ \tilde E	= \tilde e^{(0)} +\tilde e^{(2)}_{pq}\lambda_+^p\lambda_-^q,  $$
				then the functional 
				$$\mathcal W(\tilde E D^{-2m+2}) =-\frac{n-2}{24}2^nv_{n-1}\int\limits_M  \sqrt{g} R\tilde e^{(0)}. $$
			\end{lem}
			The proof is based on direct calculations using Proposition \ref{prop23}.
			\begin{equation}
				\begin{aligned}
					\mathcal W(E&D^{-2m+2})= \int\hbox{Tr\ } \left(E\int_{||\xi||=1} \mathfrak c_{2(m-1)+2}\right)\\
					&= \frac{n-2}{12}v_{n-1}\int \hbox{Tr\ }\left(E\left[2(R_{srqp}+R_{sqrp})\lambda_+^p\lambda_-^q\lambda_+^r\lambda_-^s +R-2Ric_{qp}\lambda_+^p\lambda_-^q\right]\right)
				\end{aligned}
			\end{equation}
			Next, we consider functionals of the type $\mathcal W(PD^{-2m})$, where $\sigma(P)=F^{ab}\xi_a\xi_b+G^a\xi_a+H$.
			\begin{lem}
				For $P$, such that $\sigma(P)=F^{ab}\xi_a\xi_b+G^a\xi_a+H$, we have that, if
				$$ F^{ab}= f^{(0)ab} +f^{(2)ab}_{pq}\gamma^p\gamma^q, $$
				then
				$$	\mathcal W(PD^{-2m})= v_{n-1}\int\limits_M\hbox{Tr\ }H + \frac{2^n}{24}v_{n-1}\int R(-f^{(0)aa}- f^{(2)aa}_{pp}) ,$$
				and, if
				$$ \tilde F^{ab} = \tilde f^{(0)ab} +\tilde f^{(2)ab}_{pq}\lambda_+^p\lambda_-^q,$$
				then 
				$$
				\mathcal W(\tilde PD^{-2m})= v_{n-1}\int\limits_M\hbox{Tr\ }H + \frac{2^n}{48}v_{n-1}\int\limits_M \left[ -2R \tilde f^{(0)aa} -\tilde f^{(2)aa}_{pp} R+ 2(\tilde f^{(2)ab}_{pq}+ \tilde f^{(2)ba}_{pq} )R_{paqb} \right]. 
				$$
			\end{lem}
			The proof follows directly by computation using Proposition \ref{prop23} and 
			Lemma \ref{kawalki} applied to the explicit expression:
			\begin{equation}\label{secondfunctional}
				\begin{aligned}
					\int_{||\xi||=1}\sigma_{-2m}(PD^{-2m})&= \frac{1}{6}F^{aa} \left[(R_{srqp}+R_{bqrs}) \lambda_+^p\lambda_-^q \lambda_+^r\lambda_-^s 
					+\frac12 R - R_{pq}\lambda_+^p\lambda_-^q\right]\\
					&+\frac{1}{6}F^{ab}[-R_{ab}+(R_{qapb}+R_{paqb})\lambda_+^p\lambda_-^q]    
					+H.
				\end{aligned}
			\end{equation}
			\begin{proof}[Proof of Proposition \ref{HDEin}]
				We begin with computing the  symbol of $ \hat u D \hat w D $ at $x=0$,
				where it suffices to expand the Clifford multiplication by one-forms 
				$u$ and $w$ as:
				$$ \hat u =  \gamma^p {u}_p + o({\bf 1}), \qquad  
				\hat w =  \gamma^s w_s + \gamma^s w_{sa}  x^a + o({\bf x}),$$
				and thus
				$$ 
				\begin{aligned}
					\hat u D\hat wD  = & 
					\gamma^p \gamma^q \gamma^r \gamma^s {u}_p w_r \xi_q \xi_s 
					- i  \gamma^p \gamma^q \gamma^r \gamma^s {u}_p w_{rq}  \xi_s \\
					&  -\frac{1}{3} i \gamma^p \gamma^q \gamma^r 
					\lambda_-^s \lambda_+^t \lambda_-^z  (R_{ztsq} +R_{zstq})  u_p w_{r} +o({\bf 1}).
				\end{aligned}
				$$
				Then, we use lemma \ref{funkcjonaly} for $P=\hat uD\hat wD$ and $E=\hat u\hat w$. In this case we have
				$$
				\begin{aligned}
					&E=u_p w_q\gamma^p\gamma^q, \\
					& H  = -\frac{1}{3} i \gamma^p \gamma^q \gamma^r 
					\lambda_-^s \lambda_+^t \lambda_-^z  (R_{ztsq} +R_{zstq})  u_p w_{r}\\
					& \quad=-\frac{2}{3}i\gamma^p 
					\lambda_-^q \lambda_+^r \lambda_-^s  (R_{srqt} +R_{sqrt})  u_p w_{t} 
					+ \frac{1}{3} i \gamma^p \gamma^q \gamma^r 
					\lambda_-^s \lambda_+^t \lambda_-^z  (R_{ztsr} +R_{zstr})  u_p w_{q}, \\ 
					&F^{pq}\xi_p\xi_q=  \gamma^r \gamma^p \gamma^s \gamma^q u_r w_s \xi_p\xi_q= (2u_r w_p \delta_{qs}\gamma^r\gamma^s 
					- u_r w_s \gamma^r\gamma^s \delta_{pq})\xi_p\xi_q,
				\end{aligned}
				$$
				where we used that $\gamma^p\gamma^q\xi_p\xi_q = \delta_{pq}\xi_p\xi_q$. 
				Next, we see,
				$$
				\begin{aligned} 
					&e^{(0)}=0, \qquad && e^{(2)}_{ab}=u_a w_b, \\
					& f^{(0)}=0, \qquad && f^{(2)ab}_{cd}=2u_c w_a\delta_{bd}
					-u_c w_d\delta_{ab}.
				\end{aligned}
				$$
				Finally, the contribution arising from $E$ gives:
				$$
				-\frac{n-2}{24}2^nv_{n-1}R\, u_aw_a.
				$$
				whereas the part from $F$ is,
				$$
				-\frac{2^n}{24} v_{n-1} R f^{(2)aa}_{ii}= 
				\frac{n-2}{24} 2^n v_{n-1} R\, u_a w_a.
				$$
			These two terms cancel each other, and we are left with terms that arise from  $H$. The only possible terms in $\hbox{Tr\ }H$ are linear combinations of $u_aw_aR$ and $u_aw_bRic_{ab}$, thus, we know, that the result is symmetric in $u_a,w_b$. It allows us to simplify the second term in $H$:
				$$
				H=-\frac{2}{3} i \gamma^p 
				\lambda_-^q \lambda_+^s \lambda_-^t  (R_{tsqb} +R_{tqsb})  u_p w_{b} + \frac{1}{3} i  \gamma^r 
				\lambda_-^q \lambda_+^s \lambda_-^t  (R_{tsqr} +R_{tqsr})  u_a w_{a} +\dots,
				$$
				where "$\dots$" are terms antisymmetric in $u_a,w_b$, which we can neglect. We can also insert $-i\lambda_+$ instead of the remaining $\gamma$'s because the part with $\lambda_-$ will be traceless. Now, using the lemma \ref{slady_naprzemienne} we get:
				$$
				\hbox{Tr\ }H= \frac{1}{6} \hbox{Ric}_{ab}\, u_a w_b  -\frac{1}{12} R \,u_a w_a= \frac{1}{6} G_{ab}\, u_a w_b .
				$$
				This proves the result.
			\end{proof}
			
			\subsection{Spectral closedness and torsion}
			In this section, we will prove that the Hodge-Dirac spectral triple has the property of being spectrally closed.
			\begin{thm}
				Let $T$ be an operator of order $0$ from the algebra generated by $a [D,b]$, $a,b \in C^\infty(M)$.
				Then,
				$$ \wres (T D |D|^n) = 0. $$
			\end{thm}
			\begin{proof}
				If we compute the symbol of  $TD$ at a chosen point on the manifold $M$ in normal ccordinates at
				$x=0$ we obtain,
				$$ \sigma(T D)  = T (- \gamma^p \xi_p). $$
				Next, if we combine it with Proposition \ref{prop23} we see that the symbol of order $-n$
				of $T D |D|^n$ is:
				$$ \sigma_{-n}(T D |D|^{-n}) = 0 + o(\mathbf{1}). $$
				This ends the proof.
			\end{proof}
		As a consequence, we demonstrate in \cite{DSZ23b} that the Hodge-Dirac spectral triple is torsion-free. It is interesting to study generalised Hodge-Dirac operators, which are defined through an arbitrary linear connection, are not metric compatible, and have a torsion.
			
			\appendix
			\section{Details of computations}
			\begin{lem}\label{Tra1}
				A direct computation of traces of products of $\lambda$ matrices is based on the following
				recursive formula: 
				$$
				\begin{aligned}
					\mathrm{Tr\ } &\lambda^{p_1}_+  \cdots  \lambda^{p_k}_+	\lambda^{q_1}_-  \cdots  \lambda^{q_k}_-  \\
					&=  \frac{1}{2}  \sum\limits_{j=1}^k (-1)^{k-j} \delta^{p_1 q_j}\; \mathrm{Tr\ } \,
					\bigl( \lambda^{p_2}_+  \cdots  \lambda^{p_k}_+	\lambda^{q_1}_-  \cdots    \lambda^{q_{j-1}}_-  \lambda^{q_{j+1}}_- \cdots \lambda^{q_k}_-  \bigr).
				\end{aligned}
				$$
				In particular, we have
				\begin{equation}
					\label{slady_naprzemienne}
					\begin{aligned}
						&\mathrm{Tr\ }( \lambda_+^p \lambda_-^q ) =
						2^{n-1} \delta^{pq}, 
						\\
						&\mathrm{Tr\ }( \lambda_+^{p_1} \lambda_-^{q_1}\lambda_+^{p_2} \lambda_-^{q_2} ) =
						2^{n-2}(\delta^{p_1q_1} \delta^{p_2q_2} + \delta^{p_1q_2} \delta^{p_2q_1}),
					\end{aligned}
				\end{equation}
				and
				\begin{equation}
					\begin{aligned}	
						\mathrm{Tr\ }( \lambda_+^{p_1} &\lambda_-^{q_1}\lambda_+^{p_2} \lambda_-^{q_2} \lambda_+^{p_3} \lambda_-^{q_3} ) = \\
						&=	2^n \bigl( \frac18 \sum_{\sigma\in S_3} \delta^{p_1q_{\sigma(1)}} \delta^{p_2q_{\sigma(2)}} \delta^{p_3q_{\sigma(3)}} 
						\!-\! \frac14 \delta^{p_1q_{2}} \delta^{p_2q_{3}} \delta^{p_3q_{1}} \bigr).	
					\end{aligned}	
				\end{equation}
			\end{lem}
			Next, we present the results on traces of products of $\gamma$ and $\lambda$ matrices.
			\begin{lem}
				\begin{equation}\label{slady_gamma_lambda}
					\mathrm{Tr\ }(\gamma^p\gamma^q\lambda_+^r\lambda^s_-)=
					2^{n-2}\bigl(
					2\delta^{pq}\delta^{rs}
					+\delta^{ps}\delta^{qr}
					-\delta^{pr}\delta^{qs} \bigr)
					= 2^{n-1} \bigl(\delta^{pq}\delta^{rs}
					- \frac{1}{2} \varepsilon^{pq}_{rs} \bigr),
				\end{equation}
				\begin{equation}\label{corollary2}
					\mathrm{Tr\ }(\gamma^p\gamma^q\lambda_+^r
					\lambda_-^s\lambda_+^t\lambda_-^z) =
					2^{n-2} \delta^{pq}(\delta^{rs}\delta^{tz}+
					\delta^{rz}\delta^{st})
					-2^{n-3}(\delta^{rs}\varepsilon^{pq}_{tz}
					+\delta^{st}\varepsilon^{pq}_{rz}
					+\delta^{tz}\varepsilon^{pq}_{rs}
					+\delta^{rz}\varepsilon^{pq}_{st})
				\end{equation}	
			\end{lem}
			We skip the computational proof, which is based on expressing  $\gamma$-matrices in terms of $\lambda$-matrices, 
			\begin{equation*}
				\gamma^p\gamma^q= -\lambda_+^q\lambda_-^p + \lambda_+^p\lambda_-^q+\delta^{pq}+\dots,
			\end{equation*}
			and using the results of Lemma \ref{Tra1}.
			
As a consequence, we obtain the following identities for the geometric quantities:
			%%%%%%%%%%%%%%%%%%%%%%%%%%%%%%%%%%%%%%%%%%%%%%%%%%%%%%%%
			\begin{lem}\label{kawalki}
				%In {\color{magenta} 
					In normal coordinates  around $\mathbf{x}=0$ we have the following identities		
					\begin{equation}
						\begin{aligned}
							&	\mathrm{Tr\ }( \lambda_+^{p} \lambda_-^{q} ) Ric_{pq}=
							2^{n-1} R , \\
							& 	\mathrm{Tr\ }( \lambda_+^{p} \lambda_-^{q} )(R_{paqb}+R_{qapb})=
							2^n \mathrm{Ric}_{ab} ,\\
							& 	\mathrm{Tr\ }(\lambda_+^{p} \lambda_-^{q} \lambda_+^{r} \lambda_-^{s} )(R_{srqp}+R_{sqrp})=
							-2^{n-2} R ,\\
							& 	\mathrm{Tr\ }( \lambda_+^{p} \lambda_-^{q}\lambda_+^{r} \lambda_-^{s} )\mathrm{Ric}_{rs}=
							2^{n-2} (\delta_{pq}R + \mathrm{Ric}_{pq}) ,\\
							& 	\mathrm{Tr\ }( \gamma^{p} \gamma^{q}\lambda_+^{r} \lambda_-^{s} )
							\mathrm{Ric}_{rs}=
							2^{n-1} \delta_{pq}R ,\\
							& 	\mathrm{Tr\ }( \lambda_+^{p} \lambda_-^{q}\lambda_+^{r} \lambda_-^{s} )(R_{rasb}+R_{sarb})=
							2^{n-2} (2\delta_{pq}Ric_{ab} + R_{qapb} + R_{paqb}) ,\\
							& 	\mathrm{Tr\ }( \gamma^{p} \gamma^{q}\lambda_+^{r} \lambda_-^{s} )(R_{rasb}+R_{sarb})=
							2^{n} \delta_{pq}\mathrm{Ric}_{ab} ,\\
							& 	\mathrm{Tr\ }( \lambda_+^{p} \lambda_-^{q} \lambda_+^{r} \lambda_-^{s} \lambda_+^{t} \lambda_-^{z} )(R_{ztsr}+R_{zstr})=
							2^{n-3} (-R\delta_{pq}+2\mathrm{Ric}_{pq}) ,\\
							& 	\mathrm{Tr\ }( \gamma^{p} \gamma^{q}\lambda_+^{r} \lambda_-^{s} \lambda_+^{t} \lambda_-^{z} )(R_{ztsr}+R_{zstr})=
							-2^{n-2}\delta_{pq}R. 
						\end{aligned}
					\end{equation}		
				\end{lem}   
				\begin{proof}
					Direct computation using  \eqref{slady_naprzemienne}-\eqref{slady_gamma_lambda} and the
					properties of the Riemann and Ricci tensors.
				\end{proof}
				
				\vspace{3mm}
				
				\noindent {\bf Acknowledgements:} LD acknowledges that this work is part of the project Graph Algebras partially supported by EU grant HORIZON-MSCA-SE-2021 Project 101086394. 
				
				\begin{thebibliography}{xx}
					
					\bibitem{Co80} A.~Connes, {\it C* alg\`ebres et g\'eom\'etrie diff\'erentielle},C.\,R.\,Acad.\,Sci.\,Paris S\'er.\,A-B 290 A599--A604 (1980)
					
					\bibitem{Co94} A.~Connes, {\it Noncommutative Geometry},
					Academic Press 1994
					
					\bibitem{Co96}
					A.~Connes, {\it Gravity coupled with matter and the foundation of non-commutative geometry}, 
					Commun.Math. Phys. 182, 155-176 (1996)
					
					\bibitem{Co13} A.~Connes, {\it On the spectral characterization of manifolds}
					J.Noncommut.Geom., 7 (1), 1-82 (2013)
					
					
					\bibitem{CoMo95}
					A.~Connes and H.~Moscovici, {\it The local index formula in noncommutative geometry},  
					Geom. Funct. Anal. 5, 174-243 (1995).
					
					
					\bibitem{CoTr11} A.~Connes and P.~Tretkoff, 
					{\it The Gauss-Bonnet theorem for the noncommutative two torus},	In: {\it Noncomm.\,Geom,\,Arithmetic, and Related Topics}, 141--158 J. Hopkins University Press (2011)
					
					\bibitem{DDS18}
					L.~D\k abrowski, F.~D'Andrea, A.~Sitarz,
					{\it The Standard Model in noncommutative geometry: fundamental fermions as internal forms} ,
					Lett. Math. Phys., 108, 1323 (2018)
					
					\bibitem{DSZ23}
					L.~Dabrowski, A.~Sitarz, P.~Zalecki, 
					{\it Spectral Metric and Einstein Functionals}, Adv.Math., Vol. 427, (2023) 1091286. 
					
					\bibitem{DSZ23b}
					L.~Dabrowski, A.~Sitarz, P.~Zalecki, 
					{\it in preparation}
								
					\bibitem{Gi84}
					P.~B.~Gilkey, {\it Invariance theory, the heat equation and the Atiyah-Singer index theorem},
					Publish or Perish, Dilmington, (1984).
					
					\bibitem{Gi04}
					P.~B.~Gilkey, {\it Asymptotic Formulae in Spectral
						Geometry}, Chapman \& Hall/CRC, Boca Raton, FL (2004)
					
					\bibitem{Gu85}
					V. Guillemin, {\it A new proof of Weyl's formula on the asymptotic distribution of eigenvalues}, 
					Adv. Math. 55(2), 131--160 (1985)
					
					\bibitem{Ka66} M.~Kac, 
					{\it Can One Hear the Shape of a Drum?}, American Mathematical Monthly. 73, 4: 1–23 (1966)
					
					\bibitem{KaWa95}
					W.~Kalau, M.~Walze, {\it Gravity, non-commutative geometry and
						the Wodzicki residue}, J. Geom. Phys. 16, no. 4, 327--344  (1995)
					
					\bibitem{Ka95} 	D.~Kastler, {\it The Dirac operator and gravitation}, 	Commun.\ Math.\ Phys.\  {166}, 633 (1995)
					
					\bibitem{LRV12}
					S.~Lord, A.~Rennie, J.C.~V\'arilly,
					{\it Riemannian manifolds in noncommutative geometry},
					Journal of Geometry and Physics
					62, no. 7, 1611--1638 (2012) 
					
				
					\bibitem{MMMT} D.~Mitrea, I.~Mitrea, M.~Mitrea, 
					and M.~Taylor {\it The Hodge-Laplacian}, 
					De Gruyter Studies in Mathematics, (2016)
					
	
					\bibitem{Wo87}
					M.~Wodzicki, {\it Noncommutative residue. I. Fundamentals}, 
					in K-Theory, Arithmetic and Geometry (Moscow, 1984-1986), 
					Lecture Notes in Mathematics Vol. 1289 (Springer, Berlin, 1987), pp. 320--399.
					
				\end{thebibliography}
				
				
				
			\end{document}
