% Globally allow pagebreacks in formulas:
%\allowdisplaybreaks

% Page dimensions:
\renewcommand{\baselinestretch}{1.33} % There is some debate what "one and a half lines" means. This seems reasonable to me.
%\onehalfspacing
\setlength{\textwidth}{13.8cm}
%\calclayout
%\addtolength{\oddsidemargin}{.5cm}
%\addtolength{\evensidemargin}{-.5cm}msbook

% set epigraph quote length
\setlength{\epigraphwidth}{0.6\textwidth}

% Numbering of sections should be within chapters
\numberwithin{section}{chapter}
\numberwithin{equation}{section}

%% command to do Index and emph at same time
\newcommand{\emi}[1]{\index{#1}\emph{#1}}%

%% Theorem environments
\def\defthm#1#2#3#4{
	\newtheorem{#1}[theorem]{#3}
	\newtheorem*{#1*}{#3}
	\newtheorem{#2}[theorem]{#4}
	\newtheorem*{#2*}{#4}
	\crefname{#1}{#3}{#4}
	\crefname{#2}{#4}{#4}  
}


% Theorems, propositions, lemmas

\newtheoremstyle{mythm}% 
{10pt}% Space above 
{}% Space below 
{\itshape}% Body font 
{}% Indent amount 
{\bf}%  Theorem head font 
{.}% Punctuation after theorem head 
{.5em}% Space after theorem head 
{}% 

% Definitions

\newtheoremstyle{mydef}% 
{10pt}% Space above 
{3pt}% Space below 
{}% Body font 
{}% Indent amount 
{\bf}%  Theorem head font 
{.}% Punctuation after theorem head 
{.5em}% Space after theorem head 
{}% 

% Remarks 

\newtheoremstyle{myrmk}% 
{10pt}% Space above 
{3pt}% Space below 
{}% Body font 
{}% Indent amount 
{\bf}%  Theorem head font 
{.}% Punctuation after theorem head 
{.5em}% Space after theorem head 
{}% 


\theoremstyle{mythm}
\newtheorem{theorem}{Theorem}[section]
\newtheorem*{theorem*}{Theorem}


\defthm{corollary}{corollaries}{Corollary}{Corollaries}
\defthm{lemma}{lemmata}{Lemma}{Lemmata}
\defthm{proposition}{propositions}{Proposition}{Propositions}
\defthm{prop}{props}{Proposition}{Propositions}

\theoremstyle{mydef}
\defthm{definition}{definitions}{Definition}{Definitions}

\theoremstyle{myrmk}
\defthm{remark}{remarks}{Remark}{Remarks}
\defthm{example}{examples}{Example}{Examples}
\defthm{question}{questions}{Question}{Questions}
\defthm{claim}{claims}{Claim}{Claims}
\defthm{conjecture}{conjectures}{Conjecture}{Conjectures}


% repeating a statement (useful if you want to state a main result in the introduction)
% see http://tex.stackexchange.com/a/443
\newtheorem*{replemmax}{\reptitle}
\newenvironment{replemma}[1]{%
	\def\reptitle{Lemma \ref*{#1}}%
	\begin{replemmax}}%
	{\end{replemmax}}

% Reference names 

\crefname{section}{Section}{Sections}
\crefname{theorem}{Theorem}{Theorems}

% Proofs 

\makeatletter
\renewenvironment{proof}[1][\proofname] {\par\pushQED{\qed}\normalfont\topsep6\p@\@plus6\p@\relax\trivlist\item[\hskip\labelsep\bf#1\@addpunct{.}]\ignorespaces}{\popQED\endtrivlist\@endpefalse}
\makeatother

% Some macros

\newcommand{\ie}{\text{i.e.\ }}
\newcommand{\eg}{\text{e.g.\ }}
\newcommand{\cf}{\text{cf.\,}}
\newcommand{\resp}{\text{resp.\ }}
\newcommand{\etal}{\text{et al.\ }}
\newcommand{\etals}{\text{et al.'s\ }}
\newcommand{\etc}{\text{etc.\ }}


\crefname{diagram}{diagram}{diagrams}

\newcommand\mc[1]{\hl{\textbf{#1}}}

%%%%%%%%%%%%%%%%%%%%%%%%%%%%%%%%%%%%%%%%%%%%%%%%%%%%%%%%%%%%%%%%%%%%%%%%%%%%%%

% Bibliography style: choose one and make sure you have the relevant .bst file
%\bibliographystyle{dcu}
%\bibliographystyle{leeds}
%\bibliographystyle{amsplain}
\bibliographystyle{ourbib}
%\bibliographystyle{alpha}

%%%%%%%%%%%%%%%%%%%%%%%%%%%%%%%%%%%%%%%%%%%%%%%%%%%%%%%%%%%%%%%%%%%%%%%%%%%%%%
\newcommand{\Gpd}{\ensuremath{\mathbf{Gpd}}}
\newcommand{\Gr}[1]{\ensuremath{\mathbb{#1}}}

\newcommand{\norm}[1]{\left\lVert#1\right\rVert}

\newcommand{\Ab}{\ensuremath{\mathbf{Ab}}}
\newcommand{\algS}[1]{\ensuremath{#1\text{-}\mathrm{alg}_s}}
\newcommand{\alg}[1]{\ensuremath{#1\text{-}\mathrm{alg}}}
\newcommand{\algR}[2]{\ensuremath{#1\text{-}\mathrm{alg}^{#2}}}
\newcommand{\SubAlg}{\ensuremath{\operatorname{\mathbf{SubAlg}}}}
\newcommand{\RGraph}{\ensuremath{\mathbf{RGraph}}}
%\newcommand{\Grpd}{\ensuremath{\mathbf{Grpd}}}
\newcommand{\Cat}{\ensuremath{\mathbf{Cat}}}
\newcommand{\Set}{\ensuremath{\mathbf{Set}}}
\newcommand{\ob}{\ensuremath{\operatorname{ob}}}
\newcommand{\Fib}{\ensuremath{\operatorname{Fib}}}
\newcommand{\SFib}{\ensuremath{\operatorname{SFib}}}
\newcommand{\Poly}{\ensuremath{\operatorname{Poly}}}
\newcommand{\PolyFun}{\ensuremath{\operatorname{PolyFun}}}
\newcommand{\Sp}{\ensuremath{\operatorname{Sp}}}
\newcommand{\op}{\ensuremath{\operatorname{op}}}
\newcommand{\Nat}{\ensuremath{\operatorname{Nat}}}

\newcommand{\A}{\ensuremath{\mathcal{A}}}
\newcommand{\B}{\ensuremath{\mathcal{B}}}
\newcommand{\C}{\ensuremath{\mathcal{C}}}
\newcommand{\D}{\ensuremath{\mathcal{D}}}
\newcommand{\E}{\ensuremath{\mathcal{E}}}
\newcommand{\K}{\ensuremath{\mathbf{K}}}
\newcommand{\Q}{\ensuremath{\mathbf{Q}}}
\newcommand{\X}{\ensuremath{\mathbf{X}}}
\newcommand{\V}{\ensuremath{\mathcal{V}}}
\newcommand{\R}{\ensuremath{\mathbf{R}}}


\newcommand{\W}{\ensuremath{\mathbf{W}}}
\newcommand{\id}{\ensuremath{\operatorname{id}}}
\newcommand{\dom}{\ensuremath{\operatorname{dom}}}
\newcommand{\N}{\ensuremath{\mathbb{N}}}
\newcommand{\rank}{\ensuremath{\operatorname{rank}}}
\newcommand{\im}{\ensuremath{\operatorname{im}}}
\newcommand{\U}{\ensuremath{\mathcal{U}}}
\newcommand{\Ext}{\operatorname{Ext}}

\newcommand{\Ag}{\ensuremath{\mathbb{A}}}
\newcommand{\Bg}{\ensuremath{\mathbb{B}}}
\newcommand{\Cg}{\ensuremath{\mathbb{C}}}
\newcommand{\Dg}{\ensuremath{\mathbb{D}}}

\newcommand*\quot[2]{{{#1}_{/#2}}}
\newcommand*\bquot[2]{{{\textstyle #1}\sslash_{\textstyle #2}}}

\newcommand{\Types}{\textsc{Types}}
\newcommand{\Func}{\textsc{Func}}
\newcommand{\Rel}{\textsc{Rel}}
\newcommand{\Var}{\textsc{Var}}
\newcommand{\Terms}{\textsc{Terms}}
%\newcommand{\Form}{\textsc{Form}}
\newcommand{\unit}{\mathbf{1}}
\newcommand{\FV}{\operatorname{FV}}
\newcommand{\Grpd}{\mathbf{Grpd}}
\newcommand{\PAlg}{\operatorname{P-\mathbf{Alg}}}
\newcommand{\ttype}{\operatorname{\mathtt{type}}}
\newcommand{\tctx}{\operatorname{\mathtt{ctx}}}
\newcommand{\Id}{\operatorname{Id}}
\newcommand{\Refl}{\operatorname{Refl}}
%\renewcommand{\C}{\mathbf{C}}
\newcommand{\hC}{\hat{\C}}
\newcommand{\sSet}{\mathbf{sSet}}
\newcommand{\Fam}{\mathbf{Fam}}
\newcommand{\y}[1]{\operatorname{\textsf{y}}#1}
\newcommand{\App}{\operatorname{App}}
\newcommand{\app}{\operatorname{app}}
\newcommand{\Pair}{\operatorname{Pair}}
\newcommand{\splt}{\operatorname{split}}
\renewcommand{\U}{\mathcal{U}}
\newcommand{\wU}{\widetilde{\U}}
\newcommand{\hSigma}{\hat{\Sigma}}
\newcommand{\hlambda}{\hat{\lambda}}
\newcommand{\hPi}{\hat{\Pi}}
\newcommand{\father}{\operatorname{father}}
\newcommand{\Hom}{\operatorname{Hom}}

\newcommand\defeq{=_{\mathrm{def}}}

% Bibtex hacks:
\providecommand{\noopsort}[1]{}

% Title page decls:
\makeatletter
\def\degreedate#1{\gdef\@degreedate{#1}}
% The full (unabbreviated) name of the degree
\def\degree#1{\gdef\@degree{#1}}
% The name of your college or department(eg. Trinity, Pembroke, Maths, Physics)
\def\collegeordept#1{\gdef\@collegeordept{#1}}
% The name of your University
\def\university#1{\gdef\@university{#1}}
% Defining the crest
\def\crest#1{\gdef\@crest{#1}}
% Defining the logo
\def\logo#1{\gdef\@logo{#1}}
% Defining the departmental logo
\def\deptlogo#1{\gdef\@deptlogo{#1}}
\makeatother

% Version control - by adding DRAFT DATE
\newcommand{\draftdate}{Version $\gamma$ as of \today}



%%%%%%%%%%%%%%%%%%%%%%%%%%%%%%%%%%%%%%%%%%%%%%%%%%%%%%
%%%%%%%%%%%%% Constants - fill these in for use throughout the thesis %%%%%%%%%%%%
%%%%%%%%%%%%%%%%%%%%%%%%%%%%%%%%%%%%%%%%%%%%%%%%%%%%%%
%%%%%%%%%%%%%%%%%%%%%%%%%%%%%%%%%%%%%%%%%%%%%%%%%%%%%%
%%%%%%%%%%%%% Title Page Information %%%%%%%%%%%%%%%%%%%%%%%%%%%%
%%%%%%%%%%%%%%%%%%%%%%%%%%%%%%%%%%%%%%%%%%%%%%%%%%%%%%
\title{Computability and Tiling Problems}
\author{Mark Richard Carney}
\crest{% Figure removed}
%%%%%%%%%%%%%%%%%%%%%%%%%%%%%%%%%%%%%%%%%%%%%%%%%%%%%%
%% Define these as empty to omit the two logos on the title page
%%%%%%%%%%%%%%%%%%%%%%%%%%%%%%%%%%%%%%%%%%%%%%%%%%%%%%
\logo{% Figure removed} %University Logo
\deptlogo{} %% Figure removed} % Institute Logo
%%%%%%%%%%%%%%%%%%%%%%%%%%%%%%%%%%%%%%%%%%%%%%%%%%%%%%
\collegeordept{\href{https://physicalsciences.leeds.ac.uk/info/6/school_of_mathematics}{School of Mathematics}}
\university{\href{http://www.leeds.ac.uk}{University of Leeds}}

\degree{Doctor of Philosophy}
\newcommand{\submittedtext}{{Submitted in accordance with the requirements for the degree of}}

\degreedate{October 2019}
