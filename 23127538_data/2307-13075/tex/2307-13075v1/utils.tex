%%%%%%%%%%%%%%%%%%%%%%%%%%%%%%%%%%%%%%%%%%%%%
%             CUSTOMISATIONS - TIKZ DRAWING ETC.
%%%%%%%%%%%%%%%%%%%%%%%%%%%%%%%%%%%%%%%%%%%%%

\usepackage{tikz}

\newcommand{\sampletile}[4]{\scalebox{.75}{
\tikz[scale=1.7,label distance=2.5mm]{
\draw[fill=white] (1.6,1.6) rectangle (0,0);
\filldraw[fill=white] (0,0) -- (0.8,0.8) -- (0,1.6) -- cycle;
\filldraw[fill=white] (0,0) -- (0.8,0.8) -- (1.6,0) -- cycle;
\filldraw[fill=white] (0,1.6) -- (0.8,0.8) -- (1.6,1.6) -- cycle;
\node at (0.8,0.8) [label=above:{#2},label=left:{#1},label=right:{#3},label=below:{#4}] {};}
}}

\newcommand{\sampletilefarlabels}[4]{\scalebox{.75}{
\tikz[scale=1.7,label distance=4.5mm]{
\draw[fill=white] (1.6,1.6) rectangle (0,0);
\filldraw[fill=white] (0,0) -- (0.8,0.8) -- (0,1.6) -- cycle;
\filldraw[fill=white] (0,0) -- (0.8,0.8) -- (1.6,0) -- cycle;
\filldraw[fill=white] (0,1.6) -- (0.8,0.8) -- (1.6,1.6) -- cycle;
\node at (0.8,0.8) [label=above:{#2},label=left:{#1},label=right:{#3},label=below:{#4}] {};}
}}

\newcommand{\sampletilenearlabels}[4]{\scalebox{.75}{
\tikz[scale=1.7,label distance=1.5mm]{
\draw[fill=white] (1.6,1.6) rectangle (0,0);
\filldraw[fill=white] (0,0) -- (0.8,0.8) -- (0,1.6) -- cycle;
\filldraw[fill=white] (0,0) -- (0.8,0.8) -- (1.6,0) -- cycle;
\filldraw[fill=white] (0,1.6) -- (0.8,0.8) -- (1.6,1.6) -- cycle;
\node at (0.8,0.8) [label=above:{#2},label=left:{#1},label=right:{#3},label=below:{#4}] {};}
}}


% colours in the next sampletilefill use <l,u,r,b> format as before
\newcommand{\sampletilefill}[4]{\scalebox{.75}{
\tikz[scale=1.7,label distance=2.5mm]{
\draw[fill=#3] (1.6,1.6) rectangle (0,0);
\filldraw[fill=#1] (0,0) -- (0.8,0.8) -- (0,1.6) -- cycle;
\filldraw[fill=#4] (0,0) -- (0.8,0.8) -- (1.6,0) -- cycle;
\filldraw[fill=#2] (0,1.6) -- (0.8,0.8) -- (1.6,1.6) -- cycle;
{};}
}}

\usetikzlibrary{shapes.geometric}
\newcommand\fillshape[3]{ % #1 = shape, #2 = filename of texture, #3 = includegraphics options
    \begin{scope}
        \clip #1;
        \node {% Figure removed};
    \end{scope}
    \draw[line width=1mm] #1;
}

\newcommand\fillshapesimple[1]{ % #1 = shape, #2 = filename of texture, #3 = includegraphics options
    \begin{scope}
        \clip #1;
    \end{scope}
    \draw[line width=0.2mm] #1;
}

\newcommand\lozengetile[5]{
    \begin{scope}
        \clip #1;
    \end{scope}
    \draw[line width=0.1mm] #1;
\draw[line width=0.2mm] (30:0.66) node {#2};
\draw[line width=0.2mm] (15:1.33) node {#3};
\draw[line width=0.2mm] (330:0.66) node {#5};
\draw[line width=0.2mm] (345:1.33) node {#4};
}
\newcommand\hexagontile[4]{
  \begin{scope}
    \clip (0:0) -- (180:2) -- (120:2) -- (0:0) -- (120:2) -- (60:2) -- (0:0) -- (60:2) -- (0:2) -- (0:0) -- (0:2) -- (300:2) -- (240:2) -- (180:2) -- cycle;
  \end{scope}
  \draw[line width=0.2] (0:0) -- (180:2) -- (120:2) -- (0:0) -- (120:2) -- (60:2) -- (0:0) -- (60:2) -- (0:2) -- (0:0) -- (0:2) -- (300:2) -- (240:2) -- (180:2) -- cycle;
\draw (150:1) node {$#1$};
\draw (90:1) node {$#2$};
\draw (30:1) node {$#3$};
\draw (270:1) node {$#4$};
}

% this is so I can number in align* as necessary...
\newcommand\numberthis{\addtocounter{equation}{1}\tag{\theequation}}
