\chapter{$\Sigma^1_1$-Complete Tilings}
\label{chap4}
% next resets the equation numbers to start at 1 at the start of the chapter
\setcounter{equation}{0}
\renewcommand{\theequation}{\thechapter.\arabic{equation}}

%------------------------------------------------------------------------------

\epigraph{I could be bounded in a nutshell and count myself king of infinite space.}{\textit{Hamlet}}

In this chapter we present our main results that concern infinite sets of Wang prototiles, and relate these to problems on infinite trees in Baire space. Previous work in tilings has generally considered finite sets of prototiles - and this is a natural assumption to make about things that we ostensibly only consider to be of finitely-many possibilities. 

The difference, as we shall see, is that by allowing our tilings as functions $f:\mathbb{Z}^2 \rightarrow \mathcal{S}$ to range over infinite prototiles, the original Domino Problem \ref{def:domprob} becomes equivalent, after careful construction, to whether a tree corresponding to our tiling is well-founded or ill-founded. As we found that finite sets of prototiles are equivalent to the Halting Problem, so we construct this new equivalence in this chapter.

We then extend this result to a variation of the Domino Problem - the problem of `weakly tiling' the plane, as well as an analogous notion of `strongly not tiling' the plane.

\section{Computable Trees and Computable Tilings}

In the section that follows, we will need the following in order to prove theorem \ref{thm:TILE-ILL}. First, we define what we mean by computable tilings. Recall that we represent by $\langle l,u,r,b \rangle$ the Wang prototile
\begin{center}
\sampletile{$l$}{$u$}{$r$}{$b$}
\end{center}
We define a computable set of Wang prototiles as follows:

\begin{definition}\index{computable prototile sets}\index{computable tile sets}
Let $X \subset \omega$, and $\mathcal{S}$ be a set of Wang prototiles.
\begin{itemize}
\item Let $X_\mathcal{S} = \{ \langle c_l, c_u, c_r, c_b \rangle : \langle  c_l, c_u, c_r, c_b \rangle \text{ codes some prototile in } \mathcal{S} \}$.
\item We say that $\mathcal{S}$ is \emph{computable} if $X_\mathcal{S}$ is computable.
\item We say that an $\mathcal{S}$-tiling of the plane is computable if $f_\mathcal{S} : \mathbb{Z}^2 \rightarrow \mathcal{S}$ is computable.
\item We say that $\mathcal{S}$ is \emph{total} if for every point $(x,y) \in \mathbb{Z}^2$ and a tiling function $f: \mathbb{Z}^2 \rightarrow \mathcal{S}$, $f$ is total on $\mathbb{Z}^2$, all edge conditions are met for any $\mathcal{S}$-tiling.
\end{itemize}
\end{definition}

\section{$\Pi^1_1$ Properties of Tilings}

In this section we will cover previous work on the $\Pi^1_1$ nature of specified Domino Problems that inquire about the properties of tile occurrences in planar tilings.

\subsection{Harel's $\Pi^1_1$ Tilings}

David Harel in \cite{Harel1986} was interested in translations between various kinds of computable trees. The core idea is to formulate correspondences between finitely branching and countably infinitely branching trees and infinitely branching tress, one-to-one, such that the paths along the latter become ``$\varphi$-abiding" paths of the former, for $\varphi$ being some property of infinite paths.

Harel in \cite{Harel1986} proposes the following problem relating to Wang prototile sets:

\begin{definition}[Recurring Tile Problem]\index{recurring domino problem}\label{def:RecDP}
Given a set of prototiles $\mathcal{S}$, for $t \in \mathcal{S}$, does $t$ occur infinitely often in a tiling of the lattice $\mathbb{Z}^2$?
\end{definition}

This is a variation on the standard Domino problems that we have considered so far. Rather than ask ``do there exist planar tilings?" we ask ``do any planar tilings have a given property?" The property in this case is a weaker question than ``are all $S$-tilings periodic or aperiodic?" - something we will come to discuss later in this thesis.

Harel in \cite{Harel1986} goes on to prove the following theorem:

\begin{theorem}[\cite{Harel1986}, Theorem 6.3]\label{thm:RecDomProbSigma}
The Recurring Tile Problem is $\Sigma^1_1$-complete.
\end{theorem}

We first require the following definition and lemmas from \cite{Harel1986}:

\begin{definition}
A class $A$ is \emph{$\Sigma^1_1$-hard} if there is a computable way of converting any $\Sigma^1_1$ formula into some member of $A$.
\end{definition}

\begin{definition}
A tree $T$ is an $\omega$-tree if $T \subseteq \omega^{< \omega}$. A $k$-tree is a tree $T \subseteq \{0,1,\ldots,k-1\}^{< \omega}$ for some finite $k \in \omega$. If such a \emi{$k$-tree} $T$ is bounded by some $b \in \omega$ then it is a \emi{$b$-tree}. We say that a \emph{recurrence} in a $b$-tree is the repetition of some specific $i \in \{ 0, \ldots, k-1 \}$ along an infinite path. 
\end{definition}

For graph-theoretic trees, this is equivalent to some of the non-leaf nodes being marked, and a recurrence being infinitely many marked nodes along some infinite path in the tree. 

\begin{lemma}[\cite{Harel1986}, p.230]\label{lemma:InfTreeRec}
The set $A$ of computable well-founded $\omega$-trees is computably isomorphic to the set $B$ of computable marked recurrence-free $b$-trees.
\end{lemma}

This lemma then sets the scene for the following theorem:

\begin{theorem}[\cite{Harel1986}, Lemma 6.1]\label{lemma:AequivC}
Let $A$ be the set of computable well-founded $\omega$-trees, and let $C$ be the set of enumerated notation for all Non-deterministic Turing Machines (NTMs). Then $$ A \equiv_1 C $$
\end{theorem}

Recalling our definition of 1-reducibility in definition \ref{def:mred}, and let $A \equiv_1 B$ iff $A \leq_1 B$ and $B \leq_1 A$. A proof of this is found in \cite{Harel1986}. From here we get:

\begin{corollary}[\cite{Harel1986}, Corollary 6.2]
$C$ is $\Pi^1_1$ complete.
\end{corollary}

The intuition behind these results is to set the stage that the question:

\noindent
\textbf{C1}: ``for a given NTM $U$, does $U$ re-enter its starting state $q_0$ infinitely often?" 

\noindent 
is a $\Sigma^1_1$ link to our Recurring Tile Problem above (\textbf{RTP}). The proof of \ref{thm:RecDomProbSigma} thus proceeds as follows:

\begin{proof}[Proof of \ref{thm:RecDomProbSigma}]
To first see that \textbf{RTP} is $\Sigma^1_1$, let $\mathcal{S}$ and some $t \in \mathcal{S}$ be given. Construct and NTM $M$ that begins on a blank tape by initially constructing a blank tiling of $\mathbb{Z}^2$. At each step, $M$ iterates over the $\mathbb{Z}^2$ lattice in a spiral pattern, considering each point in turn. Non-deterministically, $M$ tries to tile each position with some tile from $\mathcal{S}$. $M$ rejects if the edge conditions fail to match, and signals a successful use of the tile $t$ by re-entering its starting state $q_0$. Otherwise, $M$ will never re-enter $q_0$. Thus, $M$ has the property \textbf{C1} iff $t$ occurs infinitely often in the $\mathcal{S}$-tiling.

The rest of the proof is showing that \textbf{RTP} is $\Sigma^1_1$-hard. This is done through the following three claims. First, define \textbf{R2} as follows:

\noindent
\textbf{R2} - Given $\mathcal{S}$ and $t \in \mathcal{S}$, can $\mathcal{S}$ tile the positive quadrant of $\mathbb{Z}^2$ with $t$ occurring infinitely often and with the borderlines coloured white?

\begin{claim}\label{claim:R2}
\textbf{R2} is $\Sigma^1_1$-hard.
\end{claim}

\begin{proof}[Proof of \ref{claim:R2}]
We sketch the following proof of this claim. By theorem \ref{lemma:AequivC} we have that for an NTM $M$ that computes from the right, the question of whether it enters its initial $q_0$ infinitely often will be a $\Sigma^1_1$-hard problem, as it will be equivalent to the well-foundedness of some $\omega$-tree. 

We then construct a tile set from a scheme such that for each $M$, the tile set we build from $M$ has the property \textbf{R2} iff $M$ has the property above.

Let $M$ be given, reserving $B$ as the `blank' symbol, and let $p,q$ be states, and $s,t$ be tape symbols, all in NTM quintuples as defined in chapter 1. Our prototile set $\mathcal{S}$ will consist of tiles generated by the schema defined in the proof of theorem \ref{thm:TMTilings}.

Given our translation of $M$ into tiles preserves the recurrent properties of $M$, if $M$ enters its starting state $q_0$ infinitely often, then the tile representing this will occur infinitely often in the tiling, so $\mathcal{S}$ satisfies \textbf{R2}, with the white borders guaranteed by substituting the blank colour $B$ for plain white quadrants in our prototiles.
\end{proof}

We modify \textbf{R2} to the following statement:

\noindent
\textbf{R3} - Given $\mathcal{S}$ and $t \in \mathcal{S}$, can $\mathcal{S}$ tile the positive quadrant of $\mathbb{Z}^2$ with $t$ occurring infinitely often?

\begin{claim}\label{claim:R3}
\textbf{R3} is $\Sigma^1_1$-hard.
\end{claim}

\begin{proof}[Proof of \ref{claim:R3}]

Note that the border requirement in the previous claim was intended to force the initial starting state tile giving $q_0$ to appear in the right place. Consider the following machine problem:

\noindent
\textbf{C2} - Given NTM $M$, is there some tape configuration and state such that the following computation does not halt and re-enters $q_0$ from the right onto a blank tape cell infinitely often?

\textbf{C2} is $\Sigma^1_1$-hard by theorem \ref{lemma:AequivC} and the observation that a machine can be run from any starting tape configuration and state. We now adjust our schema prototiles as follows in order to produce prototiles for our $\mathcal{S}$ as follows:

For all symbols $s \in \Sigma$:

\begin{center}
\sampletile{$\rightarrow$}{$s$}{$\rightarrow$}{$s$} \sampletile{$\leftarrow$}{$s$}{$\leftarrow$}{$s$}
\end{center}

For symbols $s, s^\prime \in \Sigma$ and $q_i, q_j \in Q$:

\begin{center}
	\sampletile{${\rightarrow \atop q_i}$}{$s$}{$\leftarrow$}{$(s,q_i)$} \sampletile{$\rightarrow$}{$s$}{${\leftarrow \atop q_i}$}{$(s,q_i)$}
\end{center}

\begin{center}
	\sampletile{${\rightarrow \atop q_j}$}{$(s,q_i)$}{$\rightarrow$}{$s^\prime$} \sampletile{$\leftarrow$}{$(s,q_i)$}{${ \leftarrow \atop q_j}$}{$s^\prime$}
\end{center}

Fix $t$ to be 

\begin{center}
	\sampletile{$\rightarrow$}{$$B$$}{${\leftarrow \atop q_0}$}{$(B,q_0)$}
\end{center}

The addition of the arrows forces patterns of the form $$\cdots \rightarrow \rightarrow \leftarrow \leftarrow \cdots  $$

This is intended to force only one state to appear on each row in our NTM tiling. Thus $t$ occurring just once forces exactly one state per row, and so $(\mathcal{S}, t)$ satisfies \textbf{R3} iff $M$ satisfies \textbf{C2}. 
\end{proof}

To complete our proof, we need to extend these tilings out from one quadrant to full planar tilings. First, note that our NTM tapes are bi-infinite two way tapes, so we can extend our $\cdots \rightarrow \rightarrow \leftarrow \leftarrow \cdots$ pattern to the left half of the plane easily.

Extending to the upper half-plane is trickier. Note that there is nothing that requires $M$ to have infinite computations in the forwards or backwards directions by default. We can fix the backwards direction by requiring that $M$ will return repeatedly into some state $q_i$, requiring that $q_i \neq q_0$.

Likewise, we can prevent $\mathcal{S}$ from having $t$ appear infinitely often upwards but nowhere appearing downwards by having $M$ hold a counter variable that is incremented each time $M$ enters $q_0$. Thus, a planar tiling with infinitely many $q_0$ in the upper half of the grid would indicate a computation that checks the presence of increasingly smaller positive integers, which is impossible. 

Thus, for these modified machines, $M$ satisfies \textbf{C2} iff $(\mathcal{S},t)$ satisfies \textbf{RTP}. This completes our sketch of this proof for \ref{thm:RecDomProbSigma} from \cite{Harel1986}.

\end{proof}

In the following sections, we will deviate from asking if the Recurring Tile Problem from definition \ref{def:RecDP} is $\Sigma^1_1$, and instead ask if we can find some $\Pi^1_1$ properties that are equivalent to the original Domino Problem (\ref{def:domprob}).

\section{Domino Problems for Infinite Computable Sets of Prototiles}

Next, we will define our class of prototiles sets with total planar tilings as to not restrict ourselves to finite sets of prototiles. To this end, we define the set $TILE$ that will range over infinite sets of Wang prototiles.

\begin{definition}\index{$TILE$}\label{def:TILE}
\begin{align*}
TILE = \{ e : \, & \varphi_e \text{ is the characteristic function of some infinite}  \\
	      & \text{ Wang prototile set whose tilings are total in the plane.} \}
\end{align*}
\end{definition}

It is natural from our definition of $TILE$ that for any $e \in TILE$, the tiling that is generated by $e$ must be connected and infinite. 

We also define set $ILL$ which we will use later to get our $\Sigma^1_1$-completeness of $TILE$.

\begin{definition}\label{def:ILL}
$$ ILL = \{ e : \, \varphi_e \text{ is the characteristic function of an ill-founded tree } T \subseteq \omega^{< \omega} \} $$
\end{definition}
Note that by proposition \ref{prop:wellfddR}, specifically the converse argument, $ILL$ is $\Sigma^1_1$-complete.

\subsection{Filter for Computable Trees}

In order to adequately satisfy \ref{thm:TILE-ILL}, it is critical that our computable functions $\Phi_e$ do indeed actually compute trees. As such, we will need the following lemma to `filter out' the functions that do not compute trees.

\begin{lemma}[C. 2019]\label{lemma-pproc}
There is a computable $g: \omega \rightarrow \omega$ such that for every characteristic function $\varphi_e$ of some set $T \subseteq \omega^{<\omega}$:
\begin{enumerate}
\item if $\varphi_e$ is a tree, then $\varphi_{g(e)}$ is the same tree.
\item if $\varphi_e$ is total but not a tree, then $\varphi_{g(e)}$ is not total.
\item if $\varphi_e$ is not total then $\varphi_{g(e)}$ is not total.
\end{enumerate}
\end{lemma}

\begin{proof}
For any $\varphi_e$ define $g(e)$ as follows:
$$ 
    \varphi_{g(e)}(\sigma)= 
\begin{cases}
    1 & \text{if } \forall \tau \subseteq \sigma \  (\varphi_e(\tau)=1) \\ %notice the single backslash spacer before \Phi...
    0 & \text{if } \exists \tau \subseteq \sigma \text{ s.t. } \\ 
    & \ \ \ \forall \eta (\eta \subset \tau \rightarrow \varphi_e(\eta) = 1 \land \tau \subseteq \eta \rightarrow \varphi_e(\eta)=0 ) \\
    \uparrow & \text{otherwise}
\end{cases} $$
\end{proof}

\section{$\Pi^1_1$ and $\Sigma^1_1$ Domino Problems}

We will now present our results that show some equivalences between the domino problem for infinite prototile sets and well-founded trees.

\subsection{Equivalences to $TILE$}

\begin{theorem}[C. 2019]\label{thm:TILE-ILL}\index{$TILE$ equivalence to $ILL$}
\[ TILE \equiv_m ILL \]
\end{theorem}

\begin{proof}
Firstly, we note that it follows from $\Sigma^1_1$-completeness of $ILL$ that anything $ILL$ is $m$-reducible to will be $\Sigma^1_1$-complete as well, and so anything in $ILL$ will likewise be found in the set we are reducing to. Thus, we get the converse $m$-equivalence essentially `for free' from this fact and a opposite argument to that found in lemma \ref{lemma:wellfddTreesPi11}.

As such, it suffices to prove $ILL \leq_m TILE$. For this, we will follow the shape of regular $m$-reducibility proofs, and show that there is a computable function $h$ such that $$\forall e (x \in ILL \iff h(x) \in TILE)$$.

We first fix the following colours/symbols:
\begin{itemize}
\item Let $\lambda$ denote the empty string, and let $\lambda^U, \lambda^D$ be unique colours.
\item Fix $M^L_0$ and $M^R_0$ as unique colours.
\item Fix unique colours for all $M_i$ for $i \in \omega$.
\item For $j \in \{1,2,3,4\}$ and $i \in \omega$, let each $c^j_i$ be unique colours.
\item Let $\alpha \in \omega^\omega$ be an infinite string, and for all $i \in \omega$ let $\sigma_i \in \omega^{< \omega}$ denote successive initial segments of $\alpha$ of length $i$ such that $\sigma_0 \prec \sigma_1 \prec \ldots \sigma_i \prec \ldots \prec \alpha$.
\item Let $\sigma_0 = \lambda$ by this notation.
\item For $\sigma \in \omega^{< \omega}$, let $\sigma^\frown n$ denote $\sigma$ concatenated with $n$ as defined before for some $n \in \omega$, and let $| \sigma |$ denote the length of $\sigma$.
\end{itemize}
With these defined, let $e \in ILL$ be given. We will construct the following schema tiles:

We start with the \textbf{root tile}:
\begin{center}
\sampletile{$M^L_0$}{$\lambda^U$}{$M^R_0$}{$\lambda^D$}
\end{center}

Next, we require {\bf column tiles}:

\begin{center}
\sampletile{$c^1_{i+1}$}{$\sigma_i^\frown n$}{$c^2_{i+1}$}{$\sigma_i$}
\sampletile{$c^4_{i+1}$}{$\sigma_i$}{$c^3_{i+1}$}{$\sigma_i^\frown n$}
%\sampletile{$c^4_{\left| \sigma_i \right|}$}{$\sigma_i$}{$c^3_{\left| \sigma_i \right|}$}{$\sigma_i^\frown n$}
\end{center}


We also define {\bf mid-row} tiles to be:

\begin{center}
\sampletilenearlabels{$M_{i+1}$}{$c^1_{i+1}$}{$M_i$}{$c^4_{i+1}$}
\sampletilenearlabels{$M_i$}{$c^2_{i+1}$}{$M_{i+1}$}{$c^3_{i+1}$}
\end{center}

We shall additionally define the following diagonal {\bf quadrant filling} tiles:

\begin{center}
\sampletile{$c^1_{i+1}$}{$c^1_{i+1}$}{$c^1_i$}{$c^1_i$}
\sampletile{$c^2_{i}$}{$c^2_{i+1}$}{$c^2_{i+1}$}{$c^2_i$}
\sampletile{$c^3_{i}$}{$c^3_{i}$}{$c^3_{i+1}$}{$c^3_{i+1}$}
\sampletile{$c^4_{i+1}$}{$c^4_{i}$}{$c^4_{i}$}{$c^4_{i+1}$}
\end{center}

We now construct a `library' $\mathcal{S}$ from which we will select the prototiles we need. To generate $\mathcal{S}$ we take all of the colours we fixed at the start of the proof, and colour the schema tiles as follows:
\begin{itemize}
\item We colour the root tile with the tuple $\langle M^L_0, \lambda^U, M^R_0, \lambda^D \rangle$ and put this tile into $\mathcal{S}$.
\begin{itemize}
\item \textbf{NB} - our root tile has distinctions for up/down and left/right in order to prevent trivial $S_e$-tilings using only the root tile.
\end{itemize}
\item For all the $c^j_i$ and $M_i$ colour the mid-row tiles.
\begin{itemize}
\item We must be careful to put the $M^L_0$ and $M^R_0$ tiles such that they will tile from the root tile.
\item specifically, we add the tiles $\langle M_1, c^1_1, M^L_0, c^4_1 \rangle$ and $\langle M^R_0, c^2_1, M_1, c^3_1 \rangle$.
\end{itemize}
\item For all $c^j_i$ colour all of the quadrant tiles, and put these into $\mathcal{S}$.
\end{itemize}

What now remains is to colour the column tiles and add the required ones to $\mathcal{S}$. To do this we will need to take our $e$ and ensure that it has been put through our pre-processing lemma \ref{lemma-pproc} in order to ensure it is a tree.

With this done, we have an $h$ that we will now use to construct a set of prototiles $S_e \subset \mathcal{S}$ as follows:
\begin{itemize}
\item Select all of the mid-row and quadrant filling tiles, along with the root tile, and add these into $S_e$. 
\item Next add all of the column tiles for all $\sigma_n \in \omega^{< \omega}$ such that $\varphi_e(\sigma_n)=1$.
\end{itemize}

We choose all of the column tiles such that there are two copies of each $\sigma_n \text{ such that } \varphi_e(\sigma_n) = 1$; one copy going up from the root tile, with $\sigma_0 = \lambda^U$ and one going down from the root tile with $\sigma_0 = \lambda^R$.

We now want to verify that for each $e \in ILL$ we will get an $S_e$ such that there exist $S_e$ tilings of the plane, giving $h(e) \in TILE$.

To see this, we first note that the quadrant tiles, root tile, and mid-row tiles form a near-complete tiling of the plane. Without the column tiles, we can tile the left and right halves of the plane, meaning that whether or not we have a total function $\Phi^{p}: \mathbb{Z}^2 \rightarrow S_e$ (defined below) is dependant on whether this central column is fully tiled. We now show that this is dependent on there being an infinite path through the tree computed by $\varphi_e$.

So show that this is the case, let $T_e$ be the tree computed by $\varphi_e$ - this is guaranteed by lemma \ref{lemma-pproc}. Given $e \in ILL$ it follows that there is an infinite $p \in [T_e]$. Thus, for all $n \in \omega$ there is some string $\sigma_n = p \upharpoonright n$. Given we added all of these $\sigma_n$ strings into $S_e$ as tiles that cover both the up and down directions from the root tile, $\varphi_{h(e)}$ will have contained all of the tiles that represent $\sigma_0 \prec \sigma_1 \prec \sigma_2 \prec \ldots p$ - in fact, there will be precisely two copies. Given $p$ is infinite, these column tiles will thus complete our tiling, making our $S_e$-tiling total in the plane.

Indeed, taking such a $p \in [T_e]$ as our oracle, for all $x,y \in \mathbb{Z}$, and given the output of $\varphi_{h(e)}$ from above as $S_e$, we define $\Phi^p$ as a fully as a total function $$\Phi^p : \mathbb{Z}^2 \rightarrow S_e$$ which can be fully defined algorithmically as follows:

\begin{itemize}
\item For $\Phi^p(0,0)$ will return the root tile, $\langle M^L_0, \lambda^U, M^R_0, \lambda^D \rangle$
\item For $\Phi^p(x,y)$, where $x,y \neq 0$, we will return the relevant quadrant tile.
\item For $\Phi^p(x,0)$ we will return the correct middle-row tile of the form:
\begin{itemize}
\item if $x$ is positive: $\langle M_{x-1},c^2_x,M_x,c^3_x \rangle$
\item if $x$ is negative: $\langle M_{x-1},c^1_x,M_x,c^4_x \rangle$
\end{itemize}
\item For $\Phi^p(0,y)$ we will use that $\sigma = p \upharpoonright y$, and then return the correct column tile of the form:
\begin{itemize}
\item if $y$ is positive: $\langle c^1_y, \sigma, c^2_y, \sigma \upharpoonright y-1 \rangle$
\item if $y$ is negative: $\langle c^4_y, \sigma \upharpoonright y-1, c^3_y, \sigma \rangle$
\end{itemize}
\end{itemize}

To show that $h(e) \in TILE \Rightarrow e \in ILL$ we first note that if $\Phi^p$ is total, then $\varphi_e$ must also be total - as such, if there are no gaps in our $S_e$-tiling following our construction of $S_e$, then it suffices to show that we can computably recover an infinite $p$ from an $S_e$-tiling for which we can assume that $e \in ILL$.

Let $\mathcal{I}$ be the class of all $S_e$-tilings of the plane. We take one total tiling $I \in \mathcal{I}$ - clearly existing by our assumption that $h(e) \in TILE$ - and try to recover an infinite path $p \in [T_e]$, where $T_e$ is again the tree computed by $\varphi_e$. Our goal is to use the tiling to show whether or not $e \in ILL$.

The following computable method will be our attempt to extract the path $p$ from our $S_e$-tiling:
\begin{enumerate}
\item If we choose the root tile, read upwards along the column of tiles, from which we can recover a path $p$. 
\item If we choose a mid-row tile, then we follow the descending chain of $M_i$ colours to the root tile, and then go to 1.
\item If we choose a quadrant tile, then for our given $i \in \omega$ from our chosen tile:
\begin{itemize}
\item If $c^1_i$ or $c^2_i$ then follow all the tiles down to the mid-row tiles, and go to 2.
\item If $c^3_i$ or $c^4_i$ then follow all the tiles up to the mid-row tiles, and go to 2.
\end{itemize}
\end{enumerate}

If our $S_e$-tiling $I$ is total, then the resulting $\tau$ from this process is infinite and corresponds to some $p \in [T_e]$. Thus, we have shown that for $h(e) \in TILE$ we can take any $S_e$-tiling and computably recover a path demonstrating that $e \in ILL$.
\end{proof}

% Figure environment removed

We show in figure \ref{fig:ShapeTilingTILE-ILL} the overall shape of our tiling proposed in the proof of theorem \ref{thm:TILE-ILL}. The $c^j_i$'s occupy the upper left/right and lower left/right quarter planes of $\mathbb{Z}^2$, with the middle rows joining the upper/lower left quarter planes and upper/lower right quarter planes. Thus, our root tile connects the two planes with the paths from a tree coded in the upper and lower columns.

\begin{corollary}[C. 2019]
$TILE$ is $\Sigma^1_1$-Complete.
\end{corollary}

\begin{proof}
This follows immediately from the combination of facts that $TILE$ is $m$-equivalent to a $\Sigma^1_1$-complete set, namely $ILL$, which we obtain by the opposite argument shown in corollary \ref{cor:CompTreesPi11}. As such, everything expressible in $ILL$ is also expressible in $TILE$, so every $a \in \Sigma^1_1$ has some representation in $TILE$.
\end{proof}

We should point out that a key part of this proof is that we have not restricted ourselves to finite sets of prototiles, which we know from theorem \ref{thm:TMTilings} is $\Sigma^0_1$ complete. By allowing ourselves to consider infinite sets of prototiles, we have found a way to get $\Sigma^1_1$ completeness by a proof that gives an equivalence between familiar objects, namely the ill-foundedness of trees. In a sense, this result could be entirely expected.

% Figure environment removed

Figure \ref{fig:TT1} shows an example of a patch around the root tile for some $S_e$-tiling generated by the above algorithm. The first two bits of a path $\sigma$, with $\sigma \upharpoonright 2 = $`01'. Note that we can see in this diagram that if $| \sigma | < \omega$ then there will be gaps at some point going up/down from the root tile, there by such an $e$ will not be total, and so $e \notin TILE$.

\begin{definition}\label{def:WELL}
We define the set of well-founded computable trees: $$WELL = \{ e : \varphi_e \text{ is the characteristic function of a well-founded tree } T \subseteq \omega^{< \omega} \}$$
\end{definition}
Recall that by proposition \ref{prop:wellfddR} it follows that $WELL$ is $\Pi^1_1$-complete, which is an important fact we will use.

We let $\neg TILE$ be the set of computable characteristic functions of infinite sets of prototiles that do not have total tilings of plane. It is interesting that, by the same construction above, we can get that $WELL \equiv_m \neg TILE$, despite unequal complements and totality issues.

\begin{theorem}[C. 2019]\label{thm:nTILE-WELL}\index{$\neg TILE$ equivalence to $WELL$}
\[ (\neg TILE) \equiv_m WELL \]
\end{theorem}

\begin{proof}
We proceed as for the proof of \ref{thm:TILE-ILL} - it suffices to show $WELL \leq_m \neg TILE$ as $(\neg TILE) \leq_m WELL$ will follow then by $\Pi^1_1$-completeness of $WELL$ and lemma \ref{lemma:wellfddTreesPi11}. Given this, we want computable $h$ such that $$ e \in WELL \iff h(e) \in \neg TILE $$

We derive the same $S_e \subset \mathcal{S}$ as we derive in the previous proof. Thus we have an $h$ such that $\Phi^p : \mathbb{Z}^2 \rightarrow S_e$ is given for any path $p \in [T_e]$.

If we have some $e \in WELL$, then by our construction, it must be the case that $\varphi_{h(e)}$ would not give a total tiling of the plane as the well-foundedness of $T_e$ would give that there is no infinite $p \in [T_e]$. Thus, there is no set of column tiles in $S_e$ that will tile the central column of our tilings. Thus it follows that $h(e) \in \neg TILE$.

Now suppose that we have some $h(e) \in \neg TILE$, and let $\mathcal{I}$ be the class of all $S_e$-tilings of the plane. For any given $I \in \mathcal{I}$ we know that $I$ is not a total tiling of the plane, but we know that by our construction both halves of the plane about the central column will be computably tiled. Thus, the gaps in our tiling that make it non-total must be along this central column for all $I \in \mathcal{I}$.

Given this central column is composed of tiles that code paths in $[T_e]$, it must be the case that there is no output of $\varphi_e$ that is an infinite path $p \in [T_e]$. Thus it follows that if $h(e) \in \neg TILE$ then $e \in WELL$.
\end{proof}

As we shall see in the next section, this construction gives rise to some interesting implications when it comes to equivalences of free Domino Problems and infinite sets of prototiles. 

\subsection{Further Equivalences for $WELL$ and $ILL$}

It was found that the equivalences in the previous section were not the only ones  we could construct when we consider infinite sets of prototiles. Indeed, when we consider other free Domino Problems, we can prove further equivalences using a similar framework. In order to do this analysis, we need the following definitions. 

\begin{definition}\label{def:WTILE}
\begin{align*}
WTILE = \{ e : \, & \varphi_e \text{ is the char. func. of a Wang prototile set that has tilings} \\
	       & \text{that are infinite, connected, but not necessarily total} \}
\end{align*}
\end{definition}

$WTILE$ is short for \emph{`weakly-tile'}, and intuitively stands for infinitely connected, but not total tilings. This notion of \emi{weakly tiling} the plane gives us a natural notion of \emi{strongly not tiling} the plane, which we define as follows:

\begin{definition}\label{def:SNT}
\begin{align*}
SNT = \{ e : \, & \varphi_e \text{ is the char. func. of a Wang prototile set whose} \\
		& \text{connected tilings are finite} \}
\end{align*}
\end{definition}

Intuitively we can think of $WTILE$ tilings as being everything in $TILE$ but plus other tilings up to infinite connected `snakes' of tiles that are connected. Though we are now considering tilings that are no longer necessarily total, the fact that they are infinite and connected is the key property we wish to analyse.

On the other hand, $SNT$ denotes tilings that form (potentially infinitely many) disconnected patches of tiles. We can picture disconnected colonies of mould, for example, as an intuition for what these tilings can look like.

As such, prototile sets that are in $SNT$ are necessarily disconnected, whereas tilings in $WTILE$ are necessarily connected, in a graph theoretic sense. We can use the following construction to analyse tilings of infinite sets of prototiles for these properties. Again, we will use $WELL$ and $ILL$ from previous proofs as fundamental tools.

\begin{theorem}[C. 2019]\label{thm:SNT-WELL}\index{$SNT$ equivalence to $WELL$}
\[ SNT \equiv_m WELL \]
\end{theorem}

\begin{proof}
As before, we denote Wang prototiles through the 4-tuple $\langle l,u,r,b \rangle$, and for $\sigma \in \omega^{\omega}$, let $\sigma(n)$ denote the $n^{th}$ symbol of $\sigma$.

We prove these equivalences sequentially. Similarly to the previous proof, it follows from the $\Pi^1_1$-completeness of $WELL$ that for a $\Pi^1_1$ set $A$, $$(WELL \leq_m A) \rightarrow (A \equiv_m WELL)$$ As such, it suffices to show that $WELL \leq_m SNT$, as $SNT \leq_m WELL$ will follow from this, giving our $m$-equivalence.

We want some computable $g$ such that $$ e \in WELL \iff g(e) \in SNT $$ which will give us our $m$-reduction.

In order to carry out this proof, we will need to fix the following colours/symbols:
\begin{itemize}
\item Let $\lambda$ denote the empty string as before, and fix unique colours $\lambda^U, \lambda^D, \lambda^L$, and $\lambda^R$.
\item For $\sigma \in \omega^{< \omega}$ let $| \sigma |$ denote the length of $\sigma$, 
\item Let $\sigma^\frown n$ denote the concatenation of $\sigma$ with some $n \in \omega$.
\item For $j \in \{ 1,2,3,4 \}$ and $n \in \omega$ fix colours $\sigma^j_n$ for every $\sigma$.
\item Let $\sigma \in \omega^\omega$, and for all $i \in \omega$ let $\sigma_i \in \omega^{< \omega}$ denote successive initial segments of $\sigma$ of length $i$ such that $\sigma_0 \prec \sigma_1 \prec \ldots \prec \sigma$.
\item Let $\sigma_0 = \lambda$ by our notation above. 
\end{itemize}

With these colours and symbols fixed, let $e \in WELL$ be given. We construct the following schema tiles:

We start with the \textbf{root tile}:

\begin{center}
\sampletile{$\lambda^L$}{$\lambda^U$}{$\lambda^R$}{$\lambda^D$}
\end{center}

We will also need \textbf{middle column and row tiles}:

\begin{center}
\sampletile{$s^1$}{$s$}{$s^2$}{$\sigma$}
\sampletile{$\sigma$}{$s^2$}{$s$}{$s^3$}
\sampletile{$s^4$}{$\sigma$}{$s^3$}{$s$}
\sampletile{$s$}{$s^1$}{$\sigma$}{$s^4$}
\end{center}
Where $s = \sigma^\frown n$ for $\sigma \in \omega^{< \omega}$ and $n \in \omega$.

Lastly, we will also require \textbf{quadrant filling tiles}:

\begin{center}
\sampletile{$s^1_{i+1}$}{$s^1_{i+1}$}{$s^1_{i}$}{$s^1_{i}$}
\sampletile{$s^2_{i}$}{$s^2_{i+1}$}{$s^2_{i+1}$}{$s^2_{i}$}
\sampletile{$s^3_{i}$}{$s^3_{i}$}{$s^3_{i+1}$}{$s^3_{i+1}$}
\sampletile{$s^4_{i+1}$}{$s^4_{i}$}{$s^4_{i}$}{$s^4_{i+1}$}
\end{center}
Where for $j \in \{ 1,2,3,4 \}$ we have that $s^j_i = \sigma \in \omega^{< \omega}$ of length $i$, and $s^j_{i+1} = \sigma^\frown n$ for some $n \in \omega$ as before. Each colour $s^j_i$ thereby encodes some string in $\omega^{< \omega}$, and $s^j_{i+1}$ is the extension of this by 1 character, and both are initial segments of some infinite path.

We can now construct a library $\mathcal{U}$ of tiles from which we will select the relevant ones we need. To generate $\mathcal{U}$ we will take all of the colours we fixed earlier and apply them to the prototile schema above as follows:
\begin{itemize}
\item We colour the root tile with the colours we fixed to get the prototile $\langle \lambda^L, \lambda^U, \lambda^R, \lambda^D \rangle$ and put this tile into $\mathcal{U}$.
\begin{itemize}
\item As before, our root tile has unique colours for each direction to prevent trivial tilings of the plane from the root tile alone. 
\end{itemize}
\item With $j \in \{ 1,2,3,4 \}$, for each initial segment colour $s^j_n$ we fixed earlier, colour all of the possible quadrant tiles and put these into $\mathcal{U}$.
\begin{itemize}
\item For each $\sigma,\tau \in \omega^{< \omega}$, where $\tau = \sigma^\frown i$ is an ancestor for some $\sigma$ with $i \in \omega$, we fix 8 colours:
\begin{itemize}
\item[--] $\sigma^1, \sigma^2, \sigma^3, \sigma^4$
\item[--] $\tau^1, \tau^2, \tau^3, \tau^4$
\end{itemize}
\item We then proceed to create 4 prototiles:
\begin{enumerate}
\item $\langle \tau^1, \tau^1, \sigma^1, \sigma^1 \rangle$
\item $\langle \sigma^2, \tau^2, \tau^2, \sigma^2 \rangle$
\item $\langle \sigma^3, \sigma^3, \tau^3, \tau^3 \rangle$
\item $\langle \tau^4, \sigma^4, \sigma^4, \tau^4 \rangle$
\end{enumerate}
\end{itemize}
\item We also colour for every $\sigma_n \in \omega^{< \omega}$ of length $n$, and every $i \in \omega$ the following column tiles:
\begin{enumerate}
\item $\langle (\sigma_n^\frown i)^1, \sigma_n^\frown i, (\sigma_n^\frown i)^2, \sigma_n \rangle$
\item $\langle \sigma_n, (\sigma_n^\frown i)^2, \sigma_n^\frown i, (\sigma_n^\frown i)^3 \rangle$
\item $\langle (\sigma_n^\frown i)^4, \sigma_n, (\sigma_n^\frown i)^3, \sigma_n^\frown i \rangle$
\item $\langle \sigma_n^\frown i, (\sigma_n^\frown i)^1, \sigma_n, (\sigma_n^\frown i)^4 \rangle$
\end{enumerate}
\end{itemize}

We are now left with a requirement to colour the middle-row and middle-column prototiles. We again ensure that our $e \in WELL$ has been through the pre-processing lemma \ref{lemma-pproc}, which ensures there is a tree $T_e$ computed by $\varphi_e$.

Our $g$ will then construct $U_e \subset \mathcal{U}$ as follows:
\begin{enumerate}
\item Select the root tile, and add this into $U_e$.
\item Select all of the middle column and middle row tiles that correspond to each path $p \in [T_e]$ and add these also into $U_e$.
\item Select from the quadrant filling tiles with the relevant $\tau$ such that for any $\sigma \prec p \in [T_e]$, $\tau$ is the immediate ancestor $\sigma^\frown i$ for $i \in \omega$ such that $\sigma \prec \tau \prec \ldots \prec p$. We then add to this the quadrant tiles we need into $U_e$.
\end{enumerate}

The construction of the prototile set $U_e$ embeds some path $\sigma_n$, where $\varphi_e(\sigma_n) = 1$, 4 times from the root tile - each copy going one of the 4 directions up, down, left, or right, forming `spokes' from the root tile that represent a path through $T_e$. The quadrant tiles are then used to fill in the gaps between these spokes with the intention that we \emph{could} get a total $U_e$-tiling of the plane if $e \notin WELL$. As before, the root tile's empty strings are equivalent to $\sigma_0 = \lambda^L = \lambda^U = \lambda^R = \lambda^D$.

We first want to verify that $$e \in WILL \rightarrow g(e) \in SNT$$ This can be done by analysing the behaviour of the tiling function $\Psi^p : \mathbb{Z}^2 \rightarrow U_e$.

To do this, let $e \in WELL$ be given. Then we can construct $U_e$ as above, and then observe what will happen in a $U_e$-tiling of the plane. Given the well-foundedness of $\varphi_e$ means that there is no infinite $p \in [T_e]$ such that our $U_e$-tilings would have infinitely long spokes. Given this, each $U_e$-tiling will have a bound on the width and height of the tiling, and as such our tiling function $\Psi^p$ will not be total over $\mathbb{Z}^2$. 

Given this fact, $\varphi_{g(e)}$ will only generate a finite patch tiling that is connected. Thus we can say that $g(e)$ must only have connected tilings that are patches, and so we get $g(e) \in SNT$.

For the converse direction it suffices to show that $$ e \notin WELL \rightarrow g(e) \notin SNT$$ Given $e \notin WELL$ there exists an infinite $p \in [T_e]$ which we will use as our oracle. This follows by construction of $\Psi^p : \mathbb{Z}^2 \rightarrow U_e$ as a total TM as follows - let $\sigma_n = p \upharpoonright n$:
\begin{itemize}
\item For $\Psi^p(0,0)$ we return the root tile $\langle \lambda^L, \lambda^U, \lambda^R, \lambda^D \rangle$
\item For $\Psi^p(x,0)$ we return one of two tiles:
\begin{itemize}
\item If $x$ is negative: $\langle \sigma_n, \sigma^1_n, \sigma_{n-1}, \sigma^4_n \rangle$
\item If $x$ is positive: $\langle \sigma_{n-1}, \sigma^2_n, \sigma_n, \sigma^3_n \rangle$
\end{itemize}
\item For $\Psi^p(0,y)$ we return one of two tiles:
\begin{itemize}
\item If $y$ is negative: $\langle \sigma_n^4, \sigma_{n-1}, \sigma^3_n, \sigma_n \rangle$
\item If $y$ is positive: $\langle \sigma_n^1, \sigma_n, \sigma^2_n, \sigma_{n-1} \rangle$
\end{itemize}
\item For $\Psi^p(x,y)$ such that $x,y \neq 0$, we return tile for the correct quadrant such that $\sigma_{n-1}$ and $\sigma_n$ are present for $n = |x| + |y|$.
\end{itemize}

\textbf{NB} - we substitute $\lambda^L, \lambda^U, \lambda^R, \text{ and } \lambda^D$ as required for $\sigma_0$ to ensure that all of our tiles align in the plane.

With $p \in [T_e]$ infinite, given $e \notin WELL$, then $\Psi^p$ is total, which gives us immediately that there are total planar $U_e$-tilings. Thus, our connected tilings for $U_e$ are not patches, and so $g(e) \notin SNT$. 
\end{proof}

% Figure environment removed

In Figure \ref{fig:WT1} we find the proposed construction, showing the four copies of some path $\sigma$ emanating from the central root tile. The absence of any tiles to complete the edges means that these can never join together to form a complete tiling of the plane - the only way for there to be a total tiling of the plane is for $e \notin WELL$, from which our result follows.


\begin{corollary}[C. 2019]
$SNT \equiv_m WELL$ implies that $SNT$ is $\Pi^1_1$-Complete.
\end{corollary}

\begin{proof}
This follows as a consequence that $SNT$ is equivalent to a $\Pi^1_1$-complete set, $WELL$. As such, every $b \in \Pi^1_1$ will have some representation in $SNT$, and as such, $SNT$ is also $\Pi^1_1$-complete.
\end{proof}

It should be noted that the proofs of theorem \ref{thm:SNT-WELL} could have been shortened to just the tiles that enumerate the paths that we are interested in - however, we will use the construction with filler tiles later in this thesis. We shall also utilise the quadrant filling tiles in this construction in the next theorem.

Following on from theorem \ref{thm:SNT-WELL}, we asked what the relationship to $WTILE$ was, and found that we can state the following theorem:

\begin{theorem}[C. 2019]\label{thm:WTILE-ILL}\index{$WTILE$ equivalence to $ILL$}
\[ WTILE \equiv_m ILL \]
\end{theorem}

\begin{proof}
We get $WTILE \leq_m ILL$ by the $\Sigma^1_1$-completeness of $ILL$. It is sufficient to then show that $ILL \leq_m WTILE$ by our construction for the proof of \ref{thm:SNT-WELL}. As such, we want computable $g$ such that $$ e \in ILL \iff g(e) \in WTILE $$

We derive the same $U_e \subset \mathcal{U}$ as given in the proof of theorem \ref{thm:SNT-WELL}, so we have a $g$ such that $\varphi_{g(e)} : \mathbb{Z}^2 \rightarrow U_e$ computable.

By our construction, we have that if $e \in ILL$ then $\varphi_{g(e)}$ will be a total function from the $\mathbb{Z}^2$ lattice into $U_e$. The resulting tiling will have 4 infinite spokes coming from the root tile, and these are infinite connected tilings that satisfy $g(e) \in WTILE$, even without knowing that $\varphi_{g(e)}$ is total.

For the converse direction it would suffice to show that $$e \notin ILL \rightarrow g(e) \notin WTILE$$ which can be seen through the following argument. Given $e \notin ILL$ then there exists no path $p \in [T_e]$ that is infinite. Thus, when we create $U_e$ by means of $g(e)$, we must create a tile set that has only connected patch of the plane, violating the requirements for $WTILE$, thus $g(e) \notin WTILE$.
\end{proof}

\subsection{Discussion of these Results}

These results differ from previous work by \cite{Harel1986} insofar as they do not rely on any knowledge of the properties of recurrent patterns within a tiling, but rather manage to specifically equate several forms of Domino Problems on infinite sets of prototiles.

It is worth noting some of the following facts about these results:
\begin{enumerate}
\item At no point to we restrict ourselves to requiring to use every tile in a generated prototile set.
\item We do not require any special conditions on how/where our tilings start. 
\end{enumerate}

For 1, it is of interest that we do not require every tile $t \in S_e$ or $t \in U_e$ to be used at all. To this end, we have specifically added extra colours the make specific alignments and prevent trivial planar tilings - specifically from the root tiles we defined. 

For 2, these tilings can be essentially tiled without stating specific initial criteria as we have been very careful to include design elements that essentially force the hand of the tiling function into only admitting certain tilings that code the precise behaviour we want.

Note also that the classes of tilings from either of these prototile sets effectively encode the paths down the trees computed by some $e$, once $e$ has been passed through our tree-filtering lemma \ref{lemma-pproc}.

It is also worth noting that our $m$-equivalences are such that they work despite the mismatch in complements for the sets we concern ourselves with - \emph{e.g.} the complements of $TILE$ and $WTILE$ are quite different, and yet they are both $m$-equivalent to $ILL$. This shows us that infinite computable sets of Wang prototiles are not rich enough to discern the differences that we are mathematically aware of.


