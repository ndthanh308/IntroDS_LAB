\chapter*{Glossary of Sets and Constructions}
\label{chap:Glossary}
% next resets the equation numbers to start at 1 at the start of the chapter
\setcounter{equation}{0}
\renewcommand{\theequation}{\thechapter.\arabic{equation}}

We give a table that details all of the major sets and operators that are used in this thesis, for convenience and for reference.

%\begin{table}[h]
%\begin{tabular}{p{2.5cm}|p{8cm}|p{2cm}}
\begin{longtable}{p{2.5cm}|p{8.25cm}|p{1.75cm}}
\textbf{Name} & \textbf{Description} & \textbf{Thesis Ref.} \\
 \midrule
$m$-reducibility & Given two sets $A$ and $B$, $A$ is $m$-reducible to $B$, written $A \leq_m B$, if there exists some computable function $f: \omega \rightarrow \omega$ such that for all $x \in \omega$, $x \in A \iff f(x) \in B$ & \ref{def:mred}\\
\hline
Weihrauch Reducibility & Given two operators $f$ and $g$ on represented spaces, we say $f \leq_W g$, if there exist computable $H,K :\subseteq \omega^\omega \rightarrow \omega^\omega$ such that for any realizer $G \vdash g$, $F = K\langle id_{\omega^\omega}, GH \rangle$ is a realizer for $f$. & \ref{def:Weihrauch} \\
\hline
$WELL$ & The set of all indices $e$ such that $\varphi_e$ is the characteristic function of a well-founded tree $T \subseteq \omega^{<\omega}$. & \ref{def:WELL}\\
\hline
$ILL$ & The set of all indices $e$ such that $\varphi_e$ is the characteristic function of an ill-founded tree $T \subseteq \omega^{<\omega}$. & \ref{def:ILL}\\
\hline
$TILE$ & The set of all indices $e$ such that $\varphi_e$ is the characteristic function of an infinite Wang prototile set whose tilings are total in the plane. & \ref{def:TILE} \\
\hline
$WTILE$ & The set of all indices $e$ such that $\varphi_e$ is the characteristic function of an infinite Wang prototile set whose tilings are infinite, connected, but not necessarily total in the plane. & \ref{def:WTILE} \\
\hline
$SNT$ & The set of all indices $e$ such that $\varphi_e$ is the characteristic function of an infinite Wang prototile set whose connected tilings are all finite. & \ref{def:SNT}\\
\hline
$ATile$ & Set of all $e$ such that $\varphi_e$ is the characteristic function for a set of prototiles who planar tilings are all total and aperiodic. & \ref{def:ATile} \\
\hline
$PTile$ & Set of all $e$ such that $\varphi_e$ is the characteristic function for a set of prototiles who planar tilings are all total and periodic. & \ref{def:PTile} \\
\hline
$ATile_{FIN}$ & Set of all $e$ such that $\varphi_e$ is the characteristic function for a \emph{finite} set of prototiles who planar tilings are all total and aperiodic. & \ref{def:ATileFIN}\\
\hline
$PTile_{FIN}$ & Set of all $e$ such that $\varphi_e$ is the characteristic function for a \emph{finite} set of prototiles who planar tilings are all total and periodic. & \ref{def:PTileFIN}\\
\hline
AIT & The construction found in the proof of theorem \ref{thm:TILE-ILL} that creates an aperiodic prototile set given an ill-founded tree. & \ref{def:AITPIT}\\
\hline
PIT & The construction found in the proof of theorem \ref{thm:ILL-PTile} that creates an aperiodic prototile set given an ill-founded tree. & \ref{def:AITPIT}\\
\hline
$CT$ & The operator that takes some set of Wang prototiles as input and returns a total tiling of the plane. & \ref{def:ChooseTiling} \\
\hline
$CWPT$ & An operator that takes a set of Wang prototiles and returns a connected planar, but not necessarily total tiling. & \ref{ref:CWPT} \\
\hline
$CIPT$ & An operator that takes a prototile set $S$ that has total planar tilings, and returns an infinite `slice' of this tiling as a tiling of an infintie region of $\mathbb{Z}^2$. & \ref{def:CIPT} \\
\hline
$WIPT$ & An operator that takes a set of prototiles and return a tiling that has an infinite patch tiled within it, but we do not know where it is. & \ref{def:WIPT} \\
\hline
$DPW$ & The $DPW$ operator takes some set of prototiles and return a tiling that has an infinite connected patch within it. & \ref{def:DPW} \\
\hline
$C_{\omega^\omega}$ & Closed choice on Baire space - equivalent to finding a path through an ill-founded Baire space tree. & \ref{def:ClosedChoice} \\
\hline
$C_{2^\omega}$ & Closed choice on Cantor space - equivalent to Weak K\"onig's Lemma. & Sec \ref{sec:FurtherWeakTilingProblems} \\
\hline
$C_{\omega}$ & closed choice on the natural numbers - this takes a function $f: \omega \rightarrow \omega$ such that $range(f) \neq \omega$, and returns some point $n \notin range(f)$. & Sec. \ref{sec:FurtherWeakTilingProblems} \\
\end{longtable}
%\end{tabular}
%\end{table}
