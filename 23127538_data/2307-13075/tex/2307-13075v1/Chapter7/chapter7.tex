\chapter{Small ECA Tilings}
\label{chap6}
% next resets the equation numbers to start at 1 at the start of the chapter
\setcounter{equation}{0}
\renewcommand{\theequation}{\thechapter.\arabic{equation}}

%------------------------------------------------------------------------------

\epigraph{In mathematics you don’t understand things. You just get used to them.}{\textit{John von Neumann (attrib.)}}

This chapter presents a small tiling that encodes any Elementary Cellular Automaton in 15 prototiles. We also present some results about this class of automata that show that these prototile sets have interesting properties, specifically that they can be chaotic or Turing complete.

\section{Elementary Cellular Automata}

In this section, we will give formal definitions for Elementary Cellular Automata (ECAs) in preparation for coding them into small tiling sets. Our motivation for this originally was work that was ultimately carried out to its full completion in \cite{Rao2015} - aiming to find small, aperiodic tiling sets by means of coding small chaotic Elementary Cellular Automata into prototile sets. 

However, as we shall show in theorem \ref{thm:ECA-15-Hex}, we found a different way of encoding 3-ary functions as dynamical systems into prototile sets that represent their behaviour in the plane. We maintain the usual structure from previous work on coding Turing Machines into the plane - the 1-dimensional state is given left to right, with subsequent iterations going vertically.

We first give the background theory on ECAs, as well as a basic primer on the relevant pieces of chaos theory, and then proceed to detail results from Cook and Cattaneo \etal about Turing completeness and chaos in ECAs, respectively. Finally we give our representations of any ECA in prototile sets of only 15 tiles using our new construction, replete with diagrams and relevant corollaries. 

\subsection{Elementary Cellular Automata}

We will define a cellular automaton, and elementary cellular automaton (ECA) as per \cite{ECAMathworld}. They have appeared in a considerable amount of research, in areas as varied as computer science, symbolic dynamics, and as we shall see, chaos theory.

\begin{definition}
A \emi{cellular automaton} is pair $( X, R )$ where $X$ is a grid of some specific boundary topology\footnote{These grids can have joined boundaries, fixed boundaries, be bi-infinite, \etc \etc.} and $R$ is the `rule' that is applied successively to the grid. Each row is coloured based on the state of the colours on the previous row.
\end{definition}

We will specifically look at the subclass of cellular automata known as Elementary Cellular Automata, or ECAs. These were first introduced and studied by Wolfram in \cite{NKS}. 

\begin{definition}
\sloppy An \emi{elementary cellular automaton}, or ECA, is a cellular automaton $( X, R_n )$ where the rules in $R$ are derived from the binary representation of $n$. An ECA's rules for a cell at position $i$ on row $j$, written $c_{i,j}$, is determined by the triple $(c_{i-1,j-1} , c_{i,j-1} , c_{i+1,j-1})$. Thus, our rule set $R_n$ is given by a function $r_n : \{ 0,1 \}^3 \rightarrow \{ 0,1 \}$. 
\end{definition}

To acquire our rules for $R_n$ we first take the binary representation of $n$, and then send each of our 8 possible inputs sequentially to each bit of the binary representation of $n$, starting with the least significant bit.

To illustrate how this works, take $R_{30}$. We start with the 8-bit binary representation of 30, 00011110, and then map the inputs to $r_{30}$ as per table \ref{tbl:rule30}.

\begin{center}
\begin{table}[t]
\begin{tabular}{ c | c | c | c | c | c | c | c }
  111 & 110 & 101 & 100 & 011 & 010 & 001 & 000 \\
\hline
  0 & 0 & 0 & 1 & 1 & 1 & 1 & 0 \\  
\end{tabular}
\caption[Rule 30 Automaton Rules]{Rule 30 Automaton Rules}\label{tbl:rule30}
\end{table}
\end{center}

If we let our grid be the full $\mathbb{Z}^2$ lattice, then for each row $x \in \mathbb{Z}^2$, we can define $R_n : \mathbb{Z}^2 \rightarrow \mathbb{Z}^2$ as the successive application of $r_n$ to every triple $(x(i-1), x(i), x(i+1))$, for each cell $x(i) \in x$. 

\section{Some Results about ECAs}

We will define and discuss some background results for our work on ECAs and tilings. With ECA's already being an interesting and fertile area of study, we will give some background theory to the results, and then demonstrate that these results can also be realized as tiling problems by means of coding ECA's into prototile sets.

\subsection{ECAs and Chaos}

We will view ECAs as Discrete Time Dynamical Systems (DTDS) - that is, an iterated system that has discrete time steps. We write these as above, $(X,F)$, where $X$ is the \emi{phase space}, which is equipped with a distance function $d$, and a \emi{next state map} $F:X \mapsto X$, continuous on $X$ according to the topology on $X$ induced by $d$. We also assume that such a metric space $(X,d)$ is perfect - i.e. has no isolated points.

\begin{definition}[Sensitivity]\label{def:sensitivity}\index{topological sensitivity}
A DTDS $(X,F)$ is \emph{sensitive to initial conditions} if and only if there exists $\delta > 0$ such that $$ (\forall x \in X) \, (\forall \epsilon > 0) \, (\exists y \in X) \, (\exists n \in \mathbb{N}) [d(x,y) < \epsilon \wedge d(F^n(x),F^n(y)) \geq \delta] $$
\end{definition}

More intuitively, this definition states that the iterated map has the property that there exist points arbitrarily close to some point $x \in X$ that eventually separate away from $x$ by at least $\delta$. 

We will need, for our definitions of chaos, definitions of the following terms:

\begin{definition}
A dynamical system $(X,F)$ has a \emi{dense orbit} if and only if $$ (\exists x \in X) \, (\forall y \in X) \, (\forall \epsilon > 0) \, (\exists n \in \mathbb{N}) \, [d(F^n(x),y) < \epsilon] $$
\end{definition}

\begin{definition}
A dynamical system $(X,F)$ is \emi{topologically transitive} if and only if for all non-empty open subsets $U,V$ of $X$, $$ (\exists n \in \mathbb{N}) \, [F^n(U) \cap V \neq \emptyset]$$
\end{definition}

For a perfect DTDS $(X,F)$, the existence of a dense orbit necessarily implies topological transitivity. This is an important result in reference to the 1-dimensional dynamical systems we wish to represent in tilings later on - it shows us that the barrier to achieving chaotic behaviour is reassuringly low, which somewhat naturalises our results.

\begin{definition}
A dynamical system $(X,F)$ has \emi{dense periodic points} if and only if the set of all the periodic points given by $$ Per(F) = \{ x \in X : (\exists k \in \omega ) \, F^k(x) = x \} $$ is a dense subset of $X$. Specifically, $$ (\forall x \in X) \, (\forall \epsilon > 0) \, (\exists p \in Per(F)) \, [ d(x,p) < \epsilon ] $$
\end{definition}

Following on from these definitions, Devaney in \cite{Devaney1989} formulated the most well-known definition of chaos as follows:

\begin{definition}[Devaney Chaos]\index{Devaney chaos}
The dynamical system $(X,F)$ is \emph{chaotic} if
\begin{enumerate}
\item $F$ is topologically transitive,
\item $F$ has dense periodic points,
\item $F$ is sensitive to initial conditions.
\end{enumerate}
\end{definition}

Meanwhile, other formulations of chaos came about - the most notable for this work is due to Knudson \cite{Knudson1994}, which is nonperiodicity-free: 

\begin{definition}[Knudson Chaos]\index{Knudson chaos}
The dynamical system $(X,F)$ is chaotic if
\begin{enumerate}
\item $F$ has a dense orbit,
\item $F$ is sensitive to initial conditions.
\end{enumerate}
\end{definition}

This formulation that came about when Knudson proved there existed a dynamical system which is chaotic according to Devaney's definition, but which the restriction of the set to its periodic points was also Devaney Chaotic.

It will be useful later to consider similar restrictions, such as \cite{VellekoopBerglund} that demonstrates the following proposition:

\begin{proposition}[\protect{\cite[Prop. 1, p.353]{VellekoopBerglund}}]
Let $I$ be a (potentially infinite) interval - a 1-dimensional space - and $F: I \mapsto I$ be a continuous, topologically transitive map. Then
\begin{enumerate}
\item The periodic points of $F$ are dense in $I$,
\item $F$ has sensitivity to initial conditions.
\end{enumerate}
\end{proposition}

Thus, for 1-dimensional systems, topological transitivity is `enough' for a dynamical system to be chaotic. Given our ECAs are being considered as essentially 1-dimensional DTDS it becomes clear that our requirements for such a system to be chaotic are quite surprisingly minimal. 

In order to fully describe this, we need notions of `permutivity' for an ECA, which we get from \cite{CattaneoFM00}:

\begin{definition}[Permutivity]\index{permutivity}
A cellular automaton local rule $f$ is \emi{permutive} in $x_i$, for $-k \leq i \leq k$, if and only if for any given sequence $x_{-k}, \ldots , x_{i-1},x_{i+1},\ldots,x_k \in X$ we have $$ \{ f(x_{-k},\ldots,x_{i-1},x_i,x_{i+1},\ldots,x_k) : x_i \in X \} = X $$
\end{definition}

We can refine this idea to leftmost (rightmost) as follows:

\begin{definition}[Leftmost (Rightmost) Permutive]\index{left/rightmost permutive}
A local CA rule $f$ is said to be \emph{leftmost} (\emph{rightmost}) permutive if and only if there is an integer $i$, $-k \leq i \leq 0$ ($0 \leq i \leq k$) such that:
\begin{enumerate}
\item $i \neq 0$,
\item $f$ is permutive in the $i^{th}$ variable, 
\item $f$ does not depend on $x_j$ for $j < i$ ($j > i$).
\end{enumerate}
\end{definition}

As pointed out in \cite{CattaneoFM00}, for ECAs this means that when an ECA is leftmost-permutive, it follows that $$(\forall x_i, x_{i+1}) \, [f(0,x_i, x_{i+1}) \neq f(1, x_i, x_{i+1})] $$ namely, if two strings differ in the $x_{i-1}^{th}$ position, they differ in the $x_i^{th}$ position under $f$. Likewise, when an ECA is rightmost-permutive, the mirror argument follows, specifically $$(\forall x_{i-1}, x_{i}) \, [f(x_{i-1},x_i, 0) \neq f(x_{i-1}, x_i, 1)] $$

We can now use the following result from Cattaneo \etal (Cor. 3.3 in \cite{CattaneoFM00}):

\begin{corollary}[\protect{\cite[Cor. 3.2]{CattaneoFM00}}]
Let $( \mathbb{Z}^2, R_n ) $ be an ECA based on the local rule $r_n$. Then the following are equivalent:
\begin{enumerate}
\item $r_n$ is leftmost or rightmost permutive, or both.
\item $r_n$ is Devaney Chaotic.
\item $r_n$ is Knudson Chaotic
\item $r_n$ is surjective and non-trivial.
\end{enumerate}
\end{corollary}

By Table 1 and the analysis in Section 3.3 in \cite{CattaneoFM00}, it becomes clear that there exist a set of rules that exhibit chaotic behaviour, the most well known of which is $R_{30}$, having been studied in some depth originally by Wolfram in \cite{NKS}.

\subsection{ECAs and Turing Universality}

We now wish to extend results from earlier in this thesis to very small dynamical systems, for which we will need the following definitions:

\begin{definition}
A \emi{cyclic tag system} is a computational system consisting of the following arrangement:
\begin{itemize}
\item A set $P \subset 2^{< \omega}$ of \emph{productions}.
\item A finite binary string $d = d_0, d_1, \ldots d_j$ called the \emph{data string}.
\item A transformation map $$ (i, d) \rightarrow (i+1 (\mathrm{mod}\ n), (d_1, d_2, \ldots, d_k)^\frown P^{d_0}_i) $$ where $i$ is a counter, $n = |P|$, and for all $i$:
\begin{align*}
P^0_i & = \emptyset \\
P^1_i & = P_i
\end{align*}
\end{itemize}
\end{definition}

Intuitively, a cyclic tag system operates as follows:
\begin{enumerate}
\item If $d_0 = 0$, then we delete $d_0$ and do nothing.
\item If $d_0 = 1$, then we delete $d_0$ and append the $i^{th}$ member of $P$, $P_i$.
\item If $d = \emptyset$ then we halt. 
\end{enumerate}

An example computation is as follows. Let $P = \{101, 110, 10\}$ and $d = 11$, our computation is as given in table 2.
\begin{table}[]\label{tbl:cyclicTagSystem}
\begin{tabular}{ll}
$P_i$  & $d$ \\
101  & 11 \\
110  & 1101 \\
11   & 101110 \\
101  & 0111011 \\
110  & 111011 \\
11   & 11011110 \\
101  & $\ldots$
\end{tabular}
\caption{This table shows the development of a cyclic tag system for initial $d$ of $11$ and $P_i$'s in sequence as given in the text. The development of the contents of $d$ is given at each line.}
\end{table}

It is proved in \cite{CookUniversality} that a cyclic tag system is Turing Universal - this was done by showing a Universal Turing Machine can be coded into a 2-tag system, and 2-tag systems can be coded into Cyclic tag systems. The proof is omitted here, but a clear proof can be found in \cite{Neary2006}. 

In 2004, Cook proved in \cite{CookUniversality} the following theorem:

\begin{theorem}[\protect{\cite[Sec 4]{CookUniversality}}]\label{thm:CookECA}
The ECA $R_{110}$ is Turing Universal.
\end{theorem}

This is done by combination of the following theorem and Lemmas:

\begin{lemma}[\protect{\cite[Sec 3]{CookUniversality}}]
A cyclic tag system is Turing Complete.
\end{lemma}

This is a somewhat surprising result, owing to the very minimal nature of cyclic tag systems, but the proof shows that by careful construction of the production sets $P$ it is possible to emulate the tag systems, due to Post, of a small number of states easily. The proof of this coding is fairly straightforward, but is omitted here owing to length.

\begin{lemma}[\protect{\cite[Sec 4]{CookUniversality}}]
A Cyclic Tag system can be implemented in a glider system.
\end{lemma}

\begin{proof}[Sketch of proof of \ref{thm:CookECA}]
Rule 110 has the ability to carry a state of 1's and 0's left and right depending on careful setup of the strings - such patterns that shift iteratively left and right down our ECA state are called `gliders'. A `glider system' is some arrangement of these gliders such that they then propagate left and right. There are 5 glider types documented in \cite{CookUniversality}, and these are crafted into different arrangements of glider systems in order to achieve the result we are interested - specifically, coding the $P$ and $d$ of any cyclic tag system.

% Figure environment removed

By carefully implementing a glider system in the input row for an ECA, Cook was able to code Turing Machine computations into the dynamics of $R_{110}$, thereby showing this ECA to be Turing Universal. 

An additional aside, which will be useful in our discussion of ECA tilings, is that the halting state of some TM coded into $R_{110}$ is equivalent to whether the dynamics of the system become aperiodic or remain periodic, equivalent to halting or not halting, respectively.
\end{proof}

An overall schematic diagram can be found in Figure \ref{fig:CookSchema} 

However, we note that there are some cases where simply expecting aperiodicity or continued periodicity is not sufficient. Take a TM that calculates some non-repeating sequence, such as the Champernowne's Constants used earlier in this thesis. The output of this computation will necessarily be aperiodic in any given tiling encoding of this computation.

Thus we have to resolve the issue surrounding this - if our tiling is going to be aperiodic whether we have halted or not, then how can we tell if our computation is running or if it has halted?

Firstly, we note that Rule 110 is not left or right permutive, so any tiling will not naturally be aperiodic by the criteria in the previous section. We next need to note that we can stratify these two notions of `aperiodicity' by means of a straightforward argument on the underlying mechanics of our resultant tilings \emph{in vicem} of the Turing Machines and cyclic tag machines we are representing.

We note that any non-repeating computation will actually be quasi-periodic by our definition \ref{def:quasiperiodic} - a fact that follows when we observe that certain strings, namely those representing states in our Turing Machine via the set of productions $P$ in our cyclic tag system being recurrent in the tiling.

Thus, any aperiodic behaviour will be apparent from the fact that there will be no sign of our Turing computational artefacts in the ECA following entering the halt state. As such, it will either become periodic or aperiodic, but our test for the occurrence of particular words that code these will fail.

The same carries forwards into our tiling by means of looking for particular sequences of tiles - represented as finite tuples - in any resultant tiling. Given this, we can safely work with ECA Rule 110 and not worry about `losing track' of the status of our computation.

\section{Elementary Cellular Automata and Tilings}

In this section we build on work from the author's MSc thesis, \cite{MCarneyMSc}, where we proved the following theorem:

\begin{theorem}[\protect{\cite[Chap. 3]{MCarneyMSc}}]\label{thm:WangECA}
There exists a universal prototile schema consisting of 18 Wang tiles that tiles the plane according to the rules of any given ECA.
\end{theorem}

\begin{proof}

We note that we need to satisfy the following requirements:
\begin{enumerate}
\item Encode each cell in a time-space diagram for a given ECA.
\item Encode the relationships between each cell given by $R_n$.
\item Show how bits can be copied across each other in the tiling in order to emulate the action of $R_n$.
\end{enumerate}

We first construct the prototile scheme that will code the action of our ECA function given by $f_n : \{0,1\}^3 \rightarrow \{0,1\}$, given by our rule $R_n$. This scheme is as follows:

\begin{center}
\sampletilefarlabels{$a$}{$b$}{$c$}{$\substack{}{f_n(a,b,c)}$}
\end{center}

Thus, for each rule we get the following 8 prototiles, where we fill in the specific outputs for each $f_n$ to get our \emph{Rule prototiles}:

\begin{center}
\sampletilefarlabels{$0$}{$0$}{$0$}{$\substack{}{f_n(0,0,0)}$}
\sampletilefarlabels{$0$}{$0$}{$1$}{$\substack{}{f_n(0,0,1)}$}
\sampletilefarlabels{$0$}{$1$}{$0$}{$\substack{}{f_n(0,1,0)}$}
\sampletilefarlabels{$0$}{$1$}{$1$}{$\substack{}{f_n(0,1,1)}$}

\vspace{2.5mm}

\sampletilefarlabels{$1$}{$0$}{$0$}{$\substack{}{f_n(1,0,0)}$}
\sampletilefarlabels{$1$}{$0$}{$1$}{$\substack{}{f_n(1,0,1)}$}
\sampletilefarlabels{$1$}{$1$}{$0$}{$\substack{}{f_n(1,1,0)}$}
\sampletilefarlabels{$1$}{$1$}{$1$}{$\substack{}{f_n(1,1,1)}$}

\end{center}

We add to these \emph{state swapping} tiles that will take an output of $f_n$ and `swap' this bit with the cell's neighbours. We first fix the colour $B$ that will act as `blank', allowing us to line up the tiles above and below each crossover of bits from the distributor tiles (see below):

\begin{center}
\sampletile{$0$}{$B$}{$0$}{$(0,0)$} \sampletile{$0$}{$(0,0)$}{$0$}{$B$}

\vspace{2.5mm}

\sampletile{$0$}{$B$}{$1$}{$(0,1)$} \sampletile{$1$}{$(0,1)$}{$0$}{$B$}

\vspace{2.5mm}

\sampletile{$1$}{$B$}{$0$}{$(1,0)$} \sampletile{$0$}{$(1,0)$}{$1$}{$B$}

\vspace{2.5mm}

\sampletile{$1$}{$B$}{$1$}{$(1,1)$} \sampletile{$1$}{$(1,1)$}{$1$}{$B$}
\end{center}

We now need some \emph{distributor tiles} that will take an output state and distribute this information left, right, and downwards:

\begin{center}
\sampletile{$1$}{$1^{f_n}$}{$1$}{$1$}
\sampletile{$0$}{$0^{f_n}$}{$0$}{$0$}
\end{center}

Note that these tiles differentiate the upper quadrant as being specifically from the output of $f_n$ so as to prevent trivial tilings of the plane using just distributor prototiles. These tiles code exactly the cells from the original time-space diagram.

We then note that each part of the action of some ECA rule $R_n$ is now coded into our tiling:
\begin{itemize}
\item Each cell is represented in any planar tiling due to the above prototile constructions.
\item Each relationship coded by $f_n$ is represented as state swapping tiles creating a space for some rule tile, which then has the output of $f_n$ distributed for this process to repeat.
\item We do not code the upper half-plane owing to our not-knowing the previous rows of computation that took place before our input row.
\end{itemize}

Thus, we have fully represented in 18 prototiles, given by our 8 rule tiles, 8 state swapping, and 2 distributor prototiles.

The tiling process is as follows:
\begin{enumerate}
\item Code the input into a series of distributor tiles.
	\begin{itemize}
	\item We pad the input with infinitely many `0's left and right to achieve a full half-planar tiling.
	\end{itemize}
\item Place the relevant state swapping tiles between each of these.
\item Tile each successive row using the correct tilings, in order to get successive states of the ECA.
\end{enumerate}

\end{proof}

% Figure environment removed

Figure \ref{fig:ECA-Wang} shows the tiling in action, coding the first few rows of ECA rule 30, with $R_{30}$ clearly coded with the connecting tiles showing how the outputs interact with each other.

\section{A 15 Prototile ECA Tiling}

We present a tiling that codes any ECA in only 15 tiles, using an adapted hexagon-based tiling. This particular tiling lends itself to our computable trinary functions that form our $f_n$ ECA functions, and have not yet been found in the literature.

\begin{theorem}[C. 2019]\label{thm:ECA-15-Hex}
For any ECA of Rule $n$ there exists a prototile set $S_n$ of size 15 such that any tiling of the plane $T$ by $S_n$ codes each iteration of the ECA starting from the string coded by the first row. 
\end{theorem}

\begin{proof}

Broadly speaking, we require three things from our tiling of ECA rules - for a given rule $R_n$:
\begin{enumerate}
\item Encoding of each input and output of the $f_n$ for our rule $R_n$.
\item Handling of the `transfer of bits' from one represented cell to the cells lower left, lower centre, and lower right.
\item Fixing of upper half-plane boundary.
\end{enumerate}

For the purposes of this proof, we work on tiling the lower half-plane, with the lower border of the upper half-plane having colour $I$. This means that we do not have to worry about the pre-images of the inverse function $f^{-1}$ which can not be unique or even be a `Garden of Eden', meaning it is a configuration that has no pre-image. Thus simplifying the way in which we tile the plane by omitting these in the upper half plane, essentially fixing it with colour $I$.

We first present the base tiling we are going to use - horizontally aligned hexagons with diamond lozenges filling the gaps between them, as so:

\begin{center}
\begin{tikzpicture}
%\begin{scope}[xshift=0cm]\fillshapesimple{(0:1) -- (60:1) -- (120:1) -- (180:1) -- (240:1) -- (300:1) -- cycle}\end{scope}
\begin{scope}[xshift=0cm]\fillshapesimple{(0:1) -- (60:1) -- (120:1) -- (180:1) -- (240:1) -- (300:1) -- cycle}\end{scope}
\begin{scope}[xshift=2cm]\fillshapesimple{(0:1) -- (60:1) -- (120:1) -- (180:1) -- (240:1) -- (300:1) -- cycle} \end{scope}
\begin{scope}[xshift=4cm]\fillshapesimple{(0:1) -- (60:1) -- (120:1) -- (180:1) -- (240:1) -- (300:1) -- cycle} \end{scope}
\begin{scope}[xshift=6cm]\fillshapesimple{(0:1) -- (60:1) -- (120:1) -- (180:1) -- (240:1) -- (300:1) -- cycle} \end{scope}
\begin{scope}[xshift=0cm,yshift=-1.75cm]\fillshapesimple{(0:1) -- (60:1) -- (120:1) -- (180:1) -- (240:1) -- (300:1) -- cycle} \end{scope}
\begin{scope}[xshift=2cm,yshift=-1.75cm]\fillshapesimple{(0:1) -- (60:1) -- (120:1) -- (180:1) -- (240:1) -- (300:1) -- cycle} \end{scope}
\begin{scope}[xshift=4cm,yshift=-1.75cm]\fillshapesimple{(0:1) -- (60:1) -- (120:1) -- (180:1) -- (240:1) -- (300:1) -- cycle} \end{scope}
\begin{scope}[xshift=6cm,yshift=-1.75cm]\fillshapesimple{(0:1) -- (60:1) -- (120:1) -- (180:1) -- (240:1) -- (300:1) -- cycle} \end{scope}
\end{tikzpicture}
\end{center}

We present two tile schemas that we will make use of can be carried out to obtain a tile set $S_n$ for each ECA Rule $R_n$.

Firstly, we give a schema for the hexagon tiles that will code the actual rule action. For $a,b,c, f_n(a,b,c) \in \{0,1\}$ we define our tile schema:

\begin{center}
\begin{tikzpicture} % this formed the prototype for \hexagontile
\begin{scope}[xshift=0cm]\fillshapesimple{(0:0) -- (180:2) -- (120:2) -- (0:0) -- (120:2) -- (60:2) -- (0:0) -- (60:2) -- (0:2) -- (0:0) -- (0:2) -- (300:2) -- (240:2) -- (180:2) -- cycle}
\draw (150:1) node {$a$};
\draw (90:1) node {$b$};
\draw (30:1) node {$c$};
\draw (270:1) node {$f_n(a,b,c)$};
\end{scope}
\end{tikzpicture}
\end{center}
where $f_n$ is the operation of applying rule $n$ to the three input bits $a,b,c$. Note, if required we can use similar notation to the 4-tuple codes we used for Wang tiles - specifically: $\langle a,b,c,f_n(a,b,c) \rangle$

We can see that for $a,b,c \in \{0,1\}$ there are 8 prototiles that we can define as our basis for each ECA tiling. These are as follows:

\begin{center}
\begin{tikzpicture}[scale=0.6]
\hexagontile{1}{1}{1}{f_n(1,1,1)}
\begin{scope}[xshift=4.5cm]\hexagontile{1}{1}{0}{f_n(1,1,0)}\end{scope}
\begin{scope}[xshift=9cm]\hexagontile{1}{0}{1}{f_n(1,0,1)}\end{scope}
\begin{scope}[xshift=13.5cm]\hexagontile{1}{0}{0}{f_n(1,0,0)}\end{scope}
\begin{scope}[xshift=0cm, yshift=-4cm]\hexagontile{0}{1}{1}{f_n(0,1,1)}\end{scope}
\begin{scope}[xshift=4.5cm, yshift=-4cm]\hexagontile{0}{1}{0}{f_n(0,1,0)}\end{scope}
\begin{scope}[xshift=9cm, yshift=-4cm]\hexagontile{0}{0}{1}{f_n(0,0,1)}\end{scope}
\begin{scope}[xshift=13.5cm, yshift=-4cm]\hexagontile{0}{0}{0}{f_n(0,0,0)}\end{scope}
\end{tikzpicture}
\end{center}

We next define our diamond lozenges as being tiles that are vertically and horizontally quadrisected and use the following tile schema, for $s, t \in \{0,1  \}$:

\begin{center}
\begin{tikzpicture} % this formed the prototype for \lozengetile function
\begin{scope}[xshift=0cm]\fillshapesimple{(0:0) -- (60:2) -- (0:2) -- (300:2) -- (60:2) -- (0:2) -- cycle -- (300:2)}
\draw (30:0.66) node {$s$};
\draw (15:1.33) node {$t$};
\draw (330:0.66) node {$t$};
\draw (345:1.33) node {$s$};
\end{scope}
\end{tikzpicture}
\end{center}

This gives us our 4 connecting lozenges as follows: 

\begin{center}
\begin{tikzpicture}
\lozengetile{(0:0) -- (60:2) -- (0:2) -- (300:2) -- (60:2) -- (0:2) -- cycle -- (300:2)}{0}{0}{0}{0}
\begin{scope}[xshift=2.5cm]\lozengetile{(0:0) -- (60:2) -- (0:2) -- (300:2) -- (60:2) -- (0:2) -- cycle -- (300:2)}{0}{1}{0}{1}\end{scope}
\begin{scope}[xshift=5cm]\lozengetile{(0:0) -- (60:2) -- (0:2) -- (300:2) -- (60:2) -- (0:2) -- cycle -- (300:2)}{1}{0}{1}{0}\end{scope}
\begin{scope}[xshift=7.5cm]\lozengetile{(0:0) -- (60:2) -- (0:2) -- (300:2) -- (60:2) -- (0:2) -- cycle -- (300:2)}{1}{1}{1}{1}\end{scope}
\end{tikzpicture}
\end{center}

These connecting lozenges are required owing to a property of ECAs - namely, for some string $\sigma \in \{0,1\}^{<\omega}$, any bit $b_i \in \sigma$ is needed to calculate the bits $b'_{i-1},b'_i, b'_{i+1} \in \sigma'$. As such, these lozenges achieve the required `crossover' of these bits. These act in principle precisely the same as the `state swapping tile' in our previous theorem \ref{thm:WangECA}.

We will also need the following 3 `I' tiles to make our tiling `neat' and to define the first row of out tiling:

\begin{center}
\begin{tikzpicture}[scale=0.6]
\hexagontile{I}{I}{I}{0}
\begin{scope}[xshift=4.5cm]\hexagontile{I}{I}{I}{1}\end{scope}
% REDO this tile so that it is higher up, and we don't need to do any weird shifts to get it to tile in large diagrams :P
\begin{scope}[xshift=7.5cm]\lozengetile{(0:0) -- (0:2) -- (300:2) -- cycle -- (300:2) -- (0:1)}{}{}{I}{I}\end{scope}
\end{tikzpicture}
\end{center}

This will give us a flat edge for the top of the tiling, where we can now see that a tiling of the plane, with no gaps can be achieved, as shown in this diagram:

We can thus define the tiling algorithm for some ECA as follows:
\begin{enumerate}
\item Take the input for our ECA and code this using the `I' tiles.
\begin{itemize}
\item Pad the input with $\langle I, I, I, 0\rangle$ tiles as needed left and right to fill the left and right halves of our lower half-plane.
\item Ensure that the half-lozenge `I'-tiles are placed between the upper gaps between these hexagons.
\end{itemize}
\item Place the correct corresponding lozenge tiles between the hexagon tiles.
\item Place the now-defined hexagon tiles under each hexagon such that the upper 3 sides correspond to the lozenges on the upper left and upper right, and the hexagon immediately above.
\item Go to 2.
\end{enumerate}

Given this algorithm and this tile set, we can code any ECA by choosing the prescribed outputs from $f_n(x,y,z)$ from our rule $n$. Given this setup, we can see that our tiling gives a tiling of the half-plane without any holes, and such that it imitates the behaviour of any ECA.

\end{proof}

As an illustrated example, the full prototile set for Rule 30 can be found in figure \ref{fig:15TileRule30}

% Figure environment removed

Our proof of this theorem is unusual as it makes use of a non-standard planar tiling made up of hexagon and lozenge tiles - something that the author has not seen at all in the literature. This particular prototile arrangement lends itself to 3-ary iterated functions and dynamical systems slightly better than Wang tiles. Hence, they are included in this thesis as objects for potential further consideration.

\begin{corollary}[C. 2019]
There are chaotic ECA prototile sets of size 15.
\end{corollary}

\begin{proof}
This is immediate from the known properties of Rule 30, 90, etc. given in \cite{CattaneoFM00} - specifically, we can simply code these ECAs into prototiles using the scheme above and obtain a fixed-size prototile set that can code the required behaviour on a given input, such an input being given by an initial row of `I'-tiles from our original construction.
\end{proof}

\begin{corollary}[C. 2019]
There are Turing Complete prototile sets of size 15.
\end{corollary}

\begin{proof}
This corollary is immediate from the Turing completeness of Rule 110 \cite{CookUniversality} and the theorem \ref{thm:ECA-15-Hex} by the same argument given for 1. We note that we have to perform the following steps to obtain the result. Given a Turing Machine with index $e$ and a given input $x$:

\begin{enumerate}
\item Convert $\varphi_e$ to a cyclic tag system, to get $Tag_e$.
\item For $\varphi_e(x)$ we take $Tag_e$ and code this and $x$ into a single row input for our ECA.
\item Code this into the initial row `I'-tiles from our construction.
\end{enumerate}

With this done, we can allow our tiling to proceed row by row, and note that this codes each successive stage of the computation $\varphi_e(x)$ via the mapping above.
\end{proof}

% Figure environment removed

We include in figure \ref{fig:ExHexLozTiling} as a worked example of the initial few stages and columns of a Rule 30 ECA Hexagon and Lozenge tiling, demonstrating the function of the initializer tiles, the ECA hexagons, and the connecting lozenge tiles to demonstrate how an ECA can be encoded into a tiling of the plane.

\begin{conjecture}[C. 2019]
There exist ECA prototile sets of 8 tiles.
\end{conjecture}

By \cite{Rao2015} these cannot be formed from Wang tiles - this would mean that there is an aperiodic prototile set of fewer than 8 tiles, which they proved to not be the case. As such, a tiling of 8 tiles must be some other planar repeating tessellation with colours applied to different edges or areas in order to represent a prototile set of 8 tiles.
