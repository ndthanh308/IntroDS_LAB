\chapter{Aperiodicity, Tilings, and Logical Complexity}
\label{chap5}
% next resets the equation numbers to start at 1 at the start of the chapter
\setcounter{equation}{0}
\renewcommand{\theequation}{\thechapter.\arabic{equation}}

%------------------------------------------------------------------------------

\epigraph{Everything is simpler than you think and at the same time more complex than you imagine.}{\textit{Goethe (attrib.)}}

In this chapter we will explore and present results relating to tiling problems that ask about properties of total planar tilings - specifically whether they are periodic or aperiodic. We present first an overview of past results, and then provide new results inspired by our work in Chapter 3, culminating in a completeness result between periodicity/aperiodicity in infinite prototile sets and the class of problems of the form $(\Pi^1_1 \wedge \Sigma^1_1)$.

\section{Aperiodic Tilings and $\Sigma^1_1$/$\Pi^1_1$ Sets}

We will now look at aperiodicity in tilings and uncover some interesting facts about the $m$-reducibility of previously defined sets $WELL$ and $ILL$ to periodic and aperiodic tiling problems.

\subsection{Definitions of Periodic and Aperiodic Tilings}

We will use the following definitions in our analysis of aperiodic prototile sets derived from our definitions in Chapter 3.

\begin{definition}[Periodic Tilings]\label{def:PTile}
A tiling $T$ of the plane is a \emi{periodic tiling} iff there exists some non-zero vector $\mathbf{v}$ such that $\mathbf{v}$ defines a shift of $T$ such that $$ T = \mathbf{v} T $$

A set of prototiles $\mathcal{S}$ is \emph{periodic} iff it admits only periodic tilings of the plane. For computable $e$, let $PTile$ be as follows 
\begin{align*}
PTile = \{ e: & \, \varphi_e \text{ is the characteristic function for a set of prototiles} \\
	& \, \text{whose tilings are all periodic total tilings.} \} 
\end{align*}
\end{definition}

Our requirement that a periodic, set of prototile has \emph{only total} tilings that meet these criteria is how we avoid trivial periodic tilings by means of tilings that only tile some finite portion of the plane. 

Analogously we have the following definition for aperiodic tilings:

\begin{definition}\label{def:ATile}
A tiling $T$ of the plane is an \emi{aperiodic tiling} iff for any vector $\mathbf{v}$ necessary that $T \neq \mathbf{v} T$. Similarly, a set of prototiles $\mathcal{S}$ is \emph{aperiodic} iff it admits no periodic tilings of the plane.

For computable $e$, let $ATile$ be as follows:
\begin{align*}
ATile = \{ e: \varphi_e & \, \text{ is the characteristic function for a set of prototiles} \\
	&  \,\text{whose tilings are only aperiodic total tilings.}  \}
\end{align*}
\end{definition}

It is worth recalling that Simpson's equivalence of tiling problems on Wang prototiles with 2-dimensional subshifts of finite type in \cite{Simpson2007} is prophetic with respect to extending our gaze beyond domino problems and into questions of the existence of shifts of total tilings themselves.

\subsection{Overview of Aperiodicity} 

Whilst periodic tilings have been around since ancient times - of which a plethora of examples mathematical significance can be found in \cite{GrunbaumTP} - aperiodicity is relatively new. We will first discuss the origins of aperiodic tilings sets, and then set the scene and context in which some famous aperiodicity results find themselves.

\subsubsection{Origins of Aperiodic Prototile Sets}

As documented in \cite[P.520-600]{GrunbaumTP}, the study of aperiodicity in tilings did not occur until Robinson proved that such tilings must necessarily exist in 1968. Conway, Amman, and Penrose all made headways in the study of aperiodicity in tilings. One such result can be found in the following definitions and proposition - for which we shall use the presentation in \cite{Shen2010}:

\begin{definition}\label{def:macro-tile}
Let $S$ be a finite set of prototiles. Then a \emi{macro tile} is a square of size $n \times n$ filled with matching tiles from $S$.
\end{definition}

\begin{definition}
Let set of prototiles $S$ and a set of macro tiles $M$ be given. We say that $S$ \emph{implements} $M$ if any $S$-tiling can be split by horizontal and vertical cuts into macro-tiles $m \in M$.
\end{definition}

\begin{definition}
A set of prototiles $S$ is a \emi{self-similar prototile set} if it implements some macro-tile set $M$, with $M$ isomorphic to $S$, which we shall write $M \cong S$.
\end{definition}

Here, `isomorphic' means that we can find a one to one correspondence between the sets of $M$ and $S$ prototiles - that is, for some $m \in M$, we can find a corresponding $s \in S$ such that under a chosen mapping of the edge conditions of $m$, $s$ has the same edge conditions.

Note, that if $n$ exists and $S$ is self-similar, then $S$ will have total tilings of the plane, as for any  patch tiling, we can inflate the tilings with the substituted macro tilings to obtain arbitrarily large tilings of the plane by compactness. Though, we shall lose this compactness argument when we graduate from finite to infinite prototile sets.

\begin{proposition}[\protect{\cite[Sec. 4]{Shen2010}}]\label{prop:SHEN-Aperiodic}
A self-similar prototile set $S$ has only aperiodic tilings.
\end{proposition}

\begin{proof}
Proof from \cite{Shen2010}. Suppose for contradiction that a self-similar prototile set $S$ is periodic. We let $p \in \omega$ be the period of some $S$-tiling $T$. By definition, $T$ can be split uniquely into macro-tiles from $M \cong S$ by $n \times n$ cuts, for some unique $n \in \omega$. A shift by $p$ should respect this splitting, else we get a different splitting, so $p$ must be some multiple of $n$.

`Zooming out' from our tiling, by which we mean rescaling our tiling by some fixed factor, we can proceed in replacing each $M$ macro-tile by its corresponding $S$ tile, we get a $\frac{p}{n}$ shift of $T$. However, by the same reasoning $\frac{p}{n}$ must also be a multiple of n, so we can zoom out again, and continue this construction.

We must therefore conclude that $p$ is a multiple of $n^k$ for any $k$, meaning that $p$ is a zero vector. $\rightarrow \leftarrow$
\end{proof}

The classic instance of such results can be found in Penrose Tilings, specifically the presentation from \cite{Gummelt1996}, and the original article by Penrose in \cite{Penrose1979} - wherein Penrose shows how you can acquire aperiodic tilings of the plane from as few as two prototiles. Indeed, two distinct but related prototile sets are given: the Penrose Rhombi and Penrose Kite and Dart prototile sets.

Interesting tangents of study that have derived from the study of aperiodic tile sets has been found in the study of quasicrystals - crystalline lattice structures that are ordered but not periodic. Penrose tilings have been found to have given some insight into the icosahedral phases of quasicrystals - see \cite{Mackay1987}.

Their proofs of aperiodicity follow as analogous arguments to the above - by showing that the Penrose constructions `deflate' and `inflate' to copies of the tiling, we show that we can tile every arbitrary finite portion of the plane. Thus, by a basic compactness argument, we find that Penrose prototiles tile the plane. However, if they do so, then the inflation/deflation processes give the same bi-simulation argument as given by proposition \ref{prop:SHEN-Aperiodic}. As such, any Penrose tiling must then also be invariant under any linear shift, else they would fail to be self-similar in the way that there are, and so Penrose tilings are aperiodic.

There is a fantastic treatment of the underlying algebraic theory by de Bruijn in two papers: \cite{DEBRUIJN198139}, followed by \cite{DEBRUIJN198153} - both are dedicated to P\'{o}lya. The theory is quite exceptionally beautiful, but beyond the scope of this thesis to include. The essential idea that was given in this work is called the `cut and project' method, where a five-dimensional lattice is projected through a `window' onto the plane in order to acquire the corner points of a Penrose tiling. The original results can be found in \cite{DEBRUIJN198139}, with an excellent overview of this work and its relationship to actual physical phenomena can be found in the work in Au-Yang \etal \cite{AuYang2013}.

The existence of precisely 8 corner configurations in any Penrose tiling is also given in \cite{DEBRUIJN198153}, which is again work that is worthy of study but beyond the scope of this thesis. 

% Figure environment removed

In the continuation of their work we outlined above, Shen \etal in \cite{Shen2010} produced some very novel conditions under which aperiodic tilings could be found by means of fixed points - they show that it is possible to have some predicate $S$ that is isomorphic to the set of tiles $T$ that is used to implement it. This is analogous to the challenge of creating Quines in computer science - that is, computer programs whose output upon being run is to print their own source code. Just as Quines are necessarily existing, so are these Shen fixed-point tilings.

\subsubsection{Aperiodic Wang Prototiles}

As we quoted in the introduction, Simpson in \cite{Simpson2007} draws the equivalence between tiling problems in Wang prototile sets and 2-dimensional subshifts of finite type. Utilising this as our base intuition, we present now an overview of aperiodicity in Wang tiles, for which there have been some very interesting and recent developments.

Building on from this basis, the question was asked about what the \emph{smallest} aperiodic Wang prototile sets might be. The survey in \cite{Rao2015} gives a fascinating timeline: Berger originally came up with a set of 20,426 Wang prototiles that was aperiodic for his thesis. By \cite{GrunbaumTP}, a set of 24 aperiodic Wang prototiles was presented, with improvements by Robinson and Amman along the way.

% Figure environment removed

After a result by Kari \cite{Kari1996}, it was Culik who set a record in \cite{CULIK1996245} - an aperiodic set of 13 Wang prototiles, which we have included in figure \ref{fig:13-ATile}. These were derived from the states of automata transducers which can compute non-repeating reals. As such, any prototile set coding this behaviour will likewise be non-repeating, thereby aperiodic. 

The most significant breakthrough in this area has been a recent publication from Jeandel and Rao in \cite{Rao2015} where they proved the following two important results:

\begin{theorem}[\protect{\cite[Thm. 5]{Rao2015}}]
There exists an aperiodic set of 11 Wang prototiles.
\end{theorem}

% Figure environment removed

\begin{theorem}[\protect{\cite[Thm. 1]{Rao2015}}]
There is no aperiodic Wang prototile set with 10 tiles or fewer.
\end{theorem}

The proof of both of these theorems are computer assisted, and they used a series of innovative techniques to check the tilings they generated - from the simple cases of repeating patterns, through to the complicated cases that were in fact subsets of the Kari and Culik constructions above. These more advanced cases - of which there were 4 - were not computer-checkable, so the proofs and checks were carried out by hand. It transpired that each of these aperiodic tilings were coding transducers in some way, and as such were given by similar reasoning to the aperiodicity results due to Kari and Culik. We have included the 11-prototile set in figure \ref{fig:11-ATile}.

It has been postulated, and subsequently answered to a lesser degree than expected in \cite{Socolar2011}, the question ``Does there exist a single-prototile that tiles the plane aperiodically?'' The Taylor-Socolar tile detailed in \cite{Socolar2011} achieves this, but by the use of a tile that is defined with gaps between its various pieces - though tilings of the plane utilising this tile cover every point.

In general, the literature has not, however, given any consideration to infinite sets of prototiles and their periodicity or aperiodicity. However, as seen in \cite{Harel1986}, the aperiodic properties of some finite prototile sets - specifically that if a specified tile appears only finitely often in a planar tiling, then this must be an aperiodic tiling - were found to code $\Pi^1_1$ statements, indicating that perhaps this would be some interesting candidate for further analysis and study. 

\subsubsection{Quasi-periodicity of tilings}\label{subsec:quasiperiodic}

When observing the properties of Penrose tilings, it is immediate that certain patterns recur regularly, even though the overall tiling is aperiodic. Such tilings are in the class of \emi{quasi-periodic} tilings, which we define as follows, from \cite{Delvenne2004}:

\begin{definition}\label{def:quasiperiodic}
For a given prototile set $S$, $S$ is \emph{quasi-periodic} iff each $S$-tiling of the plane is of the form such that for every pattern $u$ of the tiling there is an integer $k$ such that $u$ appears in every $(k \times k)$ patch of tiles. 
\end{definition}

Where here a \emi{pattern} is any valid, finite patch of tiles that occurs in our tiling. Intuitively, something is quasiperiodic if any finite patch can be found occurring infinitely often and within a bound in any tiling. As a reference, consider a star-like pattern in a Penrose tiling. 

There is a lot of interesting work found in papers such as Delvenne and Blondel \cite{Delvenne2004}, and survey papers connecting quasicrystals to quasi-periodic tilings like Schechtman \cite{Schechtman2013}. The most interesting parts of these are the way in which Penrose tilings mimic and indeed accurately code actual physical surfaces found in Shi \etal in \cite{Shi2017} - where we can note that their 7 diagrams of the ``angles and islands around each vertex'' line up with de Bruijn's derived unique vertex configurations for Penrose tilings found in \cite{DEBRUIJN198139} and \cite{DEBRUIJN198153}. We note that, although these 7 configurations are not the 8 identified by de Bruijn, we suspect that given two of the configurations in the mathematics are identical with edge-conditions removed, they look to be identical under the microscope in \cite{Shi2017}.

Such connections are found in other quasicrystals which we alluded to previously - \eg Subramanian \etal in \cite{Subramanian2016}, Shi \etal \cite{Shi2017} and Au-Yang \etal \cite{AuYang2013} are all readily accessible physics papers that make extensive use of the developed mathematics behind Penrose tilings as quasicrystals. This is, however, a digression from the main content of this thesis.

Indeed, the work of Socolar \etal in \cite{Socolar2011} is a very interesting way of determining the dynamics of this aperiodic tiling system. We will consider more the dynamics of tilings in Chapter 6 - but it is worth noting that it is an open problem as to whether the tile-by-tile tilings of the plane due to the method in \cite{Socolar2011} does indeed lead to planar tilings.

\section{Periodicity and Aperiodicity of $ILL$}

\begin{theorem}[C. 2019]\label{thm:ILL-ATile}
$$ ILL \leq_m ATile $$
\end{theorem}

\begin{proof}
To see this fact, we note that the construction of our function $h$ in the proof of theorem \ref{thm:TILE-ILL} gives an infinite set of prototiles $\mathcal{S}$ that tiles the plane in such a way that the root tile will only occur once, and every point $(x,y)$ in the plane has some unique tile in $\mathcal{S}_e$ that covers it. As such, any ill-founded tree $e \in ILL$ coded into a $\mathcal{S}_e$ by $h$ in our given construction is necessarily aperiodic. Thus it follows that for any $e \in ILL$, our given $h(e) \in ATile$.

Conversely, any $h(e) \in ATile$ must tile the plane, and as such our $e$ must be in $ILL$ otherwise it would be a well-founded tree, and so not tile the plane as outlined in our previous proof.
\end{proof}

It was, however, found that the following additional result could also be obtained:

\begin{theorem}[C. 2019]\label{thm:ILL-PTile}
$$ ILL \leq_m PTile $$
\end{theorem}

\begin{proof}
We can obtain the result by an adapting the procedure in the proof from \ref{thm:TILE-ILL} in the following way. We require a computable $f$ such that $$ \forall e (e \in ILL \iff f(e) \in PTile ) $$

We start by defining our colours as the following:

\begin{itemize}
\item Let $\lambda$ denote the empty string, and let $\lambda^U, \lambda^D$ be unique colours.
\item Fix $M$ unique, and $U_i, D_i$ unique for all $i \in \omega$.
\item Let $\alpha \in \omega^\omega$, and for all $i \in \omega$, let $\sigma_i \in \omega^{< \omega}$ denote successive initial segments of $\sigma$ of length $i$ such that $ \sigma_0 \prec \sigma_1 \prec \ldots \prec \sigma_i \ldots \prec \alpha$. 
\item We fix for each $\sigma_i$ an `up' $\sigma^U_i$ and `down' $\sigma^D_i$ colour that will be used in the prototile set construction.
\item Let $\sigma_0 = \lambda$ as before.
\end{itemize}

With these fixed, let $e \in ILL$ be given. We will construct our prototile set from the following schema tiles:

We start with a modified \textbf{root tile}:
\begin{center}
\sampletile{$M$}{$\lambda^U$}{$M$}{$\lambda^D$}
\end{center}

Next, we require \textbf{column tiles} of the following form:
\begin{center}
\sampletile{$U_i$}{$\sigma_i^{U \frown} n$}{$U_i$}{$\sigma_i^U$} \sampletile{$D_i$}{$\sigma^D$}{$D_i$}{$\sigma_i^{D\frown} n$}
\end{center}

We then construct our prototile set $\mathcal{S}_e$ similarly to the previous proof, by colouring the above schema tiles as follows:

\begin{itemize}
\item Colour the root tile with the tuple $\langle M, \lambda^U, M, \lambda^D \rangle$ and put this into $\mathcal{S}_e$.
\begin{itemize}
\item \textbf{NB} - we still maintain the difference between the `up' and `down' variants of our empty string symbol in order to prevent trivial root-tile only tilings of the plane, though they would be undoubtedly periodic.
\end{itemize}
\item We fix some path $p \in \varphi_e$ such that $\sigma_n \prec p$ for $\sigma_n \in \omega^{< \omega}$, and add a column tile where it holds that $\varphi_e(p \upharpoonright n) = 1$.
\begin{itemize}
\item For $\sigma_0$ we use the appropriate placement of $\lambda^U$ and $\lambda^D$ as before.
\item We also select distinct colours for $\sigma^U_i$ and $\sigma^D_i$ in order that we fail to tile the plane if $e \notin ILL$.
\end{itemize}
\end{itemize}

We can now verify that for each $e \in ILL$ we get $f(e) \in PTile$. The core idea in this construction is to have infinitely many copies of our central column tilings from our previous proof, laid out in such as way that for left or right shift of our tiling, we get the same tiling back, thus $f(e)$ would be periodic. 

As before, we can define our tiling function $\Phi^p : \mathbb{Z}^2 \rightarrow S_e$ as follows:
\begin{itemize}
\item For $\Phi^p(x,0)$ return the root tile $\langle M, \lambda^U, M , \lambda^D \rangle$.
\item For $\Phi^p(x,y)$, with $\sigma = p \upharpoonright y$, 
\begin{itemize}
\item If $y > 0$ return the tile $ \langle U_y, \sigma^{U \frown} n , U_y, \sigma^U \rangle$
\item If $y < 0$ return the tile $ \langle D_y, \sigma^D , D_y ,\sigma^{D \frown} n \rangle$
\end{itemize}
\end{itemize}

To see that our tilings are periodic, note that all of our root tiles will form an infinite middle-row of tiles that can be left or right shifted. We then build up our tilings, noting that each successive column will have prototiles selected that code specifically some copy of our path $p$ upwards or downwards. Thus, every $\mathcal{S}_e$-tiling will have infinitely many leftwards or rightwards shifts. 

Thus, if $\mathbf{v}$ is a `shift right one' vector, then we have that an $\mathcal{S}_e$-tiling $T_e$ has the property $$T_e = \mathbf{v} T_e$$ meaning that $f(e) \in PTile$.

Suppose we have some $f(e) \in PTile$, then it follows that from any root tile we can extract some infinite path moving upwards that gives us that $e \in ILL$. We can also locate a root tile from any tile we select in our $\mathcal{S}_e$-tilings by moving appropriately down our $UM_i$'s or up our $DM_i$'s until a root tile is reached.

From this position we can then follow our tiling upwards in order to extract an infinite path that was given by $e$. As such, if our tiling is total and total, $e \in ILL$.
\end{proof}

% Figure environment removed


In figure \ref{fig:PTileILL} we give an example of the tiling construction for theorem \ref{thm:ILL-PTile} for the initial segment $\sigma = 01$. This illustrates the way in which we create vertical dual copies of the given path from our ill-founded tree in such a way that any left shift vector $\mathbf{l}$, or right shift vector $\mathbf{r}$ and a given $T_e$, we have that $$ \mathbf{l}T = T = \mathbf{r}T$$

% Figure environment removed

Figure \ref{fig:ShapeTilingILL-PTile} shows the overall shape of this tiling construction used in the proof of theorem \ref{thm:ILL-PTile}. This diagram is complimentary to the previous figure \ref{fig:PTileILL}.

Note that we were required to preserve the up vs. down directions of our paths, which we were not required to do before. The reason being is that we wanted to preserve that the existence of a tiling derived with $f(e)$ implies that our original $e \in ILL$. We could very well have constructed periodic tilings of $e$'s that are either in $WELL$ or $ILL$. This realisation drove the results in the next section \ref{sec:PAWell}.

\subsection{Periodicity and Aperiodicity of $WELL$}\label{sec:PAWell}

Before we carry on with the proofs in this section we will need the following tool - the ability to take disjoint unions of prototile sets. Our requirement for this construction can be outlined in the following definition and subsequent proposition:

\begin{definition}
We say that two prototile sets $S_1$ and $S_2$ have \emph{common edge meets} iff for some tile $t_i \in S_1$, with $t_i = \langle l_i, u_i, r_i, b_i \rangle $, there exists a tile $s_i \in S_2$ such that one of the following hold:
\begin{itemize}
\item $s_i = \langle r_i, \cdot, \cdot, \cdot \rangle$
\item $s_i = \langle \cdot, b_i, \cdot, \cdot \rangle$
\item $s_i = \langle \cdot, \cdot, l_i, \cdot \rangle$
\item $s_i = \langle \cdot, \cdot, \cdot, u_i \rangle$
\end{itemize}
where $\cdot$ denotes a `wildcard placeholder' for any other possible colour.
\end{definition}

We say that two prototiles $S_1$ and $S_2$ have no common edge meets if the above definition does not hold - intuitively, you cannot place any tile from $S_1$ next to any tile from $S_2$, and vice versa. The following proposition demonstrates an important consequence of two prototile sets being edge-meet disjoint.

\begin{proposition}[C. 2019]\label{prop:DisjointEdgeMeets}
If two periodic (aperiodic) prototile sets $S_1, S_2$ share no common edge meets, then their union $S_1 \cup S_2$ is also periodic (aperiodic).
\end{proposition}

\begin{proof}
Let periodic prototile sets $S_1, S_2$ be given. If $S_1$ and $S_2$ share no common edge meets, then for any selection of a tile $t \in S_1 \cup S_2$, the resultant tiling must be formed from only tiles from $S_1$ if $t \in S_1$ or $S_2$ otherwise, as the edge-meet criteria from each prototile set is incompatible. Thus any tiling from such a $S_1 \cup S_2$ is periodic.

We note that the same argument holds for $S_1$ and $S_2$ being aperiodic. 
\end{proof}

To illustrate an example of where this fails - which is essentially the canonical case that we wish to avoid - we provide the following:

\begin{example}\label{example:UnionTileSets}
Let it be given that a periodic tiling consisting of squares can be made aperiodic by the bisection of a single randomly chosen square into two rectangles. Thus we give the following example to illustrate how this can be done in Wang prototile sets, and thereby show the importance of the lack of edge-meets between prototile sets. 

Let $S_1$ be given by the prototile 
\begin{center}
\sampletilefill{red}{blue}{red}{blue}
\end{center}
and let $S_2$ be given by the prototiles
\begin{center}
\sampletilefill{red}{blue}{green}{green} \sampletilefill{green}{blue}{red}{green} \\
\vspace{2mm}
\sampletilefill{red}{green}{green}{blue} \sampletilefill{green}{green}{red}{blue}
\end{center}

Clearly both $S_1$ and $S_2$ are periodic by themselves. However, $S_1 \cup S_2$ will have tilings that, say, feature only finitely many of the patch tilings given by the prototiles in $S_2$, and would therefore be aperiodic. The same could be done by a single column of tiles from the prototile in $S_1$ being inserted into an $S_2$-tiling, which would also make it aperiodic.
\end{example}

As such, given the example above, we present a construction that provides a way of combining prototile sets, yet preserving the periodicity and aperiodicity conditions we wish to. 

\begin{definition}[Disjoint Union of Tile Sets]\index{disjoint union of prototile sets}
Let the \emph{disjoint union of prototile sets} $A$ and $B$, denoted $A \sqcup B$, be given as follows:
\begin{itemize}
\item For each prototile $t \in A$, let $t = \langle a,b,c,d \rangle$ then this gets mapped to $$ \langle a,b,c,d \rangle \mapsto \langle (1,a), (1,b), (1,c), (1,d) \rangle $$
\item For each prototile $s \in B$, let $s =\langle e,f,g,h \rangle$ then we map this similarly: $$ \langle e,f,g,h \rangle \mapsto \langle (2,e), (2,f), (2,g), (2,h) \rangle $$
\end{itemize}

Likewise, for any arbitrary number of prototile sets $S_i$ for $i \in \omega$ the disjoint union $ \bigsqcup_{i \in \omega} S_i $ is given by mapping each $t_j \in S_i$, with $t_j = \langle l_j, u_j, r_j, b_j \rangle$ by $$  \langle l_j, u_j, r_j, b_j \rangle \mapsto  \langle (i,l_j), (i,u_j), (i,r_j), (i,b_j) \rangle $$
\end{definition}

The intuition behind this disjoint union is the ability to take two sets of (potentially infinite) prototile sets and `apply a tint' to each prototile in each prototile set, thereby placing us in the position given in proposition \ref{prop:DisjointEdgeMeets}. Thus, we can talk about the tiling properties of the resultant disjoint union, but each subset will be incompatible for tiling with any others. 

Our intention is to be able to talk about the disjoint union of two prototile sets $A$ and $B$ in the following way, after proposition \ref{prop:DisjointEdgeMeets}:
\begin{itemize}
\item If both $A$ and $B$ are periodic (aperiodic) then the disjoint union $A \sqcup B$ will be periodic (aperiodic), and so will likewise belong to $PTile$ ($ATile$).
\item If $A$ is periodic and $B$ is aperiodic, or vice versa, then $A \sqcup B$ will have both periodic and aperiodic tilings and so will belong to neither $PTile$ nor $ATile$.
\end{itemize}

In our previous example \ref{example:UnionTileSets}, were we to take $S_1 \sqcup S_2$, then we would only have periodic tilings, given both $S_1$ and $S_2$ are periodic, total planar tilings, and would fail to share edge-meet conditions in $S_1 \sqcup S_2$.

Prototile sets that are not in either $PTile$ nor $ATile$ are relatively easy to find. A straightforward example is the set consisting of the following sixteen prototiles:
\begin{center}
\sampletile{$0$}{$0$}{$0$}{$0$} 
\sampletile{$0$}{$0$}{$0$}{$1$}
\sampletile{$0$}{$1$}{$0$}{$0$}
\sampletile{$0$}{$1$}{$0$}{$1$} \\
\smallskip
\sampletile{$0$}{$0$}{$1$}{$0$}
\sampletile{$0$}{$0$}{$1$}{$1$}
\sampletile{$0$}{$1$}{$1$}{$0$}
\sampletile{$0$}{$1$}{$1$}{$1$} \\
\smallskip
\sampletile{$1$}{$0$}{$0$}{$0$}
\sampletile{$1$}{$0$}{$0$}{$1$}
\sampletile{$1$}{$1$}{$0$}{$0$}
\sampletile{$1$}{$1$}{$0$}{$1$} \\
\smallskip
\sampletile{$1$}{$0$}{$1$}{$0$}
\sampletile{$1$}{$0$}{$1$}{$1$}
\sampletile{$1$}{$1$}{$1$}{$0$}
\sampletile{$1$}{$1$}{$1$}{$1$}
\end{center}

These prototiles allow us to encode two binary strings - one going vertically, and another horizontally. Thus, if we place tiles such that they encode periodic repeating strings, such as ``$0101010101\ldots$" using these prototiles in our tiling of the plane, then our tiling will clearly be periodic. 

However if we use non-repeating, aperiodic strings - such as using a Martin-L\"{o}f random string vertically and the binary version of Champernowne's constant\footnote{This is constructed by concatenating every binary number: $0110111001011101111000\ldots$} horizontally - then our tiling will be clearly aperiodic. 

Essentially, in this tiling we code two binary strings - $\sigma$ going left to right and $\tau$ going up and down. If either $\sigma$ or $\tau$ (or both) are periodic, then the tiling is periodic. Else, the tiling is aperiodic.

We will use our previous constructions, and fix the following construction names.

\begin{definition}\label{def:AITPIT}
Let the following short hand definitions be given:
\begin{itemize}
\item \textbf{AIT} (Aperiodic Ill-founded Tilings) - the construction found in the proof of theorem \ref{thm:TILE-ILL}.
\item \textbf{PIT} (Periodic Ill-founded Tilings) - the construction found in the proof of theorem \ref{thm:ILL-PTile}.
\end{itemize}
\end{definition}

Recall, our constructions here take any ill-founded tree and generate either periodic or aperiodic prototile sets as required. We shall use these constructions in the following sections in conjunction with our notion of disjoint union of prototile sets (`prototile set tinting') in order to obtain the following results.

\begin{theorem}[C. 2019]\label{thm:WELL-PTile}
$$ WELL \leq_m PTile $$
\end{theorem}

\begin{proof}
As before, we want some recursive function $k$ such that $$ e \in WELL \iff k(e) \in PTile $$

We begin by fixing some recursive ill-founded tree $R$ and feeding this through the \textbf{PIT} construction to obtain a set of prototiles $\mathcal{R}$ that has only periodic tilings of the plane for any infinite path in $R$.

We next take our $e$ and pass this through the \textbf{AIT} construction to get a prototile set $U_e$ that tiles the plane only if $e \notin WELL$. We then let our desired prototile set $S_e$ generated by this recursive method be $$ S_e = \mathcal{R} \sqcup U_e$$

If $e \in WELL$ then the only tilings of the plane will be given by $\mathcal{R}$, and as such, $k(e) \in PTile$. 

If $e \notin WELL$ then both $\mathcal{R}$ and $U_e$ will give tilings of the plane, meaning that $k(e) \notin PTile$, as it would have both periodic \emph{and} aperiodic tilings.
\end{proof}

By a nearly identical argument we shall obtain the following result:

\begin{theorem}[C. 2019]\label{thm:WELL-ATile}
$$ WELL \leq_m ATile $$
\end{theorem}

\begin{proof}
We proceed exactly as above, to construct a recursive $l$ such that $$e \in WELL \iff l(e) \in ATile$$ but with our argument switching the periodic and aperiodic constructions from our previous proof.

We fix a recursive ill-founded tree $R$ and now feed this through the \textbf{AIT} construction, giving us a new $\mathcal{R}$ we shall use. Likewise, we will take our $e$ and pass this through the \textbf{PIT} construction to get $V_e$. Our prototile set $S_e$ is now given by $$ S_e = \mathcal{R} \sqcup V_e $$

If $e \in WELL$ then as above, the only tilings of the plane will come from $\mathcal{R}$, except that this time they shall be aperiodic, and so $l(e) \in ATile$. 

Similarly, if $e \notin WELL$ then both $\mathcal{R}$ and $V_e$ will give tilings of the plane, and given $V_e$ gives periodic tilings, we have that $l(e) \notin ATile$.
\end{proof}

\subsubsection{An Alternative Proof}

We note that there exist alternative and more intuitive ways that we can prove both \ref{thm:WELL-PTile} and \ref{thm:WELL-ATile} that we shall provide here.

\begin{proof}[Alternative Proof for \ref{thm:WELL-PTile}, C. 2019]
We begin by using the construction in \ref{thm:SNT-WELL} - the finite diamond-shaped patches of tiles that will not tile the plane iff the tree whose paths it tiles is well-founded. To this tiling set, we add the following prototile schemes:

\textbf{Corner tiles:}
\begin{center}
\sampletile{$\sigma_n$}{$\sigma_n$}{$\sigma_n$}{$\sigma_n$}
\end{center}
for each $\sigma \in \omega^{<\omega}$, with $|\sigma| = n$ and $\sigma \in \varphi_e$.

\textbf{Edge Connecting tiles:}

\begin{center}
\sampletile{$\sigma^2_n$}{$\sigma^4_n$}{$\sigma^4_n$}{$\sigma^2_n$}
\sampletile{$\sigma^3_n$}{$\sigma^3_n$}{$\sigma^1_n$}{$\sigma^1_n$}
\end{center}
for each $\sigma_n$ as above.

The idea of these tiles are, as we shall see, to fill in the gaps between fragments of our original prototile set construction, and provide total and periodic tilings of the plane.

We construct our library $\mathcal{U}$ as before, and extract $U_e$ as before, adding in the requisite Corner tiles and Edge Connecting tiles, being careful to remove the quadrant filling tiles we had included so far for paths $\sigma^j_n$. We then note that we only require two pairs of quadrant tile types that will meet in the total planar tiling - $\sigma^2_n$ tiles will meet with $\sigma^4_n$ tiles, and $\sigma^3_n$ tiles will meet with $\sigma^1_n$ tiles.

The resulting $U_e$ then takes each of our previous patch tilings and allows us to join them together by the addition of the connective tiles. Thus, we are effectively tiling with our `meta-tiles' formed from the patch tilings we constructed above. 

So, we can let this above procedure be a computable function $p$. If $e \in WELL$ then $p(e)$ will construct a $U_e$, all of whose tilings are periodic total tilings of the plane. Thus $p(e) \in PTile$.

Likewise, if $e \notin WELL$ then only one path will be tiled, and will be infinite and total. However, as it will only use the root tile once in any tiling, it follows that there are no linear shifts of our tiling that can be performed. Thus, $p(e) \notin PTile$.

As such, we have \[ e \in WELL \iff p(e) \in PTile  \] which gives us our $m$-reduction \[ WELL \leq_m PTile \]
\end{proof}

\section{Completeness of $PTile$ and $ATile$}

Given we have assessed the relationship of $WELL$ and $ILL$ to tiling problems regarding periodicity and aperiodicity, it is natural to next seek some completeness for this general class of problems. In this spirit, we present the following theorem:

\begin{theorem}[C. 2019]\label{thm:X-ATile}
Let $X \subset \omega$ be in $( \Pi^1_1 \wedge \Sigma^1_1 )$, that is $$ X = \{ n : \chi(n) \wedge \psi(n) \}$$ such that $\chi \in \Sigma^1_1$ and $\psi \in \Pi^1_1$, then $$X \leq_m ATile$$
\end{theorem}

Intuitively, this proof arises from the fact that our definitions of $PTile$ and $ATile$ are both of the form ``there exists a tiling" followed by some general statement about all of the tilings given by that prototile set. 

In this proof, we will pass each statement through the periodic or aperiodic construction for the ill-founded ($\Pi^1_1$) side of the conjunction as desired. We then take the disjoint union of this with the $\Sigma^1_1$ side of the construction being passed through the opposite (a)periodic construction to obtain the result. The formal proof now follows.

\begin{proof}
To show that $X \leq_m ATile$, we want some computable $h$ such that $$n \in X \iff h(n) \in ATile$$. 

First let us define our two recursive functions $f:X \rightarrow \omega$ and $g : X \rightarrow \omega$ as follows:
\begin{itemize}
\item $f(n)$ be such that $(\varphi_{f(n)}$ is a tree $\wedge f(n) \in ILL) \leftrightarrow \chi(n)$
\item $g(n)$ be such that $(\varphi_{g(n)}$ is a tree $\wedge g(n) \in WELL) \leftrightarrow \psi(n)$
\end{itemize}

Our function $f$ holds only if the $\Sigma^1_1$ side of our formula given by $\chi(n)$ and constructs index that computes the tree $T \subseteq \omega^{<\omega}$ given by this formula, resulting in an index $f(n) \in ILL$.

Likewise the function $g$ holds if the $\Pi^1_1$ side of our formula given by $\psi(n)$ holds, and constructs index that computes the tree $T \subseteq \omega^{<\omega}$ given by this formula, resulting in an index $g(n) \in WELL$.

Now let the $U,V$ be defined as follows:
\begin{itemize}
\item $U$ is the set of prototiles obtained by passing $\varphi_{f(n)}$ through the \textbf{AIT} construction to create an aperiodic prototile set for $\varphi_{f(n)}$ being ill-founded.
\item $V$ is the set of prototiles obtained by passing $\varphi_{g(n)}$ through the \textbf{PIT} construction to create a periodic prototile set for $\varphi_{g(n)}$ not being well-founded.
\end{itemize}

Both of these constructions are given by the previous results, and so are known computable reductions. $h(n)$ be then the function that produces the prototile set that is the disjoint union $S_n = U \sqcup V$. 

These two infinite sets of prototiles have both been passed through constructions designed for total planer tilings intended for ill-founded trees. Thus, the prototile set corresponding to our well-founded prototiles, $V$, will only tile the plane if $\neg \psi(n)$ holds. Given this, we now utilise our disjoint union in obtaining $S_n$ in order to restrict the behaviour of our combined prototile sets to obtain the result we want. 

We thus have the following 4 cases: 
\begin{enumerate}
\item $\chi(n) \wedge \psi(n)$ - In this case, everything is as we would like it to be, as the only planar $S_n$-tilings will be given by $U$, which are aperiodic.
\item $\neg \chi(n) \wedge \psi(n)$ - In this case we will get no total $S_n$-tilings of the plane.
\item $\chi(n) \wedge \neg \psi(n)$ - In this case we will get both periodic and aperiodic $S_n$-tilings of the plane.
\item $\neg \chi(n) \wedge \neg \psi(n)$ - In this case we will only get periodic $S_n$-tilings of the plane.
\end{enumerate}

Given by our construction of $h$ we only get aperiodic tilings of the plane for $n$ precisely when $(\chi(n) \wedge \psi(n))$, it follows that $n \in X \rightarrow h(n) \in ATile$.

For the converse argument, take that $h(n) \in ATile$ is given. For the class of $S_e$-tilings $\mathcal{T}$ given by $h(e)$ we take some $T \in \mathcal{T}$ and ask if $T$ is total. If $T$ is a total tiling, then we can extract (as described in \ref{thm:TILE-ILL}) an infinite path corresponding to the ``$\varphi_{f(n)} \leftrightarrow \chi(n)$" part of the definition of $n \in X$. 

If $T$ is not a total tiling, then we know that we have infinitely many copies of the path given by $\varphi_{g(n)}$ corresponding to the ``$\varphi_{g(n)} \leftrightarrow \psi(n)$" part of the definition of $n \in X$. 

Thus, by examining the class of $S_n$-tilings given by $h(n) \in ATile$ we can get that $n \in X$, for any $X$ of the desired form in the theorem.
\end{proof}

\begin{theorem}[C. 2019]\label{thm:X-PTile}
For $X = \{ n : \chi(n) \wedge \psi(n) \}$, with $\chi(n) \in \Sigma^1_1$ and $\psi(n) \in \Pi^1_1$, then $$ X \leq_m PTile $$
\end{theorem}

\begin{proof}
Our proof proceeds precisely as for \ref{thm:X-ATile} in order to give a recursive $k$ such that $$ n \in X \iff k(n) \in PTile $$ except that we differ in constructing $U$ and $V$ as follows:
\begin{itemize}
\item $U$ is the set of prototiles obtained by passing $\varphi_{f(n)}$ through the \textbf{PIT} construction to create a periodic prototile set for $\varphi_{f(n)}$ being ill-founded.
\item $V$ is the set of prototiles obtained by passing $\varphi_{g(n)}$ through the \textbf{AIT} construction to create an aperiodic prototile set for $\varphi_{g(n)}$ not being well-founded.
\end{itemize}
Wherein we have essentially swapped the roles of \textbf{PIT} and \textbf{AIT} in order to achieve our result. We can then re-analyse the outcomes as follows:
\begin{enumerate}
\item $\chi(n) \wedge \psi(n)$ - In this case, we only get periodic $S_n$-tilings of the plane.
\item $\neg \chi(n) \wedge \psi(n)$ - In this case we will get no total $S_n$-tilings of the plane.
\item $\chi(n) \wedge \neg \psi(n)$ - In this case we will get both periodic and aperiodic $S_n$-tilings of the plane.
\item $\neg \chi(n) \wedge \neg \psi(n)$ - In this case we will only get aperiodic $S_n$-tilings of the plane.
\end{enumerate}

Thus, our $k$ has precisely the same properties as our previous $h$, with the periodicity properties reversed. As such, the forwards and reverse directions of our implication are precisely the same, giving our result.
\end{proof}

Once we define our constructions in these results, the entire proofs are essentially captured in the four cases. The fact that both $ATile$ and $PTile$ have interchangeably periodic and aperiodic $\Sigma^1_1$ and $\Pi^1_1$ parts was unexpected, but actually quite natural.

The background intuition for these results was the observation that the existence of a tiling, and the fact that all tilings either have exclusively or no periodic/aperiodic parts. If we allow ourselves to use quantification of sets in the analytic hierarchy as above, we obtain the following corollary:

\begin{corollary}[C. 2019]\label{cor:ATilePTileComplete}
Aperiodicity and periodicity for infinite prototile sets is $(\Sigma^1_1 \wedge \Pi^1_1)$-complete
\end{corollary}

\begin{proof}
This follows from our previous theorem \ref{thm:X-ATile} and theorem \ref{thm:X-PTile} working in tandem. Any problem given in the form $$\zeta(n) \leftrightarrow (\chi(n) \wedge \psi(n))$$ for $\chi(n) \in \Sigma^1_1$ and $\psi(n) \in \Pi^1_1$ has a representation as a tiling problem on infinite prototile sets by our constructions above, thereby having both periodic and aperiodic total tilings being given.
\end{proof}

In fact, we can choose which of aperiodic or periodic tilings we would like for our infinite prototile sets.

As an aside, the author did attempt to find other problems that share this same or similar syntactical form or structure. The closest that we could find was a definition and corollary in Bagaria \etal \cite[def. on p.6, Cor. 6.8]{Bagaria2015} wherein they show that Vop\v{e}nka's Principle for $\Sigma_{n+2}$ classes is equivalent for $(\Sigma_{n+1} \wedge \Pi_{n+1})$ classes, which naively seems to be a weaker form. However, these only work for $n \geq 1$, so are not an exact match, and indeed were superseded by the work by Bagaria \etal in \cite[Cor 4.13]{Bagaria2012}, where the result was weakened further to $\Pi_{n+1}$.\footnote{We would like to thank Dr. Andrew Brooke-Taylor for these references.} Aside from these references, it does indeed seem to be the case that very little in logic has $(\Sigma^1_1 \wedge \Pi^1_1)$ as the natural syntactic shape.

\section{Aperiodicity and Periodicity for Finite Prototile Sets}

\begin{definition}\label{def:PTileFIN}
Let the set of periodic finite prototile sets be
\begin{align*}
PTile_{FIN} = \{ e : & \, e \text{ tiles the plane from a finite set of prototiles}  \\
		     & \,  \text{ all of whose tilings are periodic} \}
\end{align*}
\end{definition}

\begin{definition}\label{def:ATileFIN}
Let the set of aperiodic finite prototile sets be
\begin{align*}
ATile_{FIN} = \{ e : & \, e \text{ tiles the plane from a finite set of prototiles}  \\
                     & \,  \text{ all of whose tilings are aperiodic} \}
\end{align*}
\end{definition}

\begin{definition}
Let a \emi{megatile} $M$ be a finite patch of tiles such that $M$ can be considered to be a tile at scale.
\end{definition}

Note, we differentiate this from a \emph{macrotile} we used earlier, as we are not interested specifically in simulating the original prototile set in our megatiles. We wish to be able to treat blocks of tiles as individual units.

\begin{proposition}[C. 2019][Rectangularisation of Megatiles]
For any non-rectangular megatile $M$ made up of Wang tiles in a periodic tiling $T$, there is a rectangular megatile $M^*$ that tiles $T$ precisely the same as $M$.
\end{proposition}

\begin{proof}
Let $\mathbf{v}$ be the periodicity vector for $T$ such that $[ \mathbf{v} T = T]$ for every non-zero $\mathbf{v}$-shift. Clearly we can rewrite $\mathbf{v}$ in the normal Cartesian orthogonal left-right, up-down basis - let $\mathbf{xy} = \mathbf{v}$.

We first select a tile $t \in T$, our tiling, and begin with the rectangle formed by one application on $t$. This rectangle will have sides of length $|\mathbf{x}| \text{ and } |\mathbf{y}|$, and will capture the translation of this one tile $t$. For each $t_i \in M$, a megatile in our periodic tiling, we can get a sequence $r_1, r_2, \ldots$ of rectangles tracking the motion of each rectangle.

We take either a column (row) of each $r_i$'s such that they overlap at the boundary. We keep appending $r_i$'s under (to the right of) each other until we get the bottom row (right-most column) matches the top row (left-most column). Once we have this, which is guaranteed by the periodicity of our tilings, we can trim the duplicated column (row) and we obtain a single rectangle that has captured all of the translations of each $t_i \in M$ under $\mathbf{v}$.

\end{proof}

The resultant rectangle in the proof has at least two opposite edges that are some permutation of an integer multiple of the $t_i \in M$. Thus, our theorem is guaranteed by the finiteness of our prototile set.

We will now explore the logical complexity of whether finite prototile sets are periodic or aperiodic. Our first result in this endeavour is somewhat unexpected:

\begin{theorem}[C. 2019]\label{thm:ATileFIN-Pi01}
$$ ATile_{FIN} \in \Pi^0_1 $$
\end{theorem}

\begin{proof}
Let $S$ be a finite prototile set, and define the following set: $$ EPTile_{FIN} = \{ e : \text{ there exist periodic tilings given by } \varphi_e \}$$ Given it is equivalent to the halting state of a TM that finds the period of some $S$-tiling $T$, specifically $$ \psi(S) = \exists s ( s \text{ is the period of an $S$-tiling } T ) $$ it naturally follows that $$ EPTile_{FIN} \in \Sigma^0_1$$

Note that this computable search across all possible tilings can proceed iteratively along a sequence of $S$-tilings, which are enumerable given $S$ is finite, given by $$ T_0, T_1, T_2, \ldots $$ We only require that our search stops once for $S$ to be in $EPTile_{FIN}$.

We now note that $\neg \psi(S)$ is equivalent to saying that our periodicity finding machine will not halt for any $S$-tiling, noting that this does not require set comprehension. Thus, $$\neg \psi(S) \in \Pi^0_1$$ and given this is equivalent to saying every $S$-tiling is aperiodic, the theorem follows by: $$ ATile_{FIN} = \overline{EPTile_{FIN}} $$

\end{proof}

\begin{theorem}[C. 2019]\label{thm:PTileFINPi11}
$$ PTile_{FIN} \in \Pi^1_1 $$
\end{theorem}

\begin{proof}
 For a any prototile $S$ and any $S$-tiling $T_S$ we have $$ S \in PTile_{FIN} \iff (\forall T_S) (\exists \mathbf{v}) [T_S = \mathbf{v} T_S] $$ We also notice that for any finite prototile set $S$, the maximal shift is given by every tile of $S$ in a line, thus a periodicity vector $\mathbf{v}$ has a maximal length determined by $| S |$. Given that $\mathbf{v}$ is bounded by the size of $S$, we get that $$ PTile_{FIN} \in \Pi^1_1 $$
\end{proof}

However, given our previous result in theorem \ref{thm:ATileFIN-Pi01}, we may consider that there is some arithmetical representation of $PTile_{FIN}$. But after some searching, we pose the following conjecture:

\begin{conjecture}[C. 2019]
$PTile_{FIN}$ has no arithmetical representation.
\end{conjecture}

The intuition for this follows from the fact that we are required to quantify over every possible $S$-tiling for some prototile set $S$, and thereby guarantee that there is no such $S$-tiling where there is no periodicity vector. As such, this would appear to consistently give $PTile_{FIN} \in \Pi^1_1$ as given above. A concrete proof that there is no arithmetical representation of $PTile_{FIN}$ has not been found, so the possibility remains open.
