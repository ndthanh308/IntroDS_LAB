\chapter{Tilings - Concepts and Results}
\label{chap3}
% next resets the equation numbers to start at 1 at the start of the chapter
\setcounter{equation}{0}
\renewcommand{\theequation}{\thechapter.\arabic{equation}}

%------------------------------------------------------------------------------

\epigraph{It is the shape that matters.}{\textit{Samuel Beckett \\ to Harold Hobson}}

This chapter presents previous results to do with the mathematical study of tiling problems. We present more general results first, and then focus on tiling problems for Wang prototiles that will occupy the rest of our study in this thesis.

\section{Tilings of the Plane}

In this chapter, we will give an overview of the notation, history, and important results concerning tiling problems. Unless otherwise indicated, we will use \cite{GrunbaumTP} and \cite{FuchsT} as our primary resources for material in this chapter.

\subsection{Preliminaries of Tilings}

We will use the following definitions of tilings in this thesis. Note we restrict ourselves to tilings on the plane $\mathbb{R}^2$. 

\begin{definition}[Tiles]
A \emi{tile} is a closed polygon that covers some finite potion of the plane.
\end{definition}

Topologically, each tile is a closed subset of the plane, and is homeomorphic to a disc. As such, we can define \emi{tilings} as follows:

\begin{definition}[Tilings]
Tilings will generally take the following forms:
\begin{itemize}
\item Tiles form a \emi{complete tiling} if the union of these subsets is the full plane.
\item Tiles form a \emi{partial tiling} if there are points in the plane that are not contained in any subset.
\end{itemize}
\end{definition}

For complete tilings, each point $p \in \mathbb{R}^2$ will find itself in one of two situations. Either we have that:
\begin{enumerate}
	\item $p$ is to the interior of at most one tile, or
	\item $p$ is on the edge join of two tiles.
\end{enumerate}
As a consequence of this, tiles in a complete tiling have pairwise disjoint interiors, and there are no gaps between the tiles in the tiling. 

To make it easier to consider the relationship between a tiling and the tiles that constitute it, we can define sets of prototiles as follows:

\begin{definition}[Prototile Sets]
For a given tiling $\mathcal{T}$,
\begin{itemize}
\item A \emi{prototile} set $\mathcal{S} \subset \mathcal{T}$ is a set of tiles such that for every tile $t \in \mathcal{T}$ there is an $s \in \mathcal{S}$ that is congruent to $t$.
\item A prototile set $\mathcal{S}$ is called \emph{minimal} if for all $s_i,s_j \in \mathcal{S}$, \[ s_i \text{ is congruent to } s_j \iff s_i = s_j \]
\end{itemize}
\end{definition}

Later in this thesis we will consider only \emph{minimal tilings}, where we have substituted geometric requirements with a regular polygonal lattice with edge conditions. But for now, we will proceed with all the above definitions.

\subsection{The Extension Theorem}

The Extension Theorem is a compactness-like argument that is an important result from the literature, a version of which will become very useful later in this volume.

We will start with some definitions for related and useful concepts we will use in theorem \ref{thm:extension}. For this section we will assume that all prototile sets are finite, although we will relax this requirement for our further work in tiling problems later in this volume.

\begin{definition}
Given a tiling $\mathcal{T}$, $t_i, t_j \in \mathcal{T}$, the \emi{Hausdorff distance} $h(t_i, t_j)$ between two tiles is defined as \[ h(t_1,t_2)  = \max\left\{ \sup_{a \in t_1} \inf_{b\in t_2} \norm{a - b} ,\sup_{b\in t_2} \inf_{a\in t_1} \norm{a - b} \right\} \]
\end{definition}

From this definition it follows that where for some tiles $t_1, t_2 \in \mathcal{T}$, we have that $h(t_1, t_2) = 0 \implies t_1 = t_2$. 

\begin{definition}[Patch Tiling]
A \emi{patch} is the union of a number of tiles covering some non-total portion of the plane $R \subset \mathbb{R}^2$.
\end{definition}

The usual intuition for patch tilings is that they are finite portions of the plane, however we will also use this wording to denote infinite connected regions of the plane that are not total. Where the context requires we will talk of `infinite patches' and `finite patches', but generally speaking, we use this looser definition of `patch tiling' than is generally used in the literature.

\begin{definition}
We say that a set of prototiles $\mathcal{S}$ \emi{tiles over} a finite subset $X$ of the plane if there is a finite patch tiling $P_\mathcal{S}$ such that for all $x \in X$, $x \subset P_\mathcal{S}$, with each $t \in P_\mathcal{S}$ congruent to some $s \in \mathcal{S}$.
\end{definition}

Where we have finite patches as a bounded tiling, these are then also topologically equivalent to a disc.

\begin{definition}
A sequence of tiles $t_1, t_2, t_3, \ldots$ \emi{converges} to a limit tile $t$ if $\lim_{i\to\infty} h(t_i,t) = 0$.
\end{definition}

\begin{definition}[Circumparameter]
$U$ is a \emi{circumparameter} of a prototile set $\mathcal{S}$ if for every $t \in \mathcal{S}$, $t$ is contained in some disc of radius $U$.
\end{definition}

\begin{definition}[Inparameter]
Analogously we have that $u$ is an \emi{inparameter} of $\mathcal{S}$ if for each $t \in \mathcal{S}$, there exists a disc of radius $u$ that can be wholly inscribed within $t$.
\end{definition}

Now that we have covered the base definitions we require for this section, we will proceed to prove some general theorems in the theory of tilings. Our aim here is to state the geometric and topological arguments that are commonly used to analyse general properties of tilings derived from finite prototile sets. We begin with the following lemmas:

\begin{lemma}[Bolzano-Weierstrass Theorem]\label{lemma:ConvSeq}
Let $S$ be a closed bounded area in $\mathbb{R}^2$, and let $z_1,z_2, z_3, \ldots $ be a sequence of points in $S$. There is a subsequence of $z_{i_1}, z_{i_2},\ldots$ that converges to some point $z \in S$.
\end{lemma}

Note, such a limit $z$ need not be unique. 

\begin{proof}
Let $z_i$ for $i \in \omega$ be our sequence $z_1, z_2, z_3, \ldots$, and let $S_0$ be a bounded region in $\mathbb{R}^2$.
	
First we bisect $S_0$. By pigeonhole principle, we have that at least one of these pieces contains infinitely many $z_i$. Call this piece $S_1$, and repeat the subdivision infinitely. The same density must apply to at least one of any subdivided region, so we can choose a sequence of pieces $S_2$, $S-3$, \ldots containing infinitely-many $z_i$ in each subsequent piece.

From our eventual infinite sequence $S_0, S_1, S_2, \ldots$ we can choose any sequence of points, with each successive $z_i$ coming from $S_i$. These points converge closer to some limit point $z$.
\end{proof}


\begin{theorem}[Selection Theorem, \protect{\cite[p.154]{GrunbaumTP}}]\label{thm:Selection}
Let $t_1, t_2, \ldots$ be an infinite sequence of tiles such that all $t_i$ are congruent - by translation and rotation - to (bounded) $t$, that is fixed. If every $t_i$ contains point $p$, then the sequence contains a convergent subsequence whose limit tile $t'$ is congruent to $t$, with $p \in t$.
\end{theorem}

\begin{proof}
Choose $t_n \cong t_0$ for each $n \in \omega$. As such, each point $p \in t_n$ identifies some point $q_n \in T$. By \ref{lemma:ConvSeq}, there is a convergent subsequence $q_{i_1}, q_{i_2}, \ldots \rightarrow q$ inside $t$. 

Intuitively, this limit point $q$ is taken from a point $p$ that is `common' to all tiles where they translated, but \emph{not} rotated, and placed over each other. When separated out, this is our sequence of $q_i$'s, where each tile is labelled spiralling out from our $t_0$ - much like the `snake' proof in classical set theory.

If we position this $t$ such that $q$ is over the coordinate $(0,0) \in \mathbb{R}^2$, then we can notice that all of our translations are rotated about $q$. So the position of each tile is the translation $q-q_i$ followed by some rotation $\alpha_{i_n}$ (mod $2\pi$).
	
As such, by using the same reasoning in lemma \ref{lemma:ConvSeq}, we can we can gather a subsequence of rotation angles $\alpha_{i_1}, \alpha_{i_2} , \ldots$ which converges (modulo $2\pi$). Let $\alpha$ be the limit of this sequence, and so $t_{i_1}, t_{i_2}, \ldots \rightarrow t'$, which is a copy of $t$ rotated by $\alpha$ and with $q$ coincident with $p$.
\end{proof}

Note, this theorem will fail if such a $p$ does not exist, for example. That said, we will use the following special case later:

\begin{corollary}[\protect{\cite[p.154]{GrunbaumTP}}]
Let $t_0, t_1, t_2, \ldots$ converge to some $t$, if $d(t_i, t) \rightarrow 0$ as $i \rightarrow \infty$. Then $t$ is congruent to $t_0$. 
\end{corollary}

We can now prove the Extension Theorem, which is a fundamental, general result about tilings.

\begin{theorem}[Tiling Extension Theorem, \protect{\cite[Thm. 3.8.1]{GrunbaumTP}}]\label{thm:extension}\index{extension theorem}
Let $\mathcal{S}$ be a finite set of prototiles - each of which is a closed topological disc. If $\mathcal{S}$ tiles over arbitrarily large discs, then there exist $\mathcal{S}$-tilings of the plane. 
\end{theorem}

The proof will follow the one found in \cite[p.151]{GrunbaumTP}.

\begin{proof}
Let $\mathcal{S}$ be a finite set of prototiles, and let $U$ be the common circumparameter, and $u$ be the common inparameter. Consider the lattice $\Lambda$ of all points who regular Cartesian coordinates are $(nu,mu)$ for $m,n \in \mathbb{Z}$. $\Lambda$ therefore has some point in each prototile in $\mathcal{S}$. Let $L_0, L_1, L_2, \ldots$ be the full sequence of these points, spiralling out from some chosen $L_0$, say $(0,0)$. 

For any positive $r \in \mathbb{N}$, let $D(L_0, r)$ be the disc of radius $r$ centred on the point $L_0$. Let $P(r)$ be the finite patch of tiles from $\mathcal{S}$ that covers $D(L_0,r)$. When $r$ is large enough for $D(L_0,r)$ to contain some $L_s$, let $t_{rs}$ denote the tile of $P(r)$ that covers $L_s$. If, however, $L_s$ lies on an edge or a vertex point, then we can choose any tile in $P(r)$ that is incident to $L_s$. 

By the Selection Theorem (\ref{thm:Selection}) we have that, given $\mathcal{S}$ is finite, the sequence $t_0, t_1, t_2, \ldots$ has a subsequence $t'_{0}, t'_{1}, \ldots$ of tiles that are congruent to $t'_0$. This sequence will also contain an infinite subsequence $S_0 = t'_{i_0}, t'_{i_1}, \ldots$ that is convergent, and whose limit tile $t'_0$ will also contain $L_0$.

We now consider the sequence of tiles $t_{r1}$ containing $L_1$, restricting attention to values of $r$ that correspond to tiles in $S_0$. We can carry out the same line of argument as we just did to acquire $S_1$ of tiles all congruent to $t_1$, containing $L_1$, and convergent to a limit tile $t'_1$. 

Let $\mathcal{T} = \{ t'_0, t'_1, t'_2, \ldots \} $, deleting any duplicates as necessary in our selection. Ultimately, we want to show that $\mathcal{T}$ forms an $\mathcal{S}$-tiling of the plane. To show this, let $p$ be any point of the plane. We want to show that $p$ belongs to at least one $t'_i$, but does not belong to the interior of any other $t'_j$. 

Let $D(p,u)$ be the disc centred at $p$, of circumparameter radius $u$. Let $L_m$ be the point of $\Lambda$ in $D(p,u)$ with greatest index. We want to restrict our attention to the sequence of finite patches $P(r)$ as $r$ ranges through value corresponding to the subsequences $S_m$, specifically $\mathcal{T}_r = \{ t_{r0}, t_{r1}, t_{r2}, \ldots t_{rm} \}$.

As $r \rightarrow \infty$, $\mathcal{T}_r$ converges to the set $\mathcal{T}' = \{ t'_0, t'_1, \ldots \}$. Since all of the tiles in $\mathcal{T}_r$ have disjoint interiors, and all contain $p$, the same is true of each member of $\mathcal{T}'$. Thus, $\mathcal{T}$ is an $\mathcal{S}$-tiling of the plane.
\end{proof}

\section{The Domino Problem}

Whilst theorem \ref{thm:extension} gives us a notion of compactness that we can express through tiles, we then come to a more general question about tilings, known as the `Domino Problem'.

\begin{definition}[the Domino Problem]\label{def:domprob}\index{domino problem}
For any given set of prototiles $S$, does there exist an $S$-tiling of the plane?
\end{definition}

By theorem \ref{thm:extension} we know that if we can extend any finite patch $S$-tiling, we can get a tiling of the plane, but the Domino Problem asks us to consider whether there is any finite patch that cannot be tiled. 

When considering the Domino Problem for various sets of tiles, it is possible to modulate various requirements on how we cover the plane. For example, we might not consider trivial sub-tilings of some $S' \subset S$, or we might permit `small' holes that are strictly smaller than any polygon $t \in S$, such that we can consider them `small enough' in the limit. 

Given this definition, it is common to consider these conditions on a Domino Problem for some prototile set, unless explicitly indicated otherwise. Given a set of prototiles $S$:

\begin{itemize}
\item We will \emph{not} require that $\forall t \in S$, $t$ is used at least once in each $S$-tiling of the plane.
\begin{itemize}
\item This requirement is sometimes used to prevent trivial sub-tilings of the plane of some $S' \subset S$, mentioned above. However, when we come to dealing with encoding a Turing Machine into prototile a set, we need to allow that our TM will not enter every state on every input.
\end{itemize}
\item We will require that, for lattice regular polygonal tilings such as Wang tiles (defined in \ref{def:Wang}) we do not admit rotations of the tiles.
\begin{itemize}
\item Although this increases our prototile sets significantly, it make our later more functional definitions much more straightforward.
\end{itemize}
\item That our tilings are complete tilings.
\end{itemize}
Though we will make use predominantly of lattice-based tilings in this thesis, we wish to prevent gap from occurring in our tilings. As such, our resultant tilings can be thought of as total functions over the plane via coverings given by mapping each point in the lattice to a copy of a prototile.

These requirements can serve as to simplify our tilings, definitions, and constructions of prototiles. As noted, although the number of prototiles will increase, the complexity of our tiling functions will significantly reduce.

However, in this thesis, we will attempt to make our tilings as `free' as possible. Although this gives us slightly larger prototile sets, it serves to give us some better insight into the equivalence between the logical complexity of some statement, the computable trees arising from these statements, and the computable prototile sets that code paths of these computable trees into planar tilings.

\subsection{Wang Tiles}

To properly analyse the Domino Problem, we wish to reduce the complexity of our tilings to some `bare minimum', in line with the requirements above. As such we will make use of Wang tiles, first introduced by Hao Wang in \cite{Wang1990}, which we define as follows:

\begin{definition}[Wang Tiles]\label{def:Wang}\index{Wang tiles}
Let \emi{Wang tiles} be square tiles, diagonally quadrisected, such that ordered 4-tuples of the form $\langle l,u,r,b \rangle$ can be represented by:

\begin{center}
\sampletile{$l$}{$u$}{$r$}{$b$}
\end{center}
Where $l,u,r,b$ each stand for left, upper, right, and bottom respectively.
\end{definition}

We keep our previous definitions of `prototiles', `prototile sets', and `tilings'. Given this prototile definition, we will need to consider what happens when the edges of our tiles are to meet. Given a set $\mathcal{S}$ of Wang prototiles:

\begin{definition}
Given two Wang tiles $w, u \in \mathcal{S}$, such that $w = \langle l_w, u_w, r_w, b_w \rangle$ and $u = \langle l_u, u_u, r_u, b_u \rangle$:
\begin{itemize}
\item The \emi{edge meets} between these tiles are the comparisons between meeting edges, where one of the following applies:
\begin{itemize}
\item $l_w$ is next to $r_u$,
\item $u_w$ is next to $b_u$,
\item $r_w$ is next to $l_u$,
\item $b_w$ is next to $u_u$
\end{itemize}
\item The \emi{match criteria} for Wang tiles are the requirements that for any edge meet, the edge symbols match. Explicitly, one of the following holds: 
\begin{itemize}
\item if $l_w$ is next to $r_u$, then $l_w = r_u$
\item if $u_w$ is next to $b_u$, then $u_w = b_u$
\item if $r_w$ is next to $l_u$, then $r_w = l_u$
\item if $b_w$ is next to $u_u$, then $b_w = u_u$
\end{itemize}
\end{itemize}
\end{definition}

% Figure environment removed

Intuitively we use the von Neumann neighbourhood surrounding the tile as the basis for our matching and placement conditions for each Wang prototile. This means that we only ever consider the 4-place valency for each tile and for each position in our $\mathbb{Z}^2$ lattice following the rules we stated above. When we come to code cellular automata, we will still only consider the von Neumann neighbourhood over the usual Moore neighbourhood.

From this construction of Wang tiles, we can now envisage our tilings as projection functions $$f_{\mathcal{S}} : \mathbb{Z}^2 \rightarrow \mathcal{S}$$ This characterisation will be useful when we explore computable tilings later in this thesis. Thus, the following definition is natural:

\begin{definition}\index{Total Wang tilings}
Given a set of Wang prototiles $\mathcal{S}$, we say that an $\mathcal{S}$-tiling of the plane is a \emi{total tiling} if for $f: \mathbb{Z}^2 \rightarrow \mathcal{S}$, and $f$ enforces the edge-meet criteria for the von Neumann neighbourhood of every point in $\mathbb{Z}^2$.
\end{definition}

Given for every point $(x,y) \in \mathbb{Z}^2$ there is some $s \in \mathcal{S}$ such that $f(x,y) = s$ and $f$ ensures that $s$ observes and meets all of the match criteria for its neighbours in the plane.

Our notion of a `total Wang tile tiling' is indeed a direct analogue for complete tilings we defined earlier. The slight change in terminology is to facilitate the intuition we will use later in this thesis that a complete tiling generated by a computable function must be total on $\mathbb{Z}^2$, and so is in this sense a total function. Thereby, total functions give total tilings, and total tilings must come from total functions.

Thus, a total tiling from a Wang prototile set is analogous to a complete tiling we considered previously. When we consider computable sets of Wang prototiles, this definition will be equivalent to a computable function $\varphi_e$ being total. 

Wang proved a version of the Extension Theorem for Wang tiles - known as \emph{Wang's theorem}. Our statement and proof are taken from \cite[p.600]{GrunbaumTP}.

\begin{theorem}[\protect{\cite[p.600]{GrunbaumTP}}]\label{thm:WangExtension}
Let $\mathcal{S}$ be a finite set of Wang prototiles. If it is possible, of arbitrarily large values of $n$, to assemble $n \times n$ blocks of tiles satisfying the edge-matching conditions, then there is an $\mathcal{S}$-tiling of the plane.
\end{theorem}

We should reiterate that we only admit \emph{translations} of Wang prototiles - we do not permit rotations of Wang prototiles into tilings of the plane. If we did, this theorem would be immediate from the Extension theorem, theorem \ref{thm:extension}. Additionally the proof will make explicit use of the face that $\mathcal{S}$ is a finite set of prototiles.

\begin{proof}
Given a set of prototiles $\mathcal{S}$, with $| \mathcal{S} | = r$. We can construct a graph-theoretic tree in the following manner. We start with a single root node $n^0$. Level 1 is formed of $n^1_1, \ldots, n^1_r$ corresponding to each of the tiles in $\mathcal{S}$. Similarly at each level $k$, we add nodes $n^k_1, \ldots,n^k_{r_k}$ corresponding to adding a ring of tiles around each of the previous blocks.

We then form the tree by joining all of level 1 nodes to the root node. For any level $k$, we connect any of the $n^k$ to the nodes in $n^{k+1}$; if, for any $n^k_i$, $n^{k+1}_j$ contains the block represented by $n^k_i$, and the tiles on the outer edge of block $n^{k+1}_j$ match all the edge-matching criteria for the exterior of the tiles represented in $n^k_i$ are met by the inner edge criteria $n^{k+1}_j$. If this holds, then $n^k_i \text{ and } n^{k+1}_j$ are connected.

Each successive block can be thought of as an extension of the previous block by an outer `square ring' of tiles from $\mathcal{S}$ that surround the outside of the block.

Thus, we can reduce the question of an $\mathcal{S}$-tiling now to whether each level $k$ is connected to each $k+1$. If the answer is in the negative, then  there exists some $n$ such that there can be no patch of Wang tiles greater than $n \times n$ that can be extended to a full planar tiling.

If the answer is in the positive, then we have created a finitely branching infinite tree. By K\"onig's Lemma, there is necessarily an infinite path through our tree. By this construction, this path corresponds to an $\mathcal{S}$-tiling of the plane. 
\end{proof}

It is worth noting that although K\"{o}nig's Lemma is utilised in this proof, this is not necessary. Given our sets of prototiles are always finite, we only actually require Weak K\"{o}nig's Lemma - that an infinite bounded-branching tree necessarily has an infinite path - for this proof with some modification of our tiling tree as follows.

\begin{proof}[Alternative Tree Construction for proof of \ref{thm:WangExtension}]
Take some finite prototile set $S$, and consider each each point on $\mathbb{Z}^2$ by spiralling out from the centre point $(0,0)$ as before for the proof of theorem \ref{thm:extension}. We can construct a tree based on the valid tiles that could be placed at each successive point based on the 1 or 2 edge criteria defined by previously placed tiles. 

This tree is bounded by the size $S$, which is finite, thereby restricting the branching of our tree. A total planar tiling also corresponds to a path through this tree by the following observations:
\begin{itemize}
\item Each level on our tree corresponds to a point in $\mathbb{Z}^2$.
\item All edge-meet criteria are met by the construction of each branch.
\end{itemize}
Thus if our tree is infinite, there must be an infinite path by WKL, meaning there is a total planar tiling.
\end{proof}

Later in this thesis, we will entertain weaker notions of tiling the plane, and will draw more equivalences with properties and principles on trees in both Baire space and Cantor space.

\section{Undecidability of the Domino Problem}

One of Hao Wang's students, Robert Berger, proved in \cite{berger1966} the undecidability of the Domino Problem for finite sets of Wang prototiles. Whilst Berger's original created a prototile set of over 6,000 tiles, we present an updated proof where sets of `universal Turing Machine prototiles' number in the few hundred.

\begin{definition}
For a set $\mathcal{S}$ of prototiles, we denote ``There exists a complete $\mathcal{S}$-tiling of the plane" by $Tile(\mathcal{S})$.
\end{definition}

Note that $Tile(S)$ immediately has a $\Sigma^0_1$ normal form as the existence of an infinite sequence $s \in S^{\omega}$, such that $s$ is a sequence of tiles that covers each point in the lower-right quarter plane in $\mathbb{Z}^2$, thereby giving a total tiling of this quarter plane.

However, given we can extend any $S$ with tiles that fill in the other three quarter-planes, we can convert this $s$ to a total planar tiling.

\begin{theorem}[\protect{\cite[Thm. 3-3]{berger1966}}]\label{thm:TMTilings}
The Domino Problem for finite Wang prototile sets is $\Sigma^0_1$-complete.
\end{theorem}

We will prove this by showing that for any Turing Machine $\varphi_e$ there exists a set of prototiles $\mathcal{S}_e$ such that $$ \varphi_e(x) \downarrow \iff \neg Tile(\mathcal{S}_e) $$ In order to do this, we will need the following machinery:

\begin{definition}\label{def:schematile}
A \emi{schema tile} is a prototile that determines a set of prototiles for given sets of colours. That is, it determines the position of colours taken from one or more sets of colours.
\end{definition}

\begin{example}[Schema Tile Example]
Let $A = \{ a_1, a_2 \}$ and $B = \{ b_1 \}$ be sets of colours. Let $t$ be the schema tile, with $i \neq j$:
\begin{center}
\sampletile{$a_i$}{$b_i$}{$a_j$}{$b_i$}
\end{center}	
The prototile set $\mathcal{S}$ generated by $t$ will consist of the following tiles:
\begin{center}
\sampletile{$a_1$}{$b_1$}{$a_2$}{$b_1$} \sampletile{$a_2$}{$b_1$}{$a_1$}{$b_1$}
\end{center}
It is worth observing that this resultant prototile set can give total planar tilings.
\end{example}

Thus, we can talk about the following progression: $$ \text{schema tile} + \text{colours} \Rightarrow \text{prototile sets} \Rightarrow \text{planar tilings} $$ By careful control of our schema tiles, we can establish the overall `shape' or `behaviour' of our prototile sets, which in turn controls some desirable feature or features of our classes of planar tilings.  

The following proof is after \cite{Boas97Conv} and \cite{MCarneyMSc}, however it has been restructured in order to match the structure of proofs later in this thesis.

\begin{proof}[Proof of \ref{thm:TMTilings}]
We construct the following schema tiles with which we can emulate Turing Machines. Let $s \in \Sigma$ be colours representing symbols, $q_i \in Q$ be colours representing machine states, and $(s,q) \in \Sigma \times Q$ be colours corresponding to each symbol matched with each state. Let $B$ be a distinguished colour representing `blank', and $H$ be distinguished colour representing the halting state.

\noindent \textbf{Symbol tiles}
\begin{center}
\sampletile{$B$}{$s$}{$B$}{$s$}
\end{center}
\textbf{Head State tiles}
\begin{center}
\sampletile{$q$}{$s$}{$B$}{$(s,q)$} \sampletile{$B$}{$s$}{$q$}{$(s,q)$}
\end{center}
\textbf{Computational tiles}
For $s,s' \in \Sigma$ and $q, q' \in Q$, permitting $s = s'$ and $q=q'$,
\begin{center}
\sampletile{$B$}{$(s,q)$}{$q'$}{$s'$} \sampletile{$q'$}{$(s,q)$}{$B$}{$s'$}
\end{center}
\textbf{Halting tile}
\begin{center}
%\sampletile{$H_L$}{$s$}{$B$}{$s'$} \sampletile{$B$}{$s$}{$H_R$}{$s'$}
\sampletile{$B$}{$s$}{$H$}{$s'$}
\end{center}

Let $\varphi_e$ be some Turing Machine composed of 5-tuples, and let $\varphi_e(x)$ be the computation that we wish to represent in our planar tilings.

We first take every symbol in our Turing program, and represent each one by some $s \in \Sigma$. We then code each symbol in the tape by a symbol tile. The `blank' representing colour $B$ serves to line up our rows into representations of configurations $c_i$ for $i \in \omega$. We now colour all of the symbol tiles with each $s \in \Sigma$, and put these into $\mathcal{S}_e$.

Next, we need to assign each of the states in $e$ to a state $q \in Q$, and we are then ready to add the Head State and Computation prototiles to $\mathcal{S}_e$. To do this, we take each $s \in \Sigma$, and each $q \in Q$, and assign colours for each `TM state' $(s,q)$. The Head State tiles will accept a state from $q$ from left or right, and will merge this information into the bottom quadrant of the prototile. 

Next, we need to look to all of the 5-tuples $(s,q,s',q',\{L,R\}) \in e$. For each $(s,q)$ taken from $\Sigma \times Q$, we look to see which of these form the first two positions of a 5-tuple. We then create a prototile for $\mathcal{S}_e$ of the form of this tuple based off the schema, placing the exit state $q'$ on the left or right according to the last position of the 5-tuple.
	
\emph{E.g.} let $(1,a,1,a,L)$ and $(1,b,0,a,R)$ be valid 5-tuples from some given $\varphi_e$. We can represent them in $\mathcal{S}_e$ by means of the computation schema tiles as follows (respectively left and right): 
\begin{center}
\sampletile{$a$}{$(1,a)$}{$B$}{$1$}  \sampletile{$B$}{$(1,b)$}{$a$}{$0$}
\end{center}
Given this we colour all the necessary computation tiles - except for any 5-tuple that enters the halting state, which we will deal with below) - remove any unnecessary head state tiles, and add all these to the symbol tiles in $\mathcal{S}_e$.

In order to complete the representation of $\varphi_e$, we need to add the halting states. These are distinctive, 5-tuples, and for any given halting 5-tuple $(s, q, s', HALT, \{ L, R \})$, we represent these as:
\begin{center}
\sampletile{$B$}{$(s,q)$}{$H$}{$s'$}
\end{center}

In order to fully represent our computation $\varphi_e(x)$ we perform the following steps:
\begin{enumerate}
\item We first take the representation of $x$ in symbols from $\Sigma$ - let this be a string of symbols $s_0, s_1, \ldots, s_k$, where $k = |x|$.
\item We take $s_0$, the initial state of our TM $q_0$, and place the following tile in the first position at co-ordinate $(0,0)$:
\begin{center}
\sampletile{$q_0$}{$s_0$}{$B$}{$(s_0,q_0)$}
\end{center}
\item We then place the respective symbol tiles for $s_1, \ldots, s_k$ to the right of this tile on what will become the representation of the first configuration $c_0$ of $\varphi_e(x)$. We can also continue tiling this entire bi-infinite row according to the symbols on the rest of the TM tape.
\begin{itemize}
\item We will later use the index on each configuration $c_i$ to map to the lower quadrants of every even row of tiles $r_{2i}$ for checking later.
\end{itemize}
\item We now continue the computation by continuing the tiling - given the prototiles in use code each part of the computation, each row can be read off as a successive stage of the computation.
\item the Halting tiles are designed that they will block the tiling from tiling the plane to the right any further.
\end{enumerate}

We can check the following facts about our tiling computation:
\begin{itemize}
\item Given our TM is not a non-deterministic TM, there will be only one choice for each computation prototile on each row.
\item Each row $r_{2i}$ will correspond to some configuration $c_i$ in our computation, with the tape configuration being readable from the top quadrants of each tile on the row.
\item Given our first row setup, there will not be more than one TM head performing the computation.
\end{itemize}

Thus, our tiling problem $Tile(\mathcal{S}_e(x))$ is also represented by the problem $$\exists s \, \{r_{2s} \in \mathcal{S}_e(x) \text{ has a hole}\}$$ which is in turn equivalent to the statement $\exists s \, \varphi_{e,s}(x) \downarrow$. As such, the Domino Problem for finite Wang prototile sets is $\Sigma^0_1$-complete.
\end{proof}

% for this - no. There is now a hole because of $H$
%\mc{TODO - should this be $\exists s \, \{r_{2s} = r_{2s+1} \in \mathcal{S}_e(x) \}$ ??}

%\begin{definition}
%Let $\mathbb{T} = \{ \mathcal{S} : \text{there exist total planar $\mathcal{S}$-tilings} \} $
%\end{definition}
%
%We can prove the following theorem:
%
%\begin{theorem}
%\[ K \leq_m \mathbb{T} \]
%\end{theorem}
%
%TODO - should this be $HALT$??

%Or more colloquially, there is no $e$ such that $Tile = \varphi_e$. Our proof will act in the spirit of \cite{berger1966}, but will follow more closely previous work by the author in \cite{MCarneyMSc}, which is itself inspired by \cite{Boas97Conv}.
%
%\begin{proof}
%\mc{TODO}
%\end{proof}

\begin{corollary}[\protect{\cite[Cor. 4-1, p.36]{berger1966}}]
The Domino Problem is undecidable.
\end{corollary}

\begin{proof}
By \ref{thm:TMTilings}, it is clear that there exists a class of prototile sets corresponding to each TM enumerated by some $e$. By our construction, $$ \neg Tile(\mathcal{S}_e(x)) \iff \varphi_e(x) \downarrow $$
	
Thus, given the Halting Problem is undecidable, then the question of whether or not the corresponding $\mathcal{S}_e$-tilings tile the plane or not is also undecidable.
\end{proof}

The above re-proof of this classic result due to Berger is intended to illustrate our proof method in later chapters.

The original proof uses much more machinery, and a large set of prototiles for a Universal Turing Machine. This simplification makes plain the equivalence much more immediately, and lays a groundwork for our later results.

We will use this equivalence in the rest of this volume when we define \emph{computable prototile sets} and \emph{computable tilings} in the next chapter.

\subsection{Universal Turing Machine and TM Tilings}

Let Universal Turing Machines (UTMs) be minimal Turing Machine symbol and state sets, such that they can effectively emulate a Turing Machine of any size. 

\begin{definition}
Let a \emi{$(x,y)$-Universal Turing Machine} $\psi$, denoted $(x,y)$-UTM, be a Turing Machine that uses precisely $x$ active non-halting states, and $y$-many symbols on the tape, such that $\psi$ is Turing Complete.
\end{definition}

As such, we can think of them as being a pre-coded minimum requirement for any Turing Machine to operate. Let $\mathcal{S}_{UTM}$ denote a `library' of all possible states and symbols given by some UTM of a given number of states and symbols.
 
Due to the succinctness of our construction, it is reasonable to ask ``how big would a Turing prototile library be?". By colouring for all possible states, symbols, and state-symbol combinations we can get the following theorem:

\begin{theorem}[\protect{\cite[Chap. 3]{MCarneyMSc}}]
There exists a set, called the \emph{library}, of prototiles $\mathcal{S}$ with $|\mathcal{S}| = 625$, such that for every $\varphi_e$ there exists a set of prototiles $P_e \subset \mathcal{S}$ such that $P_e$ is a finite set of prototiles that represents $\varphi_e$ selected from $\mathcal{S}$.
\end{theorem}

The proof of this can be found in \cite{MCarneyMSc}, and involves colouring a full library of Turing tiles with the states and symbols of a $(2,5)-UTM$, known universal Universal Turing Machine.

Indeed, if we take Smith's as-yet unpublished proof that a $(2,3)$-TM is universal, \cite{Smith2007}, then we can get the following theorem:

\begin{theorem}[C. 2019]
There is a library set of Turing Machine encoding prototiles of size 105.
\end{theorem}

The proof comes from generating colours from a set of states $|\Sigma| = 2$ and a set of symbols $|Q| = 3$, obtaining $|\Sigma \times Q| = 6$, and then applying these colours to our Turing Tile schemas, and then counting all possible compositions.

%\section{Computable Tilings}
%
%We will use the following definitions in this thesis, as we work towards the $\Sigma^1_1$-completeness theorems in later chapters.
%
%\begin{definition}\index{computable prototile sets}\index{computable tile sets}
%Let $X \subset \omega$, and $\mathcal{S}$ be a set of Wang prototiles.
%\begin{itemize}
%\item Let $X_\mathcal{S} = \{ \langle c_l, c_u, c_r, c_b \rangle : \langle  c_l, c_u, c_r, c_b \rangle \text{ codes some prototile in } \mathcal{S} \}$.
%\item We say that $\mathcal{S}$ is \emph{computable} if $X_\mathcal{S}$ is computable.
%\item We say that an $\mathcal{S}$-tiling of the plane is computable if $f_\mathcal{S} : \mathbb{Z}^2 \rightarrow \mathcal{S}$ is computable.
%\end{itemize}
%\end{definition}

\section{Implications of TM Tilings}

There are some interesting implications that arise out of the fact that every Turing Machine has a representation in tiles. We state the following processes and theorems from \cite{Cichon1983}, assuming that the definitions of Primitive Recursive Arithmetic (PRA) and Peano Arithmetic (PA) are already known:

\begin{definition}[\protect{\cite[Process 1]{Cichon1983}}]\label{def:Process1}
\begin{enumerate}
\item Given some $n \in \omega$, write this number as the sum of powers of $x$ (base-$x$ notation).
\item Increase the base of the representation by 1.
\item Subtract one from this new representation. 
\item Return to 2 and repeat this procedure.
\end{enumerate}
\end{definition}

\begin{definition}[\protect{\cite[Process 2]{Cichon1983}}]\label{def:Process2}
Same as \ref{def:Process1}, except that on step 1 we write $n$ as \emph{pure base} representation, that is we write $n$ in base $x$, and then continue this process for all the exponents.
\end{definition}

The difference between these two definitions is that process 1 (definition \ref{def:Process1}) will admit for $n = 244$ a representation of $3^5 +1$, whilst \ref{def:Process2} will go further to $3^{3+2} +1$. After one iteration of \ref{def:Process1} we get $(3^5 + 1) : \rightarrow 4^5$, whereas \ref{def:Process2} will give us $(3^{3+2} + 1) : \rightarrow 4^{4+2}$.

The algorithm in \ref{def:Process2} is due to Goodstein in 1944 in \cite{Goodstein1944}. \cite{Cichon1983} gives short, elegant proof of the following famous results originally due to Kirby and Paris \cite{ParisKirby82}:

\begin{theorem}[\protect{\cite[Thorem 1]{Cichon1983}}]\label{thm:Cichon1}
For any $n \in \omega$ and base $x$, \ref{def:Process1} terminates, but this fact is not provable in PRA.
\end{theorem}

\begin{theorem}[\protect{\cite[Thorem 2]{Cichon1983}}]\label{thm:Cichon2}
For any $n \in \omega$ and base $x$, \ref{def:Process2} terminates, but this fact is not provable in PA.
\end{theorem}

Denote by $ProvRec(PA)$ the Provably Recursive functions of PA. Cichon's \cite{Cichon1983} proof of \ref{thm:Cichon2} relies on demonstrating that some machine $\varphi_{Good}$ that computes \ref{def:Process2} is such that $$\varphi_{Good} \notin ProvRec(PA)$$ Given this fact, it is necessarily true that $$ PA \nvdash \forall n,x \, \exists s \, \varphi_{Good,s}(n,x) \downarrow = 0$$

Let $S_{Good}$ denote the Turing Machine tiling generated by the process outlined in the proof of theorem \ref{thm:TMTilings}. We get the following corollary:

\begin{corollary}[C. 2019]\label{cor:PAGoodNotTile}
It is necessarily the case that for all $n,x$ there exists an $s$ such that the row $r_{2s}$ has a hole, and so $\forall n,x \, [\neg Tile(S_{Good}(n,x))]$, however by \cite{Cichon1983} it is necessarily true that $$PA \nvdash \forall n,x \, [\neg Tile(S_{Good}(n,x))]$$
\end{corollary}

It is perhaps unexpected \emph{prima facie} that the Domino Problem would have the means to defy provability of mathematically strong theories such as PA. However, the long established relationships between tilings and computability cement that there exists sets of Wang prototiles that have interesting proof theoretic outcomes.

