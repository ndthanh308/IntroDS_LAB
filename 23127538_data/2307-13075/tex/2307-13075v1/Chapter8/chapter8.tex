\chapter{Conclusion}
\label{chap7}
% next resets the equation numbers to start at 1 at the start of the chapter
\setcounter{equation}{0}
\renewcommand{\theequation}{\thechapter.\arabic{equation}}

%------------------------------------------------------------------------------

\epigraph{Nevertheless, I repeat; we are only at the beginning. I am only a beginner. I was successful in digging up buried monuments from the substrata of the mind. But where I have discovered a few temples, others may discover a continent.}{\textit{S. Freud, \\ in an interview with G. S. Viereck.}}

Here we give an overview of the conclusions from the work presented in this thesis, and give summary of some of the open questions arising from this research.

\section{Conclusions from Results}

In Chapter 3 we presented our first results concerning the relationship between computability and tiling problems. We extended results due to Harel in \cite{Harel1986} to the general Domino Problem for infinite prototile sets. These results follow the general intuition due to Berger in \cite{berger1966} that the Domino Problem for finite prototile sets is $\Sigma^0_1$/$\Pi^0_1$ complete, so expecting that $TILE/\neg TILE$ is equivalent to $\Sigma^1_1$/$\Pi^1_1$ does fit the general intuition regarding this class of tiling problems.

We next discussed, in chapter 4 the question of whether tilings from a given prototile set are periodic or aperiodic. From this outset we found a rather unusual set for which the problems of (a)periodicity for infinite prototile sets are complete - $(\Pi^1_1 \wedge \Sigma^1_1)$ - which is a rare class of problems. Indeed, it is entirely possible that this may be weakened in subsequent work to one side of this conjunction.

The fact that $ATile_{FIN} \in \Pi^0_1$ is surprising, given we did not even have a proven existence of such prototile sets until the mid-60's. However, we state the conjecture (below) that $PTile_{FIN}$ is unlikely to be arithmetical owing to the requirement to quantify over all possible tilings for a given prototile set, despite its bound on lengths of their possible periodicity vectors.

The Weihrauch reductions presented in Chapter 5 are the first that we know of concerning tiling problems. They directly use material from previous chapters in order to show that the Domino Problems we have defined and studied are all bounded above by the closed choice principle for Baire Space, with some equivalences also being found. These give further detail to our picture of the computability aspects of Domino Problems, fleshing out the overall picture beyond the conventional view.

Finally, our results in Chapter 6 paint a picture regarding how to code tilings of 3-ary functions, using ECAs as our example. This is, sometimes, a more natural formulation of a problem, and as such the presentation of this hexagon-lozenge tiling may be useful outside of this particular class of automaton coding into prototile sets.

\section{Open Problems and Further Work}

There remain some interesting open problems that arise both from the literature surrounding this thesis, and from results in the thesis itself. 

From \cite{Rao2015} we have the following conjecture:

\begin{conjecture}
All the aperiodic Wang prototile sets generated by Kari's method are \emph{minimal aperiodic}.
\end{conjecture}

This result holds for all given prototile sets derived and demonstrated in the literature, but we did not make any progress regarding the resolution of this problem. It does, however, make a lot of sense, and would be a good result to complete the picture painted by Rao \etal. 

Recall $PTile_{FIN}$ is the set of finite prototile sets for whom all tilings are periodic, we stated the following conjecture: 

\begin{conjecture}
$PTile_{FIN}$ is not arithmetical.
\end{conjecture}

This is motivated by the need to at some point quantify over the entire class of tilings for some finite prototile set $S$ in order to assert that $S \in PTile_{FIN}$, and this need seems unavoidable. However this is not something we have yet been able to show in general. The possible vectors are bounded, which may belie some clever trick for making $PTile_{FIN}$ arithmetical, but this is thus far elusive.

Lastly, recall that $ATile_{FIN} \in \Pi^0_1$, it would seem natural to derive some notion of measure on a prototile set's tilings, in order to derive the following conjecture - an analogue of Kucera's key result (see \cite{Downey2010} for an exposition):

\begin{conjecture}[C. 2019]
For a notion of positive measure on $\mathcal{S}$-tilings, for some prototile set $\mathcal{S}$, if a tiling $T$ has positive measure:
\begin{itemize}
\item $T$ is aperiodic.
\item $T$ encodes some Martin-L\"of Random.
\end{itemize}
\end{conjecture}

However, the work to identify a suitable notion of measure was not yet undertaken. We suspect that this can be achieved by means of analysis on the `colour density' for coloured edges/Want tile quadrants.

It is also worth noting that the following conjecture is unresolved:
\begin{conjecture}
The tiling method due to Socolar in \cite{Socolar1988} does indeed lead to total planar aperiodic tilings.
\end{conjecture}
It is our strong opinion that this is true by means of an application of WKL to some additional machinery added to the construction that is presented. However, the details have not yet been worked out to see if this can be achieved.

Finally, we have our conjecture from chapter 6:

\begin{conjecture}
	There exist ECA prototile sets of 8 tiles.
\end{conjecture}

As noted there, this cannot be formed of Wang tiles, but there is likely some way of cutting a planar representation of a given ECA into a regular single prototile per part of each rule. Shapes for this result would probably resemble interlocking tilings that look like a double conjoined `H', as detailed in \cite{GrunbaumTP}.

Finally, we note that the work in Chapter 5 on Weihrauch reducibility for tiling problems as principles has the capability to be taken much further. We alluded to one, for which we gave a definition of $WIPT$, accompanied by the following estimate of $C_{\omega^\omega} \leq_W C_{2^\omega} \star C_\omega \star WIPT$.

Indeed, we consider that there are many further applications for tiling problems, in particular for dimensionality $\geq 2$ and for non-Euclidian planar tilings.

A good starting point for the latter is the result due to Beauquier, Muller, and Schupp in \cite{Beauquier1999}. Here, they showed that a tiling problem known as ``the Bar Problem'' - the question of whether a plane that has holes in it can be covered with $(1 \times n)$ `bars' - is $NP$-complete in the Euclidian plane, however in the hyperbolic plane it becomes polynomial time.

Overall, we hope that we have demonstrated some interesting results regarding tiling problems, and laid down some framework and exposition that encourages future results.
