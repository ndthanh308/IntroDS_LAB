\chapter*{Introduction}
\label{chap:intro}
% next resets the equation numbers to start at 1 at the start of the chapter
\setcounter{equation}{0}
\renewcommand{\theequation}{\thechapter.\arabic{equation}}

In this thesis we will explore the connections between tiling problems and logic, specifically in relation to, and through the lens of, computability theory. 

\section*{Background to the Thesis}

Broadly speaking, the tiling problems we study fall into two categories, for given prototile set $S$:

\begin{enumerate}
\item Domino Problems - the question of whether $S$ tiles the plane.
\item Tiling Properties - do all/any $S$-tilings have some specific property, \eg are they all periodic or aperiodic?
\end{enumerate}

We will construct well defined versions of both of these problems, and study their relationships to various areas of computability theory. 

This thesis builds on results that the author first presented in their MSc dissertation \cite{MCarneyMSc} as part of their MSc Mathematics at the University of Leeds. In that work, we presented some ways to code various results in computability, as well as elementary cellular automata, into sets of Wang prototiles.

In building on these results, we explore with much more depth the ways in which the classes of tiling problems listed above relate to various aspects of computability. We ask questions along the following lines:
\begin{itemize}
\item What are the computable parts of a given tiling problem?
\item How do tiling problems fit into existing computability hierarchies?
\end{itemize}
We also present improved versions of the Elementary Cellular Automata tilings using an original tile schema that we have constructed for this purpose.

\subsection*{Motivations}

There are some very interesting results in the literature regarding tiling problems and logic, and in general the aim is to determine both what conditions can be met by some given prototile set, and conversely whether there exist prototile sets that exhibit particular properties that are of interest. 

We will look at both finite and infinite sets of prototiles and determine results for both of these classes of possible tiling problems. Specifically, we are interested in formulating answers to the question:

\begin{center}
``What is the relative difficulty for a given problem about tile sets and tilings?''
\end{center}

This question, as the literature belies, is far from a foregone conclusion. The construction of a prototile set is intrinsically linked to the various patterns and behaviours of that set's tilings in the plane.

Given the well-studied logical strength of other combinatorial principles, we hope to expand the logical and mathematical vocabulary in this respect for tiling problems.

\subsection*{Computability and Tiling Problems}

In 1964 (see \cite{GrunbaumTP}) Wang proved that if a prototile set of Wang tiles - diagonally quadrisected square tiles - can tile any arbitrarily large finite portion of the plane, then it can tile the whole plane. This is a fairly straightforward compactness argument, and does indeed use K\"onig's lemma (cited as `K\"onig's Infinity Lemma' in \cite{GrunbaumTP}) to achieve the result, which we present in Chapter 2, Theorem \ref{thm:WangExtension}. 

Following on from this work, Wang continued to ask interesting questions regarding tiling problems. Indeed, many of the interesting results regarding tilings spawns from a conjecture due to Hao Wang in the early 60's:

\begin{conjecture}
It is necessary, as well as sufficient, that if a set of prototiles $S$ is periodic, it tiles the plane.
\end{conjecture}

Seeking an answer to this question, Berger in \cite{berger1966} formulated the first set of aperiodic Wang tiles - a prototile set consisting of 20,426 tiles that has only aperiodic tilings of the plane. This completely disproved Wang's conjecture, and demonstrated that periodicity is sufficient, but not necessary for a prototile set to tile the plane - thereby negating the conjecture.

Berger's refutation of Wang's conjecture was surprising, and laid the groundwork for further results in creating aperiodic prototile sets for a decade - the most well known of which are probably Penrose tilings. A summary of this work is given at the start of Chapter 4.  

In addition to creating the first aperiodic prototile sets, Berger was also the first to formulate the connection between Wang tilings and Turing Machines. The ultimate result was that the domino problem for finite sets of Wang prototiles, namely
\begin{center}
``Does a finite set of Wang prototiles $S$ tile the plane?''	
\end{center}
and the halting problem
\begin{center}
``Does a given Turing Machine $M$ halt on given input $x$?''
\end{center}
are equivalent, and these formed the central results of his thesis.

This equivalence was highly motivational for the current work we have regarding prototile sets and mathematical logic, as we can include the Domino Problem class of tiling problems for finite sets of prototiles as having the normal form of some $\Sigma^0_1$ formula - or the negation of one, if we desire an infinite planar tiling.

\section*{The Current Literature on Tiling Problems and Logic}

Firstly, we will summarise results in the literature that relate areas of logic to theorems and ideas about tiles, tilings, and prototile set properties and constructions.

Although Berger showed early on that Wang tiles are related to the undecidability of the Halting Problem, developments of using and studying tilings in mathematical logic is comparatively recent. 

Beginning with Harel in \cite{Harel1983}, who showed how problems of `high undecidability', \ie problems in $\Pi^1_1$, can be expressed as tiling problems. This is achieved in the plane by means of a set of carefully constructed Wang prototiles. Harel then built on this work in \cite{Harel1986} developing more full relationships between prototile sets and theorems about well/illfounded trees. Indeed, \cite{Harel1983} is cited by many texts in the field of Dynamic Logic - with Harel providing a chapter on this in the Handbook of Philosophical Logic \cite{Harel1984}. 

In `On the Convenience of Tilings' \cite{Boas97Conv}, van Emde Boas showed how various complexity classes are captured in specific tiling boundary results. Starting with an effective formulation of Turing Machines as prototile sets, van Emde Boas shows that a Wang prototile set that is unbounded vertically and horizontally is \textbf{NP}-complete, owing to the fact that a Turing Tape is realized left to right, whilst successive stages of a computation are realized vertically. Similarly, van Emde Boas continued by showing that a `corridor' tiling - a tiling that is of bounded width but unbounded height - is complete for \textbf{PSPACE}.

Following Durand's work on tilings and quasiperiodicity in \cite{Durand1999}, the work of Durand, Levin, and Shen \cite{Durand2008} showed that for every prototile set admits either no tiling or some tiling with $\mathcal{O}(n)$ Kolmogorov complexity of its $(n\times n)$-squares. Thatis to say, the string taken to describe any given square in the tiling has a complexity linearly related to the size of the square. This work was a continuation of their study of computational complexity paradigms and how they relate to tile sets and their planar tilings.

In Durand, Romashenko, and Shen \cite{Shen2010}, we find a significant development in the underlying theory of tilings - the existence of fixed point-based tilings. This work married up the work on Wang tiles with the previous work by Penrose and Amman on aperiodic Penrose tilings - see \cite[Chapters 10,11]{GrunbaumTP} for full presentations and discussions of these earlier works.

With these results in hand, recent work on $\Pi^0_1$ sets and tilings by Brown-Westrick in \cite{Westrick2017} utilised these self-similar Turing Machine tilings from \cite{Shen2010} in order to show that effectively closed subshifts of the distinct square shift are all sofic \cite[Theorem 1, 2]{Westrick2017}.

The study of tilings has, naturally from the above, been found and utilised in symbolic dynamics. A full introduction is found in the aforementioned Harel \cite{Harel1984}, with some interesting results being found recently in the work of Delvenne and Blondel \cite{Delvenne2004} where it is shown that by means of tiling problems, an analogue of Rice's theorem for computable functions is possible, giving that certain properties of dynamical systems are undecidable. As an extension to this result (Theorem 1 in \cite{Delvenne2004}), it is shown that topological entropy (as defined in \cite[Sec. 4.3, p.140]{Delvenne2004}) is undecidable for Turing Machines and tilings alike. Simpson in \cite{Simpson2007} also gave the following insight into tiling problems and their relation to mathematical logic, writing in \cite{Simpson2007} that:

\begin{quote}
``In the study of 2-dimensional subshifts of finite type, it has been useful to note that they are essentially the same thing as \emph{tiling problems} in the sense of Wang [ in \cite{Wang1990}].''
\end{quote}

Indeed, Levin's address, given as the Kolmogorov Lecture in 2005 at the University of London - see \cite{Levin2005} - gave some detail on the use of enumerable tilings in order to
prove that $2$-adic shifts and reflections can be enforced by a prototile set.

It is interesting to note that \cite{Delvenne2004} makes use of the notion of \emph{quasi-periodicity} - the property that every pattern $u$ of the tiling, there exists a $k$ such that any given $(k \times k)$ patch of tiles contains $u$. This notion is an interesting interim property that bridges the gap between fully periodic and fully aperiodic - see section \ref{subsec:quasiperiodic} for further details.

Adjacent to this work in mathematical logic, papers by Kari \cite{Kari1996} and later Culik \cite{CULIK1996245} showed how theorems about cellular automata that compute non-repeating reals can be converted into prototile sets to give very small sets of aperiodic prototiles. This work was generalised by Jeandel and Rao in \cite{Rao2015} to give the smallest possible set of aperiodic Wang prototiles, with a very small size of 11 prototiles to achieve this. They also proved through various means - both mathematically and with computational assistance - that this prototile set was smallest possible, and also had the property that if we were to remove any single tile from the prototile set, we no longer have tilings of the plane. Thereby, this prototile set either tiles aperiodically or fails to tile at all.

Having given this outline of the general view of tiling problems with respect to mathematical logic and related fields, we are now in a position to outline our contribution to this field.

\section*{Outline of the Thesis and Main Results}

Here we give an overview of the outline of the thesis, the main points in each chapter, and an account of the original work we are presenting in this volume.

\subsection*{Overview and Outline of the Thesis}

In chapter 1 we give a full background to the underlying mathematical logic and machinery we will use throughout the thesis. We give many definitions and present theorems generally without proofs, indicating sources along the way should they be necessary to the reader. We introduce precise definitions of Turing Machines as well as basic computability results that will be used later on. We also define various notions of reducibility in preparation for our work in Chapter 3.

We also give the background theory of computable trees as computable subsets of Baire space and Cantor space that form the backbone of many of our results in later chapters. We also give background results concerning the $\Pi^1_1$-completeness of Kleene's $\mathcal{O}$ which we shall use in later chapters. We finish this chapter with overview material for how computable trees, ordinals, and the arithmetical and analytic hierarchies hang together mathematically.

In chapter 2 we give an overview of core results regarding tilings and prototile sets. We give proofs of the Extension Theorem and state formally the first of our core tiling problems - the Domino Problem. We then give a proof of the undecidability of the Domino Problem by means of the computable conversion of any Turing Machine into a set of prototiles in such a way that their tilings tiling the plane iff the given Turing Machine on input $x$ does not halt. 

We introduce here the notion of a tile schema - a way of describing specific placement of colours from chosen colour sets. This allows us to describe (infinite) prototile sets by means of carefully chosen colour sets and schema tile construction such that the resultant product of combining these gives prototile sets whose tilings carry the specific properties we are looking for. Though this method may seem convoluted \textit{prima facie}, we hope to demonstrate that this technique leads in fact to quite straightforward proofs for translating various principles and concepts into the combinatorial properties of a prototile set. 

We round off this chapter by noticing some interesting corollaries and propositions arising from this fact that are of similar ilk to other results in mathematical logic - principally the fact that there exist prototile sets such that their domino problem is undecidable by Peano Arithmetic. 

In chapter 3 we state the first run of our main results - $\Pi^1_1$- and $\Sigma^1_1$-completeness of specific domino problems. We consider domino problems that require all tilings to be total, as well as domino problems that do not require total tilings, but instead only require an infinite connected patch of the plane to be tiled. To prove these results of $\Pi^1_1$ and $\Sigma^1_1$ completeness, we utilise the completeness for these classes due to wellfounded and illfounded trees. We construct tile schemas for each, and then demonstrate the completeness by means of $m$-reductions between our classes of prototile sets and ill-/well-founded trees.

With Chapter 4 we depart from domino problems, and instead consider the problems regarding whether or not the tilings for a given prototile set are all periodic, all aperiodic, or some mixture of the two. We state the fundamental results, with background references provided for this rather interesting class of problems.

We demonstrate that these notions are simultaneously $\Pi^1_1$ and $\Sigma^1_1$, as well as prove that, in fact, the questions of periodicity and aperiodicity for infinite sets of prototiles are both complete for the class of problems of the form $(\Pi^1_1 \wedge \Sigma^1_1)$. We also show that the set of all finite prototile sets whose tilings are aperiodic is $\Pi^0_1$, which is a surprising result.

Chapter 5 is an extension of this notion of computable reductions into the realm of Weihrauch reducibility. We give a feature rich presentation of the definitions and notions of Weihrauch reducibility, and state some core results. We then give intuitions for the core concepts in this theory, and proceed to derive Weihrauch equivalences between domino problems and closed choice on Baire space. 

Intuitively these results are motivated by realisation that all Wang tilings can be given by `tiling trees', first defined by Wang, for which closed choice realizers in Baire space can locate the infinite paths through, and from which we can recover a tiling of the plane. We can also consider that, given a non-deterministic prototile set - that is, for any prototile in the set there exist multiple possibilities for matching tiles in a given tiling - then having some choice principle in play is a natural conclusion. We give some exact results by means of Weihrauch equivalences.

The proposal for a new way of coding Elementary Cellular Automata (ECAs) into prototile sets is the subject of Chapter 6. Here, we demonstrate that for the 3-ary functions defining the behaviour of ECAs is naturally coded by a hexagon and lozenge based construction. With the requisite tiles to neaten up the upper edge of our tiling, we have a prototile set consisting of 15 tiles that very naturally give a way to represent the behaviour of ECAs in tilings of the half-plane by means of coding the first `input' row, and then making it such that the subsequent tilings of each row are exactly given by the underlying function of the given ECA.

We also show that such a prototile set is necessarily then chaotic and Turing Complete given correct choices for the ECA rule that we encode - Rule 30 and Rule 100 respectively for these results. Thus we have a nice and very small prototile set that carries with it a lot of possible mathematical capability.

Finally, we complete the thesis with an overview in Chapter 7 of the various open problems that we have found along the way - both in the literature and in the course of our research. We also aim to indicate the possible avenues for extending the results in this thesis further.

\subsection*{Summary of Original Work}

In this thesis, the following items are our original contributions:

\begin{itemize}
\item Our proof of theorem \ref{thm:TMTilings} is inspired by the form in \cite{Boas97Conv}, but is reshaped to match the structure of our later proofs. The observations leading up to corollary \ref{cor:PAGoodNotTile} have not been found in the literature, but are relatively straightforward to derive.
\item The results given in Chapter 3 are all original unless stated otherwise. Specifically, our main results are:
\begin{itemize}
\item Lemma \ref{lemma-pproc}
\item Theorem \ref{thm:TILE-ILL}
\item Theorem \ref{thm:nTILE-WELL}
\item Theorem \ref{thm:SNT-WELL}
\item Theorem \ref{thm:WTILE-ILL}
\end{itemize}
\item The results concerning $ATile$, $PTile$, $ATile_{FIN}$ and $PTile_{FIN}$ in Chapter 4 are all original:
\begin{itemize}
\item Theorem \ref{thm:ILL-ATile}
\item Theorem \ref{thm:ILL-PTile}
\item Theorem \ref{thm:WELL-PTile}
\item Theorem \ref{thm:WELL-ATile}
\item Theorem \ref{thm:X-ATile}
\item Theorem \ref{thm:X-PTile}
\item Corollary \ref{cor:ATilePTileComplete}
\item Theorem \ref{thm:ATileFIN-Pi01}
\item Theorem \ref{thm:PTileFINPi11}
\end{itemize}
\item The Weihrauch reductions for tiling problems in Chapter 5 are original:
\begin{itemize}
\item Theorem \ref{thm:CT-SW-ClosedC}
\item Theorem \ref{thm:CWPT-SW-ClosedC}
\item Theorem \ref{thm:CC-WKLstarCIPT}
\item Theorem \ref{thm:DPW-CC}
\end{itemize}
\item The main result in Chapter 6 is also original: Theorem \ref{thm:ECA-15-Hex}
\end{itemize}
