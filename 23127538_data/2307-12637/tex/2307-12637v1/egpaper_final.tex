\documentclass[10pt,twocolumn,letterpaper]{article}

\usepackage{iccv}
\usepackage{times}
\usepackage{epsfig}
\usepackage{graphicx}
\usepackage{amsmath}
\usepackage{mathbbol}
\usepackage{amssymb}
\usepackage{booktabs,makecell}
\usepackage{verbatim}
\usepackage{multirow}
\usepackage{authblk}
\usepackage{blindtext}


% Include other packages here, before hyperref.

% If you comment hyperref and then uncomment it, you should delete
% egpaper.aux before re-running latex.  (Or just hit 'q' on the first latex
% run, let it finish, and you should be clear).
\usepackage[breaklinks=true,bookmarks=false]{hyperref}

\iccvfinalcopy % *** Uncomment this line for the final submission

\def\iccvPaperID{7316} % *** Enter the ICCV Paper ID here
\def\httilde{\mbox{\tt\raisebox{-.5ex}{\symbol{126}}}}

% Pages are numbered in submission mode, and unnumbered in camera-ready
\ificcvfinal\pagestyle{empty}\fi

\begin{document}

%%%%%%%%% TITLE
\title{PG-RCNN: Semantic Surface Point Generation for 3D Object Detection}

\author{Inyong Koo\thanks{Denote equal contribution}
\quad Inyoung Lee\protect\CoauthorMark 
\quad Se-Ho Kim 
\quad Hee-Seon Kim 
\quad Woo-jin Jeon 
\quad Changick Kim}
\affil{
KAIST \\
Daejeon, South Korea\protect \\
\tt\small \{iykoo010, inzero24, ksh1040, hskim98, woojin.jeon337, changick\}@kaist.ac.kr
}
\newcommand\CoauthorMark{\footnotemark[\arabic{footnote}]}
%\author{Inyong Koo \quad Inyoung Lee \quad Se-Ho Kim \quad Hee-Seon Kim \quad Woo-jin Jeon \quad Changick Kim\\
%KAIST\\
%Daejeon, South Korea\\
%{\tt\small \{iykoo010, inzero24, ksh1040, hskim98, woojin.jeon337, changick\}@kaist.ac.kr}
% For a paper whose authors are all at the same institution,
% omit the following lines up until the closing ``}''.
% Additional authors and addresses can be added with ``\and'',
% just like the second author.
% To save space, use either the email address or home page, not both
%}

\maketitle
% Remove page # from the first page of camera-ready.
 \ificcvfinal\thispagestyle{empty}\fi

%%%%%%%%% ABSTRACT
\begin{abstract}
Graph Neural Networks (GNNs) have proven to be effective in processing and learning from graph-structured data.
However, previous works mainly focused on understanding single graph inputs while many real-world applications require pair-wise analysis for graph-structured data (e.g., scene graph matching, code searching, and drug-drug interaction prediction).
To this end, recent works have shifted their focus to learning the interaction between pairs of graphs.
Despite their improved performance, these works were still limited in that the interactions were considered at the node-level, resulting in high computational costs and suboptimal performance.
To address this issue, we propose a novel and efficient graph-level approach for extracting interaction representations using co-attention in graph pooling. 
Our method, Co-Attention Graph Pooling (CAGPool), exhibits competitive performance relative to existing methods in both classification and regression tasks using real-world datasets, while maintaining lower computational complexity.

\end{abstract}

%%%%%%%%% BODY TEXT

\section{Introduction}
Deep learning models have been widely used in many applications.
For example, BERT~\citep{devlin_bert_2019}, GPT-3~\citep{brown_language_2020}, and T5~\citep{raffel_exploring_2020} achieved state-of-the-art~(SOTA) results on different natural language processing~(NLP) tasks. 
For computer vision~(CV), Transformer-like models such as ViT~\citep{dosovitskiy_image_2021} and Swin Transformer~\citep{liu_swin_2021} deliver excellent accuracy performance upon multiple tasks. 


At the same time, training deep learning models has been a critical problem troubling the community due to the long training time, especially for those large models with billions of parameters~\citep{brown_language_2020}. 
In order to enhance the training efficiency, researchers propose some manually designed parallel training strategies~\citep{narayanan_efficient_2021,shazeer_mesh-tensorflow_2018,xu_gspmd_2021}. 
However, selecting, tuning, and combining these strategies require extensive domain knowledge in deep learning models and hardware environments. With the increasing diversity of modern hardware architectures~\cite{flynn_very_1966,flynn_computer_1972} and the rapid development of deep learning models, these manually designed approaches are bringing heavier burdens to developers. 
Hence, \emph{automatic parallelism} is introduced to automate the parallel strategy searching for training models.


There are two main categories of parallelism in deep learning models: inter-layer parallelism~\citep{huang_gpipe_2019,narayanan_pipedream_2019,narayanan_memory-efficient_2021,fan_dapple_2021,li_chimera_2021,lepikhin_gshard_2021,du_glam_2022,fedus_switch_2022} and intra-layer parallelism~\citep{li_pytorch_2020,narayanan_efficient_2021,rasley_deepspeed_2020,fairscale_authors_fairscale_2021}. 
Inter-layer parallelism partitions the model into disjoint sets on different devices without slicing tensors. 
Alternatively, intra-layer parallelism partitions tensors in a layer along one or more axes and distributes them across different devices.


Current automatic parallelism techniques focus on optimizing strategies within these two categories. However, they treat these two categories separately. 
Some methods~\citep{zhao_vpipe_2022,jia_exploring_2018,cai_tensoropt_2022,wang_supporting_2019,jia_beyond_2019,schaarschmidt_automap_2021,liu_colossal-auto_2023} overlook potential opportunities for inter- or intra-layer parallelism, the others optimize inter- and intra-layer parallelism hierarchically and sequentially~\citep{narayanan_pipedream_2019,fan_dapple_2021,he_pipetransformer_2021,tarnawski_efficient_2020,tarnawski_piper_2021,zheng_alpa_2022}. 
As a result, current automatic parallelism techniques often fail to achieve the global optima and instead become trapped in local optima. 
Therefore, a unified inter- and intra-layer approach is needed to enhance the effectiveness of automatic parallelism.


This paper aims to find the optimal parallelism strategy while simultaneously considering inter- and intra-layer parallelism. 
It enables us to search in a more extensive strategy space where the globally optimal solution lurk. 
However, unifying inter- and intra-layer parallelism in automatic parallelism brings us two challenges. 
Firstly, to adopt a unified perspective on the inter- and intra-layer automatic parallelism, we should not formalize them with separate formulations as prior works. Therefore, how can we express these parallelism strategies in a unified formulation? 
Secondly, previous methods take a long time to obtain the solution with a limited strategy space. Therefore, how can we ensure that the best solution can be obtained in a reasonable time while expanding the strategy space?


To solve the above challenges, we propose UniAP. For the first challenge, UniAP adopts the mixed integer quadratic programming~(MIQP)~\citep{lazimy_mixed_1982} to search for the globally optimal parallel strategy automatically. 
It unifies the inter- and intra-layer automatic parallelism in a single MIQP formulation. 
For the second challenge, our complexity analysis and experimental results show that UniAP can obtain the globally optimal solution in a significantly shorter time.


The contributions of this paper are summarized as follows: 
\begin{itemize}
    \item We propose UniAP, the first framework to unify inter- and intra-layer automatic parallelism in model training.
    \item The optimal parallel strategies discovered by UniAP exhibit scalability on training throughput and strategy searching time.
    \item The experimental results show that UniAP speeds up model training on four Transformer-like models by up to 1.70$\times$ and reduces the strategy searching time by up to 16$\times$, compared with the SOTA method.
\end{itemize}

% Figure environment removed

\section{Related Works}
\subsection{Density-aware Dehazing Methods}

In recent years, several methods \cite{zhang2021hierarchical,deng2020hardgan,guo2022image,yang2022self,wang2021haze,yi2022two,yeperceiving} have attempted to improve the dehazing performance by enabling the network to perceive haze density.

\subsubsection{Density-awareness via estimating T-map} Haze density is influenced by several factors and is inversely proportional to T-map, so some methods learn density information by estimating T-map. Lou et al. \cite{lou2020integrating} predict a T-map first for nighttime image dehazing. Zhang et al. \cite{zhang2021hierarchical} estimate a low-resolution T-map and then jointly input the feature map and the estimated T-map to a Laplacian pyramid decoder to achieve a restored image. Yang et al. \cite{yang2022self} propose a semi-supervised method that does not require paired data. The method estimates T-map, scattering coefficient, and depth to reconstruction hazy images and restores clear images. However, these methods require additional labeled data and might be inaccurate due to the complexity of practical scenes \cite{li2017aod}.
% The Haze-Aware Feature Distillation (HARD) module is designed in \cite{deng2020hardgan}, which introduces two dual-channel maps to describe the atmospheric brightness and pixel-wise spatial information of each feature channel respectively, and calculates a haze aware map through an InstanceNorm followed by a sigmoid layer. Finally it fuses the above three factors.

% Figure environment removed

\subsubsection{Density-awareness via extracting density features directly} Research works \cite{deng2020hardgan,chen2020unsupervised,yeperceiving} directly learn haze density information without estimating a T-map. Deng et al. \cite{deng2020hardgan} design a Haze-Aware Representation Distillation (HARD) module to extract global brightness and a haze-aware map. Chen et al. \cite{chen2020unsupervised} propose an attention mechanism based on dark channel prior to describe haze concentration. However, not estimating the T-map would result in a lack of a comparator to measure density. Generating intermediate results and using the information contained therein can address this issue.

\subsection{Dehazing Methods with Intermediate Results}
Considering the difficulty of recovering images directly from the haze input, dehazing methods \cite{bai2022self,chen2021desmokenet,yeperceiving,hong2022uncertainty} which generate intermediate results (or one result) inside the network to facilitate the dehazing process are proposed. Bai et al. \cite{bai2022self} first generate a reference image by a deep pre-dehazer, and then develop a progressive feature fusion module to fuse the hazy and reference features, which achieves high metrics on several datasets. Chen et al. \cite{chen2021desmokenet} first remove light and thick smoke by a Smoke Remove Network (SRN) to gain a coarse output, which is concatenated with the original input and fed to a Pixel Compensation Network (PCN) to recover the missing pixels in the thick smoke. Hong et al. \cite{hong2022uncertainty} propose an Uncertainty-Driven Dehazing Network. In this method, intermediate results are together generated with uncertainty maps for uncertainty features extraction. Ye et al. \cite{yeperceiving} also pre-generate a pseudo-haze-free image. The hazy input and the pseudo-haze-free image are concatenated to estimate a Density Encoding Matrix describing the relationship between haze density and absolute position and mixed up to the following deep layers.

Despite the above methods extracting feature from intermediate results, they do not fully consider the differences between these results and the haze inputs, especially the differences in haze density. Simple concatenation \cite{chen2021desmokenet,bai2022self} or linear summation \cite{yeperceiving} might lead the networks to rely on the uncertain learning process and lose the capture of information about the differences between the two images. In addition, the lack of a targeted design that addresses the relationship between the intermediate results and the original input leads the extracted features not fine enough and limits the dehazing performance.

Our DFR-Net improves on the aforementioned methods by exploring and refining density features through the utilization of density differences between a generated proposal image and the hazy input, thereby achieving an awareness of haze density and superior dehazing performance.




\section{PG-RCNN}
PG-RCNN is a two-stage method for 3D object detection composed of a region proposal stage and a proposal refinement stage. 
Figure \ref{fig:2_framework} illustrates the overview of the PG-RCNN framework. 
For the first stage, the region proposal network (RPN) with a voxel-based backbone generates the initial bounding box proposals. 
Our main novelty lies in the second stage, where we introduce the RoI point generation (RPG) module to create a semantic surface point cloud for each proposal. 
The RPG module aggregates the backbone voxel features in a grid and uses a Transformer \cite{transformer_vaswani} encoder to capture the global context of the RoI.
Then an MLP is individually applied to each grid point feature to output the coordinates offset and the semantic feature of the generated point.
Lastly, the detection head produces the final detection output using the bounding box refinement features extracted from the generated point clouds with semantic features.

%Lastly, the detection head uses the PointNet++ \cite{pointnet++_qi} encoder to extract refinement features from the generated point clouds with semantic features, producing the final detection output.

\subsection{Region Proposal Network}

Following many recent works \cite{pvrcnn_shi, voxelrcnn_deng, ct3d_sheng, pdv_hu}, we adopt SECOND \cite{second_yan} as our RPN for its high efficiency and recall.
%To generate 3D region proposals, we adopt SECOND \cite{second_yan}, a widely used voxel-based object detector due to its high efficiency. 
The input raw point cloud is first divided into evenly spaced voxels and gradually processed with the 3D backbone network composed of a series of sparse convolution layers, resulting in multiple scales of feature volumes.
% The downsampled feature volumes are then projected along the Z-axis and converted into a bird's-eye view (BEV) feature map, where the proposal layers produce dense prediction with the classification and box regression branch to generate initial detection output for the later refinement stage.
The downsampled feature volumes are projected along the Z-axis and converted into a bird's-eye view (BEV) feature map.
The proposal layers use the BEV feature map to produce dense predictions with the classification and box regression branches to generate initial detection output for the later refinement stage.


\subsection{RoI Point Generation Module}

 Previous approaches \cite{sienet_li, pcrgnn_zhang} leverage point-based completion models, using raw points pooled from RoI as input. 
Instead, our RPG module exploits voxel features from the 3D backbone, which contain rich context information about their surroundings.

We begin by dividing a region proposal into $G \times G \times G$ regular sub-voxels, using the center coordinates of these sub-voxels as grid points.
Then, we use a method from Voxel R-CNN \cite{voxelrcnn_deng} to aggregate voxel features at the grid points.
Specifically, a grid point $\mathbf{g}_i$ is quantified into a voxel, so that the neighboring voxels are efficiently obtained by indices translation.
Using a PointNet++ \cite{pointnet++_qi} module, we aggregate its feature $\mathbf{f}_{\mathbf{g}_i}$ from the sampled set of neighboring voxels $\Gamma_i = \{\mathbf{v}_i^1, \mathbf{v}_i^2, \cdots, \mathbf{v}_i^K\}$ as follows:
\begin{equation}
\mathbf{f}_{\mathbf{g}_{i}} = {MaxPool} \left( \{{\mathcal{A}^{agg}([\mathbf{v}_i^k - \mathbf{g}_i; \mathbf{f}_{\mathbf{v}_i^k}])}\}_{k=1}^K \right),
\end{equation}
where $\mathcal{A}^{agg}(\cdot)$ represents the MLP for feature aggregation, $\mathbf{v}_i^k - \mathbf{g}_i$ represents relative coordinates, and $\mathbf{f}_{\mathbf{v}_i^k}$ is the feature of voxel $\mathbf{v}_i^k$. The RPG module aggregates voxel features from feature volumes of the last three stages in the 3D backbone network and concatenates the multi-scale features.

%The features pooled at the grid points are independently aggregated at their position and lack RoI-level context information for estimating object shapes.
The feature pooled at each grid point contains local information about its surroundings but lacks RoI-level context information for estimating object shapes.   
%Features are independently pooled from the grid points lack RoI-level context information for estimating object shapes.
%Features independently pooled from the grid points lack RoI-level context information for estimating object shapes.
To capture the long-range dependencies between the grid points, we further process the features with a Transformer encoder.
In Section \ref{sec:4.4}, we demonstrate the effectiveness of utilizing the Transformer encoder in enhancing object detection performance.
The refined grid point feature $\Tilde{\mathbf{f}}_{\mathbf{g}_i}$ is formulated as
\begin{equation}
    \Tilde{\mathbf{f}}_{\mathbf{g}_i} = \mathcal{T}(\mathbf{f}_{\mathbf{g}_i}, \mathbf{\delta}_{\mathbf{g}_i}),
\end{equation}
where $\mathbf{\delta}_{\mathbf{g}_i}$ is the positional encoding, and $\mathcal{T}(\cdot)$ is a standard Transformer encoder. To encode positional information, we apply a shallow feedforward neural network (FFN) to the relative coordinates of the grid point with respect to the region proposal bounding box, as described in \cite{ct3d_sheng}:
\begin{equation}
    \mathbf{\delta}_{\mathbf{g}_i} = \mathcal{A}^{pos}([\mathbf{g}_i - \mathbf{r}^c; \mathbf{g}_i - \mathbf{r}^1; \mathbf{g}_i - \mathbf{r}^2; \cdots; \mathbf{g}_i - \mathbf{r}^8]),
\end{equation}
where $\mathcal{A}^{pos}(\cdot)$ represents the FFN, $\mathbf{r}^c$ is the center, and $\mathbf{r}^{1,2,\cdots,8}$ are the eight corners of the bounding box.

% Finally, an MLP is applied to $\Tilde{\mathbf{f}}_{\mathbf{g}_i}$ and outputs $[\mathbf{o}_j ; \mathbf{f}_{\mathbf{p}_i}] \in \mathbb{R}^{3+C}$
Finally, a two-layer MLP $\mathcal{A}^{gen}(\cdot)$ is applied to the refined features to generate the offset $\mathbf{o}_i$ from the grid point, as well as the semantic feature of the generated point $\mathbf{f}^{se}_{\mathbf{p}_i}$:
\begin{equation}
    [\mathbf{o}_i; \mathbf{f}^{se}_{\mathbf{p}_i}] = \mathcal{A}^{gen}(\Tilde{\mathbf{f}}_{\mathbf{g}_i}).
\end{equation}
The generated point's coordinates $\mathbf{p}_i = (x_i,y_i,z_i)$ can be calculated as  $\mathbf{g}_i + \mathbf{o}_i$, and the foreground score $s_i$ for each generated point is calculated by applying a linear projection $A$ and a sigmoid function $\sigma$ to its semantic feature, \ie, 
\begin{equation}
    s_i = \sigma(A\mathbf{f}^{se}_{\mathbf{p}_i}).
\end{equation}

\subsection{Detection Head}

%We use a PointNet-based encoder to extract shape-aware RoI features from the generated points with semantic features.
Our detection head is inspired by the design of PointRCNN \cite{pointrcnn_shi} where it uses the PointNet++ encoder to extract refinement features from semantic point clouds.
For every generated point, we obtain the local spatial feature $\mathbf{f}^{sp}_{\mathbf{p}_i}$ with an MLP $\mathcal{A}^{loc}$ as follows:
\begin{equation}
    \mathbf{f}^{sp}_{\mathbf{p}_i} = \mathcal{A}^{loc} \left( \left[x_i^c, y_i^c, z_i^c, d_i, s_i \right]\right).
\end{equation}
Here, $(x_i^c, y_i^c, z_i^c)$ are the coordinates of the generated point $\mathbf{p}_i$ in the canonical coordinates system of the bounding box, and $d_i=\sqrt{x_i^2 + y_i^2 + z_i^2}$ is the depth of the point. 
The canonical transformation facilitates robust local spatial feature learning.
However, the transformation causes the inevitable loss of points' depth information, so we append $d_i$ as an additional feature. 
The estimated foreground score $s_i$ is also appended as the feature that represents the significance of the generated point.
%The canonical transformation facilitates robust local spatial feature learning, and depth and foreground score are concatenated as additional features that implicitly represent the reliability and importance of each point.
We merge $\mathbf{f}^{sp}_{\mathbf{p}_i}$ and $\mathbf{f}^{se}_{\mathbf{p}_i}$ for each point $\mathbf{p}_i$, and feed the point set with features into the PointNet++ encoder to obtain the refinement feature for RoI $\mathbf{f}^{r}$:
\begin{equation}
   \mathbf{f}^{r} = \mathcal{P}\left( \{\mathbf{p}_i\}_{i=1}^{G^3}, \{[\mathbf{f}^{sp}_{\mathbf{p}_i};\mathbf{f}^{se}_{\mathbf{p}_i}]\}_{i=1}^{G^3}\right), 
\end{equation}
where $\mathcal{P}(\cdot)$ denotes the PointNet++ encoder taking the set of point coordinates and the corresponding feature set as input.
The RoI feature serves as the input for confidence classification and bounding box refinement branches, resulting in the final detection output.

\subsection{Training Losses}

PG-RCNN is an end-to-end model trained with the summation of the region proposal loss $\mathcal{L}_{\mathrm{RPN}}$, the proposal refinement loss $\mathcal{L}_{\mathrm{head}}$, and the point generation loss $\mathcal{L}_{\mathrm{RPG}}$:
\begin{equation}
    \mathcal{L}_{total} = \mathcal{L}_{\mathrm{RPN}} + \mathcal{L}_{\mathrm{head}} + \mathcal{L}_{\mathrm{RPG}}. 
\end{equation}

$\mathcal{L}_{\mathrm{RPN}}$ and $\mathcal{L}_{\mathrm{head}}$ are conventional training losses for two-stage detectors calculated with the outputs of the RPN and the detection head, respectively.
Both losses are composed of a classification and a regression term.
The classification targets are assigned based on the intersection-over-union (IoU) of the anchors and the proposals with the ground truth bounding boxes. Only foreground anchors and proposals contribute to the regression losses, using the regression target given by their ground truth residuals.
Focal Loss \cite{focal_yun} is used for the RPN's classification branch output, while binary cross-entropy loss is used for the detection head's confidence branch output. For the regression loss, we use the smooth-L1 loss for both losses.
%The $\mathcal{L}_{\mathrm{RPN}}$ is formulated as:
%\begin{multline}
%    \mathcal{L}_{\mathrm{RPN}} = \frac{1}{N_{fg}} \sum_j \left[ \mathcal{L}_{focal} (c_j, c_j^*) \right. \\ 
%    \left.+ \mathbb{1}(c_j^* > 1)\mathcal{L}_{smooth-L1} (r_j, r_j^*) \right],
%\end{multline}
%where $N_{fg}$ is the number of foreground anchors, and $c_j$ and $r_j$ are the 3D RPN's outputs of classification and box regression branches, $c_j^*$ and $r_j^*$ are the classification labels and ground-truth anchor residual. $\mathbb{1}(c_j^* > 1)$ indicates that foreground anchors contribute to the regression loss. Here, Focal loss \cite{focalloss} was utilized for the classification branch, and smooth-L1 loss for the box regression branch.

$\mathcal{L}_{\mathrm{RPG}}$ is an auxiliary loss term that specifically supervises point generation, calculated with the RPG module outputs:
\begin{equation}
    \mathcal{L}_{\mathrm{RPG}} = \mathcal{L}_{score} + \mathcal{L}_{offset}.
\end{equation}
$\mathcal{L}_{score}$ is a point-level segmentation loss that governs the foreground scores of generated points.
We assign segmentation labels to generated points by checking if they are inside a ground-truth bounding box.
Since we generate $G^3$ points for each proposal, calculating loss at all generated points would be computationally expensive.
We select $N_p$ points from the scene using the FPS algorithm and apply Focal Loss on the sampled points, \textit{i.e.},
\begin{equation}
    \mathcal{L}_{score} = -\frac{1}{N_p} \sum_{j}{(1-s_j)^\gamma \log{s_j}}
\end{equation}
where $s_j, j=1, 2, \cdots N_p$ are the foreground score of the sampled points. 


% Figure environment removed

On the other hand, $\mathcal{L}_{offset}$ supervises the shape of the generated point clouds. 
Since the KITTI dataset does not provide complete point clouds of the object instances, previous approaches \cite{pcrgnn_zhang, sienet_li} used external datasets such as ShapeNet \cite{shapenet} to train their point cloud completion network in advance. 
Instead, we exploit other object instances within the provided dataset to approximate the complete shape of the object.
Specifically, we use the approximation method proposed in \cite{btc_xu}.
We first search for other objects of the same class that have similar bounding boxes and point distributions.
Then we combine the point sets of two best-matching objects with the original points and produce a dense point cloud.
For cars and cyclists, we assume symmetry along the object's heading axis and mirror the points accordingly.
Figure \ref{fig:3_target} displays an example of a completed point cloud for each class.
%Figure \ref{fig:3_target} displays an example generation target for each class, and we explain the approximation method for the KITTI dataset in Section \ref{sec:4.1}.
Using the completed point clouds as generation targets, we employ Chamfer Distance on foreground proposals as follows:
\begin{multline}
    \mathcal{L}_{offset} = \frac{1}{N_{fp}}\sum_{r}\left(\dfrac{1}{|\mathbf{P}_r|} \sum_{\mathbf{x} \in \mathbf{P}_r}\min_{\mathbf{y}\in \mathbf{P}_r^*}||\mathbf{x}-\mathbf{y}||^2_2 + \right. \\
    \left. \dfrac{1}{|\mathbf{P}_r^*|} \sum_{\mathbf{y} \in \mathbf{P}_r^*}\min_{\mathbf{x}\in \mathbf{P}_r}||\mathbf{y}-\mathbf{x}||^2_2 \right),  
\end{multline}
where $N_{fp}$ is the number of foreground proposals, and $\mathbf{P}_r$ and $\mathbf{P}_r^*$ are the generated and the target point cloud of the $r$-th foreground proposal where $r=1, 2, \cdots, N_{fp}$, respectively.

%Formally, $\mathcal{L}_{\mathrm{RPG}}$ is formulated as:
%\begin{multline}
%    \mathcal{L}_{\mathrm{RPG}} = \frac{1}{N_p}\sum_{j}\mathcal{L}_{focal}(s_{j}, s_{j}^*) + \frac{1}{N_{fg}}\sum_{r}\mathcal{L}_{CD}(\mathbf{P}_r, \mathbf{P}_r^*)
%    \frac{1}{N_{fg}}\sum_{r}\left(\dfrac{1}{|\mathbf{P}_r|} \sum_{x \in \mathbf{P}_r}\min_{y\in \mathbf{P}_r^*}||x-y||^2_2 + \dfrac{1}{|\mathbf{P}_r^*|} \sum_{y \in \mathbf{P}_r^*}\min_{x\in \mathbf{P}_r}||y-x||^2_2 \right)
%\end{multline}


%In order to generate points that accurately represent the surface of objects, it is necessary to have additional supervision for learning the complete shape of the objects.
%For supervision, a dense and complete point cloud of objects is required, but the datasets \cite{kitti, waymo} used for training do not provide this.
%Therefore, other object detection models \cite{pcrgnn_zhang, sienet_li} that generate points according to objects' shape used a pre-trained point completion network using the external dataset, ShapeNet \cite{shapenet}.
%In contrast to previous approaches, we simultaneously train point generation and object detection rather than using a pre-trained model.
%Since this method requires a dense and complete target corresponding to each object, we generate the target using the method motivated by \cite{btc_xu, sparse2dense_wang}.
%First, it is necessary to find the point sets that match the original points of the object. In the case of KITTI \cite{kitti}, the points with the most similar distribution to the original points are matched among the same class. In contrast, in the case of the Waymo Open Dataset \cite{waymo} with multi-frame, points of the same object appearing in other frames are matched.
%The matching point sets found in this way are then combined with the original points.
%Assuming that Car and Cyclist are approximately symmetrical, we mirror the points about the object's heading axis.
%As a result, a target with dense and complete points is generated for each object.
%Further details on target generation are described in the supplementary.


% % Figure environment removed


% % Figure environment removed

% % Figure environment removed

% % Figure environment removed

% % Figure environment removed

% % Figure environment removed

% Figure environment removed


% Figure environment removed

% Figure environment removed


\subsection{Implementation Details}


\paragraph{Network.} In order to disentangle shape and color latent information within the hashgrids, we split the single hash table in the NeRF network architecture of Instant-NGP~\cite{mueller2022instant} into two: a density grid $\mathcal{G}^{\sigma}$ and a color grid $\mathcal{G}^c$, with the same settings as the original density grid in the open-source PyTorch implementation torch-ngp~\cite{torch-ngp}. We do this to make it possible to make fine-grained edits of one to one of the color or geometry properties without affecting the other. The rest of the network architecture remains the same, including a sigma MLP $f^\sigma$ and a color MLP $f^c$. For a spatial point $\mathbf{x}$ with view direction $\mathbf{d}$, the network predicts volume density $\sigma$ and color $c$ as follows:
\begin{align}
    \sigma, \mathbf{z} &= f^\sigma(\mathcal{G}^{\sigma}(\mathbf{x})) \\
    c &= f^c(\mathcal{G}^c(\mathbf{x}),\mathbf{z},\mathrm{SH}(\mathbf{d}))
\end{align}
where $\mathbf{z}$ is the intermediate geometry feature, and $\mathrm{SH}$ is the spherical harmonics directional encoder~\cite{mueller2022instant}. The same as Instant-NGP's settings, $f^\sigma$ has 2 layers with hidden channel 64, $f^c$ has 3 layers with hidden channel 64, and $\mathbf{z}$ is a 15-channel feature.

We compare our modified NeRF network with the vanilla architecture in the Lego scene of NeRF Blender Synthetic dataset\cite{mildenhall2020nerf}. We train our network and the vanilla network on the scene for 30,000 iterations. The result is as follows:
\begin{itemize}
    \item Ours: training time 441s, PSNR 35.08dB
    \item Vanilla: training time 408s, PSNR 34.44dB
\end{itemize}
We observe slightly slower runtime and higher quality for our modified architecture, indicating that this modification causes negligible changes.

\paragraph{Training.}
% We use Instant-NGP\fcite{NGP} as our editing framework backbone to achieve real-time editing preview. 
We select Instant-NGP~\cite{mueller2022instant} as the NeRF backbone of our editing framework.
Our implementations are based on the open-source PyTorch implementation torch-ngp~\cite{torch-ngp}. All experiments are run on a single NVIDIA RTX 3090 GPU. Note that we make a slight modification to the original network architecture. Please refer to the supplementary material for details.

During the pretraining stage, we set $\msymbol{weight_pretrain_color}=\msymbol{weight_pretrain_sigma}=1$ and the learning rate is fixed to $0.05$. During the finetuning stage, we set $\msymbol{weight_train_color} = \msymbol{weight_train_depth} = 1$ with an initial learning rate of 0.01. 
% The bit field mask of the editing space is filled so that the editing space can be fully sampled during training. 
Starting from a pretrained NeRF model, we perform 50-100 epochs of local pretraining (for about 0.5-1 seconds) and about 50 epochs of global finetuning (for about 40-60 seconds). The number of epochs and time consumption can be adjusted according to the editing type and the complexity of the scene. Note that we test our performance in the absence of tiny-cuda-nn~\cite{tiny-cuda-nn} which achieves superior speed to our backbone, which indicates that our performance has room for further optimization.
% Note that the training speed is evaluated when tiny-cuda-nn is not enabled.

\paragraph{Datasets.}
We evaluate our editing in the synthetic\Skip{lego, chair, and ship from} NeRF Blender Dataset~\cite{mildenhall2020nerf}, and the real-world captured \Skip{family and truck from}Tanks and Temples~\cite{Knapitsch2017} and \Skip{, and scan83 from} DTU~\cite{jensen2014large} datasets. We follow the official dataset split of the frames for the training and evaluation.


% Figure environment removed

% Figure environment removed

% Figure environment removed

\subsection{Experimental Results}
\label{sec-results}
% \paragraph{Comparisons of rendering quality between teacher and student network.} 


\paragraph{Qualitative NeRF editing results.} 
We provide extensive experimental results in all kinds of editing categories we design, including bounding shape (\cref{fig-bbox,fig-bbox-elf}), brushing (\cref{fig-brush}), anchor (\cref{fig-anchor}), and color (\cref{fig-teaser}). Our method not only achieves a huge performance boost, supporting instant preview at the second level but also produces more visually realistic editing appearances, such as shading effects on the lifted side in \cref{fig-brush} and shadows on the bumped surface in \cref{fig-neumesh}. Besides, results produced by the student network can even outperform the teacher labels, \eg in \cref{fig-bbox-elf} the $F^t$ output contains floating artifacts due to view inconsistency. As analyzed in \cref{sec-train}, the distillation process manages to eliminate this. We also provide an example of object transfer (\cref{fig-bbox-baby}): the bulb in the Lego scene (of Blender dataset) is transferred to the child's head in the family scene of Tanks and Temples dataset.
% \Skip{
% We evaluate our method on all the editing types we design, \ie bounding shape, brushing and anchor, respectively:
% \begin{itemize}
%     \item Bounding shape editing. As shown in \cref{fig-bbox}, we scale the warning light on the top of the Lego model, shorten the chair leg, \zjs{TBD}, and provides plausible results.
%     \item Brushing and color editing. As shown in \cref{fig-brush}, our method edits the scene according to the user's paintings (\ie a cross sign on the chair back, a heart shape on the car logo, and \zjs{TBD}). Note that our brushing method supports simultaneous geometry lifting, as shown in the ``cross'' example. Due to our shading preservation strategy in HSL space, the edited surface can contain realistic visual effects (see the shading effects of the lifted surface).
%     \item Anchor editing. As shown in \cref{fig-anchor}, our method edits the scene according to the anchor points (\ie ship's bow, bulldozer's shovel and \zjs{TBD}) and the stretching direction. The edited geometry has consistent appearance with the anchored area.
% \end{itemize}
% }

% Figure environment removed

% Figure environment removed

\paragraph{Comparisons to baselines.} Existing works have strong restrictions on editing types, which focus on either geometry editing or appearance editing, while ours is capable of doing both simultaneously. Our brushing and anchor tools can create user-guided out-of-proxy geometry structures, which no existing methods support. We make comparisons on color and texture painting supported by NeuMesh~\cite{neumesh} and Liu \etal~\cite{liu2021editing}. 

\cref{fig-neumesh} illustrates two comparisons between our method and NeuMesh~\cite{neumesh} in scribbling and a texture painting task. Our method significantly outperforms NeuMesh, which contains noticeable color bias and artifacts in the results. In contrast, our method even succeeds in rendering the shadow effects caused by geometric bumps.

\cref{fig-neumesh-mic} illustrates the results of the same non-rigid blending applied to the Mic from NeRF Blender\cite{mildenhall2020nerf}. It clearly shows that being mesh-free, We have more details than NeuMesh\cite{neumesh}, unlimited by mesh resolution.

\cref{fig-editnerf} shows an overview of the pixel-wise editing ability of existing NeRF editing methods and ours. Liu \etal~\cite{liu2021editing}'s method does not focus on the pixel-wise editing task and only supports textureless simple objects in their paper. Their method causes an overall color deterioration within the edited object, which is highly unfavorable. This is because their latent code only models the global color feature of the scene instead of fine-grained local features. Our method supports fine-grained local edits due to our local-aware embedding grids.

% \yq{describe the difference}

% \paragraph{Artistic applications (a comic on NeRF).} Based on the four example tools we implemented, we created a comic \textit{Bob the Bulb} (Fig. \ref{fig-comic}) to show the potential applications of our editing method. This might be the first artwork created with NeRF.

% \subsection{Experiments on Bounding Shape Editing}
% \subsection{Experiments on Brush Editing}
% \subsection{Experiments on Anchor Editing}
% \subsection{Experiments on Color Shape Editing}
% \subsection{Real-world Example: NeRF-Rendered Comic}


\subsection{Ablation Studies}
\label{sec-ablation}
\paragraph{Effect of the two-stage training strategy.} To validate the effectiveness of our pretraining and finetuning strategy, we make comparisons between our full strategy (3\textsuperscript{rd} row), finetuning-only (1\textsuperscript{st} row) and pretraining-only (2\textsuperscript{nd} row) in \cref{fig-ab_pre}. Our pretraining can produce a coarse result in only 1 second, while photometric finetuning can hardly change the appearance in such a short period. The pretraining stage also enhances the subsequent finetuning, in 30 seconds our full strategy produces a more complete result. However, pretraining has a side effect of local overfitting and global degradation. Therefore, our two-stage strategy makes a good balance between both and produces optimal results.

\paragraph{MLP fixing in the pretraining stage.} In \cref{fig-ab_fix}, we validate our design of fixing all MLP parameters in the pretraining stage. The result confirms our analysis that MLP mainly contains global information so it leads to global degeneration when MLP decoders are not fixed.


We proposed a machine-learning based method to approximate diagonal as well as non-diagonal elements of the Hessian of a molecule. The representation used is specific for every internal coordinates, and takes explicitly into account the bond order, which is sensible because a single point DFT calculation is computationally considerably less expensive that the explicit calculation of the Hessian.
We trained our ML model on a relatively small dataset (subset of QM7) of less than 7000 molecules. The Hessian was computed at the B3LYP/cc-pVDZ level of theory. 
The agreement between ML and DFT was satisfactory. In particular, the calculated MAPE for bond stretching force constant was below 2\%, and was particularly small for bonds involving hydrogen atoms because they point outwards and are less affected by the chemical environment. The MAPE for bending and torsion was of 5\% and 10\%, respectively. 
From the ML model trained on QM7 we were also able to predict the Hessian of some molecules representative of the QM9 dataset. The Hessian predicted in internal coordinates was then transformed into the mass-weighted Cartesian Hessian, the diagonalization of which yields the harmonic vibrational frequencies and normal modes, that can be compared to the ones calculated  explicitly from DFT.

High frequency vibrations and normal modes were predicted accurately, while lower frequency ones were not. This behaviour is analogous to the IR spectroscopy theory, where stretchings and bendings can be identified accurately, while torsion and delocalized vibrations are more difficult to be interpreted.

The approximate Hessian obtained with ML is computational inexpensive and can be used as an initial Hessian guess for geometry optimization, or in the context of alchemical geometry relaxation \cite{Domenichini2020,domenichini2022alchemical, shiraogawa2022exploration,shiraogawa2023optimization}. 
A good starting Hessian may speed up the convergence of the geometrical optimization. An in detail assessment of the performance of the ML Hessian proposed is not yet provided, but should carefully take into account many parameters on which the optimization depends, \textit{e.g.} the type of molecule, the initial geometry, the optimization algorithm, and the Hessian update scheme.



%-------------------------------------------------------------------------


% \begin{table}[ht!]
%  \caption{Ablation}
% \centering
% \begin{tabular}{l|c c c|c}
%   \hline
%    \multirow{2}{*}{Method} &
%    \multicolumn{4}{c}{3D Average Precision ($R40$)} \\
%    \cline{2-5}
%    & Easy & Mod. & Hard & mAP \\
%    \hline
%    PC-RGNN \cite{pcrgnn_zhang}& 90.94 & 81.43 & 80.45 &84.27\\
%    SIENet \cite{sienet_li}& 92.49 & 85.43 & 83.05 &86.99\\
%    \hline
%    PG-RCNN (Ours) &&&\\
%    \hline
%\end{tabular}
%\end{table}


%------------------------------------------------------------------------

\vspace{-1ex}
\section*{Acknowledgements}
This research was financially supported by the Institute of Civil Military Technology Cooperation funded by the Defense Acquisition Program Administration and Ministry of Trade, Industry and Energy of Korean government under grant No. 22-SN-AU-09


{\small
\bibliographystyle{ieee_fullname.bst}
\bibliography{egbib.bib}
}


\end{document}