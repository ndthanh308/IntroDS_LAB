

\documentclass[12pt,twoside]{amsart} 

\usepackage{amssymb,latexsym,bbm,graphicx,epsfig,epic,eepic,oldgerm,psfrag,comment}
\usepackage{amsmath,amsfonts,amscd,mathrsfs,amsthm}
\usepackage{mathtools}\DeclarePairedDelimiter{\ceil}{\lceil}{\rceil}
\usepackage{mathtools}\DeclarePairedDelimiter{\floor}{\lfloor}{\rfloor}
\usepackage{forest}

%\usepackage{calrsfs} 
\usepackage{tikz,wasysym}  \usetikzlibrary{trees}
%\usepackage{mathabx}
\usepackage{marvosym}                  
%\usepackage[pdftex]{hyperref} 
\usepackage{hyperref} 
%\usepackage{a4wide}
%\usepackage{showkeys}   
 
\usepackage{xypic} 
\input xy
\xyoption{all}             

                 
\theoremstyle{plain}
\newtheorem{theorem}{Theorem}[section]
\newtheorem{lemma}[theorem]{Lemma}                           
\newtheorem{proposition}[theorem]{Proposition}
\newtheorem{corollary}[theorem]{Corollary}
\newtheorem*{remark*}{Remark}
\newtheorem*{remarks*}{Remarks}
\newtheorem{remark}[theorem]{Remark}
\newtheorem{remarks}[theorem]{Remarks}
\newtheorem{claim}[theorem]{Claim}
\newtheorem{example}[theorem]{Example}
\newtheorem{examples}[theorem]{Examples}
\newtheorem*{example*}{Example}
\newtheorem*{examples*}{Examples}
\newtheorem{definition}[theorem]{Definition}
\newtheorem*{definition*}{Definition}
\newtheorem{observation}[theorem]{Observation}
\newtheorem{observations}[theorem]{Observations}
\newtheorem{question}[theorem]{Question}
\newtheorem{questions}[theorem]{Questions}
\newtheorem*{claim*}{Claim}
\newtheorem{conjecture}[theorem]{Conjecture}
\newtheorem{openproblem}[theorem]{Open Problem}



\numberwithin{figure}{section}
\numberwithin{equation}{section}

\newcommand{\proofend}{\hspace*{\fill} $\square$\\}
\newcommand{\diam}{\hspace*{\fill} $\Diamond$}
\newcommand{\Crr}{{{\mathcal C} \;\!\!{\it r}}}
%\newcommand{\labell}[1] {\label{#1}}
\newcommand{\nobarfrac}{\genfrac{}{}{0pt}{}}

\def\D{\operatorname{D}}
\def\ca{\operatorname{ca}}
\def\realpart{\operatorname{Re}}
\def\imagpart{\operatorname{Im}}
\def\th{\operatorname{\scriptsize th}}

\def\spec{\operatorname{spec}\1}
%\def\actionspectrum{\operatorname{action \; spectrum}\1}
\def\lengthspectrum{\operatorname{length-spec}\1}
\def\multispectrum{\operatorname{multi-spec}\1}
\def\multilengthspectrum{\operatorname{multi-length-spec}\1}
\def\SHspectrum{\operatorname{SH-spec}\1}
\def\ECHspectrum{\operatorname{{ECH-spec}}\1}
\def\ECC{\operatorname{ECC}} 
\def\ECH{\operatorname{ECH}}  
\def\GH{\operatorname{GH}}
\def\FH{\operatorname{FH}}
\def\Chek{\operatorname{Chek}}
\def\Cliff{\operatorname{Cliff}}
\def\Op{\operatorname{Op}}

\def\sECH{ {\scriptscriptstyle \operatorname{ECH}} }
\def\sEH{ {\scriptscriptstyle \operatorname{EH}} }
\def\sSH{ {\scriptscriptstyle \operatorname{SH}} }
\def\SW{\operatorname{SW}}
\def\Gr{\operatorname{Gr}}
\def\Emb{\operatorname{Emb}} 
\def\symp{\operatorname{symp}}
\def\Ch{\operatorname{Ch}}   
\def\graph{\operatorname{graph}}

\newcommand{\Bcirc}{\overset%
{\raisebox{-.3ex}[0ex][-.3ex]{\mbox{$\scriptscriptstyle\circ$}} \mskip-3mu}B}

\newcommand{\Dcirc}{\overset%
{\raisebox{-.3ex}[0ex][-.3ex]{\mbox{$\scriptscriptstyle\circ$}} \mskip1mu}\D}

\newcommand{\Ditcirc}{\overset%
{\raisebox{-.3ex}[0ex][-.3ex]{\mbox{$\scriptscriptstyle\circ$}} \mskip-5mu}D}


\newcommand{\Omegacirc}{\overset%
{\raisebox{-.3ex}[0ex][-.3ex]{\mbox{$\scriptscriptstyle\circ$}} \mskip1mu}\Omega}

\newcommand{\Omegancirc}{\overset%
{\raisebox{-.4ex}[-.0ex][-.3ex]{\mbox{$\!\!\!\2 \scriptscriptstyle\circ$}} \mskip1mu} {\Omega^{\;\!\! \raisebox{.2ex}{$\scriptstyle n$}}}}


\newcommand{\Sigmacirc}{\overset%
{\raisebox{-.3ex}[0ex][-.3ex]{\mbox{$\scriptscriptstyle\circ$}} \mskip1mu}\Sigma}

\newcommand{\Kcirc}{\overset%
{\raisebox{-.3ex}[0ex][-.3ex]{\mbox{$\scriptscriptstyle\circ$}} \mskip-3mu}K}
          % Höhe Zeilenabst                                     linkgscircshift

\newcommand{\Xcirc}{\overset%
{\raisebox{-.2ex}[0ex][-.3ex]{\mbox{$\scriptscriptstyle\circ$}} \mskip-3mu}X}


\def\P{\operatorname{P}}

\newcommand{\Pcirc}{\overset% 
{\raisebox{-.3ex}[0ex][-.3ex]{\mbox{$\scriptscriptstyle\circ$}} \mskip1mu}\P} 

\newcommand{\trcirc}{\overset%
{\raisebox{-.3ex}[0ex][-.3ex]{\mbox{$\scriptscriptstyle\circ$}} \mskip1mu}\tr}


\newcommand{\trcircn}{\overset%
{\raisebox{-.4ex}[0ex][-.3ex]{\mbox{$\!\! \scriptscriptstyle\circ$}} \mskip1mu} {\tr^{\! \raisebox{.2ex}{$\scriptstyle n$}}}}

\newcommand{\mama}{\overline \B\raisebox{0.8ex}{\scriptstyle 4}}
\newcommand{\nana}{\overline \B\raisebox{0.8ex}{\scriptstyle 2n}}


\newenvironment{brsm}{
  \bigl[ \begin{smallmatrix} }{
  \end{smallmatrix} \bigr]}
	
\newenvironment{brsmm}{
  \bigl( \begin{smallmatrix} }{
  \end{smallmatrix} \bigr)}	

\def\1{\:\!}
\def\2{\;\!}
\def\s{\smallskip}
\def\m{\medskip}
\def\eps{\varepsilon}
\def\Ker{\operatorname{ker}}
\def\im{\operatorname {im}}
\def\grad{\operatorname{grad}}
\def\Hess{\operatorname{Hess}}
\def\pt{\operatorname{pt}}
\def\Vol{\operatorname {Vol}}
\def\Volg{\operatorname {Vol}}
\def\Diff{\operatorname{Diff}}
\def\Diffcc{\operatorname{Diff^c}}
\def\Diffc0{\operatorname{Diff^c_0}}
\def\Symp{\operatorname{Symp}}
\def\Sympcc{\operatorname{Symp^c}}
\def\Sympc0{\operatorname{Symp^c_0}}
\def\Iso{\operatorname{Iso}}
\def\Int{\operatorname{Int}}
\def\Ham{\operatorname{Ham}}
\def\Hamc{\operatorname{Ham^c}}
\def\Conj{\operatorname{Conj}}
\def\rank{\operatorname{rank}}
\def\Trace{\operatorname{Trace}}
\def\length{\operatorname{length}}
\def\supp{\operatorname{supp}}
\def\idd{\operatorname{id}}
\def\tid{\tiny\operatorname{id}}
\def\ind{\operatorname{ind}}
\def\Crit{\operatorname{Crit}}
\def\var{\operatorname{var}}
\def\can{\operatorname{can}}
\def\Flux{\operatorname{Flux}}
\def\End{\operatorname{End}}
\def\width{\operatorname{width\,}}
\def\Conv{\operatorname{Conv}}
\def\loc{\operatorname{loc}}
\def\top{\operatorname{top}}
\def\h{\operatorname{h}}
\def\area{\operatorname{area}}
\def\garea{g_J\operatorname{-area}}
\def\vol{\operatorname{vol}}
\def\Cr{\operatorname{Cr}}
\def\PD{\operatorname{PD}}
\def\C{\operatorname{C}}
\def\H{\operatorname{H}}
\def\dia{\operatorname{diam}}
\def\const{\operatorname{const}}
\def\GL{\operatorname{GL}}
\def\U{\operatorname{U}}
\def\hom{\operatorname{hom}}
\def\SL{\operatorname{SL}}
\def\aa{\boldsymbol{a}}
\def\AA{\boldsymbol{A}}
\def\mm{\mathbf{m}}
\def\uu{\mathbf{u}}
\def\ww{\boldsymbol{w}}
\def\xx{\mathbf{x}}
\def\yy{\mathbf{y}}
\def\zz{\mathbf{z}}
\def\odd{\operatorname{odd}}
\def\CROSS{\operatorname{CROSS}}
\def\RS{\operatorname{RS}}
\def\CZ{\operatorname{CZ}}
\def\RFC{\operatorname{RFC}}
\def\RFH{\operatorname{RFH}}
\def\Morse{\operatorname{Morse}}
\def\ASM{\operatorname{ASM}}
\def\AS{\operatorname{AS}}
\def\AM{\operatorname{AM}}
\def\Sp{\operatorname{Sp}}
\def\Bl{\operatorname{Bl}}
\def\gec{\succcurlyeq}

\def\ga{\alpha}
\def\gb{\beta}
\def\gg{\gamma}
\def\gd{\delta}  
\def\bgd{\boldsymbol{\delta}}
\def\gve{\varepsilon}
\def\gf{\varphi}
\def\gk{\kappa}
\def\gkk{\varkappa}
\def\gl{\lambda}
\def\go{\omega}
\def\Go{\Omega}
\def\gs{\sigma}
\def\gt{\vartheta}
\def\gy{\upsilon}
\def\gv{\varrho} 
\def\gz{\zeta}

\def\gD{\Delta}
\def\gF{\Phi}
\def\gG{\Gamma}
\def\gL{\Lambda}
\def\gO{\Omega}
\def\gS{\Sigma}

\def\B{\operatorname{B}}
\def\sB{ {\scriptstyle \operatorname{B}} }
\def\sC{ {\scriptstyle \operatorname{C}} }
\def\sZ{ {\scriptstyle \operatorname{Z}} }
\def\C{\operatorname{C}}
\def\E{\operatorname{E}}
\def\M{\operatorname{M}}
%\def\hB{\operatorname{\widehat B}}
%\def\hE{\operatorname{\widehat E}}
\def\P{\operatorname{P}}
\def\T{\operatorname{T}}
\def\Z{\operatorname{Z}}

\def\EC{\scriptscriptstyle \operatorname{EC}}
\def\EB{\scriptscriptstyle \operatorname{EB}}
\def\EE{\scriptscriptstyle \operatorname{EE}}
\def\EP{\scriptscriptstyle \operatorname{EP}}
\def\PB{\scriptscriptstyle \operatorname{PB}}

%\def\ca{{\mathcal A}}
\def\cb{{\mathcal B}}
\def\cc{{\mathcal C}}
\def\cd{{\mathcal D}}
\def\ce{{\mathcal E}}
\def\cf{{\mathcal F}}
\def\cg{{\mathcal G}}
\def\ch{{\mathcal H}}
\def\ci{{\mathcal I}}
\def\cj{{\mathcal J}}
\def\ck{{\mathcal K}}
\def\cl{{\mathcal L}}
\def\cm{{\mathcal M}}
\def\cn{{\mathcal N}}
\def\co{{\mathcal O}}
\def\cp{{\mathcal P}}
\def\cq{{\mathcal Q}}
\def\cR{{\mathcal R}}
\def\cs{{\mathcal S}}
\def\ct{{\mathcal T}}
\def\cu{{\mathcal U}}
\def\cv{{\mathcal V}}
\def\cw{{\mathcal W}}

\def\bJ{{\mathbf J}}
\def\bg{{\mathbf g}}

\def\CC{\mathbb{C}}
\def\DD{\mathbb{D}}
\def\FF{\mathbb{F}}
\def\HH{\mathbb{H}}
\def\II{\mathbb{I}}
\def\NN{\mathbb{N}}
\def\PP{\mathbb{P}}
\def\QQ{\mathbb{Q}}
\def\RR{\mathbb{R}}
\def\SS{\mathbb{S}}
\def\TT{\mathbb{T}}
\def\ZZ{\mathbb{Z}}
\def\kk{\mathbb{k}}

\def\R{\operatorname{\mathbb{R}}}
\def\RP{\operatorname{\mathbb{R}P}}
\def\CP{\operatorname{\mathbb{C}P}}

\def\Log{\operatorname{Log}}


\def\pp{\partial}
\def\pr{{\rm pr}}
\def\gec{\succcurlyeq}
\def\fm{{\mathfrak m}}
\def\fp{{\mathfrak p}}
\def\sym{Sp(n;\RR)}

\def\ra{\rightarrow}
\def\ha{\hookrightarrow}
\def\Ra{\Rightarrow}
\def\Lra{\Leftrightarrow}

\newcommand{\hse}{\;{\stackrel{s\2}\hookrightarrow}\;}

\newcommand{\se}{\overset%
{\raisebox{-.2ex}[0ex][-.2ex]{\mbox{$\scriptstyle s$}} \mskip 1mu}\hookrightarrow}

\newcommand{\seq}{\;{\stackrel{s\2}=}\;}

\def\ni{\noindent}
\def\b{\bigskip}
\def\m{\medskip}

\def\id{\mbox{id}}
\def\de{\stackrel{\mbox{\scriptsize{def}}}{=}}

\def\sq{\square}
\def\tr{\triangle}
\def\proof{\noindent {\it Proof. \;}}


%%%%%%%%%%%%%%%%%%%%%%%%%%%%%%%%%%%%%%%%%%%%%%%%


\begin{document}




\title{\vspace*{0cm} On the agreement of symplectic capacities in high dimension}
\author{Dan Cristofaro-Gardiner and Richard Hind}

\date{\today}

\maketitle

\begin{abstract}

%Symplectic capacities are measurements of symplectic size.  
A theorem of Gutt-Hutchings-Ramos asserts that all normalized symplectic capacities give the same value for monotone four-dimensional toric domains.
% all normalized symplectic capacities give the same value.
% when the toric domain is monotone.  
We generalize this theorem to arbitrary dimension.  The new ingredient in our proof is the construction of symplectic embeddings of  
%To do this, we prove that 
 ``$L$-shaped" domains in any dimension into corresponding infinite cylinders; this resolves a conjecture of Gutt-Pereira-Ramos in the affirmative.

\end{abstract}

\section{Introduction}

Symplectic capacities are measurements of symplectic size: a {\em symplectic capacity} is a rule that associates to each symplectic manifold $(M,\omega)$ of a fixed dimension a nonnegative, possibly infinite, real number $c(M,\omega)$.  The assignation $(M,\omega) \to c(M,\omega)$ is subject to various axioms; perhaps the most important is the ``Monotonicity Axiom", which states that 
\[ c(M_1,\omega_1) \le c(M_2,\omega_2),\]
when $(M_1,\omega_1)$ can be symplectically embedded into $(M_2,\omega_2)$; for the other axioms, we refer the reader to \cite{chls}.  Any scalar multiple of a symplectic capacity is itself a symplectic capacity.  However, a symplectic capacity $c$ is said to be {\em normalized}  if
\[ c(B^{2n}(1)) = c(Z^{2n}(1)) = 1,\]
where $B^{2n}$ denotes the ball and $Z^{2n}$ denotes the infinite cylinder, both defined below.

While there are many symplectic capacities (see e.g. \cite{chls}), it is interesting to try to understand to what degree normalized symplectic capacities are unique.  For example, 
a conjecture due to Viterbo states that all normalized capacities give the same value on convex domains in $\mathbb{R}^{2n}$.  This is known to imply the famous Mahler conjecture \cite{vm}.   Another instance of this kind of phenomenon, which is the genesis of the current note, is the following theorem of Gutt-Hutchings-Ramos.  To set the notation,    let $\pi: \CC^n \to \R^n_{\ge 0}$ be given by $\pi(z_1, \dots , z_n) = (\pi|z_1|^2, \dots , \pi|z_n|^2)$. Then given a subset $\Omega \subset \R^n_{\ge 0}$ we can define the associated {\em toric domain} by
$$X_{\Omega} := \pi^{-1}(\Omega) \subset \CC^n.$$
The vector space $\CC^n$ has a standard symplectic structure $\omega = \frac{1}{2i} \sum dz_k \wedge d\overline{z_k}$ and so the $X_{\Omega}$ inherit a symplectic structure.
A toric domain $X_{\Omega}$ is called {\em monotone} if the outward normals at every point $\mu \in \partial \Omega \cap \R^n_{> 0}$ have nonnegative entries.  Many monotone toric domains are not convex, and many convex domains are not toric.  

%However, the following     

\begin{theorem}[\cite{ghr}]
Let $X_{\Omega}$ be a four-dimensional monotone toric domain.  Then any two normalized symplectic capacities $c_1$ and $c_2$ satisfy 
\[ c_1(X_{\Omega}) = c_2(X_{\Omega}).\]
\end{theorem}

%Many monotone toric domains are not convex, and many convex domains are not toric.    

Our main result extends this theorem to arbitrary dimension.

\begin{theorem}\label{agree}
All normalized symplectic capacities agree on monotone toric domains of any dimension.
%Let $X_{\Omega}$ be a monotone toric domain.  Then any two normalized symplectic capacities $c_1$ and $c_2$ satisfy 
%\[ c_1(X_{\Omega}) = c_2(X_{\Omega}).\]
\end{theorem}

A strictly monotone toric domain is one for which the outward normals at every point   $\mu \in \partial \Omega \cap \R^n_{\ge 0}$ have positive entries. It was shown in \cite{ghr}, Proposition 1.8, that strictly monotone toric domains are dynamically convex. The notion of dynamical convexity is invariant under symplectomorphisms, and all strictly convex domains are dynamically convex, see \cite{convex}, section 3. Abbondandolo, Bramham, Hryniewicz, and Salam\~{a}o \cite{abhs} have constructed 4 dimensional dynamically convex domains for which Viterbo's conjecture does not hold, see \cite{ghr} Remark 1.9, but we do not know if similar examples exist in higher dimension.
 
Let us now explain the key new ingredient in our proof, which is a theorem of potentially independent interest.
%To prove this, we prove another result of potentially independent interest.
%Our main result is a symplectic embedding from L-shaped domains into cylinders.
%To fix notation,
Given $r>0$ we define the ball and cylinder of capacity $r$ by
$$B(r) = X_{b(r)}, \qquad Z(r) = X_{z(r)}$$
respectively, where $b(r) = \{ x_k \ge 0 \, | \, \sum x_k < r\}$ and $z(r) = \{x_n < r\}$.
Also, given $a_1, \dots , a_n >0$, we define an {\em $L$-shaped domain} by
$$L(a_1, \dots , a_n) = X_{l(a_1, \dots , a_n)}$$
where $l(a_1, \dots , a_n) = \bigcup_{k=1}^n \{ x_k < a_k \}$.
A crucial step in the argument of Gutt-Hutchings-Ramos is to show that $L(a_1,a_2)$ can be symplectically embedded into $Z(a_1+ a_2)$.  This is proved by making use of \cite{cg}, a rather general embedding result proved using pseudoholomorphic curves and symplectic inflation.  As the techniques in \cite{cg} have no known analogue in higher dimensions, it is natural to wonder whether or not the corresponding embeddings of $L$-shaped domains in higher dimensions exist; our second result resolves this. 

%A slightly simplified version of this was conjectured in all dimensions by Gutt, Periera and Ramos in \cite{gpr}
  

%is natural to wonder whether or not   It is natural to ask

%interesting to see if one can extend this embedding to higher-dimensions

%As a result, the authors note that they do not know how to extend their result to higher dimensions.   We construct the needed symplectic embedding in arbitrary dimension through a much more ``hands-on" approach


%We will construct a symplectic embedding as follows.

\begin{theorem}\label{main} Let $r > \sum a_k$. Then there exists a symplectic embedding
$$L(a_1, \dots , a_n) \hookrightarrow Z(r).$$
\end{theorem}

%Thus, is was proven in the case $n=2$ by Gutt, Hutchings and Ramos \cite{ghr} and 
A slightly simplified version of this was conjectured in all dimensions by Gutt, Periera and Ramos in \cite[Conj. 15]{gpr}. 

\begin{remark} Combining the proof of Theorem \ref{main} with Theorem 4.3 in \cite{pvn}, it is actually possible to produce embeddings even in the case when $r = \sum a_k$. However the strict inequality is enough to derive our consequences for symplectic capacities.
\end{remark}


%The proof in \cite{ghr} used the methods of ECH however, which do not apply in higher dimension.


%The motivation in \cite{ghr} was to establish the equivalence of normailzed symplectic capacities for monotone toric domains, and Theorem \ref{main} implies that the same holds in higher dimension. 

Let us now recall why Theorem \ref{main} implies Theorem \ref{agree}.

%the argument. 
\begin{proof}  Given $X \subset \CC^n$ we define its Gromov width by $$c_G(X) = \sup \{ r>0 \, | \, B(r) \hookrightarrow X \}$$ and its cylindrical capacity by $$c_Z(X) = \inf \{ r>0 \, | \, X \hookrightarrow Z(r) \}.$$ The definition of normalized symplectic capacities implies that  $$c_G(X) \le c(X) \le c_Z(X)$$ for all such capacities $c$ and all $X \subset \CC^n$.

Thus, it suffices to show that  $c_G(X) = c_Z(X)$ for all monotone toric domains $X_{\Omega}$.

%It suffices to show that the Gromov width agrees with the cylindrical capacity.

Let $B(r)$ be the largest toric ball that embeds by inclusion.   Then there is a point $(a_1,\ldots,a_n) \in \partial \Omega$ with $a_1 + \ldots + a_n = r.$
It now suffices to show that $r$ is an upper bound on the cylindrical capacity. Because $\Omega$ is monotone, $X_{\Omega} \subset L(a_1,\ldots,a_n).$  By Theorem~\ref{main}, it follows that there is a symplectic embedding 
$X_{\Omega} \to Z(r)$
Thus, it follows that the cylindrical capacity is at most $r$, as desired.

%We have the following.

%\begin{corollary} If $X$ is a monotone toric domain then $c_G(X) = c_Z(X)$. Hence all normalized symplectic capacities coincide on monotone toric domains.
%\end{corollary}

%\begin{proof}

%[add this]

\proofend
\end{proof}

{\bf Acknowledgements.} We would like to thank Jean Gutt, Michael Hutchings and Vinicius Ramos for helpful discussions.
%, Felix Schlenk and Richard Schwartz for reading a preliminary version of the text and providing very helpful comments.   
DCG also thanks the National Science Foundation for their support under agreement DMS-2227372, and RH thanks the Simons Foundation for their support under grant no. 633715.  We also thank the Brin Mathematics Research Center at the University of Maryland for hosting a visit by RH, during which important conversations about this project occurred.

%A famous conjecture of Viterbo states that all normalized capacities coincide on convex domains. Many monotone toric domains are not convex, but also many convex domains are not toric.

\section{Construction of a symplectic embedding}

Our symplectic embedding is quite explicit, exploiting an idea from \cite[Sec. 2.1]{busehind11}.
In section \ref{two1}  we describe a general construction for embedding domains in $\CC^n$ which are invariant under the diagonal $S^1$ action. Specifically we reduce the construction to finding a 1 parameter family of embeddings in complex $(n-1)$ dimensional projective space. In section \ref{two2} we apply the construction to the domains $L(a_1, \dots , a_n)$ and $Z(r)$. This reduces Theorem \ref{main} to finding a family of Hamiltonian diffeomorphisms of $\CC P^{n-1}$, which we proceed to do.


\subsection{Symplectomorphisms of projective space and embeddings in $\CC^n$}\label{two1}


%$B(S) \subset \CC^n$ be the ball, and 
Let $p_S: \partial B(S) \to \CC P^{n-1}(S)$ be the symplectic reduction, that is, the quotient by the characteristic orbits. Hence $\CC P^{n-1}(S)$ is complex projective space and inherits a symplectic form which integrates to $S$ over complex lines.

Next let $H : \CC P^{n-1}(S) \to \RR$ and $\tilde{H}: \CC^n \to \RR$ be a smooth extension of $H \circ p_S : \partial B(S) \to \RR$. Denote by $\phi$ and $\tilde{\phi}$ the corresponding Hamiltonian diffeomorphisms of $\CC P^{n-1}(S)$ and $\CC^n$ respectively.

\begin{lemma} \label{lift}
If $z \in \partial B(S)$ we have $\tilde{\phi}(z) \in \partial B(S)$ and $p_S (\tilde{\phi}(z)) = \phi(p_S(z))$.
\end{lemma}

\begin{proof} As $\tilde{H}$ is constant along the characteristic orbits, the Hamiltonian vector field $X_{\tilde{H}}$ is tangent to $\partial B(S)$ and hence its flow preserves $\partial B(S)$, giving the first part of the statement. Also, if $z \in \partial B(S)$ we have $dp_S \, X_{\tilde{H}}(z) = X_H(p_S(z))$, and integrating gives the second part.
\proofend
\end{proof}

The main result of this section will be a parameterized version of Lemma \ref{lift}. Let $$p: \CC^n \setminus \{ 0 \} \to (0,\infty) \times \CC P^{n-1},$$ $$z \in \partial B(S) \mapsto (S, p_S(z)).$$
A family of functions $H_S : \CC P^{n-1} \to \R$ define a function $H: (0,\infty) \times \CC P^{n-1} \to \R$ by $(S, z) \mapsto H_S(z)$ which we always assume to be smooth. We will also assume there is a constant $c$ so that $H_S(z) = c$ whenever $S$ is small. Then $\tilde{H} = H \circ p$ extends smoothly to a function on $\CC^n$ with $\tilde{H}(0)=c$. It is invariant under the diagonal $S^1$ action generated by multiplication by the unit circle, that is, $\tilde{H}(e^{it}z) = \tilde{H}(z)$.



%Let $H_S$ be a smooth family of functions on $\CC P^{n-1}$ generating Hamiltonian flows on the symplectic manifolds $\CC P^{n-1}(S)$. Together these lift to a function $H$ on $\CC^n$ which is invariant under the $S^1$ action $z \mapsto e^{it} z$. 

As in the Lemma \ref{lift}, let $\phi_S$ be the Hamiltonian diffeomorphism of $\CC P^{n-1}(S)$ generated by $H_S$ and $\tilde{\phi}$ be the  Hamiltonian diffeomorphism of  $\CC^n$ generated by $\tilde{H}$.

Let $U, V \subset \CC^n$ be open sets with $V$ invariant under the $S^1$ action and with $0 \in V$, and let $U_S, V_S \subset \CC P^{n-1}(S)$ be the images of $U \cap \partial B(S)$ and $V \cap \partial B(S)$ under the projection maps $p_S : \partial B(S) \to \CC P^{n-1}(S)$. Then Lemma \ref{lift} implies the following.

\begin{corollary}\label{embed} Suppose $\phi_S(U_S) \subset V_S$ for all $S$. Then $\tilde{\phi}(U) \subset V$.
\end{corollary}

\begin{proof} We have $\tilde{\phi}(0) = 0 \in V$, so it suffices to check $\tilde{\phi}(z) \in V$ for $z \in U \setminus \{ 0 \}$. Suppose then that $z \in \partial B(S)$ with $S>0$. Then Lemma \ref{lift} says that
$$\tilde{\phi}(z) \in p^{-1}(S, \phi_S(p_S(z))) \subset p^{-1}(\{S \} \times  \phi_S(U_S)) $$ $$ \subset p^{-1} (\{S\} \times V_S) = \partial B(S) \cap V$$
as required, where the final equality follows from the $S^1$ invariance of $V$.

\end{proof}
\proofend


\subsection{An embedding of L-shaped domains}\label{two2}

In this subsection Corollary \ref{embed} is applied to prove Theorem \ref{main}, which is restated here for convenience.

\begin{proposition} Suppose $r > a_1 + \dots a_n$. Then there exists a symplectic embedding $L(a_1, \dots, a_n) \hookrightarrow Z(r)$.
\end{proposition}

\begin{proof}
{\bf Step 1.} In this step we translate the general sufficient conditions from section \ref{two1} into specific requirements for our embedding.

Following the notation from Corollary \ref{embed} we set $U = L(a_1, \dots, a_n)$ and $V = Z(r)$. These domains are invariant under our $S^1$ action, and in fact are toric, so for $S>0$ they project to toric domains $U_S, V_S \subset \CC P^{n-1}(S)$. We note that the $T^n$ action on $\CC^n$ descends to the $\CC P^{n-1}(S)$, but now has a $1$ dimensional kernel generated by the diagonal action $z \mapsto e^{it}z$. %We also note that if $S$ is sufficiently small then (as $U$ and $V$ are open neighborhoods of $0$) we have $U_S = V_S = \CC P^{n-1}(S)$.

We can use polar coordinates $R_i = \pi|z_i|^2$, $\theta_i \in \RR / \ZZ$ on $\CC^n$, so $\partial B(S) = \{ \sum R_i = S \}$. We then identify the symplectic reduction of $\partial B(S)$ with the toric manifold associated to the projection of this set to the $(R_1, \dots , R_{n-1})$ plane, a closed triangle $\Delta_S$ with vertices $(0, \dots, 0,S,0, \dots, 0)$. Under this identification, the $T^n$ fiber over $(R_1, \dots , R_n)$ in $\CC^n$ projects to the toric fiber over $(R_1, \dots , R_{n-1})$ via the map $$(\theta_1, \dots , \theta_n) \mapsto (\theta_1 - \theta_n, \dots , \theta_{n-1} - \theta_n).$$ 

Let $\mu : \CC P^{n-1}(S) \to \Delta_S$ be the moment projection.
%We can write the moment image of $\CC P^{n-1}(S)$ as $$\mu(\CC P^{n-1}(S)) = \{R_1 + \dots + R_{n-1} \le 1\}.$$ Then the moment projection of $U_S$ is the unionR
We have
$$\mu(U_S) = \bigcup_{i=1}^{n-1} \{R_i < a_i \} \bigcup \{ \sum_{i=1}^{n-1} R_i > S - a_n\}$$
and 
$$\mu(V_S) = \{ \sum_{i=1}^{n-1} R_i > S-r\}.$$

By Corollary \ref{embed}, to find a symplectic embedding $U \hookrightarrow V$ it suffices to find a smooth family of Hamiltonian functions $H_S$ on $\CC P^{n-1}(S)$ generating diffeomorphisms $\phi_S$ with  $\phi_S(U_S) \subset V_S$ for all $S$. The condition is vacuous when $S \le r$, when $V_S = \CC P^{n-1}(S)$, and so it suffices to construct $H_S$ for $S> r - \epsilon$, say, and then apply a bump function so that $H_S(z)=0$ whenever $S$ is small. 

Equivalently, by looking at the complements, it suffices to construct a family of Hamiltonian diffeomorphisms mapping $$\overline{B(S-r)} = \CC P^{n-1}(S) \setminus V_S \hookrightarrow \CC P^{n-1}(S) \setminus U_S = \{R_i \ge a_i \, \forall i , \, \sum R_i \le S - a_n\}.$$

{\bf Step 2.} In this step we show that Hamiltonian diffeomorphisms as in Step 1 do indeed exist, and we can even arrange the support to lie in the affine part $\{\sum_{i=1}^{n-1} R_i < S\} \subset \CC P^{n-1}$. We describe the diffeomorphisms for a fixed $S$, however it will be clear the generating functions can be chosen to depend smoothly on $S$ (after identifying our models of $\CC P^{n-1}(S)$ with the fixed underlying manifold).

We define $D(T) = \{ \pi|z|^2 \le T\}$ to be the closed disk in the $z$ plane, and $A(T_1, T_2) = \{ T_1 < \pi|z|^2 < T_2\}$ to be the open annulus.

Let $\phi_i$, $1 \le i \le n-1$, be Hamiltonian diffeomorphisms of the plane %with compact support in $D(S- \sum_{j \neq i} a_j)$ 
such that  $$\phi_i(D(S-r)) \subset A(a_i, S- \sum_{j \neq i} a_j).$$ %into the annulus $$A(a_i, S- \sum_{j \neq i} a_j) = \{a_i < \pi|z|^2 < S- \sum_{j \neq i} a_j \}.$$
We note that such Hamiltonians exist since $r > \sum a_i$ (implying that the disk has strictly smaller area than the annulus). Moreover we can arrange that a Hamiltonian flow $\phi_i^t$ with $\phi_i^1 = \phi_i$ satisfies $\pi |\phi^t_i(z)|^2 < \pi|z|^2 + ta_i + \epsilon$ for all $0 \le t \le 1$ and $\epsilon$ arbitrarily small.

Now consider the Hamiltonian flow of $\CC^{n-1}$ given by  $$\Phi^t: (z_1, \dots z_{n-1}) \mapsto (\phi^t_1(z_1), \dots \phi^t_{n-1}(z_{n-1})).$$
Our proof will follow from properties of $\Phi^t$.

{\bf Claim.} 
\begin{enumerate}
\item $\Phi^1(\overline{B(S-r)}) \subset \{\pi|z_i|^2 > a_i \, \forall i , \, \sum \pi|z_i|^2 < S - a_n\}$.
\item $\Phi^t(\overline{B(S-r)}) \subset \{\sum \pi|z_i|^2 < S - a_n \}$ for all $0 \le t \le 1$.
\end{enumerate}

The second part of the claim implies that $\Phi^t$ has a generating Hamiltonian function which can be cut off to have support inside $B(S)$. Hence there exists a Hamiltonian diffeomorphism of $\CC P^2(S)$ which acts on $\overline{B(S-r)}$ in the same way as $\Phi^1$. In particular the first part of the claim then implies there exists a Hamiltonian diffeomorphism of $\CC P^2(S)$ mapping $\overline{B(s-r)} = \CC P^2 \setminus V_S$ into $\CC P^2(S) \setminus U_S$ as required.

To justify the claim we suppose $(z_1, \dots z_{n-1}) \in \overline{B(S-r)}$.
Then we have $$\sum_{i=1}^{n-1} \pi |\phi^t_i (z_i)|^2 < \sum (\pi |z_i|^2 + ta_i + \epsilon) \le S - r + (n-1)\epsilon + t\sum a_i < S - a_n$$ when $\epsilon$ is sufficiently small. Here the first inequality follows from our assumptions on the $\phi^t_i$, the second holds since $\sum \pi|z_i|^2 \le S-r$ and the final innequality uses $\sum a_i < r$. This establishes statement (2). For the remainder of (1) we just recall that $\phi^1_i(z) = \phi_i(z) \subset \{ \pi |z|^2 > a_i\}$.

% Hence $\Phi$ gives a symplectic embedding from $B(S-r)$ into $B(S - a_n)$, and 
%as each $\pi |\phi_i(z_i)|^2 > a_i$ we see that $\Phi(B(S-r))$, the complement of the image of $V_S$, is contained in the complement of $U_S$.

%To conclude we observe that the symplectic embedding $\Phi$ can be realized by a Hamiltonian diffeomorphism generated by a function with support in $B(S)$, and hence is the restriction of a Hamiltonian diffeomorphism of $\CC P^{n-1}(S)$. For this, in affine space $\Phi$ is the time $1$ flow of
%$$\Phi^t(z_1, \dots ,z_{n-1}) = (\phi^t_1(z_1), \dots, \phi^t_{n-1} (z_{n-1})).$$
%Now, if $z \in B(S -r)$, then by the properties of the $\phi^t_i$ we have
%$$\sum \pi|\phi^t_i(z_i)|^2 < \sum (\pi|z_i|^2 + ta_i + \epsilon)$$  $$< S - r + (n-1)\epsilon + t\sum a_i < S - a_n + (n-1)\epsilon$$
%for $0 \le t \le 1$, and so $\Phi^t(z) \in B(S - a_n + (n-1)\epsilon)$. Hence a generating Hamiltonian can be cut off to have support inside $B(S)$ as required.

\proofend


\end{proof}





%References:

%\vspace{3 mm}

%[CG]: Symplectic embeddings from concave toric domains into convex ones

%[CCGHR]: Symplectic embeddings into concave toric domains

%[CGHR]: The asymptotics of ECH capacities

%[CGS]: Subleading asymptotics of ECH capacities

%[H1]: Quantitative embedded contact homology

%[H2]: ECH capacities and the Ruelle invariant

%[McD]: The Hofer conjecture on embedding symplectic ellipsoids

%\newline
%hi


\begin{thebibliography}{99}

\bibitem{abhs} A. Abbondandolo, B. Bramham, U. Hryniewicz and P. Salam\~{a}o, {\em Systolic ratio, index of closed orbits and convexity for tight contact forms on the three-sphere}, Compos. Math. 154 (2018), 2643--2680.


\bibitem{vm} S. Artstein-Avidan, R. Karasev and Y. Ostrover, {\em From symplectic measurements to the Mahler conjecture}, Duke Math. J. 163 (2014), 2003 - 2022.

%\bibitem{biran99} P. Biran, {\em A stability property of symplectic packing}, Invent. Math (1999)

\bibitem{busehind11} O. Buse and R. Hind, {\em Symplectic embeddings of ellipsoids in dimension greater than four}, Geom. Top. 15 (2011), 2091--2110.

%\bibitem{busehind13} O. Buse and R. Hind, {\em Ellipsoid embeddings and symplectic packing stability}, Comp. Math. (2013).

%\bibitem{busehindopshtein16} O. Buse, R. Hind and E. Opshtein, {\em Packing stability for symplectic $4$-manifolds}, Trans. AMS (2016).

%\bibitem{cgetal} D. Cristofaro-Gardiner, T. Holm, A. Mandini, and A. Pires, {\em On infinite staircases in toric symplectic four-manifolds}, arXiv preprint.
%Symplectic embeddings into toric
 


%\bibitem{ccghr} K. Choi, D. Cristofaro-Gardiner, M. Hutchings and V. Ramos, {\em Symplectic embeddings into concave toric domains}.

%\bibitem{cghr} D. Cristofaro-Gardiner, M. Hutchings and V. Ramos, {\em The asymptotics of ECH capacities}, Invent. Math

%\bibitem{cgs} D. Cristofaro-Gardiner and N. Savale, {\em Subleading asymptotics of ECH capacities}, Selecta

\bibitem{chls} K. Cieliebak, H. Hofer, J. Latschev and F. Schlenk, {\em Quantitative symplectic geometry}, Dynamics, ergodic theory, and geometry, 1-44 MSRI Publ. 54, (2007).

\bibitem{cg} D. Cristofaro-Gardiner, {\em Symplectic embeddings from concave toric domains into convex ones}, J. Diff. Geom. 112 (2019), 199--232.

\bibitem{ghr} J. Gutt, M. Hutchings and V. Ramos, {\em Examples around the strong Viterbo conjecture},
J. Fixed Point Theory Appl. 24 (2022).

\bibitem{gpr} J. Gutt, M. Pereira and V. Ramos, {\em Cube normalized symplectic capacities}, arXiv:2208.13666.

\bibitem{pvn} A. Pelayo and S. V\~{u} Ng\d{o}c, {\em Hofer's question on intermediate symplectic capacities},
Proc. Lond. Math. Soc. 110 (2015), 787--804.



%\bibitem{hales} T. Hales. {\em A proof of the Kepler conjecture}, (2005)

%\bibitem{h1} M. Hutchings, {\em Quantitative embedded contact homology}, JDG.

%\bibitem{h2} M. Hutchings, {\em ECH capacities and the Ruelle invariant}, JFPTA.

%\bibitem{mcd} D. McDuff, {\em The Hofer conjecture on embedding symplectic ellipsoids}, JDG.

%\bibitem{mp1994}  D. McDuff and L. Polterovich, {\em Symplectic packings and algebraic geometry}, Invent. Math.

%\bibitem{schlenk17} F. Schlenk, {\em Symplectic embedding problems old and new}, Survey at http://members.unine.ch/felix.schlenk/Daejeon18/Survey.embeddings.pdf

%\bibitem{Bangert} V. Bangert, \textit{On the lengths of closed geodesics on almost round spheres.} Math. Z. 191, 549--558 (1986)

%\bibitem{CDHR} V. Colin, P. Dehornoy, U. Hryniewicz, A. Rechtman, \textit{Generic properties of 3-dimensional Reeb flows: Birkhoff sections and entropy.} arXiv:2202.01506

%\bibitem{Hofer93} H. Hofer, \textit{Pseudoholomorphic curves in symplectizations with applications to the Weinstein conjecture in dimension three.} Invent. Math. 114, 515--563 (1993)

%\bibitem{props1} H. Hofer, K. Wysocki and E. Zehnder, \textit{Properties of pseudo-holomorphic curves in symplectisations. I. Asymptotics.} Ann. Inst. H. Poincar\'e C Anal. Non Lin\'eaire 13, 337--379 (1996)

%\bibitem{props2} H. Hofer, K. Wysocki and E. Zehnder, \textit{Properties of pseudo-holomorphic curves in symplectisations. II. Embedding controls and algebraic invariants.} Geom. Funct. Anal. 5, 270--328 (1995)

%\bibitem{props3} H. Hofer, K. Wysocki and E. Zehnder, \textit{Properties of pseudoholomorphic curves in symplectizations. III. Fredholm theory.} In: Topics in Nonlinear Analysis, Progr. Nonlinear Differential Equations Appl. 35, Birkhäuser, Basel, 381--475 (1999)

\bibitem{convex} H. Hofer, K. Wysocki and E. Zehnder, {\em The dynamics on three-dimensional strictly convex energy surfaces}, Ann. of Math. (2) 148 (1998), 197--289.

%\bibitem{fast} U. Hryniewicz, \textit{Fast finite-energy planes in symplectizations and applications.} Trans. Amer. Math. Soc. 364, 1859--1931 (2012) 

%\bibitem{elliptic} U. Hryniewicz, P. A. S. Salom\~ao, \textit{Elliptic bindings for dynamically convex Reeb flows on the real projective three-space.} Calc. Var. Partial Differential Equations 55, art. 43, 57 pp.
%(2016)

%\bibitem{HSW}
%U. Hryniewicz, P. A. S. Salom\~ao, K. Wysocki, \textit{Genus zero global surfaces of section for Reeb flows and a result of Birkhoff.} J. Eur. Math. Soc. (2022), published online first

%\bibitem{SiefringCPAM} R. Siefring, \textit{Relative asymptotic behavior of pseudoholomorphic half-cylinders.} Comm.Pure Appl. Math. 61, 1631--1684 (2008)

%\bibitem{SiefringMathAnn} R. Siefring, \textit{Finite-energy pseudoholomorphic planes with multiple asymptotic limits.} Math. Ann. 368, no. 1-2, 367--390 (2017)

\end{thebibliography}





\end{document}

