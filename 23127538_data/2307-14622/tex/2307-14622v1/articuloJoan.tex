
%534599965

\documentclass[12pt]{amsart}
%\documentclass{article}


\usepackage{graphicx}
\usepackage{amssymb}
\usepackage{amsthm}
\usepackage{graphics}
\usepackage{amsmath}
\usepackage{epsfig}
\usepackage{amsfonts}
\usepackage{amssymb}
\usepackage{graphicx}
\usepackage{geometry}

%\usepackage{subcaption}

%\usepackage[pagewise]{lineno}\linenumbers

\newtheorem{theorem}{Theorem}[section]
\newtheorem{lemma}[theorem]{Lemma}
\newtheorem{proposition}[theorem]{Proposition}
%\newtheorem{claim}[theorem]{Claim}

\theoremstyle{definition}
\newtheorem{definition}[theorem]{Definition}
\newtheorem{ejem}[theorem]{Example}
\newtheorem{ejer}[theorem]{Ejercicio}
\newtheorem{claim}[theorem]{Claim}
%\newtheorem{proof}{Proof}
\newtheorem{conjecture}[theorem]{Conjecture}
%\newtheorem{question}{theorem}{Question}

%\newtheorem{coro}[theorem]{Corolario}
%\newtheorem{propo}[theorem]{Proposici\'on}
%\newtheorem{teo}[theorem]{Teorema}

%\theoremstyle{remark}
%\newtheorem{remark}[theorem]{Observaci\'on}
\newcommand{\bd}{\partial}
\newcommand{\R}{\mathbb{R}}


%\usepackage{multirow}
%\usepackage{caption}


\begin{document}

\title{On the transient number of a knot}

\author{Mario Eudave-Mu\~noz}
\address{ \hskip-\parindent
Mario  Eudave-Mu\~noz\\
Instituto de Matem\'aticas\\
 Universidad Nacional Aut\'onoma de M\'exico\\ 
 MEXICO}
 \email{mario@matem.unam.mx}
 
 \author{Joan Carlos Segura-Aguilar}
\address{ \hskip-\parindent
Joan Carlos Segura-Aguilar\\
Universidad Nacional Aut\'onoma de M\'exico\\ 
 MEXICO}
 \email{joancarlos28@gmail.com}


\keywords{knot, transient number, unknotting number, tunnel number, double branched covers}

\subjclass{57K10, 57M12}


\maketitle

\begin{abstract}
The transient number of a knot $K$, denoted $tr(K)$, is the minimal number of simple arcs that have to be attached to
$K$, in order that $K$ can be homotoped to a trivial knot in a regular neighborhood of the union of $K$ and the arcs. 
We give a lower bound for $tr(K)$ in terms of the rank of the first homology group of the double branched cover of $K$.
In particular, if $t(K)=1$, then the first homology group of the double branched cover of $K$ is cyclic. Using this,
we can calculate the transient number of many knots in the tables and show that there are knots
with arbitrarily large transient number. 
\end{abstract}


\section{Introduction}.

Let $K$ be a knot in the 3-sphere and let $M$ be a submanifold of $S^3$ containing $K$. We say that $K$ is
transient in $M$ if $K$ can be homotoped within $M$ to the trivial knot in $S^3$; otherwise $K$ is called
persistent. For example, $K$ is persistent in a regular neighborhood $\mathcal{N}(K)$ of $K$, but it is transient in a 3-ball
$B$ containing $K$. Yuya Koda and Makoto Ozawa \cite{KO} proved that every knot is transient in a submanifold $M$ if 
and only if $M$ is unknotted, that is, its complement in $S^3$ is a union of handlebodies. Then Koda and Ozawa \cite{KO} 
introduced a new invariant of knots, called the transient number of $K$, which somehow measures, starting with 
$\mathcal{N}(K)$, how large must be a submanifold in which $K$ is transient. 

The transient number is defined as follows: given a knot $K$ in $S^3$, 
there is a collection of arcs $\{ \tau_1, \tau_2, \dots, \tau_n\}$, disjointly embedded in $S^3$, each $\tau_i$ 
intersecting $K$ exactly at its endpoints, such that $K$ can be homotoped in a regular neighborhood of $K$ union the arcs, 
$T=\mathcal{N}(K\cup \tau_1 \cup \dots \cup \tau_n)$, into the trivial knot.  That is, we perform crossing changes and
isotopies inside $T$, until we get the trivial knot $K'$. Note that any knot $K'$ obtained from $K$ in this way is not 
trivial in $T$, i.e. it cannot bound a disk contained in $T$, but it can be trivial in $S^3$. The transient number of $K$,
$tr(K)$, is then defined as the minimal number of arcs needed in such a system of arcs. The transient number is 
related to other knot invariants, namely $tr(K)\leq u(K)$, where $u(K)$ is the unknotting number and $tr(K)\leq t(K)$,
where $t(K)$ is the tunnel  number. It is easy to check these inequalities. For the unknotting number, given a sequence
of crossing changes that unknot $K$, consider for each crossing change an arc with endpoints in $K$ that guides
the crossing change, such that a regular neighborhood of the arc encapsulates the crossing change, then clearly 
$K$ can be made trivial in a neighborhood $T$ of $K$ union the arcs. 
For the case of tunnel number, consider a tunnel system and a
neighborhood $T$ of the union of $K$ and the arcs, such that the exterior is a handlebody. Isotope $T$ such that it looks 
like a standard handlebody in $S^3$. Then $K$ can be projected to the intersection of a plane with $T$, and
guided by this projection to a plane,
crossing changes can be performed to $K$ inside $T$ to get the trivial knot.

However a knot $K$ can have $tr(K)=1$, but $u(K)$ and $t(k)$ can be larger than one. Some examples with this
property are given in \cite{KO}. However, in that paper no example is given of a knot $K$ with $tr(K)>1$.
Homology groups of branched covers have been used to bound invariants like $u(K)$ and $t(K)$, which goes back
to the work of Wendt \cite{W}. In fact, it is well known that if $\Sigma[K]$ denotes the double branched cover of $K$, then
the rank of the group $H_1(\Sigma[K])$ gives a lower bound for $u(K)$, see \cite{W} or \cite{M}. It is also not
difficult to show that the rank of $H_1(\Sigma[K])$ is at most $2t(K)+1$; in particular it is known that if $t(K)=1$ then 
$H_1(\Sigma[K])$ is a cyclic group (though not explicitly stated, this follows from the computations
of homology of cyclic covers done in \cite{G}, or from \cite{Paz}). 

In this paper we prove that the rank of the first homology group of a branched cover of a knot give lower bounds for the
transient number. By using the Montesinos trick, it can be shown that if $K$ is a knot with $u(K)=n$, then $\Sigma[K]$
can be obtained by Dehn surgery on an $n$-component link in $S^3$, which implies then the bound for $u(K)$. 
In this paper we do a kind of generalized Montesinos trick. Our main results are the following.

\begin{theorem}\label{icubiertageneral} If $K$ is a knot in $S^3$ such that $tr(K) = n$, then the first homology group of the double branched cover of $K$ has a presentation with at most $2n + 1$ generators.
\end{theorem}

%By using the same ideas, the following generalization can be proved.

\begin{theorem}\label{icubiertageneral-p} If $K$ is a knot in $S^3$ such that $tr(K) = n$, then the first homology group of the $p$-fold branched cover of $K$ has a presentation with at most $pn + 1$ generators.
\end{theorem}

These results imply that $rank(H_1(\Sigma[K])\leq 2tr(K)+1$. Also, if 
$\Sigma_p[K]$ denotes the $p$-fold branched cover of $K$, it follows that $rank(H_1(\Sigma_p[K])\leq ptr(K)+1$. 

For the case that $tr(K)=1$, we can get a better bound. In fact,
by doing a careful calculation of the first homology group of $\Sigma[K]$, we get the following result.

\begin{theorem} \label{icubiertatransito} If $K$ is a knot in $S^3$ such that $tr(K) = 1$, then the first homology group of the double branched cover of $K$ is cyclic. \end{theorem}

Of course, these results may not be sharp. It would be interesting to find sharp bounds for these inequalities.
It would also be interesting to find bounds for the transient number depending on other classical invariants of knots.

Given any knot invariant, it is always interesting to study its behavior under connected sums of knots. We have the
following:

\begin{theorem} Let $K_1$, $K_2$ be knots in $S^3$. Then $tr (K_1\# K_2) \leq tr (K_1)+tr (K_2) + 1$. \end{theorem}

The paper is organized as follows. In Section \ref{branchedcovers} we sketch a proof that the unknotting number and 
tunnel number are bounded below by the rank of the first homology group of 
double branched covers. Then prove the main results. As part of the proofs, we show also that if $t(K)=1$,
then $H_1(\Sigma[K])$ is cyclic; this is used in the proof of Theorem \ref{icubiertatransito}. In Section \ref{examples}
we give examples of knots with large transient number and explore the transient number of knots in the tables ok KnotInfo
\cite{Knot}. In Section \ref{connectedsums} we consider the transient number of a 
connected sum of knots,  prove some facts and propose some problems. 

Through the paper we work in the piecewise linear category. To avoid cumbersome notation we 
use expressions like the double branched cover of a knot to mean the double cover of $S^3$ branched along the knot. If
$\Lambda$ is a simple closed curve in the boundary of a 3-manifold $M$, we say adding a 2-handle along $\Lambda$,
to mean that we attach a 2-handle $D^2\times I$ to $M$, such that $\partial D^2 \times I$ is identified with a regular
neighborhood of $\Lambda$ in $\partial M$, which is an annulus. Also, if $M$ and $T$ are compact 3-manifods,
with $T\subset M$, then by $M\backslash T$ we mean $M$ minus the interior of $T$, or well the closure in $M$ of $M-T$.
If $X$ is a topological space, $\vert X \vert$ denotes its number of components. 

\section{Transient number and double branched covers}\label{branchedcovers}


This section is inspired by an idea that is used to build the double branched cover of a knot with unknotting number 
equal to one. Consider a knot $K$ in $S^3$ with unknotting number equal to one. Let $\alpha$ be an arc embedded 
in $S^3$, with endpoints in $K$, such that a regular neighborhood of it encapsulates the crossing change. 
So there is an homotopy in $\mathcal{N}(K \cup \alpha)$ 
between the knot $K$ and the trivial knot, which is denoted by $K'$. Clearly this homotopy can be taken so that it is 
constant in $\mathcal{N}(K) \backslash\, \mathcal{N}(\alpha)$ and that the changes are occurring only in 
$\mathcal{N}(\alpha)$; so we assume that $K'$ is obtained from $K$ just by taking the
 two arcs $K \cap \mathcal{N}(\alpha)$ and passing one arc through the other, which would correspond to a crossing 
change in the corresponding knot diagram. Due to the above we have that 
$K\cap (S^3 \backslash \mathcal{N}(\alpha))=K' \cap (S^3 \backslash \mathcal{N}(\alpha))$.

Let $\Sigma (K')$ be the double branched cover of the knot $K'$, with covering
function given by $p : \Sigma(K') \rightarrow S^3$. Now, since $K'$ is the trivial knot, $\Sigma(K')$ is 
homeomorphic to $S^3$. We know that $\mathcal{N}(\alpha)$ is a 3-ball intersecting $K'$ in two arcs, therefore 
$p^{-1}(\mathcal{N}(\alpha))$ is a solid torus, and $p^{-1}(\partial \mathcal{N}(\alpha))$ is a surface of 
genus one. Therefore, $S^3 \backslash p^{-1}(\mathcal{N}(\alpha))$ is a double cover of 
$S^3 \backslash\, \mathcal{N}(\alpha)$ branched along $K \cap (S^3 \backslash \, \mathcal{N}(\alpha))$. So to finish building the double branched cover of the knot $K$, 
all we have to do is to refill $S^3 \backslash \, p^{-1}(\mathcal{N}(\alpha))$ appropriately.

Note that there exists a compressing disk for $\partial (\mathcal{N}(\alpha)) \backslash K$ contained in 
$\mathcal{N}(\alpha) \backslash K$; we denote this disk by $D$ (see Figure 12). 
As $K \cap D = \emptyset$ then $\vert K' \cap D\vert$ is an even number, so the curve
$\partial D$ is lifted by $p$ into two curves in $p^{-1}(\partial \mathcal{N}(\alpha))$; we
denote these curves by $\Lambda_1$ and $\Lambda_2$. Let $\Sigma'$ be the 3-manifold obtained by
adding two 2-handles to the 3-manifold $S^3 \backslash \, p^{-1}(\mathcal{N}(\alpha))$, attached along 
the curves $\Lambda_1$ and $\Lambda_2$; we denote these 
2-handles by $\overline{\Lambda}_1$ and $\overline{\Lambda}_2$ respectively. So 
$\Sigma' =[S^3 \backslash p^{-1}(\mathcal{N}(\alpha))]\cup[\overline{\Lambda}_1 \cup \overline{\Lambda}_2]$.



We know that $\Lambda_1 \cup \Lambda_2$ is a double cover of $\partial D$ with covering function given
by $p \vert_{\Lambda_1 \cup \Lambda_2}$. So we can extend the function $p \vert_{\Lambda_1 \cup \Lambda_2}$
to $\overline{\Lambda}_ 1\cup \overline{\Lambda}_2$, to get that $\overline{\Lambda}_1 \cup \overline{\Lambda}_2$ is a double cover of $\mathcal{N}(D)$. From this follows that $\Sigma'$ is a double cover of 
$[S^3 \backslash p^{-1}(\mathcal{N}(\alpha))]\cup \mathcal{N}(D)$ branched along two arcs of $K$.


We have that $\partial ([S^3\backslash \mathcal{N}(\alpha)]\cup [\mathcal{N}(D)])$ consists of two 2-spheres and 
$\partial \Sigma'$ also consists of two 2-spheres. Also, the 2-spheres of $\partial \Sigma'$ are a double cover of the 
two spheres of $\partial ([S^3\backslash \mathcal{N}(\alpha)]\cup \mathcal{N}(D)]$ branched over 
the points $K \cap \partial([S^3\backslash \mathcal{N}(\alpha)]\cup \mathcal{N}(D))$. 

Now we can fill the sphere boundary components of $\Sigma'$ with 3-balls, and extend the function $p$ to these 3-balls in order to get the double covering of $S^3$ branched along the knot $K$.

The idea described above is known as the Montesinos trick. Similar to the previous construction, 
we will build the double branched covers of knots for which we 
know the tunnel number or the transient number. 
For the case of tunnel number, note that
if $K$ has tunnel number $n$, then $K$ is contained in a genus $(n+1)$-handlebody $V$, such that its
complement is another genus $(n+1)$-handlebody $W$. By taking $\Sigma[K]$, $V$ and $W$ lift to genus
$(2n+1)$-handlebodies, that is, give a genus $2n+1$ Heegaard decomposition of $\Sigma[K]$. This shows
that $H_1(\Sigma[K])$ is an abelian group of rank at most $2n+1$.



The following lemma is a general result of coverings which we will use often. The proof is a standard argument, we omit it.

\begin{lemma} \label{conexo} Let $M$ be a given 3-manifold. Let $\Sigma$ be a double cover of $M$ with covering function $p : \Sigma \rightarrow M$; and let $C \subset M$. If $M$ is path connected and $p^{-1}(C)$ is connected 
then $\Sigma$ is connected.
\end{lemma}



The following theorem is our first important result of this section. We will see that if we know the transient number of a knot 
we can construct the double branched cover of this knot and from there calculate its first homology group.

\begin{theorem}\label{cubiertageneral} If $K$ is a knot in $S^3$ such that $tr(K) = n$, then the first homology group of the double branched cover of $K$ has a presentation with at most $2n + 1$ generators.
\end{theorem}

\begin{proof} Let $K$ be a knot in $S^3$ such that $tr(K) = n$, let $\{ \tau_1,\tau_2,...,\tau_n\}$ be a transient system for $K$, and let $T =\mathcal{N}(K\cup \tau_1 \cup \tau_2 \cup, \dots, \cup \tau_n)$, this is a genus $n+1$ handlebody. Let 
$K' \subset T$ be the trivial knot, such that $K'$ is homotopic to $K$ in $T$.

Let us define a family of compressing disks for $\partial T$ properly embedded in $T$, say 
$\{D_1, D_2, \dots , D_n, D_{n+1}\}$, which satisfy the following properties (see Figure 13):
\begin{enumerate}

\item For each $i \in \{1, 2, . . . , n\}$ the disk $D_i$ is properly embedded in $\mathcal{N}(\tau_i)$.
\item The disk $D_{n+1}$ is properly embedded in $\mathcal{N}(K)$ and is a compression disk for it.
\end{enumerate}

All of these disks are properly embedded in $T$, so we can deduce that:
\begin{enumerate}
\item The family $\{D_1, D_2, \dots, D_n, D_{n+1}\}$ is pairwise disjoint. 
\item For each $i \in \{1,2,\dots,n\}$, $\vert D_i \cup K \vert = 0$.
\item $\vert D_{n+1} \cap  K \vert = 1$.
\end{enumerate}

Let $\Sigma[K']$ be the double branched cover of $K'$ with covering function given by
$p : \Sigma[K'] \rightarrow S^3$. Note that $\Sigma[K']$ is homeomorphic to $S^3$.

\begin{claim}\label{disjointcurves} For each $i \in \{1, 2, \dots, n\}$, $p^{-1}(\partial D_i)$ has exactly two connected
components, where each connected component is a simple closed curve
in $p^{-1}(\partial T)$; whereas $p^{-1}(\partial D_{n+1})$ is a single simple closed curve in
$p^{-1}(\partial T)$. Also, all these curves are disjoint in $\partial T$. \end{claim}

\begin{proof} We know that $\vert D_{n+1} \cap K \vert = 1$ and $\vert D_i \cap K \vert = 0$ for all 
$i \in \{1,2,\dots,n\}$. As $K'$ is homotopic to $K$ in $T$, then $\vert D_{n+1}\cap K'\vert$ is an odd integer and 
$\vert D_i \cap K'\vert$ is an even integer for all $i \in \{1,2,\dots,n\}$. 
Therefore, for each $i \in \{1,2,...,n\}$ we have that $p^{-1}(\partial D_i)$ has exactly two connected
components in $p^{-1}(\partial T)$, where each connected component is a simple closed curve; and
$p^{-1}(\partial D_{n+1})$ is a simple closed connected curve in $p^{-1}(\partial T)$.
Now, since the disks of the family $\{D_1, D_2, \dots, D_{n+1}\}$ are pairwise disjoint, 
we have that all the curves are pairwise disjoint. \end{proof}

\begin{claim}\label{genus} $p^{-1}(\partial T)$ is a connected, orientable surface with Euler characteristic $-4n$ 
(and genus $2n + 1$) contained in $\Sigma[K']$. \end{claim}

\begin{proof} Note that $\partial T$ is a genus $n+1$ surface, then $\chi(\partial T) = -2n$, and therefore 
$\chi (p^{-1}(\partial T))) = 2\chi(\partial T ) = -4n$.
Since $\partial T$ is connected, $p^{-1}(\partial T)$ is a double cover of 
$\partial T$,  $\partial D_{n+1} \subset \partial T$ and $p^{-1}(\partial D_{n+1})$ is a connected curve on 
$p^{-1}(\partial T)$, then by Lemma \ref{conexo} we have that $p^{-1}(\partial T)$ is connected. Therefore 
$p^{-1}(\partial T)$ is a connected orientable surface of Euler characteristic $-4n$ (and of genus $2n + 1$).\end{proof}


\begin{claim}\label{otroconexo} $p^{-1}(\partial T \backslash \cup ^{n}_{j=1} \partial D_j)$ is connected. \end{claim}

\begin{proof} Clearly $\partial  T \backslash \cup^{n}_{j=1}\partial D_j$ is connected. We have that 
$p^{-1}(\partial T  \backslash \cup ^{n}_{j=1} \partial D_j)$ is a double cover of 
$\partial T \backslash \cup^{n}_{j=1}\partial D_j$, 
 that $\partial D_{n+1} \subset \partial T \backslash \cup ^{n}_{j=1} \partial D_j$ and that
$p^{-1}(\partial D_{n+1})$ is a connected curve on $p^{-1}(\partial T \backslash \cup ^{n}_{j=1} \partial D_j$), then using Lemma \ref{conexo} we have that $p^{-1}(\partial T \backslash \cup^{n}_{j=1}\partial D_j)$ is connected. \end{proof}



By Claim \ref{disjointcurves} we know that for each $i \in \{1,2,\dots,n\}$ the curve
$\partial D_i$ lifts, under $p$, to exactly two simple closed curves in $p^{-1}(\partial T)$. 
Let us denote by $\Lambda^{i}_1$
and $\Lambda^{i}_2$ the two liftings of $\partial D_i$ in $p^{-1}(\partial T)$, so 
$\{\Lambda^1_1,\Lambda^1_2,\Lambda^2_1,\Lambda^2_2,...,\Lambda^n_1,\Lambda^n_2\}$ is a pairwise disjoint 
collection of simple closed curves in $p^{-1}(\partial T)$. Also, $\Lambda^i_1 \cup \Lambda^i_2$ is a double cover of 
$\partial D_i$ with $p\vert_{\Lambda^i_1\cup \Lambda^i_2}$ the corresponding covering function, then the functions
$p\vert_{\Lambda^i_1}: \Lambda^i_1 \rightarrow \partial D_i$ and $p \vert_{\Lambda^i_2} : \Lambda^i_2 \rightarrow \partial D_i$ are homeomorphisms.

By Claim \ref{disjointcurves} we have that $p^{-1}(D_{n+1})$ is a simple closed curve on $p^{-1}(\partial T)$. 
Let us denote by $\Lambda$ the curve $p^{-1}(\partial D_{n+1})$. So $\Lambda$ is a double cover for $\partial D_{n+1}$ with covering function 
$p \vert_\Lambda : \Lambda \rightarrow \partial D_{n+1}$.



Let us introduce the following notations: 

\begin{itemize}

\item  $Ext(T):=S^3 \backslash T$,
\item $\Sigma [Ext(T)] := \Sigma [K'] \backslash p^{-1}(T)$,
\end{itemize}



Note that $\Sigma[Ext(T)]$ is a double cover of $Ext(T)$. 
Note also that $\partial \Sigma [Ext(T )] = p^{-1}(\partial T)$.



Let $\Sigma [Ext(K)]$ be the 3-manifold obtained from $\Sigma [Ext(T)]$ by adding a 2-handle
along each of the members of the family of curves 
$\{\Lambda^1_1, \Lambda^1_2, \Lambda^2_1, \Lambda^2_2, \dots, \Lambda^n_1, \Lambda^n_2\}$. Since the functions 
$p \vert_{\Lambda^i_r}$ are homeomorphisms for each $i \in \{1,2,\dots,n\}$ and $r \in \{1,2\}$, we can extend these 
homeomorphisms to homeomorphisms whose domains are discs whose boundaries are $\Lambda^i_r$, which map
to the disks $D_i$. We then extend these last homeomorphisms to homeomorphisms from the 2-handle added
along $\Lambda^i_r$ to $\mathcal{N}(D_i)$. With this we conclude that $\Sigma[Ext(K)]$ is a double 
cover of $Ext(T) \cup (\cup ^n_{j=1} \mathcal{N}(D_j))$. Recall that the family of disks $\{D_1, D_2, \dots, D_n\}$ was chosen 
such that $Ext(T ) \cup (\cup ^n_{j=1} \mathcal{N}(Dj ))$ is homeomorphic to $Ext(K)$. Therefore $\Sigma [Ext(K)]$ 
is a double cover of $Ext(K)$.

On the other hand, from Claim \ref{genus} we know that $p^{-1}(\partial T)$ is an orientable connected surface of 
genus $2n + 1$ and by Claim \ref{otroconexo} we know that $p^{-1}(\partial T\backslash \cup^n_{j=1} \partial D_j)$ is 
connected. Since $\{\Lambda^i_1,\Lambda^1_2,\Lambda^2_1,\Lambda^2_2,...,\Lambda^n_1,\Lambda^n_2\}$ 
consist of $2n$ curves and

$$p^{-1}(\partial T\backslash \cup ^n_{j=1} \partial D) = p^{-1}(\partial T)\backslash \cup_{i\in \{1,2,...,n\}  r\in \{1,2\}} \Lambda ^i_r ,$$

\noindent then $\partial\Sigma[Ext(K)]$ is an orientable surface of genus one.

Now, note that $\partial D_{n+1} \subset \partial Ext(K)$ since $\partial D_{n+1} \subset \partial \mathcal{N}(K)$ and 
$D_{n+1} \cap  D_i = \emptyset$ for all $i \in \{1,2,...,n\}$. Therefore we also have $\Lambda \subset \partial\Sigma[Ext(K)]$.

Let us define the 3-manifold $\Sigma[K]$ obtained from $\Sigma[Ext(K)]$ by adding a 2-handle along $\Lambda$ 
on $\partial \Sigma[Ext(K)]$, and then complete with a 3-ball so that $\Sigma[K]$ is a closed 3-manifold. 
Since $p\vert_\Lambda$ is a two-to-one covering function then we can extend this function to a function 
that goes from a disk, whose boundary is $\Lambda$, to the disk $D_{n+1}$, where this extension is two-to-one 
branched at the point $K \cap D_{n+1}$. 
This last function is then extended to a function that goes from the 2-handle added along
$\Lambda$ to 
$\mathcal{N}(D_{n+1})$, where this function is two to one branched along the arc $K \cap \mathcal{N}(D_{n+1})$. 
Finally, this last function is 
extended to the added 3-ball, thus obtaining a function that goes from $\Sigma[K]$ to $S^3$ which is two to one branched along the knot $K$. From the above we conclude that $\Sigma[K]$ is the double branched cover of $K$.


Now we know from Claim \ref{genus} that $p^{-1}(\partial T)$ is an orientable connected surface of genus $2n + 1$ contained in 
$S^3$. Since $\partial \Sigma[Ext(T )] = p^{-1} (\partial T)$ and $\Sigma[Ext(T )] \subset \Sigma[K'] = S^3$ then 
$H_1(\Sigma[Ext(T)])$ is a free abelian group of rank $2n+1$. So, let 
$H_1(\Sigma[Ext(T)])=<\theta_1,\theta_2,...,\theta_{2n+1}>$,
where $\theta_i$ for $i \in \{1,2,...,2n + 1\}$ are generators.




Thus, $H_1(\Sigma[K]) =< \theta_1,\theta_2,...,\theta_{2n+1} \, \vert \, \lambda_1^1,\lambda_1^2,\lambda_2^1, \lambda_2^2,...,\lambda_n^1,\lambda_n^2,\lambda >$,
where $\lambda$ and the $\lambda^j_r$, for $j \in \{1,2,...,n\}$ and $r \in \{1,2\}$, correspond to the homology classes in 
$H_1(\Sigma[Ext(T)])$ of the respective curves $ \Lambda$ and $\Lambda_r^j$.
\end{proof}


It should be noted that in the proof of Theorem \ref{cubiertageneral}, besides from proving
the result, we construct the double cover of $S^3$ branched along the knot for which we know the transient number.
This construction will continue to be repeated throughout this work. Theorem \ref{cubiertageneral} can be generalized to  $p$-fold branched covers, with a similar proof.

\begin{theorem}\label{cubiertageneral-p} If $K$ is a knot in $S^3$ such that $tr(K) = n$, then the first homology group of the $p$-fold branched cover of  $K$ has a presentation with at most $pn + 1$ generators.
\end{theorem}


The next lemma is a general result of algebra of groups, which we will use for the proof of Theorems \ref{cubiertatunel} 
and \ref{cubiertatransito}.

\begin{lemma}\label{grupos} Let $G_1$ and $G_2$ be abelian groups such that

$G_1 =< \theta_1,\theta_2,\theta_3 : \lambda_1,\lambda_2,\lambda_3 >$ and $G_2 =< \beta_1,\beta_2 : \delta_1,\delta_2 >$.

Let $\Psi :< \theta_1,\theta_2,\theta_3 >\rightarrow < \theta_1,\theta_2,\theta_3 >$ and 
$\Phi :< \theta_1,\theta_2,\theta_3 >\rightarrow < \beta_1,\beta_2 >$ be homomorphisms between free abelian groups such that:

$$\begin{array}{ccccc}
\Psi(\theta_1) = \theta_2 & \Psi(\lambda_1) = \lambda_2 & \vert & \Phi(\theta_1) = \beta_1 & \Phi(\lambda_1) = \delta_1 \\
 \Psi(\theta_2) = \theta_1 & \Psi(\lambda_2) = \lambda_1 & \vert & \Phi(\theta_2) = \beta_1 & \Phi(\lambda_2) = \delta_1 \\
 \Psi(\theta_3) = \theta_3  &  \Psi(\lambda_3) = \lambda_3 & \vert &  \Phi(\theta_3) = 2\beta_2 & \Phi(\lambda_3) = 2\delta_2
 \end{array}$$
 
 
 
If $\lambda_1 = x\theta_1 + y\theta_2 + z\theta_3$ and $G_2$ is the trivial group, then $G_1$ is isomorphic to $Z_{x-y}$.
\end{lemma}

\begin{proof} Let $a_{ij}$ be integers, with $i, j \in \{1, 2, 3\}$, such that: 
\begin{equation} \label{sistem1}
\begin{split}
\lambda_1 = a_{11}\theta_1 + a_{12}\theta_2 + a_{13}\theta_3 \\
\lambda_2 = a_{21}\theta_1 + a_{22}\theta_2 + a_{23}\theta_3  \\
\lambda_3 = a_{31}\theta_1 + a_{32}\theta_2 + a_{33}\theta_3
\end{split}
\end{equation}

Applying the homomorphism $\Psi$, on both sides of the previous system of equations, we obtain:

\begin{equation} \label{sistem2}
\begin{split}
\lambda_2 = \Psi(\lambda_1) = \Psi(a_{11}\theta_1 + a_{12}\theta_2 + a_{13}\theta_3) = a_{11}\theta_2 + a_{12}\theta_1 + a_{13}\theta_3 \\
\lambda_1 = \Psi(\lambda_2) = \Psi(a_{21}\theta_1 + a_{22}\theta_2 + a_{23}\theta_3) = a_{21}\theta_2 + a_{22}\theta_1 + a_{23}\theta_3 \\
\lambda_3 = \Psi(\lambda_3) = \Psi(a_{31}\theta_1 + a_{32}\theta_2 + a_{33}\theta_3) = a_{31}\theta_2 + a_{32}\theta_1 + a_{33}\theta_3
\end{split}
\end{equation}

From the system (\ref{sistem1}) and from the system obtained in (\ref{sistem2}) we get:

\begin{equation} \label{sistem3}
\begin{split}
0 = (a_{11} - a_{22})\theta_1 + (a_{12} - a_{21})\theta_2 + (a_{13} - a_{23})\theta_3 \\
0 = (a_{12} - a_{21})\theta_1 + (a_{11} - a_{22})\theta_2 + (a_{13} - a_{23})\theta_3  \\
0 = (a_{31} - a_{32})\theta_1 + (a_{32} - a_{31})\theta_2
\end{split}
\end{equation}

Since $< \theta_1,\theta_2,\theta_3 >$ is a free abelian group, then from the system in (\ref{sistem3}) we have:

$a_{11} = a_{22}, \quad a_{12} = a_{21}, \quad a_{13} = a_{23}, \quad a_{31} = a_{32}$


Then the system (\ref{sistem1}) can be rewritten as

\begin{equation} \label{sistem4}
\begin{split}
\lambda_1 = a_1\theta_1 + a_2\theta_2 + a_3\theta_3 \\
\lambda_2 = a_2\theta_1 + a_1\theta_2 + a_3\theta_3  \\
\lambda_3 = a_4\theta_1 + a_4\theta_2 + a_5\theta_3 \\
\end{split}
\end{equation}

\noindent where $a_1=a_{11}$, $a_2=a_{12}$, $a_3=a_{23}$, $a_4=a_{31}$ and $a_5=a_{33}$.
Applying the homomorphism $\Phi$ to the system (\ref{sistem4}) we obtain:

\begin{equation} \label{sistem5}
\begin{split}
\delta_1 =\Phi(\lambda_1)=\Phi(a_1\theta_1 +a_2\theta_2 +a_3\theta_3) = (a_1 +a_2)\beta_1 +2a_3\beta_2     \\
\delta_1 = \Phi(\lambda_2) = \Phi(a_2\theta_1 + a_1\theta_2 + a_3\theta_3) = (a_2 + a_1)\beta_1 + 2a_3\beta_2  \\ 
2\delta_2 = \Phi(\lambda_3) = \Phi(a_4\theta_1 + a_4\theta_2 + a_5\theta_3) = 2a_4\beta_1 + 2a_5\beta_2
\end{split}
\end{equation}

By properties of free abelian groups, we obtain from the last equation of the system (\ref{sistem5}) that:
$$\delta_2 = a_4\beta_1 + a_5\beta_2$$. 

So the system in (\ref{sistem5}) can be rewritten as:

\begin{equation} \label{sistem6}
\begin{split}
\delta_1 = (a_1 + a_2)\beta_1 + 2a_3\beta_2.  \\ 
\delta_2 = a_4\beta_1 + a_5\beta_2
\end{split}
\end{equation}

From the system (\ref{sistem6}) we see that the matrix $A$, given by: 


$$A = \begin{pmatrix} a_1+a_2  & 2a_3 \\ a_4  & a_5 \end{pmatrix}$$
is the representation matrix of the group $ G_2 = < \beta_1, \beta_2 : \delta_1, \delta_2 >$. From the system
in (\ref{sistem4}), doing an operation on rows, we see that the matrix $\tilde A$, given by:


$$\tilde A= \begin{pmatrix} a_1 & a_2 & a_3 \\ a_1+a_2 & a_1+a_2 & 2a_3 \\ a_4 & a_4 & a_5 \end{pmatrix}$$
is a representation matrix of the group $G_1$.

By Smith Normal Form Theorem, there exists matrices $S_1$ and $S_2$ of order $2 \times 2$, invertible and with integer entries such 
that the matrix $S_1AS_2$ is a diagonal matrix with integer entries. From Smith Normal Form Theorem it is also known that the 
inverse matrices of $S_1$ and $S_2$ have integer entries, therefore $det S_1 = \pm 1$ and $det S_2 = \pm 1$. Now, since $G_2$ is the trivial 
group, then $det A = \pm 1$. So the matrix $S_1AS_2$ is of the form 

\begin{equation}\label{sistem7} 
S_1AS_2 = \begin{pmatrix} \pm 1 & 0 \\ 0 & \pm 1 \end{pmatrix}
\end{equation}

From (\ref{sistem7}) we can ensure that there is a matrix $S$ of order $2 \times 2$, invertible and
with integer entries that satisfies:

\begin{equation}\label{sistem8} 
SA= \begin{pmatrix} 1 & 0 \\ 0 & 1 \end{pmatrix}
\end{equation}


Let us define the following matrix:

\begin{equation*}
\tilde{S}=
\begin{pmatrix}
1 & \begin{matrix}
0 & 0
\end{matrix}\\
\begin{matrix}
0 \\
0
\end{matrix} & S
\end{pmatrix}
\end{equation*}

Clearly the matrix $\tilde S$ has integer entries and using the result in (\ref{sistem8}) we have:

\begin{equation}\label{sistem9} 
\tilde S \tilde A= \begin{pmatrix} a_1 & a_2 &  a_3 \\ 1 & 1 &  0 \\ 0 & 0 & 1 \end{pmatrix}
\end{equation}

Using elementary operations, from the matrix in (\ref{sistem9}) we obtain:

$$\begin{pmatrix} a_1-a_2 & 0 & 0 \\ 0 & 1 &  0 \\ 0 & 0 & 1 \end{pmatrix}$$



From the above matrix we conclude that the group $< \theta_1, \theta_2, \theta_3 : \lambda_1, \lambda_2, \lambda_3 >$ is
isomorphic to $Z_{a_1-a_2}$, therefore the group $G_1$ is isomorphic to $Z_{a_1-a_2}$. \end{proof}

The following result is well known to experts. We include a proof for completeness and because it
will help us as a lemma in the proof of Theorem \ref{cubiertatransito}.

\begin{theorem} \label{cubiertatunel} If $K$ is a knot in $S^3$ such that $t(K) = 1$, then the first homology group of the double branched cover of $K$ is cyclic. \end{theorem}

\begin{proof} Let $K$ be a knot in $S^3$ such that $t(K) = 1$, and let ${\tau}$ be an unknotting tunnel for $K$. 
Let $T = \mathcal{N}(K \cup \tau)$ and 
$Ext(T ) = S^3 \backslash T$, so $Ext(T)$ is a genus two handlebody.
Since $Ext(T)$ is a handlebody, we can ensure that there exists a knot $K' \subset T$ such that $K'$ is a trivial knot in 
$S^3$ and it is homotopic with the knot $K$ in $T$. Let $\Sigma(K')$ be the double branched cover of the 
knot $K'$ and let $p : \Sigma(K') \rightarrow S^3$ be the associated covering function.
It is easy to notice, for the way it is defined $T$,  that there are meridian disks $D_1$ and $D_2$ in $T$ such that 
$\vert D_1 \cap K\vert = 0$ and 
$\vert D_2 \cap K\vert = 1$. Since $K'$ is homotopic to $K$ in $T$, then $\vert D_1 \cap K' \vert$ is an even integer and 
$\vert D_2 \cap K'\vert$ is an odd integer. Therefore $\partial D_1$ lifts, under $p$, in two simple closed curves; while 
$\partial D_2$ lifts to exactly a single simple closed curve. Let us denote by $\Lambda_1$ and 
$\Lambda_2$ the liftings of $\partial D_1$ and by $\Lambda_3$ the lifting of $\partial D_2$.
For each $i \in \{1,2,3\}$ we attach a 2-handle to $p^{-1}(Ext(T))$ along
$\Lambda_i \subset \partial(p^{-1}(Ext(T)))$; let us denote the 2-handle attached along $\Lambda_i$ by 
$\overline{\Lambda}_i$. Let $\Sigma$ be the 3-manifold obtained by attaching to $p^{-1}(Ext(T))$ the 2-handles 
$\overline{\Lambda}_i$, 
that is: $\Sigma := p^{-1}(Ext(T))\cup (\cup_{i=1}^3\overline{\Lambda}_i)$.

Let us note the following observations: 
\begin{enumerate}

\item $\partial p^{-1}(Ext(T))$ is a genus three connected surface.
\item $p^{-1}(Ext(T))$ is a double covering of $Ext(T)$. %with covering function given by $p\vert_1{p^{-1} (Ext(T ))}$.
\item The function $p$ can be extended to $\Sigma$, such that $\overline{\Lambda}_1 \cup \overline{\Lambda}_2$ is a double 
covering of $\mathcal{N}(D_1)$ and $\Lambda_3$ is a double covering of $\mathcal{N}(D_2)$ branched along 
$K \cap \mathcal{N}(D_2)$.
\item $\partial\Sigma$ is a 2-sphere.

\end{enumerate}

Let $\Sigma(K)$ be the 3-manifold obtained by attaching a 3-ball to $\Sigma$ along its
boundary. So, we can extend the covering function $p\vert_ {p^{-1}(Ext(T))} : p^{-1}(Ext(T)) \rightarrow Ext(T)$ to a covering 
function $p': \Sigma(K) \rightarrow S^3$ which branches along the knot $K$. 
Therefore $\Sigma (K)$ is the double covering of 
$S^3$ branched along $K$ with covering function given by $p'$.

We know that $Ext(T)$ is a genus two handlebody, therefore $H_1(Ext(T))$ is a free abelian group in two generators. 
Note that $\partial(p^{-1}(Ext(T)))$ is a genus three handlebody, therefore $H_1(p^{-1}(Ext(T)))$ is a free abelian group in three generators.


\begin{claim}\label{curvascubiertas} There are two connected simple closed curves in $Ext(T)$, 
denoted by $B_1$ and $B_2$, 
such that $B_1$ lifts, by $p$, in two  closed and connected simple curves, denoted by $\Theta_1$ 
and $\Theta_2$; while $B_2$ lifts, by $p$, in exactly one simple curve closed, denoted 
by $\Theta_3$. If $\beta_j$ is the homology class of $B_j$ in $H_1(Ext(T))$ and $\theta_i$ is the homology 
class of $\Theta_i$ in $H_1(p^{-1}(Ext(T)))$ for all $j \in \{1,2\}$ and $i \in \{1,2,3\}$, then
$H_1(Ext(T))=<\beta_1,\beta_2 >$ , $H_1(p^{-1}(Ext(T)))=<\theta_1,\theta_2,\theta_3 >$.
 \end{claim}

\begin{proof} Note that $Ext(T)$ is a genus two handlebody, call it $V$. Let $D$ be a disk in $V$ which splits it
in two solid tori $V_1$ and $V_2$. Note that $p^{-1}(V_i)$ double covers $V_i$, then it is either a set of
two solid tori or a solid torus that coves $V_i$ two-to-one. There are two possibilities.

\begin{enumerate} 

\item $V_1$ is covered by two solid tori, say $V^1_1$ and $V^2_1$, and $V_2$ is covered two-to-one by a solid torus $V_2'$. See Figure

\item $V_1$ and $V_2$ are covered both two-to-one  by solid tori $V_1'$ and $V_2'$. See Figure

\end{enumerate}

In Case 1, take as $B_i$, $i=1,2$, a core of the solid tori $V_i$. Clearly $B_1$ lifts to two simple closed curves
$\Theta_1$ and $\Theta_2$, which are a core of the solid tori $V^1_1$ and $V^2_1$, and $B_2$ lifts to
a simple closed curve $\Theta _3$ which is a core of the solid tori $V_2'$, and which cover two-to-one the
curve $B_2$. In this case it is clear that the homology classes of the curves satisfy the required properties.
See Figure \ref{handlebody1}.

% Figure environment removed

In Case 2, take as $B_1$ a curve that goes once around each of the cores of $V_1$ and $V_2$ and intersects
$D$ in two points. In this case $B_1$ lifts to two simple closed curves $\Theta_1$ and $\Theta_2$, each of
which goes once around $V^1_1$ and $V^2_1$. Take as $B_2$ a core of $V_1$, then clearly it lifts to a 
curve $\Theta_3$ which covers $B_2$ two-to-one. It is clear that the homology classes of the curves satisfy the required properties. See Figure \ref{handlebody2}.
\end{proof}

% Figure environment removed

We know that $p^{-1}(Ext(T))$ is a double covering of $Ext(T)$, with covering function given by the restriction of $p$. Let 
$p_*: H_1(p^{-1}(Ext(T))) \rightarrow H_1(Ext(T))$ be the
homomorphism associated with the restriction of $p$. For each $i \in \{1, 2, 3\}$, let us denote
by $\lambda_i$ the homology class in $H_1(p^{-1}(Ext(T)))$ associated to the curve $\Lambda_i$. 
Note that $H_1(\Sigma[K])=<\theta_1,\theta_2,\theta_3: \lambda_1,\lambda_2,\lambda_3>$. For
each $j \in \{1, 2\}$, let us denote by $\delta_j$ the homology class in $H_1(Ext(T))$ associated to the curve $\partial D_j$.
We have that $H_1(Ext(T))=<\beta_1,\beta_2: \delta_1, \delta_2 >$.
By choosing orientations conveniently, assume that 

\begin{equation} \label{sistem10}
\begin{split}
 p_*(\lambda_1) = \delta_1, \,\,\, p_*(\lambda_2) = \delta_1, \,\,\, p_*(\lambda_3) = 2\delta_2
\end{split}
\end{equation}



According to Claim \ref{curvascubiertas}, we have that


\begin{equation} \label{sistem11}
\begin{split}
p_*(\theta_1)= \beta_1, \,\,\, p_*(\theta_2)=\beta_1, \,\,\, p_*(\theta_3)=2\beta_2
\end{split}
\end{equation}

Let $q:p^{-1}(Ext(T)) \rightarrow p^{-1}(Ext(T))$ be the non-trivial covering transformation associated to the covering function $p\vert_{p^{-1}(Ext(T ))}$. Let 
$q_*:H_1(p^{-1}(Ext(T))) \rightarrow H_1(p^{-1}(Ext(T)))$ be  the homomorphism induced by the covering transformation $q$. By Claim \ref{curvascubiertas} we have that

\begin{equation} \label{sistem12}
\begin{split}
q_*(\theta_1) = \theta_2, \,\,\, q_*(\theta_2) = \theta_1, \,\,\, q_*(\theta_3) = \theta_3 \\
q_*(\lambda_1) = \lambda_2, \,\,\, q_*(\lambda_2) = \lambda_1, \,\,\, q_*(\lambda_3) = \lambda_3
\end{split}
\end{equation}

Then, applying Lemma \ref{grupos} directly we have that $H_1(\Sigma_2(K)) = Z_{x-y}$,
where $\lambda_1 = x\theta_1 + y\theta_2 + z\theta_3$.
\end{proof}

Now we prove the main result of this paper.

\begin{theorem} \label{cubiertatransito} If $K$ is a knot in $S^3$ such that $tr(K) = 1$ then the first homology group of the double branched cover of $K$ is cyclic. \end{theorem}

\begin{proof} Let $K$ be a knot in $S^3$ such that $tr(K) = 1$, and let $\{\tau \}$ be a transient system for the 
knot $K$. Let $T = \mathcal{N}(K \cup \tau)$ and let $K' \subset T$ be a trivial knot in $S^3$ such that $K'$ is 
homotopic to $K$ in $T$. Define also the 3-manifold $Ext(T)$ as $Ext(T) := S^3 \backslash Int(T)$.

As $\partial T$ is a genus two surface in the exterior of the knot $K'$, which is trivial, it follows that $\partial T$
is compressible in $Ext(K')$, that is, there is a compression disk $E_1$ for $\partial T$ disjoint from $K'$.

There are two possibilities for the disk $E_1$, that is: 
\begin{enumerate}

\item The disk $E_1$ is a compression disk for $\partial T$ lying in the interior of $T$; 
\item The disk $E_1$ is a compression disk for $\partial T$ lying in the exterior of $T$.
\end{enumerate}

Suppose first that we have case (1), that is, $E_1$ lies in the interior of $T$. If $E_1$ separates $T$, then by
cutting along $E_1$ we get two solid tori, one of them contains $K'$, and then there is a compression disk in the
other solid tori which is non-separating in $T$. So, we can assume that there is a compression disk $E_1$
for $\partial T$, lying in $T$, and which does not separate $T$.
%Podemos asumir que el disco B1 no separa a T, ya que de lo contrario T \\mathcal{N}B1 tendr ́ıa dos componentes conexas y K' estar ́ıa contenido en solo una de estas componentes conexas. Luego tomando un disco de compresio ́n interior a la componente de T \ \mathcal{N} B1 que no contiene a K' vemos que este u ́ltimo disco sera ́ un disco de compresi ́on interior a T, disjunto de K' y que adema ́s no separa a T.

\begin{claim} There exist a knot $K''$ and a disk $E_2$ in $T$ such that:

\begin{enumerate}
\item $E_2$ is a compression disk for $\partial T$ which is properly embedded in $T$.
\item $K''$ is a trivial knot in $S^3$ and it is homotopic to $K$ in $T$.

%\item $\vert E_1 \cap K'' \vert = 0$.			
\item $\vert E_2 \cap K'' \vert = 1$.
\end{enumerate} \end{claim}

\begin{proof} By cutting $T$ along $E_1$, we get a solid torus $V$. The knot $K'$ lies in $V$, and as $K'$
represents a primitive element in $\pi_1(T)$, it must be 
homotopic to the core of $V$.
If $V$ is knotted, then $\partial V$ is incompressible in $Ext(K')$, which is not possible, for $K'$ is the trivial knot.
Then $V$ must be a standard solid torus in $S^3$. Then $K'$ can be further homotoped to the core of $V$,
which is a trivial knot in the 3-sphere. Then there is a disk $E_2$ in $V$ such that $\vert E_2 \cap K'' \vert = 1$.
\end{proof}

Let $\Sigma[K'']$ be the double cover of $S^3$ branched along $K''$ with covering function given by $p:\Sigma[K''] \rightarrow S^3$. The disk $E_1$ and $E_2$ form a meridian disk system for $T$, and as $K''$ is disjoint from $E_1$ and 
intersects $E_2$ in one point, it follows that $p^{-1}(T)$ is a genus 3 handlebody, $p^{-1}(E_1)$ consists of two disks and 
$p^{-1}(E_2)$ consists of a single disk which covers two-to-one the disk $E_2$. Note that these disks form a
 meridian system for $p^{-1}(T)$. Let $B_i=\partial E_i$, $i=1,2$. Denote by $\Theta_1$ and $\Theta_2$ the two
 components of $p^{-1}{B_1}$, and let $\Theta_3=p^{-1}(B_2)$. As $\Sigma[K'']$ is the 3-sphere, and $p^{-1}(T)$
 is a genus 3 handlebody, it follows that the homology clases of the curves $\Theta_i$, $i=1,2,3$, generate
 $H_1(p^{-1}(Ext(T))$.

Let $\{D_1,D_2\}$ compression disks in the interior of $T$ such that $D_1$ is properly embedded in $\mathcal{N}(\tau)$ y $D_2$ is properly embedded in $\mathcal{N}(K)$, such that $\vert D_1 \cap K \vert = 0$ and 
$\vert D_2 \cap K \vert=1$. Note that the disks
$D_1$ and  $D_2$ do not separate $T$. As $K''$ is homotopic to $K$ in $T$, then $\vert D_1 \cap K'' \vert$ is an even number and $\vert D_2 \cap K'' \vert$ is an odd number. Therefore $\partial D_1$ lifts, under $p$, in two simple 
closed curves, while  $\partial D_2$ lifts exactly in a single simple closed curve. Denote by 
$\Lambda_1$ y $\Lambda_2$ the liftings of $\partial D_1$
and by $\Lambda_3$ the lifting of $\partial D_2$.
Attach 2-handles to the 3-manifold $p^{-1}(Ext(T))$ along the curves $\Lambda_i$, note that these curves lie in 
$\partial (p^{-1}(Ext(T)))$, and denote the 2-handle attached along $\Lambda_i$ by $\overline \Lambda_i$. Let $\Sigma$ be the 3-manifold obtained by attaching to $p^{-1}(Ext(T))$ the 2-handles $\overline \Lambda_i$.

%$$\Sigma := p^{-1}(Ext(T))\bigcup_{i=1}^{3} \Lambda_i$$

Note that $p^{-1}(Ext(T))$ is a doble covering of $Ext(T)$, with covering function $p'$ given by 
$p'=p\vert_{p^{-1} (Ext(T ))}$. The function $p'$ can be extended to a function 
$p':\Sigma \rightarrow Ext(T) \cup N(D_1) \cup N(D_2)$, such that $\overline \Lambda_1 \cup \overline \Lambda_2$ is a double covering of $\mathcal{N}(D_1)$ y $\overline \Lambda_3$ is a double covering of $\mathcal{N}(D_2)$ 
branched along $K \cap \mathcal{N}(D_2)$.

Note that $\partial \Sigma$ is a 2-sphere. Let $\Sigma(K)$ be the 3-manifold obtained by attaching a 3-ball to $\Sigma$
along its boundary. We can extend the covering function $p'$ to a covering function 
$\hat p: \Sigma(K) \rightarrow S^3$, which branchs along $K$. Therefore $\Sigma(K)$ is the double cover of $S^3$
branched along $K$ with covering function given by $\hat p$.

As $p^{-1}(Ext(T))$ is a double covering of $Ext(T)$, with covering function given by the restriction of $p$, 
let $p_*: H_1(p^{-1}(Ext(T))) \rightarrow H_1(Ext(T))$ be the homomorphism induced by $p$. For each $i \in \{1,2,3\}$
denote by $\lambda_i$ the homology class in $H_1(p^{-1}(Ext(T)))$ associated to the curve $\Lambda_i$.
For each $j \in \{ 1,2\}$ denote by $\delta_j$ the homology class in $H_1(Ext(T))$ associated to the curve $\partial D_j$. 
Then

\begin{equation} \label{sistem13}
\begin{split}
p_*(\lambda_1) = \delta_1, \,\,\, p_*(\lambda_2) = \delta_1, \,\,\,p_*(\lambda_3) = 2\, \delta_2 
\end{split}
\end{equation}

Note that $H_1(Ext(T))$ is a free abelian group in two generators, generated by the homology classes of 
the curves $B_1$ and $B_2$, which we denote by $\beta_1$ and $\beta_2$. As we said before, 
$H_1(p^{-1}(Ext(T)))$ is a free abelian group in there generators, generated by the homology classes of the
curves $\Theta_i$, which we denote by $\theta_i$, $i=1,2,3$. We have that

\begin{equation} \label{sistem14}
\begin{split}
H_1(Ext(T))=<\beta_1,\beta_2 >, \,\, \, H_1(p^{-1}(Ext(T)))=<\theta_1,\theta_2,\theta_3 >
\end{split}
\end{equation}

We also obtain that 

\begin{equation} \label{sistem15}
\begin{split}
H_1(Ext(T))=<\beta_1,\beta_2: \delta_1, \delta_2 >, \,\, \, H_1(\Sigma[K])=<\theta_1,\theta_2,\theta_3; \lambda_1, \lambda_2, \lambda_3 >
\end{split}
\end{equation}

\begin{equation} \label{sistem16}
\begin{split}
p_*(\theta_1) = \beta_1, \,\,\, p_*(\theta_2) = \beta_1, \,\,\, p(\theta_3) = 2\beta_2 
\end{split}
\end{equation}

Let $q: p^{-1}(Ext(T)) \rightarrow p^{-1}(Ext(T))$ be the non-trivial covering transformation, associated to the covering function $p$. Let $q_* : H_1(p^{-1}(Ext(T))) \rightarrow H_1(p{-1}(Ext(T)))$ be the homomorphism associated to the covering transformation $q$. By the way that $\theta_i$ and the $\lambda_i$ were defined we have that:

\begin{equation} \label{sistem17}
\begin{split}
q_*(\theta_1) = \theta_2, \,\,\, q_*(\theta_2) = \theta_1, \,\,\, q_*(\theta_3) = \theta_3, \\
 q_*(\lambda_1) = \lambda_2, \,\,\, q_*(\lambda_2) = \lambda_1, \,\,\, q_*(\lambda_3) = \lambda_3
\end{split}
\end{equation}
Applying Lemma \ref{grupos} we have that $H_1(\Sigma(K)) = \mathbb{Z}_{x-y}$, where $\lambda_1 = x\theta_1 + y\theta_2 + z\theta_3$.
So, we have proved that if the compression disk $E_1$ is contained in $T$, then the homology group of 
the double branched cover of $K$ is cyclic.

Now suppose that the compression disk $E_1$ is contained in $Ext(T)$. In this situation we can suppose that $Ext(T)$
is not a handlebody, for otherwise we have that $t(K) = 1$ and by Theorem \ref{cubiertatunel} we get the desired result. Suppose first that
the disk $E_1$ does not separate $Ext(T)$. Define $\Gamma= T \cup \mathcal{N}(E_1)$. As $E_1$ does not
 divide $\partial T$ then $\partial \Gamma$ is a connected genus one surface, and it must bound a solid torus.
 Then $\Gamma$ is a solid torus, for otherwise $Ext(T)$ will be a genus 2 handlebody. So, $\Gamma$ is a knotted
 solid torus and $K'$ lies on it. As $K'$ is a trivial knot, it must lie in a 3-ball contained in $\Gamma$, for otherwise there will be an incompressible torus in $Ext(K)$. In particular, $K'$ has winding number zero in $\Gamma$.
 Then $K$ is also of winding number zero in $\Gamma$, as it is homotopic to $K'$ in $T\subset \Gamma$. 
 Embed $\Gamma$ in $S^3$ such that it is an standard solid torus $V$, and such that a preferred longitude
 of $\Gamma$ goes to a preferred longitude of $V$. Let $\bar K$ be the image of $K$ in $V$. Then $K$ is
 a satellite knot with pattern given by $\bar K$. As $\bar K$ has winding number zero in $V$, it follows that
 $H_1(\Sigma[\bar K])$ is isomorphic to  $H_1(\Sigma[K])$, by \cite{Sei}.  Let $\bar T$ be the image of $T$ in $V$,
 clearly $\bar T$ is the neighborhood of $\bar K$ union a transient arc, and the exterior of $\bar T$ is the exterior
 of $V$, which is a solid torus union a 1-handle given by the image of the disk $E_1$.
This shows $\bar K$ is a tunnel number one knot
 and then $H_1(\Sigma[\bar K])$ is a cyclic group, which implies then that $H_1(\Sigma[K])$ is also cyclic.
 
 Suppose now that the disk $E_1$ separates $Ext(T)$ and that there is no non-separating compression disk in $Ext(T)$.
Let $\Gamma= T \cup \mathcal{N}(E_1)$. As $E_1$ is separating, $\partial \Gamma$ consist of two tori, 
say $S_1$ and $S_2$. Then $S_1$ bounds a solid torus $V_1$ which contains $\Gamma$, and also contains $S_2$.
Then $V_1$ is a knotted solid torus, and as $K'$ is contained in $V_1$, it must lie inside a 3-ball, and then as
in the previous case, $K$ has winding number zero in $V_1$. Embed $V_1$ in $S^3$ such that it is an standard solid 
torus $V_2$, and such that a preferred longitude of $V_1$ goes to a preferred longitude of $V_2$. Let $\bar K$ be the
image of $K$ in $V_2$. Then $K$ is
 a satellite knot with pattern given by $\bar K$. As $\bar K$ has winding number zero in $V$, it follows that
 $H_1(\Sigma[\bar K])$ is isomorphic to  $H_1(\Sigma[K])$, by \cite{Sei}.  Let $\bar T$ be the image of $T$ in $V$,
 clearly $\bar T$ is the neighborhood of $\bar K$ union a transient arc, and the exterior of $\bar T$ is the exterior
 of $V$, which is a solid torus union a manifold bounded by the image of $S_2$ plus 1-handle given by the image 
 of the disk $E_1$. It follows that $\bar K$ is a transient number one knot
 such that the exterior of the knot union a transient arc is compressible, and it has a non-separating compression disk.
 By the previous case, $H_1(\Sigma[\bar K])$ is a cyclic group, which implies then that $H_1(\Sigma[K])$ is also cyclic.
 \end{proof}
 
% From  $T$ we construct a genus one handlebody, denoted by $\Gamma$, obtained by attaching handles (2-handles or 3-handles) to $T$; this is done oin the following way: if $B_1$ does not divide $\partial T>$ then define $\Gamma= T \cup \mathcal{N}(D_1)$, as $D_1$ does not divide $\partial T$ then $\partial \Gamma$ is a connected genus one surface. If $D_1$ divides $\partial T$ then $S^3 \ Int(T \cup \mathcal{N}(D_1))$ is a 3-manifold with two connected components and as $Ext(T)$ is not a handlebody then one of these components is not a handlebody. In this case we define $\Gamma$ as the 3-manifold obtained from $S^3$ from removing the interior of $S^3 \ Int(T \cup \mathcal{N}(D_1))$ the component which is not a handlebody. Note that in any case we have that $Ext(\Gamma)$ is not a handlebody. Let $C$ be the core of $\Gamma$, so that $\Gamma = \mathcal{N}(C)$. Now as $Ext(\Gamma)$ is not a handlebody then $C$ is not a trivial knot.\end{proof}

\section{Knots with large transent number}\label{examples}

By the results of the last section we can now estimate the transient number of some knots.

\begin{theorem}\label{largetransitnumber} Let $K$ be a knot such that its double branched cover is not an homology sphere, that is, $H_1(\Sigma[K])$ is not trivial. Then
\begin{enumerate}
\item $tr(K\# K) \geq 2$:
\item $tr(K_n) \geq (n-1)/2$, where $K_n = K\# K\# \cdots \# K$, is the connected sum of $n$ copies of $K$.
\end{enumerate}
\end{theorem}

\begin{proof} it is known that $\Sigma[K_n]= \Sigma[K] \# \Sigma[K] \# \cdots \# \Sigma[K]$, the connected sum
of $n$ copies of $\Sigma[K]$. As $H_1(\Sigma[K])$ is not trivial, then $H_1(\Sigma[K])$ has rank at least
$n$. By Theorem \ref{cubiertageneral}, $tr(K_n) \geq (n-1)/2$, this shows (2). In particular 
$H_1(\Sigma[K_2])=H_1(\Sigma[K])+H_1(\Sigma[K])$, which is not cyclic, and this implies (1).
\end{proof}

This shows that there are knots with arbitrarily large transient number, which answers a question of 
Koda and Ozawa \cite{KO}.

Now we concentrate in the tables of knots up to crossing number 10.

\begin{theorem} 
\begin{enumerate}
\item The following knots have transient number $2$: $8_{18}$, $9_{35}$, $9_{37}$, $9_{40}$, $9_{41}$, $9_{46}$, $9_{47}$,
$9_{48}$, $9_{49}$, $10_{74}$, $10_{75}$, $10_{98}$, $10_{99}$, $10_{103}$, $10_{123}$, $10_{155}$, $10_{157}$.

\item The following knots have transient number at most $2$: $8_{16}$, $9_{29}$, $9_{32}$, $9_{38}$, $10_{61}$,
$10_{62}$, $10_{63}$, $10_{64}$, $10_{65}$, $10_{66}$, $10_{67}$, $10_{68}$, $10_{69}$, 
$10_{79}$, $10_{80}$, $10_{81}$, $10_{83}$, $10_{85}$, $10_{86}$, $10_{87}$, $10_{89}$,
$10_{90}$, $10_{92}$, $10_{93}$, $10_{94}$, $10_{96}$, $10_{97}$, $10_{100}$, $10_{101}$,
$10_{105}$, $10_{106}$, $10_{108}$, $10_{109}$, $10_{110}$, $10_{111}$, $10_{112}$, 
$10_{115}$, $10_{116}$, $10_{117}$, $10_{120}$, $10_{121}$, $10_{122}$, $10_{140}$, 
$10_{142}$, $10_{144}$, $10_{148}$, $10_{149}$, $10_{150}$, $10_{151}$, $10_{152}$, 
$10_{153}$, $10_{154}$, $10_{158}$, $10_{160}$, $10_{162}$, $10_{163}$, $10_{165}$.

\item Any other knot of crossing number at most 10 has transient number one.
\end{enumerate}
\end{theorem}

\begin{proof} According to the information given in KnotInfo \cite{Knot}, the knots in (1) and (2) are precisely the knots
with crossing number up to 10, whose unknotting number and tunnel number are both larger that 1. So, any other knot has unknotting number of tunnel number equal to 1, and then have transient number 1. The knots in (1) are precisely the knots 
whose double branched cover has non-cyclic first homology group, and furthermore these knots have tunnel number 2.
Therefore its transient number must be two. The knots in (2) have tunnel number two but  their double branched cover have cyclic first homology group, hence we cannot calculate the transient number yet.
\end{proof}

A similar result can be done for the knots of crossing number 11 or 12.

The following knots are interesting, for we use the homology of $p$-branched covers of a knot to determine the transient number.

\begin{theorem} The following knots have transient number $2$: $10_{99}$, $10_{123}$, $12a_{427}$, $12a_{435}$, 
$12a_{465}$, $12a_{466}$, $12a_{475}$, $12a_{647}$, $12a_{742}$, $12a_{801}$, $12a_{868}$, $12a_{975}$, 
$12a_{990}$, $12a_{1019}$, $12a_{1102}$, $12a_{1105}$, $12a_{1167}$, $12a_{1206}$, $12a_{1229}$, 
$12a_{1288}$, $12n_{518}$, $12n_{533}$, $12n_{604}$, $12n_{605}$, $12n_{642}$, $12n_{706}$, $12n_{840}$, $12n_{879}$, $12n_{888}$. 
\end{theorem}

\begin{proof} According to Theorem \ref{cubiertageneral-p}, if $K$ has $tr(K)=1$, then $rank(H_1(\Sigma_p[K]) \leq p+1$. Using this and the information given in KnotInfo \cite{Knot}, we show that these knots cannot have transient number one. 
As they have tunnel
number two, in fact must also have transient number two. Below in the Table, there is a list of the knots with the corresponding
homology group needed for the proof. For some of them, it is enough to use the homology of $\Sigma[K]$,
but not for all. A symbol $\{6,\{2,2,2,10,20,340,0,0\}\}$ means that 
$H_1(\Sigma_6[K])=\mathbb{Z}_2+\mathbb{Z}_2+\mathbb{Z}_2+\mathbb{Z}_{10}+\mathbb{Z}_{20}+\mathbb{Z}_{340}+\mathbb{Z}+\mathbb{Z}$.
\end{proof}



%\begin{center} TABLE 1 \end{center}

\begin{scriptsize}
\begin{center}
\begin{tabular}{|c|c|c|}





\hline
&$ 10_{99}$	& $\{2,\{9,9\}\},\quad \{6,\{2,2,6,6,0,0,0,0\}\}$ \\
\hline
& $10_{123}$ & 	$\{2,\{11,11\}\},\quad \{5,\{2,2,2,2,2,2,2,2\}\}$ \\
\hline
& $12a_{427}$	 & $\{2,\{15,15\}\},\quad  \{4,\{3,3,3,3,15,15\}\},\quad \{6,\{4,4,20,20,0,0,0,0\}\} $ \\
\hline
&$12a_{435}$	 & $\{2,\{3,75\}\},\quad \{6,\{2,2,8,200,0,0,0,0\}\}$ \\
\hline
& $12a_{465}$	& $\{6,\{2,2,2,2,2,2,38,9158\}\}$ \\
\hline
&$12a_{466}$ &	$\{6,\{2,2,2,2,2,2,26,5434\}\} $\\
\hline
& $12a_{475}$ &	$\{6,\{2,2,2,10,20,340,0,0\}\} $\\
\hline
& $12a_{647}$ & 	$\{2,\{3,51\}\},\{6,\{2,2,2,34,0,0,0,0\}\} $\\
\hline
%&12a_{742} &	{{2,{3,45}},{6,{5,5,15,0,0,0,0}}mmmmmmm \\

%& 12a_{801}	& {{2,{3,45}},{6,{5,5,15,0,0,0,0}}mmmmmmmm \\

&$12a_{868}$ &	$\{5,\{2,2,2,2,8,8,88,88\}\}$ \\
\hline
& $12a_{975}$	& $\{2,\{5,45\}\},\quad \{4,\{5,5,5,5,5,45\}\} $\\
\hline
&$12a_{990}$ & $\{2,\{3,75\}\}\quad \{6,\{2,2,8,200,0,0,0,0\}\} $ \\
\hline
&$12a_{1019}$	& $\{2,\{19,19\}\},\quad \{5,\{6,6,6,6,6,6,6,6\}\} $ \\
\hline
&$12a_{1102}$ &	 $\{6,\{2,2,2,2,2,2,112,34160\}\} $\\
\hline
& $12a_{1105}$	& $\{2,\{17,17\}\},\quad \{6,\{2,2,2,2,10,10,170,170\}\} $ \\
\hline
&$12a_{1167}$ &	$\{5,\{2,2,2,2,2,2,82,82\}\} $\\
\hline
%& $12a_\{1206\}$	& $\{\{2,\{7,35\}\},\quad \{5,\{11,11,11,11,11,11\}\},$mmmmmm \\

& $12a_{1229} $&	$\{5,\{2,2,2,2,8,8,8,8\}\}$ \\
\hline
& $12a_{1288}$	& $\{2,\{3,39\}\},\quad \{6,\{2,2,2,26,0,0,0,0\}\}$ \\
\hline
&$12n_{518}$ &	$\{2,\{3,21\}\}, \quad \{6,\{2,2,4,28,0,0,0,0\}\} $ \\
\hline
&$12n_{533}$	& $\{6,\{2,2,2,2,2,42,0,0\}\} $\\
\hline
&$12n_{604}$ &	$\{2,\{3,27\}\},\quad \{6,\{2,2,2,18,0,0,0,0\}\} $ \\
\hline
& $12n_{605}$	& $\{2,\{3,3\}\},\quad \{6,\{2,2,2,2,0,0,0,0\}\} $ \\
\hline
%& 12n_{642}	[3,4]	3&	{ {2,{3,3,3}},{3,{23,23}},{4,{3,51,51}},{5,{2,2,158,158}},{6,{3,9,207,207}},{7,{3527,3527}},{8,{3,6477,6477}},{9,{34753,34753}}} \\

&$12n_{706}$	&  $\{2,\{7,7\}\},\quad \{5,\{3,3,3,3,3,3,3,3\}\},\quad \{6,\{2,2,2,2,2,2,14,14\}\}$ \\
\hline
&$12n_{840} $ &	$\{6,\{2,2,2,2,2,2,10,1190\}\}$ \\
\hline
&$12n_{879}$ & 	$\{5,\{2,2,2,2,4,4,4,4\}\} $ \\
\hline
&$12n_{888}$ &	$\{2,\{3,15\}\},\quad \{6,\{2,2,2,10,0,0,0,0\}\} $ \\
\hline

\end{tabular}
\end{center}
\end{scriptsize}



\section{Transient number and connected sums}\label{connectedsums}

It is natural to consider the behavior of a knot invariant with respect to connected sums. It is easy to see that $u(K_1 \# K_2) \leq u(K_1) + u(K_2)$, and the equality is conjectured to happen. It is also not difficult to see that $t(K_1 \# K_2)\leq t(K_1)+t(K_2)+1$. There are known examples of knots with 
$t(K_1 \# K_2) = t(K_1)+t(K_2)+1$ \cite{MSY}, examples with $t(K_1 \# K_2) = t(K_1)+t(K_2)$, and examples with 
$t(K_1 \# K_2) < t(K_1)+t(K_2)$ \cite{M}. So, we can expect a similar inequality for the transient number. 

\begin{theorem} Let $K_1$, $K_2$ be knots in $S^3$. Then $tr (K_1\# K_2) \leq tr (K_1)+tr (K_2) + 1$. \end{theorem}

\begin{proof} Let $K_1$ be a knot with transient number $tr(K)=n$, and let $\{ \gamma_1,\gamma_2,\dots,\gamma_n\}$ 
be a system of transient arcs for $K_1$. Let $T_1=\mathcal{N(}K\cup \gamma_1 \cup \gamma_2 \dots \cup \gamma_n)$. 
Then $T_1$ is a genus $n+1$ handlebody with the property that $K_1$ can be homotoped in the interior of $T_1$ to the 
trivial knot in $S^3$. We can assume that the homotopy that transform $K_1$ into the trivial
knot can be realized by a sequence of ambient isotopies of $K_1$ and crossing changes. So, suppose that after making
isotopies, all crossing changes are performed simultaneously. Suppose $r$ crossing changes are performed,
numbered $1,2,\dots,r$, and for each crossing change let $\alpha_i$ be an arc with endpoints in $K_1$
which remembers the crossing change, that is, if $B_i$ is a regular neighborhood of $\alpha_i$, in fact a 3-ball that intersects $K_1$ in two unknotted arcs, then a crossing change can be performed 
inside each $B_i$ to get the trivial knot.
Make an isotopy to move $K_1$ to its original position,
and then $\{\alpha_1,\alpha_2,\dots,\alpha_r\}$ is a collection of disjoint arcs with endpoints in $K_1$ contained  in $T_1$. 
Let $\delta_1$ be an arc in $T_1$ with an endpoint in $K_1$ and the other in $\partial N(K_1)$, such that $\delta_1$ is disjoint from the arcs $\alpha_i$. 

If $K_2$ is knot with $tr(K_2)=m$, then as above there is a genus $m+1$ handlebody that is the neighborhood
of $K$ union a system of transient arcs $\{ \gamma'_1,\gamma'_2,\dots,\gamma'_m\}$, and there is a collection of 
arcs $\{ \beta_1,\dots,\beta_s\}$ that determines
crossing changes that unknot $K_2$. Let $\delta_2$ be an arc in $T_2$ with an endpoint in $K_2$ and the other in 
$\partial N(K_2)$, such that $\delta_2$ is disjoint from the arcs $\beta_i$. 


Suppose that $T_1$ and $T_2$ lie in disjoint 3-balls $C_1$ and $C_2$ contained in $S^3$. Suppose that
 $\partial T_i \cap \partial C_i$ consists of a disk $D_i$, such that the endpoint of $\delta_i$ lying in $\partial T_i$, 
 it lies in $D_i$, for $i=1,2$. Do a disk sum of $T_1$ and $T_2$, identifying $D_1$ and $D_2$, such that the endpoints of 
 $\delta_1$ and $\delta_2$ coincide. Let $\delta=\delta_1 \cup \delta_2$,
this is an arc with an endpoints in $K_1$ and $K_2$. Following $\delta$, do a band sum of $K_1$ and $K_2$.
As $K_1$ and $K_2$ lie in disjoint 3-balls, this band sum is in fact a connected sum $K_1\# K_2$. Let
$T=T_1\cup T_2$, this is a genus $n+m+2$ handlebody, and $K_1\# K_2$ can be homotoped to the trivial knot
inside it, to see that just do crossing changes following the arcs $\alpha_i$ and $\beta_j$. Now note that $T$
is the regular neighborhood of $K_1\# K_2$ and a system of $n+m+1$ arcs, that is, the $n$ arcs for a
system fo $K_1$, the $m$ arcs for a system of $K_2$, plus one more arc which is dual to the band used to
perform the connected sum of $K_1$ and $K_2$, see Figure \ref{connectedsum}. 
This shows that the transient number of $K_1\# K_2$ is at most $n+m+1$.
 \end{proof}
 
 % Figure environment removed
 In many cases we can ensure that $tr(K_1\# K_2)$ is at most $tr(K_1)+ tr(K_2)$. For example, if the arc systems
 that unknot $K_1$ and $K_2$ are disjoint from a meridian disk for $N(K_1)$ and a meridian disk for $N(K_2)$,
 then it can be shown that no more than $tr(K_1) + tr(K_2)$ arcs are needed to unknot $K_1\# K_2$. 
 
 There are examples of knots
 $K_1$, $K_2$, such that $t(K_1)=1=t(K_2)$, but $t(K_1\# K_2)=3$ \cite{MSY}. For these example, it is clear that $tr(K_1)=1=tr(K_2)$, but it is not clear what is $tr(K_1\# K_2)$.
 
 There are also examples of knots  $K_1$, $K_2$, such that $t(K_1)=2$, $t(K_2)=1$, but $t(K_1\# K_2)=2$ \cite{Mr}.
 In this case $tr(K_2)=1$ and $tr(K_1\# K_2)\leq 2$, but it is not clear whether $tr(K_1)=1$ or $2$. 
 
 It is well known that knots with unknotting number one or tunnel number one are prime, but the proofs are not so easy.
 The first proof that knots $K$ with $u(K)=1$ are prime \cite{S}, uses heavy combinatorial arguments, a second
 proof uses sutured manifold theory \cite{ST}, and a third proof depends on double branched covers and deep
 results on Dehn surgery on knots \cite{Z}. There are also two proofs that tunnel number one knots are prime,
 one uses combinatorial group theory \cite{N}, and other uses combinatorial arguments \cite{Sc}. 
 A proof that transient number one
 knots are prime would imply both, that unknotting number one and tunnel number one knots are prime, 
 so it may not be easy to prove that. However seems reasonable to conjecture the following.
 
 \begin{conjecture} If $K$ is a knot with $tr(K)=1$ then $K$ is prime.
 \end{conjecture}
 
 
Theorem \ref{largetransitnumber} (1) gives some evidence  for this conjecture.
 
\vskip30pt

\textbf{Acknowledgments.}
 This research was supported by a grant from the National Autonomous University of Mexico, UNAM-PAPIIT IN116720.

\begin{thebibliography}{90}

\bibitem{Paz}
Mar\'{\i}a de la Paz Alvarez-Scherer,
Matadores de grupos de nudos e \'{\i}ndice de indeterminaci\'on de nudos discoidales,
PhD. Thesis, Univeridad Nacional Aut\'onoma de M\'exico, 1992.

\bibitem{G} 
C. McA.Gordon,
Some aspects of classical knot theory. 
Knot Theory, Proc. Semin., Plans-sur-Bex 1977, Lect. Notes Math. 685, 1-60 (1978). 


\bibitem{KO} 
Y. Koda, M. Ozawa,
Knot homotopy in subspaces of the 3-sphere.
Pac. J. Math. 282, No. 2, 389-414 (2016). 


\bibitem{Knot}
C. Livingston and A. H. Moore, 
KnotInfo: Table of Knot Invariants, 
knotinfo.math.indiana.edu, July 16, 2023. 


\bibitem{M} 
J.M. Montesinos,
Surgery on links and double branched covers of $S^3$,
Ann. of Math. Studies, No. 84,
Princeton Univ. Press, Princeton, N.J., 1975, pp. 227–259.


\bibitem{Mr}
K. Morimoto, 
Characterization of tunnel number two knots which have the property “2+1=2”. 
Topology Appl. 64, No. 2, 165-176 (1995).

\bibitem{MSY}
K. Morimoto, M. Sakuma, Y. Yokota,
Examples of tunnel number one knots which have the property “1+1=3”. 
Math. Proc. Camb. Philos. Soc. 119, No. 1, 113-118 (1996). 

\bibitem{N}
F.H. Norwood,
Every two-generator knot is prime.
Proc. Am. Math. Soc. 86, 143-147 (1982). 

\bibitem{Sc}
M. Scharlemann,
Tunnel number one knots satisfy the Poenaru conjecture.
Topology Appl. 18, 235-258 (1984). 

\bibitem{S}
M.  Scharlemann,
Unknotting number one knots are prime.
Invent. Math. 82, 37-55 (1985). 

\bibitem{ST}
M. Scharlemann, A. Thompson,
Unknotting number, genus and companion tori.
Math. Ann. 280, No. 2, 191-205 (1988). 

\bibitem{Sei}
H. Seifert,
On the homology invariants of knots.
Quart. J. Math. Oxford Ser. (2) 1 (1950), 23-32.

\bibitem{W}  
H. Wendt, 
Die gordische Aufl\"osung von Knoten.
Math. Z. 42, 680-696 (1937). 

\bibitem{Z}
X. Zhang,
Unknotting number one knots are prime: A new proof.
Proc. Am. Math. Soc. 113, No. 2, 611-612 (1991). 

\end{thebibliography}


\end{document}


given a knot $K$ and a collection of crossing changes that unknot it, take
a little 3-ball around each of the crossings, that intersects the knot in two arcs and encapsulates the crossing change.
Now perform the crossign change, obtaining the trivial knot. Take the double branched cover of the trivial knot,
which is then $S^3$, each of the 3-balls lift to a solid torus. To get the double branched cover of $K$, remove
the interior of the solid tori, do Dehn surgery, ensuring that the new manifold double branchs cover $S^3$ along
the knot $K$. Then as $\Sigma[K]$ is obtained by Den surgery on an $n$-compponent link in $S^3$, is
clear thet  $H_1(\Sigma[K])$ is an abelian group of rank at most $n$. For the case of tunnel number, note that
if $K$ has tunnel number $n$, then $K$ is contained in a genus $(n+1)$-handlebody $V$, such that its
complement is another genus $(n+1)$-handlebody $W$. By taking $\Sigma[K]$, $V$ and $W$ lift to genus
$(2n+1)$-handlebodies, that is, give a genus $2n+1$ Heegaard decomposition of $\Sigma[K]$. This shows
that $H_1(\Sigma[K])$ is an abelian group of rank at most $2n+1$. A better bound can be found using 
fundamental groups, as the rank of $\pi_1(S^3-K)$ is less or equal to the tunnel number $t(K)$, and
then doing algebraic manipulations as in \cite{G}. 

In this paper we make a combination of the above ideas to find bound to $tr(K)$ depending on the rank of
$H_1(\Sigma[K])$. Let $\{ \tau_1, \tau_2, \dots, \tau_n\}$ be a transit system for $K$, such that $K$ can be
homotoped to the trivial knot in $T=\mathcal{N}(K\cup \tau_1 \cup \dots \cup \tau_n)$. Note that $T$ is a
genus $(n+1)$-handlebody. The complement of $T$ is 3-manifold $M$, with $\partial M$ a surface of genus
$(n+1)$. Now take $\Sigma[K']$, the double branched cover of the trivial knot, this is $S^3$. $T$ and $W$
lift to manifolds $\hat T$ and $\hat M$,  it is not clear what kind of manifolds are these, but we knot that
$\partial \hat T =\partial \hat W$ is a connected genus $2n+1$ surface. As $\hat W$ is contained in $S^3$,
$H_1(\hat W)$ is a free abelian group of rank $2n+1$. Now take a system of meridian disks in $T$, 
such that one of the disks intersect $K$ in one point and the others are disjoint from $K$. The boundaries of
these disks lift to curves on $\partial \hat M$, then by adding 2-handles to $\hat M$ along these curves,
we get $\Sigma[K]$, the double branched cover of $K$. Then $H_1(\Sigma[K])$ is a quotient of $H_1(\hat M)$,
which implies that $H_1(\Sigma[K])$ is an abelian group of rank at most $2n+1$. In fact, we show the following.                                                                                                                                           

-------------------------------------------------------------------------------------------------------------------------------------------------------


\begin{document}

\title{On the transit number of a knot}

\author{Mario Eudave-Mu\~noz}
\address{ \hskip-\parindent
Mario  Eudave-Mu\~noz\\
Instituto de Matem\'aticas\\
 Universidad Nacional Aut\'onoma de M\'exico\\ 
 MEXICO}
 \email{mario@matem.unam.mx}
 
 \author{Joan Carlos Segura-Aguilar}
\address{ \hskip-\parindent
Joan Carlos Segura-Aguilar\\
Universidad Nacional Aut\'onoma de M\'exico\\ 
 MEXICO}
 \email{lorenaarmas089@gmail.com}


\maketitle

\begin{abstract}
We show .\end{abstract}



\section{Introduction}.

\section{Transit number and connected sums}

It is natural to consider the behavior of a knot invariant with respect to connected sums. It is easy to see that $u(K_1 \# K_2) \leq u(K_1) + u(K_2)$, and the equality is conjectured to happen. It is also not difficult to see that $tn(K_1 \# K_2)\leq tn(K_1)+tn(K_2)+1$. There are known examples of knots with 
$tn(K_1 \# K_2) = tn(K_1)+tn(K_2)+1$, examples with $tn(K_1 \# K_2) = tn(K_1)+tn(K_2)$, and examples with 
$tn(K_1 \# K_2) < tn(K_1)+tn(K_2)$. So, we can expect a similar inequality for transit number. 

\begin{theorem} Let $K_1$ and $K_2$ be knots in $S^3$. Then $tr (K_1\# K_2) \leq tr (K_1)+tr (K_2) + 1$. \end{theorem}

\begin{proof} Let $K$ be a knot with transit number $tr(K)=n$, and let $\{ \gamma_1,\gamma_2,\dots,\gamma_n\}$ be a system of transit arcs for $K$. 
Let $N=N(K\bigcup \cup_{i=1} ^{n} \gamma_i)$. Then $N$ is a genus $n+1$ handlebody with the property that $K$ can be isotoped in the interior of $N$ to the trivial knot in $S^3$. We can assume that the homotopy that transform $K$ into the trivial
knot can be realized by a sequence of ambient isotopies of $K$ and crossing changes. Suppose the first crossing change is performed obtaining a knot $K_1$, and let $\alpha_1$ be an arc with endpoints in $K_1$
which remembers the crossing change, that is, if $B$ is a regular neighborhood of $\alpha_1$, in fact a 3-ball that intersects $K_1$ in two unknotted arcs, then a crossing change can be performed 
inside $B$ to recover the original knot $K$. Perform the second crossing change, obtaining a knot $K_2$, and take again an arc $\alpha_2$ with endpoints in $K_2$, which remembers the
crossing change, and this arc can be chosen to be disjoint from $\alpha_1$. Continuing in this manner, after $r$ crossing changes we obtain the trivial knot $\bar K$, and $r$ arcs
$\gamma_1,\dots,\gamma_r$ with endpoints in $\bar K$, such that by performing a crossing change in each of the neighborhoods $B_i$ of the arcs $\alpha_i$ we get back the knot $K$.
Suppose we have the knot $K$ and the collection of ball $B_i$, each intersecting $K$ in two unknotted spanning arcs.
Consider for each $B_i$ and arc $\beta_i$ with endpoins in $K$, such that after performing a crossing change in $B_i$ we get the trivial knot. Make an isotopy to move $K$ to its original position,
and then $\beta_i$ would be a collection of arcs with endpoints in $K$ contained  in $N$. Let $\delta$ be an arc in $N$ with an endpoint in $K$ and the other in $\partial N(K)$, such that $\delta$ is disjoint from the
arcs $\beta_i$. Do something similar to $K_2$, that is  \end{proof}

\section{Transit number and double branched covers}


This section is inspired by an idea that is used to build the double cover of $S^3$
branched along a knot with unknotting number equal to one. Consider a knot $K$
in $S^3$ with unknotting number equal to one. Let $\alpha$ be an arc embedded in $S^3$, with endpoints in $K$, which encapsulates the crossing change. So there is a homotopy in $\mathcal{N}(K \cup \alpha)$ 
between the knot $K$ and the trivial knot of $S^3$. We denote by $K'$ the trivial knot of $S^3$ contained in 
$\mathcal{N}(K \cup \alpha)$ and that is homotopic to $K$. Clearly this homotopy can be taken so that it is 
constant in $\mathcal{N}(K) \backslash\, \mathcal{N}(\alpha)^\circ$ and that the changes are occurring only in 
$\mathcal{N}(\alpha)$; so we assume that $K'$ is obtained from $K$ just by taking the
 two arcs $K \cap \mathcal{N}(\alpha)$ and passing one arc through the other, which would correspond to a crossing change between the respective knot diagrams. 
Due to the above we have that $K\cap (S^3 \backslash \mathcal{N}(\alpha))=K' \cap (S^3 \backslash \mathcal{N}(\alpha))$.

Let $\Sigma (K')$ be the double cover of $S^3$ branched along the knot $K'$ with covering
function given by $p : \Sigma(K') \rightarrow S^3$. Now, since $K'$ is the trivial knot in $S^3$, $\Sigma(K')$ is 
homeomorphic to $S^3$. We know that $\mathcal{N}(\alpha)$ is a 3-ball intersecting $K'$ in two arcs, and then 
$p^{-1}(\mathcal{N}(\alpha))$ is a solid torus, and $p^{-1}(\partial \mathcal{N}(\alpha))$ is a surface of 
genus one. Therefore, $S^3 \backslash \, p^{-1}(\mathcal{N}(\alpha))^\circ$ is a double cover of 
$S^3 \backslash\, \mathcal{N}(\alpha)^\circ$ branched along $K \cap (S^3 \backslash \, \mathcal{N}(\alpha))^\circ$. So to finish building the double cover of $S^3$ branched along the knot $K$, 
all we have to do is to refill $S^3 \backslash \, p^{-1}(\mathcal{N}(\alpha))^\circ$ appropriately.

Note that there exists a compressing disk for $\partial (\mathcal{N}(\alpha)) \backslash K$ contained in 
$\mathcal{N}(\alpha) \backslash K$; we denote this disk by $D$ (see Figure 12). 
As $K \cap D = \emptyset$ then $\vert K' \cap D\vert$ is an even number, so the curve
$\partial D$ is lifted by $p$ into two curves in $p^{-1}(\partial \mathcal{N}(\alpha))$; we
denote these two curves by $\Lambda_1$ and $\Lambda_2$. Let $\Sigma'$ be the 3-manifold obtained by
gluing two 2-handles to the 3-manifold $S^3 \backslash \, p^{-1}(\mathcal{N}(\alpha))^\circ$, glued over regular
neighborhoods of $\Lambda_1$ and $\Lambda_2$; we denote these 
2-handles by $\overline{\Lambda}_1$ and $\overline{\Lambda}_2$ respectively. So 
$\Sigma' =[S^3 \backslash p^{-1}(\mathcal{N}(\alpha)^\circ)]\cup[\overline{\Lambda}_1 \cup \overline{\Lambda}_2]$.



We know that $\Lambda_1 \cup \Lambda_2$ is a double cover of $\partial D$ with covering function given
by $p \vert_{\Lambda_1 \cup \Lambda_2}$. So we can extend the function $p \vert_{\Lambda_1 \cup \Lambda_2}$
to $\overline{\Lambda}_ 1\cup \overline{\Lambda}_2$, to get that $\overline{\Lambda}_1 \cup \overline{\Lambda}_2$ is a double cover of $\mathcal{N}(D)$. From this follows that $\Sigma'$ is a double cover of 
$[S^3 \backslash p^{-1}(\mathcal{N}(\alpha)^\circ)]\cup \mathcal{N}(D)$ branched along two arcs of $K$.


We have that $\partial {[S^3\backslash \mathcal{N}(\alpha)^\circ]\cup [\mathcal{N}(D)]}$ consists of two 2-spheres and 
$\partial \Sigma'$ also consists of two 2-spheres. Also, the 2-spheres of $\partial \Sigma'$ are a double cover of the 
two spheres of $\partial {[S^3\backslash \mathcal{N}(\alpha)^\circ]\cup [\mathcal{N}(D)]}$ branched over 
the points $K \cap \partial{[S^3\backslash \mathcal{N}(\alpha)^\circ]\cup [\mathcal{N}(D)]}$. 

Now we can fill the sphere boundary components of $\Sigma'$ with 3-balls, and extend the function $p$ to these 3-balls in order to get the double covering of $S^3$ branched along the knot $K$.

The idea described above is known as the Montesinos trick. Similar to the previous construction, 
we will build the double covers of $S^3$ branched over the knots for which we 
know the tunnel number or the transit number. The following lemma is a general result of coverings which we will use often. The proof is a standard argument, we omit it.

\begin{lemma} \label{conexo} Let $M$ be a given 3-manifold. Let $\Sigma$ be a double cover of $M$ with covering function 
$p : \Sigma \rightarrow M$; and let $C \subset M$. If $M$ is path connected and $p^{-1}(C)$ is connected then $\Sigma$ is connected.
\end{lemma}



The following theorem is our first important result of this section. We will see that if we know the transient number of a knot we can construct the double cover of $S^3$ branched over this knot and from there calculate its first homology group.

\begin{theorem}\label{cubiertageneral} If $K$ is a knot in $S^3$ such that $tr(K) = n$, then the first homology group of the double cover of $S^3$ branched along $K$ has a presentation with at most $2n + 1$ generators.
\end{theorem}

\begin{proof} Let $K$ be a knot in $S^3$ such that $tr(K) = n$, let $\{ \tau_1,\tau_2,...,\tau_n\}$ be a transient system for $K$, and let $T =\mathcal{N}(K\cup \tau_1 \cup \tau_2 \cup, \dots, \cup \tau_n)$. Let 
$K' \subset T$ be a trivial knot in $S^3$ such that $K'$ is homotopic to $K$ in $T$.

Let us define a family of compressing disks for $\partial T$ properly embedded in $T$, say 
$\{D_1, D_2, \dots , D_n, D_{n+1}\}$, which satisfy the following properties (see figure 13):
\begin{itemize}

\item For each $i \in \{1, 2, . . . , n\}$ the disk $D_i$ is properly embedded in $\mathcal{N}(\tau_i)$.
\item The disk $D_{n+1}$ is properly embedded in $\mathcal{N}(K)$.
\end{itemize}

All of these disks are properly embedded in $T$, so we can deduce that:
\begin{enumerate}
\item The family $\{D_1, D_2, \dots, D_n, D_{n+1}\}$ is pairwise disjoint. 
\item For each $i \in \{1,2,\dots,n\}$, $\vert D_i \cup K \vert = 0$.
\item $\vert D_{n+1} \cap  K \vert = 1$.
\end{enumerate}

Let $\Sigma[K']$ be the double cover of $S^3$ branched along $K'$ with covering function given by
$P : \Sigma[K'] \rightarrow S^3$. Note that $\Sigma[K']$ is homeomorphic to $S^3$.

\begin{claim}\label{disjointcurves} For each $i \in \{1, 2, \dots, n\}$, $P^{-1}(\partial D_i)$ has exactly two connected
components, where each connected component is a simple closed curve
in $P^{-1}(\partial T)$; whereas $P^{-1}(\partial D_{n+1})$ is a simple closed connected curve in
$P^{-1}(\partial T)$. Also, all these curves are disjoint in $\partial T$. \end{claim}

\begin{proof} We know that $\vert D_{n+1} \cap K \vert = 1$ and $\vert D_i \cap K \vert = 0$ for all 
$i \in \{1,2,\dots,n\}$. As $K'$ is homotopic to $K$ in $T$, then $\vert D_{n+1}\cap K'\vert$ is an odd integer and 
$\vert D_i \cap K'\vert$ is an even integer for all $i \in \{1,2,\dots,n\}$. 
Therefore, for each $i \in \{1,2,...,n\}$ we have that $P^{-1}(\partial D_i)$ has exactly two connected
components in $P^{-1}(\partial T)$, where each connected component is a simple closed curve; and
$P^{-1}(\partial D_{n+1})$ is a simple closed connected curve in $P^{-1}(\partial T)$.
Now, since the disks of the family $\{D_1, D_2, \dots, D_{n+1}\}$ are pairwise disjoint, 
we have that all the curves are pairwise disjoint. \end{proof}

\begin{claim}\label{genus} $P^{-1}(\partial T)$ is a connected, orientable surface with Euler characteristic $-4n$ 
(and genus $2n + 1$) contained in $\Sigma[K']$. \end{claim}

\begin{proof} $T$ is a genus $n+1$ surface, then $\chi(\partial T) = -2n$, and therefore 
$\chi (P^{-1}(\partial T))) = 2\chi(\partial T ) = -4n$.
Since $\partial T$ is connected, $P^{-1}(\partial T)$ is a double cover of 
$\partial T$,  $\partial D_{n+1} \subset \partial T$ and $P^{-1}(\partial D_{n+1})$ is a connected curve on $P^{-1}(\partial T)$, then by Lemma z\ref{conexo} we have that $P^{-1}(\partial T)$ is connected. Therefore $P^{-1}(\partial T)$ is a connected orientable surface of Euler characteristic $-4n$ (and of genus $2n + 1$).\end{proof}


\begin{claim}\label{otroconexo} $P^{-1}(\partial T \backslash \cup ^{n}_{j=1} \partial D_j)$ is connected. \end{claim}

\begin{proof} Clearly $\partial  T \backslash \cup^{n}_{j=1}\partial D_j$ is connected. We have that 
$P^{-1}(\partial T  \backslash \cup ^{n}_{j=1} \partial D_j)$ is a double cover of $\partial T \backslash \cup^{n}_{j=1}\partial D_j$, 
 that $\partial D_{n+1} \subset \partial T \backslash \cup ^{n}_{j=1} \partial D_j$ and that
$P^{-1}(\partial D_{n+1})$ is a connected curve on $P^{-1}(\partial T \backslash \cup ^{n}_{j=1} \partial D_j$), then using Lemma \ref{conexo} we have that $P^{-1}(\partial T \backslash \cup^{n}_{j=1}\partial D_j)$ is connected. \end{proof}



By Claim \ref{disjointcurves} we know that for each $i \in \{1,2,\dots,n\}$ the curve
$\partial D_i$ lifts, under $P$, to exactly two simple closed curves in $P^{-1}(\partial T)$. 
Let us denote by $\Lambda^{i}_1$
and $\Lambda^{i}_2$ the two lifting of $\partial D_i$ in $P^{-1}(\partial T)$, so 
$\{\Lambda^1_1,\Lambda^1_2,\Lambda^2_1,\Lambda^2_2,...,\Lambda^n_1,\Lambda^n_2\}$ is a pairwise disjoint family 
of simple closed curves in $P^{-1}(\partial T)$. Also $\Lambda^i_1 \cup \Lambda^i_2$ is a double cover of 
$\partial D_i$ with $P\vert_{\Lambda^i_1\cup \Lambda^i_2}$ the respective covering function, then the functions
$P\vert_{\Lambda^i_1}: \Lambda^i_1 \rightarrow \partial D_i$ and $P \vert_{\Lambda^i_2} : \Lambda^i_2 \rightarrow \partial D_i$ 
are homeomorphisms.

By Claim \ref{disjointcurves} we have that $P^{-1}(D_{n+1})$ is a simple closed curve on $P^{-1}(\partial T)$. Let us denote by $\Lambda$ the curve $P^{-1}(\partial D_{n+1})$. So $\Lambda$ is a double cover for $\partial D_{n+1}$ with covering function 
$P \vert_\Lambda : \Lambda \rightarrow \partial D_{n+1}$.



Let us introduce the following notations: 

\begin{itemize}

\item  $Ext(T):=S^3 \backslash Int(T)$,
\item $\Sigma [Ext(T)] := \Sigma [K'] \backslash P^{-1}(Int(T))$,
\end{itemize}



Note that $\Sigma[Ext(T)]$ is a double cover of $Ext(T)$ with covering function given by $P\vert _{\Sigma [Ext(T)]}$. 
Note also that $\partial \Sigma [Ext(T )] = P^{-1}(\partial T)$.



Let $\Sigma [Ext(K)]$ be the 3-manifold obtained from $\Sigma [Ext(T)]$ by gluing a 2-handle to a regular neighborhood, 
in $P^{-1}(\partial T)$, of each of the members of the family of curves 
$\{\Lambda^1_1, \Lambda^1_2, \Lambda^2_1, \Lambda^2_2, \dots, \Lambda^n_1, \Lambda^n_2\}$. Since the functions 
$P \vert_{\Lambda^i_r}$ are homeomorphisms for each $i \in \{1,2,\dots,n\}$ and $r \in \{1,2\}$, we can extend these 
homeomorphisms to homeomorphisms whose domains are discs whose boundaries are $\Lambda^i_r$, which map
to the disks $D_i$. We then extend these last homeomorphisms to homeomorphisms from the 2-handle added to the regular neighborhood of $\Lambda^i_r$ to $\mathcal{N}(D_i)$. With this we conclude that $\Sigma[Ext(K)]$ is a double cover of 
$Ext(T) \cup ^n_{j=1} \mathcal{N}(D_j)$. Recall that the family of disks $\{D_1, D_2, \dots, D_n\}$ was chosen such that 
$Ext(T ) \cup (\cup ^n_{j=1} \mathcal{N}(Dj ))$ is homeomorphic to $Ext(K)$. Therefore $\Sigma [Ext(K)]$ is a double cover of $Ext(K)$.

On the other hand, from Claim \ref{genus} we know $P^{-1}(\partial T)$ is an orientable connected surface of genus $2n + 1$ 
and by Claim \ref{otroconexo} we know that $P^{-1}(\partial T\backslash \cup^n_{j=1} \partial D_j)$ is connected. Since 
$\{\Lambda^i_1,\Lambda^1_2,\Lambda^2_1,\Lambda^2_2,...,\Lambda^n_1,\Lambda^n_2\}$ consist of $2n$ curves and

$$P^{-1}(\partial T\backslash \cup ^n_{j=1} \partial D) = P^{-1}(\partial T)\backslash \cup_{i\in \{1,2,...,n\}  r\in \{1,2\}} \Lambda ^i_r ,$$

\noindent then $\partial\Sigma[Ext(K)]$ is an orientable surface of genus one.

Now, note that $\partial D_{n+1} \subset \partial Ext(K)$ since $\partial D_{n+1} \subset \partial \mathcal{N}(K)$ and 
$D_{n+1} \cap  D_i = \emptyset$ for all $i \in \{1,2,...,n\}$. Therefore we also have $\Lambda \subset \partial\Sigma[Ext(K)]$.

Let us define the 3-manifold $\Sigma[K]$ obtained from $\Sigma[Ext(K)]$ by adding a 2-handle along a regular neighborhood of
$\Lambda$ on $\partial \Sigma[Ext(K)]$, and then complete with a 3-ball so that $\Sigma[K]$ is a closed 3-manifold. 
Since $P_\Lambda$ is a two-to-one covering function then we can extend the function $P_\Lambda$ to a function that goes from a disk, whose boundary is $\Lambda$, to the disk $D_{n+1}$, where this extension is two-to-one branched at the point $K \cap D_{n+1}$. 
This last function is then extended to a function that goes from the 2-handle added to the regular neighborhood of $\Lambda$ to 
$\mathcal{N}(D_{n+1})$, where this function is two to one branched along the arc $K \cap \mathcal{N}(D_{n+1})$. 
Finally, this last function is 
extended to the added 3-ball, thus obtaining a function that goes from $\Sigma[K]$ to $S^3$ which is two to one branched along the knot 
$K$. From the above we conclude that $\Sigma[K]$ is a double cover of $S^3$ branched along $K$.


Now we know from Claim \ref{genus} that $P^{-1}(\partial T)$ is an orientable connected surface of genus $2n + 1$ contained in 
$S^3$. Since $\partial \Sigma[Ext(T )] = P^{-1} (\partial T)$ and $\Sigma[Ext(T )] \subset \Sigma[K'] = S^3$ then 
$H_1(\Sigma[Ext(T)])$ is a free abelian group of rank $2n+1$. So, let 
$H_1(\Sigma[Ext(T)])=<\theta_1,\theta_2,...,\theta_{2n+1}>$,
where $\theta_i$ for $i \in \{1,2,...,2n + 1\}$ are generators.




Thus,
$H_1(\Sigma[K]) =< \theta_1,\theta_2,...,\theta_{2n+1} \, \vert \, \lambda_1^1,\lambda_1^2,\lambda_2^1, \lambda_2^2,...,\lambda_n^1,\lambda_n^2,\lambda >$,
where $\lambda$ and the $\lambda^j_r$, for $j \in \{1,2,...,n\}$ and $r \in \{1,2\}$, correspond to the words generated by the homology classes in $H_1(\Sigma[Ext(T)])$ of the respective curves $ \Lambda$ and $\Lambda_r^j$.
\end{proof}


It should be noted that in the proof of Theorem \ref{cubiertageneral}, besides from proving
the result, we construct the double cover of $S^3$ branched along the knot for which we know the transient number.
This construction will continue to be repeated throughout this work.
The next lemma is a general result of algebra of groups, which we will use for the proof of Theorems \ref{cubiertatunel1} and \ref{cubiertatransito1}.

\begin{lemma}\label{grupos}. Let $G_1$ and $G_2$ be abelian groups such that

$G_1 =< \theta_1,\theta_2,\theta_3 : \lambda_1,\lambda_2,\lambda_3 >$ and $G_2 =< \beta_1,\beta_2 : \delta_1,\delta_2 >$.

Let $\Psi :< \theta_1,\theta_2,\theta_3 >\rightarrow < \theta_1,\theta_2,\theta_3 >$ and 
$\Phi :< \theta_1,\theta_2,\theta_3 >\rightarrow < \beta_1,\beta_2 >$ be homomorphisms between free abelian groups such that:

$$\begin{array}{ccccc}
\Psi(\theta_1) = \theta_2 & \Psi(\lambda_1) = \lambda_2 & \vert & \Phi(\theta_1) = \beta_1 & \Phi(\lambda_1) = \delta_1 \\
 \Psi(\theta_2) = \theta_1 & \Psi(\lambda_2) = \lambda_1 & \vert & \Phi(\theta_2) = \beta_1 & \Phi(\lambda_2) = \delta_1 \\
 \Psi(\theta_3) = \theta_3  &  \Psi(\lambda_3) = \lambda_3 & \vert &  \Phi(\theta_3) = 2\beta_2 & \Phi(\lambda_3) = 2\delta_2
 \end{array}$$
 
 
 
If $\lambda_1 = x\theta_1 + y\theta_2 + z\theta_3$ and $G_2$ is the trivial group, then $G_1$ is isomorphic to $Z_{x-y}$.
\end{lemma}

\begin{proof} Let $a_{ij}$ be integers, with $i, j \in \{1, 2, 3\}$, such that: 
\begin{equation} \label{sistem1}
\begin{split}
\lambda_1 = a_{11}\theta_1 + a_{12}\theta_2 + a_{13}\theta_3 \\
\lambda_2 = a_{21}\theta_1 + a_{22}\theta_2 + a_{23}\theta_3 z  \\
\lambda_3 = a_{31}\theta_1 + a_{32}\theta_2 + a_{33}\theta_3
\end{split}
\end{equation}

Applying the homomorphism $\Psi$, on both sides of the previous system of equations, we obtain:

\begin{equation} \label{sistem2}
\begin{split}
\lambda_2 = \Psi(\lambda_1) = \Psi(a_{11}\theta_1 + a_{12}\theta_2 + a_{13}\theta_3) = a_{11}\theta_2 + a_{12}\theta_1 + a_{13}\theta_3 \\
\lambda_1 = \Psi(\lambda_2) = \Psi(a_{21}\theta_1 + a_{22}\theta_2 + a_{23}\theta_3) = a_{21}\theta_2 + a_{22}\theta_1 + a_{23}\theta_3 \\
\lambda_3 = \Psi(\lambda_3) = \Psi(a_{31}\theta_1 + a_{32}\theta_2 + a_{33}\theta_3) = a_{31}\theta_2 + a_{32}\theta_1 + a_{33}\theta_3
\end{split}
\end{equation}

Of the system (\ref{sistem1}) and from the system obtained in (\ref{sistem2}) we obtain:

\begin{equation} \label{sistem3}
\begin{split}
0 = (a_{11} - a_{22})\theta_1 + (a_{12} - a_{21})\theta_2 + (a_{13} - a_{23})\theta_3 \\
0 = (a_{12} - a_{21})\theta_1 + (a_{11} - a_{22})\theta_2 + (a_{13} - a_{23})\theta_3  \\
0 = (a_{31} - a_{32})\theta_1 + (a_{32} - a_{31})\theta_2
\end{split}
\end{equation}

Since $< \theta_1,\theta_2,\theta_3 >$ is a free abelian group, then from the system in (\ref{sistem3}) we obtain:

$a_{11} = a_{22}, \quad a_{12} = a_{21}, \quad a_{13} = a_{23}, \quad a_{31} = a_{32}$


Then rewriting the system (\ref{sistem1}) we obtain

\begin{equation} \label{sistem4}
\begin{split}
\lambda_1 = a_1\theta_1 + a_2\theta_2 + a_3\theta_3 \\
\lambda_2 = a_2\theta_1 + a_1\theta_2 + a_3\theta_3  \\
\lambda_3 = a_4\theta_1 + a_4\theta_2 + a_5\theta_3 \\
\end{split}
\end{equation}

\noindent where $a_1=a_{11}$, $a_2=a_{12}$, $a_3=a_{23}$, $a_4=a_{31}$ and $a_5=a_{33}$.
Applying the homomorphism $\Phi$ to the system (\ref{sistem4}) we obtain:

\begin{equation} \label{sistem5}
\begin{split}
\delta_1 =\Phi(\lambda_1)=\Phi(a_1\theta_1 +a_2\theta_2 +a_3\theta_3) = (a_1 +a_2)\beta_1 +2a_3\beta_2     \\
\delta_1 = \Phi(\lambda_2) = \Phi(a_2\theta_1 + a_1\theta_2 + a_3\theta_3) = (a_2 + a_1)\beta_1 + 2a_3\beta_2  \\ 
\delta_2 = \Phi(\lambda_3) = \Phi(a_4\theta_1 + a_4\theta_2 + a_5\theta_3) = 2a_4\beta_1 + 2a_5\beta_2
\end{split}
\end{equation}

By property of free abelian groups we obtain from the last equation of the system (23) that:
$$\delta_2 = a_4\beta_1 + a_5\beta_2$$. 

So the system in (\ref{sistem5}) we can rewrite:

\begin{equation} \label{sistem6}
\begin{split}
\delta_1 = (a_1 + a_2)\beta_1 + 2a_3\beta_2.  \\ 
\delta_2 = a_4\beta_1 + a_5\beta_2
\end{split}
\end{equation}

From the system (\ref{sistem6}) we see that the matrix $A$, given by: 


$$A = \begin{pmatrix} a_1+a_2  & 2a_3 \\ a_4  & a_5 \end{pmatrix}$$
is the representation matrix of the group $ < \beta_1, \beta_2 : \delta_1, \delta_2 >$. From the system
in (\ref{sistem4}), doing an operation on rows, we see that the matrix $\tilde A$, given by:


$$\tilde A= \begin{pmatrix} a_1 & a_2 & a_3 \\ a_1+a_2 & a_1+a_2 & 2a_3 \\ a_4 & a_4 & a_5 \end{pmatrix}$$
is a representation matrix of the group $< \theta_1, \theta_2, \theta_3 : \lambda_1, \lambda_2, \lambda_3 >$.

By Smith normal form theorem, there exist matrices $S_1$ and $S_2$ of order $2 \times 2$, invertible and with integer entries such 
that the matrix $S_1AS_2$ is a diagonal matrix with integer entries. From Smith normal form theorem it is also known that the 
inverse matrices of $S_1$ and $S_2$ have integer entries, therefore $det S_1 = \pm 1$ and $det S_2 = \pm 1$. Now, since $G_2$ is the trivial 
group, then $det A = \pm 1$. So the matrix $S_1AS_2$ is of the form 

$$S_1AS_2 = \begin{pmatrix} \pm 1 & 0 \\ 0 & \pm 1 \end{pmatrix}$$
From (25) we can ensure that there is a matrix $S$ of order $2 \times 2$, invertible and
with integer entries that satisfies:

$$SA= \begin{pmatrix} 0 &1 \\ 1 &0 \end{pmatrix}$$

Let us define the following matrix:

$$\tilde S = \begin{pmatrix}1 & 0 & 0 \\ 0 & S & S\\ 0 & S & S \end{pmatrix}$$
Clearly the matrix $\tilde S$ has integer entries and using the result in (26) we have:


$$\tilde S \tilde A= \begin{pmatrix} a_1 & a_2 &  a_3 \\ 1 & 1 &  0 \\ 0 & 0 & 1 \end{pmatrix}$$

Using elementary operations, from the matrix in (27) we obtain:

$$\begin{pmatrix} a_1-a_2 & 0 & 0 \\ 0 & 1 &  0 \\ 0 & 0 & 1 \end{pmatrix}$$



From the above matrix we conclude that the group $< \theta_1, \theta_2, \theta_3 : \lambda_1, \lambda_2, \lambda_3 >$ is
isomorphic to $Z_{a_1-a_2}$, therefore the group $G_1$ is isomorphic to $Z_{a_1-a_2}$. \end{proof}


In the following theorem we will prove that if a knot
has tunnel number one, then the first homology group of the double cover of $S^3$ branched along the knot is cyclic. This result, 
apart from being interesting, will help us as a lemma in the proof of Theorem \ref{cubiertatransito}.

\begin{theorem} \label{cubieretatunel} If $K$ is a knot in $S^3$ such that $t(K) = 1$ then the first homology group of the double cover of $S^3$ branched along $K$ is cyclic. \end{theorem}

\begin{proof} Let $K$ be a knot in $S^3$ such that $t(K) = 1$, let ${\tau}$ be an unknotting tunnel for $K$, so that 
$S^3 \backslash \mathcal{N}(K \cup \tau)^\circ$ is a handlebody. Let $T = \mathcal{N}(K \cup \tau)$ and 
$Ext(T ) = S^3 \backslash T^\circ$, so $Ext(T)$ is a handlebody.
Since $Ext(T)$ is a handlebody, we can ensure that there exists a knot $K' \subset T$ such that $K'$ is a trivial knot in $S^3$ and 
furthermore it is homotopic with the knot $K$ in $T$. Let $\Sigma_2(K')$ be the double cover of $S^3$ branched along the knot 
$K'$ and let $p : \Sigma_2(K') \rightarrow S^3$ be the associated covering function.
It is easy to notice, for the way it is defined $T$,  that there are meridian disks $D_1$ and $D_2$ in $T$ such that 
$\vert D_1 \cap K\vert = 0$ and 
$\vert D_2 \cap K\vert = 1$. Since $K'$ is homotopic to $K$ in $T$, then $\vert D_1 \cap K' \vert$ is an even integer and 
$\vert D_2 \cap K'\vert$ is an odd integer. Therefore $\partial D_1$ lifts, under $p$, in two simple closed curves; while 
$\partial D_2$ lifts to exactly a single simple closed curve. Let us denote by $\Lambda_1$ and 
$\Lambda_2$ the lifting of $\partial D_1$ and by $\Lambda_3$ the lifting of $\partial D_2$.
For each $i \in \{1,2,3\}$ we glue a 2-handle to $p^{-1}(Ext(T))$ over a regular neighborhood of 
$\Lambda_i \in \partial(p^{-1}(Ext(T)))$; let us denote the 2-handle attached to the regular neighborhood of $\Lambda_i$ by 
$\overline{\Lambda}_i$. Let $\Sigma$ be the 3-manifold obtained by attaching to $p^{-1}(Ext(T))$ the 2-handles 
$\overline{\Lambda}_i$, 
that is: $\Sigma := p^{-1}(Ext(T))\cup_{i=1}^3\overline{\Lambda}_i$.

Let us note the following observations: 
\begin{enumerate}

\item $\partial p^{-1}(Ext(T))$ is a genus three connected surface.
\item $p^{-1}(Ext(T))$ is a double covering of $Ext(T)$. %with covering function given by $p\vert_1{p^{-1} (Ext(T ))}$.
\item The function $p$ can be extended to $\Sigma$, such that $\overline{\Lambda}_1 \cup \overline{\Lambda}_2$ is a double 
covering of $\mathcal{N}(D_1)$ and $\Lambda_3$ is a double covering of $\mathcal{N}(D_2)$ branched along 
$K \cap \mathcal{N}(D_2)$.
\item $\partial\Sigma$ is a 2-sphere.

\end{enumerate}

Let $\Sigma(K)$ be the 3-manifold obtained by gluing a 3-ball to $\Sigma$ along its
boundary. So we can extend the covering function $p\vert_ {p^{-1}(Ext(T))} : p^{-1}(Ext(T)) \rightarrow Ext(T)$ to a covering 
function $p': \Sigma(K) \rightarrow S^3$ which branches along the knot $K$. Therefore $\Sigma (K)$ is the double covering of 
$S^3$ branched along $K$ with covering function given by $p'$.

We know that $p^{-1}(Ext(T))$ is a double covering of $Ext(T)$, with covering function given by the restriction of $p$. Let 
$p_*: H_1(p^{-1}(Ext(T))) \rightarrow H_1(Ext(T))$ be the
homomorphism associated with the restriction of $p$. For each $i \in \{1, 2, 3\}$ let us denote
by $\lambda_i$ the homology class in $H_1(p^{-1}(Ext(T)))$ associated to the curve $\lambda_i$. For
each $j \in \{1, 2\}$ let us denote by $\delta_j$ the homology class in $H_1(Ext(T))$ associated to the curve $\partial D_j$, 
such that $p_*(\lambda_1) = \delta_1$,  $p_*(\lambda_2) = \delta_1$, $p_*(\lambda_3) = 2\delta_2$.

On the other hand, we know that $Ext(T) \subset S^3$ and $\partial(Ext(T))$ is a connected surface of genus two, therefore $H_1(Ext(T))$ is a free abelian group in two generators. Likewise, we know that $p^{-1}(Ext(T)) \subset S^3$, since 
$p^{-1}(Ext(T)) \subset \Sigma_2(K')$ and as $K'$ is the trivial knot of $S^3$ then $\Sigma_2(K') = S^3$. Note that 
$\partial(p^{-1}(Ext(T)))$ is a connected surface of genus three, therefore $H_1(p^{-1}(Ext(T)))$ is a free abelian group in three generators.

\begin{claim} There are two connected simple closed curves in $Ext(T)$, denoted by $B_1$ and $B_2$, 
such that $B_1$ lifts, by $p$, in two  closed and connected simple curves, denoted by $\Theta_1$ 
and $\Theta_2$; while $B_2$ lifts, by $p$, in exactly one simple curve closed, denoted 
by $\Theta_3$. If $\beta_j$ is the homology class of $B_j$ in $H_1(Ext(T))$ and $\theta_i$ is the homology 
class of $\Theta_i$ in $H_1(p^{-1}(Ext(T)))$ for all $j \in \{1,2\}$ and $i \in \{1,2,3\}$, then
$H_1(Ext(T))=<\beta_1,\beta_2 > , H_1(p^{-1}(Ext(T)))=<\theta_1,\theta_2,\theta_3 >$. \end{claim}

\begin{proof} Let $C$ be a simple closed curve in $\partial Ext(T)$ that separates $\partial Ext(T)$ and that 
is boundary of a meridional disk in $Ext(T)$, we know
that such a curve $C$ exists since $Ext(T)$ is a handlebody. So each connected
component of $(\partial Ext(T )) \backslash C$ is a genus one surface minus a closed disk.
We also have that the curve $C$ lifts, by the covering function $p$, in
two simple closed curves, since $C$ is boundary of a compression
disk that is contained in $Ext(T)$ and therefore it is disjoint from the knot $K$. 
Let us denote these lifts by $C_1$ and $C_2$. So we have the
following two possibilities:

Figure 14: In the case that each $C_j$ divides $\partial p^{-1}(Ext(T))$

\begin{enumerate}

\item $C_j$ divides $\partial(p^{-1}(Ext(T )))$ for all $j \in \{1, 2\}$.
\item $C_j$ does not divide $\partial(p^{-1}(Ext(T )))$ for some $j \in \{1, 2\}$.

\end{enumerate}

Consider the first situation. Suppose that each of the $C_j$,
for $j \in \{1, 2\}$, divides $\partial(p^{-1}(Ext(T )))$, such that $\partial(p^{-1}(Ext(T ))) \backslash (C_1 \cup C_2)$
has three connected components. Two of these components, say $S_1$, $S_2$ is a genus one surface minus a disk, and the third
component, say $S_3$, is a genus one surface minus two disks. The tori $S_1$, $S_2$ cover, by means of $p$, 
a single connected component of $\partial((Ext(T ))) \backslash C$, while $S_3$
covers 2 to 1, the remaining connected component of  $\partial (Ext(T ))\backslash C$. Let $B_1$
and $B_2$ longitudes of $S_1$ and $S_2$ respectively. Without loss of generality assume that
$B_1$ is in the connected componet of $\partial(Ext(T )) \backslash C$ that is covered by $S_1$ and $S_2$,
and that $B_2$ is in the boundary component of $\partial(Ext(T)) \backslash C$ 
that is covered by $S_3$. As $Ext(T)$ and $P^{-1}(Ext(T))$ are handlebodies, the claim is satisfied.



Consider the second situation, in which there exists $j \in \{1,2\}$ such that $C_j$ does not divide $\partial(p^{-1}(Ext(T)))$. 
Suppose without loss of generality that
$C_1$ does not divide $\partial (p^{-1}(Ext(T)))$. Note that, if there is a simple closed curve which intersect only once
$C_1$, then, as $C$ divides $\partial Ext(T)$, such a curve have to intersect $C_2$, so $C_2$ does not divide
$\partial (p^{-1}(Ext(T)))$. Likewise, as $C$ divides $\partial Ext(T)$ then $C_1\cup C_2$ divides $\partial (p^{-1}(Ext(T )))$. 
In summary, we have that for each $j \in \{1, 2\}$, $C_j$ does not divide
$\partial (p^{-1}(Ext(T )))$; while $C_1 \cup C_2$ divides $\partial(p^{-1}(Ext(T )))$. Therefore
$\partial (p^{-1}(Ext(T ))) \backslash (C_1 \cup C_2)$ has two connected components, where each one of these
is a genus one surface minus two disks. So, each of the connected components of $\partial (p^{-1}(Ext(T ))) \backslash C_1 \cup C_2$
covers 2 to 1, by means of $p$, exactly one of the connected components of $\partial (Ext(T ) \backslash C$. Let $B_1^*$ and
$B_2$ be longitudes of each of the connected components of $\partial (Ext(T ) \backslash C$. 
Take a simple arc $\alpha$, 
in $\partial Ext(T)$, that connects the curve $B_1^*$ with the curve $B_2$ and that intersects only once the curve $C$.
Consider $N(\alpha \cup B_1^* \cup B_2$ in $T$, this is a surface with three boundary components, where one of them
intersects $C$ twice, and let $B_1$ be such a curve. As $Ext(T)$ and $P^{-1}(Ext(T))$ are handlebodies, in this 
situation the claim for $B_1$ and $B_2$ is satisfied. See Figure 15.

\end{proof}

According to Claim 4, we have that

$$P(\theta_1= \beta_1 \, P(\theta_2)=\beta_1) \, P(\theta_3)=2\beta_2$$ (29)

Let $q:p^{-1}(Ext(T)) \rightarrow p^{-1}(Ext(T))$ the covering transformation, different to the indentity function 
in $p^{-1}(Ext(T ))$, associated to the covering function $p\vert_{p^{-1}(Ext(T ))}$. Let 
$Q:H_1(p^{-1}(Ext(T))) \rightarrow H_1(p^{-1}(Ext(T)))$ the homomorphism induced by the covering transformation$q$. 
By (29) and (28) we have that: 

$$Q(\theta_1) = \theta_2 \, Q(\theta_2) = \theta_1 \, Q(\theta_3) = \theta_3$$

$$Q(\lambda_1) = \lambda_2 \, Q(\lambda_2) = \lambda_1 \, Q(\lambda_3) = \lambda3$$

Now as $H_1(S^3)$ is trivial, $H_1(Ext(T )) =< \beta_1, \beta_2 >$, furthermore $\vert D_1 \cap K \vert =0$ and
$\vert D_2 \cap K \vert = 1$, then the abelian group with the presentation  $< \beta_1,\beta_2 : \delta_1,\delta_2 >$
is the trivial group. Then, applying directly Lemma 5 we have that $H_1(\Sigma_2(K)) = Z_{x-y}$,
where $\lambda1 = x\theta_1 + y\theta_2 + z\theta_3$.
\end{proof}

This is the main result of this paper.

\begin{theorem} \label{cubiertatransito} If $K$ is a knot in $S^3$ such that $tr(K) = 1$ then the first homology group of the double cover of $S^3$ branched along $K$ is cyclic. \end{theorem}

\begin{proof} Let $K$ be a knot in $S^3$ such that $tr(K) = 1$, and let $\{\tau \}$ be a tunnel transit system for the knot $K$. Let $T = \mathcal{N}(K \cup \tau)$ and let $K' \subset T$ be a trivial knot in $S^3$ such that $K'$ is homotopic to $K$ in $T$. Define also the 3-manifold $Ext(T)$ as $Ext(T) := S^3 \ Int(T)$.

As $\partial T$is a genus two surface in the exterior of the knot $K'$, which is trivial, it follows that $\partial T$
is compressible in $Ext(K')$, that is there is a compression disk $D_1$ for $\partial T$ disjoint from $K'$.

There are two possibilities for the disk $D_1$, that is: 
\begin{enumerate}

\item The disk $D_1$ is a compression disk for $T$ lying in the interior of $T$; 
\item The disk $D_1$ is a compression disk for $T$ lying in the exterior of $T$.
\end{enumerate}
Suppose first that we have case (1), that is, $D_1$ lies in the interior of $T$.
%Podemos asumir que el disco B1 no separa a T, ya que de lo contrario T \\mathcal{N}B1 tendr ́ıa dos componentes conexas y K' estar ́ıa contenido en solo una de estas componentes conexas. Luego tomando un disco de compresio ́n interior a la componente de T \ \mathcal{N} B1 que no contiene a K' vemos que este u ́ltimo disco sera ́ un disco de compresi ́on interior a T, disjunto de K' y que adema ́s no separa a T.

\begin{claim} There exist a knot $K''$ and a disk $D_2$ in $T$ such that:

\begin{enumerate}
\item $D_2$ is a compression disk for $\partial T$ which is properly embedded in $T$.
\item $K''$ is a trivial knot in $S^3$ and it is homotopic to $K$ in $T$.

\item $\vert D_1 \cap K'' \vert = 0$.			
\item $\vert D_2 \cap K'' \vert = 1$.

\end{enumerate} \end{claim}

\begin{proof} First note that $H_1(T)$ is isomorphic to $\mathbb{Z}\times \mathbb{Z}$. Denote by $⟨K⟩$ and $⟨K'⟩$ the elements of
the group $H_1(T )$ corresponding to the homology classes of $K$ and $K'$, where clearly $⟨K⟩$ is one of the generators of $H_1(T)$. 
Let $⟨U⟩ \in H_1(T)$ such that $\{⟨K⟩,⟨U⟩\}$ is a generator set for the group $H_1(T)$. As $K$ is homotopic to $K'$ in $T$,
then $⟨K⟩ = ⟨K'⟩$. On the other hand note that $H_1(T \ \mathcal{N}(D_1))$ is isomorphic to $\mathbb{Z}$, for $B_1$ is a compression 
disk in $T$ which is non-separating. Let $V$ be the spine of $T \ \mathcal{N}(D_1)$ and let $< V >$ be the homology class of 
$V$ in $H_(T)$. Note that $⟨V ⟩$ correspond to the generator of $H_1(T \ \mathcal{N}(D_1))$, so that we can ensure that there exists
$a \in \mathbb{Z}$ such that $⟨K'⟩ = a⟨V ⟩$, so $K' \subset T \ \mathcal{N}(D_1)$ for $D_1 \cap K' = \emptyset$.
Note that $a\not= 0$, for otherwise we have that $⟨K'⟩ = 0 \in H_1(T)$, which is not possible for $⟨K'⟩ = ⟨K⟩$ and $⟨K⟩$ is a generator
of $H_1(T)$. Now, as $⟨V⟩ \in H_1(T)$, there exist $b,c \in \mathbb{Z}$ such that $⟨V⟩ = b⟨K⟩+c⟨U⟩$. Then we have that
$⟨K⟩ = ⟨K'⟩ = a⟨V ⟩ = a(b⟨K⟩ + c⟨U⟩) = (ab)⟨K⟩ + (ac)⟨U⟩$, and as $a\not= 0$, then $c=0$and >$ab=1$. We conclude that $⟨K⟩=±⟨V⟩$.
Now we see that $V$ is a trivial knot in $S^3$. Note that $K \subset T \ \mathcal{N}(D_1)$ and as $T \ \mathcal{N}(D_1)$
is homeomorphic to $\mathcal{N}(V)$ then $K'$ is a satellite knot of $V$, and as $K'$ is the trivial knot of $S^3$ then $V$ is
the trivial knot in $S^3$. From this we have that $V$ is homotopic to $K$ in $T$ for $<V > = \pm <K>$.

Taking $K''$ as the knot $V$ and taking the disk $D_2$ as a compression disk properly embedded in $T \ \mathcal{N}(D_1)$
we see that:
\begin{enumerate}

\item $D_2$ ia compression disk properly embedded in $T$.
\item $K''$ in the trivial knot in $S^3$, which is homotopic to $K$ in $T$. 
\item $\vert D_1 \cap K'' \vert = 0$, for $K'' \subset T\ \mathcal{N}(D_1)$.
\item $\vert D_2 \cap K'' \vert = 1$, for $T \ \mathcal{N}(D_1)$ is homeomorphic to $\mathcal{N}(V)$.
\end{enumerate}
\end{proof}

Let $\Sigma[K'']$ the double cover of $S^3$ branched along $K''$ with covering function given by $p:\Sigma[K ] \rightarrow S^3$.
Let $\{D_1,D_2\}$ compression disks in the interior of $T$ such that $D_1$ is properly embedded in $\mathcal{N}(\tau)$ y $D_2$
is properly embedded in $\mathcal{N}(K)$, such that $\vert D_1 \cap K \vert = 0$ and $\vert D_2 \cap K \vert=1$. Note that the disks
$D_1$ and  $D_2$ do not separate $T$. As $K''$ is homotopic to $K$ in $T$, then $\vert D_1 \cap K'' \vert$ is an even number ana
$\vert D_2 \cap K'' \vert$ is an odd number. Therefore $\partial D_1$ lift, under $p$, in two simple closed curves, while 
$\partial D_2$ lifts exactly in a single simple closed curve. Denote by $\Lambda_1$ y $\Lambda_2$ the liftings of $\partial D_1$
and by $\Lambda_3$ the lifting of $\partial D_2>$.
Attach ther 2-handles to the 3-manifold $p^{-1}(Ext(T))$ in the following way: For each $i \in \{1,2,3\}$ attach a 2-handle to a
regular neighborhood of $\Lambda_i$ in $\partial (p^{-1}(Ext(T)))$, denote the 2-handle attached to the regular neighborhood of
$\Lambda_i$ by $\over \Lambda_i$. Let $\Sigma$ the 3-manifold obtained by attaching to $p^{-1}(Ext(T))$ the 2-handles $\Lambda_i$, 
that is:

$$\Sigma_{i=1} := p^{-1}(Ext(T)) \Lambda_i$$

Observe the following:
\begin{enumerate}
\item $\partial p^{-1}(Ext(T))$ is a connected surface of genus three.
\item $p^{-1}(Ext(T))$ is a doble covering of $Ext(T)$, with covering function given by $p\vert_{p^{-1} (Ext(T ))}$.
\item $\overline \Lambda_1 \cup \overline \Lambda_2$ is a double covering of $\mathcal{N}(D_1)$ y $\over \Lambda3$
is a double covering of $\mathcal{N}(D_2)$ branched along $K \cap \mathcal{N}(D_2)$.
\item $\partial \Sigma<$ is a 2-sphere.
\end{enumerate}

Let $\Sigma_2(K)$ be the 3-manifold obtained by attaching along its boundary a 3-ball to $\Sigma$. We can extend
the covering function $p \vert_{p^{-1}(Ext(T))} : p^{-1}(Ext(T)) \rightarrow Ext(T)$ to a covering function 
$\hat p: \Sigma_2(K) \rightarrow S^3$, which branchs along $K$. Therefore $\Sigma_2(K)$ is a double cover of $S^3$
branched along $K$ with covering function given by $\hat p$.

We know that $p^{-1}(Ext(T))$ is a double covering of $Ext(T$), with covering function given by $p\vert -1$. 
Let $P: H_1(p^{-1}(Ext(T))) \rightarrow H_1(Ext(T))$ the homomorphism induced by $p\vert -1$. For each $i \in \{1,2,3\}$
denote by $\lambda_i$ the homolo class in $H_1(p^{-1}(Ext(T)))$ associated to the curve $\Lambda_i$.
For each $j \in \{ 1,2\}$ denote by $\delta_j$ to the homology class in $H_1(Ext(T))$ associated to the curve $\partial D_j$. 
Then

$$P(\lambda_1) = \delta_1 \, P(\lambda_2) = \delta_1 \, P(\lambda_3) = 2\delta_2 \, \,(30)$$

Note that $H_1(Ext(T))$ is a free abelian group in two generators. On the other hand, as $\partial p^{-1}(Ext(T))$ is a connected
surface of genus three, $p^{-1}(Ext(T)) \subset \Sigma[K'']$ and furthermore $\Sigma[K'']$ is homeomorphic $S^3$, for $K''$
is the trivial knot in $S^3$, we conclude that $H_1(p^{-1}(Ext(T)))$ is a free abelian group in there generators. 
By claim (5) we know that $D_1$ lift, by means of $p$, in two simple closed curves, denote such liftings by 
$\Theta_1$ and $\Theta_2$; while $D_2$ lifts, by means of $p$, in a single simple closed curve, denoted by $\Theta_3$. 
Such that if $\beta_j$ is the homology class of $D_j$ in $H_1(Ext(T))$ and $\theta_i$ is the homology class of $\Theta_i$ in 
$H_1(p^{-1}(Ext(T)))$ for all $j \in \{1,2\}$ and $i \in \{1,2,3\}$, then:

$$H_1(Ext(T))=<\beta1,\beta2 > \, \, H_1(p^{-1}(Ext(T)))=<\theta1,\theta2,\theta3 >$$

We also obtain that 

$$P(\theta_1) = \beta_1 \, P(\theta_2) = \beta_1 \, P(\theta_3) = 2\beta_2 \,\,(31)$$ 

Let $q: p^{-1}(Ext(T)) \rightarrow p^{-1}(Ext(T))$ the covering transformation, different to the identity function in $p^{-1}(Ext(T ))$, 
associated to the covering function $p\vert_ {p^{-1} (Ext(T ))}$.
Let $Q : H_1(p^{-1}(Ext(T))) \rightarrow H_1(p{-1}(Ext(T)))$ the homeomorphism associated to the covering transformation $q$. 
By the way that $\theta_i$ and the $\lambda_i$ were defined we have that:

$$Q(\theta_1) = \theta_2 \,\, Q(\theta_2) = \theta_1 \,\, Q(\theta_3) = \theta_3 \,\, Q(\lambda_1) = \Lambda_2 \,\, Q(\lambda2) = \lambda1 \,\, Q(\lambda3) = \lambda3$$

Now as $H_1(S^3)$ is trivial, $H_1(Ext(T )) =< \beta_1, \beta_2 >$ and furthermore $\vert D_1 \cap K \vert = 0$ and 
$\vert D_2 \cap K \vert = 1$ then the abelian group with presentation $$< \beta_1,\beta_2 : \delta_1,\delta_2 > (32)$$ is the trivial group. 
Applying Lemma (5) we have that $H_1(\Sigma_2(K)) = \mathbb{Z}_{x-y}$, where $\lambda_1 = x\theta_1 + y\theta_2 + z\theta_3$.
So, we have proved that if the disk $D_1>$ is contained in $T$, then the homology group of the double covering of $S^3$ branched
along $K$ is cyclic.

Now suppose that $D_1$ is contained in $Ext(T)$ and we will get the same result. In this situation we can suppose that $Ext(T)$
is not a handlebody, for otherwise we have that $t(K) = 1$ and by Theorem 5 we get the desired result. From  $T$ we construct
a genus one handlebody, denoted by $\Gamma$, obtained by attaching handles (2-handles or 3-handles) to $T$; this is done oin the 
following way: if $B_1$ does not divide $\partial T>$ then define $\Gamma= T \cup \mathcal{N}(D_1)$, as $D_1$ does not
 divide $\partial T$ then $\partial \Gamma$ is a connected genus one surface. If $D_1$ divides $\partial T$ then
$S^3 \ Int(T \cup \mathcal{N}(D_1))$ is a 3-manifold with two connected components and as $Ext(T)$ is not a handlebody then
one of these components is not a handlebody. In this case we define $\Gamma$ as the 3-manifold obtained from $S^3$
from removing the interior of $S^3 \ Int(T \cup \mathcal{N}(D_1))$ the component which is not a handlebody.
Note that in any case we have that $Ext(\Gamma)$ is not a handlebody. Let $C$ be the core of $\Gamma$, so that
$\Gamma = \mathcal{N}(C)$. Now as $Ext(\Gamma)$ is not a handlebody then $C$ is not a trivial knot.

it is easy to see that $K$ is homotopic in $\Gamma$ to a point in $\Gamma$. For otherwise $K'$ is a satellite of $C$ and
as $K'$ is the trivial knotthen $C$ will ba a trivial knot, which is not possible.

By the Fox embedding Theorem, we know that there exist an homeomorphism $\Omega: S^3 \rightarrow S^3$,
such that $\Omega(\Gamma) = \Gamma'$ and  $Ext(\Gamma')$ is a handlebody. Let $K''$ be a knot in $S^3$
such that $\Omega(K) = K''$. Como $Ext(\Gamma)$ is a handlebody and as $\Gamma$ is obtained by attaching handles
(2-handles or 3-handles) to $T$ then $t(K'') = 1$. Denote by
$\Sigma_2(K)$ the double cover of $S^3$ branched along $K$ and by $\Sigma_2(K'')$ the double cover of $S^2$ branched along
the knot $K''$. As $t(K ) = 1$ then by Theorem 5 we have that $H_1(\Sigma_2(K''))$ is cyclic.
If $K$ is a satellite of the knot $C$ then the knot $K''$ is the pattern of $K$ and therefore the grupo $H_1(\Sigma_2(K))$ is
isomorphic to $H_1(\Sigma_2(K''))$, from here we obtain that $H_1(\Sigma_2(K))$ is cyclic. If $K$ is not a satellie knot
of $C$, then there is a compression disk in $\Gamma'$ that does not intersect $K$: then there is a 3-ball in $\Gamma'$
that contains $K$, this implies that $K''$ is isotopic to $K$ and then $H_1(\Sigma_2(K))$ is cyclic.
\end{proof}

\end{document}


Problemas abiertos
Entre la preguntas que han dejado Koda y Ozawa en su trabajo [7], estamos interesados en algunos de los problemas planteados ah ́ı:
(a) El desanudamiento de una 3-subvariedad en una 3-variedad arbi- traria conexa y cerrada, y la existencia de un reembebimiento de Fox han sido generalizados de varias formas. Scharlemann-Thompson en [8] expusieron una versio ́n de dicha generalizaci ́on y Nakamura en [9] expuso otra versio ́n. Nos interesa saber si el teorema 1 que Koda y Ozawa probaron en [7] puede ser generalizado para una 3-subvariedad M en una 3-variedad arbitraria N.
(b) En general, dada una subvariedad M de S3, y un nudo K contenido en M, es dif ́ıcil determinar si K es transitorio o persistente en M. Hasta el momento so ́lo hay dos criterios para determinar esto. El primero de estos criterios lo expuso Letscher [6] usando homotop ́ıa persistente, homolog ́ıa persistente y el mo ́dulo de alexander persis- tente. El segundo de estos criterios lo propusieron Koda y Ozawa [7] en el cual usaron superficies incompresibles o laminaciones es- enciales contenidas en M. Uno de nuestros objetivos es el de dar nuevos criterios para determinar si un nudo es transitorio o persis- tente en M.
En el teorema 3 probamos que
tr(K1#K2) 􏰟 tr(K1) + tr(K2) + 1.
Para le caso del nu ́mero de tu ́nel sabemos que hay muchos resultado y ejemplos alrededor de la desigualdad
t(K1#K2) 􏰟 t(K1) + t(K2) + 1.
Como ejemplo de estos resultados podemos citar el trabajo de Morimoro en [12]. Usando la notaci ́on planteada por Morimoto en sus art ́ıculos, queremos saber si, al igual que para el nu ́mero de tu ́nel, existe una familia de nudos en S3 para los cuales el nu ́mero de tra ́nsito cumple con la desigualdad estricta, con la igualdad, con la propiedad 2 + 1 = 2, con la propiedad 1 + 1 = 3 etc.
Para los casos del nu ́mero de desanudamiento y el nu ́mero de tu ́nel exis- ten afirmaciones cla ́sicas conocidas; entre las cuales destacamos la sigu- iente: Si un nudo tiene nu ́mero de desanudamiento uno o nu ́mero de tu ́nel uno entonces dicho nudo es un nudo primo. Ser ́ıa interesante saber si para el caso del nu ́mero de tra ́nsito tambi ́en se satisface la afirmacio ́n ana ́loga; es decir, si un nudo tiene tra ́nsito uno entonces dicho nudo es primo.
2.
3.
39

References
[1] R. H. Fox. “On the embedding of polyhedra in 3-space”. Ann. Math, (1948), 49(2): 462-470.
[2] R.H.Bing.“Necessaryandsufficientconditionsthata3-manifoldbeS3”,Ann.ofMath. (1958), (2) 68, 17-37.
[3] J. Hass and A. Thompson, “A necessary and sufficient condition for a 3-manifold to have Heegaard genus one”, Proc. Amer. Math. Soc. (1989), 107:4, 1107-1110.
[4] D. Rolfsen. “Knots and Links”. American Mathematical Society, (1976), 17-47.
[5] R. Lickorish. “An Introduction to Knot Theory”. Springer, 1997.
[6] D. Letscher, “On Persistent Homotopy, Knotted Complexes and the Alexander Mod- ule”, ITCS’12 Proceeding of the 3rd Innovations in Theoretical Computer Science Con- ference, ACM New York, NY, USA, (2012), 428-441.
[7] Koda, Y., Ozawa, M. “Knot Homotopy in Subspaces of the 3-Sphere”.
[8] M. Scharlemann and A. Thompson, “Surfaces, submanifolds, and aligned Fox re-
imbedding in non-Haken 3-manifolds”, Proc. Amer. Math. Soc. (2005), 133:6, 1573-1580.
[9] K. Nakamura “Fox reimbedding and Bing submanifold”, Trans. Amer. Math. Soc. (2015), 367:12, 8325-8346.
[10] C. P. Roueke and B. J. Sanderson “Introduction to Piece-Linear Topology”. Springer-Verlag, 1982
[11] R. B. Sher and R. J. Daverman “Handbook of Geometric Topology”. ELSEVIER, 2002.
[12] Morimoto, Kanji “Tunnel numbers of knots”. 2016.
[13] Montesinos, Jose ́ Mar ́ıa “Surgery on links and double branched covers of S .”. 1975.
40

, remember that we want to build the double
cover of $S^3$ branched over the knot $K$. Clearly $\Sigma(K') \ p-1(Int \mathcal{N}(\alpha))$ is
a double cover of $S^3 \ Int \mathcal{N}(\alpha)$ branched over $K' \cap (S^3 \ Int \mathcal{N}(\alpha))$ with
covering function given by $p:􏰞\Sigma(K')\p-1(Int \mathcal{N}(\alpha))$. Considering the equation (18)
we can rewrite the previous statement as follows: $\Sigma(K') \ p-1(Int \mathcal{N}(\alpha))$ is
a double cover of $S^3 \ Int \mathcal{N}(\alpha)$ branched over $K \cap (S^3 \ Int \mathcal{N}(\alpha))$ with
covering function given by $p 􏰞\Sigma(K')\p-1(Int\mathcal{N}(\alpha))$. 

with covering function given by 
$p\vert_{S^3\backslash p^{-1}(Int\mathcal{N}(\alpha))}$

Now, since there exist boundary-compressing disks in 
$\mathcal{N}(\alpha)$ for arcs $K\cap \mathcal{N}(\alpha)$, then the double cap $\Sigma'$ of $[S3\ Int\mathcal{N}(\alpha)]\cup [\mathcal{N}(D)]$ branched over $K\cap (S^3\ Int\mathcal{N}(\alpha))$ can be extended to a double covering of $S^3$ branched over the knot $K$.




nte 
Facultad de Ciencias UNAM 















In the following theorem we will prove that if a knot
has tunnel number one, then the first homology group of the double cover of $S^3$ branched along the knot is cyclic. This result, 
apart from being interesting, will help us as a lemma in the proof of Theorem \ref{cubiertatransito}.

\begin{theorem} \label{cubieretatunel} If $K$ is a knot in $S^3$ such that $t(K) = 1$ then the first homology group of the double cover of $S^3$ branched along $K$ is cyclic. \end{theorem}

\begin{proof} Let $K$ be a knot in $S^3$ such that $t(K) = 1$, and let ${\tau}$ be an unknotting tunnel for $K$. 
Let $T = \mathcal{N}(K \cup \tau)$ and 
$Ext(T ) = S^3 \backslash T$, so $Ext(T)$ is a handlebody.
Since $Ext(T)$ is a handlebody, we can ensure that there exists a knot $K' \subset T$ such that $K'$ is a trivial knot in $S^3$ and 
furthermore it is homotopic with the knot $K$ in $T$. Let $\Sigma_2(K')$ be the double cover of $S^3$ branched along the knot 
$K'$ and let $p : \Sigma_2(K') \rightarrow S^3$ be the associated covering function.
It is easy to notice, for the way it is defined $T$,  that there are meridian disks $D_1$ and $D_2$ in $T$ such that 
$\vert D_1 \cap K\vert = 0$ and 
$\vert D_2 \cap K\vert = 1$. Since $K'$ is homotopic to $K$ in $T$, then $\vert D_1 \cap K' \vert$ is an even integer and 
$\vert D_2 \cap K'\vert$ is an odd integer. Therefore $\partial D_1$ lifts, under $p$, in two simple closed curves; while 
$\partial D_2$ lifts to exactly a single simple closed curve. Let us denote by $\Lambda_1$ and 
$\Lambda_2$ the lifting of $\partial D_1$ and by $\Lambda_3$ the lifting of $\partial D_2$.
For each $i \in \{1,2,3\}$ we attach a 2-handle to $p^{-1}(Ext(T))$ along
$\Lambda_i \in \partial(p^{-1}(Ext(T)))$; let us denote the 2-handle attached along $\Lambda_i$ by 
$\overline{\Lambda}_i$. Let $\Sigma$ be the 3-manifold obtained by attaching to $p^{-1}(Ext(T))$ the 2-handles 
$\overline{\Lambda}_i$, 
that is: $\Sigma := p^{-1}(Ext(T))\cup_{i=1}^3\overline{\Lambda}_i$.

Let us note the following observations: 
\begin{enumerate}

\item $\partial p^{-1}(Ext(T))$ is a genus three connected surface.
\item $p^{-1}(Ext(T))$ is a double covering of $Ext(T)$. %with covering function given by $p\vert_1{p^{-1} (Ext(T ))}$.
\item The function $p$ can be extended to $\Sigma$, such that $\overline{\Lambda}_1 \cup \overline{\Lambda}_2$ is a double 
covering of $\mathcal{N}(D_1)$ and $\Lambda_3$ is a double covering of $\mathcal{N}(D_2)$ branched along 
$K \cap \mathcal{N}(D_2)$.
\item $\partial\Sigma$ is a 2-sphere.

\end{enumerate}

Let $\Sigma(K)$ be the 3-manifold obtained by attaching a 3-ball to $\Sigma$ along its
boundary. So, we can extend the covering function $p\vert_ {p^{-1}(Ext(T))} : p^{-1}(Ext(T)) \rightarrow Ext(T)$ to a covering 
function $p': \Sigma(K) \rightarrow S^3$ which branches along the knot $K$. Therefore $\Sigma (K)$ is the double covering of 
$S^3$ branched along $K$ with covering function given by $p'$.

We know that $p^{-1}(Ext(T))$ is a double covering of $Ext(T)$, with covering function given by the restriction of $p$. Let 
$p_*: H_1(p^{-1}(Ext(T))) \rightarrow H_1(Ext(T))$ be the
homomorphism associated with the restriction of $p$. For each $i \in \{1, 2, 3\}$ let us denote
by $\lambda_i$ the homology class in $H_1(p^{-1}(Ext(T)))$ associated to the curve $\lambda_i$. For
each $j \in \{1, 2\}$ let us denote by $\delta_j$ the homology class in $H_1(Ext(T))$ associated to the curve $\partial D_j$, 
such that $p_*(\lambda_1) = \delta_1$,  $p_*(\lambda_2) = \delta_1$, $p_*(\lambda_3) = 2\delta_2$.

On the other hand, we know that $Ext(T) \subset S^3$ and $\partial(Ext(T))$ is a connected surface of genus two, therefore $H_1(Ext(T))$ is a free abelian group in two generators. Likewise, we know that $p^{-1}(Ext(T)) \subset S^3$, since 
$p^{-1}(Ext(T)) \subset \Sigma_2(K')$ and as $K'$ is the trivial knot of $S^3$ then $\Sigma_2(K') = S^3$. Note that 
$\partial(p^{-1}(Ext(T)))$ is a connected surface of genus three, therefore $H_1(p^{-1}(Ext(T)))$ is a free abelian group in three generators.

\begin{claim} There are two connected simple closed curves in $Ext(T)$, denoted by $B_1$ and $B_2$, 
such that $B_1$ lifts, by $p$, in two  closed and connected simple curves, denoted by $\Theta_1$ 
and $\Theta_2$; while $B_2$ lifts, by $p$, in exactly one simple curve closed, denoted 
by $\Theta_3$. If $\beta_j$ is the homology class of $B_j$ in $H_1(Ext(T))$ and $\theta_i$ is the homology 
class of $\Theta_i$ in $H_1(p^{-1}(Ext(T)))$ for all $j \in \{1,2\}$ and $i \in \{1,2,3\}$, then
$H_1(Ext(T))=<\beta_1,\beta_2 > , H_1(p^{-1}(Ext(T)))=<\theta_1,\theta_2,\theta_3 >$. \end{claim}

\begin{proof} Let $C$ be a simple closed curve in $\partial Ext(T)$ that separates $\partial Ext(T)$ and that 
is boundary of a meridional disk in $Ext(T)$, we know
that such a curve $C$ exists since $Ext(T)$ is a handlebody. So each connected
component of $(\partial Ext(T )) \backslash C$ is a genus one surface minus a closed disk.
We also have that the curve $C$ lifts, by the covering function $p$, in
two simple closed curves, since $C$ is boundary of a compression
disk that is contained in $Ext(T)$ and therefore it is disjoint from the knot $K$. 
Let us denote these lifts by $C_1$ and $C_2$. So we have the
following two possibilities:

Figure 14: In the case that each $C_j$ divides $\partial p^{-1}(Ext(T))$

\begin{enumerate}

\item $C_j$ divides $\partial(p^{-1}(Ext(T )))$ for all $j \in \{1, 2\}$.
\item $C_j$ does not divide $\partial(p^{-1}(Ext(T )))$ for some $j \in \{1, 2\}$.

\end{enumerate}

Consider the first situation. Suppose that each of the $C_j$,
for $j \in \{1, 2\}$, divides $\partial(p^{-1}(Ext(T )))$, such that $\partial(p^{-1}(Ext(T ))) \backslash (C_1 \cup C_2)$
has three connected components. Two of these components, say $S_1$, $S_2$ is a genus one surface minus a disk, and the third
component, say $S_3$, is a genus one surface minus two disks. The tori $S_1$, $S_2$ cover, by means of $p$, 
a single connected component of $\partial((Ext(T ))) \backslash C$, while $S_3$
covers 2 to 1, the remaining connected component of  $\partial (Ext(T ))\backslash C$. Let $B_1$
and $B_2$ longitudes of $S_1$ and $S_2$ respectively. Without loss of generality assume that
$B_1$ is in the connected componet of $\partial(Ext(T )) \backslash C$ that is covered by $S_1$ and $S_2$,
and that $B_2$ is in the boundary component of $\partial(Ext(T)) \backslash C$ 
that is covered by $S_3$. As $Ext(T)$ and $P^{-1}(Ext(T))$ are handlebodies, the claim is satisfied.



Consider the second situation, in which there exists $j \in \{1,2\}$ such that $C_j$ does not divide $\partial(p^{-1}(Ext(T)))$. 
Suppose without loss of generality that
$C_1$ does not divide $\partial (p^{-1}(Ext(T)))$. Note that, if there is a simple closed curve which intersect only once
$C_1$, then, as $C$ divides $\partial Ext(T)$, such a curve have to intersect $C_2$, so $C_2$ does not divide
$\partial (p^{-1}(Ext(T)))$. Likewise, as $C$ divides $\partial Ext(T)$ then $C_1\cup C_2$ divides $\partial (p^{-1}(Ext(T )))$. 
In summary, we have that for each $j \in \{1, 2\}$, $C_j$ does not divide
$\partial (p^{-1}(Ext(T )))$; while $C_1 \cup C_2$ divides $\partial(p^{-1}(Ext(T )))$. Therefore
$\partial (p^{-1}(Ext(T ))) \backslash (C_1 \cup C_2)$ has two connected components, where each one of these
is a genus one surface minus two disks. So, each of the connected components of $\partial (p^{-1}(Ext(T ))) \backslash C_1 \cup C_2$
covers 2 to 1, by means of $p$, exactly one of the connected components of $\partial (Ext(T ) \backslash C$. Let $B_1^*$ and
$B_2$ be longitudes of each of the connected components of $\partial (Ext(T ) \backslash C$. 
Take a simple arc $\alpha$, 
in $\partial Ext(T)$, that connects the curve $B_1^*$ with the curve $B_2$ and that intersects only once the curve $C$.
Consider $N(\alpha \cup B_1^* \cup B_2$ in $T$, this is a surface with three boundary components, where one of them
intersects $C$ twice, and let $B_1$ be such a curve. As $Ext(T)$ and $P^{-1}(Ext(T))$ are handlebodies, in this 
situation the claim for $B_1$ and $B_2$ is satisfied. See Figure 15.

\end{proof}

According to Claim 4, we have that

$$P(\theta_1= \beta_1 \, P(\theta_2)=\beta_1) \, P(\theta_3)=2\beta_2$$ (29)

Let $q:p^{-1}(Ext(T)) \rightarrow p^{-1}(Ext(T))$ the covering transformation, different to the indentity function 
in $p^{-1}(Ext(T ))$, associated to the covering function $p\vert_{p^{-1}(Ext(T ))}$. Let 
$Q:H_1(p^{-1}(Ext(T))) \rightarrow H_1(p^{-1}(Ext(T)))$ the homomorphism induced by the covering transformation$q$. 
By (29) and (28) we have that: 

$$Q(\theta_1) = \theta_2 \, Q(\theta_2) = \theta_1 \, Q(\theta_3) = \theta_3$$

$$Q(\lambda_1) = \lambda_2 \, Q(\lambda_2) = \lambda_1 \, Q(\lambda_3) = \lambda3$$

Now as $H_1(S^3)$ is trivial, $H_1(Ext(T )) =< \beta_1, \beta_2 >$, furthermore $\vert D_1 \cap K \vert =0$ and
$\vert D_2 \cap K \vert = 1$, then the abelian group with the presentation  $< \beta_1,\beta_2 : \delta_1,\delta_2 >$
is the trivial group. Then, applying directly Lemma 5 we have that $H_1(\Sigma_2(K)) = Z_{x-y}$,
where $\lambda1 = x\theta_1 + y\theta_2 + z\theta_3$.
\end{proof}

This is the main result of this paper.

\begin{theorem} \label{cubiertatransito} If $K$ is a knot in $S^3$ such that $tr(K) = 1$ then the first homology group of the double cover of $S^3$ branched along $K$ is cyclic. \end{theorem}

\begin{proof} Let $K$ be a knot in $S^3$ such that $tr(K) = 1$, and let $\{tau \}$ be a tunnel transit system for the knot $K$. Let
$T = \mathcal{N}(K \cup \tau)$ and let $K' \subset T$ be a trivial in $S^3$ such that $K'$ is homotopic to $K$ in $T$. Define also 
the 3-manifold $Ext(T)$ as $Ext(T) := S^3 \ Int(T)$.
As $K'$ is a trivial knot in $S^3$, there exists a disk $F$ embedded in $S^3$ such that $\partial F = K$ . Take the disk $F$ such that
the number of connected components of  $F \cap \partial T $ is minimal. Note that $F \cap\partial T$ is a collection disjoint simple
closed curves embedded in $F$. Take an innermost curve in  $F$. Such a curve bounds a disk $D_1$, which is a compression
disk for $T$ and it is disjoint from $K'$. 

There are two possibilities for the disk $D_1$, that is: 1) The disk $D_1$ is a compression disk for $T$ lying in the interior of $T$; 
2) it is a compression disk for $T$ lying in the exterior of $T$.

Suppose first that we have case (1), that is, $D_1$ lies in the interior of $T$.
%Podemos asumir que el disco B1 no separa a T, ya que de lo contrario T \\mathcal{N}B1 tendr ́ıa dos componentes conexas y K' estar ́ıa contenido en solo una de estas componentes conexas. Luego tomando un disco de compresio ́n interior a la componente de T \ \mathcal{N} B1 que no contiene a K' vemos que este u ́ltimo disco sera ́ un disco de compresi ́on interior a T, disjunto de K' y que adema ́s no separa a T.

\begin{claim} There exist a knot $K''$ and a disk $D_2$ in $T$ such that:

\begin{enumerate}
\item $D_2$ is a compression disk for $\partial T$ which is properly embedded in $T$.
\item $K''$ is a trivial knot in $S^3$ and it is homotopic to $K$ in $T$.

\item $\vert D_1 \cap K'' \vert = 0$.			
\item $\vert D_2 \cap K'' \vert = 1$.
\end{enumerate} \end{claim}

\begin{proof} First note that $H_1(T)$ is isomorphic to $\mathbb{Z}\times \mathbb{Z}$. Denote by $⟨K⟩$ and $⟨K'⟩$ the elements of
the group $H_1(T )$ corresponding to the homology classes of $K$ and $K'$, where clearly $⟨K⟩$ is one of the generators of $H_1(T)$. 
Let $⟨U⟩ \in H_1(T)$ such that $\{⟨K⟩,⟨U⟩\}$ is a generator set for the group $H_1(T)$. As $K$ is homotopic to $K'$ in $T$,
then $⟨K⟩ = ⟨K'⟩$. On the other hand note that $H_1(T \ \mathcal{N}(D_1))$ is isomorphic to $\mathbb{Z}$, for $B_1$ is a compression 
disk in $T$ which is non-separating. Let $V$ be the spine of $T \ \mathcal{N}(D_1)$ and let $< V >$ be the homology class of 
$V$ in $H_(T)$. Note that $⟨V ⟩$ correspond to the generator of $H_1(T \ \mathcal{N}(D_1))$, so that we can ensure that there exists
$a \in \mathbb{Z}$ such that $⟨K'⟩ = a⟨V ⟩$, so $K' \subset T \ \mathcal{N}(D_1)$ for $D_1 \cap K' = \emptyset$.
Note that $a\not= 0$, for otherwise we have that $⟨K'⟩ = 0 \in H_1(T)$, which is not possible for $⟨K'⟩ = ⟨K⟩$ and $⟨K⟩$ is a generator
of $H_1(T)$. Now, as $⟨V⟩ \in H_1(T)$, there exist $b,c \in \mathbb{Z}$ such that $⟨V⟩ = b⟨K⟩+c⟨U⟩$. Then we have that
$⟨K⟩ = ⟨K'⟩ = a⟨V ⟩ = a(b⟨K⟩ + c⟨U⟩) = (ab)⟨K⟩ + (ac)⟨U⟩$, and as $a\not= 0$, then $c=0$and >$ab=1$. We conclude that $⟨K⟩=±⟨V⟩$.
Now we see that $V$ is a trivial knot in $S^3$. Note that $K \subset T \ \mathcal{N}(D_1)$ and as $T \ \mathcal{N}(D_1)$
is homeomorphic to $\mathcal{N}(V)$ then $K'$ is a satellite knot of $V$, and as $K'$ is the trivial knot of $S^3$ then $V$ is
the trivial knot in $S^3$. From this we have that $V$ is homotopic to $K$ in $T$ for $<V > = \pm <K>$.

Taking $K''$ as the knot $V$ and taking the disk $D_2$ as a compression disk properly embedded in $T \ \mathcal{N}(D_1)$
we see that:
\begin{enumerate}

\item $D_2$ ia compression disk properly embedded in $T$.
\item $K''$ in the trivial knot in $S^3$, which is homotopic to $K$ in $T$. 
\item $\vert D_1 \cap K'' \vert = 0$, for $K'' \subset T\ \mathcal{N}(D_1)$.
\item $\vert D_2 \cap K'' \vert = 1$, for $T \ \mathcal{N}(D_1)$ is homeomorphic to $\mathcal{N}(V)$.
\end{enumerate}
\end{proof}

Let $\Sigma[K'']$ the double cover of $S^3$ branched along $K''$ with covering function given by $p:\Sigma[K ] \rightarrow S^3$.
Let $\{D_1,D_2\}$ compression disks in the interior of $T$ such that $D_1$ is properly embedded in $\mathcal{N}(\tau)$ y $D_2$
is properly embedded in $\mathcal{N}(K)$, such that $\vert D_1 \cap K \vert = 0$ and $\vert D_2 \cap K \vert=1$. Note that the disks
$D_1$ and  $D_2$ do not separate $T$. As $K''$ is homotopic to $K$ in $T$, then $\vert D_1 \cap K'' \vert$ is an even number ana
$\vert D_2 \cap K'' \vert$ is an odd number. Therefore $\partial D_1$ lift, under $p$, in two simple closed curves, while 
$\partial D_2$ lifts exactly in a single simple closed curve. Denote by $\Lambda_1$ y $\Lambda_2$ the liftings of $\partial D_1$
and by $\Lambda_3$ the lifting of $\partial D_2>$.
Attach ther 2-handles to the 3-manifold $p^{-1}(Ext(T))$ in the following way: For each $i \in \{1,2,3\}$ attach a 2-handle to a
regular neighborhood of $\Lambda_i$ in $\partial (p^{-1}(Ext(T)))$, denote the 2-handle attached to the regular neighborhood of
$\Lambda_i$ by $\over \Lambda_i$. Let $\Sigma$ the 3-manifold obtained by attaching to $p^{-1}(Ext(T))$ the 2-handles $\Lambda_i$, 
that is:

$$\Sigma_{i=1} := p^{-1}(Ext(T)) \Lambda_i$$

Observe the following:
\begin{enumerate}
\item $\partial p^{-1}(Ext(T))$ is a connected surface of genus three.
\item $p^{-1}(Ext(T))$ is a doble covering of $Ext(T)$, with covering function given by $p\vert_{p^{-1} (Ext(T ))}$.
\item $\overline \Lambda_1 \cup \overline \Lambda_2$ is a double covering of $\mathcal{N}(D_1)$ y $\over \Lambda3$
is a double covering of $\mathcal{N}(D_2)$ branched along $K \cap \mathcal{N}(D_2)$.
\item $\partial \Sigma<$ is a 2-sphere.
\end{enumerate}

Let $\Sigma_2(K)$ be the 3-manifold obtained by attaching along its boundary a 3-ball to $\Sigma$. We can extend
the covering function $p \vert_{p^{-1}(Ext(T))} : p^{-1}(Ext(T)) \rightarrow Ext(T)$ to a covering function 
$\hat p: \Sigma_2(K) \rightarrow S^3$, which branchs along $K$. Therefore $\Sigma_2(K)$ is a double cover of $S^3$
branched along $K$ with covering function given by $\hat p$.

We know that $p^{-1}(Ext(T))$ is a double covering of $Ext(T$), with covering function given by $p\vert -1$. 
Let $P: H_1(p^{-1}(Ext(T))) \rightarrow H_1(Ext(T))$ the homomorphism induced by $p\vert -1$. For each $i \in \{1,2,3\}$
denote by $\lambda_i$ the homolo class in $H_1(p^{-1}(Ext(T)))$ associated to the curve $\Lambda_i$.
For each $j \in \{ 1,2\}$ denote by $\delta_j$ to the homology class in $H_1(Ext(T))$ associated to the curve $\partial D_j$. 
Then

$$P(\lambda_1) = \delta_1 \, P(\lambda_2) = \delta_1 \, P(\lambda_3) = 2\delta_2 \, \,(30)$$

Note that $H_1(Ext(T))$ is a free abelian group in two generators. On the other hand, as $\partial p^{-1}(Ext(T))$ is a connected
surface of genus three, $p^{-1}(Ext(T)) \subset \Sigma[K'']$ and furthermore $\Sigma[K'']$ is homeomorphic $S^3$, for $K''$
is the trivial knot in $S^3$, we conclude that $H_1(p^{-1}(Ext(T)))$ is a free abelian group in there generators. 
By claim (5) we know that $D_1$ lift, by means of $p$, in two simple closed curves, denote such liftings by 
$\Theta_1$ and $\Theta_2$; while $D_2$ lifts, by means of $p$, in a single simple closed curve, denoted by $\Theta_3$. 
Such that if $\beta_j$ is the homology class of $D_j$ in $H_1(Ext(T))$ and $\theta_i$ is the homology class of $\Theta_i$ in 
$H_1(p^{-1}(Ext(T)))$ for all $j \in \{1,2\}$ and $i \in \{1,2,3\}$, then:

$$H_1(Ext(T))=<\beta1,\beta2 > \, \, H_1(p^{-1}(Ext(T)))=<\theta1,\theta2,\theta3 >$$

We also obtain that 

$$P(\theta_1) = \beta_1 \, P(\theta_2) = \beta_1 \, P(\theta_3) = 2\beta_2 \,\,(31)$$ 

Let $q: p^{-1}(Ext(T)) \rightarrow p^{-1}(Ext(T))$ the covering transformation, different to the identity function in $p^{-1}(Ext(T ))$, 
associated to the covering function $p\vert_ {p^{-1} (Ext(T ))}$.
Let $Q : H_1(p^{-1}(Ext(T))) \rightarrow H_1(p{-1}(Ext(T)))$ the homeomorphism associated to the covering transformation $q$. 
By the way that $\theta_i$ and the $\lambda_i$ were defined we have that:

$$Q(\theta_1) = \theta_2 \,\, Q(\theta_2) = \theta_1 \,\, Q(\theta_3) = \theta_3 \,\, Q(\lambda_1) = \Lambda_2 \,\, Q(\lambda2) = \lambda1 \,\, Q(\lambda3) = \lambda3$$

Now as $H_1(S^3)$ is trivial, $H_1(Ext(T )) =< \beta_1, \beta_2 >$ and furthermore $\vert D_1 \cap K \vert = 0$ and 
$\vert D_2 \cap K \vert = 1$ then the abelian group with presentation $$< \beta_1,\beta_2 : \delta_1,\delta_2 > (32)$$ is the trivial group. 
Applying Lemma (5) we have that $H_1(\Sigma_2(K)) = \mathbb{Z}_{x-y}$, where $\lambda_1 = x\theta_1 + y\theta_2 + z\theta_3$.
So, we have proved that if the disk $D_1>$ is contained in $T$, then the homology group of the double covering of $S^3$ branched
along $K$ is cyclic.

Now suppose that $D_1$ is contained in $Ext(T)$ and we will get the same result. In this situation we can suppose that $Ext(T)$
is not a handlebody, for otherwise we have that $t(K) = 1$ and by Theorem 5 we get the desired result. From  $T$ we construct
a genus one handlebody, denoted by $\Gamma$, obtained by attaching handles (2-handles or 3-handles) to $T$; this is done oin the 
following way: if $B_1$ does not divide $\partial T>$ then define $\Gamma= T \cup \mathcal{N}(D_1)$, as $D_1$ does not
 divide $\partial T$ then $\partial \Gamma$ is a connected genus one surface. If $D_1$ divides $\partial T$ then
$S^3 \ Int(T \cup \mathcal{N}(D_1))$ is a 3-manifold with two connected components and as $Ext(T)$ is not a handlebody then
one of these components is not a handlebody. In this case we define $\Gamma$ as the 3-manifold obtained from $S^3$
from removing the interior of $S^3 \ Int(T \cup \mathcal{N}(D_1))$ the component which is not a handlebody.
Note that in any case we have that $Ext(\Gamma)$ is not a handlebody. Let $C$ be the core of $\Gamma$, so that
$\Gamma = \mathcal{N}(C)$. Now as $Ext(\Gamma)$ is not a handlebody then $C$ is not a trivial knot.

it is easy to see that $K$ is homotopic in $\Gamma$ to a point in $\Gamma$. For otherwise $K'$ is a satellite of $C$ and
as $K'$ is the trivial knotthen $C$ will ba a trivial knot, which is not possible.

By the Fox embedding Theorem, we know that there exist an homeomorphism $\Omega: S^3 \rightarrow S^3$,
such that $\Omega(\Gamma) = \Gamma'$ and  $Ext(\Gamma')$ is a handlebody. Let $K''$ be a knot in $S^3$
such that $\Omega(K) = K''$. Como $Ext(\Gamma)$ is a handlebody and as $\Gamma$ is obtained by attaching handles
(2-handles or 3-handles) to $T$ then $t(K'') = 1$. Denote by
$\Sigma_2(K)$ the double cover of $S^3$ branched along $K$ and by $\Sigma_2(K'')$ the double cover of $S^2$ branched along
the knot $K''$. As $t(K ) = 1$ then by Theorem 5 we have that $H_1(\Sigma_2(K''))$ is cyclic.
If $K$ is a satellite of the knot $C$ then the knot $K''$ is the pattern of $K$ and therefore the grupo $H_1(\Sigma_2(K))$ is
isomorphic to $H_1(\Sigma_2(K''))$, from here we obtain that $H_1(\Sigma_2(K))$ is cyclic. If $K$ is not a satellie knot
of $C$, then there is a compression disk in $\Gamma'$ that does not intersect $K$: then there is a 3-ball in $\Gamma'$
that contains $K$, this implies that $K''$ is isotopic to $K$ and then $H_1(\Sigma_2(K))$ is cyclic.
\end{proof}


Maria de los Angeles Guevara-Hernandez
Gabriela Hinojosa-Palafox
Juan Pablo Diaz-Gonzalez
Bruno Cisneros de la Cruz
Max Neumann-Coto
Araceli Guzman-Tristan
Lorena Armas-Sanabria
Luis Celso Chan-Palomo
Mario Eudave-Muñoz

Jesus Rodriguez-Viorato *
Enrique Ramirez-Losada *


Muchas gracias Mario.
Creo poder probar lo que querías en el caso de 2 generadores y una relación :
1. Sea X_inf  la cubierta cíclica ibfinita del compémento de un nudo k en S3 con polinomio de Añexander Delta . Como se explica en Some aspects of classical knot theorty secciones 4 y 5, si M_2 es la doble cubierta de S3 ramificada sobre k entonces
H_1(M_2) = coker (t^2 -1) : H_1(X_inf) ---> H_1(X_inf)  = coker (t+1): H_1(X_inf) --->H_1(X_inf)
asi que si H_1(X_inf) es un Z(Z)-modulo ciclico , es decir H_1(X_inf) = Z(Z)/Belta  entonces
H_1)X_2) = Z(Z)/<Delta, t+1> = Z_Delta(-1) grupo cíclico de orden Delta(-1).

Despues le sigo para expllicar por que si el grupo de k se describe por 2 generadores y una relacion entonces H_1(X_inf) esun Z(Z)-modulo ciclico.
Saludos y gracias
Fico


Hola Fico,

Te mando aquí la carta membretada con tus datos.

Te la mando en word y en pdf.
La de word se puede modificar por si los datos no están bien.

Faltaría nomás firmarla.

Saludos,
Mario
francisco gonzalez <ficomx@yahoo.com.mx>
	
21 jun 2020, 18:55
	
para mario@matem.unam.mx, ficomx@yahoo.com.mx
Hola Mario :
No se necesita que haya una sola relacion.
Sea G:= ! y_1, y_2 :   s_1, ...,s_m ! el grupo del nudo k (dos generadores).
Sea alfa :  G ---> Z = !t :  !    la abelianización.
Es fácil ver que G puede generarse por {x_1 , a }  con alfa(x_1) = t y alfa(a) trivial (algoritmo de Euclides)l, y por tanto,
 si x_2 = ax_1,    {x_1, x_2} genera G  con alfa(x-1) = alfa(x_2) = t.
Hay entonces una presentación (x_1, x_2 :  r_1,...,r_m) de G.
Por VIII, 3.7 de Introduction to Knot Theory de Crowell-Fox,    H_1(X_inf)  como Z(Z)-modulo está generado por un solo elemento, es decir, 
es un Z(Z)-módulo ciclico.

Saludos
Fico

----- Mensaje reenviado -----
De: francisco gonzalez <ficomx@yahoo.com.mx>
Para: "mario@matem.unam.mx" <mario@matem.unam.mx>; ficomx@yahoo.com.mx <ficomx@yahoo.com.mx>
Enviado: sábado, 20 de junio de 2020 21:46:59 GMT-5
Asunto: Re: cartas

Muchas gracias Mario.
Creo poder probar lo que querías en el caso de 2 generadores y una relación :
1. Sea X_inf  la cubierta cíclica infinita del complémento de un nudo k en S3 con polinomio de Alexander Delta . Como se explica en Some aspects of classical knot theorty secciones 4 y 5, si M_2 es la doble cubierta de S3 ramificada sobre k entonces
H_1(M_2) = coker (t^2 -1) : H_1(X_inf) ---> H_1(X_inf)  = coker (t+1): H_1(X_inf) --->H_1(X_inf)
asi que si H_1(X_inf) es un Z(Z)-modulo ciclico , es decir H_1(X_inf) = Z(Z)/Belta  entonces
H_1)X_2) = Z(Z)/<Delta, t+1> = Z_Delta(-1) grupo cíclico de orden Delta(-1).


10_99	2	2	{{2,{9,9}},{3,{4,4,4,4}},{4,{9,9}},{5,{1}},{6,{2,2,6,6,0,0,0,0}},{7,{1}},{8,{9,9}},{9,{4,4,4,4}}}

10_123	2	2	{{2,{11,11}},{3,{5,5,5,5}},{4,{11,11}},{5,{2,2,2,2,2,2,2,2}},{6,{5,5,55,55}},{7,{29,29,29,29}},{8,{3,3,3,3,3,3,33,33}},{9,{5,5,5,5}}}

12a_427	2	2	{{2,{15,15}},{3,{8,8,8,8}},{4,{3,3,3,3,15,15}},{5,{11,11,11,11}},{6,{4,4,20,20,0,0,0,0}},{7,{29,29,29,29}},{8,{3,3,21,21,105,105}},{9,{152,152,152,152}}}

12a_435	2	2	{{2,{3,75}},{3,{4,4,16,16}},{4,{3,3,9,225}},{5,{121,121}},{6,{2,2,8,200,0,0,0,0}},{7,{841,841}},{8,{3,3,441,11025}},{9,{4,4,5776,5776}}

12a_465	2	2	{{2,{241}},{3,{2,2,38,38}},{4,{17,4097}},{5,{1021,1021}},{6,{2,2,2,2,2,2,38,9158}},{7,{21701,21701}},{8,{6001,1446241}},{9,{2,2,192926,192926}}}
12a_466	2	2	{{2,{209}},{3,{2,2,26,26}},{4,{209}},{5,{619,619}},{6,{2,2,2,2,2,2,26,5434}},{7,{7027,7027}},{8,{191,39919}},{9,{2,2,21554,21554}}}

12a_475	2	2	{{2,{255}},{3,{2,2,40,40}},{4,{15,15,255}},{5,{451,451}},{6,{2,2,2,10,20,340,0,0}},{7,{13,13,377,377}},{8,{255,1785,1785}},{9,{2,2,55480,55480}}}

12a_647	4	2	{{2,{3,51}},{3,{4,4,4,4}},{4,{3,3,3,51}},{5,{7,7,7,7}},{6,{2,2,2,34,0,0,0,0}},{7,{29,29}},{8,{3,3,69,1173}},{9,{4,4,508,508}}}

12a_742	2	2	{{2,{3,45}},{3,{2,2,10,10}},{4,{3,45}},{5,{29,29}},{6,{5,5,15,0,0,0,0}},{7,{43,43}},{8,{69,1035}},{9,{2,2,710,710}}}

12a_801	[3,4]	2	{{2,{3,45}},{3,{2,2,10,10}},{4,{3,45}},{5,{29,29}},{6,{5,5,15,0,0,0,0}},{7,{43,43}},{8,{69,1035}},{9,{2,2,710,710}}}

12a_868	2	2	{{2,{377}},{3,{133,133}},{4,{33,12441}},{5,{2,2,2,2,8,8,88,88}},{6,{665,250705}},{7,{63253,63253}},{8,{16401,6183177}},{9,{19,19,83923,83923}}}

12a_975	2	2	{{2,{5,45}},{3,{91,91}},{4,{5,5,5,5,5,45}},{5,{1271,1271}},{6,{1365,12285}},{7,{14351,14351}},{8,{5,15,15,15,255,765}},{9,{245791,245791}}}

12a_990	2	2	{{2,{3,75}},{3,{4,4,16,16}},{4,{3,3,9,225}},{5,{121,121}},{6,{2,2,8,200,0,0,0,0}},{7,{841,841}},{8,{3,3,441,11025}},{9,{4,4,5776,5776}}}

12a_1019	2	2	{{2,{19,19}},{3,{11,11,11,11}},{4,{5,5,95,95}},{5,{6,6,6,6,6,6,6,6}},{6,{11,11,209,209}},{7,{41,41,41,41}},{8,{5,5,95,95}},{9,{209,209,209,209}}}

12a_1102	2	2	{{2,{305}},{3,{2,2,56,56}},{4,{33,10065}},{5,{2981,2981}},{6,{2,2,2,2,2,2,112,34160}},{7,{81229,81229}},{8,{23793,7256865}},{9,{2,2,1054424,1054424}}}

12a_1105	2	2	{{2,{17,17}},{3,{10,10,10,10}},{4,{5,5,85,85}},{5,{41,41,41,41}},{6,{2,2,2,2,10,10,170,170}},{7,{13,13,13,13,13,13,13,13}},{8,{17,17,85,85,85,85}},{9,{730,730,730,730}}}

12a_1167	2	2	{{2,{313}},{3,{97,97}},{4,{17,5321}},{5,{2,2,2,2,2,2,82,82}},{6,{97,30361}},{7,{8849,8849}},{8,{2737,856681}},{9,{209617,209617}}}

12a_1206	2	2	{{2,{7,35}},{3,{5,5,20,20}},{4,{3,3,63,315}},{5,{11,11,11,11,11,11}},{6,{5,5,5,280,280}},{7,{13,13,377,377}},{8,{21,21,63,315}},{9,{5,5,380,380}}}

12a_1229	2	2	{{2,{221}},{3,{28,28}},{4,{3,663}},{5,{2,2,2,2,8,8,8,8}},{6,{56,12376}},{7,{43,43,43,43}},{8,{17,357,4641}},{9,{19684,19684}}}

12a_1288	[2,3]	2	{{2,{3,39}},{3,{4,4,4,4}},{4,{3,3,3,39}},{5,{31,31}},{6,{2,2,2,26,0,0,0,0}},{7,{113,113}},{8,{3,3,69,897}},{9,{4,4,508,508}}}

12n_518	4	2	{{2,{3,21}},{3,{4,4,8,8}},{4,{7,21,21}},{5,{41,41}},{6,{2,2,4,28,0,0,0,0}},{7,{167,167}},{8,{7,357,357}},{9,{4,4,568,568}}}

12n_533	2	2	{{2,{63}},{3,{2,2,4,4}},{4,{63}},{5,{3,3,3,3}},{6,{2,2,2,2,2,42,0,0}},{7,{1}},{8,{7,7,63}},{9,{2,2,68,68}}}

12n_604	2	2	{{2,{3,27}},{3,{4,4,4,4}},{4,{3,27}},{5,{1}},{6,{2,2,2,18,0,0,0,0}},{7,{1}},{8,{3,27}},{9,{4,4,4,4}}}

12n_605	2	2	{{2,{3,3}},{3,{4,4,4,4}},{4,{15,15}},{5,{11,11}},{6,{2,2,2,2,0,0,0,0}},{7,{29,29}},{8,{3,3,3,3,15,15}},{9,{4,4,76,76}}}

12n_642	[3,4]	3	{{2,{3,3,3}},{3,{23,23}},{4,{3,51,51}},{5,{2,2,158,158}},{6,{3,9,207,207}},{7,{3527,3527}},{8,{3,6477,6477}},{9,{34753,34753}}}

12n_706	2	2	{{2,{7,7}},{3,{2,2,2,2}},{4,{7,7}},{5,{3,3,3,3,3,3,3,3}},{6,{2,2,2,2,2,2,14,14}},{7,{1}},{8,{7,7,7,7,7,7}},{9,{34,34,34,34}}}

12n_840	2	2	{{2,{119}},{3,{2,2,10,10}},{4,{5,595}},{5,{3,3,123,123}},{6,{2,2,2,2,2,2,10,1190}},{7,{13,13,13,13}},{8,{119,595,595}},{9,{2,2,12410,12410}}}

12n_879	2	2	{{2,{143}},{3,{35,35}},{4,{3,429}},{5,{2,2,2,2,4,4,4,4}},{6,{35,5005}},{7,{1247,1247}},{8,{3,3,3,3,21,3003}},{9,{665,665}}}

12n_888	5	2	{{2,{3,15}},{3,{4,4,4,4}},{4,{3,15}},{5,{19,19}},{6,{2,2,2,10,0,0,0,0}},{7,{3,3,3,3,3,3}},{8,{51,255}},{9,{4,4,148,148}}}

We have the following.
 
 \begin{theorem} There exists knots $K_1$ and $K_2$ such that:
 \begin{enumerate}
 \item $tr(K_1)=2$.
 \item $K_2$ is any 2-bridge knot, and then $tr(K_2)=1$.
 \item $tr(K_1\# K_2) =2$.
 \end{enumerate}
 \end{theorem}
 
 \begin{proof} Let $K_1$ be the knot $9_{98}$ or $9_{99}$. It follows from \cite{Knot} that
 for any of these knots $t(K_1)=2$, and that $H_1(\Sigma[K_1])$ is the group $\mathbb{Z}_3 + \mathbb{Z}_{27}$,
or $\mathbb{Z}_{9}+\mathbb{Z}_{9}$, respectively, which is not
 cyclic. From Theorem \ref{cubiertatransito} it follows that $tr(K_1)=2$. As shown in Figure, each of this knots admits
 a Conway sphere which decompose the knot into nontrivial free tangles, at least one of them has
 an unknotted component. It follows from \cite{M} that $t(K_1\# K_2)=2$, and as 
 $H_1(\Sigma[K_1\# K_2])=H_1(\Sigma[K_1])+H_1(\Sigma[K_2])$,  which is not cyclic, then
 $tr(K_1 \# K_2)=2$.
 \end{proof}
 
 There are many other knots of 11 and 12 crossings that satisfy the conclusiones of this theorem.














