\section{Approach}
Our goal is to train robust visuomotor policies in simulation and to deploy them  to manipulation tasks on a real robot.
We first describe our visual policies in Section~\ref{sec:bc} and introduce domain randomization for visual sim-to-real transfer in Section~\ref{sec:dr}. We then present our approach for selecting parameters of domain randomization using a proxy task in Section~\ref{sec:proxy_task}.

\section{Model Architecture}

% Figure environment removed

\subsection{Overview}
\label{sec:methods:overview}
As illustrated in Figure \ref{fig:main_arch}, HUTFormer is based on a hierarchical U-net structure to \textit{generate} and \textit{utilize} multi-scale representations of traffic data.
In this subsection, we intuitively discuss each component of HUTFormer and its two-stage training strategy.

% First, we elaborate on the framework of HUTFormer.
% 
% Hierarchical Encoder
% \noindent
% \textbf{Hierarchical encoder.} 
First, we discuss the hierarchical encoder.
% 
The window Transformer layer is the basis for generating multi-scale representations, {\color{black}which calculates self-attention within a small window to limit the receptive field.}
% 
Then, segment merging acts as a pooling layer, reducing the sequence length to produce larger-scale representations.
% 
By combining them, lower layers can focus on smaller-scale features while higher layers focus on larger-scale features, thus successfully generating multi-scale features.
% 
Then, an intermediate prediction is made based on the top-layer representations.
However, considering that the top-layer features are semantically strong but coarse-grained, 
% the intermediate prediction can basically predict global changes but may fail to capture rapidly changing local details, \eg the red line in Figure \ref{figure:intro}(b).
the intermediate prediction {\color{black}may fail to capture rapidly changing local details, \eg the red line in Figure \ref{figure:intro}(b).}
% 
% , while essentially capturing global changes, may fail in periods of rapid changes, \eg the green line in Figure \ref{figure:intro}(b).
% Hierarchical Decoder

{\color{black}To address the above problem}, the hierarchical decoder aims to fine-tune the intermediate prediction by incorporating multi-scale representations.
% 
% However, as discussed in Section \ref{section:intro}, the history and future sequence in traffic forecasting tasks are not aligned.
% % 
% Thus, the feature sequences extracted by the encoder and the decoder can not be directly superimposed as regular u-net structures~\cite{2015UNet, 2021SwinUNet} do.
% 
U-net~\cite{2015UNet,2021SwinUNet} is a popular structure for utilizing multi-scale representations, especially in computer vision tasks~(\eg semantic segmentation).
In these tasks, the pixels of the input and target images are aligned, \ie models operate on the same image.
% In these tasks, the pixels of input and target image are aligned, \ie models operating on the same image.
% the input and output image are basically the same image, \ie the pixels of input and output image are aligned.
However, for traffic \textit{forecasting} tasks, 
the input and output sequences are {\color{black}not the same sequence, \ie} not aligned.
Thus, the representations generated by the encoder and the decoder cannot be directly superimposed as regular U-net structures~\cite{2015UNet, 2021SwinUNet} do for computer vision tasks.
% 
To this end, we design a cross-scale Transformer layer, which uses the representations from the decoder as \textit{queries} and the multi-scale features from the encoder as \textit{keys} and \textit{values} to retrieve information.
% 
Such a top-down pathway and lateral connects help to combine the multi-scale representations, thus enhancing the prediction accuracy.

% Efficiency Issue
In addition, HUTFormer addresses complexity issues based on an efficient input embedding strategy, which consists of segment embedding and spatial-temporal positional encoding.
% 
On the one hand, segment embedding {\color{black}reduces complexity from the temporal dimension by using} time series segments as basic input tokens.
This simple operation has significant benefits in both reducing the length of the input sequence and providing more robust semantics~\cite{2022STEP}.
% 
On the other hand, spatial-temporal positional encoding is designed to replace the standard positional encoding~\cite{2017Transformer,2021ViT} in Transformer. 
More importantly, it efficiently models the correlations among time series from the perspective of solving the indistinguishability of samples~\cite{2022STID, 2021STNorm}, 
avoiding the {\color{black}high complexity of conducting graph convolution~\cite{2019GWNet, 2018DCRNN} in the spatial dimension}.
% 
% In addition, HUTFormer reduces complexity issues from temporal and spatial dimensions based on an efficient input embedding strategy, which consists of segment embedding and spatial-temporal positional encoding.
% Overall, HUTFormer is significantly more efficient than previous works that conduct graph convolution at each time steps.
% Overall, HUTFORER works much more efficiently than previous graph convolution on each time slice.

% Finally, we propose the training strategy.
% Specifically, HUTFormer applies a two-stage training strategy.
Finally, we propose the training strategy: a two-stage strategy.
The first stage aims to train the hierarchical encoder based on the Mean Absolute Error~(MAE) between the intermediate prediction and ground truth.
In the second stage, we only train the decoder, while the parameters of the encoder are fixed to act as the multi-scale feature extractor.
% 
The reason for adopting the two-stage strategy is that traffic forecasting tasks are different from tasks that employ an end-to-end strategy (\eg semantic segmentation~\cite{2015UNet, 2021SwinUNet} and object detection~\cite{2017FPN} in computer vision).
%
Specifically, in computer vision tasks, pre-trained vision models (\eg pre-trained ResNet~\cite{2016ResNet}) usually serve as the backbone to extract multi-scale features~\cite{2017FPN}.
However, there is no pre-trained model for time series that can extract multi-scale features.
Therefore, optimizing the feature extractor~(\ie the encoder) and downstream networks~(\ie the decoder) in an end-to-end fashion may be insufficient.
The experimental results in Section \ref{section:ablation} also verify this hypothesis.
% 
Next, we introduce each component in detail.

\subsection{Input Embedding}

\textbf{Segment embedding.}
Most previous works usually use single data points as the basic input units.
However, isolated points of time series usually give less semantics~\cite{2022STEP} and are more easily affected by noise.
Therefore, HUTFormer adopts segment embedding, \ie dividing the input sequence into several segments to get the input tokens.
Specifically, given the time series $\mathbf{X}^i\in\mathbb{R}^{T \times C}$ from sensor $i$, HUTFormer divides it into $P$ non-overlapping segments of length $L$, \ie $T=P*L$. We denote the $j$th segment as $\mathbf{X}^i_j\in\mathbb{R}^{LC}$.
Then, we conduct the input embedding layer based on these segments: 
\begin{equation}
\mathbf{S}_j^i=\mathbf{W}\cdot\mathbf{X}^i_j+\mathbf{b},
\end{equation}
where $\mathbf{S}_j^i\in\mathbb{R}^{d}$ is the embedding of segments $j$ of the time series from sensor $i$, and $d$ is the hidden dimension.
$\mathbf{W}\in\mathbb{R}^{d\times (LC)}$ and $\mathbf{b}\in\mathbb{R}^{d}$ are learnable parameters shared by all segments.

In summary, applying segment embedding brings two benefits.
First, it provides more robust semantics. 
Second, it significantly reduces the sequence length to reduce computational complexity.

\noindent
\textbf{Spatial-temporal positional encoding.}
In this paper, we propose to replace the standard positional encoding in Transformer-based networks~\cite{2017Transformer, 2021ViT} with Spatial-Temporal Positional Encoding~(ST-PE).
Specifically, given the segment embedding $\mathbf{S}_j^i\in\mathbb{R}^{d}$ of segments $j$ from time series $i$, ST-PE conduct positional encoding on the spatial and temporal dimensions simultaneously:
\begin{equation}
    \mathbf{U}_j^i = \text{Linear}(\mathbf{S}_j^i \parallel \mathbf{E}^i \parallel \mathbf{T}_j^{TiD} \parallel \mathbf{T}_j^{DiW}).
    \label{eq:stpe}
\end{equation}
On the spatial dimension, we define the spatial positional embeddings $\mathbf{E}\in\mathbb{R}^{N\times d_1}$, where $N$ is the number of time series~(\ie sensors), and $d_1$ is the hidden dimension.
On the temporal dimension, we define two semantic positional embeddings,
$\mathbf{T}^{TiD}\in\mathbb{R}^{N_D\times d_2}$ and $\mathbf{T}^{DiW}\in\mathbb{R}^{N_W\times d_3}$, 
where $N_D$ is the number of time slots of a day~(determined by the sensor's sampling frequency) and $N_W=7$ is the number of days in a week. The temporal embeddings are thus shared among slots for the same time of the day and the same day of the week.
Semantic temporal positional embeddings are helpful since traffic systems usually reflect the periodicity of human society.
In addition, kindly note that all other baseline models~\cite{2018DCRNN, 2019GWNet, 2022D2STGNN, 2021AutoFormer, 2022FEDFormer, 2022Pyraformer} also use such temporal features, so there is no unfairness.
$\text{Linear}(\cdot)$ is a linear layer to reduce the hidden dimension.
$\mathbf{E}$, $\mathbf{T}^{TiD}$, and $\mathbf{T}^{DiW}$ are trainable parameters.

Embedding $\mathbf{E}$ is vital for reducing the complexity of modeling the spatial correlations between time series.
% 
This is because attaching spatial embeddings plays a similar role to GCN in terms of solving the indistinguishability of samples~\cite{2022STID}, but with two primary advantages. 
% 
% This is because attaching the spatial embeddings plays a similar role to GCN from the perspective of solving the indistinguishability of samples~\cite{2022STID, 2021STNorm}, but with two primary advantages.
On the one hand, it is more efficient than GCNs, which usually have $\mathcal{O}(N^2)$ complexity.
On the other hand, it does not generate many additional network parameters than approaches based on variable-specific modeling~\cite{2020AGCRN, 2022TriFormer}.

\subsection{Hierarchical Encoder}
\textbf{Window Transformer Layer.}
Standard Transformer Layers~\cite{2017Transformer} are designed based on the multi-head self-attention mechanism.
As shown in Figure \ref{fig:w-msa}(a), it computes the attention among all input tokens.
Therefore, each layer of the Transformer Layer has an infinite receptive field, and many works~\cite{2021Informer, 2021AutoFormer, 2022FEDFormer, 2022Pyraformer} try to capture long-term dependencies based on such a feature.

% Figure environment removed

However, the infinite receptive field makes the standard Transformer layers unable to generate multi-scale features~\cite{2021SwinTransformer}.
Inspired by recent development in computer vision~\cite{2021SwinTransformer}, we apply the window self-attention in HUTFormer to extract the hierarchical multi-scale features.
An example of window self-attention with windows size 2 is shown in Figure \ref{fig:w-msa}(b).
Window self-attention forces calculating attention inside non-overlapping windows, thereby limiting the size of the receptive field.
By replacing multi-head self-attention in standard Transformer layers~\cite{2017Transformer} with the Window Multi-head Self-Attention~(W-MSA), we present the window Transformer layer:
\begin{equation}
    \begin{aligned}
    \mathbf{H}^{in'} & = \text{W-MSA}(\text{LN}(\mathbf{H}^{in})) + \mathbf{H}^{in}, \\
    \mathbf{H}^{out} & = \text{MLP}(\text{LN}(\mathbf{H}^{in'})) + \mathbf{H}^{in'},
    \end{aligned}
\end{equation}
where $\text{LN}(\cdot)$ is the layer normalization, and $\text{MLP}(\cdot)$ is the multi-layer perceptron. $\mathbf{H}^{in}\in\mathbb{R}^{P\times d}$ and $\mathbf{H}^{out}\in\mathbb{R}^{P\times d}$ are the input and output sequences. $P$ is the sequence length, and $d$ is the hidden dimension.
By limiting the receptive field size, the window transformer layer is the basis for extracting multi-scale features.

\noindent
\textbf{Segment Merging.} To generate hierarchical multi-scale representations, we adopt segment merging, which reduces the number of tokens and increases the number of hidden dimensions as the network gets deeper. 
As illustrated in Figure \ref{fig:segment_merging}, segment merging divides the token series into non-overlapping groups of size 2, and concatenates the features within each group.

% Figure environment removed


By combining the segment merging and window transformer layer, we get the basic block of the hierarchical encoder~(\ie the blue block in Figure \ref{fig:main_arch}).
Assuming $(\mathbf{H}^i)^{l}_{enc}\in\mathbb{R}^{P^l\times d^l}$ is the representation of time series $i$ after block $l$~($l\geq 1$) of the encoder, the $(l+1)$th block is computed as:
\begin{equation}
    \begin{aligned}
    (\mathbf{H}^i)^{l'}_{enc} &= \text{SegmentMerging}((\mathbf{H}^i)^{l}_{enc}),\\
    (\mathbf{H}^i)^{l+1}_{enc} &= \text{WindowTransformer}((\mathbf{H}^i)^{(l')}_{enc}),
    \end{aligned}
\end{equation}
where $(\mathbf{H}^i)^{l+1}_{enc}\in\mathbb{R}^{P^{l+1}\times d^{l+1}}$ is the representation of time series $i$ after block $l+1$ of the encoder. 
$P^{l+1}=\frac{P^l}{2}$ is the number of tokens after $(l+1)$th layer, and $d^{l+1}=2d^{l+1}$ is the hidden dimension.

\noindent
\textbf{Prediction Layer.}
Assuming there are $S$ blocks in the encoder, HUTFormer makes an intermediate prediction with a linear layer:
\begin{equation}
    \hat{\mathbf{Y}}^i_{enc} = \text{Linear}(\mathop{\parallel}\limits_{j=1}^{P^S}(\mathbf{H}^i_j)^{S}_{enc}),
\end{equation}
where $P^S$ is the number of tokens after the $S$th block.
$\hat{\mathbf{Y}}^i\in\mathbb{R}^{T_f\times C}$ is the prediction of time series $i$.
Considering the prediction from all $N$ time series, $\hat{\mathcal{Y}}^{enc}\in\mathbb{R}^{T_f\times N\times C}$, we compute the Mean Absolute Error~(MAE) as regression loss to train the hierarchical encoder:
\begin{equation}
    \mathcal{L}_{enc} = \frac{1}{T_f N C}\sum_{j=1}^{T_f}\sum_{i=1}^{N}\sum_{k=1}^{C}|\hat{\mathcal{Y}}_{ijk}^{enc} - \mathcal{Y}_{ijk}|.
\end{equation}


\subsection{Hierarchical Decoder}
\textbf{Cross-Scale Transformer Layer.}
The hierarchical decoder aims to effectively utilize the multi-scale features, to fine-tune each segment of the intermediate prediction.
However, as discussed in Section \ref{sec:methods:overview}, the history and future sequence in traffic forecasting tasks are not aligned, making the feature sequences extracted by the encoder and the decoder cannot be directly superimposed.
Therefore, we design a cross-scale attention mechanism to select and incorporate multi-scale features.
Different from self-attention, cross-scale attention utilizes the representations of the decoder as \textit{queries} to retrieve the multi-scale features from the encoder.
For brevity, we denote $\mathbf{H}_{enc}\in\mathbb{R}^{P_{enc}\times d_{enc}}$ as the representation from the encoder and $\mathbf{H}_{dec}\in\mathbb{R}^{P_{dec}\times d_{dec}}$ as the corresponding representation from the decoder.
Then, the Cross-scale Attention~(CA) is computed as:
\begin{equation}
    \begin{aligned}
    \text{CA}(\mathbf{H}_{enc}, \mathbf{H}_{dec})&=\text{Softmax}(\frac{\mathbf{H}_{dec}(\mathbf{H}_{enc}^{'})^T}{\sqrt{d_{dec}}})\mathbf{H}_{enc}^{'},\\
\mathbf{H}_{enc}^{'}&=\text{Linear}(\mathbf{H}_{enc}),
    \end{aligned}
\end{equation}
The $\text{Linear}(\cdot)$ layer is used to transform the hidden dimension from $d_{enc}$ to $d_{dec}$.
By replacing the multi-head self-attention with Multi-head Cross-scale Attention~(MCA), we present the cross-scale Transformer layer as:
\begin{equation}
    \begin{aligned}
    \mathbf{H}^{in'}_{dec} & = \text{MCA}(\text{LN}(\mathbf{H}^{in}_{enc}, \mathbf{H}^{in}_{dec})) + \mathbf{H}^{in}_{dec}, \\
    \mathbf{H}^{out}_{dec} & = \text{MLP}(\text{LN}(\mathbf{H}^{in'}_{dec})) + \mathbf{H}^{in'}_{dec},
    \end{aligned}
\end{equation}
where $\mathbf{H}^{in}_{enc}$ is the multi-scale feature from the encoder, and $\mathbf{H}^{in}_{dec}$ is the input feature from the decoder.
$\mathbf{H}^{out}_{dec}$ is the output of the cross-scale Transformer layer.

\noindent
\textbf{Prediction Layer.} Assuming $(\mathbf{H}^i_j)^S_{dec}\in\mathbb{R}^{d_{dec}}$ is the representation of the decoder's last block~(\ie the $S$th block) for $j$th segment of $i$th time series, HUTFormer makes the final prediction for each segment with a shared linear layer:
\begin{equation}
    (\hat{\mathbf{Y}}^i_j)_{dec} = \text{Linear}((\mathbf{H}^i_j)^{S}_{dec}),
\end{equation}
where $(\hat{\mathbf{Y}}^i_j)_{dec}\in\mathbb{R}^{LC}$ is the final prediction of segment $j$ of time series $i$.
Similar to the encoder, we consider the prediction from all $P_{dec}$ segments~($P_{dec}\times L=T_f$) of all $N$ time series, $\hat{\mathcal{Y}}^{dec}\in\mathbb{R}^{T_f\times N\times C}$, and compute the MAE loss to train the hierarchical decoder:
\begin{equation}
    \mathcal{L}_{dec} = \frac{1}{T_f N C}\sum_{j=1}^{T_f}\sum_{i=1}^{N}\sum_{k=1}^{C}|\hat{\mathcal{Y}}_{ijk}^{dec} - \mathcal{Y}_{ijk}|.
\end{equation}
Kindly note that the parameters of the encoder are fixed during this stage to serve as a pre-training model for extracting robust hierarchical multi-scale representations of traffic data.

\vspace{-.1cm}

\subsection{Visuomotor policies for robotic manipulation}
\label{sec:bc}

\noindent\textbf{Problem formulation.}
We learn a policy $\pi_{\theta}(\mbf{a}_{t}\,|\,\mbf{o}_{t})$ that maps environment observation $\mbf{o}_{t}$ to a robot action $\mbf{a}_{t}$ at  timestep $t$, see Figure~\ref{fig:model_arch}.
The action $\mbf{a}_{t}=(\mbf{v}_{t}, \boldsymbol{\omega}_{t}, g_{t})$ is composed of the linear velocity $\mbf{v}_{t} \in \mathbb{R}^{3}$, the angular velocity $\boldsymbol{\omega}_{t} \in  \mathbb{R}^{3}$ of the robot end-effector and the gripper openness state $g_{t} \in \{0, 1\}$. 
The policy is executed in a closed-loop setting to perform a manipulation task. 

\noindent\textbf{Model architecture.}
Our model takes input from two RGB cameras as frames $\mbf{I}_{t} = \{\mbf{I}^1_{t}, \mbf{I}^2_t \}$ at each timestep $t$ where $\mbf{I}^{*}_{t} \in \mathbb{R}^{H \times W \times 3}$.
The cameras are placed on a wide baseline with 90 degrees relative angle to alleviate depth ambiguities and occlusions.  
The observation $o_t$ includes the last three RGB frames $(\mbf{I}_{t-2}, \mbf{I}_{t-1}, \mbf{I}_t)$ and the last three gripper proprioceptive values $\mbf{P}_{t}=[\text{pos}_t, \sin(\phi_t), \cos(\phi_t)]$, where $\text{pos}_t$ is the gripper position with respect to the robot base, 
and $\phi_t$ is the rotation angle of the gripper around the end-effector axis. 
We stack frames from each viewpoint in the channel dimension and use \mbox{ResNet-18} network~\cite{he16} to generate a feature vector of dimension $512$ for each viewpoint. 
We then concatenate feature vectors from the two viewpoints together with the proprioceptive information $\mbf{P}_{t}$.
The action $\mbf{a}_{t}$ is finally predicted by a multi-layer perceptron (MLP) composed of two layers with $512$ hidden units and a ReLU activation function each.

%\smallskip
\noindent\textbf{Training with behavior cloning.}
Given a dataset of observation-action pairs $\{(\mbf{o}_{t}, \mbf{a}_{t})\}_{t}$ obtained from expert trajectories, 
we randomly initialize our policy network
and train it using a combination of the mean-squared error (MSE) loss for velocity control $\mbf{v}_{t}$, $\boldsymbol{\omega}_{t}$, and the binary cross entropy (BCE) loss for gripper state probability $g_{t}$ defined as:
\begin{equation}
  L = \lambda L_{MSE}((\hat{\mbf{v}}_{t}, \hat{\boldsymbol{\omega}}_{t}), (\mbf{v}_{t}, \boldsymbol{\omega}_{t})) + (1-\lambda)L_{BCE}(\hat{g}_{t}, g_{t})
  \label{eqn:loss}
\end{equation}
where $\lambda$ is a hyper-parameter to balance the MSE and BCE terms, $(\mbf{v}_{t}, \boldsymbol{\omega}_{t}, g_{t})$ is the expert action and $\pi_{\theta}(o_{t}) = (\hat{\mbf{v}}_{t}, \hat{\boldsymbol{\omega}}_{t}, \hat{g}_{t})$ is the predicted action for observation $o_t$.
\vspace{-.1cm}

\subsection{Domain Randomization}
\label{sec:dr}
In order to improve sim-to-real transfer, we augment synthetic images with domain randomization (DR) for policy learning.
We investigate different visual DR components, including textures, lightning conditions, object colors and camera parameters as described below.

\noindent\textbf{Texture randomization.} 
To obtain robustness to scene appearance, we randomize the textures of the robot, table, wall and floor. 
As the ability to distinguish object colors matters in our tasks, we do not randomize the object textures. 
We compare two types of textures as illustrated in Figure~\ref{fig:textures_comp}.
The first type is procedural textures \cite{tobin17} (Figure~\ref{fig:textures_comp} top), which have random colors and follow one of the four patterns - checkers, gradient, noise and a plain color. 
The second type is a set of $1,203$ high quality and realistic textures from ambientCG assets \cite{ambientCG} (Figure~\ref{fig:textures_comp} bottom). 

\input{figures/textures_comp}

\noindent\textbf{Lighting randomization.} To achieve robustness to lighting conditions, we sample the light position uniformly in a portion of a sphere for rendering images. 
We define the sphere in spherical coordinates with a distance of the light source to the workspace center in the range $[1.0, 3.0]$ meter, azimuthal angle and polar angle in the range $[0, \pi/2]$ and $[\pi/10, 4\pi/10]$ radians. This range of parameters allows us to sample realistic light source positions around the table, e.g., we avoid sampling lights inside the table or under it.
Besides light positions, we randomize the light properties with diffuse, specular and ambient coefficients around a nominal value set to $0.3$ by adding an offset sampled from the range $[-\psi_{l}, \psi_{l}]$ where $\psi_{l}$ is a parameter to be optimized.

\noindent\textbf{Variation of object colors.} 
We treat object colors in a special way since object manipulation in some of our tasks depends on the object color.
We therefore randomize object colors by sampling the Hue, Saturation and Value (Brightness) (HSV) around their nominal value. 
We sample an offset from range $[-\phi_o, \phi_o]$ and add it to the object color expressed using HSV coordinates in $[0, 1]$. The range $\phi_o$ is selected such that it is sufficiently large to cover possible color discrepancies between real and simulated scenes, and not too large to avoid confusion between different objects with initially different colors.

\noindent\textbf{Variation of camera parameters.} Camera calibration is often imprecise especially in terms of extrinsic parameters. 
To improve robustness to slight viewpoint changes, we randomize camera positions in simulation by sampling camera angles, locations and the field of view (FOV) around default values.

\section{Secure Design of \puma}\label{sec:design}
In this section, we first present an overview of \puma, and present the protocols for secure $\gelu$ , $\softmax$, embedding, and $\layernorm$ used by \puma. Note that the linear layers such as matrix multiplication are straightforward in replicated secret sharing, so we mainly describe our protocols for non-linear layers in this manuscript.

\subsection{Overview of \puma}\label{sec:overview}
To achieve secure inference of Transformer models, \puma\ defines three kinds of roles: one model owner, one client, and three computing parties. The model owner and the client  provide their models or inputs to the computing parties (i.e., $P_0$, $P_1$, and $P_2$) in a secret-shared form, then the computing parties execute the MPC protocols and send the results back to the client. Note that the model owner and client can also act as one of the computing party, we describe them separately for generality. \eg, when the model owner acts as $P_0$, the client acts as  $P_1$, a third-party dealer acts as $P_2$, the system model becomes the same with \mpcformer~\citep{li2023mpcformer}.

During the secure inference process, a key invariant is maintained: For any layer, the computing parties always start with 2-out-of-3 replicated secret shares of the previous layer's output and the model weights, and end with 2-out-of-3 replicated secret shares of this layer's output. As the shares do not leak any information to each party, this ensures that the layers can be sequentially combined for arbitrary depths to obtain a secure computation scheme for any Transformer-based model.
%The main focus of \puma\ is to reduce the computation and communication costs between the computing parties while maintaining the desired level of security. 



\iffalse
\textbf{Threat Model.}
Following previous works~\citep{aby3,li2023mpcformer},
\puma\ resists a semi-honest (a.k.a., honest-but-curious) adversary in honest-majority~\citep{lindell2009proof}, where the adversary passively corrupts no more than one computing party. Such an adversary follows the protocol specification exactly, but may try to learn more information than permitted. Please note that \puma\ cannot protect against the extraction of information from the inference results, and the examination of mitigating solutions (\eg, differential privacy~\citep{abadi2016deep}) falls outside the scope of this study.
\fi 

\subsection{Protocol for Secure GeLU}\label{sec:gelu}
Most of the current approaches view the $\gelu$ function as a composition of smaller functions and try to optimize each piece of them, making them to miss the
chance of optimizing the private $\gelu$ as a whole. Given the $\gelu$ function:
\begin{equation}\label{eq:gelu}
\begin{split}
    \gelu(x) &= \frac{x}{2} \cdot \left(1 + \tanh \left( \sqrt{\frac{2}{\pi}} \cdot \left(x + 0.044715 \cdot x^3 \right) \right) \right)\\
    &\approx x\cdot \mathsf{sigmoid}(0.071355\cdot x^3 + 1.595769\cdot x) 
\end{split},
\end{equation}
these approaches~\citep{hao2022iron,characmpctranformer} focus either on designing efficient protocols for function $\tanh$
or using the existing MPC protocols of exponentiation and reciprocal for $\mathsf{sigmoid}$. 

However, none of current approaches have utilized the fact that $\gelu$ function is almost linear on the two sides (\ie, $\gelu(x)\approx 0$ for $x<-4$ and $\gelu(x)\approx x$ for $x>3$). 
Within the short interval $[-4,3]$ of $\gelu$,
we suggest a piece-wise approximation of low-degree polynomials is a more efficient and easy-to-implement choice for its secure protocol. Concretely, our piece-wise low-degree polynomials are shown as equation~(\ref{eq:geluapprox}):
\begin{equation}\label{eq:geluapprox}
\gelu(x)=
\begin{cases}
0, & x<-4 \\
F_0(x), & -4 \le x < -1.95 \\
F_1(x), & -1.95 \le x \le 3 \\
x, & x >3
\end{cases},
\end{equation}
where polynomials $F_0()$ and $F_1()$ are computed by library $\mathsf{numpy.ployfit}$\footnote{\url{https://numpy.org/doc/stable/reference/generated/numpy.polyfit.html}} as equation~(\ref{eq:f0f1}). Surprsingly, the above simple poly fit works very well and our $\mathsf{max\ error}< 0.01403$, $\mathsf{median\ error}< 4.41e-05$, and $\mathsf{mean\ error}< 0.00168$.
\begin{equation}\label{eq:f0f1}
\begin{cases}
F_0(x) &= -0.011034134030615728 x^3 -0.11807612951181953 x^2 \\
&- 0.42226581151983866 x -0.5054031199708174\\
F_1(x) &= 0.0018067462606141187x^6 -0.037688200365904236 x^4 \\
&+ 0.3603292692789629x^2 + 0.5x + 0.008526321541038084
\end{cases}
\end{equation}

Formally, given secret input $\share{x}$, our secure $\gelu$ protocol $\Pi_{\gelu}$ is constructed as algorithm~\ref{protocol:gelu}. 
\iffalse
\begin{itemize}
    \item The parties jointly compute
$\share{b_0}^2 = \Pi_{\mathsf{LT}}(\share{x}, 4)$,
$\share{b_1}^2 = \Pi_{\mathsf{LT}}(\share{x}, -1.95)$, and
$\share{b_2}^2 = \Pi_{\mathsf{LT}}(3, \share{x})$.

\item  Then, each $P_i$ locally compute
$\share{b_3}^2 = \share{b_1}^2 \oplus \share{b_2}^ \oplus 1$ and
$\share{b_4}^2 = \share{b_0}^2 \oplus \share{b_1}^2$

\item Finally, the parties compute and return 
$\share{b_2}^2 \cdot \share{x} + \share{b_4}^2 \cdot F_0(\share{x}) + \share{b_3}^2 \cdot F_1(\share{x})$, where polynomials $(F_0, F_1)$ can be computed easily using secure addition and multiplication (and its variants, \eg, secure square)~\citep{spu}. 
\end{itemize}
\fi 

\begin{algorithm}[tp]
\caption{Secure $\gelu$ Protocol $\Pi_{\mathsf{GeLU}}$}\label{protocol:gelu}
\begin{algorithmic}[1]
\REQUIRE
$P_i$ holds the 2-out-of-3 replicate secret share $\share{x}_i$ for $i\in \{0,1,2\}$ 
\ENSURE
$P_i$ gets the 2-out-of-3 replicate secret share $\share{y}_i$ for $i\in \{0,1,2\}$, where $y=\gelu(x)$.

\STATE $P_0$, $P_1$, and $P_2$ jointly compute
\begin{equation*}
\begin{split}
&\shareb{b_0} = \Pi_{\mathsf{LT}}(\share{x}, -4),~~~\vartriangleright b_0 = 1\{x<-4\}\\
&\shareb{b_1} = \Pi_{\mathsf{LT}}(\share{x}, -1.95),~~~\vartriangleright b_1 = 1\{x<-1.95\} \\
&\shareb{b_2} = \Pi_{\mathsf{LT}}(3, \share{x}),~~~~~~\vartriangleright b_2 = 1\{3<x\}
\end{split}
\end{equation*}
and compute 
$\shareb{z_0} = \shareb{b_0} \oplus \shareb{b_1}$,
$\shareb{z_1} = \shareb{b_1} \oplus \shareb{b_2} \oplus 1$, and $\shareb{z_2}=\shareb{b_2}$. Note that $z_0 = 1\{-4\le x < -1.95\}$, $z_1 = 1\{-1.95\le x\le 3\}$, and $z_2 = 1\{x>3\}$.

\STATE Jointly compute $\share{x^2} = \Pi_{\mathsf{Square}}(\share{x})$, $\share{x^3} = \Pi_{\mathsf{Mul}}(\share{x}, \share{x^2})$, $\share{x^4} = \Pi_{\mathsf{Square}}(\share{x^2})$, and $\share{x^6} = \Pi_{\mathsf{Square}}(\share{x^3})$.

\STATE Computing polynomials $\share{F_0(x)}$ and $\share{F_1(x)}$ based on $\{\share{x}, \share{x^2}, \share{x^3}, \share{x^4}, \share{x^6}\}$ as equation~(\ref{eq:geluapprox}) securely.


\RETURN$\share{y} = \Pi_{\mathsf{Mul_{BA}}}(\shareb{z_0}, \share{F_0(x)}) + \Pi_{\mathsf{Mul_{BA}}}(\shareb{z_1}, \share{F_1(x)})+\Pi_{\mathsf{Mul_{BA}}}(\shareb{z_2}, \share{x})$.

\end{algorithmic}
\end{algorithm}



\subsection{Protocol for Secure Softmax}\label{sec:secureatten}

In the function $\attention(\Q,\K,\V)=
\softmax(\Q \cdot \K^\mathsf{T} + \M) \cdot \V$, where $\M$ can be viewed as a bias matrix, the key challenge is computing function $\softmax$. For the sake of numerical stability, the $\softmax$ function is computed as
\begin{equation}\label{eq:softmax}
    \softmax(\x)[i]=\frac{\exp(\x[i] - \bar{x} - \epsilon)}{\sum_i \exp(\x[i] - \bar{x} - \epsilon)},
\end{equation}
where $\bar{x}$ is the maximum element of the input vector $\x$. 
For the normal plaintext softmax, $\epsilon=0$. For a two-dimension matrix, we apply equation~(\ref{eq:softmax}) to each of its row vector.

Formally, our detailed secure protocol  $\Pi_{\softmax}$ is illustrated in algorithm~\ref{protocol:softmax}, where we propose two optimizations:
\begin{itemize}
\item 
For the first optimization, we set $\epsilon$ in equation~\ref{eq:softmax} to a tiny and positive
value, e.g., $\epsilon =
10^{-6}$, so that the inputs to exponentiation
in equation~\ref{eq:softmax} are all negative. We exploit the negative operands
for acceleration. Particularly, we compute the exponentiation using the Taylor series~\citep{tan2021cryptgpu} with a simple clipping
\begin{equation}\label{eq:negexp}
\mathsf{negExp}(x) = \begin{cases}
    0, &x < T_{\exp} \\
    (1+\frac{x}{2^t})^{2^t}, &x\in [T_{\exp},0].
\end{cases}
\end{equation}
Indeed, we apply the less-than for the branch $x < T_{\exp}$
The division by $2^t$ can be achieved using
$\Pi_{\mathsf{Trunc}}^t$ since the input is already negative. Also, we can
compute the power-of-$2^t$ using $t$-step sequences of square function $\Pi_{\mathsf{square}}$ and $\Pi_{\mathsf{Trunc}}^f$. Suppose our MPC program uses
$18$-bit fixed-point precision. Then we set $T_{\exp}=-14$ given $\exp(-14) < 2^{-18}$, and empirically set $t = 5$.


\item 
Our second optimization is to reduce the number of divisions, which ultimately saves computation and communication costs.
To achieve this, for a vector $\x$ of size $n$, we have replaced the operation $\mathsf{Div}(\x, \mathsf{Broadcast}(y))$ with $\x \cdot  \mathsf{Broadcast}(\frac{1}{y})$, where $y=\sum_{i=1}^n\x[i]$. By making this replacement, we effectively reduce $n$ divisions to just one reciprocal operation and $n$ multiplications.
This optimization is particularly beneficial in the case of the $\softmax$ operation. The $\frac{1}{y}$ in the $\softmax$ operation is still large enough to maintain sufficient accuracy under fixed-point values. As a result, this optimization can significantly reduce the computational and communication costs while still providing accurate results.
\end{itemize}

\begin{algorithm}[tp]
\caption{Secure $\softmax$ Protocol $\Pi_{\softmax}$}\label{protocol:softmax}
\begin{algorithmic}[1]
\REQUIRE
$P_i$ holds the 2-out-of-3 replicate secret share $\share{\x}_i$ for $i\in \{0,1,2\}$, and $\x$ is a vector of size $n$. 
\ENSURE
$P_i$ gets the 2-out-of-3 replicate secret share $\share{\y}_i$ for $i\in \{0,1,2\}$, where $\y=\softmax(\x)$.

\STATE $P_0$, $P_1$, and $P_2$ jointly compute
$\shareb{\mathbf{b}} = \Pi_{\mathsf{LT}}(T_{\exp}, \share{\x})$ and the maximum $\share{\bar{x}} = \Pi_{\mathsf{Max}}(\share{\x})$.

\STATE Parties locally computes $\share{\hat{\x}} = \share{\x} - \share{\bar{x}} - \epsilon$, and jointly compute $\share{\z_0} = 1+  \Pi_{\mathsf{Trunc}}^t(\share{\hat{\x}})$.

\FOR{$j=1,2,\dots, t$}
\STATE $\share{\z_j} = \Pi_{\mathsf{Square}}(\share{\z_{j-1}})$.
\ENDFOR

\STATE Parties locally compute $\share{z} = \sum_{i=1}^n \share{\z[i]}$ and jointly compute $\share{1/z} = \Pi_{\mathsf{Recip}}(\share{z})$.

\STATE Parties jointly compute $\share{\z / z} = \Pi_{\mathsf{Mul}}(\share{\z}, \share{1/z})$

\RETURN $\share{\y} = \Pi_{\mathsf{Mul}_{\mathsf{BA}}}( \shareb{\mathbf{b}}, \share{\z / z})$.

\end{algorithmic}
\end{algorithm}

\subsection{Protocol for Secure Embedding}\label{sec:embed}


The current secure embedding procedure described in~\citep{li2023mpcformer} necessitates the client to  generate a one-hot vector using the token $\tokenid$ locally. This deviates from a plaintext Transformer workflow where the one-hot vector is generated inside the model. As a result, they have to carefully strip off the one-hot step from the pre-trained models, and add the step to the client side, which could be an obstacle for deployment. 



To address this issue, we propose a secure embedding design as follows. Assuming that the token $\tokenid\in [n]$ and all embedding vectors are denoted by $\E= (\e_1^T, \e_2^T, \dots, \e_n^T)$, the embedding can be formulated as $\e_{\tokenid} = \mathbf{E}[\tokenid]$. Given $(\tokenid, \E)$ are in secret-shared fashion, our secure embedding protocol $\Pi_{\mathsf{Embed}}$ works as follows:
\begin{itemize}
    \item The computing parties securely compute the one-hot vector $\shareb{\mathbf{o}}$ after receiving $\share{\tokenid}$ from the client. Specifically, $\shareb{\mathbf{o}[i]}=\Pi_{\mathsf{Eq}}(i,\share{\tokenid})$ for $i\in [n]$.
    \item The parties can compute the embedded vector via $\share{\e_{\tokenid}} = \Pi_{\mathsf{Mul_{BA}}}(\share{\E}, \shareb{\mathbf{o}})$, where  does not require secure truncation.
\end{itemize}
In this way, our $\Pi_{\mathsf{Embed}}$ does not require explicit modification of the workflow of plaintext Transformer models, at the cost of more $\Pi_{\mathsf{Eq}}$ and $\Pi_{\mathsf{Mul_{BA}}}$ operations. 



\subsection{Protocol for Secure LayerNorm}\label{sec:seclayernorm}
Recall that given a vector $\x$ of size $n$, $\layernorm(\x)[i] =  \gamma \cdot \frac{\x[i]-\mu}{\sqrt{\sigma}} + \beta$, where $(\gamma, \beta)$ are trained parameters, $\mu = \frac{\sum_{i=1}^n \x[i]}{n}$, and $\sigma = \sum_{i=1}^n (\x[i] - \mu)^2$. In MPC, the key challenge is the evaluation of the divide-square-root $\frac{\x[i]-\mu}{\sqrt{\sigma}}$ formula. To securely evaluate this formula, CrypTen sequentially executes the MPC protocols of square-root, reciprocal, and multiplication. However, we observe that $\frac{\x[i]-\mu}{\sqrt{\sigma}}$ is equal to $(\x[i]-\mu)\cdot \sigma^{-1/2}$. And in the MPC side, the costs of computing the inverse-square-root $\sigma^{-1/2}$ is similar to that of the square-root operation~\citep{rSqrt}. Besides, inspired by the second optimization of \S~\ref{sec:secureatten}, we can first compute $\sigma^{-1/2}$ and then $\mathsf{Broadcast}(\sigma^{-1/2})$ to support fast and secure $\layernorm(\x)$. And our formal protocol $\Pi_{\layernorm}$ is shown in algorithm~\ref{protocol:layernorm}.

\begin{algorithm}[tp]
\caption{Secure $\mathsf{LayerNorm}$ Protocol $\Pi_{\mathsf{LayerNorm}}$}\label{protocol:layernorm}
\begin{algorithmic}[1]
\REQUIRE
$P_i$ holds the 2-out-of-3 replicate secret share $\share{\x}_i$ for $i\in \{0,1,2\}$, and $\x$ is a vector of size $n$. 
\ENSURE
$P_i$ gets the 2-out-of-3 replicate secret share $\share{\y}_i$ for $i\in \{0,1,2\}$, where $\y=\mathsf{LayerNorm}(\x)$.

\STATE $P_0$, $P_1$, and $P_2$ compute $\share{\mu} = \frac{1}{n}\cdot \sum_{i=1}^n\share{\x[i]}$ and $\share{\sigma} = \sum_{i=1}^n \Pi_{\mathsf{Square}}(\share{\x} - \share{\mu})[i]$.

\STATE Parties jointly compute $\share{\sigma^{-1/2}} = \Pi_{\mathsf{rSqrt}}(\share{\sigma})$.

\STATE Parties jointly compute $\share{\mathbf{c}} = \Pi_{\mathsf{Mul}}((\share{\x} - \share{\mu}), \share{\sigma^{-1/2}})$

\RETURN $\share{\y} = \Pi_{\mathsf{Mul}}(\share{\gamma}, \share{\mathbf{c}}) + \share{\beta}$.

\end{algorithmic}
\end{algorithm}
\vspace{-.2cm}
\subsection{Localization as a proxy task}
\vspace{-.1cm}
\label{sec:proxy_task}

DR parameters could be selected for each task by optimizing the accuracy of the learned policies for a real robot scenario. 
Such an approach, however, is highly time-consuming as it requires policy retraining and real-world policy evaluation for each set of DR parameters.
In this work we demonstrate that DR parameters can be efficiently chosen off-line using real images recorded for a proxy task. 
For this purpose we propose the following task of cube localization.

Given RGB images of a scene with three cubes of different colors, our proxy task aims to predict 3D locations of each cube relative to the gripper, see Figures~\ref{fig:textures_comp} and  \ref{fig:robustness_setup}. 

We chose this proxy task because it requires to distinguish different colors and localize multiple objects, which enables to study DR parameters for color-dependent and multi-object manipulation policies.

To predict 3D locations of the cubes, we use a similar neural network architecture as in Section~\ref{sec:bc} but with RGB frames $\mbf{I}_{t} = \{\mbf{I}^1_{t}, \mbf{I}^2_t \}$ for one time step instead of three and no proprioceptive information at the input.
We train the model by minimizing the distance between the predicted and the ground truth positions of the cubes. 

To evaluate sim-to-real transfer for the proxy task, we train the cube localization network in simulation using different DR parameters and measure localization precision on pre-recorded images of real scenes.
Our empirical experiments in Section~\ref{sec:expr} show that the performance of this proxy task is well aligned with the real-world accuracy of diverse manipulation policies.
\vspace{-.2cm}