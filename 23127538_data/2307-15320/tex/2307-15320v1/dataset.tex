\section{Tasks and Datasets}
\vspace{-.1cm}
\label{sec:tasks}

\subsection{Manipulation tasks}
To assess the quality and robustness of our proposed sim-to-real pipeline, we propose a rich set of robotic manipulation tasks with different challenges, see  Figure~\ref{fig:overview}.

\noindent\textbf{Stacking.} The robot has to pick up a red cube on the table and stack it on top of a green cube. Both cubes have a side length of $5 \text{ cm}$. Stacking is most similar to our
proxy task and serves as a starting point to validate our sim-to-real pipeline.

\noindent\textbf{Box-retrieving.} A rectangular gray box of size $20.5\times21.5\times8 \text{ cm}^{3}$ has two compartments containing a green and a red cube respectively. The cubes have a side length of $5 \text{ cm}$. The box is on the table and its lid is closed.
The task requires a gripper to open the box lid, take out the red cube and place it on top of a green marker of size $6 \times 6 \text{ cm}^2$ outside of the box. This task is challenging because it is a long-horizon task that requires a decision which object to pick once the box is opened. 

\noindent\textbf{Assembling.} The task requires grasping a nut with a square hole located at a random position and orientation on the table, and then putting it on a pink rectangular parallelepiped of size $3.5\times3.5\times10.5 \text{  cm}^{3}$ fixed on the table. It requires gripper rotation and high-precision motion to put the nut on the screw. 


\noindent\textbf{Pushing.} The goal is to push a green cube of $4 \text{ cm}$ side length with the closed tip of the gripper to reach a square pink marker of $6 \times 6 \text{ cm}^2$ on the table. Pushing requires vision-based reactive behavior and closed-loop control to handle uncertainty of contacts between the gripper, the cube and the table. 

\noindent\textbf{Pushing-to-pick.} A red cube is surrounded by three obstacle cubes with random colors. The goal is to pick the red cube which cannot be performed directly. The gripper needs to first unlock the red cube by pushing two cubes on each side of the red cube. All the cubes are of $5 \text{ cm}$ side length. Pushing-to-pick combines challenges from pushing and picking into a long-horizon task.

\noindent\textbf{Sweeping.} 
The  goal is to sweep $14$ tiny foam cubes of $1.5 \text{ cm}$ side length lying on the table onto a  $16 \times 16 \text{  cm}^2$ green marker. To do so a squared broom of $7 \times 7 \text{ cm}^2$ is attached to the robot gripper.
Sweeping is significantly harder than pushing as it requires to sweep many small objects towards the goal while they can spread. To successfully perform the task, a policy has to perform several sweeps and recover the missing objects that did not reach the target zone, a behavior only obtained with a closed-loop system. 

\noindent\textbf{Rope-shaping.} A deformable rope of length $30  \text{ cm}$ and diameter of $3  \text{ cm}$ is in a random configuration on the table. The gripper needs to manipulate the rope to shape it into a straight line so that the ends of the rope are placed on top of two $6 \times 6 \text{ cm}^2$ pink and green markers. This task is challenging due to the manipulation of a deformable object.

\vspace{-.1cm}
\subsection{Dataset of expert demonstrations}
We leverage object information available in simulation to generate expert trajectories solving the tasks. 
Stacking, pushing-to-pick, box-retrieving and assembling can be solved with an open-loop oracle computing the solution trajectory before execution given the initial gripper and object configuration. 
However, we need to design a closed-loop oracle for pushing, rope-shaping and sweeping tasks, where trajectories need to be computed at each step to deal with the dynamics of the environment. The closed-loop oracles are finite-state machines whose states define a sub-goal to finish the task, e.g., move next to the cube, push the cube, rotate the cube, move to the initial gripper position, etc. At each step, the oracle on a particular state uses the current gripper and object configuration to compute both the next action and next oracle state until the task is completed.
The training datasets consists of $2,000$ demonstrations 
per task for stacking, box-retrieving, pushing and pushing-to-pick, and $4,000$ demonstrations per task for assembling, sweeping and rope-shaping in simulation.
We additionally collect $150$ real robot demonstrations for stacking. 
\vspace{-.15cm}