\section{Missing Proofs of Section~\ref{sec:prelim}}
\label{sec:prelims-extra}

\loodVsTmms*
\begin{proof}
%All valuations considered in this proof are additive.

First, assume $1$-out-of-$d$ MMS allocations exist for all instances with $n$ agents. Consider any instance $\mathcal{I} = ([d], M, (v_1, \ldots, v_d))$ with $d$ agents. Then, for the instance $\mathcal{I}' = ([n], M, (v_1, \ldots, v_n))$, a $1$-out-of-$d$ MMS allocation $X'$ exists. Hence, $v_i(X_i') \ge \MMS_i^d(M)$ for all $i \in [n]$. Let $X$ be an allocation of $M$ over $d$ agents where $X_i = X_i'$ for $i \in [n]$ and $X_i = \emptyset$ for $i \in [d] \setminus [n]$. Then $X$ is a $T$-MMS allocation for $\mathcal{I}$. Since the choice of $\mathcal{I}$ was arbitrary, we get that $T$-MMS allocations exist for all instances with $d$ agents.

Now assume $T$-MMS allocations exist for all instances with $d$ agents.
Consider any instance $\mathcal{I} = ([n], M, (v_1, \ldots, v_n))$.
Add $d-n$ dummy agents with arbitrary additive valuations $v_{n+1}, \ldots, v_d$.
Let $\mathcal{I}' = ([d], M, (v_1, \ldots, v_d))$ be the resulting instance.
Then a $T$-MMS allocation $X'$ exists for $\mathcal{I}'$, i.e., $v_i(X_i') \ge \MMS_i^d(M)$ for all $i \in [n]$.
Hence, $X = (X_1', \ldots, X_n')$ is a 1-out-of-$d$ MMS allocation for $\mathcal{I}$.
Since the choice of $\mathcal{I}$ was arbitrary, we get that 1-out-of-$d$ MMS allocations exist for all instances with $n$ agents.
\end{proof}

\ordNorm*
\begin{proof}
    Let $\ins$ be an arbitrary instance. We create a $d$-normalized ordered instance $\mathcal{I}'' = (N, M, V'')$ such that from any $1$-out-of-$d$ MMS allocation for $\mathcal{I}''$, one can obtain a $1$-out-of-$d$ MMS allocation for the original instance $\mathcal{I}$.

    First of all, we can ignore all agents $i$ with $\MMS^d_i=0$ since no good needs to be allocated to them. Recall that for all $i \in N$, $P^i = (P^i_1, \ldots, P^i_d)$ is a $d$-MMS partition of agent $i$. For all $i \in N$ and $g \in M$, we define $v'_{i,g} = v_i(g)/v_i(P^i_j)$ where $j$ is such that $g \in P^i_j$. Now for all $i \in N$, let $v'_i: 2^M \rightarrow \mathbb{R}_{\geq 0}$ be defined as an additive function such that $v'_i(S)=\sum_{g \in S} v'_{i,g}$.  Note that $v'_{i,g} \leq v_i(g)/\MMS^d_i(M)$ for all $g \in M$ and thus,
    \begin{align}
        v_i(S) \geq v'_i(S) \cdot \MMS^d_i(M). \label{ineq-apx}
    \end{align}
    Since $v'_i(P^i_j)=1$ for all $i \in N$ and  $j \in [d]$, $\mathcal{I}' = (N, M, V')$ is a $d$-normalized instance. If a $1$-out-of-$d$ MMS allocation exists for $\mathcal{I}$, let $X$ be one such allocation. By Inequality \eqref{ineq-apx}, $v_i(X_i) \geq v'_i(X_i) \cdot \MMS^d_i(M) \geq \MMS^d_i(M)$. Thus, every allocation that is $1$-out-of-$d$ MMS for $\mathcal{I}'$ is $1$-out-of-$d$ MMS for $\mathcal{I}$ as well.
    For all agents $i$ and $g \in [m]$, let $v''_{i,g}$ be the $g$-th number in the multiset of $\{v_i(1), \ldots, v_i(m)\}$. Let $v''_i: 2^M \rightarrow \mathbb{R}_{\geq 0}$ be defined as an additive function such that $v''_i(S)=\sum_{g \in S} v''_{i,g}$. Let $\mathcal{I''} = (N,M,V'')$. Note that $\mathcal{I}''$ is ordered and $d$-normalized.
    Barman and Krishnamurthy \cite{barman2020approximation} proved that for any allocation $X$ in $\mathcal{I''}$, there exists and allocation $Y$ in $\mathcal{I'}$ such that $v'_i(Y_i) \geq v''_i(X_i)$. Therefore, from any $1$-out-of-$d$ MMS allocation in $\mathcal{I}''$, one can obtain a $1$-out-of-$d$ MMS allocation in $\mathcal{I'}$ and as already shown before, it gives a $1$-out-of-$d$ MMS allocation for $\mathcal{I}$.
\end{proof}

\begin{comment}
\validred*
\begin{proof} Clearly, $v_a(S) \geq 3/4$. Next, we show that the MMS values of all other agents do not decrease separately for each case of $S\in\{S_1,S_2, S_3,S_4\}$. Fix agent $b \in N \setminus \{a\}$ and a MMS partition $P^b=(P^b_1, \dots, P^b_n)$. After removing $S$, we show that a partition of $M \setminus S$ exists into $(n-1)$ bundles such that the value of each bundle is at least $1$.
\begin{itemize}
    \item \textbf{$S=S_1$. } Removal of one good from $P^b$ affects exactly one bundle and each of the remaining $(n-1)$ bundles has value at least $1$. Therefore, the MMS value of $b$ doesn't decrease.
    \item \textbf{$S=S_2$. } In $P^b$, there exists a bundle with two goods from $\{1, \dots, n+1\}$ (pigeonhole principle). Let $T$ be a bundle in $P^b$ that has two goods from $\{1, \dots,n+1\}$. Let us exchange these goods with goods $n$ and $n+1$ in other bundles and arbitrarily distribute any remaining goods in $T$ among other bundles. Clearly, the value of other bundles except $T$ does not decrease, and hence the MMS value of $b$ in the reduced instance doesn't decrease.
    \item \textbf{$S=S_3$. } Similar to the proof of Case $`S = S_2$'.
    \item \textbf{$S=S_4$.} In each iteration, the lowest index bundle from $\{S_1,S_2,S_3,S_4 \}$ is picked. Therefore, $S_4$ is only picked when $v_i(S_1),v_i(S_3) < 3/4$ for all $i \in N$ which implies that $v_{i1} < 3/4$ and $v_{i(2n+1)} < 1/4$ and hence $v_i(S_4) < 1$ for all $i\in N$.

    In $P^b$, if goods $1$ and $2n+1$ are in the same bundle, removing $S_4$ and agent $a$ is a valid reduction. For the other case, if $1$ and $2n+1$ are in two different bundles, we can make two new bundles, one with $\{1,2n+1\}$ and another with all the remaining goods of the two bundles. The value of the bundle without $\{1,2n+1\}$ is at least $1$ because $v_i(S_4) < 1$ for all $i\in N$ and $\MMS_i \ge 1$.
   Hence, this is a valid reduction. \qedhere
\end{itemize}
\end{proof}
\end{comment}


\propExact*
\begin{proof}
For the first part,
    fix an agent $i$. Let $1 \in P^i_1$. By Proposition \ref{prop:trivial}, $v_i(1) \leq v_i(P^i_1)=1$.

    For the second part, by the pigeonhole principle, there exists a bundle $P^i_k$ and two goods $j, j' \in \{1, 2, \ldots, d+1\}$ such that $\{j,j'\} \subseteq P^i_k$. Without loss of generality, assume $j < j'$. We have
    \begin{align*}
        v_i(C_d) &= v_i(d) + v_i(d+1) \tag{$C_d = \{d, d+1\}$}\\
        &\leq v_i(j) + v_i(j') &\tag{$j \leq d$ and $j' \leq d+1$} \\
        &\leq v_i(P^i_k) =1. &\tag{$\{j,j'\} \in P^i_k$}
    \end{align*}

    For the third part, we have
    \begin{align*}
        1 &\geq v_i(C_d) = v_i(d) + v_i(d+1) \geq 2 v_i(d+1).
    \end{align*}
    Thus, $v_i(d+1) \leq \tf12$.
\end{proof}

\lemCsum*
\begin{proof}
For the sake of contradiction, assume the claim does not hold for some agent $i$ and let $\ell \geq 1$ be the largest index for which we have $\sum_{j = \ell}^d v_i(C_j) > d - \ell + 1$. Proposition \ref{prop:exact}(\ref{exact:2}) implies that $\ell < d$.
We have
\begin{align*}
    v_i(\ell) + v_i(2d-\ell+1) &= v_i(C_\ell) \\
    &= \sum_{j=\ell}^d v_i(C_j) - \sum_{j=\ell+1}^d v_i(C_j) \\
    &>  (d - \ell + 1) - (d - (\ell+1) + 1) &\tag{$\sum_{j = k}^d v_i(C_j) \leq d - k + 1$ for $k>\ell$} \\
    &=1.
\end{align*}
 For all $j,j' < \ell$, $v_i(j) + v_i(j') \geq v_i(\ell) + v_i(2d-\ell+1) > 1$. Therefore, $j$ and $j'$ cannot be in the same bundle in any $d$-MMS partition of $i$. For $j<\ell$, let $j \in P^i_j$. For all $j<\ell$ and $\ell \leq j' \leq 2d-\ell+1$,
 \begin{align*}
    v_i(j) + v_i(j') &\geq v_i(\ell) + v_i(2d-\ell+1) \\
    &= v_i(C_{\ell}) >1.
 \end{align*}
 Therefore, $j' \notin P_j$. Also, since $\sum_{j = \ell}^d v_i(C_j) > d - \ell + 1$, there are at least $t \geq d-\ell+2$ different bundles $Q_1, \ldots, Q_t$ in $P$ such that $Q_j \cap \{\ell, \ldots, 2d-\ell+1\} \neq \emptyset$. It is a contradiction since these $t \geq d-\ell+2$ bundles must be different from $P^i_1, \ldots P^i_{\ell-1}$.
\end{proof}
