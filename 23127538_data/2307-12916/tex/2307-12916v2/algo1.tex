\section{MMS with Agent Priority Ranking}
\label{sec:algo1}

We give an algorithm for approximate MMS allocations in the priority ranking setting.
We assume \wLoG{} that the input is $n$-normalized and ordered (see \cref{defn:normalized,defn:ordered}).
% We also assume \wLoG{} that for all $i \in [n]$, agent $i$ has priority ranking $i$.

Our algorithm first performs a few \emph{reduction} operations (phase 1) and then does bag filling (phase 2).
This approach is very similar to that of \cite{garg2021improved}.
We call our algorithm $\RBF$ (abbreviates `Reductions and Bag Filling')
and formally describe it in \cref{algo:rbf}.
We assume that the input to $\RBF$ is a pair $(\Ical, T)$, where
$\Ical \defeq ([n], [m], \Vcal)$ is an $n$-normalized fair division instance
such that $v_i(1) \ge v_i(2) \ge \ldots \ge v_i(m)$ for all $i \in [n]$,
and $T \defeq [\tau_1, \ldots, \tau_n]$ is a sequence of thresholds
such that $1 \ge \tau_1 \ge \ldots \ge \tau_n > 0$.
We say that an agent $i$ \emph{likes} a set $S$ of goods if $v_i(S) \ge \tau_i$.
We want to show that for a reasonable choice of $T$,
$\RBF$ gives each agent a bundle that they like.

For a set $S \subseteq \mathbb{Z}_{\ge 1}$ and an index $j \in [|S|]$,
let $\ordSt(S, j)$ be the $j\Th$ smallest number in $S$
(called the $j\Th$ order statistic of $S$).
For a finite set $J \subset \mathbb{Z}_{\ge 1}$, let
$\ordSt(S, J) \defeq \{\ordSt(S, j): j \in J \textrm{ and } j \le |S|\}$.

The first phase of the algorithm proceeds in multiple rounds.
In each round, we pick a set $S$ of goods and give it to an agent $i$.
The goods $S$ and agent $i$ are then removed from consideration in subsequent rounds.
This operation is called a \emph{reduction}.
Specifically, let $N$ and $M$ be the set of remaining agents and goods,
respectively, at the beginning of a round.
Let $S_1 \defeq \ordSt(M, 1)$, $S_2 \defeq \ordSt(M, \{|N|, |N|+1\})$,
$S_3 \defeq \ordSt(M, \{2|N|-1, 2|N|, 2|N|+1\})$, and $S_4 \defeq \ordSt(M, \{1, 2|N|+1\})$.
Then we find the smallest $k$ such that some agent likes $S_k$,
and the smallest $i$ such that $i$ likes $S_k$. Then we give $S_k$ to agent $i$.
If no such $k$ exists, then phase 1 ends.

\begin{algorithm}[htb]
\caption{$\RBF(\Ical, T)$:
\\ \textbf{Input:} Ordered and $n$-normalized instance $\Ical = ([n], [m], \Vcal)$
    and agent thresholds $T \defeq [\tau_1, \ldots, \tau_n]$.
\\ \textbf{Output:} (Partial) allocation $A = (A_1, \ldots, A_n)$.
}
\label{algo:rbf}
\begin{algorithmic}[1]
\State $N = [n]$
\State $M = [m]$
\While{$|N| > 0$ and $|M| > 0$}
    \State Let $S_1 \defeq \ordSt(M, 1)$.
    \State Let $S_2 \defeq \ordSt(M, \{|N|, |N|+1\})$.
    \State Let $S_3 \defeq \ordSt(M, \{2|N|-1, 2|N|, 2|N|+1\})$.
    \State Let $S_4 \defeq \ordSt(M, \{1, 2|N|+1\})$.
    \State Find the (lexicographically) smallest pair $(k, i) \in [4] \times N$ such that
        $v_i(S_k) \ge \tau_i$. Let $k$ and $i$ be $\Null$ if no such pair exists.
        \Comment{this event is called a type-$k$ reduction.}
    \If{$i$ is not $\Null$}
        \State Give $S_k$ to agent $i$.
            Set $N = N \setminus \{i\}$ and $M = M \setminus S_k$.
    \Else
        \State break
    \EndIf
\EndWhile
%
\LineComment{$|M| \ge 2|N|$ now (see \cref{thm:m-ge-2n}).}
\State $\bagFillHyp((N, M, \Vcal), T)$
\end{algorithmic}
\end{algorithm}

Then in phase 2, we perform bag filling (\cref{algo:bagFill}) on the remaining instance.
We create $|N|$ bags, where in the $i\Th$ bag, we add the $i\Th$ and $(2n-i+1)\Th$
most valuable goods (so the first $2|N|$ goods are in bags).
Then we repeatedly do the following till all agents receive a bag:
Find the smallest $i$ such that agent $i$ likes some bag, and give that bag to $i$.
If no such $i$ exists (i.e., no agent likes any bag),
add the most valuable remaining good to an arbitrary bag.

\begin{algorithm}[htb]
\caption{$\bagFill(\Ical, T)$
\\ \textbf{Input:} Ordered instance $\Ical = ([n], [m], \Vcal)$ with $m \ge 2n$,
    and agent thresholds $T \defeq [\tau_1, \ldots, \tau_n]$.
\\ \textbf{Output:} (Partial) allocation $A = (A_1, \ldots, A_n)$.
}
\label{algo:bagFill}
\begin{algorithmic}[1]
\For{$i \in [n]$}
    \State $B_i = \{i, 2n+1-i\}$.
    \State $A_i = \emptyset$.
\EndFor
\State $U_G = [m] \setminus [2n]$  \Comment{unassigned goods}
\State $U_A = [n]$  \Comment{unsatisfied agents}
\State $U_B = [n]$  \Comment{unassigned bags}
\While{$U_A \neq \emptyset$}
    \Comment{loop invariant: $|U_A| = |U_B|$}
    \State Let $i$ be the smallest in $U_A$ such that for some $k \in U_B$,
        we have $v_i(B_k) \ge \tau_i$. Let $i = \Null$ otherwise.
    \If{$i$ is not $\Null$}
        \LineComment{assign the $k\Th$ bag to agent $i$:}
        \State $A_i = B_k$
        \State $U_A = U_A \setminus \{i\}$
        \State $U_B = U_B \setminus \{k\}$
    \ElsIf{$U_G \neq \emptyset$}
        \State $g$ = most valuable good in $U_G$
        \State $k$ = arbitrary bag in $U_B$
        \LineComment{assign $g$ to the $k\Th$ bag:}
        \State $B_k = B_k \cup \{g\}$.
        \State $U_G = U_G \setminus \{g\}$
    \Else
        \State \label{alg-line:bagFill:error}\textbf{error}: we ran out of goods.
        \State Allocate the bags in $U_B$ to agents in $U_A$ arbitrarily.
        \State \Return $(A_1, \ldots, A_n)$.
        \LineComment{agents $U_A$ are not satisfied with their bags.}
    \EndIf
\EndWhile
\State \Return $(A_1, \ldots, A_n)$
\end{algorithmic}
\end{algorithm}

Our algorithm ($\RBF$) is the same as that of \cite{garg2021improved}, with a few minor differences:
\begin{enumerate}
\item We assume the input to be $n$-normalized, whereas they do not.
\item In both phases 1 and 2, we assume that when multiple agents like a bundle,
    we give it to the smallest-numbered agent, whereas they break ties arbitrarily.
\item They use the same thresholds for all agents, i.e., $\tau_1 = \ldots = \tau_n$.
\end{enumerate}

The idea behind reductions is to shrink the set of agents to consider:
we want to give each agent a bundle that they like, and since each agent selected for a reduction
obtains a bundle that she likes, we only need to worry about the remaining agents.
Reductions also shrink the set of goods, though, and it is therefore necessary to argue
that the set of remaining goods is enough to satisfy the remaining agents,
i.e., we are not giving away too many goods during reductions.

\cite{garg2021improved} showed this using the idea of \emph{valid reductions}:
a reduction is called \emph{valid} if it preserves the MMS values of the remaining agents.
Formally, they showed that if $\tau_j \le 3/4$ for all $j$,
and if a reduction involves giving $S_k$ to agent $i$, then for all $j \in N \setminus \{i\}$,
we have $\MMS_j^{|N|-1}(M \setminus S_k) \ge \MMS_j^{|N|}(M)$.
Hence, if the initial MMS value of each agent is 1
(which happens if we start with an $n$-normalized instance),
then the MMS value of each remaining agent after phase 1 is at least 1.
Since \cite{garg2021improved} aim to find $3/4$-MMS allocations,
they assume \wLoG{} that the input instance is irreducible, i.e., no reductions are possible.
They show that irreducible instances have nice properties, which they exploit to show that
$\bagFill$ outputs a $3/4$-MMS allocation when its input is irreducible.

Unfortunately, reductions in $\RBF$ are not valid, i.e.,
they can decrease the MMS value of the remaining agents.
Specifically, when $\tau_j > 3/4$ for an agent $j$,
then it can happen that $\MMS^{n-1}_j(M \setminus S_4) < \MMS^n_j(M)$.
Moreover, even for fair division instances where reductions are valid
(e.g., no type-4 reductions take place),
it is unclear how to exploit the validity of reductions in our analysis.
This is because reductions can change the priority rank of agents, e.g.,
the $(n/2)\Th$ most important agent in the original instance may end up becoming
the least important in the reduced instance if $n/2$ reductions take place.
Hence, we need to use other techniques. 

\subsection{Analysis of \texorpdfstring{$\RBF$}{RBF}}

We now show that when the thresholds are not too high, $\RBF$ outputs an allocation
where each agent $i$ gets a bundle of value at least $\tau_i$.

\begin{lemma}[follows from \cite{garg2021improved}]
\label{thm:rbf-gt}
Let $A$ be the allocation output by $\RBF(\Ical, T)$. For all $i \in [n]$,
if $\tau_i = \frac{3}{4} + \frac{1}{12n}$, then $v_i(A_i) \ge \tau_i$.
\end{lemma}

Suppose $p$ reductions happen during $\RBF$, where the $j\Th$ reduction is a type-$r_j$ reduction.
Call $R \defeq (r_1, \ldots, r_p)$ the \emph{reduction sequence}.
To analyze $\RBF$, we must understand the structure of $R$.

\begin{lemma}
\label{thm:redn-seq}
The reduction sequence $R$ for $\RBF$ is captured by the following regular expression: $(1^*2^*4^*)(32^*4^*)^*$.
Equivalently, if $n_3$ type-3 reductions happened in $\RBF$, then we can partition $R$ into
$n_3+1$ contiguous sub-lists $(R_0, \ldots, R_{n_3})$, such that all of the following hold:
\begin{enumerate}
\item $R_0$ is a sequence of zero or more 1s, followed by zero or more 2s, followed by zero or more 4s.
    ($R_0$ can be empty.)
\item For $i \in [n_3]$, the first element in $R_i$ is 3, followed by zero or more 2s, followed by zero or more 4s.
\end{enumerate}
\end{lemma}
\begin{proof}
All type-1 reductions happen together in the beginning.
This is because a different kind of reduction can only happen if a type-1 reduction is not possible.
If a type-1 reduction is not possible, then for each good $g$ and each remaining agent $i$,
$v_i(g) < \tau_i$. Hence, a type-1 reduction will never happen in the future.

Call $R_i$ the $i\Th$ \emph{phase} of \ $\RBF$.
Within each phase, all type-2 reductions happen before all type-4 reductions.
This is because a type-4 reduction happens only if a type-2 reduction is not possible.
After a type-4 reduction, the value of $S_2$ remains the same,
so a type-2 reduction will never occur again in the same phase.
\end{proof}

\begin{lemma}
\label{thm:m-ge-2n}
In $\RBF$, after all type-1 reductions have happened, we have $|M| \ge 2|N|$
throughout the execution of the algorithm.
\end{lemma}
\begin{proof}
Let $\Ical \defeq ([n], [m], \Vcal)$ be the initial fair division instance (the input to $\RBF$).

By \cref{thm:redn-seq}, all type-1 reductions happen before all other reductions.
Suppose $k$ type-1 valid reductions happened. Immediately after that,
$|M| = m - k$ and $|N| = n - k$. If $|M| < 2|N|$, then $m < 2n-k$.
For any agent $i \in [n]$, consider her partition $P \defeq (P_1, \ldots, P_n)$ for $\Ical$.
Since $m < 2n-k$, at least $k+1$ singleton bundles in $P$ exist.
Since $\Ical$ is normalized, the goods in those singleton bundles have a value of $1$ for $i$.
But at least $k+1$ type-1 reductions should have happened, which is a contradiction.
Hence, immediately after type-1 reductions, we have $|M| \ge 2|N|$.

Type-3 and type-4 reductions only happen when $|M| \ge 2|N|+1$,
so they preserve the invariant $|M| \ge 2|N|$.
\end{proof}

\begin{lemma}
\label{thm:redn-vs-bag}
In $\RBF$, after all reductions have happened, let $M_f$ and $N_f$ be
the set of remaining goods and remaining agents, respectively.
(By \cref{thm:m-ge-2n}, $|M_f| \ge 2|N_f|$.)
Then for any good $g$ in a type-$k$ reduction bundle, we have
\begin{enumerate}
\item $g > \ordSt(M_f, |N_f|)$ if $k = 2$.
\item $g > \ordSt(M_f, 2|N_f|)$ if $k = 3$.
\end{enumerate}
\end{lemma}
\begin{proof}
Suppose $p$ reductions happen during $\RBF$. For any $i \in [p]$,
let $S_i$ be the $i\Th$ reduction bundle and let $k_i$ be the type of the $i\Th$ reduction.
Let $M_i$ and $N_i$ be the set of remaining goods and remaining agents, respectively,
after $i$ reductions have happened. (Hence, $(N_0, M_0, \Vcal)$ is the original instance.)

Define proposition $P(i)$ as: For any $j \in [i]$ and any $g \in S_i$,
$g > \ordSt(M_i, |N_i|)$ if $k = 2$ and $g > \ordSt(M_i, 2|N_i|)$ if $k = 3$.

We prove $P(i)$ for all $i \in \{0\} \cup [p]$ using induction:
The base case $i = 0$ is trivial since no reductions have happened.
Now for $i \in [p]$, we want to prove $P(i-1) \implies P(i)$.

\textbf{Case 1}: $k_i = 1$:
\\ By \cref{thm:redn-seq}, no type-2 or type-3 reductions happened among the first $i$ reductions.
Hence, $P(i)$ holds trivially.

For the remaining cases, $\ordSt(M_i, |N_i|)$ and $\ordSt(M_i, 2|N_i|)$ are well-defined
because of \cref{thm:m-ge-2n}.

\textbf{Case 2}: $k_i = 2$:
\\ Then $S_i = \ordSt(M_{i-1}, \{|N_{i-1}|, |N_{i-1}|+1\})$,
$\ordSt(M_i, |N_i|) = \ordSt(M_{i-1}, |N_{i-1}|-1)$,
and $\ordSt(M_i, 2|N_i|) = \ordSt(M_{i-1}, 2|N_{i-1}|)$.
Hence, $P(i)$ holds.

\textbf{Case 3}: $k_i = 3$:
\\ Then $S_i = \ordSt(M_{i-1}, \{2|N_{i-1}|-1, 2|N_{i-1}|, 2|N_{i-1}|+1\})$,
$\ordSt(M_i, |N_i|) = \ordSt(M_{i-1}, |N_{i-1}|-1)$,
and $\ordSt(M_i, 2|N_i|) = \ordSt(M_{i-1}, 2|N_{i-1}|-2)$.
Hence, $P(i)$ holds.

\textbf{Case 4}: $k_i = 4$:
\\ Then $S_i = \ordSt(M_{i-1}, \{1, 2|N_{i-1}|+1\})$,
$\ordSt(M_i, 2|N_i|) = \ordSt(M_{i-1}, 2|N_{i-1}|-1)$,
and $\ordSt(M_i, |N_i|) = \ordSt(M_{i-1}, |N_{i-1}|)$.
Hence, $P(i)$ holds.

By mathematical induction, we get that $P(p)$ holds.
\end{proof}

\begin{lemma}[Lemma 6 in \cite{akrami2023simplification}]
\label{thm:bag-top-le-third}
In an ordered and normalized fair division instance, for all $i,k \in [n]$,
if $v_i(\{k, 2n+1-k\}) > 1$, then $v_i(2n+1-k) \le 1/3$ and $v_i(k) > 2/3$.
\end{lemma}

\begin{lemma}
\label{thm:rbf-1}
Let $A$ be the allocation output by $\RBF(\Ical, T)$.
For all $i \in [n]$, if $\tau_i = 2n/(2n+i-1)$, then $v_i(A_i) \ge \tau_i$.
\end{lemma}
\begin{proof}
Suppose for some $i \in [n]$, we have $\tau_i = 2n/(2n+i-1)$ and $v_i(A_i) < \tau_i$.
We will try to upper-bound the total value agent $i$ has for all the bundles.
To do this, we associate a \emph{charge} $c(A_j)$ with each bundle $A_j$ for $j \in [n]$
such that all of these properties hold:
\begin{enumerate}
\item The total charge of all bundles equals the total value of all bundles.
\item The charge of each bundle is less than $3\tau_i/2$.
\item If a bundle has value less than $\tau_i$ to agent $i$, then it is charged less than $\tau_i$.
\end{enumerate}

The total number of reductions that happen is at most $n-1$
(otherwise, $i$ would have gotten a bundle in some reduction).
In the instance after all reductions have happened,
let $M_f$ and $N_f$ be the set of remaining goods and remaining agents, respectively.
Let $z' \defeq v_i(\ordSt(M_f, |N_f|))$ and $z \defeq v_i(\ordSt(M_f, |N_f|+1))$.
Then $z' + z < \tau_i$ and $z' \ge z$, so $z < \tau_i/2$.

We now upper-bound the charge of each bundle in $A$.
To do this we consider three cases.

\textbf{Case 1}: type-1 and type-4 reduction bundles:
\\ Let $S$ be a type-$k$ reduction bundle, for $k \in \{1, 4\}$. Let $c(S) \defeq v_i(S)$.
Let $M$ and $N$ be the set of remaining goods and remaining agents immediately before the reduction.
Then $|N| \ge 2$, since at most $n-1$ valid reductions happen.
\begin{enumerate}
\item $k = 1$: then $v_i(S) \le v_i(1) \le 1$, since the instance is normalized.
\item $k = 4$: then a type-3 reduction was not possible. Hence,
    $v_i(S) = v_i(\ordSt(M, \{1, 2|N|+1\})) < \tau_i + \tau_i/3 = 4\tau_i/3$.
\end{enumerate}

\textbf{Case 2}: type-2 and type-3 reduction bundles:
\\ By \cref{thm:redn-seq}, we can partition the sequence of
reduction events into sub-lists $(R_0, R_1, \ldots, R_{n_3})$.
Let $S$ be a type-2 reduction bundle in $R_0$. Then $S = \{j, 2n+1-j\}$ for some $j$.
If $v_i(S) > 1$, then by \cref{thm:bag-top-le-third}, we get that $v_i(2n+1-j) \le 1/3$.
Also, $v_i(j) < \tau_i$, since no type-1 reduction was possible.
Hence, $v_i(S) < \tau_i + 1/3$ if $v_i(S) > 1$.
Since $\tau_i = 2n/(2n+i-1) \ge 2n/(3n-1) > 2/3$, we get
$1 < 3\tau_i/2$ and $\tau_i + 1/3 < 3\tau_i/2$.
Hence, $v_i(S) < 3\tau_i/2$. Let $c_i(S) \defeq v_i(S)$.

Consider the sub-list $R_t$ for some $t > 0$.
In the instance immediately before the first reduction in $R_t$,
let $M'$ and $N'$ be the set of remaining goods and agents, respectively.
Let $S_j$ be the $j\Th$ type-2 reduction bundle in $R_t$,
and let $T$ be the type-3 reduction bundle in $R_t$.
\begin{enumerate}
\item By \cref{thm:redn-vs-bag}, $\ordSt(M', 2|N'|-1) > \ordSt(M_f, 2|N_f|)$.
    Hence, $v_i(T) \le 3z < 3\tau_i/2$.
\item $S_1 = \ordSt(M', \{|N'|-1, |N'|\})$. By \cref{thm:redn-vs-bag},
    $\ordSt(M', |N'|-1) > \ordSt(M_f, |N_f|)$. Hence, $v_i(S_1) \le 2z' < 2(\tau_i - z)$.
\item For $j \ge 2$, we have $S_j = \ordSt(M', \{|N'|-j, |N'|+(j-1)\})$.
    A type-2 reduction was not possible at the beginning of $R_t$,
    so $v_i(\ordSt(M', |N'|+1)) < \tau_i/2$. Hence, $v_i(S_j) < 3\tau_i/2$.
    Let $c(S_j) \defeq v_i(S_j)$.
\end{enumerate}
If no type-2 reduction happened in $R_t$, let $c(T) \defeq v_i(T)$.

Now, suppose at least one type-2 reduction happened in $R_t$.
Let $\zeta \defeq v_i(T) + v_i(S_1) < 3z + 2(\tau_i - z) < 5\tau_i/2$.
Define $c(T)$ and $c(S_1)$ as follows:
\[ (c(T), c(S_1)) = \begin{cases}
(\frac{3}{5}\zeta, \frac{2}{5}\zeta)
    & \textrm{ if } v_i(T) > \frac{3}{5}\zeta
\\ (\frac{2}{5}\zeta, \frac{3}{5}\zeta)
    & \textrm{ if } v_i(S_1) > \frac{3}{5}\zeta
\\ (v_i(T), v_i(S_1)) & \textrm{ otherwise}
\end{cases}. \]
Let $x \defeq \max(c(T), c(S_1))$ and $y \defeq \min(c(T), c(S_1))$.
Then $x \le \frac{3}{5}\zeta < 3\tau_i/2$.
Furthermore, if $v_i(S_1) < \tau_i$, then $c(S_1) < \tau_i$.
This is because if $c(S_1) \neq v_i(S_1)$, then $c(S_1) = \frac{2}{5}\zeta < \tau_i$.
Similarly, $v_i(T) < \tau_i \implies c(T) < \tau_i$.

\textbf{Case 3}: bag filling bundles:
\\ For a bag filling bundle, its initial value is less than $\tau_i + z < 3\tau_i/2$.
If it received some goods during bag filling, then its value to $i$ before the last good
was added is less than $\tau_i$. Since no type-3 reduction is possible,
each good added to a bag has a value less than $\tau_i/3$ to agent $i$.
Hence, the bag's value is less than $4\tau_i/3$ to agent $i$.
Charge a bag equal to its value.

Now, we combine all the cases. There are $n$ bundles in the allocation.
Since agent $i$ did not get a bundle of value $\ge \tau_i$,
and she has higher priority than $n-i$ other agents, we get that
each such bundle has value less than $\tau_i$ to agent $i$,
and hence has charge less than $\tau_i$.
The remaining $i-1$ agents get a bundle of charge less than $3\tau_i/2$. Hence,
\[ n = \sum_{j=1}^n v_i(A_j) = \sum_{j=1}^n c(A_j)
    < (i-1)(3\tau_i/2) + (n-i+1)\tau_i = (\tau_i/2)(2n+i-1). \]
Therefore, $\tau_i > 2n/(2n+i-1)$. This is a contradiction.
Hence, if $\tau_i = 2n/(2n+i-1)$, then $v_i(A_i) \ge \tau_i$.
\end{proof}

\begin{theorem}
\label{thm:rbf}
Let $\Ical$ be a fair division instance and $T \defeq (\tau_1, \ldots, \tau_n)$ where for all $i \in [n]$,
\[ \tau_i \defeq \max\left(\frac{2n}{2n+i-1}, \frac{3}{4} + \frac{1}{12n}\right). \]
Let $A$ be the allocation output by $\RBF(\Ical, T)$. Then $A$ is a $T$-MMS allocation,
i.e., for all $i \in [n]$, we have $v_i(A_i) \ge \tau_i$.
\end{theorem}
\begin{proof}
Follows from \cref{thm:rbf-1,thm:rbf-gt}.
\end{proof}

\subsection{Best-of-Both-Worlds Fairness}
\label{sec:algo1:bobw}

In the best-of-both-worlds setting, the aim is to output a distribution (ideally of small support)
over allocations that is both ex-ante fair and ex-post fair.
We first show what guarantees we can get for any algorithm by assigning priority ranks randomly.

\begin{lemma}
\label{thm:cyclic-perm}
Let $T \defeq (\tau_1, \ldots, \tau_n)$ be a list of thresholds and let $\Acal$ be
an algorithm that outputs a $T$-MMS allocation for every fair division instance having $n$ agents.
For $k \in [n]$, let $A^{(k)}$ be the allocation output by algorithm $\Acal$
when agent $i$ has priority rank
\[ \begin{cases}
i + k - 1 & \textrm{ if } i + k - 1 \le n
\\ i + k - 1 - n & \textrm{ otherwise}
\end{cases}. \]
If we sample $k$ uniformly randomly from $[n]$, then $A^{(k)}$ is
ex-post $\tau_n$-MMS and ex-ante $(\frac{1}{n}\sum_{i=1}^n \tau_i)$-MMS.
\end{lemma}
\begin{proof}
For each agent $i$, her priority rank is distributed uniformly over $[n]$.
\end{proof}

Note that in \cref{thm:cyclic-perm}, if $\Acal$ is deterministic, then instead of outputting
just a random allocation, we can output the entire distribution by finding $A^{(k)}$ for all $k \in [n]$.
This is useful if the agents want to verify that the distribution is indeed ex-ante fair.

We now show that by randomizing priority ranks as in \cref{thm:cyclic-perm}
and picking thresholds as in \cref{thm:rbf}, $\RBF$ is
ex-post $(\frac{3}{4} + \frac{1}{12n})$-MMS and ex-ante $(0.82536 + \frac{1}{36n})$-MMS.

\begin{restatable}{lemma}{rthmRbfAvg}
\label{thm:rbf-avg}
For all $n \ge 1$, we get
\[ \gamma \defeq \frac{1}{n}\sum_{i=1}^n \max\left(\frac{2n}{2n+i-1}, \frac{3}{4} + \frac{1}{12n}\right)
    \ge 2\ln\left(\frac{4}{3}\right) + \frac{1}{4} + \frac{1}{36n}. \]
\end{restatable}
\begin{proof}
(Proof deferred to \cref{sec:prio-extra:calculations}).
\end{proof}

\begin{theorem}
\label{thm:rbf-rand}
Let $\Ical$ be a fair division instance having $n$ agents, $T \defeq (\tau_1, \ldots, \tau_n)$, and
$\tau_i \defeq \max\big(\frac{2n}{2n+i-1},\allowbreak \frac{3}{4} + \frac{1}{12n}\big)$ for all $i \in [n]$.
Sample $k$ uniformly randomly from $[n]$, and let agent $i$'s priority rank be
$i + k - 1$ if $i + k - 1 \le n$ and $i + k - 1 - n$ otherwise.
Then the allocation output by $\RBF(\Ical, T)$ is ex-post $(\frac{3}{4} + \frac{1}{12n})$-MMS
and ex-ante $\beta$-MMS, where
$\beta \defeq 2\ln\left(\frac{4}{3}\right) + \frac{1}{4} + \frac{1}{36n}
\approx 0.82536 + \frac{1}{36n}$.
\end{theorem}
\begin{proof}
Follows from \cref{thm:rbf,thm:cyclic-perm,thm:rbf-avg}.
\end{proof}
