\section{\texorpdfstring{$(\alpha, \beta, \gamma)$}{(alpha, beta, gamma)}-MMS Allocation}\label{sec:or}

In this section, we show the existence of $(2(1-\beta)/\beta, \beta, 3/4)$-MMS allocation for any $3/4 < \beta < 1$. Using $\beta = \sqrt{3}/2$, this implies the existence of a randomized allocation that gives each agent at least $3/4$ times her MMS value (ex-post) and at least $(34\sqrt{3} - 48)/8\sqrt{3} > 0.785$ times her MMS value in expectation (ex-ante). 

Given an instance $\ins$, without loss of generality, we assume that $\I$ is $n$-normalized and ordered, which implies that $\MMS_i = 1,\forall i\in N$. Since our approach is an extension of the Garg-Taki (GT) algorithm~\cite{garg2021improved} for the existence of $3/4$-MMS allocation, we first summarize their algorithm. GT algorithm has two phases of \emph{valid reductions} and \emph{bag filling}. In a valid reduction, the instance is reduced by removing an agent $a$ and a subset $S$ of goods such that $a$'s value for $S$ is at least $3/4$, and the MMS values of the remaining agents do not decrease, i.e., $v_a(S) \ge 3/4$ and $\MMS_i \ge 1$ for each remaining agent in the reduced instance $(N\setminus\{a\}, M\setminus\{S\}, V\setminus\{v_a\})$.

The GT algorithm utilizes the simple valid reductions with the set of goods $S_1= \{1\}$ (i.e., the highest valued good), $S_2 = \{n, n+1\}$, $S_3 = \{2n-1, 2n, 2n+1\}$, and $S_4=\{1, 2n+1\}$, in the priority order of $S_1$, $S_2$, $S_3$, and $S_4$, i.e., $S_k$ is performed only when for all $j<k$, $S_j$ is not feasible. The following lemma (proof is in Appendix~\ref{app}) shows that for all $k\in[4]$, $S_k$ is a valid reduction if performed in the priority order. 

\begin{restatable}{lemma}{validred}\cite{garg2021improved}
Let $S$ be the lowest index bundle in $S\in\{S_1,S_2,S_3,S_4\}$ for which $\{i \in N: v_i(S) \geq 3/4 \}$ is non-empty. Then, removing $S$ and agent $a$ with $v_a(S) \ge 3/4$ is a valid reduction.
\end{restatable}

Let $\I'=([n'], [m'], V')$ be the instance after all the valid reductions are performed, i.e., no more valid reductions are feasible for $\I'$. This gives some information about the values of goods shown in the following corollary. 

\begin{corollary}\label{cor:val}
If no valid reductions are feasible for $\I'=([n'], [m'], V')$, then for any agent $i\in [n']$, $v_i(1) < 3/4$, $v_i(n'+1) < 3/8$, and $v_i(2n'+1) < 1/4$. 
\end{corollary}
\begin{algorithm}[t!]
\caption{The $(2(1-\beta)/\beta, \beta, 3/4)$-MMS Algorithm}\label{algo:or}
    \textbf{Input:} $n$-normalized ordered instance $\ins$, where $\MMS_i = 1, \forall i\in N$, and $\beta\in (3/4, 1)$\\
    \textbf{Output:} $(2(1-\beta)/\beta, \beta, 3/4)$-MMS allocation $A$
\begin{algorithmic}[1]
  \State $N_1 \gets$ an arbitrary set of agents s.t. $|N_1| \le 2(1-\beta)|N|/\beta$, with a target of at least $\beta$-MMS
  \State $N_2 \gets N\setminus N_1$, set of remaining agents ($|N_2| \ge (3\beta -2)|N|/\beta$), with a target of at least $3/4$-MMS
  \State $S_1\gets \{1\}$;\ \ $S_2\gets \{|N|, |N|+1\}$;\ \ $S_3\gets \{2|N|-1, 2|N|, 2|N|+1\}$;\ \ $S_4\gets\{1, 2|N|+1\}$
  \State $T_1 \gets \{i\in N_1: \exists k\in [4], v_i(S_k) \ge \beta\}$; \ \ $T_2 \gets \{i\in N_2: \exists k\in [4], v_i(S_k) \ge 3/4\}$
  \While{$T_1 \cup T_2 \neq \emptyset$} \Comment{Valid Reductions}
	   \State $\ell\gets$ smallest index s.t. either $v_i(S_\ell) \ge \beta$, for some $i\in N_1$ or $v_i(S_\ell) \ge 3/4$, for some $i\in N_2$
	   \State $a \gets$ agent in $N_1$ with the highest value for $S_\ell$
	   \If{$v_a(S_\ell) \ge \beta$} 
	      \State $A_a\gets S_\ell$;\ $N \gets N \setminus \{a\}$;\ $M\gets M\setminus S_\ell$
	   \Else 
	      \State $a\gets$ agent in $N_2$ such that $v_a(S_\ell) \ge 3/4$
	   	 \State $A_a\gets S_\ell$;\ $N \gets N \setminus \{a\}$;\ $M\gets M\setminus S_\ell$    
	   \EndIf
	   \State $S_1\gets \{1\}$;\ \ $S_2\gets \{|N|, |N|+1\}$;\ \ $S_3\gets \{2|N|-1, 2|N|, 2|N|+1\}$;\ \ $S_4\gets\{1, 2|N|+1\}$
	   \State $T_1 \gets \{i\in N_1: \exists k\in [4], v_i(S_k) \ge \beta\}$; \ \ $T_2 \gets \{i\in N_2: \exists k\in [4], v_i(S_k) \ge 3/4\}$
  \EndWhile
  \State $n'\gets$ $|N_1 \cup N_2|$
  \State $R\gets$ $M\setminus [2n']$
  \State Initialize $n'$ bags as in \eqref{eq:B_i}
  \For{each $B_k, k\in [n']$} \Comment{Bag Filling}
  	\State $T \gets \{i\in N_1: v_i(B_k) \ge \beta\} \cup \{i\in N_2: v_i(B_k) \ge 3/4\}$
  	\While{$T = \emptyset$}
	  \State $B_k \gets B_k \cup \{g\}, g\in R$;\ $R\gets R\setminus \{g\}$ 
       \State $T \gets \{i\in N_1: v_i(B_k) \ge \beta\} \cup \{i\in N_2: v_i(B_k) \ge 3/4\}$
	\EndWhile
	  \State $a \gets$ agent in $N_1$ with the highest value for $B_k$
	  \If{$v_a(B_k) \ge \beta$} 
	     \State $A_a\gets B_k$;\ $N_1 \gets N_1 \setminus \{a\}$
	  \Else 
	     \State $a\gets$ agent in $N_2$ such that $v_a(B_k) \ge 3/4$ \Comment{$a$ exists due to while condition}
	     \State $A_a\gets B_k$;\ $N_2 \gets N_2 \setminus \{a\}$
	  \EndIf
  \EndFor
\end{algorithmic}
\end{algorithm}


In the bag filling phase, $n'$ bags are initialized using the first $[2n']$ goods as in~\eqref{eq:B_i} and each bag is filled with goods in $[m']\setminus [2n']$ until some agent has a value at least $3/4$ for the bag. Although the GT algorithm is quite simple, the main challenge is showing there are enough goods in $[m']\setminus [2n']$ to satisfy each agent with a value of at least $3/4$.

Our algorithm is described in Algorithm~\ref{algo:or}. We start with an arbitrary set $N_1$ of agents such that $|N_1|\le 2n(1-\beta)/\beta$, and our goal is to satisfy each of them with a value of at least $\beta$. $N_2$ is the set of remaining agents, and our goal is to satisfy each of them with a value of at least $3/4$. Like the GT algorithm, Algorithm~\ref{algo:or} also has two phases, valid reductions and bag filling, albeit with some crucial differences. In the valid reduction phase, we use different targets for $N_1$ and $N_2$, but we prioritize agents in $N_1$. We first check if a valid reduction is feasible for an agent in $N_1 \cup N_2$ with some $S_k, k\in[4]$. If yes, we pick the smallest feasible index, say $S_\ell$, and find an agent, say $a$, in $N_1$ with the highest value for $S_\ell$. If this value is at least $\beta$, we assign $S_\ell$ to agent $a$. Otherwise, we assign $S_\ell$ to any agent in $N_2$ with a value of at least $3/4$. 

We run the bag-filling phase on the reduced instance if no valid reductions are feasible. This phase is similar to the GT algorithm except that we again use different targets for agents in $N_1$ and $N_2$ and prioritize agents in $N_1$. For all $i \in [n']$, let $B_i$ be the $i^{\text{th}}$ initial bag (i.e., $B_i=\{i,2n'-i+1\}$) and $\hat{B}_i$ be the bag at the end of the algorithm (i.e., after the bag-filling phase).

Although Algorithm~\ref{algo:or} is a simple extension of the GT algorithm, the analysis is not straightforward because $S_4$ is not a valid reduction for $N_1$. We first show that each agent in $N_2$ will receive a bag valued at least $3/4$ at the end of the algorithm. 

%The next two lemmas show that all agents are satisfied with their respective targets, proving the algorithm's correctness.

\begin{lemma}\label{lem:N2}
Each agent in $N_2$ receives a bag valued at least $3/4$.
\end{lemma}

\begin{proof}
It easily follows from the GT algorithm because valid reductions do not decrease the MMS value of any remaining agent in $N_2$, and the bag filling does not add goods from $[m']\setminus [2n']$ to a bag when some agent in $N_2$ has a value at least $3/4$. 
\end{proof}

In the rest of the section, we show the following claim, proving the algorithm's correctness. 

\begin{lemma}\label{lem:n1}
Each agent in $N_1$ receives a bag valued at least $\beta$.
\end{lemma}

For a contradiction, suppose the bag filling phase stopped at iteration $k\le n'$ because $[m']\setminus [2n']$ is empty and an agent $a\in N_1$ has not received a bag. Since no more valid reductions are feasible, like in Corollary~\ref{cor:val}, we must have 
\begin{equation}
v_a(j) < \begin{cases} \beta & \forall j\le n'\\ \beta/2 & \forall n' < j \le 2n'\\ \beta/3 & \forall j> 2n' \end{cases}
\end{equation}

This implies that at the beginning of the bag filling phase, 
\begin{equation}\label{eqn:bk}
v_i(B_k) < 3\beta/2, \forall k\in [n']. 
\end{equation}
Further, if $\hat{B} \neq B_k$ for some $k$, then $v_a(\hat{B}) < \beta + \beta/3 = 4\beta/3$ for the bags assigned to other agents before iteration $k$ because $v_a(\hat{B}\setminus g) < \beta$, where $g$ is the last good added to $\hat{B}$ and $v_a(g) \le v_a(2n'+1) < \beta/3$. Also, $v_a(\hat{B}) < \beta$ for all bags assigned to agents in $N_2$ because we prioritize agents in $N_1$. Therefore, in the bag-filling phase, we have

\begin{equation}\label{eqn:hbk}
v_a(\hat{B}) < \begin{cases} \beta  & \text{ if } \hat{B} \text{ is assigned to an agent in } N_2\\
3\beta/2  & \text{ if } \hat{B} \text{ is assigned to an agent in } N_1  \end{cases}.
\end{equation}

During valid reductions, since we prioritize agents in $N_1$, we have 
\begin{equation}
v_a(S_{\ell}) < \beta \ \  \text{ if } S_{\ell} \text{ is assigned to an agent in } N_2 .
\end{equation}

We next bound the value of $v_a(S_{\ell})$ when it is assigned to an agent in $N_1$. We have 
\begin{equation}\label{eqn:sn1}
\begin{aligned}
v_a(S_1) & \le  1 \text{ for any reduction using } S_1 \text{ since the instance is $n$-normalized}\\
v_a(S_3) & <  3\beta/2 \text{ for any reduction using } S_3 \text{ since $S_2$ is not feasible} \\
v_a(S_4) & <  \beta + \beta/3 = 4\beta/3 \text{ for any reduction using } S_4 \text{ since $S_1$ and $S_3$ are not feasible}
\end{aligned}
\end{equation}

The only case left is reduction using $S_2$, for which we break the analysis into multiple cases. Let $S_{\ell_1}S_{\ell_2} \cdots$ for $\ell_i \in [4]$ be a series of reduction. Now, consider the transitions to $S_2$, i.e., $\cdots S_{\ell}[S_2]^{t}S_{\ell'}\cdots$, where $\ell,\ell' \neq 2$ and $t\ge 1$ denote the number of $S_2$'s between $S_{\ell}$ and $S_{\ell'}$. Let $S_2^{t'}$ denote the $t'$-th $S_2$ for $t'\in [t]$. There are three cases: 
\smallskip

\textbf{Case 1} $[S_1]^{s}[S_2]^{t}\cdots:$ Here, $S_1$ occurs exactly $s\ge 0$ times. This case can happen at most once since $S_1$'s are the highest priority, and once $S_1$ is not feasible, it remains infeasible. 
%\HA{$S_2$ is a valid reduction before any $S_4$ happens. So in Case 1, $v_a(S_2^{t'}) \leq 1$.}
\begin{lemma}[Case 1]\label{lem:s1}
$v_a(S_2^{t'}) < 3\beta/2, \forall t'\in [t]$. 
\end{lemma}

\begin{proof}
Note that $S_2^{t'} := \{n-t'+1,n+t'\}, \forall t'\in [t]$, where $n$ is the number of agents in the original instance. By the pigeonhole principle, a bundle in $a$'s MMS partition $P^a$ must contain two goods from $[n+1]$. This, together with the instance being $n$-normalized, implies that $v_{a}(S_2^1) \le 1$. For $t'\ge 2$, if $v_a(S_2^{t'}) > 1$, then the goods in $\{n-t'+2, \dots, n+t'\}$ must be in $t'-1$ different bundles in $P^a$, which implies that there must be a bundle in $P^a$ that contains at least three goods from $\{n-t'+2, \dots, n+t'\}$ by the pigeonhole principle. This further implies that $v_a(n+t') \le 1/3$ because the instance is $n$-normalized. 

Finally, since $S_1$ is not feasible when the algorithm performs $S_2$, we have $v_a(n-t'+1) < \beta, \forall t' \in [t]$ and then we have $v_a(S_2^{t'}) = v_a(n-t'+1) + v_a(n+t') < \beta + 1/3 < 3\beta/2, \forall t'\ge 2$ using $\beta>3/4$. 
\end{proof}

\textbf{Case 2} $\cdots S_4[S_2]^{t}\cdots$: This case cannot happen because $S_2$ was not feasible when $S_4$ was performed and the set of goods in $S_2$ doesn't change after an $S_4$. 
\medskip

\textbf{Case 3} $\cdots S_3[S_2]^{t}\cdots$: Let $s$ be the number of agents just before the instance is reduced using $S_3$, which implies that $S_3 = \{2s+1, 2s, 2s-1\}$ and $S_2 = \{s-1, s\}$. Let $x:= v_a(S_3)$, which implies that $v_a(2s-1) \ge x/3$. Since $S_2$ is not feasible when we used $S_3$, we have $v_a(\{s, s+1\}) < \beta$, which further implies that $x<3\beta/2$. Furthermore, we have $v_a(s+1) \ge v_a(2s-1) \ge x/3$ and $v_a(s) < \beta - x/3$. Next, we break the analysis into two subcases depending on whether there are more $S_3$ reductions later. 

\textbf{Case 3a:} If this is the last reduction with $S_3$, then we have $v_a(\{s-1, s\} < \beta + \beta - x/3 < 2\beta$. Since all later $S_2 = \{j, j'\}$'s will have a $j<s-1$ and $j'>s+1$, we have $v_a(\{j, j'\}) < \beta + \beta/2 = 3\beta/2$ for each of them. Note that this case can occur at most once. 

\textbf{Case 3b:} For the other case, if this is not the last $S_3$, then we must have $v_a(\{s-2, 2s-2\})<\beta$ otherwise, $S_2$ will always be feasible contradicting the fact that this is not the last $S_3$. This implies that $v_a(s-1) \le v_a(s-2) < \beta - v_a(2s-2) < \beta - x/3$.  Then, we have 
$$v_a(S_2^1 \cup S_3) = v_a(\{s-1, s\}) + v_a(S_3) < 2\beta - 2x/3 + x < 2\beta + x/3 < 5\beta/2.$$
Furthermore, for each of the remaining $t-1$ $S_2 = \{j, j'\}$'s, we have $j<s, j'>s$, implying $v_a(\{j, j'\} < \beta + \beta/2 = 3\beta/2$. 
The above analysis implies that 

\begin{corollary}
In Case 3, either $v_a(S_2^1 \cup S_3) < 5\beta/2$ or $v_a(S_2^1) < 2\beta$. For the remaining $t-1$ $S_2$'s, $v_a(S_2^{t'}) < 3\beta/2, \forall t'\ge 2$. 
\end{corollary}

We are now ready to prove Lemma~\ref{lem:n1}.

\begin{proof}(of Lemma~\ref{lem:n1})
Recall that we assumed for a contradiction that the bag filling phase stopped at iteration $k\le n'$ because $[m']\setminus [2n']$ is empty and an agent $a\in N_1$ has not received a bag. Lemma~\ref{lem:N2} implies that all agents in $N_2$ must have received a bag valued at least $3/4$ before this iteration. 

Since we prioritize agents in $N_1$ in both valid reductions and bag filling, we have $v_a(S) < \beta$ whenever $S$ is given to an agent in $N_2$. Further, \eqref{eqn:bk} and \eqref{eqn:hbk} imply that both $v_a(B_k)$  and $v_a(\hat{B})$ are strictly less than $3\beta/2$ at the beginning and when assigned to other agents in the bag filling phase. In valid reductions, except for Case 3 of $S_2$, \eqref{eqn:sn1} and Lemma~\ref{lem:s1} imply that $v_a(S_\ell) < 3\beta/2$. Case 3a of $S_2$ occurs at most once, which implies that for all $S_2$'s in this case except for one, say $S_2^*$, we have $v_a(S_2) < 3\beta/2$ and $v_a(S_2^*) < 2\beta$. Case 3b of $S_2$ implies that $v_a(S_3 \cup S_2^1) < 5\beta/2$ and for all other $S_2$'s we have $v_a(S_2) < 3\beta/2$. Therefore, we have

\begin{equation}\nonumber
\begin{aligned}
n = v_a(M) & = \sum_{S \text{ assigned to } i\in N_2} v_a(S) + \sum_{S \text{ assigned to } i\in N_1} v_a(S) + v_a(B_k) + \sum_{j=k+1}^{n'} v_a(B_j)\\
& < \beta\cdot (3\beta-2)n/\beta + 3\beta/2\cdot (2(1-\beta)n/\beta-(n'-k+2)) + 2\beta + v_a(B_k) + 3\beta/2\cdot (n'-k)\\
& = n - \beta + v_a(B_k), 
\end{aligned}
\end{equation}
which implies that $v_a(B_k) > \beta$, a contradiction. 
\end{proof}

