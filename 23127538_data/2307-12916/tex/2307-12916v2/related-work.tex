\subsection{Further Related Work}
Since the MMS notion and its variants have been intensively studied, we mainly focus on closely related work here. Computing the MMS value of an agent is NP-hard, but a PTAS exists \cite{woeginger1997polynomial}. For $n=2$, MMS allocations always exist \cite{bouveret2016characterizing}. For $n=3$, a series of work has improved the MMS approximation from $3/4$ \cite{procaccia2014fair} to $7/8$ \cite{amanatidis2017approximation} to $8/9$ \cite{gourves2019maximin}, and then to $11/12$~\cite{feige2022improved}.
For $n=4$, $(4/5)$-MMS allocations exist~\cite{ghodsi2018fair,babaioff2022fair}.

Babaioff \etal\ \cite{BabaioffNT21} considered $\ell$-out-of-$d$ MMS, in which the MMS value of an agent is the maximum value that can be guaranteed by partitioning goods into $d$ bundles and selecting the $\ell$ least-valuable ones. This was further studied by Segal-Halevi \cite{SegalHalevi2019TheMS, SegalHalevi2017CompetitiveEF}. Currently, the best result is the existence of $\ell$-out-of-$\lfloor (\ell+\frac{1}{2})n \rfloor$ MMS \cite{Hosseini2021OrdinalMS}. The MMS and its ordinal approximations have also been applied in the context of cake-cutting problems~\cite{Elkind2021GraphicalCC, Elkind2021KeepYD, Elkind2020MindTG, BogomolnaiaM22}.

Many works have analyzed randomly generated instances. Bouveret and Lema{\^\i}tre~\shortCite{bouveret2016characterizing}
showed that MMS allocations usually exist (for data generated randomly using uniform or Gaussian valuations).
MMS allocations exist with high probability when the valuation of each good is drawn independently and randomly from the uniform distribution on $[0, 1]$ \cite{amanatidis2017approximation} or for arbitrary distributions of sufficiently large variance \cite{kurokawa2016can}.

MMS can be analogously defined for fair division of chores where items provide negative value. Like the case of the goods, MMS allocations do not always exist for chores \cite{aziz2017algorithms}. Many papers studied approximate MMS for chores~\cite{aziz2017algorithms,barman2020approximation,huang2021algorithmic}, with the current best approximation ratio being $13/11$~\cite{huang2023reduction}. For three agents, $19/18$-MMS allocations exist~\cite{feige2022improved}. Also, ordinal MMS approximation for chores has been studied, and it is known that $1$-out-of-$\lfloor 3n/4 \rfloor$ MMS allocations exist~\cite{Hosseini2022OrdinalMS}. The chores case turns out to be easier than goods due to its close relation with the well-studied variants of bin-packing and job scheduling problems.

MMS has also been studied for non-additive valuations~\cite{MMS-XOS,ghodsi2018fair,li2021fair}. Generalizations have been studied where restrictions are imposed on the set of allowed allocations, like matroid constraints \cite{gourves2019maximin}, cardinality constraints \cite{biswas2018fair}, and graph connectivity constraints \cite{bei2022price,truszczynski2020maximin}. Stretegyproof versions of fair division have also been studied~\cite{barman2019fair,amanatidis2016truthful,amanatidis2017truthful,aziz2019Strategyproof}. MMS has also inspired other notions of fairness, like weighted MMS \cite{farhadi2019fair}, AnyPrice Share (APS) \cite{babaioff2021fair}, Groupwise MMS \cite{barman2018groupwise,chaudhury2021little}, and self-maximizing shares \cite{babaioff2022fair}.

%Very recently, Akrami \etal{}~\cite{MMS-XOS} proved the existence of randomized allocations which are $1/4$-MMS ex-ante and $1/8$-MMS ex-post when agents have XOS valuations.

\begin{comment}
Babaioff et al.~\cite{Babaioff2021BestofBothWorldsFA} studied fairness mechanisms to give all agents both ex-ante and ex-post guarantees. Namely, they give a deterministic polynomial time algorithm that computes a distribution over allocations that is ex-ante proportional and ex-post $1/2$-MMS.
\end{comment}
