\section{Introduction}

Fair allocation of resources (goods) is a fundamental problem in multiple disciples, including computer science, economics, and social choice theory, where the goal is to divide a set $M$ of $m$ goods among a set $N$ of $n$ agents in a \emph{fair} manner. This field has received significant attention since the seminal work of Steinhaus in the 1940s~\cite{steinhaus1948problem}.
When the goods are divisible, the two standard fairness notions are \emph{envy-freeness} and \emph{proportionality}, based on envy and share, respectively. In an envy-free allocation, no agent prefers another agent's allocation, and in a proportional allocation, each agent receives her proportionate share, i.e., at least a $1/n$ fraction of her value of all the goods. In the case of divisible goods, an envy-free and proportional allocation exists; see~\cite{Varian74,Foley67,AzizM16b}.

We study the discrete setting, in which each good can be given to exactly one agent. %Formally, the input consists of a set $N$ of $n$ agents, a set $M$ of $m$ indivisible goods and a valuation profile $\mathcal{V} = (v_1, \ldots, v_n)$ where 
Let $v_i: 2^M \rightarrow \mathbb{R}_{\geq 0}$ denote the valuation function of agent $i$. We assume that $v_i(.)$'s are additive, i.e., $v_i(S) = \sum_{g\in S}v_i(\{g\})$. 
%is agent $i$'s valuation function over the subsets of goods. 
Simple examples show that in the discrete case, neither envy-freeness nor proportionality can be guaranteed.\footnote{Consider an instance with two agents and one good with positive value to both agents.} This necessitates the refinement of these notions.

We consider the natural and most popular discrete analog of proportionality called \emph{maximin share} (MMS) introduced in~\cite{budish2011combinatorial}. It is also shown to be preferred by participating agents in real-life experiments~\cite{GatesGD20}. The MMS value of an agent is the maximum value she can guarantee if she divides the goods into $n$ bundles (one for each agent) and then receives a bundle with the minimum value. Formally, for a set $S$ of goods and an integer $d$, let $\Pi_d(S)$ denote the set of all partitions of $S$ into $d$ bundles. Then,
\begin{align*}
    \MMS_i^d(S) := \max_{(P_1, \ldots, P_d) \in \Pi_d(S)} \min_{j} v_i(P_j).
\end{align*}
The MMS value of agent $i$ is denoted by $\MMS_i := \MMS_i^n(M)$.
An allocation is said to be MMS if each agent receives at least their MMS value. However, MMS is an unfeasible share guarantee that cannot always be satisfied when there are more than two agents with additive valuations~\cite{procaccia2014fair, KurokawaPW18, FeigeST21}. Therefore, the MMS share guarantee needs to be relaxed, and the two natural ways are its multiplicative and ordinal approximations.

\paragraph{\bf \boldmath $\alpha$-MMS}
Since we need to lower the share threshold, a natural way is to consider $\alpha<1$ times the MMS value. Formally, an allocation $X = (X_1, \ldots, X_n)$ is $\alpha$-MMS if for each agent $i$, $v_i(X_i) \geq \alpha \cdot \MMS_i$. Earlier works showed the existence of $2/3$-MMS allocations using several different approaches~\cite{procaccia2014fair,amanatidis2017approximation,barman2020approximation,garg2019approximating,KurokawaPW18}.
Later, in a groundbreaking work~\cite{ghodsi2018fair}, the existence of $3/4$-MMS allocations was obtained through more sophisticated techniques and involved analysis. This factor was slightly improved to $3/4 + 1/(12n)$ in~\cite{garg2021improved}, then more recently to $\frac{3}{4} + \min(\frac{1}{36}, \frac{3}{16n-4})$~\cite{akrami2023simplification}, and finally to $3/4+3/3836$~\cite{Akrami2023BreakingT}. On the other hand, $\alpha$-MMS allocations need not exist for $\alpha> 1-1/n^4$~\cite{FeigeST21}.

\paragraph{\bf \boldmath $1$-out-of-$d$ MMS}
Another natural way of relaxing MMS is to consider the share value of $\MMS_i^d(M)$ for $d>n$ for each agent $i$, which is the maximum value that $i$ can guarantee if she divides the goods into $d$ bundles and then takes a bundle with the minimum value. This notion was introduced together with the MMS notion in~\cite{budish2011combinatorial}, which also shows the existence of $1$-out-of-$(n+1)$ MMS after \emph{adding excess goods}. Unlike $\alpha$-MMS, this notion is robust to small perturbations in the values of goods because it only depends on the bundles' ordinal ranking and is not affected by small perturbations as long as the ordinal ranking of the bundles does not change.%
\footnote{The $\alpha$-MMS is very sensitive to agents' precise cardinal valuations: Consider the example mentioned in \cite{Hosseini2021OrdinalMS}. Assume $n=3$ and there are four goods $g_1$, $g_2$, $g_3$ and $g_4$ with values $30$, $39$, $40$ and $41$ respectively for agent $1$. Assume the goal is to guarantee the $3/4$-MMS value of each agent. We have $\MMS_1 = 40$, and therefore any non-empty bundle satisfies $3/4$-MMS for agent $1$. However, if the value of $g_3$ gets slightly perturbed and becomes $40+\epsilon$ for any $\epsilon>0$, then $\MMS_1 > 40$ and then $3/4 \cdot \MMS_1 > 30$ and the bundle $\{g_1\}$ does not satisfy agent $1$. Thus,  the acceptability of a  bundle  (in this example, $\{g_1\}$) might be affected by an arbitrarily small perturbation in the value of an irrelevant good (i.e., $g_3$).

Observe that in this example, whether the value of $g_3$ is $40$ or $40+\epsilon$ for any $\epsilon \in \mathbb{R}$, $\{g_1\}$ is an acceptable $1$-out-of-$4$ MMS bundle for agent $1$.}

In the standard setting (i.e., without excess goods), the first non-trivial ordinal approximation was the existence of $1$-out-of-$(2n-2)$ MMS allocations~\cite{AignerHorev2019EnvyfreeMI}, which was later improved to $1$-out-of-$\lceil 3n/2 \rceil$~\cite{hosseini2021guaranteeing}, and then to the current state-of-the-art $1$-out-of-$\lfloor 3n/2 \rfloor$~\cite{Hosseini2021OrdinalMS}. On the other hand, the (non-)existence of $1$-out-of-$(n+1)$ MMS allocations is open to date.

Our first main result in the following theorem shows that $1$-out-of-$4\lceil n/3\rceil$ MMS allocations always exist, thereby improving the state-of-the-art of $1$-out-of-$d$ MMS.

\begin{theorem}\label{thm:1}
$1$-out-of-$4\lceil n/3 \rceil$ MMS allocations always exist. 
\end{theorem}

Our proof is constructive; we give a simple algorithm to achieve $1$-out-of-$4\lceil n/3 \rceil$ MMS allocations. Our algorithm just utilizes \emph{bag-filling}, where we initialize $n$ bags $\{B_1, \dots, B_n\}$ with a particular set of two goods in each. %First we initialize $n$ bags by putting $2n$ the high value goods $k$ and $n-k+1$ in bag $B_k$ for all $k \in [n]$. 
Then, we iterate over the bags, and in each round $j$, we keep adding more goods to the bag $B_j$ until its value is at least $\MMS^d_i$ for some agent $i$, who has not received a bag yet, and $d=4\ceil{\frac{n}{3}}$. Then, we allocate $B_j$ to any such agent $i$ and proceed.

We note that our algorithm is very similar to the bag-filling phase of algorithms in~\cite{garg2021improved,akrami2023simplification} which obtain $\frac{3}{4}+O(\frac1n)$-MMS allocations. However, these algorithms use another critical phase called \emph{valid reductions} before bag-filling. Valid reductions allocate a set of goods to an agent $i$ such that while $i$ gets her desired share ($\alpha\cdot \MMS_i$), the MMS values of the other agents do not drop. The bag-filling phase in these algorithms is run only when no more valid reductions are feasible. It provides strong bounds on the values of the remaining goods utilized in the analysis of the bag-filling phase. However, it is unclear how to use valid reduction in an algorithm to find $1$-out-of-$d$ MMS allocations. This is because $d$ is not the same as $n$, and our goal is to guarantee exact $1$-out-of-$d$ MMS allocations and not an $\alpha$ approximation of it. Consequently, our analysis is totally different from the ones in ~\cite{garg2021improved,akrami2023simplification}, and much more involved and challenging. We give an overview of our analysis in~\cref{sec:tec}.
%\HA{(I don't know how to say anything about the analysis without formally defining things and how to keep it short.)}
%
We also give a tight example showing that Theorem~\ref{thm:1} has the best factor possible with this approach (Section~\ref{sec:tighteg}). 

Another way to interpret $1$-out-of-$d$ MMS allocations is giving $n/d$ fraction of agents their MMS value and nothing to the remaining agents. We prove this equivalence in Section~\ref{sec:prelim}.
%
Both ordinal and multiplicative approximations focus on extremes. In $1$-out-of-$d$ MMS, some agents get nothing, and others are guaranteed their \emph{full} MMS value.  In $\alpha$-MMS, each agent receives (the same factor) $\alpha<1$ fraction of their MMS value.  As a middle ground between these two extreme notions, we introduce a general framework of \emph{approximate MMS with agent priority ranking}, where each agent has a unique \emph{priority rank} from the set $\{1, \ldots, n\}$. We are also given a list of \emph{thresholds} $T \defeq (\tau_1, \tau_2, \ldots, \tau_n)$ where $1 \ge \tau_1 \ge \ldots \ge \tau_n \ge 0$. An allocation is said to be $T$-MMS if the agent with priority rank $i$ gets a bundle of value at least $\tau_i$ times her MMS value. This framework captures both ordinal and multiplicative approximations as special cases. Namely, an $\alpha$-MMS allocation corresponds to the case where $\tau_i = \alpha$ for all $i \in [n]$, and $1$-out-of-$d$ allocations correspond to the case where the first $n/d$ thresholds in $T$ are 1 and the rest are 0. Furthermore, the $(\alpha, \beta)$-framework introduced in~\cite{hosseini2021guaranteeing}, where $\alpha$ fraction of agents receive $\beta$-MMS, is another special case where the first $n\alpha$ thresholds in $T$ are $\beta$ and the rest are 0.

%In \cref{sec:algo1}, 
Our second main result in the following theorem shows that a $T$-MMS allocation always exists where $\tau_i = \max(\frac{3}{4} + \frac{1}{12n}, \frac{2n}{2n+i-1})$ for all $i \in [n]$. 

\begin{theorem}\label{thm:2}
$T=(\tau_1, \ldots, \tau_n)$-MMS allocations exist when $\tau_i \defeq \max\left(\frac{2n}{2n+i-1}, \frac{3}{4} + \frac{1}{12n}\right)$ for all $i\in N$. 
\end{theorem}

%\jnote{Ekavya: please write technical overview of the proof of this theorem.} 

We prove Theorem~\ref{thm:2} constructively by giving an allocation algorithm. Our algorithm is almost identical to that of \cite{garg2021improved}. The main difference is that when multiple agents are eligible to receive a bundle, we give the bundle to the agent with the smallest priority rank.
%
However, despite a similar algorithm, our analysis is totally different from theirs. Our algorithm, like theirs, performs a few \emph{reduction} operations (phase 1) until the instance becomes \emph{irreducible}, and then does bag filling (phase 2). They prove that reduction operations in their algorithm are \emph{valid}, so they only need to prove their claim for irreducible instances. In our setting, reduction operations may not be valid, so we need to prove our claim for non-irreducible instances too. To do this, we bound the values of bundles from both phases together.

Another approach to get fair allocations is to use randomization.
For example, when there is a single good, giving the good to an agent selected uniformly randomly is fair. Hence, one can also consider the fairness of probability distributions of allocations.
However, randomness by itself is unsatisfactory: giving all the goods to a random agent is fair because each agent has equal opportunity, but is unfair after the randomness is realized due to the large disparity in this allocation.
To fix this, one can aim for a \emph{best-of-both-worlds} approach
(see, e.g.,~\cite{AleksandrovAGW15,freeman2020best,babaioff2022best,aziz2023best,cycle-breaking})
%\jnote{please add more citations on best-of-both worlds, e.g., Freeman et al.},
i.e., find a distribution of allocations where each allocation in the support of the distribution is fair (ex-post fairness), and the entire distribution is fair in a randomized sense (ex-ante fairness).
For example, Babaioff \etal{} \shortCite{babaioff2022best} gave an algorithm whose output is ex-post $1/2$-MMS and ex-ante proportional, i.e., it outputs a distribution over $1/2$-MMS allocations, such that each agent's expected value of her bundle is at least $v_i(M)/n$.

By ordering the agents randomly in \cref{thm:2}, we can get allocations that are $(\frac{3}{4} + \frac{1}{12n})$-MMS ex-post and $(0.8253 + \frac{1}{36n})$-MMS ex-ante. On the other hand, we show that our algorithm cannot give better than ex-ante $(0.8631+\frac{1}{2n})$-MMS, regardless of how we pick the thresholds (\cref{sec:hard1}).
The analysis of many MMS-approximation algorithms has a natural property called \emph{obliviousness}. In \cref{sec:hard2}, we show that obtaining better than ex-ante $(0.8578+\frac{1}{3n})$-MMS with our algorithm (for some choice of thresholds) requires non-oblivious proof techniques.
