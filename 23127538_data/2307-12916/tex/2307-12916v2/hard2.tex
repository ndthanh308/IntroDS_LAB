\section{Limitations of Oblivious Analysis}
\label{sec:hard2}

We start by showing an attempt to improve the upper bounds of \cref{sec:hard1}.
Although our attempt does not succeed, it reveals a shortcoming of existing
techniques (including ours) to analyze bag-filling algorithms.
We formally characterize this using a concept called \emph{obliviousness},
and study the limitations of oblivious techniques.

\subsection{Attempt to get a Hard Example}
\label{sec:hard2:ex}

Consider a fair division instance $\Ical \defeq ([n], [m], v)$ with $n \defeq 6$ agents,
where each agent $i$ has priority rank $i$. We will analyze $\RBF$ from the perspective of agent 4.
\Cref{sec:hard1} says that $\tau_4$ should be at most $0.9$ for $\RBF$ to always succeed.
We will try to improve this to $\tau_4 \le 5/6 = 0.8\overline{3}$.
According to agent 4, say the first 3 goods have value $5/6$, the next 9 goods of value $1/3$, and
the remaining $3t$ goods have value $\eps \defeq 1/6t$, for some $t \in \mathbb{Z}_{\ge 3}$.

$\Ical$ is ordered and normalized according to agent 4:
3 of her MMS bundles have 3 goods of value $1/3$ each, and each of the remaining bundles
has one good of value $5/6$ and $t$ goods of value $1/6t$.

Suppose $\tau_4 > 5/6 + 2\eps$. Then agent 4 does not want any bundle during reduction, since
$v_1(1) \le v_1(\{1, 2n+1\}) = 5/6 + \eps < \tau_4$, $v_1(\{n, n+1\}) = 2/3 < \tau_4$,
and $v_1(\{2n-1, 2n+1, 2n+1\}) = 2/3 + \eps < \tau_4$.
Suppose the other agents also do not want any of the above bundles.
Hence, no reduction events happen during $\RBF$, and $\bagFill$ starts.

According to agent 4, the first 3 bags have value $5/6 + 1/3 = 7/6$ each.
Suppose the first 3 agents also want these bags.
Since they have a smaller priority rank, these bags go to the first 3 agents.
Agent 4 does not like any of the remaining bags, since they have value $2/3$.
Suppose agents 5 and 6 also do not like these bags. Hence, we start adding goods to these bags.
Since all goods in $[m] \setminus [2n]$ have value $\eps$ to agent 4, agent 4 would want
a bag only if at least $t+3$ goods are added to it. Suppose agents 5 and 6 like a bag
iff it gets at least $t+2$ goods from $[m] \setminus [2n]$.
Since there are only $3t$ goods of value $\eps$, agent 4 does not get a bag of value $\ge \tau_4$.

At first glance, this seems to prove that $\RBF$ cannot give agent 4 more than $5/6$-MMS.
But there is a catch: we have not mentioned what the valuations and thresholds of the other agents are.
Are there valuations and thresholds for which $\RBF$ behaves as above?

When all agents have the same valuation function as agent 4, and agents 5 and 6 have thresholds
in the range $(5/6+\eps, 5/6+2\eps]$, then $\RBF$ indeed behaves as described above.
However, this threshold range is unreasonable since \cref{sec:hard1} says that
$\tau_6 \le 9/11 < 5/6$ for $\RBF$ to succeed.
Are there (ordered and normalized) valuations and reasonable thresholds for the agents such that
$\RBF$ behaves as described above? We could not resolve this question.

\subsection{Obliviousness}
\label{sec:hard2:obl}

Let us think of fair division algorithms as interactive protocols,
where the algorithm repeatedly makes value queries to agents, i.e.,
it finds out $v_i(S)$ for some agent $i$ and some set $S$ of goods,
instead of just querying the value of each good for each agent upfront.
$\RBF$ can be interpreted in this way: during the reduction phase,
it queries $v_i(S)$ for $S \in \{\{1\}, \{n, n+1\}, \{2n-1, 2n, 2n+1\}, \{1, 2n+1\}\}$,
and during the bag-filling phase, it repeatedly queries the value of some bag.

Let $\Acal$ be an algorithm for fair division, let $i^*$ be an agent,
and let $T \defeq (\tau_1, \ldots, \tau_n)$ be a list of thresholds.
Generally, when we analyze $\Acal$, we ask the following question:
\textsl{Does $\Acal$'s output give each agent $i$ at least $\tau_i$ times her MMS,
given that all agents answer value queries truthfully?}

An \emph{oblivious analysis} of $\Acal$, on the other hand, asks the following question:
\textsl{For each $i \in [n]$, does $\Acal$'s output give agent $i$ at least $\tau_i$ times her MMS
if agent $i$ answers value queries truthfully and
all other agents are free to answer value queries non-truthfully?}
Note that in an oblivious analysis, the other agents' responses are not required to be
consistent with any (additive, ordered, normalized) valuation function.
Also, we do not assume that the other agents are rational, i.e., they may answer value
queries arbitrarily, even if it hurts them.

In the oblivious setting, we wish to show that an agent $i$ can secure a certain
fraction of her MMS for any possible responses of the other agents.
This is a worst-case analysis, so imagine that the other $n-1$ agents are
controlled by an adversary (who knows $i$'s valuation function)
whose sole motive is to reduce the value of $i$'s bundle.
Furthermore, the oblivious setting is more demanding than a non-oblivious one, i.e.,
if $i$ is guaranteed $\alpha$ times her MMS in the oblivious setting,
then she also gets $\alpha$ times her MMS in the non-oblivious setting.

The analysis of bag-filling algorithms in the literature
\cite{akrami2023simplification,garg2021improved,Akrami2023BreakingT,garg2019approximating},
and our analysis in \cref{sec:algo1}, work in the oblivious setting.
In fact, it is unclear how to \emph{not} do an oblivious analysis.
In this sense, obliviousness is a hard-to-bypass barrier towards the design
of approximate MMS algorithms.

Furthermore, we can get tighter upper bounds on thresholds in the oblivious setting.
E.g., \cref{sec:hard2:ex} shows that for 6 agents,
we need $\tau_4 \le 5/6$ for $\RBF$ to succeed in the oblivious setting.
We now generalize this example to an arbitrary number of agents.

\begin{theorem}
\label{thm:hard2}
Let $T \defeq (\tau_1, \ldots, \tau_n)$ be a list of thresholds, where $n \ge 2$.
If $\RBF(\cdot, T)$ always outputs a $T$-MMS allocation in the oblivious setting,
then for each $i \in [n]$, we have
\[ \tau_i \le \max\left(\frac{5}{6}, 1-\frac{i-1}{3n}\right). \]
\end{theorem}
\begin{proof}
Let $k_1, k_2 \in [n]$ such that $k_1 + k_2 < i$ and $2k_1 + k_2 \le n$.
Let $\alpha \defeq 1 - k_1/3(n-k_2)$. Let $t \in \mathbb{Z}_{\ge 3}$
and $\eps \defeq 1/3t(n-k_2)$. Then $\alpha \in [5/6, 1-\eps)$.
Pick an arbitrary agent $i$. Assume \wLoG{} that $i \ge 2$ (otherwise theorem is trivially true).
According to agent $i$, let there be $k_1+k_2$ goods of value $\alpha$,
$2n-k_1-2k_2$ goods of value $1/3$, and $(n-k_1-k_2)^2t$ goods of value $\eps$.
Let $\tau_i > \alpha+2\eps$.

The instance is normalized for agent $i$, since the MMS partition is as follows:
\begin{enumerate}
\item $k_1$ bundles: one good of value $\alpha$ and $k_1t$ goods of value $\eps$.
\item $k_2$ bundles: one good of value $\alpha$ and one good of value $1-\alpha$.
\item $k_1$ bundles: 3 goods of value $1/3$.
\item $n-2k_1-k_2$ bundles: 2 goods of value $1/3$ and $1/3\eps$ goods of value $\eps$.
\end{enumerate}

Suppose no reduction events happen during $\RBF$.
This is possible because $v_i(\{n, n+1\}) = 2/3 < \tau_i$,
$v_i(\{2n-1, 2n, 2n+1\}) \le 2/3 + \eps < \tau_i$, and $v_i(\{1, 2n+1\}) = \alpha+\eps < \tau_i$.

Suppose $\RBF$ gives away $k_1+k_2$ bags having a good of value $\alpha$
and a good of value $1/3$ or $1-\alpha$ to the first $k_1 + k_2$ agents.
All the remaining bags initially have 2 goods of value $1/3$.
Suppose each of them receives at most $(n-k_1-k_2)t + 2$ goods of value $\eps$.
Then each of those bags has value at most $\alpha+2\eps$.
Hence, agent $i$ will not get a bag of value $\ge \tau_i$.

Hence, if $\tau_i > \alpha$, then for some large enough $t$,
$\RBF$ cannot give a bag of value $\ge \tau_i$ to agent $i$.
If $i \le n/2+1$, then set $k_1 = i-1$ and $k_2 = 0$ to get $\tau_i \le 1 - (i-1)/3n$.
Otherwise, set $k_1 = \floor{n/2}$ and $k_2 = n - 2k_1$ to get $\tau_i \le 5/6$. Hence,
\[ \tau_i \le \max\left(\frac{5}{6}, 1-\frac{i-1}{3n}\right).
\qedhere \]
\end{proof}

Our approach in \cref{sec:algo1:bobw} to get best-of-both-worlds fairness is by
running $\RBF$ with random priority ranks (as in \cref{thm:cyclic-perm}).
If $\RBF$ is $T$-MMS, this gives us ex-post $\tau_n$-MMS
and ex-ante $\frac{1}{n}(\sum_{i=1}^n \tau_i)$-MMS.
In the oblivious setting, by \cref{thm:hard1,thm:hard2},
the best ex-ante MMS guarantee we can get is
\[ \frac{1}{n}\sum_{i=1}^n \min\left(\frac{3n}{3n+i-2},
    \max\left(\frac{5}{6}, 1 - \frac{i-1}{3n}\right)\right). \]
We now show an upper-bound on this quantity.

\begin{restatable}{lemma}{rthmHardIIExa}
\label{thm:hard2-avg}
For all $n \ge 1$,
\[ \frac{1}{n}\sum_{i=1}^n \min\left(\frac{3n}{3n+i-2},
    \max\left(\frac{5}{6}, 1 - \frac{i-1}{3n}\right)\right)
    \le \frac{13}{24} + 3\ln\left(\frac{10}{9}\right) + \frac{1}{3n}. \]
\end{restatable}
\begin{proof}
(Proof deferred to \cref{sec:prio-extra:calculations}).
\end{proof}

\begin{theorem}
\label{thm:hard2-exa}
If we fix the number of agents to $n$, the ex-ante MMS guarantee obtainable from $\RBF$
in the oblivious setting by picking priority ranks randomly (as in \cref{thm:cyclic-perm})
is not better than $\beta$-MMS, where
$\beta \defeq \frac{13}{24} + 3\ln\left(\frac{10}{9}\right) + \frac{1}{3n}
\approx 0.8578 + \frac{1}{3n}$,
and the ex-post MMS guarantee is not better than $\gamma$-MMS,
where $\gamma \defeq 3n/(4n-2) = \frac{3}{4} + \frac{3}{8n-4}$.
\end{theorem}
\begin{proof}
Follows from \cref{thm:hard1,thm:hard2,thm:cyclic-perm,thm:hard1-avg}.
\end{proof}
