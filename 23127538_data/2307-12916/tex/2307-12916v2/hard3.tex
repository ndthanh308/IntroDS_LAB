\subsection{Tight Example}\label{sec:tighteg}

We now show that \cref{algo}'s guarantees are almost tight, i.e.,
it cannot give us better than 1-out-of-$\floor{(4n+1)/3}$ MMS.
Note that $\ceil{4n/3} = \floor{(4n+2)/3}$ for all $n \in \mathbb{N}$.

Specifically, we show a fair division instance where \cref{algo}'s output is not
1-out-of-$\floor{(4n-2)/3}$ MMS, even when agents have identical valuations.
This instance is similar to the tight example in \cite{akrami2023simplification},
where they show that a natural class of algorithms (containing \cref{algo})
cannot guarantee a better multiplicative approximation than $(\frac{3}{4} + \frac{3}{8n-4})$-MMS.

\begin{example}
\label{ex:1ood-hard}
Consider a fair division instance with $n$ agents, where $n \ge 2$.
Let $d \defeq \floor{(4n-2)/3} = n + \floor{(n-2)/3}$.
There are $m \defeq 2n+1+3(d-n) \le 3n-1$ goods.
All agents have the same valuation function $u$, where
\[ u(j) \defeq \begin{cases}
\displaystyle \frac{2}{3} - \frac{\ceil{j/2}}{3n} & \textrm{ if } 1 \le j \le 2n
\\ 1/3 & \textrm{ if } 2n < j \le m
\end{cases}. \]
\end{example}

\Cref{fig:1ood-hard} shows that when we run \cref{algo} on \cref{ex:1ood-hard} with $n=5$,
it does not output a 1-out-of-$d$ MMS allocation.
We formally prove this for all $n$:
in \cref{thm:1ood-hard:ord-norm}, we show that \cref{ex:1ood-hard} is ordered and normalized,
and in \cref{thm:1ood-hard:output}, we show that \cref{algo}'s output on \cref{ex:1ood-hard}
gives someone a bundle of value less than 1.
This proves that \cref{algo} does not output a 1-out-of-$d$ MMS allocation on \cref{ex:1ood-hard}.

% Figure environment removed

\begin{lemma}\label{thm:1ood-hard:ord-norm}
\Cref{ex:1ood-hard} is ordered and $d$-normalized.    
\end{lemma}
\begin{proof}
$u$ is ordered since $u(2n) = u(2n+1) = 1/3$.

Define $M \defeq (M_1, \ldots, M_d)$ as
\[ M_i \defeq \begin{cases}
\{i, 2n-1-i\} & \textrm{ if } 1 \le i \le n-1
\\ \{i+n-1, 2d+n-i, 2d-n+i+1\} & \textrm{ if } n \le i \le d
\end{cases}. \]
To show that $u$ is normalized, we show that $u(M_i) = 1$ for all $i \in [d]$.
For $i \in [n-1]$, we have
\begin{align*}
u(M_i) &= u(i) + u(2n-1-i) = \frac{4}{3} - \frac{\ceil{i/2} + \ceil{(2n-1-i)/2}}{3n}
\\ &= 1 - \frac{\ceil{i/2} + \ceil{-(i+1)/2}}{3n} = 1.
\end{align*}
For $i \in [d] \setminus [n-1]$ and $g \in M_i$, we have $g \ge 2n-1$.
Since $u(j) = 1/3$ for all $j \ge 2n-1$, we get that $u(M_i) = 1$ for $i \in [d] \setminus [n-1]$.
\end{proof}


\begin{lemma}\label{thm:1ood-hard:output}
Let $A \defeq (A_1, \ldots, A_n)$ be \cref{algo}'s output on \cref{ex:1ood-hard}.
Then $u(A_n) = 1 - 1/(3n)$.    
\end{lemma}
\begin{proof}
For $i \in [n]$, let $B_i \defeq \{i, 2n+1-i\}$ be the initial bag. Then
\begin{align*}
u(B_i) &= u(i) + u(2n+1-i) = \frac{4}{3} - \frac{\ceil{i/2} + \ceil{(2n+1-i)/2}}{3n}
\\ &= \frac{4}{3} - \frac{(n+1) + \ceil{i/2} + \ceil{-(i+1)/2}}{3n}
= 1 - \frac{1}{3n}.
\end{align*}
Since $m-2n \le n-1$ goods are left after bag initialization, and each good has value $1/3$,
the first $m-2n$ bags get 1 good each during bag filling,
and the remaining bags do not get any goods.
\end{proof}