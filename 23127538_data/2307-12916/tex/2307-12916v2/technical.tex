\subsection{Technical Overview}\label{sec:tec}
For $\alpha$-MMS problem, the algorithms for $\alpha\ge 3/4$~\cite{ghodsi2018fair,garg2021improved,akrami2023simplification} utilize the two-phase approach: \emph{valid reductions} and \emph{bag filling}. In a valid reduction, the instance is reduced by removing an agent $a$ and a subset of goods $S$ such that $v_a(S)\ge \alpha$, and the MMS values of the remaining agents do not decrease.
%(see \cref{sec:algo1} for more details).
The valid reduction phase is crucial for the bag filling to work in the analysis of these algorithms. However, it is not clear how to define valid reductions in the case of $1$-out-of-$d$ MMS because $d$ is not the same as the number of agents $n$. Therefore, we only use bag filling in our algorithm, which makes its analysis quite involved and entirely different than from $\alpha$-MMS algorithms.

\begin{algorithm}[htb]
    \caption{$1$-out-of-$4\ceil{n/3}$ MMS}
    \label{algo}
    \textbf{Input:} Ordered $4\ceil{n/3}$-normalized instance $\mathcal{I} = (N, [m], \mathcal{V})$.\\
    \textbf{Output:} Allocation $\hat{B} = (\hat{B}_1, \ldots, \hat{B}_n)$.
    \begin{algorithmic}[1]
    \For{$k \in [n]$}   \Comment{Initialization}
        \State $B_k = \{k, 2n-k+1\}$
    \EndFor
    \State $j \leftarrow 2n+1$
    \For{$k \in [n]$} \Comment{Bag-filling}
        \While{$\nexists i \in N$ s.t. $v_i(B_k) \geq 1$}
            \State $B_k \leftarrow B_k \cup \{j\}$
            \State $j \leftarrow j+1$
            \If{$j>m$}
                \State Terminate
            \EndIf
        \EndWhile
        \State Let $i \in N$ be s.t. $v_i(B_k) \geq 1$
        \State $\hat{B}_i \leftarrow B_k$
        \State $N \leftarrow N \setminus \{i\}$
    \EndFor
    %\emph{Assign the remaining goods to some agent:} \\
    \State $\hat{B}_n \leftarrow \hat{B}_n \cup (M \setminus [j])$
    \State \Return $\hat{B}$
    \end{algorithmic}
\end{algorithm}


The algorithm is described in Algorithm~\ref{algo}. Given an ordered $d$-normalized instance, we initialize $n$ bags (one for each agent) with the first $2n$ (highest valued) goods as follows.
\begin{equation}
    \label{eq:B_i}
    B_k := \{k , 2n-k+1\} \text{ for } k\in [n].
\end{equation}

See Figure \ref{fig:bags} for a better intuition. Then, we do bag-filling. That is, at each round $j$, we keep adding goods in decreasing values to the bag $B_j$ until some agent with no assigned bag values it at least $1$ (recall that $1$-out-of-$d$ MMS value of each agent is 1 in a $d$-normalized instance). Then, we allocate it to an arbitrary such agent.
%Algorithm \ref{algo} shows the pseudocode of our Algorithm.
We note that in contrast to \cite{garg2021improved,akrami2023simplification}, in the bag-filling phase we do not add arbitrary goods to the bags but we add the goods in the decreasing order of their values.

To prove that the output of Algorithm~\ref{algo} is $1$-out-of-$d$ MMS, it is sufficient to prove that we never run out of goods in any round or, equivalently, each agent receives a bag in some round. Towards contradiction, assume that agent $i$ does not receive a bag and the algorithm terminates. It can be easily argued that agent $i$'s value for at least one of the initial bags $\{B_1, \ldots, B_n\}$ must be strictly less than $1$. Let $\ell^*$ be the smallest such that $v_i(B_{\ell^*})<1$. We consider two cases based on the value of $v_i(2n-\ell^*)$. In Section \ref{positive}, we reach a contradiction assuming $v_i(2n-\ell^*) \geq 1/3$ and in Section \ref{negative}, we reach a contradiction assuming $v_i(2n-\ell^*)<1/3$.

Let $\hat{B}_j$ denote the $j$-th bag at the end of the algorithm. The overall idea is to categorize the bags into different groups and prove an upper bound on the value of each bag $(\hat{B}_j)$ for agent $i$ depending on which group it belongs to. Since $v_i(M)=n$ due to the instance being $d$-normalized, we get upper and lower bounds on the size of the groups. For example, if we know that for all bags $\hat{B}_j$ in a certain group $v_i(\hat{B}_j)<1$, we get the trivial upper bound of $n-1$ on the size of this group since $n = v_i(M) = \sum_{j \in [n]}v_i(\hat{B}_j)$.

Unfortunately, upper bounding the value of the bags is not enough to reach a contradiction in all cases. However, for these cases, we have upper and lower bounds on the size of each group, and in general, we show several additional properties to make it work.
%In these cases as well as before, we reach contradiction with the fact that $v_i(M)=n$.
For example, we obtain nontrivial upper bounds on the values of certain subsets of goods using the fact that all bundles in a $d$-MMS partition of agent $i$ have value $1$ (see Lemmas \ref{expowerful} and \ref{difficult-bound}).
