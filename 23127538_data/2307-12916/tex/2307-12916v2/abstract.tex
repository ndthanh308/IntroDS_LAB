\begin{abstract}
We consider fair division of a set of indivisible goods among $n$ agents with additive valuations using the fairness notion of maximin share (MMS). MMS is the most popular share-based notion, in which an agent finds an allocation fair to her if she receives goods worth at least her ($1$-out-of-$n$) MMS value. An allocation is called MMS if all agents receive their MMS values. However, since MMS allocations do not always exist~\texorpdfstring{\cite{KurokawaPW18}}{[Kurokawa et al., JACM'18]}, the focus shifted to investigating its ordinal and multiplicative approximations.

In the ordinal approximation, the goal is to show the existence of $1$-out-of-$d$ MMS allocations (for the smallest possible $d>n$). A series of works led to the state-of-the-art factor of $d=\lfloor3n/2\rfloor$~%
\texorpdfstring{\cite{Hosseini2021OrdinalMS}}{[Hosseini et al., J.Artif.Intell.Res.'21]}.
We show that $1$-out-of-$4\lceil n/3\rceil$ MMS allocations always exist, thereby improving the state-of-the-art of ordinal approximation. 
%\jg{We also give a tight example showing that this is the best factor possible with our approach.} 

In the multiplicative approximation, the goal is to show the existence of $\alpha$-MMS allocations (for the largest possible $\alpha < 1$), which guarantees each agent at least $\alpha$ times her MMS value. A series of works in the last decade led to the state-of-the-art factor of $\alpha = \frac{3}{4} + \frac{3}{3836}$~%
\texorpdfstring{\cite{Akrami2023BreakingT}}{[Akrami and Garg, SODA'24]}.

We introduce a general framework of \emph{approximate MMS with agent priority ranking}. We order the agents, and agents earlier in the order are considered more \emph{important}.  An allocation is said to be $T$-MMS, for a non-increasing sequence $T \defeq (\tau_1, \ldots, \tau_n)$ of numbers, if the agent at rank $i$ in the order gets a bundle of value at least $\tau_i$ times her MMS value. This framework captures both ordinal approximation and multiplicative approximation as special cases. We show the existence of $T$-MMS allocations where $\tau_i \ge \max(\frac{3}{4} + \frac{1}{12n}, \frac{2n}{2n+i-1})$ for all $i$. Furthermore, by ordering the agents randomly, we can get allocations that are $(\frac{3}{4} + \frac{1}{12n})$-MMS ex-post and $(0.8253 + \frac{1}{36n})$-MMS ex-ante. We also investigate the limitations of our algorithm and show that it does not give better than $(0.8631 + \frac{1}{2n})$-MMS ex-ante.
\end{abstract}
