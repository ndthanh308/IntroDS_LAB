\documentclass[letterpaper,11pt]{article}
\usepackage[margin=1in]{geometry}
\usepackage{amsmath,amssymb,amsthm}
\usepackage[table]{xcolor}
\usepackage{mathtools}
\usepackage{array}
\usepackage{enumitem}
\usepackage{xcolor}
\usepackage{comment}
\usepackage[hidelinks,bookmarksnumbered=true]{hyperref}
\usepackage{algorithm,algpseudocode}
\usepackage[capitalize,sort]{cleveref}
\usepackage{tikz}
\usepackage{thm-restate}
\usepackage{multirow}
\usepackage[symbol]{footmisc}
\renewcommand{\thefootnote}{\fnsymbol{footnote}}

\newtheorem{lemma}{Lemma}
\newtheorem{theorem}{Theorem}
\newtheorem{observation}{Observation}
\newtheorem{corollary}{Corollary}
\newtheorem{definition}{Definition}
\newtheorem{claim}{Claim}
\newtheorem*{claim*}{Claim}
\newtheorem{example}{Example}
\newtheorem{proposition}{Proposition}

\renewcommand{\t}[1]{\texttt{#1}}
\renewcommand{\b}[1]{{\color{blue} #1}}
\newcommand{\ul}{\underline}
\newcommand{\nf}{\nicefrac}
\newcommand{\notename}[2]{{\textcolor{red}{\footnotesize{\bf (#1:} {#2}{\bf ) }}}}
\newcommand{\noteswarning}{{\begin{center} {\Large WARNING: NOTES ON}\end{center}}}
\newcommand{\jnote}[1]{{\notename{JG}{#1}}}
\newcommand{\ST}[1]{{\notename{ST}{#1}}}
\newcommand{\HA}[1]{\textcolor{blue}{\textbf{HA:} #1}}
\newcommand{\todo}[1]{{\notename{TODO}{#1}}}
\newcommand{\jg}[1]{\textcolor{blue}{#1}}

\newcommand{\MMS}{\text{MMS}}
\newcommand{\I}{\mathcal{I}}
\newcommand{\B}{\mathcal{B}}
\newcommand{\MP}{\mathcal{P}}
\newcommand{\FRAC}{\tfrac{3}{4}+\gamma}
\DeclareMathOperator*{\argmax}{\arg\!\max}
\renewcommand{\t}[1]{\texttt{#1}}
\renewcommand{\b}[1]{{\color{blue} #1}}
\renewcommand{\r}[1]{{\color{red} #1}}
\newcommand{\bo}[1]{\b{\st{#1}}} %text out
\newcommand{\mbo}[1]{\b{\text{\sout{\ensuremath{#1}}}}} %math out
\renewcommand{\ul}{\underline}
\newcommand{\Rt}{R^{(t)}}
\newcommand{\Rk}{R^{(k)}}
\newcommand{\Tt}{T^{(t)}}
\newcommand{\Prn}{\{P_1, \dots, P_{t_3} \}}
\newcommand{\Tk}{T^{(k)}}
\newcommand{\ins}{\I=(N,M,V)}
\newcommand{\tf}[2]{\tfrac{#1}{#2}}
\newcommand{\A}{A = (A_1,A_2,\dots,A_n)}
\newcommand{\J}{(J_1 \cup J_2)}
\renewcommand{\P}[2]{P_{#1},\dots,P_{#2}}
\renewcommand{\H}{\mathcal{H}}
\renewcommand{\L}{\mathcal{L}}
\newcommand{\pair}{\{j,j'\}}
%\renewcommand{\C}{\mathcal{C}}
\newcommand{\N}[2]{N_{#1}\cup N_{#2}}
\renewcommand{\S}{\mathcal{S}}
\newcommand{\R}{\mathcal{R}}
\newcommand{\aor}{(\tf{3}{4} \lor \tf45)}
\newcommand{\ld}[2]{#1 \text{ out of } #2}
\newcommand*{\etal}{{\em et~al.}}
\newenvironment{claimproof}[1]{\emph{Proof. }\space#1}{\hfill $\blacksquare$\medskip\par}


\DeclareMathOperator*{\argmin}{\arg\!\min}

\title{Improving Approximation Guarantees for Maximin Share}
\author{Hannaneh Akrami\thanks{Max Planck Institute for Informatics and Graduiertenschule Informatik, Universit\"at des Saarlandes}\\ \texttt{\small hakrami@mpi-inf.mpg.de} \and Jugal Garg\thanks{University of Illinois at Urbana-Champaign. Supported by NSF Grant CCF-1942321}\\ \texttt{\small jugal@illinois.edu} \and Setareh Taki\thanks{Grubhub, USA}\\ \texttt{\small Staki@grubhub.com}}

\date{}


\begin{document}
\maketitle
\begin{abstract}

The Fast Reciprocal Square Root Algorithm is a well-established approximation technique consisting of two stages: first, a coarse approximation is obtained by manipulating the bit pattern of the floating point argument using integer instructions, and second, the coarse result is refined through one or more steps, traditionally using Newtonian iteration but alternatively using improved expressions with carefully chosen numerical constants found by other authors. The algorithm was widely used before microprocessors carried built-in hardware support for computing reciprocal square roots. At the time of writing, however, there is in general no hardware acceleration for computing other fixed fractional powers. This paper generalises the algorithm to cater to all rational powers, and to support any polynomial degree(s) in the refinement step(s), and under the assumption of unlimited floating point precision provides a procedure which automatically constructs provably optimal constants in all of these cases. It is also shown that, under certain assumptions, the use of monic refinement polynomials yields results which are much better placed with respect to the cost/accuracy tradeoff than those obtained using general polynomials. Further extensions are also analysed, and several new best approximations are given.

\end{abstract}

%\thispagestyle{empty}
%\newpage
%\setcounter{page}{1}
% Figure environment removed

\section{Introduction}
Automatic 3D reconstruction of clothed humans using image inputs has gained increasing significance due to its potential applications in a wide array of AR/VR scenarios. High-fidelity reconstructions typically depend on sophisticated capture systems, which are developed with dense camera arrays~\cite{collet2015high,joo2015panoptic,joo2018total}, programmable light-stages~\cite{Vlasic2009, guo2019relightables}, and depth sensors~\cite{newcombe2011kinectfusion,DoubleFusion,BodyFusion,dou2016fusion4d,newcombe2015dynamicfusion}. However, stringent capture environments equipped with complex hardware pose significant challenges for consumer-level applications.


In this context, considerable research effort has been dedicated to developing methods that allow for more flexible capture configurations, such as utilizing a few RGB inputs. Among these works, learning implicit functions \cite{iccv2020PIFu, saito2020pifuhd, hong2021stereopifu} has proven effective in achieving highly detailed reconstructions by integrating the advancements of deep neural networks. These methods employ large multi-layer perceptrons (MLPs) to predict the occupancy probability or truncated signed distance function (TSDF) value of every queried 3D point based on its associated local feature, which is extracted from images. They can recover a continuous surface at arbitrary resolutions without topology restrictions.


However, in typical MLP-based implicit networks, the occupancy or TSDF value at each location is solved independently with planar image features, rendering them less capable of addressing challenging cases such as occlusions. Consequently, these methods suffer from generalization and robustness issues, particularly when tackling strong occlusions caused by large motion or multiple interacting humans. 
Some follow-up studies  \cite{zheng2021deepmulticap,zheng2021pamir,huang2020arch} utilize an extra geometric model, SMPL~\cite{Loper2015}, to improve robustness by introducing strong shape priors. 
Their success typically relies on the assumption of geometrical similarity \cite{huang2020arch} between the shape prior and target reconstruction, making them intractable for handling complex cases with loose clothes and sensitive to errors in SMPL model fitting.



%\ping{this paragraph sounds like `TSDF is better than MLP/SMPL, and we use TSDF to solve the problem'. But in Sec 3, we are telling a different story, saying `MLP needs a 3D convolutional encoder'. We need to make these two sections consistent.}\sicong{I think in this paragraph we claim that the TSDF}


%We opt for Trucated Signed Distance Funtion (TSDF) volumetric representations as they are naturally suitable for convolution operations, which have shown remarkable performance for learning hierarchical features on 2D visual perception tasks \cite{SunXLW19}. 
%Meanwhile, TSDF also describes the gradual geometry change around shape surface, which is not reflected by occupancy volume. 

We instead revisit the 3D volumetric representation and resort to 3D convolutional neural networks (CNNs) for feature learning, due to their impressive performance in feature learning and the ability to incorporate spatial context. However, volumetric methods and 3D convolution involve discretization, which might raise concerns regarding whether a discretized volume can preserve subtle geometric details as continuous representations learned in implicit functions. We investigate the relationship between volume resolution and quantization error on synthetic data by converting target mesh objects to TSDF volumes, as shown in Figure~\ref{fig:quantization_error}. We observe that the quantization errors are significantly reduced by increasing volume resolution and become nearly negligible when reaching a relatively high resolution (e.g., 512 or higher). In other words, achieving fine-detailed reconstruction is not supposed to be restricted by the use of volume representations as long as a proper volume resolution is utilized. Therefore, we present a method with high-resolution feature volumes, e.g., 256 and 512, while traditional volumetric methods \cite{varol18_bodynet,gilbert2018volumetric} are often limited to much lower resolutions, such as 32 or 128.



On the other hand, an increase in volume resolution may lead to a cubic growth of memory overhead \cite{8100085}. Reducing memory costs while guaranteeing the granularity of volumetric representations is necessary for pursuing high-quality reconstruction. Thus, we adopt a coarse-to-fine approach and cull away irrelevant voxels to build a sparse high-resolution feature volume. At the coarse level, the network computes an initial TSDF by applying a U-Net with sparse 3D CNN \cite{3DSemanticSegmentationWithSubmanifoldSparseConvNet} on the sparse feature volume, which is carved by a visual hull. Through our experiments, it turns out that more than 95\% of the volume grids are discarded by the visual hull culling, making the sparse 3D CNN efficient. At the fine level, the network focuses on a narrow band near the zero-level set of the initial TSDF and discretizes the narrow band with smaller voxels. By employing this narrow-band culling, we further shrink the sampling space, resulting in a relatively small range of grid numbers (usually 300K--500K in our experiments) even with a high volume resolution of 512. The remaining voxels in the narrow band are associated with features that fuse high-frequency information from the computed normal maps upon the low-frequency shape from the coarse level to compute the TSDF at high resolution. The final mesh is then extracted from the TSDF using the Marching-Cube algorithm ~\cite{Lorensen87marchingcubes}.
% Different from the u-net sturcture to preserve global topology context, we then apply a shallow 3dcnn to compute the final TSDF $D_{final}$ which contain more local geometry detail.




% \ping{this paragraph can be expanded. It is an important contribution and often ignored by other works. stress on the novel idea of regressing blending weights instead of colors}

In addition to geometry, high-quality mesh texture is also a crucial factor contributing to visual appearance. Directly computing a color field in 3D space, as in \cite{iccv2020PIFu}, struggles to capture high-frequency texture details, while the neural radiance field (NeRF) \cite{yu2020pixelnerf} or the DoubleField~\cite{shao2022doublefield} require expensive per-instance optimization and are often unstable for sparse input images. In contrast, we adopt an image-based rendering approach to compute a texture atlas map, which is efficient and widely supported in existing computer graphics tools. 
Specifically, we compute a blending weight at each 3D point on the mesh surface to determine its color as a weighted average of the colors at its image projections. The blending weights can be computed at a relatively coarse resolution, e.g., 512 volume resolution in our case, and leave texture details to the high-resolution images, such as 1K or 2K. Unlike previous methods that generate blurry texturing results under sparse input, our method generalizes well on both synthetic and real data with just a few input views. 
Figure~\ref{fig:teaser} shows two examples reconstructed by our method. Despite the challenging garment, pose, and occlusion, our method recovers faithful shape, normal, and texture on the right.

%with a wide variety of poses and clothing styles, and it is also adaptive to handle input image with arbitrary resolutions.
%\sicong{For this concern we claim that when the resolution of dicretized volume meets certain threshold (which is 256 in our experiment), the quantization error can be neglected.} 



In summary, the main contributions of this paper are as follows:
\begin{itemize}
\vspace{-0.1in}
  \item 
  We revisit the 3D volumetric representation and demonstrate that it can support clothed human reconstruction with equal or even better performance compared to implicit representation. 
  \item 
  We develop a memory and computation-efficient method for high-resolution volumetric reconstruction using sophisticated sparse 3D CNN, coarse-to-fine estimation, and voxel culling by visual hull and narrow bands. 
  \item 
  We introduce a novel method to compute a texture atlas map, which captures rich appearance details from high-resolution input images.
  \item 
  We achieve impressive results on standard benchmark datasets Twindom and MultiHuman, significantly reducing the point-2-surface (P2S) precision to approximately 0.2cm from just six input views, with more than $50\%$ error reduction compared to the state-of-the-art methods, including DoubleField~\cite{shao2022doublefield} and PIFuHD~\cite{saito2020pifuhd}.
\end{itemize}
Most academic vocabulary lists have been developed in the context of English for Academic Purposes (EAP). On the whole, two categories of lists exist. One list type aims to identify academic words commonly used in EAP across disciplines, which students could be made aware of. The studies aiming to provide cross-disciplinary academic word lists usually use large corpora containing expert academic writing from various disciplines. The widely used lists of this type are the Academic Word List (AWL) \cite{coxhead2000new} and the Academic Vocabulary List (AVL) \cite{gardner2014new}. The second type of list seeks to identify discipline or field-specific words worth teaching. Various specialised lists have been developed for fields such as veterinary medicine \cite{ohashi2020esp} or nursing \cite{yang2015nursing}.

While there is a growing interest in building cross-disciplinary academic word lists for languages other than English, these academic word lists remain few. See, for example studies conducted for French \cite{cobb2004there}, Persian \cite{rezvanifirst}, Portuguese \cite{baptista2010p}, Swedish \cite{carlund2012academic}, and Norwegian \cite{johannessen2016constructing}. An explanation for this scarcity might be that academic language data sets are rare and often not freely available due to copyright. This can be especially true for low-resource languages, such as Romanian. Access to a representative corpus is crucial, as the validity and reliability of an academic word list highly depend on the quality of the data set. 

Apart from the limited availability of academic writing corpora, an additional challenge may be that there is no standard procedure for extracting academic word lists. Scholars are still exploring and testing various methodologies. For example, some studies build on the methods used for the AWL or the AVL \cite{johannessen2016constructing,rezvanifirst}. One study uses the translated version of the AVL in Portuguese as a starting point for its investigation \cite{baptista2010p}. Another study proposes a new word list extraction method different from previous ones \cite{carlund2012academic}.  

In the case of Romanian, no previous studies have compiled specialised or general academic word lists. Although in the last 10-15 years, several research institutions and projects have been involved in developing corpus resources in Romanian, relatively few have focused exclusively on general academic writing. Some of the most significant corpora recently compiled, such as ROMBAC (Romanian Balanced Annotated Corpus, see \citet{ion2012rombac}), with more than 30 million words, CoRoLa (Corpus of Contemporary Romanian Language, see \citet{mititelu2014corola}), or The Balanced Romanian Corpus (BRC, see \citet{midrigan2020resources}) cover only few disciplines or subsets: 5 sections for ROMBAC (journalism, literature, medical texts, legal texts, biographies), uneven and unfiltered distribution of resources in CoRoLa (the collection of academic writing texts is based on agreements with publishing houses and journals, without filtering of the content on quality criteria) and BRC (literary text samples, research articles, news, spoken data). The ROMBAC corpus (excluding the medical subcorpus) was already used to develop the Romanian Word List (RWL, see \citet{szabo2015introducing}), targeted at Romanian L2 learners (e.g. from the Hungarian minority in Romania). The list is a general list of words, not focused on academic language. As far as discipline-specific corpora are concerned, smaller corpora such as SiMoNERo (medical corpus, \citet{mitrofan2019monero}), BioRo \cite{mitrofan2018bioro}, PARSEME-Ro (news articles), LegalNERo (legal, \citet{paiș2021named}), MARCELL (legal, multilingual, see \citet{varadi2020marcell}), CURLICAT (multilingual, containing several domains: Economics, Education, Health, Sciences, etc., see \citet{varadi2022introducing}) have been compiled. However, apart from compiling the datasets and conducting a series of descriptive studies, no special attention is given to the lexical level. 

In this context, the EXPRES corpus (Corpus of Expert Writing in Romanian and English) is the first corpus of discipline-specific academic writing in the Romanian context (academic writing in Romanian L1 and academic writing in English L2 produced by Romanians) \cite{bucur2022expres,chitez2022write}. Covering four disciplines – Linguistics, Economics, Political Sciences, Information Technology –, the Romanian subset contains 200 open-access research articles from each domain, published in the past 5-10 years in peer-reviewed journals (see \citet{chitez2022expres}). The rigorous selection criteria \cite{rogobete2021challenges} contribute to the representativeness of the corpus, making it a suitable candidate for testing a possible Romanian Word List and narrowing it down to an Academic Word List. Furthermore, the EXPRES corpus is the first Romanian expert academic corpus available on an open-access query platform. Unlike other Romanian corpora, which offer limited access to third parties and poor resources for downloading search results or statistics, the EXPRES corpus support platform has been implemented as a cross-platform distributed web application  \cite{chitez2022expres}.

We first review some basic concepts from probability theory (see standard textbooks such as \cite{pollard2002user,williams1991probability} for a detailed treatment), 
%the background of Bayesian inference, and finally 
%We first review some basic concepts from probability theory, 
and then present the Bayesian probabilistic programming language and the normalised posterior distribution (NPD) problem.
%we consider in this work. 
Throughout the paper,
we denote by $\Nset$, $\Zset$ and $\Rset$ the sets of all natural numbers (including zero), integers, and real numbers, respectively.

\vspace{-1.5ex}
\subsection{Basics of Probability Theory}
%We assume familiarity with basic probability theory (see \cref{app:prelim} for details). 

A \emph{measurable space} is a pair $(U,\Sigma_U)$, where $U$ is a nonempty set and $\Sigma_U$ is a $\sigma$-algebra on $U$, i.e., a family of subsets of $U$ such that $\Sigma_U\subseteq \mathcal{P}(U)$ contains $\emptyset$ and is closed under complementation and countable union. Elements of $\Sigma_U$ are called \emph{measurable} sets. A function $f$ from a measurable space $(U_1,\Sigma_{U_1})$ to another measurable space $(U_2,\Sigma_{U_2})$ is \emph{measurable} if $f^{-1}(A)\in\Sigma_{U_1}$ for all $A\in\Sigma_{U_2}$.

A \emph{measure} $\mu$ on a measurable space $(U,\Sigma_U)$ is a mapping from $\Sigma_U$ to $[0,\infty]$ such that (i) $\mu(\emptyset)=0$ and (ii) $\mu$ 
%satisfies the
is countably additive:
%condition: 
for every pairwise-disjoint set sequence $\{A_n\}_{n\in\Nset}$ in $\Sigma_U$, it holds that $\mu(\bigcup_{n\in\Nset}A_n)=\sum_{n\in\Nset}\mu(A_n)$. We call the triple $(U,\Sigma_U,\mu)$ a \emph{measure space}. 
%If $\mu(U)\le 1$, we call $\mu$ a \emph{subprobability measure}. 
If $\mu(U)=1$, we call $\mu$ a \emph{probability measure}, and $(U,\Sigma_U,\mu)$ a \emph{probability space}.
The Lebesgue measure $\lambda$ is the unique measure on $(\Rset,\Sigma_{\Rset})$ satisfying $\lambda([a,b))=b-a$ for all valid intervals $[a,b)$ in $\Sigma_{\Rset}$. For each $n\in\Nset$, we have a measurable space $(\Rset^n,\Sigma_{\Rset^n})$ 
%such that there exists 
and
a unique product measure $\lambda_n$ on $\Rset^n$ satisfying $\lambda_n(\prod_{i=1}^n A_i)=\prod_{i=1}^n \lambda(A_i)$ for all $A_i\in\Sigma_{\Rset}$.


The \emph{Lebesgue} integral operator $\int$ is a partial operator that maps a measure $\mu$ on $(U,\Sigma_U)$ and a real-valued function $f$ on the same space $(U,\Sigma_U)$ to a real number or infinity, which is denoted by $\int f \mathrm{d}\mu$ or $\int f(x)\mu(\mathrm{d}x)$. 
The detailed definition of Lebesgue integral is somewhat technical, see \cite{rankin1968real,rudin1976principles} for more details. 
Given a measurable set $A\in\Sigma_U$, the integral of $f$ over $A$ is defined by $\int_A f(x)\mu(\mathrm{d} x):=\int f(x) \cdot [x\in A] \mu(\mathrm{d}x)$
%\begin{align*}
%\textstyle\int_A f(x)\mu(\mathrm{d} x):=\int f(x) \cdot [x\in A] \mu(\mathrm{d}x)
%\end{align*} 
where $[-]$ is the Iverson bracket such that $[\phi]=1$ if 
%the predicate 
$\phi$ is true, and $0$ otherwise. If $\mu$ is a probability measure, then we call the integral as the \emph{expectation} of $f$, denoted by $\expectdist{x\sim\mu;A}{f}$, or $\expv[f]$ when the scope is clear from the context.

For a measure $v$ on $(U,\Sigma_U)$, a measurable function $f:U\to \Rset_{\ge 0}$ is the \emph{density} of $v$ with respect to $\mu$ if $v(A)=\int f(x)\cdot [x\in A] \mu(\mathrm{d} x)$ for all measurable $A\in\Sigma_U$, and $\mu$ is called the \emph{reference measure} (most often $\mu$ is the Lebesgue measure). Common families of probability distributions on the reals, e.g., uniform, normal distributions, are measures on $(\Rset,\Sigma_{\Rset})$. Most often these are defined in terms of probability density functions with respect to the Lebesgue measure. That is, for each $\mu_D$ there is a measurable function $\text{pdf}_D:\Rset\to\Rset_{\ge 0}$ that determines it: $\mu_D(A):=\int_A \text{pdf}_D (\mathrm{d}\lambda) $. As we will see, density functions such as $\text{pdf}_D$ play an important role in Bayesian inference.

Given a probability space $\pspace$, a \emph{random variable} is an $\mathcal{F}$-measurable function $X: \Omega \rightarrow \Rset \cup \{+\infty,-\infty\}$. The expectation of a random variable $X$, denoted by $\expv(X)$, is the Lebesgue integral of $X$ w.r.t. $\probm$, i.e., $\int X\,\mathrm{d}\probm$. A \emph{filtration} of $\pspace$ is an infinite sequence $\{ \mathcal{F}_n \}_{n=0}^{\infty}$ such that for every $n\ge 0$, the triple $(\Omega, \mathcal{F}_n, \probm)$ is a probability space and $\mathcal{F}_n \subseteq \mathcal{F}_{n+1} \subseteq \mathcal{F}$. A \emph{stopping time} w.r.t. $\{ \mathcal{F}_n \}_{n=0}^{\infty}$ is a random variable $T: \Omega \rightarrow \Nset \cup \{0, \infty\}$ such that for every $n \geq 0$, the event \{$T \leq n$\} is in $\mathcal{F}_n$. 

A \emph{discrete-time stochastic process} is a sequence $\Gamma = \{X_n\}_{n=0}^\infty$ of random variables in $\pspace$. The process $\Gamma$ is \emph{adapted} to a filtration $\{ \mathcal{F}_n \}_{n=0}^{\infty}$, if for all $n \geq 0$, $X_n$ is a random variable in $(\Omega, \mathcal{F}_n, \probm)$. A discrete-time stochastic process $\Gamma=\{X_n\}_{n=0}^\infty$ adapted to a filtration $\{\mathcal{F}_n\}_{n=0}^\infty$ is a \emph{martingale} (resp. \emph{supermartingale}, \emph{submartingale})
if for all $n \geq 0$, $\expv(|X_n|)<\infty$ and it holds almost surely (i.e.,~with probability $1$) that
$\condexpv{X_{n+1}}{\mathcal{F}_n}=X_n$ (\mbox{resp. } $\condexpv{X_{n+1}}{\mathcal{F}_n}\le X_n$, $\condexpv{X_{n+1}}{\mathcal{F}_n}\ge X_n$).
See~\cite{williams1991probability} for details.
%Intuitively, a martingale is a discrete-time stochastic process, in which at any time $n$, the expected value $\condexpv{X_{n+1}}{\mathcal{F}_n}$ in the next step, given all previous values, is equal to the current value $X_n$. In a supermartingale, this expected value is less than or equal to the current value and a submartingale is defined conversely.
Applying martingales to qualitative and quantitative analysis of probabilistic programs is a well-studied technique~\cite{SriramCAV,ChatterjeeFG16,ChatterjeeNZ2017}.


\subsection{Bayesian Probabilistic Programming Language}

%We consider an imperative arithmetic probabilistic programming language. 
The syntax of our probabilistic programming language (PPL) is given in \cref{fig:syntax}, where the metavariables $S$, $B$ and $E$ stand for statements, boolean expressions and arithmetic expressions, respectively.   
Our PPL is imperative with the usual conditional and loop structures (i.e.,~\textbf{if} and \textbf{while}), as well as the following new structures: (a)~sample constructs of the form ``$\textbf{sample}\  D$'' that sample a value from a prescribed distribution $D$ over $\mathbb{R}$ and then assign this value to a sampling variable $r$; (b)~score statements of the form ``\textbf{score}($EW$)'' that weight the current execution with a value expressed by $EW$ (note that $\textit{pdf}(D,x)$ means the value of a probability density function w.r.t. $D$ at $x$);
%\footnote{Instead of the hard conditioning that refutes the execution when the observation mismatches the value of the sampling variable, we use the more general soft conditioning and assume the existence of a global weight variable initialized  to $1$.}
%for each program
(c)~probabilistic branching statements of the form
``$\textbf{if}\ \textbf{prob}(p)\dots$'' that lead to the then part with probability
$p\in (0,1]$ and to the else part with probability $1-p$. We also have sequential compositions (i.e., ";") and support return statements (i.e., \textbf{return}) that 
return the value of the program variable of interest. %The set of all statements is denoted by $Stmt$.
Note that $c,c_1,c_2\in\Rset$ are constants, and our language supports any distributions with continuous density functions and infinite supports, 
including but not limited to uniform and normal distributions. 



% Figure environment removed





Given a probabilistic program in our language, we distinguish two disjoint sets of variables in the program: (i) the set $\pvars$ of \emph{program variables} whose values are determined by assignments in the program (i.e., the expressions at the LHS of ``:="); (ii)~the set $\rvars$ of \emph{sampling variables} whose values are independently sampled from prescribed probability distributions each time they are accessed (i.e., each ``$\textbf{sample}\ D$" can be regarded as a sampling variable $r$). 




\begin{example}\label{ex:pedestrian-program}

%Consider the pedestrian random walk example~\cite{DBLP:conf/esop/MakOPW21}, a pedestrian is lost on a road, and she only knows that she is away from her house at most $3$ km. Thus, she starts to repeatedly walk a uniformly random distance of at most $1$ km in either direction, until reaching her house. Upon she arrives, an  odometer tells that she has walked $1.1$ km totally. However, this odometer was once broken and the measured distance is normally distributed around the true distance with standard deviation $0.1$ km. 
\cref{fig:pedestrian-program} shows a Bayesian probabilistic program written in our PPL language. In this program, the set of program variables is $\pvars=\{start,pos,dis,step\}$, and the set of sampling variables is $\rvars=\{ \textbf{sample uniform}(0,1)\}$. Each time $\textbf{sample uniform}(0,1)$ is executed, it samples a value uniformly from $[0,1]$ and then assigns the value to the variable $step$. 
%Thus, $step$ is associated with the probability distribution $\textbf{uniform}(0,1)$.
\qed


	
% Figure environment removed
\end{example}

\subsection{The Semantics of Our Programming Language}

%To relate variables with their values, we introduce the notion of valuations. 
Let $V$ be a finite set of variables with an implicit linear order over its elements. A \emph{valuation} on $V$ is a function $\pv: V \rightarrow \Rset$ that assigns a real value to each variable in $V$. We denote the set of all valuations on $V$ by $\val{V}$. For each $1\le i\le |V|$, we denote the value of the $i$-th variable (in the implicit linear order) in $\pv$ by $\pv[i]$, so that we can view each valuation as a real vector on $V$. A \emph{program} (resp. \emph{sampling}) valuation is a valuation on $\pvars$ (resp. $\rvars$), respectively. 
For the sake of convenience, we fix the notations in the following way, i.e., we always use $\pv\in\val{\pvars}$ to denote a program valuation, and $\rv\in\val{\rvars}$ to denote a sampling valuation; we also write $\pv[\mathit{ret}]$ for the value of the return variable in $\pv$. 



Below we present the semantics for our programming language. Existing semantics in the literature are either measure-\cite{DBLP:conf/lics/StatonYWHK16,LeeYRY20} or sampling-based  \cite{DBLP:conf/esop/MakOPW21,Beutner2022b}. To facilitate the development of our algorithm, we consider the \emph{transition-based} semantics~\cite{DBLP:conf/cav/ChakarovS13,DBLP:conf/popl/ChatterjeeFNH16} to our language and 
%To apply template-based algorithmic approaches to NPD problems, we consider  that 
treat each probabilistic program as a \emph{weighted probabilistic transition system} (WPTS). A WPTS extends a PTS  ~\cite{DBLP:conf/cav/ChakarovS13,DBLP:conf/popl/ChatterjeeFNH16} with weights and an initial probability distribution. 





%Below we present a variant of probabilistic transition systems \cite{DBLP:conf/cav/ChakarovS13}.
\begin{definition}
%[Weighted Probabilistic Transition Systems]
[WPTS]\label{def:wpts}
	A \emph{weighted probabilistic transition system} (WPTS) $\Pi$
	is a tuple
\begin{equation}\label{eq:wpts} 
\tag{\dag}
\Pi = (\pvars, \rvars,  L,\lin,\lout,\mu_{\mathrm{init}}, \rdvarjdis,\transset)%\win)
\end{equation}
for which:
	\begin{itemize}
		\item
		$\pvars$ and $\rvars$ are finite disjoint sets of \emph{program} and resp. \emph{sampling} variables.
%  (variables}) 
%  such that $\pvars\cap \rvars=\emptyset$.
    \item $\locs$ is a finite set of \emph{locations} 
  %or \emph{program counters} 
  with special locations $\lin,\lout\in \locs$. Informally, $\lin$ is the initial location and $\lout$ represents program termination. 
		\item
		$\mu_{\mathrm{init}}$ is the \emph{initial probability distribution} over $\mathbb{R}^{\pvars}$ with a finite support (denoted by $\supp{\mu_{\mathrm{init}}}$), 
  %from which the initial program valuation %$\valin$ is sampled, 
  while $\rdvarjdis$ is a function that assigns a probability distribution $\rdvarjdis(r)$ to each 
  %sampling variable 
  $r \in \rvars$. We call each $\pv\in\supp{\mu_{\mathrm{init}}}$ an \emph{initial program valuation}, and abuse the notation so that $\rdvarjdis$ also denotes the independent joint distribution of all $\rdvarjdis(r)$'s ($r\in \rvars$).
		\item 
		$\transset$ is a finite set of \emph{transitions} where
		each transition $\tau \in \transset$ is a tuple $\langle \loc, \phi, F_1,\dots,F_k \rangle$ such that 
(i) $\loc\in L$ is the \emph{source location} of the transition, 
%\item 
(ii) $\phi$ is the \emph{guard condition} which is a predicate over variables $\pvars$, %which serves as the \emph{guard condition}, 
and (iii) each $F_j:=\langle \loc'_j, p_j, \upd_j,\wet_j \rangle$ is called a \emph{weighted fork} for which (a) $\loc'_j\in L$ is the \emph{destination location} of the fork, (b) $p_j\in (0,1]$ is the probability of this fork, (c) $\upd_j:\Rset^{|\pvars|} \times \Rset^{|\rvars|} \rightarrow \Rset^{|\pvars|}$ is an {\em update function} that takes as inputs the current program and sampling valuations  and returns an updated program valuation in the next step, and (d) $\wet_j:\Rset^{|\pvars|} \times \Rset^{|\rvars|}\to [0,\infty)$ is a \emph{score function} that gives the likelihood weight of this fork depending on the current program and sampling valuations.	
\end{itemize}
\end{definition}


In a WPTS, we use update and score functions to model the update on the program variables and resp. the likelihood weight when running a basic block of statements in a program, respectively.  
%and use score functions to model  caused by the execution of the score statements (if exists) in this block. 
If there is no score statement in the block, then the score function is constantly $1$. 
We always assume that a WPTS $\Pi$ is \emph{deterministic} and \emph{total}, i.e., (i) there is no program valuation that simultaneously satisfies the guard conditions of two distinct transitions from the same source location, and (ii) the disjunction of the guard conditions of all the transitions from any source location is a tautology. 
The transformation from a probabilistic program into its WPTS can be done in a straightforward way (see e.g.~\cite{DBLP:journals/toplas/ChatterjeeFNH18,DBLP:conf/cav/ChakarovS13}). 

\begin{example}\label{ex:pedestrian-semantics} 
\cref{fig:pedestrian-wpts} shows the WPTS of the program in \cref{fig:pedestrian-program} which has two locations $\lin,\lout$. 
 %In the WPTS, 
The circle nodes represent locations and square nodes model the forking behavior of transitions. An edge entering a square node is labeled with the condition of its respective transition, while an edge entering a circle node stands for a fork, which is associated with its probability, update functions and score functions that marked by $w$.\footnote{Here we omit the update functions if the values of program variables are unchanged.} The value of $step$ is initialised to $0$. An the initial probability distribution $\mu_{\mathrm{init}}$ is determined by the joint distribution of $(start,pos,dis,step)$ where $start\sim uniform(0,3)$ and $pos,dis,step$ observe the Dirac measures $Dirac(\{start\})$, $Dirac(\{0\})$ and $Dirac(\{0\})$, respectively, e.g., the probability of the event ``$dis\in\{0\}$'' equals $1$. As $step$ simply receives values from a sampling variable, we neglect it in the WPTS.\qed
\end{example}

%\paragraph{Score-recursive WPTS.} 

We say that a WPTS is \emph{non-score-recursive} if for all transitions $\tau=\langle \loc, \phi, F_1,  F_2,\dots,F_k \rangle$ in the WPTS with each fork $F_j=\langle \loc'_j, p_j, \upd_j,\wet_j \rangle$ ($1\le j\le k$), we have that each score function $\wet_j$ is constantly $1$ (i.e., the multiplicative weight does not change) for every $\loc'_j\ne \lout$. Otherwise, the WPTS is \emph{score-recursive}.
Informally, a non-score-recursive WPTS has non-trivial score functions only on the transitions to the termination of a program, while a score-recursive WPTS has {\tt score} statements in the execution of the program. 
For example, the WPTS of the program in~\cref{sec3:pedestrian} is non-score-recursive as the nontrivial (i.e., score values not equal to $1$) {\tt score} statement only appears to the termination, while the WPTS of the program in \cref{sec3:phylogenetic} is score recursive since it has {\tt score} statements inside the loop body.
In the case of a non-score-recursive WPTS, we say that the WPTS is \emph{score-bounded} by a positive real $M>0$ if for every $\tau=\langle \loc, \phi, F_1, F_2,\dots,F_k \rangle$ in the WPTS with $F_j=\langle \loc'_j, p_j, \upd_j,\wet_j \rangle$ ($1\le j\le k$), we have that 
$|\wet_j|\le M$ whenever $\loc'_j=\lout$.


Given a program valuation $\mathbf{v}$ and a predicate $\phi$ over variables $\pvars$, we say that $\mathbf{v}$ \emph{satisfies} $\phi$ (written as $\mathbf{v}\models\phi$) if $\phi$ holds when the variables in $\phi$ are substituted by their values in $\mathbf{v}$. 
A \emph{state} 
%of the WPTS $\Pi$ 
is a pair $\Xi=(\loc, \pv)$ where $\loc \in L$ (resp. $\pv \in \Rset^{|\pvars|}$) represents the current location (resp. program valuation), respectively, while a \emph{weighted state} is a triple 
%$\Xi^w:=(\loc, \pv,w)$ 
$\Theta=(\loc, \pv, w)$ 
where $(\loc, \pv)$ is a state and $w\in [0,\infty)$ represents the multiplicative likelihood weight accumulated so far. 


 
%\paragraph{Semantics.} 
Below we specify the semantics of a WPTS. Consider a WPTS $\Pi$ in the form of \eqref{eq:wpts}. The semantics of $\Pi$ is formalized by the infinite sequence $\Gamma=\{\widehat{\Theta}_n=(\widehat{\loc}_n,\widehat{\pv}_n,\widehat{w}_n)\}_{n\ge 0}$ 
%of \emph{random weighted states} 
where each $(\widehat{\loc}_n,\widehat{\pv}_n,\widehat{w}_n)$ is the random weighted state at the $n$th execution step of the WPTS such that $\widehat{\loc}_n$ (resp. $\widehat{\pv}_n$, $\widehat{w}_n$) is the random variable for the location (resp. the random vector 
%of random variables 
for the program valuation, the random variable for the multiplicative likelihood weight) at the $n$th step, respectively. %The initial random state $\widehat{\Theta}_0$ is constant and equals $(\lin,\valin,\win)$. 
%its corresponding stochastic process $\Gamma:=\{\hat{\Xi}_n\}_{n\ge 0}$ on states.
The sequence $\Gamma$ starts with the initial random weighted state 
$\widehat{\Theta}_0=(\widehat{\loc}_0,\widehat{\pv}_0,\widehat{w}_0)$ such that $\widehat{\loc}_0$ is constantly $\lin$, $\widehat{\pv}_0\in \supp{\mu_\mathrm{init}}$ is sampled from the initial distribution $\mu_\mathrm{init}$ and the initial weight $\widehat{w}_0$ is constantly set to $1$\footnote{This follows the traditional setting in e.g.~\cite{Beutner2022b}.}. 
Then, given the current random weighted state $\widehat{\Theta}_n=(\widehat{\loc}_n,\widehat{\pv}_n,\widehat{w}_n)$ at the $n$th step, the next random weighted state $\widehat{\Theta}_{n+1}=(\widehat{\loc}_{n+1},\widehat{\pv}_{n+1},\widehat{w}_{n+1})$ is determined by:
(a) If $\widehat{\loc}_n=\lout$, then $(\widehat{\loc}_{n+1}, \widehat{\pv}_{n+1},\widehat{w}_{n+1})$ takes the same weighted state as $(\widehat{\loc}_n,\widehat{\pv}_n,\widehat{w}_n)$ (i.e., the next weighted state stays at the termination location $\lout$);
(b) Otherwise, $\widehat{\Theta}_{n+1}$ is determined by the following procedure:
\begin{itemize}
\item First, since the WPTS $\Pi$ is deterministic and total, we take the unique transition $\tau=\langle \hat{\loc}_n,\phi,F_1,\dots, F_k \rangle$ such that $\hat{\pv}_n\models\phi$. 
\item Second, we choose a fork $F_j=\langle \loc_j, p_j,\upd_j,\wet_j\rangle$ with probability $p_j$.
\item 
Third, we obtain a sampling valuation $\rv\in \supp{\rdvarjdis}$ 
%over the sampling variables $\rvars$ 
by sampling each $r \in \rvars$ independently w.r.t the probability distribution $\rdvarjdis(r).$
\item Finally, the value of the next random weighted state $(\widehat{\loc}_{n+1}, \widehat{\pv}_{n+1},\widehat{w}_{n+1})$ is determined as that of 
$(\loc'_j, \upd_j(\hat{\pv}_n,\rv),\widehat{w}_n\cdot \wet_j(\widehat{\pv}_n,\rv))$. 
\end{itemize}


Given the semantics, a \emph{program run} of the WPTS $\Pi$ is a concrete instance of $\Gamma$, i.e., an infinite sequence $\omega=\{\Theta_n\}_{n\ge 0}$ of weighted states where each $\Theta_n=(\loc_n,\pv_n,w_n)$ is the concrete weighted state at the $n$th step in this program run with location $\loc_n$, program valuation $\pv_n$ and multiplicative likelihood weight $w_n$. A state $(\loc,\pv)$ is called \emph{reachable} if there exists a program run $\omega=\{\Theta_n\}_{n\ge 0}$ such that $\Theta_n=(\loc,\pv,w_n)$ for some $n$. 


 
\begin{example}\label{ex:pedestrian-run}
Consider the WPTS in \cref{ex:pedestrian-semantics}. Consider an initial program valuation $(1,1,0)$ which means that the initial values of $start,pos,dis$ are $1,1,0$, respectively. Then starting from the initial weighted state $(\lin,(1,1,0),1)$, a program run w.r.t the WPTS semantics above could be 
\[
(\lin,(1,1,0),1)\to (\lin,(1,0.5,0.5),1)\to (\lin,(1,-0.1,1.1),1)\to (\lout,(1,-0.1,1.1),3.9894).\qed
\]
\end{example}

Given an initial program valuation $\valin$ of a WPTS, one could construct a probability space over the program runs by their probabilistic evolution described above and standard constructions such as general state space Markov chains~\cite{meyn2012markov}. We denote the probability measure in the probability space by $\probm_{\valin}(-)$ and the expectation operator by $\expectdist{\valin}{-}$.  



\subsection{Normalised Posterior Distribution}\label{sec2:NPD}


Before presenting the central problem of Bayesian probabilistic programming, i.e., analyzing normalised posterior distribution with our WPTS models, we introduce some technical concepts.

%\paragraph{Termination.}
\begin{definition}[Termination]
The \emph{termination time} of a WPTS
%The \emph{termination time} of the WPTS 
$\Pi$ 
%is a random variable $T$ defined on programs runs given 
is the random variable $T$ given by
%a program run  $\omega=\{\Xi_n=(\loc_n,\pv_n,w_n)\}_{n\in\Nset}$,
%\begin{align*}	
$T(\omega):=\text{min}\{n\in\Nset\mid \loc_n=\lout\}$ for every program run  $\omega=\{(\loc_n,\pv_n,w_n)\}_{n\ge 0}$
%\end{align*}
where $\text{min}\,\emptyset:=\infty$. That is, $T(\omega)$ is the number of steps a program run $\omega$ takes to reach the termination location $\lout$. A WPTS $\Pi$ is \emph{almost-surely terminating} (AST) if $\probm_{\valin}(T<\infty)=1$ for all initial program valuations $\valin\in \supp{\mu_{\mathrm{init}}}$.  
%in the case that the program run never terminates. 
\end{definition}




\begin{definition}[Expected Weights]\label{def:exp-wt}
 Given a WPTS $\Pi$ in the form of \eqref{eq:wpts}, a designated initial program valuation $\valin$ and a measurable subset $\calU\in\Sigma_{\Rset^{|\pvars|}}$, the \emph{expected weight} $\measureSem{\Pi}_{\valin}(\calU)$ 
%$\measureSem{\Pi}(\valin)$ 
%of $\Pi$ w.r.t $\pv$ 
is defined as
%$\measureSem{\Pi}_\calU(\valin):=\expectdist{\valin}{\widehat{w}_T}$. 
$\measureSem{\Pi}_{\valin}(\calU):=\expectdist{\valin}{[\widehat{\pv}_T\in \calU]\cdot\widehat{w}_T}$. 
\end{definition}

By definition, we have that $\widehat{\pv}_T$ (resp. $\widehat{w}_T$) is the random vector (resp. variable) of the program valuation (resp. the multiplicative likelihood weight) at termination, respectively. Thus, $\measureSem{\Pi}_{\valin}(\calU)$ is the expectation of $\widehat{w}_T$ 
%over all program runs 
that start from the state $(\lin,\valin,1)$ and end with $\widehat{\pv}_T\in\calU$. If $\calU=\Rset^{|\pvars|}$, the restriction of $\widehat{\pv}_T\in\calU$ can be removed.

Below we define the normalised posterior distribution (NPD) problem. %under our WPTS semantics. 

 
\begin{definition}[Normalised Posterior Distribution]\label{def:npd}
Given a WPTS $\Pi$ in the form of \eqref{eq:wpts},
%We write $\measureSem{\Pi}(\valin)$ iff $\calU=\Rset^{|\pvars|}$.)
%Then given a probability distribution $\mu$ over initial program valuations, 
the \emph{normalised posterior distribution} (NPD) $\posterior_\Pi$ of $\Pi$ 
%over $U$ 
is defined by:
\begin{align*}
\posterior_{\Pi}(\calU):=\measureSem{\Pi}(\calU)/Z_\Pi\mbox{ for all measurable subsets } \calU\in \Sigma_{\Rset^{|\pvars|}},   
\end{align*}	
where 
$\measureSem{\Pi}(\calU):=\int_{\calV} \measureSem{\Pi}_{\pv}(\calU)\cdot \mu_{\mathrm{init}}(\mathrm{d} \pv)$ is the \emph{unnormalised posterior distribution} w.r.t. $\calU$, $\calV:=\supp{\mu_{\mathrm{init}}}$, %is the support of $\mu_{\mathrm{init}}$
%is the integral of all expected weights with an initial program valuation $\pv\in U$, 
and $Z_\Pi:=\measureSem{\Pi}(\Rset^{|\pvars|})$ is the \emph{normalising constant}.  
The WPTS $\Pi$ is called \emph{integrable} 
%w.r.t a probability distribution (for initial program valuations) 
if we have $0<Z_{\Pi}<\infty$. 
%\pw{Shall we mention that $\measureSem{\Pi}_{\pv}(\calU)$ is an integrable function here?}
\end{definition}

%We call a WPTS $\Pi$ \emph{integrable} 
%w.r.t a probability distribution (for initial program valuations) 
%if the normalising constant is finite, i.e., ~$0<Z_{\Pi}<\infty$. %for any $\pv\in\val{\pvars}$. 
%Given an integrable program, we are interested in deriving lower and upper bounds on the normalised posterior distribution over some measurable set $U\in \Sigma_\Rset$.
\paragraph{Interval Bounds for NPD.} In this work, we consider the automated interval-bound analysis for NPD of a WPTS. Formally, we aim to derive an interval $[l,u]\subseteq [0,\infty)$ 
for an integrable WPTS $\Pi$ and any measurable sets $\calU\in\Sigma_{\Rset^{|\pvars|}}$ as tight as possible such that $l\le \posterior_{\Pi}(\calU) \le u$. 
%$l,u$ are called \emph{interval bounds} for the NPD $\posterior_{\Pi}(\calU)$. 
%To achieve this, in the following (\cref{sec:math}) we develop approaches to obtain interval bounds for expected weights as $\measureSem{\Pi}(\calU)$ and $Z_\Pi$ are integrations of expected weights over $\calV$. 
 



To achieve interval bounds for NPD, below we introduce the construction of a new WPTS $\Pi_\calU$ based on the original WPTS $\Pi$ and a measurable set $\calU\in \Sigma_{\Rset^{|\pvars|}}$.  

\paragraph{Construction of $\Pi_\calU$.} Consider a probabilistic program $P$ and its WPTS $\Pi$, given a measurable set $\calU\in\Sigma_{\Rset^{|\pvars|}}$, we construct a new program $P_\calU$ by adding a conditional branch of the form ``\textbf{if} $\pv_T\notin\calU$ \textbf{then} \textbf{score}($0$) \textbf{fi}'' immediately after the termination of $P$ and obtain the WPTS $\Pi_\calU$ of $P_\calU$. Therefore, $\Pi$ and $\Pi_\calU$ have the same initial probability distribution $\mu_{\mathrm{init}}$ and the same finite support $\calV=\supp{\mu_{\mathrm{init}}}$. The following proposition shows that interval-bound analysis for NPD can be reduced to interval-bound analysis for expected weights in the form $\llbracket \Pi\rrbracket_{\pv}(\Rset^{|\pvars|})$. 

\begin{proposition}\label{prop:unnorm-norm}
   Given a WPTS $\Pi$ in the form of \eqref{eq:wpts}, a measurable set $\calU\in\Sigma_{\Rset^{|\pvars|}}$ and the WPTS $\Pi_\calU$ constructed as above, we have that $\llbracket \Pi \rrbracket_{\pv}(\calU)=\llbracket \Pi_\calU\rrbracket_{\pv}(\Rset^{|\pvars|})$ for any $\pv\in\calV=\supp{\mu_{\mathrm{init}}}$. Furthermore,
   if there exist intervals $[l_1,u_1],[l_2,u_2]\subseteq [0,\infty)$ such that $\llbracket \Pi_\calU\rrbracket_{\pv}(\Rset^{|\pvars|})\in [l_1,u_1]$ and $\llbracket \Pi\rrbracket_{\pv}(\Rset^{|\pvars|})\in [l_2,u_2 ]$ for any $\pv\in\calV$, then we have two intervals $[l_\calU,u_\calU],[l_Z,u_Z]\subseteq [0,\infty)$ such that the unnormalised posterior distribution $\llbracket \Pi\rrbracket (\calU)\in [l_\calU,u_\calU]$ and the normalising constant $Z_\Pi\in [l_Z,u_Z]$. Moreover, if $\Pi$ is integrable, i.e., $[l_Z,u_Z]\subseteq (0,\infty)$, then we can obtain the NPD $\posterior_{\Pi}(\calU)\in [\frac{l_\calU}{u_Z},\frac{u_\calU}{l_Z}]$.\footnote{The interval bounds derived in this manner may be loose, but they are definitely correct.}  Note that by \cref{def:npd}, $l_\calU=\int_\calV l_1 \cdot\mu_{\mathrm{init}}(\mathrm{d} \pv)$, $u_\calU=\int_\calV u_1 \cdot\mu_{\mathrm{init}}(\mathrm{d} \pv)$, $l_Z=\int_\calV l_2 \cdot\mu_{\mathrm{init}}(\mathrm{d} \pv)$ and $u_Z=\int_\calV u_1 \cdot\mu_{\mathrm{init}}(\mathrm{d} \pv)$.

\end{proposition}

The proof of \cref{prop:unnorm-norm} is relegated to \cref{app:sec2-prop}. In the following, we will develop approaches to obtain interval bounds for expected weights.
%in the form $\llbracket \Pi \rrbracket_{\pv}(\Rset^{|\pvars|})$ where $\pv$ is an initial program valuation.











%%%%-----------------Definition/Theorems------------------%%%%
\theoremstyle{definition} %%upright text, extra space above and below
    \newtheorem{definition}{Definition}

\theoremstyle{plain} %% italic text, extra space above and below
    \newtheorem{theorem}[definition]{Theorem}
    \newtheorem{proposition}[definition]{Proposition}
    \newtheorem{lemma}[definition]{Lemma}
    \newtheorem{corollary}[definition]{Corollary}
    \newtheorem{claim}[definition]{Claim}
    \newtheorem{assumption}[definition]{Assumption}

\theoremstyle{remark} %% upright text, no extra space above or below
    \newtheorem{remark}[definition]{Remark}



%%%%-----------------Bibliography------------------%%%%

\usepackage[bibstyle=alphabetic,citestyle=alphabetic,useprefix,giveninits=true, sorting=ynt, sortcites, minbibnames=99,maxbibnames=99,backend=biber]{biblatex}  %%other styles: numeric-comp, authoryear 
    %% sorting: ynt in the text, nyt in the bibliography with additional command there
\renewbibmacro{in:}{} %% removes in: before journals
\bibliography{bibliography}  %% Name of the file with the bibliography
\emergencystretch=1em %% Adjusts the overfulls in the bibliography by allowing more space between words

%%% 
\DeclareSourcemap{
  \maps[datatype=bibtex]{
    \map[overwrite]{ %%If DOI is present, doesn't print arXiv
      \step[fieldsource=doi, final]
      \step[fieldset=url, null]
      \step[fieldset=eprint, null]
      }
     \map[overwrite]{ %% If only arXiv is present, doens't print pages and eid (eliminates duplicates)
      \step[fieldsource=eprint, final]
      \step[fieldset=pages, null]
      \step[fieldset=eid, null]
      \step[fieldset=journal, null]
    }  
  }
}

%%%%-----------------Headers------------------%%%%
    % \usepackage{fancyhdr} %% Headers
    %     \pagestyle{fancy}
    %     \fancyhf{}
    %     \fancyhead[LE,RO]{\thepage}
    %     \fancyhead[RE]{ \nouppercase{\leftmark} }
    %     \fancyhead[LO]{ \nouppercase{\rightmark} }
    %     \setlength{\headheight}{15pt}
    % \usepackage{emptypage}



%%%%-----------------Typing shortcuts------------------%%%%

    \newcommand{\Tr}[0]{\text{Tr}}
    \newcommand{\indices}{}
%% tilde variables 
    \renewcommand{\Tilde}{\widetilde}   
    \newcommand{\tc}{\widetilde{c}}
    \newcommand{\tom}{\widetilde{\omega}}
    \newcommand{\tg}{\widetilde{g}}
    \newcommand{\te}{\widetilde{e}}
    \newcommand{\txi}[1]{\widetilde{\xi}^{#1}}
    \newcommand{\tedl}{\widetilde{\underline{e}}^\dag}
    
%% Bold variables
    \newcommand{\bxi}{\boldsymbol{\xi}}
    \newcommand{\bc}{\mathbf{c}}
    \newcommand{\bom}{\boldsymbol{\omega}}
    
%% double tilde variables
    \newcommand{\ttc}{\widetilde{\tc}}
    \newcommand{\tte}{\widetilde{\te}}
    \newcommand{\ttom}{\widetilde{\tom}}
    \newcommand{\ttxi}[1]{\widetilde{\widetilde{\xi^{#1}}}}
    
%% Generic Math
    \DeclareMathOperator{\Ima}{Im}
    \newcommand{\oloc}{\Omega_{\mathrm{loc}}}
    \newcommand{\paral}{\slash\!\slash}
    \newcommand{\Ker}[1]{\mathrm{Ker}{(#1)}}
    \newcommand{\comp}[1]{\langle #1 \rangle}
    \newcommand{\X}[1]{(X_{#1})}
    \newcommand{\qsp}[2]{\,\ensuremath{\raise.5ex\hbox{$#1$}\big\slash\raise-.5ex\hbox{$#2$}}} 
    \newcommand{\pard}[2]{\frac{\delta#1}{\delta#2}}

%% BV shortcuts
    \newcommand{\FF}{\mathfrak{F}}
    \newcommand{\AKSZ}{\textsf{\tiny AKSZ}}
    \DeclareMathOperator{\BFV}{\mathit{BFV}}
    \DeclareMathOperator{\BV}{\mathit{AKSZ}}
    \newcommand{\dd}{\mathrm{d}}
    \newcommand{\ndash}{\nobreakdash-\hspace{0pt}}

%% Commands for AKSZ
    \newcommand{\Fp}[2]{\mathcal{F}_{#2}^\partial(#1)}
    \newcommand{\Sp}[2]{S_{#2}^\partial(#1)}
    \newcommand{\Qp}[2]{Q_{#2}^\partial(#1)}
    \newcommand{\varp}[2]{\varpi_{#2}^\partial(#1)}
    \newcommand{\alp}[2]{\alpha_{#2}^\partial(#1)}
    \newcommand{\uF}[1]{{\mathcal{F}}_{R}(#1)}
    \newcommand{\uFF}[1]{{\mathfrak{F}}_{R}(#1)}
    \newcommand{\uS}[1]{{S}_{R}(#1)}
    \newcommand{\uV}[1]{{\varpi}_{R}(#1)}
    \newcommand{\uQ}[1]{{Q}_{R}(#1)}
    \newcommand{\UQ}{{Q}_R}

%%filtration symbols
    \newcommand{\filt}[1]{(#1)}
    \newcommand{\filtint}[1]{M^{(#1)}}
    \newcommand{\filtBF}[1]{(#1 )}
    \newcommand{\filtintBF}[1]{M^{(#1 )}}

%%% BF non degenerate
    \newcommand{\BFnd}{BF_{*}}

%%%%-----------------For long computations------------------%%%%
    \makeatletter
        \newcommand{\zzlabel}[1]{\ifmeasuring@\else\ltx@label{#1}\fi} %%new label (necessary if amsmath is present)
    \makeatother

    \newcounter{terms}[equation] %%counter for terms in a equation
    \newcommand{\unl}[2]{\underline{#1}_{\refstepcounter{terms} \zzlabel{#2} \theterms}} %%underlines, counts a term and put the corresponding number. Put the term in the first slot and a label in the second

    \newcommand{\reft}[2]{(\ref{#1}.\ref{#2})} %%refers to a term in a equation as (equation.term)

    %\showlabels[\color{blue}]{zzlabel}  %%shows the labels of the terms


\section{1-out-of-(4n/3) MMS Algorithm}\label{sec:4n-3}
Our algorithm in Algorithm~\ref{algo} consists of \emph{initialization} and \emph{bag-filling}. First, we remark that assuming $|M| \geq 2n$ is without loss of generality. This is because we can always add dummy goods to $M$ with a value of $0$ for all the agents. The resulting instance is ordered and $d$-normalized if the original instance has these properties.

As mentioned in Section~\ref{sec:tec}, the algorithm first initialize $n$ bags as in \eqref{eq:B_i} (see Figure~\ref{fig:bags}). 
% Figure environment removed
Then, in each round $j$ of bag-filling, we keep adding goods in decreasing value to the bag $B_j$ until some agent with no assigned bag values it at least $1$. Then, we allocate it to an arbitrary such agent. 

In the rest of this section, we prove the following theorem, showing the correctness of the algorithm.

\begin{theorem}\label{thm:main}
    Given any ordered $(4n/3)$-normalized instance, Algorithm \ref{algo} returns a $1$-out-of-$(4n/3)$ MMS allocation.
\end{theorem}

To do so, it suffices to prove that we never run out of goods in bag-filling. Towards contradiction, assume that the algorithm stops before all agents receive a bundle. Let $i$ be an agent with no bundle. Let $\hat{B}_j$ be the $j$-th bundle after bag-filling.  
\begin{observation}\label{upper-bound-left}
    For all $j,k$ such that $j \leq k \leq n$, $v_i(\hat{B}_j) \leq 1 + v_i(2n-k+1)$.
\end{observation}
\begin{proof}
    Let $g$ be the good with the largest index in $\hat{B}_j$. If $g = 2n-j+1$, $v_i(\hat{B}_j \setminus \{g\}) = v_i(j)  \leq 1$ by Proposition \ref{prop:exact}(\ref{exact:1}). If $g > 2n-j+1$, meaning that $g$ was added to $\hat{B}_j$ during bag-filling, then $v_i(\hat{B}_j \setminus \{g\}) < 1$. Otherwise, $g$ would not be added to $\hat{B}_j$. Therefore, 
    \begin{align*}
        v_i(\hat{B_j}) &= v_i(\hat{B}_j \setminus \{g\}) + v_i(g) \\
        &\leq 1 + v_i(2n-k+1). &\tag{$v_i(\hat{B}_j \setminus \{g\}) \leq 1$ and $g \geq 2n-k+1$}
    \end{align*}
\end{proof}
\begin{observation}\label{upper-bound-right}
    For all $j,k$ such that $k \leq j \leq n$, $v_i(\hat{B}_j) \leq \max(1 + v_i(2n-k+1), 2v_i(k))$.
\end{observation}
\begin{proof}
    First, assume $\hat{B}_j \neq B_j$ and $g$ be the last good added to $\hat{B}_j$. We have $v_i(\hat{B}_j \setminus \{g\}) < 1$. Otherwise, $g$ would not be added to $\hat{B}_j$. Therefore, 
    \begin{align*}
        v_i(\hat{B_j}) &= v_i(\hat{B}_j \setminus \{g\}) + v_i(g) \\
        &< 1 + v_i(2n-k+1). &\tag{$v_i(\hat{B}_j \setminus \{g\}) < 1$ and $g > 2n-k+1$}
    \end{align*}
    Now assume $\hat{B}_j = B_j$. We have
    \begin{align*}
        v_i(\hat{B}_j) &= v_i(B_j) \\
        &= v_i(j) + v_i(2n-j+1) \\
        &\leq 2v_i(k). &\tag{$2n-j+1 > j \geq k$}
    \end{align*}
    Hence, $v_i(\hat{B}_j) \leq \max(1 + v_i(2n-k+1), 2v_i(k))$.
\end{proof}
\begin{observation}\label{less-than-one}
    There exists a bag $B_j$, such that $v_i(B_j) < 1$.
\end{observation}
\begin{proof}
    Otherwise, the algorithm would allocate the remaining bag with the smallest index to agent $i$.
\end{proof}

Let $\ell^*$ be the smallest such that $v_i(B_{\ell^*+1}) < 1$. I.e., $B_{\ell^*+1}$ is the leftmost bag in Figure \ref{fig:bags} with a value less than $1$ to agent $i$. In Section \ref{negative}, we reach a contradiction assuming $v_i(2n-\ell^*)<1/3$ and prove Theorem \ref{contradict-1}.
\begin{restatable}{theorem}{contradictOne}\label{contradict-1}
    If Algorithm \ref{algo} does not allocate a bag to some agent $i$, then $v_i(2n-\ell^*) \geq 1/3$ where $\ell^*$ is the smallest index such that $v_i(B_{\ell^*+1}) < 1$.
\end{restatable}

In Section \ref{positive}, we reach a contradiction assuming $v_i(2n-\ell^*) \geq 1/3$ and prove Theorem \ref{contradict-2}. 
\begin{restatable}{theorem}{contradictTwo}\label{contradict-2}
    If Algorithm \ref{algo} does not allocate a bag to some agent $i$, then $v_i(2n-\ell^*) < 1/3$ where $\ell^*$ is smallest such that $v_i(B_{\ell^*+1}) < 1$.
\end{restatable}

By Theorems \ref{contradict-1} and \ref{contradict-2}, agent $i$ who receives no bundle by the end of Algorithm \ref{algo} does not exist, and Theorem \ref{thm:main} follows.

\subsection{\boldmath $\mathbf{v_i(2n-\ell^*) \geq 1/3}$}\label{positive}
In this section we assume $v_i(2n-\ell^*) = 1/3 + x$ for $x \geq 0$. %\jnote{$x\ge 0$?}
 We define $A^+ := \{B_1, B_2, \ldots, B_{\ell^*}\}$; see Figure~\ref{k-picture}.
% Figure environment removed
\begin{observation}\label{no-change}
    For all $B_j \in A^+$, $\hat{B}_j = B_j$.
\end{observation}
\begin{proof}
    For all $B_j \in A^+$, $v_i(B_j) \geq 1$. Since $i$ did not receive any bundle, $B_j$ must have been assigned to some other agent, and no good needed to be added to $B_j$ in bag-filling since there is an agent (namely $i$) with no bag who values $B_j$ at least $1$. 
\end{proof}
\begin{observation}\label{half-bound}
    For all $j\geq 2n-\ell^*$, $v_i(j) < 1/2$.
\end{observation}
\begin{proof}
    Since $v_i(B_{\ell^*+1}) = v_i(\ell^*+1)+ v_i(2n-\ell^*) < 1$ and $v_i(2n-\ell^*) \leq v_i(\ell^*+1)$, $v_i(2n-\ell^*) < 1/2$.
    Also for all $j \geq 2n-\ell^*$, $v_i(j) \leq v_i(2n-\ell^*) < 1/2$.
\end{proof}

\begin{corollary}[of Observation \ref{half-bound}]
    $x < 1/6$.    
\end{corollary}

Let $s$ be the smallest such that either the algorithm stops at step $s+1$ or $B_{s+1}$ gets more than one good in bag-filling. 
\begin{observation}
    $s \geq \ell^*$.
\end{observation}
\begin{proof}
    For all $j < \ell^*$, $v_i(B_{j+1}) \geq 1$. Since $i$ did not receive any bundle, $B_{j+1}$, must have been assigned to another agent. Therefore, the algorithm does not stop at step $j+1$. Also, by Observation \ref{no-change}, $B_{j+1}$ gets no good in bag-filling.
\end{proof}
Let $A^1$ be the set of bags in $\{B_{\ell^*+1}, \ldots, B_s\}$ which receive exactly one good in bag-filling. Formally, $A^1 = \{B_j | \ell^* < j \leq s \text{ and } |\hat{B}_j| = 3\}$. Let $A^2 = \{B_1, B_2, \ldots, B_n\} \setminus (A^+ \cup A^1)$. 

\begin{lemma}\label{save-2}
    For all $B_j \in A^2$, $v_i(B_j) < 4/3 -2x$.
\end{lemma}
\begin{proof}
    We have 
    \begin{align*}
        1 &> v_i(B_{\ell^*+1}) \\
        &= v_i(\ell^*+1) + v_i(2n-\ell^*) \\
        &= v_i(\ell^*+1) + \frac{1}{3} + x.
    \end{align*}
    Hence, $v_i(\ell^*+1) < 2/3 - x$. Also, for $B_j \in A^2$, we have
    \begin{align*}
        v_i(B_j) &= v_i(j) + v_i(2n-j+1) \\
        &\leq 2v_i(\ell^*+1) &\tag{$2n-j+1 > j \geq \ell^*+1$ since $B_j \in A^2$} \\
        &< \frac{4}{3} - 2x.
    \end{align*}
    So if $\hat{B}_j = B_j$, the inequality holds. Now assume $\hat{B}_j \neq B_j$. This implies that $j \geq s+1$ and the algorithm did not stop at step $j$ before adding a good to $B_j$. Therefore it did not stop at step $s +1$ before adding a good to $B_{s+1}$ either. Let $g$ be the first good added to $B_{s+1}$. Since $B_{s+1}$ requires more than one good,
    \begin{align*}
        1 &> v_i(B_{s +1} \cup \{g\}) \\
        &= v_i(s+1) + v_i(n-s) + v_i(g) \\
        &\geq 2v_i(2n-\ell^*) + v_i(g) &\tag{$s+1 < n-s \leq 2n-\ell^*$} \\
        &= \frac{2}{3} + 2x + v_i(g).
    \end{align*}
    Therefore, $v_i(g) < 1/3 - 2x$. Now let $h$ be the last good added to bag $B_j$. We have
    \begin{align*}
        v_i(\hat{B}_j) &= v_i(\hat{B}_j \setminus \{h\}) + v_i(h) \\
        &< 1 + v_i(g) &\tag{$v_i(\hat{B}_j \setminus \{h\}) < 1$ and $v_i(h) \leq v_i(g)$} \\
        &< \frac{4}{3} - 2x.
    \end{align*}
\end{proof}

\begin{lemma}\label{save-all}
    For all $B_j \in A^+ \cup A^1$, $v_i(\hat{B}_j) \leq 4/3 + x$.
\end{lemma}
\begin{proof}
    First assume $B_j \in A^+$. We have $j \leq \ell^*$. Also,
    \begin{align*}
        v_i(\hat{B}_j) &= v_i(B_j) &\tag{$\hat{B}_j = B_j$} \\
        &= v_i(j) + v_i(2n-j+1) \\
        &\leq 1 + v_i(2n-\ell^*) &\tag{$v_i(j) \leq 1$ and $2n-j+1 > 2n-\ell^*$} \\
        &= \frac{4}{3} +x.
    \end{align*}
    Now assume $B_j \in A^1$. Let $g$ be the good added to bag $B_j$ in bag-filling. We have,
    \begin{align*}
        v_i(\hat{B}_j) &= v_i(B_j) + v_i(g) \\
        &< 1 + v_i(2n-\ell^*) &\tag{$v_i(B_j) < 1$ and $v_i(g) \leq v_i(2n+1) \leq v_i(2n-\ell^*)$}\\
        &= \frac{4}{3} + x. 
    \end{align*}
\end{proof}

Let $|A^1|=2n/3+\ell$. Then $|A^2| = n - \ell^* - (2n/3 + \ell) = n/3-(\ell+\ell^*)$. If $\ell + \ell^* \leq 0$, then $|A^2| \geq n/3$ and hence there are at least $n/3$ bags with value less than $4/3 -2x$ (by Lemma \ref{save-2}) and at most $2n/3$ bags with value at most $4/3 + x$ (by Lemma \ref{save-all}). Hence, 
$$v_i(M) < \frac{n}{3}(\frac{4}{3}-2x) + \frac{2n}{3}(\frac{4}{3}+x) = \frac{4n}{3}$$ 
which is a contradiction since $v_i(M) = 4n/3$. So assume $\ell + \ell^* > 0$. 

Limit the items in a $1$-out-of-$4n/3$ MMS partition $P^i=(P^i_1, \dots, P^i_{4n/3})$ of agent $i$ to $\{1, \ldots, 8n/3+\ell\}$ and let $Q$ be the set of bags in $P^i$ containing goods $\{1, \ldots, \ell^*\}$. Formally, $Q = \{ P_j^i \cap \{1, \ldots, 8n/3+\ell\}: |P_j^i \cap \{1, \ldots, \ell^*\}| \geq 1\}$. Let $t$ be the number of bags of size $1$ in $Q$.

% Figure environment removed

\begin{lemma}\label{expowerful}
    Let $t$ be the number of bags of size $1$ in $Q = \{ P_j^i \cap \{1, \ldots 8n/3+\ell\}: |P_j^i \cap \{1, \ldots \ell^*\}| \geq 1\}$. Then,
    \begin{align*}
        v_i(&\{8n/3 - 2\ell-t-2\ell^*+1, \ldots, 8n/3 + \ell\} \\
        &\cup \{t+1, \ldots, \ell^*\} \\
        &\cup \{2n-\ell^*+1, \ldots, 2n-t\}) \leq 2\ell^* + \ell -t.
    \end{align*}
\end{lemma}
The items considered in Lemma \ref{expowerful} are marked with blue in Figure \ref{colorful}. 
First, we prove that the goods mentioned in Lemma \ref{expowerful} are distinct. To that end, it suffices to prove that $8n/3 - 2\ell-t-2\ell^*+1 > 2n-t$. It follows from the fact that $2n/3+\ell+\ell^* \leq n$. 
Before proving Lemma \ref{expowerful}, let us show how to obtain a contradiction assuming this lemma holds.
Note that since there are $n/3 - \ell - \ell^*$ bags with value less than $4/3-2x$ (namely the bags in $A^2$), it suffices to prove that there exists $3(\ell + \ell^*)$ other bags with total value $4(\ell + \ell^*)$. Since the remaining $2n/3 -2\ell -2\ell^*$ bags are of value at most $4/3+x$ (by Lemma \ref{save-all}), we get 
\begin{align}
    v_i(M) < (\frac{n}{3} - \ell - \ell^*) (\frac{4}{3} - 2x) + (\frac{2n}{3} - 2\ell - 2\ell^*)(\frac{4}{3} + x) + 4 (\ell + \ell^*) = \frac{4n}{3} \label{important-ineq}
\end{align}
which is a contradiction since $v_i(M)=4n/3$.

Now consider $B = \{\hat{B}_1, \ldots, \hat{B}_{2\ell + t + 2\ell^* -2n/3}, \hat{B}_{t+1}, \ldots, \hat{B}_{\ell^*} \} \cup \hat{A}^1$ where $\hat{A}^1$ is the set of bags in $A^1$ after bag-filling. $B$ consists of $3(\ell+\ell^*)$ bags. Now we prove that $v_i(\bigcup_{B_j \in B}B_j) \leq 4(\ell+\ell^*)$. We have 
%\jnote{I think we cannot say the exact items in these bags but can argue about their values.}
\begin{align*}
    v_i(\bigcup_{\hat{B}_j \in B}\hat{B}_j) \leq v_i(&\bigcup_{B_j \in A^1} B_j) \\
    + v_i(\textcolor{red}{\{}&\textcolor{red}{1, \ldots, 2\ell + t + 2\ell^* -2n/3\}})\\
    + v_i(\textcolor{blue}{\{}&\textcolor{blue}{8n/3 - 2\ell-t-2\ell^*+1, \ldots, 8n/3 + \ell\}} \\
     &\textcolor{blue}{\cup \{t+1, \ldots, \ell^*\}} \\
     &\textcolor{blue}{\cup \{2n-\ell^*+1, \ldots, 2n-t\}}) .
\end{align*}
We bound the value of the goods marked with different colors in different inequalities.  
\begin{observation}
    For all $B_j \in A^1$, $v_i(B_j) < 1$.
\end{observation}
Since $|A^1| = 2n/3 + \ell$, 
\begin{align*}
    v_i(\bigcup_{B_j \in A^1} B_j) < 2n/3 +\ell.
\end{align*}
Also, since all goods are of value at most $1$ to agent $i$,

\begin{align*}
    \textcolor{red}{v_i(\{1, \ldots, 2\ell + t + 2\ell^* -2n/3\}) \leq 2\ell + t + 2\ell^* -2n/3}.
\end{align*}
By Lemma \ref{expowerful}, 
\textcolor{blue}{
\begin{align*}
        v_i(&\{8n/3 - 2\ell-t-2\ell^*+1, \ldots, 8n/3 + \ell\} \\
        &\cup \{t+1, \ldots, \ell^*\} \\
        &\cup \{2n-\ell^*+1, \ldots, 2n-t\}) \leq 2\ell^* + \ell -t.
\end{align*}}

By adding all the inequalities, we get
\begin{align*}
    v_i(\bigcup_{\hat{B}_j \in B}\hat{B}_j) \leq 4(\ell+\ell^*). 
\end{align*}
Hence, Inequality \eqref{important-ineq} holds, which is a contradiction. So the case of $v_i(2n-\ell^*) \geq 1/3$ cannot arise.
\contradictTwo*


In the rest of this section, we prove Lemma \ref{expowerful}.

\subsubsection{Proof of Lemma \ref{expowerful}}

To prove Lemma \ref{expowerful}, we partition the goods considered in this lemma into two parts. These parts are colored red and blue in Figure \ref{colorful-1}. We bound the value of red goods in Lemma~\ref{k-t}, i.e., $$\sum_{2n-\ell^* < j\leq 2n-t}v_i(j) + \sum_{8n/3 - 2\ell-t-2\ell^* < j \leq 8n/3 - 2\ell - 2t - \ell^*}v_i(j) < \ell^*-t,$$ and the value of the blue goods in Lemma~\ref{k+l}, i.e., $$\sum_{t < j \leq \ell^*} v_i(j) + \sum_{8n/3 - 2\ell - 2t - \ell^* < j \leq 8n/3 + \ell}v_i(j) \leq \ell^* + \ell.$$
Thereafter, we have
\begin{align*}
    v_i(&\{8n/3 - 2\ell-t-2\ell^*+1, \ldots, 8n/3 + \ell\} \\
    &\cup \{t+1, \ldots, \ell^*\} \\
    &\cup \{2n-\ell^*+1, \ldots, 2n-t\}) \\
    &= \textcolor{red}{\sum_{2n-\ell^* < j\leq 2n-t}v_i(j) + \sum_{8n/3 - 2\ell-t-2\ell^* < j \leq 8n/3 - 2\ell - 2t - \ell^*}v_i(j)} \\
    &+ \textcolor{blue}{\sum_{t < j \leq \ell^*} v_i(j) + \sum_{8n/3 - 2\ell - 2t - \ell^* < j \leq 8n/3 + \ell}v_i(j)} \\
    &< (\ell^*-t) + (\ell^*+\ell) &\tag{Lemma \ref{k-t} and \ref{k+l}}\\
    &= 2\ell^*+\ell-t,
\end{align*}
and Lemma \ref{expowerful} follows.

It suffices to prove Lemmas \ref{k-t} and \ref{k+l}. In the rest of this section, we prove these two lemmas.
% Figure environment removed
Limit the items in a $1$-out-of-$4n/3$ MMS partition of agent $i$ to $\{1, \ldots, 8n/3+\ell\}$ and let $R$ be the set of the resulting bags. Formally, for all $j \in [4n/3]$, $R_j = P_j^{i} \cap \{1, \ldots, 8n/3+\ell\}$ and $R = \{R_1, \ldots, R_{4n/3}\}$. Without loss of generality, assume $|R_1| \geq |R_2| \geq \ldots \geq |R_{4n/3}|$. Let $t$ be the number of bags of size $1$ in $R$. 

\begin{lemma}\label{powerful}
    If there exist $t$ bags of size at most $1$ in $R$, then 
    $$\sum_{1 \leq j \leq t+\ell}|R_j| \geq 3(t+\ell).$$
\end{lemma}
\begin{proof}
    Since $R_j$'s are sorted in decreasing order of their size, 
    \begin{align*}
        \sum_{1 \leq j \leq t+\ell}|R_j| \geq (t+\ell)|R_{t+\ell}|.
    \end{align*}
    Hence, if $|R_{t+\ell}| \geq 3$, then $\sum_{1 \leq j \leq t+\ell}|R_j| \geq 3(t+\ell).$ So assume $|R_{t+\ell}| \leq 2$. 
    \begin{align*}
        \frac{8n}{3} + \ell &= \sum_{1 \leq j \leq 4n/3}|R_j| \\
        &= \sum_{1 \leq j \leq t+\ell} |R_j| + \sum_{t+\ell < j \leq 4n/3 - t}|R_j| + \sum_{4n/3-t < j \leq 4n/3}|R_j| \\
        &\leq \sum_{1 \leq j \leq t+\ell} |R_j| +  (\frac{4n}{3} - 2t - \ell)|R_{t+\ell}| + t \\
        &\leq \sum_{1 \leq j \leq t+\ell} |R_j| + 2(\frac{4n}{3} - 2t - \ell) + t 
    \end{align*}
    Therefore, $$\sum_{j \in [t+\ell]} |R_j| \geq 3(\ell+t).$$
\end{proof}
\begin{lemma}\label{bound}
    $\ell+\ell^*+t \leq 4n/3$.
\end{lemma}
\begin{proof}
    We have $\ell^* + 2n/3 + \ell \leq s \leq n$. See Figure \ref{colorful} for intuition. Therefore, $\ell^* + \ell \leq n/3$. Also, $t \leq \ell^* \leq n$. Hence $\ell+\ell^*+t \leq 4n/3$.
\end{proof}
\begin{lemma}\label{k-t}
    \textcolor{red}{
    $$\sum_{2n-\ell^* < j\leq 2n-t}v_i(j) + \sum_{8n/3 - 2\ell-t-2\ell^* < j \leq 8n/3 - 2\ell - 2t - \ell^*}v_i(j) < \ell^*-t.$$
    }
\end{lemma}
\begin{proof}
    Let $B' = \{2n-\ell^*+1, \ldots, 2n-t\} \cup \{8n/3 - 2\ell-t-2\ell^*+1, \ldots, 8n/3 - 2\ell - 2t - \ell^*\}$. $|B'| = 2(\ell^*-t)$ and by Observation \ref{half-bound} for all goods $g \in B'$, $v_i(g) < 1/2$. Therefore, $v_i(B') < \ell^*-t$.
\end{proof}
\begin{lemma}\label{k+l}
    \textcolor{blue}{
    $$\sum_{t < j \leq \ell^*} v_i(j) + \sum_{8n/3 - 2\ell - 2t - \ell^* < j \leq 8n/3 + \ell}v_i(j) \leq \ell^* + \ell.$$
    }
\end{lemma}
\begin{proof}
    Recall that $\{R_1, \ldots, R_{4n/3}\}$ is the set of bags in the $1$-out-of-$4n/3$ MMS partition of agent $i$ after removing items $\{8n/3+\ell+1, \ldots, m\}$. Moreover, we know exactly $t$ of these bags have size $1$. If there is a bag $R_j = \{g\}$ for $g>t$, there must be a good $g' \in [t]$ such that $g' \in R_{j'}$ and $|R_{j'}|>1$. Swap the goods $g$ and $g'$ between $R_j$ and $R_{j'}$ as long as such good $g$ exists. Note that $v_i(R_{j'})$ can only decrease and $v_i(R_j)=v_i(g') \leq 1$. Therefore, in the end of this process for all $j \in [4n/3]$, $v_i(R_j) \leq 1$ and we can assume bags containing goods $1, \ldots, t$ are of size $1$ and bags containing goods $t+1, \ldots, \ell^*$ %\jnote{here what is $k$? should it be $8n/3 + \ell$?} 
    are of a size more than $1$. Recall that $|R_1| \geq \ldots \geq |R_{4n/3}|$. Let $T_{j}$ be the bag that contains good $j$. 
    
    Consider the bags $B=\{R_1, \ldots, R_{t+\ell}\} \cup \{T_{t+1}, \ldots, T_{\ell^*}\}$. If $|B| < \ell^*+\ell$, keep adding a bag with the largest number of items to $B$ until there are exactly $\ell^*+\ell$ bags in $B$. First we show that $B$ contains at least $3\ell + 2\ell^* + t$ goods. Namely,
    $$\sum_{S \in B} |S| \geq 3\ell + 2\ell^* + t.$$
    By Lemma \ref{powerful}, $\sum_{1 \leq j \leq t+\ell} |R_j| \geq 3(t+\ell)$. If all the remaining $\ell^*-t$ bags in $B \setminus \{R_1, \ldots, R_{t+\ell}\}$ are of size $2$, then $\sum_{S \in B} |S| \geq 3(t+\ell)+2(\ell^*-t) = 3\ell + 2\ell^* +t$. Otherwise, there is a bag in $B$ of size at most $1$; hence, all bags outside $B$ are also of size at most $1$. So we have
    \begin{align*}
        \frac{8n}{3} + \ell &= \sum_{S \in B} |S| + \sum_{S \notin B} |S| \\
        &\leq \sum_{S \in B} |S| + (\frac{4n}{3} - \ell^* - \ell).
    \end{align*}
    Therefore, 
    \begin{align*}
        \sum_{S \in B} |S| &\geq 4n/3 + 2\ell + \ell^* \\
        &\geq 3\ell + 2\ell^* +t. &\tag{Lemma \ref{bound}}
    \end{align*}
    Note that the goods $\{t+1, \ldots, \ell^*\}$ are contained in $B$ and moreover, $B$ contains at least $3\ell + 2\ell^* +t - (\ell^*-t) = 3\ell + 2t + \ell^*$ other goods. Therefore,
    \begin{align*}
        \ell^*+\ell &\geq \sum_{S \in B} v_i(S) \\
        &\geq \sum_{t < j \leq \ell^*} v_i(j) + \sum_{8n/3 - 2\ell - 2t - \ell^* < j \leq 8n/3 + \ell} v_i(j).
    \end{align*}
The last inequality follows because we used the $3\ell + 2t + \ell^*$ lowest valued goods in $[8n/3+\ell]$. 
\end{proof}    

\subsection{\boldmath $\mathbf{v_i(2n-\ell^*)<1/3}$}\label{negative}
Let $r^*$ be largest such that $v_i(B_{r^*})<1$. That is, $B_{r*}$ is the rightmost bag in Figure \ref{fig:bags} with a value less than $1$ to agent $i$. 
\begin{lemma}\label{r-star-bound}
    If $v_i(2n-r^*+1) \leq 1/3$, then $r^* < 2n/3$.
\end{lemma}
\begin{proof}
    Since $1>v_i(B_{r^*})=v_i(r^*)+v_i(2n-r^*+1)$, we have $v_i(r^*)<2/3+x$. 
    By Observation \ref{upper-bound-left}, for all $j \leq r^*$, $v_i(\hat{B}_j) \leq 4/3-x$.
    Also, by Observation \ref{upper-bound-right}, for all $j>r^*$, $v_i(\hat{B}_j)<4/3+2x$.
    Hence, we have 
    \begin{align*}
        \frac{4n}{3} &= v_i(M) \\
        &= \sum_{j \leq r^*} v_i(\hat{B}_j) + \sum_{j > r^*} v_i(\hat{B}_j) \\
        &< r^*(\frac{4}{3}-x) + (n-r^*)(\frac{4}{3}+2x) \\%&\tag{Corollary \ref{cor-r-star} and \ref{claim-r-star-1}}\\
        &= \frac{4n}{3} + x(2n-3r^*).
    \end{align*}
    Therefore, $r^* < 2n/3$.
\end{proof}
\begin{lemma}
    $v_i(2n-r^*+1) > 1/3$.
\end{lemma}
\begin{proof}
    Towards contradiction, assume $v_i(2n-r^*+1) = 1/3-x$ for $x \geq 0$. By Lemma \ref{r-star-bound}, $r^* < 2n/3$.
    \begin{claim}\label{claim-r-star}
        $\sum_{j > r^*} v_i(\hat{B}_j) < \frac{10n}{9}-r^* + \frac{2nx}{3}$.
    \end{claim}
    \begin{claimproof}
    Note that by the definition of $r^*$, for all $j > r^*$, $\hat{B}_j = B_j$.
    By Lemma \ref{lem:Csum}, $v_i(\{2n/3+r^*+1, \ldots, 2n-r^*\}) \leq 2n/3-r^*$. Also since $v_i(r^*) < 2/3+x$, $v_i(\{r^*+1, \ldots, 2n/3+r^*\}) \leq \frac{2n}{3}(\frac{2}{3}+x)$. In total, we get
    \begin{align*}
        \sum_{j > r^*} v_i(\hat{B}_j) &= \sum_{j > r^*} v_i(B_j) \\
        &= v_i(\{r^*+1, \ldots, 2n/3+r^*\}) + v_i(\{2n/3 +r^*+1, \ldots, 2n-r^*\}) \\
        &< \frac{2n}{3}(\frac{2}{3}+x) + \frac{2n}{3}-r^* \\
        &= \frac{10n}{9}-r^* + \frac{2nx}{3}.
    \end{align*}
    Therefore, Claim \ref{claim-r-star} holds. 
    \end{claimproof}
    We have 
    \begin{align*}
        \frac{4n}{3} &= v_i(M) \\
        &= \sum_{j \leq r^*} v_i(\hat{B}_j) + \sum_{j > r^*} v_i(\hat{B}_j) \\
        &< r^*(\frac{4}{3} -x) + \frac{10n}{9}-r^* + \frac{2nx}{3} &\tag{Observation \ref{upper-bound-left} and Claim \ref{claim-r-star}}\\
        &= r^*(\frac{1}{3}-x) + \frac{10n}{9} + \frac{2nx}{3}.
    \end{align*}
    Thus, 
    \begin{align*}
        \frac{2n}{9} &< r^*(\frac{1}{3}-x) + \frac{2n}{3}(x) \\
        &\leq \frac{2n}{3} \cdot \frac{1}{3}, \tag{$r^* \leq 2n/3$ by Lemma \ref{r-star-bound}}
    \end{align*}
    which is a contradiction. Hence, $v_i(2n-r^*+1) > 1/3$.
\end{proof}

Recall that $\ell^*$ be the smallest such that $v_i(B_{\ell^*+1}) < 1$, i.e., $B_{\ell^*+1}$ is the leftmost bag in Figure \ref{fig:bags} with value less than $1$ to agent $i$. Let $\ell$ be largest such that $v_i(B_\ell)<1$ and $v_i(2n- \ell +1) \leq 1/3$. Since $v_i(B_{\ell^*+1})<1$ and $v_i(2n- \ell^*) < 1/3$, such an index exists and $\ell \ge \ell^*+1$. Also, let $r$ be smallest such that $v_i(B_{r+1})<1$ and $v_i(2n-r) \geq 1/3$. Again, since $v_i(B_{r^*})<1$ and $v_i(2n- r^*+1) > 1/3$, such an index exists. We set $x := 1/3 - v_i(2n- \ell +1)$ and $y := v_i(2n-r) - 1/3$. See Figure \ref{r-l-fig}.
\begin{observation}\label{lessThan13}
    $x < 1/3$.
\end{observation}
\begin{proof}
    Towards a contradiction, assume $x=1/3$. Therefore, $v_i(2n-\ell+1)=0$. Let $k < 2n-\ell+1$ be the number of goods with a value larger than $0$ to agent $i$. Consider $(P^i_1 \cap [k], \ldots, P^i_{4n/3} \cap [k])$. There are at least $\ell$ many indices $j$ such that, $|P^i_j \cap [k]|=1$. Since $\mathcal{I}$ is $4n/3$-normalized, $v_i(1) = \ldots = v_i(\ell)=1$ which is a contradiction with $v_i(B_{\ell^*+1})<1$.
\end{proof}

\begin{observation}\label{y-bound}
    $y< 1/6$.
\end{observation}
\begin{proof}
    We have $1/3 + y = v_i(2n-r) \leq v_i(B_r)/2 < 1/2$. Thus, $y<1/6$.
\end{proof}
% Figure environment removed
\begin{corollary}[of Observation \ref{upper-bound-left}]\label{cor-upper-left}
    For all $j \leq \ell$, $v_i(\hat{B}_j) \leq 4/3-x$.    
\end{corollary}
\begin{corollary}[of Observation \ref{upper-bound-right}]\label{cor-upper-right}
    For all $j>r$, $v_i(\hat{B}_j) \leq \max(4/3-x, 4/3-2y)$.
\end{corollary}
\begin{observation}\label{middle-block}
    For all $\ell < j \leq r$, $1 \leq v_i(\hat{B}_j) < 1+x+y$.
\end{observation}
\begin{proof}
    Note that by definition of $\ell$ and $r$, for all $\ell < j \leq r$, $v_i(B_j) \geq 1$. Therefore, $\hat{B}_j=B_j$. Also, 
    \begin{align*}
        v_i(B_j) &= v_i(j) + v_i(2n-j+1) \\
        &\leq v_i(\ell) + v_i(2n-r) \tag{$\ell < j$ and $2n-r < 2n-j+1$}\\
        &< (\frac{2}{3}+x) + (\frac{1}{3}+y) \tag{$v_i(B_\ell)<1$ and $v_i(B_{r+1})<1$}\\
        &= 1+x+y.
    \end{align*}
\end{proof}
\begin{lemma}
    $r-\ell > 2n/3$.
\end{lemma}
\begin{proof}
    If $x+y \leq 1/3$, then by Corollaries \ref{cor-upper-left} and \ref{cor-upper-right} and Observation \ref{middle-block}, for all $t \in [n]$ we have $v_i(\hat{B}_t) \leq 4/3$ and for at least one bag this value is less than $1$ by Observation \ref{less-than-one}. Therefore, $v_i(M) < 4n/3$, which is a contradiction. Thus, $x+y>1/3$. 
    We have
    \begin{align*}
        \frac{4n}{3} &= v_i(M) \\
        &= \sum_{j \leq \ell} v_i(\hat{B}_j) + \sum_{\ell < j \leq r} v_i(\hat{B}_j) + \sum_{j > r} v_i(\hat{B}_j) \\
        &\leq \ell(\frac{4}{3}-x) + (r-\ell)(1+x+y) + (n-r) \max(\frac{4}{3}-x, \frac{4}{3}-2y) \\
        &\leq (r-\ell)(1+x+y) + (n-r+\ell) \max(\frac{4}{3}-x, \frac{4}{3}-2y) \tag{Corollaries \ref{cor-upper-left} and \ref{cor-upper-right} and Observation \ref{middle-block}}\\
        &= \frac{4n}{3} + (r-\ell)(x+y-\frac{1}{3}) - (n-r+\ell)\min(x,2y).
    \end{align*}
    Therefore, $(r-\ell)(x+y-1/3) \geq (n-r+\ell)\min(x,2y)$. By Observation \ref{lessThan13}, $x<1/3$ and thus, we have $x+y-1/3 < y$. Also, since $y < 1/6$ (by Observation \ref{y-bound}), we have $x+y-1/3 < x-1/6 < x/2$. Thus, $x+y-1/3 < \min(x,2y)/2$. Hence, $r-\ell > 2(n-r+\ell)$ and therefore, $r-\ell > 2n/3$.
\end{proof}
Let $r - \ell = 2n/3 + s$. Recall that $P^{i} = (P^{i}_1, \ldots, P^{i}_{4n/3})$ is an $(4n/3)$-MMS partition of $M$ for agent $i$. Since $i$ is fixed, we use $P = (P_1, \ldots, P_{4n/3})$ instead for ease of notation. For all $j \in [4n/3]$, let $g_j$ be good with the smallest index (and hence the largest value) in $P_j$. Without loss of generality, assume $g_1 < g_2 < \ldots < g_{4n/3}$. Observe that $\{1,\dots,r\} \subseteq \cup_{k\in [r]} P_k$. Let $S'$ be the set of goods in $\{r+1, \ldots, 2n-\ell\}$ that appear in the first $r$ bags in $P$. Formally, $S' = \{g \in \{r+1, \ldots, 2n-\ell\} \mid g \in \cup_{j \in [r]} P_j\}$. Let $s':=\min(|S'|,s)$. 
% Figure environment removed
\begin{restatable}{lemma}{difficultbound} \label{difficult-bound}
    $v_i(\{r-s'+1, \ldots, r\} \cup \{2n-\ell-3s+2s'+1, \ldots, 2n-\ell\}) \leq s$.
\end{restatable}
The items considered in Lemma \ref{difficult-bound} are marked with blue in Figure \ref{colorful-11}.
Before proving Lemma \ref{difficult-bound}, let us assume it holds and reach a contradiction. Since $v_i(\ell) < 1-v_i(2n-\ell+1)= 2/3+x$, we have
\begin{align}
    v_i(\{\ell+1, \ldots, r-s'\}) &< (\frac{2n}{3}+s-s')(\frac{2}{3}+x). \label{easy-1}
\end{align}
Also, since $v_i(2n-r+1) =1/3+y$,
\begin{align}
    v_i(\{2n-r+1, \ldots, 2n-\ell-3s+2s'\}) &\leq (\frac{2n}{3}-2s+2s')(\frac{1}{3}+y). \label{easy-2}
\end{align}
Therefore,
\begin{align*}
    \sum_{\ell < j \leq r} v_i(\hat{B}_j) &= \sum_{\ell < j \leq r} v_i(B_j) \\
    &= v_i(\{\ell+1, \ldots, r\} \cup \{2n-r+1, \ldots, 2n-\ell\}) \\
    &= v_i(\{\ell+1, \ldots, r-s'\}) \\
    &\indent + v_i(\{r-s'+1, \ldots, r\} \cup \{2n-\ell-3s+2s'+1, \ldots, 2n-\ell\}) \\ 
    &\indent + v_i(\{2n-r+1, \ldots, 2n-\ell-3s+2s'\}) \\
    &< (\frac{2n}{3}+s-s')(\frac{2}{3}+x) + s + (\frac{2n}{3}-2s+2s')(\frac{1}{3}+y) \tag{Inequalities \eqref{easy-1} and \eqref{easy-2} and Lemma \ref{difficult-bound}}\\
    &= \frac{2n}{3}(1+x+y)+(s-s')(x-2y)+s.
\end{align*}
Thus,
\begin{align*}
    \frac{4n}{3} &= v_i(M) \\
    &= \sum_{j \leq \ell} v_i(\hat{B}_j) + \sum_{\ell < j \leq r} v_i(\hat{B}_j) + \sum_{j>r} v_i(\hat{B}_j) \\
    &< (\ell+n-r)\max(\frac{4}{3}-x, \frac{4}{3}-2y) + \frac{2n}{3}(1+x+y)+(s-s')(x-2y)+s \tag{Corollaries \ref{cor-upper-left} and \ref{cor-upper-right}} \\
    &= (\frac{n}{3}-s)\max(\frac{4}{3}-x, \frac{4}{3}-2y) + \frac{2n}{3}(1+x+y)+(s-s')(x-2y)+s.
\end{align*}
If $x \leq 2y$, then by replacing $\max(4/3-x, 4/3-2y)$ with $4/3-x$ in the above inequality, we get
\begin{align*}
    \frac{4n}{3} &< (\frac{n}{3}-s)(\frac{4}{3}-x) + \frac{2n}{3}(1+x+y)+(s-s')(x-2y)+s \\
    &\leq \frac{n}{3}(\frac{4}{3}-x) + \frac{2n}{3}(1+x+y)+(s-s')(x-2y) \tag{$4/3-x \geq 1$}\\
    &\leq \frac{n}{3}(\frac{10}{3}+x+2y) \tag{$(s-s')(x-2y) \leq 0$} \\
    &< \frac{4n}{3}, \tag{$x \leq 1/3$ and $y<1/6$}
\end{align*}
which is a contradiction. If $2y < x$, by replacing $\max(4/3-x, 4/3-2y)$ with $4/3-2y$, we get 
\begin{align*}
    \frac{4n}{3} &< (\frac{n}{3}-s)(\frac{4}{3}-2y) + \frac{2n}{3}(1+x+y)+(s-s')(x-2y)+s \\
    &= \frac{n}{3}(\frac{10}{3}+2x) - s(\frac{1}{3}-x) - s'(x-2y) \\
    &\leq \frac{n}{3}(\frac{10}{3}+2x) &\tag{$x \leq 1/3$ and $x>2y$} \\
    &\leq \frac{4n}{3}, \tag{$x \leq 1/3$}
\end{align*} 
which is again a contradiction. Therefore, it is not possible that $v_i(2n-\ell^*) < 1/3$. Thus, Theorem \ref{contradict-1} follows.
\contradictOne*

It only remains to prove Lemma \ref{difficult-bound}. The main idea is as follows. Recall that $s'=\min(|S'|,s)$. We consider two cases for $s'$. If $s' = s$, then in order to prove Lemma \ref{difficult-bound}, we must prove $$v_i(\{r-s'+1, \ldots, r\} \cup \{2n-\ell - s'+1, \ldots, 2n-\ell\}) \leq s',$$
    which is what we do in Claim \ref{claim-2}.
    In case $s' = |S'|$, we prove 
    $$v_i(\{r-s'+1, \ldots, r\}) + v_i(S') \leq s'$$ 
    in Claim \ref{claim-22} and 
    $$v_i (\{2n-\ell-3s+2s'+1, \ldots, 2n-\ell\}) - v_i(S') \leq s-s'$$ 
    in Claim \ref{claim-3}. Adding the two sides of the inequalities implies Lemma \ref{difficult-bound}. We prove this lemma in Section \ref{proof-sec-1}.
    
\subsubsection{Proof of Lemma \ref{difficult-bound}}\label{proof-sec-1}
\difficultbound*
    Note that $\{1, \ldots, r\} \cup S' \subseteq P_1 \cup \ldots \cup P_r$. 
    For $j \in [r]$, let $Q_j = P_j \cap (\{1, \ldots, r\} \cup S')$. We begin with proving the following claim. %\ref{claim-1}.
    \begin{claim}\label{claim-1}
        There are $s'$ many sets like $Q_{j_1}, \ldots, Q_{j_{s'}}$ such that $|\cup_{k \in [s']} Q_{j_k}| \geq 2s'$ and $|\cup_{k \in [s']} Q_{j_k} \cap \{1, \ldots, r\}| \geq s'$. 
    \end{claim}
    \begin{proof}
        If $s'=0$, the claim trivially holds. Thus, assume $s' \geq 1$. By induction, we prove that for any $t \leq s'$, there are $t$ many sets like $Q_{j_1}, \ldots, Q_{j_t}$ such that $|\cup_{k \in [t]} Q_{j_k}| \geq 2t$ and $|\cup_{k \in [t]} Q_{j_k} \cap \{1, \ldots, r\}| \geq t$. 
        \paragraph{\boldmath Induction basis: $t=1$.} If there exists $Q_k$ such that $|Q_k \cap \{1, \ldots, r\}| \geq 2$, let $j_1 = k$. Otherwise, for all $k \in [r]$, we have $|Q_k \cap \{1, \ldots, r\}| = 1$. Since $s' \geq 1$, there must be an index $k$ such that $|Q_k \cap S'| \geq 1$. Let $j_1=k$.
            
        \paragraph{\boldmath Induction assumption:} There are $t$ many sets like $Q_{j_1}, \ldots, Q_{j_t}$ such that $|\cup_{k \in [t]} Q_{j_k}| \geq 2t$ and $|\cup_{k \in [t]} Q_{j_k} \cap \{1, \ldots, r\}| \geq t$. 

        Now for $t+1 \leq s'$, we prove that there are $t+1$ many sets like $Q_{j_1}, \ldots, Q_{j_{t+1}}$ such that $|\cup_{k \in [t+1]} Q_{j_k}| \geq 2t+2$ and $|\cup_{k \in [t+1]} Q_{j_k} \cap \{1, \ldots, r\}| \geq t+1$. 
        \paragraph{\boldmath Case 1: $|\cup_{k \in [t]} Q_{j_k}| \geq 2t+2$:} If $|\cup_{k \in [t]} Q_{j_k} \cap \{1, \ldots, r\}| \geq t+1$, set $j_{t+1}=k$ for an arbitrary $k \in [r] \setminus \{j_1, \ldots, j_t\}$. Otherwise, set $j_{t+1}=k$ for an index $k \in [r] \setminus \{j_1, \ldots, j_t\}$ such that $|Q_{k} \cap \{1, \dots, r\}| \ge 1$.
        \paragraph{\boldmath Case 2: $|\cup_{k \in [t]} Q_{j_k}| = 2t+1$:} If there exists $k \in [r] \setminus \{j_1, \ldots, j_t\}$, such that $|Q_k \cap [r]| \geq 1$, set $j_{t+1}=k$. Otherwise, set $j_{t+1}=k$ for any $k \in [r] \setminus \{j_1, \ldots, j_t\}$ such that $|Q_k| \geq 1$. Since $|\cup_{j \in [r]} Q_j| \geq r+s' > 2t+1$, such $k$ exists.
        \paragraph{\boldmath Case 3. $|\cup_{k \in [t]} Q_{j_k}| = 2t$ and $|\cup_{k \in [t]} Q_{j_k} \cap \{1, \ldots, r\}| \geq t+1$:} $|\cup_{k \in [r] \setminus \{j_1, \ldots, j_t\}} Q_{j_k}| \geq r+s'-2t > r-t$. Therefore, by pigeonhole principle, there exists an index $k \in [r] \setminus \{j_1, \ldots, j_t\}$ such that $|Q_k| \geq 2$. Set $j_{t+1}=k$.
        \paragraph{\boldmath Case 4. $|\cup_{k \in [t]} Q_{j_k}| = 2t$ and $|\cup_{k \in [t]} Q_{j_k} \cap \{1, \ldots, r\}| = t$:} If there exists $k \in [r] \setminus \{j_1, \ldots, j_t\}$, such that $|Q_k \cap [r]| \geq 2$, set $j_{t+1}=k$. Otherwise, for all $k \in [r] \setminus \{j_1, \ldots, j_t\}$, $|Q_k \cap [r]|=1$ since $|\cup_{k \in [t]} Q_{j_k} \cap \{1, \ldots, r\}| = t$ and $|\cup_{k \in [r]} Q_{j_k} \cap \{1, \ldots, r\}| = r$. Set $j_{t+1}=k$ for any $k \in [r] \setminus \{j_1, \ldots, j_t\}$, such that $|Q_k \cap S'| \geq 1$. Since $|\cup_{j \in [r]} Q_j \cap S'| \geq s' > t$, such $k$ exists.
    \end{proof}
Now we prove Claim \ref{claim-2}.
\begin{claim}\label{claim-2}
    $v_i(\{r-s'+1, \ldots, r\} \cup \{2n-\ell - s'+1, \ldots, 2n-\ell\}) \leq s'$.
\end{claim}
\begin{proof}
    Let $Q^1$ be the set of $s'$ most valuable goods in $\cup_{k \in [s']} Q_{j_k}$ and let $Q^2$ be the set of $s'$ least valuable goods in $\cup_{k \in [s']} Q_{j_k}$. Since $|\cup_{k \in [s']} Q_{j_k}| \geq 2s'$, $Q^1 \cap Q^2 = \emptyset$. Also, $|\cup_{k \in [s']} Q_{j_k} \cap \{1, \ldots, r\}| \geq s'$. Thus, $v_i(Q^1) \geq v_i(\{r-s'+1, \ldots, r\})$. Moreover, $v_i(Q^2) \geq v_i(\{2n-\ell - s'+1, \ldots, 2n-\ell\})$. Hence,
    \begin{align*}
        s' &= \sum_{k \in [s']} v_i(P_{j_k}) \notag\\
        &\geq \sum_{k \in [s']} v_i(Q_{j_k}) \notag \\
        &\geq v_i(\{r-s'+1, \ldots, r\} \cup \{2n-\ell - s'+1, \ldots, 2n-\ell\}). 
    \end{align*}
\end{proof}
Note that in case $s' = s$, Claim \ref{claim-2} implies Lemma \ref{difficult-bound}. Therefore, from now on, we assume $s' = |S'| < s$.
\begin{claim}\label{claim-22}
    $v_i(\{r-s'+1, \ldots, r\}) + v_i(S') \leq s'$.
\end{claim}
\begin{proof}
    The proof is similar to the proof of Claim \ref{claim-2}. Let $Q^1$ be the set of $s'$ most valuable goods in $\cup_{k \in [s']} Q_{j_k}$ and let $Q^2$ be the set of $s'$ least valuable goods in $\cup_{k \in [s']} Q_{j_k}$. Since $|\cup_{k \in [s']} Q_{j_k}| \geq 2s'$, $Q^1 \cap Q^2 = \emptyset$. Also, $|\cup_{k \in [s']} Q_{j_k} \cap \{1, \ldots, r\}| \geq s'$. Thus, $v_i(Q^1) \geq v_i(\{r-s'+1, \ldots, r\})$. Moreover, $v_i(Q^2) \geq v_i(S')$ since $s' = |S'|$. Hence,
    \begin{align*}
        s' &= \sum_{k \in [s']} v_i(P_{j_k}) \notag\\
        &\geq \sum_{k \in [s']} v_i(Q_{j_k}) \notag \\
        &\geq v_i(\{r-s'+1, \ldots, r\} \cup S')\\
        &= v_i(\{r-s'+1, \ldots, r\}) + v_i(S').
    \end{align*}
\end{proof}
\begin{claim}\label{claim-3}
    $v_i (\{2n-\ell-3s+2s'+1, \ldots, 2n-\ell\}) - v_i(S') \leq s-s'.$
\end{claim}
\begin{proof}
    Note that by definition of $S'$, the $2n-\ell-r-s'=8n/3-2r+s-s'$ goods in $\{r+1, \ldots, 2n-\ell\} \setminus S'$ are in $P_{r+1} \cup \ldots \cup P_{4n/3}$. Now for $j \in [4n/3 - r]$, let $R_j = P_{j+r} \cap \{r+1, \ldots, 2n-\ell\} \setminus S'$. Assume $|R_{j_1}| \geq \ldots \geq |R_{j_{4n/3-r}}|$. We prove 
    \begin{align}
        \sum_{k \leq s-s'} |R_{j_k}| \geq 3(s-s'). \label{R-bound}    
    \end{align}
    If $|R_{j_{s-s'+1}}| \geq 3$, Inequality \eqref{R-bound} holds. Otherwise, we have 
    \begin{align*}
        \frac{8n}{3}-2r+s-s' &= \sum_{k \in [4n/3-r]} |R_{j_k}| \\
        &= \sum_{k \leq s-s'} |R_{j_k}| + \sum_{s-s' < k \leq 4n/3-r} |R_{j_k}| \\
        &\leq \sum_{k \leq s-s'} |R_{j_k}| + 2(\frac{4n}{3}-r-s+s'). \tag{$|R_{j_k}| \leq 2$ for $k > s-s'$}
    \end{align*}
    Thus, $\sum_{k \in [s-s']} |R_{j_k}| \geq 3(s-s')$. We have
    \begin{align*}
        s-s' &= \sum_{k \in [s-s']} v_i(P_{j_k+r}) \notag\\
        &\geq \sum_{k \in [s-s']} v_i(R_{j_k}) \notag\\
        &\geq v_i (\{2n-\ell-3s+2s'+1, \ldots, 2n-\ell\}) - v_i(S'). \tag{$\lvert \cup_{k \in [s-s']} R_{j_k} \rvert \geq 3(s-s')$ and $\lvert S' \rvert =s'$}
    \end{align*}
\end{proof}
Claims \ref{claim-22} and \ref{claim-3} imply Lemma \ref{difficult-bound}.

\paragraph{Recap:} To show that a $1$-out-of-$(4n/3)$ MMS allocation exists, it suffices to prove that we never run out of goods for bag-filling in Algorithm \ref{algo}. Towards contradiction, we assumed that the algorithm stops before agent $i$ receives a bundle. By Observation \ref{less-than-one}, a bag with a value less than $1$ for agent $i$ exists. Let $\ell^*$ be the smallest such that $v_i(B_{\ell^*+1}) < 1$. In Section \ref{negative}, we reached a contradiction assuming $v_i(2n-\ell^*)<1/3$ and proved Theorem \ref{contradict-1}. In Section \ref{positive}, we reached a contradiction assuming $v_i(2n-\ell^*) \geq 1/3$ and proved Theorem \ref{contradict-2}. Therefore, no such agent $i$ exists, and all agents receive a bag by the end of Algorithm \ref{algo}. Theorem \ref{thm:main} follows.


\section{\texorpdfstring{$(\alpha, \beta, \gamma)$}{(alpha, beta, gamma)}-MMS Allocation}\label{sec:or}

In this section, we show the existence of $(2(1-\beta)/\beta, \beta, 3/4)$-MMS allocation for any $3/4 < \beta < 1$. Using $\beta = \sqrt{3}/2$, this implies the existence of a randomized allocation that gives each agent at least $3/4$ times her MMS value (ex-post) and at least $(34\sqrt{3} - 48)/8\sqrt{3} > 0.785$ times her MMS value in expectation (ex-ante). 

Given an instance $\ins$, without loss of generality, we assume that $\I$ is $n$-normalized and ordered, which implies that $\MMS_i = 1,\forall i\in N$. Since our approach is an extension of the Garg-Taki (GT) algorithm~\cite{garg2021improved} for the existence of $3/4$-MMS allocation, we first summarize their algorithm. GT algorithm has two phases of \emph{valid reductions} and \emph{bag filling}. In a valid reduction, the instance is reduced by removing an agent $a$ and a subset $S$ of goods such that $a$'s value for $S$ is at least $3/4$, and the MMS values of the remaining agents do not decrease, i.e., $v_a(S) \ge 3/4$ and $\MMS_i \ge 1$ for each remaining agent in the reduced instance $(N\setminus\{a\}, M\setminus\{S\}, V\setminus\{v_a\})$.

The GT algorithm utilizes the simple valid reductions with the set of goods $S_1= \{1\}$ (i.e., the highest valued good), $S_2 = \{n, n+1\}$, $S_3 = \{2n-1, 2n, 2n+1\}$, and $S_4=\{1, 2n+1\}$, in the priority order of $S_1$, $S_2$, $S_3$, and $S_4$, i.e., $S_k$ is performed only when for all $j<k$, $S_j$ is not feasible. The following lemma (proof is in Appendix~\ref{app}) shows that for all $k\in[4]$, $S_k$ is a valid reduction if performed in the priority order. 

\begin{restatable}{lemma}{validred}\cite{garg2021improved}
Let $S$ be the lowest index bundle in $S\in\{S_1,S_2,S_3,S_4\}$ for which $\{i \in N: v_i(S) \geq 3/4 \}$ is non-empty. Then, removing $S$ and agent $a$ with $v_a(S) \ge 3/4$ is a valid reduction.
\end{restatable}

Let $\I'=([n'], [m'], V')$ be the instance after all the valid reductions are performed, i.e., no more valid reductions are feasible for $\I'$. This gives some information about the values of goods shown in the following corollary. 

\begin{corollary}\label{cor:val}
If no valid reductions are feasible for $\I'=([n'], [m'], V')$, then for any agent $i\in [n']$, $v_i(1) < 3/4$, $v_i(n'+1) < 3/8$, and $v_i(2n'+1) < 1/4$. 
\end{corollary}
\begin{algorithm}[t!]
\caption{The $(2(1-\beta)/\beta, \beta, 3/4)$-MMS Algorithm}\label{algo:or}
    \textbf{Input:} $n$-normalized ordered instance $\ins$, where $\MMS_i = 1, \forall i\in N$, and $\beta\in (3/4, 1)$\\
    \textbf{Output:} $(2(1-\beta)/\beta, \beta, 3/4)$-MMS allocation $A$
\begin{algorithmic}[1]
  \State $N_1 \gets$ an arbitrary set of agents s.t. $|N_1| \le 2(1-\beta)|N|/\beta$, with a target of at least $\beta$-MMS
  \State $N_2 \gets N\setminus N_1$, set of remaining agents ($|N_2| \ge (3\beta -2)|N|/\beta$), with a target of at least $3/4$-MMS
  \State $S_1\gets \{1\}$;\ \ $S_2\gets \{|N|, |N|+1\}$;\ \ $S_3\gets \{2|N|-1, 2|N|, 2|N|+1\}$;\ \ $S_4\gets\{1, 2|N|+1\}$
  \State $T_1 \gets \{i\in N_1: \exists k\in [4], v_i(S_k) \ge \beta\}$; \ \ $T_2 \gets \{i\in N_2: \exists k\in [4], v_i(S_k) \ge 3/4\}$
  \While{$T_1 \cup T_2 \neq \emptyset$} \Comment{Valid Reductions}
	   \State $\ell\gets$ smallest index s.t. either $v_i(S_\ell) \ge \beta$, for some $i\in N_1$ or $v_i(S_\ell) \ge 3/4$, for some $i\in N_2$
	   \State $a \gets$ agent in $N_1$ with the highest value for $S_\ell$
	   \If{$v_a(S_\ell) \ge \beta$} 
	      \State $A_a\gets S_\ell$;\ $N \gets N \setminus \{a\}$;\ $M\gets M\setminus S_\ell$
	   \Else 
	      \State $a\gets$ agent in $N_2$ such that $v_a(S_\ell) \ge 3/4$
	   	 \State $A_a\gets S_\ell$;\ $N \gets N \setminus \{a\}$;\ $M\gets M\setminus S_\ell$    
	   \EndIf
	   \State $S_1\gets \{1\}$;\ \ $S_2\gets \{|N|, |N|+1\}$;\ \ $S_3\gets \{2|N|-1, 2|N|, 2|N|+1\}$;\ \ $S_4\gets\{1, 2|N|+1\}$
	   \State $T_1 \gets \{i\in N_1: \exists k\in [4], v_i(S_k) \ge \beta\}$; \ \ $T_2 \gets \{i\in N_2: \exists k\in [4], v_i(S_k) \ge 3/4\}$
  \EndWhile
  \State $n'\gets$ $|N_1 \cup N_2|$
  \State $R\gets$ $M\setminus [2n']$
  \State Initialize $n'$ bags as in \eqref{eq:B_i}
  \For{each $B_k, k\in [n']$} \Comment{Bag Filling}
  	\State $T \gets \{i\in N_1: v_i(B_k) \ge \beta\} \cup \{i\in N_2: v_i(B_k) \ge 3/4\}$
  	\While{$T = \emptyset$}
	  \State $B_k \gets B_k \cup \{g\}, g\in R$;\ $R\gets R\setminus \{g\}$ 
       \State $T \gets \{i\in N_1: v_i(B_k) \ge \beta\} \cup \{i\in N_2: v_i(B_k) \ge 3/4\}$
	\EndWhile
	  \State $a \gets$ agent in $N_1$ with the highest value for $B_k$
	  \If{$v_a(B_k) \ge \beta$} 
	     \State $A_a\gets B_k$;\ $N_1 \gets N_1 \setminus \{a\}$
	  \Else 
	     \State $a\gets$ agent in $N_2$ such that $v_a(B_k) \ge 3/4$ \Comment{$a$ exists due to while condition}
	     \State $A_a\gets B_k$;\ $N_2 \gets N_2 \setminus \{a\}$
	  \EndIf
  \EndFor
\end{algorithmic}
\end{algorithm}


In the bag filling phase, $n'$ bags are initialized using the first $[2n']$ goods as in~\eqref{eq:B_i} and each bag is filled with goods in $[m']\setminus [2n']$ until some agent has a value at least $3/4$ for the bag. Although the GT algorithm is quite simple, the main challenge is showing there are enough goods in $[m']\setminus [2n']$ to satisfy each agent with a value of at least $3/4$.

Our algorithm is described in Algorithm~\ref{algo:or}. We start with an arbitrary set $N_1$ of agents such that $|N_1|\le 2n(1-\beta)/\beta$, and our goal is to satisfy each of them with a value of at least $\beta$. $N_2$ is the set of remaining agents, and our goal is to satisfy each of them with a value of at least $3/4$. Like the GT algorithm, Algorithm~\ref{algo:or} also has two phases, valid reductions and bag filling, albeit with some crucial differences. In the valid reduction phase, we use different targets for $N_1$ and $N_2$, but we prioritize agents in $N_1$. We first check if a valid reduction is feasible for an agent in $N_1 \cup N_2$ with some $S_k, k\in[4]$. If yes, we pick the smallest feasible index, say $S_\ell$, and find an agent, say $a$, in $N_1$ with the highest value for $S_\ell$. If this value is at least $\beta$, we assign $S_\ell$ to agent $a$. Otherwise, we assign $S_\ell$ to any agent in $N_2$ with a value of at least $3/4$. 

We run the bag-filling phase on the reduced instance if no valid reductions are feasible. This phase is similar to the GT algorithm except that we again use different targets for agents in $N_1$ and $N_2$ and prioritize agents in $N_1$. For all $i \in [n']$, let $B_i$ be the $i^{\text{th}}$ initial bag (i.e., $B_i=\{i,2n'-i+1\}$) and $\hat{B}_i$ be the bag at the end of the algorithm (i.e., after the bag-filling phase).

Although Algorithm~\ref{algo:or} is a simple extension of the GT algorithm, the analysis is not straightforward because $S_4$ is not a valid reduction for $N_1$. We first show that each agent in $N_2$ will receive a bag valued at least $3/4$ at the end of the algorithm. 

%The next two lemmas show that all agents are satisfied with their respective targets, proving the algorithm's correctness.

\begin{lemma}\label{lem:N2}
Each agent in $N_2$ receives a bag valued at least $3/4$.
\end{lemma}

\begin{proof}
It easily follows from the GT algorithm because valid reductions do not decrease the MMS value of any remaining agent in $N_2$, and the bag filling does not add goods from $[m']\setminus [2n']$ to a bag when some agent in $N_2$ has a value at least $3/4$. 
\end{proof}

In the rest of the section, we show the following claim, proving the algorithm's correctness. 

\begin{lemma}\label{lem:n1}
Each agent in $N_1$ receives a bag valued at least $\beta$.
\end{lemma}

For a contradiction, suppose the bag filling phase stopped at iteration $k\le n'$ because $[m']\setminus [2n']$ is empty and an agent $a\in N_1$ has not received a bag. Since no more valid reductions are feasible, like in Corollary~\ref{cor:val}, we must have 
\begin{equation}
v_a(j) < \begin{cases} \beta & \forall j\le n'\\ \beta/2 & \forall n' < j \le 2n'\\ \beta/3 & \forall j> 2n' \end{cases}
\end{equation}

This implies that at the beginning of the bag filling phase, 
\begin{equation}\label{eqn:bk}
v_i(B_k) < 3\beta/2, \forall k\in [n']. 
\end{equation}
Further, if $\hat{B} \neq B_k$ for some $k$, then $v_a(\hat{B}) < \beta + \beta/3 = 4\beta/3$ for the bags assigned to other agents before iteration $k$ because $v_a(\hat{B}\setminus g) < \beta$, where $g$ is the last good added to $\hat{B}$ and $v_a(g) \le v_a(2n'+1) < \beta/3$. Also, $v_a(\hat{B}) < \beta$ for all bags assigned to agents in $N_2$ because we prioritize agents in $N_1$. Therefore, in the bag-filling phase, we have

\begin{equation}\label{eqn:hbk}
v_a(\hat{B}) < \begin{cases} \beta  & \text{ if } \hat{B} \text{ is assigned to an agent in } N_2\\
3\beta/2  & \text{ if } \hat{B} \text{ is assigned to an agent in } N_1  \end{cases}.
\end{equation}

During valid reductions, since we prioritize agents in $N_1$, we have 
\begin{equation}
v_a(S_{\ell}) < \beta \ \  \text{ if } S_{\ell} \text{ is assigned to an agent in } N_2 .
\end{equation}

We next bound the value of $v_a(S_{\ell})$ when it is assigned to an agent in $N_1$. We have 
\begin{equation}\label{eqn:sn1}
\begin{aligned}
v_a(S_1) & \le  1 \text{ for any reduction using } S_1 \text{ since the instance is $n$-normalized}\\
v_a(S_3) & <  3\beta/2 \text{ for any reduction using } S_3 \text{ since $S_2$ is not feasible} \\
v_a(S_4) & <  \beta + \beta/3 = 4\beta/3 \text{ for any reduction using } S_4 \text{ since $S_1$ and $S_3$ are not feasible}
\end{aligned}
\end{equation}

The only case left is reduction using $S_2$, for which we break the analysis into multiple cases. Let $S_{\ell_1}S_{\ell_2} \cdots$ for $\ell_i \in [4]$ be a series of reduction. Now, consider the transitions to $S_2$, i.e., $\cdots S_{\ell}[S_2]^{t}S_{\ell'}\cdots$, where $\ell,\ell' \neq 2$ and $t\ge 1$ denote the number of $S_2$'s between $S_{\ell}$ and $S_{\ell'}$. Let $S_2^{t'}$ denote the $t'$-th $S_2$ for $t'\in [t]$. There are three cases: 
\smallskip

\textbf{Case 1} $[S_1]^{s}[S_2]^{t}\cdots:$ Here, $S_1$ occurs exactly $s\ge 0$ times. This case can happen at most once since $S_1$'s are the highest priority, and once $S_1$ is not feasible, it remains infeasible. 
%\HA{$S_2$ is a valid reduction before any $S_4$ happens. So in Case 1, $v_a(S_2^{t'}) \leq 1$.}
\begin{lemma}[Case 1]\label{lem:s1}
$v_a(S_2^{t'}) < 3\beta/2, \forall t'\in [t]$. 
\end{lemma}

\begin{proof}
Note that $S_2^{t'} := \{n-t'+1,n+t'\}, \forall t'\in [t]$, where $n$ is the number of agents in the original instance. By the pigeonhole principle, a bundle in $a$'s MMS partition $P^a$ must contain two goods from $[n+1]$. This, together with the instance being $n$-normalized, implies that $v_{a}(S_2^1) \le 1$. For $t'\ge 2$, if $v_a(S_2^{t'}) > 1$, then the goods in $\{n-t'+2, \dots, n+t'\}$ must be in $t'-1$ different bundles in $P^a$, which implies that there must be a bundle in $P^a$ that contains at least three goods from $\{n-t'+2, \dots, n+t'\}$ by the pigeonhole principle. This further implies that $v_a(n+t') \le 1/3$ because the instance is $n$-normalized. 

Finally, since $S_1$ is not feasible when the algorithm performs $S_2$, we have $v_a(n-t'+1) < \beta, \forall t' \in [t]$ and then we have $v_a(S_2^{t'}) = v_a(n-t'+1) + v_a(n+t') < \beta + 1/3 < 3\beta/2, \forall t'\ge 2$ using $\beta>3/4$. 
\end{proof}

\textbf{Case 2} $\cdots S_4[S_2]^{t}\cdots$: This case cannot happen because $S_2$ was not feasible when $S_4$ was performed and the set of goods in $S_2$ doesn't change after an $S_4$. 
\medskip

\textbf{Case 3} $\cdots S_3[S_2]^{t}\cdots$: Let $s$ be the number of agents just before the instance is reduced using $S_3$, which implies that $S_3 = \{2s+1, 2s, 2s-1\}$ and $S_2 = \{s-1, s\}$. Let $x:= v_a(S_3)$, which implies that $v_a(2s-1) \ge x/3$. Since $S_2$ is not feasible when we used $S_3$, we have $v_a(\{s, s+1\}) < \beta$, which further implies that $x<3\beta/2$. Furthermore, we have $v_a(s+1) \ge v_a(2s-1) \ge x/3$ and $v_a(s) < \beta - x/3$. Next, we break the analysis into two subcases depending on whether there are more $S_3$ reductions later. 

\textbf{Case 3a:} If this is the last reduction with $S_3$, then we have $v_a(\{s-1, s\} < \beta + \beta - x/3 < 2\beta$. Since all later $S_2 = \{j, j'\}$'s will have a $j<s-1$ and $j'>s+1$, we have $v_a(\{j, j'\}) < \beta + \beta/2 = 3\beta/2$ for each of them. Note that this case can occur at most once. 

\textbf{Case 3b:} For the other case, if this is not the last $S_3$, then we must have $v_a(\{s-2, 2s-2\})<\beta$ otherwise, $S_2$ will always be feasible contradicting the fact that this is not the last $S_3$. This implies that $v_a(s-1) \le v_a(s-2) < \beta - v_a(2s-2) < \beta - x/3$.  Then, we have 
$$v_a(S_2^1 \cup S_3) = v_a(\{s-1, s\}) + v_a(S_3) < 2\beta - 2x/3 + x < 2\beta + x/3 < 5\beta/2.$$
Furthermore, for each of the remaining $t-1$ $S_2 = \{j, j'\}$'s, we have $j<s, j'>s$, implying $v_a(\{j, j'\} < \beta + \beta/2 = 3\beta/2$. 
The above analysis implies that 

\begin{corollary}
In Case 3, either $v_a(S_2^1 \cup S_3) < 5\beta/2$ or $v_a(S_2^1) < 2\beta$. For the remaining $t-1$ $S_2$'s, $v_a(S_2^{t'}) < 3\beta/2, \forall t'\ge 2$. 
\end{corollary}

We are now ready to prove Lemma~\ref{lem:n1}.

\begin{proof}(of Lemma~\ref{lem:n1})
Recall that we assumed for a contradiction that the bag filling phase stopped at iteration $k\le n'$ because $[m']\setminus [2n']$ is empty and an agent $a\in N_1$ has not received a bag. Lemma~\ref{lem:N2} implies that all agents in $N_2$ must have received a bag valued at least $3/4$ before this iteration. 

Since we prioritize agents in $N_1$ in both valid reductions and bag filling, we have $v_a(S) < \beta$ whenever $S$ is given to an agent in $N_2$. Further, \eqref{eqn:bk} and \eqref{eqn:hbk} imply that both $v_a(B_k)$  and $v_a(\hat{B})$ are strictly less than $3\beta/2$ at the beginning and when assigned to other agents in the bag filling phase. In valid reductions, except for Case 3 of $S_2$, \eqref{eqn:sn1} and Lemma~\ref{lem:s1} imply that $v_a(S_\ell) < 3\beta/2$. Case 3a of $S_2$ occurs at most once, which implies that for all $S_2$'s in this case except for one, say $S_2^*$, we have $v_a(S_2) < 3\beta/2$ and $v_a(S_2^*) < 2\beta$. Case 3b of $S_2$ implies that $v_a(S_3 \cup S_2^1) < 5\beta/2$ and for all other $S_2$'s we have $v_a(S_2) < 3\beta/2$. Therefore, we have

\begin{equation}\nonumber
\begin{aligned}
n = v_a(M) & = \sum_{S \text{ assigned to } i\in N_2} v_a(S) + \sum_{S \text{ assigned to } i\in N_1} v_a(S) + v_a(B_k) + \sum_{j=k+1}^{n'} v_a(B_j)\\
& < \beta\cdot (3\beta-2)n/\beta + 3\beta/2\cdot (2(1-\beta)n/\beta-(n'-k+2)) + 2\beta + v_a(B_k) + 3\beta/2\cdot (n'-k)\\
& = n - \beta + v_a(B_k), 
\end{aligned}
\end{equation}
which implies that $v_a(B_k) > \beta$, a contradiction. 
\end{proof}


\begin{comment}
\section{System Architecture}
\label{appendix:architecture}
\system has a novel modularized system architecture with three key components: 
\emph{StreamManager}, 
\emph{TxnManager} and \emph{TxnScheduler}. 
These components are instantiated in each thread locally.
The execution outline of \system is presented in Algorithm~\ref{alg:algo}.
Transactional stream processing is continuous and potentially never ends (Line 1$\sim$8).
The dependency resolution and execution of state transactions are separated into two non-overlapping phases by punctuations~\cite{Tucker:2003:EPS:776752.776780} (Line 2 and 5), which guarantees that no subsequent input event will have a smaller timestamp. 
Effectively, a batch of state transactions is collected during the first phase, and processed during the second phase.

In the first phase (i.e., stream processing phase), 
the \emph{StreamManager} conducts preprocessing for every input event ($e$). Similar to some prior works~\cite{tstream}, state transactions may be issued but not immediately processed during preprocessing (Line 3).
The \emph{pre\_processing} and \emph{post\_processing} functions are exposed as APIs to users.
The \emph{TxnManager} handles dependency resolution (Line 4) among state transactions and insert decomposed operations to construct a \tpg. We discuss the detailed two-phase \tpg construction process in Section~\ref{subsec:construction}.

In the second phase  (i.e., transaction processing phase), 
the \emph{TxnManager} is first involved again to refine (Line 6) the constructed \tpg with further dependency resolution.
The \emph{TxnScheduler} 
schedules operations for concurrent execution based on the constructed \tpg according to the three dimensions of scheduling decisions (Line 7). 
In particular, a scheduling decision model $M$ is instantiated based on the constructed \tpg (Line 14).
\textbf{\circled{1}} Guided by $M$, execution threads adopt an exploration strategy (Section~\ref{subsec:explore}) to explore the constructed \tpg for operations available to be scheduled constrained by dependencies. 
\textbf{\circled{2}} 
During exploration, one or multiple operations may be treated as the 
% basic 
unit of scheduling (Section~\ref{subsec:granularity}). 
Subsequently, \textbf{\circled{3}} every thread executes operation(s) in the unit of scheduling with various abort handling mechanisms (Section~\ref{subsec:abort_handling}).
Only when state transactions are processed (i.e., committed or aborted) can the associated input events be postprocessed (Line 8) by the \emph{StreamManager} based on transaction processing results.
\end{comment}

\begin{comment}
\begin{algorithm}
\footnotesize
    \KwData{$e$ \tcp{Input event}}
    \KwData{$txn_{ts}$ \tcp{State transaction}}
    \KwData{$G$ \tcp{The currently constructed TPG}}
    \While{!finish processing of input streams}{
        \eIf(\tcp*[h]{Phase 1}){\text{$e$ is not a $punctuation$}}{
                $txn_{ts}$ $\gets$ PRE\_Processing($e$)\;
                \textbf{TPG\_Construction}($G$, $txn_{ts}$)\; 
          }(\tcp*[h]{Phase 2}){
                \textbf{TPG\_Refinement}($G$)\; 
                \textbf{TXN\_Scheduling}($G$)\; 
                POST\_Processing()\;
          }
    }
    
    \SetKwFunction{FMain}{TPG\_Construction}
    \SetKwProg{Fn}{Function}{:}{}
    \Fn{\FMain{$G$, $txn_{ts}$}}{
        $O_{1..k}$ $\gets$ \textbf{Partition} $txn_{ts}$\;
        \ForEach{\text{operation $O_{i}$ $\in$ $O_{1..k}$}}{
            \textbf{Identify} its \ld\;
            $G$ $\gets$ $G$ + $O_{i}$ \;
        }
    }
    \SetKwFunction{FMain}{TPG\_Refinement}
    \SetKwProg{Fn}{Function}{:}{}
    \Fn{\FMain{$G$}}{
        \ForEach{\text{vertex $e_{i}$ $\in$ $G$}}{
            \textbf{Identify} its \td, \pd\;
        }
    }
    
    \SetKwFunction{FMain}{TXN\_Scheduling}
    \SetKwProg{Fn}{Function}{:}{}
    \Fn{\FMain{$G$}}{
        $M$ $\gets$ Instantiated with $G$;\tcp{A decision model}
        \While{!finish scheduling of $G$
        }{
          \textbf{\circled{2}} $Scheduling Unit$ $\gets$ \textbf{\circled{1}} \emph{Explore}($G$, $M$)\; 
            \textbf{\circled{3}} \emph{Execute with Abort Handling} ($Scheduling Unit$)\; 
        }
    }
  \caption{Execution Outline of \system}
  \label{alg:algo}
\end{algorithm}
\end{comment}
\bibliographystyle{alpha}
\bibliography{bibliography}
\end{document}
