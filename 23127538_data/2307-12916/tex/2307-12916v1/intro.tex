\section{Introduction}
Fair allocation of resources (goods) is a fundamental problem in multiple disciples, including computer science, economics, and social choice theory, where the goal is to divide goods among agents in a \emph{fair} manner. This field has received significant attention since the seminal work of Steinhaus in the 1940s~\cite{steinhaus1948problem}. 
When the goods are divisible, the two standard fairness notions are \emph{envy-freeness} and \emph{proportionality}, based on envy and share, respectively. In an envy-free allocation, no agent prefers another agent's allocation, and in a proportional allocation, each agent receives her proportionate share, i.e., at least a $1/n$ fraction of her value of all the goods. In the case of divisible goods, an envy-free and proportional allocation exists; see~\cite{Varian74,Foley67,AzizM16b}.

We study the discrete setting, in which each good can be given to exactly one agent. Formally, the input consists of a set $N$ of $n$ agents, a set $M$ of $m$ indivisible goods and a valuation profile $\mathcal{V} = (v_1, \ldots, v_n)$ where $v_i: 2^M \rightarrow \mathbb{R}_{\geq 0}$ is agent $i$'s valuation function over the subsets of goods. Simple examples show that in the discrete case, neither envy-freeness nor proportionality can be guaranteed.\footnote{Consider an instance with two agents and one good with positive value to both agents.} This necessitates the refinement of these notions. 

In this paper, we consider the natural and most popular discrete analog of proportionality called \emph{maximin share} (MMS) introduced in~\cite{budish2011combinatorial}. The MMS value of an agent is the maximum value she can guarantee if she divides the goods into $n$ bundles (one for each agent) and then receives a bundle with the minimum value. Formally, for a set $S$ of goods and an integer $d$, let $\Pi_d(S)$ denote the set of all partitions of $S$ into $d$ bundles. Then, 
\begin{align*}
    \MMS_i^d(S) := \max_{(P_1, \ldots, P_d) \in \Pi_d(S)} \min_{j} v_i(P_j). 
\end{align*}
The MMS value of each agent $i$ is denoted by $\MMS_i := \MMS_i^n(M)$.
An allocation is said to be MMS if each agent receives at least their MMS value. However, MMS is an unfeasible share guarantee that cannot always be satisfied when there are more than two agents with additive valuations~\cite{procaccia2014fair, KurokawaPW18, FeigeST21}. Therefore, the MMS share guarantee needs to be relaxed, and the two natural ways are its multiplicative and ordinal approximations. 
\medskip

\noindent{\bf \boldmath $\alpha$-MMS.} Since we need to lower the share threshold, a traditional way is to consider $\alpha<1$ times the MMS value. Formally, an allocation $X = \langle X_1, \ldots, X_n \rangle$ is $\alpha$-MMS if for each agent $i$, $v_i(X_i) \geq \alpha \cdot \MMS_i$. Earlier works showed the existence of $2/3$-MMS allocations using several different approaches~\cite{procaccia2014fair,amanatidis2017approximation,barman2020approximation,garg2019approximating,KurokawaPW18}. Later, in a groundbreaking work~\cite{ghodsi2018fair}, the existence of $3/4$-MMS allocations was obtained through more sophisticated techniques and involved analysis. This factor was slightly improved to $3/4 + 1/(12n)$ in~\cite{garg2021improved}, then more recently to $\frac{3}{4} + \min(\frac{1}{36}, \frac{3}{16n-4})$~\cite{simple}, and finally to $3/4+3/3836$~\cite{Akrami2023BreakingT}. On the other hand, $\alpha$-MMS allocations need not exist for $\alpha> 1-1/n^4$~\cite{FeigeST21}. 
\medskip

\noindent{\bf \boldmath $1$-out-of-$d$ MMS.} Another way of relaxing MMS is to consider the share value of $\MMS_i^d(M)$ for $d>n$ for each agent $i$, which is the maximum value that $i$ can guarantee if she divides the goods into $d$ bundles and then takes a bundle with the minimum value. This notion was introduced together with the MMS notion in~\cite{budish2011combinatorial}, which also shows the existence of $1$-out-of-$(n+1)$ MMS after \emph{adding excess goods}. Unlike $\alpha$-MMS, this notion is robust to small perturbations in the values of goods because it only depends on the bundles' ordinal ranking and is not affected by small perturbations as long as the ordinal ranking of the bundles does not change.\footnote{As mentioned in~\cite{Hosseini2021OrdinalMS}, the $\alpha$-MMS is very sensitive to agents' precise cardinal valuations: Consider the example mentioned in \cite{Hosseini2021OrdinalMS}. Assume $n=3$ and there are four goods $g_1$, $g_2$, $g_3$ and $g_4$ with values $30$, $39$, $40$ and $41$ respectively for agent $1$. Assume the goal is to guarantee the $3/4$-MMS value of each agent. We have $\MMS_1 = 40$, and therefore any non-empty bundle satisfies $3/4$-MMS for agent $1$. However, if the value of $g_3$ gets slightly perturbed and becomes $40+\epsilon$ for any $\epsilon>0$, then $\MMS_1 > 40$ and then $3/4 \cdot \MMS_1 > 30$ and the bundle $\{g_1\}$ does not satisfy agent $1$. Thus,  the acceptability of a  bundle  (in this example, $\{g_1\}$) might be affected by an arbitrarily small perturbation in the value of an irrelevant good (i.e., $g_3$). 

Observe that in this example, whether the value of $g_3$ is $40$ or $40+\epsilon$ for any $\epsilon \in \mathbb{R}$, $\{g_1\}$ is an acceptable $1$-out-of-$4$ MMS bundle for agent $1$.}

In the standard setting (i.e., without excess goods), the first non-trivial ordinal approximation was the existence of $1$-out-of-$(2n-2)$ MMS allocations~\cite{AignerHorev2019EnvyfreeMI}, which was later improved to $1$-out-of-$\lceil 3n/2 \rceil$~\cite{hosseini2021guaranteeing}, and then to the current state-of-the-art $1$-out-of-$\lfloor 3n/2 \rfloor$~\cite{Hosseini2021OrdinalMS}. On the other hand, the existence of $1$-out-of-$(n+1)$ MMS allocations is open to date. 

In this paper, we show that $1$-out-of-$\lceil 4n/3\rceil$ MMS allocations always exist, thereby improving the state-of-the-art of $1$-out-of-$d$ MMS. 
Another way to interpret $1$-out-of-$d$ MMS allocations is giving $n/d$ fraction of agents their MMS value and nothing to the remaining agents. We prove this equivalence in Section~\ref{sec:prelim}. 

Both ordinal and multiplicative approximations focus on extremes. In $1$-out-of-$d$ MMS, some agents get nothing, and others are guaranteed their \emph{full} MMS value. In $\alpha$-MMS, each agent receives (the same factor) $\alpha<1$ fraction of their MMS value. As a middle ground between these two notions, we introduce a general framework of $(\alpha, \beta, \gamma)$-MMS that guarantees $\alpha$ fraction of agents $\beta$ times their MMS values and the remaining $(1-\alpha)$ fraction of agents $\gamma$ times their MMS values. The $(\alpha, \beta, \gamma)$-MMS captures both ordinal and multiplicative approximations as special cases. Namely, an $\alpha$-MMS allocation can also be denoted by $(1,\alpha, \gamma)$-MMS for any arbitrary $\gamma$ and $1$-out-of-$d$ allocations are in correspondence with $(n/d, 1, 0)$-MMS allocations. Furthermore, the $(\alpha, \beta)$-framework introduced in~\cite{hosseini2021guaranteeing}, where $\alpha$ fraction of agents receive $\beta$-MMS, is another special case of $(\alpha, \beta, \gamma)$-MMS with $\gamma = 0$.

We show that $(2(1-\beta)/\beta, \beta, 3/4)$-MMS allocations always exist. Moreover, our algorithm can choose the $2(1-\beta)/\beta$ fraction of agents getting a $\beta>3/4$ fraction of their MMS value. Therefore, by choosing these agents randomly and using $\beta = \sqrt{3}/2$, we can guarantee each agent $3/4$ of her MMS value (ex-post) and $(17\sqrt{3}-24)/4\sqrt{3} > 0.785$ of her MMS value on average (ex-ante). 
