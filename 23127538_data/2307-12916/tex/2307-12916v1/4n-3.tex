\section{1-out-of-(4n/3) MMS Algorithm}\label{sec:4n-3}
Our algorithm in Algorithm~\ref{algo} consists of \emph{initialization} and \emph{bag-filling}. First, we remark that assuming $|M| \geq 2n$ is without loss of generality. This is because we can always add dummy goods to $M$ with a value of $0$ for all the agents. The resulting instance is ordered and $d$-normalized if the original instance has these properties.

As mentioned in Section~\ref{sec:tec}, the algorithm first initialize $n$ bags as in \eqref{eq:B_i} (see Figure~\ref{fig:bags}). 
% Figure environment removed
Then, in each round $j$ of bag-filling, we keep adding goods in decreasing value to the bag $B_j$ until some agent with no assigned bag values it at least $1$. Then, we allocate it to an arbitrary such agent. 

In the rest of this section, we prove the following theorem, showing the correctness of the algorithm.

\begin{theorem}\label{thm:main}
    Given any ordered $(4n/3)$-normalized instance, Algorithm \ref{algo} returns a $1$-out-of-$(4n/3)$ MMS allocation.
\end{theorem}

To do so, it suffices to prove that we never run out of goods in bag-filling. Towards contradiction, assume that the algorithm stops before all agents receive a bundle. Let $i$ be an agent with no bundle. Let $\hat{B}_j$ be the $j$-th bundle after bag-filling.  
\begin{observation}\label{upper-bound-left}
    For all $j,k$ such that $j \leq k \leq n$, $v_i(\hat{B}_j) \leq 1 + v_i(2n-k+1)$.
\end{observation}
\begin{proof}
    Let $g$ be the good with the largest index in $\hat{B}_j$. If $g = 2n-j+1$, $v_i(\hat{B}_j \setminus \{g\}) = v_i(j)  \leq 1$ by Proposition \ref{prop:exact}(\ref{exact:1}). If $g > 2n-j+1$, meaning that $g$ was added to $\hat{B}_j$ during bag-filling, then $v_i(\hat{B}_j \setminus \{g\}) < 1$. Otherwise, $g$ would not be added to $\hat{B}_j$. Therefore, 
    \begin{align*}
        v_i(\hat{B_j}) &= v_i(\hat{B}_j \setminus \{g\}) + v_i(g) \\
        &\leq 1 + v_i(2n-k+1). &\tag{$v_i(\hat{B}_j \setminus \{g\}) \leq 1$ and $g \geq 2n-k+1$}
    \end{align*}
\end{proof}
\begin{observation}\label{upper-bound-right}
    For all $j,k$ such that $k \leq j \leq n$, $v_i(\hat{B}_j) \leq \max(1 + v_i(2n-k+1), 2v_i(k))$.
\end{observation}
\begin{proof}
    First, assume $\hat{B}_j \neq B_j$ and $g$ be the last good added to $\hat{B}_j$. We have $v_i(\hat{B}_j \setminus \{g\}) < 1$. Otherwise, $g$ would not be added to $\hat{B}_j$. Therefore, 
    \begin{align*}
        v_i(\hat{B_j}) &= v_i(\hat{B}_j \setminus \{g\}) + v_i(g) \\
        &< 1 + v_i(2n-k+1). &\tag{$v_i(\hat{B}_j \setminus \{g\}) < 1$ and $g > 2n-k+1$}
    \end{align*}
    Now assume $\hat{B}_j = B_j$. We have
    \begin{align*}
        v_i(\hat{B}_j) &= v_i(B_j) \\
        &= v_i(j) + v_i(2n-j+1) \\
        &\leq 2v_i(k). &\tag{$2n-j+1 > j \geq k$}
    \end{align*}
    Hence, $v_i(\hat{B}_j) \leq \max(1 + v_i(2n-k+1), 2v_i(k))$.
\end{proof}
\begin{observation}\label{less-than-one}
    There exists a bag $B_j$, such that $v_i(B_j) < 1$.
\end{observation}
\begin{proof}
    Otherwise, the algorithm would allocate the remaining bag with the smallest index to agent $i$.
\end{proof}

Let $\ell^*$ be the smallest such that $v_i(B_{\ell^*+1}) < 1$. I.e., $B_{\ell^*+1}$ is the leftmost bag in Figure \ref{fig:bags} with a value less than $1$ to agent $i$. In Section \ref{negative}, we reach a contradiction assuming $v_i(2n-\ell^*)<1/3$ and prove Theorem \ref{contradict-1}.
\begin{restatable}{theorem}{contradictOne}\label{contradict-1}
    If Algorithm \ref{algo} does not allocate a bag to some agent $i$, then $v_i(2n-\ell^*) \geq 1/3$ where $\ell^*$ is the smallest index such that $v_i(B_{\ell^*+1}) < 1$.
\end{restatable}

In Section \ref{positive}, we reach a contradiction assuming $v_i(2n-\ell^*) \geq 1/3$ and prove Theorem \ref{contradict-2}. 
\begin{restatable}{theorem}{contradictTwo}\label{contradict-2}
    If Algorithm \ref{algo} does not allocate a bag to some agent $i$, then $v_i(2n-\ell^*) < 1/3$ where $\ell^*$ is smallest such that $v_i(B_{\ell^*+1}) < 1$.
\end{restatable}

By Theorems \ref{contradict-1} and \ref{contradict-2}, agent $i$ who receives no bundle by the end of Algorithm \ref{algo} does not exist, and Theorem \ref{thm:main} follows.

\subsection{\boldmath $\mathbf{v_i(2n-\ell^*) \geq 1/3}$}\label{positive}
In this section we assume $v_i(2n-\ell^*) = 1/3 + x$ for $x \geq 0$. %\jnote{$x\ge 0$?}
 We define $A^+ := \{B_1, B_2, \ldots, B_{\ell^*}\}$; see Figure~\ref{k-picture}.
% Figure environment removed
\begin{observation}\label{no-change}
    For all $B_j \in A^+$, $\hat{B}_j = B_j$.
\end{observation}
\begin{proof}
    For all $B_j \in A^+$, $v_i(B_j) \geq 1$. Since $i$ did not receive any bundle, $B_j$ must have been assigned to some other agent, and no good needed to be added to $B_j$ in bag-filling since there is an agent (namely $i$) with no bag who values $B_j$ at least $1$. 
\end{proof}
\begin{observation}\label{half-bound}
    For all $j\geq 2n-\ell^*$, $v_i(j) < 1/2$.
\end{observation}
\begin{proof}
    Since $v_i(B_{\ell^*+1}) = v_i(\ell^*+1)+ v_i(2n-\ell^*) < 1$ and $v_i(2n-\ell^*) \leq v_i(\ell^*+1)$, $v_i(2n-\ell^*) < 1/2$.
    Also for all $j \geq 2n-\ell^*$, $v_i(j) \leq v_i(2n-\ell^*) < 1/2$.
\end{proof}

\begin{corollary}[of Observation \ref{half-bound}]
    $x < 1/6$.    
\end{corollary}

Let $s$ be the smallest such that either the algorithm stops at step $s+1$ or $B_{s+1}$ gets more than one good in bag-filling. 
\begin{observation}
    $s \geq \ell^*$.
\end{observation}
\begin{proof}
    For all $j < \ell^*$, $v_i(B_{j+1}) \geq 1$. Since $i$ did not receive any bundle, $B_{j+1}$, must have been assigned to another agent. Therefore, the algorithm does not stop at step $j+1$. Also, by Observation \ref{no-change}, $B_{j+1}$ gets no good in bag-filling.
\end{proof}
Let $A^1$ be the set of bags in $\{B_{\ell^*+1}, \ldots, B_s\}$ which receive exactly one good in bag-filling. Formally, $A^1 = \{B_j | \ell^* < j \leq s \text{ and } |\hat{B}_j| = 3\}$. Let $A^2 = \{B_1, B_2, \ldots, B_n\} \setminus (A^+ \cup A^1)$. 

\begin{lemma}\label{save-2}
    For all $B_j \in A^2$, $v_i(B_j) < 4/3 -2x$.
\end{lemma}
\begin{proof}
    We have 
    \begin{align*}
        1 &> v_i(B_{\ell^*+1}) \\
        &= v_i(\ell^*+1) + v_i(2n-\ell^*) \\
        &= v_i(\ell^*+1) + \frac{1}{3} + x.
    \end{align*}
    Hence, $v_i(\ell^*+1) < 2/3 - x$. Also, for $B_j \in A^2$, we have
    \begin{align*}
        v_i(B_j) &= v_i(j) + v_i(2n-j+1) \\
        &\leq 2v_i(\ell^*+1) &\tag{$2n-j+1 > j \geq \ell^*+1$ since $B_j \in A^2$} \\
        &< \frac{4}{3} - 2x.
    \end{align*}
    So if $\hat{B}_j = B_j$, the inequality holds. Now assume $\hat{B}_j \neq B_j$. This implies that $j \geq s+1$ and the algorithm did not stop at step $j$ before adding a good to $B_j$. Therefore it did not stop at step $s +1$ before adding a good to $B_{s+1}$ either. Let $g$ be the first good added to $B_{s+1}$. Since $B_{s+1}$ requires more than one good,
    \begin{align*}
        1 &> v_i(B_{s +1} \cup \{g\}) \\
        &= v_i(s+1) + v_i(n-s) + v_i(g) \\
        &\geq 2v_i(2n-\ell^*) + v_i(g) &\tag{$s+1 < n-s \leq 2n-\ell^*$} \\
        &= \frac{2}{3} + 2x + v_i(g).
    \end{align*}
    Therefore, $v_i(g) < 1/3 - 2x$. Now let $h$ be the last good added to bag $B_j$. We have
    \begin{align*}
        v_i(\hat{B}_j) &= v_i(\hat{B}_j \setminus \{h\}) + v_i(h) \\
        &< 1 + v_i(g) &\tag{$v_i(\hat{B}_j \setminus \{h\}) < 1$ and $v_i(h) \leq v_i(g)$} \\
        &< \frac{4}{3} - 2x.
    \end{align*}
\end{proof}

\begin{lemma}\label{save-all}
    For all $B_j \in A^+ \cup A^1$, $v_i(\hat{B}_j) \leq 4/3 + x$.
\end{lemma}
\begin{proof}
    First assume $B_j \in A^+$. We have $j \leq \ell^*$. Also,
    \begin{align*}
        v_i(\hat{B}_j) &= v_i(B_j) &\tag{$\hat{B}_j = B_j$} \\
        &= v_i(j) + v_i(2n-j+1) \\
        &\leq 1 + v_i(2n-\ell^*) &\tag{$v_i(j) \leq 1$ and $2n-j+1 > 2n-\ell^*$} \\
        &= \frac{4}{3} +x.
    \end{align*}
    Now assume $B_j \in A^1$. Let $g$ be the good added to bag $B_j$ in bag-filling. We have,
    \begin{align*}
        v_i(\hat{B}_j) &= v_i(B_j) + v_i(g) \\
        &< 1 + v_i(2n-\ell^*) &\tag{$v_i(B_j) < 1$ and $v_i(g) \leq v_i(2n+1) \leq v_i(2n-\ell^*)$}\\
        &= \frac{4}{3} + x. 
    \end{align*}
\end{proof}

Let $|A^1|=2n/3+\ell$. Then $|A^2| = n - \ell^* - (2n/3 + \ell) = n/3-(\ell+\ell^*)$. If $\ell + \ell^* \leq 0$, then $|A^2| \geq n/3$ and hence there are at least $n/3$ bags with value less than $4/3 -2x$ (by Lemma \ref{save-2}) and at most $2n/3$ bags with value at most $4/3 + x$ (by Lemma \ref{save-all}). Hence, 
$$v_i(M) < \frac{n}{3}(\frac{4}{3}-2x) + \frac{2n}{3}(\frac{4}{3}+x) = \frac{4n}{3}$$ 
which is a contradiction since $v_i(M) = 4n/3$. So assume $\ell + \ell^* > 0$. 

Limit the items in a $1$-out-of-$4n/3$ MMS partition $P^i=(P^i_1, \dots, P^i_{4n/3})$ of agent $i$ to $\{1, \ldots, 8n/3+\ell\}$ and let $Q$ be the set of bags in $P^i$ containing goods $\{1, \ldots, \ell^*\}$. Formally, $Q = \{ P_j^i \cap \{1, \ldots, 8n/3+\ell\}: |P_j^i \cap \{1, \ldots, \ell^*\}| \geq 1\}$. Let $t$ be the number of bags of size $1$ in $Q$.

% Figure environment removed

\begin{lemma}\label{expowerful}
    Let $t$ be the number of bags of size $1$ in $Q = \{ P_j^i \cap \{1, \ldots 8n/3+\ell\}: |P_j^i \cap \{1, \ldots \ell^*\}| \geq 1\}$. Then,
    \begin{align*}
        v_i(&\{8n/3 - 2\ell-t-2\ell^*+1, \ldots, 8n/3 + \ell\} \\
        &\cup \{t+1, \ldots, \ell^*\} \\
        &\cup \{2n-\ell^*+1, \ldots, 2n-t\}) \leq 2\ell^* + \ell -t.
    \end{align*}
\end{lemma}
The items considered in Lemma \ref{expowerful} are marked with blue in Figure \ref{colorful}. 
First, we prove that the goods mentioned in Lemma \ref{expowerful} are distinct. To that end, it suffices to prove that $8n/3 - 2\ell-t-2\ell^*+1 > 2n-t$. It follows from the fact that $2n/3+\ell+\ell^* \leq n$. 
Before proving Lemma \ref{expowerful}, let us show how to obtain a contradiction assuming this lemma holds.
Note that since there are $n/3 - \ell - \ell^*$ bags with value less than $4/3-2x$ (namely the bags in $A^2$), it suffices to prove that there exists $3(\ell + \ell^*)$ other bags with total value $4(\ell + \ell^*)$. Since the remaining $2n/3 -2\ell -2\ell^*$ bags are of value at most $4/3+x$ (by Lemma \ref{save-all}), we get 
\begin{align}
    v_i(M) < (\frac{n}{3} - \ell - \ell^*) (\frac{4}{3} - 2x) + (\frac{2n}{3} - 2\ell - 2\ell^*)(\frac{4}{3} + x) + 4 (\ell + \ell^*) = \frac{4n}{3} \label{important-ineq}
\end{align}
which is a contradiction since $v_i(M)=4n/3$.

Now consider $B = \{\hat{B}_1, \ldots, \hat{B}_{2\ell + t + 2\ell^* -2n/3}, \hat{B}_{t+1}, \ldots, \hat{B}_{\ell^*} \} \cup \hat{A}^1$ where $\hat{A}^1$ is the set of bags in $A^1$ after bag-filling. $B$ consists of $3(\ell+\ell^*)$ bags. Now we prove that $v_i(\bigcup_{B_j \in B}B_j) \leq 4(\ell+\ell^*)$. We have 
%\jnote{I think we cannot say the exact items in these bags but can argue about their values.}
\begin{align*}
    v_i(\bigcup_{\hat{B}_j \in B}\hat{B}_j) \leq v_i(&\bigcup_{B_j \in A^1} B_j) \\
    + v_i(\textcolor{red}{\{}&\textcolor{red}{1, \ldots, 2\ell + t + 2\ell^* -2n/3\}})\\
    + v_i(\textcolor{blue}{\{}&\textcolor{blue}{8n/3 - 2\ell-t-2\ell^*+1, \ldots, 8n/3 + \ell\}} \\
     &\textcolor{blue}{\cup \{t+1, \ldots, \ell^*\}} \\
     &\textcolor{blue}{\cup \{2n-\ell^*+1, \ldots, 2n-t\}}) .
\end{align*}
We bound the value of the goods marked with different colors in different inequalities.  
\begin{observation}
    For all $B_j \in A^1$, $v_i(B_j) < 1$.
\end{observation}
Since $|A^1| = 2n/3 + \ell$, 
\begin{align*}
    v_i(\bigcup_{B_j \in A^1} B_j) < 2n/3 +\ell.
\end{align*}
Also, since all goods are of value at most $1$ to agent $i$,

\begin{align*}
    \textcolor{red}{v_i(\{1, \ldots, 2\ell + t + 2\ell^* -2n/3\}) \leq 2\ell + t + 2\ell^* -2n/3}.
\end{align*}
By Lemma \ref{expowerful}, 
\textcolor{blue}{
\begin{align*}
        v_i(&\{8n/3 - 2\ell-t-2\ell^*+1, \ldots, 8n/3 + \ell\} \\
        &\cup \{t+1, \ldots, \ell^*\} \\
        &\cup \{2n-\ell^*+1, \ldots, 2n-t\}) \leq 2\ell^* + \ell -t.
\end{align*}}

By adding all the inequalities, we get
\begin{align*}
    v_i(\bigcup_{\hat{B}_j \in B}\hat{B}_j) \leq 4(\ell+\ell^*). 
\end{align*}
Hence, Inequality \eqref{important-ineq} holds, which is a contradiction. So the case of $v_i(2n-\ell^*) \geq 1/3$ cannot arise.
\contradictTwo*


In the rest of this section, we prove Lemma \ref{expowerful}.

\subsubsection{Proof of Lemma \ref{expowerful}}

To prove Lemma \ref{expowerful}, we partition the goods considered in this lemma into two parts. These parts are colored red and blue in Figure \ref{colorful-1}. We bound the value of red goods in Lemma~\ref{k-t}, i.e., $$\sum_{2n-\ell^* < j\leq 2n-t}v_i(j) + \sum_{8n/3 - 2\ell-t-2\ell^* < j \leq 8n/3 - 2\ell - 2t - \ell^*}v_i(j) < \ell^*-t,$$ and the value of the blue goods in Lemma~\ref{k+l}, i.e., $$\sum_{t < j \leq \ell^*} v_i(j) + \sum_{8n/3 - 2\ell - 2t - \ell^* < j \leq 8n/3 + \ell}v_i(j) \leq \ell^* + \ell.$$
Thereafter, we have
\begin{align*}
    v_i(&\{8n/3 - 2\ell-t-2\ell^*+1, \ldots, 8n/3 + \ell\} \\
    &\cup \{t+1, \ldots, \ell^*\} \\
    &\cup \{2n-\ell^*+1, \ldots, 2n-t\}) \\
    &= \textcolor{red}{\sum_{2n-\ell^* < j\leq 2n-t}v_i(j) + \sum_{8n/3 - 2\ell-t-2\ell^* < j \leq 8n/3 - 2\ell - 2t - \ell^*}v_i(j)} \\
    &+ \textcolor{blue}{\sum_{t < j \leq \ell^*} v_i(j) + \sum_{8n/3 - 2\ell - 2t - \ell^* < j \leq 8n/3 + \ell}v_i(j)} \\
    &< (\ell^*-t) + (\ell^*+\ell) &\tag{Lemma \ref{k-t} and \ref{k+l}}\\
    &= 2\ell^*+\ell-t,
\end{align*}
and Lemma \ref{expowerful} follows.

It suffices to prove Lemmas \ref{k-t} and \ref{k+l}. In the rest of this section, we prove these two lemmas.
% Figure environment removed
Limit the items in a $1$-out-of-$4n/3$ MMS partition of agent $i$ to $\{1, \ldots, 8n/3+\ell\}$ and let $R$ be the set of the resulting bags. Formally, for all $j \in [4n/3]$, $R_j = P_j^{i} \cap \{1, \ldots, 8n/3+\ell\}$ and $R = \{R_1, \ldots, R_{4n/3}\}$. Without loss of generality, assume $|R_1| \geq |R_2| \geq \ldots \geq |R_{4n/3}|$. Let $t$ be the number of bags of size $1$ in $R$. 

\begin{lemma}\label{powerful}
    If there exist $t$ bags of size at most $1$ in $R$, then 
    $$\sum_{1 \leq j \leq t+\ell}|R_j| \geq 3(t+\ell).$$
\end{lemma}
\begin{proof}
    Since $R_j$'s are sorted in decreasing order of their size, 
    \begin{align*}
        \sum_{1 \leq j \leq t+\ell}|R_j| \geq (t+\ell)|R_{t+\ell}|.
    \end{align*}
    Hence, if $|R_{t+\ell}| \geq 3$, then $\sum_{1 \leq j \leq t+\ell}|R_j| \geq 3(t+\ell).$ So assume $|R_{t+\ell}| \leq 2$. 
    \begin{align*}
        \frac{8n}{3} + \ell &= \sum_{1 \leq j \leq 4n/3}|R_j| \\
        &= \sum_{1 \leq j \leq t+\ell} |R_j| + \sum_{t+\ell < j \leq 4n/3 - t}|R_j| + \sum_{4n/3-t < j \leq 4n/3}|R_j| \\
        &\leq \sum_{1 \leq j \leq t+\ell} |R_j| +  (\frac{4n}{3} - 2t - \ell)|R_{t+\ell}| + t \\
        &\leq \sum_{1 \leq j \leq t+\ell} |R_j| + 2(\frac{4n}{3} - 2t - \ell) + t 
    \end{align*}
    Therefore, $$\sum_{j \in [t+\ell]} |R_j| \geq 3(\ell+t).$$
\end{proof}
\begin{lemma}\label{bound}
    $\ell+\ell^*+t \leq 4n/3$.
\end{lemma}
\begin{proof}
    We have $\ell^* + 2n/3 + \ell \leq s \leq n$. See Figure \ref{colorful} for intuition. Therefore, $\ell^* + \ell \leq n/3$. Also, $t \leq \ell^* \leq n$. Hence $\ell+\ell^*+t \leq 4n/3$.
\end{proof}
\begin{lemma}\label{k-t}
    \textcolor{red}{
    $$\sum_{2n-\ell^* < j\leq 2n-t}v_i(j) + \sum_{8n/3 - 2\ell-t-2\ell^* < j \leq 8n/3 - 2\ell - 2t - \ell^*}v_i(j) < \ell^*-t.$$
    }
\end{lemma}
\begin{proof}
    Let $B' = \{2n-\ell^*+1, \ldots, 2n-t\} \cup \{8n/3 - 2\ell-t-2\ell^*+1, \ldots, 8n/3 - 2\ell - 2t - \ell^*\}$. $|B'| = 2(\ell^*-t)$ and by Observation \ref{half-bound} for all goods $g \in B'$, $v_i(g) < 1/2$. Therefore, $v_i(B') < \ell^*-t$.
\end{proof}
\begin{lemma}\label{k+l}
    \textcolor{blue}{
    $$\sum_{t < j \leq \ell^*} v_i(j) + \sum_{8n/3 - 2\ell - 2t - \ell^* < j \leq 8n/3 + \ell}v_i(j) \leq \ell^* + \ell.$$
    }
\end{lemma}
\begin{proof}
    Recall that $\{R_1, \ldots, R_{4n/3}\}$ is the set of bags in the $1$-out-of-$4n/3$ MMS partition of agent $i$ after removing items $\{8n/3+\ell+1, \ldots, m\}$. Moreover, we know exactly $t$ of these bags have size $1$. If there is a bag $R_j = \{g\}$ for $g>t$, there must be a good $g' \in [t]$ such that $g' \in R_{j'}$ and $|R_{j'}|>1$. Swap the goods $g$ and $g'$ between $R_j$ and $R_{j'}$ as long as such good $g$ exists. Note that $v_i(R_{j'})$ can only decrease and $v_i(R_j)=v_i(g') \leq 1$. Therefore, in the end of this process for all $j \in [4n/3]$, $v_i(R_j) \leq 1$ and we can assume bags containing goods $1, \ldots, t$ are of size $1$ and bags containing goods $t+1, \ldots, \ell^*$ %\jnote{here what is $k$? should it be $8n/3 + \ell$?} 
    are of a size more than $1$. Recall that $|R_1| \geq \ldots \geq |R_{4n/3}|$. Let $T_{j}$ be the bag that contains good $j$. 
    
    Consider the bags $B=\{R_1, \ldots, R_{t+\ell}\} \cup \{T_{t+1}, \ldots, T_{\ell^*}\}$. If $|B| < \ell^*+\ell$, keep adding a bag with the largest number of items to $B$ until there are exactly $\ell^*+\ell$ bags in $B$. First we show that $B$ contains at least $3\ell + 2\ell^* + t$ goods. Namely,
    $$\sum_{S \in B} |S| \geq 3\ell + 2\ell^* + t.$$
    By Lemma \ref{powerful}, $\sum_{1 \leq j \leq t+\ell} |R_j| \geq 3(t+\ell)$. If all the remaining $\ell^*-t$ bags in $B \setminus \{R_1, \ldots, R_{t+\ell}\}$ are of size $2$, then $\sum_{S \in B} |S| \geq 3(t+\ell)+2(\ell^*-t) = 3\ell + 2\ell^* +t$. Otherwise, there is a bag in $B$ of size at most $1$; hence, all bags outside $B$ are also of size at most $1$. So we have
    \begin{align*}
        \frac{8n}{3} + \ell &= \sum_{S \in B} |S| + \sum_{S \notin B} |S| \\
        &\leq \sum_{S \in B} |S| + (\frac{4n}{3} - \ell^* - \ell).
    \end{align*}
    Therefore, 
    \begin{align*}
        \sum_{S \in B} |S| &\geq 4n/3 + 2\ell + \ell^* \\
        &\geq 3\ell + 2\ell^* +t. &\tag{Lemma \ref{bound}}
    \end{align*}
    Note that the goods $\{t+1, \ldots, \ell^*\}$ are contained in $B$ and moreover, $B$ contains at least $3\ell + 2\ell^* +t - (\ell^*-t) = 3\ell + 2t + \ell^*$ other goods. Therefore,
    \begin{align*}
        \ell^*+\ell &\geq \sum_{S \in B} v_i(S) \\
        &\geq \sum_{t < j \leq \ell^*} v_i(j) + \sum_{8n/3 - 2\ell - 2t - \ell^* < j \leq 8n/3 + \ell} v_i(j).
    \end{align*}
The last inequality follows because we used the $3\ell + 2t + \ell^*$ lowest valued goods in $[8n/3+\ell]$. 
\end{proof}    

\subsection{\boldmath $\mathbf{v_i(2n-\ell^*)<1/3}$}\label{negative}
Let $r^*$ be largest such that $v_i(B_{r^*})<1$. That is, $B_{r*}$ is the rightmost bag in Figure \ref{fig:bags} with a value less than $1$ to agent $i$. 
\begin{lemma}\label{r-star-bound}
    If $v_i(2n-r^*+1) \leq 1/3$, then $r^* < 2n/3$.
\end{lemma}
\begin{proof}
    Since $1>v_i(B_{r^*})=v_i(r^*)+v_i(2n-r^*+1)$, we have $v_i(r^*)<2/3+x$. 
    By Observation \ref{upper-bound-left}, for all $j \leq r^*$, $v_i(\hat{B}_j) \leq 4/3-x$.
    Also, by Observation \ref{upper-bound-right}, for all $j>r^*$, $v_i(\hat{B}_j)<4/3+2x$.
    Hence, we have 
    \begin{align*}
        \frac{4n}{3} &= v_i(M) \\
        &= \sum_{j \leq r^*} v_i(\hat{B}_j) + \sum_{j > r^*} v_i(\hat{B}_j) \\
        &< r^*(\frac{4}{3}-x) + (n-r^*)(\frac{4}{3}+2x) \\%&\tag{Corollary \ref{cor-r-star} and \ref{claim-r-star-1}}\\
        &= \frac{4n}{3} + x(2n-3r^*).
    \end{align*}
    Therefore, $r^* < 2n/3$.
\end{proof}
\begin{lemma}
    $v_i(2n-r^*+1) > 1/3$.
\end{lemma}
\begin{proof}
    Towards contradiction, assume $v_i(2n-r^*+1) = 1/3-x$ for $x \geq 0$. By Lemma \ref{r-star-bound}, $r^* < 2n/3$.
    \begin{claim}\label{claim-r-star}
        $\sum_{j > r^*} v_i(\hat{B}_j) < \frac{10n}{9}-r^* + \frac{2nx}{3}$.
    \end{claim}
    \begin{claimproof}
    Note that by the definition of $r^*$, for all $j > r^*$, $\hat{B}_j = B_j$.
    By Lemma \ref{lem:Csum}, $v_i(\{2n/3+r^*+1, \ldots, 2n-r^*\}) \leq 2n/3-r^*$. Also since $v_i(r^*) < 2/3+x$, $v_i(\{r^*+1, \ldots, 2n/3+r^*\}) \leq \frac{2n}{3}(\frac{2}{3}+x)$. In total, we get
    \begin{align*}
        \sum_{j > r^*} v_i(\hat{B}_j) &= \sum_{j > r^*} v_i(B_j) \\
        &= v_i(\{r^*+1, \ldots, 2n/3+r^*\}) + v_i(\{2n/3 +r^*+1, \ldots, 2n-r^*\}) \\
        &< \frac{2n}{3}(\frac{2}{3}+x) + \frac{2n}{3}-r^* \\
        &= \frac{10n}{9}-r^* + \frac{2nx}{3}.
    \end{align*}
    Therefore, Claim \ref{claim-r-star} holds. 
    \end{claimproof}
    We have 
    \begin{align*}
        \frac{4n}{3} &= v_i(M) \\
        &= \sum_{j \leq r^*} v_i(\hat{B}_j) + \sum_{j > r^*} v_i(\hat{B}_j) \\
        &< r^*(\frac{4}{3} -x) + \frac{10n}{9}-r^* + \frac{2nx}{3} &\tag{Observation \ref{upper-bound-left} and Claim \ref{claim-r-star}}\\
        &= r^*(\frac{1}{3}-x) + \frac{10n}{9} + \frac{2nx}{3}.
    \end{align*}
    Thus, 
    \begin{align*}
        \frac{2n}{9} &< r^*(\frac{1}{3}-x) + \frac{2n}{3}(x) \\
        &\leq \frac{2n}{3} \cdot \frac{1}{3}, \tag{$r^* \leq 2n/3$ by Lemma \ref{r-star-bound}}
    \end{align*}
    which is a contradiction. Hence, $v_i(2n-r^*+1) > 1/3$.
\end{proof}

Recall that $\ell^*$ be the smallest such that $v_i(B_{\ell^*+1}) < 1$, i.e., $B_{\ell^*+1}$ is the leftmost bag in Figure \ref{fig:bags} with value less than $1$ to agent $i$. Let $\ell$ be largest such that $v_i(B_\ell)<1$ and $v_i(2n- \ell +1) \leq 1/3$. Since $v_i(B_{\ell^*+1})<1$ and $v_i(2n- \ell^*) < 1/3$, such an index exists and $\ell \ge \ell^*+1$. Also, let $r$ be smallest such that $v_i(B_{r+1})<1$ and $v_i(2n-r) \geq 1/3$. Again, since $v_i(B_{r^*})<1$ and $v_i(2n- r^*+1) > 1/3$, such an index exists. We set $x := 1/3 - v_i(2n- \ell +1)$ and $y := v_i(2n-r) - 1/3$. See Figure \ref{r-l-fig}.
\begin{observation}\label{lessThan13}
    $x < 1/3$.
\end{observation}
\begin{proof}
    Towards a contradiction, assume $x=1/3$. Therefore, $v_i(2n-\ell+1)=0$. Let $k < 2n-\ell+1$ be the number of goods with a value larger than $0$ to agent $i$. Consider $(P^i_1 \cap [k], \ldots, P^i_{4n/3} \cap [k])$. There are at least $\ell$ many indices $j$ such that, $|P^i_j \cap [k]|=1$. Since $\mathcal{I}$ is $4n/3$-normalized, $v_i(1) = \ldots = v_i(\ell)=1$ which is a contradiction with $v_i(B_{\ell^*+1})<1$.
\end{proof}

\begin{observation}\label{y-bound}
    $y< 1/6$.
\end{observation}
\begin{proof}
    We have $1/3 + y = v_i(2n-r) \leq v_i(B_r)/2 < 1/2$. Thus, $y<1/6$.
\end{proof}
% Figure environment removed
\begin{corollary}[of Observation \ref{upper-bound-left}]\label{cor-upper-left}
    For all $j \leq \ell$, $v_i(\hat{B}_j) \leq 4/3-x$.    
\end{corollary}
\begin{corollary}[of Observation \ref{upper-bound-right}]\label{cor-upper-right}
    For all $j>r$, $v_i(\hat{B}_j) \leq \max(4/3-x, 4/3-2y)$.
\end{corollary}
\begin{observation}\label{middle-block}
    For all $\ell < j \leq r$, $1 \leq v_i(\hat{B}_j) < 1+x+y$.
\end{observation}
\begin{proof}
    Note that by definition of $\ell$ and $r$, for all $\ell < j \leq r$, $v_i(B_j) \geq 1$. Therefore, $\hat{B}_j=B_j$. Also, 
    \begin{align*}
        v_i(B_j) &= v_i(j) + v_i(2n-j+1) \\
        &\leq v_i(\ell) + v_i(2n-r) \tag{$\ell < j$ and $2n-r < 2n-j+1$}\\
        &< (\frac{2}{3}+x) + (\frac{1}{3}+y) \tag{$v_i(B_\ell)<1$ and $v_i(B_{r+1})<1$}\\
        &= 1+x+y.
    \end{align*}
\end{proof}
\begin{lemma}
    $r-\ell > 2n/3$.
\end{lemma}
\begin{proof}
    If $x+y \leq 1/3$, then by Corollaries \ref{cor-upper-left} and \ref{cor-upper-right} and Observation \ref{middle-block}, for all $t \in [n]$ we have $v_i(\hat{B}_t) \leq 4/3$ and for at least one bag this value is less than $1$ by Observation \ref{less-than-one}. Therefore, $v_i(M) < 4n/3$, which is a contradiction. Thus, $x+y>1/3$. 
    We have
    \begin{align*}
        \frac{4n}{3} &= v_i(M) \\
        &= \sum_{j \leq \ell} v_i(\hat{B}_j) + \sum_{\ell < j \leq r} v_i(\hat{B}_j) + \sum_{j > r} v_i(\hat{B}_j) \\
        &\leq \ell(\frac{4}{3}-x) + (r-\ell)(1+x+y) + (n-r) \max(\frac{4}{3}-x, \frac{4}{3}-2y) \\
        &\leq (r-\ell)(1+x+y) + (n-r+\ell) \max(\frac{4}{3}-x, \frac{4}{3}-2y) \tag{Corollaries \ref{cor-upper-left} and \ref{cor-upper-right} and Observation \ref{middle-block}}\\
        &= \frac{4n}{3} + (r-\ell)(x+y-\frac{1}{3}) - (n-r+\ell)\min(x,2y).
    \end{align*}
    Therefore, $(r-\ell)(x+y-1/3) \geq (n-r+\ell)\min(x,2y)$. By Observation \ref{lessThan13}, $x<1/3$ and thus, we have $x+y-1/3 < y$. Also, since $y < 1/6$ (by Observation \ref{y-bound}), we have $x+y-1/3 < x-1/6 < x/2$. Thus, $x+y-1/3 < \min(x,2y)/2$. Hence, $r-\ell > 2(n-r+\ell)$ and therefore, $r-\ell > 2n/3$.
\end{proof}
Let $r - \ell = 2n/3 + s$. Recall that $P^{i} = (P^{i}_1, \ldots, P^{i}_{4n/3})$ is an $(4n/3)$-MMS partition of $M$ for agent $i$. Since $i$ is fixed, we use $P = (P_1, \ldots, P_{4n/3})$ instead for ease of notation. For all $j \in [4n/3]$, let $g_j$ be good with the smallest index (and hence the largest value) in $P_j$. Without loss of generality, assume $g_1 < g_2 < \ldots < g_{4n/3}$. Observe that $\{1,\dots,r\} \subseteq \cup_{k\in [r]} P_k$. Let $S'$ be the set of goods in $\{r+1, \ldots, 2n-\ell\}$ that appear in the first $r$ bags in $P$. Formally, $S' = \{g \in \{r+1, \ldots, 2n-\ell\} \mid g \in \cup_{j \in [r]} P_j\}$. Let $s':=\min(|S'|,s)$. 
% Figure environment removed
\begin{restatable}{lemma}{difficultbound} \label{difficult-bound}
    $v_i(\{r-s'+1, \ldots, r\} \cup \{2n-\ell-3s+2s'+1, \ldots, 2n-\ell\}) \leq s$.
\end{restatable}
The items considered in Lemma \ref{difficult-bound} are marked with blue in Figure \ref{colorful-11}.
Before proving Lemma \ref{difficult-bound}, let us assume it holds and reach a contradiction. Since $v_i(\ell) < 1-v_i(2n-\ell+1)= 2/3+x$, we have
\begin{align}
    v_i(\{\ell+1, \ldots, r-s'\}) &< (\frac{2n}{3}+s-s')(\frac{2}{3}+x). \label{easy-1}
\end{align}
Also, since $v_i(2n-r+1) =1/3+y$,
\begin{align}
    v_i(\{2n-r+1, \ldots, 2n-\ell-3s+2s'\}) &\leq (\frac{2n}{3}-2s+2s')(\frac{1}{3}+y). \label{easy-2}
\end{align}
Therefore,
\begin{align*}
    \sum_{\ell < j \leq r} v_i(\hat{B}_j) &= \sum_{\ell < j \leq r} v_i(B_j) \\
    &= v_i(\{\ell+1, \ldots, r\} \cup \{2n-r+1, \ldots, 2n-\ell\}) \\
    &= v_i(\{\ell+1, \ldots, r-s'\}) \\
    &\indent + v_i(\{r-s'+1, \ldots, r\} \cup \{2n-\ell-3s+2s'+1, \ldots, 2n-\ell\}) \\ 
    &\indent + v_i(\{2n-r+1, \ldots, 2n-\ell-3s+2s'\}) \\
    &< (\frac{2n}{3}+s-s')(\frac{2}{3}+x) + s + (\frac{2n}{3}-2s+2s')(\frac{1}{3}+y) \tag{Inequalities \eqref{easy-1} and \eqref{easy-2} and Lemma \ref{difficult-bound}}\\
    &= \frac{2n}{3}(1+x+y)+(s-s')(x-2y)+s.
\end{align*}
Thus,
\begin{align*}
    \frac{4n}{3} &= v_i(M) \\
    &= \sum_{j \leq \ell} v_i(\hat{B}_j) + \sum_{\ell < j \leq r} v_i(\hat{B}_j) + \sum_{j>r} v_i(\hat{B}_j) \\
    &< (\ell+n-r)\max(\frac{4}{3}-x, \frac{4}{3}-2y) + \frac{2n}{3}(1+x+y)+(s-s')(x-2y)+s \tag{Corollaries \ref{cor-upper-left} and \ref{cor-upper-right}} \\
    &= (\frac{n}{3}-s)\max(\frac{4}{3}-x, \frac{4}{3}-2y) + \frac{2n}{3}(1+x+y)+(s-s')(x-2y)+s.
\end{align*}
If $x \leq 2y$, then by replacing $\max(4/3-x, 4/3-2y)$ with $4/3-x$ in the above inequality, we get
\begin{align*}
    \frac{4n}{3} &< (\frac{n}{3}-s)(\frac{4}{3}-x) + \frac{2n}{3}(1+x+y)+(s-s')(x-2y)+s \\
    &\leq \frac{n}{3}(\frac{4}{3}-x) + \frac{2n}{3}(1+x+y)+(s-s')(x-2y) \tag{$4/3-x \geq 1$}\\
    &\leq \frac{n}{3}(\frac{10}{3}+x+2y) \tag{$(s-s')(x-2y) \leq 0$} \\
    &< \frac{4n}{3}, \tag{$x \leq 1/3$ and $y<1/6$}
\end{align*}
which is a contradiction. If $2y < x$, by replacing $\max(4/3-x, 4/3-2y)$ with $4/3-2y$, we get 
\begin{align*}
    \frac{4n}{3} &< (\frac{n}{3}-s)(\frac{4}{3}-2y) + \frac{2n}{3}(1+x+y)+(s-s')(x-2y)+s \\
    &= \frac{n}{3}(\frac{10}{3}+2x) - s(\frac{1}{3}-x) - s'(x-2y) \\
    &\leq \frac{n}{3}(\frac{10}{3}+2x) &\tag{$x \leq 1/3$ and $x>2y$} \\
    &\leq \frac{4n}{3}, \tag{$x \leq 1/3$}
\end{align*} 
which is again a contradiction. Therefore, it is not possible that $v_i(2n-\ell^*) < 1/3$. Thus, Theorem \ref{contradict-1} follows.
\contradictOne*

It only remains to prove Lemma \ref{difficult-bound}. The main idea is as follows. Recall that $s'=\min(|S'|,s)$. We consider two cases for $s'$. If $s' = s$, then in order to prove Lemma \ref{difficult-bound}, we must prove $$v_i(\{r-s'+1, \ldots, r\} \cup \{2n-\ell - s'+1, \ldots, 2n-\ell\}) \leq s',$$
    which is what we do in Claim \ref{claim-2}.
    In case $s' = |S'|$, we prove 
    $$v_i(\{r-s'+1, \ldots, r\}) + v_i(S') \leq s'$$ 
    in Claim \ref{claim-22} and 
    $$v_i (\{2n-\ell-3s+2s'+1, \ldots, 2n-\ell\}) - v_i(S') \leq s-s'$$ 
    in Claim \ref{claim-3}. Adding the two sides of the inequalities implies Lemma \ref{difficult-bound}. We prove this lemma in Section \ref{proof-sec-1}.
    
\subsubsection{Proof of Lemma \ref{difficult-bound}}\label{proof-sec-1}
\difficultbound*
    Note that $\{1, \ldots, r\} \cup S' \subseteq P_1 \cup \ldots \cup P_r$. 
    For $j \in [r]$, let $Q_j = P_j \cap (\{1, \ldots, r\} \cup S')$. We begin with proving the following claim. %\ref{claim-1}.
    \begin{claim}\label{claim-1}
        There are $s'$ many sets like $Q_{j_1}, \ldots, Q_{j_{s'}}$ such that $|\cup_{k \in [s']} Q_{j_k}| \geq 2s'$ and $|\cup_{k \in [s']} Q_{j_k} \cap \{1, \ldots, r\}| \geq s'$. 
    \end{claim}
    \begin{proof}
        If $s'=0$, the claim trivially holds. Thus, assume $s' \geq 1$. By induction, we prove that for any $t \leq s'$, there are $t$ many sets like $Q_{j_1}, \ldots, Q_{j_t}$ such that $|\cup_{k \in [t]} Q_{j_k}| \geq 2t$ and $|\cup_{k \in [t]} Q_{j_k} \cap \{1, \ldots, r\}| \geq t$. 
        \paragraph{\boldmath Induction basis: $t=1$.} If there exists $Q_k$ such that $|Q_k \cap \{1, \ldots, r\}| \geq 2$, let $j_1 = k$. Otherwise, for all $k \in [r]$, we have $|Q_k \cap \{1, \ldots, r\}| = 1$. Since $s' \geq 1$, there must be an index $k$ such that $|Q_k \cap S'| \geq 1$. Let $j_1=k$.
            
        \paragraph{\boldmath Induction assumption:} There are $t$ many sets like $Q_{j_1}, \ldots, Q_{j_t}$ such that $|\cup_{k \in [t]} Q_{j_k}| \geq 2t$ and $|\cup_{k \in [t]} Q_{j_k} \cap \{1, \ldots, r\}| \geq t$. 

        Now for $t+1 \leq s'$, we prove that there are $t+1$ many sets like $Q_{j_1}, \ldots, Q_{j_{t+1}}$ such that $|\cup_{k \in [t+1]} Q_{j_k}| \geq 2t+2$ and $|\cup_{k \in [t+1]} Q_{j_k} \cap \{1, \ldots, r\}| \geq t+1$. 
        \paragraph{\boldmath Case 1: $|\cup_{k \in [t]} Q_{j_k}| \geq 2t+2$:} If $|\cup_{k \in [t]} Q_{j_k} \cap \{1, \ldots, r\}| \geq t+1$, set $j_{t+1}=k$ for an arbitrary $k \in [r] \setminus \{j_1, \ldots, j_t\}$. Otherwise, set $j_{t+1}=k$ for an index $k \in [r] \setminus \{j_1, \ldots, j_t\}$ such that $|Q_{k} \cap \{1, \dots, r\}| \ge 1$.
        \paragraph{\boldmath Case 2: $|\cup_{k \in [t]} Q_{j_k}| = 2t+1$:} If there exists $k \in [r] \setminus \{j_1, \ldots, j_t\}$, such that $|Q_k \cap [r]| \geq 1$, set $j_{t+1}=k$. Otherwise, set $j_{t+1}=k$ for any $k \in [r] \setminus \{j_1, \ldots, j_t\}$ such that $|Q_k| \geq 1$. Since $|\cup_{j \in [r]} Q_j| \geq r+s' > 2t+1$, such $k$ exists.
        \paragraph{\boldmath Case 3. $|\cup_{k \in [t]} Q_{j_k}| = 2t$ and $|\cup_{k \in [t]} Q_{j_k} \cap \{1, \ldots, r\}| \geq t+1$:} $|\cup_{k \in [r] \setminus \{j_1, \ldots, j_t\}} Q_{j_k}| \geq r+s'-2t > r-t$. Therefore, by pigeonhole principle, there exists an index $k \in [r] \setminus \{j_1, \ldots, j_t\}$ such that $|Q_k| \geq 2$. Set $j_{t+1}=k$.
        \paragraph{\boldmath Case 4. $|\cup_{k \in [t]} Q_{j_k}| = 2t$ and $|\cup_{k \in [t]} Q_{j_k} \cap \{1, \ldots, r\}| = t$:} If there exists $k \in [r] \setminus \{j_1, \ldots, j_t\}$, such that $|Q_k \cap [r]| \geq 2$, set $j_{t+1}=k$. Otherwise, for all $k \in [r] \setminus \{j_1, \ldots, j_t\}$, $|Q_k \cap [r]|=1$ since $|\cup_{k \in [t]} Q_{j_k} \cap \{1, \ldots, r\}| = t$ and $|\cup_{k \in [r]} Q_{j_k} \cap \{1, \ldots, r\}| = r$. Set $j_{t+1}=k$ for any $k \in [r] \setminus \{j_1, \ldots, j_t\}$, such that $|Q_k \cap S'| \geq 1$. Since $|\cup_{j \in [r]} Q_j \cap S'| \geq s' > t$, such $k$ exists.
    \end{proof}
Now we prove Claim \ref{claim-2}.
\begin{claim}\label{claim-2}
    $v_i(\{r-s'+1, \ldots, r\} \cup \{2n-\ell - s'+1, \ldots, 2n-\ell\}) \leq s'$.
\end{claim}
\begin{proof}
    Let $Q^1$ be the set of $s'$ most valuable goods in $\cup_{k \in [s']} Q_{j_k}$ and let $Q^2$ be the set of $s'$ least valuable goods in $\cup_{k \in [s']} Q_{j_k}$. Since $|\cup_{k \in [s']} Q_{j_k}| \geq 2s'$, $Q^1 \cap Q^2 = \emptyset$. Also, $|\cup_{k \in [s']} Q_{j_k} \cap \{1, \ldots, r\}| \geq s'$. Thus, $v_i(Q^1) \geq v_i(\{r-s'+1, \ldots, r\})$. Moreover, $v_i(Q^2) \geq v_i(\{2n-\ell - s'+1, \ldots, 2n-\ell\})$. Hence,
    \begin{align*}
        s' &= \sum_{k \in [s']} v_i(P_{j_k}) \notag\\
        &\geq \sum_{k \in [s']} v_i(Q_{j_k}) \notag \\
        &\geq v_i(\{r-s'+1, \ldots, r\} \cup \{2n-\ell - s'+1, \ldots, 2n-\ell\}). 
    \end{align*}
\end{proof}
Note that in case $s' = s$, Claim \ref{claim-2} implies Lemma \ref{difficult-bound}. Therefore, from now on, we assume $s' = |S'| < s$.
\begin{claim}\label{claim-22}
    $v_i(\{r-s'+1, \ldots, r\}) + v_i(S') \leq s'$.
\end{claim}
\begin{proof}
    The proof is similar to the proof of Claim \ref{claim-2}. Let $Q^1$ be the set of $s'$ most valuable goods in $\cup_{k \in [s']} Q_{j_k}$ and let $Q^2$ be the set of $s'$ least valuable goods in $\cup_{k \in [s']} Q_{j_k}$. Since $|\cup_{k \in [s']} Q_{j_k}| \geq 2s'$, $Q^1 \cap Q^2 = \emptyset$. Also, $|\cup_{k \in [s']} Q_{j_k} \cap \{1, \ldots, r\}| \geq s'$. Thus, $v_i(Q^1) \geq v_i(\{r-s'+1, \ldots, r\})$. Moreover, $v_i(Q^2) \geq v_i(S')$ since $s' = |S'|$. Hence,
    \begin{align*}
        s' &= \sum_{k \in [s']} v_i(P_{j_k}) \notag\\
        &\geq \sum_{k \in [s']} v_i(Q_{j_k}) \notag \\
        &\geq v_i(\{r-s'+1, \ldots, r\} \cup S')\\
        &= v_i(\{r-s'+1, \ldots, r\}) + v_i(S').
    \end{align*}
\end{proof}
\begin{claim}\label{claim-3}
    $v_i (\{2n-\ell-3s+2s'+1, \ldots, 2n-\ell\}) - v_i(S') \leq s-s'.$
\end{claim}
\begin{proof}
    Note that by definition of $S'$, the $2n-\ell-r-s'=8n/3-2r+s-s'$ goods in $\{r+1, \ldots, 2n-\ell\} \setminus S'$ are in $P_{r+1} \cup \ldots \cup P_{4n/3}$. Now for $j \in [4n/3 - r]$, let $R_j = P_{j+r} \cap \{r+1, \ldots, 2n-\ell\} \setminus S'$. Assume $|R_{j_1}| \geq \ldots \geq |R_{j_{4n/3-r}}|$. We prove 
    \begin{align}
        \sum_{k \leq s-s'} |R_{j_k}| \geq 3(s-s'). \label{R-bound}    
    \end{align}
    If $|R_{j_{s-s'+1}}| \geq 3$, Inequality \eqref{R-bound} holds. Otherwise, we have 
    \begin{align*}
        \frac{8n}{3}-2r+s-s' &= \sum_{k \in [4n/3-r]} |R_{j_k}| \\
        &= \sum_{k \leq s-s'} |R_{j_k}| + \sum_{s-s' < k \leq 4n/3-r} |R_{j_k}| \\
        &\leq \sum_{k \leq s-s'} |R_{j_k}| + 2(\frac{4n}{3}-r-s+s'). \tag{$|R_{j_k}| \leq 2$ for $k > s-s'$}
    \end{align*}
    Thus, $\sum_{k \in [s-s']} |R_{j_k}| \geq 3(s-s')$. We have
    \begin{align*}
        s-s' &= \sum_{k \in [s-s']} v_i(P_{j_k+r}) \notag\\
        &\geq \sum_{k \in [s-s']} v_i(R_{j_k}) \notag\\
        &\geq v_i (\{2n-\ell-3s+2s'+1, \ldots, 2n-\ell\}) - v_i(S'). \tag{$\lvert \cup_{k \in [s-s']} R_{j_k} \rvert \geq 3(s-s')$ and $\lvert S' \rvert =s'$}
    \end{align*}
\end{proof}
Claims \ref{claim-22} and \ref{claim-3} imply Lemma \ref{difficult-bound}.

\paragraph{Recap:} To show that a $1$-out-of-$(4n/3)$ MMS allocation exists, it suffices to prove that we never run out of goods for bag-filling in Algorithm \ref{algo}. Towards contradiction, we assumed that the algorithm stops before agent $i$ receives a bundle. By Observation \ref{less-than-one}, a bag with a value less than $1$ for agent $i$ exists. Let $\ell^*$ be the smallest such that $v_i(B_{\ell^*+1}) < 1$. In Section \ref{negative}, we reached a contradiction assuming $v_i(2n-\ell^*)<1/3$ and proved Theorem \ref{contradict-1}. In Section \ref{positive}, we reached a contradiction assuming $v_i(2n-\ell^*) \geq 1/3$ and proved Theorem \ref{contradict-2}. Therefore, no such agent $i$ exists, and all agents receive a bag by the end of Algorithm \ref{algo}. Theorem \ref{thm:main} follows.

