\section{Preliminaries}\label{sec:prelim}
For all $t \in \mathbb{N}$, we denote the set $\{1, 2, \ldots, t\}$ by $[t]$.
A discrete fair division instance is denoted by $(N, M, \mathcal{V})$, where $N=[n]$ is the set of $n$ agents, $M=[m]$ is the set of $m$ indivisible goods and $\mathcal{V}=(v_1, \ldots, v_n)$ is the vector of agents' valuation function. For all $i \in [n]$, $v_i: 2^M \rightarrow \mathit{R}_{\geq 0}$ is the valuation function of agent $i$ over all subsets of the goods. In this paper, we assume $v_i(\cdot)$ is additive for all $i \in [n]$, i.e., $v_i(S)=\sum_{g \in S} v_i(\{g\})$ for all $S \subseteq M$. For ease of notation, we also use $v_i(g)$ and $v_{i,g}$ instead of $v_i(\{g\})$.

For a set $S$ of goods and any positive integers $d$, let $\Pi_d(S)$ denote the set of all partitions of $S$ into $d$ bundles. Then, 
\begin{align}
    \MMS_i^d(S) := \max_{P \in \Pi_d(S)} \min_{j=1}^d v_i(P_j). \label{MMS-def}    
\end{align}

Setting $d=n$, we obtain the standard MMS notion. Formally, $\MMS_i = \MMS^n_i(M)$. An allocation $X=\langle X_1, \ldots, X_n \rangle$ is a partition of the goods into $n$ bundles such that for all $i \in [n]$ agent $i$ receives $X_i$. We call an allocation $X$ \emph{$1$-out-of-$d$ MMS}, if for all agents $i$, $v_i(X_i) \geq \MMS^d_i(M)$. For each agent $i$, $d$-MMS partition of $i$ is a partition $P=(P_1, \ldots, P_d)$ of $M$ into $d$ bundles such that $\min_{j=1}^d v_i(P_j)$ is maximized. Basically, for a $d$-MMS partition $P$ of agent $i$, $\MMS_i^d(M) = \min_{j=1}^d v_i(P_j)$. Whenever $d$ is not an integer, by $1$-out-of-$d$ MMS, we mean $1$-out-of-$\lceil d \rceil$ MMS.\footnote{This is without loss of generality because we prove the existence of $1$-out-of-$4n/3$ MMS assuming that $4n/3$ is an integer. If $4n/3$ is not an integer, we can copy one of the agents $1$ or $2$ times (depending on $4n \text{ mod } 3$) so that the new instance has $n'$ agents and $4n'/3$ is an integer. Note that $\lceil 4n/3 \rceil = 4n'/3$. Since we prove the existence of $1$-out-of-$4n'/3$ MMS for the new instance, we prove the existence of an allocation that gives all the agents $i$ in the original instance their $\MMS^{4n'/3}_i(M)=\MMS^{\lceil 4n/3 \rceil}_i(M)$ value. Hence, the existence of $1$-out-of-$\lceil 4n/3 \rceil$ MMS allocations follows.} In the rest of the paper, for each agent $i$, we denote a $d$-MMS partition of $i$ by $P^i = (P^i_1, \ldots, P^i_d)$. 

An allocation $X = \langle X_1, \ldots, X_n \rangle$ is a partition of the goods into $n$ bundles such that for all $i \in [n]$, agent $i$ receives bundle $X_i$.
\begin{definition}
    Given an instance $\mathcal{I}=(N,M,\mathcal{V})$, for $0 < \alpha \leq 1$, $\beta > \gamma \geq 0$, an allocation $X$ is $(\alpha, \beta, \gamma)$-MMS if for any partition of the agents into two parts of $N_1$ and $N_2$ such that $|N_1| \leq \alpha n$ and $N_2=N \setminus N_1$, for all $i \in N_1$, $v_i(X_i) \geq \beta \cdot \MMS_i$ and for all $j \in N_2$, $v_j(X_j) \geq \gamma \cdot \MMS_j$.
\end{definition}
\begin{lemma}
    For integers $d \geq n$, $1$-out-of-$d$ MMS allocations exist for all instances with $n$ agents if and only if $(n/d,1,0)$-MMS allocations exist for all instances with $d$ agents.
\end{lemma}
\begin{proof}
    First, assume $1$-out-of-$d$ MMS allocations exist for all instances with $n$ agents. Consider any instance $\mathcal{I}= (N, M, \mathcal{V})$ with $d$ agents. Take any arbitrary partition of the agents into $N_1$ and $N_2$ such that $|N_1|=n$. Consider the instance $\mathcal{I}' = (N_1, M, \mathcal{V}_1)$ where $\mathcal{V}_1$ is the valuation profile of agents in $N_1$. Since $1$-out-of-$d$ MMS allocations exist for $\mathcal{I}'$, there exists an allocation that satisfies any $n/d$ fraction of the agents in $\mathcal{I}$ with their MMS value. 

    Now assume $(n/d,1,0)$-MMS allocations exist for all instances with $d$ agents. Consider any instance $\mathcal{I}= (N, M, \mathcal{V})$ with $n$ agents. Add $d-n$ dummy agents with arbitrary additive valuations. Let $\mathcal{I}'= (N', M, \mathcal{V}')$ be the resulting instance. Since $(n/d,1,0)$-MMS allocations exist for $\mathcal{I}'$, for any $n/d$ fraction of the agents in $N'$ and in particular for the agents in $N$, there exists an allocation that gives them their MMS value in $\mathcal{I}'$ which is equal to their $1$-out-of-$d$ MMS value in $\mathcal{I}$. 
\end{proof}
\begin{definition}
    An instance $\mathcal{I}=(N, M, \mathcal{V})$ is \emph{ordered} if there exists an ordering $[g_1, \ldots, g_m]$ of the goods such that for all agents $i$, $v_i(g_1) \geq \ldots \geq v_i(g_m)$.
\end{definition}
\begin{definition}
    An instance $\mathcal{I}=(N, M, \mathcal{V})$ is $d$-normalized if for all agents $i$, there exists a partition $P = (P_1, \ldots, P_d)$ of $M$ into $d$ bundles such that $v_i(P_j)=1$ for all $j \in [d]$. 
\end{definition}
\iffalse
\begin{lemma}
    For any $d \in \mathbb{N}$, if $1$-out-of-$d$ MMS allocations exist for $d$-normalized instances, then $1$-out-of-$d$ MMS allocations exist for all instances.
\end{lemma}
\begin{proof}
    Let $\mathcal{I}=(N,M,\mathcal{V})$ be an arbitrary instance. We construct a $d$-normalized instance $\mathcal{I}'=(N,M,\mathcal{V}')$ such that if a $1$-out-of-$d$ MMS allocation exists for $\mathcal{I}'$, then a $1$-out-of-$d$ MMS allocation exists for $\mathcal{I}$ as well. For all $i \in N$, let $P^i=(P^i_1, \ldots, P^i_d)$ be a $d$ MMS partition of agent $i$. For all $g \in M$, let $f(g)$ be such that $g \in P^i_{f(g)}$. Then set $v'_i(g) := v_i(g)/v_i(P^i_{f(g)})$. Then $v'_i(P^i_j)=1$ for all $j\in[d]$. Therefore, $\mathcal{I}'$ is a $d$-normalized instance. Now let $X$ be a $1$-out-of-$d$ MMS allocation for $\mathcal{I}'$. For all $i \in N$, we have
    \begin{align*}
        v_i(X_i) &= \sum_{g \in X_i} v_i(g) \\
        &= \sum_{g \in X_i} v'_i(g) v_i(P^i_{f(g)}) \tag{$v'_i(g) := v_i(g)/v_i(P^i_{f(g)})$} \\
        &\geq \MMS^d_i \sum_{g \in X_i} v'_i(g) \tag{$v_i(P^i_j) \geq \MMS^d_i$ for all $j \in [d]$} \\
        &= \MMS^d_i. \tag{$\sum_{g \in X_i} v'_i(g)=1$ since $X$ is a $d$ MMS partition of $\mathcal{I}'$}
    \end{align*}
    Thus, $X$ is a $1$-out-of-$d$ MMS allocation for $\mathcal{I}$.
\end{proof}
\fi
Barman and Krishnamurthy \cite{barman2020approximation} proved that when the goal is to guarantee a minimum threshold of $\alpha_i$ for each agent $i$, it is without loss of generality to assume the instance is ordered.
Akrami et al. \cite{simple} proved that when the goal is to find an approximate MMS allocation, it is without loss of generality to assume the instance is $n$-normalized and ordered. Their proof does not rely on the number of agents. Formally, for any $d \in \mathbb{N}$, when the goal is to find a $1$-out-of-$d$ MMS allocation, it is without loss of generality to assume the instance is ordered and $d$-normalized, as shown in the following lemma (proof is in Appendix~\ref{app}). %For completeness, we prove this in the appendix. \todo{...}

\begin{restatable}{lemma}{ordNorm}\label{ord-norm}
    For any $d \in \mathbb{N}$, if $1$-out-of-$d$ MMS allocations exist for $d$-normalized ordered instances, then $1$-out-of-$d$ MMS allocations exist for all instances.
\end{restatable}

From now on, even if not mentioned, we assume the instance is ordered and $d$-normalized. Without loss of generality, for all $i \in [n]$, we assume $v_i(1) \geq v_i(2) \geq \ldots \geq v_i(m)$. In Section \ref{prelim-1}, we prove some properties of ordered $d$-normalized instances for arbitrary $d$. In Section \ref{sec:4n-3}, we set $d=4n/3$ and prove $1$-out-of-$(4n/3)$ MMS allocations always exist. 

\subsection{1-out-of-d MMS}\label{prelim-1}
Recall that for a given instance $\mathcal{I}$ and integer $d$, for each agent $i$, $P^i = (P^i_1, \ldots, P^i_d)$ is a $d$-MMS partition of agent $i$.
\begin{proposition}\label{prop:trivial}
    Given a $d$-normalized instance for all $i \in N$ and $k \in [d]$, we have
    \begin{enumerate}
        \item $v_i(P^i_k)=1$, and
        \item $v_i(M)=d$.
    \end{enumerate}
\end{proposition}
We note that it is without loss of generality to assume $m \geq 2d$. Otherwise, we can add $2d-m$ dummy goods with a value of $0$ for all the agents. The normalized and ordered properties of the instance would be preserved.
Consider the bag setting with $d$ bags as follow.
\begin{equation}
    \label{eq:C_i}
    C_k := \{k , 2d-k+1\} \text{ for } k\in [d]
\end{equation}
See Figure \ref{c-bags} for more intuition. Next, we show some important properties of the values of the goods in $C_k$'s.
% Figure environment removed
\begin{proposition}\label{prop:exact}
    For all agents $i \in N$, we have
    \begin{enumerate}
        \item $v_i(1) \leq 1$, \label{exact:1}
        \item $v_i(C_d) \leq 1$, and \label{exact:2}
        \item $v_i(d+1) \leq \tf12$. \label{exact:3}
    \end{enumerate}
\end{proposition}
\begin{proof}
For the first part, 
    fix an agent $i$. Let $1 \in P^i_1$. By Proposition \ref{prop:trivial}, $v_i(1) \leq v_i(P^i_1)=1$. 

    For the second part, by the pigeonhole principle, there exists a bundle $P^i_k$ and two goods $j, j' \in \{1, 2, \ldots, d+1\}$ such that $\{j,j'\} \subseteq P^i_k$. Without loss of generality, assume $j < j'$. We have 
    \begin{align*}
        v_i(C_d) &= v_i(d) + v_i(d+1) \tag{$C_d = \{d, d+1\}$}\\
        &\leq v_i(j) + v_i(j') &\tag{$j \leq d$ and $j' \leq d+1$} \\
        &\leq v_i(P^i_k) =1. &\tag{$\{j,j'\} \in P^i_k$}
    \end{align*}

    For the third part, we have
    \begin{align*}
        1 &\geq v_i(C_d) = v_i(d) + v_i(d+1) \geq 2 v_i(d+1).
    \end{align*}
    Thus, $v_i(d+1) \leq \tf12$.
\end{proof}
\begin{lemma}
\label{lem:Csum}
For all $i \in N$ and $k \in [d]$, $\sum_{j=k}^d v_i(C_j) \le d-k+1$.
\end{lemma}
\begin{proof}
For the sake of contradiction, assume the claim does not hold for some agent $i$ and let $\ell \geq 1$ be the largest index for which we have $\sum_{j = \ell}^d v_i(C_j) > d - \ell + 1$. Proposition \ref{prop:exact}(\ref{exact:2}) implies that $\ell < d$.
We have
\begin{align*}
    v_i(\ell) + v_i(2d-\ell+1) &= v_i(C_\ell) \\
    &= \sum_{j=\ell}^d v_i(C_j) - \sum_{j=\ell+1}^d v_i(C_j) \\
    &>  (d - \ell + 1) - (d - (\ell+1) + 1) &\tag{$\sum_{j = k}^d v_i(C_j) \leq d - k + 1$ for $k>\ell$} \\
    &=1. 
\end{align*}
 For all $j,j' < \ell$, $v_i(j) + v_i(j') \geq v_i(\ell) + v_i(2d-\ell+1) > 1$. Therefore, $j$ and $j'$ cannot be in the same bundle in any $d$-MMS partition of $i$. For $j<\ell$, let $j \in P^i_j$. For all $j<\ell$ and $\ell \leq j' \leq 2d-\ell+1$, 
 \begin{align*}
    v_i(j) + v_i(j') &\geq v_i(\ell) + v_i(2d-\ell+1) \\
    &= v_i(C_{\ell}) >1.    
 \end{align*}
 Therefore, $j' \notin P_j$. Also, since $\sum_{j = \ell}^d v_i(C_j) > d - \ell + 1$, there are at least $t \geq d-\ell+2$ different bundles $Q_1, \ldots, Q_t$ in $P$ such that $Q_j \cap \{\ell, \ldots, 2d-\ell+1\} \neq \emptyset$. It is a contradiction since these $t \geq d-\ell+2$ bundles must be different from $P^i_1, \ldots P^i_{\ell-1}$.
\end{proof}
