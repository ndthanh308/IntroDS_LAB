\appendix
\section{Missing Proofs}\label{app}
\ordNorm*
\begin{proof}
    Let $\ins$ be an arbitrary instance. We create a $d$-normalized ordered instance $\mathcal{I}'' = (N, M, V'')$ such that from any $1$-out-of-$d$ MMS allocation for $\mathcal{I}''$, one can obtain a $1$-out-of-$d$ MMS allocation for the original instance $\mathcal{I}$. 
    
    First of all, we can ignore all agents $i$ with $\MMS^d_i=0$ since no good needs to be allocated to them. Recall that for all $i \in N$, $P^i = (P^i_1, \ldots, P^i_d)$ is a $d$-MMS partition of agent $i$. For all $i \in N$ and $g \in M$, we define $v'_{i,g} = v_i(g)/v_i(P^i_j)$ where $j$ is such that $g \in P^i_j$. Now for all $i \in N$, let $v'_i: 2^M \rightarrow \mathbb{R}_{\geq 0}$ be defined as an additive function such that $v'_i(S)=\sum_{g \in S} v'_{i,g}$.  Note that $v'_{i,g} \leq v_i(g)/\MMS^d_i(M)$ for all $g \in M$ and thus, 
    \begin{align}
        v_i(S) \geq v'_i(S) \cdot \MMS^d_i(M). \label{ineq-apx}
    \end{align}
    Since $v'_i(P^i_j)=1$ for all $i \in N$ and  $j \in [d]$, $\mathcal{I}' = (N, M, V')$ is a $d$-normalized instance. If a $1$-out-of-$d$ MMS allocation exists for $\mathcal{I}$, let $X$ be one such allocation. By Inequality \eqref{ineq-apx}, $v_i(X_i) \geq v'_i(X_i) \cdot \MMS^d_i(M) \geq \MMS^d_i(M)$. Thus, every allocation that is $1$-out-of-$d$ MMS for $\mathcal{I}'$ is $1$-out-of-$d$ MMS for $\mathcal{I}$ as well. 
    For all agents $i$ and $g \in [m]$, let $v''_{i,g}$ be the $g$-th number in the multiset of $\{v_i(1), \ldots, v_i(m)\}$. Let $v''_i: 2^M \rightarrow \mathbb{R}_{\geq 0}$ be defined as an additive function such that $v''_i(S)=\sum_{g \in S} v''_{i,g}$. Let $\mathcal{I''} = \langle N,M, V'' \rangle$. Note that $\mathcal{I}''$ is ordered and $d$-normalized.
    Barman and Krishnamurthy \cite{barman2020approximation} proved that for any allocation $X$ in $\mathcal{I''}$, there exists and allocation $Y$ in $\mathcal{I'}$ such that $v'_i(Y_i) \geq v''_i(X_i)$. Therefore, from any $1$-out-of-$d$ MMS allocation in $\mathcal{I}''$, one can obtain a $1$-out-of-$d$ MMS allocation in $\mathcal{I'}$ and as already shown before, it gives a $1$-out-of-$d$ MMS allocation for $\mathcal{I}$.
\end{proof}

\validred*
\begin{proof} Clearly, $v_a(S) \geq 3/4$. Next, we show that the MMS values of all other agents do not decrease separately for each case of $S\in\{S_1,S_2, S_3,S_4\}$. Fix agent $b \in N \setminus \{a\}$ and a MMS partition $P^b=(P^b_1, \dots, P^b_n)$. After removing $S$, we show that a partition of $M \setminus S$ exists into $(n-1)$ bundles such that the value of each bundle is at least $1$. 
\begin{itemize}
    \item \textbf{$S=S_1$. } Removal of one item from $P^b$ affects exactly one bundle and each of the remaining $(n-1)$ bundles has value at least $1$. Therefore, the MMS value of $b$ doesn't decrease.
    \item \textbf{$S=S_2$. } In $P^b$, there exists a bundle with two items from $\{1, \dots, n+1\}$ (pigeonhole principle). Let $T$ be a bundle in $P^b$ that has two items from $\{1, \dots,n+1\}$. Let us exchange these items with items $n$ and $n+1$ in other bundles and arbitrarily distribute any remaining items in $T$ among other bundles. Clearly, the value of other bundles except $T$ does not decrease, and hence the MMS value of $b$ in the reduced instance doesn't decrease.
    \item \textbf{$S=S_3$. } Similar to the proof of Case $`S = S_2$'.
    \item \textbf{$S=S_4$.} In each iteration, the lowest index bundle from $\{S_1,S_2,S_3,S_4 \}$ is picked. Therefore, $S_4$ is only picked when $v_i(S_1),v_i(S_3) < 3/4$ for all $i \in N$ which implies that $v_{i1} < 3/4$ and $v_{i(2n+1)} < 1/4$ and hence $v_i(S_4) < 1$ for all $i\in N$.

    In $P^b$, if items $1$ and $2n+1$ are in the same bundle, removing $S_4$ and agent $a$ is a valid reduction. For the other case, if $1$ and $2n+1$ are in two different bundles, we can make two new bundles, one with $\{1,2n+1\}$ and another with all the remaining items of the two bundles. The value of the bundle without $\{1,2n+1\}$ is at least $1$ because $v_i(S_4) < 1$ for all $i\in N$ and $\MMS_i \ge 1$.
   Hence, this is a valid reduction. \qedhere
\end{itemize}
\end{proof}

