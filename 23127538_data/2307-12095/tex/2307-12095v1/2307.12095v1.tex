\documentclass[11pt]{amsart}	
\usepackage[utf8]{inputenc}


\usepackage{graphicx}
\usepackage{subcaption}
\usepackage[usenames,dvipsnames]{pstricks}
\usepackage{epsfig}
\usepackage{pst-grad} % For gradients
\usepackage{pst-plot} % For axes
\usepackage[space]{grffile} % For spaces in paths
\usepackage{etoolbox} % For spaces in paths
\makeatletter % For spaces in paths
\patchcmd\Gread@eps{\@inputcheck#1 }{\@inputcheck"#1"\relax}{}{}
\makeatother
\usepackage{comment}
\usepackage{mathtools}
\usepackage{amsmath}
\usepackage{amssymb}
\usepackage{amsthm}
\usepackage{ifthen}
\usepackage{color}
\usepackage[UKenglish]{babel}
\usepackage{polski}

\usepackage[shortlabels]{enumitem}

\numberwithin{equation}{section}



\newcommand{\intav}[1]{\mathchoice {\mathop{\vrule width 6pt height 3 pt depth  -2.5pt
\kern -8pt \intop}\nolimits_{\kern -6pt#1}} {\mathop{\vrule width
5pt height 3  pt depth -2.6pt \kern -6pt \intop}\nolimits_{#1}}
{\mathop{\vrule width 5pt height 3 pt depth -2.6pt \kern -6pt
\intop}\nolimits_{#1}} {\mathop{\vrule width 5pt height 3 pt depth
-2.6pt \kern -6pt \intop}\nolimits_{#1}}}

\def\polhk#1{\setbox0=\hbox{#1}{\ooalign{\hidewidth\lower1.5ex\hbox{`}\hidewidth\crcr\unhbox0}}}

\newcommand{\supp}{\operatorname{supp}}
\newcommand{\trace}{\operatorname{trace}}
\newcommand{\argmin}{\operatorname{argmin}}
\newcommand{\argmax}{\operatorname{argmax}}
\renewcommand{\div}{\operatorname{div}}
\newcommand{\diag}{\operatorname{diag}}
\newcommand{\osc}{\operatorname{osc}}
\newcommand{\Ll}{{\mathcal{L}}}
\newcommand{\Aaa}{{\mathcal{A}}}
\newcommand{\Tt}{{\mathbb{T}}}
\newcommand{\E}{\mathbb E}
\newcommand{\PP}{\mathbb P}
\newcommand{\Pp}{\mathcal P}
 \newcommand{\Rr}{\mathbb R}
 \newcommand{\Zz}{\mathbb Z}
 \newcommand{\mS}{\mathbb S}
 \newcommand{\cH}{\mathcal H}
 \newcommand{\cD}{\mathcal D}
 \newcommand{\cT}{\mathcal T}
 \newcommand{\cG}{\mathcal G}
 \newcommand{\cU}{\mathcal U}
 \newcommand{\cL}{\mathcal L}
 \newcommand{\cM}{\mathcal M}
 \newcommand{\cN}{\mathcal N}
 \newcommand{\bq}{\mathbf{q}}
 \newcommand{\hH}{\hat{H}}
 \newcommand{\bG}{\bar{G}}
\newcommand{\Hh}{\bar{H}}
\newcommand{\bL}{\bar{L}}
 \newcommand{\tH}{\tilde{H}}
 \newcommand{\af}{u}
 \newcommand{\fui}{\varphi}
 \newcommand{\ep}{\epsilon}
\newcommand{\be}{\beta}
 \newcommand{\ga}{\gamma}
 \newcommand{\Ga}{\Gamma}
 \newcommand{\de}{\delta}
 \newcommand{\om}{\omega}
  \newcommand{\lam}{\lambda}
 \newcommand{\te}{u}
 \newcommand{\lV}{\left\Vert}
 \newcommand{\rV}{\right\Vert}
 \newcommand{\rip}{\rangle}
 \newcommand{\lip}{\langle}
 \newcommand{\mL}{\mathbb L}
 \newcommand{\mH}{\mathbb H}
\newcommand{\Ff}{\mathbb F}
\newcommand{\Nn}{\mathbb N}

\renewcommand{\div}{\operatorname{div}}
\newcommand{\bmo}{\operatorname{BMO}}
\newcommand{\dist}{\operatorname{dist}}
\newcommand{\rot}{\operatorname{rot}}
\newcommand{\flux}{\operatorname{flux}}
\newcommand{\grad}{\operatorname{grad}}
\newcommand{\dx}{\operatorname{d}}
\newcommand{\lap}{\bigtriangleup}
\newcommand{\dom}{\operatorname{dom}}
\newcommand{\sgn}{\operatorname{sgn}}
\newcommand{\hes}{\operatorname{Hess}}
\newcommand{\Alt}{\operatorname{Alt}}
\newcommand{\entre}{\setminus}
\newcommand{\Id}{\operatorname{Id}}
\renewcommand{\dx}{\operatorname{d}}
\newcommand{\pv}{\operatorname{p.v.}}
\newcommand{\ann}{\operatorname{ann}}
\newcommand{\graf}{\operatorname{graph}}
\newcommand{\pr}{\operatorname{proj}}
\newcommand{\tr}{\operatorname{Tr}}
\newcommand{\sop}{\operatorname{supp}}
\newcommand{\dHp}{\dfrac{\partial H}{\partial p}}
\newcommand{\rec}{\operatorname{rec}}
\newcommand{\epi}{\operatorname{epi}}
\newcommand{\llip}{\operatorname{Log-Lip}}
\newcommand{\curl}{\operatorname{curl}}
\newcommand{\ess}{\operatorname{ess}}

\def\appendixname{\empty}
\def\chaptername{\empty}
\def\figurename{Fig.}
\def\abstractname{Abstract}

\newtheorem{Theorem}{Theorem}
\newtheorem{Definition}{Definition}
\newtheorem{Lemma}{Lemma}
\newtheorem{Corollary}{Corollary}
\newtheorem{Proposition}{Proposition}

\newtheorem{assump}{}
\renewcommand\theassump{[A\arabic{assump}]}


\newtheoremstyle{break}
  {\topsep}{\topsep}%
  {\itshape}{}%
  {\bfseries}{}%
  {\newline}{}%
\theoremstyle{break}
\newtheorem{lemmabreak}{Lemma}



\theoremstyle{definition}
\newtheorem{Example}{Example}
\theoremstyle{remark}
\newtheorem{Remark}{Remark}

\newcommand{\z}{\mathbb{Z}} 
\newcommand{\re}{\mathbb{R}}
\newcommand{\tn}{\mathbb{T}^N}
\newcommand{\rn}{\mathbb{R}^N}

\def\cqd {\,  \begin{footnotesize}$\square$\end{footnotesize}}
\def\cqdt {\hspace{5.6in} \begin{footnotesize}$\blacktriangleleft$\end{footnotesize}}

\usepackage{setspace}
\setstretch{1.2}

\DeclareMathOperator*{\esssup}{ess\,sup}
\DeclareMathOperator*{\essinf}{ess\,inf}
\DeclareMathOperator*{\essosc}{ess\,osc}

\title[Fully nonlinear equations with degneracy]{Fully nonlinear equations with Muckenhoupt weights and applications }

\author[D. Jesus]{David Jesus}
\address{Dipartimento di Matematica, Universit\`a di Bologna, 
Piazza di Porta San Donato 5, 40126 Bologna, Italy}{}
\email{david.jesus2@unibo.it}



\author[Y. Sire]{Yannick Sire}
\address{Department of Mathematics,
Johns Hopkins University,
3400 N. Charles Street, Baltimore, MD 21218, U.S.A.}{}
\email{ysire1@jhu.edu}





\date{}

\begin{document}
\begin{abstract}
We investigate the regularity of viscosity solutions for fully nonlinear Hessian equations with coefficients in some $A_p$ class. We prove H\"older and higher regularity under mild assumptions on the coefficients. We use approximation techniques in the spirit of Caffarelli's seminal approach. A special attention has to be paid to the degeneracy of the coefficients and we assume some natural structural assumptions which entail applications to curvature equations on conic manifolds.   
\end{abstract}


\maketitle

\tableofcontents

\section{Introduction and main results}

\subsection{Introduction}The purpose of this paper is to understand how the degeneracy/singularity of the coefficients influences the regularity of the solution of fully nonlinear equations of Hessian  type. Elliptic equations degenerating as a Muckenhoupt weight have been extensively considered, see for example \cite{Caffarelli-Silvestre_2007, Dong-Kim_2015, Fabes-Jerison-Kenig_1982, Fabes-Kenig-Serapioni_1982, Krylov_1994, Krylov_1999, Sire-Terracini-Vita_2020, Sire-Terracini-Vita_2021}. Of particular interest is the equation in divergence form
\begin{align}\label{eq_VS}
    \mathcal{L}_au:=\div\left(|y|^aA(x,y)Du\right)=f
\end{align}
which has a close relation to the fractional Laplacian $(-\Delta)^s$ for $0<s<1$ by the famous paper of Caffarelli and Silvestre \cite{Caffarelli-Silvestre_2007}. In a series of papers with Terracini, Tortone and Vita, the second author investigated thoroughly \eqref{eq_VS} both at the regularity level (Schauder estimates in particular) \cite{Sire-Terracini-Vita_2020, Sire-Terracini-Vita_2021} and geometric properties of solutions \cite{STT2020}. Notice that the homogeneous weight $|y|^a$ is very special. In particular, in the aforementioned papers, the authors consider super-degenerate or super-singular weights, i.e. beyond the $A_2$ class with $a<-1$ or $a>1$. In these special cases, the associated measure is of course doubling but the weight is lacking several properties which are of fundamental use such as boundedness of maximal functions for instance. 

As far as regularity is concerned, when the domains are upper-half spaces, more general results on the existence and regularity estimates in weighted and mixed-norm Sobolev spaces for a similar class of linear elliptic and parabolic equations can be found first in \cite{DP19} with $\alpha \in (-1, 1)$ and then in \cite{DP20} with $\alpha \in (-1, \infty)$.  Similar results for problems with homogeneous Dirichlet boundary conditions can be found in \cite{DP-2023, DP-2021}. See also  \cite{CMP} for results on $W^{1,p}$-estimates for solutions of linear elliptic equations whose coefficients can be singular or degenerate with general $A_2$-weights but with some restrictive smallness assumption on the weighted mean oscillations of the coefficients. In the previous papers, non-divergence equations have been also considered. 

In the present paper we are interested in the fully nonlinear case only.  Let $B_1\subset \Rr^d$ be the unit ball centered at the origin; we investigate the following fully nonlinear elliptic Hessian equation
\begin{align}\label{eq_var_coef_main_form}
    F(D^2u,x)=f, \mbox{ in } B_1,
\end{align}
where $F$ is degenerate elliptic with ellipticity given by $\lambda(x)=\lambda\, \omega(x)$ and $\Lambda(x)=\Lambda\, \omega(x)$ with $0<\lambda\leq \Lambda<\infty$. The weight $\omega$ captures the degeneracy of the equation which is  described in terms of an $A_p$ weight. More precisely, we say that a function $\nu$ is $A_p$ for $1 <p <\infty$, if there exists a constant $C>0$ such that, for every ball $B$, it holds
\begin{align*}
    \left(\frac{1}{|B|}\int_B \nu(x)dx\right)\left(\frac{1}{|B|}\int_B \nu^{-\frac{p'}{p}}(x)dx\right)^\frac{p}{p'}\leq C<\infty,
\end{align*}
where $1/p+1/p'=1$.

In our case, the weight $\omega$ is assumed behave in such a way that there exists  some $b\in(0,d+1)$ for which
$$\nu(x):=\omega^{-d}(x)\dist(x,\partial B_1)^b\in A_p.$$
This allows in particular to develop regularity beyond H\"olderianity.  We comment below for some motivations to consider such types of weights. 


\subsection{Main results.} Before introducing our results, we first set the notations we will use throughout the paper.
\begin{itemize}
\item[--] If $\Omega\subset \Rr^d$ is an open subset, we denote by $\mathcal{M}(\Omega)$ the space of real-valued measurable functions defined on $\Omega$.
\item[--] If $\nu\in A_p$, we define
\begin{align*}
L^q_\nu(\Omega)=\left\{f\in \mathcal{M}(\Omega) : \|f\|_{L^q_\nu(\Omega)}:=\left(\int_\Omega |f(x)|^q\nu(x)\, dx\right)^\frac{1}{q}<\infty\right\}.
\end{align*}

\item[--] We say $u\in C^{1,\alpha}(x)$, if there exists an affine function $\ell_x(y)=a_x+b_x\cdot(y-x)$ and a positive constant $K_x$, possibly depending on $x$, such that
\begin{align*}
    ||u-\ell_x||_{L^\infty(B_r(x))}\leq K_xr^{1+\alpha}.
\end{align*}
If $u\in C^{1,\alpha}(x)$ for every $x\in \Omega$ and $|a_x|+|b_x|+4K_x\leq M$, then $u\in C^{1,\alpha}(\Omega)$ in the usual sense and
\begin{align*}
    ||u||_{C^{1,\alpha}(\Omega)}\leq M.
\end{align*}

\item[--] The number $\alpha_0$ will always be the exponent corresponding to the interior $C^{1,\alpha_0}$ regularity of solutions to the equation
\begin{align*}
F(D^2u)=0, \mbox{ in } B_1,
\end{align*}
where $F$ is any $(\lambda,\Lambda)$-elliptic operator, see \cite[Chapter 5.3]{Caffarelli-Cabre_1995}.

\item[--] It will be useful to consider the following extremal inequalities in the viscosity sense
\begin{align}\label{equation_Sw_1}
    \omega(x)\mathcal{M}^-(D^2u)\leq f,\quad \mbox{ in } B_R
\end{align}
and
\begin{align}\label{equation_Sw_2}
    \omega(x)\mathcal{M}^+(D^2u)\geq f,\quad \mbox{ in } B_R,
\end{align}
where $\mathcal{M}^\pm$ are the usual Pucci operators. We say $u\in \overline{S}_\omega(f)$ if it satisfies \eqref{equation_Sw_1} and $u\in \underline{S}_\omega(f)$ if it satisfies \eqref{equation_Sw_2}. We also define  $S_\omega(f)=\overline{S}_\omega(f)\cap \underline{S}_\omega(f)$ and $S^*_\omega(f)=\overline{S}_\omega(|f|)\cap \underline{S}_\omega(-|f|)$.

\end{itemize}


As was alluded before, one can explore regularity for degenerate equations, provided one has some structural assumptions on the degeneracy. In the following we describe the set of assumptions concerning this issue.  The first two assumptions below will be in force throughout the paper. 
\begin{assump}[Degenerate ellipticity]\label{Assumption1}
There exist constants $0<\lambda\leq \Lambda<\infty$ and a nonnegative function $\omega\in L^\infty(B_1)$ such that for every matrices $M, N\in S(d)$ with $N\geq 0$, 
\[
\lambda\, \omega(x)\,|N|\leq F(M+N,x)-F(M,x)\leq \Lambda\, \omega(x)\, |N|.
\]
We furthermore assume that $\nu:=\omega^{-d}\psi^{b}\in A_p$ for some $b\in[0,d+1)$, where $\psi(x)=(1-|x|^2)^{1/2}$.
\end{assump}



\begin{assump}[Continuity and integrability]\label{Assumption2}
    Assume that $F(M,\cdot)\in C(B_1)$ for every $M\in S(d)$ and $f\in C(B_1)\cap L^\infty(B_1)\cap L_\nu^d(B_1)$. 
\end{assump}

\begin{assump}[Homogenization]\label{Assumption3}
Let $0\in \Gamma:=\{\omega=0\}$. There exists $0\leq a\leq 1$ such that as $\mu\to 0^+$, 
\begin{align*}
F_\mu(M,x):=\mu^{-a}F(M,\mu x)
\end{align*}
converges locally uniformly to $\omega_0(x)\overline{F}(M)$, where $\omega_0\in A_p$ and $\overline{F}$ is $(\lambda,\Lambda)$-elliptic.
\end{assump}


\begin{assump}[Oscillation control]\label{Assumption4}
Let $x_0\in B_1$ and define the oscillation of $F$ at the point $x_0$ by
\begin{align*}
\beta_{F,x_0}(x):=\sup_{M\in S(d)\setminus \{0\}}\frac{|F(M,x)-F(M,x_0)|}{|M|}.
\end{align*}
Then $\beta_{F,x_0}$ is H\"older continuous.
\end{assump}

Before commenting on the assumptions, we state our main results. The first one concerns interior H\"older regularity for functions in the class $S^*_\omega(f)$. 

\begin{Theorem}[Interior H\"older continuity]\label{Theorem_holder_continuity}
Suppose \ref{Assumption1}-\ref{Assumption2} are in force and let $u\in S^*_\omega(f)$ in $B_1$. Then $u\in C^{\alpha}(B_{1/2})$ and there exists $C>0$ 
\begin{align*}
    \|u\|_{C^\alpha(B_{1/2})}\leq C\left(\|u\|_{L^\infty(B_1)}+\|f\|_{L_\nu^d(B_1)} \right),
\end{align*}
where $C$ and $\alpha$ depend on $\lambda$, $\Lambda$, $d$ and $\omega$.
\end{Theorem}

The second result amounts to the optimal pointwise  $C^{1,\alpha}$ regularity. 

\begin{Theorem}\label{Theorem_C1alpha}
    Suppose \ref{Assumption1}-\ref{Assumption4} are in force and let $u\in C(B_1)$ be a viscosity solution to
    \begin{align*}
        F(D^2u,x)=f, \mbox{ in } B_1.
    \end{align*}
    Let $x_0\in B_{1/2}$. Then:
    \begin{enumerate}
    \item If $x_0\in \Gamma$ then $u\in C^{1,\alpha}(x_0)$ and
    \begin{align*}
        \|u\|_{C^{1,\alpha}(x_0)}\leq C\left(\|u\|_{L^\infty(B_1)}+\|f\|_{L^\infty(B_1)}\right),
    \end{align*}
	where $C$ depends only on $d, \lambda, \Lambda$    and $\alpha=\min\{\alpha_0^-,1-a\}$.
	
	\item If $x_0\in \{\omega>0\}$ then $u\in C^{1,\alpha}(x_0)$ for every $\alpha<\alpha_0$ and satisfies
	\begin{align*}
\|u\|_{C^{1,\alpha}(x_0)}\leq C\left({u}\|_{L^\infty(B_1)}+{f}\|_{L^\infty(B_1)}\right)
\end{align*}
where $C=C(\alpha, x_0,\omega,\lambda,\Lambda,d)$.
    \end{enumerate}
    
\end{Theorem}




\subsection{Outline of the proofs.} In this section, we describe the roadmap to the two previous theorems and how it differs from the one in the uniformly elliptic case. We begin by obtaining a weighted ABP (Aleksandrov-Bakelman-Pucci) estimate by utilizing the degeneracy of the operator as a measure. The process involves two main steps. Firstly, we observe that dividing the equation by the degeneracy $\omega$ results in a uniformly elliptic equation, albeit with a new source term that is only measurable. Secondly, although the conventional argument for obtaining the ABP estimate in uniformly elliptic equations assumes the source term to be in $L^\infty$, the estimates themselves are independent of this norm. Thus, we can approximate the unbounded source term by a sequence of truncated functions for which we obtain the ABP estimate in the conventional way. By employing techniques from the theory of $L^p$-viscosity solutions (see \cite{Caffarelli-Crandall-Kocan-Swiech_1996}), we establish the stability of these estimates, which consequently implies the ABP. Furthermore, we utilize the new Aleksandrov estimate obtained in \cite{Krylov_2019} to refine this ABP and allow for more degenerate equations, when they degenerate as a power of the distance to the boundary. This is the topic of Section \ref{Section_ABP}.

In Section \ref{Section_Calpha}, we use the weighted ABP to obtain a weak $L^\varepsilon$ estimate and a Harnack inequality. The main difficulty stems from the fact that, unlike the Lebesgue measure, if $L:\Rr^d\to \Rr^d$ is an affine map and $U\subset \Rr^d$ is a measurable set, then the measure $\nu(L(U))$ cannot be easily expressed in terms of $\nu(U)$ and $|\det(DL)|$. Therefore, we need to use properties of the Muckenhoupt weights to control the measure of scaled and translated sets. Harnack inequality and H\"older continuity follow in the usual way.

In Section \ref{Section_unique} we prove uniqueness of solutions to the approximating equation
\begin{align*}
\begin{cases}
\omega_0(x)F(D^2u)=0, &\mbox{ in } B_1,\\
u=\varphi, &\mbox{  on } \partial B_1.
\end{cases}
\end{align*}
The proof follows similar steps to \cite[Chapter 5]{Caffarelli-Cabre_1995}, with two important distinctions. First is the fact that the ellipticity might vanish at points where we touch the convex envelope of the solution by paraboloids. And secondly, the equation does not have constant coefficients which is a crucial assumption in \cite{Caffarelli-Cabre_1995}, see Remark 1 therein. 



In Section \ref{Section_C1alpha} we obtain interior $C^{1,\alpha}$ regularity. The approach used in this paper was inspired by the paper \cite{Silvestre-Teixeira_2015}, but instead of considering the recession in the matrix, we consider it on the coefficients and use the scaling given by the degeneracy, that is, we consider
\begin{align*}
F_\mu(M,x)=\mu^{-a} F(M,\mu x).
\end{align*}
The crucial assumption to proceed with the argument was inspired by \cite[Lemma 5]{Silvestre-Teixeira_2015} and \cite[Assumption (H2)]{LST} and states that $F_\mu(M,x)$ converges to $\omega_0(x)\overline{F}(M)$ uniformly in $x$ and locally in $M$, see \ref{Assumption3}.

It is noteworthy that the usual approach to treat the case with variable coefficients is to consider a smallness assumption on the oscillation of the form
\begin{align}\label{eq_intro_beta}
\|\beta\|_{L^\infty(B_r)}\leq Cr^\gamma,
\end{align}
where $\beta$ measures the proximity between the operator $F$ and another one for which there is improved regularity. In \cite[Chapter 8]{Caffarelli-Cabre_1995}, for example, $\beta$ measures the proximity between $F(M,x)$ and $F(M,0)$. In our case, however, since the ellipticity vanishes at points where $\omega=0$, we instead measure the proximity to the operator $\omega_0(x)\overline{F}(M)$ for which there is optimal $C^{1,\alpha_0}$ regularity, see Lemma \ref{Lemma_division}. Therefore, it seems natural to consider instead
\begin{align*}
\beta(x)=\sup_{M\in S(d)}\frac{|F(M,x)-\omega_0(x)\overline{F}(M)|}{1+|M|}.
\end{align*}
And indeed \ref{Assumption3} implies \eqref{eq_intro_beta}. To see this, first note that $\omega_0(x)$ has to be positively homogeneous of degree $a$. \ref{Assumption3} implies that for every $\varepsilon>0$ there exists $\delta>0$ such that if $\mu\leq \delta$, then for every $M\in S(d)$
\begin{align*}
|\mu^{-a}F(M,\mu x)-\omega_0(x)\overline{F}(M)|\leq \varepsilon (1+|M|).
\end{align*}
Hence, for every $r\leq \delta$ and $x\in B_r$, we can take $\mu=|x|$ to obtain
\begin{align*}
\frac{1}{1+|M|}\left|F(M,x)-|x|^a\omega_0\left(\frac{x}{|x|}\right)\overline{F}(M)\right|\leq \varepsilon r^a.
\end{align*}
Using the fact that $\omega_0$ is homogeneous of degree $a$, we get
\begin{align*}
\|\beta\|_{L^\infty(B_r)}\leq \varepsilon r^a,
\end{align*}
as intended.














We now comment further on the assumptions. We start by proving that the assumption $\omega\in L^\infty(B_1)$ in \ref{Assumption1} is not restrictive. Indeed, if  $\omega$ is any function satisfying \ref{Assumption1} except $\omega\notin L^\infty$ and we call $W(x)=\max\{\omega(x),1\}$, $\overline{\omega}(x)=W^{-1}(x)\,\omega(x)$ and $\overline{F}(M,x)=W^{-1}(x)\,F(M,x)$, then we see that $\overline{\omega}\in L^\infty(B_1)$,
\[
\lambda\, \overline{\omega}(x)\,|N|\leq \overline{F}(M+N,x)-\overline{F}(M,x)\leq \Lambda\, \overline{\omega}(x)\, |N|,
\]
and $\tilde{\nu}:=\overline{\omega}^{-d}\psi^{b}\in A_p$. To see this last statement, if $\nu\in A_p$, $B$ is an arbitrary ball and we define
\begin{align*}
\overline{\nu}:=\frac{\nu}{\min\{\nu,1\}}, \quad B^+:=\{\nu\geq 1\}\cap B_1, \quad B^-:=\{\nu<1\}\cap B_1,
\end{align*}
we can check that
\begin{align*}
&\left(\frac{1}{|B|}\int_B\overline{\nu}\,dx\right)\left(\frac{1}{|B|}\int_B\overline{\nu}^{-\frac{p'}{p}}\,dx\right)^\frac{p}{p'}\\
=&\left(\frac{1}{|B|}\left[\int_{B_+}\nu\,dx+|B_-|\right]\right)\left(\frac{1}{|B|}\left[\int_{B_+}\nu^{-\frac{p'}{p}}\,dx+|B_-|\right]\right)^\frac{p}{p'}\\
=&\left(\frac{1}{|B|}\int_{B_+}\nu\,dx\right)\left(\frac{1}{|B|}\left[\int_{B_+}\nu^{-\frac{p'}{p}}\,dx+|B_-|\right]\right)^\frac{p}{p'}\\
&+\frac{|B_-|}{|B|}\left(\frac{1}{|B|}\left[\int_{B_+}\nu^{-\frac{p'}{p}}\,dx+|B_-|\right]\right)^\frac{p}{p'}\\
\leq & \left(\frac{1}{|B|}\int_{B}\nu\,dx\right)\left(\frac{1}{|B|}\left[\int_{B}\nu^{-\frac{p'}{p}}\,dx\right]\right)^\frac{p}{p'}+1\\
\leq &\,C+1.
\end{align*}
Therefore $\overline{\nu}\in A_p$. Since $0\leq \psi^{b}\leq 1$, we have $\nu\leq \tilde{\nu}\leq \overline{\nu}$ with $\nu, \overline{\nu}\in A_p$ and thus $\tilde{\nu}\in A_p$, as intended. Furthermore, $\overline{f}=W^{-1}f\in C(B_1)\cap L^\infty(B_1)$. Therefore, we limit our study to the degenerate case, since at singular points, we can divide the equation by the weight and recover a uniformly elliptic equation. This property is common in equations in non-divergence form and represents one of the main differences with equations in divergence form where the singular case is often substantially different from the degenerate one. 


    An interesting class of weights $\omega$ included in  assumption $[A1]$ is the following. Let $\Gamma\subset \Rr^d$ be a $k$-dimensional $C^{\gamma}$ surface with $k<d$ such that $\Gamma\cap \partial B_1=\emptyset$. Assume that 
\begin{align*}
    \omega(x)=\dist(x,\Gamma)^a
\end{align*}
where
\begin{align}\label{eq_interval_a}
    0<a<\frac{d-k}{d}.
\end{align}
    Note that the interval \eqref{eq_interval_a} guarantees that $\dist(x,\Gamma)^{-ad}\in A_p$ (see \cite[Theorem 1.1]{Dyda-Ihnatsyeva-Lehrback_2019}). Furthermore, if $b<(p-1)$ then also $\psi^{b}\in A_p$ and therefore $\nu(x)=\dist(x,\Gamma)^{-ad}\psi(x)^{b}\in A_p$. Another interesting example is when we instead take $\Gamma=\partial B_1$ then since $\psi$ defines a distance to the boundary of $\partial B_1$, we get $\nu(x)=\dist(x,\partial B_1)^{-da+b}$ which is in $A_p$ if $a<1+2/d$. 
    
    
Regarding \ref{Assumption3}, we first note that if $F(M,x)=\omega(x)F(M)$, then this assumption coincides with the $L^\infty$ version of assumption (H2) in \cite{LST}; check the end of Section 2 therein to find some interesting examples of weights satisfying this assumption. In the more general case where the coefficients of $F$ are not explicit, we can find a sequence $(x_n)_n$ such that $\omega(x_n)>0$ and $x_n\to 0$. Then $F_n(M):=\omega^{-1}(x_n)F(M,x_n)$ is $(\lambda,\Lambda)$-elliptic and thus, by Arzela-Áscoli's Theorem, it must converge up to a subsequence to an operator $\overline{F}$. If we assume that this limit is unique and independent of the sequence $(x_n)_n$, then \ref{Assumption3} is satisfied for this $\overline{F}$. We would like also to point out that such coefficients involving suitable powers of the distance function to a smooth lower dimensional set have been considered by David, Feneuil and Mayboroda in their investigations about elliptic measure (for second order operators) in higher co-dimension (see e.g. \cite{DFM1,DFM2,DFM3} and references therein together with subsequent works). One of the motivations of the problem under consideration here is also to develop regularity tools for the non-divergence case in this setup. 

To ensure $C^{1,\alpha}$ regularity at points where $\omega>0$, the validity of \ref{Assumption4} places us within the framework described in \cite[Section 8.2]{Caffarelli-Cabre_1995}, allowing us to employ the findings presented therein. Notice however, that this is not enough to conclude directly. Indeed, consider $F(M,x)=|x|^a \,\mbox{Tr}\, M$ for $0<a<1$ so that $\alpha_0=1$. Then $u(x)=|x|^{1+\alpha}$ is a solution to
\begin{align*}
|x|^a\Delta u=f, \mbox{ in } B_1,
\end{align*}
for $f=C|x|^{\alpha+a-1}$ which is bounded and H\"older continuous provided $\alpha\geq 1-a$. Therefore, we can not expect solutions to be more regular than $C^{1,\alpha}$ for $\alpha=1-a$. It is important to note that these counter-examples exist only in the nonhomogeneous setting, as for the homogeneous one we can divide the equation by the degeneracy, see Lemma \ref{Lemma_division}.

At last, we would like to point out that equations of the type considered in the present paper are reminiscent of fully nonlinear equations on manifolds with conic-edge singularities. We state below a corollary of the previous theorems and refer the reader to the last section for more details. 

\begin{Corollary}
Let $(\mathcal{M},g_0)$ be a closed Riemannian manifold of dimension $d$, $g=e^{2u}g_0$ be a conformal metric to $g_0$ and $A_g$ be the Schouten tensor of $g$. Assume that $g$ has the conformal divisor
\begin{align*}
D=\sum_{i=1}^q\beta_ip_i
\end{align*}
where $p_i\in \mathcal{M}$ are distinct points and $\beta_i\in (-1,0)$.

If $u\in C^{2}(\mathcal{M})$ and 
\begin{align*}
g\in \mathcal{C}_k^+:=\{g:\sigma_1(M_g)>0, ...\,, \sigma_k(M_g)>0\},
\end{align*}
then the operator for $-2\beta_i k <1$, i.e. $\beta_i \in (0,2k)$
\begin{align*}
F(D^2u,x):=\sigma_k\left( g_E^{-1}e^{-v_i}\left(-\nabla^2u+du\otimes du-\frac{|\nabla u|^2}{2}g_E \right)\right)
\end{align*}
 satisfies \ref{Assumption1}-\ref{Assumption4}. Therefore, if
\begin{align*}
|x-p_i|^{-2\beta_ik}\sigma_k\left( g_E^{-1}e^{-v_i}\left(-\nabla^2u+du\otimes du-\frac{|\nabla u|^2}{2}g_E \right)\right)=f, \mbox{ in } \Omega,
\end{align*}
then $u\in C^{1,\alpha}_{loc}(\Omega)$ and
\begin{align*}
\|u\|_{C^{1,\alpha}_{loc}(\Omega)}\leq C(\|u\|_{L^\infty(\Omega)}+\|f\|_{L^\infty(\Omega)}).
\end{align*}
\end{Corollary}
For a discussion on this topic as well as the definitions of $D$ and $\sigma_k$ , see Section \ref{Section_conic}.
\subsection*{Higher order regularity: $W^{2,p}$ estimates}
We finish this introduction with some remarks about Sobolev estimates. 
Let $u\in C(B_1)$ be a solution of 
\begin{align}\label{eq_vc1}
    \begin{cases}
        F(D^2u,x)=f,\mbox{ in } B_1,  \\
        u=\varphi, \mbox{ on } \partial B_1,
    \end{cases}
\end{align}
where $\varphi\in C(\overline{B_1})$, $f\in L^\infty(B_1)\cap C(B_1)\cap L^d_\nu(B_1)$ and
\begin{align}\label{eq_el1}
    \omega(x)\lambda |N|\leq F(M+N,x)-F(M,x)\leq \omega(x)\Lambda |N|.
\end{align}
Then one can write
\begin{align*}
    \lambda |N|\leq G(M+N,x)-G(M,x)\leq \Lambda |N|
\end{align*}
where $G(M,x):=\omega^{-1}(x)F(M,x)$ is a uniformly elliptic operator, which is merely measurable with respect to $x$.  Furthermore, $g:=\omega^{-1}f\in L^d$. Then $u$ is an $L^p$-viscosity solution of
\begin{align*}
    G(D^2u,x)=g\in L^d(B_1).
\end{align*}


Let $F_\varepsilon(M,x):=\left(F(M,\cdot)*\eta_\varepsilon\right)(x)$ be a smoothening of $F$ with respect to $x$. Then $F_\varepsilon$ satisfies \eqref{eq_el1} with $\omega_\varepsilon:=\omega * \eta_\varepsilon$ instead of $\omega$ and note that $\omega_\varepsilon(x)\geq \delta>0$, for some $\delta$ depending on $\varepsilon$ (one can prove this by contradiction, using the fact that the set $\{\omega(x)=0\}$ has measure zero). Finally, let $G_\varepsilon(M,x):=\omega_\varepsilon^{-1}(x)F_\varepsilon(M,x)$ be $(\lambda,\Lambda)$-elliptic, $g_\varepsilon:=\omega_\varepsilon^{-1} f\in C(B_{1-\varepsilon})\cap L^d(B_{1-\varepsilon})$ which converges to $g$ in $L^d$, and let $u_\varepsilon\in C(B_1)$ be the $C$-viscosity solution to the uniformly elliptic equation
\begin{align}\label{eq_approx_ue}
    \begin{cases}
    G_\varepsilon(D^2u_\varepsilon,x)=g_\varepsilon,\, \mbox{ in } B_{1-\varepsilon},  \\
        u_\varepsilon=\varphi, \mbox{ on } \partial B_1.
    \end{cases}
\end{align}
If we further assume that  $F$ is convex with respect to $M$, then we can immediately apply \cite[Theorem 7.1]{Caffarelli-Cabre_1995} and get $W^{2,p}$ estimates for \eqref{eq_approx_ue}, that is,
\begin{align*}
    \|u_\varepsilon\|_{W^{2,p}(B_{1/2})}\leq C\left(\|u_\varepsilon\|_{L^\infty(B_{1-\varepsilon})}+\|g_\varepsilon\|_{L^p(B_{1-\varepsilon})}\right).
\end{align*}
By \cite[Theorem 3.8]{Caffarelli-Crandall-Kocan-Swiech_1996}, $u_\varepsilon\to u$ as $\varepsilon\to 0$ uniformly. Arguing as in \cite[Corollary 3.10]{Caffarelli-Crandall-Kocan-Swiech_1996}, we see that $u$ satisfies this same estimate. 

Note that this result does not contradict the counter-example presented before. Indeed, if we consider $f\equiv 1$ and $\omega(x)=|x|^a$, then $\omega^{-1} f\in L^p$ for $p<1/a$ and hence, as $a\to 1^-$, $p\to 1^+$ and thus there is no $C^{1,\alpha}$ regularity available since the Sobolev inequalities don't apply. On the other hand, as $a\to 1^-$, $p\to +\infty$ and since $u\in W^{2,p}(B_{1/2})$, by the Sobolev inequalities we have $u\in C^{1,\gamma}(B_{1/2})$ for $\gamma<1-ad$, which is a worse regularity result than Theorem \ref{Theorem_C1alpha}, even for $F$ convex.


\section{Weighted ABP estimate}\label{Section_ABP}

Let $u$ solve \eqref{eq_var_coef_main_form} and assume throughout that \ref{Assumption1} and \ref{Assumption2} hold. Then the extremal inequalities \eqref{equation_Sw_1} and \eqref{equation_Sw_2} are satisfied in the viscosity sense. The goal of this section is to obtain the following weighted ABP estimate for functions in $\overline{S}_\omega(f)$.

\begin{Proposition}\label{Proposition_w_weighted_ABP}
Let $u\in \overline{S}_\omega(f)$ satisfy $u\geq 0$ in $\partial B_R$. Then
\begin{align*}
    \sup_{B_R}u^-\leq C(d,b)R^{1-b/d}\left(\int_{B_R\cap \{u=\Gamma_u\}}(f^+)^d\omega^{-d}\psi^b\, dx\right)^\frac{1}{d},
\end{align*}
where $\psi(x)=(R^2-|x|^2)^{1/2}$ and $b\in [0,d+1)$. Note that $\Gamma_u$ corresponds to the convex envelope of $u^-$ extended by $0$ to $B_{2R}$, just as in \cite{Caffarelli-Cabre_1995}.
\end{Proposition}

Before proving this result, we start by proving an equivalent one for the uniformly elliptic equation (when $\omega\equiv 1$) and then proceed by approximation. We aim at proving the following.

\begin{Proposition}\label{Proposition_weighted_ABP}
Let $u\in \overline{S}_1(f)$ satisfy $u\geq 0$ in $\partial B_R$ and $f$ be continuous and bounded. Then
\begin{align*}
    \sup_{B_R}u^-\leq C(d,b)R^{1-b/d}\left(\int_{B_R\cap \{u=\Gamma_u\}}(f^+)^d\psi^b\, dx\right)^\frac{1}{d}.
\end{align*}

\end{Proposition}

The proof of this proposition follows similar steps as \cite[Theorem 3.2]{Caffarelli-Cabre_1995} with the Aleksandrov Lemma replaced by a refined version which was obtained in \cite[Corollary 1.2]{Krylov_2019}, namely
\begin{Lemma}
Let $u\in W^{2,d}_{loc}(B_R)\cap C(\overline{B_R})$ be convex, $u\geq 0$ in $B_R$, $\psi(x)=(R^2-|x|^2)^{1/2}$ and $b\in [0,d+1)$. Then
\begin{align*}
\sup_{B_R}u^-\leq C(b,d)R^{1-b/d}\left(\int_{B_R}\psi^b\det D^2u\, dx\right)^\frac{1}{d}.
\end{align*}
\end{Lemma}

The following result follows immediately.
\begin{Lemma}
Let $u\in C(\overline{B_R})$ satisfy $u\geq 0$ on $\partial B_R$, $\Gamma_u$ be defined as in Proposition \ref{Proposition_w_weighted_ABP} and assume that $\Gamma_u\in C^{1,1}(\overline{B_R})$.

Then there exists a set $A$ such that $|B_R\setminus A|=0$, $\Gamma_u$ is second order differentiable at every $x\in A$ and
\begin{align*}
\sup_{B_R}u^-\leq C(b,d)R^{1-b/d}\left(\int_{A}\psi^b\det D^2\Gamma_u\, dx\right)^\frac{1}{d}.
\end{align*}
\end{Lemma}

We can then proceed exactly as in \cite[Lemma 3.5, Lemma 3.3]{Caffarelli-Cabre_1995} and obtain the following results. We also adopt the notation therein. First we observe that $\det D^2\Gamma_u(x)=0$ if $x\in A\setminus \{u=\Gamma_u\}$ and therefore we can take the integral only on the contact set.
\begin{Lemma}\label{Lemma_contact_int}
Let $u\in C(\overline{B_R})$ satisfy $u\geq 0$ on $\partial B_R$ and $\Gamma_u$ be as defined in Proposition \ref{Proposition_w_weighted_ABP}. Let $K>0$ and $0<r\leq R$ be constants and assume that for every $x_0\in B_R$ there exists a convex paraboloid of opening $K$ that touches $\Gamma_u$ from above at $x_0$ in $B_r(x_0)$.

Then $\Gamma_u$ is $C^{1,1}(\overline{B_R})$ and there exists a set $A$ such that $|B_R\setminus A|=0$, $\Gamma_u$ is second order differentiable at every $x\in A$ and
\begin{align*}
\sup_{B_R}u^-\leq C(b,d)R^{1-b/d}\left(\int_{A\cap \{u=\Gamma_u\}}\psi^b\det D^2\Gamma_u\, dx\right)^\frac{1}{d}.
\end{align*}
\end{Lemma}

The next result concerns the regularity of the convex envelope.
\begin{Lemma}\label{Lemma_envelope_regularity}
Let $u\in \overline{S}_1(f)$ in $B_r$. Assume that $f$ is bounded and that $\phi$ is a convex function in $B_r$ such that $0\leq \phi\leq u$ in $B_r$ and $0=\phi(0)=u(0)$. Then
\begin{align*}
	\phi(x)\leq C (\sup_{B_r} f^+)|x|^2, \quad \forall x\in B_{\nu r},
\end{align*}
where $\nu$ and $C$ are positive universal constants.
\end{Lemma}

Combining Lemmas \ref{Lemma_contact_int} and \ref{Lemma_envelope_regularity} we obtain Proposition \ref{Proposition_weighted_ABP} exactly as in \cite[Theorem 3.2]{Caffarelli-Cabre_1995}. It is important to note that even though we assume that $f\in L^\infty$, Proposition \ref{Proposition_weighted_ABP} is independent of this norm. We will now exploit this fact to prove Proposition \ref{Proposition_w_weighted_ABP}.
\begin{proof}[Proof of Proposition \ref{Proposition_w_weighted_ABP}]

We start by taking $\eta_j\in C^\infty(B_R)$ such that $|\eta_j|\leq j$, $\eta_j\to \omega^{-1}$ in $L^p(B_R)$ and pointwise a.e. By \cite[Lemma 3.1]{Caffarelli-Crandall-Kocan-Swiech_1996}, there exist functions $\phi_j\in W^{2,p}(B_R)\cap C(\overline{B_R})$ which solve
\begin{align*}
\begin{cases}
\mathcal{M}^+(D^2\phi_j)\leq f(\eta_j-\omega^{-1}), \quad \mbox{ in } B_R,\\
\phi_j=0, \hspace{1.45in} \mbox{ on } \partial B_R,
\end{cases}
\end{align*}
and satisfy
\begin{align*}
\|\phi_j\|_{L^\infty(B_R)}\leq C(R)\|f(\eta_j-\omega^{-1})\|_{L^p(B_R)}\to 0 \mbox{ as } j\to \infty.
\end{align*}
Note that 
\begin{align*}
\mathcal{M}^-(D^2u+D^2\phi_j)\leq \mathcal{M}^-(D^2u)+ \mathcal{M}^+(D^2\phi_j).
\end{align*}
Set $v=u+\phi_j$ to get 
\begin{align*}
\mathcal{M}^-(D^2v)\leq& \mathcal{M}^-(D^2u)+\mathcal{M}^+(D^2\phi_j)\\
\leq &f\omega^{-1}+f(\eta_j-\omega^{-1})\\
\leq &f\eta_j
\end{align*}
Since the right hand side is continuous, we can apply Proposition \ref{Proposition_weighted_ABP} to $v$ and get
\begin{align*}
\sup_{B_R}(u+\phi_j)^-\leq C(d,b)R^{1-b/d}\left(\int_{B_R\cap \{v=\Gamma_v\}}(f^+)^d\psi^b\eta_j^d\,dx\right)^\frac{1}{d}.
\end{align*}
Taking the limit $j\to \infty$ we get the desired inequality. We again refer to \cite[Appendix A]{Caffarelli-Crandall-Kocan-Swiech_1996} for continuity properties of the contact set $\{v=\Gamma_v\}$.

\end{proof}


\section{$C^\alpha$ Regularity}\label{Section_Calpha}


This section concerns  $C^\alpha$ regularity for functions in $S^*_\omega(f)$. To achieve this, we use the weighted Aleksandrov-Bakelman-Pucci estimate to derive a weighted $L^\varepsilon$ estimated from which Harnack inequality and $C^\alpha$ follow as usual. 


We start by noting that the barrier function constructed in \cite[Lemma 4.1]{Caffarelli-Cabre_1995} still works in our context, since $\omega\in L^\infty$. 
\begin{Lemma}\label{Lemma_barrier}
    There exists a smooth function $\varphi$ satisfying
    \begin{align}
        \label{eq1lem8}&\varphi\geq 0 \mbox{ in } \Rr^d\setminus B_{2\sqrt{d}},\\
        \label{eq2lem8}&\varphi\leq -2 \mbox{ in } Q_3,\\
        \label{eq3lem8}&\omega(x)\mathcal{M}^+(D^2\varphi)\leq C\xi,
    \end{align}
    where $0\leq\xi\leq 1$, $\xi\in C(\Rr^d)$, $\supp\xi\subset\overline{Q}_1$. Also $\varphi\geq -M$ in $\Rr^d$.
\end{Lemma}
\begin{proof}
Let $\varphi(x)=M_1-M_2|x|^{-\alpha}$ in $\Rr^d\setminus B_{1/4}$ with $\alpha=\max\{1,(d-1)\Lambda/\lambda-1\}$ and take $M_1,M_2$ such that
\begin{align*}
    \varphi=0 \mbox{ in } B_{2\sqrt{d}}\quad \mbox{and}\quad  \varphi=-2 \mbox{ in } B_{3\sqrt{d}/2},
\end{align*}
so that \eqref{eq1lem8} holds. We can extend $\varphi$ smoothly to $\Rr^d$ so that \eqref{eq2lem8} holds. This extension depends only on $d, \lambda, \Lambda$. We now check \eqref{eq3lem8}. Note that for $|x|>1/4$
\begin{align*}
    D^2\varphi(x)=M_2\alpha|x|^{-\alpha-2}\left( -(\alpha+2)\frac{x\otimes x}{|x|^2} +I \right)
\end{align*}
and so the eigenvalues are
\begin{align*}
    M_2\alpha|x|^{-\alpha-2}(-\alpha-1,1,\hdots,1).
\end{align*}
We can calculate for $|x|>1/4$,
\begin{align*}
    \omega(x)\mathcal{M}^+(D^2\varphi)=  \omega(x) M_2\alpha|x|^{-\alpha-2}\left( (d-1)\Lambda-\lambda(\alpha+1) \right)\leq 0
\end{align*}
by our choice of $\alpha$. Since $\varphi$ was extended smoothly to $\Rr^d$ in a universal way, we also have
\begin{align*}
    \omega(x)\mathcal{M}^+(D^2\varphi)(x) \leq C(d,\lambda,\Lambda,\|\omega\|_{L^\infty}) \quad \mbox{ for } |x|\leq 1/4.
\end{align*}

\end{proof}

We will also use the following Calderón-Zygmund decomposition for $\nu\in A_p$. Following notation from \cite{Caffarelli-Cabre_1995}, if $Q$ is a dyadic cube, we call $\tilde{Q}$ the predecessor of $Q$ if $Q$ can be obtained by dividing $\tilde{Q}$ into cubes of half-size.
\begin{Lemma}\label{Lemma_weighted_CZ}
    Let $A\subset B\subset Q_1$ be measurable sets and $0<\delta<1$ such that
    \begin{enumerate}
        \item $\nu(A)\leq \delta$ and
        \item If $Q$ is a dyadic cube such that $\nu(A\cap Q)>(1-\delta)\nu(Q)$ then $\tilde{Q}\subset B$.
    \end{enumerate}
    Then $\nu(A)\leq \delta \nu(B)$.
\end{Lemma}
\begin{proof}
The proof follows exactly the same steps as \cite[Lemma 4.2]{Caffarelli-Cabre_1995}, with the only important remark being the fact that since $\nu\in A_p$ then it is a doubling measure by Proposition \ref{Prop_Ap_doubling} and therefore
\begin{align*}
    f(x)=\lim_{Q\to x} \frac{1}{\nu(Q)}\int_Q f(x) \, d\nu,
\end{align*}
where $Q$ are shrinking cubes containing $x$. For a proof of this fact, see \cite[Section 2.9]{Federer_1969}.
\end{proof}

Combining these two ingredients with the weighted ABP we get the first iteration for the $L^\varepsilon$ estimate.
\begin{Lemma}\label{Lemma_first_ite}
    There exist universal constants $\varepsilon_0>0$, $0<\mu<1$ and $M>1$ such that if $u\in \overline{S}_\omega(|f|)$ in $Q_{4\sqrt{d}}$, $u\in C(\overline{Q}_{4\sqrt{d}})$ and
    \begin{align*}
        &u\geq 0 \mbox{ in } Q_{4\sqrt{d}},\\
        &\inf_{Q_3}u\leq 1,\\
        &\|f\|_{L^d_\nu(Q_{4\sqrt{d}})}\leq \varepsilon_0,
    \end{align*}
    then
    \begin{align*}
        \nu(\{u\leq M\}\cap Q_1)>\mu.
    \end{align*}
\end{Lemma}
\begin{proof}
Let $\varphi$ be the barrier function defined in Lemma \ref{Lemma_barrier} and define $v=u+\varphi\in \overline{S}_\omega(|f|+C\xi)$. By Proposition \ref{Proposition_w_weighted_ABP} applied to $v$ we get
\begin{align*}
    1\leq & C\left(\int_{B_{2\sqrt{d}}\cap \{v=\Gamma_v\}}\left(|f|+C\xi\right)^d\nu(x)\,dx\right)^\frac{1}{d}\\
    \leq &C\|f\|_{L^d_\nu(B_{4\sqrt{d}})}+C\nu(\{v=\Gamma_v\}\cap Q_1)^\frac{1}{d}.
\end{align*}
For $\varepsilon_0$ small enough, we get
\begin{align*}
    \nu(\{u\leq M\}\cap Q_1)^\frac{1}{d}\geq C
\end{align*}
as intended.

\end{proof}

Next we iterate this argument.
\begin{Lemma}\label{Lemma_Lepsilon}
    Let $u$ be as in Lemma \ref{Lemma_first_ite}. Then
    \begin{align}\label{eq_ite_mk}
        \nu(\{u>M^k\}\cap Q_1)\leq (1-\overline{\mu})^k,
    \end{align}
    where $\overline{\mu}=c^{-1-q^2}\mu^{q^2}<\mu$ and $M, \mu$ are as in Lemma \ref{Lemma_first_ite}.
    As a consequence we have
    \begin{align}\label{eq_leps}
        \nu(\{u\geq t\}\cap Q_1)\leq C_1t^{-\varepsilon}, \forall t>0.
    \end{align}
    \end{Lemma}
    \begin{proof}
    Let $A=\{u>M^k\}\cap Q_1$ and $B=\{u>M^{k-1}\}\cap Q_1$. It suffices to show that 
    \begin{align*}
        \frac{\nu(A)}{\nu(B)}\leq (1-\overline{\mu}).
    \end{align*}
    We apply Lemma \ref{Lemma_weighted_CZ}. Clearly $A\subset B\subset Q_1$ and $\nu(A)\leq \nu(\{u>M\}\cap Q_1)\leq 1-\overline{\mu}$. It remains to prove that condition \textit{(2)} of Lemma \ref{Lemma_weighted_CZ} applies, that is, if $Q=Q_{1/2^i}(x_0)$ is such that 
    \begin{align}\label{eq_contradiction_Leps}
        \nu(A\cap Q)>(1-\overline{\mu})\nu(Q),
    \end{align}
    then $\tilde{Q}\subset B$. Suppose otherwise and take $\tilde{x}\in \tilde{Q}$ such that
    \begin{align}\label{Eq_x_tilde}
        u(\tilde{x})\leq M^{k-1}
    \end{align}
    Consider the transformation
    \begin{align*}
        x=x_0+\frac{1}{2^i}y,\quad y\in Q_1,\quad x\in Q,
    \end{align*}
    and $\tilde{u}(y)=u(x)/M^{k-1}$. We claim that $\tilde{u}$ is under the hypothesis of Lemma \ref{Lemma_first_ite}. Then it holds that
    \begin{align*}
        \nu(\{\tilde{u}(y)\leq M\}\cap Q_1)>\mu
    \end{align*}
By Proposition \ref{Proposition_control_Ap} we also have
\begin{align*}
    c|\{\tilde{u}(y)\leq M\}\cap Q_1|^\frac{1}{q}\geq \frac{\nu(\{\tilde{u}(y)\leq M\}\cap Q_1)}{\nu(Q_1)}>\mu
\end{align*}
where we assume, without loss of generality, that $\nu(Q_1)=1$. Therefore
\begin{align*}
    |\{\tilde{u}(y)\leq M\}\cap Q_1|>(c^{-1}\mu)^q
\end{align*}
Note that clearly
\begin{align*}
        \frac{|\{\tilde{u}(y)\leq M\}\cap Q_1|}{|\{u(x)\leq M^k\}\cap Q|}=2^{id},
\end{align*}    
and so    
\begin{align*}
    |\{u(x)\leq M^k\}\cap Q|>(c^{-1}\mu)^q|Q|
\end{align*}
Again by Proposition \ref{Proposition_control_Ap}
\begin{align*}
    \frac{\nu(\{u(x)\leq M^k\}\cap Q)}{\nu(Q)}\geq& c^{-1}\left(\frac{|\{u(x)\leq M^k\}\cap Q|}{|Q|}\right)^q\\
    >&c^{-1-q^2}\mu^{q^2}=\overline{\mu},
\end{align*}
    which is a contradiction with \eqref{eq_contradiction_Leps}.
    
    It remains to show that $\tilde{u}$ satisfies the hypothesis of Lemma \ref{Lemma_first_ite}. Clearly
    \begin{align*}
        \tilde{u}(y)\in \overline{S}_{\tilde{\omega}(y)}(\tilde{f}(y)) \mbox{ in } Q_{4\sqrt{d}},
    \end{align*}
    where $\tilde{\omega}(y)=\omega(x)$ and $\tilde{f}(y)=f(x)/\left(2^{2i}M^{k-1}\right)$. Also
    \begin{align*}
        x\in \tilde{Q}\implies y\in Q_3.
    \end{align*}
    Hence $\tilde{u}\geq 0$ and $\inf_{Q_3}\tilde{u}\leq \tilde{u}(\tilde{x})/M^{k-1}\leq 1$ by \eqref{Eq_x_tilde}. Finally, since
    \begin{align*}
        \|\tilde{f}(y)\|_{L^d_{\tilde{\nu}(y)}(Q_{4\sqrt{d}})}=\frac{2^i}{2^{2i}M^{k-1}}\|f\|_{L^d_\nu(Q_{4\sqrt{d}})}\leq \varepsilon_0.
    \end{align*}
    \eqref{eq_leps} follows immediately from \eqref{eq_ite_mk} taking $C_1=(1-\overline{\mu})^{-1}$ and $\varepsilon$ such that $1-\overline{\mu}=M^{-\varepsilon}$.
    
    
    \end{proof}

\begin{Lemma}
Let $u\in \underline{S}_\omega(-|f|)$ in $Q_{4\sqrt{d}}$. Assume that $\nu\in A_p$ and  $f, u$ satisfy
\begin{align}\label{eq4.10}
    \|f\|_{L^d_\nu(Q_{4\sqrt{d}})}\leq \varepsilon_0,
\end{align}
and
    \begin{align}\label{eq4.11}
        \nu(\{u\geq t\}\cap Q_1)\leq C_1t^{-\varepsilon}, \quad \forall t>0.
    \end{align}
    Then there exists $M_0>1$ and $\sigma>0$ such that for $\delta:=M_0/(M_0-1/2)>1$, the following holds:
    
    If $j\geq 1$ is an integer and $x_0$ satisfies
    \begin{align*}
        &|x_0|_\infty\leq 1/4,\\
        &u(x_0)\geq \delta^{j-1}M_0,
    \end{align*}
    then
    \begin{align*}
        Q^j:=Q_{l_j}(x_0)\subset Q_1
    \end{align*}
    and
    \begin{align*}
        \sup_{Q^j}u\geq \delta^j M_0,
    \end{align*}
    where $l_j:=\sigma M_0^{-\varepsilon/(qd)}\delta^{-\varepsilon j/(qd)}$.
    
    
    
    
    
\end{Lemma}

\begin{proof}

    Let $\sigma>0$ and $M_0>1$ such that
    \begin{align}\label{eq4.17}
        \frac{1}{2}\sigma^{qd}> c^{-1} C_1  2^\varepsilon(4\sqrt{d})^{qd}
    \end{align}
    and
    \begin{align}\label{eq4.18}
        \sigma M_0^{-\varepsilon /(qd)}+c^{1+1/q^2}(C_1M_0^{-\varepsilon})^\frac{1}{q^2}\leq \frac{1}{2}
    \end{align}
    where $c$ depends only on $\nu$, and $\varepsilon, C_1$ come from \eqref{eq4.11}.
    
    It is easy to check that
    \begin{align}\label{eq4.19}
        Q_{l_j/(4\sqrt{d})}(x_0)\subset Q_{l_j(x_0)}=Q^j\subset Q_1.
    \end{align}
    By \eqref{eq4.19} and \eqref{eq4.11} we have
    \begin{align}\label{eq4.20}
        \nonumber&\nu(\{u\geq \delta^jM_0/2\}\cap Q_{l_j/(4\sqrt{d})}(x_0))\\
        \leq&\nu(\{u\geq \delta^jM_0/2\}\cap Q_1)\leq C_1 \delta^{-j\varepsilon}(M_0/2)^{-\varepsilon}.
    \end{align}
    We define the transformation 
    \begin{align*}
        x=x_0+\frac{l_j}{4\sqrt{d}}y,\quad y\in Q_{4\sqrt{d}}, \quad x\in Q^j
    \end{align*}
and the function
\begin{align*}
    v(y)=\frac{\delta M_0-\frac{u(x)}{\delta^{(j-1)}}}{(\delta-1)M_0}.
\end{align*}
This transformation defines bijections between the sets
\begin{align*}
    &x\in Q_{l_j}(x_0) [\mbox{ resp. } Q_{3l_j/(4\sqrt{d})}(x_0), Q_{l_j/(4\sqrt{d})}(x_0)]\iff\\
    \iff & y\in Q_{4\sqrt{d}}[\mbox{ resp. } Q_3, Q_1].
\end{align*}
We claim that $v$ satisfies the assumptions of Lemma \ref{Lemma_Lepsilon}. Thus, by \eqref{eq4.11},
\begin{align*}
    \nu(\{v(y)>M_0\}\cap Q_1)\leq C_1 M_0^{-\varepsilon}
\end{align*}
Since $u(x)<\delta^jM_0/2\implies v(y)>M_0$ we have
\begin{align*}
    \frac{|\{u(x)<\delta^jM_0/2\}\cap Q_{l_j/(4\sqrt{d})(x_0)}|}{|\{v(y)>M_0\}\cap Q_1|}\leq \left(\frac{l_j}{4\sqrt{d}}\right)^d.
\end{align*}
For simplicity, call $F=\{u(x)<\delta^jM_0/2\}\cap Q$, $Q=Q_{l_j/(4\sqrt{d})}(x_0)$ and $\tilde{F}=\{v(y)>M_0\}\cap Q_1$. We know that
\begin{align*}
    \nu(\tilde{F})\leq C_1 M_0^{-\varepsilon}.
\end{align*}
We will not repeatedly use Proposition \ref{Proposition_control_Ap}. First we get
\begin{align*}
    |\tilde{F}|\leq \left(c C_1 M_0^{-\varepsilon}\right)^\frac{1}{q},
\end{align*}
and therefore
\begin{align*}
    |F|\leq |Q|\left(c C_1M_0^{-\varepsilon}\right)^\frac{1}{q}.
\end{align*}
Note also that
\begin{align*}
    \nu(Q)\geq c^{-1}|Q|^q
\end{align*}
and we know by \eqref{eq4.20} that
\begin{align*}
    \nu(F^c)\leq C_1\delta^{-j\varepsilon}(M_0/2)^{-\varepsilon}.
\end{align*}
We now combine everything and get
\begin{align*}
    1=&\frac{\nu(Q)}{\nu(Q)}=\frac{\nu(F^c)+\nu(F)}{\nu(Q)}\\
    \leq &c^{-1}|Q|^{-q}C_1 \delta^{-j/\varepsilon}(M_0/2)^{-\varepsilon}+c^{1+1/q^2}(C_1M_0^{-\varepsilon})^\frac{1}{q^2}
\end{align*}
by \eqref{eq4.18}
\begin{align*}
    \frac{|Q|^q}{2}\leq c^{-1}C_1 \delta^{-j/\varepsilon}(M_0/2)^{-\varepsilon}
\end{align*}
and since 
\begin{align*}
    l_j=\sigma M_0^{-\varepsilon/(qd)}\delta^{-\varepsilon j/(qd)}
\end{align*}
we obtain
\begin{align*}
    \frac{1}{2}\sigma^{qd}\leq c^{-1} C_1 2^\varepsilon (4\sqrt{d})^{qd}
\end{align*}
which contradicts \eqref{eq4.17}.



It just remains to prove that $v$ satisfies the assumptions of Lemma \ref{Lemma_Lepsilon}. One can check that
\begin{align*}
    D^2_y v(y)=-\left( (\delta-1)M_0\delta^{j-1} \right)^{-1}\left(\frac{l_j}{4\sqrt{d}}\right)^2D^2_xu(x)
\end{align*}
and so
\begin{align*}
    \omega\left(x_0+\frac{l_j}{4\sqrt{d}}y\right)\mathcal{M}^-\left(D^2_yv(y)\right)\leq \frac{\left|f\left(x_0+\frac{l_j}{4\sqrt{d}}y\right)\right|}{(\delta-1)M_0\delta^{j-1}}\left(\frac{l_j}{4\sqrt{d}}\right)^2
\end{align*}
which we rewrite as
\begin{align*}
    \overline{\omega}(y)\mathcal{M}^-\left(D^2_yv(y)\right)\leq |\overline{f}(y)|.
\end{align*}
Call also $\overline{\nu}(y)=\overline{\omega}^{-d}(y)$ and note that $\nu$ and $\overline{\nu}$ belong to the same class $A_p$.

We now prove that $\overline{f}$ satisfies the smallness assumption \eqref{eq4.10}. Indeed a simple change of variables yields
\begin{align*}
    \|\overline{f}\|_{L^d_{\overline{\nu}}(Q_{4\sqrt{d}})}=&\left(\frac{l_j}{4\sqrt{d}}\right)\left((\delta-1)M_0\delta^{j-1}\right)^{-1}\left(\int_{Q^j}f^d(x)\omega^{-d}(x)\,dx\right)^\frac{1}{d}\\
    \leq &\left(\frac{l_j}{4\sqrt{d}}\right)\left((\delta-1)M_0\delta^{j-1}\right)^{-1} \varepsilon_0\\
    &\leq \frac{\delta^{-\varepsilon j/(qd)}}{8\sqrt{d}}\frac{2}{M_0\delta^{j-1}}\varepsilon_0\leq \varepsilon_0
\end{align*}
where we used the definition of $l_j$ and the fact that $\delta>1$ and $\delta=2(\delta-1)M_0$.













\end{proof}

The proof of Harnack inequality and thus of Theorem \ref{Theorem_holder_continuity} follows identically from \cite[Chapter 4]{Caffarelli-Cabre_1995}. We also have uniform continuity up to the boundary. The proof is also identical to \cite[Proposition 4.14]{Caffarelli-Cabre_1995}.

\begin{Proposition}[Uniform continuity up to the boundary]\label{Proposition_continuity_boundary}
    Let $u\in S_\omega(f)$ in $B_1$, $\varphi:=u_{|\partial B_1}$ and $\rho_\varphi(|x-y|)$ be a modulus of continuity of $\varphi$. Assume finally that $K>0$ is a constant such that $\|\varphi\|_{L^\infty(\partial B_1)}\leq K$ and $\|f\|_{L^d_\nu(B_1)}\leq K$.

    Then there exists a modulus of continuity $\rho_u$ of $u$ in $\overline{B_1}$ which depends only on $d, \lambda, \Lambda, K$ and $\rho_\varphi$.
\end{Proposition}

\section{Uniqueness of solutions}\label{Section_unique}

In this section, we will prove uniqueness of solutions to the following homogeneous Dirichlet problem
\begin{align*}
    \begin{cases}
        \omega(x)F(D^2u)=0,\quad \mbox{ in } B_1,\\
        u=\varphi,\quad \mbox{ on } \partial B_1.
    \end{cases}
\end{align*}

We start by defining Jensen's approximate solutions. Let $u$ be a continuous function in $B_1$ and let $U$ be an open set such that $\overline{U}\subset \Omega$. We define, for $\varepsilon>0$, the upper $\varepsilon$-envelope of $u$ with respect to $U$:
\begin{align*}
    u^\varepsilon(x_0)=\sup_{x\in \overline{U}}\left\{u(x)+\varepsilon-\frac{1}{\varepsilon}|x-x_0|^2\right\}, \quad \mbox{ for } x_0\in U.
\end{align*}
Similarly we can define the lower envelope $u_\varepsilon$ using convex paraboloids.

The following properties of $u^\varepsilon$ are proven in \cite[Lemma 5.2]{Caffarelli-Cabre_1995}. 

\begin{Lemma}\label{Lemma_properties_envelope}
    Let $x_0, x_1\in U$. Then
    \begin{enumerate}
        \item $\exists x_0^*\in \overline{U}$ such that $u^\varepsilon(x_0)=u(x_0^*)+\varepsilon-|x_0^*-x_0|^2/\varepsilon$.
        \item $u^\varepsilon(x_0)\geq u(x_0)+\varepsilon$.
        \item $|u^\varepsilon(x_0)-u^\varepsilon(x_1)|\leq (3/\varepsilon)\mbox{diam}(U)|x_0-x_1|$.
        \item $0<\varepsilon<\varepsilon'\implies u^\varepsilon(x_0)\leq u^{\varepsilon'}(x_0)$.
        \item $|x_0^*-x_0|^2\leq \varepsilon \osc_U u$.
        \item $0<u^\varepsilon(x_0)-u(x_0)\leq u(x_0^*)-u(x_0)+\varepsilon$.
    \end{enumerate}
\end{Lemma}

We will now prove the following.
\begin{Theorem}\label{Theorem_properties_envelope}
    \begin{enumerate}
        \item $u^\varepsilon\in C(U)$ and $u^\varepsilon\downarrow u$ uniformly in $U$ as $\varepsilon\to 0$.
        \item For any $x_0\in U$, there is a concave paraboloid of opening $2/\varepsilon$ that touches $u^\varepsilon$ by below at $x_0$ in $U$. Hence $u^\varepsilon$ is $C^{1,1}$ from below in $U$. In particular, $u^\varepsilon$ is punctually second order differentiable at almost every point of $U$.
        \item Suppose that $u$ is a viscosity solution of $\omega(x)F(D^2u)=0$ in $B_1$ and that $U_1$ is an open set such that $\overline{U}_1\subset U$. We then have that for $\varepsilon\leq \varepsilon_0$ (where $\varepsilon_0$ depends only on $u, U, U_1$) $u^\varepsilon$ is a viscosity subsolution of $\omega(x)F(D^2u)=0$ in $U_1$; in particular, $\omega(x)F(D^2u^\varepsilon(x))\geq 0$ a.e. in $U$.
        \end{enumerate}
\end{Theorem}
\begin{proof}
    The proof of \textit{(1)} and \textit{(2)} can be found in \cite[Theorem 5.1]{Caffarelli-Cabre_1995}. We will prove \textit{(3)}.

    Let $x_0\in U_1$ and $P(x)$ be a paraboloid that touches $u^\varepsilon$ by above at $x_0$. We wish to prove that 
    \begin{align*}
        \omega(x_0)F(D^2P)\geq 0.
    \end{align*}
    Clearly if $\omega(x_0)=0$ there is nothing to prove, so we assume that $\omega(x_0)>0$. Consider
    \begin{align*}
        Q(x)=P(x+x_0-x_0^*)+\frac{1}{\varepsilon}|x_0-x_0^*|^2-\varepsilon.
    \end{align*}
    
    By property \textit{(5)} of Lemma \ref{Lemma_properties_envelope}, we can pick $\varepsilon_0$ so small that for every $\varepsilon\leq \varepsilon_0$
    \begin{align*}
        \begin{cases}
            x_0\in U_1\implies x_0^*\in U,\\
            \omega(x_0)>0\implies \omega(x_0^*)>0.
        \end{cases}
    \end{align*}
Hence
\begin{align*}
    u(x)\leq u^\varepsilon(x+x_0-x_0^*)+\frac{1}{\varepsilon}|x_0-x_0^*|^2-\varepsilon.
\end{align*}
Therefore, again for $x$ close to $x_0^*$,
\begin{align*}
    u(x)\leq\,& P(x+x_0-x_0^*)+\frac{1}{\varepsilon}|x_0-x_0^*|^2-\varepsilon\\
    =\,&Q(x)
\end{align*}
and $u(x_0^*)=Q(x_0^*)$, since $P(x_0)=u^\varepsilon(x_0)$. Hence $Q$ touches $u$ by above at $x_0^*$, i.e.
\begin{align*}
    0\leq \omega(x_0^*)F(D^2Q)= \omega(x_0^*)F(D^2P)
\end{align*}
which implies $F(D^2P)\geq 0$ and thus
\begin{align*}
    \omega(x_0) F(D^2P)\geq 0,
\end{align*}
as intended.
    
\end{proof}

The following stability result will be used several times.

\begin{Proposition}\label{Proposition_stability}
    Let $\{\omega_k\}_k$ be a sequence of weights satisfying \ref{Assumption1}-\ref{Assumption4} and converging locally uniformly to $\omega$, $F$ be a $(\lambda,\Lambda)$-elliptic operator and $\{u_k\}_k$ be such that $\omega_k(x)F(D^2u_k)\geq 0 $ in the viscosity sense in $B_1$. Assume also that $u_k$ converges locally uniformly to $u$. Then
    \begin{align*}
        \omega(x)F(D^2u)\geq 0, \quad\mbox{ in } B_1.
    \end{align*}
\end{Proposition}
\begin{proof}
    Let $P$ touch $u$ from above at $x_0$, $B_r(x_0)\subset B_1$, and $k_0\geq 1$. Then
\begin{align*}
    P_k(x):=P(x)+\frac{1}{k}\frac{|x-x_0|^2}{2}+C_k
\end{align*}
touches $u_k$ from above at some point $x_k\in B_r(x_0)$, for some $k\geq k_0$. Hence
\begin{align*}
    \omega_k(x_k)F\left(D^2P+\frac{1}{k} I\right)\geq 0,
\end{align*}
which implies
\begin{align*}
    \Lambda \omega_k(x_k)\frac{1}{k}+\omega_k(x_k)F(D^2P)\geq 0.
\end{align*}
Letting $k\to \infty$ we get $x_k\to x_0$, $\omega_k(x_k)\to \omega(x_0)$ and thus
\begin{align*}
    \omega(x_0)F(D^2P)\geq 0,
\end{align*}
as intended.
\end{proof}

\begin{Remark}\label{Remark_closed_S}
    In particular, this implies that the classes $\overline{S}_\omega, \underline{S}_\omega$ and $S_\omega$ are closed under uniform limits in compact sets.
\end{Remark}

\begin{Theorem}\label{Theorem_uniqueness}
    Let $u$ be a subsolution of $\omega(x)F(D^2u)=0$ and $v$ be a supersolution of $\omega(x)F(D^2v)=0$ in $B_1$. Then
    \begin{align*}
        u-v\in \underline{S}_\omega, \mbox{ in } B_1.
    \end{align*}
\end{Theorem}
\begin{proof}
    The proof follows similar steps as \cite[Theorem 5.3]{Caffarelli-Cabre_1995}; however we need to use some results obtained in this paper for degenerate equations instead. Another important difference comes from the fact that our equation has variable coefficients.

    Fix $U$ and $U_1$ such that $\overline{U}_1\subset U\subset \overline{U}\subset B_1$; we will prove that for $\varepsilon>0$ small enough $u^\varepsilon-v_\varepsilon\in \underline{S}_\omega$ in $U_1$. It follows that $u-v\in \underline{S}_\omega$ in $B_1$ since $U_1$ was arbitrary, $u^\varepsilon-v_\varepsilon$ converges uniformly to $u-v$ and $\underline{S}_\omega$ is closed, see Remark \ref{Remark_closed_S}.

    To see that $u^\varepsilon-v_\varepsilon\in \underline{S}_\omega$ in $U_1$, let $P$ be a paraboloid such that $u^\varepsilon-v_\varepsilon\leq P$ in $B_r(x_0)\subset U_1$, with $r$ to be fixed, and $(u^\varepsilon-v_\varepsilon)(x_0)=P(x_0)$. It suffices to show that 
    \begin{align*}
        \omega(x_0)\mathcal{M}^+(D^2P)\geq 0.
    \end{align*}
    If $\omega(x_0)=0$ we are done, therefore assume $\omega(x_0)>0$ and choose $r>0$ so small that $B_r(x_0)\subset \{\omega>0\}$. We may assume that $\overline{B}_{2r}(x_0)\subset U$. Take $\delta>0$ and define
    \begin{align*}
        \phi(x)=v_\varepsilon(x)-u^\varepsilon(x)+P(x)+\delta|x-x_0|^2-\delta r^2.
    \end{align*}
    We have that $\phi\geq 0$ in $\partial B_r(x_0)$ and $\phi(x_0)<0$. Using \textit{(2)} in Theorem \ref{Theorem_properties_envelope}, we know that for every $x\in \overline{B}_r(x_0)$, there exists a convex paraboloid $P^x$ of opening $K$ that touches $\phi$ by above at $x$ in $B_r(x)$, where $K$ is independent of $x$. We apply Lemma \ref{Lemma_contact_int} to $\phi$ in $B_r(x_0)$ with $b=0$. Using the notation of that lemma, we have that if $x\in \overline{B}_r(x_0)\cap \{\phi=\Gamma_\phi\}$ then $P^x$ also touches $\Gamma_\phi$ by above at $x$ in $B_r(x)$. Since $\phi(x_0)<0$, we further get 
    \begin{align}\label{Equation_unique1}
        0<\int_{B_r\cap \{\phi=\Gamma_\phi\}}\det D^2\Gamma_\phi\, dx.
    \end{align}
    By \textit{(2)} in Theorem \ref{Theorem_properties_envelope} we know that there exists $A\subset B_r(x_0)$ such that $|B_r(x_0)\setminus A|=0$ and $u^\varepsilon$ and $v_\varepsilon$ (and hence $\phi$) are punctually second order differentiable in $A$. By \textit{(3)} in Theorem \ref{Theorem_properties_envelope},
    \begin{align}\label{Equation_unique2}
        F(D^2v_\varepsilon(x))\leq 0 \mbox{ and } F(D^2u^\varepsilon(x))\geq 0,\mbox{ for } x\in A.
    \end{align}
    Since $\Gamma_\phi$ is convex and $\Gamma\leq \phi$, we have that
    \begin{align}\label{Equation_unique3}
        D^2\phi(x) \mbox{ is nonnegative definite, for } x\in A\cap \{\phi=\Gamma_\phi\}
    \end{align}
    It follows from \eqref{Equation_unique1} and $|B_r(x_0)\setminus A|=0$ that
    \begin{align*}
        |\{\phi=\Gamma_\phi\}\cap A|>0
    \end{align*}
    and thus there is a point $x_1\in \{\phi=\Gamma_\phi\}\cap A$. At this point, we have by \eqref{Equation_unique2} and \eqref{Equation_unique3} that
    \begin{align*}
        0\leq & \,\omega(x_1)F(D^2u^\varepsilon(x_1))\\
        = & \,\omega(x_1)F(D^2v_\varepsilon-D^2\phi(x_1)+D^2P+2\delta I)\\
        \leq &\,\omega(x_1)F(D^2v_\varepsilon(x_1)+D^2P+2\delta I)\\
        \leq &\, \omega(x_1)\left[ F(D^2v_\varepsilon(x_1))+\Lambda \|(D^2P)^+\|-\lambda\|(D^2P)^-\|+2\Lambda \delta \right]\\
        \leq &\,\omega(x_1)\left[ \Lambda \|(D^2P)^+\|-\lambda\|(D^2P)^-\|+2\Lambda \delta \right]\\
        \leq&\, \omega(x_1)\mathcal{M}^+(D^2P)+2\Lambda \delta \omega(x_1).
    \end{align*}
    Letting $\delta \to 0$ we get
    \begin{align*}
        \omega(x_1)\mathcal{M}^+(D^2P)\geq 0
    \end{align*}
    Since $x_1\in A\subset B_r(x_0)\subset \{\omega>0\}$ we get 
    \begin{align*}
        \omega(x_0)\mathcal{M}^+(D^2P)\geq 0
    \end{align*}
    as intended.
    


    
\end{proof}

Combining Theorem \ref{Theorem_uniqueness} with the maximum principle, which follows from Proposition \ref{Proposition_w_weighted_ABP}, we immediately get the following result.
\begin{Corollary}\label{Corollary_uniqueness}
    The Dirichlet problem
    \begin{align*}
        \begin{cases}
        \omega(x)F(D^2u)=0,\quad &\mbox{ in } B_1,\\
        u=\varphi, &\mbox{ on } \partial B_1,
    \end{cases}
    \end{align*}
    has at most one solution $u\in C(B_1)$.
\end{Corollary}







\section{$C^{1,\alpha}$ Regularity}\label{Section_C1alpha}


This section concerns the local $C^{1,\alpha}$ regularity for the equation 
\begin{align}\label{eq_var_coef}
    \begin{cases}
        F(D^2u,x)=f,\quad \mbox{ in } B_1,\\
        u=\varphi \mbox{ on } \partial B_1.
    \end{cases}
\end{align}
where $F$ satisfies assumptions \ref{Assumption1}-\ref{Assumption4} and $\varphi$ has some modulus of continuity. We start by considering only the more challenging case and obtain $C^{1,\alpha}(x_0)$ regularity at points where $\omega(x_0)=0$. We assume, without loss of generality, that $x_0=0$. Assumption \ref{Assumption3} provides a tangential path from equation \eqref{eq_var_coef} to the following one
\begin{align}\label{eq_div_1}
    \omega_0(x)\overline{F}(D^2u)=0, \quad \mbox{ in } B_1,
\end{align}
where $\overline{F}$ is uniformly elliptic with constants $\lambda, \Lambda$. 



We start this section by proving 
interior $C^{1,\alpha_0}$ regularity for \eqref{eq_div_1} by the following Division Lemma, and then we use a perturbation argument to get $C^{1,\alpha}$ for \eqref{eq_var_coef}.

\begin{Lemma}[Division Lemma]\label{Lemma_division}
If $u$ is a viscosity solution to \eqref{eq_div_1}, then it is also a viscosity solution to
\begin{align}\label{eq_div_2}
    \overline{F}(D^2u)=0, \quad \mbox{ in } B_1.
\end{align}
\end{Lemma}
\begin{proof}


Let $u$ be a solution to \eqref{eq_div_1} in $B_1$ and $v$ be the unique solution of
\begin{align*}
\begin{cases}
\overline{F}(D^2v)=0, &\mbox{ in } B_{1},\\
v=\varphi,\qquad &\mbox{ on } \partial B_{1}.
\end{cases}
\end{align*}
Then clearly
\begin{align}\label{eq_div_3}
\begin{cases}
\omega(x)\overline{F}(D^2v)=0, &\mbox{ in } B_{1},\\
v=\varphi,\qquad &\mbox{ on } \partial B_{1}.
\end{cases}
\end{align}
By Corollary \ref{Corollary_uniqueness}, there exists a unique solution to \eqref{eq_div_3} and thus $v=u$ in $B_1$.

\end{proof}

The following Approximation Lemma is instrumental in our study.
\begin{Lemma}[Approximation Lemma]\label{Lemma_approximation}
Assume \ref{Assumption1}-\ref{Assumption3} hold and let $u\in C(B_1)$ be a solution to
\begin{align*}
F_\mu(D^2u,x)=f.
\end{align*}
Then, for every $\varepsilon>0$ there exists $\delta>0$ such that if $\mu\leq \delta$, $\|u\|_{L^\infty(B_1)}\leq 1$ and $\|f\|_{L^\infty(B_1)}\leq \delta$, there exists $h\in C^{1,\alpha_0}_{loc}(B_{9/10})$ solving
\begin{align*}
\overline{F}(D^2h)=0,\quad \mbox{ in } B_{9/10}
\end{align*} 
and such that
\begin{align*}
\|u-h\|_{L^\infty(B_{1/2})}\leq \varepsilon.
\end{align*}
\end{Lemma}

\begin{proof}
By contradiction, assume that there exist sequences $(u_n)_n$, $(\mu_n)_n$ and $(f_n)_n$ such that
\begin{align*}
F_{\mu_n}(D^2u_n,x)=f_n
\end{align*}
with $\mu_n\to 0$, $\|u_n\|_{L^\infty(B_1)}\leq 1$ and $f_n\to 0$; however there exists $\varepsilon_0>0$ such that, for every $h\in C_{loc}^{1,\alpha_0}(B_{9/10})$,
\begin{align}\label{eq_approx_1}
\|u_n-h\|_{L^\infty(B_{1/2})}>\varepsilon_0.
\end{align}

By \ref{Assumption3}, $F_{\mu_n}(M,x)\to \omega_0(x)\overline{F}(M)$. Since $F_{\mu_n}(M,x)$ has ellipticity  $\omega_n(x)\lambda$ and $\omega_n(x)\Lambda$ with $\omega_n(x)=\mu_n^{-a}\omega(\mu_n x)$, we easily see that $\nu_n:=\omega_n\psi^b\in A_p$ with the same constant as $\nu$. Thus, by Theorem \ref{Theorem_holder_continuity}, $u_n\in C_{loc}^\alpha$ with universal estimates. Hence,  by Arzelà-Ascoli, there exists a subsequence such that $u_n\to u_0$ locally uniformly. Arguing as in Proposition \ref{Proposition_stability} we conclude that
\begin{align*}
\omega_0(x)\overline{F}(D^2u_0)=0.
\end{align*}
By Lemma \ref{Lemma_division} we get
\begin{align*}
\overline{F}(D^2u_0)=0.
\end{align*}
Choosing $h=u_0$ in \eqref{eq_approx_1} we get a contradiction, which concludes the proof.

\end{proof}


We now check that we can assume, without loss of generality, that the smallness regime considered in Lemma \ref{Lemma_approximation} holds. For this purpose, let $\delta>0$ to be chosen later. Define the rescaling
\begin{align*}
\tilde{u}(y)=\frac{r_1^{-2}u(r_1y)}{K}
\end{align*}
with
\begin{align*}
K:=r_1^{-2}\|u\|_{L^\infty(B_1)}+\delta^{-1}\|f\|_{L^\infty(B_1)}r_1^{-a}.
\end{align*}
Then $\tilde{u}$ solves
\begin{align*}
\frac{1}{r_1^a K}F(KD^2\tilde{u},r_1y)=\frac{f(r_1y)}{Kr_1^a}
\end{align*}
which we rewrite as
\begin{align*}
\tilde{F}_{r_1}(D^2\tilde{u},y)=\tilde{f}(y).
\end{align*}
Choosing $r_1\leq \delta$, we see that we are in the smallness regime of Lemma \ref{Lemma_approximation}.




We now use the characterization of H\"older spaces to obtain the first instance of a geometric iteration.


\begin{Lemma}\label{Lemma_first_it}
Suppose \ref{Assumption1}-\ref{Assumption3} are in force and let $u\in C(B_1)$ be a solution to \eqref{eq_var_coef}. Then there exist $0<\rho\ll1$ and an affine function
\begin{align*}
\ell(x)=a+b\cdot x
\end{align*}
with $a\in \Rr$ and $b\in \Rr^d$ such that
\begin{align*}
\|u-\ell\|_{L^\infty(B_\rho)}\leq \rho^{1+\alpha}
\end{align*}
where $\alpha$ is any positive number less than $\alpha_0$ and $\rho$ depends only on universal constants and $\alpha$.
\end{Lemma}
\begin{proof}
Take $\varepsilon_0>0$ to be fixed. Note that this choice fixes $\delta>0$ via Lemma \ref{Lemma_approximation}, such that if the smallness assumptions are satisfied, then there exists $h\in C^{1,\alpha_0}$ with universal estimates such that
\begin{align*}
\|u-h\|_{L^\infty(B_{1/2})}\leq \varepsilon_0.
\end{align*}
Let 
\begin{align*}
\ell(x)=h(0)+Dh(0)\cdot x
\end{align*}
and compute
\begin{align*}
\sup_{B_\rho}|u(x)-\ell(x)|\leq&\sup_{B_\rho}|h(x)-\ell(x)|+\sup_{B_\rho}|u(x)-h(x)|\\
\leq &C\rho^{1+\alpha_0}+\varepsilon_0,
\end{align*}
where $C>0$ and $\alpha_0\in(0,1)$ are universal constants. We now make the universal choices
\begin{align*}
\rho=\left(\frac{1}{2C}\right)^\frac{1}{\alpha_0-\alpha}, \quad \varepsilon_0=\frac{1}{2}\rho^{1+\alpha}
\end{align*}
for $\alpha\in(0,\alpha_0)$ to be chosen later. This concludes the proof.
\end{proof}

The next result extends the oscillation control
from Lemma \ref{Lemma_first_it} to discrete scales $\rho^n$, for every $n\in\Nn$. Furthermore, it
provides a control on the coefficients of the approximating polynomials.

\begin{Lemma}\label{Lemma_geometric_iterations}
Suppose \ref{Assumption1}-\ref{Assumption3} are in force and let $u\in C(B_1)$ be a solution to \eqref{eq_var_coef}. Then there exists a sequence of affine functions $(\ell_n)_n$ of the form
\begin{align*}
\ell_n(x)=a_n+b_n\cdot x
\end{align*}
satisfying 
\begin{align}\label{eq_it_1}
	\|u-\ell_n\|_{L^\infty(B_{\rho^n})}\leq \rho^{n(1+\alpha)}
\end{align}
and
\begin{align}\label{eq_it_2}
|a_n-a_{n-1}|+\rho^n|b_n-b_{n-1}|\leq C\rho^{(n-1)(1+\alpha)}
\end{align}
for every $n\in \Nn$, where $C$ depends only on $d, \lambda$ and $\Lambda$, $\alpha=\min\{\alpha_0^-,1-a\}$ and $a$ is given by \ref{Assumption3}.



\end{Lemma}


\begin{proof}
We argue by induction on $n\in \Nn$. For simplicity of presentation, we split the proof into three steps.

\textbf{Step 1 -- } Let $\ell_0\equiv 0$ and set
\begin{align*}
\ell_1(x):=h(0)+Dh(0)\cdot x
\end{align*}
where $h$ is the function approximating $u$ from the proof of Lemma \ref{Lemma_approximation}. Then, owing to this lemma, \eqref{eq_it_1} and \eqref{eq_it_2} are readily verified in the case $n=1$.

\textbf{Step 2 -- } Suppose now the claim has been verified for $n=k$ and consider the case $n=k+1$. Let $v_k$ be defined as
\begin{align*}
v_k(x)=\frac{(u-l_k)}{\rho^{k(1+\alpha)}}\left(\rho^kx\right).
\end{align*}
Clearly, $v_k$ solves, for $x\in B_1$, the equation
\begin{align*}
\rho^{(1-\alpha-a)k}F\left(\rho^{(\alpha-1)k}D^2v_k,\rho^kx\right)=\rho^{(1-\alpha-a)k}f\left(\rho^kx\right).
\end{align*}
Call $F^k(M,x):=\rho^{(1-\alpha)k}F\left(\rho^{(\alpha-1)k}M,x\right)$ and note that $F^k$ still satisfies assumptions \ref{Assumption1}-\ref{Assumption2} with the same constants and one can also check that \ref{Assumption3} holds with $\overline{F}(M)$ replaced with $\rho^{(1-\alpha)k} \overline{F}(\rho^{(\alpha-1)k}M)$, which still has the same ellipticity contants $\lambda, \Lambda$. Hence we can rewrite the previous equation as
\begin{align*}
F^k_{\rho^k}\left(D^2v_k,x\right)=f_k,
\end{align*}
where $\|f_k\|_{L^\infty(B_1)}\leq \delta_0$ by our choice $\alpha\leq 1-a$.

Because $\rho^k<1$, the smallness assumption on $\lambda$ in Lemma \ref{Lemma_first_it} is still verified.

Applying Lemma \ref{Lemma_approximation}, we find $\overline{h}\in C_{loc}^{1,\alpha_0}(B_{9/10})$ such that
\begin{align*}
\|v_k-\overline{h}\|_{L^\infty(B_{1/2})}\leq \varepsilon_0,
\end{align*}
with
\begin{align*}
\rho^{(1-\alpha)k} \overline{F}\left(\rho^{(\alpha-1)k}D^2\overline{h}\right)=0.
\end{align*}
From Lemma \ref{Lemma_first_it} applied to $v_k$, we get
\begin{align*}
\sup_{B_\rho}|v_k(x)-\overline{h}(0)-D\overline{h}(x)\cdot x|\leq \rho^{1+\alpha}
\end{align*}
for $\alpha=\min\{\alpha_0^-,1-a\}$. Finally, using the definition of $v_k$, we get
\begin{align*}
\sup_{B_{\rho^{k+1}}}|u(x)-\ell_{k+1}(x)|\leq \rho^{(1+\alpha)(k+1)},
\end{align*}
where
\begin{align*}
\ell_{k+1}(x)=\ell_k(x)+\rho^{(1+\alpha)k}\left(\overline{h}(0)+D\overline{h}(0)\cdot \rho^{-k}x\right).
\end{align*}

\textbf{Step 3 -- } To complete the induction step, we must verify that the coefficients in $\ell_{k+1}$ satisfy \eqref{eq_it_2}. Notice that
\begin{align*}
\begin{cases}
a_{n+1}-a_n=\rho^{(1+\alpha)k}\overline{h}(0)\\
b_{n+1}-b_n=\rho^{\alpha k}D\overline{h}(0).
\end{cases}
\end{align*}
Since $\rho^{(1-\alpha)k} \overline{F}\left(\rho^{(\alpha-1)k}D^2\overline{h}\right)=0$, where $\rho^{(1-\alpha)k} \overline{F}\left(\rho^{(\alpha-1)k} \cdot \right)$ has the same ellipticity constants as $\overline{F}$, $\overline{h}$ enjoys the same $C^{1,\alpha_0}$ estimates as $h$, as thus \eqref{eq_it_2} is verified.

\end{proof}

To conclude Theorem \ref{Theorem_C1alpha} it remains to treat the points where $\omega(x_0)>0$. We start with the very simple observation that solutions to
\begin{align*}
F(D^2u,x_0)=0,\quad \mbox{ in } B_1
\end{align*}
belong to $C^{1,\alpha_0}(B_{1/2})$, where $\alpha_0=\alpha_0(\lambda, \Lambda, d)$ is the same as in Lemma \ref{Lemma_approximation}. To see this, note that the operator $G(M):=\omega^{-1}(x_0)F(M,x_0)$ is uniformly elliptic with constants $\lambda$ and $\Lambda$. 

We then define the rescaling 
\begin{align*}
\tilde{u}(y)=\frac{r_1^{-2}u(r_1y)}{K}
\end{align*}
with
\begin{align*}
K:=r_1^{-2}\|u\|_{L^\infty(B_1)}+\delta^{-1}\|f\|_{L^\infty(B_1)}
\end{align*}
and
\begin{align*}
r_1=dist(x_0,\Gamma)/2
\end{align*}
Then $\tilde{u}$ solves
\begin{align*}
\frac{1}{ K}F(KD^2\tilde{u},r_1y)=\frac{f(r_1y)}{K}
\end{align*}
which we rewrite as
\begin{align*}
\tilde{F}(D^2\tilde{u},x)=\tilde{f}(x),\quad \mbox{ in } B_1
\end{align*}
where $\tilde{F}$ is uniformly elliptic, $\|\tilde{u}\|_{L^\infty(B_1)}\leq 1$ and $\|\tilde{f}\|_{L^\infty(B_1)}\leq \delta$.




Combining this observation with \ref{Assumption4}, we can immediately apply \cite[Theorem 8.3]{Caffarelli-Cabre_1995} and conclude that $u\in C^{1,\alpha}(x_0)$ at each $x_0\in \{\omega>0\}$ for every $\alpha<\alpha_0$ and satisfies
\begin{align*}
\|u\|_{C^{1,\alpha}(x_0)}\leq C\left(\|\tilde{u}\|_{L^\infty(B_1)}+\|\tilde{f}\|_{L^\infty(B_1)}\right)
\end{align*}
where $C=C(\alpha, x_0,\omega,\lambda,\Lambda,d)$. Combining this result with Lemma \ref{Lemma_geometric_iterations} we immediately get Theorem \ref{Theorem_C1alpha}.



\section{Some geometric computations and applications of the regularity estimates to cone-edge equations}\label{Section_conic}

The fully nonlinear operators considered in this work appear naturally in the framework of so-called {\sl conic} or {\sl cone-edge} metrics in conformal geometry (see e.g. \cite{haoFang1,haoFang2}). We expand on this connection in this section. Let $\mathcal M$ be a closed Riemannian manifold and consider the conformal metric $g=g_u=e^{2u}g_0$ where $g_0$ is a given smooth background metric. Call $Ric_g$, $R_g$ and $A_g$ the Ricci curvature, scalar curvature and the Schouten tensor with respect to $g$, respectively. Then standard computations give 
\begin{align*}
Ric_g=&Ric_{g_0}-(d-2)\nabla^2u-\Delta u\cdot g_0+(d-2)du\otimes du-(d-2)|\nabla u|^2g_0,\\
R_g=&e^{-2u}\left(R_{g_0}-2(d-1)\Delta u-(d-1)(d-2)|\nabla u|^2\right),\\
A_g=&A_{g_0}-\nabla^2u+du\otimes du-\frac{1}{2}|\nabla u|^2g_0.
\end{align*}
Consider the conformal divisor
\begin{align*}
D=\sum_{i=1}^q\beta_ip_i
\end{align*}
where $p_i\in \mathcal M$ are distinct points and $\beta_i\in (-1,0)$. We say that $g_u$ is a cone-edge metric representing the conformal divisor $D$ if the conformal factor writes
\begin{align}\label{eq_divisor}
u(x)=\beta_i\ln|x-p_i|+v_i(x) \mbox{ as } x\to p_i \mbox{ for } i=1, ...\,, q
\end{align}
where $v_i$ are bounded functions on $\mathcal M$.

Define also the family of fully nonlinear  operators 
\begin{align*}
\sigma_k(g^{-1}A_g)=\sum_{1\leq i_1<...<i_k\leq d}\lambda_{i_1}...\lambda_{i_k}
\end{align*}
where $\lambda(A)$ are the eigenvalues of $A$.

Then the problem is to find $u\geq 0$ satisfying the singular conditions \eqref{eq_divisor}
\begin{align*}
\sigma_k(g^{-1}A_g)=c>0.
\end{align*}

Using \eqref{eq_divisor} we get, for $x\approx p_i$, $i=1, ...\,, q-1$,
\begin{align*}
Ric_g=&\,(d-2)\nabla^2u+(d-2)du\otimes du+(-\Delta u-(d-2)|\nabla u|^2),\\
R_g=&\,|x-p_i|^{-2\beta_i}e^{-v_i}(-2(d-1)\Delta u-(d-1)(d-2)|\nabla u|^2),\\
g^{-1}A_g=&\,g_E^{-1}|x-p_i|^{-2\beta_i}e^{-v_i}\left(-\nabla^2u+du\otimes du-\frac{|\nabla u|^2}{2}g_E\right),
\end{align*}
with $-2\beta_i\in(0,2)$. Therefore
\begin{align*}
\sigma_k(g^{-1}A_g)=|x-p_i|^{-2\beta_ik}\sigma_k\left( g_E^{-1}e^{-v_i}\left(-\nabla^2u+du\otimes du-\frac{|\nabla u|^2}{2}g_E \right)\right).
\end{align*}
The following natural cone condition ensures that the operator $\sigma_k$ is elliptic
\begin{align*}
g\in \mathcal{C}_k^+:=\{g:\sigma_1(M_g)>0, ...\,, \sigma_k(M_g)>0\},
\end{align*}
where 
\[
	M_g:=g_E^{-1}e^{-v_i}\left(-\nabla^2u+du\otimes du-\frac{|\nabla u|^2}{2}g_E \right).
\]
For $B\geq 0$ we have
\begin{align*}
\sigma_k(A+B)-\sigma_k(A)\geq& \sum_{j=1}^kC(j,d)\lambda_1(B)^j\sigma_{k-j}(A),\\
\sigma_k(A+B)-\sigma_k(A)\leq& \sum_{j=1}^kC(j,d)\lambda_d(B)^j\sigma_{k-j}(A).
\end{align*}
If we further assume, a priori, that $u\in C^{1,1}$ then 
\begin{align*}
\sigma_k\left( g_E^{-1}e^{-v_i}\left(-\nabla^2u+du\otimes du-\frac{|\nabla u|^2}{2}g_E \right)\right)=:F(D^2u,x),
\end{align*}
where $F$ is uniformly elliptic, since the lower order derivatives of $u$ can be considered as coefficients of $F$ and the bound on the second derivatives of $u$ implies uniform ellipticity of $F$. If finally we assume $-2\beta_i k<1$ then $\omega(x)=|x-p_i|^{-2\beta_i k}\in A_p$ locally around $p_i$ and we fall into the framework of our theorem. 


\section{Appendix}\label{Appendix}

We collect  some useful properties of Muckenhoupt weights which were used in the paper. First, we recall that these weights are also doubling.

\begin{Proposition}\label{Prop_Ap_doubling}$A_p$ implies doubling]
If $\nu\in A_p$ then $d\mu=\nu(x)dx$ is a doubling measure, that is, there exists $C>1$ such that for every ball $B_r(x)$ it holds
\begin{align*}
0<\mu(B_{r}(x))\leq C \mu(B_{r/2}(x))<\infty.
\end{align*}
\end{Proposition}

Next we state a crucial fact about these weights, which will allow us to control translations and rescalings of sets. 

\begin{Proposition}\label{Proposition_control_Ap}
Let $\nu\in A_p$, $p\in(1,\infty)$ and $B$ be either a ball or a cube. Then there exists $q\geq 1$ such that, for every $F\subset B$,
\begin{align*}
    c^{-1}\left(\frac{|F|}{|B|}\right)^q\leq \left(\frac{\nu(F)}{\nu(B)}\right)\leq c\left(\frac{|F|}{|B|}\right)^\frac{1}{q},
\end{align*}
where $c=c(\nu)>1$.
\end{Proposition}
For a proof of this fact, see \cite[Chapter 15]{Heinonen-kilpelainen-Martio_1993}.




\section*{Acknowledgements}
This work was initiated while the first author was visiting the Department of Mathematics at Johns Hopkins University. The first author acknowledges the department for their hospitality. The second author is partially support by NSF DMS grant $2154219$, " Regularity {\sl vs} singularity formation in elliptic and parabolic equations". Both authors would like to thank Jonah Duncan for very thoughtful comments about the present work. 


\begin{thebibliography}{99}

\bibitem{Caffarelli-Cabre_1995}
L. Caffarelli, and X. Cabr\'e,
{\it Fully Nonlinear Elliptic Equations}, Colloquium Publications 43, American Mathematical Society, Providence, 1995.

\bibitem{Caffarelli-Crandall-Kocan-Swiech_1996} 
L. Caffarelli, M. Crandall, M. Kocan, and A. Swiech, {\it On viscosity solutions of fully nonlinear equations with measurable ingredients}, Commun. Pure Appl. Math. 49 (4), 365–398 (1996).

\bibitem{Caffarelli-Silvestre_2007} 
L. Caffarelli, and L. Silvestre, \textit{An extension problem related to the fractional Laplacian}, Comm. Partial Differential Equations 32 (8), 1245-1260 (2007).

\bibitem{CMP} D. Cao,  T. Mengesha,  and T. Phan, {\it Weighted $W^{1,p}$ estimates for weak solutions of degenerate and singular elliptic equations}, Indiana Univ. Math. J. 67 (2018), no. 6, 2225--2277.



\bibitem{Crandall-Ishii-Lions_1992} 
M. Crandall, H. Ishii, and P.-L.Lions,  
\textit{User’s guide to viscosity solutions of second order partial differential equations}, Bull. Amer. Math. Soc. (N.S.) 27, 1-67 (1992).

\bibitem{DFM1}
G. David, J. Feneuil and S. Mayboroda, 
{\it   Dahlberg's theorem in higher co-dimension}, J. Funct. Anal. 276, no. 9, 2731-2820 (2019). 

\bibitem{DFM2}
G. David, J. Feneuil and S. Mayboroda, 
{\it    A new elliptic measure on lower dimensional sets}, Acta Math. Sin. (Engl. Ser.) 35, no. 6, 876-902 (2019).

\bibitem{DFM3}
G. David, J. Feneuil and S. Mayboroda, 
{\it     Elliptic theory for sets with higher co-dimensional boundaries}, Mem. Amer. Math. Soc. 274, no. 1346, vi+123 pp (2021). 


\bibitem{Dong-Kim_2015}
H. Dong, and D. Kim,
{\it Elliptic and parabolic equations with measurable coefficients in weighted Sobolev spaces}, Adv. Math. 274, 681–735 (2015).

 \bibitem{DP-2023} H. Dong, and T. Phan, {\it Weighted mixed-norm estimates for equations in non-divergence form with singular coefficients: the Dirichlet problem}, J. Funct. Anal. 285, no. 2, 109964.

\bibitem{DP-2021} H. Dong, and T. Phan, {\it Parabolic and elliptic equations with singular or degenerate coefficients: the Dirichlet problem}, Trans. Amer. Math. Soc. 374 (2021), 6611--6647.

\bibitem{DP19}
H. Dong, and T. Phan, {\it Regularity for parabolic equations with singular or degenerate coefficients}, Calc. Var. Partial Differential Equations 60 (2021), no. 1, Paper No. 44, 39 pp.

\bibitem{DP20}
H. Dong, and T. Phan, {\it On parabolic and elliptic equations with singular or degenerate coefficients}, to appear in Indiana U. Math. J., arXiv:2007.04385.

\bibitem{Dyda-Ihnatsyeva-Lehrback_2019}
B. Dyda, L. Ihnatsyeva, J. Lehrbäck, et al. ,
{\it Muckenhoupt Ap-properties of Distance Functions and Applications to Hardy–Sobolev -type Inequalities}, Potential Anal 50, 83–105 (2019).



\bibitem{Fabes-Jerison-Kenig_1982}
E. Fabes, D. Jerison, and C. Kenig,
{\it The Wiener test for degenerate elliptic equations}, Ann. Inst. Fourier (Grenoble) 32 (3), 151–182 (1982).

\bibitem{Fabes-Kenig-Serapioni_1982}
E. Fabes, D. Jerison, and C. Kenig,
{\it The local regularity of solutions of
degenerate elliptic equations}, Comm. Partial Differential Equations 7 (1), 77–116 (1982).

% \bibitem{Fabes-Jerison-Kenig_1982}
% E. Fabes, D. Jerison. C. Kenig,
% {\it Boundary behavior of solutions to degenerate elliptic equations},  In Conference on Harmonic Analysis in Honor of Antoni Zygmund,
% Vols. I, II. (Chicago, Ill., 1981), Wadsworth Math. Ser. Belmont, CA: Wadsworth, 577–589 (1983).

\bibitem{haoFang1}
H. Fang, and W. Wei, 
{\it $\sigma_2$ Yamabe problem on conic $4-$spheres.}
Calc. Var. Partial Differential Equations 58 (2019), no. 4, Paper No. 119, 19 pp. 

\bibitem{haoFang2}
H. Fang, and W. Wei, 
{\it $\sigma_2$ Yamabe problem on conic spheres II: Boundary compactness of the moduli.}
 Pacific J. Math. 311 (2021), no. 1, 33--51. 
 
\bibitem{Federer_1969}
H. Federer,
{\it Geometric measure theory}, Die Grundlehren der mathematischen Wissenschaften, Band 153. Springer-Verlag New York Inc., New York, 1969.

\bibitem{guoSong}
 B. Guo, and J. Song, {\it Schauder estimates for equations with cone metrics I}. Indiana Univ. Math. J. 70 (2021), no. 5, 1639--1676.

\bibitem{Heinonen-kilpelainen-Martio_1993} 
J. Heinonen, T. Kilpel\"ainen, and O. Martio, {\it Nonlinear Potential Theory of Degenerate Elliptic Equations}, Oxford University Press, 1993.


\bibitem{Krylov_1994} 
 N.V. Krylov, {\it  A Wp-theory of the Dirichlet problem for SPDE in general smooth domains}, Probab. Theory Relat. Fields, 98, 389-421 (1994).


\bibitem{Krylov_1999} 
 N.V. Krylov, {\it  Weighted Sobolev spaces and Laplace’s equation and the heat equations in a half
space}, Comm. Partial Differential Equations 24 (9–10), 1611–1653 (1999).



\bibitem{Krylov_2019} 
 N.V. Krylov, {\it  Weighted Aleksandrov estimates: PDE and stochastic versions}, Algebra
Anal. [in Russian] 31, No. 3, 134–169 (2019).

\bibitem{LST}
J. Lamboley, Y.  Sire, and E. V. Teixeira, {\it Free boundary problems involving singular weights}. 
Comm. Partial Differential Equations 45 (2020), no. 7, 758-775. 

\bibitem{Lin_1986} 
F. H. Lin, {\it Second derivative Lp estimates for elliptic equations of nondivergent
type}, Proc. Am. Math. Soc. 96 (3), 447–451 (1986).

\bibitem{Rozovskii_1990} 
B. L. Rozovskii, {\it Stochastic evolution systems}, Kluwer, Dordrecht, 1990. 

\bibitem{Silvestre-Teixeira_2015} 
L. Silvestre, and E. V. Teixeira, \textit{Regularity estimates for fully nonlinear elliptic
equations which are asymptotically convex}, Contributions to nonlinear elliptic equations and systems, Progr. Nonlinear Differential Equations Appl., 86,
Birkh\"auser/Springer, Cham,  425–438 (2015).


\bibitem{Sire-Terracini-Vita_2020} 
Y. Sire, S. Terracini, and S. Vita, {\it Liouville type theorems and regularity of solutions to degenerate or singular problems part I: even solutions}, Comm. Partial Differential Equations 46 (2), 310-361 (2020).


\bibitem{Sire-Terracini-Vita_2021} 
Y. Sire, S. Terracini, and S. Vita, {\it Liouville type theorems and regularity of solutions to degenerate or singular problems part II: odd solutions}, Mathematics in Engineering, 3 (1), 1-50 (2021).

\bibitem{STT2020}
Y. Sire, S. Terracini, and G. Tortone, {\it  On the nodal set of solutions to degenerate or singular elliptic equations with an application to s-harmonic functions}. J. Math. Pures Appl. (9) 143 (2020), 376-441. 




\end{thebibliography}




\end{document}
