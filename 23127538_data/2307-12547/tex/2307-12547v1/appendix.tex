\section{Appendix: Technical Preliminaries}

\subparagraph*{Treewidth. }
Let $G$ be a graph.  A {\em tree-decomposition} of a graph $G$ is a pair 
$(\mathbb{T},\mathcal{ X}=\{X_{t}\}_{t\in V({\mathbb T})})$,  where 
${\mathbb T}$ is a tree where every node $t\in V({\mathbb T})$ 
is assigned a subset $X_t\subseteq V(G)$, called a bag,  such that 
the following conditions hold. 
\begin{itemize}
%\setlength{\itemsep}{-2pt}
\item $\bigcup_{t\in V(\mathbb{T})}{X_t}=V(G)$,
\item for every edge $xy\in E(G)$ there is a $t\in V(\mathbb{T})$ such that  $\{x,y\}\subseteq X_{t}$, and 
\item for any $v\in V(G)$ the subgraph of $\mathbb{T}$ induced by the set  $\{t\mid v\in X_{t}\}$ is connected.
\end{itemize}

The {\em width} of a tree decomposition is $\max_{t\in V(\mathbb{T})} |X_t| -1$. The {\em treewidth} of $G$ 
is the  minimum width over all tree decompositions of $G$ and is denoted by $\tw(G)$. For a node $t \in \mathbb{T}$, let $\mathbb{T}_t$ be the subtree of $\mathbb{T}$ rooted at $t$. Then $\beta(t)$ is the subgraph of $G$ induced on $\bigcup_{t' \in \mathbb{T}_t} X_{t'}$.
 
A tree decomposition  $(\mathbb{T},\mathcal{ X})$ is called a {\em nice tree decomposition} if $\mathbb{T}$ is a tree rooted at some node $r$ where $ X_{r}=\emptyset$, each node of $\mathbb{T}$ has at most two children, and each node is of one of the following kinds:
\begin{itemize}
%%%\setlength\itemsep{-1mm}
%\setlength{\itemsep}{-2pt}
\item {\bf Introduce node}: a node $t$ that has only one child $t'$ where $X_{t}\supset X_{t'}$ and  $|X_{t}|=|X_{t'}|+1$.
\item {\bf Introduce edge node} a node $t$ labeled with an edge 
$uv$, with only one child $t'$ such that $\{u,v\}\subseteq X_{t'}=X_{t}$. 
This bag is said to introduce $uv$. 
\item {\bf  Forget vertex node}: a node $t$ that has only one child $t'$  where $X_{t}\subset X_{t'}$ and  $|X_{t}|=|X_{t'}|-1$.
\item {\bf Join node}:  a node  $t$ with two children $t_{1}$ and $t_{2}$ such that $X_{t}=X_{t_{1}}=X_{t_{2}}$.
\item {\bf Leaf node}: a node $t$ that is a leaf of $\mathbb T$, and $X_{t}=\emptyset$. 
\end{itemize}
We additionally require that every edge is introduced exactly once. 
One can  show that  a tree decomposition of width $t$ can be transformed into 
a nice tree decomposition of the same width $t$ and  with 
 $\mathcal{O}(t |V(G)|)$ nodes, see~e.g.~\cite{DBLP:books/sp/CyganFKLMPPS15}. 
%7 authors book
In this paper, we sometimes fix a vertex $v\in V(G)$ and include it in every bag of a nice tree decomposition $(\mathbb{T},\mathcal{X})$ of $G$, with the effect of the root bag and each leaf bag containing $v$. For the sake of brevity, we also call such a modified tree decomposition a nice tree decomposition. Given the tree $\mathbb{T}$ rooted at the node $r$, for any nodes $t_1,t_2 \in V(\mathbb{T})$, the distance between the two nodes in $\mathbb{T}$ is denoted by $\sf{dist}_\mathbb{T}(t,r)$.

Given two partitions $\mathbb{P}_1,\mathbb{P}_2$ of $V(G)$, $\mathbb{P}_1 \leq \mathbb{P}_2$ if $\mathbb{P}_2$ is coarser than $\mathbb{P}_1$ or, equivalently, $\mathbb{P}_1$ is finer than $\mathbb{P}_2$. In other words, each part of $\mathbb{P}_1$ is a subset of some part of $\mathbb{P}_2$. The join of the two partitions $\mathbb{P}_1 \Join \mathbb{P}_2$ is the finest partition that is coarser than both $\mathbb{P}_1$ and $\mathbb{P}_2$.

\section{Appendix: Omitted Proofs}

\begin{definition}[\kp]
Given a set $\XX=[n]$ of $n$ items with sizes $\theta_1,\ldots,\theta_n,$ values $p_1,\ldots,p_n$, capacity $b$ and target value $q$, compute if there exists a subset $\II \subseteq [n]$ such that $\sum_{i\in \II} \theta_i \leq b$ and $\sum_{i\in \II} p_i \geq q$. We denote an arbitrary instance of \kp by $(\XX,(\theta_i)_{i\in\XX}, (p_i)_{i\in\XX}, b,q)$.
\end{definition}

\begin{proof}[Proof of \Cref{thm:pa-star-npc}]
\pa clearly belongs to \NP. To show \NP-hardness, we reduce from \kp. Let $(\XX=[n],(\theta_i)_{i\in\XX}, (p_i)_{i\in\XX}, b,q)$ be an arbitrary instance of \kp. We consider the following instance $(\GG(\VV,\EE),(w(u))_{u\in\VV},(\alpha(u))_{u\in\VV},s,d)$ of \pa.
\begin{align*}
    &\VV = \{v_0, v_1, \ldots, v_n\}\\
    &\EE = \{\{v_0,v_i\}: 1\le i\le n\}\\
    &w(v_i) = \theta_i\, \alpha(v_i) = p_i;\forall i\in[n], w(v_0)=\alpha(v_0)=0;\\
    &s = b, d = q
\end{align*}
We now claim that the two instances are equivalent.

In one direction, let us suppose that the \kp instance is a \yes instance. Let $\WW\subseteq\XX$ be a solution of \kp. Let us consider $\UU=\{v_i: i\in\WW\}\cup\{v_0\}\subseteq\VV$. We observe that $\GG[\UU]$ is connected since $v_0\in\UU$. We also have
\[ w(\UU) = \sum_{i\in\WW} w(v_i) = \sum_{i\in\WW} \theta_i \le b = s,\]
and
\[ \alpha(\UU) = \sum_{i\in\WW} \alpha(v_i) = \sum_{i\in\WW} p_i \ge q = d.\]
Hence, the \pa instance is a \yes instance.

In the other direction, let us assume that the \pa instance is a \yes instance with $\UU\subseteq\VV$ be one of its solution. Let us consider a set $\WW=\{i: i\in[n], v_i\in\UU\}$. We now have
\[ \sum_{i\in\WW} \theta_i = \sum_{i\in\WW} w(v_i) = w(\UU) \le s = b, \]
and
\[ \sum_{i\in\WW} p_i = \sum_{i\in\WW} \alpha(v_i) = \alpha(\UU) \ge d=q.\]
Hence, the \kp instance is a \yes instance.
\end{proof}

% \begin{proof}[Proof of Correctness for \Cref{thm:treewidth-pa}]
% Recall that we are looking for a solution $\mathcal{U}$ that contains the fixed vertex $v$ that belongs to all bags of the tree decomposition. First, we show that a pair $(w,\alpha)$ belonging to $D[t,\mathbb{P}]$ for a node $t \in \mathbb{T}$ and a partition $\mathbb{P}$ of $X_t$ corresponds to a $\mathbb{P}$-subgraph $H$ in $\beta(t)$. Recall that $X_r = \{v\}$. Thus, this implies that a pair $(w,\alpha)$ belonging to $D[r,\mathbb{P} = (P_0 = \emptyset, P_1 = \{v\})]$ corresponds to a connected subgraph of $G$. Moreover, the output is a pair that is feasible and with the highest value. 

% In order to show that a pair $(w,\alpha)$ belonging to $D[t,\mathbb{P}]$ for a node $t \in \mathbb{T}$ and a partition $\mathbb{P}$ of $X_t$ corresponds to a $\mathbb{P}$-subgraph $H$ in $\beta(t)$, we need to consider the cases of what $t$ can be:
% \begin{enumerate}
%     \item When $t$ is a leaf node with $\ell(t) = i$, $X_t$ only contains $v$ and the update to $D$ is done such that $v$ is the corresponding subgraph to a stored pair. This is true in particular when $i =L$, the base case. From now we can assume that for a node $t$ with $\ell(t) = i < L$ all $D[t',\mathbb{P}']$ entries are correct and correspond to $\mathbb{P}'$-subgraphs in $\beta(t')$ when $\ell(t') > i$.
%     \item When $t$ is a forget vertex node, let $t'$ be the child node and $u \neq v$ be the vertex that is being forgotten. We copy pairs from $D[t',\mathbb{P}']$ depending on the structure of $\mathbb{P}'$. Since $\ell(t') > \ell(t)$, by induction hypothesis all entries in $D[t',\mathbb{P}']$ for any partition $\mathbb{P}'$ of $X_{t'}$ are feasible. From the cases considered, we copy a pair to $D[t,\mathbb{P}]$ from a $D[t',\mathbb{P}']$ only when $u$ is not part of the $\mathbb{P}'$-subgraph or is in a component of $\mathbb{P}'$ that has vertices in $X_t$. Thus, the same subgraph is a $\mathbb{P}$-subgraph in $\beta(t)$.
%     \item When $t$ is an introduce node, there is a child $t'$ we are introducing a vertex $u \neq v$ that has no adjacent edges added in $\beta(t)$. Since $\ell(t') > \ell(t)$, by induction hypothesis all entries in $D[t',\mathbb{P}']$ for any partition $\mathbb{P}'$ of $X_{t'}$ are feasible. We update pairs in $D[t,\mathbb{P}]$ from $D[t',\mathbb{P}']$ such that either $u$ is not considered as part of a $\mathbb{P}$-subgraph and the pair is certified by the $\mathbb{P}'$-subgraph, or $u$ is added to a $\mathbb{P}'$-subgraph in order to obtain a new $\mathbb{P}$-subgraph.
%     \item When $t$ is an introduce edge node, there is a child $t'$ such that $X_t = X_{t'}$ and the only difference is that two vertices $u,w$ in the bags $X_t = X_{t'}$ now have an edge in $\beta(t)$. Since $\ell(t') > \ell(t)$, by induction hypothesis all entries in $D[t',\mathbb{P}']$ for any partition $\mathbb{P}'$ of $X_{t'}$ are feasible. The updates are made in the cases when one of $u$ or $w$ are not in the intended $\mathbb{P}$-subgraph and the included pair is certified by a $\mathbb{P'}$-subgraph, or when the $u$ and $w$ belong to different components of a $\mathbb{P}'$-subgraph and the new $\mathbb{P}$-subgraph has these components merged as a single component.
%     \item When $t$ is a join node, there are two children $t_1,t_2$ such that $X_T = X_{t_1} = X_{t_2}$. Since $\ell(t_1),\ell(t_2) > \ell(t)$, by induction hypothesis all entries in $D[t_i,\mathbb{P}']$ for any partition $\mathbb{P}'$ of $X_{t_i}$ are feasible for $i \in \{1,2\}$.We update pairs in $D[t,\mathbb{P}]$ when there is a $\mathbb{P}$-subgraph in $\beta(t_1)$ and a $\mathbb{P}$-subgraph in $\beta(t_2)$ and we take the union of these two subgraphs to obtain a $\mathbb{P}$-subgraph in $\beta(t)$.
% \end{enumerate}
%  Thus in all cases of $t$, a pair added to $D[t,\mathbb{P}]$ for some partition of $X_t$ is a feasible pair. Recall that we also maintain undominated pairs at all times.

%  What remains to be shown is that an undominated feasible solution $\mathcal{U}$ of \pa in $G$ is contained in $D[r,\mathbb{P} = (P_0 = \emptyset, P_1 = \{v\})]$. Let $w$ be the weight of $\mathcal{U}$ and $\alpha$ be the value. Recall that $v \in \mathcal{U}$. For each $t$, we consider the subgraph $\beta(t) \cap \mathcal{U}$. Let $C_1,C_2,\ldots,C_m$ be components of $\beta(t) \cap \mathcal{U}$ and let for each $1\leq i \leq m, P_i = X_t \cap C_i$. Also, let $P_0 = X_t \setminus \mathcal{U}$. Consider $\mathbb{P} = (P_0,P_1,\ldots,P_m)$. The algorithm updates in $D[t,\mathbb{P}]$ the pair $(w',\alpha')$ for the subsolution $\beta(t) \cap \mathcal{U}$. Therefore, $D[r,\mathbb{P} = (\emptyset,\{v\})]$ contains the pair $(w,\alpha)$. Thus, we are done.

% \end{proof}

\begin{definition}[\pvc]
Given a graph $\GG=(\VV,\EE)$ and two integers $k$ and $\el$, compute if there exists a subset $\VV^\pr\subseteq\VV$ such that (i) $|\VV^\pr|\le k$ and (ii) there exist at least $\el$ edges whose at least one end point belong to $\VV^\pr$. We denote an arbitrary instance of \pvc by $(\GG,k,\el)$.
\end{definition}

\begin{proof}[Proof of \Cref{thm:vcs-woh}]
At a high-level, our reduction from \pvc is similar to the reduction in \Cref{thm:pa-gen-npc}. The only difference is that we will now have only one ``global vertex" instead of the ``global path" in the proof of \Cref{thm:pa-gen-npc}. Formally, our reduction is as follows.

Let $(\GG(\VV,\EE),k,\el)$ be an arbitrary instance of \pvc. We consider the following instance $(\GG^\pr(\VV^\pr,\EE^\pr),(w(u))_{u\in\VV},(\alpha(u))_{u\in\VV},s,d)$ of \pa.
\begin{align*}
    &\VV^\pr = \{u_i: i\in[n]\} \cup \{h_e: e\in \EE\}\cup\{g\}\\
    &\EE^\pr = \{\{u_i,h_e\}: i\in[n], e \in\EE, e\text{ is incident on }v_i\text{ in }\GG\}
    \cup \{\{u_i,g\}: i\in[n]\}\\
    &w(u_i) = 1, \alpha(u_i)=0, \forall i\in[n], 
    w(h_e)=0, \alpha(h_e)=1, \forall e\in\EE,\\
    &w(g)=0, \alpha(g)=0, s=k, d=\el
\end{align*}
We claim that the two instances are equivalent.

In one direction, let us suppose that the \pvc instance is a \yes instance. Let $\WW\subseteq\VV$ covers at least \el edges in \GG and $|\WW|\le k$. We consider $\UU=\{u_i: i\in[n],v_i\in\WW\}\cup\{h_e: e\in\EE, e\text{ is covered by }\WW\}\cup\{g\}\subseteq \VV^\pr$. We claim that $\GG^\pr[\UU]$ is connected. Since there is an edge between $u_i$ and $g$ for every $i\in[n]$, the induced subgraph $\GG^\pr[\{u_i: i\in[n],v_i\in\WW\}\cup\{g\}]$ is connected. Also, since $h_e$ belongs to \UU only if \WW covers $e$ in \GG, the induced subgraph $\GG^\pr[\UU]$ is connected. Now we have $w(\UU)=\sum_{i=1}^n w(u_i)\mathbbm{1}(u_i\in\UU) + w(g) + \sum_{e\in\EE, \WW\text{ covers }e} w(h_e)=|\UU|\le k$. We also have $\alpha(\UU)=\sum_{i=1}^n \alpha(u_i)\mathbbm{1}(u_i\in\UU) + \alpha(g) + \sum_{e\in\EE, \WW\text{ covers }e} \alpha(h_e)\ge\el$. Hence, the \pa instance is also a \yes instance.

In the other direction, let us assume that the \pa instance is a \yes instance. Let $\UU\subseteq\VV^\pr$ be a solution of the \pa instance. We consider $\WW=\{v_i: i\in[n]: u_i\in\UU\}$. Since $s=k$, we have $|\WW|\le k$. Also, since $d=\el$, we have $|\{h_e: e\in\EE\}\cap\UU|\ge\el$. We claim that \WW covers every edge in $\{e: e\in\EE, h_e\in\UU\}$. Suppose not, then there exists an edge $e\in\{e: e\in\EE, h_e\in\UU\}$ which is not covered by \WW. Then none of the neighbors of $h_e$ belongs to \UU contradicting our assumption that \UU is a solution and thus $\GG^\pr[\UU]$ should be connected. Hence, \WW covers at least \el edges in \GG and thus the \pvc instance is a \yes instance.

We also observe that if the size of a minimum vertex cover of the subgraph induced by any solution of the reduced \pvc instance is at most one more than the size of a minimum partial vertex cover --- the vertices in $\GG^\pr$ corresponding to a partial vertex cover and $g$ forms a vertex cover of the subgraph induced by a solution of the \pa instance. Also, all the numbers in our reduced \pa instance are at most the number of edges of the graph, and \pvc is \WOH parameterized by $k$. Hence, there is no algorithm for \pa running in time $\OO(f(vcs).\text{poly}(n,s,d))$ unless \ETH fails.
\end{proof}


\begin{proof}[Proof of \Cref{thm:path-gen-npc}]
\pathknapsack clearly belongs to \NP. To show \NP-hardness, we reduce from \hp. Let $(\GG,u,v)$ be an arbitrary instance of \hp. We consider the following instance $(\GG,(w_u)_{u\in\VV},(\alpha_u)_{u\in\VV},s,d,u,v)$ of \pathknapsack.
\[w_u=0,\alpha_u=1~\forall u\in\VV,s=0,d=n\]
We claim that the two instances are equivalent.

In one direction, let us assume that \hp is \yes. Let \PP be a Hamiltonian path between $u$ and $v$ in \GG. Then we have $w(\PP)=0\le s$ and $\alpha(\PP)=n\ge d$. Hence, the \pathknapsack instance is also a \yes instance.

In the other direction, let us assume that the \pathknapsack instance is a \yes instance. Let \PP be a path between $u$ and $v$ with $w(\PP)\le s=0$ and $\alpha(\PP)\ge d=n$. Since $\alpha(\VV)=n=\alpha(\PP)$ and there is no vertex with zero \alpha value, \PP must be a Hamiltonian path between $u$ and $v$. Hence, the \hp instance is also a \yes instance.

We observe that all the numbers in our reduced \pathknapsack instance are at most the number of vertices in the graph. Hence, our reduction shows that \pathknapsack is strongly \NPC.
\end{proof}

\begin{proof}[Proof of \Cref{thm:sol-path}]
Let $(\GG(\VV,\EE),(w(u))_{u\in\VV},(\alpha(u))_{u\in\VV},s,d,u,v)$ be an arbitrary instance of \pathknapsack and $k$ the number of vertices in the solution. We can assume without loss of generality that we know $k$ since there are only $n-1$ possible values of $k$ namely $2,3,\ldots,n$. We color every vertex uniformly randomly from a palette of $k$ colors independent of everything else. Let $\chi:\VV\longrightarrow[k]$ be the resulting coloring. For every non-empty subset $S\subseteq[k],S\ne\emptyset$ and vertex $x\in\VV$, we define a boolean variable PATH$(S,x)$ to be \true if there is a path which stars from $u$, ends at $x$, and contains exactly one vertex of every color in $S$; we call such a path $S$-colorful. If PATH$(S,x)$ is \true, then we also define $D[S,x]=\{(w,\alpha):\exists\text{ an $u$ to $x$ $S$-colorful path \PP such that }w(\PP)=w, \alpha(\PP)=\alpha\text{ for every other $u$ to $x$ $S$-colorful path \QQ, we have wither }w(\QQ)>w \text{ or }\alpha(\QQ)<\alpha\}$. For $|S|=1$, we note that PATH$(S,u)$=\true if and only if $\{\chi(u)\}=S$; PATH$(S,x)$=\false for every $x\in\VV\setminus\{u\}$ and $|S|=1$; $D[S,u]=\{(w(u),\alpha(u))\}$ if $\{\chi(u)\}=S$; $D[S,x]=\emptyset$ for every $x\in\VV\setminus\{u\}$ and $S\subseteq\VV$. We update $\text{PATH}(S,x)$ as per the following recurrence for $|S|>1$.
    \[
    \text{PATH}(S,x) = 
    \begin{cases}
        \bigvee\{\text{PATH}(S\setminus\{\chi(x)\},v): \{v,x\}\in\EE\} & \text{if } \chi(x)\in S\\
        \false & \text{otherwise}
    \end{cases}
    \]
    When we update any $\text{PATH}(S,x)$ to be \true, we update $D[S,x]$ as follows. For every $\{x,y\}\in\EE$ if $\text{PATH}(S\setminus\{\chi(x)\},y)$ is \true, then we do the following: for every $(w,\alpha)\in D[S\setminus\{\chi(x)\},y]$, we put $(w+w(x),\alpha+\alpha(x))$ in $D[S,x]$ if $w+w(x)\le s$; and we finally remove all dominated pairs from $D[S,x]$. We output \yes if $\text{PATH}([k],v)$ is \true and there exists a $(w,\alpha)\in D[[k],v]$ such that $w\le s$ and $\alpha\ge d$. Otherwise, we output \no.

    {\bf Proof of correctness:} If there does not exist any colorful path between $x$ and $y$ of length $k$, then the algorithm clearly outputs \no. Suppose now that the instance if a \yes instance. Then there exists a colorful path $\PP=(u_1(=x),u_2,\ldots,u_k(=y))$ between $x$ and $y$ such that $w(\PP)\le s$ and $\alpha(\PP)\ge d$. Let us define $S_i=\{u_j:j\in[i]\}$ for $i\in[k]$. Then $\text{PATH}(S_i,u_i)$ is \true and $D[S_i,u_i]$ either contains $(w(S_i),\alpha(S_i))$ or any pair which dominates $(w(S_i),\alpha(S_i))$ for every $i\in[k]$. Hence, the algorithm outputs \yes.

    {\bf Runtime analysis:} If there exists a colorful path between $x$ and $y$, then the algorithm finds it in time $\OO\left(2^k n^{\OO(1)}\right)$. If there exists a path between $x$ and $y$ containing $k-1$ edges (that is, $k$ vertices including $x$ and $y$), then one such path becomes colorful in the random coloring with probability at least
    \[ \frac{k!}{k^k}\ge e^{-k}. \]
    Hence, by repeating $\OO(e^k)$ times and outputting \yes if any run of the algorithm outputs \yes, the above algorithm achieves a success probability at least $2/3$. We can use the splitters to derandomize the algorithm above to obtain a deterministic algorithm for \pathknapsack which runs in time $\OO\left((2e)^k k^{\OO(\log k)}n^{\OO(1)}\right)$~\cite[Section 5.6.2]{DBLP:books/sp/CyganFKLMPPS15}.
\end{proof}