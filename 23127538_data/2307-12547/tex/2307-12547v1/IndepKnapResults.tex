\section{Results on \indpsetknapsack}

We now present our results for \indpsetknapsack. We show that \indpsetknapsack is strongly \NPC by reducing it from \indset which is defined as follows.

\begin{definition}[\indset]
Given a graph $\GG(\VV,\EE)$ and an integer $k$ compute if there exists a subset $\WW\subseteq\VV$ such that there is no edge whose both end points are in \WW and $|\WW|= k$. We denote an arbitrary instance of \indset by $(\GG,k)$.
\end{definition}

\begin{theorem}\label{thm:ind}
\indpsetknapsack is strongly \NPC.
\end{theorem}

\begin{proof}
\indpsetknapsack clearly belongs to \NP. To prove \NP-hardness, we reduce from \indset. Let $(\GG(\VV,\EE),k)$ be an arbitrary instance of \indset. We consider the instance $(\GG(\VV,\EE),(w(u))_{u\in\VV},(\alpha(u))_{u\in\VV},s,d)$ of \indpsetknapsack where
\begin{align*}
w(u)=\alpha(u)=1\;\forall u\in\VV, s=d=k
\end{align*}
We claim that the two instances are equivalent.

In one direction, let us assume that the \indset instance is a \yes instance. Let $\WW\subseteq\VV$ be an independent set of size $k$. Since $|\WW|=k$, we have $w(\WW)=\alpha(\WW)=k$. Hence, the \indpsetknapsack instance is also a \yes instance. In the other direction, let us assume that the \indpsetknapsack instance is a \yes instance. Then there exists a $\WW\subseteq\VV$ such that $w(\WW)\le k$ and $\alpha(\WW)\ge k$. This implies that we have $|\WW|\le k$ and $|\WW|\ge k$, and thus $|\WW|=k$. Moreover, \WW is an independent set in \GG. Hence, the \indset instance is a \yes instance.

We observe that all the numbers in our reduced \indpsetknapsack instance are at most the number of vertices in the graph. Hence, our reduction shows that \indpsetknapsack is strongly \NPC.
\end{proof}

We complement the hardness result of \Cref{thm:ind} by designing an algorithm for \indpsetknapsack running in time $\OO\left(2^{\tw}\cdot\text{poly}(\min\{s,d\})\right)^{\dagger^\star}$ following the idea of \Cref{thm:treewidth-pa} and the \FPT algorithm for \indset with respect to \tw~\cite{DBLP:books/sp/CyganFKLMPPS15}. We omit its proof from this shorter version due to its similarity with \Cref{thm:treewidth-pa}.

\begin{theorem}\label{thm:treewidth-indset}
There is an algorithm for \pa with runtime $2^{\tw}\cdot n^{\mathcal{O}(1)}\cdot {\sf min}\{s,d\}$ where $\tw$ is the treewidth of the input graph.
\end{theorem}

Using the technique in \Cref{thm-fptas}, we use \Cref{thm:treewidth-indset} in a black-box fashion to have the following approximation algorithm. We again omit its proof due to its similarity with \Cref{thm-fptas}.


\begin{theorem}\label{thm-fptas-indset}
There is an $(1-\eps)$ factor approximation algorithm for \pa for optimizing the value of the solution running in time $2^{\tw}\cdot \text{poly}(n,1/\eps)$ where \tw is the treewidth of the input graph.
\end{theorem}

We now consider the size $k$ of the solution as our parameter. We know that \indset is \WOH parameterized by the size of the independent set. A straight-forward reduction from \indset thus shows the following.

\begin{observation}\label{obs:ind-woh}
    \indpsetknapsack is \WOH parameterized by the size $k$ of the solution.
\end{observation}