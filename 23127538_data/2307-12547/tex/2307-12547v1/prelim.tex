\section{Preliminaries}

For any integer \el, we denote the sets $\{1,\ldots,\el\}$ and $\{0,1,\ldots,\el\}$ by $[\el]$ and $[\el]_0$ respectively. We now formally define our problems. Our first problem is \pa where the knapsack subset of items must induce a connected subgraph. We frame all our problems as decision problems so that we can directly use the framekwork of \NP-completeness.

\begin{definition}[\conknapsack]\label{def:pa}
Given an undirected graph $\GG=(\VV,\EE)$, weight of vertices $(w(u))_{u\in\VV}$, value of vertices $(\alpha(u))_{u\in\VV}$, size $s$ of knapsack, and target value $d$, compute if there exists a subset $\UU\subseteq\VV$ of vertices such that:
\begin{enumerate}
    \item \UU is connected,
    \item $w(\UU)=\sum_{u\in\UU} w(u) \le s$,
    \item $\alpha(\UU)=\sum_{u\in\UU} \alpha(u) \ge d$
\end{enumerate}
We denote an arbitrary instance of \pa by $(\GG,(w(u))_{u\in\VV},(\alpha(u))_{u\in\VV},s,d)$.
\end{definition}

In our next problem, the knapsack subset of items must be a path between two given vertices.

\begin{definition}[\pathknapsack]\label{def:path}
	Given an undirected graph $\GG=(\VV,\EE)$, weight of vertices $(w(u))_{u\in\VV}$, value of vertices $(\alpha(u))_{u\in\VV}$, size $s$ of knapsack, target value $d$, two vertices $x$ and $y$, compute if there exists a subset $\UU\subseteq\VV$ of vertices such that:
	\begin{enumerate}
		\item \UU is a path between $x$ and $y$ in $\GG$,
    \item $w(\UU)=\sum_{u\in\UU} w(u) \le s$,
    \item $\alpha(\UU)=\sum_{u\in\UU} \alpha(u) \ge d$
	\end{enumerate}
	We denote an arbitrary instance of \pathknapsack by $(\GG,(w(u))_{u\in\VV},(\alpha(u))_{u\in\VV},s,d,x,y)$.
\end{definition}

In \shortestpathknapsack, the knapsack subset of items must be a shortest path between two given vertices; everything else remains the same as \pathknapsack.

\longversion{\begin{definition}[\shortestpathknapsack]\label{def:path}
Given an undirected edge-weighted graph $\GG=(\VV,\EE,c:\EE\longrightarrow\RB_{\ge0})$, weight of vertices $(w(u))_{u\in\VV}$, value of vertices $(\alpha(u))_{u\in\VV}$, size $s$ of knapsack, target value $d$, two vertices $x$ and $y$, compute if there exists a subset $\UU\subseteq\VV$ of vertices such that:
\begin{enumerate}
	\item \UU is a shortest path between $x$ and $y$ in $\GG$,
    \item $w(\UU)=\sum_{u\in\UU} w(u) \le s$,
    \item $\alpha(\UU)=\sum_{u\in\UU} \alpha(u) \ge d$
\end{enumerate}
We denote an arbitrary instance of \shortestpathknapsack by $(\GG,(w(u))_{u\in\VV},(\alpha(u))_{u\in\VV},s,d,x,y)$.
\end{definition}
}

If not mentioned otherwise, we use $n,m,s,d,\GG$ and $\VV$ to denote respectively the number of vertices, the number of edges, the size of the knapsack, the target value, input graph, and the set of vertices in the input graph.\shortversion{ In the interest of space, we omit the preliminaries on treewidth. It is available in the appendix.}\longversion{ We now briefly recall the notion of treewidth of a graph.

\subparagraph*{Treewidth. }
Let $G$ be a graph.  A {\em tree-decomposition} of a graph $G$ is a pair 
$(\mathbb{T},\mathcal{ X}=\{X_{t}\}_{t\in V({\mathbb T})})$,  where 
${\mathbb T}$ is a tree where every node $t\in V({\mathbb T})$ 
is assigned a subset $X_t\subseteq V(G)$, called a bag,  such that 
the following conditions hold. 
\begin{itemize}
%\setlength{\itemsep}{-2pt}
\item $\bigcup_{t\in V(\mathbb{T})}{X_t}=V(G)$,
\item for every edge $xy\in E(G)$ there is a $t\in V(\mathbb{T})$ such that  $\{x,y\}\subseteq X_{t}$, and 
\item for any $v\in V(G)$ the subgraph of $\mathbb{T}$ induced by the set  $\{t\mid v\in X_{t}\}$ is connected.
\end{itemize}

The {\em width} of a tree decomposition is $\max_{t\in V(\mathbb{T})} |X_t| -1$. The {\em treewidth} of $G$ 
is the  minimum width over all tree decompositions of $G$ and is denoted by $\tw(G)$. For a node $t \in \mathbb{T}$, let $\mathbb{T}_t$ be the subtree of $\mathbb{T}$ rooted at $t$. Then $\beta(t)$ is the subgraph of $G$ induced on $\bigcup_{t' \in \mathbb{T}_t} X_{t'}$.
 
A tree decomposition  $(\mathbb{T},\mathcal{ X})$ is called a {\em nice tree decomposition} if $\mathbb{T}$ is a tree rooted at some node $r$ where $ X_{r}=\emptyset$, each node of $\mathbb{T}$ has at most two children, and each node is of one of the following kinds:
\begin{itemize}
%%%\setlength\itemsep{-1mm}
%\setlength{\itemsep}{-2pt}
\item {\bf Introduce node}: a node $t$ that has only one child $t'$ where $X_{t}\supset X_{t'}$ and  $|X_{t}|=|X_{t'}|+1$.
\item {\bf Introduce edge node} a node $t$ labeled with an edge 
$uv$, with only one child $t'$ such that $\{u,v\}\subseteq X_{t'}=X_{t}$. 
This bag is said to introduce $uv$. 
\item {\bf  Forget vertex node}: a node $t$ that has only one child $t'$  where $X_{t}\subset X_{t'}$ and  $|X_{t}|=|X_{t'}|-1$.
\item {\bf Join node}:  a node  $t$ with two children $t_{1}$ and $t_{2}$ such that $X_{t}=X_{t_{1}}=X_{t_{2}}$.
\item {\bf Leaf node}: a node $t$ that is a leaf of $\mathbb T$, and $X_{t}=\emptyset$. 
\end{itemize}
We additionally require that every edge is introduced exactly once. 
One can  show that  a tree decomposition of width $t$ can be transformed into 
a nice tree decomposition of the same width $t$ and  with 
 $\mathcal{O}(t |V(G)|)$ nodes, see~e.g.~\cite{}. 
%7 authors book
In this paper, we sometimes fix a vertex $v\in V(G)$ and include it in every bag of a nice tree decomposition $(\mathbb{T},\mathcal{X})$ of $G$, with the effect of the root bag and each leaf bag containing $v$. For the sake of brevity, we also call such a modified tree decomposition a nice tree decomposition. Given the tree $\mathbb{T}$ rooted at the node $r$, for any nodes $t_1,t_2 \in V(\mathbb{T})$, the distance between the two nodes in $\mathbb{T}$ is denoted by $\sf{dist}_\mathbb{T}(t,r)$.

Given two partitions $\mathbb{P}_1,\mathbb{P}_2$ of $V(G)$, $\mathbb{P}_1 \leq \mathbb{P}_2$ if $\mathbb{P}_2$ is coarser than $\mathbb{P}_1$ or, equivalently, $\mathbb{P}_1$ is finer than $\mathbb{P}_2$. In other words, each part of $\mathbb{P}_1$ is a subset of some part of $\mathbb{P}_2$. The join of the two partitions $\mathbb{P}_1 \Join \mathbb{P}_2$ is the finest partition that is coarser than both $\mathbb{P}_1$ and $\mathbb{P}_2$. 
}
We present our results in the next section.\shortversion{ In the interest of space, we omit proof of some of our results; they are marks $\star$.}