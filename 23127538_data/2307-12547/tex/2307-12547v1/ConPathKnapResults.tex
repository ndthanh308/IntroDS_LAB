\section{Results for \pathknapsack}

We now consider \pathknapsack. We show that \pathknapsack is strongly \NPC by reducing from \hp which is defined as follows.

\begin{definition}[\hp]
Given a graph $\GG(\VV,\EE)$ and two vertices $u$ and $v$, compute if there exists a path between $u$ and $v$ which visits every other vertex in \GG. We denote an arbitrary instance of \hp by $(\GG,u,v)$.
\end{definition}

\hp is known to be \NPC even for graphs with maximum degree three~\cite{DBLP:conf/stoc/GareyJS74}.

\begin{theorem}\shortversion{[$\star$]}\label{thm:path-gen-npc}
\pathknapsack is strongly \NPC even for graphs with maximum degree three.
\end{theorem}
\longversion{
\begin{proof}
\pathknapsack clearly belongs to \NP. To show \NP-hardness, we reduce from \hp. Let $(\GG,u,v)$ be an arbitrary instance of \hp. We consider the following instance $(\GG,(w_u)_{u\in\VV},(\alpha_u)_{u\in\VV},s,d,u,v)$ of \pathknapsack.
\[w_u=0,\alpha_u=1~\forall u\in\VV,s=0,d=n\]
We claim that the two instances are equivalent.

In one direction, let us assume that \hp is \yes. Let \PP be a Hamiltonian path between $u$ and $v$ in \GG. Then we have $w(\PP)=0\le s$ and $\alpha(\PP)=n\ge d$. Hence, the \pathknapsack instance is also a \yes instance.

In the other direction, let us assume that the \pathknapsack instance is a \yes instance. Let \PP be a path between $u$ and $v$ with $w(\PP)\le s=0$ and $\alpha(\PP)\ge d=n$. Since $\alpha(\VV)=n=\alpha(\PP)$ and there is no vertex with zero \alpha value, \PP must be a Hamiltonian path between $u$ and $v$. Hence, the \hp instance is also a \yes instance.

We observe that all the numbers in our reduced \pathknapsack instance are at most the number of vertices in the graph. Hence, our reduction shows that \pathknapsack is strongly \NPC.
\end{proof}
}
However, \pathknapsack is clearly polynomial-time solvable for trees, since there exist only one path between every two vertices in any tree.

\begin{observation}\label{obs:pk-tree-poly}
\pathknapsack is polynomial-time solvable for trees.
\end{observation}

One immediate natural question is if \Cref{obs:pk-tree-poly} can be generalized to graphs of bounded treewidth. The following result refutes the existence of any such algorithm.

\begin{theorem}\label{thm:pk-pathwidth}
\pathknapsack is \NPC even for graphs of pathwidth at most two. In particular, \pathknapsack is \PNPH parameterized by pathwidth.
\end{theorem}

\begin{proof}
We reduce from \kp to \pathknapsack. Let $(\XX=[n],(\theta_i)_{i\in\XX}, (p_i)_{i\in\XX}, b,q)$ be an arbitrary instance of \kp. We consider the following instance $(\GG(\VV,\EE),(w(u))_{u\in\VV},(\alpha(u))_{u\in\VV},s,d,u,v)$ of \pathknapsack.
\begin{align*}
\VV &= \{u_0\}\cup\{u_i,v_i,w_i: i\in[n]\}\\
\EE &= \{\{u_i,v_{i+1}\},\{u_i,w_{i+1}\}:0\le i\le n-1\}\cup\{\{u_i,v_{i}\},\{u_i,w_{i}\}:i\in[n]\}\\
w(v_i)&=\theta_i, \alpha(v_i)=p_i ~\forall i\in[n], w(u_i)=\alpha(u_i)=0~\forall i\in[n]_0, w(w_i)=\alpha(w_i)=0~\forall i\in[n]\\
s &= b, d=q, u=u_0, v=u_n
\end{align*}
\Cref{fig:enter-label} shows a schematic diagram of the reduced instance. We now claim that the two instances are equivalent.

% Figure environment removed

In one direction, let us suppose that the \kp instance is a \yes instance. Let $\II\subseteq\XX$ be such that $\sum_{i\in\II}\theta_i\le b$ and $\sum_{i\in\II}p_i\ge q$. We now consider the path \PP from $u_0$ to $u_n$ consisting of $2n$ edges where the $2i$-th vertex (from $u_0$) is $v_i$ if $i\in\II$ and $w_i$ otherwise for every $i\in[n]$; the $(2i-1)$-th vertex (from $u_0$) is $u_{i-1}$ for every $i\in[n]$ and the last vertex is $u_n$. Clearly, we have $w(\PP)=\sum_{i\in\II}\theta_i\le b=s$ and $\alpha(\PP)=\sum_{i\in\II}p_i\ge q=d$. Hence, the \pathknapsack instance is a \yes instance.

In the other direction, let us assume that the \pathknapsack instance is a \yes instance. Let \PP be a path between $u$ and $v$ with $w(\PP)\le s=b$ and $\alpha(\PP)\ge d=q$. We observe that all paths from $u_0$ to $u_n$ in \GG has $2n$ edges; $(2i-1)$-th vertex is $u_{i-1}$ and $2i$-th vertex is either $v_i$ or $w_i$ for every $i\in[n]$ and the last vertex is $u_n$. We consider the set $\II=\{i\in[n]: \text{ $(2i-1)$-th vertex is $v_i$}\}\subseteq\XX$. We now have $\sum_{i\in\II}\theta_i=w(\PP)\le s=b$ and $\sum_{i\in\II}p_i=\alpha(\PP)\ge d=q$. Hence, the \kp instance is a \yes instance.

We also observe that the pathwidth of \GG is at most two (there is a path decomposition with bag size being at most three) which proves the result.
\end{proof}

\Cref{thm:pk-pathwidth} leaves the following question open: does there exist an algorithm for \pathknapsack which runs in time $\OO(f(\tw)\cdot \text{poly}(n,s,d))$? We answer this question affirmatively in the following result. We omit its proof from this shorter version, since the algorithm is very similar to \Cref{thm:treewidth-pa}.

\begin{theorem}\label{thm:treewidth-path}
There is an algorithm for \pathknapsack with runtime $2^{\tw\log \tw}\cdot n^{\mathcal{O}(1)}\cdot {\sf min}\{s^2,d^2\}$ where $\tw$ is the treewidth of the input graph.
\end{theorem}

Using the technique in \Cref{thm-fptas}, we use \Cref{thm:treewidth-path} in a black-box fashion to have the following approximation algorithm. We again omit its proof due to its similarity with \Cref{thm-fptas}.


\begin{theorem}\label{thm-fptas-path}
There is an $(1-\eps)$ factor approximation algorithm for \pathknapsack for optimizing the value of the solution running in time $2^{\tw\log \tw}\cdot \text{poly}(n,1/\eps)$ where \tw is the treewidth of the input graph.
\end{theorem}

We next consider the size of the minimum vertex cover of the subgraph induced by the solution $\WW\subseteq\VV[\GG]$. We observe that the size of the minimum vertex cover of $\GG[\WW]$ is at least half of $|\WW|$ since there exists a Hamiltonian path in $\GG[\WW]$. Hence, it is enough to design an \FPT algorithm with parameter $|\WW|$. Our algorithm is based on color coding technique~\cite{DBLP:books/sp/CyganFKLMPPS15}.

\begin{theorem}\shortversion{[$\star$]}\label{thm:sol-path}
    There is an algorithm for \pathknapsack running in time $\OO\left((2e)^k k^{\OO(\log k)}n^{\OO(1)}\right)$ where $k$ is the number of vertices in the solution.
\end{theorem}
\longversion{
\begin{proof}
    Let $(\GG(\VV,\EE),(w(u))_{u\in\VV},(\alpha(u))_{u\in\VV},s,d,u,v)$ be an arbitrary instance of \pathknapsack and $k$ the number of vertices in the solution. We can assume without loss of generality that we know $k$ since there are only $n-1$ possible values of $k$ namely $2,3,\ldots,n$. We color every vertex uniformly randomly from a palette of $k$ colors independent of everything else. Let $\chi:\VV\longrightarrow[k]$ be the resulting coloring. For every non-empty subset $S\subseteq[k],S\ne\emptyset$ and vertex $x\in\VV$, we define a boolean variable PATH$(S,x)$ to be \true if there is a path which stars from $u$, ends at $x$, and contains exactly one vertex of every color in $S$; we call such a path $S$-colorful. If PATH$(S,x)$ is \true, then we also define $D[S,x]=\{(w,\alpha):\exists\text{ an $u$ to $x$ $S$-colorful path \PP such that }w(\PP)=w, \alpha(\PP)=\alpha\text{ for every other $u$ to $x$ $S$-colorful path \QQ, we have wither }w(\QQ)>w \text{ or }\alpha(\QQ)<\alpha\}$. For $|S|=1$, we note that PATH$(S,u)$=\true if and only if $\{\chi(u)\}=S$; PATH$(S,x)$=\false for every $x\in\VV\setminus\{u\}$ and $|S|=1$; $D[S,u]=\{(w(u),\alpha(u))\}$ if $\{\chi(u)\}=S$; $D[S,x]=\emptyset$ for every $x\in\VV\setminus\{u\}$ and $S\subseteq\VV$. We update $\text{PATH}(S,x)$ as per the following recurrence for $|S|>1$.
    \[
    \text{PATH}(S,x) = 
    \begin{cases}
        \bigvee\{\text{PATH}(S\setminus\{\chi(x)\},v): \{v,x\}\in\EE\} & \text{if } \chi(x)\in S\\
        \false & \text{otherwise}
    \end{cases}
    \]
    When we update any $\text{PATH}(S,x)$ to be \true, we update $D[S,x]$ as follows. For every $\{v,x\}\in\EE$ if $\text{PATH}(S\setminus\{\chi(x)\},v)$ is \true, then we do the following: for every $(w,\alpha)\in D_v$, we put $(w+w(x),\alpha+\alpha(x))$ in $D_x$ if $w+w(x)\le s$; and we finally remove all dominated pairs from $D_x$. We output \yes if $\text{PATH}([k],y)$ is \true and there exists a $(w,\alpha)\in D_y$ such that $w\le s$ and $\alpha\ge d$. Otherwise, we output \no.

    {\bf Proof of correctness:} If there does not exist any colorful path between $x$ and $y$ of length $k$, then the algorithm clearly outputs \no. Suppose now that the instance if a \yes instance. Then there exists a colorful path $\PP=(u_1(=x),u_2,\ldots,u_k(=y))$ between $x$ and $y$ such that $w(\PP)\le s$ and $\alpha(\PP)\ge d$. Let us define $S_i=\{u_j:j\in[i]\}$ for $i\in[k]$. Then $\text{PATH}(S_i,u_i)$ is \true and $D[S_i,u_i]$ either contains $(w(S_i),\alpha(S_i))$ or any pair which dominates $(w(S_i),\alpha(S_i))$ for every $i\in[k]$. Hence, the algorithm outputs \yes.

    {\bf Runtime analysis:} If there exists a colorful path between $x$ and $y$, then the algorithm finds it in time $\OO\left(2^k n^{\OO(1)}\right)$. If there exists a path between $x$ and $y$ containing $k-1$ edges (that is, $k$ vertices including $x$ and $y$), then one such path becomes colorful in the random coloring with probability at least
    \[ \frac{k!}{k^k}\ge e^{-k}. \]
    Hence, by repeating $\OO(e^k)$ times and outputting \yes if any run of the algorithm outputs \yes, the above algorithm achieves a success probability at least $2/3$. We can use the splitters to derandomize the algorithm above to obtain a deterministic algorithm for \pathknapsack which runs in time $\OO\left((2e)^k k^{\OO(\log k)}n^{\OO(1)}\right)$~\cite[Section 5.6.2]{DBLP:books/sp/CyganFKLMPPS15}.
\end{proof}
}
\Cref{thm:sol-path} immediately implies the following result.

\begin{corollary}\label{cor:vcs-path}
    There is an algorithm for \pathknapsack running in time $\OO\left((2e)^{2vcs} vcs^{\OO(\log vcs)}n^{\OO(1)}\right)$ where $vcs$ is the size of the minimum vertex cover of the subgraph induced by the solution.
\end{corollary}