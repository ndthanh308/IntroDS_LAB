\documentclass[12pt]{article}
\newcommand{\myparagraph}[1]{\par\smallskip\par\noindent{\bf{}#1:~}}
\newcommand{\alg}[1]{\mbox{\sf #1}}  %% Names of algorithms in small caps
\newcommand{\algsub}[1]{\mbox{\sf\scriptsize #1}}
     %% Names of algorithms in small caps to use in subscripts

\usepackage[svgnames]{xcolor}
\usepackage{amssymb}
\usepackage{amsthm}
\usepackage{amsmath}
\usepackage{todonotes}
\usepackage{fullpage}
\usepackage{charter}
\usepackage{csquotes}
\usepackage{url}
\MakeOuterQuote{"}

\usepackage{spverbatim}

\definecolor{cadmiumgreen}{rgb}{0.0, 0.42, 0.24}


\newif\ifverbose
% \verbosefalse %%% Uncomment / Comment this for viewing without / with author comments
\verbosetrue %%% Comment / Uncomment this for viewing without / with author comments
\newcommand{\revcmt}[1]  {\ifverbose {\noindent \textcolor{IndianRed}{{\bf Reviewer's comment: }{\em #1}} } \else \fi }
\newcommand{\palashrevcmt}[1]{{\color{IndianRed}#1}}
\newcommand{\response}[1]  {\ifverbose {\noindent {\bf\color{cadmiumgreen} Authors' response: }{\color{cadmiumgreen}#1}} \else \fi }
\newcommand{\change}[1]  {\ifverbose \textcolor{blue}{{\noindent {\bf Change(s) in the manuscript: }{#1}} } \else \fi }

\begin{document}

\noindent Dear Editor,
\bigskip

\noindent We would like to submit for your consideration the revision of the paper
``Knapsack: Connectedness, Path, and
Shortest-Path.''
\bigskip

We thank the reviewer for his/her comments. We have incorporated all the suggestions the reviewer has proposed in this revised version of the draft.
\bigskip

A more detailed account is enclosed in this letter. We look forward to your feedback on this version, and we thank you again for your consideration.

\bigskip
\flushright

\noindent Sincerely,

\bigskip

\noindent Authors of paper id 9
%\noindent\textcolor{blue}{Text that should not be submitted to the editor is written in blue.}
%\bigskip
\flushleft
\newpage



\section*{Reviews and Response}

\subsection*{Review 1}

\medskip
\begin{spverbatim}
----------- Overall evaluation -----------
Summary of the paper
-------------------------
This paper studies certain variants of the Knapsack problem under the lens of parameterized complexity. In particular, the variants Connected Knapsack, Path Knapsack, and Shortest Path Knapsack are considered here. All these problems ask for a solution to the Knapsack problem such that this solution also satisfies some graph theoretic constraint.

* Connected Knapsack asks for a connected subset of items which has maximum value subject to the knapsack constraint.
* Path Knapsack asks for a subset of items - all of which are adjacent vertices on a path - which has maximum value subject to the knapsack constraint.
* Shortest Path Knapsack asks for a subset of items - all of which are adjacent vertices on a shortest path - which has maximum value subject to the knapsack constraint.

The main results shown in the paper are the following:

1. Connected Knapsack and Path Knapsack are strongly NP-hard.
2. Shortest Path Knapsack is NP-hard but not strongly NP-hard.
3. Connected Knapsack and Path Knapsack admit FPT algorithms parameterized by the treewidth of the input graph.
4. Shortest Path Knapsack admits a pseudo-polynomial time algorithm.
5. All these three problems admit an FPTAS for graphs with small (i.e., at most O(log n)) treewidth.

A reduction from vertex cover shows that Connected Knapsack is strongly NP-hard. A reduction from Hamiltonian Path shows that Path Knapsack is strongly NP-hard. The FPT algorithm for Connected Knapsack is based on dynamic programming. The FPT algorithm for Path Knapsack is very similar to the FPT algorithm for Connected Knapsack. Once the FPT algorithm is assumed as a black box, the FPTAS is straightforward.

It is also shown that Path Knapsack is NP-complete even for graphs of pathwidth at most two. This is by a reduction from Knapsack and all the paths constructed in this reduction have the same length. This leads to the NP-completeness of Shortest Path Knapsack. The pseudo-polynomial time algorithm for Shortest Path Knapsack is greedy and DP based.


Evaluation
------------
I think the paper shows some reasonably interesting results for some variants of the Knapsack problem. Ref. [14] by Ito et al. (2008) studied a generalization of these problems called  the Maximum-Partition problem in series-parallel graphs - they showed an FPTAS for this more general problem - moreover, they say in their conclusions "The FPTAS for series-parallel graphs can be extended to that for partial k-trees although it would become much more complicated".

As this LATIN 2024 submission says "To the best of our knowledge, there is no follow-up work where the techniques for bounded treewidth graphs are explained". Indeed, there seems to have been no follow-up paper.

I did not find this submission particularly readable - in particular, I found it difficult to parse through the DP algorithm and its proof of correctness. I think the paper can be accepted if there is room.

\end{spverbatim}

\response{Nothing to update in the revised version based above comments.}
% \change{Our changes...}


\subsection*{Review 2}

\medskip

\begin{spverbatim}
----------- Overall evaluation -----------
This paper examines  an interesting extension of the Knapsack  problem where the items are vertices in  a graph G. The set chosen to fit  in the knapsack must in addition fulfill some property in G. The authors examine three such properties and the corresponding problems. The vertices in the knapsack should induce 1) a connected subgraph  2) a path  between two specified  vertices or 3)  a shortest path between two specified vertices.


To  my knowledge  the  problems  defined are  new,  which is  somewhat surprising given  how heavily  Knapsack and  its extensions  have been studied. The  authors present  several results  on the  three problems including  NP-completeness  status, pseudopolynomial  algorithms  for fixed treewidth and/or FPTAS for  fixed treewidth. They are summarized in Table 1. Interestingly,  using Dikstra's label-setting approach the FPTAS for Shortest Path Knapsack works for any treewidth value.


The algorithms contain some  interesting problem-specific insights but overall follow the time-tested  dynamic programming approaches used in the treewidth  literature. The submission is  reasonably well-written,
some minor comments follow.



SPECIFIC COMMENTS:

p. 1, Abstract: "Our results show".
"Suggest" would be more accurate. There are no
lower-bound proofs.

\end{spverbatim}
\change{Done.}
\begin{spverbatim}

p. 2, Section 1.1.  "We summarize our results in Table 1".

\end{spverbatim}
\change{Done.}
\begin{spverbatim}

p. 2, last sentence of Section 1.1. Similarly to the first
comment, I recommend restating as follows: "According to our
results... seems to be computationally the easiest...".

\end{spverbatim}
\change{Done.}
\begin{spverbatim}

p. 4, definition of Shortest Path Knapsack. Mention that you require
nonnegative edge weights.

\end{spverbatim}
\change{Done.}
\begin{spverbatim}

p. 5, line 1. "which is known".

\end{spverbatim}
\change{Done.}
\begin{spverbatim}

p. 7, line 5. At first read it took me a while to locate
the definition of the $\beta()$ operator, it can only be
found in the appendix.
\end{spverbatim}
\change{The preliminary section has been expanded to include required definitions of treewidth.}
\begin{spverbatim}


p. 7. Change "updation" to "update".
\end{spverbatim}
\change{Done.}
\begin{spverbatim}

p. 8, Case 5 of the algorithm. Shouldn't in the last line have
$-w_{\mathcal{P}}$, i.e., subtract as opposed to add? Similarly
for the a-values. Else you double count.
\end{spverbatim}
\change{This was a typo and has been corrected.}
\begin{spverbatim}

p. 9, Case 5 of the correctness. You should mention for clarity that
the intersection of $\beta(t_1)$ and $\beta(t_2)$ is a subset of $X_t.$

\end{spverbatim}
\change{This has been specified.}
\begin{spverbatim}

p. 13, l. 8. "and for every other $x$".

\end{spverbatim}
\change{Done.}
\begin{spverbatim}

p. 20, proof of Theorem 10. Change "wither" to "either".

\end{spverbatim}
\change{Done.}

\subsection*{Review 3}

\medskip

\begin{spverbatim}
----------- Overall evaluation -----------
In the present article, the authors study variants of the knapsack problem, where the items to choose are vertices in a graph and there are restrictions over the subgraphs induced by the set of chosen items. In particular, they consider Connected Knapsack, Path Knapsack and Shortest Path Knapsack, where the solutions (the set of chosen items) must be such that they respect the size restrictions of the knapsack while maximizing their value plus they induce a connected subgraph, a path and a shortest path, respectively. The authors prove that the three problems are NP-complete and present algorithms to solve them. These algorithms are FPT parameterized by the treewidth of input graph, the size of the knapsack and the target value of the solution. Furthermore, they provide FPTASes for these problems on graphs of bounded treewidth (and on general graphs for Shortest Path Knapsack).
The algorithm presented in Theorem 3 seems to be correct but the proof needs to be carefully rewritten before being published, since it is not entirely correct as it is (see comments below). Other parts of the article also need to be thoroughly revised (in particular: Theorem 8, Theorem 11, and the parameters defined for Theorem 5, Theorem 10 and Corollary 2).

As a result, I cannot recommend to accept the submission. However, I believe in the potential of the article and I strongly suggest the authors to consider the following comments before submitting again elsewhere.


List of comments and other corrections

- The notion of nice tree decomposition that is used in the main algorithm (Theorem 3) is the one mentioned in [1], which is a variant of the ``standard'' notion of nice tree decomposition.
It is also called edge nice tree decomposition to avoid confusion. I think this fact should be pointed out somewhere in the paper before the description of the algorithm, not only in the Appendix.

\end{spverbatim}
\change{The updates suggested to the tree decomposition portion have been made in the preliminaries.}
\begin{spverbatim}

- Throughout the text:
The choice of $\tau$ for referring to the treewidth can be confusing, since $\tau$ is usually used for the vertex cover number.

\end{spverbatim}
\change{We have replaced $\tau$ with tw.}
\begin{spverbatim}

- Throughout the text:
Revise the notation of ``-'', ``--'' and ``---'', in order to be consistent.

\end{spverbatim}
\change{Done.}
\begin{spverbatim}

- Throughout the text:
Fix the order of the references when there are multiple citations.

\end{spverbatim}
\change{Done.}
\begin{spverbatim}

- Page 1, abstract, line 10:
The complexity should be $O(2^{\tau \log\tau} \cdot \text{poly}(n) \cdot \min\{s^2, d^2\})$, where $n$ is the number of vertices of the graph.

\end{spverbatim}
\change{Done.}
\begin{spverbatim}

- Page 1, abstract, line 11:
It should be clear that $s$ is the size of the knapsack.

\end{spverbatim}
\change{Done.}
\begin{spverbatim}

- Page 1, abstract, line 16:
It should say ``is computationally the hardest''. Optionally, ``is computationally the hardest of the three problems''.

\end{spverbatim}
\change{Done.}
\begin{spverbatim}

- Page 2, Section 1.1, last line of the first paragraph:
It should say ``Table 1'' instead of ``Section 1.1''.

\end{spverbatim}
\change{Done.}
\begin{spverbatim}

- Page 2, Section 1.2, first paragraph:
The citation ``[20]'' should go directly after ``Yamada et al.''.

\end{spverbatim}
\change{Done.}
\begin{spverbatim}

- Page 4, Section 3, first paragraph:
Please revise the second sentence. Perhaps it should say ``Vertex Cover, which is known''.

\end{spverbatim}
\change{Done.}
\begin{spverbatim}

- Page 5, proof of Theorem 1, second paragraph:
There is an extra space in ``(iii )''.

\end{spverbatim}
\change{Done.}
\begin{spverbatim}

- Page 5, proof of Theorem 1, third paragraph, second sentence:
It should say ``of $\mathcal{G}$ with $|\mathcal{W}| \leq k$''.

\end{spverbatim}
\change{Done.}
\begin{spverbatim}

- Page 6, proof of Theorem 3, second sentence:
It should say ``$\subseteq$''.

\end{spverbatim}
\change{Corrected.}
\begin{spverbatim}

- Page 6, proof of Theorem 3, second paragraph, first sentence:
It should be the set of non-negative integers instead of ``$\{1, 2, \ldots, n\}$''.

\end{spverbatim}
\change{We have changed the range of the function to the set of whole numbers.}
\begin{spverbatim}

- Page 6, proof of Theorem 3, second paragraph, second and third sentences:
The definition $\ell(t) = \text{dist}_{\mathbb{T}}(t,r)$ works as well for $t = r$, there is no need to separate this case. Also, make sure it says $\text{dist}_{\mathbb{T}}(t,r)$ instead of $\text{dist}_{\mathbb{T}}\text{(t,r)}$.

\end{spverbatim}
\change{We have changed the paragraph so that $t=r$ is not read as a separate case. The notation we have used is $\sf{dist}$. }
\begin{spverbatim}

- Page 7, proof of Theorem 3, subsection ``States'', second item:
Please use a different letter, since it was mentioned before that $m$ was the number of edges. Also, it should be ``$m \leq \tau+1$'' instead of ``$m \leq \tau$'', because $\tau$ is the width of $\mathbb{T}$ and each of its bags contains $\tau+1$ vertices. This change should be done in other parts of this proof.
\end{spverbatim}
\change{We are not using the number of edges anywhere and hence have removed that notation from the preliminaries. The upper bound on m has been corrected here and everywhere else. }
\begin{spverbatim}


- Page 7, proof of Theorem 3, subsection ``Interpretation of States'':
Please revise it carefully. I understood the idea but it is not correctly explained. Also, in the definition of $\mathbb{P}$-subgraph, it should be pointed out that the number ``$m$'' of connected components of $H$ should be the same ``$m$'' as in the partition $\mathbb{P}$.

\end{spverbatim}
\change{This correction has been made. }
\begin{spverbatim}


- Page 7, proof of Theorem 3, subsection ``Dynamic Programming on $D$'', line 5:
It should say ``$0 \leq i < L$'' instead of ``$0 \leq i \leq L$''.

\end{spverbatim}
\change{Corrected. }
\begin{spverbatim}

- Page 7, proof of Theorem 3, subsection ``Dynamic Programming on $D$'', item 1:
Why is $(0,0)$ excluded in the case of $\mathbb{P} = \mathbb{P}^2_t$? According to the definition, a $\mathbb{P}$-subgraph could be an empty set (and its weight and value would be both 0).

\end{spverbatim}
\change{The proof has been updated so that each table entry is initialized and then as the algorithm goes forward the entries are updated.}
\begin{spverbatim}

- Page 7, proof of Theorem 3, subsection ``Dynamic Programming on $D$'', item 2, second sentence:
It should say ``but not to $X_t$''.
\end{spverbatim}
\change{Corrected. }
\begin{spverbatim}


- Page 7, proof of Theorem 3, subsection ``Dynamic Programming on $D$'', item 2, third sentence:
It is not clear if we take only one partition or if we iterate over all of them. It is also not clear in the proof of this case. This is an important step in the algorithm, without its clarification it cannot be considered correct.

\end{spverbatim}
\change{it is clarified that we do the procedure for any partition that satisfies the condition.}
\begin{spverbatim}


- Page 7, proof of Theorem 3, subsection ``Dynamic Programming on $D$'', item 2, last sentence:
What do you mean by ``we do not make any changes to $D[t, \mathbb{P}]$''? Note that $D[t, \mathbb{P}]$ is undefined when the algorithm starts. There are similar sentences later in the text.

\end{spverbatim}
\change{Since an initialization for each entry of the DP table has been added, this point should be clear now.}
\begin{spverbatim}

- Page 7, proof of Theorem 3, subsection ``Dynamic Programming on $D$'', item 3, fourth sentence:
It should say ``be the partition'' because it is unique.

\end{spverbatim}
\change{Corrected. }
\begin{spverbatim}


- Page 8, proof of Theorem 3, subsection ``Dynamic Programming on $D$'', item 4, second sentence:
It is not correct to say ``except for'' in this context, because the two sets are indeed equal, you are talking about differences between other objects.
\end{spverbatim}
\change{Corrected. }
\begin{spverbatim}

- Page 8, proof of Theorem 3, subsection ``Dynamic Programming on $D$'', item 5, last sentence:
It should say ``$w_1 + w_2 - w_{\mathcal{P}}$'' (both times) and ``$\alpha_1 + \alpha_2 - \alpha_{\mathcal{P}}$''.

\end{spverbatim}
\change{Corrected. }
\begin{spverbatim}

- Pages 7 to 9, proof of Theorem 3, subsections ``Dynamic Programming on $D$'' and ``Correctness of the Algorithm'', in general: It would be easier to read if these sections were merged, in such a way that the proof of each case was right after the definition of $D$ in that case.

\end{spverbatim}
\change{It is standard practice to write down the steps of the algorithm and then prove correctness. Please note that there are 2 directions for correctness: (i) that the entries are feasible, (ii) that the best solution is one of the entries. Therefore, we feel that our current layout explains this best.}
\begin{spverbatim}

- Pages 8 and 9, proof of Theorem 3, subsection ``Correctness of the Algorithm'', in general:
Please revise this section very carefully. The proofs of all the cases are not really proofs but mostly a description of the algorithm. There is an attempt of proving that the values stored in $D$ are correct (in the sense that they are feasible pairs for $t$ and $\mathbb{P}$) but no proof that there are no other undominated feasible pairs for $t$ and $\mathbb{P}$.

\end{spverbatim}
\change{The subsection has been revised. The proof for correctness of feasibility of saved entries is followed by the proof that undominated pairs are kept. Finally the proof that a best solution is part of the entries is given.}
\begin{spverbatim}

- Page 9, proof of Theorem 3, subsection ``Running time'', third sentence:
It should say ``are $O(n \cdot 2^{\tau \log \tau})$ instead of ``are at most $n \cdot 2^{\tau \log \tau}$''.

\end{spverbatim}
\change{We found several missing Order notations and have corrected them all.}
\begin{spverbatim}

- Page 9, proof of Theorem 3, subsection ``Running time'', line 7:
It should say ``Updating the DP table''. Also, that time complexity should be further explained.

\end{spverbatim}
\change{The time complexity has been explained further. Grammatical errors have been corrected.}
\begin{spverbatim}

- Page 11, first paragraph:
For a given graph, there could be different optimal solutions for the problem and they do not necessarily induce subgraphs with the same vertex cover number. As it is, this parameter is not well defined. Did you mean the minimum among all possible such values? Did you mean that you consider an extra parameter in the instance, $vcs$, and say that an instance is a YES instance if there exists a set that satisfied all the previously demanded properties and also is such that it induces a subgraph of vertex cover number equal to $vcs$? Notice that in the latter case it would be a slightly different problem and this should be pointed out.

\end{spverbatim}
\change{Our parameter is the maximum of the size of minimum vertex covers of solutions. We have updated our draft accordingly.}
\begin{spverbatim}

- Page 11, first paragraph, lines 6 and 7:
It should say ``polynomial time even for constant values'' instead of ``polynomial-time for even constant values''.

\end{spverbatim}
\change{Done.}
\begin{spverbatim}

- Page 12, first paragraph, last sentence:
The idea of the algorithm should be further explained. In which way is it similar to the other one? What do we store in the DP table? In the previous DP table, the states involved the different connected components, how is that modified for this problem?

As a consequence, I cannot verify the validity of Theorem 8. Neither of Theorem 9 because, even though in that case the similarity of the proof is clear, the time complexity depends on Theorem 8.


\end{spverbatim}
\change{The states and the interpretation of states for this case will be the same. It will just be that the connected components in this case shall be subpaths of the final solution path. Everything else remains the same and therefore, the running time will also be the same. }
\begin{spverbatim}
- Page 12, fifth paragraph and statements of Theorem 10 and Corollary 2:
Here we encounter a situation similar to the one mentioned in the previous comment about $vcs$. The definition of $vcs$ is not clear and neither is ``the number of vertices in the solution'', since different solutions may have different number of vertices. The proof of Theorem 10 makes it more confusing when it says ``We can assume without loss of generality that we know $k$ since there are only $n - 1$ possible values of $k$ namely $2, 3, \ldots, n$''.

\end{spverbatim}
\change{We can assume without loss of generality that we know $k$ since we can then start from $k=1$, increment $k$ in every iteration, till we obtain a solution. This will add only a multiplicative factor of $O(n)$ in our running time. We have now expplained this in our draft.}
\begin{spverbatim}

- Page 13, proof of Theorem 11, first paragraph, line 6:
It should say ``$\alpha$ and for every''.

\end{spverbatim}
\change{Done.}
\begin{spverbatim}

- Page 13, proof of Theorem 11, second paragraph, line 3:
What do you mean by ``we put''? This should be carefully explained because: if the new value of $d_u$ is smaller than the previous one then the pairs previously stored in $D_u$ should not be there anymore, but if the new value of $d_u$ is equal to the previous one and at the same time is equal to $d_z + c(\{z,u\})$, then we need to keep the pairs previously stored in $D_u$ and add the new ones.

\end{spverbatim}
\change{In every iteration, we first put the pairs in the appropriate cell of DP table and then we remove all dominated pairs. Hence, after finishing each iteration, the number of pairs in each cell becomes bounded. We have clarified this further in the draft.}
\begin{spverbatim}

- Page 13, proof of Theorem 11, third paragraph, lines 7 to 9:
The sentence ``The invariant holds inductively since we update the DP table of the neighbors of vertices which are labeled TRUE'' is not a proof that the table is updated correctly.

\end{spverbatim}
\change{We have added a detailed proof.}
\begin{spverbatim}

- Page 14, first paragraph, last sentence:
See the comments below about related work.

\end{spverbatim}
\change{We have updated our related work in light of your references.}
\begin{spverbatim}

- Page 16, first paragraph:
There is a problem with the definition of $\beta(t)$. As it is defined here, $\beta(t)$ is simply the subgraph of $G$ induced by $\bigcup_{t' \in V(\mathbb{T}_{t})} X_{t'}$, meaning that it contains all the edges in $G$ between vertices of these bags, and thus has no relation with the introduce edge nodes. I suggest you employ the same definitions as in [1] (those of the graphs $G_i$).

\end{spverbatim}
\change{Corrected.}
\begin{spverbatim}

- Page 16, paragraph after the itemize, last line:
It should say ``$\text{dist}_{\mathbb{T}}(t_1, t_2)$'' instead of ``$\text{dist}_{\mathbb{T}}(\text{t}, \text{r})$''. Nonetheless, this notion is defined for general graphs in the preliminaries section, so there is no need to define it again.

\end{spverbatim}
\change{We wish to keep it in. The notation used is ${\sf dist}$}
\begin{spverbatim}
- Page 16, last paragraph of Section 7:
Is this paragraph necessary?

\end{spverbatim}
\change{This has been removed.}
\begin{spverbatim}
- Page 17, Definition 6:
Please revise the phrase ``there exist at least $\ell$ edges whose at least one end point belong to $\mathcal{V}$''.

\end{spverbatim}
\change{We have rephrased it.}
\begin{spverbatim}

- Page 18, second paragraph, line 8:
It should say ``$|\mathcal{W}|$'' instead of ``$|\mathcal{U}|$''.

\end{spverbatim}
\change{Done.}
\begin{spverbatim}

- Page 18, last paragraph of the proof of Theorem 5:
Please revise completely.

\end{spverbatim}
\change{Done.}
\begin{spverbatim}

- Page 18, proof of Theorem 6 and proof of theorem 7:
``$u$'' is used to refer both to a variable to iterate over vertices and to one the special vertices in the instances, which may lead to confusion. Please rename them accordingly.

\end{spverbatim}
\change{Done.}
\begin{spverbatim}

- Page 20, first line:
It should say ``have'' instead of ``has''.

\end{spverbatim}
\change{Done.}
\begin{spverbatim}

- Page 20, third line:
It should say ``$\{i \in [n] : \text{the $(2i-1)$-th vertex in $\mathcal{P}$ is $v_i$}\}$''.

\end{spverbatim}
\change{Done.}
\begin{spverbatim}


Related work
There exist general frameworks that include different variants of knapsack problems, in particular some of them contain those problems addressed in the submitted article.
A knapsack problem where the subgraph induced by the set of chosen vertices has to be complete or independent is included in the set of problems studied in [2] for graphs of bounded thinness.
The problem Connected Knapsack can be modeled in the framework of [3] by using global properties (the ones of Sections 7.2 and 7.3). When $s$ (the size of the knapsack) is polynomial in the number of vertices of the graph, the problem is FPT parameterized by treewidth. The two other problems, Path Knapsack and Shortest-Path Knapsack, can also be modeled in that framework, but with a rather more elaborate model.
Notice, though, that this not invalidate the results of the submitted article because the complexity of the main algorithm (Theorem 3) is better since it depends on the minimum between $s$ and $d$. In particular, it turns to be an advantage for the approximation algorithm given Theorem 4.

\end{spverbatim}
\change{We have updated our related work in light of your references.}
\begin{spverbatim}

References

[1] Hans L. Bodlaender, Hirotaka Ono, and Yota Otachi. A Faster Parameterized Algorithm for Pseudoforest Deletion. In 11th International Symposium on Parameterized and Exact Computation (IPEC 2016). Leibniz International Proceedings in Informatics (LIPIcs), Volume 63, pp. 7:1-7:12, Schloss Dagstuhl - Leibniz-Zentrum für Informatik (2017).
https://doi.org/10.4230/LIPIcs.IPEC.2016.7

[2] Flavia Bonomo, Diego de Estrada. On the thinness and proper thinness of a graph. Discrete Applied Mathematics, Volume 261, 2019, Pages 78-92.
https://doi.org/10.1016/j.dam.2018.03.072

[3] Flavia Bonomo-Braberman, Carolina Lucía Gonzalez. A new approach on locally checkable problems. Discrete Applied Mathematics, Volume 314, 2022, Pages 53-80.
https://doi.org/10.1016/j.dam.2022.01.019.    
\end{spverbatim}

\end{document}
