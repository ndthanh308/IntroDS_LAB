\section{\pa}
We present our results for \pa in this section. First, we show that \pa is strongly \NPC by reducing it from \vc, which is known to be \NPC even for $3$-regular graphs~\cite[folklore]{DBLP:journals/dm/FleischnerSS10}. Hence, we do not expect a pseudo-polynomial time algorithm for \pa, unlike \kp.

\begin{definition}[\vc]Given a graph $\GG=(\VV,\EE)$ and a positive integer $k$, compute if there exists a subset $\VV' \subseteq \VV$ such that at least one end-point of every edge belongs to $\VV'$ and $|\VV'|\leq k$. We denote an arbitrary instance of \vc by $(\GG,k)$.
\end{definition}

\begin{theorem}\label{thm:pa-gen-npc}
\pa is strongly \NPC even when the maximum degree of the input graph is four.
\end{theorem}

\begin{proof}
    Clearly, \pa $\in$ \NP. We reduce \vc to \pa to prove NP-hardness. Let $(\GG=(\VV=\{v_i: i\in[n]\},\EE),$ $k)$ be an arbitrary instance of \vc where \GG is $3$-regular. We construct the following instance $(\GG^\pr=(\VV^\pr,\EE^\pr),(w(u))_{u\in\VV},(\alpha(u))_{u\in\VV},s,d)$ of \pa.
    \begin{align*}
        &\VV^\pr = \{u_i, g_i: i\in[n]\} \cup \{h_e: e\in \EE\}\\
        &\EE^\pr = \{\{u_i,h_e\}: i\in[n], e \in\EE, e\text{ is incident on }v_i\text{ in }\GG\} \\
        &\cup \{\{u_i,g_i\}: i\in[n]\} \cup \{\{g_i,g_{i+1}\}:i \in [n-1]\}\\
        &w(u_i) = 1, \alpha(u_i)=0, w(g_i)=0, \alpha(g_i)=0, \forall i\in[n]\\
        &w(h_e)=0, \alpha(h_e)=1, \forall e\in\EE, s=k, d=|\EE|
    \end{align*}
    We observe that the maximum degree of $\GG^\pr$ is at most four --- (i) the degree of $u_i$ is four for every $i\in[n]$, since \GG is $3$-regular and $u_i$ has an edge to $g_i$, (ii) the degree of $h_e$ is two for every $e\in\EE$, and (iii) the degree of $g_i$ is at most three for every $i\in[n]$ since the set $\{g_i:i\in[n]\}$ induces a path. We claim that the two instances are equivalent.

    In one direction, let us suppose that the \vc instance is a \yes instance. Let $\WW\subseteq\VV$ be a vertex cover of \GG with $|\WW|\le k$. We consider $\UU=\{u_i: i\in[n],v_i\in\WW\}\cup\{g_i: i\in[n]\}\cup\{h_e: e\in\EE\}\subseteq \VV^\pr$. We claim that $\GG^\pr[\UU]$ is connected. Since $\{g_i:i\in[n]\}$ induces a path and there is an edge between $u_i$ and $g_i$ for every $i\in[n]$, the induced subgraph $\GG^\pr[\{u_i: i\in[n],v_i\in\WW\}\cup\{g_i: i\in[n]\}]$ is connected. Since \WW is a vertex cover of \GG, every edge $e\in\EE$ is incident on at least one vertex in \WW. Hence, every vertex $h_e, e\in\EE,$ has an edge with at least one vertex in $\{u_i: i\in[n],v_i\in\WW\}$ in the graph $\GG^\pr$. Hence, the induced subgraph $\GG^\pr[\UU]$ is connected. Now we have $w(\UU)=\sum_{i=1}^n w(u_i)\mathbbm{1}(u_i\in\UU) + \sum_{i=1}^n w(g_i) + \sum_{e\in\EE} w(h_e)=|\UU|\le k$. We also have $\alpha(\UU)=\sum_{i=1}^n \alpha(u_i)\mathbbm{1}(u_i\in\UU) + \sum_{i=1}^n \alpha(g_i) + \sum_{e\in\EE} \alpha(h_e)=|\EE|$. Hence, the \pa instance is also a \yes instance.

    In the other direction, let us assume that the \pa instance is a \yes instance. Let $\UU\subseteq\VV^\pr$ be a solution of the \pa instance. We consider $\WW=\{v_i: i\in[n]: u_i\in\UU\}$. Since $s=k$, we have $|\WW|\le k$. Also, since $d=|\EE|$, we have $\{h_e: e\in\EE\}\subseteq\UU$. We claim that \WW is a vertex cover of \GG. Suppose not, then there exists an edge $e\in\EE$ which is not covered by \WW. Then none of the neighbors of $h_e$ belongs to \UU contradicting our assumption that \UU is a solution and thus $\GG^\pr[\UU]$ should be connected. Hence, \WW is a vertex cover of \GG and thus the \vc instance is a \yes instance.

    We observe that all the numbers in our reduced \pa instance are at most the number of edges of the graph. Hence, our reduction shows that \pa is strongly \NPC.
\end{proof}

 \Cref{thm:pa-gen-npc} also implies the following corollary in the framework of parameterized complexity.

\begin{corollary}\label{cor:pa-max-deg}
    \pa is \PNPH parameterized by the maximum degree of the input graph.
\end{corollary}

We next show that \pa is \NPC even for trees. For that, we reduce from the \NPC problem \kp.

\begin{definition}[\kp]
Given a set $\XX=[n]$ of $n$ items with sizes $\theta_1,\ldots,\theta_n,$ values $p_1,\ldots,p_n$, capacity $b$ and target value $q$, compute if there exists a subset $\II \subseteq [n]$ such that $\sum_{i\in \II} \theta_i \leq b$ and $\sum_{i\in \II} p_i \geq q$. We denote an arbitrary instance of \kp by $(\XX,(\theta_i)_{i\in\XX}, (p_i)_{i\in\XX}, b,q)$.
\end{definition}


\begin{theorem}\label{thm:pa-star-npc}
	\pa is \NPC even for star graphs.
\end{theorem}

\begin{proof}
\pa clearly belongs to \NP. To show \NP-hardness, we reduce from \kp. Let $(\XX=[n],(\theta_i)_{i\in\XX}, (p_i)_{i\in\XX}, b,q)$ be an arbitrary instance of \kp. We consider the following instance $(\GG(\VV,\EE),(w(u))_{u\in\VV},(\alpha(u))_{u\in\VV},s,d)$ of \pa.
\begin{align*}
    &\VV = \{v_0, v_1, \ldots, v_n\}\\
    &\EE = \{\{v_0,v_i\}: 1\le i\le n\}\\
    &w(v_i) = \theta_i\, \alpha(v_i) = p_i;\forall i\in[n], w(v_0)=\alpha(v_0)=0;\\
    &s = b, d = q
\end{align*}
We now claim that the two instances are equivalent.

In one direction, let us suppose that the \kp instance is a \yes instance. Let $\WW\subseteq\XX$ be a solution of \kp. Let us consider $\UU=\{v_i: i\in\WW\}\cup\{v_0\}\subseteq\VV$. We observe that $\GG[\UU]$ is connected since $v_0\in\UU$. We also have
\[ w(\UU) = \sum_{i\in\WW} w(v_i) = \sum_{i\in\WW} \theta_i \le b = s,\]
and
\[ \alpha(\UU) = \sum_{i\in\WW} \alpha(v_i) = \sum_{i\in\WW} p_i \ge q = d.\]
Hence, the \pa instance is a \yes instance.

In the other direction, let us assume that the \pa instance is a \yes instance with $\UU\subseteq\VV$ be one of its solution. Let us consider a set $\WW=\{i: i\in[n], v_i\in\UU\}$. We now have
\[ \sum_{i\in\WW} \theta_i = \sum_{i\in\WW} w(v_i) = w(\UU) \le s = b, \]
and
\[ \sum_{i\in\WW} p_i = \sum_{i\in\WW} \alpha(v_i) = \alpha(\UU) \ge d=q.\]
Hence, the \kp instance is a \yes instance.
\end{proof}


We complement the hardness result in \Cref{thm:pa-star-npc} by designing a pseudo-polynomial-time algorithm for \pa for trees. In fact, we have designed an algorithm with running time $2^{\OO(\tw\log \tw)}\cdot n^{\mathcal{O}(1)}\cdot {\sf min}\{s^2,d^2\}$ where the treewidth of the input graph is $\tw$. We present this algorithm next.

\subsection{Treewidth as a Parameter}
% \subsection{\pa on Bounded Treewidth Graphs}
%\subsection{Treewidth as Parameter}
% In this part, we study \pa parameterized by the treewidth of the input graph $G$. Note that since the problem is \NPC even for stars, we do not expect FPT algorithms parameterized by the treewidth of $G$. However, we design an algorithm with running time $f(k)\cdot p(n,s,d)$; here $f$ is a computable function on $k = tw(G)$, and $p$ is a polynomial dependent on $n = |V(G)|$, the target size $s$ and the target value $d$.

\begin{theorem}\label{thm:treewidth-pa}
There is an algorithm for \pa with running time $2^{\OO(\tw\log \tw)}\cdot n\cdot {\sf min}\{s^2,d^2\}$ where $n$ is the number of vertices in the input graph, $\tw$ is the treewidth of the input graph, $s$ is the input size of the knapsack and $d$ is the input target value.
\end{theorem}



\begin{proof}
Let $(G = (V_G,E_G),{(w(u))_{u \in V_G}, (\alpha(u))_{u\in V_G}}, s,d)$ be an input instance of \pa such that $\tw=tw(G)$. Let $\mathcal{U} \subseteq V_G$ be a solution subset for the input instance. For technical purposes, we guess a vertex $v \in \mathcal{U}$ --- once the guess is fixed, we are only interested in finding solution subsets $\mathcal{U}'$ that contain $v$ and $\mathcal{U}$ is one such candidate. We also consider a nice edge tree decomposition $(\mathbb{T} = (V_{\mathbb{T}},E_{\mathbb{T}}),\mathcal{X})$ of $G$ that is rooted at a node $r$, and where $v$ has been added to all bags of the decomposition. Therefore, $X_{r} = \{v\}$ and each leaf bag is the singleton set $\{v\}$.

We define a function $\ell: V_{\mathbb{T}} \rightarrow \mathbb{N}$.
%For the root $r$, $\ell(r) = 0$. 
For a vertex $t \in V_\mathbb{T}$, $\ell(t) = \sf{dist}_\mathbb{T}(t,r)$, where $r$ is the root. Note that this implies that $\ell(r) = 0$. Let us assume that the values that $\ell$ takes over the nodes of $\mathbb{T}$ is between $0$ and $L$. Now, we describe a dynamic programming algorithm over $(\mathbb{T},\mathcal{X})$ for \pa.


%%%%%%%%
    \paragraph{\textbf{States.}} We maintain a DP table $D$ where a state has the following components:
    \begin{enumerate}
    \item $t$ represents a node in $V_\mathbb{T}$.
    \item $\mathbb{P} = (P_0,\ldots,P_m), m\leq \tw+1$ represents a partition of the vertex subset $X_t$. 
    \end{enumerate}
%%%%%%
    \paragraph{\textbf{Interpretation of States.}} For a state $[t,\mathbb{P}]$, if there is a solution subset $\mathcal{U}$ let $\mathcal{U'} = \mathcal{U} \cap V(\beta(t))$. Let $\beta_{\mathcal{U'}}$ be the graph induced on $\mathcal{U'}$ in $\beta(t)$. Let $C_1,C_2,\ldots,C_m$ be the connected components of $\beta_{\mathcal{U'}}$. Note that $m\leq \tw+1$. Then in the partition $\mathbb{P} = (P_0,P_1,\ldots,P_m)$, $P_i = C_i \cap X_t, 1 \leq i \leq m$. Also, $P_0 = X_t \setminus \mathcal{U'}$. 

Given a node $t\in V_{\mathbb{T}}$, a subgraph $H$ of $\beta(t)$ is said to be a $\mathbb{P}$-subgraph if (i) the connected components $C_1,C_2,\ldots,C_m$ of $H$, $m\leq \tw+1$ are such that $P_i = C_i \cap X_t, 1 \leq i \leq m$, (ii) $P_0 = X_t \setminus H$. For each state $[t,\mathbb{P}]$, a pair $(w,\alpha)$ with $w\leq s$ is said to be feasible if there is a $\mathbb{P}$-subgraph of $\beta(t)$ whose total weight is $w$ and total value is $\alpha$. Moreover, a feasible pair $(w,\alpha)$ is said to be undominated if there is no other $\mathbb{P}$-subgraph with weight $w'$ and value $\alpha'$ such that $w' \leq w$ and $\alpha' \geq \alpha$. Please note that by default, an empty $\mathbb{P}$-subgraph has total weight $0$ and total value $0$.

For each state $[t,\mathbb{P}]$, we initialize $D[t,\mathbb{P}]$ to the list $\{(0,0)\}$. Our computation shall be such that in the end each $D[t,\mathbb{P}]$ stores the set of all undominated feasible pairs $(w,\alpha)$ for the state $[t,\mathbb{P}]$.

%%%%%%
    \paragraph{\textbf{Dynamic Programming on $D$.}} We describe the following procedure to update the table $D$. We start updating the table for states with nodes $t\in V_\mathbb{T}$ such that $\ell(t)=L$. When all such states are updated, then we move to update states where the node $t$ has $\ell(t) = L-1$, and so on till we finally update states with $r$ as the node --- note that $\ell(r) =0$. For a particular $i$, $0\leq i< L$ and a state $[t,\mathbb{P}]$ such that $\ell(t) = i$, we can assume that $D[t',\mathbb{P}']$ have been evaluated for all $t'$ such that $\ell(t')>i$ and all partitions $\mathbb{P}'$ of $X_{t'}$. Now we consider several cases by which $D[t,\mathbb{P}]$ is updated based on the nature of $t$ in $\mathbb{T}$:
    \begin{enumerate}
    \item Suppose $t$ is a leaf node. Note that by our modification, $X_t = \{v\}$. There can be only 2 partitions for this singleton set --- $\mathbb{P}_t^1 = (P_0 = \emptyset, P_1 = \{v\})$ and $\mathbb{P}_t^2 = (P_0 = \{v\}, P_1 = \emptyset)$. If $\mathbb{P} = \mathbb{P}_t^1$ then $D[t,\mathbb{P}]$ stores the pair $(w(v),\alpha(v))$ if $w(v) \leq s$ and otherwise no modification is made. If $\mathbb{P} = \mathbb{P}_t^2$ then $D[t,\mathbb{P}]$ is not modified.
    
    \item Suppose $t$ is a forget vertex node. Then it has an only child $t'$ where $X_t \subset X_{t'}$ and there is exactly one vertex $u \neq v$ that belongs to $X_{t'}$ but not to $X_t$. Let $\mathbb{P}' = (P'_0,P'_1,\ldots,P'_{m'})$ be a partition of $X_{t'}$ such that when restricted to $X_t$ we obtain the partition $\mathbb{P} = (P_0,P_1,\ldots,P_m )$. For each such partition, we shall do the following.\\ Suppose $\mathbb{P}'$ has $u \in P'_0$, then each feasible undominated pair stored in $D[t',\mathbb{P}']$ is copied to $D[t,\mathbb{P}]$. \\
    Alternatively, suppose $\mathbb{P}'$ has $u \in P'_i, i>0$ and $|P'_i| >1$. Then, each feasible undominated pair stored in $D[t',\mathbb{P}']$ is copied to $D[t,\mathbb{P}]$.\\
    Finally, suppose $\mathbb{P}'$ has $u \in P'_i, i>0$ and $P'_i = \{u\}$. Then we do not make any changes to $D[t,\mathbb{P}]$.
    
    \item Suppose $t$ is an introduce node. Then it has an only child $t'$ where $X_{t'} \subset X_{t}$ and there is exactly one vertex $u \neq v$ that belongs to $X_{t}$ but not $X_{t'}$. Note that no edges incident to $u$ have been introduced yet, and so in $\beta(t)$ $u$ is not yet connected to any other vertex. Let $\mathbb{P}' = (P'_0,P'_1,\ldots,P'_{m'})$ be the partition of $X_{t'}$ obtained from restricting the partition $\mathbb{P} = (P_0,P_1,\ldots,P_m )$ to $X_{t'}$. First, suppose $u \in P_0$. Then we copy all pairs of $D[t',\mathbb{P}']$ to $D[t,\mathbb{P}]$. \\
    Next, suppose $u \in P_i, i>0$ and $P_i = \{u\}$. Then for each pair $(w,\alpha)$ in $D[t',\mathbb{P}']$, if $w + w(u) \leq s$ we add $(w+w(u),\alpha+\alpha(u))$ to the set in $D[t,\mathbb{P}]$.\\
    Finally, let $u \in P_i, i>0$ and $|P_i| >1 $. Then we make no changes to $D[t,\mathbb{P}]$.
    
    \item Suppose $t$ is an introduce edge node. Then it has an only child $t'$ where $X_{t'} = X_{t}$, and additionally for two vertices $u,w \in X_t = X_{t'}$, the edge $\{u,w\}$ is introduced into $\beta(t)$. First, suppose $\mathbb{P} = (P_0,P_1,\ldots,P_m)$ is such that one of $u,w$ is in $P_0$. Then all pairs of $D[t',\mathbb{P}]$ are copied to $D[t,\mathbb{P}]$.\\
    Next, let $u \in P_i$, $w\in P_j$, $i\neq j \neq 0$. Then no updates are made to $D[t,\mathbb{P}]$. \\
    Finally, suppose $u,w \in P_i, i>0$. Copy to $D[t,\mathbb{P}]$ all pairs from $D[t',\mathbb{P}]$. Consider a partition $\mathbb{P}'$ where the part $P'$ contains $u$ and $P''$ contains $w$. $\mathbb{P}'$ is such that $P_i = P' \cup P''$ and any other $P_j$ with $j \neq i$ is a part in $\mathbb{P}'$. Any pair of $D[t',\mathbb{P}']$ is copied to $D[t,\mathbb{P}]$.
    \item Suppose $t$ is a join node. Then it has two children $t_1,t_2$ such that $X_t = X_{t_1} = X_{t_2}$. Consider $\mathbb{P} = (P_0,P_1,\ldots,P_m)$. Let $(w_{\mathbb{P}}, \alpha_{\mathbb{P}})$ be the total weight and value of the vertices in $\cup_{1\leq i \leq m} P_i$. Consider a pair $(w_1,\alpha_1)$ in $D[t_1,\mathbb{P}]$ and a pair $(w_2,\alpha_2)$ in $D[t_2,\mathbb{P}]$. Suppose $w_1 + w_2 - w_{\mathbb{P}} \leq s$. Then we add $(w_1 + w_2 - w_{\mathbb{P}}, \alpha_1+\alpha_2-\alpha_{\mathbb{P}})$ to $D[t,\mathbb{P}]$.
    \end{enumerate}
    
Finally, in the last step of updating $D[t,\mathbb{P}]$, we go through the list saved in $D[t,\mathbb{P}]$ and only keep undominated pairs.

The output of the algorithm is a pair $(w,\alpha)$ stored in $D[r,\mathbb{P} = (P_0 = \emptyset, P_1 = \{v\})]$ such that $w \leq s$ and $\alpha$ is the maximum value over all pairs in $D[r,\mathbb{P}]$.


%%%%%%%
    \paragraph{\textbf{Correctness of the Algorithm.}}

    Recall that we are looking for a solution $\mathcal{U}$ that contains the fixed vertex $v$ that belongs to all bags of the tree decomposition. First, we show that a pair $(w,\alpha)$ belonging to $D[t,\mathbb{P}]$ for a node $t \in V_\mathbb{T}$ and a partition $\mathbb{P}$ of $X_t$ corresponds to a $\mathbb{P}$-subgraph $H$ in $\beta(t)$. Recall that $X_r = \{v\}$. Thus, this implies that a pair $(w,\alpha)$ belonging to $D[r,\mathbb{P} = (P_0 = \emptyset, P_1 = \{v\})]$ corresponds to a connected subgraph of $G$. Moreover, the output is a pair that is feasible and with the highest value. 

    In order to show that a pair $(w,\alpha)$ belonging to $D[t,\mathbb{P}]$ for a node $t \in V_\mathbb{T}$ and a partition $\mathbb{P}$ of $X_t$ corresponds to a $\mathbb{P}$-subgraph $H$ in $\beta(t)$, we need to consider the cases of what $t$ can be:
    \begin{enumerate}
    \item When $t$ is a leaf node with $\ell(t) = i$, $X_t$ only contains $v$ and the update to $D$ is done such that $v$ is the corresponding subgraph to a stored pair. This is true in particular when $i =L$, the base case. From now we can assume that for a node $t$ with $\ell(t) = i < L$ all $D[t',\mathbb{P}']$ entries are correct and correspond to $\mathbb{P}'$-subgraphs in $\beta(t')$ when $\ell(t') > i$.
    \item When $t$ is a forget vertex node, let $t'$ be the child node and $u \neq v$ be the vertex that is being forgotten. We copy pairs from $D[t',\mathbb{P}']$ depending on the structure of $\mathbb{P}'$. Since $\ell(t') > \ell(t)$, by induction hypothesis all entries in $D[t',\mathbb{P}']$ for any partition $\mathbb{P}'$ of $X_{t'}$ are feasible. From the cases considered, we copy a pair to $D[t,\mathbb{P}]$ from a $D[t',\mathbb{P}']$ only when $u$ is not part of the $\mathbb{P}'$-subgraph or is in a component of $\mathbb{P}'$ that has vertices in $X_t$. Thus, the same subgraph is a $\mathbb{P}$-subgraph in $\beta(t)$.
    \item When $t$ is an introduce node, there is a child $t'$ we are introducing a vertex $u \neq v$ that has no adjacent edges added in $\beta(t)$. Since $\ell(t') > \ell(t)$, by induction hypothesis all entries in $D[t',\mathbb{P}']$ for any partition $\mathbb{P}'$ of $X_{t'}$ are feasible. We update pairs in $D[t,\mathbb{P}]$ from $D[t',\mathbb{P}']$ such that either $u$ is not considered as part of a $\mathbb{P}$-subgraph and the pair is certified by the $\mathbb{P}'$-subgraph, or $u$ is added to a $\mathbb{P}'$-subgraph in order to obtain a new $\mathbb{P}$-subgraph.
    \item When $t$ is an introduce edge node, there is a child $t'$ such that $X_t = X_{t'}$ and the only difference is that two vertices $u,w$ in the bags $X_t = X_{t'}$ now have an edge in $\beta(t)$. Since $\ell(t') > \ell(t)$, by induction hypothesis all entries in $D[t',\mathbb{P}']$ for any partition $\mathbb{P}'$ of $X_{t'}$ are feasible. The updates are made in the cases when one of $u$ or $w$ is not in the intended $\mathbb{P}$-subgraph and the included pair is certified by a $\mathbb{P'}$-subgraph, or when the $u$ and $w$ belong to different components of a $\mathbb{P}'$-subgraph and the new $\mathbb{P}$-subgraph has these components merged as a single component.
    \item When $t$ is a join node, there are two children $t_1,t_2$ such that $X_T = X_{t_1} = X_{t_2}$. Note that this implies that $V(\beta(t_1)) \cap V(\beta(t_2)) = X_t$. Since $\ell(t_1),\ell(t_2) > \ell(t)$, by induction hypothesis all entries in $D[t_i,\mathbb{P}']$ for any partition $\mathbb{P}'$ of $X_{t_i}$ are feasible for $i \in \{1,2\}$.We update pairs in $D[t,\mathbb{P}]$ when there is a $\mathbb{P}$-subgraph in $\beta(t_1)$ and a $\mathbb{P}$-subgraph in $\beta(t_2)$ and we take the union of these two subgraphs to obtain a $\mathbb{P}$-subgraph in $\beta(t)$.
    \end{enumerate}
    Thus in all cases of $t$, a pair added to $D[t,\mathbb{P}]$ for some partition of $X_t$ is a feasible pair. Recall that as a last step of updating $D[t,\mathbb{P}]$, we go through the entire list and keep only  undominated pairs in the list.

    What remains to be shown is that an undominated feasible solution $\mathcal{U}$ of \pa in $G$ is contained in $D[r,\mathbb{P} = (P_0 = \emptyset, P_1 = \{v\})]$. Let $w$ be the weight of $\mathcal{U}$ and $\alpha$ be the value. Recall that $v \in \mathcal{U}$. For each $t$, we consider the subgraph $\beta(t) \cap \mathcal{U}$. Let $C_1,C_2,\ldots,C_m$ be components of $\beta(t) \cap \mathcal{U}$ and let for each $1\leq i \leq m, P_i = X_t \cap C_i$. Also, let $P_0 = X_t \setminus \mathcal{U}$. Consider $\mathbb{P} = (P_0,P_1,\ldots,P_m)$. The algorithm updates in $D[t,\mathbb{P}]$ the pair $(w',\alpha')$ for the subsolution $\beta(t) \cap \mathcal{U}$. Therefore, $D[r,\mathbb{P} = (\emptyset,\{v\})]$ contains the pair $(w,\alpha)$. Thus, we are done.
%%%%%%
    \paragraph{\textbf{Running time.}} There are $n$ choices for the fixed vertex $v$. Upon fixing $v$ and adding it to each bag of $(\mathbb{T}, \mathcal{X})$ we consider the total possible number of states.  There are at most $\OO(n)\cdot 2^{\tw\log \tw}$ states. For each state, since we are keeping only undominated pairs, for each $w$ there can be at most one pair with $w$ as the first coordinate; similarly, for each $\alpha$ there can be at most one pair with $\alpha$ as the second coordinate. Thus, the number of undominated pairs in each $D[t,\mathbb{P}]$ is at most ${\sf min}\{s,d\}$. By the description of the algorithm, the maximum length of the list stored at $D[t,\mathbb{P}]$ during updation, but before the check is made for only undominated pairs, is $2^{\tw \log \tw}\cdot{\sf min}\{s^2,d^2\}$. Thus, updating the DP table at any vertex takes $2^{\tw \log \tw}\cdot{\sf min}\{s^2,d^2\}$ time. Since there are $\OO(n\cdot\tw)$ vertices in $\mathbb{T}$, the total running time of the algorithm is $2^{\OO(\tw\log \tw)}\cdot n\cdot {\sf min}\{s^2,d^2\}$.
\end{proof}


\subsection{A Fixed Parameter Fully Pseudo-polynomial Time Approximation Scheme}

We now use the algorithm in \Cref{thm:treewidth-pa} as a black-box to design an $(1-\eps)$-factor approximation algorithm for optimizing the value of the solution and running in time $2^{\tw\log \tw}\cdot \text{poly}(n,1/\eps)$.

\begin{theorem}\label{thm-fptas}
There is an $(1-\eps)$-factor approximation algorithm for \pa for optimizing the value of the solution and running in time $2^{\OO(\tw\log \tw)}\cdot \text{poly}(n,1/\eps)$ where \tw is the treewidth of the input graph.
\end{theorem}

\begin{proof}
Let $\II=(\GG=(\VV,\EE),(w(u))_{u\in\VV},(\alpha(u))_{u\in\VV},s)$ be an arbitrary instance of \pa where the goal is to output a connected subgraph \UU of maximum $\alpha(\UU)$ subject to the constraint that $w(\UU)\le s$. Without loss of generality, we can assume that $w(u)\le s$ for every $u\in\VV$. If not, then we can remove every $u\in\VV$ whose $w(u)>s$; this does not affect any solution since any vertex deleted can never be part of any solution. Let $\alpha_{\text{max}}=\max\{\alpha(u): u\in\VV\}$. We construct another instance $\II^\pr=\left(\GG=(\VV,\EE),(w(u))_{u\in\VV},(\alpha^\pr(u)=\left\lfloor \frac{n\alpha(u)}{\eps\alpha_{\text{max}}}\right\rfloor)_{u\in\VV},s\right)$ of \pa. We compute the optimal solution $\WW^\pr\subseteq\VV$ of $\II^\pr$ using the algorithm in \Cref{thm:treewidth-pa} and output $\WW^\pr$. Let $\WW\subseteq\VV$ be an optimal solution of $\II$. Clearly $\WW^\pr$ is a valid (may not be optimal) solution of \II also, since $w(\WW^\pr)\le s$ by the correctness of the algorithm in \Cref{thm:treewidth-pa}. We now prove the approximation factor of our algorithm.
\begin{align*}
     \sum_{u\in\WW^\pr}\alpha(u) &\ge \frac{\eps\alpha_{\text{max}}}{n}\sum_{u\in\WW^\pr} \left\lfloor \frac{n\alpha(u)}{\eps\alpha_{\text{max}}}\right\rfloor\\
    &\ge \frac{\eps\alpha_{\text{max}}}{n}\sum_{u\in\WW} \left\lfloor \frac{n\alpha(u)}{\eps\alpha_{\text{max}}}\right\rfloor&\text{[since $\WW^\pr$ is an optimal solution of $\II^\pr$]}\\
    &\ge \frac{\eps\alpha_{\text{max}}}{n}\sum_{u\in\WW} \left( \frac{n\alpha(u)}{\eps\alpha_{\text{max}}}-1\right)\\
    &\ge \left(\sum_{u\in\WW}\alpha(u)\right)-\eps\alpha_{\text{max}}\\
    &\ge \OPT(\II)-\eps\OPT(\II) & \text{[$\alpha_{\text{max}}\le\OPT(\II)$]}\\
    &=(1-\eps)\OPT(\II)
\end{align*}
Hence, the approximation factor of our algorithm is $(1-\eps)$. We now analyze the running time of our algorithm.

The value of any optimal solution of $\II^\pr$ is at most
\[\sum_{u\in\VV}\alpha^\pr(u) \le \frac{n}{\eps\alpha_{\text{max}}}\sum_{u\in\VV}\alpha(u)\le \frac{n}{\eps\alpha_{\text{max}}}\sum_{u\in\VV}\alpha_{\text{max}}=\frac{n^2}{\eps}. \]
Hence, the running time of our algorithm is $2^{\tw\log \tw}\cdot \text{poly}(n,1/\eps)$.
\end{proof}

\subsubsection{Other Parameters}

We next consider $vcs$, the maximum size of a minimum vertex cover of the subgraph induced by any solution of \pa, as our parameter. That is, $vcs(\II=(\GG,(w(u))_{u\in\VV},(\alpha(u))_{u\in\VV},s,d) = \max\{\text{size of minimum vertex cover of }W: W\subseteq\VV\text{ is a solution of }\II\}.$ We already know from \Cref{thm:pa-star-npc} that \pa is \NPC for star graphs. We note that $vcs$ is one for star graphs. Hence, \pa is \PNPH with respect to $vcs$, that is, there is no algorithm for \pa which runs in polynomial time even for constant values of $vsc$. However, whether there exists any algorithm with running time $\OO(f(vcs).\text{poly}(n,s,d))$, remains a valid question. We answer this question negatively in \Cref{thm:vcs-woh}. For that, we exhibit an \FPT-reduction from \pvc which is known to be \WOH parameterized by the size of the partial vertex cover we are looking for.


\begin{definition}[\pvc]
Given a graph $\GG=(\VV,\EE)$ and two integers $k$ and $\el$, compute if there exists a subset $\VV^\pr\subseteq\VV$ such that (i) $|\VV^\pr|\le k$ and (ii) there exist at least $\el$ edges whose one or both end points belong to $\VV^\pr$. We denote an arbitrary instance of \pvc by $(\GG,k,\el)$.
\end{definition}

% We know that \pvc is \WOH parameterized by $k$. 

\begin{theorem}\label{thm:vcs-woh}
There is no algorithm for \pa running in time $\OO(f(vcs).\text{poly}(n,s,d))$ unless \ETH fails.
\end{theorem}




\begin{proof}
At a high-level, our reduction from \pvc is similar to the reduction in \Cref{thm:pa-gen-npc}. The only difference is that we will now have only one ``global vertex'' instead of the ``global path'' in the proof of \Cref{thm:pa-gen-npc}. Formally, our reduction is as follows.

Let $(\GG=(\VV,\EE),k,\el)$ be an arbitrary instance of \pvc. We consider the following instance $(\GG^\pr=(\VV^\pr,\EE^\pr),(w(u))_{u\in\VV^\pr},(\alpha(u))_{u\in\VV^\pr},s,d)$ of \pa.
\begin{align*}
    &\VV^\pr = \{u_i: i\in[n]\} \cup \{h_e: e\in \EE\}\cup\{g\}\\
    &\EE^\pr = \{\{u_i,h_e\}: i\in[n], e \in\EE, e\text{ is incident on }v_i\text{ in }\GG\}
    \cup \{\{u_i,g\}: i\in[n]\}\\
    &w(u_i) = 1, \alpha(u_i)=0, \forall i\in[n], 
    w(h_e)=0, \alpha(h_e)=1, \forall e\in\EE,\\
    &w(g)=0, \alpha(g)=0, s=k, d=\el
\end{align*}
We claim that the two instances are equivalent.

In one direction, let us suppose that the \pvc instance is a \yes instance. Let $\WW\subseteq\VV$ covers at least \el edges in \GG and $|\WW|\le k$. We consider $\UU=\{u_i: i\in[n],v_i\in\WW\}\cup\{h_e: e\in\EE, e\text{ is covered by }\WW\}\cup\{g\}\subseteq \VV^\pr$. We claim that $\GG^\pr[\UU]$ is connected. Since there is an edge between $u_i$ and $g$ for every $i\in[n]$, the induced subgraph $\GG^\pr[\{u_i: i\in[n],v_i\in\WW\}\cup\{g\}]$ is connected. Also, since $h_e$ belongs to \UU only if \WW covers $e$ in \GG, the induced subgraph $\GG^\pr[\UU]$ is connected. Now we have $w(\UU)=\sum_{i=1}^n w(u_i)\mathbbm{1}(u_i\in\UU) + w(g) + \sum_{e\in\EE, \WW\text{ covers }e} w(h_e)=|\WW|\le k$. We also have $\alpha(\UU)=\sum_{i=1}^n \alpha(u_i)\mathbbm{1}(u_i\in\UU) + \alpha(g) + \sum_{e\in\EE, \WW\text{ covers }e} \alpha(h_e)\ge\el$. Hence, the \pa instance is also a \yes instance.

In the other direction, let us assume that the \pa instance is a \yes instance. Let $\UU\subseteq\VV^\pr$ be a solution of the \pa instance. We consider $\WW=\{v_i: i\in[n]: u_i\in\UU\}$. Since $s=k$, we have $|\WW|\le k$. Also, since $d=\el$, we have $|\{h_e: e\in\EE\}\cap\UU|\ge\el$. We claim that \WW covers every edge in $\{e: e\in\EE, h_e\in\UU\}$. Suppose not, then there exists an edge $e\in\{e: e\in\EE, h_e\in\UU\}$ which is not covered by \WW. Then none of the neighbors of $h_e$ belongs to \UU contradicting our assumption that \UU is a solution and thus $\GG^\pr[\UU]$ should be connected. Hence, \WW covers at least \el edges in \GG and thus the \pvc instance is a \yes instance.

We also observe that the size of a minimum vertex cover of the subgraph induced by any solution of the reduced \pa instance is at most $k+1$ --- the set consisting of the vertices in any solution of the \pa instance in $\GG^\pr$ which has weight $1$ and $g$ forms a vertex cover of the subgraph induced by the solution of the \pa instance. Also, all the numbers in our reduced \pa instance are at most the number of edges of the graph, and \pvc is \WOH parameterized by $k$. Hence, \pa is \WOH parameterized by the maximum size of the minimum vertex cover of any solution even when all the numbers are encoded in unary. Therefore, there is no algorithm for \pa running in time $\OO(f(vcs).\text{poly}(n,s,d))$ unless \ETH fails.
\end{proof}