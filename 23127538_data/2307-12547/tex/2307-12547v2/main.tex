\newcommand{\longversion}[1]{}
\newcommand{\shortversion}[1]{#1}

% This is samplepaper.tex, a sample chapter demonstrating the
% LLNCS macro package for Springer Computer Science proceedings;
% Version 2.21 of 2022/01/12
%
\documentclass[runningheads]{llncs}
%
\usepackage[T1]{fontenc}
% T1 fonts will be used to generate the final print and online PDFs,
% so please use T1 fonts in your manuscript whenever possible.
% Other font encondings may result in incorrect characters.
%
\usepackage{graphicx}
% Used for displaying a sample figure. If possible, figure files should
% be included in EPS format.
%
% If you use the hyperref package, please uncomment the following two lines
% to display URLs in blue roman font according to Springer's eBook style:
%\usepackage{color}
%\renewcommand\UrlFont{\color{blue}\rmfamily}
%

%
%%%%%%%%%%%%%%%%%%%%%%%%%%%%%%%%%%%%%%%%%%%%%%%%%%%%%%%%%%%
%\usepackage[subtle]{savetrees}
%\usepackage[margin=2cm]{geometry}
\usepackage{tikz,amsmath, amssymb,bm,color, amsthm,amsfonts}
\usetikzlibrary{positioning, calc,chains,fit,shapes}
%\usetikzlibrary{circuits.logic.US,circuits.logic.IEC,fit}
\usepackage{enumerate}
\usepackage{comment}
\usepackage{tikz}
\usepackage{graphics}
%\usepackage[cm]{fullpage}
\usepackage{longtable}
\usepackage{mdframed}
\usepackage{caption}
\usepackage{subcaption}
\usepackage{slashbox}
\usepackage{url}
\usepackage{framed}
\usepackage{array}
\usepackage{tabu}
\usepackage{lscape}
\usepackage{multirow}
\usepackage{ulem}
\usepackage{multicol}
\usepackage{placeins}
\usepackage{cite}
\usepackage{enumitem}
\usepackage{mathtools}
%\usepackage[numbers]{natbib}
%\usepackage{mathtools}
%\usepackage{authblk}

\mdfsetup{skipabove=2pt,skipbelow=2pt}
%\setlenght {\marginparwidth }{2cm}
%\usepackage{todonotes}

%\usepackage{floatrow}
%\usepackage{adjustbox}
%\setlength{\extrarowheight}{.05ex}
%\renewcommand\thesubfigure{\roman{subfigure}}


%\newtheorem{theorem}{Theorem}[section]
%\newtheorem{lemma}[theorem]{Lemma}
%\newtheorem{observation}[theorem]{Observation}
%\newtheorem{corollary}[theorem]{Corollary}
%\newtheorem{proposition}[theorem]{Proposition}
%\newtheorem{definition}[theorem]{Definition}
\newtheorem{construction}{Construction}
%\newtheorem{conjecture}{Conjecture}
%\newtheorem{remark}[theorem]{Remark}

\newcommand{\pname}[1]{\textnormal{\textsc{#1}}}
\newcommand{\cclass}[1]{\textnormal{\textsf{#1}}}
\newcommand{\nog}{nine} % no of members in the gang!
\newcommand{\nogd}{nineteen} % no of members in the gang - for deletion/completion
\newcommand{\nogl}{eighteen} % no of members in the larger gang - for editing
\newcommand{\nogld}{thirty eight} % no of members in the larger gang - for deletion/completion
\newcommand{\diffnog}{ten} %
%\newcommand{\dominatedby}{dominated by} %
%\newcommand{\dominatingset}{dominating set} %
%\newcommand{\dominates}{dominates} %
\newcommand{\simulates}{simulates} %
\newcommand{\baseset}{base} %
\newcommand{\issimulatedby}{is simulated by} %

\newcommand{\StarSAT}{\pname{8-SAT$_{\geq 6}$}}
\newcommand{\FSAT}{\pname{4-SAT$_{\geq 2}$}}
\newcommand{\FISAT}{\pname{5-SAT$_{\geq 3}$}}
\newcommand{\SIXSAT}{\pname{6-SAT$_{\geq 4}$}}
\newcommand{\ESAT}{\pname{8-SAT$_{\geq 6}$}}
\newcommand{\KSAT}{\pname{$k$-SAT$_{\geq {k-2}}$}}
\newcommand{\KSATO}{\pname{$k$-SAT}}
\newcommand{\ESATO}{\pname{8-SAT}}
\newcommand{\FSATO}{\pname{4-SAT}}
\newcommand{\FISATO}{\pname{5-SAT}}
\newcommand{\TSAT}{\pname{3-SAT}}
\newcommand{\HED}{\pname{${H}$-free Edge Deletion}}
\newcommand{\AEE}{\pname{${A}$-free Edge Editing}}
\newcommand{\AED}{\pname{${A}$-free Edge Deletion}}
\newcommand{\TSED}{\pname{$t$-star-free Edge Deletion}}
\newcommand{\ATSED}{\pname{Annotated $t$-star-free Edge Deletion}}
\newcommand{\AFSED}{\pname{Annotated $4$-star-free Edge Deletion}}
\newcommand{\FSED}{\pname{$4$-star-free Edge Deletion}}
\newcommand{\FVSED}{\pname{$5$-star-free Edge Deletion}}
\newcommand{\HEE}{\pname{${H}$-free Edge Editing}}
\newcommand{\HEC}{\pname{${H}$-free Edge Completion}}
\newcommand{\HDEE}{\pname{${H'}$-free Edge Editing}}
\newcommand{\HDDEE}{\pname{${H''}$-free Edge Editing}}
\newcommand{\HDED}{\pname{${H'}$-free Edge Deletion}}
\newcommand{\HDEC}{\pname{${H'}$-free Edge Completion}}
\newcommand{\HBEE}{\pname{${\overline{H}}$-free Edge Editing}}
\newcommand{\HBED}{\pname{${\overline{H}}$-free Edge Deletion}}
\newcommand{\HBEC}{\pname{${\overline{H}}$-free Edge Completion}}
\newcommand{\HOEDCE}{\pname{${H_1}$-free Edge Deletion(Completion/Editing)}}
\newcommand{\HEDCE}{\pname{${H}$-free Edge Deletion(Completion/Editing)}}
\newcommand{\HEEDC}{\pname{${H}$-free Edge Editing(Deletion/Completion)}}
\newcommand{\HDEEDC}{\pname{${H'}$-free Edge Editing(Deletion/Completion)}}
\newcommand{\BFED}{\pname{Bow-free Edge Deletion}}
\newcommand{\ABFED}{\pname{Annotated Bow-free Edge Deletion}}
\newcommand{\DTIS}{\pname{Distance-3 Independent Set}}
\newcommand{\SVC}{\pname{Strong Vertex Cover}}
\newcommand{\CLIQUE}{\pname{Clique}}
\newcommand{\IS}{\pname{Independent Set}}
\newcommand{\PFS}{\pname{Propagational-$f$ Satisfiability}}
\newcommand{\RHED}{\pname{Restricted ${H}$-free Edge Deletion}}
\newcommand{\RHEC}{\pname{Restricted ${H}$-free Edge Completion}}
\newcommand{\RHDED}{\pname{Restricted ${H'}$-free Edge Deletion}}
\newcommand{\RHDEC}{\pname{Restricted ${H'}$-free Edge Completion}}
\newcommand{\RHEE}{\pname{Restricted ${H}$-free Edge Editing}}
\newcommand{\PH}{$\cclass{NP} \subseteq \cclass{coNP/poly}$}
\newcommand{\NOPH}{$\cclass{NP} \not\subseteq \cclass{coNP/poly}$}
\newcommand{\LG}{\mathcal{W}}
\newcommand{\LGD}{\mathcal{W}'}
\newcommand{\LGDD}{\mathcal{W}''}


%\let\oldvee\vee
\renewcommand\vee{\boxtimes}

\newcommand\addvmargin[1]{
  \node[fit=(current bounding box),inner ysep=#1,inner xsep=0]{};
}
\setlength{\fboxrule}{0pt}

\newcommand{\defstage}[2]{% PGD Version
  \hfill\\\smallskip\noindent%
  \begin{tabularx}{\textwidth}{|l X|}%
    \hline%
    \multicolumn{2}{|l|}{\textbf{#1}}\\%
    &#2\\\hline%
  \end{tabularx}%
%  \smallskip%
}
\setlength\extrarowheight{15pt}

\newcounter{rowcntr}[table]
\renewcommand{\therowcntr}{\thetable.\arabic{rowcntr}}

% A new columntype to apply automatic stepping
\newcolumntype{N}{>{\refstepcounter{rowcntr}\therowcntr}c}

% Reset the rowcntr counter at each new tabular
\AtBeginEnvironment{longtabu}{\setcounter{rowcntr}{0}}

\newcounter{rowcntra}[table]
\renewcommand{\therowcntra}{\arabic{rowcntra}}

% A new columntype to apply automatic stepping
\newcolumntype{M}{>{\refstepcounter{rowcntra}\therowcntra}c}

% Reset the rowcntr counter at each new tabular
\AtBeginEnvironment{tabular}{\setcounter{rowcntra}{0}}

\newcommand{\NPC}{NP-Complete}


\newcommand{\highlight}[1]{\textcolor{blue}{#1}}
\newcommand{\dhanya}[1]{\textcolor{blue}{dhanya: #1}}


%\newcommand{\XCD1}[1]{\pname{$\chi_{cd}$\ensuremath{(#1)}}}
\newcommand{\XCD}{\pname{$\chi_{cd}$}}
\newcommand{\SC}{\pname{$\omega_{s}$}}

\newcommand{\CDC}{\textsc{CD-coloring}}
\newcommand{\SCP}{\textsc{Separated-Cluster}}
\newcommand{\TD}{\textsc{Total Domination}}
\newcommand{\ISP}{\textsc{Independent Set}}
\newcommand{\CC}{\textsc{Clique Cover}}
\newcommand{\TETHS}{Further, the problem cannot be solved in time \ensuremath{2^{o(|V(G)|)}}, unless the ETH fails}
%\usetikzlibrary{positioning,chains,shapes,calc}
\usetikzlibrary{fit}
\thispagestyle{empty}
\usetikzlibrary{
  graphs,
  graphs.standard
}
\begin{document}



%
\title{Knapsack: Connectedness, Path, and Shortest-Path}
%
%\titlerunning{Abbreviated paper title}
% If the paper title is too long for the running head, you can set
% an abbreviated paper title here
%
% \author{}
\author{Palash Dey\inst{1} \and
Sudeshna Kolay\inst{2} \and
Sipra Singh\inst{3}}
%
% \authorrunning{}
\authorrunning{Dey et al.}
% First names are abbreviated in the running head.
% If there are more than two authors, 'et al.' is used.
%
% \institute{\email{}}
\institute{Indian Institute of Technology Kharagpur
\email{\{palash.dey\inst{1},skolay\inst{2}\}@cse.iitkgp.ac.in,sipra.singh@iitkgp.ac.in\inst{3}}}
%

% \author{First Author\inst{1}\orcidID{0000-1111-2222-3333} \and
% Second Author\inst{2,3}\orcidID{1111-2222-3333-4444} \and
% Third Author\inst{3}\orcidID{2222--3333-4444-5555}}
% %
% \authorrunning{F. Author et al.}
% % First names are abbreviated in the running head.
% % If there are more than two authors, 'et al.' is used.
% %
% \institute{Princeton University, Princeton NJ 08544, USA \and
% Springer Heidelberg, Tiergartenstr. 17, 69121 Heidelberg, Germany
% \email{lncs@springer.com}}

\maketitle              % typeset the header of the contribution


\begin{abstract}
We study the \kp problem with graph-theoretic constraints. That is, there exists a graph structure on the input set of items of \kp and the solution also needs to satisfy certain graph theoretic properties on top of the \kp constraints. In particular, we study \pa where the solution must be a connected subset of items which has maximum value and satisfies the size constraint of the knapsack. We show that this problem is strongly \NPC even for graphs of maximum degree four and \NPC even for star graphs. On the other hand, we develop an algorithm running in time $\OO\left(2^{\tw\log\tw}\cdot\text{poly}(n)\min\{s^2,d^2\}\right)$  where $\tw,s,d,n$ are respectively treewidth of the graph, the size of the knapsack, the target value of the knapsack, and the number of items. We also exhibit a $(1-\eps)$ factor approximation algorithm running in time $\OO\left(2^{\tw\log\tw}\cdot\text{poly}(n,1/\eps)\right)$ for every $\eps>0$. We show similar results for \pathknapsack and \shortestpathknapsack, where the solution must also induce a path and shortest path, respectively. Our results suggest that \pa is computationally the hardest, followed by \pathknapsack and then \shortestpathknapsack.
\keywords{Knapsack  \and Graph Algorithms \and Parameterised Complexity \and  Approximation algorithm.}


\end{abstract}
%
%
%
\section{Introduction}
Current quantum hardware is unable to carry out universal quantum computations due to the buildup of errors that occur during the computation. 
The magnitude of the individual error is currently above the value that the Threshold Theorem requires in order to kick-start quantum error correction and fault-tolerant quantum computation~\cite[Section 10.6]{nielsen_chuang_2010}. 
Although the experimentally achieved fidelity rates are promising and the error bounds are inching closer to the required threshold, we will have to work for the foreseeable future with quantum hardware with errors that build-up during the computation.  This implies that we can only do a limited number of steps before the output of the computation has become completely uncorrelated with the intended one.

For fault-tolerant quantum computing, we repeat four steps: 
1) We apply a number of single and two-qubit quantum gates, in parallel whenever possible; 
2) We perform a syndrome measurement on a subset of the qubits; 
3) We perform fast classical computations to determine which errors have occurred and how to correct them; 
and, 4) We apply correction terms based on the classical computations.
We then repeat these four steps with a next sequence of gates. 
These four steps are essential to fault-tolerant quantum computing. 


The starting point of this work is to use the four steps outlined above, not to carry out error correction and fault-tolerant computation, but to enhance short, constant-depth, {\em uncorrected} quantum circuits that perform single qubit gates and {\em nearest-neighbor} two qubit gates. 
Since in the long run we will have to implement error-correction and fault-tolerant computation anyhow, and this is done by such a four-step process, why not make other use of this architecture? Moreover, on some of the quantum hardware platforms, these operations are already in place.
Embracing this idea we naturally arrive at the question: what is the computational power of \textit{low-depth} quantum-classical circuits organized as in the four steps outlined above? 
We thus investigate circuits that execute a small, ideally constant, number of stages, where at each stage we may apply, in parallel, single qubit gates and {\em nearest-neighbor} two qubit gates, followed by measurements, followed by low-depth classical computations of which the outcome can control quantum gates in later stages. 
It is not clear, at first, whether such circuits, especially with constant depth, can do anything remotely useful. 
But we will see that this is indeed the case: many quantum computations can be done by such circuits in constant depth. 
By parallelizing quantum computations in this way, we improve the overall computational capabilities of these circuits, as we do not incur errors on qubits that are idle, simply because qubits are not idle for a very long time. 
Furthermore, reducing the depth of quantum circuits, at the cost of increasing width, allows the circuit to be run faster even if errors occur.

The first usage of such a four-step layout, not to do error correction, but to perform computations, can be found in the paradigm of measurement-based quantum computing~\cite{gottesman1999demonstrating,raussendorf2001one,jozsa2006introduction,clark2007generalised}: 
A universal form of quantum computing where a quantum state is prepared and operations are performed by measuring qubits in different bases, depending on previous measurements and intermediate measurements.

\citeauthor{PhamSvore2013} were the first to formalize the four-step protocol for performing computations~\cite{PhamSvore2013}. They included specific hardware topologies by considering two-dimensional graphs for imposing constraints on qubit interactions. In their model, they develop circuits for particularly useful multi-qubit gates, including specifying costs in the width, number of qubits, depth, number of concurrent time steps, size, and total number of non-Identity operations.
As a result, they find an algorithm that factors integers in polylogarithmic depth.
\citeauthor{Browne:2011} showed that the main tool in the work by \citeauthor{PhamSvore2013}, the fan-out gate, can also be replaced by additional log-depth classical computations in the measurement-based quantum computing setting~\cite{Browne:2011}.

More recently, \citeauthor{Cirac:2021} introduced a scheme to implement unitary operations involving quantum circuits combined with Local Operations and Classical Communication ($\mathsf{LOCC}$) channels: $\mathsf{LOCC}$-assisted quantum circuits~\cite{Cirac:2021}. Similarly to the four-step scheme we just described, they allow for a short depth circuit to be run on the qubits, followed by one round of $\mathsf{LOCC}$, in which ancilla qubits are measured and local unitaries are applied based on the measurement outcomes. They show that in this model any 1D transitionally invariant matrix-product state (MPS) with fixed bond dimension is in the same phase of matter as the trivial state. Similar ideas can be found in~\cite{TVV_NonAbelianTopologicalOrder_2022, tantivasadakarn2021long}.

In this work, we introduce a new model, called \textit{Local Alternating Quantum-Classical Computations} ($\LAQCC$). In this model we alternate between running quantum circuits (constrained by locality), ending in the measurement of a subset of qubits, and fast classical computations based on the measurement results. The outcome of the classical computations are then used to control future quantum circuits. We allow for flexibility in this model, by giving different constraints to the power of both the quantum circuits and the classical circuits as well as the number of alternations between them. 
Most attention will be given to $\LAQCC$ containing quantum circuits of constant depth, classical circuits of logarithmic depth and at most a constant number of alternations between them. 
Any circuit constructed in this model is considered to be of constant depth. 
We restrict ourselves to logarithmic depth classical computations, as this is the first natural and non-trivial extension beyond constant-depth classical computations. 
Constant-depth classical computations do however also have an equivalent constant-depth quantum implementation.

The definition of $\LAQCC$ sharpens the original definition of \citeauthor{PhamSvore2013} by adding constraints to the intermediate classical computations. This allows us to bound the power of $\LAQCC$ from above. 

The main result of \citeauthor{Cirac:2021}, that 1D translational invariant MPS with fixed bond dimension can be prepared by $\mathsf{LOCC}$-assisted circuits, relies on local symmetries of the MPS. These symmetries allow them to prepare local states (on a constant number of qubits) and glue them together by doing one round of the appropriate entangling measurement and corrections, after which they run a round of local unitaries to get the desired result. This general scheme for preparing states that exhibit an MPS description with the appropriate local symmetries requires only geometrically local unitaries and one round of measurement and corrections an therefore is accessible in $\LAQCC$. Studying different local symmetries, known as Symmetry Protected Topological (SPT) phases of matter, to find measurement-based constant depth circuits for states is a broad ongoing field of research~\cite{TVV_NonAbelianTopologicalOrder_2022, tantivasadakarn2021long, smith2023deterministic}. 
All these schemes have a $\LAQCC$ implementation.

%$\LAQCC$-circuits also exist for general schemes of preparing local states, based on the local tensors, and gluing them together using one round of entangled measurement and corrections, based on the local symmetry. 
%The main result of \citeauthor{Cirac:2021}, that 1D translational invariant MPS with fixed bond dimension can be prepared by $\mathsf{LOCC}$-assisted circuits, relies heavily on local symmetries of the MPS and as a result also has an equivalent $\LAQCC$ implementation. 
%The corrections applied after the measurement round are local unitaries depending on the local symmetries of the MPS. 

 

%This general scheme of preparing local states, based on the local tensors, and gluing it together by doing one round of entangled measurement and corrections, based on the local symmetry, is accessible in $\LAQCC$.
Note however that \citeauthor{Cirac:2021} also suggest a circuit for the $W$-state.
This circuit uses sequentially and dependent measurement-based corrections of the ancilla qubits. 
These dependent measurements translate to sequential alternations between the quantum and classical circuits and therefore increase the total depth to linear depth, exceeding the constant-depth constraints imposed by $\LAQCC$-circuits. 

We study the power of the $\LAQCC$ model with respect to state preparation, showing that even with only constant quantum-depth and logarithmic classical depth it remains possible to prepare states with long-range entanglement.
Another surprising result is that it is unlikely that $\LAQCC$ circuits are classically simulatable. We show that any instantaneous quantum polynomial-time (IQP) circuit~\cite{Bremner2010,Shepherd2009} has an $\LAQCC$ implementation.
Classical simulation of IQP circuits implies the collapse of the polynomial hierarchy to the third level, which is not believed to be true~\cite{Bremner2017}. Therefore, we expect that $\LAQCC$ circuits are unlikely to be classically simulatable. We bound the power of $\LAQCC$ by showing that it is contained in $\QNC^1$, the class of polynomial-size, log-depth circuits.

Next, we also study the power that intermediate classical calculations can add to quantum computations, by considering a new model that alternates between polynomially many polynomial-depth quantum circuits and unbounded classical computations
We study this model by doing a complexity theoretical analysis, where we draw inspiration from the notions of complexity given by \citeauthor{RosenthalYuen:2022}, \citeauthor{MetgerYuen:2023}, and \citeauthor{Aaronson:2004}.
All three complexity notions are based on the notion of state preparation, instead of more traditional definition of complexity such as the decidability of a computational problem. 
The first two consider classes based on sequences of quantum states preparable by a polynomial-sized quantum circuit, where the circuits are uniformly generated by a computational class, for instance, the class $\mathsf{PSPACE}$, which results in the complexity class $\mathsf{StatePSPACE}$~\cite{RosenthalYuen:2022,MetgerYuen:2023}.
The third notion considers a relative complexity, where the complexity is measured between two given states, and is measured by the number of gates, from a given gate-set, required to transform one state in another state~\cite{Aaronson:2004}. 
For our definition of state preparation complexity, we drop the uniformity constraint from~\cite{RosenthalYuen:2022,MetgerYuen:2023} and define a class as $\mathsf{StateX}$, which refers to states preparable by circuits of type $\mathsf{X}$. 
As an example, if $\mathsf{X} = \QNC^0$, this results in the class $\mathsf{StateQNC^0}$, which is the set of states preparable from the $\ket{0}^n$ state by poly-size constant-depth circuits. 
This notion is similar to the relative complexity from~\cite{Aaronson:2004}, where one state is the  $\ket{0}^n$ state and instead of counting the number of gates we consider the set of states preparable by a fixed number of gates. Using this notion of complexity we show that any state preparable by an $\LAQCC^*$ circuit is also preparable by a $\mathsf{PostQPoly}$ circuit, the class of circuits of polynomial depth with an additional post-selection gate. 

All Clifford circuits have a constant-depth $\LAQCC$ implementation, implying that any stabilizer state can be implemented by a constant-depth $\LAQCC$ circuit, see Section~\ref{sec:clifford_circuits} for a proof of this statement. 
Efficient circuits for stabilizer states have been known already through measurement-based quantum computing. Therefore this paper focuses on the preparation of non-stabilizer states, and as a surprising result we find novel constant-depth protocols for four very natural classes of non-stabilizer states.
Despite the extensive research into these four classes of non-stabilizer states and the many applications of them, no efficient constant- or low-depth state preparation protocols are known yet. We specifically consider these four classes as they are all often used as initial states in other algorithms.

The first state is a uniform superposition over an arbitrary number of states. 
This state finds applications in many quantum algorithms, as they often start with a uniform superposition over multiple states. 
This superposition is often achieved by applying Hadamard gates to every qubit due to its simplicity to prepare. 
Yet, the analysis of many algorithms, such as Shor's algorithm~\cite{Shor:1997}, would benefit from a different initial superposition. 
The circuit to prepare the uniform superposition over an arbitrary number of states uses an exact version of Grover search as a subroutine, that turns a probabilistic circuit, with a known constant probability of success, into a deterministic circuit. 
We use the circuit for preparing a uniform superposition over an arbitrary number of states as a subroutine in the next two quantum state preparation protocols. 

The second state is the $W$-state, the uniform superposition over all computational basis states of Hamming-weight~$1$, a natural long-ranged entangled state that displays a fundamentally nonequivalent type of entanglement from the Greenberger–Horne–Zeilinger state~\cite{WState:2000}, for which $\LAQCC$-type constant-depth circuits were previously known~\cite{PhamSvore2013, Cirac:2021}. 
The $W$-state is often used as benchmark for new quantum hardware~\cite{Haffner2005,Neeley2010,GarciaPerez:2021}. 
A novel way to prepare the $W$-state therefore gives a new way to benchmark different quantum devices with each other. 
A circuit for preparing the $W$-state was given in~\cite{Cirac:2021}, but this implementation requires sequentially alternating measurements followed by local unitaries, which in the $\LAQCC$ model is not considered to be of constant depth. 
We improve this protocol by giving an $\LAQCC$ implementation of the $W$-state, based on a compress-uncompress method that links the one-hot and binary encoding of integers.

The third state considered is the Dicke state, a generalization of the $W$-state, a superposition over all computational basis states with Hamming-weight $k$~\cite{Dicke:1954}. 
Dicke states have relevance in various practical settings.
For instance, for quantum game theory~\cite{zdemir2007}, quantum storage~\cite{Bacon_Compress:2006,Plesch:2010}, quantum error correction~\cite{ouyang2014permutation}, quantum metrology~\cite{toth2012multipartite}, and quantum networking~\cite{prevedel2009experimental}. 
Dicke states have been used as a starting state for variational optimization algorithms, most notably Quantum Alternating Operator Ansatz (QAOA)~\cite{Hadfield2019}, to find solutions to problems such as Maximum k-vertex Cover~\cite{Brandhofer2022,cook2020quantum}.
The ground states of physical Hamiltonians describing one-dimensional chains tend to show a resemblance to Dicke states such as states resulting from the Bethe ansatz, making them an ideal starting state when investigating the ground state behavior of these Hamiltonians~\cite{TDL_BetheAnsatzDerivation:2010,B_ExcitedStateQuantumPhaseTransitions:2013,DickeTransitions:2021}. 
For instance, the algorithm by \citeauthor{van2021preparing}, who give an algorithm to prepare the Bethe ansatz eigenstates of the spin-1/2 XXZ spin chain, starts by first preparing a Dicke state~\cite{van2021preparing}. 
A Dicke-state preparation protocol based on the compress-uncompress methodology used in the $W$-state furthermore finds applications in entanglement distillation, where the entanglement of a large state is concentrated on only a few qubits. 
Efficient deterministic circuits for preparing Dicke states have been proposed by \citeauthor{bartschi2019deterministic}~\cite{bartschi2019deterministic, bartschi2022deterministic_short_depth}. 
They provide a quantum circuit of depth $\mathO(k \log(\frac{n}{k}))$, allowing arbitrary connectivity, to prepare a Dicke state, which they conjecture to be optimal when $k$ is constant. 
In this work, we provide a constant-depth $\LAQCC$ circuit below their conjectured bound already for constant $k$. 
However, this does not directly disprove their conjecture, as we allow for intermediate measurements and classical computations. 
More significantly, we even construct constant-depth $\LAQCC$ circuits for $k = \mathO(\sqrt{n})$ greatly improving their bound.
This construction extends the compress-uncompress method for the $W$-state combined with additional subroutines. 

We continue with a log-depth state preparation protocol for the Dicke-state for arbitrary $k$. 
This protocol implements an efficient transformation between the factoradic number representation and the combinatorial number representation of a positive integer. 
The combinatorial number representation relates directly to the Dicke state. 
The provided efficient transformation between number representation systems might be of independent interest. 

We conclude by modifying our protocol for preparing a Dicke-state to a protocol that prepares quantum many-body scar states in constant-depth. 
These states have low entanglement and longer coherence times than states with similar energy density.
These characteristics make many-body scar states interesting to analyze and relevant within physics.
Many-body scar states appear for instance in the AKLT model~\cite{AKLT:1987,MRBAR:2018,MRB:2018} and different spin models~\cite{SI:2019,MOBFR:2020}.
Known methods for preparing these states have polynomial-depth~\cite{Gustafson:2023}, whereas our circuit has constant depth. 

% We conclude by studying the power that intermediate classical calculations can add to quantum computations. 
% In this study, we define a new model that relaxes constant-depth quantum circuits to polynomial depth quantum circuits, log-depth classical calculations to unbounded classical computations and a constant number of alternations to a polynomial number of alternations. 
% We call this model $\LAQCC^*$. 
% We study this model by doing a complexity theoretical analysis, where we draw inspiration from the notions of complexity given by \citeauthor{RosenthalYuen:2022}, \citeauthor{MetgerYuen:2023}, and \citeauthor{Aaronson:2004}.
% All three complexity notions are based on the notion of state preparation, instead of more traditional definition of complexity such as the decidability of a computational problem. 
% The first two consider classes based on sequences of quantum states preparable by a polynomial-sized quantum circuit, where the circuits are uniformly generated by a computational class, for instance, the class $\mathsf{PSPACE}$, which results in the complexity class $\mathsf{StatePSPACE}$~\cite{RosenthalYuen:2022,MetgerYuen:2023}.
% The third notion considers a relative complexity, where the complexity is measured between two given states, and is measured by the number of gates, from a given gate-set, required to transform one state in another state~\cite{Aaronson:2004}. 
% For our definition of state preparation complexity, we drop the uniformity constraint from~\cite{RosenthalYuen:2022,MetgerYuen:2023} and define a class as $\mathsf{StateX}$, which refers to states preparable by circuits of type $\mathsf{X}$. 
% As an example, if $\mathsf{X} = \QNC^0$, this results in the class $\mathsf{StateQNC^0}$, which is the set of states preparable from the $\ket{0}^n$ state by poly-size constant-depth circuits. 
% This notion is similar to the relative complexity from~\cite{Aaronson:2004}, where one state is the  $\ket{0}^n$ state and instead of counting the number of gates we consider the set of states preparable by a fixed number of gates. Using this notion of complexity we show that any state preparable by an $\LAQCC^*$ circuit is also preparable by a $\mathsf{PostQPoly}$ circuit, the class of circuits of polynomial depth with an additional post-selection gate. 

\paragraph{Summary of results}
\begin{itemize}
    \item We give a new definition of a computational model that captures the power of the four step process: applying a constant number of layers of one- and two-qubit gates; performing a syndrome measurement; perform a fast classical computation determining corrections; apply corrections. We call this model \emph{Local Alternating Quantum Classical Computations}, or $\LAQCC$ for short. In this model we bound the allowed quantum operations, intermediate classical calculations, and number of rounds separately. In Section~\ref{sec:LAQCC_model} we define this model and give a list of operations based on results from literature contained in this computational model. In some of these operations we explicitly use that we allow for multiple, but at most constant, rounds  of corrections.
    \item  We show show that there exist $\LAQCC$ circuits that can not be weakly simulated in Section~\ref{sec:IQP_in_LAQCC}. We further show that for every $\LAQCC$ circuit there exists a $\QNC^1$ circuit simulating it perfectly, in Section~\ref{sec:LAQCC_in_QNC1}.
    \item We introduce a new type computational complexity for preparing states and show that the extension of $\LAQCC$ where we allow a polynomial number of rounds and unbounded classical computation, is contained in $\mathsf{PostQPoly}$, the class of polynomial circuits with post-selection, in Section~\ref{sec:Complexity results}.
    \item We show a protocol to prepare the uniform superposition state of size $q$ in $\LAQCC$ using $\mathO(\ceil{\log_2(q)}^2)$ qubits in Section~\ref{sec:superposition_modulo_q}. 
    \item We show a protocol to prepare the $W_n$ state in $\LAQCC$ using $\mathO(n\log(n))$ qubits in Section~\ref{sec:W_state_in_LAQCC}.
    \item We show two ways of preparing the Dicke-$(n,k)$ state. The first method is in $\LAQCC$, works up to $k = \mathO(\sqrt{n})$, uses $\mathO(n^2\log(n))$ qubits, and is found in Section~\ref{sec:dicke:small_k}. The second method is in $\LAQCC\text{-}\mathsf{LOG}$ (an extension of $\LAQCC$ allowing for logarithmic number of alterations instead of constant), works for any $k$, uses $\mathO(\text{poly}(n))$ qubits, and is found in Section~\ref{sec:Dicke_in_LAQCC_LOG}. 
    \item We extend on our $\LAQCC$ method of generating Dicke-$(n,k)$ states for $k = \mathO(\sqrt{n})$ and show a protocol to generate many-body scar states for a particular Hamiltonian in $\LAQCC$ (Section~\ref{sec:many_body_scar}). 
\end{itemize}
Summarized in a table, we provide the following state generation protocols:
\begin{table}[htb]
\centering
\begin{tabular}{l|l|l|l}
\textbf{State description} & \textbf{Width} & \textbf{Depth} & \textbf{Implementation}\\
\hline 
Uniform superposition mod $q$: $\frac{1}{\sqrt{q}} \sum_{i = 0}^{q-1}\ket{i}$ & $\mathO(\ceil{\log^2 q})$ & $\mathO(1)$ & Section~\ref{sec:superposition_modulo_q}\\

$W$-state: $\frac{1}{\sqrt{n}}\sum_{i = 0}^{n-1}\ket{e_i}$ & $\mathO(n \log n)$ & $\mathO(1)$ & Section~\ref{sec:W_state_in_LAQCC}\\

Dicke-$(n,k)$, $k = \mathO(\sqrt{n})$: $\binom{n}{k}^{-1/2}\sum_{x \in \{0,1\}^n: |x| = k} \ket{x}$ &  $\mathO(n^2\log n)$ & $\mathO(1)$ 
&Section~\ref{sec:dicke:small_k}\\

Dicke-$(n,k)$: $\binom{n}{k}^{-1/2}\sum_{x \in \{0,1\}^n: |x| = k} \ket{x}$ & $\mathO(\text{poly}(n))$ & $\mathO(\log n)$ &Section~\ref{sec:Dicke_in_LAQCC_LOG}\\

QMBS: $\ket{S_k} = \frac{1}{k! \sqrt{\mathcal N(n,k)}}(Q^\dagger)^k \ket{\Omega}$ &  $\mathO(n^2\log n)$ & $\mathO(1)$  &  Section~\ref{sec:many_body_scar}
\end{tabular}
\caption{Summary of state preparation protocols given in this paper.}
\label{tab:sate_prep}
\end{table}
In the entry for the quantum many-body scar state $Q$ denotes the raising operator and $\mathcal N(n,k)=\binom{n-k-1}{k}$. 
Section~\ref{sec:many_body_scar} will provide more details on the variables and the implementation. 

\paragraph{Organization of the paper}
\noindent We first introduce relevant preliminaries in Section~\ref{sec:preliminaries}. 
In Section~\ref{sec:LAQCC_model} we formally define the class of Local Alternating Quantum-Classical Computations ($\LAQCC$). We also show that any Clifford circuit can be implemented in constant depth $\LAQCC$ (a result based on a result from measurement-based quantum computing~\cite{jozsa2006introduction}). 
This result allows us to give many useful multi-qubit gates and routines in Section~\ref{sec:gates_created_in_LAQCC}. 
Beyond that we show that constant depth $\LAQCC$ circuits are contained in $\QNC^1$ and that any $\mathsf{IQP}$ circuit has an $\LAQCC$ implementation.
We conclude this section with an analysis of a more powerful instantiation of $\LAQCC$ and show an inclusion with respect to the class $\mathsf{PostQPoly}$, which is the class of circuits of polynomial depth with one additional post-selection gate. 
In Section~\ref{sec:state_prep_in_LAQCC} we give $\LAQCC$ circuit implementations for preparing the uniform superposition over an arbitrary number of states, the $W$-state and the Dicke state up to $k = \mathO(\sqrt{n})$. We furthermore give a log-depth circuit implementation for preparing the Dicke state for any $k$. We conclude by showing a $\LAQCC$ circuit for generating many body scar states of a particular type of Hamiltonian.


We first review some basic concepts from probability theory (see standard textbooks such as \cite{pollard2002user,williams1991probability} for a detailed treatment), 
%the background of Bayesian inference, and finally 
%We first review some basic concepts from probability theory, 
and then present the Bayesian probabilistic programming language and the normalised posterior distribution (NPD) problem.
%we consider in this work. 
Throughout the paper,
we denote by $\Nset$, $\Zset$ and $\Rset$ the sets of all natural numbers (including zero), integers, and real numbers, respectively.

\vspace{-1.5ex}
\subsection{Basics of Probability Theory}
%We assume familiarity with basic probability theory (see \cref{app:prelim} for details). 

A \emph{measurable space} is a pair $(U,\Sigma_U)$, where $U$ is a nonempty set and $\Sigma_U$ is a $\sigma$-algebra on $U$, i.e., a family of subsets of $U$ such that $\Sigma_U\subseteq \mathcal{P}(U)$ contains $\emptyset$ and is closed under complementation and countable union. Elements of $\Sigma_U$ are called \emph{measurable} sets. A function $f$ from a measurable space $(U_1,\Sigma_{U_1})$ to another measurable space $(U_2,\Sigma_{U_2})$ is \emph{measurable} if $f^{-1}(A)\in\Sigma_{U_1}$ for all $A\in\Sigma_{U_2}$.

A \emph{measure} $\mu$ on a measurable space $(U,\Sigma_U)$ is a mapping from $\Sigma_U$ to $[0,\infty]$ such that (i) $\mu(\emptyset)=0$ and (ii) $\mu$ 
%satisfies the
is countably additive:
%condition: 
for every pairwise-disjoint set sequence $\{A_n\}_{n\in\Nset}$ in $\Sigma_U$, it holds that $\mu(\bigcup_{n\in\Nset}A_n)=\sum_{n\in\Nset}\mu(A_n)$. We call the triple $(U,\Sigma_U,\mu)$ a \emph{measure space}. 
%If $\mu(U)\le 1$, we call $\mu$ a \emph{subprobability measure}. 
If $\mu(U)=1$, we call $\mu$ a \emph{probability measure}, and $(U,\Sigma_U,\mu)$ a \emph{probability space}.
The Lebesgue measure $\lambda$ is the unique measure on $(\Rset,\Sigma_{\Rset})$ satisfying $\lambda([a,b))=b-a$ for all valid intervals $[a,b)$ in $\Sigma_{\Rset}$. For each $n\in\Nset$, we have a measurable space $(\Rset^n,\Sigma_{\Rset^n})$ 
%such that there exists 
and
a unique product measure $\lambda_n$ on $\Rset^n$ satisfying $\lambda_n(\prod_{i=1}^n A_i)=\prod_{i=1}^n \lambda(A_i)$ for all $A_i\in\Sigma_{\Rset}$.


The \emph{Lebesgue} integral operator $\int$ is a partial operator that maps a measure $\mu$ on $(U,\Sigma_U)$ and a real-valued function $f$ on the same space $(U,\Sigma_U)$ to a real number or infinity, which is denoted by $\int f \mathrm{d}\mu$ or $\int f(x)\mu(\mathrm{d}x)$. 
The detailed definition of Lebesgue integral is somewhat technical, see \cite{rankin1968real,rudin1976principles} for more details. 
Given a measurable set $A\in\Sigma_U$, the integral of $f$ over $A$ is defined by $\int_A f(x)\mu(\mathrm{d} x):=\int f(x) \cdot [x\in A] \mu(\mathrm{d}x)$
%\begin{align*}
%\textstyle\int_A f(x)\mu(\mathrm{d} x):=\int f(x) \cdot [x\in A] \mu(\mathrm{d}x)
%\end{align*} 
where $[-]$ is the Iverson bracket such that $[\phi]=1$ if 
%the predicate 
$\phi$ is true, and $0$ otherwise. If $\mu$ is a probability measure, then we call the integral as the \emph{expectation} of $f$, denoted by $\expectdist{x\sim\mu;A}{f}$, or $\expv[f]$ when the scope is clear from the context.

For a measure $v$ on $(U,\Sigma_U)$, a measurable function $f:U\to \Rset_{\ge 0}$ is the \emph{density} of $v$ with respect to $\mu$ if $v(A)=\int f(x)\cdot [x\in A] \mu(\mathrm{d} x)$ for all measurable $A\in\Sigma_U$, and $\mu$ is called the \emph{reference measure} (most often $\mu$ is the Lebesgue measure). Common families of probability distributions on the reals, e.g., uniform, normal distributions, are measures on $(\Rset,\Sigma_{\Rset})$. Most often these are defined in terms of probability density functions with respect to the Lebesgue measure. That is, for each $\mu_D$ there is a measurable function $\text{pdf}_D:\Rset\to\Rset_{\ge 0}$ that determines it: $\mu_D(A):=\int_A \text{pdf}_D (\mathrm{d}\lambda) $. As we will see, density functions such as $\text{pdf}_D$ play an important role in Bayesian inference.

Given a probability space $\pspace$, a \emph{random variable} is an $\mathcal{F}$-measurable function $X: \Omega \rightarrow \Rset \cup \{+\infty,-\infty\}$. The expectation of a random variable $X$, denoted by $\expv(X)$, is the Lebesgue integral of $X$ w.r.t. $\probm$, i.e., $\int X\,\mathrm{d}\probm$. A \emph{filtration} of $\pspace$ is an infinite sequence $\{ \mathcal{F}_n \}_{n=0}^{\infty}$ such that for every $n\ge 0$, the triple $(\Omega, \mathcal{F}_n, \probm)$ is a probability space and $\mathcal{F}_n \subseteq \mathcal{F}_{n+1} \subseteq \mathcal{F}$. A \emph{stopping time} w.r.t. $\{ \mathcal{F}_n \}_{n=0}^{\infty}$ is a random variable $T: \Omega \rightarrow \Nset \cup \{0, \infty\}$ such that for every $n \geq 0$, the event \{$T \leq n$\} is in $\mathcal{F}_n$. 

A \emph{discrete-time stochastic process} is a sequence $\Gamma = \{X_n\}_{n=0}^\infty$ of random variables in $\pspace$. The process $\Gamma$ is \emph{adapted} to a filtration $\{ \mathcal{F}_n \}_{n=0}^{\infty}$, if for all $n \geq 0$, $X_n$ is a random variable in $(\Omega, \mathcal{F}_n, \probm)$. A discrete-time stochastic process $\Gamma=\{X_n\}_{n=0}^\infty$ adapted to a filtration $\{\mathcal{F}_n\}_{n=0}^\infty$ is a \emph{martingale} (resp. \emph{supermartingale}, \emph{submartingale})
if for all $n \geq 0$, $\expv(|X_n|)<\infty$ and it holds almost surely (i.e.,~with probability $1$) that
$\condexpv{X_{n+1}}{\mathcal{F}_n}=X_n$ (\mbox{resp. } $\condexpv{X_{n+1}}{\mathcal{F}_n}\le X_n$, $\condexpv{X_{n+1}}{\mathcal{F}_n}\ge X_n$).
See~\cite{williams1991probability} for details.
%Intuitively, a martingale is a discrete-time stochastic process, in which at any time $n$, the expected value $\condexpv{X_{n+1}}{\mathcal{F}_n}$ in the next step, given all previous values, is equal to the current value $X_n$. In a supermartingale, this expected value is less than or equal to the current value and a submartingale is defined conversely.
Applying martingales to qualitative and quantitative analysis of probabilistic programs is a well-studied technique~\cite{SriramCAV,ChatterjeeFG16,ChatterjeeNZ2017}.


\subsection{Bayesian Probabilistic Programming Language}

%We consider an imperative arithmetic probabilistic programming language. 
The syntax of our probabilistic programming language (PPL) is given in \cref{fig:syntax}, where the metavariables $S$, $B$ and $E$ stand for statements, boolean expressions and arithmetic expressions, respectively.   
Our PPL is imperative with the usual conditional and loop structures (i.e.,~\textbf{if} and \textbf{while}), as well as the following new structures: (a)~sample constructs of the form ``$\textbf{sample}\  D$'' that sample a value from a prescribed distribution $D$ over $\mathbb{R}$ and then assign this value to a sampling variable $r$; (b)~score statements of the form ``\textbf{score}($EW$)'' that weight the current execution with a value expressed by $EW$ (note that $\textit{pdf}(D,x)$ means the value of a probability density function w.r.t. $D$ at $x$);
%\footnote{Instead of the hard conditioning that refutes the execution when the observation mismatches the value of the sampling variable, we use the more general soft conditioning and assume the existence of a global weight variable initialized  to $1$.}
%for each program
(c)~probabilistic branching statements of the form
``$\textbf{if}\ \textbf{prob}(p)\dots$'' that lead to the then part with probability
$p\in (0,1]$ and to the else part with probability $1-p$. We also have sequential compositions (i.e., ";") and support return statements (i.e., \textbf{return}) that 
return the value of the program variable of interest. %The set of all statements is denoted by $Stmt$.
Note that $c,c_1,c_2\in\Rset$ are constants, and our language supports any distributions with continuous density functions and infinite supports, 
including but not limited to uniform and normal distributions. 



% Figure environment removed





Given a probabilistic program in our language, we distinguish two disjoint sets of variables in the program: (i) the set $\pvars$ of \emph{program variables} whose values are determined by assignments in the program (i.e., the expressions at the LHS of ``:="); (ii)~the set $\rvars$ of \emph{sampling variables} whose values are independently sampled from prescribed probability distributions each time they are accessed (i.e., each ``$\textbf{sample}\ D$" can be regarded as a sampling variable $r$). 




\begin{example}\label{ex:pedestrian-program}

%Consider the pedestrian random walk example~\cite{DBLP:conf/esop/MakOPW21}, a pedestrian is lost on a road, and she only knows that she is away from her house at most $3$ km. Thus, she starts to repeatedly walk a uniformly random distance of at most $1$ km in either direction, until reaching her house. Upon she arrives, an  odometer tells that she has walked $1.1$ km totally. However, this odometer was once broken and the measured distance is normally distributed around the true distance with standard deviation $0.1$ km. 
\cref{fig:pedestrian-program} shows a Bayesian probabilistic program written in our PPL language. In this program, the set of program variables is $\pvars=\{start,pos,dis,step\}$, and the set of sampling variables is $\rvars=\{ \textbf{sample uniform}(0,1)\}$. Each time $\textbf{sample uniform}(0,1)$ is executed, it samples a value uniformly from $[0,1]$ and then assigns the value to the variable $step$. 
%Thus, $step$ is associated with the probability distribution $\textbf{uniform}(0,1)$.
\qed


	
% Figure environment removed
\end{example}

\subsection{The Semantics of Our Programming Language}

%To relate variables with their values, we introduce the notion of valuations. 
Let $V$ be a finite set of variables with an implicit linear order over its elements. A \emph{valuation} on $V$ is a function $\pv: V \rightarrow \Rset$ that assigns a real value to each variable in $V$. We denote the set of all valuations on $V$ by $\val{V}$. For each $1\le i\le |V|$, we denote the value of the $i$-th variable (in the implicit linear order) in $\pv$ by $\pv[i]$, so that we can view each valuation as a real vector on $V$. A \emph{program} (resp. \emph{sampling}) valuation is a valuation on $\pvars$ (resp. $\rvars$), respectively. 
For the sake of convenience, we fix the notations in the following way, i.e., we always use $\pv\in\val{\pvars}$ to denote a program valuation, and $\rv\in\val{\rvars}$ to denote a sampling valuation; we also write $\pv[\mathit{ret}]$ for the value of the return variable in $\pv$. 



Below we present the semantics for our programming language. Existing semantics in the literature are either measure-\cite{DBLP:conf/lics/StatonYWHK16,LeeYRY20} or sampling-based  \cite{DBLP:conf/esop/MakOPW21,Beutner2022b}. To facilitate the development of our algorithm, we consider the \emph{transition-based} semantics~\cite{DBLP:conf/cav/ChakarovS13,DBLP:conf/popl/ChatterjeeFNH16} to our language and 
%To apply template-based algorithmic approaches to NPD problems, we consider  that 
treat each probabilistic program as a \emph{weighted probabilistic transition system} (WPTS). A WPTS extends a PTS  ~\cite{DBLP:conf/cav/ChakarovS13,DBLP:conf/popl/ChatterjeeFNH16} with weights and an initial probability distribution. 





%Below we present a variant of probabilistic transition systems \cite{DBLP:conf/cav/ChakarovS13}.
\begin{definition}
%[Weighted Probabilistic Transition Systems]
[WPTS]\label{def:wpts}
	A \emph{weighted probabilistic transition system} (WPTS) $\Pi$
	is a tuple
\begin{equation}\label{eq:wpts} 
\tag{\dag}
\Pi = (\pvars, \rvars,  L,\lin,\lout,\mu_{\mathrm{init}}, \rdvarjdis,\transset)%\win)
\end{equation}
for which:
	\begin{itemize}
		\item
		$\pvars$ and $\rvars$ are finite disjoint sets of \emph{program} and resp. \emph{sampling} variables.
%  (variables}) 
%  such that $\pvars\cap \rvars=\emptyset$.
    \item $\locs$ is a finite set of \emph{locations} 
  %or \emph{program counters} 
  with special locations $\lin,\lout\in \locs$. Informally, $\lin$ is the initial location and $\lout$ represents program termination. 
		\item
		$\mu_{\mathrm{init}}$ is the \emph{initial probability distribution} over $\mathbb{R}^{\pvars}$ with a finite support (denoted by $\supp{\mu_{\mathrm{init}}}$), 
  %from which the initial program valuation %$\valin$ is sampled, 
  while $\rdvarjdis$ is a function that assigns a probability distribution $\rdvarjdis(r)$ to each 
  %sampling variable 
  $r \in \rvars$. We call each $\pv\in\supp{\mu_{\mathrm{init}}}$ an \emph{initial program valuation}, and abuse the notation so that $\rdvarjdis$ also denotes the independent joint distribution of all $\rdvarjdis(r)$'s ($r\in \rvars$).
		\item 
		$\transset$ is a finite set of \emph{transitions} where
		each transition $\tau \in \transset$ is a tuple $\langle \loc, \phi, F_1,\dots,F_k \rangle$ such that 
(i) $\loc\in L$ is the \emph{source location} of the transition, 
%\item 
(ii) $\phi$ is the \emph{guard condition} which is a predicate over variables $\pvars$, %which serves as the \emph{guard condition}, 
and (iii) each $F_j:=\langle \loc'_j, p_j, \upd_j,\wet_j \rangle$ is called a \emph{weighted fork} for which (a) $\loc'_j\in L$ is the \emph{destination location} of the fork, (b) $p_j\in (0,1]$ is the probability of this fork, (c) $\upd_j:\Rset^{|\pvars|} \times \Rset^{|\rvars|} \rightarrow \Rset^{|\pvars|}$ is an {\em update function} that takes as inputs the current program and sampling valuations  and returns an updated program valuation in the next step, and (d) $\wet_j:\Rset^{|\pvars|} \times \Rset^{|\rvars|}\to [0,\infty)$ is a \emph{score function} that gives the likelihood weight of this fork depending on the current program and sampling valuations.	
\end{itemize}
\end{definition}


In a WPTS, we use update and score functions to model the update on the program variables and resp. the likelihood weight when running a basic block of statements in a program, respectively.  
%and use score functions to model  caused by the execution of the score statements (if exists) in this block. 
If there is no score statement in the block, then the score function is constantly $1$. 
We always assume that a WPTS $\Pi$ is \emph{deterministic} and \emph{total}, i.e., (i) there is no program valuation that simultaneously satisfies the guard conditions of two distinct transitions from the same source location, and (ii) the disjunction of the guard conditions of all the transitions from any source location is a tautology. 
The transformation from a probabilistic program into its WPTS can be done in a straightforward way (see e.g.~\cite{DBLP:journals/toplas/ChatterjeeFNH18,DBLP:conf/cav/ChakarovS13}). 

\begin{example}\label{ex:pedestrian-semantics} 
\cref{fig:pedestrian-wpts} shows the WPTS of the program in \cref{fig:pedestrian-program} which has two locations $\lin,\lout$. 
 %In the WPTS, 
The circle nodes represent locations and square nodes model the forking behavior of transitions. An edge entering a square node is labeled with the condition of its respective transition, while an edge entering a circle node stands for a fork, which is associated with its probability, update functions and score functions that marked by $w$.\footnote{Here we omit the update functions if the values of program variables are unchanged.} The value of $step$ is initialised to $0$. An the initial probability distribution $\mu_{\mathrm{init}}$ is determined by the joint distribution of $(start,pos,dis,step)$ where $start\sim uniform(0,3)$ and $pos,dis,step$ observe the Dirac measures $Dirac(\{start\})$, $Dirac(\{0\})$ and $Dirac(\{0\})$, respectively, e.g., the probability of the event ``$dis\in\{0\}$'' equals $1$. As $step$ simply receives values from a sampling variable, we neglect it in the WPTS.\qed
\end{example}

%\paragraph{Score-recursive WPTS.} 

We say that a WPTS is \emph{non-score-recursive} if for all transitions $\tau=\langle \loc, \phi, F_1,  F_2,\dots,F_k \rangle$ in the WPTS with each fork $F_j=\langle \loc'_j, p_j, \upd_j,\wet_j \rangle$ ($1\le j\le k$), we have that each score function $\wet_j$ is constantly $1$ (i.e., the multiplicative weight does not change) for every $\loc'_j\ne \lout$. Otherwise, the WPTS is \emph{score-recursive}.
Informally, a non-score-recursive WPTS has non-trivial score functions only on the transitions to the termination of a program, while a score-recursive WPTS has {\tt score} statements in the execution of the program. 
For example, the WPTS of the program in~\cref{sec3:pedestrian} is non-score-recursive as the nontrivial (i.e., score values not equal to $1$) {\tt score} statement only appears to the termination, while the WPTS of the program in \cref{sec3:phylogenetic} is score recursive since it has {\tt score} statements inside the loop body.
In the case of a non-score-recursive WPTS, we say that the WPTS is \emph{score-bounded} by a positive real $M>0$ if for every $\tau=\langle \loc, \phi, F_1, F_2,\dots,F_k \rangle$ in the WPTS with $F_j=\langle \loc'_j, p_j, \upd_j,\wet_j \rangle$ ($1\le j\le k$), we have that 
$|\wet_j|\le M$ whenever $\loc'_j=\lout$.


Given a program valuation $\mathbf{v}$ and a predicate $\phi$ over variables $\pvars$, we say that $\mathbf{v}$ \emph{satisfies} $\phi$ (written as $\mathbf{v}\models\phi$) if $\phi$ holds when the variables in $\phi$ are substituted by their values in $\mathbf{v}$. 
A \emph{state} 
%of the WPTS $\Pi$ 
is a pair $\Xi=(\loc, \pv)$ where $\loc \in L$ (resp. $\pv \in \Rset^{|\pvars|}$) represents the current location (resp. program valuation), respectively, while a \emph{weighted state} is a triple 
%$\Xi^w:=(\loc, \pv,w)$ 
$\Theta=(\loc, \pv, w)$ 
where $(\loc, \pv)$ is a state and $w\in [0,\infty)$ represents the multiplicative likelihood weight accumulated so far. 


 
%\paragraph{Semantics.} 
Below we specify the semantics of a WPTS. Consider a WPTS $\Pi$ in the form of \eqref{eq:wpts}. The semantics of $\Pi$ is formalized by the infinite sequence $\Gamma=\{\widehat{\Theta}_n=(\widehat{\loc}_n,\widehat{\pv}_n,\widehat{w}_n)\}_{n\ge 0}$ 
%of \emph{random weighted states} 
where each $(\widehat{\loc}_n,\widehat{\pv}_n,\widehat{w}_n)$ is the random weighted state at the $n$th execution step of the WPTS such that $\widehat{\loc}_n$ (resp. $\widehat{\pv}_n$, $\widehat{w}_n$) is the random variable for the location (resp. the random vector 
%of random variables 
for the program valuation, the random variable for the multiplicative likelihood weight) at the $n$th step, respectively. %The initial random state $\widehat{\Theta}_0$ is constant and equals $(\lin,\valin,\win)$. 
%its corresponding stochastic process $\Gamma:=\{\hat{\Xi}_n\}_{n\ge 0}$ on states.
The sequence $\Gamma$ starts with the initial random weighted state 
$\widehat{\Theta}_0=(\widehat{\loc}_0,\widehat{\pv}_0,\widehat{w}_0)$ such that $\widehat{\loc}_0$ is constantly $\lin$, $\widehat{\pv}_0\in \supp{\mu_\mathrm{init}}$ is sampled from the initial distribution $\mu_\mathrm{init}$ and the initial weight $\widehat{w}_0$ is constantly set to $1$\footnote{This follows the traditional setting in e.g.~\cite{Beutner2022b}.}. 
Then, given the current random weighted state $\widehat{\Theta}_n=(\widehat{\loc}_n,\widehat{\pv}_n,\widehat{w}_n)$ at the $n$th step, the next random weighted state $\widehat{\Theta}_{n+1}=(\widehat{\loc}_{n+1},\widehat{\pv}_{n+1},\widehat{w}_{n+1})$ is determined by:
(a) If $\widehat{\loc}_n=\lout$, then $(\widehat{\loc}_{n+1}, \widehat{\pv}_{n+1},\widehat{w}_{n+1})$ takes the same weighted state as $(\widehat{\loc}_n,\widehat{\pv}_n,\widehat{w}_n)$ (i.e., the next weighted state stays at the termination location $\lout$);
(b) Otherwise, $\widehat{\Theta}_{n+1}$ is determined by the following procedure:
\begin{itemize}
\item First, since the WPTS $\Pi$ is deterministic and total, we take the unique transition $\tau=\langle \hat{\loc}_n,\phi,F_1,\dots, F_k \rangle$ such that $\hat{\pv}_n\models\phi$. 
\item Second, we choose a fork $F_j=\langle \loc_j, p_j,\upd_j,\wet_j\rangle$ with probability $p_j$.
\item 
Third, we obtain a sampling valuation $\rv\in \supp{\rdvarjdis}$ 
%over the sampling variables $\rvars$ 
by sampling each $r \in \rvars$ independently w.r.t the probability distribution $\rdvarjdis(r).$
\item Finally, the value of the next random weighted state $(\widehat{\loc}_{n+1}, \widehat{\pv}_{n+1},\widehat{w}_{n+1})$ is determined as that of 
$(\loc'_j, \upd_j(\hat{\pv}_n,\rv),\widehat{w}_n\cdot \wet_j(\widehat{\pv}_n,\rv))$. 
\end{itemize}


Given the semantics, a \emph{program run} of the WPTS $\Pi$ is a concrete instance of $\Gamma$, i.e., an infinite sequence $\omega=\{\Theta_n\}_{n\ge 0}$ of weighted states where each $\Theta_n=(\loc_n,\pv_n,w_n)$ is the concrete weighted state at the $n$th step in this program run with location $\loc_n$, program valuation $\pv_n$ and multiplicative likelihood weight $w_n$. A state $(\loc,\pv)$ is called \emph{reachable} if there exists a program run $\omega=\{\Theta_n\}_{n\ge 0}$ such that $\Theta_n=(\loc,\pv,w_n)$ for some $n$. 


 
\begin{example}\label{ex:pedestrian-run}
Consider the WPTS in \cref{ex:pedestrian-semantics}. Consider an initial program valuation $(1,1,0)$ which means that the initial values of $start,pos,dis$ are $1,1,0$, respectively. Then starting from the initial weighted state $(\lin,(1,1,0),1)$, a program run w.r.t the WPTS semantics above could be 
\[
(\lin,(1,1,0),1)\to (\lin,(1,0.5,0.5),1)\to (\lin,(1,-0.1,1.1),1)\to (\lout,(1,-0.1,1.1),3.9894).\qed
\]
\end{example}

Given an initial program valuation $\valin$ of a WPTS, one could construct a probability space over the program runs by their probabilistic evolution described above and standard constructions such as general state space Markov chains~\cite{meyn2012markov}. We denote the probability measure in the probability space by $\probm_{\valin}(-)$ and the expectation operator by $\expectdist{\valin}{-}$.  



\subsection{Normalised Posterior Distribution}\label{sec2:NPD}


Before presenting the central problem of Bayesian probabilistic programming, i.e., analyzing normalised posterior distribution with our WPTS models, we introduce some technical concepts.

%\paragraph{Termination.}
\begin{definition}[Termination]
The \emph{termination time} of a WPTS
%The \emph{termination time} of the WPTS 
$\Pi$ 
%is a random variable $T$ defined on programs runs given 
is the random variable $T$ given by
%a program run  $\omega=\{\Xi_n=(\loc_n,\pv_n,w_n)\}_{n\in\Nset}$,
%\begin{align*}	
$T(\omega):=\text{min}\{n\in\Nset\mid \loc_n=\lout\}$ for every program run  $\omega=\{(\loc_n,\pv_n,w_n)\}_{n\ge 0}$
%\end{align*}
where $\text{min}\,\emptyset:=\infty$. That is, $T(\omega)$ is the number of steps a program run $\omega$ takes to reach the termination location $\lout$. A WPTS $\Pi$ is \emph{almost-surely terminating} (AST) if $\probm_{\valin}(T<\infty)=1$ for all initial program valuations $\valin\in \supp{\mu_{\mathrm{init}}}$.  
%in the case that the program run never terminates. 
\end{definition}




\begin{definition}[Expected Weights]\label{def:exp-wt}
 Given a WPTS $\Pi$ in the form of \eqref{eq:wpts}, a designated initial program valuation $\valin$ and a measurable subset $\calU\in\Sigma_{\Rset^{|\pvars|}}$, the \emph{expected weight} $\measureSem{\Pi}_{\valin}(\calU)$ 
%$\measureSem{\Pi}(\valin)$ 
%of $\Pi$ w.r.t $\pv$ 
is defined as
%$\measureSem{\Pi}_\calU(\valin):=\expectdist{\valin}{\widehat{w}_T}$. 
$\measureSem{\Pi}_{\valin}(\calU):=\expectdist{\valin}{[\widehat{\pv}_T\in \calU]\cdot\widehat{w}_T}$. 
\end{definition}

By definition, we have that $\widehat{\pv}_T$ (resp. $\widehat{w}_T$) is the random vector (resp. variable) of the program valuation (resp. the multiplicative likelihood weight) at termination, respectively. Thus, $\measureSem{\Pi}_{\valin}(\calU)$ is the expectation of $\widehat{w}_T$ 
%over all program runs 
that start from the state $(\lin,\valin,1)$ and end with $\widehat{\pv}_T\in\calU$. If $\calU=\Rset^{|\pvars|}$, the restriction of $\widehat{\pv}_T\in\calU$ can be removed.

Below we define the normalised posterior distribution (NPD) problem. %under our WPTS semantics. 

 
\begin{definition}[Normalised Posterior Distribution]\label{def:npd}
Given a WPTS $\Pi$ in the form of \eqref{eq:wpts},
%We write $\measureSem{\Pi}(\valin)$ iff $\calU=\Rset^{|\pvars|}$.)
%Then given a probability distribution $\mu$ over initial program valuations, 
the \emph{normalised posterior distribution} (NPD) $\posterior_\Pi$ of $\Pi$ 
%over $U$ 
is defined by:
\begin{align*}
\posterior_{\Pi}(\calU):=\measureSem{\Pi}(\calU)/Z_\Pi\mbox{ for all measurable subsets } \calU\in \Sigma_{\Rset^{|\pvars|}},   
\end{align*}	
where 
$\measureSem{\Pi}(\calU):=\int_{\calV} \measureSem{\Pi}_{\pv}(\calU)\cdot \mu_{\mathrm{init}}(\mathrm{d} \pv)$ is the \emph{unnormalised posterior distribution} w.r.t. $\calU$, $\calV:=\supp{\mu_{\mathrm{init}}}$, %is the support of $\mu_{\mathrm{init}}$
%is the integral of all expected weights with an initial program valuation $\pv\in U$, 
and $Z_\Pi:=\measureSem{\Pi}(\Rset^{|\pvars|})$ is the \emph{normalising constant}.  
The WPTS $\Pi$ is called \emph{integrable} 
%w.r.t a probability distribution (for initial program valuations) 
if we have $0<Z_{\Pi}<\infty$. 
%\pw{Shall we mention that $\measureSem{\Pi}_{\pv}(\calU)$ is an integrable function here?}
\end{definition}

%We call a WPTS $\Pi$ \emph{integrable} 
%w.r.t a probability distribution (for initial program valuations) 
%if the normalising constant is finite, i.e., ~$0<Z_{\Pi}<\infty$. %for any $\pv\in\val{\pvars}$. 
%Given an integrable program, we are interested in deriving lower and upper bounds on the normalised posterior distribution over some measurable set $U\in \Sigma_\Rset$.
\paragraph{Interval Bounds for NPD.} In this work, we consider the automated interval-bound analysis for NPD of a WPTS. Formally, we aim to derive an interval $[l,u]\subseteq [0,\infty)$ 
for an integrable WPTS $\Pi$ and any measurable sets $\calU\in\Sigma_{\Rset^{|\pvars|}}$ as tight as possible such that $l\le \posterior_{\Pi}(\calU) \le u$. 
%$l,u$ are called \emph{interval bounds} for the NPD $\posterior_{\Pi}(\calU)$. 
%To achieve this, in the following (\cref{sec:math}) we develop approaches to obtain interval bounds for expected weights as $\measureSem{\Pi}(\calU)$ and $Z_\Pi$ are integrations of expected weights over $\calV$. 
 



To achieve interval bounds for NPD, below we introduce the construction of a new WPTS $\Pi_\calU$ based on the original WPTS $\Pi$ and a measurable set $\calU\in \Sigma_{\Rset^{|\pvars|}}$.  

\paragraph{Construction of $\Pi_\calU$.} Consider a probabilistic program $P$ and its WPTS $\Pi$, given a measurable set $\calU\in\Sigma_{\Rset^{|\pvars|}}$, we construct a new program $P_\calU$ by adding a conditional branch of the form ``\textbf{if} $\pv_T\notin\calU$ \textbf{then} \textbf{score}($0$) \textbf{fi}'' immediately after the termination of $P$ and obtain the WPTS $\Pi_\calU$ of $P_\calU$. Therefore, $\Pi$ and $\Pi_\calU$ have the same initial probability distribution $\mu_{\mathrm{init}}$ and the same finite support $\calV=\supp{\mu_{\mathrm{init}}}$. The following proposition shows that interval-bound analysis for NPD can be reduced to interval-bound analysis for expected weights in the form $\llbracket \Pi\rrbracket_{\pv}(\Rset^{|\pvars|})$. 

\begin{proposition}\label{prop:unnorm-norm}
   Given a WPTS $\Pi$ in the form of \eqref{eq:wpts}, a measurable set $\calU\in\Sigma_{\Rset^{|\pvars|}}$ and the WPTS $\Pi_\calU$ constructed as above, we have that $\llbracket \Pi \rrbracket_{\pv}(\calU)=\llbracket \Pi_\calU\rrbracket_{\pv}(\Rset^{|\pvars|})$ for any $\pv\in\calV=\supp{\mu_{\mathrm{init}}}$. Furthermore,
   if there exist intervals $[l_1,u_1],[l_2,u_2]\subseteq [0,\infty)$ such that $\llbracket \Pi_\calU\rrbracket_{\pv}(\Rset^{|\pvars|})\in [l_1,u_1]$ and $\llbracket \Pi\rrbracket_{\pv}(\Rset^{|\pvars|})\in [l_2,u_2 ]$ for any $\pv\in\calV$, then we have two intervals $[l_\calU,u_\calU],[l_Z,u_Z]\subseteq [0,\infty)$ such that the unnormalised posterior distribution $\llbracket \Pi\rrbracket (\calU)\in [l_\calU,u_\calU]$ and the normalising constant $Z_\Pi\in [l_Z,u_Z]$. Moreover, if $\Pi$ is integrable, i.e., $[l_Z,u_Z]\subseteq (0,\infty)$, then we can obtain the NPD $\posterior_{\Pi}(\calU)\in [\frac{l_\calU}{u_Z},\frac{u_\calU}{l_Z}]$.\footnote{The interval bounds derived in this manner may be loose, but they are definitely correct.}  Note that by \cref{def:npd}, $l_\calU=\int_\calV l_1 \cdot\mu_{\mathrm{init}}(\mathrm{d} \pv)$, $u_\calU=\int_\calV u_1 \cdot\mu_{\mathrm{init}}(\mathrm{d} \pv)$, $l_Z=\int_\calV l_2 \cdot\mu_{\mathrm{init}}(\mathrm{d} \pv)$ and $u_Z=\int_\calV u_1 \cdot\mu_{\mathrm{init}}(\mathrm{d} \pv)$.

\end{proposition}

The proof of \cref{prop:unnorm-norm} is relegated to \cref{app:sec2-prop}. In the following, we will develop approaches to obtain interval bounds for expected weights.
%in the form $\llbracket \Pi \rrbracket_{\pv}(\Rset^{|\pvars|})$ where $\pv$ is an initial program valuation.











\section{\pa}
We present our results for \pa in this section. First, we show that \pa is strongly \NPC by reducing it from \vc, which is known to be \NPC even for $3$-regular graphs~\cite[folklore]{DBLP:journals/dm/FleischnerSS10}. Hence, we do not expect a pseudo-polynomial time algorithm for \pa, unlike \kp.

\begin{definition}[\vc]Given a graph $\GG=(\VV,\EE)$ and a positive integer $k$, compute if there exists a subset $\VV' \subseteq \VV$ such that at least one end-point of every edge belongs to $\VV'$ and $|\VV'|\leq k$. We denote an arbitrary instance of \vc by $(\GG,k)$.
\end{definition}

\begin{theorem}\label{thm:pa-gen-npc}
\pa is strongly \NPC even when the maximum degree of the input graph is four.
\end{theorem}

\begin{proof}
    Clearly, \pa $\in$ \NP. We reduce \vc to \pa to prove NP-hardness. Let $(\GG=(\VV=\{v_i: i\in[n]\},\EE),$ $k)$ be an arbitrary instance of \vc where \GG is $3$-regular. We construct the following instance $(\GG^\pr=(\VV^\pr,\EE^\pr),(w(u))_{u\in\VV},(\alpha(u))_{u\in\VV},s,d)$ of \pa.
    \begin{align*}
        &\VV^\pr = \{u_i, g_i: i\in[n]\} \cup \{h_e: e\in \EE\}\\
        &\EE^\pr = \{\{u_i,h_e\}: i\in[n], e \in\EE, e\text{ is incident on }v_i\text{ in }\GG\} \\
        &\cup \{\{u_i,g_i\}: i\in[n]\} \cup \{\{g_i,g_{i+1}\}:i \in [n-1]\}\\
        &w(u_i) = 1, \alpha(u_i)=0, w(g_i)=0, \alpha(g_i)=0, \forall i\in[n]\\
        &w(h_e)=0, \alpha(h_e)=1, \forall e\in\EE, s=k, d=|\EE|
    \end{align*}
    We observe that the maximum degree of $\GG^\pr$ is at most four --- (i) the degree of $u_i$ is four for every $i\in[n]$, since \GG is $3$-regular and $u_i$ has an edge to $g_i$, (ii) the degree of $h_e$ is two for every $e\in\EE$, and (iii) the degree of $g_i$ is at most three for every $i\in[n]$ since the set $\{g_i:i\in[n]\}$ induces a path. We claim that the two instances are equivalent.

    In one direction, let us suppose that the \vc instance is a \yes instance. Let $\WW\subseteq\VV$ be a vertex cover of \GG with $|\WW|\le k$. We consider $\UU=\{u_i: i\in[n],v_i\in\WW\}\cup\{g_i: i\in[n]\}\cup\{h_e: e\in\EE\}\subseteq \VV^\pr$. We claim that $\GG^\pr[\UU]$ is connected. Since $\{g_i:i\in[n]\}$ induces a path and there is an edge between $u_i$ and $g_i$ for every $i\in[n]$, the induced subgraph $\GG^\pr[\{u_i: i\in[n],v_i\in\WW\}\cup\{g_i: i\in[n]\}]$ is connected. Since \WW is a vertex cover of \GG, every edge $e\in\EE$ is incident on at least one vertex in \WW. Hence, every vertex $h_e, e\in\EE,$ has an edge with at least one vertex in $\{u_i: i\in[n],v_i\in\WW\}$ in the graph $\GG^\pr$. Hence, the induced subgraph $\GG^\pr[\UU]$ is connected. Now we have $w(\UU)=\sum_{i=1}^n w(u_i)\mathbbm{1}(u_i\in\UU) + \sum_{i=1}^n w(g_i) + \sum_{e\in\EE} w(h_e)=|\UU|\le k$. We also have $\alpha(\UU)=\sum_{i=1}^n \alpha(u_i)\mathbbm{1}(u_i\in\UU) + \sum_{i=1}^n \alpha(g_i) + \sum_{e\in\EE} \alpha(h_e)=|\EE|$. Hence, the \pa instance is also a \yes instance.

    In the other direction, let us assume that the \pa instance is a \yes instance. Let $\UU\subseteq\VV^\pr$ be a solution of the \pa instance. We consider $\WW=\{v_i: i\in[n]: u_i\in\UU\}$. Since $s=k$, we have $|\WW|\le k$. Also, since $d=|\EE|$, we have $\{h_e: e\in\EE\}\subseteq\UU$. We claim that \WW is a vertex cover of \GG. Suppose not, then there exists an edge $e\in\EE$ which is not covered by \WW. Then none of the neighbors of $h_e$ belongs to \UU contradicting our assumption that \UU is a solution and thus $\GG^\pr[\UU]$ should be connected. Hence, \WW is a vertex cover of \GG and thus the \vc instance is a \yes instance.

    We observe that all the numbers in our reduced \pa instance are at most the number of edges of the graph. Hence, our reduction shows that \pa is strongly \NPC.
\end{proof}

 \Cref{thm:pa-gen-npc} also implies the following corollary in the framework of parameterized complexity.

\begin{corollary}\label{cor:pa-max-deg}
    \pa is \PNPH parameterized by the maximum degree of the input graph.
\end{corollary}

We next show that \pa is \NPC even for trees. For that, we reduce from the \NPC problem \kp.

\begin{definition}[\kp]
Given a set $\XX=[n]$ of $n$ items with sizes $\theta_1,\ldots,\theta_n,$ values $p_1,\ldots,p_n$, capacity $b$ and target value $q$, compute if there exists a subset $\II \subseteq [n]$ such that $\sum_{i\in \II} \theta_i \leq b$ and $\sum_{i\in \II} p_i \geq q$. We denote an arbitrary instance of \kp by $(\XX,(\theta_i)_{i\in\XX}, (p_i)_{i\in\XX}, b,q)$.
\end{definition}


\begin{theorem}\label{thm:pa-star-npc}
	\pa is \NPC even for star graphs.
\end{theorem}

\begin{proof}
\pa clearly belongs to \NP. To show \NP-hardness, we reduce from \kp. Let $(\XX=[n],(\theta_i)_{i\in\XX}, (p_i)_{i\in\XX}, b,q)$ be an arbitrary instance of \kp. We consider the following instance $(\GG(\VV,\EE),(w(u))_{u\in\VV},(\alpha(u))_{u\in\VV},s,d)$ of \pa.
\begin{align*}
    &\VV = \{v_0, v_1, \ldots, v_n\}\\
    &\EE = \{\{v_0,v_i\}: 1\le i\le n\}\\
    &w(v_i) = \theta_i\, \alpha(v_i) = p_i;\forall i\in[n], w(v_0)=\alpha(v_0)=0;\\
    &s = b, d = q
\end{align*}
We now claim that the two instances are equivalent.

In one direction, let us suppose that the \kp instance is a \yes instance. Let $\WW\subseteq\XX$ be a solution of \kp. Let us consider $\UU=\{v_i: i\in\WW\}\cup\{v_0\}\subseteq\VV$. We observe that $\GG[\UU]$ is connected since $v_0\in\UU$. We also have
\[ w(\UU) = \sum_{i\in\WW} w(v_i) = \sum_{i\in\WW} \theta_i \le b = s,\]
and
\[ \alpha(\UU) = \sum_{i\in\WW} \alpha(v_i) = \sum_{i\in\WW} p_i \ge q = d.\]
Hence, the \pa instance is a \yes instance.

In the other direction, let us assume that the \pa instance is a \yes instance with $\UU\subseteq\VV$ be one of its solution. Let us consider a set $\WW=\{i: i\in[n], v_i\in\UU\}$. We now have
\[ \sum_{i\in\WW} \theta_i = \sum_{i\in\WW} w(v_i) = w(\UU) \le s = b, \]
and
\[ \sum_{i\in\WW} p_i = \sum_{i\in\WW} \alpha(v_i) = \alpha(\UU) \ge d=q.\]
Hence, the \kp instance is a \yes instance.
\end{proof}


We complement the hardness result in \Cref{thm:pa-star-npc} by designing a pseudo-polynomial-time algorithm for \pa for trees. In fact, we have designed an algorithm with running time $2^{\OO(\tw\log \tw)}\cdot n^{\mathcal{O}(1)}\cdot {\sf min}\{s^2,d^2\}$ where the treewidth of the input graph is $\tw$. We present this algorithm next.

\subsection{Treewidth as a Parameter}
% \subsection{\pa on Bounded Treewidth Graphs}
%\subsection{Treewidth as Parameter}
% In this part, we study \pa parameterized by the treewidth of the input graph $G$. Note that since the problem is \NPC even for stars, we do not expect FPT algorithms parameterized by the treewidth of $G$. However, we design an algorithm with running time $f(k)\cdot p(n,s,d)$; here $f$ is a computable function on $k = tw(G)$, and $p$ is a polynomial dependent on $n = |V(G)|$, the target size $s$ and the target value $d$.

\begin{theorem}\label{thm:treewidth-pa}
There is an algorithm for \pa with running time $2^{\OO(\tw\log \tw)}\cdot n\cdot {\sf min}\{s^2,d^2\}$ where $n$ is the number of vertices in the input graph, $\tw$ is the treewidth of the input graph, $s$ is the input size of the knapsack and $d$ is the input target value.
\end{theorem}



\begin{proof}
Let $(G = (V_G,E_G),{(w(u))_{u \in V_G}, (\alpha(u))_{u\in V_G}}, s,d)$ be an input instance of \pa such that $\tw=tw(G)$. Let $\mathcal{U} \subseteq V_G$ be a solution subset for the input instance. For technical purposes, we guess a vertex $v \in \mathcal{U}$ --- once the guess is fixed, we are only interested in finding solution subsets $\mathcal{U}'$ that contain $v$ and $\mathcal{U}$ is one such candidate. We also consider a nice edge tree decomposition $(\mathbb{T} = (V_{\mathbb{T}},E_{\mathbb{T}}),\mathcal{X})$ of $G$ that is rooted at a node $r$, and where $v$ has been added to all bags of the decomposition. Therefore, $X_{r} = \{v\}$ and each leaf bag is the singleton set $\{v\}$.

We define a function $\ell: V_{\mathbb{T}} \rightarrow \mathbb{N}$.
%For the root $r$, $\ell(r) = 0$. 
For a vertex $t \in V_\mathbb{T}$, $\ell(t) = \sf{dist}_\mathbb{T}(t,r)$, where $r$ is the root. Note that this implies that $\ell(r) = 0$. Let us assume that the values that $\ell$ takes over the nodes of $\mathbb{T}$ is between $0$ and $L$. Now, we describe a dynamic programming algorithm over $(\mathbb{T},\mathcal{X})$ for \pa.


%%%%%%%%
    \paragraph{\textbf{States.}} We maintain a DP table $D$ where a state has the following components:
    \begin{enumerate}
    \item $t$ represents a node in $V_\mathbb{T}$.
    \item $\mathbb{P} = (P_0,\ldots,P_m), m\leq \tw+1$ represents a partition of the vertex subset $X_t$. 
    \end{enumerate}
%%%%%%
    \paragraph{\textbf{Interpretation of States.}} For a state $[t,\mathbb{P}]$, if there is a solution subset $\mathcal{U}$ let $\mathcal{U'} = \mathcal{U} \cap V(\beta(t))$. Let $\beta_{\mathcal{U'}}$ be the graph induced on $\mathcal{U'}$ in $\beta(t)$. Let $C_1,C_2,\ldots,C_m$ be the connected components of $\beta_{\mathcal{U'}}$. Note that $m\leq \tw+1$. Then in the partition $\mathbb{P} = (P_0,P_1,\ldots,P_m)$, $P_i = C_i \cap X_t, 1 \leq i \leq m$. Also, $P_0 = X_t \setminus \mathcal{U'}$. 

Given a node $t\in V_{\mathbb{T}}$, a subgraph $H$ of $\beta(t)$ is said to be a $\mathbb{P}$-subgraph if (i) the connected components $C_1,C_2,\ldots,C_m$ of $H$, $m\leq \tw+1$ are such that $P_i = C_i \cap X_t, 1 \leq i \leq m$, (ii) $P_0 = X_t \setminus H$. For each state $[t,\mathbb{P}]$, a pair $(w,\alpha)$ with $w\leq s$ is said to be feasible if there is a $\mathbb{P}$-subgraph of $\beta(t)$ whose total weight is $w$ and total value is $\alpha$. Moreover, a feasible pair $(w,\alpha)$ is said to be undominated if there is no other $\mathbb{P}$-subgraph with weight $w'$ and value $\alpha'$ such that $w' \leq w$ and $\alpha' \geq \alpha$. Please note that by default, an empty $\mathbb{P}$-subgraph has total weight $0$ and total value $0$.

For each state $[t,\mathbb{P}]$, we initialize $D[t,\mathbb{P}]$ to the list $\{(0,0)\}$. Our computation shall be such that in the end each $D[t,\mathbb{P}]$ stores the set of all undominated feasible pairs $(w,\alpha)$ for the state $[t,\mathbb{P}]$.

%%%%%%
    \paragraph{\textbf{Dynamic Programming on $D$.}} We describe the following procedure to update the table $D$. We start updating the table for states with nodes $t\in V_\mathbb{T}$ such that $\ell(t)=L$. When all such states are updated, then we move to update states where the node $t$ has $\ell(t) = L-1$, and so on till we finally update states with $r$ as the node --- note that $\ell(r) =0$. For a particular $i$, $0\leq i< L$ and a state $[t,\mathbb{P}]$ such that $\ell(t) = i$, we can assume that $D[t',\mathbb{P}']$ have been evaluated for all $t'$ such that $\ell(t')>i$ and all partitions $\mathbb{P}'$ of $X_{t'}$. Now we consider several cases by which $D[t,\mathbb{P}]$ is updated based on the nature of $t$ in $\mathbb{T}$:
    \begin{enumerate}
    \item Suppose $t$ is a leaf node. Note that by our modification, $X_t = \{v\}$. There can be only 2 partitions for this singleton set --- $\mathbb{P}_t^1 = (P_0 = \emptyset, P_1 = \{v\})$ and $\mathbb{P}_t^2 = (P_0 = \{v\}, P_1 = \emptyset)$. If $\mathbb{P} = \mathbb{P}_t^1$ then $D[t,\mathbb{P}]$ stores the pair $(w(v),\alpha(v))$ if $w(v) \leq s$ and otherwise no modification is made. If $\mathbb{P} = \mathbb{P}_t^2$ then $D[t,\mathbb{P}]$ is not modified.
    
    \item Suppose $t$ is a forget vertex node. Then it has an only child $t'$ where $X_t \subset X_{t'}$ and there is exactly one vertex $u \neq v$ that belongs to $X_{t'}$ but not to $X_t$. Let $\mathbb{P}' = (P'_0,P'_1,\ldots,P'_{m'})$ be a partition of $X_{t'}$ such that when restricted to $X_t$ we obtain the partition $\mathbb{P} = (P_0,P_1,\ldots,P_m )$. For each such partition, we shall do the following.\\ Suppose $\mathbb{P}'$ has $u \in P'_0$, then each feasible undominated pair stored in $D[t',\mathbb{P}']$ is copied to $D[t,\mathbb{P}]$. \\
    Alternatively, suppose $\mathbb{P}'$ has $u \in P'_i, i>0$ and $|P'_i| >1$. Then, each feasible undominated pair stored in $D[t',\mathbb{P}']$ is copied to $D[t,\mathbb{P}]$.\\
    Finally, suppose $\mathbb{P}'$ has $u \in P'_i, i>0$ and $P'_i = \{u\}$. Then we do not make any changes to $D[t,\mathbb{P}]$.
    
    \item Suppose $t$ is an introduce node. Then it has an only child $t'$ where $X_{t'} \subset X_{t}$ and there is exactly one vertex $u \neq v$ that belongs to $X_{t}$ but not $X_{t'}$. Note that no edges incident to $u$ have been introduced yet, and so in $\beta(t)$ $u$ is not yet connected to any other vertex. Let $\mathbb{P}' = (P'_0,P'_1,\ldots,P'_{m'})$ be the partition of $X_{t'}$ obtained from restricting the partition $\mathbb{P} = (P_0,P_1,\ldots,P_m )$ to $X_{t'}$. First, suppose $u \in P_0$. Then we copy all pairs of $D[t',\mathbb{P}']$ to $D[t,\mathbb{P}]$. \\
    Next, suppose $u \in P_i, i>0$ and $P_i = \{u\}$. Then for each pair $(w,\alpha)$ in $D[t',\mathbb{P}']$, if $w + w(u) \leq s$ we add $(w+w(u),\alpha+\alpha(u))$ to the set in $D[t,\mathbb{P}]$.\\
    Finally, let $u \in P_i, i>0$ and $|P_i| >1 $. Then we make no changes to $D[t,\mathbb{P}]$.
    
    \item Suppose $t$ is an introduce edge node. Then it has an only child $t'$ where $X_{t'} = X_{t}$, and additionally for two vertices $u,w \in X_t = X_{t'}$, the edge $\{u,w\}$ is introduced into $\beta(t)$. First, suppose $\mathbb{P} = (P_0,P_1,\ldots,P_m)$ is such that one of $u,w$ is in $P_0$. Then all pairs of $D[t',\mathbb{P}]$ are copied to $D[t,\mathbb{P}]$.\\
    Next, let $u \in P_i$, $w\in P_j$, $i\neq j \neq 0$. Then no updates are made to $D[t,\mathbb{P}]$. \\
    Finally, suppose $u,w \in P_i, i>0$. Copy to $D[t,\mathbb{P}]$ all pairs from $D[t',\mathbb{P}]$. Consider a partition $\mathbb{P}'$ where the part $P'$ contains $u$ and $P''$ contains $w$. $\mathbb{P}'$ is such that $P_i = P' \cup P''$ and any other $P_j$ with $j \neq i$ is a part in $\mathbb{P}'$. Any pair of $D[t',\mathbb{P}']$ is copied to $D[t,\mathbb{P}]$.
    \item Suppose $t$ is a join node. Then it has two children $t_1,t_2$ such that $X_t = X_{t_1} = X_{t_2}$. Consider $\mathbb{P} = (P_0,P_1,\ldots,P_m)$. Let $(w_{\mathbb{P}}, \alpha_{\mathbb{P}})$ be the total weight and value of the vertices in $\cup_{1\leq i \leq m} P_i$. Consider a pair $(w_1,\alpha_1)$ in $D[t_1,\mathbb{P}]$ and a pair $(w_2,\alpha_2)$ in $D[t_2,\mathbb{P}]$. Suppose $w_1 + w_2 - w_{\mathbb{P}} \leq s$. Then we add $(w_1 + w_2 - w_{\mathbb{P}}, \alpha_1+\alpha_2-\alpha_{\mathbb{P}})$ to $D[t,\mathbb{P}]$.
    \end{enumerate}
    
Finally, in the last step of updating $D[t,\mathbb{P}]$, we go through the list saved in $D[t,\mathbb{P}]$ and only keep undominated pairs.

The output of the algorithm is a pair $(w,\alpha)$ stored in $D[r,\mathbb{P} = (P_0 = \emptyset, P_1 = \{v\})]$ such that $w \leq s$ and $\alpha$ is the maximum value over all pairs in $D[r,\mathbb{P}]$.


%%%%%%%
    \paragraph{\textbf{Correctness of the Algorithm.}}

    Recall that we are looking for a solution $\mathcal{U}$ that contains the fixed vertex $v$ that belongs to all bags of the tree decomposition. First, we show that a pair $(w,\alpha)$ belonging to $D[t,\mathbb{P}]$ for a node $t \in V_\mathbb{T}$ and a partition $\mathbb{P}$ of $X_t$ corresponds to a $\mathbb{P}$-subgraph $H$ in $\beta(t)$. Recall that $X_r = \{v\}$. Thus, this implies that a pair $(w,\alpha)$ belonging to $D[r,\mathbb{P} = (P_0 = \emptyset, P_1 = \{v\})]$ corresponds to a connected subgraph of $G$. Moreover, the output is a pair that is feasible and with the highest value. 

    In order to show that a pair $(w,\alpha)$ belonging to $D[t,\mathbb{P}]$ for a node $t \in V_\mathbb{T}$ and a partition $\mathbb{P}$ of $X_t$ corresponds to a $\mathbb{P}$-subgraph $H$ in $\beta(t)$, we need to consider the cases of what $t$ can be:
    \begin{enumerate}
    \item When $t$ is a leaf node with $\ell(t) = i$, $X_t$ only contains $v$ and the update to $D$ is done such that $v$ is the corresponding subgraph to a stored pair. This is true in particular when $i =L$, the base case. From now we can assume that for a node $t$ with $\ell(t) = i < L$ all $D[t',\mathbb{P}']$ entries are correct and correspond to $\mathbb{P}'$-subgraphs in $\beta(t')$ when $\ell(t') > i$.
    \item When $t$ is a forget vertex node, let $t'$ be the child node and $u \neq v$ be the vertex that is being forgotten. We copy pairs from $D[t',\mathbb{P}']$ depending on the structure of $\mathbb{P}'$. Since $\ell(t') > \ell(t)$, by induction hypothesis all entries in $D[t',\mathbb{P}']$ for any partition $\mathbb{P}'$ of $X_{t'}$ are feasible. From the cases considered, we copy a pair to $D[t,\mathbb{P}]$ from a $D[t',\mathbb{P}']$ only when $u$ is not part of the $\mathbb{P}'$-subgraph or is in a component of $\mathbb{P}'$ that has vertices in $X_t$. Thus, the same subgraph is a $\mathbb{P}$-subgraph in $\beta(t)$.
    \item When $t$ is an introduce node, there is a child $t'$ we are introducing a vertex $u \neq v$ that has no adjacent edges added in $\beta(t)$. Since $\ell(t') > \ell(t)$, by induction hypothesis all entries in $D[t',\mathbb{P}']$ for any partition $\mathbb{P}'$ of $X_{t'}$ are feasible. We update pairs in $D[t,\mathbb{P}]$ from $D[t',\mathbb{P}']$ such that either $u$ is not considered as part of a $\mathbb{P}$-subgraph and the pair is certified by the $\mathbb{P}'$-subgraph, or $u$ is added to a $\mathbb{P}'$-subgraph in order to obtain a new $\mathbb{P}$-subgraph.
    \item When $t$ is an introduce edge node, there is a child $t'$ such that $X_t = X_{t'}$ and the only difference is that two vertices $u,w$ in the bags $X_t = X_{t'}$ now have an edge in $\beta(t)$. Since $\ell(t') > \ell(t)$, by induction hypothesis all entries in $D[t',\mathbb{P}']$ for any partition $\mathbb{P}'$ of $X_{t'}$ are feasible. The updates are made in the cases when one of $u$ or $w$ is not in the intended $\mathbb{P}$-subgraph and the included pair is certified by a $\mathbb{P'}$-subgraph, or when the $u$ and $w$ belong to different components of a $\mathbb{P}'$-subgraph and the new $\mathbb{P}$-subgraph has these components merged as a single component.
    \item When $t$ is a join node, there are two children $t_1,t_2$ such that $X_T = X_{t_1} = X_{t_2}$. Note that this implies that $V(\beta(t_1)) \cap V(\beta(t_2)) = X_t$. Since $\ell(t_1),\ell(t_2) > \ell(t)$, by induction hypothesis all entries in $D[t_i,\mathbb{P}']$ for any partition $\mathbb{P}'$ of $X_{t_i}$ are feasible for $i \in \{1,2\}$.We update pairs in $D[t,\mathbb{P}]$ when there is a $\mathbb{P}$-subgraph in $\beta(t_1)$ and a $\mathbb{P}$-subgraph in $\beta(t_2)$ and we take the union of these two subgraphs to obtain a $\mathbb{P}$-subgraph in $\beta(t)$.
    \end{enumerate}
    Thus in all cases of $t$, a pair added to $D[t,\mathbb{P}]$ for some partition of $X_t$ is a feasible pair. Recall that as a last step of updating $D[t,\mathbb{P}]$, we go through the entire list and keep only  undominated pairs in the list.

    What remains to be shown is that an undominated feasible solution $\mathcal{U}$ of \pa in $G$ is contained in $D[r,\mathbb{P} = (P_0 = \emptyset, P_1 = \{v\})]$. Let $w$ be the weight of $\mathcal{U}$ and $\alpha$ be the value. Recall that $v \in \mathcal{U}$. For each $t$, we consider the subgraph $\beta(t) \cap \mathcal{U}$. Let $C_1,C_2,\ldots,C_m$ be components of $\beta(t) \cap \mathcal{U}$ and let for each $1\leq i \leq m, P_i = X_t \cap C_i$. Also, let $P_0 = X_t \setminus \mathcal{U}$. Consider $\mathbb{P} = (P_0,P_1,\ldots,P_m)$. The algorithm updates in $D[t,\mathbb{P}]$ the pair $(w',\alpha')$ for the subsolution $\beta(t) \cap \mathcal{U}$. Therefore, $D[r,\mathbb{P} = (\emptyset,\{v\})]$ contains the pair $(w,\alpha)$. Thus, we are done.
%%%%%%
    \paragraph{\textbf{Running time.}} There are $n$ choices for the fixed vertex $v$. Upon fixing $v$ and adding it to each bag of $(\mathbb{T}, \mathcal{X})$ we consider the total possible number of states.  There are at most $\OO(n)\cdot 2^{\tw\log \tw}$ states. For each state, since we are keeping only undominated pairs, for each $w$ there can be at most one pair with $w$ as the first coordinate; similarly, for each $\alpha$ there can be at most one pair with $\alpha$ as the second coordinate. Thus, the number of undominated pairs in each $D[t,\mathbb{P}]$ is at most ${\sf min}\{s,d\}$. By the description of the algorithm, the maximum length of the list stored at $D[t,\mathbb{P}]$ during updation, but before the check is made for only undominated pairs, is $2^{\tw \log \tw}\cdot{\sf min}\{s^2,d^2\}$. Thus, updating the DP table at any vertex takes $2^{\tw \log \tw}\cdot{\sf min}\{s^2,d^2\}$ time. Since there are $\OO(n\cdot\tw)$ vertices in $\mathbb{T}$, the total running time of the algorithm is $2^{\OO(\tw\log \tw)}\cdot n\cdot {\sf min}\{s^2,d^2\}$.
\end{proof}


\subsection{A Fixed Parameter Fully Pseudo-polynomial Time Approximation Scheme}

We now use the algorithm in \Cref{thm:treewidth-pa} as a black-box to design an $(1-\eps)$-factor approximation algorithm for optimizing the value of the solution and running in time $2^{\tw\log \tw}\cdot \text{poly}(n,1/\eps)$.

\begin{theorem}\label{thm-fptas}
There is an $(1-\eps)$-factor approximation algorithm for \pa for optimizing the value of the solution and running in time $2^{\OO(\tw\log \tw)}\cdot \text{poly}(n,1/\eps)$ where \tw is the treewidth of the input graph.
\end{theorem}

\begin{proof}
Let $\II=(\GG=(\VV,\EE),(w(u))_{u\in\VV},(\alpha(u))_{u\in\VV},s)$ be an arbitrary instance of \pa where the goal is to output a connected subgraph \UU of maximum $\alpha(\UU)$ subject to the constraint that $w(\UU)\le s$. Without loss of generality, we can assume that $w(u)\le s$ for every $u\in\VV$. If not, then we can remove every $u\in\VV$ whose $w(u)>s$; this does not affect any solution since any vertex deleted can never be part of any solution. Let $\alpha_{\text{max}}=\max\{\alpha(u): u\in\VV\}$. We construct another instance $\II^\pr=\left(\GG=(\VV,\EE),(w(u))_{u\in\VV},(\alpha^\pr(u)=\left\lfloor \frac{n\alpha(u)}{\eps\alpha_{\text{max}}}\right\rfloor)_{u\in\VV},s\right)$ of \pa. We compute the optimal solution $\WW^\pr\subseteq\VV$ of $\II^\pr$ using the algorithm in \Cref{thm:treewidth-pa} and output $\WW^\pr$. Let $\WW\subseteq\VV$ be an optimal solution of $\II$. Clearly $\WW^\pr$ is a valid (may not be optimal) solution of \II also, since $w(\WW^\pr)\le s$ by the correctness of the algorithm in \Cref{thm:treewidth-pa}. We now prove the approximation factor of our algorithm.
\begin{align*}
     \sum_{u\in\WW^\pr}\alpha(u) &\ge \frac{\eps\alpha_{\text{max}}}{n}\sum_{u\in\WW^\pr} \left\lfloor \frac{n\alpha(u)}{\eps\alpha_{\text{max}}}\right\rfloor\\
    &\ge \frac{\eps\alpha_{\text{max}}}{n}\sum_{u\in\WW} \left\lfloor \frac{n\alpha(u)}{\eps\alpha_{\text{max}}}\right\rfloor&\text{[since $\WW^\pr$ is an optimal solution of $\II^\pr$]}\\
    &\ge \frac{\eps\alpha_{\text{max}}}{n}\sum_{u\in\WW} \left( \frac{n\alpha(u)}{\eps\alpha_{\text{max}}}-1\right)\\
    &\ge \left(\sum_{u\in\WW}\alpha(u)\right)-\eps\alpha_{\text{max}}\\
    &\ge \OPT(\II)-\eps\OPT(\II) & \text{[$\alpha_{\text{max}}\le\OPT(\II)$]}\\
    &=(1-\eps)\OPT(\II)
\end{align*}
Hence, the approximation factor of our algorithm is $(1-\eps)$. We now analyze the running time of our algorithm.

The value of any optimal solution of $\II^\pr$ is at most
\[\sum_{u\in\VV}\alpha^\pr(u) \le \frac{n}{\eps\alpha_{\text{max}}}\sum_{u\in\VV}\alpha(u)\le \frac{n}{\eps\alpha_{\text{max}}}\sum_{u\in\VV}\alpha_{\text{max}}=\frac{n^2}{\eps}. \]
Hence, the running time of our algorithm is $2^{\tw\log \tw}\cdot \text{poly}(n,1/\eps)$.
\end{proof}

\subsubsection{Other Parameters}

We next consider $vcs$, the maximum size of a minimum vertex cover of the subgraph induced by any solution of \pa, as our parameter. That is, $vcs(\II=(\GG,(w(u))_{u\in\VV},(\alpha(u))_{u\in\VV},s,d) = \max\{\text{size of minimum vertex cover of }W: W\subseteq\VV\text{ is a solution of }\II\}.$ We already know from \Cref{thm:pa-star-npc} that \pa is \NPC for star graphs. We note that $vcs$ is one for star graphs. Hence, \pa is \PNPH with respect to $vcs$, that is, there is no algorithm for \pa which runs in polynomial time even for constant values of $vsc$. However, whether there exists any algorithm with running time $\OO(f(vcs).\text{poly}(n,s,d))$, remains a valid question. We answer this question negatively in \Cref{thm:vcs-woh}. For that, we exhibit an \FPT-reduction from \pvc which is known to be \WOH parameterized by the size of the partial vertex cover we are looking for.


\begin{definition}[\pvc]
Given a graph $\GG=(\VV,\EE)$ and two integers $k$ and $\el$, compute if there exists a subset $\VV^\pr\subseteq\VV$ such that (i) $|\VV^\pr|\le k$ and (ii) there exist at least $\el$ edges whose one or both end points belong to $\VV^\pr$. We denote an arbitrary instance of \pvc by $(\GG,k,\el)$.
\end{definition}

% We know that \pvc is \WOH parameterized by $k$. 

\begin{theorem}\label{thm:vcs-woh}
There is no algorithm for \pa running in time $\OO(f(vcs).\text{poly}(n,s,d))$ unless \ETH fails.
\end{theorem}




\begin{proof}
At a high-level, our reduction from \pvc is similar to the reduction in \Cref{thm:pa-gen-npc}. The only difference is that we will now have only one ``global vertex'' instead of the ``global path'' in the proof of \Cref{thm:pa-gen-npc}. Formally, our reduction is as follows.

Let $(\GG=(\VV,\EE),k,\el)$ be an arbitrary instance of \pvc. We consider the following instance $(\GG^\pr=(\VV^\pr,\EE^\pr),(w(u))_{u\in\VV^\pr},(\alpha(u))_{u\in\VV^\pr},s,d)$ of \pa.
\begin{align*}
    &\VV^\pr = \{u_i: i\in[n]\} \cup \{h_e: e\in \EE\}\cup\{g\}\\
    &\EE^\pr = \{\{u_i,h_e\}: i\in[n], e \in\EE, e\text{ is incident on }v_i\text{ in }\GG\}
    \cup \{\{u_i,g\}: i\in[n]\}\\
    &w(u_i) = 1, \alpha(u_i)=0, \forall i\in[n], 
    w(h_e)=0, \alpha(h_e)=1, \forall e\in\EE,\\
    &w(g)=0, \alpha(g)=0, s=k, d=\el
\end{align*}
We claim that the two instances are equivalent.

In one direction, let us suppose that the \pvc instance is a \yes instance. Let $\WW\subseteq\VV$ covers at least \el edges in \GG and $|\WW|\le k$. We consider $\UU=\{u_i: i\in[n],v_i\in\WW\}\cup\{h_e: e\in\EE, e\text{ is covered by }\WW\}\cup\{g\}\subseteq \VV^\pr$. We claim that $\GG^\pr[\UU]$ is connected. Since there is an edge between $u_i$ and $g$ for every $i\in[n]$, the induced subgraph $\GG^\pr[\{u_i: i\in[n],v_i\in\WW\}\cup\{g\}]$ is connected. Also, since $h_e$ belongs to \UU only if \WW covers $e$ in \GG, the induced subgraph $\GG^\pr[\UU]$ is connected. Now we have $w(\UU)=\sum_{i=1}^n w(u_i)\mathbbm{1}(u_i\in\UU) + w(g) + \sum_{e\in\EE, \WW\text{ covers }e} w(h_e)=|\WW|\le k$. We also have $\alpha(\UU)=\sum_{i=1}^n \alpha(u_i)\mathbbm{1}(u_i\in\UU) + \alpha(g) + \sum_{e\in\EE, \WW\text{ covers }e} \alpha(h_e)\ge\el$. Hence, the \pa instance is also a \yes instance.

In the other direction, let us assume that the \pa instance is a \yes instance. Let $\UU\subseteq\VV^\pr$ be a solution of the \pa instance. We consider $\WW=\{v_i: i\in[n]: u_i\in\UU\}$. Since $s=k$, we have $|\WW|\le k$. Also, since $d=\el$, we have $|\{h_e: e\in\EE\}\cap\UU|\ge\el$. We claim that \WW covers every edge in $\{e: e\in\EE, h_e\in\UU\}$. Suppose not, then there exists an edge $e\in\{e: e\in\EE, h_e\in\UU\}$ which is not covered by \WW. Then none of the neighbors of $h_e$ belongs to \UU contradicting our assumption that \UU is a solution and thus $\GG^\pr[\UU]$ should be connected. Hence, \WW covers at least \el edges in \GG and thus the \pvc instance is a \yes instance.

We also observe that the size of a minimum vertex cover of the subgraph induced by any solution of the reduced \pa instance is at most $k+1$ --- the set consisting of the vertices in any solution of the \pa instance in $\GG^\pr$ which has weight $1$ and $g$ forms a vertex cover of the subgraph induced by the solution of the \pa instance. Also, all the numbers in our reduced \pa instance are at most the number of edges of the graph, and \pvc is \WOH parameterized by $k$. Hence, \pa is \WOH parameterized by the maximum size of the minimum vertex cover of any solution even when all the numbers are encoded in unary. Therefore, there is no algorithm for \pa running in time $\OO(f(vcs).\text{poly}(n,s,d))$ unless \ETH fails.
\end{proof}
\section{Results for \pathknapsack}

We now consider \pathknapsack. We show that \pathknapsack is strongly \NPC by reducing from \hp which is defined as follows.

\begin{definition}[\hp]
Given a graph $\GG(\VV,\EE)$ and two vertices $u$ and $v$, compute if there exists a path between $u$ and $v$ which visits every other vertex in \GG. We denote an arbitrary instance of \hp by $(\GG,u,v)$.
\end{definition}

\hp is known to be \NPC even for graphs with maximum degree three~\cite{DBLP:conf/stoc/GareyJS74}.

\begin{theorem}\shortversion{[$\star$]}\label{thm:path-gen-npc}
\pathknapsack is strongly \NPC even for graphs with maximum degree three.
\end{theorem}
\longversion{
\begin{proof}
\pathknapsack clearly belongs to \NP. To show \NP-hardness, we reduce from \hp. Let $(\GG,u,v)$ be an arbitrary instance of \hp. We consider the following instance $(\GG,(w_u)_{u\in\VV},(\alpha_u)_{u\in\VV},s,d,u,v)$ of \pathknapsack.
\[w_u=0,\alpha_u=1~\forall u\in\VV,s=0,d=n\]
We claim that the two instances are equivalent.

In one direction, let us assume that \hp is \yes. Let \PP be a Hamiltonian path between $u$ and $v$ in \GG. Then we have $w(\PP)=0\le s$ and $\alpha(\PP)=n\ge d$. Hence, the \pathknapsack instance is also a \yes instance.

In the other direction, let us assume that the \pathknapsack instance is a \yes instance. Let \PP be a path between $u$ and $v$ with $w(\PP)\le s=0$ and $\alpha(\PP)\ge d=n$. Since $\alpha(\VV)=n=\alpha(\PP)$ and there is no vertex with zero \alpha value, \PP must be a Hamiltonian path between $u$ and $v$. Hence, the \hp instance is also a \yes instance.

We observe that all the numbers in our reduced \pathknapsack instance are at most the number of vertices in the graph. Hence, our reduction shows that \pathknapsack is strongly \NPC.
\end{proof}
}
However, \pathknapsack is clearly polynomial-time solvable for trees, since there exist only one path between every two vertices in any tree.

\begin{observation}\label{obs:pk-tree-poly}
\pathknapsack is polynomial-time solvable for trees.
\end{observation}

One immediate natural question is if \Cref{obs:pk-tree-poly} can be generalized to graphs of bounded treewidth. The following result refutes the existence of any such algorithm.

\begin{theorem}\label{thm:pk-pathwidth}
\pathknapsack is \NPC even for graphs of pathwidth at most two. In particular, \pathknapsack is \PNPH parameterized by pathwidth.
\end{theorem}

\begin{proof}
We reduce from \kp to \pathknapsack. Let $(\XX=[n],(\theta_i)_{i\in\XX}, (p_i)_{i\in\XX}, b,q)$ be an arbitrary instance of \kp. We consider the following instance $(\GG(\VV,\EE),(w(u))_{u\in\VV},(\alpha(u))_{u\in\VV},s,d,u,v)$ of \pathknapsack.
\begin{align*}
\VV &= \{u_0\}\cup\{u_i,v_i,w_i: i\in[n]\}\\
\EE &= \{\{u_i,v_{i+1}\},\{u_i,w_{i+1}\}:0\le i\le n-1\}\cup\{\{u_i,v_{i}\},\{u_i,w_{i}\}:i\in[n]\}\\
w(v_i)&=\theta_i, \alpha(v_i)=p_i ~\forall i\in[n], w(u_i)=\alpha(u_i)=0~\forall i\in[n]_0, w(w_i)=\alpha(w_i)=0~\forall i\in[n]\\
s &= b, d=q, u=u_0, v=u_n
\end{align*}
\Cref{fig:enter-label} shows a schematic diagram of the reduced instance. We now claim that the two instances are equivalent.

% Figure environment removed

In one direction, let us suppose that the \kp instance is a \yes instance. Let $\II\subseteq\XX$ be such that $\sum_{i\in\II}\theta_i\le b$ and $\sum_{i\in\II}p_i\ge q$. We now consider the path \PP from $u_0$ to $u_n$ consisting of $2n$ edges where the $2i$-th vertex (from $u_0$) is $v_i$ if $i\in\II$ and $w_i$ otherwise for every $i\in[n]$; the $(2i-1)$-th vertex (from $u_0$) is $u_{i-1}$ for every $i\in[n]$ and the last vertex is $u_n$. Clearly, we have $w(\PP)=\sum_{i\in\II}\theta_i\le b=s$ and $\alpha(\PP)=\sum_{i\in\II}p_i\ge q=d$. Hence, the \pathknapsack instance is a \yes instance.

In the other direction, let us assume that the \pathknapsack instance is a \yes instance. Let \PP be a path between $u$ and $v$ with $w(\PP)\le s=b$ and $\alpha(\PP)\ge d=q$. We observe that all paths from $u_0$ to $u_n$ in \GG has $2n$ edges; $(2i-1)$-th vertex is $u_{i-1}$ and $2i$-th vertex is either $v_i$ or $w_i$ for every $i\in[n]$ and the last vertex is $u_n$. We consider the set $\II=\{i\in[n]: \text{ $(2i-1)$-th vertex is $v_i$}\}\subseteq\XX$. We now have $\sum_{i\in\II}\theta_i=w(\PP)\le s=b$ and $\sum_{i\in\II}p_i=\alpha(\PP)\ge d=q$. Hence, the \kp instance is a \yes instance.

We also observe that the pathwidth of \GG is at most two (there is a path decomposition with bag size being at most three) which proves the result.
\end{proof}

\Cref{thm:pk-pathwidth} leaves the following question open: does there exist an algorithm for \pathknapsack which runs in time $\OO(f(\tw)\cdot \text{poly}(n,s,d))$? We answer this question affirmatively in the following result. We omit its proof from this shorter version, since the algorithm is very similar to \Cref{thm:treewidth-pa}.

\begin{theorem}\label{thm:treewidth-path}
There is an algorithm for \pathknapsack with runtime $2^{\tw\log \tw}\cdot n^{\mathcal{O}(1)}\cdot {\sf min}\{s^2,d^2\}$ where $\tw$ is the treewidth of the input graph.
\end{theorem}

Using the technique in \Cref{thm-fptas}, we use \Cref{thm:treewidth-path} in a black-box fashion to have the following approximation algorithm. We again omit its proof due to its similarity with \Cref{thm-fptas}.


\begin{theorem}\label{thm-fptas-path}
There is an $(1-\eps)$ factor approximation algorithm for \pathknapsack for optimizing the value of the solution running in time $2^{\tw\log \tw}\cdot \text{poly}(n,1/\eps)$ where \tw is the treewidth of the input graph.
\end{theorem}

We next consider the size of the minimum vertex cover of the subgraph induced by the solution $\WW\subseteq\VV[\GG]$. We observe that the size of the minimum vertex cover of $\GG[\WW]$ is at least half of $|\WW|$ since there exists a Hamiltonian path in $\GG[\WW]$. Hence, it is enough to design an \FPT algorithm with parameter $|\WW|$. Our algorithm is based on color coding technique~\cite{DBLP:books/sp/CyganFKLMPPS15}.

\begin{theorem}\shortversion{[$\star$]}\label{thm:sol-path}
    There is an algorithm for \pathknapsack running in time $\OO\left((2e)^k k^{\OO(\log k)}n^{\OO(1)}\right)$ where $k$ is the number of vertices in the solution.
\end{theorem}
\longversion{
\begin{proof}
    Let $(\GG(\VV,\EE),(w(u))_{u\in\VV},(\alpha(u))_{u\in\VV},s,d,u,v)$ be an arbitrary instance of \pathknapsack and $k$ the number of vertices in the solution. We can assume without loss of generality that we know $k$ since there are only $n-1$ possible values of $k$ namely $2,3,\ldots,n$. We color every vertex uniformly randomly from a palette of $k$ colors independent of everything else. Let $\chi:\VV\longrightarrow[k]$ be the resulting coloring. For every non-empty subset $S\subseteq[k],S\ne\emptyset$ and vertex $x\in\VV$, we define a boolean variable PATH$(S,x)$ to be \true if there is a path which stars from $u$, ends at $x$, and contains exactly one vertex of every color in $S$; we call such a path $S$-colorful. If PATH$(S,x)$ is \true, then we also define $D[S,x]=\{(w,\alpha):\exists\text{ an $u$ to $x$ $S$-colorful path \PP such that }w(\PP)=w, \alpha(\PP)=\alpha\text{ for every other $u$ to $x$ $S$-colorful path \QQ, we have wither }w(\QQ)>w \text{ or }\alpha(\QQ)<\alpha\}$. For $|S|=1$, we note that PATH$(S,u)$=\true if and only if $\{\chi(u)\}=S$; PATH$(S,x)$=\false for every $x\in\VV\setminus\{u\}$ and $|S|=1$; $D[S,u]=\{(w(u),\alpha(u))\}$ if $\{\chi(u)\}=S$; $D[S,x]=\emptyset$ for every $x\in\VV\setminus\{u\}$ and $S\subseteq\VV$. We update $\text{PATH}(S,x)$ as per the following recurrence for $|S|>1$.
    \[
    \text{PATH}(S,x) = 
    \begin{cases}
        \bigvee\{\text{PATH}(S\setminus\{\chi(x)\},v): \{v,x\}\in\EE\} & \text{if } \chi(x)\in S\\
        \false & \text{otherwise}
    \end{cases}
    \]
    When we update any $\text{PATH}(S,x)$ to be \true, we update $D[S,x]$ as follows. For every $\{v,x\}\in\EE$ if $\text{PATH}(S\setminus\{\chi(x)\},v)$ is \true, then we do the following: for every $(w,\alpha)\in D_v$, we put $(w+w(x),\alpha+\alpha(x))$ in $D_x$ if $w+w(x)\le s$; and we finally remove all dominated pairs from $D_x$. We output \yes if $\text{PATH}([k],y)$ is \true and there exists a $(w,\alpha)\in D_y$ such that $w\le s$ and $\alpha\ge d$. Otherwise, we output \no.

    {\bf Proof of correctness:} If there does not exist any colorful path between $x$ and $y$ of length $k$, then the algorithm clearly outputs \no. Suppose now that the instance if a \yes instance. Then there exists a colorful path $\PP=(u_1(=x),u_2,\ldots,u_k(=y))$ between $x$ and $y$ such that $w(\PP)\le s$ and $\alpha(\PP)\ge d$. Let us define $S_i=\{u_j:j\in[i]\}$ for $i\in[k]$. Then $\text{PATH}(S_i,u_i)$ is \true and $D[S_i,u_i]$ either contains $(w(S_i),\alpha(S_i))$ or any pair which dominates $(w(S_i),\alpha(S_i))$ for every $i\in[k]$. Hence, the algorithm outputs \yes.

    {\bf Runtime analysis:} If there exists a colorful path between $x$ and $y$, then the algorithm finds it in time $\OO\left(2^k n^{\OO(1)}\right)$. If there exists a path between $x$ and $y$ containing $k-1$ edges (that is, $k$ vertices including $x$ and $y$), then one such path becomes colorful in the random coloring with probability at least
    \[ \frac{k!}{k^k}\ge e^{-k}. \]
    Hence, by repeating $\OO(e^k)$ times and outputting \yes if any run of the algorithm outputs \yes, the above algorithm achieves a success probability at least $2/3$. We can use the splitters to derandomize the algorithm above to obtain a deterministic algorithm for \pathknapsack which runs in time $\OO\left((2e)^k k^{\OO(\log k)}n^{\OO(1)}\right)$~\cite[Section 5.6.2]{DBLP:books/sp/CyganFKLMPPS15}.
\end{proof}
}
\Cref{thm:sol-path} immediately implies the following result.

\begin{corollary}\label{cor:vcs-path}
    There is an algorithm for \pathknapsack running in time $\OO\left((2e)^{2vcs} vcs^{\OO(\log vcs)}n^{\OO(1)}\right)$ where $vcs$ is the size of the minimum vertex cover of the subgraph induced by the solution.
\end{corollary}
\section{\shortestpathknapsack}

We now consider \shortestpathknapsack. We observe that all the $u$ to $v$ paths in the reduced instance of \pathknapsack in the proof of \Cref{thm:pk-pathwidth} are of the same length. Hence, we immediately obtain the following result as a corollary of \Cref{thm:pk-pathwidth}.

\begin{corollary}\label{cor:shortest-pathwidth}
\shortestpathknapsack is \NPC even for graphs of pathwidth at most two and the weight of every edge is one. In particular, \shortestpathknapsack is \PNPH parameterized by pathwidth.
\end{corollary}

Interestingly, \shortestpathknapsack admits a pseudo-polynomial-time algorithm for any graph unlike \pa and \pathknapsack.

\begin{theorem}\label{thm:shortest-algo}
There is an algorithm for \shortestpathknapsack running in time $\OO((m+n\log n)\cdot\min\{s^2,(\alpha(\VV))^2\})$, where $m$ is the number of edges in the input graph.
\end{theorem}

\begin{proof}
Let $(\GG=(\VV,\EE,(c(e))_{e\in\EE}),(w(u))_{u\in\VV},(\alpha(u))_{u\in\VV},s,d,x,y)$ be an arbitrary instance of \shortestpathknapsack. We design a greedy and dynamic-programming based algorithm. For every vertex $v\in \VV$, we store a boolean marker $b_v$, the distance $\delta_v$ of $v$ from $x$, and a set $D_v=\{(w,\alpha):\exists$ an $x$ to $v$ shortest-path \PP such that $w(\PP)=w, \alpha(\PP)=\alpha,$ and for every other $x$ to $v$ shortest-path \QQ, we have either $w(\QQ)>w$ or $\alpha(\QQ)<\alpha \text{ (or both)}\}$. That is, we store undominated weight-value pairs of all shortest $x$ to $v$ paths in $D_v$. We initialize $b_x=\false, \delta_x=0, D_x=\{(w(x),\alpha(x)\}, b_u=\false, \delta_u=\infty,$ and $ D_u=\emptyset$ for every $u\in\VV\setminus\{x\}$.

Updating DP table: We pick a vertex $z=\argmin_{v\in\VV:b_v=\false}\delta_v$. We set $b_z=\true$. For every neighbor $u$ of $z$, if $\delta_u>\delta_z+c(\{z,u\})$, then we reset $D_u=\emptyset$ and set $\delta_u=\delta_z+c(\{z,u\})$. If $\delta_u=\delta_z+c(\{z,u\})$, then update $D_u$ as follows: for every $(w,\alpha)\in D_z$, we update 
$D_z$ to $(D_z\cup\{(w+w(u),\alpha+\alpha(u))\})$ if $w+w(u)\le s$. We remove all dominated pairs from $D_u$ just before finishing each iteration. If we have $b_v=\true$ for every vertex, then we output \yes if there exists a pair $(w,\alpha)\in D_y$ such that $w\le s$ and $\alpha\ge d$. Else, we output \no.

We now argue the correctness of our algorithm. Following the proof of correctness of the classical Dijkstra's shortest path algorithm, we observe that if $b_v$ is $\true$ for any vertex $v\in\VV$, its distance from $x$ is $\delta_v$~\cite{DBLP:books/daglib/0023376}. We claim that at the end of updating a table entry in every iteration, the following invariant holds: for every vertex $v\in\VV$ such that $b_v=\true$, $(k_1,k_2)\in D_v$ if and only if there exists an $x$ to $v$ undominated shortest path $\PP$ using only vertices marked $\true$ such that $w(\PP)=k_1$ and $\alpha(\PP)=k_2$.

The invariant clearly holds after the first iteration. Let us assume that the invariant holds after $i\;(>\!\!\!\!1)$ iterations; $\VV_T$ be the set of vertices which are marked $\true$ after $i$ iterations. We have $|\VV_T|=i>1$. Suppose the algorithm picks the vertex $z_{i+1}$ in the $(i+1)$-th iterationl; that is, we have $z_{i+1}=\argmin_{v\in\VV:b_v=\false}\delta_v$ when we start the $(i+1)$-th iteration. Let $\PP^*=x,\ldots,z,z_{i+1}$ be an undominated $x$ to $z_{i+1}$ shortest path using the vertices marked $\true$ only. Then we need to show that $(w(\PP^*),\alpha(\PP^*))\in D_{z_{i+1}}$. We claim that the prefix of the path $\PP^*$ from $x$ to $z$, let us call it $\QQ=x,\ldots,z$, is an undominated $x$ to $z$ shortest path using the vertices marked \true only. It follows from the standard proof of correctness of Dijkstra's algorithm~\cite{DBLP:books/daglib/0023376} that $\QQ$ is a shortest path from $x$ to $z$. Now, to show that $\QQ$ is undominated, let us assume that another shortest path $\RR$ from $x$ to $z$ dominates $\QQ$. Then the shortest path $\RR^\pr$ from $x$ to $z_{i+1}$ which is $\RR$ followed by $z_{i+1}$ also dominates $\PP^*$ contradicting our assumption that $\PP^*$ is an undominated shortest path from $x$ to $z_{i+1}$. We now observe that the iteration $j$ when the vertex $z$ was marked \true must be less than $(i+1)$ since $z$ is already marked \true in the $(i+1)$-th iteration. Now, applying induction hypothesis after the $j$-th iteration, we had $(w(\QQ),\alpha(\QQ))\in D_z$ and $(w(\PP^*),\alpha(\PP^*))\in D_{z_{i+1}}$. Also, at the end of the $j$-th iteration, the $\delta_z$ and $\delta_{z_{i+1}}$ values are set to the distances of $z$ and $z_{i+1}$ from $x$ respectively. Thus, the $D_z$ and $D_{z_{i+1}}$ are never reset to $\emptyset$ after $j$-th iteration. Also, we never remove any undominated pairs from DP tables. Since $\PP^*$ is an undominated $x$ to $z_{i+1}$ path, we always have $(w(\PP^*),\alpha(\PP^*))\in D_{z_{i+1}}$ from the end of $j$-th iteration and thus, in particular, after $(i+1)$-th iteration. Hence, invariant (i) holds for $z_{i+1}$ after $(i+1)$-th iteration. Now consider any vertex $z^\pr$ other than $z_{i+1}$ which is marked \true after $(i+1)$-th iteration. Let $\PP_1=x,\ldots,z^\pr$ be an undominated $x$ to $z^\pr$ shortest path using the vertices marked \true only. If $\PP_1$ does not pass through the vertex $z_{i+1}$, then we have $(w(\PP_1),\alpha(\PP_1))\in D_{z_{z^\pr}}$ by induction hypothesis after $i$-iterations. If $\PP_1$ passes through $z_{i+1}$, we have $\delta_{z_{i+1}}<\delta_{z^\pr}$ since all the edge weights are positive. However, this contradicts our assumption that $z^\pr$ had already been picked by the algorithm and marked \true before $z_{i+1}$ was picked. Hence, $\PP_1$ cannot use $z_{i+1}$ as an intermediate vertex. Hence, after $(i+1)$ iterations, for every vertex $v\in\VV$ such that $b_v=\true$, for every $x$ to $v$ shortest path \PP using only vertices marked \true, we have $(w(\PP),\alpha(\PP))\in D_v$.


For the other direction, let $(k_1,k_2)$ be an arbitrary pair in $D_{z_{i+1}}$ after $(i+1)$ iterations. Let $j$ be the iteration when $(k_1,k_2)$ was first included in $D_{z_{i+1}}$; $z^\pr$ be the vertex that the algorithm picked and marked \true in the $j$-th iteration. Since $(k_1,k_2)$ was included in $D_{z_{i+1}}$ in the $j$-th iteration, we must have an edge between $z^\pr$ and $z_{i+1}$. From the proof of correctness of Dijkstra's algorithm, we observe that, after the end of the $j$-th iteration, $\delta_{z_{i+1}}$ is $\delta_{z^\pr}+c(\{z^\pr,z_{i+1}\})$ which is actually the distance of $z_{i+1}$ from $x$ and was thus never decreased after $j$-th iteration of the algorithm. Hence, $D_{z_{i+1}}$ is never reset to $\emptyset$ after the $j$-th iteration and thus the pair $(k_1,k_2)$ remains in $D_{z_{i+1}}$ from $j$-th iteration till $(i+1)$-th iteration. Also, it follows from the proof of correctness of Dijkstra's algorithm that there is a shortest path from $x$ to $z_{i+1}$ using vertices marked \true in the first $j$ iterations only. In particular, there exists an undominated shortest path $\PP_1$ from $x$ to $z_{i+1}$ using vertices marked \true in the first $(i+1)$ iterations only with $w(\PP_1)=k_1$ and $\alpha(\PP_1)=k_2$. Hence, the invariant holds for $z_{i+1}$ after $(i+1)$ iterations. For every other vertex $z^\prr\in\VV$ such that $z^\prr$ is marked \true, we have already argued before that we cannot have a shortest $x$ to $z^\prr$ path using $z_{i+1}$ as an intermediate vertex. Thus, the invariant holds for $z^\prr$ thanks to induction hypothesis.


For every vertex $v\in\VV$, the cardinality of $D_v$ is at most $s$ and also at most $\alpha(\VV)$ and thus at most $\min\{s,\alpha(\VV)\}$. Implementing the above algorithm using a standard Fibonacci heap-based priority queue to find $\argmin_{v\in\VV:b_v=\false}\delta_v$ gives us a running time $\OO((m+n\log n)\cdot\min\{s^2,(\alpha(\VV))^2\})$ where $m$ is the number of edges in the graph.
\end{proof}

Clearly, \shortestpathknapsack admits a polynomial-time algorithm for trees since only one path exists between every two vertices.

\begin{observation}\label{obs:shortest-tree-poly}
\shortestpathknapsack is in \Pb for trees.
\end{observation}

Using the technique in \Cref{thm-fptas}, we use \Cref{thm:shortest-algo} as a black-box to obtain the following approximation algorithm. We again omit its proof due to its similarity with \Cref{thm-fptas}.


\begin{theorem}\label{thm-fptas-shortest-path}
There is a $\text{poly}(n,1/\eps)$ time, $(1-\eps)$ factor approximation algorithm for \shortestpathknapsack for optimizing the value of the solution.
\end{theorem}
%\section{Results on \indpsetknapsack}

We now present our results for \indpsetknapsack. We show that \indpsetknapsack is strongly \NPC by reducing it from \indset which is defined as follows.

\begin{definition}[\indset]
Given a graph $\GG(\VV,\EE)$ and an integer $k$ compute if there exists a subset $\WW\subseteq\VV$ such that there is no edge whose both end points are in \WW and $|\WW|= k$. We denote an arbitrary instance of \indset by $(\GG,k)$.
\end{definition}

\begin{theorem}\label{thm:ind}
\indpsetknapsack is strongly \NPC.
\end{theorem}

\begin{proof}
\indpsetknapsack clearly belongs to \NP. To prove \NP-hardness, we reduce from \indset. Let $(\GG(\VV,\EE),k)$ be an arbitrary instance of \indset. We consider the instance $(\GG(\VV,\EE),(w(u))_{u\in\VV},(\alpha(u))_{u\in\VV},s,d)$ of \indpsetknapsack where
\begin{align*}
w(u)=\alpha(u)=1\;\forall u\in\VV, s=d=k
\end{align*}
We claim that the two instances are equivalent.

In one direction, let us assume that the \indset instance is a \yes instance. Let $\WW\subseteq\VV$ be an independent set of size $k$. Since $|\WW|=k$, we have $w(\WW)=\alpha(\WW)=k$. Hence, the \indpsetknapsack instance is also a \yes instance. In the other direction, let us assume that the \indpsetknapsack instance is a \yes instance. Then there exists a $\WW\subseteq\VV$ such that $w(\WW)\le k$ and $\alpha(\WW)\ge k$. This implies that we have $|\WW|\le k$ and $|\WW|\ge k$, and thus $|\WW|=k$. Moreover, \WW is an independent set in \GG. Hence, the \indset instance is a \yes instance.

We observe that all the numbers in our reduced \indpsetknapsack instance are at most the number of vertices in the graph. Hence, our reduction shows that \indpsetknapsack is strongly \NPC.
\end{proof}

We complement the hardness result of \Cref{thm:ind} by designing an algorithm for \indpsetknapsack running in time $\OO\left(2^{\tw}\cdot\text{poly}(\min\{s,d\})\right)^{\dagger^\star}$ following the idea of \Cref{thm:treewidth-pa} and the \FPT algorithm for \indset with respect to \tw~\cite{DBLP:books/sp/CyganFKLMPPS15}. We omit its proof from this shorter version due to its similarity with \Cref{thm:treewidth-pa}.

\begin{theorem}\label{thm:treewidth-indset}
There is an algorithm for \pa with runtime $2^{\tw}\cdot n^{\mathcal{O}(1)}\cdot {\sf min}\{s,d\}$ where $\tw$ is the treewidth of the input graph.
\end{theorem}

Using the technique in \Cref{thm-fptas}, we use \Cref{thm:treewidth-indset} in a black-box fashion to have the following approximation algorithm. We again omit its proof due to its similarity with \Cref{thm-fptas}.


\begin{theorem}\label{thm-fptas-indset}
There is an $(1-\eps)$ factor approximation algorithm for \pa for optimizing the value of the solution running in time $2^{\tw}\cdot \text{poly}(n,1/\eps)$ where \tw is the treewidth of the input graph.
\end{theorem}

We now consider the size $k$ of the solution as our parameter. We know that \indset is \WOH parameterized by the size of the independent set. A straight-forward reduction from \indset thus shows the following.

\begin{observation}\label{obs:ind-woh}
    \indpsetknapsack is \WOH parameterized by the size $k$ of the solution.
\end{observation}
\section{Conclusion}\label{sec:conclusion}

This paper presents our empirical domain knowledge distillation framework using ChatGPT and discusses our observations from the framework application experiments in the autonomous driving domain. The key finding is that: 1) with proper design of prompt engineering and execution flow, fully automated domain knowledge (in the ontology format) distillation is possible. However, due to the randomness in the response and the butterfly effect, the quality of fully automated distillation results is not guaranteed. To address this, we develop a web-based assistant to enable manual supervision and early intervention at runtime. We hope our findings and tools inspire future research toward revolutionizing the engineering processes of knowledge-based systems across domains.

% \newpage
\bibliographystyle{splncs04}
\bibliography{references}
\shortversion{\newpage\begin{comment}
\section{System Architecture}
\label{appendix:architecture}
\system has a novel modularized system architecture with three key components: 
\emph{StreamManager}, 
\emph{TxnManager} and \emph{TxnScheduler}. 
These components are instantiated in each thread locally.
The execution outline of \system is presented in Algorithm~\ref{alg:algo}.
Transactional stream processing is continuous and potentially never ends (Line 1$\sim$8).
The dependency resolution and execution of state transactions are separated into two non-overlapping phases by punctuations~\cite{Tucker:2003:EPS:776752.776780} (Line 2 and 5), which guarantees that no subsequent input event will have a smaller timestamp. 
Effectively, a batch of state transactions is collected during the first phase, and processed during the second phase.

In the first phase (i.e., stream processing phase), 
the \emph{StreamManager} conducts preprocessing for every input event ($e$). Similar to some prior works~\cite{tstream}, state transactions may be issued but not immediately processed during preprocessing (Line 3).
The \emph{pre\_processing} and \emph{post\_processing} functions are exposed as APIs to users.
The \emph{TxnManager} handles dependency resolution (Line 4) among state transactions and insert decomposed operations to construct a \tpg. We discuss the detailed two-phase \tpg construction process in Section~\ref{subsec:construction}.

In the second phase  (i.e., transaction processing phase), 
the \emph{TxnManager} is first involved again to refine (Line 6) the constructed \tpg with further dependency resolution.
The \emph{TxnScheduler} 
schedules operations for concurrent execution based on the constructed \tpg according to the three dimensions of scheduling decisions (Line 7). 
In particular, a scheduling decision model $M$ is instantiated based on the constructed \tpg (Line 14).
\textbf{\circled{1}} Guided by $M$, execution threads adopt an exploration strategy (Section~\ref{subsec:explore}) to explore the constructed \tpg for operations available to be scheduled constrained by dependencies. 
\textbf{\circled{2}} 
During exploration, one or multiple operations may be treated as the 
% basic 
unit of scheduling (Section~\ref{subsec:granularity}). 
Subsequently, \textbf{\circled{3}} every thread executes operation(s) in the unit of scheduling with various abort handling mechanisms (Section~\ref{subsec:abort_handling}).
Only when state transactions are processed (i.e., committed or aborted) can the associated input events be postprocessed (Line 8) by the \emph{StreamManager} based on transaction processing results.
\end{comment}

\begin{comment}
\begin{algorithm}
\footnotesize
    \KwData{$e$ \tcp{Input event}}
    \KwData{$txn_{ts}$ \tcp{State transaction}}
    \KwData{$G$ \tcp{The currently constructed TPG}}
    \While{!finish processing of input streams}{
        \eIf(\tcp*[h]{Phase 1}){\text{$e$ is not a $punctuation$}}{
                $txn_{ts}$ $\gets$ PRE\_Processing($e$)\;
                \textbf{TPG\_Construction}($G$, $txn_{ts}$)\; 
          }(\tcp*[h]{Phase 2}){
                \textbf{TPG\_Refinement}($G$)\; 
                \textbf{TXN\_Scheduling}($G$)\; 
                POST\_Processing()\;
          }
    }
    
    \SetKwFunction{FMain}{TPG\_Construction}
    \SetKwProg{Fn}{Function}{:}{}
    \Fn{\FMain{$G$, $txn_{ts}$}}{
        $O_{1..k}$ $\gets$ \textbf{Partition} $txn_{ts}$\;
        \ForEach{\text{operation $O_{i}$ $\in$ $O_{1..k}$}}{
            \textbf{Identify} its \ld\;
            $G$ $\gets$ $G$ + $O_{i}$ \;
        }
    }
    \SetKwFunction{FMain}{TPG\_Refinement}
    \SetKwProg{Fn}{Function}{:}{}
    \Fn{\FMain{$G$}}{
        \ForEach{\text{vertex $e_{i}$ $\in$ $G$}}{
            \textbf{Identify} its \td, \pd\;
        }
    }
    
    \SetKwFunction{FMain}{TXN\_Scheduling}
    \SetKwProg{Fn}{Function}{:}{}
    \Fn{\FMain{$G$}}{
        $M$ $\gets$ Instantiated with $G$;\tcp{A decision model}
        \While{!finish scheduling of $G$
        }{
          \textbf{\circled{2}} $Scheduling Unit$ $\gets$ \textbf{\circled{1}} \emph{Explore}($G$, $M$)\; 
            \textbf{\circled{3}} \emph{Execute with Abort Handling} ($Scheduling Unit$)\; 
        }
    }
  \caption{Execution Outline of \system}
  \label{alg:algo}
\end{algorithm}
\end{comment}}









\end{document}
