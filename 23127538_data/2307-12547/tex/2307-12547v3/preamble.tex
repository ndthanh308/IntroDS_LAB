\usepackage{hyperref}
\usepackage{amssymb,amsmath}
\usepackage{color}
%\usepackage[colorlinks=true,linkcolor=blue,citecolor=Mahogany]{hyperref}
\usepackage{makecell}
%\usepackage{multirow}
\usepackage{mathtools}
% \usepackage{mathaccent}
%\usepackage{eulervm}
%\usepackage{graphicx}
\usepackage{bbm}
\usepackage{setspace}
%\newcommand{\eqdef}{\overset{\mathrm{def}}{=\joinrel=}}
% \newenvironment{proof}{\noindent{\em Proof:}}{ \hfill $\square$\\ }

\DeclareMathOperator*{\argmin}{arg\,min}
\DeclareMathOperator*{\argmax}{arg\,max}

\newcommand{\pl}{{\bf pl}}

%Color and Pictures
\usepackage[usenames,svgnames,dvipsnames]{xcolor}

%\longversion{
%\usepackage{fullpage}
%\usepackage{charter,eulervm}
%}

\sloppy


\usepackage{tikz}
\usetikzlibrary{arrows,positioning}


\newcommand{\defproblem}[3]{
  \vspace{1mm}
\begin{center}
\noindent\fbox{

  \begin{minipage}{\textwidth}
  \begin{tabular*}{\textwidth}{@{\extracolsep{\fill}}l} \textsc{\underline{#1}} \\ \end{tabular*}\vspace{1ex}
  {\bf{Input:}} #2  \\
  {\bf{Question:}} #3
  \end{minipage}

  }
\end{center}
  \vspace{1mm}
}

\newcommand{\defparproblem}[4]{
  \vspace{1mm}
\begin{center}
\noindent\fbox{

  \begin{minipage}{\textwidth}
  \begin{tabular*}{\textwidth}{@{\extracolsep{\fill}}lr} \textsc{#1}  & {\bf{Parameter:}} #3 \\ \end{tabular*}
  {\bf{Input:}} #2  \\
  {\bf{Question:}} #4
  \end{minipage}

  }
\end{center}
  \vspace{1mm}
}

\newdimen\prevdp
\def\leftlabel#1{\noalign{\prevdp=\prevdepth
   \kern-\prevdp\nointerlineskip\vbox to0pt{\vss\hbox{\ensuremath{#1}}}\kern\prevdp}}

%A useful package if you write url addresses:
\usepackage{url}

\renewcommand{\labelitemi}{$\vartriangleright$}

\newcommand{\SB}{\mathbb{S}}

\usepackage{enumerate}

\usepackage{xspace}
\newcommand{\NP}{\ensuremath{\mathsf{NP}}\xspace}
\newcommand{\NPC}{\ensuremath{\mathsf{NP}}\text{-complete}\xspace}
\newcommand{\NPH}{\ensuremath{\mathsf{NP}}\text{-hard}\xspace}
\newcommand{\PNPH}{para-\ensuremath{\mathsf{NP}\text{-hard}}\xspace}
\newcommand{\el}{\ensuremath{\ell}\xspace}
\newcommand{\suc}{\ensuremath{\succ}\xspace}
\newcommand{\sucb}{\ensuremath{\succ_{\text{best}}}\xspace}
\newcommand{\sucw}{\ensuremath{\succ_{\text{worst}}}\xspace}
\newcommand{\PNENP}{\ensuremath{\mathsf{P\ne NP}}\xspace}
\newcommand{\WOH}{\ensuremath{\mathsf{W[1]}}-hard\xspace}
\newcommand{\WOC}{\ensuremath{\mathsf{W[1]}}-complete\xspace}
\newcommand{\FPT}{\ensuremath{\mathsf{FPT}}\xspace}
\newcommand{\ETH}{\ensuremath{\mathsf{ETH}}\xspace}
\newcommand{\OPT}{\ensuremath{\mathsf{OPT}}\xspace}
\newcommand{\ALG}{\ensuremath{\mathsf{ALG}}\xspace}
\newcommand{\FPTAS}{\ensuremath{\mathsf{FPTAS}}\xspace}
\newcommand{\tsat}{\ensuremath{(3,\text{B}2)}-{\sc SAT}\xspace}
\newcommand{\SAT}{\ensuremath{3}-{\sc SAT}\xspace}
\newcommand{\XTC}{{\sc X3C}\xspace}
\newcommand{\Pb}{\ensuremath{\mathsf{P}}\xspace}
\newcommand{\tw}{\text{tw}\xspace}
\newcommand{\tvdp}{{\sc Two Vertex Disjoint Paths}\xspace}

\let\oldlambda\lambda
\renewcommand{\lambda}{\ensuremath{\oldlambda}\xspace}
\let\oldalpha\alpha
\renewcommand{\alpha}{\ensuremath{\oldalpha}\xspace}
\let\oldDelta\Delta
\renewcommand{\Delta}{\ensuremath{\oldDelta}\xspace}

\newcommand{\YES}{{\sc yes}\xspace}
\newcommand{\NO}{{\sc no}\xspace}
\newcommand{\yes}{{\sc yes}\xspace}
\newcommand{\no}{{\sc no}\xspace}
\newcommand{\true}{\text{{\sc true}}\xspace}
\newcommand{\false}{\text{{\sc false}}\xspace}

\newcommand{\rr}{{\sc Resource Reallocation}\xspace}
\newcommand{\ra}{{\sc Resource Allocation}\xspace}

\newcommand{\pa}{{\sc Connected Knapsack}\xspace}
\newcommand{\conknapsack}{\pa}
\newcommand{\pathknapsack}{{\sc Path Knapsack}\xspace}
\newcommand{\shortestpathknapsack}{{\sc Shortest Path Knapsack}\xspace}
\newcommand{\consepknapsack}{{\sc Connected Seperator Knapsack}\xspace}
\newcommand{\indpsetknapsack}{{\sc Independent Knapsack}\xspace}
\newcommand{\pd}{{\sc Defender's Problem}\xspace}

\newcommand{\loss}{{\text{\sc Loss}}\xspace}
%
\newcommand{\ds}{{\sc Defending Strategy}\xspace}
%
\newcommand{\partition}{\text{\sc Partition}\xspace}

\renewcommand{\AA}{\ensuremath{\mathcal A}\xspace}
\newcommand{\BB}{\ensuremath{\mathcal B}\xspace}
\newcommand{\CC}{\ensuremath{\mathcal C}\xspace}
\newcommand{\DD}{\ensuremath{\mathcal D}\xspace}
\newcommand{\EE}{\ensuremath{\mathcal E}\xspace}
\newcommand{\FF}{\ensuremath{\mathcal F}\xspace}
\newcommand{\GG}{\ensuremath{\mathcal G}\xspace}
\newcommand{\HH}{\ensuremath{\mathcal H}\xspace}
\newcommand{\II}{\ensuremath{\mathcal I}\xspace}
\newcommand{\JJ}{\ensuremath{\mathcal J}\xspace}
\newcommand{\KK}{\ensuremath{\mathcal K}\xspace}
\newcommand{\LL}{\ensuremath{\mathcal L}\xspace}
\newcommand{\MM}{\ensuremath{\mathcal M}\xspace}
\newcommand{\NN}{\ensuremath{\mathcal N}\xspace}
\newcommand{\OO}{\ensuremath{\mathcal O}\xspace}
\newcommand{\PP}{\ensuremath{\mathcal P}\xspace}
\newcommand{\QQ}{\ensuremath{\mathcal Q}\xspace}
\newcommand{\RR}{\ensuremath{\mathcal R}\xspace}
\renewcommand{\SS}{\ensuremath{\mathcal S}\xspace}
\newcommand{\TT}{\ensuremath{\mathcal T}\xspace}
\newcommand{\UU}{\ensuremath{\mathcal U}\xspace}
\newcommand{\VV}{\ensuremath{\mathcal V}\xspace}
\newcommand{\WW}{\ensuremath{\mathcal W}\xspace}
\newcommand{\XX}{\ensuremath{\mathcal X}\xspace}
\newcommand{\YY}{\ensuremath{\mathcal Y}\xspace}
\newcommand{\ZZ}{\ensuremath{\mathcal Z}\xspace}

\newcommand{\SSS}{\overline{\SS}\xspace}


\newcommand{\ov}[1]{\ensuremath{\overline{#1}}}

\newcommand{\aaa}{\ensuremath{\mathfrak a}\xspace}
\newcommand{\bbb}{\ensuremath{\mathfrak b}\xspace}
\newcommand{\ccc}{\ensuremath{\mathfrak c}\xspace}
\newcommand{\ddd}{\ensuremath{\mathfrak d}\xspace}
\newcommand{\eee}{\ensuremath{\mathfrak e}\xspace}
\newcommand{\fff}{\ensuremath{\mathfrak f}\xspace}
%\newcommand{\ggg}{\ensuremath{\mathfrak g}\xspace}
%\newcommand{\hhh}{\ensuremath{\mathfrak h}\xspace}
\newcommand{\iii}{\ensuremath{\mathfrak i}\xspace}
\newcommand{\jjj}{\ensuremath{\mathfrak j}\xspace}
\newcommand{\kkk}{\ensuremath{\mathfrak k}\xspace}
%\newcommand{\lll}{\ensuremath{\mathfrak l}\xspace}
\newcommand{\mmm}{\ensuremath{\mathfrak m}\xspace}
\newcommand{\nnn}{\ensuremath{\mathfrak n}\xspace}
\newcommand{\ooo}{\ensuremath{\mathfrak o}\xspace}
\newcommand{\ppp}{\ensuremath{\mathfrak p}\xspace}
\newcommand{\qqq}{\ensuremath{\mathfrak q}\xspace}
\newcommand{\rrr}{\ensuremath{\mathfrak r}\xspace}
\newcommand{\sss}{\ensuremath{\mathfrak s}\xspace}
\newcommand{\ttt}{\ensuremath{\mathfrak t}\xspace}
\newcommand{\uuu}{\ensuremath{\mathfrak u}\xspace}
\newcommand{\vvv}{\ensuremath{\mathfrak v}\xspace}
\newcommand{\www}{\ensuremath{\mathfrak w}\xspace}
\newcommand{\xxx}{\ensuremath{\mathfrak x}\xspace}
\newcommand{\yyy}{\ensuremath{\mathfrak y}\xspace}
\newcommand{\zzz}{\ensuremath{\mathfrak z}\xspace}

\newcommand{\EB}{\ensuremath{\mathbb E}\xspace}
\newcommand{\NB}{\ensuremath{\mathbb N}\xspace}
\newcommand{\RB}{\ensuremath{\mathbb R^+}\xspace}


\usepackage{nicefrac}
\newcommand{\ffrac}{\flatfrac}
\newcommand{\nfrac}{\nicefrac}


%\newtheorem{proposition}{\bf Proposition}
\newtheorem{observation}{\bf Observation}
% \newtheorem{theorem}{\bf Theorem}
% \newtheorem{lemma}{\bf Lemma}
% \newtheorem{corollary}{\bf Corollary}
% \newtheorem{definition}{\bf Definition}
% \newtheorem{claim}{\bf Claim}
\newtheorem{reductionrule}{\bf Reduction rule}

\newcommand{\eps}{\ensuremath{\varepsilon}\xspace}
\renewcommand{\epsilon}{\eps}

\newcommand{\ignore}[1]{}

\newcommand{\pr}{\ensuremath{\prime}}
\newcommand{\prr}{\ensuremath{{\prime\prime}}}

\renewcommand{\leq}{\leqslant}
\renewcommand{\geq}{\geqslant}
\renewcommand{\ge}{\geqslant}
\renewcommand{\le}{\leqslant}

% \renewcommand{\chi}{\ensuremath{\chi}}

% \usepackage{relsize}
\usepackage{enumitem}
\setlist[enumerate]{labelwidth=!, labelindent=0pt}


\newcommand{\greedy}{\textsc{Greedy committee}\xspace}

\newcommand{\kp}{\text{\sc Knapsack}\xspace}
\newcommand{\vc}{\text{\sc Vertex Cover}\xspace}
\newcommand{\pvc}{\text{\sc Partial Vertex Cover}\xspace}
\newcommand{\hp}{\text{\sc Hamiltonian Path}\xspace}
\newcommand{\indset}{\text{\sc Independent Set}\xspace}
%\\\\\\\\\\\\\\\\\\\\\\\\\\\\\\\\\\\\\\\\\\\\\\\\\\\\\\\\\\\\\\\\\\\\\\\\\\\\\
%-----------------------------------------------------------------------------------------
%---------------------------------- packages ---------------------------------------------
%-----------------------------------------------------------------------------------------



\usepackage{rotating}
\usepackage{amssymb}
\usepackage{latexsym}
%\usepackage[pdftex]{graphicx}
%% Added two
\usepackage{epsfig}
%\usepackage{subfigure}
\usepackage{xcolor}
\usepackage{url}
\raggedbottom
\usepackage{fancyhdr}

% added instead
\usepackage{caption}
\usepackage{subcaption}
\usepackage{array}
\usepackage{multirow}
\usepackage{enumerate}
\usepackage{pdflscape}
\usepackage{amsmath,graphicx,algorithm,algpseudocode,amsfonts}
\usepackage{epstopdf}
\usepackage{float}
\floatstyle{ruled}
\newfloat{Algorithms}{!thb}{lop}
\floatname{Algorithms}{Algorithm}

\allowdisplaybreaks[1]

% Example definitions.
% --------------------
\def\x{{\mathbf x}}
\def\L{{\cal L}}
\algnewcommand\algorithmicinput{\textbf{Input:}}
\algnewcommand\INPUT{\item[\algorithmicinput]}
\algnewcommand\algorithmicoutput{\textbf{Output:}}
\algnewcommand\OUTPUT{\item[\algorithmicoutput]}
\algnewcommand{\LineComment}[1]{\State \(\triangleright\) #1}

% \renewcommand{\baselinestretch}{1}
%\setcaptionwidth{0.9\textwidth}
\newtheorem{thm}{Theorem}
\newtheorem{rem}{Remark}
%\newtheorem{pr}{Proof}
\newtheorem{lem}{Lemma}
%\newtheorem{example}{Example}
%\usepackage{enumitem}
%\setlist[itemize]{noitemsep}
%\setlist[itemize]{itemsep=0cm}
%\setlength{\parindent}{4em}
\setlength{\parskip}{.5em}
%\renewcommand{\baselinestretch}{1.5}
\usepackage{pifont}

\usepackage{mathtools}

\newcommand\Myperm[2][^n]{\prescript{#1\mkern-2.5mu}{}P_{#2}}
\newcommand\Mycomb[2][^n]{\prescript{#1\mkern-0.5mu}{}C_{#2}}
%\\\\\\\\\\\\\\\\\\\\\\\\\\\\\\\\\\\\\\\\\\\\\\\\\\\\\\\\\\\\\\\\\\\\\\\\\\\\\
%\begin{comment}
%\usepackage{comment}
%\usepackage{indentfirst}
%\setlength{\parindent}{0cm}
%\setlength{\parskip}{.3cm}
%\parindent=1cm
%\parskip=1.2cm
%\usepackage{relsize}
%\def\textsubscript#1{\ensuremath{_{\mbox{\textscale{.6}{#1}}}}}
%\usepackage{caption}
\newcommand\tab[1][1cm]{\hspace*{#1}}
\usepackage{algorithm}
\usepackage{algpseudocode}
% \usepackage{pifont}
%\usepackage{algcompatible}
%\usepackage[compatible]{algpseudocode}
%\usepackage{algorithmicx}
%\usepackage[options ]{algorithm2e}
%\usepackage[ruled,vlined]{algorithm2e}
%\usepackage{algorithm,algpseudocode}
%\usepackage{lipsum}

%\usepackage{caption}
%%%%%%%%%%% For breakable algorithm

\makeatletter
\newenvironment{breakablealgorithm}
{% \begin{breakablealgorithm}
		\begin{center}

			\refstepcounter{algorithm}% New algorithm
			\hrule height.8pt depth0pt \kern2pt% \@fs@pre for \@fs@ruled
			\renewcommand{\caption}[2][\relax]{% Make a new \caption
				{\raggedright\textbf{\fname@algorithm~\thealgorithm} ##2\par}%
				\ifx\relax##1\relax % #1 is \relax
				\addcontentsline{loa}{algorithm}{\protect\numberline{\thealgorithm}##2}%
				\else % #1 is not \relax
				\addcontentsline{loa}{algorithm}{\protect\numberline{\thealgorithm}##1}%
				\fi
				\kern2pt\hrule\kern2pt
			}
		}{% \end{breakablealgorithm}
		\kern2pt\hrule\relax% \@fs@post for \@fs@ruled
	\end{center}
}
\makeatother
\usepackage{comment}


\renewenvironment{proof}{\paragraph{Proof:}}{\hfill$\square$}
\newenvironment{proof_sketch}{\paragraph{Proof sketch:}}{\hfill$\square$}

\usepackage{cleveref}

\crefname{theorem}{Theorem}{\bf Theorems}
\crefname{observation}{Observation}{\bf Observations}
\crefname{lemma}{Lemma}{\bf Lemmata}
\crefname{corollary}{Corollary}{\bf Corollaries}
\crefname{proposition}{Proposition}{\bf Propositions}
\crefname{definition}{Definition}{\bf Definitions}
\crefname{claim}{Claim}{\bf Claims}
\crefname{reductionrule}{Reduction rule}{\bf Reduction rules}
\usepackage{color,soul} % For highLight
%\usepackage[dvipsnames]{xcolor} % To access some named colors used with \highLight
%\usepackage{luacolor} % Required to use the lua-ul \highLight command 
%\usepackage{lua-ul} 