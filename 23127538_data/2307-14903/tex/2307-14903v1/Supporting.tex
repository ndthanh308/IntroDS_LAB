%%%%%%%%%%%%%%%%%%%%%%%%%%%%%%%%%%%%%%%%%%%%%%%%%%%%%%%%%%%%%%%%%%%%%
%% This is a (brief) model paper using the achemso class
%% The document class accepts keyval options, which should include
%% the target journal and optionally the manuscript type. 
%%%%%%%%%%%%%%%%%%%%%%%%%%%%%%%%%%%%%%%%%%%%%%%%%%%%%%%%%%%%%%%%%%%%%
\documentclass[journal=jacsat,manuscript=article]{achemso}

%%%%%%%%%%%%%%%%%%%%%%%%%%%%%%%%%%%%%%%%%%%%%%%%%%%%%%%%%%%%%%%%%%%%%
%% Place any additional packages needed here.  Only include packages
%% which are essential, to avoid problems later. Do NOT use any
%% packages which require e-TeX (for example etoolbox): the e-TeX
%% extensions are not currently available on the ACS conversion
%% servers.
%%%%%%%%%%%%%%%%%%%%%%%%%%%%%%%%%%%%%%%%%%%%%%%%%%%%%%%%%%%%%%%%%%%%%
\usepackage[version=3]{mhchem} % Formula subscripts using \ce{}
\usepackage{siunitx}
\usepackage[dvipsnames]{xcolor}
\usepackage{caption}
\usepackage{graphicx}
\usepackage{subcaption}
\DeclareCaptionLabelFormat{suppl}{#1~#2S}%

%%%%%%%%%%%%%%%%%%%%%%%%%%%%%%%%%%%%%%%%%%%%%%%%%%%%%%%%%%%%%%%%%%%%%
%% If issues arise when submitting your manuscript, you may want to
%% un-comment the next line.  This provides information on the
%% version of every file you have used.
%%%%%%%%%%%%%%%%%%%%%%%%%%%%%%%%%%%%%%%%%%%%%%%%%%%%%%%%%%%%%%%%%%%%%
%%\listfiles

%%%%%%%%%%%%%%%%%%%%%%%%%%%%%%%%%%%%%%%%%%%%%%%%%%%%%%%%%%%%%%%%%%%%%
%% Place any additional macros here.  Please use \newcommand* where
%% possible, and avoid layout-changing macros (which are not used
%% when typesetting).
%%%%%%%%%%%%%%%%%%%%%%%%%%%%%%%%%%%%%%%%%%%%%%%%%%%%%%%%%%%%%%%%%%%%%
\newcommand*\mycommand[1]{\texttt{\emph{#1}}}

%%%%%%%%%%%%%%%%%%%%%%%%%%%%%%%%%%%%%%%%%%%%%%%%%%%%%%%%%%%%%%%%%%%%%
%% Meta-data block
%% ---------------
%% Each author should be given as a separate \author command.
%%
%% Corresponding authors should have an e-mail given after the author
%% name as an \email command. Phone and fax numbers can be given
%% using \phone and \fax, respectively; this information is optional.
%%
%% The affiliation of authors is given after the authors; each
%% \affiliation command applies to all preceding authors not already
%% assigned an affiliation.
%%
%% The affiliation takes an option argument for the short name.  This
%% will typically be something like "University of Somewhere".
%%
%% The \altaffiliation macro should be used for new address, etc.
%% On the other hand, \alsoaffiliation is used on a per author basis
%% when authors are associated with multiple institutions.
%%%%%%%%%%%%%%%%%%%%%%%%%%%%%%%%%%%%%%%%%%%%%%%%%%%%%%%%%%%%%%%%%%%%%
\author{Carolina Paba}
\affiliation{Department of Physics, University of Trieste, 34127 Trieste, Italy}
\author{Virginia Dorigo}
\affiliation{Hochschule Fresenius, 65510 Idstein, Germany}
\author{Beatrice Senigagliesi}
\affiliation{Elettra Sincrotrone Trieste, 34149 Basovizza TS, Italy}
\alsoaffiliation{Current address: IINS, Bordeaux Neurocampus, 33076 Bordeaux Cedex, France}
\author{Nicolò Tormena}
\affiliation{Department of Physics, University of Durham, Durham DH1 3LE, United Kingdom}
\author{Pietro Parisse}
\affiliation{IOM-CNR, 34149 Basovizza TS, Italy}
\alsoaffiliation{Elettra Sincrotrone Trieste, 34149 Basovizza TS, Italy}
\email{parisse@iom.cnr.it}
\phone{+39 3756416}
\author{Kislon Voitchovsky}
\email{kislon.voitchovsky@durham.ac.uk}
\affiliation{Department of Physics, University of Durham, Durham DH1 3LE, United Kingdom}
\phone{+44 191 334 3615}
\author{Loredana Casalis}
\email{loredana.casalis@elettra.eu}
\affiliation{Elettra Sincrotrone Trieste, 34149 Basovizza TS, Italy}
\phone{+39 040 375 8291}
%%%%%%%%%%%%%%%%%%%%%%%%%%%%%%%%%%%%%%%%%%%%%%%%%%%%%%%%%%%%%%%%%%%%%
%% The document title should be given as usual. Some journals require
%% a running title from the author: this should be supplied as an
%% optional argument to \title.
%%%%%%%%%%%%%%%%%%%%%%%%%%%%%%%%%%%%%%%%%%%%%%%%%%%%%%%%%%%%%%%%%%%%%
\title[An \textsf{achemso} demo]
  {Supplementary Information: Lipid bilayer fluidity and degree of order regulates small EVs adsorption on model cell membrane}
%%%%%%%%%%%%%%%%%%%%%%%%%%%%%%%%%%%%%%%%%%%%%%%%%%%%%%%%%%%%%%%%%%%%%
%% Some journals require a list of abbreviations or keywords to be
%% supplied. These should be set up here, and will be printed after
%% the title and author information, if needed.
%%%%%%%%%%%%%%%%%%%%%%%%%%%%%%%%%%%%%%%%%%%%%%%%%%%%%%%%%%%%%%%%%%%%%
\abbreviations{AFM,sEV,SLB}
\keywords{American Chemical Society, \LaTeX}
%%%%%%%%%%%%%%%%%%%%%%%%%%%%%%%%%%%%%%%%%%%%%%%%%%%%%%%%%%%%%%%%%%%%%
%% The manuscript does not need to include \maketitle, which is
%% executed automatically.
%%%%%%%%%%%%%%%%%%%%%%%%%%%%%%%%%%%%%%%%%%%%%%%%%%%%%%%%%%%%%%%%%%%%%
\begin{document}
\section{Extracellular vesicles size characterisation}
For AFM analysis of individual sEVs isolated from MDA-MB-231 cell line, a freshly cleaved mica of $10\;mm$ in diameter with $0.15-0.21\;mm$ thickness (from Micro to Nano) was first incubated with $30\;\mu L$ of Poly-L-Ornithine (from Sigma-Aldrich) for $20\;min$; after an extensive washing with Milli-Q $H_2 O$, $30\;\mu L$ of sEVs were let to incubate for $20\;min$ and then gently rinsed with $50\;\mu L$ of $10\;\unit{mM}$ PBS before AFM imaging. \\
The AFM size characterisation is reported in Figure \ref{fig:Figure1s}S. From the size distribution reported in the scatter plot, sEV's typical height was $15.43\pm5.77\;nm$, with a mean diameter equal to $49.73\;\pm\;18.24\;nm$, calculated on a number of $112$ vesicles. These values resides in the typical range that can be found in literature \cite{perissinotto2020multi,ridolfi2020afm}.
\captionsetup{labelformat = suppl}
% Figure environment removed
\section{MDA-MB-231 sEVs affinity test with single component SLB}
To confirm the hypothesis of the sEV's (from MDA-MB-231 cell line) preferential mixing with the $S_o$ phase of the SLB, the vesicles have been tested in their interaction with a single component SLB composed of DOPC (Figure \ref{fig:Figure2s}Sa) or DPPC (Figure \ref{fig:Figure2s}Sb). In the first composition, as expected and reported in Figure \ref{fig:Figure2s}Sc, sEVs colocalize with the DOPC defects, without interacting with the surrounding SLB. The opposite behavior is reported in Figure \ref{fig:Figure2s}Sd for the DPPC SLB where the sEVs cover the whole area available. As previously reported, sEVs mixing with DPPC lead to the formation of an intermediate phase, regular in height, with a thickness comparable to the total thickness of the bilayer. 
\captionsetup{labelformat = suppl}
% Figure environment removed
\section{UC-MSC sEVs interaction with model membrane}
To study the differences in the sEVs interaction process as a function of the sEV's origin, sEVs isolated from the human umbilical cord-derived mesenchymal stromal cells (UC-MSCs) have been added to the SLB composed of DOPC and SM with $17\;mol \%$ of Cholesterol. From the AFM image reported in Figure \ref{fig:Figure3s}Sa, it can be observed that the sEV's mixing is localised at the interphase of the $L_o$/$L_d$ domains, forming local invaginations with an average value of $0.5\;nm$ at the level of the $L_d$ phase. This result is in agreement with the work proposed by our group \cite{perissinotto2021structural} for the same sV sample, where it was attributed to the formation of a mixed phase due to sEV fusion with the SLB. A time-resolved analysis was also performed to track the evolution of the sEV mixing with the SLB. In Figure \ref{fig:Figure3s}Sb is reported the same area was analysed after $40\;min$ from the sEV uptake. It can be noticed a positive growth of the area occupied by sEVs patches that is accompanied by a dramatic decrease in the $L_o$ domains. However, the process evolution differs from what can be observed for the same SLB composition in interaction with the sEVs isolated from the MDA-MB-231 cell line, as reported in the main text. In the latter, the $L_o$ melting was not characterized by any significative increase over the time of the area occupied by the sEV's patches, while here this phenomenon is more pronounced. This can be attributed to intrinsic differences in the sEV's composition and mechanisms of interaction with the proposed model system.\\
\captionsetup{labelformat = suppl}
% Figure environment removed
\bibliography{referencesSup}
\end{document}