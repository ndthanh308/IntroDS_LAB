%%%%%%%%%%%%%%%%%%%%%%%%%%%%%%%%%%%%%%%%%%%%%%%%%%%%%%%%%%%%%%%%%%%%%
%% This is a (brief) model paper using the achemso class
%% The document class accepts keyval options, which should include
%% the target journal and optionally the manuscript type. 
%%%%%%%%%%%%%%%%%%%%%%%%%%%%%%%%%%%%%%%%%%%%%%%%%%%%%%%%%%%%%%%%%%%%%
\documentclass[journal=jacsat,manuscript=article]{achemso}
%%\documentclass[journal=elsevier,manuscript=article]{achemso}
%%%%%%%%%%%%%%%%%%%%%%%%%%%%%%%%%%%%%%%%%%%%%%%%%%%%%%%%%%%%%%%%%%%%%
%% Place any additional packages needed here.  Only include packages
%% which are essential, to avoid problems later. Do NOT use any
%% packages which require e-TeX (for example etoolbox): the e-TeX
%% extensions are not currently available on the ACS conversion
%% servers.
%%%%%%%%%%%%%%%%%%%%%%%%%%%%%%%%%%%%%%%%%%%%%%%%%%%%%%%%%%%%%%%%%%%%%
\usepackage[version=3]{mhchem} % Formula subscripts using \ce{}
\usepackage{siunitx}
\usepackage[dvipsnames]{xcolor}
\usepackage{subcaption}

%%%%%%%%%%%%%%%%%%%%%%%%%%%%%%%%%%%%%%%%%%%%%%%%%%%%%%%%%%%%%%%%%%%%%
%% If issues arise when submitting your manuscript, you may want to
%% un-comment the next line.  This provides information on the
%% version of every file you have used.
%%%%%%%%%%%%%%%%%%%%%%%%%%%%%%%%%%%%%%%%%%%%%%%%%%%%%%%%%%%%%%%%%%%%%
%%\listfiles

%%%%%%%%%%%%%%%%%%%%%%%%%%%%%%%%%%%%%%%%%%%%%%%%%%%%%%%%%%%%%%%%%%%%%
%% Place any additional macros here.  Please use \newcommand* where
%% possible, and avoid layout-changing macros (which are not used
%% when typesetting).
%%%%%%%%%%%%%%%%%%%%%%%%%%%%%%%%%%%%%%%%%%%%%%%%%%%%%%%%%%%%%%%%%%%%%
\newcommand*\mycommand[1]{\texttt{\emph{#1}}}

%%%%%%%%%%%%%%%%%%%%%%%%%%%%%%%%%%%%%%%%%%%%%%%%%%%%%%%%%%%%%%%%%%%%%
%% Meta-data block
%% ---------------
%% Each author should be given as a separate \author command.
%%
%% Corresponding authors should have an e-mail given after the author
%% name as an \email command. Phone and fax numbers can be given
%% using \phone and \fax, respectively; this information is optional.
%%
%% The affiliation of authors is given after the authors; each
%% \affiliation command applies to all preceding authors not already
%% assigned an affiliation.
%%
%% The affiliation takes an option argument for the short name.  This
%% will typically be something like "University of Somewhere".
%%
%% The \altaffiliation macro should be used for new address, etc.
%% On the other hand, \alsoaffiliation is used on a per author basis
%% when authors are associated with multiple institutions.
%%%%%%%%%%%%%%%%%%%%%%%%%%%%%%%%%%%%%%%%%%%%%%%%%%%%%%%%%%%%%%%%%%%%%
\author{Carolina Paba}
\affiliation{Department of Physics, University of Trieste, 34127 Trieste, Italy}
\author{Virginia Dorigo}
\affiliation{Hochschule Fresenius, 65510 Idstein, Germany}
\author{Beatrice Senigagliesi}
\affiliation{Elettra Sincrotrone Trieste, 34149 Basovizza TS, Italy}
\alsoaffiliation{Current address: IINS, Bordeaux Neurocampus, 33076 Bordeaux Cedex, France}
\author{Nicolò Tormena}
\affiliation{Department of Physics, University of Durham, Durham DH1 3LE, United Kingdom}
\author{Pietro Parisse}
\affiliation{IOM-CNR, 34149 Basovizza TS, Italy}
\alsoaffiliation{Elettra Sincrotrone Trieste, 34149 Basovizza TS, Italy}
\email{parisse@iom.cnr.it}
\phone{+39 3756416}
\author{Kislon Voitchovsky}
\email{kislon.voitchovsky@durham.ac.uk}
\affiliation{Department of Physics, University of Durham, Durham DH1 3LE, United Kingdom}
\phone{+44 191 334 3615}
\author{Loredana Casalis}
\email{loredana.casalis@elettra.eu}
\affiliation{Elettra Sincrotrone Trieste, 34149 Basovizza TS, Italy}
\phone{+39 040 375 8291}
%%%%%%%%%%%%%%%%%%%%%%%%%%%%%%%%%%%%%%%%%%%%%%%%%%%%%%%%%%%%%%%%%%%%%
%% The document title should be given as usual. Some journals require
%% a running title from the author: this should be supplied as an
%% optional argument to \title.
%%%%%%%%%%%%%%%%%%%%%%%%%%%%%%%%%%%%%%%%%%%%%%%%%%%%%%%%%%%%%%%%%%%%%
\title[An \textsf{achemso} demo]
  {Lipid bilayer fluidity and degree of order regulates small EVs adsorption on model cell membrane}
%%%%%%%%%%%%%%%%%%%%%%%%%%%%%%%%%%%%%%%%%%%%%%%%%%%%%%%%%%%%%%%%%%%%%
%% Some journals require a list of abbreviations or keywords to be
%% supplied. These should be set up here, and will be printed after
%% the title and author information, if needed.
%%%%%%%%%%%%%%%%%%%%%%%%%%%%%%%%%%%%%%%%%%%%%%%%%%%%%%%%%%%%%%%%%%%%%
\abbreviations{AFM,sEVs,SLB,$L_d$,$L_o$,$S_o$,Chol,DOPC,SM,UC-MSC,MDA-MB-231,TNBC}
\keywords{American Chemical Society, \LaTeX}

%%%%%%%%%%%%%%%%%%%%%%%%%%%%%%%%%%%%%%%%%%%%%%%%%%%%%%%%%%%%%%%%%%%%%
%% The manuscript does not need to include \maketitle, which is
%% executed automatically.
%%%%%%%%%%%%%%%%%%%%%%%%%%%%%%%%%%%%%%%%%%%%%%%%%%%%%%%%%%%%%%%%%%%%%
\begin{document}

%%%%%%%%%%%%%%%%%%%%%%%%%%%%%%%%%%%%%%%%%%%%%%%%%%%%%%%%%%%%%%%%%%%%%
%% The "tocentry" environment can be used to create an entry for the
%% graphical table of contents. It is given here as some journals
%% require that it is printed as part of the abstract page. It will
%% be automatically moved as appropriate.
%%%%%%%%%%%%%%%%%%%%%%%%%%%%%%%%%%%%%%%%%%%%%%%%%%%%%%%%%%%%%%%%%%%%%
%% The abstract environment will automatically gobble the contents
%% if an abstract is not used by the target journal.
%%%%%%%%%%%%%%%%%%%%%%%%%%%%%%%%%%%%%%%%%%%%%%%%%%%%%%%%%%%%%%%%%%%%%
\begin{abstract}
Small extracellular vesicles (sEVs) are known to play an important role in the communication between distant cells and to deliver biological information throughout the body. To date, many studies have focused on the role of sEVs characteristics such as cell origin, surface composition, and molecular cargo on the resulting uptake by the recipient cell. Yet, a full understanding of the sEV fusion process with recipient cells and in particular the role of cell membrane physical properties on the uptake are still lacking. Here we explore this problem using sEVs from a cellular model of triple-negative breast cancer fusing to a range of synthetic planar lipid bilayers both with and without cholesterol, and designed to mimic the formation of ‘raft’-like nanodomains in cell membranes. Using time-resolved Atomic Force Microscopy we were able to track the sEVs interaction with the different model membranes, showing the process to be strongly dependent on the local membrane fluidity. The strongest interaction and fusion is observed over the less fluid regions, with sEVs even able to disrupt ordered domains at sufficiently high cholesterol concentration. Our findings suggest the biophysical characteristics of recipient cell membranes to be crucial for sEVs uptake regulation.\\\\
Keywords: Extracellular vesicles, model membrane, uptake, fluidity, Atomic Force Microscopy.
%
\end{abstract}
%%%%%%%%%%%%%%%%%%%%%%%%%%%%%%%%%%%%%%%%%%%%%%%%%%%%%%%%%%%%%%%%%%%%%
%% Start the main part of the manuscript here.
%%%%%%%%%%%%%%%%%%%%%%%%%%%%%%%%%%%%%%%%%%%%%%%%%%%%%%%%%%%%%%%%%%%%%
\section{Introduction}
%The plasma membrane has a crucial role in maintaining cell homeostasis\cite{tekpli2013role}.

%of its molecular composition and shape in response to external stimuli \cite{sezgin2017mystery, simons2011membrane}
%
The plasma membrane has an essential role in maintaining cell homeostasis\cite{tekpli2013role}, and can actively  regulate its molecular composition and shape in response to external stimuli\cite{sezgin2017mystery, simons2011membrane}.
%
In particular, it is known that some dynamic membrane microdomains such as lipid rafts and caveolae contribute in regulating many cellular functions including cell proliferation, survival, and intracellular signaling through the constant local redistribution of membrane lipids.\cite{lingwood2010lipid, smart1999caveolins,zajchowski2002lipid} 
%
This has major implications for protein sorting and molecular trafficking across the membrane \cite{zajchowski2002lipid}. 
%
Among the different molecular species involved, cholesterol is known to play a fundamental role for the correct functioning of membrane domains, regulating membrane fluidity and the structural integrity of lipid rafts. \cite{crane2004role, engberg2016affinity}.
%
%\textcolor{red}{compartmentalised in a tightly packed state called liquid-ordered phase ($L_o$), coexisting with the more fluid surrounding lipid bilayer in the shape of a liquid-disordered ($L_d$) phase \cite{kaiser2009order}}.
%
%the constant membranes assembling play a role in protein sorting, Cholesterol recruitment, and membrane trafficking, resulting in a cascade of signaling events \cite{zajchowski2002lipid}. 
%as it is involved in the regulation of membrane fluidity and lipid raft structural integrity, serving as a fundamental molecule for the correct functioning of these platforms \cite{crane2004role, engberg2016affinity}. 
Cholesterol depletion can lead to an increased permeability to external pathogens, signaling molecules release, and in the worst case, lipid rafts disruption with alteration of membrane thickness. This, in turn, affects the signaling pathways and can even induce programmed cell death. Cholesterol accumulation in lipid rafts is equally problematic, creating a higher sensitivity to apoptosis \cite{li2006elevated}. 
%
%In terms of cell membrane local composition and lateral organization, Cholesterol has also major effects, in particular on the modulation of intercellular communication processes, involving endocytic pathways \cite{hanzal2007lipid}. It has indeed been suggested that lipid phase order and membrane fluidity, mostly affected by perturbations of the Cholesterol levels and of the cortical actin (CA) cytoskeleton component, actively drive the gradient of molecules trafficking across the cell membrane \cite{subczynski2000physical, cordeiro2018molecular}. 
%
Finally, cholesterol modulates both the cell membrane local composition and its lateral molecular organisation, fluidity, and hence intercellular communication processes including endocytic pathways. \cite{hanzal2007lipid}.
%
One important aspect of lipid rafts - and indirectly cholesterol - is their involvement in the release and uptake of a particular class of cell-derived vesicles called extracellular vesicles (EVs). EVs are now widely accepted as nanocarriers involved in cell-cell communication \cite{huyan2020extracellular,herrmann2021extracellular,araujo2022biomedical} and have been shown to take part in many pathophysiological processes such as cancer progression and metastasis formation, cell proliferation, and stimulation of the adaptive and innate immune system \cite{becker2016extracellular,bebelman2018biogenesis}.
%
%now widely considered by the scientific community as powerful nanocarriers, involved in cell-cell communication, taking part in many pathophysiological processes such as cancer progression and metastasis formation, cell proliferation, and stimulation of the adaptive and innate immune system \cite{becker2016extracellular,bebelman2018biogenesis}. 
%
EVs are typically divided into two sub-classes, microvesicles (MVs) and exosomes which differ from their biogenesis pathway: budding of the membrane for MVs and endocytic pathway for exosomes \cite{kalluri2020biology}. Their respective size range is slightly different with MVs ($ 100 - 1000\; nm $) and exosomes ($30 - 150\;nm$) \cite{van2014particle,coumans2017methodological}. Since the different vesicles isolation methods rely on size-based separation, vesicles ranging from $30$ to $200\; nm$, (enriched in exosomes but also including small MVs) are often referred to as small Extracellular Vesicles (sEVs). sEVs have emerged as potential cancer biomarkers as their molecular composition reflects that of the originating cells; they are also considered optimal delivering nanocarriers as they mediate the communication between tumor and tumor-associated cells, escaping the immune response \cite{maia2018exosome}.  
%
%Since the different isolation methods available mainly allow for a size-based separation, vesicles ranging from $30$ to $200\;nm$, (enriched in exosomes but also including small MVs) are referenced as small Extracellular Vesicles (sEVs). sEVs have emerged as potential cancer biomarkers as their molecular composition reflects the one of the originating cells; they are also considered optimal delivering nanocarriers as they mediate the communication between tumor and tumor-associated cells, escaping the immune response \cite{maia2018exosome}. 
%
This landscape is further complicated by the presence of different pathways for sEVs to deliver their molecular cargo. One of these pathways is lipid raft mediated endocytosis, where the rafts are continuously assembling/disassembling to maintain cell homeostasis and regulate vesicle trafficking \cite{mulcahy2014routes}. 
%
Delivery of the cargo can also occur through a mechanism initiated by some degree of fusion with the target membrane, a pathway mostly adopted by viruses. This last pathway induces a level of mixing of the sEV and target membranes which become contiguous, something recently observed with umbilical cord mesenchymal stem cells (UC-MSC) sEVs from GMP production and a model membrane containing lipid rafts \cite{perissinotto2021structural}. 
%
However, details on the dynamics of the interaction pathways of sEVs with target cells and on the specific role of each molecular player are still scarce and highly debated in the literature \cite{herrmann2021extracellular, russell2019biological, french2017extracellular}. 
%
This is related to the small size and the high heterogeneity of sEVs \cite{thery2018minimal}, as well as to the high spatial and temporal resolution required for the detection of lipid rafts dynamics. Several recent studies have investigated the role of the biophysical properties of the cell membrane on vesicle fusion and agglomeration rate, showing that the mechanical properties such as membrane curvature, fluidity and rigidity, all highly regulated by cholesterol content, can affect vesicle fusion kinetics \cite{russell2019biological, grouleff2015influence, perissinotto2021structural,caselli2021plasmon}. 
% 
In this study, we follow up on the raft-based pathway for regulating vesicle uptake and investigate the interaction of single sEVs isolated from a triple-negative breast cancer cell line (TNBC) with model supported lipid bilayers (SLBs) with different fluidity, ordered nanodomains, and cholesterol concentration. Using Atomic Force Microscopy (AFM) in solution, we aim to explore sEVs fusion with membranes exhibiting quasi-physiological cholesterol concentration and compositions reflecting the raft structures of in vivo systems. In particular, we focus on lipid membranes with the coexistence of two lipid phases: a tightly packed and ordered state called liquid-ordered phase ($L_o$) made of sphingolipid and cholesterol molecules, coexisting with a more fluid and disordered phase called liquid-disordered ($L_d$) phase enriched with unsaturated phospholipid. The use of AFM enables us to track single EVs interacting and fusing with the membrane in situ and with nanoscale precision, and the subsequent evolution of the target membrane.
%
\section{Results and discussion}
\subsection{AFM of the model membrane}
Supported lipid bilayers offer a powerful model membrane platform for studying membrane–vesicles interactions. They allow for the simultaneous analysis of the SLB morphology changes with quantitative information about the surface area of the different lipid phases, height modification, and real-time observation of the vesicle fusion process with the lipidic system. In order to gain microscopic insights into the sEV fusion process, we first performed a careful topographic analysis of the multicomponent-SLBs by means of AFM. To mimic the typical organisation of cell membranes in ’lipid raft’ microdomains, the composition adopted comprises 1,2-dioleoyl-sn-glycero-3-phosphocholine (DOPC) which is a neutral and monounsaturated phospholipid ($18:1$), sphingomyelin (SM) that represents one of the most abundant sphingolipids in the plasma membrane, and is characterized by long saturated fatty acyl chains \cite{niemela2006influence}, and cholesterol (Chol) that sterically interacts with the acyl chains of other lipids and preferentially with saturated phospholipids such as SM \cite{marquardt2016Cholesterol}. This mixture is representative of the outer membrane leaflet as it contains phosphatidylcholine and sphingomyelin as its main building blocks. To monitor the lipid phase behavior and the cholesterol dependence on SLB morphology, three cholesterol molecular concentrations have been tested: $5\; mol \%$, $10\;mol \%$ and $17\;mol \%$, with DOPC and SM kept at a fixed $2:1$ ($m/m$) ratio. In the following sections the sample composition with $17\;mol\%$ will be described in more depth and compared with a bilayer that has no sterol content. The $17\;mol \%$ Chol falls in the typical biological range of $15 - 50\;\%$ for the sterol component, and offers good reproducibility and stability when performing AFM imaging in liquid conditions \cite{sullan2010Cholesterol,redondo2012influence}.
%
%This mixture has been chosen as the most representative in particular of the outer membrane leaflet as it contains phosphatidylcholine, sphingomyelin as building molecules. To monitor the lipid phase behavior and the Cholesterol dependence on lipid bilayer morphology, three Cholesterol molar concentrations have been tested, ranging from $5\;mol\%$ to $10\;mol\%$ up to $17\;mol\%$, with DOPC and SM kept at a fixed $2:1$ ($m/m$) ratio. In the following sections the sample composition with the $17\;mol\%$ will be described more deeply and compared with the opposite condition of a bilayer without the sterol component. This Chol percentage in fact falls in the typical biological range of $15-50 \%$ for the sterol component, and satisfies the needed conditions of good reproducibility and stability when performing AFM imaging in liquid conditions of LB over a rigid substrate such as mica \cite{sullan2010Cholesterol,redondo2012influence}.
%
%\subsection{Impact of lipid composition and Cholesterol on the bilayer molecular organisation}
\\
Typical examples of lipid phase separation as observed in AFM topographs are presented in Figure \ref{fig:Figure1}. 
% Figure environment removed
The formation of the liquid-ordered domains is visible for all three tested conditions, although they differ in height, area, and number. The taller $L_o$ bilayer domains exhibit a maximum diameter of around $0.5\; \mu m$ with $5\;mol\%$ Chol, a value that increases with Chol and reaches around $1.5\;\mu m$ at $17\;mol\%$. The apparent number of domains changes very little with increasing cholesterol percentage, varying from an average value of $113 \pm 7.07$ to $128 \pm 5.65$ and $135 \pm 14.36$ respectively. The increase of the area occupied by $L_o$ domains is accompanied by a decrease of their relative height to with respect to the surrounding DOPC. The total area occupied by $L_o$ domains is directly related to the Chol in agreement with the theory of the preferential mixing of cholesterol with saturated lipids such as SM. This results in the increase of the area per lipid \cite{ma2016Cholesterol, mcmullen2004Cholesterol}, and a ’cholesterol-condensing effect’ on phospholipids thickening of the $L_d$ phase and a reduced height difference with the $L_o$ domains \cite{ma2016Cholesterol, hung2007condensing}. 
%
%This result is in agreement with the theory of the preferential mixing of Cholesterol with saturated lipids such as SM, resulting in the increase of the area per lipid \cite{ma2016Cholesterol, mcmullen2004Cholesterol}, and the 'Cholesterol-condensing effect' on phospholipids, resulting in a thickening of the $L_d$ phase and a reduced height difference with the $L_o$ domains \cite{ma2016Cholesterol, hung2007condensing}. 
%
Lastly, it has been demonstrated that the transition temperature of a phospholipid system decreases with increasing cholesterol concentration \cite{redondo2012influence}. This explains the lower number of $L_o$ domains per scanned area observed for the $5\;mol\%$, where the transition starts at $30^\circ C$, when compared with the $10\;mol\%$ and $17\;mol\%$ where lower thermal fluctuations are required to promote the $L_o$ phase nucleation. While helpful to explain the AFM observations, a full description of the systems should also take into account the impact that a rigid substrate, which tends to stabilize the lipids and promote order (lower $T_c$) \cite{leidy2002ripples}, has on the bilayer, and the kinetics of the temperature control during sample cooling, which in turn influences the $L_o$ nucleation process and growth \cite{blanchette2008quantifying}.
%leading to a lowering of $T_m$ with the increase of Cholesterol amount. This would explain the higher number of $L_o$ domains per scanned area observed for the $5\;mol\%$, where the transition starts at $30^\circ C$, in comparison with the $10\;mol\%$ and $17\;mol\%$ where lower thermal fluctuations are required to promote the $L_o$ phase nucleation. However, all the considerations above should take into account that the impact that a rigid substrate has on the lipid bilayer arrangement is not negligible \cite{leidy2002ripples}, as well as the temperature control during the cooling of the sample, which strongly influences the Lo nucleation process and growth \cite{blanchette2008quantifying}. 
%
%\textcolor{red}{In this work, a cooling speed of $0.02\;\frac{^\circ C}{s}$ and a final imaging temperature of $27^\circ C$ were used}.
% Figure environment removed
\subsection{Adsorption of the sEVs and local biophysical changes}
With the model membrane system characterized, we then explore the interaction of sEVs isolated from the TNBC MDA-MB-231 cell line. The isolation protocol is reported in the Materials and Methods section, together with all the details of our model membrane systems. TNBC represents one of the most aggressive breast cancer subtypes, with a poor prognosis due to the absence of targetable receptors, high propensity for metastatic progression, and lack of effective chemotherapy treatments \cite{lee2019triple}. TNBC-derived sEVs have been thoroughly characterized in previous works from our group \cite{senigagliesi2022triple}. In particular, it was observed that TNBC-derived sEVs induce morphological as well as biomechanical phenotype changes in non-metastatic cancer cells toward higher aggressiveness. Here, once the model membrane has been characterized, a concentration set of sEVs are introduced to the aqueous solution and the system evolution followed in real time within few minutes. A representative AFM image of sEVs adsorption on the $17\;mol\%$ Chol membrane is shown in Figure \ref{fig:Figure3}a. 
% Figure environment removed
The adsorption of sEVs induce small protrusion $\sim1\;nm$ above the height over the $L_o$ domains, accompanied by a local destabilisation of the $L_o$ region at the edges of the interaction’s site. This destabilisation appears as fluid-like regions surrounding the protrusions (blue arrows in Figure \ref{fig:Figure3}) and the formation of pores confined at the level of the outer leaflet of the supported lipid bilayer. 
%
%\textcolor{red}{and fluidity has been studied since 1925 when Leathes et al. \cite{leathes1925croonian} first posed the question of how the area per lipid molecule changes with the Cholesterol addition to a phospholipidic system}. 
%Once the sample has been characterized, by performing AFM in liquid conditions with the use of a liquid perfusion holder, we were able to follow in real time the interaction of EVs with the previously described supported lipid bilayers. A representative AFM image of EV adsorption on the lipid bilayer for the composition with $17\;mol\%$ Chol, reported in Figure \ref{fig:Figure3}a, shows the formation of $1\;nm$ protrusions in height over the $L_o$ domains, while at the same time a local lipid bilayer destabilization upon EV interaction is observed at the edges of the interaction's site, with the formation of pores of a typical depth of $1.25\;nm$. 
%
Given the typical $5-6\;nm$ thickness of the membrane as measured from the SLB defects \cite{perissinotto2021structural, balgavy2001bilayer}, and the average $15.43\pm5.77\;nm$ height of sEVs when directly adsorbed on the mica substrate (according to \textcolor{blue}{Supplementary Information}) we interpret the localized protrusions as sEVs clusters whose adsorption process involves their full mixing with the SLB, and the possible molecular cargo release due to pore formation. No morphological changes were observed at the DOPC ($L_d$) level but only a local interaction with $L_o$ domains was detectable. 
%
%and given the average $10\;nm$ height of EVs when they are images over mica substrate as reported in \textcolor{blue}{supplementary Material}, we interpret the localized protrusions as clusters of EVs whose adsorption process involves their full mixing with the lipid bilayer, and the possible molecular cargo release due to pore formation. No morphological changes were observed at the DOPC level but only a local interaction with $L_o$ domains was detectable. Moreover, small round protrusions $0.5\;nm$ in height distributed along the $L_o$ domain area and surrounded by a thin ring of the DOPC layer, can also be noticed in Figure \ref{fig:Figure3}a.\\
%
To understand whether the protrusions are EV-related components or the result of the $L_o$ degradation process, a time-resolved analysis was performed to track the process evolution (Figure \ref{fig:Figure3}a-c). A drastic rearrangement of the $L_o$ domains is visible with a progressively melting into the surrounding SLB, in favor of positive growth for both the area occupied by the lipid-vesicles protrusions and SLB invaginations. After an initial step of lateral lipid redistribution with no significant morphological variations, the area occupied by $L_o$ domains progressively decreases starting from the small defects of the $L_o$ phase characterized by high curvature and evolving laterally until the melting with the $L_d$ phase expansion is completed. Simultaneously, a slight increase in the area occupied by pores and the $L_o$ phase takes place. 
%
%To understand whether the protrusions are EV-related components or the result of the $L_o$ degradation process, a time-resolved analysis was performed to evaluate the process evolution. The AFM imaging of the lipidic system evolution over time, illustrated in Figure \ref{fig:Figure3}, reported a drastic rearrangement of the Lo domains, progressively melting into the surrounding lipid bilayer, in favor of positive growth for both the area occupied by the lipid-vesicles protrusions and lipid bilayer invaginations. After an initial step of lateral lipid redistribution with no significative morphological variations, the area occupied by $L_o$ domains progressively decreases starting from the small defects of the $L_o$ phase, characterized by high curvature, and evolving in the radial direction until the melting with the $L_d$ phase is completed. At the same time, a slight increase in the area occupied by pores and the $L_o$ phase takes place.
%
The ‘melting’ effect of sEVs on planar lipid bilayer has previously been observed by our group \cite{perissinotto2021structural}, where sEVs from UC-MSC cell line were tested in the interaction with a SLB enriched with $5\;mol\%$ cholesterol. Here, sEVs lead to a dramatic fluidification of the $L_o$ phase, in contrast to the previous study where a mixing between sEVs and the $L_o$ was observed instead, with the formation of high granularity patches protruding $4\;nm$ above the SLB. These apparent differences in docking process and the resulting impact on the SLB suggests possible intrinsic differences in the EV adsorption process based on sEVs origin and cholesterol content of the target membrane. To further investigate the impact of the sEVs origins, we tested the behaviour of sEVs isolated from the UC-MSC cell line with the same target membrane containing $17\;mol\%$ Chol, resulting in qualitatively similar results to what previously reported \cite{perissinotto2021structural} (Figure 3S, \textcolor{blue}{Supplementary Information}). Given the relevance of lipid raft integrity in regulating cell proliferation, adhesion, and invasion \cite{badana2016lipid}, these results further strengthens the idea of EV potency altering the membrane properties. It also underlines the need for screening approaches that consider, other than EV’s molecular cargo and surface properties, the cell membrane molecular composition in order to be able to investigate their ability to alter the membrane properties of recipient cells, such that both faces of the interaction process can be explored.  
%
%The melting effect of EVs on lipid bilayer was already observed in previous work from our group \cite{perissinotto2021structural}, where EVs from a mesenchymal stem cell line (UC-MSC), were tested in the interaction with a planar lipid bilayer enriched with $5\;mol\%$ Cholesterol. While in our system, EVs lead to a massive fluidification of the $L_o$ phase, in that previous study a fusion between EVs and the $L_o$ phase was observed instead, with the formation of patches characterized by high granularity and protruding $4\;nm$ in height over the LB.The differences between both the docking process and the resulting impact on the LB structure alteration, opened the question of possible intrinsic differences in the EV adsorption process related to both the EV origin and the LB physical state, since the Chol content was different in the two cases ($5\;mol\%$ and $17\;mol\%$, respectively).
%To investigate the possible differences of the EV impact on model membrane reorganization depending on the EV origin, the behavior of EVs isolated from the UC-MSC cell line was tested with the same Chol concentration investigated for the EV-MDA-MB-231, and reported in the \textcolor{blue}{supplementary Material}. 
%Given the relevance of lipid raft integrity in regulating cell proliferation, adhesion, and invasion \cite{badana2016lipid}, these results underline the strength of the proposed model in discovering the EV potency in altering the membrane properties and underline the need for screening, based on the EV cell origin, the EV’s molecular cargo and surface properties, in order to be able to investigate their different function in altering the membrane properties of recipient cells.
%
%% Figure environment removed
  %  
\subsection{sEVs interaction is regulated by lipids mobility}
The previous results highlight the importance of ordered nano-domains on the adsorption and fusion of sEVs. The well-established importance of cholesterol in modulating the emergence, stability and fate of these nano-domains makes it an obvious agent for indirectly modulating sEVs uptake in recipient cells. It is however not clear at this stage to what extent the effect is physical in terms of membrane biomechanics and fluidity or chemical through specific interactions between cholesterol and adsorbing sEVs. To further study the impact of membrane fluidity on modulating the sEVs adsorption, two control compositions with $0\;\%$ Chol content were also analysed containing either DOPC and SM $2:1$ or DOPC and DPPC $2:1$ at $27^\circ C$. In these conditions, SM domains are expected to form an ordered phase also called solid-ordered ($S_o$), characterized by a higher degree of order and less fluidity, surrounded by fluid DOPC. Similarly, DPPC domains should form highly ordered gel-phase domains within the DOPC. For both membranes, AFM imaging confirms the expectations (Figure \ref{fig:Figure4}a,b), with SM forming smaller domains covering an average percentage area of $1.17\;\%$ and protruding $1.75\;nm$ over the DOPC layer, compared to bigger DPPC domains, occupying an average $2.8\;\%$ of the membrane and with a relative height of $2\;nm$ over the surrounding DOPC. The SM $S_o$ domains are also more irregular in height that DPPC, showing two different levels at $0.75\;nm$ and $1.75\;nm$ above the DOPC layer, suggesting that the phase transition of the SM during the cooling is not uniform. Indeed, the two levels can be explained by a leaflet-by-leaflet phase transition where the SM molecules in contact with the substrate solidify first \cite{rinia2001visualizing, alessandrini2014phase}. This is also consistent with the fact that DPPC displays a highly cooperative phase transition characterized by a sharp peak at the main $T_m$ in differential scanning calorimetry whereas SM shows a single endothermic peak with a wide transition range related to the heterogeneity of the fatty acids of the lipid \cite{demetzos2008differential,nyholm2003calorimetric}.
% Figure environment removed
The interactions of MDA-MB-231 sEVs with the SM $S_o$ phase is illustrated in Figure \ref{fig:Figure4}c, showing the formation of protrusions $6\;nm$ above the lipid domains. Interestingly, the sEVs clusters that co-localise with the portion of SM domains are characterized by the largest height. Moreover, the number of interaction sites per scanned area is higher compared to the membrane with $17\;mol\%$ Chol, indicating an enhanced EV interaction with the planar lipid bilayer. However, no local morphological variations can be observed over time, suggesting that the sEVs are no longer able to mix with their lipidic component with that of the SLB. A comparative experiment conducted on the DOPC/DPPC membrane displays a similar degree of order and level of saturation to the model system with SM, ruling out a chemical affinity of the sEVs with SM. Also in this case (Figure \ref{fig:Figure4}d), a specific MDA-MB-231 sEVs interaction with the ordered domains is observed. However, contrary to SM domains, a mixing with the vesicles is visible, inducing an increase of the relative height of the $S_o$ domains (profile in Figure \ref{fig:Figure4}d). This is confirmed by AFM revealing the overlapping of multiple layers and the presence of a ’vesicle-like’ morphology over the DPPC domains. To fully confirm the hypothesis of sEVs preferential mixing with high-ordered domains, two control experiments were performed using single-component SLB made of either pure DOPC or pure DPPC. The results, reported in Figure 2S of \textcolor{blue}{Supplementary Information}, confirm that sEVs do not interact with the disordered DOPC SLB, while a maximal interaction can be observed for the DPPC SLB, resulting in the SLB morphology reshaping over a larger time scale compared to the system enriched with cholesterol. These results highlight the need of lower system fluidity for the 'lipid raft' domains in order to have a fast EV adsorption process and cargo release over the SLB.  Moreover, the structural SLB modification leading to a ’lipid rafts’ fluidification further stresses the importance of the molecular orientation and packing in the recipient membrane lipids to control interaction and uptake of sEVs over time. These results pave the basis for further investigating the physicochemical mechanisms of the cell membrane, and in particular of lipid rafts as a preferential route of interaction with the sEVs.
%
%MDA-MB-231 sEVs interaction with the $S_o$ phase formed by SM is illustrated in Figure \ref{fig:Figure5}a and see the formation of protrusions $6\;nm$ in height over the lipid domains. Interestingly, the EVs clusters now form
%protrusions of $6\;nm$ in height that co-localise with the portion of SM domains characterized by the highest height. Moreover, the number of interaction sites that can be counted per scanned area is higher compared to what has been observed for the composition with $17\;mol \%$, meaning that in this case the EV interaction with the lipid bilayer is maximised. However, by the analysis over time, no local morphological variations can be observed, meaning that the EVs are no longer able to mix with the lipidic component of the LB in the absence of a certain level of fluidity. The preferential docking with the $S_o$ domains, posed the question of the eventual chemical affinity of the EVs with SM, or the possible physical nature of the EVs interaction with the LB. To understand that, a comparative analysis has been performed for the EVs interacting with the LB composed of DOPC and DPPC, which displays a similar degree of order and level of saturation to the model system with SM. Also in this case, as illustrated in Figure \ref{fig:Figure5}b, a specific EV-MDA-MB-231 sEVs interaction with the ordered domains is observed. Contrary to what can be observed in the vesicle interaction with SM domains, where the relative height of the $S_o$ phase does not change upon EV uptake, in the case of DPPC a mixing with the vesicles can be noticed, which determines an increase of the relative height of the domains, as can be appreciated by the height profile reported in Figure \ref{fig:Figure5}b. Indeed, it can be clearly observed from the AFM imaging the overlapping of multiple layers and the presence of a 'vesicle-like' morphology over the DPPC domains. 
%To support the hypothesis of the preferential mixing of EVs with high-ordered domains and the less likely interaction with a more disordered system, two control affinity test were performed as a function of time for a single-component LB made of DOPC and DPPC respectively. The results, reported in the \textcolor{blue}{supplementary Material}, confirm that EVs do not interact with the disordered LB made of DOPC, while a maximal interaction can be observed for the DPPC, resulting in a LB morphology reshaping over a larger time scale compared to the system enriched with Cholesterol.
%These results indicate the need of system fluidity in order to have a fast EV adsorption process and cargo release over the lipid bilayer. Moreover, the structural LB modification that leads to a 'lipid rafts' fluidification as a consequence, stresses also the importance of a specific orientation and packing degree of the lipidic components to increase/reduce the EV's interaction and time-scale through which the process can spread. This paves the basis for further investigating the physicochemical mechanisms of the cell membrane, and in particular of lipid rafts as a preferential route of interaction with the EVs. 
%
%developing innovative strategies to manipulate the fusion process with the target cells, underlying the need to focus the attention not only on the EVs characteristics but also in the cell membrane structural properties.
%
%% Figure environment removed
\section{Conclusions}
The development of a multi-component SLB mimicking the 'lipid-raft' structure of cell model membranes, allowed us to study the driving forces regulating the sEVs uptake for vesicles isolated from breast cancer cell lines. Our findings, based on fast AFM topographic imaging, indicate a preferential sEV affinity for the ordered lipid raft-like domains. However, the adsorption process undergoes different pathways depending on lipid bilayer composition and fluidity. Working at the submicrometric level and performing a time-resolved analysis it was possible to identify two interaction pathways. For a fluid SLB enriched with cholesterol, the adsorption process is featured by the formation of sEV clusters protruding over the outer layer of the model system. In the same frame, a pore-opening close to the interaction site occurs, followed by a fluidification step that leads to lipid raft integrity loss. Whereas, for a rigid system without cholesterol, the adsorption pathway follows the budding-fission mechanisms \cite{liu2023kinetic}, with maximal affinity with the solid-ordered domains. This alternative mechanism is described by the fusion of the vesicles with the outer layer of the model membrane and the formation of an intermediate regular lipid phase due to full lipid mixing with the vesicles. In such a rigid system, the extent of the interaction is featured by the formation of a stable state not prone to fracture, which leads to a large-scale shape modification over time. Our study provides evidence that the degree of sEV mixing with lipids is highly regulated by the vesicle origin but also by the fluidity of the SLB. Although the lipid composition is limited to a restricted choice of lipids and cholesterol range, we believe that our results provide a strong message in light of the chemical and physical forces regulating the vesicle uptake, underling that both cell membrane composition and lateral organization must be taken into consideration to rationalize sEV interaction and cargo release in the recipient cell. Moreover, it is also evident that the side effects on lipid raft integrity are not negligible as well, as it has been demonstrated that membrane domain disruption is fundamental for the regulation of molecules trafficking across the membrane and cell survival \cite{badana2016lipid}. Furthermore, it is interesting to note that this versatile platform can be applied to study the impact of surface functionalization strategies (e.g. fusogenic proteins) on the vesicle uptake pathways \cite{verta2022generation}, but it can also be easily integrated, besides cholesterol molecules, with other lipids and proteins. In particular, the reconstitution of transmembrane proteins in the proposed model would be an innovative approach for studying transmembrane proteins localization and activity, when the planar lipid bilayer is fabricated over a pore spanning membrane \cite{teiwes2021pore, muhlenbrock2020fusion}. We foresee that, with some implementation of the model, we can develop a versatile and broadly accessible platform for the investigation the sEVs uptake pathways.
\section{Experimental Section}
\subsection{sEV isolation and characterization}
For sEV isolation, MDA-MB-231 cells ($2\cdot10^6$) were grown in a $175\; \unit{\cm}^2$ flask in DMEM (Sigma-Aldrich) with $20\;\%$ FBS (EuroClone) for 3 days. The cells were then washed two times with PBS and three times with DMEM without serum. The cells were further incubated at $37^\circ \unit{\C}$. After $24 \;\unit{\hour}$ the medium was collected and centrifuged at $300\;g$ and $4^\circ \unit{\C}$ (Allegra X-22R, Beckman Coulter) for $10\;\unit{\min}$. With a $0.22 \;\unit{\mu}\unit{\m}$ filter, the supernatant was filtered, poured into Amicon Filter Units (Ultracel-PLPLHK, $100\;\unit{\kilo\dalton}$ cutoff, Merck Millipore, UFC9100) and centrifuged at $3900 g/ 4^\circ \unit{\C}$ for $20\;\unit{\min}$ (Allegra X-22R, Beckman Coulter). The samples collected were then transferred into the polypropylene (PP) ultracentrifuge tubes (Beckman Coulter, 361623), filled with PBS and centrifuged at $ 120000\;g/4^\circ \unit{\C}$ for $2\;\unit{\hour}$ in the ultracentrifuge (70.1 Ti rotor, k-factor 36, Beckman Coulter, Brea, CA, USA). After removing the supernatant, the pellets were resuspended in $200\;\unit{\mu}\unit{\liter}$ of PBS, aliquoted, and conserved at $-20\;^\circ\unit{\C}$ until usage.
\subsection{Small unilamellar vesicles preparation}
The lipids, 1,2-dioleoyl-sn-glycero-3-phosphoCholine ($18:1$ ($\Delta9-Cis$) PC), 1,2-dipalmitoyl-sn-glycero-3-phosphoCholine (DPPC, 16:1), Sphingomyelin (brain, porcine, SM), and cholesterol (ovine wool, $>98\%$), were purchased from Avanti Polar Lipids. The single lipids, suspended in chloroform, were mixed at the desired concentration and placed under vacuum overnight. The dry film was then hydrated with TRIS buffer ($10\unit{mM}$, $pH=7.4$),  to obtain a final concentration of $1 \unit{\milli\g/\milli\l}$. The lipidic mixture was sonicated for 40 min at $45^\circ \unit{C}$ and vortexed. Lastly, the resulting solution was extruded $51$ times at $40^\circ \unit{C}$ through a membrane with $100 \unit{\nm}$ pores (PC Membranes $0.1\;\unit{\mu}\unit{\m}$, Avanti Polar Lipids).
\subsection{Supported lipid bilayers preparation}
Lipids were combined in three lipid mixtures: DOPC/SM (2:1 m/m) with Chol (5, 10, 17 \unit{mol}\%), DOPC/SM and DOPC/DPPC in a fixed molar ratio of 2:1, and lastly, DOPC and DPPC alone. The obtained extruded solution was diluted in TRIS/\ce{CaCl2} buffer to a final concentration of $0.4\;\unit{\milli\g/\milli\l}$ with $2\;\unit{mM} \;\ce{CaCl2}$. For all compositions, the vesicle fusion method was adopted as a standard procedure for planar lipid bilayer preparation. The sample was deposited on a freshly cleaved mica substrate (Nano-Tec V-1 grade, $0.15-0.21\;mm$ thickness, $10\;mm$ diameter), incubated at $50^\circ \unit{C}$ for $30\;\unit{min}$, and slowly cooled to $27^\circ \unit{C}$, then extensively washed with TRIS buffer $10\;\unit{mM}$.
\subsection{Atomic Force Microscopy imaging}
AFM was performed on commercially available microscope (Cypher ES from Asylum Research), working at $27^\circ C$ in high resolution AC mode. Sharpe nitride levers ($SNL-10$ with A geometry from Bruker Corporation) were used to perform the imaging in liquid conditions. Images were acquired at $512\times512$ pixel frames at $2.44\;\unit{Hz}$.
\section{Author contributions}
C. P., L. C., K. V. and P. P. conceived and planned the
experiments. C. P. performed the atomic force
microscopy experiments and analysed the data. V. D. contributed to atomic force microscopy measurements. V.D. and B.S. contributed to EV isolation and molecular characterization. N.T. contributed to atomic force microscopy training. C. P. and L. C. took the lead in writing the
manuscript. All authors provided critical feedback and helped
shape the research, analysis and manuscript.
\section{Conflicts of interest}
There are no conflicts to declare.
\section{Acknowledgments}
The authors wish to thank M. Gimona from Paracelsus Medical University (Salzburg, Austria) for providing the EV-UC-MSC samples. We gratefully acknowledge the Structural Biology Laboratory at Elettra-Sincrotrone Trieste S.C.p.A. for the instrumentation and constant support during the cell culture experiments. We acknowledge the Soft and Bio NanoInterfaces Laboratory at Durham University for the precious and continuous support. The authors and in particular C. P. are very grateful to CERIC-ERIC for financial funding within the framework of the INTEGRA and INTEGRA's PhD project.
\newpage
\bibliography{references}
\end{document}
