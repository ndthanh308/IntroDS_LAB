\documentclass[nofootinbib,prl,superscriptaddress,a4paper]{revtex4-1}
%twocolumn
%\usepackage{scicite}
\usepackage{graphicx}% Include figure files
%\usepackage{dcolumn}% Align table columns on decimal point
\usepackage{bm,bbm}% bold math
\usepackage{xcolor}%colours
\usepackage{tcolorbox}
\usepackage{algorithm}
\usepackage{algpseudocode}
\usepackage{amsmath}
\usepackage{float}
\usepackage{booktabs} 

\usepackage[colorlinks,
            linkcolor=blue,  
            citecolor=blue,   
            urlcolor=blue,     
            breaklinks ]{hyperref}

\makeatletter
\newcommand\org@hypertarget{}
\let\org@hypertarget\hypertarget
\renewcommand\hypertarget[2]{%
  \Hy@raisedlink{\org@hypertarget{#1}{}}#2%
  }
\makeatother
\newcommand{\ket}[1]{\left\vert#1\right\rangle}
\newcommand{\bra}[1]{\left\langle #1 \right\vert}
\newcommand{\card}[1]{\left\vert #1 \right\vert}
\newcommand{\braket}[2]{\langle #1|#2\rangle}
\newcommand{\ketbra}[2]{| #1\rangle \langle #2|}
\newcommand{\xq}[1]{\textcolor{cyan}{#1}}  
\newcommand{\mpi}[1]{\textcolor{olive}{#1}}

\renewcommand{\figurename}{Supplementary Figure}
\renewcommand{\tablename}{Supplementary Table}

\setcounter{table}{0}
\renewcommand{\thetable}{\arabic{table}}
\setcounter{figure}{0}
\renewcommand{\thefigure}{\arabic{figure}}
\setcounter{equation}{0}
\renewcommand{\theequation}{S\arabic{equation}}

\begin{document}

\title{Supplementary Materials for\\
``Nonlocal photonic quantum gates over 7.0 km"}

\author{Xiao Liu}
\email{These three authors contributed equally to this work.}
\affiliation{CAS Key Laboratory of Quantum Information, University of Science and Technology of China, Hefei 230026, China}
\affiliation{CAS Center for Excellence in Quantum Information and Quantum Physics, University of Science and Technology of China, Hefei 230026, China}
\author{Xiao-Min Hu}
\email{These three authors contributed equally to this work.}
\affiliation{CAS Key Laboratory of Quantum Information, University of Science and Technology of China, Hefei 230026, China}
\affiliation{CAS Center for Excellence in Quantum Information and Quantum Physics, University of Science and Technology of China, Hefei 230026, China}
\affiliation{Hefei National Laboratory, University of Science and Technology of China, Hefei 230088, China}
\author{Tian-Xiang Zhu}
\email{These three authors contributed equally to this work.}
\author{Chao Zhang}
\author{Yi-Xin Xiao}
\author{Jia-Le Miao}
\author{Zhong-Wen Ou}
\author{Pei-Yun Li}
\affiliation{CAS Key Laboratory of Quantum Information, University of Science and Technology of China, Hefei 230026, China}
\affiliation{CAS Center for Excellence in Quantum Information and Quantum Physics, University of Science and Technology of China, Hefei 230026, China}

\author{Bi-Heng Liu}
\email{bhliu@ustc.edu.cn}
\author{Zong-Quan Zhou}
\email{zq\_zhou@ustc.edu.cn}
\author{Chuan-Feng Li}
\email{cfli@ustc.edu.cn}
\author{Guang-Can Guo}
\affiliation{CAS Key Laboratory of Quantum Information, University of Science and Technology of China, Hefei 230026, China}
\affiliation{CAS Center for Excellence in Quantum Information and Quantum Physics, University of Science and Technology of China, Hefei 230026, China}
\affiliation{Hefei National Laboratory, University of Science and Technology of China, Hefei 230088, China}




\begin{abstract}
\bigskip\bigskip\bigskip
\tableofcontents

\end{abstract}
\date{\today}
\maketitle


\subsection{1. Quantum gate teleportation protocol}

We briefly explain the protocol for the teleportation of quantum gates. A schematic diagram of a typical quantum gate teleportation is shown in Supplementary Figure \ref{fig:CNOT GATE}\textbf{a}. Assume that two parties, Alice and Bob, each have two qubits 1, 2, and 3, 4. EPR entanglement $|\Phi\rangle_{23}=\frac{1}{\sqrt{2}}\left(|00\rangle_{23}+|11\rangle_{23}\right)$ is shared between qubits 2 and 3. The purpose is to implement nonlocal CNOT operations for qubit 4 on qubit 1 with the assistance of this EPR entanglement, as well as local two-qubit operations and local single-qubit operations under classical communications. A quantum CNOT gate $C_{43}$ is performed on local qubits 3, 4, and a quantum CNOT gate $C_{21}$ is performed on local qubits 1, 2. Then, qubits 2 and 3 are measured on the $\{+,-\}$ and $\{0, 1\}$ bases, respectively. 
Finally, Alice and Bob will notify each other of the measurement results and perform the required operations on qubits 1 and 4 depending on the results. Quantum gate teleportation is realized by the following convention:



\begin{equation}
\begin{aligned}
C_{43} C_{21}\left(|\Psi\rangle_{14} \otimes|\Phi\rangle_{23}\right) & =|+0\rangle_{23} \otimes C_{41}\left(|\Psi\rangle_{14}\right) \\
& +|+1\rangle_{23} \otimes \sigma_1^x C_{41}\left(|\Psi\rangle_{14}\right) \\
& +|-0\rangle_{23} \otimes \sigma_4^z C_{41}\left(|\Psi\rangle_{14}\right) \\
& +|-1\rangle_{23} \otimes\left(-\sigma_1^x \sigma_4^z\right) C_{41}\left(|\Psi\rangle_{14}\right).
\end{aligned}
\end{equation}


% Figure environment removed



Here, we employ two degrees of freedom (DOFs) of two photons to encode the four qubits. 
The local two-qubit gates between different DOFs are inherently deterministic with only linear elements \cite{Kwiat1998Embedded}.
However, when we obtain the measurement results of qubit 2 with single-photon detectors, qubit 1 also collapses. Therefore, as shown in Supplementary Figure \ref{fig:CNOT GATE}\textbf{b}, we discard half of the measurement results of qubit 2, and the quantum gate teleportation is performed with a success probability of $50\%$. Nondestructive measurements of photonic qubits \cite{Niemietz2021Nondestructive,BrienBrienNondestructive} would be required to achieve bidirectional classical feedforward control, and thus to achieve deterministic implementations of this scheme with photons unabsorbed for further operations. 
A different strategy involves introducing additional photons as communication qubits for local two-qubit gates, enabling the control and target qubits for nonlocal quantum gates undetected and available for cascading. In this case, photon number resolving quantum nondemolition detection \cite{Nemoto2004Nearly} or other efficient nonlinear processes \cite{Chang2014Quantum} at the single-photon level are required to obtain higher success probabilities. To scale up the number of qubits and operations at each node, many other accessible DOFs including high-dimensional DOFs could be exploited for interaction \cite{Ren2013Deterministic,Imany2019High}, although practical scalability remains a great challenge \cite{Kwiat1997Hyper,DENG2017Quantum}. A more general approach is to use the interference of different photons to cascade qubits towards universal quantum computing \cite{Sleator1995Realizable} with the Knill-Laflamme-Milburn \cite{Knill2001scheme} scheme.


The realization of quantum teleportation for arbitrary controlled unitary (C-\textit{U}) gates is slightly different from that of quantum teleportation for CNOT gates.
As shown in Supplementary Figure \ref{fig:CNOT GATE}\textbf{c}, qubit 1 performs a CNOT operation on qubit 2, and qubit 3 performs a C-\textit{U} operation on qubit 4. In this way, we implement the nonlocal C-\textit{U} operation applied by qubit 1 to qubit 4. Similarly, we have

\begin{equation}
\begin{aligned}
CU_{34} C_{12}\left(|\Psi\rangle_{14} \otimes|\Phi\rangle_{23}\right)& =|0+\rangle_{23} \otimes CU_{14}\left(|\Psi\rangle_{14}\right) \\
& +|0-\rangle_{23} \otimes \sigma_1^z CU_{14}\left(|\Psi\rangle_{14}\right) \\
& +|1+\rangle_{23} \otimes U^{-1}_4 CU_{14}\left(|\Psi\rangle_{14}\right) \\
& +|1-\rangle_{23} \otimes\left(-\sigma_1^z U^{-1}_4\right) C_{14}\left(|\Psi\rangle_{14}\right),
\end{aligned}
\label{CU}
\end{equation}
with $UU^{-1}=I$. Since photonic qubits are measured with absorptive detectors, here the last two terms of Eq. \ref{CU} are dropped, and the quantum teleportation of the C-\textit{U} gate is also realized with a $50\%$ success probability.

\subsection{2. Overall experimental setup}

In this section, we briefly introduce the entire experimental setup at node A and node B (as shown in Supplementary Figure \ref{fig:scheme sm}). In Supplementary Figure \ref{fig:scheme sm}\textbf{a}, a two-photon source is generated using the spontaneous parametric down-conversion (SPDC) process, detailed in section 4 of the supplementary materials. Supplementary Figure \ref{fig:scheme sm}\textbf{b}, \textbf{d} depict two unbalanced interferometers with arm differences of 30 m, utilized for postselection of entangled states. Additionally, the interferometer in Supplementary Figure \ref{fig:scheme sm}\textbf{b} also serves as an encoding converter between path and polarization DOFs, detailed in section 9 of the supplementary materials, while the interferometer in Supplementary Figure \ref{fig:scheme sm}\textbf{d} implements the CNOT operation between A$_1$ and A$_2$, as explained in section 9 of the supplementary materials. In Supplementary Figure \ref{fig:scheme sm}\textbf{c}, a solid-state quantum memory (QM) with storage time up to 80.315 $\mu$s is employed to store 580-nm photons and await the feedforward signal from the measurement results of node B, thereby achieving an improvement in the success probability of controlled gates (see details in the main text and section 1 of the supplementary materials). The 1537-nm photons in polarization DOF are distributed through a 7.9-km field-deployed fiber to reach node B. In Supplementary Figure \ref{fig:scheme sm}\textbf{e}, the controlled gate of B$_3$ to B$_4$ in node B is implemented, as detailed in section 9 of the supplementary materials. Subsequently, the real-time measurement results of node B are fed back to node A through the classical channel. Based on the measurement results of node B, node A performs the corresponding unitary transformation ($I$ or $\sigma_x$) by a high-speed electro-optic modulator (EOM). Finally, a polarization analyzer is used to analyze the input and output quantum state results.



% Figure environment removed


\subsection{3. Laser system}

% Figure environment removed

The main laser system used in the experiment is placed at node A, with a schematic diagram shown in Supplementary Figure \ref{fig: laser system}. Two independent 580-nm lasers, both frequency-doubled from 1160-nm lasers, are employed for the SPDC source and the QM, respectively. The 580-nm laser-1 (Toptica, TA-SHG) used for quantum memory is frequency-locked to a high-finesse Fabry-Pérot (F-P) cavity, achieving a linewidth of approximately 0.3 kHz. The F-P cavity is temperature-controlled within a stability of 0.01 K and is placed in a vacuum chamber maintained by a sputter ion pump. The other 580-nm laser system (580-nm laser-2) consists of an 1160-nm narrow linewidth semiconductor laser (Toptica, DL PRO), a Raman fiber amplifier, and a single-pass frequency doubling system made by another company (Precilasers), with an output power up to 3 W. The 580-nm laser-2 is frequency stabilized by phase locking its 1160-nm seed laser to the 1160-nm seed laser in the 580-nm laser-1. The beat frequency between two 1160-nm lasers can be adjusted by changing the frequency of 1160-nm laser-2 with an acoustic optical modulator (AOM).
A 1537-nm laser is combined with the 580-nm laser-2 in a PPLN bulk crystal, producing a 421-nm laser by sum-frequency generation. The 1537-nm seed laser (Precilasers, EFL-SF-1536-S) is phase-locked to a commercial optical frequency comb (Menlo Systems, FC1500-250-ULN) and amplified by an erbium-doped fiber amplifier, achieving a linewidth of approximately 20 kHz and an output power of 5 W. The 421-nm laser, with a total output power of 200 mW, is employed as the pump laser for the SPDC process.




\subsection{4. SPDC photon source}


The pump light at 421 nm
is injected into a PPLN waveguide, leading to a nondegenerate SPDC process that produces signal photons at 580 nm and idler photons at 1537 nm. The peak pump power before the waveguide is set to approximately 10 mW and is gated into pulses according to the time sequence shown in Supplementary Figure \ref{fig:Time sequence}\textbf{b}. The 580-nm photon is reflected by a dichroic mirror (Semrock, FF775-Di01-25x36) and is filtered by a long-pass filter (Semrock, BLP01-442R-2f) and a band-pass filter centered at 580 nm with a 3-nm bandwidth. The 1537-nm photon passes through the dichroic mirror and is filtered using a filter centered at 1550 nm with a 30-nm bandwidth. We have significantly improved the collection efficiency of entangled photon sources using a 4-$f$ system comprising pairs of cylindrical lenses. Without the use of the cylindrical lenses (that is, removing parts CL@580 nm and CL@1537 nm in Supplementary Figure \ref{fig:source}), the heralding efficiency before the filtering system is $19.5\%$. After the filtering system, the heralding efficiency of the narrow-band source becomes $4.1\%$. To optimize the coupling mode, two pairs of cylindrical lenses are used to amplify the short axis of the light spot for 1537 nm and 580 nm, respectively. After shaping the light, the heralding efficiency of the source before the filtering system is increased to $27.5\%$, while the efficiency of the narrow-band source after the filtering system is $4.9\%$. This represents an overall improvement of approximately $20\%$ by employing the cylindrical lenses. 

% Figure environment removed



% Figure environment removed

The quantum correlation between two photons can be expressed as the second-order cross-correlation function $g_{12}^{(2)}(\Delta\tau)=P_{12} /\left(P_1 P_2\right)$ with zero time difference, where $P_{12}$ represents the two-photon coincidence probability in the time window $\Delta\tau$ and $P_{1} (P_{2})$ corresponds to the measured probability of a single photon of port 1 (2). In practice, the correlation between two photons can be obtained by measuring the two-photon coincidence counts, as shown in Supplementary Figure \ref{fig:g2}.
By Lorentz fitting of both sides of the peak value of the correlation function with $e^{-2 \pi \Delta v \tau}$, we estimate photon bandwidths $\Delta v$ of 178 MHz at 580 nm and 155 MHz at 1537 nm. The $g_{12}^{(2)}(\Delta\tau)$ between photon pairs at zero time difference is 36.9 in a detection window of $\Delta\tau = 6.4$ ns. We use a 30-m unbalanced interferometer to generate time-energy entanglement, and observe a visibility of $0.990\pm0.001$ for the computational basis {$|0\rangle$, $|1\rangle$} and $0.862\pm0.001$ for the Fourier basis {$|0\rangle+|1\rangle$,$|0\rangle-|1\rangle$}. This indicates that the fidelity of our initial entangled state is approximately $0.926\pm0.001$.

\subsection{5. Quantum memory}

% Figure environment removed



Among various quantum storage protocols \cite{Fleischhauer2000Dark-State, Moiseev2001Complete, Duan2001Long-distance, Afzelius2009Multimode, Ma2021Elimination},
the atomic frequency comb (AFC) \cite{Afzelius2009Multimode} scheme has the particular advantages of large storage bandwidth and high multimode capacity, which is crucial for efficient interface with quantum light sources. The $\mathrm{Eu^{3+}}$:$\mathrm{Y_2SiO_5}$ crystal has been an important candidate for quantum memory because of its extremely long optical and spin coherence lifetimes \cite{Konz2003Temperature, Jobez2014Cavity, Zhong2015Optically, Ma2021One-hour}. 
Since the hyperfine splitting of $\mathrm{^{153}Eu^{3+}}$ ions is larger than that of $\mathrm{^{151}Eu^{3+}}$ ions \cite{Timoney2013Single}, this isotope is chosen here to achieve an AFC memory with a wider bandwidth. The maximum bandwidth of the spin-wave storage \cite{Ma2021One-hour, Ma2021Elimination} can reach $29.1$ MHz for site-1 $\mathrm{^{153}Eu^{3+}}$ ions in the $\mathrm{Y_2SiO_5}$ crystal, which can be obtained based on the level structure of the $\mathrm{^{153}Eu^{3+}}$:$\mathrm{Y_2SiO_5}$ crystal (Supplementary Figure \ref{fig:Time sequence}\textbf{a}) and the analogous analytical method presented in Ref. \cite{SU2022ondemand}. 


The detailed setup of quantum memory is shown in Supplementary Figure \ref{fig:quantum memory}.
To isolate mechanical vibrations and extend the optical coherence lifetime ($T_2$), the $\mathrm{^{153}Eu^{3+}}$:$\mathrm{Y_2SiO_5}$ crystal is cooled to 3.1 K with vibration-isolated sample holder installed in a closed-cycle cryostat (Montana Instruments). The sample holder uses five springs made from stainless steel with a wire diameter of 0.4 mm, which significantly reduces the high-frequency (above kHz) vibration caused by the cold head of the cryostat. $T_2$ is measured by the two-pulse photon echo \cite{Tittel2010Photon-echo}.
As shown in Supplementary Figure \ref{fig:T2-compare}, $T_2=1.27 \pm 0.01$ ms is measured with the vibration-isolated sample holder when the peak power of the input pulse is $1.71$ mW, which is nearly independent with time, while $T_2=0.49 \pm 0.04$ ms is measured at the moment of minimum vibration without this sample holder. 
The vibration of the cryostat is indirectly measured by an acceleration sensor placed near the cold head. 
Our vibration-isolated sample holder increases the $T_2$ of $\mathrm {^{153}Eu^{3+}}$ ions by more than 2.6 times, which is crucial for the preparation of high-resolution AFCs for long-lived and multiplexed quantum storage.

% Figure environment removed

% Figure environment removed


The input mode with a center frequency of $f_0 = 516.847$ THz is generated from the SPDC source, stored in the crystal for 80.315 $\mu$s, and finally released to the local gate $C_{21}$. We use two counter-propagating modes (pump-1 and pump-2) to prepare quantum memory. The pump pulses are generated by AOMs that are controlled with an arbitrary waveform generator.
Pump-1, modulated by a double-pass AOM, is applied for the initialization and the preparation of the AFC at the center frequency of $f_0$.
Pump-2, modulated by a single-pass AOM and a double-pass AOM, is employed to achieve antihole burning at the center frequency of $f_0$.
The time sequence for the preparation of quantum memory and the whole experiment is shown in Supplementary Figure \ref{fig:Time sequence}\textbf{b}. 

The first step of the pump sequence is initializing the absorption of the crystal. A chirping pulse at the center frequency of $f_0$ with a bandwidth of 68 MHz is repeated 25 times. After the initialization process, the population in the storage bandwidth is close to empty.

The next step is antihole burning. Based on the level structure of the $\mathrm{^{153}Eu^{3+}}$:$\mathrm{Y_2SiO_5}$ crystal (Supplementary Figure \ref{fig:Time sequence}\textbf{a}) and the analysis detailed in Ref. \cite{Zhu2022On-Demand}, an absorption band with an optical depth of $9.7$ at the center frequency of $f_0$ is obtained by applying a chirping pulse at the center frequency of $f_0-166$ MHz. This chirping pulse, which has a bandwidth of 158 MHz and a duration of 8 ms, is repeated 25 times. Using this pumping scheme, we can enhance the absorption depth by obtaining many back-burning holes from the different classes of ions in the absorption band. 

The third step is the preparation of the AFC. To achieve long-lived multiplexed storage, an AFC with an extremely large number ($N$) of comb teeth ($N>1000$) is required here. In the simple parallel preparation scheme \cite{Jobez2016Towards}, the preparation pulses are the direct superposition of the complex hyperbolic secant (CHS) pulses with periodic frequency detunings $n\Delta$, where $\Delta$ is the frequency period of the AFC and $n=1,2,3,...,N$. The interference between the CHS pulses leads to a low energy utilization of the AFC-preparation pulses.
As a result, the pulse energy of such simple parallel AFC-preparation pulses is proportional to $1/N$ \cite{Businger2022Non-classical}, which would require long preparation times to prepare AFCs with a large $N$, and the efficiency is severely limited. 
To solve this problem, Ref. \cite{Schroeder1970Synthesis, Oswald2021Burning, Businger2022Non-classical} proposed adding the Schroeder phase $\Phi_n=\pi\left[\frac{n^2}{2 N}\right]$ to every CHS pulses \cite{Schroeder1970Synthesis} to evenly separate the temporal centers of every CHS pulse \cite{Businger2022Non-classical}. Such an idea greatly improves the energy utilization efficiency of the AFC preparation pulse and leads to the preparation of high-quality AFCs \cite{Businger2022Non-classical}.
The CHS pulses $A_n(t)$ for square combs used here can be written as
\begin{equation}
	A_n(t) =\operatorname{sech}[\beta (t-\frac{T_{prep}}{2})] \sin \left[2 \pi\left(f_0+(n-\frac{N+1}{2}) \Delta\right) (t-\frac{T_{prep}}{2})+ 2\pi \frac{\Delta_f}{2 \beta} \ln (\cosh (\beta(t-\frac{T_{prep}}{2})\right], 
\end{equation}
where $t \in[0, T_{prep}]$, $n=1,2,3,...,N$, $T_{prep}$ is the pulse duration, parameter $\beta$ controls the waveform of the CHS pulses, $\gamma$ is the width of the combs and $\Delta_f=\Delta-\gamma$. 
By summing the CHS pulses with separated temporal centers $A'_n(t)$, the normalized AFC-preparation pulse $A(t)$ can be written as
\begin{equation}
	\begin{aligned}
	A(t) = &\frac{1}{\text{max}(A(t))} \sum_{n=1}^{N} \sin (\Phi_n) A'_n(t) \\
	= &\frac{1}{\text{max}(A(t))} \sum_{n=1}^{N} \sin (\pi\left[\frac{n^2}{2 N}\right])
	\left\{
	\begin{aligned}
		& A_n(t+\frac{n-1}{N}T_{prep})  & {0 \leq t < \frac{N-n+1}{N}T_{prep}} \\
		& A_n(t-\frac{N-n+1}{N}T_{prep})  & {\frac{N-n+1}{N}T_{prep} \leq t < T_{prep}} 
	\end{aligned}
	\right..
\end{aligned} 
\end{equation}
This waveform can only be applied to a single-pass AOM but not to a double-pass AOM. Here, we solve this problem by extending the function to the complex field that uses $\exp(i\theta(t))$ to replace the $\sin(\theta(t))$ term in $A(t)$. Considering the desired waveform 
\begin{equation}
		f(t) = a(t)e^{i\theta(t)} \quad \quad \text{with } a(t) \geq 0, \quad \theta(t) \in[-\pi, \pi],
\end{equation}
the corresponding waveform $f'(t)$ for the double-pass AOM can be represented as
\begin{equation}
\begin{aligned}
	f'(t) = & \sqrt{a(t)}e^{i\theta(t)/2} \quad \quad \text{with }\theta(t)/2 \in[-\pi/2, \pi/2],\\
	= & \sqrt{a(t)}e^{i\theta'(t)} \quad \quad \text{with }\theta'(t) \in[-\pi, \pi]. 
\end{aligned} 
\end{equation}  
Since $\theta(t)/2$ is a discontinuous function, it is necessary to perform sign inversion on part of $\theta(t)/2$ to make the function continuous, which is denoted as $\theta'(t)$.


The optimized AFC-preparation pulse in our experiment is obtained with parameters $T_{prep}=8.018$ ms, $\beta=17.627/T_{prep}$, and $\Delta_f=\Delta/11$. The pulse is repeated 130 times, and the peak power of the AFC-preparation pulse is 142 $\mu$W before the cryostat. Finally, the 24-MHz AFC structure (Supplementary Figure \ref{fig:AFC}) is successfully prepared in the $\mathrm{^{153}Eu^{3+}}$ ion ensemble. 
Limited by the accuracy of comb preparation, the non-zero absorption between the AFC teeth results in a decrease in efficiency.

% Figure environment removed

As shown in Fig. 1\textbf{c} in the main text, the storage efficiency for 580-nm entangled photons is measured by coincidence counts triggered by paired 1537-nm photons. To reduce the time interval between the entangled photon pairs for more accurate measurement results, the data after 80.315-$\mu$s AFC storage (the blue line in Fig. 1\textbf{c}) is measured by delaying the 1537-nm entangled photons with the 7.9-km field-deployed fiber loop. The emergence of the additional small peak in the blue traces, as shown in Fig. 1\textbf{c} in the main text, can be attributed to the bandwidth of the input photons exceeding that of the AFC memory. Since the storage bandwidth of the AFC memory is square-enveloped, its Fourier transform in the temporal domain is a sinc function. This results in additional small peaks after the main peak in the temporal domain.


The bandwidth of 580-nm entangled photons is 178 MHz, which is still greater than the 24-MHz bandwidth of the memory. Therefore, the AFC memory essentially acts as a spectral filter for the quantum light source. We measure the input and the transmission of the quantum light, by preparing the absorption to a 24-MHz transparent window and a 24-MHz AFC, respectively (red and black lines in Fig. 1\textbf{c}). The random coincidences have been subtracted to accurately determine the storage efficiency for the quantum light at the bandwidth of 24 MHz. The measured AFC efficiency decays exponentially with the storage time with an effective AFC lifetime $T_{2}^{AFC} = 148 \pm 6$ $\mu$s, as shown in Supplementary Figure \ref{fig:mode_number_AFC_decay}\textbf{a}.  The storage time ($1/e$ intensity decay time) of the 24-MHz memory is 37 $\mu$s. It is worth noting that a decrease in efficiency was observed with the expansion of the AFC bandwidth, which may be ascribed to the effect of spectral diffusion. 
For the bandwidth-matched 580-nm photons, the efficiency is $3.2 \pm 0.1 \%$ for an 80.315-$\mu$s storage.


% Figure environment removed


The storage time demonstrated here is already close to the limit of excited-state storage, and longer storage times can be expected with spin-wave storage \cite{Ma2021One-hour,Ma2021Elimination}. 
We would need to choose an appropriate lambda system and select out one class of ions for implementing spin-wave storage. For the $\mathrm{^{153}Eu^{3+}}$:$\mathrm{Y_2SiO_5}$ crystals in the background of the Earth’s magnetic field, if we use a similar lambda system and apply a similar methodology as in Ref. \cite{SU2022ondemand} to design the pumping sequence, the spin-wave storage bandwidth can reach 29.1 MHz. This lambda system contains the $|\pm1/2\rangle_g \rightarrow |\pm5/2\rangle_e$ transition for the absorption of signal photons, the $|\pm5/2\rangle_g \rightarrow |\pm5/2\rangle_e$ transition for the control field, and the auxiliary $|\pm3/2\rangle_g \rightarrow |\pm3/2\rangle_e$ transition for class-cleaning and spin-polarization.
Due to the extended spin coherence time of $\mathrm{Eu^{3+}}$ in $\mathrm{Y_2SiO_5}$ crystals, the storage time for spin-wave quantum memory has been demonstrated to reach several tens of milliseconds \cite{Ortu2022Storage}, with an upper limit extending up to a few hours \cite{Zhong2015Optically,Ma2021One-hour}, which could lead to a quantum storage performance far beyond the capabilities of fiber delay loops. Moreover, spin-wave quantum memory provides on-demand retrieval \cite{Jobez2015Coherent} and enables real-time manipulation of photonic qubits \cite{Yang2018Multiplexed}, which are crucial for quantum repeaters and large-scale quantum networks \cite{Lei2023Quantum}.


We can define a single mode duration of 72 ns that completely covers the retrieved echo, there are 79 $\mu$s / 72 ns = 1097 temporal modes simultaneously stored in the memory crystal \cite{Lago-Rivera2021qp,liu2021heralded, Businger2022Non-classical}. As shown in Supplementary Figure \ref{fig:mode_number_AFC_decay}\textbf{b}, temporal multimode storage enables a linear enhancement of the rate for quantum gate teleportation.
In our implementation, the multimode capacity surpasses the performance of other ensemble-based quantum memories, such as cold atoms \cite{Parniak2017Wavevector,Pu2017Experimental} and hot vapor \cite{Chrapkiewicz2017High}. However, the storage efficiency is currently lower than that of cold atoms and hot vapor \cite{Wang2019Efficient,Ma2022High}. This is primarily caused by the weak absorption of the $\mathrm{^{153}Eu^{3+}}$:$\mathrm{Y_2SiO_5}$ crystals, and the 54\% theoretical limit posed by the AFC memory for retrieval in the forward direction \cite{Afzelius2009Multimode}. To address this limitation, cavity-enhanced AFC memory could be utilized to improve storage efficiency \cite{Afzelius2010Impedance,duranti2023efficient}.




\subsection{6. Optical losses}

Due to the low duty cycle of memory operations (3.7\%), the coincidence count rate of the entire experiment is approximately 0.042 Hz. After quantum storage, the retrieved echo has a signal-to-noise ratio of 12.6:1. The data collection times for the Deutsch-Jozsa algorithm and quantum phase estimation algorithm in our experiment were 3.8 hours and 17.1 hours, respectively.


The collection efficiency of the unfiltered SPDC source is approximately $42\%$. The heralding efficiency of the narrowband SPDC sources is $4.9\%$. The main optical losses are listed here. The peak transmittance of the cascaded etalons is approximately 63\%, both for 580 nm and 1537 nm. The detection efficiency is $80\%$ and $91\%$ for photons at 580 nm and 1537 nm, respectively. The 7.9 km field-deployed optical fiber introduces a loss of 2.2 dB, which includes a transmission loss of 1.6 dB and a coupling loss of 0.6 dB. The whole fiber channel can maintain a polarization extinction ratio of greater than 100:1 for 24 hours (see Supplementary Figure \ref{fig:polarization}).
 
The source generates energy-time entanglement, the interferometers select two time bins from the general superposition and each unbalanced interferometer introduces postselection and coupling loss. The coupling efficiency is $71\%$ for 580 nm and $62\%$ for 1537 nm.

The internal storage efficiency for 24-MHz quantum light is $3.2\%$ and the total transmission of the optical setup for the quantum memory is approximately $65\%$. In addition, there is a bandwidth match between the quantum light source and the quantum memory, and approximately $8\%$ of the generated photons are in the bandwidth of the memory.

The success probabilities or efficiencies of each step and the losses of each component are also summarized in Supplementary Table \ref{tabloss}.

\begin{table}
    \centering
    \renewcommand\arraystretch{1.22}
    \begin{tabular}{|c|c|}
\hline \textbf{Steps and components} & \textbf{Success probabilities or losses} \\
\hline
Success probability of the gate teleportation scheme & $50\%$\\
\hline
The postselection of unbalanced interferometer (for 580 nm or 1537 nm) & $50\%$\\
\hline 
Collection efficiency of the unfiltered SPDC source &  $42\%$ \\
\hline
Heralding efficiency of the narrowband SPDC sources & $4.9\%$\\
\hline
Peak transmittance of the cascaded etalons (for 580 nm or 1537 nm) & $63\%$\\
\hline
Loss of the 7.9 km field-deployed optical fiber & 2.2 dB\\
\hline
Coupling efficiency of the unbalanced interferometer (for 580 nm) & $71\%$\\
\hline
Coupling efficiency of the unbalanced interferometer (for 1537 nm) & $62\%$\\
\hline
Internal storage efficiency quantum memory for 24-MHz quantum light & $3.2\%$\\
\hline
Total transmission of the optical setup for quantum memory & $65\%$\\
\hline
Bandwidth matching efficiency between photons and the quantum memory & $8\%$\\
\hline
Detection efficiency (for 580 nm) & $80\%$\\
\hline
Detection efficiency (for 1537 nm) & $91\%$\\
\hline
\end{tabular}
    \caption{The success probabilities or efficiencies of each step and the losses of each component.}
    \label{tabloss}
\end{table}

% Figure environment removed


\subsection{7. Synchronization of the entanglement distribution system }

% Figure environment removed

We achieve synchronization between node A and node B through electro-optic and photoelectric conversion (see Supplementary Figure \ref{fig:synchronous}), with a synchronization accuracy of 220 ps. At node B, the four signals (S1, S2, S3 and S4) from the SNSPD are converted into optical pulses at wavelengths of 1550.92 nm, 1550.12 nm, 1549.32 nm, and 1548.52 nm, respectively.
These optical pulses are then transmitted through a 7.9-km optical fiber to node A, where they are converted back into electrical signals using avalanche photodiodes (APDs). The time-to-digital converter (TDC) is utilized to convert the time tag signals from both node A (G1, G2) and node B (S1, S2, S3, S4) into digital signals, enabling coincidence readout. To realize LOCC, a field programmable gate array (FPGA) is employed to determine the appropriate operation based on the signal received from node B. The electro-optical modulator (EOM) in node A performs $\sigma_x$ operation when receiving signals from S1 and S2, and $I$ operation when receiving signals from S3 and S4.

\subsection{8. Unbalanced interferometer locking}

Here, we will provide a detailed description of the phase locking methods used for both the 1537-nm and 580-nm unbalanced interferometers (see Supplementary Figure \ref{fig:lock_phase}). We use a single-photon-level reference light to lock the phase of the unbalanced interferometer. The reference light counter-propagates with the signal light to avoid introducing too much noise. Phase compensation is realized by a piezoelectric transducer (PZT), which drives a mirror to move in free space. All fibers in the interferometer are equipped with FC-APC connectors to minimize reflections. We must ensure $\tau_{\text {delay}} \gg \tau_{\text {photon}}$, where $\tau_{\text {delay}}$ corresponds to the time delay caused by the difference between the long and short arms, and $ \tau_{\text {photon}}$ corresponds to the coherence time of single photons, which is related to the photon bandwidth (155 MHz for 1537-nm photons and 24 MHz for 580-nm photons). Therefore, we use 30-m long optical fibers to build unbalanced interferometers to meet such conditions. For effective phase locking, we carefully select the reference light to have the same wavelength as the signal light.


% Figure environment removed

Without phase locking, the phase of the unbalanced interferometer varies by 2$\pi$ at a rate of approximately 0.1 Hz. We set the PZT locking frequency to 10 Hz to lock the interferometer and the phase locking can be stable for over 24 hours. 
The background noise finally introduced by the reference light is less than 100 Hz and 6 Hz for 580 nm and 1537 nm, respectively, which has negligible effects on the signal-to-noise ratio in our experiments. 



\subsection{9. Encoding converter and local CNOT operation, C-\textit{U} operation}

Energy conservation guarantees that two photons in a pair are generated at the same time, and the generation time of the photon pair is uncertain within the coherence time of the pump laser, resulting in time-energy entanglement \cite{franson1989bell,clausen2011quantum}. The time-energy entanglement is ensured as long as $\tau_{\text {pump }} \gg \tau_{\text {pair }}$, where $\tau_{\text {pump}}\left(\tau_{\text {pair}}\right)$ corresponds to the coherence time of the pump laser light (photon pair). As shown in Supplementary Figure \ref{fig:Encoding converter}, we divide photons into long paths and short paths with equal probability through HWPs (HWP1 and HWP3 ) set to 22.5 degrees and PBSs (PBS1 and PBS3). We generate long ($|L\rangle$) and short ($|S\rangle$) paths using a 30-m unbalanced interferometer. At this point, the entangled state of the two photons is

\begin{align}
    |\psi\rangle=1/\sqrt{2}(|S\rangle_2|S\rangle_3+|L\rangle_2|L\rangle_3).
\end{align}

We also use PBS2 to convert $|S\rangle$ into $|H\rangle$ and $|L\rangle$ into $|V\rangle$ at 1550 nm. The entangled state of the two photons thus becomes

\begin{align}
    |\psi\rangle=1/\sqrt{2}(|S\rangle_2|H\rangle_3+|L\rangle_2|V\rangle_3).
\end{align}

Afterwards, we distribute polarized photons to node B through an external field fiber.
At nodes A and B, we encode qubits 1, 2 and 3, 4 using photons with different degrees of freedom, and implement $C_{21}$ and $C_{43}$ operations, respectively.

% Figure environment removed



In Supplementary Figure \ref{fig:C-NOT_of_580nm_and_1537nm}, at node B, we convert the polarization DOF into path DOF through BD1. At the same time, we define $|S\rangle$ and $|L\rangle$ as path DOFs $|0\rangle$ and $|1\rangle$ at node A, and the entangled state becomes path-entangled state:

\begin{align}
    |\psi\rangle=1/\sqrt{2}(|0\rangle_2|0\rangle_3+|1\rangle_2|1\rangle_3).
\end{align}

In Supplementary Figure \ref{fig:C-NOT_of_580nm_and_1537nm}, wave plates are employed to encode qubit 1 and qubit 4 in the polarization DOF of two photons. The quantum state of the whole system can be written as
\begin{align}
    |\psi\rangle_{1234}=(\alpha_1|H\rangle+\beta_1|V\rangle)_1[(|0\rangle|0\rangle+|1\rangle|1\rangle)_{23}/\sqrt{2}] (\alpha_4|H\rangle+\beta_4|V\rangle)_4, \label{qubit1234}
\end{align}
where the subscripts represent different qubits. Qubits 1 and 2 are carried by photons at 580 nm, while qubits 3 and 4 by photons at 1537 nm.
 
In Supplementary Figure \ref{fig:C-NOT_of_580nm_and_1537nm}\textbf{a}, we directly encode the path and polarization qubits 1 and 2 in the unbalanced interferometer. In node A, we implement the CNOT operation of the path qubit on the polarization qubit by setting an HWP of $45^{\circ}$ in path 1 (if we want to achieve C-\textit{U} operation of the path to polarization DOF, we only need to use the combination of HWPs and QWPs to achieve \textit{U}-operation of polarization DOF in path 1). Afterward, we use polarization-independent beamsplitters to achieve measurements of the path DOF on $(|0\rangle+|1\rangle)/\sqrt{2}$ and $(|0\rangle-|1\rangle)/\sqrt{2}$ bases.


% Figure environment removed 

In Supplementary Figure \ref{fig:C-NOT_of_580nm_and_1537nm}\textbf{b}, we use a stable interferometer based on beam displacers (BD) to achieve path and polarization qubit encodings and operations. This interferometer is robust to noise and is suitable for out-field laboratories with poor conditions. In node B, we implement the CNOT operation of the polarization qubit on the path qubit.  
If the polarization is $H$, the path state remains unchanged, and when the polarization state is $V$, the path state flips. 


\subsection{10. The implementation of the Deutsch-Jozsa and phase estimation algorithm}

Here we will provide a detailed introduction to the experimental implementation of these two algorithms. The Deutsch-Jozsa and phase estimation algorithms rely on the constructed nonlocal CNOT and C-\textit{U} gates, respectively. In the Deutsch-Jozsa algorithm, as shown in Fig. 3\textbf{a} of the main text, we implement four operations between A$_1$ and B$_4$: identity (ID) and NOT operations represent the constant function; CNOT and zero-CNOT (ZCNOT) operations represent the balanced function. For identity (ID) and NOT operations, there is no need for the A$_1$ qubit to perform nonlocal two-qubit operations on the B$_4$ qubit. ID operation does not require any operations on A$_1$ and B$_4$, and NOT operation only requires a bit flip operation on B$_4$ (HWP@$45^{\circ}$). Entangled resources are not needed for the constant functions, so we directly distribute two photons that are in the direct product of polarization DOF to A$_1$ and B$_4$, thus achieving the two functions. For CNOT and ZCNOT operations, we can use the nonlocal CNOT operations introduced earlier. ZCNOT operation requires performing bit flipping on B$_4$ based on the CNOT gate. In the Deutsch-Jozsa algorithm, the Fourier transform ($F$) can be achieved by setting the HWP at $22.5^{\circ}$ for polarization qubit. It is necessary to perform Fourier transform and inverse Fourier transform ($F^{-1}$) on the inputs and outputs of A$_1$ and B$_4$.

The quantum phase estimation algorithm is used to estimate the phase of an operator acting on an eigenstate and is frequently used as a subroutine in other quantum algorithms, such as factorization. In this algorithm, the quantum state register consists of a unitary operator $U$ with an eigenstate $|\psi\rangle$ ($U|\psi\rangle=\mathrm{e}^{\mathrm{i}2\pi \varphi}|\psi\rangle$), and information about the unitary operator $U$ is encoded on the measurement register through multiple controlled-$U^{2^{k}}$ operations with $k$ an integer. In the phase estimation algorithm, to calculate the probability of successfully determining each bit correctly, we initially assume that the phase, denoted by $\varphi$, can be expressed as a binary number with no more than $m$ bits: $\varphi=(0.\varphi_1\varphi_2...\varphi_m000...)$. During the first iteration ($k=m$), a controlled-$U^{2^m}$ gate is applied, targeting the $m$th bit of the expansion. The probability of measuring ``0" is $P_0 = \cos^2\left[\pi\left(0.\varphi_{m-1}00\ldots\right)\right]$, which equals unity when $\varphi_m = 0$ and zero when $\varphi_m = 1$. Consequently, the first bit $\varphi_m$ is extracted deterministically. In the subsequent iteration ($k=m-1$), the measurement focuses on the $(m-1)$th bit. The phase of the first qubit before the $Z$ rotation equals $2\pi\left(0.\varphi_{m-1}\varphi_m00\ldots\right)$. If we denote the first $m$ bits of the binary expansion of $\varphi$ as $\widetilde{\varphi} = 0.\varphi_1\varphi_2\ldots\varphi_m$, there generally exists a remainder $\delta$, where $\delta<1$, defined by $\varphi = \widetilde{\varphi} + \delta 2^{-m}$. Under these conditions, the probability of measuring $\varphi_m$ is $\cos^2(\pi \delta / 2)$. If $\varphi_{m}$ is measured correctly, the probability of accurately measuring $\varphi_{m-1}$ in the second iteration becomes $\cos^2(\pi \delta / 4)$, and so forth.

The experimental setup is similar to that of the Deutsch-Jozsa algorithm, as illustrated in Fig. 3\textbf{f} in the main text. According to Eq. \ref{CU}, in order to achieve a $50\%$ success probability of C-\textit{U} gate using LOCC, we use A$_1$ to control B$_4$ instead of B$_4$ to control A$_1$. By employing the C-\textit{U} gate and Fourier transform mentioned above, we can achieve a single-cycle phase estimation algorithm. We use a liquid crystal phaser to provide feedback on the phase information measured in each cycle ($Z_{\varphi}$). After 3 cycles, we can implement a 3-bit binary phase estimation algorithm.


\subsection{11. Detailed experimental results}

Here, we provide more detailed data about the experimental results presented in the main text.

The truth table of the remote CNOT gate is detailed in the Supplementary Table \ref{table1}.

The real parts of the key matrix elements for four Bell states generated by the remote CNOT gates are provided in Eq. \ref{eq1}-\ref{eq4}.
\begin{equation}
|\Phi^+\rangle=\left(\begin{array}{rrrr}
0.429 & 0 & 0 & 0.376 \\
0 & 0.064 & 0 & 0 \\
0 & 0 & 0.064 & 0 \\
0.376 & 0 & 0 & 0.443
\end{array}\right),
\label{eq1}
\end{equation}

\begin{equation}
|\Phi^-\rangle=\left(\begin{array}{rrrr}
0.472 & 0 & 0 & -0.406 \\
0 & 0.074 & 0 & 0 \\
0 & 0 & 0.037 & 0 \\
-0.406 & 0 & 0 & 0.417
\end{array}\right),
\label{eq2}
\end{equation}

\begin{equation}
|\Psi^+\rangle=\left(\begin{array}{rrrr}
0.056 & 0 & 0 & 0 \\
0 & 0.400 & 0.360 & 0 \\
0 & 0.361 & 0.504 & 0 \\
0 & 0 & 0 & 0.040
\end{array}\right),
\label{eq3}
\end{equation}

\begin{equation}
|\Psi^-\rangle=\left(\begin{array}{rrrr}
0.077 & 0 & 0 & 0 \\
0 & 0.505 & -0.357 & 0 \\
0 & -0.357 & 0.385 & 0 \\
0 & 0 & 0 & 0.034
\end{array}\right).
\label{eq4}
\end{equation}


The detailed data of the Deutsch-Jozsa algorithm are shown in the Supplementary Table \ref{table2}.

Detailed data for the phase estimation algorithm are shown in the Supplementary Table \ref{table3}.

\begin{table}[H]
\centering 
\begin{tabular}{ccccc}
\hline & $HH$ & $HV$ & $VH$ & $VV$ \\
\hline $HH$ & $0.894\pm 0.021$ & $0.106\pm0.020$ & 0 & 0 \\
\hline $HV$ & $0.106\pm 0.021$ & $0.894\pm0.020$ & 0 & 0 \\
\hline $VH$ & 0 & 0 & $0.129\pm0.023$ & $0.889\pm0.021$ \\
\hline $VV$ & 0 & 0 & $0.871\pm0.023$ & $0.111\pm0.021$ \\
\hline
\end{tabular}
\caption{Experimental data of truth table the CNOT gate. The horizontal axis represents the input, and the vertical axis represents the output. 0 represents a photon detection probability of 0 for this setting of measurements in the experiment. Compared to the theoretical truth table, the CNOT gate has a high fidelity of $0.887\pm0.021$.}
\label{table1}
\end{table}


\begin{table}[H]
\begin{equation}
\begin{array}{|c|cccc|}
\hline & \text { ID } & \text { NOT } & \text { CNOT } & \text { ZCNOT } \\
\hline H & 0.987 & 0.993 & 0.061 & 0.086 \\
\hline V & 0.013 & 0.007 & 0.939 & 0.914 \\
\hline \text {Errors} & 0.009 & 0.007 & 0.009 & 0.009 \\
\hline
\end{array}
\nonumber
\end{equation}
\caption{Experimental data of the Deutsch-Jozsa algorithm. The table gives the probabilities for measuring the polarization state of $H$ and $V$. The horizontal axis represents the four operations, ID, NOT, CNOT, and ZCNOT. The probabilities that the results are either $H$ or $V$ determine whether the function is balanced or constant. It can be seen that for each type of function, we can make a correct judgment with a probability of at least 0.91.}
\label{table2}
\end{table}


\begin{table}[H]
\begin{equation}
\begin{array}{|c|ccc|ccc|ccc|ccc|}
\hline & 0 & 0 & 0 & 0 & 1 & 0 & 1 & 0 & 1 & 1 & 1 & 0 \\
\hline 0 & 0.876 & 0.883 & 0.913 & 0.861 & 0.097 & 0.812 & 0.182 & 0.828 & 0.153 & 0.162 & 0.257 & 0.844 \\
\hline 1 & 0.124 & 0.117 & 0.087& 0.139 & 0.903 & 0.188 & 0.818 & 0.172 & 0.847 & 0.838 & 0.743 & 0.156 \\
\hline \text {Errors} & 0.035 & 0.033 & 0.028 & 0.033 & 0.026 & 0.037 & 0.033 & 0.038 & 0.033 & 0.035 & 0.030 & 0.037 \\
\hline
\end{array}
\nonumber
\end{equation}
\caption{Experimental data of phase estimation algorithm. The phase estimation of four unitary operations ($U=I$, $Z^{1/2}$, $Z^{5/4}$, and $Z^{3/2}$) correspond to phase shifts of $\varphi$ = 0, $\pi/2$, $5\pi/4$, $3\pi/2$, respectively. These phases can be accurately represented as $\varphi=2\pi \times (0.000,0.010,0.101,0.110)$ using 3-digit binary numbers. We estimate the phase of the corresponding $Z$-phase gate with three significant digits and obtain the probability that each digit is 0 or 1 by polarization measurement of the register. When the probability for 0 or 1 is greater than 1/2, this digit is selected. According to the data, we can estimate the phase of the four $Z$-phase gates with $100\%$ accuracy.} 
\label{table3}
\end{table}


\begin{thebibliography}{10}
\expandafter\ifx\csname url\endcsname\relax
  \def\url#1{\texttt{#1}}\fi
\expandafter\ifx\csname urlprefix\endcsname\relax\def\urlprefix{URL }\fi
\providecommand{\bibinfo}[2]{#2}
\providecommand{\eprint}[2][]{\url{#2}}

\bibitem{Kwiat1998Embedded}
\bibinfo{author}{Kwiat, P.~G.} \& \bibinfo{author}{Weinfurter, H.}
\newblock \bibinfo{title}{Embedded bell-state analysis}.
\newblock \emph{\bibinfo{journal}{Physical Review A}} \textbf{\bibinfo{volume}{58}}, \bibinfo{pages}{R2623--R2626} (\bibinfo{year}{1998}).

\bibitem{Niemietz2021Nondestructive}
\bibinfo{author}{Niemietz, D.}, \bibinfo{author}{Farrera, P.}, \bibinfo{author}{Langenfeld, S.} \& \bibinfo{author}{Rempe, G.}
\newblock \bibinfo{title}{Nondestructive detection of photonic qubits}.
\newblock \emph{\bibinfo{journal}{Nature}} \textbf{\bibinfo{volume}{591}}, \bibinfo{pages}{570--574} (\bibinfo{year}{2021}).

\bibitem{BrienBrienNondestructive}
\bibinfo{author}{O'Brien, C.}, \bibinfo{author}{Zhong, T.}, \bibinfo{author}{Faraon, A.} \& \bibinfo{author}{Simon, C.}
\newblock \bibinfo{title}{Nondestructive photon detection using a single rare-earth ion coupled to a photonic cavity}.
\newblock \emph{\bibinfo{journal}{Physical Review A}} \textbf{\bibinfo{volume}{94}}, \bibinfo{pages}{043807} (\bibinfo{year}{2016}).

\bibitem{Nemoto2004Nearly}
\bibinfo{author}{Nemoto, K.} \& \bibinfo{author}{Munro, W.~J.}
\newblock \bibinfo{title}{Nearly deterministic linear optical controlled-not gate}.
\newblock \emph{\bibinfo{journal}{Physical Review Letters}} \textbf{\bibinfo{volume}{93}}, \bibinfo{pages}{250502} (\bibinfo{year}{2004}).

\bibitem{Chang2014Quantum}
\bibinfo{author}{Chang, D.~E.}, \bibinfo{author}{Vuleti{\'{c}}, V.} \& \bibinfo{author}{Lukin, M.~D.}
\newblock \bibinfo{title}{Quantum nonlinear optics --- photon by photon}.
\newblock \emph{\bibinfo{journal}{Nature Photonics}} \textbf{\bibinfo{volume}{8}}, \bibinfo{pages}{685--694} (\bibinfo{year}{2014}).

\bibitem{Ren2013Deterministic}
\bibinfo{author}{Ren, B.-C.}, \bibinfo{author}{Wei, H.-R.} \& \bibinfo{author}{Deng, F.-G.}
\newblock \bibinfo{title}{Deterministic photonic spatial-polarization hyper-controlled-not gate assisted by a quantum dot inside a one-side optical microcavity}.
\newblock \emph{\bibinfo{journal}{Laser Physics Letters}} \textbf{\bibinfo{volume}{10}}, \bibinfo{pages}{095202} (\bibinfo{year}{2013}).

\bibitem{Imany2019High}
\bibinfo{author}{Imany, P.} \emph{et~al.}
\newblock \bibinfo{title}{High-dimensional optical quantum logic in large operational spaces}.
\newblock \emph{\bibinfo{journal}{npj Quantum Information}} \textbf{\bibinfo{volume}{5}}, \bibinfo{pages}{59} (\bibinfo{year}{2019}).

\bibitem{Kwiat1997Hyper}
\bibinfo{author}{Kwiat, P.~G.}
\newblock \bibinfo{title}{Hyper-entangled states}.
\newblock \emph{\bibinfo{journal}{Journal of Modern Optics}} \textbf{\bibinfo{volume}{44}}, \bibinfo{pages}{2173--2184} (\bibinfo{year}{1997}).

\bibitem{DENG2017Quantum}
\bibinfo{author}{Deng, F.-G.}, \bibinfo{author}{Ren, B.-C.} \& \bibinfo{author}{Li, X.-H.}
\newblock \bibinfo{title}{Quantum hyperentanglement and its applications in quantum information processing}.
\newblock \emph{\bibinfo{journal}{Science Bulletin}} \textbf{\bibinfo{volume}{62}}, \bibinfo{pages}{46--68} (\bibinfo{year}{2017}).

\bibitem{Sleator1995Realizable}
\bibinfo{author}{Sleator, T.} \& \bibinfo{author}{Weinfurter, H.}
\newblock \bibinfo{title}{Realizable universal quantum logic gates}.
\newblock \emph{\bibinfo{journal}{Physical Review Letters}} \textbf{\bibinfo{volume}{74}}, \bibinfo{pages}{4087--4090} (\bibinfo{year}{1995}).

\bibitem{Knill2001scheme}
\bibinfo{author}{Knill, E.}, \bibinfo{author}{Laflamme, R.} \& \bibinfo{author}{Milburn, G.~J.}
\newblock \bibinfo{title}{A scheme for efficient quantum computation with linear optics}.
\newblock \emph{\bibinfo{journal}{Nature}} \textbf{\bibinfo{volume}{409}}, \bibinfo{pages}{46--52} (\bibinfo{year}{2001}).

\bibitem{Fleischhauer2000Dark-State}
\bibinfo{author}{Fleischhauer, M.} \& \bibinfo{author}{Lukin, M.~D.}
\newblock \bibinfo{title}{Dark-state polaritons in electromagnetically induced transparency}.
\newblock \emph{\bibinfo{journal}{Physical Review Letters}} \textbf{\bibinfo{volume}{84}}, \bibinfo{pages}{5094--5097} (\bibinfo{year}{2000}).

\bibitem{Moiseev2001Complete}
\bibinfo{author}{Moiseev, S.~A.} \& \bibinfo{author}{Kröll, S.}
\newblock \bibinfo{title}{Complete reconstruction of the quantum state of a single-photon wave packet absorbed by a doppler-broadened transition}.
\newblock \emph{\bibinfo{journal}{Physical Review Letters}} \textbf{\bibinfo{volume}{87}}, \bibinfo{pages}{173601} (\bibinfo{year}{2001}).

\bibitem{Duan2001Long-distance}
\bibinfo{author}{Duan, L.~M.}, \bibinfo{author}{Lukin, M.~D.}, \bibinfo{author}{Cirac, J.~I.} \& \bibinfo{author}{Zoller, P.}
\newblock \bibinfo{title}{Long-distance quantum communication with atomic ensembles and linear optics}.
\newblock \emph{\bibinfo{journal}{Nature}} \textbf{\bibinfo{volume}{414}}, \bibinfo{pages}{413--418} (\bibinfo{year}{2001}).

\bibitem{Afzelius2009Multimode}
\bibinfo{author}{Afzelius, M.}, \bibinfo{author}{Simon, C.}, \bibinfo{author}{de~Riedmatten, H.} \& \bibinfo{author}{Gisin, N.}
\newblock \bibinfo{title}{Multimode quantum memory based on atomic frequency combs}.
\newblock \emph{\bibinfo{journal}{Physical Review A}} \textbf{\bibinfo{volume}{79}}, \bibinfo{pages}{052329} (\bibinfo{year}{2009}).

\bibitem{Ma2021Elimination}
\bibinfo{author}{Ma, Y.-Z.} \emph{et~al.}
\newblock \bibinfo{title}{Elimination of noise in optically rephased photon echoes}.
\newblock \emph{\bibinfo{journal}{Nature Communications}} \textbf{\bibinfo{volume}{12}}, \bibinfo{pages}{4378} (\bibinfo{year}{2021}).

\bibitem{Konz2003Temperature}
\bibinfo{author}{K\"onz, F.} \emph{et~al.}
\newblock \bibinfo{title}{Temperature and concentration dependence of optical dephasing, spectral-hole lifetime, and anisotropic absorption in $\mathrm{Eu}^{3+}:\mathrm{Y}_{2}\mathrm{SiO}_{5}$}.
\newblock \emph{\bibinfo{journal}{Physical Review B}} \textbf{\bibinfo{volume}{68}}, \bibinfo{pages}{085109} (\bibinfo{year}{2003}).

\bibitem{Jobez2014Cavity}
\bibinfo{author}{Jobez, P.} \emph{et~al.}
\newblock \bibinfo{title}{Cavity-enhanced storage in an optical spin-wave memory}.
\newblock \emph{\bibinfo{journal}{New Journal of Physics}} \textbf{\bibinfo{volume}{16}}, \bibinfo{pages}{083005} (\bibinfo{year}{2014}).

\bibitem{Zhong2015Optically}
\bibinfo{author}{Zhong, M.} \emph{et~al.}
\newblock \bibinfo{title}{Optically addressable nuclear spins in a solid with a six-hour coherence time}.
\newblock \emph{\bibinfo{journal}{Nature}} \textbf{\bibinfo{volume}{517}}, \bibinfo{pages}{177--180} (\bibinfo{year}{2015}).

\bibitem{Ma2021One-hour}
\bibinfo{author}{Ma, Y.}, \bibinfo{author}{Ma, Y.-Z.}, \bibinfo{author}{Zhou, Z.-Q.}, \bibinfo{author}{Li, C.-F.} \& \bibinfo{author}{Guo, G.-C.}
\newblock \bibinfo{title}{One-hour coherent optical storage in an atomic frequency comb memory}.
\newblock \emph{\bibinfo{journal}{Nature Communications}} \textbf{\bibinfo{volume}{12}}, \bibinfo{pages}{2381} (\bibinfo{year}{2021}).

\bibitem{Timoney2013Single}
\bibinfo{author}{Timoney, N.}, \bibinfo{author}{Usmani, I.}, \bibinfo{author}{Jobez, P.}, \bibinfo{author}{Afzelius, M.} \& \bibinfo{author}{Gisin, N.}
\newblock \bibinfo{title}{Single-photon-level optical storage in a solid-state spin-wave memory}.
\newblock \emph{\bibinfo{journal}{Physical Review A}} \textbf{\bibinfo{volume}{88}}, \bibinfo{pages}{022324} (\bibinfo{year}{2013}).

\bibitem{SU2022ondemand}
\bibinfo{author}{Su, M.-X.} \emph{et~al.}
\newblock \bibinfo{title}{On-demand multimode optical storage in a laser-written on-chip waveguide}.
\newblock \emph{\bibinfo{journal}{Physical Review A}} \textbf{\bibinfo{volume}{105}}, \bibinfo{pages}{052432} (\bibinfo{year}{2022}).
\newblock \bibinfo{note}{PRA}.

\bibitem{Tittel2010Photon-echo}
\bibinfo{author}{Tittel, W.} \emph{et~al.}
\newblock \bibinfo{title}{Photon-echo quantum memory in solid state systems}.
\newblock \emph{\bibinfo{journal}{Laser \& Photonics Reviews}} \textbf{\bibinfo{volume}{4}}, \bibinfo{pages}{244--267} (\bibinfo{year}{2010}).

\bibitem{Zhu2022On-Demand}
\bibinfo{author}{Zhu, T.-X.} \emph{et~al.}
\newblock \bibinfo{title}{On-demand integrated quantum memory for polarization qubits}.
\newblock \emph{\bibinfo{journal}{Physical Review Letters}} \textbf{\bibinfo{volume}{128}}, \bibinfo{pages}{180501} (\bibinfo{year}{2022}).

\bibitem{Jobez2016Towards}
\bibinfo{author}{Jobez, P.} \emph{et~al.}
\newblock \bibinfo{title}{Towards highly multimode optical quantum memory for quantum repeaters}.
\newblock \emph{\bibinfo{journal}{Physical Review A}} \textbf{\bibinfo{volume}{93}}, \bibinfo{pages}{032327} (\bibinfo{year}{2016}).

\bibitem{Businger2022Non-classical}
\bibinfo{author}{Businger, M.} \emph{et~al.}
\newblock \bibinfo{title}{Non-classical correlations over 1250 modes between telecom photons and 979-nm photons stored in $^{171}\mathrm{Yb}^{3+}:\mathrm{Y}_{2}\mathrm{SiO}_{5}$}.
\newblock \emph{\bibinfo{journal}{Nature Communications}} \textbf{\bibinfo{volume}{13}}, \bibinfo{pages}{6438} (\bibinfo{year}{2022}).

\bibitem{Schroeder1970Synthesis}
\bibinfo{author}{Schroeder, M.}
\newblock \bibinfo{title}{Synthesis of low-peak-factor signals and binary sequences with low autocorrelation (corresp.)}.
\newblock \emph{\bibinfo{journal}{IEEE Transactions on Information Theory}} \textbf{\bibinfo{volume}{16}}, \bibinfo{pages}{85--89} (\bibinfo{year}{1970}).

\bibitem{Oswald2021Burning}
\bibinfo{author}{Oswald, R.}, \bibinfo{author}{Nevsky, A.~Y.} \& \bibinfo{author}{Schiller, S.}
\newblock \bibinfo{title}{Burning and reading ensembles of spectral holes by optical frequency combs: Demonstration in rare-earth-doped solids and application to laser frequency stabilization}.
\newblock \emph{\bibinfo{journal}{Physical Review A}} \textbf{\bibinfo{volume}{104}}, \bibinfo{pages}{063111} (\bibinfo{year}{2021}).

\bibitem{Ortu2022Storage}
\bibinfo{author}{Ortu, A.}, \bibinfo{author}{Holzäpfel, A.}, \bibinfo{author}{Etesse, J.} \& \bibinfo{author}{Afzelius, M.}
\newblock \bibinfo{title}{Storage of photonic time-bin qubits for up to 20 ms in a rare-earth doped crystal}.
\newblock \emph{\bibinfo{journal}{npj Quantum Information}} \textbf{\bibinfo{volume}{8}}, \bibinfo{pages}{29} (\bibinfo{year}{2022}).

\bibitem{Jobez2015Coherent}
\bibinfo{author}{Jobez, P.} \emph{et~al.}
\newblock \bibinfo{title}{Coherent spin control at the quantum level in an ensemble-based optical memory}.
\newblock \emph{\bibinfo{journal}{Physical Review Letters}} \textbf{\bibinfo{volume}{114}}, \bibinfo{pages}{230502} (\bibinfo{year}{2015}).
\newblock \bibinfo{note}{PRL}.

\bibitem{Yang2018Multiplexed}
\bibinfo{author}{Yang, T.-S.} \emph{et~al.}
\newblock \bibinfo{title}{Multiplexed storage and real-time manipulation based on a multiple degree-of-freedom quantum memory}.
\newblock \emph{\bibinfo{journal}{Nature Communications}} \textbf{\bibinfo{volume}{9}}, \bibinfo{pages}{3407} (\bibinfo{year}{2018}).

\bibitem{Lei2023Quantum}
\bibinfo{author}{Lei, Y.} \emph{et~al.}
\newblock \bibinfo{title}{Quantum optical memory for entanglement distribution}.
\newblock \emph{\bibinfo{journal}{Optica}} \textbf{\bibinfo{volume}{10}}, \bibinfo{pages}{1511--1528} (\bibinfo{year}{2023}).

\bibitem{Lago-Rivera2021qp}
\bibinfo{author}{Lago-Rivera, D.}, \bibinfo{author}{Grandi, S.}, \bibinfo{author}{Rakonjac, J.~V.}, \bibinfo{author}{Seri, A.} \& \bibinfo{author}{de~Riedmatten, H.}
\newblock \bibinfo{title}{Telecom-heralded entanglement between multimode solid-state quantum memories}.
\newblock \emph{\bibinfo{journal}{Nature}} \textbf{\bibinfo{volume}{594}}, \bibinfo{pages}{37--40} (\bibinfo{year}{2021}).

\bibitem{liu2021heralded}
\bibinfo{author}{Liu, X.} \emph{et~al.}
\newblock \bibinfo{title}{Heralded entanglement distribution between two absorptive quantum memories}.
\newblock \emph{\bibinfo{journal}{Nature}} \textbf{\bibinfo{volume}{594}}, \bibinfo{pages}{41--45} (\bibinfo{year}{2021}).

\bibitem{Parniak2017Wavevector}
\bibinfo{author}{Parniak, M.} \emph{et~al.}
\newblock \bibinfo{title}{Wavevector multiplexed atomic quantum memory via spatially-resolved single-photon detection}.
\newblock \emph{\bibinfo{journal}{Nature Communications}} \textbf{\bibinfo{volume}{8}}, \bibinfo{pages}{2140} (\bibinfo{year}{2017}).

\bibitem{Pu2017Experimental}
\bibinfo{author}{Pu, Y.-F.} \emph{et~al.}
\newblock \bibinfo{title}{Experimental realization of a multiplexed quantum memory with 225 individually accessible memory cells}.
\newblock \emph{\bibinfo{journal}{Nature Communications}} \textbf{\bibinfo{volume}{8}}, \bibinfo{pages}{15359} (\bibinfo{year}{2017}).

\bibitem{Chrapkiewicz2017High}
\bibinfo{author}{Chrapkiewicz, R.}, \bibinfo{author}{Dabrowski, M.} \& \bibinfo{author}{Wasilewski, W.}
\newblock \bibinfo{title}{High-capacity angularly multiplexed holographic memory operating at the single-photon level}.
\newblock \emph{\bibinfo{journal}{Physical Review Letters}} \textbf{\bibinfo{volume}{118}}, \bibinfo{pages}{063603} (\bibinfo{year}{2017}).

\bibitem{Wang2019Efficient}
\bibinfo{author}{Wang, Y.} \emph{et~al.}
\newblock \bibinfo{title}{Efficient quantum memory for single-photon polarization qubits}.
\newblock \emph{\bibinfo{journal}{Nature Photonics}} \textbf{\bibinfo{volume}{13}}, \bibinfo{pages}{346--351} (\bibinfo{year}{2019}).

\bibitem{Ma2022High}
\bibinfo{author}{Ma, L.} \emph{et~al.}
\newblock \bibinfo{title}{High-performance cavity-enhanced quantum memory with warm atomic cell}.
\newblock \emph{\bibinfo{journal}{Nature Communications}} \textbf{\bibinfo{volume}{13}}, \bibinfo{pages}{2368} (\bibinfo{year}{2022}).

\bibitem{Afzelius2010Impedance}
\bibinfo{author}{Afzelius, M.} \& \bibinfo{author}{Simon, C.}
\newblock \bibinfo{title}{Impedance-matched cavity quantum memory}.
\newblock \emph{\bibinfo{journal}{Physical Review A}} \textbf{\bibinfo{volume}{82}}, \bibinfo{pages}{022310} (\bibinfo{year}{2010}).

\bibitem{duranti2023efficient}
\bibinfo{author}{Duranti, S.} \emph{et~al.}
\newblock \bibinfo{title}{Efficient cavity-assisted storage of photonic qubits in a solid-state quantum memory} (\bibinfo{year}{2023}).
\newblock \eprint{Preprint at https://doi.org/10.48550/arXiv.2307.03509}.

\bibitem{franson1989bell}
\bibinfo{author}{Franson, J.~D.}
\newblock \bibinfo{title}{Bell inequality for position and time}.
\newblock \emph{\bibinfo{journal}{Physical Review Letters}} \textbf{\bibinfo{volume}{62}}, \bibinfo{pages}{2205} (\bibinfo{year}{1989}).

\bibitem{clausen2011quantum}
\bibinfo{author}{Clausen, C.} \emph{et~al.}
\newblock \bibinfo{title}{Quantum storage of photonic entanglement in a crystal}.
\newblock \emph{\bibinfo{journal}{Nature}} \textbf{\bibinfo{volume}{469}}, \bibinfo{pages}{508--511} (\bibinfo{year}{2011}).

\end{thebibliography}



\end{document}
