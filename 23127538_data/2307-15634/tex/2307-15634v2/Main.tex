\documentclass[nofootinbib,twocolumn,superscriptaddress,a4paper]{revtex4-1}
%twocolumn
%\usepackage{scicite}
\usepackage{graphicx}% Include figure files
%\usepackage{dcolumn}% Align table columns on decimal point
\usepackage{bm,bbm}% bold math
\usepackage{xcolor}%colours
\usepackage{tcolorbox}
\usepackage{algorithm}
\usepackage{algpseudocode}
\usepackage{amsmath}
\usepackage{booktabs} 
\usepackage[colorlinks,breaklinks]{hyperref}
\usepackage{lineno}


\makeatletter
\newcommand\org@hypertarget{}
\let\org@hypertarget\hypertarget
\renewcommand\hypertarget[2]{%
\Hy@raisedlink{\org@hypertarget{#1}{}}#2%
  }
\makeatother
\newcommand{\ket}[1]{\left\vert#1\right\rangle}
\newcommand{\bra}[1]{\left\langle #1 \right\vert}
\newcommand{\card}[1]{\left\vert #1 \right\vert}
\newcommand{\braket}[2]{\langle #1|#2\rangle}
\newcommand{\ketbra}[2]{| #1\rangle \langle #2|}
\newcommand{\xq}[1]{\textcolor{cyan}{#1}}  
\newcommand{\mpi}[1]{\textcolor{olive}{#1}}


\begin{document}



\title{Nonlocal photonic quantum gates over 7.0 km}
\author{Xiao Liu}
\email{These three authors contributed equally to this work.}
\affiliation{CAS Key Laboratory of Quantum Information, University of Science and Technology of China, Hefei 230026, China}
\affiliation{CAS Center for Excellence in Quantum Information and Quantum Physics, University of Science and Technology of China, Hefei 230026, China}
\author{Xiao-Min Hu}
\email{These three authors contributed equally to this work.}
\affiliation{CAS Key Laboratory of Quantum Information, University of Science and Technology of China, Hefei 230026, China}
\affiliation{CAS Center for Excellence in Quantum Information and Quantum Physics, University of Science and Technology of China, Hefei 230026, China}
\affiliation{Hefei National Laboratory, University of Science and Technology of China, Hefei 230088, China}
\author{Tian-Xiang Zhu}
\email{These three authors contributed equally to this work.}
\author{Chao Zhang}
\author{Yi-Xin Xiao}
\author{Jia-Le Miao}
\author{Zhong-Wen Ou}
\author{Pei-Yun Li}
\affiliation{CAS Key Laboratory of Quantum Information, University of Science and Technology of China, Hefei 230026, China}
\affiliation{CAS Center for Excellence in Quantum Information and Quantum Physics, University of Science and Technology of China, Hefei 230026, China}

\author{Bi-Heng Liu}
\email{bhliu@ustc.edu.cn}
\author{Zong-Quan Zhou}
\email{zq\_zhou@ustc.edu.cn}
\author{Chuan-Feng Li}
\email{cfli@ustc.edu.cn}
\author{Guang-Can Guo}
\affiliation{CAS Key Laboratory of Quantum Information, University of Science and Technology of China, Hefei 230026, China}
\affiliation{CAS Center for Excellence in Quantum Information and Quantum Physics, University of Science and Technology of China, Hefei 230026, China}
\affiliation{Hefei National Laboratory, University of Science and Technology of China, Hefei 230088, China}


\begin{abstract}
Quantum networks provide a prospective paradigm to connect separated quantum nodes, which relies on the distribution of long-distance entanglement and active feedforward control of qubits between remote nodes. Such approaches can be utilized to construct nonlocal quantum gates, forming building blocks for distributed quantum computing and other novel quantum applications. However, these gates have only been realized within single nodes or between nodes separated by a few tens of meters, limiting the ability to harness computing resources in large-scale quantum networks. Here, we demonstrate nonlocal photonic quantum gates between two nodes spatially separated by 7.0 km using stationary qubits based on multiplexed quantum memories, flying qubits at telecom wavelengths, and active feedforward control based on field-deployed fibers. Furthermore, we illustrate quantum parallelism by implementing the Deutsch-Jozsa algorithm and the quantum phase estimation algorithm between the two remote nodes. These results represent a proof-of-principle demonstration of quantum gates over metropolitan-scale distances and lay the foundation for the construction of large-scale distributed quantum networks relying on existing fiber channels.
\end{abstract}


\date{\today}
\maketitle



\section*{INTRODUCTION}
The development of large-scale quantum networks \cite{Kimble2008,wehner2018quantum}, comprising interconnected quantum nodes, has garnered significant attention as they provide powerful capabilities beyond the reach of classical counterparts, including global quantum communication \cite{Hensen2015Loophole,Yu2020Entanglement,vanLeent2022Entangling,Krutyanskiy2023Entanglement}, distributed quantum computing \cite{DiVincenzo1995Quantum,Ladd2010QC}, and quantum-enhanced sensing \cite{Komar2014A}. For distributed quantum computing, such network-based approaches can break the technical constraints of single quantum devices, providing an efficient method to scale up the systems \cite{Jiang2007distri,Monroe2014distri,li2024highrate}. Crucially, they can leverage the collective power of multiple remote nodes within expansive quantum networks to solve complex computational tasks efficiently \cite{Cirac1999distri,gottesman1999demonstrating,cuomo2020towards,Oh2023Distributed}, and could enable secure cloud quantum computing with completely classical clients \cite{Huang2017blind}. 

Towards distributed quantum networks, an essential function is the execution of nonlocal quantum gates across network nodes \cite{gottesman1999demonstrating,Eisert2000gate,Bartlett2003gate}, which can be implemented with the help of quantum teleportation, avoiding the direct interaction of two remote qubits. Analogous to quantum state teleportation, which has been demonstrated across diverse physical systems and implementations \cite{Pirandola2015review,Hu2023review}, quantum gate teleportation requires pre-shared entanglement between two separated nodes, local two-qubit gates, and active feedforward control of remote qubits through local operations and classical communications (LOCC). Such remote quantum gates have been realized with photonic qubits without LOCC \cite{huang2004optical1,gap2010optical2}, as well as two species of trapped ions \cite{wang2019trappedions} and two superconducting qubits \cite{Chou2018superconducting}, both within a single device. More recently, based on spin-photon quantum logic, a remote quantum gate has been demonstrated between two separated single atoms linked by a 60-m fiber in the same building \cite{Daiss2021singleatom}. A demonstration of nonlocal quantum gates over longer distances represents a great challenge in the construction of large-scale quantum computing networks. 


Here, we demonstrate remote quantum gates based on long-distance distributed photonic entanglement, LOCC enabled with long-lived quantum memories, and local two-qubit operations based on multiple-degree-of-freedom encodings on photons. We characterize the nonlocal controlled-NOT (CNOT) gate by measuring its truth table and the fidelity of four Bell states created from separable states. Furthermore, we use the established nonlocal quantum gates to execute the Deutsch-Jozsa algorithm \cite{deutsch1992rapid} and quantum phase estimation algorithm \cite{kitaev1995quantum}, demonstrating the prototype of distributed quantum computing over metropolitan-scale distances.


\section*{RESULTS}

% Figure environment removed

\textbf{Experimental Setup.} 
The main approach of our experiment is schematically shown in Fig. \ref{fig:setup}. Nondegenerate entangled photon pairsare generated through the spontaneous parametric down-conversion (SPDC) process in node A, which is located on the east campus of the University of Science and Technology of China. The signal photon at 580 nm is stored in a local quantum memory, while the idler photon at 1537 nm is sent along field-deployed fibers to node B, which is located on the foot of the mountain DaShuShan with a spatial separation of 7.0 km from node A. Here, we employ multiple degree of freedoms (DOFs) to encode four qubits with two photons to implement the teleportation-based nonlocal two-qubit gates \cite{gottesman1999demonstrating}. After performing local two-qubit gates at each node and successive LOCC, a nonlocal two-qubit gate is successfully implemented between the distant network nodes. Our protocol involves multiplexed operations in the time domain \cite{Lago-Rivera2021qp,liu2021heralded} and the gate teleportation rate is proportional to the number of stored modes in the quantum memory.

 

The nondegenerate entangled photon source is based on SPDC in a periodically-poled lithium niobate (PPLN) waveguide pumped by a laser at 421 nm, which is obtained by sum-frequency generation from stabilized and amplified lasers at 580 nm and 1537 nm. Both photons are spectrally filtered with two cascaded etalons to match the bandwidth of the quantum memory (see Section 4 in the Supplementary materials). Energy conservation guarantees that two photons are generated simultaneously, while the precise generation time of the photon pair remains uncertain within the coherence time of the pump laser, resulting in time-energy entanglement \cite{franson1989bell,clausen2011quantum}. After postselection through an unbalanced interferometer of 30 m, the entangled state $\frac{1}{\sqrt{2}}\left(\left|S_{\mathrm{s}} S_{\mathrm{i}}\right\rangle+\left|L_{\mathrm{s}} L_{\mathrm{i}}\right\rangle\right)$ is obtained, where the subscripts s and i represent the signal and idler photons, respectively. If we encode the short $\left|S_{\mathrm{s}, \mathrm{i}}\right\rangle$ and long $\left|L_{\mathrm{s}, \mathrm{i}}\right\rangle$ paths to $|0\rangle$, $|1\rangle$ path encoding, we can obtain the path-entangled state $\frac{1}{\sqrt{2}}(\left|00\right\rangle+\left|11\right\rangle)$. 
Another unbalanced interferometer (encoding converter) is used to convert 1537 nm photons from the time-energy DOF to the polarization DOF before sending them into the field-deployed ultralow-loss optical fiber (see Section 9 in the Supplementary materials). The fiber-optic cable is fixed in a protective underground duct system, maintaining low mechanical vibrations and temperature fluctuations, which enables the long-distance transmission of polarization-encoded photons. 


\textbf{Quantum memory and feedforward control.}
The quantum memory is implemented with the atomic frequency comb (AFC) protocol \cite{Afzelius2009Multimode} in a rare-earth-ion-doped crystal (REIC), i.e., 0.2\% doped $\mathrm{^{153}Eu^{3+}}$:$\mathrm{Y_2SiO_5}$ crystal. The isotope $\mathrm{^{153}Eu^{3+}}$ is chosen here to provide a larger storage bandwidth as compared to that of $\mathrm{^{151}Eu^{3+}}$ \cite{Jobez2014Cavity, Ma2021One-hour}. 
The $\mathrm{^{153}Eu^{3+}}$:$\mathrm{Y_2SiO_5}$ crystal is assembled on a close-cycle cryostat with a homemade vibration-isolated sample holder, which allows the preparation of high-resolution AFCs for long-lived and multiplexed photonic storage. The quantum memory should hold the 580-nm photons to wait for the transmission of 1537-nm photons from node A to node B (quantum communication, QC) and the feedback of successive measurement results from node B to node A (classical communication, CC). Both QC and CC are based on field-deployed optical fibers with a length of 7.9 km, which puts a lower bound for the storage time of 79 $\mu$s; therefore, we extend the storage time to 80.315 $\mu$s, which significantly outperforms previous results for photonic entanglement storage (47.7 $\mu$s in Ref. \cite{rakonjac2021entanglement}) in solid-state absorptive quantum memories \cite{clausen2011quantum, Puigibert2020Entanglement, Lago-Rivera2021qp, liu2021heralded, rakonjac2021entanglement, Businger2022Non-classical}. 
Optical memories based on REICs have shown the capability to coherently store light for 1 hour \cite{Longdell2005Stopped,Ma2021One-hour}, making them competitive among various absorptive and emissive quantum memory systems \cite{Lei2023Quantum}.

The prepared AFC has a total bandwidth of $24$ MHz, and the storage efficiency for the bandwidth-matched SPDC source is $(3.2 \pm 0.1) \%$ (Fig. \ref{fig:setup}\textbf{c}). 
Given a single mode duration of 72 ns which covers the retrieved echo peak and the input window length of  79 $\mu$s, the number of temporal modes stored in the memory is 79 $\mu$s / 72 ns = 1097, which results in a linear enhancement of the rate for quantum gate teleportation as compared to the case of employing a single-mode quantum memory  (see Supplementary Fig. 10 in Supplementary materials). 


The four qubits employed in the teleportation-based two-qubit gates are denoted as A$_1$, A$_2$ in node A and B$_3$, B$_4$ in node B, where qubits A$_2$ and B$_3$ are prepared in an entangled state in the path DOF of photons, denoted as $|\Phi\rangle_{23}=\frac{1}{\sqrt{2}}\left(|00\rangle_{23}+|11\rangle_{23}\right)$, while the control qubit and target qubit (A$_1$ and B$_4$) are encoded in the polarization DOF of photons. At node A, we use path qubit A$_2$ to perform CNOT operations on polarization qubit A$_1$, followed by measurement of the path qubit A$_2$ along the $X$ basis. At node B, we use polarization qubit B$_4$ to perform CNOT operations on path qubit B$_3$, followed by a measurement of path qubit B$_3$ along the $Z$ basis. Then node B notifies node A of the measurement results through classical communication, and node A performs $I$ or $\sigma_x$ local operations on the polarization qubit A$_1$ using an electro-optic modulator (EOM) to finalize the nonlocal CNOT gate between A$_1$ and B$_4$. In this protocol, without LOCC, the probability of success is $25\%$. The implementation of active feedforward has doubled the success probability of nonlocal two-qubit gates in our experiment. We note that a recent study has also implemented LOCC for multiplexed teleportation of quantum states between a photonic qubit and an atomic memory, linked by 1-km optical fiber, albeit within a single local node \cite{Lago-Rivera2023Long}.
These two-node demonstrations only necessitate quantum memories with preprogrammed storage times. In principle, optical fiber delay loops could provide similar functions with reduced complexity. Nevertheless, quantum memories could provide extended storage times far surpassing those of fiber delay loops and offer versatile multifunctionalities \cite{Lei2023Quantum}, rendering them indispensable components in future large-scale quantum networks.

\textbf{Characterization of the nonlocal CNOT gate.}
We use $H$ and $V$ to denote the basis states for polarization qubits A$_1$ and B$_4$. The CNOT gate acts as $|HH\rangle \rightarrow |HH\rangle$, $|HV\rangle \rightarrow |HV\rangle$, and $|VH\rangle \rightarrow |VV\rangle$, $|VV\rangle \rightarrow |VH\rangle$. To characterize this gate, we use all combinations of $|H\rangle/|V\rangle$ for the polarization qubits of node A and node B and then measure the resulting states. 
The truth table of the nonlocal CNOT gate is shown in Fig. \ref{fig:truth_table_and_witness_zhengwen}\textbf{a}, and a fidelity of $(88.7\pm2.1)\%$ compared to the ideal quantum CNOT gate is obtained. 

% Figure environment removed


To demonstrate the quantum properties of the nonlocal CNOT gate, we further use it to create entanglement between two qubits that are initially separable. We initialize qubits A$_1$ and B$_4$ into $|+\rangle|H\rangle$, $|+\rangle|V\rangle$, $|-\rangle|H\rangle$, and $|-\rangle|V\rangle$, respectively, where $|+\rangle=\frac{1}{\sqrt{2}}(|H\rangle+|V\rangle)$, $|-\rangle=\frac{1}{\sqrt{2}}(|H\rangle-|V\rangle)$. An ideal CNOT gate would generate four maximally entangled Bell states $|\Phi^+\rangle$, $|\Phi^-\rangle$, $|\Psi^+\rangle$ and $|\Psi^-\rangle$, where $|\Phi^{\pm}\rangle=|HH\rangle\pm|VV\rangle$ and $|\Psi^{\pm}\rangle=|HV\rangle\pm|VH\rangle$. 
The reconstructed density matrices ($\rho$) of the output states are provided in Fig. \ref{fig:truth_table_and_witness_zhengwen}\textbf{b}-\textbf{e}, with fidelity to expected Bell states of $\mathcal{F}(|\Phi^+\rangle)=(81.1\pm2.6)\%$, $\mathcal{F}(|\Phi^-\rangle)=(85.1\pm2.5)\%$, $\mathcal{F}(|\Psi^+\rangle)=(81.3\pm2.8)\%$ and $\mathcal{F}(|\Psi^-\rangle)=(80.2\pm2.0)\%$, respectively. The average overlap fidelity to the ideal Bell states is $(81.9\pm2.5)\%$, indicating that high-quality entanglement is generated between node A and node B by the nonlocal CNOT gate. The final generation rates of effective nonlocal gates are 0.042 Hz, primarily limited by the efficiency of entangled light sources and quantum memories (see Section 6 in Supplementary materials).


\textbf{Implementation of quantum algorithms.}
Universal quantum computing can be realized through the implementation of the demonstrated nonlocal CNOT gate in conjunction with additional local quantum gates  \cite{gottesman1999demonstrating}. Here, we execute a proof-of-principle for distributed quantum computing using two representative quantum algorithms that exploit quantum advantages: the Deutsch-Jozsa algorithm \cite{deutsch1992rapid} and the phase estimation algorithm \cite{kitaev1995quantum}. 


In the Deutsch-Jozsa problem, there are four possible functions $f$ that map one input bit ($a = 0,1$) to one output bit ($f(a) = 0,1$). These functions can be divided into constant functions ($f_{1}(a)=0$, $f_{2}(a)=1$) and balanced functions ($f_{3}(a)=a$, $f_{4}(a)=NOT\ a$). Now, there is an $N$-bit input $x$, and the distinction between the constant and balanced functions $f(x)$ can be achieved through an oracle. Classical algorithms would require querying this oracle $2^{N-1}+1$ times in the worst case for a deterministic answer. In contrast, the Deutsch-Jozsa quantum algorithm requires only a single query in all cases, leveraging inherent quantum parallelism that utilizes quantum interference for computation \cite{deutsch1992rapid,DiVincenzo1995Quantum}, in which the output states might be engineered to be a coherent superposition of states corresponding to different answers.

As shown in Fig. \ref{fig:Algorithm}\textbf{a}, in the two-qubit case \cite{deutsch1985quantum}, the operation that needs to be loaded is $|x\rangle|y\rangle \rightarrow |x\rangle|y \oplus f(x)\rangle$, where $y$ is an auxiliary qubit and $x$ is a single query qubit. Fig. \ref{fig:Algorithm}\textbf{b}-\textbf{e} presents the measured probability distributions of the $x$-register in the computational basis when the function is chosen as constant (Fig. \ref{fig:Algorithm}\textbf{b, c}) and balanced (Fig. \ref{fig:Algorithm}\textbf{d, e}). In all four cases, the experimental fidelity of identifying function classes with one measurement exceeds $91\%$.


% Figure environment removed


The quantum phase estimation algorithm \cite{kitaev1995quantum} is used to estimate the phase of an operator acting on an eigenstate and is frequently used as a subroutine in other quantum algorithms, such as factorization \cite{lanyon2007experimental,politi2009shor} and quantum chemistry \cite{lanyon2010towards}. In this algorithm, the quantum state register consists of a unitary operator $U$ with an eigenstate $|\psi\rangle$ ($U|\psi\rangle=\mathrm{e}^{\mathrm{i}2\pi \varphi}|\psi\rangle$), and information about the unitary operator $U$ is encoded on the measurement register through multiple controlled-$U^{2^{k}}$ operations with $k$ an integer. 


The accuracy of the phase estimation algorithm increases with the number of measurement registers. The estimated phase $\tilde{\varphi}$ with $m$ measurement register qubits in binary expansion is $\tilde{\varphi}= 0. \tilde{\varphi}_1 \tilde{\varphi}_2 \ldots \tilde{\varphi}_m$ \cite{kitaev1995quantum}. The circuit with $m$ measurement register qubits can be simplified to an $m$-round iterative phase estimation algorithm (IPEA) \cite{dobvsivcek2007arbitrary} with a single measurement register qubit circuit (Fig. \ref{fig:Algorithm}\textbf{f}). At the end of each iteration, the measurement register qubit is measured, which is an estimate of the $k$-th bit of $\varphi$ in the binary expansion. In the IPEA scheme, the least significant bit is first evaluated (i.e., $k$ iterates backward from $m$ to 1), and then the obtained information is fed back to the phase estimation of subsequent iterations. The iterative information transmission is achieved by rotating $Z_{\varphi}$ of the state register, and its angle is determined by the phase measurement in the previous step. 

The key challenge in implementing a distributed phase estimation algorithm is to achieve nonlocal controlled-\textit{U} (C-\textit{U}) gates. Here, we construct the nonlocal C-\textit{U} gate based on quantum gate teleportation, where a local CNOT gate $C_{21}$ is implemented in node A and $C_{43}$ in node B is changed to a local C-\textit{U} gate (see the Supplementary materials for details). As shown in Fig. \ref{fig:Algorithm}\textbf{g}-\textbf{j}, we perform the quantum phase estimation algorithm for $U=I$, $Z^{1/2}$, $Z^{5/4}$, and $Z^{3/2}$ with three iterations.
These three unitary operations each act on the state $|V\rangle$ with phase shifts $\varphi$ = 0, $\pi/2$, $5\pi/4$, and $3\pi/2$ (corresponding to 0.000, 0.010, 0.101 and 0.110 in binary form, respectively, multiplied by $2\pi$). We obtain the probabilities of each binary digit by polarization measurement of the register. The deviation between measured probabilities and theoretical results is mainly due to the imperfection of nonlocal C-\textit{U} gates. To determine the phase, each binary digit (0 or 1) is chosen based on which measurement probability exceeds 1/2.
Here, three rounds of iteration are sufficient to accurately determine the target phase chosen in this experiment. However, if the precision of the digital phase shift exceeds the number of measurement rounds (e.g., if a phase in binary form 0.1111 is determined but only three measurement rounds are conducted), a discrepancy will arise between the experimental measurements and the actual value.


\section*{DISCUSSION}
To conclude, we have demonstrated nonlocal quantum gates across two nodes separated by 7.0 km and the long-distance active feedforward control of remote
qubits enabled by long-lived quantum memories could serve as a fundamental tool in large-scale quantum networks. 
Several significant improvements are necessary prior to practical applications which include integrating essential functionalities into quantum memories, such as on-demand spin-wave storage \cite{Afzelius2009Multimode,Ma2021Elimination}, enhancing the storage efficiency to outperform optical fiber loops, as well as establishing entangled photon sources and quantum memories at separate nodes to enable remote interconnection via quantum repeaters \cite{Lago-Rivera2021qp, liu2021heralded}. 
It is particularly noteworthy that the incorporation of multiplexed and long-lived quantum memories into quantum networks also opens the opportunity for the investigation of more efficient and high-performance quantum computing systems \cite{Gouzien2021Factoring,liu2023quantum}. 
Due to the use of different DOFs of photons, the photons are measured and absorbed by detectors before feedforward control, and the highest success probability for nonlocal gate operations is 50$\%$, posing a challenge to future scalability. The development of nondestructive photonic qubit detection \cite{Niemietz2021Nondestructive,BrienBrienNondestructive} may offer solutions to these limitations. An alternative approach would be introducing additional photons as communication qubits, allowing target and control qubits retrieved from quantum memories to be available for cascading with more qubits. 
The proof-of-principle demonstration is implemented with photonic qubits, but the principle of linking distant quantum computing nodes with field-deployed fibers can be extended to other platforms such as trapped ions \cite{wang2019trappedions} and neutral atoms \cite{Daiss2021singleatom}, so that to achieve more qubits in a single node and enable deterministic operations. This approach could enable the construction of large-scale quantum computing networks, to make powerful quantum computers that harness the power of quantum communication. 

After the submission of the current work, we note that similar experiments of teleported gates were performed with two silicon T center modules separated by approximately 40 m of fiber \cite{inc2024distributed} and two trapped ion modules separated by approximately 2 m \cite{main2024distributed}.


\section*{METHODS}

\textbf{Quantum memory sample.}
We implement the AFC protocol in an isotope-enriched $\mathrm{^{153}Eu^{3+}}$:$\mathrm{Y_2SiO_5}$ crystal that has a doping concentration of 0.2\% and an isotope $\mathrm{^{153}Eu^{3+}}$ enrichment greater than 99.8\%. The dimensions of the crystal are 17.5 mm $\times$ 5 mm $\times$ 4 mm along the $b$ $\times$ $D1$ $\times$ $D2$ axes. The optical storage utilizes the $\mathrm{{^7}F{_0}\rightarrow{^5}D{_0}}$ transition for site-1 $\mathrm{^{153}Eu^{3+}}$ ions in the $\mathrm{Y_2SiO_5}$ crystal with the background of Earth’s magnetic field. The optical depth of this transition is 5.6, and the inhomogeneous linewidth of the crystal is $(4.6\pm0.1)$ GHz.

\textbf{Witness of entangled states.}
The nonlocal CNOT gate can be used to generate entanglement with separable input states. The real part of the density matrix of the entangled state is measured by a witness \cite{hu2020efficient}. The diagonal terms are given by measuring $\sigma_z^{ij}$, i.e., $\langle i j|\rho| i j\rangle=\langle(|i\rangle\langle i|\otimes|j\rangle\langle j|)\rangle$, where $\rho$ is the density matrix. The off-diagonal terms require a superposition of two subspaces to be measured, i.e., $\Re e[\langle i i|\rho| j j\rangle]=\frac{1}{4}\left(\left\langle\sigma_x^{i j} \otimes \sigma_x^{i j}\right\rangle-\left\langle\sigma_y^{i j} \otimes \sigma_y^{i j}\right\rangle\right)$ and $\Re e[\langle i j|\rho| j i\rangle]=\frac{1}{4}\left(\left\langle\sigma_x^{i j} \otimes \sigma_x^{i j}\right\rangle+\left\langle\sigma_y^{i j} \otimes \sigma_y^{i j}\right\rangle\right)$, where $\sigma_x^{a b}=|a\rangle\langle b|+| b\rangle\langle a|$ and $\sigma_y^{a b}=i|a\rangle\langle b|-i| b\rangle\langle a|$. In the experiment, four Bell states are obtained, and the real parts of some key elements of their density matrix are measured, as shown in Fig. \ref{fig:truth_table_and_witness_zhengwen}\textbf{b}-\textbf{e}. The average fidelity of four Bell states four Bell states is $(81.9\pm2.5)\%$, slightly lower than that of the initial entangled states. The reduction in fidelity is mainly attributed to the low storage efficiency of the quantum memory. The transmission through external optical fibers and the feedforward operation of LOCC also cause a slight decrease in fidelity.
\bigskip

\textbf{Data availability}
Data that support the findings of this study are available from the corresponding authors upon request.
\\

\textbf{Code availability}
The custom codes used to produce the results presented in this paper are available from the corresponding authors upon request.

\begin{thebibliography}{10}
\expandafter\ifx\csname url\endcsname\relax
  \def\url#1{\texttt{#1}}\fi
\expandafter\ifx\csname urlprefix\endcsname\relax\def\urlprefix{URL }\fi
\providecommand{\bibinfo}[2]{#2}
\providecommand{\eprint}[2][]{\url{#2}}

\bibitem{Kimble2008}
\bibinfo{author}{Kimble, H.~J.}
\newblock \bibinfo{title}{The quantum internet}.
\newblock \emph{\bibinfo{journal}{Nature}} \textbf{\bibinfo{volume}{453}}, \bibinfo{pages}{1023--1030} (\bibinfo{year}{2008}).

\bibitem{wehner2018quantum}
\bibinfo{author}{Wehner, S.}, \bibinfo{author}{Elkouss, D.} \& \bibinfo{author}{Hanson, R.}
\newblock \bibinfo{title}{Quantum internet: A vision for the road ahead}.
\newblock \emph{\bibinfo{journal}{Science}} \textbf{\bibinfo{volume}{362}}, \bibinfo{pages}{eaam9288} (\bibinfo{year}{2018}).

\bibitem{Hensen2015Loophole}
\bibinfo{author}{Hensen, B.} \emph{et~al.}
\newblock \bibinfo{title}{Loophole-free bell inequality violation using electron spins separated by 1.3 kilometres}.
\newblock \emph{\bibinfo{journal}{Nature}} \textbf{\bibinfo{volume}{526}}, \bibinfo{pages}{682--686} (\bibinfo{year}{2015}).

\bibitem{Yu2020Entanglement}
\bibinfo{author}{Yu, Y.} \emph{et~al.}
\newblock \bibinfo{title}{Entanglement of two quantum memories via fibres over dozens of kilometres}.
\newblock \emph{\bibinfo{journal}{Nature}} \textbf{\bibinfo{volume}{578}}, \bibinfo{pages}{240--245} (\bibinfo{year}{2020}).

\bibitem{vanLeent2022Entangling}
\bibinfo{author}{van Leent, T.} \emph{et~al.}
\newblock \bibinfo{title}{Entangling single atoms over 33{\thinspace}km telecom fibre}.
\newblock \emph{\bibinfo{journal}{Nature}} \textbf{\bibinfo{volume}{607}}, \bibinfo{pages}{69--73} (\bibinfo{year}{2022}).

\bibitem{Krutyanskiy2023Entanglement}
\bibinfo{author}{Krutyanskiy, V.} \emph{et~al.}
\newblock \bibinfo{title}{Entanglement of trapped-ion qubits separated by 230 meters}.
\newblock \emph{\bibinfo{journal}{Physical Review Letters}} \textbf{\bibinfo{volume}{130}}, \bibinfo{pages}{050803} (\bibinfo{year}{2023}).

\bibitem{DiVincenzo1995Quantum}
\bibinfo{author}{DiVincenzo, D.~P.}
\newblock \bibinfo{title}{Quantum computation}.
\newblock \emph{\bibinfo{journal}{Science}} \textbf{\bibinfo{volume}{270}}, \bibinfo{pages}{255--261} (\bibinfo{year}{1995}).

\bibitem{Ladd2010QC}
\bibinfo{author}{Ladd, T.~D.} \emph{et~al.}
\newblock \bibinfo{title}{Quantum computers}.
\newblock \emph{\bibinfo{journal}{Nature}} \textbf{\bibinfo{volume}{464}}, \bibinfo{pages}{45--53} (\bibinfo{year}{2010}).

\bibitem{Komar2014A}
\bibinfo{author}{K{\'o}m{\'a}r, P.} \emph{et~al.}
\newblock \bibinfo{title}{A quantum network of clocks}.
\newblock \emph{\bibinfo{journal}{Nature Physics}} \textbf{\bibinfo{volume}{10}}, \bibinfo{pages}{582--587} (\bibinfo{year}{2014}).

\bibitem{Jiang2007distri}
\bibinfo{author}{Jiang, L.}, \bibinfo{author}{Taylor, J.~M.}, \bibinfo{author}{S\o{}rensen, A.~S.} \& \bibinfo{author}{Lukin, M.~D.}
\newblock \bibinfo{title}{Distributed quantum computation based on small quantum registers}.
\newblock \emph{\bibinfo{journal}{Physical Review A}} \textbf{\bibinfo{volume}{76}}, \bibinfo{pages}{062323} (\bibinfo{year}{2007}).

\bibitem{Monroe2014distri}
\bibinfo{author}{Monroe, C.} \emph{et~al.}
\newblock \bibinfo{title}{Large-scale modular quantum-computer architecture with atomic memory and photonic interconnects}.
\newblock \emph{\bibinfo{journal}{Physical Review A}} \textbf{\bibinfo{volume}{89}}, \bibinfo{pages}{022317} (\bibinfo{year}{2014}).

\bibitem{li2024highrate}
\bibinfo{author}{Li, Y.} \& \bibinfo{author}{Thompson, J.~D.}
\newblock \bibinfo{title}{High-rate and high-fidelity modular interconnects between neutral atom quantum processors}.
\newblock \emph{\bibinfo{journal}{PRX Quantum}} \textbf{\bibinfo{volume}{5}}, \bibinfo{pages}{020363} (\bibinfo{year}{2024}).

\bibitem{Cirac1999distri}
\bibinfo{author}{Cirac, J.~I.}, \bibinfo{author}{Ekert, A.~K.}, \bibinfo{author}{Huelga, S.~F.} \& \bibinfo{author}{Macchiavello, C.}
\newblock \bibinfo{title}{Distributed quantum computation over noisy channels}.
\newblock \emph{\bibinfo{journal}{Physical Review A}} \textbf{\bibinfo{volume}{59}}, \bibinfo{pages}{4249--4254} (\bibinfo{year}{1999}).

\bibitem{gottesman1999demonstrating}
\bibinfo{author}{Gottesman, D.} \& \bibinfo{author}{Chuang, I.~L.}
\newblock \bibinfo{title}{Demonstrating the viability of universal quantum computation using teleportation and single-qubit operations}.
\newblock \emph{\bibinfo{journal}{Nature}} \textbf{\bibinfo{volume}{402}}, \bibinfo{pages}{390--393} (\bibinfo{year}{1999}).

\bibitem{cuomo2020towards}
\bibinfo{author}{Cuomo, D.}, \bibinfo{author}{Caleffi, M.} \& \bibinfo{author}{Cacciapuoti, A.~S.}
\newblock \bibinfo{title}{Towards a distributed quantum computing ecosystem}.
\newblock \emph{\bibinfo{journal}{IET Quantum Communication}} \textbf{\bibinfo{volume}{1}}, \bibinfo{pages}{3--8} (\bibinfo{year}{2020}).

\bibitem{Oh2023Distributed}
\bibinfo{author}{Oh, E.}, \bibinfo{author}{Lai, X.}, \bibinfo{author}{Wen, J.} \& \bibinfo{author}{Du, S.}
\newblock \bibinfo{title}{Distributed quantum computing with photons and atomic memories}.
\newblock \emph{\bibinfo{journal}{Advanced Quantum Technologies}} \textbf{\bibinfo{volume}{6}}, \bibinfo{pages}{2300007} (\bibinfo{year}{2023}).

\bibitem{Huang2017blind}
\bibinfo{author}{Huang, H.-L.} \emph{et~al.}
\newblock \bibinfo{title}{Experimental blind quantum computing for a classical client}.
\newblock \emph{\bibinfo{journal}{Physical Review Letters}} \textbf{\bibinfo{volume}{119}}, \bibinfo{pages}{050503} (\bibinfo{year}{2017}).

\bibitem{Eisert2000gate}
\bibinfo{author}{Eisert, J.}, \bibinfo{author}{Jacobs, K.}, \bibinfo{author}{Papadopoulos, P.} \& \bibinfo{author}{Plenio, M.~B.}
\newblock \bibinfo{title}{Optimal local implementation of nonlocal quantum gates}.
\newblock \emph{\bibinfo{journal}{Physical Review A}} \textbf{\bibinfo{volume}{62}}, \bibinfo{pages}{052317} (\bibinfo{year}{2000}).

\bibitem{Bartlett2003gate}
\bibinfo{author}{Bartlett, S.~D.} \& \bibinfo{author}{Munro, W.~J.}
\newblock \bibinfo{title}{Quantum teleportation of optical quantum gates}.
\newblock \emph{\bibinfo{journal}{Physical Review Letters}} \textbf{\bibinfo{volume}{90}}, \bibinfo{pages}{117901} (\bibinfo{year}{2003}).

\bibitem{Pirandola2015review}
\bibinfo{author}{Pirandola, S.}, \bibinfo{author}{Eisert, J.}, \bibinfo{author}{Weedbrook, C.}, \bibinfo{author}{Furusawa, A.} \& \bibinfo{author}{Braunstein, S.~L.}
\newblock \bibinfo{title}{Advances in quantum teleportation}.
\newblock \emph{\bibinfo{journal}{Nature Photonics}} \textbf{\bibinfo{volume}{9}}, \bibinfo{pages}{641--652} (\bibinfo{year}{2015}).

\bibitem{Hu2023review}
\bibinfo{author}{Hu, X.-M.}, \bibinfo{author}{Guo, Y.}, \bibinfo{author}{Liu, B.-H.}, \bibinfo{author}{Li, C.-F.} \& \bibinfo{author}{Guo, G.-C.}
\newblock \bibinfo{title}{Progress in quantum teleportation}.
\newblock \emph{\bibinfo{journal}{Nature Reviews Physics}} \textbf{\bibinfo{volume}{5}}, \bibinfo{pages}{339--353} (\bibinfo{year}{2023}).

\bibitem{huang2004optical1}
\bibinfo{author}{Huang, Y.-F.}, \bibinfo{author}{Ren, X.-F.}, \bibinfo{author}{Zhang, Y.-S.}, \bibinfo{author}{Duan, L.-M.} \& \bibinfo{author}{Guo, G.-C.}
\newblock \bibinfo{title}{Experimental teleportation of a quantum controlled-not gate}.
\newblock \emph{\bibinfo{journal}{Physical Review Letters}} \textbf{\bibinfo{volume}{93}}, \bibinfo{pages}{240501} (\bibinfo{year}{2004}).

\bibitem{gap2010optical2}
\bibinfo{author}{Gao, W.-B.} \emph{et~al.}
\newblock \bibinfo{title}{Teleportation-based realization of an optical quantum two-qubit entangling gate}.
\newblock \emph{\bibinfo{journal}{Proceedings of the National Academy of Sciences}} \textbf{\bibinfo{volume}{107}}, \bibinfo{pages}{20869--20874} (\bibinfo{year}{2010}).

\bibitem{wang2019trappedions}
\bibinfo{author}{Wan, Y.} \emph{et~al.}
\newblock \bibinfo{title}{Quantum gate teleportation between separated qubits in a trapped-ion processor}.
\newblock \emph{\bibinfo{journal}{Science}} \textbf{\bibinfo{volume}{364}}, \bibinfo{pages}{875--878} (\bibinfo{year}{2019}).

\bibitem{Chou2018superconducting}
\bibinfo{author}{Chou, K.~S.} \emph{et~al.}
\newblock \bibinfo{title}{Deterministic teleportation of a quantum gate between two logical qubits}.
\newblock \emph{\bibinfo{journal}{Nature}} \textbf{\bibinfo{volume}{561}}, \bibinfo{pages}{368--373} (\bibinfo{year}{2018}).

\bibitem{Daiss2021singleatom}
\bibinfo{author}{Daiss, S.} \emph{et~al.}
\newblock \bibinfo{title}{A quantum-logic gate between distant quantum-network modules}.
\newblock \emph{\bibinfo{journal}{Science}} \textbf{\bibinfo{volume}{371}}, \bibinfo{pages}{614--617} (\bibinfo{year}{2021}).

\bibitem{deutsch1992rapid}
\bibinfo{author}{Deutsch, D.} \& \bibinfo{author}{Jozsa, R.}
\newblock \bibinfo{title}{Rapid solution of problems by quantum computation}.
\newblock \emph{\bibinfo{journal}{Proceedings of the Royal Society of London. Series A: Mathematical and Physical Sciences}} \textbf{\bibinfo{volume}{439}}, \bibinfo{pages}{553--558} (\bibinfo{year}{1992}).

\bibitem{kitaev1995quantum}
\bibinfo{author}{Kitaev, A.~Y.}
\newblock \bibinfo{title}{Quantum measurements and the abelian stabilizer problem} (\bibinfo{year}{1995}).
\newblock \eprint{Preprint at https://arxiv.org/abs/quant-ph/9511026}.

\bibitem{Lago-Rivera2021qp}
\bibinfo{author}{Lago-Rivera, D.}, \bibinfo{author}{Grandi, S.}, \bibinfo{author}{Rakonjac, J.~V.}, \bibinfo{author}{Seri, A.} \& \bibinfo{author}{de~Riedmatten, H.}
\newblock \bibinfo{title}{Telecom-heralded entanglement between multimode solid-state quantum memories}.
\newblock \emph{\bibinfo{journal}{Nature}} \textbf{\bibinfo{volume}{594}}, \bibinfo{pages}{37--40} (\bibinfo{year}{2021}).

\bibitem{liu2021heralded}
\bibinfo{author}{Liu, X.} \emph{et~al.}
\newblock \bibinfo{title}{Heralded entanglement distribution between two absorptive quantum memories}.
\newblock \emph{\bibinfo{journal}{Nature}} \textbf{\bibinfo{volume}{594}}, \bibinfo{pages}{41--45} (\bibinfo{year}{2021}).

\bibitem{franson1989bell}
\bibinfo{author}{Franson, J.~D.}
\newblock \bibinfo{title}{Bell inequality for position and time}.
\newblock \emph{\bibinfo{journal}{Physical Review Letters}} \textbf{\bibinfo{volume}{62}}, \bibinfo{pages}{2205} (\bibinfo{year}{1989}).

\bibitem{clausen2011quantum}
\bibinfo{author}{Clausen, C.} \emph{et~al.}
\newblock \bibinfo{title}{Quantum storage of photonic entanglement in a crystal}.
\newblock \emph{\bibinfo{journal}{Nature}} \textbf{\bibinfo{volume}{469}}, \bibinfo{pages}{508--511} (\bibinfo{year}{2011}).

\bibitem{Afzelius2009Multimode}
\bibinfo{author}{Afzelius, M.}, \bibinfo{author}{Simon, C.}, \bibinfo{author}{de~Riedmatten, H.} \& \bibinfo{author}{Gisin, N.}
\newblock \bibinfo{title}{Multimode quantum memory based on atomic frequency combs}.
\newblock \emph{\bibinfo{journal}{Physical Review A}} \textbf{\bibinfo{volume}{79}}, \bibinfo{pages}{052329} (\bibinfo{year}{2009}).

\bibitem{Jobez2014Cavity}
\bibinfo{author}{Jobez, P.} \emph{et~al.}
\newblock \bibinfo{title}{Cavity-enhanced storage in an optical spin-wave memory}.
\newblock \emph{\bibinfo{journal}{New Journal of Physics}} \textbf{\bibinfo{volume}{16}}, \bibinfo{pages}{083005} (\bibinfo{year}{2014}).

\bibitem{Ma2021One-hour}
\bibinfo{author}{Ma, Y.}, \bibinfo{author}{Ma, Y.-Z.}, \bibinfo{author}{Zhou, Z.-Q.}, \bibinfo{author}{Li, C.-F.} \& \bibinfo{author}{Guo, G.-C.}
\newblock \bibinfo{title}{One-hour coherent optical storage in an atomic frequency comb memory}.
\newblock \emph{\bibinfo{journal}{Nature Communications}} \textbf{\bibinfo{volume}{12}}, \bibinfo{pages}{2381} (\bibinfo{year}{2021}).

\bibitem{rakonjac2021entanglement}
\bibinfo{author}{Rakonjac, J.~V.} \emph{et~al.}
\newblock \bibinfo{title}{Entanglement between a telecom photon and an on-demand multimode solid-state quantum memory}.
\newblock \emph{\bibinfo{journal}{Physical Review Letters}} \textbf{\bibinfo{volume}{127}}, \bibinfo{pages}{210502} (\bibinfo{year}{2021}).

\bibitem{Puigibert2020Entanglement}
\bibinfo{author}{Puigibert, M. l.~G.} \emph{et~al.}
\newblock \bibinfo{title}{Entanglement and nonlocality between disparate solid-state quantum memories mediated by photons}.
\newblock \emph{\bibinfo{journal}{Physical Review Research}} \textbf{\bibinfo{volume}{2}}, \bibinfo{pages}{013039} (\bibinfo{year}{2020}).

\bibitem{Businger2022Non-classical}
\bibinfo{author}{Businger, M.} \emph{et~al.}
\newblock \bibinfo{title}{Non-classical correlations over 1250 modes between telecom photons and 979-nm photons stored in $^{171}\mathrm{Yb}^{3+}:\mathrm{Y}_{2}\mathrm{SiO}_{5}$}.
\newblock \emph{\bibinfo{journal}{Nature Communications}} \textbf{\bibinfo{volume}{13}}, \bibinfo{pages}{6438} (\bibinfo{year}{2022}).

\bibitem{Longdell2005Stopped}
\bibinfo{author}{Longdell, J.~J.}, \bibinfo{author}{Fraval, E.}, \bibinfo{author}{Sellars, M.~J.} \& \bibinfo{author}{Manson, N.~B.}
\newblock \bibinfo{title}{Stopped light with storage times greater than one second using electromagnetically induced transparency in a solid}.
\newblock \emph{\bibinfo{journal}{Physical Review Letters}} \textbf{\bibinfo{volume}{95}}, \bibinfo{pages}{063601} (\bibinfo{year}{2005}).

\bibitem{Lei2023Quantum}
\bibinfo{author}{Lei, Y.} \emph{et~al.}
\newblock \bibinfo{title}{Quantum optical memory for entanglement distribution}.
\newblock \emph{\bibinfo{journal}{Optica}} \textbf{\bibinfo{volume}{10}}, \bibinfo{pages}{1511--1528} (\bibinfo{year}{2023}).

\bibitem{Lago-Rivera2023Long}
\bibinfo{author}{Lago-Rivera, D.}, \bibinfo{author}{Rakonjac, J.~V.}, \bibinfo{author}{Grandi, S.} \& \bibinfo{author}{Riedmatten, H.~d.}
\newblock \bibinfo{title}{Long distance multiplexed quantum teleportation from a telecom photon to a solid-state qubit}.
\newblock \emph{\bibinfo{journal}{Nature Communications}} \textbf{\bibinfo{volume}{14}}, \bibinfo{pages}{1889} (\bibinfo{year}{2023}).

\bibitem{deutsch1985quantum}
\bibinfo{author}{Deutsch, D.}
\newblock \bibinfo{title}{Quantum theory, the church--turing principle and the universal quantum computer}.
\newblock \emph{\bibinfo{journal}{Proceedings of the Royal Society of London. A. Mathematical and Physical Sciences}} \textbf{\bibinfo{volume}{400}}, \bibinfo{pages}{97--117} (\bibinfo{year}{1985}).

\bibitem{lanyon2007experimental}
\bibinfo{author}{Lanyon, B.~P.} \emph{et~al.}
\newblock \bibinfo{title}{Experimental demonstration of a compiled version of shor’s algorithm with quantum entanglement}.
\newblock \emph{\bibinfo{journal}{Physical Review Letters}} \textbf{\bibinfo{volume}{99}}, \bibinfo{pages}{250505} (\bibinfo{year}{2007}).

\bibitem{politi2009shor}
\bibinfo{author}{Politi, A.}, \bibinfo{author}{Matthews, J.~C.} \& \bibinfo{author}{O'Brien, J.~L.}
\newblock \bibinfo{title}{Shor’s quantum factoring algorithm on a photonic chip}.
\newblock \emph{\bibinfo{journal}{Science}} \textbf{\bibinfo{volume}{325}}, \bibinfo{pages}{1221--1221} (\bibinfo{year}{2009}).

\bibitem{lanyon2010towards}
\bibinfo{author}{Lanyon, B.~P.} \emph{et~al.}
\newblock \bibinfo{title}{Towards quantum chemistry on a quantum computer}.
\newblock \emph{\bibinfo{journal}{Nature Chemistry}} \textbf{\bibinfo{volume}{2}}, \bibinfo{pages}{106--111} (\bibinfo{year}{2010}).

\bibitem{dobvsivcek2007arbitrary}
\bibinfo{author}{Dob\ifmmode \check{s}\else \v{s}\fi{}\'{\i}\ifmmode~\check{c}\else \v{c}\fi{}ek, M.}, \bibinfo{author}{Johansson, G.}, \bibinfo{author}{Shumeiko, V.} \& \bibinfo{author}{Wendin, G.}
\newblock \bibinfo{title}{Arbitrary accuracy iterative quantum phase estimation algorithm using a single ancillary qubit: A two-qubit benchmark}.
\newblock \emph{\bibinfo{journal}{Physical Review A}} \textbf{\bibinfo{volume}{76}}, \bibinfo{pages}{030306} (\bibinfo{year}{2007}).

\bibitem{Ma2021Elimination}
\bibinfo{author}{Ma, Y.-Z.} \emph{et~al.}
\newblock \bibinfo{title}{Elimination of noise in optically rephased photon echoes}.
\newblock \emph{\bibinfo{journal}{Nature Communications}} \textbf{\bibinfo{volume}{12}}, \bibinfo{pages}{4378} (\bibinfo{year}{2021}).

\bibitem{Gouzien2021Factoring}
\bibinfo{author}{Gouzien, E.} \& \bibinfo{author}{Sangouard, N.}
\newblock \bibinfo{title}{Factoring 2048-bit rsa integers in 177 days with 13 436 qubits and a multimode memory}.
\newblock \emph{\bibinfo{journal}{Physical Review Letters}} \textbf{\bibinfo{volume}{127}}, \bibinfo{pages}{140503} (\bibinfo{year}{2021}).

\bibitem{liu2023quantum}
\bibinfo{author}{Liu, C.}, \bibinfo{author}{Wang, M.}, \bibinfo{author}{Stein, S.~A.}, \bibinfo{author}{Ding, Y.} \& \bibinfo{author}{Li, A.}
\newblock \bibinfo{title}{Quantum memory: A missing piece in quantum computing units} (\bibinfo{year}{2023}).
\newblock \eprint{Preprint at https://arxiv.org/abs/2309.14432}.

\bibitem{Niemietz2021Nondestructive}
\bibinfo{author}{Niemietz, D.}, \bibinfo{author}{Farrera, P.}, \bibinfo{author}{Langenfeld, S.} \& \bibinfo{author}{Rempe, G.}
\newblock \bibinfo{title}{Nondestructive detection of photonic qubits}.
\newblock \emph{\bibinfo{journal}{Nature}} \textbf{\bibinfo{volume}{591}}, \bibinfo{pages}{570--574} (\bibinfo{year}{2021}).

\bibitem{BrienBrienNondestructive}
\bibinfo{author}{O'Brien, C.}, \bibinfo{author}{Zhong, T.}, \bibinfo{author}{Faraon, A.} \& \bibinfo{author}{Simon, C.}
\newblock \bibinfo{title}{Nondestructive photon detection using a single rare-earth ion coupled to a photonic cavity}.
\newblock \emph{\bibinfo{journal}{Physical Review A}} \textbf{\bibinfo{volume}{94}}, \bibinfo{pages}{043807} (\bibinfo{year}{2016}).

\bibitem{inc2024distributed}
\bibinfo{author}{Afzal, F.} \emph{et~al.}
\newblock \bibinfo{title}{Distributed quantum computing in silicon} (\bibinfo{year}{2024}).
\newblock \eprint{Preprint at https://doi.org/10.48550/arXiv.2406.01704}.

\bibitem{main2024distributed}
\bibinfo{author}{Main, D.} \emph{et~al.}
\newblock \bibinfo{title}{Distributed quantum computing across an optical network link} (\bibinfo{year}{2024}).
\newblock \eprint{Preprint at https://doi.org/10.48550/arXiv.2407.00835}.

\bibitem{hu2020efficient}
\bibinfo{author}{Hu, X.-M.} \emph{et~al.}
\newblock \bibinfo{title}{Efficient generation of high-dimensional entanglement through multipath down-conversion}.
\newblock \emph{\bibinfo{journal}{Physical Review Letters}} \textbf{\bibinfo{volume}{125}}, \bibinfo{pages}{090503} (\bibinfo{year}{2020}).

\end{thebibliography}





\bigskip
\textbf{Acknowledgments}
This work is supported by the National Key R\&D Program of China (No. 2017YFA0304100), Innovation Program for Quantum Science and Technology (No. 2021ZD0301200), the National Natural Science Foundation of China (Nos. 12374338, 12222411, 11904357, 12174367, 12204458, 11821404, 12204459 and 62322513), Anhui Provincial Natural Science Foundation (Nos. 2108085QA26 and 2408085JX002), Fundamental Research Funds for the Central Universities, Xiaomi Young Talents Program, Anhui Province science and technology innovation project (No. 202423r06050004), the China Postdoctoral Science Foundation (2023M743400). Z.-Q.Z acknowledges the support from the Youth Innovation Promotion Association CAS. T.-X. Z. acknowledges the support from the Postdoctoral Fellowship Program of CPSF. The allocation of node B and the deployment of ultralow-loss fiber is supported by China Unicom (Anhui).\\


\textbf{Author contributions}
Z.-Q.Z., B.-H.L., and C.-F.L. designed the experiment; X.L. constructed the fiber channel, T.-X. Z. constructed the quantum memory with the help of Y.-X.X. and Z.-W.O. X.-M.H. constructed the quantum light source with the help of C.Z., X.L., J.-L.M. and P.-Y.L. X.-M.H., T.-X.Z., X.L., C. Z., B.-H.L., and Z.-Q.Z. wrote the manuscript with input from others. Z.-Q.Z., B.-H.L., C.-F.L., and G.-C.G. supervised the project. All authors discussed the experimental procedures and results.\\

\textbf{Competing interests}
The authors declare no competing interests.



\end{document}