\documentclass[nofootinbib,twocolumn,prl,superscriptaddress,a4paper]{revtex4-1}
%twocolumn
%\usepackage{scicite}
\usepackage{graphicx}% Include figure files
%\usepackage{dcolumn}% Align table columns on decimal point
\usepackage{bm,bbm}% bold math
\usepackage{xcolor}%colours
\usepackage{tcolorbox}
\usepackage{algorithm}
\usepackage{algpseudocode}
\usepackage{amsmath}

\usepackage{booktabs} 

\usepackage[colorlinks,breaklinks]{hyperref}
\makeatletter
\newcommand\org@hypertarget{}
\let\org@hypertarget\hypertarget
\renewcommand\hypertarget[2]{%
\Hy@raisedlink{\org@hypertarget{#1}{}}#2%
  }
\makeatother
\newcommand{\ket}[1]{\left\vert#1\right\rangle}
\newcommand{\bra}[1]{\left\langle #1 \right\vert}
\newcommand{\card}[1]{\left\vert #1 \right\vert}
\newcommand{\braket}[2]{\langle #1|#2\rangle}
\newcommand{\ketbra}[2]{| #1\rangle \langle #2|}
\newcommand{\xq}[1]{\textcolor{cyan}{#1}}  
\newcommand{\mpi}[1]{\textcolor{olive}{#1}}

\begin{document}

%\title{Nonlocal quantum gate between network nodes separated by 7 kilometers}
\title{Distributed quantum computing over 7.0 km}
\author{Xiao Liu}
\email{These three authors contributed equally to this work.}
\affiliation{CAS Key Laboratory of Quantum Information, University of Science and Technology of China, Hefei 230026, China}
\affiliation{CAS Center for Excellence in Quantum Information and Quantum Physics, University of Science and Technology of China, Hefei 230026, China}
\author{Xiao-Min Hu}
\email{These three authors contributed equally to this work.}
\affiliation{CAS Key Laboratory of Quantum Information, University of Science and Technology of China, Hefei 230026, China}
\affiliation{CAS Center for Excellence in Quantum Information and Quantum Physics, University of Science and Technology of China, Hefei 230026, China}
\affiliation{Hefei National Laboratory, University of Science and Technology of China, Hefei 230088, China}
\author{Tian-Xiang Zhu}
\email{These three authors contributed equally to this work.}
\author{Chao Zhang}
\author{Yi-Xin Xiao}
\author{Jia-Le Miao}
\author{Zhong-Wen Ou}
\affiliation{CAS Key Laboratory of Quantum Information, University of Science and Technology of China, Hefei 230026, China}
\affiliation{CAS Center for Excellence in Quantum Information and Quantum Physics, University of Science and Technology of China, Hefei 230026, China}

\author{Bi-Heng Liu}
\email{bhliu@ustc.edu.cn}
\author{Zong-Quan Zhou}
\email{zq\_zhou@ustc.edu.cn}
\author{Chuan-Feng Li}
\email{cfli@ustc.edu.cn}
\author{Guang-Can Guo}
\affiliation{CAS Key Laboratory of Quantum Information, University of Science and Technology of China, Hefei 230026, China}
\affiliation{CAS Center for Excellence in Quantum Information and Quantum Physics, University of Science and Technology of China, Hefei 230026, China}
\affiliation{Hefei National Laboratory, University of Science and Technology of China, Hefei 230088, China}




\begin{abstract}
Distributed quantum computing provides a viable approach towards scalable quantum computation, which relies on nonlocal quantum gates to connect distant quantum nodes, to overcome the limitation of a single device. However, such an approach has only been realized within single nodes or between nodes separated by a few tens of meters, preventing the target of harnessing computing resources in large-scale quantum networks. Here, we demonstrate distributed quantum computing between two nodes spatially separated by 7.0 km, using stationary qubits based on multiplexed quantum memories, flying qubits at telecom wavelengths, and active feedforward control based on field-deployed fiber. Specifically, we illustrate quantum parallelism by implementing Deutsch-Jozsa algorithm and quantum phase estimation algorithm between the two remote nodes. These results represent the first demonstration of distributed quantum computing over metropolitan-scale distances and lay the foundation for the construction of large-scale quantum computing networks relying on existing fiber channels.

 
\end{abstract}
\date{\today}
\maketitle

% Figure environment removed


Quantum computing promises exponential computational power and the ability to solve certain problems beyond the reach of classical computers \cite{Ladd2010QC}. The construction of practical quantum computers using single quantum devices could be limited by technical constraints, such as crosstalk-induced errors and noise, and coupling restraints between distant qubits that prevent arbitrary connection of each qubit. The development of large-scale quantum networks \cite{wehner2018quantum}, comprising interconnected quantum nodes, has garnered significant attention as they provide an efficient method to scale up the numbers of qubits. Such network-based approach can harness the power of multiple remote nodes to solve complex computational tasks more efficiently \cite{Cirac1999distri,gottesman1999demonstrating,Jiang2007distri,Monroe2014distri,cuomo2020towards}, and could enable secure cloud quantum computing with completely classical clients \cite{Huang2017blind}. 

Towards distributed quantum computing networks, an essential function is the execution of nonlocal quantum gates across the network nodes \cite{gottesman1999demonstrating,Eisert2000gate,Bartlett2003gate}, which could be implemented with the help of quantum teleportation, avoiding the direct interaction of two remote qubits. Similar to quantum state teleportation, quantum gate teleportation requires pre-shared entanglement between two separated nodes, local operations and classical communications (LOCC) and local two two-qubit gates. Such remote quantum gates have been realized with photonic qubits without LOCC \cite{huang2004optical1,gap2010optical2}, as well as two species of trapped ions \cite{wang2019trappedions} and two superconducting qubits \cite{Chou2018superconducting}, both within a single device. More recently, based on spin-photon quantum logic gates, a remote quantum gate has been demonstrated between two separated single atoms linked by a 60-m fiber in the same building \cite{Daiss2021singleatom}. A demonstration of nonlocal quantum gates over long distances becomes a mandatory step to the construction of practical quantum computing networks.

Here we demonstrate long-distance distributed quantum computing based on long-distance distributed photonic entanglement, LOCC enabled with long-lived quantum memories and local two-qubit operations based on multiple-degree-of-freedom (DOF) encodings on photons. We characterize the nonlocal control-NOT (CNOT) gate by measuring its truth table and the fidelity of four Bell states created from separable states. Furthermore, we use the nonlocal quantum gates to execute Deutsch-Jozsa algorithm \cite{deutsch1992rapid} and quantum phase estimation algorithm \cite{Kitaev1996Quantum}, demonstrating distributed quantum computing over metropolitan-scale distances.

%\section{Experimental Setup}

The main approach of our experiment is schematically shown in Fig. \ref{fig:setup}. Nondegenerate entangled photon pairs %are employed to establish the entanglement between long-lived memories and telecom flying qubits, which 
are generated through the spontaneous parametric down-conversion (SPDC) process in node A which is located on the east campus of University of Science and Technology of China. The photon at 580 nm is stored in a local quantum memory, while the other photon at 1537 nm is sent along field-deployed fibers to node B which is located on the foot of the mountain DaShuShan with a spatial separation of 7.0 km from node A. Here we employ multiple DOFs to encode four qubits with two photons, to implement the teleportation-based nonlocal two-qubit gates \cite{gottesman1999demonstrating}. After performing local two-qubit gates at each node and successive LOCC, a nonlocal two-qubit gate is successfully implemented between the distant network node. Our protocol involves multiplexed operations in the time domain and the gate teleportation rate is proportional to the number of stored modes in the quantum memory \cite{Lago-Rivera2021qp,liu2021heralded}.



The nondegenerate entangled photon source is based on SPDC in a periodically-poled lithium niobate (PPLN) waveguide pumped by a laser at 421 nm which is obtained by sum-frequency generation from stabilized and amplified laser at 580 nm and 1537 nm. Both photons are spectrally filtered with two cascaded etalons to match the bandwidth of the quantum memory. Energy conservation guarantees that two photons are generated simultaneously, while the precise generation time of the photon pair remains uncertain within the coherence time of the pump laser, resulting in time-energy entanglement \cite{franson1989bell,clausen2011quantum}. After postselection through a 30-m unbalanced interferometer, the entangled state $\frac{1}{\sqrt{2}}\left(\left|S_{\mathrm{s}} S_{\mathrm{i}}\right\rangle+\left|L_{\mathrm{s}} L_{\mathrm{i}}\right\rangle\right)$ is obtained. If we encode the short $\left|S_{\mathrm{s}, \mathrm{i}}\right\rangle$ and long $\left|L_{\mathrm{s}, \mathrm{i}}\right\rangle$ paths to $|0\rangle$, $|1\rangle$ path encoding, we can get the path entangled state $\frac{1}{\sqrt{2}}(\left|00\right\rangle+\left|11\right\rangle)$. At the same time, an unbalanced interferometer is used to convert 1537~nm photons from time-energy DOF to polarization DOF before sending into the field-deployed ultra-low-loss optical fiber. The fiber-optic cable is fixed in the underground pipeline, maintaining low mechanical and temperature vibrations, which can support the long-distance transmission of polarization-encoded photons. 

The quantum memory is implemented with atomic frequency comb (AFC) protocol \cite{Afzelius2009Multimode} in a rare-earth-ion-doped crystal, i.e., 0.2\% doped $\mathrm{^{153}Eu^{3+}}$:$\mathrm{Y_2SiO_5}$ crystal. The isotope $\mathrm{^{153}Eu^{3+}}$ is chosen here to provide a larger storage bandwidth as compared to that of $\mathrm{^{151}Eu^{3+}}$ \cite{Jobez2014Cavity, Ma2021One-hour}. %which can provide a larger bandwidth and a long optical coherence time in the optical transition $\rm{^7}F{_0}\rightarrow{^5}D{_0}$.
%Considering a high storage fidelity and a large bandwidth to match the SPDC source, the atomic frequency comb (AFC) storage protocol \cite{Afzelius2009Multimode} is applied in the storage procedure. 
The $\mathrm{^{153}Eu^{3+}}$:$\mathrm{Y_2SiO_5}$ crystal is assembled on a close-cycle cryostat with a home-made vibration-isolated sample holder, which allows the preparation of high-resolution AFC for long-lived and multiplexed photonic storage. The quantum memory should hold the 580-nm photons to wait the transmission of 1537-nm photons from node A to node B (quantum communication, QC) and the feedback of successive measurement results from node B to node A (classical communication, CC). Both QC and CC are based on field-deployed optical fibers with the length of 7.9 km which puts a lower bound for the storage time of 79 $\mu$s, therefore we extend the storage time to 80.315 $\mu$s which significantly outperforms previous results for photonic entanglement storage (47.7 $\mu$s in Ref. \cite{rakonjac2021entanglement}) in solid-state quantum memories \cite{clausen2011quantum, Tang2015Storage, Zhou2015Quantum, Puigibert2020Entanglement, Lago-Rivera2021qp, liu2021heralded, rakonjac2021entanglement, Businger2022Non-classical, Rakonjac2022Storage}.
The prepared AFC has a total bandwidth of $24$ MHz and the storage efficiency for the bandwidth-matched SPDC source is $3.2 \%$ (Fig. \ref{fig:setup} \textbf{c}). Given a single mode duration of 72 ns which covers the retrieved echo peak, the number of temporal modes stored in the memory is 80.315 $\mu$s / 72 ns = 1115, which results in a linear enhancement of the rate for quantum gate teleportation (see SM for details). 
As an overall performance benchmark, the efficiency-time-bandwidth product of the memory is $80.315$ $\mu s$ $\times 24$ MHz $\times 3.2\% =61.7$, which significantly outperforms previous results for quantum light storage (24.5 in Ref. \cite{Businger2022Non-classical}) in solid-state quantum memories \cite{clausen2011quantum, Zhou2015Quantum, Saglamyurek2016multiplexed, Puigibert2020Entanglement, Lago-Rivera2021qp, liu2021heralded, rakonjac2021entanglement, Businger2022Non-classical, Rakonjac2022Storage}.


The four qubits employed in the teleportation-based two-qubits gates are denoted as A$_1$, A$_2$ in node A and B$_3$, B$_4$ in node B, where qubits A$_2$ and B$_3$ are prepared in an entangled state in the path DOF of photons, denoted as $|\Phi\rangle_{23}=\frac{1}{\sqrt{2}}\left(|00\rangle_{23}+|11\rangle_{23}\right)$, while the control qubit and target qubit (A$_1$ and B$_4$) are encoded in the polarization DOF of photons. On node A, we use path qubit A$_2$ to perform CNOT operations on polarization qubit A$_1$, followed by measurement of the path qubit A$_2$ along the $X$ basis. On node B, we use polarization qubit B$_4$ to perform CNOT operations on path qubit B$_3$, followed by a measurement of path qubit B$_3$ along the $Z$ basis. Then node B notifies Alice of the measurement results through classical communication, and Alice performs $I$ or $\sigma_x$ local operations on the polarization qubit A$_1$ using an electro-optic modulator (EOM) to finalize the non-local CNOT gate between A$_1$ and B$_4$. The implementation of LOCC has doubled the success probability of nonlocal two-qubit gates in our experiment.


We use $H$ and $V$ to denote the basis states for polarization qubits A$_1$ and B$_4$. The CNOT gate acts as: $|HH\rangle \rightarrow |HH\rangle$, $|HV\rangle \rightarrow |HV\rangle$, $|VH\rangle \rightarrow |VV\rangle$, $|VV\rangle \rightarrow |VH\rangle$. To characterize this gate, we use all combinations of $|H\rangle/|V\rangle$ for the polarization qubits of node A and node B, then measure the resulting states on a computational basis. %We recorded about 200 clicks for each input of the logic gate. 
The truth table of the quantum gate is shown in Fig. \ref{fig:truth_table_and_witness_zhengwen} \textbf{a}, and a fidelity of $(88.7\pm2.1)\%$ compared to the ideal quantum CNOT gate is obtained. 

To demonstrate the quantum properties of the nonlocal CNOT gate, we further use it to create entanglement between two qubits that are initially separable. We initialize qubits A$_1$ and B$_4$ into $|+\rangle|H\rangle$, $|+\rangle|V\rangle$, $|-\rangle|H\rangle$, $|-\rangle|V\rangle$, respectively, where $|+\rangle=\frac{1}{\sqrt{2}}(|H\rangle+|V\rangle)$, $|-\rangle=\frac{1}{\sqrt{2}}(|H\rangle-|V\rangle)$. An ideal CNOT gate would generate four maximally entangled Bell states $|\Phi^+\rangle$, $|\Phi^-\rangle$, $|\Psi^+\rangle$ and $|\Psi^-\rangle$ where $|\Phi^{\pm}\rangle=|HH\rangle\pm|VV\rangle$ and $|\Psi^{\pm}\rangle=|HV\rangle\pm|VH\rangle$. %For each resulting state, \textcolor{red}{we record about 100 heralding photon clicks.} 
The reconstructed density matrices ($\rho$) of output states are provided in Fig. \ref{fig:truth_table_and_witness_zhengwen}, with the fidelity to expected Bell states of $F(|\Phi^+\rangle)=81.1\pm2.6\%$, $F(|\Phi^-\rangle)=85.1\pm2.5\%$, $F(|\Psi^+\rangle)=81.3\pm2.8\%$ and $F(|\Psi^-\rangle)=80.2\pm2.0\%$, respectively. The average overlap fidelity to the ideal Bell states is $81.9\pm2.5\%$, indicating that high-quality entanglement has been generated between node A and node B by the nonlocal CNOT gate.

%编译原因 先注释了。
% Figure environment removed

%编译原因 先注释了。
% Figure environment removed



Universal quantum computing can be executed with the demonstrated nonlocal CNOT gate together with other local quantum gates \cite{gottesman1999demonstrating}. Here we implement distributed quantum computing with two representative quantum algorithms: Deutsch-Josza algorithms \cite{deutsch1992rapid} and phase estimation algorithm \cite{Kitaev1996Quantum}. 

 
In Deutsch-Josza problem, there are four possible functions $f$ that map one input bit ($a = 0,1$) to one output bit ($f(a) = 0,1$). These functions can be divided into constant functions ($f_{1}(a)=0$, $f_{2}(a)=1$) and balanced functions ($f_{3}(a)=a$, $f_{4}(a)=NOT\ a$). Now there is $N$-bit input $x$, and the distinction between the constant and balanced functions $f(x)$ can be achieved through an oracle. Classic algorithms would require querying this oracle $2^{N-1}+1$ times in the worst case for a deterministic answer. On the contrary, the Deutsch-Josza quantum algorithms only require one query in all cases due to the inherent quantum parallelism.

As shown in Fig. \ref{fig:Algorithm} \textbf{a}, in the two-qubit case \cite{deutsch1985quantum}, the operation that needs to be loaded is $|x\rangle|y\rangle \rightarrow |x\rangle|y \oplus f(x)\rangle$, where $y$ is an auxiliary qubit and $x$ is a single query qubit. Fig. \ref{fig:Algorithm} presents the measured probability distributions of the $x$-register in the computational basis when the function is chosen as constant (Fig. \ref{fig:Algorithm} \textbf{b,c}) and balanced (Fig. \ref{fig:Algorithm} \textbf{d,e}). In all four cases, the experimental fidelity of identifying function classes with one measurement exceeds $91\%$.

Quantum phase estimation algorithm \cite{Kitaev1996Quantum} is used to estimate the phase of an operator acting on an eigenstate and is frequently used as a subroutine in other quantum algorithms, such as factorization \cite{lanyon2007experimental,politi2009shor} and quantum chemistry \cite{lanyon2010towards}. In this algorithm, the quantum state register consists of a unitary operator $U$ with an eigenstate $|\psi\rangle$($\mathrm{U}|\psi\rangle=\mathrm{e}^{\mathrm{i}2\pi \varphi}|\psi\rangle$), and information about the unitary operator $U$ is encoded on the measurement register through multiple controlled-$U^{2^{k}}$ operations with $k$ an integer. 


The accuracy of the phase estimation algorithm increases with the number of the measurement registers. The estimated phase $\tilde{\varphi}$ with $m$ measurement register qubits in binary expansion is $\tilde{\varphi}=0. \tilde{\varphi}_1 \tilde{\varphi}_2 \ldots \tilde{\varphi}_m$ \cite{Kitaev1996Quantum}. The circuit with $m$ measurement register qubits can be simplified to an $m$-round iterative phase estimation algorithm (IPEA) \cite{dobvsivcek2007arbitrary} with a single measurement register qubit circuit (Fig. \ref{fig:Algorithm} \textbf{f}). At the end of each iteration, the measurement register qubit is measured, which is an estimate of the $k$-th bit of $\varphi$ in the binary expansion. In the IPEA scheme, the least significant bit is first evaluated (i.e., $k$ iterates backward from $m$ to 1), and then the obtained information is feedback to the phase estimation of subsequent iterations. The iterative information transmission is achieved by rotating $Z_{\varphi}$ of the state register, and its angle is determined by the phase measurement in the previous step. 

The key challenge in implementing a distributed phase estimation algorithm is to achieve nonlocal controlled-U (C-U) gates. In principle, the combination of nonlocal CNOT gates and local gates can generate any C-U gate but this method typically requires the consumption of numerous CNOT and local gate resources \cite{divincenzo1995two}. Here, we construct the nonlocal C-U gate based on quantum gate teleportation, where a local CNOT gate $C_{12}$ is implemented in node A and $C_{34}$ in node B is changed to a local C-U gate (see SM for details). As shown in Fig. \ref{fig:Algorithm} \textbf{g,h,i,j}, we perform quantum phase estimation algorithm for $U=I$, $Z^{1/2}$, $Z^{5/4}$, and $Z^{3/2}$ with three iterations. These three unitary operations each act on state $|V\rangle$ with phase shifts $\varphi$=0, $\pi/2$, $5\pi/4$, $3\pi/2$, which are all correctly measured using three rounds of iteration, in consistent with theoretical expectations.
 
To conclude, we have demonstrated distributed quantum computing across two nodes separated by 7.0 km with the help of long-lived and multiplexed quantum memories, telecom photonic interface, and field-deployed fibers. The demonstrated long-distance nonlocal quantum gates could serve as a fundamental tool in large-scale quantum networks. This proof-of-principle demonstration is implemented with photonic qubits, but the principle of linking distant computing nodes with field-deployed fibers can be extended to other platforms such as trapped ions \cite{wang2019trappedions} and superconducting qubits \cite{Chou2018superconducting}, so that to achieve more qubits in a single node and enable deterministic operations. This approach could enable the construction of large-scale quantum computing networks, to make powerful quantum computers that harness the power of quantum communication.



\begin{acknowledgments}
This work is supported by the National Key R\&D Program of China (No. 2017YFA0304100), Innovation Program for Quantum Science and Technology (No. 2021ZD0301200), the National Natural Science Foundation of China (Nos. 12222411, 11904357, 12174367, 12204458, 11821404 and 12204459), Anhui Provincial Natural Science Foundation (No. 2108085QA26), Fundamental Research Funds for the Central Universities, USTC Tang Scholarship, Xiaomi Foundation, Science and Technological Fund of Anhui Province for Outstanding Youth (2008085J02). Z.-Q.Z acknowledges the support from the Youth Innovation Promotion Association CAS. The deployment of ultra-low-loss fibers is supported by China Unicom.
\end{acknowledgments}

\bibliography{bibliography}
\end{document}
