\documentclass[aps,pre,showpacs,
%,raggedbottom, %nobalancelastpage,
amssymb,twocolumn
%,groupedaddress
]{revtex4-2}
%\documentclass[aps,pre,showpacs,raggedbottom, %nobalancelastpage,
%amssymb,twocolumn,groupedaddress]{revtex4-2}%{revtex4}
%\usepackage{graphicx}
%\usepackage{amsmath}
%\usepackage{amsfonts}
%\usepackage{amssymb}
%\usepackage{epsfig}
%\usepackage{color}
%\usepackage{dsfont}
%\documentclass[pra,aps,showpacs,showkeys,raggedbottom,
%nobalancelastpage,
%superscriptaddress,tightenlines,amssymb,twocolumn,longbibliography]{revtex4-1}
\usepackage{graphicx}
\usepackage{amsmath}
\usepackage{amsfonts}
\usepackage{amssymb}
\usepackage{epsfig,libertine}
\usepackage[libertine]{newtxmath}
\usepackage{color}
\usepackage{dsfont, hyperref, xcolor}
\usepackage{comment}
\usepackage{enumerate}
%\usepackage{babel}[english, italian]


\newcommand{\numberset}{\mathbb}
\newcommand{\N}{\numberset{N}}
\newcommand{\Z}{\numberset{Z}}
\newcommand{\Q}{\numberset{Q}}
\newcommand{\R}{\numberset{R}}
\newcommand{\C}{\numberset{C}}
\newcommand{\der}[2] { \frac{d #1}{d #2}}
\newcommand{\abs}[1]{\left\vert#1\right\vert}
\newcommand{\set}[1]{\left\{#1\right\}}
%\newcommand{\bra}[1]{\left\langle#1\right\vert}
%\newcommand{\ket}[1]{\left\vert#1\right\rangle}
\newcommand{\Tr}[1]{\text{Tr}\left\{#1\right\}}
\newcommand{\ParTr}[2]{\text{Tr}_{#1}\left\{#2\right\}}
%\newcommand\braket[2]{\left.\left\langle#1\right|#2\right\rangle}
\newcommand{\bra}[1]{\langle#1\vert}
\newcommand{\ket}[1]{\vert#1\rangle}
\newcommand\braket[2]{\langle#1|#2\rangle}

\begin{document}


\title{Fluctuation theorems and expected utility hypothesis}

\author{Gianluca~Francica, Luca Dell'Anna}
\address{Dipartimento di Fisica e Astronomia e Sezione INFN, Universit\`{a} di Padova, via Marzolo 8, 35131 Padova, Italy}

\date{\today}

\begin{abstract}
The expected utility hypothesis is a popular concept in economics that is useful for making decisions when the payoff is uncertain.
In this paper, we investigate the implications of a fluctuation theorem in the theory of expected utility. In particular, we wonder whether entropy could serve as a guideline for gambling. We prove the existence of a bound involving the certainty equivalent which depends on the entropy produced. Then, we examine the dependence of the certainty equivalent on the entropy by looking at specific situations, for instance, the work extraction from a non-equilibrium initial state.
\end{abstract}

\maketitle

\section{Introduction}
Resources (e.g., the wealth) are commonly affected by uncertainty, which can come from our impossibility to know all the information to get a deterministic prediction. For instance, in a quantum system this uncertainty is intrinsic and cannot be reduced arbitrarily.
Thus, typically an agent aims to gain a resource which is affected by stochastic fluctuations, so that he has to make a choice (among the different procedures available) under uncertainty. A risk neutral agent will make his choice by preferring procedures giving the maximum average gain. However, if the agent takes in account the risk of his choice, he can make his choice relying on the expected utility hypothesis, first formalized by von Neumann and Morgenstern within the theory of games and economic behaviour in 1944~\cite{vonNeumann}. Thus, the choice will depend on how much the agent is risk averse. Formally, the risk aversion can be characterized in terms of a utility function, from which we get the certainty equivalent (in simple terms, the certain amount of wealth that is equivalent, from the agent's point of view, to the procedure in question). Fluctuations are not always arbitrary, e.g., the fluctuations coming from the permanent state of thermal agitation of the matter cannot violate the second law of thermodynamics. This behavior can be a consequence of the so-called fluctuation theorems (see, e.g., Refs.~\cite{jarzynski97,crooks99,campisi11}). Recently, fundamental connections between utility theories and information-theoretic quantities have been established in Ref.~\cite{Ducuara232}.
Furthermore, by focusing on the stochastic work, it was shown that the certainty equivalent for certain utility functions is equal to the R\'{e}nyi divergence if the work fluctuation theorem is satisfied~\cite{Ducuara23}.
This result suggests that the fluctuation theorems play a role in the expected utility hypothesis.
Here, we aim to investigate this role in general, in strictly analogy to the so-called thermodynamic uncertainty relations~\cite{Hasegawa19,Timpanaro19,Francica22}. These relations provide a bound for the fluctuations of observables, which involves the entropy produced in the process.
Here, by assuming that a fluctuation theorem for the joint distribution holds~\cite{Garcia10}, we find a generalization of the thermodynamic uncertainty relation which involves the certainty equivalent of a risk non-neutral agent, which reproduces the usual thermodynamic uncertainty relation when the agent is risk neutral.
This result shows how the certainty equivalent is related to the average entropy produced. However, there are situations where the certainty equivalent shows a non-trivial dependence on the entropy, which goes beyond this bound.
In particular, we illustrate this dependence in a physical system where the wealth is the work extracted in a unitary cycle from a non-equilibrium initial state.

\section{Preliminaries}
Let us introduce some expected utility theory elements~\cite{bookmicroeco,bookmicroeco2}. We consider two lotteries 1 and 2, i.e., two random variables (wealths) $x$ and $y$ with certain probability distributions. A player will prefer a lottery instead of the other, e.g., 1 instead of 2, if
\begin{equation}\label{eq.exp utility theo}
\langle u(x) \rangle > \langle u(y) \rangle
\end{equation}
where $u(x)$ is the so-called utility function, and it is defined up to affine transformations, i.e., two utility functions related by such transformation gives the same preference ordering. For a lottery, e.g., 1, and a given utility function, we define the certainty equivalent $x_{CE}$ such that
\begin{equation}
u(x_{CE}) = \langle u(x) \rangle
\end{equation}
Of course, if $u(x)$ is a strictly increasing function, Eq.~\eqref{eq.exp utility theo} reduces to $x_{CE}> y_{CE}$.
The risk aversion of the player can be characterized in terms of the utility function which he chooses. In detail, the player is risk averse if $u(x)$ is concave, risk neutral if $u(x)$ is linear and risk loving if $u(x)$ is convex. Equivalently, the player is risk averse if $x_{CE}<\langle x\rangle$, risk neutral if $x_{CE}=\langle x\rangle$ and risk loving if $x_{CE}> \langle x\rangle$.
For a utility function which is concave and strictly increasing, the risk aversion can be measured with the absolute risk aversion defined as
\begin{equation}
r_A(x) = -\frac{u''(x)}{u'(x)}
\end{equation}
which is non-negative.



\section{Certainty equivalent and entropy}
We consider a lottery with wealth $w$ which is related to a random variable $\sigma$ by the fluctuation theorem
\begin{equation}\label{eq.fluc}
\frac{p(\sigma,w)}{p(-\sigma,-w)} = e^\sigma\,,
\end{equation}
so we will refer to $\sigma$ as stochastic entropy.
We consider an agent with an arbitrary utility function $u(x)$, which can be written as $u(x)=u_e(x)+u_o(x)$, with $u_e(x)=u_e(-x)$ and $u_o(x)=-u_o(-x)$. The certainty equivalent $w_{CE}$ is defined as
\begin{equation}
u(w_{CE})= \langle u(w)\rangle\,.
\end{equation}
Our main result is the bound
\begin{equation}\label{eq.main}
\frac{(u(w_{CE})-\langle u_e(w)\rangle)^2}{\langle u_o^2(w)\rangle}\leq  \left\langle \tanh^2\left(\frac{\sigma}{2}\right) \right\rangle\leq f^2(\langle \sigma\rangle)\,,
\end{equation}
where $f$ is the inverse of $h(x)=2x\tanh^{-1} x$. Thus, the certainty equivalent is constrained by the entropy production.
To prove Eq.~\eqref{eq.main}, we note that
\begin{equation}\label{eq.1}
u(w_{CE}) = \langle u_e(w) \rangle + \langle u_o(w) \rangle\,,
\end{equation}
thus we get
\begin{equation}\label{eq.trivial}
(u(w_{CE})-\langle u_e(w)\rangle)^2= \langle u_o(w)\rangle^2\leq \langle u_o^2(w)\rangle\,,
\end{equation}
which is a trivial bound which does not involve explicitly the entropy.
From the fluctuation theorem in Eq.~\eqref{eq.fluc}, given a function $F(\sigma,w)$ we get the identity
\begin{equation}\label{eq.id}
\langle F(\sigma,w) \rangle = \langle F(-\sigma,-w)e^{-\sigma}\rangle\,,
\end{equation}
from which
\begin{eqnarray}
\langle u_o(w) \rangle &=& \frac{1}{2} \langle u_o(w) (1-e^{-\sigma})\rangle\\
 &=& \frac{1}{2} \langle u_o(w)\sqrt{1+e^{-\sigma}} \frac{(1-e^{-\sigma})}{\sqrt{1+e^{-\sigma}}}\rangle\,.
\end{eqnarray}
By using the Cauchy-Schwartz inequality we get
\begin{eqnarray}
\langle u_o(w) \rangle^2 &\leq& \frac{1}{4} \left\langle u_o^2(w)(1+e^{-\sigma})\right\rangle \left\langle \frac{(1-e^{-\sigma})^2}{1+e^{-\sigma}}\right\rangle\\
 &=& \langle u_o^2(w)\rangle \left\langle \tanh^2\left(\frac{\sigma}{2}\right) \right\rangle\,,
\end{eqnarray}
where we have used the identity in Eq.~\eqref{eq.id}.
As shown, e.g., in Ref.~\cite{Francica22}, we have the inequality
\begin{equation}
\left\langle \tanh^2\left(\frac{\sigma}{2}\right) \right\rangle  \leq f^2(\langle \sigma\rangle )\,,
\end{equation}
from which follows Eq.~\eqref{eq.main}, which is a bound tighter than the trivial one obtained from Eq.~\eqref{eq.trivial} since $0\leq f^2(x)\leq \tanh(x/2)\leq 1$, with $f(0)=0$ and $f(x)\to 1$ as $x\to \infty$.

From Eq.~\eqref{eq.main}, we see that as $\langle \sigma \rangle \to 0$, if $\langle u_o^2(w)\rangle \neq 0$ and it is finite, then $w_{CE}\to u^{-1}( \langle u_e(w)\rangle)$. E.g., if $p(\sigma,w)=\delta(\sigma) p(w)$, from the fluctuation theorem we need to have $p(w)=p(-w)$, where we have defined the marginal distribution $p(w)=\int p(\sigma,w)d\sigma$,  and this result directly follows from Eq.~\eqref{eq.1}.
If $\langle \sigma \rangle \to \infty$, from the bound we get Eq.~\eqref{eq.trivial} so that the entropy tends to do not constrain the certainty equivalent.
We note that the bound in Eq.~\eqref{eq.main} can be saturated, i.e., we get the equality for the minimal distribution of Ref.~\cite{Timpanaro19}
\begin{eqnarray}\label{eq.minimal}
\nonumber p_{min}(\sigma,w) &=& \frac{1}{2\cosh(a/2)}\big( e^{a/2}\delta(\sigma-a)\delta(w-b)\\
 && + e^{-a/2}\delta(\sigma+a)\delta(w+b)\big)
\end{eqnarray}
or for $p(\sigma,w)=\delta(\sigma)p(w)$.
For a linear utility function, Eq.~\eqref{eq.main} reduces to
\begin{equation}\label{eq.theunc}
\frac{w_{CE}^2}{\langle w^2\rangle}\leq f^2(\langle \sigma\rangle)
\end{equation}
and since $w_{CE}=\langle w \rangle$, we obtain the thermodynamic uncertainty relation of Ref.~\cite{Timpanaro19}.
Thus, for a risk neutral agent, this bound suggests that if $\langle w \rangle >0$ ($\langle w \rangle <0$) the certainty equivalent becomes smaller (larger) as the entropy decreases, when $\langle w^2\rangle$ changes slowly with the entropy.
Furthermore, Eq.~\eqref{eq.main} can be written as
\begin{equation}\label{eq.b1}
\langle u_e(w) \rangle<u(w_{CE}) \leq \langle u_e(w) \rangle + f(\langle \sigma \rangle) \sqrt{\langle u_o^2(w)\rangle}
\end{equation}
if $\langle u_o(w)\rangle >0$, or as
\begin{equation}\label{eq.b2}
\langle u_e(w) \rangle>u(w_{CE}) \geq \langle u_e(w) \rangle - f(\langle \sigma \rangle) \sqrt{\langle u_o^2(w)\rangle}
\end{equation}
if $\langle u_o(w) \rangle <0$. It is worth observing that when $\langle u_e(w) \rangle$ and $\langle u_o^2(w)\rangle$ do not change with $\langle \sigma \rangle$, the right side of Eq.~\eqref{eq.b1} decreases as the average entropy decreases, whereas the right side of Eq.~\eqref{eq.b2} increases as the average entropy decreases. Then, the bound suggests that a risk non-neutral agent prefers lotteries with small or large entropies depending on the sign of  $\langle u_o(w) \rangle$. Of course, since $w$ implicitly depends on the entropy, we can get situations where this suggestion is not satisfied. For instance, we note that for the distribution in Eq.~\eqref{eq.minimal} the bound is saturated and both $\langle u_e(w) \rangle$ and $\langle u_o^2(w)\rangle$ do not change with $\langle \sigma \rangle$, but if we add a point in the support we get
\begin{eqnarray}
\nonumber p_{3}(\sigma,w) &=& \frac{1}{1+2\cosh(a/2)}\big( e^{a/2}\delta(\sigma-a)\delta(w-b)\\
 && + e^{-a/2}\delta(\sigma+a)\delta(w+b)+\delta(\sigma)\delta(w)\big)\,,
\end{eqnarray}
so that both $\langle u_e(w) \rangle$ and $\langle u_o^2(w)\rangle$ will change with $\langle \sigma \rangle$. However they change slowly if $u_e(x)$ and $u_o(x)$ evaluated at $x=0,b$ are not too large, e.g., $\langle u_e(w)\rangle$ monotonically changes from $(u_e(0)+2u_e(b))/3$ to $u_e(b)$ as $a$ (and so the average entropy) goes from zero to infinity. Then, if $\langle u_o(w)\rangle >0$, so that Eq.~\eqref{eq.b1} holds, if $(u_e(0)+2u_e(b))/3\approx u_e(b)$, $u(w_{CE})$ will decrease as the average entropy decreases, but if $u_e(b)$ is negative and very large in modulus, $u(w_{CE})$ follows the trend of $\langle u_e(w)\rangle$ and it will increase as the average entropy decreases.

\subsection{Exponential utility function}
To be more quantitative, let us focus on the exponential utility function
\begin{equation}
u_r(t) = \frac{1}{r}(1-e^{-rt})
\end{equation}
for $r\neq 0$, and $u_0(t)=t$, which is strictly increasing. The player is risk averse for $r>0$, risk neutral for $r=0$ and risk loving for $r<0$, and the absolute risk aversion is constant and it is $r_A(x)=r$. The certainty equivalent $w_{CE}$ is such that
\begin{equation}
e^{-rw_{CE}} = \langle e^{-rw}\rangle\,.
\end{equation}
The even and odd components of the utility function are $u_e(x)=(1-\cosh(r x))/r$ and $u_o(x)=\sinh(rx)/r$.
If $\langle \sinh r w \rangle >0$ when $r>0$, e.g., when $w$ takes non-negative values (we cannot have losses), Eq.~\eqref{eq.b1} is achieved which suggests that the certainty equivalent becomes smaller as the entropy decreases.
In contrast, if $\langle \sinh r w \rangle <0$ when $r>0$ (in simple terms losses are very likely), Eq.~\eqref{eq.b2} is achieved which suggests that the certainty equivalent becomes larger as the entropy decreases. However, for large $r$, since $\langle u_e(w) \rangle$ and/or $\langle u_o^2(w)\rangle$ can change strongly with the average entropy, we can obtain the opposite situation as we have seen for a distribution with a support of three points. To give another example, we consider the probability distribution~\cite{campisi21}
\begin{equation}
p(\sigma,w)= n (g(\sigma,w)\theta(\sigma)+g(-\sigma,-w)e^\sigma\theta(-\sigma))
\end{equation}
where $g(\sigma,w)$ is a non-negative function. We focus on $g(\sigma,w)=e^{-\sigma^2-w^2+\gamma \sigma w}$. In this case, $\langle w\rangle$ has always the same sign of $\langle \sinh r w \rangle$ with $r>0$, which is equal to the sign of $\gamma$.
As shown in Fig.~\ref{fig:plotfin1}, for $\gamma>0$ ($\gamma<0$) the certainty equivalent $w_{CE}$ increases (decreases) as the entropy increases for $r$ non-positive (non-negative), whereas $w_{CE}$ decreases (increases) as the entropy increases for $r$ positive (negative) and large in modulus. Thus, every risk loving (averse) agent prefers to gamble when the entropy is large (small) when the average gain $\langle w\rangle$ is positive (negative). On the other hand, when $\langle w\rangle$ is negative (positive), this becomes true only as the agent becomes more and more risk loving (averse).
% Figure environment removed
Furthermore, even the bound suggests that a risk non-neutral agent can prefer the opposite situation than a risk neutral one, e.g., we can have $\langle w\rangle <0$ and $\langle \sinh (rw) \rangle >0$ with $r>0$, so that the bound suggests that a risk neutral agent prefers to gamble when $\langle \sigma \rangle$ is small, whereas a risk non-neutral agent prefers $\langle \sigma \rangle $ large.
This can happen for a distribution with more than three points in the support, e.g., of the form
\begin{equation}
p_{2N}(\sigma,w)=K\sum_{n=-N,n\neq 0}^N e^{a_n/2} \delta(\sigma-a_n)\delta(w-w_n)
\end{equation}
with $K^{-1}=2 \sum_{n>0} \cosh(a_n/2)$, the support such that $w_n=-w_{-n}$ and $a_n=-a_{-n}$ to ensure the distribution satisfies the fluctuation theorem. As shown in Fig.~\ref{fig:plot0} in the region where $\langle w\rangle <0$ and $\langle \sinh (rw) \rangle >0$ with $r>0$, the certainty equivalent increases when the entropy increases, as suggested by the bound. In particular, we note that for $r=1$, the  certainty equivalent is maximum for a non-zero $\langle \sigma \rangle$, corresponding to a negative $\langle w \rangle$, so that the agent prefers to gamble even if on average the losses are different from zero.
% Figure environment removed




\section{Physical example}
Let us investigate a physical situation corresponding to a work extraction protocol.
We focus on a closed quantum system made of two parties $A$ and $B$. The time-dependent Hamiltonian $H(t)=H_A(t) + H_B(t) + H_{int}(t)$ generates the unitary time evolution operator $U_{t,0}$ defined by the Schr\"{o}dinger equation $i \dot U_{t,0}=H(t) U_{t,0}$ with initial condition $U_{0,0}=\mathds{1}$. In particular we consider the case in which the interaction $H_{int}(t)$ between the systems $A$ and $B$ is turned-off at the initial time $t=0$ and final time $t=\tau$, and the parameters are changed cyclically, so that $H(0)=H(\tau)=H$ and so $H_A(0)=H_A(\tau)=H_A$ and $H_B(0)=H_B(\tau)=H_B$. The initial state is locally at equilibrium, i.e.,  $\rho(0)= e^{-\beta_A H_A}\otimes e^{- \beta_B H_B}/Z$ where $Z$ is the normalization constant $Z =Z_A Z_B$, where $Z_X=\Tr{e^{-\beta_X H_X}}$ with $X=A,B$. Since the system is thermally isolated, the work extracted is minus the change of energy, so that the average work extracted is $\langle w \rangle = \Tr{H(\rho(0)-\rho(\tau))}$, where $\rho(\tau) = U_{t,0} \rho(0)U^\dagger_{t,0}$. We note that if $\beta_A=\beta_B$ the initial state is passive, i.e., it is not possible to extract a non zero average work for any unitary evolution $U_{t,0}$.
To get a stochastic work, we perform a two-projective measurement protocol, so that we perform local measurement of the energy on the two parties $A$ and $B$ at the initial time $t=0$, then we generate the time evolution with unitary operator $U_{t,0}$ by changing the Hamiltonian in the time interval $(0,\tau)$, and at the end we perform local measurement of the energy on the two parties $A$ and $B$ at the final time $t=\tau$.
We consider $w$ as the work extracted from the total system. We have the probability distribution function
\begin{eqnarray}
\nonumber p(\sigma,w) &=& \sum p_{m m'}P_{m m' n n'}\delta(\sigma-\sigma_{m m' n n'})\\
 && \times\delta(w-\epsilon^A_m-\epsilon^B_{m'}+ \epsilon^A_n+\epsilon^B_{n'})\,,
\end{eqnarray}
where the initial populations are $p_{m m'}= \bra{\epsilon^A_m,\epsilon^B_{m'}}\rho(0) \ket{\epsilon^A_m,\epsilon^B_{m'}}$, $\ket{\epsilon^X_m}$ are eigenstates of $H_X$ with eigenvalues $\epsilon^X_m$, with $X=A,B$, and $\sigma_{m m' n n'}=\beta_A(\epsilon^A_n-\epsilon^A_m)+\beta_B(\epsilon^B_{n'}-\epsilon^B_{m'})$. The transition probability is given by $P_{m m' n n'}=\abs{\bra{\epsilon^A_n,\epsilon^B_{n'}}U_{\tau,0} \ket{\epsilon^A_m,\epsilon^B_{m'}}}^2$.
% Figure environment removed
It is easy to see that this probability distribution satisfies the fluctuation theorem in Eq.~\eqref{eq.fluc}.
In order to proceed with our study, we consider the simple case of two qubits with local Hamiltonian $H_X(t)=\omega_X \sigma^X_z/2$, with $X=A,B$, and the interaction is
\begin{equation}
H_{int}(t) = \lambda(t) ((1+\gamma)\sigma^A_x \otimes \sigma^B_x+(1-\gamma)\sigma^A_y \otimes \sigma^B_y)
\end{equation}
where $\lambda(0)=\lambda(\tau)=0$ and $\sigma_x$, $\sigma_y$ and $\sigma_z$ are the Pauli matrices. The local ground states are $\ket{\epsilon^X_1}$ with energy $\epsilon^X_1=-\omega_X/2$, with $X=A,B$, while $\ket{\epsilon^X_2}$ are the local excited states with energy $\epsilon^X_2=\omega_X/2$. We consider $\omega_A>\omega_B$, it is easy to see that the initial state is non-passive if $\beta_B/\beta_A>\omega_B/\omega_A$, so that we expect $\langle w \rangle >0$ for certain unitary evolutions.
For $\gamma=0$ the excitation number is conserved and we can have only transitions in the sector with odd parity excitation, i.e., between the states $\ket{\epsilon^A_1 \epsilon^B_2}$ and $\ket{\epsilon^A_2 \epsilon^B_1}$. In this case the support of the distribution probability  of work has only three points, which are $w=\omega_B-\omega_A,0,\omega_A-\omega_B$. In contrast, for $\gamma\neq 0$, we can have also transitions in the sector with even parity excitation, i.e., also between the states $\ket{\epsilon^A_1 \epsilon^B_1}$ and $\ket{\epsilon^A_2 \epsilon^B_2}$, so that the support of the distribution probability of work has five points, which are $w=-\omega_A-\omega_B,\omega_B-\omega_A,0,\omega_A-\omega_B,\omega_A+\omega_B$.
To characterize the risk aversion, we consider the exponential utility function $u_{r}(t)$.
As shown in Fig.~\ref{fig:plotfin3}, for $\gamma=0$ the imprint of the entropy is strong, which determines completely the trend of the certainty equivalent. However, for $\gamma\neq 0$, the certainty equivalent follows the trend of the entropy only for not too large $r$, otherwise $w_{CE}$ shows oscillations with different period with respect to the ones of $\langle \sigma\rangle$. This example shows how in general very risk averse (loving) agents cannot look to entropy alone to make their choice. On the other hand, the entropy can be a useful reference quantity for agents not too far from being risk neutral.


\section{Conclusions}
In this paper we investigate the role of fluctuation theorems in expected utility hypothesis. For this purpose, we assume that the wealth is related to a stochastic entropy by a fluctuation theorem. We prove the existence of a bound which depends on the entropy and reproduces the thermodynamic uncertainty relation for a neutral risk agent. Thus, also by using this bound, we investigate how the certainty equivalent depends on the entropy.
%Since the certainty equivalent is related to the entropy
In particular, since those quantities are related
by the fluctuation theorem, we show that there are situations  where the entropy can be a reference quantity for a gamble. However, there are other situations where entropy alone is not enough. We illustrate this with the help of a physical situation of work extraction.

\subsection*{Acknowledgements}
The authors acknowledge financial support from the project BIRD 2021 "Correlations, dynamics and topology in long-range quantum systems" of the Department of Physics and Astronomy, University of Padova and from the European Union-Next Generation EU within the National Center for HPC, Big Data and Quantum Computing (Project No. CN00000013, CN1 Spoke 10 Quantum Computing).


\begin{thebibliography}{99}

\bibitem{vonNeumann} J. von Neumann and O. Morgenstern, Theory of Games and Economic Behavior (60th Anniversary Commemorative Edition) (Princeton University Press, 2007).
\bibitem{jarzynski97} C. Jarzynski, Phys. Rev. Lett. 78, 2690 (1997).
\bibitem{crooks99} G. Crooks, Phys. Rev. E, 60, 2721 (1999).
\bibitem{campisi11} M. Campisi, P. H\"{a}nggi, and P. Talkner, Rev. Mod. Phys. 83, 771 (2011).

\bibitem{Ducuara232} A. F. Ducuara, P. Skrzypczyk, arXiv:2306.07975 (2023).
\bibitem{Ducuara23} A. F. Ducuara, P. Skrzypczyk, F. Buscemi, P. Sidajaya, V. Scarani, arXiv:2306.00449 (2023).


\bibitem{Hasegawa19} Y. Hasegawa and T. Van Vu, Phys. Rev. Lett. 123, 110602 (2019).
\bibitem{Timpanaro19} A. M. Timpanaro, G. Guarnieri, J. Goold and G. T. Landi, Phys. Rev. Lett. 123, 090604 (2019).
\bibitem{Francica22} G. Francica, Phys. Rev. E 105, 014129 (2022).

\bibitem{Garcia10} R. Garc\'{i}a-Garc\'{i}a, D. Dom\'{i}nguez, V. Lecomte, and A. B. Kolton, Phys. Rev. E 82, 030104(R) (2010).



\bibitem{bookmicroeco} D. Mas-Colell, M. Winston, and J. Green, Microeconomic Theory, Oxford, Oxford University Press, 1995.
\bibitem{bookmicroeco2} D. Kreps, A Course in Microeconomic Theory, New Jersey, Princeton University Press, 1990.


\bibitem{campisi21} M. Campisi and L. Buffoni, Phys. Rev. E 104, L022102 (2021)


\end{thebibliography}


\end{document}
