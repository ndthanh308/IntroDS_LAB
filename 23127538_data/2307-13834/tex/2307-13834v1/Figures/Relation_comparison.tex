\begin{table}[H]
\centering
{\footnotesize
\begin{tabular}{|>{\scriptsize \centering}p{2.2cm}|>{\centering} p{1.2cm}| 
>{\centering} p{1.2cm}| >{\centering} p{1.2cm}|  >{\centering} p{1.2cm}|  >{\centering} p{1.2cm} | p{1cm} |}
\hline
\multicolumn{7}{|c|}{ Frequencies used and traces required to break} \\
\hline
{\footnotesize Frequencies} & Nr. Traces & Failed Enc & Removed Traces & Max delay & Timing overhead & Sim. overhead\\
\hline
\multicolumn{7}{|c|}{\hyperref[tab:simulation_results]{Two high, two lower than half}} \\											
\hline
$f_{base}$=10 $f_1$=14.4317 $f_2$=12.9781 $f_3$=4.4021  $f_4$=3.6719 & 5500 & 0\% & 31\% & 3800 & 125\% & -7\%  \\
\hline
$f_{base}$=10 $f_1$=14.5370 $f_2$=13.0883 $f_3$=4.3611  $f_4$=3.6725 & 5000 & 0.02\% & 30\% & 4200 & 136\% & -6\%  \\
\hline
$f_{base}$=10 $f_1$=11.8182 $f_2$=10.8333 $f_3$=4.8148  $f_4$=3.5017 & 6000 & 6.3\% & 34\% & 4300 & 140\% & 6\%  \\

\hline
\multicolumn{7}{|c|}{\hyperref[tab:simulation_results]{Three low, one lower than half}} \\
\hline
$f_{base}$=10 $f_1$=9.7991 $f_2$=9.1458 $f_3$=8.9959  $f_4$=4.4254 & 4500 & 0.52\% & 32\% & 2030 & 14\% & 2\%  \\
\hline
$f_{base}$=10 $f_1$=9.4001 $f_2$=9.3001 $f_3$=9.2003  $f_4$=4.003 & 4750 & 0.07\% & 41\% & 2000 & 12\% & 6\%  \\
\hline
$f_{base}$=10 $f_1$=9.8659 $f_2$=9.4485 $f_3$=8.9527  $f_4$=4.5009 & 4500 & 0.034\% & 23\% &2800 & 23\% & 10\%  \\
\hline
\end{tabular}
}
\caption{Frequency test on hardware}
{\footnotesize Contains the same information as \autoref{tab:attack_results} but instead of comparing the relation between frequencies and the base clock it compares specific frequencies. Proving whether small changes in frequencies have a effect on the results. Two types of frequencies are tested, with three sets of frequencies for each used. 
}
\label{tab:relation_comparison}
\end{table}
