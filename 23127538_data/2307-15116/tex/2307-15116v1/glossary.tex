\newglossaryentry{computsecure}
{
	name=computationally secure,
	description={means that security is based on the assumption that a given mathematical problem is hard to solve on a computer.  Cryptosystems that provide this type of security are thus vulnerable to potential future attacks that exploit breakthroughs in software development or novel hardware}
}

\newglossaryentry{inftheosecure}
{
	name=information-theoretically secure,
	description={(also called unconditionally secure) means that security is based on information-theoretic principles. In contrast to a computationally secure cryptographic system, an information-theoretically secure system is immune even to attackers with unlimited computational power}
}

\newglossaryentry{authentication}
{
	name=authentication,
	description={refers to a method for ensuring that the identity of the claimed sender of a message is correct. Authentication requires some initial resources, e.g., a common password held by the sender and the receiver of the message}
}

\newglossaryentry{RSA}
{
	name=RSA,
	description={is a classical algorithm for \gls{publickey} that is nowadays widely used to encrypt data transmission, named after its inventors Rivest, Shamir, and Adleman \cite{Rivest1978}. RSA is \gls{computsecure}, and the problem it relies on is factoring large numbers into prime factors. Because there exists a quantum algorithm for efficiently solving this problem~\cite{Shor1994}, RSA is vulnerable to attackers with access to a (universal) quantum computer}
}

\newglossaryentry{publickey}
{
	name=public-key cryptography,
	description={is a cryptosystem that uses pairs of related keys, consisting of a public and a private key. The public key is openly distributed for others to encrypt data, which can only be decrypted by those who know the corresponding private key. Similarly, public-key cryptography enables other functionalities, such as \gls{authentication} or electronic signatures. Public-key cryptography is usually \gls{computsecure}} 
}

\newglossaryentry{OTP}
{
	name=one-time pad,
	description={(OTP) encryption is a scheme where a message and a \gls{key} are combined via binary addition. The resulting ciphertext does not reveal any information about the encrypted message, but can be decrypted with the same key. This encryption scheme is \gls{inftheosecure}}
}

\newglossaryentry{storenow}
{
	name=``store now decrypt later'' attacks,
	description={exploit the fact that encrypted data can be intercepted during transmission and stored in its encrypted form, to be decrypted once more powerful (quantum) computers are available. These attacks pose a high risk to data that has a long shelf life, like medical records or military secrets}
}

\newglossaryentry{PQC}
{
	name=post-quantum cryptography,
	description={(PQC) refers to classical cryptographic algorithms believed to remain secure even when universal quantum computers are available. The security of these algorithms relies on the assumption that a given mathematical problem is hard to solve for any computational device, including future quantum computers. Post-quantum cryptography is thus usually \gls{computsecure} but not \gls{inftheosecure}}
}

\newglossaryentry{qrepeat}
{
	name=quantum repeaters,
	description={allow to establish entanglement over long distances via a procedure called \emph{entanglement swapping}, effectively enabling the transmission of quantum information over such distances. Since they work entirely on the quantum level, they are secured by the laws of quantum theory and hence don't have to be trusted}
}

\newglossaryentry{sidechannels}
{
	name=side-channel attacks,
	description={do not target the encryption or key distribution protocol itself, but exploit deviations of the implementation from the theoretical description. This could, for example, be leaked information on timing or power consumption, or imperfections of the devices}
}


\newglossaryentry{qmemory}
{
	name=quantum memory,
	description={is the quantum-mechanical analogue of classical computer memory. It stores quantum states for later retrieval}
}

\newglossaryentry{key}
{
	name=cryptographic key,
	description={refers to a bit string that is uniformly random and secret, i.e., known only to the honest communicating parties. This string may then be used, for example, for \gls{OTP} encryption}
}

\newglossaryentry{quantitativeproof}
{
	name=quantitative security proofs,
	description={give a quantitative bound on the probability that a security breach happens (cf.~\cref{fig:quantcost})}
}

\newglossaryentry{protocolsecurity}
{
	name=protocol security,
	description={describes the theoretical security of a protocol}
}

\newglossaryentry{implementationsecurity}
{
	name=implementation security,
	description={denotes the security of a practical implementation of a protocol (which can differ from the theoretical description)}
}

\newglossaryentry{universalqcomputer}
{
	name=universal quantum computers,
	description={are quantum devices that are able to run any quantum algorithm}
}

