
    \section{Polynomial Time Algorithm for \ref{APV}} \label{sec:apv}




The primary contribution of this paper is the development of a polynomial time algorithm for \ref{APV}. To achieve this, we introduce in Section \ref{subsection:expanded} the concepts of the ``Expanded Revenue" and ``Expanded Set" of an assortment. %, which are instrumental for our analysis. 
In Section \ref{subsection:prop}, we present a polynomial time algorithm to compute the expanded set and expanded revenue and demonstrate the monotonicity and supermodularity of the expanded revenue function. Leveraging these properties, we characterize the structure of an optimal solution for \ref{APV} and present an algorithm that computes it in $O(n + T)$ time (Section \ref{subsection:algo}). Finally, in Section \ref{subsectin:lp}, we demonstrate that \ref{APV} can be formulated as a compact linear program.



% In this section, the main result we will prove is the following:

% There exists a nested optimal solution to \ref{APV}, in which every assortment offered is included in the previous one. This solution can be computed in time $\mathcal{O}(nT)$.

% \begin{theorem}
%     The problem \ref{APV} can be solved in Polynomial time
% \end{theorem}

% To achieve this, we first introduce the notion of expanded set and expanded revenue, that will be useful for our analysis. We then prove some interesting monotonicity and supermodularity properties on the expanded revenue function. Relying on these properties, we finally identify a simple structure for an optimal solution of problem \ref{APV}, which can be computed in polynomial time.



\subsection{Expanded Revenue and Expanded Set} \label{subsection:expanded}

We begin our analysis by examining \ref{APV} in the context of a single customer. In this particular scenario, the visibility constraints are given such that either $\ell_i=0$ or $\ell_i=1$. Let $A$ denote the subset of all products where $\ell_i=1$. Consequently, \ref{APV} is transformed into the problem of identifying the assortment that maximizes revenue while including $A$. This particular problem will serve as  the building block for our analysis, as it lays the foundation for understanding the general case involving  $T$ customers. Thus, it leads us to introduce the subsequent definitions that will aid us in our analysis.

% This lead us to introduce the notion of the expanded revenue,  and the expanded revenue of a set and show how we con compute them. We introduce the following defintiions.



%Because of the visibility constraints, the assortments offered under \ref{APV} will have to enforce certain products. Thus it becomes interesting to study how we can maximize the expected revenue of a set once if we force it to contain certain elements, and we optimize over the remaining ones.

%We therefore naturally introduce the following notations:


\begin{definition}[\bf Expanded revenue]
    \label{expanded revenue}
    Let $A \subseteq \mathcal{N}$. The expanded revenue of $A$, denoted as $\overline{R}(A)$, is defined as the maximum expected revenue achieved by any assortment in ${\cal N}$ that contains $A$. In particular, it is given by

    \begin{equation} \label{eq:exrev}
        \overline{R}(A) \coloneqq \max_{S \subseteq \mathcal{N} , \; A \subseteq S} R(S).
    \end{equation}
\end{definition}

The optimal solution of the maximization problem in \eqref{eq:exrev} is referred to as the expanded set of $A$. In case multiple optimal solutions exist, we break ties by selecting the optimal assortment with the largest cardinality, which we show is unique in Lemma \ref{Compute expanded set}, hence proving that the expanded set is well defined. Formally, we have the following definition. 

\begin{definition}[\bf Expanded set]
    \label{expanded set}
The expanded set of $A$, denoted as $\overline{A}$, is defined as the assortment within $\mathcal{N}$ that  maximizes the expected revenue  among all assortments containing $A$.  If multiple assortments achieve the same maximum expected revenue, $\overline{A}$ is selected as the assortment with the largest cardinality.  Mathematically, $\overline{A}$ is given by
    $$\overline{A} \coloneqq \underset{S \subseteq \mathcal{N} , \; A \subseteq S}{\arg \max}   \left\{ |S| \; : \; R(S)=  \overline R (A)   \right\}.   $$
\end{definition}


% Let $A \subseteq \mathcal{N}$. The expanded set of $A$, denoted as $\overline{A}$, is defined as the assortment within $\mathcal{N}$ that satisfies two criteria: it contains $A$, and it simultaneously achieves the maximum revenue and the maximum cardinality among all possible assortments. Mathematically, $\overline{A}$ is given by:


% Note that Problem \eqref{eq:exrev} is equivalent to \ref{APV} in the scenario where we have a single customer $(T=1)$ and $A$ is defined as the set of products that have to be shown once, i.e., $A= \{ i \in {\cal N} : \ell_i =1 \} $. In that case, $\overline A$ corresponds to the optimal assortment to this problem with the largest cardinality.

% We will use $\overline{R}$ as a set function that takes input an assortment $A \subseteq R$ and returns $\overline{R}(A)$. In the next section we study several properties of this function as well as properties of the expanded set.



Problem \eqref{eq:exrev} can be viewed as equivalent to \ref{APV} when considering a single customer scenario $(T=1)$ and defining $A$ as the set of products that need to be shown once, i.e., $A= \{ i \in {\cal N} : \ell_i =1 \}$. Thus, $\overline A$ represents the optimal assortment, with the largest cardinality, for the problem.

In our  analysis, we consider $\overline{R}$ as a set function that takes an assortment $A \subseteq \mathcal{N}$ as input and returns $\overline{R}(A)$. Note that $\overline{R}(A) = R(\overline{A})$. In the subsequent section, we delve into examining various properties of this function, as well as properties associated with the expanded set.







%$\overline{R}$ is the expanded revenue function, such that $\overline{R}(A) = R(\overline{A})$. When there are several solutions for $\overline{A}$, we define $\overline{A}$ as the one with the highest cardinal. The following lemma justifies that $\overline{A}$ is well defined and can be computed efficiently.

\subsection{Properties of the Expanded Revenue} \label{subsection:prop}

In this section, we first show that we can compute the expanded revenue and the expanded set of a given assortment in polynomial time (Lemma \ref{Compute expanded set}). Then, we show that the expanded revenue is a non-increasing function and  the expanded set is a non-decreasing function (Lemma \ref{monotonicity}). 
Finally, we show that the expanded revenue function is supermodular (Lemma \ref{supermodularity}) which is the most fundamental property for our algorithm design later in the paper. %These  structural properties will be useful in designing our algorithm to solve \ref{APV} for $T \geq 1$.




\vspace{2mm}
\noindent
{\bf Computing the expanded revenue and expanded set.}
Recall without loss of generality that $p_1 \geq \ldots \geq p_n$. We define  an assortment $S$ to be price-ordered if $S=\{1,\ldots,k\}$ for some $1 \leq k \leq n$. Essentially, a price-ordered assortment prioritizes products with high prices. It is worth noting there  are only $n$ possible price-ordered assortments.
Consider an assortment $A \subseteq {\cal N}$, and its expanded set $\overline A$. In the following lemma, we demonstrate that $\overline A$ is the union of $A$ and a price-ordered assortment. Since there are only $n$ possible price-ordered assortments, it is sufficient to compute the expected revenue of the assortments $A \cup \{1,\ldots, k\}$ for each $k \in \{1, \ldots ,n\}$. The expanded set corresponds to the assortment with the highest expected revenue. In the case of multiple assortments with the same maximum revenue, we break ties by selecting the one with the largest cardinality. Thus, the expanded set $\overline A$ can be computed in linear time, specifically $O(n)$. The expanded revenue is simply $\overline{R}(A)= R(\overline A)$. The proof of Lemma \ref{Compute expanded set} leverages some structural properties of the revenue function under MNL that are presented in Appendix \ref{apx1}.

\begin{lemma}
    \label{Compute expanded set}
    For any $ A \subseteq \mathcal{N}$, the expanded set of $A$ is given by  
    $ \overline{A} = A \cup \{i \in \mathcal{N} : p_i \geq \overline{R}(A) \}.$ Furthermore, $\overline{R}(A)$ and $\overline{A}$ can be computed in time $O(n)$.
\end{lemma}
\begin{proof}
    By definition of the expanded set, we have $A \subseteq \overline{A}$. Hence, there exists an assortment $B \subseteq \mathcal{N} \setminus A$, such that $\overline{A} = A \cup B$. Let us show that $$B = \{i \in \mathcal{N} \setminus A   \; : \; p_i \geq \overline{R}(A) \}.$$ 
    \begin{itemize}
        \item {\em Direct inclusion}: Let $i\in B$, and assume by contradiction that $p_i < \overline{R}(A) = R(A \cup B)$. It is known that, under the MNL model, when we add a product $j$ to an assortment $S$, the revenue of this assortment increases if and only of $p_j \geq R(S)$. For completeness, we provide the statement and the proof of this result in Lemma \ref{Revenue variations} in Appendix \ref{apx1}. Using this lemma implies that removing $i$ from $B$ would strictly increase the expected revenue $R(A)$, which contradicts the optimality of $A \cup B$.
        \item {\em Indirect inclusion: } Let $i \in \mathcal{N} \setminus A$ such that $p_i \geq \overline{R}(A)$, and assume by contradiction that $i \notin B$. By Lemma \ref{Revenue variations}, adding $i$ to $B$ would increase the revenue. If this increase is strict, it contradicts the optimality of $A \cup B$. If the revenue stays the same, it contradicts the definition of $\overline{A} = A \cup B$ as the optimal solution with maximum cardinality. 
        \end{itemize}
    Thus,
        $$\overline{A} = A \cup \{i \in \mathcal{N} \setminus A, p_i \geq \overline{R}(A) \}.$$
        Finally, $\overline{A}$ can be computed in time $O(n)$. Indeed, we start from $A$, then we sequentially add elements by decreasing price. At each iteration, we can compute the new revenue from the previous one in constant time by storing the current numerator and denominator, since we only need to add $p_i v_i$ to the former and $v_i$ to the latter when we reach element $i$. Finally, we pick the highest revenue set among the $n$ computed sets.
\end{proof}
    

%Next, we show that the expanded revenue is a non-increasing function and  the expanded set is a non-decreasing function.


\begin{lemma}[\bf Monotonicity]
    \label{monotonicity} 
    If $ A \subseteq B \subseteq \mathcal{N}$, then  $ \overline{A} \subseteq \overline{B}$ and $\overline{R}(A) \geq \overline{R}(B)$.   
\end{lemma}

\begin{proof}
    For $A \subseteq B \subseteq \mathcal{N}$, we have $\{S \subseteq \mathcal{N} \; :\; B \subseteq S \} \subseteq \{S \subseteq \mathcal{N} \; : \; A \subseteq S \}.$ Therefore, every feasible solution for $\max_{S \subseteq \mathcal{N} , \; B \subseteq S} R(S)$ is a feasible solution for $\max_{S \subseteq \mathcal{N} , \; A \subseteq S} R(S)$. Hence, $\overline{R}(A) \geq \overline{R}(B)$. It follows that $\{i \in \mathcal{N}, p_i \geq \overline{R}(A) \} \subseteq \{i \in \mathcal{N}, p_i \geq \overline{R}(B) \},$ and therefore $\overline{A} = A \cup \{i \in \mathcal{N}, p_i \geq \overline{R}(A) \} \subseteq B \cup \{i \in \mathcal{N}, p_i \geq \overline{R}(B) \} = \overline{B}.$
\end{proof}

%Finally, we show that  the expanded revenue function  $\overline R$ is supermodular. This  property  will play  a fundamental role later in our analysis.


\begin{lemma}[\bf Supermodularity]
    \label{supermodularity}
    The expanded revenue function $\overline{R}$ is  supermodular, i.e.,
     $$\forall A, B \subseteq \mathcal{N},  \; 
     \; \overline{R}(A \cup B) + \overline{R}(A \cap B) \geq \overline{R}(A) + \overline{R}(B).$$
\end{lemma}
\begin{proof}[Proof for Lemma \ref{supermodularity}]
In this proof, we use the following alternative definition of supermodularity. A function $f\colon \Omega\rightarrow \R$ is supermodular if and only if for all $A,B\subseteq \Omega$ such that $A\subseteq B$, and each $i\in \Omega\setminus B$, $f(B\cup\{i\})-f(B) \geq f(A\cup\{i\}) - f(A)$.

Following this definition, let $A,B\subseteq \Nc$ such that $A\subseteq B$, and let $i\in \Nc\setminus B$. The proof of this result is separated into two steps. In the first step, we show that\begin{equation}\label{eq:step1}
    R\left(\overline{B}\right) - R\left(\overline{B}\cup \overline{A\cup\{i\}}\right) \leq R\left(\overline{A}\right) - R\left( \overline{A\cup\{i\}}\right). 
\end{equation}
Subsequently, we show in the second step that\begin{equation}\label{eq:step2}
    R\left(\overline{B}\cup \overline{A\cup\{i\}}\right)- R\left(\overline{B\cup \{i\}}\right) \leq 0.
\end{equation}
The result follows directly by summing the inequalities \eqref{eq:step1} and \eqref{eq:step2} term by term.
\paragraph{Step 1.} 
Let us start with the following claim which follows from simple algebra.
\begin{claim}\label{cl:computation}
    For any assortment $S_1, S_2\subseteq \Nc$ such that $S_1\subseteq S_2$, we have\begin{equation*}
        R(S_1) - R(S_2) = \frac{1}{1+V(S_2)}\cdot \sum_{j\in S_2\setminus S_1}(R(S_1) - p_j)\cdot v_j.
    \end{equation*}
\end{claim}
\noindent Using this claim, we have
\begin{equation*}
    R\left(\overline{B}\right) - R\left(\overline{B}\cup \overline{A\cup\{i\}}\right) = \frac{1}{1+V
        \left(\overline{B}\cup \overline{A\cup\{i\}}\right)}\cdot \sum_{j\in \overline{A\cup \{i\}}\setminus \overline{B}} \left(R\left(\overline B\right) - p_j\right)v_j.
\end{equation*}
Next, we know by definition of the extended set of $B$ that $R(\overline{B}) \geq p_j$ for all $j \notin \overline{B}$. Therefore, since $V(\overline{B}\cup \overline{A\cup\{i\}})$ is trivially greater than or equal to $V(\overline{A\cup\{i\}})$, we have,
\begin{align}
    R\left(\overline{B}\right) - R\left(\overline{B}\cup \overline{A\cup\{i\}}\right) &\leq \frac{1}{1+V
        \left(\overline{A\cup\{i\}}\right)}\cdot \sum_{j\in \overline{A\cup \{i\}}\setminus \overline{B}} \left(R\left(\overline B\right) - p_j\right)v_j \notag\\
        & \leq \frac{1}{1+V
        \left(\overline{A\cup\{i\}}\right)}\cdot \sum_{j\in \overline{A\cup \{i\}}\setminus \overline{B}} \left(R\left(\overline A\right) - p_j\right)v_j,\label{eq:smallsum}
\end{align}
where the second inequality follows from Lemma \ref{monotonicity}.
Finally, we have $$\overline{A\cup \{i\}}\setminus \overline{B} \subseteq \overline{A\cup \{i\}}\setminus  \overline{A}.$$
Moreover, for every $j \in \overline{A\cup \{i\}}\setminus  \overline{A}$, $j\notin \overline{A}$, and in particular, $p_j\leq R(\overline{A})$. Therefore, by adding the missing terms to the sum in Equation \eqref{eq:smallsum}, we have
    \begin{align*}
        R\left(\overline{B}\right) - R\left(\overline{B}\cup \overline{A\cup\{i\}}\right) & \leq \frac{1}{1+V
        \left(\overline{A\cup\{i\}}\right)}\cdot \sum_{j\in \overline{A\cup \{i\}}\setminus \overline{A}} \left(R\left(\overline A\right) - p_j\right)v_j\\
        & = R\left(\overline{A}\right) - R\left(\overline{A\cup \{i\}}\right),
    \end{align*}
where the equality follows from Claim \ref{cl:computation}. This concludes the first step.
\paragraph{Step 2. }On one hand, we have $\{i\}\subseteq\overline{A\cup \{i\}}$. Therefore, we have in particular ${\{i\} \subseteq \overline{B}\cup\overline{A\cup \{i\}}}$. On the other hand,  $B\subseteq \overline{B}$ and therefore $B\subseteq \overline{B}\cup\overline{A\cup \{i\}}$. Hence $B\cup \{i\}\subseteq \overline{B}\cup\overline{A\cup \{i\}}$. Recalling that $\overline{B\cup \{i\}}$ is by definition the maximum revenue assortment containing $B\cup \{i\}$, we have$$
    R\left(\overline{B\cup \{i\}}\right) \geq R\left(\overline{B}\cup\overline{A\cup \{i\}}\right),
$$
which concludes the second step, and thereby the proof of the lemma.
% Let $m \coloneqq |\overline{A\cup\{i\}}|$, and let $j_1, \ldots, j_m$ be the elements of $\overline{A\cup\{i\}}$ in the order of decreasing prices, i.e., $p_{j_1}\geq \ldots, p_{j_m}$. Let $C_0 \coloneqq  \overline{A}, D_0 \coloneqq \overline{B}$, and for all $q=0,\ldots,m-1$, $C_{q+1} = C_q\cup \{j_{q+1}\}$ and $D_{q+1} = D_q\cup \{j_{q+1}\}$. Next, we show by induction that for all $q=0,\ldots, m-1$, we have
% \begin{enumerate}
%     \item[(i)] $R(\overline B) - R(D_q) \leq R(\overline A) - R(C_q)$;
%     \item[(ii)] $R(D_{q}) \geq p_{j_{q}}$ and $R(C_{q}) \geq p_{j_{q}}$;
%     \item[(iii)] $R(C_{q})$
% \end{enumerate}

%     \begin{equation*}
%         R(D_{q}) - R(D_{q+1}) \leq R(C_{q}) - R(C_{q+1}).
%     \end{equation*}
\end{proof}
















\subsection{Optimal Solution  for \ref{APV}} \label{subsection:algo}

In this section, we present the main technical result in this paper. In particular, we characterize the structure of an optimal solution of \ref{APV}. Our characterization relies on the supermodularity property of the expanded revenue function. Moreover, we show that we can compute such a solution in $O(n+T)$, which gives us a polynomial time algorithm to solve \ref{APV}.

\vspace{2mm}
\noindent
{\bf Optimal solution.} Consider an instance of \ref{APV}. Recall that for all $i \in {\cal N}$,  $\ell_i$ is the lower bound on the minimum number of customers for which we must offer product $i$. For $t \in \{0,1,\ldots,T \},$ we define the following sets
\begin{equation}
    L_t = \{i \in \mathcal{N}, \ell_i = t \}.
\end{equation}
Our candidate solution for \ref{APV} is given by
\begin{equation} \label{eq:sol}
    {S_t^*} = \overline{\bigcup_{t \leq u \leq T} L_u}, \quad \forall t \in \{1,\ldots,T \}.
\end{equation}


Note that $(L_t)_{0 \leq t \leq T}$ is a partition of $\cal N$. Moreover, since  $ \bigcup_{t+1 \leq u \leq T} L_u  \subseteq  \bigcup_{t \leq u \leq T} L_u$, the monotonicity property in Lemma \ref{monotonicity} implies that $S_{t+1}^* \subseteq S_{t}^*$ for any $t=0,\ldots,T-1$. Therefore, our solution has a nested structure, i.e.,
$ S_T^* \subseteq S_{T-1}^* \ldots \subseteq S_1^*.$ In the following, we prove that the assortments given by \eqref{eq:sol} are optimal for \ref{APV}. Moreover, they can be computed in polynomial time. Indeed, Lemma \ref{Compute expanded set} shows that each of them can be computed in time $O(n)$, so the entire solution can be computed in time $O(nT)$. We can further improve the running time to $O(n+T)$ as shown below.



% \begin{definition}[\bf Nested solution]
%     \label{Nested solution}
%     For each instance of the problem \ref{APV}, we define the partition $(L_t)_{0 \leq t \leq T}$ of $\mathcal{N}$ by: $\forall t \in [\![0, T]\!], \; L_t = \{j \in \mathcal{N}, \ell_j = t \}$ \\
%     We define the sequence of assortments $(S_t^*)_{1 \leq t \leq T} \coloneqq (\overline{\bigcup_{t \leq u \leq T} L_u})_{1 \leq t \leq T}$, that is $S_1^*, S_2^*, \ldots, S_T^* = \overline{L_1 \cup \ldots \cup L_T}, \overline{L_2 \cup \ldots \cup L_T}, \ldots, \overline{L_T}$
% \end{definition}

%We observe that this solution structure is nested: $S_{t+1}^* \subseteq S_t^* \;\; \forall t \in [\![1, T-1]\!] $


\begin{theorem}{}
    \label{Solution structure}
    The  sequence of assortments $(S_t^*)_{1 \leq t \leq T}$ defined in \eqref{eq:sol} is optimal for \ref{APV}. % Moreover, each permutation of these sets is also an optimal solution for \ref{APV}. 
    Moreover, such a solution can be computed in $O(n+T)$ time.
\end{theorem}




\begin{proof}
    We prove the result by induction. First, for $T= 1$, for any $i\in \Nc$, we either have $\ell_i=0$ or $\ell_i=1$. Noting that $L_1$ is the set of products $i$ such that $\ell_i=1$, \ref{APV} reduces to the problem of finding the optimal assortment that contains $L_1$, i.e., $$
        \max_{S\subseteq \Nc\text{ s.t }L_1\subseteq S}R(S),
    $$
    whose solution is $\overline{L_1}$ by definition of the expanded set. The result follows for $T=1$ by noticing that $S_1^* =\overline{L_1}$.

    Let us now prove the result for a general number of customers. Let $T\geq 2$, and assume by induction that the result is true for $T-1$, in other words, given $T-1$ customers, and for any set of visibility constraints, the optimal solution of \ref{APV} is given by the assortments defined in Equation \eqref{eq:sol}. Let us show that the result holds for $T$ customers.
    %The outline of the proof is as follows. At first, we consider an optimal solution, then using the supermodularity property, sequentially modify it into another optimal solution where at least one assortment contains $L_1$, i.e., the assortment of products which must be shown at least once.
    We denote by $A$ the set of all products which must be shown to at least one customer due to the visibility constraints, i.e., $A\coloneqq \bigcup_{t=1}^TL_t$. We start by providing the following crucial intermediary claim.
    \begin{claim}\label{cl:intermediary}
        There exists an optimal solution $S_1, \ldots, S_T$ to \ref{APV} such that $S_1 \supseteq A$. In particular, there exists an optimal solution $S_1, \ldots, S_T$ to \ref{APV} such that $S_1 =S_1^*$.
    \end{claim}
    In other words, Claim \ref{cl:intermediary} states that there exists an optimal solution $S_1,\ldots, S_T$ to \ref{APV} such that $S_1$ contains all the products that must be shown at least once due to the visibility constraints, i.e., products $i$ such that $\ell_i\geq 1$. Consequently, this allows us to focus only on those feasible solutions of \ref{APV}, which offer $S_1^*$ to customer $1$. Maximizing the revenue amongst said solutions thereby guarantees attaining the optimal objective. In the remainder of this proof, we start by showing Claim \ref{cl:intermediary}, before leveraging it to conclude our induction.
    \paragraph{Proof of Claim \ref{cl:intermediary}. }The proof of this result mainly relies on a judicious exploitation of the supermodularity property. Assume by contradiction that there exists no optimal solution of \ref{APV} such that $S_1 \supseteq A$. Let $\hat S_1, \ldots, \hat S_T$ be the optimal solution of \ref{APV} that maximizes $|\hat S_1\cap A|$. In the case of ties, we pick any arbitrary solution that maximizes $|\hat S_1\cap A|$. By the contradiction hypothesis, $A \nsubseteq \hat S_1$. In particular, there exists some product $j \in \Nc$ such that $j\in A$ and $j \notin \hat S_1$. Moreover, since $j\in A$, we know that $\ell_j\geq 1$, and therefore, $j$ must be shown at least to $1$ customer, which means that there exists some $u\in [T]\setminus \{1\}$ such that $j\in \hat S_u$. We define the following new solution to \ref{APV}: $S_1 = \hat S_1\cup \hat S_u$, $S_u = \hat S_1\cap \hat S_u$, and $S_t = \hat S_t$ for all $t\notin \{1,u\}$. First, this newly defined solution is also feasible since any product offered once in either $\hat S_1$ or $\hat S_u$ is also shown in $S_1$, and each product shown in both $\hat S_1$ and $\hat S_u$ is also shown in both $S_1$ and $S_u$. Second, $S_1,\ldots, S_T$ is also an optimal solution. Indeed, by the supermodularity property, we have $$
        R\left(\hat S_1\cup \hat S_u\right)+ R\left(\hat S_1\cap \hat S_u\right) \geq R\left(\hat S_1\right)+ R\left(\hat S_u\right).
    $$
    Therefore,$$
        \sum_{t=1}^TR\left(S_t\right) \geq \sum_{t=1}^TR\left(\hat S_t\right).
    $$
    Third, we have $|S_1\cap A| \geq |\hat S_1\cap A|+1$ since $\hat S_1\cap A\subsetneq S_1\cap A$, as the latter set contains $j$ but the former does not. This contradicts the definition the $\hat S_1,\ldots, \hat S_T$, as the optimal solution that maximizes $|\hat S_1\cap A|$, and thereby proves by contradiction that there exists a solution $S_1,\ldots, S_T$ such that $S_1 \supseteq A$.
    
    Finally we show that we can take $S_1 = S_1^*$ in particular. On one hand, the solution $A, S_2, \ldots, S_T$ is also feasible, as it is obtained by removing all the products $j$ such that $\ell_j=0$ from $S_1$, which cannot break any constraints. Therefore, noting that $\overline A = S_1^*$, the solution $S_1^*, S_2, \ldots, S_T$ is also feasible, since $A\subseteq \overline{A}$. Finally, we have\begin{align*}
        R\left(S_1^*\right)+\sum_{t=2}^T R\left(S_t\right) &= R\left(\overline A\right)+\sum_{t=2}^T R\left(S_t\right)\geq \sum_{t=1}^TR(S_t),
    \end{align*}
    where the inequality follows from the definition of the expanded set of $A$, and the fact that $A\subseteq S_1$. In conclusion, there exists an optimal solution of \ref{APV} such that $S_1 = S_1^*$.
    \paragraph{Concluding the proof of the theorem. }
    In Claim \ref{cl:intermediary}, we showed the existence of an optimal solution which offers $S_1^*$ to the first customer. Therefore, restricting the search space to only such solutions still guarantees obtaining an optimal solution. Therefore, \ref{APV} is equivalent to the following optimization problem:

        \begin{equation*}
        \begin{aligned}
         \max_{ S_2, \ldots, S_T \subseteq \mathcal{N}}  & \; \;      R(S_1^*)+\sum_{t=2}^T  R(S_t)   \\  
          s.t. \;\;   & \;\;  1+\sum_{t=2}^T  \mathbbm{1}(i \in S_t) \geq \ell_i, \;\;\; \forall i \in A,
        \end{aligned}
        \end{equation*}
    which itself reduces to the following different instance of \ref{APV}.
    \begin{equation}\label{eq:reducedinstance}
        \begin{aligned}
         \max_{ S_2, \ldots, S_T \subseteq \mathcal{N}}  & \; \;      \sum_{t=2}^T  R(S_t)   \\  
          s.t. \;\;   & \;\;  \sum_{t=2}^T  \mathbbm{1}(i \in S_t) \geq \Tilde \ell_i, \;\;\; \forall i \in \Nc,
        \end{aligned}
    \end{equation}
    where $\Tilde \ell_i = \ell_i - \mathbbm 1(i\in A)$, for all $i\in \Nc$. Noting that this last problem is an instance of \ref{APV} with $T-1$ customers, we can apply the induction hypothesis, which implies that the optimal solution is given by Equations \eqref{eq:sol}. Directly applying the formulas to this new instance implies that $S_2^*, \ldots, S_T^*$ is an optimal solution to \eqref{eq:reducedinstance}, and hence that $S_1^*,\ldots,S_T^*$ is an optimal solution for \ref{APV}.
    \paragraph{Running time. }The running time of the algorithm for APV can be improved from $O(nT)$ to $O(n+T)$. In fact, we proceed by induction. We start by computing $S_T^*$. Then, when computing $S_t^*$, we do not need to directly compute $\overline{\bigcup_{t \leq u \leq T} L_u}$. Instead, since $S_t^*\supseteq S_{t+1}^* $, we just need to check the products $\{i \in \mathcal{N}\backslash (S_{t+1}^* ) : p_i \geq \overline{R}(S_{t}^* ) \}$, which  is achieved in $O(1+|S_t^*|-|S_{t+1}^*|)$. Therefore, the running time is 
$O(1+|S_T^*|)+\sum_{t=1}^{T-1} O(1+|S_t^*|-|S_{t+1}^*|)=O(|S_1^*|+T)= O(n+T).$

\end{proof}


















%\vspace{-7mm}

\subsection{Linear Program for \ref{APV}} \label{subsectin:lp}
\vspace{-2mm}

Consider the classic unconstrained assortment problem under MNL model for a single customer 
\begin{equation}
\label{Unconstrained problem}
\begin{aligned}
\max_{S \subseteq \mathcal{N}} \quad  R(S).
\end{aligned}
\tag{\sf{AP}}
\end{equation}
It is known that 
\ref{Unconstrained problem} can be formulated as the following LP (\cite{Gallego2011AGA}),
\vspace{-3mm}
\begin{equation*}
\label{Unconstrained problem LP}
\begin{aligned}
 \max_{S \subseteq \mathcal{N}} R(S) =  \max_{(\alpha_i)_{0 \leq i \leq n}} \left\{ \sum_{i=1}^n p_i \alpha_i \quad s.t. \quad \forall i \in \mathcal{N}, 0 \leq \frac{\alpha_i}{v_i} \leq \alpha_0, \quad \sum_{i=0}^n \alpha_i = 1 \right\}.
\end{aligned}
\end{equation*}
% \textcolor{red}{Omar: review and shorten this paragraph}\\
% The fact that the \ref{Unconstrained problem} optimal solution (as outlined in Appendix \ref{apx1}) is equivalent to the optimal solution of the above LP can be seen as follows.
% Let $S$ be an optimal solution to the initial problem \ref{Unconstrained problem}. Then, we can define $\alpha_0 = \frac{1}{1 + V(S)}$ and $\forall i \in \mathcal{N}, \alpha_i = \mathbbm{1}_{\{i \in S\}} \frac{v_i}{1 + V(S)}$, which is a feasible solution to the LP. And the two objective functions have the same value. Therefore, the optimal value of the LP is greater or equal to the optimal value of the initial problem. 
% Let $(\alpha_i)_{0 \leq i \leq n}$ be an optimal solution to the LP. Then, we define $S = \{i \in \mathcal{N},~\alpha_i = v_i \alpha_0 \}$, which is feasible for the initial problem. Since there are $n+1$ variable in the LP,  we can find a solution such that at least $n+1$ constraints are tight (a solution on an extremal point of the feasibility polytope). The equality constraint will always be verified, and since within the $2n$ inequality constraints, each one is incompatible with another, we have that for each $i, \alpha_i \in \{0, \alpha_0 v_i\}$. Therefore, the two objective functions have the same value. This proves that the \ref{Unconstrained problem} optimal value is greater or equal to the optimal value of the LP.
% As a result, the two problems have the same optimal value, which proves the equivalence. %and therefore each optimal solution for one yields an optimal solution for the other.
Motivated by the structure of the above LP  and the structure of our optimal solution of \ref{APV} given in Equation \eqref{eq:sol},
we propose the following linear formulation for  \ref{APV}.

\begin{theorem}[\bf LP for \ref{APV}] \label{thm:LPformulation}
    \ref{APV} is equivalent to the following linear program:
    \begin{equation}\label{LP}\tag{\sf LP}
    \begin{aligned}
     \max_{\boldsymbol{\alpha}} \;\; & \sum_{i=1}^n p_i \sum_{t=1}^T \alpha_i^t \\
     s.t. \quad &  \sum_{i=0}^n \alpha_i^t = 1, &&\quad\forall t \in [T], \\ 
     & \alpha_i^t = v_i \alpha_0^t, &&\quad\forall i \in \mathcal{N}, \;\; \forall t \in \{1,\ldots,\ell_i\},\\
     &  0 \leq \alpha_i^t \leq v_i \alpha_0^t ,&&\quad\forall i \in \mathcal{N}, \;\; \forall t \in \{\ell_i+1,\ldots, T\}.
    \end{aligned}
    \end{equation}
\end{theorem}

\begin{proof}
    In this proof, we use $\opt^{LP}$ and $\opt^{APV}$ to denote the values of \ref{LP} and \ref{APV} respectively. Our objective is to show that these values are equal. We demonstrate, on one hand, that $\opt^{LP} \geq \opt^{APV}$, by constructing a feasible solution to \ref{LP} whose objective is greater than or equal to $\opt^{APV}$. On the other hand, we demonstrate using the same technique that $\opt^{LP} \leq \opt^{APV}$. The combination of these two inequalities directly implies the desired result.
    
    \noindent {\em \underline{First inequality}. } Recall that $(S_1^*, \ldots,S_T^*)$ is the optimal solution of \ref{APV}, whose expression is stated in Equations \eqref{eq:sol}. We introduce a solution $\boldsymbol{\alpha}$ for \ref{LP}, defined as follows, for every $t\in [T]$, \begin{align*}
        &\alpha_0^t = \frac{1}{1+V(S_t^*)},\\
        &\alpha_i^t = \frac{v_i}{1+V(S_t^*)}\cdot \mathbbm 1\left(i\in S_t^*\right)\quad\quad\text{ for } i\in \Nc.
    \end{align*}
    Let us show that $\boldsymbol \alpha$ is feasible. The first and third constraints are straightforward. Indeed, for the first constraint, we have for all $t\in [T]$, \begin{equation*}
        \sum_{i=0}^n\alpha_i^t = \frac{1}{1+V(S_t^*)} + \sum_{i\in S_t^*}\frac{v_i}{1+V(S_t^*)} =1.
    \end{equation*}
    The third constraint is also directly verified by construction, as for all $i\in \Nc$ and $t\in [T]$, we have
    \begin{equation*}
        \alpha_i^{t} = \frac{v_i}{1+V(S_t^*)}\cdot \mathbbm 1\left(i\in S_t^*\right) \leq \frac{v_i}{1+V(S_t^*)} = v_i\alpha_{0}^t.
    \end{equation*}
    Regarding the second constraint, let $i\in \Nc$ and $t\in \{1,\ldots, \ell_i\}$. We have\begin{equation*}
        i\in L_{\ell_i} \subseteq \bigcup_{t\leq u\leq T}L_u \subseteq \overline{\bigcup_{t\leq u\leq T}L_u} = S_t^*.
    \end{equation*}
    Therefore, $$
        \alpha_{i}^t = \frac{v_i}{1+V(S_t^*)} = v_i\alpha_0^t,
    $$
    which proves that the second constraint is respected, and hence shows the feasibility of the constructed solution. Finally, its objective is equal to the revenue of the sequence of assortments $S_1^*,\ldots, S_T^*$ as demonstrated by the following easy computation:\begin{equation*}
        \sum_{i=1}^n p_i\sum_{t=1}^T \alpha_i^t = \sum_{t=1}^T\sum_{i=1}^n p_i \alpha_i^t = \sum_{t=1}^T\sum_{i\in S_t^*} p_i \phi(i,S_t^*) = \sum_{t=1}^TR(S_t^*) = \opt^{APV}.
    \end{equation*}
    Since the objective of $\boldsymbol{\alpha}$ is trivially upper bounded by $\opt^{LP}$, we deduce that  ${\opt^{LP} \geq \opt^{APV}}$.

    \noindent {\em \underline{Second inequality}. } Consider an optimal basic feasible solution for \ref{LP}, denoted by $\boldsymbol{\alpha}\coloneqq(\alpha_{i}^t\,:\,i\in \Nc\cup\{0\}, t\in [T])$. \ref{LP} is a linear program with $T\cdot(n+1)$ variables. Any basic solution has at least $T\cdot(n+1)$ active constraints. Since \ref{LP} already contains $T+\sum_{i\in \Nc}\ell_i$ equality constraints, $\boldsymbol{\alpha}$ activates at least $\sum_{i\in \Nc}(T-\ell_i)$ constraints from the remaining $2\cdot \sum_{i\in \Nc}(T-\ell_i)$ inequality constraints. Moreover, since each pair of constraints $\alpha_i^t \geq 0$ and $\alpha_i^t\leq v_i\alpha_0^t$ consists on two incompatible constraints, at most $\sum_{i\in \Nc}(T-\ell_i)$ (i.e., half of the inequality constraints) can be active. Thus, $\boldsymbol\alpha$ activates exactly $\sum_{i\in \Nc}(T-\ell_i)$ inequality constraints, and we have for all $i\in \Nc$ and $t\in \{\ell_i+1,\ldots,T\}$, $\alpha_{i}^t \in \{0, v_i\alpha_0^t\}$. Using this new observation, we construct a feasible solution $(S_1,\ldots, S_T)$ for \ref{APV} as follows: for all $t\in [T]$, let $S_t \coloneqq \{i\in \Nc\,\colon\, \alpha_i^t\neq 0\}$. In particular, if $i \in S_t$, then $\alpha_i^t = v_i\alpha_0^t$. It is then easy to see that $\alpha_i^t$ is exactly the choice probability of product $i$ in assortment $S_t^*$. Indeed, for all $t\in [T]$, we have$$
        \alpha_0^t+\sum_{i\in \Nc} \alpha_i^t= \alpha_0^t+ \sum_{i\in S_t}v_i \alpha_0^t= \alpha_0^t\cdot (1+V(S_t)),
    $$
    which implies using the first constraint of \ref{LP} that $\alpha_i^0 = \phi(0,S_t)$, and thereby that $\alpha_i^t = \phi(i, S_t)$ for all $i \in S_t$.
    Next, we can easily see that the $(S_1, \ldots, S_T)$ is feasible for \ref{APV}, as any product $i\in \Nc$ is included in the first $\ell_i$ assortments, which guarantees the visibility constraints. Finally we have$$
        \opt^{LP} = \sum_{i\in \Nc}p_i \sum _{t=1}^T\alpha_i^t = \sum_{t=1}^T\sum_{i\in \Nc}p_i\phi(i,S_t) = \sum_{t=1}^TR(S_t) \leq \opt^{APV}.
    $$
    
    \noindent {\em \underline{Conclusion}. }Combining the inequalities from the two cases shows that $\opt^{APV} = \opt^{LP}$, thereby demonstrating the equivalence between the two optimization problems \ref{LP} and \ref{APV}.
\end{proof}