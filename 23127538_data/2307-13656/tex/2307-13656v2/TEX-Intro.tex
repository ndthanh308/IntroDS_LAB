
\section{Introduction}



Assortment optimization is a crucial aspect of  decision making in many industries  such as  e-retail and online advertising. In this domain, our goal is to select a subset of available products  to offer to customers in order to maximize a context-appropriate objective function, %In this context, the objective function might vary depending on the business's priorities, 
such as revenue, profit, or market share. For example, e-retailers seek to strategically select which products should be displayed to customers in order to maximize their expected revenue. Online  advertisers strategically select the most effective combination of advertisements to maximize user engagement and desired outcomes, such as click-through rates. 
Consequently, the choice of a well formed assortment is crucial due to inherent substitution effects, where a product's attractiveness depend not only on its intrinsic value but also on the concurrent alternatives presented at that time. For instance, offering a high-quality, high-priced product alongside a comparable product at a significantly lower price may result in diminishing sales for the higher-priced product, leading in turn to an unsatisfactory platform revenue. This highlights the importance of carefully selecting assortments. 





Traditionally, assortment optimization frameworks often overlook a crucial element in contemporary e-commerce: product visibility. In today's complex business landscape, where companies adhere to Service-Level Agreements (SLAs) with suppliers and prioritize sponsored product promotion, product visibility within an assortment is pivotal. 
SLAs  often define conditions for product representation, ensuring equitable visibility for each supplier's products on the platform. Moreover, the concept of sponsored products has gained traction, with brands willing to pay for prominent display and increased visibility. While these strategies influence consumer behavior,  solely focusing on products visibility without considering broader assortment optimization can lead to an imbalanced product mix, resulting in reduced customer satisfaction and overall revenue.

In this paper, we introduce the notion of {\em visibility constraints} in the context of Assortment Optimization. The purpose is to enforce a minimum display of each product, i.e., each product has to be shown at least a certain number of times in the displayed assortments. This constraint models both Service-Level Agreements and sponsored products. It can also capture the settings where the platform would like to ensure some fairness notion among vendors by ensuring that each product is given a ``fair'' chance, i.e., it is shown at least to a certain number of customers. 
Specifically, we are given a universe of substitutable products and a stream of $T$ customers. For each  customer, we have to offer an assortment from the universe of products. The customer decides to purchase one of these products, or to leave without purchasing any product (no-purchase option). We assume that the choice of the customer is governed by a Multinomial Logit (MNL) choice model. We enforce the constraint that each  product in the universe has to be shown a minimum number of times among the $T$ assortments offered. The minimum display requirement for each product is given exogenously.   Our objective is to maximize the total expected revenue from  the $T$ customers. We refer to this combinatorial optimization problem as {\em Assortment optimization Problem with Visibility} constraints, concisely denoted as \ref{APV}.

A first natural question concerns the complexity of \ref{APV}. In fact, without visibility constraints, the problem reduces to the classic unconstrained revenue maximization problem under MNL, for which we know that the optimal assortment is revenue-ordered (\cite{talluri2004revenue}) and therefore can be solved in polynomial time. However, by enforcing the visibility constraints, we might have to include certain products in the assortment that cannibalize the sales of other more profitable products because of the substitution effect.
Consequently, determining an optimal sequence of assortments for \ref{APV} is not immediately clear. 
Subsequently, a natural follow-up problem concerns the introduction of cardinality constraints on the offered assortments of products. These form of constraints are widely used in the literature \citep{Davis2013AssortmentPU, gallego2014constrained, desir2020constrained}, as they model multiple applications where vendors have limited shelf space, or limited screen size for online vendors.
Finally, an interesting question emerging from the visibility problem is the task of quantifying the revenue loss incurred by enforcing visibility constraints compared to the relaxed unconstrained problem. 
This challenge compels us to develop a pricing strategy that appropriately apportions the loss to different vendors based on the impact of their product on the overall revenue.
Such a scenario frequently occurs within the framework of SLAs. Typically, a contract between a platform and a vendor includes a clause that guarantees a certain level of visibility to the vendor's product. In return for this visibility, the vendor compensates the platform with a fee. 
 %This fee acts as a buff   er against any potential revenue loss that the platform might suffer as a consequence of providing visibility to the vendor's product. Thus, the aim is to determine a fair and proportionate fee that correlates with the revenue loss associated with each vendor's product exposure.

\subsection{Main Contributions and Technical Ideas}
    
In this paper, we introduce and study the assortment  optimization problem with visibility constraints under the Multinomial Logit choice model. Our first goal is to settle the complexity question on the positive side by developing a polynomial time algorithm for \ref{APV}. Next, we consider a natural extension of this problem, where a cardinality constraint on the offered assortment is enforced. We show that this additional constraints makes the problem fundamentally harder, rendering it strongly NP-hard, even in the special case where all the products have identical prices. Additionally, we leverage the NP-Hardness to prove that the existence of a Fully Polynomial Time Approximation Scheme (FPTAS) is precluded. Following this assessment on the hardness of the problem, we design a Polynomial Time Approximation Scheme (PTAS) for the special case of equal prices.
A subsequent goal is to quantify the revenue loss caused by the visibility constraints compared to the unconstrained assortment optimization problem and design a  strategy to  share this loss among the different vendors of the products for which visibility constraints have been enforced.  Our contributions are organized and summarized as follows.


\begin{enumerate}
    \item {\bf Polynomial time algorithm for \ref{APV}}.   
    Our main technical contribution is to design a polynomial time algorithm for \ref{APV}.  We introduce the notion of expanded revenue and expanded set of an assortment, which will play a pivotal role in designing our algorithm. We leverage structural properties of the expanded revenue function to characterize the structure of an optimal solution of \ref{APV}, and consequently design an efficient algorithm to compute it.
    
    %We start by considering the problem of static assortment optimization over $T$ customers when we enforce visibility constraints on the products available. Contrary to the unconstrained case, the introduction of the visibility constraints introduces a coupling between the different customers, because we cannot respect the global visibility constraints if we optimize each assortment independently of the others. Hence, it does not appear obvious whether this problem can be solved in polynomial time or is NP-hard. We prove that \ref{APV} can actually be solved by a polynomial time algorithm, and devise such an algorithm. Our contribution follows this structure:
    \begin{enumerate}
        \item {\bf Expanded Revenue}. In Section \ref{expanded revenue}, we introduce the expanded revenue function. Given a universe of products $\cal N$ and an assortment $A \subseteq {\cal N}$. The expanded revenue of assortment $A$ is defined as the maximum revenue of any assortment in $\cal N$ that contains $A$. The expanded set is the assortment that achieves this maximum revenue. We show that the expanded revenue function is closely related to the objective function of \ref{APV} in the case of a single customer. We provide a linear time algorithm to compute the expanded revenue.         
        %: given a subset $A$ of the $n$ products available, compute the maximal expected revenue of a subset containing $A$. We prove in Lemma \ref{Compute expanded set} that there exists a subset of maximal cardinality that is optimal for this problem, and name it the expanded set of $A$, while the expanded revenue of $A$ is the corresponding expected revenue. In addition, we prove that the expanded set can be computed in linear time with respect to the number $n$ of products.

        
        \item {\bf Monotonicity and supermodularity\footnote{A function $f: {\Omega} \rightarrow \mathbb{R}$ is supermodular if $\forall A, B \in {\Omega}, \; f(A \cup B) + f(A \cap B) \geq f(A) + f(B)$.} of the expanded revenue}. We show that the expanded revenue, as defined above, possesses some useful properties. Namely, we prove in Lemma \ref{monotonicity} a monotonicity property, i.e., we show that the expanded revenue of an assortment decreases as the assortment gets large. Then, in Lemma \ref{supermodularity}, we prove the main theoretical result on which our final algorithm relies: the supermodularity of the expanded revenue function.
        \item {\bf Our algorithm and LP formulation}. Building on the previous properties of the expanded revenue,  we finally identify in Theorem \ref{Solution structure} a very simple nested structure for an optimal solution of \ref{APV}, and devise a polynomial time algorithm to efficiently solve the problem. Additionally, we demonstrate in Theorem \ref{thm:LPformulation} that \ref{APV} can be formulated as a compact  linear program.
    \end{enumerate}


\item {\bf \ref{APV} with cardinality constraints.} We consider in Section \ref{sec:APVC} the natural extension of \ref{APV} where there is an upper bound on the number of products that we can display in each assortment.
\begin{enumerate}
    \item {\bf Hardness.} In Theorem \ref{NP-hardness}, we prove that \ref{APV}  with cardinality constraints is strongly NP-hard, even in the case of equal prices, by linking its resolution to the $3$-\texttt{PARTITION} decision problem. Moreover, we extend our proof to show that the existence of an FPTAS is precluded for this problem, unless $P=NP$.
    \item {\bf Polynomial-time approximation scheme (PTAS).} Our cornerstone algorithmic result in the case of equal prices is the design of a polynomial-time approximation scheme (PTAS), which for any fixed desired precision, returns a sequence of assortments within that degree of precision, in polynomial time. Since the strong NP-hardness rules out the existence of a Fully Polynomial Time Approximation Scheme (FPTAS), unless $P=NP$, a PTAS is the best approximation algorithm we can achieve. Our PTAS relies on a linearization of the objective function through the guessing of a carefully chosen set of parameters of the problem. Then, we solve the relaxed linear program, and we leverage its solution to compute a random approximate (integer) solution for the linearized problem, using a dependent rounding scheme. This rounded solution is subsequently shown to be, in expectation, near optimal for \ref{APV} with cardinality constraints.
\end{enumerate}



\item {\bf Price of visibility.} The introduction of visibility constraints results in a reduction in the total expected revenue compared to the unconstrained version of the assortment problem. Therefore, we aim to evaluate this revenue loss and propose a fair strategy for distributing it among vendors based on their respective contributions to the loss.
\begin{enumerate}
\item {\bf Pricing the loss.} We devise in Section \ref{subsection:share},  a pricing strategy as follows: For each product that negatively impacts the overall revenue, we charge the vendor a fraction of the loss proportional to the ratio between the negative contribution of the product and the sum of the negative contributions of all products. We demonstrate that this strategy satisfies natural fairness properties and exhibits favorable computational tractability.
\item {\bf Numerical experiments.} Additionally, we conduct in Section \ref{subsection:numeric} some numerical experiments to illustrate our findings. We analyze the influence of visibility constraints on expected revenue and sales, examine the individual effect of a single product's visibility constraint, and explore the trade-off between revenue and fees. Moreover, we show how our pricing strategy accurately captures the revenue loss attributed to the products made visible.
\end{enumerate}

\end{enumerate}
        



    
    
\subsection{Related Literature}

Assortment optimization under the Multinomial Logit (MNL) model is a well-established problem in scientific literature. Initially introduced by \cite{luce1959individual}, with subsequent works by \cite{McFadden1972ConditionalLA} and \cite{RePEc:ecm:emetrp:v:52:y:1984:i:5:p:1219-40}, the MNL choice model has gained popularity for modeling customer choices due to its simplicity in computing the choice probabilities, its predictive power and its computational tractability compared to more complex choice models. It has been extensively used in various research works  such as \cite{mahajan2001stocking,talluri2004revenue,el2021joint,Sumida2020RevenueUtilityTI,gao2021assortment,housni2023maximum}, to mention a few. 
The MNL model proves particularly useful in assortment optimization, as demonstrated by \cite{talluri2004revenue}, who showed that under the MNL model, the optimal assortment in the unconstrained setting is revenue-ordered. This means that it contains all products whose revenues exceed a certain threshold, simplifying the optimization problem by avoiding the consideration of exponentially numerous potential subsets. Moreover, \cite{Gallego2011AGA} give a linear programming formulation for the unconstrained assortment problem under MNL. \cite{Rusmevichientong2010DynamicAO} solved the version of the problem with a cardinality constraint, proving it is still solvable in polynomial time, and \cite{Dsir2014NearOptimalAF} studied more general capacity constraints, showing it is NP-hard to solve in the general case. \cite{Sumida2020RevenueUtilityTI} and \cite{Davis2013AssortmentPU} studied totally unimodular constraint structures for the assortment and showed that the resulting problem can be reformulated as a linear program. 
However, when considering mixtures of MNL models (MMNL), the assortment optimization problem becomes NP-hard even in the unconstrained setting with two classes of customers as shown in \cite{rusmevichientong2014assortment}. 




%The MNL model proves particularly useful in assortment optimization, as demonstrated by Talluri and van Ryzin (2004), who showed that under the MNL model, the optimal assortment is revenue ordered. This means that it contains all products whose revenue exceeds a certain threshold, simplifying the optimization problem by avoiding the consideration of exponentially numerous potential subsets.



    
%Assortment optimization under the Multinonmial Logit model is a well anchored problem in the scientific literature. First introduced in \cite{luce1959individual}, the Multinomial Logit choice model has become very popular to model customer choices because of its tractability, contrary to more complex models. For instance, mixtures of MNL models (MMNL) become NP-hard for the assortment optimization problem as soon as there are two classes of customers . The MNL model is particularly useful in assortment optimization, where \cite{talluri2004revenue} showed that under the MNL model, the optimal assortment is revenue ordered, which means it contains all the products whose revenue is greater than some threshold. This makes the optimization problem easy to solve in polynomial time, without having to consider the exponential number of potential subsets to offer. Assortment optimization is a significant field of research, and several constrained variations of the MNL optimal assortment have been studied. \cite{Rusmevichientong2010DynamicAO} solved the version of the problem with cardinality constraints, proving it is still solvable in polynomial time, and \cite{Dsir2014NearOptimalAF} studied more general capacity constraints, showing it is NP-hard to solve in the general case. \cite{Sumida2020RevenueUtilityTI} and \cite{Davis2013AssortmentPU} studied totally unimodular constraint structures for the assortment and showed that the resulting problem can be reformulated as a linear program. 





To the best of our knowledge, this paper is the first to study assortment optimization under MNL with visibility constraints.  The topic of visibility in assortment planning has barely been covered: \cite{chen2022fair} studied visibility under a fairness approach, trying to enforce similar visibility for products with similar characteristics, while \cite{Wang2021WhenAM} studied a version of the assortment optimization problem in which they can increase the attractiveness of some products through an advertising budget. %Nevertheless, to the best of our knowledge, visibility constraints on a minimum number of appearances have never been studied in the literature so far. 
Recently, in a concurrent work, \cite{lu2023simple} considers an assortment optimization problem under the MNL model, subject to some fairness constraints similar to the visibility constraints presented in our paper. In contrast to our model, which is deterministic and involves a predetermined number of customers, their study investigates a probabilistic version with a single customer, where random assortments can be offered. This probabilistic version is less complex than our framework. For instance, we establish that our model with cardinality constraints is strongly NP-hard and does not admit an FPTAS, whereas their probabilistic model does admit an FPTAS under cardinality constraints. Hence, the need for different optimization techniques and approaches.




In addition, the topic of assortment optimization for a stream of customers is more often studied from an online perspective, where decisions are made sequentially such as in \cite{avis2015AssortmentOO} and \cite{Cheung2017ThompsonSF}. In contrast, we study a static version of the problem, where we plan the entirety of our assortments in advance. Other versions of the problem, such as in  \cite{Li2009ASA}, consider a flow of customers with randomized preferences, to which we offer a common assortment. Finally, in revenue management, pricing problems are often considered in the sense of optimizing the selling price of each product \citep{Wang2012CapacitatedAA,Miao2018DynamicJA, Alptekinolu2015TheEC}, while selling prices are fixed for our problem,  and we instead study the question of pricing the loss generated by enforcing visibility of each product.

    
   \vspace{3mm}
    \underline{{\em Outline.}} The remainder of the paper is organized as follows. We first introduce the mathematical framework and define our problem \ref{APV} in Section \ref{sec:form}. Then, in Section \ref{sec:apv}, we devise a polynomial time algorithm for the \ref{APV} problem. In Section \ref{sec:APVC}, we consider \ref{APV} with cardinality constraints. We show that adding these constraints makes the problem strongly NP-hard even in a setting with identical prices, then we provide a PTAS for the latter setting. In Section~\ref{sec:price}, we propose a pricing strategy to charge the revenue loss to the vendors proportionally to their contribution, and illustrate our results numerically. Finally, in Section \ref{sec:conclusions}, we draw some conclusions and outline future work. 
        