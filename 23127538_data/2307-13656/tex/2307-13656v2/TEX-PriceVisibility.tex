
    \section{Price of Visibility} \label{sec:price}


In this section, we investigate the impact of visibility constraints on the total expected revenue, comparing it to the unconstrained setting where there are no visibility constraints. In Section \ref{subsection:loss}, we quantify the loss resulting from enforcing the visibility constraints. In Section \ref{subsection:share}, we introduce a novel method to distribute the loss among different products in proportion to their contribution to the overall loss. 
 Finally, in Section \ref{subsection:numeric}, we illustrate   our method through a series of numerical experiments.



\subsection{The loss due to Visibility constraints}
\label{subsection:loss}




Consider the unconstrained assortment optimization \ref{Unconstrained problem} and let $S^*$ be its optimal solution. It is known that there exists an optimal assortment that is price-ordered. This is a standard result in assortment optimization under the MNL model (see \citeauthor{talluri2004revenue} \cite{talluri2004revenue}).\footnote{We refer the reader to Appendix \ref{apx1} for further discussion of assortment optimization under MNL.}
In the absence of visibility constraints and with $T$ customers, it is optimal to offer assortment $S^*$ to each customer. Consequently, the total expected revenue in the unconstrained setting can be expressed as $T\cdot  R(S^*)$. As the unconstrained problem serves as a relaxation of \ref{APV}, it possesses a higher objective function.
In the following example, we show that enforcing the visibility constraints can imply a gap that is arbitrary bad as compared to the unconstrained setting. 




%After enforcing visibility constraints in our assortment problem, we observe a decrease in the total revenue. 

%Indeed, if we remove all the constraints, that is $\ell_i = 0 \; \forall i \in \mathcal{N}$, we come back to \ref{Unconstrained problem}, as defined in \ref{apx1}. The best solution we could hope for is $S_t = S^* \quad \forall 1 \leq t \leq T$, because it would maximize each term $R(S_t)$ independently of the others, therefore maximize the total revenue. However, because of the visibility constraints, we had to integrate in the assortments $(S_t)_{1 \leq t \leq T}$ some elements that were not in $S^*$, driving the revenue down. 




% \noindent{\bf Example.} Consider two products such that for all $i \in \{1,\ldots, n-1\}$
% $ p_i = 1, v_i = 1$ and $ \ell_i = 0$. For product $n$, let  $p_n = 0, v_n = n^2$,  and $\ell_n = T$.  We consider the setting where $n$ is large.
% To compute the optimal assortment for \ref{Unconstrained problem}. As mentioned earlier, it sufficient to evaluate the revenue of the price-ordered assortments and choose the one with the highest revenue. We have 
% $R(\{1,\ldots,k\})=  \frac{k}{1 + k} \quad \forall k \in \{1,\ldots, n-1\}$ and $R(\mathcal{N}) = \frac{n-1}{1 + n-1 + n^2} = \frac{n-1}{n + n^2} \leq \frac{n-1}{n} $. Therefore, the optimal assortment is $S^*=\{1, \ldots,n-1 \}$ and $R^*=\frac{n-1}{n}.

% On the other hand,

%  However, $\forall A \subseteq [\![1, n-1]\!], \; R(\{n\} \cup A) = \frac{1 + |A|}{1 + |A| + n^2}$, which increases with $|A|$, so the optimal solution for \ref{APV} is $S_t^* = \overline{\{n\}} = \mathcal{N} \; \forall t$. Therefore the revenue goes to zero because of the visibility constraints on product $n$, while it was close to 1 for a large value of $n$ without constraints.



\vspace{2mm}
\noindent{\bf Example.} Consider an instance of \ref{APV} with two products $n=2$ and $T$ customers. Let $ p_1 = 1, v_1 = 1, \ell_1 =0$ and  $p_2 = 0, v_2 = M, \ell_2 = T$.  We consider the setting where $M$ is large.
To compute the optimal assortment for \ref{Unconstrained problem}. As mentioned earlier, it sufficient to evaluate the revenue of the price-ordered assortments and choose the one with the highest revenue. We have 
$R(\{1\})=  \frac{1}{2} $  and $R(\{1,2\})=  \frac{1}{2+M} < R(\{1\})$ . Therefore, the optimal assortment is $S^*=\{1 \}$ and $R(S^*)=\frac{1}{2}$.

On the other hand, let $(S_1, S_2, \ldots, S_T)$ be a feasible solution for \ref{APV}. Because, $\ell_2=T$, we have to include product $2$ in every assortment $S_t$. Adding product $1$ to an assortment only increases the revenue because $R(\{1,2\})=  \frac{1}{2+M} > R(\{2\})=0$.
Therefore, it is optimal to offer product $1$ and $2$ in every assortment $S_t$ in an optimal solution of \ref{APV}. Therefore, $S_t^*=\{1,2\}$ for all $t =1,\ldots,T$. 
Now, consider the ratio $$\frac{T R(S^*)}{\sum_{t=1}^T R(S_t^*)} = \frac{T  \frac{1}{2}   }{T \frac{1}{2+M}}= \frac{M}{2}+1,$$
which goes to infinity as $M$ increases. Hence the gap can be arbitrarily large.
While in this pathological example, we see that enforcing products with very low price and high weight can drive the revenue very low, our numerical experiments in Section \ref{subsection:numeric} will illustrate how more common distributions of the prices and weight react to the enforcement of visibility constraints.

 % However, $\forall A \subseteq [\![1, n-1]\!], \; R(\{n\} \cup A) = \frac{1 + |A|}{1 + |A| + n^2}$, which increases with $|A|$, so the optimal solution for \ref{APV} is $S_t^* = \overline{\{n\}} = \mathcal{N} \; \forall t$. Therefore the revenue goes to zero because of the visibility constraints on product $n$, while it was close to 1 for a large value of $n$ without constraints.


\subsection{Sharing the loss}
\label{subsection:share}

% Consider a platform on which $n$ vendors sell a product each. Due to SLAs, the vendors impose some visibility constraints for their product on the platform, so that the situation can be modeled by the problem \ref{APV}. 


In this section, we explore a scenario where each product within our universe is associated with a specific vendor. As mentioned earlier, vendors can impose visibility constraints on their products within the platform. These constraints can be established through service level agreements or product sponsorships. However, it is important to note that enforcing these constraints may result in a decrease in the platform's revenue. To address this issue, the platform can implement a fee structure based on the vendors' contributions to the revenue loss. %This approach aims to establish a fair pricing policy that aligns with each vendor's impact on the revenue loss.
%We would like to a fair pricing policy where we first calculate the revenue loss incurred due to the constraints imposed by vendors. Then, we  charge each vendor a fraction of this loss, proportionate to their individual contribution towards the overall revenue reduction.
% In this section, we consider the situation when each product in our universe belongs to a vendor. As explained earlier vendors can impose some visibility constraint for their product on the platform according to a service level agreement or by sponsoring the product. 
% As seen previously, enforcing these constraints could cause a loss in the revenue of the platform. In return, the platform can charge the different vendors a fee depending on how much they contributed in reducing the revenue. In this perspective, a fair pricing policy would consist in computing the revenue loss due to the constraints, and then to charge to the vendors a fraction of this loss, which would be kind of proportional to their contribution in the loss.
Let $S^*$ be an optimal solution for \ref{Unconstrained problem} and let $(S_1^*, S_2^*,\ldots, S_T^*)$ be an optimal solution for \ref{APV}. We denote the revenue loss due to the visibility constraints as 
\begin{equation}
    \label{Delta loss}
    \Delta \coloneqq T \cdot R(S^*) - \sum_{t=1}^T R(S_t^*).
\end{equation}

\noindent
{\bf A first naive approach.} One approach is to allocate the loss based solely  on the parameters $\ell_i$. In this case, the proportion of the loss assigned to the vendor of product $i$ would be determined by $\frac{\ell_i}{\sum_{j=1}^n \ell_j}$. However, this distribution would not be equitable in the sense that  we should not impose any charges on a product that already belongs to the optimal set $S^*$, even if it satisfies the constraint $\ell_i > 0$. Moreover, this allocation fails to consider the impact of each product on the overall loss. For example, if there is a product with exceptionally high preference weight $v_i$ but a significantly lower price $p_i$, while other products have higher prices and lower preference weights, enforcing the visibility of the first product would drive the revenue down, whereas the others would have a lesser impact. In this scenario, the former product should be responsible for covering almost the entire revenue loss.



\noindent
{\bf Our approach.}
Let $S_1^*, \ldots, S_T^*$ be an optimal solution to \ref{APV}. First observe that  $R(S_t^*) = \frac{\sum_{i \in S_t^*} p_i v_i}{1 + \sum_{i \in S_t^*} v_i}$, implies that $$R(S_t^*) = \sum_{i \in S_t^*}  (p_i - R(S_t^*))v_i$$ for every set $S_t^*$. This decomposition of the revenue gives us which products drive the revenue down (the ones with $p_i < R(S_t^*)$) and which products increase the revenue ($p_i > R(S_t^*))$. It also shows that the contribution of  product $i$ to the revenue is proportional to the difference between the price of product $i$ and the actual revenue, as well as proportional to the preference weight $v_i$. We can then rewrite the total revenue of the assortments $(S_1^*, S_2^*, \ldots, S_T^*)$ as 
$$\sum_{t=1}^T R(S_t^*) = \sum_{t=1}^T \sum_{i \in S_t^*}  (p_i - R(S_t^*)) v_i = \sum_{i=1}^n \sum_{t=1}^T \mathbbm{1}(i \in S_t^*)  (p_i - R(S_t^*)) v_i.$$
Therefore, we view the contribution of product $i \in \mathcal{N}$ to the total revenue as $$\cont{i}\coloneqq\sum_{t=1}^T \mathbbm{1}(i \in S_t^*) (p_i - R(S_t^*)) v_i.$$

%\begin{prop}
%    \label{pricing formula}

\noindent
{\bf Pricing the loss.}
 For each product $i \in \mathcal{N}$, we propose to charge its vendor the fraction of the loss corresponding to the negative contribution of this product, divided by the sum of the negative contributions of all the products:

\begin{equation} \label{pricing formula}
  \Gamma_i \coloneqq \frac{\cont{i}^-}{\sum_{j=1}^n \cont{j}^-} \cdot \Delta
\end{equation}
where $x^- = max(-x, 0)$ is the negative part of x, and recalling that $\Delta$ is the total loss in revenue due to enforcing visibility constraints.
%\end{prop}

Next, we discuss three important properties of this strategy:
\\

\noindent
{\bf (a) Fair distribution of fees.} First, it is worth noting that the loss in revenue is exactly shared between the products whose contribution to the total revenue is negative as on one hand we have $\sum_{i \in \mathcal{N}} \Gamma_i = \Delta$, and on the other hand we have $\Gamma_i > 0$ if and only if $C_i^->0$, i.e., the product's contribution to the revenue is negative. Moreover, the fee $\Gamma_i$ for each product takes into account its actual impact on the revenue:
the first observation is that the products in $S^*$ all have a nonnegative contribution, and their vendors are consequently exempt from a fee as expected. The second observation is that for any product with a negative contribution, its impact on the revenue is magnified by a lower price and a greater preference weight. It turns out that our policy does take this observation into account, as the lower/greater the price/weight of such a product, the greater the loss it incurs to the total revenue, and thus the greater the fee imposed on the vendor. The third observation is that even if a products $i\notin S^*$ and/or if there exists some $t\in [T]$ such that $i \in S_t^*$ and $p_i\leq R(S_t^*)$, such a product can still have a positive contribution, in which case $\Gamma_i =0$, concurring with a product whose impact overall is positive. The final observation is that this pricing strategy guarantees a fair treatment of equal products. Indeed, if two distinct products $i$ and $j$ have identical prices and preference weights, then imposing similar visibility constraints for the two products implies a similar fee for the vendors.

\noindent
{\bf (b) Monotonicity of the fee.}                            
For every product $i$, the fee $\Gamma_i$ is nondecreasing as a function of $\ell_i$. In other words, if a vendor wishes to display her product more often to customers, her fee gets higher. Let us show this intuitive result mathematically. Recall that an optimal solution to \ref{APV} is given by $S_t^* = \overline{L_t\cup \cdots \cup L_T}$ for $t\in [T]$, as shown in Theorem \ref{Solution structure}. Similarly, let $\Tilde S_t$ be the optimal solution given by Equations \eqref{eq:sol} when $\ell_i$ is increased by $1$. Note that a direct consequence of the formula giving these optimal solution, we have $\Tilde S_t = S_t^*$ for all $t\neq \ell_i+1$, and $\Tilde S_{\ell_i+1} = \overline{L_t\cup\cdots\cup L_{\ell_i+1}\cup\{i\}}$. First, if $i\in S_{\ell_i+1}^*$, then nothing changes and we have $\Tilde {\cont{i}} = \cont{i}$, where $\Tilde{\cont{i}} \coloneqq \sum_{t=1}^T \mathbbm{1}(i \in \Tilde S_t) (p_i - R(\Tilde S_t)) v_i$ is product $i$'s contribution to the loss after increasing $\ell_i$. Otherwise, $\cont{i}\geq\Tilde{\cont{i}}$, and therefore $\cont{i}^-\leq\Tilde{\cont{i}}^-$. Furthermore, for every $j\neq i$, since $R(S^*_{\ell_i+1}) \geq R(\Tilde S_{\ell_i+1})$, we have $\Tilde{\cont{j}} \geq {\cont{j}}$, since the term associated with $t=\ell_i+1$ in the definition of $\cont{j}$ can only increase when $\ell_i$ is increased, while all the other terms remain unchanged. Letting $\Tilde \Delta$ and $\Tilde \Gamma_i$ be the total loss incurred and the fee imposed on vendor $i$ respectively, after $\ell_i$ is increased by $1$, we have\begin{align*}
    \Tilde \Gamma_{i} &= \frac{\Tilde C_i}{\Tilde{\cont{i}} + \sum_{j\neq i}\Tilde{\cont{j}}}\cdot \Tilde \Delta\\
    & \geq \frac{\Tilde C_i}{\Tilde{\cont{i}} + \sum_{j\neq i}{\cont{j}}}\cdot  \Delta\\
    &\geq \frac{C_i}{{\cont{i}} + \sum_{j\neq i}{\cont{j}}}\cdot  \Delta\\
    & = \Gamma_i.
\end{align*}
The third line follows from the fact that the map $x\mapsto x/{c+x}$ is nondecreasing on $[0,+\infty)$, for any $c>0$.

\noindent
{\bf (c) Computational tractability.} The fees charged are easy to compute: each of them can be computed in polynomial time ${O}(nT)$ since they only require to solve \ref{APV}. Additionally, consider the situation where a vendor is interested in knowing the fee that she needs to pay in order to increase the visibility of the product by one customer. This corresponds to the situation where we change $\ell_i$ to $\ell_i + 1$ (without changing any other ${\ell}_j$).
In this case, we only have to recompute $S_{\ell_i + 1}^*$, because the others assortments $S_t^*$ stay the same, as explained in the previous paragraph. Therefore, for a fixed product $i$, can compute efficiently the value of the fee $\Gamma_i$ for all the values of $\ell_i \in [T]\cup\{0\}$.
 %we can compute all the $\Gamma_i(\ell)$ for $\ell \in [\![0, T]\!]$ in time $\mathcal{O}(nT) + T \mathcal{O}(n) = \mathcal{O}(nT)$, instead of $\mathcal{O}(n T^2)$ if we recomputed all the $S_t^*$ every time we increase $\ell_i$. Thus, the computation all the $(\Gamma_i(\ell_i))_{0 \leq \ell_i \leq T}$ has the same complexity as the computation of a single $\Gamma_i(\ell_i)$ for one value of $\ell_i$. 
 
 %This is useful for a vendor who is willing to see how the fee they will have to pay evolves with the visibility constraint they ask for in the SLA for instance.

   

\subsection{Numerical Study}\label{subsection:numeric}

To illustrate our theoretical contributions, we perform some numerical experiments on randomly generated data.

\noindent
{\bf Our experimental setup.}
We fix the number of products equal to $n = 20$, and fix the number of customers to $T = 30$. We generate the weights $v_i$, prices $p_i$ and visibility constraints $\ell_i$ from several distributions. Namely, to generate the weights $v_i$ and prices $p_i$ we used uniform and exponential distributions with varying parameters. For the visibility constraints $\ell_i$, we ran our experiments using three distributions: we used a standard integer uniform $\mathcal{U}(\{0\}\cup[T])$, then an integer uniform $\ell_i \sim \mathcal{U}(\{\lfloor\frac{i}{n} T\rfloor,\ldots, T\})$, and finally the constant $\ell_i = T$, for all  $i \in \mathcal{N}$. Recalling  that products are ordered by non-increasing price values, the intuition behind the second distribution is that products with lower prices are likelier to receive higher visibility constraints.

For each set of parameters, we generate $1000$ samples, then we compute ratio of the optimal solution of \ref{APV} divided by $T \cdot R(S^*)$, i.e., the unconstrained optimal expected revenue (without any visibility constraints), namely 
$$\eta \coloneqq \frac{\sum_{t=1}^T R(S_t^*)}{T \times R(S^*)}.$$
This captures the fraction of the revenue that we conserve once visibility constraints are enforced.
We compute statistics on ratio $\eta$:  mean, standard deviation, minimum and maximum values.
\footnote{Initially, we made $T$ vary, but we observed that the results where almost independent of $T$ compared to the randomness of the generation of the parameters, so we kept only $T=30$. Indeed, varying $T$ just extends the stream of customers, and if the visibility constraints are extended by the same factor, it seems natural that the ratio of conserved revenue remains constant.}

\noindent
{\bf Overall results and analysis.}
Our results are gathered in Table \ref{fig:tables}. 
\begin{table}[h]
    \caption{For each distribution of $p_i, v_i$ and $\ell_i$, we compute the ratio $\eta$ between the optimal assortment of \ref{APV} and the optimal assortment of the unconstrained problem \ref{Unconstrained problem} on $1000$ samples. We provide its mean, standard deviation, minimum and maximum values.}
    \centering
\begin{tabular}{||c|c|c|c|c|c|}
\hline
\rule{0pt}{12.5pt}{}  & {}    &  \multicolumn{4}{c|}{\begin{tabular}{l} $\eta$ (in \%) \end{tabular}} \\
\hline
{Distribution of $v_i$ and $p_i$} &{Distribution of $\ell_i$} & {Mean} & {Standard deviation} & {Min} & {Max} \\
\hline
    & $\mathcal{U}(\{0\}\cup[T])$ & 82.9 & 5.7 & 58.8 & 96.8  \\
$v_i \sim \mathcal{U}(0,1), \quad p_i \sim \mathcal{U}(0,1)$  
    & $\mathcal{U}(\{\lfloor\frac{i}{n} T\rfloor,\ldots, T\})$ & 73.7 & 6.4 & 51.9 & 91.3  \\
    & $\ell_i = T$ & 71.6 & 6.9 & 47.4 & 91.6  \\

\hline
    & $\mathcal{U}(\{0\}\cup[T])$ & 71.0 & 7.3 & 43.9 & 89.2  \\
$v_i \sim \mathcal{U}(0,10), \quad p_i \sim \mathcal{U}(0,10)$  
    & $\mathcal{U}(\{\lfloor\frac{i}{n} T\rfloor,\ldots, T\})$ & 60.3 & 7.1 & 39.5 & 81.6 \\
    & $\ell_i = T$ & 58.1 & 7.6 & 33.0 & 81.0  \\

\hline
    & $\mathcal{U}(\{0\}\cup[T])$ & 64.3 & 9.4 & 34.5 & 88.1  \\
$v_i \sim \mathcal{E}(1), \quad p_i \sim \mathcal{E}(1)$  
    & $\mathcal{U}(\{\lfloor\frac{i}{n} T\rfloor,\ldots, T\})$ & 50.9 & 9.4 & 24.8 & 87.6  \\
    & $\ell_i = T$ & 48.2 & 10.1 & 20.0 & 89.1  \\

\hline
    & $\mathcal{U}(\{0\}\cup[T])$ & 49.4 & 11.2 & 16.1 & 79.1  \\
$v_i \sim \mathcal{E}(1/10), \quad p_i \sim \mathcal{E}(1/10)$  
    & $\mathcal{U}(\{\lfloor\frac{i}{n} T\rfloor,\ldots, T\})$ & 37.2 & 9.8 & 13.4 & 63.7  \\
    & $\ell_i = T$ & 34.7 & 9.4 & 11.3 & 67.1  \\

\hline
\end{tabular}

    \label{fig:tables}
\end{table}

As expected, the ratio $\eta$ decreases when we give higher visibility $\ell_i$ to the products with smaller prices $p_i$. Furthermore, we observe that the higher the range of fluctuation of the weights $v_i$ and prices $p_i$, the lower the revenue becomes once we enforce visibility constraints. This is rather intuitive, because we saw previously that the cases in which the visibility constraints can severely harm the revenue are when there are some products with extreme value of the price or weight compared to the others. However, we observe that extreme events when the revenue of \ref{APV} becomes very low are really rare since the standard deviation is rather small compared to the range of fluctuation of $\eta$. Based on our results, it seems that in most cases, as long as there are no abnormally extreme values of the weights and prices, we can always hope to keep a fraction close to half of the revenue after enforcing visibility constraints.\\


\noindent
{\bf Revenue vs market shares.} To assess the influence of the visibility constraints on the market share, we compute the optimal value of \ref{APV} for all visibility constraints $(\ell_i)_{1 \leq i \leq n}$ equal to a same value $\ell$ going from $0$ to $T$ (for $v_i \sim \mathcal{U}(0,1), p_i \sim \mathcal{U}(0,1)$). We then compute the expected sales associated to this expected revenue, and plot in Figure \ref{fig:graph1} the ratio of revenue decrease versus the ratio of sales increase by dividing them by their unconstrained value, for normalization.
% Figure environment removed

It is interesting to see that, while the revenue decreases as we enforce more visibility constraints on products, on the contrary, the sales increase. Indeed, based on the structure of the optimal nested solution we identified, $S_t^* = \overline{L_t \cup \ldots \cup L_T}$, we see that the sizes of the $S_t^*$ increase when the $\ell_i$'s increase. Furthermore, if we take a look at the sales optimization problem $$\max_{S_1, \ldots, S_T} \sum_{t=1}^T \frac{V(S_t)}{1 + V(S_t)},$$ that corresponds to $p_i = 1$ for all $i\in \Nc$, we can see that the more products we add to each $S_t$, the higher the sales since the function $x \mapsto \frac{x}{1+x}$ is increasing with respect to $x$. \\
Therefore, we can see that while visibility constraints decrease the revenue, they increase the sales, which can be interesting if the objective of the platform is to capture market shares.\\

\noindent
{\bf Marginal study of one product.}
Finally, it is interesting to illustrate the marginal effect of the visibility constraints on one product. For this purpose, we take an instance of our generated data with $v_i \sim \mathcal{E}(1), \; p_i \sim \mathcal{E}(1), \; \ell_i \sim \mathcal{U}([\![0, T]\!])$. Then, we select a particular $i \in \mathcal{N}$, and vary $\ell_i$ from $0$ to $T$ while the others $\ell_j$ stay fixed at their generated value $\ell_j$. In Figure \ref{fig:graph2}, we plot the variation of the revenue loss with respect to $\ell_i$

% Figure environment removed


We remark that the loss is convex with respect to $\ell_i$. Indeed, let   $(S_t^*)_{1 \leq t \leq T} = (\overline{L_T \cup \ldots \cup L_t})_{1 \leq t \leq T}$ be the nested solution for constraint $\ell_i$, and $\ell_i \mapsto \Delta(\ell_i)$ the revenue loss function, as defined in \eqref{Delta loss}, and considered as a function of constraint $\ell_i$ when all the other visibility constraints $(\ell_j)_{j \neq i}$ are fixed. We have 
% \begin{align*}
%     \Delta(\ell_i +1) - \Delta(\ell_i) - (\Delta(\ell_i) - \Delta(\ell_i -1)) &= \overline{R}(S_t^* \cup \{i\}) + \overline{R}(S_{t+1}^* \cup \{i\}) - 2 (\overline{R}(S_t^* \cup \{i\}) + \overline{R}(S_{t+1}^*)) + \overline{R}(S_t^*) + \overline{R}(S_{t+1}^*) \\
%     &= \overline{R}(S_{t+1}^* \cup \{i\}) + \overline{R}(S_t^*) - \overline{R}(S_t^* \cup \{i\}) - \overline{R}(S_{t+1}^*) \geq 0
% \end{align*}
$$\Delta(\ell_i +1) - \Delta(\ell_i) - (\Delta(\ell_i) - \Delta(\ell_i -1))  = \overline{R}(S_{t+1}^* \cup \{i\}) + \overline{R}(S_t^*) - \overline{R}(S_t^* \cup \{i\}) - \overline{R}(S_{t+1}^*) \geq 0,$$  where the inequality holds by supermodularity and $S_t^* \subseteq S_{t+1}^*$, which shows $\ell_i \mapsto \Delta(\ell_i)$ is convex.  


We can then, in the case where all the revenue from the sales of a product comes to its vendor, compute the expected revenue of product $i$ versus the fee its vendor will pay as the visibility constraint $\ell_i$ increases. The expected revenue of product $i$ is given by: $\sum_{t=1}^T \mathbbm{1}(i \in S_t^*) p_i \frac{v_i}{1 + V(S_t^*)}$, while the fee charged $\Gamma_i$ is defined in  \eqref{pricing formula}. This is depicted in Figure \ref{fig:graph3}.

% Figure environment removed

For this particular product, we can see that enforcing a visibility constraint $\ell_i$ is profitable until $\ell_i = 17$ (with respect to $T=30$), and becomes non profitable after. The profit associated with this constraint is the difference between the revenue and the fee, and is maximized for the value $\ell_i = 7$ (see Figure \ref{fig:graph3}). This is an example of a product that does not belong to the optimal unconstrained set $S^*$ (since the revenue is $0$ when $\ell_i = 0$, it means the product was not included), but for which adding a visibility constraint allows to boost sales and the revenue of the vendor, while paying a small fee to compensate the platform's revenue loss.

One interesting observation here is that the fee paid by the vendor of product $i$ (0.8 when $\ell_i = T = 30$ on Graph \ref{fig:graph3}) is higher than the loss caused by the introduction of $\ell_i = T = 30$ while the others products were enforced previously (loss of 0.5 as we can read on Graph \ref{fig:graph2}). However, this can be explained: as the expanded revenue function is supermodular and decreasing, the revenue decreases most when the first products are enforced, and then decreases less. Therefore, enforcing visibility constraints on product $i$ decreases less the revenue if we already have some visibility constraints on the other products than if $i$ is the first product on which the constraints are enforced. However, our pricing policy $\Gamma_i$ does not take into account such sequential order of arrival, and prices each product globally, so that two products with the same price and weight pay the same fee for identical visibility constraints.
    