    \section{Conclusions and future directions}
    \label{sec:conclusions}
    
%noindent
%{\bf Conclusion.}
In this paper, we introduced the problem of assortment optimization with visibility constraints \ref{APV}, motivated by situations in e-commerce and online advertising in which a platform aims at ensuring a minimal exposure for some or all products. We prove that this problem can be solved in polynomial time, and devise an efficient algorithm to compute an optimal solution. Our algorithm leverages the supermodularity of an altered version of the expected revenue function, that allows us to identify the nested structure of an optimal solution. 
We also consider an extension of the problem with cardinality constraints on the assortments offered. We prove that the problem becomes strongly NP-hard even under uniform prices, and that in particular, it admits no FPTAS, unless $P=NP$. We then devise a PTAS for the special case of equal prices.
Finally, we evaluate the revenue loss caused by the visibility constraints enforced, and propose a fair pricing strategy to charge each vendor a fee proportional to the contribution of its product to the revenue loss. Finally, 


\vspace{2mm}
\noindent
{\bf Future directions.}
A promising future research direction involves developing an approximation algorithm for the assortment optimization problem with visibility and cardinality constraints, considering general prices.
Furthermore, exploring the assortment problem with visibility using alternative choice models such as the Markov Chain choice model or Nested Logit represents another fruitful avenue for investigation. %Finally, an interesting direction to consider an online version of the problem, in which assortments have to be offered sequentially, as customers arrive.
