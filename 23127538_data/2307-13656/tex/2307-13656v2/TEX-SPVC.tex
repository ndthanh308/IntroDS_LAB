 \section{\ref{APV} with Cardinality Constraints}
    \label{sec:APVC}
A natural extension to \ref{APV} consists on considering cardinality constraints on the offered assortment. Cardinality constraints arise in multiple real world applications such as shelf space in brick-and-mortar stores or screen size in online stores, in which case, offering large assortments of products is infeasible. In this section we consider the constrained version of our problem, which we refer to as {\em Assortment Problem with Visibility and Cardinality constraints (\ref{APVC})}, where each customer can be shown at most $k$ products, where $k$ is a positive integer specified as part of the input. Formally, \ref{APVC} is defined as follows, 
\begin{equation}
\label{APVC}
\tag{\sf{APVC}} 
\begin{aligned}
 \max_{ S_1, \ldots, S_T \subseteq \mathcal{N}} & \; \; \sum_{t=1}^T  \frac{\sum_{i \in S_t}p_i v_i}{1 + \sum_{i \in S_t} v_i}  \\ s.t. \;\; &\sum_{t=1}^T  \mathbbm{1}(i \in S_t) \geq \ell_i, \;\;\; &&\forall i \in \mathcal{N}, \\ & |S_t| \leq k, && \forall t \in [T].
\end{aligned}
\end{equation}






First, we study the complexity of \ref{APVC}. In the next theorem, we show that \ref{APVC} is strongly NP-Hard.
\begin{theorem} \label{NP-hardness} \ref{APVC} is strongly NP-hard, even when all prices $p_i$ are equal. Moreover, there is no FPTAS for \ref{APVC}, even with equal prices.
\end{theorem}
The proof of this theorem relies on a reduction of the \texttt{3-PARTITION} problem, where the objective is to partition a set of cardinality $3T$ into $T$ subsets, each of size $3$, such that the sum of elements within each subset is identical. The full proof of this reduction is deferred to Appendix \ref{apxhard}.

Relying on the strong NP-hardness of this problem, we show that the existence of a Fully Polynomial Time Approximation Scheme (FPTAS) is precluded, unless ${P}={NP}$. Therefore, we focus in the remainder of this section on the question of designing a Polynomial Time Approximation Scheme (PTAS) for the problem, in the case of equal prices. This setting corresponds to a sales maximization problem, which is particularly useful in the case where the platform or the company's goal is to maximize the captured portion of customers.

\begin{theorem} \label{thm:PTAS}
There exists a PTAS for \ref{APVC} with equal prices.
\end{theorem}
The remainder of this section focuses on proving Theorem \ref{thm:PTAS}, by devising a PTAS for \ref{APVC} with equal weights. The proof is organized as follows. In Section \ref{subsec:discret}, we consider an instance of \ref{APVC} with equal prices, which we reduce to a ``friendlier" instance of the problem by slightly altering the weights of a particular subset of products. Importantly, we show that we only incur an $\eps$ loss due to this reduction, and therefore only focus on solving the modified instance. In Section \ref{subsec:linearsol}, we linearize our optimization problem through a carefully crafted guessing procedure, then we provide a method to obtain approximate solutions for the linearized program, thereby completing the formulation of our PTAS. Finally, we show in Section \ref{subsec:analysis} that the solution yielded by our algorithm is indeed a $(1-\eps)$-approximation for our problem.
\subsection{Discretizing the universe of products}\label{subsec:discret}
The setting with identical prices is formally defined as follows
\begin{equation}
\label{eq:SPVC}\tag{\sf{SPVC}}
\begin{aligned}
 \max_{ S_1, \ldots, S_T \subseteq \mathcal{N}} & \; \; \sum_{t=1}^T\frac{V(S_t)}{1+V(S_t)}  \\ s.t. \;\; &\sum_{t=1}^T  \mathbbm{1}(i \in S_t) \geq \ell_i, \;\;\; &&\forall i \in \mathcal{N}, \\ & |S_t| \leq k, && \forall t \in [T].
\end{aligned}
\end{equation}

In this section, we reduce the original instance of our problem into a modified instance, by rounding down the weights of a particular subset of products in our original universe. Then, we show that we incur at most an $\eps$-loss by performing this reduction.
For any product $i\in \Nc$,\begin{itemize}
    \item If $v_i\geq 1/\eps$, then $v_i$ is rounded down to $v_i^{\downarrow} = 1/\eps$. We denote the product associated with the weight $v_i^{\downarrow}$ simply by $i^{\downarrow}$.
    \item If $v_i< \eps^5$, then $v_i$ remains unchanged and we define $v_i^\downarrow = v_i$. Similarly, the associated product is denoted by $i^{\downarrow}$.
    \item If $\eps^5\leq v_i < 1/\eps$, then $v_i$ is rounded down such that $v_i/\eps^5$ meets the nearest power of $(1+\eps)$. In other words, since $\eps^5\leq v_i < 1/\eps$, there exists some $q\geq 1$ such that $\eps^5\cdot(1+\eps)^{q-1} \leq v_i < \eps^5\cdot(1+\eps)^{q}$, and $v_i$ is rounded down to $v_i^\downarrow = \eps^5\cdot(1+\eps)^{q-1}$. We denote the associated product by $i^{\downarrow}$. In particular, note that by this definition, we have \begin{equation}\label{eq:weightbound}
        v_i^\downarrow \leq v_i\leq (1+\eps)\cdot v_i^\downarrow.
    \end{equation}
\end{itemize}
Let us denote this new universe of products by $\Nc^\downarrow = \{i^\downarrow\,\colon\,i\in \Nc\}$. We invite the reader to think of ${.}^\downarrow$, when applied to a product, as a one-to-one map from the universe $\Nc$, to the modified universe $\Nc^\downarrow$, where a subset of weights have been rounded down.
In the subsequent result, we show that solving a modified instance of \ref{eq:SPVC} using the products from $\Nc^\downarrow$ yields a $(1-\eps)$-approximation of the solution on the original instance. To this purpose, we define the following new instance of \ref{eq:SPVC}, which only uses products from the universe $\Nc^\downarrow$:
\begin{equation}
\label{eq:SPVCd}\tag{{\sf SPVC}$^\downarrow$}
\begin{aligned}
 \max_{ S_1, \ldots, S_T \subseteq \mathcal{N}^\downarrow} & \; \; \sum_{t=1}^T\frac{V(S_t)}{1+V(S_t)}  \\ s.t. \;\; &\sum_{t=1}^T  \mathbbm{1}(i^\downarrow \in S_t) \geq \ell_i, \;\;\; &&\forall i^\downarrow \in \mathcal{N}^\downarrow, \\ & |S_t| \leq k, && \forall t \in [T].
\end{aligned}
\end{equation}

Let $\opt$ (resp. $\opt^\downarrow$) denote the objective of an optimal solution of \ref{eq:SPVC} (resp. \ref{eq:SPVCd}).
In the following, we show that $\opt$ and $\opt^\downarrow$ are within a $1-\eps$ fraction of one another. To this purpose, we show a stronger result in Lemma \ref{lem:epsobjective}. Indeed, for any sequence of assortments $S_1,\ldots, S_t$, we define\begin{equation*}
    \obj(S_1,\ldots, S_T) = \sum_{t=1}^T\frac{V(S_t)}{1+V(S_t)}.
\end{equation*}
For any assortment $S\subseteq \Nc$, we denote by $S^\downarrow$ its rounded counterpart, defined as $S^\downarrow = {\{i^\downarrow\,\colon\, i \in S\}}$. In the next lemma, we show that the objective achieved by any sequence of $T$ assortments in $\Nc$ is within $1-\eps$ of their rounded counterpart. In particular, letting $\opt$ denote the value of \ref{eq:SPVC} (i.e., the objective of an optimal feasible sequence of assortments), and similarly $\opt^\downarrow$ denote the value of \ref{eq:SPVCd}, this implies that $\opt$ and $\opt^\downarrow$ are also within $1-\eps$.

\begin{lemma}\label{lem:epsobjective}
    For any sequence of assortments $S_1,\ldots,S_T\subseteq \Nc$, we have $$\obj(S_1^\downarrow,\ldots, S_T^\downarrow)\leq \obj(S_1,\ldots, S_T)\leq (1+\eps)\cdot\obj(S_1^\downarrow,\ldots, S_T^\downarrow).$$ Consequently, we have $\opt^\downarrow\leq \opt\leq (1+\eps)\cdot \opt^\downarrow$.
\end{lemma}

\begin{proof}[Proof of Lemma \ref{lem:epsobjective}]
    First, since all the weights of the products in $\Nc$ are either rounded down or remain the same, then for all $t\in [T]$, $V(S_t^\downarrow)\leq V(S_t)$. Consequently, by the monotonicity of the map $x\mapsto x/(1+x)$, we have $V(S_t^\downarrow)/(1+V(S_t^\downarrow))\leq V(S_t)/(1+V(S_t))$. By summing over $t$, we obtain the first inequality $\obj(S_1^\downarrow,\ldots, S_T^\downarrow)\leq \obj(S_1,\ldots, S_T)$. Let us prove the second inequality. We show that $V(S_t)/(1+V(S_t)) \leq (1+\eps)\cdot V(S_t^\downarrow)/(1+V(S_t^\downarrow))$, for every $t \in [T]$. Let $t \in [T]$.\begin{itemize}
        \item If there exists a product $i\in S_t$ such that $v_i\geq 1/\eps$, then $v_i^\downarrow = 1/\eps$, and in turn, $V(S_t^\downarrow)\geq 1/\eps$. Therefore, \begin{equation*}
            (1+\eps)\frac{V(S_t^\downarrow)}{1+V(S_t^\downarrow)} \geq (1+\eps) \cdot \frac{1/\eps}{1+1/\eps} = 1 \geq \frac{V(S_t)}{1+V(S_t)}.
        \end{equation*}
        \item If for all $i\in S_t$, we have $v_i< 1/\eps$. Note that Equation \eqref{eq:weightbound} holds trivially when $v_i<\eps$, since $v_i^\downarrow = v_i$ by definition. Therefore, Equation \eqref{eq:weightbound} holds for every $i\in S_t$. Hence,\begin{equation*}
            \frac{v_i}{1+v_i} \leq \frac{(1+\eps)\cdot v_i^\downarrow}{1+v_i^\downarrow}. 
        \end{equation*}
    \end{itemize}
    The result follows by summing over $t = 1,\ldots,T$.
    
Noticing that if $S_1,\ldots, S_T$ is feasible for \ref{eq:SPVC}, then its rounded counterpart is also feasible for \ref{eq:SPVCd}, we also have $\opt^\downarrow\leq \opt\leq (1+\eps)\cdot \opt^\downarrow$.

    
\end{proof}
In light of Lemma \ref{lem:epsobjective}, we deduce that obtaining an approximation to the reduced problem \ref{eq:SPVCd} automatically yields an approximation to the original problem \ref{eq:SPVC}, with only an additional $\eps$ loss. In the remainder of this section, we will only focus on solving the reduced problem \ref{eq:SPVCd}, therefore, for simplicity of notation, we denote the products $1^\downarrow,\ldots,n^\downarrow$ simply by $1,\ldots, n$ respectively, and we denote their respective weights simply by the original $v_1,\ldots,v_n$ instead of the heavy notation $v_1^\downarrow, \ldots, v_n^\downarrow$. So, products $1,\ldots, n$ and their respective weights $v_1,\ldots, v_n$ will refer to those of the modified instance \ref{eq:SPVCd}.







\subsection{Linearization of the objective function and algorithm}\label{subsec:linearsol}
In this section, we aim at linearizing the optimization problem \ref{eq:SPVC}, through a guessing procedure of a certain number of parameters pertaining to the optimal solution. We start by presenting our detailed guessing procedure in Section \ref{subsec:guessing}, before leveraging the guessed parameters to construct a linearized formulation in Section \ref{subsec:linearprog}. In Section \ref{subsec:GKPS}, we present a dependent rounding scheme introduced in \cite{gandhi2002dependent}, which we use later in Section \ref{subsubsec:bipartite} to round a fractional solution of the linearized program, into a feasible integral solution for \ref{eq:SPVCd}. 
\subsubsection{Guessing procedure}\label{subsec:guessing}
 Let $S_1^*,\ldots, S_T^*$ be an (unknown) optimal solution for our problem \ref{eq:SPVCd}.
\paragraph{Customer types.}In this definition, we aim at categorizing customers with respect to the total weight of the products offered to them in an optimal solution. First we say that a customer $t\in [T]$ is {\em light} if $V(S_t^*) < \eps$. Similarly, we say that a customer is {\em heavy} if $V(S_t^*)\geq 1/\eps$. Otherwise (i.e., if $\eps\leq V(S_t^*)<1/\eps$), we say that the customer is {\em medium}. We further partition medium customers into $L$ classes $G_1, \ldots, G_L$, depending on the value of $V(S_T^*)$. Specifically, we define every class $\ell =1,\ldots,L$ as follows,\begin{equation*}
    G_{\ell} = \left\{ t\text{ medium }\,\colon\, \eps\cdot (1+\eps)^{\ell-1}\leq V(S_t^*)< \eps\cdot (1+\eps)^{\ell}\right\}.
\end{equation*}
We fix $L$ to be the smallest integer such that $\eps\cdot(1+\eps)^{L}\geq 1/\eps$. In particular, $L = O(\frac{1}{\eps}\log(\frac{1}{\eps}))$. Therefore, the classes $G_1,\ldots, G_L$ form a partition of medium customers. Additionally, we denote by $G_{light}$ and $G_{heavy}$ the sets of light and heavy customers respectively. 

\paragraph{Packing patterns.}First let us start by defining classes of products. Recall from Section \ref{subsec:discret} that for every product $i\in \Nc$ such that $v_i \geq \eps^5$, either there exists some $1\leq q\leq Q-1$ such that $v_i = \eps^5\cdot(1+\eps)^{q-1}$ or $v_i = 1/\eps$, where $Q$ is the smallest integer such that $\eps^5\cdot (1+\eps)^{Q-1}\geq 1/\eps$. Let $D_q$ denote the set of all the products of $\Nc$ such that $v_i = \eps^5\cdot (1+\eps)^{q-1}$, for every $1\leq q\leq Q-1$, and let $D_Q$ denote the set of all products of $\Nc$ such that $v_i = 1/\eps$. Similarly, let $D_0$ be the class of all products $i\in \Nc$ whose weight $v_i<\eps^5$. Note that by construction, the classes $D_0, \ldots D_Q$ form a partition of the universe of products $\Nc$. A packing pattern is a $(Q+1)$-dimensional vector, taking values in $\{0,1,2,\ldots, 1/\eps^6, \star\}$, where $1/\eps^6$ is assumed to be an integer without loss of generality. For a given customer, each entry $q=0,\ldots,Q$ of its associated packing pattern characterizes the number of products from class $D_q$ in the assortment $S_t^*$, and $\star$ should be viewed simply as a symbol which characterizes the fact that the number of products from class $q$ is (strictly) greater than $1/\eps^6$.
Using this definition, the number of packing patterns is  $$\left(\frac{1}{\eps^6}+2\right)^{Q+1} = 2^{O\left(\left(Q+1\right)\log\left(\frac{1}{\eps}\right)\right)}   =2^{O\left(\frac{1}{\eps}\log^2\left(\frac{1}{\eps}\right)\right)}.$$ 
% \paragraph{Guessing. } With all the previous definitions, we are now able to present the guessing procedure. We first start by guessing the number of customers of each type in the optimal solution $S_1^*, \ldots, S_T^*$, i.e., we guess the cardinalities $|G_{light}|,|G_{heavy}|, |G_1|,\ldots ,|G_L|$. Then for each customer type $\ell \in \{\text{light}, \text{heavy}, 1,\ldots, L\}$, and every given packing pattern  $P = (p_0, \ldots, p_Q)$, we guess the number of customers of type $\ell$, who use the packing pattern $P$, which we denote by $K_{\ell, P}$. 


\paragraph{Guessing. } With all the previous definitions, we are now able to present the guessing procedure. For each couple consisting of a class of customers $\ell \in \{\text{light}, \text{heavy}, 1,\ldots, L\}$, and a packing pattern $P$, we guess the number of customers in $G_{\ell}$ that use the packing pattern $P$. Let us call this guessed number $K_{\ell,P}$.

It is important to note that using the above guessing procedure allows us to determine, for any given customer $t\in [T]$, both her type and her packing pattern in the optimal solution, up to a permutation of $\{1,\ldots, T\}$. Indeed, since for each class $\ell\in \{\text{light}, \text{heavy}, 1,\ldots, L\}$, we know the number of customers that use each packing pattern, in particular, we know the quantity $|G_\ell|$. Hence, due to the fact that all customers are initially interchangeable, we arbitrarily assign each customer to a type according to our guess, without loss of generality. Subsequently, given our guess $K_{\ell, P}$ for every customer type $\ell$, and every packing pattern $P$, and since the customers within the same type are all interchangeable a priori, we arbitrarily assign each customer to a packing pattern, according to our guess, without loss of generality. Therefore, for each customer $t\in [T]$, we have determined both her type and her packing pattern, up to a permutation of $\{1,\ldots,T\}$.
\paragraph{Complexity of the guessing procedure.}For any couple consisting of a specific class of customers and a specific packing pattern, we guess the number of customers from said class who use said packing pattern. There are $L+2 = O(\frac{1}{\eps}\log(\frac{1}{\eps}))$ classes of customers, and $O(2^{O(\frac{1}{\eps} \log^2(\frac{1}{\eps}))})$ packing patterns. The number of couples is therefore given by $O(2^{O(\frac{1}{\eps} \log^2(\frac{1}{\eps}))})$.  Since there are $T$ customers, there are $T+1$ possible guesses for each couple consisting of a class of customers and a packing pattern. Therefore, the total number of guesses is $O(T^{2^{O(\frac{1}{\eps}\log^2(\frac{1}{\eps}))}})$.
\subsubsection{Integer linear programming formulation}\label{subsec:linearprog}
Following the detailed presentation of the guessing procedure, we are now ready to leverage the guessed parameters in order to introduce our integer linear programming formulation \ref{eq:ILP}. This formulation is crucial to our algorithm, since its approximate solution is later (Section \ref{subsec:analysis}) shown to be near optimal for \ref{eq:SPVCd}. In \cite{segev2021efficient}, the authors use a similar idea in order to obtain rounded solutions for the Santa Claus problem.

Let us start with some useful notation.
For each customer $t$, we define the lower bound $V_t$ on the $V(S_t^*)$ as follows. If $t$ is light (resp. heavy) then $V_t = 0$ (resp. $V_t = 1/\eps$), and if $V_t$ is of type $\ell$ for some $\ell = 1,\ldots, L$, then $V_t = \eps\cdot (1+\eps)^{\ell-1}$. Finally, we denote the packing pattern of customer $t$ by $(k_{0,t}, k_{1,t},\ldots, k_{Q,t})$.
Using these newly defined parameters, we define the following integer linear program:

\begin{equation}\label{eq:ILP}\tag{ILP}
 \begin{array}{rrclcl}
\displaystyle \max_{\mathbf{x}} & \multicolumn{3}{l}{\displaystyle\sum_{t\in G_{light}}\sum_{i\in \Nc}v_ix_{it}} \\
\textrm{s.t.} &\displaystyle\sum_{t=1}^T x_{it}& \geq & \ell_i,& & \forall i\in \Nc ,\\
&\displaystyle\sum_{i\in \Nc} x_{it}& \leq & k,& & \forall t\in [T],\\
&\displaystyle \sum_{i\in D_q}x_{it} & = & k_{q,t}, & & \forall t\in [T],\,\forall q=0,\ldots,Q \text{ such that } k_{q,t}\neq \star,\\
&\displaystyle \sum_{i\in D_q}x_{it} & > & \displaystyle\frac{1}{\eps^6}, & & \forall t\in [T],\,\forall q=0,\ldots,Q \text{ such that } k_{q,t}= \star,\\
&\displaystyle \sum_{i\in \Nc} v_i x_{it} & \geq & V_{t}, & & \forall t\in [T], \\
&\displaystyle \sum_{i \in \Nc} v_ix_{it} & \leq & \eps, & & \forall t\in G_{light}, \\
& x_{it} & \in & \{0,1\}, & & \forall i\in \Nc,\, \forall t\in [T].
\end{array}
\end{equation}
The last step is to present a way to approximate \ref{eq:ILP}. To this purpose, the high level idea of our method is to start by solving the linear relaxation of \ref{eq:ILP}, then to use a specific rounding of the obtained fractional solution, in order to derive a near-optimal integral solution.

Let $\mathbf{x^*}$ be an optimal solution to the linear relaxation of \ref{eq:ILP}, that is, to the linear program obtained by replacing the constraint $x_{it}\in \{0,1\}$ by the relaxed constraint $x_{it} \in [0,1]$. Similarly, let $S_1^*,\ldots, S_T^*$ be an optimal sequence of assortments for \ref{eq:SPVCd}. The following lemma relates $\mathbf{x^*}$ to the optimal solution $S_1^*,\ldots, S_T^*$.
\begin{lemma}\label{lem:LPopt}
    $$
        \sum_{t=1}^T\frac{\sum_{i\in \Nc} v_ix^*_{it}}{1+\sum_{i\in \Nc} v_ix^*_{it}} \geq (1-\eps)\cdot\sum_{t=1}^T\frac{V(S_t^*)}{1+V(S_t^*)}.
    $$
\end{lemma}
The proof of this Lemma is included in Appendix \ref{apx:LPopt}. The next step is to round $\mathbf{x^*}$ in order to obtain a feasible solution for \ref{eq:ILP}.

\subsubsection{The dependent rounding scheme}\label{subsec:GKPS}
In this section, we briefly present the dependent rounding scheme in \cite{gandhi2002dependent}, which we use to round our fractional solution into an integral one. Given a bipartite graph $(L,R,E)$ and values $x_{i,j}\in [0,1]$ for every edge $(i,j)\in E$, the authors in \cite{gandhi2002dependent} present a polynomial time rounding scheme which returns a sequence of random variables $X_{i,j}\in \{0,1\}$ representing a rounding of the values $x_{i,j}$, and which verify the following properties:
\setenumerate[1]{label={\bf (P\arabic*)}}
\begin{enumerate}
    \item\label{it:P1} {\bf Marginal distributions:} For every edge $(i,j)$, $\E(X_{i,j}) = x_{i,j}$.
    \item\label{it:P2} {\bf Degree-preservation:} The fractional degree of every edge is rounded to its floor or its ceiling, i.e., for any $i\in L\cup R$, if $d_i = \sum_{j\in L\cup R}x_{ij}$, then $\sum_{j\in L\cup R}X_{ij}\in\{\lfloor d_i\rfloor, \lceil d_i\rceil\}$.
    \item\label{it:P3} {\bf Negative correlation:} For any node $i\in L\cup R$, and any subset $S\subseteq N(i)$, where $N(i)$ if the set of neighbors of $i$, we have\begin{align*}
        &\P\left[\bigwedge_{j\in S}(X_{i,j}=0)\right]\leq \prod_{j\in S}\P\left[X_{i,j} = 0\right]\\
        \text{and}\quad \quad&\P\left[\bigwedge_{j\in S}(X_{i,j}=1)\right]\leq \prod_{j\in S}\P\left[X_{i,j} = 1\right].
    \end{align*}
\end{enumerate}
Negative correlation {\bf (P3)} is a strong property that allows us to derive Chernoff bounds, as stated in the following lemma.
\begin{lemma}[{\cite[Theorem~1.10.24][66]{doerr2020probabilistic}}]\label{lem:chernoff}
    If $X_1,\ldots,X_n$ are negatively correlated random variables, then for any vector $a=(a_1,\ldots,a_n)\in [0,1]^n$ we have$$
        \P[a(X)\leq (1-\eps)\cdot \E[a(X)]] \leq \exp\left(-\frac{\eps^2\cdot \E[a(X)]}{2}\right),
    $$
    where $a(X) = \sum_{i=1}^na_iX_i$.
\end{lemma}
Next, we construct a framework that allows us to use the dependent rounding scheme in order to derive a feasible solution for \ref{eq:ILP}.


\subsubsection{Construction of the bipartite graph and feasibility of the rounded solution}\label{subsubsec:bipartite}
Recall that the high level idea is to use the rounding procedure presented in \ref{subsec:GKPS}, in order to derive an integer solution to \ref{eq:ILP}. Noticing that the rounding procedure takes as input a bipartite graph and a sequence of edge-associated values, the first step is to interpret the solution of the linear relaxation of \ref{eq:ILP} as a sequence of edge-associated values in a carefully constructed bipartite graph.
Before presenting our proposed bipartite graph, let us introduce the two following definitions. We say that a customer $t\in [T]$ is {\em bounded} if $k_{q,t} \neq \star$, for all $q=1,\ldots, Q$ (note that $q$ starts from $1$ in this definition and not from $0$). Otherwise, we say that customer $t$ is {\em unbounded}. In other words, customer $t$ is bounded if she is not offered more than $1/\eps^5$ products from any class $D_q$ where $q\geq 1$, in the optimal solution.
\paragraph{The bipartite graph.}
We start by defining the node sets $L$ and $R$. The set $L$ is simply the set of all products in $\Nc$. Let us now describe the set of right nodes $R$. First, for every unbounded customer $t$, we introduce a node $t$ in the set $R$. Alternatively, for every bounded customer $t$, we introduce $Q+1$ nodes in the set $R$, specifically, one node $(t,q)$ for every class of products $q=0,\ldots, Q$. Therefore, $$R = \{t\,:\, t\text{ unbounded}\} \cup \left\{(t,q)\,\colon \, t \text{ bounded}, q =0,\ldots,Q\right\}.$$ Let us now describe the edge set $E$ of the bipartite graph, as well as the associated value $x_e$ of each edge $e\in E$. Let us fix a node $i\in L$ and describe all of its adjacent edges and associated values. First, $i$ neighbors every node $t$ where $t$ is an unbounded customer, and the edge $(i,t)$ is associated with the value $x^*_{it}$. Next, assume $q$ is the class of products containing $i$, then the node $i$ neighbors every node $(t,q)$ for $t\in [T]$, and the edge $(i, (t,q))$ is associated with the value $x^*_{it}$. This completes our description of the bipartite graph.
\paragraph{PTAS.} Let us summarize the steps of our PTAS. The first step is to perform the guessing procedure presented in Section \ref{subsec:guessing}, and then use our guess to construct \ref{eq:ILP}. The second step is to solve the linear relaxation of \ref{eq:ILP} using standard linear programming techniques, and obtain a fractional solution $\mathbf{x}^*$. The third step is to use $\mathbf{x}^*$ to construct the bipartite graph as described above. We can now apply the dependent rounding presented in Section \ref{subsec:GKPS} to derive a sequence of random variables $(X_{it}\,\colon\, i\in \Nc, t\in[T])$ representing rounded values. Finally, we denote by $S_1,\ldots, S_T$ the associated sequence of assortments, that is, for each $t\in [T]$, \begin{equation*}
        S_t= \{i \in \Nc \,\colon\, X_{it}=1\},
    \end{equation*}
and the algorithm returns this sequence $(S_1, \ldots, S_T)$.
\paragraph{Complexity analysis.}Since our algorithm is exhaustive, the steps presented above are applied for every possible guess. Solving the linear relaxation of \ref{eq:ILP} takes $O(n^3T^3)$ using standard linear programming algorithms (interior point method for example). Constructing the bipartite graph takes $O(nTQ)$ running time, and running the dependent rounding scheme takes $O((nTQ)^3)$. Therefore, processing each guess takes $O(n^3T^3Q^3) = O(n^3T^3\frac{1}{\eps^3}\log^3{\frac{1}{\eps}})$ running time. Finally, since there are $O(T^{2^{O(\frac{1}{\eps}\log^2(\frac{1}{\eps}))}})$ possible guesses as argued in Section \ref{subsec:guessing}, the total running time of our algorithm is:
$$
O\left(T^{2^{O\left(\frac{1}{\eps}\log^2\left(\frac{1}{\eps}\right)\right)}} \cdot n^3T^3\frac{1}{\eps^3}\log^3{\frac{1}{\eps}}\right) = O\left(n^3T^{2^{O\left(\frac{1}{\eps}\log^2\left(\frac{1}{\eps}\right)\right)}}\right),
$$
which is polynomial in the size of the input for any fixed $\eps$, confirming that our algorithm is indeed a PTAS.












\subsection{Near-optimality of the rounded solution}\label{subsec:analysis}
In this section, we show that the obtained solution $(S_1,\ldots,S_T)$ is near-optimal for \ref{eq:SPVCd}. The main result is stated in the following theorem.
% \begin{algorithm}\label{alg:PTASd}
% \caption{Polynomial-time approximation scheme for \ref{eq:SPVCd}}
% \begin{algorithmic}[1]
% \REQUIRE $\Nc$, $T$, $v_i$ for every $i\in \Nc$
% \ENSURE Sum of integers from 1 to $N$
% \STATE $S_t^* \leftarrow \emptyset$ for $t\in [T]$
% \STATE $A \leftarrow 0$
% \FOR{every guess of customer types and packing patterns}
%     \STATE Solve the linear relaxation of \ref{eq:ILP}
%     \STATE Use the the rounding in Section \ref{subsec:GKPS} to compute $S_1,\ldots, S_T$
%     \STATE $A \leftarrow \obj(S_1,\ldots,S_T)$
%     \IF{$A \geq A^*$}
%         \STATE $S_t^* \leftarrow S_t$ for $t\in [T]$
%         \STATE $A^* \leftarrow A$
%     \ENDIF
% \ENDFOR
% \RETURN $A^*$, $(S_t^*\,\colon\,t\in [T])$
% \end{algorithmic}
% \end{algorithm}

\begin{theorem}
    The sequence of assortments $S_1,\ldots, S_T$ is feasible for \ref{eq:SPVCd} and we have
    \begin{equation*}
        \E\left[\sum_{t=1}^T \frac{V(S_t)}{1+V(S_t)} \right]\geq (1-3\eps) \cdot \opt^\downarrow.
    \end{equation*}
\end{theorem}

\begin{proof}
    In this proof, we start by showing that the sequence $S_1, \ldots,S_T$ is feasible for \ref{eq:SPVCd}. Next, we show that the objective of the sequence of assortments $S_1,\ldots,S_T$ is within $1-3\eps$ of the optimal objective. Recall that $S_1^*,\ldots, S_T^*$ is an optimal sequence of assortments for \ref{eq:SPVCd}.
    
    \paragraph{Feasibility. }First, the visibility constraint is straightforward. Indeed, let $i\in \Nc$. By the first constraint of the linear relaxation of \ref{eq:ILP}, we have: $\sum_{t=1}^Tx_{it}^* \geq \ell_i$. Noticing that $\sum_{t=1}^Tx_{it}^*$ is the fractional degree of node $i$ in the bipartite graph, we have by property \ref{it:P2}, $\sum_{t=1}^TX_{it} \geq \lfloor\ell_i\rfloor = \ell_i$, which proves that the visibility constraint is respected. We now show that the cardinality constraint is respected after the rounding. Let $t\in [T]$. First, if $t$ is unbounded, then the cardinality of $S_t$ is simply given by the degree of node $t$. Since $\sum_{i=1}^nx_{it}^* \geq k$, then by property \ref{it:P2}, we have $|S_t| = \sum_{i\in \Nc}X_{it} \geq k$. Alternatively, if $t$ is bounded, for every class $q=1,\ldots,Q$, the degree of node $(t,q)$ is $\sum_{i\in D_q}
    x_{it}^*$ which is equal to $k_{q,t}$, according to the third constraint of \ref{eq:ILP}. Since $k_{q,t}$ is an integer, the degree of node $(t,q)$ remains unchanged after the rounding, according to property \ref{it:P2}. Therefore, we have almost surely:\begin{align*}
        \sum_{i\in \Nc}v_i X_{it}&= \sum_{i\in D_0}v_iX_{it} + \sum_{q=1}^Q\sum_{i\in D_q}v_iX_{it} \\ &\leq \left\lceil\sum_{i\in D_0}v_ix_{it}^*\right\rceil+ \sum_{q=1}^Q k_{q,t} \\ &= \left\lceil\sum_{i\in D_0}v_ix_{it}^*+ \sum_{q=1}^Q k_{q,t} \right\rceil\\ &= \left\lceil\sum_{i\in \Nc}v_ix_{it}^* \right\rceil\\
        &\leq \lceil k\rceil=k.
    \end{align*}
    The first inequality follows from Property \ref{it:P2}. In the equality in the third line, we use the fact that the second summand is an integer. This proves the cardinality constraint and concludes the feasibility proof. 
    \paragraph{Near-optimality. }
    We show that the contribution of every customer $t\in [T]$ to the objective function, i.e., the quantity $V(S_t)/(1+V(S_t))$ is within $1-2\eps$ of the contribution of $t$ in the optimal solution $\mathbf x^*$, in expectation.
    Let $t \in [T]$.
    \\ \noindent {\em \underline{Case 1}: If $t$ unbounded.} Let $q_t \in \{1,\ldots,Q\}$ be the class of products such that $k_{q_t,t} = \star$. We have \begin{equation*}
        \sum_{i\in D_{q_t}} v_i x_{it}^* \geq \eps^5\cdot \sum_{i\in D_{q_t}} x_{it}^* \geq \eps^5\cdot \frac{1}{\eps^6} = \frac{1}{\eps}.
    \end{equation*}
Therefore, 
    \begin{align}
        \P\left[V(S_t) \leq  \frac{1-\eps}{\eps}\right] &= 
        \P\left[\sum_{i\in \Nc}v_i X_{it} \leq  \frac{1-\eps}{\eps}\right] \notag\\& \leq \P\left[\sum_{i\in \Nc}v_i X_{it} \leq  (1-\eps)\cdot \sum_{i\in D_{q_t}} v_i x_{it}^*\right]\notag\\
        & \leq \P\left[\sum_{i\in D_{q_t}}v_i X_{it} \leq  (1-\eps)\cdot \sum_{i\in D_{q_t}} v_i x_{it}^*\right]\notag\\
        & = \P\left[\sum_{i\in D_{q_t}}X_{it} \leq  (1-\eps)\cdot \sum_{i\in D_{q_t}}x_{it}^*\right]\notag\\
        & \leq \exp\left(-\frac{\eps^2\cdot \sum_{i\in D_{q_t}}x_{it}^*}{2}\right)\notag\\
        & \leq \exp\left(-\frac{1}{2\eps^4}\right)\notag\\
        &\leq \eps.\label{eq:unboundbound}
    \end{align}
    In the second equality, we use the fact that products within the same class have the same preference weight to simplify all the weights. The inequality in the fifth line is a direct application of Lemma \ref{lem:chernoff}, and in the one in the sixth line, we use the fact that $\sum_{i\in D_{q_t}} x_{it}^*> 1/\eps^6$, since $k_{q_t, t} = \star$.
    Therefore, by conditioning on the event $\{V(S_t)>(1-\eps)/\eps\}$, and neglecting one of the terms, we have\begin{align*}
        \E\left[\frac{V(S_t)}{1+V(S_t)}\right] &\geq \P\left[V(S_t)> \frac{1-\eps}{\eps}\right] \cdot  \E\left[\frac{V(S_t)}{1+V(S_t)} \,\mid\,V(S_t)> \frac{1-\eps}{\eps}\right]\\
        &\geq (1-\eps)\cdot \frac{\frac{1-\eps}{\eps}}{1+\frac{1-\eps}{\eps}}\\
        &=(1-\eps)^2\\
        &\geq (1-2\eps)\cdot \frac{\sum_{i\in \Nc}v_ix_{it}^*}{1+\sum_{i\in \Nc}v_ix_{it}^*}.
    \end{align*}
    The second inequality follows from Equation \eqref{eq:unboundbound} and the monotonicity of $x\mapsto x/(1+x)$ on $[0,+\infty)$.


    \noindent{\em \underline{Case 2}: If $t$ is bounded and heavy or if $t$ is medium. }Then we have\begin{equation*}
    \sum_{i\in \Nc}v_iX_{it} = \sum_{i\in D_0}v_iX_{it} + \sum_{q = 1}^{Q-1}\sum_{i\in   D_q}v_iX_{it}.
\end{equation*}
    Let $W_t^{\lar} = \sum_{i\in \Nc\setminus D_0}v_iX_{it}$, and $W_t^{\sma} = \sum_{i\in D_0}v_iX_{it}$. Similarly, let $w_t^{\lar} = \sum_{i\in \Nc\setminus D_0}v_ix^*_{it}$, and $w_t^{\sma} = \sum_{i\in D_0}v_ix^*_{it}$.
\begin{claim}\label{cl:nonlightcust}
    With probability $1$, we have $W_t^{\lar} = w_t^\lar.$
        % \sum_{q = 1}^{Q-1}\sum_{i\in   D_q}v_iX_{it} \geq \sum_{q = 1}^{Q-1}\sum_{i\in   D_q}v_ix_{it}^*.

\end{claim}
The proof of this result is a direct application of the degree preservation property, and is deferred to Appendix \ref{apx:nonlightcust}.
\begin{itemize}
    \item If $w_t^{\sma}\leq \eps \cdot\sum_{i\in \Nc} v_ix_{it}^*$, then intuitively, the contribution of the products in $D_0$ represent an $\eps$-fraction of the total contribution and can therefore be neglected. Formally, we have with probability $1$
    $$V(S_t)\geq W_t^{\lar}  = w_t^{\lar}
         =  \sum_{i\in \Nc}v_i x_{it}^* - w_t^\sma\geq (1-\eps)\cdot \sum_{i\in \Nc}v_ix_{it}^*.$$
    % \begin{align*}
    %     V(S_t)&\geq W_t^{\lar} \\& = w_t^{\lar}\\
    %     & =  \sum_{i\in \Nc}v_i x_{it}^* - w_t^\sma\\
    %     &\geq (1-\eps)\cdot \sum_{i\in \Nc}v_ix_{it}^*.
    % \end{align*}
    The first equality is a consequence of Claim \ref{cl:nonlightcust}, and the second inequality follows from the case hypothesis.
    Therefore, by the monotonicity of $x\mapsto x/(1+x)$, we have with probability $1$\begin{equation}
        \frac{V(S_t)}{1+V(S_t)} \geq (1-\eps)\cdot \frac{\sum_{i\in \Nc}v_i x_{it}^*}{1+\sum_{i\in \Nc}v_i x_{it}^*}.
    \end{equation}
    \item Otherwise, we have
    \begin{align}
        \P\left[W_t^\sma \leq (1-\eps)\cdot w_t^\sma\right] & = \P\left[\sum_{i\in D_0}\frac{v_i}{\eps^5}X_{it} \leq (1-\eps)\cdot \sum_{i\in D_0}\frac{v_i}{\eps^5}x_{it}^*\right]\notag\\
        &\leq \exp\left(-\frac{\eps^2\cdot \sum_{i\in D_0}v_ix_{it}^*}{2\eps^5}\right)\notag\\
        &\leq \exp\left(-\frac{\eps \cdot \sum_{i\in \Nc}v_ix_{it}^*}{2\eps^3}\right)\notag\\
        &\leq \exp\left(-\frac{1}{2\eps}\right)\notag\\
        &\leq \eps.\label{eq:boundbound}
    \end{align}
    The first inequality follows from Lemma \ref{lem:chernoff}. The second inequality is a consequence of the case hypothesis, and third one follows from the fact that $t$ is not light, and hence that $\sum_{i\in \Nc}v_ix_{it}^* \geq \eps$.
    Therefore, by conditioning on the event $\{W_t^\sma\leq (1-\eps)\cdot w_t^\sma\}$ and neglecting one of the terms, we have\begin{align*}
        \E\left[\frac{V(S_t)}{1+V(S_t)}\right]
        &\geq \P\left[W_t^{\sma}> (1-\eps)\cdot w_t^{\sma}\right] \cdot  \E\left[ \frac{V(S_t)}{1+V(S_t)}\,\mid\,W_t^{\sma}> (1-\eps)\cdot w_t^{\sma}\right]\\
        &= \P\left[W_t^{\sma}> (1-\eps)\cdot w_t^{\sma}\right] \cdot  \E\left[ \frac{w_t^\lar+ W_t^\sma}{1+w_t^\lar+ W_t^\sma}\,\mid\,W_t^{\sma}> (1-\eps)\cdot w_t^{\sma}\right]\\
        &\geq (1-\eps) \cdot \frac{w_t^\lar+ (1-\eps)\cdot w_t^s}{1+w_t^\lar+(1-\eps)\cdot w_t^s}\\
        &\geq (1-\eps)^2.\frac{\sum_{i\in \Nc}v_ix^*_{it}}{1+\sum_{i\in \Nc}v_ix^*_{it}}\\
        & \geq (1-2\eps).\frac{\sum_{i\in \Nc}v_ix^*_{it}}{1+\sum_{i\in \Nc}v_ix^*_{it}}.
    \end{align*}
\end{itemize}
    In the equality, we apply Lemma \ref{cl:nonlightcust}. In the second inequality, we use Equation \eqref{eq:boundbound}, as well as the monotonicity of $x\mapsto x/(1+x)$.

\noindent {\em \underline{Case 3}: If $t$ is light.}
\begin{claim}\label{cl:lightcusts}
    We have $$
        \E\left[\frac{V(S_t)}{1+V(S_t)}\right] \geq (1-2\eps)\cdot \frac{\sum_{i\in \Nc}v_ix_{it}^*}{1+\sum_{i\in \Nc}v_ix_{it}^*}.
    $$
\end{claim}
Let us present the high level idea for the proof of Claim \ref{cl:lightcusts}. If $t\in [T]$ is a light customer then $\sum_{i\in \Nc} v_ix_{it}^* \leq \eps$. Therefore, the denominator $1+V(S_t)$ is in expectation equal to $1+\sum_{i\in \Nc}v_ix_{it}^* \leq 1+\eps$. The high level idea is to use this remark to approximate $V(S_t)/(1+V(S_t))$ to simply $V(S_t)$, which in expectation is equal to $\sum_{i\in \Nc}v_i x^*_{it}$. Finally, this last term is trivially lower bounded by $\sum_{i\in \Nc}v_i x^*_{it}/(1+\sum_{i\in \Nc}v_i x^*_{it})$. The formal proof is deferred to Appendix \ref{apx:nonlightcust}.


Combining cases 1, 2 and 3, and by linearity of expectation, we have\begin{align*}
    \E\left[\sum_{t =1}^T\frac{V(S_t)}{1+V(S_t)}\right] &\geq (1-2\eps)\cdot \sum_{t=1}^T \frac{\sum_{i\in \Nc}v_ix_{it}^*}{1+\sum_{i\in \Nc}v_ix_{it}^*} \\ &\geq (1-2\eps)\cdot(1-\eps)\cdot \sum_{t=1}^T\frac{V(S_t^*)}{1+V(S_t^*)} \\&\geq (1-3\eps)\cdot \sum_{t=1}^T\frac{V(S_t^*)}{1+V(S_t^*)}\\
    &=(1-3\eps)\cdot \opt^\downarrow,
\end{align*}
where the second inequality is an application of Lemma \ref{lem:LPopt}.

\end{proof}

%The proof of this theorem relies  on constructing assortments $S_1,\ldots,S_T$ in polynomial time such that the sum of the weights of the products in each assortment are close enough one from another. This way, we ensure that the value of each of them is either close to the one in an optimal set, or close to a concave upper bound of the optimum. 
    