\section{Model Formulation}\label{sec:form}
    




{\bf The MNL choice model.} Let  $\mathcal{N} \coloneqq \{1,\ldots, n\}$ be a universe of substitutable products at our disposal. 
 Each product $i \in {\cal N}$ has a  price $p_i \geq 0$.  Without loss of generality, we order the products by non-increasing prices, i.e., $p_1 \geq p_2 \geq \ldots \geq p_n$.  
An assortment of products or an offer set, is simply a subset of products $S \subseteq {\cal N}$. Additionally, the option of not selecting any product is symbolically represented as product $0$, and referred to it as the no-purchase option.  



We assume that customers make choices according to a Multinomial Logit model. Under this model, each product $i \in {\cal N}$ is associated with a preference weight $v_i >0$. Note that $v_i $ captures the attractiveness of product $i$, meaning a  high  preference weight indicates a high popularity. Without loss of generality, we use the standard convention that the no-purchase preference weight is normalized to $v_0=1$. 
We use the notation $V(S) \coloneqq \sum_{i \in S} v_i$, which is the total weight of a subset $S \subseteq \mathcal{N}$.
Under the MNL model, if we offer  an assortment $S \subseteq {\cal N}$, the customer chooses product $i$ with  probability 
 $$\phi(i, S) \coloneqq \frac{v_i}{1 + V(S)}.$$
 We refer to $\phi(i,S)$ as the choice probability of product $i$ given assortment $S$. 
 Alternatively, the customer may decide to not purchase any product, which happens with the complementary probability 
  $$\phi(0, S) \coloneqq \frac{1}{1 + V(S)}.$$
Let $R(S)$ be the expected  revenue we get from a customer if we offer assortment $S$. In particular, we have 

$$R(S) \coloneqq   \sum_{i \in S} p_i \phi(i,S) = \frac{\sum_{i \in S} p_i v_i}{1 + \sum_{i \in S} v_i}.$$ 

 






%At each  time period, a customer arrives and we need to offer them an assortment of products $S \subseteq \mathcal{N}$. The customer then decides to purchase a single product, or to leave without purchasing anything. Customers make choices according to a choice model $\phi$, i.e., $\phi(i,S)$ is the choice probability of product $i$ given an offered assortment $S$. 




%We can now define $$R(S) \coloneqq \frac{\sum_{i \in S} p_i v_i}{1 + \sum_{i \in S} v_i}$$ which corresponds to the expected revenue when we offer the set of products $S$ under the MNL model. 


\noindent
{\bf Assortment Optimization with Visibility constraints.} 
We are presented with a finite stream of $T$ customers. Each  customer $t$  will be offered an assortment $S_t$. Customers make choices according to the same MNL model, i.e., a customer decides to purchase product $i$ from assortment $S_t$ with a probability $\phi(i, S_t)$, or they may choose the no-purchase option with  probability $\phi(0, S_t)$. The expected revenue we obtain from customer $t$ is $R(S_t)$.
To ensure visibility, we impose constraints that require each product $i \in \mathcal{N}$ to be shown to at least $\ell_i$ customers. Note that the parameters $\ell_i$ are exogenous and satisfy $\ell_i \in  \{0,\ldots, T\} $ for all $i \in \mathcal{N}$.
Our objective is to determine the assortment $S_t$ to offer to each customer $t$ in order to maximize the total expected revenue while satisfying the visibility constraints. We refer to this problem as the {\em Assortment optimization Problem with Visibility constraints}  (\ref{APV}). It can be formulated as follows:




\begin{equation}
\label{APV}
\begin{aligned}
 \max_{ S_1, \ldots, S_T \subseteq \mathcal{N}}  & \; \;      \sum_{t=1}^T  R(S_t)   \\  
  s.t. \;\;   & \;\;  \sum_{t=1}^T  \mathbbm{1}(i \in S_t) \geq \ell_i, \;\;\; \forall i \in \mathcal{N}.
\end{aligned}
\tag{\sf{APV}}
\end{equation}

