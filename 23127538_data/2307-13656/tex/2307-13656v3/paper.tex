\documentclass[english,a4paper,11pt]{article}
\usepackage[a4paper,hmargin=1.0in,vmargin=1.0in]{geometry}
\usepackage[round]{natbib}
\usepackage[dvipsnames]{xcolor}
\usepackage[linktocpage=true,pagebackref=true,bookmarks=true,bookmarksopen,bookmarksnumbered]{hyperref}
\usepackage{sectsty}
\usepackage[english]{babel}
\usepackage[utf8]{inputenc}
\usepackage{amsfonts}
\usepackage{amssymb}
\usepackage{amsmath}
\usepackage{stmaryrd}
\usepackage{comment}
\usepackage{setspace}
\usepackage{bbm}
\usepackage{mathtools}
\usepackage{enumitem}
%\usepackage{authblk}
\usepackage{amsthm}
\usepackage{algorithm}
\usepackage{algorithmic}
\usepackage{tikz}
\usepackage{comment}
%\usepackage{pgfplots}
\usepackage{thmtools}
\usepackage{thm-restate}
\usepackage{booktabs}
\usepackage{array}
\usepackage{cleveref}

% MATH %%%%%%%%%%%%%%%%%%%%%%%%%%%%%%%%%%%%%%%%%%%%%%%%%%%%%%%%%%%%
\newcommand{\R}{\mathbb{R}}
\newcommand{\N}{\mathbb{N}}
\newcommand{\Q}{\mathbb{Q}}
\newcommand{\E}{\mathbb{E}}
\newcommand{\Z}{\mathbb{Z}}
\newcommand{\obj}{{\cal E}}
\newcommand{\bl}{\boldsymbol{\ell}}
\newcommand{\Nc}{\mathcal{N}}
\newcommand{\opt}{{\sf OPT}}
\renewcommand{\P}{\mathbb{P}}
\newcommand{\eps}{\epsilon}
\newcommand{\lb}{\llbracket}
\newcommand{\sma}{{s}}
\newcommand{\lar}{{l}}
\newcommand{\extr}{\overline{R}}
\newcommand{\cont}[1]{C_{#1}}
\newcommand{\LP}{{LP} }
\newcommand{\Var}{\mathrm{Var}}
\newcommand{\Cov}{\mathrm{Cov}}
\newcommand{\nota}[1]{\hat{#1}}
% \renewcommand{\overline}{\widetilde}

\renewcommand{\overline}[1]{\left\llbracket#1\right\rrbracket}

\newcommand{\black}[1]{{{\color{black}{#1}}}}

\DeclareMathOperator*{\argmin}{arg\,min}
\DeclareMathOperator*{\argmax}{arg\,max} 
\newcommand{\MyAbove}[2]{\genfrac{}{}{0pt}{}{#1}{#2}}

% THEOREM-LIKE ENV %%%%%%%%%%%%%%%%%%%%%%%%%%%%%%%%%%%s%%%%%%%%%%%%%%
\newtheoremstyle{DStheorem}% name of the style to be used
  {\topsep}% measure of space to leave above the theorem. E.g.: 3pt
  {\topsep}% measure of space to leave below the theorem. E.g.: 3pt
  {\itshape}% name of font to use in the body of the theorem
  {0pt}% measure of space to indent
  {\scshape}% name of head font
  {.}% punctuation between head and body
  { }% space after theorem head; " " = normal interword space
  {\thmname{#1}\thmnumber{ #2}\thmnote{ (#3)}}
\theoremstyle{DStheorem}
\newtheorem{theorem}{Theorem}[section]
\newtheorem{lemma}[theorem]{Lemma}
\newtheorem{claim}[theorem]{Claim}
\newtheorem{definition}[theorem]{Definition}

%\newtheorem{proposition}[theorem]{Proposition}
\newtheorem{observation}[theorem]{Observation}
\newtheorem{corollary}[theorem]{Corollary}
\renewcommand{\qedsymbol}{\hfill{\rule{2mm}{2mm}}}
\let\oldproofname=\proofname
\renewcommand{\proofname}{\rm\sc{\oldproofname}}

% MORE SETTINGS %%%%%%%%%%%%%%%%%%%%%%%%%%%%%%%%%%%%%%%%%%%%%%%%%%%%
\sectionfont{\large} \subsectionfont{\normalsize}
\allowdisplaybreaks
\onehalfspacing %\doublespacing
\makeindex
\newcommand{\changelocaltocdepth}[1]{%
  \addtocontents{toc}{\protect\setcounter{tocdepth}{#1}}%
  \setcounter{tocdepth}{#1}%
}

% REVIEW %%%%%%%%%%%%%%%%%%%%%%%%%%%%%%%%%%%%%%%%%%%%%%%%%%%%%%%%%%
% \newcounter{dannycounter}
% \newcommand{\Danny}[1]{\noindent\textcolor{black}{$\ll$\textsf{#1}$\gg$\marginpar{\tiny\bf \textcolor{black}{\refstepcounter{dannycounter}Danny~\thedannycounter}}}}

% \newcounter{omarcounter}
% \newcommand{\Omar}[1]{\noindent\textcolor{red}{$\ll$\textsf{#1}$\gg$\marginpar{\tiny\bf \textcolor{red}{\refstepcounter{omarcounter}Omar~\theomarcounter}}}}

\newcounter{marouanecounter}
\newcommand{\marouane}[1]{\noindent\textcolor{violet}{$\ll$\textsf{#1}$\gg$\marginpar{\tiny\bf \textcolor{violet}{\refstepcounter{marouanecounter}Marouane~\themarouanecounter}}}}


\newcounter{omarcounter}
\newcommand{\omar}[1]{\noindent\textcolor{red}{$\ll$\textsf{#1}$\gg$\marginpar{\tiny\bf \textcolor{red}{\refstepcounter{omarcounter}Omar~\theomarcounter}}}}




\hfuzz=5pt
\sloppy



% TITLE %%%%%%%%%%%%%%%%%%%%%%%%%%%%%%%%%%%%%%%%%%%%%%%%%%%%%%%%%%%%
\begin{document}

\begin{titlepage}

\title{Assortment Optimization with Visibility Constraints%
\thanks{An extended abstract of this paper appeared in proceedings of the 25th International Conference on Integer Programming and Combinatorial Optimization (IPCO 2024).}}
% \author{%
% Th\'eo Barr\'e\thanks{Department of Industrial Engineering and Operations Research,
% University of California
% Berkeley, CA, USA. Email: {\tt theo\_barre@berkeley.edu}.}  \and
% Omar El Housni\thanks{School of Operations Research and Information Engineering, Cornell Tech, Cornell University, NY, USA. Email: {\tt \{oe46,mi262,al748\}@cornell.edu}.}
% \and
% Marouane Ibn Brahim\footnotemark[2]}

\author{%
\quad\quad Th\'eo Barr\'e\thanks{Department of Industrial Engineering and Operations Research,
University of California
Berkeley, CA, USA. Email: {\tt theo\_barre@berkeley.edu}.}  \and
Omar El Housni\thanks{School of Operations Research and Information Engineering, Cornell Tech, Cornell University, NY, USA. Email: {\tt \{oe46,mi262,al748\}@cornell.edu}.}
\and
Marouane Ibn Brahim\footnotemark[3] \quad\quad\quad
\and
Andrea Lodi\footnotemark[3]
\and
Danny Segev\thanks{School of Mathematical Sciences and Coller School of Management, Tel Aviv University, Tel Aviv 69978, Israel. Email: {\tt segevdanny@tauex.tau.ac.il}.}  }

\date{}
\maketitle

\setcounter{page}{200}
\thispagestyle{empty}



\begin{abstract}
Motivated by  applications in e-retail and online advertising, we study  the problem of assortment optimization under visibility constraints \eqref{APV}. Here, we are given a universe of substitutable products and a stream of customers. The objective is to determine the optimal assortment of products to offer to each customer in order to maximize the total expected revenue, subject to exogenously-given {\em visibility constraints}, stating that each product should be shown to a minimum number of customers. We assume that customer choices follow a Multinomial Logit model (MNL). 

We provide a structural characterization of optimal assortments  and present \black{a linear} time algorithm for solving \ref{APV}. To this end, we introduce a novel function called the ``expanded revenue" of an assortment and establish its supermodularity; our algorithm takes advantage of this structural property. Additionally, we prove that \ref{APV}  can be formulated as a compact linear program. Next, we consider \ref{APV} with cardinality constraints, \black{which limit the maximum number of products that can be included in an assortment}. We prove \black{this problem} to be strongly NP-hard and not admitting a Fully Polynomial Time Approximation Scheme (FPTAS), even when all products have identical prices. Subsequently, we devise a Polynomial Time Approximation Scheme (PTAS) for \ref{APV} under cardinality constraints with identical prices. 
% Our algorithm starts by linearizing the objective function through a carefully crafted guessing procedure, then solves the linearized program, and finally randomly rounds the obtained solution to derive a near optimal solution for \ref{APV} with cardinality constraints.
We also examine the revenue loss resulting from the enforcement of visibility constraints, comparing it to the unconstrained problem. To offset this loss, we propose a novel strategy to distribute the loss incurred among the products subject to visibility constraints, charging each vendor an amount proportional to their product's contribution to the revenue loss.
\end{abstract}

\bigskip \noindent {\small {\bf Keywords}: Assortment Optimization, Multinomial Logit model, Visibility Constraints,  Supermodularity.}

\end{titlepage}



% CONTENTS %%%%%%%%%%%%%%%%%%%%%%%%%%%%%%%%%%%%%%%
% \setcounter{page}{200}
% \pagestyle{empty}
% \tableofcontents



% SECTIONS %%%%%%%%%%%%%%%%%%%%%%%%%%%%%%%%%%%%%%%%%%%%%%%%%%%%%%%%%%%
%\newpage
\pagestyle{plain}
\setcounter{page}{1}

%
\section{Introduction}



Assortment optimization is a crucial aspect of  decision making in many industries  such as  e-retail and online advertising. In this domain, our goal is to select a subset of available products  to offer to customers in order to maximize a context-appropriate objective function, %In this context, the objective function might vary depending on the business's priorities, 
such as revenue, profit, or market share. For example, e-retailers seek to strategically select which products should be displayed to customers in order to maximize their expected revenue. Online  advertisers strategically select the most effective combination of advertisements to maximize user engagement and desired outcomes, such as click-through rates. 
Consequently, the choice of a well formed assortment is crucial due to inherent substitution effects, where a product's attractiveness depend not only on its intrinsic value but also on the concurrent alternatives presented at that time. For instance, offering a high-quality, high-priced product alongside a comparable product at a significantly lower price may result in diminishing sales for the higher-priced product, leading in turn to an unsatisfactory platform revenue. This highlights the importance of carefully selecting assortments. 





Traditionally, assortment optimization frameworks often overlook a crucial element in contemporary e-commerce: product visibility. In today's complex business landscape, where companies adhere to Service-Level Agreements (SLAs) with suppliers and prioritize sponsored product promotion, product visibility within an assortment is pivotal. 
SLAs  often define conditions for product representation, ensuring equitable visibility for each supplier's products on the platform. Moreover, the concept of sponsored products has gained traction, with brands willing to pay for prominent display and increased visibility. While these strategies influence consumer behavior,  solely focusing on products visibility without considering broader assortment optimization can lead to an imbalanced product mix, resulting in reduced customer satisfaction and overall revenue.

In this paper, we introduce the notion of {\em visibility constraints} in the context of Assortment Optimization. The purpose is to enforce a minimum display of each product, i.e., each product has to be shown at least a certain number of times in the displayed assortments. This constraint models both Service-Level Agreements and sponsored products. It can also capture the settings where the platform would like to ensure some fairness notion among vendors by ensuring that each product is given a ``fair'' chance, i.e., it is shown at least to a certain number of customers. 
Specifically, we are given a universe of substitutable products and a stream of $T$ customers. For each  customer, we have to offer an assortment from the universe of products. The customer decides to purchase one of these products, or to leave without purchasing any product (no-purchase option). We assume that the choice of the customer is governed by a Multinomial Logit (MNL) choice model. We enforce the constraint that each  product in the universe has to be shown a minimum number of times among the $T$ assortments offered. The minimum display requirement for each product is given exogenously.   Our objective is to maximize the total expected revenue from  the $T$ customers. We refer to this combinatorial optimization problem as {\em Assortment optimization Problem with Visibility} constraints, concisely denoted as \ref{APV}.

A first natural question concerns the complexity of \ref{APV}. In fact, without visibility constraints, the problem reduces to the classic unconstrained revenue maximization problem under MNL, for which we know that the optimal assortment is revenue-ordered (\cite{talluri2004revenue}) and therefore can be solved in polynomial time. However, by enforcing the visibility constraints, we might have to include certain products in the assortment that cannibalize the sales of other more profitable products because of the substitution effect.
Consequently, determining an optimal sequence of assortments for \ref{APV} is not immediately clear. 
Subsequently, a natural follow-up problem concerns the introduction of cardinality constraints on the offered assortments of products. These form of constraints are widely used in the literature \citep{Davis2013AssortmentPU, gallego2014constrained, desir2020constrained}, as they model multiple applications where vendors have limited shelf space, or limited screen size for online vendors.
Finally, an interesting question emerging from the visibility problem is the task of quantifying the revenue loss incurred by enforcing visibility constraints compared to the relaxed unconstrained problem. 
This challenge compels us to develop a pricing strategy that appropriately apportions the loss to different vendors based on the impact of their product on the overall revenue.
Such a scenario frequently occurs within the framework of SLAs. Typically, a contract between a platform and a vendor includes a clause that guarantees a certain level of visibility to the vendor's product. In return for this visibility, the vendor compensates the platform with a fee. 
 %This fee acts as a buff   er against any potential revenue loss that the platform might suffer as a consequence of providing visibility to the vendor's product. Thus, the aim is to determine a fair and proportionate fee that correlates with the revenue loss associated with each vendor's product exposure.

\subsection{Main Contributions and Technical Ideas}
    
In this paper, we introduce and study the assortment  optimization problem with visibility constraints under the Multinomial Logit choice model. Our first goal is to settle the complexity question on the positive side by developing a polynomial time algorithm for \ref{APV}. Next, we consider a natural extension of this problem, where a cardinality constraint on the offered assortment is enforced. We show that this additional constraints makes the problem fundamentally harder, rendering it strongly NP-hard, even in the special case where all the products have identical prices. Additionally, we leverage the NP-Hardness to prove that the existence of a Fully Polynomial Time Approximation Scheme (FPTAS) is precluded. Following this assessment on the hardness of the problem, we design a Polynomial Time Approximation Scheme (PTAS) for the special case of equal prices.
A subsequent goal is to quantify the revenue loss caused by the visibility constraints compared to the unconstrained assortment optimization problem and design a  strategy to  share this loss among the different vendors of the products for which visibility constraints have been enforced.  Our contributions are organized and summarized as follows.


\begin{enumerate}
    \item {\bf Polynomial time algorithm for \ref{APV}}.   
    Our main technical contribution is to design a polynomial time algorithm for \ref{APV}.  We introduce the notion of expanded revenue and expanded set of an assortment, which will play a pivotal role in designing our algorithm. We leverage structural properties of the expanded revenue function to characterize the structure of an optimal solution of \ref{APV}, and consequently design an efficient algorithm to compute it.
    
    %We start by considering the problem of static assortment optimization over $T$ customers when we enforce visibility constraints on the products available. Contrary to the unconstrained case, the introduction of the visibility constraints introduces a coupling between the different customers, because we cannot respect the global visibility constraints if we optimize each assortment independently of the others. Hence, it does not appear obvious whether this problem can be solved in polynomial time or is NP-hard. We prove that \ref{APV} can actually be solved by a polynomial time algorithm, and devise such an algorithm. Our contribution follows this structure:
    \begin{enumerate}
        \item {\bf Expanded Revenue}. In Section \ref{expanded revenue}, we introduce the expanded revenue function. Given a universe of products $\cal N$ and an assortment $A \subseteq {\cal N}$. The expanded revenue of assortment $A$ is defined as the maximum revenue of any assortment in $\cal N$ that contains $A$. The expanded set is the assortment that achieves this maximum revenue. We show that the expanded revenue function is closely related to the objective function of \ref{APV} in the case of a single customer. We provide a linear time algorithm to compute the expanded revenue.         
        %: given a subset $A$ of the $n$ products available, compute the maximal expected revenue of a subset containing $A$. We prove in Lemma \ref{Compute expanded set} that there exists a subset of maximal cardinality that is optimal for this problem, and name it the expanded set of $A$, while the expanded revenue of $A$ is the corresponding expected revenue. In addition, we prove that the expanded set can be computed in linear time with respect to the number $n$ of products.

        
        \item {\bf Monotonicity and supermodularity\footnote{A function $f: {\Omega} \rightarrow \mathbb{R}$ is supermodular if $\forall A, B \in {\Omega}, \; f(A \cup B) + f(A \cap B) \geq f(A) + f(B)$.} of the expanded revenue}. We show that the expanded revenue, as defined above, possesses some useful properties. Namely, we prove in Lemma \ref{monotonicity} a monotonicity property, i.e., we show that the expanded revenue of an assortment decreases as the assortment gets large. Then, in Lemma \ref{supermodularity}, we prove the main theoretical result on which our final algorithm relies: the supermodularity of the expanded revenue function.
        \item {\bf Our algorithm and LP formulation}. Building on the previous properties of the expanded revenue,  we finally identify in Theorem \ref{Solution structure} a very simple nested structure for an optimal solution of \ref{APV}, and devise a polynomial time algorithm to efficiently solve the problem. Additionally, we demonstrate in Theorem \ref{thm:LPformulation} that \ref{APV} can be formulated as a compact  linear program.
    \end{enumerate}


\item {\bf \ref{APV} with cardinality constraints.} We consider in Section \ref{sec:APVC} the natural extension of \ref{APV} where there is an upper bound on the number of products that we can display in each assortment.
\begin{enumerate}
    \item {\bf Hardness.} In Theorem \ref{NP-hardness}, we prove that \ref{APV}  with cardinality constraints is strongly NP-hard, even in the case of equal prices, by linking its resolution to the $3$-\texttt{PARTITION} decision problem. Moreover, we extend our proof to show that the existence of an FPTAS is precluded for this problem, unless $P=NP$.
    \item {\bf Polynomial-time approximation scheme (PTAS).} Our cornerstone algorithmic result in the case of equal prices is the design of a polynomial-time approximation scheme (PTAS), which for any fixed desired precision, returns a sequence of assortments within that degree of precision, in polynomial time. Since the strong NP-hardness rules out the existence of a Fully Polynomial Time Approximation Scheme (FPTAS), unless $P=NP$, a PTAS is the best approximation algorithm we can achieve. Our PTAS relies on a linearization of the objective function through the guessing of a carefully chosen set of parameters of the problem. Then, we solve the relaxed linear program, and we leverage its solution to compute a random approximate (integer) solution for the linearized problem, using a dependent rounding scheme. This rounded solution is subsequently shown to be, in expectation, near optimal for \ref{APV} with cardinality constraints.
\end{enumerate}



\item {\bf Price of visibility.} The introduction of visibility constraints results in a reduction in the total expected revenue compared to the unconstrained version of the assortment problem. Therefore, we aim to evaluate this revenue loss and propose a fair strategy for distributing it among vendors based on their respective contributions to the loss.
\begin{enumerate}
\item {\bf Pricing the loss.} We devise in Section \ref{subsection:share},  a pricing strategy as follows: For each product that negatively impacts the overall revenue, we charge the vendor a fraction of the loss proportional to the ratio between the negative contribution of the product and the sum of the negative contributions of all products. We demonstrate that this strategy satisfies natural fairness properties and exhibits favorable computational tractability.
\item {\bf Numerical experiments.} Additionally, we conduct in Section \ref{subsection:numeric} some numerical experiments to illustrate our findings. We analyze the influence of visibility constraints on expected revenue and sales, examine the individual effect of a single product's visibility constraint, and explore the trade-off between revenue and fees. Moreover, we show how our pricing strategy accurately captures the revenue loss attributed to the products made visible.
\end{enumerate}

\end{enumerate}
        



    
    
\subsection{Related Literature}

Assortment optimization under the Multinomial Logit (MNL) model is a well-established problem in scientific literature. Initially introduced by \cite{luce1959individual}, with subsequent works by \cite{McFadden1972ConditionalLA} and \cite{RePEc:ecm:emetrp:v:52:y:1984:i:5:p:1219-40}, the MNL choice model has gained popularity for modeling customer choices due to its simplicity in computing the choice probabilities, its predictive power and its computational tractability compared to more complex choice models. It has been extensively used in various research works  such as \cite{mahajan2001stocking,talluri2004revenue,el2021joint,Sumida2020RevenueUtilityTI,gao2021assortment,housni2023maximum}, to mention a few. 
The MNL model proves particularly useful in assortment optimization, as demonstrated by \cite{talluri2004revenue}, who showed that under the MNL model, the optimal assortment in the unconstrained setting is revenue-ordered. This means that it contains all products whose revenues exceed a certain threshold, simplifying the optimization problem by avoiding the consideration of exponentially numerous potential subsets. Moreover, \cite{Gallego2011AGA} give a linear programming formulation for the unconstrained assortment problem under MNL. \cite{Rusmevichientong2010DynamicAO} solved the version of the problem with a cardinality constraint, proving it is still solvable in polynomial time, and \cite{Dsir2014NearOptimalAF} studied more general capacity constraints, showing it is NP-hard to solve in the general case. \cite{Sumida2020RevenueUtilityTI} and \cite{Davis2013AssortmentPU} studied totally unimodular constraint structures for the assortment and showed that the resulting problem can be reformulated as a linear program. 
However, when considering mixtures of MNL models (MMNL), the assortment optimization problem becomes NP-hard even in the unconstrained setting with two classes of customers as shown in \cite{rusmevichientong2014assortment}. 




%The MNL model proves particularly useful in assortment optimization, as demonstrated by Talluri and van Ryzin (2004), who showed that under the MNL model, the optimal assortment is revenue ordered. This means that it contains all products whose revenue exceeds a certain threshold, simplifying the optimization problem by avoiding the consideration of exponentially numerous potential subsets.



    
%Assortment optimization under the Multinonmial Logit model is a well anchored problem in the scientific literature. First introduced in \cite{luce1959individual}, the Multinomial Logit choice model has become very popular to model customer choices because of its tractability, contrary to more complex models. For instance, mixtures of MNL models (MMNL) become NP-hard for the assortment optimization problem as soon as there are two classes of customers . The MNL model is particularly useful in assortment optimization, where \cite{talluri2004revenue} showed that under the MNL model, the optimal assortment is revenue ordered, which means it contains all the products whose revenue is greater than some threshold. This makes the optimization problem easy to solve in polynomial time, without having to consider the exponential number of potential subsets to offer. Assortment optimization is a significant field of research, and several constrained variations of the MNL optimal assortment have been studied. \cite{Rusmevichientong2010DynamicAO} solved the version of the problem with cardinality constraints, proving it is still solvable in polynomial time, and \cite{Dsir2014NearOptimalAF} studied more general capacity constraints, showing it is NP-hard to solve in the general case. \cite{Sumida2020RevenueUtilityTI} and \cite{Davis2013AssortmentPU} studied totally unimodular constraint structures for the assortment and showed that the resulting problem can be reformulated as a linear program. 





To the best of our knowledge, this paper is the first to study assortment optimization under MNL with visibility constraints.  The topic of visibility in assortment planning has barely been covered: \cite{chen2022fair} studied visibility under a fairness approach, trying to enforce similar visibility for products with similar characteristics, while \cite{Wang2021WhenAM} studied a version of the assortment optimization problem in which they can increase the attractiveness of some products through an advertising budget. %Nevertheless, to the best of our knowledge, visibility constraints on a minimum number of appearances have never been studied in the literature so far. 
Recently, in a concurrent work, \cite{lu2023simple} considers an assortment optimization problem under the MNL model, subject to some fairness constraints similar to the visibility constraints presented in our paper. In contrast to our model, which is deterministic and involves a predetermined number of customers, their study investigates a probabilistic version with a single customer, where random assortments can be offered. This probabilistic version is less complex than our framework. For instance, we establish that our model with cardinality constraints is strongly NP-hard and does not admit an FPTAS, whereas their probabilistic model does admit an FPTAS under cardinality constraints. Hence, the need for different optimization techniques and approaches.




In addition, the topic of assortment optimization for a stream of customers is more often studied from an online perspective, where decisions are made sequentially such as in \cite{avis2015AssortmentOO} and \cite{Cheung2017ThompsonSF}. In contrast, we study a static version of the problem, where we plan the entirety of our assortments in advance. Other versions of the problem, such as in  \cite{Li2009ASA}, consider a flow of customers with randomized preferences, to which we offer a common assortment. Finally, in revenue management, pricing problems are often considered in the sense of optimizing the selling price of each product \citep{Wang2012CapacitatedAA,Miao2018DynamicJA, Alptekinolu2015TheEC}, while selling prices are fixed for our problem,  and we instead study the question of pricing the loss generated by enforcing visibility of each product.

    
   \vspace{3mm}
    \underline{{\em Outline.}} The remainder of the paper is organized as follows. We first introduce the mathematical framework and define our problem \ref{APV} in Section \ref{sec:form}. Then, in Section \ref{sec:apv}, we devise a polynomial time algorithm for the \ref{APV} problem. In Section \ref{sec:APVC}, we consider \ref{APV} with cardinality constraints. We show that adding these constraints makes the problem strongly NP-hard even in a setting with identical prices, then we provide a PTAS for the latter setting. In Section~\ref{sec:price}, we propose a pricing strategy to charge the revenue loss to the vendors proportionally to their contribution, and illustrate our results numerically. Finally, in Section \ref{sec:conclusions}, we draw some conclusions and outline future work. 
        

\section{Introduction}\label{sec:intro}
Assortment optimization is a crucial aspect of  decision making in many industries  such as  e-retail and online advertising. In this domain, our goal is to select a subset of available products  to offer to customers in order to maximize a context-appropriate objective function, %In this context, the objective function might vary depending on the business's priorities, 
such as revenue, profit, or market share. For example, e-retailers seek to select which products should be displayed to customers in order to maximize their expected revenue. Online  advertisers strategically select the most effective combination of advertisements to maximize user engagement and desired outcomes, such as click-through rates. 
Consequently, the choice of a well formed assortment is crucial due to inherent substitution effects, where a product's attractiveness depends not only on its intrinsic value but also on the concurrent alternatives presented at that time. For instance, offering a high-quality, high-priced product alongside a comparable product at a significantly lower price may result in diminishing sales for the higher-priced product, leading in turn to an unsatisfactory platform revenue. For a comprehensive overview of assortment optimization and its applications, we refer avid readers to 
related surveys and books \citep{kok2015assortment, Gallego2019RevenueMA}.




Traditionally, assortment optimization frameworks often overlook a crucial element in contemporary e-commerce, which is the intended exposure of specific products to customers, which we refer to as product {\em visibility}. In today's complex business landscape, where companies adhere to Service-Level Agreements (SLAs) with suppliers and prioritize sponsored promotions, product visibility within our assortment of choice is pivotal. 
SLAs  often define conditions for product representation, ensuring equitable visibility for each supplier's products on the platform. Moreover, the concept of sponsored products has gained traction in recent years, with brands willing to pay for prominent display and increased visibility. %\citep{amazonads}
While these strategies influence consumer behavior in various ways,  solely focusing on products visibility without considering the broader context of assortment optimization can lead to an imbalanced product mix, resulting in reduced customer satisfaction and overall revenue.

In this paper, we introduce the notion of {\em visibility constraints} into the classical framework of assortment optimization. In essence, our purpose is to enforce a minimum level of display, \black{asking each product to be shown} at least a certain number of times in the displayed assortments. Such constraints model both Service-Level Agreements and sponsored products. In addition, they can capture settings where the platform wishes to instill some fairness notion among vendors by ensuring that each product is given a ``fair'' chance, in the sense of being shown to at least a certain number of customers. Deferring the formal model formulation to Section \ref{sec:form}, we now provide a brief informal introduction to our model. Specifically, we are given a universe of substitutable products along with a stream of $T$ customers. For each  customer, we determine an individual assortment from the universe of products. Subsequently, the customer either decides to purchase one of these products, or chooses to leave without purchasing any product. We assume that these decisions are governed by the well-known Multinomial Logit (MNL) choice model. That is, what differentiates our work from classical models is enforcing the constraint that each  product has to be shown a minimum number of times among the $T$ offered assortments noting that this minimum display requirement is given exogenously.   Our objective is to maximize the combined expected revenue collected across the stream of customers. We refer to this newly-introduced problem as {\em assortment optimization with visibility} constraints, concisely denoted as \ref{APV}.

\black{Alternatively to our model, one could also consider a randomized approach, where the goal is to determine a distribution over assortments rather than an exact collection, as explored in the concurrent work of Lu et al. (2024). While both models have merit, focusing on a stream of customers rather than a distribution over assortments offers key advantages. First, many advertising platforms, such as Google and Facebook , employ cost-per-mille (CPM) pricing, where advertisers pay for a guaranteed level of visibility based on the number of product displays. Explicitly modeling a customer stream ensures that these visibility commitments are met, whereas a distribution-based approach does not inherently guarantee the required displays. As a result, a correction phase may be necessary at the end of the time horizon to satisfy visibility constraints. Second, in our setting, computing a deterministic solution is a more natural choice. Introducing randomness is often justified when a deterministic approach is either intractable or leads to unfavorable properties. However, our model turns out to exhibit favorable structural and algorithmic properties, as presented throughout the paper.}

From a computational standpoint, \black{it is only natural to wonder about} the complexity of including visibility constraints. In fact, without this requirement, the problem in question reduces to the classic MNL-driven assortment optimization problem, for which we know that the optimal offer set is price-ordered (\cite{talluri2004revenue}), and can therefore be computed in polynomial time. However, by enforcing visibility constraints, we necessarily have to include certain products along the sequence of displayed assortments, possibly cannibalizing the sales of more profitable products due to substitution effects.
Consequently, determining an optimal sequence of assortments  appears to be a challenging question. 
Subsequently, a practically motivated setting concerns the introduction of cardinality constraints on the offered assortments\black{, which impose an upper bound on the number of products included in the assortments}. Such constraints have extensively been studied in earlier literature \citep{Sumida2020RevenueUtilityTI, gallego2014constrained, desir2020constrained, kunnumkal2023new, leitner2024exact}, as they allow us to capture multiple applications where vendors have limited shelf space, screen size, or purchasing budget.
Finally, a fundamental question is that of quantifying the revenue loss incurred by enforcing visibility constraints compared to their relaxed unconstrained counterpart. 
This challenge compels us to develop a pricing strategy that appropriately apportions the loss to different vendors based on the impact of their product on the overall revenue.
Such a scenario frequently occurs within the framework of SLAs, where typically, vendor-platform contracts include a clause that guarantees a certain level of visibility to the vendor's product. In return for this visibility, the vendor compensates the platform with an appropriate fee. \black{In general, advertising companies such as Google, Facebook or Amazon employ various pricing schemes. For instance, two of the most common are {\em cost-per-mille} (CPM) and {\em cost-per-clic} (CPC); see e.g., \cite{asdemir2012pricing}. In a nutshell, the former pricing model charges a fee for a fixed number of ad impressions, typically per thousand, while the latter model charges a fee per interaction with the ad, usually a click. Our work aligns with the CPM framework, as we enforce visibility constraints that require a minimum number of displays for each product.
}
 %This fee acts as a buff   er against any potential revenue loss that the platform might suffer as a consequence of providing visibility to the vendor's product. Thus, the aim is to determine a fair and proportionate fee that correlates with the revenue loss associated with each vendor's product exposure.

\subsection{Main Contributions and Technical Ideas}\label{subsec:contributions}
    
In this paper, we introduce and study  assortment  optimization with visibility constraints \eqref{APV} under the Multinomial Logit choice model. Our first contribution resides in settling the complexity status of this problem, by developing \black{an exact} polynomial time algorithm. Next, we consider a natural extension, imposing a cardinality constraint on the offered assortment\black{, which limits its allowable number of products}. We show that this additional requirement makes the problem fundamentally harder, rendering it strongly NP-hard, even when all products have identical prices. This finding excludes the possibility of devising a fully polynomial time approximation Scheme (FPTAS), unless $\mathrm{P}=\mathrm{NP}$. On the positive side, we design a \black{randomized} polynomial time approximation scheme (PTAS) for the setting of equal prices.
Our subsequent goal is to quantify the revenue loss brought by visibility constraints, compared to the unconstrained assortment optimization problem, and to design a  strategy for  sharing this loss among the different vendors for which visibility constraints have been enforced.  Our main contributions can be briefly summarized as follows.


\begin{enumerate}
    \item {\bf Exact polynomial time algorithm}. In Section \ref{sec:apv}, our main technical contribution is to design an exact \black{linear} time algorithm for \ref{APV}. To this end, we introduce the notion of expanded revenue and expanded set of a given assortment, which will play a pivotal role in designing our algorithm. We leverage structural properties of the expanded revenue function to characterize the structure of optimal solutions, enabling us to develop an efficient algorithmic approach.
    
    %We start by considering the problem of static assortment optimization over $T$ customers when we enforce visibility constraints on the products available. Contrary to the unconstrained case, the introduction of the visibility constraints introduces a coupling between the different customers, because we cannot respect the global visibility constraints if we optimize each assortment independently of the others. Hence, it does not appear obvious whether this problem can be solved in polynomial time or is NP-hard. We prove that \ref{APV} can actually be solved by a polynomial time algorithm, and devise such an algorithm. Our contribution follows this structure:
    \begin{enumerate}
        \item {\bf Expanded Revenue}. \black{We define the so-called expanded revenue function as follows:} Given a universe of products $\cal N$ and an assortment $A \subseteq {\cal N}$, the expanded revenue of $A$ is defined as the maximum revenue of any assortment that contains $A$. In turn, the expanded set of $A$ is the assortment that achieves the latter revenue. We show that the expanded revenue function is closely related to the objective function of \ref{APV} for a single customer, and provide a linear time algorithm to compute the expanded revenue.         
        %: given a subset $A$ of the $n$ products available, compute the maximal expected revenue of a subset containing $A$. We prove in Lemma \ref{Compute expanded set} that there exists a subset of maximal cardinality that is optimal for this problem, and name it the expanded set of $A$, while the expanded revenue of $A$ is the corresponding expected revenue. In addition, we prove that the expanded set can be computed in linear time with respect to the number $n$ of products.

        
        \item {\bf Monotonicity and supermodularity}. We prove that the expanded revenue function possesses several useful properties. Namely, we establish a monotonicity property, showing that the expanded revenue of an assortment decreases with the addition of new products. Then, we prove the main theoretical result on which our final algorithm relies: supermodularity\footnote{
        % A function $f: {\Omega} \rightarrow \mathbb{R}$ is supermodular if for every $A, B \in {\Omega}, \; f(A \cup B) + f(A \cap B) \geq f(A) + f(B)$.
        \black{A function $f\colon \Omega\rightarrow \R$ is supermodular if $f(B\cup\{i\})-f(B) \geq f(A\cup\{i\}) - f(A)$ for all $A\subseteq B\subseteq \Omega$ and $i\in \Omega\setminus B$, .}} of the expanded revenue function.
        
        \item {\bf Combinatorial algorithm and LP formulation}. Building on the above-mentioned properties, we finally identify a simple nested structure for  optimal \ref{APV} solutions, and devise a polynomial time algorithm to efficiently compute such assortments. Additionally, we demonstrate that \ref{APV} can be formulated as a compact  linear program, thereby proposing an alternative algorithmic approach.
    \end{enumerate}


\item {\bf \ref{APV} with cardinality constraints.} In section \ref{sec:APVC}, we consider the practically-motivated extension of \ref{APV}, where there is an upper bound on the number of products that can be displayed in each assortment.
\begin{enumerate}
    \item {\bf Hardness.} First, we prove that \ref{APV}  with cardinality constraints is strongly NP-hard, even with equal prices, by linking its resolution to the $3$-\texttt{PARTITION} problem. Moreover, we extend our proof to rule out the existence of an FPTAS, unless $\mathrm{P}=\mathrm{NP}$.
    
    \item {\bf Polynomial-time approximation scheme (PTAS).} Our cornerstone algorithmic result for the equal price setting is the design of a \black{randomized} polynomial-time approximation scheme (PTAS), allowing us to approach optimal revenues within any degree of precision. \black{Specifically, for any $\eps>0$, our approximation scheme outputs a random assortment whose performance guarantee is, in expectation, within $1-\eps$ of optimal}. Our PTAS relies on a linearization of the objective function through the efficient guessing of a carefully chosen set of parameters. Then, we formulate a linear relaxation, and leverage its fractional solution to compute a random approximate (integer) solution, using a dependent rounding scheme. The latter solution is subsequently shown to be, in expectation, near optimal for \ref{APV} with cardinality constraints. \black{Some of these ideas are inspired by the work of  \cite{segev2021efficient} on rounding fractional} solutions for the Santa Claus problem.
\end{enumerate}



\item {\bf Price of visibility.} Clearly, the introduction of visibility constraints may possibly reduce the \black{optimal} expected revenue compared to the unconstrained assortment optimization problem. In Section \ref{sec:price}, we aim to quantify this revenue loss and to propose a fair strategy for distributing it among vendors based on their respective contributions to the overall loss. We devise a pricing strategy that, for each product which negatively impacts the overall revenue, charges the vendor a fraction of the loss \black{that is} proportional to the ratio between the negative contribution of the product and the sum of the negative contributions of all products. We demonstrate that this strategy satisfies natural fairness properties and exhibits favorable computational tractability.
\end{enumerate}
        



    
    
\subsection{Related Literature}

Assortment optimization under the Multinomial Logit (MNL) model has been a well-established research direction in revenue management. Initially introduced by \cite{luce1959individual}, with subsequent works by \cite{mcfadden1973conditional} and \cite{RePEc:ecm:emetrp:v:52:y:1984:i:5:p:1219-40}, the MNL  model has gained a great deal of popularity for modeling customer choices due to its structural simplicity, predictive power, ease of estimation and computational tractability compared to more complex alternatives. It has been extensively utilized in various research works  such as those of \cite{mahajan2001stocking}, \cite{talluri2004revenue}, \cite{ Rusmevichientong2010DynamicAO}, \cite{Sumida2020RevenueUtilityTI}, \cite{gao2021assortment}, \cite{housni2023maximum} and \cite{el2023joint}, to mention a few. As demonstrated by \cite{talluri2004revenue}, under the MNL model, the optimal unconstrained assortment is revenue-ordered, namely, it contains all products whose prices exceed a certain threshold, simplifying the optimization problem by avoiding the consideration of exponentially-many subsets. Moreover, \cite{Gallego2011AGA} proposed a linear programming formulation for this problem. \cite{Rusmevichientong2010DynamicAO} seem to have been the first to consider cardinality constraints, proving this setting to be solvable in polynomial time. \cite{desir2022capacitated} studied more general capacity constraints, showing they are NP-hard to address in the general case. \cite{Sumida2020RevenueUtilityTI} studied totally unimodular constraint structures in this context, and showed that the resulting problem can be reformulated as a linear program. 
However, when considering mixtures of MNL models (MMNL), the assortment optimization problem becomes NP-hard even with two classes of customers \citep{rusmevichientong2014assortment}, and $O(n^{1-\eps})$-hard to approximate in general \citep{desir2022capacitated}.




%The MNL model proves particularly useful in assortment optimization, as demonstrated by Talluri and van Ryzin (2004), who showed that under the MNL model, the optimal assortment is revenue ordered. This means that it contains all products whose revenue exceeds a certain threshold, simplifying the optimization problem by avoiding the consideration of exponentially numerous potential subsets.



    
%Assortment optimization under the Multinonmial Logit model is a well anchored problem in the scientific literature. First introduced in \cite{luce1959individual}, the Multinomial Logit choice model has become very popular to model customer choices because of its tractability, contrary to more complex models. For instance, mixtures of MNL models (MMNL) become NP-hard for the assortment optimization problem as soon as there are two classes of customers . The MNL model is particularly useful in assortment optimization, where \cite{talluri2004revenue} showed that under the MNL model, the optimal assortment is revenue ordered, which means it contains all the products whose revenue is greater than some threshold. This makes the optimization problem easy to solve in polynomial time, without having to consider the exponential number of potential subsets to offer. Assortment optimization is a significant field of research, and several constrained variations of the MNL optimal assortment have been studied. \cite{Rusmevichientong2010DynamicAO} solved the version of the problem with cardinality constraints, proving it is still solvable in polynomial time, and \cite{Dsir2014NearOptimalAF} studied more general capacity constraints, showing it is NP-hard to solve in the general case. \cite{Sumida2020RevenueUtilityTI} and \cite{Davis2013AssortmentPU} studied totally unimodular constraint structures for the assortment and showed that the resulting problem can be reformulated as a linear program. 





To the best of our knowledge, our paper is the first to study assortment optimization under visibility constraints.  The topic of visibility in assortment planning has been largely overlooked thus far: \cite{chen2022fair} studied visibility under a fairness approach, trying to enforce similar visibility for products with similar characteristics, while \cite{wang2023advertising} studied a version in which they can increase the attractiveness of some products through an advertising budget. %Nevertheless, to the best of our knowledge, visibility constraints on a minimum number of appearances have never been studied in the literature so far. 
Very recently, in a concurrent work, \cite{lu2023simple} considered an assortment optimization problem under the MNL model, subject to  fairness constraints similar to the visibility constraints presented in our paper. However, in contrast to our model, where individual assortments are offered to a predetermined number of customers, their study investigates a {\em single-customer} version, where random assortments can be offered. The latter version appears to be not as complex. For instance, as mentioned in Section \ref{subsec:contributions}, we establish that our model with cardinality constraints is strongly NP-hard and does not admit an FPTAS, whereas their model admits an FPTAS under cardinality constraints. Hence, different optimization techniques and approaches appear to be needed.




Interestingly, the topic of assortment optimization for a stream of customers has typically been studied from an online perspective, where decisions are made sequentially; see, e.g., \cite{golrezaei2014real}
% , \cite{avis2015AssortmentOO}, \cite{Cheung2017ThompsonSF}, \cite{gong2022online}, \cite{miao2022online}
and \cite{aouad2023online}. \black{In contrast, we study a non-adaptive setting, where the entirety of our assortments is planned in advance, that is, the collection of assortments in decided offline}. Other existing versions, such as those studied by  \cite{Li2009ASA}, consider a flow of customers with randomized preferences, to which we offer a common assortment. Finally, in revenue management, pricing problems are often considered in the sense of optimizing the selling price of each product \citep{Wang2012CapacitatedAA,Miao2018DynamicJA, Alptekinolu2015TheEC}, while we focus on the scenario where selling prices are fixed,  and instead, we study the question of pricing the revenue loss generated by enforcing visibility of each product.











%\section{Model Formulation}\label{sec:form}
    




{\bf The MNL choice model.} Let  $\mathcal{N} \coloneqq \{1,\ldots, n\}$ be a universe of substitutable products at our disposal. 
 Each product $i \in {\cal N}$ has a  price $p_i \geq 0$.  Without loss of generality, we order the products by non-increasing prices, i.e., $p_1 \geq p_2 \geq \ldots \geq p_n$.  
An assortment of products or an offer set, is simply a subset of products $S \subseteq {\cal N}$. Additionally, the option of not selecting any product is symbolically represented as product $0$, and referred to it as the no-purchase option.  



We assume that customers make choices according to a Multinomial Logit model. Under this model, each product $i \in {\cal N}$ is associated with a preference weight $v_i >0$. Note that $v_i $ captures the attractiveness of product $i$, meaning a  high  preference weight indicates a high popularity. Without loss of generality, we use the standard convention that the no-purchase preference weight is normalized to $v_0=1$. 
We use the notation $V(S) \coloneqq \sum_{i \in S} v_i$, which is the total weight of a subset $S \subseteq \mathcal{N}$.
Under the MNL model, if we offer  an assortment $S \subseteq {\cal N}$, the customer chooses product $i$ with  probability 
 $$\phi(i, S) \coloneqq \frac{v_i}{1 + V(S)}.$$
 We refer to $\phi(i,S)$ as the choice probability of product $i$ given assortment $S$. 
 Alternatively, the customer may decide to not purchase any product, which happens with the complementary probability 
  $$\phi(0, S) \coloneqq \frac{1}{1 + V(S)}.$$
Let $R(S)$ be the expected  revenue we get from a customer if we offer assortment $S$. In particular, we have 

$$R(S) \coloneqq   \sum_{i \in S} p_i \phi(i,S) = \frac{\sum_{i \in S} p_i v_i}{1 + \sum_{i \in S} v_i}.$$ 

 






%At each  time period, a customer arrives and we need to offer them an assortment of products $S \subseteq \mathcal{N}$. The customer then decides to purchase a single product, or to leave without purchasing anything. Customers make choices according to a choice model $\phi$, i.e., $\phi(i,S)$ is the choice probability of product $i$ given an offered assortment $S$. 




%We can now define $$R(S) \coloneqq \frac{\sum_{i \in S} p_i v_i}{1 + \sum_{i \in S} v_i}$$ which corresponds to the expected revenue when we offer the set of products $S$ under the MNL model. 


\noindent
{\bf Assortment Optimization with Visibility constraints.} 
We are presented with a finite stream of $T$ customers. Each  customer $t$  will be offered an assortment $S_t$. Customers make choices according to the same MNL model, i.e., a customer decides to purchase product $i$ from assortment $S_t$ with a probability $\phi(i, S_t)$, or they may choose the no-purchase option with  probability $\phi(0, S_t)$. The expected revenue we obtain from customer $t$ is $R(S_t)$.
To ensure visibility, we impose constraints that require each product $i \in \mathcal{N}$ to be shown to at least $\ell_i$ customers. Note that the parameters $\ell_i$ are exogenous and satisfy $\ell_i \in  \{0,\ldots, T\} $ for all $i \in \mathcal{N}$.
Our objective is to determine the assortment $S_t$ to offer to each customer $t$ in order to maximize the total expected revenue while satisfying the visibility constraints. We refer to this problem as the {\em Assortment optimization Problem with Visibility constraints}  (\ref{APV}). It can be formulated as follows:




\begin{equation}
\label{APV}
\begin{aligned}
 \max_{ S_1, \ldots, S_T \subseteq \mathcal{N}}  & \; \;      \sum_{t=1}^T  R(S_t)   \\  
  s.t. \;\;   & \;\;  \sum_{t=1}^T  \mathbbm{1}(i \in S_t) \geq \ell_i, \;\;\; \forall i \in \mathcal{N}.
\end{aligned}
\tag{\sf{APV}}
\end{equation}




\section{Model Formulation}\label{sec:form}
    




{\bf The MNL choice model.} Let  $\mathcal{N} \coloneqq \{1,\ldots, n\}$ be a universe of substitutable products at our disposal, where each product $i \in {\cal N}$ has a  price $p_i \geq 0$.  Without loss of generality, we index these products by non-increasing prices, i.e., $p_1 \geq \cdots \geq p_n$.  
An assortment of products (or an offer set), is simply a subset of products $S \subseteq {\cal N}$. Additionally, the option of not selecting any product is symbolically represented as product $0$, referred to as the no-purchase option.  



We assume that customers make their purchasing decisions according to a Multinomial Logit model. Under this model, each product $i \in {\cal N}$ is associated with a so-called preference weight $v_i >0$. This parameter captures the attractiveness of product $i$, meaning that a high  preference weight indicates a high popularity. Without loss of generality, we assume by convention that the no-purchase preference weight is normalized to $v_0=1$, and we define the total weight of an assortment $S\subseteq \Nc$ as $V(S) \coloneqq \sum_{i \in S} v_i$.
Under the MNL model, if we offer  an assortment $S \subseteq {\cal N}$ to a given customer, she chooses each product $i\in S$ with  probability 
 $$\phi(i, S) \coloneqq \frac{v_i}{1 + V(S)},$$
 \black{with convention that $\phi(i, S) = 0$ for all $i\in \Nc\setminus S$.}
 We refer to $\phi(i,S)$ as the choice probability of product $i$ given the assortment $S$. 
 Alternatively, this customer may decide to not purchase any product, which happens with the complementary probability 
  $$\phi(0, S) \coloneqq \frac{1}{1 + V(S)}.$$
Letting $R(S)$ be the expected  revenue obtained by offering the assortment $S$, we have 

$$R(S) \coloneqq   \sum_{i \in S} p_i \phi(i,S) = \frac{\sum_{i \in S} p_i v_i}{1 + \sum_{i \in S} v_i}.$$ 

 






%At each  time period, a customer arrives and we need to offer them an assortment of products $S \subseteq \mathcal{N}$. The customer then decides to purchase a single product, or to leave without purchasing anything. Customers make choices according to a choice model $\phi$, i.e., $\phi(i,S)$ is the choice probability of product $i$ given an offered assortment $S$. 




%We can now define $$R(S) \coloneqq \frac{\sum_{i \in S} p_i v_i}{1 + \sum_{i \in S} v_i}$$ which corresponds to the expected revenue when we offer the set of products $S$ under the MNL model. 


\noindent
{\bf Assortment optimization with visibility constraints.} 
Given a finite stream of $T$ customers, our solution concept consists of offering an assortment $S_t$ to each customer $t\in [T]$. As previously explained, customers make their purchasing choices according to the same MNL model, meaning that each customer independently decides to purchase product $i\in S_t$ with probability $\phi(i, S_t)$, or chooses the no-purchase option, with  probability $\phi(0, S_t)$. As such, the expected revenue we obtain from customer $t$ is $R(S_t)$.
To ensure visibility, we impose constraints that require each product $i \in \mathcal{N}$ to be shown to at least $\ell_i$ customers, assuming that the parameters $\{\ell_i\}_{i\in \Nc}$ are exogenous, with $\ell_i \leq T$.
Our objective is to determine, \black{in an offline fashion}, a collection of assortments $S_1, \ldots, S_T$ in order to maximize the total expected revenue while satisfying these visibility constraints. We refer to this problem as the {\em assortment optimization problem with visibility constraints}  (\ref{APV}), succinctly formulated as follows:




\begin{equation}
\label{APV}
\begin{aligned}
 \max_{ S_1, \ldots, S_T \subseteq \mathcal{N}}  & \; \;      \sum_{t=1}^T  R(S_t)   \\  
  \mathrm{s.t.} \;\;   & \;\;  \sum_{t=1}^T  \mathbbm{1}(i \in S_t) \geq \ell_i, \;\;\; \forall i \in \mathcal{N}
\end{aligned}
\tag{\sf{APV}}
\end{equation}





%
    \section{Polynomial Time Algorithm for \ref{APV}} \label{sec:apv}




The primary contribution of this paper is the development of a polynomial time algorithm for \ref{APV}. To achieve this, we introduce in Section \ref{subsection:expanded} the concepts of the ``Expanded Revenue" and ``Expanded Set" of an assortment. %, which are instrumental for our analysis. 
In Section \ref{subsection:prop}, we present a polynomial time algorithm to compute the expanded set and expanded revenue and demonstrate the monotonicity and supermodularity of the expanded revenue function. Leveraging these properties, we characterize the structure of an optimal solution for \ref{APV} and present an algorithm that computes it in $O(n + T)$ time (Section \ref{subsection:algo}). Finally, in Section \ref{subsectin:lp}, we demonstrate that \ref{APV} can be formulated as a compact linear program.



% In this section, the main result we will prove is the following:

% There exists a nested optimal solution to \ref{APV}, in which every assortment offered is included in the previous one. This solution can be computed in time $\mathcal{O}(nT)$.

% \begin{theorem}
%     The problem \ref{APV} can be solved in Polynomial time
% \end{theorem}

% To achieve this, we first introduce the notion of expanded set and expanded revenue, that will be useful for our analysis. We then prove some interesting monotonicity and supermodularity properties on the expanded revenue function. Relying on these properties, we finally identify a simple structure for an optimal solution of problem \ref{APV}, which can be computed in polynomial time.



\subsection{Expanded Revenue and Expanded Set} \label{subsection:expanded}

We begin our analysis by examining \ref{APV} in the context of a single customer. In this particular scenario, the visibility constraints are given such that either $\ell_i=0$ or $\ell_i=1$. Let $A$ denote the subset of all products where $\ell_i=1$. Consequently, \ref{APV} is transformed into the problem of identifying the assortment that maximizes revenue while including $A$. This particular problem will serve as  the building block for our analysis, as it lays the foundation for understanding the general case involving  $T$ customers. Thus, it leads us to introduce the subsequent definitions that will aid us in our analysis.

% This lead us to introduce the notion of the expanded revenue,  and the expanded revenue of a set and show how we con compute them. We introduce the following defintiions.



%Because of the visibility constraints, the assortments offered under \ref{APV} will have to enforce certain products. Thus it becomes interesting to study how we can maximize the expected revenue of a set once if we force it to contain certain elements, and we optimize over the remaining ones.

%We therefore naturally introduce the following notations:


\begin{definition}[\bf Expanded revenue]
    \label{expanded revenue}
    Let $A \subseteq \mathcal{N}$. The expanded revenue of $A$, denoted as $\overline{R}(A)$, is defined as the maximum expected revenue achieved by any assortment in ${\cal N}$ that contains $A$. In particular, it is given by

    \begin{equation} \label{eq:exrev}
        \overline{R}(A) \coloneqq \max_{S \subseteq \mathcal{N} , \; A \subseteq S} R(S).
    \end{equation}
\end{definition}

The optimal solution of the maximization problem in \eqref{eq:exrev} is referred to as the expanded set of $A$. In case multiple optimal solutions exist, we break ties by selecting the optimal assortment with the largest cardinality, which we show is unique in Lemma \ref{Compute expanded set}, hence proving that the expanded set is well defined. Formally, we have the following definition. 

\begin{definition}[\bf Expanded set]
    \label{expanded set}
The expanded set of $A$, denoted as $\overline{A}$, is defined as the assortment within $\mathcal{N}$ that  maximizes the expected revenue  among all assortments containing $A$.  If multiple assortments achieve the same maximum expected revenue, $\overline{A}$ is selected as the assortment with the largest cardinality.  Mathematically, $\overline{A}$ is given by
    $$\overline{A} \coloneqq \underset{S \subseteq \mathcal{N} , \; A \subseteq S}{\arg \max}   \left\{ |S| \; : \; R(S)=  \overline R (A)   \right\}.   $$
\end{definition}


% Let $A \subseteq \mathcal{N}$. The expanded set of $A$, denoted as $\overline{A}$, is defined as the assortment within $\mathcal{N}$ that satisfies two criteria: it contains $A$, and it simultaneously achieves the maximum revenue and the maximum cardinality among all possible assortments. Mathematically, $\overline{A}$ is given by:


% Note that Problem \eqref{eq:exrev} is equivalent to \ref{APV} in the scenario where we have a single customer $(T=1)$ and $A$ is defined as the set of products that have to be shown once, i.e., $A= \{ i \in {\cal N} : \ell_i =1 \} $. In that case, $\overline A$ corresponds to the optimal assortment to this problem with the largest cardinality.

% We will use $\overline{R}$ as a set function that takes input an assortment $A \subseteq R$ and returns $\overline{R}(A)$. In the next section we study several properties of this function as well as properties of the expanded set.



Problem \eqref{eq:exrev} can be viewed as equivalent to \ref{APV} when considering a single customer scenario $(T=1)$ and defining $A$ as the set of products that need to be shown once, i.e., $A= \{ i \in {\cal N} : \ell_i =1 \}$. Thus, $\overline A$ represents the optimal assortment, with the largest cardinality, for the problem.

In our  analysis, we consider $\overline{R}$ as a set function that takes an assortment $A \subseteq \mathcal{N}$ as input and returns $\overline{R}(A)$. Note that $\overline{R}(A) = R(\overline{A})$. In the subsequent section, we delve into examining various properties of this function, as well as properties associated with the expanded set.







%$\overline{R}$ is the expanded revenue function, such that $\overline{R}(A) = R(\overline{A})$. When there are several solutions for $\overline{A}$, we define $\overline{A}$ as the one with the highest cardinal. The following lemma justifies that $\overline{A}$ is well defined and can be computed efficiently.

\subsection{Properties of the Expanded Revenue} \label{subsection:prop}

In this section, we first show that we can compute the expanded revenue and the expanded set of a given assortment in polynomial time (Lemma \ref{Compute expanded set}). Then, we show that the expanded revenue is a non-increasing function and  the expanded set is a non-decreasing function (Lemma \ref{monotonicity}). 
Finally, we show that the expanded revenue function is supermodular (Lemma \ref{supermodularity}) which is the most fundamental property for our algorithm design later in the paper. %These  structural properties will be useful in designing our algorithm to solve \ref{APV} for $T \geq 1$.




\vspace{2mm}
\noindent
{\bf Computing the expanded revenue and expanded set.}
Recall without loss of generality that $p_1 \geq \ldots \geq p_n$. We define  an assortment $S$ to be price-ordered if $S=\{1,\ldots,k\}$ for some $1 \leq k \leq n$. Essentially, a price-ordered assortment prioritizes products with high prices. It is worth noting there  are only $n$ possible price-ordered assortments.
Consider an assortment $A \subseteq {\cal N}$, and its expanded set $\overline A$. In the following lemma, we demonstrate that $\overline A$ is the union of $A$ and a price-ordered assortment. Since there are only $n$ possible price-ordered assortments, it is sufficient to compute the expected revenue of the assortments $A \cup \{1,\ldots, k\}$ for each $k \in \{1, \ldots ,n\}$. The expanded set corresponds to the assortment with the highest expected revenue. In the case of multiple assortments with the same maximum revenue, we break ties by selecting the one with the largest cardinality. Thus, the expanded set $\overline A$ can be computed in linear time, specifically $O(n)$. The expanded revenue is simply $\overline{R}(A)= R(\overline A)$. The proof of Lemma \ref{Compute expanded set} leverages some structural properties of the revenue function under MNL that are presented in Appendix \ref{apx1}.

\begin{lemma}
    \label{Compute expanded set}
    For any $ A \subseteq \mathcal{N}$, the expanded set of $A$ is given by  
    $ \overline{A} = A \cup \{i \in \mathcal{N} : p_i \geq \overline{R}(A) \}.$ Furthermore, $\overline{R}(A)$ and $\overline{A}$ can be computed in time $O(n)$.
\end{lemma}
\begin{proof}
    By definition of the expanded set, we have $A \subseteq \overline{A}$. Hence, there exists an assortment $B \subseteq \mathcal{N} \setminus A$, such that $\overline{A} = A \cup B$. Let us show that $$B = \{i \in \mathcal{N} \setminus A   \; : \; p_i \geq \overline{R}(A) \}.$$ 
    \begin{itemize}
        \item {\em Direct inclusion}: Let $i\in B$, and assume by contradiction that $p_i < \overline{R}(A) = R(A \cup B)$. It is known that, under the MNL model, when we add a product $j$ to an assortment $S$, the revenue of this assortment increases if and only of $p_j \geq R(S)$. For completeness, we provide the statement and the proof of this result in Lemma \ref{Revenue variations} in Appendix \ref{apx1}. Using this lemma implies that removing $i$ from $B$ would strictly increase the expected revenue $R(A)$, which contradicts the optimality of $A \cup B$.
        \item {\em Indirect inclusion: } Let $i \in \mathcal{N} \setminus A$ such that $p_i \geq \overline{R}(A)$, and assume by contradiction that $i \notin B$. By Lemma \ref{Revenue variations}, adding $i$ to $B$ would increase the revenue. If this increase is strict, it contradicts the optimality of $A \cup B$. If the revenue stays the same, it contradicts the definition of $\overline{A} = A \cup B$ as the optimal solution with maximum cardinality. 
        \end{itemize}
    Thus,
        $$\overline{A} = A \cup \{i \in \mathcal{N} \setminus A, p_i \geq \overline{R}(A) \}.$$
        Finally, $\overline{A}$ can be computed in time $O(n)$. Indeed, we start from $A$, then we sequentially add elements by decreasing price. At each iteration, we can compute the new revenue from the previous one in constant time by storing the current numerator and denominator, since we only need to add $p_i v_i$ to the former and $v_i$ to the latter when we reach element $i$. Finally, we pick the highest revenue set among the $n$ computed sets.
\end{proof}
    

%Next, we show that the expanded revenue is a non-increasing function and  the expanded set is a non-decreasing function.


\begin{lemma}[\bf Monotonicity]
    \label{monotonicity} 
    If $ A \subseteq B \subseteq \mathcal{N}$, then  $ \overline{A} \subseteq \overline{B}$ and $\overline{R}(A) \geq \overline{R}(B)$.   
\end{lemma}

\begin{proof}
    For $A \subseteq B \subseteq \mathcal{N}$, we have $\{S \subseteq \mathcal{N} \; :\; B \subseteq S \} \subseteq \{S \subseteq \mathcal{N} \; : \; A \subseteq S \}.$ Therefore, every feasible solution for $\max_{S \subseteq \mathcal{N} , \; B \subseteq S} R(S)$ is a feasible solution for $\max_{S \subseteq \mathcal{N} , \; A \subseteq S} R(S)$. Hence, $\overline{R}(A) \geq \overline{R}(B)$. It follows that $\{i \in \mathcal{N}, p_i \geq \overline{R}(A) \} \subseteq \{i \in \mathcal{N}, p_i \geq \overline{R}(B) \},$ and therefore $\overline{A} = A \cup \{i \in \mathcal{N}, p_i \geq \overline{R}(A) \} \subseteq B \cup \{i \in \mathcal{N}, p_i \geq \overline{R}(B) \} = \overline{B}.$
\end{proof}

%Finally, we show that  the expanded revenue function  $\overline R$ is supermodular. This  property  will play  a fundamental role later in our analysis.


\begin{lemma}[\bf Supermodularity]
    \label{supermodularity}
    The expanded revenue function $\overline{R}$ is  supermodular, i.e.,
     $$\forall A, B \subseteq \mathcal{N},  \; 
     \; \overline{R}(A \cup B) + \overline{R}(A \cap B) \geq \overline{R}(A) + \overline{R}(B).$$
\end{lemma}
\begin{proof}[Proof for Lemma \ref{supermodularity}]
In this proof, we use the following alternative definition of supermodularity. A function $f\colon \Omega\rightarrow \R$ is supermodular if and only if for all $A,B\subseteq \Omega$ such that $A\subseteq B$, and each $i\in \Omega\setminus B$, $f(B\cup\{i\})-f(B) \geq f(A\cup\{i\}) - f(A)$.

Following this definition, let $A,B\subseteq \Nc$ such that $A\subseteq B$, and let $i\in \Nc\setminus B$. The proof of this result is separated into two steps. In the first step, we show that\begin{equation}\label{eq:step1}
    R\left(\overline{B}\right) - R\left(\overline{B}\cup \overline{A\cup\{i\}}\right) \leq R\left(\overline{A}\right) - R\left( \overline{A\cup\{i\}}\right). 
\end{equation}
Subsequently, we show in the second step that\begin{equation}\label{eq:step2}
    R\left(\overline{B}\cup \overline{A\cup\{i\}}\right)- R\left(\overline{B\cup \{i\}}\right) \leq 0.
\end{equation}
The result follows directly by summing the inequalities \eqref{eq:step1} and \eqref{eq:step2} term by term.
\paragraph{Step 1.} 
Let us start with the following claim which follows from simple algebra.
\begin{claim}\label{cl:computation}
    For any assortment $S_1, S_2\subseteq \Nc$ such that $S_1\subseteq S_2$, we have\begin{equation*}
        R(S_1) - R(S_2) = \frac{1}{1+V(S_2)}\cdot \sum_{j\in S_2\setminus S_1}(R(S_1) - p_j)\cdot v_j.
    \end{equation*}
\end{claim}
\noindent Using this claim, we have
\begin{equation*}
    R\left(\overline{B}\right) - R\left(\overline{B}\cup \overline{A\cup\{i\}}\right) = \frac{1}{1+V
        \left(\overline{B}\cup \overline{A\cup\{i\}}\right)}\cdot \sum_{j\in \overline{A\cup \{i\}}\setminus \overline{B}} \left(R\left(\overline B\right) - p_j\right)v_j.
\end{equation*}
Next, we know by definition of the extended set of $B$ that $R(\overline{B}) \geq p_j$ for all $j \notin \overline{B}$. Therefore, since $V(\overline{B}\cup \overline{A\cup\{i\}})$ is trivially greater than or equal to $V(\overline{A\cup\{i\}})$, we have,
\begin{align}
    R\left(\overline{B}\right) - R\left(\overline{B}\cup \overline{A\cup\{i\}}\right) &\leq \frac{1}{1+V
        \left(\overline{A\cup\{i\}}\right)}\cdot \sum_{j\in \overline{A\cup \{i\}}\setminus \overline{B}} \left(R\left(\overline B\right) - p_j\right)v_j \notag\\
        & \leq \frac{1}{1+V
        \left(\overline{A\cup\{i\}}\right)}\cdot \sum_{j\in \overline{A\cup \{i\}}\setminus \overline{B}} \left(R\left(\overline A\right) - p_j\right)v_j,\label{eq:smallsum}
\end{align}
where the second inequality follows from Lemma \ref{monotonicity}.
Finally, we have $$\overline{A\cup \{i\}}\setminus \overline{B} \subseteq \overline{A\cup \{i\}}\setminus  \overline{A}.$$
Moreover, for every $j \in \overline{A\cup \{i\}}\setminus  \overline{A}$, $j\notin \overline{A}$, and in particular, $p_j\leq R(\overline{A})$. Therefore, by adding the missing terms to the sum in Equation \eqref{eq:smallsum}, we have
    \begin{align*}
        R\left(\overline{B}\right) - R\left(\overline{B}\cup \overline{A\cup\{i\}}\right) & \leq \frac{1}{1+V
        \left(\overline{A\cup\{i\}}\right)}\cdot \sum_{j\in \overline{A\cup \{i\}}\setminus \overline{A}} \left(R\left(\overline A\right) - p_j\right)v_j\\
        & = R\left(\overline{A}\right) - R\left(\overline{A\cup \{i\}}\right),
    \end{align*}
where the equality follows from Claim \ref{cl:computation}. This concludes the first step.
\paragraph{Step 2. }On one hand, we have $\{i\}\subseteq\overline{A\cup \{i\}}$. Therefore, we have in particular ${\{i\} \subseteq \overline{B}\cup\overline{A\cup \{i\}}}$. On the other hand,  $B\subseteq \overline{B}$ and therefore $B\subseteq \overline{B}\cup\overline{A\cup \{i\}}$. Hence $B\cup \{i\}\subseteq \overline{B}\cup\overline{A\cup \{i\}}$. Recalling that $\overline{B\cup \{i\}}$ is by definition the maximum revenue assortment containing $B\cup \{i\}$, we have$$
    R\left(\overline{B\cup \{i\}}\right) \geq R\left(\overline{B}\cup\overline{A\cup \{i\}}\right),
$$
which concludes the second step, and thereby the proof of the lemma.
% Let $m \coloneqq |\overline{A\cup\{i\}}|$, and let $j_1, \ldots, j_m$ be the elements of $\overline{A\cup\{i\}}$ in the order of decreasing prices, i.e., $p_{j_1}\geq \ldots, p_{j_m}$. Let $C_0 \coloneqq  \overline{A}, D_0 \coloneqq \overline{B}$, and for all $q=0,\ldots,m-1$, $C_{q+1} = C_q\cup \{j_{q+1}\}$ and $D_{q+1} = D_q\cup \{j_{q+1}\}$. Next, we show by induction that for all $q=0,\ldots, m-1$, we have
% \begin{enumerate}
%     \item[(i)] $R(\overline B) - R(D_q) \leq R(\overline A) - R(C_q)$;
%     \item[(ii)] $R(D_{q}) \geq p_{j_{q}}$ and $R(C_{q}) \geq p_{j_{q}}$;
%     \item[(iii)] $R(C_{q})$
% \end{enumerate}

%     \begin{equation*}
%         R(D_{q}) - R(D_{q+1}) \leq R(C_{q}) - R(C_{q+1}).
%     \end{equation*}
\end{proof}
















\subsection{Optimal Solution  for \ref{APV}} \label{subsection:algo}

In this section, we present the main technical result in this paper. In particular, we characterize the structure of an optimal solution of \ref{APV}. Our characterization relies on the supermodularity property of the expanded revenue function. Moreover, we show that we can compute such a solution in $O(n+T)$, which gives us a polynomial time algorithm to solve \ref{APV}.

\vspace{2mm}
\noindent
{\bf Optimal solution.} Consider an instance of \ref{APV}. Recall that for all $i \in {\cal N}$,  $\ell_i$ is the lower bound on the minimum number of customers for which we must offer product $i$. For $t \in \{0,1,\ldots,T \},$ we define the following sets
\begin{equation}
    L_t = \{i \in \mathcal{N}, \ell_i = t \}.
\end{equation}
Our candidate solution for \ref{APV} is given by
\begin{equation} \label{eq:sol}
    {S_t^*} = \overline{\bigcup_{t \leq u \leq T} L_u}, \quad \forall t \in \{1,\ldots,T \}.
\end{equation}


Note that $(L_t)_{0 \leq t \leq T}$ is a partition of $\cal N$. Moreover, since  $ \bigcup_{t+1 \leq u \leq T} L_u  \subseteq  \bigcup_{t \leq u \leq T} L_u$, the monotonicity property in Lemma \ref{monotonicity} implies that $S_{t+1}^* \subseteq S_{t}^*$ for any $t=0,\ldots,T-1$. Therefore, our solution has a nested structure, i.e.,
$ S_T^* \subseteq S_{T-1}^* \ldots \subseteq S_1^*.$ In the following, we prove that the assortments given by \eqref{eq:sol} are optimal for \ref{APV}. Moreover, they can be computed in polynomial time. Indeed, Lemma \ref{Compute expanded set} shows that each of them can be computed in time $O(n)$, so the entire solution can be computed in time $O(nT)$. We can further improve the running time to $O(n+T)$ as shown below.



% \begin{definition}[\bf Nested solution]
%     \label{Nested solution}
%     For each instance of the problem \ref{APV}, we define the partition $(L_t)_{0 \leq t \leq T}$ of $\mathcal{N}$ by: $\forall t \in [\![0, T]\!], \; L_t = \{j \in \mathcal{N}, \ell_j = t \}$ \\
%     We define the sequence of assortments $(S_t^*)_{1 \leq t \leq T} \coloneqq (\overline{\bigcup_{t \leq u \leq T} L_u})_{1 \leq t \leq T}$, that is $S_1^*, S_2^*, \ldots, S_T^* = \overline{L_1 \cup \ldots \cup L_T}, \overline{L_2 \cup \ldots \cup L_T}, \ldots, \overline{L_T}$
% \end{definition}

%We observe that this solution structure is nested: $S_{t+1}^* \subseteq S_t^* \;\; \forall t \in [\![1, T-1]\!] $


\begin{theorem}{}
    \label{Solution structure}
    The  sequence of assortments $(S_t^*)_{1 \leq t \leq T}$ defined in \eqref{eq:sol} is optimal for \ref{APV}. % Moreover, each permutation of these sets is also an optimal solution for \ref{APV}. 
    Moreover, such a solution can be computed in $O(n+T)$ time.
\end{theorem}




\begin{proof}
    We prove the result by induction. First, for $T= 1$, for any $i\in \Nc$, we either have $\ell_i=0$ or $\ell_i=1$. Noting that $L_1$ is the set of products $i$ such that $\ell_i=1$, \ref{APV} reduces to the problem of finding the optimal assortment that contains $L_1$, i.e., $$
        \max_{S\subseteq \Nc\text{ s.t }L_1\subseteq S}R(S),
    $$
    whose solution is $\overline{L_1}$ by definition of the expanded set. The result follows for $T=1$ by noticing that $S_1^* =\overline{L_1}$.

    Let us now prove the result for a general number of customers. Let $T\geq 2$, and assume by induction that the result is true for $T-1$, in other words, given $T-1$ customers, and for any set of visibility constraints, the optimal solution of \ref{APV} is given by the assortments defined in Equation \eqref{eq:sol}. Let us show that the result holds for $T$ customers.
    %The outline of the proof is as follows. At first, we consider an optimal solution, then using the supermodularity property, sequentially modify it into another optimal solution where at least one assortment contains $L_1$, i.e., the assortment of products which must be shown at least once.
    We denote by $A$ the set of all products which must be shown to at least one customer due to the visibility constraints, i.e., $A\coloneqq \bigcup_{t=1}^TL_t$. We start by providing the following crucial intermediary claim.
    \begin{claim}\label{cl:intermediary}
        There exists an optimal solution $S_1, \ldots, S_T$ to \ref{APV} such that $S_1 \supseteq A$. In particular, there exists an optimal solution $S_1, \ldots, S_T$ to \ref{APV} such that $S_1 =S_1^*$.
    \end{claim}
    In other words, Claim \ref{cl:intermediary} states that there exists an optimal solution $S_1,\ldots, S_T$ to \ref{APV} such that $S_1$ contains all the products that must be shown at least once due to the visibility constraints, i.e., products $i$ such that $\ell_i\geq 1$. Consequently, this allows us to focus only on those feasible solutions of \ref{APV}, which offer $S_1^*$ to customer $1$. Maximizing the revenue amongst said solutions thereby guarantees attaining the optimal objective. In the remainder of this proof, we start by showing Claim \ref{cl:intermediary}, before leveraging it to conclude our induction.
    \paragraph{Proof of Claim \ref{cl:intermediary}. }The proof of this result mainly relies on a judicious exploitation of the supermodularity property. Assume by contradiction that there exists no optimal solution of \ref{APV} such that $S_1 \supseteq A$. Let $\hat S_1, \ldots, \hat S_T$ be the optimal solution of \ref{APV} that maximizes $|\hat S_1\cap A|$. In the case of ties, we pick any arbitrary solution that maximizes $|\hat S_1\cap A|$. By the contradiction hypothesis, $A \nsubseteq \hat S_1$. In particular, there exists some product $j \in \Nc$ such that $j\in A$ and $j \notin \hat S_1$. Moreover, since $j\in A$, we know that $\ell_j\geq 1$, and therefore, $j$ must be shown at least to $1$ customer, which means that there exists some $u\in [T]\setminus \{1\}$ such that $j\in \hat S_u$. We define the following new solution to \ref{APV}: $S_1 = \hat S_1\cup \hat S_u$, $S_u = \hat S_1\cap \hat S_u$, and $S_t = \hat S_t$ for all $t\notin \{1,u\}$. First, this newly defined solution is also feasible since any product offered once in either $\hat S_1$ or $\hat S_u$ is also shown in $S_1$, and each product shown in both $\hat S_1$ and $\hat S_u$ is also shown in both $S_1$ and $S_u$. Second, $S_1,\ldots, S_T$ is also an optimal solution. Indeed, by the supermodularity property, we have $$
        R\left(\hat S_1\cup \hat S_u\right)+ R\left(\hat S_1\cap \hat S_u\right) \geq R\left(\hat S_1\right)+ R\left(\hat S_u\right).
    $$
    Therefore,$$
        \sum_{t=1}^TR\left(S_t\right) \geq \sum_{t=1}^TR\left(\hat S_t\right).
    $$
    Third, we have $|S_1\cap A| \geq |\hat S_1\cap A|+1$ since $\hat S_1\cap A\subsetneq S_1\cap A$, as the latter set contains $j$ but the former does not. This contradicts the definition the $\hat S_1,\ldots, \hat S_T$, as the optimal solution that maximizes $|\hat S_1\cap A|$, and thereby proves by contradiction that there exists a solution $S_1,\ldots, S_T$ such that $S_1 \supseteq A$.
    
    Finally we show that we can take $S_1 = S_1^*$ in particular. On one hand, the solution $A, S_2, \ldots, S_T$ is also feasible, as it is obtained by removing all the products $j$ such that $\ell_j=0$ from $S_1$, which cannot break any constraints. Therefore, noting that $\overline A = S_1^*$, the solution $S_1^*, S_2, \ldots, S_T$ is also feasible, since $A\subseteq \overline{A}$. Finally, we have\begin{align*}
        R\left(S_1^*\right)+\sum_{t=2}^T R\left(S_t\right) &= R\left(\overline A\right)+\sum_{t=2}^T R\left(S_t\right)\geq \sum_{t=1}^TR(S_t),
    \end{align*}
    where the inequality follows from the definition of the expanded set of $A$, and the fact that $A\subseteq S_1$. In conclusion, there exists an optimal solution of \ref{APV} such that $S_1 = S_1^*$.
    \paragraph{Concluding the proof of the theorem. }
    In Claim \ref{cl:intermediary}, we showed the existence of an optimal solution which offers $S_1^*$ to the first customer. Therefore, restricting the search space to only such solutions still guarantees obtaining an optimal solution. Therefore, \ref{APV} is equivalent to the following optimization problem:

        \begin{equation*}
        \begin{aligned}
         \max_{ S_2, \ldots, S_T \subseteq \mathcal{N}}  & \; \;      R(S_1^*)+\sum_{t=2}^T  R(S_t)   \\  
          s.t. \;\;   & \;\;  1+\sum_{t=2}^T  \mathbbm{1}(i \in S_t) \geq \ell_i, \;\;\; \forall i \in A,
        \end{aligned}
        \end{equation*}
    which itself reduces to the following different instance of \ref{APV}.
    \begin{equation}\label{eq:reducedinstance}
        \begin{aligned}
         \max_{ S_2, \ldots, S_T \subseteq \mathcal{N}}  & \; \;      \sum_{t=2}^T  R(S_t)   \\  
          s.t. \;\;   & \;\;  \sum_{t=2}^T  \mathbbm{1}(i \in S_t) \geq \Tilde \ell_i, \;\;\; \forall i \in \Nc,
        \end{aligned}
    \end{equation}
    where $\Tilde \ell_i = \ell_i - \mathbbm 1(i\in A)$, for all $i\in \Nc$. Noting that this last problem is an instance of \ref{APV} with $T-1$ customers, we can apply the induction hypothesis, which implies that the optimal solution is given by Equations \eqref{eq:sol}. Directly applying the formulas to this new instance implies that $S_2^*, \ldots, S_T^*$ is an optimal solution to \eqref{eq:reducedinstance}, and hence that $S_1^*,\ldots,S_T^*$ is an optimal solution for \ref{APV}.
    \paragraph{Running time. }The running time of the algorithm for APV can be improved from $O(nT)$ to $O(n+T)$. In fact, we proceed by induction. We start by computing $S_T^*$. Then, when computing $S_t^*$, we do not need to directly compute $\overline{\bigcup_{t \leq u \leq T} L_u}$. Instead, since $S_t^*\supseteq S_{t+1}^* $, we just need to check the products $\{i \in \mathcal{N}\backslash (S_{t+1}^* ) : p_i \geq \overline{R}(S_{t}^* ) \}$, which  is achieved in $O(1+|S_t^*|-|S_{t+1}^*|)$. Therefore, the running time is 
$O(1+|S_T^*|)+\sum_{t=1}^{T-1} O(1+|S_t^*|-|S_{t+1}^*|)=O(|S_1^*|+T)= O(n+T).$

\end{proof}


















%\vspace{-7mm}

\subsection{Linear Program for \ref{APV}} \label{subsectin:lp}
\vspace{-2mm}

Consider the classic unconstrained assortment problem under MNL model for a single customer 
\begin{equation}
\label{Unconstrained problem}
\begin{aligned}
\max_{S \subseteq \mathcal{N}} \quad  R(S).
\end{aligned}
\tag{\sf{AP}}
\end{equation}
It is known that 
\ref{Unconstrained problem} can be formulated as the following LP (\cite{Gallego2011AGA}),
\vspace{-3mm}
\begin{equation*}
\label{Unconstrained problem LP}
\begin{aligned}
 \max_{S \subseteq \mathcal{N}} R(S) =  \max_{(\alpha_i)_{0 \leq i \leq n}} \left\{ \sum_{i=1}^n p_i \alpha_i \quad s.t. \quad \forall i \in \mathcal{N}, 0 \leq \frac{\alpha_i}{v_i} \leq \alpha_0, \quad \sum_{i=0}^n \alpha_i = 1 \right\}.
\end{aligned}
\end{equation*}
% \textcolor{red}{Omar: review and shorten this paragraph}\\
% The fact that the \ref{Unconstrained problem} optimal solution (as outlined in Appendix \ref{apx1}) is equivalent to the optimal solution of the above LP can be seen as follows.
% Let $S$ be an optimal solution to the initial problem \ref{Unconstrained problem}. Then, we can define $\alpha_0 = \frac{1}{1 + V(S)}$ and $\forall i \in \mathcal{N}, \alpha_i = \mathbbm{1}_{\{i \in S\}} \frac{v_i}{1 + V(S)}$, which is a feasible solution to the LP. And the two objective functions have the same value. Therefore, the optimal value of the LP is greater or equal to the optimal value of the initial problem. 
% Let $(\alpha_i)_{0 \leq i \leq n}$ be an optimal solution to the LP. Then, we define $S = \{i \in \mathcal{N},~\alpha_i = v_i \alpha_0 \}$, which is feasible for the initial problem. Since there are $n+1$ variable in the LP,  we can find a solution such that at least $n+1$ constraints are tight (a solution on an extremal point of the feasibility polytope). The equality constraint will always be verified, and since within the $2n$ inequality constraints, each one is incompatible with another, we have that for each $i, \alpha_i \in \{0, \alpha_0 v_i\}$. Therefore, the two objective functions have the same value. This proves that the \ref{Unconstrained problem} optimal value is greater or equal to the optimal value of the LP.
% As a result, the two problems have the same optimal value, which proves the equivalence. %and therefore each optimal solution for one yields an optimal solution for the other.
Motivated by the structure of the above LP  and the structure of our optimal solution of \ref{APV} given in Equation \eqref{eq:sol},
we propose the following linear formulation for  \ref{APV}.

\begin{theorem}[\bf LP for \ref{APV}] \label{thm:LPformulation}
    \ref{APV} is equivalent to the following linear program:
    \begin{equation}\label{LP}\tag{\sf LP}
    \begin{aligned}
     \max_{\boldsymbol{\alpha}} \;\; & \sum_{i=1}^n p_i \sum_{t=1}^T \alpha_i^t \\
     s.t. \quad &  \sum_{i=0}^n \alpha_i^t = 1, &&\quad\forall t \in [T], \\ 
     & \alpha_i^t = v_i \alpha_0^t, &&\quad\forall i \in \mathcal{N}, \;\; \forall t \in \{1,\ldots,\ell_i\},\\
     &  0 \leq \alpha_i^t \leq v_i \alpha_0^t ,&&\quad\forall i \in \mathcal{N}, \;\; \forall t \in \{\ell_i+1,\ldots, T\}.
    \end{aligned}
    \end{equation}
\end{theorem}

\begin{proof}
    In this proof, we use $\opt^{LP}$ and $\opt^{APV}$ to denote the values of \ref{LP} and \ref{APV} respectively. Our objective is to show that these values are equal. We demonstrate, on one hand, that $\opt^{LP} \geq \opt^{APV}$, by constructing a feasible solution to \ref{LP} whose objective is greater than or equal to $\opt^{APV}$. On the other hand, we demonstrate using the same technique that $\opt^{LP} \leq \opt^{APV}$. The combination of these two inequalities directly implies the desired result.
    
    \noindent {\em \underline{First inequality}. } Recall that $(S_1^*, \ldots,S_T^*)$ is the optimal solution of \ref{APV}, whose expression is stated in Equations \eqref{eq:sol}. We introduce a solution $\boldsymbol{\alpha}$ for \ref{LP}, defined as follows, for every $t\in [T]$, \begin{align*}
        &\alpha_0^t = \frac{1}{1+V(S_t^*)},\\
        &\alpha_i^t = \frac{v_i}{1+V(S_t^*)}\cdot \mathbbm 1\left(i\in S_t^*\right)\quad\quad\text{ for } i\in \Nc.
    \end{align*}
    Let us show that $\boldsymbol \alpha$ is feasible. The first and third constraints are straightforward. Indeed, for the first constraint, we have for all $t\in [T]$, \begin{equation*}
        \sum_{i=0}^n\alpha_i^t = \frac{1}{1+V(S_t^*)} + \sum_{i\in S_t^*}\frac{v_i}{1+V(S_t^*)} =1.
    \end{equation*}
    The third constraint is also directly verified by construction, as for all $i\in \Nc$ and $t\in [T]$, we have
    \begin{equation*}
        \alpha_i^{t} = \frac{v_i}{1+V(S_t^*)}\cdot \mathbbm 1\left(i\in S_t^*\right) \leq \frac{v_i}{1+V(S_t^*)} = v_i\alpha_{0}^t.
    \end{equation*}
    Regarding the second constraint, let $i\in \Nc$ and $t\in \{1,\ldots, \ell_i\}$. We have\begin{equation*}
        i\in L_{\ell_i} \subseteq \bigcup_{t\leq u\leq T}L_u \subseteq \overline{\bigcup_{t\leq u\leq T}L_u} = S_t^*.
    \end{equation*}
    Therefore, $$
        \alpha_{i}^t = \frac{v_i}{1+V(S_t^*)} = v_i\alpha_0^t,
    $$
    which proves that the second constraint is respected, and hence shows the feasibility of the constructed solution. Finally, its objective is equal to the revenue of the sequence of assortments $S_1^*,\ldots, S_T^*$ as demonstrated by the following easy computation:\begin{equation*}
        \sum_{i=1}^n p_i\sum_{t=1}^T \alpha_i^t = \sum_{t=1}^T\sum_{i=1}^n p_i \alpha_i^t = \sum_{t=1}^T\sum_{i\in S_t^*} p_i \phi(i,S_t^*) = \sum_{t=1}^TR(S_t^*) = \opt^{APV}.
    \end{equation*}
    Since the objective of $\boldsymbol{\alpha}$ is trivially upper bounded by $\opt^{LP}$, we deduce that  ${\opt^{LP} \geq \opt^{APV}}$.

    \noindent {\em \underline{Second inequality}. } Consider an optimal basic feasible solution for \ref{LP}, denoted by $\boldsymbol{\alpha}\coloneqq(\alpha_{i}^t\,:\,i\in \Nc\cup\{0\}, t\in [T])$. \ref{LP} is a linear program with $T\cdot(n+1)$ variables. Any basic solution has at least $T\cdot(n+1)$ active constraints. Since \ref{LP} already contains $T+\sum_{i\in \Nc}\ell_i$ equality constraints, $\boldsymbol{\alpha}$ activates at least $\sum_{i\in \Nc}(T-\ell_i)$ constraints from the remaining $2\cdot \sum_{i\in \Nc}(T-\ell_i)$ inequality constraints. Moreover, since each pair of constraints $\alpha_i^t \geq 0$ and $\alpha_i^t\leq v_i\alpha_0^t$ consists on two incompatible constraints, at most $\sum_{i\in \Nc}(T-\ell_i)$ (i.e., half of the inequality constraints) can be active. Thus, $\boldsymbol\alpha$ activates exactly $\sum_{i\in \Nc}(T-\ell_i)$ inequality constraints, and we have for all $i\in \Nc$ and $t\in \{\ell_i+1,\ldots,T\}$, $\alpha_{i}^t \in \{0, v_i\alpha_0^t\}$. Using this new observation, we construct a feasible solution $(S_1,\ldots, S_T)$ for \ref{APV} as follows: for all $t\in [T]$, let $S_t \coloneqq \{i\in \Nc\,\colon\, \alpha_i^t\neq 0\}$. In particular, if $i \in S_t$, then $\alpha_i^t = v_i\alpha_0^t$. It is then easy to see that $\alpha_i^t$ is exactly the choice probability of product $i$ in assortment $S_t^*$. Indeed, for all $t\in [T]$, we have$$
        \alpha_0^t+\sum_{i\in \Nc} \alpha_i^t= \alpha_0^t+ \sum_{i\in S_t}v_i \alpha_0^t= \alpha_0^t\cdot (1+V(S_t)),
    $$
    which implies using the first constraint of \ref{LP} that $\alpha_i^0 = \phi(0,S_t)$, and thereby that $\alpha_i^t = \phi(i, S_t)$ for all $i \in S_t$.
    Next, we can easily see that the $(S_1, \ldots, S_T)$ is feasible for \ref{APV}, as any product $i\in \Nc$ is included in the first $\ell_i$ assortments, which guarantees the visibility constraints. Finally we have$$
        \opt^{LP} = \sum_{i\in \Nc}p_i \sum _{t=1}^T\alpha_i^t = \sum_{t=1}^T\sum_{i\in \Nc}p_i\phi(i,S_t) = \sum_{t=1}^TR(S_t) \leq \opt^{APV}.
    $$
    
    \noindent {\em \underline{Conclusion}. }Combining the inequalities from the two cases shows that $\opt^{APV} = \opt^{LP}$, thereby demonstrating the equivalence between the two optimization problems \ref{LP} and \ref{APV}.
\end{proof}


    \section{Polynomial Time Algorithm for \ref{APV}} \label{sec:apv}




The primary contribution of this section consists of developing a polynomial time algorithm for \ref{APV}. To this end, in Section \ref{subsection:expanded}, we introduce the ``expanded revenue" of an assortment  and its ``expanded set". %, which are instrumental for our analysis. 
In Section \ref{subsection:prop}, we present a polynomial time procedure for computing these two objects and establish the monotonicity and supermodularity of the expanded revenue function. Leveraging these properties, in Section \ref{subsection:algo}, we characterize the structure of optimal solutions to \ref{APV}, showing that they can be computed in $O(n + T)$ time. Finally, in Section \ref{subsectin:lp}, we prove that \ref{APV} can be formulated as a compact linear program.



% In this section, the main result we will prove is the following:

% There exists a nested optimal solution to \ref{APV}, in which every assortment offered is included in the previous one. This solution can be computed in time $\mathcal{O}(nT)$.

% \begin{theorem}
%     The problem \ref{APV} can be solved in Polynomial time
% \end{theorem}

% To achieve this, we first introduce the notion of expanded set and expanded revenue, that will be useful for our analysis. We then prove some interesting monotonicity and supermodularity properties on the expanded revenue function. Relying on these properties, we finally identify a simple structure for an optimal solution of problem \ref{APV}, which can be computed in polynomial time.



\subsection{Expanded Revenue and Expanded Set} \label{subsection:expanded}

We begin our analysis by examining \ref{APV} in the context of a single customer. In this simple scenario, the underlying visibility constraints are given such that each product $i\in \Nc$ either must be displayed (i.e., $\ell_i = 1$) or can be displayed (i.e., $\ell_i=0$). Let $A=  \{i\in \Nc\,\colon\,\ell_i = 1\}$ denote the subset of all products that must be displayed. Consequently, in this simple setting, \ref{APV} is the problem of identifying the assortment that maximizes revenue while including $A$. This particular problem will serve as  the building block for our analysis, as it lays the foundation for understanding the general case involving  $T$ customers. Thus, it leads us to introduce the next two definitions.
% This lead us to introduce the notion of the expanded revenue,  and the expanded revenue of a set and show how we con compute them. We introduce the following defintiions.



%Because of the visibility constraints, the assortments offered under \ref{APV} will have to enforce certain products. Thus it becomes interesting to study how we can maximize the expected revenue of a set once if we force it to contain certain elements, and we optimize over the remaining ones.

%We therefore naturally introduce the following notations:


\begin{definition}[\bf Expanded revenue]
    \label{expanded revenue}
    Let $A \subseteq \mathcal{N}$. The expanded revenue $\overline{R}(A)$ of $A$ is defined as the maximum expected revenue achieved by any assortment that contains $A$. Namely,

    \begin{equation} \label{eq:exrev}
        \overline{R}(A) \coloneqq \max_{S \subseteq \mathcal{N} : A \subseteq S} R(S).
    \end{equation}
\end{definition}

An optimal solution to the maximization problem in \eqref{eq:exrev} is referred to as the expanded set of $A$. When multiple optimal solutions exist, we break ties by selecting one with the largest cardinality, which is shown to be unique in Lemma \ref{Compute expanded set}, hence proving that the expanded set is well defined. Formally, we have the following definition. 

\begin{definition}[\bf Expanded set]
    \label{expanded set}
The expanded set $\overline{A}$ of $A$ is defined as the assortment within $\mathcal{N}$ that  maximizes the expected revenue  among all assortments containing $A$.  If multiple assortments attain this optimum, $\overline{A}$ is selected as the one of largest cardinality.  In other words,
    $$\overline{A} \coloneqq \underset{S \subseteq \mathcal{N} : A \subseteq S}{\arg \max}   \left\{ |S| \; : \; R(S)=  \overline R (A)   \right\}.   $$
\end{definition}


% Let $A \subseteq \mathcal{N}$. The expanded set of $A$, denoted as $\overline{A}$, is defined as the assortment within $\mathcal{N}$ that satisfies two criteria: it contains $A$, and it simultaneously achieves the maximum revenue and the maximum cardinality among all possible assortments. Mathematically, $\overline{A}$ is given by:


% Note that Problem \eqref{eq:exrev} is equivalent to \ref{APV} in the scenario where we have a single customer $(T=1)$ and $A$ is defined as the set of products that have to be shown once, i.e., $A= \{ i \in {\cal N} : \ell_i =1 \} $. In that case, $\overline A$ corresponds to the optimal assortment to this problem with the largest cardinality.

% We will use $\overline{R}$ as a set function that takes input an assortment $A \subseteq R$ and returns $\overline{R}(A)$. In the next section we study several properties of this function as well as properties of the expanded set.



Problem \eqref{eq:exrev} can be viewed as equivalent to \ref{APV} when considering a single customer scenario $(T=1)$ and defining $A$ as the set of products that must be shown. Thus, $\overline A$ represents the optimal assortment, specifically one with the largest cardinality.

In our  analysis, we consider $\overline{R}$ as a set function that takes an assortment $A \subseteq \mathcal{N}$ as input and returns $\overline{R}(A)$, noting that $\overline{R}(A) = R(\overline{A})$.







%$\overline{R}$ is the expanded revenue function, such that $\overline{R}(A) = R(\overline{A})$. When there are several solutions for $\overline{A}$, we define $\overline{A}$ as the one with the highest cardinal. The following lemma justifies that $\overline{A}$ is well defined and can be computed efficiently.

\subsection{Properties of the Expanded Revenue Function} \label{subsection:prop}

In this section, we first show that the expanded revenue and the expanded set of a given assortment can both be computed in polynomial time (Lemma \ref{Compute expanded set}). Then, we prove that the expanded revenue is a non-increasing function and that the expanded set is a non-decreasing function (Lemma \ref{monotonicity}). 
Finally, we argue that the expanded revenue function is supermodular (Lemma \ref{supermodularity}), which is a particularly important property for our algorithm design later on. %These  structural properties will be useful in designing our algorithm to solve \ref{APV} for $T \geq 1$.




\vspace{2mm}
\noindent
{\bf Computing the expanded revenue and expanded set.}
Recall that, without loss of generality $p_1 \geq \cdots \geq p_n$. We define an assortment $S$ to be price-ordered if $S=\black{[k]}$ for some $0 \leq k \leq n$, essentially prioritizing products with high prices; clearly, there  are only $n+1$ such assortments possible.
Consider an arbitrary assortment $A \subseteq {\cal N}$ and its expanded set $\overline A$. In Lemma \ref{Compute expanded set} below, we prove that $\overline A$ is the union of $A$ and a price-ordered assortment. Since there are only $n+1$ possible price-ordered assortments, it suffices to compute the expected revenue of the assortments $A \cup \black{[k]}$, for each $k \in \black{[n]}$. The expanded set corresponds to the assortment with the highest expected revenue. As previously explained, in the case of multiple assortments with the same maximum revenue, we break ties by selecting the one with the largest cardinality. Thus, the expanded set $\overline A$ can be computed  $O(n)$ time, and the expanded revenue is simply $\overline{R}(A)= R(\overline A)$. The proof of Lemma \ref{Compute expanded set} leverages several structural properties of the MNL-based revenue function that are presented in Appendix \ref{apx1}.

\begin{lemma}
    \label{Compute expanded set}
    For any assortment $A \subseteq \mathcal{N}$, its expanded set is given by  
    $ \overline{A} = A \cup \{i \in \mathcal{N} : p_i \geq \overline{R}(A) \}.$ Furthermore, $\overline{R}(A)$ and $\overline{A}$ can be computed in time $O(n)$.
\end{lemma}
\begin{proof}
    By definition of the expanded set, we have $A \subseteq \overline{A}$. Hence, there exists an assortment $B \subseteq \mathcal{N} \setminus A$, such that $\overline{A} = A \uplus B$, \black{where $\uplus$ denotes the disjoint union of sets}. In what follows, we prove that
    $B = \{i \in \mathcal{N} \setminus A   \; : \; p_i \geq \overline{R}(A) \}.$
    \begin{itemize}
        \item {\em Direct inclusion}: Let $i\in B$, and assume by contradiction that $p_i < \overline{R}(A) = R(A \cup B)$. It is easy to verify that, under the MNL model, when we add a product $j\notin S$ to an assortment $S$, the revenue of this assortment weakly increases if and only of $p_j \geq R(S)$. For completeness, we provide the formal statement and the proof of this result in Lemma \ref{Revenue variations} (see Appendix \ref{apx1}). By this lemma, removing product $i$ from $B$ would strictly increase the expected revenue $R(A)$, which contradicts the optimality of $A \cup B$.
        \item {\em Indirect inclusion: } Let $i \in \mathcal{N} \setminus A$ be a product with $p_i \geq \overline{R}(A)$, and suppose by contradiction that $i \notin B$. By Lemma \ref{Revenue variations}, adding $i$ to $B$ would weakly increase the expected revenue of $A\cup B$. If this increase is strict, it contradicts the optimality of $A \cup B$. If the expected revenue remains unchanged, it contradicts the definition of $\overline{A} = A \cup B$ as the optimal assortment of maximum cardinality. 
        \end{itemize}
    Thus,
        $\overline{A} = A \cup \{i \in \mathcal{N} \setminus A\,\colon\, p_i \geq \overline{R}(A) \}.$
        
    Finally, $\overline{A}$ can be computed in $O(n)$ time. Indeed, starting from $A$, we sequentially add products by decreasing price. At each iteration, we can compute the new expected revenue given the previous one in constant time by storing the current numerator and denominator, since we only need to add $p_i v_i$ to the former and $v_i$ to the latter. Finally, we pick the highest revenue assortment among the $n$ computed sets.
\end{proof}
    

%Next, we show that the expanded revenue is a non-increasing function and  the expanded set is a non-decreasing function.


\begin{lemma}[\bf Monotonicity]
    \label{monotonicity} 
    If $ A \subseteq B \subseteq \mathcal{N}$, then  $ \overline{A} \subseteq \overline{B}$ and $\overline{R}(A) \geq \overline{R}(B)$.   
\end{lemma}

\begin{proof}
    It is easy to see that when $A \subseteq B \subseteq \mathcal{N}$, every feasible solution for $\max_{S \subseteq \mathcal{N} , \; B \subseteq S} R(S)$ is a feasible solution for $\max_{S \subseteq \mathcal{N} , \; A \subseteq S} R(S)$. Hence, $\overline{R}(A) \geq \overline{R}(B)$. It follows that $\{i \in \mathcal{N}, p_i \geq \overline{R}(A) \} \subseteq \{i \in \mathcal{N}, p_i \geq \overline{R}(B) \},$ and therefore, by Lemma \ref{Compute expanded set}, we have $$\overline{A} = A \cup \{i \in \mathcal{N}, p_i \geq \overline{R}(A) \} \subseteq B \cup \{i \in \mathcal{N}, p_i \geq \overline{R}(B) \} = \overline{B}.$$
\end{proof}

%Finally, we show that  the expanded revenue function  $\overline R$ is supermodular. This  property  will play  a fundamental role later in our analysis.


\begin{lemma}[\bf Supermodularity]
    \label{supermodularity}
    The expanded revenue function $\overline{R}$ is  supermodular, i.e.,
    % for every $A, B \subseteq \mathcal{N}$,
    %  $$\overline{R}(A \cup B) + \overline{R}(A \cap B) \geq \overline{R}(A) + \overline{R}(B).$$
     \black{for every $A\subseteq B \subseteq \mathcal{N}$ and $i\in \Nc\setminus B$,
     $$\overline{R}(B\cup\{i\}) - \overline{R}(B) \geq \overline{R}(A\cup\{i\}) - \overline{R}(A).$$}
\end{lemma}
\begin{proof}
% In this proof, we use the following alternative definition of supermodularity. A function $f\colon \Omega\rightarrow \R$ is supermodular if and only if $f(B\cup\{i\})-f(B) \geq f(A\cup\{i\}) - f(A)$ for all $A,B\subseteq \Omega$ such that $A\subseteq B$, and for each $i\in \Omega\setminus B$.

Let $A,B\subseteq \Nc$ be a pair of assortments with $A\subseteq B$, and let $i\in \Nc\setminus B$. The proof of this result is separated into two steps. First, we show that\begin{equation}\label{eq:step1}
    R\left(\overline{B}\right) - R\left(\overline{B}\cup \overline{A\cup\{i\}}\right) \leq R\left(\overline{A}\right) - R\left( \overline{A\cup\{i\}}\right). 
\end{equation}
Subsequently, we show that\begin{equation}\label{eq:step2}
    R\left(\overline{B}\cup \overline{A\cup\{i\}}\right)- R\left(\overline{B\cup \{i\}}\right) \leq 0.
\end{equation}
The result directly follows by summing inequalities \eqref{eq:step1} and \eqref{eq:step2}.
\paragraph{Proving inequality \eqref{eq:step1}.} 
Let us start with the following claim, \black{whose proof appears in Appendix \ref{apx:computation}}.
\begin{claim}\label{cl:computation}
    For any pair of assortments $S_1, S_2\subseteq \Nc$ with $S_1\subseteq S_2$, we have\begin{equation*}
        R(S_1) - R(S_2) = \frac{1}{1+V(S_2)}\cdot \sum_{j\in S_2\setminus S_1}(R(S_1) - p_j)\cdot v_j.
    \end{equation*}
\end{claim}
\noindent Using this claim, we have
\begin{equation*}
    R\left(\overline{B}\right) - R\left(\overline{B}\cup \overline{A\cup\{i\}}\right) = \frac{1}{1+V
        (\overline{B}\cup \overline{A\cup\{i\}})}\cdot \sum_{j\in \overline{A\cup \{i\}}\setminus \overline{B}} \left(R\left(\overline B\right) - p_j\right)v_j.
\end{equation*}
Next, we have $R(\overline{B}) \geq p_j$ for all $j \notin \overline{B}$, since if there exists some $j\notin \overline{B}$ such that $p_j>R(\overline{B})$, adding this product to $\overline{B}$ would increase its expected revenue according to Lemma \ref{Revenue variations}, which would contradict the definition of the extended set of $B$. Therefore, since $V(\overline{B}\cup \overline{A\cup\{i\}})\geq V(\overline{A\cup\{i\}})$, we have
\begin{align}
    R\left(\overline{B}\right) - R\left(\overline{B}\cup \overline{A\cup\{i\}}\right) &\leq \frac{1}{1+V
        (\overline{A\cup\{i\}})}\cdot \sum_{j\in \overline{A\cup \{i\}}\setminus \overline{B}} \left(R\left(\overline B\right) - p_j\right)v_j \notag\\
        & \leq \frac{1}{1+V
        (\overline{A\cup\{i\}})}\cdot \sum_{j\in \overline{A\cup \{i\}}\setminus \overline{B}} \left(R\left(\overline A\right) - p_j\right)v_j,\label{eq:smallsum}
\end{align}
where \black{Equation~\eqref{eq:smallsum}} follows from Lemma \ref{monotonicity}.
Finally, \black{since $\overline{A}\subseteq \overline{B}$ by again Lemma~\ref{monotonicity},} we have $$\left(\overline{A\cup \{i\}}\setminus \overline{B} \right)\subseteq \left(\overline{A\cup \{i\}}\setminus  \overline{A}\right).$$
Moreover, for every $j \in \overline{A\cup \{i\}}\setminus  \overline{A}$, $j\notin \overline{A}$, and in particular, $p_j\leq R(\overline{A})$. Therefore, by adding the missing terms to the sum in Equation \eqref{eq:smallsum}, we have
    \begin{align*}
        R\left(\overline{B}\right) - R\left(\overline{B}\cup \overline{A\cup\{i\}}\right) & \leq \frac{1}{1+V
        \left(\overline{A\cup\{i\}}\right)}\cdot \sum_{j\in \overline{A\cup \{i\}}\setminus \overline{A}} \left(R\left(\overline A\right) - p_j\right)v_j\\
        & = R\left(\overline{A}\right) - R\left(\overline{A\cup \{i\}}\right),
    \end{align*}
where the equality follows from Claim \ref{cl:computation}.
\paragraph{Proving inequality \eqref{eq:step2}. }On the one hand, we have $\{i\}\subseteq\overline{A\cup \{i\}}$. Therefore, we have in particular ${\{i\} \subseteq \overline{B}\cup\overline{A\cup \{i\}}}$. On the other hand,  $B\subseteq \overline{B}$ and therefore $B\subseteq \overline{B}\cup\overline{A\cup \{i\}}$. Hence $B\cup \{i\}\subseteq \overline{B}\cup\overline{A\cup \{i\}}$. Recalling that $\overline{B\cup \{i\}}$ is by definition the maximum revenue assortment containing $B\cup \{i\}$, we have$$
    R\left(\overline{B\cup \{i\}}\right) \geq R\left(\overline{B}\cup\overline{A\cup \{i\}}\right),
$$
% Let $m \coloneqq |\overline{A\cup\{i\}}|$, and let $j_1, \ldots, j_m$ be the elements of $\overline{A\cup\{i\}}$ in the order of decreasing prices, i.e., $p_{j_1}\geq \ldots, p_{j_m}$. Let $C_0 \coloneqq  \overline{A}, D_0 \coloneqq \overline{B}$, and for all $q=0,\ldots,m-1$, $C_{q+1} = C_q\cup \{j_{q+1}\}$ and $D_{q+1} = D_q\cup \{j_{q+1}\}$. Next, we show by induction that for all $q=0,\ldots, m-1$, we have
% \begin{enumerate}
%     \item[(i)] $R(\overline B) - R(D_q) \leq R(\overline A) - R(C_q)$;
%     \item[(ii)] $R(D_{q}) \geq p_{j_{q}}$ and $R(C_{q}) \geq p_{j_{q}}$;
%     \item[(iii)] $R(C_{q})$
% \end{enumerate}

%     \begin{equation*}
%         R(D_{q}) - R(D_{q+1}) \leq R(C_{q}) - R(C_{q+1}).
%     \end{equation*}
\end{proof}
















\subsection{Computing Optimal Assortments} \label{subsection:algo}

In this section, we characterize the structure of optimal assortments for \ref{APV}, relying on the supermodularity property of the expanded revenue function. Moreover, we explain how to compute such an assortment in $O(n+T)$ time, leading to an exact polynomial-time algorithm.

\vspace{2mm}
\noindent
{\bf Optimal solution.} Consider an arbitrary \ref{APV} instance, and recall that $\ell_i$ is the minimum number of customers for which we must offer each product $i$. For $t \in \black{\{0\}\cup[T]},$ we define 
\begin{equation}\label{eq:theells}
    L_t = \{i \in \mathcal{N}:\, \ell_i = t \},
\end{equation}
Noting that $(L_t)_{0 \leq t \leq T}$ is a partition of $\cal N$, our candidate solution for \ref{APV} is given by offering each customer $t\in [T]$ the set of products
\begin{equation} \label{eq:sol}
    {S_t^*} = \overline{\bigcup_{t \leq u \leq T} L_u}.
\end{equation}


Since  $ \bigcup_{t+1 \leq u \leq T} L_u  \subseteq  \bigcup_{t \leq u \leq T} L_u$, the monotonicity property in Lemma \ref{monotonicity} implies that $S_{t+1}^* \subseteq S_{t}^*$ for any $t=0,\ldots,T-1$. Therefore, our solution has a nested structure, where
$ S_T^* \subseteq S_{T-1}^*\subseteq \cdots \subseteq S_1^*$. In the following, we prove that these assortments are optimal for \ref{APV}, and that they can be computed in polynomial time. Indeed, Lemma \ref{Compute expanded set} shows that each $S_t^*$ can be computed in $O(n)$ time , meaning that the entire solution can be computed in $O(nT)$ time. We can further improve this running time to $O(n+T)$ as shown below.



% \begin{definition}[\bf Nested solution]
%     \label{Nested solution}
%     For each instance of the problem \ref{APV}, we define the partition $(L_t)_{0 \leq t \leq T}$ of $\mathcal{N}$ by: $\forall t \in [\![0, T]\!], \; L_t = \{j \in \mathcal{N}, \ell_j = t \}$ \\
%     We define the sequence of assortments $(S_t^*)_{1 \leq t \leq T} \coloneqq (\overline{\bigcup_{t \leq u \leq T} L_u})_{1 \leq t \leq T}$, that is $S_1^*, S_2^*, \ldots, S_T^* = \overline{L_1 \cup \ldots \cup L_T}, \overline{L_2 \cup \ldots \cup L_T}, \ldots, \overline{L_T}$
% \end{definition}

%We observe that this solution structure is nested: $S_{t+1}^* \subseteq S_t^* \;\; \forall t \in [\![1, T-1]\!] $


\begin{theorem}{}
    \label{Solution structure}
    The  sequence of assortments $(S_t^*)_{1 \leq t \leq T}$ defined in \eqref{eq:sol} is optimal for \ref{APV}. % Moreover, each permutation of these sets is also an optimal solution for \ref{APV}. 
    Moreover, such a solution can be computed in $O(n+T)$ time.
\end{theorem}




\begin{proof}
    We prove the result by induction. First, when $T= 1$, for any product $i\in \Nc$, we either have $\ell_i=0$ or $\ell_i=1$. Noting that $L_1$ is the set of products $i$ for which $\ell_i=1$, \ref{APV} reduces to the problem of computing the optimal assortment that contains $L_1$, i.e., $
        \max_{S\subseteq \Nc\colon L_1\subseteq S}R(S),
    $
    whose solution is $\overline{L_1}$ by definition of the expanded set. By noticing that $S_1^* =\overline{L_1}$, we deduce that $S_1^*$ is the optimal solution to \ref{APV}.
    
    Now consider $T\geq 2$, and assume by induction that for any set of visibility constraints, the optimal solution of \ref{APV} for $T-1$ customers is given by the assortments defined in Equation \eqref{eq:sol}. Let us show that the latter assumption holds for $T$ customers.
    %The outline of the proof is as follows. At first, we consider an optimal solution, then using the supermodularity property, sequentially modify it into another optimal solution where at least one assortment contains $L_1$, i.e., the assortment of products which must be shown at least once.
    We denote by $A$ the set of all products that must be shown to at least one customer, i.e., $A\coloneqq \bigcup_{t=1}^TL_t$. We start by providing the following crucial intermediary claim, whose proof is presented on Page~\pageref{prf:intermediary}.
    \begin{claim}\label{cl:intermediary}
        % There exists an optimal solution $S_1, \ldots, S_T$ to \ref{APV} such that $S_1 \supseteq A$. In particular,
        There exists an optimal solution $S_1, \ldots, S_T$ to \ref{APV} with $S_1 =S_1^*$.
    \end{claim}
    In other words, Claim \ref{cl:intermediary} states that there exists an optimal solution $S_1,\ldots, S_T$ to \ref{APV} such that $S_1$ contains all products that must be shown at least once, i.e., those with $\ell_i\geq 1$. Consequently, this property allows us to focus only on feasible solutions of \ref{APV}, that offer $S_1^*$ to customer $1$. Maximizing the revenue amongst these solutions thereby guarantees attaining the optimal objective, and solving \ref{APV} is in turn equivalent to solving the following optimization problem:
    \begin{equation*}
        \begin{aligned}
         \max_{ S_2, \ldots, S_T \subseteq \mathcal{N}}  & \; \;      R(S_1^*)+\sum_{t=2}^T  R(S_t)   \\  
          s.t. \;\;   & \;\;  1+\sum_{t=2}^T  \mathbbm{1}(i \in S_t) \geq \ell_i, \;\;\; \forall i \in A,
        \end{aligned}
        \end{equation*}
    which itself reduces to the following instance of \ref{APV}:
    \begin{equation}\label{eq:reducedinstance}
        \begin{aligned}
         \max_{ S_2, \ldots, S_T \subseteq \mathcal{N}}  & \; \;      \sum_{t=2}^T  R(S_t)   \\  
          s.t. \;\;   & \;\;  \sum_{t=2}^T  \mathbbm{1}(i \in S_t) \geq \Tilde \ell_i, \;\;\; \forall i \in \Nc,
        \end{aligned}
    \end{equation}
    where $\Tilde \ell_i = \ell_i - \mathbbm 1(i\in A)$, for all $i\in \Nc$. Noting that this last problem is an instance of \ref{APV} with $T-1$ customers, we can apply the induction hypothesis, which implies that the optimal solution is given by Equation \eqref{eq:sol}. Directly applying this formula implies that $S_2^*, \ldots, S_T^*$ is an optimal solution to \eqref{eq:reducedinstance}, and hence that $S_1^*,\ldots,S_T^*$ is an optimal solution for \ref{APV}.
    \paragraph{Running time. }The running time of our algorithm for APV can be improved from $O(nT)$ to $O(n+T)$ as follows. We start by computing $S_T^*$. Then, proceeding by order of decreasing customer index, when computing each $S_t^*$, we do not need to directly compute $\llbracket\bigcup_{t \leq u \leq T} L_u\rrbracket$. Instead, since $S_t^*\supseteq S_{t+1}^* $, we should only check the products $\{i \in \mathcal{N}\backslash (S_{t+1}^* ) : p_i \geq \overline{R}(S_{t}^* ) \}$\black{. Since the latter set is price ordered, we can stop checking when reaching the first item we do not include in $S^*_t$. In the end, $|S_t^*|-|S_{t+1}^*|+1$ items are checked (including the first one that is not included), eventually leading to a running time of 
$$O(1)\cdot \left(1+|S_T^*|+\sum_{t=1}^{T-1} \left(1+|S_t^*|-|S_{t+1}^*|\right)\right)=O(|S_1^*|+T)= O(n+T).$$
}
    \paragraph{Proof of Claim \ref{cl:intermediary}.\label{prf:intermediary} }The proof of this result mainly relies on exploiting the supermodularity of $\overline{R}$, as stated in Lemma \ref{supermodularity}. \black{We start by showing that there exists a solution $S_1,\ldots, S_T$ to \ref{APV} such that $S_1\supseteq A$.} Assume by contradiction that there is no optimal solution of \ref{APV} with $S_1 \supseteq A$. Let $\hat S_1, \ldots, \hat S_T$ be the optimal solution of \ref{APV} that maximizes $|\hat S_1\cap A|$; in the case of ties, we pick an arbitrary solution that maximizes $|\hat S_1\cap A|$. Since $A \nsubseteq \hat S_1$, there exists some product $j\in A\setminus S_1$. Moreover, since $j\in A$, we know that $\ell_j\geq 1$, and therefore, this product must be shown to at least one customer, which means that there exists some $u\in [T]\setminus \{1\}$ with $j\in \hat S_u$. We define the following new solution to \ref{APV}: $S_1 = \hat S_1\cup \hat S_u$, $S_u = \hat S_1\cap \hat S_u$, and $S_t = \hat S_t$ for all $t\notin \{1,u\}$. First, this newly defined solution is also feasible since any product offered once in either $\hat S_1$ or $\hat S_u$ \black{(exclusive)} is also shown in $S_1$, and each product shown in both $\hat S_1$ and $\hat S_u$ is also shown in both $S_1$ and $S_u$. Second, $S_1,\ldots, S_T$ must be an optimal solution. Indeed, by the supermodularity\footnote{\black{Here, we use the equivalent definition of supermodularity: A function $f:\Omega \rightarrow \R$ is supermodular if ${f(A\cup B)} + f(A\cap B) \geq f(A)+ f(B)$ for all $A,B\subseteq \Omega$.}} property, we have $$
        R\left(\hat S_1\cup \hat S_u\right)+ R\left(\hat S_1\cap \hat S_u\right) \geq R\left(\hat S_1\right)+ R\left(\hat S_u\right).
    $$
    Therefore,$$
        \sum_{t=1}^TR\left(S_t\right) \geq \sum_{t=1}^TR\left(\hat S_t\right).
    $$
    Third, we have $|S_1\cap A| \geq |\hat S_1\cap A|+1$ since $(\hat S_1\cap A)\subsetneq (S_1\cap A)$, as the latter set contains $j$ whereas the former does not. This contradicts the definition of $\hat S_1,\ldots, \hat S_T$, as the optimal solution that maximizes $|\hat S_1\cap A|$, and thereby proves by contradiction that there exists a solution $S_1,\ldots, S_T$ with $S_1 \supseteq A$.
    
    Finally, we show that we can in particular take $S_1 = S_1^*$. On the one hand, the solution $A, S_2, \ldots, S_T$ is also feasible, as it is obtained by removing from $S_1$ all products $j$ such that $\ell_j=0$, which \black{are absent from the visibility constraints and thus cannot violate them}. Therefore, noting that $\overline A = S_1^*$, the solution $S_1^*, S_2, \ldots, S_T$ is also feasible, since $A\subseteq \overline{A}$. Finally, we have\begin{align*}
        R\left(S_1^*\right)+\sum_{t=2}^T R\left(S_t\right) &= R\left(\overline A\right)+\sum_{t=2}^T R\left(S_t\right)\geq \sum_{t=1}^TR(S_t),
    \end{align*}
    where the latter inequality follows by definition of the expanded set of $A$, and the fact that $A\subseteq S_1$\black{, giving $R(\overline{A})\geq R(\overline{S_1})\geq R(S_1)$}. In conclusion, there exists an optimal solution $S_1,\ldots, S_T$ of \ref{APV} with $S_1 = S_1^*$.
\end{proof}


















%\vspace{-7mm}

\subsection{Linear Programming Formulation} \label{subsectin:lp}
\vspace{-2mm}
In this section, we provide a linear programming-based exact algorithm for solving \ref{APV}. For this purpose, consider the unconstrained assortment problem under the MNL model for a single customer, given by
\begin{equation}
\label{Unconstrained problem}
\begin{aligned}
\max_{S \subseteq \mathcal{N}} \quad  R(S),
\end{aligned}
\tag{\sf{AP}}
\end{equation}
where we recall that $R(S)$ is the expected revenue of the assortment $S$ in this model.
It is well-known that 
\ref{Unconstrained problem} can be formulated as the following LP (\cite{Gallego2011AGA}):
\begin{equation*}
    \begin{aligned}
     \max_{(\alpha_i)_{0 \leq i \leq n}} \;\; & \sum_{i=1}^n p_i \alpha_i \\
     s.t. \quad &  0 \leq \frac{\alpha_i}{v_i} \leq \alpha_0, &&\quad\forall i \in \mathcal{N}, \\ 
     & \sum_{i=0}^n \alpha_i = 1.
    \end{aligned}
\end{equation*}



Motivated by the structure of our optimal solution of \ref{APV}, as prescribed by Equation \eqref{eq:sol},
we propose the following linear formulation for  \ref{APV}:
  \begin{equation}\label{LP}\tag{\sf LP}
    \begin{aligned}
     \max_{\boldsymbol{\alpha}} \;\; & \sum_{i=1}^n p_i \sum_{t=1}^T \alpha_i^t \\
     s.t. \quad & \text{(A)}\quad&&  \sum_{i=0}^n \alpha_i^t = 1, &&\quad\forall t \in [T], \\ 
     &\text{(B)}\quad&& \frac{\alpha_i^t}{v_i} = \alpha_0^t, &&\quad\forall i \in \mathcal{N}, \;\; \forall t \in \black{[\ell_i]},\\
     &\text{(C)}\quad&&  0 \leq \frac{\alpha_i^t}{v_i} \leq  \alpha_0^t ,&&\quad\forall i \in \mathcal{N}, \;\; \forall t \in \black{[T]\setminus[\ell_i]}.
    \end{aligned}
    \end{equation}


\black{
\noindent In the above linear program, we invite the reader to think of the variable $\alpha_i^t$ as the probability that customer $t$ purchases product $i$, when optimal assortments are offered to customers. Our optimality proof later shows that these variables indeed correspond to the stated probabilities. Using this interpretation, the objective of \ref{LP} represents the expected revenue collected from our collection of customers. Constraint (A) ensures that every customer either purchases a product or opts for the no-purchase option, bringing the sum of probabilities of these events to $1$. For each product $i\in \Nc$, Constraint (B) links the probability that a customer $t\in[\ell_i]$ purchases product $i$ to the probability of leaving the market. In essence, it imposes that product $i$ is included in assortment $S_t$ when $t\in[\ell_i]$, which we know is optimal by virtue of Equation \eqref{eq:sol}. Finally, Constraint (C) bounds the choice probability of each product $i$ by customer $t \in [T]\setminus [\ell_i]$. We later show that using the properties of basic feasible solutions of linear programs that $\alpha_i^t$ adheres to either the upper or the lower bound, corresponding to either including product $i$ in assortment $S_t$ or excluding it.

In what follows, we use $\opt^{LP}$ and $\opt^{APV}$ to denote the optimum values of \ref{LP} and \ref{APV} respectively.
}

    
\begin{theorem} \label{thm:LPformulation}
    $\opt^{LP}=\opt^{APV}$. Moreover, given an optimal solution $\boldsymbol{\alpha}^*$ for \ref{LP}, we can construct an optimal solution $S_1^*, \ldots, S_T^*$ for \ref{APV} by taking $
        S_t^* = \{i\in \Nc\colon \alpha_i^t>0\}
    $
    for all $t\in [T]$.
\end{theorem}

\begin{proof}
     On the one hand, we prove that $\opt^{LP} \geq \opt^{APV}$, by constructing a feasible solution to \ref{LP} whose objective is at least $\opt^{APV}$. On the other hand, using the same technique, we argue that $\opt^{LP} \leq \opt^{APV}$.

    \noindent {\em {First inequality: $\opt^{LP} \geq \opt^{APV}$}. } Recall that $(S_1^*, \ldots,S_T^*)$ stands for the optimal solution of \ref{APV}, whose definition is stated in Equation \eqref{eq:sol}. We introduce a solution $\boldsymbol{\alpha}$ for \ref{LP} such that for every customer $t\in [T]$ and every option $i\in \Nc \cup \{0\}$, \begin{equation*}
        \alpha_i^t = \phi(i, S_t^*),
    \end{equation*}
    where \black{we recall that} $\phi(i,S) = 0$ for every assortment $S\subseteq \Nc$ and every product $i\in \Nc\setminus S$ by convention.
    Let us show that $\boldsymbol \alpha$ is feasible. Constraints (A) and (C) are straightforward. Indeed, for Constraint (A), we have for all $t\in [T]$, \begin{equation*}
        \sum_{i=0}^n\alpha_i^t = \frac{1}{1+V(S_t^*)} + \sum_{i\in S_t^*}\frac{v_i}{1+V(S_t^*)} =1.
    \end{equation*}
    Constraint (C) is also directly satisfied by construction, as for all $i\in \Nc$ and $t\in [T]$, we have
    \begin{equation*}
        \black{0\leq}\; \alpha_i^{t} = \frac{v_i}{1+V(S_t^*)}\cdot \mathbbm 1\left(i\in S_t^*\right) \leq \frac{v_i}{1+V(S_t^*)} = v_i\alpha_{0}^t.
    \end{equation*}
    Regarding Constraint (B), let $i\in \Nc$ and $t\in \black{[\ell_i]}$. Then
    \black{\begin{align}
        i&\in L_{\ell_i} \label{eq:1}\\ & \subseteq \bigcup_{t\leq u\leq T}L_u \label{eq:2}\\ &\subseteq \overline{\bigcup_{t\leq u\leq T}L_u}\label{eq:3}\\ & = S_t^*\label{eq:4}.
    \end{align}}
    \black{Here, Equation~\eqref{eq:1} comes from the} definition of $L_{\ell_i}$ (see Equation \eqref{eq:theells}). \black{Equation~\eqref{eq:2}} is a consequence of $t\leq \ell_i\leq T$. \black{Equation~\eqref{eq:3}} follows from the fact that the extended set of any assortment $S\subseteq \Nc$ contains $S$. \black{Finally, Equation \eqref{eq:4}} is by definition of $S_t^*$, as given by Equation \eqref{eq:sol}.
    Therefore, $$
        \alpha_{i}^t = \frac{v_i}{1+V(S_t^*)} = v_i\alpha_0^t.
    $$
    Finally, the objective value of $\boldsymbol{\alpha}$ is precisely
    \begin{equation*}
        \sum_{i=1}^n p_i\sum_{t=1}^T \alpha_i^t = \sum_{t=1}^T\sum_{i\in S_t^*} p_i \phi(i,S_t^*) = \sum_{t=1}^TR(S_t^*) = \opt^{APV}.
    \end{equation*}
    Since the objective value of $\boldsymbol{\alpha}$ forms a lower bound on $\opt^{LP}$, we deduce that  ${\opt^{LP} \geq \opt^{APV}}$.

    \noindent {\em {Second inequality: $\opt^{LP} \leq \opt^{APV}$}. } Consider an optimal basic feasible solution $\boldsymbol{\alpha}$ to \ref{LP}. Since \ref{LP} is a linear program with $T\cdot(n+1)$ variables, any basic solution has at least $T\cdot(n+1)$ active constraints. Since \ref{LP} already contains $T+\sum_{i\in \Nc}\ell_i$ equality constraints, $\boldsymbol{\alpha}$ activates at least $\sum_{i\in \Nc}(T-\ell_i)$ constraints from the remaining $2\cdot \sum_{i\in \Nc}(T-\ell_i)$ inequality constraints. Moreover, \black{for each pair of constraints $\alpha_i^t \geq 0$ and $\alpha_i^t\leq v_i\alpha_0^t$, at most one can be satisfied with equality. Therefore, at most $\sum_{i\in \Nc}(T-\ell_i)$ inequality constraints can be active.} Thus, $\boldsymbol\alpha$ activates exactly $\sum_{i\in \Nc}(T-\ell_i)$ inequality constraints, and we have $\alpha_{i}^t \in \{0, v_i\alpha_0^t\}$, for all $i\in \Nc$ and $t\in \black{[T]\setminus[\ell_i]}$. Using this new observation, we construct a feasible solution $(S_1,\ldots, S_T)$ for \ref{APV} as follows: For all $t\in [T]$, let $S_t \coloneqq \{i\in \Nc\,\colon\, \alpha_i^t\neq 0\}$. In particular, if $i \in S_t$, then $\alpha_i^t = v_i\alpha_0^t$. It is easy to verify that $\alpha_i^t$ is exactly the choice probability of product $i$ out of the assortment $S_t^*$. Indeed, for all $t\in [T]$, we have$$
        \alpha_0^t+\sum_{i\in \Nc} \alpha_i^t= \alpha_0^t+ \sum_{i\in S_t}v_i \alpha_0^t= \alpha_0^t\cdot (1+V(S_t)),
    $$
    which implies that $\alpha_i^0 = \phi(0,S_t)$ using Constraint (A) of \ref{LP}, and thereby that $\alpha_i^t = \phi(i, S_t)$ for all $i \in S_t$.
    Next, we can easily see that $(S_1, \ldots, S_T)$ is feasible for \ref{APV}, as any product $i\in \Nc$ is included in \black{each of} the first $\ell_i$ assortments, which satisfies the visibility constraints. Finally, we have$$
        \opt^{LP} = \sum_{i\in \Nc}p_i \sum _{t=1}^T\alpha_i^t = \sum_{t=1}^T\sum_{i\in \Nc}p_i\phi(i,S_t) = \sum_{t=1}^TR(S_t) \leq \opt^{APV}.
    $$
\end{proof}



% \section{\ref{APV} with Cardinality Constraints}
    \label{sec:APVC}
A natural extension to \ref{APV} consists on considering cardinality constraints on the offered assortment. Cardinality constraints arise in multiple real world applications such as shelf space in brick-and-mortar stores or screen size in online stores, in which case, offering large assortments of products is infeasible. In this section we consider the constrained version of our problem, which we refer to as {\em Assortment Problem with Visibility and Cardinality constraints (\ref{APVC})}, where each customer can be shown at most $k$ products, where $k$ is a positive integer specified as part of the input. Formally, \ref{APVC} is defined as follows, 
\begin{equation}
\label{APVC}
\tag{\sf{APVC}} 
\begin{aligned}
 \max_{ S_1, \ldots, S_T \subseteq \mathcal{N}} & \; \; \sum_{t=1}^T  \frac{\sum_{i \in S_t}p_i v_i}{1 + \sum_{i \in S_t} v_i}  \\ s.t. \;\; &\sum_{t=1}^T  \mathbbm{1}(i \in S_t) \geq \ell_i, \;\;\; &&\forall i \in \mathcal{N}, \\ & |S_t| \leq k, && \forall t \in [T].
\end{aligned}
\end{equation}






First, we study the complexity of \ref{APVC}. In the next theorem, we show that \ref{APVC} is strongly NP-Hard.
\begin{theorem} \label{NP-hardness} \ref{APVC} is strongly NP-hard, even when all prices $p_i$ are equal. Moreover, there is no FPTAS for \ref{APVC}, even with equal prices.
\end{theorem}
The proof of this theorem relies on a reduction of the \texttt{3-PARTITION} problem, where the objective is to partition a set of cardinality $3T$ into $T$ subsets, each of size $3$, such that the sum of elements within each subset is identical. The full proof of this reduction is deferred to Appendix \ref{apxhard}.

Relying on the strong NP-hardness of this problem, we show that the existence of a Fully Polynomial Time Approximation Scheme (FPTAS) is precluded, unless ${P}={NP}$. Therefore, we focus in the remainder of this section on the question of designing a Polynomial Time Approximation Scheme (PTAS) for the problem, in the case of equal prices. This setting corresponds to a sales maximization problem, which is particularly useful in the case where the platform or the company's goal is to maximize the captured portion of customers.

\begin{theorem} \label{thm:PTAS}
There exists a PTAS for \ref{APVC} with equal prices.
\end{theorem}
The remainder of this section focuses on proving Theorem \ref{thm:PTAS}, by devising a PTAS for \ref{APVC} with equal weights. The proof is organized as follows. In Section \ref{subsec:discret}, we consider an instance of \ref{APVC} with equal prices, which we reduce to a ``friendlier" instance of the problem by slightly altering the weights of a particular subset of products. Importantly, we show that we only incur an $\eps$ loss due to this reduction, and therefore only focus on solving the modified instance. In Section \ref{subsec:linearsol}, we linearize our optimization problem through a carefully crafted guessing procedure, then we provide a method to obtain approximate solutions for the linearized program, thereby completing the formulation of our PTAS. Finally, we show in Section \ref{subsec:analysis} that the solution yielded by our algorithm is indeed a $(1-\eps)$-approximation for our problem.
\subsection{Discretizing the universe of products}\label{subsec:discret}
The setting with identical prices is formally defined as follows
\begin{equation}
\label{eq:SPVC}\tag{\sf{SPVC}}
\begin{aligned}
 \max_{ S_1, \ldots, S_T \subseteq \mathcal{N}} & \; \; \sum_{t=1}^T\frac{V(S_t)}{1+V(S_t)}  \\ s.t. \;\; &\sum_{t=1}^T  \mathbbm{1}(i \in S_t) \geq \ell_i, \;\;\; &&\forall i \in \mathcal{N}, \\ & |S_t| \leq k, && \forall t \in [T].
\end{aligned}
\end{equation}

In this section, we reduce the original instance of our problem into a modified instance, by rounding down the weights of a particular subset of products in our original universe. Then, we show that we incur at most an $\eps$-loss by performing this reduction.
For any product $i\in \Nc$,\begin{itemize}
    \item If $v_i\geq 1/\eps$, then $v_i$ is rounded down to $v_i^{\downarrow} = 1/\eps$. We denote the product associated with the weight $v_i^{\downarrow}$ simply by $i^{\downarrow}$.
    \item If $v_i< \eps^5$, then $v_i$ remains unchanged and we define $v_i^\downarrow = v_i$. Similarly, the associated product is denoted by $i^{\downarrow}$.
    \item If $\eps^5\leq v_i < 1/\eps$, then $v_i$ is rounded down such that $v_i/\eps^5$ meets the nearest power of $(1+\eps)$. In other words, since $\eps^5\leq v_i < 1/\eps$, there exists some $q\geq 1$ such that $\eps^5\cdot(1+\eps)^{q-1} \leq v_i < \eps^5\cdot(1+\eps)^{q}$, and $v_i$ is rounded down to $v_i^\downarrow = \eps^5\cdot(1+\eps)^{q-1}$. We denote the associated product by $i^{\downarrow}$. In particular, note that by this definition, we have \begin{equation}\label{eq:weightbound}
        v_i^\downarrow \leq v_i\leq (1+\eps)\cdot v_i^\downarrow.
    \end{equation}
\end{itemize}
Let us denote this new universe of products by $\Nc^\downarrow = \{i^\downarrow\,\colon\,i\in \Nc\}$. We invite the reader to think of ${.}^\downarrow$, when applied to a product, as a one-to-one map from the universe $\Nc$, to the modified universe $\Nc^\downarrow$, where a subset of weights have been rounded down.
In the subsequent result, we show that solving a modified instance of \ref{eq:SPVC} using the products from $\Nc^\downarrow$ yields a $(1-\eps)$-approximation of the solution on the original instance. To this purpose, we define the following new instance of \ref{eq:SPVC}, which only uses products from the universe $\Nc^\downarrow$:
\begin{equation}
\label{eq:SPVCd}\tag{{\sf SPVC}$^\downarrow$}
\begin{aligned}
 \max_{ S_1, \ldots, S_T \subseteq \mathcal{N}^\downarrow} & \; \; \sum_{t=1}^T\frac{V(S_t)}{1+V(S_t)}  \\ s.t. \;\; &\sum_{t=1}^T  \mathbbm{1}(i^\downarrow \in S_t) \geq \ell_i, \;\;\; &&\forall i^\downarrow \in \mathcal{N}^\downarrow, \\ & |S_t| \leq k, && \forall t \in [T].
\end{aligned}
\end{equation}

Let $\opt$ (resp. $\opt^\downarrow$) denote the objective of an optimal solution of \ref{eq:SPVC} (resp. \ref{eq:SPVCd}).
In the following, we show that $\opt$ and $\opt^\downarrow$ are within a $1-\eps$ fraction of one another. To this purpose, we show a stronger result in Lemma \ref{lem:epsobjective}. Indeed, for any sequence of assortments $S_1,\ldots, S_t$, we define\begin{equation*}
    \obj(S_1,\ldots, S_T) = \sum_{t=1}^T\frac{V(S_t)}{1+V(S_t)}.
\end{equation*}
For any assortment $S\subseteq \Nc$, we denote by $S^\downarrow$ its rounded counterpart, defined as $S^\downarrow = {\{i^\downarrow\,\colon\, i \in S\}}$. In the next lemma, we show that the objective achieved by any sequence of $T$ assortments in $\Nc$ is within $1-\eps$ of their rounded counterpart. In particular, letting $\opt$ denote the value of \ref{eq:SPVC} (i.e., the objective of an optimal feasible sequence of assortments), and similarly $\opt^\downarrow$ denote the value of \ref{eq:SPVCd}, this implies that $\opt$ and $\opt^\downarrow$ are also within $1-\eps$.

\begin{lemma}\label{lem:epsobjective}
    For any sequence of assortments $S_1,\ldots,S_T\subseteq \Nc$, we have $$\obj(S_1^\downarrow,\ldots, S_T^\downarrow)\leq \obj(S_1,\ldots, S_T)\leq (1+\eps)\cdot\obj(S_1^\downarrow,\ldots, S_T^\downarrow).$$ Consequently, we have $\opt^\downarrow\leq \opt\leq (1+\eps)\cdot \opt^\downarrow$.
\end{lemma}

\begin{proof}[Proof of Lemma \ref{lem:epsobjective}]
    First, since all the weights of the products in $\Nc$ are either rounded down or remain the same, then for all $t\in [T]$, $V(S_t^\downarrow)\leq V(S_t)$. Consequently, by the monotonicity of the map $x\mapsto x/(1+x)$, we have $V(S_t^\downarrow)/(1+V(S_t^\downarrow))\leq V(S_t)/(1+V(S_t))$. By summing over $t$, we obtain the first inequality $\obj(S_1^\downarrow,\ldots, S_T^\downarrow)\leq \obj(S_1,\ldots, S_T)$. Let us prove the second inequality. We show that $V(S_t)/(1+V(S_t)) \leq (1+\eps)\cdot V(S_t^\downarrow)/(1+V(S_t^\downarrow))$, for every $t \in [T]$. Let $t \in [T]$.\begin{itemize}
        \item If there exists a product $i\in S_t$ such that $v_i\geq 1/\eps$, then $v_i^\downarrow = 1/\eps$, and in turn, $V(S_t^\downarrow)\geq 1/\eps$. Therefore, \begin{equation*}
            (1+\eps)\frac{V(S_t^\downarrow)}{1+V(S_t^\downarrow)} \geq (1+\eps) \cdot \frac{1/\eps}{1+1/\eps} = 1 \geq \frac{V(S_t)}{1+V(S_t)}.
        \end{equation*}
        \item If for all $i\in S_t$, we have $v_i< 1/\eps$. Note that Equation \eqref{eq:weightbound} holds trivially when $v_i<\eps$, since $v_i^\downarrow = v_i$ by definition. Therefore, Equation \eqref{eq:weightbound} holds for every $i\in S_t$. Hence,\begin{equation*}
            \frac{v_i}{1+v_i} \leq \frac{(1+\eps)\cdot v_i^\downarrow}{1+v_i^\downarrow}. 
        \end{equation*}
    \end{itemize}
    The result follows by summing over $t = 1,\ldots,T$.
    
Noticing that if $S_1,\ldots, S_T$ is feasible for \ref{eq:SPVC}, then its rounded counterpart is also feasible for \ref{eq:SPVCd}, we also have $\opt^\downarrow\leq \opt\leq (1+\eps)\cdot \opt^\downarrow$.

    
\end{proof}
In light of Lemma \ref{lem:epsobjective}, we deduce that obtaining an approximation to the reduced problem \ref{eq:SPVCd} automatically yields an approximation to the original problem \ref{eq:SPVC}, with only an additional $\eps$ loss. In the remainder of this section, we will only focus on solving the reduced problem \ref{eq:SPVCd}, therefore, for simplicity of notation, we denote the products $1^\downarrow,\ldots,n^\downarrow$ simply by $1,\ldots, n$ respectively, and we denote their respective weights simply by the original $v_1,\ldots,v_n$ instead of the heavy notation $v_1^\downarrow, \ldots, v_n^\downarrow$. So, products $1,\ldots, n$ and their respective weights $v_1,\ldots, v_n$ will refer to those of the modified instance \ref{eq:SPVCd}.







\subsection{Linearization of the objective function and algorithm}\label{subsec:linearsol}
In this section, we aim at linearizing the optimization problem \ref{eq:SPVC}, through a guessing procedure of a certain number of parameters pertaining to the optimal solution. We start by presenting our detailed guessing procedure in Section \ref{subsec:guessing}, before leveraging the guessed parameters to construct a linearized formulation in Section \ref{subsec:linearprog}. In Section \ref{subsec:GKPS}, we present a dependent rounding scheme introduced in \cite{gandhi2002dependent}, which we use later in Section \ref{subsubsec:bipartite} to round a fractional solution of the linearized program, into a feasible integral solution for \ref{eq:SPVCd}. 
\subsubsection{Guessing procedure}\label{subsec:guessing}
 Let $S_1^*,\ldots, S_T^*$ be an (unknown) optimal solution for our problem \ref{eq:SPVCd}.
\paragraph{Customer types.}In this definition, we aim at categorizing customers with respect to the total weight of the products offered to them in an optimal solution. First we say that a customer $t\in [T]$ is {\em light} if $V(S_t^*) < \eps$. Similarly, we say that a customer is {\em heavy} if $V(S_t^*)\geq 1/\eps$. Otherwise (i.e., if $\eps\leq V(S_t^*)<1/\eps$), we say that the customer is {\em medium}. We further partition medium customers into $L$ classes $G_1, \ldots, G_L$, depending on the value of $V(S_T^*)$. Specifically, we define every class $\ell =1,\ldots,L$ as follows,\begin{equation*}
    G_{\ell} = \left\{ t\text{ medium }\,\colon\, \eps\cdot (1+\eps)^{\ell-1}\leq V(S_t^*)< \eps\cdot (1+\eps)^{\ell}\right\}.
\end{equation*}
We fix $L$ to be the smallest integer such that $\eps\cdot(1+\eps)^{L}\geq 1/\eps$. In particular, $L = O(\frac{1}{\eps}\log(\frac{1}{\eps}))$. Therefore, the classes $G_1,\ldots, G_L$ form a partition of medium customers. Additionally, we denote by $G_{light}$ and $G_{heavy}$ the sets of light and heavy customers respectively. 

\paragraph{Packing patterns.}First let us start by defining classes of products. Recall from Section \ref{subsec:discret} that for every product $i\in \Nc$ such that $v_i \geq \eps^5$, either there exists some $1\leq q\leq Q-1$ such that $v_i = \eps^5\cdot(1+\eps)^{q-1}$ or $v_i = 1/\eps$, where $Q$ is the smallest integer such that $\eps^5\cdot (1+\eps)^{Q-1}\geq 1/\eps$. Let $D_q$ denote the set of all the products of $\Nc$ such that $v_i = \eps^5\cdot (1+\eps)^{q-1}$, for every $1\leq q\leq Q-1$, and let $D_Q$ denote the set of all products of $\Nc$ such that $v_i = 1/\eps$. Similarly, let $D_0$ be the class of all products $i\in \Nc$ whose weight $v_i<\eps^5$. Note that by construction, the classes $D_0, \ldots D_Q$ form a partition of the universe of products $\Nc$. A packing pattern is a $(Q+1)$-dimensional vector, taking values in $\{0,1,2,\ldots, 1/\eps^6, \star\}$, where $1/\eps^6$ is assumed to be an integer without loss of generality. For a given customer, each entry $q=0,\ldots,Q$ of its associated packing pattern characterizes the number of products from class $D_q$ in the assortment $S_t^*$, and $\star$ should be viewed simply as a symbol which characterizes the fact that the number of products from class $q$ is (strictly) greater than $1/\eps^6$.
Using this definition, the number of packing patterns is  $$\left(\frac{1}{\eps^6}+2\right)^{Q+1} = 2^{O\left(\left(Q+1\right)\log\left(\frac{1}{\eps}\right)\right)}   =2^{O\left(\frac{1}{\eps}\log^2\left(\frac{1}{\eps}\right)\right)}.$$ 
% \paragraph{Guessing. } With all the previous definitions, we are now able to present the guessing procedure. We first start by guessing the number of customers of each type in the optimal solution $S_1^*, \ldots, S_T^*$, i.e., we guess the cardinalities $|G_{light}|,|G_{heavy}|, |G_1|,\ldots ,|G_L|$. Then for each customer type $\ell \in \{\text{light}, \text{heavy}, 1,\ldots, L\}$, and every given packing pattern  $P = (p_0, \ldots, p_Q)$, we guess the number of customers of type $\ell$, who use the packing pattern $P$, which we denote by $K_{\ell, P}$. 


\paragraph{Guessing. } With all the previous definitions, we are now able to present the guessing procedure. For each couple consisting of a class of customers $\ell \in \{\text{light}, \text{heavy}, 1,\ldots, L\}$, and a packing pattern $P$, we guess the number of customers in $G_{\ell}$ that use the packing pattern $P$. Let us call this guessed number $K_{\ell,P}$.

It is important to note that using the above guessing procedure allows us to determine, for any given customer $t\in [T]$, both her type and her packing pattern in the optimal solution, up to a permutation of $\{1,\ldots, T\}$. Indeed, since for each class $\ell\in \{\text{light}, \text{heavy}, 1,\ldots, L\}$, we know the number of customers that use each packing pattern, in particular, we know the quantity $|G_\ell|$. Hence, due to the fact that all customers are initially interchangeable, we arbitrarily assign each customer to a type according to our guess, without loss of generality. Subsequently, given our guess $K_{\ell, P}$ for every customer type $\ell$, and every packing pattern $P$, and since the customers within the same type are all interchangeable a priori, we arbitrarily assign each customer to a packing pattern, according to our guess, without loss of generality. Therefore, for each customer $t\in [T]$, we have determined both her type and her packing pattern, up to a permutation of $\{1,\ldots,T\}$.
\paragraph{Complexity of the guessing procedure.}For any couple consisting of a specific class of customers and a specific packing pattern, we guess the number of customers from said class who use said packing pattern. There are $L+2 = O(\frac{1}{\eps}\log(\frac{1}{\eps}))$ classes of customers, and $O(2^{O(\frac{1}{\eps} \log^2(\frac{1}{\eps}))})$ packing patterns. The number of couples is therefore given by $O(2^{O(\frac{1}{\eps} \log^2(\frac{1}{\eps}))})$.  Since there are $T$ customers, there are $T+1$ possible guesses for each couple consisting of a class of customers and a packing pattern. Therefore, the total number of guesses is $O(T^{2^{O(\frac{1}{\eps}\log^2(\frac{1}{\eps}))}})$.
\subsubsection{Integer linear programming formulation}\label{subsec:linearprog}
Following the detailed presentation of the guessing procedure, we are now ready to leverage the guessed parameters in order to introduce our integer linear programming formulation \ref{eq:ILP}. This formulation is crucial to our algorithm, since its approximate solution is later (Section \ref{subsec:analysis}) shown to be near optimal for \ref{eq:SPVCd}. In \cite{segev2021efficient}, the authors use a similar idea in order to obtain rounded solutions for the Santa Claus problem.

Let us start with some useful notation.
For each customer $t$, we define the lower bound $V_t$ on the $V(S_t^*)$ as follows. If $t$ is light (resp. heavy) then $V_t = 0$ (resp. $V_t = 1/\eps$), and if $V_t$ is of type $\ell$ for some $\ell = 1,\ldots, L$, then $V_t = \eps\cdot (1+\eps)^{\ell-1}$. Finally, we denote the packing pattern of customer $t$ by $(k_{0,t}, k_{1,t},\ldots, k_{Q,t})$.
Using these newly defined parameters, we define the following integer linear program:

\begin{equation}\label{eq:ILP}\tag{ILP}
 \begin{array}{rrclcl}
\displaystyle \max_{\mathbf{x}} & \multicolumn{3}{l}{\displaystyle\sum_{t\in G_{light}}\sum_{i\in \Nc}v_ix_{it}} \\
\textrm{s.t.} &\displaystyle\sum_{t=1}^T x_{it}& \geq & \ell_i,& & \forall i\in \Nc ,\\
&\displaystyle\sum_{i\in \Nc} x_{it}& \leq & k,& & \forall t\in [T],\\
&\displaystyle \sum_{i\in D_q}x_{it} & = & k_{q,t}, & & \forall t\in [T],\,\forall q=0,\ldots,Q \text{ such that } k_{q,t}\neq \star,\\
&\displaystyle \sum_{i\in D_q}x_{it} & > & \displaystyle\frac{1}{\eps^6}, & & \forall t\in [T],\,\forall q=0,\ldots,Q \text{ such that } k_{q,t}= \star,\\
&\displaystyle \sum_{i\in \Nc} v_i x_{it} & \geq & V_{t}, & & \forall t\in [T], \\
&\displaystyle \sum_{i \in \Nc} v_ix_{it} & \leq & \eps, & & \forall t\in G_{light}, \\
& x_{it} & \in & \{0,1\}, & & \forall i\in \Nc,\, \forall t\in [T].
\end{array}
\end{equation}
The last step is to present a way to approximate \ref{eq:ILP}. To this purpose, the high level idea of our method is to start by solving the linear relaxation of \ref{eq:ILP}, then to use a specific rounding of the obtained fractional solution, in order to derive a near-optimal integral solution.

Let $\mathbf{x^*}$ be an optimal solution to the linear relaxation of \ref{eq:ILP}, that is, to the linear program obtained by replacing the constraint $x_{it}\in \{0,1\}$ by the relaxed constraint $x_{it} \in [0,1]$. Similarly, let $S_1^*,\ldots, S_T^*$ be an optimal sequence of assortments for \ref{eq:SPVCd}. The following lemma relates $\mathbf{x^*}$ to the optimal solution $S_1^*,\ldots, S_T^*$.
\begin{lemma}\label{lem:LPopt}
    $$
        \sum_{t=1}^T\frac{\sum_{i\in \Nc} v_ix^*_{it}}{1+\sum_{i\in \Nc} v_ix^*_{it}} \geq (1-\eps)\cdot\sum_{t=1}^T\frac{V(S_t^*)}{1+V(S_t^*)}.
    $$
\end{lemma}
The proof of this Lemma is included in Appendix \ref{apx:LPopt}. The next step is to round $\mathbf{x^*}$ in order to obtain a feasible solution for \ref{eq:ILP}.

\subsubsection{The dependent rounding scheme}\label{subsec:GKPS}
In this section, we briefly present the dependent rounding scheme in \cite{gandhi2002dependent}, which we use to round our fractional solution into an integral one. Given a bipartite graph $(L,R,E)$ and values $x_{i,j}\in [0,1]$ for every edge $(i,j)\in E$, the authors in \cite{gandhi2002dependent} present a polynomial time rounding scheme which returns a sequence of random variables $X_{i,j}\in \{0,1\}$ representing a rounding of the values $x_{i,j}$, and which verify the following properties:
\setenumerate[1]{label={\bf (P\arabic*)}}
\begin{enumerate}
    \item\label{it:P1} {\bf Marginal distributions:} For every edge $(i,j)$, $\E(X_{i,j}) = x_{i,j}$.
    \item\label{it:P2} {\bf Degree-preservation:} The fractional degree of every edge is rounded to its floor or its ceiling, i.e., for any $i\in L\cup R$, if $d_i = \sum_{j\in L\cup R}x_{ij}$, then $\sum_{j\in L\cup R}X_{ij}\in\{\lfloor d_i\rfloor, \lceil d_i\rceil\}$.
    \item\label{it:P3} {\bf Negative correlation:} For any node $i\in L\cup R$, and any subset $S\subseteq N(i)$, where $N(i)$ if the set of neighbors of $i$, we have\begin{align*}
        &\P\left[\bigwedge_{j\in S}(X_{i,j}=0)\right]\leq \prod_{j\in S}\P\left[X_{i,j} = 0\right]\\
        \text{and}\quad \quad&\P\left[\bigwedge_{j\in S}(X_{i,j}=1)\right]\leq \prod_{j\in S}\P\left[X_{i,j} = 1\right].
    \end{align*}
\end{enumerate}
Negative correlation {\bf (P3)} is a strong property that allows us to derive Chernoff bounds, as stated in the following lemma.
\begin{lemma}[{\cite[Theorem~1.10.24][66]{doerr2020probabilistic}}]\label{lem:chernoff}
    If $X_1,\ldots,X_n$ are negatively correlated random variables, then for any vector $a=(a_1,\ldots,a_n)\in [0,1]^n$ we have$$
        \P[a(X)\leq (1-\eps)\cdot \E[a(X)]] \leq \exp\left(-\frac{\eps^2\cdot \E[a(X)]}{2}\right),
    $$
    where $a(X) = \sum_{i=1}^na_iX_i$.
\end{lemma}
Next, we construct a framework that allows us to use the dependent rounding scheme in order to derive a feasible solution for \ref{eq:ILP}.


\subsubsection{Construction of the bipartite graph and feasibility of the rounded solution}\label{subsubsec:bipartite}
Recall that the high level idea is to use the rounding procedure presented in \ref{subsec:GKPS}, in order to derive an integer solution to \ref{eq:ILP}. Noticing that the rounding procedure takes as input a bipartite graph and a sequence of edge-associated values, the first step is to interpret the solution of the linear relaxation of \ref{eq:ILP} as a sequence of edge-associated values in a carefully constructed bipartite graph.
Before presenting our proposed bipartite graph, let us introduce the two following definitions. We say that a customer $t\in [T]$ is {\em bounded} if $k_{q,t} \neq \star$, for all $q=1,\ldots, Q$ (note that $q$ starts from $1$ in this definition and not from $0$). Otherwise, we say that customer $t$ is {\em unbounded}. In other words, customer $t$ is bounded if she is not offered more than $1/\eps^5$ products from any class $D_q$ where $q\geq 1$, in the optimal solution.
\paragraph{The bipartite graph.}
We start by defining the node sets $L$ and $R$. The set $L$ is simply the set of all products in $\Nc$. Let us now describe the set of right nodes $R$. First, for every unbounded customer $t$, we introduce a node $t$ in the set $R$. Alternatively, for every bounded customer $t$, we introduce $Q+1$ nodes in the set $R$, specifically, one node $(t,q)$ for every class of products $q=0,\ldots, Q$. Therefore, $$R = \{t\,:\, t\text{ unbounded}\} \cup \left\{(t,q)\,\colon \, t \text{ bounded}, q =0,\ldots,Q\right\}.$$ Let us now describe the edge set $E$ of the bipartite graph, as well as the associated value $x_e$ of each edge $e\in E$. Let us fix a node $i\in L$ and describe all of its adjacent edges and associated values. First, $i$ neighbors every node $t$ where $t$ is an unbounded customer, and the edge $(i,t)$ is associated with the value $x^*_{it}$. Next, assume $q$ is the class of products containing $i$, then the node $i$ neighbors every node $(t,q)$ for $t\in [T]$, and the edge $(i, (t,q))$ is associated with the value $x^*_{it}$. This completes our description of the bipartite graph.
\paragraph{PTAS.} Let us summarize the steps of our PTAS. The first step is to perform the guessing procedure presented in Section \ref{subsec:guessing}, and then use our guess to construct \ref{eq:ILP}. The second step is to solve the linear relaxation of \ref{eq:ILP} using standard linear programming techniques, and obtain a fractional solution $\mathbf{x}^*$. The third step is to use $\mathbf{x}^*$ to construct the bipartite graph as described above. We can now apply the dependent rounding presented in Section \ref{subsec:GKPS} to derive a sequence of random variables $(X_{it}\,\colon\, i\in \Nc, t\in[T])$ representing rounded values. Finally, we denote by $S_1,\ldots, S_T$ the associated sequence of assortments, that is, for each $t\in [T]$, \begin{equation*}
        S_t= \{i \in \Nc \,\colon\, X_{it}=1\},
    \end{equation*}
and the algorithm returns this sequence $(S_1, \ldots, S_T)$.
\paragraph{Complexity analysis.}Since our algorithm is exhaustive, the steps presented above are applied for every possible guess. Solving the linear relaxation of \ref{eq:ILP} takes $O(n^3T^3)$ using standard linear programming algorithms (interior point method for example). Constructing the bipartite graph takes $O(nTQ)$ running time, and running the dependent rounding scheme takes $O((nTQ)^3)$. Therefore, processing each guess takes $O(n^3T^3Q^3) = O(n^3T^3\frac{1}{\eps^3}\log^3{\frac{1}{\eps}})$ running time. Finally, since there are $O(T^{2^{O(\frac{1}{\eps}\log^2(\frac{1}{\eps}))}})$ possible guesses as argued in Section \ref{subsec:guessing}, the total running time of our algorithm is:
$$
O\left(T^{2^{O\left(\frac{1}{\eps}\log^2\left(\frac{1}{\eps}\right)\right)}} \cdot n^3T^3\frac{1}{\eps^3}\log^3{\frac{1}{\eps}}\right) = O\left(n^3T^{2^{O\left(\frac{1}{\eps}\log^2\left(\frac{1}{\eps}\right)\right)}}\right),
$$
which is polynomial in the size of the input for any fixed $\eps$, confirming that our algorithm is indeed a PTAS.












\subsection{Near-optimality of the rounded solution}\label{subsec:analysis}
In this section, we show that the obtained solution $(S_1,\ldots,S_T)$ is near-optimal for \ref{eq:SPVCd}. The main result is stated in the following theorem.
% \begin{algorithm}\label{alg:PTASd}
% \caption{Polynomial-time approximation scheme for \ref{eq:SPVCd}}
% \begin{algorithmic}[1]
% \REQUIRE $\Nc$, $T$, $v_i$ for every $i\in \Nc$
% \ENSURE Sum of integers from 1 to $N$
% \STATE $S_t^* \leftarrow \emptyset$ for $t\in [T]$
% \STATE $A \leftarrow 0$
% \FOR{every guess of customer types and packing patterns}
%     \STATE Solve the linear relaxation of \ref{eq:ILP}
%     \STATE Use the the rounding in Section \ref{subsec:GKPS} to compute $S_1,\ldots, S_T$
%     \STATE $A \leftarrow \obj(S_1,\ldots,S_T)$
%     \IF{$A \geq A^*$}
%         \STATE $S_t^* \leftarrow S_t$ for $t\in [T]$
%         \STATE $A^* \leftarrow A$
%     \ENDIF
% \ENDFOR
% \RETURN $A^*$, $(S_t^*\,\colon\,t\in [T])$
% \end{algorithmic}
% \end{algorithm}

\begin{theorem}
    The sequence of assortments $S_1,\ldots, S_T$ is feasible for \ref{eq:SPVCd} and we have
    \begin{equation*}
        \E\left[\sum_{t=1}^T \frac{V(S_t)}{1+V(S_t)} \right]\geq (1-3\eps) \cdot \opt^\downarrow.
    \end{equation*}
\end{theorem}

\begin{proof}
    In this proof, we start by showing that the sequence $S_1, \ldots,S_T$ is feasible for \ref{eq:SPVCd}. Next, we show that the objective of the sequence of assortments $S_1,\ldots,S_T$ is within $1-3\eps$ of the optimal objective. Recall that $S_1^*,\ldots, S_T^*$ is an optimal sequence of assortments for \ref{eq:SPVCd}.
    
    \paragraph{Feasibility. }First, the visibility constraint is straightforward. Indeed, let $i\in \Nc$. By the first constraint of the linear relaxation of \ref{eq:ILP}, we have: $\sum_{t=1}^Tx_{it}^* \geq \ell_i$. Noticing that $\sum_{t=1}^Tx_{it}^*$ is the fractional degree of node $i$ in the bipartite graph, we have by property \ref{it:P2}, $\sum_{t=1}^TX_{it} \geq \lfloor\ell_i\rfloor = \ell_i$, which proves that the visibility constraint is respected. We now show that the cardinality constraint is respected after the rounding. Let $t\in [T]$. First, if $t$ is unbounded, then the cardinality of $S_t$ is simply given by the degree of node $t$. Since $\sum_{i=1}^nx_{it}^* \geq k$, then by property \ref{it:P2}, we have $|S_t| = \sum_{i\in \Nc}X_{it} \geq k$. Alternatively, if $t$ is bounded, for every class $q=1,\ldots,Q$, the degree of node $(t,q)$ is $\sum_{i\in D_q}
    x_{it}^*$ which is equal to $k_{q,t}$, according to the third constraint of \ref{eq:ILP}. Since $k_{q,t}$ is an integer, the degree of node $(t,q)$ remains unchanged after the rounding, according to property \ref{it:P2}. Therefore, we have almost surely:\begin{align*}
        \sum_{i\in \Nc}v_i X_{it}&= \sum_{i\in D_0}v_iX_{it} + \sum_{q=1}^Q\sum_{i\in D_q}v_iX_{it} \\ &\leq \left\lceil\sum_{i\in D_0}v_ix_{it}^*\right\rceil+ \sum_{q=1}^Q k_{q,t} \\ &= \left\lceil\sum_{i\in D_0}v_ix_{it}^*+ \sum_{q=1}^Q k_{q,t} \right\rceil\\ &= \left\lceil\sum_{i\in \Nc}v_ix_{it}^* \right\rceil\\
        &\leq \lceil k\rceil=k.
    \end{align*}
    The first inequality follows from Property \ref{it:P2}. In the equality in the third line, we use the fact that the second summand is an integer. This proves the cardinality constraint and concludes the feasibility proof. 
    \paragraph{Near-optimality. }
    We show that the contribution of every customer $t\in [T]$ to the objective function, i.e., the quantity $V(S_t)/(1+V(S_t))$ is within $1-2\eps$ of the contribution of $t$ in the optimal solution $\mathbf x^*$, in expectation.
    Let $t \in [T]$.
    \\ \noindent {\em \underline{Case 1}: If $t$ unbounded.} Let $q_t \in \{1,\ldots,Q\}$ be the class of products such that $k_{q_t,t} = \star$. We have \begin{equation*}
        \sum_{i\in D_{q_t}} v_i x_{it}^* \geq \eps^5\cdot \sum_{i\in D_{q_t}} x_{it}^* \geq \eps^5\cdot \frac{1}{\eps^6} = \frac{1}{\eps}.
    \end{equation*}
Therefore, 
    \begin{align}
        \P\left[V(S_t) \leq  \frac{1-\eps}{\eps}\right] &= 
        \P\left[\sum_{i\in \Nc}v_i X_{it} \leq  \frac{1-\eps}{\eps}\right] \notag\\& \leq \P\left[\sum_{i\in \Nc}v_i X_{it} \leq  (1-\eps)\cdot \sum_{i\in D_{q_t}} v_i x_{it}^*\right]\notag\\
        & \leq \P\left[\sum_{i\in D_{q_t}}v_i X_{it} \leq  (1-\eps)\cdot \sum_{i\in D_{q_t}} v_i x_{it}^*\right]\notag\\
        & = \P\left[\sum_{i\in D_{q_t}}X_{it} \leq  (1-\eps)\cdot \sum_{i\in D_{q_t}}x_{it}^*\right]\notag\\
        & \leq \exp\left(-\frac{\eps^2\cdot \sum_{i\in D_{q_t}}x_{it}^*}{2}\right)\notag\\
        & \leq \exp\left(-\frac{1}{2\eps^4}\right)\notag\\
        &\leq \eps.\label{eq:unboundbound}
    \end{align}
    In the second equality, we use the fact that products within the same class have the same preference weight to simplify all the weights. The inequality in the fifth line is a direct application of Lemma \ref{lem:chernoff}, and in the one in the sixth line, we use the fact that $\sum_{i\in D_{q_t}} x_{it}^*> 1/\eps^6$, since $k_{q_t, t} = \star$.
    Therefore, by conditioning on the event $\{V(S_t)>(1-\eps)/\eps\}$, and neglecting one of the terms, we have\begin{align*}
        \E\left[\frac{V(S_t)}{1+V(S_t)}\right] &\geq \P\left[V(S_t)> \frac{1-\eps}{\eps}\right] \cdot  \E\left[\frac{V(S_t)}{1+V(S_t)} \,\mid\,V(S_t)> \frac{1-\eps}{\eps}\right]\\
        &\geq (1-\eps)\cdot \frac{\frac{1-\eps}{\eps}}{1+\frac{1-\eps}{\eps}}\\
        &=(1-\eps)^2\\
        &\geq (1-2\eps)\cdot \frac{\sum_{i\in \Nc}v_ix_{it}^*}{1+\sum_{i\in \Nc}v_ix_{it}^*}.
    \end{align*}
    The second inequality follows from Equation \eqref{eq:unboundbound} and the monotonicity of $x\mapsto x/(1+x)$ on $[0,+\infty)$.


    \noindent{\em \underline{Case 2}: If $t$ is bounded and heavy or if $t$ is medium. }Then we have\begin{equation*}
    \sum_{i\in \Nc}v_iX_{it} = \sum_{i\in D_0}v_iX_{it} + \sum_{q = 1}^{Q-1}\sum_{i\in   D_q}v_iX_{it}.
\end{equation*}
    Let $W_t^{\lar} = \sum_{i\in \Nc\setminus D_0}v_iX_{it}$, and $W_t^{\sma} = \sum_{i\in D_0}v_iX_{it}$. Similarly, let $w_t^{\lar} = \sum_{i\in \Nc\setminus D_0}v_ix^*_{it}$, and $w_t^{\sma} = \sum_{i\in D_0}v_ix^*_{it}$.
\begin{claim}\label{cl:nonlightcust}
    With probability $1$, we have $W_t^{\lar} = w_t^\lar.$
        % \sum_{q = 1}^{Q-1}\sum_{i\in   D_q}v_iX_{it} \geq \sum_{q = 1}^{Q-1}\sum_{i\in   D_q}v_ix_{it}^*.

\end{claim}
The proof of this result is a direct application of the degree preservation property, and is deferred to Appendix \ref{apx:nonlightcust}.
\begin{itemize}
    \item If $w_t^{\sma}\leq \eps \cdot\sum_{i\in \Nc} v_ix_{it}^*$, then intuitively, the contribution of the products in $D_0$ represent an $\eps$-fraction of the total contribution and can therefore be neglected. Formally, we have with probability $1$
    $$V(S_t)\geq W_t^{\lar}  = w_t^{\lar}
         =  \sum_{i\in \Nc}v_i x_{it}^* - w_t^\sma\geq (1-\eps)\cdot \sum_{i\in \Nc}v_ix_{it}^*.$$
    % \begin{align*}
    %     V(S_t)&\geq W_t^{\lar} \\& = w_t^{\lar}\\
    %     & =  \sum_{i\in \Nc}v_i x_{it}^* - w_t^\sma\\
    %     &\geq (1-\eps)\cdot \sum_{i\in \Nc}v_ix_{it}^*.
    % \end{align*}
    The first equality is a consequence of Claim \ref{cl:nonlightcust}, and the second inequality follows from the case hypothesis.
    Therefore, by the monotonicity of $x\mapsto x/(1+x)$, we have with probability $1$\begin{equation}
        \frac{V(S_t)}{1+V(S_t)} \geq (1-\eps)\cdot \frac{\sum_{i\in \Nc}v_i x_{it}^*}{1+\sum_{i\in \Nc}v_i x_{it}^*}.
    \end{equation}
    \item Otherwise, we have
    \begin{align}
        \P\left[W_t^\sma \leq (1-\eps)\cdot w_t^\sma\right] & = \P\left[\sum_{i\in D_0}\frac{v_i}{\eps^5}X_{it} \leq (1-\eps)\cdot \sum_{i\in D_0}\frac{v_i}{\eps^5}x_{it}^*\right]\notag\\
        &\leq \exp\left(-\frac{\eps^2\cdot \sum_{i\in D_0}v_ix_{it}^*}{2\eps^5}\right)\notag\\
        &\leq \exp\left(-\frac{\eps \cdot \sum_{i\in \Nc}v_ix_{it}^*}{2\eps^3}\right)\notag\\
        &\leq \exp\left(-\frac{1}{2\eps}\right)\notag\\
        &\leq \eps.\label{eq:boundbound}
    \end{align}
    The first inequality follows from Lemma \ref{lem:chernoff}. The second inequality is a consequence of the case hypothesis, and third one follows from the fact that $t$ is not light, and hence that $\sum_{i\in \Nc}v_ix_{it}^* \geq \eps$.
    Therefore, by conditioning on the event $\{W_t^\sma\leq (1-\eps)\cdot w_t^\sma\}$ and neglecting one of the terms, we have\begin{align*}
        \E\left[\frac{V(S_t)}{1+V(S_t)}\right]
        &\geq \P\left[W_t^{\sma}> (1-\eps)\cdot w_t^{\sma}\right] \cdot  \E\left[ \frac{V(S_t)}{1+V(S_t)}\,\mid\,W_t^{\sma}> (1-\eps)\cdot w_t^{\sma}\right]\\
        &= \P\left[W_t^{\sma}> (1-\eps)\cdot w_t^{\sma}\right] \cdot  \E\left[ \frac{w_t^\lar+ W_t^\sma}{1+w_t^\lar+ W_t^\sma}\,\mid\,W_t^{\sma}> (1-\eps)\cdot w_t^{\sma}\right]\\
        &\geq (1-\eps) \cdot \frac{w_t^\lar+ (1-\eps)\cdot w_t^s}{1+w_t^\lar+(1-\eps)\cdot w_t^s}\\
        &\geq (1-\eps)^2.\frac{\sum_{i\in \Nc}v_ix^*_{it}}{1+\sum_{i\in \Nc}v_ix^*_{it}}\\
        & \geq (1-2\eps).\frac{\sum_{i\in \Nc}v_ix^*_{it}}{1+\sum_{i\in \Nc}v_ix^*_{it}}.
    \end{align*}
\end{itemize}
    In the equality, we apply Lemma \ref{cl:nonlightcust}. In the second inequality, we use Equation \eqref{eq:boundbound}, as well as the monotonicity of $x\mapsto x/(1+x)$.

\noindent {\em \underline{Case 3}: If $t$ is light.}
\begin{claim}\label{cl:lightcusts}
    We have $$
        \E\left[\frac{V(S_t)}{1+V(S_t)}\right] \geq (1-2\eps)\cdot \frac{\sum_{i\in \Nc}v_ix_{it}^*}{1+\sum_{i\in \Nc}v_ix_{it}^*}.
    $$
\end{claim}
Let us present the high level idea for the proof of Claim \ref{cl:lightcusts}. If $t\in [T]$ is a light customer then $\sum_{i\in \Nc} v_ix_{it}^* \leq \eps$. Therefore, the denominator $1+V(S_t)$ is in expectation equal to $1+\sum_{i\in \Nc}v_ix_{it}^* \leq 1+\eps$. The high level idea is to use this remark to approximate $V(S_t)/(1+V(S_t))$ to simply $V(S_t)$, which in expectation is equal to $\sum_{i\in \Nc}v_i x^*_{it}$. Finally, this last term is trivially lower bounded by $\sum_{i\in \Nc}v_i x^*_{it}/(1+\sum_{i\in \Nc}v_i x^*_{it})$. The formal proof is deferred to Appendix \ref{apx:nonlightcust}.


Combining cases 1, 2 and 3, and by linearity of expectation, we have\begin{align*}
    \E\left[\sum_{t =1}^T\frac{V(S_t)}{1+V(S_t)}\right] &\geq (1-2\eps)\cdot \sum_{t=1}^T \frac{\sum_{i\in \Nc}v_ix_{it}^*}{1+\sum_{i\in \Nc}v_ix_{it}^*} \\ &\geq (1-2\eps)\cdot(1-\eps)\cdot \sum_{t=1}^T\frac{V(S_t^*)}{1+V(S_t^*)} \\&\geq (1-3\eps)\cdot \sum_{t=1}^T\frac{V(S_t^*)}{1+V(S_t^*)}\\
    &=(1-3\eps)\cdot \opt^\downarrow,
\end{align*}
where the second inequality is an application of Lemma \ref{lem:LPopt}.

\end{proof}

%The proof of this theorem relies  on constructing assortments $S_1,\ldots,S_T$ in polynomial time such that the sum of the weights of the products in each assortment are close enough one from another. This way, we ensure that the value of each of them is either close to the one in an optimal set, or close to a concave upper bound of the optimum. 
    

 \section{\ref{APV} with Cardinality Constraints}
    \label{sec:APVC}
%A natural extension to \ref{APV} consists on considering cardinality constraints on the offered assortment. Cardinality constraints arise in multiple real world applications such as shelf space in brick-and-mortar stores or screen size in online stores, in which case, offering large assortments of products is infeasible.
In this section, we consider the cardinality constrained version of our fundamental model, referred to as {\em assortment optimization with visibility and cardinality constraints (\ref{APVC})}. Here, each customer can be shown at most $k$ products, where $k$ is a positive integer specified as part of the input description. Formally, \ref{APVC} is defined as follows: 
\begin{equation}
\label{APVC}
\tag{\sf{APVC}} 
\begin{aligned}
 \max_{ S_1, \ldots, S_T \subseteq \mathcal{N}} & \; \; \sum_{t=1}^T  R(S_t)  \\ \mathrm{s.t.} \;\; &\sum_{t=1}^T  \mathbbm{1}(i \in S_t) \geq \ell_i, \;\;\; &&\forall i \in \mathcal{N}, \\ & |S_t| \leq k, && \forall t \in [T].
\end{aligned}
\end{equation}






\paragraph{Hardness results.} First, we study the \black{computational} complexity of \ref{APVC}, showing that unlike its \black{polynomially-solvable} unconstrained counterpart, \ref{APVC} is strongly NP-hard.
\begin{theorem} \label{NP-hardness} \ref{APVC} is strongly NP-hard, even when all prices are equal. Moreover, \black{\ref{APVC} does not admit an FPTAS, even with equal prices}, unless $\mathrm{P} = \rm{NP}$.
\end{theorem}
The proof of this \black{result} relies on a reduction \black{from} the \texttt{3-PARTITION} problem, where the objective is to \black{decide whether a set of $3T$ positive integers can be partitioned} into $T$ triplets, such that the sum of elements within each triplet is identical. The full proof is \black{given in} Appendix \ref{apxhard}.
To the best of our knowledge, within the assortment optimization literature, there is only one other prominent choice model for which the unconstrained variant is solvable in polynomial time, while the cardinality constrained variant is NP-hard, which is the Markov chain model \citep{blanchet2016markov, desir2020constrained}. It is worth noting that despite this similarity, \black{under the Markov Chain choice model}, the cardinality constrained assortment optimization problem is APX-hard even with equal prices \black{\cite[Theorem~5]{desir2020constrained}}, whereas \ref{APVC} is strongly NP-hard, and admits a PTAS when prices are equal, as we proceed to show next.
\paragraph{Approximation scheme.} In the remainder of this section, we focus on the question of designing a polynomial time approximation scheme (PTAS), for the special case of equal prices. This setting corresponds to a sales maximization \black{(or market share)} problem, which is particularly useful when the platform or the company's goal is to maximize the captured portion of customers (see, e.g., \cite{desir2020constrained, derakhshan2022product, asadpour2023sequential}). \black{In particular, we design an algorithm which outputs a random solution for \ref{APVC}, whose objective value is, in expectation, within $1-\eps$ of optimal.}

\begin{theorem} \label{thm:PTAS}
For any $\eps>0$, \ref{APVC} with equal prices can be approximated within factor $1-\eps$ of optimal. The running time of our algorithm is $O(n^3T^{2^{O(\frac{1}{\eps}\log^2(\frac{1}{\eps}))}})$.
\end{theorem}
In Section \ref{subsec:discret}, we show that any \ref{APVC} instance with equal prices can be \black{approximated by} a ``friendlier" instance, \black{where the weights of a particular subset of products are slightly altered}. Importantly, we only incur an $\eps$-loss due to this \black{approximation}. In Section \ref{subsec:linearsol}, \black{we write our approximate problem as an integer linear program} through a carefully crafted guessing procedure. In Section \ref{subsec:GKPS}, we elaborate on a dependent rounding scheme introduced by \cite{gandhi2006dependent}, which we subsequently utilize in Section \ref{subsubsec:bipartite} to obtain approximate solutions for the \black{above-mentioned integer linear program}, thereby completing the description of our PTAS. Finally, in Section \ref{subsec:analysis}, we show that our algorithm indeed computes a $(1-\eps)$-approximation.
\subsection{Step 1: Discretizing the Universe of Products}\label{subsec:discret}
By specializing \ref{APVC} to the setting with identical prices, our current model can be written as
\begin{equation}
\label{eq:SPVC}\tag{\sf{SPVC}}
\begin{aligned}
 \max_{ S_1, \ldots, S_T \subseteq \mathcal{N}} & \; \; \sum_{t=1}^T\frac{V(S_t)}{1+V(S_t)}  \\ \mathrm{s.t.} \;\; &\sum_{t=1}^T  \mathbbm{1}(i \in S_t) \geq \ell_i, \;\;\; &&\forall i \in \mathcal{N}, \\ & |S_t| \leq k, && \forall t \in [T].
\end{aligned}
\end{equation}

\paragraph{Weight rounding.} In what follows, we reduce an arbitrarily-structured \ref{eq:SPVC} instance  into a modified instance, by rounding down the \black{preference} weights of a particular subset of products. Then, \black{this alteration will be shown to incur} at most an $\eps$-loss.
For any product $i\in \Nc$, \black{we operate according to three cases:}\begin{itemize}
    \item If $v_i\geq 1/\eps$, then $v_i$ is rounded down to $\nota{v}_i = 1/\eps$.
    \item If $v_i< \eps^5$, then $v_i$ remains unchanged and we define $\nota{v}_i = v_i$.
    \item If $\eps^5\leq v_i < 1/\eps$, then $v_i$ is rounded down as follows: Let $\black{q_i}\geq 1$ be the unique integer for which $\eps^5\cdot(1+\eps)^{\black{q_i}-1} \leq v_i < \eps^5\cdot(1+\eps)^{\black{q_i}}$. Then $v_i$ is rounded down to $\nota{v}_i = \eps^5\cdot(1+\eps)^{\black{q_i}-1}$. Note that by this definition, we have \begin{equation}\label{eq:weightbound}
        \nota{v}_i \leq v_i\leq (1+\eps)\cdot \nota{v}_i.
    \end{equation}
\end{itemize}
For each $i\in \Nc$, we denote the product associated with the weight $\nota{v}_i$ simply by $\nota{i}$, \black{with $\nota{\Nc} = \{\nota{i}\,\colon\,i\in \Nc\}$} being our new universe of products. We invite the reader to think of the notation $\nota{\cdot}$, as a one-to-one map from the universe $\Nc$ to the modified universe $\nota{\Nc}$, where several weights have been rounded down. Similarly to \black{$V(\cdot)$, we define the total weight of an assortment of products $S\subseteq \nota{\Nc}$ by} $\nota{V}(S) \coloneqq \sum_{\nota{i} \in S} \nota{v}_i$.


\paragraph{Analysis.} We proceed by showing that \black{any approximation guarantee for the resulting instance would migrate to the original instance up to an extra multiplicative loss of $1-\eps$}. To this end, \black{note that the former instance can be written as}
\begin{equation}
\label{eq:SPVCd}\tag{$\widehat{\sf SPVC}$}
\begin{aligned}
 \max_{ S_1, \ldots, S_T \subseteq \nota{\mathcal{N}}} & \; \; \sum_{t=1}^T\frac{\nota{V}(S_t)}{1+\nota{V}(S_t)}  \\ \rm{s.t.} \;\; &\sum_{t=1}^T  \mathbbm{1}(\nota{i} \in S_t) \geq \ell_i, \;\;\; &&\forall \nota{i} \in \nota{\Nc}, \\ & |S_t| \leq k, && \forall t \in [T].
\end{aligned}
\end{equation}
Let $\opt$ and $\widehat\opt$ respectively denote the \black{optimum values of} \ref{eq:SPVC} and \ref{eq:SPVCd}. Additionally, for any sequence of assortments $S_1,\ldots, S_T\subseteq \Nc$, \black{we make use of the shorthand notation}
\begin{equation*}
    \obj(S_1,\ldots, S_T) = \sum_{t=1}^T\frac{V(S_t)}{1+V(S_t)},
\end{equation*}
and \black{similarly,} for any sequence of assortments $S_1,\ldots, S_T\subseteq \nota{\Nc}$, let 
\begin{equation*}
    \widehat{\obj}(S_1,\ldots, S_T) = \sum_{t=1}^T\frac{\nota{V}(S_t)}{1+\nota{V}(S_t)}.
\end{equation*}
In Lemma \ref{lem:epsobjective} below, \black{whose proof is presented in Appendix \ref{apx:epsobjective}}, we show that the objective achieved by any sequence of assortments $S_1,\ldots,S_T\subseteq \Nc$ is within a factor $1\pm\eps$ of the objective achieved by their rounded counterparts $\nota{S}_1,\ldots,\nota{S}_T$, where $\nota{S}_t \coloneqq {\{\nota{i}\,\colon\, i \in S_t\}}$. This claim implies that $\widehat\opt\leq \opt\leq (1+\eps)\cdot\widehat{\opt}$.


\begin{lemma}\label{lem:epsobjective}
    For any sequence of assortments $S_1,\ldots,S_T\subseteq \Nc$, we have $$\widehat{\obj}(\hat{S}_1,\ldots, \hat{S}_T)\leq \obj(S_1,\ldots, S_T)\leq (1+\eps)\cdot\widehat{\obj}(\nota{S}_1,\ldots, \nota{S}_T).$$
\end{lemma}
We defer the proof of this lemma to Appendix \ref{apx:epsobjective}.
In the remainder of this section, we focus on dealing with the reduced instance \ref{eq:SPVCd}, and for simplicity of notation, we denote \black{its underlying products} simply by $1,\ldots, n$. Similarly, their respective weights will be denoted by $v_1,\ldots,v_n$.






\subsection{Step 2: Linearization of the Objective Function}\label{subsec:linearsol}
In this section, we \black{propose a method to linearize} the optimization problem \ref{eq:SPVCd}, by guessing certain parameters pertaining to its optimal solution. We start by presenting our detailed guessing procedure, before leveraging the guessed parameters to construct a linearized formulation.


\paragraph{Customer types.} Let $S_1^*,\ldots, S_T^*$ be an (unknown) optimal solution for our instance \ref{eq:SPVCd}. \black{With respect to this solution}, we aim at categorizing customers \black{in terms of} to the total weight of the products they are offered. First, we say that customer $t\in [T]$ is {\em light} when $V(S_t^*) < \eps$. Similarly, customer $t$ is {\em heavy} when $V(S_t^*)\geq 1/\eps$. Otherwise, $\eps\leq V(S_t^*)<1/\eps$ and we say that this customer is {\em medium}. We further partition medium customers into $L$ classes $G_1, \ldots, G_L$, depending on the value of $V(S_t^*)$. Specifically, \black{each class $G_\ell$ is defined as}
\begin{equation*}
    G_{\ell} = \left\{ t\text{ medium }\,\colon\, \eps\cdot (1+\eps)^{\ell-1}\leq V(S_t^*)< \eps\cdot (1+\eps)^{\ell}\right\}.
\end{equation*}
\black{Here, $L$ is defined as} the smallest integer for which $\eps\cdot(1+\eps)^{L}\geq 1/\eps$, \black{meaning that} $L = O(\frac{1}{\eps}\log\frac{1}{\eps})$.
\black{Indeed, since $\eps\cdot(1+\eps)^{L-1} < 1/\eps$, we have
\begin{equation}\label{eq:Lorder}
L < 1 + \frac{\log{(1/\eps^2)}}{\log(1+\eps)} = O\left(\frac1\eps\log\frac1\eps\right).
\end{equation}
}As such, the classes $G_1,\ldots, G_L$ form a partition of medium customers. Additionally, we denote the sets of light and heavy customers by $G_{\rm light}$ and $G_{\rm heavy}$ respectively. 

\paragraph{Packing patterns.} Next, we proceed by defining classes of products. Recall from Section \ref{subsec:discret} that, for every product $i\in \Nc$ with $v_i \geq \eps^5$, either there exists some $1\leq q\leq Q-1$ for which $v_i = \eps^5\cdot(1+\eps)^{q-1}$ or $v_i = 1/\eps$, where $Q$ is the smallest integer such that $\eps^5\cdot (1+\eps)^{Q-1}\geq 1/\eps$. \black{In particular, we have $Q = O\left(\frac{1}{\eps}\log\frac1\eps\right)$, justified in the same manner as in Equation~\eqref{eq:Lorder}}. For every $1\leq q\leq Q-1$, let $D_q$ denote the set of products in $\Nc$ with $v_i = \eps^5\cdot (1+\eps)^{q-1}$, and let $D_Q$ denote the set of products with $v_i = 1/\eps$. Similarly, let $D_0$ be the class of products $i\in \Nc$ with $v_i<\eps^5$. By construction, the classes $D_0, \ldots D_Q$ form a partition of the universe of products $\Nc$.

\noindent Now, a packing pattern is a $(Q+1)$-dimensional vector, where each coordinate takes \black{one of the values $0,1,2,\ldots, 1/\eps^6, \star\,$}; here, $1/\eps^6$ is assumed to be an integer, without loss of generality. For a given customer $t\in [T]$, each entry $q=0,\ldots,Q$ of her associated packing pattern specifies the number of products from class $D_q$ in the assortment $S_t^*$, with the convention that $\star$ indicates that the number of products from class $D_q$ is strictly greater than $1/\eps^6$.
By this definition, the number of possible packing patterns is  $$\left(\frac{1}{\eps^6}+2\right)^{Q+1} = 2^{O(Q\log\frac{1}{\eps})}   =2^{O(\frac{1}{\eps}\log^2\frac{1}{\eps})}.$$ 
% \paragraph{Guessing. } With all the previous definitions, we are now able to present the guessing procedure. We first start by guessing the number of customers of each type in the optimal solution $S_1^*, \ldots, S_T^*$, i.e., we guess the cardinalities $|G_{light}|,|G_{heavy}|, |G_1|,\ldots ,|G_L|$. Then for each customer type $\ell \in \{\text{light}, \text{heavy}, 1,\ldots, L\}$, and every given packing pattern  $P = (p_0, \ldots, p_Q)$, we guess the number of customers of type $\ell$, who use the packing pattern $P$, which we denote by $K_{\ell, P}$. 


\paragraph{Guessing procedure.} We are now ready to present our guessing procedure. For each class of customers $\ell \in \{{\rm light}, {\rm heavy}, 1,\ldots, L\}$ and for each packing pattern $P$, we guess the number of customers in $G_{\ell}$ whose packing pattern is $P$. Let us designate this guessed value by $K_{\ell,P}$.
Since there are $L+2 = O(\frac{1}{\eps}\log\frac{1}{\eps})$ classes of customers and $O(2^{O(\frac{1}{\eps} \log^2(\frac{1}{\eps}))})$ packing patterns, the number of such pairs is $O(2^{O(\frac{1}{\eps} \log^2(\frac{1}{\eps}))})$.  \black{Due to having} $T$ customers, we have $T+1$ possible guesses for each \black{of these pairs}, meaning that the total number of guesses is $O(T^{2^{O(\frac{1}{\eps}\log^2(\frac{1}{\eps}))}})$.


It is important to note that this guessing procedure allows us to determine, for any customer $t\in [T]$, both her type and her packing pattern in the optimal solution $S_1^*,\ldots,S_T^*$, up to a permutation of $\black{[T]}$. Indeed, since for each class $\ell\in \{{\rm light}, {\rm heavy}, 1,\ldots, L\}$, we know the number of customers that \black{are characterized by} each packing pattern, we know the quantity $|G_\ell|$. Hence, since all customers are initially interchangeable, we can arbitrarily assign each customer to a type according to our guess. Subsequently, given our guess $K_{\ell, P}$ for every customer type $\ell$ and every packing pattern $P$, since the customers within the same type are interchangeable a priori, we arbitrarily assign each customer to a packing pattern, according to our guess. Consequently, for each customer $t\in [T]$, we have determined both her type and her packing pattern, up to a permutation of $\black{[T]}$.

\paragraph{Integer linear programming formulation.}\label{subsec:linearprog}
We are now ready to leverage these guessed parameters in order to introduce our integer linear programming formulation \ref{eq:ILP}. \black{In the sequel, an approximate solution to this formulation} will be shown to be near optimal for \ref{eq:SPVCd}.

Let us start with some useful notation.
For each customer $t$, we define a lower bound $V_t$ on $V(S_t^*)$ as follows. If $t$ is light then $V_t = 0$, and if $t$ is heavy, then  $V_t = 1/\eps$. Similarly, if $V_t$ is of type $\ell\in [L]$, \black{i.e., $t\in G_\ell$}, then $V_t = \eps\cdot (1+\eps)^{\ell-1}$. In addition, we denote the packing pattern of customer $t$ by $(k_{0,t}, k_{1,t},\ldots, k_{Q,t})$.
Using these newly-defined parameters, we formulate the following integer linear program:
\begin{equation}\label{eq:ILP}\tag{ILP}
 \begin{aligned}
\max_{\mathbf{x}} &&& \sum_{t\in G_{\rm light}}\sum_{i\in \Nc}v_ix_{it} \\
\textrm{s.t.}\quad&\text{(I)} &&\displaystyle\sum_{t=1}^T x_{it} \geq  \ell_i,&&\quad  \forall i\in \Nc ,\\
&\text{(II)} &&\sum_{i\in \Nc} x_{it} \leq k,&&\quad  \forall t\in [T],\\
&\text{(III)} && \sum_{i\in D_q}x_{it}  =k_{q,t}, && \quad \forall t\in [T],\,\forall q=0,\ldots,Q \text{ \black{with} } k_{q,t}\neq \star,\\
&\text{(IV)} && \sum_{i\in D_q}x_{it}  \geq \frac{1}{\eps^6}+1,  &&\quad \forall t\in [T],\,\forall q=0,\ldots,Q \text{ \black{with} } k_{q,t}= \star,\\
&\text{(V)} && \sum_{i\in \Nc} v_i x_{it}  \geq  V_{t}, & & \quad\forall t\in [T], \\
&\text{(VI)} && \sum_{i \in \Nc} v_ix_{it}  \leq  \eps, & &\quad\forall t\in G_{\rm light}, \\
& \text{(VII)} &&x_{it}  \in  \{0,1\}, & & \quad\forall i\in \Nc,\, \forall t\in [T].
\end{aligned}
\end{equation}


% \begin{equation}\label{eq:ILP}\tag{ILP}
%  \begin{aligned}
% \max_{\mathbf{x}} &&& \sum_{t\in G_{light}}\sum_{i\in \Nc}v_ix_{it} \\
% \textrm{s.t.}\quad&\text{(I)}\quad &&\displaystyle\sum_{t=1}^T x_{it} \geq  \ell_i,&&\quad  \forall i\in \Nc ,\\
% &\text{(II)} \quad&&\sum_{i\in \Nc} x_{it} \leq k,&&\quad  \forall t\in [T],\\
% &\text{(III)} \quad&& \sum_{i\in D_q}x_{it}  =k_{q,t}, && \quad \forall t\in [T],\,\forall q=0,\ldots,Q \text{ such that } k_{q,t}\neq \star,\\
% &\text{(IV)} \quad&& \sum_{i\in D_q}x_{it}  > \frac{1}{\eps^6},  &&\quad \forall t\in [T],\,\forall q=0,\ldots,Q \text{ such that } k_{q,t}= \star,\\
% &\text{(V)} \quad&& \sum_{i\in \Nc} v_i x_{it}  \geq  V_{t}, & & \quad\forall t\in [T], \\
% &\text{(VI)} \quad&& \sum_{i \in \Nc} v_ix_{it}  \leq  \eps, & &\quad\forall t\in G_{light}, \\
% & \text{(VII)} \quad&&x_{it}  \in  \{0,1\}, & & \quad\forall i\in \Nc,\, \forall t\in [T].
% \end{aligned}
% \end{equation}



Here, for each customer $t\in [T]$ and each product $i\in \Nc$, $x_{it}$ is a binary variable indicating whether product $i$ is offered to customer $t$. \black{The first observation is that the objective function only includes light customers. Intuitively, this feature is motivated by already having a lower bound on the contribution of other types of customers within Constraint (V).} Constraints (I) and (II) respectively enforce the visibility and cardinality requirements, similar to the original formulation \ref{eq:SPVCd}. Constraints (III) and (IV) guarantee that for any customer $t\in [T]$ and any product class $q\in \{0,1,\ldots,Q\}$, the number of products in $D_q$ displayed to customer $t$ is consistent with the guessed packing pattern. Constraint (V) ensures that the lower bound $V_t$ on $V(S_t^*)$ is met for every customer $t\in [T]$. Constraint (VI) stipulates that for every light customer, the weight of her assortment is at most $\eps$. Finally, Constraint (VII) specifies that the decision variables $x_{it}$ are binary.  \black{It is worth mentioning that while some guesses from our procedure may result in infeasible linear programs, their infeasibility can be easily checked prior to proceeding with the algorithm, and they can be discarded as they are encountered. This does not affect the subsequent analysis, which relies on the guarantee that at least one guess --- specifically, the one corresponding to an optimal solution of \ref{eq:SPVCd} --- is optimal. As such, enumerating all the guesses will necessarily include this optimal guess. Finally, since our algorithm selects the best encountered guess, it is guaranteed to be near-optimal}.



Let $\mathbf{x^*}$ be an optimal solution to the linear relaxation of \ref{eq:ILP}, referred to as LP, and obtained by replacing the integrality constraint $x_{it}\in \{0,1\}$ by the relaxed constraint $x_{it} \in [0,1]$. The following lemma relates $\mathbf{x^*}$ to the optimal solution $S_1^*,\ldots, S_T^*$ of \ref{eq:SPVCd}. The proof of this result is included in Appendix \ref{apx:LPopt}.
\begin{lemma}\label{lem:LPopt}
    $
        \sum_{t=1}^T\frac{\sum_{i\in \Nc} v_ix^*_{it}}{1+\sum_{i\in \Nc} v_ix^*_{it}} \geq (1-\eps)\cdot\sum_{t=1}^T\frac{V(S_t^*)}{1+V(S_t^*)}.
    $
\end{lemma}

\subsection{Dependent Rounding}\label{subsec:GKPS}
Our next step is to round $\mathbf{x^*}$ in order to obtain {an approximate solution to} \ref{eq:SPVCd}.
To this end, we briefly review the dependent rounding scheme of \cite{gandhi2006dependent}, which will be employed to round our fractional solution into an integral one. Given a bipartite graph $(U,V,E)$ and values $x_{ij}\in [0,1]$ for every edge $(i,j)\in E$, \cite{gandhi2006dependent} present a polynomial-time \black{algorithm that generates} a collection of Bernoulli random variables $\{X_{ij}\}_{(i,j)\in E}$ satisfying the next three properties:
\setenumerate[1]{label={(P\arabic*)}}
\begin{enumerate}
    \item\label{it:P1} {\bf Marginal distributions:} $\E[X_{ij}] = x_{ij}$ for every edge $(i,j)$.
    \item\label{it:P2} {\bf Degree-preservation:} The fractional degree of every vertex is rounded to its floor or to its ceiling, almost surely. Namely, \black{for all} $i\in U\cup V$, \black{letting $d_i = \sum_{j\in \delta(i)}x_{ij}$, we have $\sum_{j\in \delta(i)}X_{ij}\in\{\lfloor d_i\rfloor, \lceil d_i\rceil\}$. Here, $\delta(i)$ is the set of neighbors of vertex $i$.}
    \item\label{it:P3} {\bf Negative correlation:} For every vertex $i\in U\cup V$ and subset $S\subseteq \delta(i)$ of its neighbors
    % \begin{align*}
    %     &\P\left[\bigwedge_{j\in S}(X_{i,j}=0)\right]\leq \prod_{j\in S}\P\left[X_{i,j} = 0\right]\\
    %     \text{and}\quad \quad&\P\left[\bigwedge_{j\in S}(X_{i,j}=1)\right]\leq \prod_{j\in S}\P\left[X_{i,j} = 1\right].
    % \end{align*}
    \begin{equation*}
        \P\left[\bigwedge_{j\in S}(X_{ij}=0)\right]\leq \prod_{j\in S}\P\left[X_{ij} = 0\right]\quad
        \text{and}\quad \quad\P\left[\bigwedge_{j\in S}(X_{ij}=1)\right]\leq \prod_{j\in S}\P\left[X_{i
        j} = 1\right].
    \end{equation*}
\end{enumerate}
As pointed out by \citet[Section~3]{gandhi2006dependent}, the negative correlation {(P3)} leads to Chernoff-type bounds, as stated in the following lemma.
\begin{lemma}   [{\cite{panconesi1997randomized}}]\label{lem:chernoff}
    Suppose that $X_1,\ldots,X_n$ satisfy property \ref{it:P3}. Then, for \black{any $\eps \in [0,1]$} and any vector $a=(a_1,\ldots,a_n)\in [0,1]^n$, letting $a(X) = \sum_{i=1}^na_iX_i$, we have$$
        \P[a(X)\leq (1-\eps)\cdot \E[a(X)]] \leq \exp\left(-\frac{\eps^2\cdot \E[a(X)]}{2}\right).
    $$
\end{lemma}




\subsection{Step 3:  The Rounding Procedure}\label{subsubsec:bipartite}
Next, we \black{devise} a framework \black{for employing} the \black{above-mentioned} dependent rounding scheme in order to derive \black{an approximate solution to \ref{eq:SPVCd}}.
Noticing that this rounding procedure takes \black{an edge-weighted} bipartite graph as its input, our first step is to interpret the optimal solution ${\bf x}^*$ of LP as a collection of edge-associated values in a carefully constructed bipartite graph.
Before presenting this graph, let us introduce the next two definitions. We say that a customer $t\in [T]$ is {\em bounded} when $k_{q,t} \neq \star$, for all $q=1,\ldots, Q$, noting that $q$ starts from $1$ in this definition rather than from $0$. Otherwise, we say that customer $t$ is {\em unbounded}. In other words, customer $t$ is bounded when in her optimal assortment $S_t^*$, she is offered at most $1/\eps^5$ products from any of the product classes $D_1,\ldots, D_Q$.
\paragraph{The bipartite graph.}
We start by defining the vertex sets $U$ and $V$. The set $U$ is simply the set $\Nc$ of all products. To describe the set of vertices $V$, for every unbounded customer $t$, we introduce a vertex $t$. In addition, for every bounded customer $t$, we introduce $Q+1$ vertices, specifically, one vertex $(t,q)$ for each class of products $q=0,\ldots, Q$. Therefore, $$V = \{t\,:\, t\text{ unbounded}\} \cup \left\{(t,q)\,\colon \, t \text{ bounded}, q =0,\ldots,Q\right\}.$$
Let us now describe the edge set $E$ of this graph, as well as the weight of each edge $e\in E$. \black{To this end, every} vertex $i\in U$ is joined by an edge to every vertex $t$ of every unbounded customer $t$; the edge $(i,t)$ is associated with the value $x^*_{it}$. Next, suppose that $D_q$ is the class containing product $i$. Then $i$ is joined by an edge to every vertex $(t,q)$ for \black{every bounded customer $t$}; the edge $(i, (t,q))$ is associated with the value $x^*_{it}$. \black{It is worth noting that for every bounded customer $t$ and every product $i\in \Nc$, there exists a unique edge $(i, (t,q))\in E$ in the above-constructed bipartite graph, where $q$ corresponds to the index of the class $D_q$ containing product $i$. As such, there is a one-to-one correspondence between all edges of the form $(i, (t,q))$ in our bipartite graph and the set of pairs $\{(i,t):i\in \Nc\text{ and }t\text{ bounded}\}$. This allows us to label each edge connecting a vertex $i\in U$ to a vertex of the form $(t,q)\in V$ simply as edge $(i,t)$, without any risk of ambiguity. Adopting this simplified notation, the weight of every edge $(i,t)$ in our bipartite graph is $x_{it}^*$, for every product $i\in \Nc$ and every customer $t\in [T]$.
}
\paragraph{{\color{black} Rounding procedure.}} {\black{With respect to this bipartite graph,}} we can now apply the dependent rounding procedure of \cite{gandhi2006dependent} to derive a collection of random variables $(X_{it}\,\colon\, i\in \Nc, t\in[T])$ representing rounded values. Finally, we return the associated sequence $S_1,\ldots, S_T$ of random assortments, where $S_t= \{i \in \Nc \,\colon\, X_{it}=1\}$ for each \black{customer} $t\in [T]$.
\paragraph{Running time.}Since our algorithm is based on exhaustive enumeration, the steps presented in Sections \ref{subsec:linearsol} and \ref{subsubsec:bipartite} are applied to each possible guess considered in Section \ref{subsec:linearsol}. Solving LP takes $O((nT)^{O(1)})$ using standard linear programming algorithms. Constructing the bipartite graph takes $O(nTQ)$ time, and implementing the dependent rounding scheme requires $O(|E|\cdot(|L|+|R|)) = O(n^2T^2Q)$ time \citep[Theorem~2.3]{gandhi2006dependent}. Therefore, processing each guess takes $O((nT)^{O(1)}Q) = O((nT)^{O(1)}\cdot\frac{1}{\eps}\log{\frac{1}{\eps}})$ time, since $Q = O(\frac{1}{\eps}\log\frac{1}{\eps})$. Finally, as explained in Section \ref{subsec:linearsol}, there are $O(T^{2^{O(\frac{1}{\eps}\log^2\frac{1}{\eps})}})$ possible guesses, \black{meaning that} the overall running time of our algorithm is
$$
O\left(T^{2^{O(\frac{1}{\eps}\log^2\frac{1}{\eps})}} \cdot (nT)^{O(1)}\cdot\frac{1}{\eps}\log{\frac{1}{\eps}}\right) = O\left(n^{O(1)}T^{2^{O(\frac{1}{\eps}\log^2\frac{1}{\eps})}}\right),
$$
which is polynomial in the input size for any fixed $\eps$.












\subsection{Analysis}\label{subsec:analysis}
In this section, we prove that the sequence of random assortments $(S_1,\ldots,S_T)$ is near-optimal for \ref{eq:SPVCd}, \black{as formally} stated in the following theorem.
% \begin{algorithm}\label{alg:PTASd}
% \caption{Polynomial-time approximation scheme for \ref{eq:SPVCd}}
% \begin{algorithmic}[1]
% \REQUIRE $\Nc$, $T$, $v_i$ for every $i\in \Nc$
% \ENSURE Sum of integers from 1 to $N$
% \STATE $S_t^* \leftarrow \emptyset$ for $t\in [T]$
% \STATE $A \leftarrow 0$
% \FOR{every guess of customer types and packing patterns}
%     \STATE Solve the linear relaxation of \ref{eq:ILP}
%     \STATE Use the the rounding in Section \ref{subsec:GKPS} to compute $S_1,\ldots, S_T$
%     \STATE $A \leftarrow \obj(S_1,\ldots,S_T)$
%     \IF{$A \geq A^*$}
%         \STATE $S_t^* \leftarrow S_t$ for $t\in [T]$
%         \STATE $A^* \leftarrow A$
%     \ENDIF
% \ENDFOR
% \RETURN $A^*$, $(S_t^*\,\colon\,t\in [T])$
% \end{algorithmic}
% \end{algorithm}

\begin{theorem}\label{thm:feasibleoptimal}
    The sequence of random assortments $S_1,\ldots, S_T$ is feasible for \ref{eq:SPVCd}, with an expected \black{objective value of}
    \begin{equation*}
        \E\left[\sum_{t=1}^T \frac{V(S_t)}{1+V(S_t)} \right]\geq (1-3\eps) \cdot \widehat{\opt}.
    \end{equation*}
\end{theorem}

    In what follows, we start by showing that the sequence $S_1, \ldots,S_T$ is always feasible for \ref{eq:SPVCd}. Subsequently, we show that the expected \black{objective value} of this sequence is within factor $1-3\eps$ of optimal. We remind the reader that $S_1^*,\ldots, S_T^*$ is an optimal sequence of assortments for \ref{eq:SPVCd}.
    
    \paragraph{Feasibility.} First, to verify that the visibility constraint of every product $i\in \Nc$ \black{is satisfied}, \black{note that} by constraint (I) of LP, we have $\sum_{t=1}^Tx_{it}^* \geq \ell_i$. Noticing that $\sum_{t=1}^Tx_{it}^*$ is the fractional degree of vertex $i$ in the bipartite graph, by property \ref{it:P2}, we have $\sum_{t=1}^TX_{it} \geq \lfloor\ell_i\rfloor = \ell_i$.
    
    We now show that our cardinality constraint is always satisfied as well, meaning that each customer is offered at most $k$ products. First, for every unbounded customer $t$, the cardinality of $S_t$ is simply given by the degree of vertex $t$. Since $\sum_{i=1}^nx_{it}^* \leq k$, by property \ref{it:P2}, we have $|S_t| = \sum_{i\in \Nc}X_{it} \leq k$. For a bounded customer $t$, for every class $q=1,\ldots,Q$, the fractional degree of vertex $(t,q)$ is $\sum_{i\in D_q}
    x_{it}^*=k_{q,t}$, according to contraint (III) of \ref{eq:ILP}. Since $k_{q,t}$ is an integer, the degree of vertex $(t,q)$ remains unchanged with respect to $\bf X$, according to property \ref{it:P2}. Therefore, we almost surely have\begin{align}
        \sum_{i\in \Nc} X_{it}&= \sum_{i\in D_0}X_{it} + \sum_{q=1}^Q\sum_{i\in D_q}X_{it} \notag\\ &\leq \left\lceil\sum_{i\in D_0}x_{it}^*\right\rceil+ \sum_{q=1}^Q k_{q,t} \label{eq:5}\\ &= \left\lceil\sum_{i\in D_0}x_{it}^*+ \sum_{q=1}^Q k_{q,t} \right\rceil\label{eq:6}\\ &= \left\lceil\sum_{i\in \Nc}x_{it}^* \right\rceil \notag\\
        &\leq \lceil k\rceil \label{eq:7}\\
        &=k.\notag
    \end{align}
    Here, \black{Equation~\eqref{eq:5}} follows from Property \ref{it:P2}. \black{Equation~\eqref{eq:6} holds since} the second summand $\sum_{q=1}^Q k_{q,t}$ is an integer. \black{Equation~\eqref{eq:7}} follows from constraint (II) of LP. This proves the cardinality constraint and concludes the feasibility proof. 
    \paragraph{Near-optimality.} We \black{proceed by showing} that the expected contribution of every customer $t\in [T]$ to the objective function, i.e., $\E[V(S_t)/(1+V(S_t))]$ is within factor $1-2\eps$ of her contribution in the optimal solution $\mathbf x^*$. Formally, we have the following lemma, whose proof is provided in Appendix \ref{apx:nearoptimal}.
    \begin{lemma}\label{lem:nearoptimal}
        For every customer $t\in [T]$, we have$$
            \E\left[\frac{V(S_t)}{1+V(S_t)}\right] \geq (1-2\eps)\cdot\frac{\sum_{i\in \Nc}v_ix_{it}^*}{1+\sum_{i\in \Nc}v_ix_{it}^*}.
        $$
    \end{lemma}
We can now conclude the proof of Theorem \ref{thm:feasibleoptimal}, by observing that
     \begin{align}\E\left[\sum_{t =1}^T\frac{V(S_t)}{1+V(S_t)}\right]&\geq (1-2\eps)\cdot\sum_{t =1}^T\frac{\sum_{i\in \Nc}v_ix_{it}^*}{1+\sum_{i\in \Nc}v_ix_{it}^*} \label{eq:1024}\\
     &\geq (1-2\eps)\cdot(1-\eps)\cdot \sum_{t=1}^T\frac{V(S_t^*)}{1+V(S_t^*)} \label{eq:2048}\\&\geq (1-3\eps)\cdot \sum_{t=1}^T\frac{V(S_t^*)}{1+V(S_t^*)}\notag\\
    &=(1-3\eps)\cdot \widehat{\opt}\notag,
\end{align}
where Inequality \eqref{eq:1024} is an application of Lemma \ref{lem:nearoptimal} and Inequality \eqref{eq:2048} follows from Lemma \ref{lem:LPopt}.


%The proof of this theorem relies  on constructing assortments $S_1,\ldots,S_T$ in polynomial time such that the sum of the weights of the products in each assortment are close enough one from another. This way, we ensure that the value of each of them is either close to the one in an optimal set, or close to a concave upper bound of the optimum. 
            

%
    \section{Price of Visibility} \label{sec:price}


In this section, we investigate the impact of visibility constraints on the total expected revenue, comparing it to the unconstrained setting where there are no visibility constraints. In Section \ref{subsection:loss}, we quantify the loss resulting from enforcing the visibility constraints. In Section \ref{subsection:share}, we introduce a novel method to distribute the loss among different products in proportion to their contribution to the overall loss. 
 Finally, in Section \ref{subsection:numeric}, we illustrate   our method through a series of numerical experiments.



\subsection{The loss due to Visibility constraints}
\label{subsection:loss}




Consider the unconstrained assortment optimization \ref{Unconstrained problem} and let $S^*$ be its optimal solution. It is known that there exists an optimal assortment that is price-ordered. This is a standard result in assortment optimization under the MNL model (see \citeauthor{talluri2004revenue} \cite{talluri2004revenue}).\footnote{We refer the reader to Appendix \ref{apx1} for further discussion of assortment optimization under MNL.}
In the absence of visibility constraints and with $T$ customers, it is optimal to offer assortment $S^*$ to each customer. Consequently, the total expected revenue in the unconstrained setting can be expressed as $T\cdot  R(S^*)$. As the unconstrained problem serves as a relaxation of \ref{APV}, it possesses a higher objective function.
In the following example, we show that enforcing the visibility constraints can imply a gap that is arbitrary bad as compared to the unconstrained setting. 




%After enforcing visibility constraints in our assortment problem, we observe a decrease in the total revenue. 

%Indeed, if we remove all the constraints, that is $\ell_i = 0 \; \forall i \in \mathcal{N}$, we come back to \ref{Unconstrained problem}, as defined in \ref{apx1}. The best solution we could hope for is $S_t = S^* \quad \forall 1 \leq t \leq T$, because it would maximize each term $R(S_t)$ independently of the others, therefore maximize the total revenue. However, because of the visibility constraints, we had to integrate in the assortments $(S_t)_{1 \leq t \leq T}$ some elements that were not in $S^*$, driving the revenue down. 




% \noindent{\bf Example.} Consider two products such that for all $i \in \{1,\ldots, n-1\}$
% $ p_i = 1, v_i = 1$ and $ \ell_i = 0$. For product $n$, let  $p_n = 0, v_n = n^2$,  and $\ell_n = T$.  We consider the setting where $n$ is large.
% To compute the optimal assortment for \ref{Unconstrained problem}. As mentioned earlier, it sufficient to evaluate the revenue of the price-ordered assortments and choose the one with the highest revenue. We have 
% $R(\{1,\ldots,k\})=  \frac{k}{1 + k} \quad \forall k \in \{1,\ldots, n-1\}$ and $R(\mathcal{N}) = \frac{n-1}{1 + n-1 + n^2} = \frac{n-1}{n + n^2} \leq \frac{n-1}{n} $. Therefore, the optimal assortment is $S^*=\{1, \ldots,n-1 \}$ and $R^*=\frac{n-1}{n}.

% On the other hand,

%  However, $\forall A \subseteq [\![1, n-1]\!], \; R(\{n\} \cup A) = \frac{1 + |A|}{1 + |A| + n^2}$, which increases with $|A|$, so the optimal solution for \ref{APV} is $S_t^* = \overline{\{n\}} = \mathcal{N} \; \forall t$. Therefore the revenue goes to zero because of the visibility constraints on product $n$, while it was close to 1 for a large value of $n$ without constraints.



\vspace{2mm}
\noindent{\bf Example.} Consider an instance of \ref{APV} with two products $n=2$ and $T$ customers. Let $ p_1 = 1, v_1 = 1, \ell_1 =0$ and  $p_2 = 0, v_2 = M, \ell_2 = T$.  We consider the setting where $M$ is large.
To compute the optimal assortment for \ref{Unconstrained problem}. As mentioned earlier, it sufficient to evaluate the revenue of the price-ordered assortments and choose the one with the highest revenue. We have 
$R(\{1\})=  \frac{1}{2} $  and $R(\{1,2\})=  \frac{1}{2+M} < R(\{1\})$ . Therefore, the optimal assortment is $S^*=\{1 \}$ and $R(S^*)=\frac{1}{2}$.

On the other hand, let $(S_1, S_2, \ldots, S_T)$ be a feasible solution for \ref{APV}. Because, $\ell_2=T$, we have to include product $2$ in every assortment $S_t$. Adding product $1$ to an assortment only increases the revenue because $R(\{1,2\})=  \frac{1}{2+M} > R(\{2\})=0$.
Therefore, it is optimal to offer product $1$ and $2$ in every assortment $S_t$ in an optimal solution of \ref{APV}. Therefore, $S_t^*=\{1,2\}$ for all $t =1,\ldots,T$. 
Now, consider the ratio $$\frac{T R(S^*)}{\sum_{t=1}^T R(S_t^*)} = \frac{T  \frac{1}{2}   }{T \frac{1}{2+M}}= \frac{M}{2}+1,$$
which goes to infinity as $M$ increases. Hence the gap can be arbitrarily large.
While in this pathological example, we see that enforcing products with very low price and high weight can drive the revenue very low, our numerical experiments in Section \ref{subsection:numeric} will illustrate how more common distributions of the prices and weight react to the enforcement of visibility constraints.

 % However, $\forall A \subseteq [\![1, n-1]\!], \; R(\{n\} \cup A) = \frac{1 + |A|}{1 + |A| + n^2}$, which increases with $|A|$, so the optimal solution for \ref{APV} is $S_t^* = \overline{\{n\}} = \mathcal{N} \; \forall t$. Therefore the revenue goes to zero because of the visibility constraints on product $n$, while it was close to 1 for a large value of $n$ without constraints.


\subsection{Sharing the loss}
\label{subsection:share}

% Consider a platform on which $n$ vendors sell a product each. Due to SLAs, the vendors impose some visibility constraints for their product on the platform, so that the situation can be modeled by the problem \ref{APV}. 


In this section, we explore a scenario where each product within our universe is associated with a specific vendor. As mentioned earlier, vendors can impose visibility constraints on their products within the platform. These constraints can be established through service level agreements or product sponsorships. However, it is important to note that enforcing these constraints may result in a decrease in the platform's revenue. To address this issue, the platform can implement a fee structure based on the vendors' contributions to the revenue loss. %This approach aims to establish a fair pricing policy that aligns with each vendor's impact on the revenue loss.
%We would like to a fair pricing policy where we first calculate the revenue loss incurred due to the constraints imposed by vendors. Then, we  charge each vendor a fraction of this loss, proportionate to their individual contribution towards the overall revenue reduction.
% In this section, we consider the situation when each product in our universe belongs to a vendor. As explained earlier vendors can impose some visibility constraint for their product on the platform according to a service level agreement or by sponsoring the product. 
% As seen previously, enforcing these constraints could cause a loss in the revenue of the platform. In return, the platform can charge the different vendors a fee depending on how much they contributed in reducing the revenue. In this perspective, a fair pricing policy would consist in computing the revenue loss due to the constraints, and then to charge to the vendors a fraction of this loss, which would be kind of proportional to their contribution in the loss.
Let $S^*$ be an optimal solution for \ref{Unconstrained problem} and let $(S_1^*, S_2^*,\ldots, S_T^*)$ be an optimal solution for \ref{APV}. We denote the revenue loss due to the visibility constraints as 
\begin{equation}
    \label{Delta loss}
    \Delta \coloneqq T \cdot R(S^*) - \sum_{t=1}^T R(S_t^*).
\end{equation}

\noindent
{\bf A first naive approach.} One approach is to allocate the loss based solely  on the parameters $\ell_i$. In this case, the proportion of the loss assigned to the vendor of product $i$ would be determined by $\frac{\ell_i}{\sum_{j=1}^n \ell_j}$. However, this distribution would not be equitable in the sense that  we should not impose any charges on a product that already belongs to the optimal set $S^*$, even if it satisfies the constraint $\ell_i > 0$. Moreover, this allocation fails to consider the impact of each product on the overall loss. For example, if there is a product with exceptionally high preference weight $v_i$ but a significantly lower price $p_i$, while other products have higher prices and lower preference weights, enforcing the visibility of the first product would drive the revenue down, whereas the others would have a lesser impact. In this scenario, the former product should be responsible for covering almost the entire revenue loss.



\noindent
{\bf Our approach.}
Let $S_1^*, \ldots, S_T^*$ be an optimal solution to \ref{APV}. First observe that  $R(S_t^*) = \frac{\sum_{i \in S_t^*} p_i v_i}{1 + \sum_{i \in S_t^*} v_i}$, implies that $$R(S_t^*) = \sum_{i \in S_t^*}  (p_i - R(S_t^*))v_i$$ for every set $S_t^*$. This decomposition of the revenue gives us which products drive the revenue down (the ones with $p_i < R(S_t^*)$) and which products increase the revenue ($p_i > R(S_t^*))$. It also shows that the contribution of  product $i$ to the revenue is proportional to the difference between the price of product $i$ and the actual revenue, as well as proportional to the preference weight $v_i$. We can then rewrite the total revenue of the assortments $(S_1^*, S_2^*, \ldots, S_T^*)$ as 
$$\sum_{t=1}^T R(S_t^*) = \sum_{t=1}^T \sum_{i \in S_t^*}  (p_i - R(S_t^*)) v_i = \sum_{i=1}^n \sum_{t=1}^T \mathbbm{1}(i \in S_t^*)  (p_i - R(S_t^*)) v_i.$$
Therefore, we view the contribution of product $i \in \mathcal{N}$ to the total revenue as $$\cont{i}\coloneqq\sum_{t=1}^T \mathbbm{1}(i \in S_t^*) (p_i - R(S_t^*)) v_i.$$

%\begin{prop}
%    \label{pricing formula}

\noindent
{\bf Pricing the loss.}
 For each product $i \in \mathcal{N}$, we propose to charge its vendor the fraction of the loss corresponding to the negative contribution of this product, divided by the sum of the negative contributions of all the products:

\begin{equation} \label{pricing formula}
  \Gamma_i \coloneqq \frac{\cont{i}^-}{\sum_{j=1}^n \cont{j}^-} \cdot \Delta
\end{equation}
where $x^- = max(-x, 0)$ is the negative part of x, and recalling that $\Delta$ is the total loss in revenue due to enforcing visibility constraints.
%\end{prop}

Next, we discuss three important properties of this strategy:
\\

\noindent
{\bf (a) Fair distribution of fees.} First, it is worth noting that the loss in revenue is exactly shared between the products whose contribution to the total revenue is negative as on one hand we have $\sum_{i \in \mathcal{N}} \Gamma_i = \Delta$, and on the other hand we have $\Gamma_i > 0$ if and only if $C_i^->0$, i.e., the product's contribution to the revenue is negative. Moreover, the fee $\Gamma_i$ for each product takes into account its actual impact on the revenue:
the first observation is that the products in $S^*$ all have a nonnegative contribution, and their vendors are consequently exempt from a fee as expected. The second observation is that for any product with a negative contribution, its impact on the revenue is magnified by a lower price and a greater preference weight. It turns out that our policy does take this observation into account, as the lower/greater the price/weight of such a product, the greater the loss it incurs to the total revenue, and thus the greater the fee imposed on the vendor. The third observation is that even if a products $i\notin S^*$ and/or if there exists some $t\in [T]$ such that $i \in S_t^*$ and $p_i\leq R(S_t^*)$, such a product can still have a positive contribution, in which case $\Gamma_i =0$, concurring with a product whose impact overall is positive. The final observation is that this pricing strategy guarantees a fair treatment of equal products. Indeed, if two distinct products $i$ and $j$ have identical prices and preference weights, then imposing similar visibility constraints for the two products implies a similar fee for the vendors.

\noindent
{\bf (b) Monotonicity of the fee.}                            
For every product $i$, the fee $\Gamma_i$ is nondecreasing as a function of $\ell_i$. In other words, if a vendor wishes to display her product more often to customers, her fee gets higher. Let us show this intuitive result mathematically. Recall that an optimal solution to \ref{APV} is given by $S_t^* = \overline{L_t\cup \cdots \cup L_T}$ for $t\in [T]$, as shown in Theorem \ref{Solution structure}. Similarly, let $\Tilde S_t$ be the optimal solution given by Equations \eqref{eq:sol} when $\ell_i$ is increased by $1$. Note that a direct consequence of the formula giving these optimal solution, we have $\Tilde S_t = S_t^*$ for all $t\neq \ell_i+1$, and $\Tilde S_{\ell_i+1} = \overline{L_t\cup\cdots\cup L_{\ell_i+1}\cup\{i\}}$. First, if $i\in S_{\ell_i+1}^*$, then nothing changes and we have $\Tilde {\cont{i}} = \cont{i}$, where $\Tilde{\cont{i}} \coloneqq \sum_{t=1}^T \mathbbm{1}(i \in \Tilde S_t) (p_i - R(\Tilde S_t)) v_i$ is product $i$'s contribution to the loss after increasing $\ell_i$. Otherwise, $\cont{i}\geq\Tilde{\cont{i}}$, and therefore $\cont{i}^-\leq\Tilde{\cont{i}}^-$. Furthermore, for every $j\neq i$, since $R(S^*_{\ell_i+1}) \geq R(\Tilde S_{\ell_i+1})$, we have $\Tilde{\cont{j}} \geq {\cont{j}}$, since the term associated with $t=\ell_i+1$ in the definition of $\cont{j}$ can only increase when $\ell_i$ is increased, while all the other terms remain unchanged. Letting $\Tilde \Delta$ and $\Tilde \Gamma_i$ be the total loss incurred and the fee imposed on vendor $i$ respectively, after $\ell_i$ is increased by $1$, we have\begin{align*}
    \Tilde \Gamma_{i} &= \frac{\Tilde C_i}{\Tilde{\cont{i}} + \sum_{j\neq i}\Tilde{\cont{j}}}\cdot \Tilde \Delta\\
    & \geq \frac{\Tilde C_i}{\Tilde{\cont{i}} + \sum_{j\neq i}{\cont{j}}}\cdot  \Delta\\
    &\geq \frac{C_i}{{\cont{i}} + \sum_{j\neq i}{\cont{j}}}\cdot  \Delta\\
    & = \Gamma_i.
\end{align*}
The third line follows from the fact that the map $x\mapsto x/{c+x}$ is nondecreasing on $[0,+\infty)$, for any $c>0$.

\noindent
{\bf (c) Computational tractability.} The fees charged are easy to compute: each of them can be computed in polynomial time ${O}(nT)$ since they only require to solve \ref{APV}. Additionally, consider the situation where a vendor is interested in knowing the fee that she needs to pay in order to increase the visibility of the product by one customer. This corresponds to the situation where we change $\ell_i$ to $\ell_i + 1$ (without changing any other ${\ell}_j$).
In this case, we only have to recompute $S_{\ell_i + 1}^*$, because the others assortments $S_t^*$ stay the same, as explained in the previous paragraph. Therefore, for a fixed product $i$, can compute efficiently the value of the fee $\Gamma_i$ for all the values of $\ell_i \in [T]\cup\{0\}$.
 %we can compute all the $\Gamma_i(\ell)$ for $\ell \in [\![0, T]\!]$ in time $\mathcal{O}(nT) + T \mathcal{O}(n) = \mathcal{O}(nT)$, instead of $\mathcal{O}(n T^2)$ if we recomputed all the $S_t^*$ every time we increase $\ell_i$. Thus, the computation all the $(\Gamma_i(\ell_i))_{0 \leq \ell_i \leq T}$ has the same complexity as the computation of a single $\Gamma_i(\ell_i)$ for one value of $\ell_i$. 
 
 %This is useful for a vendor who is willing to see how the fee they will have to pay evolves with the visibility constraint they ask for in the SLA for instance.

   

\subsection{Numerical Study}\label{subsection:numeric}

To illustrate our theoretical contributions, we perform some numerical experiments on randomly generated data.

\noindent
{\bf Our experimental setup.}
We fix the number of products equal to $n = 20$, and fix the number of customers to $T = 30$. We generate the weights $v_i$, prices $p_i$ and visibility constraints $\ell_i$ from several distributions. Namely, to generate the weights $v_i$ and prices $p_i$ we used uniform and exponential distributions with varying parameters. For the visibility constraints $\ell_i$, we ran our experiments using three distributions: we used a standard integer uniform $\mathcal{U}(\{0\}\cup[T])$, then an integer uniform $\ell_i \sim \mathcal{U}(\{\lfloor\frac{i}{n} T\rfloor,\ldots, T\})$, and finally the constant $\ell_i = T$, for all  $i \in \mathcal{N}$. Recalling  that products are ordered by non-increasing price values, the intuition behind the second distribution is that products with lower prices are likelier to receive higher visibility constraints.

For each set of parameters, we generate $1000$ samples, then we compute ratio of the optimal solution of \ref{APV} divided by $T \cdot R(S^*)$, i.e., the unconstrained optimal expected revenue (without any visibility constraints), namely 
$$\eta \coloneqq \frac{\sum_{t=1}^T R(S_t^*)}{T \times R(S^*)}.$$
This captures the fraction of the revenue that we conserve once visibility constraints are enforced.
We compute statistics on ratio $\eta$:  mean, standard deviation, minimum and maximum values.
\footnote{Initially, we made $T$ vary, but we observed that the results where almost independent of $T$ compared to the randomness of the generation of the parameters, so we kept only $T=30$. Indeed, varying $T$ just extends the stream of customers, and if the visibility constraints are extended by the same factor, it seems natural that the ratio of conserved revenue remains constant.}

\noindent
{\bf Overall results and analysis.}
Our results are gathered in Table \ref{fig:tables}. 
\begin{table}[h]
    \caption{For each distribution of $p_i, v_i$ and $\ell_i$, we compute the ratio $\eta$ between the optimal assortment of \ref{APV} and the optimal assortment of the unconstrained problem \ref{Unconstrained problem} on $1000$ samples. We provide its mean, standard deviation, minimum and maximum values.}
    \centering
\begin{tabular}{||c|c|c|c|c|c|}
\hline
\rule{0pt}{12.5pt}{}  & {}    &  \multicolumn{4}{c|}{\begin{tabular}{l} $\eta$ (in \%) \end{tabular}} \\
\hline
{Distribution of $v_i$ and $p_i$} &{Distribution of $\ell_i$} & {Mean} & {Standard deviation} & {Min} & {Max} \\
\hline
    & $\mathcal{U}(\{0\}\cup[T])$ & 82.9 & 5.7 & 58.8 & 96.8  \\
$v_i \sim \mathcal{U}(0,1), \quad p_i \sim \mathcal{U}(0,1)$  
    & $\mathcal{U}(\{\lfloor\frac{i}{n} T\rfloor,\ldots, T\})$ & 73.7 & 6.4 & 51.9 & 91.3  \\
    & $\ell_i = T$ & 71.6 & 6.9 & 47.4 & 91.6  \\

\hline
    & $\mathcal{U}(\{0\}\cup[T])$ & 71.0 & 7.3 & 43.9 & 89.2  \\
$v_i \sim \mathcal{U}(0,10), \quad p_i \sim \mathcal{U}(0,10)$  
    & $\mathcal{U}(\{\lfloor\frac{i}{n} T\rfloor,\ldots, T\})$ & 60.3 & 7.1 & 39.5 & 81.6 \\
    & $\ell_i = T$ & 58.1 & 7.6 & 33.0 & 81.0  \\

\hline
    & $\mathcal{U}(\{0\}\cup[T])$ & 64.3 & 9.4 & 34.5 & 88.1  \\
$v_i \sim \mathcal{E}(1), \quad p_i \sim \mathcal{E}(1)$  
    & $\mathcal{U}(\{\lfloor\frac{i}{n} T\rfloor,\ldots, T\})$ & 50.9 & 9.4 & 24.8 & 87.6  \\
    & $\ell_i = T$ & 48.2 & 10.1 & 20.0 & 89.1  \\

\hline
    & $\mathcal{U}(\{0\}\cup[T])$ & 49.4 & 11.2 & 16.1 & 79.1  \\
$v_i \sim \mathcal{E}(1/10), \quad p_i \sim \mathcal{E}(1/10)$  
    & $\mathcal{U}(\{\lfloor\frac{i}{n} T\rfloor,\ldots, T\})$ & 37.2 & 9.8 & 13.4 & 63.7  \\
    & $\ell_i = T$ & 34.7 & 9.4 & 11.3 & 67.1  \\

\hline
\end{tabular}

    \label{fig:tables}
\end{table}

As expected, the ratio $\eta$ decreases when we give higher visibility $\ell_i$ to the products with smaller prices $p_i$. Furthermore, we observe that the higher the range of fluctuation of the weights $v_i$ and prices $p_i$, the lower the revenue becomes once we enforce visibility constraints. This is rather intuitive, because we saw previously that the cases in which the visibility constraints can severely harm the revenue are when there are some products with extreme value of the price or weight compared to the others. However, we observe that extreme events when the revenue of \ref{APV} becomes very low are really rare since the standard deviation is rather small compared to the range of fluctuation of $\eta$. Based on our results, it seems that in most cases, as long as there are no abnormally extreme values of the weights and prices, we can always hope to keep a fraction close to half of the revenue after enforcing visibility constraints.\\


\noindent
{\bf Revenue vs market shares.} To assess the influence of the visibility constraints on the market share, we compute the optimal value of \ref{APV} for all visibility constraints $(\ell_i)_{1 \leq i \leq n}$ equal to a same value $\ell$ going from $0$ to $T$ (for $v_i \sim \mathcal{U}(0,1), p_i \sim \mathcal{U}(0,1)$). We then compute the expected sales associated to this expected revenue, and plot in Figure \ref{fig:graph1} the ratio of revenue decrease versus the ratio of sales increase by dividing them by their unconstrained value, for normalization.
% Figure environment removed

It is interesting to see that, while the revenue decreases as we enforce more visibility constraints on products, on the contrary, the sales increase. Indeed, based on the structure of the optimal nested solution we identified, $S_t^* = \overline{L_t \cup \ldots \cup L_T}$, we see that the sizes of the $S_t^*$ increase when the $\ell_i$'s increase. Furthermore, if we take a look at the sales optimization problem $$\max_{S_1, \ldots, S_T} \sum_{t=1}^T \frac{V(S_t)}{1 + V(S_t)},$$ that corresponds to $p_i = 1$ for all $i\in \Nc$, we can see that the more products we add to each $S_t$, the higher the sales since the function $x \mapsto \frac{x}{1+x}$ is increasing with respect to $x$. \\
Therefore, we can see that while visibility constraints decrease the revenue, they increase the sales, which can be interesting if the objective of the platform is to capture market shares.\\

\noindent
{\bf Marginal study of one product.}
Finally, it is interesting to illustrate the marginal effect of the visibility constraints on one product. For this purpose, we take an instance of our generated data with $v_i \sim \mathcal{E}(1), \; p_i \sim \mathcal{E}(1), \; \ell_i \sim \mathcal{U}([\![0, T]\!])$. Then, we select a particular $i \in \mathcal{N}$, and vary $\ell_i$ from $0$ to $T$ while the others $\ell_j$ stay fixed at their generated value $\ell_j$. In Figure \ref{fig:graph2}, we plot the variation of the revenue loss with respect to $\ell_i$

% Figure environment removed


We remark that the loss is convex with respect to $\ell_i$. Indeed, let   $(S_t^*)_{1 \leq t \leq T} = (\overline{L_T \cup \ldots \cup L_t})_{1 \leq t \leq T}$ be the nested solution for constraint $\ell_i$, and $\ell_i \mapsto \Delta(\ell_i)$ the revenue loss function, as defined in \eqref{Delta loss}, and considered as a function of constraint $\ell_i$ when all the other visibility constraints $(\ell_j)_{j \neq i}$ are fixed. We have 
% \begin{align*}
%     \Delta(\ell_i +1) - \Delta(\ell_i) - (\Delta(\ell_i) - \Delta(\ell_i -1)) &= \overline{R}(S_t^* \cup \{i\}) + \overline{R}(S_{t+1}^* \cup \{i\}) - 2 (\overline{R}(S_t^* \cup \{i\}) + \overline{R}(S_{t+1}^*)) + \overline{R}(S_t^*) + \overline{R}(S_{t+1}^*) \\
%     &= \overline{R}(S_{t+1}^* \cup \{i\}) + \overline{R}(S_t^*) - \overline{R}(S_t^* \cup \{i\}) - \overline{R}(S_{t+1}^*) \geq 0
% \end{align*}
$$\Delta(\ell_i +1) - \Delta(\ell_i) - (\Delta(\ell_i) - \Delta(\ell_i -1))  = \overline{R}(S_{t+1}^* \cup \{i\}) + \overline{R}(S_t^*) - \overline{R}(S_t^* \cup \{i\}) - \overline{R}(S_{t+1}^*) \geq 0,$$  where the inequality holds by supermodularity and $S_t^* \subseteq S_{t+1}^*$, which shows $\ell_i \mapsto \Delta(\ell_i)$ is convex.  


We can then, in the case where all the revenue from the sales of a product comes to its vendor, compute the expected revenue of product $i$ versus the fee its vendor will pay as the visibility constraint $\ell_i$ increases. The expected revenue of product $i$ is given by: $\sum_{t=1}^T \mathbbm{1}(i \in S_t^*) p_i \frac{v_i}{1 + V(S_t^*)}$, while the fee charged $\Gamma_i$ is defined in  \eqref{pricing formula}. This is depicted in Figure \ref{fig:graph3}.

% Figure environment removed

For this particular product, we can see that enforcing a visibility constraint $\ell_i$ is profitable until $\ell_i = 17$ (with respect to $T=30$), and becomes non profitable after. The profit associated with this constraint is the difference between the revenue and the fee, and is maximized for the value $\ell_i = 7$ (see Figure \ref{fig:graph3}). This is an example of a product that does not belong to the optimal unconstrained set $S^*$ (since the revenue is $0$ when $\ell_i = 0$, it means the product was not included), but for which adding a visibility constraint allows to boost sales and the revenue of the vendor, while paying a small fee to compensate the platform's revenue loss.

One interesting observation here is that the fee paid by the vendor of product $i$ (0.8 when $\ell_i = T = 30$ on Graph \ref{fig:graph3}) is higher than the loss caused by the introduction of $\ell_i = T = 30$ while the others products were enforced previously (loss of 0.5 as we can read on Graph \ref{fig:graph2}). However, this can be explained: as the expanded revenue function is supermodular and decreasing, the revenue decreases most when the first products are enforced, and then decreases less. Therefore, enforcing visibility constraints on product $i$ decreases less the revenue if we already have some visibility constraints on the other products than if $i$ is the first product on which the constraints are enforced. However, our pricing policy $\Gamma_i$ does not take into account such sequential order of arrival, and prices each product globally, so that two products with the same price and weight pay the same fee for identical visibility constraints.
    

\section{Price of Visibility} \label{sec:price}
In this section, we investigate the impact of visibility constraints on the total expected revenue, comparing this model to the unconstrained setting where there are no visibility constraints. In Section \ref{subsection:loss}, we quantify the maximum-possible loss resulting from enforcing visibility constraints. In Section \ref{subsection:share}, we introduce a novel method to distribute the loss incurred among different products in proportion to their share of the overall loss.


\subsection{The Loss Due to Visibility Constraints}
\label{subsection:loss}
Consider the unconstrained assortment optimization problem \ref{Unconstrained problem}, which was defined in Section \ref{subsectin:lp} as
\begin{equation}
\begin{aligned}
\max_{S \subseteq \mathcal{N}} \quad  R(S),
\end{aligned}
\tag{\sf{AP}}
\end{equation}
and let $S^*$ be its optimal solution. As mentioned in Section \ref{sec:intro}, this basic version of the MNL model has a price-ordered optimal assortment (\cite{talluri2004revenue}). \black{We refer the reader to Appendix \ref{apx1} for further discussion of assortment optimization under MNL}.
In the absence of visibility constraints, it is clearly optimal to offer the assortment $S^*$ to all customers. Consequently, the total expected revenue in the unconstrained setting is simply $T\cdot  R(S^*)$.
In the following example, we show that enforcing visibility constraints can lead to an arbitrarily \black{large revenue gap in comparison with} the unconstrained setting. 




%After enforcing visibility constraints in our assortment problem, we observe a decrease in the total revenue. 

%Indeed, if we remove all the constraints, that is $\ell_i = 0 \; \forall i \in \mathcal{N}$, we come back to \ref{Unconstrained problem}, as defined in \ref{apx1}. The best solution we could hope for is $S_t = S^* \quad \forall 1 \leq t \leq T$, because it would maximize each term $R(S_t)$ independently of the others, therefore maximize the total revenue. However, because of the visibility constraints, we had to integrate in the assortments $(S_t)_{1 \leq t \leq T}$ some elements that were not in $S^*$, driving the revenue down. 




% \noindent{\bf Example.} Consider two products such that for all $i \in \{1,\ldots, n-1\}$
% $ p_i = 1, v_i = 1$ and $ \ell_i = 0$. For product $n$, let  $p_n = 0, v_n = n^2$,  and $\ell_n = T$.  We consider the setting where $n$ is large.
% To compute the optimal assortment for \ref{Unconstrained problem}. As mentioned earlier, it sufficient to evaluate the revenue of the price-ordered assortments and choose the one with the highest revenue. We have 
% $R(\{1,\ldots,k\})=  \frac{k}{1 + k} \quad \forall k \in \{1,\ldots, n-1\}$ and $R(\mathcal{N}) = \frac{n-1}{1 + n-1 + n^2} = \frac{n-1}{n + n^2} \leq \frac{n-1}{n} $. Therefore, the optimal assortment is $S^*=\{1, \ldots,n-1 \}$ and $R^*=\frac{n-1}{n}.

% On the other hand,

%  However, $\forall A \subseteq [\![1, n-1]\!], \; R(\{n\} \cup A) = \frac{1 + |A|}{1 + |A| + n^2}$, which increases with $|A|$, so the optimal solution for \ref{APV} is $S_t^* = \overline{\{n\}} = \mathcal{N} \; \forall t$. Therefore the revenue goes to zero because of the visibility constraints on product $n$, while it was close to 1 for a large value of $n$ without constraints.



\vspace{2mm}
\noindent{\bf Example.} Consider an \ref{APV} instance with two products and $T$ customers. The product prices are $p_1=1$ and $p_2=0$, their preference weights are $v_1=1$ and $v_2 = M$, and the visibility constraints are given by $\ell_1=0$ and $\ell_2=T$, where $M$ is a large constant. As mentioned earlier, \black{without visilibity constraints}, it \black{suffices} to evaluate the expected revenue of all price-ordered assortments and choose the one with the highest revenue to obtain an optimal assortment for \ref{Unconstrained problem}. Since 
$R(\{1\})=  \frac{1}{2} $, $R(\{1,2\})=  \frac{1}{2+M} < R(\{1\})$, and $R(\emptyset) = 0$, the optimal assortment is $S^*=\{1 \}$ with $R(S^*)=\frac{1}{2}$.

On the other hand, let $(S_1, S_2, \ldots, S_T)$ be a feasible solution for \ref{APV}. Since $\ell_2=T$, we have to include product $2$ in each of these assortments. Moreover, adding product $1$ to any assortment only increases our revenue since $R(\{1,2\})=  \frac{1}{2+M} > R(\{2\})=0$.
Therefore, it is optimal to offer product $1$ and $2$ in every assortment, and $S_t^*=\{1,2\}$ for all $t =1,\ldots,T$. 
As such, the ratio between the optimal expected revenue collected from $T$ customers in an unconstrained setting and the one collected when enforcing visibility constraints is
$$\frac{T \cdot R(S^*)}{\sum_{t=1}^T R(S_t^*)} = \frac{T \cdot  \frac{1}{2}   }{T \cdot\frac{1}{2+M}}= \frac{M}{2}+1,$$
which tends to infinity as $M$ increases.
In this example, we see that enforcing products with very low price and high weight can drive the expected revenue \black{to arbitrarily low values}.

 % However, $\forall A \subseteq [\![1, n-1]\!], \; R(\{n\} \cup A) = \frac{1 + |A|}{1 + |A| + n^2}$, which increases with $|A|$, so the optimal solution for \ref{APV} is $S_t^* = \overline{\{n\}} = \mathcal{N} \; \forall t$. Therefore the revenue goes to zero because of the visibility constraints on product $n$, while it was close to 1 for a large value of $n$ without constraints.


\subsection{Sharing the Expected Loss}
\label{subsection:share}

% Consider a platform on which $n$ vendors sell a product each. Due to SLAs, the vendors impose some visibility constraints for their product on the platform, so that the situation can be modeled by the problem \ref{APV}. 


In this section, we explore a scenario of a platform, where each product within our universe is associated with a specific vendor. These vendors can impose visibility constraints on their products, established through service-level agreements or product sponsorships. As observed in Section \ref{subsection:loss}, enforcing such constraints may result in decreasing the platform's revenue. To address this issue, the platform wishes to implement a fee structure based on the vendors' contributions to the overall revenue loss. 
Letting $S^*$ and $(S_1^*, S_2^*,\ldots, S_T^*)$ be optimal solutions to \ref{Unconstrained problem} and \ref{APV}, we denote the revenue loss due to visibility constraints by 
\begin{equation}
    \label{Delta loss}
    \Delta \coloneqq T \cdot R(S^*) - \sum_{t=1}^T R(S_t^*).
\end{equation}

\noindent
{\bf A naive approach.} One approach for allocating $\Delta$ is solely based on the parameters $\ell_i$, which represent the lower bounds on the products' imposed visibility. \black{Toward this objective, suppose that} the proportion assigned to the vendor of product $i$ would be determined by \black{$\ell_i/\sum_{j=1}^n \ell_j$}. This distribution is generally not equitable, in the sense that  we should not impose any charges on a product that already belongs to the optimal set $S^*$, even when $\ell_i > 0$, as each such product is already profitable for the platform, and would be offered to customers regardless of its visibility requirement. Moreover, this allocation rule fails to consider the impact of each product on the overall loss. For example, when a product has an exceptionally high preference weight $v_i$ but a significantly lower price $p_i$, while other products have higher prices and lower preference weights, enforcing visibility for the first product would drive our revenue down, whereas the others would have a lesser impact. In this scenario, the former product should in principle be responsible for covering almost the entire revenue loss.



\noindent
{\bf Our proposed approach.}
Let us first observe that \black{$R(S_t^*) = \sum_{i \in S_t^*} p_i v_i/(1 + \sum_{i \in S_t^*} v_i)$}, implying that for every customer $t\in [T]$
$$R(S_t^*) = \sum_{i \in S_t^*}  (p_i - R(S_t^*))\cdot v_i.$$
This decomposition tells us which products are driving our revenue down (those with $p_i < R(S_t^*)$) and which products are acting in the opposite direction ($p_i \geq R(S_t^*)$). It also shows that the contribution of  product $i$ to the expected revenue is proportional to the difference between the price of product $i$ and the actual revenue $R(S_t^*)$, as well as proportional to the preference weight $v_i$. We can now rewrite the total revenue of the assortments $S_1^*, S_2^*, \ldots, S_T^*$ as 
$$\sum_{t=1}^T R(S_t^*) = \sum_{t=1}^T \sum_{i \in S_t^*}  (p_i - R(S_t^*)) \cdot v_i = \sum_{i=1}^n \sum_{t=1}^T \mathbbm{1}(i \in S_t^*) \cdot (p_i - R(S_t^*))\cdot v_i.$$







Therefore, we view the collective contribution of each product $i \in \mathcal{N}$ to the total revenue over all customers as $$\cont{i}\coloneqq\sum_{t=1}^T \mathbbm{1}(i \in S_t^*) \cdot (p_i - R(S_t^*)) \cdot v_i.$$

%\begin{prop}
%    \label{pricing formula}

\noindent
{\bf Pricing the expected loss.}
 For each product $i \in \mathcal{N}$, we propose to charge its vendor the fraction of the loss corresponding to the negative contribution of this product, divided by the sum of all negative contributions, namely,

\begin{equation} \label{pricing formula}
  \Gamma_i \coloneqq \frac{\cont{i}^-}{\sum_{j=1}^n \cont{j}^-} \cdot \Delta,
\end{equation}
where $x^- = -\min(x, 0)$ is the negative part of \black{$x$}, and $\Delta$ is the total loss in revenue due to enforcing visibility constraints, \black{given by Equation~\eqref{Delta loss}}. Next, we discuss three important properties of this strategy:
\paragraph{(a) Fair distribution.} First, it is worth noting that the revenue loss $\Delta$ is exactly shared between the products whose contribution is negative. Indeed, on the one hand $\sum_{i \in \mathcal{N}} \Gamma_i = \Delta$, and on the other hand  $\Gamma_i > 0$ if and only if $C_i^->0$, i.e., the product's contribution to the revenue is negative. Moreover, the fee $\Gamma_i$ for each product takes into account its actual impact. The first observation is that all products in $S^*$ have nonnegative contributions, and their vendors are consequently exempt from a fee, as expected. The second observation is that for any product with a negative contribution, its revenue impact is amplified by a lower price and a higher preference weight. It turns out that our policy \black{accounts for this observation}, as the lower the price of such a product, the greater revenue loss it causes, and thus the higher the fee it imposes on the vendor. Similarly, the greater the preference weight of an unprofitable product (i.e., one with $p_i< R(S^*)$) the greater its fee. In the following, we say that a product $i\in \Nc$ is unprofitable with respect to an assortment $S\subseteq \Nc$ if $p_i< R(S)$. The third observation is that even if a product $i\in \Nc$ is unprofitable with respect to the optimal unconstrained assortment $S^*$, it can still have a nonnegative contribution, in which case $\Gamma_i =0$, concurring with a product whose overall impact is nonnegative. In fact, the same holds even if a product is unprofitable with respect to some assortment $S_t^*$ for $t\in [T]$, as its overall contribution may potentially be nonnegative. The final observation is that this pricing strategy guarantees a fair treatment of identical products. Indeed, if two distinct products $i$ and $j$ have identical prices and preference weights, then imposing similar visibility constraints implies a similar fee for the two vendors.
\paragraph{(b) Monotonicity.}                            
For every product $i$, the fee $\Gamma_i$ is nondecreasing as a function of $\ell_i$. In other words, if a vendor wishes to display her product more often to customers, the resulting fee becomes higher. \black{Formally, let us fix a product $i\in \Nc$, and consider the instance with $T$ customer, $\Tilde \ell_i = \ell_i+1$ and $\Tilde \ell_j = \ell_j$ for $j\neq i$, which corresponds to increasing the visibility requirement $\ell_i$ of product $i$ by $1$, while the requirement for other products remains unchanged. Let $\Tilde \Gamma_i$ be the fee imposed on product $i$ in the latter instance. Then, we have the following claim, whose proof is included in Appendix~\ref{apx:increasefee}.
\begin{claim}\label{cl:increasefee}
    $\Tilde \Gamma_i \geq \Gamma_i$.
\end{claim}
}
\paragraph{(c) Computational tractability.} The fees charged are easy to compute: Each $\Gamma_i$ can be computed in ${O}(nT)$ time, since it only requires solving \ref{APV}. Additionally, consider the situation where a vendor is interested in knowing the fee that she will be paying in order to increase the visibility of her product by one unit, corresponding to increasing $\ell_i$ to $\ell_i + 1$, without changing any other ${\ell}_j$.
In this case, we only have to recompute $S_{\ell_i + 1}^*$, since all others assortments $S_t^*$ remain the same. Therefore, for a fixed product $i$, we can efficiently compute the fee $\Gamma_i$ for all $\ell_i \in [T]\cup\{0\}$.
 %we can compute all the $\Gamma_i(\ell)$ for $\ell \in [\![0, T]\!]$ in time $\mathcal{O}(nT) + T \mathcal{O}(n) = \mathcal{O}(nT)$, instead of $\mathcal{O}(n T^2)$ if we recomputed all the $S_t^*$ every time we increase $\ell_i$. Thus, the computation all the $(\Gamma_i(\ell_i))_{0 \leq \ell_i \leq T}$ has the same complexity as the computation of a single $\Gamma_i(\ell_i)$ for one value of $\ell_i$. 
 
 %This is useful for a vendor who is willing to see how the fee they will have to pay evolves with the visibility constraint they ask for in the SLA for instance.
   
\black{\paragraph{Additional discussions.} Here, we discuss the sensitivity of the fee with respect to problem primitives, mainly the preference weights. Since the MNL choice model is a stylized model for customer choice behavior, an estimation step is applied prior to making assortment decisions, in order to determine estimates for the model's preference weights using methods such as maximum likelihood estimation or Bayesian methods; see Chapter~8 in \cite{train2009discrete} for more detail on numerical estimation procedures in choice modelling. Such estimates are by assumption, approximations of some underlying real parameters, and are inherently noisy. This noise in estimation can have an impact on the fees given to customers. Indeed, let us take a simple illustrating argument. Consider an instance with a single customer, i.e., $T=1$, and two products $1$ and $2$, with their respective prices $p_1 = 2$ and $p_2 = 1$, and their visibility requirements $\ell_1 = 0$ and $\ell_2 = 1$. According to some prior estimation procedure, the preference weights of both products are $v_1 = v_2 = 1$. Abstracting about the technical details, applying our approach to the above instance would show that $\{1,2\}$ is an optimal solution, and that the expected loss is $\Delta = 0$. Consequently, both products enjoy a fee of $0$. Now assume that due to estimation error, the true underlying parameter is $\Tilde v_1 = 1+\eps$ for some estimation error $\eps>0$, and the parameter $v_2$ was correctly estimated. Therefore, the true expected loss is $$\Tilde \Delta = \frac{\eps}{(2+\eps)(3+\eps)},$$
which is entirely incured by product $2$, that is $\Gamma_2 = \Tilde \Delta$. This shows the existence of instances where $\eps$-bounded variations on the preference weights can turn a null fee into a positive one.

It is noteworthy that while our approach outlines an explicit cost-sharing scheme, it is much richer as it quantifies the so-called contribution $C_i$ of any product $i\in \Nc$. This allows platforms to design specific cost-sharing schemes, tailored to their estimation procedures, business needs and their bottom line. For example, recalling the above example, when the platform's estimation procedure allows for an $\eps$-bounded noise, the preference weights can for instance be randomly sampled from a distribution over confidence intervals for preference weights.
}



%    \section{Conclusions and future directions}
    \label{sec:conclusions}
    
%noindent
%{\bf Conclusion.}
In this paper, we introduced the problem of assortment optimization with visibility constraints \ref{APV}, motivated by situations in e-commerce and online advertising in which a platform aims at ensuring a minimal exposure for some or all products. We prove that this problem can be solved in polynomial time, and devise an efficient algorithm to compute an optimal solution. Our algorithm leverages the supermodularity of an altered version of the expected revenue function, that allows us to identify the nested structure of an optimal solution. 
We also consider an extension of the problem with cardinality constraints on the assortments offered. We prove that the problem becomes strongly NP-hard even under uniform prices, and that in particular, it admits no FPTAS, unless $P=NP$. We then devise a PTAS for the special case of equal prices.
Finally, we evaluate the revenue loss caused by the visibility constraints enforced, and propose a fair pricing strategy to charge each vendor a fee proportional to the contribution of its product to the revenue loss. Finally, 


\vspace{2mm}
\noindent
{\bf Future directions.}
A promising future research direction involves developing an approximation algorithm for the assortment optimization problem with visibility and cardinality constraints, considering general prices.
Furthermore, exploring the assortment problem with visibility using alternative choice models such as the Markov Chain choice model or Nested Logit represents another fruitful avenue for investigation. %Finally, an interesting direction to consider an online version of the problem, in which assortments have to be offered sequentially, as customers arrive.


    \section{Concluding Remarks}
    \label{sec:conclusions}
    
%noindent
%{\bf Conclusion.}


% In this paper, we introduced the problem of assortment optimization with visibility constraints (\ref{APV}). We demonstrated that \ref{APV} is solvable in polynomial time and proved that its cardinality-constrained counterpart (\ref{APVC}) is strongly NP-hard. Additionally, we developed a PTAS for \ref{APVC} with equal prices. Furthermore, we investigated the price of visibility and proposed a fair pricing policy.

\black{We conclude this paper by presenting a number of intriguing questions and promising avenues for future research.}

{\em Algorithms for \ref{APVC} with general prices}: While we designed an approximation scheme for \ref{APVC} with identical product prices, developing non-trivial algorithms for general prices appears to be a challenging research direction. Extending our algorithmic approach to \ref{APVC} with general prices will likely require new ideas, as directly applying our current techniques does not appear to yield a PTAS. Specifically, discretizing and guessing techniques similar to those of Section \ref{subsec:linearsol} would need additional refinement to effectively handle this broader setting.

{\em Generalization to arbitrary choice models}: Investigating assortment optimization with visibility constraints under alternative choice models, such as Nested Logit, Markov Chain, or ranking-based choice models, represents another interesting area for exploration. Our current algorithms are heavily reliant on structural properties of the MNL choice model. For instance, our approach for computing the expanded set leverages the concept of revenue-ordered assortments, which does not easily generalize to other choice models. Additionally, the supermodularity of the expanded revenue is based on the formulation of the MNL choice probabilities, making its applicability to a general choice model an intriguing question for future research.
%Finally, an interesting direction to consider an online version of the problem, in which assortments have to be offered sequentially, as customers arrive.


% BIB %%%%%%%%%%%%%%%%%%%%%%%%%%%%%%%%%%%%%%%%%%%
{%\small
\addcontentsline{toc}{section}{Bibliography}
\bibliographystyle{plainnat}
\bibliography{BIB-Visibility}
}


% APPENDIX %%%%%%%%%%%%%%%%%%%%%%%%%%%%%%%%%%%%%%
\appendix
\changelocaltocdepth{1}
%    \section{Revenue Maximization Under MNL}
    \label{apx1}

In this appendix, we present for completeness two standard known results about assortment optimization under MNL. Lemma \ref{lem:nestedrevenue}, shows that the optimal assortment for \ref{Unconstrained} is revenue-ordered. This is a standard result in assortment optimization under MNL \cite[Proposition~6][25]{talluri2004revenue}.
Lemma \ref{Revenue variations} provides a necessary and sufficient conditions under which adding a given product to a given assortment increases the overall revenue.
Consider the classic unconstrained revenue maximization problem under MNL, defined as follows
\begin{equation}
\label{Unconstrained}
\begin{aligned}
\max_{S \subseteq \mathcal{N}} \quad  R(S).
\end{aligned}
\tag{\sf{AP}}
\end{equation}
%The next lemma shows that the optimal assortment for \ref{Unconstrained} is revenue-ordered. This is a standard result in assortment optimization under MNL \cite[Proposition~6][25]{talluri2004revenue}. 
\begin{lemma}\label{lem:nestedrevenue}
    \label{Unconstrained solution}
    Let $R^* = \max_{S \subseteq \mathcal{N}} R(S)$ be the optimal value of \ref{Unconstrained problem}. There exists a revenue ordered optimal solution to \ref{Unconstrained problem}, given by $S^* = \{i \in \mathcal{N}, p_i \geq R^* \}$.
\end{lemma}

Before proving Lemma \ref{lem:nestedrevenue}, we state and prove the following more general lemma, which shows that adding a product $i$ to any given assortment $S$ increases the objective if and only if the price of the added product $i$ is greater than or equal to the original revenue $R(S)$.


\begin{lemma}\label{Revenue variations}
    For any assortment $S\subseteq {\cal N}$ and any product $k \in {\cal N} \setminus S$, the three following propositions are equivalent:
$$(i) \; R(S \cup \{k\}) \geq R(S),  \qquad (ii)  \; p_k \geq R(S),  \qquad (iii) \; p_k \geq R(S\cup \{k\}). $$
\end{lemma}
    


%\begin{lemma}\label{Revenue variations}
%    For any assortment $S\subseteq {\cal N}$ and any product $k \in {\cal N} \setminus S$, the three following propositions are equivalent
 %   \begin{itemize}
  %      \item[(i)] $R(S \cup \{k\}) \geq R(S)$.
   %     \item[(ii)] $p_k \geq R(S)$.
    %    \item[(iii)] $p_k \geq R(S\cup \{k\})$.
  %  \end{itemize}
%\end{lemma}
    
\begin{proof}[Proof of Lemma \ref{Revenue variations}]
    We simply show that $R(S\cup \{k\})$ is a convex combination of $R(S)$ and $p_k$. In fact, simple algebra gives  \begin{align*}
        R\left(S\cup \{k\}\right) &= \sum_{i\in S} p_i\cdot\phi\left(i,S\cup \{k\}\right) + p_k\cdot\phi\left(k,S\cup \{k\}\right)
   % & = \sum_{i\in S} p_i\cdot\phi\left(i,S\right) \cdot \frac{1+V(S)}{1+V(S\cup \{k\})}+ p_k\cdot\phi\left(k,S\cup \{k\}\right)\\
     = R(S) \cdot \alpha + p_k \cdot (1-\alpha),
    \end{align*}
    where $\alpha = (1+V(S)) / (1+V(S\cup\{k\}))$. It is easy to see that $\alpha \in (0,1)$,  which proves that $R(S\cup \{k\})$ is a convex combination of $R(S)$ and $p_k$. Consequently, $R(S\cup \{k\})$ belongs to the closed interval whose extremities are $R(S)$ and $p_k$. In particular, we have $R(S\cup \{k\}) \geq R(S)$ if and only if $p_k \geq R(S)$, and if and only if $p_k \geq R(S\cup \{k\})$.
\end{proof}

%We can now deduce the proof of Lemma \ref{lem:nestedrevenue}.
\begin{proof}[Proof of Lemma \ref{lem:nestedrevenue}]
Let $\Tilde{S}$ be the optimal assortment of \ref{Unconstrained} with maximal cardinality. In the case of ties, let $\Tilde{S}$ be any arbitrary such assortment. In the following, we show that $S^* = \Tilde S$ (which in particular implies that there can be no ties).
 Let $i\in S^*$, and assume by contradiction that $i\notin \Tilde S$. One one hand, we know by optimality of $\Tilde S$ that $R(\Tilde S) = R^*$. On the other hand, we know that $p_i\geq R^*$ by definition of $S^*$. Therefore, $p_i\geq R(\Tilde S)$ and we can apply lemma \ref{Revenue variations}, which implies that $R(\Tilde{S}\cup \{i\})\geq R(\Tilde S)$, and hence that $\Tilde{S}\cup \{i\}$ is also an optimal assortment. Since $\Tilde{S} \subsetneq \Tilde{S}\cup \{i\}$. This contradicts the definition of $\Tilde{S}$ as the optimal assortment with maximal cardinality, and therefore shows that $i\in \Tilde S$.
 Now, let $i \in \Tilde S$. Assume by contradiction that $i\notin S^*$, i.e., $p_i < R^* = R(\Tilde S)$. In particular, this implies using Lemma \ref{Revenue variations} that $R(\Tilde S\setminus \{i\}) > R(\Tilde S)$, which contradicts the optimality of $\Tilde S$, and therefore shows that $i \in S^*$. This proves the second inclusion, and thereby completes the proof of the lemma.
\end{proof}


\section{Additional Proofs from Section \ref{sec:APVC}}\label{apx:APVC}
\subsection{Proof of Theorem \ref{NP-hardness}}
\label{apxhard}
In order to show the result, we reduce the $3$-partition problem to \ref{APVC}. In the $3$-\texttt{PARTITION} problem, the input consists on an integer $T$, and a set of $3T$ integers ${\cal A} = \{a_1,\ldots, a_{3T}\}$ such that $\sum_{i=1}^{3T}a_i=BT$ for some $B\geq 0$, and seeks to determine whether $\cal A$ can be partitioned into $T$ triplets $A_1,\ldots, A_T$, and such that the sum of elements of each triplet is $B$. This problem is known to be strongly NP-hard (see \cite{garey1975complexity} for example, $3$-\texttt{PARTITION} is referred to there as $P[3,1]$).

Given an arbitrary instance of $3$-\texttt{PARTITION}, we construct the following instance of \ref{APVC}. We consider $3T$ products, simply denoted by the universe $\Nc = \{1,\ldots,3T\}$, and such that for all $i\in \Nc$, $v_i = a_i$, $p_i = 1$ and $\ell_i = 1$. In particular, note that $V(\Nc) = BT$. Finally, let $k=3$ in our instance, the upper bound on the cardinality of any offered assortment.
A first important observation is that any feasible solution for this instance of \ref{APVC} shows each product to exactly one customer. In other words, if $S_1,\ldots ,S_T$ is a feasible solution for this instance of \ref{APVC}, then for any product $i\in \Nc$, there exists a unique $t\in [T]$ such that $i\in S_t$. Indeed, one one hand, the visibility constraints impose that each product be shown at least once, and on the other hand, since each assortment can contain at most $k=3$ products and there are $3T$ products in our universe, this implies that each product must be shown at most once, which shows the observation. A natural consequence of this observation is that for any feasible solution $S_1,\ldots, S_T$ of our instance of \ref{APVC}, we have $\sum_{t=1}^TV(S_t) = V(\Nc) = BT.$
We denote by $S^* = (S^*_1,\ldots,S^*_T)$ an optimal sequence of assortments for this instance of \ref{APVC}. In order to show the strong NP-hardness, we introduce the following claim.
\begin{claim}\label{cl:NPhard}
    The answer to the $3$-\texttt{PARTITION} is {\em yes} if and only if $\sum_{t=1}^TR(S_t^*) = T\cdot B/(1+B).$
\end{claim}
Before proving the claim, note that it directly implies the strong NP-hardness of \ref{APVC}, even in the special case of equal prices and integer preference-weights, as it shows that $3$-\texttt{PARTITION} reduces to this special case of \ref{APVC} (see {\cite[Observation 5][504]{garey1978strong}}). Let us now prove the claim.
\begin{proof}[Proof of Claim \ref{cl:NPhard}]\phantom \\

\noindent \underline{Direct implication}. Suppose that a valid $3$-\texttt{PARTITION} exists, which we denote by $A_1,\ldots, A_T$, and let $S_t = \{i \,\colon\, a_i\in A_t\}$ for $t\in [T]$. On one hand, we have$$
    \sum_{t=1}^TR(S_t^*)\geq \sum_{t=1}^TR(S_t) = T\cdot\frac{B}{1+B},
$$
and on the other hand, we have using Jensen's inequality and the concavity of the map $x\mapsto x/(1+x)$,$$
    \sum_{t=1}^TR(S_t^*) = T \sum_{t=1}^T\frac{1}{T}\cdot\frac{V(S_t^*)}{1+V(S_t^*)} \leq T\cdot \frac{\sum_{t=1}^T\frac{V(S_t^*)}{T}}{1+\sum_{t=1}^T\frac{V(S_t^*)}{T}} = T\cdot\frac{B}{1+B}.
$$
This implies that $\sum_{t=1}^TR(S_t^*) = T\cdot B/(1+B)$ and proves the direct implication.

\noindent \underline{Indirect implication}. Assume that $$\sum_{t=1}^TR(S_t^*) = T\cdot \frac{B}{1+B}.$$
On one hand, we have according to Cauchy-Schwarz inequality,$$
    \sum_{t=1}^T(1+V(S_t^*)) \cdot \sum_{t=1}^T\frac{1}{1+V(S_t^*)} \geq T^2,
$$
with equality if and only if there exists some $\lambda$ such that for all $t\in [T]$, $1+V(S_t^*) = \lambda / (1+V(S^*_t))$, which in turn is equivalent to $V(S^*_1) = V(S^*_2)=\cdots = V(S^*_T) = \sqrt{\lambda}-1$.

On the other hand, it is easy to see that $$
    \sum_{t=1}^TR(S_t^*) = T - \sum_{t=1}^T\frac{1}{1+V(S^*_t)}.
$$
Therefore, \begin{align*}
    \sum_{t=1}^T(1+V(S_t^*)) \cdot \sum_{t=1}^T\frac{1}{1+V(S_t^*)} & = \left(T+V\left(\Nc\right)\right)\cdot \left(T - \sum_{t=1}^TR(S_t^*)\right)\\
    & =  \left(T+TB\right)\cdot \left(T - T\cdot\frac{B}{1+B}\right)\\
    & = T^2,
\end{align*}
This proves the equality case in Cauchy-Schwarz inequality, and consequently that $V(S^*_1) = V(S^*_2) = \cdots = V(S^*_T)$. Finally, taking $A_t = \{a_i\,:\, i\in S_t^*\}$ for $t\in [T]$, it is easy to see that $A_1,\ldots, A_T$ is a valid $3$-\texttt{PARTITION}, which concludes the proof of the claim.
\end{proof}

In the remainder of this proof, we show that the problem does not admit an FPTAS, relying on the following claim.
\begin{claim}\label{cl:noFPTAS}
    If the answer to the $3$-\texttt{PARTITION} is {\em no}, then $$
        \sum_{t=1}^TR(S_t^*) \leq (1-\delta)\cdot T\cdot \frac{B}{1+B},
    $$
    where $\delta = \frac{2}{TB^2(B+2)}$.
\end{claim}
Finally, the non-existence of an FPTAS for this problem follows naturally. Indeed, assume for the sake of contradiction that an FPTAS exists for \ref{APV}. Then by taking $\eps < \delta$, say $\eps = \delta/2$ for example, the FPTAS would return a solution $(S_1^*,\ldots, S_T^*)$ with an objective greater than or equal to $(1-\epsilon)\cdot TB/(1+B)$ if an only if a valid partition exists. This yields an algorithm that solves the $3$-\texttt{PARTITION} problem in pseudo-polynomial time, thereby contradicting its strong NP-hardness.
\\{\em Conclusion. }\ref{APV} is strongly NP-hard, and admits no FPTAS.






\begin{proof}[Proof of Claim \ref{cl:noFPTAS}]
Suppose the answer to the $3$-\texttt{PARTITION} problem is {\em no}.
Consider the following integer program:
    \begin{equation}\label{eq:clprog}
        \begin{aligned}
         \max_{\mathbf{b} \in \N^T}  & \; \;      \sum_{t=1}^T  \frac{b_t}{1+b_t}   \\  
          s.t. \;\;   & \;\;  \sum_{t=1}^Tb_t = BT\\
          &\,\, \mathbf{b} \neq (B,B,\ldots,B).
        \end{aligned}
    \end{equation}
Intuitively, this program's objective is to partition a budget of $BT$ into $T$ bins, under the constraint that not all the bins receive the same fraction of budget, and such that each bin $t\in [T]$ yields a return of $b_t/(1+b_t)$. Let $\mathbf{b}^*$ be the optimal solution of this problem. The idea of the proof is to show that:
\begin{equation}\label{eq:doubleineq}
    \sum_{t=1}^TR(S_t^*) \leq \sum_{t=1}^T\frac{b^*_t}{1+b^*_t} = (1-\delta)\cdot \frac{TB}{1+B}.
\end{equation}
\paragraph{The inequality. }
Let $\mathbf{b} = (V(S_1^*),\ldots, V(S_T^*))$. We have $\sum_{t=1}^TV(S_t^*) = BT$, and since a valid partition does not exist, then $\mathbf{b}\neq (B,B,\ldots,B)$  (see Claim \ref{cl:NPhard}). Therefore $\mathbf{b}$ is feasible for \eqref{eq:clprog}, and we have:
$$
    \sum_{t=1}^TR(S_t^*) = \sum_{t=1}^T\frac{b_t}{1+b_t} \leq \sum_{t=1}^T\frac{b^*_t}{1+b^*_t},
$$
by the optimality of $\mathbf{b}^*$.
\paragraph{The equality. }We show that $(B+1, B-1, B,\ldots,B)$ is an optimal solution for \eqref{eq:clprog}. Towards this purpose, we start by showing that the number of indices $u$ for which $b^*_u \neq B$ is at least two, before showing by contradiction that it is exactly $2$, then we show that the entries of these two indices in $\mathbf{b}^*$ are exactly $B+1$ and $B-1$. Since the indices are all interchangeable, this shows that $(B+1, B-1, B,\ldots,B)$ is an optimal solution for \eqref{eq:clprog}.

Let ${\cal U}\subseteq [T]$ be the set of indices $u$ such that $b^*_u\neq B$, i.e., ${\cal U} = \{u\in [T]\,:\, b_u^* \neq B\}$. First, ${\cal U} \neq \emptyset$, by the second constraint of \eqref{eq:clprog}. Also, if we suppose by contradiction that $|{\cal U}| = 1$, say ${\cal U} = \{1\}$ without loss of generality, then $\sum_{t=1}^T{b^*_t} = b_1^* + (T-1)B \neq BT$, which contradicts the first constraint of \eqref{eq:clprog}. Therefore, $|{\cal U}| \geq 2$.

Suppose by contradiction that $|{\cal U}| > 2$. Since the elements of $\cal U$ cannot all be smaller, or all be greater than $B$ (otherwise the second constraint in \eqref{eq:clprog} does not hold), there exists $u_1, u_2\in [T]$ such that $b_{u_1}^* \geq B+1$ and $b_{u_2}^* \leq B-1$. Also, since $|{\cal U}| >2$, there exists some $u_3\neq u_1,u_2$, such that $b_{u_3}^* \neq B$. For simplicity, we assume without loss of generality that $u_1 = 1$, $u_2 = 2$ and $u_3 = 3$. Let $ {\mathbf{\Tilde b}} = (b_1^*-1,b_2^*+1, b_3^*,\ldots, b_T^*)$, in other words, we transfer a unit of the budget from $b_1^*$ to $b_2^*$. First, $\sum_{t=1}^T{\Tilde b}_t = \sum_{t=1}^Tb_t^* =BT$. Moreover, since $3\in {\cal U}$, $\Tilde b_3 = b_3^* \neq B$, which implies that $\mathbf{\Tilde b}$ is feasible for \eqref{eq:clprog}. Finally, we have
\begin{align*}
    \sum_{t=1}^T \frac{\Tilde b_t}{1+\Tilde b_t} &=  \frac{b^*_1-1}{1+b^*_1-1} +\frac{b^*_2+1}{1+b^*_2+1} + \sum_{t=3}^T\frac{b^*_t}{1+b^*_t}\\
    & > \frac{b^*_1}{1+b^*_1} +\frac{b^*_2}{1+b^*_2} + \sum_{t=3}^T\frac{b^*_t}{1+b^*_t}\\
    & = \sum_{t=1}^T\frac{b^*_t}{1+b_t^*},
\end{align*}
where the inequality in the second line is due to the fact that the map
$$
    f:x\mapsto \frac{b_1^*-x}{1+b_1^*-x}+\frac{b_2^*+x}{1+b_2^*+x}
$$
is monotonically (strictly) increasing on $[0,1]$ (which we show at the end of the proof for completeness). This contradicts the optimality of $\mathbf{b}^*$, and thereby shows that $|{\cal U}| =2$.

Assume without loss of generality that ${\cal U} = \{1,2\}$, and that $b^*_1 = B+\alpha$, and $b^*_2 = B-\alpha$ for some $\alpha \in \N\setminus \{0\}$. Assume by contradiction that $\alpha \geq 2$, then using the same argument as the contradiction above, the map 
$$
    x\mapsto \frac{b_1^*-x}{1+b_1^*-x}+\frac{b_2^*+x}{1+b_2^*+x}
$$
is increasing, and we can therefore transfer a unit of the budget from $1$ to $2$, and strictly increase the objective of the obtained solution. Moreover, since $\alpha\geq 2$, $b_1^*-1 \neq B$, and the second constraint in \eqref{eq:clprog} is valid, which proves the feasibility of the obtained solution, and contradicts the optimality of $\mathbf{b}^*$. Therefore, $\alpha=1$ and $(B+1, B-1, B,\ldots,B)$ is an optimal solution for \eqref{eq:clprog}.

Going back to the seeked equality \eqref{eq:doubleineq}, we have\begin{align*}
    \sum_{t=1}^T\frac{b^*_t}{1+b_t^*} &= \frac{B-1}{1+B-1} + \frac{B+1}{1+B+1} + (T-2)\cdot \frac{B}{1+B} \\
    % & = \frac{TB}{1+B} - \left(\frac{2B}{1+B} - \frac{B-1}{B} - \frac{B+1}{B+2}\right)\\
    % & = \frac{TB}{1+B}\cdot \left(1 - \frac{2}{T} + \frac{(B-1)(B+1)}{TB^2} - \frac{(B+1)^2}{TB(B+2)}\right)\\
    % & = \frac{TB}{1+B} \cdot \left(1 - \frac{2}{TB^2(B+2)}\right)\\
    & = \frac{TB}{1+B} \cdot \left(1 - \delta\right).
\end{align*}
The second equality follows from basic algebra operations. This proves Equation \eqref{eq:doubleineq}, which in turn proves the claim.
\paragraph{Leftover: Proof of the strict monotonicity of $f$. }
For all $x\in [0,1]$, recall that $$
    f(x) = \frac{b_1^*-x}{1+b_1^*-x}+\frac{b_2^*+x}{1+b_2^*+x}.
$$
We differentiate $f$, and we have for all $x\in [0,1]$
$$
    f'(x) = \frac{1}{\left(1+b_2^*+x\right)^2}-\frac{1}{\left(1+b_1^*+x\right)^2}
$$
Therefore, it is easy to see that $f'(x) > 0$ for all $x<(b^*_1-b^*_2)/2$. Since $b_1^*\geq B+1$ and $b_2^*\leq B-1$, then $(b^*_1-b^*_2)/2 \leq 1$, which proves that $f$ is monotonically strictly increasing on $[0,1]$.
\end{proof}



\subsection{Proof of Lemma \ref{lem:LPopt}}\label{apx:LPopt}
    In this proof, we show separate inequalities for light, medium, and heavy customers. Combining these inequalities then gives the desired result.
    
    \noindent {\bf Heavy customers:} If $t\in [T]$ is heavy then by the fifth constraint of \ref{eq:ILP}, we have \begin{equation*}
            \sum_{i\in \Nc} v_ix^*_{it} \geq \frac{1}{\eps}.
        \end{equation*}
        Therefore, \begin{equation}\label{eq:heavy}
            \frac{\sum_{i\in \Nc} v_ix^*_{it}}{1+\sum_{i\in \Nc} v_ix^*_{it}} \geq \frac{1}{1+\eps} \geq (1-\eps) \cdot \frac{v(S_t^*)}{1+v(S_t^*)}.
        \end{equation}
    
    \noindent {\bf Medium customers:} If $t\in G_\ell$ for some $\ell \in \{1,\ldots,L\}$, then we have
        \begin{equation*}
            \sum_{i\in \Nc} v_i x_{it} \geq \eps\cdot (1+\eps)^{\ell-1} \geq \frac{1}{1+\eps} v(S_t^*),
        \end{equation*}
        where the first inequality follows from the fifth constraint of \ref{eq:ILP}, and the second follows from the fact that $v(S_t^*) \leq \eps \cdot (1+\eps)^{\ell}$ by definition of customers of type $\ell$.
        Therefore,\begin{equation}\label{eq:medium}
            \frac{\sum_{i\in \Nc} v_ix^*_{it}}{1+\sum_{i\in \Nc} v_ix^*_{it}} \geq \frac{1}{1+\eps} \cdot \frac{v(S_t^*)}{1+v(S_t^*)} =  (1-\eps)\cdot \frac{v(S_t^*)}{1+v(S_t^*)}.
        \end{equation}
    
    \noindent {\bf Light customers:} Let $\mathbf{X^*}$ be the solution defined as follows, for every $t\in [T]$, and every $i\in \Nc$, $X^*_{it} = \mathbbm 1\{i \in S_t^*\}$. It is easy to see that by construction of \ref{eq:ILP}, the solution $\mathbf{X^*}$ is feasible for \ref{eq:ILP}, and therefore, it is also feasible for its linear relaxation. Therefore, since $\mathbf{x^*}$ is optimal for \ref{eq:ILP}, we have\begin{equation*}
            \sum_{t\in G_{light}} \sum_{i\in \Nc} v_ix^*_{it} \geq \sum_{t\in G_{light}} \sum_{i\in \Nc} v_iX^*_{it} =  \sum_{t\in G_{light}} v(S_t^*).
        \end{equation*}
        We also have for all $t\in G_{light}$, $\sum_{i\in \Nc} v_i x^*_{it}\leq \eps$ by  the sixth constraint of \ref{eq:ILP}. Therefore, \begin{align}
            \sum_{t\in G_{light}}\frac{\sum_{i\in \Nc} v_ix^*_{it}}{1+\sum_{i\in \Nc} v_ix^*_{it}}&\geq \frac{1}{1+\eps}\cdot \sum_{t\in G_{light}}\sum_{i\in \Nc} v_ix^*_{it}\notag\\
            &\geq \sum_{t\in G_{light}} v(S_t^*)\notag\\
            & \geq \sum_{t\in G_{light}}\frac{v(S_t^*)}{1+v(S_t^*)}.\label{eq:light}
        \end{align}
    We conclude the proof by combining Equations \eqref{eq:heavy}, \eqref{eq:medium} and \eqref{eq:light}, as we have\begin{equation*}
        \sum_{t=1}^T\frac{\sum_{i\in \Nc} v_ix^*_{it}}{1+\sum_{i\in \Nc} v_ix^*_{it}}\geq (1-\eps)\cdot\sum_{t=1}^T\frac{v(S_t^*)}{1+v(S_t^*)}. 
    \end{equation*}

















\subsection{Proof of Claim \ref{cl:nonlightcust}}\label{apx:nonlightcust}
We have\begin{align*}
    W_t^\lar& = \sum_{q = 1}^Q\sum_{i\in D_q}v_iX_{it}\\
    & = \sum_{q = 1}^Q \eps^5\cdot(1+\eps)^{q-1}\sum_{i\in D_q}X_{it}\\
    & = \sum_{q = 1}^Q \eps^5\cdot(1+\eps)^{q-1}\sum_{i\in D_q}x^*_{it}\\
    & = \sum_{q = 1}^Q\sum_{i\in D_q}v_ix^*_{it}\\
    & = w_t^\lar,
\end{align*}
where the second equality follows from the definition of $D_q$, and the third inequality follows from the degree preservation property \ref{it:P2}.





















% \subsection{Proof of Claim \ref{cl:lightcusts}}\label{apx:lightcust}
% First, note that the proof of Claim \ref{cl:nonlightcust} holds even for the case where $t$ is a light customer. Therefore, we can decompose $v(S_t)$ into a sum of a random variable $W_t^\sma$ representing the contribution of products in $D_0$, and a deterministic random variable $W_t^{\lar} \overset{a.s}{=}w_t^\lar$ (which will be treated as a constant from this point on), i.e.,
% \begin{equation*}
%     v(S_t) = W_t^\sma + w_t^\lar.    
% \end{equation*}
% For simplicity of notation, and only in this proof, we use the notation $W\coloneqq W_t^\sma$ and $w \coloneqq w_t^\lar$.
% This proof consists on two steps. In the first step, we lower bound the expectation of $v(S_t)/(1+v(S_t))$ with $\sum_{i\in \Nc}v_ix^*_{it}$, up to a constant $\alpha<1$, mainly using a convexity argument. In other words, we provide a constant $\alpha$ such that $$
%     v(S_t)/(1+v(S_t)) \geq \alpha\cdot \sum_{i\in \Nc}v_ix^*_{it}.
% $$
% Then, in the second step, we show that $\alpha = (1-O(\eps))$.
% \paragraph{Step 1:}
% Let $\cal Y$ be the support of the random variable $W$, i.e.,$$
%     {\cal Y} = \left\{y\in\R\,\colon\, \P\left[W= y\right]>0\right\}.
% $$
% We have
% \begin{align*}
%     \E\left[\frac{v(S_t)}{1+v(S_t)}\right] & = \E\left[\frac{W+w}{1+W+w}\right]\\
%     & = \sum_{y \in {\cal Y}} \frac{y+w}{1+y+w}\P\left[W=y\right]\\
%     & = \left(\sum_{y\in {\cal Y}} \frac{1}{1+y+w}\cdot \frac{(y+w)\cdot \P\left[W=y\right]}{w+\E\left[W\right]}\right)\cdot \left(w+ \E\left[W\right]\right).
% \end{align*}
% For each $y\in {\cal Y}$, let $f(y) = \frac{1}{1+y+w}$ and let $t_y = \frac{(y+w)\cdot \P\left[W=y\right]}{w+\E\left[W\right]}$. It is easy to see that $\sum_{y\in {\cal Y}} t_y=1$, and that $f$ is a convex function on $(0, +\infty)$. Therefore, using Jensen's inequality, and recalling that $\E[W]+w = \E[v(S_t)]$ we have\begin{align*}
%     \E\left[\frac{v(S_t)}{1+v(S_t)}\right] &\geq f\left(\sum_{y\in {\cal Y}} t_y \cdot y\right)\cdot \E\left[v(S_t)\right]\\
%     & = \alpha\cdot \E\left[v(S_t)\right],
% \end{align*}
% where $\alpha = f\left(\sum_{y\in {\cal Y}} t_y \cdot y\right)$.
% \paragraph{Step 2.} Let us now show that $\alpha = 1-O(\eps)$. We have
% \begin{align}
%         \alpha& = \frac{1}{1+w + \frac{\sum_{y\in {\cal Y}}(y+w)\cdot \P[W=y]\cdot y}{w+\E[W]}} \notag\\
%     & = \frac{1}{1+w+\frac{w\E[W] + \E[W^2]}{w+\E[W]}}\notag\\
%     & = \frac{1}{1+w+\E[W] + \frac{Var[W]}{w+\E[W]}}.\label{eq:last}
% \end{align}
% In order to show the desired property, we need to bound the denomination in Equation \eqref{eq:last}. To this purpose we have\begin{align*}
%     Var[W] & = Var\left[\sum_{i\in D_0} v_iX_{it}\right]\\
%         & \leq \sum_{i\in D_0} v_i^2\cdot Var[X_{it}]\\
%         &\leq \eps^5\cdot \sum_{i\in D_0} v_i\cdot \E[X_{it}]\\
%         &= \eps^5\cdot \E[W].
% \end{align*}
% Therefore, by replacing in Equation \eqref{eq:last}, we have\begin{align*}
%     \alpha \geq \frac{1}{1+\eps+\eps^5} \geq \frac{1}{1+2\eps}\geq 1-2\eps.
% \end{align*}

% Finally, combining steps 1 and 2, we have \begin{align*}
%     \frac{v(S_t)}{1+v(S_t)} &\geq \alpha\cdot \E[v(S_t)]\\
%                     & \geq (1-2\eps)\cdot \sum_{i\in \Nc} v_ix^*_{it}\\
%                     &\geq (1-2\eps)\cdot \frac{\sum_{i\in \Nc} v_ix^*_{it}}{1+\sum_{i\in \Nc} v_ix^*_{it}}.     
%         \end{align*}

























\subsection{Proof of Claim \ref{cl:lightcusts}}\label{apx:lightcust}
First, note that the proof of Claim \ref{cl:nonlightcust} holds even for the case where $t$ is a light customer. Therefore, we can decompose $v(S_t)$ into a sum of a random variable $W_t^\sma$ representing the contribution of products in $D_0$, and a deterministic random variable $W_t^{\lar} \overset{a.s}{=}w_t^\lar$ (which will be treated as a constant from this point on), i.e.,
\begin{equation*}
    v(S_t) = W_t^\sma + w_t^\lar.    
\end{equation*}
For simplicity of notation, and only in this proof, we use the notation $A\coloneqq W_t^\sma$ and $b \coloneqq w_t^\lar$.
This proof consists on two steps. In the first step, we lower bound the expectation of $v(S_t)/(1+v(S_t))$ with $\sum_{i\in \Nc}v_ix^*_{it}$, up to a constant $\alpha<1$, mainly using a convexity argument. In other words, we provide a constant $\alpha$ such that $$
    v(S_t)/(1+v(S_t)) \geq \alpha\cdot \sum_{i\in \Nc}v_ix^*_{it}.
$$
Then, in the second step, we show that $\alpha = (1-O(\eps))$.
\paragraph{Step 1:}
Let $\cal A$ be the support of the random variable $A$, i.e.,$$
    {\cal A} = \left\{a\in\R\,\colon\, \P\left[A= a\right]>0\right\}.
$$
We have
\begin{align*}
    \E\left[\frac{v(S_t)}{1+v(S_t)}\right] & = \E\left[\frac{A+b}{1+A+b}\right]\\
    & = \sum_{a \in {\cal A}} \frac{a+b}{1+a+b}\P\left[A=a\right]\\
    & = \left(\sum_{a\in {\cal A}} \frac{1}{1+a+b}\cdot \frac{(a+b)\cdot \P\left[A=a\right]}{b+\E\left[A\right]}\right)\cdot \left(b+ \E\left[A\right]\right).
\end{align*}
For each $a\in {\cal A}$, let $f(a) = \frac{1}{1+a+b}$ and let $z_a = \frac{(a+b)\cdot \P\left[A=a\right]}{b+\E\left[A\right]}$. It is easy to see that $\sum_{a\in {\cal A}} z_a=1$, and that $f$ is a convex function on $(0, +\infty)$. Therefore, using Jensen's inequality, and recalling that $\E[A]+b = \E[v(S_t)]$ we have\begin{align*}
    \E\left[\frac{v(S_t)}{1+v(S_t)}\right] &\geq f\left(\sum_{a\in {\cal A}} z_a \cdot y\right)\cdot \E\left[v(S_t)\right] = \alpha\cdot \E\left[v(S_t)\right],
\end{align*}
where $\alpha = f\left(\sum_{a\in {\cal A}} z_a \cdot a\right)$.
\paragraph{Step 2.} Let us now show that $\alpha = 1-O(\eps)$. We have
\begin{align}
        \alpha& = \frac{1}{1+b + \frac{\sum_{a\in {\cal A}}(a+b)\cdot \P[A=a]\cdot a}{b+\E[A]}} \notag\\
    & = \frac{1}{1+b+\frac{b\E[A] + \E[A^2]}{b+\E[A]}}\notag\\
    & = \frac{1}{1+b+\E[A] + \frac{Var[A]}{b+\E[A]}}.\label{eq:last}
\end{align}
In order to show the desired property, we need to bound the denominator in Equation \eqref{eq:last}. To this purpose we have\begin{align*}
    Var[A] & = Var\left[\sum_{i\in D_0} v_iX_{it}\right]\\
        & \leq \sum_{i\in D_0} v_i^2\cdot Var[X_{it}]\\
        &\leq \eps^5\cdot \sum_{i\in D_0} v_i\cdot \E[X_{it}]\\
        &= \eps^5\cdot \E[A].
\end{align*}
Therefore, by replacing in Equation \eqref{eq:last}, we have\begin{align*}
    \alpha \geq \frac{1}{1+\eps+\eps^5} \geq \frac{1}{1+2\eps}\geq 1-2\eps.
\end{align*}
Finally, combining steps 1 and 2, we have \begin{align*}
    \frac{v(S_t)}{1+v(S_t)} &\geq \alpha\cdot \E[v(S_t)]  \geq (1-2\eps)\cdot \sum_{i\in \Nc} v_ix^*_{it} 
                    \geq (1-2\eps)\cdot \frac{\sum_{i\in \Nc} v_ix^*_{it}}{1+\sum_{i\in \Nc} v_ix^*_{it}}.     
        \end{align*}



























    \section{MNL-based Revenue Maximization}
    \label{apx1}
For completeness, we present two well-known results in the context of unconstrained revenue maximization problem under the MNL model, defined as follows
\begin{equation}
\label{Unconstrained}\tag{\sf{AP}}
\max_{S \subseteq \mathcal{N}} \quad  R(S).
\end{equation} 
Lemma \ref{Revenue variations} provides necessary and sufficient conditions under which adding a product to a given assortment increases its expected revenue. Then, Lemma \ref{lem:nestedrevenue}, shows that the optimal assortment for \ref{Unconstrained} is price-ordered (see, e.g., \citet[Prop.~6]{talluri2004revenue}).

The next lemma shows that adding a product $j$ to any given assortment $S$ weakly increases its expected revenue if and only if the price of this product $p_j$ is at least $R(S)$, or equivalently, at least $R(S\cup\{j\})$.

\begin{lemma}\label{Revenue variations}
    For any assortment $S\subseteq {\cal N}$ and any product $j \in {\cal N} \setminus S$, the next three conditions are equivalent:
$$(i) \; R(S \cup \{j\}) \geq R(S),  \qquad (ii)  \; p_j \geq R(S),  \qquad (iii) \; p_j \geq R(S\cup \{j\}). $$
\end{lemma}


\begin{lemma}\label{lem:nestedrevenue}
    \label{Unconstrained solution}
    Let $R^* = \max_{S \subseteq \mathcal{N}} R(S)$. Then, the price-ordered assortment $S^* = \{i \in \mathcal{N}, p_i \geq R^* \}$ is optimal.
\end{lemma}




    


%\begin{lemma}\label{Revenue variations}
%    For any assortment $S\subseteq {\cal N}$ and any product $k \in {\cal N} \setminus S$, the three following propositions are equivalent
 %   \begin{itemize}
  %      \item[(i)] $R(S \cup \{k\}) \geq R(S)$.
   %     \item[(ii)] $p_k \geq R(S)$.
    %    \item[(iii)] $p_k \geq R(S\cup \{k\})$.
  %  \end{itemize}
%\end{lemma}
    
\begin{proof}[Proof of Lemma \ref{Revenue variations}]
    We show that $R(S\cup \{j\})$ is a convex combination of $R(S)$ and $p_j$. To this end, note that
    \begin{align*}
        R\left(S\cup \{j\}\right) &= \sum_{i\in S} p_i\cdot\phi\left(i,S\cup \{j\}\right) + p_j\cdot\phi\left(j,S\cup \{j\}\right)
   % & = \sum_{i\in S} p_i\cdot\phi\left(i,S\right) \cdot \frac{1+V(S)}{1+V(S\cup \{k\})}+ p_k\cdot\phi\left(k,S\cup \{k\}\right)\\
     = \alpha \cdot R(S)  + (1-\alpha)\cdot p_j,
    \end{align*}
    where $\alpha = \frac{1+V(S)}{1+V(S\cup\{j\})}$. It is easy to verify that $\alpha \in (0,1]$,  which proves that $R(S\cup \{j\})$ is a convex combination of $R(S)$ and $p_j$. Consequently, $R(S\cup \{j\})$ belongs to the closed interval whose endpoints are $R(S)$ and $p_j$. In particular, we have $R(S\cup \{j\}) \geq R(S)$ if and only if $p_j \geq R(S)$, and if and only if $p_j \geq R(S\cup \{j\})$.
\end{proof}

%We can now deduce the proof of Lemma \ref{lem:nestedrevenue}.
\begin{proof}[Proof of Lemma \ref{lem:nestedrevenue}]
Let $\Tilde{S}$ be an optimal assortment of \ref{Unconstrained} with maximal cardinality. In the following, we show that $S^* = \Tilde S$.
% , which in particular implies that the maximum-cardinality optimal assortment is unique.
We begin by arguing that $S^*\subseteq \Tilde S$. Consider some product $i\in S^*$, and assume by contradiction that $i\notin \Tilde S$. On the one hand, we know by optimality of $\Tilde S$ that $R(\Tilde S) = R^*$. On the other hand, $p_i\geq R^*$ by definition of $S^*$. Therefore, $p_i\geq R(\Tilde S)$ and we can apply \black{Lemma} \ref{Revenue variations}, which implies that $R(\Tilde{S}\cup \{i\})\geq R(\Tilde S)$. Hence, $\Tilde{S}\cup \{i\}$ is also an optimal assortment, contradicting the definition of $\Tilde{S}$ as an optimal assortment with maximal cardinality.
Now, to show that $\Tilde S\subseteq S^*$, suppose there exists a product $i\in \Tilde S$ that does not belong to $S^*$, i.e., $p_i < R^* = R(\Tilde S)$. By Lemma \ref{Revenue variations}, the latter inequality implies that $R(\Tilde S\setminus \{i\}) > R(\Tilde S)$, which contradicts the optimality of $\Tilde S$.
\end{proof}

\section{\black{Proof of Claim \ref{cl:computation}}}\label{apx:computation}
\black{Let $S_1, S_2\subseteq \Nc$ with $S_1\subseteq S_2$. Then, \begin{align*}
    R(S_2) &= \sum_{j\in S_2} \frac{p_j\cdot v_j}{1+V(S_2)} \\
    & = \sum_{j\in S_1} \frac{p_j\cdot v_j}{1+V(S_2)}  + \sum_{j\in S_2\setminus S_1} \frac{p_j\cdot v_j}{1+V(S_2)}\\
    & = R(S_1) \cdot \frac{1+V(S_1)}{1+V(S_2)} + \sum_{j\in S_2\setminus S_1} \frac{p_j\cdot v_j}{1+V(S_2)}
\end{align*}
Therefore,
\begin{align*}
    R(S_1) - R(S_2) &= R(S_1) \cdot \left(1 - \frac{1+V(S_1)}{1+V(S_2)}\right) - \sum_{j\in S_2\setminus S_1} \frac{p_j\cdot v_j}{1+V(S_2)}\\
    & = R(S_1) \cdot  \frac{\sum_{j\in S_2\setminus S_1}v_j}{1+V(S_2)} - \sum_{j\in S_2\setminus S_1} \frac{p_j\cdot v_j}{1+V(S_2)}\\
    & = \frac{1}{1+V(S_2)}\cdot \sum_{j\in S_2\setminus S_1}\left(R(S_1)  - p_j\right)\cdot v_j
\end{align*}
}

\section{Additional Proofs from Section \ref{sec:APVC}}\label{apx:APVC}
\subsection{Proof for Theorem \ref{NP-hardness}}\label{apxhard}
In order to prove that \ref{APVC} is strongly NP-hard, we reduce the $3$-\texttt{PARTITION} problem to \ref{APVC}. \black{In this setting}, the input consists of $3T$ positive integers ${\cal A} = \{a_1,\ldots, a_{3T}\}$ with $\sum_{i=1}^{3T}a_i=BT$, for some $B>0$. We are asked to determine whether $\cal A$ can be partitioned into $T$ triplets $A_1,\ldots, A_T$, such that the sum of each triplet is exactly $B$. This problem is known to be strongly NP-hard (see, e.g., \citet[Theorem~3.5]{garey1975complexity})
% $3$-\texttt{PARTITION} is referred to there as $P[3,1]$).
\paragraph{Instance construction.} Given a $3$-\texttt{PARTITION} instance of the form described above, we construct the following \ref{APVC} instance. We consider \black{$T$ customers and} $3T$ products, comprising the universe $\Nc = \{1,\ldots,3T\}$.  Each product $i\in \Nc$, has a preference weight $v_i = a_i$, a price $p_i = 1$, and its visibility requirement is $\ell_i = 1$. In particular, note that $V(\Nc) = \sum_{i=1}^{3T} a_i= BT$. Finally, the upper bound on the cardinality of any offered assortment is $k=3$.
\paragraph{Analysis.} An important observation is that any feasible solution for this instance offers each product to exactly one customer. In other words, if $S_1,\ldots ,S_T$ is a feasible solution, then for any product $i\in \Nc$, there exists a unique customer $t\in [T]$ for which $i\in S_t$. Indeed, on the one hand, the visibility constraints of $\ell_1=\ell_2= \ldots= \ell_{3T} =1$ impose that each product will be shown at least once, and on the other hand, since each assortment can contain at most $k=3$ products and there are $3T$ products in our universe, each product can be offered at most once. As a consequence, for any feasible solution $S_1,\ldots, S_T$, we have $\sum_{t=1}^TV(S_t) = V(\Nc) = BT$.

Now, let us denote by $S^* = (S^*_1,\ldots,S^*_T)$ an optimal collection of assortments for our \ref{APVC} instance. The following claim shows that a valid partition exists for the original $3$-\texttt{PARTITION} instance if and only if the expected revenue of $S^*$ exceeds a certain threshold. A close inspection of this result reveals that it directly implies the strong NP-hardness of \ref{APVC}, even with equal prices and integer-valued preference-weights.
\begin{lemma}\label{lem:NPhard}
    ${\cal A} = \{a_1,\ldots, a_{3T}\}$ is a YES-instance of $3$-\texttt{PARTITION} if and only if $\sum_{t=1}^TR(S_t^*) \geq T\cdot \frac{B}{1+B}.$
\end{lemma}
\paragraph{First direction: YES-instance $\boldsymbol{\implies \sum_{t=1}^TR(S_t^*) \geq T\cdot \frac{B}{1+B}}$.}
Suppose that $A_1,\ldots, A_T$ is a valid partition, and let $S_t = \{i\,\colon\,a_i\in A_t\}$. Then, $$
    \sum_{t=1}^TR(S_t^*) \geq \sum_{t=1}^TR(S_t)  = T\cdot \frac{B}{1+B}. 
$$
\paragraph{Second direction: NO-instance $\boldsymbol{\implies \sum_{t=1}^TR(S_t^*) \leq T\cdot \frac{B}{1+B} - \frac{2}{B(B+1)(B+2)}}$.}
Consider the following integer program:
    \begin{equation}\tag{IP}\label{eq:integerprogram}
        \begin{aligned}
         \max_{\mathbf{b} \in \N^T}  & \; \;      \sum_{t=1}^T  \frac{b_t}{1+b_t}   \\  
          {\rm s.t.} \;\;   & \;\;  \sum_{t=1}^Tb_t = BT.
        \end{aligned}
    \end{equation}
For any feasible solution $(b_1,\ldots,b_T)$, we have{\black{\begin{align}
    \sum_{t=1}^T \frac{b_t}{1+b_t}& = T\cdot \sum_{t=1}^T \frac1T\cdot \frac{b_t}{1+b_t} \notag\\ & \leq T\cdot \frac{\frac{1}{T}\cdot\sum_{t=1}^T{b_t}}{1+ \frac{1}{T}\cdot\sum_{t=1}^T{b_t}} \label{eq:15}\\ &= T\cdot \frac{B}{1+B}.\label{eq:bound}
\end{align}}}\\
\black{Inequality~\eqref{eq:15}} follows from Jensen's inequality and the concavity of $x\mapsto x/(1+x)$, and \black{Equality~\eqref{eq:bound}} is a consequence of $\sum_{t=1}^Tb_t = BT$. 
In other words, $\opt({\rm \ref{eq:integerprogram}}) \leq T\cdot B/(1+B)$.
    
Now consider any feasible sequence of assortments $(S_1,\ldots, S_T)$. Since ${\cal A}$ is a NO-instance, there exists \black{at least one pair of customers} $u,v\in [T]$ for which $V(S_u) \geq B+1$ and $V(S_v)\leq B-1$. We assume without loss of generality that $u=1$ and $v=2$. Consider the feasible solution $\mathbf b$ for \ref{eq:integerprogram} defined as follows: $b_1 = V(S_1)-1$, $b_2 = V(S_2) +1$ and $b_t = V(S_t)$ for all $t = 3,\ldots,T$.
\black{Then, to obtain an upper bound on $\sum_{t=1}^TR(S_t)$, we observe that} \begin{align}
    T\cdot \frac{B}{1+B} - \sum_{t=1}^TR(S_t) &\geq \sum_{t=1}^T\frac{b_t}{1+b_t} - \sum_{t=1}^T\frac{V(S_t)}{1+V(S_t)}\label{eq:one}\\
    & = \left(\frac{V(S_1)-1}{V(S_1)} - \frac{V(S_1)}{1+V(S_1)}\right) + \left(\frac{V(S_2)+1}{V(S_2)+2} - \frac{V(S_2)}{1+V(S_2)}\right)\label{eq:two}\\
    & = \frac{1}{(1+V(S_2))(2+V(S_2))} - \frac{1}{V(S_1)(1+V(S_1))}\notag\\
    & \geq \frac{1}{B(B+1)} - \frac{1}{(B+1)(B+2)}\label{eq:three}\\
    & = \frac{2}{B(B+1)(B+2)}.\notag
\end{align}
Here, Equation \eqref{eq:one} \black{holds since $\opt({\rm \ref{eq:integerprogram}}) \leq T\cdot \frac{B}{1+B}$, as shown above}. In Equation \eqref{eq:two}, we simply substitute the entries of $\mathbf b$. In Equation \eqref{eq:three}, we use the fact that $V(S_1)\geq B+1$ and $V(S_2) \leq B-1$.
\paragraph{Non-existence of an FPTAS.}
Suppose by contradiction that an FPTAS for \ref{APVC} exists. Letting $\eps = \frac{1}{TB^2(B+2)}$, since this quantity is pseudo-polynomial in the input size, we can obtain a $(1-\eps)$-approximation $(S_1,\ldots, S_T)$ in pseudo-polynomial time.
% By the contrapositive of the implication above, $\sum_{t=1}^TR(S_t) > T\cdot \frac{B}{1+B}-\frac{2}{B(B+1)(B+2)}$ implies that $\cal A$ is a YES-instance. Moreover, if $\cal A$ is a YES-instance, then
% \begin{equation*}
%     \sum_{t=1}^TR(S_t) \geq (1-\eps)\cdot \sum_{t=1}^TR(S_t^*) > T\cdot \frac{B}{1+B}-\frac{2}{B(B+1)(B+2)}. 
% \end{equation*}
As explained above, we have $\sum_{t=1}^TR(S_t) > T\cdot \frac{B}{1+B}-\frac{2}{B(B+1)(B+2)}$ if and only if the $3$-\texttt{PARTITION} instance ${\cal A} = \{a_1,\ldots, a_{3T}\}$ is a YES-instance. Since $(S_1,\ldots, S_T)$ is obtained in pseudo-polynomial time, this yields a pseudo-polynomial time to solve the $3$-\texttt{PARTITION} problem, which contradicts its strong NP-hardness.



















\subsection{Proof of Lemma \ref{lem:epsobjective}}\label{apx:epsobjective}
    First, since \black{$\nota{v}_i \leq v_i$ for every product $i\in\Nc$}, $\nota{V}(\nota{S}_t)\leq V(S_t)$ for all $t\in [T]$. Consequently, by the monotonicity of $x\mapsto x/(1+x)$, we have $\nota{V}(\nota{S}_t)/(1+\nota{V}(\nota{S}_t))\leq V(S_t)/(1+V(S_t))$. By summing this inequality over all customers $t\in [T]$, we obtain $\widehat{\obj}(\nota{S}_1,\ldots, \nota{S}_T)\leq \obj(S_1,\ldots, S_T)$. Now, let us prove the second inequality, stating that $V(S_t)/(1+V(S_t)) \leq (1+\eps)\cdot \nota{V}(\nota{S}_t)/(1+\nota{V}(\nota{S}_t))$, for     every $t \in [T]$.\begin{itemize}
        \item If there exists a product $i\in S_t$ with $v_i\geq 1/\eps$, then $\nota{v}_i = 1/\eps$, and in turn, $\nota{V}(\nota{S}_t)\geq 1/\eps$. Therefore, \begin{equation*}
            \frac{V(S_t)}{1+V(S_t)} \leq 1 = (1+\eps) \cdot \frac{1/\eps}{1+1/\eps} \leq (1+\eps)\cdot\frac{\nota{V}(\nota{S}_t)}{1+\nota{V}(\nota{S}_t)}
        \end{equation*}
        \item If $v_i< 1/\eps$ for all $i\in S_t$: Note that Equation \eqref{eq:weightbound} trivially holds when $v_i<\eps$, since $\nota{v}_i = v_i$ by definition. Therefore, this equation holds for every $i\in S_t$. Hence, $v_i/(1+v_i) \leq (1+\eps) \cdot \nota{v}_i/(1+\nota{v}_i)$, and our desired inequality follows by summing over $t=1,\ldots,T$.
    \end{itemize}











\subsection{Proof of Lemma \ref{lem:LPopt}}\label{apx:LPopt}
    In this proof, we establish separate inequalities for light, medium, and heavy customers. At the very end, combining these inequalities will lead to the result stated in Lemma \ref{lem:LPopt}.
    
    \paragraph{Heavy customers.} When $t\in [T]$ is a heavy customer, by constraint (V) of LP, we have $\sum_{i\in \Nc} v_ix^*_{it} \geq \frac{1}{\eps}.$
        Therefore, \begin{equation}\label{eq:heavy}
            \frac{\sum_{i\in \Nc} v_ix^*_{it}}{1+\sum_{i\in \Nc} v_ix^*_{it}} \geq \frac{1}{1+\eps} \geq (1-\eps) \cdot \frac{V(S_t^*)}{1+V(S_t^*)}.
        \end{equation}
    
    \paragraph{Medium customers.} In this case, customer $t$ belongs to one of the classes $G_1,\ldots, G_L$, say $G_{\ell}$. As a result,
        \begin{equation*}
            \sum_{i\in \Nc} v_i x^*_{it} \geq V_t = \eps\cdot (1+\eps)^{\ell-1} \geq \frac{1}{1+\eps}\cdot V(S_t^*),
        \end{equation*}
        where the first inequality follows from constraint (V) of LP, and the second inequality holds since $V(S_t^*) \leq \eps \cdot (1+\eps)^{\ell}$ by definition of the class $G_{\ell}$.
        Therefore,\begin{align}
            \frac{\sum_{i\in \Nc} v_ix^*_{it}}{1+\sum_{i\in \Nc} v_ix^*_{it}} &\geq  \frac{1}{1+\eps} \cdot \frac{V(S_t^*)}{1+\frac{1}{1+\eps}\cdot V(S_t^*)}\label{eq:2050}\\
            & \geq \frac{1}{1+\eps} \cdot \frac{V(S_t^*)}{1+V(S_t^*)}\notag\\ 
            &\geq  (1-\eps)\cdot \frac{V(S_t^*)}{1+V(S_t^*)}\label{eq:medium}.
        \end{align}
    \paragraph{Light customers.} Let $\mathbf{\hat X}$ be the vector given by $\hat X_{it} = \mathbbm 1\{i \in S_t^*\}$, for every $t\in [T]$ and $i\in \Nc$. According to our construction, it is easy to verify that $\mathbf{\hat X}$ is a feasible solution to \ref{eq:ILP}, and therefore, also to its linear relaxation LP. Therefore, since $\mathbf{x^*}$ is an optimal solution to the latter program, we have \begin{equation}\label{eq:noinspi}
            \sum_{t\in G_{\rm light}} \sum_{i\in \Nc} v_ix^*_{it} \geq \sum_{t\in G_{\rm light}} \sum_{i\in \Nc} v_i\hat X_{it} =  \sum_{t\in G_{\rm light}} V(S_t^*).
        \end{equation}
        Also, by constraint (VI) of LP, we have $\sum_{i\in \Nc} v_i x^*_{it}\leq \eps$ for all $t\in G_{\rm light}$. Therefore, \begin{align}
            \sum_{t\in G_{\rm light}}\frac{\sum_{i\in \Nc} v_ix^*_{it}}{1+\sum_{i\in \Nc} v_ix^*_{it}}&\geq \frac{1}{1+\eps}\cdot \sum_{t\in G_{\rm light}}\sum_{i\in \Nc} v_ix^*_{it}\notag\\
            &\geq (1-\eps)\cdot\sum_{t\in G_{\rm light}} V(S_t^*)\notag\\
            & \geq (1-\eps)\cdot\sum_{t\in G_{\rm light}}\frac{V(S_t^*)}{1+V(S_t^*)},\label{eq:light}
        \end{align}
    where the second inequality follows from Equation \eqref{eq:noinspi}.
    % We conclude the proof by combining inequalities \eqref{eq:heavy}, \eqref{eq:medium}, and \eqref{eq:light}, as we have\begin{equation*}
    %     \sum_{t=1}^T\frac{\sum_{i\in \Nc} v_ix^*_{it}}{1+\sum_{i\in \Nc} v_ix^*_{it}}\geq (1-\eps)\cdot\sum_{t=1}^T\frac{V(S_t^*)}{1+V(S_t^*)}. 
    % \end{equation*}











\subsection{Proof of Lemma \ref{lem:nearoptimal}}\label{apx:nearoptimal}
    \noindent {\em {Case 1}: Customer $t$ is unbounded.} Let $q_t \in \{1,\ldots,Q\}$ be a class of products for which $k_{q_t,t} = \star$; such a class indeed exists by definition of unbounded customers. Then, \black{\begin{align}
        \sum_{i\in D_{q_t}} v_i x_{it}^* &\geq \eps^5\cdot \sum_{i\in D_{q_t}} x_{it}^* \label{eq:8} \\ &\geq \eps^5\cdot \frac{1}{\eps^6} \label{eq:9}\\ & = \frac{1}{\eps}.\label{eq:number1}
    \end{align}
Inequality~\eqref{eq:8} holds since $v_i \geq \eps^5$ for every product $i\in D_{q_t}$, as  $q_t\neq 0$.} \black{Equation~\eqref{eq:9}} follows from constraint (IV) of LP.
Therefore, 
    \begin{align}
        \P\left[V(S_t) \leq  \frac{1-\eps}{\eps}\right] &= 
        \P\left[\sum_{i\in \Nc}v_i X_{it} \leq  \frac{1-\eps}{\eps}\right] \notag\\& \leq \P\left[\sum_{i\in \Nc}v_i X_{it} \leq  (1-\eps)\cdot \sum_{i\in D_{q_t}} v_i x_{it}^*\right]\label{eq:ineq1}\\
        & \leq \P\left[\sum_{i\in D_{q_t}}v_i X_{it} \leq  (1-\eps)\cdot \sum_{i\in D_{q_t}} v_i x_{it}^*\right]\notag\\
        & = \P\left[\sum_{i\in D_{q_t}}X_{it} \leq  (1-\eps)\cdot \sum_{i\in D_{q_t}}x_{it}^*\right]\label{eq:number2}\\
        & \leq \exp\left(-\frac{\eps^2\cdot \sum_{i\in D_{q_t}}x_{it}^*}{2}\right)\label{eq:ineq2}\\
        & \leq \exp\left(-\frac{1}{2\eps^4}\right)\label{eq:ineq3}\\
        &\leq \eps.\label{eq:unboundbound}
    \end{align}
    Here, Inequality \eqref{eq:ineq1} follows from Equation \eqref{eq:number1}. Equality \eqref{eq:number2} holds since all products within the same class ($D_{q_t}$) have the same preference weight. Inequality \eqref{eq:ineq2} is a direct application of Lemma \ref{lem:chernoff}. In Inequality \eqref{eq:ineq3}, we plug in $\sum_{i\in D_{q_t}} x_{it}^*> 1/\eps^6$, since $k_{q_t, t} = \star$, as established by constraint (IV) of LP.
    Therefore, \begin{align}
        \E\left[\frac{V(S_t)}{1+V(S_t)}\right] &\geq \P\left[V(S_t)> \frac{1-\eps}{\eps}\right] \cdot  \E\left[\frac{V(S_t)}{1+V(S_t)} \,\left|\,V(S_t)> \frac{1-\eps}{\eps}\right.\right]\notag\\
        &\geq (1-\eps)\cdot \frac{\frac{1-\eps}{\eps}}{1+\frac{1-\eps}{\eps}} \label{eq:10} \\ &=(1-\eps)^2 \\ &\geq (1-2\eps)\cdot \frac{\sum_{i\in \Nc}v_ix_{it}^*}{1+\sum_{i\in \Nc}v_ix_{it}^*},\notag
    \end{align}
    where \black{Equation~\eqref{eq:10}} follows from Equation \eqref{eq:unboundbound} and the monotonicity of $x\mapsto x/(1+x)$ on $[0,\infty)$.


    \noindent{\em {Case 2}: Customer $t$ is bounded and heavy, or customer $t$ is medium.} To address this case, let us recall that the random preference weight of the assortment $S_t$ offered to customer $t$ is\begin{equation*}
    V(S_t)=\sum_{i\in \Nc}v_iX_{it} = \underbrace{\sum_{i\in D_0}v_iX_{it}}_{W_t^{\sma}} + \underbrace{\sum_{q = 1}^{Q-1}\sum_{i\in   D_q}v_iX_{it}}_{W_t^{\lar}}.
\end{equation*}
    \black{We introduce the random variables $W_t^{\sma}$ and $W_t^{\lar}$, which represent the contributions to $V(S_t)$ from products in $D_0$ and in $\Nc\setminus D_0$, respectively, defined as: $
    W_t^{\sma} = \sum_{i\in D_0}v_iX_{it}$ and $W_t^{\lar} = \sum_{q = 1}^{Q-1}\sum_{i\in   D_q}v_iX_{it}.
    $}
    Similarly, let $w_t^{\sma} = \sum_{i\in D_0}v_ix^*_{it}$, and $w_t^{\lar} = \sum_{i\in \Nc\setminus D_0}v_ix^*_{it}$, be the expected values of these  random variables. \black{In the next claim, we show that the random variable $W_t^{\lar}$ is almost surely equal to its expected value. Its proof, whose deferred to Appendix~\ref{apx:nonlightcust}, is a direct application of the degree preservation property}.
\begin{claim}\label{cl:nonlightcust}
    With probability $1$, we have $W_t^{\lar} = w_t^\lar.$
\end{claim}
\begin{itemize}
    \item When $w_t^{\sma}\leq \eps \cdot\sum_{i\in \Nc} v_ix_{it}^*$, then intuitively, the contribution of the products in $D_0$ represents at most an $\eps$-fraction of the total contribution of all products to $\sum_{i\in \Nc}v_ix_{it}^*$, and can therefore be neglected. Formally, \black{with} probability $1$, we have
    \black{\begin{align}
        V(S_t)&\geq W_t^{\lar}\label{eq:11} \\  &= w_t^{\lar} \notag\\
         &=  \sum_{i\in \Nc}v_i x_{it}^* - w_t^\sma \notag \\ \label{eq:12}&\geq (1-\eps)\cdot \sum_{i\in \Nc}v_ix_{it}^*,
    \end{align}}
    where \black{Equation~\eqref{eq:11}} is a consequence of Claim \ref{cl:nonlightcust}, and \black{Equation~\eqref{eq:12}} follows from the case hypothesis.
    Therefore, by the monotonicity of $x\mapsto x/(1+x)$, we have\begin{equation}
        \frac{V(S_t)}{1+V(S_t)} \geq (1-\eps)\cdot \frac{\sum_{i\in \Nc}v_i x_{it}^*}{1+\sum_{i\in \Nc}v_i x_{it}^*}.
    \end{equation}
    \item When $w_t^{\sma}> \eps \cdot\sum_{i\in \Nc} v_ix_{it}^*$, we have
    \begin{align}
        \P\left[W_t^\sma \leq (1-\eps)\cdot w_t^\sma\right] & = \P\left[\sum_{i\in D_0}\frac{v_i}{\eps^5}X_{it} \leq (1-\eps)\cdot \sum_{i\in D_0}\frac{v_i}{\eps^5}x_{it}^*\right]\notag\\
        &\leq \exp\left(-\frac{ \sum_{i\in D_0}v_ix_{it}^*}{2\eps^3}\right)\label{eq:ineq4}\\
        &\leq \exp\left(-\frac{\sum_{i\in \Nc}v_ix_{it}^*}{2\eps^2}\right)\label{eq:ineq5}\\
        &\leq \exp\left(-\frac{1}{2\eps}\right)\label{eq:ineq6}\\
        &\leq \eps.\label{eq:boundbound}
    \end{align}
    Here, Inequality \eqref{eq:ineq4} follows from Lemma \ref{lem:chernoff}, since $v_i< \eps^5$ for all $i\in D_0$. Inequality \eqref{eq:ineq5} is a consequence of the case hypothesis. Inequality \eqref{eq:ineq6} follows from the fact that customer $t$ is not light, and hence $\sum_{i\in \Nc}v_ix_{it}^* \geq \eps$.
    Therefore, \begin{align}
        \E\left[\frac{V(S_t)}{1+V(S_t)}\right]
        &\geq \P\left[W_t^{\sma}> (1-\eps)\cdot w_t^{\sma}\right] \cdot  \E\left[ \left.\frac{V(S_t)}{1+V(S_t)}\,\right|\,W_t^{\sma}> (1-\eps)\cdot w_t^{\sma}\right]\notag\\
        &= \P\left[W_t^{\sma}> (1-\eps)\cdot w_t^{\sma}\right] \cdot  \E\left[ \left.\frac{w_t^\lar+ W_t^\sma}{1+w_t^\lar+ W_t^\sma}\,\right|\,W_t^{\sma}> (1-\eps)\cdot w_t^{\sma}\right]\label{eq:13}\\
        &\geq (1-\eps) \cdot \frac{w_t^\lar+ (1-\eps)\cdot w_t^s}{1+w_t^\lar+(1-\eps)\cdot w_t^s}\label{eq:14}\\
        &\geq (1-\eps)^2.\frac{\sum_{i\in \Nc}v_ix^*_{it}}{1+\sum_{i\in \Nc}v_ix^*_{it}}\notag\\
        & \geq (1-2\eps).\frac{\sum_{i\in \Nc}v_ix^*_{it}}{1+\sum_{i\in \Nc}v_ix^*_{it}}.\notag
    \end{align}
    In \black{Equation~\eqref{eq:13}}, we apply Claim \ref{cl:nonlightcust}. In \black{Equation~\eqref{eq:14}}, we use Equation \eqref{eq:boundbound}, as well as the monotonicity of $x\mapsto x/(1+x)$.
\end{itemize}
    

\noindent {\em {Case 3}: Customer $t$ is light.} \black{The following claim is simply a restatement of Lemma~\ref{lem:nearoptimal} when customer $t$ is light. Due to the complex nature of its proof, we defer it to Appendix~\ref{apx:lightcust}}.
\begin{claim}\label{cl:lightcusts}
    $
        \E\left[\frac{V(S_t)}{1+V(S_t)}\right] \geq (1-2\eps)\cdot \frac{\sum_{i\in \Nc}v_ix_{it}^*}{1+\sum_{i\in \Nc}v_ix_{it}^*}.
    $
\end{claim}
% At a high level, when $t$ is a light customer then $\sum_{i\in \Nc} v_ix_{it}^* \leq \eps$, meaning that the denominator $1+V(S_t)$ is equal in expectation to $1+\sum_{i\in \Nc}v_ix_{it}^* \leq 1+\eps$. We make use of this connection to approximate $V(S_t)/(1+V(S_t))$ by $V(S_t)$, which is equal in expectation to $\sum_{i\in \Nc}v_i x^*_{it}$. Finally, this last term is trivially lower bounded by $\sum_{i\in \Nc}v_i x^*_{it}/(1+\sum_{i\in \Nc}v_i x^*_{it})$.


Combining cases 1, 2 and 3, it follows that\begin{equation*}
    \E\left[\sum_{t =1}^T\frac{V(S_t)}{1+V(S_t)}\right] \geq (1-2\eps)\cdot \sum_{t=1}^T \frac{\sum_{i\in \Nc}v_ix_{it}^*}{1+\sum_{i\in \Nc}v_ix_{it}^*}.\end{equation*}






















\subsection{Proof of Claim \ref{cl:nonlightcust}}\label{apx:nonlightcust}
For the purpose of showing that $W^{\lar}_t = w_t^{\lar}$, note that \begin{align*}
    W_t^\lar& = \sum_{q = 1}^Q\sum_{i\in D_q}v_iX_{it}\\
    & = \sum_{q = 1}^Q \eps^5\cdot(1+\eps)^{q-1}\sum_{i\in D_q}X_{it}\\
    & = \sum_{q = 1}^Q \eps^5\cdot(1+\eps)^{q-1}\sum_{i\in D_q}x^*_{it}\\
    & = \sum_{q = 1}^Q\sum_{i\in D_q}v_ix^*_{it}\\
    & = w_t^\lar,
\end{align*}
where the second equality follows from the definition of $D_q$ as the set of products with weight $\eps\cdot (1+\eps)^{q-1}$, and the third inequality follows from the degree preservation property \ref{it:P2}, and the fact that $\sum_{i\in D_q}x^*_{it}$ is an integer, as stipulated by constraint (III) of \ref{eq:ILP}. Note that $k_{q,t}\neq \star$ for all $q =1,\ldots, Q$, as otherwise, customer $t$ would be heavy and unbounded, which contradicts the case hypothesis.





















% \subsection{Proof of Claim \ref{cl:lightcusts}}\label{apx:lightcust}
% First, note that the proof of Claim \ref{cl:nonlightcust} holds even for the case where $t$ is a light customer. Therefore, we can decompose $V(S_t)$ into a sum of a random variable $W_t^\sma$ representing the contribution of products in $D_0$, and a deterministic random variable $W_t^{\lar} \overset{a.s}{=}w_t^\lar$ (which will be treated as a constant from this point on), i.e.,
% \begin{equation*}
%     V(S_t) = W_t^\sma + w_t^\lar.    
% \end{equation*}
% For simplicity of notation, and only in this proof, we use the notation $W\coloneqq W_t^\sma$ and $w \coloneqq w_t^\lar$.
% This proof consists on two steps. In the first step, we lower bound the expectation of $V(S_t)/(1+V(S_t))$ with $\sum_{i\in \Nc}v_ix^*_{it}$, up to a constant $\alpha<1$, mainly using a convexity argument. In other words, we provide a constant $\alpha$ such that $$
%     V(S_t)/(1+V(S_t)) \geq \alpha\cdot \sum_{i\in \Nc}v_ix^*_{it}.
% $$
% Then, in the second step, we show that $\alpha = (1-O(\eps))$.
% \paragraph{Step 1:}
% Let $\cal Y$ be the support of the random variable $W$, i.e.,$$
%     {\cal Y} = \left\{y\in\R\,\colon\, \P\left[W= y\right]>0\right\}.
% $$
% We have
% \begin{align*}
%     \E\left[\frac{V(S_t)}{1+V(S_t)}\right] & = \E\left[\frac{W+w}{1+W+w}\right]\\
%     & = \sum_{y \in {\cal Y}} \frac{y+w}{1+y+w}\P\left[W=y\right]\\
%     & = \left(\sum_{y\in {\cal Y}} \frac{1}{1+y+w}\cdot \frac{(y+w)\cdot \P\left[W=y\right]}{w+\E\left[W\right]}\right)\cdot \left(w+ \E\left[W\right]\right).
% \end{align*}
% For each $y\in {\cal Y}$, let $f(y) = \frac{1}{1+y+w}$ and let $t_y = \frac{(y+w)\cdot \P\left[W=y\right]}{w+\E\left[W\right]}$. It is easy to see that $\sum_{y\in {\cal Y}} t_y=1$, and that $f$ is a convex function on $(0, +\infty)$. Therefore, using Jensen's inequality, and recalling that $\E[W]+w = \E[V(S_t)]$ we have\begin{align*}
%     \E\left[\frac{V(S_t)}{1+V(S_t)}\right] &\geq f\left(\sum_{y\in {\cal Y}} t_y \cdot y\right)\cdot \E\left[V(S_t)\right]\\
%     & = \alpha\cdot \E\left[V(S_t)\right],
% \end{align*}
% where $\alpha = f\left(\sum_{y\in {\cal Y}} t_y \cdot y\right)$.
% \paragraph{Step 2.} Let us now show that $\alpha = 1-O(\eps)$. We have
% \begin{align}
%         \alpha& = \frac{1}{1+w + \frac{\sum_{y\in {\cal Y}}(y+w)\cdot \P[W=y]\cdot y}{w+\E[W]}} \notag\\
%     & = \frac{1}{1+w+\frac{w\E[W] + \E[W^2]}{w+\E[W]}}\notag\\
%     & = \frac{1}{1+w+\E[W] + \frac{Var[W]}{w+\E[W]}}.\label{eq:last}
% \end{align}
% In order to show the desired property, we need to bound the denomination in Equation \eqref{eq:last}. To this purpose we have\begin{align*}
%     Var[W] & = Var\left[\sum_{i\in D_0} v_iX_{it}\right]\\
%         & \leq \sum_{i\in D_0} v_i^2\cdot Var[X_{it}]\\
%         &\leq \eps^5\cdot \sum_{i\in D_0} v_i\cdot \E[X_{it}]\\
%         &= \eps^5\cdot \E[W].
% \end{align*}
% Therefore, by replacing in Equation \eqref{eq:last}, we have\begin{align*}
%     \alpha \geq \frac{1}{1+\eps+\eps^5} \geq \frac{1}{1+2\eps}\geq 1-2\eps.
% \end{align*}

% Finally, combining steps 1 and 2, we have \begin{align*}
%     \frac{V(S_t)}{1+V(S_t)} &\geq \alpha\cdot \E[V(S_t)]\\
%                     & \geq (1-2\eps)\cdot \sum_{i\in \Nc} v_ix^*_{it}\\
%                     &\geq (1-2\eps)\cdot \frac{\sum_{i\in \Nc} v_ix^*_{it}}{1+\sum_{i\in \Nc} v_ix^*_{it}}.     
%         \end{align*}

























% \subsection{Proof of Claim \ref{cl:lightcusts}}\label{apx:lightcust}
% First, it is not difficult to verify that the proof of Claim \ref{cl:nonlightcust} also holds for light customers. Therefore, we can decompose $V(S_t)$ into a sum of a random variable $W_t^\sma$ representing the contribution of products in $D_0$, and a deterministic random variable $W_t^{\lar} \overset{a.s}{=}w_t^\lar$ (which will be treated as a constant from this point on), i.e.,
% \begin{equation*}
%     V(S_t) = W_t^\sma + w_t^\lar.    
% \end{equation*}
% For simplicity of notation, and only in this proof, we use the notation $A\coloneqq W_t^\sma$ and $b \coloneqq w_t^\lar$.
% This proof consists on two steps. In the first step, we lower bound the expectation of $V(S_t)/(1+V(S_t))$ with $\sum_{i\in \Nc}v_ix^*_{it}$, up to a constant $\alpha<1$, mainly using a convexity argument. In other words, we provide a constant $\alpha$ such that $$
%     V(S_t)/(1+V(S_t)) \geq \alpha\cdot \sum_{i\in \Nc}v_ix^*_{it}.
% $$
% Then, in the second step, we show that $\alpha = (1-O(\eps))$.
% \paragraph{Step 1:}
% Let $\cal A$ be the support of the random variable $A$, i.e.,$$
%     {\cal A} = \left\{a\in\R\,\colon\, \P\left[A= a\right]>0\right\}.
% $$
% We have
% \begin{align*}
%     \E\left[\frac{V(S_t)}{1+V(S_t)}\right] & = \E\left[\frac{A+b}{1+A+b}\right]\\
%     & = \sum_{a \in {\cal A}} \frac{a+b}{1+a+b}\P\left[A=a\right]\\
%     & = \left(\sum_{a\in {\cal A}} \frac{1}{1+a+b}\cdot \frac{(a+b)\cdot \P\left[A=a\right]}{b+\E\left[A\right]}\right)\cdot \left(b+ \E\left[A\right]\right).
% \end{align*}
% For each $a\in {\cal A}$, let $f(a) = \frac{1}{1+a+b}$ and let $z_a = \frac{(a+b)\cdot \P\left[A=a\right]}{b+\E\left[A\right]}$. It is easy to see that $\sum_{a\in {\cal A}} z_a=1$, and that $f$ is a convex function on $(0, +\infty)$. Therefore, using Jensen's inequality, and recalling that $\E[A]+b = \E[V(S_t)]$ we have\begin{align*}
%     \E\left[\frac{V(S_t)}{1+V(S_t)}\right] &\geq f\left(\sum_{a\in {\cal A}} z_a \cdot y\right)\cdot \E\left[V(S_t)\right] = \alpha\cdot \E\left[V(S_t)\right],
% \end{align*}
% where $\alpha = f\left(\sum_{a\in {\cal A}} z_a \cdot a\right)$.
% \paragraph{Step 2.} Let us now show that $\alpha = 1-O(\eps)$. We have
% \begin{align}
%         \alpha& = \frac{1}{1+b + \frac{\sum_{a\in {\cal A}}(a+b)\cdot \P[A=a]\cdot a}{b+\E[A]}} \notag\\
%     & = \frac{1}{1+b+\frac{b\E[A] + \E[A^2]}{b+\E[A]}}\notag\\
%     & = \frac{1}{1+b+\E[A] + \frac{Var[A]}{b+\E[A]}}.\label{eq:last}
% \end{align}
% In order to show the desired property, we need to bound the denominator in Equation \eqref{eq:last}. To this purpose we have\begin{align*}
%     Var[A] & = Var\left[\sum_{i\in D_0} v_iX_{it}\right]\\
%         & \leq \sum_{i\in D_0} v_i^2\cdot Var[X_{it}]\\
%         &\leq \eps^5\cdot \sum_{i\in D_0} v_i\cdot \E[X_{it}]\\
%         &= \eps^5\cdot \E[A].
% \end{align*}
% Therefore, by replacing in Equation \eqref{eq:last}, we have\begin{align*}
%     \alpha \geq \frac{1}{1+\eps+\eps^5} \geq \frac{1}{1+2\eps}\geq 1-2\eps.
% \end{align*}
% Finally, combining steps 1 and 2, we have \begin{align*}
%     \frac{V(S_t)}{1+V(S_t)} &\geq \alpha\cdot \E[V(S_t)]  \geq (1-2\eps)\cdot \sum_{i\in \Nc} v_ix^*_{it} 
%                     \geq (1-2\eps)\cdot \frac{\sum_{i\in \Nc} v_ix^*_{it}}{1+\sum_{i\in \Nc} v_ix^*_{it}}.     
%         \end{align*}







\subsection{Proof of Claim \ref{cl:lightcusts}}\label{apx:lightcust}
First, it is not difficult to verify that the proof of Claim \ref{cl:nonlightcust} also holds for light customers. Therefore, for every customer $t\in G_{\rm light}$, we can decompose $V(S_t)$ into the sum of : (1) The random variable $W_t^\sma$, and (2) The deterministic random variable $W_t^{\lar} \overset{a.s}{=}w_t^\lar$, i.e.,
$
    V(S_t) = W_t^\sma + w_t^\lar.    
$
Our proof consists on two steps. In the first step, we lower bound the expectation of $V(S_t)/(1+V(S_t))$, mainly using a convexity argument. Specifically, we prove that there exists a constant $\alpha\in (0,1]$ such that $
    \E\left[\frac{V(S_t)}{1+V(S_t)}\right] \geq \alpha\cdot \sum_{i\in \Nc}v_ix^*_{it}.
$
In the second step, we show that $\alpha = 1-O(\eps)$.
\paragraph{Step 1:}
Let $\cal W$ be the support of the random variable $W_t^\sma $, i.e., $
    {{\cal W} = \left\{w\in\R\,\colon\, \P\left[W_t^\sma= w\right]>0\right\}}.
$
Note that the support ${\cal W}$ is finite since the random variable $W_t^\sma$ is a deterministic function of the random binary vector $(X_{it}\,\colon\,i\in D_0)$, whose support is trivially finite.
We have
\begin{align*}
    \E\left[\frac{V(S_t)}{1+V(S_t)}\right] & = \E\left[\frac{W_t^\sma + w_t^\lar}{1+W_t^\sma + w_t^\lar}\right]\\
    & = \sum_{w \in {\cal W}} \frac{w+w_t^\lar}{1+w+ w_t^\lar}\P\left[W_t^\sma=w\right]\\
    & = \left(\sum_{w\in {\cal W}} \frac{1}{1+w + w_t^\lar}\cdot \frac{(w+ w_t^\lar)\cdot \P\left[W_t^\sma=w\right]}{w_t^\lar+\E\left[W_t^\sma\right]}\right)\cdot \left(w_t^\lar +\E\left[W_t^\sma\right]\right).
\end{align*}
For each $w\in {\cal W}$, let $f(w) = \frac{1}{1+w + w_t^\lar}$ and let $z_w = \frac{(w+ w_t^\lar)\cdot \P\left[W_t^\sma=w\right]}{ w_t^\lar+\E\left[W_t^\sma\right]}$. It is easy to verify that $\sum_{w\in {\cal W}} z_w=1$, and that $f$ is a convex function on $(0, +\infty)$. Therefore, using Jensen's inequality, we have\begin{align*}
    \E\left[\frac{V(S_t)}{1+V(S_t)}\right] &\geq f\left(\sum_{w\in {\cal W}} z_w \cdot w\right)\cdot \left(\E[W_t^\sma]+ w_t^\lar\right)= \underbrace{f\left(\sum_{w\in {\cal W}} z_w \cdot w\right)}_{\coloneqq \alpha}\cdot \E\left[V(S_t)\right],
\end{align*}
where the last equality holds since $\E[W_t^\sma]+ w_t^\lar = \E[V(S_t)]$.
\paragraph{Step 2.} Let us now show that $\alpha = 1-O(\eps)$. To this end, note that
% \begin{align}
%     \alpha& = \frac{1}{1+w_t^\lar + \frac{\sum_{w\in {\cal W}}(w+ w_t^\lar)\cdot \P[W_t^\sma=w]\cdot w}{w_t^\lar+\E[W_t^\sma]}} \notag\\
%     & = \frac{1}{1+w_t^\lar+\frac{w_t^\lar\cdot\E[W_t^\sma] + \E[(W_t^\sma)^2]}{w_t^\lar+\E[W_t^\sma]}}\notag\\
%     & = \frac{1}{1+ w_t^\lar+\E[W_t^\sma] + \frac{\Var[W_t^\sma]}{w_t^\lar+\E[W_t^\sma]}}.\label{eq:last}
% \end{align}
\begin{align}
    \frac1\alpha& = {1+w_t^\lar + \frac{\sum_{w\in {\cal W}}(w+ w_t^\lar)\cdot \P[W_t^\sma=w]\cdot w}{w_t^\lar+\E[W_t^\sma]}} \notag\\
    & = {1+w_t^\lar+\frac{w_t^\lar\cdot\E[W_t^\sma] + \E[(W_t^\sma)^2]}{w_t^\lar+\E[W_t^\sma]}}\notag\\
    & = {1+ w_t^\lar+\E[W_t^\sma] + \frac{\Var[W_t^\sma]}{w_t^\lar+\E[W_t^\sma]}}.\label{eq:last}
\end{align}
In order to show that $\alpha = 1-O(\eps)$, we proceed by bounding the denominator in Equation \eqref{eq:last}. For this purpose, we have\begin{align}
    \Var[W_t^\sma] & = \Var\left[\sum_{i\in D_0} v_iX_{it}\right]\notag\\
    & = \sum_{i\in D_0}v_i^2\cdot \Var\left[X_{it}\right]+ \sum_{\substack{i,j\in D_0\\ i\neq j}} v_i\cdot v_j\cdot \Cov[X_{it},X_{jt}] \label{eq:420}\\
        & \leq \sum_{i\in D_0} v_i^2\cdot \Var[X_{it}]\notag\\
        &\leq \eps^5\cdot \sum_{i\in D_0} v_i\cdot \E[X_{it}] \label{eq:421}\\
        &= \eps^5\cdot \E[W_t^\sma]\notag.
\end{align}
In Equation \eqref{eq:421}, we use the fact that for any product $i\in D_0$, $\Var[X_{it}]\leq \E[X_{it}]$ since $X_{it}$ is a Bernoulli random variable, and the fact that $v_i \leq \eps^5$.
Equation \eqref{eq:420} holds since $\Cov[X_{it}, X_{jt}]\leq 0$ for all $i,j\in D_0$, $i\neq j$. Indeed, \begin{align*}
    \Cov[X_{it}, X_{jt}] &=\E[X_{it}\cdot X_{jt}] - \E[X_{it}]\cdot \E[X_{jt}]\\
    & = \P[(X_{it} = 1)\wedge (X_{it} = 1)]-\P[X_{it} = 1]\cdot \P[X_{it} = 1]\\
    &\leq 0,
\end{align*}
where the inequality holds due to the negative correlation property \ref{it:P3}.
Therefore, by replacing in Equation \eqref{eq:last}, we have\begin{align*}
    \alpha &\geq \frac{1}{1+w_t^\lar+\E[W_t^\sma]+\frac{\eps^5\cdot \E[W_t^\sma]}{w_t^\lar+\E[W_t^\sma]}} \\
    & \geq \frac{1}{1+\eps+\eps^5} \\
    &\geq \frac{1}{1+2\eps}\\
    &\geq 1-2\eps
\end{align*}
In the second inequality, we use the fact that $w_t^\lar+\E[W_t^\sma] = \E[V(S_t)] = \sum_{i\in \Nc}v_ix_{it}^* \leq \eps$, by constraint (VI) of LP, as customer $t$ is light by the case hypothesis.


































\section{Proof of Claim~\ref{cl:increasefee}} \label{apx:increasefee}
Recall that an optimal solution to \ref{APV} is given by $S_t^* = \overline{L_t\cup \cdots \cup L_T}$ for every customer $t\in [T]$, as shown in Theorem \ref{Solution structure}. \black{Using Equations \eqref{eq:theells} and \eqref{eq:sol} to characterize the optimal solution of this updated instance, we define $\Tilde L_t = \{i\in\Nc:\Tilde \ell_i = t\}$, for all $t\in \{0\}\cup[T]$ and $\Tilde S_t = \llbracket\bigcup_{t\leq u\leq T}\Tilde L_u\rrbracket$ for all $t \in [T]$. In particular, 
    $$
    \Tilde L_t := \left\{\begin{array}{ll}
         L_t &  \quad\text{if }t\notin \{\ell_i, \ell_i+1\} \\
         L_t\setminus \{i\} & \quad\text{if } t=\ell_i\\
         L_t\cup\{i\} & \quad\text{if } t=\ell_i+1
    \end{array}\right.
    $$
    Therefore, we have $\Tilde S_t = S_t^*$ for all $t\neq \ell_i+1$, and $\Tilde S_{\ell_i+1} = \overline{L_t\cup\cdots\cup L_{\ell_i+1}\cup\{i\}}$.
}
%Similarly, let $(\Tilde S_1, \ldots, \Tilde S_T)$  be the optimal solution given by Equations \eqref{eq:sol} when $\ell_i$ is increased by $1$. Note that as a direct consequence of the formula giving this optimal solution, we have $\Tilde S_t = S_t^*$ for all $t\neq \ell_i+1$, and $\Tilde S_{\ell_i+1} = \overline{L_t\cup\cdots\cup L_{\ell_i+1}\cup\{i\}}$.
Let $\Tilde \Delta$ be the total revenue loss incurred in the modified instance, that is $\Tilde \Delta = T\cdot R(S^*) - \sum_{t=1}^T R(\Tilde S_t)$. Additionally, for any product $j\in \Nc$, let $\Tilde C_{j}$ be its contribution after increasing $\ell_i$, i.e., $\Tilde C_{j} = \sum_{t=1}^T \mathbbm{1}(j \in \Tilde S_t) (p_j - R(\Tilde S_t)) \cdot v_j$. Finally, let $\Tilde \Gamma_j = (\Tilde C_{j}^- / \sum_{m\neq j}\Tilde C_{m}^-)\cdot  \Tilde \Delta$, namely the fee imposed on vendor $j$ in the modified instance. Using these notations, let us show that $\Tilde \Gamma_i\geq \Gamma_i$.
 First, if $i\in S_{\ell_i+1}^*$, then $S_{\ell_i+1}^* = \Tilde S_{\ell_i+1}$ and we have $\Tilde C_{i} = \cont{i}$. Consequently, $\Tilde \Gamma_i  = \Gamma_i$. Otherwise, $\cont{i}\geq\Tilde C_{i}$, and therefore $C_{i}^-\leq\Tilde C_{i}^-$. Furthermore, since $R(S^*_{\ell_i+1}) \geq R(\Tilde S_{\ell_i+1})$, we have $\Tilde C_{j} \geq C_{j}$ for every $j\neq i$. Finally, we have\begin{align}
    \Tilde \Gamma_{i} &= \frac{\Tilde C_i^-}{\Tilde C_{i}^- + \sum_{j\neq i}\Tilde C_{j}^-} \cdot \Tilde \Delta\notag\\
    & \geq \frac{\Tilde C_i^-}{\Tilde C_{i}^- + \sum_{j\neq i}C_{j}^-}\cdot  \Delta\notag\\
    &\geq \frac{C_i^-}{{C_{i}^-} + \sum_{j\neq i}{C_{j}^-}}\cdot  \Delta\label{eq:2049}\\
    & = \Gamma_i,\notag
\end{align}
where Inequality \eqref{eq:2049} follows from the fact that $x\mapsto x/{(c+x)}$ is nondecreasing on $[0,+\infty)$, for any $c>0$.



%\input{TEX-Remarks}


\end{document}