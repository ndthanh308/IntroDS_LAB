% This is samplepaper.tex, a sample chapter demonstrating the
% LLNCS macro package for Springer Computer Science proceedings;
% Version 2.20 of 2017/10/04
%
\documentclass[runningheads]{llncs}
%
\usepackage{graphicx}
\usepackage[english]{babel}
\usepackage[utf8]{inputenc}
\usepackage{amsfonts}
\usepackage{amssymb}
\usepackage{amsmath}
\usepackage{stmaryrd}
\usepackage{comment}
\usepackage[margin=1in]{geometry}
\usepackage{setspace}
\usepackage{bbm}
\usepackage{mathtools}
\usepackage{enumitem}
\usepackage{authblk}
%\usepackage{amsthm}
\usepackage{algorithm, algorithmicx, algpseudocode}
\usepackage{tikz}
\usepackage{pgfplots}
%\usepackage{natbib}
\usepackage{thmtools}
\usepackage{thm-restate}
\usepackage{booktabs}
\usepackage{array}
\usepackage{bbold}

\usepackage{hyperref}
\usepackage{cleveref}

\usepackage[maxnames=999]{biblatex} %Imports biblatex package
\addbibresource{biblio.bib} %Import the bibliography file


%\usepackage[backend=biber, style=authoryear, maxnames=999]{biblatex}

\newcommand{\qedsymbol}{\hfill$\square$}
\pgfplotsset{compat=1.18}

\usepackage[fontsize=11pt]{fontsize}
\linespread{1.3}


\begin{document}
%
\title{Assortment Optimization with Visibility Constraints}%
\author{Théo Barré \inst{1} \and Omar El Housni \inst{2} \and Andrea Lodi \inst{2} 
}
\authorrunning{Théo Barré, Omar El Housni, Andrea Lodi} 

\institute{\'Ecole Polytechnique, Institut Polytechnique de Paris, Palaiseau, France \\
    \texttt{theo.barre@polytechnique.edu} \\
    \and
    School of Operations Research and Information Engineering, Cornell Tech, New York, NY \\ 
    \texttt{\{oe46,al748\}@cornell.edu} \\}
%
\maketitle              % typeset the header of the contribution
%
\begin{abstract}

Motivated by  applications in e-retail and online advertising, we study  the problem of assortment optimization under visibility constraints, that we refer to as \ref{APV}. We are given a universe of substitutable products and a stream of $T$ customers. The objective is to determine the optimal assortment of products to offer to each customer in order to maximize the total expected revenue, subject to the constraint that each product is required to be shown to a minimum number of customers. The minimum display requirement for each product is given exogenously and we refer to these constraints as {\em visibility constraints}.   We assume that customer choices follow a Multinomial Logit model (MNL). We provide a characterization of the structure of the optimal assortments  and present an efficient polynomial time algorithm for solving \ref{APV}. To accomplish this, we introduce a novel function called the ``expanded revenue" of an assortment and establish its supermodularity. Our algorithm takes advantage of this structural property. Additionally, we demonstrate that \ref{APV}  can be formulated as a compact linear program. We also examine the revenue loss resulting from the enforcement of visibility constraints, comparing it to the unconstrained version of the problem. To offset this loss, we propose a novel strategy to distribute the loss among the products subject to visibility constraints. Each vendor is charged an amount proportional to their product's contribution to the revenue loss. Finally, we present the results of our numerical experiments providing illustration of the obtained outcomes, and we discuss some preliminary results on the extension of the problem to accommodate cardinality constraints.

\keywords{Assortment Optimization  \and Multinomial Logit model \and Visibility Constraints \and  Supermodularity \and Algorithm Design.}
\end{abstract}
%
%
%
\section{Introduction}

%Assortment Optimization is a substantial field of Operations Research within Revenue Management. It consists in, given a set of available products, selecting a subset of them to offer to a customer so as to optimize a particular objective function. When being offered this assortment, the customer then decides to either purchase a product from the assortment, or leave without purchasing anything. Such a scenario is particularly used to model two business situations: retail, and online advertising. 

%In retail, the objective function to maximize is the revenue of the platform that offers the assortments. We can think for instance about Amazon trying to select which products to display when a new customer arrives and searches an item for which there exists different versions. The choice of the assortment is very important because of the substitution effect: the sales of one product do not only depend on its intrinsic value, but also on all the other alternatives to which it is compared by the customer examining the offered assortment. As an example, if we offer a high quality, high price product along with another product with similar quality but a much lower price, the first one will never be sold and the revenue of the platform will not be satisfactory. This motivates the importance of choosing wisely the assortment and makes it an optimization problem.\\
%In online advertising, the objective is to maximize the number of clicks advertisements generate, independently of which advertisement is being clicked on by the customers. This situation can be seen as a special case of the retail one, in which all products generate the same value for the company, that is the value of a click received. \\



Assortment optimization is a crucial aspect of  decision making in many industries  such as  retail and online advertising. The goal of assortment optimization is to select a subset of available products  to offer to customers in order to maximize a specific objective function. In this context, the objective function might vary depending on the business's priorities, such as maximizing revenue, profit, or market share. For example, online retails such as Amazon and Ebay seek to strategically select which products to display when a new customer searches for a product, in order to maximize the expected revenue.  The assortment choice is crucial due to the substitution effect, where a product's sales depend not only on its intrinsic value but also on the alternatives presented to the customer. For example, offering a high-quality, high-priced product alongside a comparable product at a significantly lower price may result in poor sales for the higher-priced product, leading to an unsatisfactory platform revenue. This highlights the importance of carefully selecting assortments. 
Assortment optimization also plays a vital role in the realm of online advertising, where the objective is to strategically select the most effective combination of advertisements to maximize user engagement and desired outcomes, such as click-through rates.  Through assortment optimization, advertisers can enhance the effectiveness of their online advertising campaigns, ultimately driving better results and maximizing the return on their advertising investments.


%Assortment optimization commonly relies on the utilization of random utility choice models such as Multinomial Logit model,  Nested Logit model,  Mixed logit model, etc.  Choice models are statistical frameworks that capture the decision-making process of individuals when presented with a set of alternatives. These models aim to quantify the probability of customers selecting a particular option based on various attributes, features, and contextual factors. 




Traditionally, the assortment optimization framework often misses a key component of the contemporary e-commerce business - the visibility of products. As businesses operate within the boundaries of Service-Level Agreements (SLAs) with their suppliers and are constantly seeking ways to promote sponsored products, the level of product visibility in an assortment takes on a profound significance. In fact, in the contemporary commercial landscape, businesses are embedded within a complex network of agreements and relationships. A particularly significant aspect of this network is the SLAs between businesses and their suppliers. These agreements often stipulate certain conditions about product representation, ensuring that each supplier's product gets a fair share of visibility on the platform.  Furthermore, the concept of sponsored products has gained immense traction in recent years. Brands are willing to pay a premium to ensure their products are prominently displayed and receive a higher level of visibility. It is an effective strategy to increase brand awareness, influence consumer behavior, and boost sales. However, simply maximizing the visibility of sponsored products without considering the broader assortment optimization strategy could lead to an imbalance in the product mix, resulting in lower customer satisfaction and decreased overall revenue.


In this paper, we introduce the notion of {\em visibility constraints} in the context of Assortment Optimization. The purpose is to enforce a minimum display of each product, among the assortments offered to a sequence of customers. In other words, among all the relevant products that can be selected when customers make a query, each of them has to be shown at least a certain number of times in the  displayed assortments. This constraint is notably modeling both Service-Level Agreements and sponsored products. It can also capture the settings where the platform would like to ensure some fairness notion among vendors by ensuring that each product is given a ``fair" chance, i.e., it is shown at least to a certain number of customers. 





More formally, we are given a universe of substitutable products and a stream of $T$ customers. When a customer arrives on the platform, we have to offer them an assortment from the universe of products. The customer decides to purchase one of these products, or to leave without purchasing any product  (no-purchase option). We assume that the choice of the customer is governed by a Multinomial Logit choice model. We enforce the constraint that each  product in the universe has to be shown a minimum number of times among the $T$ assortments offered. The minimum display requirement for each product is given exogenously.   Our objective is to maximize the total expected revenue from  the $T$ customers. We refer to this problem as {\em Assortment optimization Problem with Visibility} constraints, briefly denoted as \ref{APV}.

A first natural question concerns the complexity of the problem \ref{APV}. In fact, without the visibility constraints, the problem reduces to the classical unconstrained assortment optimization under Multinomial Logit, for which we know that the optimal assortment is revenue-ordered (\citeauthor{talluri2004revenue}   \cite{talluri2004revenue}) and therefore can be solved in polynomial time. However, by enforcing the visibility constraints, we might have to include  certain products in the assortment that cannibalize the sales of other more profitable products because of the substitution effect.
Consequently, determining the optimal assortments for \ref{APV} is not clear at first glance.

% placement within the sequence to mitigate detrimental effects becomes a significant concern.




%In this paper, we introduce the notion of {\em visibility} constraints in the context of Assortment Optimization scheduling. Their purpose is to enforce a minimum display of each product, among a sequence of assortments offered to customers during a specific time horizon. In other words, among all the relevant products that can be selected when a customer makes a query, each of them has to be selected at least a certain percentage of times in the assortment displayed. This constraint is notably motivated by Service-Level Agreements, contracts between the platform and the different vendors of the products it can display and sell. In these contracts, the two parts can agree that the product in question will be offered at least a certain number of times, or a certain percentage of times. In addition, one can think about sponsored products, to which the platform wants to give particular visibility, by including them with a certain regularity for instance.\\






% More precisely, we consider the situation in which at each one of $T$ time steps, a customer arrives on the platform and we have to offer them an assortment included in the same fixed $n$ products. The customer then decides to purchase one of these products, or to leave with no purchase, according to the {\em multinomial logit choice model}. Our objective is to optimize the total expected revenue over the $T$ time steps when we enforce the constraint that each of the available products has to be shown a minimum number of times among the $T$ assortments offered. We refer to this problem as {\em Assortment Problem with Visibility (\ref{APV})}.\\


%A first natural question concerns the complexity of the problem \ref{APV}. Indeed, because of these visibility constraints, we might have to include in the assortments certain products that cut the sales of other more profitable products because of the substitution effect, and it becomes interesting to decide where to put them in the sequence so as to minimize the negative impacts. \\




A subsequent issue emerging from the visibility problem is the task of quantifying and distributing the revenue loss incurred due to the enforcement of visibility constraints. This challenge compels us to develop a pricing strategy that appropriately apportions the loss to different vendors based on the impact of their product on the overall revenue.
Such a scenario frequently occurs within the framework of Service-Level Agreements. Typically, a contract between a platform and a vendor includes a clause that guarantees a certain level of visibility to the vendor's product. In return for this visibility, the vendor compensates the platform with a fee. This fee acts as a buffer against any potential revenue loss that the platform might suffer as a consequence of providing visibility to the vendor's product. Thus, the aim is to determine a fair and proportionate fee that correlates with the revenue loss associated with each vendor's product exposure.
    

     \subsection{Our Contributions}
    
In this paper, we introduce and study the assortment  optimization problem with visibility constraints under the Multinomial Logit choice model. Our first goal is to settle the complexity question on the positive side by developing a polynomial time algorithm for \ref{APV}. A subsequent goal is to quantify the revenue loss caused by the visibility constraints compared to the unconstrained assortment optimization problem and design a  strategy to  share this loss among the different vendors of the products for which visibility constraints have been enforced. Our contributions are organized and summarized as follows.


\begin{enumerate}
    \item {\bf Polynomial time algorithm for \ref{APV}}.   
    Our main technical contribution is to design a polynomial time algorithm for \ref{APV}.  We introduce the notion of expanded revenue and expanded set of an assortment, which will play a pivotal role in designing our algorithm. We leverage structural properties of the expanded revenue function to characterize the structure of an optimal solution of \ref{APV}, and consequently design an efficient algorithm to compute it.
    
    %We start by considering the problem of static assortment optimization over $T$ customers when we enforce visibility constraints on the products available. Contrary to the unconstrained case, the introduction of the visibility constraints introduces a coupling between the different customers, because we cannot respect the global visibility constraints if we optimize each assortment independently of the others. Hence, it does not appear obvious whether this problem can be solved in polynomial time or is NP-hard. We prove that \ref{APV} can actually be solved by a polynomial time algorithm, and devise such an algorithm. Our contribution follows this structure:
    \begin{enumerate}
        \item {\bf Expanded Revenue}. In Section \ref{expanded revenue}, we introduce the expanded revenue function. Given a universe of products $\cal N$ and an assortment $A \subseteq {\cal N}$. The expanded revenue of assortment $A$ is defined as the maximum revenue of any assortment in $\cal N$ that contains $A$. The expanded set is the assortment that achieves this maximum revenue. We show that the expanded revenue function is closely related to the objective function of \ref{APV} in the case of a single customer. We provide a linear time algorithm to compute the expanded revenue.         
        %: given a subset $A$ of the $n$ products available, compute the maximal expected revenue of a subset containing $A$. We prove in Lemma \ref{Compute expanded set} that there exists a subset of maximal cardinality that is optimal for this problem, and name it the expanded set of $A$, while the expanded revenue of $A$ is the corresponding expected revenue. In addition, we prove that the expanded set can be computed in linear time with respect to the number $n$ of products.

        
        \item {\bf Monotonicity and supermodularity\footnote{A function $f: {\Omega} \rightarrow \mathbb{R}$ is supermodular $\iff \forall A, B \in {\Omega}, \; f(A \cup B) + f(A \cap B) \geq f(A) + f(B)$.} of the expanded revenue}. We show that the expanded revenue, as defined above, possesses some useful properties. Namely, we prove in Lemma \ref{monotonicity} a monotonicity property, i.e., we show that the expanded revenue of an assortment decreases as the assortment gets large. Then, in Lemma \ref{supermodularity}, we prove the main theoretical result on which our final algorithm relies: the supermodularity of the expanded revenue function.
        \item {\bf Our algorithm and LP formulation}. Building on the previous properties of the expanded revenue,  we finally identify in Theorem \ref{Solution structure} a very simple nested structure for an optimal solution of \ref{APV}, and devise a polynomial time algorithm to efficiently solve the problem. Additionally, we demonstrate in Theorem \ref{Visibility problem LP} that \ref{APV} can be formulated as a compact  linear program.
    \end{enumerate}

\item {\bf Price of visibility.} The introduction of visibility constraints results in a reduction in the total expected revenue compared to the unconstrained version of the assortment problem. Therefore, we aim to evaluate this revenue loss and propose a fair strategy for distributing it among vendors based on their respective contributions to the loss.
\begin{enumerate}
\item {\bf Pricing the loss.} We devise in Section \ref{subsection:share},  a pricing strategy as follows: For each product that negatively impacts the overall revenue, we charge the vendor a fraction of the loss proportional to the ratio between the negative contribution of the product and the sum of the negative contributions of all products. We demonstrate that this strategy satisfies natural fairness properties and exhibits favorable computational tractability.
\item {\bf Numerical experiments.} Additionally, we conduct in Section \ref{subsection:numeric} some numerical experiments to illustrate our findings. We analyze the influence of visibility constraints on expected revenue and sales, examine the individual effect of a single product's visibility constraint, and explore the trade-off between revenue and fees. Moreover, we show how our pricing strategy accurately captures the revenue loss attributed to the products made visible.
\end{enumerate}

\item {\bf Cardinality Constraint Extension.} We consider in Section \ref{sec:extensions} the natural extension of \ref{APV} where there is an upper bound on the number of products that we can display in each assortment. The resulting problem becomes significantly harder and, although our investigation is still preliminary, we can already prove the two following strong results. In Theorem \ref{NP-hardness}, we prove that \ref{APV}  with cardinality constraints is strongly NP-hard, even in the case of equal prices, by linking its resolution to the $k$-partitioning decision problem. In the case where all prices are equal, we devise a $0.61$-approximation algorithm for the problem. The case of equal prices is relevant for example when the objective is to maximize the sales in e-retail or maximize the engagement in online advertisement. The proofs of these extensions are deferred to the appendix.
\end{enumerate}
        



    
    
    \subsection{Related Literature}


Assortment optimization under the Multinomial Logit (MNL) model is a well-established problem in scientific literature. Initially introduced by \citeauthor{luce1959individual} \cite{luce1959individual}, with subsequent works by \citeauthor{McFadden1972ConditionalLA}  \cite{McFadden1972ConditionalLA} and \citeauthor{RePEc:ecm:emetrp:v:52:y:1984:i:5:p:1219-40} \cite{RePEc:ecm:emetrp:v:52:y:1984:i:5:p:1219-40}, the MNL choice model has gained popularity for modeling customer choices due to its simplicity in computing the choice probabilities, its predictive power and its computational tractability compared to more complex choice models. It has been extensively used in various research works  such as \cite{mahajan2001stocking,talluri2004revenue,el2021joint,Sumida2020RevenueUtilityTI,gao2021assortment}, to mention a few. 
The MNL model proves particularly useful in assortment optimization, as demonstrated by \citeauthor{talluri2004revenue} \cite{talluri2004revenue}, who showed that under the MNL model, the optimal assortment in the unconstrained setting is revenue-ordered. This means that it contains all products whose revenues exceed a certain threshold, simplifying the optimization problem by avoiding the consideration of exponentially numerous potential subsets. Moreover, Gallego et al. \cite{Gallego2011AGA} give a linear programming formulation for the unconstrained assortment problem under MNL. Rusmevichientong et al. \cite{Rusmevichientong2010DynamicAO} solved the version of the problem with a cardinality constraint, proving it is still solvable in polynomial time, and Desir et al. \cite{Dsir2014NearOptimalAF} studied more general capacity constraints, showing it is NP-hard to solve in the general case. Sumida et al. \cite{Sumida2020RevenueUtilityTI} and Davis et al. \cite{Davis2013AssortmentPU} studied totally unimodular constraint structures for the assortment and showed that the resulting problem can be reformulated as a linear program. 
However, when considering mixtures of MNL models (MMNL), the assortment optimization problem becomes NP-hard even in the unconstrained setting with two classes of customers as shown in Rusmevichientong et al.  \cite{rusmevichientong2014assortment}. 




%The MNL model proves particularly useful in assortment optimization, as demonstrated by Talluri and van Ryzin (2004), who showed that under the MNL model, the optimal assortment is revenue ordered. This means that it contains all products whose revenue exceeds a certain threshold, simplifying the optimization problem by avoiding the consideration of exponentially numerous potential subsets.



    
%Assortment optimization under the Multinonmial Logit model is a well anchored problem in the scientific literature. First introduced in \cite{luce1959individual}, the Multinomial Logit choice model has become very popular to model customer choices because of its tractability, contrary to more complex models. For instance, mixtures of MNL models (MMNL) become NP-hard for the assortment optimization problem as soon as there are two classes of customers . The MNL model is particularly useful in assortment optimization, where \cite{talluri2004revenue} showed that under the MNL model, the optimal assortment is revenue ordered, which means it contains all the products whose revenue is greater than some threshold. This makes the optimization problem easy to solve in polynomial time, without having to consider the exponential number of potential subsets to offer. Assortment optimization is a significant field of research, and several constrained variations of the MNL optimal assortment have been studied. \cite{Rusmevichientong2010DynamicAO} solved the version of the problem with cardinality constraints, proving it is still solvable in polynomial time, and \cite{Dsir2014NearOptimalAF} studied more general capacity constraints, showing it is NP-hard to solve in the general case. \cite{Sumida2020RevenueUtilityTI} and \cite{Davis2013AssortmentPU} studied totally unimodular constraint structures for the assortment and showed that the resulting problem can be reformulated as a linear program. 





To the best of our knowledge, this paper is the first to study assortment optimization under MNL with visibility constraints.  The topic of visibility in assortment planning has barely been covered: Chen et al. \cite{chen2022fair} studied visibility under a fairness approach, trying to enforce similar visibility for products with similar characteristics, while Wang et al. \cite{Wang2021WhenAM} studied a version of the assortment optimization problem in which they can increase the attractiveness of some products through an advertising budget. %Nevertheless, to the best of our knowledge, visibility constraints on a minimum number of appearances have never been studied in the literature so far. 
In addition, the topic of assortment optimization for a stream of customers is more often studied from an online perspective, where decisions are made sequentially such as in Davis et al. \cite{avis2015AssortmentOO} and \citeauthor{Cheung2017ThompsonSF} \cite{Cheung2017ThompsonSF}. In contrast, we study a static version of the problem, where we plan the entirety of our assortments in advance. Other versions of the problem, such as in  \citeauthor{Li2009ASA} \cite{Li2009ASA}, consider a flow of customers with randomized preferences, to which we offer a common assortment. Finally, in revenue management, pricing problems are often considered in the sense of optimizing the selling price of each product, as for example in \citeauthor{Wang2012CapacitatedAA} \cite{Wang2012CapacitatedAA}, \citeauthor{Miao2018DynamicJA} \cite{Miao2018DynamicJA} and \citeauthor{Alptekinolu2015TheEC} \cite{Alptekinolu2015TheEC}, while, for our problem, selling prices are fixed and we study instead how to price the loss generated by enforcing visibility of each product.

    
   \vspace{3mm}
    \underline{{\em Outline.}} The remainder of the paper is organized as follows. We first introduce the mathematical framework and define our problem \ref{APV} in Section \ref{sec:form}. Then, in Section \ref{sec:apv}, we devise a polynomial time algorithm for the \ref{APV} problem. In Section \ref{sec:price}, we propose a pricing strategy to charge the revenue loss to the vendors proportionally to their contribution, and illustrate numerically our results. In Section \ref{sec:extensions}, we discuss extensions of our model to the case where cardinality constraints are present. Finally, in Section \ref{sec:conclusions}, we draw some conclusions and outline future work. 
        
    
    
    %%%%%%%%%%%%%%%%%%%%%%%%%%%  %%%%%%%%%%%%%%%%%%%%%%%%%%%  %%%%%%%%%%%%%%%%%%%%%%%%%%%
    \section{Model Formulation}\label{sec:form}
    




{\bf The MNL choice model.} Let  $\mathcal{N} \coloneqq \{1,\ldots, n\}$ be a universe of substitutable products at our disposal. 
 Each product $i \in {\cal N}$ has a  price $p_i \geq 0$.  Without loss of generality, we order the products by non-increasing prices, i.e., $p_1 \geq p_2 \geq \ldots \geq p_n$.  
An assortment of products or an offer set, is simply a subset of products $S \subseteq {\cal N}$. Additionally, the option of not selecting any product is symbolically represented as product $0$, and referred to it as the no-purchase option.  



We assume that customers make choices according to a Multionomial Logit model. Under this model, each product $i \in {\cal N}$ is associated with a preference weight $v_i >0$. Note that $v_i $ captures the attractiveness of product $i$, meaning a  high  preference weight indicates a high popularity. Without loss of generality, we use the standard convention that the no-purchase preference weight is normalized to $v_0=1$. 
We use the notation $V(S) \coloneqq \sum_{i \in S} v_i$, which is the total weight of a subset $S \subseteq \mathcal{N}$.
Under the MNL model, if we offer  an assortment $S \subseteq {\cal N}$, the customer chooses product $i$ with  probability 
 $$\phi(i, S) \coloneqq \frac{v_i}{1 + \sum_{j \in S} v_j}.$$
 We refer to $\phi(i,S)$ as the choice probability of product $i$ given assortment $S$. 
 Alternatively, the customer may decide to not purchase any product, which happens with the complementary probability 
  $$\phi(0, S) \coloneqq \frac{1}{1 + \sum_{j \in S} v_j}.$$

Let $R(S)$ be the expected  revenue we get from a customer if we offer assortment $S$. In particular, we have 

$$R(S) \coloneqq   \sum_{i \in S} p_i \phi(i,S) = \frac{\sum_{i \in S} p_i v_i}{1 + \sum_{i \in S} v_i}.$$ 

 






%At each  time period, a customer arrives and we need to offer them an assortment of products $S \subseteq \mathcal{N}$. The customer then decides to purchase a single product, or to leave without purchasing anything. Customers make choices according to a choice model $\phi$, i.e., $\phi(i,S)$ is the choice probability of product $i$ given an offered assortment $S$. 




%We can now define $$R(S) \coloneqq \frac{\sum_{i \in S} p_i v_i}{1 + \sum_{i \in S} v_i}$$ which corresponds to the expected revenue when we offer the set of products $S$ under the MNL model. 


\noindent
{\bf Assortment Optimization with Visibility constraints.} 
We are presented with a stream of $T$ customers. Each  customer $t$  will be offered an assortment $S_t$. These customers make choices according to the same MNL model. In other words, a customer decides to purchase product $i$ from assortment $S_t$ with a probability denoted by $\phi(i, S_t)$, or they may choose the no-purchase option with  probability $\phi(0, S_t)$. The expected revenue we obtain from customer $t$ is $R(S_t)$.

To ensure visibility, we impose constraints that require each product $i \in \mathcal{N}$ to be shown to at least $\ell_i$ customers. Note that the parameters $\ell_i$ are exogenous and satisfy $\ell_i \in  [\![0, T]\!] $ for all $i \in \mathcal{N}$.
Our objective is to determine the assortment $S_t$ to offer to each customer $t$ in order to maximize the total expected revenue while (strictly) satisfying the visibility constraints. We refer to this problem as the {\em Assortment optimization Problem with Visibility constraints}  (\ref{APV}). It can be formulated as follows:




\begin{equation}
\label{APV}
\begin{aligned}
 \max_{ S_1, \ldots, S_T \subseteq \mathcal{N}}  & \; \;      \sum_{t=1}^T  R(S_t)   \\  
  s.t. \;\;   & \;\;  \sum_{t=1}^T  \mathbb{1}(i \in S_t) \geq \ell_i, \;\;\; \forall i \in \mathcal{N}.
\end{aligned}
\tag{\sf{APV}}
\end{equation}



    \section{Polynomial Time Algorithm for \ref{APV}} \label{sec:apv}




The primary contribution of this paper is the development of a polynomial time algorithm for \ref{APV}. To achieve this, we introduce in Section \ref{subsection:expanded} the concepts of the ``Expanded Revenue" and ``Expanded Set" of an assortment, which are instrumental for our analysis. In Section \ref{subsection:prop}, we present a polynomial time algorithm to compute the expanded set and expanded revenue. Additionally, we demonstrate the monotonicity and submodularity of the expanded revenue function. Leveraging these properties, we characterize the structure of an optimal solution for \ref{APV} and present an algorithm that compute it in $O(n T)$ time. This polynomial time algorithm is given in Section \ref{subsection:algo}. Finally, in Section \ref{subsectin:lp}, we demonstrate that \ref{APV} can be formulated as a compact linear program.



% In this section, the main result we will prove is the following:

% There exists a nested optimal solution to \ref{APV}, in which every assortment offered is included in the previous one. This solution can be computed in time $\mathcal{O}(nT)$.

% \begin{theorem}
%     The problem \ref{APV} can be solved in Polynomial time
% \end{theorem}

% To achieve this, we first introduce the notion of expanded set and expanded revenue, that will be useful for our analysis. We then prove some interesting monotonicity and supermodularity properties on the expanded revenue function. Relying on these properties, we finally identify a simple structure for an optimal solution of problem \ref{APV}, which can be computed in polynomial time.



\subsection{Expanded Revenue and Expanded Set} \label{subsection:expanded}

We begin our analysis by examining \ref{APV} in the context of a single customer. In this particular scenario, the visibility constraints are given such that either $\ell_i=0$ or $\ell_i=1$. Let $A$ denote the subset of all products where $\ell_i=1$. Consequently, \ref{APV} is transformed into the problem of identifying the assortment that maximizes revenue while including $A$. This particular problem will serve as  the building block for our analysis, as it lays the foundation for understanding the general case involving  $T$ customers. Thus, it leads us to introduce the subsequent definitions that will aid us in our analysis.

% This lead us to introduce the notion of the expanded revenue,  and the expanded revenue of a set and show how we con compute them. We introduce the following defintiions.



%Because of the visibility constraints, the assortments offered under \ref{APV} will have to enforce certain products. Thus it becomes interesting to study how we can maximize the expected revenue of a set once if we force it to contain certain elements, and we optimize over the remaining ones.

%We therefore naturally introduce the following notations:


\begin{definition}[\bf Expanded revenue]
    \label{expanded revenue}
    Let $A \subseteq \mathcal{N}$. The expanded revenue of $A$, denoted as $\overline{R}(A)$, is defined as the maximum expected revenue achieved by any assortment in ${\cal N}$ that contains $A$. In particular, it is given by

    \begin{equation} \label{eq:exrev}
        \overline{R}(A) \coloneqq \max_{S \subseteq \mathcal{N} , \; A \subseteq S} R(S).
    \end{equation}

The optimal solution of the maximization problem in \eqref{eq:exrev} is referred to as the expanded set of $A$. In case multiple optimal solutions exist, we break ties by selecting the optimal assortment with the largest cardinality. Lemma \ref{Compute expanded set} will show that the expanded set is well defined. Formally, we provide the following definition. 

 
\end{definition}


\begin{definition}[\bf Expanded set]
    \label{expanded set}
The expanded set of $A$, denoted as $\overline{A}$, is defined as the assortment within $\mathcal{N}$ that  maximizes the expected revenue  among all assortments containing $A$.  If multiple assortments achieve the same maximum expected revenue, $\overline{A}$ is selected as the assortment with the largest cardinality.  Mathematically, $\overline{A}$ is given by
    $$\overline{A} \coloneqq \underset{S \subseteq \mathcal{N} , \; A \subseteq S}{\arg \max}   \left\{ |S| \; : \; R(S)=  \overline R (A)   \right\}.   $$
\end{definition}


% Let $A \subseteq \mathcal{N}$. The expanded set of $A$, denoted as $\overline{A}$, is defined as the assortment within $\mathcal{N}$ that satisfies two criteria: it contains $A$, and it simultaneously achieves the maximum revenue and the maximum cardinality among all possible assortments. Mathematically, $\overline{A}$ is given by:


% Note that Problem \eqref{eq:exrev} is equivalent to \ref{APV} in the scenario where we have a single customer $(T=1)$ and $A$ is defined as the set of products that have to be shown once, i.e., $A= \{ i \in {\cal N} : \ell_i =1 \} $. In that case, $\overline A$ corresponds to the optimal assortment to this problem with the largest cardinality.

% We will use $\overline{R}$ as a set function that takes input an assortment $A \subseteq R$ and returns $\overline{R}(A)$. In the next section we study several properties of this function as well as properties of the expanded set.



Problem \eqref{eq:exrev} can be viewed as equivalent to \ref{APV} when considering a single customer scenario $(T=1)$ and defining $A$ as the set of products that need to be shown once, i.e., $A= \{ i \in {\cal N} : \ell_i =1 \}$. Thus, $\overline A$ represents the optimal assortment, with the largest cardinality, for the problem.

In our  analysis, we consider $\overline{R}$ as a set function that takes an assortment $A \subseteq \mathcal{N}$ as input and returns $\overline{R}(A)$. Note that $\overline{R}(A) = R(\overline{A})$. In the subsequent section, we delve into examining various properties of this function, as well as properties associated with the expanded set.







%$\overline{R}$ is the expanded revenue function, such that $\overline{R}(A) = R(\overline{A})$. When there are several solutions for $\overline{A}$, we define $\overline{A}$ as the one with the highest cardinal. The following lemma justifies that $\overline{A}$ is well defined and can be computed efficiently.

\subsection{Properties of the Expanded Revenue} \label{subsection:prop}

In this section, we first show that we can compute the expanded revenue and the expanded set of a given assortment in polynomial time. Then, we show that the expanded revenue function is monotone and submodular. These  structural properties will be useful in designing our algorithm to solve \ref{APV} for $T \geq 1$.




\vspace{2mm}
\noindent
{\bf Computing the expanded revenue and expanded set.}
Recall without loss of generality that $p_1 \geq \ldots \geq p_n$. We define  an assortment $S$ to be price-ordered if $S=\{1,\ldots,k\}$ for some $1 \leq k \leq n$. Essentially, a price-ordered assortment prioritizes products with high prices. It is worth noting there  are only $n$ possible price-ordered assortments.
Consider an assortment $A \subseteq {\cal N}$, and its expanded set $\overline A$. In the following lemma, we demonstrate that $\overline A$ is the union of $A$ and a price-ordered assortment. Since there are only $n$ possible price-ordered assortments, it is sufficient to compute the expected revenue of the assortments $A \cup \{1,\ldots, k\}$ for each $k \in \{1, \ldots ,n\}$. The expanded set corresponds to the assortment with the highest expected revenue. In the case of multiple assortments with the same maximum revenue, we break ties by selecting the one with the largest cardinality. Thus, the expanded set $\overline A$ can be computed in linear time, specifically $O(n)$. The expanded revenue is simply $\overline{R}(A)= R(\overline A)$. The proof of Lemma \ref{Compute expanded set} leverages some structural properties of the revenue function under MNL that are presented in Appendix \ref{apx1}.

\begin{lemma}
    \label{Compute expanded set}
    For any $ A \subseteq \mathcal{N}$, the expanded set of $A$ is given by  
    $ \overline{A} = A \cup \{i \in \mathcal{N} : p_i \geq \overline{R}(A) \}.$ Furthermore, $\overline{R}(A)$ and $\overline{A}$ can be computed in time $\mathcal{O}(n)$.
\end{lemma}

\proof
    By definition, we have $A \subseteq \overline{A}$. Hence, we can write $\overline{A} = A \cup B$, where $B \subseteq \mathcal{N} \setminus A$. We will prove that $$B = \{i \in \mathcal{N} \setminus A   \; : \; p_i \geq \overline{R}(A) \}.$$ 
    Assume that there exists $i \in B$ such that $p_i < \overline{R}(A) = R(A \cup B)$. It is known that, under the MNL model, when we add a product $j$ to an assortment $S$, the revenue of this assortment increases if and only of $p_j \geq R(S)$. For completeness, we provide the statement and the proof of this result in Lemma \ref{Revenue variations} in Appendix \ref{apx1}. Using this lemma implies that removing $i$ from $B$ would strictly increase the expected revenue $R(A)$, which contradicts the optimality of $A \cup B$. \\
    Now, assume that there exists $i \in \mathcal{N} \setminus A$ such that $p_i \geq \overline{R}(A)$ but $i \notin B$. Again by Lemma \ref{Revenue variations}, adding $i$ to $B$ would increase the revenue. If this increase is strict, it contradicts the optimality of $A \cup B$. If the revenue stays the same, it contradicts the definition of $\overline{A} = A \cup B$ as the optimal solution with maximum cardinality.
    Therefore, $$\overline{A} = A \cup \{i \in \mathcal{N} \setminus A, p_i \geq \overline{R}(A) \}.$$
    Finally, $\overline{A}$ can be computed in time $\mathcal{O}(n)$. Indeed, we just have to start from $A$, and add elements by decreasing price. At each iteration, we can compute the new revenue from the previous one in constant time if we store the current numerator and denominator, since we only need to add $p_i v_i$ to the former and $v_i$ to the latter when we reach element $i$. Finally, we just have to pick the highest revenue set among the $n$ sets computed.
    \hfill \qedsymbol
    %We can write $\overline{R}(A) = \max_{B \subseteq \mathcal{N} \setminus A} \frac{\sum_{i \in A \cup B} p_i v_i}{1 + V(A) + V(B)}$, so $\overline{R}(A) \geq \frac{\sum_{i \in A \cup B} p_i v_i}{1 + V(A) + V(B)}$ for each $B \subseteq \mathcal{N} \setminus A$, by definition of $\overline{R}(A)$, and where $V(S) = \sum_{i \in S} v_i$. This inequality is an equality iff $B$ is optimal for $\overline{R}(A)$, and is equivalent to \\
    %$\overline{R}(A)(1 + V(A) + V(B)) +  \geq \sum_{i \in A \cup B} p_i v_i$, \\
    %$\iff \overline{R}(A) \geq  \sum_{i \in A \cup B} (p_i - \overline{R}(A)) v_i 
    %= \sum_{i \in A} (p_i - \overline{R}(A)) v_i + \sum_{i \in B} (p_i - \overline{R}(A)) v_i$\\
    %The first term is a constant, and the second term is maximal when $B = \{i \in \mathcal{N} \setminus A, p_i \geq \overline{R}(A) \}$, which is when the inequality becomes an equality. We could also remove from $B$ the items $i$ such that $p_i = \overline{R}(A)$ and still have an equality, but since we defined $\overline{A}$ as the solution with maximal cardinal, we keep all of them. \\
    %Finally, by the barycenter property, adding items from $\mathcal{N} \setminus A$ to $A$ by decreasing price will again lead to a revenue first increasing and later decreasing, and therefore Algorithm \ref{alg1} applied to items $\mathcal{N} \setminus A$ and function $S \mapsto R(A \cup S)$ (instead of $S \mapsto R(S)$) will compute $\{i \in \mathcal{N} \setminus A, p_i \geq R(\overline{A}) \}$ in linear time. Adding A to this subset gives us $\overline{A}$ and therefore solves the problem in linear time.
%\end{proof}





%\noindent
%{\bf Monotonicity.} 

Next, we show that the expanded revenue is a non-increasing function and  the expanded set is a non-decreasing function.


\begin{lemma}[\bf Monotonicity]
    \label{monotonicity} 
    For any $ A \subseteq B \subseteq \mathcal{N}$, we have  $ \overline{A} \subseteq \overline{B}$ and $\overline{R}(A) \geq \overline{R}(B)$.   

\end{lemma}

\proof
    For $A \subseteq B \subseteq \mathcal{N}$, we have $\{S \subseteq \mathcal{N} \; :\; B \subseteq S \} \subseteq \{S \subseteq \mathcal{N} \; : \; A \subseteq S \}.$ So every feasible solution for $\max_{S \subseteq \mathcal{N} , \; B \subseteq S} R(S)$ is a feasible solution for $\max_{S \subseteq \mathcal{N} , \; A \subseteq S} R(S)$. Therefore, $\overline{R}(A) \geq \overline{R}(B)$. It follows that $\{i \in \mathcal{N}, p_i \geq \overline{R}(A) \} \subseteq \{i \in \mathcal{N}, p_i \geq \overline{R}(B) \},$ and therefore $\overline{A} = A \cup \{i \in \mathcal{N}, p_i \geq \overline{R}(A) \} \subseteq B \cup \{i \in \mathcal{N}, p_i \geq \overline{R}(B) \} = \overline{B}.$
    \hfill \qedsymbol
%\end{proof}

Finally, we show that  the expanded revenue function  $\overline R$ is supermodular. This is the main property of the expanded revenue function that will play  a fundamental role later in our analysis.


\begin{lemma}[\bf Supermodularity]
    \label{supermodularity}
    The expanded revenue function $\overline{R}$ is  supermodular, i.e.,
     $$\forall A, B \subseteq \mathcal{N},  \; 
     \; \overline{R}(A \cup B) + \overline{R}(A \cap B) \geq \overline{R}(A) + \overline{R}(B).$$
\end{lemma}

\proof
    First, we provide an intermediate computation: $\forall S, B \subseteq \mathcal{N}$,
    \begin{equation}   
    \label{technical computation}
    \begin{aligned}
    R(S) - R(S \cup B) &= R(S) - R(S \cup (B \setminus S)) \\
    &= R(S) - \frac{\sum_{i \in S \cup (B \setminus S)} p_i v_i}{1 + V(S) + V(B \setminus S)} \\
    &= \frac{R(S)(1 + V(S)) + R(S)V(B \setminus S) - \sum_{i \in B \cup S} p_i v_i}{1 + V(S \cup B)} \\
    &= \frac{\sum_{i \in S} p_i v_i +R(S)V(B \setminus S) -  \sum_{i \in S} p_i v_i - \sum_{i \in B \setminus S} p_i v_i}{1 + V(S \cup B)} \\
    &=  \frac{\sum_{i \in B \setminus S} (R(S) - p_i) v_i}{1 + V(S \cup B)}.
    \end{aligned}
    \end{equation}
    Now, we show that $\forall A, B \subseteq \mathcal{N}, R(\overline{A} \cup \overline{B}) + \overline{R}(A \cap B) \geq \overline{R}(A) + \overline{R}(B)$. This is equivalent to $R(\overline{A \cap B}) - R(\overline{B}) \geq R(\overline{A}) - R(\overline{A} \cup \overline{B})$, which can be rewritten as 
    \begin{equation} \label{japan}
    \frac{\sum_{i \in \overline{B} \setminus \overline{A \cap B}} (R(\overline{A \cap B}) - p_i) v_i}{1 + V(\overline{B})} \geq \frac{\sum_{i \in \overline{A} \cup \overline{B} \setminus \overline{A}} (R(\overline{A}) - p_i) v_i}{1 + V(\overline{A} \cup \overline{B})},
    \end{equation}
    using Equation \eqref{technical computation}, and the fact that $A \cap B \subseteq B \implies \overline{A \cap B} \subseteq \overline{B}$ by Lemma \ref{monotonicity}. 
    
    We know that $\forall i \in \overline{B} \setminus \overline{A \cap B}, p_i < R(\overline{A \cap B})$. Indeed, $$\overline{A \cap B} = (A \cap B) \cup \{i \in \mathcal{N}, p_i \geq R(\overline{A \cap B}) \} \supseteq \{i \in \mathcal{N}, p_i \geq R(\overline{A \cap B}) \}.$$ For the same reason, $\forall i \in \overline{A} \cup \overline{B} \setminus \overline{A}, \; p_i < R(\overline{A})$. 
    Then, we show that $\overline{A} \cup \overline{B} \setminus \overline{A} \subseteq \overline{B} \setminus \overline{A \cap B}$. Indeed, $\overline{A} \cup \overline{B} \setminus \overline{A} \subseteq \overline{B}$, and then $A \cap B \subseteq A \implies \overline{A \cap B} \subseteq \overline{A}$ by Lemma \ref{monotonicity}. So since $\overline{A} \cup \overline{B} \setminus \overline{A}$ contains no element of $\overline{A}$, it also contains no element of $\overline{A \cap B}$. Thus, $\overline{A} \cup \overline{B} \setminus \overline{A} \subseteq \overline{B} \setminus \overline{A \cap B}$. 
    In addition, Lemma \ref{monotonicity} gives $R(\overline{A \cap B}) \geq R(\overline{A})$.
    Therefore, 
    \begin{equation} \label{jap2}
        0<\sum_{i \in \overline{A} \cup \overline{B} \setminus \overline{A}} (R(\overline{A}) - p_i) v_i \leq \sum_{i \in \overline{B} \setminus \overline{A \cap B}} (R(\overline{A \cap B}) - p_i) v_i,
    \end{equation}
     because for each term in the left sum, there is a different term in the right sum that is superior or equal to it, and the additional terms in the right sum are all positive. 
    Finally, we have
    \begin{equation} \label{jap3}
       V(\overline{B}) = \sum_{i \in \overline{B}} v_i \leq \sum_{i \in \overline{A} \cup \overline{B}} v_i = V(\overline{A} \cup \overline{B}), 
    \end{equation}
   Combining \eqref{jap2} and \eqref{jap3} give the desired inequality \eqref{japan}, which is, as mentioned earlier, equivalent to  $$\forall A, B \subseteq \mathcal{N}, R(\overline{A} \cup \overline{B}) + \overline{R}(A \cap B) \geq \overline{R}(A) + \overline{R}(B).$$
    Then, as $A \subseteq \overline{A}$ and $B \subseteq \overline{B}$, we have $A \cup B \subseteq \overline{A} \cup \overline{B}$, and therefore by Lemma \ref{monotonicity}, 
    $$\overline{R}(A \cup B) \geq \overline{R}(\overline{A} \cup \overline{B}) \geq R(\overline{A} \cup \overline{B}).$$ Therefore,
    $$R(\overline{A} \cup \overline{B}) + \overline{R}(A \cap B) \geq \overline{R}(A) + \overline{R}(B) \implies \overline{R}(A \cup B) + \overline{R}(A \cap B) \geq \overline{R}(A) + \overline{R}(B).$$
    \hfill \qedsymbol
%%\end{proof}



\subsection{Optimal Solution  for \ref{APV}} \label{subsection:algo}


In this section, we present the main technical result in this paper. In particular, we characterize the structure of an optimal solution of \ref{APV}. Our characterization relies on the supermodularity property of the expanded revenue function. Moreover, we show that we can compute such a solution in $O(nT)$, which gives us a polynomial time algorithm to solve \ref{APV}.

\vspace{2mm}
\noindent
{\bf Optimal solution.} Consider an instance of \ref{APV}. Recall that for all $i \in {\cal N}$,  $\ell_i$ is the lower bound on the minimum number of customers for which we must offer product $i$. For $t \in \{0,1,\ldots,T \},$ we define the following sets
\begin{equation}
    L_t = \{i \in \mathcal{N}, \ell_i = t \}.
\end{equation}
Our candidate solution for \ref{APV} is given by
\begin{equation} \label{eq:sol}
    {S_t^*} = \overline{\bigcup_{t \leq u \leq T} L_u}, \quad \forall t \in \{1,\ldots,T \}.
\end{equation}


Note that $(L_t)_{0 \leq t \leq T}$ is a partition of $\cal N$. Moreover, since  $ \overline{\bigcup_{t+1 \leq u \leq T} L_u } \subseteq  \overline{\bigcup_{t \leq u \leq T} L_u}$, the monotonicity property in Lemma \ref{monotonicity} implies that $S_{t}^* \subseteq S_{t+1}^*$ for any $t=0,\ldots,T-1$. Therefore, our solution has nested structure, i.e.,
$ S_1^* \subseteq S_2^* \ldots \subseteq S_T^*.$ In the following, we prove that the assortments given by \eqref{eq:sol} are optimal for \ref{APV}. Moreover, they can be computed in polynomial time. Indeed, Lemma \ref{Compute expanded set} shows that each of them can be computed in time $\mathcal{O}(n)$, so the entire solution can be computed in time $\mathcal{O}(nT)$.



% \begin{definition}[\bf Nested solution]
%     \label{Nested solution}
%     For each instance of the problem \ref{APV}, we define the partition $(L_t)_{0 \leq t \leq T}$ of $\mathcal{N}$ by: $\forall t \in [\![0, T]\!], \; L_t = \{j \in \mathcal{N}, \ell_j = t \}$ \\
%     We define the sequence of assortments $(S_t^*)_{1 \leq t \leq T} \coloneqq (\overline{\bigcup_{t \leq u \leq T} L_u})_{1 \leq t \leq T}$, that is $S_1^*, S_2^*, \ldots, S_T^* = \overline{L_1 \cup \ldots \cup L_T}, \overline{L_2 \cup \ldots \cup L_T}, \ldots, \overline{L_T}$
% \end{definition}

%We observe that this solution structure is nested: $S_{t+1}^* \subseteq S_t^* \;\; \forall t \in [\![1, T-1]\!] $


\begin{theorem}{}
    \label{Solution structure}
    The  sequence of assortments $(S_t^*)_{1 \leq t \leq T}$ defined in \eqref{eq:sol} is optimal for \ref{APV}. % Moreover, each permutation of these sets is also an optimal solution for \ref{APV}. 
    Moreover, such a solution can be computed in $O(nT)$ time.
\end{theorem}


\proof
    We reason by recurrence using the supermodularity property. 
    For $T=1$, the problem we want to solve is exactly $\max_{S \subseteq \mathcal{N} \; s.t. \; L_1 \subseteq S} R(S) = \overline{R}(L_1)$, and by definition we know that $S_1^* \coloneqq \overline{L_1}$ is an optimal solution. 
    
    Let us now take an instance of problem \ref{APV} with $T \geq 2$, and assume that the result is valid for $T-1$. For simplicity, we use  the notation  $$R_T(S_1, \ldots, S_T) \coloneqq \sum_{t=1}^T R(S_t).$$ 
    Let $\hat{S_1}, \ldots, \hat{S_T}$ be an optimal solution of our problem and let $$R_T^* \coloneqq \max_{S_1, \ldots, S_T} R_T(S_1, \ldots, S_T) = \sum_{t=1}^T R(\hat{S_t}).$$ 
    We notice that $\forall t \in [\![1, T]\!], R(\hat{S_t}) = \overline{R}(\hat{S_t})$. In fact,  by definition of expanded revenue, we have $R(\hat{S_t}) \leq \overline{R}(\hat{S_t})$. Moreover, if the inequality was strict, we could complement $\hat{S_t}$ by adding to it the elements in $\overline{\hat{S_t}} \setminus \hat{S_t}$. This would strictly increase the objective value of \ref{APV} while still being a feasible solution, which contradicts the optimality of $\hat{S_1}, \ldots, \hat{S_T}$. Hence, $R(\hat{S_t}) = \overline{R}(\hat{S_t})$ for all $t\in [\![1, T]\!]$.
    %We now apply the following algorithm:
    %\begin{quote}
    %\begin{algorithmic}
    %    \State Initialize variables $S_1, \ldots, S_T$ with $S_t \leftarrow \hat{S_t}$
    %    \For{$t$ going from 2 to $T$}
    %        \State $S_1 \leftarrow S_1 \cup \hat{S_t}$, $S_t \leftarrow \hat{S_t} \cap S_1$
    %    \EndFor
    %    \State return $\overline{S_1}, \ldots, \overline{S_T}$
    %\end{algorithmic}
    %\end{quote}

    We now iteratively build a new feasible solution for \ref{APV}. We initialize $S_t \leftarrow \hat{S_t}$ for each $t \in [\![1, T]\!]$. Then, for $t$ going from 2 to T, we apply the following operations: $S_1 \leftarrow S_1 \cup \hat{S_t}$, $S_t \leftarrow S_1 \cap \hat{S_t}$. Finally, we take $\overline{S_1}, \ldots, \overline{S_T}$ as our new solution. We will refer to this algorithmic process as {\sf Algo}.

    At each iteration of {\sf Algo}, the solution $S_1, \ldots, S_T$ remains feasible for problem \ref{APV}. Indeed, if $A$ and $B$ are two assortments, then $A \cup B$ and $A \cap B$ contain the same products and with the same number of occurrences as $A$ and $B$. This is because products that were only in $A$ are now only in $A \cup B$ but not in $A \cap B$. Similarly, products that were only in $B$ are now only in $A \cup B$ but not in $A \cap B$.
     Finally, products that were in both $A$ and $B$ are in $A \cup B$ and $A \cap B$. 
    
    %At the end of {\sf Algo}, we will construct a new optimal solution with certain properties. In fact, 
    Consider step $t$ of {\sf Algo} where the current solution is $S_1, \ldots, S_T$. From Lemma \ref{supermodularity}, the supermodularity of the expanded revenue function gives
     % the total potential revenue $\overline{R_T}(S_1, \ldots, S_T) \coloneqq \sum_{t=1}^T \overline{R}(S_t)$ of the solution increases. Indeed, 
    $$\overline{R}(S_1 \cup \hat{S_t}) + \overline{R}(S_1 \cap \hat{S_t})  \geq \overline{R}(S_1) + \overline{R}(\hat{S_t}).$$ 
    Moreover,  $\overline{R}(S_u)$ remains the same as in step $t-1$ for all $u \notin \{1, t\}$. Therefore, the sum of the expanded revenue of the assortments at each step of {\sf Algo} increases. Hence, by induction, the final assortments $S_1,S_2,\ldots,S_T$ that we obtain at the end verify 
    $$ \sum_{t=1}^T \overline{R}(S_t) \geq   \sum_{t=1}^T \overline{R}(\hat{S_t}) = R_T^*.$$ 
    Moreover,
    $R_T(\overline{S_1}, \ldots, \overline{S_T}) = \sum_{t=1}^T \overline{R}(S_t).$
    Therefore,  $\overline{S_1}, \ldots, \overline{S_T}$  is also optimal for \ref{APV}.
    %Let $S_1,\ldots,S_T$ the final assortments that we get at the end of {\sf Algo} and 
    Note that at the end of {\sf Algo} we have $S_1 = \bigcup_{1 \leq t \leq T} \hat{S_t}.$ 
    Because of the visibility constraints, we should have  $$\bigcup_{1 \leq t \leq T} L_t \subseteq \bigcup_{1 \leq t \leq T} \hat{S_t}=S_1.$$
    Finally, let $Z_1=S_1 \setminus L_0= \bigcup_{1 \leq t \leq T} L_t  $ and $Z_t=S_t$ for all $t=2,\ldots,T$. By monotonicity of the expanded revenue, Lemma \ref{monotonicity} gives $\overline{R}(Z_1) \geq \overline{R}(S_1)$, i.e., $R(\overline{Z_1}) \geq R(\overline{S_1})$, which implies
     $$R_T(\overline{Z_1}, \ldots, \overline{Z_T}) \geq  R_T^*$$
    % Finally, we remove the elements of $L_0$ from $S_1$ and the rest of the  In other words, we define the new set $S_1$ to be $S_1 :=  \bigcup_{1 \leq t \leq T} L_t$.  
    % because it only consists in removing elements from $L_0$ that were in $S_1$, which can only improve the maximum revenue $\overline{R}(S_1)$ by the monotonicity property. Therefore, we have $\overline{S_1} = \overline{\bigcup_{1 \leq t \leq T} L_t}$, 
    Hence, $\overline{Z_1}, \ldots, \overline{Z_T}$ is also an optimal solution of \ref{APV}. Note that $\sum_{t=2}^T \mathbb{1}({i \in Z_t} )\geq \ell_i -1~\forall i \in \mathcal{N}$ since all elements that need to appear at least once are already in ${Z_1} \subseteq \overline{Z_1}$.
    Hence, we can optimize $$R_{T-1}^* \coloneqq \max_{S_2, \ldots, S_T} \sum_{t=2}^T R(S_t),$$ independently of $\overline{Z_1}$ and under the constraints $\sum_{t=2}^T \mathbb{1}({i \in Z_t}) \geq \ell_i -1~\forall i \in \mathcal{N}$, which is exactly our initial problem but with $T-1$ customers. 
    By recurrence hypothesis, $ (\overline{\bigcup_{t \leq u \leq T} L_u})_{2 \leq t \leq T}$ is an optimal solution for $R_{T-1}^*.$ Moreover, $\overline{Z_1}= \overline {\bigcup_{1 \leq t \leq T} L_t } $.   
    % $(Z_t)_{1 \leq t \leq T}$ is optimal for $R_T^*$. Indeed, $(Z_1, \overline{S_2}, \ldots, \overline{S_T})$ was already optimal for $R_T^*$, and $$R_T(Z_1, \ldots, Z_T) = R(Z_1) + R_{T-1}(Z_2, \ldots, Z_T) \geq R(Z_1) + R_{T-1}(\overline{S_2}, \ldots, \overline{S_T}) = R_T^*.$$    
    Therefore, $ (\overline{\bigcup_{t \leq u \leq T} L_u})_{1 \leq t \leq T}$ is an optimal solution for \ref{APV}.
    \hfill \qedsymbol
%%\end{proof}





% \begin{corollary}
%     A solution to \ref{APV} can be computed in linear time, that is $\mathcal{O}(T n)$
% \end{corollary}

% \proof
%     Indeed, a solution is given by $S_t^* = \overline{\bigcup_{t \leq u \leq T} L_u}$ as defined in \ref{Nested solution}, and each $S_t^*$ can be computed in $\mathcal{O}(n)$ time by \ref{Compute expanded set}.
% %%\end{proof}

% Our algorithm for \ref{APV} is summarized below:

% \begin{algorithm}[H] % enter the algorithm environment 
%     \caption{Optimal algorithm for \ref{APV}} % give the algorithm a caption
%     \label{alg1} % and a label for \ref{} commands later in the document
%     \begin{algorithmic} % enter the algorithmic environment
%     \For{$t = 1, \ldots, T$}
%         \State $S_t^* \leftarrow \bigcup_{t \leq u \leq T} L_u$
%         \While{$R(S_t^*)$ is non-decreasing}
%             \State Add elements to $S_t^*$ by decreasing price
%         \EndWhile
%     \EndFor
%     \State Return $S_1^*, \ldots, S_T^*$
%     \end{algorithmic}
% \end{algorithm}




\subsection{Linear Program for \ref{APV}} \label{subsectin:lp}

In this section, we derive and prove a reformulation of \ref{APV} as a Linear Program (LP). First, consider the classical assortment problem, i.e., that without any constraints, under MNL model for a single customer 
\begin{equation}
\label{Unconstrained problem}
\begin{aligned}
\max_{S \subseteq \mathcal{N}} \quad  R(S).
\end{aligned}
\tag{\sf{AP}}
\end{equation}
It is well known that \ref{Unconstrained problem} can be formulated as the following LP (see Gallego et al. \cite{Gallego2011AGA}),

\begin{equation*}
\label{Unconstrained problem LP}
\begin{aligned}
 \max_{S \subseteq \mathcal{N}} R(S) =  \max_{(\alpha_i)_{0 \leq i \leq n}} \left\{ \sum_{i=1}^n p_i \alpha_i \quad s.t. \quad \forall i \in \mathcal{N}, 0 \leq \frac{\alpha_i}{v_i} \leq \alpha_0, \quad \sum_{i=0}^n \alpha_i = 1 \right\}.
\end{aligned}
\end{equation*}

% \textcolor{red}{Omar: review and shorten this paragraph}\\
% The fact that the \ref{Unconstrained problem} optimal solution (as outlined in Appendix \ref{apx1}) is equivalent to the optimal solution of the above LP can be seen as follows.
% Let $S$ be an optimal solution to the initial problem \ref{Unconstrained problem}. Then, we can define $\alpha_0 = \frac{1}{1 + V(S)}$ and $\forall i \in \mathcal{N}, \alpha_i = \mathbb{1}_{\{i \in S\}} \frac{v_i}{1 + V(S)}$, which is a feasible solution to the LP. And the two objective functions have the same value. Therefore, the optimal value of the LP is greater or equal to the optimal value of the initial problem. 
% Let $(\alpha_i)_{0 \leq i \leq n}$ be an optimal solution to the LP. Then, we define $S = \{i \in \mathcal{N},~\alpha_i = v_i \alpha_0 \}$, which is feasible for the initial problem. Since there are $n+1$ variable in the LP,  we can find a solution such that at least $n+1$ constraints are tight (a solution on an extremal point of the feasibility polytope). The equality constraint will always be verified, and since within the $2n$ inequality constraints, each one is incompatible with another, we have that for each $i, \alpha_i \in \{0, \alpha_0 v_i\}$. Therefore, the two objective functions have the same value. This proves that the \ref{Unconstrained problem} optimal value is greater or equal to the optimal value of the LP.
% As a result, the two problems have the same optimal value, which proves the equivalence. %and therefore each optimal solution for one yields an optimal solution for the other.

Motivated by the structure of the above LP  and the structure of our optimal solution of \ref{APV} given in Equation \eqref{eq:sol},
we propose a linear formulation for the $T$ customers assortment problem with visibility constraints \ref{APV}.

\begin{theorem}[\bf LP for \ref{APV}]
    \ref{APV} is equivalent to the following linear program:
    \begin{equation*}
    \label{Visibility problem LP}
    \begin{aligned}
     \max_{(\alpha_i^t)_{\substack{1 \leq t \leq T \\ 0 \leq i \leq n}}} & \sum_{i=1}^n p_i \sum_{t=1}^T \alpha_i^t \\
     s.t. \quad & \forall t \in [\![1, T]\!], \;\; \sum _{i=0}^n \alpha_i^t = 1, \\
     & \forall i \in \mathcal{N}, \;\; \forall t \in [\![1, \ell_i]\!], \;\; \frac{\alpha_i^t}{v_i} = \alpha_0^t, \\
     & \forall i \in \mathcal{N}, \;\; \forall t \in [\![\ell_i +1, T]\!], \;\; 0 \leq \frac{\alpha_i^t}{v_i} \leq \alpha_0^t .
    \end{aligned}
    \end{equation*}
\end{theorem}


\proof
Let $(S_1^*, \ldots, S_T^*)$ be the optimal solution of \ref{APV} given in \eqref{eq:sol}, i.e., $S_t^* = \overline{L_t \cup \ldots \cup L_T},$ for all $t=1, \ldots, T$.  We define the solution $ \alpha_0^t = \frac{1}{1 + V(S_t^*)},$ and  $\forall i \in \mathcal{N}, \quad \alpha_i^t = \mathbb{1}(\{i \in S_t^*\}) \frac{v_i}{1 + V(S_t^*)}.$ For any $ i \in \mathcal{N}$ and  $  t \in [\![1, \ell_i]\!],$ we have $  L_i \subseteq S_t^*$. Hence, our solution verify the second line of constraints of the LP. The first and third line of constraints are verified by definition of the $\alpha_i^t$. Therefore our solution is feasible for the LP and it easy to verify that the objective function of our solution is the same as the optimal objective value of \ref{APV}.  This proves that the optimal value of the LP is superior or equal to the optimal value of \ref{APV}.

Now, let $(\alpha_i^t)_{\substack{1 \leq t \leq T \\ 0 \leq i \leq n}}$ be an optimal solution to the LP. We define 
$S_t = \{i \in \mathcal{N} : \alpha_i = v_i \alpha_0 \}.$
We have $T(n+1)$ variables in the LP, so  at least $T(n+1)$ constraints are active. The $T + \sum_{i=1}^n \ell_i$ equality constraints will always be verified, which means that out of the $2 \sum_{i=1}^n (T-\ell_i)$ remaining inequality constraints, at least half of them are tight. Each lower bound inequality is incompatible with the upper bound inequality for each $i$, so we have $\forall i \in [\![1, n]\!], \alpha_i^t \in \{0, v_i \alpha_0^t$\}. Therefore, the two objective functions have the same value. Then, because at least the ${\ell}_i$ first sets contain product $i$ because of constraint $\forall i \in \mathcal{N}, \forall t \in [\![1, \ell_i]\!], \frac{\alpha_i^t}{v_i} = \alpha_0^t$, the solution $(S_1, \ldots, S_T)$ created is feasible for \ref{APV}. So the optimal value of the LP is inferior or equal to the optimal value of \ref{APV}. This proves that the two problems have the same optimal value, and each solution of the LP allows us the reconstruct a solution to \ref{APV}.
\hfill \qedsymbol
%%\end{proof}

%We observe that this linear reformulation gives us a new way to solve the problem \ref{APV} in polynomial time, even though the first solution we proposed is likely to be the most efficient because it uses all the information about the problem and runs in linear time.



    \section{Price of Visibility} \label{sec:price}


In this section, we investigate the impact of visibility constraints on the total expected revenue, comparing it to the unconstrained setting where there are no visibility constraints. In Section \ref{subsection:loss}, we quantify the loss resulting from enforcing the visibility constraints. In Section \ref{subsection:share}, we introduce a novel method to distribute the loss among different products in proportion to their contribution to the overall loss. 
 Finally, in Section \ref{subsection:numeric}, we illustrate   our method through a series of numerical experiments.



\subsection{The loss due to Visibility constraints}
\label{subsection:loss}




Consider the unconstrained assortment optimization \ref{Unconstrained problem} and let $S^*$ be its optimal solution. It is known that there exists an optimal assortment that is price-ordered. This is a standard result in assortment optimization under the MNL model (see \citeauthor{talluri2004revenue} \cite{talluri2004revenue}).\footnote{We refer the reader to Appendix \ref{apx1} for further discussion of assortment optimization under MNL.}
In the absence of visibility constraints and with $T$ customers, it is optimal to offer assortment $S^*$ to each customer. Consequently, the total expected revenue in the unconstrained setting can be expressed as $T R(S^*)$. As the unconstrained problem serves as a relaxation of \ref{APV}, it possesses a higher objective function.
In the following example, we show that enforcing the visibility constraints can imply a gap that is arbitrary bad as compared to the unconstrained setting. 




%After enforcing visibility constraints in our assortment problem, we observe a decrease in the total revenue. 

%Indeed, if we remove all the constraints, that is $\ell_i = 0 \; \forall i \in \mathcal{N}$, we come back to \ref{Unconstrained problem}, as defined in \ref{apx1}. The best solution we could hope for is $S_t = S^* \quad \forall 1 \leq t \leq T$, because it would maximize each term $R(S_t)$ independently of the others, therefore maximize the total revenue. However, because of the visibility constraints, we had to integrate in the assortments $(S_t)_{1 \leq t \leq T}$ some elements that were not in $S^*$, driving the revenue down. 




% \noindent{\bf Example.} Consider two products such that for all $i \in \{1,\ldots, n-1\}$
% $ p_i = 1, v_i = 1$ and $ \ell_i = 0$. For product $n$, let  $p_n = 0, v_n = n^2$,  and $\ell_n = T$.  We consider the setting where $n$ is large.
% To compute the optimal assortment for \ref{Unconstrained problem}. As mentioned earlier, it sufficient to evaluate the revenue of the price-ordered assortments and choose the one with the highest revenue. We have 
% $R(\{1,\ldots,k\})=  \frac{k}{1 + k} \quad \forall k \in \{1,\ldots, n-1\}$ and $R(\mathcal{N}) = \frac{n-1}{1 + n-1 + n^2} = \frac{n-1}{n + n^2} \leq \frac{n-1}{n} $. Therefore, the optimal assortment is $S^*=\{1, \ldots,n-1 \}$ and $R^*=\frac{n-1}{n}.

% On the other hand,

%  However, $\forall A \subseteq [\![1, n-1]\!], \; R(\{n\} \cup A) = \frac{1 + |A|}{1 + |A| + n^2}$, which increases with $|A|$, so the optimal solution for \ref{APV} is $S_t^* = \overline{\{n\}} = \mathcal{N} \; \forall t$. Therefore the revenue goes to zero because of the visibility constraints on product $n$, while it was close to 1 for a large value of $n$ without constraints.



\vspace{2mm}
\noindent{\bf Example.} Consider an instance of \ref{APV} with two products $n=2$ and $T$ customers. Let $ p_1 = 1, v_1 = 1, \ell_1 =0$ and  $p_2 = 0, v_2 = M, \ell_2 = T$.  We consider the setting where $M$ is large.
To compute the optimal assortment for \ref{Unconstrained problem}. As mentioned earlier, it sufficient to evaluate the revenue of the price-ordered assortments and choose the one with the highest revenue. We have 
$R(\{1\})=  \frac{1}{2} $  and $R(\{1,2\})=  \frac{1}{2+M} < R(\{1\})$ . Therefore, the optimal assortment is $S^*=\{1 \}$ and $R(S^*)=1$.

On the other hand, let $(S_1, S_2, \ldots, S_T)$ a feasible solution for \ref{APV}. Because, $\ell_2=T$, we have to include product $2$ in every assortment $S_t$. Adding product $1$ to an assortment only increases the revenue because $R(\{1,2\})=  \frac{1}{2+M} > R(\{2\})=0$.
Therefore, it is optimal to offer product $1$ and $2$ in every assortment $S_t$ in an optimal solution of \ref{APV}. Therefore, $S_t^*=\{1,2\}$ for all $t =1,\ldots,T$. 
Now, consider the ratio $$\frac{T R(S^*)}{\sum_{t=1}^T R(S_t^*)} = \frac{T}{T \frac{1}{2+M}}= M+2,$$
which goes to infinity as $M$ increases. Hence the gap can be arbitrarily large.
While in this pathological example, we see that enforcing products with very low price and high weight can drive the revenue very low, our numerical experiments in Section \ref{subsection:numeric} will illustrate how more common distributions of the prices and weight react to the enforcement of visibility constraints.

 % However, $\forall A \subseteq [\![1, n-1]\!], \; R(\{n\} \cup A) = \frac{1 + |A|}{1 + |A| + n^2}$, which increases with $|A|$, so the optimal solution for \ref{APV} is $S_t^* = \overline{\{n\}} = \mathcal{N} \; \forall t$. Therefore the revenue goes to zero because of the visibility constraints on product $n$, while it was close to 1 for a large value of $n$ without constraints.


\subsection{Sharing the loss}
\label{subsection:share}

% Consider a platform on which $n$ vendors sell a product each. Due to SLAs, the vendors impose some visibility constraints for their product on the platform, so that the situation can be modeled by the problem \ref{APV}. 


In this section, we explore a scenario where each product within our universe is associated with a specific vendor. As mentioned earlier, vendors can impose visibility constraints on their products within the platform. These constraints can be established through service level agreements or product sponsorships. However, it is important to note that enforcing these constraints may result in a decrease in the platform's revenue. To address this issue, the platform can implement a fee structure based on the vendors' contributions to the revenue loss. %This approach aims to establish a fair pricing policy that aligns with each vendor's impact on the revenue loss.
%We would like to a fair pricing policy where we first calculate the revenue loss incurred due to the constraints imposed by vendors. Then, we  charge each vendor a fraction of this loss, proportionate to their individual contribution towards the overall revenue reduction.
% In this section, we consider the situation when each product in our universe belongs to a vendor. As explained earlier vendors can impose some visibility constraint for their product on the platform according to a service level agreement or by sponsoring the product. 
% As seen previously, enforcing these constraints could cause a loss in the revenue of the platform. In return, the platform can charge the different vendors a fee depending on how much they contributed in reducing the revenue. In this perspective, a fair pricing policy would consist in computing the revenue loss due to the constraints, and then to charge to the vendors a fraction of this loss, which would be kind of proportional to their contribution in the loss.
Let $S^*$ be an optimal solution for \ref{Unconstrained problem} and let $(S_1^*, S_2^*,\ldots, S_T^*)$ be an optimal solution for \ref{APV}. We denote the revenue loss due to the visibility constraints as 
\begin{equation}
    \label{Delta loss}
    \Delta \coloneqq T R(S^*) - \sum_{t=1}^T R(S_t^*)
\end{equation}

\noindent
{\bf A first naive approach.} One approach is to allocate the loss based solely  on the parameters $\ell_i$. In this case, the proportion of the loss assigned to the vendor of product $i$ would be determined by $\frac{\ell_i}{\sum_{j=1}^n \ell_j}$. However, this distribution would not be equitable in the sense that  we should not impose any charges on a product that already belongs to the optimal set $S^*$, even if it satisfies the constraint $\ell_i > 0$. Moreover, this allocation fails to consider the impact of each product on the overall loss. For example, if there is a product with exceptionally high preference weight $v_i$ but a significantly lower price $p_i$, while other products have higher prices and lower preference weights, enforcing the visibility of the first product would drive revenue down to zero, whereas the others would have a lesser impact. In this scenario, the former product should be responsible for covering almost the entire revenue loss.



\noindent
{\bf Our approach.}
Let $S_1^*, \ldots, S_T^*$ be an optimal solution to \ref{APV}. First observe that  $R(S_t^*) = \frac{\sum_{i \in S_t^*} p_i v_i}{1 + \sum{i \in S_t^*} v_i}$, implies that $$R(S_t^*) = \sum_{i \in S_t^*}  (p_i - R(S_t^*))v_i$$ for every set $S_t^*$. This decomposition of the revenue gives us which products drive the revenue down (the ones with $p_i < R(S_t^*)$) and which products increase the revenue ($p_i > R(S_t^*))$. It also shows that the contribution of  product $i$ to the revenue is proportional to the difference between the price of product $i$ and the actual revenue, as well as proportional to the preference weight $v_i$. We can then rewrite the total revenue of the assortments $(S_1^*, S_2^*, \ldots, S_T^*)$ as 
$$\sum_{t=1}^T R(S_t^*) = \sum_{t=1}^T \sum_{i \in S_t^*}  (p_i - R(S_t^*)) v_i = \sum_{i=1}^n \sum_{t=1}^T \mathbb{1}(i \in S_t^*)  (p_i - R(S_t^*)) v_i.$$

Therefore, we view the contribution of product $i \in \mathcal{N}$ to the total revenue as $\sum_{t=1}^T \mathbb{1}(i \in S_t^*) (p_i - R(S_t^*)) v_i$.

%\begin{prop}
%    \label{pricing formula}

\noindent
{\bf Pricing the loss.}
 For each product $i \in \mathcal{N}$, we propose to charge its vendor the fraction of the loss corresponding to the negative contribution of this product, divided by the sum of the negative contributions of all the products:

\begin{equation} \label{pricing formula}
  \Gamma_i \coloneqq \frac{  \left[ \sum_{t=1}^T \mathbb{1}(i \in S_t^*) (p_i - R(S_t^*) v_i)\right]^-}{\sum_{j=1}^n \left[\sum_{t=1}^T \mathbb{1}(j \in S_t^*) (p_j - R(S_t^*) v_j)\right]^-} \cdot \Delta
\end{equation}
where $x^- = max(-x, 0)$ is the negative part of x. \\
%\end{prop}

Next, we discuss three important properties of this strategy:
\\

\noindent
{\bf (a) Fair distribution of fees.} First, it is worth noting that the revenue loss is exactly shared between the products whose contribution to the total revenue is negative:  $\sum_{i \in \mathcal{N}} \Gamma_i = \Delta$, and while $\Gamma_i \geq 0$ for every product $i$, we have $\Gamma_i > 0$ if and only if the product's contribution to the revenue is negative. The fee $\Gamma_i$ for each product takes into account its actual impact on the revenue: the higher the weight of the product and the lower its price is compared to the revenues of the sets to which it belongs, that is the more the product harms the revenue, the higher the fee becomes. For instance, we observe that products in $S^*$ have $\Gamma_i = 0$ and pay nothing as expected. Then, products not in $S^*$ but which are added to a $S_t^*$ to increase its revenue do not pay anything either (since they verify $p_i \geq R(S_t^*)$), which concurs with the fact that we are happy with adding these product to the assortment offered. Finally, a more subtle example: if a product $i$ decreases revenue $R(S_1)$ by a small amount but increases revenue $R(S_2)$ by a lot, and is not in the other assortments $S_t$, then we will have $\Gamma_i = 0$, concurring with a product whose impact is overall positive. Furthermore, the pricing strategy is fair in the sense that two products of same price and weight are charged the same fee if we enforce the same visibility constraint on them. Indeed, if $p_i = p_j$ and $v_i = v_j$ for two distinct products $i$ and $j$, then we have $\Gamma_i = \Gamma_j$ whenever $\ell_i = \ell_j$.
\\

\noindent
{\bf (b) Monotonicity of the fee.}  The fee $\Gamma_i$ increases (not necessarily strictly) with $\ell_i$. Indeed, we already know that the loss $\Delta$ increases with $\ell_i$. Then, since we can assume $S_t^* = \overline{L_t \cup \ldots \cup L_T}$ by Theorem \ref{Solution structure}, increasing $\ell_i$ by one unit will only change $S_{\ell_i +1}^*$ by going from $\overline{L_T \cup \ldots \cup L_{\ell_i +1}}$ to $\overline{L_T \cup \ldots \cup L_{\ell_i +1} \cup \{i\}}$, while the others optimal sets $S_t^*$ remain identical. Therefore, if $i$ was already in $S_{\ell_i +1}^*$, then $[\sum_{t=1}^T \mathbb{1}(i \in S_t^*) (p_i - R(S_t^*)) v_i]^-$ does not change, otherwise the term whose negative part we take decreases, and therefore the negative part increases. Furthermore, for every $j \neq i$, since $R(S_{\ell_i +1}^*)$ decreases when $\ell_i$ goes from $\ell_i$ to $\ell_i +1$, $\sum_{t=1}^T \mathbb{1}(j \in S_t^*) (p_j - R(S_t^*)) v_j$ increases, because new terms are added of increase only for $t = \ell_i +1$. We use the notation
$$a(\ell_i) = [\sum_{t=1}^T \mathbb{1}(i \in S_t^*) (p_i - R(S_t^*))v_i]^- \quad \quad \text{and} \quad \quad b(\ell_i) = \sum_{j \neq i} [\sum_{t=1}^T \mathbb{1}(j \in S_t^*) (p_j - R(S_t^*))v_j]^-,$$ and define $\Delta(\ell_i)$ and $\Gamma_i(\ell_i)$ as in \eqref{Delta loss} and \eqref{pricing formula} respectively, all of them considered as functions of the parameter $\ell_i$ when all the others $(l_j)_{j \neq i}$ are fixed. We have $\Gamma_i(\ell_i) = \frac{a(\ell_i)}{b(\ell_i) + a(\ell_i)} \Delta(\ell_i)$. So $$\Gamma_i(\ell_i +1) = \frac{a(\ell_i +1)}{b(\ell_i +1) + a(\ell_i +1)} \Delta(\ell_i +1) \geq \frac{a(\ell_i +1)}{b(\ell_i) + a(\ell_i +1)} \Delta(\ell_i) \geq \frac{a(\ell_i)}{b(\ell_i) + a(\ell_i)} \Delta(\ell_i) = \Gamma_i(\ell_i),$$ since $x \mapsto \frac{x}{c + x}$ is increasing for $x, c > 0$.
\\

\noindent
{\bf (c) Computational tractability.} The fees charged are easy to compute: each of them can be computed in polynomial time $\mathcal{O}(nT)$ since they only require to solve \ref{APV}. Additionally, consider the situation where a vendor is interested in knowing the fee that they needs to pay in order to increase the visibility of the product by one customer. This corresponds to the situation where we change $\ell_i$ to $\ell_i + 1$ (without changing any other ${\ell}_j$).
In this case, we only have to recompute $S_{\ell_i + 1}^*$, because the others $S_t^*$ stay the same, as explained in the previous paragraph. Therefore, for a fixed product $i$, can compute efficiently the value of the fee $\Gamma_i$ for all the values of $\ell_i \in [\![0, T]\!]$.
 
 %we can compute all the $\Gamma_i(\ell)$ for $\ell \in [\![0, T]\!]$ in time $\mathcal{O}(nT) + T \mathcal{O}(n) = \mathcal{O}(nT)$, instead of $\mathcal{O}(n T^2)$ if we recomputed all the $S_t^*$ every time we increase $\ell_i$. Thus, the computation all the $(\Gamma_i(\ell_i))_{0 \leq \ell_i \leq T}$ has the same complexity as the computation of a single $\Gamma_i(\ell_i)$ for one value of $\ell_i$. 
 
 %This is useful for a vendor who is willing to see how the fee they will have to pay evolves with the visibility constraint they ask for in the SLA for instance.

   

\subsection{Numerical Study}\label{subsection:numeric}

To illustrate our theoretical contributions, we perform some numerical experiments on generated data.

\noindent
{\bf Our experimental setup.}
We fix the number of products equal to $n = 20$, and fix the number of customers to $T = 30$. We generate the weights $v_i$, prices $p_i$ and visibility constraints $\ell_i$ from several distributions. Namely, to generate the weights $v_i$ and prices $p_i$ we used uniform and exponential distributions with varying parameters. For the visibility constraints $\ell_i$, we used a standard integer uniform $\mathcal{U}([\![0, T]\!])$, an integer uniform $\ell_i \sim \mathcal{U}([\![\frac{i}{n} T, T]\!])$ so that on average, products with lower prices receive much higher visibility constraints,\footnote{Recall that products are ordered for non-increasing price values.} and the constant $\ell_i = T,~\forall i \in \mathcal{N}$.

For each set of parameters, we compute over $1000$ samples the ratio of the optimal solution to \ref{APV} divided by $T \times R(S^*)$, that is the optimal expected revenue when we remove all visibility constraints, namely 
$$\eta \coloneqq \frac{\sum_{t=1}^T R(S_t^*)}{T \times R(S^*)}.$$
This captures the fraction of the revenue that we conserve once visibility constraints are enforced.
We compute statistics on the ratio $\eta$:  mean, standard deviation, minimum and maximum values.
\footnote{Initially, we made $T$ vary, but we observed that the results where almost independent of $T$ compared to the randomness of the generation of the parameters, so we kept only $T=30$. Indeed, varying $T$ just extends the stream of customers, and if the visibility constraints are extended by the same factor, it seems natural that the ratio of conserved revenue remains constant.}

\noindent
{\bf Overall results.}
Our results are gathered in Table \ref{fig:tables}. 
\begin{table}[h]
    \caption{For each distribution of $p_i, v_i$ and $\ell_i$, we compute the ratio $\eta$ between the optimal assortment of \ref{APV} and the optimal assortment of the unconstrained problem \ref{Unconstrained problem} $1000$ samples. We provide its mean, standard deviation, minimum and maximum values.}
    \centering
\begin{tabular}{||c|c|c|c|c|c|}
\hline
\rule{0pt}{12.5pt}{}  & {}    &  \multicolumn{4}{c|}{\begin{tabular}{l} $\eta$ (in \%) \end{tabular}} \\
\hline
{Distribution of $v_i$ and $p_i$} &{Distribution of $\ell_i$} & {Mean} & {Standard deviation} & {Min} & {Max} \\
\hline
    & $\mathcal{U}([\![0, T]\!])$ & 82.9 & 5.7 & 58.8 & 96.8  \\
$v_i \sim \mathcal{U}(0,1), \quad p_i \sim \mathcal{U}(0,1)$  
    & $\mathcal{U}([\![\frac{i}{n} T, T]\!])$ & 73.7 & 6.4 & 51.9 & 91.3  \\
    & $\ell_i = T$ & 71.6 & 6.9 & 47.4 & 91.6  \\

\hline
    & $\mathcal{U}([\![0, T]\!])$ & 71.0 & 7.3 & 43.9 & 89.2  \\
$v_i \sim \mathcal{U}(0,10), \quad p_i \sim \mathcal{U}(0,10)$  
    & $\mathcal{U}([\![\frac{i}{n} T, T]\!])$ & 60.3 & 7.1 & 39.5 & 81.6 \\
    & $\ell_i = T$ & 58.1 & 7.6 & 33.0 & 81.0  \\

\hline
    & $\mathcal{U}([\![0, T]\!])$ & 64.3 & 9.4 & 34.5 & 88.1  \\
$v_i \sim \mathcal{E}(1), \quad p_i \sim \mathcal{E}(1)$  
    & $\mathcal{U}([\![\frac{i}{n} T, T]\!])$ & 50.9 & 9.4 & 24.8 & 87.6  \\
    & $\ell_i = T$ & 48.2 & 10.1 & 20.0 & 89.1  \\

\hline
    & $\mathcal{U}([\![0, T]\!])$ & 49.4 & 11.2 & 16.1 & 79.1  \\
$v_i \sim \mathcal{E}(1/10), \quad p_i \sim \mathcal{E}(1/10)$  
    & $\mathcal{U}([\![\frac{i}{n} T, T]\!])$ & 37.2 & 9.8 & 13.4 & 63.7  \\
    & $\ell_i = T$ & 34.7 & 9.4 & 11.3 & 67.1  \\

\hline
\end{tabular}

    \label{fig:tables}
\end{table}

As expected, the ratio $\eta$ decreases when we give higher visibility $\ell_i$ to the products with smaller prices $p_i$. Furthermore, we observe that the higher the range of fluctuation of the weights $v_i$ and prices $p_i$, the lower the revenue becomes once we enforce visibility constraints. This is rather intuitive, because we saw previously that the cases in which the visibility constraints can severely harm the revenue are when there are some products with extreme value of the price or weight compared to the others. However, we observe that extreme events when the revenue of \ref{APV} becomes very low are really rare since the standard deviation is rather small compared to the range of fluctuation of $\eta$. Based on our results, it seems that in most cases, as long as there are not abnormally extreme values of the weights and prices, we can always hope to keep a fraction close to half of the revenue after enforcing visibility constraints.\\


\noindent
{\bf Revenue vs market shares.} To assess the influence of the visibility constraints on the market share, we compute the optimal value of \ref{APV} for all visibility constraints $(\ell_i)_{1 \leq i \leq n}$ equal to a same value $\ell$ going from $0$ to $T$ (for $v_i \sim \mathcal{U}(0,1), p_i \sim \mathcal{U}(0,1)$). We then compute the expected sales associated to this expected revenue, and plot in Figure \ref{fig:graph1} the ratio of revenue decrease versus the ratio of sales increase by dividing them by their unconstrained value, for normalization.
% Figure environment removed

It is interesting to see that, while the revenue decreases as we enforce more visibility constraints on products, on the contrary, the sales increase. Indeed, based on the structure of the optimal nested solution we identified, $S_t^* = \overline{L_t \cup \ldots \cup L_T}$, we see that the sizes of the $S_t^*$ increase when the $\ell_i$'s increase. Furthermore, if we take a look at the sales optimization problem $$\max_{S_1, \ldots, S_T} \sum_{t=1}^T \frac{V(S_t)}{1 + V(S_t)},$$ that corresponds to $p_i = 1 \; \forall i$, we can see that the more products we add to each $S_t$, the higher the sales since the function $x \mapsto \frac{x}{1+x}$ is increasing with respect to $x$. \\
Therefore, we can see that while visibility constraints decrease the revenue, they increase the sales, which can be interesting if the objective of the platform is to capture market shares.\\

\noindent
{\bf Marginal study of one product.}
Finally, it is interesting to illustrate the marginal effect of the visibility constraints on one product. For this purpose, we take an instance of our generated data with $v_i \sim \mathcal{E}(1), \; p_i \sim \mathcal{E}(1), \; \ell_i \sim \mathcal{U}([\![0, T]\!])$. Then, we select a particular $i \in \mathcal{N}$, and vary $\ell_i$ from $0$ to $T$ while the others $\ell_j$ stay fixed at their generated value $\ell_j$. In Figure \ref{fig:graph2}, we plot the variation of the revenue loss with respect to $\ell_i$

% Figure environment removed


We remark that the loss is convex with respect to $\ell_i$. Indeed, let   $(S_t^*)_{1 \leq t \leq T} = (\overline{L_T \cup \ldots \cup L_t})_{1 \leq t \leq T}$ be the nested solution for constraint $\ell_i$, and $\ell_i \mapsto \Delta(\ell_i)$ the revenue loss function, as defined in \eqref{Delta loss}, and considered as a function of constraint $\ell_i$ when all the other visibility constraints $(\ell_j)_{j \neq i}$ are fixed. We have 
% \begin{align*}
%     \Delta(\ell_i +1) - \Delta(\ell_i) - (\Delta(\ell_i) - \Delta(\ell_i -1)) &= \overline{R}(S_t^* \cup \{i\}) + \overline{R}(S_{t+1}^* \cup \{i\}) - 2 (\overline{R}(S_t^* \cup \{i\}) + \overline{R}(S_{t+1}^*)) + \overline{R}(S_t^*) + \overline{R}(S_{t+1}^*) \\
%     &= \overline{R}(S_{t+1}^* \cup \{i\}) + \overline{R}(S_t^*) - \overline{R}(S_t^* \cup \{i\}) - \overline{R}(S_{t+1}^*) \geq 0
% \end{align*}
$$\Delta(\ell_i +1) - \Delta(\ell_i) - (\Delta(\ell_i) - \Delta(\ell_i -1))  = \overline{R}(S_{t+1}^* \cup \{i\}) + \overline{R}(S_t^*) - \overline{R}(S_t^* \cup \{i\}) - \overline{R}(S_{t+1}^*) \geq 0,$$  where the inequality holds by supermodularity and $S_t^* \subseteq S_{t+1}^*$, which shows $\ell_i \mapsto \Delta(\ell_i)$ is convex.  


We can then, in the case where all the revenue from the sales of a product comes to its vendor, compute the expected revenue of product $i$ versus the fee its vendor will pay as the visibility constraint $\ell_i$ increases. The expected revenue of product $i$ is given by: $\sum_{t=1}^T \mathbb{1}(i \in S_t^*) p_i \frac{v_i}{1 + V(S_t^*)}$, while the fee charged $\Gamma_i$ is defined in  \eqref{pricing formula}. This is depicted in Figure \ref{fig:graph3}.

% Figure environment removed

For this particular product, we can see that enforcing a visibility constraint $\ell_i$ is profitable until $\ell_i = 17$ (with respect to $T=30$), and becomes non profitable after. The profit associated with this constraint is the difference between the revenue and the fee, and is maximized for the value $\ell_i = 7$ (see Figure \ref{fig:graph3}). This is an example of a product that does not belong to the optimal unconstrained set $S^*$ (since the revenue is $0$ when $\ell_i = 0$, it means the product was not included), but for which adding a visibility constraint allows to boost sales and the revenue of the vendor, while paying a small fee to compensate the platform's revenue loss.

One interesting observation here is that the fee paid by the vendor of product $i$ (0.8 when $\ell_i = T = 30$ on Graph \ref{fig:graph3}) is higher than the loss caused by the introduction of $\ell_i = T = 30$ while the others products were enforced previously (loss of 0.5 as we can read on Graph \ref{fig:graph2}). However, this can be explained: as the expanded revenue function is supermodular and decreasing, the revenue decreases most when the first products are enforced, and then decreases less. Therefore, enforcing visibility constraints on product $i$ decreases less the revenue if we already have some visibility constraints on the other products than if $i$ is the first product on which the constraints are enforced. However, our pricing policy $\Gamma_i$ does not take into account such sequential order of arrival, and prices each product globally, so that two products with the same price and weight pay the same fee for identical visibility constraints.


 \section{Extension of the Model}
    \label{sec:extensions}


A natural extension to \ref{APV} is to add cardinality constraints on the offered assortments, in particular for each $t \in [\![1, T]\!]$, the assortment $S_t$ can contain a maximum of $k$ products. This constraint becomes particularly significant in situations where there is limited space available for product display.
We refer to the assortment optimization problem, with both visibility and cardinality constraints, as {\em Assortment Problem with Visibility and Cardinality constraints (APVC)}.  The introduction of cardinality constraints significantly complicates the problem and makes it strongly NP-hard.

\begin{theorem} \label{NP-hardness}
    \ref{APVC} is strongly NP-hard, even when all prices $p_i$ are equal.
\end{theorem}


As a result, we focus on developing approximation algorithms for \ref{APVC}. Since the problem, is strongly NP-hard even in the case of uniform prices. 
We decided to focus on the version of \ref{APVC} with uniform prices. It corresponds to a situation of sales maximization, which is interesting when the goal of the platform is to capture market shares.  Our main contribution for   is to derive a polynomial time algorithm providing with a $0.61$- approximation guarantee when prices are uniform. 

\begin{theorem} \label{constant approximation}
    We obtain a $\frac{1}{\phi}$-approximation algorithm to \ref{APVC} when all prices are uniform, where $\phi \coloneqq \frac{1 + \sqrt{5}}{2}$. ($\frac{1}{\phi} \gtrsim 0.61$).
\end{theorem}


%The proof of this theorem relies  on constructing assortments $S_1,\ldots,S_T$ in polynomial time such that the sum of the weights of the products in each assortment are close enough one from another. This way, we ensure that the value of each of them is either close to the one in an optimal set, or close to a concave upper bound of the optimum. 
For completeness, we provide the proofs of Theorem \ref{NP-hardness} and Theorem \ref{constant approximation} in Appendix \ref{apx2}. 



















%     \section{Extension of the Model}
%     \label{sec:extensions}


% A natural extension to \ref{APV} is to add cardinality constraints on the offered assortments. This means that for each $t \in [\![1, T]\!]$, the assortment $S_t$ can contain a maximum of $k$ products. This constraint becomes particularly significant in situations where there is limited space available for product display.
% We refer to the assortment optimization problem for $T$ customers, with both visibility and cardinality constraints, as {\em Assortment Problem with Visbility and Cardinality constraints (APVC)}. It can be formulated as follows:

% \begin{equation}
% \label{APVC}
% \tag{\sf{APVC}}
% \begin{aligned}
%  \max_{ S_1, \ldots, S_T \subseteq \mathcal{N}}  \; \;  \left\{\sum_{t=1}^T  \frac{\sum_{i \in S_t} p_i v_i}{1 + \sum_{i \in S_t} v_i}  \quad s.t. \quad \sum_{t=1}^T  \mathbb{1}(i \in S_t) \geq \ell_i \;\; \forall i \in \mathcal{N}, \quad |S_t| \leq k \;\; \forall t \in [\![1, T]\!] \right\}.
% \end{aligned}
% \end{equation}

% First, we observe that \ref{APVC} is feasible if and only if $\sum_{i=1}^n \ell_i \leq kT$. This is a necessary condition to pack the products in $T$ sets of cardinality at most $k$. To prove the sufficiency, assuming $\sum_{i=1}^n \ell_i \leq kT$, we can construct a feasible solution as follows. Initially, all assortments $S_t$ are empty. We proceed by considering each product individually. For every product $i$, we iteratively add the required occurrences, $\ell_i$, to different assortments $S_t$, ensuring that the difference in size between any two distinct sets does not exceed $1$. Since each product appears at most $T$ times and $\sum_{i=1}^n \ell_i \leq kT$, there will consistently be available space in various assortments to accommodate all products.



% The introduction of cardinality constraints significantly complicates the problem. In our preliminary results, we provide evidence that incorporating these constraints renders the problem strongly NP-hard.

% \begin{theorem} \label{NP-hardness}
%     \ref{APVC} is strongly NP-hard, even when all prices $p_i$ are equal.
% \end{theorem}


% As a result, we focus on developing approximation algorithms for \ref{APVC}. Since the problem, is strongly NP-hard even in the case of uniform prices. 
% We decide in the following to focus on the version of \ref{APVC} with uniform prices. It corresponds to a situation of sales maximization, which is interesting when the goal of the platform is to capture market shares. We name the problem {\em Sales Problem with Visibility and Cardinality constraints (SPVC)} and it can be formulated as
% \begin{equation}
% \label{SPVC}
% \tag{\sf{SPVC}} 
% \begin{aligned}
%  \max_{ S_1, \ldots, S_T \subseteq \mathcal{N}}  \; \;  \left\{\sum_{t=1}^T  \frac{\sum_{i \in S_t} v_i}{1 + \sum_{i \in S_t} v_i}  \quad s.t. \quad \sum_{t=1}^T  \mathbb{1}(i \in S_t) \geq \ell_i \;\; \forall i \in \mathcal{N}, \quad |S_t| \leq k \;\; \forall t \in [\![1, T]\!] \right\}.
% \end{aligned}
% \end{equation}
% First, we observe that the objective function for customer $t$ is of the form $x \mapsto \frac{x}{1 + x}$ where $x$ is the weight of the assortment offered to customer $t$. This is a concave function. Hence, intuitively, an optimal solution for \ref{SPVC} should try to share the weights of products in the most balanced way among the  $T$ assortments.
% Indeed, by concavity, we know that   $$\sum_{t=1}^T \frac{V(S_t)}{1 + V(S_t)} \leq T \frac{\frac{1}{T} \sum_{t=1}^T V(S_t)}{1 + \frac{1}{T} \sum_{t=1}^T V(S_t)},$$ where the right term corresponds to an ideal solution in which the sum of the attractiveness of the products is the same in each of the $T$ assortments.
%  However, for general values of preference weights, finding such an ideal split is not possible.

% Our main contribution for  \ref{SPVC} is to derive a polynomial time algorithm providing with a $1/6$- approximation guarantee. In particular, we have the following theorem.

% \begin{theorem} \label{constant approximation}
%     We obtain a $\frac{1}{6}$-approximation algorithm to \ref{SPVC}. 
% \end{theorem}


% The proof of this theorem relies  on constructing assortments $S_1,\ldots,S_T$ in polynomial time such that the sum of the weights of the products in each assortment are close enough one from another. This way, we ensure that the value of each of them is either close to the one in an optimal set, or close to a concave upper bound of the optimum. Due to space limitations, we provide our algorithm and the corresponding proof of correctness in in Appendix \ref{apx2}. 




    \section{Conclusions and future directions}
    \label{sec:conclusions}
    
%noindent
%{\bf Conclusion.}
In this paper, we introduced the problem of assortment optimization with visibility constraints \ref{APV}, motivated by situations in e-commerce and online advertising in which a platform aims at ensuring a minimal exposure for some or all products. We prove that this problem can be solved in polynomial time, and devise an efficient algorithm to compute an optimal solution. Our algorithm leverages the supermodularity of an altered version of the expected revenue function, that allows us to identify the nested structure of an optimal solution. In addition, we evaluate the revenue loss caused by the visibility constraints enforced, and propose a fair pricing strategy to charge each vendor a fee proportional to the contribution of its product to the revenue loss. Finally, we consider an extension of the problem with cardinality constraints on the assortments offered. In our preliminary results, we prove that the problem becomes strongly NP-hard even under uniform prices, and we propose a polynomial time constant approximation in this case.


\vspace{2mm}
\noindent
{\bf Future directions.}
A promising future research direction involves developing an approximation algorithm for the assortment optimization problem with visibility and cardinality constraints, considering general prices. Although Theorem \ref{NP-hardness} rules out  the existence of an FPTAS, even under uniform prices, efforts can be directed towards enhancing the constant approximation described in Theorem \ref{constant approximation} or even devising a PTAS for the problem.
Furthermore, exploring the assortment problem with visibility using alternative choice models such as the Markov Chain choice model or Nested Logit represents another fruitful avenue for investigation. %Finally, an interesting direction to consider an online version of the problem, in which assortments have to be offered sequentially, as customers arrive.


    %%%%%%%%%%%%%%%%%%%%%%%%%%%%%%%%%%%%%%%%%%%%%%

%
% ---- Bibliography ----
%
% BibTeX users should specify bibliography style 'splncs04'.
% References will then be sorted and formatted in the correct style.
%
%\bibliographystyle{splncs04}
%\bibliography{biblio}


%\bibliography{biblio}

%  {\small
  \printbibliography
 % }


\newpage   
    \appendix
    \section{Appendix: Assortment Optimization under MNL model}
    \label{apx1}

In this appendix, we present basic results about Assortment Optimization under MNL and characterize the structure of a revenue maximizing assortment. 

Consider the problem \ref{Unconstrained problem}. The following lemma is a standard result in assortment optimization under the MNL model and is stated in various forms in \citeauthor{talluri2004revenue} \cite{talluri2004revenue}, Gallego et al.  \cite{Gallego2004ManagingFP} and \citeauthor{Rusmevichientong2012RobustAO} \cite{Rusmevichientong2012RobustAO}, as well as \citeauthor{Gallego2019RevenueMA} \cite{Gallego2019RevenueMA}. It shows that an optimal assortment for \ref{Unconstrained problem} is revenue ordered, and we recall the proof for the sake of completeness.


\begin{lemma}{}
    \label{Unconstrained solution}
    Let $R^* = \max_{S \subseteq \mathcal{N}} R(S)$ be the optimal value of problem \ref{Unconstrained problem}. There exists a revenue ordered optimal solution to \ref{Unconstrained problem}, given by $S^* = \{i \in \mathcal{N}, p_i \geq R^* \}$.
\end{lemma}

\proof
    By definition of $R^*$, $\forall S \subseteq \mathcal{N}$, $R^* \geq \frac{\sum_{i \in S} p_i v_i}{1 + \sum_{i \in S} v_i}$, with equality if and only if S is an optimal assortment. This is equivalent to $R^* (1 + \sum_{i \in S} v_i) \geq \sum_{i \in S} p_i v_i, \quad \iff R^* \geq \sum_{i \in S} (p_i - R^*) v_i$. This last right term is maximal when $S$ contains all the products $i$ such that $p_i > R^*$, possibly some products whose price $p_i = R^*$, and no products whose price is strictly inferior to $R^*$. Therefore, the first inequality $R^* \geq R(S)$ is an equality as soon as $S = \{i \in \mathcal{N}, p_i \geq R^* \}$. 
    \hfill \qedsymbol
%%\end{proof} 


In the next lemma, we show that adding a product $k$ to an assortment $S$ increases the revenue of this assortment if and only if $p_k \geq R(S)$, and removing a product $k$ from an assortment $S$ decreases the revenue of this assortment if and only if $p_k \geq R(S)$. 



\begin{lemma}
    \label{Revenue variations}
    For any $S\subseteq {\cal N}$ and $k \in {\cal N} \setminus S$, we have
    $$ R(S \cup \{k\}) \geq R(S) \iff p_k \geq R(S) \iff p_k \geq R(S \cup \{k\}).$$
\end{lemma}

\proof
    We show that $\forall S \subseteq \mathcal{N}, \forall k \notin S$, $R(S \cup \{k\})$ is a barycenter of $R(S)$ and $p_k$.
    We have $R(S \cup \{k\}) = \frac{\sum_{i \in S \cup \{k\}} p_i v_i}{1 + V(S \cup \{k\})} 
    = \frac{\sum_{i \in S} p_i v_i}{1 + V(S) + v_k} + \frac{v_k}{1 + V(S) + v_k} p_k 
    = \frac{1 + V(S)}{1 + V(S) + v_k} \frac{\sum_{i \in S} p_i v_i}{1 + V(S)} + \frac{v_k}{1 + V(S) + v_k} p_k
    = \frac{1 + V(S)}{1 + V(S) + v_k} R(S) + \frac{v_k}{1 + V(S) + v_k} p_k
    = \alpha R(S) + (1- \alpha) p_k$ with $\alpha = \frac{1 + V(S)}{1 + V(S) + v_k} \in (0,1)$. \\
    It follows that $R(S \cup \{k\}) \geq R(S) \iff p_k \geq R(S)$, and in this case, $R(S \cup \{k\}) \in [R(S), p_k]$. And reciprocally, if $R(S \cup \{k\}) \leq p_k$, then it implies that $R(S) \leq p_k$, which achieves to prove the equivalences. 
    \hfill \qedsymbol
%%\end{proof}






\section{Appendix: Proof of Theorem \ref{NP-hardness}}
\label{apxhard}

The {\em Assortment Problem with Visibility and Cardinality constraints (APVC)} can be formulated as follows:

\begin{equation}
\label{APVC}
\tag{\sf{APVC}}
\begin{aligned}
 \max_{ S_1, \ldots, S_T \subseteq \mathcal{N}}  \; \;  \left\{\sum_{t=1}^T  \frac{\sum_{i \in S_t} p_i v_i}{1 + \sum_{i \in S_t} v_i}  \quad s.t. \quad \sum_{t=1}^T  \mathbb{1}(i \in S_t) \geq \ell_i \;\; \forall i \in \mathcal{N}, \quad |S_t| \leq k \;\; \forall t \in [\![1, T]\!] \right\}.
\end{aligned}
\end{equation}


We show that \ref{APVC} can be reduced to the k-partitioning problem, which can be stated as follows: given a multiset (several elements can have the same value) $S$ of $k \times T$ positive integers, decide whether it can be partitioned into $T$ subsets of $k$ elements that all have the same sum. This problem is known to be strongly NP-hard for $k \geq 3$. This case $k=3$ is called the $3$-partition problem.\\
We now take $T \in \mathbb{N}^*$ and consider an instance $S = \{v_i, i \in [\![1, 3T]\!] \}$ of the 3-partition problem. We consider the problem 
$$\max_{\substack{(S_t)_{1 \leq t \leq T} \text{ partition of } S \\ |S_t| = 3 \quad \forall t}} \quad \sum_{t=1}^T \frac{V(S_t)}{1 + V(S_t)},$$ \\
which can be seen as a specific case of \ref{APVC}, where $k=3$, $n = 3T$, $p_i = 1$, $\ell_i = 1$ $\forall i \in \mathcal{N}$. Here, we do not make a difference between, $i$ and $v_i$. We recall that $V(S_t) := \sum_{i \in S_t} v_i$.
The objective function can be rewritten $T - \sum_{t=1}^T \frac{1}{1 + V(S_t)}$, and its maximization is equivalent to minimizing $\sum_{t=1}^T \frac{1}{1 + V(S_t)}$. However, by Cauchy-Schwarz, we know that $$\left(\sum_{t=1}^T \frac{1}{1 + V(S_t)}\right)\left(\sum_{t=1}^T 1 + V(S_t)\right) \geq T^2.$$ Thus $$\sum_{t=1}^T \frac{1}{1 + V(S_t)} \geq \frac{T^2}{T + V(S)},$$ whose right-hand term is independent of the partition $(S_t)$ chosen. We also know that this inequality is an equality if and only if $V(S_1) = V(S_2) = \ldots = V(S_T)$. Therefore, the value of 
$$\min \left\{ \sum_{t=1}^T \frac{1}{1 + V(S_t)} \quad s.t. \quad (S_t)_{1 \leq t \leq T} \text{ partition of } S,  \quad |S_t| = 3 \quad \forall 1 \leq t \leq T \right\}$$ is equal to $\frac{T^2}{T + V(S)}$ if and only if there exists a partition $(S_t)_{1 \leq t \leq T}$ of $S$ with $|S_t |=3$ such that $V(S_1) = V(S_t) \; \forall t=1,\ldots,T$, which is exactly the statement of the 3-partition problem. %Thus, if \ref{APVC} was solvable in polynomial time, this special case would be too, and verifying whether the lower bound $\frac{T^2}{T + V(S)}$ is attained would allow us to conclude on the $3$-partition problem in polynomial time. 
This proves that \ref{APVC} is strongly NP-hard. \hfill \qedsymbol








\section{Appendix: Proof of Theorem \ref{constant approximation}}
\label{apx2}


When the prices are uniform, we  refer to our  problem as {\em Sales Problem with Visibility and Cardinality constraints {\sf (SPVC)}} and it can be formulated as
\begin{equation}
\label{SPVC}
\tag{\sf{SPVC}} 
\begin{aligned}
 \max_{ S_1, \ldots, S_T \subseteq \mathcal{N}}  \; \;  \left\{\sum_{t=1}^T  \frac{\sum_{i \in S_t} v_i}{1 + \sum_{i \in S_t} v_i}  \quad s.t. \quad \sum_{t=1}^T  \mathbb{1}(i \in S_t) \geq \ell_i \;\; \forall i \in \mathcal{N}, \quad |S_t| \leq k \;\; \forall t \in [\![1, T]\!] \right\}.
\end{aligned}
\end{equation}
% First, we observe that the objective function for customer $t$ is of the form $x \mapsto \frac{x}{1 + x}$ where $x$ is the weight of the assortment offered to customer $t$. This is a concave function. Hence, intuitively, an optimal solution for \ref{SPVC} should try to share the weights of products in the most balanced way among the  $T$ assortments.
% Indeed, by concavity, we know that   $$\sum_{t=1}^T \frac{V(S_t)}{1 + V(S_t)} \leq T \frac{\frac{1}{T} \sum_{t=1}^T V(S_t)}{1 + \frac{1}{T} \sum_{t=1}^T V(S_t)},$$ where the right term corresponds to an ideal solution in which the sum of the attractiveness of the products is the same in each of the $T$ assortments.
%  However, for general values of preference weights, finding such an ideal split is not possible.

\noindent
{\bf Our constant approximation solution.} 
    Consider an instance $n, k, T, (v_i)_{1 \leq i \leq n}, \; (\ell_i)_{1 \leq i\leq n}$ of \ref{SPVC} that is feasible (i.e., such that $\sum_{i=1}^n \ell_i \leq kT$), without loss of generality we assume that the weights of the products are ordered by decreasing value: $v_1 \geq v_2 \geq \ldots \geq v_n$. 
  We define the solution $(\hat{S_t})_{1 \leq t \leq T}$ in the following way. 
 %We construct an enumeration $\varphi: [\![1, kT]\!] \to [\![1, n]\!]$, which gives the $kT$ elements (including duplicates of some products $i$ when they are used more than one time) that will be used in each set $\hat{S_t}$. 
 % Being considered as a multiset, $Im(\varphi)$ initially contains each product $i$ for a number of$ \; \ell_i$ times. Then, as long as there is space available, we add the highest weight products which are not present $T$ times yet. We organize the enumeration $\varphi$ such that the first images are all the products $1$, then all the $2$, \ldots~and finally all the $n$. Then, we define the solution $\forall t \in [\![1, T]\!], \; \hat{S_t} \coloneqq \{ \varphi(t + u T),~0 \leq u \leq k-1 \}.$ In other words, we assign $1$ every $T$ products of the enumeration to each set $\hat{S_t}$.
We construct an enumeration $\varphi: [\![1, kT]\!] \to [\![1, n]\!]$, which will be used to define each set $\hat{S_t}$ as follows. Consider all products where each product $i$ is duplicated $\ell_i$ times. Then, as long as there is space available, i.e., the total number of duplicates is less than $kT$, we add a copy of the highest weight product which is not present $T$ times yet. The resulting set of copies has exactly $kT$ copies, where each product is duplicated at most $T$ times. We order these $kT$ copies from the highest preference weights to the lowest. This gives $[\![1, kT]\!]$ the domain of the function $\varphi$.  We define the image of each copy by the function $\phi$ as the product corresponding to this copy. Then, we define the solution $\forall t \in [\![1, T]\!], \; \hat{S_t} \coloneqq \{ \varphi(t + u T),~0 \leq u \leq k-1 \}.$ In other words, we assign $1$ every $T$ products of the enumeration to each set $\hat{S_t}$.








    Formally, the construction follows Algorithm \ref{alg1}.

    \begin{algorithm}[H] % enter the algorithm environment
    \caption{Construction of approximate solution for \ref{SPVC}} % give the algorithm a caption
    \label{alg1} % and a label for \ref{} commands later in the document
    \begin{algorithmic} % enter the algorithmic environment
    \Require a feasible instance of \ref{SPVC}
    \State $r \leftarrow kT - \sum_{i=1}^n \ell_i$
    \State $c = 1$
    \For{$i$ going from $1$ to $n$}
        \For{$j$ going from $c$ to $c + \ell_i -1$}
            \State $\varphi(j) \leftarrow i$
        \EndFor
        \State $occ_i \leftarrow \ell_i$
        \State $c \leftarrow c + \ell_i$
        \While{$r > 0$ and $occ_i < T$}
            \State $\varphi(c) \leftarrow i$
            \State $c \leftarrow c + 1$
            \State $r \leftarrow r - 1$
        \EndWhile
    \EndFor
    \For{$t$ going from $1$ to $T$}
        \State $\hat{S_t} \leftarrow \{ \varphi(t + u T), \quad 0 \leq u \leq k-1 \}$
    \EndFor
    \State Return $(\hat{S_t})_{1 \leq t \leq T}$
    \end{algorithmic}
\end{algorithm}


We observe that the collection of sets $(\hat{S_t})_{1 \leq t \leq T}$ returned by Algorithm \ref{alg1} is constructed in polynomial time. Next, we present four technical lemmas, that we use later to finally prove Theorem~\ref{constant approximation}.

\begin{lemma}
    \label{feasibility}
    The solution $(\hat{S}_t)_{1 \leq t \leq T}$ constructed in Algorithm \ref{alg1} is feasible for \ref{SPVC}.
\end{lemma}

\proof
    First, each set $\hat{S}_t$ contains $k$ elements which means all the cardinality constraints are verified. Then, each element is present at least $\ell_i$ times by construction of $\varphi$. Finally, no same product $i$ is present more than once in each $\hat{S_t}$. Indeed, copies of the same product $i$ occupy $T$ or less contiguous images in the enumeration $\varphi$, so by definition no set $\hat{S}_t$ can get twice a product $i$ before all sets have it, and there are $T$ sets for at most $T$ copies of $i$. 
    \hfill \qedsymbol
%%\end{proof}






\begin{lemma}
    \label{ordering of winded solution}
For all $1 \leq t < t' \leq T$, we have  $ V(\hat{S}_{t'}) \leq V(\hat{S}_t) \leq V(\hat{S}_{t'}) + v_{\varphi{(t)}}.$
\end{lemma} 

\proof
We know   by construction that $(v_{\varphi(j)})_{1 \leq j \leq kT}$ are non-increasing, hence
$$V(\hat{S}_t') = \sum_{u=0}^{k-1} v_{\varphi(t' + uT)} \leq \sum_{u=0}^{k-1} v_{\varphi(t + uT)} = V(\hat{S}_t).$$ 
    Furthermore, $$V(\hat{S}_t) = v_{\varphi(t)} + \sum_{u=1}^{k-1} v_{\varphi(t + uT)} \leq v_{\varphi(t)} + \sum_{u=1}^{k-1} v_{\varphi(t' + (u-1)T)} \leq v_{\varphi(t)} + \sum_{u=0}^{k-1} v_{\varphi(t' + uT)} = v_{\varphi(t)} + V(\hat{S}_t'),$$ where we used the fact that $t + uT > t' + (u-1)T$ as $t' - t < T$. 
    \hfill \qedsymbol
%\end{proof}


\begin{lemma}
    \label{2 items permutation}
    Let $S_1, S_2$ be sets containing some elements from $\mathcal{N}$, assume $V(S_1) \geq V(S_2)$ wlog. Then $R(S_1) + R(S_2)$ can be increased strictly by permuting an element $i_1$ from $S_1$ and an element $i_2$ from $S_2$ if and only if: $0 < v_{i_1} - v_{i_2} < V(S_1) - V(S_2)$.
\end{lemma}

\proof
    By concavity of $x \mapsto \frac{x}{1 + x}$ over positive numbers, we know that $\forall x, a, b \geq 0$, $$\frac{x + a + b}{1 + x + a + b} + \frac{x}{1 + x} \leq \frac{x + a}{1 + x + a} + \frac{x + b}{1 + x + b}.$$ So transferring weight from one set to the other increases the sum of their revenues  if and only if $|V(S_1) - V(S_2)|$ decreases after the permutation. Let $V_1=V(S_1)$ and $V_2=V(S_2)$. Assume we permute $i_1 \in S_1$ and $i_2 \in S_2$.  Let $S_1'$ and $S_2'$ be the resulting sets. Let $V'_1=V(S'_1)$ and $V'_2=V(S'_2)$. 
     We have $V_1' = V_1 - v_{i_1} + v_{i_2}$ and $V_2' = V_2 - v_{i_2} + v_{i_1}$, so $V_1' - V_2' = V_1 - V_2 - 2(v_{i_1} - v_{i_2})$. Therefore, $V_1' - V_2' < V_1 - V_2 \iff v_{i_1} > v_{i_2}$, and $V_1' - V_2' > -(V_1 - V_2) \iff v_{i_1} - v_{i_2} < V_1 - V_2$. Thus, $|V_1' - V_2'| < |V_1 - V_2| \iff 0 < v_{i_1} - v_{i_2} < V(S_1) - V(S_2)$, hence the result.
    \hfill \qedsymbol
%\end{proof}

\begin{lemma}
    \label{weight to revenue}
    Let $S_1, S_2$ be sets containing some elements from $\mathcal{N}$, such that $V(S_1) \leq \alpha V(S_2)$ for some $\alpha > 1$. Then, $R(S_1) \leq \alpha R(S_2)$.
\end{lemma}

\proof
    $R(S_1) = \frac{V(S_1)}{1 + V(S_1)} \leq \frac{\alpha V(S_2)}{1 + \alpha V(S_2)} \leq \frac{\alpha V(S_1)}{1 + V(S_1)} = \alpha R(S_2)$, where the first inequality comes from the fact that $x \mapsto \frac{x}{1 + x}$ is non-decreasing, and the second is because $\alpha > 1$.
    \hfill \qedsymbol
%\end{proof}


\noindent {\em Proof of Theorem \ref{constant approximation}.}
Here, we prove that the solution $(\hat{S}_t)_{1 \leq t \leq T}$ constructed by  Algorithm \ref{alg1} achieves a constant approximation ratio of at least $\frac{1}{\phi} = \frac{2}{1 + \sqrt{5}}$ for \ref{SPVC}. Note that  we already proved in Lemma \ref{feasibility}   that this solution is feasible.

First, let  $OPT$ be the optimal value of \ref{SPVC}. We define $\widehat{OPT}$ as the optimal value of the following problem:
\begin{equation}
\label{SPVC hat}
\tag{\sf{$\widehat{SPVC}$}}
\widehat{OPT}=  \max_{S_1, \ldots, S_T \text{ partition of } [\![1, kT]\!], \; |S_t| = k}   \; \;  \sum_{t=1}^T R_{\varphi}(S_t)
\end{equation}
where we define $$R_{\varphi}(S_t) \coloneqq R(\{\varphi(i): \; \; i \in S_t \}), \quad \forall S_t \subseteq [\![1, kT]\!].$$
In addition, we define $$V_{\varphi}(S_t) \coloneqq V(\{ \varphi(i): \;\; i \in S_t \}), \quad \forall S_t \subseteq [\![1, kT]\!].$$
We have 
\begin{equation*}
    \begin{aligned}
        OPT &=  \max_{ S_1, \ldots, S_T \text{ sets } \subseteq \mathcal{N}}  \; \;  \left\{\sum_{t=1}^T R(S_t)  \quad s.t. \quad \sum_{t=1}^T  \mathbb{1}(i \in S_t) \geq \ell_i \;\; \forall i \in \mathcal{N}, \quad |S_t| = k \;\; \forall t \in [\![1, T]\!] \right\} \\
        &\leq \max_{ S_1, \ldots, S_T \text{ supersets } \subseteq \mathcal{N}}  \; \;  \left\{\sum_{t=1}^T R(S_t)  \quad s.t. \quad \sum_{t=1}^T  \mathbb{1}(i \in S_t) \geq \ell_i \;\; \forall i \in \mathcal{N}, \quad |S_t| = k \;\; \forall t \in [\![1, T]\!] \right\} \\
        &= \max_{S_1, \ldots, S_T \text{ partition of } [\![1, kT]\!] } \left\{ \sum_{t=1}^T R_{\varphi}(S_t) \quad s.t. \quad |S_t| = k \right\} = \widehat{OPT},
    \end{aligned}
\end{equation*}
where the first inequality comes from the fact that in supersets, we allow each element $i$ to appear several times in the same $S_t$, which is not possible in sets.  And the equality that follows is due to the fact that if there was a product $i$ that appeared a number of times strictly superior to $\ell_i$, while a product $j$ such that $v_j > v_i$ appeared strictly less than $T$ times, we could improve the objective function by replacing $i$ by $j$ once. Therefore, the only elements used in the maximum over supersets will always contain exactly the products $\{ \varphi(i): \; \; i \in [\![1, kT]\!] \}$.

We now define $\widetilde{OPT}$ as the optimal value of the following problem
\begin{equation}
    \label{SPVC tilde}
    \tag{\sf{$\widetilde{SPVC}$}}
 \widetilde{OPT} =   \max_{(s_t)_{1 \leq t \leq T} \geq 0} \quad \left\{ \sum_{t=1}^T \tilde{R}(v_{\varphi(t)} + s_t)  \quad s.t. \quad \sum_{t=1}^T s_t = \sum_{i=T+1}^{kT} v_{\varphi(i)} \right\},
\end{equation}
where we used the notation $\tilde{R}(V) \coloneqq \frac{V}{1 + V}$. In particular, the revenue function here is expressed with respect to the weight of a set and not the actual set itself, such that $R(S_t) = \tilde{R}(V_{\varphi}(S_t))$ where we recall that $V_{\varphi}(S_t) \coloneqq \sum_{i \in S_t} v_{\varphi(i)}$ for $i \in [\![1, kT]\!]$.

We show that $\widetilde{OPT} \geq \widehat{OPT}$.

Indeed, take an optimal solution $S_1, \ldots, S_t$ of \ref{SPVC hat}. After permuting the sets such that $V_{\varphi}(S_1) \geq V_{\varphi}(S_2) \geq \ldots \geq V_{\varphi}(S_T)$, we can assume $V_{\varphi}(S_t) \geq v_{\varphi(t)} \; \forall t$. Indeed, assume there exists $u$ such that $V_{\varphi}(S_u) <  v_{\varphi(u)}$, and let $u$ be the smallest index verifying this. Then, we have $V_{\varphi}(S_t) <  v_{\varphi(u)} \; \forall t \geq u$, so the elements $[\![1, u]\!]$ can only be in the sets $(S_t)_{1 \leq t \leq u-1}$. There are $u$ of them for only $u-1$ sets, therefore, one of them contains two elements of $(v_{\varphi(t)})_{1 \leq t \leq u}$. Let $S_{q}$ be such a set, and $t_1, t_2$ be such two elements. Then we have 
$$V_{\varphi}(S_q) \geq v_{\varphi(t_1)} + v_{\varphi(t_2)} \geq  v_{\varphi(t_1)} +  v_{\varphi(u)} > v_{\varphi(t_1)} + V_{\varphi}(S_u),$$
and by Lemma \ref{2 items permutation}, we can improve strictly $R_{\varphi}(S_q) + R_{\varphi}(S_u)$ by exchanging $t_1$ from $S_q$ with any element from $S_u$.
Therefore, 
\begin{equation*}
    \begin{aligned}
        \widehat{OPT} &= \max_{S_1, \ldots, S_T \text{ partition of } [\![1, kT]\!], \; |S_t| = k} \left\{ \sum_{t=1}^T R_{\varphi}(S_t) \quad s.t. \quad \forall t \in [\![1, T]\!], \; V_{\varphi}(S_t) \geq v_{\varphi(t)} \right\} \\
        &\leq \max_{s_1, \ldots, s_T \geq 0} \left\{ \sum_{t=1}^T \tilde{R}(s_t) \quad s.t. \quad  \forall t \in [\![1, T]\!], \; s_t \geq v_{\varphi(t)}, \quad \sum_{t=1}^T s_t = \sum_{i=1}^{kT} v_{\varphi(i)}  \right\}
        &= \widetilde{OPT}.
    \end{aligned}
\end{equation*}
Indeed, the inequality comes from the fact that we relax the constraint of $k$-partitioning of the items into a less restrictive constraint of weight partitioning.

%Finally, we prove that $\sum_{t=1}^T R(\hat{S_t}) \geq \frac{1}{2} \tilde{OPT}$, which will achieve to prove the theorem, since $\tilde{OPT} \geq \hat{OPT} \geq OPT$.

Now, let us  characterize the structure of an optimal solution of \ref{SPVC tilde}. Note that this problem can be interpreted as a problem of weight partition in $T$ bins represented by $V_1, \ldots, V_T$, where we enforce $V_t \geq v_{\varphi(t)}$.
Since the function $\tilde{R}$ is concave, the optimal solution can be constructed by starting with  weight $v_{\varphi(t)}$ in each bin $V_t$. Then, we add weight in the bin with smallest initial weight, until it reaches the weight of the second lowest bin. We then add weight to the two smallest bins, until it reaches the third one, and we continue until there is no more weight to share.
Therefore, the solution will be  of the following form: there exists $u \in [\![0, T-1 
 ]\!]$ such that $\forall t \in [\![1, u]\!], \; V_t = v_{\varphi(t)}$ and $\forall t \in [\![u+1, T]\!], \; V_t = v_0$ where $v_0 = \frac{1}{T-u} \sum_{i=u+1}^{kT} v_{\varphi(i)} \in [\![ v_{\varphi(u+1)}, v_{\varphi(u)} ]\!]$, thus 
$$\widetilde{OPT} = \sum_{t=1}^u \frac{v_{\varphi(t)}}{1 + v_{\varphi(t)}} + \sum_{t=u+1}^T \frac{v_0}{1 + v_0}.$$
Indeed, if there was $V_t > v_{\varphi(t)}$ and $V_{t'} < v_{\varphi(t)}$ for $t < t'$, then by concavity we would have $$\tilde{R}(v_{\varphi(t)}) + \tilde{R}(V_{t'} + (V_t - v_{\varphi(t)})) > \tilde{R}(V_t) + \tilde{R}(V_{t'}),$$ contradicting the optimality of the solution containing $V_t$ and $V_{t'}$.

Now, let us define $\alpha \coloneqq \frac{\sum_{t=1}^u R(\hat{S_t})}{\sum_{t=1}^T R(\hat{S_t})} = \frac{\sum_{t=1}^u \tilde{R}(\hat{V_t})}{\sum_{t=1}^T \tilde{R}(\hat{V_t})}$, where $\hat{V_t} \coloneqq V(\hat{S_t})$. Note that $\alpha$ represents the fraction of the revenue of our approximated solution $(\hat{S_t})_{1 \leq t \leq T}$ contained in the first $u$ sets.
In addition, we write $(\tilde{V_t})_{1 \leq t \leq T}$ an optimal solution of \ref{SPVC hat}.

We observe that to go from $(\hat{V_t})_{1 \leq t \leq T}$ to $(\tilde{V_t})_{1 \leq t \leq T}$, we have to transfer weight from the first $u$ sets to the last $T-u$ sets, since $\hat{V_t} \geq \tilde{V_t}$ for $t \leq u$ and $\hat{V_t} \leq \tilde{V_t}$ for $t \geq u+1$.
Therefore, we already have 
$$\widetilde{OPT} = \sum_{t=1}^u \tilde{R}(\tilde{V_t}) + \sum_{t=u+1}^T \tilde{R}(\tilde{V_t}) \leq \sum_{t=1}^u \tilde{R}(\hat{V_t}) + \sum_{t=u+1}^T \tilde{R}(v_{\varphi(u)}) \leq \sum_{t=1}^u \tilde{R}(\hat{V_t}) + (T-u) \tilde{R}(\hat{V_u})$$
and $R(\hat{V_u}) \leq \frac{1}{u} \sum_{t=1}^u R(\hat{S_t})$, hence
$$\frac{\tilde{OPT}}{\sum_{t=1}^T R(\hat{S_t})} \leq \alpha + \frac{T-u}{u} \alpha = \frac{T}{u} \alpha.$$
Then,  by Lemma \ref{ordering of winded solution}, we know that $\forall t \leq u, \; \hat{V_t} \leq v_{\varphi(t)} + \hat{V_T}$. Therefore, the weight that is being transferred from the first $u$ sets to the last $T-u$ ones represents at most $u \times v_T$. This shows that for each $t \geq u+1,$ we can write $\tilde{V_t} = (1 + m_t) \hat{V_t}$, with $m_t \geq 0$ such that $m_t \leq m_{t+1}$ (since $\hat{V_t} \geq \hat{V_{t+1}}$ and $\tilde{V_t} = \tilde{V_{t+1}}$), and with $\sum_{t=u+1}^T m_t \leq u$. Consequently,
\begin{equation*}
    \begin{aligned}
    \widetilde{OPT} &= \sum_{t=1}^u \tilde{R}(\tilde{V_t}) + \sum_{t=u+1}^T \tilde{R}(\tilde{V_t}) \\
    &\leq \sum_{t=1}^u \tilde{R}(\hat{V_t}) + \sum_{t=u+1}^T \tilde{R}((1+m_t) \hat{V_t}) \\
    &\leq \sum_{t=1}^u \tilde{R}(\hat{V_t}) + \sum_{t=u+1}^T (1 + m_t) \tilde{R}(\hat{V_t}) \quad \text{by Lemma \ref{weight to revenue}} \\
    &\leq \sum_{t=1}^u \tilde{R}(\hat{V_t}) + \max_{0 \leq m_{u+1} \leq  \ldots \leq m_T} \left\{ \sum_{t=u+1}^T (1 + m_t) \tilde{R}(\hat{V_t}) \quad s.t. \quad \sum_{t=u+1}^T m_t \leq u \right\}. \\
    \end{aligned}
\end{equation*}
Since the $(\tilde{R}(\tilde{V_t}))$ are decreasing, the maximum of the right term is obtained for $m_t = \frac{u}{T-u} \; \forall t \geq u+1$.
Therefore, 
$$\tilde{OPT} \leq \sum_{t=1}^u R(\tilde{S_t}) + \sum_{t=u+1}^T (1 + \frac{u}{T-u}) R(\tilde{S_t}),$$
which gives $$\frac{\tilde{OPT}}{\sum_{t=1}^T R(\hat{S_t})} \leq \alpha + \frac{T}{T-u}(1 - \alpha) = \frac{T}{T-u} - \alpha \frac{u}{T-u}.$$
This last function is affine in $\alpha$ and decreasing with $\alpha$. Furthermore, we know that $\alpha \geq \frac{u}{T}$ since the $(R(\hat{S_t}))_{1 \leq t \leq T}$ are decreasing, we have $\alpha \geq \frac{u}{T}$. Therefore, the maximum of the affine function is attained in $\alpha = \frac{u}{T}$, and consequently
$$\frac{\tilde{OPT}}{\sum_{t=1}^T R(\hat{S_t})} \leq 1 + \frac{u}{T}.$$
Combining the two inequalities on the approximation ratio, we have $$\frac{\tilde{OPT}}{\sum_{t=1}^T R(\hat{S_t})} \leq \min(\frac{T}{u}, 1 + \frac{u}{T}).$$

The function $u \mapsto \frac{T}{u}$ is decreasing from $T$ to $1$, while the function $u \mapsto 1 + \frac{u}{T}$ is increasing from $1 + \frac{1}{T}$ to $2$, therefore the maximum value of $\min(\frac{T}{u}, 1 + \frac{u}{T})$ is at their intersection point, the only positive solution of $\frac{T}{u} = 1 + \frac{u}{T}$.
This happens for $u = \frac{\sqrt{5} - 1}{2} T$, and the associated value if $\frac{1 + \sqrt{5}}{2} \coloneqq \phi$, where $\phi$ is the golden ratio.
Therefore, we have $\frac{\tilde{OPT}}{\sum_{t=1}^T R(\hat{S_t})} \leq \phi.$

This proves that $\sum_{t=1}^T R(\hat{S_t}) \geq \frac{1}{\phi} \tilde{OPT} \geq \frac{1}{\phi} OPT$, therefore $(\hat{S_t})_{1 \leq t \leq T}$ is a $\frac{1}{\phi}$ approximation of \ref{SPVC}, where $\frac{1}{\phi} = \frac{2}{1 + \sqrt{5}} \gtrsim 0.61$.

\hfill \qedsymbol





\end{document}
