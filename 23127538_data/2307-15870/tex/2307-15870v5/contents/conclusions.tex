In this paper, we have reviewed the distinct properties of SFL and proposed a novel Semi-supervised SFL system, termed \method, to perform model training on unlabeled and non-IID client data by incorporating clustering regularization.
We have theoretically and experimentally investigated the impact of global updating frequency on model convergence.
Then, we have developed a control algorithm for dynamically adjusting the global updating frequency, so as to mitigate the training inconsistency and enhance training performance.
Extensive experiments have demonstrated that \method provides a 3.8× speed-up in training time and reduces the communication cost by about 73.6\% while reaching the target accuracy, and achieves up to 5.8\% improvement in accuracy under non-IID scenarios compared to the baselines.

% we propose Semi-SFL with \textit{Clustering Regularization}, a novel system for training on unlabeled and non-IID data in the SFL world.
% Clustering regularization aim to enhances the model's performance under data heterogeneity while ensuring computational and communication efficiency, 
% the clustering of teacher features provides a comprehensive view of the feature distribution, which is robust to data skewness.
% Under non-IID scenarios, our system regularizes the student bottom models based on the clustering of teacher features, so as to enhance the generalization ability of student bottom models.
% Furthermore, we adapt the \InvReg frequency in order to favor both the \srvUp and \locReg stages, guided by our theoretical analysis on model convergence.
% Our experiments on various datasets demonstrate the improved convergence speed and accuracy of our system compared to existing methods on the SVHN, CIFAR-10, and IMAGE-100 datasets. 
% This work contributes to the development of federated learning with few labels and provides a valuable avenue for future research in this field, particularly when limited resources are available.