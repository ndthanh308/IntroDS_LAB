\section{Coloured regularity graphs}\label{sec:CRG}

In this section we describe the concept of a coloured regularity graph, which was first introduced by Marchant and Thomason~\cite{marchant2010extremal}. The idea of this structure is that it encodes a set of rules for how to edit a graph to make it satisfy some hereditary property.

A \emph{coloured regularity graph (CRG)} is a complete graph $K$ on $k$ vertices, in which each vertex is coloured either black or white, and each edge is coloured black, white or grey. Formally, we say that the vertex set $V(K)$ is partitioned into two (possible empty) sets $\text{VB}(K)$ and $\text{VW}(K)$, of black and white vertices respectively, and that the edge set $E(K)$ is partitioned into three (possibly empty) sets $\text{EB}(K)$, $\text{EW}(K)$, and $\text{EG}(K)$ of black, white, and grey edges respectively. For ease of notation, when the CRG $K$ is clear from the context, we write $\text{VB}$ to mean $\text{VB}(K)$, and analogously write $\text{VW}$,~$\text{EB}$,~$\text{EW}$,~and~$\text{EG}$. We say a CRG $K'$ is a \emph{sub-CRG} of a CRG $K$ if $K'$ can be obtained by deleting vertices of $K$.

We say a graph $H$ \emph{embeds} in a CRG $K$, and we write $H \mapsto K$, if there exists some function $\phi \colon V(H) \longrightarrow V(K)$ which, for all $u, v \in V(K)$, satisfies the following conditions.
%
\begin{enumerate}[label=(\roman*)]
    \item If $uv \in E(H)$, we have either $\phi(u) = \phi(v) \in \text{VB}$, or $\phi(u)\phi(v) \in \text{EB} \cup \text{EG}$.
    \item If $uv \not\in E(H)$, we have either $\phi(u) = \phi(v) \in \text{VW}$, or $\phi(u)\phi(v) \in \text{EW} \cup \text{EG}$.
\end{enumerate}
%
For our purposes, due to a key property of CRGs, it is most interesting to consider all those CRGs into which $H$ does not embed. Specifically, suppose we find a CRG $K$ into which a graph $H$ does not embed. Then any graph $H'$ which contains $H$ as an induced subgraph will also not embed in $K$. So, if there is a graph $G$ which embeds into $K$, then crucially, $G$ cannot contain $H$ as an induced subgraph, implying that $G \in \text{Forb}(H)$. Thus we find a very clear relationship between the class of CRGs into which a graph $H$ does not embed, and the family $\text{Forb}(H)$.

Recall that $\F(\cH)$ is the family of all forbidden subgraphs of $\cH$. Then for any hereditary property $\cH$, we let
%
\begin{equation*}
    \K(\cH) = \lcb K \colon H \not\mapsto K  \text{ for all } H \in \F(\cH) \rcb .
\end{equation*}
%
Let $K \in \K(\cH)$ for some $\cH$, and suppose $G$ does not belong to $\cH$. We can view $K$ as a set of rules by which to edit $G$, in order for it to belong to $\cH$. We begin by partitioning $V(G)$ into $k$ parts $V_1, \dots, V_k$ such that each part $V_i$ corresponds to a distinct vertex $v_i$ of $K$. The optimal sizes of these parts are yet to be determined, but do not affect the method of editing. Indeed, we add or remove edges according the following rules.
%
 \begin{enumerate}[label=(\roman*),parsep=0pt]
    \item If $v_i \in \text{VB}$, we add all edges with both endpoints in $V_i$.
    \item If $v_i \in \text{VW}$, we remove all edges with both endpoints in $V_i$.
    \item If $v_iv_j \in \text{EB}$, we add all edges with one endpoint in $V_i$ and the other in $V_j$.
    \item If $v_iv_j \in \text{EW}$, we remove all edges with one endpoint in $V_i$ and the other in $V_j$.
\end{enumerate}
%
Let $G'$ be the graph obtained from $G$ by carrying out these edits. Then $G'$ embeds into $K$, and so by our observations above we have $G' \in \cH$. Note that CRGs corresponding to the strategies outlined in \cref{sec:introduction} would belong to $\K(\Forb(C_h))$. We will see that there will be some such CRG such that the edit distance can be determined by applying these rules to edit the graph $G(n,p)$ and then counting the expected number of edge changes which would be required. 

\subsection{Measuring the edits defined by a CRG}

Suppose we are given a CRG $K \in \K(\cH)$, and we would like to count the expected proportion of edges of $G(n,p)$ which would be changed with respect to these rules, for some fixed $p$. We will define a quadratic program $g_K(p)$ which counts exactly this quantity. In order to define $g_K(p)$, we first define a matrix $\boldM_K(p)$. We label the vertices of $K$ by $v_1, \dots v_k$, and let $\boldM_K(p)$ be a $k \times k$ matrix whose entries are given by
%
\begin{equation*}
    \lsb \boldM_K(p) \rsb_{ij} = \begin{cases}
                                p & \text{ if either } i = j \text{ and } v_i \in \text{VW}, \text { or } i \neq j \text{ and } v_iv_j \in \text{EW},\\
                                1-p & \text{ if either } i = j \text{ and } v_i \in \text{VB}, \text { or } i \neq j \text{ and } v_iv_j \in \text{EB},\\
                                0 & \text{ if } i \neq j \text{ and } v_iv_j \in \text{EG}.
                            \end{cases}
\end{equation*}
%
Then we define the quadratic program
%
\begin{equation}\label{eq:g_k_p}
    g_K(p) =    \begin{cases}
                    \min & \boldx^T \boldM_K(p) \boldx\\
                    \text{s.t.} & \boldx \cdot \mathbf{1} = 1\\
                         & \boldx \geq \mathbf{0}.
                \end{cases}
\end{equation}
%
Note that the vector $\boldx$ which has exactly one entry equal to $1$ and all other entries equal to $0$ is a feasible solution to this program. Thus, there is some optimal vector $\boldx^*$ which attains the minimum in the program above, and so $g_K(p)$ has a solution for every CRG $K$. Furthermore, for any given CRG $K$, we can easily find the value of $g_K(p)$ using the method of Lagrange multipliers.

The vector $\boldx^*$ depends on the matrix $\boldM_K(p)$, which captures information about the adjacencies in $K$. Recall that every CRG $K$ defines a partition of the random graph $G(n,p)$, where the vertices of $K$ represent parts in this partition. The vector $\boldx$ assigns a weight to every vertex of $K$. In particular, for any $v \in K$, the weight $\boldx(v)$ corresponds to the proportion of vertices of $G(n,p)$ which lie in the part corresponding to $v$, and the optimal weight vector $\boldx^*$ gives the assignment of vertices of $G(n,p)$ to parts in a way which minimises the expected proportion of edge changes required. Thus, the function $g_K(p)$ measures exactly the expected proportion of edge changes of $G(n,p)$ which the CRG $K$ defines.

The following result of Alon and Stav \cite{alon2008furthest} suggests that for any hereditary property $\cH$, we can use the function $g_K(p)$ to determine the edit distance function.
%
\begin{theorem}[Alon and Stav \cite{alon2008furthest}]\label{thm:alon_stav_edit_distance_equals_inf}
Let $\cH$ be a hereditary property. Then for all $p \in [0,1]$,
\[
    \text{\emph{ed}}_\cH(p) = \inf_{K \in \K(\cH)} g_K(p).
\]
\end{theorem}
%
Marchant and Thomason \cite{marchant2010extremal} later showed that there is in fact a CRG which attains the infimum in \cref{thm:alon_stav_edit_distance_equals_inf}, that is, they showed the following.
%
\begin{theorem}[Marchant and Thomason \cite{marchant2010extremal}]\label{thm:marchant_thomason_edit_distance_equals_min}
Let $\cH$ be a hereditary property. Then for all $p \in [0,1]$,
\[
    \text{\emph{ed}}_\cH(p) = \min_{K \in \K(\cH)} g_K(p).
\]
\end{theorem}
%
This is a key result in this area, since for any hereditary property $\cH$, the problem of determining the edit distance function is reduced to instead finding the CRG $K \in \K(\cH)$ which attains the minimum in \cref{thm:marchant_thomason_edit_distance_equals_min}. Indeed, since the strategies outlined in \cref{sec:introduction} correspond to certain CRGs $K \in \K(\Forb(C_h))$, the bounds would give an upper bound for $\ed_{\Forb(C_h)}(p)$.

An implication of \cref{thm:marchant_thomason_edit_distance_equals_min} is that for any $K \in \K(\cH)$, the function $g_K(p)$ provides an upper bound to the edit distance function. Thus, rather than examining all CRGs in $\K(\cH)$, we begin by only examining those CRGs which have all grey edges, and use these to obtain an upper bound to $\text{ed}_\cH(p)$.

We denote by $K(r,s)$ the CRG on $r+s$ vertices, which has $r$ white vertices, $s$ black vertices, and all its edges are grey. For any hereditary property $\cH$, we define the \emph{clique spectrum} to be
%
\begin{equation*}
    \Gamma(\cH) = \lcb (r,s) \in \Z_{\geq 0}^2 \colon H \not\mapsto K(r,s) \text{ for all } H \in \F(\cH)\rcb.
\end{equation*}
%
The important property of $\Gamma(\cH)$ which we will be using is its monotonicity. That is, if $(r,s) \in \Gamma(\cH)$, then for all $0 \leq r' \leq r$, $0 \leq s' \leq s$, we have $(r',s') \in \Gamma(\cH)$. This follows immediately from the definition, and gives rise to important elements of $\Gamma(\cH)$ known as \emph{extreme points}, which are pairs $(r,s) \in \Gamma(\cH)$ such that $(r+1, s), (r, s+1) \not\in \Gamma(\cH)$. We denote by $\Gamma^*(\cH)$ the set of extreme points of $\Gamma(\cH)$.

We state the following useful lemma, which allows us to observe another useful property of these grey edge CRGs. Let $K$ be a CRG. We say a sub-CRG $K'$ of $K$ is a \emph{component} if every edge leaving $K'$ is grey, that is, if for all $v \in V(K')$ and all $w \in V(K \setminus K')$, we have that $vw \in \text{EG}(K)$. So, every CRG has a `decomposition' into components, that is, a partition of the vertex set such that all edges leaving the sub-CRG induced on any part in this partition are grey. Then we can state the following lemma, a result of work by Martin \cite{martin2013edit}.
%
\begin{lemma}[Martin \cite{martin2013edit}]\label{lem:components}
Let $K$ be a $CRG$ with components $K^{(1)}, \dots, K^{(\ell)}$. Then
\[
    (g_K(p))^{-1} = \sum_{i=1}^{\ell} (g_{K^{(i)}}(p))^{-1}.
\]
\end{lemma}


We can now state the following useful result, which suggests that for any grey-edge CRG $K(r,s)$, it is sufficient to know $r$ and $s$ in order to calculate the value of $g_{K(r,s)}(p)$.
%
\begin{lemma}[Martin \cite{martin2013edit}]\label{lemma:calulate_g_of_K_r_s}
%
\begin{equation*}
    g_{K(r,s)}(p) = \frac{p(1-p)}{r(1-p)+sp}.
\end{equation*}
%
\end{lemma}

For any pair $(r, s) \in \Gamma^*(\cH)$, we have $K(r,s) \in \K(\cH)$. By minimising over all the grey edge CRGs in $\K(\cH)$, we obtain an upper bound for $\text{ed}_\cH(p)$. Formally, we define
%
\begin{equation*}
    \gamma_\cH(p) = \min \lcb g_{K(r,s)}(p) \colon (r,s) \in \Gamma(\cH) \rcb = \min \lcb \frac{p(1-p)}{r(1-p)+sp} \colon (r,s) \in \Gamma(\cH) \rcb.
\end{equation*}
%
Then $\gamma_\cH(p) \geq \text{ed}_\cH(p)$. Furthermore, suppose that $(r,s) \in \Gamma^*(\cH)$. Then for all  $0 \leq r' \leq r$, $0 \leq s' \leq s$ we have $g_{K(r,s)}(p) \leq g_{K(r',s')}(p)$. Thus when calculating $\gamma_\cH(p)$, it suffices to consider only those pairs $(r,s)$ which are extreme points of the clique spectrum. 

The advantage of this is that the value of $\gamma_\cH(p)$ is determinable for any hereditary property, and thus this upper bound is easier to calculate than directly minimising the function $g_K(p)$ over all $K \in K(\cH)$. Through the course of this paper, we may refer to a CRG which `attains $\gamma_\cH(p)$' or `attains $\text{ed}_\cH(p)$' for some value of $p$, by which we mean a CRG $K$ for which $g_K(p) = \gamma_\cH(p)$, or $g_K(p) = \text{ed}_\cH(p)$, respectively. We will also define the following special type of CRG. This was originally introduced by Peck \cite{peck2013edit}. 
%
\begin{definition}\label{def:candidate_CRG}
For a hereditary property $\cH$, we say that a CRG $K$ is a \emph{candidate CRG for $\cH$} if $K \in \K(\cH)$ and $g_K(p) < \gamma_\cH(p)$.
\end{definition}
%
If the hereditary property $\cH$ is clear from the context, we omit the phrase `for $\cH$' from \cref{def:candidate_CRG}.


\subsection{The $p$-core CRGs and symmetrisation}

As we have seen previously, the edit distance function $\text{ed}_\cH(p)$ can be determined by finding the CRG $K \in \K(\cH)$ which minimises $g_K(p)$. We say a CRG $K$ is \emph{$p$-core} if $g_K(p)<g_{K'}(p)$ for any sub-CRG $K'$ of $K$. Since we know by Theorem~\ref{thm:marchant_thomason_edit_distance_equals_min} that there exists some CRG which minimises $g_K(p)$, this definition immediately implies that there exists a $p$-core CRG which minimises $g_K(p)$. Marchant and Thomason~\cite{marchant2010extremal} identified the following useful classification of $p$-core CRGs.
%
\begin{theorem}[Marchant-Thomason \cite{marchant2010extremal}]\label{thm:characterisation_of_p_core_crgs}
Let $K$ be a $p$-core CRG. Then the following holds.
\begin{enumerate}
    \item If $p = 1/2$, then all edges of $K$ are grey.
    \item If $p < 1/2$, then $\text{EB} = \emptyset$ and there are no white edges incident to white vertices.
    \item If $p > 1/2$, then $\text{EW} = \emptyset$ and there are no black edges incident to black vertices.
\end{enumerate}
\end{theorem}
%
We consider again the quadratic program $g_K(p)$ defined in~\eqref{eq:g_k_p}. Marchant and Thomason \cite{marchant2010extremal} showed that if $K$ is a $p$-core CRG, then the optimal vector for this quadratic program is in fact unique, and moreover that this optimal vector $\boldx$ contains no zero entries.

Now the vector $\boldx$ assigns to each vertex $v \in V(K)$ a weight $\boldx(v)$. We write $d_B(v)$ for the weighted degree of $v$ along black edges, that is, $d_B(V) := \sum_{u \in V(K) : uv \in EB} \boldx(u)$, and define $d_W(v)$ and $d_G(v)$ analogously for white and grey edges.  Martin \cite{martin2013edit} found the following bounds on the quantity $d_G(v)$ using symmetrisation techniques.
%
\begin{lemma}[Martin \cite{martin2013edit}]\label{lem:symmetrisation}
Let $p \in (0,1/2]$ and let $K$ be a $p$-core CRG with optimal weight function $\boldx$. Then $\boldx(v) = g_K(p)/p$ for all $v \in \text{\emph{VW}}(K)$. Moreover, for all $v \in \text{\emph{VB}}(K)$, we have
%
\begin{equation*}
   d_G(v) = \frac{p-g_K(p)}{p}+\frac{1-2p}{p}\boldx(v).
\end{equation*}
%
\end{lemma}
%
\cref{lem:symmetrisation} can be thought of as a symmetrisation lemma,  and tells us that the weight is distributed evenly among all vertices in $\VW(K)$ by the vector $\boldx$. This gives a way of determining the value of $\boldx$ at any vertex. We can use this to state the following useful lemma.
%
\begin{lemma}[Martin \cite{martin2013edit}]\label{lem:upper_bound_on_black_vertex_weight}
Let $p \in (0,1/2]$ and let $K$ be a $p$-core CRG with optimal weight function $\boldx$. Then for all $v \in \text{\emph{VB}}(K)$, we have $\boldx(v) \leq g_K(p)/(1-p)$.
\end{lemma}