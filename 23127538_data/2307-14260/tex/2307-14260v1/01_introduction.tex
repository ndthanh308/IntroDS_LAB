\section{Introduction}\label{sec:introduction}

For a fixed constant $p \in (0,1)$, what is the expected number of edges (expressed as a proportion of $\binom{n}{2}$) that must be added or removed from the binomial random graph $G(n, p)$ to eliminate all induced copies of $C_h$, the cycle of length $h$? To answer this question, we might consider the following strategies.
\begin{enumerate}
    \item We could arbitrarily partition the vertex set of $G(n, p)$ into $\lceil h/2 \rceil - 1$ sets, then add all edges which have both endvertices within the same one of these sets. By the Pigeonhole Principle any induced copy of $C_h$ within the resulting graph must have three vertices within the same set, meaning that these vertices form a triangle, giving a contradiction for $h \geq 4$. By convexity, the expected number of edge-additions needed for this is minimised when the sizes of the sets of the partition are made as equal as possible, and this expected minimum number of edits is essentially $\frac{1-p}{\lceil h/2 \rceil - 1} \binom{n}{2}$. 
    \item We could arbitrarily partition the vertex set of $G(n, p)$ into sets $W_1, B_1, \dots, B_{(\lceil h/3 \rceil - 1)}$, then delete all edges with both endvertices in $W_1$ whilst, for each $i$, adding all edges with both endvertices in $B_i$. Again the resulting graph cannot contain an induced copy of $C_h$. Indeed, each set $B_i$ must contain either one or two consecutive vertices of such a copy, and since there are fewer than $h/3$ sets $B_i$ it follows that $W_1$ must also contain two adjacent vertices of the cycle, a contradiction. Solving the appropriate quadratic program shows that the number of edges edited is minimised by taking $W_1$ to have size $(1- p) n$ and the sizes of the sets $B_i$ to be $\frac{p}{\ceil{h/3}-1}n$, and hence that the minimum number of edits is essentially $\frac{p(1-p)}{1 + (\lceil\frac{h}{3}\rceil -2) p}\binom{n}{2}$.
    \item Finally, if $h$ is odd, then we can arbitrarily partition the vertex set of $G(n, p)$ into two sets $A$ and $B$, and delete all edges with both endvertices in the same part. The resulting graph is bipartite and so contains no copies of $C_h$ (induced or otherwise), and the number of edges deleted is essentially~$\frac{p}{2}\binom{n}{2}$.
\end{enumerate}
These three strategies are illustrated in Figure~\ref{fig:Gamma_Forb_C_h}, where white vertices indicate sets in which all edges should be removed, and black vertices represent sets where all non-edges should be added as edges.

From these strategies we obtain an upper bound on the expected number of edge-edits needed to remove all induced copies of $C_h$ from $G(n, p)$ of essentially
\begin{align*}
    \begin{cases}
  \min \lcb \frac{1-p}{\lceil h/2 \rceil - 1} \binom{n}{2}, \frac{p(1-p)}{1 + (\lceil\frac{h}{3}\rceil -2) p}\binom{n}{2} \rcb & \mbox{if $h$ is even, and} \\
  \min \lcb \frac{1-p}{\lceil h/2 \rceil - 1} \binom{n}{2}, \frac{p(1-p)}{1 + (\lceil\frac{h}{3}\rceil -2) p}\binom{n}{2}, \frac{p}{2}\binom{n}{2}\rcb & \mbox{if $h$ is odd}.
\end{cases}
\end{align*}
%
Martin~\cite{martin2013edit} showed that for all $h \in \{3, \ldots, 9\}$ and $p \in [0,1]$, one of these strategies  is always `best possible', that is, the upper bound obtained above is matched by the lower bound and is in fact the correct answer. Here, the value of $p$ determines which function minimises the terms above, and therefore different strategies will be best possible depending on the value of $p$. Martin also showed that when $h = 10$, these strategies are best possible provided $p \in [1/7,1]$. Peck \cite{peck2013edit} showed that this is best possible for all odd $h \geq 3$ and $p \in [0,1]$. Peck also showed that if $h$ is even, then these strategies are best possible for $p \in [1/\ceil{\frac{h}{3}}, 1]$.

In this paper, we add to this picture, showing that in fact, for $h =10$, these strategies are also best possible for all $p \in [0,1/7]$. We also show that for $h=12$, these strategies are best possible for $p \in [0,1/4]$. Thus, we complete the picture for $h \in \{10, 12\}$. Furthermore, we show that for all even $h \geq 12$, there is a constant $p_0=p_0(h)$ such that for all $p \in [p_0, 1/\ceil{h/3}]$, these strategies are best possible. We remark that the constant $p_0$ is a function of $h$, and is significant because when $h=10$, the function gives exactly the value $1/7$. We give more formal statements of all of the above in \cref{subsec:main_results}. In fact, we prove a more general result where rather than eliminating induced cycles of some fixed length $h$, we are eliminating induced copies of the $t$-th power of this cycle. As we will see more formally in \cref{thm:edit_estimated_by_random_graph}, a result of Balogh and Martin shows that these results are more significant, because they apply to all graphs with density $p$, rather than just the random graph. We will formulate all these in terms of a quantity known as the \emph{edit distance function}.


The \emph{edit distance} is a natural metric between two graphs which counts the proportion of changes which must be made to the edge set of one in order to obtain the other. Essentially, it is a measure of how similar two graphs are. As a concept, this was first formalised by Sanfeliu and Fu \cite{SanfeliuEdit} in 1983, as a tool for pattern recognition. Here, one proposed use was as a method of using a computer to recognise lower case letters drawn by hand, and computer science has since seen many uses of this. The particular formulation we are interested in here concerns the furthest graph from some hereditary property of graphs, and was introduced independently by Alon and Stav~\cite{alon2008furthest} and Axenovich, K\'ezdy and Martin~\cite{axenovich2008editing}. 

Formally, we define the edit distance between two graphs $G$ and $G'$ on the same vertex set to be the size of the symmetric difference between their edge sets as a fraction of the total number of possible edges, that is, if $\size{V(G)}= \size{V(G')} = n$, then
%
\begin{equation*}
    \dist(G,G') = \frac{\size{E(G) \Delta E(G')}}{\binom{n}{2}}.
\end{equation*}
%
We say $\cH$ is a \emph{hereditary property of graphs} if $\cH$ is a class of graphs which is closed under taking isomorphisms and induced subgraphs. $\Forb(H)$ represents the class of all graphs $G$ which do not have $H$ as an induced subgraph. Hereditary properties can be classified in terms of their forbidden subgraphs, that is, for any hereditary property $\cH$, there is a family $\F(\cH)$ of forbidden graphs, that is,
%
\begin{equation*}
    \cH = \bigcap_{H \in \F(\cH)} \Forb(H).
\end{equation*}
%
We say a hereditary property $\cH$ is \emph{trivial} if there is an $n_0 \in \N$ such that for all $n \geq n_0$, there is no $n$-vertex graph contained in $\cH$. In other words, a class is trivial if and only if it is finite. Otherwise, we say $\cH$ is \emph{non-trivial}. 
%For instance, an example of a non-trivial hereditary property is $\Forb(C_h)$, the class of graphs with no $C_h$ as an induced subgraph.

We can extend the notion of distance between graphs to define the distance between a graph $G$ and a hereditary property $\cH$, which we define to be the minimum distance from $G$ to some graph $G'$ in $\cH$ on the same vertex set, that is,
%
\begin{equation*}
    \dist(G, \cH) = \min \lcb \dist(G, G') \colon G' \in \cH, \size{V(G)}= \size{V(G')} \rcb .
\end{equation*}
%
Problems in the area have focused on finding the maximum distance of a graph $G$ on $n$ vertices from a hereditary property $\cH$, and it was this which led to the conception of the \emph{edit distance function} by Balogh and Martin \cite{balogh2008edit}. For any $p \in [0,1]$, we define
%
\begin{equation}\label{eq:edit_distance_function}
    \ed_\cH(p) = \lim_{n \to \infty} \max \lcb \dist(G, \cH) \colon \size{V(G)} = n, \size{E(G)} = \floor{p \binom{n}{2}} \rcb,
\end{equation}
%
if this limit exists. So in other words, for any $p$, the edit distance function for a hereditary property $\cH$ tells us the furthest distance a graph of density $p$ can be from belonging to $\cH$. Balogh and Martin \cite{balogh2008edit} later generalised a result of Alon and Stav \cite{alon2008furthest} to show that the limit in \eqref{eq:edit_distance_function} does exist for all non-trivial hereditary properties $\cH$. In addition to this, they showed the following result.

% Since the quantity $\floor{p \binom{n}{2}}$ approximates the number of edges of the random graph $G(n,p)$, the definition of the edit distance function suggests an inherent connection to random graph $G(n,p)$, and in fact the following result was established by Balogh and Martin \cite{balogh2008edit}.
%
\begin{theorem}[Balogh-Martin \cite{balogh2008edit}]\label{thm:edit_estimated_by_random_graph}
\begin{equation*}
    \ed_\cH(p) = \lim_{n \to \infty} \E \lsb \dist(G(n,p), \cH) \rsb .
\end{equation*}
\end{theorem}
%
That is, asymptotically, for any $p$ and hereditary property $\cH$, we can use the random graph $G(n,p)$ to estimate the edit distance function. Hence, the strategies outlined earlier for the random graph also give upper bounds on the edit distance function from $\Forb(C_h)$. Balogh and Martin \cite{balogh2008edit} also showed that the edit distance function is continuous and concave down. Methods to determine the edit distance function $\ed_\cH(p)$ make implicit use of these properties, as well as \cref{thm:edit_estimated_by_random_graph}, as we will see in the following sections. 

The edit distance function has been studied for a range of hereditary properties of the form $\cH = \Forb(H)$, such as for $H= K_r$ (see \cite{martin2013edit}), $H=K_{s,t}$ (see for example \cite{Martin_McKay_bipartite}) and more recently when $H=G(n',p')$, for some $n'$ and $p'$ which are fixed with respect to $n$ and $p$ (see \cite{martin_riasanovsky_2022}). Axenovich and Martin \cite{Axenovich_Martin_multicolor} also extended this theory into edge-coloured graphs and directed graphs, and an interesting open question raised by Martin \cite{martin2016edit} is whether this notion of the edit distance function can be extended into the setting of hypergraphs.

\subsection{Main results}\label{subsec:main_results}

In this paper, we study  the edit distance function when $\cH = \Forb(C_h^t)$ for some $h,t \in \N$. 
Here, $C_h$ is a cycle on $h$ vertices and $C_h^t$ is defined to be the graph on the same vertex set as $C_h$ such that there is an edge between two vertices of $C_h^t$ if and only if these vertices were at distance at most $t$ in $C_h$. In particular, when $t=1$, this is just the cycle on $h$ vertices. Thus, we aim to determine $\ed_{\Forb(C_h^t)}(p)$, where $\Forb(C_h^t)$ is the class of graphs which contain no $C_h^t$ as an induced subgraph. As a very natural property to consider, this question has received a lot of interest in the past. Marchant and Thomason \cite{marchant2010extremal} determined $\ed_{\Forb(C_h)}(p)$ for all $p \in [0,1]$ when $h=4$. Martin \cite{martin2013edit} explicitly determined $\text{ed}_{\text{Forb}(C_h)}(p)$ for $h \in \{5, \dots, 9\}$. In the same work, Martin determined $\ed_{\Forb(C_{10})}(p)$ for $p \in [1/7, 1]$. Peck \cite{peck2013edit} later determined $\ed_{\Forb(C_{h})}(p)$ for all $p \in [0,1]$ when $h \geq 3$ and $h$ is odd, and for $p \in [1/\ceil{h/3},1]$ when $h \geq 4$ and $h$ is even. Berikkyzy, Martin and Peck \cite{berikkyzy2019edit} generalised this result to determine $\ed_{\Forb(C_h^t)}(p)$ for all $p \in [0,1]$ when $h \geq 2t(2t+1)+1$ and $(t+1) \nmid h$. The same authors also determined $\ed_{\Forb(C_h^t)}(p)$ for $p \in [1/\ceil{h/(2t+1)},1]$ when $h \geq 2t(2t+1)+1$ and $(t+1) \mid h$. More precisely, they showed the following.

\begin{theorem}[Berikkyzy, Martin and Peck \cite{berikkyzy2019edit}]\label{pthm:berikkyzy_powers_of_cycles}
Let $t \geq 1$ and $h \geq 2t(t+1)+1$ be positive integers, and let $\cH = \emph{\text{Forb}}(C_h^t)$.
\begin{enumerate}
    \item If $(t+1) \not|\,\, h$, then for all $p \in [0,1]$, we have
    \[
        \text{\emph{ed}}_{\text{\emph{Forb}}(C_h^t)}(p) = \min\lcb \frac{p}{t+1}, \min_{r \in \{0, 1, \dots, t\}} \lcb  \frac{p(1-p)}{r+\lb \ceil{\frac{h}{2t+1}} -r-1 \rb p} \rcb \rcb.
    \]
    \item If $(t+1) \mid h$, then for all $p \in [1/\ceil{h/(2t+1)},1]$, we have
    \[
        \text{\emph{ed}}_{\text{\emph{Forb}}(C_h^t)}(p) = \min_{r \in \{0, 1, \dots, t\}} \lcb \frac{p(1-p)}{r+\lb \ceil{\frac{h}{2t+1}} -r-1 \rb p} \rcb.
    \]
\end{enumerate}
\end{theorem}

We extend on this result for small $p$ in the case when $(t+1) \mid h$ to show the following.

\begin{theorem}\label{thm:main_result}
Let $t \geq 1$ and $h \geq 4t(2t+1)$ be integers, with $(t+1) \mid h$. Let $c_0  = \floor{(\floor{h/t}+1)/3}$, $\ell_0 = \ceil{h/(2t+1)}$, and let $p_0 = t/(c_0 \ell_0 -c_0 - \ell_0 +t+1)$. Then for all $p \in [p_0, 1/\ceil{h/(2t+1)}]$, we have that 
\[
\ed_{\Forb(C_h^t)}(p) = \frac{p(1-p)}{t+\lb \ceil{\frac{h}{2t+1}} -t-1 \rb p}.
\]
\end{theorem}

Note that when $h=10$ and $t = 1$, the value of $p_0$ in the theorem above is exactly $1/7$, and so the value of $p_0$ above result matches the previously known result of Martin \cite{martin2013edit} for $h = 10$ and $t = 1$. In fact, for $t=1$, when $h \in \{10, 12\}$, we determine the edit distance function for all $p \in [0,p_0]$, as follows.

\begin{theorem}\label{thm:main_result_C_10_12}
    \begin{enumerate}
        \item\label{item:main_result_h_10} For $p \in (0,1/7)$, we have $\ed_{\Forb(C_{10})}(p) = \frac{p(1-p)}{1+2p}$.
        \item\label{item:main_result_h_12} For $p \in (0, 1/10)$, we have $\ed_{\Forb(C_{12})}(p) = \frac{p(1-p)}{1+2p}$.
    \end{enumerate}
\end{theorem}
As a consequence of \cref{thm:main_result} and \cref{thm:main_result_C_10_12} together with the work of Martin~\cite{martin2013edit} and Peck~\cite{peck2013edit}, we now know the value of $\ed_{\Forb(C_{10})}(p)$ and $\ed_{\Forb(C_{12})}(p)$ for all $p \in [0,1]$. The remainder of the paper is organised as follows. In \cref{sec:CRG}, we introduce \emph{coloured regularity graphs}, which are the key tool we use in our proof. In \cref{sec:proof}, we prove some lemmas which we then combine to prove \cref{thm:main_result}, and \cref{thm:main_result_C_10_12}. Finally, in \cref{sec:conclusion}, we discuss exactly what needs to be done to determine the edit distance function $\ed_{\Forb(C_h^t)}(p)$ for all $p \in [0,p_0]$.
