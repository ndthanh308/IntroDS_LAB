%%%% Subsection 4 %%%%%%%%%%%%
\label{sec:4}

We have presented a mechanics-centered computational model to simulate the large deformation of knitted fabrics at compatible scale to their emerging engineering applications as wearables and soft robotics. By defining a dynamic formulation of the governing equation, with constitutive law at each yarn segment defined, we have implemented an adaptive integration scheme that can efficiently solve for the evolution of such complex system with hierarchical geometric representation. Our numerical study complimented by experimental evidence show not only the fundamental nonlinear elastic response of knitted fabrics, noticeably characterised by two-stage behaviour analogous to fiber-based biological systems, polymer networks and architected materials with geometric degrees of freedom arising across-scale, but also the wide range of initial-stage stiffness and extensibility across fabrics with varying topology. Using this model, we can extract across-scale mechanisms to support the investigation of mechanical roles of yarn dynamics on fabric extensibility and anisotropy, with measurements of inhomogeneous mechanical quantities that are not feasible from experiments. Moreover, this model can be used to explore the enormous design space of knitted fabrics, as we exploit their topology-dependent mechanical properties through varying spatial distributions. By doing so, we hope to guide the mechanical programmability of knitted fabrics towards more engineering applications as functional textiles.

