%%%% Section 1 %%%%%%%%%%%%
\label{sec:1}

Knitted fabrics are hierarchical structures that build up from microscale to macroscale, as yarns are the primary building blocks to dominate the physics and design of fabrics. The separation of these two scales (i.e., yarn-level and fabric-level) at the structural level (mesoscale) not only causes a range of interesting physical phenomena \cite{Poincloux2018, Poincloux2018a} to arise, but also provides a huge design space for functionalities in knitted fabrics beyond what their constitutive materials can achieve. Knitted fabrics represent nonbiological examples of a nonlinear elastic response characterised by a ``J-shape'' curve, as the fabric transitions from bending energy dominant region to stretching energy dominant region under uniaxial tension. The presence of fabric patterns at the mesoscale allows the whole fabric to take sustainable tensile stress under low strains on the individual yarn segments, as yarn alignment dynamics provide geometric degrees of freedom for fabric. The distinctively compliant behaviour of knitted fabrics makes them stand out as excellent scaffolds for soft robotic actuators \cite{Connolly2019,Sanchez2021,Albaugh2019}, responsive wearable devices \cite{Polygerinos2015, Granberry2019} and in vivo health monitoring devices \cite{Wicaksono2020}, where large deformation and flexible morphing without material damage is desired. The anisotropy at the fabric mesoscale can been exploited to fine tune actuation of such devices \cite{Luo2022}, where carefully selected structures can be spatially varied across the fabric, to generate morphing that provides customised fit to complex geometries. Multifunctional knitted fabrics can be created through embedding functional yarns into conventional knit structures, to further enlarge the design space of knitted fabrics to mimic morphing biological structures \cite{Han2017} and to serve as light and touch sensors \cite{Albaugh2019}, pressure sensors \cite{Luo2021}, electronic interfaces \cite{Wicaksono2017} and electronic skins \cite{Pei2019} for an unravelling future of human machine interaction. 

Currently, intuition-led strategies remain the primary approach to design  devices made of knitted fabrics. This paradigm  poses limitations on exploring the design space due to high machinery costs, training costs and material waste. A generalisation of these application-driven designs for yarn geometries, fabric structures and material variations has not yet been established. Early theoretical work on knitted structures hypothesise that the fabric shape is predominantly determined by the geometric description of knit structure rather than constitutive material. Starting from a three-dimensional parameterisation of the jersey knit pattern \cite{Peirce1947} to curvature augmented model \cite{Leaf1955}, followed by energy minimisation model \cite{Munden1959}, most geometric models of knitted fabrics are constructed through superposition of cosine and sine curves due to the smoothness and periodicity of these shape functions. With the development of spline basis functions, we can discretise yarns with sufficient accuracy and such yarn-based models \cite{Bergou2008, Kaldor2008, Cirio2014, Leaf2018} stand out compared to coarse-grained models \cite{Terzopoulos1987, Baraff1998, Breen1994, Yeoman2010} and homogenised models \cite{NarainArminSamiiJamesO2012, Yuksel2012, Liu2017, Dinh2018, Weeger2018, Wu2020, Sperl2021}, due to their capability to (i) capture mechanical behavior originating from first principles via yarn dynamics, (ii) provide quantitative measurements of geometric nonlinearities arising across scales, and (iii) integrate multiple fabric structures within complex 3D configurations, all while maintaining unconstrained extensibility of individual yarns. 

In this work, we detail the development and implementation of a mechanics-centered computational model of knitted fabrics, as a step towards mechanical programmability of textile-based metamaterials. To begin, we adopt a yarn-based model with cubic order spline basis functions \cite{Kaldor2008} that was originally applied in computer graphics to animate cloth in a qualitatively realistic manner. We are the first to extend and validate this model against machine knitted fabrics at full scale, to provide physical insights into across-scale mechanical behaviours. Our numerical approach is implemented through fully dynamic formulation of the governing equation of motion, integrated explicitly with an adaptive scheme. A key aspect of our numerical procedure is the introduction of relaxation stage before the application of external tensile forces, to account for residue stresses that are inherent in the knit fabrication process. This residue stress typically poses a challenge in generating an accurate ground truth reference state from purely the geometric description of knitted fabrics, and are lacking in existing work. We treat the dynamics at a continuous yarn as the governing mechanics, and connect yarns through translation and interpolation to form fabrics. On one hand, this approach enables the prediction of local, spatial evolution of bending and stretching energy, echoing a broader interest in how soft materials gradually adapt to applied elastic energy. As we can directly compute measurements of energy, stretch and alignment with regard to each yarn segment, we can predict micromechanical hot spots. Moreover, this model enables the interactive design of complex fabric configurations, which possess inhomogeneous mechanical fields through varying fabric structures or material properties. We can further fine tune fabric configurations with mechanical insights gained from micromechanical lenses, as we leverage the advantages of the hierarchical geometric representation. 

In addition to providing quantitatively calibrated physical results and motivating examples for more detailed study, we show evidence that structural properties, such as varying topological description of fabric patterns \cite{Markande2020} and varying spatial distribution of fabric patterns can effectively adjust the mechanical behaviour of knitted fabrics, without modifying the material properties. Moreover, we discuss the potential for using our model to further extend the design space from basic knitted fabric structures, leading towards a programmable future of functional textiles.
