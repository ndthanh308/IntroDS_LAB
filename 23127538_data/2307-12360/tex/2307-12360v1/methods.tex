%%%% Section 5 %%%%%%%%%%%%
\section{Implementation details of the numerical method}
\subsection{Numerical integration}
\label{sec:5a}
The numerical simulations in this paper are performed using a custom C++ code
that uses the OpenMP library~\cite{dagum98} for multithreading. The core of the
simulation involves integrating the ODE system for the position $\vq_k$ and
velocity $\vqd_k$ degrees of freedom described in Eqs.~\eqref{eq:ys} \&
\eqref{eq:vs}. This is solved using the fourth-order adaptive ``first same as
last'' (FSAL) Runge--Kutta method~\cite{hairer93}. This method uses five
intermediate stages, where the first four can be used construct a fourth-order
accurate solution, and the final stage can be used to construct an auxiliary
third-order accurate solution for step size selection. With the FSAL property,
the final stage can be re-used as the first stage of the next step, reducing
the total computational work. We implement adaptive step size selection via the
procedures described by Hairer et al.~\cite{hairer93}, which uses a combination
of absolute tolerance $\Atol$ and relative tolerance $\Rtol$. Initial step size
selection is also performed using the methods of Hairer et al.~\cite{hairer93}.

Adaptive integration is beneficial for our simulations, since the appropriate
step size choices vary greatly over the course of the simulation. In the
initial stages, yarn elasticity is the most important physical effect and large
timesteps can be taken. Once contact forces become important, the timestep
sizes are substantially reduced. Similar methods have been employed in
the simulation of crumpled sheets~\cite{andrejevic22}, which has comparable
behavior.

Our simulations output snapshots at equally spaced time intervals. Since the
timesteps are chosen adaptively, the integration time points will not align
with the output time points in general. To solve this issue, we make use of
dense output~\cite{hairer93} and construct a cubic Hermite interpolant of the
simulation state over each integration time step. Evaluating this interpolant
at the output time point results in a fourth-order accurate approximation of
the solution. The snapshots are outputted as binary files that contain the
complete simulation state, which can be post-processed to perform a variety
of analyses.

It is worth noting that the ODE system representing the yarn mechanics is not
infinitely differentiable, since the contact forces are discretely switched on
and off as yarns move past each other. Proving that the Runge--Kutta scheme is
fourth-order accurate requires that the mathematical solution has Taylor
expansions up to fourth order, which is not true in the case when an
integration timestep passes over a moment when a contact force is switched on
or off. Because of this, it is not possible to guarantee that the results are
fourth-order accurate. Nevertheless, we opt to use the fourth-order scheme
since it results in good accuracy and performance overall. Furthermore, Hairer
et al.~\cite{hairer93} demonstrate that in practical cases, adaptive-timestep
integrators can approach high-order accuracy even when the ODE system lacks
sufficient smoothness, since the integrator can automatically refine the
timestep when passing over a discrete switch in the ODE, minimizing the
additional error incurred.

\subsection{Linear system}
\label{sec:5b}
For a single yarn with $N$ spline segments we write $\vq\in \R^{3N+3}$ and
$\vqd\in \R^{3N+3}$ to be the vectors describing the yarn position and
velocity, respectively. From Eqs.~\eqref{eq:lagr} \& \eqref{eq:dkint} the
general equation of motion for a particular component $(\vq_k,\vqd_k)$
is given by
\begin{align}
  \frac{d}{dt} \left( [ M \vqd]_k \right) &= \Big( \nabla_{\vq_k} V(\vq) + \nabla_{\vqd_k} D(\vq,\vqd) \Big), \label{eq:lagr2a} \\
  \frac{d}{dt} \left( \vq_k \right) &= \vqd_k \label{eq:lagr2b}
\end{align}
where $M$ is a banded matrix whose components are defined by
Eq.~\eqref{eq:ke_integ}. For $2\le j \le N-2$, away from the end points, the
components of $M$ are given by
\begin{equation}
  M_{jk} = \begin{cases}
    \tfrac{151}{315} & \qquad \text{if $k=j$,} \\
    \tfrac{397}{1680} & \qquad \text{if $|k-j|=1$,} \\
    \tfrac{1}{42} & \qquad \text{if $|k-j|=2$,} \\
    \tfrac{1}{5040} & \qquad \text{if $|k-j|=3$,} \\
    0 & \qquad \text{otherwise.}
  \end{cases}
\end{equation}
Near the end points, the matrix values change, because the B-spline
functions are no longer fully contained within $\Omega$. The values for $j<2$
are given in Table \ref{tab:ke_boundary}, and the values for $j>N-2$ are
obtained via symmetry. Our code can also handle the case when either the
position or direction of the end point is fixed, which results in adjusting
the linear system to incorporate a linear algebraic constraint.

\begin{table}
  \centering
  \begin{tabular}{|c|cccccc|}
    \hline
    & $k=-1$ & $k=0$ & $k=1$ & $k=2$ & $k=3$ & $k=4$ \\
    \hline
    $j=-1$ & \nicefrac{1}{252} & \nicefrac{43}{1680} & \nicefrac{1}{84} & \nicefrac{1}{5040} & & \\
    $j=0$ & \nicefrac{43}{1680} & \nicefrac{151}{630} & \nicefrac{59}{280} & \nicefrac{1}{42} & \nicefrac{1}{5040} & \\
    $j=1$ & \nicefrac{1}{84} & \nicefrac{59}{280} & \nicefrac{599}{1260} & \nicefrac{397}{1680} & \nicefrac{1}{42} & \nicefrac{1}{5040} \\
    \hline
  \end{tabular}
  \caption{Coefficients in the matrix $M$, defined in Eq.~\eqref{eq:ke_integ}
  arising from the kinetic energy term in the Lagrangian formulation
  of the yarn dynamics.\label{tab:ke_boundary}} 
\end{table}

To solve the ODE system in Eqs.~\eqref{eq:lagr2a} \& \eqref{eq:lagr2b} it is
necessary to solve the linear system $Mq=f$ where $q\in \R^{(3N+3)\times 3}$
are entries of $\vqd$ arranged into a matrix, with the $x$, $y$, and $z$
components each in one column. $f \in \R^{(3N+3)\times 3}$ are the
corresponding source terms, from the right hand side of Eq.~\eqref{eq:lagr2a}. The
matrix $M$ remains fixed throughout the simulation, and therefore during the
initialization its LU factorization is precomputed. This accelerates the
solution of the linear system during the simulation. The LAPACK library is
used, with the LU factorization being performed using the \texttt{dgbtrf}
routine for a general banded matrix in double-precision floating point
arithmetic. The \texttt{dgbtrs} routine is then used to solve the linear
systems during the timesteps.

%\begin{table}
%  \begin{tabular}{ccccccc}
%    ,1/6.,2/3.,1/6.,0.,
%    ,7/720.,173/1260.,137/840.,29/1260.,1/5040.,
%    ,121/1260.,13093/5040.,4553/630.,9883/2520.,509/1260.,17/5040.,
%  \end{tabular}
%
%  \begin{tabular}{ccccccc}
%    ,1/2.,0.,-1/2.,0.,
%    ,43/1680.,151/630.,59/280.,1/42.,1/5040.,
%    ,1/63.,397/1680.,307/630.,149/630.,1/42.,1/5040.,
%  \end{tabular}
%
%  \begin{tabular}{cccccc}
%    ,1/6.,2/3.,1/6.,0.,
%    ,1/2.,0.,-1/2.,0.,0.,
%    ,31/5040.,587/2520.,55/72.,283/630.,239/5040.,1/2520
%  \end{tabular}
%\end{table}

\subsection{Contact sphere diameter calculations}
\label{sec:csphere}
As described in Sec.~\ref{sec:2c}, yarn--yarn contact forces are handled by
introducing $n$ equally spaced spheres of radius $d_\text{con}$ along each spline segment.
If $\dcon=d$, then the envelope formed by the spheres would be smaller
than the yarn itself. We therefore systematically choose $\dcon$ to
better approximate the yarn shape. Let $D=l/(2n)$ be the distance between
successive contact spheres. Assuming that $D$ is small relative to $d$, the
contact sphere diameter is chosen to be
\begin{equation}
  \dcon=\frac{d+\sqrt{d^2 + \tfrac23 D^2}}{2},
\end{equation}
which ensures that the average diameter of the envelope of spheres is equal to
$d$. To choose the number of contact spheres, we define a parameter $\alpha$
corresponding to the maximum allowable mean square deviation between the
contact sphere envelope and the filament diameter, which is typically set to be
several percent. Then the number of contact spheres satisfies
\begin{equation}
  n = \left\lceil \frac{l}{d\sqrt{6(1+\alpha)\alpha}}  \right\rceil,
\end{equation}
where $\lceil\cdot\rceil$ is the ceiling operator. Since the contact spheres
overlap, when considering Eq.~\eqref{eq:contact_integ}, it is necessary to
screen out the effect of interactions from neighboring spheres along the same
yarn. This is done by defining a screening number $n_\text{screen} = \lceil
\beta \dcon/D \rceil$ where $\beta$ is a dimensionless parameter. Terms in the
sum of Eq.~\ref{eq:contact_integ} are only considered if the indices of the
spheres are separated by at least $n_\text{screen}$.

\subsection{Tethering forces and initial sample generation}
\label{sec:5d}
To perform the uniaxial tension tests, tethering forces are applied to the
fabric to fix the displacement in two end regions. This procedure is similar
to the approach used in the immersed boundary method~\cite{peskin02}
to simulate fixed walls~\cite{fai18}. Specifically, two regions $D_+$ and
$D_-$ are defined, where typically $D_\pm = \{ (x,y,z)\in \R^3 \,:\, \pm
y>y_\text{fix} \}$ for a constant $y_\text{fix}$ when pulling a sample in the
$y$ direction.

During the simulation initialization, all spline segments that lie fully within
$D_+$ and $D_-$ are marked, and the reference position of each
quadrature point within each marked segment is recorded. Using
this information, the additional energy contributions
\begin{equation}
  V^t_\pm = k_t \int_\Omega I_{D_\pm}(s) \| \vy(s) - \vec{F}_\pm(\vy_\text{ref}(s),t) \|^2 ds
  \label{eq:energy_tether}
\end{equation}
are added to Eq.~\eqref{eq:lagr}, where $I_{D_\pm}(s)$ is equal to one in spline
segments marked within $D_\pm$, and zero otherwise. Here, $\vec{F}_\pm: \R^3
\to \R^3$ are time-dependent affine transformations of the reference position.
They can be used to apply the constant pulling velocity in the end regions.
The forces that are measured in the tension tests are computed as the total
force applied to the fabric in each tethered region.

As described in Sec.~\ref{sec:6e}, the experimental samples created by the knitting
machine are already under substantial tension, meaning that the yarns are much
tighter than the examples shown in Figs.~\ref{fig2:knit_patterns} \&
\ref{fig1:knit_assembly}. It is difficult to initialize the simulation in tighter
configurations directly, since any overlaps in the initial state may result in
very large initial contact forces. Because of this, we perform a preliminary
simulation to generate the samples for the tension tests. We initialize the
yarn in a loose configuration given by Eq.~\eqref{eq:basic_param_spiral}, and
then make the spline rest length $l$ a function of time, applying a linear
ramping so that
\begin{equation}
  l(t) = \begin{cases}
    l_0 - (1-\eta) \frac{t}{T_r} & \qquad \text{if $t<T_r$,} \\
    \eta l_0 & \qquad \text{if $t\ge T_r$,}
  \end{cases}
\end{equation}
where $l_0$ is the initial rest length of the yarn, $T_r$ is the duration of
the ramping, and $\eta$ is a dimensionless ratio chosen to ensure that the
final state matches the same compression as the experimental samples. During
this procedure, the two end regions are tethered using the energy contributions
in Eq.~\eqref{eq:energy_tether} to prevent the sample from curling. Since the
rest length is reduced, the affine transformations $\vec{F}_\pm$ are used to
apply a commensurable shrinkage to the tethered regions. The precise amount of
shrinkage is determined by comparing to the geometry of the
experimental samples. After this preliminary simulation is performed, the yarn
state is saved and then used as the starting configuration for the tension test
simulations.


\section{Materials and Experiments}
\subsection{Material selection}
\label{sec:6a}
Commercially available acrylic spun yarns (16/2 Vybralite Acryclic Yarns, National Spinning Co. by Peter Patchis Yarns, USA) were selected to create all knitted fabric swatches, which were mainly collected for the study of Sanchez et al.~\cite{Sanchez2023}. We performed the pull-out and bending tests on a single yarn, in order to benchmark the previously reported stretching stiffness, and to calibrate the bending stiffness of selected experimental material.
  
\subsection{Filament diameter}
\label{sec:6b}
Scanning Electron Microscope (SEM) images were taken with a Tescan Vega scanning electron microscope for the acrylic spun yarns, in order to measure the diameter of the yarn accurately. To adjust for camera orientation, we used Adobe Photoshop to calculate the projected area of the yarn segment under orthogonal projection and used that to calculate the effective yarn diameter as projected area divided by the fixed yarn length of \SI{20}{mm}. This process was repeated three times on three different yarn segments from the same material. The averaged yarn diameter, \SI{0.0524}{cm}, from experimental measurements was used to benchmark the yarn diameter used in simulation. In addition, we assume a circular and consistent cross section with this calibrated effective radius from onward for related calculations. 

\subsection{Yarn linear density}
\label{sec:6c}
We adopt yarn linear density defined as 
    \begin{equation}
    \rho = \frac{M}{l},
    \label{eq:yarn_density}
    \end{equation}
which is a standard definition adopted in the textile industry and is provided by the manufacturer as \SI{0.077}{g/m}. The equivalent term in simulation is \SI{7.7e-4}{g/cm}. 

\subsection{Characterisation of yarn properties}
\label{sec:6d}
    % Figure environment removed
Figure \ref{fig5a:yarn_pullout} illustrates the experimental set up for a pull-out test on a single yarn, in order to measure its stretching stiffness. A pull-out test of a single yarn adhered to acrylic boards with instant adhesive to prevent slippage was performed on at least three different samples, with fixed gauge length recorded to calculate tensile strain. All samples were tested at a rate of \SI{5}{mm/min} on a Universal Testing Machine (Instron 5566R), and tensile force $F$ and tensile strain $\epsilon$ were directly measured during loading processes. The stretching stiffness of a single yarn 
    \begin{equation}
    E^s = \frac{F}{\pi r^2 \epsilon} 
    \label{eq:yarn_stretch}
    \end{equation}
    first underwent a linear stage with \SI{79.0}{MPa} within \SI{5}{\%} tensile strain, followed by softening behaviour. In all simulation tests, we assumed linear elasticity and calibrated the equivalent stretching stiffness to be \SI{7.9e8}{g/(cm.s^2)}. This assumption was further testified by histogram measurement of the stretch of individual yarn segments. On average, less than \SI{20}{\%} yarn segments are stretched by more than \SI{5}{\%} even when the fabric samples are stretched by more then \SI{60}{\%}. 
    % Figure environment removed
Figure \ref{fig5b:yarn_bend} illustrates the experimental set up for a bending test on a single yarn cantilevered at one end with the free end consisting a horizontal length $\lambda$, and drops by a vertical displacement of $\delta$ due to gravity. We assume that gravity of the yarn acts as a distributed load $q=\pi r^2 \rho g$ and referring to Chakrabarti et al.~\cite{Chakrabarti2021}, the bending stiffness of such soft material can be determined from sets of fixed $\lambda$ and measured $\delta$ from:
    \begin{equation}
    E^b = \frac{q \lambda^4}{8 \delta I}.
    \label{eq:yarn_bend}
    \end{equation}
    For each bending test, we took high resolution photos of the cantilevered yarn segment, with pixel images on the background. Each pixel grid represents \SI{0.25}{mm} and post processing in Adobe Photoshop was done to ensure an orthogonal perspective. From a series of bending tests on three different yarn segments, we calibrated the bending stiffness of the used acrylic spun yarn to be \SI{0.249}{MPa} and its equivalent parameter in simulation is \SI{2.49e4}{g/(cm.s^2)}. Note that the measured bending stiffness differs from the stretching stiffness by several magnitudes due to the soft nature of used material.

%\item Artificial density scaling \\
%\item Global damping in order to remain in the quasi-static regime \\
%\item Other parameters: \\
%  \begin{itemize}
%      \item Contact parameters \\
%      \item $k_{dt}$ and $k_{dn}$ \\
%      \item Tolerance on the contact sphere envelope (0.03) \\
%    \end{itemize}

\subsection{Fabrication of knitted fabrics}
\label{sec:6e}
For reference we summarise the experimental protocol to fabricate and to test the fabric swatches from Sanchez et al.~\cite{Sanchez2023} here. In total, six samples for each knitted structure (jersey, garter 1 by 1, rib 1 by 1 and seed 1 by 1) were fabricated with the same settings on stitch spacing and machine tension on a Kniterate V-bed knitting machine S8 (Kniterate, EU), and all consisting of 41 wales by 40 courses (i.e., number of stitches along the warp and the weft directions respectively). The ``knitout'' programme \cite{McCann2017} was used to convert the Python frontend (specified by McCann et al.~\cite{McCann2016}) to Kniterate-specific machine language ``kcode'', in order to operate the knitting machine. The unit arclength of each fabric $l$ was determined by 
    \begin{equation}
    L = \frac{M}{\rho N_w N_c},
    \label{eq:yarn_arclength}
    \end{equation}
from directly measured mass of fabricated sample $M$, fixed yarn linear density from the manufacturer and number of wales $N_w$ and number of courses $N_c$. After fabrication, all samples were left at ambient conditions for 24-48 hours to enable material relaxation, as responding to residual stresses from the manufacturing process. Then, the dimensions of all samples were measured and approximated as those of samples in their reference states.

\subsection{Uniaxial tensile tests to characterise the knitted fabrics}
\label{sec:6f}
All knitted fabric samples were cyclically tested using Universal Testing Machiens (an Instron 5544A with a smaller-range load cell and an Instron 5566R with a larger-range load cell). The gauge lengths were recorded, in order to offset mismatch between previously recorded sample dimensions and ground-truth sample dimensions in stress-free states on the testing machines. These samples were pre-cycled to force magnitude of \SI{20}{N} at a fixed rate of \SI{10}{mm/min} during loading stage and \SI{20}{mm/min} during unloading, followed by two cycles at a rate of \SI{5}{mm/min} until reaching \SI{15}{N}. This upper bound on force was selected empirically, in order to prevent fabrics from plastic deformation. For our study, we selected the mechanical responses measured after pre-cycling, as they were robustly repeated through cycles for the same fabric and the same tensile loading direction. Note that the strain ranges from experiments were smaller than those from numerics presented in this study, since the former were collected from a particular cycle of the cyclic tests way before failure.

