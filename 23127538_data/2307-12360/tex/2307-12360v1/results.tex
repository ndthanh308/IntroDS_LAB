%%%% Subsection 3a %%%%%%%%%%%%
\subsection{Effect of pre-tension on validation}
\label{sec:3a}
Experimental evidence show that despite specifying the same stitch numbers along both warp and weft directions, the samples made from varying fabric patterns consist of varying dimensions. Typically, yarns within a fabric exhibit residual tension from fabrication process as they are prestressed to be straight and tight when they are fed into the carriers. Hence, it is crucial to capture an accurate reference state configuration of knitted fabric under certain pretension, in order to provide a meaningful comparison between numerical test and experimental test. Since it is hard to obtain the tension a priori, it is hard to establish a relationship between stitch numbers and fabric tightness analytically. Though Eltahan et al.~\cite{Eltahan2016} attempted to find this relationship empirically, the results are constrained by the limited dataset on a single jersey pattern. We instead propose to simulate the process of fabrics being prestressed, hence not posing any constraint on the fabric structure nor the material. By using the aspect ratio and unit arclength of simulated samples to benchmark against manufactured samples, we were able to obtain close to ground-truth configurations of all four basic weft-knitted fabrics in their reference states. During the simulation of applying pretension, we enable the fabric to relax to an equilibrium state, as we pull the ends of the continuous yarn closer in a purse-string contraction mechanism. By doing so, we allow the fabric to approach its state that naturally minimises the total bending energy and becomes tighter in configuration. 

%The applied pretension is calibrated as a contraction ratio $\lambda_{ds}$ of the unit arc length referenced from its geometric model. While keeping all unit arc lengths to be consistent across the four basic patterns investigated, we vary the target aspect ratios by varying contraction ratios in three dimensions respectively as $\lambda_x$, $\lambda_y$ and $\lambda_z$ summarized in Table \ref{tab1:prestress}.

%\begin{table}[h]
%  \caption{Coefficients used in prestressed stage.}
%  \label{tab2:pretension}
%  \centering
%  \begin{tabular}{llllllll}
%    \toprule
%    Pattern & reference $ds$ & $N$ & initialised $\lambda_{ds}$ & prestressed $\lambda_{ds}$ & $\lambda_x$ & $\lambda_y$ & $\lambda_z$ \\
%    \midrule
%    jersey & 0.034 $cm$ & 32 & 0.72 & 0.65 & 0.73 & 0.48 & 0.85 \\
%    garter 1 by 1 & 0.0298 $cm$ & 32 & 0.822 & 0.65 & 0.72 & 0.35 & 0.74 \\
%    rib 1 by 1 & 0.032 $cm$ & 32 & 0.765 & 0.65 & 0.43 & 0.59 & 0.85 \\
%    seed 1 by 1 & 0.036 $cm$ & 32 & 0.66 & 0.65 & 0.65 & 0.39 & 0.74 \\    
%  \end{tabular}
%\end{table}


%%%% Subsection 3b %%%%%%%%%%%%
\subsection{The fundamental mechanical behaviour of knitted fabrics}
\label{sec:3b}
    % Figure environment removed

    % Figure environment removed

\noindent The influence of geometric flexibility of fabric pattern on the fundamental mechanical response of knitted fabrics is well captured by our numerical model and highlighted in Fig.~ \ref{fig3b:stress_strain_warp} and Fig.~\ref{fig3b:stress_strain_weft}, which show applied uniaxial tensile stress $\sigma$ against strain $\epsilon$ in warp and weft directions respectively and the corresponding measurements validated by experiments. To offset variation in fabric dimension $S_R$ along the face formed with yarn diameter $2r$, where the load is applied on, we define tensile stress $\sigma = \frac{F}{2rS_R}$. To offset variation in fabric dimension $L_R$ along loading direction in the reference state, we define tensile strain $\epsilon = \frac{L_D-L_R}{L_R}\cdot \SI{100}{\%}$ to account for normalised extension $L_D-L_R$. We see good alignment between experiment and simulation, in terms of capturing the overall ``J-shaped'' curves (analogous to architected metamaterials \cite{Moestopo2023} and biological fiber networks \cite{Fung1996}) and the relative rigidity and extensibility of four knitted fabrics. 

%    % Figure environment removed
    
Uniaxial tensile test along warp direction demonstrates distinct two-stage regions of deformation to failure for all fabrics. Firstly, the jersey fabric, having consistent loop contacts throughout the fabric and hence possessing the simplest geometry, behaved most rigidly among the four fabrics studied during the first stage characterised by low linear stiffness (regime 1) at the magnitude of \SI{0.1}{MPa} up to \SI{30}{\%} strain. Previous experimental study reported observation of geometry reconfiguration as yarns slide through loop contacts and straighten to align more towards applied tensile load \cite{Sanchez2023}, and our study provides quantitative evidences for such yarn dynamics that are summarised in Section \ref{sec:3c}. After that, the jersey fabric transitioned to a stage where the stiffness monotonically increased by up to \SI{10} times (regime 2), during when stretching of individual yarns become noticeable and statistical measurements are discussed in Section \ref{sec:3c}. Secondly, we observe the garter fabric being the initially softest and most stretchable among the four fabrics studied, as it first undergoing a nearly linear region with the lowest slope (regime 1) at the magnitude of \SI{0.01}{MPa} up to the largest strain magnitude of \SI{60}{\%}, followed by yarn stretching (regime 2) at a distinctively higher slope at the magnitude of \SI{1}{MPa} in the data up to failure strain close to \SI{90}{\%}. On the other hand, the rib fabric being initially the most rigid and least extensible (excluding jersey), it first underwent yarn alignment with a slope almost two times higher than that for the garter fabric (regime 1), and only up to \SI{40}{\%} strain, comparably quickly followed by yarn stretching (regime 2) characterised by a much higher slope in the data up to failure at only \SI{60}{\%} strain. In addition, the seed fabric sustained the former loading stage (regime 1) with stiffness similar to that of the rib fabric up to \SI{50}{\%} strain, and transitioned to the latter loading stage (regime 2) with stiffness reaching the asymptotic magnitude of \SI{1}{MPa}.

We observe similar transition behaviour upon uniaxial loading along weft direction, as all fabrics are initially soft and stretchable, followed by strain hardening as geometric flexibility from the mesoscale patterns are exhausted. However, the relative rigidity and extensibility of fabrics are now different. Though the jersey fabric is again the most rigid and undertook strain-hardening the soonest at \SI{30}{\%} strain during loading regime 1, the relative variation in stiffness among the four fabrics during this loading regime is negligible. The rib fabric, previously representative of rigid behaviour under tension along warp direction, now becomes the initially softest and most stretchable under tension along weft direction, as it first undergoing yarn alignment of a (regime 1) at almost negligible magnitude up to more than \SI{100}{\%} strain, followed by yarn stretching (regime 2) at a noticeably higher slope at the magnitude of \SI{0.1}{MPa} in the data up to failure strain more than \SI{150}{\%}. On the other hand, the garter fabric under tension along weft direction changes to be the initially most rigid and least extensible (excluding jersey), as it first underwent yarn alignment with a slope almost two times higher than that for the rib fabric (regime 1), and only up to \SI{50}{\%} strain, comparably quickly followed by yarn stretching (regime 2) characterised by a much higher slope approaching the magnitude of \SI{1}{MPa} in the data up to failure at only close to \SI{100}{\%} strain. In addition, the seed fabric undertook mechanical behaviour close to that of the garter fabric.
%\KB{here I suggest to comment on the non-perfect agreement between exp and simulations? What are the contributing factors? Can we also compare experimental and numerical deformed shapes of the knits?}

    % Figure environment removed

It is well accepted that the precise matching between reduced-order constitutive models and experimental tests on knitted fabric is challenging \cite{Vassiliadis2007, Yeoman2010, Dinh2018, Liu2017, Weeger2018, Wu2020}. Though our simulation model recovers the general two-stage nonlinear elastic behaviour and the relative responses among the four basic weft-knitted fabrics well, we notice the limitation of our model, particularly when modelling the behaviour as fabrics transition to the stretching-dominant stage. This mainly comes from the limitation of our assumption on linear elasticity. We further discuss this limitation in Section~\ref{sec:3c}, as statistical measurements on stretch of individual yarn segment for all studied fabrics and loading conditions are collected.  


%%%% Subsection 3c %%%%%%%%%%%%
\subsection{Mechanical role of yarn dynamics on fabric extensibility and anisotropy}
\label{sec:3c}
To probe into how yarn rearrangements influence the macroscopic extensibility of knitted fabrics, we measured the projection of individual yarn segments on to the loading direction of warp in Fig.~ \ref{fig3c:rose_loady} and of weft Fig.~ \ref{fig3c:rose_loadx}. With an angular increment (bin size) of \SI{5} degrees, we can track the overall evolution of yarn segments as they align closer with the applied load as tensile strain increased from \SI{0}{\%} to \SI{120}{\%}. This quantitative evidence compliments experimental observation of reorientation of yarn segments to exploit geometric degrees of freedom within the network of connected yarn segments. Such geometric rearrangement of yarn segments rather than material stretching of yarn segments contributed to the compliant behaviour during the initial stage of fabric mechanical response. Statistical distributions of yarn segment stretch in Fig.~ \ref{fig3c:hist_loady} and Fig.~ \ref{fig3c:hist_loadx} further support the yarn reorientation mechanism during the initial loading on fabrics, as distribution peaks remain within the range for negligible segment stretch while fabrics are stretched until transitions occur. It is worth noticing that even as fabrics transition to higher strain range (close to or more than \SI{100}{\%}) and yarn segments that are most stretched to almost \SI{20}{\%}, these segments account for less than \SI{20}{\%} of all yarn segments. Therefore, it is reasonable to assume linear elasticity to measure the averaged macroscopic behaviour of fabrics, despite quantitative evidence for such inhomogeneous mechanical field of segment stretch. 

Previously we also observed anisotropy from knitted fabrics, which is topology-dependent. Taking the most representative examples as garter and rib, the former has a softer mechanical response and sustains a higher elastic strain range when subjected to tension along fabric warp direction rather than weft direction. On the contrary, the latter behaves in a stiffer manner within a lower elastic strain range when responding to tension along fabric weft direction rather than warp direction. To probe into the influence of fabric structure on anisotropy, we start from examining their geometric configurations in reference state, with reference to Fig.~ \ref{fig3c:rose_loady} and Fig.~ \ref{fig3c:rose_loadx}. The garter fabric initially has less than \SI{10}{\%} of yarn segments aligning with the warp direction within a difference of \SI{10} degrees, but has more than \SI{20}{\%} similarly close yarn alignment with the weft direction. On the other hand, the rib fabric has more than \SI{20}{\%} of yarn segments aligning similarly close to the warp direction, but just less than \SI{15}{\%} yarn alignment with the weft direction in reference state. The lower frequency of close yarn alignment with the loading direction, the more geometric degrees of freedom to exploit when yarn segments rearrange themselves to manifest applied stress within the hierarchical system, leading to the fabric behaving more compliantly under this loading direction rather than the orthogonal loading direction. Moreover, if we examine Fig.~ \ref{fig3c:hist_loady} and Fig.~ \ref{fig3c:hist_loadx}, we see that for a fabric to behave relatively softer and more compliant under a fixed loading condition, the shift of its distribution peak of yarn segment stretch takes place slower towards a higher stretch range, with fixed steps to stretch the fabric. 

    % Figure environment removed

    % Figure environment removed
    
    % Figure environment removed

    % Figure environment removed

    
%%%% Subsection 3d %%%%%%%%%%%%
\newpage
\subsection{Demonstration of the design space}
\label{sec:3d}
Here we present a gallery of multi-structure knitted fabrics with varying spatial distributions: (A) half garter 1 by 1 placed on top and half jersey placed on bottom; (B) half garter 1 by 1 placed in the middle and quarter jersey placed on both ends; (C) half rib 1 by 1 placed on top and half jersey placed on bottom (Fig.~ \ref{fig3d:multi_patterns}). We demonstrate their mechanical responses to uniaxial tension at fixed steps of uniaxial tensile strain. Utilising the distinctive mechanical properties of structures discussed in Sec.~\ref{sec:3b}, we designed asymmetric primitives, with (A) being soft and extensible and (C) being rigid and less extensible along the fabric warp direction. Moreover, even with the same volumetric composition, varying spatial distribution of structures (A and B) leads to expanded set of fabric configurations. These knitted fabrics can morph compliantly with different geometric configurations characterised by varying side profiles, showing potential application in custom fit for wearables. Moreover, the anisotropy of fabric structures make them versatile in multi-objective applications, such as soft robotics. Last but not least, the side space opened up while these knitted fabrics are stretched enables them to be implemented as responsive panels, an emerging class of architectural component. 

    % Figure environment removed
