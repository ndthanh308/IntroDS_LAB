%%%% Subsection 4 %%%%%%%%%%%%
\label{sec:4}
In summary, we first present the mechanics of knitted fabrics through micromechanical lenses from yarn dynamics. By defining a dynamic formulation of the governing equation, with simple yet adaptive constitutive law at each yarn segment, we have developed and implemented a computational model to efficiently solve for the evolution of such complex system with localised mechanical fields. Our numerical study complimented by experimental evidence show the fundamental mechanical response of knitted fabrics, noticeably characterised by J-shape behaviour analogous to hierarchical biological structures with geometric degrees of freedom arising from the separation of scales \cite{Meyers2013, Barthelat2016, Oosten2019, Zhang2021}, echoing the current wider interest in understanding how soft materials gradually adapt to applied elastic energy. In particular, we include the effect of pre-tension in our numerical procedure, in order to provide meaningful comparison with experimental measurement. In addition, we probe into the topology-dependent variation in fabric stiffness, extensibility and anisotropy by conducting parametric study on a set of representative weft-knitted fabrics. Supported by statistical measurements of inhomogeneous yarn segment stretch and alignment that are not feasible from experiments, we provide insights on the remarkable differences among weft-knitted fabrics from varying topological groups. Last but not least, we demonstrate how to apply learnt mechanical properties of varying stitch patterns to manipulate the design for targeted responses and localised compliance. Such multi-structure multi-material configurations in both 2D and 3D, of which the enormous design space can be explored by rapid deployment of our computational tool. By doing so, we hope to pave the path for systematic design of mechanically programmable fabrics and textiles beyond what their constitutive materials can achieve through demonstrations in responsive structures, wearables and soft actuators.