%%%% Subsection 3a %%%%%%%%%%%%
\subsection{Effect of pre-tension on validation}
\label{sec:3a}
Experimental evidence shows that despite specifying the same number of stitches along both warp and weft directions, the samples made from same materials but varying stitch patterns often consist of varying dimensions \cite{Allan1983, Amreeva2007, Wei2011}. In addition to slight variation in unit stitch length due to limitations in manufacturing, one key factor for this variability is the internal response of stitch patterns to pre-tension during manufacturing. Typically, yarns are prestressed to be straight and tight when they are fed into the carriers and taken down from the knitting machine. Hence, it is crucial to capture an accurate reference state configuration of knitted fabric under certain pre-tension, in order to provide a meaningful comparison between numerical test and experimental test. Since it is hard to obtain the manufacturing yarn tension a priori, it is hard to establish a relationship between the number of stitches and the fabric tightness analytically. Similar to experimental attempts by Eltahan et al.~\cite{Eltahan2016} and Martinez et al.~\cite{Martinez2021} to find this relationship empirically through regression, we propose to simulate the process of fabrics being prestressed and then relaxed, then calibrate the amount of pre-tension. After the sample is constructed, an initial simulation is performed to reduce the internal rest length of the yarn, so that its configuration becomes tighter---see ~\ref{sec:5d} for more information.

Though the numbers of stitches used in numerical tests (13 along warp and 12 along weft) are different to those used in experimental tests (41 along warp and 40 along weft), we compare the dimensionless parameters, aspect ratio of fabric and unit arclength of yarn segment, to benchmark numerically obtained configurations in the reference state against manufactured samples. In addition, we calibrate relaxation stage duration by letting the numerical samples to relax in absence of external stress until the side boundaries are curvature-neutral from the untethered boundaries, as shown in Fig.~\ref{fig3a:deform_profile}. By doing so, we were able to obtain close to ground-truth configurations of all four basic weft-knitted fabrics in their reference states.  

    % Figure environment removed


%%%% Subsection 3b %%%%%%%%%%%%
\subsection{The fundamental mechanical behaviour of knitted fabrics}
\label{sec:3b}
    % Figure environment removed

    % Figure environment removed

\noindent The influence of geometric flexibility of fabric pattern on the fundamental mechanical response of knitted fabrics is well captured by our numerical tests and highlighted in Fig.~\ref{fig3b:stress_strain_warp} and Fig.~\ref{fig3b:stress_strain_weft}, which show applied uniaxial tensile stress $\sigma$ against strain $\epsilon$ in warp and weft directions respectively and the corresponding measurements validated by experimental tests. To equalise the difference in the numbers of stitches $n_w$ used in numerically generated fabrics versus those used in machine fabricated samples, we define effective tensile stress as $\sigma = \frac{F}{n_w}$. In addition, to offset variation across varying stitch patterns in fabric dimension $S_R$ along the face formed with yarn diameter $2r$, where the load is applied on, the effective tensile stress is further adjusted to $\sigma = \frac{F}{n_w 2rS_R}$. On the other hand, to equalise fabric dimension $L_R$ along loading direction in the reference state, we define tensile strain $\epsilon = \frac{L_D-L_R}{L_R}\cdot \SI{100}{\%}$ to account for normalised extension $L_D-L_R$. We see good alignment between experiment and simulation, in terms of capturing the overall ``J-shape'' curves and the relative rigidity and extensibility of four knitted fabrics. 

%    % Figure environment removed
    
Uniaxial tensile tests along the warp direction demonstrate distinct two-stage regions of deformation to failure for all fabrics. Firstly, the jersey fabric, having consistent loop contacts throughout the fabric and hence possessing the simplest geometry, behaved most rigidly among the four fabrics studied during the first stage characterised by low linear stiffness (regime 1) at the magnitude of \SI{0.1}{MPa} up to \SI{30}{\%} strain. Previous experimental studies reported observation of geometry reconfiguration as yarns slide through loop contacts and straighten to align more towards applied tensile load \cite{Sanchez2023}, and our study provides quantitative evidence for such yarn dynamics that are summarised in Sec.~\ref{sec:3c}. After this reconfiguration, the jersey fabric transitioned to a stage where the stiffness monotonically increased by up to \SI{10} times (regime 2), during which stretching of individual yarns become prominent. Statistical measurements are discussed in Sec.~\ref{sec:3c}. Secondly, among the four fabric studied, we observe the garter fabric to be the softest and most stretchable in the initial regime when deformed in the warp direction. This fabric first  undergoes a nearly linear region with the lowest slope (regime 1) at the magnitude of \SI{0.01}{MPa} up to a transitional strain magnitude of \SI{60}{\%}. This regime is followed by yarn stretching (regime 2) at a distinctively higher slope at the magnitude of \SI{1}{MPa}, which continues up to a peak strain approaching \SI{90}{\%}. In contrast, the rib fabric was initially the most rigid and least extensible structure (excluding jersey). This fabric first underwent yarn alignment with an initial slope almost two times higher than that for the garter fabric (regime 1), and with a range only up to \SI{40}{\%} strain, quickly followed by yarn stretching (regime 2) characterised by a much higher slope in the data up to failure at only \SI{60}{\%} strain. In addition, the seed fabric sustained the former loading stage (regime 1) with stiffness similar to that of the rib fabric up to \SI{50}{\%} strain, and transitioned to the latter loading stage (regime 2) with stiffness reaching the asymptotic magnitude of \SI{1}{MPa}.

We observe similar transition behaviour upon uniaxial loading along weft direction, as all fabrics are initially soft and stretchable, followed by strain hardening as geometric flexibility from the mesoscale patterns are exhausted. However, the relative rigidity and extensibility of fabrics are now different. Though the jersey fabric is again the most rigid and undertook strain-hardening the soonest at \SI{30}{\%} strain during loading regime 1, the relative variation in stiffness among the four fabrics during this loading regime is negligible. The rib fabric, previously representative of rigid behaviour under tension along warp direction, now becomes the initially softest and most stretchable under tension along weft direction, as it first undergoes yarn alignment of a (regime 1) at almost negligible magnitude up to more than \SI{100}{\%} strain, followed by yarn stretching (regime 2) at a noticeably higher slope at the magnitude of \SI{0.1}{MPa} in the data up to failure strain more than \SI{150}{\%}. Conversely, the garter fabric under tension along weft direction is the most rigid and least extensible in the initial regime (excluding jersey), as it first underwent yarn alignment with a slope almost two times higher than that for the rib fabric (regime 1), and only had a comparably small range up to \SI{50}{\%} strain. This regime is succeeded by yarn stretching (regime 2), exhibiting a markedly higher slope nearing \SI{1}{MPa} and leading to failure at nearly \SI{100}{\%} strain. In addition, the seed fabric undertook mechanical behaviour close to that of the garter fabric.

%\KB{here I suggest to comment on the non-perfect agreement between exp and simulations? What are the contributing factors? Can we also compare experimental and numerical deformed shapes of the knits?}

It is well accepted that the precise matching between reduced-order constitutive models and experimental tests on knitted fabric is challenging \cite{Vassiliadis2007, Yeoman2010, Dinh2018, Liu2017, Weeger2018, Wu2020}. Though our model recovers the general two-stage nonlinear elastic behaviour and the relative responses among the four basic weft-knitted fabrics well, we notice the limitations, particularly when capturing the behaviour when fabrics transition between regimes. There are multiple reasons for these discrepancies, such as our treatment of the yarn as a solid elastic tube, which may not precisely capture the spun fibers in the acrylic yarn. In addition, we assume a linear stretching force response in the yarn, which may not be accurate at high strains, as acrylic yarn typically softens at high strains as characterised in ~\ref{sec:6d}. We further discuss this limitation in Sec.~\ref{sec:3c}, where we collect statistical measurements of individual yarn segment stretch for all studied fabrics and loading conditions.  

%%%% Subsection 3c %%%%%%%%%%%%
\subsection{Mechanical role of yarn dynamics on fabric extensibility and anisotropy}
\label{sec:3c}
To probe into how yarn rearrangements influence the macroscopic extensibility of knitted fabrics, we measured the projection of individual yarn segments on to the loading direction of warp in Fig.~\ref{fig3c:rose_loady} and of weft in Fig.~ \ref{fig3c:rose_loadx}. With an angular increment (bin size) of \SI{5} degrees, we tracked the overall evolution of yarn segments as they aligned closer with the applied load as tensile strain increased from \SI{0}{\%} to \SI{120}{\%}. This quantitative evidence compliments experimental observation of reorientation of yarn segments to exploit geometric degrees of freedom within the connected network. Such geometric rearrangement of yarn segments rather than material stretching of yarn segments contributes to the compliant behaviour during the initial stage of fabric mechanical response. Statistical distributions of yarn segment stretch in Figs.~\ref{fig3c:hist_loady} \& \ref{fig3c:hist_loadx} further support the yarn reorientation mechanism during the initial loading on fabrics, as distribution peaks remain within the range for negligible segment stretch while fabrics are stretched until transitions occur. It is worth noting that even as fabrics transition to higher strain ranges (near or exceeding \SI{100}{\%}) and the most stretched segments approach \SI{20}{\%}, these segments account for less than \SI{20}{\%} of all yarn segments. Therefore, it is reasonable to assume linear elasticity for the majority of yarn segments as an averaged one-time calibration of the stretching stiffness for preliminary study. However, this places challenges in addressing fabric behaviour between regimes, as inhomogeneous mechanical field of segment stretch contributes to the transition. 

Previously, we also observed anisotropy from knitted fabrics, which is topology-dependent. Considering garter and rib as representative examples, we establish the former has a softer mechanical response and sustains a higher elastic strain range when subjected to tension along fabric warp direction, while the latter behaves in a stiffer manner within a lower elastic strain range when responding to tension along fabric weft direction. To probe into the influence of fabric structure on anisotropy, we begin by examining their geometric configurations in the reference state, considering yarn segment angles in Figs.~\ref{fig3c:rose_loady} \& \ref{fig3c:rose_loadx}. The garter fabric initially has less than \SI{10}{\%} of yarn segments aligning with the warp direction within a difference of \SI{10} degrees, but has more than \SI{20}{\%} similarly close yarn alignment with the weft direction. In comparison, the rib fabric has more than \SI{20}{\%} of yarn segments closely aligned with  the warp direction in the reference state, and only less than \SI{15}{\%} yarn segments aligned with the weft direction at this stage. The lower initial frequency of alignment with the loading direction provides more geometric degrees of freedom for the yarn segments reorient themselves to manifest applied stress within the hierarchical system, making the fabric more compliant under this ``unaligned'' loading direction than the orthogonal direction. Moreover, as observed in Figs.~\ref{fig3c:hist_loady} \& \ref{fig3c:hist_loadx}, for a fabric to demonstrate softer and more compliant behavior under a fixed loading condition, the shift of its peak yarn segment stretch distribution towards a higher stretch range occurs more slowly. 

    % Figure environment removed

    % Figure environment removed
    
    % Figure environment removed

    % Figure environment removed

    
%%%% Subsection 3d %%%%%%%%%%%%
\newpage
\subsection{Demonstration of the design space}
\label{sec:3d}
%\KB{this seems to be way too qualitative for my taste. You should describe what you are showing in the Fig. Which combinations are you considering? How did use your previous results to choose these combinations? Is the behavior as expected?} 

Guided by our study on the anisotropy of basic weft-knitted structures in Sec.~\ref{sec:3b}, we can purposefully explore the design space for textile-based devices to deform to desired shapes. First, we highlight how structural variation in multi-structure knitted fabrics consisting of the same numbers of stitches along fabric warp and fabric weft leads to a wide range of compliance, when they are subject to the same tensile load in Fig.~\ref{fig3d:multi_patterns}. 
Here, both asymmetric primitive (A) and symmetric primitive (B) utilise the relatively better stretchability of garter over jersey along the loading direction (fabric warp), leaving more fabric to be distributed along the orthogonal direction that can provide localised comfortability to the wearer along fabric weft. The former provides looser fit at the fabric boundary, while the latter provides looser fit at fabric middle region. 
On the other hand, asymmetric primitive (C) shows how using a less stretchable structure (rib) along the loading direction (fabric warp) potentially enhances gripping capability of a device upon actuation, as fabric quickly curls out of plane along the direction orthogonal to actuation and forms pocketed region. This can be directly applied as responsive structures to be passively actuated and textiles to provide custom fit.

In addition, we demonstrate the adaptability of our model to vary material properties at the yarn level and the generalisability of our model to create 3D configurations, both further opening up the design space of functional textiles to composites. Fig.~\ref{fig3d:composite_patterns} shows the deformation processes of two 3D primitives made of jersey throughout with (A) having more rigid materials at both ends and (B) having softer materials at both ends, both subject to bending applied through compressed tethered boundaries. This is a demonstration of direct application in soft actuators to absorb impact. Though failure is not included in the scope of this work, we apply a custom colour map in Fig.~\ref{fig3d:composite_patterns} based on stretching energy of individual yarn segments, to highlight the capability of our model to investigate micromechanical hot spots due to inhomogeneity inherent across the whole fabric. As expected, we see the outer side stretches more than the inner side upon bending. Moreover, we see that the region consisting of more rigid material stretches less than that consisting of softer material.


    % Figure environment removed

    % Figure environment removed
