%%%% Section 2 %%%%%%%%%%%%
\subsection{Discretisation of the yarn, and initial conditions for the fabric}
\label{sec:2a}
Our work builds on previous yarn-level simulations in the computer graphics literature by Kaldor et al.~\cite{Kaldor2008}. We implemented a custom code in C++, using OpenMP for multithreading, to perform the simulations. In this section we provide a mathematical overview of the methods, and then provide additional technical and computational details in ~\ref{sec:5}.

Our simulation can handle an arbitrary number of individual yarns of diameter
$d$. The centerline of each yarn is represented by a cubic B-spline with $N$
segments. We denote $s\in \Omega = [0,N]$ to be a dimensionless coordinate
along the yarn, where the $N+1$ control points of the spline are located at
$s=0,1,2, \ldots, N$. The cubic B-spline basis function is given by
\begin{equation}
  B(s) = \begin{cases}
    \tfrac23 - \tfrac12 |s|^2 & \qquad \text{if $|s|<1$,} \\
    ||s|-2|^2 & \qquad \text{if $1\le |s| < 2$,} \\
    0 & \qquad \text{otherwise.}
  \end{cases}
\end{equation}
A family of basis functions is then given by $b_k(s) = B(s-k)$ for $k\in \mathbb{Z}$. The yarn
is then defined as
\begin{equation}
  \vy(s,t) = \sum_{k=-1}^{N+1} b_k(s) \vq_k(t),
  \label{eq:ys}
\end{equation}
where $\vq_k(t)$ are time-dependent three-dimensional functions. In
Eq.~\eqref{eq:ys}, the sum must run from $-1$ to $N+1$ in order to describe all
piecewise cubics in $C^2[0,N]$~\cite{suli_textbook}, making for $m=N+3$ terms
in total. Hence, the yarn is described by $3m=3(N+3)$ degrees of freedom stored
in a vector $\vq=(\vq_{-1},\vq_0,\ldots,\vq_{N+1})$. The velocity of the yarn
is given by
\begin{equation}
  \vv(s,t) = \dot{\vec{y}}(s,t) = \sum_{k=-1}^{N+1} b_k(s) \vqd_k(t),
  \label{eq:vs}
\end{equation}
where a dot represents a derivative with respect to $t$. The velocity of the
yarn is analogously described by a $3m$-component vector $\vqd$. The descriptions
in Eqs.~\eqref{eq:ys} \& \eqref{eq:vs} effectively decouple the spatial
and temporal dependence of the yarn motion. In its rest state, the yarn has
equal arc length $l$ between each pair of control points.

We initialise the spline at $t=0$ by specifying an initial parametric curve for its shape. Knitted fabrics are generated from interlocking loop units that are formed stitch by stitch in the weft
(horizontal) and warp (vertical) directions. Depending on the direction along
which a continuous yarn is fed in, knitted fabrics fall into two categories: weft knits and warp knits. We focus on weft-knitted fabrics in this work, because of the current interest in leveraging commercially available weft knitting machines (V-bed knitting machines) to create complex 3D devices.

A typical loop geometry that we employ is~\cite{Vassiliadis2007},
\begin{equation}
  \vy_\text{p}(w) = \left(
    \begin{array}{c}
      \lambda_x \Big( w+\sin(\pi w) \Big) \\
      \lambda_y \cos(\frac{\pi w}{2}) \\
      \lambda_z \cos(\pi w)
    \end{array}
  \right)
  \label{eq:basic_param}
\end{equation}
where $\lambda_x, \lambda_y, \lambda_z$ are scaling parameters in each
dimension that may vary independently to match target aspect ratio of generated
samples. Using the coordinate range $w\in
[w_\text{start},w_\text{end}]=[-\tfrac{\pi}{2},\tfrac{\pi}{2}]$ in
Eq.~\eqref{eq:basic_param} yields a single loop as shown in
Fig.~\ref{fig1:knit_assembly}(A). In general, $\|d\vy_\text{p}/dw\|$ will not be
constant, so that the arc length along each parametric curve will not increase
at a constant rate in $w$. Therefore, to initialize the B-spline formulation,
our simulation computes the arc length along the curve as a function of $w$,
\begin{equation}
  A(w) = \int_{w_\text{start}}^w \left| \frac{d\vy_\text{p}}{dw} \right| dw,
\end{equation}
which is evaluated using composite Gaussian quadrature. The rest arc length is
computed as $l= A(w_\text{end})/N$. Using Ridders' root-finding method, a
sequence of values $w_0=w_\text{start}, w_1, w_2,\ldots, w_N=w_\text{end}$ are
found such that $A(w_k) = kl$. These set the values of the control points in
the B-spline basis formulation, so that $\vy(k,0) =\vy_\text{p}(w_k)$, giving
$N+1$ vector equations in total. In addition, the direction of the spline at
$s=0,N$ is chosen to match the direction of the parametric curve, giving an
additional two vector equations. This gives a total of $N+3$ linear vector
equations that can be solved as a linear system to determine the $\vq_k(0)$.

\subsection{Dynamics of a yarn}
\label{sec:2b}
We consider the Lagrangian formulation to describe the dynamics of a single
yarn with $m$ control points as
\begin{equation}
  \frac{d}{dt} \left( \nabla_{\vqd_k} T \right) + \nabla_{\vq_k} V + \nabla_{\vqd_k} D =0,
  \label{eq:lagr}
\end{equation}
where $T$ is the kinetic energy, $V$ is the potential energy and $D$ is the
damping energy. The kinetic energy of the yarn is
\begin{equation}
  T(\vqd) = \frac{\rho l}{2} \int_\Omega \vv^\Trans \vv \, ds,
\end{equation}
where $\rho$ is the mass density. By referencing Eq.~\eqref{eq:lagr}, we must
evaluate
\begin{equation}
  \nabla_{\vqd_k} T = \rho l \int_\Omega (\nabla_{\vqd_k} \vv^\Trans) \vv \, ds.
  \label{eq:dkint}
\end{equation}
We define the unit mass matrix $M\in \R^{m\times m}$ with components
\begin{equation}
  M_{kj} = \int_\Omega b_k(s) b_j(s) \, ds,
  \label{eq:ke_integ}
\end{equation}
which corresponds to integrating a product of two B-spline basis functions
$b_k$ and $b_j$. Since each basis function is non-zero over four intervals,
$M_{kj}=0$ if $|k-j|>3$, and therefore $M$ is a banded matrix with three
superdiagonals and three subdiagonals. The matrix $M$ remains constant throughout the
simulation and can be precomputed. Therefore Eq.~\eqref{eq:dkint} becomes
\begin{equation}
  \frac{d}{dt} \left( \nabla_{\vqd_k} T \right) = \rho l \sum_{k=-1}^{N+1} M_{jk} \vqdd_{j}.
    \label{eq:Mqdd}
\end{equation}
The potential energy of a yarn includes several terms as
\begin{equation}
    V = V^s(\vq)+V^b(\vq)+V^g(\vq),
    \label{eq:pot_en}
\end{equation}
representing energy due to stretching, bending, and gravity. With the assumption of linear elasticity, the stretching energy is given by
\begin{equation}
	V^{s}(\vq) = \frac{E^s A l}{2} \int_\Omega \Big( \frac{\|\Vec{y'}\|}{l}-1 \Big) ^2 ds,
\end{equation}
where $E^s$ is the tensile stiffness and $A=\pi d^2/4$ is the yarn cross-sectional area. Here, the prime superscript represents a partial derivative with respect to $s$. The elastic energy of the yarn due to bending is formulated as
\begin{equation} \label{eq:bend_en}
	V^{b}(\vq) = \frac{E^b I l}{2} \int_\Omega \kappa^2 ds,
\end{equation}
where $E^b$ is the bending stiffness, $I$ represents moment of inertia of the yarn cross-section, and the local curvature $\kappa$ is defined as
\begin{equation} \label{eq:curvature}
	\kappa = \frac{\|\Vec{y}' \times \Vec{y}''\|}{\|\Vec{y}'\|^3}.
\end{equation}  
The gravitational potential energy is
\begin{equation}
  V^g(\vq) = \rho l \int_\Omega \vy^\Trans \vec{g} ds,
\end{equation}
where $\vec{g}$ is the gravitational acceleration. By referencing
Eq.~\eqref{eq:lagr}, we need to evaluate $\nabla_{\vq_k} V^s(\vq)$,
$\nabla_{\vq_k} V^b(\vq)$ and $\nabla_{\vq_k} V^g(\vq)$. Similar to evaluating
Eq.~\eqref{eq:Mqdd}, part of these complicated integrals can be precomputed and
the rest can be accurately determined using quadrature.

The damping energy in Eq.~\eqref{eq:lagr} has several components. One component
is given by
\begin{equation} \label{eq:damp_en_g}
	D^\text{iso}(\vqd) = k_{g} \int_\Omega \vv^\Trans \vv \, ds,
\end{equation}
which creates a global drag force on the yarns. In the experiments, the knitted
samples are primarily in a regime where the forces are in quasi-static
equilibrium, since the yarns have sufficient internal damping to remove any
transient inertial effects. The drag force in Eq.~\eqref{eq:damp_en_g}
serves as a simple proxy for the internal damping and accomplishes the same
goal, ensuring that the inertial effects are removed.


\subsection{Yarn--yarn interactions}
\label{sec:2c}
The contact forces between two yarns (or between two different sections of the same yarn) are critically important for simulating the knitted fabric. Without loss of generality, let $s$ and $\tilde{s}$ be coordinates ranging from 0 to 1 over two spline segments $i$ and $j$ with a contact. The energy contribution is given by
\begin{equation}
  V^\text{con}_{i,j} = l^2 \int_0^1 \int_0^1 f\left( \frac{\| \vy_i(\tilde{s}) - \vy_j(s) \|}{d} \right) \,ds \, d\tilde{s},
  \label{eq:contact_integ}
\end{equation}
where $\vy_i$ and $\vy_j$ are the spline positions on the two segments, and
\begin{equation}
  f(\delta) =
  \begin{cases}
    k (\delta-1)^2 & \qquad \text{if $0 \le \delta <1$,} \\
    0 & \qquad \text{if $\delta \ge 1$,}
  \end{cases}
\end{equation}
where $k$ is a spring constant. In addition a damping term can be incorporated,
with the form
\begin{equation} \label{eq:damp_en_c}
  D_{i,j}^\text{fri} = l^2 \int_0^1 \int_0^1 \Big( k_{dt}\| \Delta\vv_{ij}\|^2 - (k_{dt}-k_{dn})(\hat{\Vec{n}}^\Trans_{ij} \Delta\vv_{ij})^2 \Big) \,ds\, d\tilde{s},
\end{equation}
which approximates the effect of frictional sliding. Here $\Delta\vv_{ij}$ is
the relative velocity and $\hat{\Vec{n}}_{ij}$ is a normal vector in the
collision direction. The constants $k_{dt}$ and where $k_{dn}$ set the size of the effect in the tangential and normal directions, respectively.

Unlike the integrals considered in the previous section, it is difficult to
evaluate the integrals in Eqs.~\eqref{eq:contact_integ} \& \eqref{eq:damp_en_c}
efficiently and accurately. Since the integrands are non-smooth, and are
only non-zero in localised patches in the $(s,\tilde{s})$ space, Gaussian
quadrature will often give imprecise results. Because of this, we replace each
integral with sum over $n$ discrete values $\{s_1,s_2,\ldots,s_n\}$ and
$\{\tilde{s}_1,\tilde{s}_2,\ldots,\tilde{s}_n\}$ so that
\begin{equation}
  V^\text{con}_{i,j} = l^2 \sum_{\alpha=1}^n \sum_{\beta=1}^n f\left( \frac{\| \vy_i(s_\alpha) - \vy_j(s'_\beta) \|}{\dcon} \right),
  \label{eq:contact_sum}
\end{equation}
where $s_\alpha=(2\alpha-1)/n$ and $\tilde{s}_\beta= (2\beta -1)/n$. This is
equivalent to modeling contact between a discrete set of spheres, evenly
distributed along each spline segment. Similarly, Eq.~\eqref{eq:damp_en_c} is
replaced with
\begin{equation}
  D^\text{fri}_{i,j} = l^2 \sum_{\alpha=1}^n \sum_{\beta=1}^n \Big( k_{dt}\| \Delta\vv_{\alpha\beta}\|^2 - (k_{dt}-k_{dn})(\hat{\Vec{n}}^\Trans_{\alpha\beta} \Delta\vv_{\alpha\beta})^2 \Big),
\end{equation}
where $\Delta \vv_{\alpha\beta}= \vv_i(s_\alpha) - \vv_j(\tilde{s}_\beta)$ and
$\hat{\Vec{n}}_{\alpha\beta}$ is a normal vector pointing in the direction of
$\vy_i(s_\alpha)-\vy_j(s_\beta)$. The diameter $\dcon$ of the contact spheres
is chosen to be slightly larger than the yarn diameter $d$, so that the
envelope made by the spheres more precisely matches the profile of the
yarn---see \ref{sec:csphere} for more information.

To detect the adjacent spheres efficiently, the spheres are binned into an
equally-spaced rectangular grid that covers all of the yarns in the simulation.
For a given sphere, finding adjacent spheres is performed by iterating over all
spheres in nearby grid boxes, resulting in a constant, $O(1)$ computation time
per sphere. Even with this optimisation, we typically find that detecting and
computing the contact forces is the most computationally expensive step
of our simulations.

\subsection{Loop topology and fabric pattern}
\label{sec:2d}
A typical V-bed knitting machine consists of front and back beds with arrays of needles. A carriage traverses these beds, actuating the knitting needles with cams. Concurrently, yarn carriers (moved by the carriage in our case) feed yarns to be caught by needles to form stitches. Stitches formed on the front bed resemble ``knit'' stitches, while those on the back bed are akin to ``purl'' stitches in hand knitting. Based on these two basic manufacturing instructions, ``knit'' and ``purl,'' we can define a set of four representative weft-knit structures as shown in Fig.~\ref{fig2:knit_patterns} (A) jersey (all knits or purls), (B) garter 1 by 1 (knits and purls alternating every row only), (C) rib 1 by 1 (knits and purls alternating every column only) and (D) seed 1 by 1 (knits and purls alternating every row and column). 

% Figure environment removed

To be consistent with the manufacturing process~\cite{Leong2000, Spencer2001}, we propose a simple pipeline to assemble full-scale weft knitted fabrics. Fig.~\ref{fig1:knit_assembly} illustrates the process of creating a geometric model for jersey, the simplest weft knitted fabric, since the topology of all contacts within the fabric is consistent. After generating a loop unit along the standardised parametric function, we assemble a row of loops by assigning the end positions of the row along the $x$ axis (fabric weft direction), similar to how rows of stitches are formed along horizontal needle beds on the V-bed knitting machine. Secondly, we translate each row along the $y$ axis (fabric warp direction) with assigned distance from the central axis of the pattern. Note that modifications to the geometric model are required to adjust for spacing between alternating rows and/or columns in order to create more complex configurations beyond the jersey. We proposed using an additional sinusoidal function to parameterise the $z$ direction (fabric thickness direction) in Eq.~\eqref{eq:basic_param}, in order to alternate wavelength and apply a phase shift to accommodate for varying topology~\cite{Markande2020}. In addition, we specify smooth spiral curves adopting a generalised helicoid surface~\cite{Piuze2011, Wadekar2020} equivalent to extra yarns used to cast on and bind off the fabric at the top and bottom boundaries in manufacturing, in order to prevent fabric from unravelling upon free boundary conditions. These spiral curves can be described by 

%\KB{how did you arrive at these formulas? If these come from literature, please add a citation}

\begin{equation}
  \vy_\text{s}(w) = \left(
    \begin{array}{c}
      \lambda_{x,\text{sp}} (0.25 w) \\
      \lambda_{y,\text{sp}} \sin \Big( \gamma \pi (w+d) \Big) \\
      \lambda_{z,\text{sp}} \Big( 1 - \cos(0.5\pi (w+d))
    \end{array}
  \right),
  \label{eq:basic_param_spiral}
\end{equation}
where $\lambda_{x,\text{sp}}$, $\lambda_{y,\text{sp}}$ and $\lambda_{z,\text{sp}}$ being scaling factors carefully selected to produce tight spirals in order to minimise boundary effects, $\gamma=0.5$ for patterns with no variation in contact topology with respect to alternating columns (jersey and garter), $\gamma=0.25$ for patterns with variation in contact topology with alternating columns (rib and seed) and $d$ being the assigned translation distance between rows and it should not exceed the upper bound in order to ensure all rows are attached in initial configuration. At last, we connect all loose ends of yarns through an interpolation scheme to form a complete fabric solely composed of one continuous yarn.

% Figure environment removed


\subsection{Mechanics-centered simulation framework}
\label{sec:2e}
After generating knitted fabrics from previously described geometric model, we performed material characterisation tests to calibrate physical parameters relevant to the simulation summarised in Table \ref{tab1:calibration}. We then apply loading through tethered springs at a sufficiently low loading rate that is critically damped to ensure numerical stability and convergence---see \ref{sec:5} for more details.

\begin{table}[b]
  \caption{Calibration of physical parameters relevant to the simulation of knitted fabrics.}
  \label{tab1:calibration}
  \centering
  \begin{tabular}{llll}
    \hline
    Symbol & Parameter & SI in measurements & SI in simulation tests\\
    \hline
    $\rho$ & mass per unit length & \SI{0.077}{g/m} & \SI{7.7e-4}{g/cm} \\
    $l$ & unit length per stitch & \SI{7.23}{mm} & \SI{0.723}{cm} \\
    $r$ & yarn radius & \SI{524}{\micro \meter} & \SI{0.0524}{cm} \\
    $A$ & yarn cross-sectional area & $A = \pi r^2$ & \SI{8.63e-3}{cm^2} \\
    $I$ & yarn moment of inertia & $I = \pi r^4 / 4$ & \SI{5.92e-6}{cm^4} \\
    $E^s$ & yarn tensile stiffness & \SI{79.0}{MPa} & \SI{7.9e8}{g/cm s^2} \\
    $E^b$ & yarn bending stiffness & \SI{0.249}{MPa} & \SI{2.49e4}{g/cm s^2} \\
  \end{tabular}
\end{table}
