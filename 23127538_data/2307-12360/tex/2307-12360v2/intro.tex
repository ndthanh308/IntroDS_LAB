%%%% Section 1 %%%%%%%%%%%%
\label{sec:1}

Knitted fabrics are hierarchical structures that build up from yarns at the microscale, to stitch pattern structures at the mesoscale, and finally to three-dimensional fabrics at the macroscale. With yarns being the primary building blocks to dominate the physics and design of fabrics, the separation of these two scales (i.e., yarn-level and fabric-level) at the structural level not only causes a range of interesting physical phenomena \cite{Poincloux2018, Poincloux2018a} to arise, but also provides a huge design space for functionalities in knitted fabrics beyond what their constitutive materials can achieve \cite{Rout2022}. Knitted fabrics, analogous to some architected \cite{Jiang2016, Moestopo2023, Mistry2023} and bio-inspired materials \cite{Ma2017, Nepal2023}, represent nonbiological examples of a nonlinear elastic response characterised by a ``J-shape'' curve, as the fabric transitions from bending energy dominant region to stretching energy dominant region under uniaxial tension. The presence of mesoscale stitch patterns enables the fabric to take on substantial tensile stress elastically. This behaviour is attributed to the low strains on individual yarn segments and the dynamics of yarn alignment with external load, which offer geometric degrees of freedom. The distinctively compliant behaviour of knitted fabrics makes them stand out as excellent scaffolds for wearable devices \cite{Polygerinos2015, Cappello2018, lee2018knit, Granberry2019, fan2020machine, Wicaksono2020} and soft robotic actuators \cite{Connolly2019,Sanchez2021}, where large deformation and flexible morphing without material damage is desired. The anisotropy at the fabric mesoscale has been exploited to fine tune actuation of such devices \cite{Ahlquist2017,Luo2022,Sanchez2023}, where carefully selected structures can be spatially varied across the fabric, such that the fabric can shape morph to comply with complex geometries. Multifunctional knitted fabrics can be created through embedding functional yarns into conventional knit structures, to further enlarge the design space of knitted fabrics to morphing structures \cite{Abel2013, Han2017} and to serve as light and touch sensors \cite{Albaugh2019}, pressure sensors \cite{Luo2021}, electronic interfaces \cite{Wicaksono2017} and electronic skins \cite{Pei2019} in an exciting new domain of smart materials to mimic and embed intelligence. 

Currently, intuition-led strategies remain the primary approach to design devices made of knitted fabrics. This paradigm poses limitations on exploring the design space due to high machinery costs, training costs and material waste. A generalisation of these application-driven designs for yarn geometries, fabric structures and material variations has not yet been established. Early theoretical work on knitted structures started from defining the characteristic unit cells to represent stitch patterns and predominantly assumed homogeneity due to periodicity of unit cells. Starting from a three-dimensional parameterisation of the jersey knit pattern \cite{Peirce1947} to curvature augmented model \cite{Leaf1955}, followed by energy minimisation model \cite{Munden1959}, most geometric models of knitted fabrics are constructed through superposition of cosine and sine curves due to the smoothness and periodicity of these shape functions. With the development of spline basis functions, we can discretise yarns with sufficient accuracy and such yarn-based models \cite{Bergou2008, Kaldor2008, Cirio2014, Leaf2018} have key advantages compared to coarse-grained models \cite{Terzopoulos1987, Baraff1998, Breen1994, Yeoman2010} and homogenised models \cite{NarainArminSamiiJamesO2012, Yuksel2012, Liu2017, Dinh2018, Weeger2018, Wu2020, Sperl2021}, due to their capability to (i) capture mechanical behavior originating from first principles via yarn dynamics, (ii) provide quantitative measurements of geometric nonlinearity arising across scales, and (iii) vary the spatial distribution of stitch patterns and material properties of yarns to form targeted 2D and 3D configurations, all while not constraining the extensibility of individual yarn segments affinely.

To begin, we adopt a yarn-based model with cubic order spline basis functions \cite{Kaldor2008} that was originally applied in computer graphics to animate cloth in a qualitatively realistic manner. We extend this model to provide physical insights into macroscopic inhomogeneity, anisotropy and cross-scale mechanisms. We investigate the mechanical responses of representative weft-knitted samples at compatible scale to machine knitted experimental samples that are systematically characterised. Our numerical approach is implemented through fully dynamic formulation of the governing equation of motion at yarn level, integrated explicitly with a high-order adaptive scheme. A key aspect of our numerical procedure is the introduction of relaxation stage similar to experimental procedure \cite{Martinez2021} before the application of external tensile forces, to account for residual stress that is inherent in the knit fabrication process. This inherent residual stress has been experimentally investigated by the textile community as one of the dimensional properties of knitted fabrics \cite{Allan1983, Amreeva2007, Wei2011} and poses a typical challenge in generating accurate reference state of knitted fabric, if obtained purely from geometric description of its structure. After initialisation, we apply uniaxial tensile loads quasi-statically to stretch the fabric samples in simulation, up to strain thresholds compatible with experimental set-up post initial cycles. Our designed experimental procedure is compatible with bias-extension tests typically conducted to characterise textiles \cite{Cao2008}. Though experimental validation is carried out on limited sets of weft-knitted samples, our selection of stitch pattern each represents a different topological group \cite{Markande2020} that can be further explored in a more systematic approach. 

Leveraging parameterisation at yarn level, we can quantitatively investigate cross-scale mechanisms contributing to nonlinear elasticity and anisotropy of knitted fabrics. Since we treat the dynamics at a continuous yarn as the governing mechanics, allowing for the prediction of local, spatial evolution of bending and stretching energy, we can explain mechanical responses of representative stitch patterns by statistical measurements of yarn dynamics. As we can directly compute measurements of energy, deformation and alignment with regard to each yarn segment, we can predict micromechanical hot spots. 

Another main aspect of the present work is to propose design of functional and composite textiles in an instructed manner compatible to manufacturing procedure, allowing our model to be adapted for systematic digital generation of knitted configurations for targeted mechanical responses. Guided by gained physical insights, we can purposely design 2D and 3D configurations that possess localised mechanical responses through spatial distribution of stitch structures and material properties. The direct applications of mechanically programmed knitted fabrics in responsive structures, wearables and soft robotics are emphasised in this work. Though demonstrated design examples focus on one length scale, our model can also be deployed to various length scales and other fibrous  networks, opening up design for new class of textile-based mechanical metamaterials and programmable materials. 

This paper is organised as follows: we first summarise theoretical background of the geometric model and the mechanical model; in addition to providing validated numerical results and motivating examples for applications, we demonstrate that structural properties, such as varying topological description of fabric patterns and varying spatial distribution of fabric patterns can effectively adjust the mechanical behaviours of knitted fabrics, not necessarily modifying the material properties in results; moreover, we detail the development and implementation of computational model, the material selection and characterisation in following sections. 

