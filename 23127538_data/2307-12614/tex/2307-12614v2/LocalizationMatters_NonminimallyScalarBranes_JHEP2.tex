\documentclass[a4paper,11pt]{article}
%\pdfoutput=1 % if your are submitting a pdflatex (i.e. if you have images in pdf, png or jpg format)
\usepackage{jheppub}
\usepackage[T1]{fontenc} % if needed
\usepackage{amsmath}
\usepackage{multirow}



\title{\boldmath Localization of matters coupled nonminimally to gravity on a scalar thick braneworld}

\author{Muhammad Taufiqur Rohman,}
\author{Triyanta,}
\author{Agus Suroso}
\affiliation{Theoretical High Energy Physics Research Division, Faculty of Mathematics and Natural Sciences, Institut Teknologi Bandung,\\ Jalan Ganesha 10 Bandung 40132, Indonesia}

\emailAdd{m.taufiqur25@gmail.com}
\emailAdd{triyanta@itb.ac.id}
\emailAdd{suroso@itb.ac.id}

\abstract{
	We study the localization of matter that is coupled nonminimally to gravity on a thick braneworld model generated by a scalar bulk. We have conducted a review of two models of scalar thick branes. The natural mechanism is used to analyze the localization of the fields.
	Without losing the point of field localization, we examine the asymptotic behavior of the warp function on $z$ towards infinity. 
	The massless and massive modes of the nonminimally coupled scalar field are localized on the brane. In general solution, if the coupling is set to be minimal, the scalar field will be localized for the massless mode.
	In a case of nonminimally coupled vector field, the massless mode in model 1 can be localized.
	For spinor field, we consider the Ricci scalar as representing gravity as a scalar field, since spinors can be localized when coupled as a spinor-scalar field system. 
	In model 1, we found that the spinor is localized for the massless and massive modes. Whereas in model 2, we only get the localization of the massless spinor field.
	\\ 
}

\keywords{Large Extra Dimensions, Field Theories in Higher Dimensions.}


\begin{document} 
	\maketitle
	\flushbottom
	
	\section{Introduction}
	Braneworld is an extra-dimensional model that assumes a 4-dimensional universe with its matter confined within a hypersurface called a brane, while gravity is free to propagate in the extra dimension called a bulk. In this term, the matter fields in the universe should be localized on the brane. 
	The localization of gravity and various bulk matter fields is a crucial issue in braneworld theory. To recover the effective 4-dimensional gravity, the gravity zero mode must be localized on brane. In order to build up the standard model, various bulk matter fields should be localized on the brane through a natural mechanism.
	The localization mechanism arises from the reduction of 5-dimensional into 4-dimensional action, that requires finite values of normalization and mass equations throughout the extra coordinate. This is an important issue in braneworld theory since not all types of fields can be localized.
	There are two types of brane models, namely thin brane and thick brane, which we will explain later.
	As an example, some types of fields in the Randall-Sundrum (RS) model \cite{Randall1999a,Randall1999b} cannot be proven to be localized on the brane \cite{Bajc2000}. However, through an extra coordinate transformation, the Modified Randall-Sundrum (MRS) model shows better localization properties for these fields compared to the RS model \cite{Jones2013,Wulandari2017,Wulandari2019}.
	
	Many models, such as the RS, MRS, and other RS-like, are thin brane models. The brane is highly ideal because its thickness is minimal.
	They are described as a Dirac delta function in the Einstein's equation located at a certain point where the brane exists in the extra coordinate $y$.
	There are two types of RS models, namely the RS I model which defines the TeV brane (visible brane where all Standard Model particles are confined, at $y = \pi r_c$) and the Planck brane (hidden brane, at $y=0$), which are located in a compactified extra spatial dimension on an $S^1/\mathbb{Z}_2$ orbifold with a radius $r_c$. They are connected exponentially through the warp factor without introducing any new fields beyond the standard model as a solution to the hierarchy problem  \cite{Randall1999a}.  
	The other, the RS II model defines Planck brane at $y = 0$ with the Standard Model particles are confined, and the TeV brane is shifted to position $r_c \rightarrow \infty$ (non-compact extra dimension) \cite{Randall1999b}.
	Then, the MRS is a brane model with an extra coordinate transformation, $dy = e^{-k|z|} dz$, such that the proper distance for a path going from $z = 0$ to $z = \infty$ perpendicular to the brane at $z = 0$ is finite, $s = \int_{0}^{\infty} e^{-kz} dz = \frac{1}{k}$.
	Beside the better field localization, the finiteness of proper distance distinguishes the MRS from RS model with an infinite proper distance  \cite{Jones2013}.
	The localization properties of the interacting fields in the MRS model have been studied, such as scalar-vector, vector-vector, and spinor-vector configuration systems \cite{Wulandari2017}. The localization of the Yukawa interaction field in the MRS model has also been discussed \cite{Wulandari2019}.
	
	On the other hand, a thick brane is a more realistic model developed based on the existence of a minimal length scale in fundamental physics called the Planck scale. 
	In the thick brane, it is not necessary to provide the junction conditions for the gravitational field equation. The warp factor solution is a smooth function \cite{Liu2018}.
	The thick brane model can be constructed from any 5-dimensional gravity theory, which conducts the smooth warp factor.
	From the gravitational action of the brane, a set of field equations can be derived. Then, we can determine or find several parameters including the warp factor function. Thereafter, the localization of the Standard Model matters from the warp factor can be examined. Using the widely used field localization mechanism on thick brane, it is necessary to redefine the field in such that the field equation in extra coordinate is obtained and therefore the solution.
	Many models have been developed, such as thick brane models generated by pure curvature \cite{Liu2008,Guo2013,Herrera-Aguilar2010}, scalar bulk \cite{Liang2009,Bazeia2009,Gremm2000}, vector \cite{Geng2016}, and spinor field \cite{Dzhunushaliev2011}. The thick brane models are not only formulated in standard theory of gravity, but have also been extensively developed in modified gravity, such as in $f(R)$ \cite{Bazeia2014}, $f(R,T)$ gravity \cite{Bazeia2015,Gu2017,Rohman2021}, etc. 
	
	The nonminimally coupled field to gravity is an intriguing topic of discussion in quantum field theory and cosmology. As a modification of general relativity, it provides insights into cosmological and theoretical concerns, such as the late-time cosmic acceleration, cosmological inflation with boson as inflaton \cite{Bezrukov2008}, dark energy-dark matter interaction \cite{Harko2014}, and so on. 
	In the scope of extra dimension, several models of braneworld with nonminimally coupled fields to gravity have been discussed.
	The localization of matters on a nonminimally coupled bulk scalar braneworld in the scenario of (thin) RS II have been investigated in Refs \cite{Farakos2005,Farakos2006}.
	The scenario of a thick brane with nonminimal coupling between the scalar bulk and gravity has been examined. Since it is difficult to find a warp function solution analytically, a numerical method with asymptotic behavior analysis is used to obtain an approximate solution of $A(z)$. The research explores the impact of the nonminimal coupling constant $\xi$ on the structure of thick branes and the localization of gravity, fermions, scalars, and vectors \cite{Guo2012}.
	The localization of the matter field in the sine-Gordon and Double sine-Gordon thick brane models has been discussed. The bulk mass and the nonminimal coupled terms are taken into account for the action and the implications for the potential in the Schr\"odinger-like equation are obtained \cite{Moazzen2017}.
	
	Our research objective is to investigate how the nonminimally gravity-coupled terms impacts the localization of matter fields, namely scalar field with $\xi R \phi^2$, vector field with $\lambda R A_M A^M$, and spinor fields with $\eta \bar{\Psi} R \Psi$, on the two cases of scalar thick braneworld model proposed in \cite{Liang2009}. 
	The writing structure is given as follows:
	Section 2 reviews the formulation of the scalar thick brane system and its solutions, with two examples involving a superpotential.
	Section 3-5 examine the localization of nonminimally coupled scalar, vector, and spinor fields, respectively, with gravity on the scalar thick brane.
	Section 6 presents the conclusions.
	
	
	\section{Brief on a thick brane generated by scalar bulk}
	In this section, we will review a thick brane generated by a scalar field, which aims to obtain a gravitational field solution in the form of a warp factor that relates the extra dimension to the brane.
	Consider a thick brane model with 5-dimensional scalar bulk $\phi (x^M)$. The gravitational action is given by
	\begin{equation}\label{S-grav}
		S_\text{grav} = \int d^5 x \sqrt{-g} \left[-\frac{1}{4} R + \frac{1}{2} g^{MN} \partial_M \phi \partial_N \phi - V (\phi) \right] .
	\end{equation}
	The first term is the Hilbert-Einstein action in 5-dimensional spacetime, followed by the bulk scalar field lagrangian density with potential $V (\phi)$.
	A 5-dimensional background spacetime is defined by the following line element with $y$ extra-coordinate,
	\begin{equation}\label{metrik-y}
		ds^2 = e^{2A(y)} \tilde{g}_{\mu \nu} dx^\mu dx^\nu - dy^2 ,
	\end{equation}
	where $ A(y) $ is the function of the warp factor and $ \tilde{g}_{\mu \nu} $ is the brane's metric tensor. 
	The $M,N$ indices represent 5-dimensional bulk coordinates, and $\mu, \nu$ indices represent the 4-dimensional brane coordinates. 
	The scalar field is supposed to be a function of extra-coordinate only, $\phi = \phi (y)$. Therefore,
	by varying the action with respect to scalar field and metric tensor, the bulk scalar field and Einstein's equations are obtained, respectively,
	\begin{equation}
		\phi'' + 4A' \phi' - \frac{d V}{d \phi} = 0 ,
	\end{equation}
	\begin{equation}\label{Einstein-skalar}
		A'' = - \frac{1}{3} \phi'^2 , \quad A'^2 = \frac{\phi'^2}{12} - \frac{V}{6} ,
	\end{equation}
	where prime denotes a derivative with respect to the extra coordinate.
	To make it easier to get field solutions, the first-order formalism is used \cite{DeWolfe2000}. Given a scalar field potential $ V(\phi) $ related to $ W(\phi) $, called a superpotential function, as follows.
	\begin{equation}\label{W}
		V(\phi) = \frac{1}{8} \left( \frac{dW (\phi)}{d \phi} \right)^2 - \frac{2}{3} W^2(\phi),
	\end{equation}
	where $ W(\phi) $ as an arbitrary function of $ \phi $. 
	From the Einstein's equations (\ref{Einstein-skalar}) and the given potential (\ref{W}), two first-order differential equations are obtained,
	\begin{align}
		\phi' & = \pm  \frac{1}{2} \frac{dW(\phi)}{d \phi} , \label{6} \\
		A'(y) & = \pm \frac{1}{3} W(\phi(y)). \label{7}
	\end{align} 
	If an explicit superpotential $ W(\phi) $ is given in (\ref{6}), then the solution of scalar field $ \phi (y) $ is obtained. This solution conducts the second equation (\ref{7}) to get the warp function $A(y)$.
	As in Ref \cite{Liang2009}, we consider two models each obtained from a given superpotential.
	
	\subsection{Model 1: $ W(\phi) = 3a \sinh (b \phi) $}
	Consider a superpotential $ W(\phi) = 3 a \sinh (b \phi) $, where $ a $ and $ b $ are constant parameters. Using equations (\ref{6})-(\ref{7}), the warp function and bulk scalar solutions in $ y $-coordinate are obtained \cite{Liang2009,Bazeia2009}
	\begin{align}
		A(y) & = - \frac{1}{3b^2} \ln \left[\sec^2 \left(\frac{3}{2} ab^2 y \right) \right], \\ 
		\phi (y) & = \sqrt{3} \ln \left[ \sec \left(\frac{ay}{2}\right) + \tan \left(\frac{ay}{2}\right) \right] .
	\end{align}
	By performing the extra-coordinate transformation $ dz = e^{-A(y)} dy $, and the parameter $b^2$ is set to $\frac{1}{3}$ for simplicity, the relation between the $y$ and $z$ coordinates is
	\begin{equation}
		z(y) = \int \sec^2 \left(\frac{1}{2} ay\right) dy = \frac{2}{a} \tan \left(\frac{ay}{2}\right).
	\end{equation}
	Taking the inverse and substituting $y$ into solutions $A(y)$ and $\phi(y)$, the warp function and the bulk scalar become
	\begin{align}
		A(z) & = - \ln \left(1 + \frac{a^2 z^2}{4} \right), \label{warp-1}\\ 
		\phi (z) & = \sqrt{3} \ln \left( \sqrt{1 + \frac{a^2 z^2}{4}} + \frac{az}{2}  \right) . \label{bulk-1}
	\end{align}
	
	\subsection{Model 2: $ W(\phi) =  3 a \sin (b \phi) $}
	Another model, consider a superpotential $ W(\phi) =  3a \sin (b \phi) $ where $b$ is a constant parameter. In the same way, using the equations (\ref{6})-(\ref{7}), the extra-coordinate transformation, and the parameter $b^2$ is set to $\frac{2}{3}$, it leads to the solutions of the warp function and the bulk scalar in $ z $-coordinate \cite{Liang2009,Bazeia2009} 
	\begin{align}
		A(z) & = - \ln \left( q \sqrt{1 + \frac{a^2 z^2}{q^2}} \right) , \label{warp-2} \\
		\phi (z) & = \sqrt{\frac{3}{2}} \arcsin \left[\tanh \left({\rm arcsinh} \left(\frac{az}{q} \right) \right)\right], \label{bulk-2}
	\end{align}
	where $q$ is a positive integration constant.
	From the consideration of these two scalar superpotentials, the localization of the matter fields coupled nonminimally with gravity in the scalar thick brane will be examined.
	
	\section{Localization of nonminimally coupled scalar field}
	In this section, we will investigate the localization of the scalar field coupled nonminimally to gravity on a thick brane generated by scalar bulk. The action of nonminimally coupled scalar matter leads to the field equation, called Schr\"odinger-like equation. From the scalar field equation, we will analyze a massless (zero mode) and a massive fields. Their solutions must satisfy the localization conditions.
	
	\subsection{Action}
	An action of 5-dimensional scalar field $\Phi (x^M)$ coupled nonminimally to gravity is
	\begin{equation}\label{aksi-skalar}
		S_0 = \int d^5 x \sqrt{-g} \left( \frac{1}{2} g^{MN} \partial_M \Phi \partial_N \Phi + \xi R \Phi^2 \right) ,
	\end{equation}
	where $ \xi $ is a coupling constant that represents a factor that governs the degree of coupling between the bulk scalar field and the curvature of spacetime $R$. 
	Consider a braneworld metric with $z$ extra coordinate that is given in the following line element, 
	\begin{equation}\label{metrik}
		ds^2 = e^{2A(z)} \left[\tilde{g}_{\mu \nu} dx^\mu dx^\nu + dz^2 \right].
	\end{equation}
	By this metric (\ref{metrik}), the Ricci scalar is obtained
	\begin{equation}\label{Ricci}
		R = -e^{-2A} \left(8 A'' + 12 A'^2\right).
	\end{equation}
	Using the scalar field decomposition, 
	\begin{equation}\label{dekomposisi}
		\Phi (x^M) = \varphi (x^\mu) \chi (z) ,
	\end{equation}
	the action (\ref{aksi-skalar}) with metric (\ref{metrik}) can be rewritten as
	\begin{multline}
		S_0^{(5D)} = \frac{1}{2} \int d^4 x \sqrt{-\tilde{g}} \tilde{g}^{\mu \nu} \partial_\mu \varphi \partial_\nu \varphi \int_{-\infty}^\infty dz e^{3A} \chi^2 \\ + \frac{1}{2} \int d^4x \sqrt{-\tilde{g}} \varphi^2  \int_{-\infty}^\infty dz \left( e^{3A} \chi'^2 + 2 \xi e^{5A} R \chi^2  \right) .
	\end{multline}
	It can be reduced into a 4-dimensional action
	\begin{equation}
		S_0^{(4D)} = \frac{1}{2} \int d^4 x \sqrt{-\tilde{g}} \left( \tilde{g}^{\mu \nu} \partial_\mu \varphi \partial_\nu \varphi 
		+ m^2 \varphi^2 \right)
	\end{equation}
	provided that it satisfies the following normalization and mass equations
	\begin{align}
		N_0 & = \int_{-\infty}^\infty dz e^{3A} \chi^2 = 1 , \\
		m^2 & = \int_{-\infty}^\infty dz \left( e^{3A} \chi'^2 + 2 \xi e^{5A} R \chi^2 \right) < \infty .
	\end{align}
	To analyze the field localization, a new field $\tilde{\chi}$ is defined,
	\begin{equation}\label{skalar-baru}
		\chi (z) \rightarrow \tilde{\chi} (z) = e^{\frac{3}{2} A(z)} \chi (z).
	\end{equation}
	Then, the normalization and mass equations become
	\begin{align}
		N_0 & = \int_{-\infty}^\infty dz \tilde{\chi}^2 = 1 , \label{N0} \\
		m_0^2 & = \int_{-\infty}^\infty dz \left[ \tilde{\chi}'^2 - 3 A' \tilde{\chi}' \tilde{\chi} + \left( \frac{9}{4} A'^2 - 16 \xi A'' - 24 \xi A'^2\right) \tilde{\chi}^2 \right] < \infty. \label{m0}
	\end{align}
	The field $\tilde{\chi} (z)$ must satisfy these conditions in order for the scalar field $\Phi$ to be localized on the brane. The normalization equation (\ref{N0}) depends only on the field $\tilde{\chi}(z)$. This condition requires that field $\tilde{\chi}$ must converge at coordinate $z$.
	
	\subsection{Field equation}
	By varying the action (\ref{aksi-skalar}) with respect to the scalar field $\Phi$, the equation of motion for the 5-dimensional scalar field coupled nonminimally to gravity is obtained, 
	\begin{equation}
		\partial_M \left(\sqrt{-g} g^{MN} \partial_N \Phi \right) - \sqrt{-g} \xi R \Phi = 0.
	\end{equation}
	Using the decomposition (\ref{dekomposisi}), it can be rewritten into the following field equation,
	\begin{equation}
		\frac{1}{\sqrt{-\tilde{g}}} \partial_\mu \left(\sqrt{-\tilde{g}} \tilde{g}^{\mu \nu} \partial_\nu \varphi \right) \chi + \left( \chi'' + 3A' \chi' \right) \varphi - e^{2A} \xi R \varphi \chi = 0 .
	\end{equation}
	The 4-dimensional scalar field $ \varphi (x^\mu) $ obeys a massive Klein-Gordon equation, 
	\begin{equation}
		\frac{1}{\sqrt{-\tilde{g}}} \partial_\mu \left(\sqrt{-\tilde{g}} \tilde{g}^{\mu \nu} \partial_\nu \varphi \right) = m^2 \varphi .
	\end{equation}
	Thus, the equation of motion for the bulk component's scalar field is
	\begin{equation}
		\chi'' + 3A' \chi' + \left(m^2 - e^{2A} \xi R \right) \chi = 0. 
	\end{equation}
	From the Ricci scalar (\ref{Ricci}), it becomes
	\begin{equation}\label{field-equation}
		\chi'' + 3A' \chi' + \left[ m^2 + \xi \left(8 A'' + 12 A'^2\right) \right] \chi = 0 .
	\end{equation}
	From the new scalar field (\ref{skalar-baru}), the field equation (\ref{field-equation}) can be rewritten into Schr\"odinger-like equation,
	\begin{equation}\label{schrodinger-skalar}
		\left[-\partial_z^2 + V_0(z)  \right] \tilde{\chi} (z) = m^2 \tilde{\chi} (z) ,
	\end{equation}
	where the effective potential 
	\begin{equation}\label{Potensial-skalar}
		V_0(z) = \left(\frac{3}{2} - 8 \xi \right) A'' + \left(\frac{9}{4} - 12 \xi \right) A'^2 .
	\end{equation}
	This effective potential, which is dependent on the warp function $A(z)$ and the coupling constant $\xi$, is primarily influenced by the warp function.
	Here, we will examine  two models discussed in the previous section.
	
	\subsection{Field localization}
	\subsubsection{Model 1}\label{scalar-model-1}
	The warp function solution for model 1 in equation (\ref{warp-1}) leads the effective potential (\ref{Potensial-skalar}) into 
	\begin{equation}\label{potensial-skalar-1}
		V_0 (z) = -\frac{4 a^2 (16 \xi -3) \left(a^2 z^2-1\right)}{\left(a^2 z^2+4\right)^2}.
	\end{equation}
	When the effective potential at its minima takes on negative values, the scalar field can be localized on the brane.
	This potential (\ref{potensial-skalar-1}) does not vanish when $a \neq 0$ and the coupling constant $\xi \neq \frac{3}{16}$. In general, the potential gives a minimum value 
	\begin{equation}
		V_0^{\text{min}} = \begin{cases}
			\frac{1}{5} a^2 (3 - 16 \xi ) & \text{ for } \xi > \frac{3}{16} , \\
			-\frac{1}{4} a^2 (3 - 16 \xi) & \text{ for } \xi < \frac{3}{16} .
		\end{cases}
	\end{equation}
	The plot of the potential form (\ref{potensial-skalar-1}) can be seen in Figure \ref{fig:potensial-skalar-1} for the cases of coupling constants $\xi > \frac{3}{16}$ and  $\xi < \frac{3}{16}$.
	% Figure environment removed
	At $z \rightarrow \pm \infty$, the potential is $V_0(z) \rightarrow 0 $.
	The coupling constant $\xi < \frac{3}{16} $ gives the volcano potential. For $\xi = -1$, the minimum value $V_0 = -\frac{19}{4} a^2 $ at $z = 0$.
	The coupling constant $\xi =1$ gives a minimum value $V_0 = -\frac{13}{5} a^2$.
	From this potential (\ref{potensial-skalar-1}), there is possibility to localize the scalar matter on the brane for all value of coupling constant except $\xi = \frac{3}{16}$.
	
	\paragraph{Minimal coupling}
	If we consider $ \xi = 0 $ as the minimal coupling case, the massless solution is 
	\begin{equation}
		\tilde{\chi}_0 (z) = \sqrt{\frac{a}{3\pi}} \frac{16}{\left(a^2 z^2+4\right)^{3/2}}.
	\end{equation}
	This solution is normalized (\ref{N0}) and satisfies the mass equation (\ref{m0}), $m^2 = 0$, so the massless scalar field is localized on the brane \cite{Liang2009}.
	On the other hand, the massive scalar field ($m\neq 0$) solution is difficult be obtained.
	
	
	\paragraph{Asymptotic analysis}
	The localization analysis of the nonminimally coupled field cannot be obtained analytically. To simplify the calculation, we will discuss asymptotic behavior with approximation over a certain $z$, without losing the important point regarding normalization conditions.
	The asymptotic behavior of the scalar field can be examined at $z=0$ and $z \rightarrow \pm \infty$. If we consider the approximation solution in the vicinity of $z = 0$, the approximate warp function (\ref{warp-1}) and the approximate bulk scalar (\ref{bulk-1}) are given as follows 
	\begin{equation}
		A(z\rightarrow 0) = -\frac{1}{4} a^2 z^2, \quad 
		\phi(z\rightarrow 0) = \frac{\sqrt{3}}{2} a z.
	\end{equation} 
	The potential (\ref{Potensial-skalar}) in the vicinity of $z = 0$ is 
	\begin{equation}
		V_0 (z \rightarrow 0) = -\frac{1}{16} a^2 (16 \xi-3) \left(3 a^2 z^2-4\right) .
	\end{equation}
	It leads to the minimum value of the potential $V_0(z)$ for $\xi < \frac{3}{16}$, as mentioned above.
	Then, we consider the approximation at $z\rightarrow \pm \infty$ to check the behavior of the field solution $\tilde{\chi}_0 (z)$ at infinity due to the localization conditions that require finiteness. 
	The approximate warp function (\ref{warp-2}) and bulk scalar (\ref{bulk-2}) at $z \rightarrow \pm \infty$ are 
	\begin{equation}\label{A-1}
		A(z \rightarrow \pm \infty) = -\ln \left(\frac{1}{4} a^2 z^2\right) ,
	\end{equation}
	\begin{equation}
		\phi (z \rightarrow \pm \infty) = \sqrt{3} \log \left[ \frac{1}{2} \left(\sqrt{a^2 z^2}+a z\right)\right] .
	\end{equation}
	The first and second derivatives of the approximate warp function with respect to $z$ are $A' (z \rightarrow \pm \infty) = -2 z^{-1}$, and $A'' (z \rightarrow \pm \infty) = 2 z^{-2}$. 
	Therefore, the potential (\ref{Potensial-skalar}) can be written as $V_0 (z) = 4 \left(3-16\xi\right) z^{-2}$.
	For the massless ($m = 0$) scalar, the field equation (\ref{schrodinger-skalar}) gives a general solution
	\begin{equation}\label{3.26}
		\tilde{\chi}_0 (z \rightarrow \pm \infty)
		= c_1 z^{\frac{1}{2} (1 - \beta)} + c_2 z^{\frac{1}{2} (1 + \beta)} ,
	\end{equation}
	where $\beta = \sqrt{49-256 \xi}$. By choosing $c_2 = 0$, it is known that the solution (\ref{3.26}) converges to zero as $z$ increases. 
	It means that the solution $\tilde{\chi}_0$ is normalized.
	The mass equation is assumed to satisfy
	\begin{equation}\label{3.28}
		m_0^2 = \int_{-\infty}^{-\epsilon} F(\xi,z) dz + M^2 + \int_{\epsilon}^{\infty} F(\xi,z) dz = 0 ,
	\end{equation}
	where $\epsilon$ is the limit of integration in the form of a positive number. The function $F(\xi ,z)$ is the integrand of equation (\ref{m0}) with approximate $A$ (\ref{A-1}) and $\tilde{\chi}_0$ (\ref{3.26}), that is
	\begin{equation}
		F(\xi, z) = \frac{1}{4} c_1^2 \left(\beta ^2-14 \beta -512 \xi +49\right) z^{-\beta -1} ,
	\end{equation}
	and $M^2$ is a finite value which is the integration value of a function with limits from $-\epsilon$ to $\epsilon$. By integrating and solving equation (\ref{3.28}), we get
	\begin{equation}
		M^2 = \frac{c_1^2 \left(384 \xi +7 \left(\sqrt{49-256 \xi }-7\right)\right) \left(-\epsilon ^2\right)^{-\sqrt{49-256 \xi }} \left((-\epsilon )^{\sqrt{49-256 \xi }}-\epsilon ^{\sqrt{49-256 \xi }}\right)}{2 \sqrt{49-256 \xi }}.
	\end{equation}
	Since the normalization condition is satisfied, the massless scalar field nonminimally coupled to gravity is localized on the brane according to the asymptotic approximation.
	
	Another mode, the massive scalar field equation (\ref{schrodinger-skalar}) gives an approximate solution
	\begin{equation}\label{42}
		\tilde{\chi} (z \rightarrow \pm \infty) = \sqrt{z} \left[ c_1 J_{\frac{1}{2} \beta }(m z) + c_2 Y_{\frac{1}{2} \beta}(m z)\right] ,
	\end{equation}
	where $J_n (z)$ is the Bessel function of the first kind, and $Y_n(z)$ is the Bessel function of the second kind. Choosing $c_2 = 0$, the massive scalar field (\ref{42}) fulfills square integrability 
	for positive 4-dimensional scalar mass, $m > 0$.
	It satisfies the normalization condition. Therefore, the approximate massive scalar field coupled nonminimally to gravity are localized on the scalar thick brane. 
	
	\subsubsection{Model 2}
	Another model with the warp function (\ref{warp-2}) gives the effective scalar potential
	\begin{equation}\label{potensial-skalar-2}
		V_0 (z) = \frac{(16 \xi -3) \left(2 a^2 q^2-5 a^4 z^2\right)}{4 \left(a^2 z^2+q^2\right)^2}.
	\end{equation}
	Localization of the scalar field on the brane is possible when the effective potential reaches negative values at its minima.
	There are several possible values for the potential minimum, which depend on the constant parameters $\xi$ and $q$,
	\begin{equation}
		V_0^\text{min} = 
		\begin{cases}
			-\dfrac{25}{112 q^2}  \left(16 a^2 \xi -3 a^2\right); & \xi > \frac{3}{16},\ a \neq 0, \text{ and } q \neq 0 , \\
			\dfrac{1}{2 q^2} (16 a^2 \xi -3 a^2) ; &  \xi < \frac{3}{16}, \ a \neq 0, \text{ and } q \neq 0 , \\
			-\infty ; &  \xi > \frac{3}{16} , \ a \neq  0, \text{ and } q= 0 .
		\end{cases}
	\end{equation}
	Otherwise, the effective potential vanishes.
	Therefore, the potential (\ref{potensial-skalar-2}) has possibility to localize the nonminimally coupled scalar matter.
	
	\paragraph{Minimal coupling} 
	Considering $\xi = 0$, the solution for the massless scalar field is as follows. 
	\begin{equation}
		\tilde{\chi}_0 (z) = \sqrt{\frac{a}{2}} \frac{q}{\left(a^2 z^2+q^2\right)^{3/4}}.
	\end{equation}
	It can be proved that this solution is normalized and satisfies the mass equation, $m^2 = 0$, so that the massless scalar field can be localized. However, the solution of massive mode cannot be obtained analytically \cite{Liang2009}.
	
	\paragraph{Asymptotic analysis}
	In the same way as in the previous model, the asymptotic behavior is discussed.
	The warp function (\ref{warp-2}) is asymptotic to the following function as $z$ becomes very large,
	\begin{equation}\label{A-2}
		A(z\rightarrow \pm \infty) = -\ln \left(az\right) ,
	\end{equation}
	with the derivatives
	\begin{equation}
		A'(z\rightarrow \pm \infty) = - \frac{1}{z}, \quad A''(z\rightarrow \pm \infty) = \frac{1}{z^2} .
	\end{equation}
	The warp function approximation in this model is similar to the warp function approximation in the sine-Gordon brane model \cite{Cruz2016}. The localization conditions for a nonminimally coupled scalar field with gravity have been discussed in \cite{Moazzen2017}. Here, the scalar action is considered without the bulk scalar field mass term, $m_s = 0$. Since the normalization equation is independent of any function other than the scalar function $\tilde{\chi}_0$, the localization result should be similar to \cite{Moazzen2017}.
	The zero mode scalar field is localized on the brane.
	
	The massive mode of the nonminimally coupled scalar field leads to the following approximate solution
	\begin{equation}
		\tilde{\chi} (z \rightarrow \pm \infty) = \sqrt{z} \left[ c_1 J_{\frac{1}{2} \zeta}(m z)+c_2 Y_{\frac{1}{2} \zeta}(m z)\right]
	\end{equation}
	where the constant parameter $\zeta = 4 \sqrt{1-5 \xi}$. 
	This solution has similarity to the previous model in equation (\ref{42}), with the only difference in constant parameter $\zeta$. Hence, the nonminimally coupled massive scalar field in this model is also localized on the brane with the asymptotic behavior where $z \rightarrow \pm \infty$.
	
	\section{Localization of nonminimally coupled vector field}
	\subsection{Action}
	Consider an action of nonminimally coupled vector field with gravity,
	\begin{equation}
		S_1 = \int d^5x \sqrt{-g} \left(-\frac{1}{4}  g^{MN} g^{RS} F_{MR} F_{NS} + \lambda R g^{MN} A_M A_N \right) ,
	\end{equation}
	where $ \lambda $ is a coupling constant of a 5-dimensional vector field, 
	\begin{equation}\label{dekomposisi-vektor}
		A_M (x^M) = \left(A_\mu (x^M), A_z \right) = \left(a_\mu (x^\mu) \alpha(z), A_z \right), 
	\end{equation}
	and gravity represented by $R$. The field strength tensor related to the $A^M$ is $ F_{MN} = \partial_M A_N - \partial_N A_M $, with the non-zero components are
	\begin{align*}
		F_{\mu \rho} & = \alpha \partial_\mu a_\rho - \alpha \partial_\rho a_\mu = \alpha f_{\mu \rho} , \\
		F_{\mu z} & = - a_\mu \partial_z \alpha = - F_{z \mu}.
	\end{align*}
	Then, the action $ S_1 $ can be rewritten as 
	\begin{multline}
		S_1 = -\frac{1}{4} \int d^4x \sqrt{-\tilde{g}} \tilde{g}^{\mu \nu} \tilde{g}^{\rho\sigma} f_{\mu \rho} f_{\nu \sigma} \int_{-\infty}^\infty dz e^A \alpha^2 - \frac{1}{2} \int d^4x \sqrt{-\tilde{g}} \tilde{g}^{\mu \nu} a_\mu a_\nu \int_{-\infty}^\infty dz e^A (\partial_z \alpha)^2 \\
		+ \lambda \int d^4 x \sqrt{-\tilde{g}} \tilde{g}^{\mu \nu} a_\mu a_\nu \int_{-\infty}^\infty dz e^{3A} \alpha^2 R + \lambda A_z^2 \int d^4x \int_{-\infty}^\infty dz e^{3A} R
	\end{multline}
	It can be reduced into 4-dimensional action by fulfilling the localization conditions expressed by the normalization and mass equations
	\begin{align}
		N_1 & = \int_{-\infty}^\infty dz e^A \alpha^2 = 1 , \\
		m_1^2 & = \int_{-\infty}^\infty dz \left(e^A (\partial_z\alpha)^2 - 2 \lambda e^{3A} \alpha^2 R \right) \nonumber  \\
		& = \int_{-\infty}^\infty dz  e^A \left[ (\partial_z\alpha)^2 - 2 \lambda \alpha^2 \left(8A'' + 12 A'^2 \right) \right] < \infty ,
	\end{align}
	and the fifth-component of $A_M$ is chosen to be zero, $ A_z = 0 $.
	
	By defining a new field, $ \tilde{\alpha} (z) = \alpha (z) e^{\frac{A(z)}{2}} $, the above localization conditions become
	\begin{align}
		N_1  & = \int_{-\infty}^\infty dz \tilde{\alpha}^2 = 1 , \\
		m_1^2 & = \int_{-\infty}^\infty dz \left[\tilde{\alpha}'^2 - A' \tilde{\alpha}' \tilde{\alpha} + \frac{A'^2}{4} \tilde{\alpha}^2 + 2 \lambda \tilde{\alpha}^2 \left(8A'' + 12 A'^2 \right)\right] < \infty .
	\end{align}
	The $z$-dependent vector field $\tilde{\alpha}$ that satisfies both of the above equations means that it is localized on the brane.
	
	\subsection{Field equation}
	From the action $S_1$, the nonminimally coupled vector field equation is obtained,
	\begin{equation}
		\frac{1}{\sqrt{-g}} \partial_M \left(\sqrt{-g} g^{MN} g^{RS} F_{NS} + 2 \lambda A^R R \right) = 0.
	\end{equation}
	From metric (\ref{metrik}) and field decomposition (\ref{dekomposisi-vektor}), which $ a^\rho $ obeys the 4-dimensional Proca equation
	\begin{equation}
		\frac{1}{\sqrt{-\tilde{g}}} \partial_\mu \left(\sqrt{-\tilde{g}} \tilde{g}^{\mu \nu} \tilde{g}^{\rho\sigma} f_{\nu \sigma} \right) = m^2 a^\rho,
	\end{equation}
	the field equation $ \alpha(z) $ can be written as
	\begin{equation}
		\alpha'' + A' \alpha' + m^2 \alpha + 2 \lambda e^{4A} R \alpha = 0 .
	\end{equation}
	In terms of thick brane model, a new field is defined, $ \tilde{\alpha} (z) = \alpha (z) e^{\frac{A(z)}{2}} $. Therefore, the above field equation can be written as
	\begin{equation}\label{schrodinger-vektor}
		\left[ -\partial_z^2 + V_1 (z) \right] \tilde{\alpha} (z) = m^2 \tilde{\alpha} (z)
	\end{equation}
	with the effective vector potential
	\begin{equation}\label{Potensial-vektor}
		V_1(z) = \frac{1}{2} A'' + \frac{1}{4} A'^2 + 2\lambda e^{2A} \left(8A'' + 12 A'^2\right) .
	\end{equation}
	We will investigate two models, with a focus on the warp function $A(z)$, which is the most significant factor in this potential.
	
	\subsection{Field localization}
	\subsubsection{Model 1}
	The warp function solution for model 1 in equation (\ref{warp-1}) leads the effective vector potential (\ref{Potensial-vektor}) into 
	\begin{equation}\label{potensial-vektor-1}
		V_1(z) = \frac{2 \left(a^8 z^6+6 a^6 z^4+1024 a^4 \lambda  z^2-32 a^2 (32 \lambda +1)\right)}{\left(a^2 z^2+4\right)^4} .
	\end{equation}
	Plot of the effective potential (\ref{potensial-vektor-1}) is given in Figure \ref{fig:potensial-vektor-1}. It allows for the nonminimally coupled vector field to be localized, since it reaches negative values at its minima.
	% Figure environment removed

\paragraph{Minimal coupling}
Considering $\lambda = 0$, the solution for the massless vector field is as follows.
\begin{equation}
\tilde{\alpha}_0 (z) = \frac{\sqrt{2a/\pi}}{\sqrt{a^2 z^2+4}} .
\end{equation}
This solution satisfies $\int_{-\infty}^\infty \tilde{\alpha}_0^2 (z) dz = 1$ and $m_1^2 = 0$, so the massless vector field is localized\footnote{According to Reference \cite{Liang2009}, it is stated that the solution of the massless vector field does not converge over the entire range of $(-\infty, \infty)$, resulting in the emergence of a non-localized field.}.


\paragraph{Asymptotic behavior}
Due to the difficulty in obtaining an exact solution analytically, the asymptotic behavior is used.
From the approximate warp function $A(z \rightarrow \pm \infty)$ given in equation (\ref{A-1}), the effective vector potential becomes
\begin{equation}
V_1(z\rightarrow \pm \infty) = \frac{2}{z^6}  \left(\frac{1024 \lambda }{a^4}+z^4\right).
\end{equation}
From the field equation (\ref{schrodinger-vektor}), the zero mode solution is
\begin{equation}
\tilde{\alpha}_0 (z\rightarrow \pm \infty) \propto \frac{\sqrt{az}}{2^{\frac{7}{8}} \lambda^{\frac{1}{8}}} (-1)^{\frac{1}{4}} \Gamma \left(\frac{7}{4}\right) I_{\frac{3}{4}}\left(\frac{16 \sqrt{2 \lambda}}{a^2 z^2}\right) ,
\end{equation}
where $\Gamma(n)$ is a gamma function, $I_n(z)$ is a modified Bessel function of the first kind. 
This solution vanishes if the limit $z$ is taken to infinity, so it converges. Therefore, the localization of the zero mode of nonminimally coupled vector field is fulfilled.

\subsubsection{Model 2}
Consider the warp function (\ref{warp-2}), which gives the effective scalar potential
\begin{equation}\label{potensial-vektor-2}
V_1(z) = 
\frac{a^2 \left(3 a^4 z^4+q^2 \left(a^2 z^2-64 \lambda \right)+160 a^2 \lambda  z^2-2 q^4\right)}{4 \left(a^2 z^2+q^2\right)^3}
\end{equation}
For non-zero constant $a$, and coupling constant $\lambda \geq -\frac{7 q^2}{352} (q > 0)$, or $ \lambda > -\frac{7 q^2}{352} (q < 0)$, the above potential gives a minimum value
\begin{equation}
V_1 = \frac{-32 a^2 \lambda -a^2 q^2}{2 q^4}. 
\end{equation}
This allows for the field localization.

\paragraph{Minimal coupling}
Consider $\lambda = 0$, the solution of zero mode vector field is
\begin{equation}
\tilde{\alpha}_0 (z) = \frac{c_1}{\left(a^2 z^2+q^2 \right)^{1/4}} . 
\end{equation}
This zero mode vector solution is not normalized.

\paragraph{Asymptotic behavior}
In the same way as in the previous case, the field solution is analyzed using its asymptotic behavior.
If the approximate warp function is considered at $z \rightarrow \pm \infty$ as given in equation (\ref{A-2}), the effective potential for the vector field is
\begin{equation}
V_1(z\rightarrow \pm \infty) = \frac{40 \lambda }{a^2 z^4}+\frac{3}{4 z^2} .
\end{equation}
The zero mode field equation gives solution
\begin{equation}
\tilde{\alpha}_0 (z\rightarrow \pm \infty) \propto \left(-\frac{a^2z^2}{10 \lambda}\right)^{\frac{1}{4}} I_1\left(\frac{2 \sqrt{10 \lambda}}{a z}\right).
\end{equation}
This solution also converges, since it equals zero if the limit $z$ is taken to infinity. Consequently, it is possible to confine the zero mode of a nonminimally coupled vector field in this model.

\section{Localization of nonminimally coupled spinor field}
In this section, we will examine the localization of the spinor field coupled nonminimally to gravity on two models of the scalar thick brane.
As mentioned in some References such as in \cite{Wulandari2019,Liang2009,Guerrero2019}, to localize fermions on branes in five and six dimensions, scalar-fermion coupling must be introduced.

\subsection{Field equation}
An action of nonminimally coupled 5-dimensional spinor $\Psi (x^M)$ with gravity is given,
\begin{equation}\label{aksi-1/2}
S_\frac{1}{2} = \int d^5 x \sqrt{-g} \left(\bar{\Psi} i \Gamma^M D_M \Psi + \eta \bar{\Psi} R \Psi \right) ,
\end{equation}
with a coupling constant $ \eta $ and Ricci scalar $ R $.
Since we define the field in a curved space, the covariant derivative and non-zero spin connection $\omega_M $ are given as
\begin{equation}
D_M = \partial_M + \omega_M ; \quad \omega_M = \omega_\mu = -\frac{i}{2} A' \gamma_\mu \gamma_5 .
\end{equation}
The components of 5-dimensional gamma matrix in a curved space, $ \Gamma^M = \left(e^{-A} \gamma^\mu , -i e^{-A} \gamma^5 \right) $. 
From the action (\ref{aksi-1/2}), the equation of motion for the nonminimally coupled 5-dimensional spinor field is
\begin{equation}
i \Gamma^M D_M \Psi + \eta R \Psi = 0 .
\end{equation}
Using the above gamma matrix and covariant derivative, the equation of motion becomes
\begin{equation}\label{54}
\left(i \gamma^\mu \partial_\mu + \gamma^5 (\partial_z + 2A') - \eta e^A R \right) \Psi = 0. 
\end{equation}
The 5-dimensional spinor field can be decomposed into two chiral components, corresponding to its left-handed and right-handed projections.
Each component can be separated into brane and extra dimension parts, that is
\begin{equation}
\Psi(x^\mu, z) = \sum_n \left[ \psi_{Ln} (x^\mu) P_{Ln} (z) + \psi_{Rn} (x^\mu) P_{Rn} (z) \right] ; \quad n = 1,2.
\end{equation}
The 4-dimensional spinor $ \psi_{L,R} (x^\mu) $ obeys the massive Dirac equation, $ i\gamma^\mu \partial_\mu \psi_{L,R} = m \psi_{R,L} $. The matrix $ \gamma^5 $ performs $ \gamma^5 \psi_L = - \psi_L $ and $  \gamma^5 \psi_R = \psi_R  $. Therefore, the field equation (\ref{54}) becomes
\begin{subequations}
\begin{align}
	\left( \partial_z + 2A' - \eta e^A R \right) P_{Rn} = -m_n P_{Ln}, \\
	\left( \partial_z + 2A' + \eta e^A R \right) P_{Ln} = -m_n P_{Rn}.
\end{align}
\end{subequations}
The orthonormality condition from the spinor action (\ref{aksi-1/2}) is
\begin{equation}
\int_{-\infty}^\infty dz e^{4A} P_{Lm} P_{Rn} = \delta_{m} \delta_{n} .
\end{equation}
In order to form the normalization equation that depend only on the $z$-component spinor field, a new field is defined, $ \tilde{P}_{L,R} (z) = e^{2A(z)} P_{L,R} (z) $. Then, the normalization equation becomes
\begin{equation}
\int_{-\infty}^\infty dz \tilde{P}_{Lm} \tilde{P}_{Rn} = \delta_{m} \delta_{n} .
\end{equation}
Using the new spinor field, the equation of motion can be written as
\begin{equation}\label{Schrodinger-spinor}
\left[-\partial_z + V_{L,R} (z) \right] \tilde{P}_{Ln, Rn} (z) = m_n^2 \tilde{P}_{Ln, Rn} (z)
\end{equation}
where the effective potential for left-handed and right-handed spinors
\begin{subequations}
\begin{align}
	V_{L} (z) & = \eta^2 e^{-2A} \left(8A'' + 12A'^2 \right)^2 + \eta e^{-A} \left[ \left(8A''' + 24 A' A'' \right) - A' \left(8A'' + 12A'^2\right) \right] , \\
	V_{R} (z) & = \eta^2 e^{-2A} \left(8A'' + 12A'^2 \right)^2 - \eta e^{-A} \left[ \left(8A''' + 24 A' A'' \right) - A' \left(8A'' + 12A'^2\right) \right] .
\end{align}
\end{subequations}
The difference between the two potentials above is only the sign in $\eta$, that is $V_R = \left. V_L \right|_{\eta \rightarrow -\eta} $.
Then, we will examine the localization of the spinor field in two models of scalar thick brane.

\subsection{Field localization}
\subsubsection{Model 1}
From the warp function (\ref{warp-1}), the effective potential for left-handed spinor becomes
\begin{equation}\label{potensial-spinor-1}
V_L (z) = \frac{32 a^4 \eta  \left(8 a^4 \eta  z^4-16 a^2 \eta  z^2+8 \eta +5 z\right)}{\left(a^2 z^2+4\right)^2},
\end{equation}
and right-handed spinor is $ V_R (z) = \left. V_L (z) \right|_{\eta \rightarrow -\eta} $.
The potential graph of (\ref{potensial-spinor-1}) is given in Figure \ref{fig:potensial-spinor} on the left side with a variation of coupling constant, $\eta = \{1, 0.1, 0.01\}$. 
In case of the same value of $a$, the coupling constant $\eta$ affects the potential minimum value.
If $a = 1$ is chosen, the potential minimum value is: $V_L = -0.644759$ at $z = -1.07484$ for $\eta = 0.1$, $V_L = -0.0648622$ at $z = -1.13896$ for $\eta = 0.01$, $V_L = -0.0064942$ at $z = -1.15295$ for $\eta = 0.001$. The smaller value of the $\eta$ constant, the minimum $V_L$ goes to zero. For a minimum coupling $\eta = 0$, the potential vanishes. This means that free spinors cannot be localized on the brane, as is the case in the RS and MRS branes \cite{Wulandari2019}. Potential $V_R$ behaves similarly to $V_L$, with $\eta \rightarrow -\eta$.
From these potential forms, it is possible that nonminimally coupled spinor fields can be localized on the brane.

The exact solution of zero mode left- and right-handed spinors from the field equation (\ref{Schrodinger-spinor}) are
\begin{equation}\label{solusi-spinor-L}
\tilde{P}_{L0} (z) = c_1 \exp \left[ {8 a \eta \left(2 a z-5 \arctan \left(\frac{a z}{2}\right)\right)} \right] ,
\end{equation}
\begin{equation}\label{solusi-spinor-R}
\tilde{P}_{R0} (z) = c_1 \exp \left[ {-8 a \eta  \left(2 a z-5 \arctan \left(\frac{a z}{2}\right)\right)} \right] ,
\end{equation}
or $\tilde{P}_R (z) = \left. \tilde{P}_L (z) \right|_{\eta \rightarrow - \eta }$.
For simplicity, the parameter $a = 1$ is chosen. For positive $\eta$, the function $\tilde{P}_{L0} (z)$ converges only as long as $z$ is negative. If we take the limit $\tilde{P}_{L0} (z)$ on $z$ toward $\pm \infty$, the function vanishes only in the negative $z$ direction.
For negative $\eta$, the function $\tilde{P}_{L0} (z)$ converges only as long as $z$ is positive. If a limit of $z \rightarrow \pm \infty$ is taken, the function vanishes only in the positive $z$ direction.
Conversely for $\tilde{P}_{R0} (z)$. The plots of the solution (\ref{solusi-spinor-L}) and (\ref{solusi-spinor-R}) are given in Figure \ref{fig:solusi-spinor-rl}.
% Figure environment removed


\paragraph{Asymptotic analysis}
Consider the asymptotic behavior of the warp function of model 1 in equation (\ref{A-1}). The approximate effective potential (\ref{potensial-spinor-1}) is
\begin{equation}\label{VL-approx}
V_{L} (z\rightarrow \pm \infty) = 256 a^4 \eta ^2.
\end{equation}
The zero mode left-handed spinor field equation gives approximate solution
\begin{equation}
\tilde{P}_{L0} (z\rightarrow \pm \infty) = c_1 e^{-16 a^2 \eta  z} .
\end{equation}
The limit of this solution at $z\rightarrow \infty$ is zero, which means it is a convergent function.
This solution also can be normalized for $a^2 \eta > 0$.
So that localization conditions are fulfilled. For the massive mode left-handed spinor, the field equation gives a solution
\begin{equation}
\tilde{P}_{L} (z\rightarrow \pm \infty) = c_1 e^{-z \sqrt{256 a^4 \eta ^2+m^2}}
\end{equation}
which is also convergent. The massive mode also satisfies the normalization equation with finite value, for $\sqrt{256 a^4 \eta ^2+m^2} > 0$.
The right-handed spinor behaves the same as the left-handed spinor in the case of asymptotic behavior, because its approximate potential have the same form as the left-handed (\ref{VL-approx}) due to the square of coupling constant $\eta$.

The finiteness of normalization conditions are satisfied for both massless and massive mode. Therefore, the spinor fields with the approximation in this model 1 can be localized on the scalar thick brane.

\subsubsection{Model 2}
Consider the warp function (\ref{warp-2}), the effective potential for left-handed spinor becomes
\begin{equation}\label{V-66}
V_L (z) = \frac{4 a^4 \eta}{\left(a^2 z^2+q^2\right)^3} \left(100 a^4 \eta  z^4-80 a^2 \eta  q^2 z^2 
+ \left(16 q^3 z - 5 a^2 q z^3\right) \sqrt{1+\frac{a^2 z^2}{q^2}}
+16 \eta  q^4\right) .
\end{equation}
The effective potential for right-handed spinor is $ V_R (z) = \left. V_L (z) \right|_{\eta \rightarrow -\eta} $.
The solution of the equation (\ref{Schrodinger-spinor}) is difficult to obtain analytically.
The shape of potential (\ref{V-66}) is given in Figure \ref{fig:potensial-spinor}.
From the minimum value of this potential, there is a possibility of spinor field to be localized.
% Figure environment removed

\paragraph{Asymptotic analysis}
The asymptotic behavior of warp function in (\ref{A-1}), leads the approximate effective potential (\ref{V-66}) into
\begin{equation}
V_{L} (z\rightarrow \pm \infty) = \frac{20 \eta  \left(20 a^2 \eta -a\right)}{z^2} .
\end{equation}
The approximate solution of zero mode left-handed spinor is
\begin{equation}\label{5.19}
\tilde{P}_{L0} (z\rightarrow \pm \infty) = c_1 z^{\frac{1}{2}-\frac{1}{2} (40 a \eta -1)}. 
\end{equation}
If the limit $z$ is taken to infinity, this solution vanishes. 
In order to localize the zero mode left-handed spinor on the brane, $ \tilde{P}_{L0} (z \rightarrow \pm \infty) $ should satisfy the normalization condition.
Since the solution (\ref{5.19}) fulfills square integrability 
for $a \eta >\frac{3}{40}$, then the approximate left-handed spinor is localized. 
Furthermore, the right-handed spinor needs to be examined as well because the approximation potential is not the same as that of the left-handed.
The approximate solution of zero mode right-handed spinor is
\begin{equation}
\tilde{P}_{R0}^2 (z \rightarrow \pm \infty) = c_1 z^{\frac{1}{2} - \frac{1}{2} (40 a \eta +1)} .
\end{equation}
The normalization condition produces a finite value,
for $a \eta >\frac{1}{40}$.
Therefore, this model provides localized nonminimally coupled spinors on the scalar thick brane in the massless mode.

\section{Conclusions}
Our investigation has focused on how Standard Model fields nonminimally coupled to gravity are confined on a thick braneworld model generated by a scalar bulk. We consider two models of scalar thick brane each derived from a different given superpotential \cite{Liang2009}. To study the localization of these fields, we have utilized a natural mechanism. The normalization condition is obtained by reducing the 5-dimensional action to a 4-dimensional action.

We consider a scalar field coupled nonminimally to gravity with a coupling constant $\xi$. The exact solutions of the field equations are difficult to obtain, and normalization analysis cannot be performed analytically. In the normalization equation, we only need to understand how the field $\tilde{\chi} (x^5)$ behaves in the extra coordinate. In this case, a square-integrable field function is required.
Without losing the point of field localization, we examine the asymptotic behavior of the warp function on $z$ towards infinity. The massless and massive modes of the nonminimally coupled scalar field are localized in both models of the scalar thick brane. In terms of the exact solution, when the coupling is set to be minimal, the scalar field will be localized for the massless mode, and this result confirms the minimal case as stated in \cite{Liang2009}.
In the case of a nonminimally coupled vector field, it has been discovered that a vector field with a nonminimal coupling constant $\lambda$ is also localized on a scalar thick brane in the massless mode for both models.

In the context of a nonminimally coupled spinor field, we investigate the spinor-scalar field system within a curved spacetime. In this scenario, the Ricci scalar serves as a scalar field, effectively localizing the spinor field.
In Model 1, we have discovered the possibility for localization of the spinor field based on its potential behavior. Through asymptotic analysis, it has been determined that the potential exhibits the same form for both left- and right-handed spinors. Consequently, a similar approximation solution can be derived for both types of spinors. The approximate solution for the spinor can be normalized for both the massless and massive modes. As a result, the spinor field in this model is successfully normalized.
In Model 2, we have also discovered the potential for localization. However, it should be noted that we only obtain approximate solutions for the massless spinor fields in this case. Nevertheless, these approximate solutions can be successfully normalized, indicating that the massless spinor field is indeed localized in this model.

\section*{Acknowledgment}
This research is supported by PPMI-FMIPA-ITB.
	
	\begin{thebibliography}{99}
		
		\bibitem{Randall1999a}
		L. Randall and R. Sundrum, \textit{Large Mass Hierarchy from a Small Extra Dimension}, \textit{Phys. Rev. Lett.} \textbf{83} (1999) 3370.
		
		\bibitem{Randall1999b}
		L. Randall and R. Sundrum, \textit{An Alternative to Compactification}, \textit{Phys. Rev. Lett.} \textbf{83} (1999) 4690.
		
		\bibitem{Bajc2000}
		B. Bajc and G. Gabadadze,
		\textit{Localization of matter and cosmological constant on a brane in anti de Sitter space},
		\textit{Phys. Lett. B} \textbf{474} (2000) 282-291.
		
		\bibitem{Jones2013}
		P. Jones, G. Mu\~{n}oz, D. Singleton and Triyanta,
		\textit{Field localization and the Nambu\textendash Jona-Lasinio mass generation mechanism in an alternative five-dimensional brane model},
		\textit{Phys. Rev. D} \textbf{88} (2013) 025048.
		
		\bibitem{Wulandari2017}
		D. Wulandari, Triyanta, J. S. Kosasih, D. Singleton and P. Jones, 
		\textit{Localization of interacting fields in five-dimensional braneworld models},
		\textit{Int. J. Mod. Phys. A} \textbf{32} (2017) 1750191.
		
		\bibitem{Wulandari2019}
		D. Wulandari, Triyanta, J. S. Kosasih and D. Singleton,
		\textit{The Field Localization of Yukawa Interaction in a Modified Randall-Sundrum Model},
		\textit{Journal of Physics: Conf. Series} \textbf{1204} (2019) 012044.
		
		\bibitem{Liu2018}
		Y. X. Liu, Introduction to Extra Dimensions and Thick Braneworlds, \textit{Memorial Volume for Yi-Shi Duan} (2018) pg. 211-275.
		
		\bibitem{Liu2008}
		Y. Liu, X. Zhang, L. Zhang and Y. Duan, 
		\textit{Localization of matters on pure geometrical thick branes}, \textit{JHEP} 02 (2008) 067.
		
		\bibitem{Guo2013}
		H. Guo, A. Herrera-Aguilar, Y.-X. Liu, D. Malag\'on-Morej\'on and R. R. Mora-Luna, 
		\textit{Localization of bulk matter fields, the hierarchy problem and corrections to Coulomb's law on a pure de Sitter thick braneworld}, \textit{Phys. Rev. D} \textbf{87} (2013) 095011.
		
		\bibitem{Herrera-Aguilar2010}
		A. Herrera-Aguilar, D. Malag\'on-Morej\'on and R. R. Mora-Luna, 
		\textit{Localization of gravity on a de Sitter thick braneworld without scalar fields}, 
		\textit{JHEP} 11 (2010) 015.
		
		\bibitem{Liang2009}
		J. Liang, and Y. Duan, \textit{Localization of matters on thick branes}, \textit{Phys. Lett. B} \textbf{678} (2009) 491-496.
		
		\bibitem{Bazeia2009}
		D. Bazeia, A. R. Gomes and L. Losano, \textit{Gravity localization on thick branes: a numerical approach}, \textit{Int. J. Mod. Phys. A} \textbf{24} (2009) 1135-1160,
		arXiv:0708.3530v2 [hep-th].
		
		\bibitem{Gremm2000}
		M. Gremm, \textit{Four-dimensional gravity on a thick domain wall}, \textit{Phys. Lett. B} \textbf{478} (2000) 434-438.
		
		\bibitem{Geng2016}
		W. J. Geng and H. Lü, \textit{Einstein-vector gravity, emerging gauge symmetry, and de Sitter bounce}, \textit{Phys. Rev. D} \textbf{93} (2016) 044035.
		
		\bibitem{Dzhunushaliev2011}
		V. Dzhunushaliev and V, Folomeev, \textit{Spinor brane}, \textit{Gen. Relativ. Gravit.}, 43 (2011) 1253-1261.
		
		\bibitem{Bazeia2014}
		D. Bazeia, A.S. Lobão Jr., R. Menezes, A.Yu. Petrov and A. J. da Silva, \textit{Braneworld solutions for $ F(R) $ models with non-constant curvature}, \textit{Phys. Lett. B} \textbf{729}  (2014) 127-135.
		
		\bibitem{Bazeia2015}
		D. Bazeia, A.S. Lobão Jr. and R. Menezes, \textit{Thick brane models in generalized theories of gravity}, \textit{Phys. Lett. B} \textbf{743}  (2015) 98-103.
		
		\bibitem{Gu2017}
		B.-M. Gu, Y.-P. Zhang, H. Yu and Y.-X. Liu, \textit{Full linear perturbations and localization of gravity on $ f(R,T) $ brane}, \textit{Eur. Phys. J. C} \textbf{77} (2017) 115.
		
		\bibitem{Rohman2021}
		M.T. Rohman and Triyanta, 
		\textit{Localization of scalar field on $ f(R, T) $ thick Robertson-Walker brane}, 
		\textit{J. Phys.: Conf. Ser.} \textbf{1816} (2021) 012058.
		
		\bibitem{Bezrukov2008}
		F. L. Bezrukov and M. Shaposhnikov, 
		\textit{The Standard Model Higgs boson as the inflaton},
		\textit{Phys. Lett. B} \textbf{659} (2008) 703-706.
		
		\bibitem{Harko2014}
		T. Harko and F.S.N. Lobo,
		\textit{Generalized Curvature-Matter Couplings in Modified Gravity},
		\textit{Galaxies} \textbf{2014} 2, 410-465.
		
		\bibitem{Farakos2005}
		K. Farakos and P. Pasipoularides,
		\textit{Gravity-induced instability and gauge field localization},
		\textit{Phys. Lett. B} \textbf{621} (2005) 224-232.
		
		\bibitem{Farakos2006}
		K. Farakos and P. Pasipoularides,
		\textit{Second Randall-Sundrum brane world scenario with a nonminimally coupled bulk scalar field},
		\textit{Phys. Rev. D} \textbf{73} (2006) 084012.
		
		\bibitem{Guo2012}
		H. Guo, Y. Liu, Z. Zhao and F. Chen,
		\textit{Thick branes with a nonminimally coupled bulk-scalar field},
		\textit{Phys. Rev. D} \textbf{85} (2012) 124033.
		
		\bibitem{Moazzen2017}
		M. Moazzen and Z. Ghalenovi,
		\textit{Non-minimally coupled bulk scalar fields in sine-Gordon braneworld models},
		\textit{Annals of Physics} \textbf{385} (2017) 70-85.

		\bibitem{DeWolfe2000}
		O. DeWolfe, D. Z. Freedman, S. S. Gubser and A. Karch,
		\textit{Modeling the fifth dimension with scalars and gravity},
		\textit{Phys. Rev. D} \textbf{62} (2000) 046008.
		
		\bibitem{Cruz2016}
		W. T. Cruz, R. V. Maluf, L. J. S. Souza and C. A. S. Almeida, 
		\textit{Gravity localization in sine-Gordon braneworlds},
		\textit{Annals of Physics} \textbf{364} (2016) 25-34.
		
		\bibitem{Guerrero2019}
		R. Guerrero and R. O. Rodriguez,
		\textit{Fermions localization on de Sitter branes}, 
		\textit{Phys. Rev. D} \textbf{100} (2019) 104038.
		
	\end{thebibliography}
\end{document}
