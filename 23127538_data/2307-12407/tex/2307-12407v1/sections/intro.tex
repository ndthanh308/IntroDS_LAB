\section{Introduction}\label{intro}

% The force density method and form-finding
The force density method (FDM)~\cite{schek_force_1974} is a form-finding method that generates shapes in static equilibrium for meter-scale 3D structures, such as masonry vaults~\cite{panozzo_designing_2013}, cable nets~\cite{veenendaal_structural_2017} and tensegrity systems~\cite{zhang_adaptive_2006}.
A structure in static equilibrium carries loads like its self-weight or wind pressure only through internal tension and compression forces~\cite{adriaenssens_shell_2014}.
% Axial stresses are a good idea
This axial-dominant mechanical behavior enables a form-found structure to span long distances with low material usage compared to a structure that is not form-found~\cite{schlaich_shell_2018,rippmann_armadillo_2016}.

% Forward form-finding
The FDM is a \textit{forward} physics solver expressed as a function $f(\theta,G)=U$.
Given a structure modeled as a sparse graph $G$ and a set of continuous design parameters $\theta$, the FDM computes a state of static equilibrium $U$ for $G$~(Fig.~\ref{fig:fdm_forward}). 
By inputting different values of $\theta$, the FDM generates a variety of shapes in static equilibrium (Fig.~\ref{fig:fdm_variations}).
% We actually need inverse form-finding
However, in engineering practice, it is necessary to generate not any mechanically efficient 3D shape, but a feasible one that satisfies constraints arising from architectural, fabrication, or other structural requirements.

% Figure environment removed

% Example
Consider the case shown in Fig.~\ref{fig:fdm_inverse} where it is of interest to find a shape in equilibrium that is as close as possible to a target surface $\hat{U}$.
This surface may express architectural intent for a new roof or can represent the geometry of a historical masonry vault that needs to be analyzed for restoration purposes~\cite{panozzo_designing_2013,marmo_thrust_2019}.
Practical structural design, therefore, requires the solution of an \textit{inverse} form-finding problem, a mapping from~${\hat{U}\to\theta}$, where the goal is to estimate adequate values for the parameters $\theta^{\star}$ that are conducive to an equilibrium state with prescribed characteristics $\hat{U}$.

The design space of all possible shapes in static equilibrium parametrized by $\theta$ is vast, particularly as the dimensionality of these parameters grows proportionally to the hundreds or thousands of cables, bricks and blocks that compose a real-world structure.
% Solving inverse form-finding by hand is therefore intractable.
Numerical approaches based on geometric heuristics~\cite{lee_automatic_2016} or genetic algorithms~\cite{koohestani_form-finding_2012} offer limited support to navigate this high-dimensional design space towards feasible designs.
% and to automate the design form-found structures conditioned on target properties.
The current surge of differentiable physics solvers and physics-informed neural networks in structural engineering~\cite{cuvilliers_constrained_2020,chang_learning_2020,xue_jaxfem_2023,wu_jaxsso_2023,pastrana_constrained_2023} provide insights to develop new approaches to tackle inverse form-finding.

% What do we do in this paper
In this paper, we present JAX FDM, a differentiable solver to perform inverse form-finding on 3D structures modeled as pin-jointed bar networks.
JAX FDM implements the FDM in JAX~\cite{bradbury_jax_2018} and solves inverse form-finding problems by estimating adequate inputs to the FDM via gradient-based optimization.
The required forward and backward calculations are executed efficiently by running a differentiable sparse solver on a CPU or a GPU.
After presenting the theory behind our work in Section~\ref{sec:method}, we use our solver to address two inverse form-finding problems: the design of a shell structure that matches an arbitrary target shape (Section~\ref{sec:results:shell}) and the design of cable net with prescribed edge lengths (Section~\ref{sec:results:cablenet}).
JAX FDM is open-source software accessible at this URL:\;\href{https://github.com/arpastrana/jax_fdm}{https://github.com/arpastrana/jax\_fdm}.

% Figure environment removed

% Figure environment removed