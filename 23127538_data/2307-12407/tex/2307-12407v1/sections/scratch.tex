
% Traversing this design space by hand might be possible for small inverse form-finding problems, but it is unfeasible for realistic design problems.
% Solving small inverse form-finding problems by hand is an option, but the process can be laborious and error prone.
% For large problems, it is unfeasible.
% Furthemore, in rare circumstances 
% Solving inverse form-finding problems is 

% While solving small inverse form-finding problems by manually modifying is possible for small, it is a laborious and error-prone process that is feasible only for 


% and metaheuristic 


% Both approches offer limited support to efficiently navigate the high-dimensional space of all possible shapes in static equilibrium conditioned on additional target properties.


% to generate 


% The combinatorial nature of manipulating input makes form-finding by hand laborious and error prone.


% To automate the design of mechanically efficient shapes conditioned on target properties.
% rapid 
% versatile and reusable: No two structures are (or want to be) alike. 
% Current methods offer limited support to navigate the high-dimensional space of all possible structurally-efficient shapes conditioned on additional target properties.



% The FDM is can generate a plethora of mechanically efficient shapes for $G$ by inputting various combinations of values of $\mathbf{q}$ as shown in Figure X.


% The FDM can generate a plethora of mechanically efficient shapes for $G$ by inputting various combinations of values of $\mathbf{q}$ as shown in Figure X.




% Automatic design

% However, practical structural design, are often posed as inverse form finding problems.
% structural efficiency should coexist with other constraints 
% the goal is to find the parameters of fdm that are conducive a shape in static equilibrium conditioned of design constraints. 
% automatic and general solution
% However, practical structural design problems are often posed as inverse form-finding problems, where the goal is to find a shape that satisfies a set of design constraints.


% Scratch

% JAX FDM automates the design of structurally efficient shapes for 3D structures that satisfy architectural, fabrication or other mechanical goals.
% JAX FDM implements the FDM in JAX \cite{bradbury_jax_2018}, and it thus applies many of JAX's perks: automatic differentiation, vectorization parallelization, and just-in-time (JIT) compilation.
% We presented JAX FDM, an open-source solver that streamlines the design of efficient 3D load-bearing structures conditioned on target constraints.
% Structures that are mechanically efficient and also architectural, buildable and environmentally-friendly.
% Future work.
% In the future 
% End-to-end training of a model that combines of a neural network with form-finding simulations could lead to the development of robust and data-efficient neural surrogates that further accelerate the solution of inverse form-finding problems.
% Our work contributes to bridging the gap between structural analysis and machine learning research, and to the discovery and the design of new and expressive structural systems that help mitigating carbon footprint of the construction sector.


% JAX FDM can be paired with optimizers and neural network libraries in the JAX ecosystem to integrate form-finding simulations into physics-informed neural networks.
% JAX FDM enables the solution of inverse form-finding problems for discrete force networks using the force density method (FDM) and gradient-based optimization. It streamlines the integration of form-finding simulations into deep learning models for machine learning research.

% Scratch

% Finally, we illustrate the potential of JAX FDM to be seamlessly integrated with neural networks in Section \ref{sec:results:nn}.

% Our framework implements the force density method in JAX [cite jax], an auto-differentiable and hardware-accelerated array processing library written in Python.

% The framework implements the FDM in JAX [cite], and features a differentiable sparse linear solver (Section \ref{sec:method:sparse}) which, in combination with hardware acceleration provided by GPUs, expedites the solution of practical inverse form-finding tasks (Section \ref{sec:method:inverse}).
% Sparse solver

% As part of the JAX ecosystem: JAX FDM shares many of JAX perks like automatic differentiation and hardware acceleration via GPUs.

% Current approaches to solve inverse form-finding problems offer limited support in automating the design form-found structures conditioned on additional target properties.
% The combinatorial nature of the problem renders inverse form-finding by hand intractable and error-prone.
% Other approaches based on geometric heuristics and genetic algorithms have been proposed [cite], but these can be computationally inefficient and do not offer convergence guarantees on the solution of an inverse form-finding problem.
% versatile and reusable: No two structures are (or want to be) alike. 

% Why the force density method - may remove this paragraph!
% The FDM remains a well-established form-finding method for structural design despite it was first formulated fifty years ago [cite].
% The FDM is a well-established form-finding method in structural design [cite].
% One of the reasons for the success of the FDM in the literature is that it linearizes the otherwise non-linear task of calculating of shape in static equilibrium for a structure without making assumptions on the material properties of a structure, making form-finding fast and tractable.
% Another reason is that the mathematics behind FDM are general enough to make it applicable to form-find a wide range compression-only, tension-only and mixed structural systems.

% The design of buildings, bridges and towers in the building industry is an interdisciplinary process whereby engineers, architects, contractors and other stakeholders inform the design process.

% Examples in different materials: trumpf bridge and armadillo vault
% The Trumpf bridge \cite{schlaich_shell_2018} and the Armadillo vault \cite{rippmann_armadillo_2016} are contemporary examples proving the efficiency of a form-found structure: they span $28$ and $15$ meters with a thickness-to-span ratio lower than that of a chicken egg \cite{altuntas_mechanical_2010}.

% on other structures insofar they are modeled as a structure supported by our solver.
% \subsection[]{FDM as a layer in a neural network?}\label{sec:results:nn}
% Autoencoder architecture.
% \lipsum[1]

% an initial design is produced with a forward run of FDM (Figure \ref*{fig:rhoen}a).
% % Figure environment removed

% Suppose you are an engineer tasked with the structural design of a masonry vault that follows the geometry given by the project architect in a COMPAS network [cite compas], and pictured in Figure X.
% You want to calculate a compression-only shape in static equilibrium to build the vault only with clay bricks (i.e. no tensile steel reinforcement), but you wonder if there exists one such shape that is a best-fit to the architectural target. You open Python in your terminal and import \texttt{jax\_fdm.equilibrium} as \texttt{jfe}.
% This example solves a problem proposed in \cite{panozzo_designing_2013}.
% We want to calculate a compression-only thrust network that is a best-fit to the architectural target to build the structure only with clay bricks and no tensile reinforcement \cite{marmo_thrust_2019}.
% We want to calculate a compression-only thrust network that is a best-fit to the architectural target to build the structure only with clay bricks and no tensile reinforcement \cite{marmo_thrust_2019}.
% A masonry structure can be modeled as a thrust network based on Heyman's safe theorem [cite]. helpers to convert this geometry into a 
% \texttt{jax\_fdm.equilibrium}