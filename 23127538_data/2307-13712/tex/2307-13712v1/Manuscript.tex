\documentclass[11pt,a4paper]{article}
\usepackage{float}
%\usepackage{graphics,natbib,amssymb}
%\usepackage{epsfig}
\usepackage{multirow}
%\usepackage{graphics}
\usepackage[cp1252,utf8]{inputenc}
\usepackage{hyperref}
\usepackage{graphicx}
\usepackage{wrapfig}
\usepackage{amsmath}
\usepackage{xparse}
\usepackage{amsfonts}
\usepackage{amssymb}
\usepackage{relsize} %big math signs
\usepackage[top=1in, bottom=2cm, left=1.5cm, right=1.5cm]{geometry}
%\usepackage{showkeys}
\usepackage{authblk}
\usepackage{ulem}
\usepackage{physics}
%\usepackage{xparse}
\usepackage{xcolor}
\NewDocumentCommand{\tens}{t_}
{%
	\IfBooleanTF{#1}
	{\tensop}
	{\otimes}%
}
\NewDocumentCommand{\tensop}{m}
{%
	\mathbin{\mathop{\otimes}\displaylimits_{#1}}%
}
\usepackage{changepage} %to make tables/figures start from a bit left

\usepackage{afterpage}

\usepackage{placeins}

\usepackage{wrapfig}
%\usepackage{pgfplots}
\usepackage{cite} % for compact citations
\usepackage[nottoc,notlof,notlot]{tocbibind} % to put the bibliography in the ToC
%\usepackage{eepic}

%\usepackage{showkeys}
%\usepackage{setspace}
%\doublespacing




\title{\bf Mixed state entanglement measures for the dipole deformed supersymmetric Yang-Mills theory}
\vskip 1cm
\author[a]{\bf  Anirban Roy Chowdhury \thanks{iamanirban@bose.res.in}}
\author[b]{\bf  Ashis Saha \thanks{ashisphys18@klyuniv.ac.in}}
\author[c]{\bf Sunandan Gangopadhyay \thanks{sunandan.gangopadhyay@bose.res.in}}

\affil[a,c]{\textit{Department of Astrophysics and High Energy Physics\\}
\textit{S.N.~Bose National Centre for Basic Sciences,}
\textit{JD Block, Sector-III, Salt Lake, Kolkata 700106, India}}
\affil[b]{\textit{Department of Physics, University of Kalyani, Kalyani 741235, India}}

\renewcommand\Authands{, }

\date{}

\begin{document}
	\maketitle
	\begin{abstract}
		\noindent Two different entanglement measures for mixed states, namely, the entanglement of purification and entanglement negativity has been holographically computed for the dipole deformed supersymmetric Yang-Mills (SYM) theory by considering its gravity dual. The dipole deformation induces non-locality in the SYM theory which is characterized by a length-scale $a=\lambda^{\frac{1}{2}}\tilde{L}$. Furthermore, the mentioned scale of non-locality introduces three different domains of the theory, $au_{t}\le1$, $1\le au_{t}< au_{b}$, and $au_{t}\sim au_{b}$, where $au_{b}$ is the UV- cutoff. By following the holographic techniques, we have computed the entanglement entropy for a strip like subsystem of length $l$. This has been done both analytically  and numerically, and by using these results we have calculated the mutual information between two disjoint subsystems $A$ and $B$. Based upon the $E_{P}=E_{W}$ duality, the entanglement of purification ($E_{P}$) is then computed and the effects of dipole deformation in this context have been studied. Finally, we proceed to compute entanglement negativity for this theory and compare the obtained result with that of the standard SYM theory in order to get a better understanding about the effects of the non-locality.
\end{abstract}
\section{Introduction}
The gauge/gravity duality \cite{Maldacena:1997re,Witten:1998qj,Aharony:1999ti} has emerged as one of the most interesting field of study in recent times. Using this duality one can study the properties of a strongly coupled quantum field theory with the help of a classical gravitational solution. Among many applications of this duality, it gives us a systematic way to study various information theoretic quantities of the strongly coupled quantum field theory. These quantities in the field theory side have a nice description in terms of the geometric quantities in the gravity side. In this work we have computed various information theoretic quantities like entanglement entropy, mutual information and entanglement of purification for dipole deformed supersymmetric Yang-Mills theory (SYM-theory) in the holographic set up. In quantum information theory, the entanglement entropy (EE) is one of the most fundamental quantity as it efficiently measures quantum correlation for a system in pure state. We can define EE of a pure bipartite system in the following way.\\ Let us consider a pure bipartite system $A\cup B$, with Hilbert space $\mathcal{H}_{A}\tens\mathcal{H}_{B}$. So the system can be described by the density matrix $\rho_{AB}$ which is given by 
\begin{eqnarray}
	\rho_{AB}=\ket{\psi_{AB}}\tens\bra{\psi_{AB}}~;~\ket{\psi_{AB}}\in\mathcal{H}_{A}\tens\mathcal{H}_{B}~.
\end{eqnarray}
Therefore, one can define the entanglement entropy of the subsystem $A$ as \cite{Chuang:2000}
\begin{eqnarray}
	S_{A}=-\tr(\rho_{A}\log\rho_{A})~;~\rho_{A}=\tr_{B}(\rho_{AB})
\end{eqnarray}
where $\rho_{A}$ is the reduced density matrix of the subsystem $A$, which can be obtained by taking the trace of the total density matrix with respect to the subsystem $B$. We can also show that for a pure state, EE satisfies the equality $S_{A}=S_{B}$ \cite{Chuang:2000}. Furthermore, the EE corresponding to the total density matrix, that is $S_{\mathrm{tot}}=-\tr(\rho_{AB}\log\rho_{AB})$ is zero. This in turn means that one has complete information about the system. However, in the context of quantum field theory it is quite difficult to compute entanglement entropy and the procedure of computation is known as the replica technique. On the other hand, the gauge/gravity duality inspires holographic way of computing EE is remarkably simple. It was Ryu and Takyanagi who had first shown how one can compute the entanglement entropy of strongly coupled conformal field theory with the help of its gravity dual \cite{Ryu:2006bv,Ryu:2006ef,Nishioka:2009un}. This holographic formulation of computing the EE is known as the Ryu-Takayanagi (RT) prescription \cite{Ryu:2006bv,Ryu:2006ef,Nishioka:2009un}. RT prescription states the EE of a reduced density matrix of the conformal field theory is related to the area of a codimension-$2$ static minimal surface in the bulk theory. The entropy obtained from the RT- prescription is also known as the holographic entanglement entropy (HEE), which is nothing but holographic counterpart of the EE (or von-Neumann entropy). According to the RT-prescription the holographic entanglement entropy of subsystem $A$ in the boundary theory is given by 
\begin{eqnarray}
	S_{HEE}(A)=\frac{\mathrm{Area}(\Gamma^{A}_{\mathrm{min}})}{4G_N}
\end{eqnarray}
where $\Gamma^{A}_{\mathrm{min}}$ is the codimension-2 static minimal surface in the bulk associated to the subsystem $A$ at the boundary and $G_{N}$ is the Newton's gravitational constant. Keeping the idea of entanglement entropy in mind, one can define another important information theoretic quantity, known as the mutual information. Let us consider a bipartite system $A\cup B$, which is described by a density matrix $\rho_{AB}$. Now the mutual information ($I(A:B)$) between two subsystems $A$ and $B$ is given by 
\begin{eqnarray}
	I(A:B)=S(A)+S(B)-S(A\cup B)~.
\end{eqnarray}
One can show that the mutual information between two subsystems is positive and definite quantity.\\
Now we will proceed to define a measure of quantum correlation when the system under consideration is in a mixed state. Various studies in this direction suggest that entanglement of purification (EoP) is one of the suitable candidates which can measure quantum correlation in this scenario.\\
For a bipartite mixed state the entanglement of purification can be defined in the following way. Let us consider bipartite system $AB$, which is in a mixed state and the system is described by a density matrix $\rho_{AB}$. The process of purification suggests that we have to construct a pure state $\ket{\psi_{AB}}$ from the mixed state $\rho_{AB}$. This can be done by adding auxiliary degrees of freedom $(A^{\prime} B^{\prime})$ to the original system such that $A$ is entangled with $A^{\prime}$ and $B$ is entangled with the $B^{\prime}$. By adding this auxiliary degrees of freedom to the original system we can make the total system in a pure state.
\begin{eqnarray}
	\rho_{AB} = \tr_{A^{\prime}B^{\prime}}\ket{\psi}\bra{\psi};~\psi \in \mathcal{H}_{AA^{\prime}BB^{\prime}}=\mathcal{H}_{AA^{\prime}} \tens \mathcal{H}_{BB^{\prime}}
\end{eqnarray}
The states $\ket{\psi}$ are denoted as the purifications of $\rho_{AB}$.
 Now the entanglement of purification (EoP) defined as 
\cite{Terhal_2002}
\begin{eqnarray}\label{EoP}
	E_P(\rho_{AB})\equiv E_P(A,B) = \mathop{\mathrm{min}}_{\ket{\psi}}S(\rho_{AA^{\prime}});~\rho_{AA^{\prime}} = \tr_{BB^{\prime}}\ket{\psi}\bra{\psi}
\end{eqnarray}
where the minimization is taken over all possible state $\ket{\psi}$ with the property $\rho_{AB} = \tr_{A^{\prime}B^{\prime}}\ket{\psi}\bra{\psi}$. The EoP has important implications in various areas of quantum information theory, such as quantum teleportation, quantum cryptography, and quantum error correction. It has also been used to study fundamental questions in quantum mechanics, such as the quantum-to-classical transition and the nature of quantum entanglement. However, in the context of quantum field theory it is not a easy task to compute EoP. Here the gauge/gravity duality comes up with a solution. The AdS/CFT correspondence suggests that the holographic counterpart of the EoP is nothing but the minimal cross section of the entanglement wedge \cite{Takayanagi:2017knl,Nguyen:2017yqw}.
This observation has led us to the $E_P=E_W$ duality. It is to be noted that this duality is yet to be proven.
It has been observed that both $E_P(A,B)$ and $E_W(A,B)$ satisfy the following properties \cite{Takayanagi:2017knl}
\begin{eqnarray}\label{prop}
	&&~E_P(A,B) = S_{EE}(A)=S_{EE}(B);~\rho_{AB}^2=\rho_{AB}\nonumber\\
	&&~\frac{1}{2} I(A:B) \leq E_P(A,B) \leq min\left[S_{EE}(A),S_{EE}(B)\right] \nonumber\\
	&& \frac{I(A:B)+I(A:C)}{2} \leq E_P(A,B\cup C)~~.
\end{eqnarray} 
Another suitable measure of quantum correlation for mixed state scenario is known as the entanglement negativity or logarithmic negativity. To define entanglement negativity let us consider a tripartite quantum mechanical system with the subsystems denoted as $A_{1}$,$A_{2}$ and $B$. The associated conditions are $A=A_{1}\cup A_{2}$ and $B=A^{c}$ \cite{Jain:2017aqk}.  The Hilbert space of the  subsystem $A$ can be written as $\mathcal{H}=\mathcal{H}_{1}\tens\mathcal{H}_{2}$, where $\mathcal{H}_{1}$ is the Hilbert space corresponding to  the subsystem $A_{1}$ and $\mathcal{H}_{2}$ is the Hilbert space associated with the subsystem $A_{2}$. To obtain the reduced density matrix of the subsystem $A$ we have to take trace over the full density matrix of the system with respect to its complement, that is, $A^{c}=B$. This reads
\begin{equation}
	\rho_{A}=\tr_{A^{c}}(\rho_{AB})
\end{equation}
where $\rho_{AB}$ is the total (mixed) density matrix of the system. To compute the entanglement negativity (or the logarithmic negativity) we have to take the partial transpose of the reduced density matrix over one of the subsystems in the given bipartite system. We can do this in the following way. Consider $|e_{i}^{(1)}\rangle$ and $|e_{i}^{(2)}\rangle$ to be the basis of the Hilbert space associated with the subsystems $A_{1}$ and $A_{2}$ respectively. The partial transpose of the reduced density matrix with respect to $A_{2}$ is then defined as
\begin{eqnarray}
	\bra{e_{i}^{(1)}e_{j}^{(2)}}\rho_{A}^{T_{2}}\ket{e_{k}^{(1)}e_{l}^{(2)}}=\bra{e_{i}^{(1)}e_{l}^{(2)}}\rho_{A}\ket{e_{k}^{(1)}e_{j}^{(2)}}
\end{eqnarray}
where $\rho_{A}^{T_{2}}$ represents the partial transpose of the total density matrix $\rho$ with respect to $A^{c}$. Entanglement negativity measures the degree to which  $\rho_{A}^{T_{2}}$ is not positive, which signifies the term `negativity'. We denote the trace norm of the partial transposed reduced density matrix by $\parallel\rho_{A}^{T_{2}}\parallel_{1}$. Now one can define two quantities, one is negativity and another one is the entanglement negativity or logarithmic negativity. The negativity between two subsystems $A_{1}$ and $A_{2}$ is defined as \cite{Vidal:2002zz}
\begin{equation}
	\mathcal{N}(\rho)=\frac{\parallel\rho_{A}^{T_{2}}\parallel_{1}-1}{2}~.
\end{equation}
The above corresponds to the absolute value of the sum of negative eigenvalues of $\rho_{A}^{T_{2}}$. One can show that for a unentangled product state, $\mathcal{N}(\rho)$ vanishes.\\
On the other hand, entanglement negativity or logarithmic negativity between two subsystems $A_{1}$ and $A_{2}$ is defined as \cite{Vidal:2002zz}
\begin{equation}
	E_{N}(\rho)=\ln(\parallel\rho_{A}^{T_{2}}\parallel_{1})~~.
\end{equation}
The gauge/gravity duality provides us a systematic way to calculate the above mentioned quantity by using the bulk/boundary dictionary.\\
Apart from EoP and entanglement negativity, there are other quantities such as odd entropy \cite{Ghasemi:2021jiy,Basak:2022gcv}, reflected entropy \cite{Dutta:2019gen,Chu:2019etd,Basak:2022cjs} which are also measures of quantum correlation for a system in mixed state.\\
In this paper we will holographically compute the EE, MI, EoP and entanglement negativity for the dipole deformed super symmetric Yang-Mills (SYM) theory. This kind of deformation introduces a scale of non-locality in the theory. So it is important to study how the non locality influences various measures of the quantum correlations. We will show that the holographic entanglement entropy follows a volume like law instead of area law below a certain length scale. Another important fact is that, this length scale is independent of the UV-cutoff. The dipole moment is introduced through a deformation of the gauge field. 
%{which is given by a non-local expression involving the position of the gauge field and a deformation parameter.}
The deformation parameter controls the strength of the deformation, and it is a new parameter in the theory that was not present in the original supersymmetric Yang-Mills theory. The dipole deformation breaks the conformal symmetry of the theory, leading to the emergence of a new scale in the system. This new scale is related to the size of the dipole moment and affects the behaviour of the theory at both low and high energies. One of the main consequences of the dipole deformation is the emergence of a new type of interaction between the gauge fields. This new interaction, known as the dipole interaction, involves the exchange of particles that carry dipole moments. The dipole interaction is responsible for many of the new phenomena that arise in dipole deformed supersymmetric Yang-Mills theory, including the emergence of new bound states and the modification of the scattering amplitudes. Overall, the dipole deformed Supersymmetric Yang-Mills theory is a fascinating area of research that has the potential to shed light on some of the most fundamental questions in physics, including the nature of dark matter, the behaviour of black holes, and the structure of the universe at the smallest scales. Dipole deformed supersymmetric Yang-Mills theory has also been studied in the context of holographic duality, where it has been shown to be related to a deformation of the AdS/CFT correspondence. This has led to new insights into the relationship between quantum gravity and field theory, and has opened up new avenues for research in both fields.\\
The paper is organised in the following way. In section\ref{sec1}~, we have briefly discuss the dipole deformed SYM theory and its gravity dual. Then in section\ref{sec2}~, we have obtained the relation between the subsystem size and turning point in the different domains of the theory. In this section we have also computed the holographic entanglement entropy for different domains. In section\ref{sec3}~, we have computed the holographic mutual information and entanglement wedge cross section in different domains of the theory and then in section\ref{ENsec}~, we compute the entanglement negativity. We conclude in section\ref{seca}~.
\section{Dipole deformed supersymmetric Yang-Mills theory and its gravity dual }\label{sec1}
In this section, we shall briefly discuss the dipole deformation of $\mathcal{N}=4$ SYM theory along with its gravity dual in the strong coupling limit. This simple deformation is done by introducing an external dipole moment which in turn breaks the Lorentz symmetry of the system. The dipole deformation modifies the action of the standard $\mathcal{N}=4$ SYM theory by introducing a new term proportional to the product of the gauge field strength and the dipole moment. The resulting theory is still supersymmetric, which means that it has an extended symmetry that relates bosonic and fermionic degrees of freedom. One of the most interesting features of dipole deformed $\mathcal{N}=4$ SYM theory is the emergence of a new scale in the system, which is related to the size of the dipole \cite{ouyang2017semiclassical,Guica:2017mtd}. Emergence of this length scale affects the behaviour of the theory at both low and high energies. For instance, at low energies, the dipole deformation leads to the appearance of a non-trivial ground state, which breaks the supersymmetry of the theory \cite{Gursoy:2005cn,article,Bergman:2000cw,Delduc:2013qra}. This ground state is characterized by a set of vortices that are responsible for the formation of a condensate of the gauge field. At high energies, the dipole deformation affects the scattering amplitudes of the gauge bosons \cite{article,PhysRevD.107.114024}, leading to the emergence of new kinematical regions that are not present in the original theory. For instance, in the behaviour of the scattering amplitudes in the presence of a background magnetic field, where the dipole deformation leads to a modification of the Landau levels of the charged particles.\\
In the dipole deformed SYM theory the ordinary algebric product of two fields is deformed in the following way \cite{Chakravarty:2000qd,Bergman:2000cw,Dasgupta:2000ry}
 \begin{eqnarray}
 	(f{\tilde{\star}}g)(\vec{x})=f\left(\vec{x}-\frac{\vec{L_{f}}}{2}\right)g\left(\vec{x}+\frac{\vec{L_{g}}}{2}\right)	
 \end{eqnarray}
 where $\vec{L_{f}}$ and $\vec{L_{g}}$ are dipole vectors associated to the fields $f$ and $g$ respectively. In order to make this new product associative, one needs to assign dipole vector $\vec{L_{f}}+\vec{L_{g}}$ to $(f{\tilde{\star}}g)(\vec{x})$. Here, we will consider $\vec{L}=L\hat{x}$ corresponding to some fixed length scale $L$, this implies that our theory is non-local only in the $x$-direction. There is also another kind of deformation possible for the SYM theory which is known as the noncommutatitve deformation \cite{Seiberg:1999vs,Szabo:2001kg,Ardalan:1998ce}. Unlike the noncommutative SYM theory, the dipole deformed SYM theory does not exhibit any UV/IR mixing property \cite{Karczmarek:2013xxa,Chowdhury:2021idy}.\\
In the gauge/gravity duality set up, the gravitational dual of dipole-deformed SYM theory is a type IIB string theory in AdS$_5$ which contains a non-trivial dilaton and axion fields \cite{article,Kawaguchi:2014qwa,Flambaum:2015ica}. As mentioned earlier, the dipole deformation deals with the fact that one has to introduce an external dipole moment which in turn breaks the Lorentz symmetry of the theory. In the gravity dual, this deformation is realized with the presence of a non-trivial dilaton field along with a non-trivial axion field. The dilaton field is related to the coupling constant of the SYM theory, while the axion field is related to the dipole moment \cite{PhysRevD.91.111702}. The gravity dual of dipole-deformed SYM theory has been extensively studied in the literature, and several important results have been obtained. For example, it has been shown that the dual theory exhibits a nontrivial scaling behavior, which is related to the presence of the dipole moment. The holographic dual also predicts the existence of a new phase in the SYM theory, which is characterized by the breaking of the conformal symmetry and the presence of a dipole moment. This phase is known as the dipole phase. One of the important applications of the gravity dual of dipole-deformed SYM theory is the study of quark-antiquark potential. The holographic calculation of the potential shows that it has a Coulomb-like behavior at short distances, while at large distances it exhibits a linear confinement \cite{universe9030114,unknown}. The metric associated to the gravity dual of dipole deformed SYM theory (in string frame) reads \cite{Karczmarek:2013xxa,Bergman:2001rw}
\begin{eqnarray}\label{gd}
	ds^{2}&=&R^2\left[u^2\left(-dt^2+f(u)dx^{2}+dy^{2}+dz^2\right)+\frac{du^{2}}{u^2}\right]+\mathrm{metric~ on ~the ~deformed}~S^5\nonumber\\
	e^{2\phi}&=&g_{s}^{2}f(u)~;~f(u)=\frac{1}{1+a^2u^2}~;~B_{x\psi}=-\frac{1-f(u)}{\tilde{L}}=-\frac{R^2}{\alpha^{\prime}}au^{2}f(u)~
\end{eqnarray}
where $a=\lambda^{\frac{1}{2}}\tilde{L}$ with $\tilde{L}=\frac{L}{2\pi}$ is the scale of non locality in the strong coupling limit and $\phi$ is the non-zero dilaton profile. The $S_{5}$ part of the metric in the gravity dual is deformed by expressing $S_{5}$ as $S_{1}$ fibration over $\mathcal{\Bbb C\Bbb P}^{2}$ \cite{Bergman:2001rw}. $\psi$ is the global angular $1$-form of the Hopf fibration.
\section{Computation of holographic entanglement entropy}\label{sec2}
In this section we start our analysis by considering a strip-like subsystem $A$. The geometry of this subsystem is specified by the volume $V_{sub}=L^{d-2}l$, with $-\frac{l}{2}\le x\le \frac{l}{2}$, and $y,z \in \left[0,L\right]$ with $L\rightarrow\infty$. We will assume that the width in $y$ and $z$ direction is fixed and it can vary only along the $x$ direction. In order to compute the HEE, we shall follow the standard Ryu-Takayanagi prescription \cite{Ryu:2006bv,Ryu:2006ef}. We choose the parametrisation $u=u(x)$ in order to compute the surface area of the co-dimension two RT surface $\Gamma_{A}^{min}$. Furthermore, we want to mention that the gravity dual metric (given in eq(\ref{gd})) is written in the string frame. But all the computations should be done in the Einstein frame. For that we will use the following transformation 
\begin{eqnarray}
	g_{\mu\nu}^{E}\rightarrow e^{-\frac{\phi}{2}}g_{\mu\nu}^{S}
\end{eqnarray}
By using the above transformation one can easily show the following relation 
\begin{eqnarray}
	\sqrt{g_{8}^{E}}=e^{-2\phi}\sqrt{g_{8}^{S}}~.
\end{eqnarray}
Keeping these above results in mind, we now proceed to compute the HEE. This reads \cite{Karczmarek:2013xxa}
\begin{eqnarray}\label{see}
	S_{\mathrm{HEE}}&=&\frac{\mathrm{Area}(\Gamma^{A}_{\mathrm{min}})}{4G_N}\nonumber\\
	&=&\frac{2L^{2}\pi^{3}R^{8}}{4G_{N}^{(10)}g_{s}^{2}}\int_{-\frac{l}{2}}^{0}~dx~u^{2}\left(1+\frac{u^{\prime2}}{f(u)u^{4}}\right)^{\frac{1}{2}}~;~u^{\prime}=\frac{du}{dx}~.
\end{eqnarray}
It can be observed that the integrand of the above expression is independent of $x$ coordinate. Therefore $x$ has the reputation of being the the cyclic coordinate in this set up. This gives rise to following conserved quantity
\begin{eqnarray}
	\mathcal{H}=-\frac{u^{3}}{\left(1+\frac{u^{\prime2}}{f(u)u^{4}}\right)^{\frac{1}{2}}}=\mathrm{constant}\equiv c~.
\end{eqnarray}
The constant, $c$ can be fixed by using the fact that at the turning point $u=u_{t}$, the quantity $u^\prime=\frac{du}{dx}$ vanishes. This implies one can have the following differential equation
\begin{eqnarray}\label{len}
	\frac{du}{dx}=\sqrt{u^{4}f(u)\left[\left(\frac{u}{u_{t}}\right)^{6}-1\right]}~.
\end{eqnarray}
By using the above expression in eq.(\ref{see}), we obtain the result for HEE (in the dimensionless form) in terms of the bulk coordinate
\begin{eqnarray}\label{see1}
	a^{2}S_{\mathrm{HEE}}=\frac{2L^{2}\pi^{3}R^{8}}{4G_{N}^{(10)}g_{s}^{2}} (au_{t})^{2}\int_{\frac{au_{t}}{au_{b}}}^{1}\frac{dp}{p^{4}}\frac{\left(p^{2}+(au_{t}^{2})\right)^{\frac{1}{2}}}{\sqrt{1-p^{6}}}~;~ p=\frac{au_{t}}{au}~.
\end{eqnarray}
In the above, we have used the following boundary condition (which regularizes the area functional)
\begin{eqnarray}\label{bc}
	u\left(x=\pm\frac{l}{2}\right)=u_{b}=\frac{1}{\epsilon}~.
\end{eqnarray}
The above boundary condition along with the expression given in eq.(\ref{len}), we obtain the the subsystem length (in the dimensionless form) in terms of the bulk coordinate
\begin{eqnarray}\label{len1}
	\frac{l}{a}=\frac{2}{au_{t}}\int_{\frac{au_{t}}{au_{b}}}^{1} dp~\frac{p^{2}\sqrt{p^{2}+(au_{t})^{2}}}{\sqrt{1-p^{6}}}~.
\end{eqnarray}
 We now proceed to compute the subsystem size in terms of the turning point for two different domains of the theory, namely for $au_{t}\le1$ and $au_{t}\ge 1$. Furthermore, the domain $au_{t}\ge1$ once again can be divided into two regions, one is  $1\le au_{t}<au_{b}$ and the other one is $au_{t}\sim au_{b}$. A similar type of analysis for the noncommutative SYM theory was done in \cite{Chowdhury:2021idy}. In the domain $au_{t}\le1$, eq.(\ref{len1}) can be recast to the following form 
\begin{eqnarray}
	\frac{l}{a}=\frac{2}{au_{t}}\left[\int_{\frac{au_{t}}{au_{b}}}^{au_{t}}dp\frac{p^{2}\sqrt{p^{2}+(au_{t})^{2}}}{\sqrt{1-p^{6}}}+\int_{au_{t}}^{1}dp\frac{p^{2}\sqrt{p^{2}+(au_{t})^{2}}}{\sqrt{1-p^{6}}} \right]~.
\end{eqnarray}
From the above expression one can observe that in the first integral $0\le p\le au_{t}$, this implies that one can assume $\frac{p}{au_{t}}< 1$ and perform an expansion to keep terms upto $\mathcal{O}\left(\frac{p}{au_{t}}\right)^{2}$. On the other hand in the second integral we have $(au_{t})\le p\le 1$ and once again we can do an expansion by assuming $\frac{au_{t}}{p}<1$ and keep terms upto $\mathcal{O}\left(\frac{au_{t}}{p}\right)^{2}$.  Therefore under this approximation the subsystem length in terms of the turning point reads
\begin{eqnarray}\label{Len}
	\frac{l}{a}&\approx&\frac{\sqrt{\pi}}{2(au_{t})}\frac{\Gamma\left(\frac{5}{3}\right)}{\Gamma\left(\frac{7}{6}\right)}+\frac{\sqrt{\pi}}{2}\frac{\Gamma\left(\frac{4}{3}\right)}{\Gamma\left(\frac{5}{6}\right)}(au_{t})-\sum_{n=0}^{\infty}\frac{1}{\sqrt{\pi}}\frac{\Gamma\left(n+\frac{1}{2}\right)}{\Gamma\left(n+1\right)}\left[\frac{2}{(6n+4)}+\frac{1}{(6n+2)}\right](au_{t})^{6n+3}\nonumber\\
	&+&\sum_{n=0}^{\infty}\frac{1}{\sqrt{\pi}}\frac{\Gamma\left(n+\frac{1}{2}\right)}{\Gamma\left(n+1\right)}\left[\frac{2\left(1-(1/au_{b})^{6n+3}\right)}{(6n+3)}+\frac{\left(1-(1/au_{b})^{6n+5}\right)}{(6n+5)}\right](au_{t})^{6n+3}~.
\end{eqnarray} 
We can also express the turning point in terms of the subsystem length by inverting the above expression. This obtained to be
\begin{eqnarray}\label{tp}
au_{t}&\approx&\frac{\sqrt{\pi}}{2}\frac{\Gamma\left(5/3\right)}{\Gamma\left(7/6\right)}\frac{1}{\left(\frac{l}{a}\right)}\Bigg[1+\frac{\Gamma\left(4/3\right)\Gamma\left(7/6\right)}{\Gamma\left(5/6\right)\Gamma\left(5/3\right)}\left(\frac{\frac{\sqrt{\pi}}{2}\frac{\Gamma\left(5/3\right)}{\Gamma\left(7/6\right)}}{\left(\frac{l}{a}\right)}\right)^{2}\nonumber\\
&&-\frac{2\Gamma\left(7/6\right)}{\sqrt{\pi}\Gamma\left(5/3\right)}\left(\frac{\frac{\sqrt{\pi}}{2}\frac{\Gamma\left(5/3\right)}{\Gamma\left(7/6\right)}}{\left(\frac{l}{a}\right)}\right)^{3}\left(\frac{2}{15}+\frac{2}{3}\frac{1}{(au_{b})^{3}}+\frac{1}{5(au_{b})^{5}}\right)\Bigg]\nonumber\\
&\equiv& \frac{\lambda_{1}}{\left(\frac{l}{a}\right)}+\frac{\lambda_{2}}{\left(\frac{l}{a}\right)^{3}}+\frac{\lambda_{3}}{\left(\frac{l}{a}\right)^{4}}~.
\end{eqnarray}
In the last line of above expression we have introduced the following quantities
\begin{eqnarray}
	\lambda_{1}&=&\frac{\sqrt{\pi}}{2}\frac{\Gamma\left(5/3\right)}{\Gamma\left(7/6\right)}\nonumber\\
	\lambda_{2}&=&\frac{\sqrt{\pi}}{2}\frac{\Gamma\left(4/3\right)}{\Gamma\left(5/6\right)}\left(\frac{\sqrt{\pi}}{2}\frac{\Gamma\left(5/3\right)}{\Gamma\left(7/6\right)}\right)^2\nonumber\\
	\lambda_{3}&=&-\left(\frac{\sqrt{\pi}}{2}\frac{\Gamma\left(5/3\right)}{\Gamma\left(7/6\right)}\right)^{3}\left(\frac{2}{15}+\frac{2}{3}\frac{1}{(au_{b})^{3}}+\frac{1}{5(au_{b})^{5}}\right)~.
\end{eqnarray} 
On the other hand for the domain $1\le(au_{t})<au_{b}$, the subsystem length in terms of the turning point can be written down as 
\begin{eqnarray}\label{Len2}
	\frac{l}{a}&=&2\int_{\frac{au_{t}}{au_{b}}}^{1}dp\sum_{n=0}^{\infty}\sum_{m=0}^{\infty}\frac{\Gamma\left(n+\frac{1}{2}\right)}{\Gamma\left(n+1\right)\Gamma\left(m+1\right)\Gamma\left(\frac{3}{2}-m\right)}\frac{p^{6n+2m+2}}{(au_{t})^{2m}}\nonumber\\
	&=&\sum_{n=0}^{\infty}\sum_{m=0}^{\infty}\frac{\Gamma\left(n+\frac{1}{2}\right)}{\Gamma\left(n+1\right)\Gamma\left(m+1\right)\Gamma\left(\frac{3}{2}-m\right)}\frac{1}{(au_{t})^{2m}}\frac{1}{(6n+2m+3)}\left[1-\left(\frac{au_{t}}{au_{b}}\right)^{6n+2m+3}\right]~.
\end{eqnarray}
In obtaining the above result, we have used the following identities
\begin{eqnarray}
	\sqrt{1+\left(\frac{p}{au_{t}}\right)^{2}}&=&\sum_{m=0}^{\infty}\frac{\Gamma\left(\frac{3}{2}\right)}{\Gamma\left(m+1\right)\Gamma\left(\frac{3}{2}-m\right)}\left(\frac{p}{au_{t}}\right)^{2m}~; ~\left(\frac{p}{au_{t}}\right)<1\nonumber\\
	\frac{1}{\sqrt{1-p^{6}}}&=&\sum_{n=0}^{\infty}\frac{1}{\sqrt{\pi}}\frac{\Gamma\left(n+\frac{1}{2}\right)}{\Gamma\left(n+1\right)}p^{6n}~.\nonumber
\end{eqnarray}
Before proceeding further we want to mention that, we can obtian the realtion between the subsystem size in terms of the turning point for the standard $\mathcal{N}=4$ supersymmetric Yang-Mills (SYM) theory by setting $a=0$ in the eq.(\ref{Len}). This reads
\cite{Ryu:2006bv,Ryu:2006ef,Nishioka:2009un}
\begin{eqnarray}\label{lsym}
	\left(\frac{l}{a}\right)_{\mathrm{SYM}}=\frac{2}{\sqrt{\pi}(au_{t})}\sum_{n=0}^{\infty}\frac{\Gamma\left(n+\frac{1}{2}\right)}{\Gamma\left(n+1\right)}\frac{1}{(6n+4)}\left[1-\left(\frac{au_{t}}{au_{b}}\right)^{6n+4}\right]~.
\end{eqnarray}
Now we will proceed to compute the subsystem size interms of the turning point for the domain $au_{t}\sim au_{b}$. In this domain, we can approximate $f(u)$ as, $f(u)\sim(au)^{-2}$ and by using this approximation we reform the differential equation given in eq.(\ref{len}) as
\begin{eqnarray}
	dx=\frac{a}{u_{t}}\frac{\left(\frac{u_{t}}{u}\right)^{4}du}{\sqrt{1-\left(\frac{u_{t}}{u}\right)^{6}}}~.
\end{eqnarray}
Solving the above differential equation, one can obtain the following result \cite{Karczmarek:2013xxa}
\begin{eqnarray}\label{3rd}
	u=\frac{u_{t}}{\left[\cos(\frac{3x}{a})\right]^{\frac{1}{3}}}~.
\end{eqnarray} 
By using the fact $x=\frac{l}{2}$ along with the boundary condition given in eq.(\ref{bc}), the above expression leads us to the following relation \cite{Karczmarek:2013xxa}
\begin{eqnarray}\label{tp1}
	u_{t}=u_{b}\left(\cos(\frac{3l}{2a})\right)^{\frac{1}{3}}~.
\end{eqnarray}
% Figure environment removed

\noindent In Fig.(\ref{1}), we have shown the variation of the subsystem size with respect to the turning point. We have plotted both the numerical and analytically obtained results. We have chosen two different values of for the cut-off. Fig.(\ref{1}) shows that for each subsystem size there exits an unique value of the turning point and the extremal surface exists for any subsystem length. \\
Next, we compute the HEE for this strip-like subsystem and we do this for the three different domains of the theory. For $au_{t}\le1$ domain, the expression given in eq.(\ref{see1}) can be reformulated in the following way (here $\bar{S}_{\mathrm{HEE}}=\frac{g_{s}^{2}G_{N}^{(10)}}{R^{8}L^{2}\pi^{3}}S_{\mathrm{HEE}}$)
\begin{eqnarray}
	a^{2}\bar{S}_{\mathrm{HEE}}&=&\frac{(au_{t})^{2}}{2}\left[(au_{t})\int_{\frac{au_{t}}{au_{b}}}^{au_{t}}dp\frac{\left(1+\left(\frac{p}{au_{t}}\right)^{2}\right)^{\frac{1}{2}}}{p^{4}\sqrt{1-p^{6}}}+\int_{au_{t}}^{1}dp\frac{\left(1+\left(\frac{au_{t}}{p}\right)^{2}\right)^{\frac{1}{2}}}{p^{3}\sqrt{1-p^{6}}}\right]~.
\end{eqnarray}
Once again we make use of the approximations which have been used to obtain eq.\eqref{Len}. These approximations lead us to the following result of HEE for the domain $au_{t}\le1$
\begin{eqnarray}\label{SEE1}
	a^{2}\bar{S}_{\mathrm{HEE}}&\approx&a^{2}\bar{S}_{\mathrm{div}}-\frac{5}{48}+\frac{(au_{t})^{2}}{2}\sum_{n=0}^{\infty}\frac{1}{\sqrt{\pi}}\frac{\Gamma\left(n+\frac{1}{2}\right)}{\Gamma\left(n+1\right)}\frac{1}{(6n-2)}+\frac{(au_{t})^{4}}{4}\sum_{n=0}^{\infty}\frac{1}{\sqrt{\pi}}\frac{\Gamma\left(n+\frac{1}{2}\right)}{\Gamma\left(n+1\right)}\frac{1}{(6n-4)}\nonumber\\
	&+&\frac{1}{2}\sum_{n=0}^{\infty}\frac{1}{\sqrt{\pi}}\frac{\Gamma\left(n+\frac{1}{2}\right)}{\Gamma\left(n+1\right)}\left[\frac{\left(1-(1/au_{b})^{6n}\right)}{(6n-3)}+\frac{\left(1-(1/au_{b})^{6n}\right)}{2(6n-1)}-\frac{1}{2(6n-4)}-\frac{1}{(6n-2)}\right](au_{t})^{6n}~.\nonumber\\
\end{eqnarray}
The above expression of HEE contains a divergent part which is independent of the turning point and hence independent of the subsystem size. In the above given result, the subsystem size independent divergent part of the HEE has been denoted as
\begin{eqnarray}
a^{2}\bar{S}_{\mathrm{div}}=\frac{1}{6}(au_{b})^{3}+\frac{1}{4}(au_{b})~.
\end{eqnarray}
Keeping this result in mind, we would like to mention that in the limit $a\rightarrow \frac{1}{u_{b}}$ one can obtain the result for HEE corresponding to the standard SYM theory. This reads \cite{Ryu:2006bv,Ryu:2006ef,Nishioka:2009un}
\begin{eqnarray}\label{ssym}
	a^{2}\bar{S}_{\mathrm{HEE}}|_{\mathrm{SYM}}^{\mathrm{\mathrm{finite}}}=\frac{(au_{t})^{2}}{2}\sum_{n=0}^{\infty}\frac{1}{\sqrt{\pi}}\frac{\Gamma\left(n+\frac{1}{2}\right)}{\Gamma\left(n+1\right)}\frac{1}{(6n-2)}=-\frac{\sqrt{\pi}}{4}\frac{\Gamma(2/3)}{\Gamma(1/6)}(au_{t})^{2}~.
\end{eqnarray}
However, for our future purpose we will use the following results of holographic entanglement entropy in the domain $au_{t}\le 1$
\begin{eqnarray}
	a^{2}\bar{S}_{HEE}&\approx&a^{2}\bar{S}_{div}-\frac{5}{48}+\frac{14}{1000}(au_{t})^6-\frac{\sqrt{\pi}}{16}\frac{\Gamma(1/3)}{\Gamma(-1/6)}(au_{t})^{4}-\frac{\sqrt{\pi}}{4}\frac{\Gamma(2/3)}{\Gamma(1/6)}(au_{t})^{2}~.
\end{eqnarray}
We can express the above result in terms of the subsystem size by using \eqref{tp} in the above result. This reads
\begin{eqnarray}\label{hee1}
		a^{2}\bar{S}_{HEE}&\approx&a^{2}\bar{S}_{div}-\frac{5}{48}+\frac{14}{1000}\left(\frac{\lambda_{1}}{\left(\frac{l}{a}\right)}+\frac{\lambda_{2}}{\left(\frac{l}{a}\right)^{3}}+\frac{\lambda_{3}}{\left(\frac{l}{a}\right)^{4}}\right)^{6}-\frac{\sqrt{\pi}}{16}\frac{\Gamma(1/3)}{\Gamma(-1/6)}\left(\frac{\lambda_{1}}{\left(\frac{l}{a}\right)}+\frac{\lambda_{2}}{\left(\frac{l}{a}\right)^{3}}+\frac{\lambda_{3}}{\left(\frac{l}{a}\right)^{4}}\right)^{4}\nonumber\\
	&-&\frac{\sqrt{\pi}}{4}\frac{\Gamma(2/3)}{\Gamma(1/6)}\left(\frac{\lambda_{1}}{\left(\frac{l}{a}\right)}+\frac{\lambda_{2}}{\left(\frac{l}{a}\right)^{3}}+\frac{\lambda_{3}}{\left(\frac{l}{a}\right)^{4}}\right)^{2}~.
\end{eqnarray}
On the other hand for the domain $1\le au_{t}< au_{b}$, the HEE is obtained to be found 
\begin{eqnarray}\label{SEE2}
	a^{2}\bar{S}_{\mathrm{HEE}}&\approx&\frac{1}{6}\left(au_{b}^{3}-au_{t}^3\right)+\frac{1}{4}(au_{b}-au_{t})+\frac{(au_{t})^{3}}{2}\sum_{n=1}^{\infty}{\sqrt{\pi}}\frac{\Gamma\left(n+\frac{1}{2}\right)}{\Gamma\left(n+1\right)}\frac{1-\left(\frac{au_{t}}{au_{b}}\right)^{6n-3}}{(6n-3)}\nonumber\\
	&+&\frac{(au_{t})}{4}\sum_{n=1}^{\infty}\frac{1}{\sqrt{\pi}}\frac{\Gamma\left(n+\frac{1}{2}\right)}{\Gamma\left(n+1\right)}\frac{1-\left(\frac{au_{t}}{au_{b}}\right)^{6n-1}}{(6n-1)}
	+\frac{1}{2}\sum_{m=2}^{\infty}\frac{\Gamma(3/2)}{\Gamma\left(m+1\right)\Gamma\left(\frac{3}{2}-m\right)(au_{t})^{2m-3}}\frac{1-(au_{t}/au_{b})^{2m-3}}{(2m-3)}\nonumber\\
	&+&\frac{1}{2}\sum_{n=1}^{\infty}\sum_{m=2}^{\infty}\frac{1}{\sqrt{\pi}}\frac{\Gamma(3/2)}{\Gamma\left(m+1\right)\Gamma\left(\frac{3}{2}-m\right)}\frac{\Gamma\left(n+\frac{1}{2}\right)}{\Gamma\left(n+1\right)(au_{t})^{2m-3}}\frac{1}{(6n+2m-3)}\left[1-\left(\frac{au_{t}}{au_{b}}\right)^{6n+2m-3}\right]~.
\end{eqnarray}
% Figure environment removed\\
Furthermore, we would like to mention some recent work related to computation of holographic entanglement entropy can be found \cite{Saha:2019ado,Saha:2018jjb,Saha:2020fon,Park:2022fqy,Sun:2021lob,Maulik:2020tzm,Ali-Akbari:2021zsm,Bhattacharya:2022msw}.\\
In Fig(\ref{2}), we have shown the variation HEE with respect to the subsystem size. We have plotted both the numerical and analytically computed results. %\textbf{Some related works can be found \cite{Saha:2019ado,Saha:2018jjb,Saha:2020fon,Park:2022fqy,Sun:2021lob,Maulik:2020tzm,Ali-Akbari:2021zsm,Bhattacharya:2022msw}??.}\\
Finally, we now compute the HEE for the domain $au_{t}\sim au_{b}$. As we have mentioned earlier that in this domain, one can approximate $f(u)$ as $f(u)\sim (au)^{-2}$ and by using this approximation along with eq.(s)(\ref{3rd}),(\ref{see}), the HEE is found to be \cite{Karczmarek:2013xxa}
\begin{eqnarray}\label{s3rd}
	S_{\mathrm{HEE}}=\frac{2R^{8}\pi^{3}L^{2}}{4G_{N}^{(10)}g_{s}^{2}}u_{b}^{3}\left[\frac{a}{3}\sin\left(\frac{3l}{2a}\right)\right]\approx\frac{R^{8}\pi^{3}L^{2}}{4G_{N}^{(10)}g_{s}^{2}}\frac{l}{\epsilon^{3}}~.
\end{eqnarray}
We would like to make few comments now. As we have mentioned earlier, due to the presence of non-locality in a particular direction (in our case it is the $x$-direction) in the bulk metric, we have two different length scale of the theory. There exists a minimum length scale $\frac{l_{c}}{a}$ below which HEE follows the `volume-like ' law and above this minimum length scale HEE obey the usual area law. This minimum length scale can be found by equating the leading order term of eq.(\ref{s3rd}) to the divergent piece of the HEE. This in turn leads us to the following
\begin{eqnarray}
	\frac{l_{c}}{a}=\frac{2}{3}~.
\end{eqnarray}
The above obtained result shows that the minimum length scale (in the dimensionless form) above which HEE follows the area law and this length scale is independent of the UV cut-off. Furthermore, this result also indicates the fact that the dipole deformed SYM theory does not exhibit the UV/IR mixing property. In our earlier work \cite{Chowdhury:2021idy} we have shown that for the noncommutative deformation of the SYM theory, this minimum length scale (in the dimension less form) depends on the UV cut-off and NCYM theory shows the UV/IR mixing phenomena.
\section{Holographic mutual information and minimal cross section of the entanglement wedge}\label{sec3}  
In this section we have studied the holographic mutual information (HMI) and entanglement wedge cross section (EWCS) for two disjoint strip-like subsystems $A$ and $B$ of equal length $l$. Furthermore, they are separated by a length scale $d$. In this set up  the holographic mutual information can be defined as
\begin{eqnarray}\label{hmi1}
	I(A:B)=S_{\mathrm{HEE}}(A)+S_{\mathrm{HEE}}(B)-S_{\mathrm{HEE}}(A\cup B)~.
\end{eqnarray}   
Now by considering the separation between the subsystems is smaller than the length of the subsystems (that is, $\frac{d}{l}<1$), we can recast eq.(\ref{hmi1}) in the following form 
\begin{eqnarray}
	I(A:B)=2S_{\mathrm{HEE}}(l)-S_{\mathrm{HEE}}(d)-S_{\mathrm{HEE}}(2l+d)~.
\end{eqnarray}  
Here, we have used the fact that $S_{\mathrm{HEE}}(A\cup B)=S_{\mathrm{HEE}}(2l+d)+S_{\mathrm{HEE}}(d)$ for $\frac{d}{l}<1$. Another thing to mention is that as we have expressed all the results in dimensionless form so to compute the HMI we need to consider the following form \cite{Chowdhury:2021idy}
\begin{eqnarray}\label{hmi2}
	a^{2}\bar{I}(A:B)=2a^{2}\bar{S}_{\mathrm{HEE}}\left(\frac{l}{a}\right)-a^{2}\bar{S}_{\mathrm{HEE}}\left(\frac{d}{a}\right)-a^{2}S_{\mathrm{HEE}}\left(\frac{2l+d}{a}\right)
\end{eqnarray}
where $\bar{I}=\frac{g_{s}^{2}G_{N}^{(10)}}{R^{8}L^{2}\pi^{3}}I$. Similar to the HEE, we shall also compute the HMI for three different domains of the theory. To compute HMI for the domain $au_{t}\le 1$, we make use of the expressions given in eq.(\ref{hee1})\footnote{ The detailed expressions of the individual terms appearing in \eqref{hmi2} are given in the Appendix A.} . On the other hand it can be shown that in the domain $1\le au_{t}< au_{b}$ HMI is negative. Therefore in this domain the HMI is not a physical quantity anymore. Further, in the domain $au_{t}\sim au_{b}$, the HMI is a divergent quantity because in this domain the divergent part of HEE depends on the subsystem size explicitly. In the domain $au_{t}\sim au_{b}$, the HMI reads
\begin{eqnarray}\label{hmi3}
	I(A:B)=\frac{2R^{8}\pi^{3}L^{2}}{4G_{N}^{(10)}g_{s}^{2}}u_{b}^{3}\frac{a}{3}\left[2\sin\left(\frac{3l}{2a}\right)-\sin\left(\frac{3d}{2a}\right)-\sin\left(\frac{3(2l+d)}{2a}\right)\right]~.
\end{eqnarray}  
The above expression shows that the HMI contains a UV cut off so it is a divergent quantity which has been regularized with the help of the UV cut-off. One can obtain the result of HMI for the standard SYM theory by using eqs.(\ref{lsym},\ref{ssym}) in eq.(\ref{hmi2}). This results in
\begin{eqnarray}
	a^2\bar{I}(A:B)|_{\mathrm{SYM}} &=& -{\pi}^{3/2}\left(\frac{\Gamma(2/3)}{\Gamma(1/6)}\right)^3\left[\frac{2}{\left(\frac{l}{a}\right)^2}-\frac{1}{\left(\frac{d}{a}\right)^2}-\frac{1}{\left(\frac{2l+d}{a}\right)^2}\right]~.\label{HMiIR}
\end{eqnarray}
On the other hand, the holographic counterpart of entanglement of purification (EoP), known as the minimal cross section of the entanglement wedge (EWCS), can be computed by using the $E_{P}=E_{W}$ duality \cite{Takayanagi:2017knl,Jokela:2019ebz,BabaeiVelni:2019pkw,Nguyen:2017yqw}. Some recent work in this direction can be found \cite{Sahraei:2021wqn,Tamaoka:2018ned,Jeong:2019xdr,Kusuki:2019evw,Kusuki:2019hcg,Boruch:2020wbe}.\\
Similar to the set up realized for the computation of HMI, here we also consider two strip-like subsystems on the boundary $\partial M$ ($\partial M$ is the boundary of the canonical time-slice $M$ that has been considered in the gravity dual). We denote these subsystems as $A$ and $B$ with both of them having the same length $l$. Further we consider that $A$ and $B$ are separated by a distance $d$ with the condition $A\cap B =0$. The Ryu-Takayanagi surfaces corresponding to $A$, $B$ and $AB$ can be denoted as $\Gamma_A^{\mathrm{min}}$, $\Gamma_B^{\mathrm{min}}$ and $\Gamma_{AB}^{\mathrm{min}}$ respectively. The co-dimension-0 domain of entanglement wedge $M_{AB}$ is characterized by the following boundary 
\begin{eqnarray}\label{16}
	\partial M_{AB} = A \cup B \cup \Gamma_{AB}^{\mathrm{min}} =\bar{\Gamma}_A \cup \bar{\Gamma}_B
\end{eqnarray}
where $\bar{\Gamma}_A = A \cup \Gamma_{AB}^{A}$, $\bar{\Gamma}_B = B \cup \Gamma_{AB}^{B}$. In the above equation, we have used the condition $\Gamma_{AB}^{\mathrm{min}} = \Gamma_{AB}^{A} \cup \Gamma_{AB}^{B}$. In this set up, one can define the holographic entanglement entropies $S(\rho_{A \cup \Gamma_{AB}^{A}})$ and $S(\rho_{B \cup \Gamma_{AB}^{B}})$ and compute them by finding a static RT surface $\Sigma^{min}_{AB}$ with the following condition
\begin{eqnarray}
	\partial \Sigma^{\mathrm{min}}_{AB} = \partial \bar{\Gamma}_A =  \partial \bar{\Gamma}_B~.
\end{eqnarray}
The splitting condition $\Gamma_{AB}^{\mathrm{min}} = \Gamma_{AB}^{A} \cup \Gamma_{AB}^{B}$ which has been incorporated in not unique and there can be infinite number of possible choices. Further, this means that there can be infinite number of choices for the surface $\Sigma^{\mathrm{min}}_{AB}$. The EWCS is computed by minimizing the area of $\Sigma^{\mathrm{min}}_{AB}$ over all possible choices for $\Sigma^{\mathrm{\mathrm{min}}}_{AB}$. This reads \cite{Takayanagi:2017knl,Jokela:2019ebz,BabaeiVelni:2019pkw,Nguyen:2017yqw}
\begin{eqnarray}\label{18}
	E_W(\rho_{AB}) = \mathop{\mathrm{min}}_{\bar{\Gamma}_A \subset \partial M_{AB}}\left[\frac{\mathrm{Area}\left(\Sigma^{\mathrm{min}}_{AB}\right)}{4G_N}\right]~.
\end{eqnarray}
This means that to compute EWCS we have to calculate the vertical constant $x$ hypersurface with minimal area which splits $M_{AB}$ into two domains corresponding to $A$ and $B$. The time induced metric on this constant $x$ hypersurface reads
\begin{eqnarray}
	ds^{2}=R^2\left[u^2\left(dy^{2}+dz^2\right)+\frac{du^{2}}{u^2}\right]+\mathrm{metric~ on ~the ~deformed}~S^5`
\end{eqnarray}
Using the above induced metric along with the formula given in eq.(\ref{18}), the EWCS reads
\begin{eqnarray}\label{ew}
	E_{W}&=&\frac{L^{2}R^{8}\pi^{3}}{4G_{N}^{(10)}g_{s}^{2}}\int_{u_{t}(2l+d)}^{u_{t}(d)} u\sqrt{1+(au)^{2}}du\nonumber\\
	&=&\frac{L^{2}R^{8}\pi^{3}}{4G_{N}^{(10)}g_{s}^{2}}\frac{1}{3a^{2}}\left[\Bigg(1+(au_{t}(d))^{2}\Bigg)^{\frac{3}{2}}-\Bigg(1+(au_{t}(2l+d))^{2}\Bigg)^{\frac{3}{2}}\right]
\end{eqnarray}
where $au_{t}(d)$ and $au_{t}(2l+d)$ represents the turning points associated with the RT surfaces $\Gamma_{d}^{\mathrm{min}}$ and $\Gamma_{2l+d}^{\mathrm{min}}$ respectively. 
Keeping the result of EWCS (given in eq.(\ref{ew})) in mind, we now compute EWCS for different domains of the theory. In the domain $au_{t}\le 1$, we can recast the eq.(\ref{ew}) as $(\bar{E}_{W}=\frac{g_{s}^{2}G_{N}^{(10)}}{R^{8}L^{2}\pi^{3}}E_{W})$
\begin{eqnarray}\label{EWCS}
	a^{2}\bar{E}_{W}=\frac{1}{8}\left[(au_{t}(d))^{2}-(au_{t}(2l+d))^{2}\right]+\frac{1}{32}\left[(au_{t}(d))^{4}-(au_{t}(2l+d))^{4}\right]~.
\end{eqnarray}
By using the relation given in eq.\eqref{tp}, we express the above obtained result in terms of the subsystem lengths. This reads
\begin{eqnarray}
	a^{2}\bar{E}_{W}&=&\frac{1}{8}\left[\left(\frac{\lambda_{1}}{\left(\frac{d}{a}\right)}+\frac{\lambda_{2}}{\left(\frac{d}{a}\right)^{3}}+\frac{\lambda_{3}}{\left(\frac{d}{a}\right)^{4}}\right)^{2}-\left(\frac{\lambda_{1}}{\left(\frac{2l+d}{a}\right)}+\frac{\lambda_{2}}{\left(\frac{2l+d}{a}\right)^{3}}+\frac{\lambda_{3}}{\left(\frac{2l+d}{a}\right)^{4}}\right)^{2}\right]\nonumber\\
	&+&\frac{1}{32}\left[\left(\frac{\lambda_{1}}{\left(\frac{d}{a}\right)}+\frac{\lambda_{2}}{\left(\frac{d}{a}\right)^{3}}+\frac{\lambda_{3}}{\left(\frac{d}{a}\right)^{4}}\right)^{4}-\left(\frac{\lambda_{1}}{\left(\frac{2l+d}{a}\right)}+\frac{\lambda_{2}}{\left(\frac{2l+d}{a}\right)^{3}}+\frac{\lambda_{3}}{\left(\frac{2l+d}{a}\right)^{4}}\right)^{4}\right]~.	
\end{eqnarray}
In \cite{Takayanagi:2017knl} it was shown that the HMI (given in eq.(\ref{hmi2})) vanishes for a particular value of the separation length, namely the critical separation length $d_{c}$ associated to the fixed values of the subsystem sizes. At this critical value of the separation length, EWCS shows a discontinuity which represents a phase transition from the connected phase to the disconnected phase of the entanglement wedge. Upto $d=d_{c}$, connected phase is the physical whereas beyond $d > d_{c}$ disconnected phase is physical. Some recent works in this direction can be found in \cite{Saha:2021kwq,Chowdhury:2021idy,Sahraei:2021wqn,Liu:2021rks,Basu:2021awn,ChowdhuryRoy:2022dgo,Yang:2023wuw,Asadi:2022mvo,Karar:2020cvz,BabaeiVelni:2023cge,Saha:2021ohr,RoyChowdhury:2022awr,RoyChowdhury:2023eol}. We have depicted this fact in Fig.(\ref{ewcs}). We would also like to mention that EWCS and HMI both obey the following inequality
\begin{eqnarray}\label{eh}
	E_{W}\ge \frac{1}{2}I(A:B)~.
\end{eqnarray}
We have also verified the above inequality in Fig.(\ref{ewcs}). We can get the result of entanglement wedge cross section for supersymmetric Yang-Mills theory by setting $a=0$ in the eq.(\ref{EWCS}). This reads
\begin{eqnarray}\label{ewsym}
	a^{2}\bar{E}_{W}|_{\mathrm{SYM}}=\frac{1}{8}\left[(au_{t}(d))^{2}-(au_{t}(2l+d))^{2}\right]~.
\end{eqnarray}
We can recast the above result for SYM theory in terms of the subsystem size by using eq.(\ref{lsym}) in eq.(\ref{ewsym}).. This reads
\begin{eqnarray}
	a^2\bar{E}_W|_{\mathrm{SYM}} &=& \frac{1}{8} \left(2\sqrt{\pi}\frac{\Gamma(2/3)}{\Gamma(1/6)}\right)^2\left[\frac{1}{\left(\frac{d}{a}\right)^2}-\frac{1}{\left(\frac{2l+d}{a}\right)^2}\right]~.\label{EwIR}
\end{eqnarray}
% Figure environment removed\\
On the other hand in the domain $1\le au_{t}<au_{b},$ EWCS (in dimensionless form) has the following form
\begin{eqnarray}\label{ew2}
	a^{2}\bar{E}_{W}=\frac{1}{12}\left[(au_{t}(d))^{3}-(au_{t}(2l+d))^{3}\right]~.
\end{eqnarray}\\
Combining the above result with the result given in eq.(\ref{Len2}) one can show that in the domain $1\le au_{t}<au_{b}$ EWCS is  negative. Therefore in this domain EWCS is not physical. The same is also true for the HMI in this domain. Therefore one can argue that in this domain there is no connected phase available for the EWCS, as it is always in the disconnected phase.\\
Finally, we will compute the entanglement wedge cross section for the domain $au_{t}\sim au_{b}$. EWCS in this domain can be obtained by using the results given in eqs.(\ref{ew2},\ref{tp1}). This leads to the following expression
\begin{eqnarray}
	a^{2}\bar{E}_{W}=\frac{(au_{b})^{3}}{12}\left[\cos\left(\frac{3d}{2a}\right)-\cos\left(\frac{3(2l+d)}{2a}\right)\right]~.
\end{eqnarray}
The above result shows that in the domain $au_{t}\sim au_{b}$ EWCS is not a finite quantity as it diverges (which has been regularized with the help of the UV cut-off). In eq.(\ref{hmi3}) we have already show that, the HMI is also a divergent quantity in this domain. Therefore, in this domain both the HMI and EWCS have no physical relevance.\\
Now we will compare the results of HMI and EWCS for dipole defomed supersymmetric Yang-MIlls (DSYM) theory with that of the standard supersymmetric Yang-Mills (SYM) theory.
% Figure environment removed\\
In Fig(\ref{ewcs2}), we have compared the results of the EWCS and HMI for the above mentioned two theories. The blue curves (dotted curve represents HMI and the solid curve gives EWCS) represents the results of the EWCS and HMI for dipole deformed supersymmetric Yang-Mills theory in the domain $au_{t}\le 1$ and the red curves represent the results corresponding to the SYM theory. This figure clearly shows that for DSYM both the EWCS and HMI vanishes earlier than that of the SYM theory.
\section{Holograhic computation of the entanglement negativity}\label{ENsec}
Now we proceed to compute another measure of quantum correlation known as the entanglement negativity (also known as the logarithmic negativity) $(E_{N})$ which quantify entanglement for mixed states. We have already briefly discussed the concept of entanglement negativity and its significance in context of quantum information theory. However, to proceed further we now qualitatively discuss how one can compute entanglement negativity holographically.\\
We have two different proposals to holographically compute entanglement negativity.
One of the proposal states that, $E_{N}$ is related to the area of an extremal cosmic brane that terminates at the boundary of the entanglement wedge \cite{Kudler-Flam:2018qjo,Kusuki:2019zsp}. This proposal is motivated by the quantum error correcting codes and states that the logarithmic negativity is equivalent to the cross-sectional area of the entanglement wedge with a bulk correction term. However, for a general entangling surface this is difficult to compute due to the backreaction of the cosmic brane. This calculation simplifies a lot for a ball shaped subregion. In this set up, the backreaction is accounted for by an overall constant to the area of the entanglement wedge cross-section. Then it is conjectured that \cite{Kudler-Flam:2018qjo,Kusuki:2019zsp,Blanco:2013joa}
\begin{eqnarray}
	E_{N}&=&\chi_{d}\frac{E_{W}}{4G_{N}}+E_{\mathrm{bulk}}
\end{eqnarray}
where $E_{W}$ is the minimal cross-sectional area of the entanglement wedge associated with the concerned boundary region and $\chi_{d}$ is a constant which depends on the dimension of the spacetime. $E_{\mathrm{bulk}}$ is the quantum correction term corresponding to the logarithmic negativity between the bulk fields on either sides of the entanglement wedge cross-section.\\
Another proposal suggests that the entanglement negativity is given by certain combinations of co-dimension-two static minimal bulk surfaces \cite{Chaturvedi:2016rft,Chaturvedi:2016rcn,Jain:2017xsu,Jain:2017uhe,Malvimat:2018izs,Malvimat:2018cfe}. Both of these proposals reproduce the exact known result of entanglement negativity in CFT. In this paper, we follow the second proposal where entanglement negativity is given by certain combination of the static minimal surfaces in the bulk. Some recent works in this directions can be found in \cite{Rogerson:2022yim,Matsumura:2022ide,Bertini:2022fnr,Roik:2022gbb,Dong:2021oad,Bhattacharya:2021dnd,Hejazi:2021yhz,Afrasiar:2021hld,Basu:2022nds}.\\
To compute the entanglement negativity we will consider two different scenarios. First let us consider two strip-like adjacent subsystems $A$ and $B$ with lengths $l_{1}$ and $l_{2}$ with zero-overlapping. In the case of such adjacent subsystems the entanglement negativity ($E_N$) is defined as \cite{Chaturvedi:2016rft,Chaturvedi:2016rcn,Jain:2017xsu,Jain:2017uhe,Malvimat:2018izs,Malvimat:2018cfe}
\begin{equation}\label{aen}
	a^{2}\bar{E}_{N_{adj}}=\frac{3}{4}\left[a^{2}\bar{S}_{EE}\left(\frac{l_1}{a}\right)+a^{2}\bar{S}_{EE}\left(\frac{l_1}{a}\right)-a^{2}\bar{S}_{EE}\left(\frac{l_1+l_2}{a}\right)\right]
\end{equation}
where $\bar{E}_{N_{\mathrm{adj}}}=\frac{g_{s}^{2}G_{N}^{(10)}}{R^{8}L^{2}\pi^{3}}E_{N_{adj}}$ and $\bar{S}(l_{i})$ is the HEE of a subsystem of length $l_{i}$. To compute the entanglement negativity for adjacent subsystems in the domain $au_{t}\le 1$, we use eq.(\ref{hee1}). On the other hand in the domain $1\le au_{t}< au_{b}$ to compute the entanglement negativity we have to use eqs.(\ref{SEE2},\ref{Len2}). The above expression of the entanglement negativity suggests that for adjacent set up, entanglement negativity is a divergent quantity in both the domains. On the other hand, entanglement negativity for two adjacent subsystems in the domain $au_{t}\sim au_{b}$ is found to be
\begin{eqnarray}
a^{2}\bar{E}_{N_{adj}}=\frac{3}{4}\frac{(au)^3}{3}\left[\sin\left(\frac{3l_{1}}{2a}\right)+\sin\left(\frac{3l_{2}}{2a}\right)-\sin\left(\frac{3(l_{1}+l_{2})}{2a}\right)\right]~.	
\end{eqnarray}
The above result shows that entanglement negativity is divergent in the domain $au_{t}\sim au_{b}$.\\
Now we proceed to compute entanglement negativity for two disjoint subsystems. To do this we consider two disjoint subsystems $A$ and $B$ with length $l_{1}$ and $l_{2}$ respectively along with the fact that they are separated by a distance $d$. In this setup, entanglement negativity reads
\cite{Malvimat:2018ood,KumarBasak:2020viv}
\begin{eqnarray}\label{end}
	a^{2}\bar{E}_{N_{\mathrm{dis}}}=\frac{3a^{2}}{4}\left[\bar{S}_{\mathrm{HEE}}(l_{1}+x)+\bar{S}_{\mathrm{HEE}}(l_{2}+x)-\bar{S}_{\mathrm{HEE}}(l_{1}+l_{2}+x)-\bar{S}_{\mathrm{HEE}}(x)\right]~~.
\end{eqnarray}
In this set up, if we now consider a special case where we take two disjoint subsystems of equal length $l_{1}=l_{2}\equiv l$, we get the following result
\begin{eqnarray}\label{end2}
	a^{2}\bar{E}_{N_{dis}}=\frac{3}{4}\left[2a^{2}\bar{S}_{HEE}\left(\frac{l+d}{a}\right)-a^{2}\bar{S}_{HEE}\left(\frac{2l+d}{a}\right)-a^{2}\bar{S}_{HEE}\left(\frac{d}{a}\right)\right]~~.	
\end{eqnarray}
In order to compute entanglement negativity in the domain $au_{t}\le 1$, we make use of the eq(s).(\ref{hee1}\ref{end2}) \footnote{The detailed expression of the individual terms are in given in Appendix B.}. Similarly, in the domain $1\le au_{t}< au_{b}$, we can obtain the result of entanglement negativity for two disjoint subsystems by using eq(s).(\ref{Len2},\ref{SEE2}) along with eq.(\ref{end2}). The expression given in eq.(\ref{end2}) suggests that, for two disjoint subsystems entanglement negativity is a divergent free quantity because the divergent pieces of the HEEs cancelled out. Again, we would like to mention that for two disjoint subsystems entanglement negativity is a physical only for the domain $au_{t}\le 1$, and for the domain $1\le au_{t}< au_{b}$ of the theory it is not physical. One can obtain the result of entanglement negativity for two disjoint subsystems in the domain $au_{t}\sim au_{b}$ by substituting eq.(\ref{s3rd}) in eq.(\ref{end2}). This reads
\begin{eqnarray}
	a^{2}\bar{E}_{N_{\mathrm{dis}}}|_{l_1=l_2}=\frac{3}{4}\frac{(au)^3}{3}\left[2\sin\left(\frac{3(l+d)}{2a}\right)+\sin\left(\frac{3(2l+d)}{2a}\right)-\sin\left(\frac{3d}{2a}\right)\right]~.
\end{eqnarray}
For the sake of completeness of our analysis, we also provide the result of entanglement negativity of two disjoint subsystems of equal length for the standard supersymmetric Yang-Mills (SYM) theory. This we do by using eqs.(\ref{lsym},\ref{ssym}) along eq.(\ref{end2}). This results 
\begin{eqnarray}\label{Ensym}
	a^{2}\bar{E}_{N_{dis}}|^{\mathrm{SYM}}=\frac{3\sqrt{\pi}}{8}\frac{\Gamma(2/3)}{\Gamma(1/6)}\left(\frac{\sqrt{\pi}\Gamma(5/3)}{2\Gamma(7/6)}\right)^2\left[\frac{1}{\left(\frac{d}{a}\right)^{2}}+\frac{1}{\left(\frac{2l+d}{a}\right)^{2}}-\frac{1}{\left(\frac{l+d}{a}\right)^{2}}\right]~.
\end{eqnarray}
Now we graphically compare the results of entanglemnet negativity of two different theories.
% Figure environment removed\\
In Fig(\ref{en}) we have shown the variation of entanglement negativity (for two disjoint sub systems of equal length) with respect to the separation between two sub systems. The red curve represents the entanglement negativity for the dipole deformed supersymmetric Yang-Mills (DSYM) theory in the domain $au_{t}\le 1$ \footnote{We are only interested in the domain $au_{t}\le 1$ because in the other two domains of the theory entanglement negativity is not physical.} and the blue curve depicts the entanglement negativity for the usual supersymmetric Yang-Mills theory. It can be observed from the Fig.(\ref{en}), that in both the case the entanglement negativity never vanishes for any value of the separation distance between the sub systems. Therefore, we can draw the conclusion that the entanglement negativity measures the quantum correlation between two disjoint sub systems even though they are not in the connected phase \footnote{This is because the HMI and EWCS at some value of the separation distance, but entanglement negativity never vanishes for any value of the separation distance.}~.
\section{Conclusion}\label{seca}
We now present a summary of our work. In this paper we have computed various entanglement measures, such as entanglement entropy, mutual information, entanglement of purification and entanglement negativity for the dipole deformed SYM theory in $3+1$-spacetime dimensions. We have computed subsystem length $\left(\frac{l}{a}\right)$ in terms of the turning point ($au_{t}$) by introducing a systematic way. We have compared both the numerical and analytical results graphically in Fig.(\ref{1}) for two different values of the cut-off. We have observed that there exists a critical length scale $\left(\frac{l_{c}}{a}\right)$ which indicates two different domains of the theory, one is for $l< l_{c}$ and the other one is $l>l_{c}$. The region $l<l_{c}$ can be further divided into two distinct regions, one corresponds to $au_{t}\le1$ and the other one corresponds to $1<au_{t}<au_{b}$. We have also calculated subsystem size in terms of the turning points for the mentioned three different domains of the theory. We have found another interesting fact that, this critical length scale $\left(\frac{l_{c}}{a}\right)$ is independent of the UV-cutoff, which clearly indicates the fact that dipole deformed SYM theory does not show UV/IR mixing property. We can also conclude (from Fig.(\ref{1})) that for every subsystem length there exists only one extremal surface.
Then we proceed to compute the holographic entanglement entropy for a subsystem of length $l$. We have also computed the holographic entanglement entropy for the three different domains of the theory. Once again we have compared the numerical and the analytical results graphically. We have found that in the domain $l>l_{c}$, the HEE contains an universal divergent term which is independent of the turning point, hence, independent of the subsystem size. On the other hand in the domain $l<l_{c}$, we have shown that the divergent part of the holographic entanglement entropy depends on the turning point. Therefore in this domain the divergent part of HEE is not universal, it depends on the subsystem size. We have also shown that the holographic entanglement entropy follows volume law in the domain $l<l_{c}$.  We then compute the holographic mutual information for two disjoint subsystems. This is a valid physical quantity only for the domain $au_{t}\le1$ and for other two domains this is not physical as one can show that in the domain $1\le au_{t}< au_{b}$, it is negative for all subsystem sizes and in the domain $au_{t}\sim au_{b}$, it is a divergent quantity. We then proceed to compute the entanglement of purification which is a proposed suitable measure for mixed states. This can be done by using the $E_{P}=E_{W}$ duality. We have computed entanglement wedge cross section in the different domains of the theory. We have shown that the entanglement wedge cross section is a valid physical quantity only for the domain $au_{t}\le1$. We have also verified that, $a^{2}E_{W}(A:B)\ge\frac{1}{2}a^{2}I(A:B)$ is also valid for the dipole deformed supersymmetric Yang-Mills theory. We have also graphically represented this fact. We have also compared the result of entanglement wedge cross section and holographic mutual information of dipole deformed supersymmetric Yang-Mills theory with that of standard supersymmetric Yang-Mills theory graphically. This comparison shows that for dipole derformed theory both entanglement wedge cross section and holographic mutual information vanishes earlier compared to standard SYM theory. Then we proceed to compute the entanglement negativity for both adjacent and disjoint subsystems in different domains of the theory. For adjacent subsystems, we observe that entanglement negativity is always a divergent quantity. On the other hand for disjoint subsystems we have shown that entanglement negativity is physical only in the domain $au_{t}\le 1$. We have also compared the results of entanglement negativity for dipole deformed supersymmetric Yang-Mills theory and standard supersymmetric Yang-Mills theory. This comparison shows that in both the cases entanglement negativity never vanishes for any values of the separation distance.\\  
Finally, we want to mention that, it is also very interesting to compute other entanglement measures for mixed state like, odd entropy, reflected entropy, etc holographically. It is interesting to investigate the holographic subregion complexity (for pure state) and complexity of purification (for mixed states) holographically. We leave these investigations for our future work.
\section{Acknowledgement}
ARC would like to thank SNBNCBS for the Senior Research Fellowship. The authors would like to thank the organizers of $12^{th}$ \textit{Field theoretic aspects of gravity} (FTAG XII), held at BIT Mesra, as the initial stages of this work was done there.\\
 \section{Appendix A: Expression of holographic mutual information in the domain $au_{t}\le 1$}\label{8}
In this Appendix, we will provide the expressions of the individual terms appearing in the result of holographic mutual information for the domain $au_{t}\le 1$. The expression given in eq.\eqref{hmi2} suggests that we need the following expressions of the holographic entanglement entropy 
\begin{eqnarray}
	a^{2}\bar{S}_{HEE}\left(\frac{l}{a}\right)&=&a^{2}\bar{S}_{div}-\frac{5}{48}+\frac{14}{1000}\left(\frac{\lambda_{1}}{\left(\frac{l}{a}\right)}+\frac{\lambda_{2}}{\left(\frac{l}{a}\right)^{3}}+\frac{\lambda_{3}}{\left(\frac{l}{a}\right)^{4}}\right)^{6}-\frac{\sqrt{\pi}}{16}\frac{\Gamma(1/3)}{\Gamma(-1/6)}\left(\frac{\lambda_{1}}{\left(\frac{l}{a}\right)}+\frac{\lambda_{2}}{\left(\frac{l}{a}\right)^{3}}+\frac{\lambda_{3}}{\left(\frac{l}{a}\right)^{4}}\right)^{4}\nonumber\\
	&-&\frac{\sqrt{\pi}}{4}\frac{\Gamma(2/3)}{\Gamma(1/6)}\left(\frac{\lambda_{1}}{\left(\frac{l}{a}\right)}+\frac{\lambda_{2}}{\left(\frac{l}{a}\right)^{3}}+\frac{\lambda_{3}}{\left(\frac{l}{a}\right)^{4}}\right)^{2}~.
	\end{eqnarray}
    \begin{eqnarray}
	a^{2}\bar{S}_{HEE}\left(\frac{d}{a}\right)&=&a^{2}\bar{S}_{div}-\frac{5}{48}+\frac{14}{1000}\left(\frac{\lambda_{1}}{\left(\frac{d}{a}\right)}+\frac{\lambda_{2}}{\left(\frac{d}{a}\right)^{3}}+\frac{\lambda_{3}}{\left(\frac{d}{a}\right)^{4}}\right)^{6}-\frac{\sqrt{\pi}}{16}\frac{\Gamma(1/3)}{\Gamma(-1/6)}\left(\frac{\lambda_{1}}{\left(\frac{d}{a}\right)}+\frac{\lambda_{2}}{\left(\frac{d}{a}\right)^{3}}+\frac{\lambda_{3}}{\left(\frac{d}{a}\right)^{4}}\right)^{4}\nonumber\\
	&-&\frac{\sqrt{\pi}}{4}\frac{\Gamma(2/3)}{\Gamma(1/6)}\left(\frac{\lambda_{1}}{\left(\frac{d}{a}\right)}+\frac{\lambda_{2}}{\left(\frac{d}{a}\right)^{3}}+\frac{\lambda_{3}}{\left(\frac{d}{a}\right)^{4}}\right)^{2}
	\end{eqnarray}
\begin{eqnarray}
	a^{2}\bar{S}_{HEE}\left(\frac{2l+d}{a}\right)&=&a^{2}\bar{S}_{div}-\frac{5}{48}+\frac{14}{1000}\left(\frac{\lambda_{1}}{\left(\frac{2l+d}{a}\right)}+\frac{\lambda_{2}}{\left(\frac{2l+d}{a}\right)^{3}}+\frac{\lambda_{3}}{\left(\frac{2l+d}{a}\right)^{4}}\right)^{6}-\frac{\sqrt{\pi}}{16}\frac{\Gamma(1/3)}{\Gamma(-1/6)}\Bigg(\frac{\lambda_{1}}{\left(\frac{2l+d}{a}\right)}+\frac{\lambda_{2}}{\left(\frac{2l+d}{a}\right)^{3}}\nonumber\\
	&+&\frac{\lambda_{3}}{\left(\frac{2l+d}{a}\right)^{4}}\Bigg)^{4}
	-\frac{\sqrt{\pi}}{4}\frac{\Gamma(2/3)}{\Gamma(1/6)}\left(\frac{\lambda_{1}}{\left(\frac{2l+d}{a}\right)}+\frac{\lambda_{2}}{\left(\frac{2l+d}{a}\right)^{3}}+\frac{\lambda_{3}}{\left(\frac{2l+d}{a}\right)^{4}}\right)^{2}~.
\end{eqnarray}
Using the above results in eq.\eqref{hmi2}, we can obtained the result of the holographic mutual information in the domain $au_{t}\le 1$.\\
\section{Appendix B: Expression of holographic entanglement negativity in the domain $au_{t}\le 1$}\label{9}
In this Appendix, we will provide the expression of individual terms appearing in the expression of entanglement negativity. The expression of entanglement negativity for adjoint subsystems given in \eqref{aen}, suggestes that we need the following expression of holographic entanglement entropy in the domain $au_{t}\le 1$
\begin{eqnarray}
	a^{2}\bar{S}_{HEE}\left(\frac{l_1}{a}\right)&=&a^{2}\bar{S}_{div}-\frac{5}{48}+\frac{14}{1000}\left(\frac{\lambda_{1}}{\left(\frac{l_1}{a}\right)}+\frac{\lambda_{2}}{\left(\frac{l_1}{a}\right)^{3}}+\frac{\lambda_{3}}{\left(\frac{l_1}{a}\right)^{4}}\right)^{6}\nonumber\\
	&-&\frac{\sqrt{\pi}}{16}\frac{\Gamma(1/3)}{\Gamma(-1/6)}\left(\frac{\lambda_{1}}{\left(\frac{l_1}{a}\right)}+\frac{\lambda_{2}}{\left(\frac{l_1}{a}\right)^{3}}+\frac{\lambda_{3}}{\left(\frac{l_1}{a}\right)^{4}}\right)^{4}
	-\frac{\sqrt{\pi}}{4}\frac{\Gamma(2/3)}{\Gamma(1/6)}\left(\frac{\lambda_{1}}{\left(\frac{l_1}{a}\right)}+\frac{\lambda_{2}}{\left(\frac{l_1}{a}\right)^{3}}+\frac{\lambda_{3}}{\left(\frac{l_1}{a}\right)^{4}}\right)^{2}~~~\\
	a^{2}\bar{S}_{HEE}\left(\frac{l_2}{a}\right)&=&a^{2}\bar{S}_{div}-\frac{5}{48}+\frac{14}{1000}\left(\frac{\lambda_{1}}{\left(\frac{l_2}{a}\right)}+\frac{\lambda_{2}}{\left(\frac{l_2}{a}\right)^{3}}+\frac{\lambda_{3}}{\left(\frac{l_2}{a}\right)^{4}}\right)^{6}\nonumber\\
	&-&\frac{\sqrt{\pi}}{16}\frac{\Gamma(1/3)}{\Gamma(-1/6)}\left(\frac{\lambda_{1}}{\left(\frac{l_2}{a}\right)}+\frac{\lambda_{2}}{\left(\frac{l_2}{a}\right)^{3}}+\frac{\lambda_{3}}{\left(\frac{l_2}{a}\right)^{4}}\right)^{4}
	-\frac{\sqrt{\pi}}{4}\frac{\Gamma(2/3)}{\Gamma(1/6)}\left(\frac{\lambda_{1}}{\left(\frac{l_2}{a}\right)}+\frac{\lambda_{2}}{\left(\frac{l_2}{a}\right)^{3}}+\frac{\lambda_{3}}{\left(\frac{l_2}{a}\right)^{4}}\right)^{2}\\
	a^{2}\bar{S}_{HEE}\left(\frac{l_1+l_2}{a}\right)&=&a^{2}\bar{S}_{div}-\frac{5}{48}+\frac{14}{1000}\left(\frac{\lambda_{1}}{\left(\frac{l_1+l_2}{a}\right)}+\frac{\lambda_{2}}{\left(\frac{l_1+l_2}{a}\right)^{3}}+\frac{\lambda_{3}}{\left(\frac{l_1+l_2}{a}\right)^{4}}\right)^{6}\nonumber\\
	&-&\frac{\sqrt{\pi}}{16}\frac{\Gamma(1/3)}{\Gamma(-1/6)}\Bigg(\frac{\lambda_{1}}{\left(\frac{l_1+l_2}{a}\right)}+\frac{\lambda_{2}}{\left(\frac{l_1+l_2}{a}\right)^{3}}+\frac{\lambda_{3}}{\left(\frac{l_1+l_2}{a}\right)^{4}}\Bigg)^{4}\nonumber\\
	&-&\frac{\sqrt{\pi}}{4}\frac{\Gamma(2/3)}{\Gamma(1/6)}\left(\frac{\lambda_{1}}{\left(\frac{l_1+l_2}{a}\right)}+\frac{\lambda_{2}}{\left(\frac{l_1+l_2}{a}\right)^{3}}+\frac{\lambda_{3}}{\left(\frac{l_1+l_2}{a}\right)^{4}}\right)^{2}~.
\end{eqnarray}
On the other hand, for two disjoint interval we can obtain entanglement negativity by using eqs.(\ref{end2},\ref{hee1}). The expression of entanglement negativity suggests that we need the following results of holographic entanglement entropy in the domain $au_{t}\le 1$
\begin{eqnarray}
	a^{2}\bar{S}_{HEE}\left(\frac{l+d}{a}\right)&=&a^{2}\bar{S}_{div}-\frac{5}{48}+\frac{14}{1000}\left(\frac{\lambda_{1}}{\left(\frac{l+d}{a}\right)}+\frac{\lambda_{2}}{\left(\frac{l+d}{a}\right)^{3}}+\frac{\lambda_{3}}{\left(\frac{l+d}{a}\right)^{4}}\right)^{6}\nonumber\\
	&-&\frac{\sqrt{\pi}}{16}\frac{\Gamma(1/3)}{\Gamma(-1/6)}\left(\frac{\lambda_{1}}{\left(\frac{l+d}{a}\right)}+\frac{\lambda_{2}}{\left(\frac{l+d}{a}\right)^{3}}+\frac{\lambda_{3}}{\left(\frac{l+d}{a}\right)^{4}}\right)^{4}
	-\frac{\sqrt{\pi}}{4}\frac{\Gamma(2/3)}{\Gamma(1/6)}\left(\frac{\lambda_{1}}{\left(\frac{l}{a}\right)}+\frac{\lambda_{2}}{\left(\frac{l}{a}\right)^{3}}+\frac{\lambda_{3}}{\left(\frac{l}{a}\right)^{4}}\right)^{2}~~~~~
\end{eqnarray}
\begin{eqnarray}
	a^{2}\bar{S}_{HEE}\left(\frac{d}{a}\right)&=&a^{2}\bar{S}_{div}-\frac{5}{48}+\frac{14}{1000}\left(\frac{\lambda_{1}}{\left(\frac{d}{a}\right)}+\frac{\lambda_{2}}{\left(\frac{d}{a}\right)^{3}}+\frac{\lambda_{3}}{\left(\frac{d}{a}\right)^{4}}\right)^{6}-\frac{\sqrt{\pi}}{16}\frac{\Gamma(1/3)}{\Gamma(-1/6)}\left(\frac{\lambda_{1}}{\left(\frac{d}{a}\right)}+\frac{\lambda_{2}}{\left(\frac{d}{a}\right)^{3}}+\frac{\lambda_{3}}{\left(\frac{d}{a}\right)^{4}}\right)^{4}\nonumber\\
	&-&\frac{\sqrt{\pi}}{4}\frac{\Gamma(2/3)}{\Gamma(1/6)}\left(\frac{\lambda_{1}}{\left(\frac{d}{a}\right)}+\frac{\lambda_{2}}{\left(\frac{d}{a}\right)^{3}}+\frac{\lambda_{3}}{\left(\frac{d}{a}\right)^{4}}\right)^{2}
\end{eqnarray}
\begin{eqnarray}
	a^{2}\bar{S}_{HEE}\left(\frac{2l+d}{a}\right)&=&a^{2}\bar{S}_{div}-\frac{5}{48}+\frac{14}{1000}\left(\frac{\lambda_{1}}{\left(\frac{2l+d}{a}\right)}+\frac{\lambda_{2}}{\left(\frac{2l+d}{a}\right)^{3}}+\frac{\lambda_{3}}{\left(\frac{2l+d}{a}\right)^{4}}\right)^{6}-\frac{\sqrt{\pi}}{16}\frac{\Gamma(1/3)}{\Gamma(-1/6)}\Bigg(\frac{\lambda_{1}}{\left(\frac{2l+d}{a}\right)}+\frac{\lambda_{2}}{\left(\frac{2l+d}{a}\right)^{3}}\nonumber\\
	&+&\frac{\lambda_{3}}{\left(\frac{2l+d}{a}\right)^{4}}\Bigg)^{4}
	-\frac{\sqrt{\pi}}{4}\frac{\Gamma(2/3)}{\Gamma(1/6)}\left(\frac{\lambda_{1}}{\left(\frac{2l+d}{a}\right)}+\frac{\lambda_{2}}{\left(\frac{2l+d}{a}\right)^{3}}+\frac{\lambda_{3}}{\left(\frac{2l+d}{a}\right)^{4}}\right)^{2}~.
\end{eqnarray}
Now using these above expressions of holographic entanglement entropy, we can compute the entanglement negativity by using eq.(\ref{end2}) in the domain $au_{t}\le 1$.
\bibliographystyle{hephys}  
\bibliography{Reference}

\end{document}