\documentclass{lmcs}

\keywords{%
  algebraic effects,
  asynchrony,
  concurrency,
  interrupt handling,
  signals,
  promises%
}

\usepackage[T1]{fontenc}
\usepackage[utf8]{inputenc}
\usepackage{amsmath}
\usepackage{amssymb}
\usepackage{amsthm}
\usepackage{bbold} % alternative blackboard letters (amssymb already provides a set of them)
\usepackage{graphicx}
\usepackage{hyperref}
\usepackage{lineno}
\usepackage{listings}
\usepackage{mathpartir} % inference rules
\usepackage{stmaryrd}
\usepackage{mathtools} % to typeset equations in the appendix
\usepackage{thmtools}
\usepackage{upquote}
\usepackage{xcolor}
\usepackage[all]{xy}
\usepackage{fontawesome5} % for the lock symbol
\usepackage[most]{tcolorbox}

\begin{document}

%%%%%%%%%%%%%%%%%%%%%%%%%%%%%%%%%%%%%%%%%%%%%%%%%%%%

\newcommand\calF{\mathcal{F}}
\newcommand\calG{\mathcal{G}}
\newcommand\calM{\mathcal{M}}
\newcommand\calV{\mathcal{V}}
\newcommand\calU{\mathcal{U}}
\newcommand\calW{\mathcal{W}}
\newcommand\calP{\mathcal{P}}
\newcommand\calD{\mathbb{D}}
%%%%%%%%%%%%%%%%%
%% macros introduced by Luke 
\newcommand\mydef[1]{{\bf\em #1}}
%%%%%%%%%%%%%%%%%

\newcommand{\numviparams}{{| \lambda |}}
\newcommand{\scoreaccvars}[1]{s_1^{#1}, \ldots, s_{\numviparams}^{#1}}
\newcommand{\scoreaccvar}[2]{s_{#1}^{#2}}
\newcommand{\isdeterm}[1]{\text{Deterministic}({#1})}


\newcommand{\expect}[1]{\mathbb{E}\left[{#1}\right]}
\newcommand{\var}[1]{\mathbb{V}\left[ {#1} \right]}
\newcommand{\expectdist}[2]{\mathbb{E}_{#1}\left[ {#2} \right]}
\newcommand{\vardist}[2]{\mathbb{V}_{#1}\left[ {#2} \right]}
\newcommand{\cov}[2]{\mathbb{C}\text{ov}[{#1}][{#2}]}
\newcommand{\covv}[1]{\mathbb{C}\text{ov}[{#1}]}
\newcommand{\corr}[1]{\mathbb{C}\text{orr}[{#1}]}

\newcommand{\fix}[1]{\mathit{fix}\left({#1}\right)}
\newcommand{\sbr}[1]{\left\llbracket {#1} \right\rrbracket}
\newcommand{\ctxtype}[3]{{#1} \cong_\text{ctx} {#2} : {#3}}
\newcommand{\bigstep}[3]{{#1} \Downarrow_{#2} {#3}}


% PCF types
\newcommand{\bool}{\mathit{bool}}
\newcommand{\nat}{\mathit{nat}}

\newcommand{\ctx}[1]{\mathcal{C}\left[ {#1}\right] }
\newcommand{\pcft}[1]{\text{PCF}_{#1}}

\newcommand{\nfl}{\mathbb{N}_\bot}
\newcommand{\bfl}{\mathbb{B}_\bot}

% PCF constructs
\newcommand{\succc}[1]{\mathbf{succ}({#1})}
\newcommand{\succcn}[2]{\mathbf{succ}^{#1}({#2})}
\newcommand{\zero}{\mathbf{0}}
\newcommand{\zerotest}[1]{\mathbf{zero}\left({#1}\right)}
\newcommand{\pred}[1]{\mathbf{pred}\left( {#1} \right)}
\newcommand{\predn}[2]{\mathbf{pred}^{#1}\left( {#2} \right)}
\def\solvable{\#}

\newcommand{\true}{\mathbf{true}}
\newcommand{\false}{\mathbf{false}}
\newcommand{\pcffix}[1]{\mathbf{fix}\left({#1}\right)}
\newcommand{\pcffn}[3]{\mathbf{fn}~{#1}:{#2}\mathpunct{.}{#3}}
\newcommand{\pairtype}[2]{{#1} * {#2}}
\newcommand{\pairexp}[2]{\mathbf{pair}({#1}, {#2})}
\newcommand{\leftexp}[1]{\mathbf{left}({#1})}
\newcommand{\rightexp}[1]{\mathbf{right}({#1})}

\newcommand{\RationalPos}{\mathbb{Q}^{+}}

\newcommand{\meas}[1]{\mathbb{M}\left( {#1} \right) }
\newcommand{\integ}[1]{\sbr{#1}_I}

\newcommand{\notbigstep}[2]{{#1}~\cancel{\Downarrow}_{#2}}
\newcommand{\subtrace}[3]{{#1}^{{#2} \ldots {#3}}}
\newcommand{\supp}[1]{\textsf{supp}\left({#1}\right)}
\newcommand{\dom}[1]{\textsf{Dom}\left({#1}\right)}
\newcommand{\suppk}[2]{\textsf{Supp}^{#1}\left({#2}\right)}
\newcommand{\tracespace}{\bigcup_{n \in \mathbb{N}}[0, 1]^n}
\newcommand{\generictracespace}{\mathbb{T}}
\newcommand{\nnreals}{\mathbb{R}_{\geq 0}}
\newcommand{\posreals}{\mathbb{R}_{> 0}}
\newcommand{\reals}{\mathbb{R}}

\newcommand{\unrollkM}[2]{\textsf{unroll}_{#1}\left({#2}\right)}
\newcommand{\nphmcint}[5]{\Psi_\textsf{NP}\left({#1}, {#2}, {#3}, {#4}, {#5}\right)}

%SPCF constructs
\newcommand{\spcfvalues}{\Lambda^0_v}

\newcommand{\prevalueM}[1]{\textsf{value}^{-1}_{#1}(\spcfvalues{})}
\newcommand{\num}[1]{\underline{#1}}

% \theoremstyle{definition}
% \newtheorem{thm}{Theorem}
% \newtheorem{lem}{Lemma}
% \newtheorem{defn}{Definition}
% \newtheorem{conj}{Conjecture}
% \newtheorem{prop}{Proposition}

%\theoremstyle{definition}
%\newtheorem{defn}{Definition}[section]
%\newtheorem{example}[defn]{Example}
%
%
%\theoremstyle{plain}
%\newtheorem{thm}{Theorem}[section]
%\newtheorem{lem}[thm]{Lemma}
%\newtheorem{cor}[thm]{Corollary}
%\newtheorem{conj}[thm]{Conjecture}
%\newtheorem{prop}[thm]{Proposition}
%\newtheorem{remark}[thm]{Remark}

%% Proofs
%\let\oldproof\proof
%\renewcommand{\proof}{\color{blue}\oldproof}


\definecolor{codegreen}{rgb}{0,0.6,0}
\definecolor{codegray}{rgb}{0.5,0.5,0.5}
\definecolor{codepurple}{rgb}{0.58,0,0.82}
\definecolor{backcolour}{rgb}{0.95,0.95,0.92}

\lstdefinestyle{myStyle}{
    belowcaptionskip=1\baselineskip,
    breaklines=true,
    frame=none,
    basicstyle=\footnotesize\ttfamily,
    keywordstyle=\bfseries\color{green!40!black},
    commentstyle=\itshape\color{purple!40!black},
    identifierstyle=\color{blue},
    backgroundcolor=\color{gray!10!white},
    %backgroundcolor=\color{backcolour}, 
    numberstyle=\tiny\color{codegray},
    stringstyle=\color{codepurple},
    breakatwhitespace=false,                          
    keepspaces=true,                 
    numbers=left,       
    numbersep=5pt,                  
    showspaces=false,                
    showstringspaces=false,
    showtabs=false,                  
    tabsize=2,
}

% argmin/argmax
\DeclareMathOperator*{\argmax}{arg\,max}
\DeclareMathOperator*{\argmin}{arg\,min}

% Concatenation of lists
\newcommand\doubleplus{+\kern-1.3ex+\kern0.8ex}

% Program configurations
\newcommand{\tuple}[1]{\ensuremath{\langle #1 \rangle}}
% Rule based definitions
\newcommand{\Rule}[4][]{\ensuremath{\inferrule*[lab={\hypertarget{#2}{(\TirName{#2})}},#1]{#3}{#4}}}

% Calligraphic symbols
\newcommand{\calI}{{\mathcal I}} 
\newcommand{\calT}{{\mathcal T}}

%  Macro for new Y operator.
\newcommand{\yBounded}[3]{\mu^{#1}_{#2}\rvert_{#3}}

%%%%%%%%%%%%%%%%%
 
%%%%%%%%%%%%%%%%%

\newcommand{\expv}{\mathbb{E}}

\newcommand{\combTr}[2]{\left[\begin{matrix}
		#1\\
		#2
	\end{matrix} \right]}

\newcommand{\exType}[2]{\left\{\begin{matrix}
		#1\\
		#2
	\end{matrix} \right\}}
\newcommand{\myint}[1]{ [#1]}
\newcommand{\Uniform}{\ensuremath{\mathrm{Uniform}}}
\newcommand{\Normal}{\ensuremath{\mathrm{normal}}}
\DeclareMathOperator{\abs}{abs}
\DeclareMathOperator{\pdf}{pdf}

\newcommand{\intConf}[1]{\lceil#1\rceil}
\newcommand{\tr}{\boldsymbol{t}}

\newcommand{\sample}{\tt{sample}}
%\newcommand{\fix}{\texttt{fix}}
%\newcommand{\num}[1]{\underline{#1}}
\newcommand{\myif}{\texttt{if}}
\newcommand{\mylet}{\texttt{let} \, }
\newcommand{\myin}{\, \texttt{in} \,}
\newcommand{\mythen}{\, \texttt{then} \,}
\newcommand{\myelse}{\, \texttt{else} \,}
\newcommand{\score}{\tt{score}}
\newcommand{\tick}{\tt{tick}}

\newcommand{\term}{\tt{term}}
\newcommand{\pv}{\mathbf{v}}
\newcommand{\rv}{\mathbf{r}}

\newcommand{\interval}{\mathfrak{I}}

\newcommand{\typeReal}{\textbf{\textsf{R}}}

\newcommand{\symbolInt}{\myint{\cdot}}

\newcommand{\LambdaInterval}{\Lambda_{\interval}}
\newcommand{\LambdaSymbolic}{\Lambda_{\text{sym}}}

\newcommand{\toIntervalTerm}[1]{#1^{2\interval}}

%Others
\newcommand{\Sset}{\mathbb{S}}
\newcommand{\Iset}{\mathbb{I}}
\newcommand{\Rset}{\mathbb{R}}
\newcommand{\Nset}{\mathbb{N}}
\newcommand{\Zset}{\mathbb{Z}}

\newcommand{\Term}{\mathbb{T}}
\newcommand{\prob}{\mathbb{P}}
\newcommand{\expt}{\mathbb{E}}


\newcommand{\Leb}{\tt{Leb}}
\newcommand{\Red}{\tt{Red}}
\newcommand{\cost}{\text{cost}}

%\newcommand{\intervalab}[2]{\underline{[#1,#2]}}
\newcommand{\intervalab}{\underline{[a,b]}}
\newcommand{\interI}{\mathcal{I}}
\newcommand{\trans}{\mathcal{T}}

\newcommand{\iv}{\mathbb{I}}

% Programming language constructs
\newcommand{\lit}[1]{\underline{#1}}
\newcommand{\letIn}[1]{\mathsf{let}\,{#1}\,\mathsf{in}\,}
\newcommand{\fixLam}[2]{\mu {#1} {#2}.}
\newcommand{\ifElse}[3]{\mathsf{if} (#1 \le \num{0}) \, {#2} \,\mathsf{else}\, {#3}}

%%Basic notions
\newcommand{\pspace}{(\Omega,\mathcal{F},\probm)}
\newcommand{\probm}{\mathbb{P}}
\newcommand{\condexpv}[2]{{\expt}{\left[{#1} \mid {#2}\right]}}

\newcommand{\stdConf}[1]{(#1)}
%\newcommand{\intConf}[1]{\lceil#1\rceil}
%\newcommand{\intConf}[1]{(#1)}
%\newcommand{\symConf}[1]{\langle\!\langle  #1 \rangle\!\rangle}
%\newcommand\symPath[1]{(#1)}
\newcommand{\symPath}[1]{\langle\!\langle  #1 \rangle\!\rangle}
\newcommand\symConf[1]{(#1)}

\newcommand{\ifSimple}[3]{\mathsf{if}(#1, #2, #3)}
%\newcommand{\ifElse}[3]{\mathsf{if} (#1 \le 0) \, \allowbreak {#2} \, \allowbreak \mathsf{else}\, {#3}}
%\newcommand{\ifElse}[3]{\ifSimple{#1}{#2}{#3}}

%\newcommand{\trace}{\mathsf{s}}
%
%\newcommand\defn[1]{{\bf \em #1}}
\newcommand{\traces}{\mathbb{T}}
%
%\newcommand{\stdConf}[1]{(#1)}
%%\newcommand{\intConf}[1]{\lceil#1\rceil}
%\newcommand{\intConf}[1]{(#1)}
%%\newcommand{\symConf}[1]{\langle\!\langle  #1 \rangle\!\rangle}
%%\newcommand\symPath[1]{(#1)}
%\newcommand{\symPath}[1]{\langle\!\langle  #1 \rangle\!\rangle}
%\newcommand\symConf[1]{(#1)}

\newcommand{\valueSem}[1]{\mathsf{val}_{#1}} % value (semantics)
\newcommand{\weightSem}[1]{\mathsf{wt}_{#1}} % weight (semantics)
\newcommand{\measureSem}[1]{\llbracket #1 \rrbracket}
\newcommand{\posterior}{\mathsf{posterior}}


%%%%%%%%%
% 
%%%%%%%%
\newcommand{\loc}{\ell}
\newcommand{\locs}{\mathit{L}}
\newcommand{\blocs}{\mathit{L}_{\mathrm{b}}}

\newcommand{\iflocs}{\mathit{L}_{\mathrm{if}}}
\newcommand{\looplocs}{\mathit{L}_{\mathrm{while}}}

\newcommand{\alocs}{\mathit{L}_{\mathrm{a}}}
\newcommand{\wlocs}{\mathit{L}_{\mathrm{w}}}
\newcommand{\rlocs}{\mathit{L}_{\mathrm{r}}}
\newcommand{\Alocs}[1]{\mathit{L}_{\mathrm{A}}^{\mathsf{#1}}}
\newcommand{\Dlocs}{\mathit{L}_{\mathrm{nd}}}
\newcommand{\transitions}{{\rightarrow}}

%%% 
\newcommand{\plocs}{\mathit{L}_{\mathrm{p}}}
\newcommand{\tlocs}{\mathit{L}_{\mathrm{t}}}

\newcommand{\lin}{\loc_\mathrm{init}}
\newcommand{\lout}{\loc_\mathrm{out}}
\newcommand{\val}[1]{\mbox{\sl Val}_{#1}}

\newcommand{\pvars}{V_\mathrm{p}}
\newcommand{\rvars}{V_{\mathrm{r}}}
\newcommand{\pre}{\mathrm{pre}}

\newcommand{\sle}{\sqsubseteq}
\newcommand{\sge}{\sqsupseteq}

\newcommand{\lfp}{\mathrm{lfp}}
\newcommand{\gfp}{\mathrm{gfp}}

\newcommand{\rdvarjdis}{\mathcal D}
\newcommand{\sampset}{\textit{supp}}

\newcommand{\upd}{\mbox{\sl upd}}
\newcommand{\wet}{\mbox{\sl wt}}
\newcommand{\transset}{\mathfrak T}
\newcommand{\valin}{\pv_{\mathrm{init}}}
\newcommand{\ret}{\mbox{\sl ret}}

\newcommand{\win}{w_{\mathrm{init}}}

\newcommand{\sampdpd}{\overline{\Upsilon}}

\newcommand{\outmap}{\text{O}}
\newcommand{\sat}[1]{\langle #1 \rangle}
\newcommand{\monoid}{\mbox{\sl Monoid}}
\newcommand{\handelmanformat}{(\dagger)}

\newcommand{\trunc}{\mathcal{B}}

\newcommand{\ewt}{\mbox{\sl ewt}}
\newcommand{\statemap}{\text{St}}

\newcommand{\valrd}{{\mathbf{r}}}
\newcommand{\frmloc}{\ell^{\mathrm{src}}}
\newcommand{\toloc}{\ell^{\mathrm{dst}}}

\newcommand{\monomials}{\mathbf{M}}

\definecolor{codegreen}{rgb}{0,0.6,0}
\definecolor{codegray}{rgb}{0.5,0.5,0.5}
\definecolor{codepurple}{rgb}{0.58,0,0.82}
\definecolor{backcolour}{rgb}{0.95,0.95,0.92}

\definecolor{keywordColor}{rgb}{0.0,0.0,0.5} % the color of language keywords
\definecolor{rulenameColor}{rgb}{0.5,0.5,0.5} % the color of rule names

\newcommand{\sref}[2]{\hyperref[#2]{#1~\ref{#2}}} % workaround for autoref using Theorem for Lemmas and Propositions
\newcommand{\srefcase}[3]{\hyperref[#2]{#1~\ref{#2}~(#3)}} % workaround for autoref using Theorem for Lemmas and Propositions (linking to a particular case of the Lemma)

\def\sectionautorefname{Section}
\def\subsectionautorefname{Section}
\def\subsubsectionautorefname{Section}

\def\lstlanguagefiles{aeff}
\lstset{language=aeff,upquote=true}
\let\ls\lstinline

\newtcolorbox{aeffbox}{
  enhanced, 
  frame hidden, 
  borderline west = {2pt}{0pt}{lightgray}, 
  colback=white, 
  boxsep=0pt, 
  top=0pt, 
  bottom=0pt, 
  left=0pt, 
  right=0pt, 
  toprule=0pt, 
  bottomrule=0pt
}

\makeatletter
\tcbset{
  after app={%
    \ifx\tcb@drawcolorbox\tcb@drawcolorbox@breakable
    \else
      % add only when not breakabel
      \@endparenv
    \fi
  }
}

% for breakable
\appto\tcb@use@after@lastbox{\@endparenv\@doendpe}
\makeatother

\bibliographystyle{alphaurl}
%%%%%%%%%%%%%%%%%%%%%%%%%%%%%%%%%%%%%%%%%%%%%%%%%%%%

\title[Higher-Order Asynchronous Effects]{Higher-Order Asynchronous Effects\rsuper*}
\titlecomment{{\lsuper*}This paper is an extended version of our previous work~\cite{Ahman:POPL}: 
it simplifies the meta-theory, removes the reliance on general recursion for reinstalling interrupt handlers, 
adds state to reinstallable interrupt handlers, and extends the calculus with higher-order signal and interrupt 
payloads, and with dynamic process creation.}
\thanks{ 
  This project has received funding from the European Union's Horizon 2020 research and 
  innovation programme under the Marie Sk\l{}odowska-Curie grant agreement No 834146
  \raisebox{-0.05cm}{
    \hspace{-0.15cm}
    % Figure removed
    \hspace{-0.15cm}
  }.
  This material is based upon work supported by the Air Force Office of Scientific Research under 
  awards number FA9550-17-1-0326 and FA9550-21-1-0024.
}

\author[D.~Ahman]{Danel Ahman\lmcsorcid{0000-0001-6595-2756}}[a]
\author[M.~Pretnar]{Matija Pretnar\lmcsorcid{0000-0001-7755-2303}}[b,c]

\address{University of Tartu, Institute of Computer Science, Narva mnt 18, Tartu, Estonia}
\email{danel.ahman@ut.ee}

\address{University of Ljubljana, Faculty of Mathematics and Physics, Jadranska 19, Ljubljana, Slovenia}
\email{matija.pretnar@fmf.uni-lj.si}

\address{Institute of Mathematics, Physics and Mechanics,  Jadranska 19, Ljubljana, Slovenia}

\begin{abstract}
  \noindent
  We explore asynchronous programming with algebraic effects. We complement their conventional 
  synchronous treatment by showing how to naturally also accommodate asynchrony within them, 
  namely, by decoupling the execution of operation calls into signalling that an operation's implementation 
  needs to be executed, and interrupting a running computation with the operation's result, to which the 
  computation can react by installing interrupt handlers. We formalise these ideas in a small core calculus
  and demonstrate its flexibility using examples ranging from a multi-party web application, to pre-emptive
  multi-threading, to (cancellable) remote function calls, to a parallel variant of runners of algebraic effects.
  In addition, the paper is accompanied by a formalisation of the calculus's type safety proofs in \pl{Agda},
  and a prototype implementation in \pl{OCaml}.
  
\end{abstract}

\maketitle

\section{Introduction}
Current quantum hardware is unable to carry out universal quantum computations due to the buildup of errors that occur during the computation. 
The magnitude of the individual error is currently above the value that the Threshold Theorem requires in order to kick-start quantum error correction and fault-tolerant quantum computation~\cite[Section 10.6]{nielsen_chuang_2010}. 
Although the experimentally achieved fidelity rates are promising and the error bounds are inching closer to the required threshold, we will have to work for the foreseeable future with quantum hardware with errors that build-up during the computation.  This implies that we can only do a limited number of steps before the output of the computation has become completely uncorrelated with the intended one.

For fault-tolerant quantum computing, we repeat four steps: 
1) We apply a number of single and two-qubit quantum gates, in parallel whenever possible; 
2) We perform a syndrome measurement on a subset of the qubits; 
3) We perform fast classical computations to determine which errors have occurred and how to correct them; 
and, 4) We apply correction terms based on the classical computations.
We then repeat these four steps with a next sequence of gates. 
These four steps are essential to fault-tolerant quantum computing. 


The starting point of this work is to use the four steps outlined above, not to carry out error correction and fault-tolerant computation, but to enhance short, constant-depth, {\em uncorrected} quantum circuits that perform single qubit gates and {\em nearest-neighbor} two qubit gates. 
Since in the long run we will have to implement error-correction and fault-tolerant computation anyhow, and this is done by such a four-step process, why not make other use of this architecture? Moreover, on some of the quantum hardware platforms, these operations are already in place.
Embracing this idea we naturally arrive at the question: what is the computational power of \textit{low-depth} quantum-classical circuits organized as in the four steps outlined above? 
We thus investigate circuits that execute a small, ideally constant, number of stages, where at each stage we may apply, in parallel, single qubit gates and {\em nearest-neighbor} two qubit gates, followed by measurements, followed by low-depth classical computations of which the outcome can control quantum gates in later stages. 
It is not clear, at first, whether such circuits, especially with constant depth, can do anything remotely useful. 
But we will see that this is indeed the case: many quantum computations can be done by such circuits in constant depth. 
By parallelizing quantum computations in this way, we improve the overall computational capabilities of these circuits, as we do not incur errors on qubits that are idle, simply because qubits are not idle for a very long time. 
Furthermore, reducing the depth of quantum circuits, at the cost of increasing width, allows the circuit to be run faster even if errors occur.

The first usage of such a four-step layout, not to do error correction, but to perform computations, can be found in the paradigm of measurement-based quantum computing~\cite{gottesman1999demonstrating,raussendorf2001one,jozsa2006introduction,clark2007generalised}: 
A universal form of quantum computing where a quantum state is prepared and operations are performed by measuring qubits in different bases, depending on previous measurements and intermediate measurements.

\citeauthor{PhamSvore2013} were the first to formalize the four-step protocol for performing computations~\cite{PhamSvore2013}. They included specific hardware topologies by considering two-dimensional graphs for imposing constraints on qubit interactions. In their model, they develop circuits for particularly useful multi-qubit gates, including specifying costs in the width, number of qubits, depth, number of concurrent time steps, size, and total number of non-Identity operations.
As a result, they find an algorithm that factors integers in polylogarithmic depth.
\citeauthor{Browne:2011} showed that the main tool in the work by \citeauthor{PhamSvore2013}, the fan-out gate, can also be replaced by additional log-depth classical computations in the measurement-based quantum computing setting~\cite{Browne:2011}.

More recently, \citeauthor{Cirac:2021} introduced a scheme to implement unitary operations involving quantum circuits combined with Local Operations and Classical Communication ($\mathsf{LOCC}$) channels: $\mathsf{LOCC}$-assisted quantum circuits~\cite{Cirac:2021}. Similarly to the four-step scheme we just described, they allow for a short depth circuit to be run on the qubits, followed by one round of $\mathsf{LOCC}$, in which ancilla qubits are measured and local unitaries are applied based on the measurement outcomes. They show that in this model any 1D transitionally invariant matrix-product state (MPS) with fixed bond dimension is in the same phase of matter as the trivial state. Similar ideas can be found in~\cite{TVV_NonAbelianTopologicalOrder_2022, tantivasadakarn2021long}.

In this work, we introduce a new model, called \textit{Local Alternating Quantum-Classical Computations} ($\LAQCC$). In this model we alternate between running quantum circuits (constrained by locality), ending in the measurement of a subset of qubits, and fast classical computations based on the measurement results. The outcome of the classical computations are then used to control future quantum circuits. We allow for flexibility in this model, by giving different constraints to the power of both the quantum circuits and the classical circuits as well as the number of alternations between them. 
Most attention will be given to $\LAQCC$ containing quantum circuits of constant depth, classical circuits of logarithmic depth and at most a constant number of alternations between them. 
Any circuit constructed in this model is considered to be of constant depth. 
We restrict ourselves to logarithmic depth classical computations, as this is the first natural and non-trivial extension beyond constant-depth classical computations. 
Constant-depth classical computations do however also have an equivalent constant-depth quantum implementation.

The definition of $\LAQCC$ sharpens the original definition of \citeauthor{PhamSvore2013} by adding constraints to the intermediate classical computations. This allows us to bound the power of $\LAQCC$ from above. 

The main result of \citeauthor{Cirac:2021}, that 1D translational invariant MPS with fixed bond dimension can be prepared by $\mathsf{LOCC}$-assisted circuits, relies on local symmetries of the MPS. These symmetries allow them to prepare local states (on a constant number of qubits) and glue them together by doing one round of the appropriate entangling measurement and corrections, after which they run a round of local unitaries to get the desired result. This general scheme for preparing states that exhibit an MPS description with the appropriate local symmetries requires only geometrically local unitaries and one round of measurement and corrections an therefore is accessible in $\LAQCC$. Studying different local symmetries, known as Symmetry Protected Topological (SPT) phases of matter, to find measurement-based constant depth circuits for states is a broad ongoing field of research~\cite{TVV_NonAbelianTopologicalOrder_2022, tantivasadakarn2021long, smith2023deterministic}. 
All these schemes have a $\LAQCC$ implementation.

%$\LAQCC$-circuits also exist for general schemes of preparing local states, based on the local tensors, and gluing them together using one round of entangled measurement and corrections, based on the local symmetry. 
%The main result of \citeauthor{Cirac:2021}, that 1D translational invariant MPS with fixed bond dimension can be prepared by $\mathsf{LOCC}$-assisted circuits, relies heavily on local symmetries of the MPS and as a result also has an equivalent $\LAQCC$ implementation. 
%The corrections applied after the measurement round are local unitaries depending on the local symmetries of the MPS. 

 

%This general scheme of preparing local states, based on the local tensors, and gluing it together by doing one round of entangled measurement and corrections, based on the local symmetry, is accessible in $\LAQCC$.
Note however that \citeauthor{Cirac:2021} also suggest a circuit for the $W$-state.
This circuit uses sequentially and dependent measurement-based corrections of the ancilla qubits. 
These dependent measurements translate to sequential alternations between the quantum and classical circuits and therefore increase the total depth to linear depth, exceeding the constant-depth constraints imposed by $\LAQCC$-circuits. 

We study the power of the $\LAQCC$ model with respect to state preparation, showing that even with only constant quantum-depth and logarithmic classical depth it remains possible to prepare states with long-range entanglement.
Another surprising result is that it is unlikely that $\LAQCC$ circuits are classically simulatable. We show that any instantaneous quantum polynomial-time (IQP) circuit~\cite{Bremner2010,Shepherd2009} has an $\LAQCC$ implementation.
Classical simulation of IQP circuits implies the collapse of the polynomial hierarchy to the third level, which is not believed to be true~\cite{Bremner2017}. Therefore, we expect that $\LAQCC$ circuits are unlikely to be classically simulatable. We bound the power of $\LAQCC$ by showing that it is contained in $\QNC^1$, the class of polynomial-size, log-depth circuits.

Next, we also study the power that intermediate classical calculations can add to quantum computations, by considering a new model that alternates between polynomially many polynomial-depth quantum circuits and unbounded classical computations
We study this model by doing a complexity theoretical analysis, where we draw inspiration from the notions of complexity given by \citeauthor{RosenthalYuen:2022}, \citeauthor{MetgerYuen:2023}, and \citeauthor{Aaronson:2004}.
All three complexity notions are based on the notion of state preparation, instead of more traditional definition of complexity such as the decidability of a computational problem. 
The first two consider classes based on sequences of quantum states preparable by a polynomial-sized quantum circuit, where the circuits are uniformly generated by a computational class, for instance, the class $\mathsf{PSPACE}$, which results in the complexity class $\mathsf{StatePSPACE}$~\cite{RosenthalYuen:2022,MetgerYuen:2023}.
The third notion considers a relative complexity, where the complexity is measured between two given states, and is measured by the number of gates, from a given gate-set, required to transform one state in another state~\cite{Aaronson:2004}. 
For our definition of state preparation complexity, we drop the uniformity constraint from~\cite{RosenthalYuen:2022,MetgerYuen:2023} and define a class as $\mathsf{StateX}$, which refers to states preparable by circuits of type $\mathsf{X}$. 
As an example, if $\mathsf{X} = \QNC^0$, this results in the class $\mathsf{StateQNC^0}$, which is the set of states preparable from the $\ket{0}^n$ state by poly-size constant-depth circuits. 
This notion is similar to the relative complexity from~\cite{Aaronson:2004}, where one state is the  $\ket{0}^n$ state and instead of counting the number of gates we consider the set of states preparable by a fixed number of gates. Using this notion of complexity we show that any state preparable by an $\LAQCC^*$ circuit is also preparable by a $\mathsf{PostQPoly}$ circuit, the class of circuits of polynomial depth with an additional post-selection gate. 

All Clifford circuits have a constant-depth $\LAQCC$ implementation, implying that any stabilizer state can be implemented by a constant-depth $\LAQCC$ circuit, see Section~\ref{sec:clifford_circuits} for a proof of this statement. 
Efficient circuits for stabilizer states have been known already through measurement-based quantum computing. Therefore this paper focuses on the preparation of non-stabilizer states, and as a surprising result we find novel constant-depth protocols for four very natural classes of non-stabilizer states.
Despite the extensive research into these four classes of non-stabilizer states and the many applications of them, no efficient constant- or low-depth state preparation protocols are known yet. We specifically consider these four classes as they are all often used as initial states in other algorithms.

The first state is a uniform superposition over an arbitrary number of states. 
This state finds applications in many quantum algorithms, as they often start with a uniform superposition over multiple states. 
This superposition is often achieved by applying Hadamard gates to every qubit due to its simplicity to prepare. 
Yet, the analysis of many algorithms, such as Shor's algorithm~\cite{Shor:1997}, would benefit from a different initial superposition. 
The circuit to prepare the uniform superposition over an arbitrary number of states uses an exact version of Grover search as a subroutine, that turns a probabilistic circuit, with a known constant probability of success, into a deterministic circuit. 
We use the circuit for preparing a uniform superposition over an arbitrary number of states as a subroutine in the next two quantum state preparation protocols. 

The second state is the $W$-state, the uniform superposition over all computational basis states of Hamming-weight~$1$, a natural long-ranged entangled state that displays a fundamentally nonequivalent type of entanglement from the Greenberger–Horne–Zeilinger state~\cite{WState:2000}, for which $\LAQCC$-type constant-depth circuits were previously known~\cite{PhamSvore2013, Cirac:2021}. 
The $W$-state is often used as benchmark for new quantum hardware~\cite{Haffner2005,Neeley2010,GarciaPerez:2021}. 
A novel way to prepare the $W$-state therefore gives a new way to benchmark different quantum devices with each other. 
A circuit for preparing the $W$-state was given in~\cite{Cirac:2021}, but this implementation requires sequentially alternating measurements followed by local unitaries, which in the $\LAQCC$ model is not considered to be of constant depth. 
We improve this protocol by giving an $\LAQCC$ implementation of the $W$-state, based on a compress-uncompress method that links the one-hot and binary encoding of integers.

The third state considered is the Dicke state, a generalization of the $W$-state, a superposition over all computational basis states with Hamming-weight $k$~\cite{Dicke:1954}. 
Dicke states have relevance in various practical settings.
For instance, for quantum game theory~\cite{zdemir2007}, quantum storage~\cite{Bacon_Compress:2006,Plesch:2010}, quantum error correction~\cite{ouyang2014permutation}, quantum metrology~\cite{toth2012multipartite}, and quantum networking~\cite{prevedel2009experimental}. 
Dicke states have been used as a starting state for variational optimization algorithms, most notably Quantum Alternating Operator Ansatz (QAOA)~\cite{Hadfield2019}, to find solutions to problems such as Maximum k-vertex Cover~\cite{Brandhofer2022,cook2020quantum}.
The ground states of physical Hamiltonians describing one-dimensional chains tend to show a resemblance to Dicke states such as states resulting from the Bethe ansatz, making them an ideal starting state when investigating the ground state behavior of these Hamiltonians~\cite{TDL_BetheAnsatzDerivation:2010,B_ExcitedStateQuantumPhaseTransitions:2013,DickeTransitions:2021}. 
For instance, the algorithm by \citeauthor{van2021preparing}, who give an algorithm to prepare the Bethe ansatz eigenstates of the spin-1/2 XXZ spin chain, starts by first preparing a Dicke state~\cite{van2021preparing}. 
A Dicke-state preparation protocol based on the compress-uncompress methodology used in the $W$-state furthermore finds applications in entanglement distillation, where the entanglement of a large state is concentrated on only a few qubits. 
Efficient deterministic circuits for preparing Dicke states have been proposed by \citeauthor{bartschi2019deterministic}~\cite{bartschi2019deterministic, bartschi2022deterministic_short_depth}. 
They provide a quantum circuit of depth $\mathO(k \log(\frac{n}{k}))$, allowing arbitrary connectivity, to prepare a Dicke state, which they conjecture to be optimal when $k$ is constant. 
In this work, we provide a constant-depth $\LAQCC$ circuit below their conjectured bound already for constant $k$. 
However, this does not directly disprove their conjecture, as we allow for intermediate measurements and classical computations. 
More significantly, we even construct constant-depth $\LAQCC$ circuits for $k = \mathO(\sqrt{n})$ greatly improving their bound.
This construction extends the compress-uncompress method for the $W$-state combined with additional subroutines. 

We continue with a log-depth state preparation protocol for the Dicke-state for arbitrary $k$. 
This protocol implements an efficient transformation between the factoradic number representation and the combinatorial number representation of a positive integer. 
The combinatorial number representation relates directly to the Dicke state. 
The provided efficient transformation between number representation systems might be of independent interest. 

We conclude by modifying our protocol for preparing a Dicke-state to a protocol that prepares quantum many-body scar states in constant-depth. 
These states have low entanglement and longer coherence times than states with similar energy density.
These characteristics make many-body scar states interesting to analyze and relevant within physics.
Many-body scar states appear for instance in the AKLT model~\cite{AKLT:1987,MRBAR:2018,MRB:2018} and different spin models~\cite{SI:2019,MOBFR:2020}.
Known methods for preparing these states have polynomial-depth~\cite{Gustafson:2023}, whereas our circuit has constant depth. 

% We conclude by studying the power that intermediate classical calculations can add to quantum computations. 
% In this study, we define a new model that relaxes constant-depth quantum circuits to polynomial depth quantum circuits, log-depth classical calculations to unbounded classical computations and a constant number of alternations to a polynomial number of alternations. 
% We call this model $\LAQCC^*$. 
% We study this model by doing a complexity theoretical analysis, where we draw inspiration from the notions of complexity given by \citeauthor{RosenthalYuen:2022}, \citeauthor{MetgerYuen:2023}, and \citeauthor{Aaronson:2004}.
% All three complexity notions are based on the notion of state preparation, instead of more traditional definition of complexity such as the decidability of a computational problem. 
% The first two consider classes based on sequences of quantum states preparable by a polynomial-sized quantum circuit, where the circuits are uniformly generated by a computational class, for instance, the class $\mathsf{PSPACE}$, which results in the complexity class $\mathsf{StatePSPACE}$~\cite{RosenthalYuen:2022,MetgerYuen:2023}.
% The third notion considers a relative complexity, where the complexity is measured between two given states, and is measured by the number of gates, from a given gate-set, required to transform one state in another state~\cite{Aaronson:2004}. 
% For our definition of state preparation complexity, we drop the uniformity constraint from~\cite{RosenthalYuen:2022,MetgerYuen:2023} and define a class as $\mathsf{StateX}$, which refers to states preparable by circuits of type $\mathsf{X}$. 
% As an example, if $\mathsf{X} = \QNC^0$, this results in the class $\mathsf{StateQNC^0}$, which is the set of states preparable from the $\ket{0}^n$ state by poly-size constant-depth circuits. 
% This notion is similar to the relative complexity from~\cite{Aaronson:2004}, where one state is the  $\ket{0}^n$ state and instead of counting the number of gates we consider the set of states preparable by a fixed number of gates. Using this notion of complexity we show that any state preparable by an $\LAQCC^*$ circuit is also preparable by a $\mathsf{PostQPoly}$ circuit, the class of circuits of polynomial depth with an additional post-selection gate. 

\paragraph{Summary of results}
\begin{itemize}
    \item We give a new definition of a computational model that captures the power of the four step process: applying a constant number of layers of one- and two-qubit gates; performing a syndrome measurement; perform a fast classical computation determining corrections; apply corrections. We call this model \emph{Local Alternating Quantum Classical Computations}, or $\LAQCC$ for short. In this model we bound the allowed quantum operations, intermediate classical calculations, and number of rounds separately. In Section~\ref{sec:LAQCC_model} we define this model and give a list of operations based on results from literature contained in this computational model. In some of these operations we explicitly use that we allow for multiple, but at most constant, rounds  of corrections.
    \item  We show show that there exist $\LAQCC$ circuits that can not be weakly simulated in Section~\ref{sec:IQP_in_LAQCC}. We further show that for every $\LAQCC$ circuit there exists a $\QNC^1$ circuit simulating it perfectly, in Section~\ref{sec:LAQCC_in_QNC1}.
    \item We introduce a new type computational complexity for preparing states and show that the extension of $\LAQCC$ where we allow a polynomial number of rounds and unbounded classical computation, is contained in $\mathsf{PostQPoly}$, the class of polynomial circuits with post-selection, in Section~\ref{sec:Complexity results}.
    \item We show a protocol to prepare the uniform superposition state of size $q$ in $\LAQCC$ using $\mathO(\ceil{\log_2(q)}^2)$ qubits in Section~\ref{sec:superposition_modulo_q}. 
    \item We show a protocol to prepare the $W_n$ state in $\LAQCC$ using $\mathO(n\log(n))$ qubits in Section~\ref{sec:W_state_in_LAQCC}.
    \item We show two ways of preparing the Dicke-$(n,k)$ state. The first method is in $\LAQCC$, works up to $k = \mathO(\sqrt{n})$, uses $\mathO(n^2\log(n))$ qubits, and is found in Section~\ref{sec:dicke:small_k}. The second method is in $\LAQCC\text{-}\mathsf{LOG}$ (an extension of $\LAQCC$ allowing for logarithmic number of alterations instead of constant), works for any $k$, uses $\mathO(\text{poly}(n))$ qubits, and is found in Section~\ref{sec:Dicke_in_LAQCC_LOG}. 
    \item We extend on our $\LAQCC$ method of generating Dicke-$(n,k)$ states for $k = \mathO(\sqrt{n})$ and show a protocol to generate many-body scar states for a particular Hamiltonian in $\LAQCC$ (Section~\ref{sec:many_body_scar}). 
\end{itemize}
Summarized in a table, we provide the following state generation protocols:
\begin{table}[htb]
\centering
\begin{tabular}{l|l|l|l}
\textbf{State description} & \textbf{Width} & \textbf{Depth} & \textbf{Implementation}\\
\hline 
Uniform superposition mod $q$: $\frac{1}{\sqrt{q}} \sum_{i = 0}^{q-1}\ket{i}$ & $\mathO(\ceil{\log^2 q})$ & $\mathO(1)$ & Section~\ref{sec:superposition_modulo_q}\\

$W$-state: $\frac{1}{\sqrt{n}}\sum_{i = 0}^{n-1}\ket{e_i}$ & $\mathO(n \log n)$ & $\mathO(1)$ & Section~\ref{sec:W_state_in_LAQCC}\\

Dicke-$(n,k)$, $k = \mathO(\sqrt{n})$: $\binom{n}{k}^{-1/2}\sum_{x \in \{0,1\}^n: |x| = k} \ket{x}$ &  $\mathO(n^2\log n)$ & $\mathO(1)$ 
&Section~\ref{sec:dicke:small_k}\\

Dicke-$(n,k)$: $\binom{n}{k}^{-1/2}\sum_{x \in \{0,1\}^n: |x| = k} \ket{x}$ & $\mathO(\text{poly}(n))$ & $\mathO(\log n)$ &Section~\ref{sec:Dicke_in_LAQCC_LOG}\\

QMBS: $\ket{S_k} = \frac{1}{k! \sqrt{\mathcal N(n,k)}}(Q^\dagger)^k \ket{\Omega}$ &  $\mathO(n^2\log n)$ & $\mathO(1)$  &  Section~\ref{sec:many_body_scar}
\end{tabular}
\caption{Summary of state preparation protocols given in this paper.}
\label{tab:sate_prep}
\end{table}
In the entry for the quantum many-body scar state $Q$ denotes the raising operator and $\mathcal N(n,k)=\binom{n-k-1}{k}$. 
Section~\ref{sec:many_body_scar} will provide more details on the variables and the implementation. 

\paragraph{Organization of the paper}
\noindent We first introduce relevant preliminaries in Section~\ref{sec:preliminaries}. 
In Section~\ref{sec:LAQCC_model} we formally define the class of Local Alternating Quantum-Classical Computations ($\LAQCC$). We also show that any Clifford circuit can be implemented in constant depth $\LAQCC$ (a result based on a result from measurement-based quantum computing~\cite{jozsa2006introduction}). 
This result allows us to give many useful multi-qubit gates and routines in Section~\ref{sec:gates_created_in_LAQCC}. 
Beyond that we show that constant depth $\LAQCC$ circuits are contained in $\QNC^1$ and that any $\mathsf{IQP}$ circuit has an $\LAQCC$ implementation.
We conclude this section with an analysis of a more powerful instantiation of $\LAQCC$ and show an inclusion with respect to the class $\mathsf{PostQPoly}$, which is the class of circuits of polynomial depth with one additional post-selection gate. 
In Section~\ref{sec:state_prep_in_LAQCC} we give $\LAQCC$ circuit implementations for preparing the uniform superposition over an arbitrary number of states, the $W$-state and the Dicke state up to $k = \mathO(\sqrt{n})$. We furthermore give a log-depth circuit implementation for preparing the Dicke state for any $k$. We conclude by showing a $\LAQCC$ circuit for generating many body scar states of a particular type of Hamiltonian.


\section{Secure Design of \puma}\label{sec:design}
In this section, we first present an overview of \puma, and present the protocols for secure $\gelu$ , $\softmax$, embedding, and $\layernorm$ used by \puma. Note that the linear layers such as matrix multiplication are straightforward in replicated secret sharing, so we mainly describe our protocols for non-linear layers in this manuscript.

\subsection{Overview of \puma}\label{sec:overview}
To achieve secure inference of Transformer models, \puma\ defines three kinds of roles: one model owner, one client, and three computing parties. The model owner and the client  provide their models or inputs to the computing parties (i.e., $P_0$, $P_1$, and $P_2$) in a secret-shared form, then the computing parties execute the MPC protocols and send the results back to the client. Note that the model owner and client can also act as one of the computing party, we describe them separately for generality. \eg, when the model owner acts as $P_0$, the client acts as  $P_1$, a third-party dealer acts as $P_2$, the system model becomes the same with \mpcformer~\citep{li2023mpcformer}.

During the secure inference process, a key invariant is maintained: For any layer, the computing parties always start with 2-out-of-3 replicated secret shares of the previous layer's output and the model weights, and end with 2-out-of-3 replicated secret shares of this layer's output. As the shares do not leak any information to each party, this ensures that the layers can be sequentially combined for arbitrary depths to obtain a secure computation scheme for any Transformer-based model.
%The main focus of \puma\ is to reduce the computation and communication costs between the computing parties while maintaining the desired level of security. 



\iffalse
\textbf{Threat Model.}
Following previous works~\citep{aby3,li2023mpcformer},
\puma\ resists a semi-honest (a.k.a., honest-but-curious) adversary in honest-majority~\citep{lindell2009proof}, where the adversary passively corrupts no more than one computing party. Such an adversary follows the protocol specification exactly, but may try to learn more information than permitted. Please note that \puma\ cannot protect against the extraction of information from the inference results, and the examination of mitigating solutions (\eg, differential privacy~\citep{abadi2016deep}) falls outside the scope of this study.
\fi 

\subsection{Protocol for Secure GeLU}\label{sec:gelu}
Most of the current approaches view the $\gelu$ function as a composition of smaller functions and try to optimize each piece of them, making them to miss the
chance of optimizing the private $\gelu$ as a whole. Given the $\gelu$ function:
\begin{equation}\label{eq:gelu}
\begin{split}
    \gelu(x) &= \frac{x}{2} \cdot \left(1 + \tanh \left( \sqrt{\frac{2}{\pi}} \cdot \left(x + 0.044715 \cdot x^3 \right) \right) \right)\\
    &\approx x\cdot \mathsf{sigmoid}(0.071355\cdot x^3 + 1.595769\cdot x) 
\end{split},
\end{equation}
these approaches~\citep{hao2022iron,characmpctranformer} focus either on designing efficient protocols for function $\tanh$
or using the existing MPC protocols of exponentiation and reciprocal for $\mathsf{sigmoid}$. 

However, none of current approaches have utilized the fact that $\gelu$ function is almost linear on the two sides (\ie, $\gelu(x)\approx 0$ for $x<-4$ and $\gelu(x)\approx x$ for $x>3$). 
Within the short interval $[-4,3]$ of $\gelu$,
we suggest a piece-wise approximation of low-degree polynomials is a more efficient and easy-to-implement choice for its secure protocol. Concretely, our piece-wise low-degree polynomials are shown as equation~(\ref{eq:geluapprox}):
\begin{equation}\label{eq:geluapprox}
\gelu(x)=
\begin{cases}
0, & x<-4 \\
F_0(x), & -4 \le x < -1.95 \\
F_1(x), & -1.95 \le x \le 3 \\
x, & x >3
\end{cases},
\end{equation}
where polynomials $F_0()$ and $F_1()$ are computed by library $\mathsf{numpy.ployfit}$\footnote{\url{https://numpy.org/doc/stable/reference/generated/numpy.polyfit.html}} as equation~(\ref{eq:f0f1}). Surprsingly, the above simple poly fit works very well and our $\mathsf{max\ error}< 0.01403$, $\mathsf{median\ error}< 4.41e-05$, and $\mathsf{mean\ error}< 0.00168$.
\begin{equation}\label{eq:f0f1}
\begin{cases}
F_0(x) &= -0.011034134030615728 x^3 -0.11807612951181953 x^2 \\
&- 0.42226581151983866 x -0.5054031199708174\\
F_1(x) &= 0.0018067462606141187x^6 -0.037688200365904236 x^4 \\
&+ 0.3603292692789629x^2 + 0.5x + 0.008526321541038084
\end{cases}
\end{equation}

Formally, given secret input $\share{x}$, our secure $\gelu$ protocol $\Pi_{\gelu}$ is constructed as algorithm~\ref{protocol:gelu}. 
\iffalse
\begin{itemize}
    \item The parties jointly compute
$\share{b_0}^2 = \Pi_{\mathsf{LT}}(\share{x}, 4)$,
$\share{b_1}^2 = \Pi_{\mathsf{LT}}(\share{x}, -1.95)$, and
$\share{b_2}^2 = \Pi_{\mathsf{LT}}(3, \share{x})$.

\item  Then, each $P_i$ locally compute
$\share{b_3}^2 = \share{b_1}^2 \oplus \share{b_2}^ \oplus 1$ and
$\share{b_4}^2 = \share{b_0}^2 \oplus \share{b_1}^2$

\item Finally, the parties compute and return 
$\share{b_2}^2 \cdot \share{x} + \share{b_4}^2 \cdot F_0(\share{x}) + \share{b_3}^2 \cdot F_1(\share{x})$, where polynomials $(F_0, F_1)$ can be computed easily using secure addition and multiplication (and its variants, \eg, secure square)~\citep{spu}. 
\end{itemize}
\fi 

\begin{algorithm}[tp]
\caption{Secure $\gelu$ Protocol $\Pi_{\mathsf{GeLU}}$}\label{protocol:gelu}
\begin{algorithmic}[1]
\REQUIRE
$P_i$ holds the 2-out-of-3 replicate secret share $\share{x}_i$ for $i\in \{0,1,2\}$ 
\ENSURE
$P_i$ gets the 2-out-of-3 replicate secret share $\share{y}_i$ for $i\in \{0,1,2\}$, where $y=\gelu(x)$.

\STATE $P_0$, $P_1$, and $P_2$ jointly compute
\begin{equation*}
\begin{split}
&\shareb{b_0} = \Pi_{\mathsf{LT}}(\share{x}, -4),~~~\vartriangleright b_0 = 1\{x<-4\}\\
&\shareb{b_1} = \Pi_{\mathsf{LT}}(\share{x}, -1.95),~~~\vartriangleright b_1 = 1\{x<-1.95\} \\
&\shareb{b_2} = \Pi_{\mathsf{LT}}(3, \share{x}),~~~~~~\vartriangleright b_2 = 1\{3<x\}
\end{split}
\end{equation*}
and compute 
$\shareb{z_0} = \shareb{b_0} \oplus \shareb{b_1}$,
$\shareb{z_1} = \shareb{b_1} \oplus \shareb{b_2} \oplus 1$, and $\shareb{z_2}=\shareb{b_2}$. Note that $z_0 = 1\{-4\le x < -1.95\}$, $z_1 = 1\{-1.95\le x\le 3\}$, and $z_2 = 1\{x>3\}$.

\STATE Jointly compute $\share{x^2} = \Pi_{\mathsf{Square}}(\share{x})$, $\share{x^3} = \Pi_{\mathsf{Mul}}(\share{x}, \share{x^2})$, $\share{x^4} = \Pi_{\mathsf{Square}}(\share{x^2})$, and $\share{x^6} = \Pi_{\mathsf{Square}}(\share{x^3})$.

\STATE Computing polynomials $\share{F_0(x)}$ and $\share{F_1(x)}$ based on $\{\share{x}, \share{x^2}, \share{x^3}, \share{x^4}, \share{x^6}\}$ as equation~(\ref{eq:geluapprox}) securely.


\RETURN$\share{y} = \Pi_{\mathsf{Mul_{BA}}}(\shareb{z_0}, \share{F_0(x)}) + \Pi_{\mathsf{Mul_{BA}}}(\shareb{z_1}, \share{F_1(x)})+\Pi_{\mathsf{Mul_{BA}}}(\shareb{z_2}, \share{x})$.

\end{algorithmic}
\end{algorithm}



\subsection{Protocol for Secure Softmax}\label{sec:secureatten}

In the function $\attention(\Q,\K,\V)=
\softmax(\Q \cdot \K^\mathsf{T} + \M) \cdot \V$, where $\M$ can be viewed as a bias matrix, the key challenge is computing function $\softmax$. For the sake of numerical stability, the $\softmax$ function is computed as
\begin{equation}\label{eq:softmax}
    \softmax(\x)[i]=\frac{\exp(\x[i] - \bar{x} - \epsilon)}{\sum_i \exp(\x[i] - \bar{x} - \epsilon)},
\end{equation}
where $\bar{x}$ is the maximum element of the input vector $\x$. 
For the normal plaintext softmax, $\epsilon=0$. For a two-dimension matrix, we apply equation~(\ref{eq:softmax}) to each of its row vector.

Formally, our detailed secure protocol  $\Pi_{\softmax}$ is illustrated in algorithm~\ref{protocol:softmax}, where we propose two optimizations:
\begin{itemize}
\item 
For the first optimization, we set $\epsilon$ in equation~\ref{eq:softmax} to a tiny and positive
value, e.g., $\epsilon =
10^{-6}$, so that the inputs to exponentiation
in equation~\ref{eq:softmax} are all negative. We exploit the negative operands
for acceleration. Particularly, we compute the exponentiation using the Taylor series~\citep{tan2021cryptgpu} with a simple clipping
\begin{equation}\label{eq:negexp}
\mathsf{negExp}(x) = \begin{cases}
    0, &x < T_{\exp} \\
    (1+\frac{x}{2^t})^{2^t}, &x\in [T_{\exp},0].
\end{cases}
\end{equation}
Indeed, we apply the less-than for the branch $x < T_{\exp}$
The division by $2^t$ can be achieved using
$\Pi_{\mathsf{Trunc}}^t$ since the input is already negative. Also, we can
compute the power-of-$2^t$ using $t$-step sequences of square function $\Pi_{\mathsf{square}}$ and $\Pi_{\mathsf{Trunc}}^f$. Suppose our MPC program uses
$18$-bit fixed-point precision. Then we set $T_{\exp}=-14$ given $\exp(-14) < 2^{-18}$, and empirically set $t = 5$.


\item 
Our second optimization is to reduce the number of divisions, which ultimately saves computation and communication costs.
To achieve this, for a vector $\x$ of size $n$, we have replaced the operation $\mathsf{Div}(\x, \mathsf{Broadcast}(y))$ with $\x \cdot  \mathsf{Broadcast}(\frac{1}{y})$, where $y=\sum_{i=1}^n\x[i]$. By making this replacement, we effectively reduce $n$ divisions to just one reciprocal operation and $n$ multiplications.
This optimization is particularly beneficial in the case of the $\softmax$ operation. The $\frac{1}{y}$ in the $\softmax$ operation is still large enough to maintain sufficient accuracy under fixed-point values. As a result, this optimization can significantly reduce the computational and communication costs while still providing accurate results.
\end{itemize}

\begin{algorithm}[tp]
\caption{Secure $\softmax$ Protocol $\Pi_{\softmax}$}\label{protocol:softmax}
\begin{algorithmic}[1]
\REQUIRE
$P_i$ holds the 2-out-of-3 replicate secret share $\share{\x}_i$ for $i\in \{0,1,2\}$, and $\x$ is a vector of size $n$. 
\ENSURE
$P_i$ gets the 2-out-of-3 replicate secret share $\share{\y}_i$ for $i\in \{0,1,2\}$, where $\y=\softmax(\x)$.

\STATE $P_0$, $P_1$, and $P_2$ jointly compute
$\shareb{\mathbf{b}} = \Pi_{\mathsf{LT}}(T_{\exp}, \share{\x})$ and the maximum $\share{\bar{x}} = \Pi_{\mathsf{Max}}(\share{\x})$.

\STATE Parties locally computes $\share{\hat{\x}} = \share{\x} - \share{\bar{x}} - \epsilon$, and jointly compute $\share{\z_0} = 1+  \Pi_{\mathsf{Trunc}}^t(\share{\hat{\x}})$.

\FOR{$j=1,2,\dots, t$}
\STATE $\share{\z_j} = \Pi_{\mathsf{Square}}(\share{\z_{j-1}})$.
\ENDFOR

\STATE Parties locally compute $\share{z} = \sum_{i=1}^n \share{\z[i]}$ and jointly compute $\share{1/z} = \Pi_{\mathsf{Recip}}(\share{z})$.

\STATE Parties jointly compute $\share{\z / z} = \Pi_{\mathsf{Mul}}(\share{\z}, \share{1/z})$

\RETURN $\share{\y} = \Pi_{\mathsf{Mul}_{\mathsf{BA}}}( \shareb{\mathbf{b}}, \share{\z / z})$.

\end{algorithmic}
\end{algorithm}

\subsection{Protocol for Secure Embedding}\label{sec:embed}


The current secure embedding procedure described in~\citep{li2023mpcformer} necessitates the client to  generate a one-hot vector using the token $\tokenid$ locally. This deviates from a plaintext Transformer workflow where the one-hot vector is generated inside the model. As a result, they have to carefully strip off the one-hot step from the pre-trained models, and add the step to the client side, which could be an obstacle for deployment. 



To address this issue, we propose a secure embedding design as follows. Assuming that the token $\tokenid\in [n]$ and all embedding vectors are denoted by $\E= (\e_1^T, \e_2^T, \dots, \e_n^T)$, the embedding can be formulated as $\e_{\tokenid} = \mathbf{E}[\tokenid]$. Given $(\tokenid, \E)$ are in secret-shared fashion, our secure embedding protocol $\Pi_{\mathsf{Embed}}$ works as follows:
\begin{itemize}
    \item The computing parties securely compute the one-hot vector $\shareb{\mathbf{o}}$ after receiving $\share{\tokenid}$ from the client. Specifically, $\shareb{\mathbf{o}[i]}=\Pi_{\mathsf{Eq}}(i,\share{\tokenid})$ for $i\in [n]$.
    \item The parties can compute the embedded vector via $\share{\e_{\tokenid}} = \Pi_{\mathsf{Mul_{BA}}}(\share{\E}, \shareb{\mathbf{o}})$, where  does not require secure truncation.
\end{itemize}
In this way, our $\Pi_{\mathsf{Embed}}$ does not require explicit modification of the workflow of plaintext Transformer models, at the cost of more $\Pi_{\mathsf{Eq}}$ and $\Pi_{\mathsf{Mul_{BA}}}$ operations. 



\subsection{Protocol for Secure LayerNorm}\label{sec:seclayernorm}
Recall that given a vector $\x$ of size $n$, $\layernorm(\x)[i] =  \gamma \cdot \frac{\x[i]-\mu}{\sqrt{\sigma}} + \beta$, where $(\gamma, \beta)$ are trained parameters, $\mu = \frac{\sum_{i=1}^n \x[i]}{n}$, and $\sigma = \sum_{i=1}^n (\x[i] - \mu)^2$. In MPC, the key challenge is the evaluation of the divide-square-root $\frac{\x[i]-\mu}{\sqrt{\sigma}}$ formula. To securely evaluate this formula, CrypTen sequentially executes the MPC protocols of square-root, reciprocal, and multiplication. However, we observe that $\frac{\x[i]-\mu}{\sqrt{\sigma}}$ is equal to $(\x[i]-\mu)\cdot \sigma^{-1/2}$. And in the MPC side, the costs of computing the inverse-square-root $\sigma^{-1/2}$ is similar to that of the square-root operation~\citep{rSqrt}. Besides, inspired by the second optimization of \S~\ref{sec:secureatten}, we can first compute $\sigma^{-1/2}$ and then $\mathsf{Broadcast}(\sigma^{-1/2})$ to support fast and secure $\layernorm(\x)$. And our formal protocol $\Pi_{\layernorm}$ is shown in algorithm~\ref{protocol:layernorm}.

\begin{algorithm}[tp]
\caption{Secure $\mathsf{LayerNorm}$ Protocol $\Pi_{\mathsf{LayerNorm}}$}\label{protocol:layernorm}
\begin{algorithmic}[1]
\REQUIRE
$P_i$ holds the 2-out-of-3 replicate secret share $\share{\x}_i$ for $i\in \{0,1,2\}$, and $\x$ is a vector of size $n$. 
\ENSURE
$P_i$ gets the 2-out-of-3 replicate secret share $\share{\y}_i$ for $i\in \{0,1,2\}$, where $\y=\mathsf{LayerNorm}(\x)$.

\STATE $P_0$, $P_1$, and $P_2$ compute $\share{\mu} = \frac{1}{n}\cdot \sum_{i=1}^n\share{\x[i]}$ and $\share{\sigma} = \sum_{i=1}^n \Pi_{\mathsf{Square}}(\share{\x} - \share{\mu})[i]$.

\STATE Parties jointly compute $\share{\sigma^{-1/2}} = \Pi_{\mathsf{rSqrt}}(\share{\sigma})$.

\STATE Parties jointly compute $\share{\mathbf{c}} = \Pi_{\mathsf{Mul}}((\share{\x} - \share{\mu}), \share{\sigma^{-1/2}})$

\RETURN $\share{\y} = \Pi_{\mathsf{Mul}}(\share{\gamma}, \share{\mathbf{c}}) + \share{\beta}$.

\end{algorithmic}
\end{algorithm}
% !TEX root = paper.tex

\section{A Calculus for Asynchronous Effects: Values and Computations}
\label{sec:basic-calculus:computations}

Before we  focus on extensions necessary for higher-order asynchronous effects in \autoref{sec:higher-order-extensions},
we first recap \lambdaAEff, our existing core calculus for programming with first-order asynchronous effects~\cite{Ahman:POPL}.
The version we present here differs from the previous one in two aspects: we drop the reliance on general recursion, as
reinstallable interrupt handlers that we introduce in \autoref{sec:extensions:reinstallable-interrupt-handlers} are sufficient to express
all the existing examples, and we slightly modify the behaviour of
the $\tmkw{await}$ construct in order to make the meta-theory slightly simpler.

To better explain the different features of the calculus and its semantics, we split the recap of
\lambdaAEff~into a \emph{sequential} part (discussed below) and a
\emph{parallel} part (discussed in \autoref{sec:basic-calculus:processes}).

\subsection{Values and Computations}
\label{sec:basic-calculus:values-and-computations}

We base \lambdaAEff~on the fine-grain
call-by-value $\lambda$-calculus (FGCBV)~\cite{Levy:FGCBV}, and as such, it is a low-level
intermediate language to which a corresponding high-level user-facing programming language
could be compiled to---this is what happens in our prototype implementation~\cite{pretnar21:AEff}.

The syntax of terms is given in \autoref{fig:terms}, stratified into \emph{values} and \emph{computations},
as in FGCBV.
While we do not study effect inference in this paper, we equip certain terms with type annotations that
in our experience should make it possible to fully infer types.


% Figure environment removed

\paragraph{Values}

The values $V,W,\ldots$ are mostly standard. They include
variables, introduction forms for
sums and products, and functions. The only \lambdaAEff-specific value
is $\tmpromise V$, which denotes a \emph{fulfilled promise}, indicating that the promise of
handling some interrupt has been fulfilled with the value $V$.

\paragraph{Computations} The computations $M,N,\ldots$ also include all
standard terms from FGCBV:
returning values, sequencing, function
application, and elimination forms.

The first two \lambdaAEff-specific computations are \emph{signals} $\tmopout{op}{V}{M}$ and
\emph{interrupts} $\tmopin{op}{V}{M}$, where
$\opsym{op}$ is drawn from a set $\sig$ of names, $V$ is a data
payload, and $M$ is a continuation.

The next \lambdaAEff-specific computation is the \emph{interrupt handler} $\tmwith{op}{x}{M}{p}{N}$,
where $x$ is bound in $M$ and $p$ in $N$.
As discussed in the previous section, one should understand this computation as making a promise
to handle a future incoming interrupt $\opsym{op}$ by executing the computation $M$. Sub-computations of the continuation
$N$ can then explicitly await, when necessary, for this promise to be fulfilled by blocking on the \emph{promise-typed variable} $p$
using the final \lambdaAEff-specific computation term, the \emph{awaiting} construct $\tmawait{V}{x}{M}$.
It is useful to note that the $p$ used above is an ordinary variable---it just gets assigned the distinguished promise type
$\typromise X$ by the interrupt handler (as discussed in \autoref{sec:basic-calculus:type-system:computations}).

\subsection{Small-Step Operational Semantics}
\label{sec:basic-calculus:semantics:computations}

We equip \lambdaAEff~with an evaluation contexts based
small-step operational semantics,
defined using a reduction relation $M \reduces N$.
The \emph{reduction rules} and \emph{evaluation contexts} are given
in \autoref{fig:small-step-semantics-of-computations}. We discuss
the rules in detail below. Note that since we have chosen to equip effectful constructs with explicit continuations,
the evaluation contexts are used only to compress four reduction rules into a single one. If instead
we took generic versions (like seen in Section~\ref{sec:overview:runningexample})
as primitives, almost all the rules in \autoref{fig:small-step-semantics-of-computations},
apart from the ones for standard monadic computations, would need to be phrased
in terms of sequential composition (i.e., $\tmkw{let}$), leading to a notably less clear presentation.
% Figure environment removed

\paragraph{Computation Rules}
The first group includes \emph{standard reduction rules} from FGCBV, such as $\beta$-reducing function applications, sequential composition, and the standard elimination forms.
%
These rules involve standard \emph{capture avoiding substitutions} $V[W/x]$ and $M[W/x]$,
defined by straightforward mutual structural recursion.

\paragraph{Algebraicity}
This group of reduction rules \emph{propagates outwards} the signals that have been issued, interrupt handlers that have been installed,
and computations awaiting fulfilled promises. While it is not surprising that outgoing signals
behave like algebraic \emph{operation calls}, getting propagated outwards as far as possible, then it is much more curious that
the natural operational behaviour of interrupt handlers turns out to be the same. As we shall explain in \autoref{sec:conclusion},
despite using the (operating systems inspired) ``handler'' terminology, mathematically interrupt handlers are in fact a form of scoped algebraic operations~\cite{Pirog:ScopedOperations}.

In contrast to our original calculus~\cite{Ahman:POPL}, the awaiting construct also propagates outwards. Before,
awaiting a promise in any subcomputation would block the evaluation immediately, whereas now, we can do the
additional outwards propagation steps. Importantly, this does not significantly change the computational behaviour, as after
the propagation, the evaluation still blocks as long as the promise is left unfulfilled. The main difference and benefit is that
all computations awaiting for a promise variable $p$ now show this explicitly at their top-level, as they are of the form
$\tmawait{p}{x}{M}$. This change significantly simplifies
the normal forms of computations (see \autoref{sec:basic-calculus:type-safety}) and the resulting meta-theory.

In the last two algebraicity rules, and other similar ones, we assume Barendregt's variable
convention to avoid accidentally capturing free variables when extending the scope of 
binders.

\paragraph{Commutativity of Signals With Interrupt Handlers}
This rule complements the algebraicity rule for signals, by further propagating
them outwards, past any enveloping interrupt handlers. From the perspective of algebraic effects,
this rule is an example of two algebraic operations \emph{commuting}~\cite{Hyland:SumAndTensor}.
Since in this rule, the scope of $p$ contracts, the usual variable naming precautions are
not sufficient for type safety. Instead, the type system ensures that the (promise-typed)
variable $p$ cannot appear in the payload value $V$.

\paragraph{Interrupt Propagation}
The handler-operation curiosity does not end with interrupt handlers. This group of reduction rules describes how
interrupts are \emph{propagated inwards} into sub-computations. While $\tmopin{op}{V}{M}$ might look like a conventional
operation call, then its operational behaviour instead mirrors that of \emph{deep effect handling}~\cite{Plotkin:HandlingEffects},
where one also recursively descends into the computation being handled.

When designing interrupt propagation, we must ensure that each interrupt handler receives a corresponding interrupt, no matter
how deep inside the computation we install it. The first reduction rule states that
we can safely discard an interrupt when it reaches a trivial, effect-free
computation $\tmreturn W$. The second rule states that we can propagate incoming interrupts past any outward moving signals. The next
two rules describe how interrupts interact with interrupt handlers, in particular, that the former behave like effect handling
(when understanding interrupt handlers as generalised algebraic operations). On the one hand, if the interrupt
matches the interrupt handler it encounters, the corresponding handler code $M$ is executed, and the interrupt is
propagated inwards into the continuation $N$. On the other hand, if the interrupt
does not match the interrupt handler, it is simply propagated past the interrupt handler into $N$.
Finally, to simplify normal forms, we propagate interrupts inside computations awaiting fulfilled promises as well. As with the algebraicity rule,
this lets the computation take a single additional step after which the $\tmkw{await}$ construct reaches the top and blocks
the evaluation.

Note that we have given the interrupt propagation rules only for terms that are in normal form (see \autoref{lem:results-are-final}). For example, we do not push interrupts into both branches of a match.
Instead, for terms that are still reducing, interrupts remain as parts of their evaluation contexts
and wait for inner interrupt handlers to propagate outwards and meet them.

An alternative design choice for interrupt propagation would be to take inspiration from
\emph{shallow interrupt handling}~\cite{Kammar:Handlers}, and instead of always propagating
the interrupts inwards into the continuations of interrupt handlers, the programmers themselves
would have to manually (recursively) reinvoke the interrupts that need to be propagated
inwards. In addition to giving an algebraically more natural semantics (due to the relationship
with deep effect handling), our choice of allowing interrupts to always propagate inwards provides a
more predictable programming model, in which an installed interrupt handler is guaranteed to
be executed whenever a corresponding interrupt is received, no matter what other installed interrupt 
handlers may do on the way. We leave exploring a variant of \lambdaAEff~based on shallow effect 
handling, and its formal relationship to this paper, for future work.

\paragraph{Awaiting a Promise To Be Fulfilled}
In addition to the two rules for outwards propagation, the semantics of the $\tmkw{await}$ construct
includes a $\beta$-rule allowing the blocked computation $M$ to resume executing as $M[V/x]$
when the $\tmkw{await}$ in question is given a fulfilled promise $\tmpromise V$.

\paragraph{Evaluation Contexts}
The semantics allows reductions under \emph{evaluation contexts} $\E$.
Observe that  as discussed earlier, the inclusion of interrupt handlers in the evaluation contexts means that reductions
involve potentially open terms.
Also, differently from the semantics of conventional operation calls \cite{Kammar:Handlers,Bauer:EffectSystem},
our evaluation contexts include outgoing signals. As such, the \emph{evaluation context rule} allows the execution of a computation
to proceed even if a signal has not yet been propagated to its receiver, or when an interrupt has
not yet arrived. Importantly, the evaluation contexts do not include $\tmkw{await}$, so as to model its blocking behaviour.
We write $\E[M]$ for the operation of filling the hole $[~]$ in $\E$ with $M$.

\paragraph{Non-Confluence}
It is worth noting that the asynchronous design means that the operational semantics
is \emph{nondeterministic}. More interestingly, the semantics is also \emph{not confluent}.

For one source of non-confluence, let us consider two reduction sequences of a same computation,
where for better readability, we highlight the active redex for each step:
 \[
\hspace{-0.15cm}
\begin{array}{r@{\,} l}
  & \tmopin{op}{V}{\tmwith{op}{x}{(\tmwith{op'}{y}{M}{q}{\tmawait{q}{z}{M'}})}{p}{\!\highlightgray{N}}}
  \\[1ex]
  \reduces & \highlightgray{\tmopin{op}{V}{\tmwith{op}{x}{(\tmwith{op'}{y}{M}{q}{\tmawait{q}{z}{M'}})}{p}{\!\highlightwhite{N'}}}}
  \\[1ex]
  \reduces & \highlightgray{\tmlet{p}{(\tmwith{op'}{y}{M[V/x]}{q}{\tmawait{q}{z}{M'}})}{\!\highlightwhite{\tmopin{op}{V}{N'}}}}
  \\[1ex]
  \reduces & \tmwith{op'}{y}{M[V/x]}{q}{\tmawait{q}{z}{(\tmlet{p}{M'}{\tmopin{op}{V}{N'}})}}
\end{array}
\]
and
\[
\hspace{-0.15cm}
\begin{array}{r@{\,} l}
  & \highlightgray{\tmopin{op}{V}{\tmwith{op}{x}{(\tmwith{op'}{y}{M}{q}{\tmawait{q}{z}{M'}})}{p}{\!\highlightwhite{N}}}}
  \\[1ex]
  \reduces & \highlightgray{\tmlet{p}{(\tmwith{op'}{y}{M[V/x]}{q}{\tmawait{q}{z}{M'}})}{\!\highlightwhite{\tmopin{op}{V}{N}}}}
  \\[1ex]
  \reduces & \tmwith{op'}{y}{M[V/x]}{q}{\tmawait{q}{z}{(\tmlet{p}{M'}{\tmopin{op}{V}{N}})}}
\end{array}
\]
Here, both final computations are \emph{temporarily} blocked until an incoming interrupt $\opsym{op'}$
is propagated to them and the variable $q$ gets bound to a fulfilled promise. Until this happens,
it is not possible for the blocked continuation $N$ to reduce to $N'$ in the latter final computation.

Another, distinct source of non-confluence concerns the commutativity of outgoing signals with enveloping interrupt
handlers. For instance, the following composite computation
\[
\tmopin{op}{V}{{\tmwith {op} x {\tmopout{op'}{W'}{M}} p {\tmopout{op''}{W''}{N}}}}
\]
can nondeterministically reduce to either
\[
\tmopout{op'}{W'}{\tmopout{op''}{W''}{{\tmlet{p}{M}{\tmopin{op}{V}{N}}}}}
\]
if we first propagate the interrupt $\op$ inwards, or to
\[
\tmopout{op''}{W''}{\tmopout{op'}{W'}{{\tmlet{p}{M}{\tmopin{op}{V}{N}}}}}
\]
if we first propagate the signal $\op''$ outwards. As a result, in the resulting two computations,
the signals $\op'$ and $\op''$ get issued, and received by other processes, in a different order.

\paragraph{A More Efficient Operational Semantics?}
Finally, it is worth emphasising that the operational semantics we present in
this paper is meant to serve as a declarative reference semantics of
\lambdaAEff, and as a means to relate the behaviour of the program constructs
specific to \lambdaAEff~to the behaviour of conventional algebraic effects and
their handlers. As such, the semantics is clearly not as efficient as one might
desire in a real-world implementation. For instance, in the current semantics,
signals are propagated out of computations one small step at a time. Instead,
one might consider an alternative semantics in which there would be a reduction
rule to pull signals out of computations from arbitrary depths. Dually, the
propagation of interrupts into computations also happens one small step at a
time. Here one might wonder whether it could be possible to use substitution in
\lambdaAEff~to make that propagation more efficient, akin to how we currently
use substitution to propagate fulfilled promises to sub-computations. Yet
another approach could be to model signal and interrupt propagation using shared
channels, as noted in \autoref{sec:conclusion}. However, as in this paper our
focus is not on the efficiency of the semantics, we leave all such explorations
for future work.


\subsection{Type-and-Effect System}
\label{sec:basic-calculus:type-system:computations}

We equip \lambdaAEff~with a type system in the tradition of type-and-effect systems for algebraic effects and
effect handlers \cite{Bauer:EffectSystem,Kammar:Handlers}, by extending the simple type system of FGCBV
with annotations about programs' possible effects (such as issued signals and installed interrupt handlers)
in function and computation types.

\subsubsection{Types}
\label{sec:basic-calculus:type-system:computations:types}

We define types in \autoref{fig:types}, separated into ground, value, and computation types.

% Figure environment removed

As noted in \autoref{sec:basic-calculus:values-and-computations}, \lambdaAEff~is parameterised over a set
$\sig$ of signal and interrupt \emph{names}. To each such name $\op \in \sig$, we assign a \emph{signature}
$\op : A_\op$ that specifies the payload type $A_\op$ of the corresponding signal or interrupt.
%
Crucially, in order to be able to later prove that \lambdaAEff~is type-safe, we must put restrictions
on these signatures, as they classify values that may cross interrupt handler or process boundaries.
In \autoref{sec:extensions:fitch-style-modal-types}, we describe the exact reasons behind this restriction,
and propose a more flexible type system employing Fitch-style modal types~\cite{Clouston:FitchStyle}. But for the sake of exposition, we use here
the more limited approach from our previous work~\cite{Ahman:POPL}, and restrict payload types to
\emph{ground types} $A, B, \ldots$, which include base, unit, empty, product, and sum types, but importantly
exclude promise and function types.

\emph{Value types} $X,Y,\ldots$ extend ground types with function and promise types.
The \emph{function type} $\tyfun{X}{\tycomp{Y}{(\o,\i)}}$ classifies functions that take $X$-typed arguments
to computations classified by the \emph{computation type} $\tycomp{Y}{(\o,\i)}$, i.e., ones that return $Y$-typed
values, while possibly issuing signals specified by $\o$ and handling interrupts specified by $\i$.
The \emph{effect annotations} $\o$ and $\i$ are drawn from sets $O$ and $I$ whose definitions we discuss
in \autoref{sec:basic-calculus:effect-annotations}. The \lambdaAEff-specific \emph{promise type}
$\typromise{X}$ classifies promises that can be fulfilled by supplying a value of type $X$.


\subsubsection{Effect Annotations}
\label{sec:basic-calculus:effect-annotations}

We now explain how we define the sets $O$ and $I$ from which we draw the
effect annotations we use for specifying functions and computations.
%
Traditionally, effect systems for algebraic effects simply use (flat) sets of
operation names for effect annotations \cite{Bauer:EffectSystem,Kammar:Handlers}.
In \lambdaAEff, however, we need to be
more careful, because triggering an interrupt handler executes a computation
that can issue potentially different signals and handle different interrupts from the main
program, and we would like to capture this in types.

\paragraph{Signal Annotations}
First, as outgoing signals do not carry any computational data, we follow
the tradition of type-and-effect systems for algebraic effects, and define
$O$ to be the \emph{power set} $\Pow \sig$. As such, each $\o \in O$ is a subset of
the signature $\Sigma$, specifying which signals a computation might issue (this is
an over-approximation of the actually issued signals).

\paragraph{Interrupt Handler Annotations}
As observed above, for specifying installed interrupt handlers, we cannot use (flat) sets
of interrupt names as the effect annotations $\i \in I$ if we want to track the nested
(and sometimes recursive) effectful structure of interrupt handlers.

Instead, intuitively each $\i \in I$ is a
\emph{possibly infinite} nesting of partial mappings of pairs of $O$- and $I$-annotations to names in
$\sig$---these pairs of annotations classify the possible effects of the corresponding interrupt handler code.
We use the
record notation
\[
\i = \{ \op_1 \mapsto (\o_1,\i_1) , \ldots , \op_n \mapsto (\o_n,\i_n) \}
\]
to mean that $\i$ maps $\op_1, \ldots, \op_n$ to the annotations $(\o_1,\i_1), \ldots, (\o_n,\i_n)$,
while any other names in $\sig$ are unannotated, corresponding to no interrupt handlers being installed for
these other names. We write $\i\, (\op_i) = (\o_i,\i_i)$ to mean that the annotation
$\i$ maps $\op_i$ to $(\o_i,\i_i)$.

Formally,
we define $I$ as the \emph{greatest fixed point}
of a set functor $\Phi$, given by
\[
\Phi (X) \defeq \sig \Rightarrow (O \times X)_\bot
\]
where $\Rightarrow$ is exponentiation, $\times$ is Cartesian product,
and $(-)_\bot$ is the lifting operation, which we use to represent unannotated names, and which is defined
using the disjoint union as $(-) \cupdot \{\bot\}$. Formally speaking, $I$ is given
by an isomorphism $I \cong \Phi(I)$, but for presentation purposes we leave it
implicit and work as if we had a strict equality $I = \Phi(I)$.


\paragraph{Subtyping and Recursive Effect Annotations}
Both $O$ and $I$ come equipped with natural \emph{partial orders}: for $O$, $\order O$ is given simply by
subset inclusion; and for $I$, the pointwise order~$\order I$ is characterised as follows:
\[
\begin{array}{l c l}
\i \order I \i'
&
\text{iff}
&
\forall\, (\op \in \sig) \, (\o'' \in O) \, (\i'' \in I) .\, \i\, (\op) = ({\o''} , {\i''}) \implies
\\[0.5ex]
&& \exists\, (\o''' \in O) \, (\i''' \in I) .\, \i'\, (\op) = ({\o'''} , {\i'''}) \wedge \o'' \order O \o''' \wedge \i'' \order I \i'''
\end{array}
\]
%
We also use the \emph{product order} $\order {O \times I}$, defined as
$(\o,\i) \order {O \times I} (\o',\i') \defeq \o \order O \o' \wedge \i \order I \i'$.
%
In particular, we use $\order {O \times I}$ to define the subtyping
relation for \lambdaAEff's computation types.

Furthermore, both $O$ and $I$ carry a \emph{join-semilattice} structure, where
$\o \sqcup \o' \in O$ is given simply by the union of sets, while
$\i \sqcup \i' \in I$ is given pointwise as follows:
\[
(\i \sqcup \i')(\op)
~\defeq~
\begin{cases}
(\o'' \sqcup \o''' , \i'' \sqcup \i''') & \mbox{if } \i\, (\op) = (\o'',\i'') \wedge \i'\, (\op) = (\o''',\i''') \\
(\o'' , \i'') & \mbox{if } \i\, (\op) = (\o'',\i'') \wedge \i'\, (\op) = \bot \\
(\o''' , \i''') & \mbox{if } \i\, (\op) = \bot \wedge \i'\, (\op) = (\o''',\i''') \\
\bot & \mbox{if } \i\, (\op) = \bot \wedge \i'\, (\op) = \bot \\
\end{cases}
\]

Importantly, the partial orders $(O,\order O)$ and $(I,\order I)$ are both \emph{$\omega$-complete} and \emph{pointed}, i.e.,
they form \emph{pointed $\omega$-cpos}, meaning that they have least upper bounds of all increasing $\omega$-chains, and
least elements (given by the empty set $\emptyset$ and the constant $\bot$-valued mapping, respectively).
%
As a consequence, and as is well-known, \emph{least fixed points} of continuous (endo)maps on them are then guaranteed
to exist~\cite{Amadio:Domains, Gierz:ContinuousLattices}.
%
For \lambdaAEff, we are particularly interested in the least fixed points of continuous maps $f : I \to I$,
so as to specify and typecheck code examples involving reinstallable interrupt handlers, as we illustrate in
\autoref{sec:extensions:reinstallable-interrupt-handlers}.
%

We also note that if we were only interested in the type safety of \lambdaAEff, and not
in typechecking reinstallable interrupt handler examples, then we would not need $(I,\order I)$ to be \emph{$\omega$-complete},
and could have instead chosen $I$ to be the
\emph{least fixed point} of the set functor $\Phi$ defined earlier, which is what we do for simplicity in our \pl{Agda}
formalisation. In this case, each interrupt handler annotation $\i \in I$ would be a \emph{finite nesting of partial mappings}.

Finally, we envisage that any future full-fledged high-level language based on \lambdaAEff~would
allow users to define their (recursive) effect annotations in a small domain-specific language, providing
a syntactic counterpart to the domain-theoretic development we use in this paper.

\paragraph{Interrupt Actions}

We mimic the act of triggering an interrupt handler for some interrupt $\op$ on an effect annotation $(\o, \i)$ through an
\emph{action} defined as follows:
\[
\opincomp {op} {(\o , \i)}
~\defeq~
  \begin{cases}
   \left(\o \sqcup \o' , \i[\op \mapsto \bot] \sqcup \i' \right) & \mbox{if } \i\, (\op) = (\o',\i')\\
   (\o,\i) & \mbox{otherwise}
  \end{cases}
\]
If $(\o, \i)$ lists any interrupt handlers installed for $\op$, then $\i\, (\op) = (\o',\i')$,
where $(\o',\i')$ specifies the effects of said handler code. Now, when the inward propagating
interrupt reaches those interrupt handlers, it triggers the execution of the corresponding handler code,
and thus the entire interrupted computation can also issue signals in $\o'$ and handle interrupts in $\i'$.

The notation $\i[\op \mapsto \bot]$ sets $\i$ to $\bot$ at $\op$,
and leaves it unchanged elsewhere.
%
Mapping $\op$ to $\bot$ in the definition of $\tmkw{\downarrow}$ captures that the incoming interrupt triggers all the corresponding interrupt handlers that are installed in the computation that $\op$ is propagated to.

\subsubsection{Typing Rules}
\label{sect:typing-rules}

We characterise \emph{well-typed values} using the judgement $\Gamma \types V : X$
and \emph{well-typed computations} using the judgement $\Gamma \types M : \tycomp{X}{(\o,\i)}$.
In both judgements, $\Gamma$ is a \emph{typing context}.
The rules defining these judgements are respectively given in \autoref{fig:value-typing-rules} and
\ref{fig:computation-typing-rules}.

% Figure environment removed

% Figure environment removed

\paragraph{Values}

The rules for values are mostly standard.
The only \lambdaAEff-specific rule is \textsc{TyVal-Promise}, which states that in order to fulfil
a \emph{promise} of type $\typromise X$, one has to supply a value of type $X$. In the rule \textsc{TyVal-Var}, we emphasise the position of the variable in the context, as it will become important once we extend the calculus with modal types in \autoref{sec:extensions:fitch-style-modal-types}.

\paragraph{Computations}

Analogously to values, the typing rules are standard for computation terms that \lambdaAEff~inherits from FGCBV,
with the \lambdaAEff-rules additionally tracking effect information.

The first \lambdaAEff-specific typing rule \textsc{TyComp-Signal} states that in order
to issue a signal $\op$ in a computation that has type $\tycomp{X}{(\o,\i)}$, we must have $\op \in \o$ and the type of
the payload value has to match $\op$'s signature $\op : A_\op$.

The rule \textsc{TyComp-Interrupt} is used to type incoming interrupts.
In particular, when the outside world propagates an interrupt $\op$ to a computation
$M$ of type $\tycomp{X}{(\o,\i)}$, the resulting
computation $\tmopin{op}{V}{M}$ gets assigned the type $\tycomp{X}{\opincomp {op} (\o,\i)}$,
where the action $\opincomp {op} (\o,\i)$ of the interrupt $\op$ on the annotation $(\o, \i)$ is given as
discussed in \autoref{sec:basic-calculus:effect-annotations}.

The rule \textsc{TyComp-Promise} states that
the interrupt handler code $M$ has to return a fulfilled promise of type $\typromise X$, for some type $X$,
while possibly issuing signals $\o'$ and handling interrupts $\i'$, both of which are
determined by the effect annotation $\i$ of the entire computation, as
$(\o',\i') = \i\, (\op)$. The variable $p$ bound in the continuation, which sub-computations can block on
to await $\op$ to arrive and be handled, also gets assigned
the promise type $\typromise X$.

It is worth noting that we could have had $M$ simply
return values of type $X$, but at the cost of not being able to implement some of the more interesting examples,
such as the guarded interrupt handlers defined in \autoref{sec:applications:guarded-handlers}.
At the same time, for \lambdaAEff's type safety, it is
crucial that $p$ would have remained assigned the distinguished promise type $\typromise X$.

The rule \textsc{TyComp-Await} simply states that when awaiting a promise of type $\typromise X$ to be fulfilled,
the continuation~$M$ can refer to the promised value using the variable $x$ of type $X$.

Finally, the rule \textsc{TyComp-Subsume} allows \emph{subtyping}.
To simplify the presentation, we consider a limited form of subtyping, in which we
shallowly relate only effect annotations.

\subsection{Type Safety}
\label{sec:basic-calculus:type-safety}

The sequential part of \lambdaAEff~satisfies the expected type safety properties
ensuring that ``well-typed programs do not go wrong''. We split these safety properties into the usual
\emph{progress} and \emph{preservation} theorems \cite{Wright:SynAppTypeSoundness}.
We omit their proofs~\cite{Ahman:POPL} from this summary, and revisit them
in \autoref{sec:type-safety} for the extended version of \lambdaAEff,
as the proofs for the extended calculus also apply to the version summarised in this section.

The progress result states that well-typed (and sufficiently) closed computations can either make another step of
reduction, or they are already in a well-defined result form (and thus have correctly stopped reducing).
As such, we first need to define when we consider \lambdaAEff-computations
to be in \emph{result form} (commonly also called a normal form). We do so using the
judgements $\CompResult {\Psi} {M}$,
which states that $M$ has reached its final form as an isolated computation term,
and $\RunResult {\Psi} {M}$, which states that $M$ has reached the final form of a
computation running inside a process with all its signals already having been propagated to
other parallel processes (described in more detail in \autoref{sec:basic-calculus:type-safety:processes}):
\begin{mathpar}
  \coopinfer{}{
    \CompResult {\Psi} {M}
  }{
    \CompResult {\Psi} {\tmopout {op} V M}
  }
  \and
  \coopinfer{}{
    \RunResult {\Psi} {M}
  }{
    \CompResult {\Psi} {M}
  }
  \and
  \coopinfer{}{
  }{
    \RunResult {\Psi} {\tmreturn V}
  }
  \and
  \coopinfer{}{
    \RunResult {\Psi \sqcup \{p\}} {N}
  }{
    \RunResult {\Psi} {\tmwith {op} x M p N}
  }
  \and
  \coopinfer{}{
    p \in \Psi
  }{
    \RunResult {\Psi} {\tmawait p x M}
  }
\end{mathpar}
In these judgements, $\Psi$ is a set
of (promise-typed) variables $p$ that have been bound by interrupt handlers enveloping the given computation.
Intuitively, these judgements express that a computation $M$ is in a (top-level)
result form $\CompResult {\Psi} {M}$ when, considered as a tree, it has a shape in which \emph{all}
signals are towards the root, interrupt handlers are in the intermediate nodes, and
the leaves contain return values and computations that are temporarily blocked
while awaiting one of the promise-typed variables $p$ in $\Psi$ to be fulfilled.

The new reduction rules that propagate the awaiting construct out of sequencing and interrupts into the awaiting construct ensure the explicit form of all blocking computations and considerably simplify the definition of $\RunResult {\Psi} {M}$ compared to the previous version of our work~\cite{Ahman:POPL}.
%
The finality of these result forms is captured by the next lemma.

\begin{lem}
\label{lem:results-are-final}
Given $\Psi$ and $M$, such that $\CompResult {\Psi} {M}$, then there is no $N$ with $M \reduces N$.
\end{lem}

Using the result forms, the progress theorem for the sequential part of \lambdaAEff\ is as follows:

\begin{thm}[Progress for computations]
\label{thm:progress}
Given a well-typed computation
\[
  p_1 \of \typromise {X_1}, \ldots, p_n \of \typromise {X_n} \types M : \tycomp{Y}{(\o,\i)}
\]
then either
\begin{enumerate}[(a)]
  \item there exists a computation $N$, such that $M \reduces N$, or
  \item the computation $M$ is in a result form, i.e., we have $\CompResult {\{p_1, \ldots, p_n\}} {M}$.
\end{enumerate}
\end{thm}

In particular, with the empty context, we get the usual progress statement, which states that
$\types M : \tycomp{X}{(\o, \i)}$ implies that either $M \reduces N$ for some $N$
or that $\CompResult {\emptyset} {M}$ holds. This implies that any promise variable
which we are awaiting to be fulfilled must correspond to one of the installed interrupt handlers.
Additionally, the type system ensures that all outgoing signals are listed in $\o$ and all
installed interrupt handlers are specified in $\i$.

The type preservation result is standard and says that reduction preserves well-typedness.

\begin{thm}[Preservation for computations]
\label{thm:preservation}
Given a well-typed computation $\Gamma \types M : \tycomp{X}{(\o,\i)}$, such that $M$
can reduce as $M \reduces N$, then we have $\Gamma \types N : \tycomp{X}{(\o,\i)}$.
\end{thm}

% !TEX root = paper.tex

\section{A Calculus for Asynchronous Effects: Parallel Processes}
\label{sec:basic-calculus:processes}

We now describe the parallel part of \lambdaAEff. Similarly to the sequential part, we 
present the corresponding syntax, small-step semantics, 
type-and-effect system, and type safety results.

\subsection{Parallel Processes}

To keep the presentation focussed on the asynchronous use of algebraic effects, we 
consider a very simple model of parallelism: a process is either an \emph{individual computation} 
or the \emph{parallel composition} of two processes. To facilitate interactions, processes also  
contain outward propagating \emph{signals} and inward propagating \emph{interrupts}. 

In detail, the syntax of \emph{parallel processes} is given by the following grammar:
\[
  P, Q
  \bnfis \tmrun M
  \,\bnfor\! \tmpar P Q
  \,\bnfor\! \tmopout{op}{V}{P}
  \,\bnfor\! \tmopin{op}{V}{P}
\]
%
Note that processes do not include interrupt handlers---these are local to computations.

Here the number and hierarchy of processes running in parallel is fixed---a limitation that we address
in \autoref{sec:extensions:dynamic-process-creation} by introducing a means to dynamically create new processes.

\subsection{Small-Step Operational Semantics}

We equip the parallel part of \lambdaAEff~with a small-step operational semantics that  
naturally extends the semantics of \lambdaAEff's sequential part from \autoref{sec:basic-calculus:semantics:computations}.
The semantics is defined using a reduction relation $P \reduces Q$, as given in \autoref{fig:processes}.

% Figure environment removed

\paragraph{Individual Computations}
This rule states that, as processes, individual computations evolve according to the small-step
operational semantics $M \reduces N$ we defined in \autoref{sec:basic-calculus:semantics:computations}.

\paragraph{Signal Hoisting}
This rule propagates signals out of individual computations.
Note that we only hoist those signals that have propagated to the outer boundary
of a computation.

\paragraph{Broadcasting}
These rules turn outward moving signals in one process into inward moving interrupts 
for the process parallel to it, while continuing to propagate the signals outwards to any 
further parallel processes. The latter ensures that the semantics is compositional.

\paragraph{Interrupt Propagation}
These three rules simply propagate interrupts inwards into individual computations, 
into all branches of parallel compositions, and past any issued signals.

\paragraph{Evaluation Contexts}
Analogously to the semantics of computations, the semantics of processes presented here also 
includes an evaluation context rule, which allows reductions under \emph{evaluation contexts} 
$\F$. Observe that compared to the evaluation contexts for computations, those for processes
are more standard, in the sense that they do not bind variables. 

\subsection{Type-and-Effect System}
\label{sec:basic-calculus:processes:type-and-effect-system}

Analogously to its sequential part, we also equip \lambdaAEff's parallel part with a type-and-effect system.

\paragraph{Types} The \emph{process types} are designed to match their parallel structure, and 
are given by
\[
  \text{$\tyC$, $\tyD$}
  \bnfis \tyrun X \o \i
  \bnfor \typar \tyC \tyD
\]
Namely, $\tyrun X \o \i$ is a type of an individual computation of type $\tycomp{X}{(\o,\i)}$, and $\typar \tyC \tyD$
is the type of the parallel composition of two processes that respectively have types $\tyC$ and $\tyD$.

\paragraph{Typing Judgements}
\emph{Well-typed processes} are characterised using the judgement
$\Gamma \vdash P : \tyC$. The typing rules are given in \autoref{fig:process-typing-rules}.
While our processes are not currently higher-order, we allow 
non-empty contexts $\Gamma$ to model using libraries and top-level function definitions.

% Figure environment removed

The rules \textsc{TyProc-Run} and \textsc{TyProc-Par} capture the earlier 
intuition about the types of processes matching their parallel structure. The rules 
\textsc{TyProc-Signal} and \textsc{TyProc-Interrupt} are similar to the corresponding 
computation typing rules from \autoref{fig:computation-typing-rules}.

The \emph{signal annotations} of a process type used in \textsc{TyProc-Signal} are calculated as
\[
\mathsf{signals\text{-}of}(\tyrun{X}{\o}{\i}) ~\defeq~ \o
\qquad\qquad
\mathsf{signals\text{-}of}(\typar{\tyC}{\tyD}) ~\defeq~ \mathsf{signals\text{-}of}(\tyC) \sqcup \mathsf{signals\text{-}of}(\tyD)
\]
and the \emph{action of interrupts} on process types extends the action on effect annotations as
\[
\opincomp{op}(\tyrun{X}{\o}{\i}) 
~\defeq~
X \att (\opincomp {op} {(\o , \i)})
\qquad\qquad
\opincomp{op}(\typar{\tyC}{\tyD}) 
~\defeq~
\typar{(\opincomp{op}{\tyC})}{(\opincomp{op}{\tyD})}
\]
by propagating the interrupt towards the types of individual computations. 

It is worth noting that \autoref{fig:process-typing-rules} does not include an analogue  
of the computation subtyping rule \textsc{TyComp-Subsume}. This choice is 
deliberate because as we shall see below, \emph{process types reduce}
in conjunction with the processes they are assigned to, and the outcome   
of process type reduction is generally neither a sub- nor supertype of the original type.


\subsection{Type Safety}
\label{sec:basic-calculus:type-safety:processes}

We conclude summarising the meta-theory of \lambdaAEff~by stating the type safety 
of its parallel part. Analogously to \autoref{sec:basic-calculus:type-safety}, 
we once again split type safety into separate \emph{progress} 
and \emph{preservation} results, and relegate their proofs to \autoref{sec:type-safety}.

We characterise the \emph{result forms} of processes 
by defining two judgements, $\ProcResult P$ and $\ParResult P$, 
and by using the judgement $\RunResult {\Psi} {M}$ from 
\autoref{sec:basic-calculus:type-safety}, as follows:
\begin{mathpar}
  \coopinfer{}{
    \ProcResult {P}
  }{
    \ProcResult {\tmopout {op} V P}
  }
  \qquad
  \coopinfer{}{
    \ParResult {P}
  }{
    \ProcResult {P}
  }
  \qquad
  \coopinfer{}{
    \RunResult {\emptyset} {M}
  }{
    \ParResult {\tmrun M}
  }
  \qquad
  \coopinfer{}{
    \ParResult P \\
    \ParResult Q
  }{
    \ParResult {\tmpar P Q}
  }
\end{mathpar}
%
These judgements express that a process $P$ is in a (top-level) 
result form $\ProcResult {P}$ when, considered as a tree, it has a shape in which 
\emph{all} signals are towards the root, parallel compositions are in 
the intermediate nodes, and individual computation results are at the leaves. 
Importantly, the computation results $\RunResult {\emptyset} {M}$ we use in this definition 
are those from which all signals have been propagated out of 
(as discussed in \autoref{sec:basic-calculus:type-safety}). 

Again, these result forms are operationally final, as captured by the next lemma.

\begin{lem}
\label{lem:results-are-final:processes}
Given a process $P$, such that $\ProcResult {P}$, then there is no $Q$ such that $P \reduces Q$.
\end{lem}

We are now ready to state the progress theorem for the parallel part of \lambdaAEff,
which applies to closed processes and takes the expected form:

\begin{thm}[Progress for processes]
  \label{thm:procprogress}
  Given a well-typed process $\types P : \tyC$, then either
  \begin{enumerate}[(b)]
    \item there exists a process $Q$, such that $P \reduces Q$, or
    \item the process $P$ is already in a (top-level) result form, i.e., we have $\ProcResult {P}$.
  \end{enumerate}
\end{thm}

The preservation theorem for processes that we state below is somewhat non-standard since term reductions 
also evolve effect annotations. In particular, the broadcast rule
\[
  \tmpar{\tmopout{op}{V}{P}}{Q} \reduces \tmopout{op}{V}{\tmpar{P}{\tmopin{op}{V}{Q}}}
\]
and its symmetric counterpart from \autoref{fig:processes} introduce new 
inward propagating interrupts in their right-hand sides that originally do not exist in their left-hand sides. As a result, 
compared to the types one assigns to the left-hand sides of these reduction rules, the types assigned to 
their right-hand sides will need to feature corresponding type-level actions of these interrupts.
We formalise this idea using a \emph{process type reduction} relation $\tyC \tyreduces \tyD$:
\[
  \coopinfer{}{
  }{
    \tyrun{X}{\o}{\i} \tyreduces \tyrun{X}{\o}{\i} 
  }
  \quad
  \coopinfer{}{
  }{
    X \att \opincompp {ops} {(\o , \i)} \tyreduces X \att \opincompp {ops} {(\opincomp {op} {(\o , \i)})}
  }
  \quad
  \coopinfer{}{
    \tyC \tyreduces \tyC' \\
    \tyD \tyreduces \tyD'
  }{
    \typar{\tyC}{\tyD} \tyreduces \typar{\tyC'}{\tyD'}
  }
\]
where we write $\opincompp {ops} {(\o , \i)}$ for a recursively defined \emph{action of a list of interrupts} on $(\o , \i)$:
\[
\opincompp {[]} {(\o , \i)} ~\defeq~ (\o , \i)
\qquad
\opincompp {(\op :: \opsym{ops})} {(\o , \i)} ~\defeq~ \opincomp {op} {(\opincompp {ops} (\o , \i))}
\]

Intuitively, $\tyC \tyreduces \tyD$ describes how process types reduce by being acted upon by 
freshly arriving interrupts.
It is important that we introduce interrupts under an arbitrary enveloping sequence of interrupt actions, 
and not simply as $X \att {(\o , \i)} \tyreduces X \att (\opincomp {op} {(\o , \i)})$,
because we want to ensure that these actions preserve type reductions (see~\srefcase{Lemma}{lem:type-reduction}{3}),
which in turn ensures type preservation of reductions under arbitrary evaluation contexts~$\F$.

Using the process type reduction relation, we state the preservation theorem for the parallel part of 
\lambdaAEff~as follows:

\begin{thm}[Preservation for processes]
  \label{thm:procpreservation}
  Given a well-typed process $\Gamma \types P : \tyC$, such that $P$ can reduce as 
  $P \reduces Q$, then there exists a process type $\tyD$, such 
  that the process type $\tyC$ can reduce as $\tyC \tyreduces \tyD$, 
  and we can type the resulting process as $\Gamma \types Q : \tyD$.
\end{thm}

%!TEX root = ../main.tex
\section{Extensions and Further Results}\label{app:extensions}


\subsection{Removing the necessity for minimal chunk length via stronger synthesis oracle}\label{sec:no_min_chunk_length}

\begin{theorem}\label{thm:simpler_TVC_thm} Support we replace \Cref{asm:iss_body} in \Cref{prop:TVC_main} with the assumption that our trajectory oracle produces \emph{entire sequences of gains} $\sfk_{1:T}$ which satisfy time-varying incremental stability (\Cref{defn:tiss}) on the whole trajectory. Then, 
\begin{itemize}
\item The conclusion of \Cref{prop:TVC_main} holds
\item we no longer need the condition $\tauc \ge c_3$; taking $\tauc = 1$ suffices.
\item The constants $c_1,c_2$ depend only on $\cgamma$ and $\cbargamma$. That is, $\cxi$ and terms associated with $\betaiss$ can be vaucuously large.  
\end{itemize}
Analogoues, if we replace \Cref{asm:tis} in \Cref{prop:TVC_main_general} with the assumption that $\sfk_{1:T}$ satisfies the time-varying incremental stability condition, then 
\begin{itemize}
\item The conclusion of \Cref{prop:TVC_main_general} holds
\item We no longer need the condition $\tauc \ge c_3$; taking $\tauc = 1$ suffices. Moreover, we can replace the condition \eqref{eq:eps_cond_general} of $\epsilon$ with the simpler condition $\epsilon \le \cgamma$.
\item Lastly, one can replace $ \epsilon_1 = 2\betaiss(2\gammaiss(\epsilon),0) $  in \eqref{eq:TVC_main} with the term $\epsilon_1 = \gammaiss(\epsilon)$. 
\end{itemize}
\end{theorem}
The proof of \Cref{thm:simpler_TVC_thm} follows by replacing \Cref{lem:iss_ips} with the following simpler lemma that recapitulates \citet[Proposition 3.1]{pfrommer2022tasil}, and propogating the argument through the proof.
\begin{lemma}\label{lem:iss_simpler_lemma} Consider two consistent trajectories $(\bx_{1:T+1},\bu_{1:T})$ and $(\bx_{1:T+1}',\bu_{1:T}')$, as well as sequences of primitive controller $\sfk_{1:T},\sfk_{1:T}'$,  such that $\bx_1 = \bx_1'$, and $\bu_t = \sfk_t(\bx_t)$, $\bu_t' = \sfk_t'(\bx_t')$. Suppose that 
\begin{align}
\max_{t}\sup_{\bx:\|\bx - \bx_t\| \le \gammaiss(\epsilon)} \|\sfk_t(\bx)-\sfk_t'(\bx_t)\| \le \epsilon.
\end{align}
Then, $\max_t \|\bu_t - \bu_t'\| \le \epsilon$ and $\max_t \|\bx_t - \bx_t'\| \le \gammaiss(\epsilon)$
\end{lemma}


\subsection{Noisy Dynamics}\label{ssec:noisy_dynamics}


\newcommand{\seqw}{\mathsf{w}}
\newcommand{\Pnoiseh}[1][h]{\distfont{P}_{\mathrm{noise},#1}}
\newcommand{\Fnoise}[1][h]{F^{\mathrm{noise}}}

We can directly extend our imitation guarantees in the composite MDP to settings with noise:
\begin{align}
\seqs_{h+1} \sim \Fnoise_h(\seqs_h,\seqa_h,\seqw_h), \quad \seqw_h \sim \Pnoiseh, \label{eq:with_noise}
\end{align}
where the noises are idependent of states and of each other. Indeed, \eqref{eq:with_noise} can be directly reduced to the no-noise setting by lifting ``actions'' to pairs $(\seqa_h,\seqw_h)$, and policies $\pi$ to encompass their distribution of actions, and over noise. 

Another approach is instead to condition on the noises $\seqw_{1:H}$ first, and treat the noise-conditioned dynamics as deterministic. Then one can take expectation over the noises and conclude. The advantage of this approach is that the couplings constructed thereby is that the trajectories experience identical sequences of noise with probability one. 

Extending the control setting to incorporate noise is doable but requires more effort:
\begin{itemize}
	\item If the \emph{demonstrations are noiseless}, then one can still appeal to the synthesis oracle to synthesis stabilizing gains. However, one needs to (ever so slightly) generalize the proofs of the various stability properties (e.g. IPS in \Cref{prop:ips_instant}) to accomodate system noise. 
	\item If the demonstrations themselves have noise, one may need to modify the synthesis oracle setup somewhat. This is because the synthesis oracle, if it synthesizes stabilizing gains, will attempt to get the learner to stabilize to a noise-perturbed trajectory. This can perhaps be modified by synthesizing controllers which stabilize to smoothed trajectories, or by collecting demonstrations of desired trajectories (e.g. position control), and stabilizing to the these states than than to actual states visited in demonstrations. 
\end{itemize}



\subsection{Robustness to Adversarial Perturbations}
\newcommand{\seqe}{\mathsf{e}}
\newcommand{\Fadv}{F^{\mathtt{adv}}}
\newcommand{\piadv}{\pi^{\mathtt{adv}}}

Our results can accomodate an even more general framework where there are both noises as well  adversarial perturbations. We explain this generalization in the composite MDP. 

Specifical, consider a space $\cE$ of adversarial perturbations, as well as $\cW$ of noises as above. We may posite a dynamics function $\Fadv: \cS \times \cA \times \cW \times \cA \to \cS$, and consider the evolution of an imitator policy $\pihat$ under the adversary
\begin{align}
\shat_{h+1} &= \Fadv_h(\shat_h,\seqahat_h,\seqw_h,\seqe_h),  \quad \seqw_h \sim \Pnoiseh\\
&\ahat_h \sim \pihat_h(\seqs_h)\\
&\seqe_h \sim \piadv_h(\shat_{1:h},\seqa_{1:h},\seqw_{1:h},\seqe_{1:h-1}), \\
&\shat_{1} \sim \piadv_0(\seqs_1), \quad \seqs_1 \sim \Dinit.
\end{align}

By constrast, we can model the demonstrator trajectory as arising from noisy, but otherwise unperturbed trajectories:
\begin{align}
\sstar_{h+1} \sim \Fadv_h(\sstar_h,\seqa^\star_h,\seqw_h,0), \quad \seqw_h \sim \Pnoiseh, \quad \seqa_h^\star \sim \pist_h(\sstar_h), \quad \sstar_1 \sim \Dinit. \label{eq:pist_unpert}
\end{align}
To reduce the composite-MDP in \Cref{sec:analysis}, we can view the combination of adverary $\piadv$ and imitator $\pihat$ as a combined policy, and the $\pist$ with zero augmentation as another policy; here, we would them treat actions as $\tilde \seqa = (\seqa,\seqe)$. Then, one can consider modified senses of stability which preserve trajectory tracking, as well as a modification of $\distA$ to a function measuring distances between $\tilde \seqa = (\seqa,\seqe)$ and $\tilde \seqa' = (\seqa',\seqe')$. The extension is rather mechanical, and we fit details. Note further that, by including a $\piadv_0(\seqs_1)$, we can modify the analysis to allow for subtle differences in initial state distribution. This would in turn require strengthening our stability asssumptions to allow stability to initial state (e.g., the definition of incremental stability as exposited by \cite{pfrommer2022tasil}).


\subsection{Deconvolution Policies and Total Variation Continuity}\label{app:deconv_smooth}

While our strongest guarantees hold for the replica policies, where we add noise both as a data augmentation at training time \emph{and} at test time, many practitioners have seen some success with the deconvolution policies where noise is only added at training time.  We note that \Cref{prop:IS_general_body} holds when the learned policy is TVC; without noise at training time this certainly will not hold when the expert policy is not TVC.  We show here that the deconvolution expert policy is TVC under mild assumptions, which lends some credence to the empirical success of deconvolution policies.

Precisely, we show that, under reasonable conditions, deconvolution is total variation continuous.  In particular, suppose that $\mu \in \Delta(\rr^d)$ is a Borel probabilty measure and $p$ is a density with respect to $\mu$.  Further suppose that $Q$ is a density with respect to the Lebesgue measure on $\rr^d$.  Suppose that $\bx \sim p$, $\bw \sim Q$, and let $\xtil = \bx + \bw$.  Denote the deconvolution measure of $\bx$ given $\xtil$ as $p(\cdot | \xtil)$. We show that this measure is continuous in $\tv$.  
\begin{proposition}\label{prop:deconvolutioncontinuity}
    Let $\bx, \bx' \in \rr^d$ be fixed, let $p: \rr^d \to \rr$ denote a probability density, and let $Q: \rr^d \to \rr$ denote a function such that $\nabla^2 Q$ and $\nabla \log Q$ exist and are continuous on the set
    \begin{align}
        \cX  = \left\{ (1- t) \xtil + t \xtil' - x | \bx \in \supp p \text{ and } t \in [0,1] \right\}
    \end{align}
    Then it holds that
    \begin{align}
        \tv\left( p(\cdot | \xtil), p(\cdot | \xtil') \right) \leq \norm{\xtil - \xtil'} \cdot \sup_{\bx \in \cX} \norm{\nabla \log Q(\bx)}.
    \end{align}
\end{proposition}
By \Cref{cor:tv_two}, any policy composed with the total variation kernel is thus total variation continuous with a linear $\gamtvc$; moreover, the Lipschitz constant is given by the maximal norm of the score of the noise distribution.  For example, if $Q$ is the density of a Gaussian with variance $\sigma^2$, then $\gamtvc(u) \leq \frac{\sup_{\cX} \norm{\bx}}{\sigma^2}$ is dimension independent.
\begin{remark}
    Note that our notation is intentionally different from that in the body to emphasize that this is a general fact about abstract probability measures.  We may intantiate the guarantee in the control setting of interest by letting $\bx = \pathm$ and consider $Q$ to be a Gaussian (for example) kernel.  In this case, we see that the deconvolution policy of \Cref{defn:dec_cond} is automatically TVC.
\end{remark}


To prove \Cref{prop:deconvolutioncontinuity}, we begin with the following lemma:
\begin{lemma}\label{lem:gradientposterior}
    Let $\xtil \in \rr^d$ be fixed and suppose that $\nabla \log Q(\xtil - \bx) $ exists for all $\bx \in \supp p$.  Then, for all $\bx \in \supp p$, it holds that $\nabla_{\xtil} p(\bx | \xtil)$ exists.  Furthermore,
    \begin{align}
        \int \norm{\nabla p(\bx | \xtil)} d \mu(\bx) \leq 2\sup_{\bx \in \supp p} \norm{\nabla \log Q(\xtil - \bx)},
    \end{align}
    where the gradient above is with respect to $\xtil$.
\end{lemma}
\begin{proof}
    We begin by noting that if $\nabla \log Q(\xtil - \bx)$ exists, then so does $\nabla Q(\xtil - \bx)$.  By Bayes' rule,
    \begin{align}
        p(\bx | \xtil) = \frac{p(\bx) Q(\xtil - \bx)}{\int Q(\xtil - \bx') p(\bx') d \mu(\bx')}.
    \end{align}
    We can then compute directly that
    \begin{align}
        \nabla p(\bx | \xtil) &= \frac{p(\bx) \nabla Q(\xtil - \bx)}{\int Q(\xtil - \bx')  p(\bx') d \mu(\bx')} - \frac{p(\bx) Q(\xtil - \bx) \cdot \int \nabla Q(\xtil - \bx')  p(\bx') d \mu(\bx')}{\left( \int Q(\xtil - \bx') p(\bx') d \mu(\bx') \right)^2},
    \end{align}
    where the exchange of the gradient and the integral is justified by Lebesgue dominated convergence and the assumption of differentiability of $Q$ and thus existence is ensured.  We have now that
    \begin{align}
        \norm{\nabla p(\bx|\xtil)} &= \frac{p(\bx) Q(\xtil - \bx)}{\int Q(\xtil - \bx') p(\bx') d \mu(\bx')} \cdot \norm{  \nabla \log Q(\xtil - \bx) - \frac{\int \nabla Q(\xtil - \bx') p(\bx') d \mu(\bx')}{\int Q(\xtil - \bx') p(\bx') d \mu(\bx')} } \\
        &= \frac{p(\bx) Q(\xtil - \bx)}{\int Q(\xtil - \bx') p(\bx') d \mu(\bx')} \cdot \norm{  \nabla \log Q(\xtil - \bx) - \frac{\int (\nabla \log Q(\xtil - \bx')) \cdot Q(\xtil - \bx) p(\bx') d \mu(\bx')}{\int Q(\xtil - \bx') p(\bx') d \mu(\bx')} } \\
        &\leq \left( \sup_{\bx \in \supp p} \norm{\nabla \log Q(\xtil - \bx)} \right) \cdot \frac{p(\bx) Q(\xtil - \bx)}{\int Q(\xtil - \bx') p(\bx') d \mu(\bx')} \cdot \left( 1 +  \frac{\int Q(\xtil - \bx) p(\bx') d \mu(\bx')}{\int Q(\xtil - \bx) p(\bx') d \mu(\bx')}\right) \\
        &= \left( 2\sup_{\bx \in \supp p} \norm{\nabla \log Q(\xtil - \bx)} \right) \cdot \frac{p(\bx) Q(\xtil - \bx)}{\int Q(\xtil - \bx') p(\bx') d \mu(\bx')}.
    \end{align}
    Now, integrating over $\bx$ makes the second factor 1, concluding the proof.
\end{proof}
We will now make use of the theory of Dini derivatives (\citep{hagood2006recovering}) to prove a bound on total variation.
\begin{lemma}\label{lem:ftcdini}
    For fixed $\xtil, \xtil'$ and $0 \leq t \leq 1$, let the upper Dini derivative
    \begin{align}
        D^+ \tv(p(\cdot | \xtil), p(\cdot | \xtil_t)) = \limsup_{h \downarrow 0} \frac{\tv(p(\cdot | \xtil), p(\cdot| \xtil_{t + h})) - \tv(p(\cdot | \xtil), p(\cdot | \xtil_t))}{h},
    \end{align}
    where
    \begin{align}
        \xtil_t = (1 -t) \xtil + t \xtil'.
    \end{align}
    If $\nabla \log Q(\xtil_t - \bx)$ exists and is finite for all $\bx \in \supp p$ and $t \in [0,1]$, then
    \begin{align}
        \tv(p(\cdot | \xtil), p(\cdot | \xtil')) \leq \int_0^1 D^+ \tv\left( p(\cdot | \xtil), p(\cdot | \xtil_t) \right) d t. \label{eq:ftcdini}
    \end{align}
\end{lemma}
\begin{proof}
    We compute:
    \begin{align}
        2 \abs{ \tv(p(\cdot | \xtil), p(\cdot| \xtil_{t + h})) - \tv(p(\cdot | \xtil), p(\cdot | \xtil_t)) } &= \abs{\int  \abs{p(\bx | \xtil) - p(\bx| \xtil_{t+h})} - \abs{p(\bx|\xtil) - p(\xtil_{t})} d \mu(\bx)} \\
        &\leq \int \abs{p(\bx | \xtil_{t+h}) - p(\bx | \xtil_t)} d \mu(\bx). \label{eq:tvtriangle}
    \end{align}
    Observe that by the assumption on $Q$ and \Cref{lem:gradientposterior}, $p(\bx | \xtil_t)$ is differentiable and thus continuous in $\xtil_t$.  We therefor see that the function
    \begin{align}
         t \mapsto \tv(p(\cdot | \xtil), p(\cdot | \xtil_t))
    \end{align}
    is continuous as $\xtil_t$ is linear in $t$.  By \citet[Theorem 10]{hagood2006recovering}, \eqref{eq:ftcdini} holds.
\end{proof}
We now bound the Dini derivatives:
\begin{lemma}\label{lem:diniderivativeupperbound}
    Let $\xtil, \xtil' \in \rr^d$ such that for all $t \in [0,1]$it holds that
    \begin{align}
        \sup_{\bx \in \supp p} \abs{\frac{d^2}{d t^2}\left( p(\bx | \xtil_t) \right)} = C < \infty,
    \end{align}
    where the derivative is applied on $\xtil_t$.  If the assumptions of \Cref{lem:gradientposterior,lem:diniderivativeupperbound} hold, then
    \begin{align}
        D^+ \tv(p(\cdot | \xtil), p(\cdot | \xtil_t)) \leq \norm{\xtil - \xtil'} \cdot \sup_{\substack{\bx \in \supp p \\ t \in [0,1]}} \norm{\nabla \log Q(\xtil_t - x)}.
    \end{align}
\end{lemma}
\begin{proof}
    By definition,
    \begin{align}
        D^+ \tv(p(\cdot | \xtil), p(\cdot | \xtil_t)) = \limsup_{h \downarrow 0} \frac{\tv(p(\cdot | \xtil), p(\cdot| \xtil_{t + h})) - \tv(p(\cdot | \xtil), p(\cdot | \xtil_t))}{h}.
    \end{align}
    Fix some $t$ and some small $h$.  By \eqref{eq:tvtriangle}, it holds that
    \begin{align}
        \abs{\tv(p(\cdot | \xtil), p(\cdot| \xtil_{t + h})) - \tv(p(\cdot | \xtil), p(\cdot | \xtil_t))} \leq \frac 12 \cdot \int \abs{p(\bx | \xtil_{t + h}) - p(\bx | \xtil_t)} d \mu(\bx).
    \end{align}
    By Taylor's theorem, it holds that
    \begin{align}
        p(\bx| \xtil_{t+h}) - p(\bx|\xtil_t) = h \cdot \frac{d}{d t}\left( p(\bx|\xtil_t) \right) + h^2 \cdot \frac{d^2}{dt^2}\left( p(\bx|\xtil_{t'}) \right)
    \end{align}
    for some $t' \in [0,1]$.  By the chain rule, we have
    \begin{align}
        \frac{d}{d t}\left( p(\bx|\xtil_t) \right)  &= \inprod{\xtil' - \xtil}{\nabla p(\bx | \xtil_t)},
    \end{align}
    and thus,
    \begin{align}
        \abs{p(\bx | \xtil_{t + h}) - p(\bx | \xtil_t)}  \leq h \cdot \norm{\xtil - \xtil'} \cdot \norm{\nabla p(\bx | \xtil_t)} + h^2 C
    \end{align}
    Now, applying \Cref{lem:gradientposterior} and plugging into the previous computation concludes the proof.
\end{proof}
We are finally ready to state and prove our main result:

\begin{proof}[Proof of \Cref{prop:deconvolutioncontinuity}]
    Note that
    \begin{align}
        \frac{d^2}{d t^2}\left( p(\bx | \xtil_t) \right) = \left( \xtil - \xtil' \right)^T \nabla^2 p(\bx | \xtil_t) (\xtil - \xtil')
    \end{align}
    and thus is bounded if and only if $\nabla^2 p(\bx | \xtil_t)$ is bounded.  An elementary computation shows that if $\nabla^2 Q$ exists and is continuous on $\cX$, then $\nabla^2 p(\bx | \xtil_t)$ is bounded in operator norm on $\cX$.  Thus the assumption in \Cref{lem:diniderivativeupperbound} holds.  Applying \Cref{lem:ftcdini} then concludes the proof.
\end{proof}


In this section we explore a few applications of the techniques
introduced in section~\ref{sec:mth}. First we consider the application
of the reweighting technique to an optimization problem. 
Second, we consider the application in Bayesian inference to obtain
the dependence of predictions on the parameters that characterize the
prior distribution.

\subsection{Applications in optimization}
\label{sec:opt}

As an example application of an optimization process we will consider
the probability density function
\begin{equation}
  p_\theta(x) = \frac{1}{\mathcal Z}\exp \left\{ -S(x;\theta) \right\}\,, \qquad \left(  \mathcal Z = \int {\rm d} x\, e^{-S(x:\theta)} \right) \,.
\end{equation}
with
\begin{equation}
  S(x; \theta) = \frac{1}{\theta_1^2+1} \left( x_1^2 + x_1^4 \right) + \frac{1}{2}x_2^2 + \theta_2 x_1x_2\,.
\end{equation}

The shape of $S(x;\theta)$ is inspired in the action of a quantum
field theory in zero dimensions, where $x_1$ and $x_2$ are two fields
with coupling $\theta_2$, while $\theta_1$ is related to the mass of
the field $x_1$.
Expectation values with respect to $p_\theta(x)$ are functions of
the parameters $\theta$.  

% Figure environment removed

As an example we consider the problem of minimizing $\mathbb
E_\theta[x_1^2 + x_2^2]$ (i.e. 
finding the values for $\theta$ that make $\mathbb
E_\theta[x_1^2+x_2^2]$ minimum). 
We have implemented two flavours of Stochastic Gradient Descent (SGD): the first -basic- one, having a constant learning rate, and the second one being the well-known
%both a basic stochastic gradient descent (with constant learning rate) and the
ADAM algorithm \cite{kingma2017adam}. It is worth noting at this
point that as a general concept, SGD implies a stochastic (but
unbiased) evaluation of the gradients of the objective function at
every iteration. While in typical applications in the ML community,
where the task is to fit some dataset, this is done by evaluating the
gradients at different random batches of the data, the present example
is different in that no data is involved. In this case, every
iteration of the SGD evaluates the gradients on the different Monte
Carlo samples used to approximate the objective function  $\mathbb
E_\theta[x_1^2 + x_2^2]$.  

%These algorithms require stochastic evaluations both of the functionand its gradient at arbitrary values of the parameters $\theta$. 
%Here we perform these evaluations via Monte Carlo sampling: we use a
Here we consider a simple implementation of the Metropolis Hastings algorithm in order to
first produce the samples $\{x^{\alpha}\}_{\alpha=1}^N \sim p_\theta(x)$. 
Second, we determine the reweighted expectation value truncated at
first order
\begin{equation}
  \frac{\sum w(x^{\alpha};\tilde\theta) \left[ (x^{\alpha}_1)^2+ (x_2^{\alpha})^2 \right]}{\sum w(x^{\alpha};\tilde \theta)} \approx \bar O + \bar O_i \epsilon_i\,,  
  \qquad \left( w(x^{\alpha};\theta) = e^{S(x^{\alpha};\theta) - S(x^{\alpha};\tilde \theta)}  \right)\,,
\end{equation}
where $\tilde \theta_i =  \theta_i + \epsilon_i$. 
This quantity gives an stochastic estimate of the function value
\begin{equation}
  \bar O = \frac{1}{N}\sum_{i=1}^N [x_1^{\alpha}]^2 + [x_2^{\alpha}]^2\,,
\end{equation}
and its derivatives
\begin{equation}
 \bar O_i \approx \frac{\partial \mathbb E_\theta[x_1^2+x_2^2]}{\partial \theta_i}\,.
\end{equation}

Figure~\ref{fig:sgd} shows the result of the optimization process. 
As the iteration count increases the function is driven to its minima,
while the values of the parameters approach the optimal values
$\theta_1^{\rm opt} = \theta_2^{\rm opt} = 0$. 

It is worth mentioning that in this particular example only $1000$ samples
were used at each step to estimate the loss function and its
derivatives. 
If one decides to use a larger number of samples (say $10^5$), the
value of the parameter $\theta_2$ is determined with a much better precision. 
Note that the direction associated
with $\theta_2$ is much flatter, and therefore its value affects much
less value of the loss function.

\subsection{An application in Bayesian inference}
\label{sec:bayesian}
The purpose of statistical inference is to determine properties of the
underlying statistical distribution of a dataset
$D=\{x_{i},y_{i}\}_{i=1}^{N}$. In many
  cases, the independent variables $x_i$ are fixed, and all the
  stochasticity is captured by the dependent variables $y_i$. As
  such,  
the data is assumed to be sampled from a certain model, specified by
the \textit{likelihood}, 
$p(y|x,\phi)$, which depends on a set of parameters $\phi$.
The Bayesian paradigm attributes a level of confidence to the model by
introducing the \textit{prior} 
$p_{\theta}(\phi)$, \textit{i.e.} an a priori distribution of the
models parameters, where in this context $\theta$ play the role of the
hyper-parameters specifying the prior. Following Bayes' rule, the
\textit{posterior} distribution $p_{\theta}(\phi|D)$ is computed
as\footnote{The normalization factor, 
  $p_{\theta}(D)$, called the evidence, or marginal likelihood, is $\phi$-independent
  and represents the probability distribution of the observed data, given the model.}:
\begin{equation}
  \label{eq:bayes}
  p_{\theta}(\phi|D) \propto p(D|\phi) p_{\theta}(\phi)~.
%  ~~~~
%  p(D|\phi) = \prod_{i=1}^N p(y_i|x_i,\phi)~
\end{equation}
The likelihood of the whole dataset, $p(D|\phi)$, is computed assuming independent data points following a Gaussian distribution:
\begin{equation}
  p(D|\phi) = \prod_{i=1}^{N}\mathcal N(y_{i}|f(x_{i};\phi),\sigma_{i})\,,
  \label{eq:likelihood}
\end{equation}
where $\sigma_i$ are the  uncertainties of the corresponding observations $y_i$ (and assumed here to be given), while the mean of the Gaussian is given by $f(x_i;\phi)$. 
% The posterior in expr.(\ref{eq:bayes}) is the distribution of $\phi$ given the observed data and assumptions.
From a practical standpoint, in addition to the normalization being,
in general, unknown, the usual complexity of the posterior
distribution makes this possibly highly dimensional integral difficult
to compute. The use of Monte Carlo techniques, in particular of the
HMC, is typical in this context.
We focus below on two types of predictions: 1) The variance of the
model parameters $\delta\phi_j^2 = \mathbb{E}_{p_\theta}[\phi_j^2] -
(\mathbb{E}_{p_\theta}[\phi_j])^2$, where $j=1,...,d$, being $d$ the
dimension of $\phi$, and 2) the variance of the output mean $\delta
f_t^2 = \mathbb{E}_{p_\theta}[f_t^2] -
(\mathbb{E}_{p_\theta}[f_t])^2$, where $f_t$ is a shorthand notation
for the output mean $f(x_t;\phi)$, evaluated at a new ``test''
datapoint $x_t$ \footnote{Note that $E_{p_\theta}[f_t]$ is analogous
  to the so-called ``predictive distribution'' of Bayesian inference,
  however here we focus on the expected value of the prediction mean,
  instead of the expected value of the likelihood of $y(x_t)$
  itself.}. 


We are interested in studying the dependence of these quantities on the choice of
hyperparameters $\theta$ that characterize the prior distributions. In
particular we will consider the case of Gaussian priors, and determine
the dependence of our predictions with the width of this Gaussian.

\subsubsection{Model and data set}

We generate a synthetic dataset (cf. Figure \ref{fig:dataset}) by defining the points on an irregular grid in the range $x_i\in[-1.0;1.0]$, such that
\begin{equation}
  y_i = f(x_i;\phi_{\rm true}) + \sigma_i\epsilon~,
\end{equation}
where the mean is a 3rd degree polynomial, $f(x;\phi)=\phi_0+\phi_1x + \phi_2x^2 + \phi_3x^3$, with $\phi_{\rm true} = (1,1,1,1)$; $\epsilon\sim{\cal N}(0,1)$ is sampled from a standard Gaussian, and we consider a heteroscedastic dataset by defining a noise $\sigma_i$ dependent on $x_i$. We adopt the same model in order to  make inference on the parameters $\phi$. 

% Figure environment removed

%The likelihood reads
%\begin{equation}
%  p(D|\phi) = \prod_{i=1}^{N}\mathcal N(y_{i}|f(x_{i},\phi),\sigma_{i})\,,
%\end{equation}
%where $\mathcal N(\mu|\sigma)$ is the usual Gaussian distribution of
%mean $\mu$ and variance $\sigma^2$. As a model we choose a third
%degree polynomial $f(x,\phi) = \phi_{0} + \phi_{1}x + \phi_{2}x^{2}+\phi_{3}x^{3}$. 
%Note that this is the model that was also used to obtain the dataset.

The prior distribution is also chosen as a Gaussian, $\phi\sim {\cal N}(\mu_p,\sigma_p)$. 
For simplicity we choose the priors centered on the ``correct'' values
of the model (i.e. $\mu_p =\phi_{\rm true}$), while we keep the width 
$\sigma_p$ as a hyperparameter to study the dependence on\footnote{This is a simplified setup for the sake of illustration, given the methodological scope of this work. Nonetheless, it is straightforward to apply the method to the situation where we are interested in studying the dependence on both parameters $\mu_p$ and $\sigma_p$ simultaneously, or in general on the joint set of hyperparameters of the model. }.

For any choice of the prior width $\sigma_p$ we can obtain a prediction by
generating $N$ samples $\{\phi^{(\alpha)}\}^N_{\alpha=1}$ according to the distribution
$p_{\theta}(\phi|D)$ computed from \cref{eq:bayes}.  
  
\subsubsection{Reweighting approach}
\label{sec:bayesianhmc}

The reweighting method takes $N$ samples
$\{\phi_{i}^{({\alpha})}\}_{\alpha=1}^{N}$ obtained at
$\sigma_{p}=\sigma_{p}^{*}$ and computes the reweighted average using
$\tilde\sigma_{p}=\sigma_{p}^{*}+\epsilon$ in \cref{eq:rw}.  

For each sample $\phi^{(\alpha)}$, the reweighting factor becomes a polynomial expansion in
$(\sigma-\sigma_{p}^{*})$  
\begin{equation}
  \label{eq:rw bi}
  \tilde w_{\alpha}(\epsilon) = \frac{p_{\mu,\sigma_{p}^{*}+\epsilon}(\phi_{\alpha}|D)}{p_{\mu,\sigma_{p}^{*}}(\phi_{\alpha}|D)}.
\end{equation}
Notice that the zeroth order of \cref{eq:rw bi} is one, such that the zeroth order result corresponds to the usual Monte Carlo point estimate for $\delta\phi_{0}(\sigma_{p}^{*})$.

In order to generate these samples, we used the standard HMC algorithm. 
The equations of motion are
\begin{align}
  &H_{\theta}(\phi,\pi) = \frac{\pi^{2}}{2} - \log(p_{\theta}(\phi|D)),\\
	&\dot\phi_{j} = \pi_{j},\\
	&\dot \pi_{j} = - \frac{1}{\sigma_{p}^{2}}(\phi_{j} - (\mu_p)_j) + \sum_{i=0}^{N}\frac{1}{\sigma_{i}^{2}}\left( y_{i} - f(x_{i},\phi) \right)(x_{i})^{j},
\end{align}
where $\pi=\{\pi_{0},\pi_{1},\pi_{2},\pi_{3}\}$ are the momenta conjugated to $\phi$.
Note that all $\phi$-independent terms can be dropped from the
equations of motion, namely the normalization of $p_{\theta}(\phi|D)$
is not needed.
The eom were solved numerically using a fourth-order symplectic
integrator \cite{OMELYAN2003272} providing a high acceptance rate in
the Metropolis-Hastings step even with a coarse integration.  

The chosen integration step-size was $\varepsilon = 0.001$, while the
trajectory length was uniformly sampled in the interval $[0,100]\times
\varepsilon$\footnote{Due to the quadratic form of the Hamiltonian,
  the phase space of this system is cyclic. The algorithm is ergodic
  only if the trajectory length is
  randomized \cite{RHMC2017}.}.
%Taking into account the conclusions from \cref{sec:nspt}, the average trajectory length is approximately tuned such that the variance is minimized.  

% In the following, all of the Monte Carlo chains correspond to half million thermalized trajectories.

%All the predictions are a function of the hyperparameter $\sigma^*$ and
%we would, generically, be interested in this dependence.
%As for the quantity to study we focus on the uncertainty of the
%average value for $\phi_{i}$, $\delta\phi_{i} =
%\mathbb{E}_{p_{\theta}}[\phi_{i}^{2}]-\mathbb{E}_{p_{\theta}}[\phi_{i}]^{2}$,
%and analogously the uncertainty for the prediction of a new point
%$x_{n}$.  

\subsubsection{Hamiltonian perturbative expansion}

Following the procedure in \cref{sec:nspt}, the Monte Carlo samples
$\{(\tilde\phi_{j})^{\alpha}\}_{\alpha=1}^{N},~j=0,1,2,3$ were
obtained with the modified HMC algorithm for some values of
$\sigma_{p}^{*}$. 
We used the same parameters for the HMC as described in the previous
section. In particular our acceptances were so close to 100\% that any
bias due to the missing accept/reject step is negligible. 
We checked this hypothesis by further performing another simulation
with a coarser value of the integration step and finding completely
compatible results.


\subsubsection{Results}

\begin{table}[t]
  \centering
  \caption{Results for the expansion coefficients of the variance,
    ${\delta\phi^{2}_{j,n}}$ for $\sigma_{p}^{*}=0.3$
    from the reweighting and hamiltonian expansion.}
\scalebox{0.9}{
  \begin{tabular}{cccccccc}
	% \toprule
     & & \multicolumn{6}{c}{$n$} \\\cmidrule{3-8}
     & & 0 & 1 & 2 & 3 & 4 & 5  \\
    \midrule
\multirow{2}{*}{$\delta\phi^{2}_{0,n}$} & RW &    0.00014705(86) &    0.0001384(63) &    -0.000248(29) &     0.000367(62) &     -0.00071(51) &      -0.0003(12)  \\
                  & HAD &   0.00014705(86) &    0.0001365(34) &   -0.0002850(60) &     0.000311(20) &     0.000178(77) &     -0.00115(26)  \\

    \midrule
\multirow{2}{*}{$\delta\phi^{2}_{1,n}$} & RW &     0.01099(15) &       0.0285(12) &      -0.0450(58) &        0.032(13) &         0.04(10) &        -0.61(25)  \\
                  & HAD &     0.01099(15) &      0.02787(69) &      -0.0518(11) &       0.0248(38) &        0.189(16) &       -0.700(46)  \\
    \midrule
\multirow{2}{*}{$\delta\phi^{2}_{2,n}$} & RW &      0.008938(74) &      0.00830(28) &      -0.0283(10) &       0.0850(39) &       -0.234(18) &        0.603(78) \\
                  & HAD &     0.008938(74) &      0.00817(15) &     -0.02789(42) &       0.0849(13) &      -0.2505(44) &        0.726(15)  \\
    \midrule
\multirow{2}{*}{$\delta\phi^{2}_{3,n}$} & RW &     0.03617(59) &      0.1205(51) &       -0.182(24) &        0.050(61) &         0.63(42) &         -4.0(12)  \\
                  & HAD &     0.03617(59) &       0.1177(30) &      -0.2052(42) &        0.020(16) &        1.132(66) &        -4.02(19)  \\
    \bottomrule
    \label{tab:variance phi0}
  \end{tabular}
  }
\end{table}

Here we compare the predictions for the average model parameters
$\phi$ and their dependence on the prior width $\sigma$. In particular
we focus on the variance of the model parameters $\delta\phi^{2}_j$,
since these are the quantities most sentitive to the prior width (i.e. 
very thin priors result in small variance for the model
parameters). We have fixed $\sigma^* = 0.3$, but similar conclusions
are obtained for other values.  

The results of the Monte Carlo average for $\delta\tilde\phi^{2}_i$ and
its derivatives with respect to $\sigma$ are
shown in \cref{tab:variance phi0}. 
Results labeled ``RW'' use the reweighting method, while results
labeled ``HAD'' use the Hamiltonian approach. 

It is obvious that results using the Hamiltonian approach are more
precise:
the uncertainties in the derivatives, $\delta\phi^{2}_{i,n},n\neq 0$, are
smaller for the Hamiltonian approach, despite the statistics being the
same. The difference is larger for higher order derivatives: the
approach based on reweighting struggles to get a signal for the fourth
and fifth derivatives, while the Hamiltonian approach is able to
obtain even the fifth derivative with a few percent precision. 
This fits our expectations (see section~\ref{sec:hamilt-appr-repar}). 
\noindent\newline\newline
On the other hand, for our second quantity of analysis $\delta f_t^2$ (i.e. the variance of the prediction mean), Figure~\ref{fig:ypred} shows the results of the dependence on $\sigma_p$, where we have fixed $x_t=0.5$.

%dependence of the variance of the
%parameter prediction
%\begin{equation}
%  y_{\text{pred}}(x_{n})=\mathbb{E}_{p_{\theta}}[f(x_{n},\phi)] = \int d\phi p_{\theta}(\phi|D) f(x,\phi).
%\end{equation}
%at $x = 0.5$ with respect to $\sigma_P$.
The Hamiltonian approach gives visually results with a reduced variance,
similar to the results presented in table~\ref{tab:variance phi0}.

% Figure environment removed



%%% Local Variables:
%%% mode: latex
%%% TeX-master: "paper"
%%% End:

\section{Conclusion and Future Work}
In this work, I design corruption-robust algorithms for the Lipschitz contextual search problem. I present the \emph{agnostic checking} technique and demonstrate its effectiveness in designing corruption-robust algorithms. There are several open problems for future research. First, in the algorithm I propose for pricing loss, the schedule for agnostic checks is fixed upfront. Can the learner design an adaptive checking schedule for the pricing loss? Second, this work assumes the learner has knowledge of the Lipschitz constant $L$. Can the learner design efficient no-regret algorithms without knowledge of $L$? 

\bibliography{references}

\appendix

\begin{comment}
\section{System Architecture}
\label{appendix:architecture}
\system has a novel modularized system architecture with three key components: 
\emph{StreamManager}, 
\emph{TxnManager} and \emph{TxnScheduler}. 
These components are instantiated in each thread locally.
The execution outline of \system is presented in Algorithm~\ref{alg:algo}.
Transactional stream processing is continuous and potentially never ends (Line 1$\sim$8).
The dependency resolution and execution of state transactions are separated into two non-overlapping phases by punctuations~\cite{Tucker:2003:EPS:776752.776780} (Line 2 and 5), which guarantees that no subsequent input event will have a smaller timestamp. 
Effectively, a batch of state transactions is collected during the first phase, and processed during the second phase.

In the first phase (i.e., stream processing phase), 
the \emph{StreamManager} conducts preprocessing for every input event ($e$). Similar to some prior works~\cite{tstream}, state transactions may be issued but not immediately processed during preprocessing (Line 3).
The \emph{pre\_processing} and \emph{post\_processing} functions are exposed as APIs to users.
The \emph{TxnManager} handles dependency resolution (Line 4) among state transactions and insert decomposed operations to construct a \tpg. We discuss the detailed two-phase \tpg construction process in Section~\ref{subsec:construction}.

In the second phase  (i.e., transaction processing phase), 
the \emph{TxnManager} is first involved again to refine (Line 6) the constructed \tpg with further dependency resolution.
The \emph{TxnScheduler} 
schedules operations for concurrent execution based on the constructed \tpg according to the three dimensions of scheduling decisions (Line 7). 
In particular, a scheduling decision model $M$ is instantiated based on the constructed \tpg (Line 14).
\textbf{\circled{1}} Guided by $M$, execution threads adopt an exploration strategy (Section~\ref{subsec:explore}) to explore the constructed \tpg for operations available to be scheduled constrained by dependencies. 
\textbf{\circled{2}} 
During exploration, one or multiple operations may be treated as the 
% basic 
unit of scheduling (Section~\ref{subsec:granularity}). 
Subsequently, \textbf{\circled{3}} every thread executes operation(s) in the unit of scheduling with various abort handling mechanisms (Section~\ref{subsec:abort_handling}).
Only when state transactions are processed (i.e., committed or aborted) can the associated input events be postprocessed (Line 8) by the \emph{StreamManager} based on transaction processing results.
\end{comment}

\begin{comment}
\begin{algorithm}
\footnotesize
    \KwData{$e$ \tcp{Input event}}
    \KwData{$txn_{ts}$ \tcp{State transaction}}
    \KwData{$G$ \tcp{The currently constructed TPG}}
    \While{!finish processing of input streams}{
        \eIf(\tcp*[h]{Phase 1}){\text{$e$ is not a $punctuation$}}{
                $txn_{ts}$ $\gets$ PRE\_Processing($e$)\;
                \textbf{TPG\_Construction}($G$, $txn_{ts}$)\; 
          }(\tcp*[h]{Phase 2}){
                \textbf{TPG\_Refinement}($G$)\; 
                \textbf{TXN\_Scheduling}($G$)\; 
                POST\_Processing()\;
          }
    }
    
    \SetKwFunction{FMain}{TPG\_Construction}
    \SetKwProg{Fn}{Function}{:}{}
    \Fn{\FMain{$G$, $txn_{ts}$}}{
        $O_{1..k}$ $\gets$ \textbf{Partition} $txn_{ts}$\;
        \ForEach{\text{operation $O_{i}$ $\in$ $O_{1..k}$}}{
            \textbf{Identify} its \ld\;
            $G$ $\gets$ $G$ + $O_{i}$ \;
        }
    }
    \SetKwFunction{FMain}{TPG\_Refinement}
    \SetKwProg{Fn}{Function}{:}{}
    \Fn{\FMain{$G$}}{
        \ForEach{\text{vertex $e_{i}$ $\in$ $G$}}{
            \textbf{Identify} its \td, \pd\;
        }
    }
    
    \SetKwFunction{FMain}{TXN\_Scheduling}
    \SetKwProg{Fn}{Function}{:}{}
    \Fn{\FMain{$G$}}{
        $M$ $\gets$ Instantiated with $G$;\tcp{A decision model}
        \While{!finish scheduling of $G$
        }{
          \textbf{\circled{2}} $Scheduling Unit$ $\gets$ \textbf{\circled{1}} \emph{Explore}($G$, $M$)\; 
            \textbf{\circled{3}} \emph{Execute with Abort Handling} ($Scheduling Unit$)\; 
        }
    }
  \caption{Execution Outline of \system}
  \label{alg:algo}
\end{algorithm}
\end{comment}

\end{document}
