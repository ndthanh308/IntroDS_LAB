\documentclass{article}



\usepackage{arxiv}

\usepackage[utf8]{inputenc} % allow utf-8 input
\usepackage[T1]{fontenc}    % use 8-bit T1 fonts
\usepackage{booktabs}       % professional-quality tables
\usepackage{amsfonts}       % blackboard math symbols
\usepackage{nicefrac}       % compact symbols for 1/2, etc.
\usepackage{microtype}      % microtypography
\usepackage{lipsum}		% Can be removed after putting your text content
\usepackage{graphicx}
\usepackage{doi}


\title{Wide Field-of-View, Large-Area Long-wave Infrared Silicon Metalenses}

%\date{September 9, 1985}	% Here you can change the date presented in the paper title
%\date{} 					% Or removing it

\author{Hung-I Lin\thanks{These authors contribute equally to this work}\\
Massachusetts Institute of Technology\\
Cambridge, MA \\
\And
Jeffrey Geldmeier\footnotemark[1] \\
Lockheed Martin Corporation\\
Orlando, FL \\
\And
Erwan Baleine\footnotemark[1] \\
Lockheed Martin Corporation\\
Orlando, FL \\
\And
Fan Yang\footnotemark[1] \\
Massachusetts Institute of Technology\\
Cambridge, MA \\
\And
Sensong An \\
Massachusetts Institute of Technology\\
Cambridge, MA \\
\And
Ying Pan \\
Massachusetts Institute of Technology\\
Cambridge, MA \\
\And
Clara Rivero-Baleine\thanks{Correspondence} \\
Lockheed Martin Corporation\\
Orlando, FL \\
\texttt{rivcar21@hotmail.com} \\
\And
Tian Gu\footnotemark[2] \\
Massachusetts Institute of Technology\\
Cambridge, MA \\
\texttt{gutian@mit.edu} \\
\And
Juejun Hu \\
Massachusetts Institute of Technology\\
Cambridge, MA \\
}

% Uncomment to remove the date
%\date{}

% Uncomment to override  the `A preprint' in the header
%\renewcommand{\headeright}{Technical Report}
%\renewcommand{\undertitle}{Technical Report}
%\renewcommand{\shorttitle}{\textit{arXiv} Template}

\begin{document}
\maketitle

\begin{abstract}
Long-wave infrared (LWIR, 8-12 $\mu m$ wavelengths) is a spectral band of vital importance to thermal imaging. Conventional LWIR optics made from single-crystalline Ge and chalcogenide glasses are bulky and fragile. The challenge is exacerbated for wide field-of-view (FOV) optics, which traditionally mandates multiple cascaded elements that severely add to complexity and cost. Here we designed and experimentally realized a LWIR metalens platform based on bulk Si wafers featuring 140$^\circ$ FOV. The metalenses, which have diameters exceeding 4 cm, were fabricated using a scalable wafer-level process involving photolithography and deep reactive ion etching. Using a metalens-integrated focal plane array, we further demonstrated wide-angle thermal imaging.
\end{abstract}


% keywords can be removed
\keywords{Wide field-of-view optics, metasurface, metalens, LWIR, fisheye lens}


\section{Introduction}
LWIR, which coincides with the peak blackbody emission wavelengths of near-room-temperature objects, is strategically important to wide-ranging imaging applications spanning nigh vision, remote sensing, robotics, industrial process monitoring, building inspection, automotive sensing, gas detection, and beyond. Since most classical optical materials such as oxide glasses and polymers become opaque at LWIR due to phonon absorption, traditional LWIR optics resort to specialty materials such as single-crystalline Ge and chalcogenide glasses. These materials either incur a high cost to manufacture (Ge), or are mechanically fragile (chalcogenide glasses). Moreover, classical refractive optics made from these materials (Ge in particular) are temperature-sensitive due to thermo-optic focal drift. The challenges are further exacerbated when it comes to applications demanding a wide field-of-view (WFOV), since classical WFOV infrared optics entail a compound lens architecture comprising multiple (in general 4 or more) stacked optical elements to suppress coma aberration\cite{aburmad2014panoramic}. As a result, even LWIR lenses with a moderate FOV of around 60$^\circ$ each cost well above \$1,000 off-the-shelf.

Optical metasurfaces provide an alternative to classical refractive optics through modulation of the amplitude, phase, and polarization state of the wavefront using sub-wavelength nanostructures customarily termed meta-atoms \cite{yu2011light,ni2012broadband,capasso2018future,kamali2018review,aieta2012aberration,yang2023metasurface,west2014all,khorasaninejad2017metalenses,lalanne2017metalenses,tseng2018metalenses, gu2023reconfigurable}. While a large collection of metalenses have been implemented at visible and near-infrared wavelengths, relatively few demonstrations targeted the LWIR regime. Pioneering work by several groups have realized silicon-based LWIR metalenses \cite{fan2018high, huang2021long, li2022largest,hou2022lightweight}. Using Si as the metasurface material is advantageous in that it is amenable to large-area wafer-level manufacturing processes, and that deep reactive ion etching (DRIE) can produce high aspect ratio Si meta-atom structures ideal for large optical phase coverage and potentially dispersion engineering \cite{shan2022design}. Si wafers prepared using the common Czochralski method, however, are known to exhibit a strong optical absorption band centering at 9 $\mu m$ wavelength due to the presence of oxygen impurity \cite{conwell1952properties}. To mitigate the issue, Ge coupled with a ZnS antireflection layer has been adopted for metalens fabrication to suppress parasitic absorption across the LWIR band \cite{nalbant2022transmission}. The challenge of coma aberration suppression and expanding the FOV has nonetheless not been tackled by these pioneering investigations except in a recent report\cite{wirth2023large}. More recently, metalens arrays comprising five lens, each covering a sub-section of the FOV, have been implemented to demonstrate LWIR imaging spanning a horizontal FOV exceeding 60$^\circ$ upon image stitching in post-processing \cite{zhao2023wide}. The approach is however hardly scalable to WFOV applications, as a large FOV (e.g. 100$^\circ$) in both horizontal and vertical directions would require tens of individual metalenses, severely curtailing the optical throughput while escalating system complexity.

In this paper, we report the design and experimental demonstration of a WFOV metalens covering 140$^\circ$ circular FOV. The metalens assumed a simple architecture consisting of an optical aperture stop and a single-layer metasurface patterned in a float-zone Si wafer. The choice of float-zone Si contributes to suppression of the oxygen impurity absorption band while still enabling full leverage of industry-standard Si fabrication processes. Compared to other WFOV designs such as quadratic phase\cite{pu2017nanoapertures,martins2020metalenses,chen2020chip,zhang2020numerical,lassalle2021imaging,zhou2022metasurface,zhang2022design} and doublet metalenses\cite{arbabi2016miniature,groever2017meta,huang2021achromatic,martins2022fundamental,tang2020achromatic,kim2020doublet}, the present architecture is simple and yet does not compromise the imaging quality or optical efficiency \cite{yang2023wide}.

The rest of this paper is organized as follows. We will start with formulating the overarching analytical design approach. Two metalens designs were derived using the method, with an air gap and a ZnSe spacer, respectively. The former features a simpler construction whereas the latter has the advantage of enhanced FOV and imaging quality as predicted by our analytical theory and validated via numerical simulations. We then proceed to describe the fabrication protocols as well as experimental characterization of both metalenses.


\section{Analytical WFOV metalens design}
The WFOV metalens architecture consisting of an aperture and an all-silicon metasurface is schematically illustrated in Fig. \ref{fig:1}a \cite{shalaginov2020single,yang2021design,shalaginov2022metasurface}. The monochromatic phase profiles of the metasurfaces are defined following \cite{yang2023understanding}:

\begin{equation}
\phi(s)=(\frac{2\pi}{\lambda})\int_0^s -(sin\alpha(s)+\frac{s-d}{\sqrt{f^2+(s-d)^2}})ds
\label{eq:1}
\end{equation}

The corresponding RMS wavefront error is given by:

\begin{equation}
\sigma\approx\frac{3nL^2D^3|s-d|}{160\left(f^2+\left(s-d\right)^2\right)(L^2+s^2)^{\frac{3}{2}}}
\label{eq:2}    
\end{equation}

We note that the numerator in the expression above contains the factor $|s-d|$, which corresponds to the transverse offset between the incident position of the chief ray on the metasurface and the corresponding focal spot position (i.e., image height). In an image-space telecentric configuration, the term vanishes, yielding optimal image quality. Optimizing the metalens design therefore involves engineering its image height vs. incident angle relation to mimic the telecentric configuration. This can be accomplished by changing the refractive index $n_{sub}$ of the spacer material, leveraging refraction at the air-spacer front surface as a practical means to modify the image height. Following this rationale, we examine the dependence of the RMS wavefront error for various spacer material refractive indices and thicknesses (Fig. \ref{fig:1}b).

Guided by the theoretical insight, we have chosen ZnSe ($n_{sub} = 2.40$ at 10.6 $\mu m$) as the spacer material. An air-gap design ($n_{sub} = 1$) was also implemented for comparison. The detailed design parameters are tabulated in Table \ref{tab:1} and Fig. \ref{fig:1}c plots the phase profiles of the designs.

\begin{table}[ht]
\caption{Metalens design parameters} 
\label{tab:1}
\begin{center} 
\begin{tabular}{|l|l|l|l|l|}
\hline
\rule[-1ex]{0pt}{3.5ex}  & Wavelength & Aperture size & Air-gap/ZnSe thickness & Si substrate thickness \\
\hline
\rule[-1ex]{0pt}{3.5ex} Air-gap & 10.6 $\mu m$ & 6 $mm$ & 7.5 $mm$ & 675 $\mu m$ \\
\hline
\rule[-1ex]{0pt}{3.5ex} ZnSe & 10.6 $\mu m$ & 10 $mm$ & 44 $mm$ & 675 $\mu m$ \\
\hline
\rule[-1ex]{0pt}{3.5ex} & Focal length & Metalens size & Image plane size & FOV \\
\hline
\rule[-1ex]{0pt}{3.5ex} Air-gap & 12 $mm$ & 32 $mm$ & 21 $mm$ & 90$^\circ$ \\
\hline
\rule[-1ex]{0pt}{3.5ex} ZnSe & 20 $mm$ & 48 $mm$ & 38 $mm$ & 140$^\circ$ \\
\hline
\end{tabular}
\end{center}
\end{table}

% Figure environment removed

Next we translate the phase functions into actual metasurface layouts. The all-Si meta-atom structure is depicted in Fig. \ref{fig:2}a inset, which is composed of 12 $\mu m$ tall pillars with a 4 $\mu m$ pitch etched into float-zone Si wafers. Full-wave electromagnetic simulations were performed using the Lumerical FDTD solver, and the transmittance and phase delay of the meta-atoms at 10.6 $\mu m$ wavelength as functions of the pillar diameter are shown in Figs. \ref{fig:2}a-b. Eight meta-atoms with approximately $\frac{\pi}{4}$ step size in phase were chosen to construct the metasurfaces. To optimize transmittance while suppressing phase error, we invoked a figure-of-merit function as the criterion to choose the meta-atom diameters \cite{shalaginov2021reconfigurable,yang2022reconfigurable}. The eight selected meta-atom designs are summarized in Table \ref{tab:2}.

% Figure environment removed

\begin{table}[ht]
\caption{Meta-atom diameter, transmittance, and phase delay} 
\label{tab:2}
\begin{center} 
\begin{tabular}{|l|l|l|l|l|l|l|l|l|}
\hline
\rule[-1ex]{0pt}{3.5ex} Meta-atom index & 1 & 2 & 3 & 4 & 5 & 6 & 7 & 8 \\
\hline
\rule[-1ex]{0pt}{3.5ex} Phase [$^\circ$] & 0 & 44 & 94 & 137 & 176 & 219 & 274 & 318 \\
\hline
\rule[-1ex]{0pt}{3.5ex} Transmittance & 0.76 & 0.72 & 0.76 & 0.84 & 0.80 & 0.68 & 0.67 & 0.76 \\
\hline
\rule[-1ex]{0pt}{3.5ex} Diameter [$\mu m$] & 1.50 & 1.75 & 1.94 & 2.05 & 2.13 & 2.22 & 2.34 & 2.42 \\
\hline
\end{tabular}
\end{center}
\end{table}

Based on the meta-atom characteristics, performances of the WFOV metalenses can be numerically evaluated using the Kirchhoff diffraction integral. The transverse and longitudinal focal spot intensity profiles of the two WFOV metalens designs at several different angles of incidence (AOIs) are presented in Fig. \ref{fig:3} and Fig. \ref{fig:4}, respectively. The modulation transfer functions (MTFs) at different spatial frequencies were obtained through Fourier transform of the simulated point-spread-functions (PSFs) and are shown in Figs. \ref{fig:5}a-b.

% Figure environment removed

% Figure environment removed

% Figure environment removed

We now characterize the focusing efficiencies and Strehl ratios of the WFOV metalenses at 10.6 $\mu m$ wavelength using the numerical results in Figs. \ref{fig:3} and \ref{fig:4}. Here the focusing efficiency is defined as the fraction of power encircled within a diameter equaling five times the focal spot full-width-at-half-maximum (FWHM), normalized by the total incident power \cite{yang2021design}. Figures \ref{fig:5}c-d plot the two parameters as functions of AOI. The air-gap metalens has a focusing efficiency of 53$\%$ and a Strehl ratio of 0.63, both averaged over AOIs across the 90$^\circ$ FOV, whereas the ZnSe-spacer metalens claims a focusing efficiency of 50$\%$ and a Strehl ratio of 0.86, similarly averaged over AOIs throughout the entire 140$^\circ$ FOV. The enhanced focusing performance of the ZnSe-spacer lens over the air-gap design, evidenced by its diffraction-limited performance (Strehl ratio $>$ 0.8) over an extended FOV of 140$^\circ$, validates our theoretical prediction.



\section{Metalens fabrication}
2 $\mu m$ thick SiN films were deposited by plasma-enhanced chemical vapor deposition (STS PECVD) on 675 $\mu m$ thick float zone Si wafers as a hard mask for DRIE. To define the metasurface patterns, a negative-tone photoresist (AZ nLOF 2035) was spin-coated onto the substrates at 3000 revolutions per minute (rpm). The resist was soft-baked at 115 $^\circ$C for 1 minute, exposed on an MLA150 Maskless Aligner, and then post-exposure baked at 115 $^\circ$C for 1 minute. The photoresist was developed by immersing the sample in Microposit MF-319 developer for 1 minute, followed by rinsing in deionized water. To etch the SiN hard mask, dry etching was performed using dual gas inlets with a mixture of SF\textsubscript{6} and C\textsubscript{4}F\textsubscript{8} (STS ICP RIE). The Bosch process was subsequently used to etch the Si meta-atoms (SPTS Rapier DRIE) before hard mask removal via buffered HF (BHF) wet etching to complete the fabrication process. To mitigate scalloping which is commonly associated with the Bosch process, we optimize 1) the gas flow ratio, which balances the etching and passivation effects; and 2) the etching loop time, which dictates the spatial period and severity of scalloping (Supplementary Information). Figure \ref{fig:6} shows images of the metasurfaces showing a low-roughness sidewall profile with minimal scalloping fabricated using the optimized parameters.

% Figure environment removed

\section{Metalens characterization and thermal imaging demonstration}

The PSFs of the metalenses were characterized experimentally using a setup schematically depicted in Fig. \ref{fig:7}a. A collimated 10.6 $\mu m$ CO\textsubscript{2} laser (L4GST, Access) was first attenuated and then expanded before being directed at the metalens and subsequently focused onto a microbolometer (Boson, FLIR). For field angles away from $0^\circ$, the metalens and the detector were tilted together, and the microbolometer also required lateral movement to span the much larger image plane. Several examples of the metalens PSFs at different AOIs are presented in Figs. \ref{fig:4}d-f and Figs. \ref{fig:5}d-f, respectively, showing excellent agreement between the measurement results and simulations. The monochromatic MTF performance of the metalenses at 10.6 $\mu m$ was measured using the collimated CO\textsubscript{2} laser and an interferometric wavefront sensor (SID4 DWIR, Phasics) schematically depicted in Fig. \ref{fig:7}b. An aperture was placed in front of the metalenses in order to limit the collimated light bundle to the entrance pupil size. The Phasics camera was set to the focal mode, allowing measurements to be collected after direct diffraction from the lens. For the ZnSe metalens, an additional pinhole was placed near the focal plane to remove the zero order component. The corresponding MTFs are shown in Fig. \ref{fig:7}c-d, indicating diffraction-limited performance in agreement with simulation predictions.

% Figure environment removed

Finally, we demonstrated wide-angle thermal imaging using the metalens. The ZnSe metalens was mounted onto a microbolometer (Boson, FLIR) to form a thermal infrared camera. The experimental setup is shown in Fig. \ref{fig:8}a. A hot plate was placed 0.53 meter away in front of the WFOV metalens to act as a LWIR illumination source. A card board 1.26 meters in length and perforated with 'EXTREME LOCKHEED MARTIN' patterns was placed in between the hot plate and the metalens, which blocked the blackbody radiation from the hot plate in all areas except within the inverse 'EXTREME LOCKHEED MARTIN' pattern. A 10.5 $\mu m$ filter with 0.2 $\mu m$ bandwidth was placed in front of the sensor. The captured image is presented in Fig. \ref{fig:8}b covering 100$^\circ$ FOV. Since the image size is larger than the image sensor area, the bolometer had to be laterally translated with respect to the metalens and the resulting image sections were stitched to form Fig. \ref{fig:8}b. We further compared the imaging results with those using an unfiltered blackbody source to quantify the impact of chromatic aberration, and the details were elaborated in the Supplementary Information.

% Figure environment removed

\section{Conclusion}

In this paper, we reported the design and experimental demonstration of metalenses operating at the LWIR band with an ultra-wide FOV of 140$^\circ$. An analytical model was used to rationally guide the design of the metasurfaces as well as the lens spacer material choice. Following the designs, centimeter-scale metasurfaces were fabricated on float-zone silicon wafers using large-area photolithographic patterning and optimized DRIE protocols. Experimental characterization of the metalenses validated our theoretical design and demonstrated thermal imaging using a metalens-integrated infrared camera. Benefiting from its simple architecture, scalable fabrication process, and exceptional wide-FOV imaging capability, the WFOV metalens technology potentially offers an appealing alternative to existing LWIR compound lens optics for applications covering night vision, spectroscopic sensing, environmental monitoring, and many others.

\section{Acknowledgements}%% if any
The authors acknowledge nanofabrication and characterization facility support provided by MIT.nano and the Center for Nanoscale Systems at Harvard University.

\section{Funding}%% if any
This work was supported by Lockheed Martin Corporation Internal Research and Development and Defense Advanced Research Projects Agency Defense Sciences Office Program: EXTREME Optics and Imaging (EXTREME) under agreement number HR00111720029. The views, opinions and/or findings expressed are those of the authors and should not be interpreted as representing the official views or policies of the Department of Defense or the US Government.

% if your bibliography is in bibtex format, use those commands:
\bibliographystyle{unsrt}
\bibliography{main}      
\end{document}