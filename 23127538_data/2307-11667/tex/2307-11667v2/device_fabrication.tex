All devices are coplanar waveguide resonators fabricated using the same designs as reported by~\citeauthor{Kopas2022}~\cite{Kopas2022}. Nominally identical designs and fabrication procedures were used for all samples.  Prior to etching, the samples were cleaned via ultrasonication in toluene, acetone, methanol, and isopropanol, then patterned using optical lithography and AZ-P4330-RS photoresist. The films were etched in a single 4 minute CF$_{4}$/N$_{2}$ Inductively Coupled Plasma – Reactive Ion Etch (Panasonic E640).  Since the Ta films were deposited on sapphire substrates, the etches did not produce any trenching into the substrate. After etching, the resist was submerged in AZ 300 T stripper at 80 $^{\circ}$C for 1 hour.  After stripping, the samples were diced and again cleaned ultrasonically in toluene, acetone, methanol, and isopropanol before being wire bonded for cryogenic microwave measurement. Optical images of the resonators are shown in Fig.~\ref{fig:3Z_optical}.

Inverse coupling quality factors, $1/Q_c$, of the fabricated resonators are presented in Supplementary Materials Table 1 and range from $1.18 \times 10^{-6}$ to $6.61 \times 10^{-6}$ across all devices. This is a larger spread of values with a trend towards smaller coupling factors than the simulated $1/Q_c$ values of these designs, which ranged from $1.95 \times 10^{-6}$ to $2.02 \times 10^{-6}$.~\cite{Kopas2022} This variation is likely due to a slight over etch of the devices during fabrication, which is congruent with a thinner measured conductor width than the lithography designs used (design: $6 \mu\, \mathrm{m}$, measured: $5.5 \mu\,\mathrm{m}$).

% Figure environment removed