% Ta Growth and Characterization
%
% Figure environment removed

%
% Figure environment removed
%

Ta films of nominal 200 nm thickness were deposited on $c$-axis sapphire wafers (2" diameter, 550\,$\mu$m thickness, from Hefei Keijing Materials Technology) using dc magnetron sputtering. Prior to deposition, the as-purchased substrates were cleaned via ultrasonication in acetone, isopropanol, and deionized water for 5 min each and blown dry with nitrogen of purity 4N. To deposit thin films with different grain sizes, two different substrate temperatures $T=400\,^{\circ}$C (sample label `SGS' or 
`small grain size') and $T=500\,^{\circ}$C (sample label `LGS' or `larger grain size') during the deposition were chosen, while other deposition parameters (background pressure $\leqslant 1 \times 10^{-7}$\,Torr, argon pressure 3\,mTorr, deposition power 150\,W, deposition rate 3.6\,nm/min) were not changed. The deposition was carried out without the use of a seed layer. A $T=600\,^{\circ}$C sample was also grown, but no increase in grain size was detected, so this sample is not included in the detailed film comparison.

The large-scale diffraction spectrum of the films (Fig.\,\ref{fig:structural_chemical_characterization}(a) inset)  is dominated by a set of two peaks, which can be associated with diffraction at the [110] and [220] planes of the $\alpha$-Ta[220] phase. A comparably small diffraction signal, which rises just above the background signal, is detected at 2$\Theta\approx33.7^{\circ}$ that can be associated with the [002]-diffraction of the tantalum $\beta$-phase. Our observations indicate the Ta films prepared for this study predominantly nucleate in the $\alpha$-Ta phase. This finding is consistent with previous reports on 200\,nm thick $\alpha$-Ta films on $c$-axis sapphire, which were deposited under comparable conditions~\cite{Place2021}. The close-up view of the  $\alpha$-Ta[110] peaks for the `SGS2' and `LGS2' samples is shown in the main panel of Fig.~\ref{fig:structural_chemical_characterization}(a). The diffraction peak of the 'LGS2' sample ($\sigma=0.4^{\circ}$) has a smaller full-width-half-maximum $\sigma$ compared to that of the 'SGS2' sample ($\sigma=0.5^{\circ}$). While this observation is indicative of a larger average grain size in the 'LGS2' sample, we note that the Scherrer equation is less suited to quantitatively analyze the grain size in this case, owing to the grain shape anisotropy and significant grain size variations (see AFM measurements below). We further observed a small deviation in the [002]-diffraction angle both between the 'SGS2' (2$\Theta=38.1^{\circ}$) and 'LGS2' (2$\Theta=38.3^{\circ}$) sample, as well as with respect to the nominal bulk value (2$\Theta=38.505^{\circ}$). This can be attributed to the presence of strain in the thin film structure, which appears slightly more pronounced in the `SGS2' sample.

To characterize the crystalline grain size of the Ta films deposited at different substrate temperatures, we carried out atomic force microscopy (AFM) measurements (tapping mode). The resulting AFM topographies for samples `SGS2' and `LGS2' are shown in Fig.\,\ref{fig:structural_chemical_characterization}(b). Both topographies are characterized by elongated crystalline grains oriented along the hexagonal basal plane of the sapphire surface, consistent with previous reports~\cite{Alegria2023}. Moreover, the grains of `LGS2' exhibit a visibly larger grain size area $G$ than those of `SGS2', consistent with our expectations in light of the substrate temperatures during deposition. To quantify these grain size differences, we applied a watershed algorithm~\cite{rabbani2015} to determine $G$, which is an average across several 1\,$\mu$m$^2$ surface areas per sample and several samples for each deposition condition. This approach was previously applied to quantify grain sizes of Nb films~\cite{Premkumar2021}. We obtain $G=924\pm51 \,\mathrm{nm}^2$ for the `SGS2' and $G=1700\pm29\,\mathrm{nm}^2$ for the `LGS2' sample, respectively. Interestingly, the average grain size $G=1732\pm92\,\mathrm{nm}^2$ of samples deposited at a substrate temperature $T=600\,^{\circ}$C is comparable to that of the $T=500\,^{\circ}$C deposition~\cite{SI}.

To detect the possible influence of the crystalline grain size on the surface oxide structure, we performed X-ray photoelectron spectroscopy (XPS) measurements (Kraxios Ultra DLD; X-ray source: Al K$\alpha$ line $E=1486.6\,$eV) on the `SGS2' and `LGS2' samples. We note that these samples did not undergo surface treatment to remove native surface oxides prior to XPS measurements. The resulting XPS spectra in Fig.\,\ref{fig:structural_chemical_characterization}(c) show the photo-electron count as a function of the electron binding energy for the Ta-4f core level. The spectra are dominated by a four peak structure, which is predominantly composed of the spin-orbit split Ta$^0$ and Ta$^{5+}$ doublets that can be assigned to the metallic Ta bulk and the Ta$_2$O$_5$ at the film surface, respectively~\cite{mcguire1973core, himpsel1984core}. 

We quantify the relative contributions of the different Ta oxidation states to the observed XPS spectra by applying a least-squares fit based on Gaussian profiles. We find a three doublet structure composed of six Gaussians, as shown in Fig.\,\ref{fig:structural_chemical_characterization}(d), can most accurately describe these spectra. The additional third doublet exhibits a core level shift of $\approx1.1\,$eV and can be assigned to the Ta$^{3+}$ oxidation state of the Ta$_2$O$_3$ suboxide~\cite{himpsel1984core}. The resulting relative contributions of Ta, Ta$^{3+}$, and Ta$^{5+}$ obtained from these fits are shown in Table\,\ref{tab:xps} and reveal a near identical chemical structure of the tantalum film surface for both samples. this is consistent with their almost identical XPS spectra ({\em cf.}~Fig.~\ref{fig:structural_chemical_characterization}(c)). The relative spectral weight of the Ta$^0$ and Ta$^{5+}$ peaks at the given incident X-ray energy is in close agreement with that found in previous XPS studies of tantalum films and indicates a surface oxide thickness of approximately 2\,nm~\cite{Place2021}.

\begin{table}
    \caption{\label{tab:xps}Relative atomic concentration of different tantalum oxidation states in the 'SGS2' and 'LGS2' samples as obtained from fits to the XPS spectra.}
    \begin{ruledtabular}
        \begin{tabular}{lcc}
         & Small grain size & Larger grain size\\
        \hline
        Oxidation state & atomic \% & atomic \% \\
        \hline
        Ta$^0$ & 18 & 20\\
        Ta$^{3+}$  & 17 & 18 \\
        Ta$^{5+}$ & 65 & 62 \\
        \end{tabular}
    \end{ruledtabular}
\end{table}
