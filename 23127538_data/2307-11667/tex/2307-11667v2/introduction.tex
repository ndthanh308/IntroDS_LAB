% Introduction
Superconducting qubits are a promising avenue for scalable quantum computing devices due to their high-fidelity operation~\cite{kjaergaard_superconducting_2020,Zhang2021, Hyyppa2022, Howard2022}. Recent advances in qubit design, packaging, and control have shrunken the gap toward their practical use~\cite{preskill2018quantum, arute2019quantum}. Still, dielectric losses due to bulk substrates, surface oxides, and amorphous or defect-ridden material interfaces limit the coherence of superconducting qubits and ancillary devices~\cite{Place2021,Crowley2023,Read2022}. Microscopically, materials loss is largely associated with the excitation of two-level systems (TLS) that dominate microwave losses in the technologically relevant range of low temperatures and single-photon numbers~\cite{Muller2019,McRae2020}. Materials engineering has been identified as a leading route for improvement of superconducting qubit coherence by reducing the effect of TLS~\cite{siddiqi2021engineering}.

Recent works demonstrate improved qubit performance when $\mathrm{\alpha}$-phase tantalum (Ta) replaces niobium (Nb) as the superconducting thin film base layer for device fabrication~\cite{Place2021,Wang2022}. These findings are further supported by loss measurements of superconducting microwave resonators~\cite{Crowley2023,Alegria2023,Shi2022,Lozano2022} and it is believed that the loss reduction is afforded by the simple oxide structure of the Ta film surface~\cite{Place2021}. Further evidence for this is suggested by recent work capping Nb films with Ta for improved qubit performance~\cite{Bal2023}.

A detailed materials study of Nb-based qubits links the bulk properties of the polycrystalline films to qubit losses~\cite{Premkumar2021}. Small crystalline grain sizes were found to correlate with increased qubit losses, which could arise from TLS present at the subsurface grain boundary oxides in Nb films ~\cite{Premkumar2021}. Hence, the grain size of the superconducting base layer has recently been debated as a promising process parameter to further minimize microwave losses in Ta films. Moreover, controlled A/B-testing studies would be desirable to firmly establish this relation and it remains unknown whether grain size effects on microwave losses extend to resonators based on Ta films, whose surface and subsurface oxide structure differs from that of Nb films.

The goal of this work is to probe the relationship between grain size and microwave losses for $\alpha$-Ta films grown on c-axis sapphire, a substrate commonly used for Ta growth~\cite{Place2021,Alegria2023}. To this end, we perform microwave loss measurements of coplanar waveguide resonators made from magnetron-sputtered $\alpha$-Ta films with large and small grain sizes. We compare the losses of both types of films across thirty resonators from multiple chips and report no statistical difference between the performance of films with small and larger grain sizes. In combination with results from the chemical and crystallographic thin film characterizations, our observations indicate that grain size does not play a significant role in microwave losses for $\alpha$-Ta films across the tested parameter regime.