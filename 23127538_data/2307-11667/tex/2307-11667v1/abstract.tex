% % ABSTRACT
In recent years, the implementation of thin-film Ta has led to improved coherence times in superconducting circuits. Efforts to further optimize this materials set have become a focus of the subfield of materials for superconducting quantum computing. It has been previously hypothesized that grain size could be correlated with device performance. In this work, we perform a comparative grain size experiment with $\alpha$-Ta on $c$-axis sapphire. Our evaluation methods include both room-temperature chemical and structural characterization and cryogenic microwave measurements, and we report no statistical difference in device performance between small- and larger-grain-size devices with grain sizes of 924 nm$^2$ and 1700 nm$^2$, respectively. These findings suggest that grain size is not correlated with loss in the parameter regime of interest for Ta grown on c-axis sapphire, narrowing the parameter space for optimization of this materials set.