% Discussion
An enduring hypothesis in the superconducting qubit community has been that larger grain size in superconducting thin films is an indication of improved device performance. The simple and stable oxide structure of Ta differs from that of Nb, where~\citet{Premkumar2021} reported that smaller grain size films exhibited higher concentrations of suboxides in interface regions, resulting in measurably higher losses in their Nb resonators. In this study, we show that smaller grain size does not induce significant low power loss in Ta thin films grown between 400 and 500\,°C on c-axis sapphire. 

Microwave measurements of low power loss suggest that there is no statistically significant difference between the intrinsic TLS losses of the two grain size Ta thin films. Chemical and structural analysis support this interpretation, as the surface chemistry obtained by XPS is nearly identical for the two films. This distinguishes densely packed Ta films with their simple Ta$_2$O$_5$ surface oxide structure~\cite{Place2021} from Nb films~\cite{Premkumar2021} for which subsurface grain boundary oxides contribute a grain size dependent TLS channel. Following this train of thought, we expect qubits and resonators fabricated from Ta films to exhibit more uniform microwave losses than those fabricated from Nb. At the same time, non-negligible concentrations of Ta$^{3+}$ species found in our XPS measurements indicates the presence of Ta$_2$O$_3$ suboxides at the Ta metal-Ta$_2$O$_5$ interface consistent with a recent report~\cite{mclellan2023}. Interestingly, we also detect practical limits within which to tune the grain size of [110]-oriented $\alpha$-Ta films deposited on $c$-axis sapphire: Deposition at substrate temperatures below 400\,°C favors the formation of the unwanted $\beta$-phase~\cite{knepper2006effect, myers2013beta} whereas grain size does not respond to an increase of substrate temperature in excess of 500\,°C in our study. Thus, it would be interesting to explore other substrates or sapphire surface orientations to promote larger grain sizes up to the formation of single-crystalline Ta films. On the other hand, our study suggests more sophisticated materials engineering efforts that focus on the reduction of TLS losses at the immediate metal-air surface rather than on the optimization of bulk properties, such as grain size, are required to further reduce microwave losses below those reported in this and other recent studies~\cite{Lozano2022, Alegria2023}. These efforts will benefit from targeted A/B testing studies, such as is presented here, to address the vast materials and processing parameter space in order to maximize state-of-the-art superconducting qubit performance.