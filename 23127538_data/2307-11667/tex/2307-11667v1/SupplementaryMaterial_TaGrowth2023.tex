%% ****** Start of file TaGrowth2023.tex ****** %

\documentclass[%
aip,
superscriptaddress,
% jmp,
% bmf,
% sd,
% rsi,
amsmath,amssymb,
%preprint,%
reprint,%
%author-year,%
%author-numerical,%
% Conference Proceedings
]{revtex4-1}
\usepackage{braket}
\usepackage{graphicx}% Include figure files
\usepackage{dcolumn}% Align table columns on decimal point
\usepackage{bm}% bold math
%\usepackage[mathlines]{lineno}% Enable numbering of text and display math
%\linenumbers\relax % Commence numbering lines

\usepackage[utf8]{inputenc}
\usepackage[T1]{fontenc}
\usepackage{mathptmx}
\usepackage{placeins}  % tools for figure and table placement
\usepackage{bibentry}

\usepackage{amsmath}
\usepackage{amsfonts, bbm, accents}

\usepackage[usenames,dvipsnames]{xcolor}
\usepackage{svg}
\usepackage[breaklinks=true]{hyperref}
\usepackage{lipsum}
%getting rid of hyperref's boxes.
%From:http://tex.stackexchange.com/a/51349
\hypersetup{
  colorlinks   = true, %Colours links instead of boxes
  urlcolor     = blue, %Colour for external hyperlinks
  linkcolor    = blue, %Colour of internal links
  citecolor   =  MidnightBlue %Colour of citations
}

\usepackage{soul} % for strikethrough \st 
% Edit DEFS
\newcommand\red{\color{red}}
\newcommand\green{\color{ForestGreen}}
\newcommand\blue{\color{blue}}
\newcommand\purp{\color{Plum}}

% Color comments for each author
\newcommand{\nrm}[1]{{\color{purple}[NM: {#1}]}}
\newcommand{\crhm}[1]{{\color{ForestGreen}[CRHM: {#1}]}}


%%%
\usepackage[capitalise]{cleveref} % must be loaded after hyperref
\crefformat{equation}{Eq.~(#2#1#3)} % These change 'equation' to Eq., more PRA-style
\crefformat{section}{Sec.~#2#1#3} % These change 'equation' to Eq., more PRA-style
\Crefformat{equation}{Equation~(#2#1#3)}
\crefformat{figure}{Fig.~#2#1#3}
\crefrangeformat{equation}{Eqs.~#3(#1)#4--#5(#2)#6}
\Crefformat{section}{Section~#2#1#3}
%%%

\begin{document}

% Use the \preprint command to place your local institutional report number 
% on the title page in preprint mode.
% Multiple \preprint commands are allowed.
%\preprint{}

\title{Supplementary Material: Grain size in low loss superconducting Ta thin films on c-axis sapphire}

\author{Sarah Jones*}
\affiliation{
Department of Electrical, Computer, and Energy Engineering, University of Colorado Boulder, Boulder, Colorado 80309, USA}
\author{Nicholas Materise*}
\affiliation{
Department of Physics, Colorado School of Mines, Golden, Colorado 80401, USA}
\author{Ka Wun Leung*}
\affiliation{
Department of Physics, The Hong Kong University of Science and Technology, Clear Water Bay, Kowloon, Hong Kong SAR, China}
\author{Brian Isakov}
\affiliation{
Department of Electrical, Computer, and Energy Engineering, University of Colorado Boulder, Boulder, Colorado 80309, USA}
\author{Xi Chen}
\affiliation{
Department of Physics, The Hong Kong University of Science and Technology, Clear Water Bay, Kowloon, Hong Kong SAR, China}
\author{Andr\'as Gyenis}
\affiliation{
Department of Electrical, Computer, and Energy Engineering, University of Colorado Boulder, Boulder, Colorado 80309, USA}
\author{Berthold Jaeck}
\email[]{bjaeck@ust.hk}
\affiliation{
Department of Physics, The Hong Kong University of Science and Technology, Clear Water Bay, Kowloon, Hong Kong SAR, China}
\affiliation{
HKUST IAS Center for Quantum Technologies, The Hong Kong University of Science and Technology, Clear Water Bay, Kowloon, Hong Kong SAR, China}
\author{Corey Rae H. McRae}
\email[]{coreyrae.mcrae@colorado.edu}
\affiliation{
Department of Electrical, Computer, and Energy Engineering, University of Colorado, Boulder, Colorado 80309, USA}
\affiliation{ 
National Institute of Standards and Technology, Boulder, Colorado 80305, USA
}%
\affiliation{ 
Department of Physics, University of Colorado, Boulder, Colorado 80309, USA
}%

\date{\today}

\pacs{}% insert suggested PACS numbers in braces on next line

\maketitle %\maketitle must follow title, authors, abstract and \pacs

%=================================================

% Wide Scans
\section{Ta grain size for T=600\,°C deposition}
We used atomic force microscopy in tapping mode to study the grain size distribution of the tantalum films deposited at a substrate temperature of T=600\,°C. A typical topography is shown in Fig.~\ref{fig:600degGS}, with an average grain size of $G=1732\pm92\,\mathrm{nm}^2$.

% Figure environment removed


\section{Wide scans}
We measured the broadband responses of the devices over multiple cooldowns to compare the variability of the background (Fig.~\ref{fig:widescan}). We measured $S_{21}$, the microwave transmission coefficient, through each device's transmission line with 65,001 points and 100 kHz IF bandwidth to resolve most of the resonator dips over the 4--8 GHz range. Repeated measurements of the same devices show significant similarities, as shown by the two blue curves (SG2) and the two green curves (LGS1). Broad resonances can be attributed to a low-Q spurious environmental mode, which is present for all but LGS2 (purple curve).

% Extracted resonator parameters
\section{Extracted resonator parameters}
Table~\ref{tab:resmeas} summarizes the parameters extracted from fitting the TLS and power-independent loss of each resonator. The uncertainties are the 95\% confidence intervals returned by a least squares fitting routine. 

Figure~\ref{fig:LPminusHP} compares the loss metrics $F \delta_{\mathrm{TLS}}^0$ with $\delta_{\mathrm{LP}}$, $\delta_{\mathrm{HP}}$, and $\delta_{\mathrm{LP}}-\delta_{\mathrm{HP}}$. There is excellent agreement between $F \delta_{\mathrm{TLS}}^0$ and $\delta_{\mathrm{LP}}-\delta_{\mathrm{HP}}$, where all measurements lay along the line of 1:1 correlation. The mismatch between $F \delta_{\mathrm{TLS}}^0$ and $\delta_{\mathrm{LP}}$ highlights that one cannot extract the intrinsic TLS loss with the low power loss alone; the power-independent loss is necessary to accurately compute TLS loss. The scatter in the center plot indicates a lack of correlation between power-independent loss and TLS loss.

% Figure environment removed

% Figure environment removed

\begin{table*}[t]
\caption{\label{tab:resmeas}Parameters extracted from cryogenic microwave measurements of Ta on Al2O3 coplanar waveguide (CPW) resonators. Values are given with their 95$\%$ confidence intervals where available. $f_0$: resonance frequency. $1/Q_{i,\mathrm{HP}}$: inverse high power internal quality factor. $F \delta_{\mathrm{TLS}}^{0}$: resonator-induced intrinsic TLS loss. $1/Q_{i,\mathrm{LP}}$: inverse low power internal quality factor. $1/Q_c$: inverse coupling quality factor at high power. Surface treatment labels correspond to small grain size (SGS*) and large grain size (LGS*) devices.}
\begin{ruledtabular}
\begin{tabular}{ccccccccc}
Surface treatment & $f_0$ (GHz) & $1/Q_{i,\mathrm{HP}}$ ($\times 10^{-6}$) & $F \delta_{\mathrm{TLS}}^{0}$ ($\times 10^{-6}$) & $1/Q_{i,\mathrm{LP}}$ ($\times 10^{-6}$) & $1/Q_c$ ($\times 10^{-6}$) & $n_c$ & $\beta$\\
\hline
LGS1 & 5.704 & 0.05 $\pm$ 0.01 & 1.73 $\pm$ 0.02 & 1.9 $\pm$ 0.3 & 4.658 $\pm$ 0.004 & 0.14 $\pm$ 0.02 & 0.197 $\pm$ 0.005 \\
Cooldown 1 & 6.112 & 0.057 $\pm$ 0.005 & 1.69 $\pm$ 0.04 & 1.5 $\pm$ 0.1 & 5.702 $\pm$ 0.002 & 0.19 $\pm$ 0.06 & 0.24 $\pm$ 0.02 \\
\hline
LGS1 & 4.531  & 0.037 $\pm$ 0.008 & 1.84 $\pm$ 0.02 & 1.89 $\pm$ 0.06 & 1.871 $\pm$ 0.004 & 1.8 $\pm$ 0.2 & 0.237 $\pm$ 0.005 \\
Cooldown 3 & 4.915 & 2.86 $\pm$ 0.02 & 2.38 $\pm$ 0.02 & 5.3 $\pm$ 0.2 & 2.649 $\pm$ 0.009 & 0.34 $\pm$ 0.03 & 0.195 $\pm$ 0.002 \\
\hline
LGS2 & 4.501 & 0.10 $\pm$ 0.02 & 1.73 $\pm$ 0.03 & 1.79 $\pm$ 0.1 & 1.199 $\pm$ 0.004 & 0.4 $\pm$ 0.1 & 0.175 $\pm$ 0.009 \\
Cooldown 4 & 4.884 & 0.48 $\pm$ 0.03 & 5.01 $\pm$ 0.05 & 5.6 $\pm$ 0.4 & 1.46 $\pm$ 0.01 & 0.17 $\pm$ 0.04 & 0.151 $\pm$ 0.006 \\
 & 5.267 & 0.110 $\pm$ 0.009 & 2.38 $\pm$ 0.05 & 2.5 $\pm$ 0.2 & 1.892 $\pm$ 0.004 & 0.10 $\pm$ 0.04 & 0.18 $\pm$ 0.01 \\
 & 5.661 & 0.19 $\pm$ 0.02 & 1.98 $\pm$ 0.02 & 2.1 $\pm$ 0.3 & 3.28 $\pm$ 0.01 & 1.0 $\pm$ 0.2 & 0.212 $\pm$ 0.008 \\
 & 6.068 & 0.17 $\pm$ 0.01 & 1.73 $\pm$ 0.02 & 1.9 $\pm$ 0.1 & 2.088 $\pm$ 0.005 & 3.6 $\pm$ 0.8 & 0.21 $\pm$ 0.01 \\
 & 6.459 & 0.13 $\pm$ 0.02 & 1.60 $\pm$ 0.03 & 1.8 $\pm$ 0.2 & 6.14 $\pm$ 0.03 & 1.4 $\pm$ 0.4 & 0.21 $\pm$ 0.01 \\
 & 6.844 & 1.70 $\pm$ 0.02 & 1.75 $\pm$ 0.02 & 3.4 $\pm$ 0.3 & 5.71 $\pm$ 0.02 & 1.0 $\pm$ 0.2 & 0.207 $\pm$ 0.008 \\
 & 7.250 & 0.19 $\pm$ 0.02 & 2.28 $\pm$ 0.03 & 2.4 $\pm$ 0.4 & 3.76 $\pm$ 0.01 & 0.13 $\pm$ 0.03 & 0.201 $\pm$ 0.008 \\
\hline\hline
SGS1 & 4.488 & 0.22 $\pm$ 0.01 & 1.41 $\pm$ 0.03 & 1.56 $\pm$ 0.08 & 1.184 $\pm$ 0.004 & 1.5 $\pm$ 0.3 & 0.236 $\pm$ 0.008 \\
Cooldown 2 & 5.682 & 5.92 $\pm$ 0.01 & 1.68 $\pm$ 0.03 & 8 $\pm$ 1 & 1.631 $\pm$ 0.002 & 0.23 $\pm$ 0.05 & 0.24 $\pm$ 0.01 \\
& 6.476 & 2.535 $\pm$ 0.006 & 1.64 $\pm$ 0.03 & 4.3 $\pm$  0.4 & 1.663 $\pm$ 0.002 & 0.32 $\pm$ 0.07 & 0.26 $\pm$ 0.01 \\
\hline
SGS1 & 4.487$^{\ast}$ & 0.38 $\pm$ 0.02 & 2.16 $\pm$ 0.09 & 3.3 $\pm$ 0.7 & 1.244 $\pm$ 0.008 & 9 $\pm$ 7 & 0.27 $\pm$ 0.06  \\
Cooldown 4 & 4.902 & 0.21 $\pm$ 0.01 & 2.31 $\pm$ 0.03 & 2 $\pm$ 1 & 1.85 $\pm$ 0.01 & 15 $\pm$ 3 & 0.20 $\pm$ 0.01 \\
& 5.267 & 0.128 $\pm$ 0.010 & 2.64 $\pm$ 0.04 & 2.5 $\pm$ 0.4 & 1.68 $\pm$ 0.01 & 3.4 $\pm$ 0.9 & 0.29 $\pm$ 0.02 \\
& 5.681$^{\ast}$ & 5.86 $\pm$ 0.06 & 2.06 $\pm$ 0.06 & 7.8 $\pm$ 1.0 & 1.576 $\pm$ 0.004 & 0.5 $\pm$ 0.2 & 0.25 $\pm$ 0.02  \\
& 6.037 & 0.395 $\pm$ 0.007 & 1.28 $\pm$ 0.02 & 1.7 $\pm$ 0.3 & 1.659 $\pm$ 0.004 & 0.9 $\pm$ 0.2 & 0.28 $\pm$ 0.02 \\
& 6.476$^{\ast}$ & 1.96 $\pm$ 0.01 & 1.77 $\pm$ 0.03 & 3.8 $\pm$ 0.7 & 1.522 $\pm$ 0.006 & 0.9 $\pm$ 0.3 & 0.27 $\pm$ 0.02  \\
& 6.833 & 0.29 $\pm$ 0.02 & 2.52 $\pm$ 0.08 & 2.9 $\pm$ 0.04 & 1.99 $\pm$ 0.02 & 0.13 $\pm$ 0.06 & 0.21 $\pm$ 0.02 \\
& 7.245 & 0.39 $\pm$ 0.01 & 1.75 $\pm$ 0.04 & 2.1 $\pm$ 0.3 & 1.269 $\pm$ 0.008 & 0.9 $\pm$ 0.2 & 0.29 $\pm$ 0.02 \\
\hline
SGS2 & 4.499 & 0.65 $\pm$ 0.01 & 2.5 $\pm$ 0.2 & 2.9 $\pm$ 0.5 & 2.106 $\pm$ 0.006 & 0.020 $\pm$ 0.018 & 0.21 $\pm$ 0.02 \\
Cooldown 4 & 4.907 & 0.49 $\pm$ 0.01 & 3.28 $\pm$ 0.08 & 3.9 $\pm$ 0.3 & 2.502 $\pm$ 0.007 & 0.210 $\pm$ 0.090 & 0.21 $\pm$ 0.02 \\
 & 5.292 & 0.537 $\pm$ 0.007 & 1.93 $\pm$ 0.03 & 2.4 $\pm$ 0.2 & 3.789 $\pm$ 0.005 & 1.0 $\pm$ 0.3 & 0.27 $\pm$ 0.02 \\
 & 6.074 & 0.93 $\pm$ 0.02 & 2.13 $\pm$ 0.03 & 2.9 $\pm$ 0.3 & 6.609 $\pm$ 0.008 & 0.20 $\pm$ 0.05 & 0.26 $\pm$ 0.01 \\
 & 6.460 & 0.54 $\pm$ 0.02 & 1.52 $\pm$ 0.02 & 2.1 $\pm$ 0.4 & 5.91 $\pm$ 0.01 & 0.16 $\pm$ 0.04 & 0.20 $\pm$ 0.01 \\
 & 6.591 & 1.52 $\pm$ 0.03 & 3.11 $\pm$ 0.06 & 4 $\pm$ 1 & 54.9 $\pm$ 0.2 & 0.10 $\pm$ 0.03 & 0.24 $\pm$ 0.02 \\
 & 7.196 & 0.96 $\pm$ 0.02 & 1.92 $\pm$ 0.05 & 2.9 $\pm$ 0.5 & 4.25 $\pm$ 0.03 & 0.04 $\pm$ 0.02 & 0.21 $\pm$ 0.02
\end{tabular}
\end{ruledtabular}
\end{table*}

% % Further Statistical Analyses
% \section{Further Statistical Analyses}
% Figure~\ref{fig:bootstrapped_histograms} shows histograms of the prior (measured) and estimated posterior (computed) probability distributions of the intrinsic TLS loss of the large and small grain size resonators. To estimate the posterior distributions, we performed a Bayesian bootstrapping analysis, resampling the prior distribution with replacement 10000 times~\cite{Rubin1981}. The samples are weighted by the continuous Dirichlet distribution and with non-zero weights assigned only to the observed priors. For a more technical discussion of the Bayesian bootstrap and classical bootstrap, see Ref.~\cite{Job2018}.
% 
% This method of estimating the posterior mean assumes that all values of the measured quantity have been observed. In the small grain size measurements, it appears that this assumption is valid, but the presence of an outlier in the large grain size measurements suggests otherwise. We present this approach here as a guideline for future statistical analysis where the number of measured devices for A and B (small and large grain size in this work) exhibit more dense filling near the mean, i.e. the sample (prior) distributions lack outliers and are representative of population (posterior) distributions.

% % Figure environment removed

% Microwave Setup
\section{Microwave Setup}
In Fig.~\ref{fig:wiring_diagram}, we give a schematic of the microwave wiring in our Janis JDry 250 dilution refrigerator. A Keysight PNA N5222B vector network analyzer (VNA) transmits signals down input lines A and B, with 60 dB of discrete attenuation supplied by XMA 2082-6418-dB-CRYO cryogenic attenuators. We estimate an additional 10 dB of attenuation from internal and external line losses.

Two Quinstar QCE-060400CM00 circulators separate input and output paths outside of two six-to-one Radiall 583 microwave switches. Additional directionality on the output lines is achieved with two pairs of 20 dB Quinstar QCI-080090XM00 isolators. Two LNF-LNC4\_8C high electron mobility transistor (HEMT) amplifiers with average noise temperatures of 1.5 K and gain of 40 dB, mounted at the 3 K stage, set the noise floor of our experiments. Room temperature low noise amplifiers (Miteq AFS4-04001200-48-20P-4), provide an additional 30 dB of gain on the receiver side. DC-blocks (Mini-Circuits BLKD-183-S+) decouple DC currents from the input and output lines at room temperature. 
%
% Figure environment removed

\section*{References}
% Create the reference section using BibTeX:
\bibliography{TaGrowth2023.bib}

\end{document}
%
% ****** End of file aiptemplate.tex ******