\documentclass[11pt,oneside,a4paper,reqno]{amsart}

%\usepackage[left=1.3in,right=1.3in,bottom=1.2in,top=1.2in]{geometry}
%\usepackage[left=1in,right=1in,bottom=1.2in,top=1.2in]{geometry}
%-------Packages---------
\usepackage{tikz}

%\usepackage{amsthm,amsmath,amstext,mathtools,graphicx}

\renewcommand{\baselinestretch}{1.1}

\usepackage{hyperref}
%\usepackage{refcheck}
%\hypersetup{
%colorlinks,
%citecolor=green,
%filecolor=black,
%linkcolor=blue,
%urlcolor=blue,
%}
%\usepackage{bbold}
\usepackage{bm}
%\usepackage{bbold}
\usepackage{amsmath}
%\usepackage{mathrsfs}  
\usepackage{amsthm}
\usepackage{amssymb}
%\usepackage{authblk}
%\usepackage{graphicx}

\newcommand{\mathscr}{\mathcal}

\newcommand{\mn}{\mathrm{min}}
\newcommand{\mx}{\mathrm{max}}
\newcommand{\half}{\frac{1}{2}}
\newcommand{\calc}{\mathcal{C}}
\renewcommand{\proofname}{{\bf Proof}}


\newcommand{\rr}{{\mathbb R}}
\newcommand{\cc}{{\mathbb C}}
\newcommand{\sss}{{\mathbb S}}




\numberwithin{equation}{section}

\DeclareMathOperator{\dom}{dom}
\DeclareMathOperator{\spec}{spec}
\DeclareMathOperator{\supp}{supp}
\DeclareMathOperator{\ran}{ran}
\DeclareMathOperator{\pv}{p.v.}

\newcommand{\specd}{\spec_\mathrm{disc}}
\newcommand{\spece}{\spec_\mathrm{ess}}
\newcommand{\specp}{\spec_\mathrm{p}}

\newcommand{\cD}{{\mathcal D}}
\newcommand{\cF}{{\mathcal F}}
\newcommand{\cS}{{\mathcal S}}

\newcommand{\eps}{\varepsilon}




\def\F{\mathcal{F}_{\overline{x}}}
\def\var{\varphi}


%%%%%%%%%%%%%%%%%%%les ensembles%%%%%%%%%%%%%%%%
\def\R{{\mathord{I\!\! R}}}
\def\C{{\mathord{I\!\!\!\!C}}}
\def\I{{\mbox{\rm I$\!$I}}}
\def\N{{\mbox{\rm I$\!$N}}}
%\def\mathbb{C}{{\cal C}}
\def\H{{\cal H}}
%\def \mathbb{C} {{\mathbb{C}}}

\newtheorem{theorem}{Theorem}[section]
\newtheorem{proposition}[theorem]{Proposition}
\newtheorem{lemma}[theorem]{Lemma}
\newtheorem{corollary}[theorem]{Corollary}

\theoremstyle{definition}
\newtheorem{definition}[theorem]{Definition}
\newtheorem{remark}[theorem]{Remark}
\newtheorem{notation}[theorem]{Notation}
\newtheorem{question}[theorem]{Question}

%\theoremstyle{theorem}
%\makeatletter
%\makeatother

%\def\theequation{\arabic{section}.\arabic{equation}}
%\numberwithin{equation}{section}
%\bibliographystyle{plain}
%--------Meta Data: Fill in your info------

\title[]{On the self-adjointness of two-dimensional relativistic shell interactions}

\author{Badredine Benhellal, Konstantin Pankrashkin, Mahdi Zreik}

\address{(B. Benhellal) Carl von Ossietzky Universit\"at Oldenburg, Institut f\"ur Mathematik, 26111 Oldenburg, Germany, E-mail:
\url{badreddine.benhellal@uol.de}}

\address{(K. Pankrashkin) Carl von Ossietzky Universit\"at Oldenburg, Institut f\"ur Mathematik, 26111 Oldenburg, Germany,
E-mail: \url{konstantin.pankrashkin@uol.de}}

\address{(M. Zreik) Institut de Math\'ematiques de Bordeaux, UMR 5251, Universit\' e de Bordeaux 33405 Talence Cedex,  France\\ and Departamento de Matem\'aticas, Universidad del Pa\' is Vasco, Barrio Sarriena s/n 48940 Leioa, Spain, E-mail:
\url{mahdi.zreik@math.u-bordeaux.fr}, \url{mahdi.zreik@ikasle.ehu.eus}}

\subjclass[2010]{81Q10, 81V05, 35P15, 58C40}

\keywords{}

%======================================================================================================================================================================================
	\begin{document}
	\begin{abstract}  We study the self-adjointness of the two-dimensional Dirac operator coupled with electrostatic and Lorentz scalar shell interactions of constant strength $\varepsilon$ and $\mu$ supported on a closed Lipschitz curve. Namely, we present several new explicit ranges of $\varepsilon$ and $\mu$ for which there is a unique self-adjoint realization with domain included into $H^{\frac{1}{2}}$. A more precise analysis is carried out
	for curvilinear polygons, which allows one to take the corner openings into account. Compared to the preceding works 
	on this topic, two new technical ingredients are employed: the explicit use of the Cauchy transform on non-smooth curves and the explicit characterization of the Fredholmness for singular integral operators.
	\end{abstract}
	
\maketitle	
%\tableofcontents


\tableofcontents

\section{Introduction}

Dirac operators with $\delta$-interactions supported on general hypersurfaces have been actively studied since the appearance of the paper \cite{AMV1}. Due to the presence of distributional coefficients, the self-adjointness of such operators requires special attention, and it was seen by many authors (primarily for the three-dimensional case) that the self-adjointness domain can be dependent on the coupling constants and the smoothness properties of the hypersurface and that it may lead to unusual spectral properties \cite{b1,b2,BB2,BB3,BP}. The paper \cite{BHOP} initiated the study of the two-dimensional case, and for the case of smooth curves a very complete spectral picture could be found, which was extended in \cite{CLMT} to a more general class of interactions. Much less attention was given to the case of non-smooth surfaces and curves. In the present work, we discuss the self-adjointness of two-dimensional Dirac operators with  $\delta$-interactions supported on closed Lipschitz curves (in particular, on curvilinear polygons). Our results complement those obtained in the recent papers \cite{BHSS,PV} and provide precise ranges of coupling constants and corner openings for which the domain of self-adjointness can be given explicitly. Compared to the preceding works, we employ two new technical ingredients: the explicit use of the Cauchy transform on non-smooth curves and a characterization of the Fredholmness for boundary integral operators using the approach of \cite{VYS}.

Now let us pass to precise formulations. Through the text, we use the Pauli matrices
\[
\sigma_{1}=\left(\begin{array}{cc}
0 & 1 \\
1 & 0
\end{array}\right) ,\quad \sigma_{2}=\left(\begin{array}{cc}
0 & -i \\
i & 0
\end{array}\right) ,\quad \sigma_{3}=\left(\begin{array}{cc}
1& 0 \\
0 & -1
\end{array}\right)
\]
and denote by $\sigma_0$ the $2\times 2$ identity matrix. The anticommutation relations
\begin{align}\label{anticmu}
    \lbrace \sigma_{j},\sigma_{k}\rbrace = \sigma_{j}\sigma_{k}+\sigma_{k}\sigma_{j}=2\delta_{jk}\sigma_0 \text{ for all }j,k\in \lbrace 1,2,3\rbrace
\end{align}
are well known. Let $m\in\rr$. The two-dimensional Dirac operator with mass $m$  is the formally self-adjoint differential expression
\[
D:\ C^\infty_0(\rr^2,\cc^2)\ni f\ \mapsto\  -i(\sigma_1\partial_1 f+\sigma_2\partial_2 f)+m\sigma_3 f\in C^\infty_0(\rr^2,\cc^2),
\]
and it naturally extends to a continuous linear map in the space of distributions $\cD'(\Omega,\cc^2)$ for any open set $\Omega\subset\rr^2$.
It is well known that the operator
\begin{equation}
	\label{adom}
A f\mapsto Df, \quad \dom A=H^1(\rr^2,\cc^2),
\end{equation}
(the free two-dimensional Dirac operator),  is self-adjoint in $L^2(\rr^2,\cc^2)$ and has the absolutely continuous spectrum
\[
\spec A=\big(-\infty,-|m|\big]\,\cup\big[|m|,+\infty\big),
\]
and it occupies a central place in relativistic quantum mechanics \cite{Tha}. We will be interested in the study of some special perturbations of $A$.


Namely, let $\Omega_+\subset\rr^2$ be a non-empty bounded open set with Lipschitz boundary. Denote
\[
\Sigma:=\partial\Omega_+,\quad \Omega_-:=\rr^2\setminus\overline{\Omega_+}.
\]
For $(\eps,\mu)\in\rr^2$ we would like to discuss self-adjoint realizations in $L^2(\rr^2,\cc^2)$ 
of operators given formally by
\begin{equation}
	\label{dirac2}
f\mapsto Df+(\eps\sigma_0 + \mu \sigma_3)\delta_\Sigma f,
\end{equation}
where $\delta_\Sigma$ is the Dirac $\delta$-distribution supported on $\Sigma$. The last summand can be considered as an idealized model of a relativistic potential concentrated on $\Sigma$, and the constant $\eps$ resp. $\mu$ measures the strength of the electrostatic resp. Lorentz scalar part of the interaction.
The formal expression \eqref{dirac2} can be given a more rigorous meaning as follows.
 First, for any non-empty open set $\Omega\subset\rr^2$ consider the space
\[
H(\sigma,\Omega):=\Big\{f\in L^2(\Omega,\cc^2):\ Df\in L^2(\Omega,\cc^2)
\Big\},
\]	
which is just the domain of the maximal realization of $D$ in $L^2(\Omega,\cc^2)$ and becomes a Hilbert space if equipped with the scalar product
\[
\langle f,g\rangle_{H(\sigma,\Omega)}:=\langle f,g\rangle_{L^2(\Omega,\cc^2)}+\langle Df,Dg\rangle_{L^2(\Omega,\cc^2)}.
\]
For $s>0$ let $H^s(\Omega,\cc^2)$ be the usual fractional Sobolev spaces of order $s$ on $\Omega$ (consisting of $\cc^2$-valued functions), and we set
\[
H^s(\sigma,\Omega):=H(\sigma,\Omega)\cap H^s(\Omega,\cc^2),
\]
which is a Hilbert space with the scalar product
\[
\langle f,g\rangle_{H^s(\sigma,\Omega)}:=\langle f,g\rangle_{H(\sigma,\Omega)}+\langle f,g\rangle_{H^s(\Omega,\cc^2)}.
\]

For what follows it will be convenient to use the identification
\[
H(\sigma,\rr^2\setminus\Sigma)\simeq H(\sigma,\Omega_+)\oplus H(\sigma,\Omega_-),
\quad
f\simeq (f_+,f_-),
\]
with $f_\pm$ being the restriction of $f$ on $\Omega_\pm$, as well as the analogous identifications for $H^s(\rr^2\setminus\Sigma,\cc^2)$
and $H^s(\sigma,\rr^2\setminus\Sigma)$. We will also use the shorthand notation
\[
\sigma\cdot x:=x_1\sigma_1+x_2\sigma_2,\quad x=(x_1,x_2)\in\rr^2;
\]
from the anticommutation relations \eqref{anticmu} one easily obtains $(\sigma\cdot x)^2=|x|^2\sigma_0$ for all $x\in\rr^2$.

It is known that for any $f\in H(\sigma,\rr^2\setminus\Sigma)$ the boundary traces $(\sigma\cdot \nu)f_\pm$ on $\Sigma$ are well-defined as functions in $H^{-\frac{1}{2}}(\Sigma)$; remark that we keep the same symbols for the boundary traces for better readability. Denote by $\delta_\Sigma f$ the distribution
\[
\langle \delta_\Sigma f, \varphi\rangle :=\int_\Sigma \dfrac{f_+ + f_-}{2}\,\varphi \,\mathrm{d} s, \quad \varphi\in C^\infty_c(\rr^2),
\]
where $\mathrm{d}s$ means the integration with respect to the arclength. An application of the jump formula shows the identity
\begin{align}\label{distjump}
D f= (Df_+)\oplus (Df_-)+i(\sigma\cdot\nu )(f_+-f_-)\delta_\Sigma,
\end{align}
where $\nu=(\nu_1,\nu_2)$ is the unit normal on $\Sigma$ pointing to $\Omega_-$. Then it follows that the right-hand side of \eqref{dirac2}
belongs to $L^2(\rr^2,\cc^2)$ if and only if $f$ satisfies the transmission condition
\begin{align}\label{TC}
	 (\eps \sigma_0 + \mu\sigma_{3})\frac{f_+ + f_-}{2}+  i(\sigma\cdot\nu )(f_+-f_-) = 0 \,\text{ on } \Sigma.
\end{align}

Therefore, as a first attempt, it is natural to consider the following operator realizations of the expression \eqref{dirac2} in $L^2(\rr^2,\cc^2)$:
\begin{itemize}
	\item the maximal realization $B_\mx$ with the domain
	\[
	\dom B_\mx:=\big\{f\in H(\sigma,\rr^2\setminus\Sigma):\ f \text{ satisfies \eqref{TC}}\big\},
	\]
	\item the minimal realization $B_\mn$ with the domain
\begin{align*}
\dom B_\mn&:=\dom B_\mx\cap H^1(\rr^2\setminus \Sigma,\cc^2)\\
&\equiv\big\{f\in H^1(\rr^2\setminus\Sigma,\cc^2):\ f \text{ satisfies \eqref{TC}}\big\}.
\end{align*}
\end{itemize}
It is standard to see that $B_\mn$ is symmetric with $B_\mn^*=B_\mx$, therefore, $B_\mn\subset B\subset B_\mx$ for any self-adjoint realization $B$ of \eqref{dirac2}. Nevertheless, an explicit description of the self-adjoint realizations turns out to be an involved problem depending on both $(\eps,\mu)$ and the regularity of $\Sigma$.

The most attention was given to the case of $C^2$-smooth $\Sigma$, see \cite{BHSS} and references therein.
Namely, if $\eps^2-\mu^2\ne 4$, then $B_\mn=B_\mx=:B$, and the spectrum of $B$ consists of the spectrum of the free Dirac operator $A$ and at most finitely many discrete eigenvalues in $(-|m|,|m|)$. For $\eps^2-\mu^2=4$ the operator $B_\mn$ is not closed, but $\overline{B_\mn}=B_\mx$, so $B_\mn$ is at least essentially self-adjoint (so there is a unique self-adjoint realization), but the loss of  regularity leads to peculiar spectral effects (e.g. new pieces of the essential spectrum), see \cite{BHOP,BHSS,BP}. Remark that \cite{BHSS,CLMT} actually consider more general interactions
by admitting so-called anomalous magnetic couplings which are not covered by the above framework.

If $\Sigma$ has corners, one has, in general, $\overline{B_\mn}\subsetneq B_\mx$, which means that there are infinitely many self-adjoint realizations \cite{OP}. The work \cite{OP} suggested that the $H^\frac{1}{2}$ regularity should be more natural for the case of non-smooth $\Sigma$. Namely, let
\[
B\equiv B_{\eps,\mu}
\]
be the restriction of $B_\mx$ to $\dom B_\mx\cap H^{\frac{1}{2}}(\rr^2\setminus\Sigma,\cc^2)$, i.e.
\begin{equation}
	\label{bdom}
	\begin{aligned}
		B&:\ f\simeq(f_+,f_-)\mapsto (D f_+,Df_-),\\
		\dom B&:=\Big\{
		f\in H^\half(\sigma,\rr^2\setminus\Sigma):\ f\text{ satisfies \eqref{TC}}
		\Big\}.
	\end{aligned}	
\end{equation}
Due to the standard Sobolev traces theorem, the one-sided traces of functions from $\dom B$ on $\Sigma$ belong to $L^2(\Sigma,\cc^2)$, so the integration by parts shows that $B$ is a symmetric operator. The main result of \cite{PV} reads as follows: if $\Sigma$ is a curvilinear polygon (a piecewise $C^2$-smooth closed curve, with finitely many corners and without cusps), $\eps=0$ and $|\mu|<2$, then $B$ is self-adjoint. The recent work \cite{BHSS}  presents an extensive study of the case of general
compact  Lipschitz curves $\Sigma$ by reducing the self-adjointness to the Fredholmness of some boundary integral operator (see also \cite{AMV1, BB3} for the three-dimensional case):
we summarize the essential components of the constructions in Section~\ref{secprep}. Nevertheless, the self-adjoint conditions obtained in \cite{BHSS} for our case are quite implicit as they depend on the (unknown) spectra of some boundary integral operators.



In the present work we extend the results of both \cite{BHSS} and \cite{PV} by providing new very explicit conditions for the self-adjointness of $B$ in terms of the parameters $(\eps,\mu)$ and the geometry of $\Sigma$. Namely, we show that $B$ is self-adjoint in the following cases:
\begin{itemize}
	\item[(A)] The curve $\Sigma$ is Lipschitz and $|\eps|\le |\mu|$ (Corollary \ref{bleq}),
	\item[(B)] The curve $\Sigma$ is $C^1$-smooth and $\eps^2-\mu^2\ne 4$ (Theorem \ref{thmc1}),
	\item[(C)] The curve $\Sigma$ is a curvilinear polygon (with $C^1$-smooth edges and without cusps) and
	\[
	\eps^2-\mu^2<\frac{1}{m (\omega)} \text{ or } \eps^2-\mu^2>16 m(\omega),
	\]
	where the constant $m(\omega)$ only depends on the sharpest corner $\omega$ of $\Sigma$ (Theorem \ref{thm52}).
	
	The value of $m(\omega)$ is not known explicitly for all $\omega$, but some bounds can be obtained,
	and each of the conditions
	\begin{itemize}
		\item[(i)] $\eps^2-\mu^2<2$ or $\eps^2-\mu^2>8$  (without additional geometric assumptions),
		\item[(ii)] $\eps^2-\mu^2\ne 4$ if each angle $\theta$ of $\Sigma$ (measured inside $\Omega_+$)  satisfies
		\[
		\dfrac{\pi}{2}\le \theta\le\dfrac{3\pi}{2},
		\]
	\end{itemize}	
	guarantees the self-adjointness of $B$ (Corollary \ref{momega}).
\end{itemize}

The case (B) is formally contained in (C.ii), but the proofs are very different, so we prefer to consider these two situations separately.


\begin{remark} If the operator $B$ is self-adjoint, a standard analysis shows that its essential spectrum coincides with the spectrum of the free Dirac operator $A$ and that the discrete spectrum is at most finite \cite[Proposition 3.8]{BHOP}. While all constructions of~\cite{BHOP} are formally for smooth $\Sigma$, the proof of this specific result only uses the compact embedding of $H^s(\Omega)$ to $L^2(\Omega)$ for $s>0$ and bounded open sets $\Omega\subset\rr^2$ with Lipschitz boundaries.
\end{remark}	

\begin{remark}\label{rmk2}
An additional useful property is that for any $(\eps,\mu)$ with $|\eps|\ne|\mu|$ 
the operator $B_{\eps,\mu}$ is unitarily equivalent to $B_{-\frac{4\eps}{\eps^2-\mu^2},-\frac{4\mu}{\eps^2-\mu^2}}$. Namely,
a simple direct computation shows that
\[
	B_{\eps,\mu}U=UB_{-\frac{4\eps}{\eps^2-\mu^2},-\frac{4\mu}{\eps^2-\mu^2}}
\]
	for the unitary linear map $U:\ L^2(\rr^2,\cc^2)\to L^2(\rr^2,\cc^2)$
	defined by
	\[
	U:\ (f_+,f_-)\mapsto (f_+,-f_-),
	\] see \cite[Propositon 4.8]{BHOP}. In particular, the self-adjointness of $B_{-\frac{4\eps}{\eps^2-\mu^2},-\frac{4\mu}{\eps^2-\mu^2}}$ is equivalent to the self-adjoitness of $B_{\eps,\mu}$,
which will be used in the last proof steps.	
\end{remark}



\section{Preparations for the proof}\label{secprep}

We will need some constructions related to the free Dirac operator $A$ in \eqref{adom}. Most of these required results were already obtained in \cite{BHOP,BHSS} and we simply present them
in an adapted form.


First of all, we consider the Cauchy transform on $\Sigma$, i.e. the linear operator $C_\Sigma:\mathit{L}^2(\Sigma)  \longrightarrow \mathit{L}^2(\Sigma)$  defined through the complex line integration
\begin{equation*}%\label{Cauchy1}
	C_\Sigma g(x):= \dfrac{i}{2\pi}\,\pv\int_{\Sigma}\dfrac{g(y)}{x-y}\,\mathrm{d}y,\quad g\in L^2(\Sigma),\quad x\in \Sigma,
\end{equation*}
and understood in the Cauchy principal value sense. It is a classical result that $C_\Sigma$ is well-defined and bounded \cite{CMM}. Moreover, if one considers the analytic function
\[
F_g:\ \cc\setminus\Sigma\simeq\rr^2\setminus\Sigma\ni x\mapsto \dfrac{i}{2\pi}\,\pv\int_{\Sigma}\dfrac{g(y)}{x-y}\,\mathrm{d}y,\quad g\in L^{2}(\Sigma),
\]
then Plemelj-Sokhotski formulas are valid:
\[
F_g(x)=\pm\frac{g(x)}{2}+C_\Sigma g(x) \text{ for a.e. $x\in\Sigma$},
\]
where the value on the left-hand side is understood as the non-tangential limit  \cite[p.~108]{jj}.

Denote by $K_j$ the modified Bessel functions of order $j$. For $z\in\cc\setminus\spec A$ consider the function $\phi_z:\rr^2\to M_{2\times 2}(\cc)$ given by
\begin{multline*}
    \phi_z (x):= \dfrac{1}{2\pi} K_{0}\big( \sqrt{ m^{2} - z^{2}} |x| \big)\big( m\sigma_{3} + z\sigma_0\big)\\
    {} + i\dfrac{\sqrt{ m^{2} - z^{2}}}{2\pi |x|} K_{1}\big( \sqrt{ m^{2} - z^{2}} |x| \big)( \sigma\cdot x).
\end{multline*}
It will be convenient to admit the additional value $z=m$ by setting
\begin{align*}
     \phi_m (x):=\frac{i}{2\pi}\left(\begin{array}{cc}
0& \dfrac{1}{x_1+ix_2} \\
\dfrac{1}{x_1-ix_2} & 0
\end{array}\right).
\end{align*}
Using the asymptotic expansions of $K_j$ one obtains
\begin{equation} 
	 \label{fifi}
	\phi_z(x) = \phi_m(x) + h_1(x) \log|x| +h_2(x).
\end{equation}
with continuous functions $h_j$, see \cite[Lemma 3.3]{BHOP} for details.

For all admissible $z$ the function $\phi_z$ is a fundamental solution of $D-z$,
and it gives rise to several (singular) integral operators.


Namely, consider the layer potentials $\Phi_z$ for $D-z$ (with $z\in\cc\setminus\spec A$)
\begin{align*}%\label{Phi}
\begin{split}
\Phi_{z}:&\  L^2(\Sigma,\cc^{2})  \longrightarrow \mathit{L}^2(\rr^2,\cc^2),\\
\Phi_{z}g(x) =&\int_{\Sigma} \phi_{z}(x-y)g(y)\,\mathrm{d}s(y), \quad  x\in\rr^2\setminus\Sigma,
\end{split}
\end{align*} 
where we recall that $\mathrm{d}s$ means the integration with respect to the arclength.
Observe that  $\phi_{z}(x)^{\ast}=\phi_{\Bar{z}}(-x)$ for all $x$. Let $\gamma:H^\half(\rr^2,\cc^2)\to L^2(\Sigma,\cc^2)$ be the Sobolev trace operator (which is a bounded linear operator), then for any $u \in L^2(\rr^2, \cc^2)$ and $g \in L^2(\Sigma, \cc^2)$ one has, using Fubini's theorem,
\begin{align*}
	\langle \Phi_{\Bar{z}} g, u \rangle_{L^2(\rr^2, \mathbb{C}^2)}&=\int_{\rr^2} \Big\langle \int_\Sigma \phi_{\Bar{z}}(x-y)g(y)\,\mathrm{d}s(y), u(x)\Big\rangle_{\cc^2}\,\mathrm{d}x\\
	&=  \int_\Sigma \Big\langle g(y), \int_{\rr^2} \phi^{\ast}_{\Bar{z}}(x-y)u(x)\mathrm{d} x \Big\rangle_{\cc^2}\,\mathrm{d}s(y),\\
	&= \big\langle g, \gamma (A- z)^{-1}u\big\rangle_{L^2(\Sigma,\cc^2)}.
\end{align*} 
This shows that $\Phi_{\Bar z}=\big(\gamma (A- z)^{-1}\big)^*$ is bounded, and by replacing $z$ with $\Bar z$ one obtains the 
useful identity
\begin{equation}
	\label{figz}
	\Phi_z^*=\gamma (A- \bar{z})^{-1},\quad z\in\cc\setminus\spec A.
\end{equation}
Now let $\varphi\in C^\infty_0(\rr^2,\cc^2)$ and $h\in L^2(\Sigma,\cc^2)$, then
\begin{align*}
	\big\langle \Phi_z h,(D-\Bar z)\varphi\big\rangle_{L^2(\rr^2,\cc^2)}&=
	\big\langle h,\Phi_z^*(D-\Bar z)\varphi\big\rangle_{L^2(\Sigma,\cc^2)}\\
	&=\big\langle h,\gamma(D-\Bar z)^{-1}(D-\Bar z)\varphi\big\rangle_{L^2(\Sigma,\cc^2)}\\
	&=\big\langle h,\gamma\varphi\big\rangle_{L^2(\Sigma,\cc^2)},
\end{align*}
and it follows that $(D-z)\Phi_z h=0$ in $\cD'(\rr^2\setminus\Sigma)$.
In particular,
\[
\ran \Phi_z\subset \ker(B_\mx -z)\subset\dom B_\mx.
\]
In fact, for any $z\in\cc\setminus\spec A$ one has the stronger property \cite[Lemma 4.2]{BHSS}:
\begin{equation}\label{bounPhi}
	\Phi_z:\ L^2(\Sigma,\cc^2)\to H^\half(\sigma,\rr^2\setminus\Sigma) \text{ is bounded.} 
\end{equation}	


For all admissible $z$ consider the singular integral operator
\[
\calc_{z}: L^2(\Sigma,\cc^2)\longrightarrow L^2(\Sigma,\cc^2)
\]
given by
\begin{align*}%\label{Cauchy}
  \calc_{z}g(x) &= \pv \int_{\Sigma} \phi_{z}(x-y)g(y)\,\mathrm{d}s(y),\quad x\in\Sigma.
\end{align*}
To summarize its properties we introduce the tangent vector field 
\[
\tau=(\tau_1,\tau_2):=(-\nu_2,\nu_1)
\]
on $\Sigma$ and denote
\[
t:= \text{the operator of multiplication by $\tau_1+i \tau_2$ in $L^2(\Sigma)$}.
\]
Then
\begin{equation}\label{Cauchy2}
\begin{aligned}
		C_\Sigma t^*g(x)&= \dfrac{i}{2\pi}\,\pv\int_{\Sigma}\dfrac{g(y)}{(x_1-y_1) -i(x_2-y_2)}\,\mathrm{d}s(y),\\
		t C_\Sigma^*g(x)&= \dfrac{i}{2\pi}\,\pv\int_{\Sigma}\dfrac{g(y)}{(x_1-y_1) +i(x_2-y_2)}\,\mathrm{d}s(y),\quad x\in\Sigma,
	\end{aligned}
\end{equation}
and
\begin{equation}\label{IdenC}
\calc_m= 
	  \begin{pmatrix} 
	       0& C_\Sigma t^*\\ 
	tC_\Sigma^* & 0 \, 
	   \end{pmatrix}.
\end{equation}
Therefore, the boundedness of $C_\Sigma$ implies the boundedness of $\calc_m$. In addition, the expansion \eqref{fifi} shows that
$\calc_z-\calc_m$ is an integral operator with a Hilbert-Schmidt kernel, in particular,
\[
\calc_z-\calc_m:\ L^2(\Sigma,\cc^2)\to L^2(\Sigma,\cc^2) \text{ is compact for any $z\in\cc\setminus \spec A$,}
\]
which also shows the well-definedness  and boundedness of $\calc_z$ for all admissible $z$.

Let $\gamma_\pm:H^\half(\Omega_\pm)\to L^2(\Sigma)$ be the Sobolev trace operators,
and for any $f\in H^\half(\rr^2\setminus\Sigma)$ we set
\[
\gamma_\pm f:=\gamma_\pm f_\pm,
\]
then one has the so-called jump formula   
\begin{align}\label{traceof Phi}
	\gamma_\pm \Phi_{z}g=  \left(\mp\frac{i}{2}\sigma\cdot\nu + \calc_z\right)g, \quad g\in L^2(\Sigma,\cc^2).
\end{align}
In \cite[Proposition 3.5]{BHOP} the jump formula was proved under the formal assumption that $\Sigma$ is $C^\infty$ smooth, but the same proof applies to our case as well, as the Plemelj-Sokhotski formula used in the proof
also holds for closed Lipschitz curves. From the jump formula \eqref{traceof Phi} one obtains
\[
g= i(\sigma\cdot\nu)\Big[\gamma_+ \Phi_z g-\gamma_-\Phi_z g \Big],
\quad
g\in L^2(\Sigma,\cc^2),
\]
which shows the injectivity of $\Phi_z$. Further direct consequences of the jump formula are the identities
\begin{equation}
	\label{jump2}
	\begin{aligned}
\gamma_+ \Phi_z g -\gamma_-\Phi_z g&=-i(\sigma\cdot\nu)g,\\
\dfrac{\gamma_+\Phi_z g +\gamma_-\Phi_z g}{2}&=\calc_z g,\quad g\in L^2(\Sigma,\cc^2).
\end{aligned}
\end{equation}

For $z\in(\cc\setminus\spec A)\cup\{m\}$ consider the bounded linear operator
\[
\Theta_z:=I+(\eps\sigma_0+\mu\sigma_3)\calc_{z}:\ L^2(\Sigma,\cc^2)\to L^2(\Sigma,\cc^2),
\]
which is closely related to the operator $B$ from \eqref{bdom} as follows:

\begin{lemma} \label{lem-discr}
For any $z\in\cc\setminus \spec A$	there holds
$\ker(B-z)=\Phi_z\ker\Theta_z$, in particular,
$\dim\ker(B-z)=\dim\ker\Theta_z$.
\end{lemma}

\begin{proof} 
Remark that the last assertion follows from the injectivity of $\Phi_z$.
	
	
	Let $z\in\cc\setminus \spec A$ and $g\in\ker \Theta_z$. Denote $f:=\Phi_{z}g$, then $f\in\ker(B_\mx-z)$
	due to the above properties of $\Phi_z$. We need to show $f\in\dom B$.
	By \eqref{bounPhi} we have already $f\in H^\half(\sigma,\rr^2\setminus\Sigma)$. By \eqref{jump2}
	we have
\begin{align*}
		(\eps\sigma_0+\mu \sigma_3)	&\dfrac{\gamma_+\Phi_z g +\gamma_-\Phi_z g}{2}
		+i(\sigma\cdot\nu)\big(\gamma_+ \Phi_z g -\gamma_-\Phi_z g\big)\\
		&=(\eps\sigma_0+\mu \sigma_3)\calc_z g+i(\sigma\cdot\nu)\big(-i (\sigma\cdot\nu)\big)g\\
		&=(\eps\sigma_0+\mu \sigma_3)\calc_z g+g=\Theta_z g=0.
\end{align*}
Hence, $f\in\ker(B-z)$. This shows the inclusion $\Phi_z\ker \Theta_z\subset \ker(B-z)$.
	
	
Now let $z\in\cc\setminus \spec A$ and $f\in \ker (B-z)$. Due to \eqref{distjump} we have
	\begin{equation}
		\label{temp5}
	(D-z)f=(B-z)f+i(\sigma\cdot\nu)(f_+ - f_-)\delta_{\Sigma}.
	\end{equation}
Let $\cF:\cS'(\rr^2)\to\cS'(\rr^2)$	be the Fourier transform. For any $\psi\in\cS'(\rr^2)$ we have 
\[
\cF (D-z)\psi=(\sigma\cdot\xi+m\sigma_3-z\sigma_0)\cF \psi.
\]
The matrix $\sigma\cdot\xi+m\sigma_3-z\sigma_0$ is invertible for any $\xi\in\rr^2$ and has polynomial entries, which shows that $D-z:\cS'(\rr^2)\to\cS'(\rr^2)$ is injective. As the function $\phi_z\in \cS'(\rr^2)$ is a fundamental solution of $D-z$, from \eqref{temp5} one obtains
\[
f=\phi_z\ast \big[i(\sigma\cdot\nu)(f_+ - f_-)\delta_{\Sigma}\big].
\]
Due to $f\in\dom B$ we have $f_\pm\in H^\half(\Omega_\pm,\cc^2)$, and, hence 

\[
g:=i(\sigma\cdot\nu)(\gamma_+ f - \gamma_- f)\in L^2(\Sigma,\cc^2).
\] Then
\[
f=\phi_z\ast g=\int_\Sigma\phi_z(\cdot-y)g(y)\,\mathrm{d}s(y)\equiv \Phi_z g.
\]
With the help of \eqref{jump2} we obtain
\begin{align*}
	0&=(\eps\sigma_0+\mu\sigma_3)\dfrac{\gamma_+ f + \gamma_- f}{2}
	+i(\sigma\cdot\nu)(\gamma_+ f - \gamma_- f)\\
	&=(\eps\sigma_0+\mu\sigma_3)\calc_{z} g+ g=\Theta_z g,
\end{align*}
which implies $g\in \ker\Theta_z$. Hence, $\ker(B-z)\subset\Phi_z\ker\Theta_z$.	
\end{proof}

For the sake of completeness, we include the proof of the following important statement (which is based on similar ideas):
\begin{lemma}\label{lem22}
The operator $C_\Sigma^2-\frac{1}{4}$ is compact in $L^2(\Sigma,\cc^2)$.	
\end{lemma}

\begin{proof}
Let $h\in L^2(\Sigma,\cc^2)$ and $z\in\cc\setminus\spec A$. Consider $f:=\Phi_z h$,
then $(D-z)f=0$ in $\Omega_\pm$. Consider further the function
\[
\widetilde f: \ \rr^2\ni x\mapsto \begin{cases}
	f(x), & x\in \Omega_+,\\
	0, & \text{otherwise}.
	\end{cases}
\]
One has $\gamma_+\widetilde f=\gamma_+ f$ and $\gamma_- \widetilde f=0$, with $(D-z)\widetilde f=0$ in $\Omega_\pm$,
and \eqref{distjump} gives
\[
(D-z)\widetilde f=i(\sigma\cdot \nu)(\gamma_+\widetilde f-\gamma_-\widetilde f)\delta_\Sigma\equiv i(\sigma\cdot \nu)\gamma_+ f\,\delta_\Sigma \text{ in } \cD'(\rr^2),
\]
which implies $\widetilde f=\phi_z\ast \big[i(\sigma\cdot \nu)\gamma_+ f\,\delta_\Sigma\big]\equiv
\Phi_z i(\sigma\cdot \nu)\gamma_+ f$. In particular,
\begin{equation}
	\label{fff3}
\Phi_z i(\sigma\cdot \nu)\gamma_+ f =f=\Phi_z h \text{ in } \Omega_+.
\end{equation}
Remark that by the construction of $f$ we have
\[
\gamma_+f=\Big(-\dfrac{i (\sigma\cdot\nu )}{2}+\calc_z\Big) h.
\]
Use this last equality in \eqref{fff3} and then apply $\gamma_+$ on the both parts, then one arrives at
\[
\Big(-\frac{i(\sigma\cdot\nu)}{2}+\calc_z\Big)i(\sigma\cdot \nu)\Big(-\dfrac{i (\sigma\cdot\nu )}{2}+\calc_z\Big) h=\Big(-\dfrac{i (\sigma\cdot\nu )}{2}+\calc_z\Big) h,
\]
which after a simple algebra takes the form
\[
\calc_zi(\sigma\cdot\nu)\calc_z h=-\dfrac{i(\sigma\cdot\nu)}{4}\, h,
\]
and results in the identity
\begin{equation}
	\label{cc2}
	\Big(\calc_z (\sigma\cdot\nu)\Big)^2=-\frac{1}{4}\, I.
\end{equation}
The identities are well-known for the three-dimensional case \cite[Lemma 3.3]{AMV1}, but we gave a complete argument
to stay self-contained. Further remark that
\[
\sigma\cdot\nu=\begin{pmatrix}
	0 & n^*\\
	n & 0
\end{pmatrix},
\]
where $n$ is the operator of multiplication by $\nu_1+i\nu_2$. Using \eqref{IdenC} we write
\[
\calc_z=\begin{pmatrix} 
		0& C_\Sigma t^*\\ 
		tC_\Sigma^* & 0 \, 
	\end{pmatrix}+
	M_0
\]
with a compact operator $M_0$. We have $t^*n=-i I$, so the substitution into \eqref{cc2} gives,
with some compact operators $M_j$,
\[
-\dfrac{1}{4}\, I=
\left[\begin{pmatrix}
	-iC_\Sigma & 0\\
	0 &tC_\Sigma^* n^*
\end{pmatrix}+M_1\right]^2=\begin{pmatrix}
-C_\Sigma^2 & 0\\
0 & (tC_\Sigma^* n^*)^2
\end{pmatrix}
+M_2,
\]
and the upper left block gives the sought result.
\end{proof}





\section{Case $|\eps|=|\mu|$}


We first consider the self-adjointness of $B$ for $|\eps|=|\mu|$.


\begin{theorem}\label{thm31}
	The operator $B$ in \eqref{bdom} is self-adjoint for $|\eps|=|\mu|$.
\end{theorem}

\begin{proof}
In the case $\eps=\mu=0$ we have obviously $B=A$. From now on let
\[
\mu=\pm \eps \text{ with }\eps\ne 0.
\]
Consider the following maps
\begin{align*}
P_+:&\ L^2(\Sigma)\ni f\mapsto \begin{pmatrix}
	f\\
	0
\end{pmatrix}\in L^2(\Sigma,\cc^2),\\
P_-:&\ L^2(\Sigma)\ni f\mapsto \begin{pmatrix}
	0\\
	f
\end{pmatrix}\in L^2(\Sigma,\cc^2),
\end{align*}
and their adjoints
\begin{align*}
P_+^*:&\ L^2(\Sigma,\cc^2)\ni \begin{pmatrix} f_1\\f_2\end{pmatrix}\mapsto f_1 \in L^2(\Sigma),\\
P_-^*:&\ L^2(\Sigma,\cc^2)\ni \begin{pmatrix} f_1\\f_2\end{pmatrix}\mapsto f_2 \in L^2(\Sigma).
\end{align*}
We set
\[
P:=P_\pm \text{ for }\eps=\pm\mu.
\]


As the operator $B$ is symmetric, it is sufficient to show that $\ran(B-z)=L^2(\rr^2,\cc^2)$ for any $z\in\cc\setminus\rr$.
For that, we will explicitly construct the inverse $(B-z)^{-1}$.

Let $z\in \cc\setminus\rr$. As $B$ is symmetric, $\ker(B-z)=\{0\}$, and Lemma \ref{lem-discr} implies $\ker \Theta_z=\{0\}$.
Remark that in the present case, we have
\begin{gather*}
\Theta_z=I+2\eps P P^*\calc_z,
\quad
\Theta_zP =P +2\eps P P^*\calc_zP\equiv 2\eps P \lambda_z\\
\text{for }
\lambda_z:=\dfrac{1}{2\eps}\,I+P^*\calc_zP:\ L^2(\Sigma,\cc^2)\to L^2(\Sigma,\cc^2)\equiv \dfrac{1}{2\eps}\,I+(z\pm m)S_z
\end{gather*}
with the operator $S_z:L^2(\Sigma)\to L^2(\Sigma)$ given by
\[
(S_z g)(x):=\dfrac{1}{2\pi}\int_\Sigma K_0\big(\sqrt{m^2-z^2}|x-y|\big) g(y)\,\mathrm{d}s(y),\quad x\in\Sigma,\quad g\in L^2(\Sigma).
\]
The integral kernel of $S_z$ has a logarithmic singularity on the diagonal, therefore, $S_z$ is Hilbert-Schmidt (in particular, compact). It follows that $\lambda_z$ is a Fredholm operator of index zero. From the injectivity of $\Theta_z$
and $P$ one obtains the injectivity of $\lambda_z$, and it follows that $\lambda_z:L^2(\Sigma)\to L^2(\Sigma)$ is bijective.

Now we are going to show that the operator
\[
R(z):=(A-z)^{-1}-\Phi_z P \lambda_z^{-1} P^*\Phi^*_{\Bar z},
\]
is the inverse of $B-z$. Let $v\in L^2(\rr^2,\cc^2)$. Due to \eqref{figz} one has
\[
f:=R(z)v\in H^\half(\rr^2\setminus\Sigma,\cc^2).
\]
Using the jump formulas \eqref{jump2} we obtain
\begin{align*}
	\dfrac{\gamma_+ f + \gamma_- f}{2}&=\gamma(A-z)^{-1}v-\calc_z P \lambda_z^{-1} P^*\Phi^*_{\Bar z} v
	\equiv \Phi^*_{\Bar z} v-\calc_z P \lambda_z^{-1} P^*\Phi^*_{\Bar z} v,\\
	\gamma_+ f - \gamma_- f&=i(\sigma\cdot\nu)P \lambda_z^{-1} P^*\Phi^*_{\Bar z} v.
\end{align*}
We have then
\begin{align*}
	(\eps\sigma_0+\mu\sigma_3)&\dfrac{\gamma_+ f+\gamma_- f}{2}+i(\sigma\cdot\nu)(\gamma_+-\gamma_- f)\\
	&\equiv
	2\eps P P^*\dfrac{\gamma_+ f+\gamma_- f}{2}+i(\sigma\cdot\nu)(\gamma_+-\gamma_- f)\\
	&=
	2\eps P P^*\big( \Phi^*_{\Bar z} v-\calc_z P \lambda_z^{-1} P^*\Phi^*_{\Bar z} v\big)
	+i(\sigma\cdot\nu)i(\sigma\cdot\nu)P \lambda_z^{-1} P^*\Phi^*_{\Bar z} v\\
	&=2\eps P P^*\big( \Phi^*_{\Bar z} v-\calc_z P \lambda_z^{-1} P^*\Phi^*_{\Bar z} v\big)-P \lambda_z^{-1} P^*\Phi^*_{\Bar z} v\\
	&=P\big( 2\eps I -2\eps P^*\calc_z P \lambda_z^{-1}-\lambda_z^{-1}\big) P^*\Phi^*_{\Bar z} v,
\end{align*}
while
\begin{align*}
2\eps -2\eps P^*\calc_z P \lambda_z^{-1}-\lambda_z^{-1}&=2\eps I-2\eps \Big( P^*\calc_z P +\frac{1}{2\eps}\,I\Big)\lambda_z^{-1}\\
&=2\eps I-2\eps \lambda_z \lambda_z^{-1}=0.
\end{align*}
This shows that $f$ satisfies the transmission condition \eqref{TC} and, therefore, $f\in\dom B$.

Further, in $\cD'(\rr^2\setminus \Sigma,\cc^2)$ we have $(D-z)\Phi_z P \lambda_z^{-1} P^*\Phi^*_{\Bar z}v=0$, therefore,
\[
(B-z)f=(D-z)f=(D-z)(A-z)^{-1} v=(A-z)(A-z)^{-1}v=v,
\]
which shows $R(z)=(B-z)^{-1}$.
\end{proof}

\section{Case $|\eps|\ne|\mu|$}

For $|\eps|\ne|\mu|$ the matrix $\eps\sigma_0+\mu\sigma_3$ is invertible, with
\[
(\eps\sigma_0+\mu\sigma_3)^{-1}=\dfrac{1}{\eps^{2} - \mu^{2}}(\eps\sigma_0 - \mu\sigma_3),
\]
and it will be more convenient to consider the auxiliary bounded linear operators
\begin{align*}%\label{Lambda}
    \Lambda_z:=\dfrac{1}{\eps^{2} - \mu^{2}}(\eps\sigma_0 - \mu\sigma_3) + \calc_z\equiv
    (\eps \sigma_0+\mu\sigma_3)^{-1} \Theta_z
\end{align*}
for $z\in(\cc\setminus\spec A)\cup\{m\}$. The symmetry property $\phi_{z}(y-x)^{\ast}=\phi_{\overline{z}}(x-y)$ entails that
both $\calc_z$ and $\Lambda_z$ are self-adjoint for real admissible $z$. 


The following assertion can be viewed as a simplified version of the results of \cite{BHSS}, and this is the entry point
for the subsequent analysis:

\begin{theorem}\label{prop21}
Let $|\eps|\ne|\mu|$ such that the operator $\Lambda_a$ is Fredholm for some $a\in(\cc\setminus\spec A)\cup\{m\}$, then the operator $B$ in \eqref{bdom} is self-adjoint.
\end{theorem}

\begin{proof}
As the domain and the self-adjointness of $B$ are independent of the choice of $m$ (which just adds a bounded symmetric perturbation), it is convenient to assume $m>0$.
	
Let $\Lambda_a$ be Fredholm. As noted above, for any $z\in\cc\setminus\spec A$ the difference $\Lambda_z-\Lambda_a\equiv \calc_z-\calc_a$ is a compact operator, and it follows that $\Lambda_z$ is also Fredholm and has the same index as $\Lambda_a$.

Now let $z\in(-m,m)\subset\cc\setminus\spec A$, then $\Lambda_z$ is self-adjoint. From the Fredholmness and the self-adjointness, it follows that the index of $\Lambda_z$ is zero. We have just seen above that the index is independent of $z$, so $\Lambda_z$ is Fredholm of index zero for all $z\in\cc\setminus\spec A$.

As $B$ is symmetric, and in order to show its self-adjointness it is sufficient to show that $\ran (B-z)=L^2(\rr^2,\cc^2)$ for all $z\in\cc\setminus\rr$. We will do it by constructing explicitly the inverse $(B-z)^{-1}$ defined on $L^2(\rr^2,\cc^2)$.

Let $z\in\cc\setminus\rr$. As $B$ is symmetric, there holds $\ker(B-z)=\{0\}$. By Lemma \ref{lem-discr} one obtains $\ker \Lambda_z=\{0\}$. As $\Lambda_z$ is Fredholm of index zero, one has $\ran \Lambda_z=L^2(\Sigma,\cc^2)$,
so $\Lambda_z:L^2(\Sigma,\cc^2)\to L^2(\Sigma,\cc^2)$ is bijective with a bounded inverse.
Consider the bounded linear operator
\begin{align*}\label{Kresolvent1}
	R(z)= (A-z)^{-1} - \Phi_{z}\Lambda_{z}^{-1} \Phi^{\ast}_{\overline{z}}:\ L^2(\rr^2,\cc^2)\to L^2(\rr^2,\cc^2).
\end{align*}
We are going to show that $R(z)=(B-z)^{-1}$.

Let $v\in L^2(\rr^2,\cc^2)$. Due to \eqref{figz} one has
\[
f:=R(z)v\in H^\half(\rr^2\setminus\Sigma,\cc^2).
\]
Using \eqref{jump2} we obtain
\begin{align*}
\dfrac{\gamma_+f + \gamma_- f}{2}&=\gamma(A-z)^{-1} v -\calc_z \Lambda_{z}^{-1} \Phi^*_{\overline{z}}v
=\Phi^*_{\overline{z}}v-\calc_z \Lambda_{z}^{-1} \Phi^*_{\overline{z}}v,\\
\gamma_+ f - \gamma_- f&=i\,(\sigma\cdot\nu) (\Lambda_{z} )^{-1} \Phi^*_{\overline{z}}v.
\end{align*}
Then
\begin{align*}
	(\eps \sigma_0 + &\mu\sigma_{3})\frac{\gamma_+f + \gamma_-f}{2}+  i(\sigma\cdot\nu )(\gamma_+f - \gamma_-f)\\
	&=\Big[(\eps \sigma_0 + \mu\sigma_{3})(I-\calc_z \Lambda_z^{-1})-\Lambda_z^{-1}\Big] \Phi^*_{\overline{z}}v,
\end{align*}
while
\begin{align*}
	(\eps \sigma_0 + \mu\sigma_{3})(I-\calc_z \Lambda_z^{-1})-\Lambda_z^{-1}&=
	\Big[ 
	(\eps \sigma_0 + \mu\sigma_{3})(\Lambda_z-\calc_z)-I	
	\Big]\Lambda_z^{-1}\\
	&=	
	\Big[(\eps \sigma_0 + \mu\sigma_{3})\dfrac{1}{\eps^{2} - \mu^{2}}(\eps\sigma_0 - \mu\sigma_3)-I\Big]\Lambda_z^{-1}\\
	&	=(I-I)\Lambda_z^{-1}=0.
\end{align*}
This shows that $f$ satisfies the transmission condition \eqref{TC}, i.e. $f\in\dom B$.
In addition, in ${\mathcal D}'(\rr^2\setminus\Sigma)$ we have $(D-z)\Phi_{z}\Lambda_{z}^{-1} \Phi^{\ast}_{\overline{z}}=0$, therefore,
\begin{align*}
(B-z)f&=(D-z)f=(D-z)R(z)v\\
&=(D-z)(A-z)^{-1}=(A-z)(A-z)^{-1}v=v,
\end{align*}
which shows the required identity $R(z)=(B-z)^{-1}$.
\end{proof}


The following lemma gives a precise range of $(\eps,\mu)$ for which $B$ is self-adjoint without additional assumptions on $\Sigma$.

\begin{theorem}\label{thm43} Assume that  $|\eps| <|\mu|$, then $B$ is self-adjoint.
\end{theorem}
\begin{proof}By Proposition \ref{prop21} it is sufficient to show that $(\eps^2-\mu^2)\Lambda_m$ is Fredholm. Using \eqref{IdenC} we represent
\begin{align*}
	(\eps^2-\mu^2)\Lambda_m&=(\eps\sigma_0-\mu\sigma_3)+(\eps^2-\mu^2)\calc_m\\
	&=(\eps\sigma_0-\mu\sigma_3)+(\eps^2-\mu^2)\begin{pmatrix} 
		0& C_\Sigma t^*\\ 
		tC_\Sigma^* & 0 
	\end{pmatrix}=\eps\sigma_0  + \Gamma,\\
	\text{with }
	\Gamma&:=\begin{pmatrix}
		-\mu& (\eps^2-\mu^2) C_\Sigma t^*\\ 
		(\eps^2-\mu^2)tC_\Sigma^* & \mu
	\end{pmatrix}.
\end{align*}	
Remark that $\Gamma$ is self-adjoint and
\[
\Gamma^2=\mu^2+(\eps^2-\mu^2)^2\begin{pmatrix}
	C_\Sigma C_\Sigma^* & 0\\
	0 & t C_\Sigma^* C_\Sigma t^*
\end{pmatrix}.
\]
The last term is a non-negative operator, which shows
\[
\spec(\Gamma^2)\subset [\mu^2,\infty),
\quad
\spec \Gamma \cap \big(-|\mu|,|\mu|\big)=\emptyset.
\]
Therefore, if $|\eps|<|\mu|$, then the operator
\[
(\eps^2-\mu^2)\Lambda_m\equiv\eps+\Gamma:\ L^2(\Sigma,\cc^2)\to L^2(\Sigma,\cc^2)
\]
is an isomorphism and, in particular, Fredholm.	
\end{proof}

By summarizing Theorems \ref{thm31} and \ref{thm43} we arrive at
\begin{corollary}\label{bleq}
	The operator $B$ is self-adjoint for any $(\eps,\mu)$ with $|\eps|\le|\mu|$.
\end{corollary}


Remark that the preceding discussion is valid without any additional assumptions on $\Sigma$ (i.e. only assumes that $\Sigma$ is Lipschitz). Under stronger geometric assumptions one can indeed enlarge the range of parameters for which the self-adjointness is guaranteed. The following result follows implicitly from the machinery of \cite{BHSS}, but we prefer to give an explicit formulation with a direct argument.

\begin{theorem}\label{thmc1}
	If $\Sigma$ is $C^1$-smooth and $\eps^2-\mu^2\ne 4$, then $B$ is self-adjoint.
\end{theorem}


\begin{proof}
	The case $|\eps|=|\mu|$ is already covered by Theorem \ref{thm31}, so from now on assume $|\eps|\ne|\mu|$.	
	By Proposition \ref{prop21} it is sufficient to show that $\Lambda_m$ is Fredholm.
	Due to the self-adjointness of $\Lambda_m$ this is equivalent to
	\begin{equation}
		\label{lmm}
		0\notin\spece(\eps^2-\mu^2)\Lambda_m.
	\end{equation}
	Using \eqref{IdenC} we represent
	\begin{align*}
		(\eps^2-\mu^2)\Lambda_m&=(\eps\sigma_0-\mu\sigma_3)+(\eps^2-\mu^2)\calc_m\\
		&=(\eps\sigma_0-\mu\sigma_3)+(\eps^2-\mu^2)\begin{pmatrix} 
			0& C_\Sigma t^*\\ 
			tC_\Sigma^* & 0 
		\end{pmatrix}=\eps\sigma_0  + \Gamma,\\
		\text{with }
		\Gamma&:=\begin{pmatrix}
			-\mu I& (\eps^2-\mu^2) C_\Sigma t^*\\ 
			(\eps^2-\mu^2)tC_\Sigma^* & \mu I
		\end{pmatrix}.
	\end{align*}
	By \cite[Theorem 3.2]{Lan} the operator $C_\Sigma-C_\Sigma^*$ is compact, therefore,
	\[
	\Gamma=\begin{pmatrix}
		-\mu& (\eps^2-\mu^2) C_\Sigma t^*\\ 
		(\eps^2-\mu^2)tC_\Sigma & \mu
	\end{pmatrix}+M_0
	\]
	with some compact operator $M_0$. Using Lemma \ref{lem22} we obtain,
	with some compact operators $M_1$ and $M_2$,
	\begin{align*}
		\Gamma^2&=\mu^2+(\eps^2-\mu^2)^2\begin{pmatrix}
			C_\Sigma^2 & 0\\
			0 & t C_\Sigma^2 t^*
		\end{pmatrix}+M_1\\
		&\equiv
		\mu^2+\dfrac{(\eps^2-\mu^2)^2}{4}\begin{pmatrix}
			I & 0\\
			0 & I
		\end{pmatrix}+M_2.
	\end{align*}
	It follows that
	\[
	\spece (\Gamma^2)=\mu^2+\dfrac{(\eps^2-\mu^2)^2}{4},
	\]
	and the self-adjointness of $\Gamma$ implies
	\[
	\spece\Gamma\in\bigg\{\,
	-\sqrt{\mu^2+\dfrac{(\eps^2-\mu^2)^2}{4}},
	\sqrt{\mu^2+\dfrac{(\eps^2-\mu^2)^2}{4}}\,
	\bigg\}.
	\]
	Due to the above identity $(\eps^2-\mu^2)\Lambda_m=\eps+\Gamma$ the condition \eqref{lmm} is equivalent to 
	\[
	|\eps|\ne\sqrt{\mu^2+\dfrac{(\eps^2-\mu^2)^2}{4}},
	\text{ i.e. }
	\eps^2-\mu^2\ne \dfrac{(\eps^2-\mu^2)^2}{4}
	\]
	which reduces to $\eps^2-\mu^2\ne4$.
\end{proof}





\section{Fredholmness for curvilinear polygons}

{}From now assume that $\Sigma$ is a piecewise $C^1$-smooth Lipschitz curve, with finitely many corner points $a_1,\dots,a_n$. For each corner $a_j$, let
\[
\theta_j\in(0,2\pi)\setminus\{ \pi \}
\]
be the non-oriented interior angle of $\Sigma$ at the point $a_j$ measured inside $\Omega_+$. 
Our main goal is to give a complete characterization of the values of $\eps$ and $\mu$ for which the operators $\Lambda_z$ are Fredholm in $L^2(\Sigma,\cc^2)$. To do so, we are going to implement the technique proposed by Shelepov \cite{VYS}. 
Remark that some components of the approach implicitly appear in other works \cite{bolt,cs}.

Actually the work \cite{VYS} also applies to the so-called Radon curves, which are more general than curvilinear polygons, but we prefer to restrict our attention to the case of piecewise $C^1$-smooth curves in order to avoid a series of involved definitions. Let us first describe the general scheme of \cite{VYS}.

Denote
\[
\sss:=\big\{x\in\rr^2:\, |x|=1\big\}
\]
and let $M_k$ be the space of $k\times k$ complex matrices. Let 
\[
G:\ \rr\times\rr\times\sss\times\sss\times\sss\to M_k
\]
be a matrix-valued function whose entries $G_{i,j}$ are Lipschitz (with respect to all variables) and such that
for some $C>0$ one has
\begin{align}\label{condi1}
    \big|G_{ij}(x,y,\xi,\eta,\zeta)\big|\leq C\Big( \big|\langle \xi,\zeta\rangle\big|+\big|\langle \eta,\zeta\rangle\big|\Big)
\end{align}
for all $(x,y,\xi,\eta,\zeta)$.

Consider the bounded integral operator  $T: L^2(\Sigma,\cc^k)\to L^2(\Sigma,\cc^k)$,
\begin{gather*}%\label{eqint}
	T g(x) = \int_{\Sigma} \dfrac{1}{|x-y|} \,G\Big(x,y,\nu(x),\nu(y),\dfrac{x-y}{|x-y|}\Big) g(y)\, \mathrm{ds}(y), \\ x,y\in\Sigma, 	\quad g\in L^2(\Sigma,\cc^k).\nonumber
\end{gather*}
We assume without loss of generality that each connected component of $\Sigma$ is oriented in the anticlockwise sense.
Fix a corner point $a$ on $\Sigma$ with an interior angle $\theta$. A small arc of $\Sigma$ around $a$
is separated by $a$ into two nonempty parts $\Gamma_+$ and $\Gamma_-$ that project in one-to-one fashion on the one-sided tangents to $\Sigma$ at $a$, and denote the projections by $\overline{\Gamma_+}$ and $\overline{\Gamma_-}$ respectively. Let $\tau_+$ and $\tau_-$ be the unit vectors along $\overline{\Gamma_+}$ and $\overline{\Gamma_-}$ directed away from the corner $a$, and let $\nu_+(a)$ and $\nu_-(a)$ be the corresponding one-sided limits of the inner normal to $\Sigma$ at $a$. We then denote by $\tau= -\tau_-$ the unit vector of the left positive tangent to $\Sigma$ at $a$ and by $\nu(a)=\nu_-(a)$ the vector obtained from $\tau$ by a counterclockwise rotation through the angle $\pi/2$, see Figure \ref{fig1}. Finally, we will use the parameters
  \[
  \xi:=\eta + \dfrac{i}{2},\quad \eta\in\rr.
  \]  
  
% Figure environment removed

Following \cite{VYS}, we define a function $\zeta:\rr\to\rr$ and  matrix-valued functions
\[
H^{(j)}_{a}:\ \rr+\frac{i}{2}\to M_k,\quad j\in\{1,2\},
\]
by
\begin{align*}
    \zeta(t) &= \dfrac{\big(e^{-\frac{t}{2}}\cos \theta - e^{\frac{t}{2}}\big)\tau - \nu e^{-\frac{t}{2}}\sin \theta }{\sqrt{e^{t} + e^{-t\mathstrut} - 2\cos\theta} },\\
    H^{(1)}_{a} (\xi)&= \int_{-\infty}^{\infty}\dfrac{e^{(i\xi+1/2)t}}{\sqrt{e^{t} + e^{-t} - 2\cos \theta}}\, G\Big(a,a,\nu,-\tau \sin \theta -\nu \cos \theta,\zeta(-t)\Big)\mathrm{d}t, \\
    H^{(2)}_{a} (\xi) &= \int_{-\infty}^{\infty}\dfrac{e^{(i\xi+1/2)t}}{\sqrt{e^{t} + e^{-t} - 2\cos(\theta)}}\, G\Big(a,a,-\tau \sin \theta -\nu \cos \theta,\nu, -\zeta(t)\Big)\mathrm{d}t,
\end{align*}
and set 
\begin{align*}%\label{delta}
    \Delta_a(\xi) = \det \Big (\sigma_0 - H^{(1)}_{a}(\xi)\,H^{(2)}_{a}(\xi)\Big),\quad \xi \in \rr+\frac{i}{2}.
\end{align*}
The following result was shown in \cite[Theorem 2]{VYS}:

\begin{proposition}\label{VYS} The operator $I-T$ is Fredholm in $L^{2}(\Sigma,\cc^{2})$ if and only if 
    \begin{align*}
        \Delta_{a_{j}}(\xi) \neq 0 \text{ for all }  \xi\in\rr+\frac{i}{2}
        \text{ and all corners $a_1,...,a_n$ of $\Sigma$.}  
    \end{align*}
\end{proposition}


We are now going to apply this machinery to our particular situation. For $\theta\in(0,2\pi)$ consider the function
\[
M_\theta:\ \rr\ni x\mapsto \dfrac{\cosh\big((\pi-\theta)x\big)}{2\big( 1+\cosh (\pi x)\big)}\in\rr,
\]
and denote
\[
m(\theta):=\sup_{x\in\rr}M_\theta(x).
\]
We have the obvious symmetry
\begin{equation}
	\label{msym}
m(\theta)=m(2\pi-\theta) \text{ for any } \theta\in(0,2\pi).
\end{equation}
The following elementary properties of $m$ will be needed as well:
\begin{proposition}\label{prop53}
	For any $\omega\in(0,\pi)$	there holds
	\begin{equation}
		\label{mmm}
		\dfrac{1}{4}\le m(\omega)\le\dfrac{1}{2}.
	\end{equation}
	Moreover, the function $\omega\mapsto m(\omega)$ is non-increasing, with
	\begin{equation}
		\label{mm1}
		\lim_{\omega\to 0^+}m(\omega)=\dfrac{1}{2}
	\end{equation}
	and
	\begin{equation}
		\label{mm2}
		m(\omega)=\frac{1}{4} \text{ for all } \omega\in\Big[\frac{\pi}{2},\pi\Big).
	\end{equation}
	
\end{proposition}
\begin{proof}
	For any $|a|\le|b|$ we have $\cosh a\le\cosh b$. It follows that for any $x\in\rr$ there holds
	\[
	\dfrac{1}{4}=M_\omega(0)\le M_\omega(x)=\dfrac{\cosh\big((\pi-\omega)x\big)}{2\big( 1+\cosh (\pi x)\big)}\le
	\dfrac{\cosh(\pi x)}{2\big( 1+\cosh (\pi x)\big)}\le \dfrac{1}{2},
	\]
	which gives \eqref{mmm}. For $0<\omega\le \omega'<\pi$ and any $x\in\rr$ one has
	\[
	M_{\omega'}(x)=
	\dfrac{\cosh\big((\pi-\omega')x\big)}{2\big( 1+\cosh (\pi x)\big)}\le \dfrac{\cosh\big((\pi-\omega)x\big)}{2\big( 1+\cosh (\pi x)\big)}=M_\omega(x),
	\]
	so taking the supremum over all $x$ one shows $m(\omega')\le m(\omega)$, i.e. $m$ is non-increasing.
	In addition, for any fixed $x$ the function $\theta\mapsto M_\theta(x)$ is non-increasing too. It follows
	\begin{align*}
		\lim_{\omega\to 0^+}m(\omega)&=\sup_{\omega\in(0,\pi)}m(\omega)=\sup_{\omega\in(0,\pi)}\sup_{x\in\rr} M_\omega(x)\\
		&=\sup_{x\in\rr}\sup_{\omega\in(0,\pi)} M_\omega(x)=\sup_{x\in\rr}\lim_{\omega\to 0^+}M_\omega(x)\\
		&=\sup_{x\in\rr}\lim_{\omega\to 0^+}\dfrac{\cosh\big((\pi-\omega)x\big)}{2\big( 1+\cosh (\pi x)\big)}
		=\sup_{x\in\rr}\dfrac{\cosh(\pi x)}{2\big( 1+\cosh (\pi x)\big)}=\dfrac{1}{2}.
	\end{align*}
	
	We further remark that for any $\omega\in(0,\pi)$ the function $M_\theta$ is even, and for any $x\ge 0$ one has
	\begin{align*}
		M'_\omega(x)&=\dfrac{1}{2\big( 1+\cosh(\pi x)\big)^2}\Big[(\pi-\omega)\sinh\big((\pi-\omega)x\big)\big(1+\cosh(\pi x)\big)\\
		&\qquad-\pi \cosh\big((\pi-\omega)x\big)\sinh(\pi x)\Big]\\
		&\equiv\dfrac{\pi \big(1+\cosh(\pi x)\big)\cosh\big((\pi-\omega)x\big) }{2\big( 1+\cosh(\pi x)\big)^2} N_\omega(x)
	\end{align*}
	with
	\begin{align*}
		N_\omega(x)&:=\dfrac{\pi-\omega}{\pi} \,\dfrac{\sinh\big((\pi-\omega)x\big)}{\cosh\big((\pi-\omega)x\big)}
		-\dfrac{\sinh(\pi x)}{1+\cosh(\pi x)}
		\\[\smallskipamount]
		&\equiv \dfrac{\pi-\omega}{\pi} \,\dfrac{\sinh\big((\pi-\omega)x\big)}{\cosh\big((\pi-\omega)x\big)}-\dfrac{\sinh \dfrac{\pi x}{2}}{\cosh \dfrac{\pi x\mathstrut}{2}}\\[\smallskipamount]
		&\equiv  \dfrac{\pi-\omega}{\pi} \tanh\big((\pi-\omega)x\big)- \tanh \dfrac{\pi x}{2}.
	\end{align*}
	The function $[0,\infty)\ni a\mapsto \tanh a$ in increasing, therefore, $N_\omega(x)<0$ for all $x>0$ and $\omega\in\big[\frac{\pi}{2},\pi\big)$, and then $M_\omega'(x)<0$ for the same $x$ and $\omega$.
	Then for each $\omega\in\big[\frac{\pi}{2},\pi\big)$ the function $M_\omega$ is decreasing on $(0,+\infty)$, and by parity
	its maximum is located at the origin, i.e.
	\[
	m(\omega)=\sup_{x\in\rr}M_\omega(x)=M_\omega(0)=\dfrac{1}{4} \text{ for all }\omega\in\Big[\frac{\pi}{2},\pi\Big). \qedhere
	\]
\end{proof}

\begin{remark}
	The condition for $\omega$ in \eqref{mm2} is not expected to be optimal. A rough numerical simulation indicates that
	\[
	\min \Big\{ \omega\in(0,\pi):\ m(\omega)=\frac{1}{4}\Big\}\simeq 0.3\,\pi.
	\]
\end{remark}


Using the above preparations we arrive at the main result:

\begin{theorem}\label{thm52}
Denote by $\omega$ the smallest angle of $\Sigma$, defined by
\[
\omega:=\min_{j\in\{1,\dots,n\}} \min\{\theta_j,2\pi-\theta_j\}\in(0,\pi).
\]
If 
\begin{equation}
	\label{emu1}
	\eps^2-\mu^2<\dfrac{1}{m(\omega)} \text { or } \eps^2-\mu^2>16 m(\omega),
\end{equation}
then the operator $B$ is self-adjoint.
\end{theorem}

\begin{proof} 
As the case $|\eps|\le |\mu|$ is already covered by Corollary \ref{bleq}, for the rest of the proof we assume
\[
|\eps|>|\mu|.
\]
By Theorem \ref{prop21} it is sufficient to show that $\Lambda_m$ is Fredholm, which is in turn equivalent to the Fredholmness of the operator
\[
\Theta_m\equiv (\eps\sigma_0+\mu\sigma_3)\Lambda_m\equiv I+(\eps\sigma_0+\mu\sigma_3)\calc_m:\ L^2(\Sigma,\cc^2)\to L^2(\Sigma,\cc^2).
\]
Eq. \eqref{IdenC} for $\calc_m$ gives the representation
\begin{align*}
    \Theta_m g(x) = g - \int_{\Sigma} \dfrac{1}{|x-y|} G\Big(x,y,\nu(x),\nu(y),\dfrac{x-y}{|x-y|}\Big) g(y)\mathrm{ds}(y)
\end{align*}
with $g\in L^2(\Sigma,\cc^2)$ and the $2\times2$ matrix function $G$ defined by 
\begin{align*}
    G\Big(x,y,\nu(x),\nu(y),\dfrac{x-y}{|x-y|}\Big)= -\dfrac{i}{2\pi}\left(\begin{array}{cc}
0& (\eps + \mu)\dfrac{\overline{x} - \overline{y}}{|x-y|} \\\\
(\eps - \mu)\dfrac{x-y}{|x-y|} & 0
\end{array}\right)
\end{align*}
for $x,y\in\Sigma$, where the integral representations in \eqref{Cauchy2} were used. The entries of $G$ are obviously Lipschitz
and satisfy \eqref{condi1}, so the above machinery is applicable to the analysis of $\Theta_m$.

Let $a$ be a corner point of $\Sigma$ with an interior angle $\theta$, then 
\begin{align*}
    G\big(a,a, \nu,-\tau\sin \theta-\nu \cos \theta, \zeta(-t)\big)&=-\dfrac{i}{2\pi}\begin{pmatrix}
0& (\eps + \mu)\overline{\zeta(-t)} \\\\
(\eps - \mu)\zeta(-t) & 0
\end{pmatrix},\\
    G(a,a,-\tau\sin \theta -\nu \cos \theta,\nu, -\zeta(t))&=\dfrac{i}{2\pi}\begin{pmatrix}
0& (\eps + \mu)\overline{\zeta(t)} \\\\
(\eps - \mu)\zeta(t) & 0
\end{pmatrix}
\end{align*}
where one uses the usual identification $\rr^2\ni(x_1,x_2)=x\simeq x=x_1+ix_2\in\cc$. We have
\[
i\xi+1=i\bar{\xi} \text{ for all }\xi\in\rr+\frac{i}{2},
\]
and one easily see that the matrices $H^{(1)}_{a}$ and $H^{(2)}_{a}$ for this specific case have the form
\begin{align*}
    H^{(1)}_{a}(\xi)&= \left(\begin{array}{cc}
0& (\eps + \mu)A_{\bar{\tau},\bar{\nu}}  \\ \\
(\eps - \mu)A_{\tau,\nu} & 0
\end{array}\right),\\
    H^{(2)}_{a}(\xi)&= \left(\begin{array}{cc}
0    &    (\eps + \mu)B_{\bar{\tau},\overline{\nu}} \\\\
(\eps - \mu)B_{\tau,\nu}    &    0
\end{array}\right),
\end{align*}
where $A_{\tau,\nu}$ and $B_{\tau,\nu} $ are given by 
\begin{align*}
A_{\tau,\nu}= \int_{-\infty}^{+\infty}\dfrac{\big(e^{i\bar{\xi}t}\cos (\theta) - e^{i\xi t}\big)\tau - e^{i\bar{\xi}t}\sin (\theta)\,\nu}{e^{t} + e^{-t} - 2\mathrm{cos}(\theta)}\,\mathrm{d}t,    \\
B_{\tau,\nu}= \int_{-\infty}^{+\infty}\dfrac{\big(e^{i\xi t}\cos (\theta) - e^{i\bar{\xi}t}\big)\tau - e^{i\xi t}\sin (\theta)\,\nu}{e^{t} + e^{-t} - 2\cos(\theta)}\,\mathrm{d}t. 
\end{align*}
Hence, applying the change of variable $x = e^t$, we can rewrite $A_{\tau,\nu}$ and $B_{\tau,\nu}$ as follows
\begin{align*}
A_{\tau,\nu}= \int_{0}^{+\infty}\dfrac{(x^{i\bar{\xi}}\mathrm{cos}(\theta) - x^{i\xi})\tau - x^{i\bar{\xi}}\mathrm{sin}(\theta)\,\nu}{x^{2} + 2x\mathrm{cos}(\pi - \theta) + 1}\,\mathrm{d}x,
\end{align*}
\begin{align*}
B_{\tau,\nu}= \int_{0}^{+\infty}\dfrac{(x^{i\xi}\mathrm{cos}(\theta) - x^{i\bar{\xi}})\tau - x^{i\xi}\mathrm{sin}(\theta)\,\nu}{x^{2} + 2x\mathrm{cos}(\pi - \theta) + 1}\,\mathrm{d}x.
\end{align*}
Now recall that for all $b>0$, $0<|\omega|<\pi$ and $0<\mathrm{Re}(\alpha)<2$ one has
\begin{align*}
  \int_{0}^{+\infty} \dfrac{x^{\alpha - 1}}{x^{2} + 2bx\,\cos (\omega) + b^{2}}\,\mathrm{d}x = -\pi b^{\alpha -2} \dfrac{1}{\sin (\omega)}\dfrac{1}{\sin (\alpha\pi)}\sin\big((\alpha-1)\omega\big),
\end{align*}
see the formula (12) in \cite[p. 327]{GR}. Applying this formula with  $b=1$ and  $\omega=\pi-\theta$, one obtains that
 \begin{align*}
     A_{\tau,\nu}&=\dfrac{i}{2\sin (\theta)}\Bigg[\Big(\cos(\theta)\dfrac{\sinh \big(\overline{\xi}(\pi - \theta)\big)}{\sinh(\xi\pi)} - \dfrac{\sinh\big(\xi(\pi - \theta)\big)}{\sinh(\overline{\xi}\pi)}\Big)\tau\\
     &\qquad - \sin(\theta) \dfrac{\sinh\big(\overline{\xi}(\pi - \theta)\big)}{\sinh(\xi\pi)} \nu\Bigg],\\
    B_{\tau,\nu}=&\dfrac{-i}{2 \sin
(\theta)}\Bigg[\Big(\cos(\theta)\dfrac{\sinh\big(\xi(\pi - \theta)\big)}{\sinh(\overline{\xi}\pi)} - \dfrac{\sinh\big(\overline{\xi}(\pi - \theta)\big)}{\sinh(\xi\pi)}\Big)\tau\\
&\qquad - \mathrm{sin}(\theta) \dfrac{\mathrm{sinh}(\xi(\pi - \theta))}{\mathrm{sinh}(\overline{\xi}\pi)} \nu\Bigg].
\end{align*}
Consequently, the product $H^{(1)}_{a}(\xi)H^{(2)}_{a}(\xi)$ yields
\begin{align*}
    H^{(1)}_{a}(\xi)H^{(2)}_{a}(\xi)=\dfrac{\eps^{2} - \mu^{2}}{4\mathrm{sin}^{2}(\theta)} \,\times \,S(\xi)\, \sigma_0,
\end{align*}
where $S(\xi)$ is given by 
\begin{align*}
 S(\xi) &= 2 \,\dfrac{\sinh\big(\bar{\xi}(\pi - \theta)\big)}{\sinh(\xi\pi)} \dfrac{\sinh\big(\xi(\pi - \theta)\big)}{\sinh(\bar{\xi}\pi)}\\
    &\qquad  - \cos(\theta)\Bigg(  \dfrac{\sinh^{2}\big(\xi(\pi - \theta)\big)}{\sinh^{2}(\bar{\xi}\pi)} + \dfrac{\sinh^{2}\big(\bar{\xi}(\pi - \theta)\big)}{\sinh^{2}(\xi\pi)}\Bigg).
\end{align*}
Using the trigonometric identity
\begin{align*}
    \cosh(x\pm i y) = \cosh(x)\cos(y) \pm i\, \sinh(x)\sin(y),\quad \text{ for all } x,y\in\rr,
\end{align*}
and a straightforward computation we transform the above expression for $S(\xi)$ to
\begin{align*}
    S(\xi) = \dfrac{2\,\sin^{2}(\theta)\cosh \big(2\eta(\pi - \theta)\big)}{ \big(1+ \cosh (2\pi\eta)\big)}
    \text{ with } \xi=\eta+\frac{i}{2}.
\end{align*}
Thus,
\begin{align*}
\Delta_{a}(\xi) = \Bigg( 1 - (\eps^2 - \mu^2)\, \dfrac{\cosh\big(2\eta(\pi - \theta)\big)}{2 \big(1+ \cosh(2\pi\eta)\big)}\Bigg)^{2}=\Big(1-(\eps^2-\mu^2)M_\theta(2\eta)\Big)^2,
\end{align*}
and the condition $\Delta_a(\xi)\ne 0$ for all $\xi$ is equivalent to
\begin{equation}
	\label{mtx}
M_\theta(x)\ne \dfrac{1}{\eps^2-\mu^2} \text{ for all } x\in\rr.
\end{equation}
Remark that for any $\theta\in(0,2\pi)$ one has
\[
M_\theta(x)\ge 0 \text{ for all $x\in\rr$,}
\quad
\lim_{x\to\pm\infty}M_\theta(x)=0,
\]
then the condition \eqref{mtx} is satisfied if any only if (recall that $|\eps|>|\mu|$ by assumption)
\[
\dfrac{1}{\eps^2-\mu^2}>m(\theta):=\sup_{x\in\rr}M_\theta(x),
\quad\text{i.e.}
\quad
\eps^2-\mu^2<\frac{1}{m(\theta)}.
\]
Thus, for each corner point  $a_j$ we have shown the equivalence
\begin{equation}
	\label{dax}
\Delta_{a_j}(\xi)\ne 0 \text{ for all } \xi\in\rr+\frac{i}{2}
\text{ if and only if }
\eps^2-\mu^2<\frac{1}{m(\theta_j)}.
\end{equation}

Using the symmetry and monotonicity properties of $m$, see \eqref{msym} and Proposition \ref{prop53}, we conclude that
that $\Theta_m$ is Fredholm if and only if
\[
\eps^2-\mu^2<\min_{j\in\{1,\dots,n\}}\frac{1}{m(\theta_j)}=\frac{1}{\max_{j\in\{1,\dots,n\}} m(\theta_j)}=\dfrac{1}{m(\omega)},
\]
which is a sufficient condition for the self-adjointness of $B\equiv B_{\eps,\mu}$ and gives the first half of \eqref{emu1}.

By applying the above result to $\widetilde B:=B_{-\frac{4\eps}{\eps^2-\mu^2},-\frac{4\mu}{\eps^2-\mu^2}}$ we see that $\widetilde B$ is self-adjoint for
\[
\Big(-\frac{4\eps}{\eps^2-\mu^2}\Big)^2-\Big(-\frac{4\mu}{\eps^2-\mu^2}\Big)^2\le \frac{1}{m(\omega)},
\]
which holds for $\eps^2-\mu^2> 16 m(\omega)$. As the self-adjointness of $\widetilde B$ is equivalent to the self-adjointness of $B$ (see Remark \ref{rmk2}), we obtain the second half of \eqref{emu1}.
\end{proof}


By combining Theorem \ref{thm52} with Proposition \ref{prop53} we obtain:	
\begin{corollary}\label{momega}
Let $\Sigma$ be a curvilinear polygon (with $C^1$-smooth edges and without cusps). Assume that one of the following three conditions holds:
\begin{itemize}
	\item[(a)] $\eps^2-\mu^2<2$,
	\item[(b)] $\eps^2-\mu^2>8$,
	\item[(c)] $\eps^2-\mu^2\ne 4$ and the interior angles $\theta_j$ of $\Sigma$ satisfy
\[
\dfrac{\pi}{2}\le \theta_j\le \dfrac{3\pi }{2} \text{ for all } j\in\{1,\dots,n\},
\]
\end{itemize} 
then $B$ is self-adjoint.
\end{corollary}	
We finish this paper by pointing out the following remark.
\begin{remark} In the proof of Theorem \ref{thm52} one sees that for $\eps^2-\mu^2>16 m(\omega)$ the operator $B$ is self-adjoint but the operators $\Lambda_z$ are not Fredholm. This shows that the converse of Theorem \ref{prop21} does not hold.
\end{remark}
 

\section*{Acknowledgments}
BB and KP were supported  by the Deutsche Forschungsgemeinschaft (DFG, German Research Foundation) -- 491606144. MZ was partially supported by the Universidad del Pais Vasco/EHU \& BCAM (Spain) project PES 22/50 and by the UBGRS international mobility grant  (France). This work was prepared during a visit of MZ to the Carl von Ossietzky Universit\"at Oldenburg in May--June 2023, and he would like to thank the Institut f\"ur Mathematik for the warm hospitality.


%Biblio
\begin{thebibliography}{999}
	\small	\setlength\itemsep{4pt}

\bibitem{AMV1} N. Arrizabalaga, A. Mas, and L. Vega, {\em Shell interactions for Dirac operators}, J. Math. Pures Appl. {\bf 102}:4 (2014) 617--639.

\bibitem{b1} J. Behrndt, P. Exner, M. Holzmann, V. Lotoreichik, \emph{On Dirac operators in $\mathbb{R}^3$
with electrostatic and Lorentz scalar $\delta$-shell interactions.} Quantum Stud. Math. Found. {\bf 6} (2019) 295--314.

\bibitem{b2} J. Behrndt, M. Holzmann, \emph{On Dirac operators with electrostatic $\delta$-shell interactions of critical strength.}
J. Spectr. Theory {\bf 10} (2020) 147--184.


\bibitem{BHOP} J. Behrndt, M. Holzmann, T. Ourmi\` eres-Bonafos, K. Pankrashkin, {\em Two-dimen\-sional Dirac operators with singular interactions supported on closed curves.} J. Funct. Anal. {\bf 279} (2020) 108700.

\bibitem{BHSS} J. Behrndt, M. Holzmann, C. Stelzer, G. Stenzel, {\em Boundary triples and Weyl functions for Dirac operators with singular interactions.} Preprint arXiv:2211.05191 (2022).

\bibitem{BB2} B. Benhellal, {\em Spectral properties of the Dirac operator coupled with $\delta$-shell interactions}.  Lett Math Phys 112, 52 (2022).

\bibitem{BB3} B. Benhellal, {\em Spectral analysis of Dirac operators with singular interactions supported on the boundaries of rough domains}, J. Math. Phys. {\bf 63} (2022) 011507.

\bibitem{BP} B. Benhellal, K. Pankrashkin, {\em Curvature contribution to the essential spectrum of Dirac operators with critical shell interactions}. Preprint arXiv:2211.10264 (2022).

\bibitem{bolt} M. Bolt, A. Raich, {\em The Kerzman–Stein operator for piecewise continuously differentiable regions}.
Compl. Var. Elliptic. Equ. {\bf 60} (2015) 478--492.

\bibitem{CLMT} B. Cassano, V. Lotoreichik, A. Mas, M. Tu\v{s}ek, {\em General $\delta$-shell interactions for the two-dimensional Dirac operator: self-adjointness and approximation}. Rev. Mat. Iberoam. {\bf 39}:4 (2023) 1443--1492.

\bibitem{CMM} R. R. Coifman, A. McIntosh, Y. Meyer, {\em L'intégrale de Cauchy définit un opérateur borné sur $L^2$ pour les courbes lipschitziennes}. Ann. of Math. (2) {\bf 116}:2 (1982) 361--387.


\bibitem{cs} M. Costabel, E. Stephan, {\em A direct boundary integral equation method for transmission problems}.
J. Math. Anal. Appl. {\bf 106} (2013) 367--413.

\bibitem{GR} I. S. Gradshteyn, I. M. Ryzhik, {\em Table of integrals, series, and products}, 4th ed., Fizmatgiz, Moscow, 1963; English Transl., Academic Press, 1965.

\bibitem{jj}
J.-L. Journ\'e, {\em Calder\'on-Zygmund operators, pseudo-differential operators and the Cauchy integral of Calder\'on}. Lecture Notes in Math., vol. 994, Springer, Berlin-New York, 1983.

\bibitem{Lan} L. Lanzani, {\em Szeg\"o projection versus potential theory for non-smooth planar domains}. Indiana Univ. Math. J. {\bf 48}:2 (1999) 537--556

\bibitem{OP} T. Ourmi\` eres-Bonafos, F. Pizzichillo, {\em Self-adjointness of Dirac operators with infinite mass boundary conditions in sectors}. 
Ann. Henri Poincar\'e {\bf 19}:5 (2018) 1465--1487.

\bibitem{PV} F. Pizzichillo and H. Van Den Bosch, { \em Self-adjointness of two dimensional Dirac operators on corner domains}. J. Spectr. Theory {\bf 11} (2021) 1043--1079.

\bibitem{VYS} V. Yu. Shelepov, {\em On the index and spectrum of integral operators of potential type along Radon curves}. Math. USSR Sb. {\bf 70} (1991) 175--203.

\bibitem{Tha} B. Thaller, {\em The Dirac equation}, Text and Monographs in Physics, Springer-Verlag, Berlin, 1992.

\end{thebibliography}
\end{document}