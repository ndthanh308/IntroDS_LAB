
\begin{table}[b]
   \begin{center}
      \small
      \centering
      \def\arraystretch{1.2}
      \begin{tabular}{l | c c c}
         \toprule
          & $\text{mADE}_6$ & $\text{mFDE}_6$ & $\text{MR}_6$ \\
         \midrule
         w/o Iterative Att. & 0.673 & 0.971 & 0.086 \\
         w/o Dual Subgraph & 0.671 & 0.960 & 0.086 \\
         ADAPT & \textbf{0.668} & \textbf{0.948} & \textbf{0.083} \\
         \bottomrule
      \end{tabular}
   \end{center}
   \vspace{-0.2cm}
   \caption{\textbf{Single-Agent Ablation Study on Argoverse (Val.).} This table shows the effect of iterative attention and dual subgraph on the performance of single-agent prediction on the Argoverse validation set.}
   \label{tab:supp_ablation}
\end{table}

\section{Quantitative Results}
\label{sec:quantitative}

In this section, we present an additional ablation study to justify some minor design choices. Specifically, we investigated the effect of iterative \vs sequential order in interaction (\figref{fig:attention_orders}) and the effect of using two separate subgraphs for encoding agents and lanes. The results in \tabref{tab:supp_ablation} show that the iterative attention blocks outperform their sequential counterpart. This implies that updating intermediate features at each iteration, as opposed to the attention order used in LaneGCN~\cite{Liang2020ECCV}, leads to a better understanding of the relationship between agents and lanes. %
Furthermore, the use of separate polyline subgraphs for lanes and agents, which is in contrast to prior work~\cite{Gu2021ICCV, Gao2020CVPR}, produces better results. Overall, our decision choices on the architecture improve performance with better feature encoding.