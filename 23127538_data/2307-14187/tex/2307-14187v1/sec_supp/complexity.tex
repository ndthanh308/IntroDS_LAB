
% Figure environment removed


\section{Computational Complexity}
\label{sec:complexity}
In this section, we provide a comparison of the computational complexity according to the attention operations used in the existing approaches. We first define variables that define the number of elements. $N$, $M$, and $T$ correspond to the number of agents, lane elements, and time steps, respectively. $T$ can be decomposed into two variables, $T_p$ and $T_f$, which refer to past and future time steps, respectively. In general, the number of agents dominates the computation, then, the number of lanes followed by the fixed number of time steps, \eg $T=50$~($N>M>T$). While the number of lanes $M$ stays mostly uniform across scenes, the number of agents $N$ might vary significantly even for the same scene.

As addressed in the SceneTransformer~\cite{Ngiam2022ICLR}, directly applying attention to both time and agent axes results in high overhead, with the computational complexity of $\mathcal{O}((NT + M)^2)$ where $N$ is the number of agents, $M$ is the number of lane segments and $T$ is the number of time steps including both past and future. SceneTransformer reduces it to $\mathcal{O}(NT^2 + N^2T + NTM)$ with factorized attention over time and agent axes. 

Autobot~\cite{Girgis2022ICLR} does not include lane elements in their factorized attention steps. Contrary to SceneTransformer, their encoding and decoding phases consider only past and future time steps, respectively, resulting in the complexity of $\mathcal{O}(N{T_p}^2 + N^2{T_p} + N{T_f}^2 + N^2{T_f})$ where $T_p$ denotes the number of past time steps and $T_f$ denotes the number of future time steps.

HiVT~\cite{Zhou2022CVPR} does not use the standard multi-axis factorized attention but embraces a more efficient type of temporal interaction by considering only one agent for each time step and attending to only one feature over different time steps. Since HiVT follows an agent-centric approach and calculates agent features independently from each other, considering only one agent in their local scene does not result in information loss. However, the agent-centric approach comes with the overhead of $N$ runs of the same procedure. Considering scene normalization for each agent and global interaction in the end, HiVT has the overall complexity of
$\mathcal{O}(N^2{T_p} + N{T_p}^2 + NM)$. 

ADAPT has a clear advantage in terms of computational complexity over the existing approaches. Our computation is not bounded by $T$ as our subgraphs in vectorized encoder handle the temporal reasoning. Since we calculate the attention over only agents and lanes, ADAPT has the complexity of $\mathcal{O}(N^2 + NM + M^2)$ resulting from the attention operations in the interaction modeling. Removing the number of time steps $T$ out of the equation is the main reason behind the efficiency gain of ADAPT.

