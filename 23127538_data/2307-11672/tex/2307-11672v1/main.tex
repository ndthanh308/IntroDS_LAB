\documentclass{article}


% if you need to pass options to natbib, use, e.g.:
\PassOptionsToPackage{numbers, compress}{natbib}
% before loading neurips_2023


% ready for submission
% \usepackage{neurips_2023}

% to compile a preprint version, e.g., for submission to arXiv, add add the
% [preprint] option:
    \usepackage[preprint]{neurips_2023}


% to compile a camera-ready version, add the [final] option, e.g.:
%     \usepackage[final]{neurips_2023}


% to avoid loading the natbib package, add option nonatbib:
%    \usepackage[nonatbib]{neurips_2023}

\usepackage[utf8]{inputenc} % allow utf-8 input
\usepackage[T1]{fontenc}    % use 8-bit T1 fonts
\usepackage{hyperref}       % hyperlinks
\usepackage{url}            % simple URL typesetting
\usepackage{booktabs}       % professional-quality tables
\usepackage{amsfonts}       % blackboard math symbols
\usepackage{nicefrac}       % compact symbols for 1/2, etc.
\usepackage{microtype}      % microtypography
\usepackage{xcolor}         % colors
\usepackage{amsmath}
\usepackage{wrapfig}
\usepackage{graphicx}
\usepackage{subcaption}
\usepackage{multirow,makecell}
\usepackage{algorithm}
\usepackage{algpseudocode}
\usepackage{enumitem}
\usepackage{bm}
\usepackage{amsthm}

\newcommand{\Krik}[1]{\textcolor{cyan}{Krik: #1}}

\bibliographystyle{plainnat}
%\title{Adaptive Test-Time Defense with Useful Features}
%\title{Theory guided Adaptive Test-Time Defense with Useful Features }
\title{Fast Adaptive Test-Time Defense with Robust Features}
%\title{}
\DeclareMathOperator*{\argmax}{argmax}
\newcommand{\comment}[2]{\textcolor{red}{[#1: #2]}}
% The \author macro works with any number of authors. There are two commands
% used to separate the names and addresses of multiple authors: \And and \AND.
%
% Using \And between authors leaves it to LaTeX to determine where to break the
% lines. Using \AND forces a line break at that point. So, if LaTeX puts 3 of 4
% authors names on the first line, and the last on the second line, try using
% \AND instead of \And before the third author name.


\author{%
Anurag Singh\thanks{Equal contribution. This work was partly done when AS was in TU Munich}$^{ \,\,\,,1}$ \\
  %Max Planck Institute for Intelligent Systems \\ Tübingen, Germany \\
  {\tt anurag.singh@cispa.de} \\\And
  Mahalakshmi Sabanayagam$^{ *,2}$  \\
  %School of Computation, Information and Technology \\ Technical University of Munich, Germany \\
  {\tt sabanaya@cit.tum.de} \\\And
  Krikamol Muandet$^{1}$ \\
  % Max Planck Institute for Intelligent Systems \\ Tübingen, Germany \\
  {\tt muandet@cispa.de} \\\And
  Debarghya Ghoshdastidar$^{2}$ \\
  % School of Computation, Information and Technology \\ Technical University of Munich, Germany \\
  {\tt ghoshdas@cit.tum.de} \\
  $^1$ CISPA--Helmholtz Center for Information Security, Saarbrücken, Germany \\
$^2$ School of Computation, Information and Technology, Technical University of Munich \\
  %
  %Debarghya Ghoshdastidar\thanks{Use footnote for providing further information
  %  about author (webpage, alternative address)---\emph{not} for acknowledging
 %   funding agencies.} \\
 % Department of Computer Science\\
 % Cranberry-Lemon University\\
 % Pittsburgh, PA 15213 \\
 % \texttt{hippo@cs.cranberry-lemon.edu} \\
  % examples of more authors
  % \And
  % Coauthor \\
  % Affiliation \\
  % Address \\
  % \texttt{email} \\
  % \AND
  % Coauthor \\
  % Affiliation \\
  % Address \\
  % \texttt{email} \\
  % \And
  % Coauthor \\
  % Affiliation \\
  % Address \\
  % \texttt{email} \\
  % \And
  % Coauthor \\
  % Affiliation \\
  % Address \\
  % \texttt{email} \\
}
\newtheorem{theorem}{Theorem}
\newtheorem{corollary}[theorem]{Corollary}
\newtheorem{lemma}[theorem]{Lemma}
\newtheorem{prop}[theorem]{Proposition}
\newtheorem{definition}{Definition}
\newtheorem{remark}{Remark}

\usepackage{multirow,rotating}
\begin{document}


\maketitle


\begin{abstract}
%The robustness of deep learning models to adversarial examples is a crucial property of deployable models. 
Adaptive test-time defenses are used to improve the robustness of deep neural networks to adversarial examples. However, existing methods significantly increase the inference time due to additional optimization on the model parameters or the input at test time. In this work, we propose a novel adaptive test-time defense strategy that is easy to integrate with any existing (robust) training procedure without additional test-time computation. Based on the notion of robustness of features that we present, the key idea is to project the trained models to the most robust feature space, thereby reducing the vulnerability to adversarial attacks in non-robust directions. 
We theoretically show that the top eigenspace of the feature matrix are more robust for a generalized additive model and support our argument for a large width neural network with the Neural Tangent Kernel (NTK) equivalence.
We conduct extensive experiments on CIFAR-10 and CIFAR-100 datasets for several robustness benchmarks, including the state-of-the-art methods in RobustBench, and observe that the proposed method outperforms existing adaptive test-time defenses at much lower computation costs. 
\end{abstract}
\section{Introduction}
Current quantum hardware is unable to carry out universal quantum computations due to the buildup of errors that occur during the computation. 
The magnitude of the individual error is currently above the value that the Threshold Theorem requires in order to kick-start quantum error correction and fault-tolerant quantum computation~\cite[Section 10.6]{nielsen_chuang_2010}. 
Although the experimentally achieved fidelity rates are promising and the error bounds are inching closer to the required threshold, we will have to work for the foreseeable future with quantum hardware with errors that build-up during the computation.  This implies that we can only do a limited number of steps before the output of the computation has become completely uncorrelated with the intended one.

For fault-tolerant quantum computing, we repeat four steps: 
1) We apply a number of single and two-qubit quantum gates, in parallel whenever possible; 
2) We perform a syndrome measurement on a subset of the qubits; 
3) We perform fast classical computations to determine which errors have occurred and how to correct them; 
and, 4) We apply correction terms based on the classical computations.
We then repeat these four steps with a next sequence of gates. 
These four steps are essential to fault-tolerant quantum computing. 


The starting point of this work is to use the four steps outlined above, not to carry out error correction and fault-tolerant computation, but to enhance short, constant-depth, {\em uncorrected} quantum circuits that perform single qubit gates and {\em nearest-neighbor} two qubit gates. 
Since in the long run we will have to implement error-correction and fault-tolerant computation anyhow, and this is done by such a four-step process, why not make other use of this architecture? Moreover, on some of the quantum hardware platforms, these operations are already in place.
Embracing this idea we naturally arrive at the question: what is the computational power of \textit{low-depth} quantum-classical circuits organized as in the four steps outlined above? 
We thus investigate circuits that execute a small, ideally constant, number of stages, where at each stage we may apply, in parallel, single qubit gates and {\em nearest-neighbor} two qubit gates, followed by measurements, followed by low-depth classical computations of which the outcome can control quantum gates in later stages. 
It is not clear, at first, whether such circuits, especially with constant depth, can do anything remotely useful. 
But we will see that this is indeed the case: many quantum computations can be done by such circuits in constant depth. 
By parallelizing quantum computations in this way, we improve the overall computational capabilities of these circuits, as we do not incur errors on qubits that are idle, simply because qubits are not idle for a very long time. 
Furthermore, reducing the depth of quantum circuits, at the cost of increasing width, allows the circuit to be run faster even if errors occur.

The first usage of such a four-step layout, not to do error correction, but to perform computations, can be found in the paradigm of measurement-based quantum computing~\cite{gottesman1999demonstrating,raussendorf2001one,jozsa2006introduction,clark2007generalised}: 
A universal form of quantum computing where a quantum state is prepared and operations are performed by measuring qubits in different bases, depending on previous measurements and intermediate measurements.

\citeauthor{PhamSvore2013} were the first to formalize the four-step protocol for performing computations~\cite{PhamSvore2013}. They included specific hardware topologies by considering two-dimensional graphs for imposing constraints on qubit interactions. In their model, they develop circuits for particularly useful multi-qubit gates, including specifying costs in the width, number of qubits, depth, number of concurrent time steps, size, and total number of non-Identity operations.
As a result, they find an algorithm that factors integers in polylogarithmic depth.
\citeauthor{Browne:2011} showed that the main tool in the work by \citeauthor{PhamSvore2013}, the fan-out gate, can also be replaced by additional log-depth classical computations in the measurement-based quantum computing setting~\cite{Browne:2011}.

More recently, \citeauthor{Cirac:2021} introduced a scheme to implement unitary operations involving quantum circuits combined with Local Operations and Classical Communication ($\mathsf{LOCC}$) channels: $\mathsf{LOCC}$-assisted quantum circuits~\cite{Cirac:2021}. Similarly to the four-step scheme we just described, they allow for a short depth circuit to be run on the qubits, followed by one round of $\mathsf{LOCC}$, in which ancilla qubits are measured and local unitaries are applied based on the measurement outcomes. They show that in this model any 1D transitionally invariant matrix-product state (MPS) with fixed bond dimension is in the same phase of matter as the trivial state. Similar ideas can be found in~\cite{TVV_NonAbelianTopologicalOrder_2022, tantivasadakarn2021long}.

In this work, we introduce a new model, called \textit{Local Alternating Quantum-Classical Computations} ($\LAQCC$). In this model we alternate between running quantum circuits (constrained by locality), ending in the measurement of a subset of qubits, and fast classical computations based on the measurement results. The outcome of the classical computations are then used to control future quantum circuits. We allow for flexibility in this model, by giving different constraints to the power of both the quantum circuits and the classical circuits as well as the number of alternations between them. 
Most attention will be given to $\LAQCC$ containing quantum circuits of constant depth, classical circuits of logarithmic depth and at most a constant number of alternations between them. 
Any circuit constructed in this model is considered to be of constant depth. 
We restrict ourselves to logarithmic depth classical computations, as this is the first natural and non-trivial extension beyond constant-depth classical computations. 
Constant-depth classical computations do however also have an equivalent constant-depth quantum implementation.

The definition of $\LAQCC$ sharpens the original definition of \citeauthor{PhamSvore2013} by adding constraints to the intermediate classical computations. This allows us to bound the power of $\LAQCC$ from above. 

The main result of \citeauthor{Cirac:2021}, that 1D translational invariant MPS with fixed bond dimension can be prepared by $\mathsf{LOCC}$-assisted circuits, relies on local symmetries of the MPS. These symmetries allow them to prepare local states (on a constant number of qubits) and glue them together by doing one round of the appropriate entangling measurement and corrections, after which they run a round of local unitaries to get the desired result. This general scheme for preparing states that exhibit an MPS description with the appropriate local symmetries requires only geometrically local unitaries and one round of measurement and corrections an therefore is accessible in $\LAQCC$. Studying different local symmetries, known as Symmetry Protected Topological (SPT) phases of matter, to find measurement-based constant depth circuits for states is a broad ongoing field of research~\cite{TVV_NonAbelianTopologicalOrder_2022, tantivasadakarn2021long, smith2023deterministic}. 
All these schemes have a $\LAQCC$ implementation.

%$\LAQCC$-circuits also exist for general schemes of preparing local states, based on the local tensors, and gluing them together using one round of entangled measurement and corrections, based on the local symmetry. 
%The main result of \citeauthor{Cirac:2021}, that 1D translational invariant MPS with fixed bond dimension can be prepared by $\mathsf{LOCC}$-assisted circuits, relies heavily on local symmetries of the MPS and as a result also has an equivalent $\LAQCC$ implementation. 
%The corrections applied after the measurement round are local unitaries depending on the local symmetries of the MPS. 

 

%This general scheme of preparing local states, based on the local tensors, and gluing it together by doing one round of entangled measurement and corrections, based on the local symmetry, is accessible in $\LAQCC$.
Note however that \citeauthor{Cirac:2021} also suggest a circuit for the $W$-state.
This circuit uses sequentially and dependent measurement-based corrections of the ancilla qubits. 
These dependent measurements translate to sequential alternations between the quantum and classical circuits and therefore increase the total depth to linear depth, exceeding the constant-depth constraints imposed by $\LAQCC$-circuits. 

We study the power of the $\LAQCC$ model with respect to state preparation, showing that even with only constant quantum-depth and logarithmic classical depth it remains possible to prepare states with long-range entanglement.
Another surprising result is that it is unlikely that $\LAQCC$ circuits are classically simulatable. We show that any instantaneous quantum polynomial-time (IQP) circuit~\cite{Bremner2010,Shepherd2009} has an $\LAQCC$ implementation.
Classical simulation of IQP circuits implies the collapse of the polynomial hierarchy to the third level, which is not believed to be true~\cite{Bremner2017}. Therefore, we expect that $\LAQCC$ circuits are unlikely to be classically simulatable. We bound the power of $\LAQCC$ by showing that it is contained in $\QNC^1$, the class of polynomial-size, log-depth circuits.

Next, we also study the power that intermediate classical calculations can add to quantum computations, by considering a new model that alternates between polynomially many polynomial-depth quantum circuits and unbounded classical computations
We study this model by doing a complexity theoretical analysis, where we draw inspiration from the notions of complexity given by \citeauthor{RosenthalYuen:2022}, \citeauthor{MetgerYuen:2023}, and \citeauthor{Aaronson:2004}.
All three complexity notions are based on the notion of state preparation, instead of more traditional definition of complexity such as the decidability of a computational problem. 
The first two consider classes based on sequences of quantum states preparable by a polynomial-sized quantum circuit, where the circuits are uniformly generated by a computational class, for instance, the class $\mathsf{PSPACE}$, which results in the complexity class $\mathsf{StatePSPACE}$~\cite{RosenthalYuen:2022,MetgerYuen:2023}.
The third notion considers a relative complexity, where the complexity is measured between two given states, and is measured by the number of gates, from a given gate-set, required to transform one state in another state~\cite{Aaronson:2004}. 
For our definition of state preparation complexity, we drop the uniformity constraint from~\cite{RosenthalYuen:2022,MetgerYuen:2023} and define a class as $\mathsf{StateX}$, which refers to states preparable by circuits of type $\mathsf{X}$. 
As an example, if $\mathsf{X} = \QNC^0$, this results in the class $\mathsf{StateQNC^0}$, which is the set of states preparable from the $\ket{0}^n$ state by poly-size constant-depth circuits. 
This notion is similar to the relative complexity from~\cite{Aaronson:2004}, where one state is the  $\ket{0}^n$ state and instead of counting the number of gates we consider the set of states preparable by a fixed number of gates. Using this notion of complexity we show that any state preparable by an $\LAQCC^*$ circuit is also preparable by a $\mathsf{PostQPoly}$ circuit, the class of circuits of polynomial depth with an additional post-selection gate. 

All Clifford circuits have a constant-depth $\LAQCC$ implementation, implying that any stabilizer state can be implemented by a constant-depth $\LAQCC$ circuit, see Section~\ref{sec:clifford_circuits} for a proof of this statement. 
Efficient circuits for stabilizer states have been known already through measurement-based quantum computing. Therefore this paper focuses on the preparation of non-stabilizer states, and as a surprising result we find novel constant-depth protocols for four very natural classes of non-stabilizer states.
Despite the extensive research into these four classes of non-stabilizer states and the many applications of them, no efficient constant- or low-depth state preparation protocols are known yet. We specifically consider these four classes as they are all often used as initial states in other algorithms.

The first state is a uniform superposition over an arbitrary number of states. 
This state finds applications in many quantum algorithms, as they often start with a uniform superposition over multiple states. 
This superposition is often achieved by applying Hadamard gates to every qubit due to its simplicity to prepare. 
Yet, the analysis of many algorithms, such as Shor's algorithm~\cite{Shor:1997}, would benefit from a different initial superposition. 
The circuit to prepare the uniform superposition over an arbitrary number of states uses an exact version of Grover search as a subroutine, that turns a probabilistic circuit, with a known constant probability of success, into a deterministic circuit. 
We use the circuit for preparing a uniform superposition over an arbitrary number of states as a subroutine in the next two quantum state preparation protocols. 

The second state is the $W$-state, the uniform superposition over all computational basis states of Hamming-weight~$1$, a natural long-ranged entangled state that displays a fundamentally nonequivalent type of entanglement from the Greenberger–Horne–Zeilinger state~\cite{WState:2000}, for which $\LAQCC$-type constant-depth circuits were previously known~\cite{PhamSvore2013, Cirac:2021}. 
The $W$-state is often used as benchmark for new quantum hardware~\cite{Haffner2005,Neeley2010,GarciaPerez:2021}. 
A novel way to prepare the $W$-state therefore gives a new way to benchmark different quantum devices with each other. 
A circuit for preparing the $W$-state was given in~\cite{Cirac:2021}, but this implementation requires sequentially alternating measurements followed by local unitaries, which in the $\LAQCC$ model is not considered to be of constant depth. 
We improve this protocol by giving an $\LAQCC$ implementation of the $W$-state, based on a compress-uncompress method that links the one-hot and binary encoding of integers.

The third state considered is the Dicke state, a generalization of the $W$-state, a superposition over all computational basis states with Hamming-weight $k$~\cite{Dicke:1954}. 
Dicke states have relevance in various practical settings.
For instance, for quantum game theory~\cite{zdemir2007}, quantum storage~\cite{Bacon_Compress:2006,Plesch:2010}, quantum error correction~\cite{ouyang2014permutation}, quantum metrology~\cite{toth2012multipartite}, and quantum networking~\cite{prevedel2009experimental}. 
Dicke states have been used as a starting state for variational optimization algorithms, most notably Quantum Alternating Operator Ansatz (QAOA)~\cite{Hadfield2019}, to find solutions to problems such as Maximum k-vertex Cover~\cite{Brandhofer2022,cook2020quantum}.
The ground states of physical Hamiltonians describing one-dimensional chains tend to show a resemblance to Dicke states such as states resulting from the Bethe ansatz, making them an ideal starting state when investigating the ground state behavior of these Hamiltonians~\cite{TDL_BetheAnsatzDerivation:2010,B_ExcitedStateQuantumPhaseTransitions:2013,DickeTransitions:2021}. 
For instance, the algorithm by \citeauthor{van2021preparing}, who give an algorithm to prepare the Bethe ansatz eigenstates of the spin-1/2 XXZ spin chain, starts by first preparing a Dicke state~\cite{van2021preparing}. 
A Dicke-state preparation protocol based on the compress-uncompress methodology used in the $W$-state furthermore finds applications in entanglement distillation, where the entanglement of a large state is concentrated on only a few qubits. 
Efficient deterministic circuits for preparing Dicke states have been proposed by \citeauthor{bartschi2019deterministic}~\cite{bartschi2019deterministic, bartschi2022deterministic_short_depth}. 
They provide a quantum circuit of depth $\mathO(k \log(\frac{n}{k}))$, allowing arbitrary connectivity, to prepare a Dicke state, which they conjecture to be optimal when $k$ is constant. 
In this work, we provide a constant-depth $\LAQCC$ circuit below their conjectured bound already for constant $k$. 
However, this does not directly disprove their conjecture, as we allow for intermediate measurements and classical computations. 
More significantly, we even construct constant-depth $\LAQCC$ circuits for $k = \mathO(\sqrt{n})$ greatly improving their bound.
This construction extends the compress-uncompress method for the $W$-state combined with additional subroutines. 

We continue with a log-depth state preparation protocol for the Dicke-state for arbitrary $k$. 
This protocol implements an efficient transformation between the factoradic number representation and the combinatorial number representation of a positive integer. 
The combinatorial number representation relates directly to the Dicke state. 
The provided efficient transformation between number representation systems might be of independent interest. 

We conclude by modifying our protocol for preparing a Dicke-state to a protocol that prepares quantum many-body scar states in constant-depth. 
These states have low entanglement and longer coherence times than states with similar energy density.
These characteristics make many-body scar states interesting to analyze and relevant within physics.
Many-body scar states appear for instance in the AKLT model~\cite{AKLT:1987,MRBAR:2018,MRB:2018} and different spin models~\cite{SI:2019,MOBFR:2020}.
Known methods for preparing these states have polynomial-depth~\cite{Gustafson:2023}, whereas our circuit has constant depth. 

% We conclude by studying the power that intermediate classical calculations can add to quantum computations. 
% In this study, we define a new model that relaxes constant-depth quantum circuits to polynomial depth quantum circuits, log-depth classical calculations to unbounded classical computations and a constant number of alternations to a polynomial number of alternations. 
% We call this model $\LAQCC^*$. 
% We study this model by doing a complexity theoretical analysis, where we draw inspiration from the notions of complexity given by \citeauthor{RosenthalYuen:2022}, \citeauthor{MetgerYuen:2023}, and \citeauthor{Aaronson:2004}.
% All three complexity notions are based on the notion of state preparation, instead of more traditional definition of complexity such as the decidability of a computational problem. 
% The first two consider classes based on sequences of quantum states preparable by a polynomial-sized quantum circuit, where the circuits are uniformly generated by a computational class, for instance, the class $\mathsf{PSPACE}$, which results in the complexity class $\mathsf{StatePSPACE}$~\cite{RosenthalYuen:2022,MetgerYuen:2023}.
% The third notion considers a relative complexity, where the complexity is measured between two given states, and is measured by the number of gates, from a given gate-set, required to transform one state in another state~\cite{Aaronson:2004}. 
% For our definition of state preparation complexity, we drop the uniformity constraint from~\cite{RosenthalYuen:2022,MetgerYuen:2023} and define a class as $\mathsf{StateX}$, which refers to states preparable by circuits of type $\mathsf{X}$. 
% As an example, if $\mathsf{X} = \QNC^0$, this results in the class $\mathsf{StateQNC^0}$, which is the set of states preparable from the $\ket{0}^n$ state by poly-size constant-depth circuits. 
% This notion is similar to the relative complexity from~\cite{Aaronson:2004}, where one state is the  $\ket{0}^n$ state and instead of counting the number of gates we consider the set of states preparable by a fixed number of gates. Using this notion of complexity we show that any state preparable by an $\LAQCC^*$ circuit is also preparable by a $\mathsf{PostQPoly}$ circuit, the class of circuits of polynomial depth with an additional post-selection gate. 

\paragraph{Summary of results}
\begin{itemize}
    \item We give a new definition of a computational model that captures the power of the four step process: applying a constant number of layers of one- and two-qubit gates; performing a syndrome measurement; perform a fast classical computation determining corrections; apply corrections. We call this model \emph{Local Alternating Quantum Classical Computations}, or $\LAQCC$ for short. In this model we bound the allowed quantum operations, intermediate classical calculations, and number of rounds separately. In Section~\ref{sec:LAQCC_model} we define this model and give a list of operations based on results from literature contained in this computational model. In some of these operations we explicitly use that we allow for multiple, but at most constant, rounds  of corrections.
    \item  We show show that there exist $\LAQCC$ circuits that can not be weakly simulated in Section~\ref{sec:IQP_in_LAQCC}. We further show that for every $\LAQCC$ circuit there exists a $\QNC^1$ circuit simulating it perfectly, in Section~\ref{sec:LAQCC_in_QNC1}.
    \item We introduce a new type computational complexity for preparing states and show that the extension of $\LAQCC$ where we allow a polynomial number of rounds and unbounded classical computation, is contained in $\mathsf{PostQPoly}$, the class of polynomial circuits with post-selection, in Section~\ref{sec:Complexity results}.
    \item We show a protocol to prepare the uniform superposition state of size $q$ in $\LAQCC$ using $\mathO(\ceil{\log_2(q)}^2)$ qubits in Section~\ref{sec:superposition_modulo_q}. 
    \item We show a protocol to prepare the $W_n$ state in $\LAQCC$ using $\mathO(n\log(n))$ qubits in Section~\ref{sec:W_state_in_LAQCC}.
    \item We show two ways of preparing the Dicke-$(n,k)$ state. The first method is in $\LAQCC$, works up to $k = \mathO(\sqrt{n})$, uses $\mathO(n^2\log(n))$ qubits, and is found in Section~\ref{sec:dicke:small_k}. The second method is in $\LAQCC\text{-}\mathsf{LOG}$ (an extension of $\LAQCC$ allowing for logarithmic number of alterations instead of constant), works for any $k$, uses $\mathO(\text{poly}(n))$ qubits, and is found in Section~\ref{sec:Dicke_in_LAQCC_LOG}. 
    \item We extend on our $\LAQCC$ method of generating Dicke-$(n,k)$ states for $k = \mathO(\sqrt{n})$ and show a protocol to generate many-body scar states for a particular Hamiltonian in $\LAQCC$ (Section~\ref{sec:many_body_scar}). 
\end{itemize}
Summarized in a table, we provide the following state generation protocols:
\begin{table}[htb]
\centering
\begin{tabular}{l|l|l|l}
\textbf{State description} & \textbf{Width} & \textbf{Depth} & \textbf{Implementation}\\
\hline 
Uniform superposition mod $q$: $\frac{1}{\sqrt{q}} \sum_{i = 0}^{q-1}\ket{i}$ & $\mathO(\ceil{\log^2 q})$ & $\mathO(1)$ & Section~\ref{sec:superposition_modulo_q}\\

$W$-state: $\frac{1}{\sqrt{n}}\sum_{i = 0}^{n-1}\ket{e_i}$ & $\mathO(n \log n)$ & $\mathO(1)$ & Section~\ref{sec:W_state_in_LAQCC}\\

Dicke-$(n,k)$, $k = \mathO(\sqrt{n})$: $\binom{n}{k}^{-1/2}\sum_{x \in \{0,1\}^n: |x| = k} \ket{x}$ &  $\mathO(n^2\log n)$ & $\mathO(1)$ 
&Section~\ref{sec:dicke:small_k}\\

Dicke-$(n,k)$: $\binom{n}{k}^{-1/2}\sum_{x \in \{0,1\}^n: |x| = k} \ket{x}$ & $\mathO(\text{poly}(n))$ & $\mathO(\log n)$ &Section~\ref{sec:Dicke_in_LAQCC_LOG}\\

QMBS: $\ket{S_k} = \frac{1}{k! \sqrt{\mathcal N(n,k)}}(Q^\dagger)^k \ket{\Omega}$ &  $\mathO(n^2\log n)$ & $\mathO(1)$  &  Section~\ref{sec:many_body_scar}
\end{tabular}
\caption{Summary of state preparation protocols given in this paper.}
\label{tab:sate_prep}
\end{table}
In the entry for the quantum many-body scar state $Q$ denotes the raising operator and $\mathcal N(n,k)=\binom{n-k-1}{k}$. 
Section~\ref{sec:many_body_scar} will provide more details on the variables and the implementation. 

\paragraph{Organization of the paper}
\noindent We first introduce relevant preliminaries in Section~\ref{sec:preliminaries}. 
In Section~\ref{sec:LAQCC_model} we formally define the class of Local Alternating Quantum-Classical Computations ($\LAQCC$). We also show that any Clifford circuit can be implemented in constant depth $\LAQCC$ (a result based on a result from measurement-based quantum computing~\cite{jozsa2006introduction}). 
This result allows us to give many useful multi-qubit gates and routines in Section~\ref{sec:gates_created_in_LAQCC}. 
Beyond that we show that constant depth $\LAQCC$ circuits are contained in $\QNC^1$ and that any $\mathsf{IQP}$ circuit has an $\LAQCC$ implementation.
We conclude this section with an analysis of a more powerful instantiation of $\LAQCC$ and show an inclusion with respect to the class $\mathsf{PostQPoly}$, which is the class of circuits of polynomial depth with one additional post-selection gate. 
In Section~\ref{sec:state_prep_in_LAQCC} we give $\LAQCC$ circuit implementations for preparing the uniform superposition over an arbitrary number of states, the $W$-state and the Dicke state up to $k = \mathO(\sqrt{n})$. We furthermore give a log-depth circuit implementation for preparing the Dicke state for any $k$. We conclude by showing a $\LAQCC$ circuit for generating many body scar states of a particular type of Hamiltonian.


%-------------------------------------------------------------------------------
\section{Related Works} \label{works}
%-------------------------------------------------------------------------------
\parab{Internet and Datacenter multicast.} Multicast has been widely applied in large-scale Internet applications, such as Internet broadcast \cite{iptv}, video conferencing \cite{chen2011celerity}, and multiplayer games \cite{cho2009enabling}, \etc Prior works for the Internet \cite{chiang2018online, huang2016multicast,diab2020oktopus,ren2018optimal, diab2022yeti} mostly focus on the multicast routing, \ie, to find promising multicast paths, inside ISPs. For instance, Yeti~\cite{diab2022yeti} supports multicast routing with traffic engineering and service chaining requirements for large-scale ISPs. Yeti creates labels representing forwarding information for multicast graphs and processes these labels to forward packets to targeted paths. Although there are a bunch of prior works on the Internet, most of them merely provide best-effort delivery, which only works for applications without reliability requirement. 

There are some works~\cite{widmer2001extending, rizzo2000pgmcc} aim to provide reliability for datacenter applications upon approaches with best-effort delivery. However, existing reliable multicast solutions mainly adopt a TCP-like software stack and cannot meet the demand for high-speed communication in datacenters. In contrast, \sys  leverages the advanced RDMA stacks to process multicast traffic, providing high-speed reliable communication.

%And there are few works for in-fabric multicast in datacenters with technical details~\cite{sharp}. \todo{software datacenter multicast.} 
%MTRSA \cite{huang2016multicast} is a multi-tree routing algorithm attempting to optimize multicast routing paths inside ISP with consideration of node and link capacity. \cite{ren2018optimal} aims to find the optimal path to forward traffic while preserving the ordered access of a sequence of network services. These services are usually deployed as virtual functions on core routers in the ISP network. 

\parab{Multicast scalability.}
Datacenter applications impose a demand for high scalability. As the traditional IP multicast~\cite{crowcroft1988multicast}, along with its native group management, IGMP and tree construction protocol, PIM~\cite{estrin1998protocol}, are poor in scalability, many works~\cite{shahbaz2019elmo, diab2022orca, li2013scaling} attempt to address the scalability issue, \ie, supporting as much as possible multicast groups. For example, Elmo~\cite{shahbaz2019elmo} encodes the routing link of a multicast tree into rules formatted as packet header. Thus Elmo switch only needs to maintain rule parsing logic, reducing the total switch-maintained states. Orca~\cite{diab2022orca} utilizes the large memory space of the server, making servers assist in forwarding packets, reducing the switch's burden on maintaining states. 

These works that address the scalability issue are orthogonal with the \sys design. Our goal in this work is to provide a general multicast protocol with prominent RDMA features and reliability guarantee rather than compressing the switch-maintained states. As mentioned before, \sys can support at least 1K multicast groups using 0.92MB space, which is acceptable for a majority of multicast applications in datacenters. \sys can support even more multicast groups when getting extended further upon these works.

%\parab{Group communication.}
%The group communication in the datacenter is not limited to one-to-many and many-to-many. There are various patterns \cite{wan2020rat, rashidi2021enabling}, such as all-gather, reduce-scatter, and all-reduce. \sys can be extended to support these group communication patterns, and we leave this as our future work.

\section{Method} \label{method_hybridaugment}
In this section, we formally define the problem, motivate our work and then present our proposed techniques.


\subsection{Preliminaries}
Let $\mathcal{F}(x;W)$ be an image classification CNN trained on the training set $\mathcal{T}_\text{train} = (x_{i}, y_{i})^{N}_{i=1}$  with $N$ samples, where $x$ and $y$ correspond to images and labels. The clean accuracy (CA) of $\mathcal{F}(x;W)$ is formally defined as its accuracy over a clean test set $\mathcal{T}_\text{test} = (x_{j}, y_{j})^{M}_{j=1}$. Assume two operators ${A}(\cdot)$ and ${C}(c, s)$ that adversarially attacks or corrupts a given set of images with the corruption category $c$ and severity $s$, respectively.  Let $A\mathcal{T}_\text{test}$ and $C\mathcal{T}_\text{test}$ be the adversarially attacked and corrupted versions of $\mathcal{T}_\text{test}$, and let $\mathcal{F}(x;W)$ have a robust accuracy (RA) on $A\mathcal{T}_\text{test}$ and a corruption accuracy (CRA) on $C\mathcal{T}_\text{test}$. 
The aim is to fit $\mathcal{F}(x;W)$ such that the model gains robustness (\ie. increased RA and CRA compared its the baseline version), while retaining (or improving) the clean accuracy of its baseline version trained without robustness concerns.


\noindent \textbf{What we know.} Our work builds on the following crucial observations: i) CNNs favour high-frequency content \cite{wang2020high}, ii) adversaries and corruptions often reside in high-frequency \cite{wang2020towards}, iii) images are dominated by low-frequency \cite{Saikia_2021_ICCV} and iv) models relying on low-frequency components are more robust \cite{li2022robust,wang2020towards}. The robustness-accuracy trade-off is visible; low-frequency reliant models are more robust, but tend to miss out on clean accuracy brought by the high-frequency components. 

\subsection{HybridAugment}
We hypothesize that a \textit{sweet spot} in the robustness-accuracy trade-off can be found. Unlike the \textit{hard} approaches that completely rule out the reliance on high-frequency components (i.e. low-pass filters), we propose to \textit{reduce} the reliance on them. To this end, we adopt a data augmentation approach that aims to diversify $\mathcal{T}_\text{train}$ by an operation $\mathcal{HA(\cdot)}$. Keeping the strong relation intact between labels and low-frequency content (i.e. labels come from low-frequency-component image), we propose to swap high and low-frequency components of images in a batch on-the-fly. Unlike \cite{mukai2022improving}, we \textit{do not} restrict the images to belong to the same class; this diversifies the training distribution even further while preserving the image semantics. We call this basic version of our approach \textit{HybridAugment}, which corresponds to: 
%
\begin{equation} \label{hybrid_augment_paired}
    \mathcal{HA_{P}}(x_{i}, x_{j}) = \mathcal{LF}(x_{i}) + \mathcal{HF}(x_{j})
\end{equation}
%
where $x_{i}$ is the input image and $x_{j}$ is a randomly sampled image from the whole training set, which we simply sample from the mini batch at each training iteration in practice. $\mathcal{HF}$ and $\mathcal{LF}$ operators select the high and low-frequency components of an input image, for which we use:
%
\begin{equation} \label{eq:cutoff}
\begin{split}
    \mathcal{LF}(x) = GaussBlur(x) \\
    \mathcal{HF}(x) = x - \mathcal{LF}(x)
    \end{split}
\end{equation}
%
where $GaussBlur$ is used as a low-pass filter. Note that a similar outcome is possible by using Discrete Fourier Transforms (DFT), swapping the frequency bands and then applying Inverse DFT (IDFT). We find the gaussian blur operation to be faster and better in practice. 


Inspired from \cite{chen2021amplitude}, in addition to the image-pair scheme in Eq.~\ref{hybrid_augment_paired}, we propose a single image variant of \textit{HybridAugment}. In the single image variant, instead of combining two images, $x_i$ and $x_{j}$ are obtained by applying randomly sampled augmentations to a single image. The single image variant $\mathcal{HA_{S}}$ can therefore be defined as 
%
\begin{equation} \label{hybrid_augment_single}
    \mathcal{HA_{S}}(x_{i}) = \mathcal{LF}(Aug(x_{i})) + \mathcal{HF}(\hat{Aug}(x_{i}))
\end{equation}
%
where $Aug$ and $\hat{Aug}$ correspond to two sets of randomly sampled augmentation operations. Note that paired and single versions can work in tandem ($\mathcal{HA_{PS}}$), and actually outperform single or paired image versions. 


\subsection{HybridAugment++}


The frequency analysis is a vast literature, however, two core aspects often stand out; frequency-band analysis (i.e. low, high) and the decomposition of signals into amplitude and phase. \textit{HybridAugment} covers the former and shows competitive results in various benchmarks (see Section \ref{sec:exp_hybridaugment}). The latter is investigated in $\mathcal{APR}$ \cite{chen2021amplitude}, where phase is shown to be the more relevant component for correct classification, and training models based on their phase labels and swapping amplitude components of images randomly lead to more robust models. Note that frequency-band and phase/amplitude discussions are arguably orthogonal, since frequency, phase and amplitude provide distinct characterizations of a signal: intuitively speaking, frequency, phase and amplitude can be seen as the separation of visual patterns in terms of scale, location and significance. 


We hypothesize these two approaches can be complementary; a model reliant on low-frequency and spatial information (i.e. phase) can further improve robustness. Inspired by the successes of cascaded augmentation methods \cite{hendrycks2019augmix,wang2021augmax,calian2022defending}, we unify these two core aspects into a single, hierarchical augmentation method. We refer to this method as \textit{HybridAugment++} and define its paired version as:
%
\begin{equation}
  \mathcal{HA_{P}}^{++}(x_{i}, x_{j}, x_{z}) = \mathcal{APR_{P}}(\mathcal{LF}(x_{i}), x_{z}) + \mathcal{HF}(x_{j})
\end{equation}
%
where $x_{i}$, $x_{j}$ and $x_{z}$ are images sampled from the same batch. Here, $\mathcal{APR_{P}}$~\cite{chen2021amplitude} is defined as
\begin{equation}
    \mathcal{APR_{P}}(x_{i}, x_{z}) = \mathcal{IDFT}(A_{x_{z}} \otimes e^{i. P_{x_{i}}}) \\
\end{equation}
%
where $\otimes$ is element-wise multiplication, $A$ is the amplitude and $P$ is the phase component. Similar to $\mathcal{HA}$ and $\mathcal{APR}$, we also define a single-image version of \textit{HybridAugment++} as
%
\begin{equation}
 \mathcal{HA_{S}}^{++}(x_{i}) = \mathcal{APR_{S}}(\mathcal{LF}(Aug(x_{i}))) + \mathcal{HF}(\hat{Aug}(x_{i}))
\end{equation}
%
where $\mathcal{APR_{S}}$~\cite{chen2021amplitude} is defined as
%
\begin{equation}
\mathcal{APR_{S}}(x_{i}) = \mathcal{IDFT}\left(A_{\bar{Aug}(x_{i})} \otimes e^{i. P_{\overline{Aug}\left(x_{i}\right)}}\right)    
\end{equation}
%
where $Aug$, $\hat{Aug}$, $\bar{Aug}$ and $\overline{Aug}$ are different sets of randomly sampled augmentation operations. Note that we essentially propose a framework; one can use different single and paired image augmentations, either individually or together, and can still achieve competitive results (see ablations in Section \ref{sec:exp_hybridaugment}). There are also other alternatives, such as swapping phase/amplitude first and then performing $\mathcal{HA}$, but we observe poor performance in practice; dividing the phase component into frequency-bands is not interpretable as frequencies of the phase component are not well defined. The pseudo-code of our methods can be found in the supplementary material.




%\section{Motivation}
\label{sec:motivation}

IGNORE THIS FILE, WILL DO IN INTRO

\section{Experiments}
% \haizhou{Follow the same way of introduction as we did in Section2.}
% \noindent In this section, we will introduce datasets and experimental setups that we used. Then we evaluate our method, other self-supervised methods, and supervised methods under different distribution shifts (\ie, concept shifts and covariate shifts) under common settings (\ie, transductive, inductive settings). It has to note that we focus on node-level tasks (\eg, node classification) in this work. As for graph-level tasks, we leave it as our future work and some simple experiments can be found in Appendix~\ref{app:graph_classification}. 
In this section, we first introduce the experimental setup including datasets, training, and evaluation protocol in Section~\ref{sec:dataset}~and~\ref{sec:unsupervised}. 
% Next, we present our experimental setup and conduct extensive experiments to evaluate our method in Section~\ref{sec:unsupervised}. 
We then perform an ablation study to demonstrate the effectiveness of each proposed component in Section~\ref{sec:ablation}. 
Additionally, we analyze the impact of important hyper-parameters in Section~\ref{sec:sensitivity}. 
Subsequently, we integrate our method with various encoding models, showcasing the model-agnostic nature of our recipe in Section~\ref{sec:other_models}. 
Finally, we provide some qualitative results such as feature visualization in Section~\ref{sec:vis}.
It is important to note that we focus on node-level tasks (\eg, node classification) in this work. As for graph-level tasks, we leave it as our future work, while some simple experiments are also provided in Appendix~\ref{app:graph_classification}.

\subsection{Datasets}\label{sec:dataset}
There exist some benchmarks for evaluating graph out-of-distribution generalization~\cite{good,ji2022drugood,gds}. 
Among them, GOOD~\cite{good} is the most representative and comprehensive benchmark that curates more diverse graph datasets with diverse tasks, including single/multi-task graph classification, graph regression, and node classification involving more distribution shifts (\ie, concept shifts and covariate shifts). Hence in this work, we follow the evaluation protocol proposed in \cite{good}. Furthermore, we validate the effectiveness of our method in the datasets (\ie, Amazon-Photo, Elliptic) that are used in EERM~\cite{eerm}. The statistics and detailed introduction to these datasets can be found in Table~\ref{tab:dataset} and Appendix~\ref{app:datasets}.

\begin{table*}[htp]
\caption{The descriptions of datasets. ``Domain-Level'' means splitting by graphs, ``Time-Aware'' denotes splitting according to chronological order.``Word'' and ``Degree'' represent splitting according to word diversity and node degree respectively. ``Language'' means splitting by user language, suggesting the prediction should not be impacted by the language the user use. ``University'' denotes splitting according to the domain university, implying that the prediction of webpages should be based on word contents and link connections rather than university features. ``Color'' means that nodes are split according to node differences in covariate shift and color-label correlations in concept shift.}
\label{tab:dataset}
\centering
\begin{tabular}{cccccccc}
\toprule
Datasets     & Network Type        & \#Nodes & \#Edges & \#Attributes &\#Classes& Train/Val/Test Split     & Metric   \\
% Cora         & Artificial Transformation & 2,703   &         &              &         &                      & Accuracy \\
Amazon-Photo\footnotemark
             & Co-purchasing network      & 7,650   & 119,081   & 755          & 10      & Domain-Level         & Accuracy \\
Elliptic\footnotemark  
             & Bitcoin transactions       & 203,769 & 234,355   & 165          & 2       & Time-Aware           & F1-Score \\
GOOD-Cora    & Scientific publications    & 19,793  & 126,842   & 8,710         & 70      & Word/Degree          & Accuracy \\
% GOOD-Arxiv   & arXiv papers               & 169,343 & 2,315,598 & 128          & 40      & Time/Degree          & Accuracy \\
GOOD-Twitch  & Gamer network              & 34,120  & 892,346   & 128          & 2       & Language             & ROC-AUC  \\
GOOD-CBAS    & A BA-house graph           & 700     & 3,962     & 4             & 4       & Color                & Accuracy \\
GOOD-WebKB   & Webpage network            & 617     & 1,138     & 1,703         & 5       & University           & Accuracy \\
\bottomrule
\end{tabular}
\end{table*}
\footnotetext[5]{This dataset is adopted from~\cite{yang2016revisiting}. \cite{eerm} constructs ten graphs with different environment id’s for each graph.} 
\footnotetext[6]{The original is available on \hyperlink{https://www.kaggle.com/ellipticco/elliptic-data-set}{https://www.kaggle.com/ellipticco/elliptic-data-set}}

\subsection{Unsupervised Representation Learning}\label{sec:unsupervised}
\subsubsection{Transductive Setting}~\label{sec:trans}
% \noindent\textbf{Baselines.}\quad We conduct experiments with 12 baselines which consist of three categories: supervised methods and self-supervised generative methods, self-supervised contrastive methods. Specifically, we compare with three supervised baselines: empirical risk minimization~(ERM)~\cite{erm}, invariant risk minimization (IRM)~\cite{irm}, and a recent proposed graph OOD method dubbed EERM~\cite{eerm}. We also compare various unsupervised node-level representation learning methods: three self-supervised generative methods including GAE~\cite{gae}, VGAE~\cite{gae}, GraphMAE~\cite{gmae} and seven self-supervised contrastive methods: DGI~\cite{dgi}, MVGRL~\cite{mvgrl}, GRACE~\cite{grace}, RoSA~\cite{rosa}, BGRL~\cite{bgrl}, COSTA~\cite{costa}, SwAV~\cite{swav}. The descriptions of these methods can be found in Appendix~\ref{app:baselines}.
In this subsection, we focus on validating our proposed algorithm under the transductive setting, where the test nodes will participate in message passing~\cite{gilmer2017neural} during training following~\cite{good}. 

\noindent\textbf{Baselines.} We conduct experiments with 12 baselines from three categories: (i)~supervised methods, including empirical risk minimization~(\textbf{ERM})~\cite{erm}, invariant risk minimization (\textbf{IRM})~\cite{irm}, and a recent proposed graph OOD method \textbf{EERM}~\cite{eerm}; (ii)~self-supervised generative methods including Graph Autoencoder (\textbf{GAE})~\cite{gae}, Variational Graph Autoencoder (\textbf{VGAE})~\cite{gae}, Self-Supervised Masked Graph Autoencoders (\textbf{GraphMAE})~\cite{gmae}; (iii)~self-supervised contrastive methods including Deep Graph Infomax (\textbf{DGI})~\cite{dgi}, Contrastive Multi-View Representation Learning on Graphs (\textbf{MVGRL})~\cite{mvgrl}, Deep Graph Contrastive Representation Learning (\textbf{GRACE})~\cite{grace}, A Robust Self-Aligned Framework for Node-Node Graph Contrastive Learning (\textbf{RoSA})~\cite{rosa}, Bootstrapped Representation Learning on Graphs (\textbf{BGRL})~\cite{bgrl}, Covariance-Preserving Feature Augmentation for Graph Contrastive Learning (\textbf{COSTA})~\cite{costa}, Unsupervised Learning of Visual Features by Contrasting Cluster Assignments (\textbf{SwAV})~\cite{swav}. The detailed descriptions of these baselines can be found in Appendix~\ref{app:baselines}.

\noindent\textbf{Experimental setup.} We use the same graph encoder across different datasets for a fair comparison following~\cite{good}. We use grid search to find other hyper-parameters (\eg, learning rate, epochs) for different methods. For all experiments, we select the best checkpoints for ID and OOD tests according to results on ID and OOD validation sets following~\cite{good}, respectively. Experimental details and hyper-parameter selections are provided in Appendix~\ref{app:hyper}. For evaluating unsupervised methods, a linear classifier will be built on the frozen trained encoder after finishing pre-training. The reported results are the mean performance with standard deviation after 10 runs following~\cite{good}.

\noindent\textbf{Analysis.}\quad Based on the experimental results listed in Table~\ref{tab:trans_concept} and \ref{tab:trans_covariate}, we can draw the following conclusions: firstly, we find strong self-supervised methods (\eg, GRACE, BGRL, COSTA) are more robust to distribution shifts (concept shift in Table~\ref{tab:trans_concept} and covariate shift in Table~\ref{tab:trans_covariate}) compared to supervised methods. For instance, on GOOD-CBAS and GOOD-WebKB datasets, GRACE surpasses the best supervised method by large margins (over 6\% absolute improvement). Interestingly, we find the methods designed for OOD generalization (\ie, IRM) and graph OOD generalization (\ie, EERM) do not attain superior performance than the standard ERM on most of the datasets. For example, EERM shows superior OOD performance compared to ERM in only one experiment, and IRM outperforms ERM in four out of ten experiments across the conducted evaluations. This phenomenon is also observed in \cite{good,ahuja2020empirical,rosenfeld2021risks}, showcasing the challenge of achieving invariant prediction in non-Euclidean graph settings. 

Furthermore, our method surpasses other SOTA self-supervised methods on the OOD test set of all datasets by a considerable margin while achieving comparable performance in the in-distribution test set. For instance, on small datasets such as GOOD-CBAS and GOOD-WebKB, our method outperforms GRACE\footnote{MARIO is built up on GRACE according to our recipe. So, we make a comparison with GRACE here.} by over 2\% absolute accuracy on the OOD test set. On larger datasets such as GOOD-Cora and GOOD-Twitch, our method still outperforms other methods which shows its superiority. For instance, under covariate shift, MARIO surpasses other methods by over 7\% absolute accuracy on the GOOD-Twitch OOD test set. These statistics prove the effectiveness of our design.


\begin{table*}[htp]
\caption{Experimental results of all methods under concept shift. The bold font means the top-1 performance and the underline represents the second performance across the unsupervised methods. 'ID' represents in-distribution test performance and 'OOD' means out-of-distribution test performance. (OOM: out-of-memory on a GPU with 24GB memory)}
\label{tab:trans_concept}
\centering
\scalebox{0.95}{
\begin{tabular}{l|cc|cc|cc|cc|cc}
\toprule
\toprule
\multirow{3}{*}{concept shift} & \multicolumn{4}{c|}{GOOD-Cora}                   & \multicolumn{2}{c|}{GOOD-CBAS} & \multicolumn{2}{c|}{GOOD-Twitch} & \multicolumn{2}{c}{GOOD-WebKB} \\
                           & \multicolumn{2}{c}{word} & \multicolumn{2}{c|}{degree}& \multicolumn{2}{c|}{color}    & \multicolumn{2}{c|}{language}   & \multicolumn{2}{c}{university} \\
                           & ID         & OOD         & ID          & OOD          & ID            & OOD           & ID             & OOD            & ID            & OOD            \\
\midrule
ERM                        & 66.38±0.45 & 64.44±0.18  & 68.60±0.40  & 60.76±0.34   & 89.79±1.39    & 83.43±1.19    & 80.80±1.00     & 56.92±0.92     & 62.67±1.53    & 26.33±1.09     \\
IRM                        & 66.42±0.41 & 64.29±0.31  & 68.57±0.35  & 61.45±0.24   & 89.64±1.21    & 82.29±1.14    & 78.87±1.04     & 59.30±1.79     & 62.67±1.10    & 26.88±1.42     \\
EERM                       & 65.10±0.44 & 62.45±0.19  & 66.95±0.44  & 56.58±0.25   & 79.07±2.12    & 64.50±1.01    & OOM            & OOM            & 62.50±2.01    & 28.07±3.23      \\
\midrule
% Random-Init                & 37.53±1.74 & 32.12±1.24  & 37.82±1.71  & 27.74±1.14   &               &               &                &                & 60.33±2.21    & 27.07±1.70     \\
GAE                        & 60.65±0.89 & 58.00±0.55  & 62.59±1.11  & 53.44±0.80   & 75.28±1.36    & 68.07±2.05    & 81.25±0.81     & 51.51±1.05     & 62.17±3.34    & 25.78±1.85     \\
VGAE                       & 63.19±0.53 & 60.35±0.47  & 61.65±0.66  & 54.28±0.28   & 76.50±0.50    & 59.07±0.56    & 80.46±0.53     & 55.56±4.53     & 62.50±2.38    & 24.40±2.57     \\
GraphMAE                   & \underline{66.44±0.46} & \underline{64.87±0.30}  & 67.95±0.46  & 59.41±0.39   & 89.14±0.89    & 82.93±0.93    & 80.05±0.64     & 59.38±1.49     & 61.83±3.37    & 29.27±2.15     \\
DGI                        & 63.33±0.56 & 60.71±0.49  & 65.93±1.02  & 55.83±0.53   & 91.22±1.47    & 85.00±1.66    & 80.05±0.87     & 59.16±1.88     & 61.83±2.83    & 28.63±1.92      \\
MVGRL                      & OOM        & OOM         & OOM         & OOM          & 88.57±1.15    & 76.50±1.17    & OOM            & OOM            & 62.00±3.79    & 28.26±4.20     \\
GRACE                      & 65.61±0.61 & 63.92±0.44  & \textbf{68.59±0.35}  & 60.15±0.45   & 92.00±1.39    & 88.64±0.67    & \textbf{83.43±0.63}     & \underline{60.45±1.46}     & 64.00±3.43    & \underline{34.86±3.43}  \\
RoSA                       & 64.06±0.67 & 62.44±0.39  & 67.07±0.65  & 57.68±0.44   & 90.78±2.27    & 85.93±2.14    & 82.39±0.42     & 57.45±2.16     & 64.17±4.10    & 32.20±2.15     \\
BGRL                       & 65.18±0.43 & 63.43±0.45  & 66.83±0.80  & 59.63±0.38   & 92.36±1.16    & 87.14±1.60    & 82.52±0.60     & 55.48±1.48     & 63.67±2.33    & 31.47±3.43     \\
COSTA                      & 65.05±0.80 & 62.37±0.45  & 66.76±0.87  & 55.73±0.36   & \underline{93.50±2.62}    & \underline{89.29±3.11}    & 83.15±0.30 & 55.03±3.22     & 61.66±2.58    & 32.39±2.13 \\
% ArCL                       &            &             & 67.64±0.57  & 59.71±0.44   &               &               &                &                & 65.00±3.94    & 35.41±1.97 \\      
SwAV                       & 62.22±0.53 & 59.79±0.53  & 64.65±0.94  & 55.06±0.39   & 89.00±0.79    & 81.72±0.66    & \underline{83.32±0.15}     & 59.69±1.97     & \underline{65.17±3.76}    & 29.36±2.01    \\
\midrule
MARIO                       & \textbf{67.11±0.46} & \textbf{65.28±0.34}  & \underline{68.46±0.40}  & \textbf{61.30±0.28}   & \textbf{94.36±1.21}    & \textbf{91.28±1.10}    & 82.31±0.54     & \textbf{63.33±1.72}     & \textbf{65.67±2.81}    & \textbf{37.15±2.37}     \\
\bottomrule
\end{tabular}}
\end{table*}

\begin{table*}[htp]
\caption{Experimental results of all methods under covariate shift. The bold font means the top-1 performance and the underline represents the second performance across the unsupervised methods. 'ID' represents in-distribution test performance and 'OOD' means out-of-distribution test performance. (OOM: out-of-memory on a GPU with 24GB memory)}
\label{tab:trans_covariate}
\centering
\scalebox{0.95}{
\begin{tabular}{l|cc|cc|cc|cc|cc}
\toprule
\toprule
\multirow{3}{*}{covariate shift} & \multicolumn{4}{c|}{GOOD-Cora}                                   & \multicolumn{2}{c|}{GOOD-CBAS} & \multicolumn{2}{c|}{GOOD-Twitch} & \multicolumn{2}{c}{GOOD-WebKB} \\
                           & \multicolumn{2}{c}{word} & \multicolumn{2}{c|}{degree}& \multicolumn{2}{c|}{color}    & \multicolumn{2}{c|}{language}   & \multicolumn{2}{c}{university} \\
                           & ID         & OOD         & ID          & OOD          & ID            & OOD           & ID             & OOD            & ID            & OOD            \\
\midrule
ERM                        & 70.50±0.41 & 64.69±0.33  & 72.46±0.49  & 55.53±0.50   & 92.00±3.08    & 77.57±1.29    & 70.98±0.41     & 49.35±5.09     & 39.34±1.79    & 14.52±3.14   \\
IRM                        & 70.48±0.26 & 64.53±0.57  & 71.98±0.34  & 53.72±0.46   & 90.86±2.41    & 78.86±1.67    & 69.81±0.95     & 49.11±2.82     & 38.52±3.30    & 13.97±2.80     \\
EERM                       & OOM        & OOM         & OOM         & OOM          & 65.00±2.57    & 57.43±3.60    & OOM            & OOM            & 46.07±4.55    & 27.40±7.65     \\
\midrule
GAE                        & 56.63±0.79 & 48.93±0.93  & 66.30±0.88  & 34.01±0.87   & 73.00±2.16    & 60.86±3.01    & 67.24±1.23     & 47.65±2.49     & 45.08±6.32    & 28.02±6.29    \\
VGAE                       & 62.02±0.66 & 54.12±0.86  & 69.41±0.57  & 44.20±1.29   & 62.29±2.04    & 63.29±1.11    & 66.99±1.43     & \underline{50.48±4.58}     & 48.85±4.68    & 20.87±6.69     \\
GraphMAE                   & 68.14±0.43 & 64.00±0.33  & \textbf{73.36±0.56}  & 53.75±0.55   & 67.28±3.03    & 67.28±1.49    & 68.84±1.20     & 48.02±2.79     & 48.03±4.34    & 30.00±8.09     \\
DGI                        & 60.85±0.75 & 57.03±0.67  & 68.97±0.41  & 41.75±0.88   & 69.57±4.09    & 59.71±3.43    & 68.43±1.05     & 44.83±1.61     & 48.52±5.04    & 21.11±7.50     \\
MVGRL                      & OOM        & OOM         & OOM         & OOM          & 65.00±1.94    & 64.15±0.77    & OOM            & OOM           & \textbf{54.10±5.39}    & 16.59±6.51     \\
GRACE                      & \underline{68.77±0.33} & \underline{64.21±0.41}  & 72.69±0.34  & \underline{56.10±0.63}   & \underline{93.57±1.83}    & \underline{89.29±3.40}    & \underline{71.12±0.87} & 46.21±1.54 & 49.67±5.82    & 28.10±4.68    \\
RoSA                       & 68.19±0.56 & 62.48±0.61  & 71.04±0.62  & 52.72±0.79   & 84.71±4.14    &79.14±3.51     & 70.58±0.36     & 45.83±1.72     & 52.30±4.24    & \underline{34.24±7.92}     \\
BGRL                       & 67.23±0.43 & 61.33±0.36  & 72.11±0.39  & 49.15±0.73   & 89.00±2.56    & 79.86±3.29    & \textbf{71.43±0.53}     & 43.86±0.94     & 51.80±5.55    & 30.32±7.61    \\
COSTA                      & 65.28±0.60 & 60.33±0.53  & 70.65±0.62  & 54.03±0.28   & 92.29±1.59    & 82.71±2.74    & 69.29±1.37     & 49.07±2.13     & 50.49±3.01    & 29.84±4.75   \\
SwAV                       & 63.29±1.01 & 56.98±0.94  & 70.27±0.73  & 43.00±0.52   & 89.57±1.12    & 81.43±1.69    & 69.19±0.93     & 49.37±2.96     & 49.84±4.82    & 30.55±6.72   \\
\midrule
MARIO                       & \textbf{69.99±0.54} & \textbf{65.06±0.34}  & \underline{72.73±0.43}  & \textbf{57.73±0.45}  & \textbf{94.57±2.46}    & \textbf{91.00±2.48}     & 68.31±0.78 & \textbf{57.37±1.37}     & \underline{53.94±3.23}    & \textbf{35.24±4.98}   \\
\bottomrule
\end{tabular}}

\end{table*}

\subsubsection{Inductive Setting}
In this subsection, we conduct experiments under the inductive settings, where the test nodes are kept unseen during training. This setting is more suitable for domain generalization.
% But we think it is more convincing that conduct experiments under inductive settings which means test nodes are unseen during training. This setting is more appropriate for domain generalization.

\noindent\textbf{Baselines:} For GOOD-WebKB and GOOD-CBAS datasets, we adopt ERM, IRM, GraphMAE, and GRACE as our baselines. And for Amazon-Photo and Elliptic datasets, we select ERM, EERM, and GRACE as our baselines.

\noindent\textbf{Experimental setup:} For GOOD-WebKB and GOOD-CBAS datasets, we use the same model configuration in Section~\ref{sec:trans}.
% Besides, we add experiments on Amazon-Photo dataset~\cite{yang2016revisiting} and Elliptic~\cite{elliptic} dataset in this subsection. 
For Amazon-Photo dataset~\cite{yang2016revisiting} and Elliptic~\cite{elliptic} dataset, they consist of many snapshots (training data and testing data use different snapshots) which are naturally inductive. For Amazon-Photo dataset, we use 2-layer GCN~\cite{gcn} as the encoder and for elliptic dataset, we use 5-layer GraphSAGE~\cite{sage} as encoder following~\cite{eerm}.

% Figure environment removed

\noindent\textbf{Analysis:}
According to Figure~\ref{fig:amazon},\ref{fig:elliptic},\ref{fig:ind_con},\ref{fig:ind_cov}, we can draw following conclusions:
firstly, based on Figure~\ref{fig:amazon}, it is evident that our method outperforms other representative supervised and self-supervised methods on all test graphs (T1$\sim$T8). This superiority is reflected in the larger median value of our method compared to others. For instance, MARIO achieves over a 3\% absolute improvement compared to ERM in terms of the mean value of eight median values. Additionally, our method demonstrates higher stability across different random initializations, as indicated by the closer proximity of the first and third quartile values to the median value~(\eg, the difference of first and third quartile values of ERM, EERM, GRACE and MARIO are 4.2, 3.3, 6.7 and 1.0 on T8 respectively which indicates MARIO is much more stable than other methods). Furthermore, our method exhibits consistent performance across different graphs (\eg, The standard deviation of median values on T1$\sim$T8 for ERM, EERM, GRACE, and MARIO are 0.4, 1.1, 1.2, and 0.3, respectively.), indicating its robustness to environmental variations and its ability to extract invariant features: $g(G^e) \approx g(G^{e'})$ for all $e, e' \in \mathcal{E}^\text{train}$. In summary, our method showcases enhanced OOD generalization capabilities.
% $g(G^e)g(G^e^\prime)$ where $any e, e^\prime in \mathcal{E}^{train}$

Secondly, from the results presented in Figure~\ref{fig:elliptic}, we can observe that our method averagely harvests 10.9\% absolute improvement over GRACE and 12.5\% absolute improvement over EERM in terms of F1 scores on Elliptic dataset. This demonstrates the effectiveness of our method in handling distribution shifts and improving performance compared to existing approaches. It is worth noting that GRACE's performance worsens over time, indicating its inability to handle distribution shifts effectively. In contrast, our method consistently achieves better F1 scores, except for T9, which is caused by the dark market shutdown occurred after T7~\cite{elliptic}. The emergence of such an event introduces significant variations in data distributions, which subsequently results in performance degradation for all methods. Indeed, this event serves as an unpredictable external factor that introduces significant challenges for models trained on limited training data. The results indicate that the performance heavily depends on available training data. Nonetheless, our approach outperforms other methods even in such an extreme case. This highlights the effectiveness of our method in addressing distribution shifts and improving generalization performance.

Finally, based on the observations from Figure~\ref{fig:ind_con} and Figure~\ref{fig:ind_cov} MARIO demonstrates the best performances on both ID and OOD test sets for GOOD-WebKB and GOOD-CBAS datasets, under both concept shift and covariate shift. Notably, MARIO outperforms other methods by more than 3\% and 10\% absolute improvement on GOOD-WebKB and GOOD-CBAS, respectively, under covariate shift. We can draw similar conclusions as discussed in Section~\ref{sec:trans}. Even under the inductive setting, our method continues to demonstrate excellent OOD generalization capabilities and achieves comparable or even improved in-distribution test performance. These statistical results further validate the effectiveness of our method in handling distribution shifts and enhancing generalization performance.

Overall, the observations we have made provide strong evidence of the great capacity of our method for handling distribution shifts, validating its effectiveness and potential for real-world applications.



% Figure environment removed

% Figure environment removed


% Figure environment removed


\subsection{Ablation Studies}\label{sec:ablation}
\noindent Table~\ref{tab:aba} provides a detailed analysis of the effect of each component according to our proposed recipe for improving OOD generalization in graph contrastive learning. Let's examine the different variants of our method and their impact on performance.
Specifically, MARIO~(w/o ad) represents MARIO without  adversarial augmentation. MARIO~(w/o cmi) denotes we only maximize the mutual information between positive pairs without considering conditional mutual information. MARIO~(w/o cmi, ad) means a vanilla graph contrastive method that is similar to GRACE. 

From Table~\ref{tab:aba}, we can find MARIO~(w/o cmi) lags far behind MARIO on OOD test set which demonstrates appropriately minimizing the redundant information (\ie, conditional mutual information) is essential to improve OOD generalization of GCL methods. And adversarial augmentation can also boost OOD generalization because it can approximately serve as a supermum operator to learn more invariant features  discussed in Section~\ref{sec:aug}. Based on the analysis of these variants, it is evident that the proposed improvements on data augmentation and contrastive loss in the recipe are both effective in enhancing graph OOD generalization. Each component contributes to the overall performance improvement, and their combination leads to a stronger self-supervised graph learner in terms of graph OOD generalization. 

In short, the findings from Table~\ref{tab:aba} support the rationale behind your proposed recipe and provide empirical evidence of the effectiveness of each proposed component. By incorporating these enhancements, our method achieves superior performance in handling distribution shifts and improving graph OOD generalization in graph contrastive learning.
\begin{table*}[htp]
\caption{Ablation studies for MARIO by masking each component.}
\label{tab:aba}
\centering
\scalebox{0.9}{
\begin{tabular}{l|cc|cc|cc|cc|cc}
\toprule
\toprule
\multirow{3}{*}{concept shift} & \multicolumn{4}{c|}{GOOD-Cora}                       & \multicolumn{2}{c|}{GOOD-CBAS} & \multicolumn{2}{c|}{GOOD-Twitch} & \multicolumn{2}{c}{GOOD-WebKB} \\
                           & \multicolumn{2}{c}{word} & \multicolumn{2}{c|}{degree}& \multicolumn{2}{c|}{color}    & \multicolumn{2}{c|}{language}   & \multicolumn{2}{c}{university} \\
                           & ID         & OOD         & ID          & OOD          & ID            & OOD           & ID             & OOD            & ID            & OOD            \\
\midrule
MARIO                      & \textbf{67.11±0.46} & \textbf{65.28±0.34}  & \textbf{68.46±0.40}  & \textbf{61.30±0.28}      & \textbf{94.36±1.21}  & \textbf{91.28±1.10}    & 82.31±0.54     & \textbf{63.33±1.72}     & \textbf{65.67±2.81}    & \textbf{37.15±2.37}     \\
MARIO(w/o ad)              & 66.23±0.53 & 64.02±0.18  & 67.88±0.38  & 60.46±0.29   & 93.21±1.25    & 90.29±0.91    & 82.42±0.73     & 60.50±1.02     & 64.83±2.83    & 36.51±3.25    \\
MARIO(w/o cmi)             & 65.32±0.60 & 63.51±0.32  & 68.14±0.32  & 61.19±0.34   & 94.15±1.23    & 90.57±1.96    & \textbf{82.51±0.56}     & 61.41±2.63     & 64.50±4.35    & 35.78±2.53     \\
MARIO(w/o cmi, ad)         & 64.67±0.55 & 63.11±0.32  & 67.95±0.65  & 60.01±0.57   & 93.36±1.66    & 89.64±1.73    & 81.90±0.75     & 60.12±1.60     & 64.17±3.67    & 34.13±2.38     \\
\bottomrule
\end{tabular}}
\end{table*}
% & 65.32±0.60 & 63.51±0.32 exchange 64.67±0.55 & 63.11±0.32
% 68.14±0.32       id ood test: 60.95±0.43       ood ood test: 61.19±0.34


\subsection{Sensitivity Analysis}\label{sec:sensitivity}
\noindent In this subsection, we will analyze some important hyper-parameters of our method. We conduct sensitivity analysis on GOOD-WebKB dataset with concept shift, we chose two sensitive hyper-parameters (\ie, the coefficient $\gamma$ of condition mutual information in Equation~\ref{equ:cmi} and the number of prototypes $|C|$ in Equation~\ref{equ:pq}). The coefficient of CMI range in $[0.001, 0.01, 0.1, 0.5, 1]$ and the number of prototypes $|C|$ ranges in $[10, 50, 100, 200, 300]$. From Figure~\ref{fig:sensitivity}, we can observe that $\gamma$ reaches 0.1 and $|C|$ reaches 100 or 200 can achieve the best OOD test accuracy. Both higher and lower values of $\gamma$ result in suboptimal performance. This finding aligns with previous research such as DIB~\cite{dib}, indicating that an appropriate compression level is crucial for achieving optimal performance. Extremely high or low compression values are not ideal. 

Regarding the number of prototypes $|C|$, based on the results shown in Figure~\ref{fig:sensitivity}, it is found that setting $|C|=100$ leads to the best performance in terms of OOD test accuracy. This choice provides a moderate number of pseudo labels, which is beneficial for the learning process. 

Based on the sensitivity analysis, we determined that setting $\gamma=0.1$ and $|C|=100$ on most datasets. These hyperparameter values strike a balance between compression level and the number of prototypes, resulting in improved graph OOD generalization.
% Figure environment removed


\subsection{Integrated with Other Models}\label{sec:other_models}
% Figure environment removed

\begin{table}[htp]
\caption{Results of different learning approaches with different encoding models (\ie, GCN, GraphSAGE, GAT).}
\label{tab:others}
\centering
\scalebox{0.9}{
\begin{tabular}{cc|cc|cc}
\toprule
\toprule
\multirow{3}{*}{Model}& \multirow{3}{*}{Method} & \multicolumn{2}{c|}{GOOD-CBAS} & \multicolumn{2}{c}{GOOD-WebKB} \\
                & & \multicolumn{2}{c|}{color}    & \multicolumn{2}{c}{university} \\
                &   & ID          & OOD         & ID          & OOD            \\
\midrule
\multirow{3}{*}{GCN} 
&ERM               & 89.79±1.39 & 83.43±1.19  &  62.67±1.53 & 26.33±1.09         \\
&GRACE             & 92.00±1.39 & 88.64±0.67  &  64.00±3.43 & 34.86±3.43        \\
&MARIO             & 94.36±1.21 & 91.28±1.10  &  65.67±2.81 & 37.15±2.37        \\ \bottomrule
\multirow{3}{*}{SAGE} 
&ERM               & 95.07±1.51 & 75.14±1.19  & 73.67±2.08  & 46.33±3.42       \\
&GRACE             & 95.29±1.11 & 74.43±2.36  & 70.50±5.06  & 49.54±3.83        \\
&MARIO             & 96.00±1.07 & 76.29±3.01  & 71.00±3.82  & 51.74±4.63        \\ \bottomrule
\multirow{3}{*}{GAT} 
&ERM               & 78.64±3.63 & 72.93±2.64  & 61.33±3.71  & 28.99±2.63        \\
&GRACE             & 84.57±1.79 & 78.36±1.60  & 59.50±2.36  & 35.78±3.26        \\
&MARIO             & 84.93±1.95 & 80.43±1.89  & 62.17±4.78  & 38.17±3.10        \\
\bottomrule
\end{tabular}}
\end{table}



\noindent In the subsection, we demonstrate the model-agnostic nature of the recipe by integrating it with various graph neural network (GNN) models, including GCN, GraphSAGE, and GAT.

From Table~\ref{tab:others}, it can be observed that regardless of the specific GNN model used as the encoder, our method consistently achieves the best performance on the OOD test set. This indicates the effectiveness and robustness of our method across different GNN models.
By achieving superior performance across different GNN models, MARIO demonstrates its versatility and ability to improve the OOD generalization of various graph neural models. This highlights the broad applicability and effectiveness of our recipe in enhancing the performance of different GNN encoders.

Furthermore, we integrate our recipe with other GCL methods in Appendix~\ref{app:other_methods}. The results demonstrate our recipe can boost the OOD generalization ability of various GCL methods which means our recipe can serve as a plug-in for many current classical GCL methods.

% Figure environment removed

\subsection{Visualization}\label{sec:vis}
\subsubsection{Metric Score Curves}
We present metric score curves for ERM and MARIO, including training, ID validation, ID testing, OOD validation, and OOD testing accuracy, in Figure~\ref{fig:curve2}. Notably, MARIO demonstrates superior convergence with approximately 10\% absolute improvement on the OOD test set compared to ERM. Furthermore, MARIO effectively narrows the performance gap between in-distribution and out-of-distribution performance, showcasing its efficacy in enhancing OOD generalization for graph data. More metric score curves can be found in Appendix~\ref{app:curves}.


\subsubsection{Feature Visualization}
In order to assess the quality of learned embeddings, we adopt t-SNE~\cite{tsne} to visualize the node embedding on GOOD-Cora dataset (concept shift in word domain) using random-init of GCN, EERM, GRACE, and MARIO, where different classes have different colors in Figure~\ref{fig:vis}. For clarity, we select eight classes with the largest number of nodes to enhance the informativeness and interpretability of the visualization. We can observe that the 2D projection of node embeddings learned by MARIO has a better separation of clusters, which indicates the model can help learn representative features for downstream tasks. It has to note that we depict both ID nodes and OOD nodes in the same figure. 

Besides, we also separately visualize ID nodes and OOD nodes in the different figures in the Appendix~\ref{app:feature}. And we can find MARIO performs a clearer separation of clusters whether on ID nodes or OOD nodes compared to other methods.



%\section{Robustness is learned first}
%In addition to top features in the eigenspace being both robust and useful, our analysis also justifies that robust features are learned first both in a random features model and also in NTK. 
%\begin{corollary}
%The prediction at time $t$ i.e. $f(x)_t$ can be written as the sum of NTK features with exponential decay depending on the eigenvalue of the feature $f(x)_t= \sum_{i=1}^{n}f_{(i)}(x)(1-e^{-\gamma t\lambda_i})$. 
%\label{corollary:ntkfeaturefirstlearned}
%\end{corollary}
%Given  Thus at a fixed time during training, robust features converge first.
\section{Conclusion and Future Work}
In this work, I design corruption-robust algorithms for the Lipschitz contextual search problem. I present the \emph{agnostic checking} technique and demonstrate its effectiveness in designing corruption-robust algorithms. There are several open problems for future research. First, in the algorithm I propose for pricing loss, the schedule for agnostic checks is fixed upfront. Can the learner design an adaptive checking schedule for the pricing loss? Second, this work assumes the learner has knowledge of the Lipschitz constant $L$. Can the learner design efficient no-regret algorithms without knowledge of $L$? 
\section*{Acknowledgements}%

The authors express their gratitude and a fond thought to Hassan \ak, who
with Gabriella Pasi set out to define a fuzzy version of OSF logic. This
paper originates from their work on the definition of similarity-based
unification for OSF terms, extending the approach of
\cite{AitKaciPasi2020}.



%\newpage
\bibliography{citations}

\newpage
\appendix
\section{Proofs of the Main Results}

In this section, we prove Theorem \ref{thm:compute-robustness}, and related results, Corollaries \ref{cor:robust_score}--\ref{cor:information} and Remark \ref{rem:thm-robust-linear}.

\subsection{Proof of Theorem \ref{thm:compute-robustness}}
\label{pf:sf_lower_bound}
\begin{proof}
%\textbf{To prove Theorem \ref{thm:compute-robustness}} and subsequent results, 
Recall that we assume 
$y = h(\mathbf{x}) + \bm{\epsilon} = \bm{\beta}^\top \phi(\mathbf{x}) + \bm{\epsilon}$, where $\bm{\epsilon} \in \mathbb{R}^C$ has independent coordinates, each satisfying $\mathbb{E}[\epsilon_c] = 0$, $\mathbb{E}[\epsilon_c^2] \leq \sigma^2$ for all $c \in \{1,\ldots,C\}$.
The features for which we wish to compute robustness are of the form $f = \bm{M}\phi$ where $\bm{M}$ is a linear map.

We are interested in robustness with respect to the $c$-th component, which is computed as
\begin{align}
s_{\mathcal{D},\bm{\beta},c}(f) &= \mathbb{E}_{(\mathbf{x},y)\sim \mathcal{D}}\left[\inf\limits_{||\tilde{\mathbf{x}}-\mathbf{x}||_2 \leq \Delta} y_c \bm{\beta}_c^\top  f(\tilde{\mathbf{x}})\right] 
\nonumber\\
&= \mathbb{E}_{(\mathbf{x},y)\sim \mathcal{D}}\left[y_c \bm{\beta}_c^\top  f(\mathbf{x})\right] 
+\mathbb{E}_{(\mathbf{x},y)\sim \mathcal{D}}\left[\inf\limits_{||\tilde{\mathbf{x}}-\mathbf{x}||_2 \leq \Delta} y_c \bm{\beta}_c^\top \big( f(\tilde{\mathbf{x}}) - f(\mathbf{x})\big)\right]  
\label{eq:robust-decomp}
\end{align}

%
We compute the first term exactly as
\begin{align*}
   \mathbb{E}_{(\mathbf{x},y)\sim \mathcal{D}}\left[y_c \bm{\beta}_c^\top  f(\mathbf{x})\right] 
   &= \mathbb{E}_{\mathbf{x},\epsilon_c}\left[ (\bm{\beta}_c^\top \phi(\mathbf{x}) + \epsilon_c) \bm{\beta}_c^\top  f(\mathbf{x})\right] 
   & (\text{since } \mathbb{E}[\epsilon_c] = 0),
   \\&= \mathbb{E}_{\mathbf{x}}\left[ \bm{\beta}_c^\top \phi(\mathbf{x}) \phi(\mathbf{x})^\top \bm{M} \bm{\beta}_c\right] 
   \\&= \bm{\beta}_c^\top \Sigma \bm{M} \bm{\beta}_c,
   &(\Sigma = \mathbb{E}_\mathbf{x} \left[\phi(\mathbf{x})\phi(\mathbf{x})^\top\right]).
\end{align*}
%
For the second term in \eqref{eq:robust-decomp}, we aim to derive a lower bound. Observe that
\begin{align*}
    y_c \bm{\beta}_c^\top \big( f(\tilde{\mathbf{x}}) - f(\mathbf{x})\big)
    &= y_c \bm{\beta}_c^\top \bm{M} \big( \phi(\tilde{\mathbf{x}}) - \phi(\mathbf{x})\big)
    \\&\geq - |y_c|\cdot \Vert\bm{\beta}_c\Vert_{\mathcal{H}} \cdot \Vert \bm{M}\big(\phi(\tilde{\mathbf{x}}) - \phi(\mathbf{x})\big)\Vert_\mathcal{H}
    \\&\geq - |y_c|\cdot \Vert\bm{\beta}_c\Vert_{\mathcal{H}} \cdot \Vert\bm{M} \Vert_{op} \cdot \Vert \phi(\tilde{\mathbf{x}}) - \phi(\mathbf{x})\Vert_\mathcal{H}.
\end{align*}
Using $L$-Lipschitzness of $\phi$, we have $\Vert \phi(\tilde{\mathbf{x}}) - \phi(\mathbf{x})\Vert_\mathcal{H} \leq L||\tilde{\mathbf{x}}-\mathbf{x}||_2 \leq L\Delta$. Hence, the second term in \eqref{eq:robust-decomp} can be bounded from below as
\begin{align*}
    \mathbb{E}_{(\mathbf{x},y)\sim \mathcal{D}}\left[\inf\limits_{||\tilde{\mathbf{x}}-\mathbf{x}||_2 \leq \Delta} y_c \bm{\beta}_c^\top \big( f(\tilde{\mathbf{x}}) - f(\mathbf{x})\big)\right] &= - \Vert\bm{M} \Vert_{op} \cdot \Vert\bm{\beta}_c\Vert_{\mathcal{H}} \cdot L\Delta \cdot \mathbb{E}_{\mathbf{x},\epsilon_c}\left[ |y_c|\right]\\
    &\geq - \Vert\bm{M} \Vert_{op} \cdot \Vert\bm{\beta}_c\Vert_{\mathcal{H}} \cdot L\Delta \cdot \mathbb{E}_{\mathbf{x},\epsilon_c}\left[ |\bm{\beta}_c^\top \phi(\mathbf{x}) + \epsilon_c|\right]  
\end{align*}
Finally, using Jensen's inequality, we can write
\begin{align*}
   \mathbb{E}_{\mathbf{x},\epsilon_c}\left[ |\bm{\beta}_c^\top \phi(\mathbf{x}) + \epsilon_c|\right] \leq \sqrt{\mathbb{E}_{\mathbf{x},\epsilon_c}\left[ (\bm{\beta}_c^\top \phi(\mathbf{x}) + \epsilon_c)^2\right] }
   \leq \sqrt{\sigma^2+ \bm{\beta}_c^\top \Sigma \bm{\beta}_c}.
\end{align*}
Combining the above computation leads to 
\begin{align*}
s_{\mathcal{D},\bm{\beta},c}(f) ~\geq~ \bm{\beta}_c^\top \Sigma \bm{M}  \bm{\beta}_c - L \Delta \Vert \bm{M}\Vert_{op} \Vert \bm{\beta}_c\Vert_\mathcal{H} \sqrt{\sigma^2 + \bm{\beta}_c^\top \Sigma \bm{\beta}_c},
\end{align*}
which proves Theorem \ref{thm:compute-robustness}.
\end{proof}

%%%%%
\subsection{Proof of Remark \ref{rem:thm-robust-linear}}
\label{pf:sf_lower_bound_tight}
\begin{proof}
The proof requires assumption of a Gaussian model, i.e., $\mathbf{x} \sim \mathcal{N}(0,\Sigma)$ and $\epsilon_c\sim \mathcal{N}(0,\sigma^2)$. Since the feature map is assumed to be linear, $\phi(\mathbf{x}) =\mathbf{x}$, it follows that $y_c = \bm{\beta}_c^\top \mathbf{x} + \epsilon_c$ is also Gaussian $y_c \sim \mathcal{N}(0,\sigma^2 + \bm{\beta}_c^\top \Sigma \bm{\beta}_c)$ and hence, $|y_c|$ is half-normal distributed.

Now recall that the first term in \eqref{eq:robust-decomp} can be computed exactly as $\bm{\beta}_c^\top \Sigma \bm{M} \bm{\beta}_c$. To compute the second term in \eqref{eq:robust-decomp}, note that
\begin{align*}
    \inf\limits_{||\tilde{\mathbf{x}}-\mathbf{x}||_2 \leq \Delta} y_c \bm{\beta}_c^\top \big( f(\tilde{\mathbf{x}}) - f(\mathbf{x})\big)
    &= \inf\limits_{||\tilde{\mathbf{x}}-\mathbf{x}||_2 \leq \Delta} y_c \bm{\beta}_c^\top \bm{M} (\tilde{\mathbf{x}}- \mathbf{x})
\end{align*}
and the infimum is achieved when the difference is aligned with $\bm{M}^\top \bm{\beta}_c$, that is,  $\tilde{\mathbf{x}} = \mathbf{x} \pm \Delta \frac{\bm{M}^\top \bm{\beta}_c}{\Vert \bm{M}^\top \bm{\beta}_c \Vert_\mathcal{H}}$.
The sign depends on the sign of $y_c$, which leads to the second term in \eqref{eq:robust-decomp} compute to
\begin{align*}
    \mathbb{E}_{(\mathbf{x},y)\sim \mathcal{D}}\left[\inf\limits_{||\tilde{\mathbf{x}}-\mathbf{x}||_2 \leq \Delta} y_c \bm{\beta}_c^\top \big( f(\tilde{\mathbf{x}}) - f(\mathbf{x})\big)\right]
    =  - \mathbb{E}_{\mathbf{x},\epsilon_c} \big[ |y_c|  \big] \cdot \Delta \cdot \Vert \bm{M}^\top \bm{\beta}_c \Vert_\mathcal{H}.
\end{align*}
Since $|y_c|$ is half-normal, $\mathbb{E}[|y_c|] = \sqrt{2/\pi}\sqrt{\sigma^2 + \bm{\beta}_c^\top\Sigma\bm{\beta}_c}$, while $\Vert\bm{M}^\top \bm{\beta}_c \Vert_\mathcal{H} \leq \Vert\bm{M}\Vert_{op} \Vert \bm{\beta}_c \Vert_\mathcal{H}$, with the inequality being tight when $\bm{\beta}_c$ is the eigenvector of $\bm{M}$, corresponding to the largest eigenvalue.
\end{proof}

%%%%%
\subsection{Proofs of Corollary \ref{cor:robust_score} and Corollary \ref{cor:information}}
\label{pf:robust_info_features}
\begin{proof}
In what follows, we restrict the linear map $\bm{M}$ as $\bm{M} = \tilde{\mathbf{U}}\tilde{\mathbf{U}}^\top = \sum_{i=1}^K \mathbf{u}_i\mathbf{u}_i^\top$ where $\tilde{\mathbf{U}} =[ \mathbf{u}_1, \ldots, \mathbf{u}_K]$ is an orthonormal matrix of basis for a $K$-dimensional subspace.
Since the operator norm $\Vert \bm{M} \Vert_{op} =1$ for projection matrix, the problem of finding the most robust subspace corresponds to maximising $\bm{\beta}_c^\top \Sigma \bm{M} \bm{\beta}_c = \sum_{i=1}^K \bm{\beta}_c^\top \Sigma \mathbf{u}_i\mathbf{u}_i^\top \bm{\beta}_c$. 


Note that if $(\lambda,\mathbf{u})$ is an eigenpair of $\Sigma$, then $\bm{\beta}_c^\top \Sigma \mathbf{u}\mathbf{u}^\top \bm{\beta}_c = \lambda (\bm{\beta}_c^\top \mathbf{u})^2$.
Hence, if we restrict the choice of $\mathbf{u}_1,\ldots,\mathbf{u}_K$ to the eigenvectors of $\Sigma$, the optimal projection is obtained by choosing the $K$ eigenvectors for which the robustness score $s_c(\mathbf{u}) = \lambda (\bm{\beta}_c^\top \mathbf{u})^2$ are largest. So the claim of Corollary \ref{cor:robust_score} holds only if the projections are restricted to eigenspaces of $\Sigma$.
The claim of Corollary \ref{cor:information} follows along the same line as the information of the feature $f = \bm{M}\phi$ can be computed as $\rho_{\mathcal{D},\bm{\beta},c}(f) = \bm{\beta}_c^\top \Sigma \bm{M} \bm{\beta}_c$. 
For the case of $\bm{M} = \tilde{\mathbf{U}}\tilde{\mathbf{U}}^\top$ where $\tilde{\mathbf{U}}$ is matrix of $K$ eigenvectors of $\Sigma$, we have $\rho_{\mathcal{D},\bm{\beta},c}(f) = \sum_{i=1}^K \lambda_i (\bm{\beta}_c^\top \mathbf{u}_i)^2$. Hence, if the search is restricted to eigenspaces of $\Sigma$, the most robust features also correspond to the most informative ones.
\end{proof}


\paragraph{Robust and informative features over all possible $K$-dimensional subspaces.}
If we consider $\bm{M} = \tilde{\mathbf{U}}\tilde{\mathbf{U}}^\top$ for any $\tilde{\mathbf{U}} =[ \mathbf{u}_1,\ldots,\mathbf{u}_K]$ with orthonormal columns, as assumed in Corollary \ref{cor:robust_score}, then 
\begin{align}
    \bm{\beta}_c^\top \Sigma \bm{M} \bm{\beta}_c
= \sum_{i=1}^K \bm{\beta}_c^\top \Sigma \mathbf{u}_i\mathbf{u}_i^\top \bm{\beta}_c
= \text{Trace}\left(\tilde{\mathbf{U}} \bm{\beta}_c \bm{\beta}_c^\top \Sigma \tilde{\mathbf{U}}^\top \right)
= \text{Trace}\left(\tilde{\mathbf{U}} \Sigma \bm{\beta}_c \bm{\beta}_c^\top \tilde{\mathbf{U}}^\top \right), \nonumber
\end{align}
where the last equality follows from taking transpose. Hence, the resulting maximisation problem can be written as 
\begin{align}
\underset{\tilde{\mathbf{U}}}{\max} ~\bm{\beta}_c^\top \Sigma \bm{M} \bm{\beta}_c 
~\equiv~ \underset{\tilde{\mathbf{U}}}{\max} ~\text{Trace}\left(\tilde{\mathbf{U}} \bm{\beta}_c \bm{\beta}_c^\top \Sigma \tilde{\mathbf{U}}^\top \right)
~\equiv~ \underset{\tilde{\mathbf{U}}}{\max} ~\text{Trace}\left(\tilde{\mathbf{U}} \mathbf{B}_c \tilde{\mathbf{U}}^\top \right),
\label{eq:alternate-max}
\end{align}
where $\mathbf{B}_c = \frac12 (\bm{\beta}_c \bm{\beta}_c^\top \Sigma + \Sigma \bm{\beta}_c \bm{\beta}_c^\top)$.
The above trace maximisation problem corresponds to finding the $K$ dominant eigenvectors of the matrix $\mathbf{B}_c$.
This leads to an alternative to Algorithm \ref{alg:cap} for finding robust projections for the test-time defense. 
The approach comprises of computing the dominant eigenvectors $\tilde{\mathbf{U}}_c$ of the matrix $\mathbf{B}_c$ for every class component $c\in\{1,\ldots,C\}$ and defining the robust output as $\tilde{h}(\mathbf{x}) = [\bm{\beta}_1^\top \tilde{\mathbf{U}}_1\tilde{\mathbf{U}}_1^\top \phi(\mathbf{x}),\ldots,\bm{\beta}_C^\top \tilde{\mathbf{U}}_C\tilde{\mathbf{U}}_C^\top \phi(\mathbf{x})]$.
The approach would result in theoretically more robust projections, but suffers computationally since it requires $(C+1)$ eigendecompositions instead of only one eigendecomposition in Algorithm \ref{alg:cap}. 
Hence, it has $O(C)$ more one-time computation than Algorithm \ref{alg:cap}, but with identical inference time.
The conclusion of Corollary \ref{cor:information} that the most robust features, obtained from the maximisation in \eqref{eq:alternate-max}, are also the most informative features still holds in this case.

\subsection{Dynamics of robust feature learning under GAM}
In this short analysis, we argue that if the trained model is a Generalized Additive Model (GAM), $h(\mathbf{x}) = \bm{\beta}^T \phi(x)$, the test-time defense of Algorithm \ref{alg:cap} could also be replicated through an early stopping of the training process. In other words, we argue that the components of $\bm{\beta}_c^T \phi(x)$ along the robust features---the eigen directions for which $s(\bm{u}) = \lambda (\bm{\beta}_c^\top \bm{u})^2$ are higher---are learned earlier.

For simplicity of analysis, we consider only the learning for $c$-th components, which corresponds to the following regression problem under GAM: Given training sample $\mathcal{D}_{\text{train}} := \{(\mathbf{x_i},y_i)\}_{i=1}^n \subseteq \mathcal{X}\times \mathbb{R}$, minimize the squared loss 
\[\underset{\bm{b}\in\mathbb{R}^p}{\text{minimize}} ~\frac{1}{2n} \sum_{i=1}^n \Vert y_i - \bm{b}^\top \phi(\mathbf{x}_i) \Vert_2^2.\]

\subsection{Proof of Proposition~\ref{prop:gd_top_features}}
\label{pf:dynamics_gam}
\begin{proof}
The optimal solution for $\bm{b}$ for the above problem when population squared loss is minimized is given by $\bm{\beta}_c = (\Phi \Phi^\top)^{-1} \Phi \bm{y}$, where $\Phi = [\phi(\mathbf{x}_1), \ldots, \phi(\mathbf{x}_n)]$ and $\bm{y}=[y_1 \ldots y_n]^\top$.
Furthermore, if the above optimisation is solved using gradient descent with learning rate $\eta >0$ and initialisation $\bm{b}^{(0)} = 0$, the parameters $\bm{b}^{(t)}$ are learned over the iterations as
\begin{align}
    \bm{b}^{(t)} &= \left( I - \dfrac{\eta}{n} \Phi\Phi^\top \right)\bm{b}^{(t-1)} + \dfrac{\eta}{n} \Phi \bm{y} \nonumber \\
    &= \sum_{k=0}^{t-1} \left( I - \dfrac{\eta}{n} \Phi\Phi^\top \right)^k \dfrac{\eta}{n}\Phi \bm{y} \nonumber \\
    &= \sum_{k=0}^{t-1} \left( I - \dfrac{\eta}{n} \Phi\Phi^\top \right)^k \cdot \dfrac{\eta}{n}\Phi\Phi^\top \bm{\beta}_c \label{eq:beta_dyn},
\end{align}
%\[\bm{b}^{(t)} ~=~ \left( I - \eta \Phi\Phi^\top \right)\bm{b}^{(t-1)} + \eta \Phi \bm{y} ~=~ \eta \sum_{k=0}^{t-1} \left( I - \eta \Phi\Phi^\top \right)^k \Phi \bm{y} ~=~ \eta \sum_{k=0}^{t-1} \left( I - \eta \Phi\Phi^\top \right)^k \cdot \Phi\Phi^\top \bm{\beta}_c,\]
with $\bm{b}^{(t)} \to \bm{\beta}_c =(\Phi \Phi^\top)^{-1} \Phi \bm{y}$ as $t\to\infty$. Suppose the eigen decomposition $\Sigma_{\text{train}} = \frac{1}{n}\Phi\Phi^\top$ is given by  $\Sigma_{\text{train}} = {\mathbf{U}} \text{diag}(\bm{\lambda}){\mathbf{U}}^\top = \sum\limits_{i=1}^p \lambda_i \bm{u}_i\bm{u}_i^\top$. Hence, \eqref{eq:beta_dyn} becomes
\begin{align}
    \eqref{eq:beta_dyn} &= \sum_{k=0}^{t-1} \left(\mathbf{U}\mathbf{U}^\top  - \eta \mathbf{U} \text{diag}(\bm{\lambda})\mathbf{U}^\top \right)^k \cdot \eta \mathbf{U} \text{diag}(\bm{\lambda})\mathbf{U}^\top \bm{\beta}_c \nonumber \\
    &= \sum_{k=0}^{t-1} \eta \mathbf{U} (I - \eta \text{diag}(\bm{\lambda}))^k \text{diag}(\bm{\lambda})  \mathbf{U}^\top \bm{\beta}_c \nonumber \\
    \bm{b}^{(t)} &= \sum_{i=1}^p (1-(1-\eta {\lambda_i})^t) \bm{u_i} \bm{u_i}^\top \bm{\beta}_c \label{eq:beta_t_dyn}
\end{align}
From \eqref{eq:beta_t_dyn}, it is clear that $\bm{u}_i$ directions are learnt in the order of $\lambda_i$. That is large $\lambda_i$ learned early during the training since $(1-(1-\eta {\lambda_i})^t)$ is decreasing and at fixed $t$, the eigendirection $\bm{u}_i$ with the largest $\lambda_i$ is learned first. This proves that the direction with maximum variance is learned first.
When the top eigendirection $\bm{u}_i$ aligns with the true signal $\bm{\beta}_c$, $\bm{u}_i$ will be the most robust direction as well. Hence, the top directions based on descending order of $\lambda$ is more robust if the directions align with the true underlying signal.
\end{proof}

\subsection{Connection to Neural Tangent Kernel features}
\label{sec:ntk}
We first briefly discuss NTK and the NTK features before proving the Proposition~\ref{prop:robust_top}.

\paragraph{Neural Tangent Kernels (NTKs) and NTK features.}
\citet{jacotntk, arora2019exact, gregyang2019scaling} show the equivalence of training a large width neural network by gradient descent to a deterministic kernel machine called Neural Tangent Kernel.
In the context of adversarial attacks and robustness, \cite{tsilivis2022can} propose a method to generate adversarial examples using NTK and show transferability of the attack to the finite width neural network counterpart successfully.
Additionally, the authors define NTK features using the eigenspectrum of the NTK gram matrix and observe that the robust features correspond to the top of the eigenspectrum and learned first during training. 
%
%Recently, \citet{tsilivis2022can} defined features using NTK gram matrix and empirically observed that the features corresponding to the top spectrum of NTK are more robust and learned first during training.
In the following, we define the NTK and NTK features and show its equivalence to our robust feature definition along with the proof that the robust NTK features correspond to the top of the spectrum.
The NTK gram matrix $\mathbf{\Theta} \in \mathbb{R}^{n\times n}$ is between all pairs of datapoints and the NTK between $\mathbf{x_i}$ and $\mathbf{x_j}$ for a network that outputs $f(\mathbf{w},\mathbf{x})$ at data point $\mathbf{x} \in \mathbb{R}^d$ parameterized by $\mathbf{w} \in \mathbb{R}^p$ is defined by the gradient of the network with respect to $\mathbf{w}$ as
\begin{equation}
     \mathbf{\Theta}(\mathbf{x_i},\mathbf{x_j}) = \mathbb{E}_{\mathbf{w}\sim \mathcal{N}(0,\mathbf{I}_p)}[\nabla_\mathbf{w} f(\mathbf{w},\mathbf{x_i})^T\nabla_\mathbf{w} f(\mathbf{w},\mathbf{x_j})].
\end{equation}
For an extremely large width network, gradient descent optimization with least square loss is equivalent to kernel regression, the kernel being the NTK. Formally, for a data $\mathbf{x}$, the converged network output in the large width limit is $f(\mathbf{w},\mathbf{x})=\mathbf{\Theta}(\mathbf{x},\mathbf{X})^{T}\mathbf{\Theta}(\mathbf{X},\mathbf{X})^{-1}\mathbf{Y} $.
\citet{tsilivis2022can} define NTK features using the eigendecomposition of $\mathbf{\Theta}(\mathbf{X},\mathbf{X}) = \sum_{i=1}^{n}\lambda_i \mathbf{v_i}\mathbf{v_i}^{T}$ as

\begin{align}
    f(\mathbf{w},\mathbf{x}) &= \mathbf{\Theta}(\mathbf{x},\mathbf{X})^{T}\mathbf{\Theta}(\mathbf{X},\mathbf{X})^{-1}\mathbf{Y}
    = \sum_{i=1}^{n}\lambda_i^{-1}\mathbf{\Theta}(\mathbf{x},\mathbf{X})^{T} \mathbf{v_i}\mathbf{v_i}^{T}\mathbf{Y} := \sum_{i=1}^{n}f^{ker}_{i}(\mathbf{x})
\label{eq:ntk_features}
\end{align}


where $f^{ker}_{i}(\mathbf{x}) \in \mathbb{R}^C$ is the $i$-th NTK feature of $\mathbf{x}$. Note that $f^{ker}_{i}$ is in accordance to our feature definition.
We prove the empirical observation that the top spectrum-induced NTK features $f^{ker}$ are more robust in the following. 
% \begin{prop}[Robustness lies at the top.]
% \label{prop:robust_top}
%     Let feature $f^{ker}_{i}$ be Lipschitz continuous in gradient of NTK with respect to $\mathbf{x}$ and an adversarial perturbation $\bm{\delta}$ for input $\mathbf{x}$ such that $||\bm{\delta}||_p \leq \Delta$. Then,
%     %of max $\Delta$ radius around input $\mathbf{x}$ i.e. $\delta\in \mathcal{B}(\Delta)(\mathbf{x})$ under $\ell_p$-norm. We can say that
%     $||f^{ker}_{i}(\mathbf{x}+\bm\delta)- f^{ker}_{i}(\mathbf{x})||_2\leq\Theta(\frac{1}{\lambda_i})$.
% \end{prop}
%Proposition~\ref{prop:robust_top} shows that the NTK features corresponding to the higher eigenvalues are more robust to adversarial perturbations, and hence robustness lies at the top. 
%This complements and strengthens the empirical observation in \citet{tsilivis2022can}. The proof is provided in appendix.


\subsection{Proof of Proposition \ref{prop:robust_top}}

\begin{proof}
Suppose that the NTK feature $f_i^{ker}$ is $L$-Lipschitz continuous in gradient of NTK with respect to $\mathbf{x}$. Then, we have 
\begin{align}
\begin{split}
  \left\Vert \nabla_\mathbf{x}\mathbf{\Theta}(\mathbf{x}+\bm\delta, \mathbf{X}) - \nabla_\mathbf{x}\mathbf{\Theta}(\mathbf{x}, \mathbf{X}) \Vert_2 \leq L \Vert \bm\delta \right\Vert_2.   
\end{split}
\label{eq:ntk_feature_grad_lip}
\end{align}
Recall that we can write the $i$-th NTK feature as $f^{ker}_{i}(\mathbf{x}) := \lambda_i^{-1}\mathbf{\Theta}(\mathbf{x},\mathbf{X})^{\top} \mathbf{v_i}\mathbf{v_i}^{\top}\mathbf{Y}$. 
Bounding $\Vert f^{ker}_{i}(\mathbf{x}+\bm\delta)-f^{ker}_{i}(\mathbf{x}) \Vert_2$ by Taylor's expansion and applying \eqref{eq:ntk_feature_grad_lip} yield
\begin{flalign}
\Vert f^{ker}_{i}(\mathbf{x}+\bm\delta) - f^{ker}_{i}(\mathbf{x}) \Vert_2 
&\stackrel{(a)}{=}\left\Vert \lambda_i^{-1}\bm\delta^{\top} \nabla_{\mathbf{x}}\mathbf{\Theta}(\mathbf{x}, \mathbf{X})\mathbf{v_i}\mathbf{v_i}^{\top}\mathbf{Y} + \lambda_i^{-1}\mathbf{R} \mathbf{v_i}\mathbf{v_i}^{\top}\mathbf{Y} \right\Vert_2 &
(\text{Where }\mathbf{R} \text{ : remainder}) \nonumber \\
&\stackrel{(b)}{\leq} \left\Vert \lambda_i^{-1}\bm\delta^{\top} \nabla_{\mathbf{x}}\mathbf{\Theta}(\mathbf{x}, \mathbf{X})\mathbf{v_i}\mathbf{v_i}^{\top}\mathbf{Y} + \dfrac{\lambda_i^{-1}L}{2}\Vert \bm\delta \Vert_2 \mathbf{v_i}\mathbf{v_i}^{\top}\mathbf{Y} \right\Vert_2 
&(\text{from \eqref{eq:ntk_feature_grad_lip}}) \nonumber \\
&\leq \lambda_i^{-1} \left\Vert \left(\bm\delta^{\top} \nabla_{\mathbf{x}}\mathbf{\Theta}(\mathbf{x}, \mathbf{X}) + \dfrac{L}{2}\Vert \bm\delta \Vert_2 \right) \mathbf{v_i}\mathbf{v_i}^{\top}\mathbf{Y} \right\Vert_2 \nonumber \\
&= \Theta \left(\dfrac{1}{\lambda_i}\right) \nonumber
\end{flalign}
where $(a)$ follows from the Taylor's expansion of $f^{ker}_{i}(\mathbf{x}+\bm\delta)$ where $\mathbf{R}$ is the remainder terms and $(b)$ follows from \eqref{eq:ntk_feature_grad_lip}, i.e., $\mathbf{R}\leq (L/2) \Vert \bm\delta \Vert_2$.
\end{proof}

\paragraph{Empirical validation: Top NTK features are indeed robust.}
To verify Proposition~\ref{prop:robust_top}, we construct a sanity experiment using a simple 1-layer NN $f(\mathbf{x})=\frac{1}{d}\mathbf{w}^T\mathbf{x}$ with parameters $\mathbf{w}\in \mathbb{R}^{d}$ initialized from $\mathcal{N}(\mathbf{0}, \mathbf{I}_d)$. 
Let the data dimension $d$ be $100$, the number of training samples $n$ be $1000$ and the data is sampled from a Gaussian $\mathcal{N}(\mathbf{0}, \Sigma)$ where the covariance $\Sigma$ is a diagonal spiked matrix, that is, $\Sigma_{11} := 1+\sqrt{d/n}$ and $\Sigma_{ii}:=1 \, \forall i \ne 1$.
%zero mean gaussian. However, in order to obtain an isolated eigenvector with a large eigenvalue we alter the isotropic covariance matrix with $cov_{0,0}=1+\sqrt{\frac{d}{n_{train}}}$. 
We then construct NTK features from the spectral decomposition of the exact NTK. Plot 3 of Figure~\ref{fig:ntk} shows the norm of difference in the original, and adversarially perturbed NTK features with respect to the eigenvalues of the NTK spectrum for different perturbation strengths of $\Delta=\{0.01, 0.05, 0.1\}$. This validates our 
theory that the NTK features corresponding to the large eigenvalues are more robust and hence remain closer to the original feature even when perturbed.

% Figure environment removed

%%%%%%%%%%%%%%%
\section{Experiments}\label{app:exp}

\subsection{Parameters for different algorithms}
We set the parameters to the standard values in the literature. Refer to RobustBench for most of the attack parameters.
\begin{enumerate}
    \item PGD: We perform PGD with the standard parameters in Table~\ref{tab:parameters_pgd} to have an overall high strength PGD attack.% $\ell_{\infty}$ and $\ell_2$ PGD attack with standard $\epsilon=8/255$ and $\epsilon=0.5$, respectively, with the attack step size $\epsilon/4$ and $40$ iterations for $\ell_{\infty}$ and $\epsilon/5$ and $100$ iterations for $\ell_2$ to have an overall high strength PGD attack. 
    \begin{table}[H]
        \centering
        \caption{\textbf{Parameters for PGD.} We use these parameters for both training and attack.}
        \resizebox{0.5\linewidth}{!}{
        \begin{tabular}{lcccc}
        \toprule
            Dataset & $\ell_p$ & $\epsilon$ & step size & iteration \\
            \midrule
            \multirow{2}{*}{CIFAR-10, CIFAR-100}
             & $\ell_\infty$ & $8/255$ & $\epsilon/4$ & $40$ \\
             & $\ell_2$ & $0.5$ & $\epsilon/5$ & $100$ \\
             \multirow{1}{*}{tiny ImageNet}
             & $\ell_\infty$ & $4/255$ & $\epsilon/4$ & $40$ \\
             %& $\ell_2$ & $0.5$ & $\epsilon/5$ & $100$ \\
            \bottomrule
        \end{tabular}}
        \label{tab:parameters_pgd}
    \end{table}
    \item APGD-CE, APGD-DLR: We perform standard $\ell_{\infty}$ perturbation with the budget $\epsilon=8/255$. 
    \item For Adversarial training in Table ~\ref{tab:rfi_calib} we use same parameters for PGD, IAT, CW and TRADES as used in PGD attack from Table~\ref{tab:parameters_pgd}
\end{enumerate}

\subsection{Details of benchmarking baseline methods} \label{app:exp_details}
We perform benchmarking of our test-time defense on multiple SOTA methods that achieves adversarial robustness in the model. For our analysis of RFI on CIFAR-10 in table~\ref{tab:rfi_calib} we used PGD~\cite{madry2018towards}, Interpolated Adversarial Training~\cite{lamb2019interpolated}, Carlini-Warger Loss~\cite{carlini2017towards} and TRADES~\cite{zhang2019theoretically} to adversarially train the baseline model. In the case of Robust CIFAR-10~\cite{featuresnotbugs}, we only replaced the standard CIFAR-10 dataset with the robust dataset. In general for PGD, IAT and C\&W attacks the adversarial training works as generating an adversarial example using the underlying attack and the objective is to minimize loss on these adversarial examples. PGD attack uses gradient descent to iteratively maximize the classification loss with respect to the input while projecting the perturbed example into the norm ball defined for the attack. IAT uses a joint objective that minimizes the classification loss of perturbed examples generated from PGD or any other attack along with classification loss on clean data with MixUP~\cite{mixup}. We use Robust CIFAR-10 proposed in ~\cite{featuresnotbugs}, although is not an adversarial training method but rather the final dataset from a procedure to only retain robust features in the dataset. \citet{featuresnotbugs} disentangle the robust and non-robust features by creating a one-to-one mapping of an input to its robustified image. From an adversarially pretrained backbone (ResNet-18 using PGD $\ell_2-$norm and $\epsilon=0.25$) linear layer features are extracted for the natural image and also from a noise input. Then by minimizing the distance between these two representations in the input space over the noise, an image that only retains robust features of the original input is obtained.

For training using all these baseline adversarial training methods, we set the batch size as $128$. We use SGD with momentum as the optimizer where we set the momentum to $0.1$, we also set the weight decay to $0.0002$. We run our training for $200$ epochs and set the learning rate schedule as starting with $0.1$ for the first $100$ epochs and then we reduce the learning rate by $90$ percent every $50$ epochs. For calibration using temperature scaling~\citep{guo2017calibration}, we take the trained model and optimize for the temperature parameter. The standard deviation in all the cases of calibrated models is reported by loading the pretrained models and 5 runs of calibration. Hence, there is no standard deviation for the non-calibrated models, and we also do not report the standard deviation for the SoTA models directly loaded from RobustBench.

\subsection{Adaptive attack performance of RFI on Expectation Over Transformation (EOT) attack using ResNet-18 for CIFAR-10}

Expectation Over Transformation (EOT) is a procedure to synthesize examples that are adversarial over a chosen distribution of transformations \citep{athalye2018synthesizing}. This procedure is shown to generate adversarial examples that are more robust to noise, distortions and affine transformations, and are consistently adversarial to the neural networks. EOT as an adversarial attack is observed to be stronger \citep{tramer2020adaptive} where a randomized transformation is applied to an input $\mathbf{x}$ before being fed into a classifier. RFI can be easily integrated into the neural network classifier in such settings by computing the transformation matrix $\Tilde{\mathbf{U}}$ in RFI by applying random transformations to the training samples to ensure a similar distribution of the train and test sets. 

We evaluate ResNet-18 using all the training settings considered in Table~\ref{tab:basicexp} on CIFAR-10 for Expectation Over Transformation (EOT) as an adaptive attack. The hyperparameters are the same as considered for adaptive attack evaluation in Section~\ref{ss:rfi_robustness_small_models}.
We evaluate $\ell_\infty$ and $\ell_2$ attacks with $\epsilon=8/255$ budget, $\epsilon/4$ step size and $40$ iterations, and $0.5$ budget, $\epsilon/5$ step size and $100$ iterations, respectively. For RFI, we set $K=10$.
We observe that \emph{RFI improves the performance by $1$ to $2\%$ consistently for EOT attack as well.}

\begin{table*}[ht!]
    \centering
    \caption{\textbf{Adaptive attack performance of RFI on Expectation over Transformation (EOT) attack}. We consider $\ell_\infty$ (step size $\epsilon/4$, $40$ iterations) and $\ell_2$ (step size $\epsilon/5$, $100$ iterations) attack on CIFAR-10 with ResNet-18. %$\ell_\infty$ attack with step size $\epsilon/4$ and $40$ iterations. $\ell_2$ attack with size $\epsilon/5$ and $100$ iterations. 
    RFI improves robustness by {\color{applegreen}{$\mathbf{1}$ to $\mathbf{2\%}$}} as shown in $\%$ Gain column.
    %Results for CIFAR-100 and tiny ImageNet are in Appendix (Tables~\ref{tab:cifar100table} and \ref{tab:tinyImagenet}).
    }
    \vspace{-0.1cm}
    \resizebox{\linewidth}{!}{
    \begin{tabular}{{@{}lccccccccccc@{}}}
    \toprule 
    \multirow{2.5}{*}{Training} & \multicolumn{3}{c}{Clean} &
    \phantom{} & 
    \multicolumn{3}{c}{$\ell_\infty(\epsilon=\frac{8}{255})$} & \phantom{}&\multicolumn{3}{c}{$\ell_2 (\epsilon=0.5)$} \\
    \cmidrule{2-4} \cmidrule{6-8}\cmidrule{10-12}
    %&
        & Method & +RFI & $\%$ Gain  &&  Method & +RFI  & $\%$ Gain && Method & +RFI & $\%$ Gain\\ \midrule 
        PGD &  \textbf{81.08} \tiny{$\pm$ 0.01} & 80.49 \tiny{$\pm$ 0.08} & \color{brightmaroon}\textbf{-0.59} && 36.01 \tiny{$\pm$ 0.01} & \textbf{37.85} \tiny{$\pm$ 0.02}& \color{applegreen}\textbf{+1.84} && 35.52 \tiny{$\pm$ 0.01}  & \textbf{36.65} \tiny{$\pm$ 0.01}& \color{applegreen}\textbf{+1.13}\\
         IAT &  \textbf{90.32} \tiny{$\pm$ 0.01} & 89.89 \tiny{$\pm$ 0.01}& \color{brightmaroon}\textbf{-0.43} && 26.92 \tiny{$\pm$ 0.00} & \textbf{28.30} \tiny{$\pm$ 0.01}& \color{applegreen}\textbf{+1.38} && 30.30 \tiny{$\pm$ 0.00}  & \textbf{31.47} \tiny{$\pm$ 0.02}& \color{applegreen}\textbf{+1.17}\\
         C\&W & \textbf{77.55} \tiny{$\pm$ 0.03} & 77.50 \tiny{$\pm$ 0.02}& \color{brightmaroon}\textbf{-0.05} && 22.51 \tiny{$\pm$ 0.02} & 
         \textbf{23.88} \tiny{$\pm$ 0.03}& \color{applegreen}\textbf{+1.37} && 25.71 \tiny{$\pm$ 0.01}  & \textbf{26.95} \tiny{$\pm$ 0.03}& \color{applegreen}\textbf{+1.24}\\
         TRADES& \textbf{79.17} \tiny{$\pm$ 0.02} & 79.02 \tiny{$\pm$ 0.01}& \color{brightmaroon}\textbf{-0.15} && 47.20 \tiny{$\pm$ 0.01} & \textbf{47.98} \tiny{$\pm$ 0.01}& \color{applegreen}\textbf{+0.78} && 48.55 \tiny{$\pm$ 0.02}  & \textbf{49.61} \tiny{$\pm$ 0.01}& \color{applegreen}\textbf{+1.06}\\
        \bottomrule
    \end{tabular}}
    % \vspace{-0.2cm}
    \label{tab:rfi_eot}
\end{table*}


\subsection{Adaptive attack performance of RFI on calibrated ResNet-18 for CIFAR-10}
We evaluate ResNet-18 using all the training settings considered in Table~\ref{tab:basicexp} on CIFAR-10 for calibrated models. The hyperparameters are the same as non-calibrated setting.
We evaluate $\ell_\infty$ and $\ell_2$ attacks with $\epsilon=8/255$ budget, $\epsilon/4$ step size and $40$ iterations, and $0.5$ budget, $\epsilon/5$ step size and $100$ iterations, respectively. For RFI, we set $K=10$.
We observe that \emph{RFI improves the performance by $4$ to $9\%$ for calibrated models.}
\begin{table*}[ht!]
    \centering
    \caption{\textbf{Adaptive attack performance of RFI on calibrated models} using temperature scaling. We consider $\ell_\infty$ (step size $\epsilon/4$, $40$ iterations) and $\ell_2$ (step size $\epsilon/5$, $100$ iterations) attack on CIFAR-10 with ResNet-18. %$\ell_\infty$ attack with step size $\epsilon/4$ and $40$ iterations. $\ell_2$ attack with size $\epsilon/5$ and $100$ iterations. 
    RFI improves robustness by {\color{applegreen}{$\mathbf{4}$ to $\mathbf{9\%}$}} as shown in $\%$ Gain column.
    %Results for CIFAR-100 and tiny ImageNet are in Appendix (Tables~\ref{tab:cifar100table} and \ref{tab:tinyImagenet}).
    }
    \vspace{-0.1cm}
    \resizebox{\linewidth}{!}{
    \begin{tabular}{{@{}lccccccccccc@{}}}
    \toprule 
    \multirow{2.5}{*}{Training} & \multicolumn{3}{c}{Clean} &
    \phantom{} & 
    \multicolumn{3}{c}{$\ell_\infty(\epsilon=\frac{8}{255})$} & \phantom{}&\multicolumn{3}{c}{$\ell_2 (\epsilon=0.5)$} \\
    \cmidrule{2-4} \cmidrule{6-8}\cmidrule{10-12}
    %&
        & Method & +RFI & $\%$ Gain  &&  Method & +RFI  & $\%$ Gain && Method & +RFI & $\%$ Gain\\ \midrule 
        Standard & \textbf{95.20} \tiny{$\pm$ 0.08}  & 88.20 \tiny{$\pm$ 0.10} & \color{brightmaroon}\textbf{-7.00} && 2.01 \tiny{$\pm$ 0.38} & \textbf{6.83} \tiny{$\pm$ 0.22} & \color{applegreen}\textbf{+4.82} && 2.58 \tiny{$\pm$ 0.62}  & \textbf{10.21} \tiny{$\pm$ 0.81} & \color{applegreen}\textbf{+7.63}\\
        Robust CIFAR-10 & {78.70} \tiny{$\pm$ 0.04} & \textbf{78.73} \tiny{$\pm$ 0.06} & \color{applegreen}\textbf{+0.03} && 3.81 \tiny{$\pm$ 0.14} & \textbf{8.03} \tiny{$\pm$ 0.21} & \color{applegreen}\textbf{+4.22} && 9.10 \tiny{$\pm$ 0.92}  & \textbf{11.21} \tiny{$\pm$ 0.68} & \color{applegreen}\textbf{+2.11}\\
         PGD &  \textbf{83.11} \tiny{$\pm$ 0.02} & 82.32 \tiny{$\pm$ 0.08} & \color{brightmaroon}\textbf{-0.79} && 42.96 \tiny{$\pm$ 0.75} & \textbf{50.08} \tiny{$\pm$ 0.88}& \color{applegreen}\textbf{+7.12} && 56.48 \tiny{$\pm$ 0.42}  & \textbf{62.13} \tiny{$\pm$ 0.92}& \color{applegreen}\textbf{+5.65}\\
         IAT &  \textbf{91.24} \tiny{$\pm$ 0.10} & 90.83 \tiny{$\pm$ 0.08}& \color{brightmaroon}\textbf{-0.41} && 46.22 \tiny{$\pm$ 0.10} & \textbf{51.34} \tiny{$\pm$ 0.83}& \color{applegreen}\textbf{+5.12} && 63.48 \tiny{$\pm$ 0.96}  & \textbf{71.12} \tiny{$\pm$ 0.29}& \color{applegreen}\textbf{+7.64}\\
         C\&W & \textbf{84.36} \tiny{$\pm$ 0.10} & 83.32 \tiny{$\pm$ 0.05}& \color{brightmaroon}\textbf{-1.03} && 41.62 \tiny{$\pm$ 0.90} & 
         \textbf{50.48} \tiny{$\pm$ 1.07}& \color{applegreen}\textbf{+8.86} && 56.63 \tiny{$\pm$ 0.68}  & \textbf{63.21} \tiny{$\pm$ 0.72}& \color{applegreen}\textbf{+6.58}\\
         TRADES& \textbf{81.11} \tiny{$\pm$ 0.01} & 79.38 \tiny{$\pm$ 0.04}& \color{brightmaroon}\textbf{-1.73} && 53.67 \tiny{$\pm$ 0.43} & \textbf{58.20} \tiny{$\pm$ 0.61}& \color{applegreen}\textbf{+4.53} && 62.12 \tiny{$\pm$ 0.28}  & \textbf{68.47} \tiny{$\pm$ 0.32}& \color{applegreen}\textbf{+6.35}\\
        \bottomrule
    \end{tabular}}
    % \vspace{-0.2cm}
    \label{tab:rfi_calib}
\end{table*}

\subsection{Adaptive attack performance of RFI for CIFAR-100 and tiny ImageNet}
We evaluate both calibrated and non-calibrated ResNet-18 using all the adversarial training setting considered in Table~\ref{tab:rfi_calib} on CIFAR-100 since standard training would not result in robust model. We also consider tiny ImageNet dataset that has $100,000$ training and $10,000$ validation samples with $200$ classes and ResNet-50 pretrained adversarially on ImageNet. 
We evaluate $\ell_\infty$ attack with $\epsilon=8/255$ and $\epsilon=4/255$ for CIFAR-100 and tiny ImageNet, respectively. The attack budget is standard, taken from RobustBench.
For RFI, we set $K=100$ and $200$ (number of classes) for CIFAR-100 (Table~\ref{tab:cifar100table}) and tiny ImageNet (Table~\ref{tab:tinyImagenet}), respectively.
$\%$ Gain in tables is between Calibration+RFI and the base method.

\begin{table}[h]
    \centering
    \caption{\textbf{Adaptive attack performance of RFI on non-calibrated and calibrated models.} Robust performance evaluation of RFI on CIFAR-100 with ResNet-18 (step size $\epsilon/4$ and $40$ iterations). RFI improves the performace on an average by \color{applegreen}$\mathbf{4\%}$.}
    \vspace{-0.1cm}
    \resizebox{\linewidth}{!}{
    \begin{tabular}{{@{}lcccccccccccc@{}}}
    \toprule 
    \multirow{2.5}{*}{Training} & \multicolumn{5}{c}{Clean} &
    \phantom{} & 
    \multicolumn{5}{c}{$\ell_\infty (\epsilon=\frac{8}{255})$}\\
    \cmidrule{2-6} \cmidrule{8-12}
    %&
        & Method & +RFI & +Calibration &+Calibration+RFI & $\%$ Gain &&  Method & +RFI  & +Calibration &+Calibration+RFI& $\%$ Gain\\ \midrule 
         PGD &  {55.30} & 55.27 & \textbf{55.82} & 55.08& \color{brightmaroon}\textbf{-0.22} &&20.08 &{20.91} & 21.86 & \textbf{25.96}& \color{applegreen}\textbf{+5.88}\\
         IAT & \textbf{58.94} & 58.88  & 58.86 & 58.09 & \color{brightmaroon}\textbf{-0.85}&& 22.56 & {23.58} & 23.04 & \textbf{26.72}& \color{applegreen}\textbf{+4.16}\\
         C\&W & \textbf{49.36} & 49.31  & 49.30 & 49.02 & \color{brightmaroon}\textbf{-0.34}&&     
         10.44 & {11.86}  & 11.28 & \textbf{14.72}&\color{applegreen}\textbf{+4.28}\\
         TRADES&\textbf{55.17} &55.11  & \textbf{55.17} & 55.10 & \color{brightmaroon}\textbf{-0.07}&& 28.25 &{28.56}  & 28.43 & \textbf{30.91}&\color{applegreen}\textbf{+2.66}\\
        \bottomrule
    \end{tabular}}
    % \vspace{-0.2cm}
    \label{tab:cifar100table}
\end{table}
In the case of tiny ImageNet, we subsampled $100$ training samples per class instead of using the full training set for computing the transformation matrix $\tilde{\bm{U}}$ of the feature covariance due to the computation time, and evaluated the clean and robust performances on the $10,000$ validation samples. The results are given in Tables~\ref{tab:cifar100table} and \ref{tab:tinyImagenet}.
We observe that \emph{RFI consistenly improves the adversarial performance on the datasets with a very small drop in the clean performance.} Thus this shows RFI generalizes to larger datasets as well. Furthermore, we would like to draw the attention that \emph{our method improves the performance even with a small subsample of the dataset.}
\begin{table}[h]
    \centering
     \caption{\textbf{Adaptive attack performance of RFI on non-calibrated and calibrated models.} Robust performance evaluation of RFI on tiny ImageNet with ResNet-50 (step size $\epsilon/4$ and $40$ iterations). RFI improves robustness even on large datasets.}
     \vspace{-0.1cm}
    \resizebox{\linewidth}{!}{
    \begin{tabular}{{@{}lcccccccccccc@{}}}
    \toprule 
    \multirow{2.5}{*}{Training} & \multicolumn{5}{c}{Clean} &
    \phantom{} & 
    \multicolumn{5}{c}{$\ell_\infty (\epsilon=\frac{4}{255})$}\\
    %&
    \cmidrule{2-6} \cmidrule{8-12}
        & Method & +RFI  & +Calibration &+Calibration+RFI & $\%$ Gain &&  Method & +RFI  & +Calibration &+Calibration+RFI& $\%$ Gain\\ \midrule 
         PGD &  \textbf{62.42} & 62.39  & 62.40 & 62.32 &\color{brightmaroon}\textbf{-0.10}&&33.38 &\textbf{33.50}  & 33.43 & 34.27&\color{applegreen}\textbf{+0.89}\\
         % IAT & \textbf{58.94} & 58.88 && 22.56 & \textbf{23.58}\\
         % C\&W attack & \textbf{49.36} & 49.31 &&     
         % 10.44 & \textbf{11.86} \\
         % TRADES&\textbf{55.17} &55.11 && 28.25 &\textbf{28.56} \\
        \bottomrule
    \end{tabular}}
    % \vspace{-0.2cm}
    \label{tab:tinyImagenet}
\end{table}




\subsection{Adaptive attack performance of RFI on state-of-the-art models from RobustBench}

\begin{table*}[t]
    \centering
    \caption{\textbf{Adaptive attack performance evaluation of RFI on state-of-the-art methods.} We apply APGD-CE, APGD-DLR and RobustBench attacks on CIFAR-10 and CIFAR-100. The inference time for RFI is $1\times$, whereas Anti-adv and SODEF are $8\times$ and $2\times$, respectively. There is no standard deviation as the trained models are directly from RobustBench. While RFI improves the robustness to AutoAttack {\color{applegreen}\textbf{upto}} $\color{applegreen}\mathbf{1.5\%}$ without calibration, SODEF and Anti-adv results in {\color{warningyellow}\textbf{no ($\mathbf{<0.1\%}$)}} or {\color{brightmaroon}\textbf{decrease}} in robustness consistently.}
    \resizebox{\linewidth}{!}{
    \begin{tabular}{@{}lclcccccc@{}}
         \toprule 
         %Dataset & 
         & Base Method & Defense & Clean & APGD-CE  & APGD-DLR  &FAB & Square & AutoAttack\\
         \midrule
         \multirow{15}{*}{\begin{sideways}CIFAR-10\end{sideways}}&
          %\midrule
          \multirowcell{4}{~\citet{carmon2019unlabeled}\\WideResNet-28-10} & None & \textbf{89.69} &  61.82 & 60.85 & 60.18 & 66.51 & 59.53\\
          &
          &Anti-adv & \textbf{89.69} & 61.81 & 60.89 & 60.11 & 66.58 & \color{brightmaroon}\textbf{58.70}\\
          &
          &SODEF & 89.68 & 60.20 & 60.72 & 58.04 & 65.28 & \color{brightmaroon}\textbf{57.23}\\
          &
          &RFI ($K=10$) & 89.60 & 62.38 & 61.58 & 60.21 & 66.59 & 60.72\\
          &
          &RFI (opt. $K=20$) & 89.60 & \textbf{62.45} & \textbf{61.60}& \textbf{60.38}& \textbf{66.90} & \color{applegreen}\textbf{61.02}\\
         \cmidrule{2-9}
         &\multirowcell{4}{~\citet{engstrom2019adversarial}\\ResNet-50} & None & \textbf{87.03} & 51.75 & 60.10 & 49.90 & 58.00  & 49.25\\
          &
          &Anti-adv & 87.00 & 51.62 & 59.95& 49.84 & 58.06 &\color{warningyellow}\textbf{49.20}\\
          &
          &SODEF & 86.95 & 50.01 & 58.20& 48.64 & 56.68 &\color{brightmaroon}\textbf{47.92}\\
          &
          &RFI ($K=10$)&87.01 & 51.86 & 61.84 & 51.28 & 58.07 & 50.75\\
          &
          & RFI (opt. $K=15$) & \textbf{87.03} & \textbf{51.94} & \textbf{61.90} & \textbf{51.46} & \textbf{58.12} & \color{applegreen}\textbf{50.98}\\
         \cmidrule{2-9}
         &
         \multirowcell{4}{~\citet{rice2020overfitting}\\WideResNet-34-10}  & None & 85.34 & 50.12 & 56.80 & 53.87 & 56.88 & 53.42\\
          &
          &Anti-adv  & \textbf{85.40} & 50.10 &  57.50 & 53.90 & 57.00 & \color{brightmaroon}\textbf{50.98}\\
          &
          &SODEF& 85.10 & 50.60 & 56.50& 53.72 & 56.21 & \color{brightmaroon}\textbf{50.09}\\
          &
          &RFI($K=10$) & 85.30 & 51.19 & 58.55 & 53.98 & \textbf{57.13} & 54.64\\
          &
          &RFI (opt. $K=35$) & 85.30 & \textbf{51.62} &  \textbf{58.97} & \textbf{54.12}& \textbf{57.13} & \color{applegreen}\textbf{54.86} \\
          \cmidrule{2-9}
          %\midrule
         &
         \multirowcell{4}{~\citet{wang2023better}\\WideResNet-28-10}  & None & \textbf{92.44} & 70.23 &  67.82 & 67.41 & 73.13 & 67.31\\
          &
          &Anti-adv  & \textbf{92.44} & 68.90 &  65.91 & 67.55 & 73.20 & \color{brightmaroon}\textbf{66.52}\\
          &
          &SODEF & 92.01 & 67.53 & 65.08 & 65.93 & 73.01 & \color{brightmaroon}\textbf{64.20}\\
          &
          &RFI ($K=10$) & 92.33 & 70.32 & 67.86 & \textbf{67.82} & 73.52 & 67.29\\
          &
          & RFI (opt. $K=20$) & 92.34 & \textbf{70.36} & \textbf{67.90} & \textbf{67.82} & \textbf{73.54} & \color{applegreen}\textbf{67.50}\\
         %  \cmidrule{2-9}
         %  &
         % \multirowcell{4}{~\citet{debenedetti2023light}\\XCiT-S12}  & None & \textbf{90.06} & - &  -& - & - & 56.14\\
         %  &
         %  &Anti-adv  & \textbf{-} & - &  - & - & - & \color{brightmaroon}\textbf{-}\\
         %  &
         %  &SODEF & - & - & - & - & - & \color{brightmaroon}\textbf{-}\\
         %  &
         %  &RFI ($K=10$) & - & - & - & - & - & \textbf{-}\\
         %  &
         %  & RFI (opt. $K=20$) &- & \textbf{-} & \textbf{-} & \textbf{-} & \textbf{-} & \color{applegreen}\textbf{-}\\
          \midrule
         %\hline 
         %\textbf{CIFAR-100} \\
         \multirow{15}{*}{\begin{sideways}CIFAR-100\end{sideways}}&
         \multirowcell{4}{~\citet{pang2022robustness}\\WideResNet-28-10} & None & \textbf{63.66} & 35.29 & 31.71& 31.32 & 35.70 & 31.08\\
          &
          &Anti-adv & 63.41 & 32.50 & 30.32 & 31.30 & 35.76 & \color{brightmaroon}\textbf{30.10}\\
          &
          &SODEF & 63.08 & 30.96 & 29.54 & 31.44 & 32.27 &\color{brightmaroon}\textbf{30.56}\\
          &
          &RFI ($K=100$)& 63.01 & 36.03 & \textbf{31.95}& 31.88 & 35.79 & 31.29\\
          &
          & RFI (opt. $K=115$) & 63.10 & \textbf{36.07} & \textbf{31.95} & \textbf{31.96} & \textbf{35.88} & \color{applegreen}\textbf{31.91}\\
          \cmidrule{2-9}
          &
         \multirowcell{4}{~\citet{addepalli2022efficient}\\ResNet-18} & 
         None & \textbf{65.45} & 33.49 & 28.55  & 28.00 & 33.70 & 27.67\\
          &
          &Anti-adv & 65.38 & 30.92 & 26.61 & 27.92 & 33.61 & \color{brightmaroon}\textbf{26.01}\\
          &
          &SODEF & 65.23 & 29.37 & 26.90 & 24.62 & 29.60 & \color{brightmaroon}\textbf{26.53}\\
          &
          &RFI $(K=\text{opt. } K = 100)$& 65.41 & \textbf{34.09} & \textbf{29.18} & \textbf{28.10} & \textbf{33.79} & \color{applegreen}\textbf{27.80}\\
          %&
          %&RFI (opt. $K=100$) & 65.41 & \textbf{34.09} & \textbf{29.10} & \textbf{27.80}\\
          \cmidrule{2-9}
         %\midrule
         &
         \multirowcell{4}{~\citet{rice2020overfitting}\\PreActResNet-18}  & None & \textbf{53.83} & 20.83 & 20.46 & \textbf{23.82} & 19.29 & 18.95\\
          &
          &Anti-adv  & \textbf{53.83} & 20.78 &  20.06 & 23.49 & 19.27 & \color{warningyellow}\textbf{18.97}\\
          &
          &SODEF & \textbf{53.83} & 18.50 & 19.20 & 19.66 & 16.05 & \color{brightmaroon}\textbf{16.92}\\
          &
          &RFI ($K=100$)& 53.70 & 21.10 & 20.98 & 20.93 & 18.13 & 19.23\\
          &
          &RFI (opt. $K=150$) & 53.75 & \textbf{21.18} & \textbf{21.10} & 21.03 & \textbf{19.53} &\color{applegreen}\textbf{19.46}\\
           \cmidrule{2-9}
          %\midrule
         &
         \multirowcell{4}{~\citet{wang2023better}\\WideResNet-28-10}  & None & \textbf{72.58} & 44.04 &  39.78 & 39.19 & 44.46 & 38.83\\
          &
          &Anti-adv  & 72.57 & 42.98 &  38.10 & 36.85 & 44.49 & \color{brightmaroon}\textbf{34.01}\\
          &
          &SODEF & 72.34 & 38.10 & 36.95 & 34.82 & 44.42 & \color{brightmaroon}\textbf{32.29}\\
          &
          &RFI ($K=100$)& 72.55 & 44.37 & 39.91 & 39.68 & 44.50 & 39.10\\
          &
          & RFI (opt. $K=115$) & 72.55 & \textbf{44.51} & \textbf{39.96} & \textbf{39.81} & \textbf{44.53} &\color{applegreen}\textbf{39.13}\\
          \midrule
         %\hline 
         %\textbf{CIFAR-100} \\
         \multirow{4}{*}{\begin{sideways}ImageNet\end{sideways}}&
         \multirowcell{2}{~\citet{salman2020adversarially}\\ResNet-50} & 
         None & \textbf{64.02} & 38.32 & 34.02  & 34.35 & 49.52 & -\\
          &
          & Dynamic RFI & 63.91 & \textbf{38.48} & \textbf{34.68} & \textbf{34.68}& \textbf{49.98} & -\\
          \cmidrule{2-9}
          &\multirowcell{2}{~\citet{salman2020adversarially}\\WideResNet-50-2} & 
         None & \textbf{68.46} & 40.67 & 37.09  & 37.81 & 54.61 & -\\
          &
          & Dynamic RFI & 68.41 & \textbf{40.84} & \textbf{37.56} & \textbf{38.12}& \textbf{54.78} & -\\
         \bottomrule
    \end{tabular}
    }
    %\caption{(Base/Ours) Indicate the performance of adversarial training methods after adding our adaptive test time defense on APGD-CE and APGD-DLR attacks ~\cite{croce2020reliable}}
    % \vspace{-0.3cm}
    \label{tab:benchmarkcomparison_app}
\end{table*}

For table~\ref{tab:benchmarkcomparison} we benchmark our test-time defense on multiple recent SoTA methods for CIFAR-10, CIFAR-100 and ImageNet. For all our baseline methods we obtain the model weights from RobustBench~\cite{croce2020reliable}. We update the weights of the last linear layer of the models using RFI and benchmark the updated models. 
We also report the performance for optimal $K$ in RFI.
We note that the Expected Calibration Error (ECE) for the SoTA models are very small as shown in Table~\ref{tab:mce_sota} (already well calibrated), hence we do not explicitly calibrate in Table~\ref{tab:benchmarkcomparison_app}.
Moreover, the results in Table~\ref{tab:benchmarkcomparison} show that calibration will only further improve robustness with RFI. Therefore, we do conservative analysis of RFI on the SoTA models.
For \citet{salman2020adversarially} on ImageNet we compute with and without dynamic RFI and not Anti-Adv and SODEF since it increase the inference costs of the evaluation such that we could no longer run experiments with our computational resources. Also we do not report AutoAttack since it requires all 4 attacks i.e. APGD-CE, APGD-DLR, FAB and Square to be executed sequentially which is outside the scope of max runtime of our resources.
Nevertheless, we observe that \emph{RFI improves the robustness reliably $\sim 1.5\%$ on average on non-calibrated SoTA models.} Importantly, SODEF and Anti-adv reduces the robustness performance especially on AutoAttack which is inline to the findings of \citet{croce2020reliable}.
%We provide our implementation of benchmarking experiments baselines used in Table~\ref{tab:benchmarkcomparison_app}.\citet{carmon2019unlabeled} show that additional $500,000$ unlabeled data improve robustness on CIFAR-10. \citet{engstrom2019adversarial} argue that the limitation of a network to represent input data in high-level features causes two semantically different images to have similar representation and propose training with an adversarial prior improves robustness. \citet{rice2020overfitting} suggest early stopping-based training to overcome this robust overfitting. \citet{wang2023better} propose use of class-conditioned diffusion models to generate additional labeled data for adversarial training and observe state-of-the-art performance using additional $50$ million synthetic images in training.
%\citet{pang2022robustness} ..
\begin{table}[h]
    \centering
    \caption{\textbf{Expected Calibration Error (ECE) of the SoTA models} are very small, hence already well calibrated.}
    \vspace{-0.1cm}
    \resizebox{0.85\linewidth}{!}{
    \begin{tabular}{{@{}ccccccc@{}}}
    \toprule 
     \multicolumn{3}{c}{CIFAR-10} &
    \phantom{} & 
    \multicolumn{3}{c}{CIFAR-100} \\
    \cmidrule{1-3} \cmidrule{5-7}
    %&
       Method & ECE & ECE after Calibration && Method & ECE & ECE after Calibration \\ \midrule 
        \multirowcell{2}{\citet{carmon2019unlabeled}\\WideResNet-28-10} & 4.310     &    \textbf{0.328} && \multirowcell{2}{\citet{pang2022robustness}\\WideResNet-28-10} & 0.364  & \textbf{0.142}  \\ \\
        \multirowcell{2}{\citet{engstrom2019adversarial}\\ResNet-50} &  0.091   & \textbf{0.065}  && \multirowcell{2}{\citet{addepalli2022efficient}\\ResNet-18} & 0.418   & \textbf{0.347}  \\\\
        \multirowcell{2}{\citet{rice2020overfitting}\\WideResNet-34-10} & 0.074     &  \textbf{0.037}  &&  \multirowcell{2}{\citet{rice2020overfitting}\\PreActResNet-18} & 0.138   & \textbf{0.074} \\\\
        \multirowcell{2}{\citet{wang2023better}\\WideResNet-28-10} & 0.145   & \textbf{0.039}  && \multirowcell{2}{\citet{wang2023better}\\WideResNet-28-10} & 0.366    & \textbf{0.290} \\\\
        \bottomrule
    \end{tabular}}
    % \vspace{-0.2cm}
    % \vspace{-0.6cm}
    \label{tab:mce_sota}
\end{table}

\subsection{Transferability Study}


We conduct a more detailed transferability of attack analysis on CIFAR-10 using ResNet-18 and on CIFAR-100 using PreActResNet-18. Here, we generated adversarial examples with respect to the base model and all the defences and evaluated the robustness of different adaptive defences under all the adversary cases (Transfer attacks). 
Then we present the results for calibrated ResNet-18 on CIFAR-10 in Table~\ref{tab:rfi_calib_transferability} which completes the analysis together with the results from Table~\ref{tab:transfercifar10}. 
We observe that the robustness of the calibrated model with RFI is on par with the base calibrated model. Moreover, when attacked with examples from RFI integrated model, the base model performs worse. 
Notably, \emph{the decrease in robust performance of the base method is much more than the decrease of the performance of RFI when evaluated on adversary from the base method.}
This shows RFI's goodness and further confirms the absence of an obfuscated gradient in RFI. Similar observation using a SoTA model on CIFAR-100 are in Table \ref{tab:transfer_attack_def_comparison} (Appendix).
In contrast, transfer attacks from base model on SODEF show a significant drop in robustness (Section 4.6.2 of \citet{kang2021stable}) and on Anti-adv render the defense ineffective (Section 3.8 of \citet{croce2022evaluating}). 
These results further highlight the soundness of RFI.
%Further, we study the adversary strength from RFI by accessing the base method against adversarial samples from RFI, showing \emph{RFI is a stronger adversary in both non-calibrated models} in Appendix~\ref{app:rfi_stronger_adv}.
\subsubsection{RFI results in stronger adversary against transfer from other defenses}
\label{app:rfi_stronger_adv}
In the set of experiments, we evaluate all combinations of transfer attacks on CIFAR-100 and PreActResNet-18~\cite{rice2020overfitting} in Table~\ref{tab:transfer_attack_def_comparison}. We compare the transferability of all adaptive test-time defenses to base model and within themselves by using adversarial examples generated with one defense attacking another defense. The general observation and expectation is that the model performance is affected the most when the adversarial examples are created using the same model, i.e., adaptive attack. This observation holds in our experiments too. The most interesting and impressive observation is that \emph{RFI outperforms all other methods in almost all the cases, even when adversarial examples are generated from base model + RFI}. Notice that SODEF and Anti-adv suffer the most when adversarial examples are generated from the respective models, unlike RFI showing the impressive robustness of our method.

\begin{table*}[h]
\caption{\textbf{Transfer attack on non-calibrated PreActResNet-18 for CIFAR-100.} RFI outperforms in all the cases and also generates the strongest adversary for the base model. }
\parbox{.5\linewidth}{
    \centering
    \resizebox{0.9\linewidth}{!}{
    \begin{tabular}{lcccc}
    \multicolumn{5}{c}{Adversarial Examples are generated from \textbf{Method (Rice et al)}}\\
    \toprule
    Attack & Method  & +AntiAdv &+SODEF &+RFI\\
    \midrule 
    APGD-CE & 20.83  &20.06  & 27.13  & \textbf{27.30}\\
    APGD-DLR & 20.46 & 20.52 & 29.33  & \textbf{29.53}\\
    FAB & 19.29 & 19.28  & 35.38  & \textbf{35.90}\\
    Square &  23.82 & 23.58 & 36.83 & \textbf{36.88}\\   
    AutoAttack & 18.95 & 18.97 & 26.09 & \textbf{26.43}\\
    \bottomrule
    \end{tabular}}
    
}
\hfill
\parbox{.5\linewidth}{
    \centering
    \resizebox{0.9\linewidth}{!}{
    \begin{tabular}{lcccc}
    \multicolumn{5}{c}{Adversarial Examples are generated from \textbf{Method+SODEF}}\\
    \toprule
    Attack & Method  & +AntiAdv &+SODEF &+RFI\\
    \midrule 
    APGD-CE &  32.99&\textbf{37.32}& 18.50 & 37.30\\
    APGD-DLR & 33.65 &\textbf{38.34}& 19.20 & 38.30\\
    FAB & 39.67  &48.11 &16.05 & \textbf{48.12}\\
    Square & 39.59  &48.20  & 19.66 & \textbf{48.22}  \\    
    AutoAttack & 32.76  & 33.16 & 15.69 & \textbf{37.23}\\
    \bottomrule
    \end{tabular}}
}
\parbox{.5\linewidth}{
    \centering
    \resizebox{0.9\linewidth}{!}{
    \begin{tabular}{lcccc}
    \multicolumn{5}{c}{Adversarial Examples are generated from \textbf{Method+AntiAdv}}\\
    \toprule
    Attack & Method  & +AntiAdv &+SODEF &+RFI\\
    \midrule 
    APGD-CE & 20.59 &20.58 & \textbf{27.31} & 26.65\\
    APGD-DLR & 20.39 & 20.49 & \textbf{28.92} & 28.53\\
    FAB & 19.27 & 19.27 & 35.80 & \textbf{38.69}\\
    Square & 23.60 & 23.49 & 37.41 & \textbf{39.04}\\      
    AutoAttack & 18.98&  18.96 & 25.61 & \textbf{26.15} \\
    \bottomrule
    \end{tabular}}
}
\hfill
\parbox{.5\linewidth}{
    \centering
    \resizebox{0.9\linewidth}{!}{
    \begin{tabular}{lcccc}
    \multicolumn{5}{c}{Adversarial Examples are generated from \textbf{Method+RFI}}\\
    \toprule
    Attack & Method  & +AntiAdv &+SODEF &+RFI\\
    \midrule 
    APGD-CE & 14.70 & 18.31 & 18.40 & \textbf{21.18}\\
    APGD-DLR & 14.12 & 18.30 & 19.21 & \textbf{21.10}\\
    FAB & 12.76 & 14.12 &  14.70 & \textbf{18.13}\\
    Square &  16.29 & 18.95 & 19.50 & \textbf{20.93}\\       
    AutoAttack & 12.55 & 16.50 & 16.92 & \textbf{19.46}\\
    \bottomrule
    \end{tabular}}
}
\label{tab:transfer_attack_def_comparison}
\end{table*}
\subsection{Transfer attack: RFI with calibration is on par with the base model} 
Results on transfer attacks, where we assess the performance of RFI against adversarial samples generated from the base, for calibrated and on CIFAR-10 with Resnet 18 backbone are in Tables~\ref{tab:rfi_calib_transferability}. 
Notably, \emph{RFI demonstrates comparable robustness to the base model}, ensuring that gradient obfuscation is \emph{not} at play in RFI and affirming that it reliably improves the model robustness. 
Moreover, the transferability of adversary from RFI leads to a degradation in robustness for the base model, suggesting that \emph{RFI acts as an on-par adversary to the base }(refer to +RFI rows of the left subtable in the Table ~\ref{tab:rfi_calib_transferability}). We hypothesize that the attack from base model and attack from base model + RFI affect different semantics or examples such that on average both are on par post-calibration. As expected the adversarial samples from the base method + RFI are more powerful and reduce the robustness of the base method to a greater extent than vice versa.
% (Table~\ref{tab:basicexp_transferability_1}). Comparing `Method' columns in Tables~\ref{tab:basic_cifar10} and \ref{tab:basicexp_transferability_1}, \emph{we note that while the loss in robust performance is comparatively smaller for TRADES which is in line with observation in \citet{croce2022evaluating},  C\&W loses its robustness completely suggesting the brittleness of C\&W defense.} On the other hand, the adversarial samples from the base method are much weaker and RFI is very robust, achieving the best performance (Table~\ref{tab:basicexp_transferability_2}). The improvement in robustness compared to adaptive attack is between $2-5\%$ for the different training defenses. 


\begin{comment}
\begin{table}[h]
\caption{\textbf{Transfer attack performance of RFI on calibrated models.} Setting same as Table~\ref{tab:rfi_calib}.}
    \vspace{-0.1cm}
\parbox{.49\linewidth}{
    \centering
    \resizebox{\linewidth}{!}{
    \begin{tabular}{|c|cc|}
    \multicolumn{3}{c}{Base method training: $\ell_\infty$ with $\epsilon=8/255$}\\
    \hline
    \backslashbox{Adversary \\ from Base }{Evaluation}
    &\makebox{\quad Base \quad}&\makebox{+RFI}\\\hline
    PGD & 42.96 \tiny{$\pm$ 0.75} & 41.08 \tiny{$\pm$ 0.00} \\
    IAT & 46.22 \tiny{$\pm$ 0.10} & 43.10 \tiny{$\pm$ 0.00}\\
    C\&W &41.62 \tiny{$\pm$ 0.90} &39.21 \tiny{$\pm$ 0.00}\\
    TRADES & 53.67 \tiny{$\pm$ 0.43}&52.01 \tiny{$\pm$ 0.00}\\\hline
    \end{tabular}}
}
    \hfill
\parbox{.5\linewidth}{
    \centering
    \resizebox{0.99\linewidth}{!}{
    \begin{tabular}{|c|cc|}
    \multicolumn{3}{c}{Base method training: $\ell_2$ with $\epsilon=0.5$}\\
    \hline
    \backslashbox{Adversary \\ from Base }{Evaluation}
    &\makebox{\quad Base \quad}&\makebox{+RFI}\\\hline
    PGD & 56.48 \tiny{$\pm$ 0.42} & 54.31 \tiny{$\pm$ 0.00} \\
    IAT & 63.48 \tiny{$\pm$ 0.96} & 62.01 \tiny{$\pm$ 0.00}\\
    C\&W & 56.63 \tiny{$\pm$ 0.68} & 54.90 \tiny{$\pm$ 0.00}\\
    TRADES & 62.12 \tiny{$\pm$ 0.23}& 59.91 \tiny{$\pm$ 0.00}\\\hline
    \end{tabular}}
}
\vspace{-0.2cm}
    \label{tab:rfi_calib_transferability_1}
\end{table}
\end{comment}
\begin{table}[h]
    \centering
    \caption{\textbf{Transfer attack performance of RFI on calibrated models.} RFI is on par with the base, ensuring reliable robustness improvement without gradient obfuscation. Setting same as Table~\ref{tab:rfi_calib}. The decrease in robustness of the base model is much more than the robustness of RFI when evaluated on the adversary from the base.}
    \parbox{0.65\linewidth}{
    \centering
    \begin{tabular}{{@{}lccccc@{}}}
    \multicolumn{6}{c}{Adversary generated from \textbf{base model+RFI} } \\
    \toprule 
    \multirow{2.5}{*}{Training} & \multicolumn{2}{c}{$\ell_\infty (\epsilon = \frac{8}{255})$} &
    \phantom{} & 
    \multicolumn{2}{c}{$\ell_2 (\epsilon = 0.5)$} \\
    \cmidrule{2-3} \cmidrule{5-6}
    %&
        & Method & +RFI &&  Method & +RFI \\ \midrule 
        PGD & 42.85 \tiny{$\pm$ 0.12}     &    \textbf{50.08} \tiny{$\pm$ 0.88} && 57.18 \tiny{$\pm$ 0.54}   & \textbf{62.13}  \tiny{$\pm$ 0.92} \\
        IAT &  47.92 \tiny{$\pm$ 0.31}   & \textbf{51.34} \tiny{$\pm$ 0.83} && 64.38 \tiny{$\pm$ 0.33}   & \textbf{71.12} \tiny{$\pm$ 0.29} \\
        C\&W & 40.73 \tiny{$\pm$ 0.64}     &  \textbf{50.48} \tiny{$\pm$ 1.07} && 55.96 \tiny{$\pm$ 0.88}   & \textbf{63.21}  \tiny{$\pm$ 0.72} \\
        TRADES &  55.43 \tiny{$\pm$ 0.42}   & \textbf{58.20} \tiny{$\pm$ 0.61} && 64.34 \tiny{$\pm$ 0.40}   & \textbf{68.47} \tiny{$\pm$ 0.32} \\
        \bottomrule
    \end{tabular}}
    \hfill
    \parbox{0.65\linewidth}{
    \centering
    \begin{tabular}{{@{}lccccc@{}}}
    \multicolumn{6}{c}{Adversary generated from \textbf{base method}.} \\
    \toprule 
    \multirow{2.5}{*}{Training} & \multicolumn{2}{c}{$\ell_\infty (\epsilon = \frac{8}{255})$} &
    \phantom{} & 
    \multicolumn{2}{c}{$\ell_2 (\epsilon = 0.5)$} \\
    \cmidrule{2-3} \cmidrule{5-6}
    %&
        & Method & +RFI &&  Method & +RFI \\ \midrule 
        PGD & \textbf{42.96} \tiny{$\pm$ 0.75}     &    41.38 \tiny{$\pm$ 0.48} && \textbf{56.48} \tiny{$\pm$ 0.42}   & 54.28  \tiny{$\pm$ 0.62} \\
        IAT &  \textbf{46.22} \tiny{$\pm$ 0.10}   & 43.44 \tiny{$\pm$ 0.21} && \textbf{63.48} \tiny{$\pm$ 0.96}   & 62.19 \tiny{$\pm$ 0.09} \\
        C\&W & \textbf{41.62} \tiny{$\pm$ 0.90}     &  39.10 \tiny{$\pm$ 0.81} && \textbf{56.63} \tiny{$\pm$ 0.68}   & 55.19  \tiny{$\pm$ 0.91} \\
        TRADES &  \textbf{53.67} \tiny{$\pm$ 0.43}   & 52.88 \tiny{$\pm$ 0.33} && \textbf{62.12} \tiny{$\pm$ 0.28}   & 59.85 \tiny{$\pm$ 0.97} \\
        \bottomrule
    \end{tabular}}
    \label{tab:rfi_calib_transferability}
\end{table}



\subsection{Static vs Dynamic RFI on calibrated model}
\label{app:static_dyn_calib} 
We extend the study of static vs dynamic RFI to calibrated models in this section using the same setup as Section~\ref{ss:static_dyn_rfi}, where we consider pretrained ResNet-18 on CIFAR-10 by applying PGD ($\ell_\infty, \epsilon=8/255$) and ($\ell_2, \epsilon=0.5$) in transfer attack setting i.e. generate adversarial examples from the base method. For the dynamic setting we compute the covariance batch-wise to compute $\tilde{U}$ with the input. Table~\ref{tab:rfi_static_adaptive_extra} shows \emph{static is better than dynamic RFI similar to non-calibrated setting}.
\begin{table}[h]
    \centering
    \caption{\textbf{Additional Comparison of static and dynamic/adaptive RFI on calibrated model showing static RFI is better than dynamic RFI.} Setting same as Table~\ref{tab:rfi_calib}. Adversarial examples are generated from the base model for fair comparison.}
    \resizebox{0.6\linewidth}{!}{
    \begin{tabular}{{@{}lcccccccc@{}}}
    \toprule 
    \multirow{2.5}{*}{Training} & \multicolumn{2}{c}{Clean} & \phantom{}& \multicolumn{2}{c}{$\ell_\infty (\epsilon=\frac{8}{255})$} &
\phantom{} & 
\multicolumn{2}{c}{$\ell_2 (\epsilon=0.5)$}\\
\cmidrule{2-3} \cmidrule{5-6} \cmidrule{8-9}
%&
& Static & Dynamic && Static & Dynamic &&  Static & Dynamic\\ \midrule
        Standard &  10.36 & \textbf{11.65} &&20.08 & 11.64 &&\textbf{20.91} & 12.43\\
         Robust CIFAR-10 & \textbf{78.78} & 75.23 && \textbf{15.41} & 12.89 && \textbf{17.38} & 16.32 \\
         PGD & 83.22 & 82.86 && 46.02 & \textbf{46.83}  &&58.81 & \textbf{59.23}\\
         IAT & 91.26 & \textbf{91.35} && \textbf{49.06} & 48.53 && \textbf{66.67} & 66.28\\
         C\&W & 84.97 & 83.01 && \textbf{45.48} & 43.98 &&     
         \textbf{58.95} & 57.82 \\
         TRADES& 80.76 & 78.98 && \textbf{54.33} & 
         53.58 && \textbf{65.23} & 65.00 \\
        \bottomrule
    \end{tabular}}
    \vspace{-0.2cm}
    \label{tab:rfi_static_adaptive_extra}
\end{table}

\subsection{Static RFI is Optimal}
\label{app:static_vs_dynamic}
In the case of dynamic RFI implementation, one needs to know when to apply the transformation as the model should be static for the attacker and adapted only for the defender. This poses implementation difficulty as the situation is mostly unknown in practice. Hence, we explore different variants of RFI in a dynamic setting where we compute the covariance matrix and eventually the transformation matrix $\Tilde{U}$ using the full validation set or single test input. 
We observe that \emph{the dynamic RFI is only marginally better than the static RFI} when full validation set is used in Table~\ref{tab:rfi_full_val_one_test} (a). 
Similarly, we present the result for single test input in Table~\ref{tab:rfi_full_val_one_test} where \emph{the method shows improvement in clean performance} since it is only a normalization of the feature representation. We perform these comparisons to highlight the fact that these hypothetical variants of dynamic RFI which work with information of validation set are also not significantly better than static RFI, thereby implying that \textbf{static RFI is indeed the optimal way of selecting $\tilde{U}$ as indicated by our theory}.

\begin{table}[h]
\caption{\textbf{Static RFI is the optimal approach.} RFI with covariance matrix calculated using different approaches.}
\parbox{.5\linewidth}{
    \centering
    \resizebox{0.95\linewidth}{!}{
    \begin{tabular}{{@{}lcccccccc@{}}}
    \multicolumn{9}{c}{(a) RFI with covariance matrix calculated using complete validation set} \\
    \toprule
        \multirow{2.5}{*}{Training} & \multicolumn{2}{c}{Clean} & \phantom{}& \multicolumn{2}{c}{$\ell_\infty (\epsilon=\frac{8}{255})$} &
        \phantom{} & 
        \multicolumn{2}{c}{$\ell_2 (\epsilon=0.5)$}\\
        \cmidrule{2-3} \cmidrule{5-6} \cmidrule{8-9}
        %&
        & Method & +RFI && Method & +RFI &&  Method & +RFI\\ \midrule
        %\multirow{5}{*}{CIFAR-10}&
        Standard & \textbf{95.28} & 88.53 && 1.02  & \textbf{9.35}  && 0.39 & \textbf{11.73}\\
        Robust CIFAR-10 & 78.69 & \textbf{78.80} && 1.30 & \textbf{11.21} && 9.63 & \textbf{12.56}\\
         PGD & \textbf{83.53} & 83.29 && 42.20  & \textbf{43.82} && 54.61 & \textbf{56.13}\\
         IAT & \textbf{91.86} & 91.32 && 44.76  & \textbf{47.65}  && 62.53 & \textbf{64.88}\\
         C\&W & \textbf{85.11} & 85.06 && 40.01 & \textbf{43.48} && 55.02 & \textbf{57.83}\\
         TRADES &  \textbf{81.13} & 80.97 && 51.70 &\textbf{54.29} && 60.03  & \textbf{61.79}\\
         \bottomrule
    \end{tabular}}
    }
    % \label{tab:rifar_full_val}
\hfill
\parbox{.5\linewidth}{
    \centering
    \resizebox{0.95\linewidth}{!}{
    \begin{tabular}{{@{}lcccccccc@{}}}
    \multicolumn{9}{c}{(b) RFI with covariance matrix calculated using single test input} \\
    \toprule
        \multirow{2.5}{*}{Training} & \multicolumn{2}{c}{Clean} & \phantom{}& \multicolumn{2}{c}{$\ell_\infty (\epsilon=\frac{8}{255})$} &
        \phantom{} & 
        \multicolumn{2}{c}{$\ell_2 (\epsilon=0.5)$}\\
        \cmidrule{2-3} \cmidrule{5-6} \cmidrule{8-9}
        %&
        & Method & +RFI && Method & +RFI &&  Method & +RFI\\ \midrule
        %\multirow{5}{*}{CIFAR-10}&
        Standard & \textbf{95.28} & 90.10 && 1.02  & \textbf{10.81}  && 0.39 & \textbf{12.16}\\
        Robust CIFAR-10 & 78.69 & \textbf{78.70} && 1.30 & \textbf{11.88} && 9.63 & \textbf{12.87}\\
         PGD & \textbf{83.53} & 83.52 && 42.20  & \textbf{44.08} && 54.61 & \textbf{56.53}\\
         IAT & \textbf{91.86} & 91.86 && 44.76  & \textbf{47.95}  && 62.53 & \textbf{65.01}\\
         C\&W & \textbf{85.11} & 85.11 && 40.01 & \textbf{43.48} && 55.02 & \textbf{58.09}\\
         TRADES &  \textbf{81.13} & 81.09 && 51.70 &\textbf{54.78} && 60.03  & \textbf{62.17}\\
         \bottomrule
    \end{tabular}}
    }
    % \label{tab:rifar_full_val_single}
    \label{tab:rfi_full_val_one_test}
    \vspace{-0.1cm}
\end{table}

\subsection{Ablation study}
\subsubsection{Effect of $K$}
Neural Collapse is a phenomenon in which the penultimate feature of each class collapses to its mean after the training error reaches zero. This implies that there is principally only $C=\#classes$ number of feature vectors, one for each class. Hence, we suggest setting $K$ to number of classes. We also justify it experimentally in Figure~\ref{fig:ablation_k}. Additional experiments for large-scale models used in Tables~\ref{tab:benchmarkcomparison} and \ref{tab:benchmarkcomparison_app} with respect to CIFAR-10 and CIFAR-100 also show drop in eigenvalues at the number of classes across models, justifying our choice for $K$. We also extend the ablation study on $K$ to report the best performance in Figure~\ref{fig:ablation_k} and the optimal $K$ row in \ref{tab:benchmarkcomparison_app}. %We have now extended it to all the large-scale models for CIFAR-10 and CIFAR-100 from the Robustbench benchmark in Figure ~\ref{fig:ablation}. 
%As mentioned in the draft, the choice of $K$ is generally equal to the number of classes since the representation of each class collapses to one mean vector. This choice of $K$ is observed empirically to be effective as it improves the robust performance over the baseline across all the settings in general. 
Note that the optimal $K$ for robust performance is not the best for standard performance as we are choosing only the top-most informative features (Corollary~\ref{cor:information}). 
% Figure environment removed
% Figure environment removed




\subsubsection{Effect of step size in PGD}
We chose $\epsilon/4$ and $\epsilon/5$ for step sizes in $\ell_\infty$ and $\ell_2$, respectively, following the benchmarks in several works in RobustBench. 
The other common choice for the step size is proportional to the iterations, that is, $2\epsilon/40$ and $
2\epsilon/100$ for $\ell_\infty$ and $\ell_2$, respectively. 
We reevaluated the models in Table~\ref{tab:rfi_calib} with and without RFI for these step sizes and the results are in Table~\ref{tab:step_size_pgd}, showing that \emph{RFI is better than the base model}, in line with the observations in the previous experiments.

\begin{table}[H]
    \centering
    \caption{\textbf{RFI is more robust than the base model irrespective of the step size in PGD.} $2\epsilon/40$ for $\ell_{\infty}$ and $2\epsilon/100$ for $\ell_2$.}
    \begin{tabular}{{@{}lcccccc@{}}}
    \toprule 
    \multirow{2.5}{*}{Training} & \multicolumn{2}{c}{$\ell_\infty (\epsilon=\frac{8}{255})$} &
    \phantom{} & 
    \multicolumn{2}{c}{$\ell_2 (\epsilon=0.5)$}\\
    %&
    \cmidrule{2-3} \cmidrule{5-6}
    & Method & +RFI &&  Method & +RFI\\ \midrule 
    Standard & 0.03   &\textbf{9.73}&& 3.67 &\textbf{14.13}\\
    PGD & 44.44 & \textbf{45.48} && 57.77 &\textbf{58.97}\\
    IAT  & 45.91 & \textbf{48.26} && 66.26   & \textbf{67.73} \\
    Robust CIFAR10 & 7.14 & \textbf{15.57} && 12.94  & \textbf{17.15}\\ CW Attack & 38.89 & \textbf{41.53} && 51.20 & \textbf{54.45}\\
    TRADES  & 52.90  & \textbf{54.10} &&  61.66  & \textbf{63.35}\\

        \bottomrule
    \end{tabular}
    \label{tab:step_size_pgd}
\end{table}


%\subsection{Time comparison}




\begin{comment}
\begin{table*}[t]
    \centering
    \caption{\textbf{Robust performance evaluation of RFI on state-of-the-art methods.} We apply APGD-CE, APGD-DLR and RobustBench attacks on both CIFAR-10 and CIFAR-100. The inference time for RFI is $1\times$, whereas Anti-adv and SODEF are $8\times$ and $2\times$, respectively. }
    \resizebox{\linewidth}{!}{
    \begin{tabular}{@{}lclcccccc@{}}
         \toprule 
         %Dataset & 
         & Base Method & Adaptive Defense & Clean & APGD-CE  & APGD-DLR  &FAB & Square & AutoAttack\\
         \midrule
         \multirow{15}{*}{\begin{sideways}CIFAR-10\end{sideways}}&
         \multirowcell{4}{~\citet{carmon2019unlabeled}\\WideResNet-28-10} & None & \textbf{89.69} &  61.82 & 60.85 & 60.18 & 66.51 & 59.53\\
          &
          &Anti-adv \cite{alfarra2022combating}& \textbf{89.69} & 61.81 & 60.89 & 60.11 & 66.58 & 58.70\\
          &
          &SODEF \cite{kang2021stable}& 89.68 & 60.20 & 60.72 & 58.04 & 65.28 & 57.23\\
          &
          &RFI ($K=10$) & 89.60 & 62.38 & 61.58 & 60.21 & 66.59 & 60.72\\
          &
          &RFI (opt. $K=20$) & 89.60 & \textbf{62.45} & \textbf{61.60}& \textbf{60.38}& \textbf{66.90} & \textbf{61.02}\\
         \cmidrule{2-9}
         &
         \multirowcell{4}{~\citet{rice2020overfitting}\\WideResNet-34-10}  & None & 85.34 & 50.12 & 56.80 & 53.87 & 56.88 & 53.42\\
          &
          &Anti-adv \cite{alfarra2022combating} & \textbf{85.40} & 50.10 &  57.50 & 53.90 & 57.00 & 50.98\\
          &
          &SODEF \cite{kang2021stable}& 85.10 & 50.60 & 56.50& 53.72 & 56.21 & 50.09\\
          &
          &RFI($K=10$) & 85.30 & 51.19 & 58.55 & 53.98 & \textbf{57.13} & 54.64\\
          &
          &RFI (opt. $K=35$) & 85.30 & \textbf{51.62} &  \textbf{58.97} & \textbf{54.12}& \textbf{57.13} & \textbf{54.86} \\
          \cmidrule{2-9}
          %\midrule
         &
         \multirowcell{4}{~\citet{wang2023better}\\WideResNet-28-10}  & None & \textbf{92.44} & 70.23 &  67.82 & 67.41 & 73.13 & 67.31\\
          &
          &Anti-adv \cite{alfarra2022combating} & \textbf{92.44} & 68.90 &  65.91 & 67.55 & 73.20 & 66.52\\
          &
          &SODEF \cite{kang2021stable}& 92.01 & 67.53 & 65.08 & 65.93 & 73.01 & 64.20\\
          &
          &RFI ($K=10$) & 92.33 & 70.32 & 67.86 & \textbf{67.82} & 73.52 & 67.29\\
          &
          & RFI (opt. $K=20$) & 92.34 & \textbf{70.36} & \textbf{67.90} & \textbf{67.82} & \textbf{73.54} & \textbf{67.50}\\
          %\cline{2-7}
          \midrule
         %\hline 
         %\textbf{CIFAR-100} \\
         \multirow{15}{*}{\begin{sideways}CIFAR-100\end{sideways}}&
         \multirowcell{4}{~\citet{addepalli2022efficient}\\ResNet-18} & 
         None & \textbf{65.45} & 33.49 & 28.55  & 28.00 & 33.70 & 27.67\\
          &
          &Anti-adv \cite{alfarra2022combating}& 65.38 & 30.92 & 26.61 & 27.92 & 33.61 & 26.01\\
          &
          &SODEF \cite{kang2021stable}& 65.23 & 29.37 & 26.90 & 24.62 & 29.60 & 26.53\\
          &
          &RFI $(K=\text{opt. } K = 100)$& 65.41 & \textbf{34.09} & \textbf{29.18} & \textbf{28.10} & \textbf{33.79} & \textbf{27.80}\\
          %&
          %&RFI (opt. $K=100$) & 65.41 & \textbf{34.09} & \textbf{29.10} & \textbf{27.80}\\
          \cmidrule{2-9}
         %\midrule
         &
         \multirowcell{4}{~\citet{rice2020overfitting}\\PreActResNet-18}  & None & \textbf{53.83} & 20.83 & 20.46 & \textbf{23.82} & 19.29 & 18.95\\
          &
          &Anti-adv \cite{alfarra2022combating} & \textbf{53.83} & 20.78 &  20.06 & 23.49 & 19.27 & 18.97\\
          &
          &SODEF \cite{kang2021stable}& \textbf{53.83} & 18.50 & 19.20 & 19.66 & 16.05 & 16.92\\
          &
          &RFI ($K=100$)& 53.70 & 21.10 & 20.98 & 20.93 & 18.13 & 19.23\\
          &
          &RFI (opt. $K=150$) & 53.75 & \textbf{21.18} & \textbf{21.10} & 21.03 & \textbf{19.53} &\textbf{19.46}\\
           \cmidrule{2-9}
          %\midrule
         &
         \multirowcell{4}{~\citet{wang2023better}\\WideResNet-28-10}  & None & \textbf{72.58} & 44.04 &  39.78 & 39.19 & 44.46 & 38.83\\
          &
          &Anti-adv \cite{alfarra2022combating} & 72.57 & 42.98 &  38.10 & 36.85 & 44.49 & 34.01\\
          &
          &SODEF \cite{kang2021stable}& 72.34 & 38.10 & 36.95 & 34.82 & 44.42 & 32.29\\
          &
          &RFI ($K=100$)& 72.55 & 44.37 & 39.91 & 39.68 & 44.50 & 39.10\\
          &
          & RFI (opt. $K=115$) & 72.55 & \textbf{44.51} & \textbf{39.96} & \textbf{39.81} & \textbf{44.53} &\textbf{39.13}\\
          \midrule
         %\hline 
         %\textbf{CIFAR-100} \\
         \multirow{4}{*}{\begin{sideways}ImageNet\end{sideways}}&
         \multirowcell{2}{~\citet{salman2020adversarially}\\ResNet-50} & 
         None & \textbf{64.02} & 38.32 & 34.02  & 34.35 & 49.52 & -\\
          &
          & Dynamic RFI & 63.91 & \textbf{38.48} & \textbf{34.68} & \textbf{34.68}& \textbf{49.98} & -\\
          \cmidrule{2-9}
          &\multirowcell{2}{~\citet{salman2020adversarially}\\WideResNet-50-2} & 
         None & \textbf{68.46} & 40.67 & 37.09  & 37.81 & 54.61 & -\\
          &
          & Dynamic RFI & 68.41 & \textbf{40.84} & \textbf{37.56} & \textbf{38.12}& \textbf{54.78} & -\\
         \bottomrule
    \end{tabular}
    }
    %\caption{(Base/Ours) Indicate the performance of adversarial training methods after adding our adaptive test time defense on APGD-CE and APGD-DLR attacks ~\cite{croce2020reliable}}
    \label{tab:benchmarkcomparison_app}
\end{table*}
\end{comment}


\subsection{Conceptual ideas similar to RFI}

\paragraph{Low dimensional last layer.}
Similar to comparing RFI on last layer vs on intermediate layer in Section~\ref{ss:rfi_last_intermediate}, here we compare RFI and directly training a network with $K$ neurons in the last layer.
We consider two ResNet-18 models with an additional fully connected hidden layer of size $512$ and $10$, respectively, and are trained with PGD. We apply RFI only to the larger model with $512$ neurons and reduce the dimension to $10$, and compare the performances in terms of clean and robust accuracies in both cases. The results are presented in Table~\ref{tab:low_dim}, showing that \emph{RFI is more robust compared to imposing a low dimensional last layer.}

\begin{table}[H]
    \centering
    \caption{\textbf{RFI is more robust compared to imposing a low dimensional last layer.} ResNet-18 with last hidden layer size $10$ and $512$. RFI done on model with $512$ hidden layer. }
    % \resizebox{0.8\linewidth}{!}{
    \begin{tabular}{lccc}
    \toprule
         & +hidden layer=10 & +hidden layer=512 & +hidden layer=512 + RFI  \\
         \midrule
         Clean & 83.71 & \textbf{84.13} & \color{applegreen}84.05\\
         Robust ($\ell_\infty$)  & 42.43 & 42.73 & \color{applegreen}\textbf{43.53} \\
         \bottomrule
    \end{tabular}%}
    \label{tab:low_dim}
\end{table}

We further argue qualitatively why setting low dimension layers is not equivalent to RFI as follows.
Firstly, overparameterization is shown empirically to be the key for both generalization \cite{brutzkus2019larger} and robustness \cite{madry2018towards}. Especially in the case of CNN, there is an empirical understanding to build the network with more than one fully connected layer after the convolution layers starting with larger widths to generalize well \cite{bengio2012practical}. These findings oppose the idea of having low dimension for the last hidden layer.
Secondly, there are similar insights from the sparsity of neural networks -- a smaller subnetwork with similar performance can be obtained by sparifying the network, called a lottery ticket \cite{frankle2018lottery}. Once known, lottery tickets can be trained from scratch to reach similar performance as the original network. However, it is not possible to obtain the ticket simply by setting hyperparameters for a smaller network from the beginning.
Finally, we emphasize that with RFI the last hidden-layer dimension is reduced by a large amount in comparison to the actual model. For example, in CIFAR-10, ResNet-50 with $2048$ dimensions is reduced to $10(=K)$. So, the network with $10$ dimension conventionally would not help generalization, which is conclusively established in the above experiment.

\subsection{Visualization of robust and non-robust features}
\label{app:feature_vis}
We obtain the visualizations of robust and non-robust features for an input $\mathbf{x}$ by solving $$\arg\min_{\tilde{\mathbf{x}}} || \Phi(\tilde{\mathbf{x}}) - \Phi(\mathbf{x})\mathbf{U}\mathbf{U}^T ||_2 $$ where $\mathbf{U}$ is top $K$ eigenvectors based on $s_c(.)$ for robust features and all eigenvectors except top $K$ for non-robust features. The objective is solved using gradient descent. Figure~\ref{fig:feature_vis} shows the visualizations of features for a few classes in CIFAR-10 using the PGD adversarially trained ResNet-18 model. `Robust $K=10$' and `Non-robust $K=10$' columns are obtained by setting $\mathbf{U}$ to the top $K$ eigenvectors and everything except the top $K$ eigenvectors based on $s_c(.)$, respectively. The columns top and bottom $100$ eigenvectors are obtained by setting $\mathbf{U}$ to the top and bottom $100$ eigenvectors based on the eigenvalues.
The feature visualizations show that robust and top eigenvectors result in more similar features. The interesting observation is that the non-robust and bottom eigenvectors are equally noisy and might have some useful information that reflects the drop in clean performance. Nevertheless, it is not possible to argue based on the visual interpretation of the features since the difference is primarily coming from the eigenspace of the feature covariance.
% While the non-robust features appear to be more noisy than the robust features visually, it is also possible that the non-robust features have some useful information that reflects the drop in clean performance.
% Figure environment removed


\begin{comment}

\paragraph{Eigenvalue truncation of intermediary layers}
We evaluate effectiveness of performing RFI on intermediate layers by truncating the last but one hidden layer of ResNet-18 and evaluate the PGD trained model considered in Table 1. This hidden layer has 512*4*4 convolution which we project to 10*4*4 using RFI procedure and the results are shown below. Doing RFI on the last hidden layer shows the best performance.
\begin{table}[H]
    \centering
    \resizebox{\linewidth}{!}{
    \begin{tabular}{ccc}
    \toprule
         no RFI & RFI on last layer & RFI on last but one layer  \\
         \midrule
          42.20 & 43.29 & 36.06\\
         \bottomrule
    \end{tabular}}
    \caption{ResNet-18 model. }
    \label{tab:rfi_intermediary}
\end{table}


\end{comment}
%\begin{comment}
\section{System Architecture}
\label{appendix:architecture}
\system has a novel modularized system architecture with three key components: 
\emph{StreamManager}, 
\emph{TxnManager} and \emph{TxnScheduler}. 
These components are instantiated in each thread locally.
The execution outline of \system is presented in Algorithm~\ref{alg:algo}.
Transactional stream processing is continuous and potentially never ends (Line 1$\sim$8).
The dependency resolution and execution of state transactions are separated into two non-overlapping phases by punctuations~\cite{Tucker:2003:EPS:776752.776780} (Line 2 and 5), which guarantees that no subsequent input event will have a smaller timestamp. 
Effectively, a batch of state transactions is collected during the first phase, and processed during the second phase.

In the first phase (i.e., stream processing phase), 
the \emph{StreamManager} conducts preprocessing for every input event ($e$). Similar to some prior works~\cite{tstream}, state transactions may be issued but not immediately processed during preprocessing (Line 3).
The \emph{pre\_processing} and \emph{post\_processing} functions are exposed as APIs to users.
The \emph{TxnManager} handles dependency resolution (Line 4) among state transactions and insert decomposed operations to construct a \tpg. We discuss the detailed two-phase \tpg construction process in Section~\ref{subsec:construction}.

In the second phase  (i.e., transaction processing phase), 
the \emph{TxnManager} is first involved again to refine (Line 6) the constructed \tpg with further dependency resolution.
The \emph{TxnScheduler} 
schedules operations for concurrent execution based on the constructed \tpg according to the three dimensions of scheduling decisions (Line 7). 
In particular, a scheduling decision model $M$ is instantiated based on the constructed \tpg (Line 14).
\textbf{\circled{1}} Guided by $M$, execution threads adopt an exploration strategy (Section~\ref{subsec:explore}) to explore the constructed \tpg for operations available to be scheduled constrained by dependencies. 
\textbf{\circled{2}} 
During exploration, one or multiple operations may be treated as the 
% basic 
unit of scheduling (Section~\ref{subsec:granularity}). 
Subsequently, \textbf{\circled{3}} every thread executes operation(s) in the unit of scheduling with various abort handling mechanisms (Section~\ref{subsec:abort_handling}).
Only when state transactions are processed (i.e., committed or aborted) can the associated input events be postprocessed (Line 8) by the \emph{StreamManager} based on transaction processing results.
\end{comment}

\begin{comment}
\begin{algorithm}
\footnotesize
    \KwData{$e$ \tcp{Input event}}
    \KwData{$txn_{ts}$ \tcp{State transaction}}
    \KwData{$G$ \tcp{The currently constructed TPG}}
    \While{!finish processing of input streams}{
        \eIf(\tcp*[h]{Phase 1}){\text{$e$ is not a $punctuation$}}{
                $txn_{ts}$ $\gets$ PRE\_Processing($e$)\;
                \textbf{TPG\_Construction}($G$, $txn_{ts}$)\; 
          }(\tcp*[h]{Phase 2}){
                \textbf{TPG\_Refinement}($G$)\; 
                \textbf{TXN\_Scheduling}($G$)\; 
                POST\_Processing()\;
          }
    }
    
    \SetKwFunction{FMain}{TPG\_Construction}
    \SetKwProg{Fn}{Function}{:}{}
    \Fn{\FMain{$G$, $txn_{ts}$}}{
        $O_{1..k}$ $\gets$ \textbf{Partition} $txn_{ts}$\;
        \ForEach{\text{operation $O_{i}$ $\in$ $O_{1..k}$}}{
            \textbf{Identify} its \ld\;
            $G$ $\gets$ $G$ + $O_{i}$ \;
        }
    }
    \SetKwFunction{FMain}{TPG\_Refinement}
    \SetKwProg{Fn}{Function}{:}{}
    \Fn{\FMain{$G$}}{
        \ForEach{\text{vertex $e_{i}$ $\in$ $G$}}{
            \textbf{Identify} its \td, \pd\;
        }
    }
    
    \SetKwFunction{FMain}{TXN\_Scheduling}
    \SetKwProg{Fn}{Function}{:}{}
    \Fn{\FMain{$G$}}{
        $M$ $\gets$ Instantiated with $G$;\tcp{A decision model}
        \While{!finish scheduling of $G$
        }{
          \textbf{\circled{2}} $Scheduling Unit$ $\gets$ \textbf{\circled{1}} \emph{Explore}($G$, $M$)\; 
            \textbf{\circled{3}} \emph{Execute with Abort Handling} ($Scheduling Unit$)\; 
        }
    }
  \caption{Execution Outline of \system}
  \label{alg:algo}
\end{algorithm}
\end{comment}
\end{document}
