\documentclass[10pt,twocolumn,letterpaper]{article}

\usepackage{iccv}
\usepackage{times}
\usepackage{epsfig}
\usepackage{graphicx}
\usepackage{amsmath}
\usepackage{amssymb}


%%%% custom packages
\usepackage{placeins}
\usepackage{float}
% \usepackage{subfigure}
\usepackage{url}
% \usepackage{hyperref}
\usepackage{booktabs}
\usepackage{subcaption}
\usepackage{cite}
\usepackage{xcolor}
\usepackage{adjustbox}
\usepackage{algorithm}
\usepackage[noend]{algpseudocode}


\newcommand{\dalle}{DALL·E 2}
\newcommand{\ours}{RICo}

\usepackage[pagebackref=true,breaklinks=true,letterpaper=true,colorlinks,bookmarks=false]{hyperref}

\iccvfinalcopy % *** Uncomment this line for the final submission

\def\iccvPaperID{} % *** Enter the ICCV Paper ID here
\def\httilde{\mbox{\tt\raisebox{-.5ex}{\symbol{126}}}}

% Pages are numbered in submission mode, and unnumbered in camera-ready
\ificcvfinal\pagestyle{empty}\fi

\begin{document}

%%%%%%%%% TITLE
\title{RICo: Rotate-Inpaint-Complete for Generalizable Scene Reconstruction
}

\author{Isaac Kasahara, Shubham Agrawal, Selim Engin, Nikhil Chavan-Dafle, Shuran Song, and Volkan Isler\\ 
Samsung AI Center, New York\\
}

\maketitle
% Remove page # from the first page of camera-ready.
\ificcvfinal\thispagestyle{empty}\fi


\DeclareRobustCommand{\vinote}[1]{{\textcolor{blue}{VI: #1}}{}}
\DeclareRobustCommand{\ncdnote}[1]{{\textcolor{magenta}{NCD: #1}}{}}
\DeclareRobustCommand{\sanote}[1]{{\textcolor{green}{SA: #1}}{}}
\DeclareRobustCommand{\iknote}[1]{{\textcolor{red}{IK: #1}}{}}
\DeclareRobustCommand{\senote}[1]{{\textcolor{orange}{SE: #1}}{}}
\DeclareRobustCommand{\shubhamnote}[1]{{\textcolor{green}{Shubham: #1}}{}}
\DeclareRobustCommand{\todo}[1]{{\textcolor{brown}{\textbf{TODO: #1}}}{}}

%%%%%%%%%%%%%%%%%%%%%%%%%%%%%%%%%%%%%%%%%%%%%%%%%%%%%%%%%%%%%%%%%%%%%%%%%%%%%%%%
\begin{abstract}

General scene reconstruction refers to the task of estimating the full 3D geometry and texture of a scene containing previously unseen objects. In many practical applications such as AR/VR, autonomous navigation, and robotics, only a single view of the scene may be available, making the scene reconstruction a very challenging task. 
In this paper, we present a method for scene reconstruction by structurally breaking the problem into two steps: rendering novel views via inpainting and 2D to 3D scene lifting. Specifically, we leverage the generalization capability of large language models (\dalle{}) to inpaint the missing areas of scene color images rendered from different views. Next, we lift these inpainted images to 3D by predicting normals of the inpainted image and solving for the missing depth values.  By predicting for normals instead of depth directly, our method allows for robustness to changes in depth distributions and scale.
With rigorous quantitative evaluation, we show that our method outperforms multiple baselines while providing generalization to novel objects and scenes.

\end{abstract}

\section{Introduction}

% Figure environment removed



The understanding of 3D scene geometry is essential for many down-stream applications.  In robotics, it allows for accurate manipulation and motion planning considering the surrounding environment.  In the field of augmented reality, it allows for better mapping and rendering to bridge the virtual world to the real world.  With smartphones and robots that are equipped with high quality depth sensors, the task of 3D scene reconstruction is becoming feasible in these domains. 
%
These depth sensors allow for accurate reconstruction of the observed parts of the scene. However, to reconstruct the unseen parts, we must use prior information conditioned on the observed information. The missing information in the input image combined with the diversity in shapes, sizes, and depth distribution of the household objects presents a major challenge for scene reconstruction in-the-wild. 
%
In this paper, we study this problem in a general setting, where the goal is to reconstruct a complex scene with multiple novel objects, given only one RGB-D image of the scene.  
%



We present our method Rotate-Inpaint-Complete (\ours{}), which predicts both the 3D geometry and the texture of the unseen parts of the scene in the input image by leveraging the inpainting capabilities of large visual-language models.
%
Given an RGB-D image of a scene, first we generate novel views (RGB and depth images) by rotating and then projecting the input scene. Then we use a surface-aware masking method to select regions in the image to allow us to inpaint utilizing the powerful 2D inpainting capabilities of \dalle{}~\cite{ramesh2022hierarchical} for exposing the potential object geometry not visible in the input image. 
%
Finally, we optimize the depth images using the input depth values and occlusion boundaries and normals estimated from the inpainted images. These inpainted and completed novel RGB-D views provide the reconstructed scene geometry as a fused pointcloud with associated textures.
To mitigate the object hallucination and spatial inconsistency of predictions from \dalle{}, we integrate algorithmic features such as filtering inpainting outputs and enforcing consistency across viewpoints into our method that play a crucial role for generalizable, yet accurate and robust scene reconstruction.

%-------------------------------------------------------

We demonstrate our method on cluttered scenes with unseen household objects and categories. Through a series of rigorous quantitative experiments, we show that our approach outperforms baseline methods in settings where no training data is available.

% \subsection{Statement of Contributions}
In short, the contributions of this paper can be summarized as follows. \textit{i)} We present an integrated approach for scene completion of unseen objects under occlusion and clutter, by solving the problem through novel view inpainting and 2D to 3D scene lifting. \textit{ii)} We develop a method for selectively inpainting regions in the novel views of the input scene that enables synthesis of consistent 2D geometry. \textit{iii)} We train a 2D to 3D lifting method on the YCB-V~\cite{xiang2018posecnn} dataset and demonstrate the generalization capability on novel household objects and categories which is crucial for maintaining the generalization capability of our integrated scene reconstruction method.



\section{Related Work}


\textbf{Scene Reconstruction:}
% Scene reconstruction refers to the problem of estimating the 3D geometry of an environment, usually containing multiple objects, from a single image.
While single-object reconstruction is a well-studied problem, full-scene reconstruction is explored in limited settings. Previously works in scene reconstruction were focused either on room scale~\cite{song2017semantic, dai2018scancomplete} or in autonomous driving settings~\cite{cheng2021s3cnet, rist2021semantic, cao2022monoscene} where the scene geometries are usually more structured. In this work, we focus on an object-level scale, specifically tabletop environments, where objects can be in cluttered configurations. While methods like ~\cite{gkioxari2019mesh, popov2020corenet, irshad2022centersnap} % Mesh R-CNN~\cite{gkioxari2019mesh}, CoReNet~\cite{popov2020corenet}, and CenterSnap~\cite{irshad2022centersnap} 
show an accurate reconstruction of objects at the scene level, they do not generalize to novel category objects. Recently,~\cite{wu2023multiview} introduced a method for reconstructing 3D geometries of objects and scenes of unseen categories, and demonstrated generalization capability to objects in-the-wild. However, different from our setting, they mostly focus on isolated objects and scenes with little to no clutter. In contrast, our method can reconstruct geometries and textures of complex scenes with objects from novel categories under heavy occlusions, as we show in our experiments.

\textbf{Inpainting:}
% Inpainting is the process of filling in missing areas of images.  
%Traditionally, inpainting methods made use of image priors such as self-similarity for tasks like image restoration, where gaps with missing or corrupt values to be filled are usually small holes.
% Traditional inpainting methods made use of image priors to fill small holes.% where gaps with missing or corrupt values to be filled are usually small holes.
% Deep learning methods, on the other hand, using large amounts of training data achieved remarkable success for inpainting images with semantically consistent contents.
% Deep learning methods, on the other hand, achieved remarkable success for inpainting images with semantically consistent contents.
% Deep generative methods like Generative adversarial networks~\cite{goodfellow2020generative}, for example, were shown to handle many challenging inverse problems, including image denoising~\cite{chen2018image}, super-resolution~\cite{ledig2017photo}, and inpainting~\cite{pathak2016context, iizuka2017globally, zhao2021large}.
% \todo{While inpainting was typically done using, ... .  Deep learning methods such as ... also demonstrated improvement with inpainting capabilities in similar images the models had been trained on
% \url{https://openaccess.thecvf.com/content_cvpr_2017/papers/Yeh_Semantic_Image_Inpainting_CVPR_2017_paper.pdf}
% ~\cite{yeh2017semantic}.}
While traditional inpainting methods made use of hand-crafted image priors to fill small gaps for tasks like image restoration \cite{elharrouss2020image}, deep generative methods like Generative Adversarial Networks (GAN) ~\cite{goodfellow2020generative} have shown remarkable success for tasks like image denoising~\cite{chen2018image}, super-resolution~\cite{ledig2017photo}, and inpainting~\cite{pathak2016context, iizuka2017globally, zhao2021large}. However, GAN models are known for potentially unstable training for large datasets \cite{weng2021diffusion}. More recently, resulting from the growth of visual language diffusion models, which can be efficiently trained on internet-scale datasets, inpainting through image diffusion has shown great generalization capabilities to many different objects and scenes~\cite{ramesh2022hierarchical}.
% also taken off~\cite{ramesh2022hierarchical}. These models, having been trained on millions of images, have demonstrated their ability to generalize to many different objects, scenes, and styles.
In this paper, we develop a process to use a visual language diffusion model for inpainting cluttered scenes involving heavy occlusions.


\textbf{Text-to-3D Synthesis: }
% Another area that is currently rapidly being explored is that of shape completion using visual-language models with a combination of neural radiance fields (NeRF)~\cite{mildenhall2021nerf}.
Recent papers such as~\cite{jain2022zero, poole2022dreamfusion, lin2022magic3d} have demonstrated the ability to generate 3D models of very diverse objects from merely a text description. Despite the realistic appearance, these generated objects are not grounded to any real-world geometry.
% , so may not be as useful when estimating the shape of real objects.
To overcome this limitation, ~\cite{xu2022neurallift, melas2023realfusion} extended these methods to reconstruct based on a ground truth reference image. These papers demonstrate high accuracy on individual objects, but do not demonstrate the ability to reconstruct multiple objects in cluttered scenes. Moreover, their runtime is a concern, as optimizing neural radiance fields ~\cite{mildenhall2021nerf} can take up to an hour. ~\cite{nichol2022point} attempts to produce faster results by optimization without NeRF but are still limited to single object reconstruction and do not directly generalize to our setting. % While better optimized for speed, this method is still limited in focus to only single object reconstruction and does not directly generalize to the setting we study in this paper.

% \vspace{-3mm}

\section{Method}



% % Figure environment removed


% Method overview

In this section, we present our method Rotate-Inpaint-Complete, or \ours{}, for generalizable reconstruction of a 3D scene containing multiple objects, given a single RGB-D image of the scene.

\ours{} takes in as input an RGB-D image \(\mathcal{I} = (\mathbf{I}, \mathbf{D}) \in \mathbb{R} ^{ H \times W \times 4} \) and outputs a color point cloud \(S \in \mathbb{R} ^{ N \times (3 + 3)} \), where $N$ is the number of predicted points in the scene.
%
Our method consists of three main components: 1) An inpainting step that takes in an RGB-D image \(\mathcal{I}\) and outputs an inpainted RGB image \(\hat{\mathbf{I}}_i\) from a novel viewpoint $\mathbf{T}_i \in SE(3)$. 2) A depth completion component that takes in the inpainted RGB image \(\hat{\mathbf{I}}_i\) as well as an incomplete depth image \(\bar{\mathbf{D}}_i \) rendered from the viewpoint $\mathbf{T}_i$, and outputs a completed depth \(\hat{\mathbf{D}}_i\) at that viewpoint. 3) A viewpoint selection and consistency filtering method that utilizes the above two components to generate completed RGB-D images at rotated novel views and uses them to reconstruct the scene. We explain each of these components in detail next.



\subsection{Inpainting}

This section describes the inpainting process, as well as the intermediate steps taken before and after to go from the input image \(\mathbf{I}\) to \(\hat{\mathbf{I}}_i\) at a novel viewpoint $\mathbf{T}_i$.

\subsubsection{Rotate and Project RGB-D Image}
Given an RGB-D image of a scene and the camera intrinsics, we deproject the image into a point cloud in the camera frame. This point cloud is then projected onto a novel viewpoint $\mathbf{T}_i$ and the resulting image is masked using our Surface-Aware Masking method (SAM), which we describe in detail in the following section. The projection from this new viewpoint creates a new RGB-D image $\bar{\mathbf{I}}_i$ with missing RGB and depth information as seen in Figure~\ref{fig:method-overview}. Small holes of the missing RGB values are filled with a naive inpainting algorithm~\cite{telea2004image} by inpainting pixels that are covered after a morphological closing operation of kernel size 5 is applied to the mask. The larger missing areas are left for the deep inpainting module.

% \subsubsection{Background Mask Generation}
\subsubsection{Surface-Aware Masking}
\label{sec:bg-mask}

% Figure environment removed

In order for inpainting to work properly, a mask covering the areas to inpaint needs to be generated. After projecting to the new camera frame, any 3D space possible to be reconstructed needs to be represented as an inpainting mask in the 2D image. This issue can be seen in Figure~\ref{fig:sam_fig} as the table takes up pixels we may want to fill in with the bottle. To solve this problem, a 3D frustum is generated from the original camera and depth image. For every pixel in the original camera frame, a ray is cast from the camera through each point in the projected point cloud from \(\mathcal{I}\). Once the ray has passed through its respective point, it is used to generate a list of points along the ray from that depth onward with $m$ points of equal spacing $c$. This is done for every ray, and from this process results a point cloud covering the potential space that the 3D scene could possibly fill. This point cloud is then converted to a mesh, and when the point cloud from the RGB-D image \(\mathcal{I}\) is rotated to novel views, the mesh is rotated with it. Finally, when projecting back to the camera frame after rotation, points that are occluded by the mesh are discarded. Any blank pixels are then used as the 2D inpainting mask to be filled when passed to the inpainting step. This procedure of generating the final image and mask is detailed in our technical report (see Appendix) and its outputs are shown in Figure~\ref{fig:method-overview}, with the green pixels representing the inpainting mask.

% %%%%%%%%%%%%%%%%% surface-aware masking pseudocode

% \begin{algorithm}
% % \begin{adjustbox}{width=\columnwidth,center}
% \caption{\textsc{Surface-Aware Masking (SAM)}}

% \begin{algorithmic}
% \Require{Input RGB-D image $\mathcal{I} = (\mathbf{I}, \mathbf{D})$, intrinsics $\mathbf{K}$, new viewpoint $\mathbf{T}_i$}
% \State $U \leftarrow$ Subsample pixels from a uniform grid in $\mathcal{I}$
% \State $X \leftarrow \{\}$ \Comment{initialize an empty point set.}
% \ForAll {$\mathbf{u} \in U$}
% \State $\mathbf{x} \leftarrow \mathbf{D}(\mathbf{u}) \mathbf{K}^{-1} \mathbf{u}$ \Comment{deprojection of $\mathbf{u}$ to 3D point $\mathbf{x}$.}
% \For {$i \leftarrow 1$ to $m$}
%     \State $\mathbf{p} \leftarrow \mathbf{x} + i \cdot c \cdot \mathbf{K}^{-1} \mathbf{u}$
%     \State $X \leftarrow X \cup \{\mathbf{p}\}$ \Comment{set of points with equal spacing.}
% \EndFor
% \EndFor
% \State $\mathcal{M} \leftarrow$ Mesh($X$) \Comment{surface triangulation to create a mesh.}
% \State $\bar{\mathbf{I}}_i, \bar{\mathbf{D}}_i \leftarrow$ Reprojection of $\mathbf{I}, \mathbf{D}$ in camera viewpoint $\mathbf{T}_i$, where missing values are set to 0.
% \State $\widetilde{\mathbf{D}}_i \leftarrow$ Depth map rendering of $\mathcal{M}$ in camera $\mathbf{T}_i$
% \State $M \leftarrow \mathbf{0}_{H \times W}$ \Comment{initialize the mask image as zeros.}
% \ForAll {$\mathbf{u} \in M$}
% \State $M(\mathbf{u}) \leftarrow 1$ if $\bar{\mathbf{D}}_i(\mathbf{u}) = 0 \vee \bar{\mathbf{D}}_i(\mathbf{u}) > \widetilde{\mathbf{D}}_i(\mathbf{u})$ 
% \EndFor


% \State \textbf{return} $M, \bar{\mathbf{D}}_i$ 
% \end{algorithmic}
% \label{algo:surfawaremask}

% \end{algorithm}


% %%%%%%%%%%%%%%%%%




\subsubsection{Diffusion-based Inpainting}
\label{sec:diff-inpaint}
Once these preprocessing steps have been completed, we pass the processed image and a mask of areas to be filled in to the inpainting algorithm. We use \dalle{}\cite{ramesh2022hierarchical} for image inpainting since it demonstrates the ability to produce the most realistic results. This model takes in the incomplete image \(\bar{\mathbf{I}}_i\), the mask generated in the previous step \(M\), and an input prompt \(P\) that describes the context of the image in words. 
% When generating inpainted images for all our scenes, 
For prompt, we pass the RGB image $\mathbf{I}$ to a deep captioning model ~\cite{li2022blip} and prefix the generated caption with \textit{``A photo of''}. 
%We use prompts generated from a deep captioning model~\cite{li2022blip} preceded with \textit{``A photo of"}. 
We also explore using a more specific and generic prompt in our ablation experiments (Table~\ref{tab:prompt}).
The output from this inpainting method is an image  \(\hat{\mathbf{I}}_i\) that now contains estimated areas from the diffusion model. Figure~\ref{fig:method-overview} shows an example before and after inpainting with \dalle{}.

% \subsubsection{Inpainting Ranking Step}
% The resulting images from the inpainting method may vary in terms of their perceived realism for every new generation. We use it to our advantage by generating multiple inpainted images for the same incomplete image and mask. These inpainted images are then compared against the input prompt \(P\) by encoding them to the CLIP embedded space~\cite{DBLP:journals/corr/abs-2103-00020}. The image containing the highest similarity is chosen as the final inpainted image \(\hat{\mathbf{I}}_i\) at viewpoint $\mathbf{T}_i$.

\subsection{Depth Completion}
We use a method proposed in~\cite{zhang2018deepdepth} for generating a complete depth image \(\hat{\mathbf{D}}_i\) from an incomplete depth image \(\bar{\mathbf{D}}_i\) and its corresponding RGB image. This method estimates the normals and occlusion boundaries from the RGB image, and optimizes for the complete depth by utilizing the estimated normals, occlusion boundaries, and incomplete depth.

\subsubsection{Normals and Occlusion Boundaries Prediction}
In order to obtain estimations for the normals and occlusion boundaries, we train Deeplabv3+ with DRN-D-54 in the same manner as in~\cite{sajjan2020clear}. 
% Sajjan et al.
The ground truth normals and occlusion boundaries are obtained using the depth images from the YCB-V training dataset~\cite{xiang2018posecnn}, the YCB-V synthetic dataset \cite{denninger2020blenderproc, hodan2020bop}, and the HomebrewedDB synthetic dataset~\cite{kaskman2019homebreweddb}. 
% The occlusion boundaries are also obtained by using the ground truth depth from the same datasets and trained in the same manner as~\cite{sajjan2020clear}.

\subsubsection{Optimize for Depth}
Given the incomplete depth, the estimated normals from the image, and estimated occlusion boundaries, we solve for the completed depth. The main idea behind this method in~\cite{zhang2018deepdepth} is that the areas with missing depth can be computed by tracing along the estimated normals from areas of known depth with the occlusion boundaries acting as barriers where normals should not be traced across. Formally we solve a system of equations to minimize an error \(E\), where \(E\) is defined as \(E=\lambda_DE_D + \lambda_SE_S + \lambda_NE_NB\). Here, \(E_D\) is the distance between the ground truth and estimated depth, \(E_S\) influences nearby pixels to have similar depths, \(E_N\) measures the consistency of estimated depth and estimated normal values, and \(B\) weights the normal values based on the probability that it is a boundary. We use the same \( \lambda_D,\lambda_S,\lambda_N \) values as in~\cite{sajjan2020clear}. % Sajjan et al

\subsection{Scene Completion}
This section describes the complete process we follow to reconstruct a 3D scene from a single RGB-D image.

% \subsubsection{Point Cloud Rotation} \label{sec:pcl_rotation}
% Our main method consists of first deprojecting the original RGB-D image into a point cloud. The original image will be referred to as \(\mathbf{I}_0\). Next, we rotate the point cloud around its mean along the world $z$-axis (or perpendicular to the ground plane) by angle \(\theta\). The rotated point cloud is then projected back into the original camera plane. This results in an incomplete RGB image as well as an incomplete depth image. The RGB image is then inpainted, and used to complete the depth image as described in the depth completion section above. This new viewpoint can be seen in Figure~\ref{fig:figure-3} as \(\mathbf{T}_1\). We denote the completed RGB-D image from this viewpoint as \(\hat{\mathbf{I}}_1\). We then take the image \(\mathbf{I}_0\), and repeat this process, this time rotating by angle \(2\times\theta\) to obtain viewpoint \(\mathbf{T}_2\) and image \(\hat{\mathbf{I}}_2\). This whole process is repeated two more times with \(-\theta\) and \(2\times-\theta\) as rotation values to obtain viewpoints \(\mathbf{T}_3\) and \(\mathbf{T}_4\) respectively. Resulting from these steps, we obtain four completed novel views of the scene [\(\hat{\mathbf{I}}_1\), \(\hat{\mathbf{I}}_2\), \(\hat{\mathbf{I}}_3\), \(\hat{\mathbf{I}}_4\)].

\subsubsection{Viewpoint Selection} \label{sec:pcl_rotation}
% Our main method consists of first deprojecting the original RGB-D image into a point cloud. The original image will be referred to as \(\mathbf{I}_0\). We define a sphere with center


For diffusion-based inpainting, ``known" pixels, i.e., the non-masked areas, guide the prediction of the unknown masked areas. We refer to the known pixels as context pixels and define the context ratio \(C\) for any given image as \(C = (\# context~pixels)\: /\: (\# all\: pixels)\). This ratio gives us some indication about how accurately the inpainting model will be able to fill in the missing areas. With a low \(C\), many areas are unknown and inpainting will struggle, and with a high \(C\) inpainting will do well but only fill in minimal information. An example of different context ratio values can be seen in Figure~\ref{fig:figure-3}.

We then design our viewpoint selection process to search for a context ratio that will allow for accurate inpainting. To do this, we define a sphere with a center as the mean of the input point cloud, and the radius as the distance between the center and the initial camera location. Then from the starting viewing angle, we rotate in various directions along this sphere away from the starting position. At each step in this rotation, we project the input point cloud onto the new camera location. Using this newly projected image, we compute the context \(C\) of the image. If the \(C\) is closest to our chosen context threshold \(C^*\), we use that viewpoint $\mathbf{T}_i$ as next to inpaint. We repeat this process for \(V\) evenly spaced directions we traverse along the sphere as visualized in Figure~\ref{fig:figure-3}, where both \(C^*\) and \(V\) are chosen using the experiment described in Section~\ref{sec:implementation}.

% as seen in  Figure~\ref{fig:figure-3} and then complete those views with RGB inpainting and depth completion as described in the sections above.

% \subsubsection{Enforcing Consistency Across Viewpoints}\label{sec:consistency}
% The final step in our method involves combining these generated viewpoints while enforcing consistency across them. One drawback of utilizing \dalle{} for inpainting real objects, is its inconsistent completion of objects as well as the creation of objects that are not originally in the scene. To combat this issue, we filter for consistent predictions across viewpoints. The final prediction is achieved by first deprojecting the RGB-D images \(\hat{\mathbf{I}}_1\) and \(\hat{\mathbf{I}}_2\) from viewpoint \(\mathbf{T}_1\) and \(\mathbf{T}_2\) respectively back into the world frame as point clouds. The intersection of these point clouds are taken, and this intersection of points is added to our final prediction. The same process is taken with images \(\hat{\mathbf{I}}_3\) and \(\hat{\mathbf{I}}_4\) from their viewpoints. With our final prediction being the point cloud from viewpoint \(\mathbf{T}_0\), the intersection between point clouds from \(\mathbf{T}_1\) and \(\mathbf{T}_2\), as well as \(\mathbf{T}_3\) and \(\mathbf{T}_4\). By taking only the points that both views predict are there, we find that our final output point cloud of the completed scene \({S}\) contains more accurate geometry and color.

\subsubsection{Enforcing Consistency Across Viewpoints}\label{sec:consistency}
{The final step in our method involves combining these generated viewpoints while enforcing consistency across them. One drawback of utilizing \dalle{} for inpainting real objects, is its inconsistent completion of objects as well as the hallucination of objects that are not originally in the scene. To combat this issue, we filter for consistent predictions across viewpoints. The final prediction is achieved by first deprojecting the RGB-D images from each viewpoint \(\mathbf{T}_i\) back into the original camera frame as point clouds. We then apply the following \textit{consistency rule} across all the generated points: If a predicted point from one viewpoint has a predicted point within a 1cm radius from at least two other viewpoints we keep that point, otherwise we remove that point from our final prediction. This rule allows us to keep points that only multiple viewpoints predict. We then combine all filtered points to obtain our final output point cloud of the completed scene \({S}\), which contains more accurate geometry and color than without filtering as seen in Table~\ref{tab:ablations} and Figure~\ref{fig:figure-1}.

% Figure environment removed

% Figure environment removed

% \newpage
\section{Experiments}

In this section, we evaluate the performance of \ours{} for single view RGB-D scene reconstruction task. % on and on the out-of-training distribution dataset.
 We also report ablation studies for understanding the dependence of our method on (1) prompt specificity, (2) inpainting model, and (3) consistency filtering.



\textbf{Implementation Details: }
\label{sec:implementation}
The inpainting step of our algorithm is based on OpenAI's \dalle{} API.
% Our inpainting algorithm is done by utilizing OpenAI's API for \dalle{}.
For our implementation of SAM, we choose a spacing value $c$ of 0.01 meters with $m$ = 100 points for generating our rays. For choosing the number of viewpoints/viewpoint directions \(V\) as well as the context ratio \(C^*\) described in our method section, we perform a parameter search using 4 held out validation scenes from the YCB-V test set. We test using 6, 8, 10, and 12 viewpoints as well as a value of 0.3, 0.4, 0.5, 0.6, and 0.7 for our context threshold. We found that 10 views and 0.4 as a context threshold gave us the best accuracy on the validation set. 12 views and 0.4 as a context threshold performed similarly, but in the interest of runtime we use 10 for the final method. %Results from this parameter search can be seen in Table~\ref{tab:grid-search} (see Appendix \cite{kasahara2023rico}). 
The full parameter grid search are provided in our appendix. For our module that enforces consistency between synthesized views, we choose a threshold of 0.01 meters when computing the intersection between points in the viewpoints point clouds.




% \begin{table}[t]
% \begin{adjustbox}{width=\columnwidth,center}
% \centering
% % \toprule
% \begin{tabular}{|c|c|c|c|c|c|}
%   \hline
%    & 0.3 & 0.4 & 0.5 & 0.6 & 0.7 \\
%   \hline
%   6 &  &  &  &  & \\
%   \hline
%   8 &  &  &  &  & \\
%   \hline
%   10 &  &  &  &  & \\
%   \hline
% % \bottomrule
% \end{tabular}
% \end{adjustbox}
%  \caption{Comparison of different parameters for our method on 4 validation scenes of the YCB-V~\cite{xiang2018posecnn} dataset using Chamfer Distances (CD) to indicate better performance.}
%     \label{tab:sota-comparison}
% \end{table}

% \begin{table}[h!]
% \vspace{-2mm}
% % \begin{adjustbox}{width=\columnwidth,center}
% \centering
% \begin{tabular}{l|c c c c c}
% \toprule
% % \diagbox{\(V\)}{\(C^*\)} & 0.3 & 0.4 & 0.5 & 0.6 & 0.7 \\
% \(V\)~$\vert$~\(C^*\) & 0.3 & 0.4 & 0.5 & 0.6 & 0.7 \\
% \midrule
% 6 & 0.218 & 0.242 & 0.255 & 0.244 & 0.238 \\
% 8 & 0.247 & 0.268 & 0.269 & 0.251 & 0.236 \\
% 10 & 0.265 & \textbf{0.277} & 0.272 & 0.251 & 0.235 \\
% \bottomrule
% \end{tabular}
% % \end{adjustbox}
%  \caption{Experiment using different values for context \(C^*\) and number of viewpoints \(V\) for our method on 4 validation scenes of the YCB-V~\cite{xiang2018posecnn} dataset using IoU to indicate better performance.}
%     \label{tab:grid-search}
%     \vspace{-5mm}
% \end{table}

\textbf{Datasets:}
We trained our depth completion model using the YCB-V training dataset \cite{xiang2018posecnn}. For testing, we test on $8$ unseen scenes from the YCB-V test set, and select $5$ RGB-D images from each of the scenes, i.e., we test on a total of $40$ RGB-D images in total. For ground truth point clouds, we deproject the RGB-D frames of the scene and concatenate them together. We also place the ground truth meshes in the scene for the objects and convert those to point clouds before concatenating them as well. Finally, we crop this point cloud around the ground truth meshes with a 10cm buffer as the RGB-D frames may contain floors and walls far away that we are not interested in reconstructing. This creates our final ground truth point cloud covering the majority of the scene with full geometry of the objects in the scene.

To demonstrate our model's capabilities of generalizing to unseen objects and to entirely new datasets, we also compare our method on the HOPE dataset \cite{tyree2022hope}. HOPE test set only contains individual RGB-D images and is unusable for generating full scene point cloud. Instead, we use HOPE training dataset for evaluation, as the train set contains RGB-D video and cluttered tabletop scenes with novel objects. The dataset has 10 scenes, and we again sample 5 frames per scene for a total of 50 RGB-D test images. Ground truth point clouds are obtained similarly to the YCB-V dataset.

% \subsection{Metrics}
% We evaluate our method on standard 3D reconstruction metrics: 

% \textbf{Intersection-over-Union (IoU)}:
% We voxelize the ground truth and predicted point clouds at a fixed resolution and compute the IoU score by dividing the number of voxels that intersect to that of their union. In our experiments, we evaluate all the methods at the same grid resolution of $100^3$ after rescaling the predictions and ground truth to fit into the unit cube. \textbf{Chamfer Distance (CD)}: Chamfer distance is commonly used to measure the similarity between two point sets and is defined as:\vspace{-2mm}
% \begin{equation}
%     CD(X, Y) = \frac{1}{|X|} \sum_{\mathbf{x} \in X} \min_{\mathbf{y} \in Y} ||\mathbf{x} - \mathbf{y}||_2
% \end{equation}
% We separately report $CD(S, S^*)$ and $CD(S^*, S)$, as well as their their sum. $CD(S, S^*)$ measures how close the reconstructed points from $S$ are to the ground truth points $S^*$, whereas $CD(S^*, S)$ computes how well the ground truth shape is covered. \textbf{F-Score}: Following ~\cite{tatarchenko2019single}, we also report F-Score$@1\%$ which is a measure for the percentage of the surface points that were reconstructed correctly.


% \subsection{Comparison with State of the Art}
\textbf{Baselines:}
We compare \ours{} against four baselines: Convolutional Occupancy Networks (CON) ~\cite{peng2020convolutional} is a 3D scene reconstruction method that inputs a sparse point cloud. We use their pre-trained model for \textit{Synthetic Indoor Scene dataset} where similar to our YCB-V and HOPE datasets, they place multiple ShapeNet ~\cite{shapenet2015} objects in indoor scenes. CoReNet ~\cite{popov2020corenet} is a multi-object shape estimator that inputs an RGB image and estimates a mesh. We compare against CoReNet's pre-trained model qualitatively since its predictions lack scale information. ShellNet ~\cite{chavan2022simultaneous} is trained for single object reconstruction. Given a scene depth image and object instance mask, ShellNet produces reconstruction for the object instance. We re-implemented ShellNet's architecture and trained it with Mask R-CNN~\cite{DBLP:journals/corr/HeGDG17} as the segmentation network on YCB-V dataset.
% ShellNet input a mask of the object and the depth map of the scene. We reimplemented ShellNet's architecture and trained it with Mask R-CNN~\cite{DBLP:journals/corr/HeGDG17} as the segmentation network on YCB objects.
Finally, we compare against CenterSnap ~\cite{irshad2022centersnap}, a multi-object point cloud prediction method. CenterSnap inputs an RGB-D image and predicts point clouds for each object in the scene. Similar to CenterSnap's original training, we first train it on YCB-V synthetic dataset \cite{denninger2020blenderproc, hodan2020bop}, then fine-tune it on the YCB-V real training dataset \cite{xiang2018posecnn}. Since CenterSnap and ShellNet only predict the point clouds for objects and not the rest of the scene, for a fair evaluation, we concatenate their outputs with deprojected point cloud from the input RGB-D image.


% \subsection{Results}
% % Figure environment removed

% Figure environment removed

% \begin{table}[t]
% \begin{adjustbox}{width=\columnwidth,center}
% \centering
% \begin{tabular}{l | c c c c c}
% \toprule
%  \textbf{Method} & IoU $\uparrow$ & F-Score $\uparrow$ & CD$(S^*, S)$ $\downarrow$ & CD$(S, S^*)$ $\downarrow$ & CD $\downarrow$ \\
% \toprule
% \multicolumn{6}{c}{YCB-V~\cite{xiang2018posecnn}} \\
% \midrule
%  CON~\cite{peng2020convolutional} & 0.041 & 0.218 & 0.040 & 0.015 & 0.055  \\ 
%  ShellNet~\cite{chavan2022simultaneous} & 0.155 & 0.495 & 0.024 & \textbf{0.009} & 0.033 \\
%  CenterSnap~\cite{irshad2022centersnap} & 0.149 & 0.483 & \textbf{0.023} & 0.013 & 0.036 \\
%  \ours{} (Ours) & \textbf{0.202} & \textbf{0.567} & \textbf{0.023} & \textbf{0.009} & \textbf{0.032} \\
% \midrule
% \multicolumn{6}{c}{HOPE~\cite{lin2021fusion}} \\
% \midrule
% CON~\cite{peng2020convolutional} & 0.051 & 0.224 & 0.078 & 0.019 & 0.097 \\ 
%  ShellNet~\cite{chavan2022simultaneous} & 0.134 & 0.433 & 0.040 & \textbf{0.006} & 0.045  \\
%  CenterSnap~\cite{irshad2022centersnap} & 0.126  & 0.411 & 0.043 & 0.007 & 0.050 \\
%  \ours{} (Ours) & \textbf{0.207} & \textbf{0.55} & \textbf{0.033}  & 0.007 & \textbf{0.040}  \\
% \bottomrule
% \end{tabular}
% \end{adjustbox}
%  \caption{Comparison of methods for the task of 3D scene completion on the YCB-V~\cite{xiang2018posecnn} and HOPE~\cite{lin2021fusion} datasets. Higher numbers for the IoU and F-score metrics, and lower numbers for the Chamfer Distances (CD) indicate better performance.}
%     \label{tab:sota-comparison}
% \end{table}

\begin{table}[t]
\begin{adjustbox}{width=\columnwidth,center}
\centering
\begin{tabular}{l | c c c c c}
\toprule
 \textbf{Method} & IoU $\uparrow$ & F-Score $\uparrow$ & CD$(S^*, S)$ $\downarrow$ & CD$(S, S^*)$ $\downarrow$ & CD $\downarrow$ \\
\toprule
\multicolumn{6}{c}{YCB-V~\cite{xiang2018posecnn}} \\
\midrule
 CON~\cite{peng2020convolutional} & 0.087 & 0.354 & 0.036 & 0.014 & 0.050 \\ 
 ShellNet~\cite{chavan2022simultaneous} & 0.224 & 0.607 & 0.019 & 0.012 & 0.031 \\
 CenterSnap~\cite{irshad2022centersnap} & 0.225 & 0.622 & 0.019 & \textbf{0.009} & \textbf{0.028} \\
 \ours{} (Ours) & \textbf{0.294} & \textbf{0.661} & \textbf{0.018} & 0.010 & \textbf{0.028} \\
\midrule
\multicolumn{6}{c}{HOPE~\cite{lin2021fusion}} \\
\midrule
CON~\cite{peng2020convolutional} & 0.086 & 0.279 & 0.094 & 0.035 & 0.128 \\ 
 ShellNet~\cite{chavan2022simultaneous} & 0.185 & 0.523 & 0.035 & 0.013 & 0.047  \\
 CenterSnap~\cite{irshad2022centersnap} & 0.180  & 0.526 & 0.037 & 0.006 & 0.042 \\
 \ours{} (Ours) & \textbf{0.290} & \textbf{0.649} & \textbf{0.031}  & \textbf{0.005} & \textbf{0.036}  \\
\bottomrule
\end{tabular}
\end{adjustbox}
 \caption{Comparison of methods for the task of 3D scene completion on YCB-V~\cite{xiang2018posecnn} and HOPE~\cite{lin2021fusion}. Higher numbers for the IoU and F-score metrics, and lower numbers for the Chamfer Distances (CD) indicate better performance.}
    \label{tab:sota-comparison}
    \vspace{-7mm}
\end{table}

% % Figure environment removed

% % \subsection{Results}
% % Figure environment removed

% \subsubsection{Qualitative Results}
% Qualitative results
Table~\ref{tab:sota-comparison} shows quantitative evaluations on within-training-distribution YCB-V dataset \cite{xiang2018posecnn} and out-of-training-distribution HOPE dataset \cite{tyree2022hope}. On YCB-V dataset, \ours{} is able to outperform CON and ShellNet on all 3D scene reconstruction metrics. CON takes as input a sparse point-cloud of the scene. When major parts of the input point clouds are missing, as the common case for single-view RGB-D point clouds, CON fails to infer those regions. ShellNet is trained to predict back-side depth image for the detected object. We notice that with varying viewing directions, ShellNet backside depths are either too thin or too thick resulting in low performance. MaskRCNN's failure to detect objects also directly contributed to lower performance for ShellNet. CenterSnap inputs RGB-D image and predicts object shapes via a multi-step procedure allowing CenterSnap to learn strong shape and pose priors for objects within training distribution. This allowed CenterSnap to perform strongly on YCB-V objects as it was trained on them, but we noticed it struggles with cases of occluded objects. \ours{} which is trained without ground truth object pose or shape supervision is able to match or outperform the baselines in all metrics. Fig. \ref{fig:qualitative-baselines} shows a qualitative comparison with baselines.

On the out-of-distribution HOPE dataset, \ours{} is able to outperform all baselines by an even larger margin. This shows that our normal and occlusion boundary-based depth completion method generalizes well to unseen novel scenes. Figure~\ref{fig:qualitative-geometry} shows qualitative results on these datasets. %while % Figure~\ref{fig:qualitative-geometry} shows qualitative results for our method on the HOPE and YCB-V datasets.

% Figure environment removed

As a byproduct, our method also produces novel views of unseen multi-object scenes from a single RGB-D image. Figure~\ref{fig:qualitative-inpainting} shows our method compared to the ground truth. We show that by combining our masking method with \dalle{}'s inpainting capability, realistic novels views can be generated for multiple unseen objects.

% % Figure environment removed



\subsection{Ablation Studies}


\begin{table}[ht!]
\centering
 \begin{tabular}{l | c c c} 
 \hline
 \textbf{Method} & IoU $\uparrow$ & F-Score $\uparrow$ & CD $\downarrow$ \\
 \hline
 \ours{} (S) & 0.262 & 0.613 & 0.038 \\
 \ours{} (G) & 0.261 & 0.613 & 0.038 \\
 \ours{} (Ours) & \textbf{0.290} & \textbf{0.649} & \textbf{0.036}  \\
 \hline
 \end{tabular}
 \caption{Prompt specificity  results  on  HOPE  dataset \cite{tyree2022hope}: \ours{} (S) denotes  our  model  with  scene  specific  prompt, \ours{} (G) uses ``household objects on a table" as the prompt for all scenes, and \ours{} (Ours) uses an image caption generator~\cite{li2022blip}.}
 \label{tab:prompt}
 \vspace{-3mm}
\end{table}

\textbf{Prompt Reliance:} Image diffusion models tend to heavily rely on the input prompt. To test our methods robustness, we performed an experiment using a general prompt (G), ``a photo of household objects on a table", for every scene to see how much performance degrades. We also use a specific prompt (S) where using the ground truth list of objects we list out every object on the table as the prompt. Table~\ref{tab:prompt} shows that our method does not largely depend on the type of prompt. We hypothesize that our view selection method retains enough surrounding context information in the input RGB image required for the inpainting model to inpaint successfully.


% \label{sec:ablation}
% \textbf{Prompt Specificity}: In our method, we use a deep image captioning method that generated a prompt from the original input image for inpainting. To investigate whether the performance is robust to prompt specificity, we also compare our method using a specific and a general prompt across all scenes. For the specific prompt, with a scene containing $n$ objects with object labels $obj_{i}$, we create the prompt as ``a $obj_1$, $obj_{2}$ $\cdots$ $obj_{n-1}$, and $obj_{n}$ on a table''. This list of objects is limited to the first 10 objects for larger scenes. We call this ablation \ours{} (S). For the general prompt, we use ``household objects on a table" for every scene and call this ablation \ours{} (G). Table~\ref{tab:prompt} shows that our method does not largely depend on the type of prompt. While image diffusion models generally heavily depend on the input prompt for creating images, we hypothesize that our solution where we maximally retain the information present in the input RGB image, provides enough surrounding context and hence, does not require a very specific prompt.





% % \textbf{Viewpoint Angles}: Here, we ablate our method for different values for the angle of rotation between viewpoints \(\theta\) (Section~\ref{sec:pcl_rotation}). We experimented with \(\theta\) values of 10, 20, and 30 degrees and chose the best performing \(\theta\) of 20 degrees as our final \(\theta\). Note that even with the larger \(\theta\) of 30 degrees, our method's performance does not change a lot.


% % \begin{table}[h!]
% % \begin{adjustbox}{width=\columnwidth,center}
% % \centering
% %  \begin{tabular}{l | c c c c c} 
% %  \hline
% %  \textbf{Method} & IoU $\uparrow$ & F-Score $\uparrow$ & CD$(S^*, S)$ $\downarrow$ & CD$(S, S^*)$ $\downarrow$ & CD $\downarrow$ \\
% %  \hline
% %  \ours{} (10) & 0.296  & 0.698 & \textbf{0.038} & 0.006 & 0.044  \\
% %  \ours{} (20) & \textbf{0.298} & \textbf{0.702} & 0.039 & 0.004 & \textbf{0.043}  \\
% %   \ours{} (30) & 0.296 & 0.699 & 0.04 & \textbf{0.003}  & \textbf{0.043}  \\
% %  \hline
% %  \end{tabular}
% %   \end{adjustbox}
% %  \caption{Viewpoint angles study on the HOPE~\cite{lin2021fusion} dataset. 10, 20, and 30 denote the angle of rotation \(\theta\) of the point cloud between inpainting steps as described in Section~\ref{sec:pcl_rotation}}
% % \end{table}

% \textbf{Inpainting Model}: For our method we use Dalle-2 for the inpainting portion as we found qualitatively it performed very well at completing objects in 2D. We also test out Stable Diffusion 2's~\cite{Rombach_2022_CVPR} inpainting model as an open-source alternative. We found that while Stable Diffusion 2's reconstructions were not as accurate as seen in Table~\ref{tab:inpainting}, it could still be used as a viable alternative if necessary.


% \begin{table}[h!]
% \begin{adjustbox}{width=\columnwidth,center}
% \centering
%  \begin{tabular}{l | c c c c c} 
%  \hline
%  \textbf{Method} & IoU $\uparrow$ & F-Score $\uparrow$ & CD$(S^*, S)$ $\downarrow$ & CD$(S, S^*)$ $\downarrow$ & CD $\downarrow$ \\
%  \hline
%  \ours{} (SD-2) & 0.265 & 0.620 & 0.033 & \textbf{0.005} & 0.037 \\
%   \ours{} (Ours) & \textbf{0.290} & \textbf{0.649} & \textbf{0.031}  & \textbf{0.005} & \textbf{0.03624}  \\
%  \hline
%  \end{tabular}
%   \end{adjustbox}
%  \caption{Inpainting method study shown on the HOPE~\cite{lin2021fusion} dataset. (Ours) refers to our main method which utilizes OpenAI's Dalle-2 model, and (SD-2) refers to Stable Diffusion 2's inpainting model.}
% \label{tab:inpainting}
% \end{table}

\textbf{Inpainting Model:}
We substitute in Stable Diffusion 2's~\cite{Rombach_2022_CVPR} inpainting model as an open-source alternative to \dalle{} in Table~\ref{tab:ablations}. We find that while accuracy decreases, it is still a viable inpainting substitute for our method.

% Figure environment removed

% Ablation Experiments

\textbf{Consistency Filtering:}
We test our method without applying the consistency filtering step by just combining all predicted viewpoints in Table~\ref{tab:ablations}. This caused a substantial decrease in accuracy as any hallucinated object is kept in.


\begin{table}[t]
\begin{adjustbox}{width=\columnwidth,center}
\centering
 \begin{tabular}{l | c c c c c} 
 \hline
 \textbf{Method} & IoU $\uparrow$ & F-Score $\uparrow$ & CD$(S^*, S)$ $\downarrow$ & CD$(S, S^*)$ $\downarrow$ & CD $\downarrow$ \\
 \hline
 \ours{} (SD-2) & 0.265 & 0.620 & 0.033 & \textbf{0.005} & 0.037 \\
  \ours{} (No Filter) & 0.271 & 0.574 & \textbf{0.017} & 0.030 & 0.047 \\ 
  \ours{} (Ours) & \textbf{0.290} & \textbf{0.649} & 0.031  & \textbf{0.005} & \textbf{0.036}  \\
 \hline
 \end{tabular}
  \end{adjustbox}
 \caption{Result of swapping out various parts of our method shown on the HOPE~\cite{lin2021fusion} dataset. (Ours) utilizes OpenAI's \dalle{} model and our consistency filtering method, (SD-2) uses Stable Diffusion 2's inpainting model, and (No Filter) refers to our method without filtering.}
\label{tab:ablations}
\vspace{-5mm}
\end{table}

% \textbf{Consistency Filtering}: Table~\ref{tab:consistency} shows results for our method without consistency filtering where we simply concatenate all predicted point clouds (Section \ref{sec:consistency}). \ours{} (No Filter) has lower performance compared to \ours{} (Ours) as \dalle{} does not produce consistent inpainting results across different views and often produce objects which are not in the scene. \ours{} with consistency filtering is able to aggregate coherent information from multiple inpainted views.

% \begin{table}[h!]
% \begin{adjustbox}{width=\columnwidth,center}
% \centering
%  \begin{tabular}{l | c c c c c} 
%  \hline
%  \textbf{Method} & IoU $\uparrow$ & F-Score $\uparrow$ & CD$(S^*, S)$ $\downarrow$ & CD$(S, S^*)$ $\downarrow$ & CD $\downarrow$ \\
%  \hline
%  \ours{} (No Filter) & 0.271 & 0.574 & \textbf{0.017} & 0.030 & 0.047 \\ 
%   \ours{} (Ours) & \textbf{0.290} & \textbf{0.649} & 0.031 & \textbf{0.005} & \textbf{0.036}  \\
%  \hline
%  \end{tabular}
%  \end{adjustbox}
%  \caption{Consistency filtering results on the HOPE~\cite{lin2021fusion} dataset. \ours{} (No Filter) refers to reconstructions without any consistency filtering (Section~\ref{sec:consistency}).}
% \label{tab:consistency}
% \end{table}

% % \textbf{Surface-Aware Masking}: Figure~\ref{fig:sam_fig} 



% \newpage
\section{Conclusion}
\section{Discussion}
\label{disc}
The results have shown that it is possible to improve the results of an existing
fuzzer by applying a generative deep learning model and a reinforcement learning
model to the fuzzer.

% Figure environment removed

The distribution of chosen actions by the DDQN agents gives interesting insights
into the performance of the model and is highlighted in \Cref{disc:distribution}.
For instance, for the $\mathcal{C}2$ agents, it seems like the low performing
models got stuck with a policy that predicts simple render HTML tags over and
over again whereas the well performing models all have the "input" tag in their
top ten of actions taken. In contrast to that the $\mathcal{C}4$ agents achieve
a similar performance with a more balanced tag distribution. Nonetheless, the low
performing models also show an unbalanced policy with a tendency to 'render only'
tags, like 'a'. Furthermore, computing the 'Kullback Leibler Divergence'\cite{kullback1951information} between the
distributions shows that the well performing action policies have a smaller distance
to each other than to the bad performing ones. For example, the distance between
the training runs six and seven of the $\mathcal{C}2$ models have a distance of
$\approx 2.3211$, whereas the distance between training run six and eight is 
$\approx 9.9995$. The training runs six, seven and eight had a total code coverage
performance of $55,594$, $54,874$ and $36,678$ basic blocks respectively.
The difference in policies is also highlighted by \Cref{disc:distribution} where
the policies of runs two to seven build a similarity block and policy eight is
clearly separated from that block. 

The rewards returned by the VMs indicate a high instability in the training process.
They potentially vary significantly in a few training steps. This effect explains 
why a smaller learning rate worked better during all training runs,
especially with growing model complexity. The initial data collection for the DDQN
agent also indicated that it is beneficial to reuse the TCN embedding weights in
the DDQN agent and disable training of the embedding.



% \begin{enumerate}
%     \item improved results
%     \item check distribution of tags
%     \item high fluctuation in vm rewards might be linked to why a small learning rate works better
% \end{enumerate}

\bibliography{references}
\bibliographystyle{unsrt}




\end{document}
