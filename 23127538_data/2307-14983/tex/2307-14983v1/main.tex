\documentclass[12pt]{article}
\usepackage{authblk}
\usepackage[english]{babel}
\usepackage{amssymb}
\usepackage{amsmath}
\allowdisplaybreaks
\usepackage{amsbsy}
\usepackage{epsfig}
\usepackage{amsthm} 
\usepackage{bbold}         
\usepackage{tikz,tikz-feynman}
\usepackage{textcomp, gensymb}
\usepackage{stackrel}
\usepackage[font=small,labelfont=bf]{caption}
\usepackage{cancel}
\usepackage{xcolor}
\usepackage{cite}
\usepackage{hyperref}
\usetikzlibrary{positioning}
\usetikzlibrary{arrows}
\usetikzlibrary{calc}
\tikzset{shifted path/.style args={from #1 to #2 by #3}{insert path={
let \p1=($(#1.east)-(#1.center)$),
\p2=($(#2.east)-(#2.center)$),\p3=($(#1.center)-(#2.center)$),
\n1={veclen(\x1,\y1)},\n2={veclen(\x2,\y2)},\n3={atan2(\y3,\x3)} in
(#1.{\n3+180+asin(#3/\n1)}) to (#2.{\n3-asin(#3/\n2)})
}}}

\usepackage[capitalise]{cleveref}
% Änderung E

\crefname{section}{sec.}{secs.}
\crefname{table}{tab.}{tabs.}
\crefname{figure}{fig.}{figs.}
\crefname{equation}{eq.}{eqs.}
\crefname{appendix}{Appendix\ }{Appendix\ }


\textwidth 16.5cm
\oddsidemargin -0.5cm
\evensidemargin -0.5cm
\textheight 22cm
\topmargin -1cm
\definecolor{darkgreen}{rgb}{0,0.5,0}
\definecolor{darkred}{rgb}{0.7,0,0}
%% ****** Start of file apstemplate.tex ****** %
%%
%%
%%   This file is part of the APS files in the REVTeX 4 distribution.
%%   Version 4.1r of REVTeX, August 2010
%%
%%
%%   Copyright (c) 2001, 2009, 2010 The American Physical Society.
%%
%%   See the REVTeX 4 README file for restrictions and more information.
%%
%
% This is a template for producing manuscripts for use with REVTEX 4.0
% Copy this file to another name and then work on that file.
% That way, you always have this original template file to use.
%
% Group addresses by affiliation; use superscriptaddress for long
% author lists, or if there are many overlapping affiliations.
% For Phys. Rev. appearance, change preprint to twocolumn.
% Choose pra, prb, prc, prd, pre, prl, prstab, prstper, or rmp for journal
%  Add 'draft' option to mark overfull boxes with black boxes
%  Add 'showpacs' option to make PACS codes appear
%  Add 'showkeys' option to make keywords appear
%\documentclass[aps,print,superscriptaddress]{revtex4-1}
%\documentclass[aps,prc,reprint,superscriptaddress,nofootinbib]{revtex4-1}
\documentclass[aps,prc,superscriptaddress,reprint,nofootinbib]{revtex4-1}

%\documentclass[aps,prc,preprint,superscriptaddress]{revtex4-1}
%\documentclass[aps,prl,reprint,groupedaddress]{revtex4-1}
\usepackage{graphicx}
\usepackage{multirow}
\usepackage{color}
%\usepackage{CJK}

%% The amssymb package provides various useful mathematical symbols
\usepackage{amssymb}
\usepackage{bbm}
\usepackage{amsmath}
\usepackage{color}
\usepackage{lineno}
\usepackage{enumitem}
\usepackage{scalerel,stackengine}
%\usepackage{dcolumn}
\usepackage{comment}
\usepackage[normalem]{ulem}
\usepackage{diagbox}

\def\jl#1{\textcolor{blue}{#1}} % JL comments

% You should use BibTeX and apsrev.bst for references
% Choosing a journal automatically selects the correct APS
% BibTeX style file (bst file), so only uncomment the line
% below if necessary.




\stackMath
\newcommand\reallywidehat[1]{%
\savestack{\tmpbox}{\stretchto{%
  \scaleto{%
    \scalerel*[\widthof{\ensuremath{#1}}]{\kern-.6pt\bigwedge\kern-.6pt}%
    {\rule[-\textheight/2]{1ex}{\textheight}}%WIDTH-LIMITED BIG WEDGE
  }{\textheight}%
}{0.5ex}}%
\stackon[1pt]{#1}{\tmpbox}%
}

\bibliographystyle{apsrev4-1}


\begin{document}
%\begin{CJK*}{GB}{song}
% Use the \preprint command to place your local institutional report
% number in the upper righthand corner of the title page in preprint mode.
% Multiple \preprint commands are allowed.
% Use the 'preprintnumbers' class option to override journal defaults
% to display numbers if necessary
%\preprint{}

%Title of paper
\title{Testing the validity of the surface approximation for reactions induced by weekly bound nuclei with a fully quantum-mechanical model}

% repeat the \author .. \affiliation  etc. as needed
% \email, \thanks, \homepage, \altaffiliation all apply to the current
% author. Explanatory text should go in the []'s, actual e-mail
% address or url should go in the {}'s for \email and \homepage.
% Please use the appropriate macro foreach each type of information

% \affiliation command applies to all authors since the last
% \affiliation command. The \affiliation command should follow the
% other information
% \affiliation can be followed by \email, \homepage, \thanks as well.
\author{Junzhe Liu}
%\homepage[]{Your web page}
%\thanks{}

\affiliation{School of Physics Science and Engineering, Tongji University, Shanghai 200092, China.}






\author{Jin Lei}
\email[]{jinl@tongji.edu.cn}
%\homepage[]{Your web page}
%\thanks{}

%\altaffiliation{Present address: Institute of Nuclear and Particle Physics, and Department of Physics and Astronomy, Ohio University, Athens, Ohio 45701, USA}
%\homepage[]{Your web page}
%\thanks{}
\affiliation{School of Physics Science and Engineering, Tongji University, Shanghai 200092, China.}
% \affiliation{Institute for Advanced Study of Tongji University, Shanghai 200092, China}


\author{Zhongzhou Ren}
%\homepage[]{Your web page}
%\thanks{}

\affiliation{School of Physics Science and Engineering, Tongji University, Shanghai 200092, China.}




%Collaboration name if desired (requires use of superscriptaddress
%option in \documentclass). \noaffiliation is required (may also be
%used with the \author command).
%\collaboration can be followed by \email, \homepage, \thanks as well.
%\collaboration{}
%\noaffiliation


\begin{abstract}
We confirm the validity of surface approximation for breakup reactions with a fully quantum-mechanical model proposed by Ichimura, Austern, and Vincent (IAV). Analogous to the semi-classical picture, we introduce radial cut-offs to scattering waves in the IAV framework, which we refer to as IAV-cut. Systematic calculations for $^6$Li and deuteron induced nonelastic breakup reactions at different incident energy are performed, and the comparison between IAV and IAV-cut results are done. The good agreement between IAV and IAV-cut for $^{6}$Li induced reactions indicates the fact that they are insensitive to the interior part of the scattering wave function, thereby validating the semi-classical picture. But for deuteron induced breakup reactions, result of IAV-cut shows a suppression on the cross section, indicating their strong dependence to the interior wave functions.
\end{abstract}



% insert suggested PACS numbers in braces on next line
\pacs{24.10.Eq, 25.70.Mn, 25.45.-z}
% insert suggested keywords - APS authors don't need to do this
%\keywords{}
\date{\today}%
%\maketitle must follow title, authors, abstract, \pacs, and \keywords
\maketitle

%\end{CJK*}

% body of paper here - Use proper section commands
% References should be done using the \cite, \ref, and \label commands



% If in two-column mode, this environment will change to single-column
% format so that long equations can be displayed. Use
% sparingly.
%\begin{widetext}
% put long equation here
%\end{widetext}

% figures should be put into the text as floats.
% Use the graphics or graphicx packages (distributed with LaTeX2e)
% and the \includegraphics macro defined in those packages.
% See the LaTeX Graphics Companion by Michel Gooses, Erastian Raft,
% and Frank Mittelbach for instance.
%
% Here is an example of the general form of a figure:
% Fill in the caption in the braces of the \caption{} command. Put the label
% that you will use with \ref{} command in the braces of the \label{} command.
% Use the figure* environment if the figure should span across the
% entire page. There is no need to do explicit centering.

% % Figure environment removed

% Surround figure environment with turnpage environment for landscape
% figure
% \begin{turnpage}
% % Figure environment removed
% \end{turnpage}

% tables should appear as floats within the text
%
% Here is an example of the general form of a table:
% Fill in the caption in the braces of the \caption{} command. Put the label
% that you will use with \ref{} command in the braces of the \label{} command.
% Insert the column specifiers (l, r, c, d, etc.) in the empty braces of the
% \begin{tabular}{} command.
% The ruledtabular enviroment adds doubled rules to table and sets a
% reasonable default table settings.
% Use the table* environment to get a full-width table in two-column
% Add \usepackage{longtable} and the longtable (or longtable*}
% environment for nicely formatted long tables. Or use the the [H]
% placement option to break a long table (with less control than
% in longtable).
% \begin{table}%[H] add [H] placement to break table across pages
% \caption{\label{}}
% \begin{ruledtabular}
% \begin{tabular}{}
% Lines of table here ending with \\
% \end{tabular}
% \end{ruledtabular}
% \end{table}

% Surround table environment with turnpage environment for landscape
% table
% \begin{turnpage}
% \begin{table}
% \caption{\label{}}
% \begin{ruledtabular}
% \begin{tabular}{}
% \end{tabular}
% \end{ruledtabular}
% \end{table}
% \end{turnpage}

\section{Introduction \label{sec:intro}}
The breakup of a nucleus into two or more fragments
is an important mechanism among various channels of nuclear reactions. With the recent advancements in radioactive beam facilities, measuring the breakup reactions of rare atomic nuclei is now feasible~\cite{abel2019isotope}. This development has greatly enhanced our understanding of nuclear properties such as binding energy, spectroscopic factors, and angular momentum~\cite{PhysRevLett.103.262501}.
Presently, coupled-channel methods which can deal with the excitation of the internal freedom has been widely used to calculate the cross section of rare nuclei induced reactions~\cite{HAGINO2022103951}.

In some experiments, weakly bound  nuclei are produced to bombard with a target, ultimately fragmenting into two separate components. 
From an experimental perspective, determining all particles simultaneously and specifying the  final states of each fragment are challenging. 
Alternatively, if the experiment is designed to detect only one of the fragments inclusively, the process can be simplified to $a + A \rightarrow b + B^*$, where the projectile $a$ is assumed to have a two-body structure $(b+x)$ and $B^*$ represents any possible state of the $x + A$ system. This process of inclusive breakup has been extensively studied in experimental research~\cite{Duan22,Wang21,yang21,DiPietro19,Duan20}.
If the three particles, $b$, $x$, and $A$, remain in their ground state after the breakup, the corresponding process is referred to as elastic breakup (EBU). Breakup accompanied by target excitation, fusion between $x$ and $A$, and any possible mass rearrangement between $x$ and $A$ is referred to as nonelastic breakup (NEB). Precise calculations for NEB are necessary, for example, to exam the semi-classical approach which has been widely applied to knockout reaction~\cite{AUMANN2021103847}, and in the surrogate method applied to study nuclear synthesis and the chemical evolution of stars~\cite{RevModPhys.84.353}. Therefore, the evaluation of NEB cross sections is of great value both theoretically and experimentally.

In 1985, Hussein and McVoy (HM) derived one of the earliest closed-form formulae for the inclusive breakup cross section~\cite{HUSSEIN1985124}. HM's derivation provided deep insight through the summation over all $x$-$A$ states. By utilizing the Glauber approximation to analyze scattering waves, they obtained an appealing and intuitive form with a clear probability interpretation of the breakup reaction. This evaluation of the NEB cross section is exclusively dependent on the asymptotic properties ($S$-matrix) between the fragments ($b$ or $x$) and the target. This is a consequence of employing the semi-classical Glauber approximation. The HM model, along with structure calculations, finds extensive application in spectroscopic studies of one-nucleon removal reactions~\cite{TOSTEVIN2001320,PhysRevC.90.057602,AUMANN2021103847}.
Furthermore, D. Baye and colleagues developed the dynamical eikonal model to address dissociation cross sections~\cite{PhysRevLett.95.082502}. Rather than employing the adiabatic approximation used in the standard eikonal model for phase shift evaluation, they numerically solve a semi-classical time-dependent Schrödinger equation using straight-line trajectories. This model has found application in investigating reactions involving halo nuclei~\cite{PhysRevC.81.024606,PhysRevC.70.064605}.

The transfer to continuum (TC) model is another successful semi-classical approach used to evaluate the NEB cross section~\cite{PhysRevC.38.1776}. In the TC model, the transfer amplitude between the initial and final states is calculated using a time-dependent approach~\cite{PhysRevC.63.044604}. This transfer amplitude is calculated by using the asymptotic part of the initial bound state and the final continuum state. The main principle of this semi-classical approximation is to utilize the classical trajectory for approximating the relative motion between the projectile and the target. This semi-classical TC method has been widely applied to large numbers of break up reactions, from stable to exotic projectiles~\cite{PhysRevC.44.1559,BONACCORSO20181}.

In spite of the tremendous success attained by the previously mentioned semi-classical models, research has already been conducted to establish a quantum-mechanical model. In the early 1980s, Udagawa and Tamura (UT)~\cite{PhysRevC.24.1348}  developed their NEB formalism using DWBA, while Austern and Vincent (AV)~\cite{PhysRevC.23.1847} carried out a similar derivation. After a long-standing dispute between these two groups, the equivalence of these two derivations has finally been proved in Ichimura, Austern, and Vincent (IAV)'s work~\cite{PhysRevC.32.431}. Due to the computational limitations, this model is not implemented numerically until recently~\cite{PhysRevC.92.044616,Potel15,Carlson2016}, and its validity has finally been tested through numbers of applications~\cite{PhysRevLett.122.042503,PhysRevLett.123.232501}.
This fully quantum-mechanical model starts from the effective three-body Hamiltonian, making no assumptions on the trajectory, and maintain the conservation laws naturally.
The EBU is a process that all fragments and targets are properly separated, which allows us to assume that the cross section of this process depends solely on the asymptotic part of wave functions. Therefore, the EBU cross sections are unaffected by the interior part of wave functions~\cite{baur1986breakup}. 
Nevertheless, this conclusion may not apply to NEB, because NEB contains the fusion channel between the fragments and the target, and the calculation requires a short-range imaginary part of the optical potential to describe this fusion process.
However, HM model with Glauber approximation ignores the inner part of the scattering wave function due to the semi-classical approximation, which lacks a direct comparison to fully quantum-mechanical approaches. Here we present a study on this surface approximation\footnote{In this study, we refer to surface approximation as a procedure that relies on the asymptotic properties of the scattering wave function.}, by introducing a radial cut-off to the scattering functions, where no other semi-classical assumptions need to be taken. We refer to this cut-off method in the IAV framework as IAV-cut. By varying the cut-off radius, we  study the sensitivity to the inner wave functions, and thus test the validity of these semi-classical interpretations of reaction processes. 

The paper is organized as follows. In Sec.~\ref{sec:theory} we review the formalism of the IAV model and raise our surface approximation through the radial cut-off. In Sec.~\ref{sec:app} we apply this cut-off to several inclusive reactions induced by $^{6}$Li and deuterons. Finally, in Sec.~\ref{con} we summarize the main results of this work and outline some future developments. 

\section{\label{sec:theory}Theoretical framework}
In this section, we briefly review the IAV model~\cite{AUSTERN1987125,PhysRevC.32.431} and define our corresponding surface approximation in the IAV model.

The inclusive breakup reaction under study takes the form
\begin{equation}
    a(=b+x)+A \rightarrow b+ B^{*},
    \label{eq1}
\end{equation}
where the projectile $a$ has a two body structure $(b+x)$, $b$ is the detected particle, and $B^{*}$ denotes any possible final state of the $x+A$ system.
In  IAV model, fragment $b$ is called the spectator, and fragment $x$ is called the participant.

The IAV model gives the NEB cross section
\begin{equation}
    \left.\frac{\mathrm{d}^2\sigma}{\mathrm{d}\Omega_b \mathrm{d}E_b}\right|_{\textbf{post}}^{\textbf{NEB}}
    =
    -\frac{2}{\hbar v_a}\rho_b(E_b) 
    \langle 
    \psi_x (\boldsymbol{k_b})|
    W_x 
    |\psi_x (\boldsymbol{k_b})
    \rangle,
    \label{eq:IAV}
\end{equation}
where 
$v_a$ is the projectile-target relative velocity, 
$\rho_b(E_b)=\mu_b k_b/[(2\pi)^3\hbar^2]$ is the density of states for particle $b$, $\mu_b$ and $k_b$ are the reduced mass and wave number, respectively, 
$W_x$ is the imaginary part of $U_{x}$ which describes $x$+$A$ elastic scattering, 
$\psi_x$ is the so-called $x$-channel wave function which is obtained by solving the inhomogeneous differential equation
\begin{equation}
    (E_x-K_x-U_{x})\psi_x(\boldsymbol{k_b},\boldsymbol{r_x})
    =
    \langle \boldsymbol{r_x}
    \chi_b^{(-)}(\boldsymbol{k_b})
    |V_{\mathrm{post}}|
    \chi_a^{(+)}\phi_a\rangle,
    \label{eq:inhomo}
\end{equation}
where $E_x=E-E_b$, 
$K_x$ is the kinetic energy operator for relative motion between fragment $x$ and target $A$,
$\chi_b^{(-)}$ is the scattering wave function with incoming boundary condition describing the scattering of $b$ in the final channel with respect to the $x$+$A$ subsystem, $V_{\mathrm{post}}=V_{bx}+U_{bA}-U_{bB}$ is the post form transition operator, where $V_{bx}$ is the potential binding two clusters $b$ and $x$ in the initial composite nucleus $a$, $U_{bA}$ is the fragment-target optical potential, $U_{bB}$ is the
optical potential in the final channel,
$\chi_a^{(+)}$ is the distorted-wave describing the $a$+$A$ elastic scattering with an outgoing boundary condition, and $\phi_a$ is the initial ground-state of the projectile $a$.
To simplify the calculations, we
ignore intrinsic spins. 
As for the angle integrated NEB cross section, we have the partial wave expansion form of Eq.~(\ref{eq:IAV}),
\begin{equation}
\begin{aligned}
    \left. \frac{\mathrm{d}\sigma}{\mathrm{d}E_b}\right|_{\textbf{post}}^{\textbf{NEB}}
    =
    &-\frac{1}{2\pi\hbar v_a}\rho_b(E_b) 
    \frac{1}{2l_{bx}+1}\\
    &\times
   \sum_{l_al_bl_x} \int \mathrm{d}r_x r_x^2|\mathcal{R}_{l_al_bl_x}(r_x)|^2 W_x(r_x), 
\end{aligned}
\label{eq:expansion}
\end{equation}
where $\mathcal{R}_{l_a l_b l_x}$ represents the radial part in the partial wave expansion of $\psi_x$. The variables $l_a$, $l_b$, $l_{bx}$ and $l_x$ represent the relative angular momenta between $a$ and $A$, $b$ and $B^*$, $b$ and $x$, and $x$ and $A$, respectively. Specifically, $l_{bx}$ is determined by the initial bound state of the projectile, while the maximum values of $l_a$, $l_b$, and $l_x$ are selected to ensure the convergence of the cross section.
More details of the IAV model can be found in Ref.~\cite{PhysRevC.92.044616} and its Appendix.


When the Coulomb interaction is taken into consideration, the incoming and outgoing distorted wave have the partial wave expansions,
\begin{equation}
    \langle r_a l_a m_a |\chi_a^{(+)} (\boldsymbol{k_a})\rangle
    =
    \frac{4\pi}{k_a r_a} i^{l_a}e^{i\sigma_{l_a}}
    u_{l_a}(r_a)
    \left [Y_{l_a}^{m_a}(\hat{k_a})\right]^{*},
\end{equation}
\begin{equation}
    \langle \chi_b^{(-)}(\boldsymbol{k_b}) |r_b l_b m_b \rangle
    =
    \frac{4\pi}{k_b r_b} i^{-l_b}e^{i\sigma_{l_b}}
    u_{l_b}(r_b)
    Y_{l_b}^{m_b}(\hat{k_b}),
\end{equation}
where $k_a$ and $k_b$ are the relative wave numbers of the incident and outgoing channel, $\sigma_{l_a}$ and $\sigma_{l_b}$ are the Coulomb phase shift. One important numerical task is to determine the radial wave function $u_{l_a}$($u_{l_b}$), and our surface approximation method focus on the cut-off of these wave functions. In particular, we set the radial wave function $u_{l_a}$($u_{l_b}$) to zero below a specific cut-off radius. 

The surface approximation in nuclear reactions often means using the asymptotic behavior of the wave function to calculate the cross section~\cite{PhysRevC.8.1084,kasano1982new}. 
Instead of solving the radial Schrödinger equations, others use eikonal method to calculate the phase shift ~\cite{HUSSEIN1985124}.  According to the unitarity of $S$-matrix, the cross section of NEB, which represents the absorption of participant $x$ by target $A$, can be expressed using $S$-matrices~\cite{pampus1978inclusive}.  The key point of this kind of surface approximation is to extract the reaction information from the asymptotic behavior ($S$-matrix). In another word, only the exterior part of the scattering wave function influences the cross sections. 
However, in the IAV model, the key step is to solve the differential inhomogeneous equation to obtain the $x$-channel wave function. This wave function is subsequently used for computing the NEB cross section.
As suggested by Baur~\cite{baur1986breakup}, one suitable surface approximation for the IAV model may be 
replacing the wave function with some suitable form. The simplest one is the asymptotic form
\begin{equation}
    u_l(r)
    \approx
    \frac{i}{2}\left [ H^{(-)}_l(r)-S_lH^{(+)}_l(r) \right ].
\end{equation}
However, due to the irregularity of this asymptotic form at the origin, it can not be implemented numerically.
In order to prevent the divergence at the origin and maintain the boundary condition of wave function, we introduce a radial cut-off to the scattering wave functions both in the entrance and the exit channel consistently, which is mentioned by Baur~\cite{baur1986breakup} as well,

\begin{equation}
    u_{l_a}(r)=u_{l_b}(r)=0   \qquad r<R_{\mathrm{cut}},
\end{equation}
where $R_{\mathrm{cut}}$ is the cut-off radius which is chosen according to the interaction radius of optical potential.
It is important to note that implementing this cut-off will result in a discontinuity in the wave functions $u_{l_a}(r)$ and $u_{l_b}(r)$ at the cut-off radius.
In the subsequent discussion, we will refer to the calculation using this cut-off method as IAV-cut.
 
The angular integrated NEB cross section can be directly obtained by using the radial component of the $x-$channel wave function, $\mathcal{R}_{l_a l_b l_x}$. Therefore, it is crucial to investigate the impact of the cut-off, especially when summing over $l_b$ and $l_x$ and only retaining the dependence on $l_a$, which represents the angular momentum between the projectile and target. We denote this new radial part wave function as $R_{l_a}(r_x)$, and its modulus square has the relation to $\mathcal{R}_{l_al_bl_x}(r_x)$
\begin{equation}
\label{eq:rla}
    |R_{l_a}(r_x)|^2 = \sum_{l_b l_x}|\mathcal{R}_{l_al_bl_x}(r_x)|^2.
\end{equation}
Then the angular integrated NEB cross section can be obtained by 
\begin{equation}
\begin{aligned}
    \left. \frac{\mathrm{d}\sigma}{\mathrm{d}E_b}\right|_{\textbf{post}}^{\textbf{NEB}}
    =
    &-\frac{1}{2\pi\hbar v_a}\rho_b(E_b) 
    \frac{1}{2l_{bx}+1}\\
    &\times
   \sum_{l_a} \int \mathrm{d}r_x r_x^2|R_{l_a}(r_x)|^2 W_x(r_x), 
\end{aligned}
\label{eq:expansion1}
\end{equation}


\section{\label{sec:app}Application}

In this section, we present systematic calculations for the inclusive breakup induced by $^6\mathrm{Li}$ and deuteron projectiles and compare the results of IAV with IAV-cut.
The choice of cut-off radii is critical for our implementation. A cut-off radius that is too small will have little impact due to the removal of only a small part of the wave function. However, a cut-off radius that is too large will obscure the interacting details between the two nuclei, leading to a significant decrease in the cross section.
In the global optical potential model~\cite{COOK1982153,PhysRevC.9.2010,PhysRevC.74.044615,koning2003local}, the radius parameter that we used in the current study takes the form
\begin{equation}
    R_0=r_{0}\times A_T^{1/3},
\end{equation}
where $r_0$ is the geometric parameter of optical potential, $A_T$ is the mass number of the target. 
The parameter $R_0$ represents the effective range of the nuclear force. The mass number of the projectile is often omitted in the fitting of the global optical potential. Consequently, we select the cut-off radius to be of a similar order of magnitude as $R_0$. This selection of the cut-off radius accounts for the variability in the effective range of interaction across different target nuclei.


\subsection{Convergence of the numerical method}
As previously mentioned, introducing a radial cut-off for the wave function leads to a discontinuity at the cut-off radius $R_{\mathrm{cut}}$. In numerical calculations, the Gaussian quadrature method is widely used to save computing time by integrating the wave function. However, accurately capturing this discontinuity at the cut-off radius $R_{\mathrm{cut}}$ often requires additional integration quadrature points. Consequently, using an excessive number of grid points in the integration significantly increases computer memory usage and computation time. Moreover, the Gaussian quadrature method is characterized by having more grid points at the upper and lower limits of the integration interval compared to the equally spaced grid points of Simpson's rule and the trapezoidal rule. 
As a result of implementing a radial cut-off for the wave function, somes quadrature points near the origin in the Gaussian method do not contribute to the overall integration result since the integrand becomes zero due to the cut-off.
This leads to the waste of our computational resources. To improve the numerical efficiency, when evaluating the source term, which is the inhomogeneous term on the right-hand side of Eq.~(\ref{eq:inhomo}), we choose the integral\footnote{More details of the evaluation of this source term can be seen in the Eq.~(13) of Ref.~\cite{PhysRevC.97.034628} and its appendix.
The variable $r_{bx}$ in the Eq.~(13) of Ref.~\cite{PhysRevC.97.034628} is expressed by the coordinate set $(r_x, r_b)$ as presented in Eq. (A5) in the appendix. The choice of $r_b$ starting from $R_{\mathrm{cut}}$ will break the continuity of wave function in $r_{bx}$ as well, so a test of convergence is necessary. } region to start from $R_{\mathrm{cut}}$, rather than setting the wave functions to zero and carrying out the integral from the origin. By reselecting the integration interval in this way, we no longer calculate the parts that do not contribute to the cross section, thus achieving rapid convergence of the integration result with fewer integration points.

To test the validity of this method, we consider the $^{28}\mathrm{Si}(d,p X)$ reaction at the incident kinetic energy of 30 MeV in the lab frame and relative kinetic energy between $p$ and $^{29}\mathrm{Si}^*$ of 10 MeV in the CM frame. We plotted the differential cross section $\mathrm{d}\sigma/\mathrm{d}E$ in Fig.~\ref{converge} as a function of the number of Gaussian quadrature points. The figure shows that increasing the number of quadrature points from 25 to 50 results in a sharp drop in the cross section, but further increasing the number of points leads to little change in the cross section and good convergence. Our test also demonstrated that to achieve the same convergence using Simpson's rule (shown as the dotted horizontal line), we needed to employ a minimum of 1000 grids, which is significantly greater than the number of quadrature points ($>$100) required for the Gaussian method. The good convergence shown in the figure supports our choice of the integral region. Similar rapid convergence can be achieved in other reaction systems as well.
% Figure environment removed
\subsection{Application to $(^6\mathrm{Li},\alpha X)$}
Since the Glauber approximation is widely used in the heavy ion induced knockout reactions~\cite{doi:10.1146/annurev.nucl.53.041002.110406,TOSTEVIN2001320}, we present studies on these reactions with IAV-cut to test the validity of the surface approximation.
Here we consider the calculations for the $^{208}\mathrm{Pb}(^6\mathrm{Li},\alpha X)$ reaction.
We treat $^6$Li as $\alpha$+$d$ cluster in the following discussion.
The incoming channel optical potential, which describes the $^6\mathrm{Li}$+$^{208}$Pb elastic scattering, is taken from Ref.~\cite{COOK1982153}. Besides, the $\alpha$+$
^{210}$Bi$^{*}$ and $\alpha$+$^{208}\mathrm{Pb}$ interaction are adopted from Ref.~\cite{PhysRevC.9.2010}, and the $d$+$^{208}\mathrm{Pb}$ interaction is taken from Ref.~\cite{PhysRevC.74.044615}.
The potential binding fragments $\alpha$ and $d$ in the initial composite projectile is assumed to take the Woods-Saxon (WS) form with the following parameter set: $a_v=0.7$ fm and $r_v=1.15$ fm. The depth of this WS potential is fitted to reproduce the experimental binding energy of $^6\mathrm{Li}$.

The nominal
Coulomb barrier for this system is around $30.1$ MeV~\cite{PhysRevC.66.041602}.
The model space needed for converged solutions of the
IAV model contains partial waves $l\le90$ in the $^6\mathrm{Li}$+$^{208}\mathrm{Pb}$ and $\alpha$+$^{210}\mathrm{Bi}^{*}$ relative
motion, and $l\le40$ in the $d$+$^{208}\mathrm{Pb}$ channel at $E_{\mathrm{lab}}=100$ MeV.
The model space we chose was large enough to ensure the convergence of NEB cross section. 
For the $^{208}\mathrm{Pb}(^6\mathrm{Li},\alpha X)$ reaction, the radius parameter of the imaginary part of the optical potential between $^6$Li and $^{208}$Pb is $R_0=9.08$ fm~\cite{COOK1982153}, so we choose the cut-off radius to be 4 fm, 6 fm, and 10 fm according to the previous discussion on the selection of cut-off parameter. It is important to note that a cut-off is applied consistently to both the incoming channel scattering wave function of $^6$Li+$^{208}$Pb and the outgoing channel scattering wave function of $\alpha$+$^{210}$Bi$^{*}$.
The differential cross section of this system at $E_{\mathrm{lab}}=100$ MeV as a function of the  outgoing  energy of $\alpha$ particles in CM frame is presented in Fig.~\ref{dsde}.
The solid line corresponds to results from the IAV model, while the dot-dashed, dashed, and dotted lines represent the cases for IAV-cut with cut-off radii of 4 fm, 6 fm, and 10 fm, respectively. First we notice that the four curves share the same shape, and the peaks are located around the same outgoing energy. 
We observe that for outgoing kinetic energies lower than 50 MeV or higher than 75 MeV, the four curves almost overlap, suggesting that the effect of the cut-off radius is minimal.
However, between 50 MeV and 75 MeV, an increasing difference can be observed as the cut-off radius increases. Nevertheless, even in the worst case (i.e., with a 10 fm cut-off), the difference is still smaller than the typical experimental uncertainty.
Interestingly, we can also see that the difference between the 4 fm cut and 6 fm cut cases is very small and almost invisible in the figure. This indicates that the corresponding part from 4 to 6 fm of the wave function does not affect the cross section significantly.
These results suggest that IAV-cut produces satisfactory outcomes compared to the original IAV calculation. 

% Figure environment removed



% Figure environment removed
 
 
% Figure environment removed

Given the substantial importance of the angular momentum dependency of cross sections, we proceeded to examine the partial wave distribution of the cross sections and the effects of this cut-off method on the cross sections.
The projectile-target angular momentum distribution of integrated NEB cross section for the same reaction is shown in Fig.~\ref{sla100}.
It is observed that the peaks are determined at approximately the same value of $l_a$ which stands for the relative angular momentum between $^{6}\mathrm{Li}$ and ${^{208}\mathrm{Pb}}$, and all curves exhibit the same bell-shaped distribution as described in~\cite{lei2021comparison}.
The cross section shows a strong absorption effect of the $^6$Li+$^{208}$Pb interaction for low partial waves ($l_a\leq20$), thus leading to zero NEB cross section.
Furthermore, the impact of the cut-off is only evident for partial waves within the range $20\leq l_a\leq55$, with higher partial waves remaining unaffected.

Mathematically, wave functions corresponding to large angular momentum states are equal to zero at sufficiently small radii because it is hard to penetrate into the high centrifugal barrier. As a result, setting the inner part of these wave functions to zero will not change the calculation of the cross section.
These results illustrate a consistent agreement between cases using various radial cut-offs and the direct calculation based on the IAV model. This confirmation validates the utilization of the surface approximation in this reaction. In this context, the term surface approximation implies that the NEB cross section remains unaffected by the inner part of the incoming scattering wave of $^6$Li+$^{208}$Pb and the outgoing scattering wave of $\alpha$+$^{210}$Bi$^{*}$.


We pick out the $l_a=$ 20, 48, and 55 cases and draw $|R_{l_a}|^2$ at relative outgoing kinetic energy $E_{\mathrm{\alpha}}=64$ MeV in CM frame in Figs.~\ref{LiR2} (a), (b) and (c), respectively. The partial waves with $l_a=20$, 48, and 55 belong to distinct regions as discussed in Fig. 3: strong absorption where the NEB cross section is close to zero, the maximum value of $\sigma_l$ where the difference between the results obtained by the IAV model and IAV-cut is significant, and the region where the centrifugal barrier plays an influential role. The solid lines represent the IAV results, while the dotted lines denote the IAV-cut results with 10 fm cut-offs. 
First, the difference between wave functions shown in Fig.~\ref{LiR2}(a) does not exhibit a significant difference.
Due to the strong absorption effect of the $^6$Li+$^{208}$Pb interaction, the probability flux for the low angular momentum component is removed to the fusion channel, resulting in a relatively small wave function for the breakup process. Thus, this low partial wave makes a minimal contribution to the NEB cross section. As for the $l_a=48$ case shown in Fig.~\ref{LiR2}(b), 
a small difference occurs in the range of 15 $\sim$ 25 fm and near 5 fm,  while leaving its asymptotic parts unaltered. Since the EBU cross section is evaluated via the boundary conditions ($S$-matrices)~\cite{pampus1978inclusive}, this surface approximation is valid for EBU calculations as well.
Additionally, as shown in Fig.~\ref{LiR2}(c), no apparent difference appears in the wave functions at $l_a=55$. 
This finding is consistent with the results in Fig.~\ref{sla100}, which also does not exhibit clear changes in the high partial wave component.

As discussed in Eq.~(\ref{eq:expansion1}), NEB cross sections can be evaluated by  $|R_{l_a}|^2W_x$, which is the product of the modulus squre of wave function and the imaginary part of the $d$+$^{208}\mathrm{Pb}$ optical potential. The products for the $l_a=20$, 48 and 55 cases are presented in Figs.~\ref{LiR2} (d), (e) and (f), respectively. The solid lines represent the IAV results, while the dotted lines denote the IAV-cut results with 10 fm cut-offs. It can be observed that for all the three partial waves in Figs.~\ref{LiR2} (d), (e) and (f), value of the product goes to zero rapidly in the region $r>15$ fm, owing to the short-range characteristic of the potential, thus making no contribution to the cross section. As a result, any changes on the wave function outside 15 fm will have no impact on the calculation of NEB cross section. The difference shown in panel (d) is significant; however, the magnitude of the quantity depicted in the figure is small, thus resulting in a relatively negligible NEB cross section. Panels (e) and (f) show no apparent difference between the IAV result and the IAV-cut result for the $l_a=$48 and 55 cases. This observation supports the conclusion that a cut-off on the scattering wave functions $u_{l_a}$ and $u_{l_b}$ does not alter the calculation of NEB cross sections.



\subsection{Application to $(d,pX)$}
On the other hand, we study the deuteron induced inclusive breakup reactions to further investigate the validity of the surface approximation.
First, we carry out the calculation for the $^{208}\mathrm{Pb}(d,p X)$ reaction at $E_{\mathrm{lab}}=70$ MeV.
The proton-target and neutron-target interactions
were adopted from the global parametrization of Koning and Delaroche (KD02)~\cite{koning2003local}. The incoming channel interaction between deuteron and the target is adopted from Ref.~\cite{PhysRevC.74.044615}. For the interaction binding the proton and the neutron in the projectile, we considered the Gaussian form
\begin{equation}
    V(r)=V_0\exp(-r^2/a^2),
\end{equation}
where $a=1.484$ fm, and $V_0$ is fitted to reproduce the experimental binding energy of deuteron.
The model space needed for converged solutions contains partial waves $l\le38$ in the $d$+$^{208}\mathrm{Pb}$ and $p$+$^{209}\mathrm{Pb}^*$ relative
motion, and $l\le18$ in the $n$+$^{208}\mathrm{Pb}$ channel at $E_{\mathrm{lab}}=70$ MeV. 

% Figure environment removed

% Figure environment removed


Similar to the $^6$Li induced cases, here we examine the partial wave dependence of the NEB cross section for this reaction. Fig.~\ref{dslx} presents integrated NEB cross section as a function of $d$+$^{208}\mathrm{Pb}$ relative angular momentum. The solid line corresponds to results from the IAV model, while the dot-dashed, dashed, and dotted lines represent the cases for IAV-cut with cut-off radii of 4 fm, 6 fm, and 8 fm, respectively.
The radius parameter of the imaginary part of the optical potential between $d$ and $^{208}$Pb is $R_0=7.87$ fm~\cite{PhysRevC.74.044615}. 
From this figure, we observed that the difference between lines only occurs in the low partial waves ($l\leq20$). When the cut-off radius increases, the variation in the difference also increases. Compared to the $^6$Li induced reactions, this deuteron induced reaction is more sensitive to inner part of the scattering wave, because the suppression on cross section is enhanced gradually when the cut-off radius increases.
Furthermore, in contrast to previous cases of $^6$Li, no strong absorption effect is observed due to the significant contribution of the low partial wave component to the NEB cross section.
These result illustrates that, without a strong absorption effect, the cut-off of the low partial wave component  will finally manifest in the NEB cross section.

To account for this difference compared to the previous $^6$Li induced cases, we also depicted $|R_{l_a}|^2$ which in this case is the modulus square of the radial part of $n$+$^{208}\mathrm{Pb}$ wave function for $l_a=$ 10, 15, and 20 for this reaction at the outgoing kinetic energy $E_p=$ 38 MeV in CM frame. The results are shown in Fig.~\ref{dR2}.
For the $l_a=10$ and $l_a=15$ cases shown in Figs.~\ref{dR2}(a) and (b), there is a noticeable difference in the wave function obtained from the IAV model and the IAV-cut. Specifically, the wave function obtained from IAV-cut within 10 fm shows a significant reduction. This effect is particularly prominent in the $l_a=10$ case.
This demonstrates that, contrary to the previous situation, the inner part of the $n$+$^{208}\mathrm{Pb}$ wave function with low angular momentum is highly sensitive to the interior part of the scattering functions $u_{l_a}$ and $u_{l_b}$. 
However, in the $l_a=20$ case, there is no clear distinction between the results obtained from the IAV model and the IAV-cut. This is due to the presence of a strong centrifugal barrier, which reduces the scattering wave function inside the barrier to almost zero.


The products of the imaginary part of the $n$+$^{208}\mathrm{Pb}$ potential and modulus square of $n$+$^{208}\mathrm{Pb}$ wave function are also presented in Figs.~\ref{dR2} (d), (e) and (f). Similar to the $^6$Li induced cases, the values of the product go to zero rapidly beyond  the effective range of nuclear force. 
It can be observed from Figs.~\ref{dR2} (d) and (e) that the results obtained from the IAV-cut are significantly suppressed compared to those obtained from the IAV model. This suppression leads to a reduction in the cross section for NEB in the IAV-cut.
Consistent to the results in Fig.~\ref{dslx}, no obvious difference can be seen in Fig.~\ref{dR2} (f) for high partial wave components. These results explain the suppression of cross section in Fig.~\ref{dslx}. Surface approximation for this deuteron induced reaction is not as appropriate as the previous $^6$Li induced cases where the strong absorption effect occurs. Nevertheless, the asymptotic behavior of the wave functions remains unchanged, which indicates that this surface approximation is still valid for EBU calculation.

In the previous $^6$Li induced case, where the IAV-cut and IAV models yield almost identical results, the validity of the surface approximation is supported for both the scattering wave functions $u_{l_a}$ and $u_{l_b}$. However, in the case of the deuteron, it is highly important to investigate whether the cross section suppression observed in the IAV-cut results from the cut-off of both wave functions or only one.
We computed the integrated cross section by exclusively applying a cut-off to either the entrance channel scattering wave function $u_{l_a}$ or the exit channel scattering wave function $u_{l_b}$.
The results are presented in Table~\ref{tab}.
The first column represents the cut-off radius, the second column displays the cross section with a cut-off applied exclusively to the incoming channel wave function $u_{l_a}$, while the third column shows the cross section with a cut-off applied exclusively to the exit channel wave function $u_{l_b}$. The fourth column presents the results obtained by consistently applying cut-offs to both the incoming and exit channel wave functions.

\begin{table}[h]
\begin{tabular}{c|c|c|c}
\hline
\diagbox{cut-off}{$\sigma_\mathrm{NEB}$ (mb)}{method}&  cut $u_{l_a}$  & cut $u_{l_b}$     & cut both   \\
\hline
4~fm &  469   &   470   &  467  \\
6~fm &  449   &   441   &  440  \\
8~fm &  413   &   428   &  410  \\
\hline
\end{tabular}
\caption{Integrated cross section of $^{208}\mathrm{Pb}(d,p X)$ reaction at $E_{\mathrm{lab}}=70$ MeV for different cut-off  radius.} 
\label{tab}
\end{table}
The original IAV result of the integrated cross section is 486 mb.
Table~\ref{tab} shows that the cross sections obtained from cutting $u_{l_a}$, $u_{l_b}$, and cutting both of them are of comparable magnitudes for different cut-off radii. This indicates that the cut-off behaves similarly for both $u_{l_a}$ and $u_{l_b}$, suggesting that no specific cut-off exhibits dominance over another. And applying a cut-off to either $u_{l_a}$ or $u_{l_b}$ will ultimately reduce the cross section when compared to the IAV model.
In another word, both scattering wave functions concurrently contribute to the overall influence on the cross section.
This can be further discussed in a theoretical perspective.
Since the wave functions in the entrance and exit channels are represented in two different sets of Jacobi coordinate, calculating the inhomogeneous 
terms in Eq.(\ref{eq:inhomo}) requires coordinate transformation between these two sets. 
When calculating integrals, the two coordinate variables are not completely orthogonal. 
Therefore, when the product of two scattering wave functions from both the incoming and outgoing channels is integrated, any cut-off applied to one wave function will also exclude the region near the origin of the other wave function. Consequently, the excluded region of the other wave function will not contribute to the final determination of the cross section, regardless of whether this part is cut or not.
In summary, the contribution of the internal parts of the incident and outgoing scattering wave functions cannot be separated when calculating the cross section.

\subsection{\label{dis}Discussion}

Based on the previous calculations and a comparison between the IAV and IAV-cut models, we conclude that the surface approximation is valid for $^6$Li-induced breakup reactions but does not yield satisfactory results for deuteron-induced cases.
In order to further investigate the validity of the surface approximation in the IAV framework, we performed systematic calculations of $^6$Li and deuterons induced inclusive breakup reactions considering different incident energies and target masses. 

We carry out the calculations for 
the $^{28}\mathrm{Si}(d,p X)$ reactions at $E_{\mathrm{lab}}=$2, 6, 10, 20, 30, 60, and 100 MeV,
the $^{208}\mathrm{Pb}(d,p X)$ reactions at $E_{\mathrm{lab}}=$ 20, 30, 50, 70, and 100 MeV,  
the $^{28}\mathrm{Si}(^6\mathrm{Li},\alpha X)$ at $E_{\mathrm{lab}}=$5,  20 ,30, 40, 50, and 100 MeV, 
and $^{208}\mathrm{Pb}(^6\mathrm{Li},\alpha X)$ reactions at $E_{\mathrm{lab}}=$30, 40, 60, 80 and 100 MeV. 
The numerical computation of the NEB cross section using the IAV model are heavy tasks. This is primarily due to the slow convergence of wave functions for many partial waves, the large memory required to store these wave functions, and the need for more grid points to ensure the convergence of the numerical integration. As a consequence, our systematic analysis is computationally intensive and has reached our computing limitations. Thus, we were only able to examine a restricted range of incident energies and target masses, and our conclusions are applicable only under these limited circumstances.

Here we introduce the relative deviation of the integrated NEB cross section to quantify the difference between IAV and IAV-cut:
\begin{equation}
    \delta=\frac{|\sigma_{\mathrm{IAV}}-\sigma_{\mathrm{cut}}|}{\sigma_{\mathrm{IAV}}}\times 100\%,
\end{equation}
where $\sigma_{\mathrm{cut}}$ is the integrated NEB cross section calculated with IAV-cut and $\sigma_{\mathrm{IAV}}$ is result computed directly with the IAV model.

Figure~\ref{deltad} shows the relative deviation for the deuteron-induced cases discussed above at different incident energies. Panel (a) and (b) correspond to the $^{28}\mathrm{Si}(d,pX)$ and $^{208}\mathrm{Pb}(d,pX)$ cases, respectively. The cut-off radii are selected based on the optical potential parameters and are included in the figures. We use 2 and 4~fm cut-offs for the $^{28}\mathrm{Si}(d,pX)$ reactions and 4, 6, and 8~fm cut-offs for $^{208}\mathrm{Pb}(d,pX)$ reactions. The line with circle, square, diamond, and plus points represents the cases with 2, 4, 6, and 8~fm cut-offs, respectively.
This figure demonstrates an upward trend as the incident kinetic energy increases. With increasing kinetic energy, the relative deviations become more considerable, ranging from a few percent to as much as 25$\%$ in the most extreme scenario. Moreover, the relative deviations still remain moderate for low incident energy near or below the Coulomb barrier, where the strong Coulomb force prevents the wave function from penetrating the interior region. Therefore, setting these wave functions to zero would not affect the calculation of the NEB cross section. Additionally, the figure shows that there is a decrease in the relative deviations with a 2~fm cut-off in panel (a) and a 4~fm cut-off in panel (b). In other words, surface approximation with these cut-offs is still valid in the NEB calculation.

% Figure environment removed

Figure~\ref{deltali} illustrates the relative deviation for the $^6$Li induced cases mentioned above at different incident energies.
Panel (a) and (b) present the $^{28}\mathrm{Si}(^6\mathrm{Li},\alpha X)$ and $^{208}\mathrm{Pb}(^6\mathrm{Li},\alpha X)$, respectively. 
We use 2 and 4~fm cut-offs for the $^{28}\mathrm{Si}(^6\mathrm{Li},\alpha X)$ reactions and 4, 6, and 10~fm cut-offs for $^{208}\mathrm{Pb}(^6\mathrm{Li},\alpha X)$ reactions. The lines with circle, square, diamond, and star points represent the cases with 2, 4, 6, and 10~fm cut-offs, respectively.
In Fig.~\ref{deltali} (a), we can observe a decreasing trend in the relative deviation. Specifically, as the kinetic energy increases, the projectile follows a classical trajectory, and a strong absorption effect of the $^6$Li+$^{28}$Si interaction occurs, both of which are key assumptions in semi-classical approaches.
Figures~\ref{deltali}(a) and (b) both demonstrate that the overall relative deviations of the cross section are less than 5$\%$, which is within the typical experimental uncertainty, confirming the validity of surface approximation in these systems. The comparison between Fig.~\ref{deltad} and Fig.~\ref{deltali} reveals that, in the considered circumstances, the surface approximation for $(d,p)$ reactions is not as applicable as it is for $(^6\mathrm{Li},\alpha X)$ reactions. This is evidenced by the deviation values in Fig.~\ref{deltad} being 20\% higher than those in Fig.~\ref{deltali}.

 % Figure environment removed


To further investigate the differences caused by $^6$Li and deuteron-induced breakup reactions, we reintroduce the angular dependence of the incoming scattering wave function in order to make a direct comparison with the semi-classical trajectory picture.
The expansion can be written as
\begin{equation}
    \chi_a^{(+)}(\boldsymbol{r})=\sum_l i^l(2l+1)\frac{u_l(r)}{kr}P_l(\cos\theta),
\end{equation}
where the Coulomb phase shift has already been inserted into the radial function $u_l$. 

Figure~\ref{fig:lipolar} depicts the modulus square of the wave function $|\chi_a^{(+)}|^2$ in the x-z plane for the elastic scattering of $^6\mathrm{Li}$+$^{208}\mathrm{Pb}$ reaction at three different bombarding energies. This figure is arranged as a heatmap in polar coordinate, where lighter colors represent larger probability of finding a particle according to the probability interpretation of wave function. At 30 MeV, $^6$Li is incapable of penetrating $^{208}$Pb and instead is diffracted before reaching the target. A clearly defined classical trajectory becomes apparent in Figs.~\ref{fig:lipolar}(b) and (c), because
as the scattering angle decreases, the peak of probability density forms a straight line.
Additionally, Figs.~\ref{fig:lipolar}(b) and (c) demonstrate that the probability of detecting a particle at forward region is exceptionally low because there are no bright points in the forward region. This lack of probability in the forward region demonstrates the strong absorption effect of the $^6$Li+$^{208}$Pb interaction, in which low angular momentum components are fully fused into the target and do not contribute to the NEB cross section.  These results affirms the validity of introducing radial cut-offs at high incident energies in IAV framework.

As a comparison, the wave function for the elastic scattering of $d$+$^{208}$Pb is depicted in Fig.~\ref{fig:dpolar}. Unlike the previous case, there is a strong interference in the forward angle in all the panels of Fig.~\ref{fig:dpolar}, showing strong wave-like characteristics. This strong distortion of the scattering wave functions lack a correspondence to the classical trajectory picture, and thus the surface approximation fails. Besides, there is no evidence of a strong absorption effect of the $d$+$^{208}$Pb interaction, because there are many light points in the forward angle. 

Comparing wave functions of these reactions makes it straightforward to establish the validity of surface approximation based on whether the incoming channel elastic scattering process has a clear correspondence to a classical trajectory.

% Figure environment removed

% Figure environment removed


\section{\label{con}Conclusion}
We present a study on the nonelastic breakup reactions induced by weekly bound nuclei with a fully quantum-mechanical model from Ichimura, Austern, and Vincent. Corresponding to the classical picture of trajectory in reaction processes, we introduce a radial cut-off to investigate the validity of the surface approximation on a fully quantum-mechanical basis. With a proper selection of the cut-off radius, we apply this surface approximation to 
the $(^6\mathrm{Li},\alpha X)$ and $(d,p X)$ reactions. 

We observed that the approximated cross sections computed with cut-offs for the $(^{6}\mathrm{Li},\alpha X)$ reactions exhibit good overall agreement with the accurate calculations. These results indicate that NEB cross section is insensitive to the inner wave function, and the semi-classical picture is valid. For $(d,pX)$ reactions, a non-negligible loss of cross-sections was observed after the cut-off, suggesting a strong dependence on the inner wave functions at low energies in the IAV framework. Setting the inner wave function to zero in $^{6}\mathrm{Li}$ induced reactions has little effect on the cross-section due to a strong absorption of small angular momentum components in the entrance channel. However, in the case of deuteron induced reactions, the distortion caused by the nuclear and the Coulomb forces does not correspond to a semi-classical trajectory picture. Consequently, there is a relatively stronger dependence on the inner scattering wave function in the IAV framework.

However, these conclusions are based on a very limited set of systems. Further studies involving higher incident energies and more targets are called for. 
We plan to optimize our computer code so that we can conduct calculations for reactions involving heavier targets and higher energy, which are currently beyond our computing capability.

\begin{acknowledgments}
This work has been supported by National Natural Science Foundation of China (Grants No.12105204, No.12035011, and No.11975167), by the Fundamental Research Funds for the Central Universities.
\end{acknowledgments}

\bibliography{SA.bib}
\end{document}



%
% ****** End of file apstemplate.tex ******
\tilde{}
\newcommand{\dZphi}{\delta Z_\phi}
\newcommand{\dZpsi}{\delta Z_\psi}
\newcommand{\ee}{\mathrm{e}}
\newcommand{\ii}{\mathrm{i}}
\newcommand{\WP}[1]{{\cyan WP: #1}}
\newcommand{\WPadd}[1]{{\cyan #1}}
\newcommand{\WPout}[1]{\sout{\cyan #1}}
\newcommand{\WPrep}[2]{\sout{\cyan #1} {\cyan #2}}
\newcommand{\AB}[1]{{\color{blue} \textbf{AB:}  #1}}
\newcommand{\AAZ}[1]{{\color{darkgreen} \textbf{AA:} #1}}
\newcommand{\KKn}[1]{{\color{darkred} \textbf{KK:} #1}}
\newcommand{\HH}[1]{{\color{cyan} HH: #1}}

\newcommand{\oD}{\overline{D}}
\newcommand{\oG}{\overline{G}}
\newcommand{\oL}{\overline{\Lambda}}
\newcommand{\op}{\overline{\phi}}
\newcommand{\oPi}{\overline{\Pi}}
\newcommand{\oV}{\overline{V}}

%------------------------------------------------------------
% For "Feynman Diagrams"
\tikzstyle{Gamma}=[circle,draw=black,fill=black,thick,inner sep=0pt,minimum size=6mm]
\tikzstyle{VEV}=[circle,draw=black,thick,inner sep=0pt,minimum size=2mm]
\tikzstyle{Pi}=[rectangle,draw=black,fill=gray,thick,inner sep=0pt,minimum size=6mm]
\tikzstyle{Sigma}=[rectangle,draw=black,fill=cyan,thick,inner sep=0pt,minimum size=6mm]
\tikzstyle{VR}=[rectangle,draw=black,thick,inner sep=0pt,minimum size=6mm]

%-------------------------------------------------------------
%-------------------------------------------------------------
% Authors+Affiliation
\title{\textbf{Renormalized equations of motions 
for scalars and fermions in the 2PI formalism}}

\author{A.~Banik\,\thanks{E-mail: \href{mailto:amitayus.banik@uni-wuerzburg.de}{amitayus.banik@uni-wuerzburg.de}}}
\author{H.~Hinrichsen\thanks{E-mail: \href{mailto:hinrichsen@physik.uni-wuerzburg.de}{hinrichsen@physik.uni-wuerzburg.de}}}
\author{W.~Porod\thanks{E-mail: \href{mailto:porod@physik.uni-wuerzburg.de}{porod@physik.uni-wuerzburg.de}}}
\affil{\textit{Institut f\"{u}r Theoretische Physik und Astrophysik, Universit\"{a}t W\"{u}rzburg, D-97074 W\"{u}rzburg, Germany}}

\date{} 
%----------------------------------------------

\begin{document}

\maketitle
\begin{abstract}
We present on shell-scheme for the 2PI formalism
with a particular focus on the renormalized
equations of motion. We first revisit the
so-called Hartree approximation where
we give the counterterms for both the broken
and unbroken phase. Moreover, we give
explicit formulas for the renormalized
three- and four-point functions in the broken phase.
We then turn to the sunset approximation, with only scalars and then including fermions. We give explicit formulas for the 
wavefunction and mass counterterms. Moreover,
we show that, in particular, the two-point functions
can be obtained numerically in a fast converging
scheme even for large couplings of order one.
\end{abstract}


\section{Introduction}
The vast majority of physical phenomena, which are vital to our understanding of nature, take place out of equilibrium. Non-equilibrium processes can be found in a wide range of physical domains ranging from particle physics and cosmology to astrophysics and condensed matter systems. A few examples are heavy-ion collisions performed, for instance, at the Large Hadron Collider (LHC) with the aim to produce quark-gluon plasma, the generation of density fluctuations during inflation and the explosive particle production at the end of inflation, as well as phase transitions in the early universe or in condensed matter systems. These phenomena also go beyond the reach of standard perturbation theory, meaning one needs to resort to non-peturbative methods for consistent results.

The challenge of addressing models beyond perturbation theory and tackling out-of-equilibrium aspects can be fulfilled using functional integral techniques based on $n$-particle irreducible ($n$PI) effective actions. In this work, we focus on the two-particle irreducible (2PI) formalism \cite{Jackiw:1974cv,Cornwall:1974vz}, where the expectation value of the field (one-point function) and the propagator (two-point function) constitute the dynamical degrees of freedom. 

Resummation schemes based on the 2PI effective action have been known to show better convergence properties in comparison to other methods, such as in bosonic finite temperature field theory \cite{Blaizot:2000fc,Andersen:2004re,Berges:2004hn} and inflationary preheating \cite{Arrizabalaga:2004iw}. Out-of-equilibrium properties can be formally studied, whereby the resummation feature of the 2PI formalism allows one to obtain approximations uniform in time \cite{PhysRevD.37.2878, Ivanov:1998nv, Berges:2004yj, Berges:2015kfa}. Furthermore, approximations within the 2PI formalism have been shown to be consistent with (global) conservation laws stemming from Noether's theorem, thus guaranteeing charge and energy conservation \cite{Arrizabalaga:2005tf,Berges:2010nk}. 
Finally, using the equations of motion of the 2PI formalism
one can obtain, in principle, the transport equations
\cite{Konstandin:2013caa,Jukkala:2019slc}
which are an important ingredient in the study 
of cosmological phase transitions, see for e.g.~\cite{Prokopec:2003pj,Prokopec:2004ic,Konstandin:2013caa}.


By definition, approximation schemes based on the 2PI formalism involve  resummation of the two-point function to all orders in perturbation theory, when one chooses a particular truncation for the expansion, such as restricting to a given loop order. Despite this selective resummation, 2PI methods show renormalizability through local counterterms, as demonstrated in \cite{Blaizot:2003br, Blaizot:2003an, Carrington:2014lba, Carrington:2017lry} for a symmetric scalar theory with a quartic coupling. The procedure to obtain the counterterms for $\text{O}(N)$ symmetric theories has been explored in \cite{Patkos:2008ik, Patkos:2008sg, Pilaftsis:2013xna, Pilaftsis:2015cka, Pilaftsis:2017enx}. Progress has also been with regard to renormalizability with the inclusion of 
abelian gauge bosons and consistency with Ward identities \cite{Berges:2004hn, Reinosa:2006cm, Reinosa:2007vi, Reinosa:2009tc, Oliveira:2022bar}, as well as the inclusion of fermions \cite{Reinosa:2005pj}. A systematic approach
for the renormalization of the 2PI action 
been presented in \cite{Berges:2005hc} focusing
on scalars.

An aspect which has to our knowledge not been
discussed so far is how to obtain an on-shell
renormalization scheme for the 2PI formalism.
Our aim
is to obtain this for a coupled system of fermions
and scalars.
We will do this with a particular focus on
the renormalized equations of motions which 
in turn, can be used as starting point for transport
equations in cosmological phase transitions.
In the present work, we demonstrate the applicability of the techniques in \cite{Berges:2005hc} to various truncations of the 2PI effective action at two-loop order, and extend the techniques to incorporate fermions in the 2PI formalism. Our main aim is to present the renormalized equations of motion, after which we focus on the necessary steps to obtain the various counterterms.

The paper is organized as follows: in Section \ref{sec:eqm}, 
we first recall relevant features of the 2PI formalism, focusing in particular on obtaining the renormalized equations of motion.
It is well-known that these equations contain 
several counterterms which need to be fixed by appropriate renormalization conditions. To this end, 
we present a suitable on-shell scheme in  
 Section~\ref{sec:gen_renorm}, 
which allows for connecting to physical observables in a straightforward fashion. 
We then apply this scheme to various two-loop truncations of the 2PI effective action.
We start in Section \ref{sec:scalars} with the well-known Hartree approximation. We find, in particular, 
that all counterterms are finite
and give explicit 
relations between the broken and unbroken phases for 
these counterterms. 
In the Hartree approximation, one can give 
analytic expressions for all interesting quantities, though
this does not hold in other cases.
Therefore, we proceed to the so-called scalar sunset 
approximation to demonstrate the intricacies and how 
to resolve them. This model also provides a simple 
example containing a so-called memory integral in the 
equation of motion which are important for the 
thermalisation of a system, see e.g.~\cite{Berges:2015kfa}.
Within this framework, we obtain the gap equation to 
obtain the renormalized two-point function and solve it 
using an iterative procedure. We demonstrate that
this procedure converges very fast even for large 
couplings.

In Section \ref{sec:fermions}, we take the fermionic sunset approximation, which serves as the simplest example to include fermions in the 2PI formalism. Here, we obtain coupled system of gap equations for the fermionic and scalar propagators which we again solve using the aforementioned iterative procedure, finding sufficiently fast convergence for an $\mathcal{O}(1)$ Yukawa coupling. We find that the inclusion of fermions renders none of the counterterms finite. Finally, in Section \ref{sec:outlook}, we present our conclusions and give an outlook. 

%----------------------------------------------------------------------------------
\section{The 2PI Formalism and Renormalized Equations of Motion}
\label{sec:eqm}
%----------------------------------------------------------------------------------

In order to fix notation, let us first summarize some key aspects of the 2PI formalism in the 
example of a single scalar field~\cite{Cornwall:1974vz}, with the generating functional as starting point
%
\begin{align}
Z[J_1,J_2] = \int {\cal D}\varphi \exp\left(\ii S[\varphi] + \ii \int_x J_1(x) \varphi(x) + \frac{\ii}{2}\int_{xy}  \varphi(x) J_2(x,y) \varphi(y)  \right)\,,
\end{align}
%
where $x$ and $y$ denote vectors in a $d$-dimensional Minkowski space with the mostly minus signature and $J_1$ and $J_2$ are two external currents suitably shifted in order to account for a Gaussian initial state. Since one is interested in the non-equilibrium evolution, where a final state is not known, the temporal integration in
%
\begin{align}
\int_x \equiv \int_{\mathcal{C}} dx_0\int d^{d-1} x
\end{align}
%
is carried out along a time-ordered Keldysh contour $\mathcal{C}$. In terms of the 
cumulant-generating functional $W[J_1,J_2] = - \ii \ln Z[J_1,J_2]$, the macroscopic field and the connected two-point correlator, defined as $\phi(x)=\langle\varphi(x)\rangle_{\mathcal C}$ and  $G(x,y)=\langle\varphi(x)\varphi(y)\rangle_{\mathcal C}$, are given by
%
\begin{equation}
\phi(x) = \frac{\delta  W[J_1,J_2]}{\delta J_1(x)} \,, \qquad
G(x,y) = 2 \frac{\delta  W[J_1,J_2]}{\delta J_2(x,y)} - \phi(x) \phi(y) \,.
\end{equation}
%
It is more practical to describe the system by the Legendre transform of $W[J_1,J_2]$, the so-called 2PI effective action
%
\begin{align}
\Gamma^{\text{2PI}}[\phi,G] &= W[J_1,J_2] - \int_x  \frac{\delta W[J_1,J_2]}{\delta J_1(x)}J_1(x) - \int_{xy}  \frac{\delta W[J_1,J_2]}{\delta J_2(x,y)}J_2(x,y)\\
&= W[J_1,J_2] - \int_x  \phi(x)J_1(x) - \frac{1}{2} \int_{xy} \Bigl( \phi(x)\phi(y)+ G(x,y) \Bigr)J_2(y,x)
\end{align}
%
which allows the currents to be expressed as
%
\begin{equation}
J_1(x)=-\frac{\delta \Gamma^{\text{2PI}}[\phi,G] }{\delta\phi}-\int_y J_2(x,y) \phi(y)\,,\qquad J_2(x,y) = -2 \frac{\delta \Gamma^{\text{2PI}}[\phi,G]}{\delta G(x,y)}\,.
\end{equation}
%
The  effective action can be split into $ \Gamma^{\text{2PI}}[\phi,G] = \Gamma_1^{\text{2PI}}[\phi,G] + \Gamma_2^{\text{2PI}}[\phi,G]$, where
%
\begin{equation}
\Gamma_1^{\text{2PI}}[\phi,G]=S[\phi] + \frac{\ii}{2}    \tr\left[\ln G^{-1} \right] 
  + \frac{\ii}{2} \tr\left[\tilde G^{-1}_\phi G \right] -  \underbrace{\frac{\ii}{2} \tr\left[G^{-1} G \right] }_{=\textrm{const}}
\end{equation}
%
comprises the classical action and all 1-loop contributions while $\Gamma_2^{\text{2PI}}[\phi,G]$ accounts for higher contributions from 2-loop onward. ``$\tr$'' refers to integration over the space-time variables. Here, 
\begin{align}
\tilde G^{-1}_\phi = -\ii \frac{\delta^2S[\phi]}{\delta\phi(x)\delta\phi(y)}
\end{align} 
denotes the classical inverse propagator and the trace stands for integration along the Keldysh contour. In practical calculations, $\Gamma_2^{\text{2PI}}[\phi,G]$ is only evaluated up to given loop order.

In view of the renormalization procedure, it is useful to split the action into a free and an interacting part $S[\phi] = S_0[\phi] + S_{\text{int}}[\phi]$ and to decompose $\tilde G^{-1}_{\phi}=\tilde G^{-1}_{0}+\tilde G^{-1}_{\phi,\textrm{int}}$ \cite{Berges:2005hc}. Correspondingly one can split the effective action into $\Gamma^{\text{2PI}}=\Gamma_0^{\text{2PI}}+\Gamma_{\textrm{int}}^{\text{2PI}}$, where
%
\begin{align}
\Gamma_0^{\text{2PI}}[\phi,G]&=
S_0[\phi] + \frac{\ii}{2}    \tr\left[\ln G^{-1} \right] + \frac{\ii}{2}\tr[\tilde G^{-1}_{0} G] \,, \\
\Gamma_\textrm{int}^{\text{2PI}}[\phi,G]&=
S_{\text{int}}[\phi] +  \frac{\ii}{2}\tr[\tilde G^{-1}_{\phi,\text{int}} G] + \Gamma_2[\phi,G] + \text{const.}
\end{align}
%
The stationarity conditions determine the physical one- and two-point functions $\overline{\phi}$ and $\overline{G}$ in the absence of external sources. i.e.~$J_1=0$ and $J_2=0$:
%
\begin{align}
\label{eq:extr_Gamma_phi}
\frac{\delta \Gamma^{\text{2PI}}[\phi,G]}{\delta \phi(x)} \bigg|_{\overline{\phi},\overline{G}} &= 
\frac{\delta S_{0}[\phi,G]}{\delta \phi(x)} \bigg|_{\overline{\phi}}+\frac{\delta \Gamma^{\text{2PI}}_{\text{int}}[\phi,G]}{\delta \phi(x)} \bigg|_{\overline{\phi},\overline{G}} = 0 \,, \\ 
\frac{\delta \Gamma^{\text{2PI}}[\phi,G]}{\delta G(x,y)} \bigg|_{\overline{\phi},\overline{G}} &=
- \frac{\ii}{2} \overline{G}^{-1}(x,y) + \frac{\ii}{2} \tilde G^{-1}_0(x,y) +
\frac{\delta \Gamma^{\text{2PI}}_{\text{int}}[\phi,G]}{\delta G(x,y)} \bigg|_{\overline{\phi},\overline{G}} =
0 \,.
\label{eq:extr_Gamma_G}
\end{align}  
We note for completeness that these conditions imply that the sources need to be redefined order-by-order
once a perturbative evaluation is performed, as has already been pointed out in \cite{Jackiw:1974cv}
in the context of usual 1PI effective action.

At this stage, it is useful to introduce the self-energy
%
\begin{align}
\label{eq:def_Pi}
\overline{\Pi}(x,y) &=  2 \ii   \frac{\delta \Gamma^{\text{2PI}}_{\text{int}}[\phi,G]}{\delta G(x,y)}\bigg|_{\overline{\phi},\overline{G}} 
\end{align}
%
which can be decomposed into a local part, proportional to $\delta_{\cal C}(x-y)$, and a non-local one, see below. The index ${\cal C}$ indicates that  the Delta-distribution takes values along the Keldysh contour.

The one- and two-point functions are obtained from the equations of motions (EOMs) once the corresponding boundary conditions are specified. More concretely, the EOM for $\phi$ is given by \cref{eq:extr_Gamma_phi} whereas the one for the two point function can be obtained from \cref{eq:extr_Gamma_G}  by convoluting it with  $G(y,z)$ yielding
\begin{align}
\label{eq:eom_G_generic}
(\square_x + m^2) G(x,z) + \int_y  \overline{\Pi}(x,y) G(y,z) &= \delta_{\cal C}(x-z) \,.
\end{align}

Fermions can be treated in an analogous way \cite{Cornwall:1974vz,Berges:2004yj}. 
We will denote the Dirac field by $\psi(x)$ and the 
corresponding two-point function by $D(x,y)$. With 
the assumption that the fermionic field does not 
acquire a vacuum expectation value (VEV), 
we obtain the generic equation of motion for $D(x)$ as
\begin{align}
\label{eq:eom_D_generic}
(\ii \myslash{\partial} - M) D(x,z) + \int_y  \overline{\Sigma}(x,y) D(y,z) &= \delta_{\cal C}(x-z) \,,
\end{align}
where $\overline{\Sigma}(x,y)$ is the fermionic self-energy given by
%
\begin{align}
\label{eq:def_Sigma}
\overline{\Sigma}(x,y) &=  \ii   \frac{\delta \Gamma^{\text{2PI}}_{\text{int}}[\phi,G,D]}{\delta  D(x,y)}\bigg|_{\overline{\phi},\overline{G},\overline{D}} \,. 
\end{align}
Later, we will renormalize the fields based on a procedure that we will outline, retaining the assumption that $\psi$ does not get a VEV.

We take the following classical action as a starting point
\begin{align}
S[\phi,\psi] &=  S_0[\phi,\psi] + S_{\text{int}}[\phi,\psi] \\
&= \int_x \Bigg\{ \frac{1}{2} \partial_\mu \phi(x) \partial^\mu \phi(x) - \frac{m^2}{2} \phi^2(x) + \bar{\psi}(x)(\ii \myslash{\partial} - M) \psi(x)  \nonumber \\ 
&  \qquad \qquad 
- \frac{\alpha}{3!} \phi^3(x) -\frac{\lambda}{4!} \phi^4(x)  - g \bar{\psi}(x)\psi(x) \phi(x)\Bigg\} 
\label{eq:classical_action}
\end{align}
where in the last step we have split the action into a free and an interaction part corresponding to the first and second line of \eqref{eq:classical_action}. We will include all contributions of the effective action up to two-loop order which is sufficient to detail all the intricacies of the renormalization, in particular in the fermionic sector. The 2PI action up to this order is given by
\begin{align}
 \Gamma_{\text{2PI}}[\phi,G,D] =&
\int_x \Bigg\{ \frac{1}{2} \partial_\mu \phi(x) \partial^\mu \phi(x) - \frac{m^2}{2} \phi^2(x) 
- \frac{\alpha}{3!} \phi^3(x) -\frac{\lambda}{4!} \phi^4(x) \nonumber \\
  - &\frac{1}{2} \left( \square_x + m^2 \right) G(x,y) \big|_{x=y} - \frac{1}{2} \alpha \, \phi(x) \, G(x,x)
  - \frac{1}{8} \lambda G^2(x,x)
  - \frac{1}{4} \lambda \,\phi^2(x) \,  G(x,x) 
  \nonumber \\
+ &  \text{tr}\left[\left(\ii \myslash{\partial}_x - M\right) 
 D(x,y)\big|_{x=y}\right] - g \, \phi(x) \, \text{tr}[D (x,x)]
 \Bigg\} \nonumber \\
+ &   \int_x \int_y \left[\frac{\ii}{12} (\alpha + \lambda \phi(x)) (\alpha + \lambda \phi(y)) \,G^3(x,y)
- \frac{\ii}{2} g^2 G(x,y) \, \text{tr}\left[D(x,y)D(y,x)\right] 
\right] \,.
\label{eq:Gamma_2PI_unrenormalized}
\end{align}
where we have already used the fact that terms linear 
in $\psi$ vanish as it does not acquire a VEV. We 
note for completeness, that the last line will give 
rise to the so-called ``memory integrals''. We stress 
that \eqref{eq:Gamma_2PI_unrenormalized} represents 
the unrenormalized action and that 
\eqref{eq:extr_Gamma_phi}, \eqref{eq:eom_G_generic} 
and  \eqref{eq:eom_D_generic} hold for both the 
unrenormalized and renormalized action.

Eventually, we are interested in obtaining the renormalized EOMs. It has been shown in \cite{Berges:2005hc} that one needs not only to renormalize the couplings appearing in the classical action $S$ but also the ones which couple one-point functions to two-point functions, as well as the couplings between two-point functions only. The reason is, that the two-point functions are resummed propagators within the 2PI formalism implying orders in perturbation theory get mixed. However, it has been shown on general grounds in ref.~\cite{Berges:2005hc}, that  this is done in a particular way. This allows one to carry out the renormalization such that only a finite number of counterterms are needed.

As is customary, we first define the renormalized fields from the bare ones, using their respective wave-function renormalizations, as
\begin{align}
&\phi(x) = Z^{\frac{1}{2}}_{\phi,2} \phi_R(x) \,, \quad \quad
G(x,y) = Z_{\phi,0} \, G_R(x,y) \,, \nonumber \\  
&\psi(x) = Z^{\frac{1}{2}}_{\psi,2} \psi_R(x)\,,\quad \quad  D(x,y) = Z_{\psi,0} \, D_R(x,y)\,,
\end{align}
where we have indicated the number of fields $\phi\, (\psi)$ associated with a term by the index $i$ in $Z_{\phi,i} \,(Z_{\psi,i})$. 
We adopt this same notation for the mass and coupling 
counterterms, and obtain
\begin{align}
\label{eq:counterterms}
\begin{array}{ll}
Z_{\phi,2} m^2 = m^2_R + \delta m^2_2 \,,
&
Z_{\phi,2}^{\frac{i}{2}} Z_{\phi,0}^{\frac{3-i}{2}} \alpha = \alpha_R + \delta \alpha_{i} \quad (i=0,1,3) 
\\[4mm] 
Z_{\phi,0} m^2 = m^2_R + \delta m^2_0 \,,
& Z_{\phi,2}^{j/2} Z_{\phi,0}^{(4-j)/2} \lambda = \lambda_{R} + \delta \lambda_j\quad (j=0,2,4)\,, \\[4mm]
Z_{\psi,0} M = M_R + \delta M_0 \,, 
&Z_{\psi,0}  Z_{\phi,2}^{\frac{k}{2}} Z_{\phi,0}^{\frac{1-k}{2}} g = g_{R} + \delta g_k \quad (k=0,1) \,.
\end{array}
\end{align} 
We note for completeness, that one needs an additional counterterm in the action to cancel 
loop-induced contributions to the effective action which are linear in $\phi_R$, i.e.
\begin{align}
- \int_x \delta t_1 \, \phi_R(x) \,.
\end{align}
 
At this stage, we can already give the renormalized equations of motions
\begin{align}
&\left[ (1+\delta Z_{\phi,2})\square_x  +
\widehat{m}^2_2(x) \right]\phi_R(x) = - \delta t_1 - \frac{(\alpha_R + \delta \alpha_1)}{2} G_R(x,x) - (g_R + \delta g_1) \text{tr}[D_R(x,x)] \,,
\label{eq:eom_phi} \\[2mm]
%
&\left[ (1+\delta Z_{\phi,0})\square_x  +
\widehat{m}^2_0(x) 
\right] G_R(x,y)  = \delta_{\cal C}(x-y)  - \int_z \Pi(x,z)  G_R(z,y)   \,,
\label{eq:eom_G} \\[2mm]
%
& \left[\ii\left(1+\delta Z_{\psi,0}\right)\myslash{\partial}_x
 -\widehat{M}_0(x) \right]
 D_R(x,y)  = \delta_{\cal C}(x-y)  - \int_z \Sigma(x,z)   D_R(z,y)  \,,
\label{eq:eom_D}
\end{align}
with
\begin{align}
%
& \widehat{m}^2_2(x) = m^2_R + \delta m^2_2 
+ \frac{1}{2} (\alpha_R + \delta \alpha_3) \phi_R(x)
+ \frac{1}{6} (\lambda_R + \delta \lambda_4) \phi^2_R(x) + \frac{1}{2} (\lambda_R + \delta \lambda_2) G_R(x,x)\,, \\[3mm]
%
& \widehat{m}^2_0(x)=
 m^2_R + \delta m^2_0 + \frac{1}{4}(\lambda_R + \delta \lambda_0) G_R(x,x)+ \frac{1}{2} (\alpha_R + \delta \alpha_3) \phi_R(x)   + \frac{1}{4} (\lambda_R + \delta \lambda_2)  \phi^2_R(x)\,, \\[3mm]
& \widehat{M}_0(x) = \left(M_R+\delta M_0\right)
+ (g_R+\delta g_1)\phi_R(x) \,,\\[3mm]
%
& \Pi(x,z) = -\frac{\ii \left[(\alpha_R + \delta \alpha_0)+(\lambda_R +\delta \lambda_1)\phi_R(x)\right]\, \left[(\alpha_R + \delta \alpha_0)+(\lambda_R +\delta \lambda_1)\phi_R(z)\right]}{4} \,G^2_R(x,z) \nonumber \\[2mm]
& \qquad \qquad  +\frac{\ii2(g_0 + \delta g_0)^2}{2} \,\text{tr}[D_R(x,z)D_R(z,x)] \,,\\[3mm]
%
&\Sigma(x,z) = \frac{\ii (g_R + \delta g_0)^2}{2} \, G_R(x,z) \left[ D_R(x,z) + D_R(z,x) \right]\,.
\end{align}
Here, we have split the self-energies $\overline{\Pi}(x,z)$ and $\overline{\Sigma}(x,z)$ into non-local contributions $\Pi(x,z)$ and $\Sigma(x,z)$ and local ones which are absorbed in $\widehat{m}^2_0(x)$ and $\widehat{M}_0(x)$, respectively. Note that the integral on the right sides of \eqref{eq:eom_G} and \eqref{eq:eom_D} are the previously mentioned memory integrals. 

One can solve \eqref{eq:eom_phi}--\eqref{eq:eom_D} numerically for a given set of initial conditions. However, before one can start such a task, one must determine the yet unspecified counterterms. To this end, we use an on-shell scheme, detailed in the next section.

%-------------------------------------------------------------------------------
\section{Renormalization: Generic Aspects}
\label{sec:gen_renorm}
It is known that the Bogoliubov-Parasiuk-Hepp-Zimmermann (BPHZ) procedure \cite{10.1007/BF02392399, Hepp:1966eg, Zimmermann:1969jj}, which 
is used in standard QFT to determine the structure of 
the divergences, does not suffice in the case of the 2PI 
formalism (see, for e.g., \cite{Berges:2005hc} and 
references therein). This is caused by the resummed 
nature of the 2-point functions $G$ and $D$. The 
solution to this challenge is the auxiliary vertex 
functions which can be resummed such that a 
consistent renormalization with only a finite number 
of counterterms is possible. 

The corresponding 2PI kernels, which enter the corresponding Bethe-Salpeter equations (BSE) are \cite{Berges:2005hc,Reinosa:2009tc}
\begin{align}
\oL^{(4)}(x_1,x_2,x_3,x_4) &\equiv 4 \frac{\delta^2 \Gamma^{\text{2PI}}_{\text{int}}}{\delta G(x_1,x_2) \delta G(x_3,x_4)} \bigg|_{\op,\oG,\oD}\,, \\
\Lambda_{\psi\psi}(x_1,x_2,x_3,x_4)_{(ab),(cd)} &\equiv -\frac{\delta^2 \Gamma^{\text{2PI}}_{\text{int}}}{\delta D^{ba}(x_1,x_2)\,\delta D^{cd}(x_3,x_4)} \bigg|_{\op,\oG,\oD}\,, \\
%
\Lambda^{(4)}_{\psi\phi}(x_1,x_2,x_3,x_4)_{ab} &\equiv -2\frac{\delta^2 \Gamma^{\text{2PI}}_{\text{int}}}{\delta D^{ba}(x_1,x_2)\,\delta G(x_3,x_4)}\bigg|_{\op,\oG,\oD} \\
& = \Lambda^{(4)}_{\phi\psi}(x_3,x_4,x_1,x_2)_{ab} \equiv -2\frac{\delta^2 \Gamma^{\text{2PI}}_{\text{int}}}{\delta G(x_3,x_4)\, \delta D^{ab}(x_1,x_2)}\bigg|_{\op,\oG,\oD}\,,
\end{align}
where we have denoted spinor indices through lowercase Latin alphabets. At this stage, we still need to take into account that a loop with four internal fermions can generate a divergent contribution to the quartic scalar coupling. This can be taken care of by introducing the following modified kernel for scalars
\begin{align}
\tilde{\Lambda}_{\phi\phi}(x_1,x_2,x_3,x_4) & = \oL^{(4)}(x_1,x_2,x_3,x_4) \nonumber \\
& - \ii \int_{y_1\dots y_4} \text{tr}\left[\oD(y_1,y_3)\, \overline{\Lambda}^{(4)}_{\phi\psi}(x_1,x_2,y_1,y_2) \,\oD(y_2,y_4)\,\overline{\Lambda}_{\psi \phi}(y_3,y_4,x_3,x_4) \right] \nonumber\\
    &+ \ii\int_{y_1\dots y_8}\text{tr}\bigg[\oD(y_1,y_3)\,\Lambda^{(4)}_{\phi\psi}(x_1,x_2,y_1,y_2)\,\oD(y_2,y_4)V_{\psi\psi}(y_3,y_4,y_5,y_6)\nonumber \\ 
    & \qquad \qquad \oD(y_5,y_7)\,\Lambda^{(4)}_{\psi \phi}(y_7,y_8,x_3,x_4)\,\oD(y_6,y_8)\bigg] \,.
\end{align}
where the trace in the second line (``tr") is over the spinor indices. Now, we are in the position
to define the vertex functions which are 
given by the following Bethe-Salpeter equations
\begin{align}
\label{eq:BSEVbar}
\oV^{(4)}(x_1,x_2,x_3,x_4) & 
= \tilde{\Lambda}_{\phi\phi}(x_1,x_2,x_3,x_4) \nonumber \\
&+ \frac{\ii}{2} \int_{y_1\dots y_4}
\tilde{\Lambda}_{\phi\phi}(x_1,x_2,y_1,y_2) \oG(y_1,y_3) \oG(y_2,y_4) 
\overline{V}^{(4)}(y_3,y_4,x_3,x_4)\,, \\
\label{eq:BSEVpsibar}
%
V_{\psi\psi} (x_1,x_2,x_3,x_4)_{ab,cd} &= \Lambda_{\psi \psi}(x_1,x_2,x_3,x_4)_{ab,cd} \nonumber \\
    &+\ii \int_{y_1\dots y_4} \Lambda_{\psi \psi}(x_1,x_2,y_1,y_2)_{ab,ef}\oD(y_1,y_3)_{eg} \nonumber\\
  &\hspace{10mm}V_{\psi\psi} (y_3,y_4,x_3,x_4)_{ef,cd}\oD(y_2,y_4)_{fb}\, \,, \\
V_{\psi\phi}^{(4)} (x_1,x_2,x_3,x_4)_{ab} &= \Lambda^{(4)}_{\psi \phi}(x_1,x_2,x_3,x_4)_{ab} \nonumber \\
    &+\ii \int_{y_1\dots y_4} \Lambda^{(4)}_{\psi \phi}(x_1,x_2,y_1,y_2)_{ae}\oD(y_1,y_3)_{ef} \nonumber\\
  &\hspace{10mm}V^{(4)}_{\psi\phi} (y_3,y_4,x_3,x_4)_{fb}\oG(y_2,y_4)\, \,.  
\end{align}

In the study of the vacuum structure of a theory, it is useful to consider any $\phi$ and evaluate
the corresponding two-point function (Green's function) $G(\phi)$. This leads to a 2PI effective action $\Gamma[\phi] = \Gamma_{\text{2PI}}[\phi, G(\phi)]$ which gives rise to a 2PI improved effective potential. When studying
phase transitions in the 2PI formalism, this is the object to consider when looking for the vacuum states of the theory.
The two-point functions in this case can be evaluated from the stationarity condition \eqref{eq:extr_Gamma_G}, leading to the gap equation for the scalar propagator
\begin{align}
 \overline{G}^{-1}(x,y; \phi) &= G^{-1}_0(x,y; \phi) - \overline{\Pi}(x,y; \phi) 
 \label{eq:gap_eqn_G}
\end{align}
and similarly one for the fermionic propagator
\begin{align}
 \overline{D}^{-1}(x,y; \phi) &= D^{-1}_0(x,y; \phi) - \overline{\Sigma}(x,y; \phi) \,.
 \label{eq:gap_eqn_D}
\end{align}
From this perspective, the stationarity condition for the scalar field becomes
\begin{align}
\Gamma^{(1)}(x) &\equiv \frac{\delta \Gamma^{\text{2PI}}}{\delta \phi(x)}\bigg|_{\overline{\phi},\oG,\oD} \stackrel{!}{=} 0  \\
&=  \frac{\delta \Gamma^{\text{2PI}}}{\delta \phi(x)}\bigg|_{\overline{\phi},\oG,\oD} + \int_{y_1,y_2}\frac{\delta\Gamma^{\text{2PI}}}{\delta G(y_1,y_2)}\bigg|_{\overline{\phi},\oG,\oD}\frac{\delta G(y_1,y_2)}{\delta\phi(x)} + \int_{y_1,y_2}\text{tr}\bigg\{\frac{\delta\Gamma^{\text{2PI}}}{\delta D(y_1,y_2)}\bigg|_{\overline{\phi},\oG,\oD}\frac{\delta D(y_1,y_2)}{\delta\phi(x)}\bigg\}
\end{align}
where $\Gamma^{(1)}$ denotes the physical one-point function. The second and third terms of the last line vanish due to the stationarity conditions. 

Moreover, we will require the $n$-point functions
\begin{align}
\Gamma^{(n)}(x_1,\dots,x_n) = \frac{\delta^n \Gamma^{\text{2PI}}}{\delta \phi(x_1) \dots \delta \phi(x_n)}\bigg|_{\overline{\phi},\oG,\oD} \,.
\end{align}
These are related to the derivatives of the 2PI generating functional $\Gamma^{\text{2PI}}$. However, one has to take care since the two-point function obtained from \eqref{eq:gap_eqn_G} depends on $\phi$ and thus the chain rule must be used. As a consequence, a system of coupled integral equations emerges. In general, these equations have the following form
\begin{align}
\label{eq:npoint2}
\frac{\delta^n \ii \Gamma}{\delta \phi(x_1) \dots \delta \phi(x_n)} = &
{\cal A}^{(n)}(x_1,\dots,x_n) \nonumber \\
+& \int_{z_1\dots z_4} \frac{\delta^2 \ii \Gamma^{\text{2PI}}_{\text{int}}}{\delta \phi(x_1) G(z_1,z_2)}\bigg|_{\op,\oG,\overline{D}} 
\oG(z_1,z_3) \frac{\delta^{n-1} \oPi(z_3,z_4)}{\delta \phi(x_2) \dots \delta \phi(x_n)}
\oG(z_4,z_2) \nonumber \\
+&\int_{z_1 \dots z_4}\text{tr}\bigg\{\frac{\delta^2\Gamma^{\text{2PI}}_{\text{int}}}{\delta \phi(x_1)\delta D(z_1,z_2)}\bigg|_{\overline{\phi},\overline{G},\overline{D}}\overline{D}(z_1,z_3)\frac{\delta^{n-1} \overline{\Sigma}(z_3,z_4)}{\delta\phi(x_2)\dots\delta\phi(x_n)}\overline{D}(z_4,z_2)\bigg\}
% + \text{fermions} 
\\
\frac{\delta^n \overline{\Pi}(y_1,y_2)}{\delta \phi(x_1) \dots \delta \phi(x_n)} = &
{\cal B}^{(n)}(y_1,y_2,x_1,\dots,x_n) \nonumber \\
+& \int_{z_1\dots z_4} \frac{\delta^2 2\ii \Gamma^{\text{2PI}}_{\text{int}}}{\delta G(y_1,y_2) \,\delta G(z_1,z_2)}\bigg|_{\op,\oG,\oD} 
\oG(z_1,z_3) \frac{\delta^{n-1} \oPi(z_3,z_4)}{\delta \phi(x_1) \dots \delta \phi(x_n)}
\oG(z_4,z_2)  \nonumber \\
+& \int_{z_1\dots z_4}\text{tr}\bigg\{ \frac{\delta^2 2\ii \Gamma^{\text{2PI}}_{\text{int}}}{\delta G(y_1,y_2) \,\delta D(z_1,z_2)}\bigg|_{\op,\oG,\oD} 
\oD(z_1,z_3) \frac{\delta^{n-1} \overline{\Sigma}(z_3,z_4)}{\delta \phi(x_1) \dots \delta \phi(x_n)}
\oD(z_4,z_2) \bigg\}
\label{eq:Pi_derivatives}
% + \text{fermions}
\\
\frac{\delta^n \overline{\Sigma}(y_1,y_2)}{\delta \phi(x_1) \dots \delta \phi(x_n)} = &
{\cal C}^{(n)}(y_1,y_2,x_1,\dots,x_n) \nonumber \\
+& \int_{z_1\dots z_4} \frac{\delta^2 \ii \Gamma^{\text{2PI}}_{\text{int}}}{\delta D(y_1,y_2) \,\delta G(z_1,z_2)}\bigg|_{\op,\oG,\oD} 
\oG(z_1,z_3) \frac{\delta^{n-1} \oPi(z_3,z_4)}{\delta \phi(x_1) \dots \delta \phi(x_n)}
\oG(z_4,z_2)  \nonumber \\
+& \int_{z_1\dots z_4} \frac{\delta^2 \ii \Gamma^{\text{2PI}}_{\text{int}}}{\delta D(y_1,y_2) \,\delta D(z_1,z_2)}\bigg|_{\op,\oG,\oD} 
\oD(z_1,z_3) \frac{\delta^{n-1} \overline{\Sigma}(z_3,z_4)}{\delta \phi(x_1) \dots \delta \phi(x_n)}
\oD(z_4,z_2) 
\label{eq:Sigma_derivatives}
\end{align}
The functions ${\cal A}^{(n)}$, ${\cal B}^{(n)}$ and ${\cal C}^{(n)}$ contain both various derivatives of 
$\Gamma^{\text{2PI}}_{\text{int}}$ with respect to $\phi$, $G$  and $D$ at the stationarity point $(\op,\oG,\oD)$ \cite{Berges:2005hc}.
Moreover, they contain lower derivatives $\delta^k \oPi/\delta \phi^k$ and 
$\delta^k \overline{\Sigma}/\delta \phi^k$ ($k=1,\dots,n-1$). 
Consequently, $n$-point functions are expressed solely via parts of the 2PI generating functional and their derivatives. This comes at the expense of infinite resummations of self-energy diagrams and their derivatives. A formal solution to the self-energy equations can be given in terms of the various vertex functions.
Using these, the solutions to \eqref{eq:Pi_derivatives} and \eqref{eq:Sigma_derivatives} are given by
%
\begin{align}
\frac{\delta^n \overline{\Pi}(y_1,y_2)}{\delta \phi(x_1) \dots \delta \phi(x_n)} = &
{\cal B}^{(n)}(y_1,y_2,x_1,\dots,x_n) \nonumber \\
+& \frac{\ii}{2} \int_{z_1\dots z_4} \oV^{(4)}(x_1,x_2,z_1,z_2) \oG(z_1,z_3)
 {\cal B}^{(n)}(z_3,z_4,x_1,\dots,x_n)
\oG(z_4,z_2) \nonumber \\
+& \ii \int_{z_1\dots z_4} \text{tr}\left\{ V^{(4)}_{\phi\psi}(x_1,x_2,z_1,z_2) \oD(z_1,z_3)
 {\cal C}^{(n)}(z_3,z_4,x_1,\dots,x_n)
\oD(z_4,z_2)\right\}\,.
\label{eq:Pi_derivatives_sol} 
\end{align}

\begin{align}
\frac{\delta^n \overline{\Sigma}(y_1,y_2)}{\delta \phi(x_1) \dots \delta \phi(x_n)} = &
{\cal C}^{(n)}(y_1,y_2,x_1,\dots,x_n) \nonumber \\
+& \ii \int_{z_1\dots z_4}  V_{\psi\phi}^{(4)}(x_1,x_2,z_1,z_2) 
 {\cal B}^{(n)}(z_3,z_4,x_1,\dots,x_n)\oG(z_1,z_3)
\oG(z_4,z_2) \nonumber \\
+& \ii \int_{z_1\dots z_4}  V_{\psi\psi}(x_1,x_2,z_1,z_2) \oD(z_1,z_3)
 {\cal C}^{(n)}(z_3,z_4,x_1,\dots,x_n)
\oD(z_4,z_2) \,.
\label{eq:Sigma_derivatives_sol} 
\end{align}
From these equations, one can then obtain solutions to \eqref{eq:npoint2}.
%
In addition, we would require an auxiliary scalar vertex function 
\begin{align}
V^{(4)}(x_1,x_2,x_3,x_4)  
&= \Lambda^{(4)}(x_1,x_2,x_3,x_4) \nonumber
\\
&\qquad + \frac{\ii}{2} \int_{y_1\dots y_4}
\Lambda^{(4)}(x_1,x_2,y_1,y_2) \oG(y_1,y_3) \oG(y_2,y_4) 
\oV^{(4)}(y_3,y_4,x_3,x_4) \,,
\label{eq:BSEV}
\end{align}
with
\begin{align}
\Lambda^{(4)}(x_1,x_2,x_3,x_4) &\equiv 2 \frac{\delta^3 \Gamma^{\text{2PI}}_{\text{int}}}{\delta G(x_1,x_2) \delta \phi(x_3) \delta \phi(x_4)} \bigg|_{\op,\oG,\oD} \,.
\end{align}

It was mentioned in the previous section that actual calculations require an approximation of the 2PI generating functional. In terms of renormalizability, there are certain requirements to be met when choosing the diagrams for $\Gamma^{\text{2PI}}_{\text{int}}$, which becomes apparent when using the BPHZ analysis for the identification of necessary subtractions. Therein, all diagrams are examined for possible sub-divergences, by employing Weinberg’s theorem \cite{Weinberg:1959nj}. Whenever a part of a diagram is potentially divergent, this part is shrunk to an effective vertex. The new topology created this way must be part of the renormalized $\Gamma^{\text{2PI}}_{\text{int}}$, with the effective vertex replaced by counterterms in order to cancel divergences. 

Effectively, this means that we are dealing with a truncation in perturbation theory where some parts have been resummed, implying that one has to take different mass and coupling constant counterterms, depending on the combination of one- and two-point functions connecting to a `vertex'. According to \cite{Berges:2005hc}, renormalization may be carried out in the vacuum at temperature $T=0$. It is most convenient to work in momentum space, which allows us to employ the usual techniques to determine the various counterterms in an on-shell scheme. 
We extend the usual on-shell conditions of standard QFT, see for e.g.~\cite{Denner:1991kt},
to the 2PI formalism as follows
\begin{align}
\label{eq:renor_cond_scalar}
Z_{\phi,0} G^{-1}_R(p) \bigg|_{p^2=m^2_R} &= Z_{\phi,2} \Gamma^{(2)}_\phi(p) \bigg|_{p^2=m^2_R} = 0 \,,\\ 
Z_{\phi,0}\frac{\partial}{\partial p^2} G^{-1}_R(p) \bigg|_{p^2=m^2_R} &= Z_{\phi,2} \frac{\partial}{\partial p^2}\Gamma^{(2)}_\phi(p) \bigg|_{p^2=m^2_R} = 1 \,,
\end{align}
\begin{align}
\label{eq:renor_cond_fermion}
Z_{\psi,0} \,D^{-1}_R(p) u(p) \bigg|_{p^2=M^2_R} = 0 \,,\quad 
\lim_{p^2\to M^2_R} Z^{-1}_{\psi,0} \left(\frac{\myslash{p} + M_R}{p^2-M^2_R}\right)D^{-1}_R(p)u(p) =  u(p) \,,
\end{align}
\begin{align}
\label{eq:renor_cond_fourpt}
Z_{\phi,0}^{2}\,\overline{V}^{(4)}(p_1,p_2,p_3,p_4) \bigg|_{p^2_i=m^2_R}  &= 
Z_{\phi,2}\,Z_{\phi,0}\,  V^{(4)}(p_1,p_2,p_3,p_4) \bigg|_{p^2_i=m^2_R} \nonumber \\
&= Z_{\phi,2}^{2}\,\Gamma^{(4)}(p_1,p_2,p_3,p_4) \bigg|_{p^2_i=m^2_R} =
-\lambda_R    \,,
\end{align}
\begin{equation}
\label{eq:renor_cond_threept}
Z_{\phi,0}\, Z_{\phi,2}^{\frac{1}{2}}\,V^{(3)}(p_1,p_2,p_3) \bigg|_{p^2_2= p^2_3 = m^2_R} = Z_{\phi,2}^{\frac{3}{2}}\, \Gamma^{(3)}(p_1,p_2,p_3) \bigg|_{p^2_2= p^2_3 = m^2_R}  = -\alpha_R \,,
\end{equation}
\begin{equation}
\label{eq:renor_cond_yukawa}
Z_{\psi,2} Z_{\phi,2}^{\frac{1}{2}}\, \bar{u}(p_1) V^{(3)}_{\psi\phi}(p_1,p_2,p_3) u(p_2) \bigg|_{p^2_2 = M^2_R,\,p^2_3 =m^2_R} = -g_R
\end{equation}
where $u(p)$ is an on-shell spinor and $\bar{u}(p)$ is its Dirac conjugate. $V^{(3)}$ and $V^{(3)}_{\psi\phi}$ denote pure scalar and fermion-fermion-scalar three-point vertices respectively, that will be discussed in the relevant section.
Note that we use the same symbols to denote the $n$-point functions in configuration and momentum space indicated only by the arguments. 
The momentum arguments are not independent, but related via momentum conservation, for e.g. $p_1 + p_2 = p_3 + p_4$ for the four-point functions and $p_1 = p_2 +p_3$ for the three-point functions.

Note that we have made a choice to implement the same renormalization conditions for the physical two-point functions of the fields and the propagators, as in \cite{Pilaftsis:2013xna, Pilaftsis:2017enx}. We do the same for the physical $n$-point functions of the scalar fields, and the corresponding vertex functions. One may also choose to renormalize these functions to different coupling constants, such as in \cite{Kainulainen:2021eki}, but these correspond to shifts in the various counterterms.

%----------------------------------------------------------------------------
\section{Renormalization with only Scalars}
\label{sec:scalars}
%----------------------------------------------------------------------------

We begin with scalars, which can be considered to be a limit where the fermions are so heavy that they need to be integrated out. This allows us to detail the intricacies of the renormalization procedure, without the complication of additional Lorentz structures. As a novel aspect, we present here not only the counterterms for the symmetric phase, but also in the broken one.

%----------------------------------------------------------------------------
%Hartree Approximation
%----------------------------------------------------------------------------
\subsection{Revisiting the Hartree Approximation}
\label{sec:hartree}
This case has been treated in literature already several times \cite{Blaizot:2003br,Blaizot:2003an,Pilaftsis:2017enx} and is summarized here to exemplify some of the details involved. Moreover, we have obtained new results, which to our knowledge have not been presented in literature so far, such as finite contributions to the wave function renormalization.  

In this approximation, one sets $\alpha=g=0$ and takes only the leading
contribution to $\Gamma_2$ into account. Expressing $\Gamma^{\text{2PI}}_{\text{int}}$ in terms of renormalized quantities, we get
\begin{align}
&\Gamma^{\text{2PI}}_{\text{int}}[\phi_R,G_R] =
- \frac{1}{2}\int_{x} (\delta Z_{\phi,0} \square_x + \delta m^2_0) G_R(x,y) \big|_{x=y} 
- \frac{1}{2}\int_x \phi_R(x)  (\delta Z_{\phi,2} \square_x + \delta m^2_2) \phi_R(x)
  \nonumber \\ 
& \qquad - \int_x \left[ \frac{1}{8}(\lambda_R + \delta \lambda_0) G^2_R(x,x)
   + \frac{1}{4} (\lambda_R + \delta \lambda_2) G_R(x,x) \phi^2_R(x)
   + \frac{1}{4!} (\lambda_R + \delta \lambda_4) \phi^4_R(x)\right] \,.
\end{align}
This gives the four-point kernel 
\begin{equation}
    \overline{\Lambda}^{(4)} = 4\, \frac{\delta^2 \Gamma^{\text{2PI}}_{\text{int}}}{\delta G_R(k)\delta G_R(q)} = -(\lambda_R + \delta \lambda_0) \,,
    \label{kernelhartree}
\end{equation}
which is evidently independent of the (external) momentum. Thus, the BSE~\eqref{eq:BSEVbar} simplifies considerably within the Hartree approximation. 
As the vertex function is an infinite resummation of iterations of this kernel, stitched together by loops with two propagators (see Fig. \ref{fig:bseqn_ht}), the only origin of a momentum dependence in $\overline{V}^{(4)}$ is from the loop function depending on the sum of external momenta $p = p_1 + p_2$. For brevity, we will thus use the notation $\overline{V}^{(4)}(p)\equiv \overline{V}^{(4)}(p_1,p_2,p_3,p_4)$ throughout this section. Accordingly, the BSE is expressed as
\begin{align}
    \overline{V}^{(4)}(p) = -(\lambda_R + \delta \lambda_0) \, -\, \frac{i}{2}(\lambda_R+\delta \lambda_0)\,\overline{V}^{(4)}(p)\int_q G_R(q)  G_R(p+q) \,.
\label{eq:VRp_hartree}    
\end{align} 
At this point, it is convenient to calculate $p^2$ in the center of mass (COM) system. This gives the familiar result
\begin{equation}
    p^2 = (p_1 + p_2)^2 = 4E_{\ast}^2 \,,
\end{equation}
where $2E_{\ast}$ is the COM energy. Let $p^2_{\ast} = 4m^2_R $ be the renormalization point corresponding to the COM three-momentum $|\vec{p_{\ast}}| = 0$ and let
\begin{equation}
	\oV^{(4)}(p_{\ast}) = - \lambda_R\,.
	\label{eqn:oV4_rc_ht}
\end{equation}
% Figure environment removed
This allows us to solve exactly for the counterterm
\begin{align}
\delta \lambda_0 = -\lambda_R + \frac{\lambda_R}{1-\frac{1}{2}\lambda_R \mathcal{I}(p^2_{\ast})}\quad \text{with} \quad \mathcal{I}(p^2) = i\int_q G_R(q)G_R(p+q) \,,
\label{ctlambda0_hartree}
\end{align}
which we can insert in \eqref{eq:VRp_hartree} to obtain the vertex function
\begin{equation}
    \overline{V}^{(4)}(p) =-\frac{\lambda_R}{1-\frac{\lambda_R}{2}\left(\mathcal{I}(p^2_{\ast})-\mathcal{I}(p^2)\right)}\,.
    \label{eqn:vbar4_ht}
\end{equation}
From this, we can immediately observe that $\overline{V}^{(4)}(p)$ is finite as any potential divergences present in the loop integral $\mathcal{I}(p^2)$ cancels in the difference $\mathcal{I}(p^2_{\ast})-\mathcal{I}(p^2)$. Furthermore, the renormalization of the auxiliary vertex $V(p)$ proceeds along similar lines as 
\begin{align}
\Lambda^{(4)} = 2 \,\frac{\delta^3 \Gamma^{\text{2PI}}_{\text{int}}}{\delta^2 \phi_R\,\delta G_R(q)}
= - (\lambda_R + \delta \lambda_2) \,.
\end{align}

\noindent Using this in \eqref{eq:BSEVbar} and the result \eqref{eqn:vbar4_ht}, along with the renormalization condition \eqref{eqn:oV4_rc_ht}, we obtain
\begin{align}
    \delta \lambda_2 = \delta \lambda_0 \quad \text{ and } \quad
    V^{(4)}(p) = \overline{V}^{(4)}(p)  \,.
    \label{eqn:ctl2_ht}
\end{align}
We have made a choice to implement the same renormalization conditions for both vertex functions. We will later do the same for the propagator and the two-point function of the scalar field, as in \cite{Pilaftsis:2013xna, Pilaftsis:2017enx}, as well for the physical $n$-point functions of the scalar fields corresponding to the vertex functions. One may also choose to renormalise these functions to different parameters, such as in \cite{Kainulainen:2021eki}, but these correspond to shifts in the various counterterms.  

The determination of these counterterms via the BSEs allows us to treat the sub-divergences, which are not accounted for by the usual BHPZ procedure, that would appear in the renormalization of the two-point function \cite{Blaizot:2003an,Berges:2005hc}, for which we now turn to the gap equation
\begin{align}
    i G^{-1}_R(p) &=  (p^2 - m^2_R)  - i\overline{\Pi}(p^2) \nonumber \\[2mm]
    &=   (p^2 - m^2_R) + (\delta Z_{\phi,0} \, p^2 - \delta m^2_0) - \frac{(\lambda_R + \delta \lambda_2)}{2} \phi_R^2  
    -\frac{(\lambda_R + \delta \lambda_0)}{2} \int_q G_R(q)  \,,
\label{eq:GR_Hartree}    
\end{align}
where $p$ here is the external momentum. Having determined $\delta \lambda_0$ and $\delta \lambda_2$, we seek for the counterterms $\delta Z_{\phi,0}$ and $\delta m_0^2$ which can be used to treat the divergences that can be accounted for by the BHPZ procedure. Enforcing the on-shell renormalization conditions \eqref{eq:renor_cond_scalar}, we have 
\begin{align}
\label{eq:hartree_dm0}
i G^{-1}_R(p) \bigg|_{p^2=m^2_R} &= - i \overline{\Pi}(p^2)  \bigg|_{p^2=m^2_R} \stackrel{!}{=} 0 \,,\\[3mm]
i\frac{\partial}{\partial p^2 } G^{-1}_R(p) \big|_{p^2=m^2_R} &=   1 - i\frac{\partial}{\partial p^2 } \overline{\Pi}(p^2)  \big|_{p^2=m^2_R} \stackrel{!}{=} 1 \,.
\label{eq:hartree_dZ0}
\end{align}
From \eqref{eq:hartree_dZ0}, we have
\begin{equation}
	\delta Z_{\phi,0} = 0 
\end{equation}
and then \eqref{eq:hartree_dm0} gives
\begin{equation}
	\delta m^2_0 = - \frac{\lambda_R + \delta \lambda_2 }{2} \phi^2_R - \frac{\lambda_R + \delta \lambda_0}{2}\int_q G_R(q)\,.
\end{equation}
If we substitute this back into \eqref{eq:GR_Hartree}, we obtain 
\begin{align}
	i G_R^{-1}(p^2) = p^2 - m^2_R\,,
	\label{eqn:prop_ht}
\end{align}
or, in other words, the full propagator is identically the bare one. Consequently, all integrals over propagators can be expressed in terms of the well-known Passarino-Veltman functions \cite{Denner:1991kt}
\begin{align}
    &\int_q G_R(q) = \frac{1}{16\pi^2}A_0(m^2_R) \\
    &\mathcal{I}(p) \equiv i\int_q G_R(q) G_R(p+q) = \frac{1}{16\pi^2}B_0(p^2,m^2_R,m^2_R) \,,
\end{align}
For the counterterms, we thus find
\begin{align}
\label{eqn:os_dl0_exp}
&\delta \lambda_0 = -\lambda_R - 32\pi^2 \epsilon + \mathcal{O}(\epsilon^2) = \delta \lambda_2 \,, \\
&\delta Z_0 = 0 + O(\epsilon)\,, \quad \text{and} \quad 
\delta m^2_0 = -m^2_R  + O(\epsilon) \,,
\label{eq:Hartree_dZ0_dm20}
\end{align}
and also the expression for the vertex function,
\begin{equation}
	\oV^{(4)}(p) = -\frac{\lambda_R}{1-\frac{\lambda_R}{32\pi^2}\left[B_0(4m^2_R,m^2_R,m^2_R) - B_0(p^2,m^2_R,m^2_R)\right]}\,.
	\label{eqn:ov4_os_ht}
\end{equation}
We mention, for completeness, that we did not assume $\phi_R=0$, i.e. these relations hold true in the unbroken and broken phases. Furthermore, in the unbroken phase we can exploit the $\mathbb{Z}_2$ symmetry of the Hartree approximation and make use of the identity \cite{Berges:2005hc}
\begin{align}
    \frac{\delta^2 \Gamma^{\text{2PI}}_{\text{int}}}{\delta \phi_R^2}
    \bigg|_{\phi_R=0} + (\delta Z_2 \, p^2 - \delta m^2_2)
=2\frac{\delta \Gamma^{\text{2PI}}_{\text{int}}}{\delta G_R}
    \bigg|_{\phi_R=0} + (\delta Z_0 \, p^2 - \delta m^2_0)
\label{eq:symphase_ct}
\end{align}
to relate the counterterms for the propagator and the field. Due to the fact that we have imposed the same renormalization conditions for $\Gamma^{(2)}$ and $G^{-1}_R$  (c.f. \eqref{eq:renor_cond_scalar}), we find
\begin{align}
    \delta Z_{\phi,2} = \delta Z_{\phi,0} \quad \text{ and } \quad
    \delta m^2_0 =  \delta m^2_2 \,.
\label{eq:equality_dZ_dm2}    
\end{align}

We now demonstrate explicitly that these relations do not hold for $\phi_R\ne 0$. It is useful at this point to introduce a diagrammatic representation of the various integral equations that we will encounter, which is based on the one in \cite{Berges:2005hc}. The basic building blocks are given in Fig. \ref{fig:graphics}.
% Figure environment removed

We first examine the physical two-point function, for which we have the following integral equation
\begin{align}
\Gamma^{(2)}(x_1,x_2) &\equiv \frac{\delta^2 \Gamma^{\text{2PI}}}{\delta \phi_R(x_1) \delta \phi_R(x_2) } \nonumber \\
&=   \ii G^{-1}_{0,R}(x_1,x_2) +  
\frac{\delta^2 \Gamma^{\text{2PI}}_{\text{int}}}{\delta \phi_R(x_1) \delta \phi_R(x_2) } \bigg|_{G_R} \nonumber \\
&+ \int_{y_1, y_2, y_3, y_4} 
\frac{\delta^2 \Gamma^{\text{2PI}}_{\text{int}}}{\delta \phi_R(x_1) \delta G_R(y_1,y_2) } \bigg|_{G_R} G_R(y_1,y_3) 
\frac{\delta \overline{\Pi}_R(y_3,y_4)}{ \delta \phi_R(x_2) }G_R(y_4,y_2) \,.
\label{eqn:phys2pta}
\end{align}
Using the elements presented in Fig. \ref{fig:graphics}, this equation reads in graphical form
\begin{equation}
\frac{\delta^2 \Gamma^{\text{2PI}}}{\delta \phi_R(x_1) \delta \phi_R(x_2) }   =  \ii G^{-1}_{0,R}(x_1,x_2) +  
%
 \begin{tikzpicture}[scale=2,baseline={([yshift=-.4ex]current bounding box.center)}]
\node (action2) at (0,-2) [Gamma] {}; 
\node (l1) at (-0.5,-2) [VEV] {}; 
\node (l2) at (0.5,-2) [VEV] {}; 
\draw[-] (l1) node[left] {1} -- (action2);
\draw[-] (action2) -- (l2) node[right] {2} ;
 \end{tikzpicture}
%
 + \frac{1}{2}
 \begin{tikzpicture}[scale=2,baseline={([yshift=-.4ex]current bounding box.center)}]
\node (action2) at (0,0) [Gamma] {}; 
\node (l1) at (-0.5,0) [VEV] {}; 
\node (threepoint) at (0.6,0) [Pi] {}; 
\node (l2) at (1.1,0) [VEV] {}; 
\draw[-] (l1) node[left] {1} -- (action2);
\draw[-] (threepoint) -- (l2) node[right] {2} ;
\draw[double distance=4mm,thick] (action2) -- (threepoint);
 \end{tikzpicture}
\label{phys2pta}
\end{equation}
\noindent We can replace the derivative of the self-energy w.r.t. a field expectation value $\delta \overline{\Pi}/\delta \phi_R$ using the following graphical representation (c.f. \eqref{eq:Pi_derivatives_sol})
\begin{equation}
\label{eq:dPidphi}
 \begin{tikzpicture}[scale=2,baseline={([yshift=-.4ex]current bounding box.center)}]
\node (threepoint) at (3,0) [Pi] {}; 
\coordinate (tra) at (2.6,0);
\draw[double distance=4mm,thick] (threepoint) -- (tra) node[right,above]{\hspace*{-2mm}1\,}
node[right,below]{\hspace*{-2mm}2\,};
\node (l3) at (3.4,0) [VEV] {}; 
\draw[-] (threepoint) -- (l3) node[right] {3} ;
 \end{tikzpicture}
 %
=
% 
 \begin{tikzpicture}[scale=2,baseline={([yshift=-.4ex]current bounding box.center)}]
\node (threepoint) at (3,0) [Gamma] {}; 
\coordinate (tra) at (2.6,0);
\draw[double distance=4mm,thick] (threepoint) -- (tra) node[right,above]{\hspace*{-2mm}1\,}
node[right,below]{\hspace*{-2mm}2\,};
\node (l3) at (3.4,0) [VEV] {}; 
\draw[-] (threepoint) -- (l3) node[right] {3} ;
 \end{tikzpicture}
% 
+ \frac{1}{2} \quad
% 
 \begin{tikzpicture}[scale=2,baseline={([yshift=-.4ex]current bounding box.center)}]
\node (fourpoint) at (3,0) [VR] {\,$\overline{V}^{(4)}$\,}; 
\node (threepoint) at (3.5,0) [Gamma] {}; 
\coordinate (tra) at (2.6,0);
\draw[double distance=4mm,thick] (fourpoint) -- (tra) node[right,above]{\hspace*{-2mm}1\,}
node[right,below]{\hspace*{-2mm}2\,};
\node (l3) at (4,0) [VEV] {}; 
\draw[-] (threepoint) -- (l3) node[right] {3} ;
\draw[double distance=4mm,thick] (fourpoint) -- (threepoint);
\end{tikzpicture}
\end{equation}

\noindent Inserting this into the integral equation \eqref{phys2pta}, we obtain in momentum space
\begin{align}
\Gamma^{(2)}(p) 
&= (p^2-m^2_R) + (\delta Z_{\phi,2}\,p^2- \delta m^2_2)  - \frac{1}{2}(\lambda_R+\delta\lambda_4)\phi^2_R  - \frac{1}{2}(\lambda_R+\delta\lambda_2) \int _q G_R(q)   \nonumber \\[2mm]
	& \qquad -\frac{1}{8}(\lambda_R+\delta\lambda_2)^2 \left[\mathcal{I}(p^2)\right]
	\, \phi^2_R + \frac{1}{16}(\lambda_R+\delta\lambda_2)^2 \left[\mathcal{I}(p^2)\right]^2 \overline{V}^{(4)}(p) \, \phi^2_R \nonumber \\[2mm]
&= \left[(1+\delta Z_{\phi,2})p^2-2 m^2_R - \delta m^2_2\right] -\frac{1}{2}(\lambda_R+\delta\lambda_4)\phi^2_R \nonumber \\[2mm]
& \qquad + \frac{\lambda_R \phi^2_R}{4}
\left\{1- \frac{\lambda_R}{32\pi^2} \left[B_0(p^2_{\ast},m^2_R,m^2_R)- B_0(p^2,m^2_R,m^2_R)\right] \right\}^{-1} + O(\epsilon) \,.
\end{align}
Imposing the same renormalization conditions as for $G_R$, \eqref{eq:renor_cond_scalar}, this yields
\begin{align}
\label{eqn:dZ2_ht}
\delta Z_{\phi,2} &= \frac{\lambda_R^2 \phi^2_R}{128\pi^2}\,\dot{B}_0(m^2_R,m^2_R,m^2_R)\left\{1- \frac{\lambda_R}{32\pi^2} \left[B_0(p^2_{\ast},m^2_R,m^2_R)- B_0(m^2_R,m^2_R,m^2_R)\right]\right\}^{-1}  \\[2mm]
\delta m^2_2 &= - m^2_R  + m^2_R \delta Z_{\phi,2} - \frac{\left(\lambda_R + \delta \lambda_4\right)
\phi^2_R}{2}  \nonumber \\[2mm] 
&\qquad \qquad + \frac{\lambda_R \phi^2_R}{4} \left\{1 - \frac{\lambda_R}{32\pi^2}\left[B_0(p_{\ast}^2,m^2_R,m^2_R)- B_0(m^2_R,m^2_R,m^2_R)\right] \right\}^{-1} \,,
\label{eqn:dm2_ht}
\end{align}
where 
\begin{equation*}
\dot{B}_0(p^2,m^2_R,m^2_R) = \frac{\partial B_0(q^2,m^2_R,m^2_R)}{\partial q^2} \bigg|_{q^2 = p^2}\,.
\end{equation*} 
One recovers the equality with the corresponding counterterms for $G_R$ in \eqref{eq:equality_dZ_dm2} when $\phi_R = 0$, as claimed in \cite{Berges:2005hc}.

Finally, one needs to determine $\delta\lambda_4$. The treatment of the physical four-point function proceeds in the same manner as for the two-point function, but now a lot more diagrams contribute. An easy way to obtain the four-point function is to take the diagrammatical expression of the two-point function and then take two more derivatives with respect to the field $\phi_R$. Application of the chain rule due to the $\phi_R$ dependence of $G_R$ generates a variety of topologies. Here, we will first give the overview over all the topologies that contribute, in order to generalise to additional couplings, and perform truncations appropriate to the Hartree approximation. For the sake of brevity, the four ``legs'' representing the external space-time points, previously denoted in the diagrams with numerical indices, are now suppressed and it is understood that all permutations of $x_2$, $x_3$ and $x_4$ contribute, as long as the resulting diagrams are not equivalent. An example of a diagram, where permutations must be considered, is 
\begin{equation}
 \begin{tikzpicture}[scale=2,baseline={([yshift=-.4ex]current bounding box.center)}]
\node (fourpoint) at (0,0.3) [Gamma] {}; 
\node (threepoint) at (0,-0.3) [Pi] {}; 
\node (l1) at (-0.4,0.3) [VEV] {}; 
\node (l2) at (0.,0.7) [VEV] {}; 
\node (l3) at (0.4,0.3) [VEV] {}; 
\node (l4) at (0.,-0.7) [VEV] {}; 
\draw[double distance=4mm,thick] (fourpoint) -- (threepoint);
\draw[-] (fourpoint) -- (l1);
\draw[-] (fourpoint) -- (l2);
\draw[-] (fourpoint) -- (l3);
\draw[-] (threepoint) -- (l4);
\end{tikzpicture}
%
=
%
 \begin{tikzpicture}[scale=2,baseline={([yshift=-.4ex]current bounding box.center)}]
\node (fourpoint) at (0,0.3) [Gamma] {}; 
\node (threepoint) at (0,-0.3) [Pi] {}; 
\node (l1) at (-0.4,0.3) [VEV] {}; 
\node (l2) at (0.,0.7) [VEV] {}; 
\node (l3) at (0.4,0.3) [VEV] {}; 
\node (l4) at (0.,-0.7) [VEV] {}; 
\draw[double distance=4mm,thick] (fourpoint) -- (threepoint);
\draw[-] (fourpoint) -- (l1) node[left] {1};
\draw[-] (fourpoint) -- (l2) node[above] {4};
\draw[-] (fourpoint) -- (l3) node[right] {3};
\draw[-] (threepoint) -- (l4) node[below] {2};
\end{tikzpicture}
%
+
%
 \begin{tikzpicture}[scale=2,baseline={([yshift=-.4ex]current bounding box.center)}]
\node (fourpoint) at (0,0.3) [Gamma] {}; 
\node (threepoint) at (0,-0.3) [Pi] {}; 
\node (l1) at (-0.4,0.3) [VEV] {}; 
\node (l2) at (0.,0.7) [VEV] {}; 
\node (l3) at (0.4,0.3) [VEV] {}; 
\node (l4) at (0.,-0.7) [VEV] {}; 
\draw[double distance=4mm,thick] (fourpoint) -- (threepoint);
\draw[-] (fourpoint) -- (l1) node[left] {1};
\draw[-] (fourpoint) -- (l2) node[above] {4};
\draw[-] (fourpoint) -- (l3) node[right] {2};
\draw[-] (threepoint) -- (l4) node[below] {3};
\end{tikzpicture}
%
+
%
 \begin{tikzpicture}[scale=2,baseline={([yshift=-.4ex]current bounding box.center)}]
\node (fourpoint) at (0,0.3) [Gamma] {}; 
\node (threepoint) at (0,-0.3) [Pi] {}; 
\node (l1) at (-0.4,0.3) [VEV] {}; 
\node (l2) at (0.,0.7) [VEV] {}; 
\node (l3) at (0.4,0.3) [VEV] {}; 
\node (l4) at (0.,-0.7) [VEV] {}; 
\draw[double distance=4mm,thick] (fourpoint) -- (threepoint);
\draw[-] (fourpoint) -- (l1) node[left] {1};
\draw[-] (fourpoint) -- (l2) node[above] {3};
\draw[-] (fourpoint) -- (l3) node[right] {2};
\draw[-] (threepoint) -- (l4) node[below] {4};
\end{tikzpicture}
\end{equation}
It is important to note that the index 1 is not part of the permutation. This happens because the diagrams are obtained by a subsequent differentiation with respect to the four fields $\phi_R(x_1)$, \dots\,, $\phi_R(x_4)$. The first self-energy box only appears in the second differentiation and thus the 1-index is always attached to the $\Gamma^{\text{2PI}}_{\text{int}}$ blob. In the subsequent discussion, the leg to the left side is always considered to correspond to the 1-index. At first glance, this might seem to not be in line with the symmetry properties of the 4-point function, such as $\Gamma^{(4)}(x_1, x_2 ,x_3, x_4) = \Gamma^{(4)} (x_2, x_1, x_3, x_4)$ and so on. However, these properties are only hidden in the above case and are apparent once the identities of the self-energy boxes are inserted. 

The diagrams that in principal contribute to the 4-point function are
\begin{align}
\Gamma^{(4)} \equiv \frac{\delta^4 \Gamma}{\delta \phi^4} = & \quad
%
 \begin{tikzpicture}[scale=2,baseline={([yshift=-.4ex]current bounding box.center)}]
\node (fourpoint) at (0,0) [Gamma] {}; 
\node (l1) at (-0.4,0) [VEV] {}; 
\node (l2) at (0.,0.4) [VEV] {}; 
\node (l3) at (0.4,0) [VEV] {}; 
\node (l4) at (0.,-0.4) [VEV] {}; 
\draw[-] (fourpoint) -- (l1);
\draw[-] (fourpoint) -- (l2);
\draw[-] (fourpoint) -- (l3);
\draw[-] (fourpoint) -- (l4);
\end{tikzpicture}
%
+
%
\frac{1}{2} \,\,
 \begin{tikzpicture}[scale=2,baseline={([yshift=-.4ex]current bounding box.center)}]
\node (v1) at (0,0.3) [Gamma,fill=red] {}; 
\node (v2) at (0,-0.3) [Pi] {}; 
\node (l1) at (-0.4,0.3) [VEV] {}; 
\node (l2) at (0.,0.7) [VEV] {}; 
\node (l3) at (0.4,0.3) [VEV] {}; 
\node (l4) at (0.,-0.7) [VEV] {}; 
\draw[double distance=4mm,thick] (v1) -- (v2);
\draw[-] (v1) -- (l1);
\draw[-] (v1) -- (l2);
\draw[-] (v1) -- (l3);
\draw[-] (v2) -- (l4);
\end{tikzpicture}
%
+
%
\frac{1}{4} \,\,
 \begin{tikzpicture}[scale=2,baseline={([yshift=-.4ex]current bounding box.center)}]
\node (v1) at (0,0.3) [Gamma,fill=red] {}; 
\node (v2) at (0,-0.3) [Pi] {}; 
\node (v3) at (0.6,0.3) [Pi] {}; 
\node (l1) at (-0.4,0.3) [VEV] {}; 
\node (l2) at (0.,0.7) [VEV] {}; 
\node (l3) at (0.9,0.3) [VEV] {}; 
\node (l4) at (0.,-0.7) [VEV] {}; 
\draw[double distance=4mm,thick] (v1) -- (v2);
\draw[double distance=4mm,thick] (v1) -- (v3);
\draw[-] (v1) -- (l1);
\draw[-] (v1) -- (l2);
\draw[-] (v3) -- (l3);
\draw[-] (v2) -- (l4);
\end{tikzpicture}
%
+
%
\frac{1}{8} \,\,
 \begin{tikzpicture}[scale=2,baseline={([yshift=-.4ex]current bounding box.center)}]
\node (v1) at (0,0.3) [Gamma,fill=red] {}; 
\node (v2) at (0,-0.3) [Pi] {}; 
\node (v3) at (0.6,0.3) [Pi] {}; 
\node (v4) at (0.0,0.9) [Pi] {}; 
\node (l1) at (-0.4,0.3) [VEV] {}; 
\node (l2) at (0.,1.3) [VEV] {}; 
\node (l3) at (0.9,0.3) [VEV] {}; 
\node (l4) at (0.,-0.7) [VEV] {}; 
\draw[double distance=4mm,thick] (v1) -- (v2);
\draw[double distance=4mm,thick] (v1) -- (v3);
\draw[double distance=4mm,thick] (v1) -- (v4);
\draw[-] (v1) -- (l1);
\draw[-] (v4) -- (l2);
\draw[-] (v3) -- (l3);
\draw[-] (v2) -- (l4);
\end{tikzpicture}
%
\nonumber \\
%
& +
%
 \begin{tikzpicture}[scale=2,baseline={([yshift=-.4ex]current bounding box.center)}]
\node (v1) at (0,0.3) [Gamma] {}; 
\node (v2) at (0,-0.3) [Pi] {}; 
\node (v3) at (0.6,0.3) [Pi] {}; 
\node (l1) at (-0.4,0.3) [VEV] {}; 
\node (l2) at (0.,0.7) [VEV] {}; 
\node (l3) at (0.9,0.3) [VEV] {}; 
\node (l4) at (0.,-0.7) [VEV] {}; 
\draw (v1) -- (v2);
\draw (v1) -- (v3);
\draw (v2) -- (v3);
\draw[-] (v1) -- (l1);
\draw[-] (v1) -- (l2);
\draw[-] (v3) -- (l3);
\draw[-] (v2) -- (l4);
\end{tikzpicture}
%
+
%
\frac{1}{2} \,\,
 \begin{tikzpicture}[scale=2,baseline={([yshift=-.4ex]current bounding box.center)}]
\node (v1) at (0,0.3) [Gamma,fill=red] {}; 
\node (v2) at (0,-0.3) [Pi] {}; 
\node (v3) at (0.6,0.3) [Pi] {}; 
\node (v4) at (0.0,0.9) [Pi] {}; 
\node (l1) at (-0.4,0.3) [VEV] {}; 
\node (l2) at (0.,1.3) [VEV] {}; 
\node (l3) at (0.9,0.3) [VEV] {}; 
\node (l4) at (0.,-0.7) [VEV] {}; 
\draw (v1) -- (v2);
\draw (v1) -- (v3);
\draw (v2) -- (v3);
\draw[double distance=4mm,thick] (v1) -- (v4);
\draw[-] (v1) -- (l1);
\draw[-] (v4) -- (l2);
\draw[-] (v3) -- (l3);
\draw[-] (v2) -- (l4);
\end{tikzpicture}
%
+
%
 \begin{tikzpicture}[scale=2,baseline={([yshift=-.4ex]current bounding box.center)}]
\node (v1) at (0,0.3) [Gamma] {}; 
\node (v2) at (0,-0.3) [Pi] {}; 
\node (v3) at (0.6,0.3) [Pi] {}; 
\node (v4) at (0.0,0.9) [Pi] {}; 
\node (l1) at (-0.4,0.3) [VEV] {}; 
\node (l2) at (0.,1.3) [VEV] {}; 
\node (l3) at (0.9,0.3) [VEV] {}; 
\node (l4) at (0.,-0.7) [VEV] {}; 
\draw (v1) -- (v2);
\draw (v4) -- (v3);
\draw (v2) -- (v3);
\draw (v1) -- (v4);
\draw[-] (v1) -- (l1);
\draw[-] (v4) -- (l2);
\draw[-] (v3) -- (l3);
\draw[-] (v2) -- (l4);
\end{tikzpicture}
%
\nonumber \\
%
& + \frac{1}{2} \,\,
%
 \begin{tikzpicture}[scale=2,baseline={([yshift=-.4ex]current bounding box.center)}]
\node (v1) at (0,0.3) [Gamma] {}; 
\node (v2) at (0.6,0.3) [Pi] {}; 
\node (l1) at (-0.4,0.3) [VEV] {}; 
\node (l2) at (0.,0.7) [VEV] {}; 
\node (l3) at (0.9,0.3) [VEV] {}; 
\node (l4) at (0.6,-0.1) [VEV] {}; 
\draw[double distance=4mm,thick] (v1) -- (v2);
\draw[-] (v1) -- (l1);
\draw[-] (v1) -- (l2);
\draw[-] (v2) -- (l3);
\draw[-] (v2) -- (l4);
\end{tikzpicture}
%
+
%
\frac{1}{4} \,\,
 \begin{tikzpicture}[scale=2,baseline={([yshift=-.4ex]current bounding box.center)}]
\node (v1) at (0,0.3) [Gamma,fill=red] {}; 
\node (v3) at (0.6,0.3) [Pi] {}; 
\node (v4) at (0.0,0.9) [Pi] {}; 
\node (l1) at (-0.4,0.3) [VEV] {}; 
\node (l2) at (0.,1.3) [VEV] {}; 
\node (l3) at (0.9,0.3) [VEV] {}; 
\node (l4) at (0.6,-0.1) [VEV] {}; 
\draw[double distance=4mm,thick] (v1) -- (v4);
\draw[double distance=4mm,thick] (v1) -- (v3);
\draw[-] (v1) -- (l1);
\draw[-] (v4) -- (l2);
\draw[-] (v3) -- (l3);
\draw[-] (v3) -- (l4);
\end{tikzpicture}
%
+
%
 \begin{tikzpicture}[scale=2,baseline={([yshift=-.4ex]current bounding box.center)}]
\node (v1) at (0,0.3) [Gamma] {}; 
\node (v3) at (0.6,0.3) [Pi] {}; 
\node (v4) at (0.0,0.9) [Pi] {}; 
\node (l1) at (-0.4,0.3) [VEV] {}; 
\node (l2) at (0.,1.3) [VEV] {}; 
\node (l3) at (0.9,0.3) [VEV] {}; 
\node (l4) at (0.6,-0.1) [VEV] {}; 
\draw (v1) -- (v3);
\draw (v4) -- (v3);
\draw (v1) -- (v4);
\draw[-] (v1) -- (l1);
\draw[-] (v4) -- (l2);
\draw[-] (v3) -- (l3);
\draw[-] (v3) -- (l4);
\end{tikzpicture}
%
\nonumber \\
%
& + \frac{1}{2} \,\,
%
 \begin{tikzpicture}[scale=2,baseline={([yshift=-.4ex]current bounding box.center)}]
\node (v1) at (0,0.3) [Gamma] {}; 
\node (v2) at (0.6,0.3) [Pi] {}; 
\node (l1) at (-0.4,0.3) [VEV] {}; 
\node (l2) at (0.6,0.7) [VEV] {}; 
\node (l3) at (0.9,0.3) [VEV] {}; 
\node (l4) at (0.6,-0.1) [VEV] {}; 
\draw[double distance=4mm,thick] (v1) -- (v2);
\draw[-] (v1) -- (l1);
\draw[-] (v2) -- (l2);
\draw[-] (v2) -- (l3);
\draw[-] (v2) -- (l4);
\end{tikzpicture} \,\,\,.
\label{eq:dGamma4_dphi4}
\end{align}
The diagrams, where the $\Gamma^{\text{2PI}}_{\text{int}}$ blob is marked in red, do not contribute in the Hartree approximation due to the limited number of terms contained in $\Gamma^{\text{2PI}}_{\text{int}}$. This reduces the number of diagrams one has to consider by about half but this only works for the 2PI kernels and not for the self-energy boxes which adhere to their own integral equations. We have seen that derivatives of the self-energy with respect to one, two and three fields $\phi_R$ are present. The single derivative is given in \eqref{eq:dPidphi}. The other two can be expressed using diagrams that contain only the first derivative of the self-energy. 
We will continue to mark $\Gamma^{\text{2PI}}_{\text{int}}$ in red if the contribution vanishes in the Hartree approximation, and permutations of field indices are implied wherever they lead to non-equivalent topologies.
\begin{align}
 \begin{tikzpicture}[scale=2,baseline={([yshift=-.4ex]current bounding box.center)}]
\node (v1) at (0.0,0.) [Pi] {}; 
\node (l1) at (-0.4,0) {}; 
\node (l2) at (0.4,0.) [VEV] {}; 
\node (l3) at (0,-0.4) [VEV] {}; 
\draw[double distance=4mm,thick] (l1) -- (v1);
\draw[-] (v1) -- (l2);
\draw[-] (v1) -- (l3);
\end{tikzpicture}
%
= & \,\,
%
 \begin{tikzpicture}[scale=2,baseline={([yshift=-.4ex]current bounding box.center)}]
\node (v1) at (0.0,0.) [Gamma] {}; 
\node (l1) at (-0.4,0) {}; 
\node (l2) at (0.4,0.) [VEV] {}; 
\node (l3) at (0,-0.4) [VEV] {}; 
\draw[double distance=4mm,thick] (l1) -- (v1);
\draw[-] (v1) -- (l2);
\draw[-] (v1) -- (l3);
\end{tikzpicture}
%
+ \frac{1}{2} \,
%
 \begin{tikzpicture}[scale=2,baseline={([yshift=-.4ex]current bounding box.center)}]
\node (v1) at (0.0,0.) [Gamma] {}; 
\node (v2) at (-0.6,0.) [VR] {\,$\overline{V}^{(4)}$\,}; 
\node (l1) at (-1.,0) {}; 
\node (l2) at (0.4,0.) [VEV] {}; 
\node (l3) at (0,-0.4) [VEV] {}; 
\draw[double distance=4mm,thick] (l1) -- (v2);
\draw[double distance=4mm,thick] (v1) -- (v2);
\draw[-] (v1) -- (l2);
\draw[-] (v1) -- (l3);
\end{tikzpicture}
% 
\nonumber \\ & 
%
+ \frac{1}{2} \,
%
 \begin{tikzpicture}[scale=2,baseline={([yshift=-.4ex]current bounding box.center)}]
\node (v1) at (0.0,0.) [Gamma,fill=red] {}; 
\node (v3) at (0.0,-0.6) [Pi] {}; 
\node (l1) at (-0.4,0) {}; 
\node (l2) at (0.4,0.) [VEV] {}; 
\node (l3) at (0,-1.0) [VEV] {}; 
\draw[double distance=4mm,thick] (l1) -- (v1);
\draw[double distance=4mm,thick] (v1) -- (v3);
\draw[-] (v1) -- (l2);
\draw[-] (v3) -- (l3);
\end{tikzpicture}
%
+ \frac{1}{4} \,
%
 \begin{tikzpicture}[scale=2,baseline={([yshift=-.4ex]current bounding box.center)}]
\node (v1) at (0.0,0.) [Gamma,fill=red] {}; 
\node (v2) at (-0.6,0.) [VR] {\,$\overline{V}^{(4)}$\,}; 
\node (v3) at (0.0,-0.6) [Pi] {}; 
\node (l1) at (-1.,0) {}; 
\node (l2) at (0.4,0.) [VEV] {}; 
\node (l3) at (0,-1.0) [VEV] {}; 
\draw[double distance=4mm,thick] (l1) -- (v2);
\draw[double distance=4mm,thick] (v1) -- (v2);
\draw[double distance=4mm,thick] (v1) -- (v3);
\draw[-] (v1) -- (l2);
\draw[-] (v3) -- (l3);
\end{tikzpicture}
% 
\nonumber \\ & 
%
+ \frac{1}{4} \,
%
 \begin{tikzpicture}[scale=2,baseline={([yshift=-.4ex]current bounding box.center)}]
\node (v1) at (0.0,0.) [Gamma,fill=red] {}; 
\node (v3) at (0.0,-0.6) [Pi] {}; 
\node (v4) at (0.6,0.) [Pi] {}; 
\node (l1) at (-0.4,0) {}; 
\node (l2) at (1.,0.) [VEV] {}; 
\node (l3) at (0,-1.0) [VEV] {}; 
\draw[double distance=4mm,thick] (l1) -- (v1);
\draw[double distance=4mm,thick] (v1) -- (v3);
\draw[double distance=4mm,thick] (v1) -- (v4);
\draw[-] (v4) -- (l2);
\draw[-] (v3) -- (l3);
\end{tikzpicture}
%
+ \frac{1}{8} \,
%
 \begin{tikzpicture}[scale=2,baseline={([yshift=-.4ex]current bounding box.center)}]
\node (v1) at (0.0,0.) [Gamma,fill=red] {}; 
\node (v2) at (-0.6,0.) [VR] {\,$\overline{V}^{(4)}$\,}; 
\node (v3) at (0.0,-0.6) [Pi] {}; 
\node (v4) at (0.6,0.) [Pi] {}; 
\node (l1) at (-1.,0) {}; 
\node (l2) at (1.,0.) [VEV] {}; 
\node (l3) at (0,-1.0) [VEV] {}; 
\draw[double distance=4mm,thick] (l1) -- (v2);
\draw[double distance=4mm,thick] (v1) -- (v2);
\draw[double distance=4mm,thick] (v1) -- (v3);
\draw[double distance=4mm,thick] (v1) -- (v4);
\draw[-] (v4) -- (l2);
\draw[-] (v3) -- (l3);
\end{tikzpicture}
% 
\nonumber \\ & 
%
+  \,
%
 \begin{tikzpicture}[scale=2,baseline={([yshift=-.4ex]current bounding box.center)}]
\node (v1) at (0.0,0.) [Gamma] {}; 
\node (v3) at (0.0,-0.6) [Pi] {}; 
\node (v4) at (0.6,0.) [Pi] {}; 
\node (l1) at (-0.4,0) {}; 
\node (l2) at (1.,0.) [VEV] {}; 
\node (l3) at (0,-1.0) [VEV] {}; 
\draw[double distance=4mm,thick] (l1) -- (v1);
\draw (v1) -- (v3);
\draw (v1) -- (v4);
\draw (v3) -- (v4);
\draw[-] (v4) -- (l2);
\draw[-] (v3) -- (l3);
\end{tikzpicture}
%
+ \frac{1}{2} \,
%
 \begin{tikzpicture}[scale=2,baseline={([yshift=-.4ex]current bounding box.center)}]
\node (v1) at (0.0,0.) [Gamma] {}; 
\node (v2) at (-0.6,0.) [VR] {\,$\overline{V}^{(4)}$\,}; 
\node (v3) at (0.0,-0.6) [Pi] {}; 
\node (v4) at (0.6,0.) [Pi] {}; 
\node (l1) at (-1.,0) {}; 
\node (l2) at (1.,0.) [VEV] {}; 
\node (l3) at (0,-1.0) [VEV] {}; 
\draw[double distance=4mm,thick] (l1) -- (v2);
\draw[double distance=4mm,thick] (v1) -- (v2);
\draw (v1) -- (v3);
\draw (v1) -- (v4);
\draw (v3) -- (v4);
\draw[-] (v4) -- (l2);
\draw[-] (v3) -- (l3);
\end{tikzpicture}
\label{eq:d2_Pi_dphi2}
\end{align}
For the triple derivative $d^3 \overline{\Pi} / d \phi^3_R$, it is more convenient to not insert instances of the double derivative $d^3 \overline{\Pi} / d \phi^2_R$ as the expressions get too lengthy otherwise. One simply needs to substitute the corresponding diagrams from \eqref{eq:d2_Pi_dphi2} at places where boxes with two field derivatives appear. One then finds
\begin{align}
 \begin{tikzpicture}[scale=2,baseline={([yshift=-.4ex]current bounding box.center)}]
\node (v1) at (0.0,0.) [Pi] {}; 
\node (l1) at (-0.4,0) {}; 
\node (l2) at (0.4,0.) [VEV] {}; 
\node (l3) at (0,-0.4) [VEV] {}; 
\node (l4) at (0,0.4) [VEV] {}; 
\draw[double distance=4mm,thick] (l1) -- (v1);
\draw[-] (v1) -- (l2);
\draw[-] (v1) -- (l3);
\draw[-] (v1) -- (l4);
\end{tikzpicture}
%
= & \,\,
%
 \begin{tikzpicture}[scale=2,baseline={([yshift=-.4ex]current bounding box.center)}]
\node (v1) at (0.0,0.) [Gamma,fill=red] {}; 
\node (l1) at (-0.4,0) {}; 
\node (l2) at (0.4,0.) [VEV] {}; 
\node (l3) at (0,-0.4) [VEV] {}; 
\node (l4) at (0,0.4) [VEV] {}; 
\draw[double distance=4mm,thick] (l1) -- (v1);
\draw[-] (v1) -- (l2);
\draw[-] (v1) -- (l3);
\draw[-] (v1) -- (l4);
\end{tikzpicture}
%
+ \frac{1}{2} \,
%
 \begin{tikzpicture}[scale=2,baseline={([yshift=-.4ex]current bounding box.center)}]
\node (v1) at (0.0,0.) [Gamma,fill=red] {}; 
\node (v2) at (-0.6,0.) [VR] {\,$\overline{V}^{(4)}$\,}; 
\node (l1) at (-1.,0) {}; 
\node (l2) at (0.4,0.) [VEV] {}; 
\node (l3) at (0,-0.4) [VEV] {}; 
\node (l4) at (0,0.4) [VEV] {}; 
\draw[double distance=4mm,thick] (l1) -- (v2);
\draw[double distance=4mm,thick] (v1) -- (v2);
\draw[-] (v1) -- (l2);
\draw[-] (v1) -- (l3);
\draw[-] (v1) -- (l4);
\end{tikzpicture}
% 
\nonumber \\ & 
%
+ \frac{1}{2} \,
%
 \begin{tikzpicture}[scale=2,baseline={([yshift=-.4ex]current bounding box.center)}]
\node (v1) at (0.0,0.) [Gamma,fill=red] {}; 
\node (v3) at (0.0,-0.6) [Pi] {}; 
\node (l1) at (-0.4,0) {}; 
\node (l2) at (0.4,0.) [VEV] {}; 
\node (l3) at (0,-1.0) [VEV] {}; 
\node (l4) at (0,0.4) [VEV] {}; 
\draw[double distance=4mm,thick] (l1) -- (v1);
\draw[double distance=4mm,thick] (v1) -- (v3);
\draw[-] (v1) -- (l2);
\draw[-] (v3) -- (l3);
\draw[-] (v1) -- (l4);
\end{tikzpicture}
%
+ \frac{1}{4} \,
%
 \begin{tikzpicture}[scale=2,baseline={([yshift=-.4ex]current bounding box.center)}]
\node (v1) at (0.0,0.) [Gamma,fill=red] {}; 
\node (v2) at (-0.6,0.) [VR] {\,$\overline{V}^{(4)}$\,}; 
\node (v3) at (0.0,-0.6) [Pi] {}; 
\node (l1) at (-1.,0) {}; 
\node (l2) at (0.4,0.) [VEV] {}; 
\node (l3) at (0,-1.0) [VEV] {}; 
\node (l4) at (0,0.4) [VEV] {}; 
\draw[double distance=4mm,thick] (l1) -- (v2);
\draw[double distance=4mm,thick] (v1) -- (v2);
\draw[double distance=4mm,thick] (v1) -- (v3);
\draw[-] (v1) -- (l2);
\draw[-] (v3) -- (l3);
\draw[-] (v1) -- (l4);
\end{tikzpicture}
% 
\nonumber \\ & 
%
+ \frac{1}{4} \,
%
 \begin{tikzpicture}[scale=2,baseline={([yshift=-.4ex]current bounding box.center)}]
\node (v1) at (0.0,0.) [Gamma,fill=red] {}; 
\node (v3) at (0.0,-0.6) [Pi] {}; 
\node (v4) at (0.6,0.) [Pi] {}; 
\node (l1) at (-0.4,0) {}; 
\node (l2) at (1.,0.) [VEV] {}; 
\node (l3) at (0,-1.0) [VEV] {}; 
\node (l4) at (0,0.4) [VEV] {}; 
\draw[double distance=4mm,thick] (l1) -- (v1);
\draw[double distance=4mm,thick] (v1) -- (v3);
\draw[double distance=4mm,thick] (v1) -- (v4);
\draw[-] (v4) -- (l2);
\draw[-] (v3) -- (l3);
\draw[-] (v1) -- (l4);
\end{tikzpicture}
%
+ \frac{1}{8} \,
%
 \begin{tikzpicture}[scale=2,baseline={([yshift=-.4ex]current bounding box.center)}]
\node (v1) at (0.0,0.) [Gamma,fill=red] {}; 
\node (v2) at (-0.6,0.) [VR] {\,$\overline{V}^{(4)}$\,}; 
\node (v3) at (0.0,-0.6) [Pi] {}; 
\node (v4) at (0.6,0.) [Pi] {}; 
\node (l1) at (-1.,0) {}; 
\node (l2) at (1.,0.) [VEV] {}; 
\node (l3) at (0,-1.0) [VEV] {}; 
\node (l4) at (0,0.4) [VEV] {}; 
\draw[double distance=4mm,thick] (l1) -- (v2);
\draw[double distance=4mm,thick] (v1) -- (v2);
\draw[double distance=4mm,thick] (v1) -- (v3);
\draw[double distance=4mm,thick] (v1) -- (v4);
\draw[-] (v4) -- (l2);
\draw[-] (v3) -- (l3);
\draw[-] (v1) -- (l4);
\end{tikzpicture}
% 
\nonumber \\ & 
%
%
+ \frac{1}{8} \,
%
 \begin{tikzpicture}[scale=2,baseline={([yshift=-.4ex]current bounding box.center)}]
\node (v1) at (0.0,0.) [Gamma,fill=red] {}; 
\node (v3) at (0.0,-0.6) [Pi] {}; 
\node (v4) at (0.6,0.) [Pi] {}; 
\node (v5) at (0.,0.6) [Pi] {}; 
\node (l1) at (-0.4,0) {}; 
\node (l2) at (1.,0.) [VEV] {}; 
\node (l3) at (0,-1.0) [VEV] {}; 
\node (l4) at (0,1.) [VEV] {}; 
\draw[double distance=4mm,thick] (l1) -- (v1);
\draw[double distance=4mm,thick] (v1) -- (v3);
\draw[double distance=4mm,thick] (v1) -- (v4);
\draw[double distance=4mm,thick] (v1) -- (v5);
\draw[-] (v4) -- (l2);
\draw[-] (v3) -- (l3);
\draw[-] (v5) -- (l4);
\end{tikzpicture}
%
+ \frac{1}{16} \,
%
 \begin{tikzpicture}[scale=2,baseline={([yshift=-.4ex]current bounding box.center)}]
\node (v1) at (0.0,0.) [Gamma,fill=red] {}; 
\node (v2) at (-0.6,0.) [VR] {\,$\overline{V}^{(4)}$\,}; 
\node (v3) at (0.0,-0.6) [Pi] {}; 
\node (v4) at (0.6,0.) [Pi] {}; 
\node (v5) at (0.,0.6) [Pi] {}; 
\node (l1) at (-1.,0) {}; 
\node (l2) at (1.,0.) [VEV] {}; 
\node (l3) at (0,-1.0) [VEV] {}; 
\node (l4) at (0,1.) [VEV] {}; 
\draw[double distance=4mm,thick] (l1) -- (v2);
\draw[double distance=4mm,thick] (v1) -- (v2);
\draw[double distance=4mm,thick] (v1) -- (v3);
\draw[double distance=4mm,thick] (v1) -- (v4);
\draw[double distance=4mm,thick] (v1) -- (v5);
\draw[-] (v4) -- (l2);
\draw[-] (v3) -- (l3);
\draw[-] (v5) -- (l4);
\end{tikzpicture}
% 
\nonumber \\ & 
%
+  \,
%
 \begin{tikzpicture}[scale=2,baseline={([yshift=-.4ex]current bounding box.center)}]
\node (v1) at (0.0,0.) [Gamma,fill=red] {}; 
\node (v3) at (0.0,-0.6) [Pi] {}; 
\node (v4) at (0.6,0.) [Pi] {}; 
\node (l1) at (-0.4,0) {}; 
\node (l2) at (1.,0.) [VEV] {}; 
\node (l3) at (0,-1.0) [VEV] {}; 
\node (l4) at (0,0.4) [VEV] {}; 
\draw[double distance=4mm,thick] (l1) -- (v1);
\draw (v1) -- (v3);
\draw (v1) -- (v4);
\draw (v3) -- (v4);
\draw[-] (v4) -- (l2);
\draw[-] (v3) -- (l3);
\draw[-] (v1) -- (l4);
\end{tikzpicture}
%
+ \frac{1}{2} \,
%
 \begin{tikzpicture}[scale=2,baseline={([yshift=-.4ex]current bounding box.center)}]
\node (v1) at (0.0,0.) [Gamma,fill=red] {}; 
\node (v2) at (-0.6,0.) [VR] {\,$\overline{V}^{(4)}$\,}; 
\node (v3) at (0.0,-0.6) [Pi] {}; 
\node (v4) at (0.6,0.) [Pi] {}; 
\node (l1) at (-1.,0) {}; 
\node (l2) at (1.,0.) [VEV] {}; 
\node (l3) at (0,-1.0) [VEV] {}; 
\node (l4) at (0,0.4) [VEV] {}; 
\draw[double distance=4mm,thick] (l1) -- (v2);
\draw[double distance=4mm,thick] (v1) -- (v2);
\draw (v1) -- (v3);
\draw (v1) -- (v4);
\draw (v3) -- (v4);
\draw[-] (v4) -- (l2);
\draw[-] (v3) -- (l3);
\draw[-] (v1) -- (l4);
\end{tikzpicture}
%
%
\nonumber \\ &
%
+  \frac{1}{2} \,
%
 \begin{tikzpicture}[scale=2,baseline={([yshift=-.4ex]current bounding box.center)}]
\node (v1) at (0.0,0.) [Gamma,fill=red] {}; 
\node (v3) at (0.0,-0.6) [Pi] {}; 
\node (v4) at (0.6,0.) [Pi] {}; 
\node (v5) at (0.,0.6) [Pi] {}; 
\node (l1) at (-0.4,0) {}; 
\node (l2) at (1.,0.) [VEV] {}; 
\node (l3) at (0,-1.0) [VEV] {}; 
\node (l4) at (0,1.) [VEV] {}; 
\draw[double distance=4mm,thick] (l1) -- (v1);
\draw[double distance=4mm,thick] (v1) -- (v5);
\draw (v1) -- (v3);
\draw (v1) -- (v4);
\draw (v3) -- (v4);
\draw[-] (v4) -- (l2);
\draw[-] (v3) -- (l3);
\draw[-] (v5) -- (l4);
\end{tikzpicture}
%
+ \frac{1}{4} \,
%
 \begin{tikzpicture}[scale=2,baseline={([yshift=-.4ex]current bounding box.center)}]
\node (v1) at (0.0,0.) [Gamma,fill=red] {}; 
\node (v2) at (-0.6,0.) [VR] {\,$\overline{V}^{(4)}$\,}; 
\node (v3) at (0.0,-0.6) [Pi] {}; 
\node (v4) at (0.6,0.) [Pi] {}; 
\node (v5) at (0.,0.6) [Pi] {}; 
\node (l1) at (-1.,0) {}; 
\node (l2) at (1.,0.) [VEV] {}; 
\node (l3) at (0,-1.0) [VEV] {}; 
\node (l4) at (0,1.) [VEV] {}; 
\draw[double distance=4mm,thick] (l1) -- (v2);
\draw[double distance=4mm,thick] (v1) -- (v2);
\draw[double distance=4mm,thick] (v1) -- (v5);
\draw (v1) -- (v3);
\draw (v1) -- (v4);
\draw (v3) -- (v4);
\draw[-] (v4) -- (l2);
\draw[-] (v3) -- (l3);
\draw[-] (v5) -- (l4);
\end{tikzpicture}
%
\nonumber 
\end{align}
%
\begin{align}
%
& +   \,
%
 \begin{tikzpicture}[scale=2,baseline={([yshift=-.4ex]current bounding box.center)}]
\node (v1) at (0.0,0.) [Gamma] {}; 
\node (v3) at (0.0,-0.6) [Pi] {}; 
\node (v4) at (0.6,0.) [Pi] {}; 
\node (v5) at (0.,0.6) [Pi] {}; 
\node (l1) at (-0.4,0) {}; 
\node (l2) at (1.,0.) [VEV] {}; 
\node (l3) at (0,-1.0) [VEV] {}; 
\node (l4) at (0,1.) [VEV] {}; 
\draw[double distance=4mm,thick] (l1) -- (v1);
\draw (v1) -- (v5);
\draw (v1) -- (v3);
\draw (v5) -- (v4);
\draw (v3) -- (v4);
\draw[-] (v4) -- (l2);
\draw[-] (v3) -- (l3);
\draw[-] (v5) -- (l4);
\end{tikzpicture}
%
+ \frac{1}{2} \,
%
 \begin{tikzpicture}[scale=2,baseline={([yshift=-.4ex]current bounding box.center)}]
\node (v1) at (0.0,0.) [Gamma] {}; 
\node (v2) at (-0.6,0.) [VR] {\,$\overline{V}^{(4)}$\,}; 
\node (v3) at (0.0,-0.6) [Pi] {}; 
\node (v4) at (0.6,0.) [Pi] {}; 
\node (v5) at (0.,0.6) [Pi] {}; 
\node (l1) at (-1.,0) {}; 
\node (l2) at (1.,0.) [VEV] {}; 
\node (l3) at (0,-1.0) [VEV] {}; 
\node (l4) at (0,1.) [VEV] {}; 
\draw[double distance=4mm,thick] (l1) -- (v2);
\draw[double distance=4mm,thick] (v1) -- (v2);
\draw (v1) -- (v5);
\draw (v1) -- (v3);
\draw (v5) -- (v4);
\draw (v3) -- (v4);
\draw[-] (v4) -- (l2);
\draw[-] (v3) -- (l3);
\draw[-] (v5) -- (l4);
\end{tikzpicture}
% 
\nonumber \\ & 
%
+ \frac{1}{2} \,
%
 \begin{tikzpicture}[scale=2,baseline={([yshift=-.4ex]current bounding box.center)}]
\node (v1) at (0.0,0.) [Gamma,fill=red] {}; 
\node (v3) at (0.0,-0.6) [Pi] {}; 
\node (l1) at (-0.4,0) {}; 
\node (l2) at (0.4,-0.6) [VEV] {}; 
\node (l3) at (0,-1.0) [VEV] {}; 
\node (l4) at (0,0.4) [VEV] {}; 
\draw[double distance=4mm,thick] (l1) -- (v1);
\draw[double distance=4mm,thick] (v1) -- (v3);
\draw[-] (v3) -- (l2);
\draw[-] (v3) -- (l3);
\draw[-] (v1) -- (l4);
\end{tikzpicture}
%
+ \frac{1}{4} \,
%
 \begin{tikzpicture}[scale=2,baseline={([yshift=-.4ex]current bounding box.center)}]
\node (v1) at (0.0,0.) [Gamma,fill=red] {}; 
\node (v2) at (-0.6,0.) [VR] {\,$\overline{V}^{(4)}$\,}; 
\node (v3) at (0.0,-0.6) [Pi] {}; 
\node (l1) at (-1.,0) {}; 
\node (l2) at (0.4,-0.6) [VEV] {}; 
\node (l3) at (0,-1.0) [VEV] {}; 
\node (l4) at (0,0.4) [VEV] {}; 
\draw[double distance=4mm,thick] (l1) -- (v2);
\draw[double distance=4mm,thick] (v1) -- (v2);
\draw[double distance=4mm,thick] (v1) -- (v3);
\draw[-] (v3) -- (l2);
\draw[-] (v3) -- (l3);
\draw[-] (v1) -- (l4);
\end{tikzpicture}
% 
\nonumber \\ & 
%
%
+ \frac{1}{4} \,
%
 \begin{tikzpicture}[scale=2,baseline={([yshift=-.4ex]current bounding box.center)}]
\node (v1) at (0.0,0.) [Gamma,fill=red] {}; 
\node (v3) at (0.0,-0.6) [Pi] {}; 
\node (v5) at (0.0,0.6) [Pi] {}; 
\node (l1) at (-0.4,0) {}; 
\node (l2) at (0.4,-0.6) [VEV] {}; 
\node (l3) at (0,-1.0) [VEV] {}; 
\node (l4) at (0,1.) [VEV] {}; 
\draw[double distance=4mm,thick] (l1) -- (v1);
\draw[double distance=4mm,thick] (v1) -- (v3);
\draw[double distance=4mm,thick] (v1) -- (v5);
\draw[-] (v3) -- (l2);
\draw[-] (v3) -- (l3);
\draw[-] (v5) -- (l4);
\end{tikzpicture}
%
+ \frac{1}{8} \,
%
 \begin{tikzpicture}[scale=2,baseline={([yshift=-.4ex]current bounding box.center)}]
\node (v1) at (0.0,0.) [Gamma,fill=red] {}; 
\node (v2) at (-0.6,0.) [VR] {\,$\overline{V}^{(4)}$\,}; 
\node (v3) at (0.0,-0.6) [Pi] {}; 
\node (v5) at (0.0,0.6) [Pi] {}; 
\node (l1) at (-1.,0) {}; 
\node (l2) at (0.4,-0.6) [VEV] {}; 
\node (l3) at (0,-1.0) [VEV] {}; 
\node (l4) at (0,1.) [VEV] {}; 
\draw[double distance=4mm,thick] (l1) -- (v2);
\draw[double distance=4mm,thick] (v1) -- (v2);
\draw[double distance=4mm,thick] (v1) -- (v3);
\draw[double distance=4mm,thick] (v1) -- (v5);
\draw[-] (v3) -- (l2);
\draw[-] (v3) -- (l3);
\draw[-] (v5) -- (l4);
\end{tikzpicture}
% 
\nonumber \\ & 
%
+  \,
%
 \begin{tikzpicture}[scale=2,baseline={([yshift=-.4ex]current bounding box.center)}]
\node (v1) at (0.0,0.) [Gamma] {}; 
\node (v3) at (0.0,-0.6) [Pi] {}; 
\node (v4) at (0.6,0.) [Pi] {}; 
\node (l1) at (-0.4,0) {}; 
\node (l2) at (1.,0.) [VEV] {}; 
\node (l3) at (0,-1.0) [VEV] {}; 
\node (l4) at (0.6,0.4) [VEV] {}; 
\draw[double distance=4mm,thick] (l1) -- (v1);
\draw (v1) -- (v3);
\draw (v1) -- (v4);
\draw (v3) -- (v4);
\draw[-] (v4) -- (l2);
\draw[-] (v3) -- (l3);
\draw[-] (v4) -- (l4);
\end{tikzpicture}
%
+ \frac{1}{2} \,
%
 \begin{tikzpicture}[scale=2,baseline={([yshift=-.4ex]current bounding box.center)}]
\node (v1) at (0.0,0.) [Gamma] {}; 
\node (v2) at (-0.6,0.) [VR] {\,$\overline{V}^{(4)}$\,}; 
\node (v3) at (0.0,-0.6) [Pi] {}; 
\node (v4) at (0.6,0.) [Pi] {}; 
\node (l1) at (-1.,0) {}; 
\node (l2) at (1.,0.) [VEV] {}; 
\node (l3) at (0,-1.0) [VEV] {}; 
\node (l4) at (0.6,0.4) [VEV] {}; 
\draw[double distance=4mm,thick] (l1) -- (v2);
\draw[double distance=4mm,thick] (v1) -- (v2);
\draw (v1) -- (v3);
\draw (v1) -- (v4);
\draw (v3) -- (v4);
\draw[-] (v4) -- (l2);
\draw[-] (v3) -- (l3);
\draw[-] (v4) -- (l4);
\end{tikzpicture} \,\,.
\label{eq:d3_Pi_dphi3}
\end{align}
Even though many of the diagrams vanish in the Hartree approximation, there are still quite a few to evaluate. However, before doing so explicitly, one can show that some of them only contribute at $O(\epsilon)$. This can be seen as follows: potential topologies that can contribute in any of the $\Gamma^{\text{2PI}}_{\text{int}}$ blobs that appear in loops and do not vanish is $\sim (\lambda_R + \delta\lambda_n)$, $n = 0, 2$. Due to the shifts \eqref{eqn:os_dl0_exp}, this gives a factor that is always $O(\epsilon)$. Other factors of $\epsilon$ come from the loop integrals: one- and two-point integrals are $O(\epsilon^{-1})$ and $n$-point functions with $n \ge 3$ are $O(1)$. Finally, the vertex function $\overline{V}^{(4)}$ is of course $O(1)$. Thus, one can count the powers of $\epsilon$ for each diagram and, as it turns out, many diagrams only contribute at $O(\epsilon)$ to the total four-point function. 

This analysis can be first applied to the self-energy boxes, revealing that $\delta^n \overline{\Pi} / \delta \phi^n_R$ $(n = 1, 2, 3)$ are all $O(1)$. In \eqref{eq:d2_Pi_dphi2}, one finds that only the second and last terms are finite and the other two non-vanishing ones give $O(\epsilon)$ contributions. In the case of \eqref{eq:d3_Pi_dphi3}, an analogous analysis reveals that only two contribute: the fourteenth and the last one. Finally, we return to the four-point function \eqref{eq:dGamma4_dphi4} and applying the same analysis, we find that the first, eighth and last diagrams contribute at $O(1)$. 

There are 3- and 4-point loop integrals among these diagrams which, in dimensional regularisation, are the well-known $C_0$ and $D_0$ functions
\begin{align}
C_0(p_1,p_2) &=  (16\pi^2) \int_q G_R(q) G_R(q+p_1)  G_R(q+p_2) \\ 
D_0(p_1,p_2,p_3) &= -i (16\pi^2) \int_q G_R(q) G_R(q+p_1)  G_R(q+p_2) G_R(q+p_4)  \,.
\end{align}
Note that both functions are ultraviolet finite and explicit expressions can be found, for example in \cite{Denner:1991kt}. 

Assembling the different contributions, with the momentum assignments $p_{1,2}$ incoming and $p_{3,4}$ outgoing, we obtain
\begin{align}
\label{eqn:gamma4_ht}
&\Gamma^{(4)}(p_1,p_2,p_3,p_4) = -(\lambda_R + \delta \lambda_4)
+ \lambda_R \left[J(p_1+p_2) +J(p_1-p_3) +J(p_1-p_4) \right] \nonumber \\[2mm]
& + \frac{\lambda^3_R \phi^2_R}{16\pi^2} 
\big[J(p_1+p_2) J(p_3)J(p_4)C_0(p_1+p_2,p_4) + J(p_1-p_3)J(p_2)J(p_4)C_0(p_1-p_3,p_4) \nonumber \\[2mm]
& \hspace{13mm}+ J(p_1-p_4) J(p_2) J(p_3) C_0(p_1-p_4,p_3) + J(p_1) J(p_2) J(p_3+p_4) C_0(p_1,p_3+p_4)  \nonumber \\[2mm]
& \hspace{13mm}+ J(p_1) J(p_3) J(p_2-p_4)C_0(p_1,p_2-p_4) + J(p_1)J(p_4)J(p_2-p_3)C_0(p_1,p_2-p_3)  \big]\nonumber \\[2mm]
& - \frac{\lambda^4_R \phi^4_R}{16\pi^2} \,J(p_1) J(p_2) J(p_3) J(p_4) \big[D_0(p_2,p_1+p_2,p_3) + D_0(p_2,p_2-p_3,p_3)+ D_0(p_2,p_2-p_4,p_4)\big] \nonumber \\[2mm]
& + \frac{\lambda^5_R \phi^4_R}{(16\pi^2)^2} \,J(p_1) J(p_2) J(p_3) J(p_4) \big[ J(p_1+p_2) C_0(p_1,p_1+p_2) C_0(p_3+p_4,p_4)  \nonumber \\[2mm]
& \hspace{5mm} +  J(p_1-p_3) C_0(p_1,p_1-p_3) C_0(p_4-p_2,p_4) +  J(p_1-p_4) C_0(p_1,p_1-p_4) C_0(p_3-p_2,p_3)  \big] \,,
\end{align} 
where we have introduced
\begin{align}
J(p) \equiv \left[ 1- \frac{\lambda_R}{32\pi^2} \left( B_0(p^2_{\ast},m^2_R,m^2_R) - B_0(p^2,m^2_R,m^2_R)
\right) \right]^{-1}
\end{align}
which originates from the vertex function. The counterterm, $\delta \lambda_4$ can then easily be obtained by the renormalization condition \eqref{eq:renor_cond_fourpt}.

For completeness, one should also note that only in the broken phase $(\phi_R \neq 0)$, there exists a physical three-point function, for which we have the following topologies
\begin{align}
\Gamma^{(3)} \equiv \frac{\delta^3 \Gamma}{\delta \phi^3} = & \quad
%
 \begin{tikzpicture}[scale=2,baseline={([yshift=-.4ex]current bounding box.center)}]
\node (fourpoint) at (0,0) [Gamma] {}; 
\node (l1) at (-0.4,0) [VEV] {}; 
\node (l2) at (0.,0.4) [VEV] {}; 
\node (l3) at (0.4,0) [VEV] {}; 
\draw[-] (fourpoint) -- (l1);
\draw[-] (fourpoint) -- (l2);
\draw[-] (fourpoint) -- (l3);
\end{tikzpicture}
%
+
%
\frac{1}{2} \,\,
 \begin{tikzpicture}[scale=2,baseline={([yshift=-.4ex]current bounding box.center)}]
\node (v1) at (0,0.3) [Gamma] {}; 
\node (v3) at (0.6,0.3) [Pi] {}; 
\node (l1) at (-0.4,0.3) [VEV] {}; 
\node (l2) at (0.,0.7) [VEV] {}; 
\node (l3) at (0.9,0.3) [VEV] {}; 
\draw[double distance=4mm,thick] (v1) -- (v3);
\draw[-] (v1) -- (l1);
\draw[-] (v1) -- (l2);
\draw[-] (v3) -- (l3);
\end{tikzpicture}
\nonumber \\[8mm]
&+
\frac{1}{4} \,\,
 \begin{tikzpicture}[scale=2,baseline={([yshift=-.4ex]current bounding box.center)}]
\node (v1) at (0,0.3) [Gamma, fill=red] {}; 
\node (v2) at (0.6,0.3) [Pi] {}; 
\node (v3) at (0.0,0.9) [Pi] {}; 
\node (l1) at (-0.4,0.3) [VEV] {}; 
\node (l2) at (0.,1.3) [VEV] {}; 
\node (l3) at (0.9,0.3) [VEV] {}; 
\draw[double distance=4mm,thick] (v1) -- (v2);
\draw[double distance=4mm,thick] (v1) -- (v2);
\draw[double distance=4mm,thick] (v1) -- (v3);
\draw[-] (v1) -- (l1);
\draw[-] (v3) -- (l2);
\draw[-] (v2) -- (l3);
\end{tikzpicture}
%
%
+
%
 \begin{tikzpicture}[scale=2,baseline={([yshift=-.4ex]current bounding box.center)}]
\node (v1) at (0,0.3) [Gamma] {}; 
\node (v2) at (0,-0.3) [Pi] {}; 
\node (v3) at (0.6,0.3) [Pi] {}; 
\node (l1) at (-0.4,0.3) [VEV] {}; 
\node (l2) at (0.9,0.3) [VEV] {}; 
\node (l3) at (0.,-0.7) [VEV] {}; 
\draw (v1) -- (v2);
\draw (v1) -- (v3);
\draw (v2) -- (v3);
\draw[-] (v1) -- (l1);
\draw[-] (v3) -- (l2);
\draw[-] (v2) -- (l3);
\end{tikzpicture}
%
\nonumber \\[8mm]
%
& + \frac{1}{2} \,\,
%
 \begin{tikzpicture}[scale=2,baseline={([yshift=-.4ex]current bounding box.center)}]
\node (v1) at (0,0.3) [Gamma] {}; 
\node (v2) at (0.6,0.3) [Pi] {}; 
\node (l1) at (-0.4,0.3) [VEV] {}; 
\node (l2) at (0.9,0.3) [VEV] {}; 
\node (l3) at (0.6,-0.1) [VEV] {}; 
\draw[double distance=4mm,thick] (v1) -- (v2);
\draw[-] (v1) -- (l1);
\draw[-] (v2) -- (l2);
\draw[-] (v2) -- (l3);
\end{tikzpicture} \,\,\,\,.
\label{eq:dGamma3_dphi3}
\end{align}
In this case, we work with the convention that $p_1$ is the incoming four-momentum and $p_2$ and $p_3$ are outgoing. One then obtains the expression
\begin{align}
    \Gamma^{(3)}(p_1,p_2,p_3) &= -(\lambda_R+\delta\lambda_4)\phi_R + \frac{1}{2} \lambda_R\, \phi_R \left[J(p_1)+J(p_2)+J(p_3)\right]  \nonumber \\ &
    \quad - \frac{1}{2}\lambda^3_R\,\phi^3_R\, J(p_1)J(p_2)J(p_3)\frac{\left[C_0(p_1,p_2)+C_0(p_1,p_3)\right]}{16\pi^2} +\mathcal{O}(\epsilon)\,,
\end{align}
which is fully determined from the counterterm $\delta \lambda_4$.

%---------------------------------------------------------------------------------

%----------------------------------------------------------------------------------
%Trilinear Coupling
%----------------------------------------------------------------------------------
\subsection{Scalar Sunset Approximation}
Introducing a trilinear coupling, the effective action is now given by
\begin{align}
&\Gamma^{\text{2PI}}_{\text{int}}[\phi_R,G_R] =
- \frac{1}{2}\int_{x} (\delta Z_{\phi,0} \square_x + \delta m^2_0) G_R(x,y) \big|_{x=y}
- \frac{1}{2}\int_x \phi_R(x)  (\delta Z_{\phi,2} \square_x + \delta m^2_2) \phi_R(x)
  \nonumber \\ 
& - \int_x \left[ \frac{1}{8}(\lambda_R + \delta \lambda_0) G^2_R(x,x)
   + \frac{1}{4} (\lambda_R + \delta \lambda_2) G_R(x,x) \phi^2_R(x)
   + \frac{1}{4!} (\lambda_R + \delta \lambda_4) \phi^4_R(x)
\right] \nonumber \\
& - \int_x \delta t_1 \phi_R(x) - \int_x \left[\frac{1}{2}(\alpha_R + \delta \alpha_1) \phi_R(x)G_R(x,x)
   + \frac{1}{3!} (\alpha_R + \delta \alpha_3) \phi^3_R(x)
\right] \nonumber \\
& +\frac{\ii}{12} \int_x \int_y \left[(\alpha_R + \delta \alpha_0)+(\lambda_R + \delta \lambda_1)\phi(x)\right]\left[(\alpha_R + \delta \alpha_0)+(\lambda_R + \delta \lambda_1)\phi(y)\right] \,G^3_R(x,y)
\end{align}
In this case, we carry out the renormalization at non-vanishing field expectation value $\phi_R \neq 0$, as there is no symmetry in the model. 

We may set the counterterms, $\delta \alpha_0 = \delta \lambda_1 = 0$, which is possible as they amount to finite renormalizations at this level of the 2PI truncation; more specifically, the first non-trivial contribution to these are obtained when one includes the basketball diagram \cite{Patkos:2008ik,Pilaftsis:2013xna, Pilaftsis:2015cka}. We then obtain the following four-point kernels which are now momentum dependent
\begin{align}
    &\oL^{(4)}(p_1,p_2,p_3,p_4) = -(\lambda_R + \delta \lambda_0) + 2i(\alpha_R + \lambda_R \phi_R)^2 G_R(p_3-p_1) \,,
\label{eq:4ptkalpha1} \\[4mm]
    &\Lambda^{(4)}(p_1,p_2,p_3,p_4) = -(\lambda_R + \delta \lambda_2) + i \lambda_R^2 \int_q G_R(q)G_R(p+q) \nonumber \\[2mm]
    &\qquad \qquad \equiv -(\lambda_R + \delta \lambda_2) + \lambda_R^2\,\mathcal{I}(p)\,,
    \label{eq:4ptkalpha2}
\end{align}
where $p =p_1 +p_2$. In addition, we also define the following three-point kernel 
\begin{align}
    \Lambda^{(3)}(p_1,p_2,p_3) = 2 \frac{\delta^2 \Gamma^{\text{2PI}}_{\text{int}}}{\delta \phi_R\, \delta G_R(p_1)} = -(\alpha_R+\delta \alpha_1) +  \lambda_R (\alpha_R + \lambda_R\phi_R) \,\mathcal{I}(p_1)\,.
        \label{eq:3ptkalpha}
\end{align}
The corresponding BSE for its resummation is 
\begin{align}
	&V^{(3)}(p_1,p_2,p_3) = \Lambda^{(3)}(p_1,p_2,p_3) \nonumber \\[2mm]
	&\quad +  \frac{i}{2}\int_q \Lambda^{(3)}(p_1,p_2,p_3)  G_R(q) G_R(p_1+q)\,\overline{V}^{(4)}(q+p_1,-q,p_2,p_3)\,.
	\label{eqn:v3_ss}
\end{align}
%
We now implement appropriate renormalization conditions in order to determine the various coupling constant counterterms. For the four-point vertices, we continue to work with the convention that the we have $p_{1,2}$ as incoming and $p_{3,4}$ as outgoing momenta. We first have from \eqref{eq:BSEVbar},
\begin{align}
    &\overline{V}^{(4)}(p_1,p_2,p_3,p_4) = \overline{\Lambda}^{(4)}(p_1,p_2,p_3,p_4) \nonumber \\[2mm]
    &\qquad \qquad  + \frac{i}{2}\int_q \overline{\Lambda}^{(4)}(p_1,p_2,q+p,-q)\, G_R(q) G_R(p+q)\,\overline{V}^{(4)}(q+p,-q,p_3,p_4)\,.
    \label{4ptvalpha}
\end{align} 
Let us set the renormalization condition,
\begin{equation}
    \overline{V}^{(4)}(p_{1 \ast},p_{2 \ast},p_{3 \ast},p_{4 \ast}) = -\lambda_R + 2i (\alpha_R+\lambda_R\phi_R)^2 \,G_R(p_{3\ast}-p_{1\ast})\,,
    \label{eq:v4alpha_rc}
\end{equation}
which accounts for the fact that four scalars may scatter directly via the quartic coupling, or by a scalar exchange through two trilinear couplings. Like in the Hartree approximation, we work in the COM system, which is characterised by a COM momentum and scattering angle.
In this case, the second term induces a dependence
on the scattering angle and, thus, we set $|\vec{p_{\ast}}| = m_R$ and $\theta_{\ast} = \pi$ as our renormalization point. 
Using \eqref{eq:v4alpha_rc} and solving for $\delta \lambda_0$, we obtain
\begin{equation}
    \delta \lambda_0  = -\lambda_R + \frac{\lambda_R -  (\alpha_R+\lambda_R\phi_R)^2\, I_2(p_{\ast},p_{2 \ast},p_{3 \ast},p_{4 \ast})}{1+\frac{1}{2}I_1(p_{\ast},p_{3 \ast},p_{4 \ast})} \,,
    \label{eq:dllambda0_tri}
\end{equation}
where we have defined the following integrals over the vertex functions:
\begin{align}
    \label{eqn:vertexintegral1}
    I_1(p,k,r) &= i \int_q G_R(q)G_R(p+q)\overline{V}^{(4)}(q+p,-q,k,r) \,,\\[2mm]
     \label{eqn:vertexintegral2}
    I_2(p,k,r,l) &= \int_q G_R(q) G_R(q+p)G_R(q+k)G_R(q+r)\overline{V}^{(4)}(q+p,-q,r,l) \,.
\end{align}
Note that the functions $I_1$ and $I_2$ are at most logarithmically divergent and finite respectively, which we can ascertain by counting the number of propagators involved. The counterterm $\delta \lambda_0$ from \eqref{eq:dllambda0_tri} is hence discerned as finite. Plugging this back into \eqref{eq:4ptkalpha1} and \eqref{4ptvalpha}, we obtain the following expression for the four-point function
\begin{align}
    &\overline{V}^{(4)}(p_1,p_2,p_3,p_4) =-\lambda_R\left(\frac{1+\frac{1}{2}I_1(p,p_{3},p_{4})}{1+\frac{1}{2}I_1(p_{\ast},p_{3 \ast},p_{4 \ast})}\right)
    + 2i(\alpha_R+\lambda_R\phi_R)^2 G_R(p_3 - p_1) \nonumber \\[2mm]
    &+ (\alpha_R+\lambda_R\phi_R)^2 \left[I_2(p,p_2,p_3,p_4) - I_2 (p_{\ast},p_{3 \ast},p_{4 \ast})
    \left(\frac{1+\frac{1}{2}I_1(p,p_{3},p_{4})}{1+\frac{1}{2}I_1(p_{\ast},p_{3 \ast},p_{4 \ast})}\right)\right]\,,
    \label{v4Ralpha}
\end{align}
which is discerned to be finite from the arguments related to the loop integrals presented above.

As our model does not possess the $\mathbb{Z}_2$ symmetry in the Hartree case, our starting point to determine the counterterm $\delta \lambda_2$ is the following BSE 
\begin{align}
      &V^{(4)}(p_1,p_2,p_3,p_4) = \Lambda^{(4)}(p_1,p_2,p_3,p_4) \nonumber \\[2mm]
    &\qquad \qquad  + \frac{i}{2}\int_q \Lambda^{(4)}(p_1,p_2,q+p,-q) G_R(q) G_R(q+p) \overline{V}^{(4)}(q+p,-q,p_3,p_4) \,.
\label{eqn:v4aux_ss}
\end{align}
From the definition \eqref{eq:4ptkalpha2}, $\Lambda^{(4)}$ does not depend on the integrating loop momentum, which simplifies matters. We impose the renormalization condition
\begin{equation}
    V^{(4)}(p_{1\ast},p_{2\ast},p_{3\ast},p_{4\ast}) = -\lambda_R
\end{equation}
and this leads to the following result
\begin{equation}
    \delta \lambda_2  = -\lambda_R + \lambda^2_R\,\mathcal{I}(p_{\ast})+ \frac{\lambda_R}{1+\frac{1}{2}I_1(p_{\ast},p_{3 \ast},p_{4\ast})}\,\,.
    \label{eqn:dellambda2_tri}
\end{equation}
Firstly, we notice that $\delta \lambda_2 \neq \delta \lambda_0$ unlike in the Hartree approximation, even if we set $\alpha = 0$. This is because, in the scalar sunset approximation, the effective trilinear coupling $\sim \lambda_R \phi_R$ gives a contribution which is the second term of \eqref{eqn:dellambda2_tri}. The very same term introduces a divergence in $\delta \lambda_2$ due to the loop integral $\mathcal{I}(p) = \ii\int_q G_R(q) G_R(q+p)$.

We now examine the three-point function \eqref{eqn:v3_ss}, which has a similar structure to $V^{(4)}$. Firstly, let us describe the kinematics: we take two of the scalars with four-m$p_2$ and $p_3$ to be on-shell. By the conservation of four-momentum, we can calculate $p_1$ as 
\begin{equation}
    p^2_1 = (p_2 + p_3)^2 = 4(|\vec{p}|^2 + m^2_R) \,,
\end{equation}
where we work in the COM frame for $p_2$ and $p_3$ so 
that these scalars are produced back-to-back. For our renormalization condition, we fix the four vectors as follows
\begin{equation}
    p_{2\ast} = \left[\sqrt{2}m_R,\,\vec{p_{\ast}}\right]\,,\quad 
    p_{3\ast} = \left[\sqrt{2}m_R,\,-\vec{p_{\ast}}\right]\,\quad
    p_{1\ast} = \left[2\sqrt{2}m_R,0\right]    
\end{equation}
with $|\vec{p_{\ast}}| = m_R$ and require
\begin{equation}
    V^{(3)}(p_{1 \ast},p_{2\ast},p_{3\ast}) = -\alpha_R \,.
\end{equation}
This gives us the following relation
\begin{equation}
    \delta \alpha_1  = -\alpha_R + \lambda_R (\alpha_R + \lambda_R\phi_R)\,\mathcal{I}(p^2_{1 \ast}) + \frac{\alpha_R}{1+\frac{1}{2}I_1(p_{1 \ast},p_{2 \ast},p_{3\ast})}\,.
    \label{eqn:delalpha1}
\end{equation}
This counterterm is also divergent for the same reason as $\delta \lambda_2$. 

Turning now to the gap equation, with $p$ being the external momentum, this reads
\begin{align}
    i G^{-1}_R(p)  
    &= (p^2 - m^2_R) + (\delta Z_{\phi,0} \, p^2 - \delta m^2_0) - (\alpha_R+\delta\alpha_1)\phi_R  - \frac{(\lambda_R + \delta \lambda_2)}{2}  \phi_R^2 \nonumber \\[2mm]
    &\qquad  - \frac{(\lambda_R + \delta \lambda_0)}{2}\, \mathcal{T} + \frac{(\alpha_R + \lambda_R \phi_R)^2}{2}\, \mathcal{I}(p)  \,,
\label{eqn:gap_ss}
\end{align}
where we have the one-point integral
\begin{equation}
	\mathcal{T} = \int_q G_R(q)\,.
\end{equation}
Using the same on-shell renormalization conditions in the Hartree approximation, as defined in \eqref{eq:hartree_dm0} and \eqref{eq:hartree_dZ0}, we first pick up a finite contribution to the wave function renormalization given by
\begin{equation}
    \delta Z_{\phi,0} = -\frac{(\alpha_R +\lambda_R \phi_R)^2}{2}\, \frac{\partial \mathcal{I}(p)}{\partial p^2}\bigg|_{p^2 = m^2_R} \,.
    \label{delZ0_alpha}
\end{equation}
The derivative eliminates any divergence present and therefore, $\delta Z_{\phi,0}$ is finite in the scalar sunset approximation. We can then determine the mass counterterm
\begin{align}
    \delta m^2_0 &=  \delta Z_{\phi,0} m^2_R - (\alpha_R+\delta\alpha_1)\phi_R - \frac{(\lambda_R + \delta \lambda_2)}{2} \phi_R^2  -\frac{(\lambda_R+\delta \lambda_0)}{2}\, \mathcal{T} \nonumber \\[2mm] 
    &\qquad \qquad + \frac{(\alpha_R +\lambda_R \phi_R)^2}{2}\,\mathcal{I}(p)\big|_{p^2 =m^2_R} \,.
\end{align}
This counterterm is no longer finite like in the Hartree approximation: the reason for this is that we have noted that the counterterms $\delta \lambda_2$ and $\delta \alpha_1$ are divergent. We plug back $\delta m^2_0$ and $\delta Z_{\phi,0}$ into \eqref{eqn:gap_ss} to obtain the following integral equation to ascertain the propagator
\begin{align}
    iG^{-1}_R(p) &= (p^2-m^2_R)\left[1 -\frac{(\alpha_R +\lambda_R \phi_R)^2}{2}\, \frac{\partial \mathcal{I}(q)}{\partial q^2}\bigg|_{q^2 = m^2_R}\right] +\frac{(\alpha_R +\lambda_R \phi_R)^2}{2}\left[\mathcal{I}(p) - \mathcal{I}(q)\big|_{q^2 =m^2_R}\right]\,.
    \label{eq:gap_alpha}
\end{align}
In this form, the propagator is manifestly finite due to potential divergences dropping out in the difference of the loop integral $\mathcal{I}$ or in its differentiation. However, $G_R$ cannot be given in an explicit form but needs to be determined by solving \eqref{eq:gap_alpha}, for which we use an iterative approach, described as follows:
\begin{enumerate}
	\item Initialising with the free propagator, we evaluate the loop integrals in \eqref{eq:gap_alpha} to yield the first iteration of the propagator in terms of Passarino-Veltman functions
	\begin{align}
    	i(G^{-1}_R(p))^{(0)} &= (p^2-m^2_R)\left[1 -\frac{(\alpha_R +\lambda_R \phi_R)^2}{32\pi^2}\, \dot{B}_0(m^2_R,m^2_R,m^2_R)\right]\nonumber \\[2mm]
	&\qquad +\frac{(\alpha_R +\lambda_R \phi_R)^2}{32\pi^2}\left[B_0(p^2,m^2_R,m^2_R)-B_0(m^2_R,m^2_R,m^2_R)\right]\,.
	\label{eqn:propss_1st}
	\end{align}
	\item For the next iteration, convert \eqref{eqn:propss_1st} to Euclidean space. Then, use a numerical implementation of the loop integral $\mathcal{I}(p)$ given as \cite{Pilaftsis:2013xna}
\begin{align}
    \int_q G^{\rm{E}}_R(||q||) &G^{\text{E}}_R (||p+q||) = \nonumber \\
    & \frac{1}{8\pi^3 p^2} \int^{\Lambda}_0 dq \,q\,G^{\rm{E}}_R(q) \int^{\text{min}\{|(q+||p||)|,\Lambda\}}_{|(q-||p||)|} du\, u \,\sqrt{-\lambda (u^2, q^2, ||p||^2)}\, G^{\rm{E}}_R(u) \,,
\end{align}
where the superscript E denotes Euclidean propagators and the momenta are in Euclidean space. We denote $\lambda(x,y,z) = x^2 +y^2 +z^2 -2xy -2yz -2zx$ as the K\"{a}ll\'{e}n function and $||x||$ as the Euclidean norm of $x$. Finally, $\Lambda$ denotes the UV cutoff which, in principle, tends to infinity, but for our numerical implementation needs to be sufficiently large as compared to the scalar mass, the only relevant mass scale. 
	\item Generate a set of points for this iteration of the propagator and interpolate these to obtain the propagator at this iteration. 
	\item Repeat now from step (2) until the iteration converges to the desired accuracy. As a criterion, we use the relative difference between successive iterations.
\end{enumerate}

% Figure environment removed


Our results in Fig. \ref{fig:scalardiff_alpha} show that there is indeed convergence for this iterative procedure as relative differences between successive iterations continue to get smaller for the very large couplings chosen. It would suffice, for smaller couplings hence, to use the first iteration of the propagator for practical purposes. We have also checked the dependence on the UV cutoff, $\Lambda$, and find it to be rather small, with the relative difference being at most $\mathcal{O}(10^{-5})$, even when choosing a cutoff two orders of magnitude higher.

For the determination of the remaining counterterms, we examine various derivatives of the 2PI effective potential. For example, the tadpole counterterm is straightforwardly obtained by taking a single derivative of the effective action w.r.t. $\phi_R$. The stationarity condition then gives
\begin{align}
    \Gamma^{(1)} &= -\delta t_1 - (m^2_R +\delta m^2_2) \phi_R - \frac{(\alpha_R + \delta \alpha_3)}{2}\phi^2_R - \frac{(\alpha_R + \delta \lambda_4)}{6}\phi^2_R \nonumber \\[2mm]
    &\quad - \frac{1}{2}\left[(\alpha_R+\delta \alpha_1) + (\lambda_R + \delta \lambda_4)\phi_R\right]\,\mathcal{T} + \frac{\lambda_R \left(\alpha_R + \lambda_R \phi_R\right)}{6} \,\mathcal{S} \stackrel{!}{=} 0   \,\,,
    \label{eqn:tadpole_ss}
\end{align}
where 
\begin{equation}
	\mathcal{S} = i \int_p \int_q G_R(p)G_R(q)G_R(p+q)\,.
\end{equation}
The tadpole counterterm cannot be determined without first ascertaining the missing field counterterms. The easiest of these to find is $\delta m^2_2$, for which we look at $\Gamma^{(2)}$. Using equations \eqref{eqn:phys2pta} and \eqref{eq:dPidphi} we obtain 
\begin{align}
    &\Gamma^{(2)}(p) = (p^2-m^2_R) + (\delta Z_{\phi,2} p^2  -\delta m^2_2) - (\alpha_R + \delta \alpha_3)\phi_R -\frac{1}{2}(\lambda_R+\delta\lambda_4) \phi^2_R \nonumber \\[2mm]
    & \quad  -\frac{1}{2}(\lambda_R + \delta \lambda_2)\mathcal{T} + \frac{\lambda^2_R}{6} \,\mathcal{S}   \nonumber \\[2mm]
    &\quad - \frac{1}{8}\left[(\alpha_R + \delta\alpha_1) + (\lambda_R+\delta\lambda_2)\phi_R  - \frac{\lambda_R(\alpha_R + \lambda_R\phi_R)}{3}\mathcal{I}(p)\right]^2 \,\mathcal{I}(p)  \nonumber \\[2mm]
    &\quad + \frac{1}{16}\bigg\{\left[(\alpha_R + \delta\alpha_1) + (\lambda_R+\delta\lambda_2)\phi_R  - \frac{\lambda_R(\alpha_R + \lambda_R\phi_R)}{3}\mathcal{I}(p)\right]^2 \nonumber \\[2mm]
    &\quad \quad \quad \quad \quad \quad\int_q \int_k G_R(p+q)G_R(q) \overline{V}^{(4)}(p+q,-q,p+k,-k) G_R(p+k)G_R(k) \bigg\}  \,,
    \label{eqn:scalar2pt_tri}
\end{align}
where the terms in parenthesis for the third and last lines appear from the replacement of the various building blocks in \eqref{eqn:phys2pta} and \eqref{eq:dPidphi}. The earlier analysis in the Hartree approximation does not carry over as  $\delta \lambda_2 \neq \delta \lambda_0$, and moreover, $\delta \lambda_2$ is now contains a divergent part. We can determine the counterterms $\delta Z_{\phi,2}$ and $\delta m^2_2$ with the on-shell renormalization conditions, to give
\begin{align}
    &\delta Z_{\phi,2} =  \frac{1}{8}\left[(\alpha_R + \delta\alpha_1) + (\lambda_R+\delta\lambda_2)\phi_R  - \frac{\lambda_R(\alpha_R + \lambda_R\phi_R)}{3}\mathcal{I}(p)\big|_{p^2=m^2_R}\right]\nonumber \\[2mm]
    &\Bigg\{\bigg\{-\frac{\lambda_R(\alpha_R+\lambda_R\phi_R)}{3}\mathcal{I}(p)\big|_{p^2=m^2_R}\,\frac{\partial \mathcal{I}(p)}{\partial p^2}\bigg|_{p^2 = m^2_R}   \nonumber \\[2mm]
    &\quad + \left[(\alpha_R + \delta\alpha_1)+ (\lambda_R+\delta\lambda_2)\phi_R  - \frac{\lambda_R(\alpha_R + \lambda_R\phi_R)}{3}\mathcal{I}(p)\big|_{p^2=m^2 _R}\right]\frac{\partial \mathcal{I}(p)}{\partial p^2}\bigg|_{p^2 = m^2_R} \bigg\}  \nonumber \\[2mm]
    &+\frac{1}{2}\bigg\{-\frac{\lambda_R(\alpha_R+\lambda_R\phi_R)}{3}\mathcal{I}(p)\big|_{p^2=m^2_R}\,\frac{\partial \mathcal{I}_V(p)}{\partial p^2}\bigg|_{p^2 = m^2_R}  \nonumber \\[2mm]
    &\quad + \left[(\alpha_R + \delta\alpha_1) + (\lambda_R+\delta\lambda_2)\phi_R  - \frac{\lambda_R(\alpha_R + \lambda_R\phi_R)}{3}\mathcal{I}(p)\big|_{p^2=m^2_R}\right]\frac{\partial \mathcal{I}_V(p)}{\partial p^2}\bigg|_{p^2 = m^2_R} \bigg\}\Bigg\}\,,
\end{align}

\begin{align}
    &\delta m^2_2 = m^2_R\,\delta Z_{\phi,2}  - (\alpha_R+\delta\alpha_3)\phi_R - \frac{1}{2}(\lambda_R+\delta\lambda_4)\phi^2_R -\frac{1}{2}(\lambda_R + \delta \lambda_2)\mathcal{T} + \frac{\lambda^2_R}{6} \,\mathcal{S}  \nonumber \\[2mm]
    & - \frac{1}{8}\left[(\alpha_R + \delta\alpha_1) + (\lambda_R+\delta\lambda_2)\phi_R  - \frac{\lambda_R(\alpha_R + \lambda_R\phi_R)}{3}\mathcal{I}(p)\big|_{p^2=m^2_R}\right]^2 \,\mathcal{I}(p)\big|_{p^2=m^2_R}  \nonumber \\[2mm]
    &+ \frac{1}{16}\left[(\alpha_R + \delta\alpha_1) + (\lambda_R+\delta\lambda_2)\phi_R  - \frac{\lambda_R(\alpha_R + \lambda_R\phi_R)}{3}\mathcal{I}(p)\big|_{p^2=m^2_R}  \right]^2\mathcal{I}_V(p)\big|_{p^2=m^2_R} \,,
\end{align}
where 
\begin{equation}
	\mathcal{I}_V(p) = \int_q \int_k G_R(p+q)G_R(q) \overline{V}^{(4)}(p+q,-q,p+k,-k) G_R(p+k)G_R(k) \,.
\end{equation}
On close inspection, we can discern that these counterterms would not be finite like in the Hartree approximation, due to the coupling constant counterterms being $\mathcal{O}(\epsilon^{-1})$. Furthermore, it is also obvious that the equalities $\delta Z_{\phi,2} = \delta Z_{\phi,0}$ and $\delta m^2_2 = \delta m^2_0$ no longer hold true, as we had in the Hartree approximation.

Finally, the only undetermined counterterms are 
$\delta \alpha_3$ and $\delta \lambda_4$, which are needed to completely express $\delta t_1$ and $\delta m^2_2$. 
The corresponding formulas are very lengthy and, thus we give here the corresponding diagrammatic representation.
%Here is d^3Gamma/dphi^3 
\begin{align}
\Gamma^{(3)} \equiv \frac{\delta^3 \Gamma}{\delta \phi^3} = & \quad
%
 \begin{tikzpicture}[scale=2,baseline={([yshift=-.4ex]current bounding box.center)}]
\node (fourpoint) at (0,0) [Gamma] {}; 
\node (l1) at (-0.4,0) [VEV] {}; 
\node (l2) at (0.,0.4) [VEV] {}; 
\node (l3) at (0.4,0) [VEV] {}; 
\draw[-] (fourpoint) -- (l1);
\draw[-] (fourpoint) -- (l2);
\draw[-] (fourpoint) -- (l3);
\end{tikzpicture}
%
+
%
\frac{1}{2} \,\,
 \begin{tikzpicture}[scale=2,baseline={([yshift=-.4ex]current bounding box.center)}]
\node (v1) at (0,0.3) [Gamma] {}; 
\node (v3) at (0.6,0.3) [Pi] {}; 
\node (l1) at (-0.4,0.3) [VEV] {}; 
\node (l2) at (0.,0.7) [VEV] {}; 
\node (l3) at (0.9,0.3) [VEV] {}; 
\draw[double distance=4mm,thick] (v1) -- (v3);
\draw[-] (v1) -- (l1);
\draw[-] (v1) -- (l2);
\draw[-] (v3) -- (l3);
\end{tikzpicture}
%
+
%
\frac{1}{4} \,\,
 \begin{tikzpicture}[scale=2,baseline={([yshift=-.4ex]current bounding box.center)}]
\node (v1) at (0,0.3) [Gamma] {}; 
\node (v2) at (0.6,0.3) [Pi] {}; 
\node (v3) at (0.0,0.9) [Pi] {}; 
\node (l1) at (-0.4,0.3) [VEV] {}; 
\node (l2) at (0.,1.3) [VEV] {}; 
\node (l3) at (0.9,0.3) [VEV] {}; 
\draw[double distance=4mm,thick] (v1) -- (v2);
\draw[double distance=4mm,thick] (v1) -- (v2);
\draw[double distance=4mm,thick] (v1) -- (v3);
\draw[-] (v1) -- (l1);
\draw[-] (v3) -- (l2);
\draw[-] (v2) -- (l3);
\end{tikzpicture}
%
+
%
 \begin{tikzpicture}[scale=2,baseline={([yshift=-.4ex]current bounding box.center)}]
\node (v1) at (0,0.3) [Gamma] {}; 
\node (v2) at (0,-0.3) [Pi] {}; 
\node (v3) at (0.6,0.3) [Pi] {}; 
\node (l1) at (-0.4,0.3) [VEV] {}; 
\node (l2) at (0.9,0.3) [VEV] {}; 
\node (l3) at (0.,-0.7) [VEV] {}; 
\draw (v1) -- (v2);
\draw (v1) -- (v3);
\draw (v2) -- (v3);
\draw[-] (v1) -- (l1);
\draw[-] (v3) -- (l2);
\draw[-] (v2) -- (l3);
\end{tikzpicture}
%
\nonumber \\[8mm]
%
& + \frac{1}{2} \,\,
%
 \begin{tikzpicture}[scale=2,baseline={([yshift=-.4ex]current bounding box.center)}]
\node (v1) at (0,0.3) [Gamma] {}; 
\node (v2) at (0.6,0.3) [Pi] {}; 
\node (l1) at (-0.4,0.3) [VEV] {}; 
\node (l2) at (0.9,0.3) [VEV] {}; 
\node (l3) at (0.6,-0.1) [VEV] {}; 
\draw[double distance=4mm,thick] (v1) -- (v2);
\draw[-] (v1) -- (l1);
\draw[-] (v2) -- (l2);
\draw[-] (v2) -- (l3);
\end{tikzpicture} \,\,\,\,.
\label{eq:Gamma3_alpha}
\end{align}

%Here is d^4Gamma/dphi^4
\begin{align}
\Gamma^{(4)} \equiv \frac{\delta^4 \Gamma}{\delta \phi^4} = & \quad
%
 \begin{tikzpicture}[scale=2,baseline={([yshift=-.4ex]current bounding box.center)}]
\node (fourpoint) at (0,0) [Gamma] {}; 
\node (l1) at (-0.4,0) [VEV] {}; 
\node (l2) at (0.,0.4) [VEV] {}; 
\node (l3) at (0.4,0) [VEV] {}; 
\node (l4) at (0.,-0.4) [VEV] {}; 
\draw[-] (fourpoint) -- (l1);
\draw[-] (fourpoint) -- (l2);
\draw[-] (fourpoint) -- (l3);
\draw[-] (fourpoint) -- (l4);
\end{tikzpicture}
%
+
%
\frac{1}{4} \,\,
 \begin{tikzpicture}[scale=2,baseline={([yshift=-.4ex]current bounding box.center)}]
\node (v1) at (0,0.3) [Gamma] {}; 
\node (v2) at (0,-0.3) [Pi] {}; 
\node (v3) at (0.6,0.3) [Pi] {}; 
\node (l1) at (-0.4,0.3) [VEV] {}; 
\node (l2) at (0.,0.7) [VEV] {}; 
\node (l3) at (0.9,0.3) [VEV] {}; 
\node (l4) at (0.,-0.7) [VEV] {}; 
\draw[double distance=4mm,thick] (v1) -- (v2);
\draw[double distance=4mm,thick] (v1) -- (v3);
\draw[-] (v1) -- (l1);
\draw[-] (v1) -- (l2);
\draw[-] (v3) -- (l3);
\draw[-] (v2) -- (l4);
\end{tikzpicture}
%
+
%
\frac{1}{8} \,\,
 \begin{tikzpicture}[scale=2,baseline={([yshift=-.4ex]current bounding box.center)}]
\node (v1) at (0,0.3) [Gamma] {}; 
\node (v2) at (0,-0.3) [Pi] {}; 
\node (v3) at (0.6,0.3) [Pi] {}; 
\node (v4) at (0.0,0.9) [Pi] {}; 
\node (l1) at (-0.4,0.3) [VEV] {}; 
\node (l2) at (0.,1.3) [VEV] {}; 
\node (l3) at (0.9,0.3) [VEV] {}; 
\node (l4) at (0.,-0.7) [VEV] {}; 
\draw[double distance=4mm,thick] (v1) -- (v2);
\draw[double distance=4mm,thick] (v1) -- (v3);
\draw[double distance=4mm,thick] (v1) -- (v4);
\draw[-] (v1) -- (l1);
\draw[-] (v4) -- (l2);
\draw[-] (v3) -- (l3);
\draw[-] (v2) -- (l4);
\end{tikzpicture}
%
\nonumber \\
%
& +
%
 \begin{tikzpicture}[scale=2,baseline={([yshift=-.4ex]current bounding box.center)}]
\node (v1) at (0,0.3) [Gamma] {}; 
\node (v2) at (0,-0.3) [Pi] {}; 
\node (v3) at (0.6,0.3) [Pi] {}; 
\node (l1) at (-0.4,0.3) [VEV] {}; 
\node (l2) at (0.,0.7) [VEV] {}; 
\node (l3) at (0.9,0.3) [VEV] {}; 
\node (l4) at (0.,-0.7) [VEV] {}; 
\draw (v1) -- (v2);
\draw (v1) -- (v3);
\draw (v2) -- (v3);
\draw[-] (v1) -- (l1);
\draw[-] (v1) -- (l2);
\draw[-] (v3) -- (l3);
\draw[-] (v2) -- (l4);
\end{tikzpicture}
%
+
%
\frac{1}{2} \,\,
 \begin{tikzpicture}[scale=2,baseline={([yshift=-.4ex]current bounding box.center)}]
\node (v1) at (0,0.3) [Gamma] {}; 
\node (v2) at (0,-0.3) [Pi] {}; 
\node (v3) at (0.6,0.3) [Pi] {}; 
\node (v4) at (0.0,0.9) [Pi] {}; 
\node (l1) at (-0.4,0.3) [VEV] {}; 
\node (l2) at (0.,1.3) [VEV] {}; 
\node (l3) at (0.9,0.3) [VEV] {}; 
\node (l4) at (0.,-0.7) [VEV] {}; 
\draw (v1) -- (v2);
\draw (v1) -- (v3);
\draw (v2) -- (v3);
\draw[double distance=4mm,thick] (v1) -- (v4);
\draw[-] (v1) -- (l1);
\draw[-] (v4) -- (l2);
\draw[-] (v3) -- (l3);
\draw[-] (v2) -- (l4);
\end{tikzpicture}
%
+
%
 \begin{tikzpicture}[scale=2,baseline={([yshift=-.4ex]current bounding box.center)}]
\node (v1) at (0,0.3) [Gamma] {}; 
\node (v2) at (0,-0.3) [Pi] {}; 
\node (v3) at (0.6,0.3) [Pi] {}; 
\node (v4) at (0.0,0.9) [Pi] {}; 
\node (l1) at (-0.4,0.3) [VEV] {}; 
\node (l2) at (0.,1.3) [VEV] {}; 
\node (l3) at (0.9,0.3) [VEV] {}; 
\node (l4) at (0.,-0.7) [VEV] {}; 
\draw (v1) -- (v2);
\draw (v4) -- (v3);
\draw (v2) -- (v3);
\draw (v1) -- (v4);
\draw[-] (v1) -- (l1);
\draw[-] (v4) -- (l2);
\draw[-] (v3) -- (l3);
\draw[-] (v2) -- (l4);
\end{tikzpicture}
%
\nonumber \\
%
& + \frac{1}{2} \,\,
%
 \begin{tikzpicture}[scale=2,baseline={([yshift=-.4ex]current bounding box.center)}]
\node (v1) at (0,0.3) [Gamma] {}; 
\node (v2) at (0.6,0.3) [Pi] {}; 
\node (l1) at (-0.4,0.3) [VEV] {}; 
\node (l2) at (0.,0.7) [VEV] {}; 
\node (l3) at (0.9,0.3) [VEV] {}; 
\node (l4) at (0.6,-0.1) [VEV] {}; 
\draw[double distance=4mm,thick] (v1) -- (v2);
\draw[-] (v1) -- (l1);
\draw[-] (v1) -- (l2);
\draw[-] (v2) -- (l3);
\draw[-] (v2) -- (l4);
\end{tikzpicture}
%
+
%
\frac{1}{4} \,\,
 \begin{tikzpicture}[scale=2,baseline={([yshift=-.4ex]current bounding box.center)}]
\node (v1) at (0,0.3) [Gamma] {}; 
\node (v3) at (0.6,0.3) [Pi] {}; 
\node (v4) at (0.0,0.9) [Pi] {}; 
\node (l1) at (-0.4,0.3) [VEV] {}; 
\node (l2) at (0.,1.3) [VEV] {}; 
\node (l3) at (0.9,0.3) [VEV] {}; 
\node (l4) at (0.6,-0.1) [VEV] {}; 
\draw[double distance=4mm,thick] (v1) -- (v4);
\draw[double distance=4mm,thick] (v1) -- (v3);
\draw[-] (v1) -- (l1);
\draw[-] (v4) -- (l2);
\draw[-] (v3) -- (l3);
\draw[-] (v3) -- (l4);
\end{tikzpicture}
%
+
%
 \begin{tikzpicture}[scale=2,baseline={([yshift=-.4ex]current bounding box.center)}]
\node (v1) at (0,0.3) [Gamma] {}; 
\node (v3) at (0.6,0.3) [Pi] {}; 
\node (v4) at (0.0,0.9) [Pi] {}; 
\node (l1) at (-0.4,0.3) [VEV] {}; 
\node (l2) at (0.,1.3) [VEV] {}; 
\node (l3) at (0.9,0.3) [VEV] {}; 
\node (l4) at (0.6,-0.1) [VEV] {}; 
\draw (v1) -- (v3);
\draw (v4) -- (v3);
\draw (v1) -- (v4);
\draw[-] (v1) -- (l1);
\draw[-] (v4) -- (l2);
\draw[-] (v3) -- (l3);
\draw[-] (v3) -- (l4);
\end{tikzpicture}
%
\nonumber \\
%
& + \frac{1}{2} \,\,
%
 \begin{tikzpicture}[scale=2,baseline={([yshift=-.4ex]current bounding box.center)}]
\node (v1) at (0,0.3) [Gamma] {}; 
\node (v2) at (0.6,0.3) [Pi] {}; 
\node (l1) at (-0.4,0.3) [VEV] {}; 
\node (l2) at (0.6,0.7) [VEV] {}; 
\node (l3) at (0.9,0.3) [VEV] {}; 
\node (l4) at (0.6,-0.1) [VEV] {}; 
\draw[double distance=4mm,thick] (v1) -- (v2);
\draw[-] (v1) -- (l1);
\draw[-] (v2) -- (l2);
\draw[-] (v2) -- (l3);
\draw[-] (v2) -- (l4);
\end{tikzpicture} \,\,\,.
\label{eq:Gamma4_alpha}
\end{align}
Referring now to \eqref{eq:d2_Pi_dphi2} and \eqref{eq:d3_Pi_dphi3}, we first enlist the various non-vanishing contributions of the derivative of the self-energy w.r.t. $\phi_R$ that would be required.
\begin{align}
 \begin{tikzpicture}[scale=2,baseline={([yshift=-.4ex]current bounding box.center)}]
\node (v1) at (0.0,0.) [Pi] {}; 
\node (l1) at (-0.4,0) {}; 
\node (l2) at (0.4,0.) [VEV] {}; 
\node (l3) at (0,-0.4) [VEV] {}; 
\draw[double distance=4mm,thick] (l1) -- (v1);
\draw[-] (v1) -- (l2);
\draw[-] (v1) -- (l3);
\end{tikzpicture}
%
= & \,\,
%
 \begin{tikzpicture}[scale=2,baseline={([yshift=-.4ex]current bounding box.center)}]
\node (v1) at (0.0,0.) [Gamma] {}; 
\node (l1) at (-0.4,0) {}; 
\node (l2) at (0.4,0.) [VEV] {}; 
\node (l3) at (0,-0.4) [VEV] {}; 
\draw[double distance=4mm,thick] (l1) -- (v1);
\draw[-] (v1) -- (l2);
\draw[-] (v1) -- (l3);
\end{tikzpicture}
%
+ \frac{1}{2} \,
%
 \begin{tikzpicture}[scale=2,baseline={([yshift=-.4ex]current bounding box.center)}]
\node (v1) at (0.0,0.) [Gamma] {}; 
\node (v2) at (-0.6,0.) [VR] {\,$\overline{V}^{(4)}$\,}; 
\node (l1) at (-1.,0) {}; 
\node (l2) at (0.4,0.) [VEV] {}; 
\node (l3) at (0,-0.4) [VEV] {}; 
\draw[double distance=4mm,thick] (l1) -- (v2);
\draw[double distance=4mm,thick] (v1) -- (v2);
\draw[-] (v1) -- (l2);
\draw[-] (v1) -- (l3);
\end{tikzpicture}
% 
\nonumber \\ & 
%
+ \frac{1}{2} \,
%
 \begin{tikzpicture}[scale=2,baseline={([yshift=-.4ex]current bounding box.center)}]
\node (v1) at (0.0,0.) [Gamma] {}; 
\node (v3) at (0.0,-0.6) [Pi] {}; 
\node (l1) at (-0.4,0) {}; 
\node (l2) at (0.4,0.) [VEV] {}; 
\node (l3) at (0,-1.0) [VEV] {}; 
\draw[double distance=4mm,thick] (l1) -- (v1);
\draw[double distance=4mm,thick] (v1) -- (v3);
\draw[-] (v1) -- (l2);
\draw[-] (v3) -- (l3);
\end{tikzpicture}
%
+ \frac{1}{4} \,
%
 \begin{tikzpicture}[scale=2,baseline={([yshift=-.4ex]current bounding box.center)}]
\node (v1) at (0.0,0.) [Gamma] {}; 
\node (v2) at (-0.6,0.) [VR] {\,$\overline{V}^{(4)}$\,}; 
\node (v3) at (0.0,-0.6) [Pi] {}; 
\node (l1) at (-1.,0) {}; 
\node (l2) at (0.4,0.) [VEV] {}; 
\node (l3) at (0,-1.0) [VEV] {}; 
\draw[double distance=4mm,thick] (l1) -- (v2);
\draw[double distance=4mm,thick] (v1) -- (v2);
\draw[double distance=4mm,thick] (v1) -- (v3);
\draw[-] (v1) -- (l2);
\draw[-] (v3) -- (l3);
\end{tikzpicture}
% 
\nonumber \\ & 
%
+ \frac{1}{4} \,
%
 \begin{tikzpicture}[scale=2,baseline={([yshift=-.4ex]current bounding box.center)}]
\node (v1) at (0.0,0.) [Gamma] {}; 
\node (v3) at (0.0,-0.6) [Pi] {}; 
\node (v4) at (0.6,0.) [Pi] {}; 
\node (l1) at (-0.4,0) {}; 
\node (l2) at (1.,0.) [VEV] {}; 
\node (l3) at (0,-1.0) [VEV] {}; 
\draw[double distance=4mm,thick] (l1) -- (v1);
\draw[double distance=4mm,thick] (v1) -- (v3);
\draw[double distance=4mm,thick] (v1) -- (v4);
\draw[-] (v4) -- (l2);
\draw[-] (v3) -- (l3);
\end{tikzpicture}
%
+ \frac{1}{8} \,
%
 \begin{tikzpicture}[scale=2,baseline={([yshift=-.4ex]current bounding box.center)}]
\node (v1) at (0.0,0.) [Gamma] {}; 
\node (v2) at (-0.6,0.) [VR] {\,$\overline{V}^{(4)}$\,}; 
\node (v3) at (0.0,-0.6) [Pi] {}; 
\node (v4) at (0.6,0.) [Pi] {}; 
\node (l1) at (-1.,0) {}; 
\node (l2) at (1.,0.) [VEV] {}; 
\node (l3) at (0,-1.0) [VEV] {}; 
\draw[double distance=4mm,thick] (l1) -- (v2);
\draw[double distance=4mm,thick] (v1) -- (v2);
\draw[double distance=4mm,thick] (v1) -- (v3);
\draw[double distance=4mm,thick] (v1) -- (v4);
\draw[-] (v4) -- (l2);
\draw[-] (v3) -- (l3);
\end{tikzpicture}
% 
\nonumber \\ & 
%
+  \,
%
 \begin{tikzpicture}[scale=2,baseline={([yshift=-.4ex]current bounding box.center)}]
\node (v1) at (0.0,0.) [Gamma] {}; 
\node (v3) at (0.0,-0.6) [Pi] {}; 
\node (v4) at (0.6,0.) [Pi] {}; 
\node (l1) at (-0.4,0) {}; 
\node (l2) at (1.,0.) [VEV] {}; 
\node (l3) at (0,-1.0) [VEV] {}; 
\draw[double distance=4mm,thick] (l1) -- (v1);
\draw (v1) -- (v3);
\draw (v1) -- (v4);
\draw (v3) -- (v4);
\draw[-] (v4) -- (l2);
\draw[-] (v3) -- (l3);
\end{tikzpicture}
%
+ \frac{1}{2} \,
%
 \begin{tikzpicture}[scale=2,baseline={([yshift=-.4ex]current bounding box.center)}]
\node (v1) at (0.0,0.) [Gamma] {}; 
\node (v2) at (-0.6,0.) [VR] {\,$\overline{V}^{(4)}$\,}; 
\node (v3) at (0.0,-0.6) [Pi] {}; 
\node (v4) at (0.6,0.) [Pi] {}; 
\node (l1) at (-1.,0) {}; 
\node (l2) at (1.,0.) [VEV] {}; 
\node (l3) at (0,-1.0) [VEV] {}; 
\draw[double distance=4mm,thick] (l1) -- (v2);
\draw[double distance=4mm,thick] (v1) -- (v2);
\draw (v1) -- (v3);
\draw (v1) -- (v4);
\draw (v3) -- (v4);
\draw[-] (v4) -- (l2);
\draw[-] (v3) -- (l3);
\end{tikzpicture}
\label{eq:d2Pi_dphi2_alpha}
\end{align}


\begin{align}
 \begin{tikzpicture}[scale=2,baseline={([yshift=-.4ex]current bounding box.center)}]
\node (v1) at (0.0,0.) [Pi] {}; 
\node (l1) at (-0.4,0) {}; 
\node (l2) at (0.4,0.) [VEV] {}; 
\node (l3) at (0,-0.4) [VEV] {}; 
\node (l4) at (0,0.4) [VEV] {}; 
\draw[double distance=4mm,thick] (l1) -- (v1);
\draw[-] (v1) -- (l2);
\draw[-] (v1) -- (l3);
\draw[-] (v1) -- (l4);
\end{tikzpicture}
%
= & \,\,
%
+ \frac{1}{4} \,
%
 \begin{tikzpicture}[scale=2,baseline={([yshift=-.4ex]current bounding box.center)}]
\node (v1) at (0.0,0.) [Gamma] {}; 
\node (v3) at (0.0,-0.6) [Pi] {}; 
\node (v4) at (0.6,0.) [Pi] {}; 
\node (l1) at (-0.4,0) {}; 
\node (l2) at (1.,0.) [VEV] {}; 
\node (l3) at (0,-1.0) [VEV] {}; 
\node (l4) at (0,0.4) [VEV] {}; 
\draw[double distance=4mm,thick] (l1) -- (v1);
\draw[double distance=4mm,thick] (v1) -- (v3);
\draw[double distance=4mm,thick] (v1) -- (v4);
\draw[-] (v4) -- (l2);
\draw[-] (v3) -- (l3);
\draw[-] (v1) -- (l4);
\end{tikzpicture}
%
+ \frac{1}{8} \,
%
 \begin{tikzpicture}[scale=2,baseline={([yshift=-.4ex]current bounding box.center)}]
\node (v1) at (0.0,0.) [Gamma] {}; 
\node (v2) at (-0.6,0.) [VR] {\,$\overline{V}^{(4)}$\,}; 
\node (v3) at (0.0,-0.6) [Pi] {}; 
\node (v4) at (0.6,0.) [Pi] {}; 
\node (l1) at (-1.,0) {}; 
\node (l2) at (1.,0.) [VEV] {}; 
\node (l3) at (0,-1.0) [VEV] {}; 
\node (l4) at (0,0.4) [VEV] {}; 
\draw[double distance=4mm,thick] (l1) -- (v2);
\draw[double distance=4mm,thick] (v1) -- (v2);
\draw[double distance=4mm,thick] (v1) -- (v3);
\draw[double distance=4mm,thick] (v1) -- (v4);
\draw[-] (v4) -- (l2);
\draw[-] (v3) -- (l3);
\draw[-] (v1) -- (l4);
\end{tikzpicture}
% 
\nonumber \\ & 
%
%
+ \frac{1}{8} \,
%
 \begin{tikzpicture}[scale=2,baseline={([yshift=-.4ex]current bounding box.center)}]
\node (v1) at (0.0,0.) [Gamma] {}; 
\node (v3) at (0.0,-0.6) [Pi] {}; 
\node (v4) at (0.6,0.) [Pi] {}; 
\node (v5) at (0.,0.6) [Pi] {}; 
\node (l1) at (-0.4,0) {}; 
\node (l2) at (1.,0.) [VEV] {}; 
\node (l3) at (0,-1.0) [VEV] {}; 
\node (l4) at (0,1.) [VEV] {}; 
\draw[double distance=4mm,thick] (l1) -- (v1);
\draw[double distance=4mm,thick] (v1) -- (v3);
\draw[double distance=4mm,thick] (v1) -- (v4);
\draw[double distance=4mm,thick] (v1) -- (v5);
\draw[-] (v4) -- (l2);
\draw[-] (v3) -- (l3);
\draw[-] (v5) -- (l4);
\end{tikzpicture}
%
+ \frac{1}{16} \,
%
 \begin{tikzpicture}[scale=2,baseline={([yshift=-.4ex]current bounding box.center)}]
\node (v1) at (0.0,0.) [Gamma] {}; 
\node (v2) at (-0.6,0.) [VR] {\,$\overline{V}^{(4)}$\,}; 
\node (v3) at (0.0,-0.6) [Pi] {}; 
\node (v4) at (0.6,0.) [Pi] {}; 
\node (v5) at (0.,0.6) [Pi] {}; 
\node (l1) at (-1.,0) {}; 
\node (l2) at (1.,0.) [VEV] {}; 
\node (l3) at (0,-1.0) [VEV] {}; 
\node (l4) at (0,1.) [VEV] {}; 
\draw[double distance=4mm,thick] (l1) -- (v2);
\draw[double distance=4mm,thick] (v1) -- (v2);
\draw[double distance=4mm,thick] (v1) -- (v3);
\draw[double distance=4mm,thick] (v1) -- (v4);
\draw[double distance=4mm,thick] (v1) -- (v5);
\draw[-] (v4) -- (l2);
\draw[-] (v3) -- (l3);
\draw[-] (v5) -- (l4);
\end{tikzpicture}
% 
\nonumber \\ & 
%
+  \,
%
 \begin{tikzpicture}[scale=2,baseline={([yshift=-.4ex]current bounding box.center)}]
\node (v1) at (0.0,0.) [Gamma] {}; 
\node (v3) at (0.0,-0.6) [Pi] {}; 
\node (v4) at (0.6,0.) [Pi] {}; 
\node (l1) at (-0.4,0) {}; 
\node (l2) at (1.,0.) [VEV] {}; 
\node (l3) at (0,-1.0) [VEV] {}; 
\node (l4) at (0,0.4) [VEV] {}; 
\draw[double distance=4mm,thick] (l1) -- (v1);
\draw (v1) -- (v3);
\draw (v1) -- (v4);
\draw (v3) -- (v4);
\draw[-] (v4) -- (l2);
\draw[-] (v3) -- (l3);
\draw[-] (v1) -- (l4);
\end{tikzpicture}
%
+ \frac{1}{2} \,
%
 \begin{tikzpicture}[scale=2,baseline={([yshift=-.4ex]current bounding box.center)}]
\node (v1) at (0.0,0.) [Gamma] {}; 
\node (v2) at (-0.6,0.) [VR] {\,$\overline{V}^{(4)}$\,}; 
\node (v3) at (0.0,-0.6) [Pi] {}; 
\node (v4) at (0.6,0.) [Pi] {}; 
\node (l1) at (-1.,0) {}; 
\node (l2) at (1.,0.) [VEV] {}; 
\node (l3) at (0,-1.0) [VEV] {}; 
\node (l4) at (0,0.4) [VEV] {}; 
\draw[double distance=4mm,thick] (l1) -- (v2);
\draw[double distance=4mm,thick] (v1) -- (v2);
\draw (v1) -- (v3);
\draw (v1) -- (v4);
\draw (v3) -- (v4);
\draw[-] (v4) -- (l2);
\draw[-] (v3) -- (l3);
\draw[-] (v1) -- (l4);
\end{tikzpicture}
%
\nonumber \\ &
%
+  \frac{1}{2} \,
%
 \begin{tikzpicture}[scale=2,baseline={([yshift=-.4ex]current bounding box.center)}]
\node (v1) at (0.0,0.) [Gamma] {}; 
\node (v3) at (0.0,-0.6) [Pi] {}; 
\node (v4) at (0.6,0.) [Pi] {}; 
\node (v5) at (0.,0.6) [Pi] {}; 
\node (l1) at (-0.4,0) {}; 
\node (l2) at (1.,0.) [VEV] {}; 
\node (l3) at (0,-1.0) [VEV] {}; 
\node (l4) at (0,1.) [VEV] {}; 
\draw[double distance=4mm,thick] (l1) -- (v1);
\draw[double distance=4mm,thick] (v1) -- (v5);
\draw (v1) -- (v3);
\draw (v1) -- (v4);
\draw (v3) -- (v4);
\draw[-] (v4) -- (l2);
\draw[-] (v3) -- (l3);
\draw[-] (v5) -- (l4);
\end{tikzpicture}
%
+ \frac{1}{4} \,
%
 \begin{tikzpicture}[scale=2,baseline={([yshift=-.4ex]current bounding box.center)}]
\node (v1) at (0.0,0.) [Gamma] {}; 
\node (v2) at (-0.6,0.) [VR] {\,$\overline{V}^{(4)}$\,}; 
\node (v3) at (0.0,-0.6) [Pi] {}; 
\node (v4) at (0.6,0.) [Pi] {}; 
\node (v5) at (0.,0.6) [Pi] {}; 
\node (l1) at (-1.,0) {}; 
\node (l2) at (1.,0.) [VEV] {}; 
\node (l3) at (0,-1.0) [VEV] {}; 
\node (l4) at (0,1.) [VEV] {}; 
\draw[double distance=4mm,thick] (l1) -- (v2);
\draw[double distance=4mm,thick] (v1) -- (v2);
\draw[double distance=4mm,thick] (v1) -- (v5);
\draw (v1) -- (v3);
\draw (v1) -- (v4);
\draw (v3) -- (v4);
\draw[-] (v4) -- (l2);
\draw[-] (v3) -- (l3);
\draw[-] (v5) -- (l4);
\end{tikzpicture}
%
\nonumber  \\
& +   \,
%
 \begin{tikzpicture}[scale=2,baseline={([yshift=-.4ex]current bounding box.center)}]
\node (v1) at (0.0,0.) [Gamma] {}; 
\node (v3) at (0.0,-0.6) [Pi] {}; 
\node (v4) at (0.6,0.) [Pi] {}; 
\node (v5) at (0.,0.6) [Pi] {}; 
\node (l1) at (-0.4,0) {}; 
\node (l2) at (1.,0.) [VEV] {}; 
\node (l3) at (0,-1.0) [VEV] {}; 
\node (l4) at (0,1.) [VEV] {}; 
\draw[double distance=4mm,thick] (l1) -- (v1);
\draw (v1) -- (v5);
\draw (v1) -- (v3);
\draw (v5) -- (v4);
\draw (v3) -- (v4);
\draw[-] (v4) -- (l2);
\draw[-] (v3) -- (l3);
\draw[-] (v5) -- (l4);
\end{tikzpicture}
%
+ \frac{1}{2} \,
%
 \begin{tikzpicture}[scale=2,baseline={([yshift=-.4ex]current bounding box.center)}]
\node (v1) at (0.0,0.) [Gamma] {}; 
\node (v2) at (-0.6,0.) [VR] {\,$\overline{V}^{(4)}$\,}; 
\node (v3) at (0.0,-0.6) [Pi] {}; 
\node (v4) at (0.6,0.) [Pi] {}; 
\node (v5) at (0.,0.6) [Pi] {}; 
\node (l1) at (-1.,0) {}; 
\node (l2) at (1.,0.) [VEV] {}; 
\node (l3) at (0,-1.0) [VEV] {}; 
\node (l4) at (0,1.) [VEV] {}; 
\draw[double distance=4mm,thick] (l1) -- (v2);
\draw[double distance=4mm,thick] (v1) -- (v2);
\draw (v1) -- (v5);
\draw (v1) -- (v3);
\draw (v5) -- (v4);
\draw (v3) -- (v4);
\draw[-] (v4) -- (l2);
\draw[-] (v3) -- (l3);
\draw[-] (v5) -- (l4);
\end{tikzpicture}
% 
\nonumber \\ & 
%
+ \frac{1}{2} \,
%
 \begin{tikzpicture}[scale=2,baseline={([yshift=-.4ex]current bounding box.center)}]
\node (v1) at (0.0,0.) [Gamma] {}; 
\node (v3) at (0.0,-0.6) [Pi] {}; 
\node (l1) at (-0.4,0) {}; 
\node (l2) at (0.4,-0.6) [VEV] {}; 
\node (l3) at (0,-1.0) [VEV] {}; 
\node (l4) at (0,0.4) [VEV] {}; 
\draw[double distance=4mm,thick] (l1) -- (v1);
\draw[double distance=4mm,thick] (v1) -- (v3);
\draw[-] (v3) -- (l2);
\draw[-] (v3) -- (l3);
\draw[-] (v1) -- (l4);
\end{tikzpicture}
%
+ \frac{1}{4} \,
%
 \begin{tikzpicture}[scale=2,baseline={([yshift=-.4ex]current bounding box.center)}]
\node (v1) at (0.0,0.) [Gamma] {}; 
\node (v2) at (-0.6,0.) [VR] {\,$\overline{V}^{(4)}$\,}; 
\node (v3) at (0.0,-0.6) [Pi] {}; 
\node (l1) at (-1.,0) {}; 
\node (l2) at (0.4,-0.6) [VEV] {}; 
\node (l3) at (0,-1.0) [VEV] {}; 
\node (l4) at (0,0.4) [VEV] {}; 
\draw[double distance=4mm,thick] (l1) -- (v2);
\draw[double distance=4mm,thick] (v1) -- (v2);
\draw[double distance=4mm,thick] (v1) -- (v3);
\draw[-] (v3) -- (l2);
\draw[-] (v3) -- (l3);
\draw[-] (v1) -- (l4);
\end{tikzpicture}
% 
\nonumber \\ & 
%
%
+ \frac{1}{4} \,
%
 \begin{tikzpicture}[scale=2,baseline={([yshift=-.4ex]current bounding box.center)}]
\node (v1) at (0.0,0.) [Gamma] {}; 
\node (v3) at (0.0,-0.6) [Pi] {}; 
\node (v5) at (0.0,0.6) [Pi] {}; 
\node (l1) at (-0.4,0) {}; 
\node (l2) at (0.4,-0.6) [VEV] {}; 
\node (l3) at (0,-1.0) [VEV] {}; 
\node (l4) at (0,1.) [VEV] {}; 
\draw[double distance=4mm,thick] (l1) -- (v1);
\draw[double distance=4mm,thick] (v1) -- (v3);
\draw[double distance=4mm,thick] (v1) -- (v5);
\draw[-] (v3) -- (l2);
\draw[-] (v3) -- (l3);
\draw[-] (v5) -- (l4);
\end{tikzpicture}
%
+ \frac{1}{8} \,
%
 \begin{tikzpicture}[scale=2,baseline={([yshift=-.4ex]current bounding box.center)}]
\node (v1) at (0.0,0.) [Gamma] {}; 
\node (v2) at (-0.6,0.) [VR] {\,$\overline{V}^{(4)}$\,}; 
\node (v3) at (0.0,-0.6) [Pi] {}; 
\node (v5) at (0.0,0.6) [Pi] {}; 
\node (l1) at (-1.,0) {}; 
\node (l2) at (0.4,-0.6) [VEV] {}; 
\node (l3) at (0,-1.0) [VEV] {}; 
\node (l4) at (0,1.) [VEV] {}; 
\draw[double distance=4mm,thick] (l1) -- (v2);
\draw[double distance=4mm,thick] (v1) -- (v2);
\draw[double distance=4mm,thick] (v1) -- (v3);
\draw[double distance=4mm,thick] (v1) -- (v5);
\draw[-] (v3) -- (l2);
\draw[-] (v3) -- (l3);
\draw[-] (v5) -- (l4);
\end{tikzpicture}
% 
\nonumber \\ & 
%
+  \,
%
 \begin{tikzpicture}[scale=2,baseline={([yshift=-.4ex]current bounding box.center)}]
\node (v1) at (0.0,0.) [Gamma] {}; 
\node (v3) at (0.0,-0.6) [Pi] {}; 
\node (v4) at (0.6,0.) [Pi] {}; 
\node (l1) at (-0.4,0) {}; 
\node (l2) at (1.,0.) [VEV] {}; 
\node (l3) at (0,-1.0) [VEV] {}; 
\node (l4) at (0.6,0.4) [VEV] {}; 
\draw[double distance=4mm,thick] (l1) -- (v1);
\draw (v1) -- (v3);
\draw (v1) -- (v4);
\draw (v3) -- (v4);
\draw[-] (v4) -- (l2);
\draw[-] (v3) -- (l3);
\draw[-] (v4) -- (l4);
\end{tikzpicture}
%
+ \frac{1}{2} \,
%
 \begin{tikzpicture}[scale=2,baseline={([yshift=-.4ex]current bounding box.center)}]
\node (v1) at (0.0,0.) [Gamma] {}; 
\node (v2) at (-0.6,0.) [VR] {\,$\overline{V}^{(4)}$\,}; 
\node (v3) at (0.0,-0.6) [Pi] {}; 
\node (v4) at (0.6,0.) [Pi] {}; 
\node (l1) at (-1.,0) {}; 
\node (l2) at (1.,0.) [VEV] {}; 
\node (l3) at (0,-1.0) [VEV] {}; 
\node (l4) at (0.6,0.4) [VEV] {}; 
\draw[double distance=4mm,thick] (l1) -- (v2);
\draw[double distance=4mm,thick] (v1) -- (v2);
\draw (v1) -- (v3);
\draw (v1) -- (v4);
\draw (v3) -- (v4);
\draw[-] (v4) -- (l2);
\draw[-] (v3) -- (l3);
\draw[-] (v4) -- (l4);
\end{tikzpicture} \,\,.
\label{eq:d3Pi_dphi3_alpha}
\end{align}
While performing the analysis of the diagrams, one may note the following: $\delta \lambda_0$ is still finite and differentiation w.r.t. to three propagators yields $\sim (\alpha_R + \lambda_R \phi_R)^2$ which is also finite. However, all other counterterms, besides $\delta \lambda_0$ are not finite, and hence one needs to be careful when such quantities multiply divergent loop integrals. The explicit form of the counterterms $\delta \alpha_3$ and $\delta \lambda_4$ are obtained by placing the renormalization conditions 
\begin{equation}
	\Gamma^{(3)}(p_{1 \ast},p_{2 \ast},p_{3 \ast}) = -\alpha_R\,,\qquad \Gamma^{(4)}(p_{1 \ast},p_{2 \ast},p_{3 \ast},p_{4 \ast}) = -\lambda_R\,.
\end{equation}
For the numerical evaluation of the three- and
four-point functions, one can then use the same
numerical procedure as outlined above for $G$.



%---------------------------------------------------------------------------- 

%----------------------------------------------------------------------------
%Yukawa Theory
%----------------------------------------------------------------------------
\section{Fermionic Sunset Approximation}
\label{sec:fermions}

Including fermions, we obtain the following effective action
\begin{align}
&\Gamma^{\text{2PI}}_{\text{int}}[\phi_R,G_R, D_R] = 
- \frac{1}{2}\int_{x} (\delta Z_{\phi,0} \square_x + \delta m^2_0) G_R(x,y) \big|_{x=y} 
- \frac{1}{2}\int_x \phi_R(x)  (\delta Z_{\phi,2} \square_x + \delta m^2_2) \phi_R(x)
  \nonumber \\ 
& - \int_x \left[\frac{1}{8}(\lambda_R + \delta \lambda_0) G^2_R(x,x)
   + \frac{1}{4} (\lambda_R + \delta \lambda_2) G_R(x,x) \phi^2_R(x)
   + \frac{1}{4!} (\lambda_R + \delta \lambda_4) \phi^4_R(x)
\right] \nonumber \\
&- \int_x \left[\frac{1}{2}(\alpha_R + \delta \alpha_1) \phi_R(x)G_R(x,x)
   + \frac{1}{3!} (\alpha_R + \delta \alpha_3) \phi^3_R(x)
\right] \nonumber \\
& +\frac{\ii}{12} \int_x \int_y \left(\alpha_R +\lambda_R\phi(x)\right)\left(\alpha_R +\lambda_R\phi(y)\right) \,G^3_R(x,y) \nonumber \\
& +\int_x(i\delta Z_{\psi,0}\myslash{\partial}_x-\delta M_0)D_R(x,y)\big|_{x=y} \nonumber \\
& -\frac{\ii}{2}\int_x \int_y (g_R+\delta g_0)^2 G_R(x,y)\text{tr}[D_R(x,y)D_R(y,x)] - \int_x \delta t_1 \phi_R(x)\nonumber \\
& -\int_x \left[(g_R+\delta g_1)\phi_R(x)\text{tr}[D_R (x,x)]\right] \,.
\label{gammaint_fermions}
\end{align}
where we have already set the corresponding coupling counterterms for the scalar sunset contribution to 0. We retain the counterterm for the Yukawa coupling corresponding to the fermionic sunset diagram, $\delta g_0$, for the moment. Note that as long as the fermionic mass $M_R$ does not vanish, we require an additional trilinear coupling $\alpha_R$ to account for potential divergences in scalar three-point functions with a fermionic loop. We can then obtain the following additional kernels involving fermions and scalars \cite{Reinosa:2005pj}
\begin{equation}
        \Lambda_{\psi \psi}(p_1,p_2,p_3,p_4)_{ab,cd} \equiv -\frac{\delta^2 \Gamma^{\text{2PI}}_{\text{int}}}{\delta D^{ba}_{R}(p)\,\delta D^{cd}_{R}(q)} = i(g_R+\delta g_0)^2 \,\delta_{db}\,G_R(p_3-p_1)\,\delta_{ac}\,,
\end{equation}
    
\begin{align}
        \Lambda^{(4)}_{\psi \phi}(p_1,p_2,p_3,p_4)_{ab} \equiv -2\frac{\delta^2 \Gamma^{\text{2PI}}_{\text{int}}}{\delta D^{ba}_R(p)\,\delta G_R(q)} = 2i(g_R+\delta g_0)^2 \, D_{R}(p_1-p_3)_{ab} \,,
%        = \left[\Lambda^{(4)\,\phi \psi}(p,q)\right]^{T} 
\end{align}

\begin{align}
        \left(\Lambda^{(3)}_{\psi \phi}\right)_{ab} \equiv -\frac{\delta^2 \Gamma^{\text{2PI}}_{\text{int}}}{\delta \phi_R \,\delta D^{ba}_R(p)}
        = -(g_R+\delta g_1) \delta_{ab} \,,
\end{align}
alongside the scalar kernels that we had in the scalar sunset approximation, c.f. \eqref{eq:4ptkalpha1}, \eqref{eq:4ptkalpha2} and \eqref{eq:3ptkalpha}. For now, we have denoted the spinor indices by lowercase Latin characters. The Kronecker deltas refer to the identity matrix in spinor space.

% Figure environment removed


We proceed by defining first the governing BSE for the kernel $\Lambda_{\psi \psi}$ as
\begin{align}
    &V_{\psi\psi} (p_1,p_2,p_3,p_4)_{ab,cd} = \Lambda_{\psi \psi}(p_1,p_2,p_3,p_4)_{ab,cd} \nonumber \\[2mm]
    &\quad +i \int_q \Lambda_{\psi \psi}(p_1,p_2,q+p,-q)_{ab,ef}D_R(q)_{eg} V_{\psi\psi} (q+p,-q,p_3,p_4)_{g h,c d}D_R(p+q)_{hf}\nonumber \\[3mm]
    &= i(g_R+\delta g_0)^2 \delta_{db}\,G_R(p_3-p_1)\,\delta_{ac} \nonumber\\[2mm]
    &\quad - (g_R+\delta g_0)^2 \int_q G_R(q+p_2) D_R(q)_{ae} V_{\psi\psi} (q+p,-q,p_3,p_4)_{ef,cd}D_R(p+q)_{fb} \,,
\end{align}
and $p = p_1 + p_2$. This vertex function is essentially a resummation of ladder diagrams contributing to $t$-channel $\psi \overline{\psi} \rightarrow \psi \overline{\psi}$ scattering via the exchange of a scalar propagator. On inspection, we can easily discern that $V_{\psi \psi}$ is finite by counting the number of propagators. The counterterm $\delta g_0$ hence accounts for a finite renormalization which we determine using the condition
\begin{equation}
    V_{\psi\psi}(p_{1\ast},p_{2\ast},p_{3\ast},p_{4\ast})_{ab,cd} = i g_R^2 \,\delta_{db}\,G_R(p_{3\ast}-p_{1\ast})\,\delta_{ac}\,.
\end{equation}
where we continue to work in the COM frame to give the renormalization conditions. Imposing this and appropriately contracting the spinor indices (with $\sum_{a}\delta_{aa} = 4$) we obtain the following expression for $\delta g_0$
\begin{equation}
    \delta g_0 = -g_R + \frac{g_R}{\sqrt{1 - \frac{1}{4}G^{-1}_R(p_{3\ast}-p_{1\ast})\,F_2}}
\end{equation}
where 
\begin{equation}
    F_2 = i \int_q G_R(q+p_{2 \ast})D_R(q)_{ae}  V_{\psi\psi}(q+ p_{\ast},-q,p_{3\ast},p_{4\ast})_{ef,ab}D_R(p_{\ast}+q)_{fb} \,.
\end{equation}
Note that $\delta g_0$ is a finite renormalization, i.e. $\mathcal{O}(1)$ and therefore, does not modify divergent structures. Thus, for convenience, we choose to set $\delta g_0 = 0$ from this point onward.

Let us consider the BSE for the kernel $\Lambda^{(4)}_{\psi \phi}$, 
\begin{align}
    &V^{(4)}_{\psi\phi} (p_1,p_2,p_3,p_4)_{ab} = \Lambda_{\psi \phi}(p_1,p_2,p_3,p_4)_{ab} \nonumber \\[2mm]
    &\qquad   +i \int_q \Lambda^{(4)}_{\psi \phi}(p_1,p_2,q+p,-q)_{ae}D_R(q)_{ef} V^{(4)}_{\psi\phi}(q+p,-q,p_3,p_4)_{f b}G_R(p+q)
\end{align}
This is again finite by power counting. We will use this now for the three-point kernel $\Lambda^{(3)}_{\psi \phi}$ to define a BSE of the form
\begin{align}
    &V^{(3)}_{\psi\phi} (p_1,p_2,p_3)_{ab} = \left(\Lambda^{(3)}_{\psi \phi}\right)_{ab} + i\, \left(\Lambda^{(3)}_{\psi \phi}\right)_{ac}\int_q D_R(q)_{cd} V^{(4)}_{\psi\phi}(q+p_1,-q,p_2,p_3)_{db}G_R(p_1+q)\,.
\end{align}
We will make use of this now to determine the counterterm $\delta g_1$. With the renormalization condition
\begin{equation}
	V^{(3)}_{\psi\phi}(p_{1\ast})_{(\alpha\beta)} = -g_R\, \delta_{\alpha\beta} \,
\end{equation}
we then obtain
\begin{equation}
	\delta g_1 = - g_R +  \frac{g_R}{1+\frac{1}{4}F_3}\,,
\end{equation}
where 
\begin{equation}
    F_3  =  i \int_q \text{tr}\left[D_R(q)V^{(4)}_{\psi\phi}(q+p_{1\ast},-q,p_{2\ast},p_{3\ast})\right]G_R(q+p_{1\ast})\,,
\end{equation}
and where appropriate contraction of the spinor indices leads to the resultant trace.

% Figure environment removed

In dealing with the scalar four-point vertex functions, we have to treat the possibility of divergences introduced by fermionic loops. To this end, with $\oL^{(4)}$ as the base, we build the following ``modified scalar kernel'' using $\Lambda^{(4)}_{\psi\phi}$ and the four-point vertex $V_{\psi\psi}$
\begin{align}
    &\tilde{\Lambda}_{\phi\phi} (p_1,p_2,p_3,p_4) = 
    \oL^{(4)}(p_1,p_2,p_3,p_4) \nonumber \\[2mm]
    &\qquad \qquad  - \ii \int_q D_R(p+q)_{db}\, \Lambda^{(4)}_{\psi\phi}(q+p,-q,p_3,p_4)_{ba}\,D_R(q)_{ac}\,\Lambda^{(4)}_{\psi\phi}(p_1,p_2,p+q-q)_{cd}  \nonumber\\[2mm]
    &\qquad \qquad  + \ii \int_q\int_r \bigg\{ D_R(q+p)_{fb}\,\Lambda^{(4)}_{\psi\phi}(p_1,p_2,p+q,-q)_{ba}\,D_R(q)_{ae}  \nonumber \\[2mm]
    & \qquad\qquad \qquad V^{\psi\psi}(p+q,-q,r+p,-r)_{ef,gh} D_R(r)_{gc}\,\Lambda^{(4)}_{\psi\phi}(r+p,-r,p_3,p_4)_{cd}\,D_R(r+p)_{dh} \bigg\}\nonumber \\[4mm]
    &= \oL^{(4)}(p_1,p_2,p_3,p_4) -4\ii g_R^4 \int_q \text{tr}\left[D_R(q+p)D_R(q+p_1)D_R(q)D_R(q+p_3)\right] \nonumber \\[2mm]
    & \qquad + 4\ii g_R^4 \int_q\int_r \text{tr}\bigg[D_R(q+p)D_R(q+p_1)D_R(q)V^{\psi\psi}(p+q,-q,r+p,-r) \nonumber \\
    & \qquad \qquad \qquad \qquad \qquad \qquad \qquad \qquad  D_R(r)D_R(r+p_3)D_R(r)\bigg]\,,
\end{align}
where in the second equality, we have contracted the spinor indices to obtain the trace. We can now iterate this four-point scalar kernel via its usual BSE
\begin{align}
    \overline{V}^{(4)}(p_1,p_2,p_3,p_4) &= \tilde{\Lambda}_{\phi\phi}(p_1,p_2,p_3,p_4) \nonumber \\[2mm]
    &\quad + \frac{\ii}{2}\int_q \tilde{\Lambda}_{\phi\phi}(p_1,p_2,q+p,-q)\, G_R(q) G_R(p+q)\,\overline{V}^{(4)}(q+p,-q,p_3,p_4) \,.
    \label{eqn:modvertex_fermions}
\end{align}
The quartic scalar coupling divergences generated will then be absorbed into the counterterm $\delta\lambda_0$. Imposing the same renormalization condition as in the scalar sunset approximation, c.f. \eqref{eq:v4alpha_rc}, we solve for the counterterm to obtain
\begin{align}
    \delta \lambda_0 &= -\lambda_R + \frac{\lambda_R -  (\alpha_R+\lambda_R\phi_R)^2\, I_2}{1+\frac{1}{2}I_1}  \nonumber \\[2mm]
    &\qquad \qquad +4 g_R^4 \left[\frac{F_4-\frac{1}{2}F_{4V}}{1+\frac{1}{2}I_1}\right] -4 g_R^4 \left[\frac{F_V-\frac{1}{2}F_{VV}}{1+\frac{1}{2}I_1}\right]\,,
\end{align}
where $I_{1,2}$ are the same integrals in \eqref{eqn:vertexintegral1} and \eqref{eqn:vertexintegral2}, defined at the renormalization point, and the following notation has been introduced for the new loop integrals pertaining to the fermions
\begin{equation}
    F_4 = \ii\int_q \text{tr}\left[D_R(q+p_{\ast})D_R(q+p_{2 \ast})D_R(q)D_R(q+p_{4 \ast})\right]\,,
\end{equation}
    
\begin{equation}
    F_{4V} = \int_q \int_r \text{tr}\left[D_R(q+p_{\ast})D_R(q+p_{2 \ast})D_R(q)D_R(q-r)\right]G_R(r)G_R(p_{\ast}+r)\overline{V}^{(4)}(r+p_{\ast},p_{3\ast})\,,
\end{equation}
    
\begin{align}
    F_{V} = \ii\int_q \int_r \text{tr}\bigg[D_R(q+p_{\ast})D_R(q+p_{2 \ast})D_R(q)&V_{\psi\psi}(p_{\ast}+q,r+p_{\ast}) \nonumber \\
    &D_R(r)D_R(r+p_{4 \ast})D_R(r+p_{\ast})\bigg]\,,
\end{align}
    
\begin{align}
    F_{VV} &= \int_k \int_q \int_r \bigg\{\text{tr}\bigg[D_R(q+p_{\ast})D_R(q+p_{2 \ast})D_R(q)V_{\psi\psi}(p_{\ast}+q,r+p_{\ast})\nonumber \\
& \quad \quad \quad D_R(r)D_R(r-k)D_R(r+p_{\ast})\bigg] 
 G_R(k)G_R(k+p_{\ast})\overline{V}^{(4)}(k+p_{\ast},p_{3 \ast})\bigg\}\,.
\end{align}  
Note that $F_V$ and $F_{VV}$ are both finite, and $F_4$ and $F_{4V}$ are logarithmically divergent. The remaining counterterms are $\delta \lambda_2$ and $\delta \alpha_1$ are obtained in the exact same manner as in the scalar sunset case, and the expressions are the same \eqref{eqn:dellambda2_tri} and \eqref{eqn:delalpha1}, with the vertex function now being defined with the modified four-point scalar kernel to include the contributions from fermions.

We can now turn to the formulation of the gap equations to determine the form of the fermionic and scalar propagators. These are given by
\begin{align}
    iG^{-1}_R(p) &= p^2 -m^2_R- \ii\overline{\Pi}(p) \nonumber \\[2mm]
    &= (p^2 - m^2_R) + (\delta Z_{\phi,0} \, p^2 - \delta m^2_0) - (\alpha_R+\delta\alpha_1)\phi_R  - \frac{(\lambda_R + \delta \lambda_2)}{2}  \phi_R^2 \nonumber \\[2mm]
    &\qquad  - \frac{(\lambda_R + \delta \lambda_0)}{2}\, \mathcal{T} + \frac{(\alpha_R + \lambda_R \phi_R)^2}{2}\, \mathcal{I}(p) - \ii g^2_R\,\int_q \text{tr}[D_R(q)D_R(p+q)] \nonumber \\[2mm]
    &\equiv (p^2 - m^2_R) + (\delta Z_{\phi,0} \, p^2 - \delta m^2_0) - (\alpha_R+\delta\alpha_1)\phi_R  - \frac{(\lambda_R + \delta \lambda_2)}{2}  \phi_R^2 \nonumber \\[2mm]
    &\qquad  - \frac{(\lambda_R + \delta \lambda_0)}{2}\, \mathcal{T} + \frac{(\alpha_R + \lambda_R \phi_R)^2}{2}\, \mathcal{I}(p) - g^2_R\left[p^2 \mathcal{F}_1(p) + \mathcal{F}_2(p)\right]
    \label{eqn:scalargap_os}
\end{align}

\begin{align}
    iD^{-1}_R(p) &= \myslash{p} - M_R-\ii\overline{\Sigma}(\myslash{p}) \nonumber \\[2mm]
    &= \myslash{p} - M_R + (\delta Z_{\psi,0}\,\myslash{p} -\delta M_0) -  (g_R+\delta g_1)\phi_R - \ii g_R^2\int_q D_R(p+q)G_R(q) \nonumber \\[2mm]
    &\equiv \myslash{p} - M_R + (\delta Z_{\psi,0}\,\myslash{p} -\delta M_0)  -  (g_R+\delta g_1)\phi_R - g_R^2 \left[X(p) \myslash{p} + Y(p)\right] \,,
    \label{eqn:fermiongap_os}
\end{align}
and are evidently coupled. For the loop integrals related to fermions, we have decomposed them into forms based on possible Lorentz structures. We now impose the appropriate on-shell renormalization conditions to obtain the counterterms related to the propagators. For the scalar propagator, we have
\begin{equation}
    \delta Z_{\phi,0} =  -\frac{(\alpha_R+\lambda_R \phi_R)^2}{2}\,\frac{\partial \mathcal{I}(p)}{\partial p^2}\bigg|_{p^2=m^2_R} + g_R^2 \left[\mathcal{F}_1(p) + m^2_R\frac{\partial \mathcal{F}_1(p)}{\partial p^2}  + \frac{\partial \mathcal{F}_2(p)}{\partial p^2}\right]\bigg|_{p^2=m^2_R} \,,
    \label{eqn:delZ_scalar}
\end{equation}
\begin{align}
    \delta m^2_0 &=  \delta Z_{\phi,0} m^2_R - (\alpha_R+\delta\alpha_1)\phi_R - \frac{(\lambda_R + \delta \lambda_2)}{2} \phi_R^2  -\frac{(\lambda_R+\delta \lambda_0)}{2}\, \mathcal{T}  \nonumber \\[2mm]
    &\qquad + \frac{(\alpha_R +\lambda_R \phi_R)^2}{2}\,\mathcal{I}(p)\big|_{p^2=m^2_R} - g_R^2 \left[m^2_R \mathcal{F}_1(p)  + \mathcal{F}_2(p)\right]\big|_{p^2=m^2_R}\,,
\end{align}
and for the fermionic propagator
\begin{equation}
    \delta Z_{\psi,0} = g_R^2 X(p)\big|_{p^2 = M^2_R}  + 2g_R^2\, M_R\,\left[M_R \frac{\partial X(p)}{\partial p^2} + \frac{\partial Y(p)}{\partial p^2}\right] \bigg|_{p^2=M^2_R}\,,
    \label{eqn:delZ_ferm}
\end{equation}

\begin{align}
    \delta M_0 &=  \delta Z_{\psi,0} M_R - (g_R+\delta g_1)\phi_R - g_R^2 [M_R \,X(p) + Y(p)]\big|_{p^2 = M^2_R}   \,\,.
\end{align}
Note that the counterterms for the scalar and fermionic wave function renormalizations are not finite, due to the loop integrals not always being subtracted. 

We proceed now in the same manner as in the scalar sunset approximation: we substitute these counterterms into the gap equations to obtain the following coupled integral equations to solve for the propagators
\begin{align}
    &iG^{-1}_R(p) = (p^2-m^2_R)\left[1 -\frac{(\alpha_R +\lambda_R \phi_R)^2}{2}\, \frac{\partial \mathcal{I}(q)}{\partial q^2} +g^2_R\left(m^2_R \frac{\partial \mathcal{F}_1(q)}{\partial q^2}  +\frac{\partial \mathcal{F}_1(q)}{\partial q^2}\right)\right]\bigg|_{q^2=m^2_R} \nonumber \\[2mm] 
    & \qquad +\frac{(\alpha_R +\lambda_R \phi_R)^2}{2}\left[\mathcal{I}(p) - \mathcal{I}(q)\big|_{q^2=m^2_R}\right] \nonumber \\[2mm] 
    & \qquad - g^2_R \left[p^2\left(\mathcal{F}_1(p) - \mathcal{F}_1(q)\big|_{q^2=m^2_R}\right) + \left(\mathcal{F}_2(p) - \mathcal{F}_2(q)\big|_{q^2=m^2_R}\right)\right]\,,
    \label{scalargapeqn}
\end{align}

\begin{align}
    iD^{-1}_R(p) &= (\myslash{p}-M_R)\left[1+2g_R^2 M_R\,\left(M_R \frac{\partial X(q)}{\partial q^2} + \frac{\partial Y(q)}{\partial q^2}\right)\right]\bigg|_{q^2=M^2_R} \nonumber \\[2mm]
    &\qquad  - g^2_R \left[\myslash{p} \left(X(p)-X(q)\big|_{q^2=M^2_R}\right) + \left(Y(p^2)-Y(q)\big|_{q^2=M^2_R}\right)\right] \nonumber \\[2mm]
    &\equiv W(p)\myslash{p} - Z(p) \,\,.
    \label{fermiongapeqn}
\end{align}
These are manifestly finite as divergences drop out due to the subtraction from a fixed point or differentiation of the divergent functions. In the last step for the fermionic propagator, we have defined the following quantities 
\begin{align}
    &W(p) = 1 - g^2_R \left(X(p)-X(q)\big|_{q^2=M^2_R}\right) + 2 g^2_R M_R \left(M_R \frac{\partial X(q)}{\partial q^2} + \frac{\partial Y(q)}{\partial q^2}\right)\bigg|_{q^2=M^2_R} \,, \\[3mm]
    &Z(p) = M_R -   g^2_R \left(Y(p)-Y(q)\big|_{q^2=M^2_R}\right) -2 g^2_R M^2_R \left(M_R \frac{\partial X(q)}{\partial q^2} + \frac{\partial Y(q)}{\partial q^2}\right)\bigg|_{q^2=M^2_R} \,,
\end{align}
according to which we can explicitly write down the expressions for $X(p)$ and $Y(p)$,
\begin{equation}
	X(p) = i\int_q W(p+q) G_R(q) \,,\qquad Y(p) = i \int_q Z(p+q) G_R(q)\,.
\end{equation}
For the trace that appears in the scalar propagator, we resolve this as
\begin{align}
    &\text{tr}[D_R(q)D_R(p+q)] = \text{tr}\left\{\frac{i}{W(q)\myslash{q} - Z(q)}\, \frac{i}{W(p+q)(\myslash{p}+\myslash{q}) - Z(p+q)} \right\} \nonumber \\[2mm]
    &= \frac{\text{tr}\left\{[W(q)\myslash{q} + Z(q)][W(p+q)(\myslash{p}+\myslash{q}) + Z(p+q)] \right\}}{[W^2(q)q^2 - Z^2(q)]\,[W^2(p+q)(p+q)^2 - Z^2(p+q)]} \nonumber \\[2mm]
    &= 4 \,\frac{(p\cdot q + q^2)W(q)W(p+q) + Z(q)Z(p+q)}{[W^2(q)q^2 - Z^2(q)]\,[W^2(p+q)(p+q)^2 - Z^2(p+q)]} \nonumber \\[2mm]
    &= p^2\left\{\frac{-2\,W(q) W(p+q)}{[W^2(q)q^2 -Z^2(q)][W^2(p+q)(p+q)^2 -Z^2(p+q)]}\right\}\nonumber\\[2mm] 
    &\qquad + 2\left\{\frac{((p+q)^2 -q^2)W(q) W(p+q)+ Z(q) Z(p+q)}{[W^2(q)q^2 -Z^2(q)][W^2(p+q)(p+q)^2 -Z^2(p+q)]}\right\} 
\end{align}
where in the second last step we have used $ 2 p\cdot q = (p+q)^2 - p^2 - q^2$. This finally gives the expressions for the functions $\mathcal{F}_1$ and $\mathcal{F}_2$
\begin{align}
	&\mathcal{F}_1(p) = - 2\ii \int_q \frac{W(q) W(p+q)}{[W^2(q)q^2 -Z^2(q)][W^2(p+q)(p+q)^2 -Z^2(p+q)]}\,, \\[2mm]
	&\mathcal{F}_2(p) =  2\ii \int_q \frac{((p+q)^2 -q^2)W(q) W(p+q)+ Z(q) Z(p+q)}{[W^2(q)q^2 -Z^2(q)][W^2(p+q)(p+q)^2 -Z^2(p+q)]}\,.
\end{align}
%
We now have the entire setup to solve the gap equations, which we approach in the same iterative manner outlined for the scalar sunset approximation, beginning with the free propagators. This yields
\begin{align}
    &X^{(1)}(p^2) =  \frac{1}{16\pi^2}\left[B_1(p^2,m^2_R,M^2_R)+B_0(p^2,m^2_R,M^2_R)\right] \,,\\[4mm]
    &Y^{(1)}(p^2) = \frac{1}{16\pi^2}\left[M_R \, B_0(p^2,m^2_R,M^2_R)\right] \,, \\[4mm]
    &i(G^{-1}_R(p))^{(1)} = p^2 \bigg\{1 -\frac{g_R^2}{8\pi^2}\left[B_0(p^2,M^2_R,M^2_R)-B_0(m^2_R,M^2_R,M^2_R)\right] \nonumber \\[2mm]
    & \qquad \qquad - \frac{g_R^2}{4\pi^2}\left(M^2_R-\frac{m^2_R}{2}\right)\dot{B}_0(m^2_R,M^2_R,M^2_R) -\frac{(\alpha_R +\lambda_R \phi_R)^2}{32\pi^2}\, \dot{B}_0(m^2_R,m^2_R,m^2_R)\bigg\} \nonumber \\[2mm]
    & -m^2_R \bigg\{1- \frac{g_R^2}{8\pi^2}\frac{M^2_R}{m^2_R}\left[B_0(p^2,M^2_R,M^2_R)-B_0(m^2_R,M^2_R,M^2_R)\right]
    \nonumber \\[2mm]
    & \qquad \qquad -\frac{g_R^2}{4\pi^2}\left(M^2_R-\frac{m^2_R}{2}\right)\dot{B}_0(m^2_R,M^2_R,M^2_R) +\frac{(\alpha_R +\lambda_R \phi_R)^2}{32\pi^2}\, \dot{B}_0(m^2_R,m^2_R,m^2_R)\bigg\} \nonumber \\[2mm]
    &\qquad \qquad + \frac{(\alpha_R +\lambda_R \phi_R)^2}{32\pi^2}\left[B_0(p^2,m^2_R,m^2_R)-B_0(m^2_R,m^2_R,m^2_R)\right]\,,
\end{align}
which are converted to Euclidean space for the next iteration. We focus on the case of a large Yukawa coupling, $g_R = 2$, and set smaller couplings pertaining to scalars. We again see that the relative difference between successive iterations drops by about two orders of magnitude for the scalar propagator (Fig. \ref{fig:scalardiffyukawa}) and the functions $W(p)$ (Fig. \ref{fig:fermionwdiff}) and $Z(p)$ (Fig. \ref{fig:fermionzdiff}), demonstrating the convergence of this approach. The spikes correspond to cases where the corresponding difference vanishes.

% Figure environment removed


% Figure environment removed

% Figure environment removed

We now outline the procedure to obtain the remaining counterterms, which pertain to the scalar field. We must start with the generic expression involving field derivatives of the effective action and include the relevant modifications due to the presence of fermions. 

Starting with the scalar one-point function, we have 
\begin{align}
    \frac{\delta\Gamma^{\text{2PI}}}{\delta\phi_R(x_1)} = &\frac{\delta \Gamma^{\text{2PI}}}{\delta \phi_R(x_1)}\bigg|_{\phi_R,\,G_R,\,D_R}  + \int_{y_1,y_2}\frac{\delta\Gamma^{\text{2PI}}}{\delta G_R(y_1,y_2)}\bigg|_{\phi_R,\,G_R,\,D_R} \frac{\delta G_R(y_1,y_2)}{\delta\phi_R(x_1)}  \nonumber \\
    &+   \int_{y_1,y_2}\text{tr}\bigg\{\frac{\delta\Gamma^{\text{2PI}}}{\delta D_R(y_1,y_2)}\bigg|_{\phi_R,\,G_R,\,D_R} \frac{\delta D_R(y_1,y_2)}{\delta\phi_R(x_1)}\bigg\} \nonumber 
\end{align}
where the second and third terms are obtained from the chain rule. The stationarity conditions of $\Gamma^{\text{2PI}}$,
\begin{equation}
	\frac{\delta\Gamma^{\text{2PI}}}{\delta G_R}\bigg|_{\phi_R,\,G_R,\,D_R} = \frac{\delta\Gamma_{\text{2PI}}}{\delta D_R}\bigg|_{\phi_R,\,G_R,\,D_R} \stackrel{!}{=} 0 \,,
\end{equation}
cause these to drop out. We thus obtain, by converting to momentum space, 
\begin{align}
&    \Gamma^{(1)} = \frac{\delta \Gamma^{\text{2PI}}}{\delta \phi_R}\bigg|_{p^2 =0} = -\delta t_1 - (m^2_R +\delta m^2_2) \phi_R - \frac{(\alpha_R + \delta \alpha_3)}{2}\phi^2_R - \frac{(\lambda_R + \delta \lambda_4)}{6}\phi^3_R \nonumber \\[2mm]
    &\quad - \frac{1}{2}\left[(\alpha_R+\delta \alpha_1) + (\lambda_R + \delta \lambda_2)\phi_R\right]\,\mathcal{T} + \frac{\lambda_R \left(\alpha_R + \lambda_R \phi_R\right)}{6} \,\mathcal{S} -(g_R+\delta g_1)\int_q \text{tr}\left[D_R(q)\right] \stackrel{!}{=} 0   \,\,,
    \label{eqn:gamma1_mincond}
\end{align}
which is the same minimisation condition as in the scalar sunset, but with a new contribution from the fermionic tadpole, which is the last term in the second line. 

Consider the physical two-point function of the scalar field,
\begin{align}
    &\Gamma^{(2)}(x_1,x_2) \equiv \frac{\delta^2\Gamma^{\text{2PI}}}{\delta\phi(x_1)\delta\phi(x_2)} \nonumber\\[2mm]
    &= \frac{\delta^2 \Gamma^{\text{2PI}}}{\delta \phi_R(x_1)\delta \phi_R(x_2)}\bigg|_{\phi_R,\,G_R,\,D_R} + \int_{y_1,...,y_4}\frac{\delta^2\Gamma^{\text{2PI}}}{\delta\phi_R(x_1)\delta G_R(y_1,y_2)}\bigg|_{\phi_R,\,G_R,\,D_R}\frac{\delta G_R(y_1,y_2)}{\delta\phi_R(x_2)} \nonumber \\[3mm]
    &\qquad \qquad \qquad \qquad \qquad \qquad  +   \int_{y_1,...,y_4}\text{tr}\bigg\{\frac{\delta^2\Gamma^{\text{2PI}}}{\delta \phi(x_1)\delta D_R(y_1,y_2)}\bigg|_{\phi_R,\,G_R,\,D_R}\frac{\delta D_R(y_1,y_2)}{\delta\phi_R(x_2)}\bigg\} \nonumber  \\[4mm]
    &= iG^{-1}_{0,R} + \frac{\delta^2\Gamma^{\text{2PI}}_{\text{int}}}{\delta\phi_R(x_1)\delta\phi_R(x_2)}\bigg|_{\phi_R,\,G_R,\,D_R} \nonumber \\[3mm]
    &\qquad \qquad + \int_{y_1,...,y_4}\frac{\delta^2\Gamma^{\text{2PI}}_{\text{int}}}{\delta\phi_R(x_1)\delta G_R(y_1,y_2)}\bigg|_{\phi_R,\,G_R,\,D_R}G_R(y_1,y_3)\frac{\delta\overline{\Pi}(y_3,y_4)}{\delta\phi_R(x_2)}G_R(y_4,y_2) \nonumber \\[3mm]
    &\qquad \qquad +\int_{y_1,...,y_4}\text{tr}\bigg\{\frac{\delta^2\Gamma^{\text{2PI}}_{\text{int}}}{\delta \phi_R(x_1)\delta D_R(y_1,y_2)}\bigg|_{\phi_R,\,G_R,\,D_R}D_R(y_1,y_3)\frac{\delta \overline{\Sigma}(y_3,y_4)}{\delta\phi_R(x_2)}D_R(y_4,y_2)\bigg\}  \,.
    \label{eqn:scal2pt_ferm}
\end{align}
The last line gives explicit fermionic contributions by virtue of $\delta \overline{\Sigma}/\delta \phi_R$. In a similar manner to \eqref{eq:dPidphi}, it can be expanded out using the following diagrammatic equation 
\begin{equation}
\frac{\delta \overline{\Sigma}(x_1,x_2)}{\delta \phi_R(x_3)} \equiv
\begin{tikzpicture}[scale=2,baseline={([yshift=-.4ex]current bounding box.center)}]
\node (threepoint) at (3,0) [Sigma] {}; 
\coordinate (tra) at (2.6,0);
\draw[double distance=4mm,thick,cyan] (threepoint) -- (tra) node[right,above]{\hspace*{-2mm}1\,}
node[right,below]{\hspace*{-2mm}2\,};
\node (l3) at (3.4,0) [VEV] {}; 
\draw[-] (threepoint) -- (l3) node[right] {3} ;
 \end{tikzpicture}
 %
=
% 
 \begin{tikzpicture}[scale=2,baseline={([yshift=-.4ex]current bounding box.center)}]
\node (threepoint) at (3,0) [Gamma] {}; 
\coordinate (tra) at (2.6,0);
\draw[double distance=4mm,thick,cyan] (threepoint) -- (tra) node[right,above]{\hspace*{-2mm}1\,}
node[right,below]{\hspace*{-2mm}2\,};
\node (l3) at (3.4,0) [VEV] {}; 
\draw[-] (threepoint) -- (l3) node[right] {3} ;
 \end{tikzpicture}
% 
+ \quad
% 
\begin{tikzpicture}[scale=2,baseline={([yshift=-.4ex]current bounding box.center)}]
\node (fourpoint) at (3,0) [VR] {\,$V_{\psi\psi}$\,}; 
\node (threepoint) at (3.5,0) [Gamma] {}; 
\coordinate (tra) at (2.6,0);
\draw[double distance=4mm,thick,cyan] (fourpoint) -- (tra) node[right,above]{\hspace*{-2mm}1\,}
node[right,below]{\hspace*{-2mm}2\,};
\node (l3) at (4,0) [VEV] {}; 
\draw[-] (threepoint) -- (l3) node[right] {3} ;
\draw[double distance=4mm,thick,cyan] (fourpoint) -- (threepoint);\,,
\end{tikzpicture}
+ \quad
% 
\begin{tikzpicture}[scale=2,baseline={([yshift=-.4ex]current bounding box.center)}]
\node (fourpoint) at (3,0) [VR] {\,$V^{(4)}_{\psi\phi}$\,}; 
\node (threepoint) at (3.5,0) [Gamma] {}; 
\coordinate (tra) at (2.6,0);
\draw[double distance=4mm,thick,cyan] (fourpoint) -- (tra) node[right,above]{\hspace*{-2mm}1\,}
node[right,below]{\hspace*{-2mm}2\,};
\node (l3) at (4,0) [VEV] {}; 
\draw[-] (threepoint) -- (l3) node[right] {3} ;
\draw[double distance=4mm,thick] (fourpoint) -- (threepoint);\,,
\end{tikzpicture}
\end{equation}
where we have the new building blocks involving fermions. cyan lines indicate fermionic propagators. Note the appearance of the vertex functions $V_{\psi\psi}$ and $V_{\psi\phi}$ now, which appear as a solution to the self-energy equation of $\delta \overline{\Sigma}/\delta \phi_R$, in the similar manner how $\overline{V}^{(4)}$ appeared as solution to $\delta \overline{\Pi}/\delta \phi_R$. We stress that the $\oV^{(4)}$ used is \eqref{eqn:modvertex_fermions} with the modified four-point scalar kernel. Furthermore, note that correspondingly $\delta \overline{\Pi}/\delta \phi_R$ is modified from \eqref{eq:dPidphi} and gains the additional part
\begin{equation}
\begin{tikzpicture}[scale=2,baseline={([yshift=-.4ex]current bounding box.center)}]
\node (threepoint) at (3,0) [Pi] {}; 
\coordinate (tra) at (2.6,0);
\draw[double distance=4mm,thick] (threepoint) -- (tra) node[right,above]{\hspace*{-2mm}1\,}
node[right,below]{\hspace*{-2mm}2\,};
\node (l3) at (3.4,0) [VEV] {}; 
\draw[-] (threepoint) -- (l3) node[right] {3} ;
 \end{tikzpicture}
 %
=
% 
 \begin{tikzpicture}[scale=2,baseline={([yshift=-.4ex]current bounding box.center)}]
\node (threepoint) at (3,0) [Gamma] {}; 
\coordinate (tra) at (2.6,0);
\draw[double distance=4mm,thick] (threepoint) -- (tra) node[right,above]{\hspace*{-2mm}1\,}
node[right,below]{\hspace*{-2mm}2\,};
\node (l3) at (3.4,0) [VEV] {}; 
\draw[-] (threepoint) -- (l3) node[right] {3} ;
 \end{tikzpicture}
% 
+ \quad
% 
\begin{tikzpicture}[scale=2,baseline={([yshift=-.4ex]current bounding box.center)}]
\node (fourpoint) at (3,0) [VR] {\,$\oV^{(4)}$\,}; 
\node (threepoint) at (3.5,0) [Gamma] {}; 
\coordinate (tra) at (2.6,0);
\draw[double distance=4mm,thick] (fourpoint) -- (tra) node[right,above]{\hspace*{-2mm}1\,}
node[right,below]{\hspace*{-2mm}2\,};
\node (l3) at (4,0) [VEV] {}; 
\draw[-] (threepoint) -- (l3) node[right] {3} ;
\draw[double distance=4mm,thick] (fourpoint) -- (threepoint);\,,
\end{tikzpicture}
+ \quad
% 
\begin{tikzpicture}[scale=2,baseline={([yshift=-.4ex]current bounding box.center)}]
\node (fourpoint) at (3,0) [VR] {\,$V^{(4)}_{\phi\psi}$\,}; 
\node (threepoint) at (3.5,0) [Gamma] {}; 
\coordinate (tra) at (2.6,0);
\draw[double distance=4mm,thick] (fourpoint) -- (tra) node[right,above]{\hspace*{-2mm}1\,}
node[right,below]{\hspace*{-2mm}2\,};
\node (l3) at (4,0) [VEV] {}; 
\draw[-] (threepoint) -- (l3) node[right] {3} ;
\draw[double distance=4mm,thick,cyan] (fourpoint) -- (threepoint);\,,
\end{tikzpicture}
\end{equation}
where $V^{(4)}_{\phi\psi}$ is the transpose of the vertex function $V^{(4)}_{\psi\phi}$. We continue now with the diagrammatic analysis and look at the 3-point and 4-point functions, required to obtain the counterterms $\delta \alpha_3$ and $\delta \lambda_4$. Besides the scalar contributions in \eqref{eq:Gamma3_alpha} and \eqref{eq:Gamma4_alpha}, we have in addition the following ones from the fermions
\begin{align}
\Gamma^{(3)}\bigg|_{\text{fermions}} ~=~ 
%
\begin{tikzpicture}[scale=2,baseline={([yshift=-.4ex]current bounding box.center)}]
\node (v1) at (0,0.3) [Gamma] {}; 
\node (v2) at (0,-0.3) [Sigma] {}; 
\node (v3) at (0.6,0.3) [Sigma] {}; 
\node (l1) at (-0.4,0.3) [VEV] {}; 
\node (l2) at (0.9,0.3) [VEV] {}; 
\node (l3) at (0.,-0.7) [VEV] {}; 
\draw[-,cyan] (v1) -- (v2);
\draw[-,cyan] (v1) -- (v3);
\draw[-,cyan] (v2) -- (v3);
\draw[-] (v1) -- (l1);
\draw[-] (v3) -- (l2);
\draw[-] (v2) -- (l3);
\end{tikzpicture}
%
+ \,\,
%
 \begin{tikzpicture}[scale=2,baseline={([yshift=-.4ex]current bounding box.center)}]
\node (v1) at (0,0.3) [Gamma] {}; 
\node (v2) at (0.6,0.3) [Sigma] {}; 
\node (l1) at (-0.4,0.3) [VEV] {}; 
\node (l2) at (0.9,0.3) [VEV] {}; 
\node (l3) at (0.6,-0.1) [VEV] {}; 
\draw[double distance=4mm,thick,cyan] (v1) -- (v2);
\draw[-] (v1) -- (l1);
\draw[-] (v2) -- (l2);
\draw[-] (v2) -- (l3);
\end{tikzpicture} \,\,\,\,,
\label{eq:Gamma3_fermions}
\end{align}

%Here is d^4Gamma/dphi^4
\begin{align}
\Gamma^{(4)}\bigg|_{\text{fermions}} ~=~ 
%
 \begin{tikzpicture}[scale=2,baseline={([yshift=-.4ex]current bounding box.center)}]
\node (v1) at (0,0.3) [Gamma] {}; 
\node (v2) at (0,-0.3) [Sigma] {}; 
\node (v3) at (0.6,0.3) [Sigma] {}; 
\node (v4) at (0.0,0.9) [Sigma] {}; 
\node (l1) at (-0.4,0.3) [VEV] {}; 
\node (l2) at (0.,1.3) [VEV] {}; 
\node (l3) at (0.9,0.3) [VEV] {}; 
\node (l4) at (0.,-0.7) [VEV] {}; 
\draw[-,cyan] (v1) -- (v2);
\draw[-,cyan] (v4) -- (v3);
\draw[-,cyan] (v2) -- (v3);
\draw[-,cyan] (v1) -- (v4);
\draw[-] (v1) -- (l1);
\draw[-] (v4) -- (l2);
\draw[-] (v3) -- (l3);
\draw[-] (v2) -- (l4);
\end{tikzpicture}
%
+
%
 \begin{tikzpicture}[scale=2,baseline={([yshift=-.4ex]current bounding box.center)}]
\node (v1) at (0,0.3) [Gamma] {}; 
\node (v3) at (0.6,0.3) [Sigma] {}; 
\node (v4) at (0.0,0.9) [Sigma] {}; 
\node (l1) at (-0.4,0.3) [VEV] {}; 
\node (l2) at (0.,1.3) [VEV] {}; 
\node (l3) at (0.9,0.3) [VEV] {}; 
\node (l4) at (0.6,-0.1) [VEV] {}; 
\draw[-,cyan] (v1) -- (v3);
\draw[-,cyan] (v4) -- (v3);
\draw[-,cyan] (v1) -- (v4);
\draw[-] (v1) -- (l1);
\draw[-] (v4) -- (l2);
\draw[-] (v3) -- (l3);
\draw[-] (v3) -- (l4);
\end{tikzpicture}
%
+
%
 \begin{tikzpicture}[scale=2,baseline={([yshift=-.4ex]current bounding box.center)}]
\node (v1) at (0,0.3) [Gamma] {}; 
\node (v2) at (0.6,0.3) [Sigma] {}; 
\node (l1) at (-0.4,0.3) [VEV] {}; 
\node (l2) at (0.6,0.7) [VEV] {}; 
\node (l3) at (0.9,0.3) [VEV] {}; 
\node (l4) at (0.6,-0.1) [VEV] {}; 
\draw[double distance=4mm,thick,cyan] (v1) -- (v2);
\draw[-] (v1) -- (l1);
\draw[-] (v2) -- (l2);
\draw[-] (v2) -- (l3);
\draw[-] (v2) -- (l4);
\end{tikzpicture} \,\,\,.
\label{eq:Gamma4_fermions}
\end{align}
A factor of $\frac{1}{2}$ does not appear for the fermionic self-energy insertions as this quantity is not defined with a factor of 2 (compare \eqref{eq:def_Pi} and \eqref{eq:def_Sigma}). Now, we list the non-vanishing fermionic contributions to the derivatives of the scalar self-energy which would be inserted in to the above equations
\begin{align}
 \begin{tikzpicture}[scale=2,baseline={([yshift=-.4ex]current bounding box.center)}]
\node (v1) at (0.0,0.) [Pi] {}; 
\node (l1) at (-0.4,0) {}; 
\node (l2) at (0.4,0.) [VEV] {}; 
\node (l3) at (0,-0.4) [VEV] {}; 
\draw[double distance=4mm,thick] (l1) -- (v1);
\draw[-] (v1) -- (l2);
\draw[-] (v1) -- (l3);
\end{tikzpicture}
\Bigg|_{\text{fermions}} ~=~ &  
%
+  \,
%
 \begin{tikzpicture}[scale=2,baseline={([yshift=-.4ex]current bounding box.center)}]
\node (v1) at (0.0,0.) [Gamma] {}; 
\node (v3) at (0.0,-0.6) [Sigma] {}; 
\node (v4) at (0.6,0.) [Sigma] {}; 
\node (l1) at (-0.4,0) {}; 
\node (l2) at (1.,0.) [VEV] {}; 
\node (l3) at (0,-1.0) [VEV] {}; 
\draw[double distance=4mm,thick] (l1) -- (v1);
\draw[-,cyan] (v1) -- (v3);
\draw[-,cyan] (v1) -- (v4);
\draw[-,cyan] (v3) -- (v4);
\draw[-] (v4) -- (l2);
\draw[-] (v3) -- (l3);
\end{tikzpicture}
%
+ \,
%
 \begin{tikzpicture}[scale=2,baseline={([yshift=-.4ex]current bounding box.center)}]
\node (v1) at (0.0,0.) [Gamma] {}; 
\node (v2) at (-0.6,0.) [VR] {\,$V^{(4)}_{\phi\psi}$\,}; 
\node (v3) at (0.0,-0.6) [Sigma] {}; 
\node (v4) at (0.6,0.) [Sigma] {}; 
\node (l1) at (-1.,0) {}; 
\node (l2) at (1.,0.) [VEV] {}; 
\node (l3) at (0,-1.0) [VEV] {}; 
\draw[double distance=4mm,thick] (l1) -- (v2);
\draw[double distance=4mm,thick,cyan] (v1) -- (v2);
\draw[-,cyan] (v1) -- (v3);
\draw[-,cyan] (v1) -- (v4);
\draw[-,cyan] (v3) -- (v4);
\draw[-] (v4) -- (l2);
\draw[-] (v3) -- (l3);
\end{tikzpicture}
\label{eq:d2Pi_dphi2_fermions}
\end{align}

\begin{align}
 \begin{tikzpicture}[scale=2,baseline={([yshift=-.4ex]current bounding box.center)}]
\node (v1) at (0.0,0.) [Pi] {}; 
\node (l1) at (-0.4,0) {}; 
\node (l2) at (0.4,0.) [VEV] {}; 
\node (l3) at (0,-0.4) [VEV] {}; 
\node (l4) at (0,0.4) [VEV] {}; 
\draw[double distance=4mm,thick] (l1) -- (v1);
\draw[-] (v1) -- (l2);
\draw[-] (v1) -- (l3);
\draw[-] (v1) -- (l4);
\end{tikzpicture}\Bigg|_{\text{fermions}}
%
~=~ & 
%
 \begin{tikzpicture}[scale=2,baseline={([yshift=-.4ex]current bounding box.center)}]
\node (v1) at (0.0,0.) [Gamma] {}; 
\node (v3) at (0.0,-0.6) [Sigma] {}; 
\node (v4) at (0.6,0.) [Sigma] {}; 
\node (v5) at (0.,0.6) [Sigma] {}; 
\node (l1) at (-0.4,0) {}; 
\node (l2) at (1.,0.) [VEV] {}; 
\node (l3) at (0,-1.0) [VEV] {}; 
\node (l4) at (0,1.) [VEV] {}; 
\draw[double distance=4mm,thick] (l1) -- (v1);
\draw[-,cyan] (v1) -- (v5);
\draw[-,cyan] (v1) -- (v3);
\draw[-,cyan] (v5) -- (v4);
\draw[-,cyan] (v3) -- (v4);
\draw[-] (v4) -- (l2);
\draw[-] (v3) -- (l3);
\draw[-] (v5) -- (l4);
\end{tikzpicture}
%
+
%
 \begin{tikzpicture}[scale=2,baseline={([yshift=-.4ex]current bounding box.center)}]
\node (v1) at (0.0,0.) [Gamma] {}; 
\node (v2) at (-0.6,0.) [VR] {\,$V^{(4)}_{\phi\psi}$\,}; 
\node (v3) at (0.0,-0.6) [Sigma] {}; 
\node (v4) at (0.6,0.) [Sigma] {}; 
\node (v5) at (0.,0.6) [Sigma] {}; 
\node (l1) at (-1.,0) {}; 
\node (l2) at (1.,0.) [VEV] {}; 
\node (l3) at (0,-1.0) [VEV] {}; 
\node (l4) at (0,1.) [VEV] {}; 
\draw[double distance=4mm,thick] (l1) -- (v2);
\draw[double distance=4mm,thick,cyan] (v1) -- (v2);
\draw[-,cyan] (v1) -- (v5);
\draw[-,cyan] (v1) -- (v3);
\draw[-,cyan] (v5) -- (v4);
\draw[-,cyan] (v3) -- (v4);
\draw[-] (v4) -- (l2);
\draw[-] (v3) -- (l3);
\draw[-] (v5) -- (l4);
\end{tikzpicture}
% 
\nonumber \\ & 
%
+  \,
%
 \begin{tikzpicture}[scale=2,baseline={([yshift=-.4ex]current bounding box.center)}]
\node (v1) at (0.0,0.) [Gamma] {}; 
\node (v3) at (0.0,-0.6) [Sigma] {}; 
\node (v4) at (0.6,0.) [Sigma] {}; 
\node (l1) at (-0.4,0) {}; 
\node (l2) at (1.,0.) [VEV] {}; 
\node (l3) at (0,-1.0) [VEV] {}; 
\node (l4) at (0.6,0.4) [VEV] {}; 
\draw[double distance=4mm,thick] (l1) -- (v1);
\draw[-,cyan] (v1) -- (v3);
\draw[-,cyan] (v1) -- (v4);
\draw[-,cyan] (v3) -- (v4);
\draw[-] (v4) -- (l2);
\draw[-] (v3) -- (l3);
\draw[-] (v4) -- (l4);
\end{tikzpicture}
%
+
%
 \begin{tikzpicture}[scale=2,baseline={([yshift=-.4ex]current bounding box.center)}]
\node (v1) at (0.0,0.) [Gamma] {}; 
\node (v2) at (-0.6,0.) [VR] {\,$V^{(4)}_{\phi\psi}$\,}; 
\node (v3) at (0.0,-0.6) [Sigma] {}; 
\node (v4) at (0.6,0.) [Sigma] {}; 
\node (l1) at (-1.,0) {}; 
\node (l2) at (1.,0.) [VEV] {}; 
\node (l3) at (0,-1.0) [VEV] {}; 
\node (l4) at (0.6,0.4) [VEV] {}; 
\draw[double distance=4mm,thick] (l1) -- (v2);
\draw[double distance=4mm,thick,cyan] (v1) -- (v2);
\draw[-,cyan] (v1) -- (v3);
\draw[-,cyan] (v1) -- (v4);
\draw [-,cyan](v3) -- (v4);
\draw[-] (v4) -- (l2);
\draw[-] (v3) -- (l3);
\draw[-] (v4) -- (l4);
\end{tikzpicture} \,\,,
\label{eq:d3Pi_dphi3_fermions}
\end{align}
 
and finally, those of the fermionic self-energy
 \begin{align}
 \begin{tikzpicture}[scale=2,baseline={([yshift=-.4ex]current bounding box.center)}]
\node (v1) at (0.0,0.) [Sigma] {}; 
\node (l1) at (-0.4,0) {}; 
\node (l2) at (0.4,0.) [VEV] {}; 
\node (l3) at (0,-0.4) [VEV] {}; 
\draw[double distance=4mm,thick,cyan] (l1) -- (v1);
\draw[-] (v1) -- (l2);
\draw[-] (v1) -- (l3);
\end{tikzpicture}
%
~=~ & \,\,
%
 \begin{tikzpicture}[scale=2,baseline={([yshift=-.4ex]current bounding box.center)}]
\node (v1) at (0.0,0.) [Gamma] {}; 
\node (v3) at (0.0,-0.6) [Sigma] {}; 
\node (v4) at (0.6,0.) [Pi] {}; 
\node (l1) at (-0.4,0) {}; 
\node (l2) at (1.,0.) [VEV] {}; 
\node (l3) at (0,-1.0) [VEV] {}; 
\draw[double distance=4mm,thick,cyan] (l1) -- (v1);
\draw[double distance=4mm,thick,cyan] (v1) -- (v3);
\draw[double distance=4mm,thick] (v1) -- (v4);
\draw[-] (v4) -- (l2);
\draw[-] (v3) -- (l3);
\end{tikzpicture}
%
+ \,
%
 \begin{tikzpicture}[scale=2,baseline={([yshift=-.4ex]current bounding box.center)}]
\node (v1) at (0.0,0.) [Gamma] {}; 
\node (v2) at (-0.6,0.) [VR] {\,$V_{\psi\psi}$\,}; 
\node (v3) at (0.0,-0.6) [Sigma] {}; 
\node (v4) at (0.6,0.) [Pi] {}; 
\node (l1) at (-1.,0) {}; 
\node (l2) at (1.,0.) [VEV] {}; 
\node (l3) at (0,-1.0) [VEV] {}; 
\draw[double distance=4mm,thick,cyan] (l1) -- (v2);
\draw[double distance=4mm,thick,cyan] (v1) -- (v2);
\draw[double distance=4mm,thick,cyan] (v1) -- (v3);
\draw[double distance=4mm,thick] (v1) -- (v4);
\draw[-] (v4) -- (l2);
\draw[-] (v3) -- (l3);
\end{tikzpicture}
% 
+ \,
%
 \begin{tikzpicture}[scale=2,baseline={([yshift=-.4ex]current bounding box.center)}]
\node (v1) at (0.0,0.) [Gamma] {}; 
\node (v2) at (-0.6,0.) [VR] {\,$V^{(4)}_{\psi\phi}$\,}; 
\node (v3) at (0.0,-0.6) [Sigma] {}; 
\node (v4) at (0.6,0.) [Sigma] {}; 
\node (l1) at (-1.,0) {}; 
\node (l2) at (1.,0.) [VEV] {}; 
\node (l3) at (0,-1.0) [VEV] {}; 
\draw[double distance=4mm,thick,cyan] (l1) -- (v2);
\draw[double distance=4mm,thick] (v1) -- (v2);
\draw[double distance=4mm,thick,cyan] (v1) -- (v3);
\draw[double distance=4mm,thick,cyan] (v1) -- (v4);
\draw[-] (v4) -- (l2);
\draw[-] (v3) -- (l3);
\end{tikzpicture}
\label{eq:d2sig_dphi2}
\end{align}


\begin{align}
 \begin{tikzpicture}[scale=2,baseline={([yshift=-.4ex]current bounding box.center)}]
\node (v1) at (0.0,0.) [Sigma] {}; 
\node (l1) at (-0.4,0) {}; 
\node (l2) at (0.4,0.) [VEV] {}; 
\node (l3) at (0,-0.4) [VEV] {}; 
\node (l4) at (0,0.4) [VEV] {}; 
\draw[double distance=4mm,thick,cyan] (l1) -- (v1);
\draw[-] (v1) -- (l2);
\draw[-] (v1) -- (l3);
\draw[-] (v1) -- (l4);
\end{tikzpicture}
%
= & \,\,
 \begin{tikzpicture}[scale=2,baseline={([yshift=-.4ex]current bounding box.center)}]
\node (v1) at (0.0,0.) [Gamma] {}; 
\node (v3) at (0.0,-0.6) [Sigma] {}; 
\node (v5) at (0.0,0.6) [Pi] {}; 
\node (l1) at (-0.4,0) {}; 
\node (l2) at (0.4,-0.6) [VEV] {}; 
\node (l3) at (0,-1.0) [VEV] {}; 
\node (l4) at (0,1.) [VEV] {}; 
\draw[double distance=4mm,thick,cyan] (l1) -- (v1);
\draw[double distance=4mm,thick,cyan] (v1) -- (v3);
\draw[double distance=4mm,thick] (v1) -- (v5);
\draw[-] (v3) -- (l2);
\draw[-] (v3) -- (l3);
\draw[-] (v5) -- (l4);
\end{tikzpicture}
%
+ \,
%
 \begin{tikzpicture}[scale=2,baseline={([yshift=-.4ex]current bounding box.center)}]
\node (v1) at (0.0,0.) [Gamma] {}; 
\node (v2) at (-0.6,0.) [VR] {\,$V_{\psi\psi}$\,}; 
\node (v3) at (0.0,-0.6) [Sigma] {}; 
\node (v5) at (0.0,0.6) [Pi] {}; 
\node (l1) at (-1.,0) {}; 
\node (l2) at (0.4,-0.6) [VEV] {}; 
\node (l3) at (0,-1.0) [VEV] {}; 
\node (l4) at (0,1.) [VEV] {}; 
\draw[double distance=4mm,thick,cyan] (l1) -- (v2);
\draw[double distance=4mm,thick,cyan] (v1) -- (v2);
\draw[double distance=4mm,thick,cyan] (v1) -- (v3);
\draw[double distance=4mm,thick] (v1) -- (v5);
\draw[-] (v3) -- (l2);
\draw[-] (v3) -- (l3);
\draw[-] (v5) -- (l4);
\end{tikzpicture}
+ \,
%
 \begin{tikzpicture}[scale=2,baseline={([yshift=-.4ex]current bounding box.center)}]
\node (v1) at (0.0,0.) [Gamma] {}; 
\node (v2) at (-0.6,0.) [VR] {\,$V^{(4)}_{\psi\phi}$\,}; 
\node (v3) at (0.0,-0.6) [Sigma] {}; 
\node (v5) at (0.0,0.6) [Sigma] {}; 
\node (l1) at (-1.,0) {}; 
\node (l2) at (0.4,-0.6) [VEV] {}; 
\node (l3) at (0,-1.0) [VEV] {}; 
\node (l4) at (0,1.) [VEV] {}; 
\draw[double distance=4mm,thick,cyan] (l1) -- (v2);
\draw[double distance=4mm,thick] (v1) -- (v2);
\draw[double distance=4mm,thick,cyan] (v1) -- (v3);
\draw[double distance=4mm,thick,cyan] (v1) -- (v5);
\draw[-] (v3) -- (l2);
\draw[-] (v3) -- (l3);
\draw[-] (v5) -- (l4);
\end{tikzpicture}\,.
\label{eq:d3sig_dphi3}
\end{align}

\noindent With these, one can analyse the scalar $n$-point functions with fermionic contributions and obtain the field counterterms. Moreover, we note that this diagrammatic analysis extends to higher truncations of the 2PI effective action involving fermions. 


%--------------------------------------------------------------
\section{Conclusions and Outlook}
\label{sec:outlook}

We have investigated an on-shell scheme for the
2PI formalism with a particular focus on the
equations of motion, the motivation being that
one can obtain from these in principle the
transport equations which are relevant, for example, while studying
cosmological phase transitions. After an outline of 
the generic procedure, we have
revisited in a first step the so-called Hartree approximation
as one can obtain in this case analytic formulas for
all interesting quantities. 
We have given the
relation between the counterterms in the broken and
unbroken phases. Moreover, we have given 
the formulas for three- and four-point functions
in the broken phase. A particular feature of this approximation 
is, that all counterterms are finite and the resumed two-point function has the same form as the tree-level one. Neither of these hold when going beyond
this approximation.

In a second step, we have first allowed for an
additional independent 
trilinear scalar coupling in the classical 
action which induces the so-called scalar sunset 
diagram at two-loop level. We have used this toy model 
to give the explicit procedure on how to obtain the 
on-shell counterterms in a more complicated system. Moreover, 
we have shown that the two-point function $G$ can
be evaluated numerically in a fast converging iteration. This in turn serves as input for the
calculation of the renormalized three- and four-point functions. 

For a Yukawa theory, the equations of motion
for the scalar and fermionic two-point functions
are coupled. We have demonstrated that the numerical
procedure used in the pure scalar case works also
for this coupled system even for $\mathcal{O}(1)$ couplings.
We have given explicitly the counterterms for
the wave function and mass renormalization, whereas
in the case of the coupling counterterms, we give a 
diagrammatic form which can easily be translated into 
formulae, which are however very lengthy. 

The procedure
outlined in this paper can easily be extended to the
case of multiple scalars and fermions. It
can also be extended to include gauge fields
which we will discuss in a forthcoming paper.
From our results, one can easily get the counterterms
for other renormalization schemes such as
$\overline{\text{MS}}$. This scheme is widely
used for the calculation of the effective potential,
which is an important tool for the study of phase transitions. 

%----------------------------------------------------------------------------------
\section*{Acknowledgements}
We thank K.~Kainulainen and O.~Koskivaara for discussions and
Ch.~Gross for collaboration in the early stage of 
this project. This work has been supported by the 
DFG, project nrs.\ PO-1337/8-1 and HI 744/10-1.

%--------------------------------------------------------------

%----------------------------------------------------------------------------------
\bibliographystyle{h-physrev5}
\bibliography{lit}

\end{document}
 