%\documentclass[aps,prb,groupedaddress,twocolumn,superscriptaddress]{revtex4}
\documentclass[aps,prl,twocolumn,titlepage,nofootinbib]{revtex4}
%\documentclass[aps,prb,reprint,preprintnumbers]{revtex4-1}
\usepackage{graphicx}
\usepackage{bm}
\usepackage{epsfig}
\usepackage{ulem}
\usepackage{color}
\usepackage{mathrsfs}
\usepackage{dcolumn}
\usepackage{setspace}
\usepackage{array}
\usepackage{amsmath}
\usepackage{amssymb}
\usepackage{dsfont}
\usepackage{gensymb}
\usepackage{dcolumn}
\usepackage{multirow}
\usepackage{bibentry,natbib}
\usepackage{booktabs}
\newcommand{\QD}[1]{{\color{red}{\textbf{[QD: #1]}}}}
\begin{document}
%\bibliographystyle{unsrt}
\title{Angular Momentum-Dependent Spectral Shift in Chiral Vacuum Cavities}
\author{Qing-Dong Jiang}
\email{qingdong.jiang@sjtu.edu.cn}
\affiliation{{}\\ Tsung-Dao Lee Institute \& School of Physics and Astronomy, Shanghai Jiao Tong University, Shanghai 200240, China}
\begin{abstract}
Based on a previously proposed unitary transformation for cavity quantum electrodynamics, we investigate the spectral shift of an atom induced by quantum fluctuations in a chiral vacuum cavity. Remarkably, we find an intriguing angular momentum-dependent shift in the spectra of bound states. Our approach surpasses conventional perturbative calculations and remains valid even in the strong-coupling limit. In addition, we establish a cavity-interaction picture for calculating the chiral vacuum Rabi oscillation in the strong-coupling limit for a generic central potential, without using the rotating wave approximation. The anomalous spectral shift revealed in this study possesses both fundamental and practical significance and could be readily observed in experiments.
\end{abstract}
%\preprint{MIT-CTP/xx}
\maketitle

%%%Introduction%%%

{\it \color{blue}{Introduction.}}---Vacuum is not void; instead, it is full of quantum fluctuations with virtal particles constantly being created and annihilated. The vacuum quantum fluctuations give rise to a plethora of well-known phenomena, including the Casimir effect \cite{casimir1948attraction,bordag2001new,milton2004casimir,plunien1986casimir}, the Lamb shift \cite{lamb1947fine,bethe1947electromagnetic,maclay2020history}, anomalous magnetic moment \cite{PhysRev.82.664,weinberg1995quantum}, vacuum Rabi oscillations \cite{jaynes1963comparison,vacuumrabi}, and spontaneous photon emission \cite{dalibard1982vacuumradiation}. 
In addition to these fundamental effects, physicists have directly probed the electromagnetic fluctuations within a vacuum cavity (often referred to as a ``dark cavity") \cite{riek2015direct,benea2019electric}. The cavity offers a notable advantage as it allows for significant amplification of the quantum fluctuations by squeezing the cavity volume \cite{PhysRevA.43.398,ikuta2021cavity,garziano2015multiphoton,shapiro2015dynamical}.
This amplification can be described by the  Hamiltonian of quantum electrodynamics for a single mode within a cavity of volume $V$:
$$
\hat H_{\text{cavity}}=\frac{\epsilon_0 V}{2}(E^2+c^2 B^2)=\frac{\epsilon_0 V}{2}({\dot A}^2+c^2 |\bold k\times\bold A|^2)
$$
This Hamiltonian can be directly mapped onto the Hamiltonian of a harmonic oscillator 
$$\hat H_{\text{HO}}=\frac{m}{2} (\dot x^2+\Omega^2x^2)
$$
by the substitution $A\mapsto x $, $\epsilon_0 V\mapsto m$, and $ck\equiv \omega_c\mapsto \Omega$.
The ground state of a harmonic oscillator exhibits quantum fluctuations, quantified by the root-mean-square standard deviation of position: $\Delta x = \left(\frac{\hbar}{2\Omega m}\right)^{\frac{1}{2}}$. This shows that the vacuum fluctuations of the vector potential are given by $\Delta A=(\hbar/2\omega_c\epsilon_0 V)^{\frac{1}{2}}$. Clearly, reducing the cavity volume can amplify quantum fluctuations significantly.

In recent years, researchers have achieved remarkable success in creating extremely small cavities, approaching the nanoscale \cite{nanocavisong,epstein2020far,bylinkin2021real}. These advancements have paved the way for exploring the realm of strong light-matter coupling across various setups. 
Comparing to the Floquet method (i.e., engineering material properties with electromagnetic radiations), using cavity quantum fluctuations for material property engineering has obvious advantages \cite{schlawin2022cavity,hubener2021engineering,bloch2022strongly,PhysRevApplied.18.044011,schafer2018ab,bacciconi2023first,mercurio2023photon}:
i) Within the cavity, the interactions between light and matter surpass the limitations imposed by classical light-matter interaction bounded by the fine structure constant. As a result, the properties of materials can be deeply tailored in cavities. 
ii) Engineering materials and molecules within vacuum cavities is superior to the Floquet method where external electromagnetic radiation heat up the system and destroy quantum effects.
iii) Cavity quantum fluctuations allow for the engineering of material properties in an equilibrium manner, in sharp contrast to the Floquet method that drives the system out of equilibrium, resulting in transient and complex physical properties. 
Over the past several years, researchers have presented pioneering proposals, with some already realized, to utilize cavity quantum fluctuations for engineering material conductivity \cite{rokaj2022free,moddel2021casimir}, inducing anomalous superconductivity \cite{sentef2018cavity,schlawin2019cavity,curtis2019cavity,thomas2019exploring}, changing band structure \cite{PhysRevB.104.155307,appugliese2022breakdown,rokaj2023topological}, and even modulating chemical reactivity \cite{PhysRevLett.116.238301,flick2017atoms,galego2017many,galego2019cavity,schafer2019modification,altman2021quantum,PhysRevB.103.165412}. These advancements exemplify the profound potential of cavity quantum fluctuations in tailoring material properties.


Nevertheless, despite its advantages, using cavity quantum fluctuations to control quantum states of matter faces a major obstacle. Unlike real electric or magnetic fields, most quantum fluctuations inherently maintain parity symmetry (PS) and time-reversal symmetry (TRS). As a result, their ability to manipulate material and molecular properties is constrained. To induce substantial changes in material properties, it becomes essential to encode symmetry breaking into quantum fluctuations.
In recent years, multiple works have shown the impact of discrete symmetry breaking on phenomena induced by quantum fluctuations. Notable examples include symmetry breaking induced anomalous Casimir forces \cite{butcher2012casimir,jiang2019axial}, chiraity selection in chemical reactions \cite{ke2022can,riso2022strong}, and topological phase transition \cite{espinosa2014semiconductor,PhysRevB.99.235156,jiang2023engineering}. 
A recent work by Wilczek and the author \cite{jiang2019quantum} highlighted the combined power of symmetry breaking and quantum fluctuations. It shows that symmetry breaking can be transmitted from materials to their vicinity by vacuum quantum fluctuations. The vacuum in proximity to a symmetry-broken material was referred to as its {\it Quantum Atmosphere}. 

In this Letter, we present what, to the best of our knowledge, are the first fully quantum mechanical predictions of angular momentum(AM)-dependent spectral shifts, induced by the quantum fluctuations in a chiral cavity. Chiral cavities can be feasibly realized using magneto-optical materials, which has been extensively studied recently  \cite{hubener2021engineering,voronin2022single,baranov2023toward}. Because chiral cavities break time-reversal symmetry, what we are calculating is actually the spectral shift induced by the TRS-broken quantum atmosphere in a chiral cavity. 
It is worth noting that while our primary focus centers on the AM-dependent spectral shift, our analytical calculation of the cavity-Lamb (CL) shift in the strong coupling limit should be equally valuable. In the last part, we establish framework---the cavity-interaction picture---to calculate time-dependent phenomena. This enables us to compute chiral vacuum Rabi oscillation in the strong coupling limit for a generic central potential, without relying on the rotating wave approximation.

{\it \color{blue}{Chiral unitary transformation and cavity-induced potential shift.}}---
To set the stage, we examine the generic Hamiltonian 
\begin{equation}\label{hamilton1}
\hat H= \frac{1}{2m}\left(\hat{\bold p}-q\hat{\bold A}\right)^2+V(\bold r)+\hbar \omega_c\hat a^\dagger \hat a,
\end{equation}
which captures the interaction between a charged particle (with mass $m$, charge $q$) and a photonic mode in a cavity. 
We assume an external single-particle potential, $V(\bold r)$, and a single photonic mode with frequency $\omega_c$, where $\hat a$ and $\hat a^\dagger$ are the annihilation and creation operators of photons, respectively. Generalization to multimode cases is straightforward.
The vector potential $\hat{\bold A}$ can be expressed as $\hat{\bold A}=A_0\left(\boldsymbol \varepsilon^* \hat a^\dagger+\boldsymbol\varepsilon \hat a\right)$, where $\boldsymbol \varepsilon$ represents the polarization of the cavity photonic modes. The mode amplitude is  $A_0=\sqrt{\frac{\hbar}{2\epsilon_0 V\omega_c}}$, with $V$ the cavity volume. 
In the context of cavity quantum electrodynamics, a dimensionless parameter $g=\sqrt{\frac{(q A_0)^2}{m\hbar \omega_c}}$ is commonly used to quantify the strength of the light-matter coupling. The regime $10^{-1} \leqslant g \leqslant 1$ is referred to as strong coupling. For $g \geqslant 1$, it is referred to as deep strong coupling, indicating even stronger interaction between light and matter\footnote{The terminology differs slightly from the standard context of quantum optics, where the term ``strong coupling" typically refers to reversible interactions between photons in the cavity mode and the atom. Here, ``strong coupling" and ``weak coupling" signify the relative strength of the light-matter coupling in comparison to the cavity mode energy.}.
In this letter, we focus on the chiral cavity case, where the photonic polarization is $\boldsymbol\varepsilon=\frac{1}{\sqrt{2}}\left(\mathbf{e_x}+i\mathbf{e_y}\right)$, and $\mathbf{e_{x(y)}}$ represents the unit vector in the x(y)-direction. 
A recent seminal advancement in cavity quantum electrodynamics is the ability to decouple matter and light degrees of freedom, either in the weak or strong coupling limits, through a special unitary transformation \cite{ashida2021cavity,PhysRevB.107.195104}. This transformation is elegantly achieved by applying the unitary operator:
\begin{equation}
\hat U=\exp\left[-i \frac{\xi}{\hbar} \hat{\bold p}\cdot \hat{\boldsymbol\pi}\right],~{\text{with}}~\xi=\frac{g}{1+g^2}\sqrt{\frac{\hbar}{m\omega_c}}
\end{equation}
to the original Hamiltonian $H_C$, where $\hat{\boldsymbol \pi}=i\left(\boldsymbol \varepsilon^* \hat a^\dagger-\boldsymbol \varepsilon \hat a\right)$ is the photonic momentum operator. The parameter $\xi$ is chosen to eliminate the linear light-matter coupling term ($\hat{\bold p}\cdot \hat{\bold A}$). Remarkably, this unitary transformation yields an equivalent yet formally much neater Hamiltonian:
\begin{eqnarray}\label{hamiltonU}
\hat H^\prime(\xi)&=&\hat U^\dagger \hat H \hat U\nonumber\\
&=&\frac{\hat{\bold p}^2}{2m_{\text{eff}}}+V\left(\bold r+\xi \hat{\boldsymbol \pi}+\frac{\xi^2}{2\hbar }\hat{\bold p}\times \bold{e_z}\right)+\hbar \omega_{\text{eff}}\hat a^\dagger\hat a\nonumber\\
\end{eqnarray}
with the renormalized mass $m_{\text{eff}}=m (1+g^2)$ and the effective cavity frequency $\omega_{\text{eff}}=\omega_c(1+g^2)$.
It is important to note that the light-matter coupling is fully encapsulated in the shifted single-particle potential, offering a key advantage of the transformed Hamiltonian (Eq.\eqref{hamiltonU}).
Several remarks are in order to better understand the above transformation: 
\begin{itemize}
\item[i)] The key advantage of Eq.\eqref{hamiltonU} is that the light-matter coupling is fully encoded in the shifted single-particle potential. 

\item[ii)]
The coupling parameter $\xi=\frac{g}{1+g^2}\sqrt{\frac{\hbar}{m\omega_c}}$ approaches zero not only in the weak-coupling limit ($g\rightarrow 0$) but also in the strong-coupling limit ($g\rightarrow \infty$). 

\item[iii)]
Cavity light-matter interactions lead to an increase in both the effective mass of the particle ($m_{\text{eff}}> m$) and the effective mode frequency ($\omega_{\text{eff}} >\omega_c$). 
\end{itemize}
These features allow the application of perturbation theory (in terms of $\xi$) to investigate strong light-matter coupling in cavities. 
In the following sections, we explore several prominent effects induced by chiral cavities, including angular momentum-dependent spectral shift, the cavity Lamb shift, and chiral vacuum Rabi oscillations.

{\it \color{blue}{Cavity QED renormalized spectra.}}---
We now examine the influence of quantum fluctuations in a cavity on the spectral shift of a bound state governed by the Hamiltonian:
\begin{eqnarray}
\hat H=\frac{\left(\boldsymbol{\hat p}-q\hat{\bold A}\right)^2}{2m}+V(r)+\hbar \omega_c \hat a^\dagger \hat a. 
\end{eqnarray}
where $V(r)$ is a central potential and $m$ is the bare mass of the particle.
By applying a cavity unitary transformation to the Hamiltonian, we obtain:
\begin{eqnarray}
\hat H^\prime=\frac{\hat{\bold p}^2}{2m_{\text{eff}}}+\hat V\left(r\right)+\hbar \omega_{\text{eff}}\,\hat a^\dagger\hat a+\Delta {\hat V}.
\end{eqnarray}
In this expression, $\Delta \hat{V}$ is the perturbative potential for small $\xi$, which can be further expanded to second order as:
\begin{eqnarray}
\Delta{\hat V}&=&\hat V\left(\bold r+\boldsymbol{\hat \tau_c}\right)-\hat V\left(r\right)\nonumber\\
&\approx& \boldsymbol{\hat \tau_c}\cdot \boldsymbol\nabla V(r)
+\frac{1}{2}\left(\boldsymbol{\hat \tau_c}\cdot \boldsymbol\nabla\right)^2V(r).
\end{eqnarray}
Here, $\boldsymbol{\hat{\tau_c}} = \xi \hat{\boldsymbol{\pi}} + \frac{\xi^2}{2\hbar} \hat{\bold{p}} \times \bold{e_z}$. This formulation applies to both weak and strong light-matter interactions, as emphasized. In the intermediate light-matter coupling strength, where perturbative expansion is not valid, Eq.(5) may still offer advantages in certain situations \cite{hubener2021engineering}.
With the above preparation, we can employ perturbation theory using the unperturbed states, which are the product states of the n-th bound state $|\psi_n\rangle$ and the cavity vacuum state (zero photon) $|0\rangle_{\text{cav}}$, i.e.,
$|\Psi_{n}\rangle=|\psi_n\rangle\otimes |0\rangle_{\text{cav}}$, where $|\psi_n\rangle$ represents the bound state of a particle with an effective mass $m_{\text{eff}}$ in a central potential $V(r)$. While the effective mass approaches the bare mass $m$ in the weak-coupling limit, it significantly deviates from the bare mass in the strong light-matter coupling regime.
The first-order perturbation calculation yields the energy shift:
\begin{eqnarray}
\Delta E_{n}=\langle \Psi_{n}|\Delta \hat V |\Psi_{n}\rangle=\Delta E_{n}^{\text{AM}}+\Delta E_{n}^{\text{CL}},
\end{eqnarray}
where $\Delta E_n^{\text{AM}}$ and $\Delta E_n^{\text{CL}}$ are the angular momentum (AM)-dependent shift and the cavity-Lamb (CL) shift, respectively. They are given by
\begin{eqnarray}
\label{ameq}\Delta E_n^{\text{AM}}&=&\frac{\xi^2}{2\hbar}\langle \psi_n|\frac{1}{r}\frac{dV(r)}{dr}\hat L_z|\psi_n \rangle\\
\label{cleq}\Delta E_n^{\text{CL}}&=&\frac{\xi^2}{4}\langle \psi_n |\nabla^2 V(r)|\psi_n \rangle.
\end{eqnarray}
The expressions, Eq.\,(8) and Eq.\,(9), are the key findings of this letter and remain applicable in both the weak and strong coupling regimes.
Eq.(8) indicates that the cavity quantum fluctuations indeed encode the breaking of time-reversal symmetry. 
This is because, in the presence of time-reversal symmetry, states with opposite angular momentum, $l_z=\pm 1$, would have the same energy. 
Notably, reversing the cavity's chirality induces a sign change in the AM-dependent spectral shift, showing the essential importance of the cavity's chirality.
In what follows, we will provide examples to illustrate these two key formulas and demonstrate they predict directly measurable effects.

{\it \color{blue}{Anomalous spectral shift in two examples.}}---
We evaluate the AM-dependent shift and the CL shift in two examples: the Hydrogen atom and the two-dimensional harmonic oscillator. Let us first focus on the spectral shift of the Hydrogen atom model to gain physical understanding. 
For the Hydrogen atom, with the potential $V(r)=-k/r$ (where $k\equiv e^2/4\pi \epsilon_0$), we determine the spectral shifts for each energy level. 
By substituting the eigen function of the Hydrogen atom into the formulas, we obtain the spectral shifts of the bound state $|\psi_{n,l,l_z} \rangle$, where $n$, $l$, and $m$ are the principal, azimuthal, and magnetic quantum numbers, respectively:
\begin{eqnarray}
\Delta E_{n,l,l_z}^{\text{AM}}&=&
\frac{l_z\,\xi^2 \,k}{2 a_{\rm eff}^3 n^3 l(l+\frac{1}{2})(l+1)}; \\
\Delta E_{n,l,l_z}^{\text{CL}}&=&
\frac{\pi \xi^2\, k}{n^3 a_{\rm eff}^3}\,\delta_{l,0}\,\delta_{l_z,0}.
\end{eqnarray}
where $a_{\rm eff}=4\pi\epsilon_0 \hbar^2/m_{\rm eff}e^2$ is the effective Bohr radius. In these calculations, we have used the relations $\langle 1/r^3\rangle=1/a_0^3 n^3 l(l+1/2)(l+1)$ and $\nabla^2 V=4\pi k \delta(r)$.
The spectral shifts can be easily estimated. For instance, the AM-dependent spectral shift of the first excited state with angular momentum $l=1$ and $l_z=\pm 1$ is given by $\Delta E_{2,1,\pm 1}= \pm \left(\frac{\xi}{a_{\text{eff}}}\right)^2 \frac{{\text{Ry}}}{24}\frac{m_{\text{eff}}}{m}\approx 0.3\, \text{meV}$, where $g=0.01$, $\omega_c=10^{16} s^{-1}$, and $\rm Ry$ is the Rydberg energy. This estimation assumes a single-mode scenario, but it can be extended to include multiple modes. Furthermore, we can recover the Lamb shift by considering a large cavity (i.e., weak light-matter coupling limit) and integrating over all possible mode frequencies.
It yields
\begin{eqnarray}
\Delta E^{\rm Lamb}&=&\sum_n\frac{\hbar}{m\omega_{c,n}}\frac{g^2}{2} \langle \Psi_n|\nabla^2 V(r) |\Psi_n\rangle\nonumber\\
&=&\frac{1}{8\epsilon_0\pi^2}\int d\omega_c   \frac{\hbar}{\omega_{c}}\frac{q^2}{m^2} \langle \Psi_n|\nabla^2 V(r) |\Psi_n\rangle\nonumber\\
&=&\frac{\hbar q^2}{8\epsilon_0\pi^2 m_{\rm eff}^2}\ln{\frac{1}{\pi \alpha}} \langle \Psi_n|\nabla^2 V(r) |\Psi_n\rangle.
\end{eqnarray}
Here, in line with Hans Bethe's approach \cite{bethe1947electromagnetic}, we have regularized the non-relativistic theory by selecting $\hbar \omega_{\text{min}}=\hbar c \pi/a_0$ (where $a_0$ represents the Bohr radius) as the smallest energy scale and $\hbar \omega_{\text{max}}=mc^2$ as the largest energy scale. We remark that the derivation of the Lamb shift closely resembles Theodore A. Welton's approach in the weak-coupling limit \cite{welton1948some}.

Next, we consider the spectral shift of a two-dimensional (2D) harmonic oscillators governed by the Hamiltonian $\hat H=\frac{p_x^2+p_y^2}{2m}+\frac{m}{2}\omega^2\left(x^2+y^2\right)$. This Hamiltonian exhibits rotational symmetry and commutes with the angular momentum operator along the z-axis, $\hat L_z$. By introducing the annihilation operator 
$\hat a_{R(L)}=\left[\sqrt{\frac{m\omega}{\hbar}} (x\pm i y)+i\frac{p_x\pm p_y}{\sqrt{m\hbar\omega}}\right]/2$,
one can rewrite the Hamiltonian and angular momentum operator in terms of number operators $\hat n_{L(R)}=\hat a_{R(L)}^\dagger \hat a_{R(L)}$:
\begin{eqnarray}
    \hat H_{\rm HO}=\left(\hat n_R +\hat n_L  +1\right)\hbar \omega;~
    \hat L_z=\hbar \left(\hat n_R  -\hat n_L  \right).
\end{eqnarray}
Here $\hat H_{\rm HO}$ and $\hat L_z$ share the common set of eigenstates
\begin{equation}
    |\phi_{n_R,n_L}\rangle=\frac{1}{\sqrt{n_R!n_L!}}(a_R^\dagger)^{n_R}(a_L^\dagger)^{n_L}|\phi_{0,0}\rangle,
\end{equation}
where $n_R$ and $n_L$ are integers that characterize an eigenstate.
According to the Eq.(8), the AM-dependent spectral shift of the state $|\phi_{n_R,n_L}\rangle$ is given by
\begin{eqnarray}
\Delta E^{\text{AM}}=\frac{\xi^2}{2}m \omega^2\left(n_R-n_L\right). 
\end{eqnarray}
For the ground state of the 2D quantum Harmonic oscillator, $\langle \hat L_z\rangle_n=0$, and AM-dependent spectral shift vanishes. However, a spectral gap of size $m\omega^2\xi^2/2 $ emerges for the originally degenerated first excited states (i.e., $|\phi_{1,0}\rangle$ and $|\phi_{0,1}\rangle$) with different angular momentum. The CL shift remains a constant due to $\nabla^2 V(r)=m\omega^2$ in this special case. 
In addition to the two prototypical examples, our approach is applicable to a wide range of real experimental systems \cite{khitrova2006vacuum,toida2013vacuum}. For instance, one could measure the spectral shift of Rydberg atoms, superconducting circuits, quantum dots or excitons in transition-metal dichalcogenides, which can be effectively described by an hydrogen atom model \cite{PhysRevA.56.1443,gramich2014lamb,zhou2015berry,PhysRevLett.115.166802}.

{\it \color{blue}{Cavity interaction picture and polaritonic vacuum oscillation.}}---
Spectral shifts and spontaneous emission are interconnected consequences of quantum fluctuations. In a vacuum cavity, an excited atom can spontaneously emit and reabsorb a cavity photon, a phenomenon known as vacuum Rabi oscillation \cite{fox2006quantum,PhysRevLett.76.1800,meunier2005rabi}.
In this section, we investigate vacuum Rabi oscillation in a chiral cavity, examining both the weak and strong light-matter coupling regimes. To proceed, we introduce the {\it cavity interaction picture}: In the cavity-interaction picture, quantum states and operators are defined as follows:
$|\Psi(t)\rangle_I=e^{{i\hat H_0 t}/{\hbar}}|\Psi(t)\rangle$ and $\Delta \hat V_{\mathrm{I}}(t)=e^{i \hat H_0 t} \Delta \hat V(t) e^{-i \hat H_0 t}$, where $|\Psi(t)\rangle$ and $\Delta \hat V(t)$ represent the quantum state and operator in Schrodinger picture.
In sharp contrast to the transitional interaction picture, the Hamiltonian $H_0$ in cavity-interaction picture includes potential $V(r)$. 
The interaction picture is highly useful for studying time-dependent phenomena. The wave function in the cavity interaction picture evolves according to $|\Psi(t)\rangle_{I}=\hat U_I(t,0)|\Psi(0)\rangle_{I}$, where the unitary evolution operator is given by
\begin{eqnarray}
\hat U_I(t,0)=\mathcal T\left\{\exp \left[-\frac{i}{\hbar}\int_0^t d\tau \Delta \hat V_I(\tau)\right]\right\}
\end{eqnarray}
where $\hat U_I(t,0)$ represents the time evolution operator and $\mathcal T$ stands for time ordering operator.

Based on the cavity interaction picture, let us consider a two-level system consisting of an excited state $|e\rangle$ and a ground state $|g\rangle$ within a vacuum cavity. Specifically, we focus on the lowest two levels of the combined system, which are represented by the product states $|\Psi_1 \rangle =|e\rangle |0\rangle_{\rm cav} $ and $|\Psi_2 \rangle =|g\rangle |1\rangle_{\rm cav} $, where $|0\rangle_{\rm cav}$ and $|1\rangle_{\rm cav}$ correspond to the cavity photon states with zero and one photon, respectively. The scattering matrix is then given by
\begin{eqnarray}
\Delta \hat V_{\mathrm{I}}(t)=\left(\begin{array}{cc}
\gamma_{11} & \gamma_{12}e^{-i \tilde \omega t}  \\
\gamma_{21}e^{i \tilde\omega t} &\gamma_{22}
\end{array}\right)
\end{eqnarray}
where $\gamma_{ij}=\langle \Psi_i| \Delta \hat V|\Psi_j\rangle$ and $\tilde\omega=\omega_2-\omega_1$ represent the spectral gap between the two states.
In the first-order approximation of the scattering matrices, the unitary evolution operator is given by
\begin{eqnarray}
\hat U_{I}(t, 0)=1-\frac{i}{\hbar}  \left(\begin{array}{cc}
\gamma_{11}t & \gamma_{12} \frac{\sin \left(\tilde{\omega} t\right)}{\tilde{\omega}} e^{-i \tilde{\omega} t} \\
\gamma_{21} \frac{\sin \left(\tilde{\omega} t\right)}{\tilde{\omega}}e^{i \tilde{\omega} t} & \gamma_{22}t
\end{array}\right)+\dots \nonumber
\end{eqnarray}
If the system is initially prepared in the first excited state, i.e.,
$|\Psi(0)\rangle_{I}=\left(0,~1\right)^T$,
then the wave function at later times is given by $|\Psi(t)\rangle_{I}=\hat U_I(t, 0) |\Psi(0)\rangle_{I}$.
Therefore, the probability of finding the system in the ground state after time $t$ is
\begin{equation}
    P_{1\rightarrow 2}\left(t, \tilde{\omega}\right) \equiv\left|\left\langle \Psi_2| \Psi_I(t)\right\rangle\right|^2=\gamma_{12}^2 \frac{\sin ^2\left(\tilde{\omega} t\right)}{\hbar^2\tilde{\omega}^2}
\end{equation}
where the scattering matrix is represented by
\begin{eqnarray}\label{scatmatr}
\gamma_{12}^{\rm AM}&=&-i\frac{\xi}{\sqrt{2}} \langle e |\left(\partial_x+i\partial_y\right)V(r)|g\rangle\nonumber\\
&=&-i\frac{\xi}{\sqrt{2}} \langle e |\frac{d V(r)}{d r}  e^{i\theta}|g\rangle. 
\end{eqnarray}
Without relying on the rotating wave approximation, which is commonly used in the Jaynes-Cummings model, we derived a general result that is applicable to both the weak and strong light-matter coupling regimes. Eq.~\eqref{scatmatr} shows that the scattering matrix connects quantum states with magnetic quantum numbers that differ by exactly one $\hbar$.  This finding indicates that a chiral photon is emitted and re-absorbed in a chiral vacuum cavity.
 

{\it \color{blue}{Concluding remarks.}}---
In our analysis, we focused on examining the spectrum of a single atom. Nevertheless, it is worth noting that our approach works for many-body systems, wherein collective enhancement can be anticipated. For example, for a group of electrons subjected to a confining potential $V(r)$, the spectral shift scales with the total angular momentum of all electrons, i.e., $\sum_i\langle \hat L_z^{(i)} \frac{1}{r}\frac{d V}{dr}\rangle$, given that all electrons coherently couple to a single cavity mode.
Additionally, we should address the valid range of our perturbation theory. To apply our theory to cases involving strong light-matter coupling, it is necessary for the shift parameter $\xi$ to be considerably smaller than the typical length scale within the system. For example, when applying our theory to atomic spectra, we require $\xi/a_{\rm eff}\ll 1$, where $a_{\rm eff}$ is the aforementioned effective Bohr radius of an electron with an effective mass, $m_{\rm eff}$. While for an electron in a vacuum, meeting this requirement becomes challenging
for strong cavity light-matter coupling due to $\xi/a_{\rm eff}=g\alpha \sqrt{\frac{mc^2}{\hbar \omega_c}}$, electrons in small-band semiconductors (e.g. InSb) can have a very small mass ($\sim 0.01 m_e$) and should easily satisfy this condition for $g\sim 0.1$. 

In conclusion, we have successfully developed a perturbation theory applicable to both weak and strong light-matter coupling regimes, uncovering an AM-dependent chiral spectral shift in chiral cavities. We determined the AM-dependent spectral shift and CL shift for two specific examples, demonstrating that the effect is robust and detectable in experimental settings. Furthermore, we have established the foundation for cavity time-dependent perturbation theory, enabling us to calculate chiral vacuum Rabi oscillations for arbitrary central potentials in the regime of strong light-matter coupling.


{\textit{Acknowledgement}.---}
We gratefully acknowledge previous collaborations F. Wilczek on this subject. We also appreciate the insightful discussions and helpful comments from Hans Hansson, Jianhui Zhou and Yi-Zhuang You. 
Q.-D. Jiang was sponsored by Pujiang Talent Program 21PJ1405400 and TDLI starting up grant. %
%\bibliographystyle{unsrt}
%\bibliography{refs.bib}

%%%Reference%%%Reference%%%Reference%%e

\bibliographystyle{apsrev4-1}
\bibliography{ref}


\end{document}
