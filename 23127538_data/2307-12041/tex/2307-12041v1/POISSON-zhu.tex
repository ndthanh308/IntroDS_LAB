%\documentclass[journal]{IEEEtran}
\documentclass[twocolumn]{IEEEtran}
%\documentclass{sigh-alternate-05-2015}

%\bibliographystyle{IEEEtran}

%\usepackage{multirow}
%\usepackage{threeparttable}
\usepackage{color}
%\usepackage{nopageno}
\usepackage{amsmath}
\usepackage{amssymb}
\usepackage{graphicx}
\IEEEoverridecommandlockouts{}
\usepackage{subfigure}
\usepackage{algorithm}
\usepackage{array}
\usepackage[noend]{algpseudocode}
\usepackage{caption}
\usepackage{amsthm}
\usepackage{cases}


\newtheorem{Theorem}{\textbf{Theorem}}
\newtheorem{Corollary}{\textbf{Corollary}}
\newtheorem{Proposition}{\textbf{Proposition}}

\newtheorem{Remark}{Remark}
\newtheorem{Lemma}{\textbf{Lemma}}

%\newtheorem{Corollary}{Corollary}
%\newtheorem{Proof}{\textbf{Proof}}
%\newtheorem{Problem}{Problem}
%\newtheorem{Definition}{Definition}
\iffalse
\setlength{\textheight}{10.2in} \setlength{\textwidth}{7.86in}
\setlength{\columnsep}{0.1in} %\setlength{\footheight}{0.0in}
%\setlength{\topmargin}{-0.5in} \setlength{\headheight}{-0.0in}
\setlength{\topmargin}{-0.7in} \setlength{\headheight}{-0.0in}
\setlength{\headsep}{0.0in} \setlength{\oddsidemargin}{-.68in}
\setlength{\parindent}{1pc} \tabcolsep=2.6pt
\fi


\begin{document}

\title{Analytical Solution of Poisson's Equation \\ with Application to VLSI Global Placement}

\author{
\normalsize \mbox{Wenxing Zhu$^{1}$,}~\mbox{Zhipeng Huang$^{1}$,}~\mbox{Jianli Chen$^{1}$,}~~and~\mbox{Yao-Wen Chang$^{2,3}$}\\
       \normalsize $^{1}$Center for Discrete Mathematics and Theoretical Computer Science, Fuzhou University, Fuzhou 350108, China\\
    \normalsize $^{2}$Graduate Institute of Electronics Engineering, National Taiwan University, Taipei 106, Taiwan\\
       \normalsize $^{3}$Department of\ Electrical Engineering, National Taiwan University, Taipei 106, Taiwan\\
       \normalsize wxzhu@fzu.edu.cn; N155420004@fzu.edu.cn; jlchen@fzu.edu.cn;  ywchang@ntu.edu.tw
}


\maketitle

%\baselineskip=10.8pt
%
%\small

%%%%%%%%%%%%%%%%%%%%%%%%%%%%%%%%%%%%%%%%%%%%%%%%%%%%%%%%%%%%%%%%%%
% Abstract

%%%%%%%%%%%%%%%%%%%%%%%%%%%%%%%%%%%%%%%%%%%%%%%%%%%%%%%%%%%%%%%%%%
\begin{abstract}
Poisson's equation has been used in VLSI global placement for  describing the potential field caused by a given charge density distribution.
Unlike previous global placement methods that solve Poisson's equation numerically, in this paper, we provide an analytical solution of the equation to calculate the potential energy of an electrostatic system. The analytical solution is derived based on the separation of variables method and an exact density function to model the block distribution in the placement region, which is an infinite series and converges absolutely. Using the analytical solution, we give a fast computation scheme of Poisson's equation and develop an effective and efficient global placement algorithm called \emph{Pplace}. Experimental results show that our Pplace achieves smaller placement wirelength than ePlace and NTUplace3. With the pervasive applications of Poisson's equation in scientific fields, in particular, our effective, efficient, and robust computation scheme for its analytical solution can provide substantial impacts on these fields.
\end{abstract}

%\keywords{Standard cell; Multi-row height standard cell}.
%We first propose Based on the density function, we directly solve Poisson's equation to obtain an analytical solution


%%%%%%%%%%%%%%%%%%%%%%%%%%%%%%%%%%%%%%%%%%%%%%%%%%%%%%%%%%%%%%%%%%%%%%%%%%
% Introduction
%%%%%%%%%%%%%%%%%%%%%%%%%%%%%%%%%%%%%%%%%%%%%%%%%%%%%%%%%%%%%%%%%%%%%%%%%%
\section{Introduction}
\label{sec:introduction}
%%%%%%%%%%%%%%%%%%%%%%%%%%%%%%%%%%%%%%%%%%%%%%%%%%%%%%%%%%%%%%%%%%%%%%%%%%
% Introduction
%%%%%%%%%%%%%%%%%%%%%%%%%%%%%%%%%%%%%%%%%%%%%%%%%%%%%%%%%%%%%%%%%%%%%%%%%%

In VLSI physical design, the placement problem is to place circuit blocks into a fixed die such that no block overlaps with the others and some cost metric (e.g., wirelength) is optimized. As the technology node enters the deep nanometer era with billion-transistor integration, the performance of a placement engine becomes dominant on the overall quality of an EDA tool~\cite{RefEplace}. As a result, many placers have been developed recently. There are three major types of placement algorithms~\cite{RefEss}: annealing-based methods, partitioning-based methods, and analytical-based techniques. Recent studies show that analytical placers typically achieve the
best placement quality with good scalability.


In analytical placement, one key ingredient is the technique to reduce overlaps among blocks for obtaining an evenly distributed placement. Many overlap reduction methods have been proposed in literature for analytical placement, e.g., partitioning, cell shifting, frequency control, assignment, bell-shaped density control, Helmholtz density control, and Poisson density control. A detailed description of these methods can be found in~\cite{RefEss}. Among them, the Poisson density control was adopted in several leading placers, e.g., FDP~\cite{RefSFDP}, Kraftwerk~\cite{RefKW2}, mFAR~\cite{RefmFAR}, and ePlace~\cite{RefEplace}. %The density is treated as the electrostatic potential.


ePlace~\cite{RefEplace} used the Poisson's equation during global placement and achieved the best wirelength among all academic placers for the ISPD 2005 \cite{Ref2005} and ISPD 2006 \cite{Ref2006} benchmarks.
It first treated each block as a positive charge and the block density as a potential energy constraint.
Then, ePlace used Poisson's equation to model the electric potential and field distribution by a given charge density distribution, where
the Neumann boundary condition and the compatibility condition were set according to placement characteristics.
Poisson's equation is a partial differential equation (PDE).
By solving this PDE using the spectral method with a fast Fourier transform (FFT),
ePlace can rapidly compute an electric potential and field distribution,
thus spreading blocks quickly while minimizing the wirelength.

Poisson's equation is a PDE commonly used in many fields such as electrostatics, computer science,
mechanical engineering, theoretical physics, astronomy, chemistry, and so on.
For example, Poisson's equation on the rectangular domain can model a finite element system with bicubic Hermite basis functions~\cite{RefFE}.
%In particular, Poisson's equation is often used to describe the potential field caused by a given charge or mass density distribution.
%As a result, the electrostatic field or gravity can be calculated with different boundary conditions for different applications.
%
Poisson's equation has also been widely used in electronic design automation. In addition to placement, for example, the authors in~\cite{RefGPU} formulated a traditional linear system as a special two-dimensional Poisson equation for large-scale power grid networks. In nanoscale double-gate CMOS, by solving Poisson's and Schr$\ddot{o}$dinger's equations, the authors in~\cite{RefNDCMOS} obtained a fully physical model for leakage distribution.
By using 3D-Poisson's equation, the work~\cite{Ref3D1} presented an analytical subthreshold model for trigate MOSFETs with thin BOX.
% the work~\cite{Ref3D2} studied a specific case of abrupt heterojunction bipolar transistors, and


Solution methods for Poisson's equation can be divided into two categories: analytical solutions and numerical solutions~\cite{RefPDE}.
An analytical solution of a PDE is exactly the PDE solution, which may not be a closed-form solution \cite{RefAna}. In contrast, a numerical solution is obtained by some numerical methods, e.g., finite element methods, numerical approximation, interpolation methods, and so on. Since a numerical solution only approximates the PDE solution, such a numerical solution inevitably incurs some numerical error, and needs convergence analysis. For a half-space eigenstrain problem, for example, many previous works were proposed to reduce  numerical error as much as possible; in contrast, an  analytical solution for a PDE was presented finally in~\cite{Refa1}. Generally, if an  analytical solution can be found for a PDE, then it is intrinsically superior to a numerical one on no numerical error, easy for fast computation and  application.
%{\color{red}(delete this paragraph,our method is not analytical solution)}

Existing placers based on the Poisson density control used numerical methods to solve the Poisson's equation. In Kraftwerk, the authors used the geometric multi-grid solver DiMEPACK~\cite{RefDiMEPACK} to solve the Poisson's equation. In ePlace, the authors computed an electric potential and field distribution based on a general fast Fourier transform (FFT) package~\cite{RefFFTL}. However, Poisson's equation with different boundary conditions has different characteristics, and thus its solutions are different. Generally, it is a great challenge to obtain an analytical solution of a PDE, which even could not be possible.

For VLSI placement, ePlace \cite{RefEplace} has shown great advantages in distributing blocks and optimizing solution quality simultaneously. A natural question arises: can we further reduce the numerical error in ePlace and improve its solution quality?  Based on the electrostatic system in ePlace, in this paper, we attempt to derive an analytical solution of Poisson's equation and provide an effective and efficient mathematical model for VLSI placement.  The major contributions of our work are summarized as follows:
\begin{itemize}

%\item  Unlike ePlace that uses a series of uniform bins to control block overlaps, we present an exact density function to model the blocks distribution.
%
%
%
%\item  Based on the exact density function, we directly solve Poisson's equation to obtain an analytical solution, which is an infinite series. We prove the convergence of the infinite series. As a result, there is no numerical error between our solution and the electrostatic system.

\item We directly solve Poisson's equation to obtain an analytical solution, which is an infinite series. We also prove the convergence of this infinite series. %Unlike ePlace that uses a series of uniform bins to control block overlaps, we present an exact density function for VLSI placement in order to make the analytical solution computable.


\item  Based on the analytical solution, we propose a fast computation scheme for Poisson's equation in  $O(n\log n)$ time, where $n$ is the number of circuit blocks. And then, we develop an effective and efficient global placer called \emph{Pplace} (\textbf{P}oisson's equation with \textbf{a}nalytical solution and also an extension of ePlace).


\item To validate the effectiveness, efficiency, and robustness of our Pplace placer,
      we performed comparative studies with two leading placers on wirelength
      optimization, ePlace and NTUplace3~\cite{RefNtu3}, based on the ISPD 2005~\cite{Ref2005} and ISPD 2006~\cite{Ref2006} benchmarks which
      were used to evaluate placers in the wirelength-driven contest.
      Experimental results show that our Pplace placer can achieve smaller placement wirelength than ePlace. By replacing the density control in NTUplace3, we further embed the fast computation of Poisson's equation into NTUplace3, and reduced its total wirelength by 11\%, a significant improvement never achieved before.


\item With the pervasive applications of Poisson's equation in scientific fields, in particular, our effective, efficient, and robust computation scheme for its analytical solution can provide substantial impacts on these fields.
\end{itemize}

It should be noted that, we evaluate our Pplace placer based on wirelength
and runtime because it is easier to see the effects of the core engine of
the examined algorithm using these fundamental placement metrics. Our
techniques are readily applicable to placement with other considerations such as
routability and timing.
%Our techniques readily apply to placement with other considerations such as routability and timing.

The remainder of this paper is organized as follows. We present some preliminaries in Section~\ref{sec:problem_statement}. In Section~\ref{sec:analytical solution and its convergence}, we show our exact density control of the placement problem, derive an analytical solution of  Poisson's equation, and provide a fast computation scheme of the equation.
 Section~\ref{sec:our placement algorithm} describes our placement algorithm.
Experimental results are reported in Section~\ref{sec:experimental results}.
Finally, Section~\ref{sec:conclusion} gives some conclusions.


%%%%%%%%%%%%%%%%%%%%%%%%%%%%%%%%%%%%%%%%%%%%%%%%%%%%%%%%%%%%%%%%%%%%%%%%%%
% section{Preliminaries}
%%%%%%%%%%%%%%%%%%%%%%%%%%%%%%%%%%%%%%%%%%%%%%%%%%%%%%%%%%%%%%%%%%%%%%%%%%

\section{Preliminaries} \label{sec:problem_statement}

In this section, we describe the placement problem and explain the application of Poisson's equation in ePlace.

\subsection{Problem Statement}

% = {\{e_1, e_2, e_3, \ldots, e_m \}}
Given $n$ blocks and $r$ nets, the circuit of the VLSI placement problem can be modelled as a hypergraph $G(V, E)$, where blocks are denoted by vertices $V = {\{v_1, v_2, v_3, \ldots, v_n\}}$, and nets are denoted by hyperedges $E = {\{e_1, e_2, e_3, \ldots, e_r \}}$.  Let $(w_i, h_i)$ be the width and height of  block $v_i$, and $(x_i, y_i)$ be the coordinate of the center of the block, respectively, $i=1, 2, \cdots, n$. Let the placement region be $[0, W]\times [0, H]$. The VLSI placement problem intends to determine the best position for each block  such that no block overlaps with the others, and the total wirelength is minimized \cite{RefEss}:
 \begin{equation}\label{eq:placement}
\min  \ W(v)~~\textmd{s.t.} ~~\textmd{no overlaps among blocks},
\end{equation}
where $W(v)$ is the total wirelength, which is calculated by the half-perimeter wirelength (HPWL)~\cite{RefEss}.


%%%%%%%%%%%%%%%%%%%%%%%%%%%%%%%%%%%%%%%%%%%%%%%%%%%%%%%%%%%%%%%%%%%%%%%%%%
% subsection{Possion's Equation in ePlace}
%%%%%%%%%%%%%%%%%%%%%%%%%%%%%%%%%%%%%%%%%%%%%%%%%%%%%%%%%%%%%%%%%%%%%%%%%%
\subsection{Poisson's Equation in ePlace}\label{review ePlace}
%%%%%%%%%%%%%%%%%%%%%%%%%%%%%%%%%%%%%%%%%%%%%%%%%%%%%%%%%%%%%%%%%%%%%%%%%%
% subsection{Possion's Equation in ePlace}
%%%%%%%%%%%%%%%%%%%%%%%%%%%%%%%%%%%%%%%%%%%%%%%%%%%%%%%%%%%%%%%%%%%%%%%%%%

 ePlace models the placement problem as a two-dimensional electrostatic system.
 According to the positions of all blocks in the placement region, the distributions of electric potential $\psi(x,y)$ and the electric field $\boldsymbol{\xi}(x,y)$ are determined, where $\boldsymbol{\xi}(x, y)=(\xi_x,\xi_y)=-\nabla\psi(x,y)$.  Then, each block $i$ is seen as a positive charge, where the area $A_i$ of block $i$ is regarded as the electric quantity $q_i$ of  charge $i$. Let
$\psi _i=\psi(x_i, y_i)$ and $\boldsymbol{\xi} _i=\boldsymbol{\xi}(x_i, y_i)$ denote the electric potential and electric field at charge $i$, respectively. Then the electric force $\boldsymbol{F}_i=q_i\boldsymbol{\xi} _i$ leads the movement of charge $i$.
 Thus, the system potential energy is defined as $N(v)=\frac {1}{2}\sum_{i\in V}N_i$, where $N_i=q_i\psi _i$ represents the potential energy of charge $i$. Finally, ePlace changes the density control in the constraint of problem~(\ref{eq:placement}) to the system of  total potential energy $N(v)=0$.


Based on Gauss's law, a partial differential equation with Poisson's equation, boundary condition, and the compatibility condition in ePlace \cite{RefEplace} is used:
\begin{subnumcases}{}\label{equ:ePlace-PDE-a}
\bigtriangledown \bullet \bigtriangledown \psi (x,y)=-\rho (x,y), & \\ \label{equ:ePlace-PDE-b}
\hat{n}\bullet \bigtriangledown \psi (x,y)=0,(x,y)\in \partial \mathbf{R},&\\ \label{equ:ePlace-PDE-c}
\iint_\mathbf{R}\rho (x,y)dxdy=\iint_\mathbf{R}\psi (x,y)dxdy=0.&
\end{subnumcases}
Equation (\ref{equ:ePlace-PDE-a}) gives Poisson's equation, where $\rho(x, y)$ is the density function.
Equation (\ref{equ:ePlace-PDE-b}) is the Neumann boundary condition, where
$\partial \mathbf{R}$ and $\hat{n}$ denote the boundary and the outer normal vector of the design range $\mathbf{R}$, respectively.
Equation (\ref{equ:ePlace-PDE-b}) is used for preventing a block from running out of the boundary.
Equation (\ref{equ:ePlace-PDE-c}) is the compatibility condition, which makes the system of equations a unique solution.


By dividing the placement region into $m \times m$ bins, ePlace uses local smoothness to get an $m\times m$ density matrix $\rho$, in which each element is calculated using the total area of blocks covered by the corresponding bin \cite{RefEplace}. Let $u$ and $p$ denote integer indexes, and the frequency components are defined as $\omega_{u}=\frac{2\pi u}{m}$ and $\omega_{p}=\frac{2\pi p}{m}$, respectively. The coefficient of each wave function of a discrete cosine transform is denoted by $a_{u,p}$ as follows:
%  \vspace{-0.1cm}
  \begin{equation}
    a_{u,p}=\frac{1}{m^2}\sum _{l=0}^{m-1}\sum _{j=0}^{m-1}\rho(l,j)\cos(\omega_{u}l)\cos(\omega_{p}j).
    \label{equ:ePlace-a}
%    \vspace{-0.1cm}
\end{equation}
Then, the potential value is calculated by
%\vspace{-0.2cm}
  \begin{equation}
    \psi(l,j)=\sum _{u=0}^{m-1}\sum _{p=0}^{m-1}\frac{a_{u,p}}{\omega_{u}^2+\omega_{p}^2}\cos(\omega_{u}l)\cos(\omega_{p}j),
    \label{equ:ePlace-psi}
%    \vspace{-0.2cm}
\end{equation}
where $\psi(l,j)$ is the electric potential at the center of bin $b_{lj}$. In ePlace \cite{RefEplace},  $\boldsymbol{\xi}(l,j)$ is calculated as follows:
\begin{equation}
\boldsymbol{\xi}(l,j)=
  \left\{
   \begin{aligned}
   &\xi _{x}=\sum _{u=0}^{m-1}\sum _{p=0}^{m-1}\frac{a_{u,p}\omega_{u}}{\omega_{u}^2+\omega_{p}^2}\sin(\omega_{u}l)\cos(\omega_{p}j),  \\
   &\xi _{y}=\sum _{u=0}^{m-1}\sum _{p=0}^{m-1}\frac{a_{u,p}\omega_{p}}{\omega_{u}^2+\omega_{p}^2}\cos(\omega_{u}l)\sin(\omega_{p}j).
   \end{aligned}
   \label{equ:ePlace-ksi}
   \right.
  \end{equation}
Equations  \eqref{equ:ePlace-a}, \eqref{equ:ePlace-psi} and \eqref{equ:ePlace-ksi} are used in the global placement stage of ePlace. After global placement, detail placement and legalization are called to obtain a final placement result.

%%%%%%%%%%%%%%%%%%%%%%%%%%%%%%%%%%%%%%%%%%%%%%%%%%%%%%%%%%%%%%%%%%%%%%%%%%
% Analytical Solution and Its Convergence
%%%%%%%%%%%%%%%%%%%%%%%%%%%%%%%%%%%%%%%%%%%%%%%%%%%%%%%%%%%%%%%%%%%%%%%%%%
\section{Analytical Solution of Poisson's equation} \label{sec:analytical solution and its convergence}
%%%%%%%%%%%%%%%%%%%%%%%%%%%%%%%%%%%%%%%%%%%%%%%%%%%%%%%%%%%%%%%%%%%%%%%%%%
% Analytical Solution and Its Convergence
%%%%%%%%%%%%%%%%%%%%%%%%%%%%%%%%%%%%%%%%%%%%%%%%%%%%%%%%%%%%%%%%%%%%%%%%%%




%In this section, we first describe an exact density function for VLSI placement. Then, based on  the function, we derive an analytical solution of Poisson's equation. Based on the solution, finally, we propose a fast computation scheme of Poisson's equation.
 In this section, we first derive an analytical solution of the Poisson's equation. Then, we prove the convergence of the analytical solution and analyze its time complexity. Based on the analytical solution, finally, we propose a fast computation scheme of Poisson's equation.
%%%%%%%%%%%%%%%%%%%%%%%%%%%%%%%%%%%%%%%%%%%%%%%%%%%%%%%%%%%%%%%%%%%%%%%%%%
% subsection{Exact Density Control}
%%%%%%%%%%%%%%%%%%%%%%%%%%%%%%%%%%%%%%%%%%%%%%%%%%%%%%%%%%%%%%%%%%%%%%%%%%
%\subsection{Exact Density Function}
%%%%%%%%%%%%%%%%%%%%%%%%%%%%%%%%%%%%%%%%%%%%%%%%%%%%%%%%%%%%%%%%%%%%%%%%%%
% subsection{Exact Density Control}
%%%%%%%%%%%%%%%%%%%%%%%%%%%%%%%%%%%%%%%%%%%%%%%%%%%%%%%%%%%%%%%%%%%%%%%%%%


%In this subsection, we introduce an exact density function $\rho (\hat{x},\hat{y})$ for the VLSI placement problem, which is different from ePlace~\cite{RefEplace} in which the density function is approximated by a density matrix.
%
%
%
%Define the density function of block $i$ in the $x$-direction as
%\begin{equation*}
%\theta _i(\hat{x})=
%  \left\{
%   \begin{aligned}
%   &0,~~~~~&&|\hat{x}-x_i|>\frac {w_i}{2},  \\
%   &1,~~~~~&&|\hat{x}-x_i|\leq \frac {w_i}{2}.
%   \end{aligned}
%   \right.
%  \end{equation*}
%Similarly, we can define the density function of block $i$ in the $y$-direction as $\theta _i(\hat{y})$. Then the density function of block $i$  is $\theta_i(\hat{x})\theta_i(\hat{y})$, and the density function of all blocks is
%\begin{equation}
%\rho (\hat{x},\hat{y})=\sum _{i=1}^{n}\theta _i(\hat{x})\theta _i(\hat{y}),
%\label{equ:global density}
%\end{equation}
%where $n$ is the number of blocks.
%
% In order to satisfy Equation (\ref{equ:ePlace-PDE-c}), we redefine $\rho (\hat{x},\hat{y})$ as
%\begin{equation}
%\begin{aligned}
%\rho (\hat{x},\hat{y})&\triangleq\rho (\hat{x},\hat{y})-\frac{1}{WH}\iint_\mathbf{R}\rho (\hat{x},\hat{y})d\hat{x}d\hat{y}\\
%&=\sum _{i=1}^{n}\theta _i(\hat{x})\theta _i(\hat{y})-\frac{1}{WH}\sum _{i=1}^{n}w_{i}h_{i}.
%\end{aligned}
%\label{equ:rho}
%\end{equation}
%
%
%Using the exact density function, Poisson's equation, the relative boundary condition and  the compatibility condition (\ref{equ:ePlace-PDE-a})--(\ref{equ:ePlace-PDE-c}) reduce to
%\begin{subnumcases}{}\label{equ:PDE-a}
%   \bigtriangledown \bullet \bigtriangledown \psi (\hat{x},\hat{y})=-\rho (\hat{x},\hat{y}),&  \\\label{equ:PDE-b}
%   \hat{n}\bullet \bigtriangledown \psi (\hat{x},\hat{y})=0,(\hat{x},\hat{y})\in \partial \mathbf{R},&\\\label{equ:PDE-c}
%   \iint_\mathbf{R}\psi (\hat{x},\hat{y})d\hat{x}d\hat{y}=0.&
%\end{subnumcases}



%%%%%%%%%%%%%%%%%%%%%%%%%%%%%%%%%%%%%%%%%%%%%%%%%%%%%%%%%%%%%%%%%%%%%%%%%%
% subsection{Analytical Solution of Poisson's Equation}
%%%%%%%%%%%%%%%%%%%%%%%%%%%%%%%%%%%%%%%%%%%%%%%%%%%%%%%%%%%%%%%%%%%%%%%%%%
\subsection{Analytical Solution of Poisson's Equation}\label{3b}
%%%%%%%%%%%%%%%%%%%%%%%%%%%%%%%%%%%%%%%%%%%%%%%%%%%%%%%%%%%%%%%%%%%%%%%%%%
% subsection{Analytical Solution of Poisson's Equation}
%%%%%%%%%%%%%%%%%%%%%%%%%%%%%%%%%%%%%%%%%%%%%%%%%%%%%%%%%%%%%%%%%%%%%%%%%%

Similar to ePlace~\cite{RefEplace}, we have the continuous Poisson's equation, the relative boundary condition and the compatibility condition
\begin{subnumcases}{}\label{equ:PDE-a}
   \bigtriangledown \bullet \bigtriangledown \psi (\hat{x},\hat{y})=-\rho (\hat{x},\hat{y}),&  \\\label{equ:PDE-b}
   \hat{n}\bullet \bigtriangledown \psi (\hat{x},\hat{y})=0,(\hat{x},\hat{y})\in \partial \mathbf{R},&\\\label{equ:PDE-c}
   \iint_\mathbf{R}\psi (\hat{x},\hat{y})d\hat{x}d\hat{y}=0.&
\end{subnumcases}
In this subsection, we give an analytical solution of Poisson's equation with the relative boundary condition and the compatibility condition (\ref{equ:PDE-a})--(\ref{equ:PDE-c}).



By separation of variables, the density function in Equation (\ref{equ:PDE-a}) can be expanded as
\begin{equation}
\rho (\hat{x},\hat{y})=\sum _{u=0}^{\infty}\sum _{p=0}^{\infty}b_{u,p}X_u(\hat{x})Y_p(\hat{y}),
\label{td:density}
\end{equation}
and the potential can be calculated by
\begin{equation}
\psi (\hat{x},\hat{y})=\sum _{u=0}^{\infty}\sum _{p=0}^{\infty}a_{u,p}X_u(\hat{x})Y_p(\hat{y}).
\label{td:psi}
\end{equation}

According to Equations (\ref{equ:PDE-a}), (\ref{equ:PDE-b}), (\ref{td:density}) and (\ref{td:psi}), we have
%\begin{subnumcases}{}\label{td:PDE-a}
%   \frac{\partial^{2}X_u(\hat{x})Y_p(\hat{y})}{\partial \hat{x}^{2}}+\frac{\partial^{2}X_u(\hat{x})Y_p(\hat{y})}{\partial \hat{y}^{2}}=-\frac{b_{u,p}}{a_{u,p}}X_u(\hat{x})Y_p(\hat{y})&  \\\label{td:PDE-b}
%   \frac{\partial a_{u,p}X_u(\hat{x})Y_p(\hat{y})}{\partial \hat{x}}|_{\hat{x}=0,W}=\frac{\partial a_{u,p}X_u(\hat{x})Y_p(\hat{y})}{\partial \hat{y}}|_{\hat{y}=0,H}=0&
%\end{subnumcases}
\begin{equation}
  \left\{
   \begin{aligned}
   &\frac{\partial^{2}X_u(\hat{x})Y_p(\hat{y})}{\partial \hat{x}^{2}}+\frac{\partial^{2}X_u(\hat{x})Y_p(\hat{y})}{\partial \hat{y}^{2}}=-\frac{b_{u,p}}{a_{u,p}}X_u(\hat{x})Y_p(\hat{y}),  \\
   &\frac{\partial a_{u,p}X_u(\hat{x})Y_p(\hat{y})}{\partial \hat{x}}|_{\hat{x}=0,W}=0,\\
   &\frac{\partial a_{u,p}X_u(\hat{x})Y_p(\hat{y})}{\partial \hat{y}}|_{\hat{y}=0,H}=0.
   \end{aligned}
   \right.
   \label{td:PDE}
  \end{equation}
If Equations (\ref{td:PDE}) are satisfied, $\psi (\hat{x},\hat{y})$ is the solution of Poisson's equation.
We set $a_{u,p}\neq 0$ and $\alpha_{u,p}=-\frac{b_{u,p}}{a_{u,p}}$, thus
\begin{subnumcases}{}\label{td:1-a}
  X_u^{''}(\hat{x})Y_p(\hat{y})+X_u(\hat{x})Y_p^{''}(\hat{y})=\alpha_{u,p}X_u(\hat{x})Y_p(\hat{y})\hspace{1cm} &  \\\label{td:1-b}
   \frac{\partial X_u(\hat{x})Y_p(\hat{y})}{\partial \hat{x}}|_{\hat{x}=0,W}=\frac{\partial X_u(\hat{x})Y_p(\hat{y})}{\partial \hat{y}}|_{\hat{y}=0,H}=0&
\end{subnumcases}

We set $\frac{X_u^{''}}{X_u}=-(\frac{Y_p^{''}}{Y_p}+\alpha_{u,p})=-\beta_{u,p}^{2}, \beta_{u,p} >0$, thus we have
\begin{subnumcases}{}\label{td:2-a}
   X_u^{''}+\beta_{u,p}^{2}X_u=0,&  \\\label{td:2-b}
   X_u^{'}(0)=X_u^{'}(W)=0,~~0\leq\hat{x}\leq W.&
\end{subnumcases}
and
\begin{subnumcases}{}\label{td:3-a}
   Y_p^{''}+(\alpha_{u,p}-\beta_{u,p}^{2})Y_p=0,&  \\\label{td:3-b}
   Y_p^{'}(0)=Y_p^{'}(H)=0,~~0\leq\hat{y}\leq H.&
\end{subnumcases}

The solution of Equation (\ref{td:2-a}) is
\begin{subnumcases}{}\label{td:2-a-1}
   X_{u}(\hat{x})=C_{1}cos(\beta_{u,p}\hat{x})+C_{2}sin(\beta_{u,p}\hat{x}),&  \\\label{td:2-a-2}
   X_{u}^{'}(\hat{x})=-C_{1}\beta_{u,p}sin(\beta_{u,p}\hat{x})+C_{2}\beta_{u,p}cos(\beta_{u,p}\hat{x}).\hspace{0.7cm}&
\end{subnumcases}
According to the boundary conditions (\ref{td:2-b}), we can get $\beta_{u,p}=\frac{u\pi}{W}$ and $X_{u}(\hat{x})=C_{1}cos(\frac{u\pi}{W}\hat{x})$.

%%%%指出与eplace的表达式的不同?

Similarly, the solution of Equation (\ref{td:3-a}) is $Y_{p}(\hat{y})=D_{1}cos(\frac{p\pi}{H}\hat{y})$.
According to the superposition principle, the solution of the original problem is
\begin{equation}
\psi (\hat{x},\hat{y})=\sum _{u=0}^{\infty}\sum _{p=0}^{\infty}a_{u,p}\cos(\frac {u\pi}{W}\hat{x})\cos(\frac {p\pi}{H}\hat{y}).
\label{equ:psi}
% \vspace{-0.1cm}
\end{equation}
Unlike Equation~(\ref{equ:ePlace-psi}), Equation~(\ref{equ:psi}) is a global function which can be used to calculate exactly the potential value of any point.



%Based on the formulation in~\cite{RefFFT}, for Equations (\ref{equ:PDE-a}) and (\ref{equ:PDE-b}), we can obtain their solution of the form
%%\vspace{-0.2cm}
%\begin{equation}
%\psi (\hat{x},\hat{y})=\sum _{u=0}^{\infty}\sum _{p=0}^{\infty}a_{u,p}\cos(\frac {u\pi}{W}\hat{x})\cos(\frac {p\pi}{H}\hat{y}).
%\label{equ:psi}
%% \vspace{-0.1cm}
%\end{equation}
In order to calculate the coefficients $a_{u,p}$, we substitute Equation (\ref{equ:psi}) into Poisson's Equation (\ref{equ:PDE-a}).
By calculating $\bigtriangledown \bullet \bigtriangledown \psi (\hat{x},\hat{y})$, we can get another expression of the density function $\rho(\hat{x},\hat{y})$ as

\vspace{-0.2cm}
\begin{equation}\label{eq:p_xy}
    \small
        \begin{aligned}
    \rho(\hat{x},\hat{y}) = & \sum\limits^{\infty}_{p=1}\frac{p^2\pi^2}{H^2}a_{0,p}\cos(\frac{p\pi \hat{y}}{H})+\sum\limits^{\infty}_{u=1}\frac{u^2\pi^2}{W^2}a_{u,0}\cos(\frac{u\pi \hat{x}}{W})\\
    &+\sum\limits^{\infty}_{u=1}\sum\limits^{\infty}_{p=1}(\frac{u^2\pi^2}{W^2}+\frac{p^2\pi^2}{H^2})a_{u,p}\cos(\frac{u\pi \hat{x}}{W})\cos(\frac{p\pi \hat{y}}{H}).\\
    \end{aligned}
 %   \vspace{-0.1cm}
\end{equation}
Multiplying $\cos(\frac{\mu\pi}{W}\hat{x})\cos(\frac{\eta\pi}{H}\hat{y})$ ($\mu,\eta\geq0$) to both sides of Equation (\ref{eq:p_xy}), and making integration, we have
\begin{equation}
    \begin{aligned}
    &~~~~\iint_R\rho(\hat{x},\hat{y})\cos(\frac{\mu\pi}{W}\hat{x})\cos(\frac{\eta\pi}{H}\hat{y})d\hat{x}d\hat{y}\\
    &=\sum_{p=1}^\infty \frac{p^2\pi^2}{H^2}a_{0,p}\iint_R\cos(\frac{\mu\pi}{W}\hat{x})\cos(\frac{p\pi}{H}\hat{y})\cos(\frac{\eta\pi}{H}\hat{y})d\hat{x}d\hat{y}\\
    &+\sum_{u=1}^\infty \frac{u^2\pi^2}{W^2}a_{u,0}\iint_R\cos(\frac{u\pi}{W}\hat{x})\cos(\frac{\mu\pi}{W}\hat{x})\cos(\frac{\eta\pi}{H}\hat{y})d\hat{x}d\hat{y}\\
    &+\sum_{u=1}^\infty \sum_{p=1}^\infty (\frac{u^2\pi^2}{W^2}+\frac{p^2\pi^2}{H^2})a_{u,p}\iint_R\cos(\frac{u\pi}{W}\hat{x})\cos(\frac{\mu\pi}{W}\hat{x})\\
    &\cos(\frac{p\pi}{H}\hat{y})\cos(\frac{\eta\pi}{H}\hat{y})d\hat{x}d\hat{y}.
    \end{aligned}
    \label{equ:coscos}
\end{equation}


In Equation (\ref{equ:coscos}), the integration area $R=(0, W)\times (0, H)$. Hence, in the right-hand-side of Equation (\ref{equ:coscos}), by the orthogonality of the trigonometric functions, the first term takes a none-zero value only at $\mu=0$ and $p=\eta$, the second term only at $\eta=0$ and $u=\mu$,  and the third term only at $p=\eta$ and $u=\mu$.


Thus for $\mu\geq1$ and $\eta=0$, Equation (\ref{equ:coscos}) reduces to
\begin{equation*}
 \iint_R\rho(\hat{x},\hat{y})\cos\frac{\mu\pi \hat{x}}{W}d\hat{x}d\hat{y}=\frac{\mu^2\pi^2}{W^2}a_{\mu,0} \iint_R\cos^2\frac{\mu\pi \hat{x}}{W}d\hat{x}d\hat{y},
\end{equation*}
and we get
\begin{equation*}
a_{\mu,0}=\frac{2W}{\mu^2\pi^2H}\iint_R\rho(\hat{x},\hat{y})\cos(\frac{\mu\pi}{W}\hat{x})d\hat{x}d\hat{y}.
\end{equation*}



Similarly, we can get the coefficients of $a_{0,\eta}$ and $a_{\mu,\eta}$. To satisfy Equation (\ref{equ:PDE-c}), by Equation \eqref{equ:psi}, we have $a_{0,0} = 0$. Thus these coefficients can be summarized as follows:
\begin{subnumcases}{}\label{equation:a-a}
a_{0,0}=0,~~~~~~~~~~~~~~~~~~~~~~~~~~~~~~~~~~~~~~~~~~~~~~~~~  \\\label{equation:a-b}
a_{0,\eta}=\frac{2H}{\eta^{2}\pi^{2}W}\iint_\mathbf{R}\rho(\hat{x},\hat{y})\cos\frac{\eta\pi \hat{y}}{H}d\hat{x}d\hat{y},~(\eta\geq 1)\hspace{0.7cm}\\\label{equation:a-c}
a_{\mu,0}=\frac{2W}{\mu^{2}\pi^{2}H}\iint_\mathbf{R}\rho(\hat{x},\hat{y})\cos\frac{\mu\pi \hat{x}}{W}d\hat{x}d\hat{y},~(\mu\geq 1)\\
a_{\mu,\eta}=\frac{4WH}{(\mu^{2}H^{2}+\eta^{2}W^{2})\pi^{2}}\cdot\notag~~~~~~~~~(\mu\geq 1, \eta\geq 1)\\\label{equation:a-d}
~~~~~~~\iint_\mathbf{R}\rho(\hat{x},\hat{y})\cos\frac{\mu\pi \hat{x}}{W}\cos\frac{\eta\pi \hat{y}}{H}d\hat{x}d\hat{y}.
\end{subnumcases}



In order to make the Equations (\ref{equation:a-a})--(\ref{equation:a-d}) computable, we need a separable global density function.
Therefore, we introduce an exact density function $\rho (\hat{x},\hat{y})$ for the VLSI placement problem.
%, which is different from ePlace~\cite{RefEplace} in which the density function is approximated by a density matrix.
 Define the density function of a block $i$ in the $x$-direction as
\begin{equation*}
\theta _i(\hat{x})=
  \left\{
   \begin{aligned}
   &0,~~~~~&&|\hat{x}-x_i|>\frac {w_i}{2},  \\
   &1,~~~~~&&|\hat{x}-x_i|\leq \frac {w_i}{2}.
   \end{aligned}
   \right.
  \end{equation*}
Similarly, we can define the density function of block $i$ in the $y$-direction as $\theta _i(\hat{y})$. Then the density function of block $i$  is $\theta_i(\hat{x})\theta_i(\hat{y})$, and the density function of all blocks is
\begin{equation}
\rho (\hat{x},\hat{y})=\sum _{i=1}^{n}\rho_i (\hat{x},\hat{y})=\sum _{i=1}^{n}\theta _i(\hat{x})\theta _i(\hat{y}),
\label{equ:global density}
\end{equation}
where $n$ is the number of blocks.

 In order to satisfy Equation (\ref{equ:PDE-c}), we redefine $\rho (\hat{x},\hat{y})$ as
\begin{equation}
\begin{aligned}
\rho (\hat{x},\hat{y})&\triangleq\rho (\hat{x},\hat{y})-\frac{1}{WH}\iint_\mathbf{R}\rho (\hat{x},\hat{y})d\hat{x}d\hat{y}\\
&=\sum _{i=1}^{n}\theta _i(\hat{x})\theta _i(\hat{y})-\frac{1}{WH}\sum _{i=1}^{n}w_{i}h_{i}.
\end{aligned}
\label{equ:rho}
\end{equation}


Furthermore, for $\mu,\eta\geq 1$, we substitute Equation (\ref{equ:rho}) into Equation (\ref{equation:a-d}), and get
\begin{equation}
\begin{aligned}
&a_{\mu,\eta}\\
=&\frac{4WH}{(\mu^{2}H^{2}+\eta^{2}W^{2})\pi^{2}}(\sum _{i=1}^{n}\iint_R\rho_i (\hat{x},\hat{y})\cos\frac{\mu\pi \hat{x}}{W}\cos\frac{\eta\pi \hat{y}}{H}d\hat{x}d\hat{y} \\
&~~~-\frac{1}{WH}\sum _{i=1}^{n}w_{i}h_{i}\int^{W}_{0}\cos\frac{\mu\pi \hat{x}}{W}d\hat{x}\int^{H}_{0}\cos\frac{\eta\pi \hat{y}}{H}d\hat{y})\\
=&\frac{4W^2H^2}{(\mu^{2}H^{2}+\eta^{2}W^{2})\mu\eta\pi^{4}}\sum_{i=1}^{n}(\sin\frac {\mu\pi(x_i+\frac {w_i}{2})}{W}\\
&-\sin\frac {\mu\pi(x_i-\frac {w_i}{2})}{W})\cdot (\sin\frac {\eta\pi(y_i+\frac {h_i}{2})}{H}-\sin\frac {\eta\pi(y_i-\frac {h_i}{2})}{H}).
\end{aligned}\label{equ:aup}
\end{equation}

Similarly, we have
\begin{equation}
a_{\mu,0}=\frac{2W^2}{\mu^3\pi^3H}\sum_{i=1}^n h_{i}\left(\sin\frac{\mu\pi(x_{i}+\frac{w_{i}}{2})}{W}-\sin\frac{\mu\pi(x_{i}-\frac{w_{i}}{2})}{W}\right),
\label{equ:au0}
\end{equation}
and
\begin{equation}
a_{0,\eta}=\frac{2H^2}{\eta^3\pi^3W}\sum_{i=1}^n w_{i}\left(\sin\frac{\eta\pi(y_{i}+\frac{h_{i}}{2})}{H}-\sin\frac{\eta\pi(y_{i}-\frac{h_{i}}{2})}{H}\right).
\label{equ:a0p}
\end{equation}


Note that $a_{u,p}$ is calculated by the integral of the exact density function (\ref{equ:rho}) of the VLSI placement problem, which is more accurate than the discrete calculation in ePlace.



By Gauss's law, the electric field $\boldsymbol{\xi}(\hat{x},\hat{y})$ equals the negative gradient of the potential function $\psi (\hat{x},\hat{y})$.
 Based on  $\psi (\hat{x},\hat{y})$ in Equation~(\ref{equ:psi}), we have $\boldsymbol{\xi} (\hat{x},\hat{y})=(\xi _{\hat{x}},\xi _{\hat{y}})$, where
\begin{equation}
  \left\{
   \begin{aligned}
   &\xi _{\hat{x}}=\sum _{u=0}^{\infty}\sum _{p=0}^{\infty}\frac {u\pi}{W}a_{u,p}\sin(\frac {u\pi}{W}\hat{x})\cos(\frac {p\pi}{H}\hat{y}),  \\
   &\xi _{\hat{y}}=\sum _{u=0}^{\infty}\sum _{p=0}^{\infty}\frac {p\pi}{H}a_{u,p}\cos(\frac {u\pi}{W}\hat{x})\sin(\frac {p\pi}{H}\hat{y}).
   \end{aligned}
   \right.
   \label{equ:ksi}
  \end{equation}

From the methodology viewpoint, the major difference between our continuous analytical solution and the discrete solution in ePlace is that, ePlace uses the spectral method and the general fast Fourier transform to solve the PDE in Equations (\ref{equ:ePlace-PDE-a})--(\ref{equ:ePlace-PDE-c}), while we directly obtain an analytical solution of the PDE in Equations (\ref{equ:PDE-a})--(\ref{equ:PDE-c}).

% In addition, for an electrostatic system, a charge (block) is affected by all other charges (blocks). The influence factors consist of the distance of every two charges and their electric quantity (area). In ePlace, the distance and the electric quantity are approximated by the center coordinate of bins and the bin density, respectively. In contrast, we consider the position and area of each block to generate an electrostatic system in Equations~\eqref{equ:aup}-\eqref{equ:a0p}. As a result, we can exactly calculate the electric potential and field at all coordinates in the placement region by Equations~\eqref{equ:psi} and \eqref{equ:ksi}.
%这一段太长了,是不是有些可以放在图上说明?



 Fig. \ref{fig:Aprogress} gives an example to demonstrate the difference between our analytical solution and the numerical solution generated by the spectral method. We take the same number of items for two series, and compare their differences. For an electrostatic system, a charge (block) is affected by all other charges (blocks). In Fig. \ref{FigA1c}, we can exactly calculate the electric potential at all coordinates in the placement region by Equations~\eqref{equ:psi}, which can reflect subtle density changes. In Fig. \ref{FigA1d}, the distribution of potential is related to the division of bin structure. The potential value of each point is not continuous and it does not reflect the details.


From the derivation process and the figure, it is obvious that our analytical solution can get a more beautiful and accurate potential distribution than the spectral method. Hence the direct solving method is more conducive to the movement of the block.


% Figure environment removed

%%%%%%%%%%%%%%%%%%%%%%%%%%%%%%%%%%%%%%%%%%%%%%%%%%%%%%%%%%%%%%%%%%%%%%%%%%
% subsection{Convergence of  Analytical Solution}
%%%%%%%%%%%%%%%%%%%%%%%%%%%%%%%%%%%%%%%%%%%%%%%%%%%%%%%%%%%%%%%%%%%%%%%%%%
\subsection{Convergence of Analytical Solution}\label{section:convegence}
%%%%%%%%%%%%%%%%%%%%%%%%%%%%%%%%%%%%%%%%%%%%%%%%%%%%%%%%%%%%%%%%%%%%%%%%%%
% subsection{Convergence of  Analytical Solution}
%%%%%%%%%%%%%%%%%%%%%%%%%%%%%%%%%%%%%%%%%%%%%%%%%%%%%%%%%%%%%%%%%%%%%%%%%%

Since  $\psi(\hat{x},\hat{y})$ in Equation~(\ref{equ:psi}) is an infinite series,
we  prove that  $\psi (\hat{x},\hat{y})$ is convergent.


\begin{Lemma}
    The infinite series $\sum\limits_{p=1}^{\infty}\frac{1}{p^{3}}$ and $\sum\limits_{u=1}^{\infty}\sum\limits_{p=1}^{\infty}\frac {1}{u^3pH^{2}+up^3W^{2}}$ are convergent.
    \label{lemma1}
\end{Lemma}

\begin{Theorem}
    The infinite series $\psi(\hat{x},\hat{y})$ is absolutely convergent.
    \label{theorem1}
\end{Theorem}
\begin{proof}


Note that
 \begin{equation*}
\begin{aligned}
&-4n\leq\sum_{i=1}^{n}\left(\sin\frac {u\pi(x_i+\frac {w_i}{2})}{W}-\sin\frac {u\pi(x_i-\frac {w_i}{2})}{W}\right) \cdot\\
&~~~~~~~~~~~~~~~\left(\sin\frac {p\pi(y_i+\frac {h_i}{2})}{H}-\sin\frac {p\pi(y_i-\frac {h_i}{2})}{H}\right)\leq4n.
\end{aligned}
\end{equation*}
Then for $a_{u,p}$, $u,p\geq 1$, by Equation \eqref{equ:aup},



\begin{equation*}
%   \begin{aligned}
|a_{u,p}|\leq\frac {4W^{2}H^{2}}{up(u^2H^{2}+p^2W^{2})\pi ^4}4n=\frac {16W^{2}H^{2}n}{up(u^2H^{2}+p^2W^{2})\pi ^4}.
%   \end{aligned}
  \end{equation*}
Similarly, for the other two cases, $u=0,p\geq 1$ and $u\geq 1,p=0$, we have
    $$|a_{0,p}|\le\frac{4H^2}{p^3\pi^3W}\sum_{i=1}^n w_{i}, $$
and
    $$|a_{u,0}|\le\frac{4W^2}{u^3\pi^3H}\sum_{i=1}^n h_{i}.$$
Thus
\begin{equation}
   \begin{aligned}
|\psi(\hat{x},\hat{y})|&=|\sum _{u=0}^{\infty}\sum _{p=0}^{\infty}a_{u,p}\cos\frac {u\pi\hat{x}}{W}\cos\frac {p\pi\hat{y}}{H}|\leq\sum _{u=0}^{\infty}\sum _{p=0}^{\infty}|a_{u,p}|\\
&\le\sum _{p=1}^{\infty}|a_{0,p}|+\sum _{u=1}^{\infty}|a_{u,0}|+\sum _{u=1}^{\infty}\sum _{p=1}^{\infty}|a_{u,p}|\\
&\leq\frac {4H^{2}\sum\limits_{i=1}^{n}w_{i}}{W\pi ^3}\sum\limits_{p=1}^{\infty}\frac{1}{p^{3}}+\frac {4W^{2}\sum\limits_{i=1}^{n}h_{i}}{H\pi ^3}\sum_{u=1}^{\infty}\frac{1}{u^{3}}\\
&~~~+\frac {16W^{2}H^{2}n}{\pi ^4}\sum _{u=1}^{\infty}\sum _{p=1}^{\infty}\frac {1}{u^3pH^{2}+up^3W^{2}}. \label{upper}
   \end{aligned}
  \end{equation}



By Lemma \ref{lemma1}, the above three infinite series are convergent. Hence there exists a convergent upper bound of $|\psi(\hat{x},\hat{y})|$, and thus $\psi(\hat{x},\hat{y})$ is absolutely convergent.
\end{proof}


According to Theorem \ref{theorem1}, we only need to calculate a partial sum of $\psi(\hat{x},\hat{y})$ for actual computation. Furthermore,
$\boldsymbol{\xi}(\hat{x},\hat{y})$ is the negative gradient of $\psi (\hat{x},\hat{y})$, and similarly it also can be calculated by a partial sum of $\boldsymbol{\xi}(\hat{x},\hat{y})$.
Since the denominators in Equation~\eqref{upper} contain $u^3$ or $p^3$, $\psi(\hat{x},\hat{y})$ converges quickly.
Hence we only need to take the first $K$ terms in $\psi(\hat{x},\hat{y})$ to compute a partial sum, which would be a nearly exact solution.

%%%%%%%%%%%%%%%%%%%%%%%%%%%%%%%%%%%%%%%%%%%%%%%%%%%%%%%%%%%%%%%%%%%%%%%%%%
% subsection{Time Complexity of Calculating Partial Sum}
%%%%%%%%%%%%%%%%%%%%%%%%%%%%%%%%%%%%%%%%%%%%%%%%%%%%%%%%%%%%%%%%%%%%%%%%%%
\subsection{Time Complexity of Calculating a Partial Sum}
%%%%%%%%%%%%%%%%%%%%%%%%%%%%%%%%%%%%%%%%%%%%%%%%%%%%%%%%%%%%%%%%%%%%%%%%%%
% subsection{Time Complexity of Calculating Partial Sum}
%%%%%%%%%%%%%%%%%%%%%%%%%%%%%%%%%%%%%%%%%%%%%%%%%%%%%%%%%%%%%%%%%%%%%%%%%%


In this subsection, we analyze the time complexity of calculating a partial sum of our analytical solution.
First, calculating each coefficient $a_{u,p}$ (Equations (\ref{equ:aup}), (\ref{equ:au0}), and (\ref{equ:a0p})) requires $O(n)$ time, where $n$ is the number of blocks.
Second, the electric potential $\psi (\hat{x},\hat{y})$ and the field $\boldsymbol{\xi}(\hat{x},\hat{y})$ can be computed by Equations (\ref{equ:psi}) and (\ref{equ:ksi}), respectively.
By the proof of Theorem~\ref{theorem1}, $u$ and $p$ are taken from 0 to $K$.
%, where $K$ can be considered a constant.
To obtain the partial sums of $\psi(\hat{x},\hat{y})$ and $\boldsymbol{\xi}(\hat{x},\hat{y})$, we need to calculate $K^2$ coefficients $a_{u,p}$.
Hence, computing all the coefficients $a_{u,p}$ requires $O(K^2n)$ time.


After computing all the coefficients, we must calculate $\psi (\hat{x},\hat{y})$ and $\boldsymbol{\xi}(\hat{x},\hat{y})$.
For each block $i$, $\psi_i (\hat{x},\hat{y})$ and $\boldsymbol{\xi}_i (\hat{x},\hat{y})$ denote the electric potential and the field at this block, respectively.
For $\psi_i (\hat{x},\hat{y})$ and $\boldsymbol{\xi}_i (\hat{x},\hat{y})$, we substitute $(x_i,y_i)$ into variables $(\hat{x},\hat{y})$.
According to Equations (\ref{equ:psi}) and (\ref{equ:ksi}), computing the electric potential and the field for each block requires $O(K^2)$ time.
With $n$ blocks for the problem, the total time of calculating $\psi (\hat{x},\hat{y})$ and $\boldsymbol{\xi} (\hat{x},\hat{y})$ is $O(K^2n)$.
Hence, the time complexity of computing the partial sums of the analytical solution is $O(K^2n)$.



 %%%%%%%%%%%%%%%%%%%%%%%%%%%%%%%%%%%%%%%%%%%%%%%%%%%%%%%%%%%%%%%%%%%%%%%%%%
% subsection{Fast Solution of Poisson's Equation}
%%%%%%%%%%%%%%%%%%%%%%%%%%%%%%%%%%%%%%%%%%%%%%%%%%%%%%%%%%%%%%%%%%%%%%%%%%

 \subsection{Fast Computation Scheme of Poisson's Equation} \label{sec:MALM_for_GP}

  %%%%%%%%%%%%%%%%%%%%%%%%%%%%%%%%%%%%%%%%%%%%%%%%%%%%%%%%%%%%%%%%%%%%%%%%%%
% subsection{Fast Solution of Poisson's Equation}
%%%%%%%%%%%%%%%%%%%%%%%%%%%%%%%%%%%%%%%%%%%%%%%%%%%%%%%%%%%%%%%%%%%%%%%%%%


Calculating a value of the partial sum needs $O(K^2n)$ time, where $K$ cannot be too small.
As a result, it is time-consuming, especially for the placement problem with millions of blocks.
In this subsection, we give a fast computation scheme of Poisson's equation based on the
analytical solution derived in Subsection~\ref{3b}.

First, we divide a chip region into $m\times m$ uniform bins,
labeled as bin $b_{lj}$, $l, j=0, 1,\cdots, m-1$. The density of bin $b_{lj}$ is
\begin{equation*}
P_{l,j}=\frac{\mbox{Total~area~of~blocks~in~bin}~b_{lj}}{\mbox{Area~of~bin}~b_{lj}}.
\end{equation*}
The area of block $i$ in bin $b_{lj}$ is determined by the block size,
which is inversely proportional to the distance between the center point of block $i$ and bin $b_{lj}$.
It is similar to the local smoothness and density scaling in ePlace.
This density expression does not differentiate between  macros and  standard cells.
Specifically, we have the following density function:
\begin{equation*}
  \rho_{l,j}(\hat{x},\hat{y})=\left\{
   \begin{aligned}
   &P_{l,j},        ~~~~~|\hat{x}-x_{l}|\le\frac{w_{b}}{2} ~~\mbox{and}~~ |\hat{y}-y_{j}|\le\frac{h_{b}}{2},  \\
   &0,~~~~~~~~~\mbox{otherwise},
   \end{aligned}
   \right.
  \end{equation*}
where $\rho_{l,j}(\hat{x},\hat{y})$ represents the global density function of bin $b_{lj}$.
Let $(x_{l},y_{j})$ denote the coordinate of the center point of bin $b_{lj}$.
Then, we accumulate the density functions of all bins, which approximates the density distribution in Equation (\ref{equ:global density}). The density function of the placement region is
 \begin{equation}
 \rho(\hat{x},\hat{y})=\sum\limits_{l=0}^{m-1}\sum\limits_{j=0}^{m-1}\rho_{l,j}(\hat{x},\hat{y})-\frac{1}{m^2}\sum\limits_{l=0}^{m-1}\sum\limits_{j=0}^{m-1}P_{l,j}.
 \label{equ:rho-dis}
 \end{equation}

Substituting Equation (\ref{equ:rho-dis}) into Equation (\ref{equation:a-b}) based on $a_{u,p}$ in Equations (\ref{equation:a-b}), (\ref{equation:a-c}), and (\ref{equation:a-d}), for the case where $u=0$ and $p\geq 1$, we get
\begin{equation}
   \begin{aligned}
&a_{0,p}\\
&=\frac{2H}{p^2\pi^2W}\iint_{R}\rho(\hat{x},\hat{y})\cos(\frac{p\pi}{H}\hat{y})d\hat{x}d\hat{y}\\
&=\frac{2H^2w_{b}}{p^3\pi^3W}\sum_{l,j=0}^{m-1}P_{l,j}\left(\sin\frac{p\pi(y_{j}+\frac{h_{b}}{2})}{H}
   -\sin\frac{p\pi(y_{j}-\frac{h_{b}}{2})}{H}\right).
   \label{equ:aupbin}
   \end{aligned}
\end{equation}
Note that $x_{l}$ and $y_j$ are determined by the bin sizes and the placement region. Substituting $x_{l}=\frac{W}{m}(l+\frac{1}{2}),y_{j}=\frac{H}{m}(j+\frac{1}{2})$ into Equation (\ref{equ:aupbin}), we get
\begin{equation*}
   \begin{aligned}
a_{0,p}&=\frac{2H^2w_{b}}{p^3\pi^3W}\sum_{l=0}^{m-1}\sum_{j=0}^{m-1}2P_{l,j}\cos(\frac{p\pi}{H}y_{j})\sin(\frac{h_b}{2}\frac{p\pi}{H})\\
&=\frac{4H^2}{p^3\pi^3m}\sin(\frac{p\pi}{2m})\sum_{l=0}^{m-1}\sum_{j=0}^{m-1}P_{l,j}\cos(\frac{p(j+\frac{1}{2})\pi}{m}).
   \end{aligned}
   \label{a0p-dis}
\end{equation*}
The other coefficients can be calculated similarly. Then, all the coefficients $a_{u,p}$ can be summarized as follows:


\begin{subnumcases}{}\label{equation:disa-a}
a_{0,0}=0,  \\\label{equation:disa-b}
a_{0,p}=\frac{4H^2}{p^3\pi^3m}\sin\frac{p\pi}{2m}\sum_{l=0}^{m-1}\sum_{j=0}^{m-1}P_{l,j}\cos\frac{p(j+\frac{1}{2})\pi}{m},\hspace{0.8cm}\\\label{equation:disa-c}
a_{u,0}=\frac{4W^2}{u^3\pi^3m}\sin\frac{u\pi}{2m}\sum_{l=0}^{m-1}\sum_{j=0}^{m-1}P_{l,j}\cos\frac{u(l+\frac{1}{2}\pi}{m}),\\
a_{u,p}=\frac{16W^2H^2}{up(u^2H^2+p^2W^2)\pi^4}\sin\frac{u\pi}{2m}\sin\frac{p\pi}{2m}\notag\\\label{equation:disa-d}
~~~~~~~~\cdot\sum_{l=0}^{m-1}\sum_{j=0}^{m-1}P_{l,j}\cos\frac{u(l+\frac{1}{2})\pi}{m}\cos\frac{p(j+\frac{1}{2})\pi}{m}.
\end{subnumcases}

For each bin $b_{lj}$, the electric potential $\psi(\hat{x},\hat{y})$ in Equation~(\ref{equ:psi}) can be recalculated by
\begin{equation*}
   \begin{aligned}
   &\psi(l,j)=\sum_{u=0}^\infty\sum_{p=0}^\infty a_{u,p}\cos(\frac{u\pi}{W}x_{l})\cos(\frac{p\pi}{H}y_{j})\\
   &~~~~~~~~=\sum_{u=0}^\infty\sum_{p=0}^\infty a_{u,p}\cos(\frac{u(l+\frac{1}{2})\pi}{m})\cos(\frac{p(j+\frac{1}{2})\pi}{m}).\\
   \end{aligned}
         \label{equ:dis-psi}
  \end{equation*}

By Theorem \ref{theorem1}, the infinite series $\psi(\hat{x},\hat{y})$ is absolutely convergent.  Thus, we can make the following approximation:
\begin{equation}
   \psi(l,j)\approx\sum_{u=0}^{K}\sum_{p=0}^{K} a_{u,p}\cos(\frac{u(l+\frac{1}{2})\pi}{m})\cos(\frac{p(j+\frac{1}{2})\pi}{m}).
            \label{equ:dis-psi1}
  \end{equation}
And the electric field $\boldsymbol{\xi}(\hat{x},\hat{y})$ in Equation (\ref{equ:ksi}) can be approximated by
\begin{equation}
  \left\{
   \begin{aligned}
   &\xi _{\hat{x}}\approx\sum_{u=0}^{K}\sum_{p=0}^{K}\frac {u\pi}{W}a_{u,p}\sin(\frac{u(l+\frac{1}{2})\pi}{m})\cos(\frac{p(j+\frac{1}{2})\pi}{m}),  \\
   &\xi _{\hat{y}}\approx\sum_{u=0}^{K}\sum_{p=0}^{K}\frac {p\pi}{H}a_{u,p}\cos(\frac{u(l+\frac{1}{2})\pi}{m})\sin(\frac{p(j+\frac{1}{2})\pi}{m}).
   \end{aligned}
   \right.
   \label{equ:dis-ksi}
\end{equation}


To make a good tradeoff between runtime and solution quality, we present the above fast computation scheme of Poisson's equation based on our analytical solution. Taking $K=m-1$ and $m=\sqrt{n}$, we can use the fast Fourier transform~\cite{RefFFTL} to quickly compute the values of Equations (\ref{equation:disa-b}), (\ref{equation:disa-c}), (\ref{equation:disa-d}), (\ref{equ:dis-psi1}), and (\ref{equ:dis-ksi}).



Fig. \ref{fig:Bprogress} shows the difference between our fast computation scheme and the numerical solution by the spectral method.
The density calculation and local smoothing technology in both the methods are the same. In the two images Fig. \ref{FigB1c} and Fig. \ref{FigB1d},
the overall trends are consistent. However, there are some differences between Fig. \ref{FigB1c} and Fig. \ref{FigB1d}, e.g., at the coordinate $(0,0)$. Fig. \ref{FigB1d} shows the error of  Fig. \ref{FigB1c} minus Fig. \ref{FigB1d}. According to Section \ref{3b}, our analytical solution is an accurate potential distribution, and the figure obtained by our fast computation scheme is closer to our analytical solution in detail.

%Therefore, we compare with Fig. \ref{FigB1b}.
%In addition to the solution method,
%Overall,. However, the
%, which indicates that our fast computation scheme is more accurate.
% Figure environment removed


We  analyze the complexity of our fast computation scheme of Poisson's equation.
In Equations (\ref{equation:disa-a})--(\ref{equation:disa-d}),
ignoring the coefficients for the summations,
we first calculate an $m\times m$ coefficient matrix of $a_{u,p}^{'}$ as follows:
\begin{subnumcases}{}
a'_{0,0}=0,  \\
a'_{0,p}=\sum_{l=0}^{m-1}\sum_{j=0}^{m-1}P_{l,j}\cos\frac{p(j+\frac{1}{2})\pi}{m},\\
a'_{u,0}=\sum_{l=0}^{m-1}\sum_{j=0}^{m-1}P_{l,j}\cos\frac{u(l+\frac{1}{2})\pi}{m},\\
a'_{u,p}=\sum_{l=0}^{m-1}\sum_{j=0}^{m-1}P_{l,j}\cos\frac{u(l+\frac{1}{2})\pi}{m}\cos\frac{p(j+\frac{1}{2})\pi}{m}.\hspace{1cm}
\end{subnumcases}
By invoking the FFT library only once, the coefficient matrix can be calculated. In addition, it takes $m^2$ time to update the coefficient matrix of $a'_{u,p}$ to $a_{u,p}$.
As a result, calculating all coefficients $a_{u,p}$ takes $O(m^2\log m^2)+O(m^2)=O(n\log n)$ time.



After calculating all the coefficients, we can calculate $\psi(l,j)$ and $\boldsymbol{\xi}(l,j)$ by using the inverse fast Fourier transform,
which
also takes $O(n\log n)$ time.
Note that $\psi(l,j)$ and $\boldsymbol{\xi}(l,j)$ are the respective electric potential and field values of bin $b_{lj}$.
According to the position of each block,
the electric potential and field values of block $i$ can be approximated by weighting the corresponding values of its surrounding bins, which takes $O(n)$ time. So the total time complexity is $O(n\log n)$.




%%%%%%%%%%%%%%%%%%%%%%%%%%%%%%%%%%%%%%%%%%%%%%%%%%%%%%%%%%%%%%%%%%%%%%%%%%
% Proposed Algorithm
%%%%%%%%%%%%%%%%%%%%%%%%%%%%%%%%%%%%%%%%%%%%%%%%%%%%%%%%%%%%%%%%%%%%%%%%%%
\section{Framework of Our Pplace Algorithm}
\label{sec:our placement algorithm}
%%%%%%%%%%%%%%%%%%%%%%%%%%%%%%%%%%%%%%%%%%%%%%%%%%%%%%%%%%%%%%%%%%%%%%%%%%
% Proposed Algorithm
%%%%%%%%%%%%%%%%%%%%%%%%%%%%%%%%%%%%%%%%%%%%%%%%%%%%%%%%%%%%%%%%%%%%%%%%%%


%\vspace{-0.3cm}
% Figure environment removed

Based on the ePlace framework, we develop our Pplace placement flow as in Figure~\ref{fig:algorithm}.
%First, we call a QP Solver to obtain an initial placement.
%Before global placement, we insert fillers in the chip region
%to compact the blocks in the placement region.
%A filler is a movable standard cell, which does not connect to any other blocks.
%The total area of the fillers $A_f$ is
%%\vspace{-0.1cm}
%\begin{equation*}
%  A_f = \rho_tA_w-A_m  \cite{RefEplace},
%%  \vspace{-0.1cm}
%\end{equation*}
%where $\rho_t$ represents the target density,
%$A_w$ and $A_m$ denote the white space and the total area of movable blocks, respectively.
%The height of a filler is its row height, and the width is the average width of standard cells.
%Like ePlace, we add fillers randomly into the placement region such that the chip density
%reaches the target density.
%The number of added fillers $n_f$ is equal to $A_f$ divided by the average area of  fillers:
%\begin{equation*}
%  n_f = \frac{A_f}{\mbox{row-height}\cdot \mbox{avg.cell-witdh}}.
%\end{equation*}
%
%  In Problem (\ref{eq:placement}), $W(v)$ is not differentiable, which is hard to optimize directly.
%    We apply the log-sum-exp wirelength \cite{RefLSE} to approximate the total HPWL.
%    The LSE wirelength function in the $x$-direction is
% \begin{equation*}
%   LSE_x(v)=\gamma\sum_{e\in E}\left(\ln{\sum_{i\in e}\exp(\frac{x_i}{\gamma})} + \ln{\sum_{i\in e}\exp(\frac{-x_i}{\gamma})}\right),
%  \end{equation*}
% where $\gamma$ is a smoothing parameter. Then we reformulate the placement problem as in ePlace:
We reformulate the placement problem as in ePlace \cite{RefEplace}:
\begin{equation}
\operatorname*{min}\limits_{v}f(v) = W_{LSE}(v)+\lambda N(v).
\label{problem}
\end{equation}

In Problem (\ref{problem}), we consider both of the given blocks and added fillers.
Given the positions of blocks and fillers,
an exact density function can be calculated by Equation~(\ref{equ:rho}).
Using the analytical solution of Poisson's equation,
we can get the electric potential $\psi(\hat{x},\hat{y})$ and field $\boldsymbol{\xi}(\hat{x},\hat{y})$
by Equations (\ref{equ:psi}) and (\ref{equ:ksi}), respectively.


%However, runtime is very important to a placer for the VLSI placement problem.
To speed up our Pplace algorithm given in Figure~\ref{fig:algorithm},
we apply the fast computation scheme of Poisson's equation
described in Subsection~\ref{sec:MALM_for_GP}.
By dividing the placement region into $m\times m$ uniform bins,
 the global density function is approximated by Equation (\ref{equ:rho-dis}).
Then, the electric potential $\psi(l,j)$ and field $\boldsymbol{\xi}(l,j)$ are approximated by Equations (\ref{equ:dis-psi1}) and (\ref{equ:dis-ksi}), respectively.
For each block, as a result,
the electric potential and field can be calculated by weighting the corresponding values of its surrounding bins.


At the $k$-th iteration, as described in Section~\ref{review ePlace},
we can quickly compute the energy (density) gradient $\nabla N^{k}=-q\boldsymbol{\xi}^k$ by using the FFT library~\cite{RefFFTL}.
With the gradient of the LSE function, we can get the gradient of the objective function $\nabla f(v^{k})$.


For fair comparison between the numerical solution method in ePlace and our fast computation scheme of Poisson's equation,
we use the same optimization method and parameters as in ePlace.
At each iteration, as in ePlace, we use Nesterov's method to solve the unconstrained minimization problem
by only one iteration and get a new solution $(x^{k+1}, y^{k+1})$,
and then update the penalty parameter $\lambda$.
If the overflow ratio meets the termination criterion,
then the algorithm is terminated.
After obtaining a global placement result,
the legalization and detailed placement solvers provided by NTUplace3 \cite{RefNtu3}
are called to obtain a final placement result.



%%%%%%%%%%%%%%%%%%%%%%%%%%%%%%%%%%%%%%%%%%%%%%%%%%%%%%%%%%%%%%%%%%%%%%%%%%
% Experimental Results
%%%%%%%%%%%%%%%%%%%%%%%%%%%%%%%%%%%%%%%%%%%%%%%%%%%%%%%%%%%%%%%%%%%%%%%%%%
\section{Experimental Results} \label{sec:experimental results}
%%%%%%%%%%%%%%%%%%%%%%%%%%%%%%%%%%%%%%%%%%%%%%%%%%%%%%%%%%%%%%%%%%%%%%%%%%
% Experimental Results
%%%%%%%%%%%%%%%%%%%%%%%%%%%%%%%%%%%%%%%%%%%%%%%%%%%%%%%%%%%%%%%%%%%%%%%%%%


In this section, we first compare our proposed Pplace placer with ePlace.
Then, we embed the density calculation method into NTUplace3 by replacing its density control method to show our robustness.
We implemented our Pplace placer using the C++ programming language and ran the program in a single-thread mode on a Linux machine with 3.20GHz Intel Core i5 6500 CPU and 16GB memory.
In our experiments, we used the ISPD 2005~\cite{Ref2005} and ISPD 2006 benchmark suites~\cite{Ref2006} for our comparative studies based on
wirelength and runtime \emph{because it is easier to see the effects of the algorithms with these fundamental placement metrics}.
Note that we do not use more recent placement contest benchmarks because these contests were mainly on routability optimization with various constraints (e.g., fence regions); further
ePlace was reported to achieve the best published wirelength based on these benchmarks.


Table \ref{table:exp_result} and Table \ref{table:exp_result.06} show the circuit statistics of the ISPD 2005 and ISPD 2006 benchmarks,
for which the problem sizes range from 211K to 2508K.
The numbers of blocks, nets, standard cells, macros and target density for the benchmark are denoted by ``\#Blocks'', ``\#Nets'', ``\#Std. Cells'', ``\#Mac.'' and ``\#Density'', respectively.
%And the ICCAD 2012 routability-driven placement contest benchmarks~\cite{Ref:ICCAD12} are used in Table \ref{table:exp_result.12}, for which the number of movable blocks ranges from 442K to 1271K.



\subsection{Comparison with ePlace}


In the ePlace work~\cite{RefEplace}, it shows that ePlace achieves the shortest total wirelength for all the eight benchmarks compared with ten academic placers. In addition, it is the fastest among all nonlinear (non-quadratic) placers.
For fair comparison on \emph{global placement},
we use the legalization and detailed placement methods by NTUplace3~\cite{RefNtu3} in our placer,
the same as in ePlace.

Since we only have the ePlace binary code, but do not have its source code,
it is not feasible to directly compare the numerical method in ePlace with our fast computation scheme of Poisson's equation
based on the ePlace version in~\cite{RefEplace}.
As a result, we implemented the numerical solution by spectral method and named it the ``num-spectral''.

Using the same legalization and detailed placement methods,
the HPWL and runtime comparisons among ePlace, ``num-spectral'' and our Pplace placer are reported in Table~\ref{table:exp_result} and Table~\ref{table:exp_result.06}, respectively.
In the tables, ``GP-HWPL'' gives the HPWL wirelength after \emph{global placement},
``GP-CPU'' the corresponding runtime,
``HWPL'' the HPWL wirelength of the final placement, ``sHWPL'' the scaled HPWL,
and ``CPU (s)'' the runtime in seconds of the total running time of a placer.
The bottom row gives the normalized wirelength and runtime ratios based on our results,
and the best results among the three placers are marked in bold.

\begin{table*}[htbp]
  \centering
  %\fontsize{6.7pt}{8.2pt}\selectfont
  \caption{Comparisons among ePlace, ``num-spectral,'' and ours based on the ISPD 2005 Contest benchmarks~\cite{Ref2005}.}
  \scalebox{0.6}[1]{
  \vspace{0.0cm}
  \begin{tabular}{|c|c|c|c|c|c|c|c|c|c|c|c|c|c|c|c|c| }
  \hline
       \multicolumn{1}{|c|}{\empty} &
       \multicolumn{4}{c|}{Benchmark Statistics} &
       \multicolumn{4}{c|}{ePlace} &
       \multicolumn{4}{c|}{``num-spectral''} &
       \multicolumn{4}{c|}{Our Pplace}
       \\ \cline{2-17}
  Benchmark &\#Blocks &\#Nets &\#Std. Cells &\#Mac. &GP-HPWL&GP-CPU&HPWL &CPU (s)&GP-HPWL&GP-CPU&HPWL &CPU (s)&GP-HPWL&GP-CPU&HPWL &CPU (s)\\
  \hline
   adaptec1    &211447  &221142  &210904  &543  &73118044&135 &74643953          &164           &73604755&110&75989399  &\bfseries132  &72148076&120&\bfseries73387186   &147   \\
   adaptec2    &255023  &266009  &254457  &566  &83534323&187 &84863690          &228           &81206401&156&85600480  &200  &80249273&158&\bfseries83033741   &\bfseries195   \\
   adaptec3    &451650  &466758  &450927  &723  &193730691&612 &196395438         &688           &191055370&448&198117119 &\bfseries531  &189583003&567&\bfseries193577147  &640  \\
   adaptec4    &496045  &515951  &494716  &1329 &178102879&657 &179148257         &738           &181105154&569&184729214 &627  &174548043&448&\bfseries177299566  &\bfseries524   \\
   bigblue1    &278164  &284479  &277604  &560  &89625096&262 &90985255          &\bfseries300  &92122364&275&91608281  &315  &89133405&298&\bfseries89739424   &332            \\
   bigblue2    &557866  &577235  &534782  &23084&139657694&371 &141819172         &\bfseries495  &141215231&433&146551686 &537  &137121253&438&\bfseries140828061  &548            \\
   bigblue3    &1096812 &1123170 &1093034 &3778 &296504804&1143 &308004959         &1330          &311820347&870&311618882 &\bfseries1061 &300830027&1013&\bfseries306369240  &1204  \\
   bigblue4    &2177353 &2229886 &2169183 &8170 &739782972&2300 &756310869&\bfseries2804 &772231562&3386&774801191 &3891 &744151236&2817&\bfseries754594872           &3187           \\ \hline
   Normalized  &        &        &        &     &1.01    &1.07    &1.01             &1.07         &1.03&0.99&1.03     &0.99&1.00&1.00&1.00                  &1.000      \\

  \hline
  \end{tabular}
  \label{table:exp_result}
  }
\end{table*}

Compared with ePlace, from  Table~\ref{table:exp_result}, our Pplace placer achieves better HPWL results on all the eight benchmarks.
On average, Pplace achieves 1\% smaller final HPWL and is 7\% faster than ePlace.
Pplace also achieves 3\% smaller HPWL than ``num-spectral'' in almost the same running time.
Our HPWL improvement over ``num-spectral'' justifies the effectiveness of our fast computation scheme for the
analytical solution of Poisson's equation, because the only difference between ``num-spectral'' and Pplace
lies in the solution methods of Poisson's equation
(i.e., a numerical method by the spectral method and an analytical solution for ours).
In particular, the improved normalized values for global placement ``GP-HWPL'' and for final placement ``HPWL''
are almost the same, revealing that our algorithm is effective and efficient for global placement optimization.


Figure~\ref{fig:progress} illustrates the block distribution during global placement by Pplace on the adaptec1 benchmark.
Note that fillers are not shown in order to better observe the movements of blocks (marked in blue).

% Figure environment removed



We also tested our algorithm on the ISPD 2006 benchmarks \cite{Ref2006}, and compare the results with ePlace in Table \ref{table:exp_result.06}.
The experimental results show that, our Pplace placer also achieves better HPWL and sHPWL  on all the eight benchmarks.
On average, Pplace achieves respectively 1\% and 1\% shorter HPWL and sHPWL than ePlace.
\begin{table*}[htbp]
  \centering
  %\fontsize{6.7pt}{8.2pt}\selectfont
  \caption{Comparisons among ePlace, ``num-spectral,'' and ours based on the ISPD 2006 Contest benchmarks~\cite{Ref2006}.}
  \scalebox{0.7}[1]{
  \vspace{0.0cm}
  \begin{tabular}{|c|c|c|c|c|c|c|c|c|c|c|c|c|c| }
  \hline
       \multicolumn{1}{|c|}{\empty} &
       \multicolumn{4}{c|}{Benchmark Statistics} &
       \multicolumn{3}{c|}{ePlace} &
       \multicolumn{3}{c|}{``num-spectral''} &
       \multicolumn{3}{c|}{Our Pplace}
       \\ \cline{2-14}
  Benchmark &\#Blocks &\#Std. Cells &\#Mac. &\#Density &HPWL&sHPWL &CPU (s)&HPWL&sHPWL &CPU (s)&HPWL&sHPWL &CPU (s)\\
  \hline
   adaptec5    &843128  &842482  &646  &50\%   &394695282 &397557738&\bfseries1265 &403286002&406217085 &1266 &392207460 &\bfseries397487761 &1466   \\
   newblue1    &330474  &330037  &401  &80\%   &62304524  &62451418 &367  &62448920 &62922805  &\bfseries245  &58129002  &\bfseries58585324  &323   \\
   newblue2    &441516  &436516  &5000  &90\%  &181378387 &182354956&376  &186214953&188717090 &\bfseries353  &181379584 &\bfseries182182053 &396  \\
   newblue3    &494011  &482833  &11178  &80\% &265770119 &265991914&588  &266567897&269089963 &609  &263713231 &\bfseries264092804 &\bfseries565   \\
   newblue4    &646139  &642717  &3422  &50\%  &272832375 &276140253&770  &261597190&286468349 &786  &264451809 &\bfseries268925895 &\bfseries679            \\
   newblue5    &1233058  &1228177  &4881  &50\%&489738331 &492808559&1978 &485240543&497299421 &2200 &484023312 &\bfseries489359921 &\bfseries1801            \\
   newblue6    &1255039 &1248150 &6889 &80\%   &462673606 &464468022&\bfseries1919 &472367792&474858888 &2441 &462220925 &\bfseries463979824 &2416  \\
   newblue7    &2507954 &2481372 &26582 &80\%  &987346891 &989865614&\bfseries3246 &986219208&1007701401&4705 &987060153 &\bfseries989058803&3670           \\ \hline
   Normalized  &        &        &        &    &1.01     &1.01    &0.99&1.02    &1.03     &1.05&1.00&1.00                 &1.00      \\

  \hline
  \end{tabular}
  \label{table:exp_result.06}
  }
\end{table*}



%In Figure~\ref{Fig1a}, the global placement starts from an initial solution with very close blocks.
%In Fig. \ref{Fig1b} and Fig. \ref{Fig1c}, the blocks are rapidly spread to the entire placement region.
%Then, as shown in Fig. \ref{Fig1d}, the blocks are moved roughly to their final locations.


\subsection{Robustness of Our Fast Computation Scheme}

NTUPlace3~\cite{RefNtu3} is a high-quality, robust placer with a multilevel framework.
To examine the robustness of our method,
we replace the density control in NTUplace3 with our fast computation scheme of Poisson's equation.
Also, the density gradient calculation in NTUPlace3 is changed to the electric field.
For convenience, the modified NTUplace3 is called MNTU3-multilevel.
Table~\ref{table:exp_result1} reports the experimental results of NTUPlace3 and MNTU3-multilevel.



\begin{table}[htbp]
%\setlength{\abovecaptionskip}{0.cm}
%\setlength{\belowcaptionskip}{-1.20cm}
  \centering
  %\fontsize{6.7pt}{8.2pt}\selectfont
  \caption{Comparisons of the HPWL and Runtime (seconds) among NTUplace3, MNTU3-multilevel, and MNTU3-flat.}
  \scalebox{0.6}[1]{
  \vspace{0.0cm}
  \begin{tabular}{|c|c|c|c|c|c|c|c|c| }
  \hline
       \multicolumn{1}{|c|}{\empty} &
       \multicolumn{2}{c|}{NTUplace3} &
       \multicolumn{2}{c|}{MNTU3-multilevel}&
       \multicolumn{2}{c|}{MNTU3-flat }
       \\ \cline{2-7}
  Benchmark         &HPWL  &CPU(s) &HPWL &CPU(s)&HPWL  &CPU(s) \\
  \hline
   adaptec1    &80291340  &183           &74860458  &\bfseries125    &\bfseries74696137   &137 \\
   adaptec2    &90180949  &226           &85048037  &\bfseries163    &\bfseries83314442   &173 \\
   adaptec3    &233773979 &\bfseries501  &201084986 &530    &\bfseries196870086  &592\\
   adaptec4    &215015818 &602           &181371097 &554    &\bfseries178493350  &\bfseries520\\
   bigblue1    &98653718  &349           &91868762  &\bfseries239    &\bfseries89746505   &322 \\
   bigblue2    &158269695 &876           &144905354 &461    &\bfseries144599935  &\bfseries399\\
   bigblue3    &346325425 &1044 &327982683 &\bfseries862  &\bfseries311977354  &1123\\
   bigblue4    &829086990 &3125 &817964751 &\bfseries2235  &\bfseries799215445           &2347 \\  \hline
   Normalized  &1.11   &1.27          &1.02   &0.93  &1.00             &1.00 \\
  \hline
  \end{tabular}
  \label{table:exp_result1}
  }
\end{table}

From the results, MNTU3-multilevel achieves 9\% smaller HPWL and 37\% shorter time than NTUplace3.
These results not only justify the effectiveness of our fast computation scheme of Poisson's equation,
but also the robustness of our analytical solution of Poisson's equation for a different placer.

To further demonstrate the robustness of our solution of Poisson's equation,
we turned off the multilevel clustering in MNTU3-multilevel and named this method MNTU3-flat.
From Table~\ref{table:exp_result1}, MNTU3-flat achieves 11\% smaller HPWL and 27\% shorter runtime than NTUplace3,
and 2\% smaller HPWL than MNTU3-multilevel in 7\% longer runtime.
The results again shows the robustness of our solution of Poisson's equation.

%{\color{red}
%As shown in ePlace \cite{RefEplace}, the suboptimal cell clustering would cause quality loss.
%In order to illustrate the robustness of our method in the multilevel framework and avoid the quality loss as much as possible, we multiply the initial penalty parameter by 0.01 at the beginning of each level.
%According to this operation, the HPWL result of MNTU3(multilevel) can be better than MNTU3(flat).
%During the improvement of results, many iterations are repeated in MNTU3(multilevel).
%Therefore, the runtime is longer.

%}


%%%%%%%%%%%%%%%%%%%%%%%%%%%%%%%%%%%%%%%%%%%%%%%%%%%%%%%%%%%%%%%%%%%%%%%%%%
% Conclusions
%%%%%%%%%%%%%%%%%%%%%%%%%%%%%%%%%%%%%%%%%%%%%%%%%%%%%%%%%%%%%%%%%%%%%%%%%%
\section{Conclusions} \label{sec:conclusion}
%%%%%%%%%%%%%%%%%%%%%%%%%%%%%%%%%%%%%%%%%%%%%%%%%%%%%%%%%%%%%%%%%%%%%%%%%%
% Conclusions
%%%%%%%%%%%%%%%%%%%%%%%%%%%%%%%%%%%%%%%%%%%%%%%%%%%%%%%%%%%%%%%%%%%%%%%%%%

Unlike previous global placement methods that solve Poisson's equation numerically, we have derived an analytical solution and a fast computation scheme of Poisson's equation for distributing circuit blocks effectively and efficiently.
The analytical solution is an infinite series, which converges absolutely.
Nevertheless, calculating a partial sum of the analytical function takes $O(K^2n)$ time.
In order to quickly solve Poisson's equation,
we have divided the placement region into uniform bin grids,
and used the fast Fourier transform to approximately computing the analytical solution.
Using the fast computation scheme of Poisson's equation,
we have developed the new placer Pplace.
Experimental results on the ISPD 2005 and ISPD 2006 benchmarks have shown that Pplace with the fast computation scheme of Poisson's equation is effective, efficient, and robust.
Future work should be extending the methods in this paper to consider the VLSI placement problem with different constraints, e.g., routability. Moreover, it would be interesting using the solution method in this paper to other partial differential equations for EDA and other applications.
With the pervasive applications of Poisson's equation in scientific fields, in particular,
our effective, efficient, and robust computation scheme for its analytical solutions
can provide substantial impacts on these fields.

%to apply the solutions in this paper to other EDA and optimization problems, and
%%%%%%%%%%%%%%%%%%%%%%%%%%%%%%%%%%%%%%%%%%%%%%%%%%%%%%%%%%%%%%%%%%%%%%%%%%
% References
%%%%%%%%%%%%%%%%%%%%%%%%%%%%%%%%%%%%%%%%%%%%%%%%%%%%%%%%%%%%%%%%%%%%%%%%%%

\bibliographystyle{abbrv}
\bibliography{sigproc}

%\bibliography{sigproc}


\end{document}
