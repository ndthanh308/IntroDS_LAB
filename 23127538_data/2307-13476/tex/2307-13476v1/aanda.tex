%                                                                 aa.dem
% AA vers. 9.1, LaTeX class for Astronomy & Astrophysics
% demonstration file
%                                                       (c) EDP Sciences
%-----------------------------------------------------------------------
%
%\documentclass[onecolumn]{aa} % for a paper on 1 column  
%\documentclass[longauth]{aa} % for the long lists of affiliations 
%\documentclass[letter]{aa} % for the letters 
%\documentclass[bibyear]{aa} % if the references are not structured 
%                              according to the author-year natbib style

%
\documentclass{aa}  
%\setcitestyle{authoryear,open={(},close={)}}
%\addbibresource{sample.bib} %Imports bibliography fil

%
\usepackage{graphicx}
%%%%%%%%%%%%%%%%%%%%%%%%%%%%%%%%%%%%%%%%
\usepackage{txfonts,textcomp}
%\usepackage{xcolor}
%\usepackage{amsmath}
\usepackage{nccmath}
\usepackage{hyperref}
%\usepackage{cleveref}
%\usepackage{multirow}
%\usepackage{diagbox}
%\usepackage{siunitx}
%%%%%%%%%%%%%%%%%%%%%%%%%%%%%%%%%%%%%%%%
%\usepackage[options]{hyperref}
% To add links in your PDF file, use the package "hyperref"
% with options according to your LaTeX or PDFLaTeX drivers.

\newcommand*{\new}{\textcolor{cyan}}
\newcommand*{\snew}{\textcolor{orange}}
\newcommand*{\vote}{\textcolor{purple}}
\newcommand*{\gram}{\textcolor{green}}
\newcommand*{\K}{\textcolor{blue}}
%
\begin{document} 


   \title{Themal Sunyaev-Zeldovich measurements and cosmic infrared background leakage mitigation combining upcoming ground-based telescopes}


   \author{M. Charmetant
          \inst{1}
          \and
          J. Erler\inst{1,2}\fnmsep
          }

   \institute{Argelander-Institut f\"ur Astronomie, Universit\"at Bonn,
              Auf dem H\"ugel 71, 53121 Bonn, Germany\\
              \email{mcharmetant@astro.uni-bonn.de}
         \and
             Deutsches Zentrum f\"ur Luft- und Raumfahrt e.V. (DLR) Projekttr\"ager, Joseph-Beuys-Allee 4, 53113 Bonn, Germany\\
             }

   \date{Received: 31 December 2022/Accepted: 5 June 2023}


% \abstract{}{}{}{}{} 
% 5 {} token are mandatory
 
\abstract
    {The Fred Young Submillimeter Telescope (FYST) and the Simons Observatory Large Aperture Telescope (SO\ LAT) will deliver unprecedented high-resolution measurements of microwave sky emissions. Notably, one of those microwave sky emissions, the thermal Sunyaev-Zeldovich (tSZ) signal, is an essential probe for cluster astrophysics and cosmology. However, an obstacle to its measurement is contamination by the cosmic infrared background (CIB), especially at high frequencies.}
  % context heading (optional)
  % {} leave it empty if necessary  
   {Our goal is to assess the detection and purity of tSZ power spectrum measurements from these two telescopes. We demonstrate that FYST's high-frequency coverage helps lower CIB contamination and improves signal detection.}
  % aims heading (mandatory)
   {We simulated the various components of the microwave sky at the frequencies, sensitivities, and beam sizes of the upcoming SO LAT and FYST telescopes using full-sky Hierarchical Equal Area isoLatitude Pixelisation (HEALPix) map templates from the Websky simulations and the Python Sky Model (PySM). We used a map-based internal linear combination (ILC) and a constrained ILC (CILC) to extract the tSZ signal and compute residual noises to assess CIB contamination and signal recovery.}
  % methods heading (mandatory)
   {We find that the CIB's residual noise power spectrum in the ILC-recovered tSZ is lowered by $\sim 35\%$ on average over the scales $\ell \in [500,5000]$ when SO LAT and FYST are combined compared to when SO LAT is used alone. We find that when using CILC to deproject CIB, the combined abilities of SO LAT and FYST offer a large $\ell \in [1800,3500]$ window in which the recovered tSZ power spectrum is not noise dominated.}
  % results heading (mandatory)
   {}
  % conclusions heading (optional), leave it empty if necessary 
   {}

   \keywords{galaxies: clusters: general -  methods: data analysis - cosmic microwave background - cosmology: observations - diffuse radiation
               }
   \titlerunning{Thermal Sunyaev-Zeldovich measurements and cosmic infrared background leakage mitigation}
   \authorrunning{Maude Charmetant} 
   \maketitle 
%
%-------------------------------------------------------------------
\section{Introduction}

The cosmic microwave background (CMB) photons we observe have been influenced by their interaction with the baryonic matter as they travel through the Universe. These interactions modify the CMB photons and are called secondary anisotropies of the CMB. The study of these secondary anisotropies allows for the extraction of information about the Large Scale Structure (LSS) of the Universe as well as about the primary anisotropies, the matter content of the Universe. One of the secondary anisotropies, the thermal Sunyaev-Zeldovich (tSZ) effect, is due to the inverse Compton scatting of the CMB photons by the hot electrons in the intracluster medium (ICM) of galaxy clusters. This effect is a crucial tool for cluster detection and was first used by the South Pole Telescope (SPT) \citep{Staniszewski_2009} and the Atacama Cosmology Telescope (ACT) collaborations \citep{Menanteau_2010} and later by the \textit{Planck} collaboration \citep{Planck_clusters} to construct cluster catalogues. The tSZ effect can be used to measure cluster properties, such as temperature \citep{Jens_2018}. It is also crucial for cosmology, as it has a strong dependence on matter clustering $\sigma_{8}$ and matter density $C_{\ell}^{yy} \propto \sigma _{8}^{8.1}\Omega_{\mathrm{m}}^{3.2}$ \citep{Komatsu_2002,Planck_2016d,Bolliet_2018}. By fitting the tSZ power spectrum extracted from data with models of the signal and its contaminants, assuming a Gaussian likelihood function, and applying a Markov chain Monte Carlo (MCMC) analysis with a fixed mass bias $b$, one can constrain these cosmological parameters \citep{Planck_2014_PS}. This implies that a small improvement in the tSZ power spectra statistical significance or a reduction of the bias due to residual contamination from other signals will lead to a significant improvement in measurements of $\sigma _{8}$ and $\Omega_{\mathrm{m}}$. Therefore, a clean, uncontaminated tSZ map is essential. 

The cosmic infrared background (CIB) is the emission from stars and galaxies absorbed and re-emitted by dust in the infrared domain. It is the main contaminant in tSZ reconstruction and is also a tracer of the matter distribution and the LSS \citep{Hurier_2015}. The CIB contamination, or leakage, originates from the spatial correlation between CIB and tSZ and is also due to the component separation techniques,  such as internal linear combination \citep[ILC;][]{Bennett_2003,Eriksen_2004,Park_2007,Kim_2008,Delabrouille_2009}, not being able to perfectly separate the signal from the many other astrophysical emissions present in the sky. Previous studies on CIB leakage in the \textit{Planck} tSZ map made by \cite{Planck_2016d} showed that the amplitude of the leakage increases when the CIB originates from higher redshift populations. The dependence of the leakage on the redshift of the CIB population comes from the fact that the separation methods of components primarily focus on reducing the main contaminant, which at high frequencies is  Galactic thermal dust, and is thus less efficient at reducing the high-z CIB contaminants. This could lead to a bias in tSZ estimation for high-z clusters, as mentioned in \cite{Vikram_2017}. Another study by \cite{Alonso_2018}  showed that the error bars on the tSZ profile are not much improved by the deprojection (i.e. nulling) of the CIB with more constrained component separation methods such as constrained internal linear combination \citep[CILC;][]{Remazeilles_2011}. Dust contamination can reduce the number of detected galaxy clusters by up to $\sim 9\%$ within z $\in[0.3,0.8]$ \citep{Melin_2018}. Similarly, \citep{Zubeldia_2022} demonstrated that CIB contamination in the tSZ signal can lead to $\sim 20\%$ fewer clusters being detected and proposed a publicly available code to produce CIB-free cluster catalogues. 

%, photometric redshift originating from the PSZ2 that were determined using the \texttt{redMaPPer} algorithm \citep{Planck_clusters}.

In the past three decades, several space missions, such as the COsmic Background Explorer \citep[\textit{COBE;}][]{COBE_1994}, the Wilkinson Microwave Anisotropy Probe  \citep[\textit{WMAP;}][]{Bennett_2003}, and \textit{Planck} \citep{Planck_2014},  have observed the sky at microwave frequencies and extracted full-sky maps of the CMB, tSZ, CIB, and other astrophysical components. Upcoming ground-based experiments, such as those using the Simons Observatory Large Aperture Telescope \citep[SO LAT;][]{SO_2019} and the Fred Young Submillimeter Telescope \citep[FYST;][]{CCAT_2021}, will allow for unprecedented measurements of the tSZ and CIB thanks to their high spatial resolution and high sensitivities. 

In this work, our focus is on quantifying the benefits of combining future experiments, such as SO LAT and FYST to reduce the CIB leakage in ILC-reconstructed tSZ maps. A cleaner tSZ map will allow better constraints on the cosmological parameters $\sigma _{8}$ and $\Omega_{\mathrm{m}}$ to be drawn, more accurate cluster counts and detection, and more accurate measurements of the properties of clusters, such as their temperature. Without mitigation, CIB leakage along the line of sight (LoS) could systematically bias measurements of the tSZ effect and lead to higher tSZ measurements, or it could lower such measurements by filling the characteristic tSZ decrement at frequencies below 217 GHz \citep{Schaan_2021}. Obtaining less contaminated signals by combining microwave experiments is common and will be part of the observation strategies of future experiments. For example, previous studies, such as \cite{SO_2019}, have shown that SO LAT will achieve precise measurement of the tSZ power spectrum when combined with \textit{Planck} data. As \textit{Planck} observations are not contaminated by the atmosphere, this combination allows SO LAT to mitigate such contamination on large scales. Similarly, \cite{Melin_2021} constructed a bigger cluster catalogue by combining \textit{Planck} and the South Pole Telescope (SPT-SZ) data. Our work also encourages the combination of all available data from previous and upcoming CMB experiments.

This paper is structured as follows: In Section 2, we formalise the tSZ and CIB description in our study. In Section 3, we describe the modelling of high-resolution maps of the different microwave sky emissions. In particular, we first explain how we generated Galactic foregrounds, the extragalactic components, and their processing. Then, we detail the generation of noises and the handling of beams and masks. This is followed by a conclusion of the output of the sky modelling. In Section 4, we summarise the methods used to extract the tSZ effect, starting with the ILC and followed by CILC. Next, we explain beam convolution handling and the treatment of full-sky maps, and we provide an overview of the methods used. We present a detailed description of our results and an overview in Section 5. Section 6 contains a summary and discussion of the above-mentioned results. 

\section{Formalism}

\subsection{The thermal Sunyaev-Zeldovich effect}

The thermal tSZ effect, predicted by \cite{SZ_1970,SZ_1972}, is the inverse Compton scattering of CMB photons on the hot electrons present in the ICM. A consequence of this effect is the distortion of the CMB blackbody spectrum, due to the shift of the low frequency (corresponding to low energies) CMB photons to higher frequencies (corresponding to higher energies). The temperature change of the CMB due to the tSZ effect can be expressed as: 

\begin{ceqn}
\begin{align}
\label{eqn:1}
    \frac{\Delta T_{\mathrm{CMB}}}{T_{\mathrm{CMB}}} = yf(x_{\nu}) ,
\end{align}
\end{ceqn}

where $T_{\mathrm{CMB}}=2.72548\pm 0.00057$ K \citep{Fixsen_2009} is the temperature of the CMB and $y$ is the dimensionless Compton-y parameter measuring the electron pressure integrated over the LoS proper distance $l$, 

\begin{ceqn}
\begin{align}
    y = \frac{\sigma _{\mathrm{T}}}{m_{\mathrm{e}}c^{2}} \int  P_{\mathrm{e}}(l) \mathrm{d}l,
\end{align}
\end{ceqn}

where $\sigma _{\mathrm{T}}$ is the Thomson cross-section, $m_{\mathrm{e}}$ the electron mass, and $P_{\mathrm{e}}(l)=n_{\mathrm{e}}(l)k_{\mathrm{B}}T_{\mathrm{e}}(l)$ the electron pressure composed of the electron density $n_{\mathrm{e}}$, the Boltzmann constant $k_{\mathrm{B}}$, and the electron temperature $T_{\mathrm{e}}$. In Eq. \eqref{eqn:1}, $f(x_{\nu})$ is the spectral shape of the tSZ \citep{SZ_1972,SZ_1980}: 

\begin{ceqn}
\begin{align}
    f(x_{\nu}) = x_{\nu}\coth \left( \frac{x_{\nu}}{2} \right) -4 = \left ( x_{\nu} \frac{e^{x_{\nu}}+1}{e^{x_{\nu}}-1} - 4 \right ),
\label{eqn:tsz_sed}
\end{align}
\end{ceqn}

where $x_{\nu}=h\nu /k_{\mathrm{B}}T_{\mathrm{CMB}}$, $h$ being the Planck constant and $\nu$ the frequency. The frequency spectrum of the tSZ is characterised by a decrement in the intensity change for frequencies lower than $\nu \sim 217$ GHz and an increment at higher frequencies. To express the frequency variations induced by the tSZ effect in terms of an intensity variation, we use the conversion factor to pass from units of thermodynamic temperature, ($\mathrm{K}_{\mathrm{CMB}}$) \citep{Planck_2014_KCMB} in Eq. \eqref{eqn:1}, to intensity units (Jy/sr),

\begin{ceqn}
\begin{align}
 \Delta I_{\nu} =  \frac{\partial \mathrm{B}_{\nu}(\mathrm{T})}{\partial \mathrm{T}} \Delta \mathrm{T} = \frac{I_{0}}{\mathrm{T}_{\mathrm{CMB}}}h(x_{\nu})\Delta \mathrm{T},
\label{eqn:conv_f}
\end{align}
\end{ceqn}

where $\mathrm{B}_{\nu}(T)$ is the Planck blackbody spectrum at a fixed frequency $\nu$, $I_{0}=2(k_{\mathrm{B}}T_{\mathrm{CMB}})^{3}/(hc)^{2}\approx 270$ $\mathrm{MJysr^{-1}}$, $c$ is the speed of light, and 

\begin{ceqn}
\begin{align}
    h(x_{\nu}) = \frac{x_{\nu}^{4}e^{x_{\nu}}}{(e^{x_{\nu}}-1)^{2}}.
\label{eqn:conv}
\end{align}
\end{ceqn}

Therefore, the tSZ intensity change is given by

\begin{ceqn}
\begin{align}
    \frac{\Delta  I_{\nu}}{I_{0}} = yh(x_{\nu})f(x_{\nu}).
\label{eqn:tsz_MJy}
\end{align}
\end{ceqn}
 

More details on the Sunyaev-Zeldovich (SZ) effects can be found in the review by \cite{Mroczkowski_2019}. In this study, we place ourselves in a simplistic approach and do not take into account the relativistic correction of the tSZ effect \citep{Wright_1979,Itoh_1998,Pointecouteau_1998,Chluba_2012}, even though \textit{Planck} data has shown that this leads to an underestimation of the amplitude of its power spectrum $C_ {\ell}^{yy}$ \citep{Remazeilles_2018}.

\subsection{The cosmic infrared background}

The CIB is the extragalactic integrated emission produced by all stars and galaxies during their formation and throughout their life. The CIB frequency spectrum peaks in the infrared regime because the majority of sources are far and their light is redshifted by the expansion of the Universe but also because some of this light is absorbed by dust and re-emitted into the infrared regime. This existence of the CIB was predicted by \cite{Partridge_1967} and first detected by  \cite{Puget_1996,Fixsen_1998,Hauser_1998}. Contrary to the tSZ, one frequency spectrum is not sufficient to fully model it. The CIB spectral energy distribution (SED) depends on the redshift of sources.  It can be approximated as a modified blackbody spectrum: 

\begin{ceqn}
\begin{align}
 I_\nu=A_{\mathrm{dust}}\nu ^{\beta }\mathrm{B}_{\nu}(T_{\mathrm{dust}}),
\end{align}
\end{ceqn}

where $A_{\mathrm{dust}}$ is the amplitude of the dust emission, $\beta$ is the dust emissivity spectral index, and $T_{\mathrm{dust}}$ is the temperature of the dust grains. The CIB is a tracer of the entire history of star formation in the Universe. One characteristic of the CIB of particular importance for our study is its non-zero correlation with the tSZ signal. Indeed, some of the unresolved dusty star-forming galaxies constituting the CIB are found in galaxy clusters. Therefore, the spatial correlation between tSZ and CIB is non-zero, and it was measured in \cite{Reichardt_2012,Dunkley_2013,George_2015,Reichardt_2021,Choi_2020}. 
  

\section{Modelling the microwave sky} \label{des_skymodel}

We set up a pipeline called Skymodel to create mock high-resolution maps of the microwave sky that would be observed in future experiments. Our pipeline is based on Hierarchical Equal Area isoLatitude Pixelisation (HEALPix) template maps of the microwave sky from existing simulations as well as the most recent observations of the microwave sky. The microwave sky components are usually separated into two categories: the emissions coming from the Galaxy, called Galactic foregrounds, and the extragalactic components. Our Skymodel deals differently with these two categories, using an existing Python package for the Galactic foregrounds and HEALPix maps from mock simulations for the extragalactic ones.

\subsection{Galactic foregrounds}

Galactic foregrounds are the various emissions originating from the Milky Way and are thus in the foreground of extragalactic emissions. We considered the following Galactic foregrounds: Galactic synchrotron; free-free emission; anomalous microwave emission (AME), which is supposedly produced by spinning dust \citep{AME_2018}; and thermal emission from Galactic dust grains.

All of the above-mentioned components were generated using the Python Sky Model (PySM) \citep{Ben_2017}. The PySM\footnote{\url{https://github.com/bthorne93/PySM_public}} package is based on Galactic foreground template maps obtained from \cite{Planck_2016_GF} that were extracted using the {\small COMMANDER} algorithm \citep{Eriksen_2004,Eriksen_2008,Planck_2016_COMMANDER,Planck_2020} on \textit{Planck} 2015 raw data, data from \textit{WMAP,} and $408 \, \mathrm{MHz}$ all-sky survey by \cite{Haslam_1981,Haslam_1982}, reprocessed by \cite{Remazeilles_2015}. The original data maps were at a HEALPix \citep{Healpy_2005} pixel resolution of $\mathrm{pix}_{\mathrm{size}}=13.74$' ($N_\mathrm{side} = 256$). The resolution of a HEALPix pixel is defined as the square root of the pixel area. The PySM upgrades them to $6.9$' ($N_\mathrm{side} = 512$) and adds small-scale fluctuations generated randomly by extrapolating the power spectra of those maps \citep{Miville-Deschenes_2007}. We generated each foreground component at the desired frequency using the basic spectral model of PySM (i.e. "s1", "f1", "d1", and "a1"). These maps were then upgraded to a higher $N_\mathrm{side}=4096$ and smoothed with a  $10\arcmin$ full width at half maximum (FWHM) Gaussian to mitigate pixelation artefacts (see Fig. \ref{fig:PySM}).

% Figure environment removed

\subsection{The extragalactic components} \label{sec_extragal}

The extragalactic components include the CMB, tSZ, CIB, and the kinematic Sunyaev-Zeldovich (kSZ) effect. These components were generated at various frequencies using template maps provided by the Websky extragalactic CMB mocks \citep{CITA_2020}, which are HEALPix full-sky maps of the extragalactic sky at $N_\mathrm{side} = 4096$ ($\mathrm{pix}_{\mathrm{size}}=0.86$'). The large-scale structure of the Websky simulations was obtained through a peak-patch and 2nd-order Lagrangian perturbation theory (2LPT) simulation with $12288^3$ particles and a box size of $15.4 \, h^{-1} \, \mathrm{Gpc}$. The authors assumed a flat $\Lambda$ cold dark matter ($\Lambda$CDM) cosmology with $h=0.68$, $\Omega_\mathrm{cdm}=0.261$, $\Omega_{b}=0.049$, $\Omega_{\Lambda}=0.69$, $n_{s}=0.965$, and $\sigma _{8}=0.81$. The simulations can be downloaded from their website.\footnote{\url{https://mocks.cita.utoronto.ca/index.php/Websky_Extragalactic_CMB_Mocks}} An overview of the provided template maps, their frequencies, and units can be seen in Table \ref{tab:skymodel}. 

To generate the tSZ map, \cite{CITA_2020} projected the pressure profiles from hydrodynamical simulations \citep{Battaglia_2012} onto the mass-Peak Patch halo catalogue. They did not consider field contributions, and the minimum halo mass considered to compute their pressure profile was $\sim 1\times 10^{13}M_{\odot}$. Their tSZ power spectrum is in good agreement with the \textit{Planck} data \citep{CITA_2020}. Websky provides all-sky, dimensionless Compton-$y$ maps (linked to the tSZ effect through Eq. \eqref{eqn:1} or \eqref{eqn:tsz_MJy}), which can be multiplied by the tSZ spectrum in Eq. \eqref{eqn:tsz_sed} and \eqref{eqn:conv} to obtain frequency maps of the tSZ. The tSZ signal was simulated by Websky in its non-relativistic limit.

The CIB was generated using the halo occupation distribution model proposed by \cite{Shang_2012} and with the parameters given in \cite{Viero_2013} to fit the CIB power spectrum with the \textit{Herschel} observations. Within this model, the rest-frame SED of a source is a modified blackbody at low frequencies and a power law at high frequencies. The CIB's SED depends on the source redshift, halo mass, and frequency at which it is observed. The Websky halo catalogue is populated with central and satellite galaxies. The field contribution was not considered. The Websky simulations generated CIB maps at the \textit{Planck} High-Frequency Instrument (HFI) frequencies (see Table \ref{tab:skymodel}), and in particular, the $545$ GHz CIB power spectrum was normalised to fit the \textit{Planck} $545$ GHz CIB power spectrum \citep{Planck_2014} at $\ell = 500$. 

Because of the redshift dependence of the CIB distribution, a decorrelation was expected between the CIB at different frequency channels. The decorrelation was determined in the Websky simulations \citep{CITA_2020} by computing the Pearson product-moment correlation coefficients on the power spectrum of the CIB maps, evaluated at the \textit{Planck} HFI frequency channels. The correlation matrix between the different CIB maps provided by the Websky simulations is illustrated in Fig. \ref{fig:CITA_corr}. The Websky CIB decorrelation values are in agreement with \cite{Planck_2014} and \cite{Lenz_2019}. 

% Figure environment removed

In this work, we extrapolated the CIB model to arbitrary frequencies by fitting each pixel of the existing Websky CIB maps with the following modified blackbody:

\begin{ceqn}
\begin{align}
I_{\nu} = A_{\mathrm{CIB},\nu _{0}} \, \left(\frac{\nu}{\nu_0}\right)^{\beta+3} \frac{e^{\frac{h\nu_0}{k_\mathrm{B}T_{\mathrm{dust}}}}-1}{e^{\frac{h\nu}{k_\mathrm{B}T_{\mathrm{dust}}}}-1},
\label{eqn:mbb_sed}
\end{align}
\end{ceqn}

where $\nu_0 = 353$ GHz. This was done using the Scipy "curve\_fit()" function to derive three all-sky parameter value maps for the variables $\mathrm{A}_{\mathrm{CIB}}$, $T_{\mathrm{dust}}$, and $\beta$. These full-sky maps were injected in Eq. \eqref{eqn:mbb_sed} to produce Websky-like CIB intensity maps at new frequencies.


In the first approximation, we assumed this exact modified black body across frequencies, which is not fully realistic since the CIB is composed of different sources at different redshifts, and their SED is not the same. The Websky simulation includes cross-correlation between the tSZ and CIB, which is in agreement with the \textit{Planck} data results (see Fig. 11 of \cite{CITA_2020}).

For the CMB component, the Websky simulations provided us with two files containing random realisations of the primary T/Q/U $a_{\ell m}$ coefficients, one being unlensed and one being lensed. The lensed file was obtained using the unlensed $a_{\ell m}$ file and the CMB lensing convergence from $0<z<1100$ in the Born approximation. From those $a_{\ell m}$, we could reconstruct a lensed and an unlensed CMB map in $\mu \mathrm{K}_{\mathrm{CMB}}$. Using those maps and the conversion factor (Eq. \eqref{eqn:conv_f}), we could obtain a CMB map at any desired frequency. In the first approximation, we used the unlensed CMB. For the kSZ component, the situation is analogous to the tSZ one. The Websky templates provided frequency-independent maps of the kSZ, which can be seen as a ‘$y_{kSZ}$-map’. We multiplied the maps by the conversion factor in Eq. \eqref{eqn:conv_f} to get a kSZ map at any frequency. The Websky simulations do currently not provide radio point source (RPS) templates, but the Websky team will add them in a future release. This extragalactic component is therefore not accounted for in our study. The different extragalactic full-sky map templates were all converted to the same unit, that is, either temperature units ($\mathrm{K}_{\mathrm{CMB}}$) or intensity units (Jy/sr) units, using Eq. \eqref{eqn:conv_f}.

\begin{table*}[ht]
\begin{center}
\caption{Summary of the base extragalactic templates of the Websky simulations. The templates are generally composed of either one full-sky map or a set of full-sky maps at various frequencies. For the CMB component, the Websky simulations provided a set of $a_{\ell m}$ that are either lensed (L) or unlensed (UL).}
\begin{tabular}{c|c|c|c}
\centering Extragalactic templates from Websky & Type of data & Frequencies (GHz) &  Units \\
\hline
\hline
\rule{0pt}{10pt} \centering Cosmic infrared background (CIB) & HEALPix map &  \{143,217,353,545,857\}
& [MJy/sr] \\
\hline
\rule{0pt}{10pt} \centering  Radio point sources (RPSs) & - & - & - \\ 
\hline
\rule{0pt}{10pt} \centering Cosmic microwave background (CMB) & $\{ a_{\ell m} \}_{L}$ or $\{ a_{\ell m} \}_{UL}$ & - & [$\mu \mathrm{K}_{\mathrm{CMB}}$] 
 \\
\hline
\rule{0pt}{10pt} \centering  thermal Sunyaev-Zeldovich (tSZ) & HEALPix map & Compton-y & [$\emptyset$] \\ 
\hline
\rule{0pt}{10pt} \centering kinematic Sunyaev-Zeldovich (kSZ) & HEALPix map & - & [$\mu \mathrm{K}_{\mathrm{CMB}}$] \\
\label{tab:skymodel}
\end{tabular}
\end{center}
\end{table*}

\subsection{Future experiments}

We simulated the microwave sky as will be seen by two future ground-based telescopes.
One of these telescopes is the FYST of the Cerro Chajnantor Atacama Telescope prime (CCAT-prime) collaboration \citep{CCAT_2021}. It is an upcoming 6 m diameter telescope that will be located at 5600 m on the mountain of Cerro Chajnantor in Chile. Due to its very low precipitable water vapour, this exceptional site will provide one of the best atmospheric transmissions for ground-based microwave observations to date \citep{Bustos_2014}. The crossed-Dragone optical design proposed by \citep{Niemack_2016} will deliver a high-throughput, wide-FoV telescope capable of rapidly scanning large areas of the sky. The first light is expected in 2024.

The second telescope of interest will be part of the Simons Observatory \citep[SO;][]{SO_2019}. The SO will be composed of three refracting 0.4 m small aperture telescopes (SATs) and one cross-Dragone 6 m large aperture telescope (LAT). The four instruments will be located at 5300 m on Cerro Toco in the Atacama Desert in Chile. 

We studied what benefits may result from the combination of FYST with the SO LAT. To do so, we used PySM and the Websky mock sky to simulate the microwave sky at frequencies and sensitivities, meaning the smallest signal a telescope can detect above the random background noise, as well as the beam sizes of future observations conducted with these telescopes, see Table \ref{tab:fsb}. 


\begin{table*}[h!]
    \centering
     \caption{Frequencies, beam sizes in FWHM, and baseline instrumental sensitivities for SO LAT and FYST.}
    \begin{tabular}{cc|cc|cc}
    & & \multicolumn{2}{c|}{SO LAT} & \multicolumn{2}{c}{FYST} \\
    Frequency & FWHM & \multicolumn{2}{c|}{Sensitivity} & \multicolumn{2}{c}{Sensitivity} \\
    (GHz) & (arcmin) & ($\mu K_{CMB}$-arcmin) & ($\mu K_{RJ}$-arcmin) & ($\mu K_{CMB}$-arcmin) & ($\mu K_{RJ}$-arcmin) \\
    \hline
    \hline
      93 & 2.2 & 8 & 6.42 & -- & --  \\
      145 & 1.4 & 10 & 5.96 & -- & -- \\
      220 & 1.0 &  -- &  -- & 15 &  4.86 \\
     225 & 1.0 & 22 & 6.80 & -- & -- \\
     279 & 0.9 & 54 & 9.68 & -- & -- \\
      280 & 0.9 & -- & -- & 27 & 4.79 \\
      350 & 0.6 & -- & -- & 105 & 8.38 \\
     405 & 0.5 & -- & -- & 372 & 15 \\
     860 & 0.2 & -- & -- & $5.7\times 10 ^{5}$ & 33.93 \\
    \end{tabular}
    \label{tab:fsb}
\end{table*}



\subsection{Noises, masks, and beams}

Our Skymodel pipeline includes two types of noise: white noise, to reproduce the instrumental noise, and atmospheric red noise, which is present for ground-based experiments. The noise implementation is based on the SO Collaboration \citep{SO_2019} noise curves forecasted and adapted for CCAT-prime by \citep{Choi_2020_Noise}:

\begin{ceqn}
\begin{align}
    N_\ell = N_\mathrm{red} \left( \frac{\ell}{\ell_\mathrm{knee}} \right)^{\alpha_{\mathrm{knee}}} + N_\mathrm{white},
\label{eqn:red_noise}
\end{align}
\end{ceqn}

where $\ell_\mathrm{knee} = 1000$ and $\alpha_\mathrm{knee} = -3.5$. The values for $N_\mathrm{red}$ and $N_\mathrm{white}$ for the SO LAT and FYST are included in the Skymodel. For this reason, the atmospheric noise model can only be computed at valid SO LAT frequencies or FYST frequencies (see Table \ref{tab:fsb}) and takes the form of a series of power spectra, as defined by Eq. (\ref{eqn:red_noise}) and shown in Fig. \ref{fig:red_noise}. Using those power spectra, the Skymodel generates full-sky map realisations of the atmospheric noise or atmospheric and instrumental noise. This is done through the HEALPy function `synfast()', which generates map realisations from the power spectrum while assuming the field to be Gaussian. This was done for each frequency band. In reality, neighbouring frequency bands will observe similar regions of the sky at the same time. Thus, the atmospheric noise is partially correlated. We adopted a $90$\% correlation between each pair of neighbouring frequency channels as suggested in \cite{SO_2019}. Fixing the `synfast()' seed allowed for the production of a $100$\% correlated map for two neighbouring channels. Another seed and a correction factor can be applied to the power spectrum to produce a down-weighted correlated map and an uncorrelated map of one of the two channels. This new map and the fully correlated map can be linearly combined to produce a $90$\% correlated map using

\begin{ceqn}
\begin{align}  
    M^{90\%}(\nu) = h(0.9^{2}C_{\ell}^{n,100\%}(\nu))+h((1-0.9)^{2}C_{\ell}^{n,0\%}(\nu)),
\label{eqn:corr_noise}
\end{align}
\end{ceqn}

where $M$ is the final map at the frequency $\nu$, which will be $90$\% correlated with its neighbouring channel. The HEALPy synfast() function is represented by $h$, and $C_{\ell}^{n}$ is the atmospheric noise curve, which is $100$\% correlated when produced from the same seed as its neighbouring channel and  $0$\% when produced from another seed. 

The frequency bands are broad and extended over some frequency range, and their transmission is not perfect but varies over their extent. In the first approximation, we considered the frequency bands to be delta functions with perfect transmission.

% Figure environment removed

In addition to the maps of astrophysical emission and noise, the Skymodel uses survey masks. The masks are stored at $N_\mathrm{side} = 256$ and were upgraded to $N_\mathrm{side}=4096$ in order to correspond to the Websky map resolution. In this study, we used a mask that excludes the Galactic plane with a sky fraction of $f_{\mathrm{sky}}=0.6$ (see Fig. \ref{fig:masks}). This mask was derived in \cite{Jens_2018} by masking the 40\% brightest pixels of a \textit{Planck} Galactic emission map. This is an initial approximation, as in reality SO LAT and FYST will observe a smaller fraction of the sky.


% Figure environment removed

The Skymodel pipeline also simulates the effect of telescope beams using the Gaussian FWHMs given in Table \ref{tab:fsb}. The product of the Skymodel is a full-sky HEALPix map containing all the Galactic foreground components, the extragalactic components, instrumental white noise, and atmospheric red noise at a given frequency and assuming a given beam size (See \ref{AnnexeSky}).

\section{Method to extract thermal Sunyaev-Zeldovich map}

To study the power spectrum measurement of the tSZ effect for the SO LAT and FYST telescopes, a tSZ map needs to be extracted from the simulated microwave sky. The sky is composed of various Galactic and extragalactic emissions that are considered to be contaminants of our signal of interest. Component separation methods permit the recovery of a signal from the multi-component microwave sky. 
Internal linear combination (ILC) is a  multi-frequency component separation technique that was first used by \cite{Bennett_2003} to extract CMB maps from \textit{WMAP} data and later refined by \cite{Eriksen_2004}. 
Multiple flavours of ILC exist. In their paper, \cite{Tegmark_2003} suggested applying it in Fourier space rather than in map space. Needlet ILC \citep[NILC;][]{Delabrouille_2009} processes the maps in both spatial and harmonic space by proposing to filter each multi-frequency map by a set of $\ell$-dependent window functions called needlets in order to apply multiple ILCs on different multipole intervals. This makes the ILC weights $\ell$-dependent, allowing them to take advantage of the scale-dependent distribution of astrophysical components to minimise the noise and better extract the signal of interest. The Modified ILC Algorithm \citep[MILCA;][]{Hurier_2013} differentiates itself by allowing multiple constraints on the Astrophysical emissions in order to reduce their contamination. Scale-discretised directional wavelet ILC \citep[SILC;][]{Rogers_2016} filters the map with $\ell$ windows that are direction dependent. 

In this work, we used ILC and CILC methods in a map-based space. It is also possible to use these methods directly in Fourier space, which is computationally faster but does not allow for spatial separation of signals. Fourier space ILC/CILC only has information on the power spectra of the component in $\ell$-space and no spatial information on the localisation of the signal over the full-sky map.

\subsection{Internal linear combination}

Internal linear combination is widely used in tSZ studies \citep{Bennett_2003,Fornazier_2022}. Its main advantage is that it is a so-called "blind" method because only the SED of the component of interest must be known, and no information on the contaminants is needed.
The ILC supposes that (1) the maps are composed of a linear combination of astrophysical components and noise, and
(2) the astrophysical components are uncorrelated. 
%\end{enumerate}

A set of $N_{\mathrm{obs}}$ observed map $m_{i}(p)$ at a frequency $i$ and a pixel $p$ can be expressed as

\begin{ceqn}
\begin{align}
m_{i}(p) = a_{i}s(p) + n_{i}(p),
\label{eqn:ILC_ini}
\end{align}
\end{ceqn}

where $a_{i}$ is the amplitude of the frequency spectrum of the signal of interest $s$ at each frequency $i$. The noise at a given frequency $i$ and pixel $p$ is represented as $n_{i}(p)$. 

%\begin{ceqn}
%\begin{align}
%M(p) = As(p) + N(p),
%\label{eqn:ILC_ini_vec}
%\end{align}
%\end{ceqn}

%where the vector containing all the maps at each frequency $i$, $M=[m_{i_{0}}(p),...,m_{i_{N_{\mathrm{obs}}}}(p)]$, is the linear combination of $A=[a_{i_{0}},...,a_{i_{N_{\mathrm{obs}}}}]$ which is known as the mixing vector and for which $a_{i}$ contain the amplitude of the frequency spectrum of the signal of interest $s$ at each frequency $i$. $N=[n_{i_{0}}(p),...,n_{i_{N_{\mathrm{obs}}}}(p)]$ is the vector containing all the other astrophysical components that are not of interest and the instrumental noise at each frequency $i$. 

To retrieve the signal of interest $s$, an estimator $\hat{s}_{\mathrm{ILC}}$ is defined as

\begin{ceqn}
\begin{align}
\hat{s}_{\mathrm{ILC}}(p)=\sum_{i}\omega_{i}m_{i}(p),
\label{eqn:ILC_est}
\end{align}
\end{ceqn}

where $\omega_{i}$ is the ILC weight at the frequency $i$. The ILC weights are defined such that they follow two conditions. One of them is that in order to obtain an unbiased estimate, the ILC weights must have a unit response to the frequency spectrum of the signal of interest $s$ characterised by $a_{i}$ at each frequency $i$ (see Eq. \ref{eqn:ILC_cond1}).
 
\begin{ceqn}
\begin{align}
\sum_{i}\omega_{i}a_{i}=1. 
\label{eqn:ILC_cond1}
\end{align}
\end{ceqn}

The other condition is that the minimisation of the presence of the unwanted astrophysical contaminants and the instrumental noise in the reconstructed map of the signal $s$ is achieved through the condition of minimum variance of the reconstructed map, VAR($\hat{s}_{\mathrm{ILC}}(p)$). As shown in \cite{Eriksen_2004}, because the components are uncorrelated, we have VAR($\hat{s}_{\mathrm{ILC}}(p)$)$=\sum_{k}\sum_{j}\hat{c_{jk}}\omega_{j}\omega_{k}$, where $\hat{c_{jk}}$ is the value of the covariance of the observed maps between the frequencies $j$ and $k$. Thus, the minimum variance condition is given by 

\begin{ceqn}
\begin{align}
\frac{\partial}{\partial \omega_{i}} \sum_{k}\sum_{j}\hat{c_{jk}}\omega_{j}\omega_{k}=0.
\label{eqn:ILC_cond2}
\end{align}
\end{ceqn}
    
The minimum variance under the constraint of Eq. \eqref{eqn:ILC_cond2} can be achieved using the Lagrange multiplier method to solve the above equation. The solution gives the following ILC weights: 

\begin{ceqn}
\begin{align}
\Vec{W}^{T} = \frac{\Vec{A}^{T}\Vec{\hat{C}}^{-1}}{\Vec{A}^{T}\Vec{\hat{C}}^{-1}\Vec{A}},
\label{eqn:ILC_w}
\end{align}
\end{ceqn}
where $\Vec{W}$ is the vector containing the ILC weights, and $\Vec{A}$ is the vector containing the frequency spectrum information of the signal of interest and is called `the mixing vector'. The covariance matrix of the set of maps is represented by $\Vec{\hat{C}}$. The map containing our signal of interest is given by 


\begin{ceqn}
\begin{align}
\hat{s}_{ILC}(p)=\sum _{i} \omega _{i}^{T}m_{i}(p).
\label{eqn:ILC_f}
\end{align}
\end{ceqn}
From that, we can define the residual noise ($r_{N}$), that is, the noise left over by the ILC in the reconstructed map. This noise can be defined as the difference between the ILC-recovered map and the true signal : 

\begin{ceqn}
\begin{align}
r_{N} = \hat{s}_{ILC}(p) - \sum _{i} \omega _{i}^{T} a_{i}s(p) = \sum _{i} \omega _{i}^{T} n_{i}(p).
\label{eqn:ILC_residual}
\end{align}
\end{ceqn}

The contaminants are not absent from the recovered signal map. Moreover, the low variance contaminants might not be reduced in the resulting map. This becomes a problem when assumption two of the ILC is not fully verified and our target signal is spatially correlated with another emission, such as for the tSZ and CIB. This is the case between CIB and tSZ. The remaining CIB signal in the map can bias the ILC estimate of the tSZ. One way to reduce this effect is through CILC.

\subsection{Constrained internal linear combination}

The CILC method was first introduced by \cite{Remazeilles_2011}. It adds one extra constraint to the ILC approach. If the spectral shape of one of the contaminants is known, it can be taken into account in the modelling such that instead of Eq. \eqref{eqn:ILC_ini}, it is supposed that  

\begin{ceqn}
\begin{align}
m_{i}(p) = a_{i}s(p) + b_{i}q(p) + n_{i}(p),
\label{eqn:CILC_ini_vec}
\end{align}
\end{ceqn}
where $s(p)$ is the signal of interest and $q(p)$ is the contaminant we want to suppress. Respectively, $a_{i}$ and $b_{i}$ are the intensity of the frequency spectrum evaluated at a frequency $i$ of the signal and contaminant. All the remaining astrophysical contaminants and the instrumental noise at a given frequency $i$ and pixel $p$ are contained by  $n_{i}(p)$. The CILC weights are defined such that 

\begin{ceqn}
\begin{align}
\sum_{i}\omega_{i}a_{i} = 1 \ \mathrm{and} \ \sum_{i}\omega_{i}b_{i}= 0.
\label{eqn:CILC_cond1}
\end{align}
\end{ceqn}

The CILC weights have a unit response to the SED of the signal of interest $s$ and a null response to the SED of the astrophysical contaminant $q$ we want to suppress so that this contaminant is absent from the reconstructed signal. 

The condition of minimum variance (see Eq. \eqref{eqn:ILC_cond2}) stays the same, and only the expression of the weights is changed to take into account the two conditions imposed by the CILC,

\begin{ceqn}
\begin{align}
\Vec{W}^{T} = \frac{(\Vec{B}^{T}\Vec{\hat{C}}^{-1}\Vec{B})\Vec{A}^{T}\Vec{\hat{C}}^{-1}-(\Vec{A}^{T}\Vec{\hat{C}}^{-1}\Vec{B})\Vec{B}^{T}\Vec{\hat{C}}^{-1}}{(\Vec{A}^{T}\Vec{\hat{C}}^{-1}\Vec{A})(\Vec{B}^{T}\Vec{\hat{C}}^{-1}\Vec{B})-(\Vec{A}^{T}\Vec{\hat{C}}^{-1}\Vec{B})^{2}},
\label{eqn:CILC_w}
\end{align}
\end{ceqn}

where $\Vec{W}$ is the vector containing the CILC weights, $\Vec{B}$ is the mixing vector of the contaminant, $\Vec{A}$ is the mixing vector of the signal, and $\Vec{\hat{C}}$ is the covariance matrix of the set of maps. Those weights are then applied to the set of multi-frequency maps using Eq. \eqref{eqn:ILC_f} to recover a map containing the signal of interest free from the nulled contaminant and with minimum contamination from other astrophysical signals and instrumental noise. The CILC method can be generalised to null more than one contaminant if their SED is known. However, these additional constraints reduce the degree of freedom available for the condition of minimum variance to be verified \citep{Hurier_2013}, thus generating increased noise for the CILC compared to the ILC. Similar to the ILC, the residual noise can be defined as the difference between the CILC-recovered map and the signal of interest (see Eq. \ref{eqn:ILC_residual}), except that the weights are given by Eq. \eqref{eqn:CILC_w}.

\subsection{Beam convolution}

The set of multi-frequency maps was smoothed with a Gaussian kernel to the lowest frequency channel resolution so that they could be linearly combined by the ILC. No meaningful physical information is available on scales smaller than the resolution of each map. When linearly combining the maps, the upper limit of meaningful physical information was therefore given by the lowest resolution map.  

More sophisticated component separation methods, such as NILC, allows the original resolution of each frequency map to be kept. Indeed, the NILC weights are a function of the multipole $\ell$ for each frequency map and can therefore be set to zero when the scale exceeds the resolution of the map. Setting the weights to 0 above a given $\ell$, allows the lower-resolution maps to contribute to the signal reconstruction only up to their beam-limited scale, leaving the higher-resolution maps to be the contributors of signal reconstruction on smaller scales.

We note that when simulating the microwave sky, only the astrophysical components and atmospheric noise are limited by the instrument beam. Instrumental noise is not beam limited. The inclusion of the beam was done using HEALPy to smooth the maps $a_{\ell,m}$ in the multipole space and multiply them with a $\ell$-dependent Gaussian kernel. Our set of multi-frequency maps can be expressed as

\begin{ceqn}
\begin{align}
a_{\ell,m}(m_{i}^{s}) = a_{i}{b_{i}(\ell)}a_{\ell,m}(s) + {b_{i}(\ell)}a_{\ell,m}(c_{i}) + a_{\ell,m}(n_{i}),
\label{eqn:ILC+beam}
\end{align}
\end{ceqn}
where $a_{\ell,m}(m_{i}^{s})$ are the $a_{\ell,m}$ coefficient of the smoothed map at the frequency $i$. The Gaussian beam in the multipole space at the frequency $i$ is indicated as $b_{i}(l).$  Next, $a_{\ell,m}(s)$ is the $a_{\ell,m}$ decomposition of component of interest map, and $a_{\ell,m}(c_{i})$ represents all astrophysical contaminants and the atmospheric noise at a frequency $i$. Finally, $a_{\ell,m}(n_{i})$ is the instrumental noise $a_{\ell,m}$ at the frequency $i$. In general, to apply an ILC, all maps have to be smoothed down to the lowest resolution. This is necessary to avoid adding information from other higher resolution, higher frequencies maps that are smaller than the beam size of some of the maps into the recovered map. The smoothing of the maps to the lowest map resolution was done using HEALPy, and this process can be formalised as

\begin{ceqn}
\begin{align}
a_{\ell,m}(m_{i}^{*}) = a_{i}b_{\mathrm{f}}(\ell)a_{\ell,m}(s) + b_{\mathrm{f}}(\ell)a_{\ell,m}(c) +\frac{b_{\mathrm{f}}(\ell)}{b_{i}(\ell)}a_{\ell,m}(n_{i}),
\label{eqn:ILC+beam2}
\end{align}
\end{ceqn}
where $a_{\ell,m}(m_{i}^{*})=(b^{\mathrm{f}}(\ell)/b_{i}(\ell))a_{\ell,m}(m_{i}^{s})$ are the $a_{\ell,m}$ coefficient of the map at a frequency $i$ smoothed down to the lowest resolution. The Gaussian kernel at this desired `lowest' resolution is indicated at  $b^{\mathrm{f}}$. The consequence of the necessary smoothing to the lowest resolution is an increase of the instrumental white noise contribution in each frequency map by a factor $b_{\mathrm{f}}(\ell)/b_{i}(\ell)$ because for all the frequency channels, except the lowest one, we have $b_{\mathrm{f}}(\ell) \textgreater b_{i}(\ell) $.  

\subsection{Full-sky treatment}

The microwave full-sky maps are composed of many different astrophysical signals. Some of those signals are spatially localised. The Galactic foregrounds are mainly localised around the Galactic plane. When applying the map-based ILC to a full-sky map, instead of applying it straightforwardly to the multi-frequency set of maps, thus getting only one set of weights for the full sky and equally treating regions that are strongly contaminated by Galactic foregrounds and regions that are less contaminated, we tessellated the sky into different zones and applied an ILC on each of these zones separately. This resulted in a set of different weights for each region so that they are always optimised for the level of contamination present in their zone. This has been done previously in various studies \citep{Tegmark_2003}. The most common tessellation approach is to separate the sky into zones of Galactic contamination. 

To tessellate the sky, we used the native nested HEALPix tessellation (see Fig. \ref{fig:masks}). This nested scheme tessellates the sky originally into 12 equal-area fields. The sides of each of those fields can be further divided to create more equal-area cells of smaller sizes. Using HEALPix nested scheme to create sub-fields \citep{Healpy_2005}, we chose to tessellate the sky using $N_\mathrm{side} = 4$, which gives $N_\mathrm{pix} = 12\times N_\mathrm{side}^{2}=192$ fields, each covering an area of $\approx 215 \mathrm{deg}^{2}$. In general, tessellation can create border effects along the limits of the fields. In \cite{Eriksen_2004}, the border effects were attenuated by smoothing the ILC weights. In this work, we do not correct the border effects. 

The Galactic plane was masked (see Fig. \ref{fig:masks}) with a  $f_ {sky} \approx 0.6$ Galactic mask map to reduce foreground contamination and better extract the tSZ effect. To compute power spectra on a masked map, we used PyMaster \citep{Alonso_2019}, which corrects the power spectra for the masking of part of the sky. In this work, we do not apodise the masks, as it is computationally very intensive at our high resolution and does not make a significant difference in the resulting power spectrum (see Fig. \ref{fig:tess_mask_apo}).
The full treatment which includes, creation, tessellation, masking and ILC is performed at the map level. Only when a tSZ full-sky map was extracted was the power spectrum of the map computed using spherical harmonics.  

\subsection{Debiasing} \label{sec_debias}

The instrumental white noise biasing of the tSZ power spectrum extraction was estimated for a $4000$ h observation time \citep{Choi_2020}. Therefore, the mean of this noise could be computed and removed to debias the tSZ reconstruction. This was done by generating random realisation of the instrumental white noise for each frequency at the sensitivities given in Table \ref{tab:fsb} and computing the projected noise by multiplying the ILC weights from the tSZ reconstruction with the noise maps and summing them. The power spectrum of this noise was then subtracted from the ILC-extracted Compton-y ($y_{\mathrm{ILC}}$) power spectrum. All the following power spectrum results were debiased for instrumental white noise. 

Similarly, as the CMB power spectrum is well-known, we could generate random realisation from the Websky CMB power spectrum to debias for the CMB residual noise. In the first approximation, this study presents an idealised case where the CMB power spectrum is perfectly known and its residual is removed. With real data, this CMB debiasing will be limited by the uncertainties on the recovered CMB power spectrum. 

\subsection{Overview of steps}

In this section, we describe our tSZ extraction method and illustrate it in Fig. \ref{fig:scheme_method}. 

First, using the Skymodel based on the Websky simulation and PySM, we generated multi-frequency maps of the microwave sky at the frequencies, sensitivities, and beam sizes of SO LAT and FYST (see Table \ref{tab:fsb}). The set of maps is in the HEALPix format and uses intensity $\mathrm{MJy \, sr}^{-1}$ units. 

Second, the simulated maps were all smoothed down to the resolution of the lowest frequency channel used in our analysis.

Third, a Galactic mask that excludes 40\% of the sky (see Fig. \ref{fig:masks}) was applied to the set of multi-frequency maps.

Fourth, each map was tessellated into smaller equal-area regions using the HEALPix nested format. 

Fith, an ILC was applied to each multi-frequency set of the tessellated area to obtain the Compton-y patches. The patches were reassembled to obtain the full-sky Compton-y map. The covariance matrix was computed only on the unmasked data by our ILC.

Sixth, the power spectrum of the recovered Compton-y map was computed using PyMaster and debiasing, and the residual noises were computed using Eq.\eqref{eqn:ILC_residual}


% Figure environment removed


\section{Results}

To quantify how much CIB could leak into the reconstructed tSZ maps for future experiments and access the detectability of the tSZ power spectrum, we used the microwave sky model detailed in Sec. \ref{des_skymodel}. We simulated a set of multi-frequency maps at the frequencies, sensitivities, and beam sizes for SO LAT and FYST (see Table \ref{tab:fsb}) with atmospheric noise. We applied an ILC or CILC component separation method to extract a tSZ map and looked at the ratios between the different noises and signal power spectra. We restricted ourselves to the scales $\ell \in [500,5000]$. We chose this lower limit to avoid the cosmic variance and the upper limit because of the $2.2'$ resolution of our HEALPix maps, which allowed us to resolve scales up to $\ell \sim 4909$. 


\subsection{Internal linear combination results} \label{ILC_results}

By construction, the weights of an ILC must verify the first condition, which is to have a unit response to the SED of the tSZ effect (see Eq. \eqref{eqn:ILC_cond1}). This first condition means that the element-wise product of weights ($\omega_{i}$) times the mixing vector ($a_{i}$) is an indicator of the fractional contribution of each frequency band to the recovery of the tSZ signal. It can be negative since the frequency spectrum of the tSZ effect is negative below $217$ GHz. The weights attributed by the ILC can themselves be negative for optimisation purposes, with the only constraint being that the sum over all frequencies must be one. The value of the weights is given by the mean of the 192 weights (one per tessellated field), and the error bars are given by the standard deviation of those weights. The top panel of Fig. \ref{fig:w_Atmo} shows that for SO LAT alone (orange dots), the 93 GHz channel is the main contributor to the tSZ reconstruction by a factor $\omega_{93}a_{93}\approx 1.4$. This channel overestimates the tSZ signal, and the SO LAT 145 GHz channel was used by the ILC to correct this. As a result, the 145 GHz channel contributes roughly by a factor $\omega_{145}a_{145}\approx -0.4$ to the tSZ reconstruction. This result is not surprising, as those frequency bands are located at the tSZ decrement that is characteristic of the signal in the microwave sky and where the values of the mixing vector are negative. When SO LAT and FYST are combined (SO LAT+FYST; blue dots), the SO LAT 93 GHz and 145 GHz channels remain the main contributors to the tSZ reconstruction and have similar values, $\omega_{93}a_{93}\approx 1.45$ and $\omega_{145}a_{145}\approx -0.45$. Other channels from SO LAT and FYST do not contribute to the reconstruction. When FYST is alone (green dots), the main contributor to the tSZ reconstruction is the 280 GHz channel, by a factor $\omega_{280}a_{280}\approx 1.1$, and the 350 GHz channel, by $\omega_{350}a_{350}\approx 0.05$. The 220 GHz and 405 GHz channels were used to mitigate the over evaluation of the tSZ signal with the respective factors $\omega_{220}a_{220}\approx -0.05$ and $\omega_{405}a_{405}\approx -0.1$. 

The second condition that has to be verified by the ILC weights is to ensure that the variance of the recovered map has been minimised (see Eq. \eqref{eqn:ILC_cond2}). The bottom panel of Fig. \ref{fig:w_Atmo} shows the values of the weights. In combination with the top panel, we concluded that the weights associated with some frequency bands that do not contribute to the tSZ signal reconstruction but have a high absolute value contribute to the noise minimisation. In particular, the 220 GHz channel located close to the tSZ null is often attributed a high weight by the ILC even though it does not contribute to the tSZ reconstruction. This is because it is the only map containing all the noise contaminants and almost no signal. It is thus the perfect map for noise minimisation. But this is not the case in Fig. \ref{fig:w_Atmo}, except for when FYST is alone (green dots), because of the presence of atmospheric noise in our mock sky. The strong atmospheric noise contaminates the 220 GHz band, making it less exploitable by the ILC (see the comparison with Fig. \ref{fig:w_Planck} for a case without atmospheric noise). Figure \ref{fig:w_Atmo} shows that in the presence of all the microwave emissions and atmospheric noise, the SO LAT frequency channels that probe the tSZ decrement, 93 GHz and 145 GHz, drive both the tSZ reconstruction and the noise minimisation of the tSZ signal.

% Figure environment removed

However, the weights do not tell us anything about the CIB contamination or the detectability of the tSZ signal. For that, we turned to the power spectrum and the residual ILC noise power spectra. The ILC output is a map containing the Compton-y signal and residual noises (Eq. \ref{eqn:ILC_residual}). Using PyMaster, the power spectra of this map and the residual noises can be computed and compared to the input-pure Compton-y map power spectra from the Websky simulations. We defined $D_{\ell}^{XX} = \ell(\ell+1)C_{\ell}^{XX}/2\pi$ as the power spectra of the quantity $X$  or of the residual noise $r_{X}$ of the quantity $X$. For SO LAT alone (see Fig. \ref{fig:PS_Atmo_debias} top-left panel) the dominant noise residual on small scales is the CIB residual ($r_{\mathrm{CIB}}$), which starts to dominate around $\ell \sim 3700$. The cumulative foreground residual ($r_{f}$) and the kSZ residual ($r_{\mathrm{kSZ}}$) are both subdominants to the input signal by around one order of magnitude over the scales $\ell \in [500,5000]$.
The atmospheric noise residual ($r_{\mathrm{Atmo}}$) dominates the input signal for $\ell < 1400$ and $\ell > 4500$. 

In the combined case of SO LAT+FYST (see right panel of Fig. \ref{fig:PS_Atmo_debias}), the situation is similar to the case where SO LAT is alone except that the atmospheric noise residual ($r_{\mathrm{Atmo}}$) is higher when FYST is present. This could be due to higher atmospheric noise being present in the high-frequency channels of FYST (see Fig. \ref{fig:red_noise}). The CIB noise residual power spectrum is significantly lower for SO LAT+FYST. It starts to dominate above the input signal around $\ell \sim 4200,$ while for SO LAT alone it was around  $\ell \sim 3700$. We concluded that adding FYST to SO LAT reduces the CIB noise residual.

The lower panel of Fig. \ref{fig:PS_Atmo_debias} shows the extent the input Compton-y power spectrum is dominated by the total ILC noise residual as a function of the scale. 
The total noise residual ($r_{tot}$) power spectrum dominates over the input signal on nearly all scales except around $\ell \in [2100,2700]$
for SO LAT alone. This is because of our debiasing of the CMB on those critical scales. Overall, SO LAT alone has a better input/noise ratio on scales $\ell<2500$ because of its slightly lower atmospheric noise residual, and SO LAT+FYST does better than SO LAT alone on scales $\ell>3000$ because of its lower CIB noise residual.

% Figure environment removed


\subsection{Constrained internal linear combination results} \label{CILC_results}

Constrained internal linear combination can be used to ‘null’ the contribution of a component of the microwave sky using its frequency spectrum. Nullifying several components was tested. 

Before any debiasing, the CMB is the main contaminant of the recovered Compton-y power spectra on large scales, so it seemed natural to deproject (i.e. ‘null’) it. However, the additional constraints on the ILC weights (see Eq. \eqref{eqn:CILC_ini_vec}) led to a smaller solution space for the weights to extract the signal, which resulted in a ‘noise penalty’, that is, higher noise in the recovered signal map. In this particular case of a mock sky containing all Galactic foregrounds and extragalactic emission, excluding RPSs, and applying a CILC to deproject CMB, the ILC weights are so constrained to null CMB that they become less efficient at reducing the second main contaminant, the atmospheric noise contamination, which blows up. As a consequence, the total residual noise dominates the input signal even more, by up to one order of magnitude on small scales (see Annexe \ref{AnnexeCILC}). Deprojecting the CMB is therefore not a recommended option for our study. This effect was already reported in Fig. 36 of \cite{SO_2019}, but in their study, the blowing up of the noise was mitigated on large scales by the addition of \textit{Planck} data.

Because of its high-frequency coverage, we expected FYST to better probe the CIB, which is the dominant contaminant above $\nu > 300$ GHz, and therefore better target CIB contamination and remove it. Deprojecting the CIB using CILC seemed the natural option to take full advantage of the potential constraining power of the FYST high-frequency channels. We note that a perfect deprojection of the CIB is not possible because its signal is not composed of only one SED. We adopted the modified blackbody spectrum given in Eq. \eqref{eqn:mbb_sed}, with $\mathrm{A}_{\mathrm{CIB}}=1$, $T_{\mathrm{dust}}=10$ K, and $\beta =1.6$. Parameter values were calculated by computing the average for each of the parameter maps derived in Sec.\ref{sec_extragal}.  

The top panel of Figure \ref{fig:PS_Atmo_CILC} shows that the CIB noise residual ($r_{\mathrm{CIB}}$) power spectrum is reduced by the CILC deprojection for both SO LAT and SO LAT+FYST, but it is not zero because the CIB is composed of multiples spectra and only one is used for the deprojection. It also shows that the atmospheric noise residual ($r_{\mathrm{Atmo}}$) power spectrum increases for SO LAT and decreases for the combined SO LAT+FYST when the CIB is deprojected (see Fig. \ref{fig:PS_Atmo_debias} for comparison). This can be explained by the higher degrees of freedom available with the higher number of frequency channels for SO LAT+FYST, making the CILC noise penalty have a less penalising effect in comparison to when SO LAT is alone. The lower panel of the figure shows that SO LAT+FYST offers a large $\ell \in [1700,3500]$ window over which the Compton-y signal is not noise dominated, while for SO LAT alone, the signal remains noise dominated (exclusively by the atmospheric noise). For SO LAT alone or for SO LAT+FYST combined, compared to the ILC case, the ratio between the input target signal and the resulting noise is not much higher in general (Fig.\ref{fig:PS_Atmo_debias}). This was also seen when applying our pipeline to \textit{Planck} data, as the recovered Compton-y map amplitude is roughly the same as the ILC and CILC. This is not surprising because we also saw that the ILC and CILC weights are very similar. This observation of the CILC not providing a much lower noise bias Compton-y map compared to the ILC was already reported by \cite{Alonso_2018}. 

% Figure environment removed


\subsection{Overview}

Figure \ref{fig:ratio_final} gives an overview of the relative amplitude changes for the power spectrum of residual noises of each astrophysical component when combining SO LAT and FYST and shows a comparison to SO LAT alone in the two previous cases applying an ILC (in blue) and a CILC deprojecting the CIB (in orange). To see how much SO LAT+FYST reduces residual noise power spectra in comparison to SO LAT alone, we took the ratio of the residual power spectra of each component in both cases:

\begin{ceqn}
\begin{align}
R = \frac{C_{\ell}^{XX}(\mathrm{SO\ LAT+FYST})}{C_{\ell}^{XX}(\mathrm{SO\ LAT})},
\label{eqn:ILC+beam2}
\end{align}
\end{ceqn}

where $C_{\ell}^{XX}(\mathrm{SO\ LAT+FYST})$ is the power spectrum of the component $X$ or of the residual noise $r_{X}$ for SO LAT + FYST and $C_{\ell}^{XX}(\mathrm{SO\ LAT})$ is the same quantity but for SO LAT alone. For the extracted Compton-y map ($y_{\mathrm{ILC}}$), the blue violin plot shows a mean improvement of around $ 4\%$ for SO LAT+FYST compared to SO LAT alone but with a large scatter. Especially at $\ell < 2800$, SO LAT performs better (see bottom panel of Fig. \ref{fig:PS_Atmo_debias}). When deprojecting the CIB (orange violin plot), we moved to a mean improvement of around $ 8\%$ in the ILC-extracted Compton-y map for SO LAT+FYST compared to SO LAT alone. In this case, the scatter is also much smaller than before because of the high number of degrees of freedom offered by the combined SO LAT+FYST frequency bands, and  the CILC noise penalty is much lower for SO LAT+FYST compared to SO LAT alone, making the combination strictly superior to extract a less biased tSZ power spectrum in the CILC case. 

For the residual CIB noise ($r_{\mathrm{CIB}}$), the power spectrum is on average around $35\%$ lower when combining SO LAT and FYST in the ILC case. In the CILC case, it drops to $\sim 7\%$, which might not be surprising because the CILC focusses on nulling as much CIB as possible no matter the frequencies. The addition of FYST does reduce the CIB residual even more but not as much as without the null constraint. Another reason for the drop in improvement when combining SO LAT and FYST in the CILC case might be the higher atmospheric noise at the high frequencies where the CIB signal dominates, undermining the advantage of FYST to constrain and remove CIB.

In the ILC case, the CMB residual noise ($r_{\mathrm{CMB}}$) power spectrum is reduced by $\sim 22\%$ when combining SO LAT with FYST compared to SO LAT alone. Because the ILC cannot separate CMB and kSZ due to their identical spectral frequency distribution, the residual kSZ noise ($r_{\mathrm{kSZ}}$) is also suppressed when combining SO LAT and FYST, on average by around $ 18\%$. When deprojecting CIB with a CILC, adding FYST to SO LAT also reduces the CMB and kSZ residual noises but only by $\sim 5\%$ over all scales on average. 

The sum of all galactic foreground noise residuals is defined as the cumulative foreground residual ($r_{f}$). Its noise power spectrum is improved on average by $\sim 10\%$ for SO LAT+FYST compared to SO LAT in the ILC case. In the CILC case, the cumulative foreground residual power spectrum is reduced by $\sim 30\%$ when combining FYST with SO LAT. This could be due to the additional constraint on the CIB dust spectrum, nulling part of the Galactic dust contribution, which is more easily nulled with FYST high-frequency coverage. 

The FYST high-frequency channels are more contaminated by the atmosphere. For this reason, the atmospheric red noise residual ($r_{\mathrm{Atmo}}$) power spectrum is $\sim 30\%$ higher when SO LAT and FYST are combined rather than when SO LAT is alone in the ILC case. When deprojecting the CIB, the atmospheric noise residual becomes slightly lower $\sim 1\%$ for SO LAT+FYST compared to SO LAT alone. This is due to the noise penalty generated by the additional constraint in the CILC worsening the atmospheric noise for  LAT alone compared to SO LAT+FYST. Because SO LAT+FYST  benefits from additional frequency bands and therefore more degrees of freedom, it is more resilient to the inevitable noise increase caused by additional spectral constraints.

% Figure environment removed

\section{Summary and discussion}

Using full-sky HEALPix template maps from the Websky simulations \citep{CITA_2020} and PySM, we created a pipeline that allowed us to simulate the high-resolution maps of the different microwave sky emissions at any beam, frequency, and sensitivity. Based on the \cite{Choi_2020_Noise} noise modelling, our pipeline also simulates atmospheric red noise at the locations of the Simons Observatory and FYST.

We simulated the sky that would be seen during experiments at these locations and used a map-based ILC and CILC to extract back the thermal Sunyaev-Zeldovich (tSZ) effect from this multi-frequency, multi-component sky. To minimise Galactic foreground contamination, we used a Galactic mask derived from \textit{Planck} (See Fig. \ref{fig:masks}). To optimise our ILC weights in order to take advantage of the spatial distribution of some of the microwave emissions, we tessellated the sky using the HEALPix equal-area nested tessellation scheme. We made use of CILC to deproject the CIB by using a simple modified blackbody spectrum to represent the CIB frequency variation and constrain the ILC weights to nullify it. We then used PyMaster to compute the power spectrum and its associated residual noise power spectra, which are the noise coming from all the other microwave sky emissions left by ILC and CILC in the recovered map.


We note that our study does have certain shortcomings that should be taken into account. For example, the Websky simulations do not yet offer a map of RPSs, and these extragalactic components are therefore absent from our study. The RPSs are an important contaminant of the tSZ signal, as they are located in galaxy clusters. If RPSs are not  considered, they can fill up the tSZ decrement and bias an estimation towards a lower estimated tSZ signal. A solution to this problem is to mask the RPSs, but this also lowers the estimated tSZ power spectrum, as it removes parts of the tSZ signal. This effect especially impacts measurements of the tSZ signal below 90 GHz, which we do not study here. A study from \cite{Holder_2002} showed that RPS subtraction can lead to an underestimation of the tSZ power spectrum of between $\sim 30\%$ to $\sim 50\%$, depending on the scale. We did not consider the relativistic correction of the tSZ effect, which can lead to a significant underestimation of the signal \citep{Remazeilles_2018}. This correction is left to future studies. Regarding the method, our tessellation scheme can lead to border effects that were reported by \cite{Eriksen_2004} but not accounted for. Our pipeline was tested on \textit{Planck} data with and without tessellation and only a marginal (i.e. $<5\%)$ difference could be seen in the power spectrum amplitude. Applying ILC in map-based space also has the disadvantage of the necessary resolution downgrade of all maps to the lower beam channel, which is 2.2' for SO LAT, thus reducing the benefits of FYST's higher resolution channels. This also prevented us from using \textit{Planck} data that are essential to lowering atmospheric and CMB contamination on large scales. This was addressed by performing CMB debiasing of the residual CMB noise in the extracted Compton-y map. A Fourier space or needlet ILC implementation is left for future work.    

We find that when using an ILC to extract tSZ, the combination of SO LAT with FYST reduces the CIB residual noise power spectrum by $\sim 35\%$,  on average, on scales $\ell \in [500,5000]$ when compared to SO LAT alone. For the CMB and kSZ components, the addition of FYST to SO LAT reduces the residual noise by  $\sim 22\%$ on average, with a strong scale dependence for the CMB, as the addition of FYST mainly helped on large scales. For the kSZ component, the reduction observed when adding FYST to SO LAT is not scale dependent. Adding FYST also helps reduce the Galactic foreground contamination by, on average, $\sim 10\%$ on scales $\ell \in [500,5000]$. However, FYST's high-frequency channels are more affected by atmospheric noise due to poorer transmission. Therefore, the residual noise power spectrum coming from the atmosphere is on average $30\%$ higher when SO LAT and FYST are combined compared to when SO LAT is alone. This greatly affects the total residual noise budget, lowering the benefits of adding FYST to SO LAT for the tSZ recovery to only an average $\sim 4\%$, and on larger scales, SO LAT alone performs better. We find that over $\ell \in [2100,2700]$, the ILC-extracted tSZ power spectrum is not noise dominated for SO LAT alone. But on smaller scales, it is extremely biased, and its noise budget is dominated by the CIB. When  SO LAT is combined with FYST, the CIB noise residual dominates much later $\ell \sim 4200$, but because the atmospheric noise is higher, there is no window over which the ILC-extracted tSZ power spectrum is not noise dominated. 

We note that the CIB residual noise reduction that FYST brings to SO LAT is not something that SO LAT could achieve on its own by simply having a longer survey time. This was tested by reducing the instrumental white noise in SO LAT bands by two, and we observed a reduction of the ILC residual instrumental noise in the reconstructed Compton-y map, but the other noise components remained the same. This ‘boosted’ version of SO LAT alone did not have a lower CIB residual power spectrum compared to the planned SO LAT or SO LAT plus FYST combination.

Because tSZ and CIB have a non-zero correlation in the Websky simulations and because FYST's high frequencies probe the CIB better, we used CILC to deproject the CIB component, modelling it with a simple one-component modified blackbody spectrum. Compared to the ILC case, adding FYST to SO LAT lowers the CIB noise residual by only $\sim 7\%$. The CMB and kSZ noise residuals are also reduced by $\sim 5\%$ when combining SO LAT and FYST when using CILC. However, the cumulative foregrounds are suppressed by $\sim 30\%$, and atmospheric noise is a bit lower for the SO LAT and FYST combination. Despite a lower reduction of the CIB residual noise when combining SO LAT and FYST, the CIB deprojection case offers a large  $\ell \in [1800,3500]$ window where the CILC-extracted Compton-y map is not noise dominated after instrumental noise and CMB debiasing. 

\begin{acknowledgements}
The authors would like to thank Kaustuv Basu for his guidance and advice throughout the project. Srinivasan Raghunathan, Jacques Delabrouille, and Frank Bertoldi for their time, valuable insights, and numerous helpful discussions. The referee for useful comments and numerous improvements to the draft. The editor Laura Pentericci for her advice and help on the submission process. Michael Coronado for his language editing work which significantly helped improve the manuscript. The CCAT-prime collaboration for their feedback on this work and Colin Hill for his help. Laila Linke for her useful comments on this draft. We furthermore acknowledge support from the International Max-Planck Research School (IMPRS) and the Bonn-Cologne Graduate School of Physics and Astronomy (BCGS). The simulations of the CIB used in this paper were developed by the Websky Extragalactic CMB Mocks team, with the continuous support of the Canadian Institute for Theoretical Astrophysics (CITA), the Canadian Institute for Advanced Research (CIFAR), and the Natural Sciences and Engineering Council of Canada (NSERC), and were generated on the GPC supercomputer at the SciNet HPC Consortium. SciNet is funded by the Canada Foundation for Innovation under the auspices of Compute Canada, the Government of Ontario, Ontario Research Fund – Research Excellence, and the University of Toronto.
\end{acknowledgements}


% WARNING
%-------------------------------------------------------------------
% Please note that we have included the references to the file aa.dem in
% order to compile it, but we ask you to:
%
% - use BibTeX with the regular commands:
%   \bibliographystyle{aa} % style aa.bst
%   \bibliography{Yourfile} % your references Yourfile.bib
%
% - join the .bib files when you upload your source files
%-------------------------------------------------------------------
% for the bibliography, at the end
%                                                                 aa.dem
% AA vers. 9.1, LaTeX class for Astronomy & Astrophysics
% demonstration file
%                                                       (c) EDP Sciences
%-----------------------------------------------------------------------
%
%\documentclass[referee]{aa} % for a referee version
%\documentclass[onecolumn]{aa} % for a paper on 1 column  
%\documentclass[longauth]{aa} % for the long lists of affiliations 
%\documentclass[letter]{aa} % for the letters 
%\documentclass[bibyear]{aa} % if the references are not structured 
%                              according to the author-year natbib style

%
\documentclass[bibyear]{aa}  

\usepackage{natbib}
\setcitestyle{notesep={, },aysep={}}
\bibpunct{(}{)}{;}{a}{}{,} % to follow the A&A style
%
\usepackage{graphicx}
\usepackage{longtable}
\usepackage{tabularx}
\usepackage{supertabular}
\usepackage{tabularray}
\usepackage{xspace}
\usepackage{booktabs,caption}
\usepackage[flushleft]{threeparttable}
%%%%%%%%%%%%%%%%%%%%%%%%%%%%%%%%%%%%%%%%
\usepackage{txfonts}
%%%%%%%%%%%%%%%%%%%%%%%%%%%%%%%%%%%%%%%%
\usepackage{hyperref}
%\hypersetup{
%    colorlinks=true,
 %   linkcolor=blue,
  %  filecolor=blue,      
   % urlcolor=blue,
    %pdftitle={Overleaf Example},
    %pdfpagemode=FullScreen,
    %}
    
    \hypersetup{colorlinks=true,allcolors=[rgb]{0,0,1}}
\usepackage{amsmath}

\newcommand{\lya}{\mbox{Lyman-$\alpha$}\xspace} 
\newcommand{\xlae}{\mbox{$X_{\textrm{LAE}}$}\xspace}
%\newcommand{\ttpaper}{\mbox{Thai et al. (submitted.)}\xspace}
\newcommand{\ttpaper}{\mbox{\cite{Thai2023}}\xspace}
\newcommand{\mm}{\mbox{$\,\mu\mathrm{m}$}\xspace}
\newcommand{\hypz}{\mbox{\textit{New-Hyperz}}\xspace}
%\newcommand{\ewlya}{$\mathrm{EW_{Ly\alpha}}$ }
\newcommand{\ewlya}{\mbox{$\mathrm{EW_{Ly\alpha}}$}\xspace}
\newcommand{\sfrlya}{\mbox{$\mathrm{SFR_{Ly\alpha}}$}\xspace}
\newcommand{\sfruv}{\mbox{$\mathrm{SFR_{UV}}$}\xspace}
\newcommand{\fuv}{\mbox{$\mathrm{f_{esc,UV}}$}\xspace}
\newcommand{\flya}{\mbox{$\mathrm{f_{esc,Ly\alpha}}$}\xspace}


\defcitealias{GdlV2019LAELF}{dLV19}
\defcitealias{GdlV2020LAEfrac}{dLV20}
\defcitealias{richard2021atlas}{R21}
\defcitealias{AC2022LLAMAS}{C22}
% To add links in your PDF file, use the package "hyperref"
% with options according to your LaTeX or PDFLaTeX drivers.
%
\begin{document} 
   \title{Evolution of the Lyman-$\alpha$ emitting fraction and UV properties of lensed star-forming galaxies between $2.9<z<6.7$}
 
   \titlerunning{Evolution of the Lyman-$\alpha$ emitting fraction and properties of lensed galaxies between $2.9<z<6.7$}

 
   
   \author{I. Goovaerts\inst{1}\and R. Pello\inst{1}\and T. T. Thai\inst{1,2,3}\and P. Tuan-Anh\inst{2,3}\and J. Richard\inst{4}\and A. Claeyssens\inst{5}\and E. Carinos\inst{1}\and G. de la Vieuville\inst{6}\and J. Matthee\inst{7}}
          

   \institute{Aix Marseille Université, CNRS, CNES, LAM (Laboratoire d’Astrophysique de Marseille), UMR 7326, 13388 Marseille, France\\
              \email{ilias.goovaerts@lam.fr}
         \and     
         Department of Astrophysics, Vietnam National Space Center, Vietnam Academy of Science and Technology, 18 Hoang Quoc Viet, Hanoi, Vietnam
         \and
         Graduate University of Science and Technology, VAST, 18 Hoang Quoc Viet, Cau Giay, Vietnam
         \and
         Univ Lyon, Univ Lyon1, Ens de Lyon, CNRS, Centre de Recherche Astrophysique de Lyon UMR5574, F-69230, Saint-Genis-Laval, France
        \and
        Department of Astronomy, Oskar Klein Centre, Stockholm University, AlbaNova University Centre, SE-106 91 Stockholm, Sweden
         \and
         7 Avenue Cuvier, F-78600, Maisons-Laffitte, France
         \and
         Department of Physics, ETH Zürich, Wolfgang-Pauli-Strasse 27, 8093 Zürich, Switzerland
         }

   %\date{Received XXX; accepted XXX}
 
  \abstract
  % context heading (optional)
  % {} leave it empty if necessary  
   {Faint galaxies are theorised to have played a major, perhaps dominant role in reionising the Universe. Their properties as well as the \lya emitter fraction, \xlae, could provide useful insight into this epoch.}
  % aims heading (mandatory)
   {We use four clusters of galaxies from the Lensed Lyman-alpha MUSE Arcs Sample (LLAMAS) which also have deep HST photometry to select a population of intrinsically faint Lyman Break Galaxies (LBGs) and \lya Emitters (LAEs). We study the interrelation between these two populations, their properties, and the fraction of LBGs that display \lya emission.}
  % methods heading (mandatory)
   {The use of lensing clusters allows us to access an intrinsically faint population of galaxies, the largest such sample collected for this purpose: 263 LAEs and 972 LBGs between redshifts of 2.9 and 6.7, \lya luminosities between $39.5\lesssim \mathrm{log(L_{Ly\alpha}) (erg\,s^{-1})}\lesssim42$ and absolute UV magnitudes between $-22\lesssim\mathrm{M_{1500}}\lesssim-12$. As well as matching LAEs and LBGs, we define an LAE+continuum sample for the LAEs which match with a continuum object that is not selected as LBG. Additionally, with the use of MUSE integral field spectroscopy, we detect a population of LAEs completely undetected in the continuum.}
  % results heading (mandatory)
   {We find a redshift evolution of \xlae in line with literature results, with diminished values above $z=6$. This has been taken to signify an increasingly neutral intervening IGM. When inspecting this redshift evolution with different limits on \ewlya and $\mathrm{M_{1500}}$ we find that \xlae for the UV-brighter half of our sample is higher than \xlae for the UV-fainter half, a difference which increases at higher redshift. This is a surprising result and can be interpreted as the presence of a population of low \lya equivalent width (\ewlya), UV-bright galaxies situated in reionised bubbles and overdensities. This result is especially interesting in the context of similar, UV-bright, low \ewlya objects recently detected in and around the epoch of reionisation. We extend to intrinsically fainter objects the previously observed trends of LAEs among LBGs as galaxies with high star-formation rates and low dust content, as well as the strongest LAEs having in general fainter $\mathrm{M_{1500}}$ and steeper UV slopes.}
  % conclusions heading (optional), leave it empty if necessary 
  {}
   
   \keywords{galaxies: high redshift --
                dark ages, reionisation, first stars --
                gravitational lensing: strong
               }
   
   \maketitle
%
%-------------------------------------------------------------------

\section{Introduction}
\label{sect:intro}
After the Dark Ages, the neutral gas in the intergalactic medium (IGM) was reionised: the last phase transition undergone by the Universe. This process ended with the hydrogen in the IGM ionised, around $z=6$ \citep{fan2006reionisation,mcgreer2015reionisation,Planck2018reionisation,lu2022subarureionisation}. There are two main candidates thought to be responsible for this process, star-forming galaxies (SFGs) \citep{bouwens2015UVLF, Finkelstein_2015, Livermore_2017} and active galactic nuclei (AGN) \citep{MadauHardtAGN_2015, Grazian2018AGN}. The influence of AGN is likely to be small \citep{Onoue2017AGN, parsa2018noAGNcontribution, Mcgreer2018AGN, JiangAGN2022}. Currently, the favoured candidate is SFGs, particularly faint SFGs \citep{robertson2013reionisation,Bouwens2015planckreionisation,Stark2016review}, although the possibly significant contribution of bright SFGs is still debated \citep{naidu2020brightlya, Mathee2022brightlya}.\\
\indent In order to study these intrinsically faint SFGs, one of the most powerful tools is \lya emission. Galaxies that exhibit \lya emission are called \lya emitters (LAEs). The strength of the \lya line for equivalent widths greater than, for instance, $25\,\AA$, allows us to identify intrinsically faint and/or high redshift galaxies. 
In recent years, this has been exploited from several different angles in order to learn more about galaxies at this epoch as well as the state of the IGM and hence reionisation itself (see, for example, the reviews by \citealt{Stark2016review,dijkstra2016LAEreview,robertson2022Reionisationreview} and references therein).\\
\indent Additionally, the use of lensing allows us to probe fainter galaxies than in blank field surveys, down to \lya luminosities of $\sim10^{39}\,\mathrm{erg\,s^{-1}}$ \citep{bina2016muse,smit2017lensedemissionlines,GdlV2019LAELF,GdlV2020LAEfrac,richard2021atlas,AC2022LLAMAS}, the last four of which are henceforth \citetalias{GdlV2019LAELF}, \citetalias{GdlV2020LAEfrac}, \citetalias{richard2021atlas} and \citetalias{AC2022LLAMAS}. This gives us direct access to the faint populations that are, as mentioned, currently the favoured candidates for the main contributor to reionisation. This avenue has been explored in recent studies with small to medium sample sizes (\citetalias{GdlV2020LAEfrac}; \citealt{fuller2020LAEfrac}). Lensing, however, comes with compromises on sample size and volume of the Universe probed. Even more recently, samples of more significant size (hundreds of objects) have become available, such as the Lensed Lyman-alpha MUSE Arcs Sample (LLAMAS) \citepalias{richard2021atlas,AC2022LLAMAS}. \\
\indent In order to investigate reionisation and the role of SFGs, the fraction of Lyman Break Galaxies (LBGs) that exhibit \lya emission (henceforth referred to as the LAE fraction or $X_{\textrm{LAE}}$) is particularly interesting. The Lyman break is caused by absorption of photons at wavelengths shorter than $912\, \AA$ by neutral hydrogen gas around the galaxy and up to $1216\, \AA$ by the Lyman forest along the line of sight. This can be used to search for galaxies photometrically by using the `drop-out' technique: galaxies will `drop out' of filters bluewards of the Lyman break. \lya emission is scattered by neutral hydrogen in the IGM, ISM and circum-galactic medium (CGM) so whether or not \lya emission is detected from LBGs gives us information about the content of these media, in particular how ionised they are, which has the potential to help in reconstructing the timeline and scale of reionisation (see, for example \citealt{mason2018LAEfrac,arrabal2018LAEfrac,kusakabe2020,leonova2022prevalence,bolan2022neutralfraction}). Studies show a drop in the prevalence of LAEs among LBGs (\xlae) above $z=6$, suggesting the increasing neutrality of the IGM before this time and supporting the established reionisation timeline (\citealt{stark2010keckLAEfrac,stark2011LAEfrac,pentericci2011laefrac/z=7LBG,caruana2014LAEfrac,deBarros2017LAEfraction,pentericci2018LAEfrac,caruana2018,hoag2019reionisation_z=7.7}; \citetalias{GdlV2020LAEfrac}). 
However there are significant uncertainties associated with both the measurement of \xlae and its usage as a probe of the reionisation history. The evolution of \xlae with redshift could also be due to the inherent evolution of one or both of the populations considered, rather than solely as a consequence of the changing state of the IGM \citep{bolton2013IGMlya_alternate,mesinger2015IGMlya}. Progress has been made in understanding the impact of the ISM, CGM and dust attenuation on \lya emission \citep{verhamme2008lyadustsim,zheng2010lya_radiativetransfer,dijkstra2011lyadetectability,kakiichi2016lyamorphology_reionisation} however there is still significant debate on the physics of \lya photon escape, in simulations as well as observations, at different redshifts and how this impacts LAE visibility and hence \xlae \citep{dayal2011lyavisibility,matthee2016calymha_lyaesc,hutter2014lyavisibility,sobral2019predictinglyaesc,smith2022lyatransmission_sim} .\\
\indent Additionally, \xlae has shown some dependence on absolute rest-frame UV magnitude \citep{stark2010keckLAEfrac,stark2011LAEfrac,schaerer2011LAEfrac,schenker2014lyafracfeasibility,kusakabe2020} and significantly on the \lya EW cut above which \xlae is calculated \citep{stark2010keckLAEfrac,caruana2018,kusakabe2020}. It is crucial to be aware of these factors when comparing results in the literature. \\
\indent Equally important are the possible biases introduced by the different methods of selecting both the UV `parent sample' and the LAE sample. When collecting the parent sample using the Lyman break, there is no standard way of performing the selection of these galaxies. Some samples are selected by colour--colour cuts such as \citealt{stark2010keckLAEfrac,pentericci2011laefrac/z=7LBG,bouwens2015UVLF,pentericci2018LAEfrac,bouwens2021UVLF,yoshioka2022LAEfrac}, some by photometric redshifts (\cite{caruana2018,fuller2020LAEfrac,kusakabe2020}, \citetalias{GdlV2020LAEfrac}) (although this probably doesn't have a large effect on the sample as the procedure to find the photometric redshift relies on the Lyman Break in the same way as the colour--colour cuts \citepalias{GdlV2020LAEfrac}). The colour--colour cuts used depend on the instrument and bands used to observe the sample as well as the depth of the observations, leading to different cuts for each study. \\
\indent The exact selection using photometric redshifts is also down to the authors of each individual study, such as the signal-to-noise required for a detection as well as how to deal with the probability distributions provided by most photometric redshift codes. \\
\indent Once the parent sample has been selected, the way in which the search for LAEs is conducted is also not standardised. Some authors select LAEs based on Narrow-Band photometry \citep{arrabal2018LAEfrac,yoshioka2022LAEfrac}, some search for \lya emission among their UV--selected sample using multi-object slit spectroscopy \citep{stark2010keckLAEfrac,stark2011LAEfrac,pentericci2011laefrac/z=7LBG,deBarros2017LAEfraction,pentericci2018LAEfrac,fuller2020LAEfrac} and some use Integral Field Unit (IFU) spectroscopy (\citealt{caruana2018}; \citetalias{GdlV2020LAEfrac};  \citealt{kusakabe2020}). \\
\indent Using slit spectroscopy to search a UV-selected sample for \lya emission can be problematic, as the \lya emission is not always centred on the UV emission, in fact it has often been found to have an offset. \citetalias{AC2022LLAMAS} report a median offset of $\Delta_{\mathrm{Ly\alpha-UV}}=0.66\pm0.14\,\mathrm{kpc}$ and \citet{hoag2019lyaoffset} find an offset corresponding to  $0''.25$ at $z=4.5$. Therefore, slit spectroscopy may not see the \lya emission at all, or may miss some flux from extended emission.\\
\indent The completeness of the different samples is also an important factor to take into account in these studies. Completeness is a correction made to the amount of objects observed to account for those present in the field but not observed. This correction depends on many factors and is different for each study. Several different approaches have been employed in the literature. \citet{stark2010keckLAEfrac} consider the completeness of their \lya detections by inserting fake emission lines across their spectra and attempting to detect them using the original detection process, a method which has been evolved into the complex \lya completeness treatments seen in IFU studies such as \citet{kusakabe2020} and \citetalias{GdlV2020LAEfrac}. \citetalias{GdlV2020LAEfrac} and this study involve the extra complication engendered by lensing fields (see \citetalias{GdlV2020LAEfrac} and \ttpaper). \citetalias{GdlV2020LAEfrac} show that the inclusion of the LAE completeness correction is significant to the calculation of \xlae. \\
\indent The study can be performed on a UV--complete subsample of the LBG population such as in \citet{kusakabe2020} but it is common to assume that one's LBG selection is highly complete for the signal-to-noise requirements imposed in the selection process. This is also an assumption sometimes made for the \lya selection. For studies not involving lensing, one can easily calculate the apparent magnitude at which one becomes incomplete in the UV selection at the 10\% and 50\% level such as in \citet{arrabal2018LAEfrac}.\\
\indent In light of these discrepancies in the different studies, it is not surprising that there is significant disagreement on the precise values and evolution of \xlae further than a general consensus that \xlae rises from lower redshifts to a redshift of $\sim$6, after which it sees a decrease at higher redshifts (see, among others: \citealt{pentericci2011laefrac/z=7LBG,stark2011LAEfrac,deBarros2017LAEfraction,arrabal2018LAEfrac,caruana2018,fuller2020LAEfrac}; \citetalias{GdlV2020LAEfrac}).\\
\indent In addition to the reasons previously mentioned, the scatter in the results from these various studies can come from issues related to sample size, as well as the exact sample used to calculate the fraction. For example, the inclusion limits on $M_{\mathrm{1500}}$ are not homogeneous across all studies, neither are the inclusion limits on the \lya equivalent width. \\
\indent In this paper we investigate faint star-forming galaxies towards the epoch of reionisation ($2.9<z<6.7$), observed behind four lensing clusters in the Hubble Frontier Fields (HFF hereafter; \cite{lotz2017frontier}), specifically chosen for being efficient enhancers of such high-redshift objects. We select LAEs from MUSE IFU spectroscopy and LBGs from deep Hubble photometry and photometric redshifts. We present the largest LAE and LBG combined sample of lensed, intrinsically faint galaxies used for this purpose to date. Our sample reaches as faint as $\mathrm{M_{1500}}\sim-12$ and $\mathrm{log(L_{Ly\alpha}/erg\,s^{-1})\sim39.5}$ after correcting for lensing magnification. This is similar to the latest LBG selection in the HFF from \cite{Bouwens2022UVLF2z9}. Having blindly selected these populations of galaxies, independently of each other, we compare their UV and \lya-derived properties and explore the fraction of LAEs among the LBG population, its redshift evolution and UV magnitude dependence.\\
\indent In Section~\ref{sect:data} we cover the data used from MUSE and the HFF as well as the specific selection criteria we apply for our samples of LAEs and LBGs. We outline the photometric redshifts used and subsequently the blind matching of both populations. Section~\ref{sect:results} outlines our results pertaining to derived properties: equivalent widths, UV slopes and star formation rates, as well as the interrelation between the two populations, including several approaches to the fraction of LAEs among the LBG population. We discuss the implications for reionisation and the properties of these high-redshift galaxies. In Section \ref{sect:conclusion} we summarise our findings and offer perspectives for future surveys.\\
\indent The Hubble constant used throughout this paper is $H_0=70\,\mathrm{km\,s^{-1}Mpc^{-1}}$ and the cosmology: $\Omega_{\Lambda}=0.7$, $\Omega_{\mathrm{m}}=0.3$. All EWs and UV slopes are converted to their rest-frame values and all magnitudes are given in the AB system \citep{OkeGunn1983}. All values of the UV absolute magnitude, defined at $1500\,\AA$ rest frame, $\mathrm{M_{1500}}$, and \lya luminosity are given corrected for magnification. 

%--------------------------------------------------------------------
\section{Data and Population Selection}
\label{sect:data}
For this study, we combine Multi--Unit Spectroscopic Explorer (MUSE; \cite{Bacon_et_al_2010_SPIE}) IFU observations with the deepest Hubble Space Telescope (HST) photometry available in lensing clusters. The Hubble Frontier Fields clusters \citep{lotz2017frontier} are ideal for this work, specifically, we use Abell 2744, Abell 370, Abell S1063 and MACS 0416 (henceforth A2744,  A370, AS1063 and M0416). The MUSE observations of these clusters are taken from the data collected in \citetalias{richard2021atlas}, from which the LAEs are selected to form the LLAMA Sample \citepalias{AC2022LLAMAS}. The complementary HST data are from the HFF-DeepSpace program (PI: H. Shipley)\footnote{\url{http://cosmos.phy.tufts.edu/~danilo/HFF/Download.html}}.

\begin{table*}
    \caption{Details of each lensing cluster used in this study.}
    $$
    \begin{array}{p{0.07\linewidth}p{0.1\linewidth} p{0.1\linewidth}p{0.06\linewidth}p{0.08\linewidth}p{0.08\linewidth}p{0.05\linewidth}p{0.05\linewidth}p{0.05\linewidth}p{0.07\linewidth}}
        \hline
        \hline
        \noalign{\smallskip}
        Cluster & RA & DEC & Redshift & MUSE Exposure Time (Hrs) & Co-volume (Mpc\textsuperscript{3}) & LAE only & LBG only & LAE + LBG & LAE + continuum\\
        \hline
        \noalign{\smallskip}
        \textbf{A2744} & 00:14:20.702  & -30:24:00.63 & 0.308 & 3.5-7 & 10500 & 24 & 294 & 47 & 50\\
        \noalign{\smallskip}
        \hline
        \noalign{\smallskip}
        \textbf{A370} & 02:29:53.122 & -01:34:56.14 & 0.375 & 1.5-8.5 & 5350 & 8 & 267 & 20 & 14\\
        \noalign{\smallskip}
        \hline
        \noalign{\smallskip}
        \textbf{AS1063} & 22:48:43.975 & -26:05:08.00 & 0.348 & 3.9 & 1970 & 6 & 159 & 7 & 8\\
        \noalign{\smallskip}
        \hline
        \noalign{\smallskip}
        \textbf{M0416S} & 04:16:09.144 & -24:04:02.95 & 0.397 & 11-15 & 1670 & 4 & 81 & 14 & 16\\
        \noalign{\smallskip}
        \hline
        \noalign{\smallskip}
        \textbf{M0416N} & 04:16:09.144 & -24:04:02.95 & 0.397 & 17 & 3420 & 13 & 71 & 12 & 20\\
        \noalign{\smallskip}
        \hline
        \noalign{\smallskip}
        \textbf{Total} &  &  &  &  & 22910 & 55 & 872 & 100 & 108\\
        \noalign{\smallskip}
        \hline
        \hline
        \vspace{0.5cm}
    \end{array}
    $$
    \label{table:data_details}
    \vspace{-5mm}
    \begin{tablenotes}
      \small
      \item \textbf{Notes.} We show the number of each sample group in each cluster (see text). The different exposure times for three of the clusters come from the range of exposure times for different MUSE pointings for these clusters.
    \end{tablenotes}
\end{table*}

% Figure environment removed

\subsection{MUSE LAEs: LLAMAS}
\label{sect:LAE}

\subsubsection{LAE Selection}
\label{sect:laeselection}
The MUSE data we use for this work is part of the Lensed Lyman-Alpha MUSE Arcs Sample (LLAMAS) \citepalias{AC2022LLAMAS}. These observations were part of the MUSE Guaranteed Time Observing (GTO) program and are comprehensively described in \citetalias{AC2022LLAMAS} as well as \citetalias{richard2021atlas}. The catalogues and lens models for the four clusters used in this work are available online\footnote{\url{https://cral-perso.univ-lyon1.fr/labo/perso/johan.richard/MUSE_data_release/}}. We summarise here the details of the reduced MUSE datacubes and the LAE selection process. \\
The MUSE datacubes contain the flux and variance over a $1\times1\,\mathrm{arcmin}^2$ field of view, with spatial and spectral pixel scales of $0.2''$ and $1.25\,\AA$. The spectral range covers $4750\,\AA$ to $9350\,\AA$, allowing the detection of \lya between redshifts of 2.9 and 6.7. Integration times on the clusters we use for this work vary between 2 and 14 hours, with different pointing configurations (see below). The clusters are all in the range $0.3<z<0.4$, providing magnifications useful to amplify sources in the MUSE \lya redshift range. Details of the four clusters are given in Table~\ref{table:data_details}. Full details covering all 17 LLAMAS clusters can be found in \citetalias{AC2022LLAMAS} and \ttpaper. \\
\indent In order to detect line emission sources such as LAEs, \citetalias{richard2021atlas} follow the prescription laid out in \citet{PW2020MUSEdata}, using the MUSELET software (\citealt{piqueras2019asp})\footnote{\url{https://mpdaf.readthedocs.io/en/latest/muselet.html}}, which is used on MUSE Narrow Band (NB) images. Subsequently, the Source Inspector package \citep{bacon2022musedatareleaseII} is used to identify sources and assign their redshifts. This package allows users to cycle through a list of sources with all the relevant information: spectra, NB images, HST counterparts, MUSELET results and redshift suggestions. With this information, users can decide on a redshift for an object as well as assigning each source a redshift-confidence level from 1 to 3. A confidence level of 1 denotes a tentative redshift and a confidence level of 3 denotes a redshift with a high confidence. For this study we only use LAEs with confidence levels of 2 and 3, indicating secure redshifts. In our sample -- A2744, A370, AS1063, M0416 -- there are 263 such LAEs. \\ 
\indent Magnifications of sources are assigned with the use of the parametric mass distribution models in \citetalias{richard2021atlas} (for A2744, A370 and M0416, for AS1063, the lens model comes from \citealt{beauchesne2023AS1063lensmodel}) and the LENSTOOL software \citep{kneib1996lenstooloriginal,jullo2007lenstool,jullokneib2009lenstool2,kneibnew2011lenstool}. These models are well constrained by the large number of multiple images with strong spectroscopic redshifts (levels 2 and 3 as described above) in these clusters. \citetalias{richard2021atlas} estimate a typical statistical error of $1\%$ of the mass profile of these clusters. The models in turn give the magnifications of the sources used in this study, which range from 0.8 (demagnified) up to 137. Most sources have magnifications between 1.5 and 25. While these lensing models are well understood and benefit from many multiple image systems as constraints, small systematic uncertainties related to the lens model used can still persist due to the particular choice of mass distribution (see, for example \citealt{Meneghetti2017HFF,Acebron2017lensing_systematics,Acebron2018strong_lensing_analysis,Furtak2021cluster_stellasmassfunc}). \\


\subsubsection{LAE Flux Determination}
\label{sect:laeflux}
The \lya flux for the LAEs in our sample was extracted using one of two different methods.\\
The main method employed in the LLAMA Sample is the line profile fitting procedure described in \citetalias{AC2022LLAMAS}. For the subsample used in this work (263 LAEs), all those behind A2744 and 20 behind the other three clusters (roughly half the subsample), we use a method involving SExtractor \citep{EB96SEx} following the procedure outlined in \citetalias{GdlV2019LAELF}. We give a brief description of both methods.\\
\indent The line profile fitting method utilises three steps: spectral fitting, NB image construction and spectral extraction. The first step involves fitting an asymmetric Gaussian function to the \lya peak with the EMCEE package from \citet{foreman2013emcee}, using 8 walkers and 10,000 iterations (in double peaked cases, both peaks are considered separately and their fluxes combined after the extraction is complete).  The peak position of the \lya line, flux, full width at half maximum (FWHM) and asymmetry of the source are all fitted. The second step takes the result for the peak position and creates a NB image around the LAE. The continuum around the LAE in redward and blueward bands of width $24\,\AA$ is subtracted from the \lya flux. The NB bandwidth is optimised such that the SNR of the \lya peak is maximised in an aperture of radius $0.7\arcsec$. Finally, utilising this new NB image of the LAE, a new extraction is performed from the MUSE datacube, ensuring that as much of the \lya emission from the galaxy and surrounding halo is extracted. The process is repeated twice more, each time using the latest NB image and extraction.\\
\indent The second method is employed for A2744 and in the three other clusters for the cases where the first method fails to fit the \lya flux. This happens in very low signal-to-noise cases or sources close to the edge of the datacube. This method uses SExtractor on NB images as described in \citetalias{GdlV2019LAELF}. The NB images in question are those in which each LAE is detected. Three sub-cubes are extracted from the main datacube, one encompasing the \lya emission and two either side of it (spectrally) each of width $20\,\AA$. These cubes are averaged to form a continuum image, which is subtracted pixel-by-pixel from an image formed by averaging the cube containing the \lya emission. SExtractor is then run on this new continuum-subtracted image and the {\tt FLUX\_AUTO} parameter is used to estimate the fluxes of the LAEs. To deal with faint sources or those with an extended, low surface brightness, SExtractor can progressively loosen the detection conditions (using the {\tt DETECT\_THRESH} and {\tt DETECT\_MINAREA} parameters) so that a flux can also be extracted for these sources. \\
Fig. 4 of \ttpaper shows the comparison between the two methods of extracting the \lya flux. In general, the two methods agree well. In some cases, the line profile fitting method estimates a larger flux due to its enhanced appreciation of the line complexity. \\
\indent In Fig.~\ref{fig:LAELBGzm1500dist} we show the properties of our sample of LAEs in red. In order to have a similar derivation of $\mathrm{M_{1500}}$ for all our LAEs, regardless of detection in HST photometry, values of $\mathrm{M_{1500}}$ are calculated from the filter closest to the $1500\,\AA$ rest frame wavelength (where available, and the $2\sigma$ upper limit of the continuum taken from this filter where not), and corrected for magnification. \lya luminosities are derived from the fluxes described above and also magnification-corrected. In terms of \lya luminosity we probe roughly between $39.5<\mathrm{log_{10}(L_{Ly\alpha}/erg\,s^{-1})}<42.5$, with decreased statistics near the faintest and brightest limits. This faint population can at present only be accessed with lensing clusters; the typical limit in blank fields lies around $\mathrm{log_{10}(L_{Ly\alpha})}\sim42$ (see for example, \citet{herenz2019MUSELAELF}), down to $\mathrm{log_{10}(L_{Ly\alpha})}\sim41.5$ in the MUSE Ultra Deep Field \citep{drake2017MUSELAELF}.

\subsection{HFF data}
\label{sect:HFF}
The HST observations of the clusters used in this work are from the Hubble Frontier Fields program (ID: 13495, P.I: J. Lotz). In particular, we use the photometric catalogues of the HFF-DeepSpace Program \citep{shipley2018hff}. As part of this program, the authors collected homogeneous photometry across the HFF, including deep $K_{S}$-band imaging ($2.2$\mm) from the HAWK-I on the Very Large Telescope (VLT) and Keck-I MOSFIRE instruments \cite{Brammer2016KeckHFF}. Additionally included are all available data from the two Spitzer/IRAC channels at 3.6\mm and 4.5\mm. The two Spitzer/IRAC channels at 5.8\mm and 8\mm were judged to be too noisy and excluded from the SED fitting process (see Section~\ref{sect:LBG_selection}). The details of the filters used, and their depths, for each cluster can be found in Table~\ref{table:filter_depths}. The bright cluster galaxies (BCGs) and intra-cluster light (ICL) are subtracted by \cite{shipley2018hff} for improved photometry of background sources in these very crowded fields. The detection image for each of these clusters is made up of a combination of the F814W, F105W, F125W, F140W and F160W bands (PSF-matched to the F814W band), after the modelling out of the BCGs. \\
\indent The area that MUSE observed for each of these clusters is fully contained within the HST area, so we cut the HST area we consider to that of the MUSE data. The resulting effective (lens-corrected) co-volume, derived from LENSTOOL source-plane projections, is $22910\,\mathrm{Mpc^{3}}$ over the redshift range $2.9<z<6.7$.

\begin{table}
    \caption{$5\sigma$ limiting magnitudes of each filter for the four clusters.}
    $$
    \begin{array}{ p{0.15\linewidth} p{0.1\linewidth} p{0.1\linewidth} p{0.12\linewidth} p{0.1\linewidth} }
        \hline
        \hline
        \noalign{\smallskip}
        \textbf{Filter} & \textbf{A2744} & \textbf{A370} & \textbf{AS1063} & \textbf{M0416} \\
        \hline
        \noalign{\smallskip}
        F225W & -- & -- & 24.0  &  24.4 \\
        \hline
        \noalign{\smallskip}
        F275W & 26.0 & 25.7 & 25.9 & 25.7 \\
        \hline
        \noalign{\smallskip}
        F336W & 26.7 & 26.2 & 26.3 & 26.2\\
        \hline
        \noalign{\smallskip}
        F390W & -- & -- & 25.4 & 25.7\\
        \hline
        \noalign{\smallskip}
        F435W & 27.5 & 27.2 & 27.3 & 27.5\\
        \hline
        \noalign{\smallskip}
        F475W & -- & 26.6 & 25.8 & 26.2\\
        \hline
        \noalign{\smallskip}
        F606W & 27.7 & 27.3 & 27.5 & 27.7\\
        \hline
        \noalign{\smallskip}
        F625W & -- & 25.6 & 25.5 & 25.9\\
        \hline
        \noalign{\smallskip}
        F775W & -- & -- & 25.5 & 25.6\\
        \hline
        \noalign{\smallskip}
        F814W & 27.8 & 27.5 & 27.7 & 27.9\\
        \hline
        \noalign{\smallskip}
        F850LP & -- & -- & 25.2 & 25.2\\
        \hline
        \noalign{\smallskip}
        F105W & 27.9 & 27.2 & 27.6 & 27.7\\
        \hline
        \noalign{\smallskip}
        F110W & -- & 26.6 & 26.4 & 26.3\\
        \hline
        \noalign{\smallskip}
        F125W & 27.4 & 27.0 & 27.4 & 27.6\\
        \hline
        \noalign{\smallskip}
        F140W & 27.4 & 27.1 & 27.3 & 27.7\\
        \hline
        \noalign{\smallskip}
        F160W & 26.8 & 26.9 & 26.9 & 27.4\\
        \hline
        \noalign{\smallskip}
        K\textsubscript{S} & 25.1 & 24.9 & 25.1 & 25.1\\
        \hline
        \noalign{\smallskip}
        IRAC1 & 24.1 & 23.6 & 23.6 & 24.7\\ 
        \hline
        \noalign{\smallskip}
        IRAC2 & 24.3 & 23.7 & 24.2 & 24.2\\
        \hline
        \hline
    \end{array}
    $$
    %\vspace{-5mm}
    \begin{tablenotes}
      \small
    \item \textbf{Notes.} Values have been calculated from the properties of the sample and adopted for the SED fitting described in Section.~\ref{sect:LBG_selection}. Dashes indicate missing filters for a particular cluster.
    \end{tablenotes}
    \label{table:filter_depths}
\end{table}

\subsection{LBG selection}
\label{sect:LBG_selection}
We calculate photometric redshifts and probability distributions (hereafter $P(z)$) in order to perform our LBG selection. We use the package \hypz \citep{MB00hyperz} to estimate redshifts and probability distributions. This package uses a standard $\chi^2$ minimisation technique to fit template galaxy Spectral Energy Distributions (SEDs) to photometric data points. It has been optimised for the redshift determination of high-redshift galaxies so is ideal for our purpose. We use a suite of template spectra to fit the photometric data: the four template spectra from \citet{CWW1980colors}, two Starburst99 models with nebular emission \citep{leitherer1999starburst99}: a single burst model and a constant star-formation rate model (where each spans five metallicities and 37 stellar population ages), and seven models adapted from \citet{BandC03SED}. Included in these seven are a star formation burst model, five exponentially decaying models with timescales between 1 and $30\,\mathrm{Gyr}$ and a constant star formation model. The redshift range we use in \hypz is 0 to 8 with a step in redshift of 0.03. The Calzetti extinction law \citep{Calzetti00dust} is used to account for internal extinction with values of $A_v$ allowed to vary between 0 and 1.5 magnitudes. No priors on galaxy luminosity were introduced during this process, avoiding any bias introduced by lensing magnification.\\ 
\indent For a galaxy to be included in the LBG sample for this study we demand that the best solution for the photometric redshift lies in the range $2.9<z<6.7$. We also accept any candidates not in this range which have $1\sigma$ errors overlapping with it. In practice, this accounts for just 14 objects. To make sure the detection is real, we also demand a $5\sigma$ detection in at least one HST filter. \hypz provides a redshift probability distribution, $P(z)$, which can have two peaks, in which case there are two redshift solutions for that particular galaxy. We accept galaxies into the sample if only one of the peaks is in the required range, as long as $60\,\%$ of the integrated $P(z)$ also lies in this range. We can compare the photometric redshift results for those objects also selected as LAEs to the spectroscopic redshift determined from the peak of the \lya emission. Fig.~\ref{fig:z_compare} shows this comparison. The dashed lines indicate regions constrained by $|z_{spec}-z_{phot}|<0.15(1+z_{spec})$ and $|z_{spec}-z_{phot}|<0.05(1+z_{spec})$. We encode the apparent magnitude in the F125W filter in the colourbar to demonstrate that the instances in which \hypz performs poorly (i.e. instances outside of the outer region specified above) tend to be fainter objects with magnitudes $\gtrsim28$.\\
\indent Finally, we manually inspect our LBG sample using the HST images, photometry, SED fitting from \hypz and the LENSTOOL package with the lens models described in Section~\ref{sect:laeselection}. This procedure is designed to identify multiple images in the LBG sample caused by the gravitational lensing, as well as to remove any obvious spurious detection; regions of noise or contamination erroneously selected as LBGs. Multiple images are identified using a combination of LENSTOOL predictions, redshift determinations and SEDs from \hypz, and visual inspection of the HST images. LENSTOOL predictions, redshift estimates and colours of objects have to match for those objects to be designated a multiple image system. The colours used are $\mathrm{F814W-F606W}$, $\mathrm{F105W-F814W}$, $\mathrm{F140W-F105W}$, all of which have to match to within $2\sigma$ for the objects in question to be designated multiple images. From each system, a representative LBG image is chosen. This image is the least contaminated, or, in the case of a match with an LAE multiple image system, the image chosen to represent the LBG multiple system matches the selected LAE image (see Section~\ref{sect:matching}). \\
% Figure environment removed

\subsection{Matching populations}

\label{sect:matching}
Having blindly selected our LAE and LBG samples, we compare them, using a matching radius of $0\farcs5$, to see which objects are selected as both. The results are shown in Table~\ref{table:data_details}. \\
We select a population labelled LAE+continuum, which denotes objects that are selected as LAEs, for which we see a continuum in the HST images, but where this continuum fails to meet the selection criteria for our LBG sample. Mostly, these galaxies fail on the SNR criteria outlined in the previous subsection, indicating a faint continuum or a noisy area in the HST images. Nevertheless, we keep these objects as, thanks to the \lya emission, we know that there is a high-redshift galaxy at these positions. In the previous work of this nature solely on A2744, \citetalias{GdlV2020LAEfrac} simply included these objects in the LAE+LBG sample, relaxing the signal-to-noise criterion for these objects, seeing as the presence of an object (as well as its redshift) was known thanks to the \lya emission. Here however, we keep the distinction between these continuum detections with \lya emission and our LBG sample which fulfil all the criteria laid out in the previous subsection. Details on the inclusion of this sample into the analysis of the LAE fraction, \xlae, are given in Section~\ref{sect:LAEfraction}.\\
\indent For the LAEs, a `best' image is chosen, a process described fully in \ttpaper. This allows us to chose representative images that are minimally impacted by neighbours, have high signal-to-noise and moderate magnification. When selecting LBGs, we keep the image corresponding to the best LAE image selected by \ttpaper where possible. However, in some cases we chose another image, because that particular LAE image is selected as LBG rather than continuum only. We ensure that the image chosen is always of similar quality. A modification of this nature is rare and we impose it in less than 10 systems across the whole LAE sample. The original \lya flux and magnification of the source, as given in the LLAMAS catalogues, is used for our analysis on the properties of these objects, however their designation as LAE+continuum or LAE+LBG can change based on this. \\

\subsection{Completeness Determination}
\label{sect:LAEcompleteness}
As covered in Section~\ref{sect:intro}, the completeness of the populations considered in such a study is very important. This correction to the number of sources detected aims to reflect the number of sources that are missed in the detection process and thus the number of sources actually present in the field of view. \\
The completeness methods used for the LAEs in this work are described in \ttpaper and summarised here. Following the procedure first laid out in \citetalias{GdlV2019LAELF}, sources are treated individually in this computation. Each source's brightness distribution profile, both in the spatial and spectral dimensions, is modelled and randomly injected 500 times into the NB layer of the original MUSE datacube where its \lya emission reaches a maximum. This process is performed in the image plane, in order to recreate as closely as possible the actual process of detection with MUSELET. The completeness of the source is the number of times (out of 500) it is successfully detected and extracted. \\
\indent Since it is the local noise that likely decides whether or not such an injected source is detected, to account for variations in the local noise, \ttpaper change the size of the NB image used to re-detect the simulated sources from $30\arcsec\times30\arcsec$ to $80\arcsec\times80\arcsec$. This was found to improve the extraction of the source when that source has close neighbours. The mean completeness value found for our sample is 0.72 and the standard deviation is 0.34. \\
Finally, each source's contribution to the LAE fraction is corrected by a value $1/\mathrm{C_i}$ where $\mathrm{C_i}$ is the completeness value of that source. In effect: a source with a completeness value of 0.5 contributes 2 LAEs to the calculation of the LAE fraction. 

%--------------------------------------------------------------------
\section{Results}
\label{sect:results}

% Figure environment removed

% Figure environment removed

% Figure environment removed

\subsection{UV properties of the Populations}
\label{sect:UVprop}
In order to ascertain the similarities and differences between our populations of high redshift galaxies, it is useful to look at the physical properties we derive from the HST photometry and their relation to \lya emission. This can also help to disseminate how these properties tie in with the LAE fraction (see Section~\ref{sect:laeflux}). Additionally, we can appraise any differences from the established trends for these high redshift galaxies that may appear in our sample of faint, lensed galaxies.\\
\indent To evaluate this relation to \lya emission, the \lya equivalent width, \ewlya, is an important property to derive. We can then compare this to UV-derived properties. In order to calculate \ewlya values for our sample, first we derive UV slopes by fitting the photometric points starting above the location of the \lya emission (irrespective of whether or not we detect it for a given galaxy) and including all the filters up to $2600\,\AA$ in the rest-frame, adapting the approach used in \citet{castellano2012Beta,schenker2014lyafracfeasibility,bouwens2015UVLF}. We fit a power law, $F_{\lambda}\propto\lambda^{\beta}$, where $\beta$ is the UV slope, to the photometry. We choose this method as it does not rely on SED fitting with a set of templates which have specific allowed values of $\beta$.\\
\indent Subsequently, by using this photometrically fitted UV slope for each object, we can ascertain the continuum flux level beneath the \lya emission. Hence we calculate the \lya equivalent width (\ewlya) by dividing the \lya flux by the continuum flux level, corrected to its rest-frame level. In this process we take into account the error on the UV slope resulting from the fitting process, as well as the error associated with the \lya flux (see Section~\ref{sect:LAE}).\\
\indent For the objects with no associated continuum, the upper limits of the continuum are taken from the filter closest to the $1500\,\AA$ rest-frame emission. For the objects with no associated \lya emission, the detection limits of the \lya emission are used.\\
\indent The EW distribution with redshift for all our sample is shown in Fig.~\ref{fig:EWvsz}. The dashed black line denotes the typical EW inclusion limit for objects to the calculation of \xlae (see Section~\ref{sect:LAEfraction}). The percentage of objects above this limit is $44\%$ and $50\%$ for LAE+LBG objects and LAE+continuum objects respectively. This is to be expected as the objects selected as continuum but not LBG are in general fainter, giving rise to higher values of \ewlya.\\ 
\indent In Fig.~\ref{fig:EWvszM1500} we compare \ewlya to UV magnitude for our LAE+LBG sample. UV absolute magnitude, $\mathrm{M_{1500}}$, is calculated from the filter closest to the $1500\,\AA$ rest-frame emission. We see a rise in \ewlya towards fainter $\mathrm{M_{1500}}$, in agreement with results reported in \citet{stark2010keckLAEfrac,deBarros2017LAEfraction,kusakabe2020}. We note that above $M_{1500}=-16$ this graph is populated by high EW, continuum-undetected LAEs (shown by orange triangles). We do not include these objects in the binned averages as these objects are not LBGs and the continuum values are estimated upper limits, however these objects indicate that this trend in EW likely increases to even fainter magnitudes than is populated by our current LBG selection. As the spatial extent of \lya emission correlates with the size of the galaxy and hence $\mathrm{M_{1500}}$ (\citealt{wisotzki2016extendedlyahaloes,leclercq2017lyahaloes}; \citetalias{AC2022LLAMAS}), the trend in Fig.~\ref{fig:EWvszM1500} is often seen as a natural result. \\
\indent We also compare \ewlya to the UV slope (Fig.~\ref{fig:EWvsbeta}). In general, high-EW LAEs are found to have bluer slopes \citep{stark2010keckLAEfrac,schenker2014lyafracfeasibility,deBarros2017LAEfraction}. We report a slight increase ($<1\sigma$) in average EW in bins between $\beta=-1.0$ and $\beta=-3.0$. However, similarly to \citet{deBarros2017LAEfraction} we find \ewlya values as high as $\sim100\,\AA$ across the sample. \\
\indent  We derive star-formation rates from \lya (\sfrlya) and from the UV continuum (\sfruv), based on the relations given in \citet{kennicutt1998Schmidtlaw} and the factor of 8.7 between $\mathrm{L(H\alpha)}$ and $\mathrm{L(Ly\alpha)}$, assuming a Salpeter IMF \citep{salpeter1955IMF} and constant star formation.
\sfrlya is a good lower limit on the intrinsic SFR as \lya flux can be lost due to dust attenuation or a more or less opaque IGM \citep{zheng2010lya_radiativetransfer,gronke2021lyatransmission}. However due to the use of an IFU we have no impact on \lya flux from slit losses.\\
\indent We compare the two measures of SFR for our LAE+LBG sample in Fig.~\ref{fig:SFR}, plotting the line of one to one ratio and the actual median ratio (\sfrlya/\sfruv) found in our sample. This median ratio (0.35), is well below the one to one ratio, indicating that in most cases the escape fraction of UV photons (the fraction of photons at $1500\,\AA$ rest frame that escape the galaxy), \fuv, exceeds that of \lya photons (the fraction of \lya photons that escape the galaxy), \flya. This is less apparent in UV-fainter objects, where there is a greater fraction above the \fuv = \flya line. This result is in line with previous findings in \citet{ando2006stronglyadef,schaerer2011LAEfrac} and \citetalias{GdlV2020LAEfrac}. The explanation offered previously by \citet{ando2006stronglyadef} and \citetalias{GdlV2020LAEfrac} relates to the likely difference between the UV-bright and UV-faint galaxies. If the UV-bright galaxies have had more time to evolve than UV-fainter objects, it is likely that they are chemically more complex and have a larger amount of dust. This would decrease the escape of \lya photons and hence result in a smaller measured \sfrlya. We return to this point having compared the LAE fraction for the UV-bright and UV-faint halves of our sample in Section~\ref{sect:LAEfraction}.\\
\indent Subsequently, we compare $\mathrm{M_{1500}}$ and \sfruv to the UV slopes of both samples (LBG and LAE+LBG) between $-3.0<\beta<-1.0$: Fig.~\ref{fig:SFRvsbeta}. In this graph, we calculate binned averages as in Figs.~\ref{fig:EWvszM1500} and \ref{fig:EWvsbeta}, however for this plot the averages are computed for bins of equal population. For the LBG only sample, there is a slight increase in \sfruv with steeper UV slopes, this is to be expected as steeper values of $\beta$ generally indicate populations that are more intensely star forming.\\
From the binned averages we can see that for the population also selected as LAEs, the galaxies are distributed towards higher values of \sfruv and steeper UV slopes. This result supports previous findings (in various redshift ranges, as well as in simulations) that LAEs are in general dust-poor and are more intensely star forming than objects only selected as LBGs \citep{verhamme2008lyadustsim,hayes2011redshiftevofdust,hayes2013lyadust,sobral2019predictinglyaesc,santos2020lyaUVproperties}.\\
\indent We extend this genre of analysis to our large sample of intrinsically faint galaxies, finding the same trends; LBGs also selected as LAEs have on average higher \sfruv and steeper UV slopes. High-\ewlya objects are on average fainter in absolute UV magnitude with steeper UV slopes, and those fainter in UV absolute magnitude tend to have \flya approaching \fuv or in some cases above, while this is not seen for UV-brighter LAEs. All these trends are seen with significant scatter, to be expected given the intrinsically faint nature of the sample, the limits of the photometry (albeit the deepest available) and those objects that have a complex star formation history and physical structure, for which our assumption of constant star formation is less valid. Combined, our results reinforce the picture of high-redshift LAEs as UV-faint, intensely star forming, dust-poor galaxies and show that these trends hold for galaxies down to $\mathrm{M_{1500}\sim-12}$.




% Figure environment removed

% Figure environment removed
 
\subsection{The Interrelation between the LAE and LBG populations}
\label{sect:interrelation}

% Figure environment removed

We define four samples: objects identied as LBG only (henceforth LBG only), objects identified as LAE only (henceforth LAE only), objects identified as both LAE and LBG (LAE + LBG) and objects identified as LAE which have a continuum detected in the photometry from HST which itself does not meet our criteria for the LBG sample (LAE+continuum). These samples have 872, 55, 100 and 108 members respectively. \\
\indent The sample is dominated by A2744 with $46\%$ of the LAEs and $35\%$ of the LBGs; this is a large field of view with long MUSE exposures (see \citetalias{richard2021atlas}, \citetalias{AC2022LLAMAS} and \ttpaper), however, as many fields of view as possible are valuable additions to the sample. The redshift distribution and the $M_{1500}$ distribution of the sample are displayed in Fig.~\ref{fig:M1500_z_distribution}. We find the same result as \citetalias{GdlV2020LAEfrac} when considering the relative importance of LAEs which are totally undetected in the continuum. Their increasing presence is clear when looking at the faint region of the $\mathrm{M_{1500}}$ distribution, even when using the deepest HST photometry. This is crucial to note for any study wishing to catalogue the contribution to reionisation of all star-forming objects at high-redshift. This effect is naturally expected to strengthen with decreasing depth of photometry.  


\subsubsection{LAE fraction}
\label{sect:LAEfraction}

% Figure environment removed


Before discussing \xlae and its relevance to reionisation, we first outline what is typically meant by \xlae. The typical qualitative interpretation is `the fraction of UV-selected Lyman Break Galaxies that display Lyman--$\alpha$ emission'. This stems from the technique of spectroscopic follow-up on a photometric preselection of targets (e.g. \citealt{stark2010keckLAEfrac,stark2011LAEfrac}). In the case of this study, both LAE and LBG selections are performed blindly. With an IFU, we also detect LAEs that have no continuum counterpart, objects that would clearly be missed in the aforementioned type of study.\\
\indent Additionally, limits are typically placed on the inclusion of objects when calculating \xlae. The most common limits are \ewlya$>25\,\AA$ and $\mathrm{M_{1500}}>-20.25$ (\citealt{stark2011LAEfrac,schenker2014lyafracfeasibility,deBarros2017LAEfraction,mason2018LAEfrac,pentericci2018LAEfrac,arrabal2018LAEfrac}; \citetalias{GdlV2020LAEfrac} among others), however \cite{caruana2014LAEfrac} use a limit of $75\,\AA$. Other authors such as \cite{stark2010keckLAEfrac,cassata2015vimosLAEfrac,caruana2018,kusakabe2020,fuller2020LAEfrac} investigate different EW and $\mathrm{M_{1500}}$ or apparent magnitude ranges to assess the EW and UV magnitude evolution of \xlae, with the exact ranges depending on the study: the depths available and whether or not lensing is involved. In blank fields, most commonly authors investigate the bright and the faint halves of their sample, split at $\mathrm{M_{1500}}=-20.25$ (e.g. \citealt{cassata2015vimosLAEfrac,deBarros2017LAEfraction,arrabal2018LAEfrac}). Alternatively, where sample sizes allow, studies bin results in $\mathrm{M_{1500}}$ from $-22$ to $-18$ (e.g. \citealt{caruana2018,kusakabe2020}).
These are important distinctions to be aware of when comparing results from the literature. This discrepancy between the exact inclusion limits studies use, as well as the observational differences (as covered in the introduction) is likely a significant factor in the ongoing debate over the exact evolution of \xlae with redshift, \ewlya and $\mathrm{M_{1500}}$ (see Fig.~\ref{fig:LAEfrac20.25}).\\
\indent In this work we provide a perspective based on blind selections of both populations with an IFU and deep photometry with which we can assess more fully the landscape of these high-redshift galaxies. We compare the inclusion of objects only selected as LBG as well as the LAE+continuum sample described in Section~\ref{sect:matching}. Given that we detect LAEs with no continuum counterpart we also discuss the effect this has on \xlae. We investigate different $M_{1500}$ and \ewlya inclusion limits to provide context on the differing results in the literature. This last point is additionally motivated by the use of lensing fields in this study, meaning that we effectively probe a different population than blank field studies. It is therefore appropriate to investigate different regions of the \ewlya and $\mathrm{M_{1500}}$ space.\\
In Fig.~\ref{fig:LAEfrac20.25} we compare our results to results from the literature in the classical manner with inclusion limits of $\mathrm{M_{1500}}>-20.25$ and \ewlya$>25\,\AA$. The redshift bins used are $2.5<z<3.5$, $3.5<z<4.5$, $4.5<z<5.5$ and $5.5<z<7.0$. For each bin, we include four values of \xlae: the fraction of \lya emitters among LBGs (yellow dots), the fraction of \lya emitters among LBGs with the completeness corrections described in Section~\ref{sect:LAEcompleteness} (red dots) and the same two measurements including the LAE+continuum objects (yellow and red stars) (see Section~\ref{sect:matching}).\\ 
\indent Despite significant uncertainties, we see roughly the same trend as many results in the literature, with \xlae decreasing around $z=6$. The decrease of \xlae shown in our data is quicker than many equivalent measurements in the literature. If one carries forward the assumption that \xlae efficiently probes the neutral hydrogen content, this result suggests that the IGM quickly becomes neutral after $z=6$.\\
\indent The objects we detect only as LAEs (making up $\sim21\%$ of our LAE sample) would push these fractions higher, were they to be included. Considering that these objects are extremely faint in terms of UV magnitude and high in terms of \ewlya, they would make an appreciable difference in the calculation of the fraction (in a similar way to the LAE+continuum sample shown by red and yellow stars in Fig.~\ref{fig:LAEfrac20.25}). There is little reason to suspect that with deeper photometry in the region that sees the rest frame UV emission, the continuum would remain undetected for these objects.\\
\indent The high values of \xlae that we see in the first (low redshift) bin in Fig.~\ref{fig:LAEfrac20.25} are likely due to the fact that the Lyman break is harder to detect at these redshifts given the filters at our disposal. In order to check this, we perform a simulation, creating 100000 fake LBG sources using Starburst99 templates \citep{leitherer1999starburst99}. We add realistic noise to the data and spread them equally throughout the redshift and UV magnitude space probed by our study. We then attempt to select LBGs following the original selection criteria laid out in Section~\ref{sect:LBG_selection}. In the first redshift bin ($2.5<z<3.5$) the amount of LBGs selected versus the number of simulated LBGs is drastically lower than the other bins, at $<10\%$. It is worth noting that this simulation was computed for AS1063: a `best case scenario', given that AS1063 has the most filters available in the short wavelength range $(<606\,\mathrm{nm})$. For sources at redshifts between 2.5 and 3, the break would appear between $\sim320\,\mathrm{nm}$ and $\sim365\,\mathrm{nm}$. The blue filters have a shallower depth (for AS1063, F225W, F275W, F336W F390W have $5\sigma$ limiting magnitudes $\sim24.0, 25.9, 26.3, 25.4$ respectively) than the redder HST filters and due to this and their scarcity (A2744 only has F275W and F336W), we struggle to recover all the LBGs in this bin (for complete filter depth information see Table~\ref{table:filter_depths}). This can also explain why in Fig.~\ref{fig:LAELBGzm1500dist} we can see a trough in the distribution at the lower end of our redshift range, why \xlae is higher than many literature estimates in the lowest redshift bin and forms part of the reason why we see little evolution between redshifts of 3 and 5. However our results are still in agreement with the literature results at $1\sigma$. Effects like this are not unexpected considering the varying nature of LBG (and LAE) selection.\\
\indent The errors on \xlae as displayed in Figs.~\ref{fig:LAEfrac20.25} and \ref{fig:LAEfrac18} contain several components. As mentioned, there is a big impact on sample inclusion from the \ewlya limits of $25\,\AA$. Therefore it is important to quantify an error based on this. We perform 1000 Monte Carlo trials of \xlae, sorting the \ewlya values in their error bars, assumed to be Gaussian. We evolve an error estimate for \xlae based on the 16th and 84th percentile of the resulting range of results for \xlae. \\
Secondly, for the fractions to which we apply our LAE completeness correction, this correction comes with an uncertainty (see Section~\ref{sect:LAEcompleteness}) which is propagated together with the error described above and the Poissonian error count on the number of LBGs and LAEs detected. This Poissonian error in \xlae is given by: $\sigma_{P}=\left(N_{LAE}/N_{LBG}^2 + N_{LAE}^2/N_{LBG}^3\right)^{1/2}$. \\
\indent The effect of the sample inclusion limits on \ewlya are clear to see from Fig.~\ref{fig:LAEfrac20.25}, particularly in the third redshift bin. We reinforce an appreciation of this as an important matter in the calculation of \xlae and the interpretation of plots such as Fig.~\ref{fig:LAEfrac20.25}. For this study, this source of error means that it is not possible to describe a statistically significant evolution of \xlae until the highest-redshift bin. \\
\indent Similarly, the inclusion of the LAE+continuum objects in the calculation of \xlae (denoted by red and yellow stars in Fig.~\ref{fig:LAEfrac20.25}) has an appreciable effect: an average absolute difference in \xlae of 0.05 for points without completeness correction and 0.07 for those with completeness correction. This is a reflection of the impact of LBG selection in the final calculation of \xlae, particularly the SNR criterion imposed in the selection. \\
Finally, we can see from Fig.~\ref{fig:LAEfrac20.25} that the completeness correction applied to the number of LAEs plays an important role in the calculation of \xlae (see the red points in Fig.~\ref{fig:LAEfrac20.25}). Without this correction, it is also difficult to address the redshift evolution of \xlae. \\  
% Figure environment removed
\indent There is interesting and unsettled debate on the relative evolution of \xlae between the UV-bright and UV-faint populations (see \citealt{stark2010keckLAEfrac,oesch2015spectroscopic,deBarros2017LAEfraction,stark2017luminouslyalpha,kusakabe2020}). In particular, this debate has consequences at higher redshift, $z\gtrsim6$, on what type of objects can be seen in \lya emission despite an increasingly neutral IGM. Relations between the UV magnitude and \lya emission at these redshifts may hold clues as to how the IGM is reionised and which objects are primarily responsible for this. In order to assess this, we split our sample in UV absolute magnitude at $\mathrm{M_{1500}}=-18$ (as this limit splits our sample roughly in half) and recalculate \xlae for the bright and faint portions. Fig.~\ref{fig:LAEfrac18} shows this comparison. We note that the $\mathrm{M_{1500}}$ ranges into which our data is split differ from many similar comparisons in the literature due to the lensed nature of our sample. We efficiently probe down to $\mathrm{M_{1500}}\sim-12$, much fainter than studies conducted in blank fields. By contrast, we have few objects as bright as $\mathrm{M_{1500}}=-22$. \\
\indent In the upper panel of Fig.~\ref{fig:LAEfrac18}, there is no statistical difference between the bright and faint halves of the sample until the final redshift bin ($5.5<z<7.0$), however we do see a rising trend in the bright half and a decreasing trend in the faint half. In order to investigate this further we remove the \ewlya limit of $25\,\AA$, resulting in the lower panel of Fig.~\ref{fig:LAEfrac18}. Based both on the result from Fig.~\ref{fig:EWvszM1500} and indications from the literature \citep{curtis2012lyafrac,stark2017luminouslyalpha}, it is likely that UV-bright galaxies exhibit \lya with small EWs: close to or below the $25\,\AA$ limit, due to spectral and spatial dispersion of the \lya photons from such systems \citep{mason2018LAEfrac}. \\
\indent This is supported by the clear distinction we see in the lower panel of Fig.~\ref{fig:LAEfrac18}. This result suggests that, while UV-fainter objects generally have higher \lya EWs, the LAE fraction is greater among UV-brighter objects, albeit with those objects exhibiting smaller \ewlya. Our findings are especially interesting in the context of the number of low \ewlya, UV-bright objects reported towards and into the epoch of reionisation (\citealt{curtis2012lyafrac,oesch2015spectroscopic,stark2017luminouslyalpha,mason2018brightlya,mason2018LAEfrac,larson2022reionisation,bunker2023GNz11,Witten2023lya_inreionisation}, see also \citealt{ando2006stronglyadef}). This result is often taken as surprising given that the increasingly neutral IGM should block much of the \lya emission at redshifts greater than 6. However, it is possible that with an increasingly neutral IGM, the more luminous galaxies ionise a surrounding bubble, making it easier for \lya emission to escape. These galaxies are also more likely to be situated in reionised overdensities \citep{Matthee2015brightlyareionise,mason2018brightlya,Witten2023lya_inreionisation}. \citet{Witten2023lya_inreionisation}, cataloguing 8 LAEs within the epoch of reionisation, most of which have small \ewlya ($<25\,\AA$), also suggest that mergers and tidal interactions with neighbours may be responsible for enhanced \lya visibility during the epoch of reionisation. Reionised bubbles created by these processes and these bright galaxies would allow the \lya photons to redshift out of resonance by the time they encounter significant neutral hydrogen, and hence would not be absorbed or scattered.\\
\indent More detailed, individual analysis of individual objects would be necessary to firmly support these conclusions for our high-redshift objects but the difference seen in the highest redshift bins of the graphs in Fig.~\ref{fig:LAEfrac18} broadly supports the enhanced visibility of UV-bright, low-\ewlya LAEs in the epoch of reionisation.\\
\indent We caution the reader that we detect a population of high-EW LAEs undetected in the continuum, and if these objects were to be detected with deeper photometry, they could modify this result as they would be UV-faint with high \ewlya. However, they are roughly equally spread over the redshift range probed so may leave the trends in Fig.~\ref{fig:LAEfrac18} unchanged. \\
\indent We plot the completeness corrected results in Fig.~\ref{fig:LAEfrac18} in grey scale, darker grey representing the UV-bright portion of the sample and lighter grey representing the UV-faint portion. Due to the smaller sample sizes in these magnitude-split bins, these corrections have large errors where one or a few LAEs with a small completeness values, accounting for many completeness-corrected LAEs, can dominate the determination of \xlae for a given bin. The lower panel of Fig.~\ref{fig:LAEfrac18} provides an example: \xlae for the bright half of the sample in the bin $4.5<z<5.5$ is at 1.19 i.e. accounting for the completeness correction, there are more UV-bright LAEs than LBGs in this bin. Noting that this is caused by individual highly incomplete objects, we do not interpret this as meaningful in the context of \xlae with respect to the evolution of the IGM and reionisation. On the other hand, we note that the completeness-corrected trends also broadly support the conclusions outlined earlier in this section, particularly in the highest-redshift bins, where the difference between the UV-bright and UV-faint portions of the sample is even more pronounced. 
 

%----------------------------------------------------------------- 
\section{Conclusions}
\label{sect:conclusion}
 We present an assessment of the interrelation of the faint \lya Emitter and Lyman Break Galaxy population viewed with MUSE IFU spectroscopy and deep HST photometry. We can access faint populations unseen in blank fields through the magnification provided by four lensing clusters, pushing LBG detections down to $\mathrm{M_{1500}}\sim-12$ and LAE detections down to \lya luminosities of $\mathrm{log(L_{Ly\alpha})\sim39\,erg\,s^{-1}}$. We find LAEs with no detected continuum counterpart, actors that play an increasingly important role in the regime fainter than $M_{1500}\sim-16$. We summarise our main results as such:\\
\indent -- Our results for \xlae agree with findings in the literature and if we accept that it is unlikely that the evolution of one or both populations considered makes a significant difference to \xlae, our faint sample supports the conclusions of brighter studies, that the Universe is reionising at $z=6$ and beyond. However based on the results of this study, we find little to no evolution between redshifts of 3 and 5. In part this is likely due to the greater effect of LBG selection incompleteness in the lower redshift regions. The scatter in this redshift range in the literature is likely due to issues related to the LBG selection and population completeness as discussed in Section~\ref{sect:LAEfraction}.\\
\indent -- We compare the evolution of \xlae for the bright and faint halves of our sample, split at $\mathrm{M_{1500}}=-18$. We find a different trend between the two; \xlae for the bright half rises towards $z=6$ and \xlae for the faint half falls. The difference in \xlae between the UV-bright and UV-faint half is statistically significant in the highest-redshift bin, indicating that this effect increases as the Universe becomes more neutral.\\
\indent -- To further investigate this, we remove the $25\,\AA$ limit in the calculation of \xlae. Without this limit, a clear distinction is seen between the bright and faint halves of the sample, suggesting that while high-EW LAEs tend to be UV-faint, there is a population of low-EW LAEs among the UV-bright population. This can account for some of the newly detected, low-EW, UV-bright objects around and in the epoch of reionisation. \\
\indent -- When considering the UV properties of our faint samples, we see the typical picture of LAEs as high-SFR, low-dust, UV-faint galaxies, extending previously observed trends to very faint UV magnitudes and \lya luminosities. The strongest trend is seen when comparing \ewlya to $\mathrm{M_{1500}}$ (Fig.~\ref{fig:EWvszM1500}). UV-fainter objects on average display larger \ewlya, a trend which likely continues down to $\mathrm{M_{1500}}\sim-12$ based on our sample.\\
\indent -- We also extend, to fainter luminosities and with a larger sample, the trend that UV-brighter galaxies tend to exhibit greater \fuv than \flya, whereas UV-fainter galaxies are distributed more towards the \fuv = \flya line, sometimes even displaying \flya > \fuv. This has been attributed to UV-bright populations being older, dustier and more chemically evolved, decreasing their \lya emission (\citealt{ando2006stronglyadef}; \citetalias{GdlV2020LAEfrac}). This is tentatively reported for our sample, particularly as there are significant uncertainties associated with the \sfruv values of the fainter galaxies.\\
\indent -- Even with the deepest HST photometry available, the presence of continuum-undetected LAEs remains important at UV magnitudes fainter than $\mathrm{M_{1500}}=-16$. Surveys with JWST can help to shed light on the effect of much deeper photometry, albeit at different wavelengths than the HST.

%-----------------------------------------------------------------

\begin{acknowledgements}
This work is done based on observations made with ESO Telescopes
at the La Silla Paranal Observatory under programme IDs 060.A-9345, 092.A-0472,
094.A-0115, 095.A-0181, 096.A-0710, 097.A0269, 100.A-0249, and
294.A-5032. Also based on observations obtained with the NASA/ESA Hubble
Space Telescope, retrieved from the Mikulski Archive for Space Telescopes
(MAST) at the Space Telescope Science Institute (STScI). STScI is operated by
the Association of Universities for Research in Astronomy, Inc. under NASA
contract NAS 5-26555. 
All plots in this paper were created using Matplotlib (Hunter 2007). 
Part of this work was supported by the French CNRS, the Aix-Marseille University, the French Programme National de Cosmologie et Galaxies (PNCG) of CNRS/INSU with INP and IN2P3, co-funded by CEA and CNES.
This work also received support from the French government under the France 2030 investment plan, as part of the Excellence Initiative of Aix-Marseille University - A*MIDEX (AMX-19-IET-008 - IPhU).
Financial support from the World Laboratory, the Odon Vallet Foundation and VNSC is gratefully acknowledged. Tran Thi Thai was funded by Vingroup JSC and supported by the Master, PhD Scholarship Programme of Vingroup Innovation Foundation (VINIF), Institute of Big Data, code VINIF.2022.TS.107.        
\end{acknowledgements}

\bibliographystyle{aa} % style aa.bst
\bibliography{bib.bib} % your references Yourfile.bib
\end{document}



%\bibliographystyle{aa}

\begin{appendix}

\section{\textit{Planck} data} \label{AnnexeP}

\subsection{Internal linear combination and constrained internal linear combination weights}  \label{AnnexeP_w}

We applied our pipeline to \textit{Planck} High-Frequency instrument (HFI) Data Release 2\footnote{\url{https://irsa.ipac.caltech.edu/data/Planck/release_2/all-sky-maps/}}  \citep[DR2;][]{Planck_2016d} in order to see how well we could recover the true $y$ using an ILC. We also considered a CILC to deproject CIB by constraining the weights so that they have a null response to the frequency spectrum of the CIB (see Eq. \eqref{eqn:CILC_w}). However, the CIB deprojection is not perfect, because the CIB frequency spectrum depends on the population of the sources and their redshift. 

Looking at the ILC weights gave us information about which channel contributes the most to extracting the tSZ signal and minimising the noise. These were the two conditions that the weights had to satisfy (see Eq. \eqref{eqn:ILC_cond1} and \eqref{eqn:ILC_cond2}).
The ILC weights ponderated by their respective frequency mixing vector value showed the fractional contribution of each frequency channel to the recovered y. The fractional contribution of the $\sim 220$GHz band was null because the value of the mixing vector at this frequency is null. 

For a map-based ILC applied to \textit{Planck} data (see blue points in Fig. \ref{fig:w_Planck}), we concluded from the weight that the 143 GHz channel is the one that contributes the most, with a factor of $\omega _{143}a_{143}\approx 0.7$, to the reconstructed final map as well as to minimising the noise. The 217 GHz channel does not contribute to the final reconstructed $y$ map because the tSZ signal is null at this frequency. However, its high $\omega _{217}$ value makes it the main contributor to noise minimisation. This was expected, as the tSZ signal is absent at this frequency and the map contains all the noise contaminants. Next are the contributions of the $353$ GHz channel and the $545$ GHz channel. The contribution values of those channels are consistent with previous studies \citep{Jens_2018}. In the case where a CILC is applied to \textit{Planck} data to deproject the CIB (orange points, Fig. \ref{fig:w_Planck}), we saw that the weights are almost identical to the ILC ones. This is expected, as \cite{Alonso_2018} also found that deprojecting CIB with a CILC does not result in much less CIB contamination in the reconstructed tSZ profile, arguing that this shows the ILC already does a sufficient job of cleaning the CIB.   

% Figure environment removed

\subsection{Processing of the maps}

We tested the consistency and effect of each of the post-processing steps applied on the map (see Fig. \ref{fig:tess_mask_apo}). The blue curve in the figure shows the ILC-extracted power spectrum computed with a Galactic mask and without tessellation (see Fig. \ref{fig:masks}), while the lighter blue dotted curve shows the same but  with apodisation. The black curve shows the case with the tessellation of the sky and the ILC being applied to each rhombus separately. The grey dotted line shows when apodisation was also applied. For our particular case of tSZ extraction, the mask apodisation only lowers the power spectra of the maps by less than $5\%$. Comparing the power spectrum with tessellation (black) and without (blue) showed that tessellation helps reduce the noise on large scales, up to $\ell \approx 2000$, by a factor of approximately two. At $\ell = 2500$, it increases the noise by around 15\%. This can be understood as the size of the rhombus being small enough to break down regions highly contaminated by unwanted astrophysical components on large scales and to weigh down such regions in order to better recover the tSZ signal. But the rhombus size might be too big to optimally deal with smaller region noise. Moreover, with 192 total rhombuses over the sky, we might be creating border effects on smaller scales and thus increasing the noise residual on those scales. The scales ($\ell > 2200$) are also affected by the $\sim 9.66$' beam.

% Figure environment removed


\section{Constrained ILC: Deprojecting the cosmic microwave background}\label{AnnexeCILC}

When applying an ILC to extract back a Compton-$y$ map, the CMB is the dominant noise residual ($r_{\mathrm{CMB}}$) on large scales (see Fig.\ref{fig:dCMBvsNone}). Using CILC to deproject the CMB, the residual CMB noise is suppressed by three orders of magnitude. However, the additional constraints imposed on the weights to deproject the CMB led to a smaller solution space and therefore increased noise. This led to a one order of magnitude increase of the atmospheric residual noise ($r_{\mathrm{Atmo}}$), which worsens the estimate of the extracted Compton-$y$ power spectrum, especially on small scales, by more than one order of magnitude. This is commented on extensively in Sec. \ref{CILC_results} and was tested using our Skymodel pipeline (see Sec. \ref{des_skymodel}).

 
% Figure environment removed

\section{ILC weights} \label{Annexew}

In our study, we tesselated the sky into 192 rhombi of equal area using the HEALPix nested scheme. This sky separation allowed us to better
adapt to the spatially dependant contaminant of some Astrophysical emissions, such as the Galactic foregrounds. The ILC we applied on each zone resulted in one weight per frequency and per rhombus. An example of a full sky weights map produced and used by our ILC can be seen in Fig. \ref{fig:maps_weights}. 

% Figure environment removed

\section{Simulated maps} \label{AnnexeSky}

In this work, we used the Skymodel pipeline based on PySM and the Websky simulations to generate mock maps of the microwave sky at the frequencies, sensitivities, and instrument beam size of the upcoming SO LAT and FYST (see Table \ref{tab:fsb}). The atmospheric red noise was generated using the noise curves of \cite{Choi_2020_Noise} and the red noise of the neighbouring frequency channels of a given telescope were correlated at 90\% using Eq. \ref{eqn:corr_noise}. The resulting simulated microwave sky maps used can be seen in Fig. \ref{fig:mock-maps-sky}. 

% Figure environment removed


%\twocolumn

\end{appendix}

\end{document}
