\section{\sysname Overview}
\label{sec: overview}

In this section, we examine \sysname at a high level and give its system and security models, as well as several potential applications. A more detailed security analysis is deferred to Appendix \ref{app: security}.

\subsection{System Architecture}
\label{subsec: over-arch}

\sysname aims to be self-contained, managing all identities and credentials under a decentralized system and giving users full sovereignty over their identities and credentials. To achieve this, \sysname provides management of credentials and identities in a highly autonomous manner.

\spar{Credential presentation and verification}
Figure \ref{fig: overview-architecture} illustrates the credential issuance and verification process in \sysname, including the core participants in the ecosystem and the data flow between them. A credential is a collection of cryptographically verifiable claims created by the issuer and granted to the holder. The holder keeps all their credentials and presents them to verifiers for identification or authentication. \sysname's credential system emphasizes the following essential requirements: \cirn{182} \textit{selective disclosure}, \cirn{183} \textit{credential non-transferability}, \cirn{184} \textit{identity privacy}, and \cirn{185} \textit{proof non-forwardability}.

Credentials often contain multiple claims, while only a few are necessary for a single verification. Directly presenting the credential would result in the verifier obtaining far more information than required. \sysname introduces \cirn{182} \textit{selective disclosure} by allowing a holder to prove to a verifier that it holds a credential containing specific claims while not revealing other contents.

To offer \cirn{183} \textit{credential non-transferability}, \sysname forces the presenter to prove that it holds the private key corresponding to the credential subject, ensuring that the presenter is indeed the holder. Nevertheless, the holder must disclose its identifier for verification, which results in the verifier establishing a \textit{identifier-based correlation}. To provide \cirn{184} \textit{identity privacy}, \sysname uses information from the Verifiable Data Registry (VDR) to identify the presenter, without disclosing its identifier to the verifier.

Since proofs for verification are publicly accessible, a malicious verifier can impersonate the holder by forwarding proofs. None of the existing identity systems address this issue, allowing the verifier who gets the public inputs to replay the presentation process. \sysname introduces two methods to assure \cirn{185} \textit{proof non-forwardability} by binding the verifier in every presentation.

% Figure environment removed

\spar{Sybil-resistant identifier management}
Another essential building block of \sysname is the identifier management system. Specifically, \sysname tries to build a robust and secure identity system in terms of \textit{Sybil resistance}.

Sybil attack is one of the most severe problems faced by identity systems. Traditional approaches based on collateral or offline identification fail to solve the problem but add more burden to users \cite{borge2017proof, buterin2017casper}. Instead, \sysname seeks to solve the problem without relying on any centralized entity. To this end, \sysname uses user-held credentials as the basis for identity assessment, since people with more fulfillment records can be considered more trustworthy than those with empty ones. As a naive example, an airdrop provider using \sysname would only grant airdrop to those with credentials for at least three participations in the recent campaign initiated by itself \cite{weyl2022decentralized}.

The above mechanism, as a bootstrap, is insufficient to mitigate Sybil attacks. Users may utilize multiple identifiers as pseudonyms\footnote{In the following, we will use the term \textit{pseudonym} as an informal surrogate for \textit{identifier}, if not otherwise specified.} in different campaigns to ensure identity privacy. Meanwhile, a user may simultaneously use credentials scattered across multiple pseudonyms to provide a proof of claims. A critical insight is that the association of identifiers occurs at the moment of presenting credentials under multiple identifiers. To take advantage of this information, \sysname requires an additional \textit{proof of association} when a user presents these credentials to ensure that all identifiers have been associated and committed to the blockchain.

This \textit{identifier association} mechanism enables multiple identifiers held by the same user to be implicitly aggregated. Users do not need to disclose exactly which identifiers are associated, thus eliminating the risk of identity privacy breaches. Proof of association is required for each verification, making associated identifiers valid only once for each Sybil-sensitive campaign. As a user becomes deeply involved in the system and accumulates credentials under different identifiers, most of its identifiers will eventually be associated. Meanwhile, the cost to launch a Sybil attack will increase significantly, as an adversary will need to continuously engage in campaigns to cultivate its identifiers.

\spar{Key recovery and revocation}
Most existing schemes rely on centralized trusted parties to host and recover lost private keys \cite{maram2021candid,jarecki2014round}, which requires additional security assurances. Instead, \sysname eliminates dependencies on any authorities or guardians, providing key recovery and revocation using the information committed to the public registry but only known to the holder.

\sysname builds an identifier association mechanism that associates and records multiple identifiers owned by the same user on the blockchain. The crux is that only the user knows which identifiers are associated within a particular on-chain commitment. Therefore, the user can prove to the VDR the knowledge of an associated identifier, indicating its ownership of a specific identifier whose key was lost. Once confirmed, the VDR could refresh the public key as the user requires. Another advantage of this mechanism is the ability to proactively update the key in case of a compromise without urgently transferring assets to prevent loss.

\subsection{System and Security Model}
\label{subsec: over-model}

We outline the system and security model of \sysname and describe its workflow and security requirements.

\spar{System model}
As shown in Figure \ref{fig: overview-architecture}, \sysname considers a decentralized identity system consisting of \textit{issuers}, \textit{holders}, and \textit{verifiers}. Each entity $\mathcal{E}$ can create key pairs $(\mathsf{pk}_\mathcal{E},\mathsf{sk}_\mathcal{E})$ and publish the public key $\mathsf{pk}_\mathcal{E}$ to a decentralized ledger called a \textit{verifiable data registry (VDR)}. Compatible with the Decentralized Identifier (DID) specification \cite{world2022decentralized}, in \sysname, the VDR generates a DID document containing the public key $\mathsf{pk}_\mathcal{E}$ and publishes an identifier $\mathsf{id}_\mathcal{E}$ that is addressable to that document.

A \textit{credential} in \sysname is a collection of cryptographically verifiable claims given by the issuer to the holder, who presents it to the verifier for identification or authentication. Similar to the Verifiable Credential (VC) specification \cite{world2022verifiable}, each credential in \sysname contains the identifiers $\mathsf{id}_\mathcal{I}, \mathsf{id}_\mathcal{H}$ of the issuer $\mathcal{I}$ and the holder $\mathcal{H}$, as well as a set of claims $\{s_1,s_2,\dots,s_n\}$, where each $s_i$ is a key-value pair that represents an attribute-value mapping. A credential should also contain the issuer's signature $\sigma_\mathcal{I}$, generated by the issuer using its private key $\mathsf{sk}_\mathcal{I}$ to sign the rest of the credential. In addition, non-mandatory fields such as issuance and expiration time are also allowed, which would be omitted in subsequent narratives.

\spar{Security model}
We assume that all entities in \sysname, including the holder, issuer, and verifier, are semi-honest, i.e., they follow the protocol when interacting but save all intermediate states of the execution. We consider a static malicious adversary $\mathcal{A}$. The compromised entity deviates from the protocol arbitrarily and reveals its state to $\mathcal{A}$.

Intuitively, the issuer and the verifier would attempt to conspire to infer the identity of a credential presenter. The verifier would attempt to forward a presented credential to other verifiers to impersonate the holder and pass the validation. The holder would distribute its credentials to others to make them pass the verification, but would not transfer the private keys, as this is equivalent to giving up all on-chain assets. We assume that the verifier is rational and willing to prevent Sybil attacks when needed.

We assume that the VDR is implemented by a smart contract on the blockchain, and thus its security comes from the standard assumption of the blockchain, i.e., it is tamper-proof and consistent. In addition, we assume that all entities can view all transaction records on the blockchain. We assume that network communications between entities do not leak the contents to third parties and that all data maintained by entities are secure and private.

\subsection{Applications and Use Cases}
\label{subsec: over-app}

The strong privacy and security provided by \sysname enable Web3 to map the relationship between participants and identities, tracking the flow of identification in a privacy-preserving way. We list several examples of the innovation \sysname brings to the Web3 ecosystem.

\spar{Identification and Know Your Customer (KYC)}
Laws in various jurisdictions require service providers to meet strict compliance responsibilities to identify or qualify their customers, for example, through KYC. Authentication often requires the user to provide personally identifiable information (PII), which increases the risk of identity privacy breaches \cite{xu2020blockchain}. An example is that a service provider only wants to verify that the user is of a certain age. Yet, current authentication process requires users to provide a full credential, revealing information about names, ID numbers, and other sensitive information far more than needed. Selective disclosure provided by \sysname can offer a more privacy-protective authentication process without changing the established protocols on the issuing side. The user only needs to provide minimized data to convince the verifier.

\spar{Community convening and on-chain democracy}
Web3's community convening currently relies on a token distribution mechanism called airdrops, which crudely gives users tokens to get the community off the ground \cite{wahby2020airdrop}. This mechanism has proven to be vulnerable to Sybil attacks as it lacks identification of participants. It has been shown that Sybil attacks in the Web3 ecosystem cannot be completely eliminated but can only be mitigated or resisted by expending significant capital \cite{platt2021sybil}. Nevertheless, the identifier association mechanism introduced by \sysname makes Sybil attacks much more costly while not requiring any compromise from honest users. This more robust anti-Sybil mechanism makes on-chain democracy and quadratic funding in the Web3 ecosystem truly possible, allowing for smoother and more secure interaction in trustless environments.

\spar{Accountability}
Despite giving users unprecedented identity and credential sovereignty, existing decentralized identity systems also lose accountability. One of the most apparent manifestations is that most of them are unable to actively revoke credentials, much less discipline participants for malicious behaviors \cite{abraham2020revocable}. \sysname can provide more robust accountability without changing the underlying Web3 trustless architecture, empowering issuers to revoke the credentials issued by it proactively. Moreover, \sysname's identity mechanism allows the judiciary to restrict or block all identifiers of malicious users. Together with the identifier association mechanism, \sysname can provide more granular restrictions and disciplinary actions against malicious users without any compromise on privacy.