\section{Security Analysis}
\label{app: security}

In Appendix \ref{app: security}, we give security analysis for the protocols involved in \sysname. We will demonstrate several attack games between a challenger and an adversary $\mathcal{A}$, where the challenger follows a fixed protocol while the adversary $\mathcal{A}$ can follow an arbitrary but efficient protocol.

\subsection{Credential Presentation}
\label{subapp: security-credential}

Credential presentation in \sysname has the properties of \textit{correctness}, \textit{soundness} (cf. Theorem \ref{thm: credential-presentation-correctness-soundness}), \textit{credential privacy} (cf. Theorem \ref{thm: credential-presentation-privacy}) and \textit{identifier indistinguishability} (cf. Theorem \ref{thm: credential-identity-indistinguishability}).

\begin{theorem}[Correctness and Soundness]
\label{thm: credential-presentation-correctness-soundness}
Let $s\in\mathsf{cred}$ be a claim about a presenter $\mathcal{H}$ satisfying $\phi(s)=1$. If $\mathcal{H}$ generates a proof $\pi_\mathsf{c}$ correctly for Relation \ref{eq: relation-basic} and interacts with a verifier $\mathcal{V}$ according to the protocol in Section \ref{subsec: cred-selective}, the verifier $\mathcal{V}$ will accept the presentation with overwhelming probability. If $\phi(s)\neq 1$, $\mathcal{V}$ will accept the presentation with negligible probability.
\end{theorem}

\begin{proof}[Proof sketch]
We first argue the correctness of credential presentation. $\mathcal{H}$ constructs a proof $\pi_\mathsf{c}$ with respect to Relation \ref{eq: relation-basic} using a claim $s$ which satisfies $\phi(s)=1$. The correctness of the presentation follows directly from the completeness of the NIZK scheme $\mathsf{\Pi}$ (cf. Definition \ref{def: zk-completeness}).

Next, we argue the soundness of credential presentation, i.e., $\mathcal{H}$ cannot use a $s$ such that $\phi(s)\neq 1$ to construct a valid proof $\pi_\mathsf{c}$ with respect to Relation \ref{eq: relation-basic} and convince $\mathcal{V}$ to accept the presentation. Assuming that $\mathcal{V}$ accepts the proof $\pi_\mathsf{c}$ at the end of the interaction, then by Definition \ref{def: zk-soundness} there must exist with overwhelming probability an extractor $\mathsf{Ext}$ that outputs a tuple of witnesses $(\mathsf{cred}', u_\mathsf{c}')$ satisfying:
	\begin{equation}
	\label{eq: relation-proof-credential-soundness}
		\begin{aligned}
			& \mathsf{\Gamma}.\mathsf{Open}(\mathsf{cred}', c_\mathsf{c}, u_\mathsf{c}')=1\\
			& \land\quad \mathsf{VerSign}(\mathsf{pk}_\mathcal{I}, \sigma_\mathcal{I},\mathsf{cred}')=1.
		\end{aligned}
	\end{equation}
	The case must be $\mathsf{cred}'\neq\mathsf{cred}$ and there exists a $s'\in\mathsf{cred}'$ such that $\phi(s')=1$. However, since the signature $\sigma_\mathcal{I}$ with respect to the credential has been made public, this breaks the correctness of the signature verification algorithm $\mathsf{VerSign}$. On the other hand, since the commitment $c_\mathsf{c}$ of the credential has been made public, this also breaks the binding property of the commitment scheme $\mathsf{\Gamma}$ (cf. Definition \ref{def: binding}).
\end{proof}

In the following, we use an attack game to define credential privacy during the presentation. Our definition captures the intuition that no efficient adversary $\mathcal{A}$ has access to any knowledge about credentials and claims other than the public inputs.

\begin{game}[Credential Privacy]
\label{game: credential-privacy}
Given an efficient adversary $\mathcal{A}$, we define two experiments, Experiment $0$ and Experiment $1$. For $b=0,1$, we define:\\
\noindent\textbf{Experiment} $b$:
\begin{itemize}
	\item The challenger randomly sample a $u_\mathsf{c}\overm{R}\leftarrow\mathbb{Z}_p$ and commits to $\mathsf{cred}$ by $c_\mathsf{c} \overm{R}\leftarrow \mathsf{\Gamma}.\mathsf{Commit}(\mathsf{cred},u_\mathsf{c})$.
	\item The challenger computes $\pi_\mathsf{c}$ as follows:
	\begin{itemize}
		\item If $b=0$, calculates as in Equation \ref{eq: proof-credential-selective-disclosure}.
		\item If $b=1$, runs a simulator on the public inputs by:
			\[
				\pi_\mathsf{c}
					\overm{R}\leftarrow
					\mathsf{Sim}\big(
						R,
						(c_\mathsf{c}, \sigma_\mathcal{I}, \mathsf{id}_\mathcal{I}, \mathsf{pk}_\mathcal{I}, \phi)
					\big),
			\]
			where $R$ refers to Relation \ref{eq: relation-basic}.
	\end{itemize}
	\item The challenger sends $\pi_\mathsf{c}$ to $\mathcal{A}$.
	\item The adversary $\mathcal{A}$ computes and outputs $\hat{b}\in\{0,1\}$.
\end{itemize}

For $b=0,1$, let $E_b$ be the event that $\mathcal{A}$ outputs $1$ in Experiment $b$. We define $\mathcal{A}$'s advantage with respect to $\mathsf{\Pi}$ as
\[
	\mathrm{CPP}\mathsf{adv}[\mathcal{A}, \mathsf{\Pi}] := \big\lvert \Pr[E_0] - \Pr[E_1] \big\rvert.
\]
\end{game}

\begin{theorem}[Credential Privacy]
\label{thm: credential-presentation-privacy}
Given a NIZK scheme $\mathsf{\Pi}$ satisfying Definition \ref{def: zk-snark}, the value of $\mathrm{CPP}\mathsf{adv}[\mathcal{A}, \mathsf{\Pi}]$ is negligible for all efficient adversaries $\mathcal{A}$.
\end{theorem}

\begin{proof}[Proof sketch]
The theorem follows directly from Definition \ref{def: zk-zero-knowledge}, which constructs a simulator $\mathsf{Sim}$ to model the transcript of proof $\pi_\mathsf{c}$. Since $\mathsf{\Pi}$ has zero knowledge, the transcript generated by $\mathsf{Sim}$ is necessarily indistinguishable from $\pi_\mathsf{c}$, indicating that $\mathrm{CPP}\mathsf{adv}[\mathcal{A}, \mathsf{\Pi}]\leq\mathsf{negl}(\lambda)$ holds for all efficient adversaries $\mathcal{A}$. Thus the protocol for credential presentation does not reveal any information about $\mathsf{cred}$ other than the statements contained in the public inputs.
\end{proof}

%In the following, we use an attack game to define identity indistinguishability during the credential presentation. This definition indicates that no efficient adversary $\mathcal{A}$ can infer or distinguish the identities of the presenter given the transcripts and proofs during interactions.

\begin{game}[Identitfier Indistinguishability]
\label{game: credential-identifier-indistinguishability}
Given an efficient adversary $\mathcal{A}$, we define two experiments, Experiment $0$ and Experiment $1$. For $b=0,1$, we define:\\
\noindent\textbf{Experiment} $b$:
\begin{itemize}
	\item The challenger computes $\pi_\mathsf{c}$ as follows:
		\begin{itemize}
			\item If $b=0$, samples $u_\mathsf{c}^{(0)},u_\mathsf{c}^{(1)}\overm{R}\leftarrow\mathbb{Z}_p$, commits to $\mathsf{cred}$ by $c_\mathsf{c}^{(0)} \overm{R}\leftarrow \mathsf{\Gamma}.\mathsf{Commit}(\mathsf{cred},u_\mathsf{c}^{(0)})$ and $c_\mathsf{c}^{(1)} \overm{R}\leftarrow \mathsf{\Gamma}.\mathsf{Commit}(\mathsf{cred},u_\mathsf{c}^{(1)})$, and calculates two proofs $\pi_\mathsf{c}^{(0)}$ and $\pi_\mathsf{c}^{(1)}$ as in Equation \ref{eq: proof-credential-selective-disclosure}.
			\item If $b=1$, samples $u_\mathsf{c}^{(0)},u_\mathsf{c}^{(1)}\overm{R}\leftarrow\mathbb{Z}_p$, commits to two credentials by $c_\mathsf{c}^{(0)} \overm{R}\leftarrow \mathsf{\Gamma}.\mathsf{Commit}(\mathsf{cred}^{(0)},u_\mathsf{c}^{(0)})$ and $c_\mathsf{c}^{(1)} \overm{R}\leftarrow \mathsf{\Gamma}.\mathsf{Commit}(\mathsf{cred}^{(1)},u_\mathsf{c}^{(1)})$, both of which are issued by $\mathcal{I}$ and contain a claim $s$ such that $\phi(s)=1$ but with different holder identifiers $\mathsf{id}_\mathcal{H}^{(0)}$ and $\mathsf{id}_\mathcal{H}^{(1)}$, and calculates two proofs $\pi_\mathsf{c}^{(0)}$ and $\pi_\mathsf{c}^{(1)}$ as in Equation \ref{eq: proof-credential-selective-disclosure}.
		\end{itemize}
	\item The challenger sends $\pi_\mathsf{c}^{(0)}$ and $\pi_\mathsf{c}^{(1)}$ to $\mathcal{A}$.
	\item The adversary $\mathcal{A}$ computes and outputs $\hat{b}\in\{0,1\}$.
\end{itemize}

For $b=0,1$, let $E_b$ be the event that $\mathcal{A}$ outputs $1$ in Experiment $b$. We define $\mathcal{A}$'s identity distinguish advantage with respect to $\mathsf{\Pi}$ and $\mathsf{\Gamma}$ as
\[
	\mathrm{IDD}\mathsf{adv}[\mathcal{A}, \mathsf{\Pi}, \mathsf{\Gamma}] := \big\lvert \Pr[E_0] - \Pr[E_1] \big\rvert.
\]
\end{game}

\begin{theorem}[Identifier Indistinguishability]
\label{thm: credential-identity-indistinguishability}
Given a NIZK scheme $\mathsf{\Pi}$ satisfying Definition \ref{def: zk-snark} and a commitment scheme $\mathsf{\Gamma}$ satisfying Definition \ref{def: commitment}, the value of $\mathrm{IDD}\mathsf{adv}[\mathcal{A}, \mathsf{\Pi},  \mathsf{\Gamma}]$ is negligible for all efficient adversaries $\mathcal{A}$.
\end{theorem}

\begin{proof}
We argue this with a series of hybrid games.

\spar{Hybrid $\mathbf{H}_1$}
This is equivalent to Experiment $0$ of Attack Game \ref{game: credential-identifier-indistinguishability}.

\spar{Hybrid $\mathbf{H}_2$}
This is the same as \textbf{Hybrid} $\mathbf{H}_1$, except that in this game, instead of using the same credential $\mathsf{cred}$, the challenger uses two credentials $\mathsf{cred}^{(0)}$ and $\mathsf{cred}^{(1)}$ issued by $\mathcal{I}$ with the same holder identifier $\mathsf{id}_\mathcal{H}$ and claim $s$ such that $\phi(s)=1$. The challenger commits to them by
	\begin{align*}
		& c_\mathsf{c}^{(0)}
			\overm{R}\leftarrow
			\mathsf{\Gamma}.\mathsf{Commit}(
				\mathsf{cred}^{(0)}, u_\mathsf{c}^{(0)}
			),\\
		& c_\mathsf{c}^{(1)}
			\overm{R}\leftarrow
			\mathsf{\Gamma}.\mathsf{Commit}(
				\mathsf{cred}^{(1)}, u_\mathsf{c}^{(1)}
			)
	\end{align*}
	using two randomly sampled $u_\mathsf{c}^{(0)}$ and $u_\mathsf{c}^{(1)}$, and calculates two proofs  $\pi_\mathsf{c}^{(0)}$ and $\pi_\mathsf{c}^{(1)}$ as in Equation \ref{eq: proof-credential-selective-disclosure}. Given the hiding property of the commitment scheme $\mathsf{\Gamma}$ (cf. Definition \ref{def: hiding}), the commitments $c_\mathsf{c}^{(0)}$ and $c_\mathsf{c}^{(1)}$ in $\mathbf{H}_1$ and $\mathbf{H}_2$ are indistinguishable. Since the NIZK scheme $\mathsf{\Pi}$ has zero knowledge (cf. Definition \ref{def: zk-zero-knowledge}), the proofs $\pi_\mathsf{c}^{(0)}$ and $\pi_\mathsf{c}^{(1)}$ are also indistinguishable in $\mathbf{H}_1$ and $\mathbf{H}_2$. Thus, the execution of $\mathbf{H}_1$ and $\mathbf{H}_2$ are indistinguishable from the view of $\mathcal{A}$.

\spar{Hybrid $\mathbf{H}_3$}
This is the same as \textbf{Hybrid} $\mathbf{H}_2$, except that in this game, the challenger uses two credentials $\mathsf{cred}^{(0)}$ and $\mathsf{cred}^{(1)}$ issued by $\mathcal{I}$ with different holder identifiers $\mathsf{id}_\mathcal{H}^{(0)}$ and $\mathsf{id}_\mathcal{H}^{(1)}$, and claim $s$ such that $\phi(s)=1$. The challenger commits to the credentials by
	\begin{align*}
		& c_\mathsf{c}^{(0)}
			\overm{R}\leftarrow
			\mathsf{\Gamma}.\mathsf{Commit}(
				\mathsf{cred}^{(0)}, u_\mathsf{c}^{(0)}
			),\\
		& c_\mathsf{c}^{(1)}
			\overm{R}\leftarrow
			\mathsf{\Gamma}.\mathsf{Commit}(
				\mathsf{cred}^{(1)}, u_\mathsf{c}^{(1)}
			)
	\end{align*}
	using two randomly sampled $u_\mathsf{c}^{(0)}$ and $u_\mathsf{c}^{(1)}$, and calculates two proofs  $\pi_\mathsf{c}^{(0)}$ and $\pi_\mathsf{c}^{(1)}$ as in Equation \ref{eq: proof-credential-selective-disclosure}. The commitments $c_\mathsf{c}^{(0)}$, $c_\mathsf{c}^{(1)}$ and the proofs $\pi_\mathsf{c}^{(0)}$ and $\pi_\mathsf{c}^{(1)}$ are indistinguishable in $\mathbf{H}_2$ and $\mathbf{H}_3$ given the hiding property of $\mathsf{\Gamma}$ and zero-knowledge property of $\mathsf{\Pi}$ (cf. Definition \ref{def: zk-zero-knowledge} and \ref{def: hiding}). All other elements in the transcripts of both games are the same, indicating that the execution of $\mathbf{H}_2$ and $\mathbf{H}_3$ are indistinguishable from the view of $\mathcal{A}$.

$\mathbf{H}_3$ is Experiment $1$ in the Attack Game \ref{game: credential-identifier-indistinguishability}. Thus we prove that Experiment $0$ and Experiment $1$ are indistinguishable from the perspective of $\mathcal{A}$, i.e., $\mathrm{IDD}\mathsf{adv}[\mathcal{A}, \mathsf{\Pi}, \mathsf{\Gamma}]\leq \mathsf{negl}(\lambda)$ holds for all efficient adversaries $\mathcal{A}$. Thus, $\mathcal{A}$ cannot infer the identities of a presenter during credential presentation.
\end{proof}

\subsection{Identifier Association}
\label{subapp: security-identifier}

Identifier association and update in \sysname has the properties of \textit{correctness}, \textit{soundness} (cf. Theorem \ref{thm: identifier-association-correctness-soundness}), and \textit{identifier unlinkability} (cf. Theorem \ref{thm: identity-identitfier-unlinkability}).

\begin{theorem}[Correctness and Soundness]
\label{thm: identifier-association-correctness-soundness}
Let
\[
	\mathsf{id}_\mathcal{H}
		\in
		\left\{
			\mathsf{id}_\mathcal{H}^{(1)}, \dots, \mathsf{id}_\mathcal{H}^{(n)}
		\right\}
\]
be an identifier, and all $\mathsf{id}_\mathcal{H}^{(i)}$ have been registered and associated in the VDR by
\[
	\mathsf{id}_\mathsf{a}
		= H_1\left(
			\mathsf{id}_\mathcal{H}^{(1)},\dots,\mathsf{id}_\mathcal{H}^{(n)}
		\right).
\]
If $\mathcal{H}$ generates the proof $\pi_\mathsf{pre}$ with respect to $\mathsf{id}_\mathcal{H}$ correctly for Relation \ref{eq: relation-id-presentation} and interacts with the verifier $\mathcal{V}$ according to the protocol in Section \ref{subsec: identity-id-presentation}, $\mathcal{V}$ will accept the identifier presentation with overwhelming probability. If $\mathsf{id}_\mathcal{H}$ is not equal to any of the $\mathsf{id}_\mathcal{H}^{(i)}$, or $\mathsf{id}_\mathsf{a}$ is not registered or associated in the VDR according to the protocol in Section \ref{subsec: identity-id-association}, $\mathcal{V}$ will accept with negligible probability.
\end{theorem}

\begin{proof}[Proof sketch]
The correctness of the association identifier presentation is obvious. If $\mathcal{H}$ provides a $\mathsf{id}_\mathcal{H}$ that satisfies Relation \ref{eq: relation-id-presentation}, then by the completeness of $\mathsf{\Pi}$ (cf. Definition \ref{def: zk-completeness}), it must be able to construct a proof $\pi_\mathsf{pre}$ that convinces $\mathcal{V}$ to accept $\mathsf{id}_\mathcal{H}$.

Next, we argue for soundness. Assuming that $\mathcal{V}$ accepts $\pi_\mathsf{pre}$ at the end of interaction, then by Definition \ref{def: zk-soundness}, there must exist with overwhelming probability an extractor $\mathsf{Ext}$ that outputs a tuple of witnesses $\big( \rho_\mathsf{a}', \mathsf{id}_\mathsf{a}', \{\mathsf{id}_\mathcal{H}'^{(i)} \} \big)$ such that:
	\begin{equation}
	\label{eq: relation-identity-association-soundness}
		\begin{aligned}
			& \mathsf{M}.\mathsf{Verify}(r_\mathsf{a}, \mathsf{id}_\mathsf{a}', \rho_\mathsf{a}') = 1\\
			& \land\quad \mathsf{id}_\mathsf{a}' = H_1\left(\mathsf{id}_\mathcal{H}'^{(1)},\dots,\mathsf{id}_\mathcal{H}'^{(n)}\right)\\
			& \land\quad n_{\mathsf{a}\varepsilon} = H_1 \left( \mathsf{id}_\mathcal{H}'^{(1)}, \dots, \mathsf{id}_\mathcal{H}'^{(n)}, \mathsf{id}_\varepsilon \right) \\
			& \land\quad \mathsf{id}_\mathcal{H} \in \left\{ \mathsf{id}_\mathcal{H}'^{(1)}, \dots, \mathsf{id}_\mathcal{H}'^{(n)} \right\}.
		\end{aligned}
	\end{equation}
	On this basis, if $\mathsf{id}_\mathcal{H}$ is not equal to any of the $\mathsf{id}_\mathcal{H}^{(i)}$, then the case must be:
	\[
		\left\{ \mathsf{id}_\mathcal{H}'^{(1)}, \dots, \mathsf{id}_\mathcal{H}'^{(n)} \right\}
		\neq
		\left\{ \mathsf{id}_\mathcal{H}^{(1)}, \dots, \mathsf{id}_\mathcal{H}^{(n)} \right\}.
	\]
	Since $\mathsf{id}_\varepsilon$ is the public input generated by $\mathcal{V}$, the statement
	\[
		n_{\mathsf{a}\varepsilon} = H_1 \left( \mathsf{id}_\mathcal{H}'^{(1)}, \dots, \mathsf{id}_\mathcal{H}'^{(n)}, \mathsf{id}_\varepsilon \right)
	\]
	in Relation \ref{eq: relation-identity-association-soundness} breaks the collision resistance of the hash function $H_1$. On the other hand, if $\mathsf{id}_\mathsf{a}$ is not registered or associated with the VDR, then the case must be:
	\[
		\begin{aligned}
			& \mathsf{id}_\mathsf{a}' = \mathsf{id}_\mathsf{a} = H_1\left(\mathsf{id}_\mathcal{H}^{(1)}, \dots, \mathsf{id}_\mathcal{H}^{(n)}\right)\\
			& \land\quad \mathsf{M}. \mathsf{Verify}(r_\mathsf{a}, \mathsf{id}_\mathsf{a}', \rho_\mathsf{a}') = 1,
		\end{aligned}
	\]
	and this apparently breaks the correctness of the Merkle tree construction.
\end{proof}

%In the following, we use an attack game to define the identity privacy of associated identifier presentation. Our definition indicates that no efficient adversary $\mathcal{A}$ can link a presenter's identifier in two campaigns given transcripts and proofs during interactions.

\begin{game}[Identity Unlinkability]
\label{game: identity-identifier-unlinkability}
Given an efficient adversary $\mathcal{A}$, we define two experiments, Experiment $0$ and Experiment $1$. For $b=0,1$, we define:\\
\noindent\textbf{Experiment} $b$:
\begin{itemize}
	\item The challenger proceeds as follows:
		\begin{itemize}
			\item If $b = 0$, registers and associates an identifier $\mathsf{id}^{(0)}$ according to the protocol in Section \ref{subsec: identity-id-association}, generating an associated identifier $\mathsf{id}_\mathsf{a}$ and recording it in the Merkle tree of the VDR.
			\item If $b = 1$, registers two identifiers $\mathsf{id}^{(0)}$ and $\mathsf{id}^{(1)}$ and derives two associated identifiers $\mathsf{id}_\mathsf{a}^{(0)}$ and $\mathsf{id}_\mathsf{a}^{(1)}$ separately, recording them in the Merkle tree of the VDR.
		\end{itemize}
	\item The adversary $\mathcal{A}$ randomly samples $\mathsf{id}_\varepsilon^{(0)}, \mathsf{id}_\varepsilon^{(1)} \overm{R}\leftarrow \mathbb{Z}_p$ and sends them to the challenger.
	\item Upon receiving $\mathsf{id}_\varepsilon^{(0)}$ and $\mathsf{id}_\varepsilon^{(1)}$, the challenger presents proceeds as follows:
		\begin{itemize}
			\item If $b = 0$, presents $\mathsf{id}^{(0)}$ in both campaigns;
			\item If $b = 1$, presents $\mathsf{id}^{(0)}$ and $\mathsf{id}^{(1)}$ in these two campaigns, respectively.
		\end{itemize}
		The challenger generates two proofs $\pi_\mathsf{pre}^{(0)}$ and $\pi_\mathsf{pre}^{(1)}$ for Relation \ref{eq: relation-id-presentation} in the two campaigns, sending them to $\mathcal{A}$, along with the public inputs $r_\mathsf{a}^{(0)}$, $r_\mathsf{a}^{(1)}$,  $n_{\mathsf{a}\varepsilon}^{(0)}$ and $n_{\mathsf{a}\varepsilon}^{(1)}$.
	\item The adversary $\mathcal{A}$ computes and outputs $\hat{b}\in\{0,1\}$.
\end{itemize}

For $b=0,1$, let $E_b$ be the event that $\mathcal{A}$ outputs $1$ in Experiment $b$. We define $\mathcal{A}$'s identity linkage advantage with respect to $\mathsf{\Pi}$ as
\[
	\mathrm{IDL}\mathsf{adv}[\mathcal{A}, \mathsf{\Pi}] := \big\lvert \Pr[E_0] - \Pr[E_1] \big\rvert.
\]
\end{game}

\begin{theorem}[Identity Unlinkability]
\label{thm: identity-identitfier-unlinkability}
Given a NIZK scheme $\mathsf{\Pi}$ satisfying Definition \ref{def: zk-snark}, the value of $\mathrm{IDL}\mathsf{adv}[\mathcal{A}, \mathsf{\Pi}]$ is negligible for all efficient adversaries $\mathcal{A}$.
\end{theorem}

\begin{proof}
We argue this with a series of hybrid games.

\spar{Hybrid $\mathbf{H}_1$}
This is equivalent to Experiment $0$ of Attack Game \ref{game: identity-identifier-unlinkability}.

\spar{Hybrid $\mathbf{H}_2$}
This is the same as \textbf{Hybrid} $\mathbf{H}_1$, except that in this game, instead of using the same $\mathsf{id}^{(0)}$, the challenger uses two identifiers $\mathsf{id}^{(0)}$ and $\mathsf{id}^{(1)}$ which have been aggregated into the same associated identifier $\mathsf{id}_\mathsf{a}$. The challenger registers and associates $\mathsf{id}^{(0)}$ and $\mathsf{id}^{(1)}$ according to the protocol in Section \ref{subsec: identity-id-association}, and generates a proof $\pi_\mathsf{a}$ to convince the VDR about the association. Given the zero-knowledge property of $\mathsf{\Pi}$ (cf. Definition \ref{def: zk-zero-knowledge}) and the randomness of the $\mathsf{\Pi}.\mathsf{Prove}$ algorithm, the proofs $\pi_\mathsf{pre}^{(0)}$ and $\pi_\mathsf{pre}^{(1)}$ are indistinguishable between $\mathbf{H}_1$ and $\mathbf{H}_2$. Given the collision-resistance of $H_1$, $n_{\mathsf{a}\varepsilon}^{(0)}$ and $n_{\mathsf{a}\varepsilon}^{(1)}$ are indistinguishable between the two games with overwhelming probability. The indistinguishability of $\mathsf{id}_\mathcal{H}$ is guaranteed by Relation \ref{eq: relation-holder-identity} and \ref{eq: relation-vdr} (cf. Theorem \ref{thm: credential-identity-indistinguishability}). All other elements in the transcripts of both games are the same, indicating that the execution of $\mathbf{H}_1$ and $\mathbf{H}_2$ are indistinguishable from the view of $\mathcal{A}$.

\spar{Hybrid $\mathbf{H}_3$}
This is the same as \textbf{Hybrid} $\mathbf{H}_2$, except that in this game, the challenger uses two identifiers $\mathsf{id}^{(0)}$ and $\mathsf{id}^{(1)}$ which have been aggregated into two different associated identifiers $\mathsf{id}_\mathsf{a}^{(0)}$ and $\mathsf{id}_\mathsf{a}^{(1)}$. The challenger involves in the two-phase interaction with the VDR, generating two proofs $\pi_\mathsf{a}^{(0)}$ and $\pi_\mathsf{a}^{(1)}$ to convince it for the associations. Given the zero-knowledge property of $\mathsf{\Pi}$ and the collision-resistance of $H_1$, the proofs $\pi_\mathsf{pre}^{(0)}$, $\pi_\mathsf{pre}^{(1)}$ and nullifiers $n_{\mathsf{a}\varepsilon}^{(0)}$, $n_{\mathsf{a}\varepsilon}^{(1)}$ are indistinguishable between $\mathbf{H}_2$ and $\mathbf{H}_3$. Also, the indistinguishability of $\mathsf{id}_\mathcal{H}$ is guaranteed by Relation \ref{eq: relation-holder-identity} and \ref{eq: relation-vdr}. Thus, the execution of $\mathbf{H}_2$ and $\mathbf{H}_3$ are indistinguishable from the view of $\mathcal{A}$.

$\mathbf{H}_3$ is actually Experiment $1$ in the Attack Game \ref{game: identity-identifier-unlinkability}. We can conclude that Experiment $0$ and $1$ are indistinguishable from the perspective of $\mathcal{A}$, i.e., $\mathrm{IDL}\mathsf{adv}[\mathcal{A},\mathsf{\Pi}]\leq\mathsf{negl}(\lambda)$ holds for all efficient adversaries $\mathcal{A}$. Thus, $\mathcal{A}$ cannot link the identifiers of a presenter in two campaigns.
\end{proof}

\subsection{Key Recovery}
\label{subapp: security-key}

Key recovery mechanism in \sysname has the properties of \textit{correctness}, \textit{soundness} (cf. Theorem \ref{thm: key-recovery-correctness-soundness}), and \textit{key privacy} (cf. Theorem \ref{thm: key-recovery-privacy}).

\begin{theorem}[Correctness and Soundness]
\label{thm: key-recovery-correctness-soundness}
Let
\[
	\mathsf{id}_\mathcal{H}
		\in
		\left\{
			\mathsf{id}_\mathcal{H}^{(1)}, \dots, \mathsf{id}_\mathcal{H}^{(n)}
		\right\}
\]
be an identifier, and all $\mathsf{id}_\mathcal{H}^{(i)}$ have been registered and associated in the VDR by
\[
	\mathsf{id}_\mathsf{a}
		= H_1\left(
			\mathsf{id}_\mathcal{H}^{(1)},\dots,\mathsf{id}_\mathcal{H}^{(n)}
		\right).
\]
If $\mathcal{H}$ generates the proof $\pi_\mathsf{key}$ with respect to $\mathsf{id}_\mathcal{H}$ correctly for Relation \ref{eq: relation-key-refresh-associated} and \ref{eq: relation-key-refresh-key-pair}, and interacts with the verifier $\mathcal{V}$ according to the protocol in Section \ref{subsec: key-refresh}, $\mathcal{V}$ will accept the identifier presentation with overwhelming probability. If $\mathsf{id}_\mathcal{H}$ is not equal to any of the $\mathsf{id}_\mathcal{H}^{(i)}$, or $\mathsf{id}_\mathsf{a}$ is not registered or associated in the VDR according to the protocol in Section \ref{subsec: identity-id-association}, $\mathcal{V}$ will accept with negligible probability.
\end{theorem}

\begin{proof}[Proof sketch]
The argument for correctness is essentially similar to the proof of Theorem \ref{thm: identifier-association-correctness-soundness}. If $\mathcal{H}$ provides an associated identifier $\mathsf{id}_\mathsf{a}$ accommodating $\mathsf{id}_\mathcal{H}$ and satisfying Relation \ref{eq: relation-key-refresh-associated}, then by the completeness of $\mathsf{\Pi}$ (cf. Definition \ref{def: zk-completeness}), it must be able to construct a proof $\pi_\mathsf{key}$ to convince $\mathcal{V}$ and update a valid public key $\mathsf{pk}_\mathcal{H}'$ to the VDR.

Next, we argue for soundness. Notice that Relation \ref{eq: relation-key-refresh-associated} is essentially the same as Relation \ref{eq: relation-id-presentation}, except that the latter computes the nullifier using a campaign-specific $\mathsf{id}_\varepsilon$, which does not affect the soundness. Referring to the proof of Theorem \ref{thm: identifier-association-correctness-soundness}, assuming that $\mathcal{V}$ accepts $\pi_\mathsf{key}$ at the end of the interaction, there must exist an extractor $\mathsf{Ext}$ capable of outputting a tuple of witnesses $\big( \rho_\mathsf{a}', \mathsf{id}_\mathsf{a}', \{\mathsf{id}_\mathcal{H}'^{(i)} \},\mathsf{sk}_\mathcal{H}'' \big)$ such that:
	\[
		\begin{aligned}
			& \mathsf{M}.\mathsf{Verify}(r_\mathsf{a}, \mathsf{id}_\mathsf{a}', \rho_\mathsf{a}') = 1\\
			& \land\quad \mathsf{id}_\mathsf{a}' = H_1\left(\mathsf{id}_\mathcal{H}'^{(1)},\dots,\mathsf{id}_\mathcal{H}'^{(n)}\right)\\	
			& \land\quad n_\mathsf{a} = H_1 \left( \mathsf{id}_\mathcal{H}'^{(1)}, \dots, \mathsf{id}_\mathcal{H}'^{(n)}, 1 \right) \\
			& \land\quad \mathsf{id}_\mathcal{H} \in \left\{ \mathsf{id}_\mathcal{H}'^{(1)}, \dots, \mathsf{id}_\mathcal{H}'^{(n)} \right\}\\
			& \land\quad \mathsf{pk}_\mathcal{H}'=\mathsf{PubKeyGen}(\mathsf{sk}_\mathcal{H}'').
		\end{aligned}
	\]

If $\mathsf{id}_\mathcal{H}$ is not equal to any of the $\mathsf{id}_\mathcal{H}^{(i)}$, or $\mathsf{id}_\mathsf{a}$ is not registered or associated with the VDR, the security properties of cryptographic primitives will be broken, as shown in the proof of Theorem \ref{thm: identifier-association-correctness-soundness}. On the other hand, it is natural to assume that the public key derivation function $\mathsf{PubKeyGen}(\cdot)$ is a bijection, and there is no case where one public key corresponds to multiple private keys.
\end{proof}

\begin{game}[Key Privacy]
\label{game: key-recovery-privacy}
Given an efficient adversary $\mathcal{A}$, we define two experiments, Experiment $0$ and Experiment $1$. For $b=0,1$, we define:\\
\noindent\textbf{Experiment} $b$:
\begin{itemize}
	\item The challenger retrieves an identifier $\mathsf{id}_\mathcal{H}$ and an associated identifier $\mathsf{id}_\mathsf{a}$ that satisfies Relation \ref{eq: relation-key-refresh-associated} from the VDR.
	\item The challenger computes $\pi_\mathsf{key}$ as follows:
		\begin{itemize}
			\item If $b=0$, randomly samples a $\mathsf{sk}_\mathcal{H}'\overm{R}\leftarrow\mathcal{SK}$ from the private key space $\mathcal{SK}$, calculates $\mathsf{pk}_\mathcal{H}'\leftarrow\mathsf{PubKeyGen}(\mathsf{sk}_\mathcal{H}')$, and constructs a proof $\pi_\mathsf{key}$ as in Equation \ref{eq: proof-key-recovery}.
			\item If $b=1$, randomly samples a $\mathsf{pk}_\mathcal{H}' \overm{R}\leftarrow\mathcal{PK}$ from the public key space $\mathcal{PK}$ and runs a simulator on the public inputs by:
				\[
					\pi_\mathsf{key}
						\overm{R}\leftarrow
						\mathsf{Sim}\big(
							R,
							(r_\mathsf{a}, n_\mathsf{a}, n_\mathsf{reg}', \mathsf{id}_\mathcal{H}, \mathsf{pk}_\mathcal{H}')
						\big),
				\]
				where $R$ refers to Relation \ref{eq: relation-key-refresh-associated} and \ref{eq: relation-key-refresh-key-pair}.
		\end{itemize}
	\item The challenger sends $\pi_\mathsf{key}$ to $\mathcal{A}$.
	\item The adversary $\mathcal{A}$ computes and outputs $\hat{b}\in\{0,1\}$.
\end{itemize}
For $b=0,1$, let $E_b$ be the event that $\mathcal{A}$ outputs $1$ in Experiment $b$. We define $\mathcal{A}$'s advantage with respect to $\mathsf{\Pi}$ as
	\[
		\mathrm{KRP}\mathsf{adv}[\mathcal{A}, \mathsf{\Pi}] := \big\lvert \Pr[E_0] - \Pr[E_1] \big\rvert.
	\]
\end{game}

\begin{theorem}[Key Privacy]
\label{thm: key-recovery-privacy}
Given a NIZK scheme $\mathsf{\Pi}$ satisfying Definition \ref{def: zk-snark}, the value of $\mathrm{KRP}\mathsf{adv}[\mathcal{A}, \mathsf{\Pi}]$ is negligible for all efficient adversaries $\mathcal{A}$.
\end{theorem}

\begin{proof}[Proof sketch]
This theorem follows directly from Definition \ref{def: zk-zero-knowledge}, where a simulator $\mathsf{Sim}$ is constructed to model the transcripts between counterparties. Since $\mathsf{\Pi}$ has zero knowledge, the transcripts generated by $\mathsf{Sim}$ should be indistinguishable from $\pi_\mathsf{key}$ generated using the true witnesses, i.e., a valid $\mathsf{sk}_\mathcal{H}'$ corresponding to $\mathsf{pk}_\mathcal{H}'$, indicating that $\mathrm{KRP}\mathsf{adv}[\mathcal{A}, \mathsf{\Pi}]\leq \mathsf{negl}(\lambda)$ holds for all efficient adversaries $\mathcal{A}$.
\end{proof}
