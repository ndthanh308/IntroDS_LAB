\section{Introduction}

Blockchain technology and decentralized ledgers have given rise to Web3, a new architecture with unprecedented flexibility and innovation. The vision for Web3 is to eliminate the reliance on centralized services and authoritative participants and achieve an open ecosystem with decentralized characteristics \cite{weyl2022decentralized}. However, current Web3 solutions lack the ability to accurately depict relationships between identities and participants and still rely on centralized Web3 infrastructure, such as social media and hosting platforms, to provide identity and credential management \cite{kuperberg2019blockchain}.

To overcome this limitation, efforts have been made to provide Web3-native identity systems that allow users to manage their identity and credentials using decentralized techniques such as blockchain \cite{khovratovich2017sovrin}. These solutions enable users to manage all credentials from issuers rather than delegating them to a centralized authority. Subsequently, users can autonomously present credentials to verifiers for identification or authentication, without the involvement of issuers \cite{naik2020uport}. Unlike traditional centralized schemes, users retain full sovereignty over their credentials, identifiers, and digital assets in these systems, while service providers are also relieved from the responsibility and risk of managing private data.

Despite their promising vision, decentralized identity schemes currently in place are still plagued by the following problems, limiting their widespread use and replacement of their centralized competitors:
\begin{itemize}
	\item \textbf{Credential and identity privacy}: Although supporting \textit{selective disclosure} of credentials, existing schemes fail to achieve both credential and identity privacy. These solutions require users to expose identifiers for identification, enabling verifiers to construct \textit{identifier-based correlation} to infer relationships between identities and claims \cite{khovratovich2017sovrin}. Additionally, the publicly verifiable proofs enable malicious parties to impersonate credential holders to other verifiers by \textit{proof forwarding}.
	\item \textbf{Sybil-resistance}: Web3 applications like airdrops and decentralized autonomous organizations (DAOs) heavily rely on identity systems that can withstand \textit{Sybil attacks}. However, users can generate multiple identifiers with meager costs to attain a disproportionate impact. Existing schemes either raise the threshold by \textit{enforcing collaterals}\cite{borge2017proof}, which increases the time or monetary costs; or seek data from authoritative parties for \textit{deduplication} \cite{maram2021candid}, which applies only to certain scenarios with trustworthy sources. Sybil attacks pose a significant challenge yet to be solved precisely and gracefully.
	\item \textbf{Key management}: In decentralized systems, users control identifiers and credentials using private keys which are difficult to keep but easy to lose. To recover from key loss, certain schemes entrust users' keys to \textit{centralized services}, while others use \textit{social recovery} mechanisms that split keys and distribute shares to guardians through secret sharing \cite{ur2019trust, maram2021candid}. These schemes rely on robust trust relationships and pose potential security risks due to malicious behaviors of trusted parties.
\end{itemize}

\subsection{Contributions}
\label{subsec:intro-contribution}

To address the above issues encountered by existing systems, we introduce \sysname, a privacy-preserving and Sybil-resistance decentralized identity scheme that supports key management and recovery. \sysname is a Web3-native identity scheme that eliminates reliance on centralized authorities and requires no endorsement from Web2 services.

\sysname allows holders to perform \textit{selective disclosure} of credentials in a data-minimized manner, proving that credentials satisfy arbitrary predicates. \sysname emphasizes credential and identity privacy, eliminating \textit{identifier-based correlation} by utilizing information stored in the decentralized registry for identification. Verifiers cannot associate credentials with specific identifiers or public keys besides knowing that a credential belongs to the holder. Moreover, \sysname empowers the holder to qualify the verifier with \textit{non-forwardable proofs}, preventing malicious proof forwarding and impersonation.

To address Sybil attacks prevalent in identity systems, \sysname leverages credentials to evaluate user identities. On top of this, \sysname incorporates a mechanism called \textit{identifier association}, forcing holders to aggregate identifiers for improved Sybil resistance. Verifiers can confirm that identifiers from the same holder are associated and committed to the blockchain, ensuring that no one can generate new identifiers for an unfair advantage. Compared to existing work, \sysname provides accurate and general Sybil resistance without relying on external sources or imposing extra burdens on benign users. Most importantly, as participants engage with this mechanism, launching Sybil attacks becomes increasingly difficult.

To resolve the issue of key loss and enable \textit{key recovery}, \sysname permits users to prove \textit{ownership of identifiers} to the decentralized registry using on-chain commitments known only to it. The registry refreshes the identifier's public key, allowing the user to regain control with a newly generated private key. Except for cryptographic assumptions, \sysname relies on no trusted relationships between entities. In addition to key loss, \sysname can also prevent asset compromise resulting from \textit{private key leakage}.

In summary, \sysname represents a significant improvement over existing decentralized identity schemes by addressing issues that have long plagued this domain. Our contributions are as follows:
\begin{itemize}
	\item We introduce \sysname, a decentralized identity scheme that enables \textit{selective disclosure} of credentials with respect to arbitrary predicates. It eliminates \textit{identifier-based correlations} and provides unparalleled credential and identity privacy. It also enables users to qualify verifiers with \textit{non-forwardable proofs}, preventing proof forwarding and misuse.
	\item We design an anti-Sybil mechanism named \textit{identifier association} that compels users to publicly aggregate and commit to their identifiers. This mechanism offers accurate and robust Sybil resistance without relying on external data sources or requiring collaterals from participants.
	\item We devise a novel \textit{key recovery scheme} that empowers users to prove their ownership of identifiers to the registry using public commitment, and actively \textit{refresh key pairs} to regain control of key-lost accounts. This mechanism can also prevent asset compromise caused by \textit{private key leakage} in addition to key loss.
\end{itemize}

\subsection{Related Work}
\label{subsec:intro-related}

\spar{Decentralized identity}
Decentralized identity schemes are widely used in the Web3 ecosystem for their ability to mitigate the risks of aggregation and leakage of private information associated with centralized and federated identities \cite{khovratovich2017sovrin, naik2020uport, khalsa2022holonym}. The Decentralized Identity Foundation and the W3C's proposed Decentralized Identifier (DID) attempt to standardize this mechanism by clarifying the authentication process and participating entities \cite{world2022decentralized}. In addition, Weyl et al. implement a similar function in another form, abstracting the user's identity as a \textit{soul} and storing the associated proofs and claims in a so-called \textit{soul-bound token} \cite{weyl2022decentralized}.

Closely related to decentralized identity are verifiable credentials issued by the issuer to the user and later presented to a verifier to prove a claim \cite{world2022verifiable}. A body of work already exists to allow users to prove claims without revealing identifiers or other identifying information through anonymous credentials, thus preventing the disclosure of user identities due to collaboration between malicious parties involved \cite{yang2019decentralized}.

Except for a few implementations that do not emphasize privacy, most of these efforts can provide some degree of selective disclosure \cite{camenisch2004signature}. However, they still require the user to disclose identifiers to complete authentication actively, and thus may lead to identity-claim correlation due to conspiracy. Other anonymization schemes, while eliminating identity-based correlations, can face credential transfer issues due to the absence of user identification.

\spar{Sybil-resistance}
Sybil attack is a common problem for identity schemes. In Web3 scenarios such as airdrops and DAOs, many services need to be protected against Sybil attacks and induced strategic behaviors \cite{victor2020address, wang2019decentralized}. A malicious user can gain a disproportionate advantage by generating multiple pseudonyms.

Various attempts have been made to address this problem, such as proving that a user is indeed a genuine entity by having them participate in certain offline activities \cite{borge2017proof}. Some systems force users to provide a certain amount of collateral to raise the cost to launch a Sybil attack \cite{buterin2017casper}. Other schemes attempt to enlist the help of centralized services, such as forcing users to authenticate their accounts with centralized social media to create an identity in the Web3 system \cite{maram2021candid, khalsa2022holonym}.

It turns out that none of these attempts solve the problem well. The point is that these mechanisms significantly raise the barrier to entry for identity systems. They can mitigate Sybil attacks only to a certain extent; for example, an attacker can still bypass proof of personhood by bribing others to register on their behalf \cite{platt2021sybil}.

\spar{Key and credential recovery}
Key loss or theft is one of the most significant risks facing identity systems, especially with decentralized schemes, which often means a complete loss of digital assets and substantial financial losses. Hardware wallets aggressively attempt to solve this problem, using dedicated devices to keep and manage all private keys \cite{rezaeighaleh2019new}. Another option is to encode private keys using easy-to-remember mnemonics and record them on a physically secure medium \cite{palatinus2013bip}. However, these traditional schemes are vulnerable to single points of failure, as they can still fail due to loss or theft of the physical medium.

Social recovery is another common solution to this problem, allowing users to authorize a set of \textit{guardians} to be able to modify the key later according to a specific procedure \cite{weyl2022decentralized}. However, this mechanism relies on a trust relationship with the guardians and requires some offline means to identify the user. Some schemes utilize secret sharing to split and share keys with guardians and encrypt the shares to prevent potential complicity \cite{bagherzandi2011password, jarecki2014round}. Such mechanisms do nothing more than shift the reliance on the private key to another key but do not solve the problem.