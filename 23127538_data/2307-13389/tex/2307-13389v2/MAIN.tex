\documentclass[11 pt,reqno]{amsart}

\usepackage[utf8]{inputenc}
\setcounter{MaxMatrixCols}{10}



% \textwidth 15.5truecm
% \topmargin -2truecm
% \textheight 24truecm
% \oddsidemargin .7truecm
\usepackage[utf8x]{inputenc}
\usepackage{amscd}
\usepackage{amsmath}
%\usepackage{amssymb}
\usepackage{amsthm}

\usepackage[colorinlistoftodos]{todonotes}

\usepackage{thmtools}
\usepackage{thm-restate}
\usepackage{mathtools}
\usepackage[full]{complexity}
\usepackage{longtable}

%\usepackage[usenames,dvipsnames]{xcolor}
\usepackage{xcolor}
% % tables
 \usepackage{array}

\usepackage{bbm}
\usepackage{comment}
\usepackage{enumerate}
\usepackage{floatrow}


\usepackage{parallel,enumitem}

\usepackage{xspace}
\usepackage{paralist}
\usepackage{xifthen}
\usepackage{url}
\usepackage{csquotes}
% \usepackage{graphicx}
\usepackage{wrapfig}
\usepackage{multirow}
\usepackage[binary-units=true]{siunitx}

\usepackage{tikz}
\usetikzlibrary{trees,decorations,arrows,automata,shadows,positioning,plotmarks,backgrounds,shapes}
\usetikzlibrary{calc,matrix,fit,petri,decorations.markings,decorations.pathmorphing,patterns,intersections,decorations.text}
\usepackage{pgfplots}
\usepackage{pgfplotstable}

\tikzstyle{mystate}=[state,inner sep=3pt,minimum size=20pt,line width=0.2mm]
\tikzstyle{fstate}=[state,accepting,inner sep=2pt,minimum size=3pt]
\tikzstyle{istate}=[state,initial,inner sep=2pt,minimum size=3pt]
\tikzstyle{mysquare}=[inner sep=3pt,minimum size=15pt,line width=0.2mm]
\tikzstyle{fmysquare}=[inner sep=3pt,minimum size=15pt,line width=0.5mm,accepting]
\newcommand{\SFSAutomatEdge}[5]{\path[->](#1) edge[#4,line width=0.2mm] node[#5] {\ensuremath{#2}} (#3);}
\usepackage{subcaption}
\usepackage{tabularx}
\usepackage{booktabs}
\usepackage{xfrac}

\usepackage{etoc}
\etocsettocdepth{3}

% \usepackage{minitoc}

% \usepackage{titletoc}
% 
% \newcommand\DoToC{%
%   \startcontents
%   \printcontents{}{2}{\textbf{Contents}\vskip3pt\hrule\vskip5pt}
%   \vskip3pt\hrule\vskip5pt
% }

\lstMakeShortInline[columns=fixed]@
% Figure environment removed
\lstDeleteShortInline@

In this section, we describe how we collect examples for learning repair strategies without any version-controlled data. Specifically, we first detect \safeprogs and corresponding witnesses using \sawitnessfull (witnesses are sanitizers and guards that protect from vulnerabilities)  in Section~\ref{subsec:sa-witness}. Using these witness annotations, we generate unsafe programs and \textit{edits} from the \safeprog using a \textbf{witness-removal} step (Section ~\ref{subsec:witness-removal}). In the following, we define terminology for the \astree  data-structure we operate on. 


\astree refers to the abstract syntax tree representation of programs, augmented with data flow edges and annotations for sources, sinks, sanitizers, guards, witnesses etc. 
An \astree is a five-tuple 
$\langle \mathcal{N},\mathcal{V},\mathcal{T},\mathcal{E}, \mathcal{A} \rangle$, where:
\begin{enumerate}
\item
$\mathcal{N}=\{\mathit{id}_0,\ldots\mathit{id}_n\}$  is a set of nodes, where  $\mathit{id_i}\in\mathbb{N}$ for 
$ 0 \leq i \leq n$.
\item
$\mathcal{V}$ is a map from nodes to program snippets
represented as strings. For a node $n$, we have that $\mathcal{V}(n)$ is a string representing the code snippet associated with $n$
\item
$\mathcal{T}$ is a map from nodes to their types defined by 
 \sa~\cite{codeqlast}. For example, \callexpr is the type of a node representing a function call, \indexexpr is the type of a node representing an array index, and \blockstmt is the type of a node representing a basic block of statements.
\item
$\mathcal{E}$ is a set of directed edges.
Each edge is of the form $(n_1,n_2,\edgetype,z)$, where
$n_1$ is a source node, $n_2$ is a target node, 
$\edgetype \in \{\T{SynParent}, \T{SynChild}, \T{SemParent},
\T{SemChild} \}$ denotes the relationship from 
$n_1$ to $n_2$, as one of syntactic parent, syntactic child, semantic parent or semantic child,
and $z\in\mathbb{Z}$ is the index of $n_2$ among $n_1's$ children if this edge is a child edge, and $-1$ if the edge is a parent edge. 
\item
$\mathcal{A}$ is a set of annotations associated with each node. The annotations are from the set $\{\T{source},
\T{sink},\T{sanitizer},\T{guard}$,\T{witness}\}. We also refer to annotations using predicates or relations. For instance, for a node $n$, if an annotation  $\T{source}$ is present, we say that
the predicate $\T{source}(n)$ is true.
\end{enumerate}

%\setlength{\grammarindent}{5em} % increase separation between LHS/RHS

% Figure environment removed



A {\em traversal} or a {\em path} in an \astree is a sequence of edges $e_0,\ldots,e_{i-1},e_i,\ldots ,e_k$ such that the target node of $e_{i-1}$ is also the source node of $e_i$, for all $i\in\{1,\ldots,k\}$. That is, $e_{i-1}$ is of the form $(\_,n,\_,\_)$ and $e_i$ is of the form $(n,\_,\_,\_,\_)$. The source node of $e_0$ is the source of this path and the target node of $e_k$ is the target of the path.


\lstMakeShortInline[columns=fixed]@
%Note that these additional edges can capture long-range dependencies in programs. E.g. edge 4 in Figure ~\ref{fig:unsafememberex} links two nodes across the function boundaries. 
Figure~\ref{fig:example1-pdg} depicts a partial \pdg corresponding to the unsafe program in Figure~\ref{fig:unsafememberex}. Each oval corresponds to an \astree-node containing a type $\tau$ and an associated value. The dark edges denote the syntactic child edges. For example, the oval with value @foo(data)@ is an \astree-node with type \callexpr and has two children -- @foo@ and @data@, both with the type \varexpr. 
%Similarly, the \blockstmt node on the top refers to the function body between Line~\ref{lst:line:handlers-run} and Line~\ref{lst:line:handlers-run-end} in Figure ~\ref{fig:unsafememberex}. As the body of a function block can contain a variable number of children, we link to @handlers[callerId](data);@ as the k-th child of the \blockstmt. 
The semantic child edges are at the bottom in cyan. These edges correspond to the ones depicted in cyan in Figure ~\ref{fig:unsafememberex}. 
\lstDeleteShortInline@

%TODO:FIX THIS

%With this simplification, 
If $\prog$ is an \pdg then
we use  $\prog.\mathtt{source}$ to denote the source node, $\prog.\mathtt{sink}$ to denote the sink node, and $\prog.\mathtt{witness}$ to denote the witness node.
If the program has several sources, sinks and sanitizers then we generate a separate \pdg for each $(\mathtt{source},\mathtt{witness},\mathtt{sink})$ triple.
For a node $n$, its syntactic parent is $n.\mathtt{parent}$, syntactic children are $n.\mathtt{children}$, semantic parent is $n.\mathtt{semparent}$, and semantic children are $n.\mathtt{semchildren}$.

%\input{ql.tex}

\subsection{Static Analysis Witnessing}
\label{subsec:sa-witness}

\newcommand{\DMethodjudge}[1]{\texttt{#1(}\checknextarga}

% Figure environment removed

%\naman{TODO - sell this more as technique to work with any \sa tool ; our master query is a general framework implemented in \codeql that can work for any vulnerability -- easily extendable to other languages }
In this section, we show how to repurpose \sa tools to generate witnesses.
\sa tools perform dataflow analysis to check for rule-violations in programs. They use pattern matching to identify known sources, sinks, sanitizers, and guards. For commercial tools, these patterns are implemented (and continuously updated) manually by developers and encode this domain knowledge. Next, 
%these patterns are used to detect sources, sinks, sanitizers, and guards in programs and
\sa checks if there exists a flow between a source and a sink that does not cross a sanitizer or guard. We capture this formally in Figure~\ref{fig:judgements} (top two rules), and explain the notation used in it below.

\sa tools encode domain knowledge about the vulnerability by annotating nodes as \T{Source}, \T{Sink}, \T{Sanitizer}, and \T{Guard}. %These relations operate on the set of dataflow nodes in the programs.
So \DMethod{Source}{\I{n}}\ is true iff the node \I{n} is a \textit{source} node for a vulnerability. Next, \sa tools perform dataflow analysis by defining the relation \DMethod{SemChild}{$n_1$}{$n_2$}\ which is true iff there is a \taintpropedge between $n_1$ and $n_2$. Then the \DMethod{Vulnerability}{$n_1$}{$n_2$}\ relation can be defined as:
\begin{enumerate}
    \item $n_1$ and $n_2$ are source and sink nodes (\DMethod{Source}{$n_1$}\ and \DMethod{Sink}{$n_2$}\ are true)
    \item There exists a \textit{path} between $n_1$ and $n_2$ which is free of sanitizers or guards (\DMethod{SanGuardFree*}{$n_1$}{$n_2$}\ is true). A path is free of sanitizers and guards iff every \textit{edge} in the \textit{path} is free of sanitizers and guards. An edge between $n_1$ and $n_2$ is considered free of sanitizers and guards (\DMethod{SanGuardFree}{$n_1$}{$n_2$}\ is true) iff $(n_1, n_2, \_, \T{SemChild}) \in \mathcal{E}$ and neither of $n_1$ or $n_2$ is a sanitizer or a guard
\end{enumerate}

Here, we make the following observation - \emph{this domain knowledge present in these annotations and relations is helpful beyond just detecting vulnerabilities}. For instance, simply using the sanitizer relation allows us to query the different kinds of sanitizers domain experts have specified. We use this observation to discover \emph{\safeprogs} i.e., programs having a source, sink, and a sanitizer or guard that \textit{blocks} the \taintprop or, in simpler terms, make the program safe. In addition, we also detect the corresponding sanitizers or guards in the programs and refer to them as \textit{witnesses} because they serve as the evidence of making the program safe. We call this procedure \sawitnessfull (abbreviated as \sawitness). 
We define this as the \T{Witness} relation in Figure~\ref{fig:judgements} (bottom two rules). Specifically, \DMethod{Witness}{$n_1$}{$n_3$}{$n_2$}\ is defined as:
\begin{enumerate}
    \item $n_1$ and $n_2$ are source and sink nodes (\DMethod{Source}{$n_1$}\ and \DMethod{Sink}{$n_2$}\ are true)
    \item There exists a node $n_3$ such that it satisfies \DMethod{SanGuardInMid}{$n_1$}{$n_3$}{$n_2$}. \DMethod{SanGuardInMid}{$n_1$}{$n_3$}{$n_2$}\ is true iff there exists a \T{SemChild}
    %\naga{notation for flow inconsistent with (2) above} 
    path between $n_1$, $n_3$, between $n_3$ and $n_2$, with the additional constraint of $n_3$ being a sanitizer or guard. 
\end{enumerate}

The difference between the \T{Vulnerability} relation (which \sa populates) and \T{Witness} relations (which we want to find) is highlighted in {\color{red} red} and {\color{ForestGreen} green}. Notice that while defining the \T{Witness} relation, we simply use the existing relations that define the \T{Vulnerability} relation. Thus, we argue that \sawitness can be implemented on top of \sa by using the intermediate relations that \sa is computing.
%for every pair of source and sink, they track taint through a taint-flow analysis. If there is a flow from a source to a sink that does not go through a sanitizer or guard, then the source-sink pair is reported as vulnerable.

%We make the following observation - \emph{the patterns defined by experts encodes domain knowledge which can be used for use cases beyond just detecting vulnerabilities}. For instance, we can use the sanitizer patterns to search for all sanitizers in source-code. In this work, we use this idea to detect \safeprogs, which we define as programs having a source, sink, and a sanitizer or guard that blocks the \unsure{flow} or in other words, makes the program safe.  \naman{highlighted part of Figure somethings shows the difference between semantics of witnessing vs traditional semantics}

%We realize the following -- the set of patterns of sources, sinks, and sanitizers are useful beyond detecting vulnerabilities. We override the existing static analysis query that detects unsafe programs and use these encoded sanitizers for detecting sanitizers and guards in programs. Specifically, in the existing query that detects unsafe programs, we modify the taint-propagation steps to propagate taints through sanitizers and guards and then use static analysis to then find these dataflows containing sanitizers and guards. Thus, we can directly find the safe programs containing these \textit{witnesses} of safety. 
%Once such a dataset is collected, we use these witnesses to convert safe  to unsafe  and thus obtain paired examples for learning repair strategies (Section~\ref{subsec:witness-removal}). 

\lstMakeShortInline[columns=fixed]@
%We instantiate our \sawitness technique using \codeql~\cite{a}. It is an open-source \sa tool that allows implementing custom static analysis as queries in a high-level object-oriented extension of datalog. These queries usually contain a \Verb|select from where| statement that allows querying the program database. \codeql maintains these patterns of sources, sinks, sanitizers, and guards using \Verb"Configuration" classes. Consider an example of a simplified \Verb"Configuration" for \xss vulnerability in Figure~\ref{fig:configuration}. It defines a set of predicates @isSource@, @isSink@, @isSanitizer@, and @isGuard@. These predicates are written manually by \codeql authors and improved through rich community support\footnote{\url{https://github.com/github/codeql}}. With this configuration, vulnerabilities are reported by selecting source-sink pairs such that the @cfg.hasFlow@ predicate is true for the source, and the sink. This predicate is internally defined by \codeql and uses the patterns defined in the configuration to check for the presence of vulnerability-causing dataflows. %\spsays{Showing corresponding programs will be useful}

%Now, we demonstrate the static-analysis-witnessing approach for collecting examples of \safeprog and witnesses in Figure~\ref{fig:safe-configuration}. Specifically, we inherit from the existing configuration, using the same @isSource@ and @isSink@ predicates while overriding the @isSanitizer@ and @isGuard@ predicates to @none()@. This ensures that all the source and sink pairs are detected independent of the presence of sanitizers/guards between them. Finally, to detect our witnesses, we define the @isWitness@ predicate which uses the @isSanitizer@ and @isGuard@ predicates from the original configuration. Specifically, witnesses are defined as sanitizers/guards that lie between a source-sink pair. Finally, to report \safeprog and witnesses, the @cfg.hasFlow@ predicate is used to select all valid source-sink pairs and the corresponding witnesses are detected via the @isWitness@ predicate. Note that Figure~\ref{fig:configuration-vs-safe-configuration} depicts the key idea behind our approach in a simplified view. In practice, additional measures need to block the taint propagation internally and we share the actual \codeql queries used as part of the Appendix~\ref{app:codeql-queries}.


\subsection{Witness Removal}
\label{subsec:witness-removal}

We obtain \safeprogs and witnesses by applying \sawitness to a snapshot of a codebase. Recall that the witnesses block the flow between a source and a sink and thus help make programs  \textit{safe}. Hence, removing these witnesses will make the programs unsafe. Recall also that the witnesses are either sanitizing functions of the form @sanitize(taintedVar)@ or guards of the form @if checkSafe(taintedVar) {executeSink(taintedVar)}@. %Usually, they are used only for ensuring the safety of programs and are not critical to the functionality of programs. Therefore, 
We implement witness-removal perturbations  that precisely remove the guard-checks and sanitizer-functions. Note that our goal here is to generate unsafe programs and corresponding edits that enable learning repair strategies that insert such witnesses. So, while we generate the unsafe programs by perturbation, they should look structurally similar to natural unsafe programs written by the developers, otherwise the repair strategies learned on this artificially generated data through perturbations would not generalize to code in the wild. 
%At the same time, minor syntactic-semantic issues in parts of unsafe programs not directly relevant to the vulnerability or repair do not impact learning.
\lstDeleteShortInline@

% Figure environment removed

\lstMakeShortInline[columns=fixed]@

\input{witnessremoval.tex}

We use \rmSan and \rmGuard functions to programmatically remove the witnesses. A high-level sketch of these functions is illustrated in Figure~\ref{fig:remove-functions}. The functions use the structure of the corresponding \astree (node types $\tau$) to decide how to remove witnesses. Consider the \rmGuard function. It first computes the parent (\witnesspar) and grand-parent (\witnessparpar) of the witness guard condition. Then if the type of \witnesspar is \ifstmt (i.e., program is of the form @if (witness) body@ then we modify the \astree edge from \witnessparpar and \witnesspar to instead point to the body of the \ifstmt (index 1 child is body of \ifstmt). Similarly, if the type of \witnesspar is \binaryexpr with operator @&&@ (i.e. of the form @if (otherCond && guard)@ or @if (guard && otherCond)@) then we again modify the edge from \witnessparpar and \witnesspar to instead point to the non-guard child of \binaryexpr (@otherCond@ in the example). Note that since \binaryexpr has 3 children, the index of non-guard child is index of guard-child subtracted from 2. 
Figure~\ref{fig:witness-removal} depicts this removal on the \astree level, where the syntactic edges in red are removed and the syntactic edges in green are inserted.
In the end, the functions returns a tuple of the \pdg of the unsafe program ($\prog_{unsafe}$), \pdg of the safe program ($\prog_{safe}$)
and an edit object (\edit) which stores


\begin{enumerate}
    \item \astree for the removed witness (referred to as \editprog)
    \item location in the \pdg where the witness is removed (referred to as editloc
    %\naga{shouldn't it be editloc to be consistent with (1)?} 
    or \editloc)
    %\item an enum (\insertsc or \replace) depending on whether \concedit is inserted or replaced 
\end{enumerate}

Since $\prog_{unsafe}$ and edit-object can generate the safe program, we only propagate the unsafe programs and edits as the output of this step. Applying \rmGuard function to the safe program in Figure~\ref{fig:safememberex} removes the \ifstmt on Line~\ref{lst:line:fix-start} while preserving the @handlers[callerId](data);@ statement and in fact produces the unsafe program in Figure~\ref{fig:unsafememberex}. Additionally, it  returns the removed witness guard  @if handlers.hasOwnProperty(data.id){ ... }@ as the \editprog and \blockstmt (blue oval in Figure~\ref{fig:example1-pdg}) as the edit location \edit.editloc. Figure~\ref{fig:example1-editprog} shows the \astree for the \editprog containing the \ifstmt. 
The dashed line and dark circle correspond to the \textit{removed} \astree edge between the \blockstmt and the \expr @handlers[callerId](data)@. 

Note that Figure~\ref{fig:remove-functions} provides a high-level sketch of witness-removal and elides over implementation details that are required to make it work for real \js programs. We discuss these issues in the implementation section (Section~\ref{subsec:impl:witness-removal}).% and include the full implementation as part of supplementing source code\naga{we should make sure we are doing these, else remove this sentence}. 
%. In practice, we need implement such decisions more carefully to cover other traditional cases in which guards occur and we document them in the supplementing source code.
\lstDeleteShortInline@

%\naman{add examples $\dots$ } \spsays{do we re-run codeql on this generated bad program? -- NO (naman)}





% New commands
\newcommand{\incr}{\,\mathrm{d}}
\newcommand{\set}[1]{\{#1\}}
\newcommand{\diff}[2]{\frac{\mathrm{d}{#1}}{\mathrm{d}{#2}}}
\newcommand{\pdiff}[2]{\frac{\partial{#1}}{\partial{#2}}}
\newcommand{\ndiff}[3][]{\frac{\mathrm{d}^{#1}{#2}}{\mathrm{d}{#3}^{#1}}}
\newcommand{\npdiff}[3][]{\frac{\partial^{#1}{#2}}{\partial{#3}^{#1}}}
\newcommand{\R}{\mathbb R}

% New environments
\newtheorem{remark}{Remark}

\begin{document}

\begin{abstract}
   In this paper, we study Lagrangian submanifolds of the pseudo-nearly Kähler $\Sl\times\Sl$.
   First, we show that they split into four different classes depending on their behaviour with respect to a certain almost product structure on the ambient space.
   Then, we give a complete classification of totally geodesic Lagrangian submanifolds of this space. 
\end{abstract}
\maketitle
\section{Introduction}

\vspace{3 ex}

The concept of a nearly Kähler manifold was introduced in 1959 by Tachibana in \cite{tachibana}. In the 1970s, Gray extended the study of nearly Kähler manifolds in \cite{gray1,gray2}  and together with Hervella they showed the importance of these spaces in \cite{gray3}. Later on, the work of Nagy \cite{nagy} exhibited the special role of homogeneous six-dimensional nearly Kähler manifolds.
It was not until 2002 that Butruille \cite{butruille} classified, up to isometries, all homogeneous Riemannian six-dimensional strictly nearly Kähler manifolds, these being $\Ss^6$, $\Ss^ 3\times \Ss^ 3$, $\mathbb{C}P^3$ and the space $F(\mathbb{C}^3)$ of full flags in $\mathbb{C}^3$.
Six similar examples of pseudo-Riemannian nearly Kähler manifolds (see diagram below), or pseudo-nearly Kähler manifolds for short, were introduced in \cite{ines,Schafer}, but a complete classification has not been found yet.

% \vspace{-0.25cm}
\[
\begin{tikzcd}[ampersand replacement=\&, row sep=small, column sep=small]
    \Ss^3\times\Ss^3\ar[dd] \&
    \Ss^6\ar[dd] \&
    \mathbb{C}P^3\arrow[dr, to path=|- (\tikztotarget)]
    \arrow[ddr, to path=|- (\tikztotarget)]
    \&
    \&
    F(\mathbb{C}^3)\arrow[dr, to path=|- (\tikztotarget)]
    \arrow[ddr, to path=|- (\tikztotarget)] \& \\
    \&
    \&
    \&
    \frac{\SO^+(2,3)}{\U(1,1)} \&  
    \&
    \frac{\SU(2,1)}{\U(1)\times\U(1)}  \\
    \Sl\times\Sl \&
    \Ss^6_4 \&
     \&
    \frac{\SO^+(4,1)}{\U(2)}\&
     \&
    \frac{\Slt}{\R^* \times \SO(2)}
\end{tikzcd}
\]
Here, $\Ss^6_4$ means the elements of $\R^7_4$ with length 1. The group $\SO(p,q)$ is not connected in general, thus the symbol $+$ represents the connected component of the identity. 
% The subgroup $\R^*\cdot\SO(2)$ of $\Slt$ is given by the elements of the form
% \[
% \begin{pmatrix}
%     \lambda \cos\theta & \lambda \sin\theta & 0\\
%     -\lambda \sin \theta & \lambda \cos\theta & 0\\
%     0 & 0 & \tfrac{1}{\lambda^2}
% \end{pmatrix}
% \]


% \begin{minipage}
% % Figure environment removed
    % \end{minipage}
    % \vspace{-0.6cm}
    
The pseudo-nearly Kähler $\Sl\times\Sl$ is an analogue of $\Ss^ 3\times \Ss^ 3$, as the split quaternions can be used to replace the quaternions to describe its geometric structure. 

When studying submanifolds of an almost Hermitian manifold, two types of submanifolds attract our attention: the almost complex and the totally real submanifolds. 
The former are those submanifolds whose tangent spaces are preserved by the almost complex structure of the ambient space. 
The latter are those whose tangent spaces are mapped into the normal spaces by the almost complex structure of the ambient space.
If a totally real submanifold has half the dimension of the ambient manifold, it is called a Lagrangian submanifold. 

In \cite{wang} the authors showed that any Lagrangian submanifold of a six-dimensional strictly nearly Kähler manifold that has parallel second fundamental form, must have vanishing second fundamental form, i.e. it has to be totally geodesic. The tools the authors used, can also be applied in the pseudo-Riemannian case (see \cite{Schafer}). In the same paper, the authors gave a complete classification of totally geodesic Lagrangian submanifolds of the nearly Kähler $\Ss^ 3\times \Ss^ 3$.

In this work, we study Lagrangian submanifolds of the pseudo-nearly Kähler $\Sl\times\Sl$ and we divide them in four different groups, according to their behavior with respect to an almost product structure on $\Sl\times\Sl$.
If we add the condition of being totally geodesic, we find that these submanifolds can only be part of the first group, also called Lagrangian submanifolds of diagonalizable type.
The full classification of totally geodesic Lagrangian submanifolds of $\Sl\times\Sl$ is as follows.
% The classification is similar to the one in \cite{wang} for $\Ss^3\times\Ss^3$, but with one extra case.
\begin{theorem}\label{maintheorem}
Any totally geodesic Lagrangian submanifold of the pseudo-nearly Kähler $\Sl\times\Sl$ is congruent to the image of one of the following maps, possibly restricted to an open subset:

% Let $f:M\to\Sl\times\Sl$ be an isometric immersion of a three-dimensional manifold into the pseudo-nearly Kähler $\Sl\times\Sl$. Then $f$ is a totally geodesic Lagrangian immersion if and only if it is congruent to an open subset of one of the following examples:
\begin{enumerate}
    \item $f\colon\Sl\to\Sl\times\Sl\colon u\mapsto(\id_2,u)$, \label{map1}
        \item $f\colon\Sl\to\Sl\times\Sl\colon u\mapsto(u,\ii u\ii)$, \label{map2}
    \item $f\colon\Sl\to\Sl\times\Sl\colon u\mapsto(u,\kk u\kk)$, \label{map3}
\end{enumerate}
where $\id_2,\ii,\kk$ are the matrices
\[
\id_2=\begin{pmatrix}
1&0\\
0&1\\
\end{pmatrix}, \ \ \ \ \ \ \ii=\begin{pmatrix}
0&1\\
-1&0\\
\end{pmatrix}, \ \ \ \ \ \
\kk=\begin{pmatrix}
1&0\\
0&-1
\end{pmatrix}.
\]

Conversely, the maps \emph{(\ref{map1})}, \emph{(\ref{map2})} and \emph{(\ref{map3})} are totally geodesic Lagrangian immersions.
\end{theorem}

In \cite{wang}, the authors stated that any totally geodesic Lagrangian submanifold of $\Ss^3\times\Ss^3$ is congruent to an immersion of a list of 6 examples. 
This classification can be simplified, using isometries equivalent to the ones given in \eqref{isoslsl}. 
This way, the list is reduced to just two examples, similar to immersions \eqref{map1} and \eqref{map2} of Theorem \ref{maintheorem}. 
Hence, immersion \eqref{map3} is a new example, which arises from the pseudo-Riemannian nature of $\Sl\times\Sl$.


We can understand the geometry of the immersions of Theorem \ref{maintheorem} via the identification of $\Sl$ with the anti-de Sitter space $H_1^3(-1)$; namely via the map  
\begin{equation*}
    H_1^3(-1)\subset \R_2^4\to\Sl:(x_0,x_1,x_2,x_3)\mapsto \begin{pmatrix}
    x_0-x_2 &x_3-x_1\\
    x_3+x_1 & x_0+x_2
\end{pmatrix},
\end{equation*} which is an isometry between $H^3_1(-1)$ and $\Sl$ with the metric introduced in Equation \eqref{prodsl2} below. Here, $\R^4_2$ denotes $\R^4$ equipped with the indefinite inner product 
\[\li x,y\ri=-x_0y_0-x_1y_1+x_2y_2+x_3y_3,\] 
and the three-dimensional anti-de Sitter space with constant sectional curvature $c<0$ is defined as $H^3_1(c)=\{x\in\R^4:\li x,x\ri=1/c\}$.



Note that the three immersions of Theorem \ref{maintheorem} induce essentially different Riemannian structures on $\Sl$. The first immersion induces a metric with constant sectional curvature, that is homothetic to the standard metric. The second immersion turns $\Sl$ into an anti-de Sitter space with a Berger-like metric stretched in a timelike direction, and the third immersion turns it again into an anti-de Sitter space with a Berger-like metric, but now stretched in a spacelike direction. These metrics have been studied more generally in \cite{calvaruso_helix_2023} and \cite{calvaruso_metrics_2014}.

The paper is organised as follows. In Section \ref{preliminaries} we recall the nearly Kähler structure of $\Sl\times\Sl$ and we give a comparison with the product metric. In Section \ref{lagrangiansubmanifolds} we state some properties of Lagrangian submanifolds and we divide them into four groups.
Finally, in Section \ref{totallygeodesic} we classify, up to congruence, all totally geodesic Lagrangian submanifolds of the nearly Kähler $\Sl\times\Sl$.


\section{The pseudo-nearly Kähler \texorpdfstring{$\Sl\times\Sl$}{SL(2,R)xSl(2,R)}} \label{preliminaries}
We recall the geometric structure of the pseudo-nearly Kähler $\Sl\times\Sl$. Most of this section is based on \cite{Ghandour}.
\subsection{Nearly Kähler manifolds}  A nearly Kähler manifold is an almost Hermitian manifold for which the tensor $G(X,Y)=(\tilde{\nabla}_XJ)Y$ is skew symmetric. Here, $J$ is the almost complex structure and $\tilde{\nabla}$ is the Levi-Civita connection. For a (pseudo-)nearly Kähler manifold, the tensor $G$ satisfies
\begin{equation}
    g(G(X,Y),Z)+g(G(X,Z),Y)=0, \ \ \ \ \ G(X,JY)+JG(X,Y)=0,\label{nkprop}
\end{equation}
where $g$ is the metric and $X$, $Y$ and $Z$ are vector fields on the manifold.
As an immediate consequence we have 
\begin{equation}
    g(G(X,Y),JZ)+g(G(X,Z),JY)=0.
    \label{gnormal}
\end{equation}

\subsection{The manifold \texorpdfstring{$\Sl$}{SL(2,R)}}Now consider the following non-degenerate indefinite inner product on $\R^4$:
\begin{equation*}
    \langle a,b\rangle =-\frac{1}{2}(a_1b_4-a_2b_3-a_3b_2+a_4b_1). 
\end{equation*}
We can identify the $2\times 2$ real matrices space $M(2,\R)$ with $\R^4$, so the above inner product can be seen as 
\begin{equation}
    \langle a,b\rangle=-\frac{1}{2}\operatorname{Trace}(\operatorname{adj}(a)b),\label{prodsl2}
\end{equation}
where $\operatorname{adj}(a)$ is the adjugate matrix of $a$. 
The real special linear group $\Sl$ is the set of all the $2\times 2$ real matrices with determinant 1, which turns out to be 
\[
\Sl=\{a\in M(2,\R):\langle a,a\rangle=-1\},
\]
thus $\Sl$ inherits the Lorentzian metric in \eqref{prodsl2} from $M(2,\R)\cong\R^4_2$, which has constant sectional curvature $-1$. This way, we can think of $\Sl$ as the anti-de Sitter space $H_1^{3}(-1)$. The tangent space of $\Sl$ at a matrix $a$ can be written as
\begin{equation}
T_a\Sl =\{ a\alpha: \alpha\in \mathfrak{sl}(2,\R)\},    \label{tangentspace}
\end{equation}
where $\mathfrak{sl}(2,\R)$ is the Lie algebra of $\Sl$, namely the set of all the matrices in $M(2,\R)$ with vanishing trace.

The three-dimensional vector space $\mathfrak{sl}(2,\R)$ is spanned by the pseudo-orthonormal basis composed by the so called split quaternions $\ii,\jj,\kk$, given by
\begin{equation*}
\ii=\begin{pmatrix}
0&1\\
-1&0\\
\end{pmatrix}, \ \ \ \ \ \ \jj=\begin{pmatrix}
0&1\\
1&0\\
\end{pmatrix}, \ \ \ \ \ \
\kk=\begin{pmatrix}
1&0\\
0&-1
\end{pmatrix}.\label{splitquaterions}
\end{equation*}

\subsection{The nearly Kähler structure on \texorpdfstring{$\Sl\times\Sl$}{SL(2,R)xSL(2,R)}}
We define an almost complex structure $J$ on $\Sl\times\Sl$ by

\[
J(a\alpha,b\beta)=\frac{1}{\sqrt{3}} (a(\alpha-2\beta),b(2\alpha-\beta)),
\]
for $\alpha,\beta\in\mathfrak{sl}(2,\R)$. It is well known that with any given (indefinite) metric and almost complex structure we can construct a metric in such a way that the almost complex structure is compatible with the new metric. Starting from the product metric, we get an explicit expression for the new metric:
\begin{equation*}
    g((a\alpha,b\beta),(a\gamma,b\delta))=\frac{2}{3}\langle(a\alpha,b\beta),(a\gamma,b\delta)\rangle-\frac{1}{3}\langle(a\beta,b\alpha),(a\gamma,b\delta)\rangle,
\end{equation*}
for $\alpha,\beta,\gamma,\delta\in\mathfrak{sl}(2,\R) $, and $\li,\ri$ is the product metric of $\Sl\times\Sl$ associated to the metric of $\Sl$ given in (\ref{prodsl2}).

As above we denote the covariant derivative of $J$ by $G$ and we can see that $G$ is skew symmetric, so $\Sl\times\Sl$ together with $J$ and $g$ forms a pseudo-nearly Kähler manifold. As a consequence $G$ satisfies (\ref{nkprop}).


\subsection{The almost product structure \texorpdfstring{$P$}{P}}
Consider the almost product structure $P$ on $\Sl\times\Sl$ defined by
\begin{equation}
    P(a\alpha,b\beta)=(a\beta,b\alpha). \label{prodstructuredef}
\end{equation}
% The next lemma states some properties of this operator.

The tensor $P$ has the following properties:

\begin{equation}
    \begin{split}
        &P^2=\id, \ \ \ \ \ \ \ \ \ \ \ \ \ \ \ \ \ \ \ \ \ g(PX,PY)=g(X,Y),\\
        &PJ=-JP, \ \ \ \ \ \ \ \ \ \ \ \ \ \ \ \ \ g(PX,Y)=g(X,PY),
    \end{split}\label{proppe}
\end{equation}
for any vector fields $X,Y$ on $\Sl\times\Sl$.


The tensor $G$ has an explicit expression which is given in the following proposition.
\begin{proposition}[Proposition 3.2 in \cite{Ghandour}]
Let $X=(a\alpha,b\beta),Y=(a\gamma,b\delta)\in T_{(a,b)}(\Sl\times\Sl)$. Then
\begin{equation*}
    G(X,Y)=\frac{2}{3\sqrt{3}}(a(-\alpha\times\gamma-\alpha\times\delta+\gamma\times\beta+2\beta\times\delta),b(-2\alpha\times\gamma+\alpha\times\delta-\gamma\times\beta+\beta\times\delta)),
\end{equation*}
where $\times$ is defined on $\mathfrak{sl}(2,\R)$ by $\alpha\times\beta=\frac{1}{2}(\alpha\beta-\beta\alpha)$.
\label{tensorg}
\end{proposition}
Notice that using this proposition now we can easily see that 
\begin{equation}
    PG(X,Y)+G(PX,PY)=0.
    \label{pyg}
\end{equation}
Also from \cite{Ghandour} we see that $\Sl\times\Sl$ has constant type $-\tfrac{2}{3}$. That is 
\begin{equation}
    \begin{split}
     g(G(X,Y),G(Z,W))&=-\tfrac{2}{3}\big(g(X,Z)g(Y,W)-g(X,W)g(Y,Z)\\
     &\quad+g(JX,Z)g(Y,JW)-g(JX,W)g(Y,JZ)\big).
    \end{split}
    \label{constanttype}
    \end{equation}
We denote by $\tilde{\nabla}$ the Levi-Civita connection on $\Sl\times\Sl$ associated to $g$.  The curvature tensor $\tilde{R}$ of $\Sl\times\Sl$ associated to $\tilde{\nabla}$ is given by
\begin{equation}
        \begin{split}
            \Tilde{R}(U,V)W&=-\tfrac{5}{6}\Big(g(V,W)U-g(U,W)V\Big)\\
            &\quad-\tfrac{1}{6}\Big(g(JV,W)JU-g(JU,W)JV-2g(JU,V)JW\Big)\\
            &\quad-\tfrac{2}{3}\Big(g(PV,W)PU-g(PU,W)PV\\
            &\quad+g(JPV,W)JPU-g(JPU,W)JPV\Big).\label{curv}
        \end{split}
    \end{equation}

\subsection{The isometries of \texorpdfstring{$\Sl \times\Sl$}{SL(2,R)xSL(2,R)}} The contents of this subsection cannot be found in the literature. However, Moruz and Vrancken showed in \cite{properties} a similar result for $\Ss^3\times\Ss^3$, which can be reproduced for $\Sl\times\Sl$.

The nearly Kähler metric $g$ may also be defined from a homogeneous point of view. 
Consider the triple product $\Sl\times\Sl\times\Sl$ with the product metric that arises from the one in \eqref{prodsl2}, and the submersion 
$\pi:\Sl\times\Sl\times\Sl\to\Sl\times\Sl$ given by $\pi(a,b,c)=(ac^{-1},bc^{-1})$. 
This way, $g$ is the metric on $\Sl\times\Sl$ that makes $\pi$ into pseudo-Riemannian submersion.

The connected component of the isometry group of $\Sl\times\Sl$ is 
\[\isoo(\Sl\times\Sl)=\Sl\times\Sl\times\Sl,\] 
where an element $\phi_{(a,b,c)}$ acts on a point $(p,q)$ by $\phi_{(a,b,c)}(p,q)=(apc^{-1},bqc^{-1})$.
 Consequently, $\Sl\times\Sl$ is a homogeneous pseudo-Riemannian manifold:
\[
    \Sl\times\Sl=\frac{\Sl\times\Sl\times\Sl}{\Delta\Sl},
\]
where $\Delta \Sl=\{(a,a,a):a\in\Sl\}$.

From permutations of the factors of $\Sl\times\Sl\times\Sl$ we obtain six different isometries of $\Sl\times\Sl$:
% \begin{tasks}[label=(\arabic*)](2)
% \task \ $\Psi_{0,0}(p,q)=(p,q)$, 
% \task \ $\Psi_{1,0}(p,q)=(q,p)$,
% \task \ $\Psi_{0,2\pi/3}(p,q)=(p q^{-1},q^{-1})$,
% \task \ $\Psi_{1,2\pi/3}(p,q)=(q^{-1},p q^{-1})$,
% \task \ $\Psi_{0,4\pi/3}(p,q)=(q p^{-1},p^{-1})$,
% \task \ $\Psi_{1,4\pi/3}(p,q)=(p^{-1},q p^{-1})$.
% \end{tasks}
\begin{equation}
    \label{isoslsl}
    \begin{alignedat}{2}
        &\Psi_{0,0}(p,q)=(p,q), 
        &&\Psi_{1,0}(p,q)=(q,p),\\
        &\Psi_{0,2\pi/3}(p,q)=(p q^{-1},q^{-1}),
        &&\Psi_{1,2\pi/3}(p,q)=(q^{-1},p q^{-1}),\\
        &\Psi_{0,4\pi/3}(p,q)=(q p^{-1},p^{-1}),\qquad\qquad
        &&\Psi_{1,4\pi/3}(p,q)=(p^{-1},q p^{-1}).
    \end{alignedat}
\end{equation}
Each one of these isometries is in a different connected component of $\iso(\Sl\times\Sl)$ and satisfies
\[
    J \circ d\Psi_{\kappa,\tau}=(-1)^\kappa  d\Psi_{\kappa,\tau}\circ J,
% \]
% and
% \[
\ \ \ \ \ \ \ P\circ d \Psi_{\kappa,\tau}=d\Psi_{\kappa,\tau}\circ(\cos\tau P+\sin \tau J P).
\]







% By making compositions with the isometries $\phi_{abc}$, we obtain six different connected components of $\iso(\Sl\times\Sl)$. 


% Note that the following maps are isometries of $\Sl\times\Sl$:
% \begin{equation}
% \begin{split}
%     \phi_1(p,q)=(q,p), \ \ & \ \phi_2(p,q)=(p^{-1},qp^{-1}), \ \ \ \phi_{abc}(p,q)=(apc,bqc),
% \end{split}    \label{isoslsl}
% \end{equation}
% where $a$, $b$ and $c$ are matrices in $\Sl$. By making different compositions of these isometries, we obtain six different connected components of $\operatorname{Iso}(\Sl\times\Sl)$, the isometry group. By similar means as in \cite{properties}, we get that the only three almost product structures satisfying \eqref{proppe}, \eqref{pyg} and~\eqref{curv} are $\cos \tau P+\sin \tau JP$ with $\tau=0$, $2\pi/3$ or $4\pi/3$. Then, we can characterise each different connected component by the values $\tau$ and $\kappa$, where $(-1)^\kappa$ is $1$ if the isometry commutes with $J$, or~$-1$ if it anticommutes with $J$. We denote isometries of the connected components by $\Psi_{\tau,\kappa}$.
    
    \subsection{Comparison with the product metric}
Let $\li,\ri$ be the product metric associated to the metric given in (\ref{prodsl2}), with Levi-Civita connection $\nabla^E$. Here $E$ stands for Euclidean, as the product metric is inherited from $\R^8_4$. We can write $\li,\ri$ in terms of the nearly Kähler metric $g$ and the almost product structure $P$ as
\begin{equation}
    \li X,Y\ri=2 g(X,Y)+g(X,PY), \label{prodmetric}
\end{equation}
    and the connection $\nabla^E$ in terms of the pseudo-nearly Kähler connection $\tilde\nabla$, $J,P$ and $G$ as
\begin{equation}
    \nabla^E_XY=\tilde{\nabla}_XY+\frac{1}{2}(JG(X,PY)+JG(Y,PX)).\label{relprodkal}
\end{equation}
A natural almost product structure for a product manifold that is compatible with the product metric is $Q$, given by
\[
Q(X_1,X_2)=(-X_1,X_2).
\]
Both product structures are related by
\begin{equation}
    QX=-\frac{1}{\sqrt{3}}(2PJX-JX).\label{prodQ}
\end{equation}
Let $D$ be the Euclidean connection for $\R^8_4$, then for $(a,b)\in\Sl\times\Sl$ we have:
\begin{equation*}
\begin{split}
        D_XY&=\nabla_X^EY+\frac{\langle D_XY,(a,b)\rangle}{\langle (a,b),(a,b)\rangle}(a,b)+\frac{\langle D_XY,(-a,b)\rangle}{\langle (-a,b),(-a,b)\rangle}(-a,b)\\
        &=\nabla_X^EY-\frac{1}{2}\langle D_XY,(a,b)\rangle(a,b)-\frac{1}{2}\langle D_XY,(-a,b)\rangle(-a,b).\\
\end{split}
\end{equation*}
As $\langle Y,(a,b)\ri=0$, this implies
\begin{equation}
\begin{split}
        D_XY&=\nabla_X^EY+\frac{1}{2}\langle X,Y\rangle(a,b)+\frac{1}{2}\langle Y,QX\rangle(-a,b). \label{connectionr8}
\end{split}
\end{equation}

\section{Lagrangian submanifolds of \texorpdfstring{$\Sl\times\Sl$}{SL(2,R)xSL(2,R)}}\label{lagrangiansubmanifolds}

We start by recalling some general facts from submanifold theory.
Let $f:M\to N$ be a non-degenerate pseudo-Riemannian immersion. The Gauss formula gives a relation between the Levi-Civita connection of the ambient space and the Levi-Civita connection of the submanifold as follows:
\begin{equation}
    \tilde{\nabla}_XY=\nabla_XY+h(X,Y),\label{gaussformula}
\end{equation}
where $X,Y$ are vector fields on $M$ and $h$ is a symmetric bilinear normal form called the second fundamental form. 
If $h$ vanishes everywhere then $M$ is said to be totally geodesic.

 A relation between $\tilde{\nabla}$ and the normal connection, is provided by the Weingarten formula:
 \[
 \tilde{\nabla}_X\xi=-S_{\xi}X+\nabla_X^{\bot}\xi
 \]
where $X$ and $\xi$ are tangent and normal vector fields on $M$, respectively. For a normal vector field $\xi$, the tensor $S_\xi$ is a symmetric (with respect to $g$) tensor called the shape operator, which satisfies $g(S_\xi X,Y)=g(h(X,Y),\xi)$ and it is linear at $\xi$ and $X$.

The Gauss and Codazzi equations are given by
\begin{equation*}
    \begin{split}
        \left(\tilde{R}(X,Y)Z\right)^{\top}&=R(X,Y)Z+S_{h(X,Z)}Y-S_{h(Y,Z)}X,\\
        \left(\tilde{R}(X,Y)Z\right)^{\bot}&=(\overline{\nabla}_Xh)(Y,Z)-(\overline{\nabla}_Yh)(X,Z),
    \end{split}
\end{equation*}
where $(\overline{\nabla}_Xh)(Y,Z)=\nabla_X^{\bot}h(Y,Z)-h(\nabla_XY,Z)-h(Y,\nabla_XZ)$ and $\tilde{R},R$ are the curvature tensors of $N$ and $M$, respectively.

Let us now consider Lagrangian submanifolds. 
A submanifold $M^n$ of an almost Hermitian manifold $(N^{2n},g,J)$ is said to be Lagrangian if $J$ maps tangent spaces of $M$ into normal spaces, and vice versa.
We will also assume that the submanifold is non-degenerate, i.e. there does not exist any vector field $X$ on $M$ such that $g(X,Y)=0$ for all vector fields $Y$ on $M$.

Given a Lagrangian immersion into a (pseudo-)nearly Kähler manifold,  we extract from \cite{Schafer} some properties of the tensor $G$ and the second fundamental form:
\begin{equation}
    \begin{split}
    g(G(X,Y),Z)&=0,\\
    g(h(X,Y),JZ)&=g(h(X,Z),JY).\\
    \end{split}
    \label{lagrprop}
\end{equation}

The proof of the following result can be found in \cite{Schafer} as well.
\begin{proposition}
Any Lagrangian submanifold of a six-dimensional strictly pseudo-nearly Kähler  \hyphenation{ma-ni-fold} manifold is orientable and minimal. \label{minimal}
\end{proposition}
% Recall that a submanifold is said to be minimal if the trace of $S_\xi $ is zero for all normal vector fields $\xi$.

Consider again the tensor $P$. As the submanifold is Lagrangian, we can write the tangent bundle of the ambient space as $T(\Sl\times\Sl)=TM\oplus JTM$, so there exist two endomorphisms $A,B:TM\rightarrow TM$ such that $P|_M=A+JB$. From Equation (\ref{proppe}) we deduce that these operators are symmetric with respect to $g$, commute with each other and $A^2+B^2=\id$. 
Then, the Gauss and Codazzi equations follow from (\ref{curv}) as:
\begin{equation}
    \begin{split}
        R( X,Y)Z&=-\tfrac{5}{6}\big(g(Y,Z) X-g( X,Z)Y\big)\\
            &\quad -\tfrac{2}{3}\big(g(AY,Z)A X-g(A X,Z)AY+g(BY,
            Z)B X\\
            &\quad-g(B X,Z)BY\big)-S_{h( X,Z)}Y+S_{h(Y,Z)} X,
            \label{Gauss}
    \end{split}
\end{equation}
\begin{equation}
\begin{split}
    (\overline{\nabla}_ X h)(Y,Z)-(\overline{\nabla}_Y h)( X,Z)&=-\tfrac{2}{3}\big(g(AY,Z)JB X-g(A X,Z)JBY\\
    &\quad-g(BY,Z)JA X+g(B X,Z)JAY\big).\label{Codazzi}
\end{split}
\end{equation}




A basis $\{e_1,e_2,e_3\}$ of $\R^3_1$ is said to be $\Delta_i$-orthonormal if the matrix of inner products is given by $\Delta_i$, where
\begin{equation*}
\Delta_1=\begin{pmatrix}
    -1 & 0 & 0 \\
    0 & 1 & 0 \\
    0  & 0 & 1 \\
\end{pmatrix},\ \ \ \
\Delta_2=\begin{pmatrix}
    0 & 1 & 0 \\
    1 & 0 & 0 \\
    0  & 0 & 1 \\
\end{pmatrix},\ \ \ \ 
\Delta_3=\begin{pmatrix}
    1 & 0 & 0 \\
    0 & -1 & 0 \\
    0  & 0 & 1 \\
\end{pmatrix} .
\end{equation*}

A positive oriented frame on a Lorentzian manifold $M$ is said to be a $\Delta_i$-orthonormal frame if it is a $\Delta_i$-orthonormal basis at each point.

Notice that symmetric operators no longer necessarily diagonalize in pseudo-Riemannian ambient spaces, so we have to resort to a result from \cite{Magid} which we will adapt to our particular case.
\begin{lemma}
Let $A$ and $B$ be two symmetric operators with respect to a Lorentzian metric on a three-dimensional vector space $V$. Assume that they commute and that $A^2 + B^2 = \id$. Then $A$ and $B$ must take one of the following forms, with respect to a $\Delta_i$-orthonormal basis.
\begin{table}[H] %this centers the table on the page
  \makebox[\textwidth]{
  \begin{tabular}{l l l l}
    \columntag{(1)}{case:8.1} \label{cc1} &
    $A = \begin{pmatrix}
    \cos 2\theta_1 & 0 & 0 \\
    0 & \cos 2\theta_2 & 0 \\
    0 & 0 & \cos 2\theta_3
    \end{pmatrix}$, & \hspace*{0.5 cm}&
    $B = \begin{pmatrix}
     \sin 2\theta_1 & 0 & 0 \\
    0 & \sin 2\theta_2 & 0 \\
    0 & 0 & \sin 2\theta_3
    \end{pmatrix}$, \\[4ex]
     \multicolumn{4}{l}{with $\Delta_i=\Delta_1$
    and $\theta_1,\theta_2,\theta_3\in[0,\pi)$. \hspace*{9 cm}}
    % \\[2ex]
\end{tabular}
}
\end{table}

    \begin{table}[H] %this centers the table on the page
        \centering
        \makebox[\textwidth]{
        \begin{tabular}{l l l}
     \columntag{(2)}{case:8.1b} \label{cc1b}&
    $A = \begin{pmatrix}
    \cosh \lambda & 0 & 0 \\
    0 & \cosh \lambda & 0 \\
    0 & 0 & \cos 2\theta
    \end{pmatrix}$, &
    $B = \begin{pmatrix}
   0& \sinh \lambda  & 0 \\
    -\sinh \lambda & 0&0 \\
    0 & 0 & \sin 2\theta
    \end{pmatrix}$, \\[4ex]
     \multicolumn{3}{l}{with $\Delta_i=\Delta_1$,  $\lambda\in\R$
    and $\theta\in[0,\pi)$.}
    \\[2ex]
      \columntag{(3)}{case:8.1c} \label{cc1c}&
    $A = \begin{pmatrix}
    \varepsilon_1 & 0 & 0 \\
    0 & \varepsilon_1 & 0 \\
    0 & 0 & \varepsilon_1
    \end{pmatrix}$, &
    $B = \begin{pmatrix}
   0& \varepsilon_2  & 0 \\
   0 & 0&0 \\
    0 & 0 & 0
    \end{pmatrix}$, \\[4ex]
     \multicolumn{3}{l}{with $\Delta_i=\Delta_2$ and $\varepsilon_1,\varepsilon_2\in \{-1,1\}$.}
    \\[2ex]
    \columntag{(4)}{case:8.2} \label{cc2} &
    $A = \begin{pmatrix}
    \cos 2\theta_1 & \varepsilon & 0 \\
    0 & \cos 2\theta_1 & 0 \\
    0 & 0 & \cos 2\theta_2
    \end{pmatrix}$, &
    $B = \begin{pmatrix}
    \sin 2\theta_1 & \frac{-(c^2 + 2 \varepsilon\cos 2\theta_1)}{2 \sin 2\theta_1} & c \\
    0 & \sin 2\theta_1 & 0 \\
    0 & c & \sin 2\theta_2
    \end{pmatrix}$, 
    \\[4ex]
    \multicolumn{3}{l}{  with $\Delta_i=\Delta_2$, 
    $\theta_1,\theta_2 \in [0,\pi)$, $\theta_1 \ne 0,\pi/2$, $\varepsilon=\pm1$ and $c \in \mathbb R$.
    If $c \ne 0$, then $\cos 2\theta_1 = \cos 2\theta_2$}\\[1ex] \multicolumn{3}{l}{and $\sin 2\theta_1 = -\sin 2\theta_2$.}
    \\[2ex]
    \columntag{(5)}{case:8.3} \label{cc3} &
    $A = \begin{pmatrix}
    -\varepsilon & \varepsilon & 0 \\ 0 & -\varepsilon & 0 \\ 0 & 0 & -\varepsilon
    \end{pmatrix}$, &
    $B = \begin{pmatrix}
    0 & t & \sqrt{2} \\
    0 & 0 & 0 \\
    0 & \sqrt{2} & 0
    \end{pmatrix}$,
    \\[4ex]
    \multicolumn{3}{l}{ with $\Delta_i=\Delta_2$, $\varepsilon=\pm1$ and $t \in \R$.}
    \\[2ex]
 %    % \\[2ex] here it breaks
 %    \end{tabular}
 %  }
 %  \end{table}
 % \begin{table}[H] %this centers the table on the page
 %  \centering
 %  \makebox[\textwidth]{
 %  \begin{tabular}{l l l l} %here ends the breake
    \columntag{(6)}{case:8.4}  \label{cc4}&
    $A = \begin{pmatrix}
    \cos 2\theta & 0 & 1\\
    0 & \cos 2\theta & 0 \\
    0 & 1 & \cos 2\theta
    \end{pmatrix}$, &
    $B = \begin{pmatrix}
    \sin 2\theta & -(\csc 2\theta)^3/2 & -\cot 2\theta \\
    0 & \sin 2\theta & 0 \\
    0 & -\cot 2\theta & \sin 2\theta
    \end{pmatrix}$,
    \\[4ex]
    \multicolumn{3}{l}{with $\Delta_i=\Delta_2$, and $\theta \ne 0,\pi/2$.}
    \\[2ex]
%         \end{tabular}
%   }
%   \end{table}
%  \begin{table}[H] %this centers the table on the page
%   \centering
%   \makebox[\textwidth]{
%   \begin{tabular}{l l l l} %here ends the breake
    \columntag{(7)}{case:8.5} \label{cc5} &
    $A = \begin{pmatrix}
    s \cos 2\theta_1 & x & 0 \\
    -x & s \cos 2\theta_1 & 0 \\
    0 & 0 & \cos 2\theta_2
    \end{pmatrix}$, &
    $B = \begin{pmatrix}
    s \sin 2\theta_1 & y & 0 \\
    -y & s \sin 2\theta_1 & 0 \\
    0 & 0 & \sin 2\theta_2
    \end{pmatrix}$,
    \\[4ex]
    \multicolumn{3}{l}{ with $\Delta_i=\Delta_3$, $s=\sqrt{1+x^2+y^2}$, $x \ne 0$, $y\sin 2\theta_1 = -x \cos 2\theta_1$ and $\theta_1,\theta_2 \in [0,\pi)$.}
  \end{tabular}
  }
\end{table}

\label{propABcruda}
\end{lemma}





\begin{proof}
In \cite{Magid} it is shown that in a real $3$-dimensional vector space $V$ equipped with a Lorentzian metric, two symmetric linear transformations $A$ and $B$ that commute, can be put into one of the following forms with respect to $\Delta_i$-orthonormal bases:
% \begin{align}
%     A &=\begin{pmatrix}
%             \lambda_1 & 0 & 0 \\
%             0 & \lambda_2 & 0 \\
%             0 & 0 & \lambda_3 \\
%         \end{pmatrix}, &&
%     \text{with $\Delta_{i}=\Delta_1$}, &&\label{case1A} \\
%     A &=\begin{pmatrix}
%             \lambda_1 & 1 & 0 \\
%             0 & \lambda_1 & 0 \\
%             0 & 0 & \lambda_2 \\
%         \end{pmatrix}, &&
%     \text{with $\Delta_{i}=\Delta_2$},\label{case2,3A} \\
%     A &=\begin{pmatrix}
%             \lambda & 0 & 1 \\
%             0 & \lambda & 0 \\
%             0 & 1 & \lambda \\
%         \end{pmatrix}, && \text{with $\Delta_{i}=\Delta_2$}, &&\label{case4A}  \\
%                A&=\begin{pmatrix}
%             \alpha & \beta & 0 \\
%             -\beta & \alpha & 0 \\
%             0 & 0 & \lambda \\
%         \end{pmatrix}, && \text{with $\Delta_{i}=\Delta_3$},  && \beta\neq 0.  \label{case5A} 
% \end{align}
% If $B$ is another symmetric operator on $V$, such that $[A,B]=0$, then we have that $A$ and $B$ will take one of the forms listed below. We separate the cases when $A$ is diagonal.


\begin{align}
    A &=\begin{pmatrix}
            \lambda_1 & 0 & 0 \\
            0 & \lambda_2 & 0 \\
            0 & 0 & \lambda_3 \\
        \end{pmatrix},  && 
    B =\begin{pmatrix}
        \mu_1 & 0 & 0 \\
        0 & \mu_2 & 0 \\
        0 & 0 & \mu_3 \\
        \end{pmatrix},  && 
    \text{with $\Delta_{i}=\Delta_1$}, &&\label{case1} \\
        A &=\begin{pmatrix}
            \lambda_1 & 0 & 0 \\
            0 & \lambda_1 & 0 \\
            0 & 0 & \lambda_2 \\
        \end{pmatrix},  && 
    B =\begin{pmatrix}
        \mu_1 & \mu_2 & 0 \\
        -\mu_2 & \mu_1& 0 \\
        0 & 0 & \mu_3 \\
        \end{pmatrix},  && 
    \text{with $\Delta_{i}=\Delta_1$}, && \mu_2\neq0,\label{case1to4} \\
    %     A &=\begin{pmatrix}
    %         \lambda_1 & 0 & 0 \\
    %         0 & \lambda_2 & 0 \\
    %         0 & 0 & \lambda_2 \\
    %     \end{pmatrix},  && 
    % B =\begin{pmatrix}
    %     \mu_1 & 0 & 0 \\
    %     0 & \mu_2 & \mu_3 \\
    %     0 & \mu_3& \mu_4 \\
    %     \end{pmatrix},  && 
    % \text{with $\Delta_{i}=\Delta_1$}, && \lambda_1\neq\lambda_2 \label{case1} \\
    %     A &=\begin{pmatrix}
    %         \lambda & 0 & 0 \\
    %         0 & \lambda & 0 \\
    %         0 & 0 & \lambda \\
    %     \end{pmatrix},  && 
    % B =\begin{pmatrix}
    %     \mu_1 & 0 & 0 \\
    %     0 & \mu_2 & 0 \\
    %     0 & 0 & \mu_3 \\
    %     \end{pmatrix},  && 
    % \text{with $\Delta_{i}=\Delta_1$}, &&\label{case1} \\
   A &=\begin{pmatrix}
            \lambda & 0 & 0 \\
            0 & \lambda & 0 \\
            0 & 0 & \lambda \\
        \end{pmatrix}, && 
    B =\begin{pmatrix}
        \mu_1 & \varepsilon & 0 \\
        0 & \mu_1 & 0 \\
        0 & 0 & \mu_2 \\
        \end{pmatrix},  && 
    \text{with $\Delta_{i}=\Delta_2$}, &&\label{case1to2} 
    \end{align}
    \begin{align}
   A &=\begin{pmatrix}
            \lambda & 0 & 0 \\
            0 & \lambda & 0 \\
            0 & 0 & \lambda \\
        \end{pmatrix}, && 
    B =\begin{pmatrix}
        \mu & 0 & 1 \\
        0 & \mu & 0 \\
        0 & 1 & \mu \\
        \end{pmatrix},  && 
    \text{with $\Delta_{i}=\Delta_2$}, &&\label{case1.3b} \\
% \end{align}
% \begin{align}
    A &=\begin{pmatrix}
            \lambda_1 & \varepsilon & 0 \\
            0 & \lambda_1 & 0 \\
            0 & 0 & \lambda_2 \\
        \end{pmatrix}, && 
    B =\begin{pmatrix}
        \mu_1 & b & c \\
        0 & \mu_1 & 0 \\
        0 & c & \mu_2 \\
        \end{pmatrix}, &&
    \text{with $\Delta_{i}=\Delta_2$}, && c\lambda_1=c\lambda_2, \label{case2,3} \\
    A &=\begin{pmatrix}
            \lambda & 0 & 1 \\
            0 & \lambda & 0 \\
            0 & 1 & \lambda \\
        \end{pmatrix}, &&
    B =\begin{pmatrix}
        \mu & b & c \\
        0 & \mu & 0 \\
        0 & c & \mu \\
        \end{pmatrix}, && \text{with $\Delta_{i}=\Delta_2$}, &&\label{case4} \\
               A&=\begin{pmatrix}
            \alpha & \beta & 0 \\
            -\beta & \alpha & 0 \\
            0 & 0 & \lambda \\
        \end{pmatrix}, &&
        B=\begin{pmatrix}
        \gamma & \delta & 0 \\
        -\delta & \gamma & 0 \\
        0 & 0 & \mu \\
        \end{pmatrix}, && \text{with $\Delta_{i}=\Delta_3$},  && \beta\neq 0,  \label{case5} 
\end{align}
with $\varepsilon=\pm1$.
We will analyse the equation $A^2+B^2=\id$ on each type of matrix separately.

\textit{Type 1 \emph{(\ref{case1})}}:
 Computing $A^2+B^2=\id$ in (\ref{case1}) immediately yields Case \ref{case:8.1} of Lemma \ref{propABcruda}.

\textit{Type 2 \emph{(\ref{case1to4})}}:
It follows from $A^2+B^2=\id$ that $\mu_1=0$, $\lambda_1^2-\mu_2^2=1$ and $\lambda^2_2+\mu_3^2=1$. Hence, we get Case \ref{case:8.1b} of Lemma \ref{propABcruda}.

\textit{Type 3 \emph{(\ref{case1to2})}}:
Case \ref{case:8.1c} is immediate from computing $A^2+B^2=\id$,  as we obtain that $\mu_1=0$. It follows that $\lambda=\pm 1$ and that $\mu_2=0$. 
This is Case \ref{case:8.1c} of Lemma \ref{propABcruda}.

\textit{Type 4 \emph{(\ref{case1.3b})}}:
We easily see that under no conditions $A^2+B^2$ can be equal to the identity in this case.

\textit{Type 5 \emph{(\ref{case2,3})}}:
Computing $A^2+B^2=\id$ in (\ref{case2,3}) yields the equations
\begin{align}
      \lambda_1^2+\mu_1^2&=1,\label{1eq}\\ 
        \lambda_2^2+\mu_2^2&=1, \label{2eq}\\
        2\varepsilon\lambda_1+2b\mu_1+c^2&=0, \label{3eq}\\ 
        c(\mu_1+\mu_2)&=0.\label{4eq}
\end{align}

Suppose $c\neq 0$. Then because of Equation (\ref{4eq}) and $c\lambda_1=c\lambda_2$ we have that $\lambda_1=\lambda_2$ and $\mu_1=-\mu_2$. 
If $\mu_1=0$ then (\ref{1eq}) and (\ref{3eq}) imply that $\lambda_1=\lambda_2=-\varepsilon$ and $c=\sqrt{2}$, which is Case \ref{case:8.3} of Lemma \ref{propABcruda}. 
If instead $\mu_1\neq 0$, by Equation (\ref{1eq}) we can write $\lambda_1=\cos2\theta_1$ and $\mu_1=\sin2\theta_1$, then by (\ref{3eq}) we have $b=-(c^2+2\varepsilon\cos2\theta_1)/(2\sin2\theta_2)$, which leave us with Case \ref{case:8.2} of Lemma~\ref{propABcruda}. 

Now suppose that $c=0$. If $\mu_1=0$ then we get a contradiction from equations (\ref{1eq}) and (\ref{3eq}). 
Therefore $\mu_1$ must be different from zero, and from (\ref{1eq}), (\ref{2eq}) and (\ref{3eq}) we get that $\lambda_1=\cos2\theta_1$, $\lambda_2=\cos2\theta_2$, $\mu_1=\sin2\theta_1$, $\mu_2=\sin2\theta_2$ and $b=-\varepsilon\cot2\theta_1$, which is again Case \ref{case:8.2} of Lemma \ref{propABcruda}.

\textit{Type 6 \emph{(\ref{case4})}}: Computing $A^2+B^2=\id$ yields
\begin{align}
    \lambda^2+\mu^2=1,\label{eqqq1}\\
    \lambda+c\mu=0, \label{eqqq2} \\
     2 b \mu +c^2+1=0. \label{eqqq3} 
\end{align}
From (\ref{eqqq1}) and (\ref{eqqq2}) we easily see that $\mu\neq 0$, thus $c=-\lambda/\mu$. By replacing this in (\ref{eqqq3}) we get that $b=-1/(2\mu^3)$. Finally by (\ref{eqqq1}) we have that $\lambda=\cos2\theta$, $\mu=\sin2\theta$, $c=-\cot2\theta$ and $b=-\csc(2\theta)^3/2$, which is Case \ref{case:8.4} in Lemma \ref{propABcruda}.

\textit{Type 7 \emph{(\ref{case5})}}: Again, we compute $A^2+B^2=\id$ in (\ref{case5}), from which follows
\begin{align}
    \alpha^2-\beta^2+\gamma^2-\delta^2=1, \label{c51} \\
    \alpha\beta+\gamma\delta=0, \label{c52}\\
    \lambda^2+\gamma^2=1. \label{c53}
\end{align}
We can transform (\ref{c51}) into
\begin{equation*}
    \begin{split}
     \left(\frac{\alpha}{\sqrt{1+\beta^2+\delta^2}}\right)^2+\left(\frac{\gamma}{\sqrt{1+\beta^2+\delta^2}}\right)^2=1,\\
    \end{split}
\end{equation*}
thus we obtain that $\alpha=\sqrt{1+\beta^2+\delta^2}\cos\theta$ and $ \gamma=\sqrt{1+\beta^2+\delta^2}\sin\theta$ for some $\theta$. 
Equation~(\ref{c52}) becomes $\sqrt{1+\beta^2+\delta^2}(\beta\cos\theta+\delta\sin\theta)=0$, hence we can conclude that $\delta=-\beta\cot\theta$ since if $\sin\theta=0$ then $\beta=0$, which contradicts the last condition of (\ref{case5}).
 Renaming $x=\beta$, $y=\delta$ and $s=\sqrt{1+x^2+y^2}$ we obtain Case \ref{case:8.5} of Lemma \ref{propABcruda}.
\end{proof}


\begin{remark}
    In \cite{Magid}, only the case $\varepsilon=1$ in equations \eqref{case1to2} and \eqref{case2,3}. However, the case $\varepsilon=-1$ is essentially different, as there is no change of basis preserving the metric that can take one case into the other.
\end{remark}







% \begin{proof}
% In \cite{Magid} we see that for a real $3$-dimensional vector space $V$ equipped with a
% Lorentzian metric, two symmetric linear transformations $A$ and $B$ such that $[A,B]=0$, can be put into one of the following four forms with respect to $\Delta_i$-orthonormal bases:
% \begin{align}
%     A &=\begin{pmatrix}
%             \lambda_1 & 0 & 0 \\
%             0 & \lambda_2 & 0 \\
%             0 & 0 & \lambda_3 \\
%         \end{pmatrix},  && 
%     B =\begin{pmatrix}
%         \mu_1 & 0 & 0 \\
%         0 & \mu_2 & 0 \\
%         0 & 0 & \mu_3 \\
%         \end{pmatrix},  && 
%     \text{with $\Delta_{i}=\Delta_1$}, &&\label{case1} \\
%     A &=\begin{pmatrix}
%             \lambda_1 & 1 & 0 \\
%             0 & \lambda_1 & 0 \\
%             0 & 0 & \lambda_2 \\
%         \end{pmatrix}, && 
%     B =\begin{pmatrix}
%         \mu_1 & b & c \\
%         0 & \mu_1 & 0 \\
%         0 & c & \mu_2 \\
%         \end{pmatrix}, &&
%     \text{with $\Delta_{i}=\Delta_2$}, && c\lambda_1=c\lambda_2, \label{case2,3} \\
%     A &=\begin{pmatrix}
%             \lambda & 0 & 1 \\
%             0 & \lambda & 0 \\
%             0 & 1 & \lambda \\
%         \end{pmatrix}, &&
%     B =\begin{pmatrix}
%         \mu & b & c \\
%         0 & \mu & 0 \\
%         0 & c & \mu \\
%         \end{pmatrix}, && \text{with $\Delta_{i}=\Delta_2$}, &&\label{case4} 
%         % \\
%             \end{align}
%     \begin{align}
%                A&=\begin{pmatrix}
%             \alpha & \beta & 0 \\
%             -\beta & \alpha & 0 \\
%             0 & 0 & \lambda \\
%         \end{pmatrix}, &&
%         B=\begin{pmatrix}
%         \gamma & \delta & 0 \\
%         -\delta & \gamma & 0 \\
%         0 & 0 & \mu \\
%         \end{pmatrix}, && \text{with $\Delta_{i}=\Delta_3$},  && \beta\neq 0.  \label{case5} 
% \end{align}
% We will analyse the equation $A^2+B^2=\id$ on each type of matrix separately.

% \textit{Type 1 \emph{(\ref{case1})}}:

% Computing $A^2+B^2=\id$ in (\ref{case1}) immediately yields Case (1) in Lemma \ref{propABcruda}.

% \textit{Type 2 \emph{(\ref{case2,3})}}:

% Computing $A^2+B^2=\id$ in (\ref{case2,3}) yields the equations
% \begin{align}
%       \lambda_1^2+\mu_1^2&=1,\label{1eq}\\ 
%         \lambda_2^2+\mu_2^2&=1, \label{2eq}\\
%         2\lambda_1+2b\mu_1+c^2&=0, \label{3eq}\\ 
%         c(\mu_1+\mu_2)&=0.\label{4eq}
% \end{align}
% Suppose $c\neq 0$. Then because of the last condition on (\ref{case2,3}) and Equation (\ref{4eq}) we have that $\lambda_1=\lambda_2$ and $\mu_1=-\mu_2$. 
% If $\mu_1=0$ then (\ref{1eq}) and (\ref{3eq}) equations imply that $\lambda_1=\lambda_2=-1$ and $c=\sqrt{2}$, which is Case (3) in Lemma \ref{propABcruda}. 
% If instead $\mu_1\neq 0$, by Equation (\ref{1eq}) we can write $\lambda_1=\cos2\theta_1$ and $\mu_1=\sin2\theta_1$, then by (\ref{3eq}) $b=-(c^2+2\cos2\theta_1)/2\sin2\theta_2$, which leave us with Case (2) of Lemma \ref{propABcruda}. 

% \noindent
% Now suppose that $c=0$. If $\mu_1=0$ then we get a contradiction from equations (\ref{1eq}) and (\ref{3eq}). Therefore $\mu_1$ must be different from zero, and from (\ref{1eq}), (\ref{2eq}) and (\ref{3eq}) we get that $\lambda_1=\cos2\theta_1$, $\lambda_2=\cos2\theta_2$, $\mu_1=\sin2\theta_1$, $\mu_2=\sin2\theta_2$ and $b=-\cot(2\theta_1)$, which is again Case (2) of Lemma \ref{propABcruda}.

% \textit{Type 3 \emph{(\ref{case4})}}: Consider now (\ref{case4}). Computing $A^2+B^2=\id$ yields
% \begin{align}
%     \lambda^2+\mu^2=1,\label{eqqq1}\\
%     \lambda+c\mu=0, \label{eqqq2} \\
%      2 b \mu +c^2+1=0. \label{eqqq3} 
% \end{align}
% From (\ref{eqqq1}) and (\ref{eqqq2}) we easily see that $\mu\neq 0$, thus $c=-\lambda/\mu$. By replacing this in (\ref{eqqq3}) we get that $b=-1/(2\mu^3)$. Finally by (\ref{eqqq1}) we have that $\lambda=\cos2\theta$, $\mu=\sin2\theta$, $c=-\cot2\theta$ and $b=-\csc(2\theta)^3/2$, which is Case (4) in Lemma \ref{propABcruda}.

% \textit{Type 4 \emph{(\ref{case5}})}: Again, we compute $A^2+B^2=\id$ in (\ref{case5}), from which it follows
% \begin{align}
%     \alpha^2-\beta^2+\gamma^2-\delta^2=1, \label{c51} \\
%     \alpha\beta+\gamma\delta=0, \label{c52}\\
%     \lambda^2+\gamma^2=1. \label{c53}
% \end{align}
% We can transform (\ref{c51}) into
% \begin{equation*}
%     \begin{split}
%      \left(\frac{\alpha}{\sqrt{1+\beta^2+\delta^2}}\right)^2+\left(\frac{\gamma}{\sqrt{1+\beta^2+\delta^2}}\right)^2=1,\\
%     \end{split}
% \end{equation*}
% thus we obtain that $\alpha=\sqrt{1+\beta^2+\delta^2}\cos(\theta)$ and $ \gamma=\sqrt{1+\beta^2+\delta^2}\sin(\theta)$ for some $\theta$. Equation~(\ref{c52}) becomes $\sqrt{1+\beta^2+\delta^2}(\beta\cos(\theta)+\delta\sin(\theta))=0$, hence we can conclude that $\delta=-\cot(\theta)\beta$ since if $\sin(\theta)=0$ then $\beta=0$, which contradicts the last condition of (\ref{case5}). Renaming $x=\beta$, $y=\delta$ and $s=\sqrt{1+x^2+y^2}$ we obtain Case (4) in Lemma \ref{propABcruda}.
% \end{proof}








Now, fixing a point $p\in M$ we know that there exists a basis of $T_pM$ as in Lemma \ref{propABcruda}. Then we can extend locally this basis to a $\Delta_i$-orthonormal frame $\{E_1,E_2,E_3\}$ on $M$.

From the first equation of (\ref{lagrprop}) it follows that $G(X,Y)$ is a normal vector to $M$ for any tangent vectors $X,Y$ on $M$. Let $\{E_1,E_2,E_3\}$ be a $\Delta_i$-orthonormal frame of the tangent space of $M$.
By (\ref{gnormal}) and (\ref{constanttype}) we get $JG(E_j,E_k)=\varepsilon \sqrt{\tfrac{2}{3}}E_l$ with $\varepsilon=\pm1$ depending on $\Delta_i$, as the following table shows.  

\begin{equation}
    \begin{tblr}{|c|c|c|c|}
\hline
      &   \Delta_1 &   \Delta_2  & \Delta_3  \\   
     \hline
     JG(E_1,E_2)  &  \sqrt{\frac{2}{3}}E_3  &  \sqrt{\frac{2}{3}}E_3 & \sqrt{\frac{2}{3}}E_3 \\
     JG(E_1,E_3)  &  -\sqrt{\frac{2}{3}}E_2  &  -\sqrt{\frac{2}{3}}E_1  & \sqrt{\frac{2}{3}}E_2 \\
     JG(E_2,E_3)  &  -\sqrt{\frac{2}{3}}E_1  &  \sqrt{\frac{2}{3}}E_2   & \sqrt{\frac{2}{3}}E_1 \\
     \hline
\end{tblr}\label{tabla}
\end{equation}

With this information we can state a stronger version of Lemma $\ref{propABcruda}$.
 \begin{lemma}
 Let $M$ be a Lagrangian submanifold of $\Sl\times\Sl$ and $P$ the almost product structure given in $(\ref{prodstructuredef})$. 
 Then there exists a Lagrangian submanifold $N$ congruent to $M$ such that the restriction of $P$ to $N$ can be written as $P|_N=A+JB$, where $A,B:TN\to TN$ must have one of the following forms, with respect to a $\Delta_i$-orthonormal frame $\{E_1,E_2,E_3\}$.
\begin{table}[H]
  \centering
  \makebox[\textwidth]{
  \begin{tabular}{l l l l}
    \columntag{(1)}{case:10.1} &
    $A = \begin{pmatrix}
    \cos 2\theta_1 & 0 & 0 \\
    0 & \cos 2\theta_2 & 0 \\
    0 & 0 & \cos 2\theta_3
    \end{pmatrix}$, & \hspace*{2 cm}&
    $B = \begin{pmatrix}
    \sin 2\theta_1 & 0 & 0 \\
    0 & \sin 2\theta_2 & 0 \\
    0 & 0 & \sin 2\theta_3
    \end{pmatrix}$, 
    \\[4ex]
    \multicolumn{4}{l}{
    with $\Delta_i=\Delta_1$ and $\theta_1 + \theta_2 + \theta_3 = 0$ modulo $\pi$.}
    % \\[2ex]
\end{tabular}}
\end{table}    
    \begin{table}[H]
        \centering
        \makebox[\textwidth]{
        \begin{tabular}{l l l}
     \columntag{(2)}{case:10.2} &
    $A = \begin{pmatrix}
    \cos 2\theta_1 & 1 & 0 \\
    0 & \cos 2\theta_1 & 0 \\
    0 & 0 & \cos 2\theta_2
    \end{pmatrix}$, &
    $B = \begin{pmatrix}
    \sin 2\theta_1 & -\cot 2\theta_1 & 0 \\
    0 & \sin 2\theta_1 & 0 \\
    0 & 0 & \sin 2\theta_2
    \end{pmatrix}$, 
    \\[4ex]
    \multicolumn{3}{l}{
     with $\Delta_i=\Delta_2$, $2\theta_1 + \theta_2 = 0$ modulo $\pi$ and $\theta_1 \ne 0, \pi/2$.}
    \\[2ex]
    \columntag{(3)}{case:10.3} &
    $A = \begin{pmatrix}
    -\frac12 & 0 & 1 \\ 0 & -\frac12 & 0 \\ 0 & 1 & -\frac12
    \end{pmatrix}$, &
    $B = \pm \begin{pmatrix}
    \frac{\sqrt 3}{2} & \frac{-4}{3\sqrt 3} & \frac{1}{\sqrt 3} \\
    0 & \frac{\sqrt 3}{2} & 0 \\
    0 & \frac{1}{\sqrt 3} & \frac{\sqrt 3}{2}
    \end{pmatrix}$,
    \\[4 ex]
    \multicolumn{3}{l}{with $\Delta_i=\Delta_2$.}
    \\[2ex]
    \columntag{(4)}{case:10.4} &
    $A = \begin{pmatrix}
    \cosh\lambda \cos 2\theta_1 & \sinh\lambda\sin\theta_2 & 0 \\
    -\sinh\lambda\sin\theta_2 & \cosh\lambda \cos 2\theta_1 & 0 \\
    0 & 0 & \cos 2\theta_2
    \end{pmatrix}$, & \\[4ex]
    & $B = \begin{pmatrix}
    \cosh\lambda \sin 2\theta_1 & \sinh\lambda\cos\theta_2 & 0 \\
    -\sinh\lambda\cos\theta_2 & \cosh\lambda \sin 2\theta_1 & 0 \\
    0 & 0 & \sin 2\theta_2
    \end{pmatrix}$,&
    \\[4ex]
    \multicolumn{3}{l}{ with $\Delta_i=\Delta_3$, $2\theta_1 + \theta_2 = 0$ modulo $\pi$, $\theta_2 \ne 0,\pi$ and $\lambda \ne 0$.}
  \end{tabular}
  }
\end{table}
\label{propAB}
 \end{lemma}
We will call $\theta_i$, $\theta$ and $\lambda$ angle functions. 
Lagrangian submanifolds such that $P$ takes the form of Case \ref{case:10.1} of Lemma \ref{propAB} are said to be of diagonalizable type.

 \begin{proof}
The next equation follows from applying $P$ to (\ref{tabla}), using (\ref{pyg}) and the fact that $PJ=-JP$:
\begin{equation}
PE_i=-\alpha JPG(E_j,E_k)=\alpha JG(PE_j,PE_k), \label{pei} 
\end{equation}
where $\alpha$ can be either $\sqrt{\tfrac{3}{2}}$ or $-\sqrt{\tfrac{3}{2}}$, depending on which case of (\ref{tabla}) we are in. For Case \ref{case:8.1}, Case \ref{case:8.1b} and Case \ref{case:8.5} of Lemma \ref{propABcruda} the triple $(i,j,k)$ is a permutation of $(1,2,3)$. For cases \ref{case:8.1c} to \ref{case:8.4} of Lemma \ref{propABcruda} this is not true anymore, and $i$ is equal either to $j$ or to $k$.
We consider the seven cases of Lemma \ref{propABcruda} applied to Equation (\ref{pei}), one by one.

\textit{Case \emph{\ref{case:8.1}}.} Using a similar procedure as in \cite{Dioos}, we get $\theta_1+\theta_2+\theta_3=0$ modulo $\pi$.
 
\textit{Case \emph{\ref{case:8.1b}}.} Taking $i=3$, $j=1$ and $k=2$ in \eqref{pei} and looking at the component of $JE_3$ we conclude that $\theta=0$.
Now we apply to $M$ the isometry $\Psi_{4\pi/3,1}$ given in $\eqref{isoslsl}$. The tensor $P$ restricted to $\Psi_{4\pi/3,1}(M)$ can be written as $\tilde{A}+J\tilde{B}$, where 
\[
\tilde{A}=-\frac{1}{2}A+\frac{\sqrt{3}}{2}B, \ \ \ \ \tilde{B}=\frac{\sqrt{3}}{2}A+\frac{1}{2}B.
\]
This is Case \ref{case:10.4} of Lemma \ref{propAB} with $\theta_1=\theta_2=\tfrac{\pi}{3}$.

\textit{Case \emph{\ref{case:8.1c}}.}
With $i=1$, $j=1$ and $k=3$ in \eqref{pei} we obtain $\varepsilon_1=1$.
As there is no possible change of basis that can transform $\varepsilon_2$ into $1$, we apply either $\Psi_{2\pi/3,1}$ or $\Psi_{4\pi/3,1}$ to $M$, depending on its value.
 Proceeding in the same way as in the previous case, we restrict $P$ to the image of the isometry. 
 After a change of basis, we get Case \ref{case:10.2} of Lemma \ref{propAB}, with $\theta_1=\theta_2=\tfrac{\pi}{3}$ if $\varepsilon_2=1$ and $\theta_1=\theta_2=\tfrac{2\pi}{3}$ if $\varepsilon_2=1$.


% Proceeding in the same way as in for Case \ref{case:8.1b} we get that $P$ restricted to $\phi_2(M)$ takes the form of Case \ref{case:10.2} in Lemma \ref{propAB}, with $\theta_1=\theta_2=\tfrac{\pi}{3}$.

 \textit{Case \emph{\ref{case:8.2}}}.
Taking $i=1$, $j=1$ and $k=1$ in (\ref{pei}) yields
\[
\cos2\theta_1=\cos2(\theta_1+\theta_2),\ \ \ \ \ \sin2\theta_1=-\sin2(\theta_1+\theta_2).
\]
Thus,
\begin{equation*}
\begin{split}
\cos2(2\theta_1+\theta_2)&=\cos2\theta_1\cos2(\theta_1+\theta_2)-\sin2\theta_1\sin2(\theta_1+\theta_2)\\
&=\cos^22(\theta_1+\theta_2)+\sin^22(\theta_1+\theta_2)\\
&=1.
\end{split}    
\end{equation*}
Then $2\theta_1+\theta_2=0$ mod $\pi$. 
Looking at the component of $E_3$ in (\ref{pei}) when taking $i=2$, $j=2$ and $k=3$ we obtain $c\sin2\theta_1=0$, therefore $c=0$ as $\sin2\theta_1\neq 0$.

If $\varepsilon=1$, we have Case \ref{case:10.2} of Lemma~\ref{propAB}.

If $\varepsilon=-1$, we would like to transform it into $1$. We apply either $\Psi_{2\pi/3,0}$ or $\Psi_{2\pi/3,1}$ to $M$ and restrict $P$ to the image. The component $\tilde{A}_{12}$ is $1+\sqrt{3} \cot 2 \theta_1$ or $1-\sqrt{3} \cot 2 \theta_1$. 
These two values cannot be negative at the same time, thus we can always choose one to be positive. After a change of basis, we get Case \ref{case:10.2} of Lemma \ref{propAB}.

\textit{Case \emph{\ref{case:8.3}}}.
On the left hand side of (\ref{pei}) we have $PE_1=-E_1$ and on the right hand side we have
\begin{equation*}
\begin{split}
    -\sqrt{\frac{3}{2}}JG(PE_1,PE_3)=-\sqrt{\frac{3}{2}}JG(-E_1,-E_3)=E_1,
\end{split}    
\end{equation*}
which is a contradiction. Therefore, there is no Lagrangian submanifold with such a frame.

\textit{Case \emph{\ref{case:8.4}}.}
Taking the component of $E_3$ on both sides of (\ref{pei})
with $i=2$, $j=2$ and  $k=3$ yields $-2\cos2\theta=1$.
Hence $\theta=\tfrac{\pi}{3}$ or $\theta=\tfrac{2}{3}\pi$. That is
\[
    A=\begin{pmatrix}
    -\frac{1}{2} & 0 & 1 \\
    0 & -\frac{1}{2} & 0 \\
    0  & 1 & -\frac{1}{2} \\
\end{pmatrix}, \ \ 
B=\pm\begin{pmatrix}
    \frac{\sqrt{3}}{2} & -\frac{4}{3\sqrt{3}} & \frac{1}{\sqrt{3}} \\
    0 & \frac{\sqrt{3}}{2} & 0 \\
    0  & \frac{1}{\sqrt{3}} & \frac{\sqrt{3}}{2} \\
\end{pmatrix}.
\]

\textit{Case \emph{\ref{case:8.5}}}.
Taking $i=1$, $j=2$ and $k=3$ in (\ref{pei}) gives us
\begin{align*}
    \cos2\theta_1&=\cos2(\theta_1+\theta_2),\\
       \sin2\theta_1&=-\sin2(\theta_1+\theta_2).
\end{align*}
Therefore as in the previous cases we have $\cos2(2\theta_1+\theta_2)=1$. We take $i=2$, $j=1$ and $k=3$ in (\ref{pei}) and we look at the components of $E_1,JE_1$:
\begin{equation}
    \begin{split}
        x&=-x\cos2\theta_2+y\sin2\theta_2,\\ y&=x\sin2\theta_2+y\cos2\theta_2.\\
    \end{split}\label{reflection}
\end{equation}
Note that the equations in (\ref{reflection}) can be written as the equation $Rv=v$ where $v=(x,y)^t$ and
\begin{equation*}
    R=\begin{pmatrix}
    \cos(\pi-2\theta_2) &  \sin(\pi-2\theta_2)\\
    \sin(\pi-2\theta_2) & -\cos(\pi-2\theta_2)\\
    \end{pmatrix}.
    \end{equation*}
Hence, $v$ is an eigenvector of the matrix $R$ associated to the eigenvalue $1$.
Now, $R$ is a reflection in the plane with respect to the straight line  $s\mapsto s(\cos(\frac{\pi}{2}-\theta_2),\sin(\frac{\pi}{2}-\theta_2))^t=s(\sin\theta_2,\cos\theta_2)^t$, therefore $v$ lies in that subspace. 
So $x=\sinh\lambda\sin\theta_2$, $y=\sinh\lambda\cos\theta_2$ for some $\lambda\in\R$ different from zero. 
Finally, replacing $x$ and $y$ in the matrices of Case \ref{cc5} of Lemma \ref{propABcruda} yields what we desired.
\end{proof}


\section{Totally geodesic Lagrangian submanifolds}\label{totallygeodesic}
Let $M$ be a Lagrangian submanifold of $\Sl\times\Sl$ and let $\{E_1,E_2,E_3\}$ be a frame given by Lemma \ref{propAB}.
As the normal space is spanned by $\{JE_1,JE_2,JE_3\}$, we may define functions $\omega_{ij}^k$ and $h_{ij}^k$ by $\nabla_{E_i}E_j=\sum_k\omega_{ij}^kE_k$ and $h(E_i,E_j)=\sum_kh_{ij}^kJE_k$.
From the second equation of (\ref{lagrprop}) and the compatibility of the connection with the metric, we obtain the following symmetries.

First, for frames $\{E_i\}$ associated to $\Delta_1$ and $\Delta_3$ we have
\[
\delta_k\omega_{ij}^k=-\delta_j\omega_{ik}^j, \ \ \ \ \ \ \ h_{ij}^k=h_{ji}^k=\delta_j\delta_kh_{ik}^j,
\]
where 
\[\delta_i=g( E_i,E_i).\]
This implies that $\omega_{ij}^j=0$ for all $i,j=1,2,3$.

If the frame is associated to $\Delta_2$ we get
 \[\omega_{ij}^k=-\omega_{i\widehat{k}}^{\widehat{j}},\ \ \ \ \ \ \ h_{ij}^k=h_{ji}^k=h_{i\widehat{k}}^{\widehat{j}}\] 
 where $\widehat{2}=1$, $\widehat{1}=2$ and $\widehat{3}=3$. As before, we have that $\omega_{i3}^3=0$. 
 Also,  if $j=1$, $k=2$ or $j=2$, $k=1$ then $\omega_{ij}^k=0$.
 
The almost product structure $P$ is not integrable, but we have a neat expression for its covariant derivative. Namely, using an pseudo-orthonormal frame we can check $(\Tilde{\nabla}_XP)JY=J(\Tilde{\nabla}_XP)Y$, from which it follows that 
\begin{equation}
(\Tilde{\nabla}_XP)Y=\frac{1}{2}(JG(X,PY)+JPG(X,Y)). \label{nablap}
\end{equation}

Equation (\ref{nablap}) is useful since it gives conditions on the components $h_{ij}^k$ and $\omega_{ij}^k$. As said equation depends on $P$, we are forced to divide between the four cases of Lemma \ref{propAB}.

\subsection{Lagrangian submanifolds of diagonalizable type}

Equation (\ref{nablap}) for Case \ref{case:10.1} of Lemma \ref{propAB} yields the following lemma.

\begin{lemma}\label{lemmacase1}Let $M$ be a Lagrangian submanifold of $\Sl\times\Sl$. Assume that the operators $A,B$ take the form of the first case in Lemma \emph{\ref{propAB}} with respect to a $\Delta_1$-orthonormal frame $\{E_1,E_2,E_3\}$. Except for $h_{12}^3$, all the components of the second fundamental form are given by the derivatives of the angle functions $\theta_1,\theta_2$ and $\theta_3$:
\begin{equation}
    E_i(\theta_j)=-\delta_i\delta_jh_{jj}^i,
    \label{anglederi}
\end{equation}
where $\delta_i=g(E_i,E_i)$. Also 
\begin{equation}
   h_{ij}^k\cos(\theta_j-\theta_k)=(\tfrac{1}{\sqrt{6}}\delta_k\varepsilon_{ij}^k-\omega_{ij}^k)\sin(\theta_j-\theta_k),\label{sffc}
\end{equation}
for $j\neq k$.
\end{lemma}
\begin{proof}
Taking $X=E_1$ and $Y=E_2$ in (\ref{nablap}) and comparing the components in $E_1,E_2,E_3,JE_1,JE_2$ and $JE_3$ yields the following six equations
\begin{align*}
    (h_{11}^2 \cos(\theta_1 - \theta_2) +  \omega_{11}^2 \sin(\theta_1 - \theta_2)) \sin(\theta_1 + \theta_2)&=0,\\
    (h_{11}^2 \cos(\theta_1 - \theta_2) + \omega_{11}^2 \sin(\theta_1 - \theta_2))\cos(\theta_1 + \theta_2) &=0,\\
    (h_{22}^1 - E_1(\theta_2)) \sin(2 \theta_2)&=0,\\
    (h_{22}^1 - E_1(\theta_2))\cos(2 \theta_2) &=0,\\
    ( h_{12}^3 \cos(\theta_2 - \theta_3) + (-\tfrac{1}{\sqrt{6} }+ \omega_{12}^3) \sin(\theta_2 - \theta_3)) \sin(\theta_2 + \theta_3)&=0,\\
    ( h_{12}^3 \cos(\theta_2 - \theta_3) + (-\tfrac{1}{\sqrt{6}} + \omega_{12}^3) \sin(\theta_2 - \theta_3))\cos(\theta_2 + \theta_3) &=0.
\end{align*}
Since sine and cosine never vanish at the same time (\ref{anglederi}) and (\ref{sffc}) hold for $i=1$ and $j=2$. The other equations follow in a similar way, by choosing different $X$ and $Y$ in (\ref{nablap}).
\end{proof}


\begin{proposition}
Let $M$ be a Lagrangian submanifold of $\Sl\times\Sl$ of diagonalizable type. If the angle functions $\theta_i$ are constant and $h_{12}^3=0$, then $M$ is totally geodesic. Conversely, if the submanifold is totally geodesic then the angles are constant.    
\end{proposition}
\begin{proof}
    By Lemma \ref{lemmacase1}, if the angle functions are constant then $h_{ii}^j=0$ for all $i,j=1,2,3$. Using the symmetries of $h_{ij}^k$ we conclude that the submanifold is totally geodesic. The converse is immediate by Lemma \ref{lemmacase1}.
\end{proof}

Now notice that
\begin{equation}
    \begin{split}
        -\delta_kE_k(h_{jj}^i)+\delta_{i}E_i(h_{jj}^k)&=\delta_k\delta_i\delta_jE_k(E_i(\theta_j))-\delta_{i}\delta_k\delta_jE_i(E_k(\theta_j)))\\
        &=\delta_k\delta_i\delta_j[E_k,E_i](\theta_j)\\
        &=\delta_k\delta_i\delta_j (\nabla_{E_k}E_i-\nabla_{E_i}E_k)(\theta_j)\\
        &=\delta_k\delta_i\delta_j\sum_l(\omega_{ki}^l-\omega_{ik}^l)E_l(\theta_j)\\
        &=\delta_k\delta_i\sum_l\delta_l(\omega_{ik}^l-\omega_{ki}^l)h_{jj}^l.\\
    \end{split}
    \label{compati}
\end{equation}


\begin{proposition}\label{twoanglesequal}
Let $M$ be a Lagrangian submanifold of $\Sl\times\Sl$ of diagonalizable type. If two angles are equal modulo $\pi$, then $M$ is totally geodesic.
\end{proposition}
\begin{proof}
We assume that $\theta_1=\theta_2$ mod $\pi$. From (\ref{anglederi}) we get $-h_{22}^i=h_{11}^i$ and $h_{i1}^2=0$, for all $i$. Thus $h_{11}^1=h_{11}^2=h_{22}^1=h_{22}^2=h_{12}^3=0$. By Proposition \ref{minimal} the submanifold $M$ is minimal, then $-h_{11}^i+h_{22}^i+h_{33}^i=0$ for $i=1,2,3$. Hence, $h_{33}^1$ and $h_{33}^2$ also vanish. The remaining components are related by $h_{11}^3=-h_{22}^3$ and  $h_{33}^3=2h_{11}^3$. Taking $i=2,j=3,k=1$ in (\ref{compati}) we obtain $0=(\omega_{21}^3-\omega_{12}^3)h_{11}^3$. Computing both sides of the Codazzi equation in (\ref{Codazzi}) with $X=E_1,Y=E_2,Z=E_2$ yields
\[
0=h_{11}^3(\sqrt{\tfrac{2}{3}}-3\omega_{12}^3+\omega_{21}^3).
\]
Suppose $h_{11}^3\neq 0$. Then $\omega_{12}^3=\omega_{21}^3=\tfrac{1}{\sqrt{6}}$ and hence by (\ref{sffc}) we have $0=(-\tfrac{1}{\sqrt{6}}-\tfrac{1}{\sqrt{6}})\sin(\theta_1-\theta_3)$, so $\theta_1=\theta_3$ mod $\pi$. 
By (\ref{anglederi}) $h_{11}^3=-h_{33}^3=0$, which is a contradiction. 
Thus, $h_{11}^3$ must be zero and therefore the submanifold is totally geodesic. 
The proof is similar (up to signs) for a different choice of the pair of angles.
\end{proof}

\begin{lemma}\label{isometrias}
 Let $f:M\to\Sl\times\Sl$ be a Lagrangian immersion into the pseudo-nearly Kähler $\Sl\times\Sl$, and assume that $M$ is of diagonalizable type with $\Delta_1$-orthonormal frame $\{E_1,E_2,E_3\}$. 
 Let $\theta_1,\theta_2,\theta_3$ be their respective angle functions. 
 Then the Lagrangian immersions $\Psi_{1,0}\circ f$ , $\Psi_{1,4\pi/3}\circ f$, where $\Psi_{\kappa,\tau}$ are the isometries given in \emph{(\ref{isoslsl})}, are also of diagonalizable type and their respective angle functions are
\[
\theta_i^{(1)}=\pi-\theta_i, \ \ \ \ \theta_i^{(2)}=\frac{\pi}{3}-\theta_i.
\]

\end{lemma}
\begin{proof}
      The proofs of Theorem 3 and Theorem 4 of \cite{constantangles} can be replicated for this case, taking the adjugate matrix instead of the conjugate of a quaternion.
\end{proof}
\begin{lemma}\label{trescasos}
Consider a totally geodesic Lagrangian submanifold of diagonalizable type of the pseudo-nearly Kähler $\Sl\times\Sl$. 
After a possible permutation of the angles, we have one of the following:
\begin{enumerate}
        \item $(2\theta_1,2\theta_2,2\theta_3)=(\tfrac{4\pi}{3},\tfrac{4\pi}{3},\tfrac{4\pi}{3})$, \label{diag1}
        \item $(2\theta_1,2\theta_2,2\theta_3)=(0,\pi,\pi)$, \label{diag2}
        \item $(2\theta_1,2\theta_2,2\theta_3)=(\pi,\pi,0)$. \label{diag3}
\end{enumerate}

\end{lemma}
\begin{proof}
      Taking $X=E_i$, $Y=E_j$ and $Z=E_j$ in the Codazzi equation yields $\sin(2(\theta_i-\theta_j))=0$, thus all pairs of angles $2\theta_i$ and $2\theta_j$ are equal up to a multiple of $\pi$. This, together with the fact that the sum of the angles is equal to zero modulo $2\pi$,  implies that $6\theta_i=0$ modulo $\pi$ for all $i$. Therefore, replacing $E_2$ with $E_3$ and $E_3$ with $-E_2$ if necessary, we obtain the following possibilities.

      \vspace{0.2 cm}
      
        \begin{tasks}[label=(\arabic*)](2)
            \task \ $(2\theta_1,2\theta_2,2\theta_3)=(\tfrac{4\pi}{3},\tfrac{4\pi}{3},\tfrac{4\pi}{3})$,\label{c1}
            \task \ $(2\theta_1,2\theta_2,2\theta_3)=(\tfrac{2\pi}{3},\tfrac{2\pi}{3},\tfrac{2\pi}{3})$,\label{c2}
            \task \ $(2\theta_1,2\theta_2,2\theta_3)=(0,0,0)$,\label{c3}
            \task \ $(2\theta_1,2\theta_2,2\theta_3)=(0,\pi,\pi)$,\label{c4}
            \task \ $(2\theta_1,2\theta_2,2\theta_3)=(\pi,\pi,0)$,\label{c5}
            \task \ $(2\theta_1,2\theta_2,2\theta_3)=(\tfrac{\pi}{3},\tfrac{\pi}{3},\tfrac{4\pi}{3})$,\label{c6}
            \task \ $(2\theta_1,2\theta_2,2\theta_3)=(\tfrac{4\pi}{3},\tfrac{\pi}{3},\tfrac{\pi}{3})$,\label{c7}
            \task \ $(2\theta_1,2\theta_2,2\theta_3)=(\tfrac{2\pi}{3},\tfrac{5\pi}{3},\tfrac{5\pi}{3})$,\label{c8}
            \task \ $(2\theta_1,2\theta_2,2\theta_3)=(\tfrac{5\pi}{3},\tfrac{5\pi}{3},\tfrac{2\pi}{3})$.\label{c9}
            \task []
        \end{tasks}

        Suppose that $M$ is a Lagrangian submanifold of diagonalizable type with angles as in \ref{c3}.
         Then by Lemma \ref{isometrias}, applying the isometry $\Psi_{1,4\pi/3}$ in (\ref{isoslsl}) produces a congruent Lagrangian submanifold with angles as in \ref{c1}. 
         We can apply a similar argument to see that cases \ref{c2}, \ref{c3}, \ref{c6}, \ref{c7}, \ref{c8} and \ref{c9} are congruent to one of the cases \ref{c1}, \ref{c4} or \ref{c5}.
          Hence we may restrict our attention to the latter cases. 
         Notice that in cases \ref{c4} and \ref{c5} we cannot do a permutation with $E_1$ since, unlike $E_2$ and $E_3$, this vector is timelike. 
         Later on we will prove that these cases are not isometric.
\end{proof}

We proceed by giving three examples of totally geodesic Lagrangian submanifolds of the nearly Kähler $\Sl\times\Sl$. 

Consider the frame $\{X_1,X_2,X_3\}$  given by 
\begin{equation}
    \begin{split}
        X_1(a)&=a\ii,
        \ \ \ \ \ 
        X_2(a)=a\jj,
        \ \ \ \ \ 
        X_3(a)=a\kk,
    \end{split}\label{frameX}
\end{equation}
for which 
\[
-\langle X_1,X_1 \rangle =\langle X_2,X_2\rangle=\langle X_3,X_3\rangle=1.
\]
where $\li,\ri$ is the metric given in (\ref{prodsl2}). Then it is a $\Delta_1$-orthonormal frame.



\begin{example} \label{example1} Consider the immersion of $\Sl$ into $\Sl\times\Sl$ given by
\[
f\colon \Sl\to \Sl\times\Sl\colon u\mapsto (\id_2,u). 
\]
Let $X_1$, $X_2$ and $X_3$ be the vector fields on $\Sl$ given in Equation (\ref{frameX}). We have
\[
df(X_1(u))=(0,u\ii)_{(\id_2,u)}, \ \ \ df(X_2(u))=(0,u\jj)_{(\id_2,u)}, \ \ \ df(X_3(u))=(0,u\kk)_{(\id_2,u)}.
\]
Then we compute 
\[
Pdf(X_1(u))=(\ii,0)_{(\id_2,u)}, \ \ \ Pdf(X_2(u))=(\jj,0)_{(\id_2,u)}, \ \ \ Pdf(X_3(u))=(\kk,0)_{(\id_2,u)},
\]
and
\begin{equation*}
\begin{split}
    Jdf(X_1(u))=-\frac{1}{\sqrt{3}}(2\ii,u\ii)_{(\id_2,u)}, \\
    Jdf(X_2(u))=-\frac{1}{\sqrt{3}}(2\jj,u\jj)_{(\id_2,u)}, \\ 
    Jdf(X_3(u))=-\frac{1}{\sqrt{3}}(2\kk,u\kk)_{(\id_2,u)}. \\
\end{split}    
\end{equation*}
From this, it immediately follows that $g(df(X_i(u)),Jdf(X_j(u)))=0$ for all $i,j=1,2,3$. Therefore $f$ is a Lagrangian immersion.
Moreover, we notice that
\[
Pdf(X_i(u))=-\tfrac{1}{2}df(X_i(u))-\tfrac{\sqrt{3}}{2}Jdf(X_i(u),
\]
hence the angle functions are constant and all are equal to $\tfrac{4}{3}\pi$, as in Case (\ref{diag1}) of Lemma \ref{trescasos}. Using Proposition \ref{twoanglesequal} we conclude that $f$ is totally geodesic.
\end{example}

\begin{example}\label{example2}
Consider the immersion of $\Sl$ into $\Sl\times\Sl$ given by 
\[f:\Sl\to\Sl\times\Sl\colon u\mapsto(u,\ii u\ii).\]
We may compute
\[
df(X_1(u))=(u\ii,-\ii u)_{(u,\ii u\ii)}, \ \ \ df(X_2(u))=(u\jj,-\ii u\kk)_{(u,\ii u\ii)}, \ \ \ df(X_3(u))=(u\kk,\ii u\jj)_{(u,\ii u\ii)}.
\]
By the definition of $J$ we get
\begin{equation*}
    \begin{split}
        Jdf(X_1(u))&=-\frac{1}{\sqrt{3}}(u\ii,\ii u)_{(u,\ii u\ii)},\\
        Jdf(X_2(u))&=\sqrt{3}(u\jj,\ii u\kk)_{(u,\ii u\ii)},\\
        Jdf(X_3(u))&=\sqrt{3}(u\kk,-\ii u\jj)_{(u,\ii u\ii)}.\\
    \end{split}
\end{equation*}
Hence, we can easily compute $g(df(X_i(u),Jdf(X_j(u)))=0$ for $i,j=1,2,3$, therefore $f$ is a Lagrangian immersion.
Moreover, we have
\begin{equation*}
    \begin{split}
        Pdf(X_1(u))&=(u\ii,-\ii u)_{(u,\ii u\ii)}=df(X_1(u)), \\
        Pdf(X_2(u))&=(-u\jj,\ii u\kk)_{(u,\ii u\ii)}=-df(X_2(u)), \\
        Pdf(X_3(u))&=(-u\kk,-\ii u\jj)_{(u,\ii u\ii)}=-df(X_3(u)).
    \end{split}
\end{equation*}
 Consequently, the angle functions are constant and equal to $(0,\pi,\pi)$, as in Case (\ref{diag2}) of Lemma \ref{trescasos}. By Proposition (\ref{twoanglesequal}) $f$ is totally geodesic.
\end{example}

\begin{example}\label{example3}
 Consider the immersion of $\Sl$ into $\Sl\times\Sl$ given by
\[
f\colon \Sl\to \Sl\times\Sl\colon u\mapsto (u,\kk u\kk). 
\]
We have
\[
df(X_1(u))=(u\ii,-\kk u\jj)_{(u,\kk u\kk )}, \ \ \ df(X_2(u))=(u\jj,-\kk u\ii)_{(u,\kk u\kk )}, \ \ \ df(X_3(u))=(u\kk,\kk u)_{(u,\kk u\kk)}.
\]
By definition of  $J$ we obtain
\begin{equation*}
    \begin{split}
        Jdf(X_1(u))&=\sqrt{3}(u\ii,\kk u\jj)_{(u,\kk u\kk)},\\
        Jdf(X_2(u))&=\sqrt{3}(u\jj,\kk u\ii)_{(u,\kk u\kk)},\\
        Jdf(X_3(u))&=\frac{1}{\sqrt{3}}(-u\kk, u\kk)_{(u,\kk u\kk)}.\\
    \end{split}
\end{equation*}
Hence we can easily compute $g(df(X_i(u),Jdf(X_j(u)))=0$ for $i,j=1,2,3$, therefore $f$ is a Lagrangian immersion. Moreover we have that
\begin{equation*}
    \begin{split}
        Pdf(X_1(u))&=(-u\ii,\kk u\jj)_{(u,\kk u\kk)}=-df(X_1(u)), \\
        Pdf(X_2(u))&=(-u\jj,\kk u\ii)_{(u,\kk u\kk)}=-df(X_2(u)), \\
        Pdf(X_3(u))&=(u\kk,\kk u)_{(u,\kk u\kk)}=df(X_3(u)).
    \end{split}
\end{equation*}
Therefore, the angle functions are constant and equal to $(\pi,\pi,0)$, corresponding to Case (\ref{diag3}) of Lemma~\ref{trescasos}. By Proposition (\ref{twoanglesequal}) $f$ is totally geodesic.
\end{example}

% \noindent
\begin{remark}
    The immersion $f:\Sl\to\Sl\times\Sl$ given by $f:u\mapsto (u,\jj u\jj)$ is a Lagrangian immersion congruent to Example \ref{example3}. Indeed, take the isometry of $\Sl\times\Sl$ given by $\phi:(p,q)\mapsto(cpc,cqc)$, where $c$ is the matrix in $\Sl$ given by $e^{-\tfrac{\pi}{4} \kk}$. After an easy computation we get that
    \[
    \phi\circ f (u)=(cuc,c\jj u \jj c)=(v,\kk v\kk),
    \]
    where $v=cuc$.

    % This is true for any immersion of the type $u\mapsto (u,\ell u \ell)$, with $\ell$ a unit-spacelike matrix in $\slf$. The same holds if $\li\ell,\ell\ri=-1$, with this immersion being congruent to $\map$
\end{remark}

We have now proven that the immersions in Theorem \ref{maintheorem} are Lagrangian. To complete the proof of this theorem, it remains to be shown that any totally geodesic Lagrangian immersion is locally isometric to one of the Examples \ref{example1}-\ref{example3}, depending on the possible angle functions in Lemma \ref{trescasos}. This is what we prove in the upcoming propositions.
      \begin{proposition}\label{idu}
      Let $M$ be a totally geodesic Lagrangian submanifold of the pseudo-nearly Kähler $\Sl\times\Sl$ of diagonalizable type. Assume that  $(2\theta_1,2\theta_2,2\theta_3)=(\tfrac{4\pi}{3},\tfrac{4\pi}{3},\tfrac{4\pi}{3})$. Then $M$ is locally congruent to the submanifold
      $\Sl\to\Sl\times\Sl\colon u\mapsto (\id_2,u)$. 
      \end{proposition}
      \begin{proof}
            By Case \ref{case:10.1} of Proposition \ref{propAB}, we have that $A=-\tfrac{1}{2}\id$ and $B=-\frac{\sqrt{3}}{2}$ are multiples of the identity we have that $PX=-\tfrac{1}{2}X-\tfrac{\sqrt{3}}{2}JX$ for any vector field $X$ tangent to $M$. It follows immediately that $QX=X$ where $Q$ is the almost product structure given in (\ref{prodQ}). Using the Gauss equation we can compute the curvature tensor of $M$ and also the sectional curvature, which is equal to $-\tfrac{3}{2}$. Then $M$ is locally isometric to $\Sl$ equipped with the metric $\tfrac{2}{3}g_0$,
            where $g_0$ is the metric defined in (\ref{prodsl2}). Now write $f=(p,q)$. By the definition of $Q$, we have that
            \begin{equation*}
                (dp(v),0)=\tfrac{1}{2}(df(v)-Qdf(v))=0, \ \ \ (0,dq(v))=\tfrac{1}{2}(df(v)+Qdf(v))=df(v), \ \ \ v\in T\Sl.
            \end{equation*}
            hence $p$ should be a constant matrix in $\Sl$. The previous equation also implies that $dq$ is a linear isomorphism, then $q$ is a local diffeomorphism. Therefore, we may assume that $q(u)$ is actually equal to $u$. Applying an isometry of $\Sl\times\Sl$ we may assume that $p$ is equal to $\id_2$.
      \end{proof}
      
      
      
         
% Example \ref{example2} of a totally geodesic immersion is locally isometric to a Lorentzian Berger sphere, which is stretched in the direction of the timelike component. 
The immersion in Example \ref{example2} is the immersion of $\Sl$ with a Berger-like metric, stretched in the timelike direction. 
We can construct such a metric by taking on $\Sl$ the frame $\{X_1,X_2,X_3\}$ given in (\ref{frameX}), and the metric $\tilde{g}$ given by
\[
\tilde{g}(X,Y)=\tfrac{4}{\kappa}\left(\li X,Y \ri+(1-\tfrac{4\tau^2}{\kappa})\li X,X_1\ri\li Y,X_1\ri\right),
\]
where $\kappa>0$ and $\tau$ are constants and $\li,\ri$ is the metric on $\Sl$ given in (\ref{prodsl2}). It follows from a straightforward computation that
\[
[X_1,X_2]=2X_3,\ \ \  [X_1,X_3]=-2X_2, \ \ \ [X_2,X_3]=-2X_1.
\]
We take the vector fields 
\begin{equation}
    \tilde{E}_1=\tfrac{\kappa}{4\tau}X_1,\ \tilde{E}_2=\tfrac{\sqrt{\kappa}}{2}X_2\ \text{and}\ \tilde{E}_3=\tfrac{\sqrt{\kappa}}{2}X_3,\label{lorentzbergframe}
\end{equation}
which form a $\Delta_1$-orthonormal frame on $\Sl$ with respect to the metric $\tilde{g}$. We denote the Levi-Civita connection associated to $\tilde{g}$ by $\nabla^\sim$. It follows from the Koszul formula that $\nabla^\sim_{\tilde{E}_i}\tilde{E}_i=0$ and that
\begin{align*}
        \nabla^\sim_{\tilde{E}_1}\tilde{E}_2&=(\frac{\kappa }{2 \tau }-\tau )\tilde{E}_3, &&    \nabla^\sim_{\tilde{E}_2}\tilde{E}_3=-\tau\tilde{E}_1,\\
            \nabla^\sim_{\tilde{E}_1}\tilde{E}_3&=(\tau-\frac{\kappa }{2 \tau })\tilde{E}_2, &&        \nabla^\sim_{\tilde{E}_3}\tilde{E}_1=\tau\tilde{E}_2,\\
                    \nabla^\sim_{\tilde{E}_2}\tilde{E}_1&=-\tau\tilde{E}_3, &&        \nabla^\sim_{\tilde{E}_3}\tilde{E}_2=\tau\tilde{E}_1.
\end{align*}
The following proposition is a particular case of Theorem 1.7.18 of \cite{wolf}, where it is proved for an arbitrary manifold equipped with a affine connection.
\begin{proposition}\label{propositionconstants}
    Let $N^n$ and $\tilde{N}^n$ be pseudo-Riemannian manifolds with Levi-Civita connections $\nabla$ and $\tilde{\nabla}$, respectively. Suppose that there exist constants $c_{ij}^k$, $i,j,k\in {1,\dots,n}$ such that for all $p\in N$ and $\tilde{p}\in\tilde{N}$ there exist pseudo-orthonormal frames $\{F_1,\ldots,F_n\}$ around $p$, $\{\tilde{F}_1,\dots,\tilde{F}_n\}$  around $\tilde{p}$ with the same signatures such that $\nabla_{F_i}F_j=\sum_{i=1}^n c_{ij}^kF_k$, $\tilde{\nabla}_{\tilde{F}_i}\tilde{F}_j=\sum_{i=1}^nc_{ij}^k \tilde{F}_k$. Then there exists a local isometry that maps a neighbourhood of $p$ into a neighbourhood of $\tilde{p}$ and $\{F_i\}$ to $\{\tilde{F}_i\}$.
\end{proposition}
We will  use this proposition to show that cases (\ref{diag2}) and (\ref{diag3}) of Lemma \ref{trescasos} are locally $\Sl$ with Berger-like metrics.
\begin{proposition}\label{uuk}
            Let $M$ be a totally geodesic Lagrangian submanifold of the pseudo-nearly Kähler $\Sl\times\Sl$, of diagonalizable type. Assume that $(2\theta_1,2\theta_2,2\theta_3)=(0,\pi,\pi)$. Then $M$ is locally isometric to the submanifold $\Sl\to\Sl\times\Sl\colon u\mapsto (u,\ii u\ii)$. 
\end{proposition}
\begin{proof}
        Let $\{E_1,E_2,E_3\}$ be a $\Delta_1$-orthonormal frame such that $JG(E_1,E_2)=\sqrt{\tfrac{2}{3}}E_3$ and $A$ and $B$ take the form in \ref{case:10.1} of Lemma \ref{propAB} with $(2\theta_1,2\theta_2,2\theta_3)=(0,\pi,\pi)$. Thus
        \begin{equation}
             PE_1=E_1, \ \ \ \ PE_2=-E_2, \ \ \ PE_3=-E_3.      \label{Pzeropipi}
        \end{equation}
        From Lemma \ref{lemmacase1} we obtain that $\omega_{i1}^j=0$ for $i,j=1,2,3$. We also get
\[
\omega_{21}^3=-\frac{1}{\sqrt{6}},\ \ \ \  \omega_{32}^1=\frac{1}{\sqrt{6}}.
\]

From the Gauss Equation (\ref{Gauss}) we obtain the following equations:
\begin{equation}
    \begin{split}
       E_1(\omega_{22}^3)-E_2(\omega_{12}^3)+\left(\omega_{12}^3+\frac{1}{\sqrt{6}}\right) \omega_{33}^2=0,\\
       -E_1(\omega_{33}^2)-E_3(\omega_{12}^3)+\left(\omega_{12}^3+\frac{1}{\sqrt{6}}\right) \omega_{22}^3=0,\\
       E_2(\omega_{33}^2)+E_3(\omega_{22}^3)-\sqrt{\frac{2}{3}} \omega_{12}^3-(\omega_{22}^3)^2-(\omega_{33}^2)^2+\frac{5}{3}=0.\\
    \end{split}\label{equationstg0pipi}
\end{equation}
Define the 1-form $\omega$ by 
\[
\omega(E_1)=-\omega_{12}^3+\frac{5}{\sqrt{6}}, \ \ \ \omega(E_2)=-\omega_{22}^3, \ \ \ \ \omega(E_3)=\omega_{33}^2.
\]
Using (\ref{equationstg0pipi}) we can prove that $\omega$ is closed. Hence there exists a local function $\varphi$ such that $d\varphi=\omega$.
Now define the new frame 
\[
F_1=-E_1,\ \ \ \  F_2=-\cos\varphi E_2-\sin\varphi E_3,\ \ \ \ \ F_3=\sin\varphi E_2-\cos\varphi E_3.
\]
 This new frame is still $\Delta_1$-orthonormal which satisfies (\ref{Pzeropipi}) and $JG(F_1,F_2)=\sqrt{\tfrac{2}{3}}F_3$.  We have
\begin{align*}
            \nabla_{F_1}F_1&=0&&         \nabla_{F_1}F_2=-\frac{5}{\sqrt{6}}F_3,&&         \nabla_{F_1}F_3=\frac{5}{\sqrt{6}}F_2,\\
            \nabla_{F_2}F_1&=\frac{1}{\sqrt{6}}F_3, &&       \nabla_{F_2}F_2=0,  &&        \nabla_{F_2}F_3=\frac{1}{\sqrt{6}}F_1,\\
            \nabla_{F_3}F_1&=-\frac{1}{\sqrt{6}}F_2,  && \nabla_{F_3}F_2=-\frac{1}{\sqrt{6}}F_1, &&  \nabla_{F_3}F_3=0.
\end{align*}
By Proposition \ref{propositionconstants} we have that $M$ is locally isometric to $\Sl$ with a Berger-like metric stretched in the timelike direction  with $\tau=-\frac{1}{\sqrt{6}}$ and $\kappa=2$.
Now using (\ref{relprodkal}) we may write
\begin{align}
      \nabla^E _{F_1}F_1&=0, &&  \nabla^E _{F_1}F_2=-\sqrt{\frac{3}{2}}F_3, &&        \nabla^E _{F_1}F_3=\sqrt{\frac{3}{2}}F_2,\nonumber \\
      \nabla^E _{F_2}F_1&=\sqrt{\frac{3}{2}}F_3, &&    \nabla^E _{F_2}F_2=0, &&         \nabla^E _{F_2}F_3=\frac{1}{\sqrt{6}}F_1, \label{connF}\\
      \nabla^E _{F_3}F_1&=-\sqrt{\frac{3}{2}}F_2, && \nabla^E _{F_3}F_2=-\frac{1}{\sqrt{6}}F_1, &&         \nabla^E _{F_3}F_3=0.\nonumber
\end{align}
where $\nabla^E$ is the Levi-Civita connection associated to the product metric. We can identify the frame $\{F_i\}_i$ on $\Sl$ with the frame given in (\ref{lorentzbergframe}), i.e.,
\begin{equation}
     F_1=-\sqrt{\frac{3}{2}}X_1,\ \ \ \  F_2=\frac{1}{\sqrt{2}}X_2, \ \ \ \ F_3=\frac{1}{\sqrt{2}}X_3.\label{identificationberger}
\end{equation}

Now writing the immersion $f=(p,q)$ and $df(F_i)=D_{F_i}f=(p\alpha_i,q\beta_i)$, where $\alpha_i,\beta_i$ are matrices in $\mathfrak{sl}(2,\R)$, we obtain
\begin{equation}
\beta_1=\alpha_1, \ \ \ \ \ \beta_2=-\alpha_2, \ \ \ \ \ \beta_3=-\alpha_3.\label{eqalfabeta}
\end{equation}
because of Equation (\ref{Pzeropipi}).

It follows from (\ref{prodmetric}) that $\alpha_i$ are mutually orthogonal and their lengths are given by
\[
\li \alpha_1,\alpha_1\ri=-\frac{3}{2}, \ \ \ \ \li\alpha_2,\alpha_2\ri=\li\alpha_3,\alpha_3\ri=\frac{1}{2}.
\]
Thus we can write the Lorentzian cross products as
\begin{equation*}
    \alpha_1\times\alpha_2=\varepsilon \sqrt{\frac{3}{2}}\alpha_3, \ \ \ \alpha_2\times\alpha_3=-\varepsilon \frac{1}{\sqrt{6}}\alpha_1,  \ \ \ \ \ \alpha_3\times\alpha_1=\varepsilon \sqrt{\frac{3}{2}}\alpha_2,
\end{equation*}
where $\varepsilon=\pm1$.
We compute
\[
D_{F_i}D_{F_j}f=(p\alpha_i\times\alpha_j+\li\alpha_i,\alpha_j\ri p+pF_i(\alpha_j),\ q\beta_i\times\beta_j+\li\beta_i,\beta_j\ri q+q F_i(\beta_j)),
\]
therefore applying (\ref{connectionr8}) it follows that
\[
\nabla^E_{F_i}F_j=(p\alpha_i\times\alpha_j+pF_i(\alpha_j),\ q\beta_i\times\beta_j+ q F_i(\beta_j)).
\]
Comparing the above equation with (\ref{connF}) we obtain
\begin{align*}
        F_1(\alpha_1)&=0,     &&    F_2(\alpha_1)=\sqrt{\frac{3}{2}}(1+\varepsilon)\alpha_3, && F_3(\alpha_1)=-\sqrt{\frac{3}{2}}(1+\varepsilon)\alpha_3,\\
    F_1(\alpha_2)&=-\sqrt{\frac{3}{2}}(1+\varepsilon)\alpha_3, &&    F_2(\alpha_2)=0,&&  F_3(\alpha_2)=-\frac{1}{\sqrt{6}}(1+\varepsilon)\alpha_1,\\ 
    F_1(\alpha_3)&=\sqrt{\frac{3}{2}}(1+\varepsilon)\alpha_2,&&    F_2(\alpha_3)=\frac{1}{\sqrt{6}}(1+\varepsilon)\alpha_1,&&    F_3(\alpha_3)=0. 
\end{align*}
Making use of (\ref{identificationberger}) yields
\begin{align*}
      X_1(\alpha_1)&=0, &&    X_2(\alpha_1)=\sqrt{3}(1+\varepsilon)\alpha_3, &&     X_3(\alpha_1)=-\sqrt{3}(1+\varepsilon)\alpha_3,\\
          X_1(\alpha_2)&=(1+\varepsilon)\alpha_3, &&    X_2(\alpha_2)=0,    &&  X_3(\alpha_2)=-\frac{1}{\sqrt{3}}(1+\varepsilon)\alpha_1,\\
              X_1(\alpha_3)&=-(1+\varepsilon)\alpha_2, &&    X_2(\alpha_3)=\frac{1}{\sqrt{3}}(1+\varepsilon)\alpha_1, &&    X_3(\alpha_3)=0.
\end{align*}
We can write the same equations for $\beta_i$. Taking into account (\ref{eqalfabeta}), we conclude $\varepsilon$ must be equal to $-1$. Therefore $\alpha_i$ is constant for all $i$. 
We can choose an isometry on $\Sl$, namely conjugation by a matrix $c$ in $\Sl$, such that
\begin{equation*}
    \alpha_1=-\sqrt{\frac{3}{2}}c\ii c^{-1},\ \ \ \  \alpha_2=\frac{1}{\sqrt{2}}c\jj c^{-1},\ \ \ \  \alpha_3=\frac{1}{\sqrt{2}}c\kk c^{-1}.
\end{equation*}
After choosing initial conditions $f(\id_2)=(\id_2,\id_2)$, we obtain that the unique solution of the system $D_{F_i}f=(p\alpha_i,q\beta_i)$ is $p(u)=cuc^{-1}$, $q(u)=-c\ii u\ii c^{-1}$. Taking an isometry on the pseudo-nearly Kähler $\Sl\times\Sl$ finally yields $p(u)=u$ and $q(u)=\ii u\ii$.
\end{proof}



The third example of a totally geodesic immersion is locally isometric to $\Sl$ with a Berger-like metric, which is stretched in the direction of a spacelike component. We can construct such a metric by taking on $\Sl$ the frame $\{X_1,X_2,X_3\}$, and the metric $\tilde{g}$ given by
\[
\tilde{g}(X,Y)=\tfrac{4}{\kappa}\left(\li X,Y \ri+(\tfrac{4\tau^2}{\kappa}-1)\li X,X_3\ri\li Y,X_3\ri\right),
\]
where $\kappa$, $\tau$ are constants and $\li,\ri$ is the metric on $\Sl$ given in (\ref{prodsl2}). It follows from a straightforward computation that
\[
[X_1,X_2]=2X_3,\ \ \  [X_1,X_3]=-2X_2, \ \ \ [X_2,X_3]=-2X_1.
\]
We take the vector fields $\tilde{E}_1=\tfrac{\sqrt{\kappa}}{2}X_1$, $\tilde{E}_2=\tfrac{\sqrt{\kappa}}{2}X_2$ and $\tilde{E}_3=\tfrac{\kappa}{4\tau}X_3$, which form a pseudo-orthonormal frame on $\Sl$ with respect to the metric $\tilde{g}$. We denote $\nabla^\sim$ as the Levi-Civita connection associated to $\tilde{g}$. It follows from the Koszul formula that $\nabla^\sim_{\tilde{E}_i}\tilde{E}_i=0$ and that
\begin{align*}
    \nabla^\sim_{\tilde{E}_1}\tilde{E}_2&=\tau\tilde{E}_3, && \nabla^\sim_{\tilde{E}_2}\tilde{E}_3=-\tau\tilde{E}_1,\\
    \nabla^\sim_{\tilde{E}_1}\tilde{E}_3&=-\tau\tilde{E}_2, && \nabla^\sim_{\tilde{E}_3}\tilde{E}_1=(\frac{\kappa}{2\tau}-\tau)\tilde{E}_2,\\
        \nabla^\sim_{\tilde{E}_2}\tilde{E}_1&=-\tau\tilde{E}_3, && \nabla^\sim_{\tilde{E}_3}\tilde{E}_2=(\frac{\kappa}{2\tau}-\tau)\tilde{E}_1.
\end{align*}


\begin{proposition}\label{ujuj}
            Let $M$ be a totally geodesic Lagrangian submanifold of the pseudo-nearly Kähler $\Sl\times\Sl$. Assume that $(2\theta_1,2\theta_2,2\theta_3)=(\pi,\pi,0)$. Then $M$ is locally isometric to the submanifold $\Sl\to\Sl\times\Sl\colon u\mapsto (u,\kk u\kk)$. 
\end{proposition}

\begin{proof}The proof is similar to the proof of Proposition \ref{uuk}, but with some minor differences. First we take a frame $\{E_1,E_2,E_3\}$ such that $A$ and $B$ take the form of Case \ref{case:10.1} of Lemma \ref{propAB}. Then we take the closed 1-form given by
\[
\omega(E_1)=-\omega_{11}^2, \ \ \ \omega(E_2)=-\omega_{22}^1, \ \ \ \ \omega(E_3)=-\omega_{32}^1+\frac{5}{\sqrt{6}}.
\]
We define the frame $F_i$ as
\[
F_1=\cosh(\varphi) E_1+\sinh(\varphi)E_2,\ \ \ \  F_2=\sinh(\varphi)E_1+\cosh(\varphi)E_2,\ \ \ \ \ F_3=-E_3,
\]
where $\varphi$ is a local function such that $d\varphi=\omega$.

As before we write $df(F_i)=D_{F_i}f=(p\alpha_i,q\beta_i)$, where $\alpha_i,\beta_i$ are matrices in $\mathfrak{sl}(2,\R)$. They satisfy $\alpha_1=-\beta_1$, $\alpha_2=-\beta_2$ and $\alpha_3=\beta_3$. Again we may choose $c$ in $\Sl$, such that
\begin{equation*}
    \alpha_1=\frac{1}{\sqrt{2}}c\ii c^{-1},\ \ \ \  \alpha_2=\frac{1}{\sqrt{2}}c\jj c^{-1},\ \ \ \  \alpha_3=-\sqrt{\frac{3}{2}}c \kk c^{-1}.
\end{equation*}
After choosing initial conditions $f(\id_2)=(\id_2,\id_2)$, we obtain that the unique solution of the system $D_{F_i}f=(p\alpha_i,q\beta_i)$ is $p(u)=cuc^{-1}$, $q(u)=c\kk u\kk c^{-1}$. Taking an isometry on the pseudo-nearly Kähler $\Sl\times\Sl$ finally yields $p(u)=u$, $q(u)=\kk u\kk$.
\end{proof}

 
% \begin{remark}
% The immersion given by $f:\Sl\to\Sl\times\Sl$, $u\mapsto(u,\jj u\jj)$ is isometric to the immersion given in Proposition \ref{ujuj} but not congruent. An easy way to see this, is assuming that there exist a isometry $\phi$ of $\Sl\times\Sl$ that maps $(u,\kk u\kk)$ to $(u,\jj u \jj)$ and taking the vector field $X(u,\kk u\kk)=(u\ii,\kk u \kk \kk)$ with $g(X,X)=\tfrac{2}{3}$. Then $d\phi (X)=(u\ii, \jj u\kk \jj)=(u\ii, -\jj u \jj\kk)$ and $g(d\phi(X),d\phi(X))=2$.
% \end{remark}


\subsection{Lagrangian submanifolds of the non-diagonalizable types}
In this subsection we show that there do not exist totally geodesic Lagrangian submanifolds in types (2), (3) and (4) of Lemma \ref{propAB}.

\begin{lemma}\label{caseee2}
There are no totally geodesic immersions into the nearly Kähler $\Sl\times\Sl$ corresponding to Case \emph{\ref{case:10.2}} of Lemma \emph{\ref{propAB}}.
\end{lemma}
\begin{proof}
Suppose that $M$ is a totally geodesic Lagrangian submanifold of $\Sl\times\Sl$ associated to Case \ref{case:10.2} of Lemma \ref{propAB}. The left hand side of the Codazzi equation in (\ref{Codazzi}) for a totally geodesic submanifold is always zero. Computing the right hand side for $X=E_1$, $Y=E_2$ and $Z=E_2$ yields  
\[
-\frac{4}{3} ( \sin 2 \theta_1+ \cos 2 \theta_1 \cot 2 \theta_1)JE_1
\]
which cannot be zero, therefore a contradiction.
\end{proof}

\begin{lemma}\label{caseee3}
There are no totally geodesic immersions into the nearly Kähler $\Sl\times\Sl$ corresponding to Case \emph{\ref{case:10.3}} of Lemma \emph{\ref{propAB}}.
\end{lemma}
\begin{proof}
Suppose that $M$ is a totally geodesic Lagrangian submanifold of $\Sl\times\Sl$ associated to Case \ref{case:10.3} of Lemma \ref{propAB}. 
As in the previous lemma, the left hand side  of the Codazzi equation is zero. 
The component in the direction of $JE_1$ of the right hand side of the Codazzi equation for Case \ref{case:10.3} with $X=E_1,Y=E_2,Z=E_2$ is $\pm\tfrac{8}{9 \sqrt{3}}$, which is  a contradiction.  
\end{proof}


\begin{lemma}\label{caseee4}
There are no totally geodesic immersions into the nearly Kähler $\Sl\times\Sl$ corresponding to Case \emph{\ref{case:10.4}} of Lemma \emph{\ref{propAB}}.
\end{lemma}
\begin{proof}
Suppose that $M$ is a totally geodesic Lagrangian submanifold of $\Sl\times\Sl$ associated to Case \ref{case:10.4} of Lemma \ref{propAB}. The left hand side of the Codazzi equation is always zero for totally geodesic submanifolds and the right hand side of the Codazzi equation for $X=E_1, Y=E_2,Z=E_2$ is
\[
\frac{2}{3} \sinh (2 \lambda ) \cos (2 \theta_1+\theta_2)JE_2
\]
which is not zero since $\lambda$ must be different from zero and $2\theta_1+\theta_2$ is equal to zero modulo $\pi$.
\end{proof}
\subsection{Proof of the main result}
We adapt a lemma from \cite{reckziegel} to Lagrangian submanifolds:
 \begin{lemma}\label{michael}
     Let $\mathcal{F}_1,\mathcal{F}_2\colon M\to \Sl\times\Sl$ 
     be two Lagrangian immersions into the pseudo-nearly Kähler $\Sl\times\Sl$ of diagonalizable type. Let $E_i$ be a $\Delta_1$-orthonormal frame on $M$. Let $F_i$ and $\tilde{F}_i$ be frames along $\mathcal{F}_1$ and $\mathcal{F}_2$, respectively, such that $d\mathcal{F}_1(E_i)=F_i$ and $d\mathcal{F}_2(E_i)=\tilde{F}_i$. If $\mathcal{F}_1(p_o)=\mathcal{F}_2(p_o)$, $F_i(p_o)=\tilde{F}_i(p_o)$ and $\omega_{ij}^k=\tilde{\omega}_{ij}^k$, $h_{ij}^k=\tilde{h}_{ij}^k$, then $\mathcal{F}_1$ and $\mathcal{F}_2$ are locally congruent.
     \end{lemma}

   \begin{proof}[Proof of Theorem \ref{maintheorem}]
     In Examples \ref{example1}, \ref{example2} and \ref{example3} we showed that the three immersions of Theorem \ref{maintheorem} are totally geodesic and Lagrangian. 
     
     Let $M$ be a totally geodesic Lagrangian submanifold of the pseudo-nearly Kähler $\Sl\times\Sl$.
     By Lemma \ref{propAB}, there are four cases to consider.
     In Lemmas~\ref{caseee2}, \ref{caseee3} and \ref{caseee4} we proved that there are no totally geodesic Lagrangian submanifolds in Cases \ref{case:10.2}, \ref{case:10.3} and \ref{case:10.4} of Lemma \ref{propAB}.
     
     In Lemma \ref{trescasos} we have seen that any totally geodesic Lagrangian submanifold of $\Sl\times\Sl$ of diagonalizable type is congruent to a submanifold with angle functions $(\tfrac{4\pi}{3},\tfrac{4\pi}{3},\tfrac{4\pi}{3})$, $(\pi,\pi,0)$ or $(0,\pi,\pi)$.
     In Proposition \ref{idu}, we showed that all totally geodesic Lagrangian submanifolds with angle functions $(\tfrac{4\pi}{3},\tfrac{4\pi}{3},\tfrac{4\pi}{3})$ are locally congruent.
     The totally geodesic Lagrangian submanifolds with angle functions $(0,\pi,\pi)$ or $(\pi,\pi,0)$ are classified up to isometries by Propositions \ref{uuk} and~\ref{ujuj}. 
     % We also can see that they are not isometric to each other since the first one is $\Sl$ with constant sectional curvature $-\tfrac{3}{2}$, the second is $\Sl$ with a Berger-like metric stretched in the timelike direction and the third one also $\Sl$ with a Berger-like metric sphere but stretched in a spacelike direction.
     
    
     It remains to prove that any totally geodesic Lagrangian submanifold of $\Sl\times\Sl$ of diagonalizable type with angle functions $(0 ,\pi,\pi)$ is congruent to Example $\ref{example2}$ and those with angle functions $(\pi,\pi,0)$ are congruent to Example $\ref{example3}$.
     Before proving this, we will show that if $(a_o\alpha,b_o\beta)$  is a tangent vector on $M$ which is part of the $\Delta_1$-orthonormal frame that diagonalizes $P$ in Lemma~\ref{propAB}, then $\alpha$ and $\beta$ are linearly dependent.
     
          Suppose that $\alpha$ and $\beta$ are linearly independent.
     Then on the one hand we have
     \[
     P(a_o\alpha,b_o\beta)=(a_o\beta,b_o\alpha)
     \]
     and on the other hand
     \[
   P(a_o\alpha,b_o\beta)=\cos 2\theta_i (a_o\alpha,b_o\beta)+\frac{\sin 2\theta_i }{\sqrt{3}}(a_o (\alpha-2\beta),b_o(2\alpha-\beta)).
     \]
Thus, we get that $\frac{2 \sin 2 \theta_i}{\sqrt{3}}=1$ and $\frac{2 \sin 2 \theta_i}{\sqrt{3}}=-1$, a contradiction. Therefore $\alpha$ and $\beta$ are multiples.

     We use Lemma \ref{michael} to prove that $M$ is locally congruent to one of the three examples of Theorem~\ref{maintheorem}.
     We have to show that given a totally geodesic Lagrangian immersion $f\colon M\to\Sl\times\Sl$, $p_o\in M$, $\{\tilde{E}_1,\tilde{E}_2,\tilde{E}_3\}$ an $\Delta_1$-orthonormal frame at $p_o$, there exists an isometry $\phi$ of $\Sl\times\Sl$ such that $\phi(f(p_o))=(\id_2,\id_2)$, the frame $d(\phi\circ f)(\tilde{E}_i)$ is equal to one of the frames in Examples \ref{example2} and \ref{example3} and the components of the second fundamental form and of the connection coincide for one of the immersions.
     Suppose that $f(p_o)=(a_o,b_o)$.
     Take the isometry $\phi\colon (a,b)\mapsto (c a_o^{-1}
     \, a \, c^{-1},c b_o^{-1} \, b\, c^{-1})$ with $c\in\Sl$, then $\phi(a_o,b_o)=(\id_2,\id_2)$ and $d\phi(a_o\alpha,b_o\beta)=(c\alpha\, c^{-1},c\beta \, c^{-1})$ for any $\alpha,\beta\in\slf$.
     We can assume that $\beta=\varepsilon \alpha$, where $\varepsilon$ is equal to $0$, $1$ or $-1$. 
     As conjugation by an element of $\Sl$ is an isometry of $\slf$, we can choose $c$ in a convenient way, taking $\alpha$ to an arbitrary element of $\slf$. Therefore, we can take the frame at $(a_o,b_o)$ to one of the frames in Examples \ref{example2} and \ref{example3}, depending on the value of $\varepsilon$.

The last step is to prove that the components of the second fundamental form and of the connection are equal to those from Examples \ref{example2} and \ref{example3}.
As the submanifold is totally geodesic, the functions $h_{ij}^k$ are all equal to zero. The components of the connection $\omega_{ij}^k$ are determined as we can see for the frame $F_i$ in the proofs of Propositions \ref{uuk} and \ref{ujuj}.
     
     
     % there exists an isometry $\phi$ of $\Sl\times\Sl$ such that $\phi$ maps any totally geodesic Lagrangian immersion to in
     
     
     
     % The congruence follows from Lemma \ref{michael}, as $h_{ij}^k=0$ and the components $\omega_{ij}^k$ are determined as we can see for the frame $F_i$ in the proofs of Proposition \ref{uuk} and Proposition \ref{ujuj}. It only suffices to prove that there exists one point $(p$
     

  \end{proof}
  
\section*{Acknowledgements}
The authors would like to Professor Luc Vrancken for the valuable discussions regarding this work.


\bibliographystyle{abbrv}
\bibliography{tg2.bib}


% \vspace{1 cm}



\end{document}
