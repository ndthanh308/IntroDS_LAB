\documentclass{article}


% if you need to pass options to natbib, use, e.g.:
%     \PassOptionsToPackage{numbers, compress}{natbib}
% before loading neurips_2023


% ready for submission
\usepackage[preprint]{neurips_2023}



% to compile a preprint version, e.g., for submission to arXiv, add add the
% [preprint] option:
%     \usepackage[preprint]{neurips_2023}


% to compile a camera-ready version, add the [final] option, e.g.:
%     \usepackage[final]{neurips_2023}


% to avoid loading the natbib package, add option nonatbib:
%    \usepackage[nonatbib]{neurips_2023}


\usepackage[utf8]{inputenc} % allow utf-8 input
\usepackage[T1]{fontenc}    % use 8-bit T1 fonts
\usepackage{hyperref}       % hyperlinks
\usepackage{url}            % simple URL typesetting
\usepackage{booktabs}       % professional-quality tables
\usepackage{amsfonts}       % blackboard math symbols
\usepackage{nicefrac}       % compact symbols for 1/2, etc.
\usepackage{microtype}      % microtypography
\usepackage{xcolor}         % colors
\usepackage{tikz}

\usepackage{wrapfig}
\usepackage{graphicx}

% AMS Packages
\usepackage{amsmath}
\usepackage{amsthm}
\usepackage{amssymb}

% Unicode
\usepackage[utf8]{inputenc}
\usepackage{hyperref}
\hypersetup{
	unicode,
%	colorlinks,
%	breaklinks,
%	urlcolor=cyan, 
%	linkcolor=blue, 
	pdfauthor={Author One, Author Two, Author Three},
	pdftitle={A simple article template},
	pdfsubject={A simple article template},
	pdfkeywords={article, template, simple},
	pdfproducer={LaTeX},
	pdfcreator={pdflatex}
}



% Vietnamese
%\usepackage{vntex}

% Natbib
%\usepackage[sort&compress,numbers,square]{natbib}
%\bibliographystyle{mplainnat}

% Theorem, Lemma, etc
\theoremstyle{plain}
\newtheorem{theorem}{Theorem}
\newtheorem{corollary}[theorem]{Corollary}
\newtheorem{lemma}[theorem]{Lemma}
\newtheorem{claim}{Claim}[theorem]
\newtheorem{axiom}[theorem]{Axiom}
\newtheorem{conjecture}[theorem]{Conjecture}
\newtheorem{fact}[theorem]{Fact}
\newtheorem{hypothesis}[theorem]{Hypothesis}
\newtheorem{assumption}[theorem]{Assumption}
\newtheorem{proposition}[theorem]{Proposition}
\newtheorem{criterion}[theorem]{Criterion}
\theoremstyle{definition}
\newtheorem{definition}[theorem]{Definition}
\newtheorem{example}[theorem]{Example}
\newtheorem{remark}[theorem]{Remark}
\newtheorem{problem}[theorem]{Problem}
\newtheorem{principle}[theorem]{Principle}

%NEW COMMANDS


\newcommand{\skill}{{\text skill}}
\newcommand{\cloze}{{\text {\sc cloze}}}
\newcommand{\gen}{{\text {\sc gen}}}
\usepackage{graphicx, color}
\graphicspath{{fig/}}

%\usepackage[linesnumbered,ruled,vlined,commentsnumbered]{algorithm2e} % use algorithm2e for typesetting algorithms
\usepackage{algorithm, algpseudocode} % use algorithm and algorithmicx for typesetting algorithms
\usepackage{mathrsfs} % for \mathscr command

\usepackage{lipsum}

\title{A Theory for Emergence of Complex Skills in Language Models}


% The \author macro works with any number of authors. There are two commands
% used to separate the names and addresses of multiple authors: \And and \AND.
%
% Using \And between authors leaves it to LaTeX to determine where to break the
% lines. Using \AND forces a line break at that point. So, if LaTeX puts 3 of 4
% authors names on the first line, and the last on the second line, try using
% \AND instead of \And before the third author name.
\begin{document}



\author{
Sanjeev Arora\textsuperscript{1, 2},
Anirudh Goyal\textsuperscript{2}\\
}

\footnotetext[1]{Princeton University}
\footnotetext[2]{Google DeepMind}



\maketitle

%\let\footnote\relax\footnotetext{ \textsuperscript{1} Princeton %University, \textsuperscript{2} Google DeepMind \\
%Corresponding authors:  \texttt{arora@cs.princeton.edu, anirudhgoyal9119@gmail.com} 
%}
%\maketitle


\begin{abstract}

A major driver of AI products today is the fact that new skills emerge in language models when their parameter set and training corpora are scaled up.  This phenomenon is poorly understood, and a mechanistic explanation via mathematical analysis of gradient-based training seems difficult. The current paper takes a different approach, analysing emergence using the famous  (and empirical) Scaling Laws of LLMs and a simple statistical framework. Contributions include: (a) A statistical framework that relates cross-entropy loss of LLMs to competence on the basic skills that underlie language tasks. (b) Mathematical analysis showing that the Scaling Laws imply a strong form of inductive bias that allows the pre-trained model to learn very efficiently. We informally call this {\em slingshot generalization} since naively viewed it appears to give competence levels at skills that violate usual generalization theory. (c)  A key example of slingshot generalization, that competence at executing tasks involving $k$-tuples of skills emerges essentially at the same scaling and same rate as competence on the elementary skills themselves.  
%The findings may have implications for discussions on AI alignment and safety, providing insight into predictable and unexpected model properties. The study could also inform the development of next-generation training protocols for Large Language Models (LLMs).
\end{abstract}

%\tableofcontents
	
	
% Figure environment removed

\section{Introduction}
Automatic 3D reconstruction of clothed humans using image inputs has gained increasing significance due to its potential applications in a wide array of AR/VR scenarios. High-fidelity reconstructions typically depend on sophisticated capture systems, which are developed with dense camera arrays~\cite{collet2015high,joo2015panoptic,joo2018total}, programmable light-stages~\cite{Vlasic2009, guo2019relightables}, and depth sensors~\cite{newcombe2011kinectfusion,DoubleFusion,BodyFusion,dou2016fusion4d,newcombe2015dynamicfusion}. However, stringent capture environments equipped with complex hardware pose significant challenges for consumer-level applications.


In this context, considerable research effort has been dedicated to developing methods that allow for more flexible capture configurations, such as utilizing a few RGB inputs. Among these works, learning implicit functions \cite{iccv2020PIFu, saito2020pifuhd, hong2021stereopifu} has proven effective in achieving highly detailed reconstructions by integrating the advancements of deep neural networks. These methods employ large multi-layer perceptrons (MLPs) to predict the occupancy probability or truncated signed distance function (TSDF) value of every queried 3D point based on its associated local feature, which is extracted from images. They can recover a continuous surface at arbitrary resolutions without topology restrictions.


However, in typical MLP-based implicit networks, the occupancy or TSDF value at each location is solved independently with planar image features, rendering them less capable of addressing challenging cases such as occlusions. Consequently, these methods suffer from generalization and robustness issues, particularly when tackling strong occlusions caused by large motion or multiple interacting humans. 
Some follow-up studies  \cite{zheng2021deepmulticap,zheng2021pamir,huang2020arch} utilize an extra geometric model, SMPL~\cite{Loper2015}, to improve robustness by introducing strong shape priors. 
Their success typically relies on the assumption of geometrical similarity \cite{huang2020arch} between the shape prior and target reconstruction, making them intractable for handling complex cases with loose clothes and sensitive to errors in SMPL model fitting.



%\ping{this paragraph sounds like `TSDF is better than MLP/SMPL, and we use TSDF to solve the problem'. But in Sec 3, we are telling a different story, saying `MLP needs a 3D convolutional encoder'. We need to make these two sections consistent.}\sicong{I think in this paragraph we claim that the TSDF}


%We opt for Trucated Signed Distance Funtion (TSDF) volumetric representations as they are naturally suitable for convolution operations, which have shown remarkable performance for learning hierarchical features on 2D visual perception tasks \cite{SunXLW19}. 
%Meanwhile, TSDF also describes the gradual geometry change around shape surface, which is not reflected by occupancy volume. 

We instead revisit the 3D volumetric representation and resort to 3D convolutional neural networks (CNNs) for feature learning, due to their impressive performance in feature learning and the ability to incorporate spatial context. However, volumetric methods and 3D convolution involve discretization, which might raise concerns regarding whether a discretized volume can preserve subtle geometric details as continuous representations learned in implicit functions. We investigate the relationship between volume resolution and quantization error on synthetic data by converting target mesh objects to TSDF volumes, as shown in Figure~\ref{fig:quantization_error}. We observe that the quantization errors are significantly reduced by increasing volume resolution and become nearly negligible when reaching a relatively high resolution (e.g., 512 or higher). In other words, achieving fine-detailed reconstruction is not supposed to be restricted by the use of volume representations as long as a proper volume resolution is utilized. Therefore, we present a method with high-resolution feature volumes, e.g., 256 and 512, while traditional volumetric methods \cite{varol18_bodynet,gilbert2018volumetric} are often limited to much lower resolutions, such as 32 or 128.



On the other hand, an increase in volume resolution may lead to a cubic growth of memory overhead \cite{8100085}. Reducing memory costs while guaranteeing the granularity of volumetric representations is necessary for pursuing high-quality reconstruction. Thus, we adopt a coarse-to-fine approach and cull away irrelevant voxels to build a sparse high-resolution feature volume. At the coarse level, the network computes an initial TSDF by applying a U-Net with sparse 3D CNN \cite{3DSemanticSegmentationWithSubmanifoldSparseConvNet} on the sparse feature volume, which is carved by a visual hull. Through our experiments, it turns out that more than 95\% of the volume grids are discarded by the visual hull culling, making the sparse 3D CNN efficient. At the fine level, the network focuses on a narrow band near the zero-level set of the initial TSDF and discretizes the narrow band with smaller voxels. By employing this narrow-band culling, we further shrink the sampling space, resulting in a relatively small range of grid numbers (usually 300K--500K in our experiments) even with a high volume resolution of 512. The remaining voxels in the narrow band are associated with features that fuse high-frequency information from the computed normal maps upon the low-frequency shape from the coarse level to compute the TSDF at high resolution. The final mesh is then extracted from the TSDF using the Marching-Cube algorithm ~\cite{Lorensen87marchingcubes}.
% Different from the u-net sturcture to preserve global topology context, we then apply a shallow 3dcnn to compute the final TSDF $D_{final}$ which contain more local geometry detail.




% \ping{this paragraph can be expanded. It is an important contribution and often ignored by other works. stress on the novel idea of regressing blending weights instead of colors}

In addition to geometry, high-quality mesh texture is also a crucial factor contributing to visual appearance. Directly computing a color field in 3D space, as in \cite{iccv2020PIFu}, struggles to capture high-frequency texture details, while the neural radiance field (NeRF) \cite{yu2020pixelnerf} or the DoubleField~\cite{shao2022doublefield} require expensive per-instance optimization and are often unstable for sparse input images. In contrast, we adopt an image-based rendering approach to compute a texture atlas map, which is efficient and widely supported in existing computer graphics tools. 
Specifically, we compute a blending weight at each 3D point on the mesh surface to determine its color as a weighted average of the colors at its image projections. The blending weights can be computed at a relatively coarse resolution, e.g., 512 volume resolution in our case, and leave texture details to the high-resolution images, such as 1K or 2K. Unlike previous methods that generate blurry texturing results under sparse input, our method generalizes well on both synthetic and real data with just a few input views. 
Figure~\ref{fig:teaser} shows two examples reconstructed by our method. Despite the challenging garment, pose, and occlusion, our method recovers faithful shape, normal, and texture on the right.

%with a wide variety of poses and clothing styles, and it is also adaptive to handle input image with arbitrary resolutions.
%\sicong{For this concern we claim that when the resolution of dicretized volume meets certain threshold (which is 256 in our experiment), the quantization error can be neglected.} 



In summary, the main contributions of this paper are as follows:
\begin{itemize}
\vspace{-0.1in}
  \item 
  We revisit the 3D volumetric representation and demonstrate that it can support clothed human reconstruction with equal or even better performance compared to implicit representation. 
  \item 
  We develop a memory and computation-efficient method for high-resolution volumetric reconstruction using sophisticated sparse 3D CNN, coarse-to-fine estimation, and voxel culling by visual hull and narrow bands. 
  \item 
  We introduce a novel method to compute a texture atlas map, which captures rich appearance details from high-resolution input images.
  \item 
  We achieve impressive results on standard benchmark datasets Twindom and MultiHuman, significantly reducing the point-2-surface (P2S) precision to approximately 0.2cm from just six input views, with more than $50\%$ error reduction compared to the state-of-the-art methods, including DoubleField~\cite{shao2022doublefield} and PIFuHD~\cite{saito2020pifuhd}.
\end{itemize} 
 

We first review some basic concepts from probability theory (see standard textbooks such as \cite{pollard2002user,williams1991probability} for a detailed treatment), 
%the background of Bayesian inference, and finally 
%We first review some basic concepts from probability theory, 
and then present the Bayesian probabilistic programming language and the normalised posterior distribution (NPD) problem.
%we consider in this work. 
Throughout the paper,
we denote by $\Nset$, $\Zset$ and $\Rset$ the sets of all natural numbers (including zero), integers, and real numbers, respectively.

\vspace{-1.5ex}
\subsection{Basics of Probability Theory}
%We assume familiarity with basic probability theory (see \cref{app:prelim} for details). 

A \emph{measurable space} is a pair $(U,\Sigma_U)$, where $U$ is a nonempty set and $\Sigma_U$ is a $\sigma$-algebra on $U$, i.e., a family of subsets of $U$ such that $\Sigma_U\subseteq \mathcal{P}(U)$ contains $\emptyset$ and is closed under complementation and countable union. Elements of $\Sigma_U$ are called \emph{measurable} sets. A function $f$ from a measurable space $(U_1,\Sigma_{U_1})$ to another measurable space $(U_2,\Sigma_{U_2})$ is \emph{measurable} if $f^{-1}(A)\in\Sigma_{U_1}$ for all $A\in\Sigma_{U_2}$.

A \emph{measure} $\mu$ on a measurable space $(U,\Sigma_U)$ is a mapping from $\Sigma_U$ to $[0,\infty]$ such that (i) $\mu(\emptyset)=0$ and (ii) $\mu$ 
%satisfies the
is countably additive:
%condition: 
for every pairwise-disjoint set sequence $\{A_n\}_{n\in\Nset}$ in $\Sigma_U$, it holds that $\mu(\bigcup_{n\in\Nset}A_n)=\sum_{n\in\Nset}\mu(A_n)$. We call the triple $(U,\Sigma_U,\mu)$ a \emph{measure space}. 
%If $\mu(U)\le 1$, we call $\mu$ a \emph{subprobability measure}. 
If $\mu(U)=1$, we call $\mu$ a \emph{probability measure}, and $(U,\Sigma_U,\mu)$ a \emph{probability space}.
The Lebesgue measure $\lambda$ is the unique measure on $(\Rset,\Sigma_{\Rset})$ satisfying $\lambda([a,b))=b-a$ for all valid intervals $[a,b)$ in $\Sigma_{\Rset}$. For each $n\in\Nset$, we have a measurable space $(\Rset^n,\Sigma_{\Rset^n})$ 
%such that there exists 
and
a unique product measure $\lambda_n$ on $\Rset^n$ satisfying $\lambda_n(\prod_{i=1}^n A_i)=\prod_{i=1}^n \lambda(A_i)$ for all $A_i\in\Sigma_{\Rset}$.


The \emph{Lebesgue} integral operator $\int$ is a partial operator that maps a measure $\mu$ on $(U,\Sigma_U)$ and a real-valued function $f$ on the same space $(U,\Sigma_U)$ to a real number or infinity, which is denoted by $\int f \mathrm{d}\mu$ or $\int f(x)\mu(\mathrm{d}x)$. 
The detailed definition of Lebesgue integral is somewhat technical, see \cite{rankin1968real,rudin1976principles} for more details. 
Given a measurable set $A\in\Sigma_U$, the integral of $f$ over $A$ is defined by $\int_A f(x)\mu(\mathrm{d} x):=\int f(x) \cdot [x\in A] \mu(\mathrm{d}x)$
%\begin{align*}
%\textstyle\int_A f(x)\mu(\mathrm{d} x):=\int f(x) \cdot [x\in A] \mu(\mathrm{d}x)
%\end{align*} 
where $[-]$ is the Iverson bracket such that $[\phi]=1$ if 
%the predicate 
$\phi$ is true, and $0$ otherwise. If $\mu$ is a probability measure, then we call the integral as the \emph{expectation} of $f$, denoted by $\expectdist{x\sim\mu;A}{f}$, or $\expv[f]$ when the scope is clear from the context.

For a measure $v$ on $(U,\Sigma_U)$, a measurable function $f:U\to \Rset_{\ge 0}$ is the \emph{density} of $v$ with respect to $\mu$ if $v(A)=\int f(x)\cdot [x\in A] \mu(\mathrm{d} x)$ for all measurable $A\in\Sigma_U$, and $\mu$ is called the \emph{reference measure} (most often $\mu$ is the Lebesgue measure). Common families of probability distributions on the reals, e.g., uniform, normal distributions, are measures on $(\Rset,\Sigma_{\Rset})$. Most often these are defined in terms of probability density functions with respect to the Lebesgue measure. That is, for each $\mu_D$ there is a measurable function $\text{pdf}_D:\Rset\to\Rset_{\ge 0}$ that determines it: $\mu_D(A):=\int_A \text{pdf}_D (\mathrm{d}\lambda) $. As we will see, density functions such as $\text{pdf}_D$ play an important role in Bayesian inference.

Given a probability space $\pspace$, a \emph{random variable} is an $\mathcal{F}$-measurable function $X: \Omega \rightarrow \Rset \cup \{+\infty,-\infty\}$. The expectation of a random variable $X$, denoted by $\expv(X)$, is the Lebesgue integral of $X$ w.r.t. $\probm$, i.e., $\int X\,\mathrm{d}\probm$. A \emph{filtration} of $\pspace$ is an infinite sequence $\{ \mathcal{F}_n \}_{n=0}^{\infty}$ such that for every $n\ge 0$, the triple $(\Omega, \mathcal{F}_n, \probm)$ is a probability space and $\mathcal{F}_n \subseteq \mathcal{F}_{n+1} \subseteq \mathcal{F}$. A \emph{stopping time} w.r.t. $\{ \mathcal{F}_n \}_{n=0}^{\infty}$ is a random variable $T: \Omega \rightarrow \Nset \cup \{0, \infty\}$ such that for every $n \geq 0$, the event \{$T \leq n$\} is in $\mathcal{F}_n$. 

A \emph{discrete-time stochastic process} is a sequence $\Gamma = \{X_n\}_{n=0}^\infty$ of random variables in $\pspace$. The process $\Gamma$ is \emph{adapted} to a filtration $\{ \mathcal{F}_n \}_{n=0}^{\infty}$, if for all $n \geq 0$, $X_n$ is a random variable in $(\Omega, \mathcal{F}_n, \probm)$. A discrete-time stochastic process $\Gamma=\{X_n\}_{n=0}^\infty$ adapted to a filtration $\{\mathcal{F}_n\}_{n=0}^\infty$ is a \emph{martingale} (resp. \emph{supermartingale}, \emph{submartingale})
if for all $n \geq 0$, $\expv(|X_n|)<\infty$ and it holds almost surely (i.e.,~with probability $1$) that
$\condexpv{X_{n+1}}{\mathcal{F}_n}=X_n$ (\mbox{resp. } $\condexpv{X_{n+1}}{\mathcal{F}_n}\le X_n$, $\condexpv{X_{n+1}}{\mathcal{F}_n}\ge X_n$).
See~\cite{williams1991probability} for details.
%Intuitively, a martingale is a discrete-time stochastic process, in which at any time $n$, the expected value $\condexpv{X_{n+1}}{\mathcal{F}_n}$ in the next step, given all previous values, is equal to the current value $X_n$. In a supermartingale, this expected value is less than or equal to the current value and a submartingale is defined conversely.
Applying martingales to qualitative and quantitative analysis of probabilistic programs is a well-studied technique~\cite{SriramCAV,ChatterjeeFG16,ChatterjeeNZ2017}.


\subsection{Bayesian Probabilistic Programming Language}

%We consider an imperative arithmetic probabilistic programming language. 
The syntax of our probabilistic programming language (PPL) is given in \cref{fig:syntax}, where the metavariables $S$, $B$ and $E$ stand for statements, boolean expressions and arithmetic expressions, respectively.   
Our PPL is imperative with the usual conditional and loop structures (i.e.,~\textbf{if} and \textbf{while}), as well as the following new structures: (a)~sample constructs of the form ``$\textbf{sample}\  D$'' that sample a value from a prescribed distribution $D$ over $\mathbb{R}$ and then assign this value to a sampling variable $r$; (b)~score statements of the form ``\textbf{score}($EW$)'' that weight the current execution with a value expressed by $EW$ (note that $\textit{pdf}(D,x)$ means the value of a probability density function w.r.t. $D$ at $x$);
%\footnote{Instead of the hard conditioning that refutes the execution when the observation mismatches the value of the sampling variable, we use the more general soft conditioning and assume the existence of a global weight variable initialized  to $1$.}
%for each program
(c)~probabilistic branching statements of the form
``$\textbf{if}\ \textbf{prob}(p)\dots$'' that lead to the then part with probability
$p\in (0,1]$ and to the else part with probability $1-p$. We also have sequential compositions (i.e., ";") and support return statements (i.e., \textbf{return}) that 
return the value of the program variable of interest. %The set of all statements is denoted by $Stmt$.
Note that $c,c_1,c_2\in\Rset$ are constants, and our language supports any distributions with continuous density functions and infinite supports, 
including but not limited to uniform and normal distributions. 



% Figure environment removed





Given a probabilistic program in our language, we distinguish two disjoint sets of variables in the program: (i) the set $\pvars$ of \emph{program variables} whose values are determined by assignments in the program (i.e., the expressions at the LHS of ``:="); (ii)~the set $\rvars$ of \emph{sampling variables} whose values are independently sampled from prescribed probability distributions each time they are accessed (i.e., each ``$\textbf{sample}\ D$" can be regarded as a sampling variable $r$). 




\begin{example}\label{ex:pedestrian-program}

%Consider the pedestrian random walk example~\cite{DBLP:conf/esop/MakOPW21}, a pedestrian is lost on a road, and she only knows that she is away from her house at most $3$ km. Thus, she starts to repeatedly walk a uniformly random distance of at most $1$ km in either direction, until reaching her house. Upon she arrives, an  odometer tells that she has walked $1.1$ km totally. However, this odometer was once broken and the measured distance is normally distributed around the true distance with standard deviation $0.1$ km. 
\cref{fig:pedestrian-program} shows a Bayesian probabilistic program written in our PPL language. In this program, the set of program variables is $\pvars=\{start,pos,dis,step\}$, and the set of sampling variables is $\rvars=\{ \textbf{sample uniform}(0,1)\}$. Each time $\textbf{sample uniform}(0,1)$ is executed, it samples a value uniformly from $[0,1]$ and then assigns the value to the variable $step$. 
%Thus, $step$ is associated with the probability distribution $\textbf{uniform}(0,1)$.
\qed


	
% Figure environment removed
\end{example}

\subsection{The Semantics of Our Programming Language}

%To relate variables with their values, we introduce the notion of valuations. 
Let $V$ be a finite set of variables with an implicit linear order over its elements. A \emph{valuation} on $V$ is a function $\pv: V \rightarrow \Rset$ that assigns a real value to each variable in $V$. We denote the set of all valuations on $V$ by $\val{V}$. For each $1\le i\le |V|$, we denote the value of the $i$-th variable (in the implicit linear order) in $\pv$ by $\pv[i]$, so that we can view each valuation as a real vector on $V$. A \emph{program} (resp. \emph{sampling}) valuation is a valuation on $\pvars$ (resp. $\rvars$), respectively. 
For the sake of convenience, we fix the notations in the following way, i.e., we always use $\pv\in\val{\pvars}$ to denote a program valuation, and $\rv\in\val{\rvars}$ to denote a sampling valuation; we also write $\pv[\mathit{ret}]$ for the value of the return variable in $\pv$. 



Below we present the semantics for our programming language. Existing semantics in the literature are either measure-\cite{DBLP:conf/lics/StatonYWHK16,LeeYRY20} or sampling-based  \cite{DBLP:conf/esop/MakOPW21,Beutner2022b}. To facilitate the development of our algorithm, we consider the \emph{transition-based} semantics~\cite{DBLP:conf/cav/ChakarovS13,DBLP:conf/popl/ChatterjeeFNH16} to our language and 
%To apply template-based algorithmic approaches to NPD problems, we consider  that 
treat each probabilistic program as a \emph{weighted probabilistic transition system} (WPTS). A WPTS extends a PTS  ~\cite{DBLP:conf/cav/ChakarovS13,DBLP:conf/popl/ChatterjeeFNH16} with weights and an initial probability distribution. 





%Below we present a variant of probabilistic transition systems \cite{DBLP:conf/cav/ChakarovS13}.
\begin{definition}
%[Weighted Probabilistic Transition Systems]
[WPTS]\label{def:wpts}
	A \emph{weighted probabilistic transition system} (WPTS) $\Pi$
	is a tuple
\begin{equation}\label{eq:wpts} 
\tag{\dag}
\Pi = (\pvars, \rvars,  L,\lin,\lout,\mu_{\mathrm{init}}, \rdvarjdis,\transset)%\win)
\end{equation}
for which:
	\begin{itemize}
		\item
		$\pvars$ and $\rvars$ are finite disjoint sets of \emph{program} and resp. \emph{sampling} variables.
%  (variables}) 
%  such that $\pvars\cap \rvars=\emptyset$.
    \item $\locs$ is a finite set of \emph{locations} 
  %or \emph{program counters} 
  with special locations $\lin,\lout\in \locs$. Informally, $\lin$ is the initial location and $\lout$ represents program termination. 
		\item
		$\mu_{\mathrm{init}}$ is the \emph{initial probability distribution} over $\mathbb{R}^{\pvars}$ with a finite support (denoted by $\supp{\mu_{\mathrm{init}}}$), 
  %from which the initial program valuation %$\valin$ is sampled, 
  while $\rdvarjdis$ is a function that assigns a probability distribution $\rdvarjdis(r)$ to each 
  %sampling variable 
  $r \in \rvars$. We call each $\pv\in\supp{\mu_{\mathrm{init}}}$ an \emph{initial program valuation}, and abuse the notation so that $\rdvarjdis$ also denotes the independent joint distribution of all $\rdvarjdis(r)$'s ($r\in \rvars$).
		\item 
		$\transset$ is a finite set of \emph{transitions} where
		each transition $\tau \in \transset$ is a tuple $\langle \loc, \phi, F_1,\dots,F_k \rangle$ such that 
(i) $\loc\in L$ is the \emph{source location} of the transition, 
%\item 
(ii) $\phi$ is the \emph{guard condition} which is a predicate over variables $\pvars$, %which serves as the \emph{guard condition}, 
and (iii) each $F_j:=\langle \loc'_j, p_j, \upd_j,\wet_j \rangle$ is called a \emph{weighted fork} for which (a) $\loc'_j\in L$ is the \emph{destination location} of the fork, (b) $p_j\in (0,1]$ is the probability of this fork, (c) $\upd_j:\Rset^{|\pvars|} \times \Rset^{|\rvars|} \rightarrow \Rset^{|\pvars|}$ is an {\em update function} that takes as inputs the current program and sampling valuations  and returns an updated program valuation in the next step, and (d) $\wet_j:\Rset^{|\pvars|} \times \Rset^{|\rvars|}\to [0,\infty)$ is a \emph{score function} that gives the likelihood weight of this fork depending on the current program and sampling valuations.	
\end{itemize}
\end{definition}


In a WPTS, we use update and score functions to model the update on the program variables and resp. the likelihood weight when running a basic block of statements in a program, respectively.  
%and use score functions to model  caused by the execution of the score statements (if exists) in this block. 
If there is no score statement in the block, then the score function is constantly $1$. 
We always assume that a WPTS $\Pi$ is \emph{deterministic} and \emph{total}, i.e., (i) there is no program valuation that simultaneously satisfies the guard conditions of two distinct transitions from the same source location, and (ii) the disjunction of the guard conditions of all the transitions from any source location is a tautology. 
The transformation from a probabilistic program into its WPTS can be done in a straightforward way (see e.g.~\cite{DBLP:journals/toplas/ChatterjeeFNH18,DBLP:conf/cav/ChakarovS13}). 

\begin{example}\label{ex:pedestrian-semantics} 
\cref{fig:pedestrian-wpts} shows the WPTS of the program in \cref{fig:pedestrian-program} which has two locations $\lin,\lout$. 
 %In the WPTS, 
The circle nodes represent locations and square nodes model the forking behavior of transitions. An edge entering a square node is labeled with the condition of its respective transition, while an edge entering a circle node stands for a fork, which is associated with its probability, update functions and score functions that marked by $w$.\footnote{Here we omit the update functions if the values of program variables are unchanged.} The value of $step$ is initialised to $0$. An the initial probability distribution $\mu_{\mathrm{init}}$ is determined by the joint distribution of $(start,pos,dis,step)$ where $start\sim uniform(0,3)$ and $pos,dis,step$ observe the Dirac measures $Dirac(\{start\})$, $Dirac(\{0\})$ and $Dirac(\{0\})$, respectively, e.g., the probability of the event ``$dis\in\{0\}$'' equals $1$. As $step$ simply receives values from a sampling variable, we neglect it in the WPTS.\qed
\end{example}

%\paragraph{Score-recursive WPTS.} 

We say that a WPTS is \emph{non-score-recursive} if for all transitions $\tau=\langle \loc, \phi, F_1,  F_2,\dots,F_k \rangle$ in the WPTS with each fork $F_j=\langle \loc'_j, p_j, \upd_j,\wet_j \rangle$ ($1\le j\le k$), we have that each score function $\wet_j$ is constantly $1$ (i.e., the multiplicative weight does not change) for every $\loc'_j\ne \lout$. Otherwise, the WPTS is \emph{score-recursive}.
Informally, a non-score-recursive WPTS has non-trivial score functions only on the transitions to the termination of a program, while a score-recursive WPTS has {\tt score} statements in the execution of the program. 
For example, the WPTS of the program in~\cref{sec3:pedestrian} is non-score-recursive as the nontrivial (i.e., score values not equal to $1$) {\tt score} statement only appears to the termination, while the WPTS of the program in \cref{sec3:phylogenetic} is score recursive since it has {\tt score} statements inside the loop body.
In the case of a non-score-recursive WPTS, we say that the WPTS is \emph{score-bounded} by a positive real $M>0$ if for every $\tau=\langle \loc, \phi, F_1, F_2,\dots,F_k \rangle$ in the WPTS with $F_j=\langle \loc'_j, p_j, \upd_j,\wet_j \rangle$ ($1\le j\le k$), we have that 
$|\wet_j|\le M$ whenever $\loc'_j=\lout$.


Given a program valuation $\mathbf{v}$ and a predicate $\phi$ over variables $\pvars$, we say that $\mathbf{v}$ \emph{satisfies} $\phi$ (written as $\mathbf{v}\models\phi$) if $\phi$ holds when the variables in $\phi$ are substituted by their values in $\mathbf{v}$. 
A \emph{state} 
%of the WPTS $\Pi$ 
is a pair $\Xi=(\loc, \pv)$ where $\loc \in L$ (resp. $\pv \in \Rset^{|\pvars|}$) represents the current location (resp. program valuation), respectively, while a \emph{weighted state} is a triple 
%$\Xi^w:=(\loc, \pv,w)$ 
$\Theta=(\loc, \pv, w)$ 
where $(\loc, \pv)$ is a state and $w\in [0,\infty)$ represents the multiplicative likelihood weight accumulated so far. 


 
%\paragraph{Semantics.} 
Below we specify the semantics of a WPTS. Consider a WPTS $\Pi$ in the form of \eqref{eq:wpts}. The semantics of $\Pi$ is formalized by the infinite sequence $\Gamma=\{\widehat{\Theta}_n=(\widehat{\loc}_n,\widehat{\pv}_n,\widehat{w}_n)\}_{n\ge 0}$ 
%of \emph{random weighted states} 
where each $(\widehat{\loc}_n,\widehat{\pv}_n,\widehat{w}_n)$ is the random weighted state at the $n$th execution step of the WPTS such that $\widehat{\loc}_n$ (resp. $\widehat{\pv}_n$, $\widehat{w}_n$) is the random variable for the location (resp. the random vector 
%of random variables 
for the program valuation, the random variable for the multiplicative likelihood weight) at the $n$th step, respectively. %The initial random state $\widehat{\Theta}_0$ is constant and equals $(\lin,\valin,\win)$. 
%its corresponding stochastic process $\Gamma:=\{\hat{\Xi}_n\}_{n\ge 0}$ on states.
The sequence $\Gamma$ starts with the initial random weighted state 
$\widehat{\Theta}_0=(\widehat{\loc}_0,\widehat{\pv}_0,\widehat{w}_0)$ such that $\widehat{\loc}_0$ is constantly $\lin$, $\widehat{\pv}_0\in \supp{\mu_\mathrm{init}}$ is sampled from the initial distribution $\mu_\mathrm{init}$ and the initial weight $\widehat{w}_0$ is constantly set to $1$\footnote{This follows the traditional setting in e.g.~\cite{Beutner2022b}.}. 
Then, given the current random weighted state $\widehat{\Theta}_n=(\widehat{\loc}_n,\widehat{\pv}_n,\widehat{w}_n)$ at the $n$th step, the next random weighted state $\widehat{\Theta}_{n+1}=(\widehat{\loc}_{n+1},\widehat{\pv}_{n+1},\widehat{w}_{n+1})$ is determined by:
(a) If $\widehat{\loc}_n=\lout$, then $(\widehat{\loc}_{n+1}, \widehat{\pv}_{n+1},\widehat{w}_{n+1})$ takes the same weighted state as $(\widehat{\loc}_n,\widehat{\pv}_n,\widehat{w}_n)$ (i.e., the next weighted state stays at the termination location $\lout$);
(b) Otherwise, $\widehat{\Theta}_{n+1}$ is determined by the following procedure:
\begin{itemize}
\item First, since the WPTS $\Pi$ is deterministic and total, we take the unique transition $\tau=\langle \hat{\loc}_n,\phi,F_1,\dots, F_k \rangle$ such that $\hat{\pv}_n\models\phi$. 
\item Second, we choose a fork $F_j=\langle \loc_j, p_j,\upd_j,\wet_j\rangle$ with probability $p_j$.
\item 
Third, we obtain a sampling valuation $\rv\in \supp{\rdvarjdis}$ 
%over the sampling variables $\rvars$ 
by sampling each $r \in \rvars$ independently w.r.t the probability distribution $\rdvarjdis(r).$
\item Finally, the value of the next random weighted state $(\widehat{\loc}_{n+1}, \widehat{\pv}_{n+1},\widehat{w}_{n+1})$ is determined as that of 
$(\loc'_j, \upd_j(\hat{\pv}_n,\rv),\widehat{w}_n\cdot \wet_j(\widehat{\pv}_n,\rv))$. 
\end{itemize}


Given the semantics, a \emph{program run} of the WPTS $\Pi$ is a concrete instance of $\Gamma$, i.e., an infinite sequence $\omega=\{\Theta_n\}_{n\ge 0}$ of weighted states where each $\Theta_n=(\loc_n,\pv_n,w_n)$ is the concrete weighted state at the $n$th step in this program run with location $\loc_n$, program valuation $\pv_n$ and multiplicative likelihood weight $w_n$. A state $(\loc,\pv)$ is called \emph{reachable} if there exists a program run $\omega=\{\Theta_n\}_{n\ge 0}$ such that $\Theta_n=(\loc,\pv,w_n)$ for some $n$. 


 
\begin{example}\label{ex:pedestrian-run}
Consider the WPTS in \cref{ex:pedestrian-semantics}. Consider an initial program valuation $(1,1,0)$ which means that the initial values of $start,pos,dis$ are $1,1,0$, respectively. Then starting from the initial weighted state $(\lin,(1,1,0),1)$, a program run w.r.t the WPTS semantics above could be 
\[
(\lin,(1,1,0),1)\to (\lin,(1,0.5,0.5),1)\to (\lin,(1,-0.1,1.1),1)\to (\lout,(1,-0.1,1.1),3.9894).\qed
\]
\end{example}

Given an initial program valuation $\valin$ of a WPTS, one could construct a probability space over the program runs by their probabilistic evolution described above and standard constructions such as general state space Markov chains~\cite{meyn2012markov}. We denote the probability measure in the probability space by $\probm_{\valin}(-)$ and the expectation operator by $\expectdist{\valin}{-}$.  



\subsection{Normalised Posterior Distribution}\label{sec2:NPD}


Before presenting the central problem of Bayesian probabilistic programming, i.e., analyzing normalised posterior distribution with our WPTS models, we introduce some technical concepts.

%\paragraph{Termination.}
\begin{definition}[Termination]
The \emph{termination time} of a WPTS
%The \emph{termination time} of the WPTS 
$\Pi$ 
%is a random variable $T$ defined on programs runs given 
is the random variable $T$ given by
%a program run  $\omega=\{\Xi_n=(\loc_n,\pv_n,w_n)\}_{n\in\Nset}$,
%\begin{align*}	
$T(\omega):=\text{min}\{n\in\Nset\mid \loc_n=\lout\}$ for every program run  $\omega=\{(\loc_n,\pv_n,w_n)\}_{n\ge 0}$
%\end{align*}
where $\text{min}\,\emptyset:=\infty$. That is, $T(\omega)$ is the number of steps a program run $\omega$ takes to reach the termination location $\lout$. A WPTS $\Pi$ is \emph{almost-surely terminating} (AST) if $\probm_{\valin}(T<\infty)=1$ for all initial program valuations $\valin\in \supp{\mu_{\mathrm{init}}}$.  
%in the case that the program run never terminates. 
\end{definition}




\begin{definition}[Expected Weights]\label{def:exp-wt}
 Given a WPTS $\Pi$ in the form of \eqref{eq:wpts}, a designated initial program valuation $\valin$ and a measurable subset $\calU\in\Sigma_{\Rset^{|\pvars|}}$, the \emph{expected weight} $\measureSem{\Pi}_{\valin}(\calU)$ 
%$\measureSem{\Pi}(\valin)$ 
%of $\Pi$ w.r.t $\pv$ 
is defined as
%$\measureSem{\Pi}_\calU(\valin):=\expectdist{\valin}{\widehat{w}_T}$. 
$\measureSem{\Pi}_{\valin}(\calU):=\expectdist{\valin}{[\widehat{\pv}_T\in \calU]\cdot\widehat{w}_T}$. 
\end{definition}

By definition, we have that $\widehat{\pv}_T$ (resp. $\widehat{w}_T$) is the random vector (resp. variable) of the program valuation (resp. the multiplicative likelihood weight) at termination, respectively. Thus, $\measureSem{\Pi}_{\valin}(\calU)$ is the expectation of $\widehat{w}_T$ 
%over all program runs 
that start from the state $(\lin,\valin,1)$ and end with $\widehat{\pv}_T\in\calU$. If $\calU=\Rset^{|\pvars|}$, the restriction of $\widehat{\pv}_T\in\calU$ can be removed.

Below we define the normalised posterior distribution (NPD) problem. %under our WPTS semantics. 

 
\begin{definition}[Normalised Posterior Distribution]\label{def:npd}
Given a WPTS $\Pi$ in the form of \eqref{eq:wpts},
%We write $\measureSem{\Pi}(\valin)$ iff $\calU=\Rset^{|\pvars|}$.)
%Then given a probability distribution $\mu$ over initial program valuations, 
the \emph{normalised posterior distribution} (NPD) $\posterior_\Pi$ of $\Pi$ 
%over $U$ 
is defined by:
\begin{align*}
\posterior_{\Pi}(\calU):=\measureSem{\Pi}(\calU)/Z_\Pi\mbox{ for all measurable subsets } \calU\in \Sigma_{\Rset^{|\pvars|}},   
\end{align*}	
where 
$\measureSem{\Pi}(\calU):=\int_{\calV} \measureSem{\Pi}_{\pv}(\calU)\cdot \mu_{\mathrm{init}}(\mathrm{d} \pv)$ is the \emph{unnormalised posterior distribution} w.r.t. $\calU$, $\calV:=\supp{\mu_{\mathrm{init}}}$, %is the support of $\mu_{\mathrm{init}}$
%is the integral of all expected weights with an initial program valuation $\pv\in U$, 
and $Z_\Pi:=\measureSem{\Pi}(\Rset^{|\pvars|})$ is the \emph{normalising constant}.  
The WPTS $\Pi$ is called \emph{integrable} 
%w.r.t a probability distribution (for initial program valuations) 
if we have $0<Z_{\Pi}<\infty$. 
%\pw{Shall we mention that $\measureSem{\Pi}_{\pv}(\calU)$ is an integrable function here?}
\end{definition}

%We call a WPTS $\Pi$ \emph{integrable} 
%w.r.t a probability distribution (for initial program valuations) 
%if the normalising constant is finite, i.e., ~$0<Z_{\Pi}<\infty$. %for any $\pv\in\val{\pvars}$. 
%Given an integrable program, we are interested in deriving lower and upper bounds on the normalised posterior distribution over some measurable set $U\in \Sigma_\Rset$.
\paragraph{Interval Bounds for NPD.} In this work, we consider the automated interval-bound analysis for NPD of a WPTS. Formally, we aim to derive an interval $[l,u]\subseteq [0,\infty)$ 
for an integrable WPTS $\Pi$ and any measurable sets $\calU\in\Sigma_{\Rset^{|\pvars|}}$ as tight as possible such that $l\le \posterior_{\Pi}(\calU) \le u$. 
%$l,u$ are called \emph{interval bounds} for the NPD $\posterior_{\Pi}(\calU)$. 
%To achieve this, in the following (\cref{sec:math}) we develop approaches to obtain interval bounds for expected weights as $\measureSem{\Pi}(\calU)$ and $Z_\Pi$ are integrations of expected weights over $\calV$. 
 



To achieve interval bounds for NPD, below we introduce the construction of a new WPTS $\Pi_\calU$ based on the original WPTS $\Pi$ and a measurable set $\calU\in \Sigma_{\Rset^{|\pvars|}}$.  

\paragraph{Construction of $\Pi_\calU$.} Consider a probabilistic program $P$ and its WPTS $\Pi$, given a measurable set $\calU\in\Sigma_{\Rset^{|\pvars|}}$, we construct a new program $P_\calU$ by adding a conditional branch of the form ``\textbf{if} $\pv_T\notin\calU$ \textbf{then} \textbf{score}($0$) \textbf{fi}'' immediately after the termination of $P$ and obtain the WPTS $\Pi_\calU$ of $P_\calU$. Therefore, $\Pi$ and $\Pi_\calU$ have the same initial probability distribution $\mu_{\mathrm{init}}$ and the same finite support $\calV=\supp{\mu_{\mathrm{init}}}$. The following proposition shows that interval-bound analysis for NPD can be reduced to interval-bound analysis for expected weights in the form $\llbracket \Pi\rrbracket_{\pv}(\Rset^{|\pvars|})$. 

\begin{proposition}\label{prop:unnorm-norm}
   Given a WPTS $\Pi$ in the form of \eqref{eq:wpts}, a measurable set $\calU\in\Sigma_{\Rset^{|\pvars|}}$ and the WPTS $\Pi_\calU$ constructed as above, we have that $\llbracket \Pi \rrbracket_{\pv}(\calU)=\llbracket \Pi_\calU\rrbracket_{\pv}(\Rset^{|\pvars|})$ for any $\pv\in\calV=\supp{\mu_{\mathrm{init}}}$. Furthermore,
   if there exist intervals $[l_1,u_1],[l_2,u_2]\subseteq [0,\infty)$ such that $\llbracket \Pi_\calU\rrbracket_{\pv}(\Rset^{|\pvars|})\in [l_1,u_1]$ and $\llbracket \Pi\rrbracket_{\pv}(\Rset^{|\pvars|})\in [l_2,u_2 ]$ for any $\pv\in\calV$, then we have two intervals $[l_\calU,u_\calU],[l_Z,u_Z]\subseteq [0,\infty)$ such that the unnormalised posterior distribution $\llbracket \Pi\rrbracket (\calU)\in [l_\calU,u_\calU]$ and the normalising constant $Z_\Pi\in [l_Z,u_Z]$. Moreover, if $\Pi$ is integrable, i.e., $[l_Z,u_Z]\subseteq (0,\infty)$, then we can obtain the NPD $\posterior_{\Pi}(\calU)\in [\frac{l_\calU}{u_Z},\frac{u_\calU}{l_Z}]$.\footnote{The interval bounds derived in this manner may be loose, but they are definitely correct.}  Note that by \cref{def:npd}, $l_\calU=\int_\calV l_1 \cdot\mu_{\mathrm{init}}(\mathrm{d} \pv)$, $u_\calU=\int_\calV u_1 \cdot\mu_{\mathrm{init}}(\mathrm{d} \pv)$, $l_Z=\int_\calV l_2 \cdot\mu_{\mathrm{init}}(\mathrm{d} \pv)$ and $u_Z=\int_\calV u_1 \cdot\mu_{\mathrm{init}}(\mathrm{d} \pv)$.

\end{proposition}

The proof of \cref{prop:unnorm-norm} is relegated to \cref{app:sec2-prop}. In the following, we will develop approaches to obtain interval bounds for expected weights.
%in the form $\llbracket \Pi \rrbracket_{\pv}(\Rset^{|\pvars|})$ where $\pv$ is an initial program valuation.










 
\section{Cross-Entropy vs Excess Cross-Entropy: A Clarifying Example} %A Clarifying example}
\label{sec:excessentropy}

Thinking about emergence and Scaling Laws, researchers sometimes get confused as follows: {\em ``The loss decreases by a tiny amount when we increase $D$ from $10^{11}$ to $10^{12}$. According to~(\ref{eqn:scaling})  this  changes cross-entropy by a tiny amount. Why does it lead to big changes in macroscopic behavior?''} 
Let us understand the flaw in this reasoning.
%begins the theory. % actually 

Language has an inherent (i.e., irreducible) cross-entropy  ---arising from existence of many possible correct choices for the next word in an average place in text---and this
is captured  by the $A$ term \footnote{Here we're assuming that as the model and dataset size tend to infinity in tandem, the model will perfectly learn the language distribution.} in (\ref{eqn:scaling}). No model, however good, can achieve lower cross-entropy than $A$. Below, we argue that rate at which the model makes (arising from its misunderstandings of the presented text) are connected to {\em excess} cross-entropy over this minimum amount, which is captured by the second and third terms of (\ref{eqn:scaling}). This excess does reduce noticeably from scaling: e.g., when $N, D$ are increased by a factor of $10$ it reduces by roughly $(10)^{0.28} \approx 2$  . %Thus scaling leads to significant reductions in excess cross-entropy!

\paragraph{Excess Cross Entropy Drives Learning:} 
 We present a key idea in our theory:  reduction in excess cross-entropy drives improvement on language tasks. 
We illustrate using a classic example  from~\cite{winograd1971procedures} that inspired the {\em Winograd Schema Challenge(WSC)}~\cite{levesque2012winograd}:

\noindent {\tt The city councilmen refused the demonstrators a permit because they feared violence.} 

Here the pronoun {\tt they} is ambiguous--- grammar rules allow it to refer to either {\tt demonstrators} or {\tt city councilmen}. Winograd pointed out that disambiguating it (i.e., anaphora resolution) requires world knowledge that is
 unavailable in the  text itself, namely that demonstrations can get violent, and city councilmen don't like violence. 
The WSC contains many such examples from varied contexts, thus testing the model's world knowledge and common sense. 

A key idea in designing testbeds for language understanding such as WSC is  the {\bf Cloze Procedure}\footnote{Although cloze procedures allow testing most language skills, they get a bit artificial for testing quality of language productions, or testing understanding of irony, because the setup requires presenting multiple choices, one of which already explains the joke to the model.}, popular also for testing language development in children~\cite{brownassessment}. To test the model's understanding of {\tt they} in this sentence, 
 we can append a {\em prompt}: {\tt Q. Who feared violence?}. This is followed by either a blank, or a choice of multiple answers:  {\tt A. city councilmen. B. demonstrators.}   For WSC examples, even though a human would be hundred percent sure of the answer, language models circa 2016 were roughly $50/50$ confused between the two options. 

 % (blank)} or,  
%\label{eqn:WSC}
%\end{eqnarray*}

In the above example, the human is $100\%$ certain of the answer, which implies their cross-entropy here is $\log 1$, namely $0$. However if the model is split $50$-$50$ between the two options this implies it has cross-entropy  $\log 2$, all of which is {\em excess cross entropy}! Given the frequency of ambiguous pronouns in usual English, one concludes that a model that has not learned pronoun disambiguation will display huge excess cross-entropy at many places in text\footnote{Note that quantifying ``excess'' cross-entropy requires  humans in the picture. It is not possible to look at the model's cross-entropy loss on the $i$th word  according to (\ref{eqn:CEloss}) and know---without asking humans--- whether it is inherent cross-entropy or excess. } and thus reductions in excess cross-entropy will tend to squeeze out such errors. 

Of course, text corpora  do not normally contain such artificial cloze questions. But one could imagine that the model's basic misunderstanding of the above type could, often, lead to prediction mistakes in neighboring text. Our theory will make this precise in Section~\ref{sec:basictheory}.



\section{Scaling Law implies Slingshot Generalization}
\label{sec:slingshot}
This section is a warmup for our theory, highlighting that the Scaling Law  implies a strong inductive bias ---learning capability that can seem shockingly high from naive application of learning theory. We use a simplistic thought experiment that assumes the Scaling Law continues to hold for text data derived from multiple sources. (This seems roughly true  in practice.) 

 Assume there are $S$ elementary skills in language, and each piece of text applies exactly one of these skills. % they are  viewed as a disjoint union of $k$ equal-sized sub-languages, where the text in the $i$th language is representative of application of the $i$'th skill. 
Then  training and test datasets consists of $S$ (roughly) equal portions, one per sub-language. 
If $D$ is the total dataset size then the size per sub-language is $D/S$. Section~\ref{sec:excessentropy} suggests that excess cross entropy roughly captures the rate of learning, so the average error on applying the {\em union} of $S$ skills on test tasks (involving any of the $S$ skills) is the per-word excess cross-entropy of the model trained on the full language, which scales as $1/D^{0.28}$ according to the Scaling Law. (While this is the average per word, for at least $1/2$ of the the sub-languages the per-word cross-entropy is at most twice of this quantity.) 
By contrast, if we were to train the model  using just data from a single skill, the scaling would be $S^{0.28}/D^{0.28}$, which is worse by a multiplicative
factor $S^{0.28}$, provided the model size is the same in all cases.   Since $S$ is presumably large, training on the full dataset gives hugely better performance on individual skills than if we trained separate models on individual skills.
We call this effect  {\em  Slingshot Generalization}\footnote{Economists might call this phenomenon {\em increasing returns to scale}.} and it requires
no mechanistic understanding of gradient descent or transformers. 

The above effect is closely related to  {\em Effective Data Transfer}, a concept used to empirically quantify efficacy of pre-training. 
Suppose we use a small labeled dataset of size $n$ for supervised  training of a classifier by fine-tuning a pretrained LLM. This usually yields final accuracy much better than training the classifier using vanilla supervised training on  the labeled dataset. The improvement can be thought of as an effective transfer of knowledge (= additional equivalent labeled data) from the model's pretraining. Our discussion here  begins to make this precise as a form of inductive bias. 

\textbf{View from Learning theory:}  Naive application of classical learning theory implies that if the model is trained on a single skill (i.e., sub-language) then its  error on downstream tasks on that skill should be at least the reciprocal of the square root of dataset size, which is $1/\sqrt{D/S}$. However, the above analysis suggests that when the model is trained on the {\em union} of $S$ sub-languages, Scaling Law makes the error on the average skill decrease as  $1/D^{0.28}$. When $S \gg D^{0.44}$, we have $1/\sqrt{D/S} \gg 1/D^{0.28}$, which suggests at first sight that the Scaling Law must violate learning theory, at least in this toy example. This inference is erroneous, however.  Learning theory carves out an exception for inductive bias of the model --- the error estimate of
$1/\sqrt{D/S}$ for training on a single sub-language assumes no inductive bias\footnote{To give a trivial example, if we were to commence training with a model that already has minimum training and test loss, then for sure its error is zero, not $1/\sqrt{D/S}.$}.
The correct conclusion is that training using the union of  $S$ datasets provides a strong inductive bias for learning on each individual dataset. 
This inductive bias presumably arises from the model's architecture and the fact that its parameters can aggregate structural information from all $S$ sub-languages, which improves learning on individual languages. (This effect is intuitively clear to humans: learning your $6$th language is easier than learning your $2$nd.)




\section{Mathematical Framework}
\label{sec:basictheory}

We give a mathematical framework for thinking about skills and how they might relate to language comprehension tasks such as pronoun disambiguation. First, it is assumed that language comprehension involves a set of skills, though the theory will not need to know a precise list.  (Scholars have  discovered and named thousands of skills. Well-trained transformers have   undoubtedly discovered many more that remain unnamed.)
Next, the theory will assume scaling laws such as (\ref{eqn:scaling}) and thus not need to reason about  training and generalization. Instead, it can reason directly about the model's behavior on the test distribution, i.e., the distribution from which the training data was drawn. We assume this test distribution is structured as a  long unordered list of text-pieces, each with an associated measure\footnote{Text-pieces should be thought of as having a size between a paragraph to a few pages, drawn from a longer corpus. To allow good prediction for the model, the text-piece could include ancillary text that preceded it the longer corpus. The model need not do predictions for the words in this ancillary text but can use it to make predictions on the text-piece.}  Traditional cross-entropy loss is averaged using this associated measure. 
%Comprehending individual text-pieces  will be assumed to require a small subset of all skills. 
%Comprehending a particular text-piece may require a set of skills. 
%Now we  make some mild  assumptions about the test pieces and the underlying skills. 

\begin{definition}[Text piece] The test corpus for the model is viewed as being divided into  {\em text-pieces}, each consisting of $C_{test}$ tokens.  There is also a measure $\mu_2()$ on  these text-pieces, with $\mu_2(t)$ denoting the measure of text-piece $t$. The usual cross-entropy loss is computed by weighting text-pieces with respect to this measure. 
\end{definition}

Now we make some assumptions. We assume that the model's ``comprehension'' of a text piece is  testable via suitable cloze questions analogous to the Winograd example in Section~\ref{sec:excessentropy}. Specifically, we assume that an (unknown) process {\sc cloze} has been used to add such  cloze questions  to the text pieces at test time. These are clearly-marked multiple-choice questions in simple English that the model has to answer. Note that the training corpus did not contain such cloze questions, so this is a simple form of distribution shift at test time. The prediction loss on cloze questions does not require  predicting the location or contents of the cloze question ---it only requires  selecting the  correct answer to the  multiple-choice cloze question. 

We allow the process {\sc cloze} to tailor the questions to the model being tested. Thus the next assumption is reasonable. 

\begin{assumption}\label{assum:proportionalloss}[Cloze Sufficiency Assumption:]
{\em  The pre-trained model's average (multiclass) prediction loss on Cloze questions --- where the average is taken over the distribution of text pieces-- closely tracks (within a small multiplicative factor like 1.1)  the excess cross-entropy of the model on classical next-word prediction.} 
\end{assumption}
\noindent{\bf Note:} As discussed in Section~\ref{sec:excessentropy}, if the cloze question is assumed to be perfectly answerable by a human then any incorrect answers  by the model  can be interpreted analogously excess cross entropy. Our assumption  amounts to saying that mistakes on cloze questions closely capture the excess entropy of the model as defined in (\ref{eqn:CEloss}).  The next theorem, shows that there {\em exists} a set of cloze questions (albeit fairly artificial) where the excess cross-entropy  of the model's answer tracks the overall excess cross-entropy on next-word prediction. 

\begin{theorem}
If a model's  excess entropy at the $i$th place in text is $\epsilon$ then there is a cloze question with binary answer such that the probability that the model answers it incorrectly is at most $\sqrt{2\epsilon}$.
\end{theorem}
\begin{proof}
    The proof involves Pinsker's Inequality (wikipedia version) which relates variation distance and KL divergence. As in Section~\ref{sec:excessentropy} let $p_i()$ be the humans' probability distribution for the $i+1$th word in the text piece and $q_i()$ be the model's distribution. 
    The probability that the  human and the model give different answers is the variation distance    between the two distributions, which is the maximum (over all subsets $A$ of words) of  $\sum_{w \in A} (p_i(w) - q_i(w))$. 
    Let  $A_{i+1}$ denote the subset for which the previous expression is maximised. The cloze question consists of replacing word $w_{i+1}$ in the text with  the question: {\em Is the next word among the words listed in option (a) or in option (b)}, where option (a) lists words in $A_{i+1}$ and (b) lists words in $\overline{{A}_{i+1}}$. 
   The theorem now follows from Pinsker's inequality.
    %Suppose $w_{i+1}$ is the actual next word in the text. Then the {\sc cloze} procedure 
    %tosses a biased coin with $\Pr[\text{heads}] =p_i(w_{i+1})$ and if it comes up heads inserts the following cloze question here ``{\em What is the next word? (A) $w_{i+1}$ (B) something else''.} Of course, that both human and model are quite likely to choose $B$ but the expected excess cross-entropy of the model in this situation is $p_i(w_{i+1})\log p_i(w_{i+1})/q_i(w_{i+1})$. 
\end{proof}

\subsection{Skills: A Statistical View}

Language is assumed to have an underlying set  $S$ of {\em skills}. Every text-piece $t$ has an associated set of skills that are required for comprehending it. The theory allows this set of skills to be quite large ---it only needs to be (a fair bit) smaller than the number of text-pieces in the distribution (an enormous number).  

\begin{definition}[skill graph]  
A {\em skill graph} is a bipartite graph $(S, T, E)$ where nodes in $S$ correspond to skills, 
nodes in $T$ correspond to text-pieces,  and  $(s, t)$ is in the edge set $ E$ if ``comprehending'' text-piece $t$  (i.e., answering its associated cloze questions) requires using skill $s$. (See Figure~\ref{fig:skillgraph})
\end{definition}

It is important to realize that  we are interested in  quantifying the model's {\em competence} on a skill. For example, while the above definition assumes there the distribution of text-pieces includes those whose comprehension requires the skill ``anaphora resolution,''    a language model (or even human individuals!) will in general be unable to apply the  skill correctly in all text pieces. Thus ``competence on anaphora resolution'' is not $0/1$ ---instead it is quantified as the fraction of text-pieces associated with this skill whose cloze questions were correctly answered by the model. Quantifying the success rate of this (in other words, the model's capabilities) is the goal of the rest of the paper.


\iffalse \begin{wrapfigure}{r}{0.5\textwidth}
    \centering
    % Figure removed
    \caption{Skill Graph.}
    \label{fig:skillgraph}
\vspace{-3mm}
\end{wrapfigure}
\fi 


%In this full generality no theory seems possible,  since the full skill set and the skill graph are unknown to us humans. So 
% text-piece.

%Second,  we make an assumption that the skill graph has random edges. 

The final element of our theory is that the skill-graph has  random edges, as made precise in Definition~\ref{def:nature}. To understand why this makes sense, we recall Winograd's example: {\em The city councilmen refused the demonstrators a permit because they feared violence}.  Winograd implicitly assumes that the trickiest skill needed here is pronoun/anaphora  resolution, but of course, applying that skill in this context requires other skills: understanding of causality (i.e., interpretation of ``because'') as well as world knowledge about ``city councilmen,''  ``permit,'' ``demonstrators,'' etc.  This example highlights the fact that if we were to look at random  text-pieces that require pronoun disambiguation, we would encounter random real-world scenarios, whose comprehension requires very different set of skills. Moreover, the scenarios (and hence the relevant skills) could have different probabilities of occurring in the corpus. 
%\footnote{While our framework is suggested as a plausible way to think about a diverse text corpus, it seems a particularly good match for productions of a chat agent, which has to generate short pieces of text in response to a user query.}. 

For simplicity we assume that each text-piece requires exactly $k$ skills for some $k$, and this set was drawn by iid sampling from an underlying measure on the set of skills.
(Thinking of  $k$  as a random variable is natural but will not be considered here.) 
The next definition  formalizes the above framework in form of a {\em skill cluster}. 


\begin{definition}[Degree-$k$ skill cluster] \label{def:nature} This is a skill graph $(S, T, E)$ where the collection of text pieces is generated by ``nature'' by applying the following process: pick a subset of $k$ skills via iid sampling from an underlying measure $\mu_1$ on skills, and then use a procedure {\sc gen} to create a text-piece $t$ whose comprehension requires these skills, as well as a measure $\mu_2(t)$ associated\footnote{Note that the measure on text-pieces has to have the correct marginals  e.g., the $\mu_2$-measure of all text-pieces containing a skill $s$ is $\mu_1(s)$. There are many measures satisfying this weak condition, since the number of text pieces is way larger than the number of skills.} with this text piece $t$.  Then nature uses process {\sc cloze} to add cloze prompts to test comprehension on $t$. The {\em prediction loss}  on the text-piece is the cross-entropy loss on predicting the answers to the cloze questions in it. The average prediction loss over all text-pieces is computed with respect to the measure $\mu_2()$. %with the generated text-piece.
We call the skill-graph thus produced  a {\em degree-$k$ skill cluster}.
%associated with is the skill graph  produces by this process. \qed
\end{definition}
%\noindent{\bf Note:}  
Now we formalize a simple model of what the full text corpus looks like. More complicated extensions of this framework (e.g., considering a hierarchy among corpora) are left for future work. 
\begin{definition}(Text corpus) The text corpus consists of many skill clusters (e.g., math, newspapers, science, coding, etc.) $(S, T_1, E_1), (S, T_2, E_2),\ldots$  which share the same underlying set of skills $S$ but have disjoint sets of text-pieces $T_1, T_2, \ldots$ that are generated as in Definition~\ref{def:nature}.
\end{definition} 
 





%, which have the advantage 


Definition~\ref{def:nature} allows us to define ``competence on a skill'' in the more familiar setting of statistical learning theory, specifically by letting us associate a statistical task with it. The task involves predicting answers to cloze questions in a sub-distribution of text pieces that contain that skill.  Our emergence theory will apply to the family of tasks of the next definition.

\begin{definition}[Competence on Skills] \label{defn:statsviewskill}
In the setting of  Definition~\ref{def:nature}, for each skill cluster and each skill $s \in S$ {\em statistical task  $\tau_{s}$ corresponding to $s$ and this cluster} is defined as follows. The  learner is given a text-piece  created by sampling $s_1, \ldots, s_{k-1}$  via iid sampling $(k-1)$ times from measure $\mu_1$, and applying  {\sc gen} and {\sc cloze} to the skill-tuple $(s, s_1, \ldots, s_{k-1})$ to convert it into a text piece $t$ with an associated measure $\mu_2(t)$ (but the measure is re-scaled so that the total measure of the inputs to this task $\tau_s$ is $1$). 
The {\em error rate} of the model at the statistical tasks is the  expected prediction loss  on text-pieces drawn from the above distribution.  Since error rate is between $0$ and $1$, the {\em competence} refers to $(1 - \text{error rate})$.

For every  $k'$-tuple of skills $(s_1, s_2,\ldots, s_{k'})$ (where $k' \leq k$)  the  statistical task 
$\tau_{s_1, s_2,\ldots, s_{k'}}$ corresponding to that $k$'-tuple is similarly defined. The inputs to the task are generated by completing the $k'$-tuple to a $k$-tuple $\vec{s}$ by iid sampling  of $k- k'$ additional skills from $\mu_1$  and then using {\sc gen} and {\sc cloze} to convert it into a text-piece.

Competence on the $k'$-tuple is defined just as above.
\end{definition}
\noindent{\bf Note:} The definition involves the $k$-tuple being picked by iid sampling from $\mu_1$ which, in principle, allows a skill to be picked twice. However, the probability of picking the same skill twice scales as $O(1/|S|)$. Since the set of  skills $S$ is assumed to be large, the distribution is almost the same as sampling distinct $k$-tuples of skills. The small difference of $O(1/|S|)$ between the two methods will not affect any of the random graph theory calculations. 

To illustrate with an example, if comprehending a text-piece involves $5$ skills, then that text-piece will appear in $5$ statistical tasks corresponding to individual skills, ${5 \choose 2}$ tasks corresponding to pairs of skills, and so on.  However, our method of measuring the loss incurred on these statistical tasks implicitly assumes that if the model incorrectly answered this cloze question (i.e., it assigned significant probability to the wrong answer), then that loss was incurred in {\em all} these statistical tasks. This accounting is conservative ---it ignores the possibility that a model could have perfect on skills $1$ to $4$ but still have incorrectly answered the cloze question because of, say, shaky   understanding of skill $5$. But this conservative accounting has the significant benefit of obviating the need for a mathematical formulation of what skills are, and what  it means to combine skills ---which is unformulated, as earlier noted. In summary, Definition~\ref{defn:statsviewskill}  can be thought of as a lower bound on the model's true ``competence''  individual skills. Note this notion of competence also does not capture out-of-distribution generalization (i.e. predict well when the distribution of text pieces changes). 



\section{Analysis of Emergence (uniform cluster)}
\label{subsec:emergence}

Having set up a framework for modeling skills and (via Assumption~\ref{assum:proportionalloss}) connecting them to the cross-entropy loss of the model, we have  arrived at a core mathematical issue around emergence: {\em As the  model's excess cross entropy goes down (due to scaling), this improves the model's performance on cloze tasks inserted in the test stream. 
How does this improve  competence on the skills as well as on tuples of skills --in other words, performance on the associated  cloze questions?}


This section analyzes a simple setting where the test-stream consists of a single degree-$k$ skill cluster, and the skills are uniformly distributed and so are the text-pieces---in other words, the  distributions $\mu_1$ and $\mu_2$ in Definition~\ref{def:nature} are uniform. Section~\ref{subsec:measure} will extend the analysis to the general setting.  
The calculations below only require the total number of skills to be much less than the support size of the distribution of text---in other words, the set of skills can be extremely large.

\paragraph{Key Hurdle:}  We point out the naive but incorrect way to reason about this.  Since each text piece is connected to a random $k$-tuple  of skills, say $\vec{s}$, one is tempted to reason about emergence via linearity of expectations, specifically, the following relation about prediction loss, where ``expectation'' is just average over text-pieces/skills with respect to their measure: 
\begin{equation}\label{eqn:incorrectreln}
    k \cdot E_t[\text{loss}(t)] = E_s[\text{failure rate of statistical task}~\tau_{s}].~~~ (\text{\bf Incorrect!})
\end{equation}
To see that this is incorrect,  let $Y$ be the subset of such text pieces where the model makes mistakes on cloze questions. This $Y$ depends upon the skill graph, and the unknown processes {\sc gen} and {\sc cloze} of Definition~\ref{def:nature}, which assign measure to text pieces in an unknown way that may introduce arbitrary correlations. Since the model ``saw'' part of the test stream (namely, the portion corresponding to training data) it has picked some  information about the skill cluster. Thus at the end of training, locations of errors in the test stream  --i.e., the set $Y$--- depend upon the skill-cluster, and since we lack understanding of $Y$ the analysis has to treat it as arbitrary. In other words, our analysis is allowed to assume an upper bound on the test loss, but the text-pieces on which this loss occurs form an arbitrary subset that depends upon the graph structure. In particular, (\ref{eqn:incorrectreln}) cannot be inferred. This is the key mathematical hurdle and our proof will surmount it using  random graph theory.



Let's say the model {\em makes a  mistake} on a text-piece if the total prediction loss on all the  cloze-questions of that text-piece is at least $1/2$ (which is the kind of error incurred if the incorrect answer is chosen with noticeable probability  on even a  single cloze question). If the average cross-entropy loss for the text-pieces is $\delta$ we conclude $Y$ consists of at most $2\delta$ fraction of text pieces.  The following result guarantees that statistical tasks corresponding to most skills do not assign significant probability to text pieces in $Y$ --in other words, the model has good performance on statistical tasks connected with these skills. The theorem follows from (and is a simple rephrasing of) Lemma~\ref{lem:mixing} in the appendix. 
\begin{theorem}[Basic]\label{corr:emerge1}
 Let $\alpha, \beta, \theta >0, \beta >1, \alpha \beta <1, \theta <1$ satisfy
\begin{equation} \label{eqn:mix2a}
    H(\theta) + k\theta \left( H(\beta \alpha)  - \beta \alpha  \log \frac{1}{\alpha} - (1- \beta \alpha)  \log (\frac{1}{1-\alpha})\right)<0
\end{equation} and  the distribution on skills and text pieces be uniform in the skill-cluster. Then irrespective of the details of {\sc gen} and {\sc cloze} processes, the following property holds for every subset $Y$ of text pieces  that contains at least $\theta$ fraction of text pieces:  at least $1-\alpha$ fraction of skills have at most $\beta \theta  k N_1/N_2$ edges to $Y$ (in other words, at  most $\beta$ times the number of edges a skill would be {\em expected} to have to text-pieces in $Y$).
 \end{theorem}
 Note that as the model is scaled up, $\theta$ will go down and the set $Y$ containing erroneous answers on cloze questions will shrink.  Our analysis kicks in only once  $\theta$ drops below $1$. In terms of the emergence phenomenon, this corresponds to first signs of improvement on downstream tasks once the model's loss drops below some threshold.
 
  Since edges between a skill node $s$ and set $Y$ correspond to errors in the statistical task $\tau_s$,  Theorem~\ref{corr:emerge1} is giving an upper bound on the prediction error in statistical tasks corresponding to $(1-\alpha)$ fraction of skills. 

 \begin{definition}[performance curve]
The contour plot (i.e., the boundary) of the region of $\alpha, \beta$ combinations satisfying Theorem~\ref{corr:emerge1} is called a {\em performance curve} and denoted $C_{(k,\theta)}$. A performance curve $C$ is {\em better} than another curve $C'$ if for every $\alpha, \beta$ on $C$ there is a corresponding point $(\alpha, \beta')$ on $C'$ for $\beta' >\beta$.
 \end{definition}


 Figure~\ref{fig:sub1} gives {\em performance curves}, i.e., the contour plot of the set of
$\alpha, \beta$ combinations satisfying Theorem~\ref{corr:emerge1} for a given $\theta, k$.     The horizontal axis plots $(1-\alpha)$ and the vertical axis plots $\beta \theta$, so  point $(0.8, 0.16)$ on a curve    means at least $0.8$ fraction of skills have at most $0.16$ fraction of their edges in the ``error set'' $Y$ (hence $0.84$ fraction of their edges are outside the error set). The emergence curves shift down noticeably (i.e., imply emergence of more skills) as we increase $k$. The next lemma shows this trend always holds; follows from the fact that 
$H(\theta)/\theta$ is a decreasing function in the interval $(0, 1)$. 

\begin{lemma}[Monotonicity] If $\theta' < \theta$ then the performance curve for $\theta', k$ lies below that for $\theta, k$. 

If $k' > k$ then  the performance curve of $\theta, k'$ lies  below that for $k, \theta$.
\end{lemma}


% Figure environment removed

\subsection{The tensorization argument}
While the above method yields performance curves,  better curves can be derived via a tensorization argument. 
Consider the following {\em $k'$-wise recombination} operation on the test stream. First randomly partition the test stream into subsets of size $k'$, and then concatenate  the $k'$ text pieces within each subset to create a larger piece of text that we refer to as a ``$k'$-piece,'' and whose measure  is the sum of the measures of the component test-pieces. All cloze questions for the old test-pieces are retained and no new cloze questions are inserted. Clearly, if the error of the model per average text-piece was $\delta$, then the  error per average $b$-piece is $k'\delta$. 
However, each $k'$-piece is now using  a random $k'k$-tuple of skills. Importantly, this set of $k'k$ skills consists of iid draws from the skill distribution. In other words, Theorem~\ref{corr:emerge1} now becomes the following.


\begin{corollary}[tensorization] \label{corr:emerge2} In the same setting as Theorem~\ref{corr:emerge1}, for integer $k' \in [2, 1/\theta]$ the conclusion of that theorem holds also for $\alpha, \beta$ pairs satisfying
\begin{equation} \label{eqn: tensor1}
 H(k'\theta) + kk'\theta \left( H(\beta \alpha)  - \beta \alpha  \log \frac{1}{\alpha} - (1- \beta \alpha)  \log (\frac{1}{1-\alpha})\right) < 0
\end{equation}   
Furthermore, if $H(k'\theta) < k'H(\theta) $ the emergence curve from this expression dominates that derived from Theorem~\ref{corr:emerge1}.
\end{corollary}

\subsubsection{Emergence for $k'$-tuples of skills}
\label{subsec:ktuples}
Now we estimate the model's emergence curve for statistical tasks corresponding to $k'$-tuples for $k'\leq k$.
The basic idea is to consider $k'$-tuples of skills as `composite-skills,' and then re-do the calculation.


\noindent{\bf 2nd estimate (better):} Consider the following {\em $k'$-wise recombination} operation on the test stream. First randomly partition the test stream into subsets of size $k'$, and then concatenate  the $k'$ text pieces within each subset to create a larger piece of text that we refer to as a ``$k'$-piece.''  All cloze questions for the old test-pieces are retained and no new cloze questions are inserted. Clearly, if the error of the model per average text-piece was $\delta$, then the  error per average $b$-piece is $k'\delta$. 
However, each $k'$-piece is now using  a random $k'k$-tuple of skills,   which we can alternatively view as $k$  random $k'$-tuples. Thus viewing $k'$-tuples of skills as `composite skills' we can use this as the skill set in  the setting of Theorem~\ref{corr:emerge1}, which gives us an easy corollary quantifying the   performance on tasks corresponding to $k'$-tuples of skills. % Thus the same proof yields the following.
%we have the following Corollary of Lemma~\ref{lem:mixing}.

\begin{lemma}[Emergence for $k'$-tuples of skills] \label{corr:emerge2}
     Consider the skill-graph $(S', T', E)$ where $S'$ consists of all $k'$-tuples of skills, $T'$ consists of $k'$-pieces, and $E$ consists of $(s', t')$ where $s'$ is a $k'$-tuple of skills and $t'$ is a $k'$-piece where this tuple of skills is used. Let $Y$ consist of $\theta$ fraction of $k'$-pieces. Then for any $\alpha, \beta >0, \beta >1, \alpha \beta <1$ satisfying (\ref{eqn:mix1}) there are at least $1-\alpha$ fraction of $k'$-tuples of skills that have at most $\alpha \beta \theta \theta N_1$ $\beta \theta$ fraction of their edges connected to $Y$.
\end{lemma}

The next corollary presents a somewhat surprising general principle that's also hinted at in caption of Figure~\ref{fig:mixing_lemma}. Assume (for simplicity) a Chinchilla-like scaling law that $10$x up-scaling leads to  factor $2$ reduction in excess entropy. If a model is considered to have reasonable performance on individual skills at current scaling, then after further up-scaling of $10x$ one would see similar reasonable performance on skill-pairs, and scaling up by yet another $10$x after that will yield similar reasonable performance on $4$-tuples of skills, etc. 
Note that these are {\em provable lower bounds} on performance gains---actual gains could  be higher. 
Figure~\ref{fig:mixing_lemma} illustrates the phenomenon.
%from scaling may  stronger boost on performance. 
%The alludes to this Corollary. 

\begin{corollary} \label{corr:emerge3} When the model $M_1$ with loss $\delta$ is scaled up (e.g., as per equation~(\ref{eqn:scaling}))  so that the new model $M_2$ has loss  $\delta/k'$,  then the performance curve inferred by our method for $k'$-tuples of skills using $M_2$ is identical to the curve inferred for individual skills on model $M_1$.
\end{corollary}  
\begin{proof}
    As noted above, a loss of $\delta$ still allows the model to make significant mistakes on $2\delta$ fraction of test pieces, which we denote by $\theta$. Thus Theorem~\ref{corr:emerge1} describes the performance curve for skills. 
    Making the loss drop to $\delta/k'$ but creating $k'$-pieces makes the fraction of errors $\theta =2\delta$ again. (Note that ``error'' now means an erroneous answer  on {\em any} cloze question in the entire $k'$-piece ---again, this is a conservative definition of error.) Applying Lemma~\ref{corr:emerge2} we get the same emergence curve as Theorem~\ref{corr:emerge1}.
\end{proof}


%See also Section~\ref{sec:digits} for a toy example.











\vspace{-2mm}
\section{Emergence analysis with general  measure on text and skills}
\label{subsec:measure}
\vspace{-2mm}

Now we turn to analysis of the general setting of Definition~\ref{def:nature} where text piece $t$ has measure $\mu_2(t)$ and  skill $s$ has measure $\mu_1(s)$.  In this setup, our lemma statements (e.g., Lemma~\ref{lem:mixing} as well as the ones in Sections~\ref{subsec:emergence} and \ref{subsec:ktuples}) hold -----the claim is the same but with cardinalities replaced by measure!

\begin{theorem}[Emergence of skills and $k$'-tuples of skills] \label{thm:genmeasure} Let $Y$ be any subset of text pieces consisting of text pieces with total measure $\theta$, and every text-piece has measure substantially less than $\theta$.  Let $\alpha, \beta >0, \beta >1, \alpha \beta <1$ satisfy
\begin{equation} \label{eqn:mix2a}
    H(\theta) + k\theta (H(\beta \alpha)  - \beta \alpha  \log \frac{1}{\alpha} - (1- \beta \alpha)  \log (\frac{1}{1-\alpha}))<0
\end{equation}
 Then the measure of skills that have at most $\beta \theta$ fraction of their edges connected to $Y$ is at least $1-\alpha$.

 For $k'$-tuples of skills the statement of Lemma~\ref{corr:emerge2}  holds with the same modification of cardinality to ``measure.''
 \end{theorem}
\begin{proof}
  The measure $\mu_1$ on skills is trivial to reason about  by just replacing each skill $s$ by a  number of copies that is proportional to $\mu_1(s)$. This converts the measure to a uniform measure ---specifically, $k$ iid draws  from this uniform measure are equivalent to $k$ iid  draws from the  $\mu_1$.
  
For the measure $\mu_2(\cdot)$ on texts, the above trick doesn't work. Recall that a text-piece is connected in the skill graph to a random $k$-tuple of skills. If we try to replace $\mu_2()$ with a uniform measure by replacing the text piece with identical copies, then these copies must still all connect to the {\em same}  subset of $k$ skills ---meaning these connections are correlated and not random. We need a more subtle argument. The key part in the proof of Lemma~\ref{lem:mixing} is where we
choose  random subset of text-pieces, $Y$  whose size is $\theta |T|$ and subset $Z$ of skills of size $\alpha |S|$, and then upper bound by (\label{eqn:mix1}) the expectation of the event that the latter has more than $\alpha \beta \theta k$ fraction of its edges going to
$Y$.  In presence of measure $\mu_2()$ let's pick $Y$ as follows: Independently pick text-pieces, choosing $t$ with probability $\theta \mu_2(t)$. (Note: $|Y|$ is  tightly concentrated around $\theta |T|$.) We still pick $Z$ randomly as before. Then we apply Jensen's Inquality on the same calculation to end up with the same upper bound as before. See Lemma~\ref{lem:mixing+measure} in the Appendix.
\end{proof}










\vspace{-2mm}
\subsection{Extending theory to multiple clusters}
\label{subsec:multipleclusters}
\vspace{-2mm}
Above we assumed a single skill cluster in the language. Real-life text  might contain multiple skill clusters. For example,  standard corpora must contain a large skill cluster involving pieces of  ``everyday'' text pieces 
and a set of basic language skills and  world knowledge needed to comprehend them. Smaller clusters may correspond to specialized topics, e.g., finance, science, mathematical reasoning, etc.  We assume each piece of text appears in only one cluster but skills may appear in different clusters. When each text-piece appears in a single cluster, the  analysis of Section~\ref{sec:slingshot}) continues to apply.  The overall loss is the weighted sum of measure of text in the individual clusters. Thus overall reduction in loss will drive emergence within individual clusters. But lacking any mechanistic insight, our theory cannot predict the rate at which loss decrease (and hence emergence) happens within clusters. This pertains to the point made earlier in the paper about lack of detailed study of scaling laws for different kinds of corpora, as well as for training on mixes of corpora.


%We later discuss what it might mean for a skill to appear in different clusters. 

%\begin{lemma}[Skill emergence with multiple clusters]
 %   TBD.  As loss goes down, it may go down at different rates in different clusters. Loss decrease within each cluster drives emergence of skills within that cluster.
%\end{lemma}

We leave a more fine-grained analysis, including possibly allowing hierarchical structure in clusters, for future work. As usual, simpler settings probably give the main insight. 
\iffalse 

\section{Toy Illustration} \label{sec:digits}



Our theory can explain the surprising phenomenon that the inductive bias of pretraining implies that combinations of skills emerge as naturally as the individual skills. Now we give a simple experiment illustrating such a phenomenon involving vision tasks on pretrained ViT models.

%We conducted an experiment to test the theory of emergence using a pre-trained Vision Transformer (ViT-CLIP) model on a custom dataset. The dataset was designed to assess the emergence of combinations of skills in vision tasks.

\begin{wrapfigure}{r}{0.5\textwidth}
    %\centering
    \vspace{1mm}
    % Figure removed
    \caption{Composite image setup. Number of underlying skills is $10$ and the model learns to apply all $4$ skills needed for the composite image.}
    \label{fig:4tuple}
    \vspace{-2mm}
\end{wrapfigure}



Our labeled dataset consisted of $10^3$ composite images created from random selecting four images from the MNIST dataset of handwritten digit images and putting each in one quadrant of the composite image. The composite image was assigned a fixed label, which is the label of one of the four digits, randomly selected.  This is the ``target'' label, while the remaining three digits were considered ``background'' labels not made available during training.



Supervised training on this dataset used a linear probe on top of input  embeddings of the composite images output by a pre-trained Vision Transformer (ViT-CLIP) model (pre-trained on a custom image dataset). The training used mini-batch SGD. Testing was done using held out images, which were composites constructed from  MNIST images that had not been used to create the training set. This also ensured the model faced novel composite images during evaluation. 
%The goal was to predict the label of the target digit when presented with individual composite images. A subset of the images was held out for testing to evaluate the model's generalization ability.

 %This required the model to generalize its learning to correctly identify the target digit in previously unseen images.

 


The final classifier, being softmax, can  be used to output top-$4$ labels just as easily as as top-$1$. Doing so achieved approximately $92$\% accuracy in classifying the four-digit tuples even though trained using only $10^3$ examples labeled with a single digit.  Full fine-tuning of the model yielded around $95$\% accuracy with $10^3$ labeled examples.

To understand why this happened, note that since the provided label is fixed by  randomly picking one of the four digits in the image, the  optimum softmax output should learn to give   equal logit values to labels of all four digits present in the image.
Figure~\ref{fig:4tuple} describes the skill-cluster implicit here.


\fi 


\iffalse 


Below, for $S \subseteq V_2$ we define $\rho(S) = \frac{|S|}{|V_2|}$ and for $T \subseteq V_2$ we define $\rho(T) = \frac{|T|}{|V_1|}$.
Also $\Gamma(s)$ denotes the neighbor set of $s$ in $V_1$. 

{\sc also need to write down asymptotics wrt $|V_1|, |V_2|$}

\subsection{Deriving Emergence} 

The important thing about the next lemma is that it holds for {\em every} $T$ and not just the average or most $T$ (which would not suffice to derive emergence). 
\begin{lemma} For any fixed $\beta$, and every $T \subseteq V_1$ with $\rho(T) =\beta$ the following is true for $1- \epsilon_1(\beta)$ fraction of $s \in V_2$:
\begin{equation}
    \rho(\Gamma(s) \cap T) \geq \beta - \epsilon_2(\beta)
\end{equation}
where $\epsilon_1(), \epsilon_2()$ are functions of $\beta$ as well as $|V_1|, |V_2|$ that go to $0$ as  the sizes of $V_1, V_2$ are increased. 
\end{lemma}
\begin{proof}
Mixing lemma for random bipartite graphs.
\end{proof}
\fi 




%{\sc sketch} Let $\mu(t) =$ excess cross-entropy on piece of text $t$. Scaling forces $\mu(V_2)$ down. Now apply the corollary to conclude that for almost all skills the excess cross-entropy goes down.


\iffalse \section{Statistical Formalization of Skills and Explanation of Emergence}
\label{sec:basictheory}

Our theory relies on scaling laws to ignore issues of training and generalization, and thus has the luxury of reasoning directly about test distribution, i.e., the distribution from which the training data was drawn.   Furthermore, this distribution is assumed to consist of long stream of text-pieces with piece $t$ having an associated measure $\mu(t)$. 
%(which do not need to be semantically related).  % to perform its  {\em test task} on it. 

\begin{definition}[Test stream and skill graph] Test data  consists of an arbitrarily long stream of {\em text-pieces}, each consisting of $C_{test}$ tokens, and an associated {\em test task}. Language has an underlying set $V_2$ of {\em skills}. Performing well on the test task associated with a text-piece $t$ requires using a subset of $V_2$.

The {\em skill graph} is a bipartite graph $(V_1, V_2, E)$ where nodes in $V_1$ correspond to text-pieces in the test stream,  nodes in $V_2$ correspond to skills, and $(v_1, v_2) \in E$ if solving the task in text-piece $v_1$ uses skill $v_2$. \end{definition}
Note that skill set $V_2$ is long and presumably hard to catalog  even for humans. We are interested here in skills as they might be defined using language models, of which we know even less. 



% LLMs are thought of as  can now be thought of as a bipartite graph.
%\begin{definition}[Skill graph]     \end{definition}
In this full generality (with skill set and the graph being unknown), no theory seems possible. So we make a (mild)  assumption about how the  test stream and skill graph are related. To motivate the assumptions, we return to Winograd's example: {\em The city councilmen refused the demonstrators a permit because they feared violence}. The Winograd challenge implicitly assumes that the trickiest skill needed here is pronoun/anaphora  resolution, but of course, applying that skill requires other skills: understanding of causality (i.e., interpretation of ``because'') as well as world knowledge about ``citycouncilmen,''  ``permit,'' ``demonstrators,'' etc.  We quickly realize, as Winograd did, that skills are intertwined and mixed up when they get used in a piece of text. This random mixing of skills is an essential feature captured in our model. 


  In the next definition note that we make no assumptions about process ${\mathcal N}$, nor about the measures $\mu_1, \mu_2$. For simplicity of notation,   our exposition treats a distribution on a finite set as a multiset.

\begin{definition}[Text generation assumption and degree-$k$ skill cluster] \label{def:nature} There is an underlying distribution $\mu_1$ on skills and a distribution $\mu_2$ on text pieces. There is an underlying process (``nature'') ${\mathcal N}$ that, given a subset $s$ of skills, can generate a text piece ${\mathcal N}(s)$ whose associated task requires the skills in the given subset, and also associates a measure $\mu_2(t)$ with the generated text-piece.

A {\em degree-$k$ skill cluster} is generated as follows.  Nature draws a subset of skills $s$ by sampling $k$ times independently from measure $\mu_1$. Then it produces a text-piece $t$  from the process ${\mathcal N}(v_2)$, and associates a measure $\mu_2(t)$ with this text-piece. Then the edges connecting $t$ to each of the $k$ skills in $v_2$ are added to the graph. (In particular,  every node in $V_1$ has degree $k$.)
\end{definition}

\begin{wrapfigure}{r}{0.5\textwidth}
    \centering
    % Figure removed
    \caption{Skill Graph.}
    \label{fig:skillgraph}
\vspace{-3mm}
\end{wrapfigure}

Now we see that ``skill'' can be brought down to familiar setting of statistical learning theory:  a skill involves a distribution on text pieces, and ability to compute a desired answer for each text-piece drawn from the distribution. 

\begin{definition}[Statistical view of Skill]
In the degree-$k$ cluster of  Definition~\ref{def:nature}, define for each skill $v_2 \in V_2$ the {\em statistical task  corresponding to $v_2$} as follows. Randomly pick a  
$k$-tuple $s$ of skills  that contains $v_2$ (i.e., by iid sampling $(k-1)$ times from $\mu_1$), and then solve the prediction task corresponding to $t ={\mathcal N}(s)$).

For each  $k'$-tuple of skills for  $k' \leq k$ the  statistical task corresponding to that $k$'-tuple is simimlarly defined  ---sample a text-piece that relies upon those $k'$ skills, and do the statistical task on it. 
%{\em Failure rate} of the model on a skill-tuple is the average cross-entropy loss on the corresponding statistical task.
\end{definition}


%A {\em skill cluster} is a set of skills and set of pieces of text where each text involves a random combination of some subset of skills in the cluster. Thus the corresponding skill graph is a random bipartite graph. 
\iffalse 
\begin{definition}[Degree-$k$ Skill Cluster] A {\em degree-$k$ skill cluster}  consists of a skill graph $(V_1, V_2, E)$ where $V_2$ correspond to  subsets of $k$ skills appearing with some multiplicity, and for each $s \in V_2$ the text-piece in $V_2$  
\end{definition}
\fi 


%We point out that the above framework has in effect recast ``skills'' as a statistical task. 

\subsection{Connecting skills, cross-entropy, and emergence}

While the framework above is fairly general and could apply in many settings, we focus from now on on emergence phenomena in LLMs trained in the standard setting with a text corpus. 
%We first do a simplified presentation of our theory of Emergence, and flesh it out with more realistic assumptions in Sections~\ref{subsec:ktuples}~\ref{subsec:multipleclusters} and~\ref{subsec:measure}.  
Emergence is usually quantified by performance on various tasks. Our theory will concern skills that are testable using Cloze prompts introduced in Section~\ref{sec:excessentropy} (and used already in theory of LLMs~\cite{saunshiexplore20}). Many language comprehension skills --such as the Winograd task---are testable via Cloze prompts and testing using cloze prompts is also common in evaluations of language development in children~\cite{brown2020language}.%, which have the advantage 

\noindent{\bf Assumptions in our Theory:} {\em (1) Tasks associated with text-pieces in the test stream involve answering cloze questions. (2) The model's error on these tasks, when measured using cross-entropy loss (in context of prediction), tracks the overall excess cross-entropy of the model.} 

The second assumption is in effect saying that the model's gaps of understanding ---quantified as excess cross-entropy---can be extracted out via suitable cloze prompts. 

%Two questions arise: {\em (Question 1)} What are skills and who decides the complete list of skills being used in a piece of text? {\em (Question 2)} Who adds suitable cloze prompts to test the model's  capabilities at these skills? 
  Thus to understand emergence we arrive at the core mathematical issue: as scaling reduces the model's loss, how does this improve  performance on skills as well as on tuples of skills? But some technical issues arise. 
  
  First: {\em if a text-piece involves $k$ skills and the model gives the wrong answer to the cloze prompt for that text-piece, then which of the $k$ skills did it fail at?} For our theory, it will suffice to only consider the case when the model succeeded in {\em correctly} answering the prompt, in which case we assume it correctly applied  {\em all} skills that were needed in this sentence. (In other words, when the model fails,  our calculation will not need to assign blame to individual skills.) 


 Second: {\em Cloze prompts do not naturally appear in text. Who is inserting them and which skills are being tested?}  Formally, we address the above formal hurdles via  the concept of an {\em Omniscient Rater}, who is assumed to know the full catalog of skills as well as where they appear in the text.  The rater is allowed to add cloze prompts in appropriate places in the test data where those skills appear. Answering the prompts adds no excess cross-entropy for the human. The model is not penalized for cross-entropy loss on words in the cloze prompts  but it incurs cross-entropy loss for its prediction in the blank slot/multiple choice occuring in the cloze prompt. The next definition makes this precise.%is made clear in the following definition.

{\sc stopped edits here}


\begin{definition}[Test stream and Model's Failure Rate] Test data  consists of an arbitrarily long stream of {\em text-pieces}, each consisting of $C_{test}$ tokens. (The theory considers different $C_{test}$ values.)  The ominiscient rater has inserted  cloze prompts  at suitable places.  The {\em failure rate} of the model is its average cross-entropy loss at the cloze prompts.

Language is assumed to have an underlying measure $\mu()$, meaning each text-piece $t$  has an associated probability $\mu(t)$.
\end{definition}
\noindent{\bf Note:} For simplicity  our exposition below assumes a uniform measure on text; Section~\ref{subsec:measure} sketches how to  extend the analysis when a general measure exists on text-pieces and skills. 

Intuitively, the failure rate of the model measures how well it has picked up language skills that occur in natural text.

\begin{assumption}[Excess cross-entropy vs failure rate] \label{assum:proportionalloss} The average cross-entropy loss of the model per cloze question scales roughly in proportion to excess per-token cross-entropy loss on the full text.     
\end{assumption} 
\noindent{\bf Note:} This assumption intuitively says that the additional text in the cloze prompt is simple and thus unambiguous for a model that is minimally competent. The model cannot predict the cloze questions themselves,  but after it reads them it is able to understand them.  It is penalized only for its answer to the cloze questions,  which are assumed to collectively condense out the overall difficulty of understanding the text-piece. 

%\end{asummption}

%\begin{definition}[Failure rate on skill]
%Competence on a $k'$-tuple of skills is measured analogously.
%\end{definition}

\subsection{Deriving Emergence}
\label{subsec:emergence}

 By Assumption~\ref{assum:proportionalloss}, the average cross-entropy loss on the cloze questions tracks the excess cross entropy of the language model. Thus as the model is scaled up, the loss on cloze-questions will go down.  Since each text piece is connected to a random $k$-tuple of skills, one is tempted to reason about emergence via linearity of expectations, specifically, the following relation: 
\begin{equation}\label{eqn:incorrectreln}
    k \cdot E_t[\text{loss}(t)] = E_s[\text{failure rate of}~s].~~~ (\text{\bf Incorrect!})
\end{equation}
This is incorrect!  Let $Y$ be the subset of such text pieces where the model makes mistakes on cloze questions. This $Y$ depends upon the skill-cluster graph, which the model ``saw''  during training (at least, the portion corresponding to training data).  Thus the graph cannot be treated as random after $Y$ has been revealed and (\ref{eqn:incorrectreln}) cannot be infered. Our proof will surmount this mathematical hurdle using random graph theory. 

Let's say the model {\em makes a  mistake} on a text-piece if the cross-entropy loss on the included cloze-questions is at least $1/2$ (which is the kind of error associated with incorrect answer being chosen with noticeable probability  on a multiple-choice question). Since average cross-entropy loss for the average text-piece is $\delta$ we conclude $Y$ consists of at most $2\delta$ fraction of text pieces. The following result guarantees that many individual skills do not get used too often in text pieces in $Y$ --in other words, the model has good performance on statistical tasks connected with these skills.

\begin{corollary}(to Lemma~\ref{lem:mixing}) \label{corr:emerge1}
   Viewing the skill-cluster as a degree-$k$ bipartite graph, let $Y$ consist of $\theta$ fraction of text pieces.  For any $\alpha, \beta >0, \beta >1, \alpha \beta <1$ satisfying (\ref{eqn:mix1}) there are at least $1-\alpha$ fraction of skills that have at most $\beta \theta$ fraction of their edges connected to $Y$.
 \end{corollary}
Figure~\ref{fig:sub1} gives performance curves, i.e. 
$\alpha, \beta$ combinations satisfying Corollary~\ref{corr:emerge1}.  The horizontal axis plots $(1-\alpha)$ and the vertical axis plots $\beta \theta$, so  point $(0.8, 0.16)$ on a curve    means at least $0.8$ fraction of skills have at most $0.16$ fraction of their edges in the ``error set'' $Y$ (hence $0.84$ fraction of their edges are outside the error set). 
% Figure environment removed

\iffalse
\begin{wrapfigure}{r}{0.5\textwidth}
  \centering
  \vspace{-10mm}
  % Figure removed
  \caption{Performance Curves: We give $\alpha, \beta$ combinations satisfying Corollary~\ref{corr:emerge1} for $k=8$ when $\theta=0.05, 0.1, 0.2$.  The horizontal axis plots $(1-\alpha)$ and the vertical axis plots $\beta \theta$. For example, the point $(0.8, 0.16)$ on the red curve means at least $0.8$ fraction of skills have at most $0.16$ fraction of their edges in the ``error set'' $Y$.  Section~\ref{subsec:ktuples} clarifies that these curves also describe the model's performance curve for $t$-tuples of skills for for $\theta =0.05$ and $t=1, 2, 4$ respectively . Thus for example the blue curve describes performance on $4$-tuples of skills when $\theta =0.05$.}
  \label{fig:mixing_lemma}
  \vspace{-4mm}
\end{wrapfigure}
\fi

\subsection{Emergence for $k'$-tuples of skills}
\label{subsec:ktuples}
Now we estimate the model's failure rate on statistical tasks corresponding to $k'$-tuples for $k'\leq k$.

\noindent{\bf Naive estimate:} This consists of observing that a random $k$-tuple is a union of $k/k'$ random $k'$ tuples. Considering $k'$-tuples as `new-skills,' we get a skill-graph  with degree $k/k'$, and Corollary~\ref{corr:emerge1} can let us derive performance curves for $k'$-tuples. However, this is a weak and sometimes trivial estimate.

\noindent{\bf 2nd estimate (better):} Consider a {\em $t$-wise recombination} operation on the test stream. First it randomly divides the test stream into subsets of size $t$, and then it concatenates  the $t$ text pieces within each subset to create a larger piece of text that we refer to as a ``$t$-piece.''  All cloze questions for the old test-pieces are retained and no new cloze questions are inserted. Clearly, if the error of the model per average text-piece was $\delta$, then the  error per average $t$-piece is $t\delta$. 
However, each $t$-piece is now using  a random $tk$-tuple of skills,   which we can alternatively view as $k$  random $t$-tuples. Thus we can use the random graph theory to derive performance on $t$-tuples of skills. 
%we have the following Corollary of Lemma~\ref{lem:mixing}.

\begin{corollary}(Emergence for $t$-tuples of skills) \label{corr:emerge2}
     In setting of Lemma~\ref{lem:mixing} consider the bipartite random graph of degree $k$ where $V_1$ consists of $t$-pieces and  $V_2$ consists of $t$-tuples of skills.  Let $Y$ consist of $\theta$ fraction of $t$-pieces. Then for any $\alpha, \beta >0, \beta >1, \alpha \beta <1$ satisfying (\ref{eqn:mix1}) there are at least $1-\alpha$ fraction of $t$-tuples of skills that have at most $\beta \theta$ fraction of their edges connected to $Y$.
\end{corollary}

The next corollary presents a somewhat surprising general principle (alluded to in caption of Figure~\ref{fig:mixing_lemma}): if the model is considered to have reasonable performance on individual skills, then scaling it up by one order of magnitude will yield same reasonable performance on skill-pairs, and scaling up by yet another order of magnitude will yield same performance on $4$-tuples of skills, etc. (Recall that the Chinchilla law predicts roughly a factor $2$ reduction in excess cross-entropy with one magnitude of up-scaling.)
%another scaling by one order of magnitude will yield reasonable performance on quadruples, and so on. We caution 
Note that these are {\em provable bounds} on performance gains---actual gains could  be higher.
%from scaling may  stronger boost on performance. 
%The alludes to this Corollary. 

\begin{corollary} \label{corr:emerge2} When the model $M_1$ with loss $\delta$ is scaled up (e.g., as per equation~(\ref{eqn:scaling})) so that its loss drops from $\delta$ to $\delta/t$ then the performance curve inferred by our method for $t$-tuples of skills using $M_2$ is identical to the curve inferred for individual skills on $M_1$.
\end{corollary}  
\begin{proof}
    As noted above, a loss of $\delta$ still allows the model to make significant mistakes on $2\delta$ fraction of test pieces, which we denote by $\theta$. Thus Corollary~\ref{corr:emerge1} describes the performance curve for skills. 
    Making the loss drop to $\delta/t$ but creating $t$-pieces makes the fraction of errors $\theta =2\delta$ again. (Note that ``error'' now means an erroneous answer  on {\em any} cloze question in the entire $t$-piece.) Applying Corollary~\ref{corr:emerge2} we get the same emergence curve. 
\end{proof}




\iffalse 
\vspace{-4mm}
\textbf{Discussion of Paucity of stimulus.} Figure~\ref{fig:mixing_lemma} illustrates that our theory predicts interesting performance curves for statistical tasks involving a constant fraction of 4-tuples, although they are slightly behind the performance on individual skills. Moreover, according to the Scaling Law and Corollary~\ref{corr:emerge2}, after scaling by two orders of magnitude, the performance on 4-tuples of skills catches up with that of individual skills. This substantial improvement in performance on most 4-tuples of skills highlights the strong inductive bias inherent in the scaling laws. Notably, this improvement occurred with a constant factor increase in the size of the dataset, while the number of $4$-tuples of skills grows as the $4$th power of the number of skills. The model learns combinations of skills because it only observed skills being applied within random subsets of other skills. See also Section~\ref{sec:digits}.
\fi 

%Figure~\ref{fig:mixing_lemma} shows that our theory implies nontrivial performance curves on  statistical tasks corresponding to a constant fraction of $4$-tuples, while lagging somewhat behind the performance on individual skills. Furthermore,  the Scaling Law and Corollary~\ref{corr:emerge2} imply that after two orders of magnitude of scaling,  the performance on $4$-tuples of skills catches up with individual skills, . This significant better performance on many/most $4$-tuples of skills highlights the strong inductive bias implicit in the scaling laws. The improvement happened with a constant factor increase in data-set size  whereas the number of skill combinations scales as the $4$th power of number of skills.  The model learns combinations of skills  because it only sees skills being applied in context of  random subsets of other skills, and the  model has no option but to learn this ability. See Section~\ref{sec:digits} as well.

 \iffalse 
The following easy lemma shows that failure on the average skill in the cluster tracks the average cross-entropy loss on text pieces.
\fi 


%{\sc need a different term than competence since this quantity goes up with loss.}



\iffalse 

\subsection{Deriving Emergence for skills}




This section assumes  the corpus consists of a single Skill Cluster; see Section~\ref{subsec:multipleclusters} for the  case with multiple clusters. Our analysis will proceed by tracking the excess cross-entropy loss on the test stream, which by Assumption~\ref{assum:proportionalloss} gets reduced as the  model size and dataset size are scaled up. Note that Lemma~\ref{lem:lossvsfailure} already implies that average failure rate of the model improves as we scale up. But improvement in average performance in principle does not rule out large holes in the model's capabilities on the skills. For example if the excess failure rate per text-piece is $0.5$,  this does not rule out high failure rate on half the skills.  We now show that improvement happens across a broad set of skills. This uses random graph theory from Section~\ref{subsec:randomgraphs}. The hurdle is that the reduction of loss happens on an arbitrary subset of the test stream, and Lemma~\ref{lem:mix} applies to every  $Y$ and every $Z$.



%The following lemma shows that almost all skills in the cluster improve in the process. This relies upon Lemma~\ref{lem:measure}.

{\sc Lemma to be rewritten} 
 \begin{lemma} When failure rate in a skill cluster is less than $\delta$ then there is a set of size $(1-\alpha)$ fraction of skills whose failure rate is at most $(1-\beta(1- 2\delta)$,  where $(\alpha, \beta)$ satisfy (\ref{eqn:mix2}) for $\theta = 1-2\delta$.
 \end{lemma}

Now we argue that emergence happens for many $k$-tuples of skills, which for exposition we refer to as a ``meta-skill.''  (Emergence of $k'$-tuples for $k' <k$ is similarly derived.) It again relies upon Lemma~\ref{lem:mix}.  The skill-graph for $k$-tuples is completely analogous, except the  ``new skills'' now correspond to $k$-tuples of the skills. Now each text piece
only depends on a {\em single} randomly-chosen ``new skill.'' In other words, this is just the subcase $k=1$ of the skill-cluster, and Lemma~\ref{lem:measure} continues to apply. 


\begin{lemma} \label{lem:ktuple}
When the cross-entropy loss on the test stream for the skill cluster is $\theta$ then for  at least 
$1-\alpha$ fraction of $k$-tuples of skills, the failure on that $k$-tuple  is at least $\beta\theta$,
where where $(\alpha, \beta)$ satisfy (\ref{eqn:mix2}) upon setting $k=1$.
\end{lemma}

\noindent{\bf Discussion:} The theory predicts equally fast emergence on $k$-tuples and individual skills. However, part of the reason is our decision to blame the error made on a cloze task equally to all skills. 
\fi 

\vspace{-2mm}
\subsection{Allowing a measure on text and another measure on skills}
\label{subsec:measure}
\vspace{-2mm}

The above setup involved a uniform distribution on skills and on text-pieces. In practice, text pieces have different probabilities and very likely the skills too. We denote by $\mu(t)$ the probability of text-piece $t$, and by $\rho(s)$ the probability of skill $s$. Each text-piece has edges to $k$ skills, where the $k$-tuple of skills is chosen via $k$ independent draws\footnote{Concretely, this corresponds to a generative process for text where nature generates a text-piece by first randomly picking a $k$-tuple of skills via iid sampling on $\rho()$ and then generates a text piece $t$ that uses that $k$-tuple of skills, and then assigns it the measure $\mu(t)$.} according to measure $\rho(\cdot)$.  In this setup, our lemma statements (e.g., Lemma~\ref{lem:mixing} as well as the ones in Sections~\ref{subsec:emergence} and \ref{subsec:ktuples}) hold but instead of cardinality of sets we need to use their measure. 

\noindent{\bf Allowing measure on skills.} The measure $\rho$ on skills is trivial to reason about  by just replacing each skill $s$ by a  number of copies that is proportional to $\rho(s)$. This converts the measure to a uniform measure ---specifically, random draw from this uniform measure is equivalent to random draw from measure $\rho$.   

\noindent{\bf Allowing measure on texts.} For the measure $\mu(t)$ on texts, the above trick doesn't work. Recall that a text-piece is connected in the skill graph to a random $k$-tuple of skills. If we try to replace $\mu()$ with a uniform measure by replacing the text piece with identical copies, then these copies must still all connect to the {\em same}  subset of $k$ skills ---meaning we no longer have a random graph. We need a more subtle argument. 

The key part is the proof of Lemma~\ref{lem:mixing} where 
choose  random subset $Y \subseteq V_1$ of size $\theta V_1$ and subset $Z \subseteq V_2$ of size $\alpha N_2$, and then upper bound by (\label{eqn:mix1}) the expectation of the event that the latter has more than $\alpha \beta \theta k$ fraction of its edges going to
$Y$.  In presence of measure $\mu()$ let's pick $Y \subseteq V_1$ as follows: Independently pick text-pieces, choosing $t$ with probability $\theta \mu(t)$. (Note: $|Y|$ is  tightly concentrated around $\theta N_1$.) We still pick $Z$ randomly as before. Then we apply Jensen's Inquality on the same calculation to end up with the same upper bound as before. See Appendix.

\iffalse 
\begin{lemma}
If ${\mathcal T}$ is the set of all text-pieces, then each measure $\mu(t)$ on text pieces can be $(1+\epsilon)$-approximated by a weighted sum $\alpha_i \mu_i(t)$ where each $\mu_i$ is the uniform measure on some subset of ${\mathcal T}$.
\end{lemma}
\fi 
\vspace{-2mm}
\subsection{Extending theory to multiple clusters}
\label{subsec:multipleclusters}
\vspace{-2mm}
Above we assumed a single skill cluster in the language. Real-life text  might contain multiple skill clusters. For example,  standard corpora must contain a large skill cluster involving pieces of  ``everyday'' text pieces 
and a set of basic language skills and  world knowledge needed to comprehend them. Smaller clusters may correspond to specialized topics, e.g., finance, science, mathematical reasoning, etc.  We assume each piece of text appears in only one cluster but skills may appear in different clusters. The disjoint clusters of text is reminiscent of the setup considered in Section~\ref{sec:slingshot}). The overall loss is the weighted sum of measure of text in the individual clusters. Thus overall reduction in loss will drive emergence within individual clusters, but lacking any mechanistic insight, our theory cannot predict the rate at which loss decrease (and hence emergence) happens within clusters. 


%We later discuss what it might mean for a skill to appear in different clusters. 

%\begin{lemma}[Skill emergence with multiple clusters]
 %   TBD.  As loss goes down, it may go down at different rates in different clusters. Loss decrease within each cluster drives emergence of skills within that cluster.
%\end{lemma}

We leave a more fine-grained analysis, including possibly hierarchical structure in clusters, for future work. We do not attempt it here because the basic insights about emergence of skills are clearer in the simple settings discussed above. 


\section{Toy Illustration} \label{sec:digits}
Our theory can explain the surprising phenomenon that the inductive bias of pretraining implies that combinations of skills emerge as naturally as the individual skills. Now we give a simple experiment illustrating such a phenomenon involving vision tasks on pretrained ViT models.

%We conducted an experiment to test the theory of emergence using a pre-trained Vision Transformer (ViT-CLIP) model on a custom dataset. The dataset was designed to assess the emergence of combinations of skills in vision tasks.





Our labeled dataset consisted of $10^3$ composite images created from random selecting four images from the MNIST dataset of handwritten digit images and putting each in one quadrant of the composite image. The composite image was assigned a fixed label, which is the label of one of the four digits, randomly selected.  This is the ``target'' label, while the remaining three digits were considered ``background'' labels not made available during training.

\begin{wrapfigure}{r}{0.5\textwidth}
    %\centering
    \vspace{-4mm}
    % Figure removed
    \caption{Composite image setup. Number of underlying skills is $10$ and the model learns to apply all $4$ skills needed for the composite image.}
    \label{fig:4tuple}
    \vspace{-2mm}
\end{wrapfigure}

Supervised training on this dataset used a linear probe on top of input  embeddings of the composite images output by a pre-trained Vision Transformer (ViT-CLIP) model (pre-trained on a custom image dataset). The training used mini-batch SGD. Testing was done using held out images, which were composites constructed from  MNIST images that had not been used to create the training set. This also ensured the model faced novel composite images during evaluation. 
%The goal was to predict the label of the target digit when presented with individual composite images. A subset of the images was held out for testing to evaluate the model's generalization ability.

 %This required the model to generalize its learning to correctly identify the target digit in previously unseen images.

The final classifier, being softmax, can  be used to output top-$4$ labels just as easily as as top-$1$. Doing so achieved approximately $92$\% accuracy in classifying the four-digit tuples even though trained using only $10^3$ examples labeled with a single digit.  Full fine-tuning of the model yielded around $95$\% accuracy with $10^3$ labeled examples.

To understand why this happened, note that since the provided label is fixed by  randomly picking one of the four digits in the image, the  optimum softmax output should learn to give   equal logit values to labels of all four digits present in the image.
Figure~\ref{fig:4tuple} describes the skill-cluster implicit here.



\iffalse 


Below, for $S \subseteq V_2$ we define $\rho(S) = \frac{|S|}{|V_2|}$ and for $T \subseteq V_2$ we define $\rho(T) = \frac{|T|}{|V_1|}$.
Also $\Gamma(s)$ denotes the neighbor set of $s$ in $V_1$. 

{\sc also need to write down asymptotics wrt $|V_1|, |V_2|$}

\subsection{Deriving Emergence} 

The important thing about the next lemma is that it holds for {\em every} $T$ and not just the average or most $T$ (which would not suffice to derive emergence). 
\begin{lemma} For any fixed $\beta$, and every $T \subseteq V_1$ with $\rho(T) =\beta$ the following is true for $1- \epsilon_1(\beta)$ fraction of $s \in V_2$:
\begin{equation}
    \rho(\Gamma(s) \cap T) \geq \beta - \epsilon_2(\beta)
\end{equation}
where $\epsilon_1(), \epsilon_2()$ are functions of $\beta$ as well as $|V_1|, |V_2|$ that go to $0$ as  the sizes of $V_1, V_2$ are increased. 
\end{lemma}
\begin{proof}
Mixing lemma for random bipartite graphs.
\end{proof}
\fi 




%{\sc sketch} Let $\mu(t) =$ excess cross-entropy on piece of text $t$. Scaling forces $\mu(V_2)$ down. Now apply the corollary to conclude that for almost all skills the excess cross-entropy goes down.
\fi 



%\section{Lemmata about Random Bipartite Graphs}
%\subsection{Key Lemma about Random Bipartite Graphs}

%% -*- mode: LaTeX; fill-column: 78; -*-

\section{Concluding Remarks}
\label{sec:conclusions}

In this paper, we presented a novel SMC algorithm, \EventDPOR, tailored to the
characteristics of event-driven multi-threaded programs running under the SC
semantics. The algorithm was proven correct and optimal for event-driven
programs in which the variable accesses of events do not depend on how their
execution is interleaved with other threads.

We have implemented \EventDPOR in the \Nidhugg tool, and we will open-source
our implementation.
%
With a wide range of event-driven programs, we have shown that \EventDPOR
incurs only a moderate constant overhead over its baseline implementation
(\OptimalDPOR), it is exponentially faster than existing state-of-the-art SMC
algorithms in time and number of traces examined on programs where events'
actions do not conflict, and does not suffer from performance degradation
caused by having to examine
% a significant number of
non-serializable executions.
%
%% \bjcom{Should we include:
%% Moreover, in our benchmarks, also those that are not non-branching,
%% \EventDPOR explores only the optimal number of executions, and never
%% had to resort to a potentially expensive decision procedure.}

\EventDPOR assumes that handlers can process their events in arbitrary order.
Directions for future work include to retarget \EventDPOR for event-driven
programs with other policies (e.g., FIFO), and for specific event-driven
execution models.


%%%%%%%%%%%%%%%%%%%%%%%%%%%%%%%%%%%%%%%%%%%%%%%%%%%%%%%%%%%%

\bibliographystyle{plainnat}
\bibliography{refs}


\begin{comment}
\section{System Architecture}
\label{appendix:architecture}
\system has a novel modularized system architecture with three key components: 
\emph{StreamManager}, 
\emph{TxnManager} and \emph{TxnScheduler}. 
These components are instantiated in each thread locally.
The execution outline of \system is presented in Algorithm~\ref{alg:algo}.
Transactional stream processing is continuous and potentially never ends (Line 1$\sim$8).
The dependency resolution and execution of state transactions are separated into two non-overlapping phases by punctuations~\cite{Tucker:2003:EPS:776752.776780} (Line 2 and 5), which guarantees that no subsequent input event will have a smaller timestamp. 
Effectively, a batch of state transactions is collected during the first phase, and processed during the second phase.

In the first phase (i.e., stream processing phase), 
the \emph{StreamManager} conducts preprocessing for every input event ($e$). Similar to some prior works~\cite{tstream}, state transactions may be issued but not immediately processed during preprocessing (Line 3).
The \emph{pre\_processing} and \emph{post\_processing} functions are exposed as APIs to users.
The \emph{TxnManager} handles dependency resolution (Line 4) among state transactions and insert decomposed operations to construct a \tpg. We discuss the detailed two-phase \tpg construction process in Section~\ref{subsec:construction}.

In the second phase  (i.e., transaction processing phase), 
the \emph{TxnManager} is first involved again to refine (Line 6) the constructed \tpg with further dependency resolution.
The \emph{TxnScheduler} 
schedules operations for concurrent execution based on the constructed \tpg according to the three dimensions of scheduling decisions (Line 7). 
In particular, a scheduling decision model $M$ is instantiated based on the constructed \tpg (Line 14).
\textbf{\circled{1}} Guided by $M$, execution threads adopt an exploration strategy (Section~\ref{subsec:explore}) to explore the constructed \tpg for operations available to be scheduled constrained by dependencies. 
\textbf{\circled{2}} 
During exploration, one or multiple operations may be treated as the 
% basic 
unit of scheduling (Section~\ref{subsec:granularity}). 
Subsequently, \textbf{\circled{3}} every thread executes operation(s) in the unit of scheduling with various abort handling mechanisms (Section~\ref{subsec:abort_handling}).
Only when state transactions are processed (i.e., committed or aborted) can the associated input events be postprocessed (Line 8) by the \emph{StreamManager} based on transaction processing results.
\end{comment}

\begin{comment}
\begin{algorithm}
\footnotesize
    \KwData{$e$ \tcp{Input event}}
    \KwData{$txn_{ts}$ \tcp{State transaction}}
    \KwData{$G$ \tcp{The currently constructed TPG}}
    \While{!finish processing of input streams}{
        \eIf(\tcp*[h]{Phase 1}){\text{$e$ is not a $punctuation$}}{
                $txn_{ts}$ $\gets$ PRE\_Processing($e$)\;
                \textbf{TPG\_Construction}($G$, $txn_{ts}$)\; 
          }(\tcp*[h]{Phase 2}){
                \textbf{TPG\_Refinement}($G$)\; 
                \textbf{TXN\_Scheduling}($G$)\; 
                POST\_Processing()\;
          }
    }
    
    \SetKwFunction{FMain}{TPG\_Construction}
    \SetKwProg{Fn}{Function}{:}{}
    \Fn{\FMain{$G$, $txn_{ts}$}}{
        $O_{1..k}$ $\gets$ \textbf{Partition} $txn_{ts}$\;
        \ForEach{\text{operation $O_{i}$ $\in$ $O_{1..k}$}}{
            \textbf{Identify} its \ld\;
            $G$ $\gets$ $G$ + $O_{i}$ \;
        }
    }
    \SetKwFunction{FMain}{TPG\_Refinement}
    \SetKwProg{Fn}{Function}{:}{}
    \Fn{\FMain{$G$}}{
        \ForEach{\text{vertex $e_{i}$ $\in$ $G$}}{
            \textbf{Identify} its \td, \pd\;
        }
    }
    
    \SetKwFunction{FMain}{TXN\_Scheduling}
    \SetKwProg{Fn}{Function}{:}{}
    \Fn{\FMain{$G$}}{
        $M$ $\gets$ Instantiated with $G$;\tcp{A decision model}
        \While{!finish scheduling of $G$
        }{
          \textbf{\circled{2}} $Scheduling Unit$ $\gets$ \textbf{\circled{1}} \emph{Explore}($G$, $M$)\; 
            \textbf{\circled{3}} \emph{Execute with Abort Handling} ($Scheduling Unit$)\; 
        }
    }
  \caption{Execution Outline of \system}
  \label{alg:algo}
\end{algorithm}
\end{comment}

\end{document}