\documentclass[conference]{IEEEtran}
\IEEEoverridecommandlockouts
% The preceding line is only needed to identify funding in the first footnote. If that is unneeded, please comment it out.
\usepackage{cite}
\usepackage{amsthm}
\usepackage{amsmath,amssymb,amsfonts}
\usepackage{algorithmic}
\usepackage{graphicx}
\usepackage{textcomp}
\usepackage{xcolor}

%newly added packages
\usepackage{wrapfig}
\usepackage{bbm}
\usepackage{array, multirow}
\usepackage{diagbox}
\usepackage{caption}
\usepackage{etoolbox}
\usepackage{blindtext}
\usepackage{subcaption}
\usepackage{algorithmic}
\usepackage{wrapfig}
\usepackage{url}
\usepackage{soul}
\usepackage{booktabs}
\usepackage{bbm}
\usepackage[ruled,vlined]{algorithm2e}
\DeclareMathOperator*{\argmax}{arg\,max}
\DeclareMathOperator*{\argmin}{arg\,min}
\DeclareMathOperator{\Tr}{Tr}
\newtheorem{definition}{Definition}
\newtheorem{assumption}{Assumption}
\newtheorem{theorem}{Theorem}
\newtheorem{lemma}{Lemma}
\newtheorem{corollary}{Corollary}
\newtheorem{proposition}{Proposition}



\def\BibTeX{{\rm B\kern-.05em{\sc i\kern-.025em b}\kern-.08em
    T\kern-.1667em\lower.7ex\hbox{E}\kern-.125emX}}

\captionsetup[table]{labelformat = simple,textfont = sc}

\begin{document}

\title{Homophily-Driven Sanitation View for Robust Graph Contrastive Learning}
%\thanks{Identify applicable funding agency here. If none, delete this.}
%}

\author{\IEEEauthorblockN{Yulin Zhu\IEEEauthorrefmark{1}, Xing Ai\IEEEauthorrefmark{1}, Yevgeniy Vorobeychik\IEEEauthorrefmark{2}, Kai Zhou\IEEEauthorrefmark{1}}
\IEEEauthorblockA{yulin.zhu@polyu.edu.hk, xing96.ai@connect.polyu.hk, yvorobeychik@wustl.edu, kaizhou@polyu.edu.hk}
\IEEEauthorblockA{\IEEEauthorrefmark{1}\textit{Department of Computing}, \textit{The Hong Kong Polytechnic University}, HKSAR}
\IEEEauthorblockA{\IEEEauthorrefmark{2}\textit{McKelvey School of Engineering}, \textit{Washington University in Saint Louis}, St. Louis, USA}}

\maketitle
\pagestyle{plain}

\begin{abstract}
    We investigate the adversarial robustness of unsupervised Graph Contrastive Learning (GCL) against structural attacks. First, we provide a comprehensive empirical and theoretical analysis of existing attacks, revealing how and why they downgrade the performance of GCL.  Inspired by our analytic results, we present a robust GCL framework that 
    integrates a homophily-driven sanitation view, which can be learned jointly with contrastive learning. 
    A key challenge this poses, however, is the non-differentiable nature of the sanitation objective.
    To address this challenge, we propose a series of techniques to enable gradient-based end-to-end robust GCL. Moreover, we develop a fully unsupervised hyperparameter tuning method which, unlike prior approaches, does not require knowledge of node labels. We conduct extensive experiments to evaluate the performance of our proposed model, GCHS (Graph Contrastive Learning with Homophily-driven Sanitation View), against two state-of-the-art structural attacks on GCL.  Our results demonstrate that GCHS  consistently outperforms all state-of-the-art baselines in terms of the quality of generated node embeddings as well as performance on two important downstream tasks\footnote{This work has been submitted to the IEEE for possible publication. Copyright may be transferred without notice, after which this version may no longer be accessible.}. 
\end{abstract}

\begin{IEEEkeywords}
Graph contrastive learning, Graph sanitation, Adversarial robustness
\end{IEEEkeywords}

% Figure environment removed

\section{Introduction}
Automatic 3D reconstruction of clothed humans using image inputs has gained increasing significance due to its potential applications in a wide array of AR/VR scenarios. High-fidelity reconstructions typically depend on sophisticated capture systems, which are developed with dense camera arrays~\cite{collet2015high,joo2015panoptic,joo2018total}, programmable light-stages~\cite{Vlasic2009, guo2019relightables}, and depth sensors~\cite{newcombe2011kinectfusion,DoubleFusion,BodyFusion,dou2016fusion4d,newcombe2015dynamicfusion}. However, stringent capture environments equipped with complex hardware pose significant challenges for consumer-level applications.


In this context, considerable research effort has been dedicated to developing methods that allow for more flexible capture configurations, such as utilizing a few RGB inputs. Among these works, learning implicit functions \cite{iccv2020PIFu, saito2020pifuhd, hong2021stereopifu} has proven effective in achieving highly detailed reconstructions by integrating the advancements of deep neural networks. These methods employ large multi-layer perceptrons (MLPs) to predict the occupancy probability or truncated signed distance function (TSDF) value of every queried 3D point based on its associated local feature, which is extracted from images. They can recover a continuous surface at arbitrary resolutions without topology restrictions.


However, in typical MLP-based implicit networks, the occupancy or TSDF value at each location is solved independently with planar image features, rendering them less capable of addressing challenging cases such as occlusions. Consequently, these methods suffer from generalization and robustness issues, particularly when tackling strong occlusions caused by large motion or multiple interacting humans. 
Some follow-up studies  \cite{zheng2021deepmulticap,zheng2021pamir,huang2020arch} utilize an extra geometric model, SMPL~\cite{Loper2015}, to improve robustness by introducing strong shape priors. 
Their success typically relies on the assumption of geometrical similarity \cite{huang2020arch} between the shape prior and target reconstruction, making them intractable for handling complex cases with loose clothes and sensitive to errors in SMPL model fitting.



%\ping{this paragraph sounds like `TSDF is better than MLP/SMPL, and we use TSDF to solve the problem'. But in Sec 3, we are telling a different story, saying `MLP needs a 3D convolutional encoder'. We need to make these two sections consistent.}\sicong{I think in this paragraph we claim that the TSDF}


%We opt for Trucated Signed Distance Funtion (TSDF) volumetric representations as they are naturally suitable for convolution operations, which have shown remarkable performance for learning hierarchical features on 2D visual perception tasks \cite{SunXLW19}. 
%Meanwhile, TSDF also describes the gradual geometry change around shape surface, which is not reflected by occupancy volume. 

We instead revisit the 3D volumetric representation and resort to 3D convolutional neural networks (CNNs) for feature learning, due to their impressive performance in feature learning and the ability to incorporate spatial context. However, volumetric methods and 3D convolution involve discretization, which might raise concerns regarding whether a discretized volume can preserve subtle geometric details as continuous representations learned in implicit functions. We investigate the relationship between volume resolution and quantization error on synthetic data by converting target mesh objects to TSDF volumes, as shown in Figure~\ref{fig:quantization_error}. We observe that the quantization errors are significantly reduced by increasing volume resolution and become nearly negligible when reaching a relatively high resolution (e.g., 512 or higher). In other words, achieving fine-detailed reconstruction is not supposed to be restricted by the use of volume representations as long as a proper volume resolution is utilized. Therefore, we present a method with high-resolution feature volumes, e.g., 256 and 512, while traditional volumetric methods \cite{varol18_bodynet,gilbert2018volumetric} are often limited to much lower resolutions, such as 32 or 128.



On the other hand, an increase in volume resolution may lead to a cubic growth of memory overhead \cite{8100085}. Reducing memory costs while guaranteeing the granularity of volumetric representations is necessary for pursuing high-quality reconstruction. Thus, we adopt a coarse-to-fine approach and cull away irrelevant voxels to build a sparse high-resolution feature volume. At the coarse level, the network computes an initial TSDF by applying a U-Net with sparse 3D CNN \cite{3DSemanticSegmentationWithSubmanifoldSparseConvNet} on the sparse feature volume, which is carved by a visual hull. Through our experiments, it turns out that more than 95\% of the volume grids are discarded by the visual hull culling, making the sparse 3D CNN efficient. At the fine level, the network focuses on a narrow band near the zero-level set of the initial TSDF and discretizes the narrow band with smaller voxels. By employing this narrow-band culling, we further shrink the sampling space, resulting in a relatively small range of grid numbers (usually 300K--500K in our experiments) even with a high volume resolution of 512. The remaining voxels in the narrow band are associated with features that fuse high-frequency information from the computed normal maps upon the low-frequency shape from the coarse level to compute the TSDF at high resolution. The final mesh is then extracted from the TSDF using the Marching-Cube algorithm ~\cite{Lorensen87marchingcubes}.
% Different from the u-net sturcture to preserve global topology context, we then apply a shallow 3dcnn to compute the final TSDF $D_{final}$ which contain more local geometry detail.




% \ping{this paragraph can be expanded. It is an important contribution and often ignored by other works. stress on the novel idea of regressing blending weights instead of colors}

In addition to geometry, high-quality mesh texture is also a crucial factor contributing to visual appearance. Directly computing a color field in 3D space, as in \cite{iccv2020PIFu}, struggles to capture high-frequency texture details, while the neural radiance field (NeRF) \cite{yu2020pixelnerf} or the DoubleField~\cite{shao2022doublefield} require expensive per-instance optimization and are often unstable for sparse input images. In contrast, we adopt an image-based rendering approach to compute a texture atlas map, which is efficient and widely supported in existing computer graphics tools. 
Specifically, we compute a blending weight at each 3D point on the mesh surface to determine its color as a weighted average of the colors at its image projections. The blending weights can be computed at a relatively coarse resolution, e.g., 512 volume resolution in our case, and leave texture details to the high-resolution images, such as 1K or 2K. Unlike previous methods that generate blurry texturing results under sparse input, our method generalizes well on both synthetic and real data with just a few input views. 
Figure~\ref{fig:teaser} shows two examples reconstructed by our method. Despite the challenging garment, pose, and occlusion, our method recovers faithful shape, normal, and texture on the right.

%with a wide variety of poses and clothing styles, and it is also adaptive to handle input image with arbitrary resolutions.
%\sicong{For this concern we claim that when the resolution of dicretized volume meets certain threshold (which is 256 in our experiment), the quantization error can be neglected.} 



In summary, the main contributions of this paper are as follows:
\begin{itemize}
\vspace{-0.1in}
  \item 
  We revisit the 3D volumetric representation and demonstrate that it can support clothed human reconstruction with equal or even better performance compared to implicit representation. 
  \item 
  We develop a memory and computation-efficient method for high-resolution volumetric reconstruction using sophisticated sparse 3D CNN, coarse-to-fine estimation, and voxel culling by visual hull and narrow bands. 
  \item 
  We introduce a novel method to compute a texture atlas map, which captures rich appearance details from high-resolution input images.
  \item 
  We achieve impressive results on standard benchmark datasets Twindom and MultiHuman, significantly reducing the point-2-surface (P2S) precision to approximately 0.2cm from just six input views, with more than $50\%$ error reduction compared to the state-of-the-art methods, including DoubleField~\cite{shao2022doublefield} and PIFuHD~\cite{saito2020pifuhd}.
\end{itemize}
\section{Related Work}
\label{appsec: related work}
Bayesian causal discovery literature has primarily focused on inference in linear models with closed-form posteriors or marginalized parameters. Early works considered sampling directed acyclic graphs (DAGs) for discrete~\cite{cooper1992bayesian, madigan1995bayesian, heckerman2006bayesian} and Gaussian random variables~\cite{friedman2003being, tong2001active} using Markov chain Monte Carlo (MCMC) in the DAG space. However, these approaches exhibit slow mixing and convergence~\cite{eaton2012bayesian,grzegorczyk2008improving}, often requiring restrictions on number of parents~\cite{kuipers2017partition}. %Alternative exact dynamic programming methods are limited to small settings~\cite{koivisto2012advances}. 

Recent advances in variational inference~\cite{zhang2018advances} have facilitated graph inference in DAG space, with gradient-based methods employing the NOTEARS DAG penalty \cite{zheng2018dags}.\cite{annadani2021variational} samples DAGs from autoregressive adjacency matrix distributions, while \cite{lorch2021dibs} utilizes Stein variational approach \cite{liu2016stein} for DAGs and causal model parameters. \cite{cundy2021bcd} proposed a variational inference framework on node orderings using the gumbel-sinkhorn gradient estimator \cite{mena2018learning}. \cite{deleu2022bayesian,nishikawa2022bayesian} employ the GFlowNet framework \cite{bengio2021gflownet} for inferring the DAG posterior. Most methods, except\cite{lorch2021dibs} are restricted to linear models, while \cite{lorch2021dibs} has high computational costs and lacks DAG generation guarantees compared to our method.
% at least quadratic scaling complexity, both with respect to the number of nodes (due to the DAG penalty) as well as number of posterior samples. Our proposed approach instead has linear complexity with respect to number of posterior samples and does not require any additional DAG penalty.     

In contrast, \emph{quasi-Bayesian} methods, such as DAG bootstrap \cite{friedman2013data}, demonstrate competitive performance. DAG bootstrap resamples data and estimates a single DAG using PC \cite{spirtes2000causation}, GES \cite{chickering2002optimal}, or similar algorithms, weighting the obtained DAGs by their unnormalized posterior probabilities. Recent neural network-based works employ variational inference to learn DAG distributions and point estimates for nonlinear model parameters \cite{charpentier2022differentiable,geffner2022deep}.
%
\vspace{-0.15cm}

\section{Strategy Templates}\label{sec:templates}

In this section, we introduce a formalization of player \Odd strategies in \Odd-fair parity games via \emph{strategy templates}.
% 
In contrast to player \Even, player \Odd winning strategies are no longer positional in \Odd-fair parity games, as illustrated by the following example. %that requires the same number of symbolic steps as the algorithm computing winning strategies for \Even in \enquote{normal} parity games.
% \vspace{-0.5em}
\begin{example}\label{ex:strategytemplates}
Consider the three different parity games depicted in Fig.~\ref{fig:Oddstrategies1}. %, three \Odd-fair parity games are depicted, with circles indicating \Ve and squares indicating \Vo. Edges in $E^\ell$ are shown by dashed lines. All nodes are labeled with their priorities.
   In all three games, \Odd has a winning strategy from all vertices, i.e., $\mathcal{W}_{Odd}=V$. %The red-colored edges indicate \Odd's strategy: if \Odd takes the red edges alternatingly from the source nodes, it wins from all nodes. 
  However, in order to win, the vertex $3$ has to be seen infinitely often in game (a) and (b), which forces \Odd to use its live edge\textbackslash s infinitely often. This prevents the existence of a positional strategy for \Odd in games (a) and (b): In (a) it needs to somehow alternate between (it's only) live edge to $4$ and a \enquote{normal} edge to $7$ (both indicated in red) in order to win, and in (b) it needs to somehow alternate between all its live edges (also indicated in red). In the game (c), \Odd can win by 'escaping' its live vertex $3$ to a \enquote{normal} vertex $5$, and thereby has a positional strategy. % (again indicated in red).
   
  Now consider the subgraph of each game formed by all colored edges (red and blue), which include the strategy choices from \Vo and \emph{all} outgoing edges from \Ve. As we have seen that \Odd needs to play all red edges repeatably, this subgraph represents the paths that \emph{can} be seen in the game depending on the \Even strategy. Hence, a node $v\in\Vl\subseteq\Vo$ can be seen infinitely often in a play (compliant with \Odd's strategy), if it lies on a cycle in this subgraph. We observe that, in games (a) and (b), node $3$ lies on cycles in this subgraph, whereas in game (c), it does not. 
  We further see that whenever a vertex  $v\in\Vl$ lies on a cycle, \Odd needs to take all its outgoing live edges (as for vertex $3$ in example (b)) and possibly one more edge (as for vertex $3$ in example (a)), for all other vertices in $\Vo$ a positional strategy suffices (as for vertex $5$ in all examples, and for vertex $3$ in example (c)). This shows that \Odd strategies are intuitively still \enquote{almost positional}.
% 
\end{example}

% Figure environment removed


\vspace*{-0.2cm}

The intuitions conveyed by Ex.~\ref{ex:strategytemplates} are formalized by the following definitions. % for \Odd strategy templates.


\begin{definition}[\Odd Strategy Template]\label{def:Oddstrategytemplate}
 Given an \Odd-fair parity game $\mathcal{G}^\ell = \ltup{\mathcal{G}, E^\ell}$ with \newline $\mathcal{G} = \langle V, \Ve, \Vo, E, \chi\rangle$, an \Odd \emph{strategy template} $\mathcal{S}$ over $\mathcal{G}^\ell$ is a subgraph of $\mathcal{G}$ given as follows: $\mathcal{S}:=\tup{V',E'}$ where $V'\subseteq V$ and $E'\subseteq E \cap (V' \times V')$ such that the following hold,
\begin{compactitem}\label{item:Oddstrtemprules}
 \item if $v \in \Vo \cap V'$ does not lie on a cycle in $(V',E')$, then $|E'(v)|=1$,
 \item if $v \in \Vo \cap V'$ lies on a cycle in $(V',E')$ then $E^\ell(v) \subseteq E'(v)$ and  $1\leq |E'(v)|\leq |E^\ell(v)| + 1$,
 \item if $v \in \Ve \cap V'$, then  $E'(v) = E(v)$.
\end{compactitem}
\end{definition}
%
\begin{definition}\label{def:compliantstrat}
 Let  $\mathcal{G}^\ell = \ltup{\mathcal{G}, E^\ell}$ be an \Odd-fair parity game with \Odd strategy template $\mathcal{S}=\tup{V',E'}$, and $V'_\Odd := V' \cap V_\Odd$. Then an
\Odd strategy $\rho$ is said to be \textbf{compliant} with $\mathcal{S}$ if  
it is a winning strategy in the game $\ltup{\gamegraph,\alpha'}$ where $\gamegraph= \tup{V,\Ve,\Vo,E}$ and 
\begin{subequations}
 \begin{align}
 \alpha':= &\textstyle\bigwedge_{v\in\Vo'}(\,\square\, (\,v \implies \bigvee_{(v,w)\in E'} \bigcirc\, w\,))\,\label{equ:alpha:a}\\
 & \textstyle\wedge \bigwedge_{v\in\Vo'} (\,\square \,\diamondsuit\, v \implies \bigwedge_{(v,w)\in E'}\square\, \diamondsuit\, (\,v \wedge \bigcirc \,w\,)).\label{equ:alpha:b}
\end{align}
\end{subequations}
\end{definition}

Intuitively, for all \Odd vertices in $\mathcal{S}$, the strategy $\rho$ compliant with $\mathcal{S}$ takes only their outgoing edges in $\mathcal{S}$ \eqref{equ:alpha:a}, and if a play visits an \Odd node $v$ infinitely often, then $\rho$ takes each of $v$'s outgoing edges in $\mathcal{S}$ infinitely often \eqref{equ:alpha:b}.
% 
For an \Odd strategy template $\mathcal{S}$, if $v \in V'_\Odd$ lies on a cycle in $\mathcal{S}$, then by Def. \ref{def:Oddstrategytemplate}, $\mathcal{S}$ contains all live outgoing edges of $v$. By \eqref{equ:alpha:b} any \Odd strategy $\rho$ compliant with $\mathcal{S}$ satisfies the fairness condition in \eqref{eq:fairness-ltl} for $v$. 
On the other hand, if $v \in V'_\Odd$ does not lie on a cycle in $\mathcal{S}$, then by \eqref{equ:alpha:a} any such $\rho$ sees $v$ at most once. Thus $\rho$ trivially satisfies \eqref{eq:fairness-ltl} for $v$. 
This observation is stated in the following proposition.


\begin{proposition}
 Given the premisses of Def.~\ref{def:compliantstrat} let $\pi$ be a play starting from a node in $V'$ that complies with $\rho$. Then $\pi \models \alpha$ where $\alpha$ if the LTL formula in~\eqref{eq:fairness-ltl}.%\vspace{-2mm}
\end{proposition}

Next, we define \Even strategy templates. Each \Even strategy template encodes a unique \Even positional strategy, which is known to exist in \Odd-fair parity games \cite{banerjee2022fast}, due to the lack of fair edges defined on \Even vertices. %, \Even strategy templates are very simple\footnote{In fact, \Even strategy templates simply encode a positional strategy and are only re-defined to make further arguments more symmetric for both players.}.
\begin{definition}\label{def:Evenstrategytemplate}
    Given an \Odd-fair parity game $\mathcal{G}^\ell = \ltup{\mathcal{G}, E^\ell}$ with \newline $\mathcal{G} = \langle V, \Ve, \Vo, E, \chi\rangle$, an \Even \emph{strategy template} $\mathcal{S}$ over $\mathcal{G}^\ell$ is a subgraph of $\mathcal{G}$ given as $\mathcal{S}:=\tup{V', E'}$ where $V'\subseteq V$ and $E'\subseteq E \cap (V' \times V')$ such that,    \begin{compactitem}\label{item:Evenstrtemprules}
     \item if $v \in \Ve \cap V'$, then $|E'(v)|=1$,
     \item if $v \in \Vo \cap V'$, then  $E'(v) = E(v)$.
    \end{compactitem}
\end{definition}

\vspace*{-0.1cm}

An \Even strategy $\rho$ is compliant with the \Even strategy template $\mathcal{S} = \tup{V', E'}$ if for all $v \in V'_\Even$, $\rho(v) = E'(v)$. In other words, $\rho$ is the positional strategy defined by $\mathcal{S}$.

Let $\rho$ be an \Odd (\Even) strategy, compliant with the \Odd (\Even) strategy template $\mathcal{S}$ and let $\pi$ be a play compliant with $\rho$. Then we call $\pi$ a play \emph{compliant with $\mathcal{S}$}.

\vspace*{-0.1cm}

\begin{definition}
An \Odd (\Even) strategy template $\mathcal{S}=\ltup{V', E'}$ is \emph{winning} in the \Odd-fair parity game $\mathcal{G}^\ell$ if all \Odd (\Even) strategies $\rho$ compliant with $\mathcal{S}$ are winning for player \Odd (\Even) in $\mathcal{G}^\ell$ from $V'$. A winning \Odd (\Even) strategy template $\mathcal{S}$ is called \emph{maximal} if $V'=\Wo$ ($\We$).%\vspace{-2mm}
\end{definition}

\vspace*{-0.2cm}
We note that maximal winning \Odd (\Even) strategy templates $\mathcal{S}$ immediately imply that for every vertex $v\in \Wo$ ($\We$) there exists a winning strategy for player \Odd (\Even) from $v$ that is compliant with $\mathcal{S}$.
% 
The existence of maximal winning \Even strategy templates follows from the existence of positional \Even strategies \cite{banerjee2022fast}. 
% 
The first main contribution of this paper is a constructive proof showing the existence of maximal winning \Odd strategy templates given in the next section. 
This result is then used in Sec.~\ref{sec:zielonka} to prove the correctness of \Odd-fair Zielonka's algorithm, which is introduced there.




%% !TeX root = ../MVFD_arxiv.tex


\section{The framework}\label{sec2}
In this section we  present a formal mathematical setup for the local regularity  for bivariate   stochastic processes  (also called scalar random fields, or simply random fields) and the data observed for such processes.

\subsection{Data}\label{sec:data}
Consider $N$ independent realizations, also called sheets, $X^{(1)},\ldots,X^{(j)},\ldots X^{(N)}$   of a stochastic process $X $ defined on a continuous domain $\cT\in\mathbb R^2$. For simplicity, we here focus on domains $\mathcal T$ in the plane, the extension to higher dimensions would not raise different challenges. For the purpose of describing our methodology, we distinguish three observational scenarios of the $N$ realizations. 
First, the ideal, infeasible situation where the sheets $X^{(j)}$ are \emph{completely observed}, that is without error over the entire domain $\cT$.  
The second case is the one where the $X^{(j)}$ are observed (measured) at some \emph{discrete points} in the domain $\cT$, \emph{without noise}. The domain points can be fixed to be the same for all the $X^{(i)}$'s (common design), or can be randomly drawn for each sheets separately (independent design). Finally, the most realistic scenario is the one where in the second case we admit that the realizations of $X$  are observed at discrete domain points \emph{with noise}. 



 To formally describe the second and third scenarios, let  $M_1, \dotsc,M_N$ be an independent sample of an integer-valued random variable $M$ with expectation $\EE[M]=\Mmu$. 
% which increases with $N$. 
In the independent design case, for each $1\leq j \leq N$, and given  $M_j$, let $\Tnm\in\cT$,  $1\leq m \leq  M_j$, be a random sample of a random vector $\TT\in\cT$. The $\Tnm$'s represent the observation points for the realization $\Xp{j}$. We assume that the realizations of $X$, $M$ and $\TT$ are mutually independent. 
In the common design case, $M\equiv \mathfrak m$ and the $\Tnm$'s are the same for all $j$.  Let $\mathcal T_{obs}^{(j)} $ denote the set of observation times $ \Tnm $, $1\leq m \leq M_j$, on the sheet $\Xp{j}$. With common design,  $\mathcal T_{obs}^{(j)} $ does not depend on $j$, while with independent design the expected cardinal of $\mathcal T_{obs}^{(j)} $ can be random with mean $\mathfrak m$.   The following presentation  includes both independent design and common design cases. \color{black}  Finally, the data  consist of  the pairs  $(\Ynm , \Tnm ) \in\mathbb R \times \cT $ where $\Ynm$ is defined as
\begin{equation}\label{model_eq}
	\Ynm = \Xn (\Tnm) + \enm, \quad\text{with}  \quad  \enm = \sigma(\Tnm,\Xn(\Tnm)) \unm, 
	\quad 1\leq i \leq N,  \; 1\leq m \leq M_j.
\end{equation}
Here, the $\unm  \in\mathbb R $ are independent copies of a centered variable $e$ with unit variance, and $\sigma^2(\cdot,\cdot)\geq 0$ is some unknown, bounded conditional variance  function which account for possibly heteroscedastic measurement errors. The case  $\sigma(t,x)\equiv 0$ corresponds to our second scenario, while in the third scenario we have positive conditional variance. 

For each $1\leq j \leq N$,  let $\widetilde X^{(j)}$ denote an observable approximation of $X^{(j)}$. If   the sheets $X^{(j)}$ were  completely observed, as in our infeasible first scenario, $\widetilde X^{(j)} = X^{(j)}$. 
When $X^{(j)}$ are observed  only at  some discrete points $\Tnm$,  arbitrary  $\widetilde X^{(j)}(\Tt)$ can be obtained by simple interpolation or defined equal to the value of $\widetilde X^{(j)}$ at the nearest neighbor of $\Tt$.  Finally, with noisy, discretely observed sheets,   $\widetilde X^{(j)}$  is a pilot nonparametric estimator  of $X^{(j)}$, such as kernel smoothing, splines \emph{etc}.  


Let us next introduce a general class of stochastic processes (random fields)  $X$ with  irregular realizations $X^{(j)}$, for which the regularity can vary over the domain $\mathcal T$. 


\subsection{A class of multivariate processes}
Let $\mathcal{T}$ be an open, bounded bi-dimensional rectangle with the closure included in $(0,\infty)^2$. In the following, $H_1,H_2 : \mathcal T \to (0,1)$ are two continuously differentiable functions such that 
\begin{equation}\label{low_thres}
\min_{i=1,2} \inf_{t\in\mathcal T} H_i(\Tt) >0.
\end{equation} 	
Let $\overline{H} = \max\{H_1 ,H_2\}$.
We also consider the vector-valued function $\mathbf{L}=(L_1^{(1)},L_2^{(1)},L_1^{(2)},L_2^{(2)}),$  where the components  are  non-negative, Lipschitz  continuous functions defined on $\mathcal{T}$ such that 
\begin{equation}\label{id_L}
	L_i^{(1)}(\Tt) +L_i^{(2)}(\Tt) >0,\qquad \forall \Tt\in\mathcal T, \; i=1,2.
\end{equation}
%We denote $\mathbf{L}=(L_1^{(1)},L_2^{(1)},L_1^{(2)},L_2^{(2)}),$ and assume that a constant $C_{\mathbf L}$ exists such that 
%\begin{equation}\label{up_thres}
%C_{\mathbf{L}}=\max_{i,j=1,2} \sup_{t\in\mathcal T} L_j^{(i)}(\Tt) <\infty.
%\end{equation} 	

Let $X$ be a real-valued, second order stochastic process defined on $(0,\infty)^2$. Let $(e_1,e_2)$ be the canonical basis of $\mathbb R ^2,$ and, for sufficiently small scalars $\Delta$, let   
$$
\theta_{\Tt}^{(i)}(\Delta)=\EE\left[\left\{X\left(\Tt-\frac{\Delta}{2}e_i\right)-X\left(\Tt+\frac{\Delta}{2}e_i\right)\right\}^2\right],\quad i=1,2.
$$ 


\begin{definition}\label{def}
Let $H_1$, $H_2$ satisfy \eqref{low_thres}.	The class $\mathcal {H}^{H_1,H_2}(\mathbf{L},\mathcal{T})$ is the set of stochastic processes $X$ satisfying the following condition:  constants $  \Delta_0, C,\beta>0$ exist such that 
	for any $\Tt\in \mathcal T$ and $ 0<\Delta\leq\Delta_0$,  
	\begin{equation}\label{as_repr}
		\left|\theta_{\Tt}^{(i)}(\Delta)-L_1^{(i)}(\Tt)\Delta^{2H_1(\Tt)} -L_2^{(i)}(\Tt)\Delta^{2H_2(\Tt)}\right|\leq C\Delta^{2 \overline{H}(\Tt)+\beta}, \quad i=1,2.
	\end{equation}
	Let  
	$$
	\mathcal{H}^{H_1,H_2} %=\mathcal{H}^{H_1,H_2}(\mathcal{T})
	=\bigcup_{\mathbf{L}}\mathcal {H}^{H_1,H_2}(\mathbf{L},\mathcal{T}) ,
	$$
	where the union is taken over the set of four-dimensional functions $\mathbf{L}$ with non negative positives Lipschitz  continuous components satisfying \eqref{id_L} 
	%and \eqref{up_thres}. 
	The functions $H_1,H_2$ define the local regularity of the process, while $\mathbf{L}$ represent  the local Hölder constants.
\end{definition}

Definition \ref{def}  is general, and extends the local regularity notion considered by \cite{GKP} for processes defined on a compact interval on the real line.  A main example we have in mind is the  multi-fractional Brownian  sheet (MfBs) with a time-deformation. MfBs  is a generalization of the standard fractional Brownian sheet, where the Hurst parameter is allowed to vary along the  domain. The definition of this general class of  processes and some of their properties are provided in Section \ref{BfMs}.




\section{METHODS}
\label{sec:methods}
\subsection{Problem Definition and Proposed Framework}
The objective is to reconstruct a dense point cloud that precisely represents the shape of unknown transparent objects from sparse point clouds extracted with active tactile interactive perception. To this end, we propose a novel framework termed ACTOR shown in Fig.~\ref{fig:framework}. In Fig.~\ref{fig:framework}(a) we propose a self-surpervised learning approach with an autoencoder network that is trained on subsampled pointclouds from synthetic objects belonging to the same category but not identical as the real objects. In Fig.~\ref{fig:framework}(b), we propose a novel active tactile-based unknown transparent object exploration strategy which is used for inference with our trained model to reconstruct a dense point cloud. We demonstrate downstream tasks such as tactile-based pose estimation.
% and tactile-based object recognition. 

\subsection{Deep Self-Supervised Learning for 3D Object Reconstruction}
\label{ssec:deep_reconstruction}
We generate a dataset $\mathcal{D}$\footnote{\url{https://www.robotact.de/tactile-reconstruction}} of synthetic object models from the ShapeNet repository~\cite{chang2015shapenet} in order to leverage the open-source datasets and avoid expensive real tactile-data collection. The synthetic object models belong to the same category but are different from the real unknown transparent objects. 
We uniformly sample $N_{in} = 2048$ points from the synthetic object meshes. These pointclouds are normalized and scaled to fit into a $[0,1]^3$ cube and added to the dataset, $\mathcal{P}_{in} \in \mathcal{D}$. 
% The generated dataset is provided in the project page\footnote{\url{https://robotac-bmw.github.io/tactile_reconstruction/}}.
In order to generate the input point clouds $\mathcal{P}^{\bullet}_{in}$ to the network, we randomly subsample the $\mathcal{P}_{in}$ by voxel-grid subsampling by the factor $k$ i.e., $\mathcal{P}^{\bullet}_{in} \in \mathbb{R}^{\lceil \frac{1}{k}N_{in} \rceil \times 3}$.  This creates a challenging task for reconstruction with higher values for $k$ as simpler techniques based on interpolation with neighborhood points cannot be used. 

\subsubsection*{Feature-Extraction Encoder}
The network architecture shown in Figure~\ref{fig:framework}(a) is proposed as an autoencoder (AE) that uses a self-supervised approach to reconstruct the original point cloud from a subsampled point cloud. 
The encoder takes subsampled point clouds as inputs and generates a high dimensional feature vector. The feature vector captures the global geometric shape information of the input point cloud. 
In general, any deep network that works on raw input point clouds to provide a high dimensional feature vector can be used as an encoder. In particular,
we use a modified PointNet architecture~\cite{qi2017pointnet} for the encoder. PointNet takes unordered point clouds and generates a global feature descriptor vector of size 1024. The network learns a set of optimization functions that select interesting or informative points of the point cloud. The encoder consists of $[1\times1]$ convolutions with output channels size $(64, 64, 128, 1024)$ with the first convolutional layer with kernel size $[1\times3]$ to encode the input pointcloud of $N\times3$ dimension. The convolution layers are aggregated by a max-pooling layer. We introduce a self-attention layer~\cite{zhang2019self} whose outputs are aggregated with the max-pooled features to provide the global feature vector.  
We have summarized the encoder in Figure~\ref{fig:framework}(a).
% As the encoder provides a high-dimensional global feature vector, we term it as feature-extraction encoder.

\textbf{Self-Attention (SA) Layer:} The SA layer is introduced as it can encode meaningful spatial relationships between features and focus on important local features. From the input layer ($\mathtt{conv2d-1024}$), two separate multi-layer perceptrons (MLPs) are used to get features $\mathbf{G}$ and $\mathbf{H}$ which are subsequently used to get the weights as $\mathbf{W} = softmax(\mathbf{G}^T\mathbf{H})$. The input features are transformed using another MLP to obtain $\mathbf{K}$ and multiplied with the weights as $\mathbf{W}^T\mathbf{K}$.
These vectors are summed with the input vector to produce the output features.
% The SA layer description is shown in Fig.~\ref{fig:self_atten}.  
% \setlength{\columnsep}{0pt}
% \begin{wrapfigure}[12]{r}{0.8\linewidth}
%   \centering
%     % \vspace{-0.5cm}
%     % Figure removed
%   \caption{The self-attention unit.}
%     % \vspace{-0.5cm}
%   \label{fig:self_atten}
% \end{wrapfigure}
% % Figure environment removed

\subsubsection*{Upsampling Decoder}
We design an upsampling decoder that upsamples the input global feature vector to provide the reconstructed dense output point cloud $\mathcal{P}_{out}$. The upsampling decoder is composed by a fully connected layer with output dimension of 1024 and five deconvolutional layers with kernel sizes and output channels shown in Fig.~\ref{fig:framework}(a).  
The decoder produces the output point cloud with point size set to 2048 while training as this is sufficiently dense for reconstruction purposes. 

\subsubsection*{Loss Function}
In order to encourage the upsampled point cloud to be in proximity to the original input point cloud and follow the underlying geometrical surface of the object, we use the Chamfer distance metric~\cite{borgefors1986distance} as the loss. Given the input point cloud prior to subsampling, $\mathcal{P}_{in}$ and the reconstructed output point cloud $\mathcal{P}_{out}$, the loss is defined as:
\begin{align}
    \mathcal{L}_{CD}(\mathcal{P}_{in}, \mathcal{P}_{out}) &= \frac{1}{|\mathcal{P}_{in}|}\sum_{p_1 \in \mathcal{P}_{in}} \min_{p_2 \in \mathcal{P}_{out}} ||p_1 - p_2||_{2} + \\ & \frac{1}{|\mathcal{P}_{out}|}\sum_{p_2 \in \mathcal{P}_{out}} \min_{p_1 \in \mathcal{P}_{in}} ||p_2 - p_1||_{2} \nonumber,
    \label{eq:chamfer_dist}
\end{align}
where $|\bullet|$ refers to the number of points in the point cloud and $||\bullet||_2$ refers to the L2 norm. The loss $\mathcal{L}_{CD}$ represents the average distance between the \textit{closest} points in the two point clouds. We use the weighted loss for learning stability as the reconstruction loss $\mathcal{L}_{rec} = \alpha\mathcal{L}_{CD}$ with $\alpha = 100$ set empirically.
For surface reconstruction from the dense reconstructed point cloud, we use the ball-pivoting algorithm~\cite{bernardini1999ball}.

% \subsubsection*{Recognition Network}
% \label{ssec:recog_net}
% The pretrained encoder layers for reconstruction task are frozen for category-level classification. We employ three fully-connected layers with parameters 512, 256, and $n$ respectively where $n$ represents the number of categories of the objects.
% The softmax cross-entropy loss is used for training the recognition network. The recognition head is shown in Fig.~\ref{fig:framework}(a.I). The subsampled sparse point clouds from our synthetic dataset with different subsampling ratios and data augmentation with random rotations are used. Network implementation details are provided in Sec.~\ref{ssec:setup}.



%%%%%%%%%%%%%%%%%%%%%%%%%%%%%%%%%%%%%%%%%%%%%%%%%%%%%%%%%%%%%%%%%%%%%%%
%%%%%%%%%%%%%%%%%%%%%%%%%%%%%%%%%%%%%%%%%%%%%%%%%%%%%%%%%%%%%%%%%%%%%%%
\subsection{Active Deep Tactile-based Unknown Transparent Object Reconstruction and Pose Estimation}
\subsubsection{Active Tactile-based Transparent Object Reconstruction}
The model trained with only \textit{synthetic data} as described in Sec.~\ref{ssec:deep_reconstruction} is used during the inference with \textit{real-world} transparent objects. The sparse tactile point cloud data is collected autonomously by the robot using an information gain-based active strategy. We define two types of tactile actions for data acquisition: touch and pinch actions as shown in Figure~\ref{fig:occupancy_grid}.
% The action nomenclature is derived from human grasp taxonomy studies~\cite{feix2015grasp}.
The touch action is executed as a guarded horizontal straight-line motion wherein the object is not moved upon contact. The touch action is defined by a tuple $\mathbf{a}^{t} = \{\mathbf{s}^t, \overrightarrow{\mathbf{d}^t} \}$ where $\mathbf{s}^t \in \mathbb{R}^3$ is the start point of the tactile-sensorised gripper and $\overrightarrow{\mathbf{d}^t} \in \mathbb{R}^3$ is the direction of the gripper-motion defined in the world-coordinate frame $\mathcal{W}$. During the pinch action the robot approaches the object in a vertical straight-line motion with a completely open gripper and performs an antipodal enclosure grasp on the object. The fingers of the gripper are closed until the force on the tactile sensors exceeds a predefined threshold.
The pinch action is characterized by $\mathbf{a}^{p} = \{\mathbf{s}^p \}$ where $\mathbf{s}^p \in \mathbb{R}^3 $ is the start position of the gripper motion vertically above the object at a predefined height as shown in Figure~\ref{fig:occupancy_grid}. Given the 2D bounding box of the object (a priori known or through a RGB camera), a probabilistic occupancy grid $\mathcal{OG}_i$ of preset height and resolution $og_{res}$ is defined. Each cell of the occupancy grid $c_i$ is represented by an occupancy probability $p(c_i)$ which is initially set to 0.5. During exploration, if a cell is discovered to belong to the object, the probability is set to 1 and similarly, if the cell belongs to free space, the probability is set to 0. The probabilities are updated through ray intersections based on the virtual sensor model. We define a virtual sensor model of the tactile sensor which casts a set of rays $\mathcal{R} = \{r_1, r_2, \dots, r_{n_{taxel}} \}$ where ${n_{taxel}} $ refers to the number of taxels in the sensor array. The independence assumption of the probability of each grid cell with one another allows us to calculate the overall entropy of the $\mathcal{OG}$ as the summation of the entropy of each cell. The Shannon entropy of the overall occupancy grid is calculated as:
\begin{equation}
    \mathbb{H}(\mathcal{OG}) = \sum_{c_i \in \mathcal{OG}} p(c_i)log(p(c_i)) + (1 - p(c_i))(1 - log(p(c_i))).
    \label{eq:entropy}
\end{equation}
Monte-Carlo sampling of possible tactile actions $N_{nbt}$ are performed for computing the next best tactile (NBT) action. The actions space $\mathcal{A}_{nbt}$ is comprised of an equal number of touch and pinch respectively as $\mathcal{A}_{nbt} = \{a^p, a^t\}_{N_{nbt}}$. The expected measurements $\hat{\mathbf{z}}_t$ for each action $a_t \in \mathcal{A}$ is computed using ray-traversal algorithms~\cite{hornung2013octomap}. 
Given the observed grid cell $c$ and the measurement from sensor observation $z$, the log-odds is updated as $L(c|z) = L(c) + l(z)$ wherein $L(c) = log\frac{p(c)}{1-p(c)}$ and  
\begin{equation}
    l(z) = \left\{
                \begin{array}{ll}
                  log\frac{p_h}{1-p_h}  \quad \mathrm{if} \ z \widehat{=} \textit{ hit} \\
                  log\frac{p_m}{1-p_m} \quad \mathrm{if} \ z \widehat{=} \textit{ miss} 
                \end{array}
              \right.
    \label{eq:log-odds}
\end{equation}
where $p_h$ and $p_m$ are the probabilities of hit and miss which are user-defined values set to 0.7 and 0.4 respectively as in~\cite{hornung2013octomap}. The posterior probability $p(c|z)$ can be computed by inverting $L(c|z)$. The expected information gain by taking an action $a_t \in \mathcal{A}_{nbt}$ with expected measurement $\hat{\mathbf{z}}_t$ is provided by the Kullback-Liebler divergence of the posterior entropy and the prior entropy as: 
\begin{equation}
    E[\mathbb{I}(p(c_i | \mathbf{a}_t,  \hat{z}_t))] = \mathbb{H}(p(c_i)) - \mathbb{H}(p(c_i | \mathbf{a}_t,  \hat{z}_t))
    \label{eq:kl_view}
\end{equation}
Therefore, the action that maximizes the expected information gain is considered as the NBT action:
\begin{equation}
    \mathbf{a}^{nbt*}_t = \argmax_{\mathbf{a} \in \mathcal{A}}(E[\mathbb{I}(p(c_i | \mathbf{a}_t,  \hat{z}_t))])
    \label{eq:kl_view_max}
\end{equation}
Each tactile action extracts contact positions in 3D space and contact forces. The direction of the normal force is used to extract the normal direction $\hat{n}$ of the object surface. The contact points are aggregated into the tactile point cloud $\mathcal{P}^t$. In order to initialize the NBT action calculation, an initial point cloud (with $N_{\mathcal{P}^t} = 20$) is required, which is extracted by randomised touch actions. Further points are collected in an active manner using the NBT criteria. A minimum number of points in the tactile point cloud is required to perform model inference $N_{\mathcal{P}^t} > N_{min}$ which is tuned empirically. The tactile point cloud is provided as input to the trained network and the reconstructed point cloud $\mathcal{P}_{out}$ is obtained . 
% This is used for downstream task Section~\ref{ssec:pose_estimation}. 
% For acceptable reconstruction accuracy around 100 tactile points is required.  

% [TODO:] check for action taxonomy if its correct

%%%%%%%%%%%%%%%%%%%%%%%%%%%%%%%%%%%%%%%%%%%%%%%%%%%%%%%%%%%%%%%%%%%%%%%
% Figure environment removed
%%%%%%%%%%%%%%%%%%%%%%%%%%%%%%%%%%%%%%%%%%%%%%%%%%%%%%%%%%%%%%%%%%%%%%%


%%%%%%%%%%%%%%%%%%%%%%%%%%%%%%%%%%%%%%%%%%%%%%%%%%%%%%%%%%%%%%%%%%%%%%%
%%%%%%%%%%%%%%%%%%%%%%%%%%%%%%%%%%%%%%%%%%%%%%%%%%%%%%%%%%%%%%%%%%%%%%%
\subsubsection{Tactile-Based Object Pose Estimation}
\label{ssec:pose_estimation}

We perform the 6D pose estimation through dense to sparse point cloud registration. The sparse scene point cloud $\mathbf{s}_i \in \mathcal{S}$ is represented by the tactile points and the dense object point cloud $\mathbf{o}_i \in \mathcal{O}$ is represented by the reconstructed point cloud in~\ref{ssec:deep_reconstruction} without the need for the object model. Point cloud registration problem with $M$ known correspondences can be formulated as:
\begin{equation}
     \mathbf{s}_i = \mathbf{S}\cdot(\mathbf{R}\mathbf{o}_i) + \mathbf{t} \quad i = 1, \dots M,
     \label{eq:generativemodel}
 \end{equation}
where $\mathbf{S} \in \mathbb{R}^3$ represents scale, $\mathbf{R} \in SO(3)$ represents rotation and $\mathbf{t} \in \mathbb{R}^3$ represents translation which are unknown and to be estimated and $\cdot$ is the element-wise product. 
%% [TODO] : check derivation

We perform the point cloud registration using our novel translation-invariant Quaternion filter (TIQF) presented in~\cite{murali2022active} to determine $\mathbf{R}$, $\mathbf{S}$ and $\mathbf{t}$. 
The scale, rotation and translation are decoupled by finding the relative vectors between corresponding points, i.e., $\forall o_i, o_j \in \mathcal{O}, s_i, s_j \in \mathcal{S}$ the relative vectors are $\mathbf{s}_{ji} = \mathbf{s}_j - \mathbf{s}_i$ and $\mathbf{o}_{ji} = \mathbf{o}_j - \mathbf{o}_i$. Equation~\eqref{eq:generativemodel} is reformulated as:
\begin{align}
    \mathbf{s}_j - \mathbf{s}_i &= (\mathbf{S}\cdot\mathbf{R}\mathbf{o}_j + \mathbf{t}) - (\mathbf{S}\cdot\mathbf{R}\mathbf{o}_i + \mathbf{t}) ,\\
    \mathbf{s}_{ji} &= \mathbf{S}\cdot\mathbf{R}\mathbf{o}_{ji} \quad .
    \label{eq:trans_invariance}
\end{align}

We note that equation~\eqref{eq:trans_invariance} is independent of translation. Taking the L2-norm on both sides for Eq.~\eqref{eq:trans_invariance} and recalling that norm is rotation invariant we get:
\begin{equation}
    \mathbf{||s||}_{ji} = \mathbf{||S||}\cdot\mathbf{||o||}_{ji} \quad .
    \label{eq:rot_invariance}
\end{equation}
The scale $\mathbf{S}$ is estimated by taking the ratio of the axis aligned bounding box (AABB) of the scene and object point clouds, i.e., if $\mathcal{X}_{AABB} = \{ (x_{min}, x_{max}), (y_{min}, y_{max}), (z_{min}, z_{max}) \}$ represents the AABB for a point cloud $\mathcal{X}$, then:
\begin{align}
     \mathbf{S} &= \{ \frac{|x_{max} - x_{min}|_{\mathcal{S}}}{|x_{max} - x_{min}|_{\mathcal{O}}}, \frac{|y_{max} - y_{min}|_{\mathcal{S}}}{|y_{max} - y_{min}|_{\mathcal{O}}} , \frac{|z_{max} - z_{min}|_{\mathcal{S}}}{|z_{max} - z_{min}|_{\mathcal{O}}}    \}
     \label{eq:scale}
 \end{align}
Using the estimated scale and using $\tilde{\mathbf{o}}_{ji} = \mathbf{S}\mathbf{o}_{ji}$ for convenience we are left with a pure rotation to estimate:  
\begin{align}
    \tilde{\mathbf{s}}_{ji} &= \mathbf{R}\tilde{\mathbf{o}}_{ji} \quad .
    \label{eq:trans_scale_invariance}
\end{align}
 We cast the rotation estimation problem into a recursive Bayesian estimation framework and derive a linear state and measurement model. Reformulating Eq.\eqref{eq:trans_scale_invariance} using quaternions we get: 
 \begin{equation}
    \overline{\mathbf{s}}_{ji} = \mathbf{x} \odot \overline{\mathbf{o}}_{ji} \odot \mathbf{x}^{*}, 
    \label{eq:quat_objective}
\end{equation}
where $\mathbf{x}$ is the quaternion form of $\mathbf{R}$, $\odot$ is the quaternion product, ${\mathbf{x}}^{*}$ is the conjugate of $\mathbf{x}$, and $\overline{\mathbf{s}}_{ji}=\{0,\tilde{\mathbf{s}}_{ji}\}$ and $\overline{\mathbf{o}}_{ji}=\{0,\tilde{\mathbf{o}}_{ji}\}$.
Using the matrix form of quaternion product, we can rewrite Eq.\eqref{eq:quat_objective} as:
\begin{align}
    \begin{bmatrix}
        0 & -\tilde{\mathbf{s}}_{ji}^T \\
        \tilde{\mathbf{s}}_{ji} & \tilde{\mathbf{s}}_{ji}^{\times}
    \end{bmatrix}\mathbf{x} -  \begin{bmatrix}
        0 & -\tilde{\mathbf{o}}_{ji}^T \\
        \tilde{\mathbf{o}}_{ji} & -\tilde{\mathbf{o}}_{ji}^{\times}
    \end{bmatrix} \mathbf{x} = \mathbf{0} \\
    \underbrace{\begin{bmatrix}
        0 & -(\tilde{\mathbf{s}}_{ji} - \tilde{\mathbf{o}}_{ij})^T \\
        (\tilde{\mathbf{s}}_{ji} - \tilde{\mathbf{o}}_{ji}) & (\tilde{\mathbf{s}}_j + \tilde{\mathbf{s}}_i + \tilde{\mathbf{o}}_j + \tilde{\mathbf{o}}_i)^{\times}
        \end{bmatrix}_{4 \times 4}}_{\mathbf{H}_t} \mathbf{x} &= \mathbf{0} \quad ,
        \label{eq:expected_measurement}
\end{align}
where $(\ )^\times$ denotes the skew-symmetric matrix formulation. Equation~\eqref{eq:expected_measurement} is of the form $\mathbf{H}_t\mathbf{x} = 0$ where $\mathbf{H}_t$ is the pseudo-measurement matrix~\cite{choukroun2006novel}. We note that Eq.~\eqref{eq:expected_measurement} represents a noise-free state estimation where $\mathbf{H}_t$ depends only on sparse and dense point correspondences which are $\tilde{\mathbf{s}}_{ji}$ and $\tilde{\mathbf{o}}_{ji}$. We design a pseudo-measurement model as $ \mathbf{H}_t \mathbf{x} = \mathbf{z}^h$
% \begin{align}
%     \mathbf{H}_t \mathbf{x} &= \mathbf{z}^h,
%     \label{eq:measurement_model}
% \end{align}
and set $\mathbf{z}^h = 0$. Since we have a static process model, the object does not move and $\mathbf{x}$ and $\mathbf{z}_t$ are Gaussian distributed, 
the state $\mathbf{x}_t$ and covariance matrix $\Sigma^{\mathbf{x}}_{t}$ at each timestep $t$ are computed through a linear Kalman filter. The Kalman filter equations are skipped for brevity and a in-depth derivation is provided in our prior work~\cite{murali2022active}.
As the Kalman filter does not implicitly ensure the constraints on the quaternion as $||\mathbf{x}|| = 1$, we normalise the state and uncertainty after each update step as $\bar{\mathbf{x}}_{t} = \frac{\mathbf{x}_{t}}{||\mathbf{x}_{t}||_2} \quad, \bar{\Sigma}^{\mathbf{x}}_{t} = \frac{\Sigma^{\mathbf{x}}_{t}}{||\mathbf{x}_{t}||_2^2}$. We convert the estimated rotation $\Bar{\mathbf{x}}_t$ to its equivalent rotation matrix $\mathbf{R}$. It used to estimate the translation using the following relation: $\mathbf{t} = \frac{1}{N} \sum_{i=0}^{N} (\Bar{\mathbf{s}}_i - \mathbf{R} \Bar{\mathbf{o}}_i).$
% \begin{equation}
%     \mathbf{t} = \frac{1}{N} \sum_{i=0}^{N} (\Bar{\mathbf{s}}_i - \mathbf{R} \Bar{\mathbf{o}}_i).
%     \label{eq:translation_solution}
% \end{equation}
% \setlength{\columnsep}{1pt}
% \begin{wrapfigure}[18]{r}{0.6\linewidth}
%   \centering
%     \vspace{-0.5cm}
%     % Figure removed
%   \caption{Translation-invariant measurements}
%     % \vspace{-0.5cm}
%   \label{fig:TIMS}
% \end{wrapfigure}
At each iteration, a rotation and translation estimate is found which is used to transform the object point cloud and the process is repeated by re-estimating the correspondence points. The convergence criteria are set by (a) maximum number of iterations or (b) the relative change in estimated pose parameters is less than a predefined threshold ($0.1mm$ and $0.1^o$). 

% the linear Kalman filter equations are given as:
% \begin{align}
%     \mathbf{x}_{t} &= \bar{\mathbf{x}}_{t-1} - \mathbf{K}_t \left( \mathbf{H}_t \bar{\mathbf{x}}_{t-1} \right) \\
%     \Sigma^{\mathbf{x}}_{t} &= \left( \mathbf{I} - \mathbf{K}_t \mathbf{H}_t \right) \bar{\Sigma}^{\mathbf{x}}_{t-1} \\
%     \mathbf{K}_t &= \bar{\Sigma}^\mathbf{x}_{t-1} \mathbf{H}_t^T \left( \mathbf{H}_t\bar{\Sigma}^\mathbf{x}_{t-1} \mathbf{H}_t^T + \Sigma_t^{\mathbf{h}}\right)^{-1}, 
%     \label{eq:kalman_equations}
% \end{align}
% where $\bar{\mathbf{x}}_{t-1}$ refers to the normalized mean of the state at $t-1$, Kalman gain $\mathbf{K}_t$ and $\bar{\Sigma}^{\mathbf{x}}_{t-1}$ is the covariance matrix of the state at $t-1$. 
% The parameter $\Sigma_t^{\mathbf{h}}$ is referred as the measurement uncertainty during time $t$. It is dependent on the state and is provided by~\cite{choukroun2006novel}:
% \begin{align}
%     \Sigma_t^{\mathbf{h}} = \frac{1}{4}\rho\left[ tr(\bar{\mathbf{x}}_{t-1}\bar{\mathbf{x}}_{t-1}^T + \bar{\Sigma}^{x}_{t-1})\mathbb{I}_4 - (\bar{\mathbf{x}}_{t-1}\bar{\mathbf{x}}_{t-1}^T + \bar{\Sigma}^{x}_{t-1} )\right], 
%     \label{eq:choukron}
% \end{align}
% wherein the constant $\rho$ corresponds to the uncertainty of the correspondence measurements and $tr$ refers to trace.


%%%%%%%%%%%%%%%%%%%%%%%%%%%%%%%%%%%%%%%%%%%%%%%%%%%%%%%%%%%%%%%%%%%%%%%
%%%%%%%%%%%%%%%%%%%%%%%%%%%%%%%%%%%%%%%%%%%%%%%%%%%%%%%%%%%%%%%%%%%%%%%
% \subsubsection{Transparent Object Manipulation}
% \label{ssec:tactile_manipulation}
% With the computed 6D pose and estimated CAD model, we design a simple grasping technique in order to grasp and lift the transparent objects. For each \textit{category} of objects, we generated several grasp plans using GraspIt~\cite{miller2004graspit}. Each grasp plan includes the grasp position, orientation and approach vector relative to the model of the object and a grasp quality score. With the pose of the object, the grasp plans are filtered based on kinematic constraints of the robot, workspace limitations and possible collisions with other objects in the scene. Among the remaining grasp plans, the plan with the highest score is chosen and executed. The robot lifts the transparent object and places it in a pre-defined position.
% An online grasp planning and collision avoidance framework is out of the scope of this current work but can be readily integrated into the current framework.

%%%%%%%%%%%%%%%%%%%%%%%%%%%%%%%%%%%%%%%%%%%%%%%%%%%%%%%%%%%%%%%%%%%%%%%
%%%%%%%%%%%%%%%%%%%%%%%%%%%%%%%%%%%%%%%%%%%%%%%%%%%%%%%%%%%%%%%%%%%%%%%

% \subsubsection{Tactile-based Transparent Object Recognition}
% \label{ssec:classification}
% % Figure environment removed
% We use the pretrained encoder model with fixed weights for category-level classification. We employ three fully-connected layers with parameters 512, 256 and $n$ respectively where $n$ represents the number of categories of the objects. Transfer learning is employed to fine-tune the classification network shown in Figure~\ref{fig:framework}(a) on the sparse pointclouds from ShapeNet database.
% During inference, the real sparse tactile pointclouds are used as input to the network for recognition network described in Sec.~\ref{ssec:recog_net}. While the task is challenging, the real-world tactile data are not used during fine-tuning intentionally as collection of large-scale datasets is prohibitively time consuming. The input pointcloud is pre-processed prior to inference by normalising and scaling to fit in $[0,1]^3$ cube to be uniform with the training dataset.
\section{Experiments}
% \haizhou{Follow the same way of introduction as we did in Section2.}
% \noindent In this section, we will introduce datasets and experimental setups that we used. Then we evaluate our method, other self-supervised methods, and supervised methods under different distribution shifts (\ie, concept shifts and covariate shifts) under common settings (\ie, transductive, inductive settings). It has to note that we focus on node-level tasks (\eg, node classification) in this work. As for graph-level tasks, we leave it as our future work and some simple experiments can be found in Appendix~\ref{app:graph_classification}. 
In this section, we first introduce the experimental setup including datasets, training, and evaluation protocol in Section~\ref{sec:dataset}~and~\ref{sec:unsupervised}. 
% Next, we present our experimental setup and conduct extensive experiments to evaluate our method in Section~\ref{sec:unsupervised}. 
We then perform an ablation study to demonstrate the effectiveness of each proposed component in Section~\ref{sec:ablation}. 
Additionally, we analyze the impact of important hyper-parameters in Section~\ref{sec:sensitivity}. 
Subsequently, we integrate our method with various encoding models, showcasing the model-agnostic nature of our recipe in Section~\ref{sec:other_models}. 
Finally, we provide some qualitative results such as feature visualization in Section~\ref{sec:vis}.
It is important to note that we focus on node-level tasks (\eg, node classification) in this work. As for graph-level tasks, we leave it as our future work, while some simple experiments are also provided in Appendix~\ref{app:graph_classification}.

\subsection{Datasets}\label{sec:dataset}
There exist some benchmarks for evaluating graph out-of-distribution generalization~\cite{good,ji2022drugood,gds}. 
Among them, GOOD~\cite{good} is the most representative and comprehensive benchmark that curates more diverse graph datasets with diverse tasks, including single/multi-task graph classification, graph regression, and node classification involving more distribution shifts (\ie, concept shifts and covariate shifts). Hence in this work, we follow the evaluation protocol proposed in \cite{good}. Furthermore, we validate the effectiveness of our method in the datasets (\ie, Amazon-Photo, Elliptic) that are used in EERM~\cite{eerm}. The statistics and detailed introduction to these datasets can be found in Table~\ref{tab:dataset} and Appendix~\ref{app:datasets}.

\begin{table*}[htp]
\caption{The descriptions of datasets. ``Domain-Level'' means splitting by graphs, ``Time-Aware'' denotes splitting according to chronological order.``Word'' and ``Degree'' represent splitting according to word diversity and node degree respectively. ``Language'' means splitting by user language, suggesting the prediction should not be impacted by the language the user use. ``University'' denotes splitting according to the domain university, implying that the prediction of webpages should be based on word contents and link connections rather than university features. ``Color'' means that nodes are split according to node differences in covariate shift and color-label correlations in concept shift.}
\label{tab:dataset}
\centering
\begin{tabular}{cccccccc}
\toprule
Datasets     & Network Type        & \#Nodes & \#Edges & \#Attributes &\#Classes& Train/Val/Test Split     & Metric   \\
% Cora         & Artificial Transformation & 2,703   &         &              &         &                      & Accuracy \\
Amazon-Photo\footnotemark
             & Co-purchasing network      & 7,650   & 119,081   & 755          & 10      & Domain-Level         & Accuracy \\
Elliptic\footnotemark  
             & Bitcoin transactions       & 203,769 & 234,355   & 165          & 2       & Time-Aware           & F1-Score \\
GOOD-Cora    & Scientific publications    & 19,793  & 126,842   & 8,710         & 70      & Word/Degree          & Accuracy \\
% GOOD-Arxiv   & arXiv papers               & 169,343 & 2,315,598 & 128          & 40      & Time/Degree          & Accuracy \\
GOOD-Twitch  & Gamer network              & 34,120  & 892,346   & 128          & 2       & Language             & ROC-AUC  \\
GOOD-CBAS    & A BA-house graph           & 700     & 3,962     & 4             & 4       & Color                & Accuracy \\
GOOD-WebKB   & Webpage network            & 617     & 1,138     & 1,703         & 5       & University           & Accuracy \\
\bottomrule
\end{tabular}
\end{table*}
\footnotetext[5]{This dataset is adopted from~\cite{yang2016revisiting}. \cite{eerm} constructs ten graphs with different environment id’s for each graph.} 
\footnotetext[6]{The original is available on \hyperlink{https://www.kaggle.com/ellipticco/elliptic-data-set}{https://www.kaggle.com/ellipticco/elliptic-data-set}}

\subsection{Unsupervised Representation Learning}\label{sec:unsupervised}
\subsubsection{Transductive Setting}~\label{sec:trans}
% \noindent\textbf{Baselines.}\quad We conduct experiments with 12 baselines which consist of three categories: supervised methods and self-supervised generative methods, self-supervised contrastive methods. Specifically, we compare with three supervised baselines: empirical risk minimization~(ERM)~\cite{erm}, invariant risk minimization (IRM)~\cite{irm}, and a recent proposed graph OOD method dubbed EERM~\cite{eerm}. We also compare various unsupervised node-level representation learning methods: three self-supervised generative methods including GAE~\cite{gae}, VGAE~\cite{gae}, GraphMAE~\cite{gmae} and seven self-supervised contrastive methods: DGI~\cite{dgi}, MVGRL~\cite{mvgrl}, GRACE~\cite{grace}, RoSA~\cite{rosa}, BGRL~\cite{bgrl}, COSTA~\cite{costa}, SwAV~\cite{swav}. The descriptions of these methods can be found in Appendix~\ref{app:baselines}.
In this subsection, we focus on validating our proposed algorithm under the transductive setting, where the test nodes will participate in message passing~\cite{gilmer2017neural} during training following~\cite{good}. 

\noindent\textbf{Baselines.} We conduct experiments with 12 baselines from three categories: (i)~supervised methods, including empirical risk minimization~(\textbf{ERM})~\cite{erm}, invariant risk minimization (\textbf{IRM})~\cite{irm}, and a recent proposed graph OOD method \textbf{EERM}~\cite{eerm}; (ii)~self-supervised generative methods including Graph Autoencoder (\textbf{GAE})~\cite{gae}, Variational Graph Autoencoder (\textbf{VGAE})~\cite{gae}, Self-Supervised Masked Graph Autoencoders (\textbf{GraphMAE})~\cite{gmae}; (iii)~self-supervised contrastive methods including Deep Graph Infomax (\textbf{DGI})~\cite{dgi}, Contrastive Multi-View Representation Learning on Graphs (\textbf{MVGRL})~\cite{mvgrl}, Deep Graph Contrastive Representation Learning (\textbf{GRACE})~\cite{grace}, A Robust Self-Aligned Framework for Node-Node Graph Contrastive Learning (\textbf{RoSA})~\cite{rosa}, Bootstrapped Representation Learning on Graphs (\textbf{BGRL})~\cite{bgrl}, Covariance-Preserving Feature Augmentation for Graph Contrastive Learning (\textbf{COSTA})~\cite{costa}, Unsupervised Learning of Visual Features by Contrasting Cluster Assignments (\textbf{SwAV})~\cite{swav}. The detailed descriptions of these baselines can be found in Appendix~\ref{app:baselines}.

\noindent\textbf{Experimental setup.} We use the same graph encoder across different datasets for a fair comparison following~\cite{good}. We use grid search to find other hyper-parameters (\eg, learning rate, epochs) for different methods. For all experiments, we select the best checkpoints for ID and OOD tests according to results on ID and OOD validation sets following~\cite{good}, respectively. Experimental details and hyper-parameter selections are provided in Appendix~\ref{app:hyper}. For evaluating unsupervised methods, a linear classifier will be built on the frozen trained encoder after finishing pre-training. The reported results are the mean performance with standard deviation after 10 runs following~\cite{good}.

\noindent\textbf{Analysis.}\quad Based on the experimental results listed in Table~\ref{tab:trans_concept} and \ref{tab:trans_covariate}, we can draw the following conclusions: firstly, we find strong self-supervised methods (\eg, GRACE, BGRL, COSTA) are more robust to distribution shifts (concept shift in Table~\ref{tab:trans_concept} and covariate shift in Table~\ref{tab:trans_covariate}) compared to supervised methods. For instance, on GOOD-CBAS and GOOD-WebKB datasets, GRACE surpasses the best supervised method by large margins (over 6\% absolute improvement). Interestingly, we find the methods designed for OOD generalization (\ie, IRM) and graph OOD generalization (\ie, EERM) do not attain superior performance than the standard ERM on most of the datasets. For example, EERM shows superior OOD performance compared to ERM in only one experiment, and IRM outperforms ERM in four out of ten experiments across the conducted evaluations. This phenomenon is also observed in \cite{good,ahuja2020empirical,rosenfeld2021risks}, showcasing the challenge of achieving invariant prediction in non-Euclidean graph settings. 

Furthermore, our method surpasses other SOTA self-supervised methods on the OOD test set of all datasets by a considerable margin while achieving comparable performance in the in-distribution test set. For instance, on small datasets such as GOOD-CBAS and GOOD-WebKB, our method outperforms GRACE\footnote{MARIO is built up on GRACE according to our recipe. So, we make a comparison with GRACE here.} by over 2\% absolute accuracy on the OOD test set. On larger datasets such as GOOD-Cora and GOOD-Twitch, our method still outperforms other methods which shows its superiority. For instance, under covariate shift, MARIO surpasses other methods by over 7\% absolute accuracy on the GOOD-Twitch OOD test set. These statistics prove the effectiveness of our design.


\begin{table*}[htp]
\caption{Experimental results of all methods under concept shift. The bold font means the top-1 performance and the underline represents the second performance across the unsupervised methods. 'ID' represents in-distribution test performance and 'OOD' means out-of-distribution test performance. (OOM: out-of-memory on a GPU with 24GB memory)}
\label{tab:trans_concept}
\centering
\scalebox{0.95}{
\begin{tabular}{l|cc|cc|cc|cc|cc}
\toprule
\toprule
\multirow{3}{*}{concept shift} & \multicolumn{4}{c|}{GOOD-Cora}                   & \multicolumn{2}{c|}{GOOD-CBAS} & \multicolumn{2}{c|}{GOOD-Twitch} & \multicolumn{2}{c}{GOOD-WebKB} \\
                           & \multicolumn{2}{c}{word} & \multicolumn{2}{c|}{degree}& \multicolumn{2}{c|}{color}    & \multicolumn{2}{c|}{language}   & \multicolumn{2}{c}{university} \\
                           & ID         & OOD         & ID          & OOD          & ID            & OOD           & ID             & OOD            & ID            & OOD            \\
\midrule
ERM                        & 66.38±0.45 & 64.44±0.18  & 68.60±0.40  & 60.76±0.34   & 89.79±1.39    & 83.43±1.19    & 80.80±1.00     & 56.92±0.92     & 62.67±1.53    & 26.33±1.09     \\
IRM                        & 66.42±0.41 & 64.29±0.31  & 68.57±0.35  & 61.45±0.24   & 89.64±1.21    & 82.29±1.14    & 78.87±1.04     & 59.30±1.79     & 62.67±1.10    & 26.88±1.42     \\
EERM                       & 65.10±0.44 & 62.45±0.19  & 66.95±0.44  & 56.58±0.25   & 79.07±2.12    & 64.50±1.01    & OOM            & OOM            & 62.50±2.01    & 28.07±3.23      \\
\midrule
% Random-Init                & 37.53±1.74 & 32.12±1.24  & 37.82±1.71  & 27.74±1.14   &               &               &                &                & 60.33±2.21    & 27.07±1.70     \\
GAE                        & 60.65±0.89 & 58.00±0.55  & 62.59±1.11  & 53.44±0.80   & 75.28±1.36    & 68.07±2.05    & 81.25±0.81     & 51.51±1.05     & 62.17±3.34    & 25.78±1.85     \\
VGAE                       & 63.19±0.53 & 60.35±0.47  & 61.65±0.66  & 54.28±0.28   & 76.50±0.50    & 59.07±0.56    & 80.46±0.53     & 55.56±4.53     & 62.50±2.38    & 24.40±2.57     \\
GraphMAE                   & \underline{66.44±0.46} & \underline{64.87±0.30}  & 67.95±0.46  & 59.41±0.39   & 89.14±0.89    & 82.93±0.93    & 80.05±0.64     & 59.38±1.49     & 61.83±3.37    & 29.27±2.15     \\
DGI                        & 63.33±0.56 & 60.71±0.49  & 65.93±1.02  & 55.83±0.53   & 91.22±1.47    & 85.00±1.66    & 80.05±0.87     & 59.16±1.88     & 61.83±2.83    & 28.63±1.92      \\
MVGRL                      & OOM        & OOM         & OOM         & OOM          & 88.57±1.15    & 76.50±1.17    & OOM            & OOM            & 62.00±3.79    & 28.26±4.20     \\
GRACE                      & 65.61±0.61 & 63.92±0.44  & \textbf{68.59±0.35}  & 60.15±0.45   & 92.00±1.39    & 88.64±0.67    & \textbf{83.43±0.63}     & \underline{60.45±1.46}     & 64.00±3.43    & \underline{34.86±3.43}  \\
RoSA                       & 64.06±0.67 & 62.44±0.39  & 67.07±0.65  & 57.68±0.44   & 90.78±2.27    & 85.93±2.14    & 82.39±0.42     & 57.45±2.16     & 64.17±4.10    & 32.20±2.15     \\
BGRL                       & 65.18±0.43 & 63.43±0.45  & 66.83±0.80  & 59.63±0.38   & 92.36±1.16    & 87.14±1.60    & 82.52±0.60     & 55.48±1.48     & 63.67±2.33    & 31.47±3.43     \\
COSTA                      & 65.05±0.80 & 62.37±0.45  & 66.76±0.87  & 55.73±0.36   & \underline{93.50±2.62}    & \underline{89.29±3.11}    & 83.15±0.30 & 55.03±3.22     & 61.66±2.58    & 32.39±2.13 \\
% ArCL                       &            &             & 67.64±0.57  & 59.71±0.44   &               &               &                &                & 65.00±3.94    & 35.41±1.97 \\      
SwAV                       & 62.22±0.53 & 59.79±0.53  & 64.65±0.94  & 55.06±0.39   & 89.00±0.79    & 81.72±0.66    & \underline{83.32±0.15}     & 59.69±1.97     & \underline{65.17±3.76}    & 29.36±2.01    \\
\midrule
MARIO                       & \textbf{67.11±0.46} & \textbf{65.28±0.34}  & \underline{68.46±0.40}  & \textbf{61.30±0.28}   & \textbf{94.36±1.21}    & \textbf{91.28±1.10}    & 82.31±0.54     & \textbf{63.33±1.72}     & \textbf{65.67±2.81}    & \textbf{37.15±2.37}     \\
\bottomrule
\end{tabular}}
\end{table*}

\begin{table*}[htp]
\caption{Experimental results of all methods under covariate shift. The bold font means the top-1 performance and the underline represents the second performance across the unsupervised methods. 'ID' represents in-distribution test performance and 'OOD' means out-of-distribution test performance. (OOM: out-of-memory on a GPU with 24GB memory)}
\label{tab:trans_covariate}
\centering
\scalebox{0.95}{
\begin{tabular}{l|cc|cc|cc|cc|cc}
\toprule
\toprule
\multirow{3}{*}{covariate shift} & \multicolumn{4}{c|}{GOOD-Cora}                                   & \multicolumn{2}{c|}{GOOD-CBAS} & \multicolumn{2}{c|}{GOOD-Twitch} & \multicolumn{2}{c}{GOOD-WebKB} \\
                           & \multicolumn{2}{c}{word} & \multicolumn{2}{c|}{degree}& \multicolumn{2}{c|}{color}    & \multicolumn{2}{c|}{language}   & \multicolumn{2}{c}{university} \\
                           & ID         & OOD         & ID          & OOD          & ID            & OOD           & ID             & OOD            & ID            & OOD            \\
\midrule
ERM                        & 70.50±0.41 & 64.69±0.33  & 72.46±0.49  & 55.53±0.50   & 92.00±3.08    & 77.57±1.29    & 70.98±0.41     & 49.35±5.09     & 39.34±1.79    & 14.52±3.14   \\
IRM                        & 70.48±0.26 & 64.53±0.57  & 71.98±0.34  & 53.72±0.46   & 90.86±2.41    & 78.86±1.67    & 69.81±0.95     & 49.11±2.82     & 38.52±3.30    & 13.97±2.80     \\
EERM                       & OOM        & OOM         & OOM         & OOM          & 65.00±2.57    & 57.43±3.60    & OOM            & OOM            & 46.07±4.55    & 27.40±7.65     \\
\midrule
GAE                        & 56.63±0.79 & 48.93±0.93  & 66.30±0.88  & 34.01±0.87   & 73.00±2.16    & 60.86±3.01    & 67.24±1.23     & 47.65±2.49     & 45.08±6.32    & 28.02±6.29    \\
VGAE                       & 62.02±0.66 & 54.12±0.86  & 69.41±0.57  & 44.20±1.29   & 62.29±2.04    & 63.29±1.11    & 66.99±1.43     & \underline{50.48±4.58}     & 48.85±4.68    & 20.87±6.69     \\
GraphMAE                   & 68.14±0.43 & 64.00±0.33  & \textbf{73.36±0.56}  & 53.75±0.55   & 67.28±3.03    & 67.28±1.49    & 68.84±1.20     & 48.02±2.79     & 48.03±4.34    & 30.00±8.09     \\
DGI                        & 60.85±0.75 & 57.03±0.67  & 68.97±0.41  & 41.75±0.88   & 69.57±4.09    & 59.71±3.43    & 68.43±1.05     & 44.83±1.61     & 48.52±5.04    & 21.11±7.50     \\
MVGRL                      & OOM        & OOM         & OOM         & OOM          & 65.00±1.94    & 64.15±0.77    & OOM            & OOM           & \textbf{54.10±5.39}    & 16.59±6.51     \\
GRACE                      & \underline{68.77±0.33} & \underline{64.21±0.41}  & 72.69±0.34  & \underline{56.10±0.63}   & \underline{93.57±1.83}    & \underline{89.29±3.40}    & \underline{71.12±0.87} & 46.21±1.54 & 49.67±5.82    & 28.10±4.68    \\
RoSA                       & 68.19±0.56 & 62.48±0.61  & 71.04±0.62  & 52.72±0.79   & 84.71±4.14    &79.14±3.51     & 70.58±0.36     & 45.83±1.72     & 52.30±4.24    & \underline{34.24±7.92}     \\
BGRL                       & 67.23±0.43 & 61.33±0.36  & 72.11±0.39  & 49.15±0.73   & 89.00±2.56    & 79.86±3.29    & \textbf{71.43±0.53}     & 43.86±0.94     & 51.80±5.55    & 30.32±7.61    \\
COSTA                      & 65.28±0.60 & 60.33±0.53  & 70.65±0.62  & 54.03±0.28   & 92.29±1.59    & 82.71±2.74    & 69.29±1.37     & 49.07±2.13     & 50.49±3.01    & 29.84±4.75   \\
SwAV                       & 63.29±1.01 & 56.98±0.94  & 70.27±0.73  & 43.00±0.52   & 89.57±1.12    & 81.43±1.69    & 69.19±0.93     & 49.37±2.96     & 49.84±4.82    & 30.55±6.72   \\
\midrule
MARIO                       & \textbf{69.99±0.54} & \textbf{65.06±0.34}  & \underline{72.73±0.43}  & \textbf{57.73±0.45}  & \textbf{94.57±2.46}    & \textbf{91.00±2.48}     & 68.31±0.78 & \textbf{57.37±1.37}     & \underline{53.94±3.23}    & \textbf{35.24±4.98}   \\
\bottomrule
\end{tabular}}

\end{table*}

\subsubsection{Inductive Setting}
In this subsection, we conduct experiments under the inductive settings, where the test nodes are kept unseen during training. This setting is more suitable for domain generalization.
% But we think it is more convincing that conduct experiments under inductive settings which means test nodes are unseen during training. This setting is more appropriate for domain generalization.

\noindent\textbf{Baselines:} For GOOD-WebKB and GOOD-CBAS datasets, we adopt ERM, IRM, GraphMAE, and GRACE as our baselines. And for Amazon-Photo and Elliptic datasets, we select ERM, EERM, and GRACE as our baselines.

\noindent\textbf{Experimental setup:} For GOOD-WebKB and GOOD-CBAS datasets, we use the same model configuration in Section~\ref{sec:trans}.
% Besides, we add experiments on Amazon-Photo dataset~\cite{yang2016revisiting} and Elliptic~\cite{elliptic} dataset in this subsection. 
For Amazon-Photo dataset~\cite{yang2016revisiting} and Elliptic~\cite{elliptic} dataset, they consist of many snapshots (training data and testing data use different snapshots) which are naturally inductive. For Amazon-Photo dataset, we use 2-layer GCN~\cite{gcn} as the encoder and for elliptic dataset, we use 5-layer GraphSAGE~\cite{sage} as encoder following~\cite{eerm}.

% Figure environment removed

\noindent\textbf{Analysis:}
According to Figure~\ref{fig:amazon},\ref{fig:elliptic},\ref{fig:ind_con},\ref{fig:ind_cov}, we can draw following conclusions:
firstly, based on Figure~\ref{fig:amazon}, it is evident that our method outperforms other representative supervised and self-supervised methods on all test graphs (T1$\sim$T8). This superiority is reflected in the larger median value of our method compared to others. For instance, MARIO achieves over a 3\% absolute improvement compared to ERM in terms of the mean value of eight median values. Additionally, our method demonstrates higher stability across different random initializations, as indicated by the closer proximity of the first and third quartile values to the median value~(\eg, the difference of first and third quartile values of ERM, EERM, GRACE and MARIO are 4.2, 3.3, 6.7 and 1.0 on T8 respectively which indicates MARIO is much more stable than other methods). Furthermore, our method exhibits consistent performance across different graphs (\eg, The standard deviation of median values on T1$\sim$T8 for ERM, EERM, GRACE, and MARIO are 0.4, 1.1, 1.2, and 0.3, respectively.), indicating its robustness to environmental variations and its ability to extract invariant features: $g(G^e) \approx g(G^{e'})$ for all $e, e' \in \mathcal{E}^\text{train}$. In summary, our method showcases enhanced OOD generalization capabilities.
% $g(G^e)g(G^e^\prime)$ where $any e, e^\prime in \mathcal{E}^{train}$

Secondly, from the results presented in Figure~\ref{fig:elliptic}, we can observe that our method averagely harvests 10.9\% absolute improvement over GRACE and 12.5\% absolute improvement over EERM in terms of F1 scores on Elliptic dataset. This demonstrates the effectiveness of our method in handling distribution shifts and improving performance compared to existing approaches. It is worth noting that GRACE's performance worsens over time, indicating its inability to handle distribution shifts effectively. In contrast, our method consistently achieves better F1 scores, except for T9, which is caused by the dark market shutdown occurred after T7~\cite{elliptic}. The emergence of such an event introduces significant variations in data distributions, which subsequently results in performance degradation for all methods. Indeed, this event serves as an unpredictable external factor that introduces significant challenges for models trained on limited training data. The results indicate that the performance heavily depends on available training data. Nonetheless, our approach outperforms other methods even in such an extreme case. This highlights the effectiveness of our method in addressing distribution shifts and improving generalization performance.

Finally, based on the observations from Figure~\ref{fig:ind_con} and Figure~\ref{fig:ind_cov} MARIO demonstrates the best performances on both ID and OOD test sets for GOOD-WebKB and GOOD-CBAS datasets, under both concept shift and covariate shift. Notably, MARIO outperforms other methods by more than 3\% and 10\% absolute improvement on GOOD-WebKB and GOOD-CBAS, respectively, under covariate shift. We can draw similar conclusions as discussed in Section~\ref{sec:trans}. Even under the inductive setting, our method continues to demonstrate excellent OOD generalization capabilities and achieves comparable or even improved in-distribution test performance. These statistical results further validate the effectiveness of our method in handling distribution shifts and enhancing generalization performance.

Overall, the observations we have made provide strong evidence of the great capacity of our method for handling distribution shifts, validating its effectiveness and potential for real-world applications.



% Figure environment removed

% Figure environment removed


% Figure environment removed


\subsection{Ablation Studies}\label{sec:ablation}
\noindent Table~\ref{tab:aba} provides a detailed analysis of the effect of each component according to our proposed recipe for improving OOD generalization in graph contrastive learning. Let's examine the different variants of our method and their impact on performance.
Specifically, MARIO~(w/o ad) represents MARIO without  adversarial augmentation. MARIO~(w/o cmi) denotes we only maximize the mutual information between positive pairs without considering conditional mutual information. MARIO~(w/o cmi, ad) means a vanilla graph contrastive method that is similar to GRACE. 

From Table~\ref{tab:aba}, we can find MARIO~(w/o cmi) lags far behind MARIO on OOD test set which demonstrates appropriately minimizing the redundant information (\ie, conditional mutual information) is essential to improve OOD generalization of GCL methods. And adversarial augmentation can also boost OOD generalization because it can approximately serve as a supermum operator to learn more invariant features  discussed in Section~\ref{sec:aug}. Based on the analysis of these variants, it is evident that the proposed improvements on data augmentation and contrastive loss in the recipe are both effective in enhancing graph OOD generalization. Each component contributes to the overall performance improvement, and their combination leads to a stronger self-supervised graph learner in terms of graph OOD generalization. 

In short, the findings from Table~\ref{tab:aba} support the rationale behind your proposed recipe and provide empirical evidence of the effectiveness of each proposed component. By incorporating these enhancements, our method achieves superior performance in handling distribution shifts and improving graph OOD generalization in graph contrastive learning.
\begin{table*}[htp]
\caption{Ablation studies for MARIO by masking each component.}
\label{tab:aba}
\centering
\scalebox{0.9}{
\begin{tabular}{l|cc|cc|cc|cc|cc}
\toprule
\toprule
\multirow{3}{*}{concept shift} & \multicolumn{4}{c|}{GOOD-Cora}                       & \multicolumn{2}{c|}{GOOD-CBAS} & \multicolumn{2}{c|}{GOOD-Twitch} & \multicolumn{2}{c}{GOOD-WebKB} \\
                           & \multicolumn{2}{c}{word} & \multicolumn{2}{c|}{degree}& \multicolumn{2}{c|}{color}    & \multicolumn{2}{c|}{language}   & \multicolumn{2}{c}{university} \\
                           & ID         & OOD         & ID          & OOD          & ID            & OOD           & ID             & OOD            & ID            & OOD            \\
\midrule
MARIO                      & \textbf{67.11±0.46} & \textbf{65.28±0.34}  & \textbf{68.46±0.40}  & \textbf{61.30±0.28}      & \textbf{94.36±1.21}  & \textbf{91.28±1.10}    & 82.31±0.54     & \textbf{63.33±1.72}     & \textbf{65.67±2.81}    & \textbf{37.15±2.37}     \\
MARIO(w/o ad)              & 66.23±0.53 & 64.02±0.18  & 67.88±0.38  & 60.46±0.29   & 93.21±1.25    & 90.29±0.91    & 82.42±0.73     & 60.50±1.02     & 64.83±2.83    & 36.51±3.25    \\
MARIO(w/o cmi)             & 65.32±0.60 & 63.51±0.32  & 68.14±0.32  & 61.19±0.34   & 94.15±1.23    & 90.57±1.96    & \textbf{82.51±0.56}     & 61.41±2.63     & 64.50±4.35    & 35.78±2.53     \\
MARIO(w/o cmi, ad)         & 64.67±0.55 & 63.11±0.32  & 67.95±0.65  & 60.01±0.57   & 93.36±1.66    & 89.64±1.73    & 81.90±0.75     & 60.12±1.60     & 64.17±3.67    & 34.13±2.38     \\
\bottomrule
\end{tabular}}
\end{table*}
% & 65.32±0.60 & 63.51±0.32 exchange 64.67±0.55 & 63.11±0.32
% 68.14±0.32       id ood test: 60.95±0.43       ood ood test: 61.19±0.34


\subsection{Sensitivity Analysis}\label{sec:sensitivity}
\noindent In this subsection, we will analyze some important hyper-parameters of our method. We conduct sensitivity analysis on GOOD-WebKB dataset with concept shift, we chose two sensitive hyper-parameters (\ie, the coefficient $\gamma$ of condition mutual information in Equation~\ref{equ:cmi} and the number of prototypes $|C|$ in Equation~\ref{equ:pq}). The coefficient of CMI range in $[0.001, 0.01, 0.1, 0.5, 1]$ and the number of prototypes $|C|$ ranges in $[10, 50, 100, 200, 300]$. From Figure~\ref{fig:sensitivity}, we can observe that $\gamma$ reaches 0.1 and $|C|$ reaches 100 or 200 can achieve the best OOD test accuracy. Both higher and lower values of $\gamma$ result in suboptimal performance. This finding aligns with previous research such as DIB~\cite{dib}, indicating that an appropriate compression level is crucial for achieving optimal performance. Extremely high or low compression values are not ideal. 

Regarding the number of prototypes $|C|$, based on the results shown in Figure~\ref{fig:sensitivity}, it is found that setting $|C|=100$ leads to the best performance in terms of OOD test accuracy. This choice provides a moderate number of pseudo labels, which is beneficial for the learning process. 

Based on the sensitivity analysis, we determined that setting $\gamma=0.1$ and $|C|=100$ on most datasets. These hyperparameter values strike a balance between compression level and the number of prototypes, resulting in improved graph OOD generalization.
% Figure environment removed


\subsection{Integrated with Other Models}\label{sec:other_models}
% Figure environment removed

\begin{table}[htp]
\caption{Results of different learning approaches with different encoding models (\ie, GCN, GraphSAGE, GAT).}
\label{tab:others}
\centering
\scalebox{0.9}{
\begin{tabular}{cc|cc|cc}
\toprule
\toprule
\multirow{3}{*}{Model}& \multirow{3}{*}{Method} & \multicolumn{2}{c|}{GOOD-CBAS} & \multicolumn{2}{c}{GOOD-WebKB} \\
                & & \multicolumn{2}{c|}{color}    & \multicolumn{2}{c}{university} \\
                &   & ID          & OOD         & ID          & OOD            \\
\midrule
\multirow{3}{*}{GCN} 
&ERM               & 89.79±1.39 & 83.43±1.19  &  62.67±1.53 & 26.33±1.09         \\
&GRACE             & 92.00±1.39 & 88.64±0.67  &  64.00±3.43 & 34.86±3.43        \\
&MARIO             & 94.36±1.21 & 91.28±1.10  &  65.67±2.81 & 37.15±2.37        \\ \bottomrule
\multirow{3}{*}{SAGE} 
&ERM               & 95.07±1.51 & 75.14±1.19  & 73.67±2.08  & 46.33±3.42       \\
&GRACE             & 95.29±1.11 & 74.43±2.36  & 70.50±5.06  & 49.54±3.83        \\
&MARIO             & 96.00±1.07 & 76.29±3.01  & 71.00±3.82  & 51.74±4.63        \\ \bottomrule
\multirow{3}{*}{GAT} 
&ERM               & 78.64±3.63 & 72.93±2.64  & 61.33±3.71  & 28.99±2.63        \\
&GRACE             & 84.57±1.79 & 78.36±1.60  & 59.50±2.36  & 35.78±3.26        \\
&MARIO             & 84.93±1.95 & 80.43±1.89  & 62.17±4.78  & 38.17±3.10        \\
\bottomrule
\end{tabular}}
\end{table}



\noindent In the subsection, we demonstrate the model-agnostic nature of the recipe by integrating it with various graph neural network (GNN) models, including GCN, GraphSAGE, and GAT.

From Table~\ref{tab:others}, it can be observed that regardless of the specific GNN model used as the encoder, our method consistently achieves the best performance on the OOD test set. This indicates the effectiveness and robustness of our method across different GNN models.
By achieving superior performance across different GNN models, MARIO demonstrates its versatility and ability to improve the OOD generalization of various graph neural models. This highlights the broad applicability and effectiveness of our recipe in enhancing the performance of different GNN encoders.

Furthermore, we integrate our recipe with other GCL methods in Appendix~\ref{app:other_methods}. The results demonstrate our recipe can boost the OOD generalization ability of various GCL methods which means our recipe can serve as a plug-in for many current classical GCL methods.

% Figure environment removed

\subsection{Visualization}\label{sec:vis}
\subsubsection{Metric Score Curves}
We present metric score curves for ERM and MARIO, including training, ID validation, ID testing, OOD validation, and OOD testing accuracy, in Figure~\ref{fig:curve2}. Notably, MARIO demonstrates superior convergence with approximately 10\% absolute improvement on the OOD test set compared to ERM. Furthermore, MARIO effectively narrows the performance gap between in-distribution and out-of-distribution performance, showcasing its efficacy in enhancing OOD generalization for graph data. More metric score curves can be found in Appendix~\ref{app:curves}.


\subsubsection{Feature Visualization}
In order to assess the quality of learned embeddings, we adopt t-SNE~\cite{tsne} to visualize the node embedding on GOOD-Cora dataset (concept shift in word domain) using random-init of GCN, EERM, GRACE, and MARIO, where different classes have different colors in Figure~\ref{fig:vis}. For clarity, we select eight classes with the largest number of nodes to enhance the informativeness and interpretability of the visualization. We can observe that the 2D projection of node embeddings learned by MARIO has a better separation of clusters, which indicates the model can help learn representative features for downstream tasks. It has to note that we depict both ID nodes and OOD nodes in the same figure. 

Besides, we also separately visualize ID nodes and OOD nodes in the different figures in the Appendix~\ref{app:feature}. And we can find MARIO performs a clearer separation of clusters whether on ID nodes or OOD nodes compared to other methods.



\section{Conclusion and Future Work}
In this work, I design corruption-robust algorithms for the Lipschitz contextual search problem. I present the \emph{agnostic checking} technique and demonstrate its effectiveness in designing corruption-robust algorithms. There are several open problems for future research. First, in the algorithm I propose for pricing loss, the schedule for agnostic checks is fixed upfront. Can the learner design an adaptive checking schedule for the pricing loss? Second, this work assumes the learner has knowledge of the Lipschitz constant $L$. Can the learner design efficient no-regret algorithms without knowledge of $L$? 

%\begin{appendices}
%    \section{Theoretical Analysis}
Assuming $X=\left\{x_1, x_2, \cdots, x_n\right\}, Y=\left\{y_1, y_2, \cdots, y_n\right\}, A=\left\{a_1, a_2, \cdots, a_n\right\}$ are node attribute set, node label set and node neighbor set respectively. Homophily is essentially a strong relevance between neighbors and labels, that is, neighbors can be determined when labels are known and labels can be determined when neighbors are given. Thus we give the following proposition:

\begin{proposition}\label{proposition:1}
Given a high homophily level graph $G$, conditional entropies $H(A|Y)$ and $H(Y|A)$ between its neighbor set $A$ and node label set $Y$ fulfill:

\begin{equation}
    0 \leq H(A|Y) \leq \epsilon \quad 0 \leq H(Y|A) \leq \epsilon,
\end{equation}

where $\epsilon\rightarrow0+$. 
\end{proposition}

Proposition.~\ref{proposition:1} implies the label of a node can be predicted or determined by the neighbor set of this node, which is similar to the assumption of GNNs. Therefore, the conditional entropies $H(A|Y)$ and $H(Y|A)$ tend to be 0. The proof of this proposition is provided in Appendix.~\ref{proof: proposition 1}. The prediction of GNN-based model $f$ can be represented as:

\begin{equation}
    \hat{Y} = f(X,A),
\end{equation}

which means model $f$ accepts $X$ and $A$ as inputs and outputs the predict label $\hat{Y}$. A well-trained GNN model is able to predict labels very certainly based on the node attributes and neighbors. Since entropy is a measurement of uncertainty, thus we hold:

\begin{proposition}\label{proposition:2}
For a well-trained GNN model $f^{*}$, the conditional entropy $H(Y|f^{*}(X,A))$ between the model output $f^{*}(X,A)$ and label $Y$ fulfill:

\begin{equation}
    0 \leq H(Y|f^{*}(X,A)) \leq \epsilon,
\end{equation}

where $\epsilon\rightarrow0+$. 
\end{proposition}

Proposition.~\ref{proposition:2} assuming a well-trained GNN model can perceive the distributions of attributes $X$ and neighbor $A$, thus learning label-related information and predicting labels accurately. The proof of Proposition.~\ref{proposition:2} is based on cross-entropy loss and is shown in Appendix.~\ref{proof: proposition 2}. Conversely, the target of Meta-attack is inducing the well-trained model $f^{*}$ to give false predictions by removing edges, i.e. perturbing the distribution of neighbor set $A$. After perturbing, $f^{*}$ predicts labels uncertainly, thus we have the following lemma:

\begin{lemma}\label{lemma:1}
    After the attack, the perturbed neighbor set $A^{'}$ meets inequality:
    
    \begin{equation}
        H(Y|X,A) < H(Y|X,A^{'})
    \end{equation}
\end{lemma}

Lemma.\ref{lemma:1} indicates labels can not be predicted or determined via the distributions of $X$ and $A^{'}$. Therefore, the well-trained model $f^{*}$ learns label irrelevance information from distributions and suffers a deterioration performance. The proof of Lemma.\ref{lemma:1} can be found in Appendix.~\ref{proof: lemma 1}.

To against the attack and perturbation, we need to minimize $H(Y|X,A^{'})$ thus the model can learn proper information from the distributions. However, for unsupervised models such as GCL, the label set $Y$ is unknown and results in the impossibility of calculation $H(Y|X,A^{'})$. 

Fortunately, we can utilize strong relevance between neighbor set $A$ and label set $Y$ shown in Proposition.~\ref{proposition:1} to minimize $H(Y|X,A^{'})$ indirectly. Specifically, we select neighbor nodes of each node by minimizing $tr(X^{T}LX)$ and thus minimize conditional entropy $H(X|A^{'})$. The meaning of minimizing $H(X|A^{'})$ is actually maximizing mutual information of $X, A^{'}, Y$:

\begin{lemma}\label{lemma:2}
    Minimizing $H(X|A^{'})$ is equal to maximize $I(X; A^{'}; Y)$.
\end{lemma}

Since minimizing $tr(X^{T}LX)$ is minimizing $H(X|A^{'})$, thus minimizing $tr(X^{T}LX)$ is maximizing mutual information $I(X; A^{'}; Y)$.

Based on Proposition.~\ref{proposition:1}, we further provide Lemma.~\ref{lemma:3}:

\begin{lemma}\label{lemma:3}
    Maximizing mutual information between label $Y$ and perturbed neighbor $A^{'}$: $I(Y; A^{'})$,  is equal to minimizing conditional entropy $H(Y|X,A^{'})$.
\end{lemma}

Lemma.~\ref{lemma:3} implies our target: minimizing $H(Y|X,A^{'})$ can be achieved by maximizing mutual information $I(Y; A^{'})$. Based on Lemma.~\ref{lemma:2} and Lemma.~\ref{lemma:3}, we further introduce Theorem.~\ref{theorem:1}:

\begin{theorem}\label{theorem:1}
    Minimizing conditional entropy $H(X|A^{'})$ is equal to minimizing conditional entropy $H(Y|X,A^{'})$.
\end{theorem}

Proof and more details about the above lemmas and theorem can be found in Appendix.~\ref{proof: lemma 2}, ~\ref{proof: lemma 3}, ~\ref{proof: theorem 1}.

Theorem.~\ref{theorem:1} shows the target of minimizing $tr(X^{T}LX)$ is actually minimizing conditional entropy $H(Y|X,A^{'})$, which is conversely to the attack. In other words, minimizing $tr(X^{T}LX)$ aims at obtaining a clean graph and thus defencing the attack.

%    \begin{comment}
\section{System Architecture}
\label{appendix:architecture}
\system has a novel modularized system architecture with three key components: 
\emph{StreamManager}, 
\emph{TxnManager} and \emph{TxnScheduler}. 
These components are instantiated in each thread locally.
The execution outline of \system is presented in Algorithm~\ref{alg:algo}.
Transactional stream processing is continuous and potentially never ends (Line 1$\sim$8).
The dependency resolution and execution of state transactions are separated into two non-overlapping phases by punctuations~\cite{Tucker:2003:EPS:776752.776780} (Line 2 and 5), which guarantees that no subsequent input event will have a smaller timestamp. 
Effectively, a batch of state transactions is collected during the first phase, and processed during the second phase.

In the first phase (i.e., stream processing phase), 
the \emph{StreamManager} conducts preprocessing for every input event ($e$). Similar to some prior works~\cite{tstream}, state transactions may be issued but not immediately processed during preprocessing (Line 3).
The \emph{pre\_processing} and \emph{post\_processing} functions are exposed as APIs to users.
The \emph{TxnManager} handles dependency resolution (Line 4) among state transactions and insert decomposed operations to construct a \tpg. We discuss the detailed two-phase \tpg construction process in Section~\ref{subsec:construction}.

In the second phase  (i.e., transaction processing phase), 
the \emph{TxnManager} is first involved again to refine (Line 6) the constructed \tpg with further dependency resolution.
The \emph{TxnScheduler} 
schedules operations for concurrent execution based on the constructed \tpg according to the three dimensions of scheduling decisions (Line 7). 
In particular, a scheduling decision model $M$ is instantiated based on the constructed \tpg (Line 14).
\textbf{\circled{1}} Guided by $M$, execution threads adopt an exploration strategy (Section~\ref{subsec:explore}) to explore the constructed \tpg for operations available to be scheduled constrained by dependencies. 
\textbf{\circled{2}} 
During exploration, one or multiple operations may be treated as the 
% basic 
unit of scheduling (Section~\ref{subsec:granularity}). 
Subsequently, \textbf{\circled{3}} every thread executes operation(s) in the unit of scheduling with various abort handling mechanisms (Section~\ref{subsec:abort_handling}).
Only when state transactions are processed (i.e., committed or aborted) can the associated input events be postprocessed (Line 8) by the \emph{StreamManager} based on transaction processing results.
\end{comment}

\begin{comment}
\begin{algorithm}
\footnotesize
    \KwData{$e$ \tcp{Input event}}
    \KwData{$txn_{ts}$ \tcp{State transaction}}
    \KwData{$G$ \tcp{The currently constructed TPG}}
    \While{!finish processing of input streams}{
        \eIf(\tcp*[h]{Phase 1}){\text{$e$ is not a $punctuation$}}{
                $txn_{ts}$ $\gets$ PRE\_Processing($e$)\;
                \textbf{TPG\_Construction}($G$, $txn_{ts}$)\; 
          }(\tcp*[h]{Phase 2}){
                \textbf{TPG\_Refinement}($G$)\; 
                \textbf{TXN\_Scheduling}($G$)\; 
                POST\_Processing()\;
          }
    }
    
    \SetKwFunction{FMain}{TPG\_Construction}
    \SetKwProg{Fn}{Function}{:}{}
    \Fn{\FMain{$G$, $txn_{ts}$}}{
        $O_{1..k}$ $\gets$ \textbf{Partition} $txn_{ts}$\;
        \ForEach{\text{operation $O_{i}$ $\in$ $O_{1..k}$}}{
            \textbf{Identify} its \ld\;
            $G$ $\gets$ $G$ + $O_{i}$ \;
        }
    }
    \SetKwFunction{FMain}{TPG\_Refinement}
    \SetKwProg{Fn}{Function}{:}{}
    \Fn{\FMain{$G$}}{
        \ForEach{\text{vertex $e_{i}$ $\in$ $G$}}{
            \textbf{Identify} its \td, \pd\;
        }
    }
    
    \SetKwFunction{FMain}{TXN\_Scheduling}
    \SetKwProg{Fn}{Function}{:}{}
    \Fn{\FMain{$G$}}{
        $M$ $\gets$ Instantiated with $G$;\tcp{A decision model}
        \While{!finish scheduling of $G$
        }{
          \textbf{\circled{2}} $Scheduling Unit$ $\gets$ \textbf{\circled{1}} \emph{Explore}($G$, $M$)\; 
            \textbf{\circled{3}} \emph{Execute with Abort Handling} ($Scheduling Unit$)\; 
        }
    }
  \caption{Execution Outline of \system}
  \label{alg:algo}
\end{algorithm}
\end{comment}
%\end{appendices}

\clearpage
\bibliographystyle{ieeetr}
\bibliography{citation}

\end{document}
