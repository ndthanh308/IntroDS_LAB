\documentclass{article}
\usepackage[top=1in, bottom=1in, left=1in, right=1in]{geometry}
\usepackage{amsmath}
\usepackage{graphicx}
\usepackage[colorlinks=true, allcolors=gray]{hyperref}
\usepackage{palatino} % font
\usepackage[numbers,sort&compress]{natbib}
\usepackage{bm}
\usepackage{float}
\usepackage{xspace}
\usepackage{xcolor}
\usepackage{caption}
\usepackage[acronym]{glossaries}
\usepackage[bottom]{footmisc} % force footnote to the bottom



\setlength{\parskip}{\baselineskip}

\setlength{\parindent}{0pt}

\renewcommand{\floatpagefraction}{.8}

\graphicspath{{./}{./figures}{./figures_si}}

\newcommand*{\mw}[1]{{\color{magenta} (Mingjian: #1)}}

\newcommand*{\fref}[1]{Fig.~\ref{#1}}
\newcommand*{\tref}[1]{Table~\ref{#1}}
\newcommand*{\eref}[1]{Eq.~\eqref{#1}}
\newcommand*{\sref}[1]{Section~\ref{#1}}
\newcommand*{\aref}[1]{Appendix~\ref{#1}}
\newcommand*{\onlinecite}[1]{Ref.~\citenum{#1}}

\newcommand*{\net}{MatTen\xspace}
\newacronym{si}{ESI}{Electronic Supplementary Information}





\title{An Equivariant Graph Neural Network for the Elasticity Tensors of All Seven Crystal Systems}

\author{
\normalsize{Mingjian Wen$^1$\footnote{Email: mjwen@uh.edu},
Matthew K.\ Horton$^{2,3}$,
Jason M.\ Munro$^2$,
Patrick Huck$^4$,
Kristin A.\ Persson$^{5,6}$
}\\
\footnotesize{$^1$ Chemical and Biomolecular Engineering, University of Houston, Houston, 77204, TX, USA} \\
\footnotesize{$^2$ Materials Sciences Division, Lawrence Berkeley National Laboratory, Berkeley, 94720, CA, USA} \\
\footnotesize{$^3$ Microsoft Research, Redmond, 98052, WA, USA} \\
\footnotesize{$^4$  Energy Technologies Area, Lawrence Berkeley National Laboratory, Berkeley, 94720, CA, USA } \\
\footnotesize{$^5$  Molecular Foundry, Lawrence Berkeley National Laboratory, Berkeley, 94720, CA, USA} \\
\footnotesize{$^6$ Department of Materials Science and Engineering, University of California, Berkeley, Berkeley, 94720, CA, USA}
\vspace{-8mm}
}



\date{}

\begin{document}
\maketitle


\begin{abstract}

    The elasticity tensor is a fundamental material property that describes the elastic response of a material to external force.
    The availability of full elasticity tensors for inorganic crystalline compounds, however, is limited due to experimental and computational challenges.
    Here, we report the materials tensor (\net) model for rapid and accurate estimation of the full fourth-rank elasticity tensors of crystals.
    Based on equivariant graph neural networks, \net satisfies the two essential requirements for elasticity tensors: independence of the frame of reference and preservation of material symmetry.
    Consequently, it provides a unified treatment of elasticity tensors for all seven crystal systems across diverse chemical spaces, without the need to deal with each separately.
    \net was trained on a dataset of first-principles elasticity tensors garnered by the Materials Project over the past several years (we are releasing the data herein) and has broad applications in predicting the isotropic elastic properties of polycrystalline materials, examining the anisotropic behavior of single crystals, and discovering new materials with exceptional mechanical properties.
    Using \net, we have discovered a hundred new crystals with extremely large maximum directional Young's modulus and eleven polymorphs of elemental cubic metals with unconventional spatial orientation of Young's modulus.

\end{abstract}




\section{Introduction}
\label{sec:intro}





All materials exhibit elastic behavior under small external loads and return to their original shape upon the release of these loads \cite{hetnarski2016mathematical}.
The elasticity tensor provides a fundamental and complete description of a material's response to such loads.
It offers a lens to examine the inherent strength of the bonding in a material and enables the understanding, analyzing, and designing of many macroscopic physical properties of materials.
Commonly employed mechanical properties (for instance, Young's modulus and Poisson's ratio), thermal properties (for instance, thermal conductivity) and acoustic properties (for instance, sound velocity) can be mathematically derived from the elasticity tensor.
These properties have long been leveraged, for example, by materials scientists to search for ultrahard materials \cite{kaner2005designing,mansouri2018machine} and by geophysicists to interpret seismic data \cite{anderson1968some,karki2001high}.
More recently, the anisotropic elastic behavior of inorganic solid electrolytes has been found to play a decisive role in determining the stability of electrodeposition at the interfaces in solid-state batteries \cite{monroe2004effect,ahmad2017stability}.
Moreover, in solid-state synthesis, one would want to utilize the elasticity tensor to determine the local stability of a material, so as to avoid unsuccessful synthesis of materials that spontaneously transform into different structures \cite{mouhat2014necessary,tolborg2022free}.


In spite of the importance, elasticity tensor data from experimental measurement is limited to a small number of materials.
For example, for inorganic crystalline compounds, experimental data is only on the order of a few hundred, considering entries both tabulated in handbooks and scattered in individual papers \cite{de2015charting}.
The major difficulty lies in preparing large enough single crystals for accurate experimental measurement using current techniques such as resonant acoustic spectroscopy \cite{du2017facile}.
In the past decade, efficient and reliable electronic structure calculation methods such as density functional theory (DFT) \cite{lejaeghere2016reproducibility} with automation frameworks \cite{ong2013python, jain2015fireworks, mathew2017atomate} have enabled high-throughput computational investigation of materials.
Using this approach in the Materials Project \cite{jain2013commentary}, we produced an initial dataset of elasticity tensors for 1181 crystals in 2015 \cite{de2015charting}, which has been expanded over time to 10276, which we now release as a new dataset with this work.
Nevertheless, this only accounts for 6.6\% of the more than 154000 crystals in the Materials Project, let alone the even greater number of crystals recorded in the Inorganic Crystal Structure Database (ICSD) \cite{hellenbrandt2004inorganic} and predicted by universal interatomic potentials \cite{chen2022universal}.


It is therefore no surprise that machine learning (ML) has gathered substantial interest as a means to develop efficient surrogate models for the prediction of elastic properties.
In a nutshell, state-of-the-art ML models for elastic properties encode compositional information \cite{wang2021compositionally,dunn2020benchmarking,de2021materials} and/or structural information \cite{dunn2020benchmarking,de2021materials,chen2019graph,choudhary2021atomistic} in a material as feature vectors and then map them to a target using some regression algorithms.
This approach is adopted in many existing works for learning elastic properties of, e.g., alloys \cite{mukhamedov2021machine,vazquez2022efficient,linton2023machine, pasini2023graph} and polycrystals \cite{karimi2023prediction, hestroffer2023graph}.
They are, however, limited to derived scalar elastic properties such as bulk modulus and shear modulus, and separate models are built for each derived property.
Ideally, one would hope to predict the full elasticity tensor, from which all other elastic properties can be derived.
Along this direction, there have been attempts to predict individual tensor components \cite{ahmad2018machine, revi2021machine}.
These models are great first steps, but essentially they still predict separate scalars in a specific coordinate system and thus unavoidably violate the transformation rules of tensors.

Geometric machine learning \cite{bronstein2021geometric} such as equivariant graph neural networks (GNNs) \cite{thomas2018tensor,satorras2021n,batzner2022e3,takamoto2022towards,equiformer_v2} and equivariant kernel methods \cite{grisafi2018symmetry, veit2020predicting} can directly operate in the space of high-rank tensors and adhere to their transformation rules.
The main use case is still for scalar molecular and materials properties, but a couple of works have explored the application in predicting tensorial properties such as the molecular dipole moment \cite{pmlr-v139-schutt21a,veit2020predicting}, magnetic moment \cite{li2023deep}, and dielectric response \cite{grisafi2018symmetry}.
Other applications that do not directly predict final tensorial targets have also successfully taken advantage of internal tensorial representations to learn scalar fields such as molecular electron density \cite{unke2021se, rackers2023recipe}.
Although these efforts focus on scalars or low-rank tensors, they demonstrate the viability of direct machine learning of the full fourth-rank elasticity tensor.

In this work, we develop the Materials Tensor (\net) model for a rapid and accurate estimate of the fourth-rank elasticity tensors of inorganic compounds.
Our model, based on equivariant GNNs, takes a crystal structure as input and outputs its full elasticity tensor with all symmetry-related transformation rules automatically satisfied.
Other elastic properties such as bulk modulus and shear modulus can then be derived from the full elasticity tensor.
The model satisfies two essential symmetry requirements for elasticity tensors:
\emph{independence of the frame of reference}, meaning that the choice of a specific coordinate system does not affect the model output, and \emph{preservation of material symmetry}, meaning that the symmetry in a crystal is captured and reflected in the output elasticity tensor.
The capabilities of \net are demonstrated via the study of both isotropic and anisotropic elastic properties.
Using MatTen, we screened the Materials Project database for the identification of materials with a large maximum directional Young's modulus.
On average, the values of the newly found materials are more than three times larger than existing data, as verified by first-principles calculations.
In addition, we have identified 11 unconventional polymorphs of elemental cubic metals whose maximum directional Young's moduli are in the $\langle 100 \rangle$ directions.



\section{Results and Discussion}
\label{sec:results}


\subsection{Symmetry and irreducible representation of the elasticity tensor}
\label{sec:et:sym}

% Figure environment removed

The elasticity tensor $\bm C$ is a fourth-rank tensor that fully characterizes the elastic behavior of a material.
Given that it is the second derivative of the total elastic energy with respect to the strain tensor and that the strain tensor is symmetric \cite{fedorov1968theory,nye1985physical}, the elasticity tensor possesses major symmetry $C_{ijkl} = C_{klij}$ and minor symmetry $C_{ijkl}= C_{jikl} = C_{ijlk}$ (in indicial notation, where $i,j,k,l \in \{1, 2, 3\}$).
Consequently, only 21 of the 81 components of $\bm C$ are independent.
It is convenient to write the elasticity tensor in a contracted matrix $\bm c$ ($c_{\alpha\beta}$, where $\alpha,\beta \in \{1,2,\dots,6\}$) with a pairs of indices $ij$ in the tensor notation replaced with a single index $\alpha$ in the matrix notation:
$11\rightarrow1$;
$22\rightarrow2$;
$33\rightarrow3$;
$23, 32\rightarrow4$;
$13, 31\rightarrow5$;
and
$12, 21\rightarrow6$.
This Voigt matrix \cite{voigt1910lehrbuch} is a $6 \times 6$ matrix symmetric about the main diagonal, reflecting the fact that the elasticity tensor has 21 independent components.

The intrinsic material symmetry of a crystal can further reduce the number of independent components \cite{nye1985physical,mouhat2014necessary}.
For example, copper is a cubic crystal with mirror planes and three-fold rotation axes (point group m$\bar3$m), and such symmetry results in a number of only three independent components.
Formally, the material symmetry imposes the following constraints on the elasticity tensor \cite{forte1996symmetry,tadmor2012continuum}:
\begin{equation}\label{eq:mat:sym}
    C_{ijkl} = Q_{ip} Q_{jq} Q_{kr} Q_{ls} C_{pqrs},
\end{equation}
where $\bm Q \in G \subset SO(3)$, and $G$ is the material symmetry group of the crystal, which is a subgroup of the rotation group $SO(3)$.
An interesting question is: how many unique symmetry classes exist and what is the number of independent components in each class?

It turns out that there exists only eight distinct classes (\fref{fig:tensor:components}), proved via a purely algebraic approach by directly identifying the equivalence classes corresponding to \eref{eq:mat:sym} \cite{forte1996symmetry}.
Of the eight classes, one is for isotropic materials, and each of the other seven corresponds to a crystal system \cite{chadwick2001new,tadmor2012continuum}.
In our opinion, there is still significant confusion on this topic.
For example, the categorization by Wallace \cite{wallace1972thermodynamics} and populated by Nye~\cite{nye1985physical}, which incorrectly gives two unique classes for each of the tetragonal and trigonal cases (Fig.~S1 in the \gls{si}), is still widely cited in recent works \cite{singh2021mechelastic,li2022elast,ran2023velas}.
We refer to Section 6.5 of \onlinecite{tadmor2012continuum} for a historical note on the development of the categorization.


The Voigt matrix provides a visual way to observe the material symmetry of a crystal reflected in the elasticity tensor (\fref{fig:tensor:components}).
The values of the matrix components, however, depend on the choice of the coordinate system and do not show any systematic pattern upon coordinate transformation \cite{itin2013constitutive,itin2020irreducible}, making it difficult to build predictive models for elasticity tensors.
This can be overcome by the \emph{harmonic decomposition} \cite{backus1970geometrical}, where the space of all elasticity tensors is factored into the direct sum of irreducible representations of $SO(3)$.
Consequently, any elasticity tensor can be written in the form,
\begin{equation}\label{eq:C:decomp}
    \bm C = h_1 (\lambda)  + h_2(\eta) + h_3 (\bm A) + h_4 (\bm B) +  h_5(\bm H) ,
\end{equation}
where $\lambda$ and $\eta$ (scalars) are the \emph{isotropic} part, $\bm A$ and $\bm B$ (second-rank traceless symmetric tensors) are the \emph{deviatoric} part, and $\bm H$ (fourth-rank traceless symmetric tensor) is the \emph{harmonic} part \cite{backus1970geometrical}.
The harmonic decomposition has two advantageous characteristics.
First, for a given $\bm C$, the values of $\lambda, \eta,\bm A, \bm B, \bm H $ and the functions $h_1, \dots, h_5$ are uniquely determined \cite{backus1970geometrical,forte1996symmetry} (see \gls{si} for their expressions).
Second, each part in \eref{eq:C:decomp} transforms in a known manner with respect to $SO(3)$ operations, enabling the construction of predictive models that leverage recent advancements in geometric deep learning.


\subsection{Equivariant graph neural networks for high-rank tensors}


% Figure environment removed


The \net model captures the structure--property relationship of crystalline materials.
It takes a crystal structure as input, represents it as a three-dimensional crystal graph, performs feature updates on the crystal graph, and finally outputs a tensor property of the material built according to the reverse process of the harmonic decomposition in \eref{eq:C:decomp}.


In the GNN model (\fref{fig:schematic}), the input crystal is represented as a graph  $\mathcal{G} (V, E)$,
with atoms as the nodes $V$ and bonds as the edges $E$.
The feature $F_i \in V$ characterizes atom $i$, and the initial value of $F_i$ is obtained by encoding the atomic number $Z_i$ using a one-hot scheme.
A bond/edge between two atoms is created if the distance $\| \vec{\bm r}_{ij} \| $ is smaller than a cutoff value, where $\vec{\bm r}_{ij}$ denotes the distance vector between atoms $i$ and $j$.
Periodic boundary conditions are considered when constructing the bonds, using super cell vectors.
The distance vector $\vec{\bm r}_{ij}$ is separated into two parts:
the unit vector  $\hat{\bm r}_{ij}$ from atom $i$ to atom $j$ and the scalar distance $r_{ij}$ between them.
The former is expanded on real spherical harmonics $Y_m^{l}$, and the latter is
expanded on the Bessel radial basis functions \cite{gasteiger2020dimenet}.
In sum, these embedding modules extract structural information  (coordinates of atoms, atomic numbers, and super cell vectors)  from the crystal and provide them to the interaction blocks.

The interaction blocks iteratively refine the atom features via convolution operations.
The architecture of the interaction block follows the design of Tensor Field Network \cite{thomas2018tensor} and NequIP \cite{batzner2022e3}.
Unlike many existing GNNs for molecules and crystals \cite{xie2018crystal,chen2019graph,wen2020bondnet,wen2022rxnrep} that utilize scalar features, here, the atom feature $F_i$ is a set of scalars, vectors, and high-rank tensors.
Formally, it is a geometric object consisting of a direct sum of irreducible representations of the $SO(3)$ rotation group \cite{thomas2018tensor,batzner2022e3}.
There are two major benefits of using such geometric features.
First, they are incorporated as inductive bias which can improve model accuracy and reduce the amount of training data.
Second, from them, one can easily construct other physical tensors such as the elasticity tensor in this work.

The convolution on these geometric objects is an equivariant function, meaning that if the input atom feature $F$ to the convolution is transformed under a rotation in $SO(3)$, the output is transformed accordingly.
This is achieved via the tensor product convolution by updating the atom feature in the $(k+1)_\mathrm{th}$ interaction block from that in the $k_\mathrm{th}$ interaction block,
\begin{equation}\label{eq:convolution}
    F_i^{k+1} =  \sum_{j\in \mathcal{N}_i} R(r_{ij}) Y(\hat{\bm r}_{ij})\otimes F_j^{k},
\end{equation}
where $\mathcal {N}_i$ denotes the set of neighboring atoms for atom $i$, $R$ indicates a multilayer perceptron (MLP) on the radial basis expansion of $r_{ij}$, and $Y$ indicates the spherical harmonics expansion of $\hat{\bm r}_{ij}$.
The tensor product $\otimes$ between $Y$ and $F_j^k$ is a bilinear map, which is a generalization of the outer product of two vectors.
The product output is decomposed back onto the irreducible representations, and the entire operation is equivariant
\cite{thomas2018tensor}.
The use of an MLP makes the convolution learnable.
After a skip connection \cite{he2016deep}, the feature $F_i^{k+1}$ is passed through a nonlinear activation function and finally normalized using an equivariant normalization function \cite{e3nnpaper}.

The output head maps the refined features from the interaction blocks to the target materials tensor of interest.
First, the features of all atoms are aggregated to obtain a representation of the crystal.
For intensive properties such as the elasticity tensor, meaning that the property value does not depend on the size of the system, we adopt the mean pooling by averaging the features such that the representation of the crystal is independent of the number of atoms.
Next, an appropriate subset of the pooled irreducible representations that correspond to the target tensor of interest is selected and then assembled as the model output.
For the elasticity tensor, the selected ones are two scalars, two second-rank traceless symmetric tensors, and a fourth-rank harmonic tensor.
They are assembled to an elasticity tensor according to \eref{eq:C:decomp}.

Overall, \net is a function $\bm C = f(x)$ that maps a crystal structure $x$ to its elasticity tensor $\bm C$.
The function $f$ is equivariant to the $SO(3)$ group transformation, that is,
for any $x$ and $g \in  SO(3)$, we have $ D_y(g) f(x) = f( D_x(g) x) $, where $D_x(g)$ and $D_y(g)$ are rotation matrices parameterized by $g$ for the crystal structure and the elasticity tensor, respectively.
This ensures that the model can produce an elasticity tensor $\bm C$ that respects the orientation of the input crystal structure.
In other words, the choice of a specific coordinate system does not affect the model output; if the coordinate system is rotated, the output tensor rotates accordingly.
This \emph{independence of the frame of reference} characteristic is an indispensable property for models that predict tensors.
In addition, any such model should also \emph{preserve the material symmetry} of the crystal.
By construction, \net guarantees the material symmetry reflected in the elasticity tensor.
Concretely, if the predicted elasticity tensor is represented as a Voigt matrix, the symmetry and number of independent components in \fref{fig:tensor:components} are automatically maintained for all seven crystal systems (proof in the \gls{si}).
For example, for a cubic crystal, the model guarantees that there are only three independent components $c_{11}=c_{22}=c_{33}$, $c_{12}=c_{13}=c_{23}$, and $c_{44}=c_{55} =c_{66}$ and that all other components are zero.



\subsection{Elastic properties of polycrystals}
\label{sec:scalar:prop}

% Figure environment removed

The \net model directly outputs the full elasticity tensor.
To assess its performance, we computed several commonly used elastic properties for polycrystals from the elasticity tensor.
\fref{fig:error:scalar} illustrates the results on the moduli obtained using the Hill average scheme \cite{hill1952elastic}.
The DFT reference values have a range of  $4\sim442$~GPa,  $3\sim375$~GPa, and $9\sim878$~GPa for the bulk modulus $K$, shear modulus $G$, and Young's modulus $E$, respectively.
The formulas to obtain the moduli are given in \sref{sec:methods}, and their statistics and distribution are given in Figs.~S4--S7 in the \gls{si}.
The predictions of \net closely align with the DFT reference values along the entire ranges, achieving mean absolute errors (MAEs) of 7.37~GPa, 8.38~GPa, and 20.59~GPa for $K$, $G$, and $E$, respectively.
To connect the MAE values to practical applications, let's examine the error in strain caused by the MAE in Young's modulus.
For example, at $E=128.4$~GPa (the mean of DFT references), an error of $20.59$~GPa will lead to a relative error of 19\% in the strain (calculation given in the \gls{si}).
While different applications necessitate varied accuracy, a relative error of 19\% in the strain can be acceptable, given that noncontact experimental techniques for strain measurement such as the digital image correlation method \cite{reu2018dic} have a typical error of $\sim$10\%.


For comparison, we trained two additional models.
The first is a variant of \net, where the tensor output head of \net is replaced with a scalar output head, referred to as MatSca hereafter.
The second is the AutoMatminer algorithm, an automated machine learning pipeline designed for predicting scalar materials properties \cite{dunn2020benchmarking}.
We evaluated their performance in predicting the elastic moduli, and the results are listed in \tref{tab:mae:mad}.
Both \net and MatSca have smaller MAEs than AutoMatminer across all three moduli, owning to the effectiveness of the underlying neural networks in learning materials properties from structures.
The performance of \net and MatSca are comparable.
However, it is worth highlighting that while an individual MatSca model was trained for each modulus, a single \net model successfully produced all the elastic moduli, demonstrating the versatility of the \net model.

Upon closer examination of \fref{fig:error:scalar}A--C, we have identified some inconsistencies in the predictions.
All crystals in the DFT dataset have positive moduli, the predicted moduli by \net, however, occasionally yield negative values, indicating that the associated crystal is elastically unstable.
This is an inherent challenge faced by machine learning regression models in general, albeit with physical inductive biases embedded in the model such as the symmetry requirements in \net.
The number of crystals with negative predicted moduli remains minimal, accounting for only 3, 2, and 2 out of the 1021 test data for bulk, shear, and Young's moduli, respectively.
The moduli alone, however, do not provide a comprehensive understanding.
For a crystal to be elastically stable, the sufficient and necessary condition is that the Voigt matrix should be positive definite \cite{mouhat2014necessary}.
We checked this and found that 25 crystals in the test set do not satisfy this condition.
The majority of them are due to the incorrect prediction of the relative magnitudes of the diagonal and off-diagonal components of the Voigt matrix.
A breakdown of the errors is provided in the \gls{si}.
Nevertheless, this is not a concern in practical use; one can filter out the negative ones if desired.


\begin{table}
    \caption{Performance of the models in predicting the bulk, shear, and Young's
        moduli.
        The value in a pair of parentheses is the standard deviation obtained from an ensemble of five models trained with different initializations.
        MAE: mean absolute error; MAD: mean absolute deviation.
    }
    \label{tab:mae:mad}
    \centering
    \begin{tabular}{@{\extracolsep{5pt}}ccccccc}
        \hline
                     & \multicolumn{2}{c}{$K$ (GPa)}
                     & \multicolumn{2}{c}{$G$ (GPa)}
                     & \multicolumn{2}{c}{$E$ (GPa)}                                                                              \\
        \cline{2-3} \cline{4-5} \cline{6-7}
                     & MAE                           & MAE/MAD       & MAE         & MAE/MAD       & MAE          & MAE/MAD       \\
        \hline
        \net         & 7.37 (0.10)                   & 0.130 (0.002) & 8.38 (0.16) & 0.280 (0.005) & 20.59 (0.35) & 0.275 (0.005) \\
        MatSca       & 7.32 (0.09)                   & 0.129 (0.002) & 8.63 (0.07) & 0.288 (0.002) & 19.87 (0.43) & 0.265 (0.006) \\
        AutoMatminer & 9.84 (0.34)                   & 0.174 (0.006) & 9.27 (0.32) & 0.309 (0.011) & 22.10 (0.77) & 0.295 (0.024) \\
        \hline\end{tabular}
\end{table}


To assess how \net performs for different elastic properties as well as for different crystal systems, we computed the scaled error, $\text{SE} = \text{MAE}/\text{MAD}$, in which MAE and MAD are the mean absolute error and mean absolute deviation, respectively (see \sref{sec:methods}).
MAD quantifies the distance of reference values to their mean, and larger MAD means the reference values are more scattered.
A model that makes accurate predictions for each data point will have an SE of 0, and a model that always predicts the mean of the dataset will have an SE of exactly 1.
Comparing between properties, we see from \fref{fig:error:scalar}d that the SE of bulk modulus is smaller than those of shear modulus and Young's modulus across all crystal systems.
This suggests that bulk modulus is easier to predict, in agreement with existing observations \cite{chen2019graph,dunn2020benchmarking,wang2021compositionally,de2021materials}.
Next, we compare between crystal systems.
The dataset used for model development has an uneven distribution in terms of the number of materials in each crystal system (Fig.~S2 in the \gls{si}).
For example, it contains 4217 cubic crystals, fewer than 800 trigonal and monoclinic crystals, and only 60 triclinic crystals.
As a result, the SE and the error bar of the predicted moduli are larger for trigonal, monoclinic, and triclinic crystals in general (\fref{fig:error:scalar}d).
Despite the slightly higher errors, it is notable that the model can still perform well for the crystal systems with a low presence in the training data, particularly for triclinic crystals.
This is primarily because \net internally treats all crystals the same, enabling crystal systems with fewer data to leverage the abundant data from other crystal systems and acquire enhanced representations.
This type of transferability is not possible with models that are built separately for each crystal system.

The elastic moduli have values across different orders from near zero to hundreds (Figs.~S4--S7 in the \gls{si}).
To mitigate the challenge of learning values across a broad spectrum, some existing models  \cite{dunn2020benchmarking,chen2019graph,wang2021compositionally,de2021materials}
adopt the approach of predicting the logarithm of the moduli.
Unlike these models, \net directly predicts the full tensor without any data transformation, and all other elastic properties (including their logarithms) can be computed from it.
The logarithms of the bulk, shear, and Young's moduli obtained from \net are comparable to those that learn in the logarithmic space
(Table~S1 in the \gls{si}).
Further, the performance of \net can be further improved with additional training data.
The MAE almost decreases linearly with the logarithm of the number of data used to train the model (Fig.~S10 in the \gls{si}).
Finally, it is also possible to predict the full elasticity tensor by separately modeling its non-zero independent components \cite{ahmad2018machine, revi2021machine}.
\net performs much better than this approach thanks to its ability to
deal with all crystal systems within a united framework (further discussion in the \gls{si}).



\subsection{Anisotropic elastic properties}
\label{sec:3d:prop}


% Figure environment removed


Crystals are inherently anisotropic, and thus their elastic properties can vary depending on the direction of measurement.
This anisotropy arises from a crystal's structure, including the symmetry of the lattice and the arrangement of the atoms.
\net predicts the full elasticity tensor, and, for the first time with a machine learning model, we are able to investigate the anisotropic elastic behavior of crystals.
We focus our discussion on the directional dependence of Young's modulus (further results on shear modulus given in the \gls{si}).


Young's modulus $E$ discussed in \sref{sec:scalar:prop} is an averaged property for isotropic polycrystals.
But for single crystals, Young's modulus depends on the direction along which the strain/stress is applied and measured.
Given the elasticity tensor $C_{ijkl}$ (equivalently, the compliance tensor $S_{ijkl}$, see \sref{sec:methods}), the directional Young's modulus can be computed as \cite{nye1985physical,ran2023velas}
\begin{equation}\label{eq:Ed}
    E_\text{d} ({\bm n})  =(n_i n_j n_k n_l S_{ijlk})^{-1},
\end{equation}
where $\bm n$ is a unit vector that specifies the direction of measurement.
The direction dependence of Young's modulus can be visualized with a three-dimensional plot (\fref{fig:3d:E}a).
Interactive visualization can be obtained via, for example, the \verb|elate| package \cite{gaillac2016elate}.
Alternatively, via a spherical coordinates transformation:
$\bm n = [\sin\theta\cdot\cos\varphi, \sin\theta\cdot\sin\varphi, \cos\theta]$ (\fref{fig:3d:E}c), it can be represented in two dimensions (\fref{fig:3d:E}d, with a Robinson map projection \cite{robinson1974new}).
Such plots make it easier to visually investigate the anisotropic characteristics of Young's modulus.
For example, for the cubic rocksalt CaS crystal (\fref{fig:3d:E}b), the maximum directional Young's modulus $E_\text{d}^\text{max}$ is along the $\langle 100 \rangle$ directions (for instance, $\theta = 90^\circ$ and $\varphi = 0^\circ$),
while the minimum $E_\text{d}^\text{min}$ is along the the $\langle 111 \rangle$ directions (for instance, $\theta = 54.7^\circ$ and $\varphi = 45^\circ$).
In fact, for cubic crystals such as CaS, the extreme values of $E_\text{d}$ are guaranteed to occur in these two directions \cite{nye1985physical}.
\eref{eq:Ed} can be simplified as $E_\text{d}(\bm n) = [S_{1111} - 2(S_{1111} - S_{1122} - 2S_{2323})(n_1^2n_2^2 + n_2^2n_3^2 + n_3^2n_1^2)]^{-1}$ for cubic crystals, expressed in terms of their three independent elasticity tensor components.
It can be mathematically shown that, if
\begin{equation}\label{eq:E:max:dir}
    S_{1111} - S_{1122} - 2S_{2323} < 0,
\end{equation}
Young's modulus achieves its maximum $E_\text{d}^\text{max}$ in the $\langle 100 \rangle$ directions and minimum in the $\langle 111 \rangle$ directions;
otherwise, if $S_{1111} - S_{1122} - 2S_{2323} > 0$, the two extremes switch directions (derivation given in the \gls{si}).
\eref{eq:E:max:dir} is satisfied by CaS, and thus we observe the maximum and minimum in the $\langle 100 \rangle$ and $\langle 100 \rangle$ directions, respectively.
In addition, $E_\text{d}$ of CaS possesses symmetry (for example, 3-fold rotational axis along the cube diagonals) consistent with a cubic crystal, further confirming that the predicted elasticity tensor \emph{preserves material symmetry}.


To quantitatively assess the ability of \net in predicting anisotropic elastic properties, we measured the error between the model predicted directional Young's modulus $E_\text{d}^\text{pred}$ and the DFT reference $E_\text{d}^\text{ref}$.
For CaS, \net prediction closely follows DFT reference,
with a maximum under-prediction of $8.7$~GPa along the $\langle 111 \rangle$ directions and a maximum over-prediction of 9.4~GPa along the $\langle 100 \rangle$ directions (Fig.~S14 in the \gls{si}).
In addition to this example crystal, we calculated the error for the entire test set, computed as $D = D_s \cdot D_v$ for each crystal.
The value of the error is $ D_v = \int_\theta\int_\varphi  \vert \Delta \text{E}_d \vert \, \text{d}\theta \text{d}\varphi$, where $ \Delta E_d (\theta, \varphi) = E_\text{d}^\text{pred} (\theta, \varphi)  - E_\text{d}^\text{ref} (\theta, \varphi)$ and $\vert\cdot\vert$ denotes the absolute value.
The sign of the error is $D_s=+1$ if$  \int_\theta\int_\varphi  \Delta \text{E}_d  \, \text{d}\theta \text{d}\varphi > 0 $, and $D_s = -1$ otherwise.
Put differently, the value $D_v$ quantifies the average deviation from the DFT reference, while the sign $D_s$ characterizes whether the overall prediction is larger than the DFT reference. The integration over $\theta$ and $\varphi$ is performed using the Chebyshev quadratures, which uniformly distribute the integration points on the sphere and can avoid biasing specific points \cite{beentjes2015quadrature}.
The distribution of the prediction error $D$ of $E_\text{d}$ for the test set is plotted in \fref{fig:3d:E}e.
It has a Gaussian-like shape, and it almost overlaps with that of isotropic Young's modulus obtained using the Hill average.
Similar behaviors are observed for shear modulus and MAEs by crystal systems (Figs.~S15 and S16 in the \gls{si}).
These observations suggest that, on average, \net performs equally well for the anisotropic and averaged isotropic elastic properties.
Accurate prediction of anisotropic properties offers a comprehensive understanding of the elastic characteristics exhibited by crystals, enabling the discovery of new materials through the utilization of these predictive capabilities.


\subsection{Screening of crystals with extreme properties}

% Figure environment removed

The maximum of the directional Young's modulus, $E_\text{d}^\text{max}$, characterizes the smallest possible deformation of a crystal under applied external loading.
It helps in the selection and orientating of materials to minimize shape change to guarantee the reliability of precision devices such as micro-electromechanical systems \cite{huang2012mems}.

We have also applied the \net model to screen for crystals with large  $E_\text{d}^\text{max}$.
We first filtered crystals from the Materials Project database based on their energy above the convex hull values, selecting those with a value of $\leq 50$~meV/atom.
This energy determines the thermodynamic stability of a crystal and has been shown to correlate with the synthesizability of crystals \cite{sun2016thermodynamic,bartel2022review}.
We further narrowed down the selection to crystals with fewer than 50 atomic sites to reduce computational cost and remove crystals already present in the dataset used to develop the model.
This resulted in 53480 crystals for further analysis.
Next, we employed \net to predict the full elasticity tensors for these crystals and compute their $E_\text{d}^\text{max}$.
The top 100 crystals with the highest  $E_\text{d}^\text{max}$ were then supplied for DFT computation to obtain their elasticity tensors.
\fref{fig:max:E} presents a histogram of $E_\text{d}^\text{max}$ for the identified 100 crystals.
Their $E_\text{d}^\text{max}$ values all fall at the tail of the distribution of $E_\text{d}^\text{max}$ from the training data.
Quantitatively, the mean of $E_\text{d}^\text{max}$ for the identified crystals is 606~GPa, while that for the training data is 174~GPa, corresponding to the expected value for a randomly selected material.
This demonstrates the effectiveness of leveraging \net to screen for materials with extreme elastic properties.
The newly identified crystals are provided in Data Availability.

In addition, we have identified 11 unconventional polymorphs of elemental cubic metals regarding the direction of the extreme values of Young's modulus.
Five of them have already been experimentally synthesized (four stable ground-state polymorphs and one metastable polymorph \cite{bartel2022review}), and the other six are metastable polymorphs that have not yet been experimentally observed.
It is believed that $E_\text{d}^\text{max}$ is along the $\langle 111 \rangle$ directions (meaning \eref{eq:E:max:dir} is not satisfied) for all cubic metals except Mo \cite{nye1985physical}.
We suspect that there exist other polymorphs of elemental cubic metals satisfying the criterion in \eref{eq:E:max:dir}.
To test this, we performed a screening of the Materials Project database.
From the dataset used to develop \net, we have identified six such crystals.
Mo is among them, and the other five are V (one polymorph), Cr (two polymorphs), and W (two polymorphs).
For crystals not in the dataset, we first used \net to predict their elasticity tensors and then selected the 18 crystals that meet the verification criterion using DFT.
DFT predicted 12 crystals satisfying the criterion; six are elastically unstable structures whose Voigt matrix has negative eigenvalues \cite{mouhat2014necessary, tolborg2022free},
and the other six successful ones are Mn, Na, K, Cs, Rh, and Tl (one polymorph for each).
Among the 11 newly identified crystals, polymorphs of alkali metals Na, K, and Cs have DFT calculated $S_{1111} - S_{1122} - 2S_{2323}$ values much more negative than that of Mo,
but they are all hypothetical crystals that have not been experimentally synthesized yet.
Five polymorphs are indeed experimentally observed, and they are all neighbors of Mo in the periodic table, namely V, Cr, Mn, and W.
Among them, four are thermodynamically stable ground-state polymorphs, and one polymorph of Cr is metastable.
Crystal structures, ground-state information, and elasticity tensors of the 11 unconventional polymorphs are provided in Data Availability and Table~S3 in the \gls{si}.


\section{Conclusions}
\label{sec:discussion}

A model such as \net that can predict the full elasticity tensors of inorganic crystalline compounds across crystal systems and chemical species brings new possibilities to probe and design materials with targeted mechanical properties.
\net has several unique characteristics:
1).\ it learns the full elasticity tensor and automatically handles all symmetry requirements, without the need to build separate models for individual components of the tensor or for each crystal system;
2).\ any elastic properties such as the bulk, shear, and Young's moduli can be computed from the predicted elasticity tensor, leading to a unified framework for modeling elasticity; and
3).\ it allows for the exploration of anisotropic elastic behaviors (not possible with existing machine learning models), demonstrated by screening for crystals with extreme directional Young's modulus.



It should be noted, however, that a crucial aspect regarding the practical use of the model relies on the robustness of the input structure.
Given two structures of a crystal where the atomic coordinates are slightly different, we would expect the elastic properties to be similar.
If, otherwise, the model is extremely sensitive to the input, then it is not ideal for practical application.
Extra work such as highly accurate DFT structure relaxation is needed before applying the model for predictions.
We tested \net by using structures directly queried from the Materials Project database and structures with tighter geometry optimization.
The MAE of $E_\text{d}^\text{max}$ between using the two types of structures is 6.55~GPa (Fig.~S17 in the \gls{si}), more than three times smaller than the MAE (22.36 GPa) of $E_\text{d}^\text{max}$ between model predictions and DFT references.
This suggests that \net is robust enough to its input, and reasonably optimized structures (for example, from online databases) would not introduce extra error larger than the intrinsic error in the model.

The \net model is not limited to inorganic crystalline compounds and even elasticity tensors in general.
The elastic behavior of other classes of materials such as two-dimensional layered materials and molecular crystals play a significant role in determining their functionality, and \net can be directly applied to model their elasticity tensors.
Of course, a curated dataset of reference elasticity tensors is needed.
Such data already exists, for example, in the Computational 2D Materials Database (C2DB) \cite{gjerding2021recent}.
Moreover, besides elasticity tensor, \net can be applied to other tensorial properties of materials.
These can be broadly categorized into two classes: material-level property and atom-level property.
While the former means a single tensor for each crystal, the latter means a separate tensor for each atom in the crystal.
Other material-level properties such as piezoelectric
and dielectric tensors can be modeled by updating the output head as in \fref{fig:schematic} to use the corresponding irreducible representations of the tensor of interest (for example, a single second-rank symmetric matrix for the dielectric tensor).
For atom-level properties such as the neutron magnet resonance (NMR) tensor, instead of using a mean pooling to aggregate atom features, one can directly map the features of an atom to a tensor for that atom.
Using \net, we have conducted such an analysis for NMR tensors of silicon oxides and found that \net significantly outperforms both historic analytical models and other machine learning models
by more than 50\% for isotropic and anisotropic NMR chemical shift \cite{venetos2023machine}.

One potential limitation of the proposed approach is the reliance on a relatively large dataset to develop the model.
We have curated a dataset of 10276 elasticity tensors which took millions of CPU hours to obtain.
Such large datasets for other tensorial properties may not be readily available, but they begin to emerge.
For example, the Materials Project has about 3000 piezoelectric and 7000 dielectric tensors \cite{de2015database,petousis2017high}.
This amount of data might still be a good start to training faithful models, given that piezoelectric and dielectric are third- and second-rank tensors, respectively, which are much simpler than the fourth-rank elasticity tensor.
In fact, for the second-rank NMR tensor, we only used a dataset of 421 crystals to obtain the best-performing model \cite{venetos2023machine}.
Another possibility is to apply a transfer learning approach, where the model is first trained on a different property with large data (for instance, elasticity tensor) and then finetuned on the target property of interest (for instance, piezoelectric tensor).
A limitation of the trained model can come from the data.
The data consists of DFT calculations of perfect single crystals with relatively small super cells at a temperature of 0~K.
Given that the efficacy of the model is intrinsically tied to the scope of the training data,
it is imperative to exercise caution when applying the model
to scenarios that extend beyond these parameters.
For example, the model is not appropriate for crystals with defects, such as vacancies, dislocations, and grain boundaries.
Additionally, it is not advisable to directly employ the model for estimating the mechanical properties at finite temperate, especially for those materials, like metallic alloys, which exhibit a pronounced temperature dependency.


\section{Experimental}
\label{sec:methods}

\subsection*{Data generation}

The elasticity tensors were computed by a liner fitting of the stresses and strains obtained from DFT calculations using the Vienna Ab Initio Simulation Package (VASP) \cite{kresse1993ab}.
The calculations follow the same procedures discussed in \onlinecite{de2015charting}, using \verb|PREC=Accurate|, a tight convergence criterion of \verb|EDIFF=1e-6|, an energy cutoff of \verb|ENCUT=700 eV|, and a $k$-points density of 64~\AA$^{-3}$ in the reciprocal space to sample the Brillouin zone.
Two additional improvements are made.
First, to get more precise stresses for calculating the elasticity tensor, the projection operators in VASP are evaluated in the reciprocal space, that is, the setting \verb|LREAL=False| was adopted.
Second, to reduce numerical error in the calculations, the stresses are symmetrized according to the crystal symmetry.
The entire workflow was implemented in the open-source \verb|atomate| package \cite{mathew2017atomate}.

\subsection*{Model architecture}

A crystal structure is converted to a graph using a distance-based approach, where an edge is created between a pair of atoms if their distance is smaller than a cutoff radius $r_\text{cut}$.
Periodic boundary conditions are considered in the graph construction.

For atom $a$, its atomic number $Z_a$ is embedded as a vector with $c$ components
using a one-hot encoding to obtain the initial atom feature $F_a$.
The unit vector $\hat{\bm r}_{ij}$ from atom $i$ to atom $j$ is expanded using a spherical harmonics basis consisting up to a degree of $l=4$.
(Explicitly, this corresponds to the ``0e + 1o + 2e + 3o + 4e'' irreducible representations in \verb|e3nn| notation \cite{e3nnpaper}).
The distance $r_{ij}$ between atoms $i$ and $j$ is expanded into a vector $R$ using the radial basis functions \cite{gasteiger2020dimenet},
\begin{equation}
    \text{RBF}_n(d) =  \sqrt{\frac{2}{r_\text{cut}}} \frac{\sin(\frac{n\pi}{r_\text{cut}} d)}{d}
\end{equation}
where $n = 1,2,\dots,$ is an index of the radial basis.


With the atom features $F$, the spherical harmonics expansion of $\hat{\bm r}_{ij}$, and the radial basis expansion $R$ of $r_{ij}$ obtained from the embedding layers, the interaction block performs tensor product--based convolution to refine the atom features.
This is achieved via \eref{eq:convolution}, more specifically \cite{thomas2018tensor},
\begin{equation} \label{eq:conv:indicial}
    \mathcal{L}_{acm_o}^{l_o} (\{\vec{\bm r}_{ab}\}, \{F^{l_i}_{bcm_i}\})
    = \sum_{m_i, m_f}C^{(l_o,m_o)}_{(l_f, m_f)(l_i, m_i)}
    \sum_{b\in \mathcal{N}_a}
    R_c^{(l_f, l_i)}(r_{ab}) Y_{l_f}^{m_f}(\hat{\bm r}_{ab})F^{l_i}_{bcm_i},
\end{equation}
where $a$ denotes the center atom, $b$ denotes all its neighbors $\mathcal{N}_b$ within the cutoff $r_\text{cut}$;
$l$ is an integer indicating the degree of the spherical harmonic function, and $m = -l, \dots, l$;
the subscripts $i$, $o$, and $f$ indicate input, output, and filter, respectively;
$c$ is the channel index (for example, for the embedding layer, it indicates the components of the one-hot encoding);
$C$ denotes the Clebsch-Gordan coefficients;
and finally, $R_c^{(l_f, l_i)}$ are learnable multilayer perceptrons (MLPs), which take the RBF expansion as the input and contain most of the parameters of the model.
Essentially, this combines the atom features of neighbors $b$ to be the new atom features of the center atom $a$, in the same spirit of a message-passing graph neural network.
A major characteristic of \eref{eq:conv:indicial} is that the use of the spherical harmonics and the Clebsch-Gordan coefficients together imply that convolutions are equivariant \cite{thomas2018tensor}.

The atom features $F$ is also passed through a self-interaction,
\begin{equation}
    \mathcal{S}_{acm}^l(F_{ac'm}^l) =  \sum_{c'} W_{cc'}^l F_{ac'm}^l ,
\end{equation}
where $W_{cc'}^l$ are learnable model parameters.
The updated atom features are then obtained as
\begin{equation}
    F^{k+1} =\mathcal{L}(F^k) + \mathcal{S}(F^k),
\end{equation}
where $F^{k}$ denotes the features in interaction block $k$, and $F^{k+1}$ the features in the next interaction block.
Indeed, $F^{k+1}$ are further passed through a nonlinearity and a normalization functions.
For each scalar part $s$ in $F$, the nonlinearity is chosen to be the SiLU function \cite{hendrycks2016gaussian},
\begin{equation}
    \text{SiLU}(s) = \sigma(s)s,
\end{equation}
and for each non-scalar part $\bm t$ in $F$, the gated nonlinearity \cite{weiler20183d} is adopted,
\begin{equation}
    G(\bm t) =   \sigma(x) \bm t,
\end{equation}
where $\sigma$ is the sigmoid function, and $x$ is a scalar obtained from  \eref{eq:conv:indicial} by setting $l_o = 0$ and $m_o=0$.
Finally, the equivariant batch normalization \cite{e3nnpaper} is applied to the features to avoid gradient vanishing or exploding.

The readout head aggregates the features of individual atoms to obtain a representation of the material via a mean pooling,
\begin{equation}
    F_\text{mat} = \frac{1}{N}\sum_a^N F_a,
\end{equation}
where $F_a$ denotes the features of atom $a$ for a crystal of a total number of $N$ atoms.
From $F_\text{mat}$, the appropriate irreducible representations (``2x0e + 2x2e + 4e'' in \verb|e3nn| notation) that correspond to the elasticity tensor (two scalars, two second-rank traceless symmetric tensors, and one fourth-rank traceless symmetric tensor) are selected to construct the elasticity tensor according to \eref{eq:C:decomp}.


\subsection*{Model training}

The dataset of 10276 elasticity tensors is split into three subsets for training, validation, and testing with a split ratio of 8:1:1.
A random split with stratification is adopted where each of the seven crystal systems is separately treated in the split.
The model parameters are optimized using the training set, model hyperparameters are determined based on model performance on the validation set,
and error analysis is performed using the test set unless otherwise stated.
We train the model with the Adam optimizer to minimize a mean-squared-error loss function
$ L = \sum_i^B \| \bm C_i - \bm C_i^\text{ref} \|^2 $
with a mini-batch size $B$ of 32.
Note, $\bm C_i$ denotes the irreducible representation of the model predicted elasticity tensor with 21 components (see \eref{eq:C:decomp}), but not the Cartesian tensor with 81 components.
Similarly, $\bm C_i^\text{ref}$ denotes the corresponding reference DFT values.
The learning rate is set to 0.01, and a reduce-on-plateau learning rate scheduler is used, which decreases the learning rate by a factor of 0.5 if the validation error does not decrease for 50 epochs.
The training stops when the validation error does not decrease for 150 consecutive epochs, and a maximum of 1000 epochs are allowed for the optimization.
We performed a grid search to obtain model hyperparameters such as the $r_\text{cut}$ and $c$.
Search ranges and their optimal values are listed in Table~S4 in the \gls{si}.

Ten-fold cross validation is also performed to test the effects of different data splits.
Figs.~S11 and S12 in the \gls{si} present the results, and \net is not sensitive to data splits.
Detailed information on the training, validation, and test split, as well as the ten-fold split is given in the released dataset (see Data Availability).


\subsection*{AutoMatmainer training}

Automatminer is a machine learning pipeline that automatically featurizes the crystals and selects the appropriate features to train a set of machine learning algorithms \cite{dunn2020benchmarking}.
The best-performing algorithm is used as the final model.
For all the results reported in this work, the \verb|production| preset is adopted.
It was found that the gradient boost, random forest, and extra trees algorithms can all be selected as the best-performing model, depending on the target elastic property and the initialization of the parameters in the automatic pipeline.
For each target elastic property, the reported results are obtained by averaging over multiple runs, each with a different initialization.

\subsection*{Compliance tensor}


The compliance tensor $\bm S$ is a fourth-rank tensor defined from the inverse stress-strain relation $\bm \epsilon = \bm S \bm \sigma$, where $\bm\epsilon$ and $\bm\sigma$ are the second-rank strain tensor and stress tensor, respectively.
The compliance tensor in Voigt notation $\bm s$ can be obtained as the inverse of the elasticity tensor Voigt matrix,
\begin{equation}
    \bm s = \bm c^{-1},
\end{equation}
which is a 6 by 6 symmetric matrix.
The relationships between the components of the full compliance tensor $S_{ijkl}$ and the Vogit matrix $s_{\alpha\beta}$ are \cite{nye1985physical}:
\begin{equation}
    \begin{aligned}
        S_{ijkl}  & = s_{\alpha\beta},\, \text{when $\alpha$ and $\beta$ are 1, 2, or 3,}       \\
        2S_{ijkl} & = s_{\alpha\beta},\, \text{when either $\alpha$ or $\beta$ are 4, 5, or 6,} \\
        4S_{ijkl} & = s_{\alpha\beta},\, \text{when both $\alpha$ and $\beta$ are 4, 5, or 6.}  \\
    \end{aligned}
\end{equation}



\subsection*{Averaged elastic moduli of polycrystals}

Given the elasticity tensor of a single crystal, the averaged bulk, shear, and Young's moduli of polycrystalline materials can be computed using different average schemes.
The Voigt average assumes that the strain is the same in each grain, equal to the macroscopically applied strain \cite{voigt1910lehrbuch}.
The Voigt bulk modulus is
\begin{equation}
    K_V =  [(c_{11} + c_{22} + c_{33}) + 2(c_{12} + c_{23} + c_{31})]/9,
\end{equation}
and the Voigt shear modulus is
\begin{equation}
    G_V =  [
    (c_{11} + c_{22} + c_{33})
    -  (c_{12} + c_{23} + c_{31})
    + 3(c_{44} + c_{55} + c_{66}) ]
    / 15  .
\end{equation}
The Reuss average assumes that the stress is the same in each grain, equal to the macroscopically applied stress \cite{reuss1929berechnung}.
The Reuss bulk modulus is
\begin{equation}
    K_R =  1 / [(s_{11} + s_{22} + s_{33}) +  2(s_{12} + s_{23} + s_{31})] ,
\end{equation}
and Reuss the shear modulus is
\begin{equation}
    G_R =  15 /[
    4(s_{11} + s_{22} + s_{33})
    - 4(s_{12} + s_{23} + s_{31})
    + 3(s_{44} + s_{55} + s_{66})
    ].
\end{equation}
The Voigt and Reuss averages are the two extreme cases.
The Hill average takes their arithmetic mean and is considered the most accurate in a wide range of experimental conditions \cite{hill1952elastic}.
The Hill bulk modulus is
\begin{equation}
    K_H = (K_V + K_R)/2,
\end{equation}
and the Hill shear modulus is
\begin{equation}
    G_H = (G_V + G_R)/2.
\end{equation}
Given the bulk modulus and the shear modulus (from any of the Voigt, Reuss, and Hill schemes), Young's modulus can be computed as  \cite{anand2022introduction}
\begin{equation}
    E = 9KG/(3K + G).
\end{equation}
In this work, we report the bulk modulus $K=K_H$, shear modulus $G=G_H$, and Young's modulus $E = 9K_HG_H/(3K_H + G_H)$ from the Hill average scheme.


\subsection*{Scaled error}

The mean absolute error (MAE) and mean absolute deviation (MAD) are defined as
$\text{MAE} = \frac{1}{N_1} \sum_i^{N_1} |y_i - y_i^\text{pred}| $
and
$\text{MAD} = \frac{1}{N_2} \sum_i^{N_2} |(y_i - \bar y)|$,
in which $y_i$ is the reference value of data point $i$, $y_i^\text{pred}$ is the model prediction for the data point, and $\bar y$ is the average of all reference values.
The numbers $N_1$ and $N_2$ do not necessarily need to be the same.
This is the case in \fref{fig:error:scalar} and \tref{tab:mae:mad}, where $N_1$ is the number of crystals in a specific crystal system and $N_2$ is the total size of the test set.
The scaled error (SE) is then computed as $\text{SE} = \text{MAE} / \text{MAD}$.





\subsection*{Software implementation}
The \net model was implemented using the \verb|e3nn| package \cite{e3nnpaper} built on top of \verb|PyTorch| \cite{paszke2019pytorch},
and the training was performed using \verb|Pytorch-Lightning| \cite{lightning}.
The DFT calculations were performed using the \verb|atomate| workflow \cite{mathew2017atomate} and all crystal structure processing was performed using the Python Materials Genomics (\verb|pymatgen|) \cite{ong2013python}.
Directional Young's modulus was obtained using the \verb|elate| package \cite{gaillac2016elate}.



\section*{Data Availability}
The elasticity tensors used for model development,
the 100 new crystals with large maximum directional Young's modulus, and the elemental cubic metals are available at \url{https://doi.org/10.5281/zenodo.8190849}
The elasticity tensors are also available from the Materials Project database via the web interface at \url{https://materialsproject.org}
or the API at \url{https://api.materialsproject.org}.


\section*{Code Availability}
The \net model and training scripts are released as an open-source repository at \url{https://github.com/wengroup/matten}.

\section*{Acknowledgements}

The method development was supported by the National Science Foundation under Grant No.\ 2316667 and the startup funds from the Presidential Frontier Faculty Program at the University of Houston.
Support for software and data infrastructure development was provided by the U.S.\ Department of Energy, Office of Science, Office of Basic Energy Sciences, Materials Sciences and Engineering Division under contract No.\ DE-AC02-05-CH11231 (Materials Project program KC23MP).
This work used computational resources provided by
the Research Computing Data Core at the University of Houston,
the National Energy Research Scientific Computing Center (NERSC), a U.S.\ Department of Energy Office of Science User Facility operated under contract No.\ DE-AC02-05CH11231,
and the Lawrencium computational cluster resource provided by the IT Division at the Lawrence Berkeley National Laboratory (Supported by the Director, Office of Science, Office of Basic Energy Sciences, of the U.S.\ Department of Energy under contract No.\ DE-AC02-05CH11231).


\section*{Author Contributions}

Conceptualization, investigation, software, visualization, and writing - original draft: M.W.;
data curation and formal analysis: M.W., M.K.H., J.M.M., and P.H.;
writing - review \& editing: M.W., M.K.H., J.M.M., P.H., and K.A.P.;
project administration: M.W.;
funding acquisition: M.W. and K.A.P.


\section*{Conflicts of Interests}

There are no conflicts to declare.



\bibliographystyle{unsrtnat}
%\bibliography{main.bib}
\documentclass[a4paper,11pt]{article}
\pdfoutput=1 % if your are submitting a pdflatex (i.e. if you have
             % images in pdf, png or jpg format)

%\usepackage[utf8]{inputenc}
%\usepackage{mathrsfs, amssymb, amsmath}  
%\usepackage{comment}
%\usepackage{dcolumn}
%\usepackage{multirow}
%\usepackage{color}
%\usepackage{amsfonts,amssymb,amsmath, txfonts}
%\usepackage{float}

\usepackage{jcappub} % for details on the use of the package, please
                     % see the JCAP-author-manual

\usepackage[T1]{fontenc} % if needed

\hypersetup{ linktoc=all,
    colorlinks=true, linkcolor={blue},  
       citecolor={red}, urlcolor={darkred}
}
\definecolor{Redgreen}{RGB}{153,76,0}
\definecolor{vividviolet}{rgb}{0.62, 0.0, 1.0}
\definecolor{green}{RGB}{11,98,17}
\definecolor{darkgreen}{RGB}{40,150,65}
\definecolor{darkblue}{rgb}{0,0,0.3}
\definecolor{darkred}{rgb}{0.7,0,0}

\def\blue{\textcolor{blue}}
\def\red{\textcolor{red}}
\def\be{\begin{equation}}
\def\ee{\end{equation}}
\def\bea{\begin{eqnarray}}
\def\eea{\end{eqnarray}}


\title{MCMC Marginalisation Bias and $\Lambda$CDM tensions}
%\title{Overcoming bias in MCMC marginalisation to elucidate $\Lambda$CDM tensions}
%\title{Temp}

%%Markov Chain Monte Carlo

%% %simple case: 2 authors, same institution
%% \author{A. Uthor}
%% \author{and A. Nother Author}
%% \affiliation{Institution,\\Address, Country}

% more complex case: 4 authors, 3 institutions, 2 
\author[a]{Eoin \'O Colg\'ain}
\author[b]{Saeed Pourojaghi}
\author[b, c]{M. M. Sheikh-Jabbari}
\author[a]{Darragh Sherwin}

% The "\note" macro will give a warning: "Ignoring empty anchor..."
% you can safely ignore it.

\affiliation[a]{Atlantic Technological University, Ash Lane, Sligo, Ireland}
\affiliation[b]{School of Physics, Institute for Research in Fundamental Sciences (IPM), P.O.Box 19395-5531, Tehran, Iran}
\affiliation[c]{The Abdus Salam ICTP, Strada Costiera 11, I-34014 Trieste, Italy}

% e-mail addresses: one for each author, in the same order as the authors
\emailAdd{eoin.ocolgain@atu.ie}
\emailAdd{pourojaghi@ipm.ir}
\emailAdd{jabbari@theory.ipm.ac.ir}
\emailAdd{darragh.sherwin@research.atu.ie}




\abstract{Probability distributions become non-Gaussian when the flat $\Lambda$CDM model is fitted to redshift binned data in the late Universe. We explain mathematically why this non-Gaussianity arises and confirm that Markov Chain Monte Carlo (MCMC) marginalisation leads to biased inferences in observational Hubble data (OHD). In particular, in high redshift bins we find that $\chi^2$ minima, as identified from both least squares fitting and the MCMC chain, fall outside of the $1 \sigma$ confidence intervals. We resort to profile distributions to correct this bias. Doing so, we observe that $z \gtrsim 1$ cosmic chronometer (CC) data currently prefers a non-evolving (constant) Hubble parameter over a Planck-$\Lambda$CDM cosmology at $\sim 2 \sigma$. We confirm that both mock simulations and profile distributions agree on this significance. Moreover, on the assumption that the Planck-$\Lambda$CDM cosmological model is correct, using profile distributions we confirm  a $> 2 \sigma$ discrepancy with Planck-$\Lambda$CDM in a combination of  CC and baryon acoustic oscillations (BAO) data beyond $ z \sim 1.5$ that was noted earlier through comparison of least square fits of observed and mock data.}



\begin{document}
\maketitle
\flushbottom

\section{Introduction}
\label{sec:intro}
The flat $\Lambda$CDM model is the minimal model that fits Cosmic Microwave Background (CMB) data. Remarkably, CMB data from the Planck satellite \cite{Planck:2018vyg} constrains the $\Lambda$CDM model to sub-percent errors, thereby not only providing the strongest constraints, but also a concrete prediction for cosmological probes in the late Universe. The unmitigated success of the $\Lambda$CDM model is that CMB, Type Ia supernovae (SN) \cite{Riess:1998cb, Perlmutter:1998np} and baryon acoustic oscillations (BAO) \cite{Eisenstein:2005su} agree on a $\Lambda$CDM Universe that is approximately $30 \%$ matter. Thus, one key prediction of the Planck-$\Lambda$CDM model agrees across early and late Universe cosmological probes. Given this non-trivial agreement, any discrepancies that arise elsewhere constitute challenging puzzles. 

Nevertheless, one cannot define any \textit{model} for a dynamical system, especially a complicated system like the Universe, using data from a cosmic snapshot.\footnote{Here, we mean CMB data with an effective redshift $z \sim 1100$.} At best, one has a \textit{prediction} and not a model. In recent years, key predictions of Planck data have been challenged by late Universe determinations of the Hubble constant $H_0$ \cite{Riess:2021jrx, Freedman:2021ahq, Pesce:2020xfe, Blakeslee:2021rqi, Kourkchi:2020iyz} and the $S_8:= \sigma_8 \sqrt{\Omega_m/0.3}$ parameter \cite{HSC:2018mrq, KiDS:2020suj, DES:2021wwk, Boruah:2019icj, Said:2020epb}. Given the diversity of the late Universe probes (see reviews \cite{Perivolaropoulos:2021jda, Abdalla:2022yfr}), it is highly unlikely that any single systematic can be found to explain the discrepancies. That being said, in astrophysics one can never preclude systematics; 3 decades after Phillips' seminal paper \cite{Phillips:1993ng}, we are still debating an ad hoc correction for the mass of the host galaxy in Type Ia SN \cite{NearbySupernovaFactory:2018qkd, Kang:2019azh, Brout:2020msh, Lee:2021txi}. Bearing in mind that Type Ia SN are one of our best understood cosmological probes, one quickly understands that any systematics debate may be endless. 

Thus, it is far more expedient to assume that the $\Lambda$CDM model is breaking down and to look for tell-tale signatures of model breakdown. If signatures cannot be found, one arrives at a contradiction, and revisits the assumption that the model is breaking down. For physicists, \textit{model breakdown comes about when model fitting parameters return discrepant values at different time slices or epochs}. Translated into astronomy, this equates to discrepant cosmological parameters in different redshift ranges. The usual $H_0, S_8$ tensions  may also be viewed in the same light: a discrepancy between high and low redshift inferences/measurements of the parameters \cite{Perivolaropoulos:2021jda, Abdalla:2022yfr}. Nevertheless, early and late Universe observables are typically not the same, so one is confronted with a rich set of potential systematics. 

Within the context of $\Lambda$CDM tensions, it was recently observed that the integration constant from the Friedmann equations, aka the Hubble constant $H_0$, picks up redshift dependence whenever our model assumption - required to close the Friedmann equations - disagrees with the Hubble parameter $H(z)$ extracted from observations \cite{Krishnan:2020vaf, Krishnan:2022fzz}. %\footnote{One is free to speculate about the nature of the missing physics \cite{Liao:2020zko, Montani:2023xpd}.} 
Similarly, $\rho_{m0}=H_0^2\Omega_m$, an integration constant of the matter continuity equation, implies matter density $\Omega_m$ is a mathematically constant quantity. 
These are irrefutable predictions from mathematics, i. e. a prediction that is \textit{robust to systematics}. However, observationally $H_0$ and $\Omega_{m}$ are model fitting parameters and nothing precludes them picking up redshift dependence (except of course if one assumes they do not!), and providing a signature of model breakdown. If this happens in the late Universe within the $\Lambda$CDM model, $H_0$ is correlated with matter density $\Omega_m$, 
while $\Omega_m$ is correlated with $S_8 \propto \sigma_8 \sqrt{\Omega_m}$. Thus, there is at least one simple scenario, namely redshift evolution of cosmological parameters in the late Universe, where ``$H_0$ tension'' and ``$S_8$ tension'' are not independent and simply symptoms of $\Lambda$CDM model breakdown. 

The next relevant question is, where is the evidence for evolving cosmological parameters in the late Universe? Starting with strong lensing time delay \cite{Wong:2019kwg, Millon:2019slk},\footnote{Systematics are explored in \cite{Millon:2019slk} and the descending trend is not an obvious systematic. The lensed system RXJ1131-1231 \cite{Sluse:2003iy}, which partly drives the trend, has recently been re-analysed using spatially resolved stellar kinematics of the host galaxy \cite{Shajib:2023uig}, and the higher $H_0$ value remains robust, admittedly with inflated errors. As TDCOSMO project to analyse 40 lenses, the prospect of a discovery of a descending $H_0$ trend assuming the $\Lambda$CDM model remain strong.} descending trends of $H_0$ with redshift have been reported in Type Ia SN \cite{Dainotti:2021pqg, Colgain:2022nlb, Colgain:2022rxy,  Malekjani:2023dky, Hu:2022kes, Jia:2022ycc} and combinations of data sets \cite{Krishnan:2020obg, Dainotti:2022bzg}. On the other hand, larger values of $\Omega_m$ have been noted in high redshift observables, primarily quasars (QSOs) \cite{Risaliti:2015zla, Risaliti:2018reu, Lusso:2020pdb, Yang:2019vgk, Khadka:2020vlh, Khadka:2020tlm, Khadka:2021xcc, Pourojaghi:2022zrh},\footnote{Just as with Type Ia SN, the systematics of QSOs are being investigated \cite{Zajacek:2023qjm}.} but also Type Ia SN \cite{Colgain:2022nlb, Colgain:2022rxy, Malekjani:2023dky, Pasten:2023rpc} (see also \cite{Wagner:2022etu, Sakr:2023hrl}). Note, as emphasised earlier, if $H_0$ evolves at the background level, correlated fitting parameters are expected to also evolve. Moreover, mock analysis within the $\Lambda$CDM setting reveals that evolution of best fit $(H_0, \Omega_m)$ parameters cannot be precluded, and conversely possesses a finite likelihood, in either observational Hubble data (OHD) \textit{or} angular diameter distance data \textit{or} luminosity distance data \cite{Colgain:2022tql}. We stress that this result \textit{rests on mock analysis}; it represents a purely mathematical statement about the $\Lambda$CDM model that is independent of systematics. 

Separately, at the perturbative level, redshift evolution of $S_8$ or $\sigma_8$ has been reported in galaxy cluster number counts and Lyman-$\alpha$ spectra \cite{Esposito:2022plo}, $f \sigma_8$ constraints from peculiar velocities and redshift space distortions (RSD) 
 \cite{Adil:2023jtu}, comparison between weak \cite{HSC:2018mrq, KiDS:2020suj, DES:2021wwk} and CMB lensing \cite{ACT:2023dou, ACT:2023kun}. What is important here is that these observations appear to restrict the evolution in $S_8$ to the late Universe. In \cite{ACT:2023ipp} the possibility was raised that \textit{``tracers at higher redshift and probing larger scales prefer higher $S_8$''}.\footnote{There are also conflicting observations of high redshift $\sigma_8$ or $S_8$ values that are lower than Planck in the late Universe \cite{Miyatake:2021qjr, Alonso:2023guh}, so either this trend is not universal, or systematics are at play.} Nevertheless, one can argue against evolution with scale on the grounds that cosmic shear \cite{HSC:2018mrq, KiDS:2020suj, DES:2021wwk}, which is sensitive to smaller scales (larger $k$), and peculiar velocity constraints \cite{Boruah:2019icj, Said:2020epb}, which are sensitive to larger scales (smaller $k$), both prefer lower values of $S_8$. Moreover, both galaxy clusters and Lyman-$\alpha$ spectra are expected to probe similar scales.\footnote{We thank Matteo Viel for correspondence on this point.} Thus, if systematics are not impacting results, then redshift evolution is the only point of agreement in the observations \cite{Esposito:2022plo, Adil:2023jtu, HSC:2018mrq, KiDS:2020suj, DES:2021wwk, ACT:2023dou, ACT:2023kun, ACT:2023ipp}. Note also that redshift is more fundamental than scale in FLRW cosmology; one must solve the Friedmann equations in either time or redshift before one contemplates any discussion of scale.  

 The purpose of this letter is to revisit the analysis presented in \cite{Colgain:2022rxy,Colgain:2022tql}, where the evidence for evolution was quantified on the basis of mock simulations and not Markov Chain Monte Carlo (MCMC), the technique most familiar in cosmology. The fundamental problem is that once one bins low redshift data and studies evolution of cosmological parameters with bin redshift, one quickly encounters projection effects in MCMC analyses. These effects are not just the preserve of exotic models \cite{Herold:2021ksg, Gomez-Valent:2022hkb, Meiers:2023gft}, such as Early Dark Energy (EDE) \cite{Poulin:2018cxd, Niedermann:2019olb}, and happen in the simplest model when one bins data. The most striking demonstration of the resulting bias is that the peaks of MCMC posteriors no longer coincide with the minimum of the likelihood (see \cite{Gomez-Valent:2022hkb}). Ultimately, this bias is expected  because one is working in a regime of the $\Lambda$CDM model with non-Gaussian probability distributions   \cite{Colgain:2022tql}  (see also \cite{Colgain:2022rxy}).

 The structure of this paper is as follows. In section \ref{sec:MCMC_bias} we confirm the bias in MCMC marginalisation. In section \ref{sec:PD} we introduce profile distributions (PDs) \cite{Gomez-Valent:2022hkb} as a means of addressing the bias and confirm that the statistical significance of discrepancies from mock simulations agree well with PD analysis. In section \ref{sec:tension}, we revisit and confirm the high redshift OHD tensions reported in \cite{Colgain:2022rxy}. We end in section \ref{sec:discussion} with concluding remarks. 
 %A short appendix is also added on Fisher matrix for $\Lambda$CDM mdoel. 

\section{A bias in MCMC marginalisation}
\label{sec:MCMC_bias}
In this section we illustrate a bias in MCMC marginalisation that arises in the (flat) $\Lambda$CDM model when data is binned by redshift. This bias can be traced to a regime of the $\Lambda$CDM model with non-Gaussian distributions and is independent of systematics  \cite{Colgain:2022rxy, Colgain:2022tql}. 

\subsection{Mathematical Foundations}
\label{sec:math}
Consider an exercise where one bins OHD and confronts it to the $\Lambda$CDM Hubble Parameter $H(z)$ in the late Universe, a setting where the radiation sector can be safely decoupled. In high redshift bins ($z \gg 0$) in the matter-dominated regime, the Hubble parameter becomes insensitive to the dark energy (DE) sector: 
\be
\label{eq:lcdm}
H(z) = H_0 \sqrt{1-\Omega_m + \Omega_m (1+z)^3} \xrightarrow[z \gg 0]{} H_0 \sqrt{\Omega_m} (1+z)^{\frac{3}{2}}.  
\ee
More concretely, taking $z \rightarrow \infty$ we see that data can only constrain the combination $\rho_{m0}=H_0^2{\Omega_m}$. For \textit{hypothetical} data in a redshift bin with effective redshift $z = \infty$, this means that one can only constrain the combination $\Omega_m h^2$ ($h:= H_0/100)$, but $H_0$ and $\Omega_m$ remain unconstrained. Alternatively put, for any given $\Omega_m h^2$ constraint, there is an infinite number of corresponding $(H_0, \Omega_m)$ pairs. Translated into a probability density function (PDF), this is simply the statement that in a very high redshift bin at $z = \infty$, one expects uniform or flat distributions for $H_0$ and $\Omega_m$ with the model (\ref{eq:lcdm}).  

Of course, observed data resides at finite $z$ and not $z = \infty$. As a result, one does not encounter \textit{exactly} flat PDFs in $H_0$ and $\Omega_m$ at high redshift, but \textit{almost} flat PDFs. More important to us is the observation that these PDFs must flatten in a non-Gaussian manner. To appreciate this fact, we observe that high redshift OHD only constrains $\Omega_m h^2$ well.\footnote{Note that observables like SN or QSO that measure $D_L(z)=c (1+z)\int_0^z \textrm{d} z'/H(z')$ are mainly sensitive to the low redshift part of $H(z)$, i. e. the combination $H_0^2 (1-\Omega_m)$, and in this sense they are complementary to the OHD data which is more sensitive to high redshift part of $H(z)$, $H_0^2\Omega_m$. The complementarity can be demonstrated by combining $H(z)$ and $D_{L}(z)$ constraints and checking that one recovers mock data input parameters in all redshift bins \cite{Colgain:2022tql}. } For this reason, best fit parameters are constrained to a $\Omega_m h^2 = \textrm{constant}$ curve in the $(H_0, \Omega_m)$-plane. The almost flat $H_0$ and $\Omega_m$ PDFs can only arise if this curve stretches in the $(H_0, \Omega_m)$-plane. As a result of this stretching, one ends up with a relatively uniform distribution on a curve. At the extremes of the curve, one finds a distribution of large $H_0$ values, which do not differ greatly in $\Omega_m$, and they get projected to a peak at small values on the $\Omega_m$ axis. Conversely, at the other end of the curve, one finds a distribution of small $\Omega_m$ values, which do not differ greatly in $H_0$, and they get projected onto a peak at large values on the $H_0$ axis.  This is a ``projection effect'' in common cosmology parlance.  It is driven by the irrelevance of the DE sector at high redshift and the constraint $\Omega_m h^2 = \textrm{constant}$ from the $\Lambda$CDM model (\ref{eq:lcdm}). Together these features distort the distribution away from a Gaussian configuration. 

Thus, simply by binning and fitting OHD to the $\Lambda$CDM model one enters a non-Gaussian regime as the effective redshift of the bin increases. This effect, which is expected from the purely mathematical arguments above, has been confirmed in mock data \cite{Colgain:2022rxy, Colgain:2022tql}, and in line with expectations, we demonstrate that it impacts MCMC inferences with observed data in the next subsection.  

% Figure environment removed

\subsection{Cosmic Chronometer (CC) Data}
\label{sec:CCbias}
Here we work with OHD from the cosmic chronometer (CC) program \cite{Jimenez:2001gg}. Concretely, we work with 34 $H(z)$ constraints spanning the redshift range $0.07 \leq z \leq 1.965$ \cite{Stern:2009ep, Moresco:2012jh, Zhang:2012mp, Moresco:2016mzx, Ratsimbazafy:2017vga, Borghi:2021rft, Jiao:2022aep, Tomasetti:2023kek}. We illustrate the data in Fig.~\ref{fig:CC}, where it is consistent with Fig. 9 of \cite{Tomasetti:2023kek} {modulo the fact that we have an additional data point at $z = 0.8$, which is not independent. See Table 1.1 of \cite{Moresco:2023zys}. While CC data may eventually be good enough to arbitrate on Hubble tension \cite{Moresco:2023zys}, the data is not good enough on its own to do cosmology. To put this comment in context, we observe that the errors in Fig.~\ref{fig:CC} do not include systematic errors (see \cite{Moresco:2020fbm} for an account of the systematics). As a result the constraints we get on cosmological parameters will be underestimated. Thus, from our perspective the data in Fig.~\ref{fig:CC} is simply some representative cosmological data in the OHD class.}

\paragraph{Methodology:} We impose a low redshift cut-off on the OHD $z_{\textrm{min}}$, removing all data points with redshifts $z_i < z_{\textrm{min}}$, and then extremising the $\chi^2$ likelihood, 
\be
\label{eq:chi2}
\chi^2 = Q^{T} \cdot C^{-1} \cdot Q, 
\ee
where $C$ is the covariance matrix, which is simply the square of the $H_i$ errors on the diagonal, and $Q$ is the vector, 
\be
\label{eq:Q}
Q_i = H_i - H_{\textrm{model}}(z_i), 
\ee
where $H_i:=H(z_i)$ denotes OHD and $H_{\textrm{model}}(z)$ is the model (\ref{eq:lcdm}) without the high redshift limit. The best fit $(H_0, \Omega_m)$ parameters correspond to the minumum of the $\chi^2$, while on the assumption of Gaussian errors, we estimate the errors from a Fisher matrix (appendix \ref{sec:fisher}). In parallel, we perform MCMC marginalisation through \textit{emcee} \cite{Foreman-Mackey:2012any}. More concretely, subject to the priors $H_0 \in [0, 200 ]$ and $\Omega_m \in [ 0, 1]$, the latter restricting us to a physical regime, we record $16^{\textrm{th}}$, $50^{\textrm{th}}$ and $84^{\textrm{th}}$ percentiles for MCMC posteriors, as is common practice with Gaussian distributions. Thus, both techniques are tailored to Gaussian posteriors, yet non-Gaussianities will be evident in MCMC posteriors. By comparing the output from these two techniques in Table \ref{tab:LCDM_CC} for different values of $z_{\textrm{min}}$ we observe that error estimates from Fisher matrix and MCMC quickly disagree as $z_{\textrm{min}}$ increases. 

From Table \ref{tab:LCDM_CC}, we see that MCMC inferences lead to non-Gaussian $1 \sigma$ confidence intervals, where in line with the expectations from \cite{Colgain:2022tql}, $H_0$ errors are larger for smaller values, and $\Omega_m$ errors are larger for larger values, respectively. This is expected if the $H_0$ and $\Omega_m$ posteriors are peaked at larger and smaller values, respectively, in line with our earlier mathematical argument. Only for the full data set with $z_{\textrm{min}} = 0$  do we find reasonable agreement between the Fisher matrix and MCMC $1 \sigma$ confidence intervals. As can be seen from the lopsided MCMC confidence intervals, the non-Gaussianity becomes more pronounced with increasing $z_{\textrm{min}}$. Interestingly, beyond $z_{\textrm{min}} = 1$, the minimum of the $\chi^2$ falls outside of the MCMC $1 \sigma$ confidence intervals. Nevertheless, by evaluating the MCMC chains on the $\chi^2$ likelihood (\ref{eq:chi2}), we confirm that the parameters corresponding to the minimum $\chi^2$ value are tracking the best fit. Note, the peak of the MCMC posterior is no longer a measure of goodness of fit and inferences have become biased in a regime of model parameter space where distributions are expected to be inherently non-Gaussian. Our analysis here underscores potential problems with a blind MCMC analysis with the traditional $16^{\textrm{th}}$, $50^{\textrm{th}}$ and $84^{\textrm{th}}$ percentiles.       



\begin{table}[htb]
    \centering
    \begin{tabular}{c|c|c|c|c|c}
    \rule{0pt}{3ex} $z_{\textrm{min}}$ & \# CC & \multicolumn{2}{c}{Fisher Matrix}  & \multicolumn{2}{|c}{MCMC} \\
    \hline
    \rule{0pt}{3ex} & & $H_0$ (km/s/Mpc) & $\Omega_m$ & $H_0$ (km/s/Mpc) & $\Omega_m$ \\
    \hline
    \rule{0pt}{3ex} $0$ & $34$ & $68.14 \pm 3.07$ & $0.320 \pm 0.059$ & $67.76^{+3.03}_{-3.09}$  ($68.12$) & $0.328^{+0.065}_{-0.055}$ ($0.321$) \\
    \hline 
    \rule{0pt}{3ex} $0.2$ & $27$ & $65.03 \pm 6.65$ & $0.368 \pm 0.118$ & $63.05^{+6.64}_{-7.23}$ ($64.98$) & $0.405^{+0.170}_{-0.111}$ ($0.369$) \\
    \hline 
    \rule{0pt}{3ex} $0.4$ & $22$ & $62.42 \pm 8.38$ & $0.411 \pm 0.161$ & $59.54^{+8.30}_{-8.22}$ ($62.39$) & $0.470^{+0.229}_{-0.151}$ ($0.411$)\\
    \hline 
    \rule{0pt}{3ex} $0.6$ & $15$ & $59.83 \pm 17.21$ & $0.454 \pm 0.338$ & $56.45^{+13.16}_{-9.33}$ ($59.86$) & $0.526^{+0.288}_{-0.225}$ ($0.453$) \\
    \hline 
    \rule{0pt}{3ex} $0.7$ & $14$ & $79.11 \pm 19.40$ & $0.222 \pm 0.162$ & $67.59^{+19.19}_{-16.57}$ ($79.18$) & $0.344^{+0.344}_{-0.178}$ ($0.222$) \\
    \hline 
    \rule{0pt}{3ex} $0.8$ & $11$ & $103.97 \pm 24.94$ & $0.097 \pm 0.088$ & $82.43^{+28.33}_{-27.03}$ ($104.02$) & $0.206^{+0.357}_{-0.131}$ ($0.096$) \\
    \hline 
    \rule{0pt}{3ex} $1$ & $8$ & $150.37 \pm 31.21$ & $0.010 \pm 0.035$ & $108.92^{+33.94}_{-44.47}$ ($150.38$) & $0.087^{+0.304}_{-0.068}$ ($0.010$) \\
    \hline 
    \rule{0pt}{3ex} $1.2$ & $7$ & $154.35 \pm 42.95$ & $0.006 \pm 0.042$ & $83.07^{+48.52}_{-32.19}$ ($154.47$) & $0.194^{+0.439}_{-0.159}$ ($0.006$) \\
    \hline 
    \rule{0pt}{3ex} $1.4$ & $4$ & $125.41 \pm 79.55$ & $0.039 \pm 0.132$ & $65.32^{+44.88}_{-20.30}$ ($125.44$) & $0.320^{+0.423}_{-0.250}$ ($0.039$) \\
    \hline 
    \rule{0pt}{3ex} $1.5$ & $3$ & $36.12 \pm 72.69$ & $1.000 \pm 4.269$ & $55.19^{+34.64}_{-14.73}$ ($36.16$) & $0.393^{+0.387}_{-0.283}$ ($0.999$)
    \end{tabular}
    \caption{Comparison between Fisher matrix and MCMC analysis for CC data with a low redshift cut-off $z_{\textrm{min}}$. We record the number of data points, the extremum of the $\chi^2$ and $1 \sigma$ confidence interval estimated from the Fisher matrix,  $16^{\textrm{th}}$, $50^{\textrm{th}}$ and $84^{\textrm{th}}$ percentiles from MCMC posteriors corresponding to $1 \sigma$ confidence intervals, and the minimum $\chi^2$ from the MCMC chain in brackets. MCMC marginalisation exhibits non-Gaussian $1 \sigma$ confidence intervals, and for $z_{\textrm{min}} > 1$, the minimum value of the $\chi^2$ from the MCMC chain falls outside of this interval. The latter tracks the best fit up to small numbers in line with expectations. }
    \label{tab:LCDM_CC}
\end{table}

\subsection{Features in CC Data}
\label{sec:features}
Once one accounts for biases, it is clear from Table \ref{tab:LCDM_CC} that there are trends in CC data when it is binned. Starting from $z_{\textrm{min}} = 0$ through to $z_{\textrm{min}} = 0.6$ we see a decreasing trend in best fit values of $H_0$ (also central $H_0$ values from MCMC), which is compensated by a increasing trend in $\Omega_m$ best fit values. From Fig.~\ref{fig:CC} it is difficult to visibly discern any trend from the raw data. From $z_{\textrm{min}} = 0.7$ through to $z_{\textrm{min}} = 1.4$, there is in contrast a preference for larger $H_0$ and smaller $\Omega_m$ values. This trend is evident from the raw data, where at higher redshifts one sees large scatter and large fractional errors in the data. For $z_{\textrm{min}} = 1$, it is clear that the best fit line in magenta corresponding to $(H_0, \Omega_m) = (150.4, 0.01)$ (Table \ref{tab:LCDM_CC}) is closer to horizontal line than the Planck-$\Lambda$CDM cosmology in red. To be more explicit, for $z_{\textrm{min}} = 0$, $\rho_{m0}:=H_0^2\Omega_m\simeq 1500$ which is close to the Planck value, whereas for $z_{\textrm{min}} = 1$, $\rho_{m0}\simeq 225$. The sharp drop in $\rho_{m0}$ means the magenta line should be almost horizontal. For $z_{\textrm{min}} = 1.5$, we switch to an opposite regime of parameter space with unexpectedly low and high values of $H_0$ and $\Omega_m$, respectively, a trend which is evident in the data, but there are only three data points. Despite, the small number of data points, the tendency for smaller $H_0$ and larger $\Omega_m$ inferences within $\Lambda$CDM cosmology at high redshifts has been documented across three independent observables \cite{Colgain:2022rxy}. We will come back to this claim in section \ref{sec:tension}. Finally, it is worth noting that for large $z_{\textrm{min}}$ and samples with few data points, one expects broad MCMC posteriors. These posteriors are severely impacted by the prior on $\Omega_m$, as is evident from Table \ref{tab:LCDM_CC}. 

For the moment we leave physical speculations to the discussion and return to the trend in CC data above $z=1$ favouring less evolution in the Hubble parameter than the Planck-$\Lambda$CDM model. We would like to quantify the significance of this trend, but since we are working in a non-Gaussian regime of the model, we can expect both Fisher matrix and MCMC to give biased results. In Fig.~\ref{fig:CCsplit1} we show MCMC posteriors for $z>1$ CC data in blue alongside posteriors for low redshift ($z < 1$) CC data, which is simply added to aid comparison and also highlight the Gaussianity of the low redshift posteriors. One notes that the peaks of the $z > 1$ distributions are a little displaced from to the values minimising the $\chi^2$. However, the emergence of the lower peak in the $H_0$ posterior at $H_0 \sim 50$ km/s/Mpc has the hallmarks of a projection effect. To appreciate this, note that the configurations in the blue curve in the top left corner of the 2D posterior are projected onto the lower $H_0$ peak. Moreover, if one shifts the $H_0$ peak from $H_0 \sim 150$ to $H_0 \sim 50$ km/s/Mpc while maintaining $\Omega_m \sim 0$, this shifts the magenta curve in Fig. \ref{fig:CC} outside of all the data points, so the lower $H_0$ peak is a phantom artefact unrelated to the goodness of fit. We also observe a shift in the higher $H_0$ peak away from the minimum of the $\chi^2$.

Ignoring these features, one could attempt to interpret the overlap in the 2D posteriors in Fig. \ref{fig:CCsplit1}. Doing so, one may conclude that low and high redshift CC data are consistent within $1 \sigma$. However, since Hubble tension is a 1D problem (local $H_0$ determinations are insensitive to other parameters), to compare with locally observed values of $H_0$ one needs to project onto the $H_0$ axis. Alternatively put, Hubble tension is a problem in 1D posteriors. Projecting onto the $H_0$ axis by determining $16^{\textrm{th}}$, $50^{\textrm{th}}$ and $84^{\textrm{th}}$ percentiles, one sees from Table \ref{tab:LCDM_CC} that the $z_{\textrm{min}} = 1$ MCMC confidence interval encloses the $z_{\textrm{min}} = 0$ central values within $1 \sigma$,\footnote{Note, removing the eight high redshift data points from the $z_{\textrm{min}} = 0$ sample will not shift the central values much.} but not the point in parameter space that best fits the data!


% Figure environment removed



Evidently, given the non-Gaussian posteriors, care is required when interpreting the significance of the trend towards a non-evolving (horizontal) $H(z)$ at higher redshifts in Fig.~\ref{fig:CC}. We cannot use the errors from the Fisher matrix as we are clearly in a non-Gaussian regime, whereas MCMC inferences are impacted by projection effects to the extent that the minimum of the $\chi^2$ (confirmed from the MCMC chain) falls outside of the $1 \sigma$ confidence interval. For this reason, we resort to mock simulations. While this may seem a little redundant if we are going to employ profile distributions in section \ref{sec:PD}, there is motivation for this exercise. In \cite{Colgain:2022rxy} the significance of a descending $H_0$/increasing $\Omega_m$ trend with effective redshift in OHD, Type Ia SN and QSOs was estimated to be a $\sim 3 \sigma$ effect on the basis of combining $\sim 2 \sigma$ effects in each of the \textit{independent} data sets using Fisher's method. Here, working with the same data throughout, we can directly compare the significance of a discrepancy estimated through mock simulations from the significance of a discrepancy estimated through profile distributions. In particular, we will address the question: how significant is a constant $H(z)$ with $z_{\textrm{min}}=1$ (8 data points) against the Planck consistent cosmology favoured by the full data set ($z_{\textrm{min}}=0$ entry in Table \ref{tab:LCDM_CC})? Note, the significance will be overestimated due to missing systematic uncertainties (see \cite{Moresco:2020fbm}), but we can still make comparison between the two techniques.

\paragraph{{Mock simulations:}} To address this question using mock simulations, we begin with the MCMC chains for the full sample. For each entry in the MCMC chain (approximately 15,000 entries in total), we generate a new realisation of the 8 high redshift data points $(z > 1)$ that are by construction statistically consistent with both the best fits from the full sample and also the Planck-$\Lambda$CDM values \cite{Planck:2018vyg}. More concretely, for each $(H_0, \Omega_m)$ entry in our MCMC chain, we displace the data points to the corresponding $\Lambda$CDM Hubble parameter before generating new data points in a normal distribution where the errors serve as standard deviations. We then fit back the $\Lambda$CDM model to each realisation of the mock data and record the best fit $(H_0, \Omega_m)$ values, which give us a distribution of expected $(H_0, \Omega_m)$ best fits. The distributions are presented in Fig.~\ref{fig:CCsims} alongside the best fits from observed data. Throughout, we assume canonical values $(H_0, \Omega_m) = (70, 0.3)$ for the initial guess of the fitting algorithm. Best fits can saturate our bounds, i. e. $\Omega_m = 0$ and $\Omega_m = 1$, and this leads to an unsightly pile up of best fits at $\Omega_m = 0$ and $\Omega_m = 1$ in Fig.~\ref{fig:CCsims} \cite{Colgain:2022rxy}. It is important to retain all the configurations, otherwise one is not accounting for the probability that a best fit falls outside our priors. As a consistency check, we see that the median or 50$^{\textrm{th}}$ percentile, $(H_0, \Omega_m) = (68.32, 0.321)$ agrees well with the mock input parameters, thereby demonstrating that there are an equal number of best fits with values above and below the injected parameters in the mocks. We find that probability of a more extreme (larger) $H_0$ value to be $p = 0.022$, while the probability of a more extreme (smaller) $\Omega_m$ value to be $p = 0.035$, respectively. Converted into a Gaussian statistic, these correspond to $2 \sigma$ and $1.8 \sigma$, respectively, for a one-sided normal distribution. Thus, on the basis of mock simulations, we estimate the non-evolving constant $H(z)$ with $z_{\textrm{min}} = 1$ as a $\sim 2 \sigma$ effect. In the next section we will recover this number more or less from the profile distribution analysis. 

% Figure environment removed


\section{Profile Distributions}
\label{sec:PD}
Having explained the mathematics behind the bias, which gives rise to a projection effect, in subsection \ref{sec:math}, and having illustrated how it affects MCMC inferences in subsection \ref{sec:CCbias} - the minimum of the $\chi^2$ may fall outside of $1 \sigma$ confidence intervals - we turn to profile distributions (PDs) \cite{Gomez-Valent:2022hkb}, an extension of the profile likelihood, e. g. \cite{Trotta:2017wnx}, in order to address the bias. Consider two sets of parameters $\theta_1$ and $\theta_2$ and a normalised distribution $\mathcal{P}(\theta_1, \theta_2)$. The basic idea \cite{Gomez-Valent:2022hkb} is to study the ratio 
\be
\label{R}
R(\theta_1) = \frac{\tilde{\mathcal{P}}(\theta_1)}{\max_{\theta_1} \tilde{\mathcal{P}}(\theta_1) } = \frac{\tilde{\mathcal{P}}(\theta_1)}{\max_{\theta_1, \theta_2} \mathcal{P}(\theta_1, \theta_2) },  
\ee
where $\tilde{\mathcal{P}}(\theta_1)$ is the PD, defined to be the maximum of $\mathcal{P}$ for each $\theta_1$ along the $\theta_2$ direction: 
\be
\label{PD}
\tilde{\mathcal{P}} (\theta_1) = \max_{\theta_2} \mathcal{P}(\theta_1, \theta_2). 
\ee
The advantage of this approach is that $R(\theta_1)$ can serve as a probability distribution function (up to an overall normalization), however we do not need to perform any integration, so $R(\theta_1)$ is not prone to volume or projection effects. At this juncture, given the simplicity of our setup with only two parameters $(H_0, \Omega_m)$, we can be more explicit. Consider the probability distribution,   
\be
\mathcal{P}(\theta_1, \theta_2) = \exp \left( - \frac{1}{2} \chi^2(\theta_1, \theta_2) \right), 
\ee
where $\theta_i \in \{H_0, \Omega_m \}$  and $\chi^2(H_0, \Omega_m)$ is our earlier likelihood (\ref{eq:chi2}). The maximum value of $\mathcal{P}$ occurs for the minimum value of $\chi^2$ from the MCMC chain, $\mathcal{P}_{\textrm{max}} = e^{-\frac{1}{2} \chi^2_{\textrm{min}}}$. In this concrete setting, the PD becomes 
\be
\tilde{\mathcal{P}}(\theta_1) = e^{-\frac{1}{2} \chi^2_{\textrm{min}}(\theta_1)}, 
\ee
where $\chi^2_{\textrm{min}}(\theta_1)$ denotes the minimum value of the $\chi^2$ along the $\theta_2$ direction for a fixed $\theta_1$ value. It should not be confused with the overall minimum $\chi^2_{\textrm{min}}$, which can be extracted easily from the MCMC chain. In practice, one can also determine $\chi^2_{\textrm{min}}(\theta_1)$ from the MCMC chain by breaking the $\theta_1$ direction up into bins and finding the minimum of the $\chi^2$ for each bin. Having done so, we are in a position to define a PDF \cite{Gomez-Valent:2022hkb}: 
\be
\label{eq:w}
w(\theta_1) = \frac{e^{-\frac{1}{2} \chi^2_{\textrm{min}}(\theta_1)}}{\int e^{-\frac{1}{2} \chi^2_{\textrm{min}}(\theta_1)} \, \textrm{d} \theta_1} = \frac{R(\theta_1)}{\int R(\theta_1) \, \textrm{d} \theta_1}, 
\ee
where in the second equality we have divided top and bottom by $\mathcal{P}_{\textrm{max}} = e^{-\frac{1}{2} \chi^2_{\textrm{min}}}$. As a result, $R(\theta_1) = e^{-\frac{1}{2} \Delta \chi_{\textrm{min}}^2}$, where $\Delta \chi^2_{\textrm{min}} := \chi_{\textrm{min}}^2(\theta_1) - \chi^2_{\textrm{min}}$, so that $R(\theta_1)$ peaks at $R(\theta_1) = 1$. Note that $\int_{-\infty}^{+\infty} w(\theta_1) \, \textrm{d} \theta_1 = 1$ by construction, so $w(\theta_1)$ describes a properly normalised PDF. Thus we can identify the $1 \sigma, 2 \sigma$ and $3 \sigma$ confidence intervals corresponding to the 68\%, 95\% and 99.7\% confidence level, respectively, by simply identifying $\theta_1^{(1)}$ and $\theta_1^{(2)}$ such that \cite{Gomez-Valent:2022hkb}
\be
\label{eq:wsigma}
\int_{\theta_1^{(1)}}^{\theta_1^{(2)}} w(\theta_1) \, \textrm{d} \theta_1 = I, \quad w(\theta_1) = w(\theta_2), \quad I \in \{0.68, 0.95, 0.997\}. 
\ee
We will outline how these conditions can most easily be satisfied when we turn to explicit examples. 

Our first port of call is making sure that the PD methodology gives sensible results. This can be best judged by applying it to the CC data with $z_{\textrm{min}} = 0$, since this is where we expect a distribution closest to a Gaussian distribution, as is evident from the agreement between Fisher matrix and MCMC results in Table \ref{tab:LCDM_CC}. In particular, we will be interested in a comparison between $1 \sigma$ confidence intervals to make sure that (\ref{eq:wsigma}) is not underestimating or overestimating the $1 \sigma$ confidence interval. 

% Figure environment removed

We start by running a long MCMC chain (100,000 iterations) in order to ensure bins are well populated, and begin by analysing $\theta_1 = H_0$ with $\theta_2 = \Omega_m$. From the MCMC chain we identify the smallest and largest value of $H_0$ in the chain and break up this range into approximately 200 uniform bins, which we label using the $H_0$ value at the centre of the bin. We omit any empty bins. One can increase the number of bins by simply running a longer MCMC chain. In each $H_0$ bin we identify the minimum value of the $\chi^2$, $\chi^2_{\textrm{min}}(H_0)$, and calculate $R(H_0)$. One then repeats the steps for $\Omega_m$. In Fig.~\ref{fig:R_zmin0} we plot $R(H_0)$ against $H_0$ and $R(\Omega_m)$ against $\Omega_m$, noting that the distributions are Gaussian to first approximation. 

Since the distributions from the MCMC chain are sparse in the tails, empty bins are evident in Fig.~\ref{fig:R_zmin0}. Nevertheless, with 200 bins, modulo any empty bins, we have sufficient density of points to calculate the total area under the $R(H_0)$ and $R(\Omega_m)$ curve using Simpson's rule. Any concern about precision can simply be mitigated by running a longer MCMC chain and increasing the number of bins. 
One may directly use $R(H_0)\leq 1$ and $R(\Omega_m)\leq 1$   to find $68$, $95$ and $99.7$ percentiles,  respectively corresponding to $1 \sigma, 2 \sigma$ and $3 \sigma$ confidence intervals. Consider $F_\kappa:= \int_{R\geq \kappa} R (\theta_1) \, \textrm{d} \theta_1$, where $\kappa \leq 1$. Observe that $F_{\kappa=1}=0$ and $F_{\kappa=0}:=F_0=\int R(\theta_1) \textrm{d} \, \theta_1$. Then move $\kappa$ through and terminate the process when $F_\kappa/F_0$ is equal to $0.68$, $0.95$ and $0.997$. This gives the corresponding range for $\theta_1$ that defines the confidence interval.
Working with the precision afforded to us by approximately 200 bins, the $H_0$ and $\Omega_m$ $1 \sigma$ confidence intervals are presented in Fig.~\ref{fig:R_zmin0} and the first entry in Table \ref{tab:LCDM_CC_PD}. The outcome is in excellent agreement with both Fisher matrix and MCMC analysis. In particular, a mild non-Gaussianity in $\Omega_m$ is evident in both Fig.~\ref{fig:R_zmin0} and the errors. 
Thus, we have succeeded in recovering results in the (almost) Gaussian regime that are consistent with Fisher matrix and MCMC analysis and this provides an important check of the methodology.  

% Figure environment removed

We now apply the same PD methodology to the non-Gaussian regime where MCMC marginalisation leads to biased results. To be concrete, we focus on the eight data points in the range $1 < z < 2$ where a non-evolving $H(z)$ trend is evident in the raw data in Fig.~\ref{fig:CC}. Our goal here is to quantify the disagreement with the full data set, where one infers $H_0 \sim 68$ km/s/Mpc and $\Omega_m \sim 0.32$. A similar exercise was performed in subsection \ref{sec:features} with mock simulations and the disagreement was estimated to be approximately $2 \sigma$. Repeating the steps outlined above for the CC data with $z_{\textrm{min}} = 1$ we find the distributions in Fig.~\ref{fig:R_zmin1}. The first observation is that the distributions are non-Gaussian, but a comparison to the MCMC posteriors from the same data in blue in Fig.~\ref{fig:CCsplit1} reveals that there is no secondary $H_0$ peak at $H_0 \sim 50$ km/s/Mpc. Thus, we confirm the secondary peak to be a projection effect. That being said, the primary $H_0$ peak from Fig.~\ref{fig:CCsplit1} has shifted to the dashed line corresponding to the minimum of the $\chi^2$, since the peak of the distribution and $\chi^2$ minimum agree by construction. Comparing the blue $\Omega_m$ distribution from Fig.~\ref{fig:CCsplit1} to the $R(\Omega_m)$ distribution in Fig.~\ref{fig:R_zmin1}, we see that the peak is close to $\Omega_m = 0$ and that the tails continue to $\Omega_m = 1$. In both plots we see that there is a non-zero probability of inferring $\Omega_m = 1$. In some sense, this is not so surprising, the reason being that one is free to adopt generous priors for $H_0$, so that probability of large and small $H_0$ values is zero, but the priors on $\Omega_m$ in the flat $\Lambda$CDM model are restricted. For this reason, as a distribution spreads one invariably finds that distributions are impacted by the $\Omega_m$ priors.\footnote{Note, this is a problem for the flat $\Lambda$CDM model. In particular, one may easily find that the peak of the $\Omega_m$ distribution is larger than $\Omega_m=1$, as is the case with Hubble Space Telescope SN with redshifts $z > 1$ in the Pantheon+ sample \cite{Malekjani:2023dky}.}

It is evident from Fig.~\ref{fig:R_zmin1} that any tension that exists is confined to the $H_0$ parameter. Moreover, since there may be only one binned value of $\Omega_m$ below the $R(\Omega_m)$ peak, at the precision afforded to us by 200 bins, the $R(\Omega_m)$ distribution in Fig.~\ref{fig:R_zmin1} is essentially one-sided and the $1 \sigma$ confidence interval stretches beyond $\Omega_m \sim 0.32$, so there is no disagreement in the $\Omega_m$ parameter. Nevertheless, in the $H_0$ parameter we see that $H_0 \sim 68$ km/s/Mpc, the value favoured by the full data set is just under $2 \sigma$ removed from the peak. The main point here is that, as is obvious from the raw data, current CC data with $z > 1$ has a preference for a non-evolving Hubble parameter $H(z)$ with a large constant $H_0 \sim 150$ km/s/Mpc. The disagreement is just under $2 \sigma$, more accurately $1.9 \sigma$ from $R(H_0)$, and only $0.9 \sigma$ from $R(\Omega_m)$. Although this may not be a serious discrepancy, essentially because of the poor data quality (8 data points), this disagreement supports the $\sim 2 \sigma$ discrepancy seen in the mock simulations. It should be borne in mind that systematic uncertainties have been omitted and these will reduce this discrepancy once properly propagated. Given the agreement between the PD and mock simulation analysis, there is nothing to suggest that the three independent trends highlighted in \cite{Colgain:2022rxy} across OHD, Type Ia SN and QSOs are not \textit{bona fide} disagreements and that redshift evolution is present in the sample. The task remains to combine them at the level of a $\chi^2$ likelihood instead of combining them using Fisher's method on the basis that they are independent probabilities. We leave this exercise for a forthcoming paper, but revisit the tension in OHD data in the following section.  %\ref{sec:tension}. 
For completeness, in Table \ref{tab:LCDM_CC_PD} we perform a reanalysis of CC data subsets with the PD approach and record the $1 \sigma$ intervals.  

\begin{table}[htb]
    \centering
    \begin{tabular}{c|c|c|c}
    \rule{0pt}{3ex} $z_{\textrm{min}}$ & \# CC & \multicolumn{2}{c}{PD}  \\
    \hline
    \rule{0pt}{3ex} & & $H_0$ (km/s/Mpc) & $\Omega_m$ \\
    \hline
    \rule{0pt}{3ex} $0$ & $34$ & $68.15^{+3.04}_{-3.11}$ & $0.320^{+0.065}_{-0.055}$ \\
    \hline 
    \rule{0pt}{3ex} $0.2$ & $27$ & $65.03^{+6.52}_{-7.03}$ & $0.368^{+0.167}_{-0.110}$ \\
    \hline 
    \rule{0pt}{3ex} $0.4$ & $22$ & $62.42^{+7.78}_{-8.74}$ & $0.411^{+0.236}_{-0.113}$ \\
    \hline
    \rule{0pt}{3ex} $0.6$ & $15$ & $59.75^{+11.73}_{-13.97}$ & $0.455^{+0.355}_{-0.160}$ \\
    \hline
    \rule{0pt}{3ex} $0.7$ & $14$ & $79.10^{+16.42}_{-20.56}$ & $0.222^{+0.386}_{-0.117}$ \\
    \hline
    \rule{0pt}{3ex} $0.8$ & $11$ & $103.94^{+22.88}_{-28.54}$ & $0.097^{+0.378}_{-0.074}$ \\
    \hline
    \rule{0pt}{3ex} $1$ & $8$ & $150.35^{+17.12}_{-35.95}$ & $ < 0.339$ \\
    \hline
    \rule{0pt}{3ex} $1.2$ & $7$ & $154.26^{+14.88}_{-54.82}$ & $ < 0.570$ \\
    \hline
    \rule{0pt}{3ex} $1.4$ & $4$ & $124.81^{+35.38}_{-52.60}$ & $ < 0.661$ \\
    \hline
    \rule{0pt}{3ex} $1.5$ & $3$ & $36.11^{+72.87}_{-2.43}$ & $ > 0.354$
    \end{tabular}
    \caption{Same as Table \ref{tab:LCDM_CC} but with the PD methodology in lieu of Fisher matrix and MCMC analysis. The high redshift $R(\Omega_m)$ distributions are typically one-sided, so one encounters $1 \sigma$ upper and lower bounds.}
    \label{tab:LCDM_CC_PD}
\end{table}




\section{A tension with Planck}
\label{sec:tension}
A $2 \sigma$ ($p = 0.021$) tension with Planck has been reported in OHD through best fits and mock simulations in \cite{Colgain:2022rxy}. In particular, it was noted that a combination of 7 CC and BAO data points above $z = 1.45$ resulted in a $(H_0, \Omega_m) = (37.8, 1)$ best fit, where in line with analysis here, an $\Omega_m \in [0, 1]$ uniform prior was assumed. Based on mock simulations, the probability of such a best fit configuration arising by chance in mocks assuming input parameters consistent with Planck was estimated to be $p = 0.021$ \cite{Colgain:2022rxy}. A similar best fit appears in the last entry of Table \ref{tab:LCDM_CC} and Table \ref{tab:LCDM_CC_PD}, but there is no tension with Planck within the errors, even with our PD analysis, because CC data is inherently of poorer quality than BAO data. One further difference between the analysis is that \cite{Colgain:2022rxy} imposes a Gaussian Planck prior $\Omega_m h^2 = 0.1430 \pm 0.0011$ \cite{Planck:2018vyg} \footnote{This prior essentially prevents high redshift CC data from tracking a non-evolving $H(z)$.} to fix the high redshift behaviour of $H(z)$, whereas our analysis here so far has not introduced a prior. 

% Figure environment removed

Nevertheless, armed with a new PD methodology, we are in a position to revisit the earlier result and see if we can recover the $2 \sigma$ tension with Planck. Since \cite{Colgain:2022rxy} made use of older BAO data, here we replace QSO and Lyman-$\alpha$ BAO with the latest eBOSS results \cite{Hou:2020rse, Neveux:2020voa, duMasdesBourboux:2020pck}. Moreover, we work directly with the $D_{H}/r_d$ constraints and do not invert them. This entails assuming a value for the radius of the sound horizon, which we take to be the Planck value, $r_d = 147.09 \pm 0.26$ Mpc \cite{Planck:2018vyg}. In addition, we reinstate the prior $\Omega_m h^2 = 0.1430 \pm 0.0011$, so that the only difference with \cite{Colgain:2022rxy} is simply to update OHD BAO to the latest constraints. We stress that the priors we introduce are consistent with the Planck cosmology, so \textit{they cannot be driving any disagreement}. Moreover, the $\Omega_m h^2$ prior restricts one to a curve in the $(H_0, \Omega_m)$, but it cannot dictate where one is on the curve, this is done by the remaining 3 CC and 3 BAO data points.  

We again marginalise over the free parameters $(H_0, \Omega_m, r_d)$ with MCMC. In Fig.~\ref{fig:CC_BAO_MCMC} we present the posteriors. While $r_d$ is Gaussian and peaked on our Planck prior, as expected, the $\Omega_m$ posterior is peaked at $\Omega_m \sim 0.6$ and the fact that the fall off in the distribution is gradual beyond the peak leads to a pile up of configurations in the top left corner of the $(H_0, \Omega_m)$-plane. This fall off continues beyond $\Omega_m = 1$ and if the prior is relaxed, the $H_0$ peak shifts to smaller values. So,  once again all the hallmarks of projection effects are present. That being said, given the sharp fall off in the $\Omega_m$ distribution to smaller $\Omega_m$ values, some tension appears to be evident with the Planck values (dashed lines). 

% Figure environment removed

We now run the MCMC chain through our PD methodology. From Fig.~\ref{fig:CC_BAO}, we can see that the $R(H_0)$ and $R(\Omega_m)$ distributions prefer smaller values of $H_0$ and larger values of $\Omega_m$. The peak of the distributions occurs at $H_0 = 42.40$ km/s/Mpc and $\Omega_m = 0.795$.  The lone dot in the $R(H_0)$ distribution at low values of $H_0$ tells us that the distribution falls off sharply below $H_0 = 40$ km/s/Mpc. Note, since we employed generous uniform priors $H_0 \in [0, 200]$, the priors are not impacting the $R(H_0)$ distribution, so it is expected that the distribution falls off to zero on both sides. In contrast, the $R(\Omega_m)$ distribution is one-sided and fails to fall off in the direction of larger values within the uniform priors $\Omega_m \in [0, 1]$. The tension with Planck falls between $2 \sigma$ and $3 \sigma$. By integrating the PDF as far as the black lines corresponding to the Planck values in Fig.~\ref{fig:CC_BAO}, we estimate that the Planck $H_0$ is located at $2.1 \sigma$ from the peak, while the Planck $\Omega_m$ value is $2.5 \sigma$ from the peak.

The main take-away from this section is that OHD data comprising CC and BAO data points beyond $z=1.45$ is inconsistent with the Planck cosmology at in excess of $2 \sigma$. We have employed Planck priors to arrive at this result, but these priors cannot drive the disagreement. Moreover, independent analysis based on least squares fitting and mock simulations presented in \cite{Colgain:2022rxy} also points to a $2 \sigma$ tension, albeit with less up-to-date high redshift BAO data. In summary, different methodologies agree on a $2 \sigma$ discrepancy with Planck, which is robust to interchanging older and newer BAO data. 

\section{Concluding remarks}
\label{sec:discussion}
A $\chi^2$ likelihood is a metric or measure of how well a model fits data. The point in model parameter space that fits the data the best possesses the lowest $\chi^2$. Once one has identified this point, the problem remains to establish $1 \sigma$, $2 \sigma$, etc, confidence intervals in parameter space. In cosmology and astrophysics, MCMC is the prevailing technique for estimating confidence intervals. Its great advantage is that it allows one to i) globally sample the parameter space and ii) arrive at posteriors that serve as an estimate of the errors even with non-Gaussian distributions. In contrast, if one minimises the $\chi^2$ by gradient descent, there is always a risk that one ends up in a local minimum, i. e. the global minimum is missed, while error estimation through Fisher matrix assumes any distribution is Gaussian. The appeal of MCMC marginalisation is that it is widely applicable. However, the point of this paper is that limitations exist, even in the simplest model. 

Indeed, what happens when the MCMC posterior no longer tracks points in parameter space that fit the data better? Traditionally, volume effects are seen as the preserve of higher-dimensional models, e. g. \cite{Herold:2021ksg, Gomez-Valent:2022hkb, Meiers:2023gft}, but projection effects also occur in the minimal $\Lambda$CDM model when one fits the model to data binned by redshift in the late Universe \cite{Colgain:2022tql}. As explained in \cite{Colgain:2022tql}, this ``projection effect'' is driven by OHD, $H(z_i)$, and angular diameter or luminosity distance data, $D_{A}(z_i)$ or $D_{L}(z_i)$, {respectively} only constraining the combinations $\Omega_m h^2$ and $ (1-\Omega_m) h^2$ well, with high redshift data $z_i \gg 0$. In practice, this restricts MCMC configurations to constant $\Omega_m h^2$ and constant $(1-\Omega_m) h^2$ curves in the $(H_0, \Omega_m)$ plane, and as the curves stretch due to DE or matter being less well constrained in high redshift bins, projection effects lead to shifts in the peaks of MCMC posteriors and the emergence of non-Gaussian tails \cite{Colgain:2022tql}. We stress that one sees the same effect in PDFs of best fit $(H_0, \Omega_m)$ parameters in a large number of mock data realisations \cite{Colgain:2022tql}, so the problem is more general than MCMC; there is an inherent bias in the $\Lambda$CDM model when one fits it to redshift binned $H(z)$ \textit{or} $D_{A}(z)$ \textit{or} $D_{L}(z)$ data. Within MCMC, one sees this effect in the errors, but also in the drift of the parameters corresponding to the $\chi^2$ minimum outside of the $1 \sigma$ confidence intervals. Highlighting this (expected) bias in MCMC using OHD is the opening salvo (result) of this paper.     

Why should one care? This is evidently only a problem if one bins data and confronts the $\Lambda$CDM model. First, note that some data sets are inherently binned. For example, effective redshifts are assigned to CC and BAO analysed in a given redshift bin, while each strongly lensed system constitutes its own bin. Working with binned data is unavoidable. Secondly, $\Lambda$CDM tensions point to a problem with the $\Lambda$CDM model once the tensions become widespread and persistent. As explained in \cite{Krishnan:2020vaf}, if the minimal $\Lambda$CDM model is too simple, one expects redshift evolution of $\Lambda$CDM cosmological parameters as it is confronted to redshift binned data. Hints of these trends are now evident in $H_0$ \cite{Wong:2019kwg, Millon:2019slk, Dainotti:2021pqg, Colgain:2022nlb, Colgain:2022rxy, Malekjani:2023dky, Hu:2022kes, Jia:2022ycc, Krishnan:2020obg, Dainotti:2022bzg}, $\Omega_m$ \cite{Risaliti:2015zla, Risaliti:2018reu, Lusso:2020pdb, Yang:2019vgk, Khadka:2020vlh, Khadka:2020tlm, Khadka:2021xcc, Pourojaghi:2022zrh, Colgain:2022nlb, Colgain:2022rxy, Malekjani:2023dky, Pasten:2023rpc, Sakr:2023hrl} and $S_8$/$\sigma_8$ \cite{Esposito:2022plo, Adil:2023jtu, ACT:2023dou, ACT:2023kun} (also \cite{Miyatake:2021qjr, Alonso:2023guh}) across a host of different observables. This evolution is an expected hallmark of model breakdown, which must happen at some redshift if systematics are not universally at play. 

The main problem with redshift dependent $\Lambda$CDM cosmological parameters\footnote{There is a separate interpretation problem as the cosmology literature works with  parameters ``defined today''. In more mathematical language, this is simply the statement that one solves an ordinary differential equation (ODE), namely the Friedmann equation or equivalent, by specifying an integration constant, e.g. $H_0 = H(z=0)$ or $\rho_m(z=0)=\rho_{m0}=H_0^2\Omega_{m}$. However, this is a mathematical statement and it still needs to be confirmed observationally that $H_0$ or $\rho_{m0}$ are \textit{bona fide} constants. This cannot be \textit{a priori} assumed, because it is mathematical prediction of the model. If the model is correct, a constant $H_0$ and $\Omega_m$  will be supported by the data. See \cite{Krishnan:2020vaf} for further discussion.} is one needs to assign a statistical significance to any trend. At a purely practical level, this entails constructing bins centered on different redshifts and identifying discrepancies in $\Lambda$CDM parameters between bins, \textit{ideally in the same observable}, so that the potential systematics are under greatest control. As demonstrated both mathematically and observationally with the CC data in section \ref{sec:MCMC_bias}, MCMC marginalisation leads to biased inferences when one bins the data. In this paper we have resorted to profile distributions \cite{Gomez-Valent:2022hkb} to overcome this bias and have applied the technique to a setting where $\Lambda$CDM distributions are expected to be non-Gaussian for the reasons outlined above and in section \ref{sec:MCMC_bias}. This new technique, provides a complementary perspective that confirms the least square fits of observed and mock data presented in \cite{Colgain:2022nlb, Colgain:2022rxy, Malekjani:2023dky}, where evidence for redshift evolution in $H_0$ and $\Omega_m$ was presented. Regardless of the methodology, the objective is to drill down on the prevailing \textit{assumption} that cosmological parameters are constants. \textit{In the era of tensions in cosmology, nothing can be assumed, especially noting that the tensions are in essence showing an example of evolution of these parameters with redshift.}

More concretely, in this paper with both mock simulations and profile distributions we have shown that high redshift CC data has a preference for a non-evolving $H(z)$ over Planck-$\Lambda$CDM at approximately $\sim 2 \sigma$. This trend, which constitutes the second result of the paper, is unquestionable, as it is visible in the data. Note, we have not propagated systematic uncertainties, so the significance will be less when these are properly propagate. Nevertheless, low and high redshift CC data currently have a preference for different $\Lambda$CDM cosmological parameters. This is important because if the CC program is claiming an 8\% constraint on the Hubble constant, $H_0 = 66.7 \pm 5.5$ km/s/Mpc \cite{Moresco:2023zys}, it is imperative that \textit{all subsets of the data are consistent with this result}. If they are not, then we are staring at either systematics or model breakdown. Admittedly, demanding self-consistency of subsets of a data set confronted to a model is a high bar, but it is important that data sets result in overlapping constraints on $\Lambda$CDM parameters, otherwise this makes cosmological inferences moot. Note, the $\Lambda$CDM model is largely only well tested in the DE dominated regime $z \lesssim 1$ and at very high redshifts $z \sim 1100$, which leaves a wide expanse of redshifts to be explored in order to confirm or refute the model. Given the existing $\Lambda$CDM tensions \cite{Perivolaropoulos:2021jda, Abdalla:2022yfr}, and the hints of evolution in $H_0$, $\Omega_m$ and $S_8$ across assorted probes in the late Universe $z \lesssim 5$, it would be surprising if all discrepancies could be explained away by systematics.\footnote{We are open to the possibility, we just consider it a bad bet at the moment. The odds can of course change as observations improve.}

As an aside, it is intriguing that CC data has a preference for larger best fit values of $H_0$ and smaller best fit values of $\Omega_m$ beyond $z_{\textrm{min}} = 0.7$, as this is traditionally the transition redshift between decelerated and accelerated expansion. % where $\ddot{a} = 0$. 
Moreover, at higher redshifts $z \sim 2.3$, there is not only a longstanding anomaly in Lyman-$\alpha$ BAO \cite{duMasdesBourboux:2020pck}, but QSOs also show a preference for a lower luminosity distance, $D_{L}(z)$, relative to Planck-$\Lambda$CDM \cite{Risaliti:2015zla, Risaliti:2018reu}. Translated into $\Lambda$CDM parameters, this corresponds to conversely larger $\Omega_m$ values, e. g.  \cite{Yang:2019vgk, Khadka:2020vlh, Khadka:2020tlm, Khadka:2021xcc, Pourojaghi:2022zrh}, and consequently smaller $H_0$ values. Thus, the emerging probes CC and QSOs  \cite{Moresco:2022phi} do not appear to be in sync on high redshift $\Lambda$CDM inferences. Nevertheless, neither may be inconsistent with the anomaly in Lyman-$\alpha$ BAO. Relative to Planck-$\Lambda$CDM, Lyman-$\alpha$ BAO prefers \textit{smaller} values of $D_{M}(z) := c \int_{0}^z 1/H(z^{\prime}) \, \textrm{d} z$ and \textit{smaller} values of $H(z)$ (larger values of $D_{H}(z) := c/H(z)$).\footnote{In this statement we assumed the Planck value $r_d \sim 147$ Mpc \cite{Planck:2018vyg} If we reinstate the radius of the sound horizon in these expressions, one recognises that changing the sound horizon, as advocated by early Universe resolutions to Hubble tension, cannot consistently address the Lyman-$\alpha$ BAO anomaly. In general, even for the Planck-$\Lambda$CDM sound horizon, one cannot get both a smaller $D_{M}(z)$ and smaller $H(z)$ from a strictly increasing function, such as the $\Lambda$CDM $H(z)$. As a result, deviations from the Planck-$\Lambda$CDM model that address this anomaly are expected to lead to wiggles in $H(z)$ \cite{Akarsu:2022lhx}, which are unsurprisingly seen in data reconstructions \cite{Zhao:2017cud, Wang:2018fng, Escamilla:2021uoj}. Finally, evolution in $H_0, \Omega_m$ discussed here cannot be explained or accommodated by early resolutions to Hubble tension relying on a change in the $r_d$ at very high $z$.}. If CC data prefer less evolution in $H(z)$ in the matter-dominated regime, then this is consistent with the preference for a smaller $H(z)$ from Lyman-$\alpha$ BAO. Furthermore, QSO data prefers smaller luminosity distances $D_{L}(z)$ relative to Planck, which are consistent with the smaller $D_{M}(z) \propto D_{L}(z)$ values preferred by Lyman-$\alpha$ BAO. Thus, even if CC and QSOs appear to be showing diverging behaviour in the cosmological parameters $(H_0, \Omega_m)$, this may still turn out to be consistent with Lyman-$\alpha$ BAO. We await future DESI \cite{DESI:2023ytc} data releases to ascertain if the non-evolving $H(z)$ trend in high redshift CC data is physical or not. 

Finally, we come to our third and main result outlined in section \ref{sec:tension}. We have revisited a $\sim 2 \sigma$ tension between high redshift CC and BAO data reported in \cite{Colgain:2022rxy}, where the significance was estimated through mock simulations. Here, we have upgraded the BAO data to the latest constraints and again  recover a $>2 \sigma$ discrepancy in $(H_0, \Omega_m)$ with different methodology. This provides a consistency check that there is evolution in OHD between low and high redshifts in the late Universe. Note, this evolution runs contrary to the non-evolving $H(z)$ seen in high redshift CC data because it assumes Planck has accurately constrained the high redshift behaviour of the Hubble parameter in (\ref{eq:lcdm}). Nevertheless, both with and without a Planck prior on $\Omega_m h^2$, evolution at $ \gtrsim 2 \sigma$ is evident in OHD data. It should be stressed that evolution is evident in PDFs of best fit $\Lambda$CDM parameters fitted to a large number of Planck-$\Lambda$CDM mocks \cite{Colgain:2022tql}, so evolution in observed data can be expected. It is imperative to revisit the remaining observations in \cite{Colgain:2022rxy, Malekjani:2023dky} in order to confirm the significance of $\sim 2 \sigma$ hints of evolution found separately in Type Ia SN and QSO data sets. 




\acknowledgments
We would like to thank Adri\`a G\'omez-Valent for discussions and comments on the draft. We thank Gabriela Marques, Mike Hudson and Matteo Viel for related discussions on late Universe evolution in $S_8$. E\'OC thanks Yonsei University and Asia Pacific Center for Theoretical Physics for hospitality. 
This article/publication is based upon work from COST Action CA21136 – “Addressing observational tensions in cosmology with systematics and fundamental physics (CosmoVerse)”, supported by COST (European Cooperation in Science and Technology). SP and MMShJ acknowledge SarAmadan grant No. ISEF/M/401332. MMShJ thanks the support from ICTP associates office (under Senior Associate program) and ICTP HECAP section for hospitality.  


\appendix
\section{Fisher Matrix}
\label{sec:fisher}
Consider the $\chi^2$ (\ref{eq:chi2}). 
Defining $H_{\textrm{model}}(z) = H_0 \sqrt{1-\Omega_m + \Omega_m (1+z)^3}$ and $Q_i$ as in \eqref{eq:Q}, we can now work out the derivatives
\begin{equation}
    \begin{split}
\partial_{H_0} Q_i &= -\sqrt{1-\Omega_m + \Omega_m (1+z_i)^3}, \\  \partial_{\Omega_m} Q_i &= - \frac{1}{2} H_0 (z_i^3 + 3 z_i^2 + 3 z_i)/\sqrt{1-\Omega_m + \Omega_m (1+z_i)^3}, \\
\partial^2_{H_0} Q_i &= 0, \\
\partial_{H_0} \partial_{\Omega_m} Q_i &= - \frac{1}{2} (z_i^3 + 3 z_i^2 + 3 z_i)/\sqrt{1-\Omega_m + \Omega_m (1+z_i)^3}, \\
\partial^2_{\Omega_m} Q_i =& \frac{1}{4} H_0 (z_i^3 + 3 z_i^2 + 3 z_i)^2/(1-\Omega_m + \Omega_m (1+z_i)^3)^{\frac{3}{2}}.      
    \end{split}
\end{equation}
We can then define the Fisher matrix 
\be
F_{ij} = \frac{1}{2} \frac{\partial^2 \chi^2(H_0, \Omega_m)}{\partial p_i \partial p_j}
\ee
where $p_i \in \{ H_0, \Omega_m \}$. Note that the Fisher matrix is evaluated on the best fit parameters. The result is a $2 \times 2$ matrix, which one inverts and the estimated errors are the square root of the diagonal entries. 








\begin{thebibliography}{99}

\bibitem{Planck:2018vyg}
N.~Aghanim \textit{et al.} [Planck],
``Planck 2018 results. VI. Cosmological parameters,''
Astron. Astrophys. \textbf{641} (2020), A6
% doi:10.1051/0004-6361/201833910
%[arXiv:1807.06209 [astro-ph.CO]].

\bibitem{Riess:1998cb}
A.~G.~Riess \textit{et al.} [Supernova Search Team],
``Observational evidence from supernovae for an accelerating universe and a cosmological constant,''
Astron. J. \textbf{116} (1998), 1009-1038
% doi:10.1086/300499
%[arXiv:astro-ph/9805201 [astro-ph]].
%13031 citations counted in INSPIRE as of 02 Feb 2021

\bibitem{Perlmutter:1998np}
S.~Perlmutter \textit{et al.} [Supernova Cosmology Project],
``Measurements of $\Omega$ and $\Lambda$ from 42 high redshift supernovae,''
Astrophys. J. \textbf{517} (1999), 565-586
% doi:10.1086/307221
%[arXiv:astro-ph/9812133 [astro-ph]].
%13057 citations counted in INSPIRE as of 02 Feb 2021

\bibitem{Eisenstein:2005su}
D.~J.~Eisenstein \textit{et al.} [SDSS],
``Detection of the Baryon Acoustic Peak in the Large-Scale Correlation Function of SDSS Luminous Red Galaxies,''
Astrophys. J. \textbf{633} (2005), 560-574
%doi:10.1086/466512
%[arXiv:astro-ph/0501171 [astro-ph]].
%3380 citations counted in INSPIRE as of 08 Oct 2020

\bibitem{Riess:2021jrx}
A.~G.~Riess, W.~Yuan, L.~M.~Macri, D.~Scolnic, D.~Brout, S.~Casertano, D.~O.~Jones, Y.~Murakami, L.~Breuval and T.~G.~Brink, \textit{et al.}
``A Comprehensive Measurement of the Local Value of the Hubble Constant with 1 km s$^{?1}$ Mpc$^{?1}$ Uncertainty from the Hubble Space Telescope and the SH0ES Team,''
Astrophys. J. Lett. \textbf{934} (2022) no.1, L7
%doi:10.3847/2041-8213/ac5c5b
%[arXiv:2112.04510 [astro-ph.CO]].
%370 citations counted in INSPIRE as of 09 Jan 2023

\bibitem{Freedman:2021ahq}
W.~L.~Freedman,
``Measurements of the Hubble Constant: Tensions in Perspective,''
Astrophys. J. \textbf{919} (2021) no.1, 16
%doi:10.3847/1538-4357/ac0e95
%[arXiv:2106.15656 [astro-ph.CO]].
%179 citations counted in INSPIRE as of 09 Jan 2023

\bibitem{Pesce:2020xfe}
D.~W.~Pesce, J.~A.~Braatz, M.~J.~Reid, A.~G.~Riess, D.~Scolnic, J.~J.~Condon, F.~Gao, C.~Henkel, C.~M.~V.~Impellizzeri and C.~Y.~Kuo, \textit{et al.}
%``The Megamaser Cosmology Project. XIII. Combined Hubble constant constraints,''
Astrophys. J. Lett. \textbf{891} (2020) no.1, L1
%doi:10.3847/2041-8213/ab75f0
%[arXiv:2001.09213 [astro-ph.CO]].
%96 citations counted in INSPIRE as of 12 Jul 2021

\bibitem{Blakeslee:2021rqi}
J.~P.~Blakeslee, J.~B.~Jensen, C.~P.~Ma, P.~A.~Milne and J.~E.~Greene,
%``The Hubble Constant from Infrared Surface Brightness Fluctuation Distances,''
Astrophys. J. \textbf{911} (2021) no.1, 65
%doi:10.3847/1538-4357/abe86a
%[arXiv:2101.02221 [astro-ph.CO]].
%11 citations counted in INSPIRE as of 12 Jul 2021

\bibitem{Kourkchi:2020iyz}
E.~Kourkchi, R.~B.~Tully, G.~S.~Anand, H.~M.~Courtois, A.~Dupuy, J.~D.~Neill, L.~Rizzi and M.~Seibert,
%``Cosmicflows-4: The Calibration of Optical and Infrared Tully\textendash{}Fisher Relations,''
Astrophys. J. \textbf{896} (2020) no.1, 3
%doi:10.3847/1538-4357/ab901c
%[arXiv:2004.14499 [astro-ph.GA]].
%15 citations counted in INSPIRE as of 12 Jul 2021

\bibitem{HSC:2018mrq}
C.~Hikage \textit{et al.} [HSC],
``Cosmology from cosmic shear power spectra with Subaru Hyper Suprime-Cam first-year data,''
Publ. Astron. Soc. Jap. \textbf{71}, 43  (2019).
%doi:10.1093/pasj/psz010

\bibitem{KiDS:2020suj}
M.~Asgari \textit{et al.} [KiDS],
``KiDS-1000 Cosmology: Cosmic shear constraints and comparison between two point statistics,''
Astron. Astrophys. \textbf{645} (2021), A104
%doi:10.1051/0004-6361/202039070
%[arXiv:2007.15633 [astro-ph.CO]].
%113 citations counted in INSPIRE as of 18 Aug 2021

\bibitem{DES:2021wwk}
T.~M.~C.~Abbott \textit{et al.} [DES],
``Dark Energy Survey Year 3 results: Cosmological constraints from galaxy clustering and weak lensing,''
Phys. Rev. D \textbf{105} (2022) no.2, 023520
%doi:10.1103/PhysRevD.105.023520
%[arXiv:2105.13549 [astro-ph.CO]].
%519 citations counted in INSPIRE as of 14 Jul 2023

\bibitem{Boruah:2019icj}
S.~S.~Boruah, M.~J.~Hudson and G.~Lavaux,
``Cosmic flows in the nearby Universe: new peculiar velocities from SNe and cosmological constraints,''
Mon. Not. Roy. Astron. Soc. \textbf{498} (2020) no.2, 2703-2718
%doi:10.1093/mnras/staa2485
%[arXiv:1912.09383 [astro-ph.CO]].
%54 citations counted in INSPIRE as of 14 Jul 2023

\bibitem{Said:2020epb}
K.~Said, M.~Colless, C.~Magoulas, J.~R.~Lucey and M.~J.~Hudson,
``Joint analysis of 6dFGS and SDSS peculiar velocities for the growth rate of cosmic structure and tests of gravity,''
Mon. Not. Roy. Astron. Soc. \textbf{497} (2020) no.1, 1275-1293
%doi:10.1093/mnras/staa2032
%[arXiv:2007.04993 [astro-ph.CO]].
%49 citations counted in INSPIRE as of 14 Jul 2023

\bibitem{Perivolaropoulos:2021jda}
L.~Perivolaropoulos and F.~Skara,
``Challenges for \ensuremath{\Lambda}CDM: An update,''
New Astron. Rev. \textbf{95}, 101659  (2022).
%doi:10.1016/j.newar.2022.101659
%\href{https://arxiv.org/abs/2105.05208}{2105.05208}

\bibitem{Abdalla:2022yfr}
E.~Abdalla, G.~Franco Abell\'an, A.~Aboubrahim, A.~Agnello, O.~Akarsu, Y.~Akrami, G.~Alestas, D.~Aloni, L.~Amendola and L.~A.~Anchordoqui, \textit{et al.}
``Cosmology intertwined: A review of the particle physics, astrophysics, and cosmology associated with the cosmological tensions and anomalies,''
JHEAp \textbf{34}, 49  (2022).
%doi:10.1016/j.jheap.2022.04.002
%\href{https://arxiv.org/abs/2203.06142}{2203.06142}

\bibitem{Phillips:1993ng}
M.~M.~Phillips,
``The absolute magnitudes of Type IA supernovae,''
Astrophys. J. Lett. \textbf{413} (1993), L105-L108
%doi:10.1086/186970
%1245 citations counted in INSPIRE as of 24 Aug 2021

\bibitem{NearbySupernovaFactory:2018qkd}
M.~Rigault \textit{et al.} [Nearby Supernova Factory],
``Strong Dependence of Type Ia Supernova Standardization on the Local Specific Star Formation Rate,''
Astron. Astrophys. \textbf{644} (2020), A176
%doi:10.1051/0004-6361/201730404
%[arXiv:1806.03849 [astro-ph.CO]].
%143 citations counted in INSPIRE as of 20 Jul 2023

\bibitem{Kang:2019azh}
Y.~Kang, Y.~W.~Lee, Y.~L.~Kim, C.~Chung and C.~H.~Ree,
``Early-type Host Galaxies of Type Ia Supernovae. II. Evidence for Luminosity Evolution in Supernova Cosmology,''
Astrophys. J. \textbf{889} (2020) no.1, 8
%doi:10.3847/1538-4357/ab5afc
%[arXiv:1912.04903 [astro-ph.GA]].
%56 citations counted in INSPIRE as of 20 Jul 2023

\bibitem{Brout:2020msh}
D.~Brout and D.~Scolnic,
``It\textquoteright{}s Dust: Solving the Mysteries of the Intrinsic Scatter and Host-galaxy Dependence of Standardized Type Ia Supernova Brightnesses,''
Astrophys. J. \textbf{909} (2021) no.1, 26
%doi:10.3847/1538-4357/abd69b
%[arXiv:2004.10206 [astro-ph.CO]].
%82 citations counted in INSPIRE as of 20 Jul 2023

\bibitem{Lee:2021txi}
Y.~W.~Lee, C.~Chung, P.~Demarque, S.~Park, J.~Son and Y.~Kang,
``Evidence for strong progenitor age dependence of type Ia supernova luminosity standardization process,''
Mon. Not. Roy. Astron. Soc. \textbf{517} (2022) no.2, 2697-2708
%doi:10.1093/mnras/stac2840
%[arXiv:2107.06288 [astro-ph.GA]].
%5 citations counted in INSPIRE as of 20 Jul 2023


\bibitem{Krishnan:2020vaf}
C.~Krishnan, E.~\'O~Colg\'ain, M.~M.~Sheikh-Jabbari and T.~Yang,
``Running Hubble Tension and a H0 Diagnostic,''
Phys. Rev. D \textbf{103} (2021) no.10, 103509
%doi:10.1103/PhysRevD.103.103509
%[arXiv:2011.02858 [astro-ph.CO]].
%65 citations counted in INSPIRE as of 14 Jul 2023 

\bibitem{Krishnan:2022fzz}
C.~Krishnan and R.~Mondol,
``$H_0$ as a Universal FLRW Diagnostic,''
[arXiv:2201.13384 [astro-ph.CO]].
%12 citations counted in INSPIRE as of 14 Jul 2023

%\bibitem{Liao:2020zko}
%K.~Liao, A.~Shafieloo, R.~E.~Keeley and E.~V.~Linder,
%``Determining Model-independent H 0 and Consistency Tests,''
%Astrophys. J. Lett. \textbf{895} (2020) no.2, L29
%doi:10.3847/2041-8213/ab8dbb
%[arXiv:2002.10605 [astro-ph.CO]].
%51 citations counted in INSPIRE as of 14 Jul 2023

%\bibitem{Montani:2023xpd}
%G.~Montani, M.~De Angelis, F.~Bombacigno and N.~Carlevaro,
%``Metric $f(R)$ gravity with dynamical dark energy as a paradigm for the Hubble Tension,''
%[arXiv:2306.11101 [gr-qc]].
%1 citations counted in INSPIRE as of 14 Jul 2023

\bibitem{Wong:2019kwg}
K.~C.~Wong, S.~H.~Suyu, G.~C.~F.~Chen, C.~E.~Rusu, M.~Millon, D.~Sluse, V.~Bonvin, C.~D.~Fassnacht, S.~Taubenberger and M.~W.~Auger, \textit{et al.}
``H0LiCOW \textendash{} XIII. A 2.4 per cent measurement of H0 from lensed quasars: 5.3\ensuremath{\sigma} tension between early- and late-Universe probes,''
Mon. Not. Roy. Astron. Soc. \textbf{498} (2020) no.1, 1420-1439
%doi:10.1093/mnras/stz3094
%[arXiv:1907.04869 [astro-ph.CO]].
%804 citations counted in INSPIRE as of 18 May 2023

\bibitem{Millon:2019slk}
M.~Millon, A.~Galan, F.~Courbin, T.~Treu, S.~H.~Suyu, X.~Ding, S.~Birrer, G.~C.~F.~Chen, A.~J.~Shajib and D.~Sluse, \textit{et al.}
``TDCOSMO. I. An exploration of systematic uncertainties in the inference of $H_0$ from time-delay cosmography,''
Astron. Astrophys. \textbf{639} (2020), A101
%doi:10.1051/0004-6361/201937351
%[arXiv:1912.08027 [astro-ph.CO]].
%114 citations counted in INSPIRE as of 18 May 2023

\bibitem{Sluse:2003iy}
D.~Sluse, J.~Surdej, J.~F.~Claeskens, D.~Hutsemekers, C.~Jean, F.~Courbin, T.~Nakos, M.~Billeres and S.~V.~Khmil,
``A Quadruply imaged quasar with an optical Einstein ring candidate: 1RXS J113155.4-123155,''
Astron. Astrophys. \textbf{406} (2003), L43-L46
%doi:10.1051/0004-6361:20030904
%[arXiv:astro-ph/0307345 [astro-ph]].
%83 citations counted in INSPIRE as of 14 Jul 2023

\bibitem{Shajib:2023uig}
A.~J.~Shajib, P.~Mozumdar, G.~C.~F.~Chen, T.~Treu, M.~Cappellari, S.~Knabel, S.~H.~Suyu, V.~N.~Bennert, J.~A.~Frieman and D.~Sluse, \textit{et al.}
``TDCOSMO. XIII. Improved Hubble constant measurement from lensing time delays using spatially resolved stellar kinematics of the lens galaxy,''
Astron. Astrophys. \textbf{673} (2023), A9
%doi:10.1051/0004-6361/202345878
%[arXiv:2301.02656 [astro-ph.CO]].
%3 citations counted in INSPIRE as of 18 May 2023

\bibitem{Dainotti:2021pqg}
M.~G.~Dainotti, B.~De Simone, T.~Schiavone, G.~Montani, E.~Rinaldi and G.~Lambiase,
``On the Hubble constant tension in the SNe Ia Pantheon sample,''
Astrophys. J. \textbf{912}, 150  (2021).
%doi:10.3847/1538-4357/abeb73


\bibitem{Colgain:2022nlb}
E.~\'O~Colg\'ain, M.~M.~Sheikh-Jabbari, R.~Solomon, G.~Bargiacchi, S.~Capozziello, M.~G.~Dainotti and D.~Stojkovic,
``Revealing intrinsic flat \ensuremath{\Lambda}CDM biases with standardizable candles,''
Phys. Rev. D \textbf{106}, L041301  (2022).
%doi:10.1103/PhysRevD.106.L041301

\bibitem{Colgain:2022rxy}
E.~\'O~Colg\'ain, M.~M.~Sheikh-Jabbari, R.~Solomon, M.~G.~Dainotti and D.~Stojkovic,
``Putting Flat $\Lambda$CDM In The (Redshift) Bin,''
[arXiv:2206.11447 [astro-ph.CO]].
%42 citations counted in INSPIRE as of 14 Jul 2023

%\cite{Colgain:2022tql}
%\bibitem{Colgain:2022tql}
%E.~\'O.~Colg\'ain, M.~M.~Sheikh-Jabbari and R.~Solomon,
%``High redshift \ensuremath{\Lambda}CDM cosmology: To bin or not to bin?,''
%Phys. Dark Univ. \textbf{40} (2023), 101216
%doi:10.1016/j.dark.2023.101216
%[arXiv:2211.02129 [astro-ph.CO]].
%10 citations counted in INSPIRE as of 25 Jul 2023


\bibitem{Malekjani:2023dky}
M.~Malekjani, R.~M.~Conville, E.~\'O.~Colg\'ain, S.~Pourojaghi and M.~M.~Sheikh-Jabbari,
``Negative Dark Energy Density from High Redshift Pantheon+ Supernovae,''
[arXiv:2301.12725 [astro-ph.CO]].
%13 citations counted in INSPIRE as of 17 Jul 2023

\bibitem{Hu:2022kes}
J.~P.~Hu and F.~Y.~Wang,
``Revealing the late-time transition of H0: relieve the Hubble crisis,''
Mon. Not. Roy. Astron. Soc. \textbf{517}, 576  (2022).

\bibitem{Jia:2022ycc}
X.~D.~Jia, J.~P.~Hu and F.~Y.~Wang,
``Evidence of a decreasing trend for the Hubble constant,''
Astron. Astrophys. \textbf{674} (2023), A45
%doi:10.1051/0004-6361/202346356
%[arXiv:2212.00238 [astro-ph.CO]].
%10 citations counted in INSPIRE as of 17 Jul 2023

\bibitem{Krishnan:2020obg}
C.~Krishnan, E.~\'O~Colg\'ain, Ruchika, A.~A.~Sen, M.~M.~Sheikh-Jabbari and T.~Yang,
``Is there an early Universe solution to Hubble tension?,''
Phys. Rev. D \textbf{102} (2020) no.10, 103525
%doi:10.1103/PhysRevD.102.103525
%[arXiv:2002.06044 [astro-ph.CO]].
%69 citations counted in INSPIRE as of 17 Jul 2023

\bibitem{Dainotti:2022bzg}
M.~G.~Dainotti, B.~De Simone, T.~Schiavone, G.~Montani, E.~Rinaldi, G.~Lambiase, M.~Bogdan and S.~Ugale,
``On the Evolution of the Hubble Constant with the SNe Ia Pantheon Sample and Baryon Acoustic Oscillations: A Feasibility Study for GRB-Cosmology in 2030,''
Galaxies \textbf{10}, 24  (2022).
%doi:10.3390/galaxies10010024

\bibitem{Risaliti:2015zla}
G.~Risaliti and E.~Lusso,
``A Hubble Diagram for Quasars,''
Astrophys. J. \textbf{815} (2015), 33
%doi:10.1088/0004-637X/815/1/33
%[arXiv:1505.07118 [astro-ph.CO]].
%146 citations counted in INSPIRE as of 16 Jun 2023

\bibitem{Risaliti:2018reu}
G.~Risaliti and E.~Lusso,
``Cosmological constraints from the Hubble diagram of quasars at high redshifts,''
Nature Astron. \textbf{3}, 272  (2019).

\bibitem{Lusso:2020pdb}
E.~Lusso, G.~Risaliti, E.~Nardini, G.~Bargiacchi, M.~Benetti, S.~Bisogni, S.~Capozziello, F.~Civano, L.~Eggleston and M.~Elvis, \textit{et al.}
``Quasars as standard candles III. Validation of a new sample for cosmological studies,''
Astron. Astrophys. \textbf{642}, A150  (2020).


\bibitem{Yang:2019vgk}
T.~Yang, A.~Banerjee and E.~\'O~Colg\'ain,
``Cosmography and flat $\Lambda$CDM tensions at high redshift,''
Phys. Rev. D \textbf{102}, 123532  (2020).

\bibitem{Khadka:2020vlh}
N.~Khadka and B.~Ratra,
``Using quasar X-ray and UV flux measurements to constrain cosmological model parameters,''
Mon. Not. Roy. Astron. Soc. \textbf{497}, 263  (2020).


\bibitem{Khadka:2020tlm}
N.~Khadka and B.~Ratra,
``Determining the range of validity of quasar X-ray and UV flux measurements for constraining cosmological model parameters,''
Mon. Not. Roy. Astron. Soc. \textbf{502}, 6140  (2021).


\bibitem{Khadka:2021xcc}
N.~Khadka and B.~Ratra,
``Do quasar X-ray and UV flux measurements provide a useful test of cosmological models?,''
Mon. Not. Roy. Astron. Soc. \textbf{510}, 2753  (2022).
%doi:10.1093/mnras/stab3678

\bibitem{Pourojaghi:2022zrh}
S.~Pourojaghi, N.~F.~Zabihi and M.~Malekjani,
``Can high-redshift Hubble diagrams rule out the standard model of cosmology in the context of cosmography?,''
Phys. Rev. D \textbf{106}, 123523  (2022).


\bibitem{Zajacek:2023qjm}
M.~Zaja\v{c}ek, B.~Czerny, N.~Khadka, R.~Prince, S.~Panda, M.~L.~Mart\'\i{}nez-Aldama and B.~Ratra,
``Extinction biases quasar luminosity distances determined from quasar UV and X-ray flux measurements,''
[arXiv:2305.08179 [astro-ph.GA]].
%0 citations counted in INSPIRE as of 17 Jul 2023

\bibitem{Pasten:2023rpc}
E.~Past\'en and V.~H.~C\'ardenas,
``Testing \ensuremath{\Lambda}CDM cosmology in a binned universe: Anomalies in the deceleration parameter,''
Phys. Dark Univ. \textbf{40} (2023), 101224
%doi:10.1016/j.dark.2023.101224
%[arXiv:2301.10740 [astro-ph.CO]].

\bibitem{Wagner:2022etu}
J.~Wagner,
``Casting the $H_0$ tension as a fitting problem of cosmologies,''
[arXiv:2203.11219 [astro-ph.CO]].
%5 citations counted in INSPIRE as of 28 Jul 2023

\bibitem{Sakr:2023hrl}
Z.~Sakr,
``One matter density discrepancy to alleviate them all or further trouble for $\Lambda$CDM model,''
[arXiv:2305.02846 [astro-ph.CO]].
%0 citations counted in INSPIRE as of 24 Jul 2023


\bibitem{Colgain:2022tql}
E.~\'O~Colg\'ain, M.~M.~Sheikh-Jabbari and R.~Solomon,
``High redshift \ensuremath{\Lambda}CDM cosmology: To bin or not to bin?,''
Phys. Dark Univ. \textbf{40} (2023), 101216
%doi:10.1016/j.dark.2023.101216
[arXiv:2211.02129 [astro-ph.CO]].
%10 citations counted in INSPIRE as of 28 Jun 2023

\bibitem{Esposito:2022plo}
M.~Esposito, V.~Ir\v{s}i\v{c}, M.~Costanzi, S.~Borgani, A.~Saro and M.~Viel,
``Weighing cosmic structures with clusters of galaxies and the intergalactic medium,''
Mon. Not. Roy. Astron. Soc. \textbf{515}, 857  (2022).
%doi:10.1093/mnras/stac1825
[arXiv:2202.00974 [astro-ph.CO]].

\bibitem{Adil:2023jtu}
S.~A.~Adil, \"O.~Akarsu, M.~Malekjani, E.~\'O~Colg\'ain, S.~Pourojaghi, A.~A.~Sen and M.~M.~Sheikh-Jabbari,
``$S_8$ increases with effective redshift in $\Lambda$CDM cosmology,''
[arXiv:2303.06928 [astro-ph.CO]].
%1 citations counted in INSPIRE as of 14 Jul 2023

\bibitem{ACT:2023dou}
F.~J.~Qu \textit{et al.} [ACT],
``The Atacama Cosmology Telescope: A Measurement of the DR6 CMB Lensing Power Spectrum and its Implications for Structure Growth,''
[arXiv:2304.05202 [astro-ph.CO]].
%10 citations counted in INSPIRE as of 14 Jul 2023

\bibitem{ACT:2023kun}
M.~S.~Madhavacheril \textit{et al.} [ACT],
``The Atacama Cosmology Telescope: DR6 Gravitational Lensing Map and Cosmological Parameters,''
[arXiv:2304.05203 [astro-ph.CO]].
%10 citations counted in INSPIRE as of 14 Jul 2023

\bibitem{ACT:2023ipp}
G.~A.~Marques \textit{et al.} [ACT and DES],
``Cosmological constraints from the tomography of DES-Y3 galaxies with CMB lensing from ACT DR4,''
[arXiv:2306.17268 [astro-ph.CO]].
%0 citations counted in INSPIRE as of 14 Jul 2023

\bibitem{Miyatake:2021qjr}
H.~Miyatake, Y.~Harikane, M.~Ouchi, Y.~Ono, N.~Yamamoto, A.~J.~Nishizawa, N.~Bahcall, S.~Miyazaki and A.~A.~Plazas Malag\'on,
``First Identification of a CMB Lensing Signal Produced by 1.5~Million Galaxies at z\ensuremath{\sim}4: Constraints on Matter Density Fluctuations at High Redshift,''
Phys. Rev. Lett. \textbf{129} (2022) no.6, 061301
%doi:10.1103/PhysRevLett.129.061301
[arXiv:2103.15862 [astro-ph.CO]].
%7 citations counted in INSPIRE as of 25 Jul 2023

\bibitem{Alonso:2023guh}
D.~Alonso, G.~Fabbian, K.~Storey-Fisher, A.~C.~Eilers, C.~Garc\'\i{}a-Garc\'\i{}a, D.~W.~Hogg and H.~W.~Rix,
``Constraining cosmology with the Gaia-unWISE Quasar Catalog and CMB lensing: structure growth,''
[arXiv:2306.17748 [astro-ph.CO]].
%0 citations counted in INSPIRE as of 25 Jul 2023


\bibitem{Herold:2021ksg}
L.~Herold, E.~G.~M.~Ferreira and E.~Komatsu,
``New Constraint on Early Dark Energy from Planck and BOSS Data Using the Profile Likelihood,''
Astrophys. J. Lett. \textbf{929} (2022) no.1, L16
%doi:10.3847/2041-8213/ac63a3
%[arXiv:2112.12140 [astro-ph.CO]].
%43 citations counted in INSPIRE as of 17 Jul 2023

\bibitem{Gomez-Valent:2022hkb}
A.~G\'omez-Valent,
``Fast test to assess the impact of marginalization in Monte~Carlo analyses and its application to cosmology,''
Phys. Rev. D \textbf{106} (2022) no.6, 063506
%doi:10.1103/PhysRevD.106.063506
%[arXiv:2203.16285 [astro-ph.CO]].
%20 citations counted in INSPIRE as of 11 Jul 2023

\bibitem{Meiers:2023gft}
M.~Meiers, L.~Knox and N.~Sch\"oneberg,
``Exploration of the Pre-recombination Universe with a High-Dimensional Model of an Additional Dark Fluid,''
[arXiv:2307.09522 [astro-ph.CO]].
%0 citations counted in INSPIRE as of 22 Jul 2023

\bibitem{Poulin:2018cxd}
V.~Poulin, T.~L.~Smith, T.~Karwal and M.~Kamionkowski,
``Early Dark Energy Can Resolve The Hubble Tension,''
Phys. Rev. Lett. \textbf{122} (2019) no.22, 221301
%doi:10.1103/PhysRevLett.122.221301
%[arXiv:1811.04083 [astro-ph.CO]].
%608 citations counted in INSPIRE as of 17 Jul 2023

\bibitem{Niedermann:2019olb}
F.~Niedermann and M.~S.~Sloth,
``New early dark energy,''
Phys. Rev. D \textbf{103} (2021) no.4, L041303
%doi:10.1103/PhysRevD.103.L041303
[arXiv:1910.10739 [astro-ph.CO]].
%140 citations counted in INSPIRE as of 24 Jul 2023

\bibitem{Jimenez:2001gg}
R.~Jimenez and A.~Loeb,
``Constraining cosmological parameters based on relative galaxy ages,''
Astrophys. J. \textbf{573} (2002), 37-42
%doi:10.1086/340549
%[arXiv:astro-ph/0106145 [astro-ph]].
%598 citations counted in INSPIRE as of 28 Jun 2023

\bibitem{Stern:2009ep}
D.~Stern, R.~Jimenez, L.~Verde, M.~Kamionkowski and S.~A.~Stanford,
``Cosmic Chronometers: Constraining the Equation of State of Dark Energy. I: H(z) Measurements,''
JCAP \textbf{02} (2010), 008
%doi:10.1088/1475-7516/2010/02/008
%[arXiv:0907.3149 [astro-ph.CO]].
%740 citations counted in INSPIRE as of 20 May 2022

\bibitem{Moresco:2012jh}
M.~Moresco, A.~Cimatti, R.~Jimenez, L.~Pozzetti, G.~Zamorani, M.~Bolzonella, J.~Dunlop, F.~Lamareille, M.~Mignoli and H.~Pearce, \textit{et al.}
``Improved constraints on the expansion rate of the Universe up to z\textasciitilde{}1.1 from the spectroscopic evolution of cosmic chronometers,''
JCAP \textbf{08} (2012), 006
%doi:10.1088/1475-7516/2012/08/006
%[arXiv:1201.3609 [astro-ph.CO]].
%508 citations counted in INSPIRE as of 20 May 2022

\bibitem{Zhang:2012mp}
C.~Zhang, H.~Zhang, S.~Yuan, T.~J.~Zhang and Y.~C.~Sun,
``Four new observational $H(z)$ data from luminous red galaxies in the Sloan Digital Sky Survey data release seven,''
Res. Astron. Astrophys. \textbf{14} (2014) no.10, 1221-1233
%doi:10.1088/1674-4527/14/10/002
%[arXiv:1207.4541 [astro-ph.CO]].
%425 citations counted in INSPIRE as of 20 May 2022

\bibitem{Moresco:2016mzx}
M.~Moresco, L.~Pozzetti, A.~Cimatti, R.~Jimenez, C.~Maraston, L.~Verde, D.~Thomas, A.~Citro, R.~Tojeiro and D.~Wilkinson,
``A 6\% measurement of the Hubble parameter at $z\sim0.45$: direct evidence of the epoch of cosmic re-acceleration,''
JCAP \textbf{05} (2016), 014
%doi:10.1088/1475-7516/2016/05/014
%[arXiv:1601.01701 [astro-ph.CO]].
%505 citations counted in INSPIRE as of 17 May 2022

\bibitem{Ratsimbazafy:2017vga}
A.~L.~Ratsimbazafy, S.~I.~Loubser, S.~M.~Crawford, C.~M.~Cress, B.~A.~Bassett, R.~C.~Nichol and P.~V\"ais\"anen,
``Age-dating Luminous Red Galaxies observed with the Southern African Large Telescope,''
Mon. Not. Roy. Astron. Soc. \textbf{467} (2017) no.3, 3239-3254
%doi:10.1093/mnras/stx301
%[arXiv:1702.00418 [astro-ph.CO]].
%162 citations counted in INSPIRE as of 17 May 2022

\bibitem{Borghi:2021rft}
N.~Borghi, M.~Moresco and A.~Cimatti,
``Toward a Better Understanding of Cosmic Chronometers: A New Measurement of H(z) at z \ensuremath{\sim} 0.7,''
Astrophys. J. Lett. \textbf{928} (2022) no.1, L4
%doi:10.3847/2041-8213/ac3fb2
%[arXiv:2110.04304 [astro-ph.CO]].
%10 citations counted in INSPIRE as of 17 May 2022

\bibitem{Jiao:2022aep}
K.~Jiao, N.~Borghi, M.~Moresco and T.~J.~Zhang,
``New Observational H(z) Data from Full-spectrum Fitting of Cosmic Chronometers in the LEGA-C Survey,''
Astrophys. J. Suppl. \textbf{265} (2023) no.2, 48
%doi:10.3847/1538-4365/acbc77
%[arXiv:2205.05701 [astro-ph.CO]].
%14 citations counted in INSPIRE as of 17 Jul 2023

\bibitem{Tomasetti:2023kek}
E.~Tomasetti, M.~Moresco, N.~Borghi, K.~Jiao, A.~Cimatti, L.~Pozzetti, A.~C.~Carnall, R.~J.~McLure and L.~Pentericci,
``A new measurement of the expansion history of the Universe at z=1.26 with cosmic chronometers in VANDELS,''
[arXiv:2305.16387 [astro-ph.CO]].
%1 citations counted in INSPIRE as of 28 Jun 2023

\bibitem{Moresco:2023zys}
M.~Moresco,
``Addressing the Hubble tension with cosmic chronometers,''
[arXiv:2307.09501 [astro-ph.CO]].
%0 citations counted in INSPIRE as of 24 Jul 2023

\bibitem{Moresco:2020fbm}
M.~Moresco, R.~Jimenez, L.~Verde, A.~Cimatti and L.~Pozzetti,
``Setting the Stage for Cosmic Chronometers. II. Impact of Stellar Population Synthesis Models Systematics and Full Covariance Matrix,''
Astrophys. J. \textbf{898} (2020) no.1, 82
%doi:10.3847/1538-4357/ab9eb0
[arXiv:2003.07362 [astro-ph.GA]].
%57 citations counted in INSPIRE as of 28 Jul 2023

\bibitem{Foreman-Mackey:2012any}
D.~Foreman-Mackey, D.~W.~Hogg, D.~Lang and J.~Goodman,
``emcee: The MCMC Hammer,''
Publ. Astron. Soc. Pac. \textbf{125} (2013), 306-312
%doi:10.1086/670067
%[arXiv:1202.3665 [astro-ph.IM]].
%3393 citations counted in INSPIRE as of 17 Jul 2023


\bibitem{Hou:2020rse}
J.~Hou, A.~G.~S\'anchez, A.~J.~Ross, A.~Smith, R.~Neveux, J.~Bautista, E.~Burtin, C.~Zhao, R.~Scoccimarro and K.~S.~Dawson, \textit{et al.}
``The Completed SDSS-IV extended Baryon Oscillation Spectroscopic Survey: BAO and RSD measurements from anisotropic clustering analysis of the Quasar Sample in configuration space between redshift 0.8 and 2.2,''
Mon. Not. Roy. Astron. Soc. \textbf{500} (2020) no.1, 1201-1221
%:10.1093/mnras/staa3234
%[arXiv:2007.08998 [astro-ph.CO]].
%135 citations counted in INSPIRE as of 28 Jun 2023

\bibitem{Neveux:2020voa}
R.~Neveux, E.~Burtin, A.~de Mattia, A.~Smith, A.~J.~Ross, J.~Hou, J.~Bautista, J.~Brinkmann, C.~H.~Chuang and K.~S.~Dawson, \textit{et al.}
``The completed SDSS-IV extended Baryon Oscillation Spectroscopic Survey: BAO and RSD measurements from the anisotropic power spectrum of the quasar sample between redshift 0.8 and 2.2,''
Mon. Not. Roy. Astron. Soc. \textbf{499} (2020) no.1, 210-229
%doi:10.1093/mnras/staa2780
%[arXiv:2007.08999 [astro-ph.CO]].
%133 citations counted in INSPIRE as of 28 Jun 2023

\bibitem{duMasdesBourboux:2020pck}
H.~du Mas des Bourboux, J.~Rich, A.~Font-Ribera, V.~de Sainte Agathe, J.~Farr, T.~Etourneau, J.~M.~Le Goff, A.~Cuceu, C.~Balland and J.~E.~Bautista, \textit{et al.}
``The Completed SDSS-IV Extended Baryon Oscillation Spectroscopic Survey: Baryon Acoustic Oscillations with Ly\ensuremath{\alpha} Forests,''
Astrophys. J. \textbf{901} (2020) no.2, 153
%doi:10.3847/1538-4357/abb085
%[arXiv:2007.08995 [astro-ph.CO]].
%172 citations counted in INSPIRE as of 28 Jun 2023

\bibitem{Trotta:2017wnx}
R.~Trotta,
``Bayesian Methods in Cosmology,''
[arXiv:1701.01467 [astro-ph.CO]].
%96 citations counted in INSPIRE as of 18 Jul 2023

\bibitem{Moresco:2022phi}
M.~Moresco, L.~Amati, L.~Amendola, S.~Birrer, J.~P.~Blakeslee, M.~Cantiello, A.~Cimatti, J.~Darling, M.~Della Valle and M.~Fishbach, \textit{et al.}
``Unveiling the Universe with emerging cosmological probes,''
Living Rev. Rel. \textbf{25} (2022) no.1, 6
%doi:10.1007/s41114-022-00040-z
%[arXiv:2201.07241 [astro-ph.CO]].
%71 citations counted in INSPIRE as of 16 Jun 2023

\bibitem{DESI:2023ytc}
G.~Adame \textit{et al.} [DESI],
``The Early Data Release of the Dark Energy Spectroscopic Instrument,''
%doi:10.5281/zenodo.7964161
[arXiv:2306.06308 [astro-ph.CO]].
%13 citations counted in INSPIRE as of 26 Jul 2023

\bibitem{Akarsu:2022lhx}
O.~Akarsu, E.~\'O~Colg\'ain, E.~\"Ozulker, S.~Thakur and L.~Yin,
``Inevitable manifestation of wiggles in the expansion of the late Universe,''
Phys. Rev. D \textbf{107} (2023) no.12, 123526
%doi:10.1103/PhysRevD.107.123526
%[arXiv:2207.10609 [astro-ph.CO]].
%6 citations counted in INSPIRE as of 17 Jul 2023

\bibitem{Zhao:2017cud}
G.~B.~Zhao, M.~Raveri, L.~Pogosian, Y.~Wang, R.~G.~Crittenden, W.~J.~Handley, W.~J.~Percival, F.~Beutler, J.~Brinkmann and C.~H.~Chuang, \textit{et al.}
``Dynamical dark energy in light of the latest observations,''
Nature Astron. \textbf{1} (2017) no.9, 627-632
%doi:10.1038/s41550-017-0216-z
%[arXiv:1701.08165 [astro-ph.CO]].
%356 citations counted in INSPIRE as of 17 Jul 2023

\bibitem{Wang:2018fng}
Y.~Wang, L.~Pogosian, G.~B.~Zhao and A.~Zucca,
``Evolution of dark energy reconstructed from the latest observations,''
Astrophys. J. Lett. \textbf{869} (2018), L8
%doi:10.3847/2041-8213/aaf238
%[arXiv:1807.03772 [astro-ph.CO]].
%92 citations counted in INSPIRE as of 17 Jul 2023

\bibitem{Escamilla:2021uoj}
L.~A.~Escamilla and J.~A.~Vazquez,
``Model selection applied to reconstructions of the Dark Energy,''
Eur. Phys. J. C \textbf{83} (2023) no.3, 251
%doi:10.1140/epjc/s10052-023-11404-2
%[arXiv:2111.10457 [astro-ph.CO]].
%13 citations counted in INSPIRE as of 17 Jul 2023

\end{thebibliography}
\end{document}



\end{document}