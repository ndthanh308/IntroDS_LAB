 \documentclass[reqno]{amsart}

\usepackage{amsfonts}
\usepackage{amssymb}
\usepackage{amsmath}
\usepackage{enumitem}
\usepackage{xcolor}
\usepackage{todonotes}

\setenumerate{label={\rm (\alph{*})}}
\usepackage{bbm,bm,euscript,mathrsfs}
\usepackage{paper_diening}
\usepackage{graphicx}
%\usepackage{showkeys}
\usepackage[top=1in, bottom=1.25in, left=1.10in, right=1.10in]{geometry}
%%%%%%%%%%%%%%%%%%%%%%%%%%%%%%%%%%%%%%%%%%%%%%%%%%%%%%%%
\usepackage{hyperref}%added by Pei
\usepackage{cases, color,amsthm}%added by Pei
\usepackage{upref}%added by Pei
\hypersetup{linkcolor=blue, colorlinks=true,citecolor = red}%added by Pei
%%%%%%%%%%%%%%%%%%%%%%%%%%%%%%%%%%%%%%%%%%%%%%%%%%%%%%%%
%\usepackage{refcheck}

\allowdisplaybreaks

\numberwithin{equation}{section}

\usepackage{tikz-cd}
\usetikzlibrary{lindenmayersystems}
\usetikzlibrary{decorations.pathreplacing}

\DeclareMathOperator{\dist}{dis}
\DeclareMathOperator{\Bog}{Bog}
\DeclareMathOperator{\tr}{tr}
\DeclareMathOperator{\Div}{div}
\DeclareMathOperator{\divy}{div_{\bf y}}
\DeclareMathOperator{\diver}{div}
\newcommand{\R}{\mathbb R}
\newcommand{\N}{\mathbb N}
\newcommand{\dd}{\mathrm d}
\newcommand{\dt}{\,\mathrm{d} t}
\newcommand{\s}{\mathbb S}
\newcommand{\ep}{\bfvarepsilon}
\newcommand{\eps}{\bfvarepsilon}

\renewcommand{\bfB}{B}

\DeclareMathOperator{\Test}{{\mathscr{F}}^{\Div}}
\newcommand{\Testeta}{{\mathscr{F}}^{\Div}_{\eta}}
\newcommand{\Testzeta}{{\mathscr{F}}^{\Div}_{\eta}}
\newcommand{\Testetar}{\Test_{\eta^\varrho}}



%\newcommand{\forall} {{\hbox{$\hskip 11mm \text{ for all } \;$}}}
%added by Pei
%%%%%%%% New commands %%%%%%%%

\newcommand{\bx}{\mathbf{x}}
\newcommand{\by}{\mathbf{y}}
\newcommand{\bz}{\mathbf{z}}
\newcommand{\bq}{\mathbf{q}}
\newcommand{\bu}{\mathbf{v}}
\newcommand{\vu}{\mathbf{v}}
\newcommand{\bv}{\mathbf{v}}
\newcommand{\bn}{\mathbf{n}}
\newcommand{\be}{\mathbf{e}}
\newcommand{\vc}[1]{{\bf #1}}

\newcommand{\dy}{\, \mathrm{d}\mathbf{y}}
\newcommand{\dq}{\, \mathrm{d} \mathbf{q}}
\newcommand{\dx}{\, \mathrm{d} \mathbf{x}}
\newcommand{\dqt}{\,\mathrm{d}\mathbf{q}\, \mathrm{d}t}
\newcommand{\dqxt}{\,\mathrm{d}\mathbf{q}\,\mathrm{d}\mathbf{x}\, \mathrm{d}t}

\newcommand{\divx}{\mathrm{div} }
\newcommand{\divq}{\mathrm{div}_{\mathbf{q}}}
\newcommand{\divqi}{\mathrm{div}_{\mathbf{q}_i}}
\newcommand{\divqj}{\mathrm{div}_{\mathbf{q}_j}}
\newcommand{\nabx}{\nabla }
\newcommand{\naby}{\nabla_{\mathbf{y}}}
\newcommand{\nabq}{\nabla_{\mathbf{q}}}
\newcommand{\nabqr}{\nabla_{\mathbf{q}_r}}
\newcommand{\nabql}{\nabla_{\mathbf{q}_l}}
\newcommand{\nabqi}{\nabla_{\mathbf{q}_i}}
\newcommand{\nabqj}{\nabla_{\mathbf{q}_j}}
\newcommand{\Delx}{\Delta }
\newcommand{\Dely}{\Delta_{\mathbf{y}}}

\newcommand{\D}{\mathrm{D}}
\newcommand{\dH}{\,\mathrm{d}\mathbf{y}}
\newcommand{\dxt}{\,\mathrm{d}x\,\mathrm{d}t}
\newcommand{\ds}{\,\mathrm{d}\sigma}
\newcommand{\dxs}{\,\mathrm{d}x\,\mathrm{d}\sigma}
\newcommand{\Oeta}{\Omega_\eta}

\newtheorem{theorem}{Theorem}[section]
\newtheorem{lemma}[theorem]{Lemma}
\newtheorem{proposition}[theorem]{Proposition}
\newtheorem{corollary}[theorem]{Corollary}
\newtheorem{remark}[theorem]{Remark}

\theoremstyle{definition}
\newtheorem{definition}[theorem]{Definition}
\newtheorem{example}[theorem]{Example}
\newtheorem{exercise}[theorem]{Exercise}

\newcommand{\db}[1]{\textcolor[rgb]{0.00,0.00,1.00}{  #1}}
\newcommand{\seb}[1]{\textcolor[rgb]{0.00,0.60,0.20}{  #1}}
\begin{document}

\title[Serrin for FSI]{Ladyzhenskaya-Prodi-Serrin condition for fluid-structure interaction systems}
%{The Serrin condition for a fluid-structure interaction system}

%    Information for first author
\author{Dominic Breit}
\address{Institute of Mathematics, TU Clausthal, Erzstra\ss e 1, 38678 Clausthal-Zellerfeld, Germany}
\email{dominic.breit@tu-clausthal.de}

\author{Prince Romeo Mensah}
\address{Institute of Mathematics, TU Clausthal, Erzstra\ss e 1, 38678 Clausthal-Zellerfeld, Germany}
\email{prince.romeo.mensah@tu-clausthal.de}

\author{Sebastian Schwarzacher}
\address{Department of Mathematical Analysis,
	Faculty of Mathematics and Physics,
	Charles University,
	Sokolovská 83,
	186 75 Praha 8, Czech Republic}
\address{and}
\address{Department of Mathematics, 
	Analysis and Partial Differential Equations, 
	Uppsala University, 
	L\"agerhyddsv\"agen 1,
	752 37 Uppsala, Sweden}
\email{schwarz@karlin.mff.cuni.cz}

\author{Pei Su}
\address{Department of Mathematical Analysis,
	Faculty of Mathematics and Physics,
	Charles University,
	Sokolovská 83,
	186 75 Praha 8, Czech Republic}
\email{peisu@karlin.mff.cuni.cz}
%\urladdr{URL}
%\thanks{}


%\thanks{Support information for the second author.}

%    General info
\subjclass[2020]{35B65, 35Q74, 35R37, 76D03, 74F10, 74K25}

\date{\today}

%\dedicatory{This paper is dedicated to our advisors.}

\keywords{Incompressible Navier-Stokes system, Viscoelastic shell equation, Fluid-Structure interaction, Strong solutions, Weak-strong uniqueness, Stability estimates.}

\begin{abstract}

We consider the interaction of a viscous incompressible fluid with a flexible shell in three space dimensions. The fluid is described by the three-dimensional incompressible Navier--Stokes equations in a domain that is changing in  accordance with the motion of the structure. The displacement of the latter evolves along a visco-elastic shell equation. Both are coupled through kinematic boundary conditions and the balance of forces.


We prove a counterpart of the classical Ladyzhenskaya-Prodi-Serrin condition yielding conditional regularity and uniqueness of a solution. %The only additional assumption that we require on the structure is that it is continuously differentiable in space. This is just an instant of regularity more than a weak solution enjoys.    

Our result is a consequence of the following three ingredients which might be of independent interest: {\bf (i)} the existence of local strong solutions, {\bf (ii)} an acceleration estimate (under the Serrin assumption) ultimately controlling the second-order energy norm, and {\bf (iii)} a weak-strong uniqueness theorem.
The first point, and to some extent, the last point were previously known for the case of elastic plates, which means that the relaxed state is flat. We extend these results to the case of visco-elastic shells, which means that more general reference geometries are considered such as cylinders or spheres. The second point, i.e. the acceleration estimate for three-dimensional fluids is new even in the case of plates.
\end{abstract}

\maketitle

\section{Introduction}
When three-dimensional Navier-Stokes equations are considered, only conditional smoothness and uniqueness of weak solutions are known for large data and time. The condition is that the fluid velocity satisfies some integrability \textit{beyond the natural energy estimate} that overcomes a certain scaling, namely\footnote{Here and later, we use $I=(0,T)$ as the time interval. Further notations can be found in the next section.}
\begin{align}
\label{eq:serrin}
\bu\in L^r(I;L^s(\Omega)), \qquad\tfrac{2}{r}+\tfrac{3}{s}= 1, \qquad 2\leq r<\infty. 
\end{align}
The above criterion is known as the
\textit{Ladyzhenskaya-Prodi-Serrin condition}, referring to the works by Prodi \cite{prodi1959} and Serrin \cite{serrin1962,serrin1963} on proving conditional uniqueness as well as that of Ladyzhenskaya \cite{lady} showing conditional regularity of solutions. Summarizing this means, if \eqref{eq:serrin} is satisfied the solution is {\bf a)} smooth, and {\bf b)} unique within all {\em weak-solutions satisfying an energy inequality}. The latter property is often referred to as {\em weak-strong-uniqueness}.


Many authors have since contributed to the generalization of this criterion~\cite{beale1984remark, beirao1995new, chen2006space, escauriaza, kozono2002critical, kozono2004bilinear, kozono2000limiting}. In particular, in recent years, seminal studies related to the borderline case $s=3$ indicate that the condition could potentially be sharp~\cite{ABC22,escauriaza,Sve14,JiaSve15}. As the {\em physical indications of non-uniqueness} are usually (necessarily) present in fluid-structure interaction problems, it seems worthwhile investigating how far the seminal work by Ladyzhenskaya, Prodi, Serrin and many others still holds true in this framework. {\em The aim of this paper is to advance this theory to the framework of elastic deformable shells interacting with the incompressible Navier-Stokes equation.} A second more practical motivation of our study is its potential application for numerical analysis. Indeed, the analysis here in particular shows that strong solutions are attractors, which is a first step towards convergence results. See Remark~\ref{rem:stab} for more details.

In the context of a fluid-structure interaction problem the domain of the fluid varies with time with respect to the evolution of the structure. Hence an estimate for the difference of two solutions cannot be directly obtained even when both solutions are smooth. This is already the case when a single rigid body is moving inside the fluid. For that regime, rather recently, the Ladyzhenskaya-Prodi-Serrin condition has been extended for the motion of a rigid ball immersed in a viscous incompressible fluid ~\cite{chemetov2019weak,maity2023uniqueness, muha2022regularity}. In that context falls also the uniqueness result for weak solutions in two dimensions~\cite{GlaSue15}.

The situation becomes even more dramatic, when flexible materials are considered that change the domain in an asymmetric fashion.

In this work we study {\em curved reference configurations} (see Figure~\ref{fig:2}). The most prominent reference geometries for shells are cylinders, that relate for example to the very relevant application of blood-flow or balls, relating for example to the motion of a balloon. But certainly many more complicated reference geometries may appear in applications. 
%\textcolor{red}{It turned out that mathematically speaking the approach of \cite{schwarzacher2022weak} was not easily applicable to the case of shells. We, therefore, follow a different approach.} 
In short, we derive the following three novel results to be found in sections three, four and five that might each be of independent interest:
\begin{enumerate}
\item[Section 3] {\bf Local strong solutions.} We show the existence of a smooth solution for short times.
\item[Section 4] {\bf The acceleration estimate.} Here we show that as long as the fluid velocity satisfies \eqref{eq:serrin} and the displacement of the shell stays $C^1$ in space, the solutions satisfies some extra smoothness. This section relates to  the conditional smoothness of solutions {\bf a)}.
\item[Section 5] {\bf Weak-strong uniqueness.} Finally, in this section, it is shown that the constructed smooth solution is unique in the regime of weak solutions satisfying an energy estimate and possessing a bi-Lipschitz-in-space shell displacement; hence, conditional uniqueness is shown {\bf b)}.
\end{enumerate} 

The only additional assumption for a weak solution to be smooth and unique that we require on the shell displacement is that it is $C^1$ in space. As we will explain below, this is just an instant of regularity more than a weak solution enjoys.

The first point and, to some extent, the last point above were previously known for the case of elastic plates, which means that the relaxed state is flat. 
The latter one is also the first weak-strong uniqueness result in the context of fluid-structure interaction with flexible structure~\cite{schwarzacher2022weak}. The proof there relies heavily on the fact that the reference configuration is flat.
The second point above, the acceleration estimate for three-dimensional fluids, is new even in the case of plates.


\subsection{Analysis of fluid-structure interactions}
The results presented here strongly connect to previous works on fluid-structure interactions involving elastic structures interacting with an unsteady three-dimensional viscous incompressible fluid. Most results are on the existence theory. We refer to~\cite{sunny} for an overview of the setting considered in this paper and to~\cite{kaltenbacher} for various subjects in fluid-solid interactions. We may broadly classify these body of work into the construction of strong solutions and weak solutions for a viscous fluid interacting with an elastic structure.

For weak solutions,  a semi-group approach is used in \cite{barbu2007existence} in the construction of global-in-time weak solutions  for the interaction between a stationary elastic solid  immersed in a  viscous incompressible fluid, where the interaction happens at the boundary of the solid. The same authors then show in \cite{barbu2008smoothness} that for smooth enough data,  the weak solutions constructed in \cite{barbu2007existence} become smooth.
The existence of a weak solution is also shown in  \cite{boulakia2007existence} for a regularized three-dimensional elastic structure immersed in
an incompressible viscous fluid contained in a fixed bounded connected domain. These solutions exist
as long as deformations of the elastic solid are sufficiently small  and no collisions occurs between the structure and the boundary. However, large translations and rotations of the structure are accounted for. 
The authors in \cite{chambolle2005existence} used a Galerkin method to show the existence of a weak solution to the  three-dimensional  Navier--Stokes equations
coupled with a two-dimensional elastic plate model that is modified to include viscous effects. This weak solution exists as long as  the structure does not touch the fixed part of the fluid boundary. The viscous effects incorporated in the plate model is then removed in \cite{grandmont2008existence} by passing to the limit as the coefficient modelling the viscoelasticity tends to zero.
  The seminal work \cite{coutand2005motion} explores the motion of the linear Kirchhoff elastic solid material
 inside an incompressible viscous fluid. A topological fixed-point argument is used to construct a local-in-time weak solution which is then shown to be regular and unique. 
The authors in \cite{LeRu} study the interaction of an incompressible Newtonian fluid with a linearly elastic Koiter shell. Here, the fluid's boundary is described by the mid-section of the shell and the authors show the existence of weak solutions, without self-intersections of the shell, using an Aubin--Lions type argument. Eventually, an existence result for the fully nonlinear Koiter shell model has been proved in \cite{MuSc}.
%A model for blood flow through arteries are explored in \cite{MuCa1, muha2016existence} where a semi-discrete in time approximation method is used to construct weak solutions.

When it comes to strong solutions, short time existence and uniqueness of solutions in Sobolev spaces are studied in \cite{cheng2007navier, CS} for  a viscous incompressible fluid interacting with a nonlinear thin elastic  shell. The shell equation for the former \cite{cheng2007navier} is modelled by the nonlinear Saint-Venant-Kirchhoff constitutive law, whereas that of the latter \cite{CS}  is modelled by the nonlinear Koiter shell model.  In \cite{coutand2006interaction}, however, the authors prove the existence of a unique local strong solution, without restriction on the size of the data, when the elastic structure is now governed by quasilinear elastodynamics. 
%A simplified model for describing blood flow through viscoelastic arteries is explored in \cite{GraHil}. Here a global-in-time strong solution is constructed for the coupling of a one-dimensional viscoelastic beam equation and the two-dimensional incompressible Navier--Stokes equation. 
%In particular, the authors show that contact between the viscoelastic wall and the bottom of the fluid cavity does not occur in finite time.
In \cite{ignatova2014well}, the elastic structure is modelled by a damped wave equation  with  additional boundary stabilization terms. For sufficiently small initial data, subject to said boundary stabilization terms,
global-in-time existence of strong solutions and exponential decay of the solutions are shown. 
The free boundary fluid-structure interaction problem consisting of a Navier--Stokes equation and a wave equation defined in two different but adjacent domains is studied in \cite{kukavica2012solutions}. A local strong solution is constructed under suitable compatibility conditions for the data.
Another local-in-time strong existence result is \cite{DRR} where the viscous Newtonian fluid is now interacting with an elastic structure modelled by a nonlinear damped shell equation.
Finally, a local strong solution is constructed for the motion of a linearly elastic Lam\'e solid moving in a viscous fluid in \cite{raymond2014fluid}. For the problem \eqref{1}--\eqref{interfaceCond} below, the only available local existence results deal with the case of a flat geometry, see \cite{CS} and \cite{DRR}, while the existence of global strong solutions in 2D is proved in \cite{GraHil} for flat geometry and in \cite{Br} for linear elastic shells in general geometries. A corresponding result for the 3D case is not yet known.

%\subsection{Conditional regularity for the Navier--Stokes equations}
%Although strong solutions of the Navier--Stokes equation are only known to exist on a short time interval or locally in time, there are several conditions under which these solutions can be extended in time until a possible blow-up occurs. The most prominent one is \textit{Serrin's criterion} \cite{serrin1962,serrin1963}
%\begin{align*}
%\bu\in L^r(I;L^s(\Omega)), \qquad\tfrac{2}{r}+\tfrac{3}{s}= 1, \qquad 2\leq r<\infty 
%\end{align*} 
%linking integrability in time of the fluid's velocity with integrability in space. 


%On the one hand, we know that a finite-energy-weak solution of the Navier--Stokes equation exist globally in time but with no information on uniqueness. On the other hand, for a more regular data, a unique local strong solution exists for the Navier-Stokes equation. Serrin's weak-strong uniqueness theorem ensures  that the finite-energy-weak solution coincides with the local strong solution
%once the latter exists, leading to uniqueness for this class of weak solutions.  
% The extension of  Serrin's weak-strong uniqueness result was thus unclear until recently when the authors in
%\cite{chemetov2019weak} obtained a weak-strong uniqueness result for a fluid-rigid body system.
% For a fluid-rigid body system filling the entire $3$-dimensional space, the authors in \cite{maity2023uniqueness} show that any two weak solutions of the system coincides if one solution satisfy an energy inequality and the other solution satisfy the Serrin condition including the end-point case (under a smallness assumption on the fliud's velocity). A similar result for a fluid-rigid body system in a bounded domain had earlier been shown
%in \cite{muha2021uniqueness}  under the assumption that the  rigid body does not touch the boundary of the fluid. 
%Under the additional condition that the time derivative of both the translational velocity and the angular velocity of the rigid body are essentially bounded in time, they show in \cite{muha2022regularity} that the solution is actually smooth. Both results exclude the critical Serrin condition.
%For a fluid interacting with its flexible domain, the authors in \cite{schwarzacher2022weak}  tackles the problem for a fluid interacting with an elastic plate.






\subsection{The fluid-structure interaction problem}
We are interested in the interaction of an incompressible fluid with a flexible shell where the shell  reacts to the surface forces induced by the fluid and deforms the spatial reference domain $\Omega \subset \mathbb{R}^3$ to $\Omega_{\eta(t)}$ with respect to a coordinate transform $\bfvarphi_{\eta(t)}$ (see Figure \ref{fig:2} for the typical situation). The deformed domain $\Omega_{\eta}$ is defined in Subsection~\ref{ssec:geom} in a precise way. %to a coordinate transform $\bfvarphi_{\eta(t)}:\Omega\to \partial \Omega_{\eta(t)}$ defined in Subsection~\ref{ssec:geom}.
  We assume that the shell is visco-elastic. This means that besides the fluid forces, it is driven by its {\em elastic} properties and its {\em viscosity}. The reference model here is the {\em linearized Koiter shell model}, but also linearized versions of von Karman shells or pure bending shells are models that can be treated by the methods here. Following~\cite{LeRu,CanMuh13}, we find that the elastic part of the equation for the solid becomes $\alpha\Dely^2\eta+B \eta $, where $B$ is a linear second-order differential operator. Similarly, the part related to the viscosity of the shell becomes $\gamma\Dely^2\partial_t\eta + B'\partial_t\eta $, where $B'$ is another linear second-order differential operator. To simplify reading, we  take a form of the equation that contains only parts of the contributions of elasticity and viscosity, which are essential for the analysis to be performed. In particular, we reduce the elastic part to $\alpha\Dely^2\eta$ and the viscous part to $-\gamma\Dely\partial_t\eta$. We observe that the reduction is with no loss of generality, which certainly would not be the case if {\em non-linear} Koiter shell models were considered as in~\cite{breit2021incompressible,CanMuh13,MuSc}. 

%\todo{We have used $\eta_1$ for the weak-strong uniqueness part, for $\partial_t\eta(0)$ we use $\eta_*$ and I have changed this everywhere before weak-strong part}
 Accordingly, the shell function $\eta:(t, \by)\in I \times \omega \mapsto   \eta(t,\by)\in \mathbb{R}$ with $I=(0,T)$ for some $T>0$ solves
 \begin{equation}\label{1}
\left\{\begin{aligned}
& \varrho_s\partial_t^2\eta -\gamma\partial_t\Dely \eta + \alpha\Dely^2\eta=g-\bn^\intercal\bm{\tau}\circ\bm{\varphi}_\eta\bn_\eta
 \vert \mathrm{det}(\naby \bm{\varphi}_{\eta})\vert
&\text{ for all }  (t,\by)\in I\times\omega
 ,\\
&\eta(0,\by)=\eta_0(\by), \quad (\partial_t\eta)(0, \by)=\eta_*(\by)
&\text{ for all } \by\in\omega.
 	\end{aligned}\right.
 \end{equation}
Here, $\omega \subset \mathbb{R}^2$ is such that there is $\bfvarphi_\eta :\omega\to \partial \Omega_\eta$ that parametrizes  the boundary of the reference domain $\Omega$. The parameters $\varrho_s,\gamma$ and $\alpha$ are positive constants and the function $g:(t, \by)\in I \times \omega \mapsto  g(t,\by)\in   \mathbb{R}$ is a given forcing term. The vectors $\bn$ and $\bfn_\eta$ are the normal vectors  of the reference boundary and of the deformed boundary, respectively, whereas $\bftau$ denotes the Cauchy stress of the fluid given by Newton's rheological law, that is
$\bftau=\mu\big(\nabx\bu+(\nabx\bu)^\intercal\big)-\pi\mathbb I_{3\times 3}$. The positive constant $\mu$ represents the viscosity coefficient. Also, $\mathbf{v}:(t, \mathbf{x})\in I \times \Oeta \mapsto  \mathbf{v}(t, \mathbf{x}) \in \mathbb{R}^3$, the velocity field and $\pi:(t, \mathbf{x})\in I \times \Oeta \mapsto  \pi(t, \mathbf{x}) \in \mathbb{R}$, the pressure function are the unknown functions for the fluid whose motion is governed by the Navier--Stokes equations
 \begin{equation}\label{2}
\left\{\begin{aligned}
 &\varrho_f\big(\partial_t \bu  + (\mathbf{v}\cdot \nabx)\mathbf{v} \big)
 = 
 \mu\Delx \bu -\nabx\pi+ \bff &\text{ for all }(t,\bx)\in I\times\Omega_\eta,\\
 &\Div \bu=0&\text{ for all }(t,\bx)\in I \times\Omega_\eta,\\
 &\bu(0,\bx)=\bu_0(\bx) &\text{ for all } \bx\in \Omega_{\eta_0},
 \end{aligned}\right.
 \end{equation}
where $\varrho_f$ is a positive constant representing the density of the fluid and the function $\bff:(t, \mathbf{x})\in I \times \Oeta \mapsto  \bff(t, \mathbf{x}) \in \mathbb{R}^3$ is a given volume force. The equations \eqref{1} and \eqref{2} are coupled through the kinematic boundary condition
\begin{align}
\label{interfaceCond}
\bu\circ \bfvarphi_\eta=\partial_t\eta\bfn \quad\text{ for all } (t,\by)\in I\times \omega.
\end{align} 


% Figure environment removed


%\subsection{Higher order estimates} Finally, we turn to higher order estimates which were previously not even available in simpler geometries. 
%First of all, we consider the parabolic Stokes system in irregular domains and prove a parabolic counterpart of \eqref{4} under the same minimal assumptions on the regularity of the boundary, see Theorem \ref{thm:stokesunsteady}. Eventually, we turn to the parabolic Stokes system in moving domains in Theorem \ref{thm:stokesunsteadymoving}. Finally, this can be applied to
%\eqref{1} moving the convective term to the right-hand side.
%With the regularity from \eqref{eq:www} at hand we obtain
%\begin{align*}
%\partial_t\bu\in L^2(I;W^{s,2}(\Omega_\eta)),\quad\bu\in L^2(I;W^{2+s,2}(\Omega_\eta)),\quad \pi\in L^2(I;W^{1+s,2}(\Omega_\eta)),
%\end{align*}
%for all $s<\frac{1}{2}$.
%Translating this into an information for the Cauchy stress $\bftau$ and using \eqref{1} finally yields
%\begin{align*}
%\partial_t\eta\in L^\infty(I;W^{s,2}(\omega)),\quad \partial_t\eta\in L^2(I;W^{s+1,2}(\omega)),\quad \eta\in L^\infty(I;W^{s+2,2}(\omega)),
%\end{align*}
%for all $s<2$. From this we obtain
%\begin{align*}
%\eta\in L^2(I;W^{4+s,2}(\omega)),\quad 
%\partial_t^2\eta\in L^2(I;W^{s,2}(\omega)),
%\end{align*}
%for all $s<1$ arguing by difference quotients for the first assertion. With the improved regularity of $\partial_t^2\eta$ at hand, we
%turn again to the momentum equation.
%Iterating this procedure we thus obtain estimates for the velocity field, the pressure and the shell displacement in Sobolev spaces of arbitrary high order.
%The details are given in Theorem \ref{thm:higher}.


%There is a large body of work on fluid-struture interaction problems with rigid structures and a viscous fluid described by the Navier--Stokes equation \cite{bravin2019energy, conca2000Existence, desjardins1999Existence, desjardins2000weak, feireisl2003motion, grandmont2000existence, heslacollision, hillairet2007lack, hillairet2010blow, hillairet2009collisions, hoffmann1999motion, muha2021uniqueness, san2002global, takahashi2004global, takahashi2003analysis} 
%
%\noindent There is a large body of work on fluid-structure interaction problems involving elastic structures interacting with an unsteady viscous incompressible fluid. We may broadly classify these body of work into the construction of strong solutions and weak solutions for a viscous fluid interacting with an elastic structure.
%\\
%For weak solutions,  a semigroup approach is used in \cite{barbu2007existence} in the construction of global-in-time weak solutions  for the interaction between a stationary elastic solid  immersed in a  viscous incompressible fluid and where the interaction happens at the boundary of the solid. The same authors then show in \cite{barbu2008smoothness} that for smooth enough data,  the weak solutions constructed in \cite{barbu2007existence} becomes smooth.
%A weak solution is also shown to exists in \cite{boulakia2007existence} for a regularized three-dimensional elastic structure immersed in
%an incompressible viscous fluid contained in a fixed bounded connected domain. These solutions exists 
%as long as deformations of the elastic solid are sufficiently small  and no collisions occurs between the structure and the boundary. However, large translations and rotations of the structure are accounted for. 
%The authors in \cite{chambolle2005existence} used a Galerkin method to show the existence of a weak solution to the  three-dimensional  Navier-Stokes equations
%coupled with a two-dimensional elastic plate model that is modified to include viscous effects. This weak solution exists so long as  the structure does not touch the fixed part of the fluid boundary. The viscous effects incorporated in the plate model is then removed in \cite{grandmont2008existence} by passing to the limit as the coefficient modelling the viscoelasticity tends to zero.
%  The seminal work \cite{coutand2005motion} explores the motion of   the linear Kirchhoff elastic solid material
% inside an incompressible viscous fluid. A topological fixed-point argument is used to construct a local-in-time weak solution which is then shown to be regular and unique. 
%The authors in \cite{LeRu} studies the interaction of an incompressible Newtonian fluid with a linearly elastic Koiter shell. Here, the fluid's boundary is described by the mid-section of the shell and the authors show the existence of weak solutions, without self-intersections of the shell, using an Aubin--Lion type argument. 
%A model for blood flow through arteries are explored in \cite{MuCa1, muha2016existence} where a semi-discrete in time approximation method is used to construct weak solutions.
%\\
%When it comes to strong solutions, a short time existence and uniqueness of solutions in Sobolev spaces are studied in \cite{cheng2007navier, CS} for  a viscous incompressible fluid interacting with a nonlinear thin elastic  shell. The shell equation for the former \cite{cheng2007navier} is modeled by  nonlinear Saint-Venant-Kirchhoff constitutive law whereas that of the latter \cite{CS}  is modeled by the nonlinear Koiter shell model.  In \cite{coutand2006interaction}, however, the authors prove the existence of a unique local strong solutions, without restriction on the size of the data, when the elastic structure is now governed by quasilinear elastodynamics. A simplified model for describing blood flow through viscoelastic arteries is explored in \cite{GraHil}. Here a global-in-time strong solution is constructed for the coupling of a one-dimensional viscoelastic beam equation and the two-dimensional incompressible Navier--Stokes equation. 
%In particular, the authors show that contact between the viscoelastic wall and the bottom of the fluid cavity does not occur in finite time.
%In \cite{ignatova2014well}, the elastic structure is modelled by a damped wave equation  with  additional boundary stabilization terms. For sufficiently small initial data, subject to said boundary stabilization terms,
%global-in-time existence of   strong solutions and exponential decay of the solutions are shown. 
%The free boundary fluid-structure interaction problem consisting of a Navier--Stokes equation and a wave equation defined in two different but adjacent domains is studied in \cite{kukavica2012solutions}. A local strong solution is constructed under suitable compatibility conditions for the data.
%Another local-in-time strong existence result is \cite{DRR} where the viscous Newtonian fluid is now interacting with an elastic structure modeled by a nonlinear damped shell equation.
%Finally, a local strong solution is constructed for the motion of a linearly elastic Lam\'e solid moving in a viscous fluid in \cite{raymond2014fluid}.




\subsection{Main result}
The main motivation for the present work is to prove an analog of the results by Serrin, Prodi and Ladyzhenskaya for the fluid-structure interaction problem \eqref{1}--\eqref{interfaceCond}.
The here presented result is a summary of Theorem \ref{thm:main}, where the statement can be found in its full extent. 

A weak solutions $(\eta,\bu)$ to \eqref{1}--\eqref{interfaceCond} can be constructed to satisfy the energy inequality and thus
\begin{align}
\sup_{I}\|\partial_t\eta\|_{L^2_\by}^2+\sup_{I}\|\Dely\eta\|_{L^2_\by}^2+\int_{I}\|\partial_t\naby\eta\|_{L^2_\by}^2\dt<\infty.\label{eq:apriorieta0}
\end{align}
We speak about a strong solution if all quantities in \eqref{1} and \eqref{2} are $L^2$-functions in space-time. The precise definitions can be found in Definitions \ref{def:weakSolution} and \ref{def:strongSolution}. 
\begin{theorem}[Shells]\label{thm:mainsimple}
Let $(\bu,\eta)$ be a weak solution to \eqref{1}--\eqref{interfaceCond}. Suppose that we
have
\begin{align}\label{eq:regu'0}
\bu&\in L^r(I;L^s(\Omega_\eta)),\quad \tfrac{2}{r}+\tfrac{3}{s}\leq1,\\
\eta&\in L^\infty(I;C^{1}(\omega)).\label{eq:regeta''0}
\end{align}
Then $(\bu,\eta)$ is a strong solution to \eqref{1}--\eqref{interfaceCond}.
Moreover, $(\bu,\eta)$  is unique in the class of weak solutions satisfying the energy inequality and are in $L^\infty(I;C^{0,1}(\omega))$.
\end{theorem}
We emphasis that the only additional assumption for the structure is given by $L^\infty(I;C^{1}(\omega))$ for conditional smoothness or $L^\infty(I;C^{0,1}(\omega))$ for the uniqueness class of the strong solution. Note that this is only an instant of regularity more than a weak solution enjoys as it belongs to $L^\infty(I;W^{2,2}(\omega))$, see \eqref{eq:apriorieta0}. Further note that the spaces $W^{2,2}(\omega),C^{0,1}(\omega)$ and $C^1(\omega)$ even have the same index in 2D, but the embedding $W^{2,2}(\omega)\hookrightarrow C^{0,1}(\omega)$ just fails. This extra assumption is, however, essential for the approach presented here. Hence it remains {\em a challenging open problem of whether the uniqueness regime (of a strong solution) can be extended to all energy preserving weak-solutions in the case of a curved reference geometry}. In contrast, in case the reference geometry is flat (the plate case), a direct approach for weak-strong uniqueness is available for which the $C^{0,1}(\omega)$ assumption is not necessary~\cite{schwarzacher2022weak}. As the theory presented here is in particular valid for plates, we have the following corollary.
\begin{corollary}[Plates]\label{cor:mainsimple}
Let $(\bu,\eta)$ be a weak solution of \eqref{1}--\eqref{interfaceCond} with {\em flat reference geometry}. Suppose that we
have
\begin{align*}%\label{eq:regu'0c}
\bu&\in L^r(I;L^s(\Omega_\eta)),\quad \tfrac{2}{r}+\tfrac{3}{s}\leq1,\\
\eta&\in L^\infty(I;C^{1}(\omega)).%\label{eq:regeta''0c}
\end{align*}
Then $(\bu,\eta)$ is a strong solution of \eqref{1}--\eqref{interfaceCond}.
Moreover, $(\bu,\eta)$ is unique in the class of weak solutions satisfying an energy inequality.\footnote{Strictly speaking the weak-strong uniqueness result in ~\cite{schwarzacher2022weak} considers elastic plates. However, the presence of dissipation in the structure equation does not change the argument at all.}
\end{corollary}

\begin{remark}[Stability and convergence of numerical schemes]
\label{rem:stab}
{\rm 
Related to the weak-strong uniqueness results is a stability estimate (see Section \ref{sec:weakStrong}). It comes naturally, as the difference of two solutions is estimated. Hence a further result of this work is that any solution satisfying \eqref{eq:regu'0} and \eqref{eq:regeta''0} are actually attractors in the respective uniqueness class.

Stability are of particular importance for numeric applications. Indeed they form the first step in order to show that the difference between a discrete solution and the continuous solution decreases (with a rate), provided that the continuous solution is unique, as was shown for plates in 2D~\cite{SchSheTum23,schwarzacher2022weak}.
}
\end{remark}

%\seb{STABILITY REMARK! RELEVANCE FOR NUMERIC.}


The formal proof for the regularity of Theorem \ref{thm:mainsimple} consists in proving an acceleration estimate which combines and extends the results in~\cite{Br, GraHil}. In
order to appreciate the moving boundary, the correct test function for the momentum equation is -- roughly speaking -- the material derivative $\partial_t\bv+\bfv\cdot\nabla\bv$ combined with the test-function $\partial_t^2\eta$ for the structure equation. A key tool is eventually to estimate $\nabla^2\bv$ (as well as $\nabla\pi$) by means of $\partial_t\bv+\bfv\cdot\nabla\bv$. This can be done by means of a steady Stokes theory for irregular domains proved in \cite{Br} (applied to the domain $\Omega_{\eta(t)}$ for fixed $t$). It strongly requires a boundary with a small local Lipschitz constant and thus \eqref{eq:regeta''0} is essentially needed here. Otherwise, such a regularity estimate is not known and probably not even expected.
Furthermore, in order to avoid the appearance of the pressure function, an extension operator from $\omega$ to $\Omega_\eta$ has been used in \cite{GraHil} which is at the same time solenoidal and satisfies a homogeneous Neumann-type boundary condition. The construction of the latter is only possible for a flat reference domain. Hence, we introduce the pressure function and work with a more common extension operator which does not preserve solenoidability.
The advantage of the latter is that it has the usual regularization property (it ``gains'' the differentiability which is lost by the trace theorem, see Section \ref{sec:ext}) different from the solenoidal extensions used in \cite{GraHil}, \cite{LeRu} and \cite{MuSc}.
A major difference between the 2D and 3D cases is that one has to use the full strength of this operator to compensate for the worsening embeddings.

In order to make this argument rigorous, we work with a strong solution to \eqref{1}--\eqref{interfaceCond}.
Thus, we prove the existence of a local strong solution in Theorem
\ref{thm:fluidStructureWithoutFK}. 
With  a strong solution at hand, we can justify the estimates mentioned above. To close the argument (see the proof of Theorem \ref{thm:main}), we have to compare weak and strong solutions by means of a weak-strong uniqueness result.
The difficulty of the latter is that one needs to compare two velocity fields which are a priori defined on different (time-changing) domains. Nevertheless,
such a result
has been established very recently in \cite{schwarzacher2022weak} for linear elastic plates. The key idea is to transform the strong solution into the domain of the weak solution and then estimate their difference. When doing so, the strong solution loses its solenoidal character which must be corrected to avoid the appearance of the pressure function in the weak formulation. In the case of a flat geometry as in \cite{schwarzacher2022weak} this can be done by an explicit construction, but our situation is more complicated. We thus work with a Bogovkij-operator for moving domains~\cite{KamSchSpe20}. 
Its properties crucially hinge on the spatial Lipschitz continuity of the moving boundary and thus require
that the weak solution is $L^\infty(C^{0,1})$. For details on the criticality of Lipschitz regularity see~\cite{SaaSch21}, where estimates for the Bogovkij-operator in rough time-dependent domains are studied.

\iffalse
In this work, we are ultimately interested in constructing strong solutions. In order to perform this task, we collect in Section \ref{sec:prelim} some notations, definitions and set-up the functional analytic framework for the rest of our work. Furthermore, we make precise the various notions of the solution that we shall explore.

Then we devote Section \ref{sec:loc} to the construction of a local-in-time solution that solves our coupled fluid-structure problem \eqref{1}--\eqref{interfaceCond} almost everywhere in space-time. This is the first main result and it is given by Theorem \ref{thm:fluidStructureWithoutFK}.
Here, we follow the classical approach to construct strong solutions of the fluid-structure problem. This involves transforming the coupled system to its reference domain and then linearising the resulting system on said reference domain to allow for easy derivation of a priori estimates. Finally, we apply a fixed-point argument to get our local solution for the original nonlinear coupled system on its original domain. Although the technique in the construction is standard, the novelty here is that we work on a general $3$-dimensional reference domain as opposed to a flat reference domain where the structure
can only move in the vertical direction. Considering a flat reference domain was the standard practice until the recent result \cite{Br} for a general $2$-dimensional reference domain.

A drawback in the use of a fixed-point argument in our construction of strong solutions in Section \ref{sec:loc} is the undesirable relationship between the bounds obtained and the local time. Nevertheless, we take the solution constructed in Section \ref{sec:loc} as an approximate solution to build upon in order to obtain a Ladyzhenskaya-Prodi-Serrin condition for our fluid-structure interaction problem. This is the subject of Section \ref{sec:reg} and leads to our second main result, i.e. Theorem \ref{prop2}. Here, we obtain a uniform-in-time bound for the strong solution under the minimal assumption that the velocity of the fluid satisfies Ladyzhenskaya-Prodi-Serrin condition and that the shell is Lipschitz in space. As far as we are aware, this is first such result for a fluid interacting with beam with no further restriction on the structure and its reference configuration.

In our final section, Section \ref{sec:weakStrong}, we present a weak-strong uniqueness result for the fluid-structure problem \eqref{1}--\eqref{interfaceCond} above. Here,
since the velocity fields for the weak and strong solutions are defined on different spatial geometries, it is not clear how to make sense of a difference estimate for the velocities  defined on the ''difference of their spatial geometries''. For this reason, we follow the approach of \cite{schwarzacher2022weak} where the system for the strong solution is transformed through a change of variable to the domain of the weak solution and obtaining a difference estimate on the domain of the weak solution. However, we do not  require any further assumption on the time derivative of the strong solution constructed in Sections \ref{sec:loc} and \ref{sec:weakStrong} as was required in \cite{schwarzacher2022weak}.
 \fi
\subsection{Organization of the paper}
We introduce in Section \ref{sec:prelim}, some notations, definitions and the functional analytic framework. In particular, we give the definition of the notion of a weak and a strong solution for the system \eqref{1}--\eqref{interfaceCond}. We then construct in Section \ref{sec:loc}, the local strong solution by linearizing the system and employing the Banach fixed-point argument. Section \ref{sec:reg} is devoted to proposing the conditions of Serrin type to obtain the acceleration estimate. In Section \ref{sec:weakStrong}, we focus on showing the weak-strong uniqueness result. Finally, we give in Section \ref{summary} a short summary to formulate the main result by collecting the key elements of the previous sections.



\section{Preliminaries}
\label{sec:prelim}
\subsection{Conventions}
For simplicity, we set all physical constants in \eqref{1}--\eqref{interfaceCond} to 1. The analysis is not affected as long as they are strictly positive.
For two non-negative quantities $f$ and $g$, we write $f\lesssim g$  if there is a $c>0$ such that $f\leq\,c g$. Here $c$ is a generic constant which does not depend on the crucial quantities. If necessary, we specify particular dependencies. We write $f\approx g$ if both $f\lesssim g$ and $g\lesssim f$ hold. In the notation for function spaces (see next subsection), we do not distinguish between scalar- and vector-valued functions. However, vector-valued functions will usually be denoted in bold case.
For simplicity, we supplement \eqref{1} with periodic boundary conditions and identify $\omega$ (which represents the complete boundary of $\Omega$) with  $(0,1)^2$. We consider periodic function spaces for zero-average functions.
It is only a technical matter to consider \eqref{1} on a nontrivial subset of $\partial\Omega$ together with zero boundary conditions for $\eta$ and $\naby\eta$ instead of considering \eqref{1} on $(0,1)^2$, see e.g. \cite{LeRu} or \cite{BrSc} for the corresponding geometrical set-up.
We shorten the time interval $(0,T)$ by $I$.

\subsection{Classical function spaces}
Let $\mathcal O\subset\R^3$ be open.
The function spaces of continuous or $\alpha$-H\"older-continuous functions, $\alpha\in(0,1)$,
 are denoted by $C(\overline{\mathcal O})$ or $C^{0,\alpha}(\overline{\mathcal O})$ respectively, where $\overline{\mathcal O}$ is the closure of $\mathcal O$. Similarly, we write $C^1(\overline{\mathcal O})$ and $C^{1,\alpha}(\overline{\mathcal O})$.
We denote  by $L^p(\mathcal O)$ and $W^{k,p}(\mathcal O)$ for $p\in[1,\infty]$ and $k\in\mathbb N$, the usual Lebesgue and Sobolev spaces over $\mathcal O$. 
%For a bounded domain $\mathcal O$, the space $L^p_\perp(\mathcal O)$ denotes the subspace of  $L^p(\mathcal O)$ of functions with zero mean, that is $(f)_{\mathcal O}:=\dashint_{\mathcal O}f\dx:=\mathcal L^n(\mathcal O)^{-1}\int_{\mathcal O}f\dx=0$.
For a bounded domain $\mathcal O$,  the notation $(f)_{\mathcal O}:=\dashint_{\mathcal O}f\dx:=\mathcal L^3(\mathcal O)^{-1}\int_{\mathcal O}f\dx$ represents the mean or average value of $f\in L^p(\mathcal O)$.
 We denote by $W^{k,p}_0(\mathcal O)$, the closure of the smooth and compactly supported functions in $W^{k,p}(\mathcal O)$. If $\partial\mathcal O$ is regular enough, this coincides with the functions vanishing $\mathcal H^{2}$ -a.e. on $\partial\mathcal O$. 
 We also denote by $W^{-k,p}(\mathcal O)$ the dual of $W^{k,p}_0(\mathcal O)$.
  Finally, we consider subspaces
$W^{1,p}_{\Div}(\mathcal O)$ and $W^{1,p}_{0,\Div}(\mathcal O)$ of divergence-free vector fields which are defined accordingly. The space $L^p_{\Div}(\mathcal O)$ is defined as the closure of the set of smooth and compactly supported solenoidal functions in $L^p(\mathcal O).$ We will use the shorthand notations $L^p_\bx$ and $W^{k,p}_\bx$ in the case of $3$-dimensional domains (typically spaces defined over $\Omega\subset\R^3$ or $\Omega_\eta\subset\R^3$) and   
$L^p_\by$ and $W^{k,p}_\by$ for $2$- dimensional sets (typcially spaces of periodic functions defined over $\omega\subset\R^{2}$). 
For any pair of separable Banach spaces $(X,\|\cdot\|_X)$ and $(Y,\|\cdot\|_Y)$ with $X\subset Y$, we write $X\hookrightarrow Y$ if $X$ is continuously embedded in $Y$, that is $\Vert\cdot\Vert_Y\lesssim \Vert \cdot\Vert_X$.
Since we only consider functions on $\omega$ with periodic boundary conditions and zero mean values, we have the following equivalences
\begin{align*}
\|\cdot\|_{W^{1,2}_\by}\approx \|\nabla_\by\cdot\|_{L^2_\by},\quad \|\cdot\|_{W^{2,2}_\by}\approx \|\Dely\cdot\|_{L^2_\by},\quad \|\cdot\|_{W^{4,2}_\by}\approx \|\Dely^2\cdot\|_{L^2_\by}.
\end{align*}
For a separable Banach space $(X,\|\cdot\|_X)$, we denote by $L^p(I;X)$, the set of (Bochner-) measurable functions $u:I\rightarrow X$ such that the mapping $t\mapsto \|u(t)\|_{X}$ belongs to $L^p(I)$. 
The set $C(\overline{I};X)$ denotes the space of functions $u:\overline{I}\rightarrow X$ which are continuous with respect to the norm topology on $(X,\|\cdot\|_X)$. For $\alpha\in(0,1]$ we write
$C^{0,\alpha}(\overline{I};X)$ for the space of H\"older-continuous functions with values in $X$. The space $W^{1,p}(I;X)$ consists of those functions from $L^p(I;X)$ for which the distributional time derivative belongs to $L^p(I;X)$ as well. The space $W^{k,p}(I;X)$ is defined accordingly.
We use the shorthand $L^p_tX$ for $L^p(I;X)$. For instance, we write $L^p_tW^{1,p}_\bx$ for $L^p(I;W^{1,p}(\mathcal O))$. Similarly, $W^{k,p}_tX$ stands for $W^{k,p}(I;X)$.
%{\color{blue}
%Finally, for any three Bochner space $W^{k_1,p_1}_tX_1$, $W^{k_2,p_2}_tX_2$ and $W^{k_3,p_3}_tY$
%% with $X_1\cap X_2\subset Y$
%, we write 
%\begin{align*}
%T^q\,[ W^{k_1,p_1}_tX_1 \cap W^{k_2,p_2}_tX_2]\hookrightarrow W^{k_3,p_3}_tY
%\end{align*}
%if $\Vert \cdot\Vert_{W^{k_3,p_3}_tY}\lesssim T^q\big(\Vert \cdot\Vert_{W^{k_1,p_1}_tX_1}+\Vert \cdot\Vert_{W^{k_2,p_2}_tX_2}\big)$.
%}

\subsection{Fractional differentiability and Sobolev mulitpliers}
For $p\in[1,\infty)$, the fractional Sobolev space (Sobolev-Slobodeckij space) with differentiability $s>0$ with $s\notin\mathbb N$ will be denoted by $W^{s,p}(\mathcal O)$. For $s>0$, we write $s=\lfloor s\rfloor+\lbrace s\rbrace$ with $\lfloor s\rfloor\in\N_0$ and $\lbrace s\rbrace\in(0,1)$.
 We denote by $W^{s,p}_0(\mathcal O)$, the closure of the smooth and compactly supported functions in $W^{s,p}(\mathcal O)$. For $s>\frac{1}{p}$ this coincides with the functions vanishing $\mathcal H^{n-1}$ -a.e. on $\partial\mathcal O$ provided that $\partial\mathcal O$ is regular enough. We also denote by $W^{-s,p'}(\mathcal O)$, for $s>0$ and $p,p'\in[1,\infty)$, with $\frac{1}{p}+\frac{1}{p'}=1$, the dual of $W^{s,p}_0(\mathcal O)$. Similar to the case of unbroken differentiabilities above, we use the shorthand notations $W^{s,p}_\bx$  and $W^{s,p}_\by$. 
We will denote by $\bfB^s_{p,q}(\R^n)$, the standard Besov spaces on $\R^n$ with differentiability $s>0$, integrability $p\in[1,\infty]$ and fine index $q\in[1,\infty]$. They can be defined (for instance) via Littlewood-Paley decomposition leading to the norm $\|\cdot\|_{\bfB^s_{p,q}(\R^n)}$. %The Besov-norm is then given as
%$\|\cdot\|_{\bfB^s_{p,q}(\R^m)}=|\cdot|_{\bfB^s_{p,q}(\R^m)}+\|\cdot\|_{L^p(\R^m)}$.
 We refer to \cite{RuSi} and \cite{Tr,Tr2} for an extensive description. 
For a bounded domain $\mathcal O\subset\R^n$, the Besov spaces $\bfB^s_{p,q}(\mathcal O)$ are defined as the restriction of functions from $\bfB^s_{p,q}(\R^n)$, that is
 \begin{align*}
 \bfB^s_{p,q}(\mathcal O)&:=\{f|_{\mathcal O}:\,f\in \bfB^s_{p,q}(\R^n)\},\\
 \|g\|_{\bfB^s_{p,q}(\mathcal O)}&:=\inf\{ \|f\|_{\bfB^s_{p,q}(\R^n)}:\,f|_{\mathcal O}=g\}.
 \end{align*}
 If $s\notin\mathbb N$ and $p\in(1,\infty)$ we have $\bfB^s_{p,p}(\mathcal O)=W^{s,p}(\mathcal O)$.
 
In accordance with \cite[Chapter 14]{MaSh}, the Sobolev multiplier norm  is given by
\begin{align}\label{eq:SoMo}
\|\varphi\|_{\mathcal M(W^{s,p}(\mathcal O))}:=\sup_{\mathbf{u} :\,\|\mathbf{u}\|_{W^{s-1,p}(\mathcal O)}=1}\|\nabla\varphi\cdot\mathbf{u}\|_{W^{s-1,p}(\mathcal O)},
\end{align}
where $p\in[1,\infty]$ and $s\geq1$.
The space $\mathcal M(W^{s,p}(\mathcal O))$ of Sobolev multipliers is defined as those objects for which the $\mathcal M(W^{s,p}(\mathcal O))$-norm is finite. By mathematical induction with respect to $s$, one can prove for Lipschitz-continuous functions $\varphi$ that membership to $\mathcal M(W^{s,p}(\mathcal O))$,  in the sense of \eqref{eq:SoMo}, implies that
\begin{align}\label{eq:SoMo'}
\sup_{w:\,\|w\|_{W^{s,p}(\mathcal O)}=1}\|\varphi \,w\|_{W^{s,p}(\mathcal O)}<\infty.
\end{align}
The quantity \eqref{eq:SoMo'} also serves as customary definition of the Sobolev multiplier norm in the literature but \eqref{eq:SoMo} is more suitable for our purposes.
Note that in our applications, we always assume that the functions in question are Lipschitz continuous so that the implication above holds true.

Let us finally collect some useful properties of Sobolev multipliers.
By \cite[Corollary 14.6.2]{MaSh} we have
\begin{align}\label{eq:MSa}
\|\phi\|_{\mathcal M(W^{s,p}(\R^{n}))}\lesssim\|\nabla\phi\|_{L^{\infty}(\R^n)},
\end{align}
provided that one of the following conditions holds:
\begin{itemize}
\item $p(s-1)<n$ and $\phi\in \bfB^{s}_{\varrho,p}(\R^n)$ with $\varrho\in\big[\frac{n}{s-1},\infty\big]$;
\item $p(s-1)=n$ and $\phi\in\bfB^{s}_{\varrho,p}(\R^n)$ with $\varrho\in(p,\infty]$.
\end{itemize}
Note that the hidden constant in \eqref{eq:MSa} depends on the $\bfB^{s}_{\varrho,p}(\R^n)$-norm of $\phi$.
By \cite[Corollary 4.3.8]{MaSh}, it holds
\begin{align}\label{eq:MSb}
\|\phi\|_{\mathcal M(W^{s,p}(\R^n))}\approx
\|\nabla\phi\|_{W^{s-1,p}(\R^n)}, 
\end{align}
for $p(s-1)>n$. 
 Finally, we note the following rule about the composition with Sobolev multipliers which is a consequence of \cite[Lemma 9.4.1]{MaSh}. For open sets $\mathcal O_1,\mathcal O_2\subset\R^n$, $u\in W^{s,p}(\mathcal O_2)$ and a Lipschitz continuous function $\bfphi:\mathcal O_1\rightarrow\mathcal O_2$ with Lipschitz continuous inverse and $\bfphi\in \mathcal M(W^{s,p}(\mathcal O_1))$ we have
\begin{align}\label{lem:9.4.1}
\|u\circ\bfphi\|_{W^{s,p}(\mathcal O_1)}\lesssim \|u\|_{W^{s,p}(\mathcal O_2)}
\end{align}
with constant depending on $\bfphi$. Using Lipschitz continuity
of $\bfphi$ and $\bfphi^{-1}$, estimate \eqref{lem:9.4.1} is obvious for $s\in(0,1]$. The general case can be proved by mathematical induction with respect to $s$.
%Finally, we denote by $W^{\alpha,p}(0,T;X)$ the fractional Sobolev space in time, i.e. the subspace of the Bochner space $L^{p}(0,T;X)$
%consisting of the functions having finite $W^{\alpha,p}(0,T;X)$-norm.

\subsection{Function spaces on variable domains}
\label{ssec:geom}
 The spatial domain $\Omega$ is assumed to be an open bounded subset of $\mathbb{R}^3$ with smooth boundary $\partial\Omega$ and an outer unit normal ${\bfn}$. We assume that
 $\partial\Omega$ can be parametrised by an injective mapping ${\bfvarphi}\in C^k(\omega;\R^3)$ for some sufficiently large $k\in\N$. We suppose for all points $\by=(y_1,y_2)\in \omega$ that the pair of vectors  
%$\mathbf{a}_i(\by):= 
$\partial_i {\bfvarphi}(\by)$, $i=1,2,$ are linearly independent.
%If $n=2$ the corresponding assumption simply asks for $\partial_\by\bfvarphi=\partial_y\bfvarphi$ not to vanish.
 For a point $\bx$ in the neighborhood
of $\partial\Omega$, we define the functions $\by$ and $s$ by  
\begin{align*}
 \by(\bx)=\arg\min_{\by\in\omega}|\bx-\bfvarphi(\by)|,\quad s(\bx)=(\bx-\by(\bx))\cdot\bfn(\by(\bx)).
 \end{align*}
Moreover, we define the projection $\bfp(\bx)=\bfvarphi(\by(\bx))$. We define $L>0$ to be the largest number such that $s,\by$ and $\bfp$ are well-defined on $S_L$, where
\begin{align}\label{eq:boundary1}
S_L=\{\bx\in\R^n:\,\mathrm{dist}(\bx,\partial\Omega)<L\}.
\end{align}
Due to the smoothness of $\partial\Omega$ for $L$ small enough we have $\abs{s(\bx)}=\min_{\by\in\omega}|\bx-\bfvarphi(\by)|$ for all $\bx\in S_L$. This implies that $S_L=\{s\bfn(\by)+\by:(s,\by)\in (-L,L)\times \omega\}$.
For a given function $\eta : I \times \omega \rightarrow\R$ we parametrise the deformed boundary by
\begin{align*}
{\bfvarphi}_\eta(t,\by)={\bfvarphi}(\by) + \eta(t,\by){\bfn}(\by), \quad \,\by \in \omega,\,t\in I.
\end{align*}
By possibly decreasing $L$, one easily deduces from this formula that $\Omega_{\eta}$ does not degenerate, that is
\begin{equation}\label{eq:1705}
\begin{aligned}
\partial_1\bfvarphi_\eta\times\partial_2\bfvarphi_\eta(t,\by)\neq0,\quad
 \bfn(\by)\cdot\bfn_{\eta(t)}(\by)&>0,\quad \,\by \in \omega,\,t\in I,
 \end{aligned}
\end{equation}
provided that $\sup_t\|\eta\|_{W^{1,\infty}_{\by}}<L$. Here $\bfn_{\eta(t)}$
is the normal of the domain $\Omega_{\eta(t)}$
 defined through
\begin{equation}\label{eq:2612}
\partial\Omega_{\eta(t)}=\set{{\bfvarphi}(\by) + \eta(t,\by){\bfn}(\by):\by\in \omega}.
\end{equation}
With the abuse of notation we define deformed space-time cylinder $I\times\Omega_\eta=\bigcup_{t\in I}\set{t}\times\Omega_{\eta(t)}\subset\R^{4}$.
The corresponding function spaces for variable domains are defined as follows.
\begin{definition}{(Function spaces)}
For $I=(0,T)$, $T>0$, and $\eta\in C(\overline{I}\times\omega)$ with $\|\eta\|_{L^\infty(I\times\omega)}< L$ we define for $1\leq p,r\leq\infty$
\begin{align*}
L^p(I;L^r(\Omega_\eta))&:=\big\{v\in L^1(I\times\Omega_\eta):\,\,v(t,\cdot)\in L^r(\Omega_{\eta(t)})\,\,\text{for a.e. }t,\,\,\|v(t,\cdot)\|_{L^r(\Omega_{\eta(t)})}\in L^p(I)\big\},\\
L^p(I;W^{1,r}(\Omega_\eta))&:=\big\{v\in L^p(I;L^r(\Omega_\eta)):\,\,\nabla v\in L^p(I;L^r(\Omega_\eta))\big\}.
\end{align*}
\end{definition}
\noindent 
In order to establish a relationship between the 
fixed domain and the time-dependent domain, we introduce the Hanzawa transform $\bm{\Psi}_\eta : \Omega \rightarrow\Omega_\eta$ defined by
\begin{equation}
\label{map}
\bfPsi_\eta(\bx)
=
 \left\{
  \begin{array}{lr}
    \mathbf{p}(\bx)+\big(s(\bx)+\eta(\by(\bx))\phi(s(\bx))\big)\bn(\by(\bx)) &\text{if dist}(\bx,\partial\Omega)<L,\\
    \bx &\text{elsewhere}.
  \end{array}
\right.
\end{equation}
for any $\eta:\omega\rightarrow (-L,L)$. Here $\phi\in C^\infty(\mathbb R)$ is such that 
$\phi\equiv 0$ in a neighborhood of $-L$ and $\phi\equiv 1$ in a neighborhood of $0$. The other variables  $\mathbf{p}$, $s$ and $\bn$ are as defined earlier in this Section \ref{ssec:geom}.
Due to the size of $L$, we find that $\bfPsi_\eta$ is a homomorphism such that $\bfPsi_\eta|_{\Omega\setminus S_L}$ is the identity. Furthermore, $\bm{\Psi}_\eta$ together with its inverse\footnote{It exists provided that we choose $\phi$ such that $|\phi'|<L/\alpha$.} 
 $\bm{\Psi}_\eta^{-1} : \Omega_\eta \rightarrow\Omega$  possesses the following properties, see \cite{Br} for details. If for some $\alpha,R>0$, we assume that
\begin{align*}
\Vert\eta\Vert_{L^\infty_\by}
+
\Vert\zeta\Vert_{L^\infty_\by}
< \alpha <L \qquad\text{and}\qquad
\Vert\naby\eta\Vert_{L^\infty_\by}
+
\Vert\naby\zeta\Vert_{L^\infty_\by}
<R
\end{align*}
holds, then for any  $s>0$, $\varrho,p\in[1,\infty]$ and for any $\eta,\zeta \in B^{s}_{\varrho,p}(\omega)\cap W^{1,\infty}(\omega)$, we have that
\begin{align}
\label{210and212}
&\Vert \bm{\Psi}_\eta \Vert_{B^s_{\varrho,p}(\Omega\cup S_\alpha)}
+
\Vert \bm{\Psi}_\eta^{-1} \Vert_{B^s_{\varrho,p}(\Omega\cup S_\alpha)}
 \lesssim
1+ \Vert \eta \Vert_{B^s_{\varrho,p}(\omega)},
\\
\label{211and213}
&\Vert \bm{\Psi}_\eta - \bm{\Psi}_\zeta  \Vert_{B^s_{\varrho,p}(\Omega\cup S_\alpha)} 
+
\Vert \bm{\Psi}_\eta^{-1} - \bm{\Psi}_\zeta^{-1}  \Vert_{B^s_{\varrho,p}(\Omega\cup S_\alpha)} 
\lesssim
 \Vert \eta - \zeta \Vert_{B^s_{\varrho,p}(\omega)}
\end{align}
and
\begin{align}
\label{218}
&\Vert \partial_t\bm{\Psi}_\eta \Vert_{B^s_{\varrho,p}(\Omega\cup S_\alpha)}
\lesssim
 \Vert \partial_t\eta \Vert_{B^{s}_{ \varrho,p}(\omega)},
\qquad
\eta
\in W^{1,1}(I;B^{s}_{\varrho,p}(\omega))
\end{align}
holds uniformly in time with the hidden constants depending only on the reference geometry, on $L-\alpha$ and $R$. 
%The construction of $\bm{\Psi}_\eta$ can be found in \cite[pages 210, 211]{LeRu}. Note that variable domains in \cite{LeRu} are defined via functions $\zeta:\partial\Omega\rightarrow\R$ rather than functions $\eta:\omega\rightarrow\R$ (clearly, one can link them by setting
%$\zeta=\eta\circ\bfvarphi^{-1}$). 




\subsection{Extension and smooth approximation on variable domains}\label{sec:ext}
In this subsection, we construct an extension operator
which extends functions from $\omega$ to the moving domain $\Omega_\eta$ for a given function $\eta$ defined on $\omega$. We follow \cite[Section 2.3]{BrScF}.
Since $\Omega$ is assumed to be sufficiently smooth, it is well-known that there is an extension operator $\mathscr F_\Omega$ which extends functions from $\partial\Omega$ to $\R^3$ and satisfies
\begin{equation*}\label{2.17a}
\mathscr F_\Omega:W^{\sigma,p}(\partial\Omega)\rightarrow W^{\sigma+1/p,p}(\R^3),
\end{equation*}
for all $p\in[1,\infty]$ and all $\sigma>0$, as well as $\mathscr F_\Omega v|_{\partial\Omega}=v$. Now we define $\mathscr F_\eta$ by 
\begin{align}\label{eq:2401b}
\mathscr F_\eta b=\mathscr F_\Omega ((b\bfn)\circ\bfvarphi^{-1})\circ{\bfPsi}_\eta^{-1},\quad b\in W^{\sigma,p}(\omega),
\end{align}
where $\bfvarphi$ is the function in the parametrization of  $\partial\Omega$.
If $\eta$ is regular enough, $\mathscr F_\eta$ behaves as a classical extension. 
%To be more precise, we can use the formula
%\begin{align*}
%\nabla\mathscr F_\eta b&=\nabla\mathscr F_\Omega ((b\bfn)\circ\bfvarphi^{-1})\circ{\bfPsi}_\eta^{-1}\nabla{\bfPsi}_\eta^{-1},
%%\nabla^2\mathscr F_\eta b&=\nabla^2\mathscr F_\Omega ((b\nu)\circ\bfvarphi^{-1})\circ\overline{\bfPsi}_\eta^{-1}\nabla\overline{\bfPsi}_\eta^{-1}\nabla\overline{\bfPsi}_\eta^{-1}+\nabla\mathscr F_\Omega ((b\nu)\circ\bfvarphi^{-1})\circ\overline{\bfPsi}_\eta^{-1}\nabla^2\overline{\bfPsi}_\eta^{-1},\\
%%\partial_t\mathscr F_\eta b&=\nabla\mathscr F_\Omega ((b\nu)\circ\bfvarphi^{-1})\circ\overline{\bfPsi}_\eta^{-1}\partial_t\overline{\bfPsi}_\eta^{-1},
%\end{align*}
%estimate \eqref{est:psi-1eta} and \eqref{2.17a} to 
We obtain the following Lemma which is a version of \cite[Lemma 2.2]{Br}, but also includes differentiabilities larger than 1.
\begin{lemma}\label{lem:3.8}
Let $\sigma>0$ and $p\in[1,\infty]$.
Let $\eta\in C^{0,1}(\omega)$ with $\|\eta\|_{L^\infty_\by}<\alpha<L$. Suppose further that $\eta\in B^{\sigma+1/p}_{\varrho,p}(\omega)$, where $p$ and $\varrho$ are related as in \eqref{eq:MSa} and \eqref{eq:MSb}.
 The operator
$\mathscr F_\eta$ defined in \eqref{eq:2401b} satisfies
\begin{align*}
\mathscr F_\eta: W^{\sigma,p}(\omega)\rightarrow W^{\sigma+1/p,p}(\Omega\cup S_\alpha)
\end{align*}
and $(\mathscr F_\eta b)\circ\bfvarphi_\eta=b\bfn$ on $\omega$ for all $b\in W^{\sigma,p}(\omega)$. In particular, we have
\begin{align*}
\|\mathscr F_\eta b\|_{W^{\sigma+1/p,p}(\Omega\cup S_\alpha)}\lesssim\|b\|_{W^{\sigma,p}(\omega)},
\end{align*}
where the hidden constant depends only on $\Omega,p,\sigma$, $\|\naby\eta\|_{L^\infty_\by}$, $\|\eta\|_{B^\sigma_{\varrho,p}}$ and $L-\alpha$.
\end{lemma}
\begin{proof}
On account of \eqref{210and212}
we have  $\bfPsi_\eta^{-1}\in B^{\sigma+1/p}_{\varrho,p}(\Omega\cup S_\alpha)$ as well.
By \eqref{eq:MSa} and \eqref{eq:MSb} this implies that $\bfPsi_\eta^{-1} \in \mathcal M(W^{\sigma+1/p,p}(\Omega\cup S_\alpha))$. Now 
\eqref{lem:9.4.1} becomes applicable and we obtain
\begin{align*}
\|\mathscr F_\eta b\|_{W^{\sigma+1/p,p}(\Omega\cup S_\alpha)}&\lesssim \|\mathscr F_\Omega ((b\bfn)\circ\bfvarphi^{-1})\|_{W^{\sigma+1/p,p}(\Omega)}\\
&\lesssim \|(b\bfn)\circ\bfvarphi^{-1})\|_{W^{\sigma,p}(\partial\Omega)}
\lesssim\|b\|_{W^{\sigma,p}(\omega)},
\end{align*}
which yields the claim.
\end{proof}
%\textcolor{blue}{
Finally, we prove the following smooth approximation result. For that we also require a solenoidal extension. The proof has been provided in \cite[Proposition 3.3]{MuSc}.
\begin{lemma}
\label{prop:musc}
For a given $\eta\in L^\infty(I;W^{1,2}( \omega ))$ with $\|\eta\|_{L^\infty_{t,y}}<\alpha<L$, there are linear operators
\begin{align*}
\mathscr K_\eta:L^1( \omega )\rightarrow\mathbb R,\quad
\Testzeta:\{\xi\in L^1(I;W^{1,1}( \omega )):\,\mathscr K_\eta(\xi)=0\}\rightarrow L^1(I;W^{1,1}_{\Div}(\Omega\cup S_{\alpha} )),
\end{align*}
such that the tuple $(\Testzeta(\xi-\mathscr K_\eta(\xi)),\xi-\mathscr K_\eta(\xi))$ satisfies
\begin{align*}
\Testzeta(\xi-\mathscr K_\eta(\xi))&\in L^\infty(I;L^2(\Omega_\eta))\cap L^2(I;W^{1,2}_{\Div}(\Omega_\eta)),\\
\xi-\mathscr K_\eta(\xi)&\in L^\infty(I;W^{2,2}( \omega ))\cap  W^{1,\infty}(I;L^{2}( \omega )),\\
\mathrm{tr}_\eta (\Testzeta&(\xi-\mathscr K_\eta(\xi))=\xi-\mathscr K_\eta(\xi),\\
\Testzeta(\xi-\mathscr K_\eta&(\xi))(t,x)=0 \text{ for } (t,x)\in I \times (\Omega \setminus S_{\alpha})
\end{align*}
provided that we have $\xi\in L^\infty(I;W^{2,2}( \omega ))\cap  W^{1,\infty}(I;L^{2}(\omega))$.
In particular, we have the estimates%\todo{Need $W^{2,p}$ and $W^{1,p}$!!}
\begin{align*}%\label{musc1}
\|\Testzeta(\xi-\mathscr K_\eta(\xi))\|_{L^q(I;W^{1,p}(\Omega \cup S_{\alpha}  ))}
&\lesssim \|\xi\|_{L^q(I;W^{1,p}( \omega ))}+\|\xi\nabla \eta\|_{L^q(I;L^{p}( \omega ))},\\
%\label{musc2}
\|\partial_t\Testzeta(\xi-\mathscr K_\eta(\xi))\|_{L^q(I;L^{p}( \Omega\cup S_{\alpha}))}
&\lesssim \|\partial_t\xi\|_{L^q(I;L^{p}( \omega ))}+\|\xi\partial_t \eta\|_{L^q(I;L^{p}( \omega ))},
\\
\|\Testzeta(\xi-\mathscr K_\eta(\xi))\|_{L^q(I;W^{2,p}(\Omega \cup S_{\alpha}  ))}
&\lesssim \|\xi\|_{L^q(I;W^{2,p}( \omega ))}+\|\xi\nabla^2 \eta\|_{L^q(I;L^{p}( \omega ))}
\\
&\quad+\|\abs{\nabla \xi}\abs{\nabla \eta}\|_{L^q(I;L^{p}( \omega ))}+\|\abs{\xi}\abs{\nabla \eta}^2\|_{L^q(I;L^{p}( \omega ))}\\
&\quad+\|\xi\nabla \eta\|_{L^q(I;L^{p}( \omega ))},
\\
%\label{musc3}
\|\partial_t\Testzeta(\xi-\mathscr K_\eta(\xi))\|_{L^q(I;W^{1,p}( \Omega\cup S_{\alpha}))}
&\lesssim \|\partial_t\xi\|_{L^q(I;W^{1,p}( \omega ))}+\|\xi\partial_t \nabla \eta\|_{L^q(I;L^{p}( \omega ))}
\\
&\quad+\|\abs{\partial_t \xi}\abs{\nabla \eta}\|_{L^q(I;L^{p}( \omega ))}+\|\abs{\nabla \xi}\abs{\partial_t \eta}\|_{L^q(I;L^{p}( \omega ))}
\\
&\quad+\|\xi\abs{\partial_t\eta}\abs{\nabla \eta}\|_{L^q(I;L^{p}( \omega ))},
\end{align*}
for any $p\in (1,\infty),q\in(1,\infty]$.
\end{lemma}
With the help of Lemma \ref{prop:musc} we obtain the following.
\begin{lemma}\label{lem:smooth}
For any tuple $(\eta,\bfv)$ belonging to the class
\begin{align}\label{eq:2206}\begin{aligned}
& L^2 \big(I; W^{2,\infty}(\omega) \big) \cap W^{1,2}(I;W^{2,2}(\omega)) %\textcolor{blue}{\cap L^2(I;W^{4,2}(\omega))}
\times L^2 \big(I; W^{1,2}_{\Div}(\Omega_{\eta}) \big)\cap C^0(\overline I;L^2(\Omega_{\eta}) )
%W^{1,2} \big(I; L^{2}(\Omega_{\eta}) \big)\cap  L^2 \big(I; W^{2,2}(\Omega_{\eta}) \big)\cap C^0(\overline I;W^{1,2}_{\Div}(\Omega_{\eta}) ),
\end{aligned}
\end{align}
and satisfying $\bfv\circ\bfvarphi_\eta=\partial_t\eta\bn$ on $\omega$, where $\|\eta\|_{L^\infty_{t,y}}<\alpha<L$ there is a sequence $(\eta_n,\bfv_n)$
which belongs to the class \eqref{eq:2206} and satisfies additionally
\begin{align*}
\eta_n\in C^\infty(I\times \omega),\quad \bfv_n\in W^{1,2} \big(I; W^{1,2}_{\Div}(\Omega_{\eta})\big),
\end{align*}
and $\bfv_n\circ\bfvarphi_{\eta}=(\partial_t\eta_n-\mathscr K_\eta(\partial_t\eta_n))\bfn$ on $\omega$, which converges to $(\eta,\bfv)$ strongly and has uniform bounds in the spaces given in \eqref{eq:2206}.
\end{lemma}
\begin{proof}
The proof is strongly related to \cite[Section 6]{MuSc}. We define
\begin{align*}
\tilde\bfv_0=\bfv-\Testzeta&(\partial_t\eta\bfn),
\end{align*}
which we use as decomposition. 
Note that $\partial_t\eta\bfn$ is the trace of the divergence free function $\bfv$, then it is not difficult to derive that (please refer to \cite[Lemma 6.3]{MuSc} for more details)
\begin{align*}
\mathscr K_\eta(\partial_t\eta\bfn)=0.
\end{align*}
This implies that $\Testzeta(\partial_t\eta\bfn)$ is well defined.  Further $\tilde\bfv_0$ has zero trace on $\partial\Omega_\eta$. 

Each part can be smoothly approximated. The first part uses $\eta_n$ as a smooth approximation of $\eta$ for instance by convolution in space and time. This also makes $\partial_t\eta_n$ and $\partial_t^2\eta_n$ smooth. Thus by Lemma \ref{prop:musc} we obtain
\begin{align*}
\Testzeta&(\partial_t\eta_n\bfn)\in W^{1,2} \big(I; W^{1,2}(\Omega_{\eta}) \big)\cap  L^2 \big(I; W^{2,2}(\Omega_{\eta}) \big)\cap C^0(\overline I;W^{1,2}(\Omega_{\eta}) ).
\end{align*}
Moreover, the tuple $(\eta_n,\Testzeta(\partial_t\eta_n-\mathscr K_\eta(\partial_t\eta_n))\bfn))$ converges to the expected limit with respect to the topology from \eqref{eq:2206}. As was already realized in \cite{MuSc,schwarzacher2022weak} this part of the approximation only uses the regularity of $\eta$ related to the energy estimate.

It is for the approximation  of $\tilde\bfv_0$ that more regularity has to be assumed for $\eta$. This is where  we use the Piola transformation $\mathscr{T}_{\eta}$ that changes the support of a function from $\Omega$ to $\Omega_\eta$ without changing the divergence.
%We define vector fields $\tilde\bfY_\ell$ by solving the Stokes system in $\Omega$ with boundary datum $(\tilde Y_\ell\bn)\circ\bfvarphi^{-1}$. Since the $\tilde Y_\ell$'s are smooth, the regularity of the $\tilde\bfY_\ell$'s is only limited by the regularity of $\partial\Omega$ (the same holds for the $\tilde\bfX_\ell$'s). 
%Now define pointwise in $t$
%\begin{align*}
%\bfX_k:=\tilde\bfX_k\circ \bfPsi_{\mathscr R_\kappa\zeta}^{-1},\quad \bfY_k:=\tilde\bfY_k\circ\bfPsi_{\mathscr R_\kappa\zeta}^{-1}.
%\end{align*}
%By Lemma \ref{lem:diffeo} we still know that $\bfX_k$ and $\bfY_k$ belong to the class $C^3(\overline{\Omega}_{\mathscr R_\kappa\zeta}(t))$.
%Obviously, $(\bfX_k)_{k\in\N}$ forms a basis of $W^{1,2}_0(\Omega_{\regkap \zeta}(t)).$
%Now we choose an enumeration $(\tilde\bfomega_\ell)_{\ell\in\N}$ of $(\tilde\bfX_\ell)_{\ell\in\N}\cup (\tilde\bfY_\ell)_{\ell\in\N}$. In return we associate
%$w_\ell:=(\tilde\bfomega_\ell|_{\partial\Omega} \bn)\circ\bfvarphi$. 
%Obviously, we obtain a basis $(\tilde\bfomega_\ell)_{\ell\in\N}$ of $W^{1,2}(\Omega)$
%and a basis $(w_\ell)_{\ell\in\N}$ of $W^{2,2}(\omega)$. Clearly,
%the set
%\begin{align}\label{eq:2206B}
%\mathrm{Span}\big(\{(\varphi w_\ell,\varphi\tilde\bfomega_\ell):\,\,\varphi \in C^\infty(\overline I),\,\ell\in\N\}\big)
%\end{align}
%is dense in 
%\begin{align*}
% &W^{1,2}\big(I; W^{2,2}(\omega) \big)\cap   \cap W^{2,2}(I;L^2(\omega))\textcolor{blue}{\cap L^2(I;W^{4,2}(\omega))}\\ &\qquad\times W^{1,2} \big(I; L^{2}(\Omega) \big)\cap  L^2 \big(I; W^{2,2}(\Omega) \big)\cap C^0(\overline I;W^{1,2}(\Omega) ).
%\end{align*}
%does the desired provided we are able to transform the velocity fields to the moving domain. #
%we employ the Piola transform
It is defined as
\begin{align*}
%\label{piolaTransform}
\mathscr{T}_{\eta}\bfw
=
\big(\nabx  \bm{\Psi}_{\eta}(\mathrm{det}\nabx  \bm{\Psi}_{\eta})^{-1}
\bfw
\big)\circ \bm{\Psi}_{\eta}^{-1}\text{ with inverse }\mathscr{T}_{\eta}^{-1}\bfw
=
\big((\nabx  \bm{\Psi}_{\eta})^{-1}(\mathrm{det}\nabx  \bm{\Psi}_{\eta}) 
\bfw
\big)\circ \bm{\Psi}_{\eta} .
%, \quad t\in \overline{I}
\end{align*}
Please note that the derivatives of $\mathscr{T}_{\eta}\bfw$ and $\mathscr{T}_{\eta}^{-1}\bfw$ can naturally be bounded by the respective derivatives of $\eta$. Hence as $\eta\in L^2(I,W^{2,\infty}(\omega))\cap W^{1,2}(I;W^{2,2}(\omega))$ we find for any $p,q\in [1,\infty]$ that 
\begin{align*}
\mathscr{T}_{\eta}:  W^{1,2}(I; W^{1,2}_{0,\Div}(\Omega)&\rightarrow  W^{1,2} \big(I; W^{1,2}_{0,\Div}(\Omega_\eta) \big),
\\
\mathscr{T}_{\eta}: L^p \big( I; L^q(\Omega)\big)&\to L^p\big( I; L^{q}(\Omega_\eta)\big),
\\
\mathscr{T}_{\eta}: L^2 \big( I; W^{1,2}_{0,\Div}(\Omega)\big)\cap L^\infty\big(I; L^{2}(\Omega)\big)&\rightarrow L^2 \big(I; W^{1,2}_{0,\Div}(\Omega_\eta) )\cap L^\infty\big( I; L^{2}(\Omega_\eta)\big),
%\\
%\mathscr{T}_{\eta}:  W^{1,2} \big(I; L^2(\Omega) \big)\cap L^\infty(I;L^{2+\sigma}(\Omega))&\rightarrow  W^{1,2} \big(I; L^2(\Omega_\eta) \big),
%\\  
%C^0(\overline I;W^{1,2}_{0,\Div}(\Omega) )&\rightarrow C^0(\overline I;W^{1,2}_{0,\Div}(\Omega_\eta) ),
\end{align*}
with uniform bounds. The same bounds hold for $\mathscr{T}_{\eta}^{-1}$. Estimates yielding these continuities can be shown by direct computations. Note, for instance, that the first estimate requires that $\nabla \eta\in L^\infty(I\times \omega)$.

 The approximation is then defined by first considering 
 $$\mathscr{T}_{\eta}^{-1}\tilde\bfv_0\in L^2 \big(I; W^{1,2}_{0,\Div}(\Omega) )\cap W^{1,2} \big(I; L^2(\Omega) \big).$$ 
 This function can now be smoothly approximated by a sequence $(\mathscr{T}_{\eta}^{-1}\tilde\bfv_0)_n\subset W^{1,2}(I;C^\infty_{c,\Div}(\Omega))$, where $(\mathscr{T}_{\eta}^{-1}\tilde\bfv_0)_n\to\mathscr{T}_{\eta}^{-1}\tilde\bfv_0$ almost everywhere as $n\rightarrow\infty$. This can be achieved by cutting-off the boundary, then applying a Bogovskij operator to make the function solenoidal and convoluting in time-space. We remark that all these operations are linear and continuous in the spaces regarded here. Now we fix
 \[
 \tilde\bfv_{0,n}:=\mathscr{T}_{\eta}(\mathscr{T}_{\eta}^{-1}\tilde\bfv_0)_n.
 \] 
 Now, by construction we have $\tilde\bfv_{0,n}\in W^{1,2} \big(I; W^{1,2}_{\Div}(\Omega_{\eta})\big)$, as required. Furthermore, we find that a.e.
 \begin{align*}
 \abs{\nabla \tilde\bfv_{0,n}}\lesssim \abs{(\mathscr{T}_{\eta}^{-1}\tilde\bfv_0)_n}+\abs{\nabla(\mathscr{T}_{\eta}^{-1}\tilde\bfv_0)_n}
 \end{align*}
with a hidden constant depending on $\abs{\nabla^2\eta}$.
 Hence we obtain that
 \begin{align*}
 \norm{\nabla \tilde\bfv_{0,n}}_{L^2(I,L^2(\Omega_\eta))}&\lesssim \norm{\nabla(\mathscr{T}_{\eta}^{-1}\tilde\bfv_0)_n}_{L^2(I,L^2(\Omega_\eta))}+\norm{(\mathscr{T}_{\eta}^{-1}\tilde\bfv_0)_n}_{L^\infty(I;L^2(\Omega_\eta))}
 \\
 &\lesssim \norm{\tilde\bfv_0}_{L^2(I,W^{1,2}(\Omega_\eta))}+\norm{\tilde\bfv_0}_{ L^\infty(I;L^2(\Omega_\eta))},
 \end{align*}
with a hidden constant depending on $\norm{\nabla \eta}_{L^\infty(I\times\omega)}$ and $\norm{\nabla^2 \eta}_{L^2(I;L^\infty(\omega))}$.
%
%of a vector field $\mathbf{w}$, with $\bfPsi_\eta$ given in \eqref{map}. 
%Due to our assumptions on $\eta$ as well as \eqref{210and212}, the mapping $\bfw\mapsto \bfw\circ \bfPsi_\eta$ maps boundedly
%\begin{align*}
%W^{1,2} \big(I; L^{2}(\Omega)\big)&\rightarrow W^{1,2} \big(I; L^{2}(\Omega_{\eta})  \big);\\
%  L^2 \big(I; W^{2,2}(\Omega) \big)&\rightarrow  L^2 \big(I; W^{2,2}(\Omega_\eta) \big),\\  
%C^0(\overline I;W^{1,2}_{0,\Div}(\Omega) )&\rightarrow C^0(\overline I;W^{1,2}_{0,\Div}(\Omega_\eta) ).
%\end{align*}
%%The Piola transform is invertible with inverse
%%\begin{align}
%%\label{piolaTransformInverse}
%%\mathscr{T}_{\eta}^{-1}\bfw
%%=
%%\big((\nabx  \bm{\Psi}_{\eta})^{-1}(\mathrm{det}\nabx  \bm{\Psi}_{\eta}) 
%%\bfw
%%\big)\circ \bm{\Psi}_{\eta} 
%%%, \quad t\in \overline{I}
%%.
%%\end{align}
%Now we procect the function $\mathscr{T}_{\eta}^{-1}\tilde\bfv_0$ onto the linear hull of $\bfomega_1,\dots,\bfomega_n$ obtaining a function $\tilde\bfv_{0,n}$. Finally, we set $\tilde\bfv_{0,n}:=\mathscr{T}_{\eta}\tilde\bfv_{0,n}$. 
Combining the above, the sequence
$(\eta_n,\tilde\bfv_{0,n}+\Testzeta((\partial_t\eta_n-\mathscr K_\eta(\partial_t\eta_n))\bfn))$ has  the desired properties. 
\end{proof}


\subsection{The concept of solutions}
In this subsection, we introduce the notions of a solution to \eqref{1}--\eqref{interfaceCond} that are under consideration. We start with the definition of a weak solution.
\begin{definition}[Weak solution] \label{def:weakSolution}
Let $(\bff, g, \eta_0, \eta_*, \bu_0)$ be a dataset such that
\begin{equation}
\begin{aligned}
\label{dataset}
&\bff \in L^2\big(I; L^2_{\mathrm{loc}}(\mathbb{R}^3)\big),\quad
g \in L^2\big(I; L^2(\omega)\big), \quad
\eta_0 \in W^{2,2}(\omega) \text{ with } \Vert \eta_0 \Vert_{L^\infty( \omega)} < L, 
\\
& 
\eta_* \in L^2(\omega), \qquad \bu_0\in L^2_{\mathrm{\Div}}(\Omega_{\eta_0}) \text{ is such that }\bu_0\circ\bfvarphi_{\eta_0} =\eta_* \bfn\text{ on $\omega$}.
\end{aligned}
\end{equation} 
We call the tuple
$(\eta,\bu)$
a weak solution to the system \eqref{1}--\eqref{interfaceCond} with data $(\bff, g, \eta_0,  \eta_*,\bu_0)$ provided that the following holds:
\begin{itemize}
\item[(a)] The structure displacement $\eta$ satisfies
\begin{align*}
\eta \in W^{1,\infty} \big(I; L^2(\omega) \big)\cap W^{1,2} \big(I; W^{1,2}(\omega) \big)\cap  L^\infty \big(I; W^{2,2}(\omega) \big) \quad \text{with} \quad \Vert \eta \Vert_{L^\infty(I \times \omega)} <L,
\end{align*}
as well as $\eta(0)=\eta_0$ and $\partial_t\eta(0)=\eta_*$.
\item[(b)] The velocity field $\bu$ satisfies
\begin{align*}
 \bu \in L^\infty \big(I; L^2(\Omega_{\eta}) \big)\cap  L^2 \big(I; W^{1,2}_{\Div}(\Omega_{\eta}) \big) \quad \text{with} \quad 
% \mathrm{tr}_\eta \bu =\partial_t\eta \, \bm{\nu} 
\bu\circ\bfvarphi_\eta =\partial_t \eta {\bfn}\quad\text{on}\quad I\times\omega,
\end{align*}
as well as $\bu(0)=\bu_0$.
%\item[(c)] The pressure $p$ satisfies
%$$p\in L^2(\omega;L^2(\Omega_\eta)).$$
\item[(c)] For all  $(\phi, {\bfphi}) \in C^\infty(\overline{I}\times\omega) \times C^\infty(\overline{I} ;C^\infty_{\Div}(\R^3))$ with $\phi(T,\cdot)=0$, ${\bfphi}(T,\cdot)=0$ and $\bfphi\circ\bfvarphi_\eta =\phi {\bfn}$ on $I\times\omega$, we have
\begin{equation*}
\begin{aligned}
&\int_I  \frac{\mathrm{d}}{\dt}\bigg(\int_\omega \partial_t \eta \, \phi \dy
+
\int_{\Oeta}\bu  \cdot {\bfphi}\dx
\bigg)\dt 
\\
&=\int_I  \int_{\Oeta}\big(  \bu\cdot \partial_t  {\bfphi} + \bu \otimes \bu: \nabla {\bfphi} 
 -  
\nabla \bu:\nabla {\bfphi} +\bff\cdot{\bfphi} \big) \dx\dt
%\\&
%-\int_I  \int_{\Oeta}\big(   
%\nabla \bu:\nabla {\bfphi} -\bff\cdot{\bfphi} \big) \dx\dt
\\
&\quad+
\int_I \int_\omega \big(\partial_t \eta\, \partial_t\phi-\partial_t\naby\eta\cdot\naby \phi+
 g\, \phi-\Dely\eta\,\Dely \phi \big)\dy\dt.
 \end{aligned}
\end{equation*}
\item[(d)] For all $t\in I$, we have
\begin{equation*}\label{energyEst}
\begin{aligned}
&\mathcal{E}(t)
+
\int_0^t\int_\omega\vert\partial_t\naby\eta \vert^2\dy\ds
+
\int_0^t
 \int_{\Omega_{\eta(\sigma)}}\vert \nabla \bu \vert^2 \dx\ds
\\& \leq
 \mathcal{E}(0)
+\frac{1}{2}
\int_0^t \int_\omega  g\partial_t\eta\dy\ds
 + \frac{1}{2}\int_0^t \int_{\Omega_{\eta(\sigma)}}  \bu\cdot \mathbf{f} \dx\ds .
\end{aligned}
\end{equation*}
where
\begin{equation*}
\mathcal{E}(t)
:=
\frac{1}{2}
\int_\omega
\big(
\vert \partial_t\eta(t)\vert^2 + \vert \Dely\eta(t)\vert^2
\big)\dy
+
\frac{1}{2}
\int_{\Omega_{\eta(t)}}\vert   \bu(t) \vert^2 \dx.
\end{equation*}
\end{itemize}
\end{definition}

The existence of a weak solution can be shown as in \cite{LeRu}. The term $\partial_t\Dely\eta$ is not included there, but it does not alter the arguments. 
Note that here, we use a pressure-free formulation (that is, with test-function satisfying additionally $\Div\bfphi=0$). If the solution possesses more regularity,
the pressure can be recovered by solving
%\begin{align}
%\label{eq:press}
%\begin{aligned}
%-\Delta \pi&= \diver \big(\partial_t \bu +\bu\cdot\nabla\bu-\Delta\big)\text{ in }[0,T]\times \Omega_\eta
%\\
%&\pi= \frac{1}{\bn^\intercal\bn_\eta}\Big(\vert \mathrm{det}(\naby \bm{\varphi}_{\eta})\vert^{-1}\varrho_s\partial_t^2\eta -\gamma\partial_t\Dely \eta + \alpha\Dely^2\eta-g)\circ\varphi_{\eta}^{-1}+\bn^\intercal\mu(\nabx\bu+(\nabx\bu)^\intercal)\bn_\eta\Big) \text{ on }[0,T]\times \partial\Omega_\eta
%\end{aligned}
%\end{align}
\begin{align}
\label{eq:press}
\begin{aligned}
-\Delta \pi&= \diver \big(\partial_t \bu +(\bu\cdot\nabla)\bu-\Delta\bu -\bff\big)\qquad&\text{ in }I\times \Omega_\eta,
\\
\pi  &= \big(\vert \mathrm{det}(\naby \bm{\varphi}_{\eta})\vert^{-1}(\partial_t^2\eta - \partial_t\Dely \eta +\Dely^2\eta-g)\bn^\intercal\bn_\eta\big)\circ\bm{\varphi}_{\eta}^{-1}
&%\text{ on }I\times \partial\Omega_\eta.
\\&\quad+\bn\circ\bm{\varphi}_{\eta}^{-1}\big(\nabx\bu+(\nabx\bu)^\intercal \big)\big)\bn_\eta\circ\bm{\varphi}_{\eta}^{-1} &\text{ on }I\times \partial\Omega_\eta.
\end{aligned}
\end{align}
%where $\mathrm{tr}(\cdot)$ denotes the trace of a matrix.
%For $\mathcal O\subset\R^n$ open and bounded with normal $\bfn_{\mathcal O}$ 
%we denote by $\Delta^{-1}_{\mathcal O}\Div$ the solution operator to the equation
%\begin{equation*}
%	\left\{\begin{aligned}
%&\Delta h=\Div\bfg\quad&\text{in}\quad\mathcal O,\\
%&\bfn_{\mathcal O}\cdot(\nabla h-\bfg)=0\quad&\text{on}\quad\partial\mathcal O.
%\end{aligned}\right.
%\end{equation*}
%where $\nu_{\mathcal O}$ denotes the normal of $\partial\mathcal O$.
%where $\Delta_\eta^{-1}$ is the solution operator to the Laplace equation on $\Omega_\eta$ with respect to homogeneous Neumann boundary conditions. 
 Setting
$\pi(t)=\pi_0(t)+c_\pi(t)$, where $\int_{\Omega_{\eta(t)}}\pi_0(t)\dx=0$ and $c_\pi=\pi-\pi_0$ is constant in space and 
testing the structure equation with 1 we obtain
\begin{equation}
\begin{aligned}\label{eq:pressure}
c_\pi(t)\int_{\omega}\bfn\cdot\bfn_\eta|\det(\naby\bfvarphi_\eta)|\dy
&=
\int_{\omega}\bfn\big(\nabla\bu+(\nabla\bu)^\intercal-\pi_0\mathbb I_{3\times 3}\big)\circ\bfvarphi_\eta\bfn_\eta|\det(\naby\bfvarphi_\eta)|\dy
\\&\quad+\int_\omega\partial_t^2\eta\dy-\int_\omega g\dy.
\end{aligned}
\end{equation}
Since $\Omega_\eta$ is $C^1$ uniformly in time, the operator $\Delta$ has the usual regularity and uniqueness properties for $C^1$ domains. In particular, it allows for a unique solution in $L^2$, if the right hand side is in $W^{-2,2}$ and the boundary value in $W^{-\frac{1}{2},2}$ or for a unique solution in $W^{1,2}$, provided that its boundary value is in $W^{\frac{1}{2},2}$ and the right hand side is in $W^{-1,2}$. Moreover, in this particular case, the solution of \eqref{eq:press} satisfies 
\begin{align*}
-\nabla \pi=\partial_t \bu +(\bu\cdot\nabla)\bu-\Delta\bu-\bff,
\end{align*}
distributionally which implies that
\begin{align*}
\int_I\int_{\Omega_\eta}|\nabla\pi|^2\dx\dt&\lesssim \int_I\int_{\Omega_\eta}\big(|\partial_t\bv\vert^2+|(\bu\cdot\nabla)\bu)|^2+|\Delta\bu|^2+\vert\bff\vert^2\big)\dx\dt
\\&\lesssim
\int_I\|\partial_t\bu\|^2_{L^{2}(\Omega_\eta)}\dt
+
\bigg( \int_I\|\bu\|^4_{L^4(\Omega_\eta)}\bigg)^{\frac{1}{2}}\bigg(\int_I\|\nabla\bu\|^4_{L^{4}(\Omega_\eta)}\dt\bigg)^{\frac{1}{2}}
\\&\quad
+\int_I\|\nabla^2\bu\|^2_{L^{2}(\Omega_\eta)}\dt
+\int_I\|\bff\|^2_{L^{2}(\Omega_\eta)}\dt,
\end{align*}
whenever the right hand side is finite, independent of the boundary value of $\pi$ in \eqref{eq:press}.
%holds by Ladyzhenskaya's inequality (using again that $\Omega_\eta$ is Lipschitz uniformly in time). Hence we have $\pi\in L^2(I;W^{1,2}(\Omega_\eta))$ provided that the right-hand side is finite (which is the case if $\bu$ and $\nabla\bu$ belong to $L^4$ in space-time). 
This is the case for a strong solution defined as follows.


\begin{definition}[Strong solution] \label{def:strongSolution}
We call the triple $(\eta,\bu,\pi)$ a strong solution to \eqref{1}--\eqref{interfaceCond} provided that $(\eta,\bu)$ is a weak solution to \eqref{1}--\eqref{interfaceCond}, which satisfies
$$
\eta \in W^{1,\infty} \big(I; W^{1,2}(\omega) \big)\cap W^{1,2} \big(I; W^{2,2}(\omega) \big)\cap  L^\infty \big(I; W^{3,2}(\omega) \big) \cap W^{2,2}(I;L^2(\omega))\cap L^2(I;W^{4,2}(\omega)),$$
$$ \bu \in W^{1,2} \big(I; L^{2}(\Omega_{\eta}) \big)\cap  L^2 \big(I; W^{2,2}(\Omega_{\eta}) \big),\quad\pi\in  L^2 \big(I; W^{1,2}(\Omega_{\eta}) \big).
$$
%and we have $\pi\in L^2(I;W^{1,2}(\Omega_\eta)$%=\nabla\Delta_{\Omega_\eta}^{-1}\Div((\bu \cdot\nabla )\bu-\Delta\bv)$.
%defined via \eqref{eq:press}.
\end{definition}

For a strong solution $(\eta,\bu,\pi)$ the momentum equation holds in the strong sense, that is we have
 \begin{align} 
 \partial_t \bu+(\bu\cdot\nabla)\bu&=\Delta\bu-\nabla \pi+\bff&\label{1'}
 \end{align}
a.e. in $I\times\Omega_\eta$. The shell equation together with the regularity properties above yield $\eta\in L^2(I;W^{4,2}(\omega))$.  Hence the shell equation holds in the strong sense as well, that is, we have
 \begin{align}\label{2'}
\ \partial_t^2\eta-\partial_t\Dely\eta+\Dely^2\eta=g-\bn^\intercal\bm{\tau}\circ\bm{\varphi}_\eta\bn_\eta
\vert \mathrm{det}(\naby \bm{\varphi}_\eta)\vert
 \end{align}
a.e. in $I\times\omega$. 
Note that for a strong solution, the Cauchy stress $\bftau=\nabla\bu+(\nabla\bu)^\intercal-\pi\mathbb I_{3\times 3}$ possesses enough regularity to be evaluated at the moving boundary (this is due to the trace theorem and the uniform Lipschitz continuity of $\Omega_\eta$).
%However, it is occasionally convenient to work with the weak formulation as well. The latter reads as 
%% only holds in the weak sense (because we do not know that $\eta\in L^2(I;W^{4,2}(\omega))$), that is, we have
%\begin{align}\label{2'}
%\int_I \int_\omega \big(\partial_t \eta\, \partial_t\phi-\partial_t\partial_y\eta\,\partial_y \phi-
% g\, \phi \big)\dy\dt-\int_I\int_\omega \partial_y^2\eta\,\partial_y^2 \phi\dy\dt=-\int_I\int_\omega\bfn \bftau\circ\bfvarphi_\eta\bfn_\eta\,\dd y_{\bfn_\eta}\phi\dy\dt.
%\end{align}
%for all $\phi\in C^\infty(\overline I\times\omega)$.




\subsection{The Stokes equations in non-smooth domains}
In this section, we present the necessary framework to parametrise
the boundary of the underlying domain $\Omega\subset\R^3$ by local maps of a certain regularity. This yields, in particular, a rigorous definition of a  $\mathcal M^{s,p}$-boundary. We follow the presentation from \cite{Br} (see also \cite{Br2}).

We assume that $\partial\Omega$ can be covered by a finite
number of open sets $\mathcal U^1,\dots,\mathcal U^\ell$ for some $\ell\in\mathbb N$, such that
the following holds. For each $j\in\{1,\dots,\ell\}$ there is a reference point
$\by^j\in\R^3$ and a local coordinate system $\{\be^j_1,\be^j_2,\be_3^j\}$ (which we assume
to be orthonormal and set $\mathcal Q_j=(\be_1^j|\be_2^j |\be_3^j)\in\mathbb R^{3\times 3}$), a function
$\varphi_j:\mathbb R^{2}\rightarrow\mathbb R$
%$$\varphi_j:\big(\mathbb H^j=y^j+\mathrm{span}\{e^j_1,\dots,e_{n-1}^j\}\big)\cap B_{r_j}(y_j)\rightarrow\mathbb R$$
%and a function $\Psi_j:\mathbb H^j\cap B_{r_j}(y^j)\rightarrow\mathbb R$ (
and $r_j>0$
with the following properties:
\begin{enumerate}[label={\bf (A\arabic{*})}]
\item\label{A1} There is $h_j>0$ such that
$$\mathcal U^j=\{\bx=\mathcal Q_j\bz+\by^j\in\mathbb R^3:\,\bz=(\bz',z_3)\in\R^3,\,|\bz'|<r_j,\,
|z_3-\varphi_j(\bz')|<h_j\}.$$
\item\label{A2} For $\bx\in\mathcal U^j$ we have with $\bz=\mathcal Q_j^\intercal(\bx-\by^j)$
\begin{itemize}
\item $\bx\in\partial\Omega$ if and only if $z_3=\varphi_j(\bz')$;
\item $\bx\in\Omega$ if and only if $0<z_3-\varphi_j(\bz')<h_j$;
\item $\bx\notin\Omega$ if and only if $0>z_3-\varphi_j(\bz')>-h_j$.
\end{itemize}
\item\label{A3} We have that
$$\partial\Omega\subset \bigcup_{j=1}^\ell\mathcal U^j.$$
\end{enumerate}
% Let $x_0\in\partial\Omega$ be boundary point. For simplicity of presentation we can assume by translation and rotation that $x_0=0$ and that the outer normal at~$x_0$ is pointing in the negative $x_n$-direction. Now, let $\phi:\mathbb R^{n-1}\rightarrow\mathbb R$ be the local map describing the boundary on the neighborhood~$U$ of~$0$, that is we have 
In other words, for any $\bx_0\in\partial\Omega$ there is a neighborhood $U$ of $\bx_0$ and a function $\phi:\mathbb R^{2}\rightarrow\mathbb R$ such that after translation and rotation\footnote{By translation via $\by^j$ and rotation via $\mathcal Q_j$ we can assume that $\bx_0=\bm{0}$ and that the outer normal at~$\bx_0$ is pointing in the negative $x_3$-direction.}
 \begin{equation*}\label{eq:3009}
 U \cap \Omega = U \cap G,\quad G = \set{(\bx',x_3)\in \R^3 \,:\, \bx' \in \R^{2}, x_3 > \phi(\bx')}.
 \end{equation*}
% for some $R>0$. Clearly, $\varphi$ and $R$ depend on $x_0$.
 The regularity of $\partial\Omega$ will be described by means of local coordinates as just described.
 \begin{definition}\label{def:besovboundary}
 Let ${\mathcal{O}}\subset\R^3$ be a bounded domain, $s>0$ and $1\leq \rho,q\leq\infty$. We say that $\partial{\mathcal{O}}$ belongs to the class $\bfB^s_{\rho,q}$ if there is $\ell\in\mathbb N$ and functions $\varphi_1,\dots,\varphi_\ell\in\bfB^s_{\rho,q}(\mathbb R^{2})$ satisfying \ref{A1}--\ref{A3}.
 \end{definition}
 % Let $L$ denote the local Lipschitz constant of~$\phi$.
Clearly, a similar definition applies for a Lipschitz boundary (or a $C^{1,\alpha}$-boundary with $\alpha\in(0,1)$) by requiring that $\varphi_1,\dots,\varphi_\ell\in W^{1,\infty}(\mathbb R^{2})$ (or $\varphi_1,\dots,\varphi_\ell\in C^{1,\alpha}(\mathbb R^{2})$). We say that the local Lipschitz constant of $\partial{\mathcal{O}}$, denoted by $\mathrm{Lip}(\partial{\mathcal{O}})$, is (smaller or) equal to some number $L>0$ provided that the Lipschitz constants of $\varphi_1,\dots,\varphi_\ell$ are not exceeding $L$. 
%Our main result depends on the assumption of a sufficiently small Lipschitz constant. While this seems rather restrictive at first glance, it appears quite natural when looking closer. Indeed, it holds, for instance, if the regularity
%of $\partial\Omega$ is better than Lipschitz (such as $C^{1,\alpha}$ for some $\alpha>0$). By means of the transformations $\mathcal Q_j$ introduced above, we can assume that the reference point $y^j$ in question is the origin and that $\nabla\varphi_j(0)=0$. Choosing $r_j$ in \ref{A1} small enough (which can be achieved simply by allowing more sets in the cover $\mathcal U^1,\dots,\mathcal U^l$) we have
%\begin{align*}
%|\nabla\varphi_j(z')|=|\nabla\varphi_j(z')-\nabla\varphi_j(0)|\leq\,r_j^\alpha[\nabla\varphi_j]_{C^\alpha}\ll 1
%\end{align*}
%for all $z'$ with $|z'|\leq r_j$.
\iffalse
In order to describe the behaviour of functions close to the boundary, we need to extend the functions $\varphi_1,\dots,\varphi_\ell$  from \ref{A1}--\ref{A3} to the half space
$\mathbb H := \set{\bm{\xi} = (\bm{\xi}',\xi_3)\,:\, \xi_3 > 0}$. Hence we are confronted with the task of extending a function~$\phi\,:\, \R^{2}\to \R$ to a mapping $\Phi\,:\, \mathbb H \to \R^3$ that maps the $\bm{0}$-neighborhood in~$\mathbb H$ to the $\bx_0$-neighborhood in~$\Omega$. The mapping $(\bm{\xi}',0) \mapsto (\bm{\xi}',\phi(\bm{\xi}'))$ locally maps the boundary of~$\mathbb H$ to the one of~$\partial \Omega$. We extend this mapping using the extension operator of Maz'ya and Shaposhnikova~\cite[Section 9.4.3]{MaSh}. Let $\zeta \in C^\infty_c(B_1(\bm{0}'))$ with $\zeta \geq 0$ and $\int_{\R^{2}} \zeta(\bx')\dx'=1$. Let $\zeta_t(\bx') := t^{-2} \zeta(\bx'/t)$ denote the induced family of mollifiers. We define the extension operator 
\begin{align*}
  (\mathcal{T}\phi)(\bm{\xi}',\xi_3)=\int_{\R^{2}} \zeta_{\xi_3}(\bm{\xi}'-\by')\phi(\by')\dy',\quad (\bm{\xi}',\xi_3) \in \mathbb H,
\end{align*}
where~$\phi:\R^2\to \R$ is a Lipschitz function with Lipschitz constant~$L$.
% and~$\zeta \in C_0^\infty(\Omega)$ with~$\zeta\ge 0$ and~$\int_{\setR^d}\zeta(t)\dt=1$. 
Then the estimate 
\begin{equation*}\label{est:ext}
  \norm{\nabla (\mathcal{T} \phi)}_{W^{s,p}(\setR^{3})}\le c\norm{\nabla \phi}_{W^{s-\frac 1 p,p}(\setR^{2})}
\end{equation*}
follows from~\cite[Theorem 8.7.2]{MaSh}. Moreover, \cite[Theorem 8.7.1]{MaSh} yields
\begin{equation*}\label{eq:MS}
\|\mathcal T\phi\|_{\mathcal M^{s,p}(\mathbb H)}\lesssim \|\phi\|_{\mathcal M^{s-\frac{1}{p},p}(\R^{2})}.
\end{equation*}
It is shown in \cite[Lemma 9.4.5]{MaSh} that (for sufficiently large~$N$, i.e., $N \geq c(\zeta) L+1$) the mapping
\begin{align*}
  \alpha_{z'}(z_3) \mapsto N\,z_3+(\mathcal{T} \phi)(\bz',z_3)
\end{align*}
is one to one for every $\bz' \in \setR^{2}$ and the inverse is Lipschitz with gradient
bounded by $(N-L)^{-1}$.
%\comment[Lars]{Can we say $N \geq 2L +1$?}
Now, we define the mapping~$\bfPhi\,:\, \mathbb H \to \R^3$ as a rescaled version of the latter one by setting
\begin{equation*}\label{eq:Phi}
  \bfPhi(\bm{\xi}',\xi_3)
  :=
    \big(\bm{\xi}',
    \alpha_{\xi_3}(\bm{\xi}')\big) = 
    \big(\bm{\xi}',
    \,\xi_3 + (\mathcal{T} \phi)(\bm{\xi}',\xi_3/N)\big).
\end{equation*}
%\comment[Lars]{We could rescale $\xi_d$ as $\xi_d/N$. This results in
%  \begin{align*}
%    \alpha_{\xi'}(\xi_d) \mapsto \xi_d+(\mathcal{T} \phi)(\xi',\xi_d/N)
%    \\
%    \Phi(\xi',\xi_d)
%    &:=
%      \big(\xi',
%      \alpha_{\xi_d/N}(\xi')\big) = 
%      \big(\xi',
%      \xi_d + (\mathcal{T} \phi)(\xi',x_d/N)\big)
%  \end{align*}
%}%
Thus, $\bfPhi$ is one-to-one (for sufficiently large~$N=N(L)$) and we also define its inverse $\bfPsi := \bfPhi^{-1}$.
%Additionally, it maps a $0$-neighborhood to a  $0$-neighborhood.
 The Jacobian matrix of the mapping $\bfPhi$ is
\begin{align*}%\label{J}
  J = \nabla \bfPhi = 
  \begin{pmatrix}
    \mathbb I_{2\times 2}&0
    \\
    \partial_{\bm{\xi}'} (\mathcal{T}  \phi)& 1+ 1/N\partial_{\xi_3}\mathcal{T}  \phi
  \end{pmatrix}.
\end{align*}
%If the Lipschitz constant if $\varphi$ is small enough the same is true for $\mathcal T\varphi$ due to \eqref{est:ext} (with $s=0$ and $\rho=q=\infty$).
%Hence we can assume that
Since 
$\abs{\partial_{\xi_3}\mathcal{T}  \phi} \leq L$, we have 
\begin{align*}%\label{eq:detJ}
\frac{1}{2} < 1-L/N \leq \abs{\det(J)} \leq 1+L/N\leq 2,
\end{align*}
where we used that $N$ is larger than $L$. Finally, we note the implication
\begin{align} \label{eq:SMPhiPsi}
\bfPhi\in\mathcal M^{s,p}(\mathbb H)\,\,\Rightarrow \,\,\bfPsi\in \mathcal M^{s,p}(\mathbb H),
\end{align}
which holds, for instance, if $\bfPhi$ is Lipschitz continuous, cf. \cite[Lemma 9.4.2]{MaSh}.\fi

After these preparations let us consider the steady Stokes system
\begin{equation}\label{eq:Stokes}
\left\{\begin{aligned}
&\Delta \bu-\nabla\pi=-\bff,	\\
&\Div\bu=0,\\
&\bu|_{\partial{\mathcal{O}}}=\bu_{\partial},
\end{aligned}\right.
\end{equation}
in a domain ${\mathcal{O}}\subset\R^3$ with unit normal $\bfn$. The result given in the following theorem is a maximal regularity estimate for the solution of \eqref{eq:Stokes} in terms of the right-hand side. The boundary data under minimal assumption on the regularity of $\partial\mathcal O$ is obtained in \cite[Theorem 3.1]{Br}. 
\begin{theorem}\label{thm:stokessteady}
Let $p\in(1,\infty)$, $s\geq 1+\frac{1}{p}$ and 
\begin{align*}%\label{eq:SMp}
\varrho\geq p\quad\text{if}\quad p(s-1)\geq 3,\quad \varrho\geq \tfrac{2p}{p(s-1)-1}\quad\text{if}\quad p(s-1)< 3,
\end{align*}
 such that 
$3\big(\frac{1}{p}-\frac{1}{2}\big)+1\leq  s$.
 Suppose that ${\mathcal{O}}$ is a $\bfB^{\theta}_{\varrho,p}$-domain for some $\theta>s-1/p$ with locally small Lipschitz constant, $\bff\in W^{s-2,p}({\mathcal{O}})$ and $\bu_{\partial}\in W^{s-1/p,p}(\partial{\mathcal{O}})$ with $\int_{\partial{\mathcal{O}}}\bu_\partial\cdot\bfn\,\dd\mathcal H^{2}=0$. Then there is a unique solution to \eqref{eq:Stokes} and we have
\begin{align*}%\label{eq:main}
\|\bu\|_{W^{s,p}({\mathcal{O}})}+\|\pi\|_{W^{s-1,p}({\mathcal{O}})}\lesssim\|\bff\|_{W^{s-2,p}({\mathcal{O}})}+\|\bu_{\partial}\|_{W^{s-1/p,p}(\partial{\mathcal{O}})}.
\end{align*}
\end{theorem}
%\todo{As in the proof of Theorem 3.3, precisely the formula (3.22), we might need the estimate for unsteady Stokes system as well?}

\begin{remark}\label{rem:stokes}
{\rm
In Section \ref{sec:reg} we have to apply Theorem \ref{thm:stokessteady} to the domain $\mathcal O={\Omega}_{\eta(t)}$ for a fixed $t$.
Here $\eta:I\times\omega\rightarrow\R$ and $\partial\Omega_{\eta(t)}$ is parametrised via the function $\bfvarphi_{\eta(t)}$ defined on $\omega$, see Section \ref{ssec:geom} for details.
 We exclude self-intersection and degeneracy by assumption (in particular, $\partial_1\bfvarphi_\eta\times\partial_2\bfvarphi_\eta\neq 0$ such that $\bfn_{\eta(t)}$ is well-defined). Given $\bx_0\in \partial{\Omega}_{\eta(t)}$, for some $t\in I$ fixed, we rotate the coordinate system such that $\bfn_{\eta(t)}(\by(\bx_0))=(0,0,1)^\intercal$. Accordingly, it holds
\begin{align*}\mathrm{det}\big(\naby\widetilde\bfvarphi_{\eta(t)}\big)=1,\quad\widetilde\bfvarphi_\eta=\begin{pmatrix}\varphi^1_\eta\\\varphi^2_\eta\end{pmatrix}.
\end{align*}
Hence the function $\widetilde\bfvarphi_{\eta(t)}$ is invertible in a neighborhood $\mathcal U$ of $\by(\bx_0)$. We define in
$\widetilde\bfvarphi_{\eta(t)}(\mathcal U)$ the function
\begin{align*}%\label{eq:phi}
\bfphi(z)
=\begin{pmatrix}z\\ \phi(z))\end{pmatrix}
=\begin{pmatrix}z\\ \varphi^3_{\eta(t)}((\widetilde\bfvarphi_{\eta(t)})^{-1}(z))\end{pmatrix}.
\end{align*}
It describes the boundary $\partial{\Omega}_{\eta(t)}$ close to $\bx_0$.
One easily checks that $\partial_z\phi(z_0)=0$ such that $\partial_z\phi$ is small close to $z_0$. Suppose now that $\eta\in L^\infty(\overline{I};B^{\theta}_{\varrho,p}\cap C^{1}(\omega))$, $\theta>2-1/p$, where $p$ and $\varrho$ are related to \eqref{eq:MSa} and  \eqref{eq:MSb} and that $\displaystyle\sup_I \mathrm{Lip}(\partial\Omega_{\eta(t)})$ is sufficiently small. Then we conclude that 
\begin{align*}%\label{eq:Br0}
\|\phi\|_{\mathcal M(W^{2-1/p,p}_\by)}\leq \delta,\quad \|\phi\|_{W^{1,\infty}_\by}\leq \delta,
\end{align*}
holds uniformly in time for some sufficiently small $\delta$ in a neighbourhood of $z_0$ (using also $\phi(z_0)=0$ and $\partial_z\phi(z_0)=0$).
% In the framework of Theorem \ref{thm:main}
%we have $\eta\in L^\infty(I;W^{2,2}(\omega))$ and ${\Omega}_{\eta(t)}$ is defined in accordance with \eqref{eq:2612}. We must argue that $\partial{\Omega}_\eta\in \bfB^{s}_{2,2}({\mathcal{O}})$ (in the sense of Definition \ref{def:besovboundary}) for some $s>\frac{3}{2}$ and has a small local Lipschitz constant (both uniformly in time). While the Besov regularity is initially clear, we have to introduce local coordinates to control the Lipschitz constant appropriately. Eventually, we must check the Besov regularity again.
%Given $x_0\in \partial{\Omega}_{\eta(t)}$ for some $t\in I$ fixed we can rotate the coordinate system such that $\bfn_{\eta(t)}(y(x_0))=(0,1)^\intercal$ (recall that $\bfn_{\eta(t)}$ is well-defined since $\partial_y\bfvarphi_\eta\neq 0$ by assumption). Accordingly, it holds
%\begin{align*}\partial_y\bfvarphi_{\eta(t)}(y(x_0))=\begin{pmatrix}\partial_y\varphi^1_{\eta(t)}(y(x_0))\\\partial_y\varphi^2_{\eta(t)}(y(x_0))\end{pmatrix}=\begin{pmatrix}1\\0\end{pmatrix}.
%\end{align*}
%Hence the function $\varphi^1_{\eta(t)}$ is invertible in a neighborhood $\mathcal U$ of $y(x_0)$. We define in
%$\varphi^1_{\eta(t)}(\mathcal U)$ the function
%\begin{align*}
%\widetilde\bfvarphi_{x_0}(z)=\begin{pmatrix}z\\ \widetilde\varphi_{x_0}(z))\end{pmatrix}=\begin{pmatrix}z\\ \varphi^2_{\eta(t)}((\varphi^1_{\eta(t)})^{-1}(z))\end{pmatrix}.
%\end{align*}
%It describes the boundary $\partial{\Omega}_{\eta(t)}$ close to $x_0$.
%One easily checks with $z_0=\varphi^1_{\eta(t)}(y(x_0))$ that $\partial_z\widetilde\varphi_{x_0}(z_0)=0$ such that $\partial_z\widetilde\varphi$ is small close to $z_0$. Also, we obtain from the chain rule and the one-dimensional Sobolev embedding that
%$\widetilde\varphi_{x_0}\in W^{2,2}$ and hence $\widetilde\varphi_{x_0}\in W^{s,2}=\bfB^s_{2,2}$ for all $s\in(1,2)$ in a neighborhood of $z_0$.
 }
\end{remark}

\subsection{Universal Bogovskij}

Bogovskij operators are natural to be considered in star-shaped domains. As Lipschitz domains are unions of star-shaped domains, for some time Bogovskij operators are available on Lipschitz domains. Recently the concept of universal Bogovskij operators was introduced in \cite{KamSchSpe20}. Observe that the same Bogovskij operator actually can be used for a family of domains, as long as the Lipschitz constant is controlled. This allows to use a (locally) steady operator to correct the divergence in the time-changing domains.

More precisely in \cite[Corollary 3.4]{KamSchSpe20} the following statement was shown:
\begin{theorem} \label{thm:ndBog1}
	Let $\Sigma \subset \R^{n-1}$ be a bounded Lipschitz-domain, $M > \gamma > 0$, $C_L>0$, $b\in C_0^\infty(\Sigma \times [0,\gamma])$ with unit integral. Then there exists a linear, universal Bogovskij operator $\Bog: C_0^\infty(\Sigma \times [0,M]) \to C_0^\infty(\Sigma \times [0,M];\R^{n-1})$ such that for any $C_L$-Lipschitz function (i.e. with Lipschitz constant $C_L$) $\eta:\Sigma \to [\gamma,M]$ and $\Omega_\eta := \{(\bx',x_n) \in \Sigma \times [0,M]: 0< x_n < \eta(\bx')\}$ the operator $\Bog$ maps $C_0^\infty(\Omega_\eta)$ to $C_0^\infty(\Omega_\eta;\R^n)$ with $\Div\Bog f = f - b \int f \dx$. 
	In addition,
	\[\norm{\Bog(f)}_{W^{s+1,p}( \Omega_\eta;\R^n )} \leq C_B^{s,p}\norm{f}_{W^{s,p}( \Omega_\eta)},\]
	for all $ 1<p<\infty $ and $ s \ge 0$
	with $C_B^{s,p}$ only depending on $s$, $p$, $\operatorname{diam}(\Sigma), C_L,\gamma $ and the Lipschitz properties of $\Sigma$. %
\end{theorem}
In order to make this operator admissible for our needs we introduce the following version. 
\begin{theorem} \label{thm:ndBog}
	There is a universal Bogovskij operator, such that for all $\Omega_\eta$ defined through \eqref{eq:2612} with $\norm{\nabla \eta}_\infty\leq C_L$, $\norm{\eta}_\infty\leq L$ and $b\in C_0^\infty(\Omega\setminus S_L)$ with unit integral
	\[
	\Bog: C_0^\infty(\Omega_\eta) \to C_0^\infty(\Omega_\eta;\R^n)\text{ with }\Div\Bog f = f - b \int f \dx.
	\]
	In addition,
	\[\norm{\Bog(f)}_{W^{s+1,p}( \Omega_\eta;\R^n )} \leq C_B^{s,p}\norm{f}_{W^{s,p}( \Omega_\eta)},\]
	for all $ 1<p<\infty $ and $ s \ge 0$
	with $C_B^{s,p}$ only depending on $L,\varphi,C_L$.
\end{theorem}

\begin{proof}
	The proof is by now standard. One covers the domain $S_L$ with balls of finite overlap, such that on each ball all possible functions $\eta$ can be written as a graph. On these sets one may apply Theorem~\ref{thm:ndBog1}. Hence the partition of unity argument introduced in~\cite[Section 3.1]{SaaSch21} allows to construct the desired operator.
	%% The proof goes along the same lines as in . If we again denote $a:= \tfrac{\gamma}{2L}$, then we can split $\Sigma$ into $O((\operatorname{diam}(\Sigma)/a)^{n-1})$ overlapping sets $(\Sigma_k)_k$ which are each star-shaped with respect to a $n-1$-dimensional cube of side-length $a/c$, where all the constants except $a$ depend only on the boundary of $\Sigma$. We can also assume that $\int_{\Omega_\eta} f \, dx= 0$.
	%% 
	%% Then by the same considerations as before $\Omega_\eta$ is covered by open sets $\Omega_k := \Sigma_k \times [0,M] \cap \Omega_\eta$ which are each again star-shaped w.r.t.\ an $n$-dimensional cube $Q_k$, moreover, we may assume that $Q_k$ are pairwise disjoint. We can then pick a partition of unity $(\psi_k)_k$ subordinate to $(\Sigma_k)_k$ and smooth functions of unit integral $(\phi_k)_k$ each supported in $Q_k$.
	%% 
	%%Now, we may apply the Proposition \ref{prop:reference} to obtain operators $\Bog_k$ that map $C_0^\infty(\Omega_k)$ to $C_0^\infty(\Omega_k;\R^n)$ with $\nabla \cdot \Bog_k f = f$ in case $f$ has mean zero over $\Omega_k$.
	%% 
	%% Hence if we set $T_k f (x) := \psi_k f(x) - \phi_k \int_{\Omega_k} f(y)\psi_k(y) dy$ then a short calculation reveals that
	% \begin{align*}
	%  \Bog f := \sum_k \Bog_k T_k f - \Bog'\left( \sum_k \phi_k \int_{\Omega_k} f(y)\psi_k(y) dy\right)
	% \end{align*}
	% is the desired operator, where $\Bog'$ denotes the Bogovskij-operator on $\Sigma \times [0,\gamma]$. %
\end{proof} 

\begin{remark}[Time-derivative of Bogovskij operator]\label{time-bog}
{\rm Please observe that the same operator can be applied to a time-changing function with time-changing support. The operator then automatically has zero trace on the variable support of the function. In particular for $\sup_t\norm{\nabla \eta(t)}_\infty\leq C_L$, $\sup_t\norm{\eta(t)}_\infty\leq L$, we find that
 $\partial_t \mathcal{B}(f\chi_{\Omega_\eta})=\mathcal{B} (\partial_tf\chi_{\Omega_\eta})$ and $\mathcal{B}(\partial_t f\chi_{\Omega_\eta})=0$ on $\partial\Omega_\eta$.
}
\end{remark}





\section{Local strong solutions}
\label{sec:loc}
Our goal in this section is to construct a local-in-time strong solution of \eqref{1}--\eqref{interfaceCond}. The main theorem is the following:
\begin{theorem}
\label{thm:fluidStructureWithoutFK}
Suppose that the dataset
$(\bff, g, \eta_0, \eta_*, \bu_0)$
satisfies \eqref{dataset} and in addition
\begin{align}
\label{datasetImproved}
\bff\in L^2(I; L^2_{\rm loc}(\mathbb{R}^{3})), \quad
g\in L^2(I; L^2(\omega)),\quad \eta_0 \in W^{3,2}(\omega), \quad \eta_* \in W^{1,2}(\omega), \quad \bu_0 \in W^{1,2}_{\divx}(\Omega_{\eta_0}).
\end{align}
There is a time $T^*>0$ such that there exists a unique strong solution to \eqref{1}--\eqref{interfaceCond} in the sense of Definition \ref{def:strongSolution}.
\end{theorem}


The main ideas to prove Theorem \ref{thm:fluidStructureWithoutFK} are as follows.
\begin{itemize}
\item We transform the fluid-structure system to its reference domain.
\item We then linearize the resulting system on the reference domain and obtain estimates for the linearized system.
\item We construct a contraction map for the linearized problem (by choosing the end time
small enough) which gives the local solution to the system on its original/actual domain.
\end{itemize}
This is reminiscent of the approach in \cite{Br,GraHil,GraHilLe,Le}.

\subsection{Transformation to reference domain}\label{Sectionlinear}
For a solution $( \eta, \bu,  \pi )$  of \eqref{1}--\eqref{interfaceCond}, we set $\overline{\pi}=\pi\circ \Psi_\eta$ and 
$\overline{\bu}=\bu\circ \Psi_\eta$
and define
\begin{equation*}\label{matrices}
\begin{aligned}
\mathbf{A}_\eta=J_\eta\big( \nabx \bfPsi_\eta^{-1}\circ \bfPsi_\eta \big)^\intercal\nabx \bfPsi_\eta^{-1}\circ \bfPsi_\eta,&\\
\mathbf{B}_\eta=J_\eta \left(\nabx \bfPsi_\eta^{-1}\circ \bfPsi_\eta\right)^\intercal,&\\
h_\eta(\overline{\bu})=\big( \mathbf{B}_{\eta_0}-\mathbf{B}_\eta\big):\nabx \overline{\bu},&
\\
\mathbf{H}_\eta(\overline{\bu}, \overline{\pi})
=\
\big( \mathbf{A}_{\eta_0}-\mathbf{A}_\eta\big)\nabx \overline{\bu}
-
\big( \mathbf{B}_{\eta_0}-\mathbf{B}_\eta\big) \overline{\pi},&
\\
\mathbf{h}_\eta(\overline{\bu})
=
(J_{\eta_0}-J_\eta)\partial_t \overline{\bu}
-
J_\eta \nabx\overline{\bu}\cdot \partial_t \bfPsi_\eta^{-1}\circ \bfPsi_\eta -\mathbf{B}_\eta\nabx\overline{\bfv}~\overline{\bu}
+
J_\eta  \bff \circ \bfPsi_\eta,
\end{aligned}
\end{equation*}
where $J_\eta=\mathrm{det}(\nabla\bfPsi_\eta)$.
Exactly as in the two-dimensional case considered in  \cite[Lemma 4.2]{Br} we obtain the following result.
\begin{theorem}
\label{thm:transformedSystem}
Suppose that the dataset
$(\bff, g, \eta_0, \eta_*, \bu_0)$
satisfies \eqref{dataset} and \eqref{datasetImproved}.
Then $( \eta, \bu,  \pi )$ is a strong solution to \eqref{1}--\eqref{interfaceCond} in the sense of Definition \ref{def:strongSolution}, if and only if $( \eta, \overline{\bu},  \overline{\pi} )$ is a strong solution of
\begin{align}
\label{contEqAloneBar}
\mathbf{B}_{\eta_0}:\nabx \overline{\bu}= h_\eta(\overline{\bu}),
\\
\partial_t^2\eta - \partial_t\Dely \eta + \Dely^2\eta
=
g+\bn^\intercal \big[\mathbf{H}_\eta(\overline{\bu}, \overline{\pi})-\mathbf{A}_{\eta_0}  \nabx\overline{\bu} +\mathbf{B}_{\eta_0}\overline{\pi}\big]\circ\bm{\varphi} \bn ,
\label{shellEqAloneBar}
\\
J_{\eta_0}\partial_t \overline{\bu}  -\divx(\mathbf{A}_{\eta_0}  \nabx\overline{\bu}) 
 +\divx(\mathbf{B}_{\eta_0}\overline{\pi}) 
 = 
\mathbf{h}_\eta(\overline{\bu})-
\divx  \mathbf{H}_\eta(\overline{\bu}, \overline{\pi})
\label{momEqAloneBar}
\end{align}
with  $\overline{\bu}  \circ \bm{\varphi}  =(\partial_t\eta)\bn$ on $I\times \omega$.
\end{theorem}
%\begin{proof}
%The proof is a direct adaptation of the $2$-dimensional proof of \cite[Lemma 4.2]{Br}.
%\end{proof}

\subsection{The linearized problem}
In this section, we let $(g, \eta_0, \eta_*, \bu_0)$ be as before in Theorem \ref{thm:transformedSystem}. In addition, we consider $(h,\mathbf{h}, \mathbf{H})$ such that 
\begin{equation}
\begin{aligned}
\label{differentHdata}
&h\in L^2\big(I;W^{1,2}(\Omega)\big) \cap W^{1,2}\big(I;W^{-1,2}(\Omega)\big)\cap \{h(0,\bx)=0\},
\\
&\mathbf{h} \in L^2(I\times\Omega),\quad \mathbf{H}\in L^2(I;W^{1,2}(\Omega)),
\end{aligned}
\end{equation}
and study the following linear system
\begin{align}
\label{contEqAloneBarLinear}
\mathbf{B}_{\eta_0}:\nabx \overline{\bu}= h,
\\
 \partial_t^2\eta - \partial_t\Dely \eta +  \Dely^2\eta
=
g+\bn^\intercal \big[\mathbf{H} -\mathbf{A}_{\eta_0}  \nabx\overline{\bu} +\mathbf{B}_{\eta_0}\overline{\pi}\big]\circ\bm{\varphi} \bn ,
\label{shellEqAloneBarLinear}
\\
J_{\eta_0}\partial_t \overline{\bu}  -\divx(\mathbf{A}_{\eta_0}  \nabx\overline{\bu}) 
 +\divx(\mathbf{B}_{\eta_0}\overline{\pi}) 
 = 
\mathbf{h}-
\divx  \mathbf{H} 
\label{momEqAloneBarLinear}
\end{align}
with  $\overline{\bu}  \circ \bm{\varphi}  =(\partial_t\eta)\bn$ on $I\times \omega$. It is important to note that $\mathbf{B}_{\eta_0}$ and $\mathbf{A}_{\eta_0} $ are time-independent.

\begin{proposition}
\label{thm:transformedSystemLinear}
Suppose that the dataset
$(g, \eta_0, \eta_*, \overline{\bu}_0, h, \mathbf{h},\mathbf{H})$
satisfies \eqref{dataset}, \eqref{datasetImproved} and \eqref{differentHdata}.
Then there exists a strong solution $( \eta, \overline{\bu},  \overline{\pi} )$ of \eqref{contEqAloneBarLinear}--\eqref{momEqAloneBarLinear} such that
\begin{equation}
\begin{aligned}
\label{energyEstLinear}
&\sup_I\int_\omega
\big(\vert \partial_t\naby \eta\vert^2 
+
\vert \naby\Dely \eta\vert^2
\big)
\dy
+
\sup_I\int_\Omega\vert\nabx \overline{\bu}\vert^2\dx
\\&\quad+
\int_I\int_\omega
\big(\vert \partial_t\Dely \eta \vert^2 + \vert \partial_t^2 \eta\vert^2+| \Dely^2\eta|^2
 \big)\dy\dt
 +
\int_I\int_\Omega\big( \vert \nabx^2\overline{\bu}\vert^2 +\vert \partial_t\overline{\bu} \vert^2  +\vert \overline{\pi}\vert^2 + \vert \nabx \overline{\pi}\vert^2
 \big)\dx\dt
 \\&\lesssim
 \int_\omega\big( \vert \eta_*\vert^2
 +
 \vert \naby\eta_*\vert^2
 +
 \vert \Dely\eta_0\vert^2
 +
  \vert \naby\Dely\eta_0\vert^2
  \big)\dy
  +
  \int_\Omega \big(\vert
  \overline{\bu}_0\vert^2
  +
   \vert\nabx\overline{\bu}_0\vert^2 \big)\dx
   \\&\quad+
   \int_I\Vert \partial_t h \Vert_{W^{-1,2}(\Omega)}^2\dt
%   +
% \int_I\int_\omega\big( \vert g\vert^2 + {\color{red}{\vert \naby g\vert^2 }}
%  \big)\dy\dt
   +
 \int_I\int_\omega \vert g\vert^2 \dy\dt
  \\&\quad+
  \int_I\int_\Omega\big(
  \vert h\vert^2
  +
  \vert \nabx h\vert^2
  +
   \vert \mathbf{h}\vert^2 +
  \vert \mathbf{H}\vert^2
  +
  \vert \nabx\mathbf{H}\vert^2
  \big)
 \dx\dt.
\end{aligned}
\end{equation}
\end{proposition}
%\todo{I did not find $\nabla_{\by}g$ on the right of the estimates in the proof}
\begin{proof}
The solution can be constructed by a finite-dimensional Galerkin approximation.
 Let us explain how to construct the Galerkin basis. By solving  eigenvalue problems of the Stokes operator we construct a smooth orthogonal basis $(\tilde \bfX_\ell)_{\ell\in\N}$
of $W^{1,2}_{0,\Div}(\Omega_{\eta_0})$. Further by solving eigenvalue problems for the Laplace operator we construct a smooth orthogonal basis $(Y_\ell)_{\ell\in\N}$ of $$\bigg\{\zeta\in W^{2,2}(\omega):\,\,\int_{\Omega_{\eta_0}}\zeta\circ\bfvarphi_{\eta_0}^{-1}\,\dd\mathcal H^2\bigg\}.$$ Then we set $\bfX_\ell:=\tilde\bfX_\ell\circ\bfPsi_{\eta_0}$, which yields a basis of
\begin{align}\label{eq:2306}
\{\bfw\in W^{1,2}_0(\Omega):\,\,\mathbf B_{\eta_0}:\nabla\bfw=0\}.
\end{align}
We define vector fields $\tilde\bfY_\ell$ by setting $\tilde \bfY_\ell=\mathscr F^{\Div}_\Omega(( Y_\ell \bfn)\circ\bfvarphi^{-1})$, where $\mathscr F^{\Div}_\Omega$ is a solenoidal extension operator. It can be constructed by means of a standard extension and a Bogosvkii correction and thus maps $W^{k,2}(\omega)\rightarrow W^{k,2}(\R^n)$ for $k\in\N$ such that the $\tilde\bfY_\ell$'s are smooth. The functions
$\bfY_\ell:=\tilde\bfY_\ell\circ\bfPsi_{\eta_0}$ also belong to the function space in \eqref{eq:2306}. Now we choose an enumeration $(\bfomega_\ell)_{\ell\in\N}$ of $(\bfX_\ell)_{\ell\in\N}\cup (\bfY_\ell)_{\ell\in\N}$. %In return we associate $w_\ell:=(\bfomega_\ell|_{\partial\Omega} \bfn)\circ\bfvarphi$%
Note that we use $(Y_\ell)_{\ell\in\N}$ as the Ansatz space for $\partial_t\eta$ and not $\eta$, as this is what is related to the fluid-velocity. The terms depending on $\eta$ are hence taken as primitive of the discretization of $\partial_t\eta$.
 %Here one can work the couples $(w_\ell,\tilde\bfomega_\ell)$, $\ell\in\N$, as given in the proof of Lemma \ref{lem:smooth}. Note that they satisfy the correct boundary conditions by construction. 
We now give a formal proof of estimate \eqref{energyEstLinear}, which can be made rigorous with the help of the Galerkin approximation.
%Now define pointwise in $t$
%\begin{align*}
%\bfX_k:=\tilde\bfX_k\circ \bfPsi_{\mathscr R_\kappa\zeta}^{-1},\quad \bfY_k:=\tilde\bfY_k\circ\bfPsi_{\mathscr R_\kappa\zeta}^{-1}.
%\end{align*}
%By Lemma \ref{lem:diffeo} we still know that $\bfX_k$ and $\bfY_k$ belong to the class $C^3(\overline{\Omega}_{\mathscr R_\kappa\zeta}(t))$.
%Obviously, $(\bfX_k)_{k\in\N}$ forms a basis of $W^{1,2}_0(\Omega_{\regkap \zeta}(t)).$

Consider the pair of test functions $(\partial_t \eta, \overline{\bu})$ (which can be used as a test-function on the Galerkin level by construction) for
 \eqref{shellEqAloneBarLinear} and  \eqref{momEqAloneBarLinear}
 respectively. 
We use the ellipticity of $\mathbf{A}_{\eta_0} $,
\begin{equation}
\begin{aligned}
\label{3.8}
&\sup_I
\int_\omega\big(\vert \partial_t\eta\vert^2
+
\vert  \Dely \eta\vert^2 \big)\dy
 +
 \int_I\int_\omega
\vert \partial_t \naby \eta\vert^2 \dy\dt
 +
 \sup_I
 \int_\Omega\vert  \overline{\bu}\vert^2\dx
+
 \int_I\int_\Omega\vert  \nabx\overline{\bu}\vert^2\dx\dt
\\
&\lesssim
\int_\omega\big(\vert \eta_*\vert^2
+
\vert \Dely \eta_0\vert^2 \big)\dy
 +
 \int_I\int_\omega
\vert  g\vert^2 \dy\dt
  +
   \int_\Omega\vert \overline{\bu}_0\vert^2\dx
+
 \int_I\int_\Omega\big(\vert  \mathbf{h}\vert^2 
 +
 \vert  \mathbf{H}\vert^2 
 \big)
 \dx\dt
\\&
\quad+
 \int_I\int_\Omega\big(\vert  h\vert^2 
+
 \vert  \overline{\pi}\vert^2 
 \big)
 \dx\dt.
\end{aligned}
\end{equation}
If we now consider  $(\partial_t^2\eta,\partial_t \overline{\bu})$ as test functions (which can be used as a test-function on the Galerkin level since $B_{\eta_0}$ is time independent) for \eqref{shellEqAloneBarLinear} and \eqref{momEqAloneBarLinear} respectively, then we obtain
\begin{equation*}
\begin{aligned}
\label{3.9}
&
 \int_I\int_\omega
\vert \partial_t^2 \eta\vert^2 \dy\dt
 +
 \frac{1}{2}
 \int_\omega \vert \partial_t \naby\eta\vert^2
\dy
 +
 \int_I\int_\Omega\vert \partial_t \overline{\bu}\vert^2\dx\dt
 +
 \frac{1}{2}
 \int_\Omega\vert \nabx\overline{\bu}\vert^2\dx
\\
&=
\frac{1}{2}
 \int_\omega \vert \naby\eta_*\vert^2\dy
 + 
 \int_I\int_\omega
 \big( g -
  \Dely^2 \eta
 +\bn^\intercal \mathbf{H}\circ \bm{\varphi}\bn
 \big)  \partial_t^2\eta
  \dy\dt
\\&
\quad +
 \frac{1}{2}
 \int_\Omega\vert \nabx\overline{\bu}_0\vert^2\dx
 +
\int_I\int_\Omega\big( \mathbf{h}
-
\divx \mathbf{H} 
 \big)\cdot\partial_t \overline{\bu}
 \dx\dt
+
 \int_I\int_\Omega
 \overline{\pi}\partial_th
 \dx\dt
 .
\end{aligned}
\end{equation*}
We note that
\begin{equation}
\begin{aligned}
\label{3.10}
 -\int_I\int_\omega
   \Dely^2 \eta
 \cdot \partial_t^2\eta
  \dy\dt
  &=
  \int_I\int_\omega
 \partial_t( \naby \Dely \eta
 \cdot \partial_t  \naby\eta)
  \dy\dt
  +
  \int_I\int_\omega
  \vert
 \partial_t  \Dely \eta
 \vert^2
  \dy\dt
  \\&
  \leq
  \frac{1}{4}
  \sup_I\int_\omega
 \vert \partial_t  \naby\eta
 \vert^2
  \dy
  +
  \sup_I\int_\omega
 \vert \naby \Dely \eta
 \vert^2
  \dy
  +
  \int_I\int_\omega
  \vert
 \partial_t  \Dely \eta
 \vert^2
  \dy\dt,
\end{aligned}
\end{equation}
and we thus obtain
\begin{equation}
\begin{aligned}
\label{3.11}
&
 \int_I\int_\omega
\vert \partial_t^2 \eta\vert^2 \dy\dt
 + 
 \sup_I\int_\omega \vert \partial_t \naby\eta\vert^2
\dy
 +
 \int_I\int_\Omega\vert \partial_t \overline{\bu}\vert^2\dx\dt
 + 
 \sup_I\int_\Omega\vert  \nabx\overline{\bu}\vert^2\dx
\\
&\lesssim
 \int_\omega \vert \naby\eta_*\vert^2\dy
+
  \sup_I\int_\omega
 \vert \naby \Dely \eta
 \vert^2
  \dy
   + 
 \int_I\int_\omega
 \big(\vert  g \vert^2
 +
 \vert
 \partial_t  \Dely \eta
 \vert^2
 \big)
  \dy\dt
  \\&\quad + 
\int_I\Vert
 \mathbf{H}\Vert_{W^{1,2}_{\bx}}^2\dt
 +
 \int_\Omega\vert \nabx\overline{\bu}_0\vert^2\dx
 +
\int_I\int_\Omega\big( \vert \mathbf{h}\vert^2
+
\vert
 \nabx\mathbf{H}\vert^2 
 \big) \dx\dt
 \\&
\quad+
 \int_I\Vert \partial_t h\Vert_{W^{-1,2}_\bx}^2\dt
 +
 \int_I\Vert\overline{\pi}\Vert_{W^{1,2}_\bx}^2\dt,
\end{aligned}
\end{equation}
where we have used the trace theorem
\begin{align*}
%\label{3.12}
 \int_I\int_{\partial\Omega}
 \vert
   \mathbf{H}\vert^2
  \dd\mathcal{H}^2\dt
  \lesssim
   \int_I\Vert
 \mathbf{H}\Vert_{W^{1,2}_{\bx}}^2\dt.
\end{align*}
We now test \eqref{shellEqAloneBarLinear}  with $-\partial_t \Dely\eta$ (which can be used as a test-function on the Galerkin level since we used the eigenvalues of the Laplace equation $-\Dely Y_\ell=\lambda_\ell Y_\ell$), to obtain
\begin{equation}
\begin{aligned}
\label{3.13}
&\int_\omega\big( \vert \partial_t \naby\eta\vert^2
+
\vert  \naby\Dely\eta\vert^2
\big)
\dy
   + 
 \int_I\int_\omega
 \vert
 \partial_t \Dely \eta
 \vert^2
  \dy\dt\\
 &\lesssim
 \int_\omega\big( \vert \naby\eta_*\vert^2
+
\vert   \naby\Dely\eta_0\vert^2
\big)
\dy
+ 
\int_I\int_\omega
\vert g \vert^2
\dy\dt
  \\& 
   \quad+
 \int_I\int_\omega
 \big\vert\bn^\intercal\big( \mathbf{H} -\mathbf{A}_{\eta_0}  \nabx\overline{\bu} +\mathbf{B}_{\eta_0} \overline{\pi}\big)\circ\bm{\varphi}  \bn\,\partial_t \Dely\eta\big\vert
  \dy\dt.
\end{aligned}
\end{equation}
For the last term, we have
\begin{equation}
\begin{aligned}
\label{findPressBar}
 &\int_I\int_\omega
 \big\vert\bn^\intercal\big( \mathbf{H} -\mathbf{A}_{\eta_0}  \nabx\overline{\bu} +\mathbf{B}_{\eta_0} \overline{\pi}\big)\circ\bm{\varphi}  \bn\,\partial_t \Dely\eta\big\vert
  \dy\dt
\\&\lesssim
 \int_I
 \big(\Vert \mathbf{H}\Vert_{W^{1/2,2}(\partial\Omega)} + \Vert  \nabx\overline{\bu}\Vert_{W^{1/2,2}(\partial\Omega)} 
 +
\Vert \overline{\pi}\Vert_{W^{1/2,2}(\partial\Omega)}
 \big)\Vert\partial_t \Dely\eta\Vert_{W^{-1/2,2}_{\by}}
 \dt
 \\&\lesssim
 \int_I
  \big(\Vert \mathbf{H}\Vert_{W^{1,2}_{\bx}} + \Vert  \nabx\overline{\bu}\Vert_{W^{1,2}_{\bx}} 
 +
\Vert \overline{\pi}\Vert_{W^{1,2}_{\bx}}
 \big)\Vert\partial_t  \eta\Vert_{W^{3/2,2}_{\by}}
 \dt
  \\&\lesssim
 \int_I
  \big(\Vert \mathbf{H}\Vert_{W^{1,2}_{\bx}} + \Vert   \nabx\overline{\bu}\Vert_{W^{1,2}_{\bx}} 
 +
\Vert \overline{\pi}\Vert_{W^{1,2}_{\bx}}
 \big)\Vert\partial_t  \eta\Vert_{W^{1,2}_{\by}}^{1/2}
 \Vert\partial_t  \eta\Vert_{W^{2,2}_{\by}}^{1/2}
 \dt
  \\&\leq \kappa
 \int_I
 \big(\Vert \mathbf{H}\Vert_{W^{1,2}_{\bx}}^2 + \Vert  \nabx\overline{\bu}\Vert_{W^{1,2}_{\bx}}^2 
 +
\Vert \overline{\pi}\Vert_{W^{1,2}_{\bx}}^2
+ 
\Vert\partial_t  \Dely\eta\Vert_{L^2_{\by}}^2
 \big)\dt
 +
 c(\kappa)
 \int_I
 \Vert\partial_t  \eta\Vert_{W^{1,2}_{\by}}^2
 \dt,
\end{aligned}
\end{equation}
where $\kappa>0$ is arbitrary. Here, we have used the equivalent relation
\begin{align}
\label{equivNorm}
\Vert f\Vert_{W^{2,2}_{\by}}
\lesssim
\Vert \Dely f\Vert_{L^{2}_{\by}}
\lesssim
\Vert f\Vert_{W^{2,2}_{\by}},
\end{align}
which holds for any $f\in W^{2,2}_{\by}$. The second inequality in \eqref{equivNorm} is straightforward whereas the first inequality follows from the estimate $\Vert\naby^2 f\Vert_{L^2_{\by}}
\lesssim
\Vert \Dely f\Vert_{L^{2}_{\by}}$ that is derived by solving the trivial elliptic equation $\Dely f=\Dely f$.
\\
Moreover, notice that by using \eqref{3.8} to estimate the last term in \eqref{findPressBar}, we obtain from \eqref{3.13} and \eqref{3.8} that
\begin{equation}
\begin{aligned}
\label{3.16}
&\sup_I\int_\omega\big( \vert \partial_t \naby\eta\vert^2
+
\vert  \naby\Dely\eta\vert^2
\big)
\dy
   + 
 \int_I\int_\omega
 \vert
 \partial_t \Dely \eta
 \vert^2
  \dy\dt
\\&
\leq
 \kappa
 \int_I
 \big(  \Vert  \nabx\overline{\bu}\Vert_{W^{1,2}_{\bx}}^2 
 +
\Vert \overline{\pi}\Vert_{W^{1,2}_{\bx}}^2
  \big)\dt
  +
c(\kappa)
 \int_\omega\big( \vert \eta_*\vert^2+\vert \naby\eta_*\vert^2
+
\vert \Dely \eta_0\vert^2+
\vert   \naby\Dely\eta_0\vert^2
\big)
\dy
  \\&  \quad+
  c(\kappa)
   \int_\Omega\vert \overline{\bu}_0\vert^2\dx 
   + 
   c(\kappa)
 \int_I\int_\omega
  \vert g \vert^2
  \dy\dt
 +
 c(\kappa)
 \int_I\int_\Omega\big(\vert  h\vert^2+\vert  \mathbf{h}\vert^2 
 +
 \vert  \mathbf{H}\vert^2 
  +
 \vert \nabx \mathbf{H}\vert^2 
 \big)
 \dx\dt.
\end{aligned}
\end{equation}
To find an estimate for the pressure term in \eqref{3.16}, we decompose it into   $\overline{\pi}=\overline{\pi}_0 +c_{\overline{\pi}}$ where $\int_\Omega \overline{\pi}_0\dx=0$ and $c_{\overline{\pi}}$ is only dependent of time. We therefore deduce from
\eqref{shellEqAloneBarLinear} that
\begin{align*}
 c_{\overline{\pi}}\int_\omega\bn^\intercal\mathbf{B}_{\eta_0}\circ\bm{\varphi} \bn 
\dy
=
\int_\omega
\big(
 \partial_t^2\eta 
 -
  g
-
\bn^\intercal \big[
\mathbf{B}_{\eta_0}  \overline{\pi}_0
+
 \mathbf{H} -\mathbf{A}_{\eta_0}  \nabx\overline{\bu} \big]\circ\bm{\varphi} \bn
\big)\dy,
\end{align*}
where we used the zero-mean property of $\eta$.
Since $\mathbf{B}_{\eta_0}$ is uniformly elliptic, it follows from the above and Poincar\'e's inequality that
\begin{equation}
\begin{aligned}
\label{317}
\int_I
\Vert  \overline{\pi} \Vert_{W^{1,2}_{\bx}}^2\dt 
&\lesssim
\int_I
\big(
\Vert \nabx\overline{\pi} \Vert_{L^2_{\bx}}^2 
+
\Vert  \overline{\pi}_0 \Vert_{L^2_{\bx}}^2\big)\dt 
+
\int_I
( c_{\overline{\pi}})^2 \dt
\\&\lesssim
\int_I
\big(
\Vert \nabx\overline{\pi} \Vert_{L^2_{\bx}}^2 
+
\Vert \overline{\pi}_0 \Vert_{L^2_{\bx}}^2\big)\dt  
+
\int_I\int_\omega
\big(
\vert \partial_t^2 \eta\vert^2 
+
\vert  g\vert^2 
\big)\dy\dt
 \\&\quad+ 
  \int_I
 \big(\Vert \pi_0\Vert_{W^{1,2}_{\bx}}^2 
 +
 \Vert \mathbf{H}\Vert_{W^{1,2}_{\bx}}^2 + \Vert  \nabx\overline{\bu}\Vert_{W^{1,2}_{\bx}}^2
 \big)\dt,
\end{aligned}
\end{equation}
where we have used the trace theorem,
\begin{equation*}
\begin{aligned} 
& \int_I
 \big(\Vert \pi_0\Vert_{L^2(\partial\Omega)}^2 
 +
 \Vert \mathbf{H}\Vert_{L^2(\partial\Omega)}^2 + \Vert \nabx\overline{\bu}\Vert_{L^2(\partial\Omega)}^2
 \big)\dt
 \\&
 \lesssim
 \int_I
 \big(\Vert \pi_0\Vert_{W^{1,2}_{\bx}}^2 
 +
 \Vert \mathbf{H}\Vert_{W^{1,2}_{\bx}}^2 + \Vert  \nabx\overline{\bu}\Vert_{W^{1,2}_{\bx}}^2
 \big)\dt.
\end{aligned}
\end{equation*}
Collecting the inequalities from \eqref{3.8}, \eqref{3.11}, \eqref{3.16} and \eqref{317}, we conclude that
\begin{equation}
\begin{aligned}
\label{319}
&\sup_I\int_\omega
\big(\vert \partial_t \naby \eta\vert^2 
+
\vert  \naby\Dely \eta\vert^2
\big)
\dy
+
\sup_I\int_\Omega\vert  \nabx \overline{\bu}\vert^2\dx
\\&\quad+
\int_I\int_\omega
\big(\vert \partial_t \Dely \eta \vert^2 
+
\vert \partial_t \naby \eta\vert^2
+ \vert \partial_t^2 \eta\vert^2
 \big)\dy\dt
 +
\int_I\int_\Omega \vert \partial_t \overline{\bu} \vert^2  
  \dx\dt
 \\&\lesssim
 \int_\omega\big( \vert \eta_*\vert^2
 +
 \vert \naby\eta_*\vert^2
 +
 \vert \Dely\eta_0\vert^2
 +
  \vert \naby\Dely\eta_0\vert^2
  \big)\dy
  +
  \int_\Omega \big(\vert
  \overline{\bu}_0\vert^2
  +
   \vert\nabx\overline{\bu}_0\vert^2 \big)\dx
   \\& 
   \quad+
 \int_I\int_\omega  \vert  g\vert^2  
 \dy\dt
 +
  \int_I\int_\Omega\big(
  \vert  h\vert^2
  +
   \vert  \mathbf{h}\vert^2 +
  \vert  \mathbf{H}\vert^2
  +
  \vert  \nabx\mathbf{H}\vert^2
  \big)
 \dx\dt
 \\&
 \quad+
 \int_I\Vert \partial_th\Vert_{W^{-1,2}_\bx}^2\dt
 +
 \kappa
 \int_I
 \big(\Vert \pi_0\Vert_{W^{1,2}_{\bx}}^2 
  + \Vert   \nabx\overline{\bu}\Vert_{W^{1,2}_{\bx}}^2
  \big)\dt.
\end{aligned}
\end{equation}
Our next goal is to estimate the $\kappa$-terms above to get \eqref{energyEstLinear}. For this, we transform
\eqref{contEqAloneBarLinear} and \eqref{momEqAloneBarLinear} by applying $\bm{\Psi}_{\eta_0}^{-1}$ to them. By setting $\underline{\bu}:=\overline{\bu} \circ \bm{\Psi}_{\eta_0}^{-1}$ and $\underline{\pi}:=\overline{\pi} \circ \bm{\Psi}_{\eta_0}^{-1}$, we obtain
\begin{equation*}
	\left\{\begin{aligned}
&\divx \underline{\bu}=h\circ \bm{\Psi}_{\eta_0}^{-1},
\\&
 \partial_t \underline{\bu}  - \Delx\underline{\bu}
 + \nabx \underline{\pi} 
 = 
 J_{\eta_0}^{-1}\big(
 \mathbf{h}-
\divx  \mathbf{H} \big)\circ \bm{\Psi}_{\eta_0}^{-1},
\end{aligned}\right.
\end{equation*}
in $I \times \Omega_{\eta_0}$ with  $\underline{\bu}  \circ \bm{\varphi}_{\eta_0}  =(\partial_t\eta)\bn$   on $I\times \omega$. Based on the maximal regularity theorem for the classical unsteady Stokes system (Please refer to, for instance, \cite{solonnikov1977}), we obtain
%\todo{It seems we lost the norm of $\underline \bu_0$ in the stokes estimate below} 
\begin{equation*}
\begin{aligned}
&	\int_I\int_{\Omega_{\eta_0}}\big(\vert
 \partial_t \underline{\bu}\vert^2+ \vert \nabx^2\underline{\bu}
 \vert^2
 +
 \vert
 \nabx \underline{\pi} \vert^2\big)\dx\dt\\
 &\lesssim
 \int_I\Vert \partial_t \eta\Vert_{W^{3/2,2}_\by}^2\dt
 +
 \int_I\int_{\Omega_{\eta_0}}
\vert
\nabx( h\circ \bm{\Psi}_{\eta_0}^{-1})
\vert^2\dx\dt
 \\
 &\quad+
 \int_I\int_{\Omega_{\eta_0}}
 \big(\vert
 \mathbf{h}\circ \bm{\Psi}_{\eta_0}^{-1}
\vert^2
+
\vert(\divx  \mathbf{H} )\circ \bm{\Psi}_{\eta_0}^{-1}\vert^2\big)\dx\dt +\int_{\Omega_{\eta_0}}|\nabla\underline \bu_0|^2\dx .
\end{aligned}
\end{equation*}
We now transform back to $\Omega$, 
 interpolate the regularity for the shell and use \eqref{equivNorm}, we obtain for any $\kappa>0$,
\begin{equation*}
\begin{aligned}
	&	\int_I\int_{\Omega}\big(\vert
 \partial_t \overline{\bu}\vert^2+ \vert \nabx^2\overline{\bu}
 \vert^2
 +
 \vert
  \nabx \overline{\pi} \vert^2\big)\dx\dt\\
  & \leq
 \kappa
 \int_I\Vert \partial_t\Dely \eta\Vert_{L^2_\by}^2\dt
 +
 c(\kappa)
 \int_I\Vert \partial_t \eta\Vert_{W^{1,2}_\by}^2\dt
 \\
 &\quad+c
 \int_I\int_{\Omega}
 \big(\vert\nabx h\vert^2 + 
 \vert
 \mathbf{h}
\vert^2
+
\vert    \nabx\mathbf{H}  \vert^2\big)\dx\dt 
+
c\int_{\Omega }|\nabla\overline \bu_0|^2\dx.
\end{aligned}
\end{equation*}
If we now combine this with \eqref{319}, then we obtain the desired estimate \eqref{energyEstLinear} (note that one can finally control $\Dely^2\eta$ by means of equation \eqref{shellEqAloneBarLinear}).
% with $h=0$. 
%To demonstrate how one removes the constraint $h=0$,  
%{\color{blue}
%let us first consider  this inhomogeneous Stokes-like system
%\begin{equation}
%\begin{aligned}
%\label{removeh0}
%\divx(\mathbf{B}_{\eta_0}\overline{p})
%-\divx(\mathbf{A}_{\eta_0}\nabx\overline{\mathbf{u}})=\mathbf{g}, \qquad \mathbf{B}_{\eta_0}:\nabx\overline{\mathbf{u}}=h, \qquad \overline{\mathbf{u}}\big\vert_{\partial\Omega}=0
%\end{aligned}
%\end{equation}
%in $\Omega$ for a given function $h:\Omega\rightarrow\mathbb{R}$ and a smooth force  $\mathbf{g}:\Omega\rightarrow\mathbb{R}^3$. By setting $\underline{p}=\overline{p}\circ\bm{\Psi}_{\eta_0}^{-1}$ and $\underline{\mathbf{u}}=\overline{\mathbf{u}}\circ \bm{\Psi}_{\eta_0}^{-1}$, equation \eqref{removeh0} is transformed into
%\begin{equation}
%\begin{aligned}
%\label{removeh1}
%\nabx\underline{p}
%-\Delx\underline{\mathbf{u}}=J_{\eta_0}^{-1}
% \mathbf{g}\circ \bm{\Psi}_{\eta_0}^{-1}, \qquad \divx\underline{\mathbf{u}}=h\circ \bm{\Psi}_{\eta_0}^{-1}, \qquad \underline{\mathbf{u}}\big\vert_{\partial\Omega_{\eta_0}}=0
%\end{aligned}
%\end{equation}
%in $\Omega_{\eta_0}$. By the \textit{weak} and \textit{strong} maximal regularity result for the Stokes system, see \cite{Ga}, we obtain from \eqref{removeh1} that
%\begin{equation}
%\begin{aligned}
%\label{removeh2}
%&\int_{\Omega_{\eta_0}}\vert\partial_t\underline{\mathbf{u}}\vert^2\dx\lesssim  \Vert J_{\eta_0}^{-1}
% (\partial_t \mathbf{g})\circ \bm{\Psi}_{\eta_0}^{-1}\Vert_{W^{-2,2}(\Omega_{\eta_0})}^2
%+
%\Vert(\partial_t h)\circ \bm{\Psi}_{\eta_0}^{-1}\Vert_{W^{-1,2}(\Omega_{\eta_0})}^2,
%%
%%\int_{\Omega_{\eta_0}}\vert\nabx\underline{\mathbf{u}}\vert^2\dx\lesssim  \int_{\Omega_{\eta_0}}\vert h\circ \bm{\Psi}_{\eta_0}^{-1}\vert^2\dx, 
%%\qquad
%\\&\int_{\Omega_{\eta_0}}\vert\nabx^2\underline{\mathbf{u}}\vert^2\dx\lesssim  
%\int_{\Omega_{\eta_0}}\vert J_{\eta_0}^{-1}
% \mathbf{g}\circ \bm{\Psi}_{\eta_0}^{-1}\vert^2\dx
%+
%\int_{\Omega_{\eta_0}}\vert\nabx( h\circ \bm{\Psi}_{\eta_0}^{-1})\vert^2\dx.
%\end{aligned}
%\end{equation}
%%see \cite[Exercise IV.1.1.]{Ga}.
%%\todo{The first estimate needs to be
%%$\int_{\Omega_{\eta_0}}\vert\partial_t\underline{\mathbf{u}}\vert^2\dx\lesssim  \Vert(\partial_t h)\circ \bm{\Psi}_{\eta_0}^{-1}\Vert_{W^{-1,2}(\Omega_{\eta_0})}^2\dx$. Check if this maximal regularity in negative space holds true and under what condition
%%} 
%If we now transform back to the system \eqref{removeh0} on $\Omega$ and we let $\bm{\mathcal{A}}_{\eta_0}^{-1}$ denotes the solution operator that maps $h$ to $\overline{\mathbf{u}}$, then by using the smoothness of $\bm{\Psi}_{\eta_0}^{-1}$ and $J_{\eta_0}^{-1}$, it follows from \eqref{removeh2} that
%\begin{equation}
%\begin{aligned}
%\label{removeh3}
%&\int_{\Omega}\vert\partial_t\bm{\mathcal{A}}_{\eta_0}^{-1}h\vert^2\dx
%\lesssim  
%\Vert\partial_t \mathbf{g}\Vert_{W^{-2,2}(\Omega)}^2
%+
%\Vert\partial_t h\Vert_{W^{-1,2}(\Omega)}^2,
%%
%%\int_{\Omega}\vert\nabx\bm{\mathcal{A}}_{\eta_0}^{-1}h\vert^2\dx\lesssim  \int_{\Omega}\vert h \vert^2\dx, 
%%\qquad
%\\&
%\int_{\Omega}\vert\nabx^2\bm{\mathcal{A}}_{\eta_0}^{-1}h\vert^2\dx
%\lesssim  
%\int_{\Omega}\vert\mathbf{g} \vert^2\d
%+
%\int_{\Omega}\vert\nabx  h \vert^2\dx.
%\end{aligned}
%\end{equation}
%%\todo{The first estimate needs to be
%%$\int_{\Omega}\vert\partial_t\bm{\mathcal{A}}_{\eta_0}^{-1}h\vert^2\dx\lesssim  \Vert\partial_t h\Vert_{W^{-1,2}(\Omega)}^2\dx$. Check if this maximal regularity in negative space holds true and under what condition
%%} 
%With this information in hand, we can now deal with our actual problem of obtaining \eqref{energyEstLinear} without the condition that $h=0$. Indeed, we note that if $\overline{\bu}$ satisfies \eqref{contEqAloneBarLinear}--\eqref{momEqAloneBarLinear}, then $\overline{\bu}-\bm{\mathcal{A}}_{\eta_0}^{-1}h$ satisfies a similar problem with zero right-hand side in the corresponding equation for \eqref{contEqAloneBarLinear} and where $J_{\eta_0}\partial_t\bm{\mathcal{A}}_{\eta_0}^{-1}h$ appears as an additional term on the right-hand side of the corresponding equation for \eqref{momEqAloneBarLinear}. If we now combine the argument above for the Stokes-like system with the fact that the embedding
%\begin{align*}
%L^2(I;W^{2,2}(\Omega))\cap W^{1,2}(I;L^2(\Omega))\hookrightarrow C(\overline{I};W^{1,2}(\Omega))
%\end{align*}
%is continuous, we obtain
%\begin{equation}
%\begin{aligned}
%\sup_I\int_\Omega \big\vert\nabx\bm{\mathcal{A}}_{\eta_0}^{-1}h  \big\vert^2\dx
%&\lesssim
%\int_I\int_\Omega \big\vert\partial_t\bm{\mathcal{A}}_{\eta_0}^{-1}h  \big\vert^2\dx\dt
%+
%\int_I\int_\Omega \big\vert\nabx^2\bm{\mathcal{A}}_{\eta_0}^{-1}h  \big\vert^2\dx\dt
%\\
%&\lesssim
%\int_I\Vert\partial_t h\Vert_{W^{-1,2}(\Omega)}^2\dt
%+
%\int_I\int_\Omega\vert\nabla h \vert^2\dx\dt
%+
%  \int_I\int_\Omega\big(
%   \vert \mathbf{h}\vert^2 
%  +
%  \vert \nabx\mathbf{H}\vert^2
%  \big)
% \dx\dt
% \\&
% +
%\end{aligned}
%\end{equation}
%}
\end{proof}


\subsection{Fixed-point argument}
Based on Proposition \ref{thm:transformedSystemLinear}, 
in this section, we study the existence of the solution of the nonlinear system \eqref{contEqAloneBar}--\eqref{momEqAloneBar}, by employing the Banach fixed-point argument. 
We assume that the triplet $(\zeta, \overline{\mathbf{w}}, \overline{q})$ are given and we wish to solve
\begin{align}
\label{contEqAloneBarFixed}
\mathbf{B}_{\eta_0}:\nabx \overline{\bu}= h_\zeta(\overline{\mathbf{w}}),
\\
\partial_t^2\eta - \partial_t\Dely \eta +  \Dely^2\eta
=
g+\bn^\intercal \big[\mathbf{H}_\zeta(\overline{\mathbf{w}}, \overline{q})-\mathbf{A}_{\eta_0}  \nabx\overline{\bu} +\mathbf{B}_{\eta_0}\overline{\pi}\big]\circ\bm{\varphi} \bn ,
\label{shellEqAloneBarFixed}
\\
J_{\eta_0}\partial_t \overline{\bu}  -\divx(\mathbf{A}_{\eta_0}  \nabx\overline{\bu}) 
 +\divx(\mathbf{B}_{\eta_0}\overline{\pi}) 
 = 
\mathbf{h}_\zeta(\overline{\mathbf{w}})-
\divx  \mathbf{H}_\zeta(\overline{\mathbf{w}}, \overline{q})
\label{momEqAloneBarFixed}
\end{align}
with  $\overline{\bu}  \circ \bm{\varphi}  =(\partial_t\eta)\bn$ on $I_*\times \omega$. Here, $I_*:=(0,T_*)$ is to be determined later. We define the space
\begin{align*}
X_{I_*}:=&\left(W^{1,\infty}\big(I_*;W^{1,2}(\omega)  \big)  \cap L^{\infty}\big(I_*;W^{3,2}(\omega)  \big) \cap W^{1,2}\big(I_*;W^{2,2}(\omega)  \big)
\cap W^{2,2}\big(I_*;L^{2}(\omega)  \big)\right)
\\&
\times
\left(L^\infty \big(I_*; W^{1,2}(\Omega ) \big)\cap  W^{1,2}\big(I_*;L^{2}(\Omega)  \big)\cap L^{2}\big(I_*;W^{2,2}(\Omega)  \big)\right)
\times
L^2\big(I_*;W^{1,2}(\Omega)  \big),
\end{align*}
equipped with the norm
\begin{align*}
\Vert (\zeta,\overline{\mathbf{w}}, \overline{q}) \Vert_{X_{I_*}}^2
&:=
\sup_{I_*}\int_\omega\big( \vert \partial_t\zeta\vert^2 +  \vert \partial_t\naby\zeta\vert^2 +  \vert\Dely\zeta \vert^2 +  \vert\naby\Dely\zeta \vert^2 \big)\dy
\\&\quad+
\int_{I_*}\int_\omega\big( \vert \partial_t\naby\zeta\vert^2 +  \vert \partial_t\Dely\zeta\vert^2 +  \vert\partial_t^2\zeta \vert^2 \big)\dy\dt
\\&
\quad+\sup_{I_*}\int_\Omega\big( \vert \overline{\mathbf{w}}\vert^2 +  \vert \nabx \overline{\mathbf{w}} \vert^2  \big)\dx
\\&\quad+
\int_{I_*}\int_\Omega\big( \vert \nabx \overline{\mathbf{w}}\vert^2 +  \vert \nabx^2 \overline{\mathbf{w}} \vert^2 +  \vert\partial_t  \overline{\mathbf{w}} \vert^2
+\vert \overline{q}\vert^2 +\vert \nabx\overline{q}\vert^2 \big)\dx\dt.
\end{align*}


Let $B_R^{X_{I_*}}$ be a ball defined as
\begin{align*}
B_R^{X_{I_*}}:= \big\{ (\zeta,\overline{\mathbf{w}}, \overline{q})\in X_{I_*}
\text{ with } \quad \zeta(0)=\eta_0, \quad \partial_t\zeta(0) = \eta_*,\quad \overline{\mathbf{w}}(0)=\overline \bu_0  \, :\, \Vert(\zeta,\overline{\mathbf{w}}, \overline{q})\Vert_{X_{I_*}}^2\leq R \big\},
\end{align*}
for some $R>0$ large enough. 

\begin{theorem}\label{fixedpoint}
	There exists a time $T>0$ and $R>0$, such that the map $\mathcal{T}$, defined by 
	\begin{equation*}
	\begin{aligned}
	\mathcal{T}:B_R^{X_{I_*}}&\rightarrow B_R^{X_{I_*}}\\
	(\zeta, \overline{\mathbf{w}}, \overline{q})&\mapsto(\eta,\overline{\bu}, \overline{\pi}),
	\end{aligned}
	\end{equation*}
 is a contraction map, which thereby possesses a fixed point in $X_{I_*}$.
\end{theorem}
\begin{proof}
%To show the mapping  $\mathcal{T}:B_R^{X_{I_*}} \rightarrow  B_R^{X_{I_*}}$, we need to show that for any $(\zeta, \overline{\mathbf{w}}, \overline{q}) \in B_R^{X_{I_*}}$, we have that 
%\begin{align}
%\label{ballToBall}
%\Vert\mathcal{T}(\zeta, \overline{\mathbf{w}}, \overline{q})\Vert_{X_{I_*}}^2
%=
%\Vert(\eta, \overline{\bu}, \overline{\pi})\Vert_{X_{I_*}}^2\leq R.
%\end{align}
%Indeed, using \eqref{energyEstLinear}, we deduce that \eqref{contEqAloneBarFixed}-\eqref{momEqAloneBarFixed} satisfies
% \todo{we don't need to minus $\zeta,\overline{\mathbf{w}}$ terms in the first line (3.28), actually it should be $\eta, \overline\bu$ form and all the norms in $X_{I_*}$ can be controlled by adding energy estimate (3.10) and (3.9) together; the right of (3.18) is OK}
%\begin{equation}
%\begin{aligned}
%&\Vert(\eta, \overline{\bu}, \overline{\pi})\Vert_{X_{I_*}}^2
%-\sup_{I_*}\int_\omega\big( \vert \partial_t\zeta\vert^2  +  \vert\Dely\zeta \vert^2\big)\dy
%-
%\int_{I_*}\int_\omega  \vert \partial_t\naby\zeta\vert^2\dy\dt
%-\sup_{I_*}\int_\Omega\vert \overline{\mathbf{w}}\vert^2 \dx
%-
%\int_{I_*}\int_\omega \vert \nabx \overline{\mathbf{w}}\vert^2\dx\dt
%\\
%&\lesssim
% \int_\omega\big( \vert \eta_*\vert^2
% +
% \vert \naby\eta_*\vert^2
% +
% \vert \Dely\eta_0\vert^2
% +
%  \vert \naby\Dely\eta_0\vert^2
%  \big)\dy
%  +
%  \int_\Omega \big(\vert
%  \overline{\bu}_0\vert^2
%  +
%   \vert\nabx\overline{\bu}_0\vert^2 \big)\dx
%   \\&\quad+
%   \int_{I_*}\Vert \partial_t h_\zeta(\overline{\mathbf{w}}) \Vert_{W^{-1,2}(\Omega)}^2\dt
%   +
% \int_{I_*}\int_\omega\big( \vert g\vert^2 + \vert \naby g\vert^2 
%  \big)\dy\dt
%  \\&\quad+
%  \int_{I_*}\int_\Omega\big(
%  \vert  h_\zeta(\overline{\mathbf{w}})\vert^2
%  +
%  \vert \nabx h_\zeta(\overline{\mathbf{w}})\vert^2
%  +
%   \vert \mathbf{h}_\zeta(\overline{\mathbf{w}})\vert^2 +
%  \vert \mathbf{H}_\zeta(\overline{\mathbf{w}}, \overline{q})\vert^2
%  +
%  \vert \nabx\mathbf{H}_\zeta(\overline{\mathbf{w}}, \overline{q})\vert^2
%  \big)
% \dx\dt.
%\end{aligned}
%\end{equation}
%By using $(\zeta, \overline{\mathbf{w}}, \overline{q}) \in B_R^{X_{I_*}}$, we can show that \todo{this seems not trivial we need to put more details} 
%\begin{align*}
%&\sup_{I_*}\int_\omega\big( \vert \partial_t\zeta\vert^2  +  \vert\Dely\zeta \vert^2\big)\dy
%+
%\int_{I_*}\int_\omega  \vert \partial_t\naby\zeta\vert^2\dy\dt
%+\sup_{I_*}\int_\Omega\vert \overline{\mathbf{w}}\vert^2 \dx
%+
%\int_{I_*}\int_\omega \vert \nabx \overline{\mathbf{w}}\vert^2\dx\dt
%\\&
%+
%\int_{I_*}\Vert \partial_t h_\zeta(\overline{\mathbf{w}}) \Vert_{W^{-1,2}(\Omega)}^2\dt
%+
%  \int_{I_*}\int_\Omega\big(
%   \vert  h_\zeta(\overline{\mathbf{w}})\vert^2
%  +
%  \vert \nabx h_\zeta(\overline{\mathbf{w}})\vert^2
%  \big)
% \dx\dt
% \\&
% +
%  \int_{I_*}\int_\Omega\big(
%   \vert \mathbf{h}_\zeta(\overline{\mathbf{w}})\vert^2 +
%  \vert \mathbf{H}_\zeta(\overline{\mathbf{w}}, \overline{q})\vert^2
%  +
%  \vert \nabx\mathbf{H}_\zeta(\overline{\mathbf{w}}, \overline{q})\vert^2
%  \big)
% \dx\dt
% \lesssim T_*^{1/n}R,
%\end{align*}
%for some $n\in\mathbb{N}$ (see the contraction argument below with a choice of second solution and set that is identically zero). 
%Let $\tilde{c}$ be the maximum of the constants in the two inequalities above and let
%\begin{equation}
%\begin{aligned}
%f_0:= &\int_\omega\big( \vert \eta_*\vert^2
% +
% \vert \naby\eta_*\vert^2
% +
% \vert \Dely\eta_0\vert^2
% +
%  \vert \naby\Dely\eta_0\vert^2
%  \big)\dy
%  +
%  \int_\Omega \big(\vert
%  \overline{\bu}_0\vert^2
%  +
%   \vert\nabx\overline{\bu}_0\vert^2 \big)\dx
%   \\&
%   +
% \int_I\int_\omega\big( \vert g\vert^2 + \vert \naby g\vert^2 
%  \big)\dy\dt+1.
%\end{aligned}
%\end{equation}
%For $R>0$ large enough, so that in particular $\tilde{c}f_0<R$, the choice of  $
%T_*^{1/n}=\frac{1}{\tilde{c}}-\frac{f_0}{R}>0$
%yields our desired claim \eqref{ballToBall}.
We would like to show that the map $\mathcal{T}$ defined above maps the ball $B_R^{X_{I_*}} $ into itself and that for any $(\zeta_i, \overline{\mathbf{w}}_i, \overline{q}_i)\in B_R^{X_{I_*}}$, for $i=1,2$, we can find $\rho<1$ such that
\begin{align*}
	\Vert \mathcal{T}(\zeta_1, \overline{\mathbf{w}}_1, \overline{q}_1)
	-
	\mathcal{T}(\zeta_2, \overline{\mathbf{w}}_2, \overline{q}_2)\Vert_{X_{I_*}}\leq \rho \Vert (\zeta_1, \overline{\mathbf{w}}_1, \overline{q}_1)-(\zeta_2, \overline{\mathbf{w}}_2, \overline{q}_2)\Vert_{X_{I_*}}.
\end{align*}
To present the proof clearly, we divide it in the following two steps.
\\ \\
{\bf Step 1:} {\em We show that $\mathcal{T}:B_R^{X_{I_*}} \rightarrow  B_R^{X_{I_*}}$, i.e. the ball $B_R^{X_{I_*}}$ is $\mathcal{T}$-invariant.}
To do this, we need to show that for any $(\zeta, \overline{\mathbf{w}}, \overline{q}) \in B_R^{X_{I_*}}$, we have 
\begin{align}
\label{ballToBall}
\Vert\mathcal{T}(\zeta, \overline{\mathbf{w}}, \overline{q})\Vert_{X_{I_*}}^2
=
\Vert(\eta, \overline{\bu}, \overline{\pi})\Vert_{X_{I_*}}^2
\leq R.
\end{align}
Indeed, according to \eqref{energyEstLinear}, we deduce the following estimate from \eqref{contEqAloneBarFixed}--\eqref{momEqAloneBarFixed} 
\begin{equation}
\begin{aligned}
\label{energyEstLinearForBall}
\Vert(\eta, \overline{\bu}, \overline{\pi})\Vert_{X_{I_*}}^2
&\lesssim
\mathcal{D}(g, \eta_0, \eta_*,\overline{\bu}_0)
 +
   \int_{I_*}\big(\Vert \partial_t h_{\zeta}(\overline{\mathbf{w}}) \Vert_{W^{-1,2}(\Omega)}^2\dt
 +
  \Vert h_{\zeta}(\overline{\mathbf{w}})\Vert_{W^{1,2}(\Omega)}^2
  \big)
 \dt
 \\&
\qquad  
+
\int_{I_*}\big(
   \Vert \mathbf{h}_{\zeta}(\overline{\mathbf{w}})\Vert_{L^{2}(\Omega)}^2 +
  \Vert \mathbf{H}_{\zeta}(\overline{\mathbf{w}}, \overline{q})\Vert_{W^{1,2}(\Omega)}^2
  \big)
 \dt
 \\&
 =:\mathcal{D}(g, \eta_0,\eta_*,\overline{\bu}_0)+\overline{K}_1+\overline{K}_2+\overline{K}_3+\overline{K}_4,
\end{aligned}
\end{equation}
where
\begin{equation*}
\begin{aligned}
		\mathcal{D}(g, \eta_0,\eta_*,\overline{\bu}_0)
		&=
		\Vert \eta_*\Vert_{W^{1,2}(\omega)}^2
		+
		\Vert \eta_0\Vert_{W^{3,2}(\omega)}^2
		+
		\Vert
		\overline{\bu}_0\Vert_{W^{1,2}(\Omega)}^2
		+
		\int_{I_*} \Vert g\Vert_{L^2(\omega)}^2  \dt.
\end{aligned}
\end{equation*}
Recalling the regularity assumption in \eqref{datasetImproved}, we choose $R>0$ large enough, such that
\begin{align}
\label{cDataR}
c \mathcal{D}(g, \eta_0, \eta_*,\overline{\bu}_0)
\leq \frac{1}{2}R,
\end{align}
where $c>0$ is the constant in \eqref{energyEstLinearForBall}. We show in what follows that the sum of the $\overline{K}_i$ is also bounded by $\frac{1}{2}R$ which will give the estimate \eqref{ballToBall}.
 
To estimate $\overline{K}_1$, we have
\begin{align*}
\partial_t h_{\zeta}(\overline{\mathbf{w}} )
&=
 (\mathbf{B}_{\eta_0}-\mathbf{B}_{\zeta}):\partial_t\nabx\overline{\mathbf{w}} 
 -
  (\partial_t\mathbf{B}_{\zeta}):\nabx\overline{\mathbf{w}}.
\end{align*}
We notice that the continuous embedding 
\begin{align}\label{eq:emb1}
L^\infty(I_*;W^{3,2}(\omega))\cap W^{1,2}(I_*;W^{2,2}(\omega))\hookrightarrow C^{0, 1/8}(I_*; W^{11/4, 2}(\omega))\hookrightarrow
L^\infty(I_*;W^{1,\infty}(\omega)),
\end{align}
scales with $T_*^{1/8}$. 
We thereby obtain from \eqref{210and212}--\eqref{218} that
\begin{equation}
\begin{aligned}
\label{k1ax}
\int_{I_*}\Vert (\mathbf{B}_{\eta_0}-\mathbf{B}_{\zeta}):\partial_t\nabx \overline{\mathbf{w}} \Vert_{W^{-1,2}_{\bx}}
^2\dt
&\lesssim
\int_{I_*}\Vert  \mathbf{B}_{\eta_0}-\mathbf{B}_{\zeta}\Vert_{L^\infty_\bx}^2\Vert\partial_t \overline{\mathbf{w}} \Vert_{L^2_\bx}^2
\dt
\\&\lesssim 
\sup_{I_*}\Vert  \eta_0 - \zeta\Vert_{W^{1,\infty}_\by}^2\int_{I_*}\Vert  \partial_t \overline{\mathbf{w}}  \Vert_{L^2_\bx}^2
\dt
\\&
\lesssim
T^{1/4}_*
\Vert (\zeta, \overline{\mathbf{w}}, \overline{q})\Vert_{X_{I_*}}^2.
\end{aligned}
\end{equation}
On the other hand, due to the continuous embedding
\begin{align*}
W^{2,2}(I_*;L^2(\omega))\cap W^{1,2}(I_*;W^{2,2}(\omega))\hookrightarrow W^{5/4,2}(I_\ast;W^{3/2,2}(\omega)) \hookrightarrow
W^{1,4}(I_*;W^{1,4}(\omega)),
\end{align*}
it follows from H\"older inequality that
\begin{equation}
\begin{aligned}
\label{k1cx}
\int_{I_*}\Vert\partial_t \mathbf{B}_{\zeta}:\nabx \overline{\mathbf{w}}\Vert_{W^{-1,2}_\bx}^2
\dt
&\lesssim
\int_{I_*}\Vert\partial_t \mathbf{B}_{\zeta}:\nabx \overline{\mathbf{w}} \Vert_{L^{4/3}_\bx}^2
\dt
\\&\lesssim
\int_{I_*}\Vert  \partial_t\mathbf{B}_{\zeta}\Vert^2_{L^4_\bx}\Vert\nabx \overline{\mathbf{w}} \Vert_{L^2_\bx}^2
\dt
\\&\lesssim T^{1/2}_*
\bigg(\int_{I_*} \Vert  \partial_t\zeta\Vert_{W^{1,4}_\by}^4\dt\bigg)^\frac{1}{2}
\sup_{I_*}\Vert  \nabx\overline{\mathbf{w}}\Vert_{L^2_\bx}^2
\\&
\lesssim
T^{1/2}_*
\Vert (\zeta, \overline{\mathbf{w}}, \overline{q})\Vert_{X_{I_*}}^2.
\end{aligned}
\end{equation} 
Combining \eqref{k1ax} with \eqref{k1cx}, we have
\begin{equation}
\begin{aligned}
\label{k1finalx}
\overline{K}_1
\lesssim T^{1/2}_*
\Vert (\zeta, \overline{\mathbf{w}}, \overline{q}) \Vert_{X_{I_*}}^2.
\end{aligned}
\end{equation}

To estimate $\overline{K}_2$, we note that
\begin{equation*}
	\begin{aligned}
\vert
 h_{\zeta}(\overline{\mathbf{w}})
\vert
+
\vert
\nabx h_{\zeta}(\overline{\mathbf{w}})
\vert
&\lesssim
\vert (\mathbf{B}_{\eta_0}-\mathbf{B}_{\zeta}) :\nabx \overline{\mathbf{w}}
\vert
+
\vert (\mathbf{B}_{\eta_0}-\mathbf{B}_{\zeta}):\nabx^2 \overline{\mathbf{w}}
\vert
\\&
\quad+
\vert \nabx(\mathbf{B}_{\eta_0}-\mathbf{B}_{\zeta}):\nabx \overline{\mathbf{w}}
\vert.
\end{aligned}
\end{equation*}
Using  the argument in \eqref{eq:emb1} again,
we derive from \eqref{210and212}--\eqref{218} that
\begin{equation}
\begin{aligned}
\label{K2ax}
&\int_{I_*}\Vert   (\mathbf{B}_{\eta_0}-\mathbf{B}_{\zeta}) :\nabx \overline{\mathbf{w}} \Vert_{L^2_\bx}^2\dt
+
\int_{I_*}\Vert   (\mathbf{B}_{\eta_0}-\mathbf{B}_{\zeta}): \nabx^2  \overline{\mathbf{w}} \Vert_{L^2_\bx}^2\dt
\\&\lesssim
\sup_{I_*}\Vert    \eta_0 - \zeta \Vert_{W^{1,\infty}_\by}^2
\int_{I_*} \Vert\nabx^2 \overline{\mathbf{w}} \Vert_{L^2_\bx}^2\dt
\\&\lesssim
T^{1/4}_*
\Vert (\zeta, \overline{\mathbf{w}}, \overline{q}) \Vert_{X_{I_*}}^2.
\end{aligned}
\end{equation}
According to the continuous embeddings:
\begin{align}
&L^\infty(I_*;W^{3,2}(\omega)) \hookrightarrow
L^\infty(I_*;W^{2,4}(\omega)), 
\label{eq:emb2}
\end{align}
and
\begin{align}
\begin{aligned}
\,W^{1,2}(I_*;L^2(\Omega))\cap L^2(I_*;W^{2,2}(\Omega))
&\hookrightarrow W^{1/8,2}(I_*;W^{7/4,2}(\Omega))
\hookrightarrow
L^2(I_*;W^{1,4}(\Omega)),
\end{aligned}
\label{eq:emb3}
\end{align}
where the latter scales with $T^{1/8}$,
it follows that
\begin{equation}
\begin{aligned}
\label{K2bx}
\int_{I_*}\Vert  \nabx (\mathbf{B}_{\eta_0}-\mathbf{B}_{\zeta}) \nabx \overline{\mathbf{w}} \Vert_{L^2_\bx}^2\dt
&\lesssim
\sup_{I_*}\Vert   \eta_0 - \zeta \Vert_{W^{2,4}_\by}^2
\int_{I_*} \Vert\nabx  \overline{\mathbf{w}}\Vert_{L^4_\bx}^2\dt
\\&\lesssim
T^{1/4}_*
\Vert (\zeta, \overline{\mathbf{w}}, \overline{q}) \Vert_{X_{I_*}}^2.
\end{aligned}
\end{equation}
We obtain from \eqref{K2ax} and \eqref{K2bx} that
\begin{align}
\label{k2finalx}
\overline{K}_2
%:=
%\int_I\Vert\nabx[ h_{\zeta_1}(\overline{\mathbf{w}}_1)- h_{\zeta_2}(\overline{\mathbf{w}}_2)]\Vert_{L^2(\Omega)}^2\dt
\lesssim
 T^{1/4}_* 
\Vert (\zeta,\overline{\mathbf{w}}, \overline{q}) \Vert_{X_{I_*}}^2.
\end{align} 

To estimate $\overline{K}_3$, let us recall that
\begin{align*}
\mathbf{h}_{\zeta}(\overline{\mathbf{w}})
=
(J_{\eta_0}-J_{\zeta})\partial_t \overline{\mathbf{w}}
+
J_{\zeta} \nabx \overline{\mathbf{w}}  \partial_t \Psi_{\zeta}^{-1}\circ \Psi_{\zeta} 
+
\mathbf{B}_\zeta \nabx \overline{\mathbf{w}}\,\overline{\mathbf{w}}
+
J_{\zeta} \bff\circ \Psi_{\zeta}.
\end{align*}
Due to the continuous embeddings \eqref{eq:emb1}, 
it follows from the definition $J_\eta=\det(\nabx \Psi_\eta)$ and \eqref{210and212}--\eqref{218}  that
\begin{equation}
\begin{aligned}
\label{K3ax}
\int_{I_*}\Vert   (J_{\eta_0}-J_{\zeta})\partial_t \overline{\mathbf{w}}\Vert_{L^2_\bx}^2\dt
&\lesssim
\sup_{I_*}\Vert  \eta_0 - \zeta\Vert_{W^{1,\infty}_\by}
\int_{I_*}\Vert \partial_t\overline{\mathbf{w}} \Vert_{L^2_\bx}^2\dt
\\&\lesssim
T^{1/4}_*
\Vert (\zeta,\overline{\mathbf{w}}, \overline{q})  \Vert_{X_{I_*}}^2.
\end{aligned}
\end{equation}
By using the embeddings
\begin{align*}
&L^\infty(I_*;W^{3,2}(\omega))\hookrightarrow L^\infty(I_*;W^{1,\infty}(\omega)),
\\&
W^{1,\infty}(I_*;W^{1,2}(\omega))
\cap
W^{1,2}(I_*;W^{2,2}(\omega))\hookrightarrow W^{1,4}(I_*; W^{3/2, 2}(\omega)) \hookrightarrow
W^{1,4}(I_*;L^{\infty}(\omega)),
\end{align*}
we obtain that
\begin{equation}
\begin{aligned}
\label{K3bx}
&\int_{I_*}\Vert   J_{\zeta}  \nabx \overline{\mathbf{w}} \partial_t \Psi_{\zeta}^{-1}\circ \Psi_{\zeta} \Vert_{L^2_\bx}^2\dt
\\&\lesssim
\int_{I_*}\big(1+\Vert   \zeta
\Vert_{W^{1,\infty}_\by}^2
\big)
\Vert \nabx \overline{\mathbf{w}} \Vert_{L^2_\bx}^2
\Vert \partial_t\zeta \Vert_{L^\infty_\by}^2 \dt
\\&
\lesssim
T^{1/2}_*
\sup_{I_*}\big(1+\Vert   \zeta
\Vert_{W^{1,\infty}_\by}^2
\big)
\sup_{I_*}
\Vert \nabx \overline{\mathbf{w}} \Vert_{L^2_\bx}^2
\bigg(\int_{I_*}
\Vert \partial_t\zeta \Vert_{L^\infty_\by}^4
\dt\bigg)^\frac{1}{2}
\\&
\lesssim
T^{1/2}_*
\Vert (\zeta,\overline{\mathbf{w}}, \overline{q})  \Vert_{X_{I_*}}^2.
\end{aligned}
\end{equation}
Also, by using the embedding \eqref{eq:emb3}
we obtain
\begin{equation}
\begin{aligned}
\label{K3dx}
\int_{I_*}\Vert   \mathbf{B}_{\zeta} \nabx \overline{\mathbf{w}}\, \overline{\mathbf{w}}_1  \Vert_{L^2_\bx}^2\dt
&\lesssim
\int_{I_*}\Vert   \zeta  \Vert_{W^{1,\infty}_\by}^2
\Vert
\nabx \overline{\mathbf{w}} \Vert_{L^4_\bx}^2
\Vert \overline{\mathbf{w}}  \Vert_{L^4_\bx}^2\dt
\\&\lesssim 
\sup_{I_*}\Vert   \zeta  \Vert_{W^{3,2}_\by}^2
\sup_{I_*}
\Vert \overline{\mathbf{w}}  \Vert_{W^{1,2}_\bx}^2
\int_{I_*}\Vert  \nabx\overline{\mathbf{w}} \Vert_{L^{4}_\bx}^2
\dt
\\&
\lesssim
T^{1/4}_*
\Vert (\zeta,\overline{\mathbf{w}}, \overline{q})  \Vert_{X_{I_*}}^2.
\end{aligned}
\end{equation}
Next, by using \eqref{eq:emb1}, we obtain
\begin{equation}
\begin{aligned}
\label{K3fx}
\int_{I_*}\Vert 
J_{\zeta} \bff\circ \Psi_{\zeta} 
 \Vert_{L^2_\bx}^2\dt
&
\lesssim 
\sup_{I_*}
\Vert
\zeta  
\Vert_{W^{1,\infty}_\by}^2\int_{I_*} \Vert \bff \Vert_{L^2_\bx}^2
\dt
\\&
\lesssim
T^{1/4}_*
\Vert (\zeta,\overline{\mathbf{w}}, \overline{q})  \Vert_{X_{I_*}}^2.
\end{aligned}
\end{equation}
It follows from \eqref{K3ax}--\eqref{K3fx} that
\begin{equation}
\begin{aligned}
\label{k3finalx}
\overline{K}_3
\lesssim  T^{1/2}_* 
\Vert (\zeta, \overline{\mathbf{w}}, \overline{q})\Vert_{X_{I_*}}^2.
\end{aligned}
\end{equation}

Our next goal is to estimate $\overline{K}_4$. Since
\begin{equation}
\begin{aligned}
\nonumber
\mathbf{H}_{\zeta}(\overline{\mathbf{w}}, \overline{q})
=
(\mathbf{A}_{\eta_0} -\mathbf{A}_{\zeta})\nabx\overline{\mathbf{w}}
+
(\mathbf{B}_{\eta_0}-\mathbf{B}_{\zeta}) \overline{q},
\end{aligned}
\end{equation}
due to the continuous embeddings  \eqref{eq:emb1}, \eqref{eq:emb3} and \eqref{eq:emb2},
%\\&W^{1,2}(I_*;L^2(\Omega)) \cap
%L^2(I_*;W^{2,2}(\Omega))
%\hookrightarrow
%L^4(I_*;W^{1,4}(\Omega)),
%\\&
%{\color{blue}
%W^{1,2}(I_*;L^2(\Omega))\cap L^2(I_*;W^{2,2}(\Omega))\hookrightarrow
%C^{0,1/4}(\overline{I}_*;W^{7/4,2}(\Omega))
%\hookrightarrow
%L^2(I_*;W^{1,4}(\Omega))
%},
%\\&
%{\color{blue}
%L^2(I_*;W^{3,2}(\omega))\cap W^{1,2}(I_*;W^{2,2}(\omega))\hookrightarrow C^{0,1/4}(\overline I_*;W^{9/4,2}(\omega))\hookrightarrow
%L^\infty(I_*;W^{1,\infty}(\omega)),
%}
it follows from \eqref{210and212}--\eqref{218} that
\begin{equation}
\begin{aligned}
\label{K4ax}
&\int_{I_*}\Vert (\mathbf{A}_{\eta_0} -\mathbf{A}_{\zeta})\nabx \overline{\mathbf{w}} \Vert_{W^{1,2}_\bx}^2\dt\\
&\lesssim
\int_{I_*}\Vert \nabx(\mathbf{A}_{\eta_0} -\mathbf{A}_{\zeta})  \Vert_{L^4_\bx}^2\Vert\nabx\overline{\mathbf{w}} \Vert_{L^4_\bx}^2\dt
+
\int_{I_*}\Vert  \mathbf{A}_{\eta_0} -\mathbf{A}_{\zeta}\Vert_{L^\infty_\bx}^2
\Vert\nabx^2\overline{\mathbf{w}} \Vert_{L^2_\bx}^2\dt
\\&\lesssim 
\sup_{I_*}\Vert  \eta_0 - \zeta \Vert_{W^{2,4}_\by}^2
\int_{I_*} \Vert\nabx \overline{\mathbf{w}} \Vert_{L^4_\bx}^2\dt
+
\sup_{I_*}\Vert  \eta_0 - \zeta \Vert_{W^{1,\infty}_\by}^2
\int_{I_*}\Vert\nabx^2\overline{\mathbf{w}}\Vert_{L^2_\bx}^2\dt
\\&\lesssim
 T^{1/4}_* 
\Vert (\zeta, \overline{\mathbf{w}}, \overline{q})\Vert_{X_{I_*}}^2.
\end{aligned}
\end{equation}
Next, we use the embedding
\begin{align*}
L^2(I_*;W^{3,2}(\omega))\cap W^{1,2}(I_*;W^{2,2}(\omega))\hookrightarrow W^{2/3,2}(I_*;W^{7/3,2}(\omega))
\hookrightarrow L^\infty(I_*;W^{2,3}(\omega)),
\end{align*}
where the latter scales with $T_*^{1/6}$,
and $W^{2,3}(\omega)\hookrightarrow W^{1,\infty}(\omega)$
to obtain
\begin{equation}
\begin{aligned}
\label{K4bx}
&\int_{I_*}\Vert (\mathbf{B}_{\eta_0}-\mathbf{B}_{\zeta}) \overline{q} \Vert_{W^{1,2}_\bx}^2\dt\\
&\lesssim
\int_{I_*}\Vert \nabx(\mathbf{B}_{\eta_0}-\mathbf{B}_{\zeta}) \Vert_{L^3_\bx}^2
\Vert
\overline{q} \Vert_{L^6_\bx}^2\dt+
\int_{I_*}\Vert \mathbf{B}_{\eta_0}-\mathbf{B}_{\zeta} \Vert_{L^\infty_\bx}^2
\Vert
\nabx\overline{q} \Vert_{L^2_\bx}^2\dt
\\
&\lesssim 
\sup_{I_*}\Vert  \eta_0 -  \zeta \Vert_{W^{2,3}_\by}^2
\int_{I_*}
\Vert
\overline{q} \Vert_{W^{1,2}_\bx}^2\dt
+
\sup_{I_*}\Vert  \eta_0 - \zeta \Vert_{W^{1,\infty}_\by}^2
\int_{I_*}
\Vert
 \overline{q} \Vert_{W^{1,2}_\bx}^2\dt
 \\&\lesssim
T^{1/3}_*
\Vert (\zeta, \overline{\mathbf{w}}, \overline{q})\Vert_{X_{I_*}}^2.
\end{aligned}
\end{equation}
By using \eqref{K4ax} and \eqref{K4bx}, it follows that
\begin{align}
\label{k4finalx}
\overline{K}_4
\lesssim
T^{1/3}_*
\Vert (\zeta,\overline{\mathbf{w}}, \overline{q})  \Vert_{X_{I_*}}^2.
\end{align}
Collecting the estimates \eqref{k1finalx}, \eqref{k2finalx}, \eqref{k3finalx} and \eqref{k4finalx}, we have shown that
\begin{align}
\label{contractionEstx}
\Vert \mathcal{T}(\zeta, \overline{\mathbf{w}}, \overline{q})
\Vert_{X_{I_*}}^2
=
\Vert(\eta, \overline{\bu}, \overline{\pi})\Vert_{X_{I_*}}^2
&\leq 
c  T^{1/2}_*  
\Vert (\zeta,\overline{\mathbf{w}}, \overline{q})  \Vert_{X_{I_*}}^2.
\end{align}
Since $(\zeta,\overline{\mathbf{w}}, \overline{q})\in B_R^{X_{I_*}}$, by choosing $T_*$ in $I_*=(0,T_*)$ so that $ T^{1/2}_* <\tfrac{c^{-1}}{2}$, we find that the right-hand side of \eqref{contractionEstx} is bounded by $R/2$. This, together with \eqref{cDataR}, implies \eqref{ballToBall}.
\\ \\
{\bf Step 2:} {\em We prove that $\mathcal{T}$ is a contraction map.} To show the contraction property, we denote $\eta_{12}:=\eta_1-\eta_2$, $\overline{\bu}_{12}:=\overline{\bu}_1-\overline{\bu}_2$ and $\overline{\pi}_{12}:=\overline{\pi}_1-\overline{\pi}_2$. 
We derive the system that $(\eta_{12}, \overline \bu_{12}, \overline \pi_{12})$ satisfies, which reads
\begin{align}
\label{contEqAloneBarFixed12}
\mathbf{B}_{\eta_0}:\nabx \overline{\bu}_{12}= h_{\zeta_1}(\overline{\mathbf{w}}_1)- h_{\zeta_2}(\overline{\mathbf{w}}_2),
\end{align}
and
\begin{equation}
\begin{aligned}
\varrho_s\partial_t^2\eta_{12} -\gamma\partial_t\Dely \eta_{12} + \alpha\Dely^2\eta_{12}
&=
\bn^\intercal \big[
-\mathbf{A}_{\eta_0}  \nabx\overline{\bu}_{12} +\mathbf{B}_{\eta_0}\overline{\pi}_{12}\big]\circ\bm{\varphi} \bn 
\\&
\quad+
\bn^\intercal \big[\mathbf{H}_{\zeta_1}(\overline{\mathbf{w}}_1, \overline{q}_1)-\mathbf{H}_{\zeta_2}(\overline{\mathbf{w}}_2, \overline{q}_2)
\big]\circ\bm{\varphi} \bn ,
\label{shellEqAloneBarFixed12}
\end{aligned}
\end{equation}
and
\begin{equation}
\begin{aligned}
J_{\eta_0}\partial_t \overline{\bu}_{12}  -\divx(\mathbf{A}_{\eta_0}  \nabx\overline{\bu}_{12}) 
 &+\divx(\mathbf{B}_{\eta_0}\overline{\pi}_{12}) 
 = 
\mathbf{h}_{\zeta_1}(\overline{\mathbf{w}}_1)-\mathbf{h}_{\zeta_2}(\overline{\mathbf{w}}_2)
\\&-
\divx  \mathbf{H}_{\zeta_1}(\overline{\mathbf{w}}_1, \overline{q}_1)
+
\divx  \mathbf{H}_{\zeta_2}(\overline{\mathbf{w}}_2, \overline{q}_2).
\label{momEqAloneBarFixed12}
\end{aligned}
\end{equation}
Since the left-hand side of \eqref{contEqAloneBarFixed12}, \eqref{shellEqAloneBarFixed12} and \eqref{momEqAloneBarFixed12} are all linear as functions of $(\eta_{12},\overline{\bu}_{12},\overline{\pi}_{12})$, it suffices to estimate their right-hand sides and substitute these estimates into the corresponding right-hand terms in \eqref{energyEstLinear}. Precisely, we estimate the following integrals:
\begin{align*}
&K_1:=
\int_{I_*}\Vert \partial_t [h_{\zeta_1}(\overline{\mathbf{w}}_1)- h_{\zeta_2}(\overline{\mathbf{w}}_2)] \Vert_{W^{-1,2}(\Omega)}^2\dt,
\\&
K_2:=
  \int_{I_*}\Vert h_{\zeta_1}(\overline{\mathbf{w}}_1)- h_{\zeta_2}(\overline{\mathbf{w}}_2)\Vert_{W^{1,2}(\Omega)}^2\dt,
 \\&
K_3:=
  \int_{I_*}\Vert \mathbf{h}_{\zeta_1}(\overline{\mathbf{w}}_1)- \mathbf{h}_{\zeta_2}(\overline{\mathbf{w}}_2)\Vert_{L^2(\Omega)}^2\dt,
 \\&
K_4:=
\int_{I_*}\Vert \mathbf{H}_{\zeta_1}(\overline{\mathbf{w}}_1, \overline{q}_1)- \mathbf{H}_{\zeta_2}(\overline{\mathbf{w}}_2, \overline{q}_2)\Vert_{W^{1,2}(\Omega)}^2\dt.
\end{align*}
The estimates for these $K_i$'s can be obtained in the same manner as their corresponding $\overline{K}_i$'s above when showing that the mapping $\mathcal{T}$ maps the ball into itself. However, we proceed to give a summary of the various estimates.

To estimate $K_1$, we need some preliminary estimates. Recalling the definition of $h_{\eta}(\overline\bu)$ at the beginning of Subsection \ref{Sectionlinear}, we write
\begin{align*}
\partial_t[h_{\zeta_1}(\overline{\mathbf{w}}_1)
-
h_{\zeta_2}(\overline{\mathbf{w}}_2)]
&=
 (\mathbf{B}_{\eta_0}-\mathbf{B}_{\zeta_1}):\partial_t\nabx( \overline{\mathbf{w}}_1
-
\overline{\mathbf{w}}_2)
+
(\mathbf{B}_{\zeta_2}-\mathbf{B}_{\zeta_1}):\partial_t \nabx\overline{\mathbf{w}}_2
\\
&\quad +\partial_t(\mathbf{B}_{\eta_0}-\mathbf{B}_{\zeta_1}):\nabx( \overline{\mathbf{w}}_1
-
\overline{\mathbf{w}}_2)
+
\partial_t(\mathbf{B}_{\zeta_2}-\mathbf{B}_{\zeta_1}):\nabx \overline{\mathbf{w}}_2.
\end{align*}
Now, just as in \eqref{k1ax}, we obtain from
 \eqref{210and212}--\eqref{218} and the continuous embedding \eqref{eq:emb1} that
%\begin{align}\label{eq:emb1}
%L^\infty(I_*;W^{3,2}(\omega))\cap W^{1,2}(I_*;W^{2,2}(\omega)\hookrightarrow
%L^\infty(I_*;W^{1,\infty}(\omega)),
%\end{align}
%which scales with $T_*^{1/8}$,
%that
\begin{equation*}
\begin{aligned}
\label{k1a}
&\int_{I_*}\big(\Vert (\mathbf{B}_{\eta_0}-\mathbf{B}_{\zeta_1}):\partial_t\nabx( \overline{\mathbf{w}}_1
-
\overline{\mathbf{w}}_2) \Vert_{W^{-1,2}_{\bx}}
^2
+
\Vert (\mathbf{B}_{\zeta_1}-\mathbf{B}_{\zeta_2})\partial_t\nabx
\overline{\mathbf{w}}_2 \Vert_{W^{-1,2}_{\bx}}^2
\big)
\dt
%\\&\lesssim
%\int_{I_*}\Vert  \mathbf{B}_{\eta_0}-\mathbf{B}_{\zeta_1}\Vert_{L^\infty_\bx}^2\Vert\partial_t( \overline{\mathbf{w}}_1
%-
%\overline{\mathbf{w}}_2) \Vert_{L^2_\bx}^2
%\dt
%\\&\lesssim 
%\sup_{I_*}\Vert  \eta_0 - \zeta_1\Vert_{W^{1,\infty}_\by}^2\int_{I_*}\Vert  \partial_t( \overline{\mathbf{w}}_1
%-
%\overline{\mathbf{w}}_2) \Vert_{L^2_\bx}^2
%\dt
\\&
\lesssim
T^{1/4}_*
\Vert (\zeta_1, \overline{\mathbf{w}}_1, \overline{q}_1)-(\zeta_2, \overline{\mathbf{w}}_2, \overline{q}_2)\Vert_{X_{I_*}}^2.
\end{aligned}
\end{equation*}
%Similarly,
%\begin{equation}
%\begin{aligned}
%\label{k1b}
%\int_{I_*}\Vert (\mathbf{B}_{\zeta_1}-\mathbf{B}_{\zeta_2})&\partial_t\nabx
%\overline{\mathbf{w}}_2 \Vert_{W^{-1,2}_{\bx}}^2
%\dt
%\lesssim
%T^{1/4}_*
%\Vert (\zeta_1, \overline{\mathbf{w}}_1, \overline{q}_1)-(\zeta_2, \overline{\mathbf{w}}_2, \overline{q}_2)\Vert_{X_{I_*}}^2.
%\end{aligned}
%\end{equation}
Also, as in \eqref{k1cx}, we obtain
%\begin{align*}
%W^{2,2}(I_*;L^2(\omega))\cap W^{1,2}(I_*;W^{2,2}(\omega))\hookrightarrow W^{5/4,2}(I_\ast;W^{3/2,2}(\omega)) \hookrightarrow
%W^{1,4}(I_*;W^{1,4}(\omega)),
%\end{align*}
%it follows that
\begin{equation*}
\begin{aligned}
\label{k1c}
&\int_{I_*}\big(\Vert\partial_t (\mathbf{B}_{\eta_0}-\mathbf{B}_{\zeta_1}):\nabx( \overline{\mathbf{w}}_1
-
\overline{\mathbf{w}}_2) \Vert_{W^{-1,2}_\bx}^2
+
\Vert\partial_t (\mathbf{B}_{\zeta_1}-\mathbf{B}_{\zeta_2}):\nabx
\overline{\mathbf{w}}_2 \Vert_{W^{-1,2}_\bx}^2
\big)
\dt
%\\
%&\lesssim
%\int_{I_*}\Vert\partial_t \mathbf{B}_{\zeta_1}:\nabx( \overline{\mathbf{w}}_1
%-
%\overline{\mathbf{w}}_2) \Vert_{L^{4/3}_\bx}^2
%\dt
%\\
%&\lesssim
%\int_{I_*}\Vert  \partial_t\mathbf{B}_{\zeta_1}\Vert^2_{L^4_\bx}\Vert\nabx( \overline{\mathbf{w}}_1
%-
%\overline{\mathbf{w}}_2) \Vert_{L^2_\bx}^2
%\dt
%\\&\lesssim T^{1/2}_*
%\bigg(\int_{I_*} \Vert  \partial_t\zeta_1\Vert_{W^{1,4}_\by}^4\dt\bigg)^\frac{1}{2}
%\sup_{I_*}\Vert  \nabx( \overline{\mathbf{w}}_1
%-
%\overline{\mathbf{w}}_2) \Vert_{L^2_\bx}^2
\\&
\lesssim
T^{1/2}_*
\Vert (\zeta_1, \overline{\mathbf{w}}_1, \overline{q}_1)-(\zeta_2, \overline{\mathbf{w}}_2, \overline{q}_2)\Vert_{X_{I_*}}^2,
\end{aligned}
\end{equation*} 
%Similarly
%\begin{equation}
%\begin{aligned}
%\label{k1d}
%&\int_{I_*}\Vert\partial_t (\mathbf{B}_{\zeta_1}-\mathbf{B}_{\zeta_2}):\nabx
%\overline{\mathbf{w}}_2 \Vert_{W^{-1,2}_\bx}^2
%\dt
%%\\&\lesssim T^{1/2}_*
%%\bigg(\int_{I_*} \Vert  \partial_t(\zeta_1-\zeta_2)\Vert_{W^{1,4}_\by}^4\dt\bigg)^\frac{1}{2}
%%\sup_{I_*}\Vert  \nabx
%%\overline{\mathbf{w}}_2\Vert_{L^2_\bx}^2
%%\\&
%\lesssim
%T^{1/2}_*
%\Vert (\zeta_1, \overline{\mathbf{w}}_1, \overline{q}_1)-(\zeta_2, \overline{\mathbf{w}}_2, \overline{q}_2)\Vert_{X_{I_*}}^2.
%\end{aligned}
%\end{equation}
%It follows from \eqref{k1a}-\eqref{k1d} that
and thus,
\begin{equation}
\begin{aligned}
\label{k1final}
K_1
%\int_I\Vert \partial_t[h_{\zeta_1}(\overline{\mathbf{w}}_1)
%-
%h_{\zeta_2}(\overline{\mathbf{w}}_2)] \Vert_{W^{-1,2}(\Omega)}
%\dt
\lesssim T^{1/2}_*
\Vert (\zeta_1, \overline{\mathbf{w}}_1, \overline{q}_1)-(\zeta_2, \overline{\mathbf{w}}_2, \overline{q}_2)\Vert_{X_{I_*}}^2.
\end{aligned}
\end{equation}
To estimate $K_2$, we first note that
\begin{equation*}
	\begin{aligned}
&\vert
 h_{\zeta_1}(\overline{\mathbf{w}}_1)
-
h_{\zeta_2}(\overline{\mathbf{w}}_2)
\vert
+
\vert
\nabx[h_{\zeta_1}(\overline{\mathbf{w}}_1)
-
h_{\zeta_2}(\overline{\mathbf{w}}_2)]
\vert
\\
&\lesssim
\vert (\mathbf{B}_{\eta_0}-\mathbf{B}_{\zeta_1}) :\nabx( \overline{\mathbf{w}}_1
-
\overline{\mathbf{w}}_2)
\vert
+
\vert
(\mathbf{B}_{\zeta_1}-\mathbf{B}_{\zeta_2}) :\nabx\overline{\mathbf{w}}_2
\vert
\\&
\quad+
\vert (\mathbf{B}_{\eta_0}-\mathbf{B}_{\zeta_1}) :\nabx^2( \overline{\mathbf{w}}_1
-
\overline{\mathbf{w}}_2)
\vert
+
\vert
(\mathbf{B}_{\zeta_1}-\mathbf{B}_{\zeta_2}) :\nabx^2\overline{\mathbf{w}}_2
\vert
\\&
\quad+
\vert \nabx(\mathbf{B}_{\eta_0}-\mathbf{B}_{\zeta_1}):\nabx( \overline{\mathbf{w}}_1
-
\overline{\mathbf{w}}_2)
\vert
+
\vert
\nabx(\mathbf{B}_{\zeta_1}-\mathbf{B}_{\zeta_2}):\nabx \overline{\mathbf{w}}_2
\vert.
\end{aligned}
\end{equation*}
Similar to \eqref{K2ax}, we use \eqref{eq:emb1} and \eqref{210and212}--\eqref{218} and obtain
\begin{equation}
\begin{aligned}
\label{K2a}
&\int_{I_*}\big(\Vert   (\mathbf{B}_{\eta_0}-\mathbf{B}_{\zeta_1}) :\nabx( \overline{\mathbf{w}}_1
-
\overline{\mathbf{w}}_2)\Vert_{L^2_\bx}^2
+
\Vert  (\mathbf{B}_{\zeta_1}-\mathbf{B}_{\zeta_2}) :\nabx\overline{\mathbf{w}}_2\Vert_{L^2_\bx}^2
\big)\dt
\\
&\quad+
\int_{I_*}\big(\Vert   (\mathbf{B}_{\eta_0}-\mathbf{B}_{\zeta_1}) :\nabx^2( \overline{\mathbf{w}}_1
-
\overline{\mathbf{w}}_2)\Vert_{L^2_\bx}^2
+
\Vert  (\mathbf{B}_{\zeta_1}-\mathbf{B}_{\zeta_2}) :\nabx^2\overline{\mathbf{w}}_2\Vert_{L^2_\bx}^2
\big)\dt
%\\&\lesssim
%\sup_{I_*}\Vert    \eta_0 - \zeta_1 \Vert_{W^{1,\infty}_\by}^2
%\int_{I_*} \Vert\nabx^2( \overline{\mathbf{w}}_1
%-
%\overline{\mathbf{w}}_2)\Vert_{L^2_\bx}^2\dt+
%\sup_{I_*}\Vert   \zeta_1 - \zeta_2 \Vert_{W^{1,\infty}_\by}^2
%\int_{I_*} \Vert\nabx^2\overline{\mathbf{w}}_2\Vert_{L^2_\bx}^2\dt
\\&\lesssim
T^{1/4}_*
\Vert (\zeta_1, \overline{\mathbf{w}}_1, \overline{q}_1)-(\zeta_2, \overline{\mathbf{w}}_2, \overline{q}_2)\Vert_{X_{I_*}}^2.
\end{aligned}
\end{equation}
%Recalling \eqref{210and212}-\eqref{218} and the continuous embeddings:
%\begin{equation}\label{eq:emb2}
%L^\infty(I_*;W^{3,2}(\omega) \hookrightarrow
%L^\infty(I_*;W^{2,4}(\omega)), 
%\end{equation}
%as well as
%\begin{align}
%\begin{aligned}
%\,W^{1,2}(I_*;L^2(\Omega))\cap L^2(I_*;W^{2,2}(\Omega))
%\hookrightarrow W^{1/8,2}(I_*;W^{7/4,2}(\Omega))\hookrightarrow
%L^2(I_*;W^{1,4}(\Omega)),
%\end{aligned}
%\label{eq:emb3}
%\end{align}
%where the latter scales with $T^{1/8}$,
%it follows that
As in \eqref{K2bx}, we obtain
\begin{equation}
\begin{aligned}
\label{K2b}
 &\int_{I_*}\big(\Vert  \nabx (\mathbf{B}_{\eta_0}-\mathbf{B}_{\zeta_1})\nabx( \overline{\mathbf{w}}_1
-
\overline{\mathbf{w}}_2)\Vert_{L^2_\bx}^2
+
\Vert  \nabx(\mathbf{B}_{\zeta_1}-\mathbf{B}_{\zeta_2}) \nabx\overline{\mathbf{w}}_2\Vert_{L^2_\bx}^2
\big)\dt
%\\
%&\lesssim
%\sup_{I_*}\Vert   \eta_0 - \zeta_1\Vert_{W^{2,4}_\by}^2
%\int_{I_*} \Vert\nabx( \overline{\mathbf{w}}_1
%-
%\overline{\mathbf{w}}_2)\Vert_{L^4_\bx}^2\dt
%+
%\sup_{I_*}\Vert   \zeta_1 - \zeta_2 \Vert_{W^{2,4}_\by}^2
%\int_{I_*} \Vert\nabx\overline{\mathbf{w}}_2\Vert_{L^4_\bx}^2\dt
\\&\lesssim
T^{1/4}_*
\Vert (\zeta_1, \overline{\mathbf{w}}_1, \overline{q}_1)-(\zeta_2, \overline{\mathbf{w}}_2, \overline{q}_2)\Vert_{X_{I_*}}^2.
\end{aligned}
\end{equation}
We obtain from \eqref{K2a} and \eqref{K2b} that
\begin{align}
\label{k2final}
K_2
%:=
%\int_I\Vert\nabx[ h_{\zeta_1}(\overline{\mathbf{w}}_1)- h_{\zeta_2}(\overline{\mathbf{w}}_2)]\Vert_{L^2(\Omega)}^2\dt
\lesssim
 T^{1/4}_* 
\Vert (\zeta_1,\overline{\mathbf{w}}_1, \overline{q}_1) - (\zeta_2,\overline{\mathbf{w}}_2, \overline{q}_2) \Vert_{X_{I_*}}^2.
\end{align} 
To estimate $K_3$, we need some preliminary estimates. First of all, note that
\begin{align*}
\vert
\mathbf{h}_{\zeta_1}(\overline{\mathbf{w}}_1)
-
\mathbf{h}_{\zeta_2}(\overline{\mathbf{w}}_2)
\vert
&\lesssim
\vert (J_{\eta_0}-J_{\zeta_1})\partial_t( \overline{\mathbf{w}}_1
-
\overline{\mathbf{w}}_2)
\vert
+
\vert
(J_{\zeta_1}-J_{\zeta_2})\partial_t \overline{\mathbf{w}}_2
\vert
\\&
\quad+
\big\vert
J_{\zeta_1} \nabx (\overline{\mathbf{w}}_1 -\overline{\mathbf{w}}_2 ) \partial_t \Psi_{\zeta_1}^{-1}\circ \Psi_{\zeta_1} 
\big\vert
+
\big\vert
\big(J_{\zeta_1}
-
J_{\zeta_2} \big) \nabx \overline{\mathbf{w}}_2  \partial_t \Psi_{\zeta_1}^{-1}\circ \Psi_{\zeta_1} 
\big\vert
\\&
\quad+
\big\vert
J_{\zeta_2} \nabx \overline{\mathbf{w}}_2  \partial_t \big(\Psi_{\zeta_1}^{-1}\circ \Psi_{\zeta_1} 
-
\Psi_{\zeta_2}^{-1}\circ \Psi_{\zeta_2} \big)
\big\vert
\\&
\quad+
\big\vert
J_{\zeta_1}\big( \nabx \Psi_{\zeta_1}^{-1}\circ \Psi_{\zeta_1}\big)^\intercal \nabx (\overline{\mathbf{w}}_1 -\overline{\mathbf{w}}_2 ) \overline{\mathbf{w}}_1 
\big\vert
\\&
\quad+
\big\vert
\big[J_{\zeta_1}\big( \nabx \Psi_{\zeta_1}^{-1}\circ \Psi_{\zeta_1}\big)^\intercal
-
J_{\zeta_2}\big( \nabx \Psi_{\zeta_2}^{-1}\circ \Psi_{\zeta_2}\big)^\intercal \big] \nabx \overline{\mathbf{w}}_2  \overline{\mathbf{w}}_1 
\big\vert
\\&
\quad+
\big\vert
J_{\zeta_2}\big( \nabx \Psi_{\zeta_2}^{-1}\circ \Psi_{\zeta_2}\big)^\intercal \nabx \overline{\mathbf{w}}_2  (\overline{\mathbf{w}}_1-\overline{\mathbf{w}}_2)
\big\vert
\\&
\quad+
\vert
J_{\zeta_1} (\bff\circ \Psi_{\zeta_1} -
\bff\circ \Psi_{\zeta_2}  )
\vert
+
\vert
(J_{\zeta_1}-J_{\zeta_2}) 
\bff\circ \Psi_{\zeta_2}  
\vert.
\end{align*}
%Due to the continuous embeddings \eqref{eq:emb1}, 
%it follows from the definition $J_\eta=\det(\nabx \Psi_\eta)$ and \eqref{210and212}-\eqref{218}  that
As in \eqref{K3ax}, using \eqref{eq:emb1} and
\eqref{210and212}--\eqref{218} we have
\begin{equation}
\begin{aligned}
\label{K3a}
&\int_{I_*}\big(\Vert   (J_{\eta_0}-J_{\zeta_1})\partial_t( \overline{\mathbf{w}}_1
-
\overline{\mathbf{w}}_2)\Vert_{L^2_\bx}^2
+
\Vert  (J_{\zeta_1}-J_{\zeta_2})\partial_t
\overline{\mathbf{w}}_2\Vert_{L^2_\bx}^2
\big)\dt
%\\&\lesssim
%\sup_{I_*}\Vert  \eta_0 - \zeta_1\Vert_{W^{1,\infty}_\by}
%\int_{I_*}\Vert \partial_t(\overline{\mathbf{w}}_1 - \overline{\mathbf{w}}_2) \Vert_{L^2_\bx}^2\dt
%+
%\sup_{I_*}\Vert  \zeta_1 - \zeta_2\Vert_{W^{1,\infty}_\by}
%\int_{I_*}\Vert \partial_t \overline{\mathbf{w}}_2 \Vert_{L^2_\bx}^2\dt
\\&\lesssim
T^{1/4}_*
\Vert (\zeta_1,\overline{\mathbf{w}}_1, \overline{q}_1) - (\zeta_2,\overline{\mathbf{w}}_2, \overline{q}_2) \Vert_{X_{I_*}}^2.
\end{aligned}
\end{equation}
Similar to \eqref{K3bx}, we obtain
\begin{equation}
\begin{aligned}
\label{K3b}
&\int_{I_*}\Vert   J_{\zeta_1}  \nabx (\overline{\mathbf{w}}_1 -\overline{\mathbf{w}}_2 ) \partial_t \Psi_{\zeta_1}^{-1}\circ \Psi_{\zeta_1} \Vert_{L^2_\bx}^2\dt
+
\int_{I_*}\big\Vert  (J_{\zeta_1} 
-
J_{\zeta_2}) \nabx \overline{\mathbf{w}}_2  \partial_t \Psi_{\zeta_1}^{-1}\circ \Psi_{\zeta_1} 
\big\Vert_{L^2_\bx}^2\dt
\\&\quad+
\int_{I_*}\big\Vert  J_{\zeta_2} \nabx \overline{\mathbf{w}}_2  \partial_t \big(\Psi_{\zeta_1}^{-1}\circ \Psi_{\zeta_1} 
-
\Psi_{\zeta_2}^{-1}\circ \Psi_{\zeta_2} \big)
\big\Vert_{L^2_\bx}^2\dt
%\\&\lesssim
%\int_{I_*}\big(1+\Vert   \zeta_1
%\Vert_{W^{1,\infty}_\by}^2
%\big)
%\Vert \nabx (\overline{\mathbf{w}}_1 -\overline{\mathbf{w}}_2 )\Vert_{L^2_\bx}^2
%\Vert \partial_t\zeta_1 \Vert_{L^\infty_\by}^2 \dt
%\\&
%\lesssim
%T^{1/2}_*
%\sup_{I_*}\big(1+\Vert   \zeta_1
%\Vert_{W^{1,\infty}_\by}^2
%\big)
%\sup_{I_*}
%\Vert \nabx (\overline{\mathbf{w}}_1 -\overline{\mathbf{w}}_2 )\Vert_{L^2_\bx}^2
%\bigg(\int_{I_*}
%\Vert \partial_t\zeta_1 \Vert_{L^\infty_\by}^4
%\dt\bigg)^\frac{1}{2}
\\&
\lesssim
T^{1/2}_*
\Vert (\zeta_1,\overline{\mathbf{w}}_1, \overline{q}_1) - (\zeta_2,\overline{\mathbf{w}}_2, \overline{q}_2) \Vert_{X_{I_*}}^2.
\end{aligned}
\end{equation}
%Similarly,
%\begin{equation}
%\begin{aligned}
%\label{K3c}
%&\int_{I_*}\big\Vert  (J_{\zeta_1} 
%-
%J_{\zeta_2}) \nabx \overline{\mathbf{w}}_2  \partial_t \Psi_{\zeta_1}^{-1}\circ \Psi_{\zeta_1} 
%\big\Vert_{L^2_\bx}^2\dt
%\\&\quad+
%\int_{I_*}\big\Vert  J_{\zeta_2} \nabx \overline{\mathbf{w}}_2  \partial_t \big(\Psi_{\zeta_1}^{-1}\circ \Psi_{\zeta_1} 
%-
%\Psi_{\zeta_2}^{-1}\circ \Psi_{\zeta_2} \big)
%\big\Vert_{L^2_\bx}^2\dt
%\\&
%\lesssim
%T^{1/2}_*
%\Vert (\zeta_1,\overline{\mathbf{w}}_1, \overline{q}_1) - (\zeta_2,\overline{\mathbf{w}}_2, \overline{q}_2) \Vert_{X_{I_*}}^2.
%\end{aligned}
%\end{equation}
Next, as in \eqref{K3dx},
%Also, by using the embedding \eqref{eq:emb3}
we obtain
\begin{equation}
\begin{aligned}
\label{K3d}
&\int_{I_*}\Vert   J_{\zeta_1}\big( \nabx \Psi_{\zeta_1}^{-1}\circ \Psi_{\zeta_1}\big)^\intercal \nabx (\overline{\mathbf{w}}_1 -\overline{\mathbf{w}}_2 ) \overline{\mathbf{w}}_1  \Vert_{L^2_\bx}^2\dt
\\&\quad+
\int_{I_*}\big\Vert  
\big[J_{\zeta_1}\big( \nabx \Psi_{\zeta_1}^{-1}\circ \Psi_{\zeta_1}\big)^\intercal
-
J_{\zeta_2}\big( \nabx \Psi_{\zeta_2}^{-1}\circ \Psi_{\zeta_2}\big)^\intercal \big] \nabx \overline{\mathbf{w}}_2  \overline{\mathbf{w}}_1   \big\Vert_{L^2_\bx}^2\dt
\\&\quad
+\int_{I_*}\Vert
J_{\zeta_2}\big( \nabx \Psi_{\zeta_2}^{-1}\circ \Psi_{\zeta_2}\big)^\intercal \nabx \overline{\mathbf{w}}_2  (\overline{\mathbf{w}}_1-\overline{\mathbf{w}}_2)
\Vert_{L^2_\bx}^2\dt
%\\&\lesssim
%\int_{I_*}\Vert   \zeta_1  \Vert_{W^{1,\infty}_\by}^2
%\Vert
%\nabx (\overline{\mathbf{w}}_1 -\overline{\mathbf{w}}_2 )\Vert_{L^4_\bx}^2
%\Vert \overline{\mathbf{w}}_1  \Vert_{L^4_\bx}^2\dt
%\\&\lesssim 
%\sup_{I_*}\Vert   \zeta_1  \Vert_{W^{3,2}_\by}^2
%\sup_{I_*}
%\Vert \overline{\mathbf{w}}_1  \Vert_{W^{1,2}_\bx}^2
%\int_{I_*}\Vert  \nabx(\overline{\mathbf{w}}_1
%-
%\overline{\mathbf{w}}_2) \Vert_{L^{4}_\bx}^2
%\dt
\\&
\lesssim
T^{1/4}_*
\Vert (\zeta_1,\overline{\mathbf{w}}_1, \overline{q}_1) - (\zeta_2,\overline{\mathbf{w}}_2, \overline{q}_2) \Vert_{X_{I_*}}^2.
\end{aligned}
\end{equation}
%Similarly,
%\begin{equation}
%\begin{aligned}
%\label{K3e}
%&\int_{I_*}\big\Vert  
%\big[J_{\zeta_1}\big( \nabx \Psi_{\zeta_1}^{-1}\circ \Psi_{\zeta_1}\big)^\intercal
%-
%J_{\zeta_2}\big( \nabx \Psi_{\zeta_2}^{-1}\circ \Psi_{\zeta_2}\big)^\intercal \big] \nabx \overline{\mathbf{w}}_2  \overline{\mathbf{w}}_1   \big\Vert_{L^2_\bx}^2\dt
%\\&\quad
%+\int_{I_*}\Vert
%J_{\zeta_2}\big( \nabx \Psi_{\zeta_2}^{-1}\circ \Psi_{\zeta_2}\big)^\intercal \nabx \overline{\mathbf{w}}_2  (\overline{\mathbf{w}}_1-\overline{\mathbf{w}}_2)
%\Vert_{L^2_\bx}^2\dt
%\\&
%\lesssim
%T^{1/4}_*
%\Vert (\zeta_1,\overline{\mathbf{w}}_1, \overline{q}_1) - (\zeta_2,\overline{\mathbf{w}}_2, \overline{q}_2) \Vert_{X_{I_*}}^2.
%\end{aligned}
%\end{equation}
By following the same argument as in \eqref{K3fx}, we obtain
\begin{equation}
\begin{aligned}
\label{K3f}
&\int_{I_*}\big(\Vert 
J_{\zeta_1} (\bff\circ \Psi_{\zeta_1} -
\bff\circ \Psi_{\zeta_2}  )
 \Vert_{L^2_\bx}^2
 +
 \Vert 
(J_{\zeta_1}-J_{\zeta_2}) 
\bff\circ \Psi_{\zeta_2} 
 \Vert_{L^2_\bx}^2
 \big)\dt
%&
%\lesssim 
%\sup_{I_*}
%\Vert
%J_{\zeta_1}
%\Vert_{L^\infty_\bx}^2
%\sup_{I_*}
%\Vert
%\zeta_1 -\zeta_2 
%\Vert_{L^\infty_\by}^2\int_{I_*} \Vert \bff \Vert_{L^2_\bx}^2
%\dt
\\&
\lesssim
T^{1/4}_*
\Vert (\zeta_1,\overline{\mathbf{w}}_1, \overline{q}_1) - (\zeta_2,\overline{\mathbf{w}}_2, \overline{q}_2) \Vert_{X_{I_*}}^2.
\end{aligned}
\end{equation}
%Similarly, we have
%\begin{equation}
%\begin{aligned}
%\label{K3g}
%\int_{I_*}\Vert 
%(J_{\zeta_1}-J_{\zeta_2}) 
%\bff\circ \Psi_{\zeta_2} 
% \Vert_{L^2_\bx}^2\dt
%%&
%%\lesssim
%%T_*
%%\sup_{I_*} \Vert \bff \Vert_{L^2_\bx}^2
%%\sup_{I_*}
%%\Vert
%%\Psi_{\zeta_2}^{-1} 
%%\Vert_{L^\infty_\bx}^2
%%\sup_{I_*}
%%\Vert
%%\zeta_1 -\zeta_2 
%%\Vert_{W^{1,\infty}_\by}^2
%%\\&
%\lesssim
%T^{1/4}_*
%\Vert (\zeta_1,\overline{\mathbf{w}}_1, \overline{q}_1) - (\zeta_2,\overline{\mathbf{w}}_2, \overline{q}_2) \Vert_{X_{I_*}}^2.
%\end{aligned}
%\end{equation}
It follows from \eqref{K3a}--\eqref{K3f} that
\begin{equation}
\begin{aligned}
\label{k3final}
K_3
%=
%\int_I\Vert \mathbf{h}_{\zeta_1}(\overline{\mathbf{w}}_1)
%-
%\mathbf{h}_{\zeta_2}(\overline{\mathbf{w}}_2) \Vert_{L^2(\Omega)}^2
%\dt
\lesssim  T^{1/2}_* 
\Vert (\zeta_1, \overline{\mathbf{w}}_1, \overline{q}_1)-(\zeta_2, \overline{\mathbf{w}}_2, \overline{q}_2)\Vert_{X_{I_*}}^2.
\end{aligned}
\end{equation}
Our next goal is to estimate $K_4$. First of all, note that
\begin{equation}
\begin{aligned}
\nonumber
\vert \mathbf{H}_{\zeta_1}(\overline{\mathbf{w}}_1, \overline{q}_1)- \mathbf{H}_{\zeta_2}(\overline{\mathbf{w}}_2, \overline{q}_2)\vert
&\lesssim
\vert (\mathbf{A}_{\eta_0} -\mathbf{A}_{\zeta_1})\nabx(\overline{\mathbf{w}}_1 - \overline{\mathbf{w}}_2)\vert
+
\vert (\mathbf{A}_{\zeta_1} -\mathbf{A}_{\zeta_2})\nabx \overline{\mathbf{w}}_2\vert
\\&
\quad+
\vert (\mathbf{B}_{\eta_0}-\mathbf{B}_{\zeta_1}) (\overline{q}_1 - \overline{q}_2)\vert
+
\vert (\mathbf{B}_{\zeta_1}-\mathbf{B}_{\zeta_2}) \overline{q}_2\vert,
\end{aligned}
\end{equation}
holds uniformly. By the same argument as in \eqref{K4ax}, by using \eqref{eq:emb1}, \eqref{eq:emb3} and \eqref{eq:emb2},
%\begin{align*}
%&L^\infty(I_*;W^{3,2}(\omega))
%\hookrightarrow
%L^\infty(I_*;W^{2,4}(\omega)),
%\\&W^{1,2}(I_*;L^2(\Omega)) \cap
%L^2(I_*;W^{2,2}(\Omega))
%\hookrightarrow
%L^4(I_*;W^{1,4}(\Omega)),
%\\&
%{\color{blue}
%W^{1,2}(I_*;L^2(\Omega))\cap L^2(I_*;W^{2,2}(\Omega))\hookrightarrow
%C^{0,1/4}(\overline{I}_*;W^{7/4,2}(\Omega))
%\hookrightarrow
%L^2(I_*;W^{1,4}(\Omega))
%},
%\\&
%{\color{blue}
%L^2(I_*;W^{3,2}(\omega))\cap W^{1,2}(I_*;W^{2,2}(\omega))\hookrightarrow C^{0,1/4}(\overline I_*;W^{9/4,2}(\omega))\hookrightarrow
%L^\infty(I_*;W^{1,\infty}(\omega)),
%}
%\end{align*}
we obtain from \eqref{210and212}--\eqref{218} that
\begin{equation}
\begin{aligned}
\label{K4a}
&\int_{I_*}\big(\Vert (\mathbf{A}_{\eta_0} -\mathbf{A}_{\zeta_1})\nabx(\overline{\mathbf{w}}_1 - \overline{\mathbf{w}}_2)\Vert_{W^{1,2}_\bx}^2\dt
+
\Vert (\mathbf{A}_{\zeta_1} -\mathbf{A}_{\zeta_2})\nabx  \overline{\mathbf{w}}_2\Vert_{W^{1,2}_\bx}^2
\big)\dt
%\\
%&\lesssim
%\int_{I_*}\Vert \nabx(\mathbf{A}_{\eta_0} -\mathbf{A}_{\zeta_1})  \Vert_{L^4_\bx}^2\Vert\nabx(\overline{\mathbf{w}}_1 - \overline{\mathbf{w}}_2)\Vert_{L^4_\bx}^2\dt
%\\&\quad+
%\int_{I_*}\Vert  \mathbf{A}_{\eta_0} -\mathbf{A}_{\zeta_1}\Vert_{L^\infty_\bx}^2
%\Vert\nabx^2(\overline{\mathbf{w}}_1 - \overline{\mathbf{w}}_2)\Vert_{L^2_\bx}^2\dt
%\\&\lesssim 
%\sup_{I_*}\Vert  \eta_0 - \zeta_1  \Vert_{W^{2,4}_\by}^2
%%\bigg(\int_{I_*}\Vert\nabx(\overline{\mathbf{w}}_1 - \overline{\mathbf{w}}_2)\Vert_{L^4_\bx}^4\dt\bigg)^\frac{1}{2}
%\int_{I_*} \Vert\nabx(\overline{\mathbf{w}}_1 - \overline{\mathbf{w}}_2)\Vert_{L^4_\bx}^2\dt
%\\&\quad+
%\sup_{I_*}\Vert  \eta_0 - \zeta_1 \Vert_{W^{1,\infty}_\by}^2
%\int_{I_*}\Vert\nabx^2(\overline{\mathbf{w}}_1 - \overline{\mathbf{w}}_2)\Vert_{L^2_\bx}^2\dt
\\&\lesssim
 T^{1/4}_* 
\Vert (\zeta_1, \overline{\mathbf{w}}_1, \overline{q}_1)-(\zeta_2, \overline{\mathbf{w}}_2, \overline{q}_2)\Vert_{X_{I_*}}^2.
\end{aligned}
\end{equation}
%Similarly,
%\begin{equation}
%\begin{aligned}
%\label{K4a1}
%\int_{I_*}\Vert (\mathbf{A}_{\zeta_1} -\mathbf{A}_{\zeta_2})\nabx  \overline{\mathbf{w}}_2\Vert_{W^{1,2}_\bx}^2\dt
%&\lesssim
% T^{1/4}_* 
%\Vert (\zeta_1, \overline{\mathbf{w}}_1, \overline{q}_1)-(\zeta_2, \overline{\mathbf{w}}_2, \overline{q}_2)\Vert_{X_{I_*}}^2.
%\end{aligned}
%\end{equation}
%Next, we use the embedding
%\begin{align*}
%L^2(I_*;W^{3,2}(\omega))\cap W^{1,2}(I_*;W^{2,2}(\omega))\hookrightarrow W^{2/3,2}(I_*;W^{7/3,2}(\omega))
%\hookrightarrow L^\infty(I_*;W^{2,3}(\omega)),
%\end{align*}
%where the latter embedding scales with $T_*^{1/6}$.
%Using \eqref{eq:emb1} with scale $T_*^{1/8}$ again, we have
%%\todo{need detailed interplation for the estimate for the first integral and explain $T_*^{1/2}$}
Finally, we adopt the approach leading to \eqref{K4bx} and obtain
\begin{equation}
\begin{aligned}
\label{K4b}
&\int_{I_*}\big(\Vert (\mathbf{B}_{\eta_0}-\mathbf{B}_{\zeta_1}) (\overline{q}_1 - \overline{q}_2)\Vert_{W^{1,2}_\bx}^2
+
\int_{I_*}\Vert (\mathbf{B}_{\zeta_1}-\mathbf{B}_{\zeta_2})  \overline{q}_2\Vert_{W^{1,2}_\bx}^2
\big)\dt
%\\
%&\lesssim
%\int_{I_*}\Vert \nabx(\mathbf{B}_{\eta_0}-\mathbf{B}_{\zeta_1}) \Vert_{L^3_\bx}^2
%\Vert
%\overline{q}_1 - \overline{q}_2 \Vert_{L^6_\bx}^2\dt+
%\int_{I_*}\Vert \mathbf{B}_{\eta_0}-\mathbf{B}_{\zeta_1} \Vert_{L^\infty_\bx}^2
%\Vert
%\nabx(\overline{q}_1 - \overline{q}_2)\Vert_{L^2_\bx}^2\dt
%\\
%&\lesssim 
%\sup_{I_*}\Vert  \eta_0 -  \zeta_1  \Vert_{W^{2,3}_\by}^2
%\int_{I_*}
%\Vert
%\overline{q}_1 - \overline{q}_2 \Vert_{L^4_\bx}^2\dt+
%\sup_{I_*}\Vert  \eta_0 - \zeta_1 \Vert_{W^{1,\infty}_\by}^2
%\int_{I_*}
%\Vert
% \overline{q}_1 - \overline{q}_2\Vert_{W^{1,2}_\bx}^2\dt
 \\&\lesssim
T^{1/3}_*
\Vert (\zeta_1, \overline{\mathbf{w}}_1, \overline{q}_1)-(\zeta_2, \overline{\mathbf{w}}_2, \overline{q}_2)\Vert_{X_{I_*}}^2.
\end{aligned}
\end{equation}
%Similarly,
%\begin{equation}
%\begin{aligned}
%\label{K4b}
%\int_{I_*}\Vert (\mathbf{B}_{\zeta_1}-\mathbf{B}_{\zeta_2})  \overline{q}_2\Vert_{W^{1,2}_\bx}^2\dt
%&\lesssim
%T^{1/3}_*
%\Vert (\zeta_1, \overline{\mathbf{w}}_1, \overline{q}_1)-(\zeta_2, \overline{\mathbf{w}}_2, \overline{q}_2)\Vert_{X_{I_*}}^2.
%\end{aligned}
%\end{equation}
By using \eqref{K4a} and \eqref{K4b}, it follows that
\begin{align}
\label{k4final}
K_4
%:=
%\int_I\Vert \mathbf{H}_{\zeta_1}(\overline{\mathbf{w}}_1, \overline{q}_1)- \mathbf{H}_{\zeta_2}(\overline{\mathbf{w}}_2, \overline{q}_2)\Vert_{W^{1,2}(\Omega)}^2\dt
\lesssim
T^{1/3}_*
\Vert (\zeta_1,\overline{\mathbf{w}}_1, \overline{q}_1) - (\zeta_2,\overline{\mathbf{w}}_2, \overline{q}_2) \Vert_{X_{I_*}}^2.
\end{align}
By collecting the estimates \eqref{k1final}, \eqref{k2final}, \eqref{k3final} and \eqref{k4final} together, we have
\begin{align*}
%\label{contractionEst}
\Vert \mathcal{T}(\zeta_1, \overline{\mathbf{w}}_1, \overline{q}_1)
-
\mathcal{T}(\zeta_2, \overline{\mathbf{w}}_2, \overline{q}_2)\Vert_{X_{I_*}}^2
&\leq 
c  T^{1/2}_*  \Vert (\zeta_1, \overline{\mathbf{w}}_1, \overline{q}_1)-(\zeta_2, \overline{\mathbf{w}}_2, \overline{q}_2)\Vert_{X_{I_*}}^2.
\end{align*}
Choosing $T_*$ in $I_*=(0,T_*)$ so that $ T^{1/2}_* <c^{-1}$ yields the desired contraction property.
%{\color{blue}
%
%To show the mapping  $\mathcal{T}:B_R^{X_{I_*}} \rightarrow  B_R^{X_{I_*}}$, we need to show that for any $(\zeta, \overline{\mathbf{w}}, \overline{q}) \in B_R^{X_{I_*}}$, we have that 
%\begin{align}
%\label{ballToBall}
%\Vert(\eta, \overline{\bu}, \overline{\pi})\Vert_{X_{I_*}}^2
%=
%\Vert\mathcal{T}(\zeta, \overline{\mathbf{w}}, \overline{q})\Vert_{X_{I_*}}^2
%\leq R.
%\end{align}
%Indeed, for $(\zeta_1, \overline{\mathbf{w}}_1, \overline{q}_1)=(\zeta, \overline{\mathbf{w}}, \overline{q})$ and $(\zeta_2, \overline{\mathbf{w}}_2, \overline{q}_2)=(0,\bm{0},0)$, since the former lives in $ B_R^{X_{I_*}}$, we immediately obtain \eqref{ballToBall} from \eqref{contractionEst}.
%}
\end{proof}


\section{The acceleration estimate}
\label{sec:reg}
In this section, we prove the acceleration estimate for a solution satisfying the Serrin condition. In order to make the proof rigorous, we work with a strong solution, the existence of which is  guaranteed locally in time by Theorem \ref{thm:fluidStructureWithoutFK}. Eventually, we compare weak and strong solution by means of Theorem \ref{thm:weakstrong} and compare the result for a weak solution in Theorem \ref{thm:main}.
%In this Section, we shall extend the strong solution obtained in Theorem \ref{fixedpoint} to the time interval $(0, T)$ for every $T>0$, by 
%proposing the additional condition for $\bu$ in Serrin type. 
Let $(\bu,\eta)$ be a strong solution \eqref{1}--\eqref{interfaceCond} in the sense of Definition \ref{def:strongSolution} with data $(\bff, g, \eta_0, \eta_*, \bu_0)$, which is in particular a weak solution (see Definition \ref{def:weakSolution}) and thus satisfies the standard energy estimates
\begin{align}
\sup_{I^\ast}\|\bu\|_{L^2_\bx}^2+\int_{I^\ast}\|\nabla\bu\|_{L^2_\bx}^2\dt\lesssim\,C_0,\label{eq:aprioriu}\\
\sup_{I^\ast}\|\partial_t\eta\|_{L^2_\by}^2+\sup_{I^\ast}\|\Dely\eta\|_{L^2_\by}^2+\int_{I^\ast}\|\partial_t\naby\eta\|_{L^2_\by}^2\dt\lesssim\,C_0,\label{eq:apriorieta}
\end{align}
where
\begin{align*}
 C_0&:=\|\bu_0\|_{L^2_\bx}^2+\|\eta_\ast\|_{L^2_\by}^2+\|\Dely\eta_0\|_{L^2_\by}^2+\int_{I^\ast}\|\bff\|_{L^2_\bx}^2\dt+\int_{I^\ast}\|g\|_{L^2_\by}^2\dt.
\end{align*} 
The following acceleration estimate (or second order energy estimate) is one of the core results of the paper and directly leads to the main result in Theorem \ref{thm:main}. Under the Serrin condition, it holds uniformly in time allowing us to extend the local solution globally in time.
%\todo{I think all the time interval $I_*$ in Proposition 4.1  and the proof should be changed to $I$}
\begin{theorem}\label{prop2}
Suppose that the dataset
$(\bff, g, \eta_0, \eta_*, \bu_0)$
satisfies \eqref{dataset} and
\eqref{datasetImproved}.
Suppose that $(\eta,\bu)$ is a strong solution to \eqref{1}--\eqref{interfaceCond} in the sense of Definition \ref{def:strongSolution}. 
Furthermore, for some $r\in[2,\infty)$ and $s\in(3,\infty]$ we set
%\todo{weak solution is pressure-free! Introduce pressure!}
\begin{align}\label{eq:regu}
C_1&:=\|\bu\|_{L^r(I;L^s(\Omega_\eta))},\quad \tfrac{2}{r}+\tfrac{3}{s}\leq1,\\
C_2&:=\|\eta\|_{ L^\infty(I;C^{1}(\omega))}.\label{eq:regeta}
\end{align}
Finally, suppose that there is no degeneracy in the sense of \eqref{eq:1705}.
%\todo{Are these conditions is optimal? (4.5) is related to the first inequality in (4.8), which might used only in the estimate of the second term in $V$, but I did not see it is in a right way. Otherwise, we can delete (4.5) }
 Then we have the estimate
 \begin{equation}
\begin{aligned}
\label{est:reg}
&\sup_{I_\ast}\int_\omega
\big(\vert \partial_t\naby \eta\vert^2 
+
\vert \naby\Dely \eta\vert^2
\big)
\dy
+
\sup_{I_\ast}\int_{\Omega_\eta}\vert\nabx  \bu \vert^2\dx
\\&\quad+
\int_{I_\ast}\int_\omega
\big(\vert \partial_t\Dely \eta \vert^2 + \vert \partial_t^2 \eta\vert^2
 \big)\dy\dt
 +
\int_{I_\ast}\int_{\Omega_\eta}\big( \vert \nabx^2 \bu\vert^2 +\vert \partial_t \bu \vert^2  + \vert \nabx  \pi\vert^2
 \big)\dx\dt
 \\&\lesssim
 \int_\omega\big( 
 \vert \naby\eta_*\vert^2
 +
  \vert \naby\Dely\eta_0\vert^2
  \big)\dy
  +
  \int_{\Omega_{\eta_0}} 
   \vert\nabx \bu_0\vert^2 \dx
   \\&\quad+ \int_{I_\ast}\int_{\Omega_\eta}  \vert \bff\vert^2  \dx\dt+
 \int_{I_\ast}\int_\omega  \vert g\vert^2  \dy\dt,
\end{aligned}
\end{equation}
%\todo{not find $\nabla_{\by} g$ on the right again; it seems the $\bff$ norm is in the constant $1$, so why not put $g$ inside as well, or write both directly on the right not using $1$ }
where the hidden constant depends only on $C_0, C_1$ and $C_2$.
\end{theorem}
\begin{proof}
We use
$
\bfphi=\partial_t\bu+\mathscr F_\eta(\partial_t\eta\bfn)\cdot\nabla\bu
$
and $\phi=\partial_t^2\eta$ as test functions for the fluid and shell equations respectively. Here $\mathscr F_\eta$ is the extension operator introduced in Section \ref{sec:ext}. %Note that we have
%\todo{check this!}
%\begin{align}
%\|\mathscr F_\eta(\partial_t\eta\bfn)\|_{W^{1/2,2}_\bx}&\lesssim \|\partial_t\eta\|_{W^{\varepsilon,2}_\by},\quad
%\|\mathscr F_\eta(\partial_t\eta\bfn)\|_{W^{3/2,2}_\bx}\lesssim \|\partial_t\eta\|_{W^{1,2}_\by},\label{eq:feta}
%\end{align}
%as a consequence of Lemma \ref{lem:3.8}. This implies that for all $\overline q<\infty$,
%\begin{align}\label{eq:2010}
%\sup_{I_\ast}\|\mathscr F_\eta(\partial_t\eta\bfn)\|_{L^{3}_\bx}^2+
%\int_{I_\ast}\|\mathscr F_\eta(\partial_t\eta\bfn)\|_{L^{\overline q}_\bx}^2\dt
%\lesssim C_0
%\end{align}
%with $C_0$ given in \eqref{eq:apriorieta}.
 From the momentum equation in the strong form \eqref{1'}, we obtain for $t\in I^\ast$
\begin{align*}
&\int_{0}^t\int_{\Omega_\eta}\big(\partial_t\bu+\bu\cdot\nabla\bu\big)\cdot\big(\partial_t\bu+\mathscr F_\eta(\partial_t\eta\bfn)\cdot\nabla\bu\big)\dx\ds\\
&=\int_{0}^t\int_{\Omega_\eta}\Div\bftau\cdot\big(\partial_t\bu+\mathscr F_\eta(\partial_t\eta\bfn)\cdot\nabla\bu\big)\dx\ds+\int_{0}^t\int_{\Omega_\eta}\bff\cdot\big(\partial_t\bu+\mathscr F_\eta(\partial_t\eta\bfn)\cdot\nabla\bu)\big)\dx\ds,
\end{align*}
where $\bftau=\nabla\bu+\nabla\bu^\intercal-\pi\mathbb I_{3\times 3}$ is the Cauchy stress.
We now aim at  integrating by parts in the first term on the right-hand side
obtaining
\begin{align}\nonumber
&\int_{0}^t\int_{\Omega_\eta}\Div\bftau\cdot\big(\partial_t\bu+\mathscr F_\eta(\partial_t\eta\bfn)\cdot\nabla\bu\big)\dx\ds\\&=-\tfrac{1}{2}\int_{\Omega_\eta}|\nabla\bu|^2\dx+\tfrac{1}{2}\int_{\Omega_\eta}|\nabla\bu_0|^2\dx+\int_{I_\ast}\int_{\partial\Omega_\eta}(\partial_t\eta\bfn)\circ\bfvarphi_\eta^{-1}\cdot\bfn_\eta\circ\bfvarphi_\eta^{-1}|\nabla\bu|^2\dd\mathcal H^2\dt\nonumber\\
&\quad-\int_{0}^t\int_{\Omega_\eta}\nabla\bu:\nabla\big(\mathscr F_\eta(\partial_t\eta\bfn)\cdot\nabla\bu\big)\dx\ds+\int_{0}^t\int_{\Omega_\eta}\pi\,\Div\big(\mathscr F_\eta(\partial_t\eta\bfn)\nabla\bu\big)\dx\ds\nonumber\\
&\quad-\int_0^t\int_\omega \bfF\,\partial_t^2\eta\dy\ds\label{eq:well}
\end{align}
with 
\begin{equation}\label{Fdef}
\bfF=-\bfn^\intercal \bftau\circ\bfvarphi_\eta\bn_\eta|\det(\naby\bfvarphi_\eta)|.\end{equation}
Note that we also used Reynold's transport theorem (applied to $\int_{\Omega_{\eta(t)}}|\nabla\bu(t)|^2\dx$).
Although all the terms in equation \eqref{eq:well} are well-defined for a strong solution $(\eta,\bu)$ this is not true for its derivation. Hence we apply Lemma \ref{lem:smooth} to obtain a smooth approximation which fully justifies \eqref{eq:well} after passing to the limit.


Multiply the structure equation \eqref{2'} by $\partial_t^2\eta$ we obtain from the formal computation \eqref{3.10} that
\begin{equation*}
	\begin{aligned}
&\int_{I_\ast}\int_\omega|\partial_t^2\eta|^2\dy\dt
+
\sup_{I_\ast}\int_\omega|\partial_t\naby\eta|^2\dy
\\&\lesssim
\int_\omega|\naby\eta_*|^2\dy
+
\sup_{I_\ast}\int_\omega|\naby\Dely\eta|^2\dy+\int_{I_\ast}\int_\omega|\partial_t\Dely\eta|^2\dy\dt+\int_{I_\ast}\int_\omega(g+\bfF)\,\partial_t^2\eta\dy\dt.
\end{aligned}
\end{equation*}
It can be made rigorous by means of a spatial regularization argument. Since we consider periodic boundary conditions a spatial convolution can be applied without further difficulty.
Combining both, using Young's inequality and writing
$$\mathscr F_\eta(\partial_t\eta\bfn)\cdot\nabla\bu=\bu\cdot\nabla\bu+\mathscr F_\eta(\partial_t\eta\bfn)\cdot\nabla\bu-\bu\cdot\nabla\bu,$$
we derive that
%\todo{It seems the some terms are missed in (4.10) for instance two green terms, please check again}
\begin{align}\label{eq:0302}
\begin{aligned}
&\sup_{I_\ast}\int_{\Omega_\eta}|\nabla\bu|^2\dx+\int_{I_\ast}\int_{\Omega_\eta}|\partial_t\bu+\bu\cdot\nabla\bu|^2\dx\dt
+\int_{I_\ast}\int_\omega|\partial_t^2\eta|^2\dy\dt+\sup_{I_\ast}\int_\omega|\partial_t\naby\eta|^2\dy
\\
&\lesssim\int_{I_\ast}\int_{\Omega_\eta}|\bu\cdot\nabla\bu|^2\dx\dt+\int_{I_\ast}\int_{\partial\Omega_\eta}(\partial_t\eta\bfn)\circ\bfvarphi_\eta^{-1}\cdot\bfn_\eta\circ\bfvarphi_\eta^{-1}|\nabla\bu|^2\dd\mathcal H^2\dt\\
&\quad-\int_{I_\ast}\int_{\Omega_\eta}\big(\partial_t\bu+\bu\cdot\nabla\bu\big)\cdot\big(\mathscr F_\eta(\partial_t\eta\bfn)\cdot\nabla\bu\big)\dx\dt\\
&\quad-\int_{I_\ast}\int_{\Omega_\eta}\nabla\bu:\big(\mathscr F_\eta(\partial_t\eta\bfn)^\intercal\nabla^2\bu+\nabla\mathscr F_\eta(\partial_t\eta\bfn)\nabla\bu^\intercal\big)\dx\dt\\
&\quad+\int_{I_\ast}\int_{\Omega_\eta}\pi\,\Div\big(\mathscr F_\eta(\partial_t\eta\bfn)\nabla\bu\big)\dx\dt+\int_{I_*}\int_{\Omega_\eta}\bff\cdot \mathscr F_{\eta}(\partial_t\eta\bfn)\nabla\bu\dx\dt\\
&\quad+\int_{I_\ast}\int_{\Omega_\eta}|\bff|^2\dx\dt+\int_{\Omega_{\eta_0}}|\nabla\bu_0|^2\dx%\\&+\int_{I_\ast}\int_{\Omega_{\mathfrak r^\varrho\eta^\varrho}}(\bu-\mathcal R^\varrho\bu)\cdot\nabla\bu\cdot(\bu+\mathscr F_\eta(\partial_t\eta\bfn)\cdot\nabla\bu)\dxs
+\int_\omega|\naby\eta_*|^2\dy+\sup_{I_\ast}\int_\omega|\naby\Dely\eta|^2\dy\\
&\quad+\int_{I_\ast}\int_\omega|\partial_t\Dely\eta|^2\dy\dt+\int_{I_\ast}\int_\omega|g|^2\dy\dt\\
%{{\color{green}+\int_{I_*}\int_{\partial\Omega_\eta}\tau\cdot \mathscr{F}_\eta(\partial_t\eta\bfn)\nabla\bu\dd \mathcal{H}^2\dt
&=:\mathrm{I}+\dots+\mathrm{XII}.
\end{aligned}
\end{align}
Notice that in \eqref{eq:0302} the last six terms are already uncritical. In particular, XI will be obtained on the left hand side via a second test, see \eqref{eq:secondtest} below. Hence we start the estimate of the terms $\mathrm{I}$--$\mathrm{VI}$. 

In order to control the first term, we make use of Theorem \ref{thm:stokessteady}. Its application to the moving domain $\Omega_\eta$ has been justified in Remark \ref{rem:stokes}, we thereby have
\begin{align}\label{eq:regstokes}
\|\bu\|_{W^{2,2}_\bx}+\|\pi\|_{W^{1,2}_\bx}\lesssim\|\partial_t\bu+\bu\cdot\nabla\bu\|_{L^2_\bx}+\|\bff\|_{L^2_\bx}+\|\partial_t\eta\|_{W^{3/2,2}_\by},
\end{align}
uniformly in time with a constant depending on $C_2$ from \eqref{eq:regeta}. Note that we also used the estimate
\begin{align}\label{again}
\|\partial_t\eta\bfn\circ \bfvarphi_\eta^{-1}\|_{W^{3/2,2}_\by}\lesssim\|\partial_t\eta\|_{W^{3/2,2}_\by},
\end{align}
which is a consequence of \eqref{eq:apriorieta} and the definition ${\bfvarphi}_\eta={\bfvarphi}+ \eta{\bfn}$. In fact,
$\bfvarphi_\eta^{-1}$ is uniformly bounded in time in the space
of Sobolev multipliers on $W^{3/2,2}(\omega)$ by \eqref{eq:MSa} and \eqref{eq:MSb}  (together with the assumption $\partial_1\bfvarphi_\eta\times\partial_2\bfvarphi_\eta \neq0$) since $\eta$ is uniformly bounded in time even in $W^{2,2}(\omega)$ by \eqref{eq:apriorieta}.
Hence the transformation rule \eqref{lem:9.4.1} applies.
For every $\kappa>0$ and for $s\in(3,\infty]$, we estimate by using Sobolev's inequality (recalling that $\partial\Omega_\eta$ is Lipschitz uniformly in time with a constant controlled by $C_2$, cf. \eqref{eq:regeta}) and \eqref{eq:aprioriu}
%\todo{here it might be interesting to consider if the critical case $r=\infty$ and $s=3$ can be achieved, maybe this need other tools}
\begin{align*}%\label{eq:reg1}
\begin{aligned}
\mathrm{I}&\leq \int_{I_\ast}\|\bu\|^2_{L^s_\bx}\|\nabla\bu\|_{L^{\frac{2s}{s-2}}_\bx}^2\dt
\leq\,c\int_{I_\ast}\|\bu\|_{L^s_\bx}^2\|\nabx\bu\|^{\frac{2s-6}{s}}_{L^{2}_\bx}\|\nabla\bu\|_{W^{1,2}_\bx}^{\frac{6}{s}}\dt\\
&\leq\,c(\kappa)\int_{I_\ast}\|\bu\|_{L^s_\bx}^{\frac{2s}{s-3}}\|\nabx\bu\|^{2}_{L^{2}_\bx}\dt+\kappa\int_{I_\ast}\|\nabla\bu\|_{W^{1,2}_\bx}^2\dt\\
%&\leq \,c\int_{I_\ast}\|\bu\|^2_{W^{1,2}_x}\big(\|\partial_t\bu+\bu\cdot\nabla\bu\|_{L^2_x}+\|\bff\|_{L^2_x}+\|\partial_t\eta\|_{W^{3/2,2}_y}\big)\dt\\
&\leq \,c(\kappa)\int_{I_\ast}\|\bu\|_{L^s_\bx}^{\frac{2s}{s-3}}\|\nabx\bu\|^{2}_{L^{2}_\bx}\dt+\kappa\int_{I_\ast}\big(\|\partial_t\bu+\bu\cdot\nabla\bu\|_{L^2_\bx}^2+\|\bff\|^2_{L^2_\bx}+\|\partial_t\eta\|^2_{W^{3/2,2}_\by}\big)\dt,
\end{aligned}
\end{align*}
where the first part of the $\kappa$-term can be absorbed in the left-hand side of \eqref{eq:0302}. 
Note that $r:=\frac{2s}{s-3}\in[2,\infty)$ since $s\in(3,\infty]$. The resulting constant depends on $C_1$ from \eqref{eq:regu}.

For the boundary integral $\mathrm{II}$ on the right-hand side of \eqref{eq:0302}, we have
\begin{equation*}\label{estimateII}
\begin{aligned}
\mathrm{II}&\leq \int_{I_*}\lVert\nabla\bu\rVert_{L^4(\partial\Omega_{\eta})}\lVert\nabla\bu\rVert_{L^{\frac{8}{3}}(\partial\Omega_{\eta})}\lVert\partial_t\eta\circ\bfvarphi_\eta^{-1}\rVert_{L^{\frac{8}{3}}(\partial\Omega_\eta)} \dt\\
&\lesssim  \int_{I_*}\lVert\nabla\bu\rVert_{W^{1/2, 2}(\partial\Omega_\eta)}\lVert\nabla\bu\rVert_{W^{1/4, 2}(\partial\Omega_\eta)}\lVert\partial_t\eta\circ\bfvarphi_\eta^{-1}\rVert_{W^{1/4,2}(\partial\Omega_\eta)}\dt\\
&\lesssim \int_{I_*}\lVert\nabla\bu\rVert_{W^{1,2}_\bx}\lVert\nabla\bu\rVert_{W^{3/4,2}_\bx}\lVert\partial_t\eta\rVert_{W^{1/4,2}_\by}\dt\\
&\lesssim \int_{I_*}\lVert\nabla\bu\rVert_{W^{1,2}_\bx}^{\frac{7}{4}}\lVert\nabla\bu\rVert_{L^2_\bx}^{\frac{1}{4}}\lVert\partial_t\eta\rVert_{L^2_\by}^{\frac{3}{4}}\lVert\partial_t\eta\rVert_{W^{1,2}_\by}^{\frac{1}{4}}\dt\\
&\leq \kappa\int_{I_*}\lVert\nabla\bu\rVert_{W^{1,2}_\bx}^2\dt+C(\kappa)\int_{I_*}\lVert\partial_t\eta\rVert_{W^{1,2}_\by}^2\lVert\nabla\bu\rVert_{L^2_\bx}^2\dt,
\end{aligned}
\end{equation*}
where we used that $\lVert\partial_t\eta\rVert_{L^2_\by}$ is uniformly bounded in time (see \eqref{eq:apriorieta}) and the embeddings $W^{1/2,2}(\partial\Omega_\eta)\hookrightarrow L^4(\partial\Omega_\eta)$ and $W^{1/4,2}(\partial\Omega_\eta)\hookrightarrow L^{8/3}(\partial\Omega_\eta)$, as well as the following interpolation inequalities:
\begin{equation}\label{interalways}
\begin{aligned}
\lVert f\rVert_{W^{\frac{1}{4},2}(\omega)}&\lesssim \lVert f\rVert_{L^2(\omega)}^{\frac{3}{4}}\lVert f\rVert_{W^{1,2}(\omega)}^{\frac{1}{4}},\\
\lVert f\rVert_{W^{\frac{3}{4},2}(\Omega_\eta)}&\lesssim \lVert f\rVert_{L^2(\Omega_\eta)}^{\frac{1}{4}}\lVert f\rVert_{W^{1,2}(\Omega_\eta)}^{\frac{3}{4}}.
\end{aligned}
\end{equation}

For the third term $\mathrm{III}$, we first have
\begin{align*}
\mathrm{III}&\leq\,\kappa\int_{I_\ast}\|\partial_t\bu+\bu\cdot\nabla\bu\|_{L^2_\bx}^2\dt
+c(\kappa)\int_{I_\ast}\|\nabla\bu\mathscr F_\eta(\partial_t\eta\bfn)\|^2_{L^2_\bx}\dt.
\end{align*}
We have the estimate for the second integrals above that
\begin{equation*}
\begin{aligned}
&\int_{I_\ast}\|\nabla\bu\mathscr F_\eta(\partial_t\eta\bfn)\|^2_{L^2_\bx}\dt\leq \int_{I_*}\lVert \mathscr F_\eta(\partial_t\eta\bfn)\rVert_{L^4_\bx}^2\lVert \nabla\bu\rVert_{L^4_\bx}^2\dt\\
&\lesssim \int_{I_\ast}\lVert \mathscr F_\eta(\partial_t\eta\bfn)\rVert_{W^{3/4,2}_\bx}^2\lVert\nabla\bu\rVert_{L^2_\bx}^{\frac{1}{2}}\lVert\nabla\bu\rVert_{W^{1,2}_\bx}^{\frac{3}{2}}\dt\\
&\lesssim \int_{I_*}\lVert\partial_t\eta\rVert_{W^{1/4, 2}_\by}^2\lVert\nabla\bu\rVert_{L^2_\bx}^{\frac{1}{2}}\lVert\nabla\bu\rVert_{W^{1,2}_\bx}^{\frac{3}{2}}\dt\\
&\lesssim \int_{I_*}\lVert\partial_t\eta\rVert_{W^{1,2}_\by}^{\frac{1}{2}}\lVert\nabla\bu\rVert_{L^2_\bx}^{\frac{1}{2}}\lVert\nabla\bu\rVert_{W^{1,2}_\bx}^{\frac{3}{2}}\dt\\
&\leq \kappa \int_{I_*}\lVert\nabla\bu\rVert_{W^{1,2}_\bx}^2\dt+C(\kappa)\int_{I_*}\lVert\partial_t\eta\rVert_{W^{1,2}_\by}^2\lVert\nabla\bu\rVert_{L^2_\bx}^2\dt.
\end{aligned}
\end{equation*}
Here we used the 2D interpolation in \eqref{interalways} again and 
$$\lVert\mathscr F_\eta(\partial_t\eta\bfn)\rVert_{W^{3/4,2}_\bx}\lesssim \lVert\partial_t\eta\rVert_{W^{1/4,2}_\by}, $$ 
and the inequalities:
\begin{equation}\label{reatinter}
\begin{aligned}
\lVert f\rVert_{L^4(\Omega_\eta)}&\lesssim \lVert f\rVert_{W^{3/4, 2}(\Omega_\eta)},\\
\lVert f\rVert_{L^4(\Omega_\eta)}&\lesssim \lVert f\rVert_{L^2(\Omega_\eta)}^{\frac{1}{4}}\lVert f\rVert_{W^{1,2}(\Omega_\eta)}^{\frac{3}{4}}.
\end{aligned}
\end{equation}
Thus we have the estimate for $\mathrm{III}$:
\begin{equation*}\label{IIestimate}
\mathrm{III}\leq \kappa\int_{I_\ast}\left(\|\partial_t\bu+\bu\cdot\nabla\bu\|_{L^2_\bx}^2+\lVert\nabla\bu\rVert_{W^{1,2}_\bx}^2\right)\dt+C(\kappa) \int_{I_*}\lVert\partial_t\eta\rVert_{W^{1,2}_\by}^2\lVert\nabla\bu\rVert_{L^2_\bx}^2\dt.
\end{equation*}

Now we consider estimating the integral $\mathrm{IV}$. For this we have 
\begin{align}
&\int_{I_*}\int_{\Omega_\eta}\nabla\bu:\mathscr F_\eta(\partial_t\eta\bfn)^\intercal\nabla^2\bu\dx\dt\nonumber\\
&\leq \int_{I_*}\lVert\nabla^2\bu\rVert_{L^2_\bx}\lVert\nabla\bu\rVert_{L^4_\bx}\lVert\mathscr F_\eta(\partial_t\eta\bfn)\rVert_{L^4_\bx}\dt\nonumber\\
&\lesssim \int_{I_*}\lVert\nabla^2\bu\rVert_{L^2_\bx}\lVert\nabla\bu\rVert_{L^2_\bx}^{\frac{1}{4}}\lVert\nabla\bu\rVert_{W^{1,2}_\bx}^{\frac{3}{4}}\lVert\mathscr F_\eta(\partial_t\eta\bfn)\rVert_{W^{3/4,2}_\bx}\dt \nonumber\\
&\lesssim \int_{I_*}\lVert\bu\rVert_{W^{2,2}_\bx}^{\frac{7}{4}}\lVert\nabla\bu\rVert_{L^2_\bx}^{\frac{1}{4}}\lVert\partial_t\eta\rVert_{W^{1/4, 2}_\by}\dt\nonumber\\
&\lesssim \int_{I_*}\lVert\bu\rVert_{W^{2,2}_\bx}^{\frac{7}{4}}\lVert\nabla\bu\rVert_{L^2_\bx}^{\frac{1}{4}}\lVert\partial_t\eta\rVert_{W^{1, 2}_\by}^{\frac{1}{4}}\dt\nonumber\\
&\leq \kappa\int_{I_*}\lVert\bu\rVert_{W^{2,2}_\bx}^2\dt+C(\kappa)\int_{I_*}\lVert\partial_t\eta\rVert_{W^{1,2}_\bx}^2\lVert\nabla\bu\rVert_{L^2_\bx}^2\dt,\label{IV1}
\end{align}
where we used the energy estimate \eqref{eq:apriorieta} and the interpolation inequality \eqref{interalways} and \eqref{reatinter} again. Then for the second integral in $\mathrm{IV}$, we also have
\begin{equation}\label{IV2}
\begin{aligned}
&\int_{I_*}\int_{\Omega_\eta}\nabla\bu:\nabla\mathscr F_\eta(\partial_t\eta\bfn)\nabla\bu^\intercal\dx\dt\\
&\lesssim \int_{I_*}\lVert\nabla\bu\rVert_{L^4_\bx}\lVert\nabla\bu^\intercal\rVert_{L^4_\bx}\lVert\nabla\mathscr F_\eta(\partial_t\eta\bfn)\rVert_{L^2_\bx}\dt\\
&\lesssim \lVert\nabla\bu\rVert_{L^2_\bx}^{\frac{1}{4}}\lVert\nabla\bu\rVert_{W^{1,2}_\bx}^{\frac{3}{4}}\lVert\nabla\bu^\intercal\rVert_{L^2_\bx}^{\frac{1}{4}}\lVert\nabla\bu^\intercal\rVert_{W^{1,2}_\bx}^{\frac{3}{4}}\lVert\mathscr F_\eta(\partial_t\eta\bfn)\rVert_{W^{1,2}_\bx}\dt\\
&\lesssim \int_{I_*}\lVert\bu\rVert_{W^{2,2}_\bx}^{\frac{3}{2}}\lVert\bu\rVert_{W^{1,2}_\bx}^{\frac{1}{2}}\lVert\partial_t\eta\rVert_{W^{1/2,2}_\by}\dt\\
&\lesssim\int_{I_*} \lVert\bu\rVert_{W^{2,2}_\bx}^{\frac{3}{2}}\lVert\bu\rVert_{W^{1,2}_\bx}^{\frac{1}{2}}\lVert\partial_t\eta\rVert_{W^{1,2}_\by}^{\frac{1}{2}}\dt\\
&\leq \kappa\int_{I_*}\lVert \bu\rVert_{W^{2,2}_\bx}^2\dt+C(\kappa)\int_{I_*}\int_{I_*}\lVert\partial_t\eta\rVert_{W^{1,2}_\by}^2\lVert\bu\rVert_{W^{1,2}_\bx}^2\dt,
\end{aligned}
\end{equation}
where we used the interpolation:
$$\lVert \partial_t\eta\rVert_{W^{1/2, 2}_\by}\lesssim \lVert\partial_t\eta\rVert_{L^2_\by}^{\frac{1}{2}}\lVert\partial_t\eta\rVert_{W^{1,2}_\by}^{\frac{1}{2}}. $$
Putting the estimate \eqref{IV1} and \eqref{IV2} together, we arrive at
\begin{equation*}\label{IV}
\mathrm{IV}\leq \kappa\int_{I_*}\lVert \bu\rVert_{W^{2,2}_\bx}^2\dt+C(\kappa)\int_{I_*}\lVert\partial_t\eta\rVert_{W^{1,2}_\by}^2\lVert\bu\rVert_{W^{1,2}_\bx}^2\dt.
\end{equation*}

To estimate $\mathrm{V}$, we note that it can be written as
\begin{align*}
	\mathrm{V}=-\int_{I_\ast}\int_{\Omega_\eta}\nabla\pi\cdot\mathscr F_\eta(\partial_t\eta\bfn)\nabla\bu\dx\dt+\int_{I_\ast}\int_{\partial\Omega_\eta}\pi\,\mathscr F_\eta(\partial_t\eta\bfn)\nabla\bu\,\bfn_\eta\circ\bfvarphi_\eta^{-1}\,\dd\mathcal H^2\dt.
\end{align*}
And we have the estimates for the two integrals on the right-hand side above one by one. Firstly, we derive that
\begin{equation}\label{fsimilar}
\begin{aligned}
&\int_{I_\ast}\int_{\Omega_\eta}\nabla\pi\cdot\mathscr F_\eta(\partial_t\eta\bfn)\nabla\bu\dx\dt\\
&\leq \int_{I_*}\lVert\nabla\pi\rVert_{L^2_\bx}\lVert\mathscr F_\eta(\partial_t\eta\bfn)\rVert_{L^4_\bx}\lVert\nabla\bu\rVert_{L^4_\bx}\dt\\
&\lesssim \int_{I_*}\lVert\nabla\pi\rVert_{L^2_\bx}\lVert\mathscr F_\eta(\partial_t\eta\bfn)\rVert_{W^{3/4,2}_\bx}\lVert\nabla\bu\rVert_{L^2_\bx}^{\frac{1}{4}}\lVert\nabla\bu\rVert_{W^{1,2}_\bx}^{\frac{3}{4}}\dt\\
&\lesssim \int_{I_*}\lVert \nabla\pi\rVert_{L^2_\bx}\lVert\partial_t\eta\rVert_{W^{1/4,2}_\by}\lVert\nabla\bu\rVert_{L^2_\bx}^{\frac{1}{4}}\lVert\nabla\bu\rVert_{W^{1,2}_\bx}^{\frac{3}{4}}\dt\\
&\lesssim \int_{I_*}\lVert\nabla\pi\rVert_{L^2_\bx}\lVert\partial_t\eta\rVert_{W^{1,2}_\by}^{\frac{1}{4}}\lVert\nabla\bu\rVert_{L^2_\bx}^{\frac{1}{4}}\lVert\nabla\bu\rVert_{W^{1,2}_\bx}^{\frac{3}{4}}\dt\\
&\lesssim \kappa\int_{I_*}\left(\lVert\nabla\pi\rVert_{L^2_\bx}^2+\lVert\nabla\bu\rVert_{W^{1,2}_\bx}^2\right)\dt+C(\kappa)\int_{I_*}\lVert\partial_t\eta\rVert_{W^{1,2}_\by}^2\lVert\nabla\bu\rVert_{L^2_\bx}^2\dt.
\end{aligned}
\end{equation}
Then, by using similar technique, we also have 
\begin{equation*}
\begin{aligned}
&\int_{I_\ast}\int_{\partial\Omega_\eta}\pi\,\mathscr F_\eta(\partial_t\eta\bfn)\nabla\bu\,\bfn_\eta\circ\bfvarphi_\eta^{-1}\,\dd\mathcal H^2\dt\\
&\leq \int_{I_*}\lVert\pi\rVert_{L^4(\partial\Omega\eta)}\lVert\mathscr F_\eta(\partial_t\eta\bfn)\rVert_{L^{8/3}(\partial\Omega_\eta)}\lVert\nabla\bu\rVert_{L^{8/3}(\partial\Omega_\eta)}\dt\\
&\lesssim \int_{I_*}\lVert\pi\rVert_{W^{1/2,2}(\partial\Omega_\eta)}\lVert\mathscr  F_{\eta}(\partial_t\eta\bfn)\rVert_{W^{1/4,2}(\partial\Omega_\eta)}\lVert\nabla\bu\rVert_{W^{1/4, 2}(\partial\Omega_\eta)}\dt\\
&\lesssim \int_{I_*}\lVert\pi\rVert_{W^{1,2}_\bx}\lVert\partial_t\eta\rVert_{W^{1/4,2}_\by}\lVert\nabla\bu\rVert_{W^{3/4,2}_\bx}\dt\\
&\lesssim \int_{I_*}\lVert\pi\rVert_{W^{1,2}_\bx}\lVert\partial_t\eta\rVert_{W^{1,2}_\by}^{\frac{1}{4}}\lVert\nabla\bu\rVert_{L^2_\bx}^{\frac{1}{4}}\lVert\nabla\bu\rVert_{W^{1,2}_\bx}^{\frac{3}{4}}\dt\\
&\leq \kappa\int_{I_*}\left(\lVert\pi\rVert_{W^{1,2}_\bx}^2+\lVert\bu\rVert_{W^{2,2}_\bx}^2\right)\dt+C(\kappa)\int_{I_*}\lVert\partial_t\eta\rVert_{W^{1,2}_\by}^2\lVert\nabla\bu\rVert_{L^2_\bx}^2\dt.
\end{aligned}
\end{equation*}
Therefore, we obtain the estimate for $\mathrm{V}$ as follows:
\begin{equation*}\label{Vesti}
\begin{aligned}
\mathrm{V}\leq \kappa\int_{I_*}\left(\lVert\pi\rVert_{W^{1,2}_\bx}^2+\lVert\bu\rVert_{W^{2,2}_\bx}^2\right)\dt+C(\kappa)\int_{I_*}\lVert\partial_t\eta\rVert_{W^{1,2}_\by}^2\lVert\nabla\bu\rVert_{L^2_\bx}^2\dt.
\end{aligned}
\end{equation*}

Notice that the sixth integral on the right-hand side of \eqref{eq:0302}, i.e. $\mathrm{VI}$, can be treated similarly as \eqref{fsimilar}. 
Observe from the estimate of $\mathrm{V}$ above, we must estimate here also the $L^2$-norm of the pressure, for which we use
\eqref{eq:pressure} (noticing that $\int_{\omega}\bfn\cdot\bfn_\eta|\partial_y\bfvarphi_\eta|\dy$ is strictly positive by our assumption of non-degeneracy). We have
\begin{equation*}
	\begin{aligned}
\int_{I_\ast}\|\pi\|^2_{W^{1,2}_\bx}\dt&\lesssim \int_{I_\ast}\|\nabla\pi\|^2_{L^{2}_\bx}\dt+\int_{I_\ast}c_\pi^2\dt\\
&\lesssim \int_{I_\ast}\|\nabla\pi\|^2_{L^{2}_\bx}\dt+\int_{I_\ast}\int_\omega|\partial_t^2\eta|^2\dy\dt+\int_{I_\ast}\int_\omega|g|^2\dy\dt\\
&\quad+\int_{I_\ast}\|\pi_0\|_{L^{2}(\partial\Omega_\eta)}^2\dt+\int_{I_\ast}\|\nabla\bu\|_{L^{2}(\partial\Omega_\eta)}^2\dt.
\end{aligned}
\end{equation*}
 Moreover, for any $\epsilon\in(0,1/2)$ the last term above can be estimated as                                                   
\begin{align*}
\int_{I_\ast}\|\nabla\bu\|_{L^{2}(\partial\Omega_\eta)}^2\dt
&\lesssim \int_{I_\ast}\|\nabla\bu\|_{W^{1/2+\epsilon,2}(\Omega_\eta)}^2\dt\\
&\lesssim\int_{I_\ast}
\|\nabla\bu\|_{L^{2}(\Omega_\eta)}^{1-2\epsilon}
\| \bu\|_{W^{2,2}(\Omega_\eta)}^{1+2\epsilon}\dt\\
&\leq \kappa\int_{I_\ast}\big(\|\partial_t\bu+\bu\cdot\nabla\bu\|_{L^2_\bx}^2+\|\bff\|^2_{L^2_\bx}+\|\partial_t\eta\|^2_{W^{3/2,2}_\by}\big)\dt
+c(\kappa)\int_{I_\ast}\|\nabla\bu\|_{L^{2}_\bx}^{2}\dt,
\end{align*}
whereas by using Poincar\'e inequality,
\begin{align*}
\int_{I_\ast}\|\pi_0\|_{L^{2}(\partial\Omega_\eta)}^2\dt
\lesssim\int_{I_\ast}\|\nabla\pi_0\|_{L^{2}_\bx}^{2}\dt+\int_{I_\ast}\|\pi_0\|_{L^{2}_\bx}^{2}\dt\lesssim \int_{I_\ast}\|\nabla\pi_0\|_{L^{2}_\bx}^{2}\dt=\int_{I_\ast}\|\nabla\pi\|_{L^{2}_\bx}^{2}\dt,
\end{align*}
where we used that $(\pi_0)_{\Omega_{\eta}}=0$ by definition.
At this stage, the integrals on the
pressure in the above can now be controlled by means of \eqref{eq:regstokes}.


Combining all the above estimates, choosing $\kappa$ small enough and using \eqref{eq:apriorieta} once more we conclude that
\begin{align}\label{eq:reg2}
\begin{aligned}
&\sup_{I_\ast}\int_{\Omega_\eta}|\nabla\bu|^2\dx+\int_{I_\ast}\int_{\Omega_\eta}|\partial_t\bu+\bu\cdot\nabla\bu|^2\dx\dt
+\int_{I_\ast}\int_\omega|\partial_t^2\eta|^2\dx\dt
+
\sup_{I_\ast}\int_\omega|\partial_t\naby\eta|^2\dy 
\\
&\lesssim 
\int_{I_\ast}\|\bff\|_{L^2_\bx}^2\dt+\|\nabla\bu_0\|_{L^2_\bx}^2
+
\int_{I_\ast}\|g\|_{L^2_\by}^2\dt+\|\naby\eta_*\|^2_{L^2_\by}+\int_{I_\ast}\|\nabla\bu\|_{L^{2}_\bx}^{2}\dt
\\&
\quad+\int_{I_\ast}
\|\bu\|_{L^s_\bx}^{\frac{2s}{s-3}}\|\nabx\bu\|^{2}_{L^{2}_\bx}\dt 
+
\int_{I_\ast}\|\partial_t\eta\|_{W^{1,2}_\by}^2\|\nabla\bu\|_{L^2_\bx}^2\dt+\int_{I_*}\lVert\partial_t\eta\rVert_{W^{1,2}_\by}^2\dt
\\
&\quad+
\sup_{I_\ast}\|\naby\Dely\eta\|_{L^2_\by}^2
+\int_{I_\ast}\|\partial_t\Dely\eta\|_{L^2_\by}^2\dt.
\end{aligned}
\end{align}
In the above we used the interpolation for the structure norm $\lVert\partial_t\eta\rVert_{W^{3/2}_\by}^2$:
\begin{equation}\label{interpoeta}
 \lVert\partial_t\eta\rVert_{W^{3/2}_\by}^2\leq \lVert\partial_t\eta\rVert_{W^{1,2}_\by}\lVert\partial_t\eta\rVert_{W^{2,2}_\by}\leq \kappa\lVert\partial_t\eta\rVert_{W^{2,2}_\by}^2+C(\kappa)\lVert\partial_t\eta\rVert_{W^{1,2}_\by}^2.
\end{equation}

Testing the structure equation by $\partial_t\Dely\eta$ yields\footnote{This test can be rigorously performed by mollifying the structure equation and multiplying it with the mollified test-function.}
\begin{align}
\label{eq:secondtest}
\begin{aligned}
&\frac{1}{2}\sup_{I_\ast}\int_\omega|\partial_t\naby\eta|^2\dy+\int_{I_\ast}\int_\omega|\partial_t\Dely\eta|^2\dy\dt+\frac{1}{2}\sup_{I_\ast}\int_\omega|\naby\Dely\eta|^2\dy\\&=\frac{1}{2}\int_\omega|\naby\eta_*|^2\dy+\frac{1}{2}\int_\omega|\naby\Dely\eta_0|^2\dy-\int_{I_\ast}\int_\omega(g+\bfF)\,\partial_t\Dely\eta\dy\dt,
\end{aligned}
\end{align}
where $\bfF$ has been introduced in \eqref{Fdef}.

Using a similar argument as in \eqref{again} to control $\bfF$ by $\bftau$ and as in $\mathrm{I}-\mathrm{III}$ and $\mathrm{V}$ to estimate $\bftau$, we have
\begin{align*}
&\int_{I_\ast}\int_\omega\bfF\cdot\partial_t\Dely\eta\dy\dt\\
&\leq\int_{I_\ast}\|\bfF\|_{W^{1/2,2}(\omega)}\|\partial_t\Dely\eta\|_{W^{-1/2,2}(\omega)}\dt\\
&\lesssim\int_{I_\ast}\|\bftau\|_{W^{1/2,2}(\partial\Omega_\eta)}\|\partial_t\eta\|_{W^{3/2,2}(\omega)}\dt
\\
&\lesssim\int_{I_\ast}\big(\|\partial_t\bu+\bu\cdot\nabla\bu\|_{L^{2}_\bx}+\|\bff\|_{L^2_\bx}+\|\partial_t\eta\|_{W^{3/2,2}_\by}+\|\partial_t^2\eta\|_{L^2_\by}+\|g\|_{L^2_\by}\big)\|\partial_t\eta\|_{W^{3/2,2}_\by}\dt
\\
&\leq\kappa\int_{I_\ast}\big(\|\partial_t\bu+\bu\cdot\nabla\bu\|_{L^{2}_\bx}^2+\|\bff\|_{L^2_\bx}^2+\|\partial_t^2\eta\|_{L^2_\by}^2+\|g\|_{L^2_\by}^2\big)\dt+c(\kappa)\int_{I_\ast}\|\partial_t\eta\|_{W^{3/2,2}_\by}^2\dt
\\
&\leq\kappa\int_{I_\ast}\big(\|\partial_t\bu+\bu\cdot\nabla\bu\|_{L^{2}_\bx}^2+\|\partial_t^2\eta\|_{L^2_\by}^2+\|\partial_t\Dely\eta\|_{L^2_\by}^2+\|\bff\|_{L^2_\bx}^2\big)\dt
\\
&\quad+c(\kappa)\int_{I_\ast}\|\partial_t\eta\|_{W^{1,2}_\by}^2\dt+c(\kappa)\int_{I_\ast}\|g\|_{L^{2}_\by}^2\dt,
\end{align*}
where we used the interpolation \eqref{interpoeta} and \eqref{equivNorm} in the last step.
Hence we derive from \eqref{eq:apriorieta} and Gr\"onwall inequality that
\begin{align}\label{eq:reg3}
\begin{aligned}
&\sup_{I_\ast}\int_\omega|\partial_t\naby\eta|^2\dy+\int_{I_\ast}\int_\omega|\partial_t\Dely\eta|^2\dy\dt+\sup_{I_\ast}\int_\omega|\naby\Dely\eta|^2\dy\\&\leq\kappa\int_{I_\ast}\big(\|\partial_t\bu+\bu\cdot\nabla\bu\|_{L^{2}_\bx}^2 +\|\partial_t^2\eta\|_{L^2_\by}^2\big)\dt+ c(\kappa)\tilde C_0,
\end{aligned}
\end{align}
where 
\begin{align*}
\tilde C_0=C_0+\int_\omega|\naby\eta_*|^2\dy+\int_\omega|\naby\Dely\eta_0|^2\dy.
\end{align*}
Combining \eqref{eq:reg2} and \eqref{eq:reg3}, we arrive at 
\begin{align*}
&\sup_{I_\ast}\int_{\Omega_\eta}|\nabla\bu|^2\dx+\int_{I_\ast}\int_{\Omega_\eta}|\partial_t\bu+\bu\cdot\nabla\bu|^2\dx\dt
+\int_{I_\ast}\int_\omega|\partial_t^2\eta|^2\dy\dt\\
&\quad+\sup_{I_\ast}\int_\omega|\partial_t\naby\eta|^2\dy+\int_{I_\ast}\int_\omega|\partial_t\Dely\eta|^2\dy\dt+\sup_{I_\ast}\int_\omega|\naby\Dely\eta|^2\dy
\\
&\lesssim \int_{I_\ast}
\|\bu\|_{L^s_\bx}^{\frac{2s}{s-3}}\|\nabx\bu\|^{2}_{L^{2}_\bx}\dt+
\int_{I_\ast}\|\partial_t\eta\|_{W^{1,2}_\by}^2\|\nabla\bu\|_{L^2_\bx}^2\dt+C.
\end{align*}
Note that the condition \eqref{eq:regu} and \eqref{eq:apriorieta} imply that $\int_{I_\ast}\big(\|\bu\|_{L^s_\bx}^{\frac{2s}{s-3}}+\|\partial_t\eta\|_{W^{1,2}_\by}^2\big)\dt\leq\,c$ with a constant $c$ depending on $C_1$. Therefore, we obtain from Gr\"onwall's lemma that
\begin{align}\label{eq:reg4}
\begin{aligned}
\sup_{I_\ast}\int_{\Omega_\eta}&|\nabla\bu|^2\dx+\int_{I_\ast}\int_{\Omega_\eta}|\partial_t\bu+\bu\cdot\nabla\bu|^2\dx\dt
+\int_{I_\ast}\int_\omega|\partial_t^2\eta|^2\dy\dt\leq\,c,\\
\sup_{I_\ast}\int_\omega&\big(|\partial_t\naby\eta|^2
+
|\naby\Dely\eta|^2
\big)\dy
+
\int_{I_\ast}\int_\omega|\partial_t\Dely\eta|^2\dy\dt \leq c.
\end{aligned}
\end{align}
We can now use the momentum equation and \eqref{eq:regstokes} again to obtain (recall \eqref{again})
\begin{align}\label{eq:reg5}
\begin{aligned}
&\int_{I_\ast}\int_{\Omega_\eta}|\nabla^2\bu|^2\dx\dt+\int_{I_\ast}\int_{\Omega_\eta}|\nabla\pi|^2\dx\dt\\&\leq\,c\int_{I_\ast}\int_{\Omega_\eta}|\partial_t\bu+\bu\cdot\nabla\bu|^2\dx\dt+\int_{I_*}\|\bff\|_{L^2_\bx}^2\dt
+\int_{I_\ast}\|\partial_t\eta\|_{W^{3/2,2}_\by}^2\dt\leq\,c.
\end{aligned}
\end{align}
At this point, we notice that the only term required to obtain \eqref{est:reg} is a uniform-in-time bound for $\int_{I_\ast}\int_{\Omega_\eta}|\partial_t\bu|^2\dx\dt$. Since by \eqref{eq:reg5}, all the terms on the right-hand side of the momentum equation \eqref{2} are squared integrable in space-time, our desired estimate follows once we show that the convective term $\bu\cdot\nabx\bu$ is also squared integrable in space-time. Note that a bound for the sum $\partial_t\bv+\bu\cdot\nabx\bu$ as given in \eqref{eq:reg5} does not directly yield a bound for $\bu\cdot\nabx\bu$. Nevertheless, combining \eqref{eq:reg4}, \eqref{eq:reg5} and Sobolev embedding, we thereby obtain that
\begin{align*}
\int_{I_\ast}&\int_{\Omega_\eta}|\bu\cdot\nabla\bu|^2\dx\dt
\lesssim
\int_{I_\ast}\Vert \bu\Vert_{L^4_\bx}^2\Vert\nabx \bu\Vert_{L^4_\bx}^2\dt
\lesssim
\sup_{I_\ast}\Vert \bu\Vert_{W^{1,2}_\bx}^2\int_{I_\ast}\Vert\nabx \bu\Vert_{W^{1,2}_\bx}^2\leq c,
\end{align*}
which completes the proof.
\end{proof}

\begin{remark}
{\rm 
We remark here that the conditions we proposed in Theorem \ref{prop2} is the minimal assumption for the conditional strong solution for the fluid-structure system \eqref{1}--\eqref{interfaceCond}. The Serrin condition for the velocity of the fluid \eqref{eq:regu} is crucial in the estimate of the convective term and the Lipschitz condition for the structure \eqref{eq:regeta} plays an important role in the steady Stokes estimate related to the pressure estimate.
}
\end{remark}



\section{Weak-strong uniqueness}
\label{sec:weakStrong}
In this section, we are interested in the weak-strong uniqueness of the solutions for 
the fluid-structure interaction system \eqref{1}--\eqref{interfaceCond}. 
We aim to compare two solutions
$(\bu_1,\eta_1)$ and $(\bu_2,\eta_2)$, where $(\bu_1,\eta_1)$ is a weak solution satisfying \eqref{eq:regeta}  and $(\bu_2,\eta_2)$ is a strong solution, i.e. satisfies \eqref{est:reg}. Since the fluid domain depends on the deformation 
of the shell, we have to transfer the {\bf strong} solution by means of a change of 
variables to the {\bf weak} domain. We transform $\bu_2$ and $\pi_2$ (note that we have a pressure for the strong solution but not for the weak one) to the domain of the weak solution (that is $\Omega_{\eta_1}$) by setting
\begin{align}
\label{mapTwoToOne}
\underline{\bu}_2:=\bu_2\circ\bfPsi_{\eta_2-\eta_1},\quad \underline\pi_2:=\pi_2\circ\bfPsi_{\eta_2-\eta_1},
\quad \underline{\mathbf f}_2:=\mathbf f_2\circ\bfPsi_{\eta_2-\eta_1},
\end{align}
where the Hanzawa transform $\bm{\Psi}_{\eta_2-\eta_1} : \Omega_{\eta_1} \rightarrow\Omega_{\eta_2}$ is defined similarly as in \eqref{map}. With this information, we are now in the position to state our main result in this section.
\begin{theorem}\label{thm:weakstrong}
Let $(\bu_1,\eta_1)$ be a weak solution of \eqref{1}--\eqref{interfaceCond} with data $(\bff_1, g_1, \eta_{0,1}, \eta_{*,1}, \bu_{0,1})$ in the sense of Definition \ref{def:weakSolution} and let $(\bu_2,\eta_2)$ be a strong solution of \eqref{1}--\eqref{interfaceCond} with data $(\bff_2, g_2, \eta_{0,2}, \eta_{*,2}, \bu_{0,2})$ in the sense of Definition \ref{def:strongSolution}.
Suppose further that
%\todo{weak solution is pressure-free! Introduce pressure!}
\begin{equation*}\label{eq:regeta'}
\eta_1\in L^\infty(I;C^{0,1}(\omega))
\end{equation*}
and define 
$\underline{\bu}_2$ and $\underline{\mathbf f}_2$ in accordance with \eqref{mapTwoToOne}. Assume that
$\mathbf f_1,\underline{\mathbf f}_2\in L^2(I;L^2(\Omega_{\eta_1}))$ and $g_1,g_2\in L^2(I;L^2(\omega))$.
Then we have
\begin{equation}\label{final}
\begin{aligned}
&\sup_{t\in I}\int_{\Omega_{\eta_1(t)}}|\bu_1(t)-\underline{\bu}_2(t)|^2\dx+\sup_{t\in I}\int_\omega\big(|\partial_t(\eta_1-\eta_2)(t)|^2
+
|\Dely(\eta_1-\eta_2)(t)|^2\big)\dy
\\
&\qquad\qquad\qquad\qquad+\int_I
\int_{\Omega_{\eta_1(\sigma)}}|\nabla(\bu_1-\underline{\bu}_2)|^2\dx\dt 
+
\int_I\int_\omega|\partial_t\nabla_{\by}(\eta_1-\eta_2)|^2\dy\dt
\\
&\lesssim \int_{\Omega_{\eta_{0,1}}}|\bu_1(0)-\underline{\bu}_2(0)|^2\dx+\int_\omega|\partial_t(\eta_1-\eta_2)(0)|^2\dy+\int_\omega|\Dely(\eta_1-\eta_2)(0)|^2\dy
\\
&\qquad\qquad\qquad\qquad +\int_I\int_{\Omega_{\eta_1}}| \mathbf f_1-\underline {\mathbf f}_2|^2\dx\dt+\int_I\int_\omega| g_1-g_2|^2\dy\dt.
\end{aligned}
\end{equation}
\end{theorem}
\begin{proof}
It turned out more suitable to perform the uniqueness and stability analysis on the weaker geometry given by $\eta_1$. We therefore transfer the strong solution $(\eta_2,v_2)$ to the geometry given by $\eta_1$.
With the transformation \eqref{mapTwoToOne} in hand, we obtain the equations for $(\underline \bu_2, \eta_2)$ in $\Omega_{\eta_1}$ as follows:
\begin{align}
\label{contEqAloneBar'}
&\mathbf{B}_{\eta_2-\eta_1}:\nabx \underline{\bu}_2= 0,
\\
&\partial_t^2\eta_2 - \partial_t\Dely \eta_2 + \Dely^2\eta_2
=
g_2-\bn^\intercal \big[\mathbf{A}_{\eta_2-\eta_1}\nabx \underline{\bu}_2
-\mathbf{B}_{\eta_2-\eta_1} \underline{\pi}_2\big]\circ\bm{\varphi}_{\eta_1}\bn_{\eta_1} ,
\label{shellEqAloneBar'}
\\
&\partial_t \underline{\bu}_2  -\Delx\underline{\bu}_2
 +\nabla\underline{\pi}_2 
 = 
\mathbf{h}_{12}(\underline{\bu}_2)+
\divx  \big[\big(\mathbf{A}_{\eta_2-\eta_1}-\mathbb I_{3\times 3}\big)\nabx \underline{\bu}_2
+\big(\mathbb I_{3\times 3}-\mathbf{B}_{\eta_2-\eta_1} \big)\underline{\pi}_2\big],
\label{momEqAloneBar'}
\end{align}
where
\begin{align*}
%\mathbf{H}_{\eta_1-\eta_2}(\underline{\bu}_2, \underline{\pi}_2)
%=\
%\mathbf{A}_{\eta_1-\eta_2}\nabx \underline{\bu}_2
%-\mathbf{B}_{\eta_1-\eta_2} \underline{\pi}_2,
%\\
\mathbf{h}_{12}(\underline{\bu}_2)
&=
(\mathbb I_{3\times 3}-J_{\eta_2-\eta_1})\partial_t \underline{\bu}_2
-
 J_{\eta_2-\eta_1}\nabx \underline{\bu}_2 \partial_t \bfPsi_{\eta_2-\eta_1}^{-1}\circ \bfPsi_{\eta_2-\eta_1} 
-
\mathbf B_{\eta_2-\eta_1}\nabx \underline{\bu}_2  \underline{\bu}_2
+
J_{\eta_2-\eta_1}  \underline{\mathbf f}_2,
\end{align*}
and the matrices $\mathbf{A}_{\eta_2-\eta_1}$ and $\mathbf{B}_{\eta_2-\eta_1}$ are similarly defined as in Subsection \ref{Sectionlinear} by replacing the subscript $\eta$ by $\eta_2-\eta_1$, respectively. 

Next we introduce a suitable bogovskij operator for our setting:
\[
\Bog_{\eta_1}(f):=\Bog(f\chi_{\Omega_{\eta_1}}),
\] 
where $\Bog$ is defined in Theorem~\ref{thm:ndBog}, depending on $\norm{\eta_1}_\infty=:L$ and $\norm{\nabla\eta_1}_\infty=:C_L$ only. Please note that this is the point where the additional Lipschitz assumption of the weak solution is crucially needed.

To obtain the difference estimate, we would like to test the equation for $(\bu_1-\underline\bu_2,\eta_1-\eta_2)$ by the pair $(\bu_1-\underline\bu_2+\mathrm{Bog}_{\eta_1}(\Div\underline\bu_2),\partial_t(\eta_1-\eta_2))$.
%\footnote{Please note that in $\Omega_{\eta_1}$, $\Div(\underline\bu_2)=\Div(\underline\bu_2-\bu_1)$, which possesses a zero mean value, as $}
 However, $\bu_1$ is not smooth enough to qualify as a test function for the weak equation. We thus consider the following procedure:
In the first step, we use the energy inequality for $(\bu_1,\eta_1)$, that is
\begin{equation*}
\begin{aligned}
\label{energyEst'}
&\tfrac{1}{2}\int_\omega\vert\partial_t\eta_1 \vert^2\dy+\tfrac{1}{2}\int_\omega\vert\Dely\eta_1 \vert^2\dy
+
\int_0^t\int_\omega\vert\partial_t\naby\eta_1 \vert^2\dy\ds
\\&\quad+ \tfrac{1}{2} \int_{\Omega_{\eta_1(t)}}\vert\bu_1 \vert^2 \dx+
\int_0^t
 \int_{\Omega_{\eta_1(\sigma)}}\vert \nabla \bu_1 \vert^2 \dx\ds
\\& \leq \tfrac{1}{2}\int_\omega\vert\partial_t\eta_1(0) \vert^2\dy
 +\tfrac{1}{2}\int_\omega\vert\Dely\eta_1(0) \vert^2\dy+\tfrac{1}{2} \int_{\Omega_{\eta_1(0)}}\vert\bu_1(0) \vert^2 \dx\\
&\quad+
\int_0^t \int_\omega  g_1\partial_t\eta_1\dy\ds
 +\int_0^t \int_{\Omega_{\eta_1(\sigma)}} \bu_1\cdot \mathbf{f}_1 \dx\ds .
\end{aligned}
\end{equation*}
Next observe that
\begin{align*}
\int_0^t\int_\omega\Dely\eta_1\,\partial_t\Dely \eta_2\dy\ds
=-\int_0^t\int_\omega\partial_t\eta_1\cdot\Dely^2 \eta_2\dy\ds + 
\left[\int_\omega\Dely\eta_1\Dely \eta_2\dy\right]^{\sigma=t}_{\sigma=0},
\end{align*}
an identity that can be rigorously shown by using convolution  in space. This implies, by testing the equation for $(\bu_1,\eta_1)$ with  $(-\underline \bu_2+\mathrm{Bog}_{\eta_1}(\Div\underline\bu_2),-\partial_t\eta_2)$, that 
\begin{align*}
&\int_\omega \left(-\Dely\eta_1(t)\Dely\eta_2(t)-\partial_t \eta_1(t) \, \partial_t\eta_2(t)\right)\dy +\int_\omega \Dely\eta_1(0)\Dely\eta_2(0)\dy
\\&\quad-\int_{\Omega_{\eta_1(t)}} \bu_1(t)  
\cdot (\underline\bu_2(t)-\mathrm{Bog}_{\eta_1}(\Div\underline\bu_2(t)))\dx
\\
&\quad+
\int_0^t \int_\omega \big(\partial_t \eta_1\, \partial_t^2 \eta_2-\partial_t\naby\eta_1\cdot\partial_t\naby  \eta_2+
 g_1\, \partial_t\eta_2+\partial_t\eta_1\Dely^2 \eta_2 \big)\dy\ds
 \\
 &=
-\int_\omega \partial_t \eta_1(t) \, \partial_t\eta_2(t)\dy
-\int_{\Omega_{\eta_1(t)}} \bu_1(t)  
\cdot (\underline\bu_2(t)-\mathrm{Bog}_{\eta_1}(\Div\underline\bu_2(t)))\dx
\\
&\quad+
\int_0^t \int_\omega \big(\partial_t \eta_1\, \partial_t^2 \eta_2-\partial_t\naby\eta_1\cdot\partial_t\naby  \eta_2+
 g_1\, \partial_t\eta_2-\Dely\eta_1\,\partial_t\Dely \eta_2 \big)\dy\ds.
\\
&=-\int_\omega \partial_t \eta_1(0) \, \partial_t\eta_2(0) \dy
+
\int_{\Omega_{\eta_1(0)}} \bu_1(0)  \cdot (-\underline \bu_2(0)+\mathrm{Bog}_{\eta_1}(\Div\underline\bu_2(0)))\dx
\\
&\quad+\int_0^t  \int_{\Omega_{\eta_1(\sigma)}} \big(  \bu_1\cdot \partial_t (-\underline\bu_2+\mathrm{Bog}_{\eta_1}(\Div\underline\bu_2)) 
 \big) + \bu_1 \otimes \bu_1: \nabla  \mathrm{Bog}_{\eta_1}(\Div\underline\bu_2))
 \big) \dx\ds\\
&\quad+\int_0^t  \int_{\Omega_{\eta_1(\sigma)}}(\bu_1\cdot\nabla)\bu_1\cdot\underline\bu_2\dx\ds-\int_0^t  \int_{\partial\Omega_{\eta_1(\sigma)}} |\bu_1|^2\bfn\cdot\partial_t\eta_2 \bn_{\eta_1 }\circ\bfvarphi^{-1}_ {\eta_1} \dd\mathcal H^2\ds
%\\&
%-\int_I  \int_{\Oeta}\big(   
%\nabla \bu:\nabla {\bfphi} -\bff\cdot{\bfphi} \big) \dx\dt
\\
&\quad+\int_0^t  \int_{\Omega_{\eta_1(\sigma)}} \big( 
 -  
\nabla \bu_1:\nabla  (-\underline\bu_2+\mathrm{Bog}_{\eta_1}(\Div\underline\bu_2))+\bff_1\cdot (-\underline\bu_2+\mathrm{Bog}_{\eta_1}(\Div\underline\bu_2)) \big) \dx\ds.
\end{align*}
Finally, we multiply the (strong) equation for $(-\underline\bu_2,-\eta_2)$ by $(\bu_1-\underline\bu_2+\mathrm{Bog}_{\eta_1}(\Div\underline\bu_2),\partial_t(\eta_1-\eta_2))$. This implies after integration by parts
\begin{align*}
&\int_\omega \left(\frac{\abs{\partial_t \eta_2(t)}^2}{2}+\frac{\abs{\Dely\eta_2(t)}^2}{2}\right)\dy -\int_0^t\int_\omega \partial_t\naby\eta_2\cdot\partial_t\naby(\eta_1-\eta_2)\dy\dt
+\tfrac{1}{2} \int_{\Omega_{\eta_1(t)}}|\underline\bu_2 |^2 \dx
\\
&\quad-\tfrac{1}{2}  \int_0^t\int_{\partial\Omega_{\eta_1(\sigma)}}\bfn\circ\bfvarphi^{-1}_ {\eta_1}\cdot\partial_t\eta_1 \bn_{\eta_1 }\circ\bfvarphi^{-1}_ {\eta_1}|\underline\bu_2 |^2 \dd \mathcal H^2\dd \sigma
-\int_0^t\int_{\Omega_{\eta_1(\sigma)}}\partial_t\underline\bu_2 \cdot\bu_1 \dx\ds
\\
&\quad+ \int_0^t\int_{\Omega_{\eta_1(\sigma)}}\nabla \underline\bu_2:\nabla(\underline\bu_2-\bu_1)\dx\ds
%-\int_0^t\int_\omega \partial_t^2 \eta_2 \partial_t\eta_1\dy\dt-\int_0^t\int_\omega \Dely^2 \eta_2 \partial_t\eta_1\dy\dt
\\
&=\tfrac{1}{2} \int_{\Omega_{\eta_1(0)}}|\underline\bu_2(0) |^2 \dx+\int_\omega \left(\frac{\abs{\partial_t \eta_2(0)}^2}{2}+\frac{\abs{\Dely\eta_2(0)}^2}{2}\right)\dy 
%-\int_\omega\partial_t\eta_2(0))\,\partial_t(\eta_1-\eta_2)(0)\dy
\\
&\quad-\int_0^t\int_{\Omega_{\eta_1(\sigma)}}\mathbf h_{12}(\underline\bu_2) \cdot\big(\bu_1- \underline\bu_2+ \mathrm{Bog}_{\eta_1}(\Div\underline\bu_2)\big)\dx\ds
\\
&\quad-
\int_0^t\int_{\Omega_{\eta_1(\sigma)}}\underline\bu_2 \cdot \partial_t\mathrm{Bog}_{\eta_1}(\Div\underline\bu_2)\dx\ds+
\int_{\Omega_{\eta_1}}\underline\bu_2 \cdot \mathrm{Bog}_{\eta_1}(\Div\underline\bu_2)\dx
\\
%&-{\color{blue}{\int_0^t\int_ {\partial\Omega_{\eta_1(\sigma)}}\underline\bu_2\cdot\mathrm{Bog}_{\eta_1}(\Div\underline\bu_2)\bn_{\eta_1}\partial_t\eta_1\bn\circ\bfvarphi_{\eta_1}^{-1}\,\dd \mathcal H^2\dd \sigma}}\\
&\quad-
\int_{\Omega_{\eta_1(0)}}\underline\bu_2(0) \cdot \mathrm{Bog}_{\eta_1(0)}(\Div\underline\bu_2(0))\dx
+ \int_0^t\int_{\Omega_{\eta_1(\sigma)}}\nabla \underline\bu_2:\nabla\mathrm{Bog}_{\eta_1}(\Div\underline\bu_2)\dx\ds
\\
&\quad+\int_0^t\int_{\Omega_{\eta_1(\sigma)}}\big(\mathbf A_{\eta_2-\eta_1}-\mathbb I_{3\times 3}\big)\nabla\underline\bu_2:\nabla(\bu_1-\underline \bu_2+\mathrm{Bog}_{\eta_1}(\Div\underline\bu_2))\dx\ds\\
&\quad+\int_0^t\int_{\Omega_{\eta_1(\sigma)}}\big(\mathbb I_{3\times 3}-\mathbf B_{\eta_2-\eta_1}\big)\underline \pi_2:\nabla(\bu_1-\underline \bu_2+\mathrm{Bog}_{\eta_1}(\Div\underline\bu_2))\dx\ds\\
&\quad+\int_0^t\int_\omega\partial_t^2\eta_2~\partial_t\eta_1\dy\ds-\int_0^t\int_\omega\naby\Dely\eta_2~\partial_t\naby\eta_1\dy\ds-\int_0^t\int_\omega g_2\partial_t(\eta_1-\eta_2)\dy\ds\\
&\quad +\int_0^t\int_\omega\partial_t\eta_2~\partial_t\eta_1\dy\ds-\int_\omega\partial_t\eta_2(t)~\partial_t\eta_1(t)\dy.
%\\
%&\textcolor{blue}{\quad+\int_0^t \int_\omega \big(\partial^2_t \eta_2\, \partial_t(\eta_1-\eta_2)+\partial_t\naby\eta_2\cdot \partial_t\naby(\eta_1-\eta_2) \big)\dy\ds}\\
%&\textcolor{blue}{\quad-\int_\omega \partial_t \eta_2(t)\, \partial_t(\eta_1-\eta_2)(t)\dy+ \int_\omega \partial_t \eta_2(0)\, \partial_t(\eta_1-\eta_2)(0)\dy}\\
%%&\quad-\int_0^t \int_\omega \big(
%% g_2\, \partial_t(\eta_1-\eta_2)-\Dely\eta_2\, \partial_t\Dely(\eta_1-\eta_2) \big)\dy\ds\\
%&\textcolor{blue}{\quad-\int_0^t \int_\omega \big(
% g_2\, \partial_t(\eta_1-\eta_2)+\partial_t\Dely\eta_2\, \Dely(\eta_1-\eta_2) \big)\dy\ds}\\
%&\textcolor{blue}{\quad+\int_\omega\Dely\eta_2(t)\, \Dely(\eta_1-\eta_2)(t) \dy-\int_\omega\Dely\eta_2(0)\, \Dely(\eta_1-\eta_2)(0) \dy.}
\end{align*}

%\seb{It can be seen that only some terms involving $\partial_t\eta_1$ are not well defined here. This can be overcome by decomposing
%$\bu_1=(\bu_1-\Test_{\eta_1}(\partial_t\eta_1))+\Test(\partial_t\eta_1)$. Now taking $(\eta_1)_\epsilon$ as a standard mollification of $\eta_1$ allows to rigorously use
%$((\eta_1)_\epsilon, \Test_{\eta_1}((\eta_1)_\epsilon))$ and $(0,(\bu_1-\Test_{\eta_1}(\partial_t\eta_1)))$ as test functions.
%In this sense one can then rewrite the last two lines of the previous equality as
%\begin{align*}
%-\int_0^t \int_\omega \big(\partial_t \eta_2\, \partial_t^2(\eta_1-\eta_2)-\partial_t\naby\eta_2\cdot \partial_t\naby(\eta_1-\eta_2) \big)\dy\ds\\
%-\int_0^t \int_\omega \big(
% g_2\, \partial_t(\eta_1-\eta_2)-\Dely\eta_2\, \partial_t\Dely(\eta_1-\eta_2) \big)\dy\ds.
%\end{align*}
% and obtain} 
Combining the above 
%using that for any two functions $a_1,a_2$
%\begin{align*}
%\int_\omega\abs{a_1-a_2}^2\dy%\Big|^t_0
%=
%\int_\omega \abs{a_1}^2-2a_2a_1+\abs{a_2}^2\dy%\Big|^t_0
%\end{align*}
we find that
\begin{align}
&\tfrac{1}{2}\int_{\Omega_{\eta_1(t)}}|\bu_1(t)-\underline{\bu}_2(t)|^2\dx+\int_0^t
\int_{\Omega_{\eta_1(\sigma)}}|\nabla(\bu_1-\underline{\bu}_2)|^2\dx\ds\nonumber\\
&\quad+\tfrac{1}{2}\int_\omega|\partial_t(\eta_1-\eta_2)(t)|^2\dy+\int_0^t\int_\omega|\partial_t\nabla_{\by}(\eta_1-\eta_2)|^2\dy\ds+\tfrac{1}{2}\int_\omega|\Dely(\eta_1-\eta_2)(t)|^2\dy\nonumber\\
&\leq 
\tfrac{1}{2}\int_{\Omega_{\eta_1(0)}}|\bu_1(0)-\underline{\bu}_2(0)|^2\dx+\tfrac{1}{2}\int_\omega|\partial_t(\eta_1-\eta_2)(0)|^2\dy+\tfrac{1}{2}\int_\omega|\Dely(\eta_1-\eta_2)(0)|^2\dy\nonumber\\
&\quad-\int_0^t\int_{\Omega_{\eta_1(\sigma)}}\big((\mathbb I_{3\times 3}-J_{\eta_2-\eta_1})\partial_t \underline{\bu}_2\big)\cdot\big(\bu_1- \underline\bu_2+ \mathrm{Bog}_{\eta_1}(\Div\underline\bu_2)\big)\dx\ds\nonumber\\
&\quad+\int_0^t\int_{\Omega_{\eta_1(\sigma)}}\big(
 J_{\eta_2-\eta_1}\nabx \underline{\bu}_2 \partial_t \bfPsi_{\eta_2-\eta_1}^{-1}\circ \bfPsi_{\eta_2-\eta_1}\big)\cdot\big(\bu_1- \underline\bu_2+ \mathrm{Bog}_{\eta_1}(\Div\underline\bu_2)\big)\dx\ds\nonumber\\
&\quad+\int_0^t\int_{\Omega_{\eta_1(\sigma)}}\nabx \underline{\bu}_2  \underline{\bu}_2\cdot\big(\bu_1- \underline\bu_2+ \mathrm{Bog}_{\eta_1}(\Div\underline\bu_2)\big)\dx\ds\nonumber
%+{\color{blue}{\int_0^t\int_{\Omega_{\eta_1(\sigma)}}(\bu_1 \cdot\nabla) \bu_1\cdot (\underline\bu_2-\mathrm{Bog}_{\eta_1}(\Div\underline\bu_2)) \big) \dx\ds}}
\\
&\quad+\tfrac{1}{2} \int_0^t\int_{\partial\Omega_{\eta_1(\sigma)}}\bfn\circ\bfvarphi^{-1}_ {\eta_1}\cdot\partial_t\eta_1 \bn_{\eta_1 }\circ\bfvarphi^{-1}_ {\eta_1}|\underline\bu_2 |^2 \dd \mathcal H^2\ds\nonumber\\
&\quad- \int_0^t\int_{\partial\Omega_{\eta_1(\sigma)}}\bfn\circ\bfvarphi^{-1}_ {\eta_1}\cdot\partial_t\eta_2 \bn_{\eta_1 }\circ\bfvarphi^{-1}_ {\eta_1}| \bu_1 |^2 \dd \mathcal H^2\ds\nonumber
\\
&\quad
+\int_0^t\int_{\Omega_{\eta_1(\sigma)}}\big( 
\mathbf B_{\eta_2-\eta_1}- \mathbb I
_{3\times 3}\big)\nabx \underline{\bu}_2  \underline{\bu}_2\cdot\big(\bu_1- \underline\bu_2+ \mathrm{Bog}_{\eta_1}(\Div\underline\bu_2)\big)\dx\ds\nonumber\\
&\quad+
\int_0^t\int_{\Omega_{\eta_1(\sigma)}}(\bu_1-\underline\bu_2) \cdot \partial_t\mathrm{Bog}_{\eta_1}(\Div\underline\bu_2)\dx\ds-
\int_{\Omega_{\eta_1(t)}}(\bu_1-\underline\bu_2) \cdot \mathrm{Bog}_{\eta_1}(\Div\underline\bu_2)\dx\nonumber\\
&\quad+
\int_{\Omega_{\eta_1(0)}}(\bu_1-\underline\bu_2)(0) \cdot \mathrm{Bog}_{\eta_1(0)}(\Div\underline\bu_2(0))\dx
- \int_0^t\int_{\Omega_{\eta_1(\sigma)}}\nabla(\bu_1- \underline\bu_2):\nabla\mathrm{Bog}_{\eta_1}(\Div\underline\bu_2)\dx\ds\nonumber
\\
&\quad+\int_0^t\int_{\Omega_{\eta_1(\sigma)}}\big(\mathbf A_{\eta_2-\eta_1}-\mathbb I_{3\times 3}\big)\nabla\underline\bu_2:\nabla(\bu_1-\underline \bu_2+\mathrm{Bog}_{\eta_1}(\Div\underline\bu_2))\dx\ds\nonumber\\
&\quad+\int_0^t\int_{\Omega_{\eta_1(\sigma)}}\big(\mathbb I_{3\times 3}-\mathbf B_{\eta_2-\eta_1}\big)\underline \pi_2:\nabla(\bu_1-\underline \bu_2+\mathrm{Bog}_{\eta_1}(\Div\underline\bu_2))\dx\ds\nonumber\\
&\quad+\int_0^t \int_\omega ( g_1-g_2)\partial_t(\eta_1-\eta_2)\dy\ds
 +\int_0^t \int_{\Omega_{\eta_1(\sigma)}}  (\bu_1-\underline \bu_2)\cdot (\mathbf{f}_1-\underline{\mathbf f}_2)\dx\ds\nonumber\\
 &\quad+\int_0^t \int_{\Omega_{\eta_1(\sigma)}} (\mathbb I_{3\times 3}-J_{ \eta_2-\eta_1})\underline{\mathbf f}_2\cdot (\bu_1-\underline \bu_2)\dx\ds+\int_0^t\int_{\Omega_{\eta_1}}(\mathbb I_{3\times 3}-J_{\eta_2-\eta_1})\underline{\mathbf f}_2\mathrm{Bog}_{\eta_1}(\Div \underline \bu_2)\dx\ds\nonumber\\
 &\quad+\int_0^t\int_{\Omega_{\eta_1(\sigma)}}\bu_1\otimes \bu_1:\nabla\mathrm{Bog}_{\eta_1}(\Div\underline \bu_2)\dx\ds+\int_0^t\int_{\Omega_{\eta_1(\sigma)}}\bu_1\cdot\nabla\bu_1\cdot\underline \bu_2\dx\ds\nonumber\\
 &\quad+\int_0^t\int_{\Omega_{\eta_1(\sigma)}}(\mathbf f_1-\underline{\mathbf f}_2)\mathrm{Bog}_{\eta_1}(\Div\underline \bu_2)\dx\ds\nonumber\\
 %&-{\color{blue}{\int_0^t\int_{\partial\Omega_{\eta_1}}\underline \bu_2\cdot \mathrm{Bog}_{\eta_1}(\Div\underline \bu_2)\partial_t\eta_1\cdot \bn_{\eta_1}\cdot \bn\circ\bfvarphi_{\eta_1}^{-1}\dd \mathcal H^2\ds}}\\
 &=:\sum_{i=1}^{22} R_{i}.\label{mainineq}
\end{align}
Note that $R_1$, $R_2$ and $R_3$ are in good form and so we start the estimate for the remaining terms on the right side of \eqref{mainineq}.
For the terms including the forcing terms, we estimate as follows:
\begin{align*}
&R_{12}+R_{16}+R_{17}+R_{18}+R_{19}+R_{22}\\
&\leq
 \delta\int_0^t\left(\lVert\bu_1-\underline \bu_2\rVert_{L^2(\Omega_{\eta_1})}^2+\lVert\nabla(\bu_1-\underline \bu_2)\rVert_{L^2(\Omega_{\eta_1})}^2 +\lVert\partial_t\eta_1-\partial_t\eta_2\rVert_{L^2(\omega)}^2\right)\ds\\
 &\quad+C(\delta)\int_0^t\lVert\underline {\mathbf f}_2\rVert_{L^2(\Omega_{\eta_1})}^2\lVert\eta_1-\eta_2\rVert_{W^{2,2}(\omega)}^2\ds+C(\delta)\int_0^t\left(\lVert g_1-g_2\rVert_{L^2(\omega)}^2+\lVert \mathbf f_1-\underline {\mathbf f}_2\rVert_{L^2(\Omega_{\eta_1})}^2\right)\ds
 \\
 &\quad+C(\delta)\lVert \bu_1(0)-\underline \bu_2(0)\rVert_{L^2(\Omega_{\eta_1(0)})}^2.
\end{align*}

Using the fact that $\mathbb I_{3\times 3}-J_{\eta_2-\eta_1}\sim \nabla_{\by}(\eta_2-\eta_1)$ we estimate $R_4$:
\begin{align*}%\label{415}
R_4&\lesssim 
\int_0^t\lVert\mathbb I_{3\times 3}-J_{\eta_2-\eta_1}\rVert_{L^4(\Omega_{\eta_1(\sigma)})}\lVert\partial_t\underline \bu_2\rVert_{L^2(\Omega_{\eta_1(\sigma)})}\lVert\bu_1- \underline\bu_2+ \mathrm{Bog}_{\eta_1}(\Div\underline\bu_2)\rVert_{L^4(\Omega_{\eta_1(\sigma)})}\ds\\
&\lesssim \int_0^t\lVert\mathbb I_{3\times 3}-J_{\eta_2-\eta_1}\rVert_{W^{1,2}(\Omega_{\eta_1(\sigma)})}\lVert\partial_t\underline \bu_2\rVert_{L^2(\Omega_{\eta_1(\sigma)})}\lVert\bu_1- \underline\bu_2+ \mathrm{Bog}_{\eta_1}(\Div\underline\bu_2)\rVert_{W^{1,2}(\Omega_{\eta_1(\sigma)})} \ds \\
&\leq \delta\int_0^t\lVert\nabla(\bu_1-\underline \bu_2)\rVert_{L^2(\Omega_{\eta_1(\sigma)})}^2\ds +C(\delta) \int_0^t\lVert\partial_t\underline \bu_2\rVert_{L^2(\Omega_{\eta_1(\sigma)})}^2\lVert\eta_1-\eta_2\rVert_{W^{2,2}(\omega)}^2\ds,
\end{align*}
where we used the estimate:
\begin{align*}
\lVert \mathrm{Bog}_{\eta_1}(\Div\underline \bu_2)\rVert_{W^{1,2}(\Omega_{\eta_1})}=\lVert \mathrm{Bog}_{\eta_1}(\Div(\underline \bu_2-\bu_1))\rVert_{W^{1,2}(\Omega_{\eta_1})}\lesssim \lVert\bu_1-\underline \bu_2\rVert_{W^{1,2}(\Omega_{\eta_1})}.
\end{align*}
According to the properties of the map $\bfPsi_{\eta_2-\eta_1}$ discussed in Section \ref{ssec:geom}, we continue estimating $R_5$:
\begin{align*}%\label{R5}
R_5
& \lesssim \int_0^t\lVert\nabla\underline \bu_2\rVert_{L^4(\Omega_{\eta_1(\sigma)})}\lVert\partial_t\bfPsi_{\eta_2-\eta_1}^{-1}\circ\bfPsi_{\eta_2-\eta_1}\rVert_{L^4(\Omega_{\eta_1(\sigma)})}\lVert\bu_1-\underline \bu_2\rVert_{L^2(\Omega_{\eta_1(\sigma)})}\ds\\
&\lesssim \int_0^t\lVert \underline \bu_2\rVert_{W^{2,2}(\Omega_{\eta_1(\sigma)})}\lVert\partial_t(\eta_1-\eta_2)\rVert_{L^4(\omega)}\lVert\bu_1-\underline \bu_2\rVert_{L^2(\Omega_{\eta_1(\sigma)})}\ds\\
&\leq\delta\int_0^t\lVert\partial_t\nabla_{\by}(\eta_1-\eta_2)\rVert_{L^2(\omega)}^2\ds+C(\delta)\int_0^t\lVert\underline \bu_2\rVert_{W^{2,2}(\Omega_{\eta_1(\sigma)})}^2\lVert\bu_1-\underline \bu_2\rVert_{L^2(\Omega_{\eta_1(\sigma)})}^2\ds.
\end{align*}
For the $R_9$ term, we use the fact that $\Vert\nabx\underline{\bv}_2\Vert_{L^2_\bx}$ is essentially bounded in time and $\mathbb I_{3\times 3}-\mathbf B_{\eta_1-\eta_2}\sim \nabla_{\by}(\eta_1-\eta_2)$ to obtain
\begin{align*}
R_9
&\lesssim\int_0^t\Vert I_{3\times 3}-\mathbf B_{\eta_1-\eta_2}\Vert_{L^4(\Omega_{\eta_1(\sigma)})}\Vert\underline{\bv}_2\Vert_{L^\infty(\Omega_{\eta_1(\sigma)})}
\Vert \bv_1-\underline{\bv}_2\Vert_{L^4(\Omega_{\eta_1(\sigma)})}\ds\\
&\leq\delta\int_0^t\Vert \nabx(\bv_1-\underline{\bv}_2)\Vert_{L^2(\Omega_{\eta_1(\sigma)})}^2\ds+C(\delta)\int_0^t
\Vert\underline{\bv}_2\Vert_{W^{2,2}(\Omega_{\eta_1(\sigma)})}^2
\Vert {\eta_1-\eta_2}\Vert_{W^{2,2}(\omega)}^2
\ds.
\end{align*}
We now start the estimate for $R_{10}$. Using the properties of the Bogovskij operator (please refer to Remark \ref{time-bog}), we have 
\begin{align*}%\label{R10es}
R_{10}&\leq 
\int_0^t\lVert\bu_1-\underline \bu_2\rVert_{L^6(\Omega_{\eta_1(\sigma)})}\lVert(\mathbb I_{3\times 3}-\mathbf B_{\eta_2-\eta_1})\partial_t \nabla\underline \bu_2\rVert_{W^{-1, \frac{6}{5}}(\Omega_{\eta_1(\sigma)})}\ds\\
&\quad+\int_0^t\lVert\bu_1-\underline \bu_2\rVert_{L^6(\Omega_{\eta_1(\sigma)})}\lVert\partial_t(\mathbf B_{\eta_2-\eta_1})\nabla\underline \bu_2\rVert_{W^{-1, \frac{6}{5}}(\Omega_{\eta_1(\sigma)})}\ds\\
&\lesssim \int_0^t\lVert\bu_1-\underline \bu_2\rVert_{W^{1,2}(\Omega_{\eta_1(\sigma)})}\lVert\mathbb I_{3\times 3}-\mathbf B_{\eta_2-\eta_1}\rVert_{W^{1,2}(\Omega(\eta_1(\sigma))}\lVert\partial_t\underline \bu_2\rVert_{L^2(\Omega_{\eta_1(\sigma)})}\\
&\quad+\int_0^t\lVert\bu_1-\underline \bu_2\rVert_{W^{1,2}(\Omega_{\eta_1(\sigma)})}\lVert\partial_t(\eta_1-\eta_2)\rVert_{L^2(\omega)}\lVert\nabla\underline \bu_2\rVert_{W^{1,2}(\Omega_{\eta_1(\sigma)})}\ds\\
&\leq \delta \int_0^t\lVert\nabla(\bu_1-\underline\bu_2)\rVert_{L^2(\Omega_{\eta_1(\sigma)})}^2+C(\delta)\int_0^t\lVert\partial_t\underline \bu_2\rVert_{L^2(\Omega_{\eta_1(\sigma)})}^2\lVert\eta_1-\eta_2\rVert_{W^{2,2}(\omega)}^2\ds\\
&\quad+C(\delta)\int_0^t\lVert\underline \bu_2\rVert_{W^{2,2}(\Omega_{\eta_1(\sigma)})}^2\lVert\partial_t(\eta_1-\eta_2)\rVert_{L^2(\omega)}^2\ds.
\end{align*}
To estimate $R_{11}$, we rewrite $\Div\underline \bu_2$ as follows:
\begin{equation}\label{divv2}
\Div\underline \bu_2=\mathbb I_{3\times 3}:\nabla\underline \bu_2=(\mathbb I_{3\times 3}-\mathbf B_{\eta_2-\eta_1}):\nabla\underline \bu_2,
\end{equation}
where we take into account the divergence free condition for $\underline \bu_2$ on $\Omega_{\eta_1}$ derived in \eqref{contEqAloneBar'}.
With \eqref{divv2}, we have
\begin{align*}%\label{R11est}
R_{11}
&\leq \lVert\bu_1-\underline\bu_2\rVert_{L^2(\Omega_{\eta_1})}\lVert\mathrm{Bog}\left((\mathbb I_{3\times 3}-\mathbf B):\nabla\underline\bu_2\right)\rVert_{L^2(\Omega_{\eta_1})}\\
&\leq \lVert\bu_1-\underline\bu_2\rVert_{L^2(\Omega_{\eta_1})}\lVert(\mathbb I_{3\times 3}-\mathbf B):\nabla\underline\bu_2\rVert_{W^{-1,2}(\Omega_{\eta_1})}\\
&\leq \delta\lVert\bu_1-\underline\bu_2\rVert_{ L^2(\Omega_{\eta_1})}^2+C(\delta)\lVert(\mathbb I_{3\times 3}-\mathbf B):\nabla\underline\bu_2\rVert_{ L^{\frac{6}{5}}(\Omega_{\eta_1})}^2\\
&\leq \delta\lVert\bu_1-\underline\bu_2\rVert_{ L^2(\Omega_{\eta_1})}^2+C(\delta)\lVert\nabla\underline\bu_2\rVert_{ L^2(\Omega_{\eta_1})}^2\lVert\nabla_{\by}(\eta_1-\eta_2)\rVert_{L^3(\omega)}^2\\
&\leq \delta\lVert\bu_1-\underline\bu_2\rVert_{ L^2(\Omega_{\eta_1})}^2+C(\delta)\lVert\nabla_{\by}(\eta_1-\eta_2)\rVert_{L^2(\omega)}^{\frac{4}{3}}\lVert\nabla_{\by}(\eta_1-\eta_2)\rVert_{W^{1,2}(\omega)}^{\frac{2}{3}}\\
&\leq \delta\lVert\bu_1-\underline\bu_2\rVert_{ L^2(\Omega_{\eta_1})}^2+\varepsilon\lVert\nabla_{\by}(\eta_1-\eta_2)\rVert_{W^{1,2}(\omega)}^2+C(\delta, \varepsilon)\lVert\nabla_{\by}(\eta_1-\eta_2)\rVert_{L^2(\omega)}^2\\
&\leq \delta\lVert\bu_1-\underline\bu_2\rVert_{ L^2(\Omega_{\eta_1})}^2+\varepsilon\lVert\nabla_{\by}(\eta_1-\eta_2)\rVert_{W^{1,2}(\omega)}^2\\
&\qquad\qquad \qquad +C(\delta, \varepsilon)\lVert\nabla_{\by}(\eta_1-\eta_2)\rVert_{L^2(0, t;L^2(\omega))}\lVert\naby(\eta_1-\eta_2)\rVert_{W^{1,2}(0,t; L^2(\omega)} 
\\
&\leq \delta\lVert\bu_1-\underline\bu_2\rVert_{ L^2(\Omega_{\eta_1})}^2+\varepsilon\lVert\nabla_{\by}(\eta_1-\eta_2)\rVert_{W^{1,2}(\omega)}^2+\nu\lVert\naby(\eta_1-\eta_2)\rVert_{W^{1,2}(0, t; L^2(\omega))}^2
\\
&\qquad\qquad\qquad +C(\delta, \varepsilon, \nu)\lVert\eta_1-\eta_2\rVert_{L^2(0, t; W^{2,2}(\omega))}^2,
\end{align*}
where we used interpolation in time as follows:
\begin{align*}
\lVert\nabla_{\by}(\eta_1-\eta_2)\rVert_{L^\infty(0, t; L^2(\omega))}\lesssim \lVert\nabla_{\by}(\eta_1-\eta_2)\rVert_{L^2(0, t; L^2(\omega))}^{\frac{1}{2}}\lVert\nabla_{\by}(\eta_1-\eta_2)\rVert_{W^{1,2}(0, t; L^2(\omega))}^{\frac{1}{2}}. 
\end{align*}
Using \eqref{divv2}, we have the estimate for $R_{13}$:
\begin{align*}%\label{R13e}
R_{13}&\leq \int_0^t\lVert\nabla(\bu_1-\underline \bu_2)\rVert_{L^2(\Omega_{\eta_1(\sigma)})}\lVert\nabla\mathrm{Bog}_{\eta_1}(\Div\underline \bu_2)\rVert_{L^2(\Omega_{\eta_1(\sigma)})}\ds\\
&\lesssim\int_0^t\lVert\nabla(\bu_1-\underline \bu_2)\rVert_{L^2(\Omega_{\eta_1(\sigma)})}\lVert(\mathbb I_{3\times 3}-\mathbf B_{\eta_2-\eta_1}):\nabla\underline \bu_2\rVert_{L^2(\Omega_{\eta_1(\sigma)})}\\
&\lesssim \int_0^t\lVert\nabla(\bu_1-\underline \bu_2)\rVert_{L^2(\Omega_{\eta_1(\sigma)})}\lVert\nabla_{\by}(\eta_1-\eta_2)\rVert_{L^4(\omega)}\lVert\nabla\underline \bu_2\rVert_{L^4(\Omega_{\eta_1(\sigma)})}\\
&\leq \delta\int_0^t\lVert\nabla(\bu_1-\underline \bu_2)\rVert_{L^2(\Omega_{\eta_1(\sigma)})}^2+C(\delta)\int_0^t\lVert\underline \bu_2\rVert_{W^{2,2}(\Omega_{\eta_1(\sigma)})}^2\lVert\eta_1-\eta_2\rVert_{W^{2,2}(\omega)}^2,
\end{align*}
where we also used 
\begin{align*}
\lVert\nabla\mathrm{Bog}_{\eta_1}(\Div\underline\bu_2)\rVert_{L^2(\Omega_{\eta_1(\sigma)})}\lesssim \lVert\Div\underline \bu_2\rVert_{L^2(\Omega_{\eta_1(\sigma)})}. 
\end{align*}
Recalling the regularity $\underline \pi_2\in L^2(I,W^{1,2}(\Omega_{\eta_1}))$and $\underline \bu_2\in L^2(I,W^{2,2}(\Omega_{\eta_1}))$, we estimate $R_{14}$ and $R_{15}$ in a similar way. Notice that $\mathbb I_{3\times 3}-\mathbf B_{\eta_1-\eta_2}\sim \nabla_{\by}(\eta_1-\eta_2)$ and $\mathbf A_{\eta_2-\eta_2}-\mathbb I_{3\times 3}\sim \nabla_{\by}(\eta_1-\eta_2)$, we thus have 
\begin{equation*}\label{14est}
R_{14}\leq \delta\int_0^t\lVert\nabla(\bu_1-\underline \bu_2)\rVert_{L^2(\Omega_{\eta_1(\sigma)})}^2\ds+C(\delta)\int_0^t\lVert\underline \bu_2\rVert_{W^{2,2}(\Omega_{\eta_1(\sigma)})}^2\lVert\eta_1-\eta_2\rVert_{W^{2,2}(\omega)}^2\ds,
\end{equation*}
and 
\begin{equation*}\label{15es}
R_{15}\leq \delta\int_0^t\lVert\nabla(\bu_1-\underline \bu_2)\rVert_{L^2(\Omega_{\eta_1(\sigma)})}^2\ds +C(\delta) \int_0^t\lVert\underline \pi_2\rVert_{W^{1,2}(\Omega_{\eta_1(\sigma)})}^2\lVert\eta_1-\eta_2\rVert_{W^{2,2}(\omega)}^2\ds.
\end{equation*}
Now we deal with the estimate of $R_6$, $R_7$, $R_8$ and $R_{20}$, $R_{21}$ together. We first take an integration by part of $R_{20}$ and obtain
\begin{equation*}
\begin{aligned}
R_{20}=-\int_0^t\int_{\Omega_{\eta_1(\sigma)}}\bu_1\cdot \nabla\bu_1\cdot \mathrm{Bog}_{\eta_1}(\Div\underline \bu_2)\dx\ds,
\end{aligned}
\end{equation*}
where we do not have the boundary term since the Bogovskij operator vanishes on the boundary. We then rewrite $R_6$ in the following way:
\begin{align*}
R_6
%&=\int_0^t\int_{\Omega_{\eta_1(\sigma)}}(\underline \bu_2-\bu_1)\nabla\underline \bu_2(\bu_1-\underline\bu_2+\mathrm{Bog}_{\eta_1}(\Div\underline \bu_2))\dx\ds\\
%&+\int_0^t\int_{\Omega_{\eta_1(\sigma)}}\bu_1\nabla\underline\bu_2(\bu_1-\underline\bu_2+\mathrm{Bog}_{\eta_1}(\Div\underline \bu_2))\dx\ds\\
%&=\int_0^t\int_{\Omega_{\eta_1(\sigma)}}(\underline \bu_2-\bu_1)\nabla\underline \bu_2(\bu_1-\underline\bu_2+\mathrm{Bog}_{\eta_1}(\Div\underline \bu_2))\dx\ds\\
%&+\int_0^t\int_{\Omega_{\eta_1(\sigma)}}\bu_1\nabla(\underline\bu_2-\bu_1)(\bu_1-\underline\bu_2+\mathrm{Bog}_{\eta_1}(\Div\underline\bu_2))\dx\ds\\
%&+\int_0^t\int_{\Omega_{\eta(\sigma)}}\bu_1\nabla\bu_1(\bu_1-\underline\bu_2+\mathrm{Bog}_{\eta_1}(\Div\underline\bu_2))\dx\ds\\
&=\int_0^t\int_{\Omega_{\eta_1(\sigma)}}(\underline \bu_2-\bu_1)\nabla\underline \bu_2(\bu_1-\underline\bu_2+\mathrm{Bog}_{\eta_1}(\Div\underline \bu_2))\dx\ds\\
&\quad+\int_0^t\int_{\Omega_{\eta_1(\sigma)}}(\bu_1-\underline\bu_2)\nabla(\underline\bu_2-\bu_1)(\bu_1-\underline\bu_2+\mathrm{Bog}_{\eta_1}(\Div\underline \bu_2))\dx\ds\\
&\quad+\int_0^t\int_{\Omega_{\eta_1(\sigma)}}\underline\bu_2\nabla(\underline\bu_2-\bu_1)(\bu_1-\underline\bu_2+\mathrm{Bog}_{\eta_1}(\Div\underline \bu_2))\dx\ds\\
&\quad+\int_0^t\int_{\Omega_{\eta_1(\sigma)}}\bu_1\nabla\bu_1(\bu_1-\underline\bu_2+\mathrm{Bog}_{\eta_1}(\Div\underline\bu_2))\dx\ds\\
&=R_{6,0}+R_{6,1}+R_{6,2}+R_{6,3}.
\end{align*}
For $R_{6,0}$, we have
\begin{align*}
R_{6,0}&\leq \int_0^t\lVert\underline\bu_2-\bu_1\rVert_{L^4(\Omega_{\eta_1(\sigma)})}\lVert\nabla\underline\bu_2\rVert_{L^2(\Omega_{\eta_1(\sigma)})}\lVert\bu_1-\underline\bu_2+\mathrm{Bog}_{\eta_1}(\Div\underline\bu_2)\rVert_{L^4(\Omega_{\eta_1(\sigma)})}\ds\\
&\lesssim \int_0^t\lVert\underline\bu_2-\bu_1\rVert_{L^4(\Omega_{\eta_1(\sigma)})}^2\lVert\nabla\underline\bu_2\rVert_{L^2(\Omega_{\eta_1(\sigma)})}\ds\\
&\lesssim\int_0^t\lVert\underline\bu_2-\bu_1\rVert_{L^2(\Omega_{\eta_1(\sigma)})}^{\frac{1}{2}}\lVert\nabla(\bu_1-\underline\bu_2)\rVert_{L^2(\Omega_{\eta_1(\sigma)})}^{\frac{3}{2}}\ds\\
&\leq \delta\int_0^t\lVert\nabla(\bu_1-\underline\bu_2)\rVert_{L^2(\Omega_{\eta_1(\sigma)})}^2\ds+C(\delta)\int_0^t\lVert\bu_1-\underline\bu_2\rVert_{L^2(\Omega_{\eta_1(\sigma)})}^2\ds,
\end{align*}
where we used the fact that $\underline\bu_2\in L^\infty(I, W^{1,2}(\Omega_{\eta_1}))$ and the interpolation inequality in $3D$:
$$\lVert\underline\bu_2-\bu_1\rVert_{L^4(\Omega_{\eta_1(\sigma)})}\lesssim \lVert\underline\bu_2-\bu_1\rVert_{L^2(\Omega_{\eta_1(\sigma)})}^{\frac{1}{4}}\lVert\nabla(\bu_1-\underline\bu_2)\rVert_{L^2(\Omega_{\eta_1(\sigma)})}^{\frac{3}{4}}. $$
Taking an integration by parts with respect to space, we estimate $R_{6,1}$ as
\begin{align*}
R_{6,1}&=-\int_0^t\int_{\Omega_{\eta_1(\sigma)})}\frac{1}{2}\nabla|\bu_1-\underline\bu_2|^2\cdot(\bu_1-\underline\bu_2+\mathrm{Bog}_{\eta_1}(\Div\underline\bu_2))\dx\ds\\
&=-\frac{1}{2}\int_0^t\int_{\partial\Omega_{\eta_1}}|\bu_1-\underline\bu_2|^2\bn\cdot(\partial_t\eta_1-\partial_t\eta_2)\bn_{\eta_1 }\circ\bfvarphi_{\eta_1}^{-1}\dd \mathcal H^2\ds,
\end{align*}
where we have used the fact that after integration by parts,
\begin{align*}
\divx(\bu_1-\underline\bu_2+\mathrm{Bog}_{\eta_1}(\Div\underline\bu_2))=\divx(-\underline\bu_2+\mathrm{Bog}_{\eta_1}(\Div\underline\bu_2))=0\qquad \text{in}~~~~~~~ \Omega_{\eta_1}.
\end{align*}
The estimate of $R_{6,2}$ is straightforward and we get
\begin{align*}
R_{6,2}&\leq \int_0^t\lVert\nabla(\bu_1-\underline\bu_2)\rVert_{L^2(\Omega_{\eta_1(\sigma)})}\lVert\underline\bu_2\rVert_{L^\infty(\Omega_{\eta_1(\sigma)})}\lVert\bu_1-\underline\bu_2+\mathrm{Bog}_{\eta_1}(\Div\underline\bu_2)\rVert_{L^2(\Omega_{\eta_1(\sigma)})}\ds\\
&\leq \delta\int_0^t\lVert\nabla(\bu_1-\underline\bu_2)\rVert_{L^2(\Omega_{\eta_1(\sigma)})}^2\ds+C(\delta)\int_0^t\lVert\underline \bu_2\rVert_{W^{2,2}(\Omega_{\eta_1(\sigma)})}^2\lVert\bu_1-\underline\bu_2\rVert_{L^2(\Omega_{\eta_1(\sigma)})}^2\ds.
\end{align*}
Adding $R_6$, $R_7$, $R_8$ and $R_{20}$ together, $R_{21}$, we arrive at
\begin{align}
\nonumber
&R_6+R_7+R_8+R_{20}+R_{21}\\
%&\leq \delta\int_0^t\lVert\nabla(\bu_1-\underline\bu_2)\rVert_{L^2(\Omega_{\eta_1(\sigma)})}^2\ds+C(\delta)\int_0^t\left(1+\lVert\underline \bu_2\rVert_{W^{2,2}(\Omega_{\eta_1(\sigma)})}^2\right)\lVert\bu_1-\underline\bu_2\rVert_{L^2(\Omega_{\eta_1(\sigma)})}^2\ds\\
%&+ \int_0^t\int_{\Omega_{\eta_1(\sigma)}}\bu_1\cdot\nabla\bu_1\bu_1\dx\ds-\frac{1}{2}\int_0^t\int_{\partial\Omega_{\eta_1}}|\bu_1-\underline\bu_2|^2\bn\cdot(\partial_t\eta_1-\partial_t\eta_2)\bn_{\eta_1 }\circ\bfvarphi_{\eta_1}^{-1}\dd \mathcal H^2\ds\\
%&+\tfrac{1}{2} \int_0^t\int_{\partial\Omega_{\eta_1(\sigma)}}\bfn\cdot\partial_t\eta_1 \bn_{\eta_1 }\circ\bfvarphi^{-1}_ {\eta_1}|\underline\bu_2 |^2 \dd \mathcal H^2\ds- \int_0^t\int_{\partial\Omega_{\eta_1(\sigma)}}\bfn\cdot\partial_t\eta_2 \bn_{\eta_1 }\circ\bfvarphi^{-1}_ {\eta_1}| \bu_1 |^2 \dd \mathcal H^2\ds\\
\nonumber&\leq \delta\int_0^t\lVert\nabla(\bu_1-\underline\bu_2)\rVert_{L^2(\Omega_{\eta_1(\sigma)})}^2\ds+C(\delta)\int_0^t\left(1+\lVert\underline \bu_2\rVert_{W^{2,2}(\Omega_{\eta_1(\sigma)})}^2\right)\lVert\bu_1-\underline\bu_2\rVert_{L^2(\Omega_{\eta_1(\sigma)})}^2\ds\\
\nonumber&\quad+\int_0^t\int_{\partial\Omega_{\eta_1}}\bfn\circ\bfvarphi_{\eta_1}^{-1}\left(\frac{1}{2}|\bu_1|^2\partial_t\eta_1-\frac{1}{2}|\bu_1-\underline\bu_2|^2(\partial_t\eta_1-\partial_t\eta_2)\right)\bfn_{\eta_1}\circ\bfvarphi_{\eta_1}^{-1}\dd \mathcal{H}^2\ds\\
\label{sumbound}&\quad+\int_0^t\int_{\partial\Omega_{\eta_1}}\bfn\circ\bfvarphi_{\eta_1}^{-1}\left(\frac{1}{2}|\underline\bu_2|^2\partial_t\eta_1-|\bu_1|^2\partial_t\eta_2\right)\bfn_{\eta_1}\circ\bfvarphi_{\eta_1}^{-1}\dd \mathcal{H}^2\ds,
\end{align}
where we use an integration by parts for the following term
$$\int_0^t\int_{\Omega_{\eta_1(\sigma)}}\bu_1\cdot\nabla\bu_1\bu_1\dx\ds=\int_0^t\int_{\partial\Omega_{\eta_1}}\frac{1}{2}\bfn|\bu_1|^2\partial_t\eta_1\bfn_{\eta_1}\circ\bfvarphi_{\eta_1}^{-1}\dd \mathcal{H}^2\ds.  $$
To deal with the boundary terms on the right side of \eqref{sumbound}, we notice that
\begin{equation}\label{boundaryrew}
\begin{aligned}
&\frac{1}{2}|\bu_1|^2\partial_t\eta_1-\frac{1}{2}|\bu_1-\underline\bu_2|^2(\partial_t\eta_1-\partial_t\eta_2)+\frac{1}{2}|\underline\bu_2|^2\partial_t\eta_1-|\bu_1|^2\partial_t\eta_2\\
&=-\frac{1}{2}|\bu_1-\underline\bu_2|^2\partial_t\eta_2-\underline\bu_2(\bu_1-\underline\bu_2)\partial_t\eta_2+(\bu_1-\underline\bu_2)\underline\bu_2(\partial_t\eta_1-\partial_t\eta_2)+\underline\bu_2^2(\partial_t\eta_1-\partial_t\eta_2)\\
&=: \mathrm{I+II+III+IV}.
\end{aligned}
\end{equation}
Recalling the definition of the map $\bfvarphi_{\eta_1}:\omega\to \partial\Omega_{\eta_1}$ and using the boundary conditions of $\bu_1$ and $\underline\bu_2$, we note that the boundary integration of $\mathrm{II}$ and $\mathrm{IV}$ in \eqref{boundaryrew} satisfy
\begin{equation*}
\begin{aligned}
&\int_0^t\int_{\partial\Omega_{\eta_1}}\bfn\circ\bfvarphi_{\eta_1}^{-1}\cdot (\mathrm{II}+\mathrm{IV})\bfn_{\eta_1}\circ\bfvarphi_{\eta_1}^{-1}\dd \mathcal{H}^2\ds\\
&=-\int_0^t\int_\omega\partial_t\eta_2\bfn(\partial_t\eta_1\bfn-\partial_2\eta_2\bfn)\partial_t\eta_2\dy\ds+\int_0^t\int_{\omega}|\partial_t\eta_2\bfn|^2(\partial_t\eta_1-\partial_t\eta_2)\dy\ds\\
&=0.
\end{aligned}
\end{equation*}
Then we estimate the remaining two boundary integrals in \eqref{sumbound} which are related to $\mathrm{I}$ and $\mathrm{III}$. We first have 
\begin{equation*}
\begin{aligned}
&\int_0^t\int_{\partial\Omega_{\eta_1}}\bfn\circ\bfvarphi_{\eta_1}^{-1}\cdot~ \mathrm{I}~\bfn_{\eta_1}\circ\bfvarphi_{\eta_1}^{-1}\dd \mathcal{H}^2\ds\\
&\leq 
\int_0^t\lVert\bu_1-\underline\bu_2\rVert_{L^2(\partial\Omega_{\eta_1})}^2\lVert\partial_t\eta_2\bfn_{\eta_1}\circ\bfvarphi_{\eta_1}^{-1}\rVert_{L^\infty(\partial\Omega_{\eta_1})}\ds\\
&\lesssim \int_0^t\lVert\bu_1-\underline\bu_2\rVert_{W^{\frac{1}{6}, 2}(\partial\Omega_{\eta_1})}^2\lVert\partial_t\eta_2\rVert_{W^{\frac{5}{3}, 2}(\omega)}\ds\\
&\lesssim \int_0^t\lVert\bu_1-\underline\bu_2\rVert_{W^{\frac{2}{3}, 2}(\Omega_{\eta_1})}^2\lVert\partial_t\eta_2\rVert_{W^{\frac{5}{3}, 2}(\omega)}\ds\\
&\lesssim \int_0^t\lVert\bu_1-\underline\bu_2\rVert_{L^2(\Omega_{\eta_1})}^{\frac{2}{3}}\lVert\nabla(\bu_1-\underline\bu_2)\rVert_{L^2(\Omega_{\eta_1})}^{\frac{4}{3}}\lVert\partial_t\eta_2\rVert_{W^{1,2}(\omega)}^{\frac{1}{3}}\lVert\partial_t\eta_2\rVert_{W^{2,2}(\omega)}^{\frac{2}{3}}\ds\\
&\leq \delta\int_0^t\lVert\nabla(\bu_1-\underline\bu_2)\rVert_{L^2(\Omega_{\eta_1})}^2\ds+C(\delta)\int_0^t\lVert\partial_t\eta_2\rVert_{W^{2,2}(\omega)}^2\lVert\bu_1-\underline\bu_2\rVert_{L^2(\Omega_{\eta_1})}^2\ds,
\end{aligned}
\end{equation*}
where we used the embedding $W^{\frac{5}{3},2}(\partial\Omega_{\eta_1})\hookrightarrow L^\infty(\partial\Omega_{\eta_1})$ and the interpolation inequalities:
\begin{equation*}
\begin{aligned}
\lVert\bu_1-\underline\bu_2\rVert_{W^{\frac{2}{3}, 2}(\Omega_{\eta_1})}&\lesssim \lVert \bu_1-\underline\bu_2\rVert_{L^2(\Omega_{\eta_1})}^{\frac{1}{3}}\lVert\bu_1-\underline\bu_2\rVert_{W^{1,2}(\Omega_{\eta_1})}^{\frac{2}{3}},\\
\lVert\partial_t\eta_2\rVert_{W^{\frac{5}{3}, 2}(\omega)}&\lesssim \lVert\partial_t\eta_2\rVert_{W^{1,2}(\omega)}^{\frac{1}{3}}\lVert\partial_t\eta_2\rVert_{W^{2,2}(\omega)}^{\frac{2}{3}}.
\end{aligned}
\end{equation*}
We also have the estimate for $\mathrm{III}$:
\begin{align*}
&\int_0^t\int_{\partial\Omega_{\eta_1}}\bfn\circ\bfvarphi_{\eta_1}^{-1}\cdot~ \mathrm{III}~\bfn_{\eta_1}\circ\bfvarphi_{\eta_1}^{-1}\dd \mathcal{H}^2\ds\\
&\leq \int_0^t\lVert\bu_1-\underline\bu_2\rVert_{L^2(\partial\Omega_{\eta_1})}\lVert\underline\bu_2\rVert_{L^\infty(\partial\Omega_{\eta_1})}\lVert(\partial_t\eta_1-\partial_t\eta_2)\bfn_{\eta_1}\circ\bfvarphi_{\eta_1}^{-1}\rVert_{L^2(\partial\Omega_{\eta_1})}\ds\\
&\lesssim\int_0^t\lVert\bu_1-\underline\bu_2\rVert_{W^{\frac{1}{2}, 2}(\partial\Omega_{\eta_1})}\lVert\underline\bu_2\rVert_{W^{\frac{3}{2}, 2}(\partial\Omega_{\eta_1})}\lVert\partial_t\eta_1-\partial_t\eta_2\rVert_{L^2(\omega)}\ds\\
&\lesssim \int_0^t\lVert\bu_1-\underline\bu_2\rVert_{W^{1,2}(\Omega_{\eta_1})}\lVert\underline\bu_2\rVert_{W^{2,2}(\Omega_{\eta_1})}\lVert\partial_t\eta_1-\partial_t\eta_2\rVert_{L^2(\omega)}\ds\\
&\leq \delta\int_0^t\lVert\bu_1-\underline\bu_2\rVert_{W^{1,2}(\Omega_{\eta_1})}^2\ds+C(\delta)\int_0^t\lVert\underline\bu_2\rVert_{W^{2,2}(\Omega_{\eta_1})}^2\lVert\partial_t\eta_1-\partial_t\eta_2\rVert_{L^2(\omega)}^2\ds.
\end{align*}
Putting all the estimates together and taking the supremum with respect to time on  both sides of \eqref{mainineq}, we obtain from Gr\"onwall's lemma that for every $T>0$, \eqref{final} holds.
\end{proof}

%%%%%%%%%%%%%%%%%%%%%%%%%%%%%%%
%Dominic Estimates below
%%%%%%%%%%%%%%%%%%%%%%%%%%%%%%%
%
%
%********************************************************************************************
%
% One of the critical parts is the convective term
%\begin{align*}
%\int_0^t\int_{\Omega_{ \eta_1}}&\big(\nabla\bu_1\bu_1-\mathbf B_{\eta_1-\eta_2}\circ \bfPsi_{\eta_1-\eta_2} \nabx \overline{\bu}_2  \underline{\bu}_2\big)\,(\bu_1-\underline\bu_2+\mathrm{Bog}_{\eta_1}(\Div\underline\bu_2))\dxs\\
%&=\int_0^t\int_{\Omega_{ \eta_1}}\big(\nabla\bu_1\bu_1- \nabx \overline{\bu}_2  \underline{\bu}_2\big)\,(\bu_1-\underline\bu_2+\mathrm{Bog}_{\eta_1}(\Div\underline\bu_2))\dxs\\&+\int_0^t\int_{\Omega_{ \eta_1}}\big(\mathbb I_{3\times 3}-\mathbf B_{\eta_1-\eta_2}\circ \bfPsi_{\eta_1-\eta_2} \big)\nabx \underline{\bu}_2  \underline{\bu}_2\big)\,(\bu_1-\underline\bu_2+\mathrm{Bog}_{\eta_1}(\Div\underline\bu_2))\dxs\\
%=:\mathrm{I}+\mathrm{II},
%\end{align*}
%where 
%\begin{align*}
%\mathrm{I}
%&=\int_0^t\int_{\Omega_{ \eta_1}}\nabla(\bu_1-\underline\bu_2)\bu_1\,(\bu_1-\underline\bu_2)\dxs\\
%&+\int_0^t\int_{\Omega_{ \eta_1}}\nabla\underline\bu_2(\bu_1-\underline{\bu}_2)  \,(\bu_1-\underline\bu_2)\dxs\\
%&+\int_0^t\int_{\Omega_{ \eta_1}}\big(\nabla\bu_1\bu_1- \nabx \underline{\bu}_2  \underline{\bu}_2\big)\,\mathrm{Bog}_{\eta_1}(\Div\underline\bu_2)\dxs.
%\end{align*}
%Integrate by parts in the first term and use $\Div\bu_1=0$. For the second one
%\begin{align*}
%\int_0^t\int_{\Omega_{ \eta_1}}\nabla\underline\bu_2(\bu_1-\underline{\bu}_2)  \,(\bu_1-\underline\bu_2)\dxs&\leq \int_0^t\|\nabla\underline\bu_2\|_{L^3_x}\|\bu_1-\underline\bu_2\|_{L^6_x}\|\bu_1-\underline\bu_2\|_{L^2_x}\ds\\
%&\leq\delta \int_0^t\|\bu_1-\underline\bu_2\|_{W^{1,2}_x}^2\ds+c(\delta)\int_0^t\|\nabla\underline\bu_2\|_{L^3_x}^2\|\bu_1-\underline\bu_2\|_{L^2_x}^2\ds.
%\end{align*}
%For $\delta$ small enough the first term can be absorbed while the second one can be handled by Gronwall's lemma since $\nabx\bu_2\in L^2_t(L^3_x)$ (in fact, we even have $\nabx\bu_2\in L^2_t(L^6_x)$). Furthermore by the properties of $\mathrm{Bog}$ (gain one derivative) and
%\begin{align*}
%\Div\underline\bu_2=\mathbf B_{\eta_1-\eta_2}:\nabx\bu_2\circ \bfPsi_{\eta_1-\eta_2}=(\mathbf B_{\eta_1-\eta_2}-\mathbb I_{3\times 3}):\nabx\bu_2\circ \bfPsi_{\eta_1-\eta_2}
%\end{align*}
%we have for $p\in(3,6)$ and $q$ large enough
%\begin{align*}
%\int_0^t\int_{\Omega_{ \eta_1}}\big(\nabla\bu_1\bu_1- \nabx \underline{\bu}_2  \underline{\bu}_2\big)\,\mathrm{Bog}_{\eta_1}(\Div\underline\bu_2)\dxs&\lesssim \int_0^t\|\nabx(\bu_1-\underline\bu_2)\|_{L^2_x}\|\bu_1\|_{L^2_x}\|\Div\underline\bu_2\|_{L^p_x}\ds\\&+\int_0^t\|\nabx\underline\bu_2\|_{L^2_x}\|\bu_1-\underline\bu_2\|_{L^2_x}\|\Div\underline\bu_2\|_{L^p_x}\ds\\
%&\lesssim \int_0^t\|\nabx(\bu_1-\underline\bu_2)\|_{L^2_x}\|\Div\underline\bu_2\|_{L^p_x}\ds\\&+\int_0^t\|\bu_1-\underline\bu_2\|_{L^2_x}\|\Div\underline\bu_2\|_{L^p_x}\ds\\
%&\lesssim \delta\int_0^t\|\bu_1-\underline\bu_2\|_{W^{1,2}_x}^2\ds+c(\delta)\int_0^t\|\nabla_y(\eta_1-\eta_2)\|^2_{L^q_y}\|\nabx\bu_2\|_{L^6_x}^2\ds\\
%&\lesssim \delta\int_0^t\|\bu_1-\underline\bu_2\|_{W^{1,2}_x}^2\ds+c(\delta)\int_0^t\|\eta_1-\eta_2\|^2_{W^{2,2}_y}\|\nabx\bu_2\|_{L^6_x}^2\ds.
%\end{align*}
%Again we can absorb and apply Gronwall. The term $\mathrm II$ is similar using $\mathbb I_{3\times 3}-\mathbf B_{\eta_1-\eta_2}\sim \nabla_y(\eta_1-\eta_2)$
%
%We must also estimate the term
%\begin{align*}
%\mathrm{III}:=\int_0^t\int_{ \Omega_{\eta_1}}(\bu_1-\underline\bu_2)\cdot\partial_t\mathrm{Bog}_{\eta_1}(\Div\underline\bu_2)\dxs,
%\end{align*}
%where by continuity properties of $\mathrm{Bog}$ (check!!)
%\begin{align*}
%\mathrm{III}&\leq\int_0^t\int_{ \Omega_{\eta_1}}\|\bu_1-\underline\bu_2\|_{L_x^6}\|\partial_t\mathrm{Bog}_{\eta_1}(\Div\underline\bu_2)\|_{L^{6/5}_x}\ds\\
%&\lesssim\int_0^t\int_{ \Omega_{\eta_1}}\|\bu_1-\underline\bu_2\|_{L_x^6}\|\partial_t(\Div\underline\bu_2)\|_{W^{-1,6/5}_x}\ds\\
%&\lesssim\int_0^t\int_{ \Omega_{\eta_1}}\|\bu_1-\underline\bu_2\|_{L_x^6}\|(\mathbb I_{3\times 3}-\mathbf B_{\eta_1-\eta_2})\partial_t\nabla\bu_2\|_{W^{-1,6/5}_x}\ds\\
%&+\int_0^t\int_{ \Omega_{\eta_1}}\|\bu_1-\underline\bu_2\|_{L_x^6}\|\partial_t\mathbf B_{\eta_1-\eta_2}\partial_t\bu_2\|_{W^{-1,6/5}_x}\ds\\
%&\lesssim\int_0^t\int_{ \Omega_{\eta_1}}\|\bu_1-\underline\bu_2\|_{W_x^{1,2}}\|\eta_1-\eta_2\|_{W^{2,2}_y}\|\partial_t\bu_2\|_{L^2_x}\ds\\
%&\lesssim \delta\int_0^t\|\bu_1-\underline\bu_2\|_{W^{1,2}_x}^2\ds+c(\delta)\int_0^t\|\eta_1-\eta_2\|^2_{W^{2,2}_y}\|\partial_t\bu_2\|_{L^2_x}^2\ds.
%\end{align*}

\begin{remark}{\rm
The estimate from Theorem \ref{thm:weakstrong} also applies when the forcing in the momentum equation is in divergence form, that is
 \begin{equation*}
\left\{\begin{aligned}
& \partial_t^2\eta -\partial_t\Dely \eta + \Dely^2\eta=g-\bn^\intercal\big(\bm{\tau}+\mathbf F\big)\circ\bm{\varphi}_\eta\bn_\eta
 \vert \mathrm{det}(\naby \bm{\varphi}_\eta)\vert
&\text{ for all }  (t,\by)\in I\times\omega,\\
 &\partial_t \bu  + (\mathbf{v}\cdot \nabx)\mathbf{v} 
 = 
 \Delx \bu -\nabx\pi+ \Div\bfF &\text{ for all }(t,\bx)\in I\times\Omega_\eta,\\
 &\Div \bu=0&\text{ for all }(t,\bx)\in I \times\Omega_\eta,
\end{aligned}\right.
\end{equation*}
for some $\mathbf F:I\times\Omega_\eta\rightarrow\R^{3\times 3}$.
In this case we obtain the estimate
\begin{align}
\nonumber
&\sup_{t\in I}\int_{\Omega_{\eta_1(t)}}|\bu_1(t)-\underline{\bu}_2(t)|^2\dx+\sup_{t\in I}\int_\omega\big(|\partial_t(\eta_1-\eta_2)(t)|^2
+
|\Dely(\eta_1-\eta_2)(t)|^2\big)\dy
\\
\nonumber&\qquad\qquad\qquad\qquad+\int_I
\int_{\Omega_{\eta_1(\sigma)}}|\nabla(\bu_1-\underline{\bu}_2)|^2\dx\dt 
+
\int_I\int_\omega|\partial_t\nabla_{\by}(\eta_1-\eta_2)|^2\dy\dt\\
\nonumber&\lesssim \int_{\Omega_{\eta_1(0)}}|\bu_1(0)-\underline{\bu}_2(0)|^2\dx+\int_\omega|\partial_t(\eta_1-\eta_2)(0)|^2\dy+\int_\omega|\Dely(\eta_1-\eta_2)(0)|^2\dy\\
&\qquad\qquad\qquad\qquad +\int_I\int_{\Omega_{\eta_1}}| \mathbf F_1-\underline {\mathbf F}_2|^2\dx\dt+\int_I\int_\omega| g_1-g_2|^2\dy\dt,\nonumber
\end{align}
where $\underline{\mathbf F}_2:=\bfF_2\circ \bfPsi_{\eta_2-\eta_1}$.}
\end{remark}

\section{The main result}\label{summary}
In the following, we formulate the desired conditional regularity and uniqueness result for \eqref{1}--\eqref{interfaceCond}, which implies Theorem~\ref{thm:mainsimple} and Corollary~\ref{cor:mainsimple}.
Its proof follows by combining
Theorems \ref{thm:fluidStructureWithoutFK},  \ref{prop2} and \ref{thm:weakstrong}.
\begin{theorem}\label{thm:main}
Let $T>0$ be given. Suppose that the dataset
$(\bff, g, \eta_0, \eta_*, \bu_0)$
satisfies \eqref{dataset} and
\eqref{datasetImproved}.
Let $(\bu,\eta)$ be a weak solution of \eqref{1}--\eqref{interfaceCond} in the sense of Definition \ref{def:weakSolution}. Suppose that we
have
\begin{align}\label{eq:regu'}
\bu&\in L^r(I;L^s(\Omega_\eta)),\quad \tfrac{2}{r}+\tfrac{3}{s}\leq1,\\
\eta&\in L^\infty(I;C^{1}(\omega)).\label{eq:regeta''}
\end{align}
Then $(\bu,\eta)$ is a strong solution in the sense of Definition \ref{def:strongSolution} on $I = (0, t)$, where $t < T$ only in case $\Omega_{\eta(s)}$ approaches a self-intersection when $s\rightarrow t$ or it degenerates\footnote{Self-intersection and degeneracy are excluded if $\sup_t\|\eta\|_{W^{1,\infty}_{y}}<L$, cf. \eqref{eq:boundary1} and \eqref{eq:1705}.} (namely, if $\displaystyle\lim_{s\rightarrow t}(\partial_1\bfvarphi_\eta\times \partial_2\bfvarphi_\eta)(s,\by)=0$ or $\displaystyle\lim_{s\rightarrow t}\bfn(\by)\cdot\bfn_{\eta(s)}(\by)=0$ for some $\by\in\omega$). 
Moreover, $(\bu,\eta)$  is unique in the class of weak solutions with deformation in $L^\infty(I,C^{0,1}(\omega))$.
\end{theorem}
\begin{proof}
Consider first the problem on the interval $(0,T^\ast)$ in which the strong solution exists by Theorem \ref{thm:fluidStructureWithoutFK}. On account of \eqref{eq:regeta''} Theorem \ref{thm:weakstrong} applies and thus both solutions coincide. Hence the strong solution satisfies \eqref{eq:regu'} (with a constant independent of $T^\ast$). Thus we obtain the estimate
from Theorem \ref{prop2}. Now we can apply Theorem \ref{thm:fluidStructureWithoutFK} to obtain a strong solution on the interval $(T^\ast,2T^\ast)$ with initial data $\bfu(T^\ast),\eta(T^\ast),\partial_t\eta(T^\ast)$.
This procedure can now be repeated until the moving boundary approaches a self-intersection or degenerates (that is $(\partial_1\bfvarphi_\eta\times \partial_2\bfvarphi_\eta)(T,\by)=0$ for some $\by\in\omega$).
\end{proof}

\section*{Acknowledgments}
S. Schwarzacher and P. Su are partially supported by the ERC-CZ Grant CONTACT LL2105 funded by the Ministry of Education, Youth and Sport of the Czech Republic and the University Centre UNCE/SCI/023 of Charles University. S. Schwarzacher also acknowledges the support of the VR
Grant 2022-03862 of the Swedish Science Foundation.


\section*{Compliance with Ethical Standards}
\smallskip
\par\noindent
{\bf Conflict of Interest}. The authors declare that they have no conflict of interest.


\smallskip
\par\noindent
{\bf Data Availability}. Data sharing is not applicable to this article as no datasets were generated or analysed during the current study.


\begin{thebibliography}{[M]}
%\bibitem{Am} H. Amann: Compact embeddings of vector-valued Sobolev and Besov
%spaces, Glass. Mat., III. Ser. 35 (55), 161--177. (2000)
%\bibitem{Br} D. Breit: \emph{Existence theory for generalized Newtonian fluids.}
%Mathematics in Science and Engineering. Elsevier/Academic Press, London. (2017)
%\bibitem{BBD} D. Breit, A. Kh Balci, L. Diening; Global Besov estimates for nonlinear elliptic problems.
\bibitem{ABC22}
D. Albritton, E. Bru{\'e}, and M. Colombo,
\newblock Non-uniqueness of {L}eray solutions of the forced {N}avier-{S}tokes
  equations.
\newblock {\em Annals of Mathematics}, 196(1), 415--455 (2022)


\bibitem{barbu2007existence}
V. Barbu, Z. Gruji\'{c}, I. Lasiecka, and A. Tuffaha, Existence of the
  energy-level weak solutions for a nonlinear fluid-structure interaction
  model.
\newblock In: Fluids and waves, \emph{Contemp. Math.}, vol. 440, pp. 55--82.
  Amer. Math. Soc., Providence, RI (2007)

\bibitem{barbu2008smoothness}
V. Barbu, Z. Gruji\'{c}, I. Lasiecka, and A. Tuffaha, Smoothness of weak
  solutions to a nonlinear fluid-structure interaction model.
\newblock {\em Indiana Univ. Math. J.}, {57}(3), 1173--1207 (2008)

\bibitem{beale1984remark}
J. T. Beale, T. Kato, and A. Majda, Remarks on the breakdown of smooth solutions for the  
$3${D} Euler equations.
 \newblock {\em Comm. Math. Phys.}, 94(1), 61-66 (1984)

\bibitem{beirao1995new}
H. Beir\~ao da Veiga, A new regularity class for the Navier-Stokes equations in $\mathbb{R}^n$. 
{\em Chinese Annals of Mathematics. Series B.}, 16. (1995)


\bibitem{Br} D. Breit, Regularity results in 2D fluid-structure interaction. {\em Math. Ann.}, 
doi:10.1007/s00208-022-02548-9 (2022)

\bibitem{Br2} D. Breit, Partial boundary regularity for the Navier-Stokes equations in irregular
  domains. Preprint at arXiv:2208.00415v2 (2022)
  
\bibitem{breit2021incompressible} D. Breit and P. R. Mensah, An incompressible polymer fluid interacting with a Koiter shell. {\em J. Nonlinear Sci.}, 31, 1-56 (2021)

\bibitem{BrSc} D. Breit and S. Schwarzacher, Compressible fluids interacting with a linear-elastic shell. {\em Arch. Rational Mech. Anal.}, 228, 495--562 (2018)
\bibitem{BrScF} D. Breit and S. Schwarzacher, {Navier--Stokes--Fourier fluids interacting with elastic shells.} {\em Ann. Sc. Norm. Super. Pisa Cl. Sci.}, (5)
Vol. XXIV, 619-690 (2023)

\bibitem{boulakia2007existence}
M. Boulakia, Existence of weak solutions for the three-dimensional motion of
  an elastic structure in an incompressible fluid.
\newblock {\em J. Math. Fluid Mech.}, {9}(2), 262--294 (2007) 

\bibitem{sunny}
S. {\v{C}}ani{\'c},
\newblock {\em Moving boundary problems.}
\newblock Bulletin of the American Mathematical Society, 2020.

% 
%\bibitem{BS} R. M. Brown, Z. Shen; 
%Estimates for the Stokes Operator in Lipschitz Domains
%Indiana Univ. Math. J. 44(4), 1183--1206. (1995)

\bibitem{chambolle2005existence}
A. Chambolle, B. Desjardins, M. J. Esteban, and C. Grandmont, Existence of weak
  solutions for the unsteady interaction of a viscous fluid with an elastic
  plate.
\newblock {\em J. Math. Fluid Mech.}, {7}(3), 368--404 (2005)

\bibitem{chemetov2019weak}
N. V. Chemetov, B. Muha, and \v{S}. Ne\v{c}asov\'{a}, Weak-strong uniqueness for fluid-rigid body interaction problem with slip boundary condition. \newblock {\em J. Math. Phys.}, 60(1) (2019)



\bibitem{chen2006space}
Q. Chen and Z. Zhang, Space-time estimates in the Besov spaces and the Navier-Stokes equations.
\newblock {\em Methods Appl. Anal.}, 13(1), 107-122 (2006)

\bibitem{cheng2007navier}
C.H.A. Cheng, D. Coutand, and S. Shkoller, Navier-{S}tokes equations interacting
  with a nonlinear elastic biofluid shell.
\newblock {\em SIAM J. Math. Anal.}, {39}(3), 742--800 (2007)

\bibitem{CS}
C.H.A. Cheng and S. Shkoller, The interaction of the 3{D} {N}avier-{S}tokes
  equations with a moving nonlinear {K}oiter elastic shell.
\newblock {\em SIAM J. Math. Anal.}, {42}(3), 1094--1155 (2010)
%
%\bibitem{CiMa} A. Cianchi, V. G. Maz'ya: Second-Order Two-Sided Estimates in
%Nonlinear Elliptic Problems. Arch Rational Mech. Anal. 229 569--599. (2018)
 
\bibitem{coutand2005motion}
D. Coutand and S. Shkoller, Motion of an elastic solid inside an incompressible
  viscous fluid.
\newblock {\em Arch. Ration. Mech. Anal.}, {176}(1), 25--102 (2005)

\bibitem{coutand2006interaction}
D. Coutand and S. Shkoller, The interaction between quasilinear elastodynamics
  and the {N}avier-{S}tokes equations.
\newblock {\em Arch. Ration. Mech. Anal.}, {179}(3), 303--352 (2006)
 
%\bibitem{DeSa} R. Denk, J. Saal: $L^p$-theory for a fluid-structure interaction model.  Z. Angew. Math. Phys. 71, 158. (2020)
%\bibitem{DHP} W. Desch, M. Hieber, J. Prüss: $L^p$-theory of the Stokes equation in a half space. J. Evol. Equ.
%1, 115--142. (2001)



\bibitem{GlaSue15}
O. Glass and F. Sueur,
\newblock Uniqueness results for weak solutions of two-dimensional fluid-solid
  systems.
\newblock {\em Arch. Ration. Mech. Anal.}, 218(2), 907--944 (2015)


% \bibitem{FKV} E. Fabes, C. Kenig, G. Verchota: The Dirichlet problem for the Stokes system on Lipschitz bounded domains. Duke Math. J. 57, 769--792. (1988)
%\bibitem{Q1}L.  Formaggia, J.-F. Gerbeau, F. Nobile, A. Quarteroni;
%On the coupling of 3D and 1D Navier--Stokes equations for flow problems in compliant vessels. 
%Comput. Meth.
%Appl. Mech. Eng. 191 (6-7), 561--582. (2001)
%\bibitem{Ga} G. P. Galdi: An Introduction to the Mathemaical Theory of the Navier--Stokes equations. Steady-Sate Problems. 2nd Edition. Springer Monographs in Mathematics. Springer, New York Dordrecht Heidelberg London. (2011) 

\bibitem{grandmont2008existence}
C. Grandmont, Existence of weak solutions for the unsteady interaction of a
  viscous fluid with an elastic plate.
\newblock {\em SIAM J. Math. Anal.}, {40}(2), 716--737 (2008)

\bibitem{GraHil}
C. Grandmont and M. Hillairet, Existence of global strong solutions to a
  beam-fluid interaction system.
\newblock {\em Arch. Ration. Mech. Anal.}, {220}(3), 1283--1333 (2016)

%\bibitem{LS} Ladyzhenskaya, O. A.; Seregin, G. A. On partial regularity of suitable weak solutions to the three-dimensional Navier-Stokes equations. J. Math. Fluid Mech. 1 (1999), no. 4, 356--387.
\bibitem{GraHilLe} C. Grandmont, M. Hillairet, and J. Lequeurre,
Existence of local strong solutions to fluid-beam and fluid-rod interaction systems. 
{\em Ann. I. H. Poincar\'e--AN}, 36, 1105--1149 (2019)
%\bibitem{HS} Hieber, Matthias; Saal, Jürgen: The Stokes equation in the $L^p$-setting: well-posedness and regularity properties. Handbook of mathematical analysis in mechanics of viscous fluids, 117--206, Springer, Cham. (2018)

\bibitem{ignatova2014well}
M. Ignatova, I. Kukavica, I. Lasiecka, and A. Tuffaha, On well-posedness and
  small data global existence for an interface damped free boundary
  fluid-structure model.
\newblock {\em Nonlinearity}, {27}(3), 467--499 (2014)


\bibitem{escauriaza}
L. Iskauriaza, G. A. Serëgin, and V. Sverak, $ L_{3,\infty}$-solutions of Navier-Stokes equations and backward uniqueness.
{\em Uspekhi Mat. Nauk.}, 58 (2(350)), 3–44 (2003)


\bibitem{Sve14}
H. Jia and V. Sverak,
\newblock Local-in-space estimates near initial time for weak solutions of the
{N}avier-{S}tokes equations and forward self-similar solutions.
\newblock {\em Inventiones mathematicae}, 196:233--265 (2014)


\bibitem{JiaSve15}
H. Jia and V. Sverak,
\newblock Are the incompressible 3{D} {N}avier--{S}tokes equations locally ill-posed
  in the natural energy space?
\newblock {\em Journal of Functional Analysis}, 268(12), 3734--3766 (2015)

\bibitem{kaltenbacher}
B.~Kaltenbacher, I.~Kukavica, I.~Lasiecka, R.~Triggiani, A.~Tuffaha, and J.~T.
  Webster,
\newblock {\em Mathematical Theory of Evolutionary Fluid-Flow Structure
  Interactions}.
\newblock Oberwolfach Seminars. Springer International Publishing, 2018.



\bibitem{KamSchSpe20}
M. Kampschulte, S. Schwarzacher, and G. Sperone,
Unrestricted deformations of thin elastic structures interacting with fluids.
\newblock {\em Journal de Math{\'e}matiques Pures et Appliqu{\'e}es}, 173, 96--148 (2023)

\bibitem{kozono2002critical}
H. Kozono, T. Ogawa, and Y. Taniuchi, The critical Sobolev inequalities in Besov spaces and regularity criterion to some semi-linear evolution equations.
\newblock {\em Mathematische Zeitschrift}, 242, 251–278 (2002)

\bibitem{kozono2004bilinear}
H. Kozono and Y. Shimada, Bilinear estimates in homogeneous Triebel-Lizorkin spaces and the Navier-Stokes equations. 
\newblock {\em Math. Nachr.}, 276, 63-74 (2004)

\bibitem{kozono2000limiting}
H. Kozono and Y. Taniuchi, Limiting Case of the Sobolev Inequality in $\mathrm{BMO}$, with Application to the Euler Equations. 
\newblock {\em Commun. Math. Phys.}, 214, 191–200 (2000)



\bibitem{kukavica2012solutions}
I. Kukavica and A. Tuffaha, Solutions to a fluid-structure interaction free
  boundary problem.
\newblock {\em Discrete Contin. Dyn. Syst.}, {32}(4), 1355--1389 (2012)

\bibitem{lady}
O. A. Ladyzhenskaya, Uniqueness and smoothness of generalized solutions of Navier-Stokes equations. {\em Zap. Nau\^vcn.
Sem. Leningrad. Otdel. Mat. Inst. Steklov. (LOMI)}, 5, 169–185 (1967)

\bibitem{LeRu} D. Lengeler and M. \Ruzicka,
Weak Solutions for an incompressible {N}ewtonian fluid interacting with a {K}oiter type shell. 
{\em Arch. Rat. Mech. Anal.}, 211(1), 205--255 (2014)


\bibitem{Le} J. Lequeurre,
Existence of strong solutions to a fluid–structure system. 
{\em SIAM J. Math. Anal.}, 43(1), 389--410 (2011)


\bibitem{DRR} D. Maity, J. P. Raymond, and A. Roy,
 Maximal-in-time existence and uniqueness of strong solution of a 3{D} fluid-structure interaction model. 
{\em  SIAM J. Math. Anal.}, 52, no. 6, 6338--6378 (2020)
 
 
\bibitem{maity2023uniqueness} D. Maity and T. Takahashi, Uniqueness and regularity of weak solutions of a fluid-rigid body interaction system under the Prodi-Serrin condition.  hal-04075090 (2023)

%
%\bibitem{Q2} A. Quarteroni, M. Tuveri, A. Veneziani; 
%Computational vascular fluid dynamics: problems, models and methods. 
%it Comput. Vis. Sci. 2(4), 163--197. (2000)
\bibitem{MaSh} V. G. Maz'ya and T. O. Shaposhnikova, {\em Theory of Sobolev multipliers}, volume 337 of
Grundlehren der Mathematischen Wissenschaften [Fundamental Principles of Mathematical Sciences].
Springer-Verlag, Berlin. With applications to differential and integral operators (2009)
%\bibitem{Mi} S. Mitra: Local Existence of Strong Solutions of a Fluid--Structure Interaction Model. J. Math. Fluid Mech. 22: 60. (2020)


\bibitem{CanMuh13} B. Muha and S. \v{C}ani\'c, Existence of a weak solution to a nonlinear fluid-structure interaction problem modeling the flow of an incompressible, viscous fluid in a cylinder with deformable walls. {\em Arch. Rational Mech. Anal.}, 207 (3), 919--968 (2013)

%\bibitem{muha2016existence}
%B. Muha, S. \v{C}ani\'{c}: Existence of a weak solution to a fluid-elastic
%  structure interaction problem with the {N}avier slip boundary condition.
%\newblock J. Differential Equations {260}(12), 8550--8589 (2016)
\bibitem{muha2022regularity}
B. Muha, \v{S}. Ne\v{c}asov\'{a}, and A. Rado\v{s}evi\'{c}, On the regularity of weak solutions to the fluid-rigid body interaction problem. arXiv preprint arXiv:2211.03080 (2022).

%\bibitem{muha2021uniqueness}
%B. Muha, \v{S}. Ne\v{c}asov\'{a}, A. Rado\v{s}evi\'{c}:
%A uniqueness result for 3D incompressible fluid-rigid body interaction problem. 
%\newblock J. Math. Fluid Mech., 23(1):Paper No. 1, 39, 2021.


\bibitem{MuSc} B. Muha and S. Schwarzacher, Existence and regularity for weak solutions for a fluid interacting with a non-linear shell in 3D. {\em Ann. I. H. Poincar\'e -- AN}, 39, 1369--1412 (2022)
%\bibitem{NRL}
%N\"{a}gele, P., \Ruzicka, M., Lengeler, D.: Functional setting
%  for unsteady problems in moving domains and applications.
% Complex Var. Elliptic Equ. {62}(1), 66--97. (2017)
\bibitem{prodi1959} 
G. Prodi, Un teorema di unicit\`a per le equazioni di Navier-Stokes. {\em Ann. Mat. Pura Appl.}, (4), 48:173–182,
(1959)


\bibitem{raymond2014fluid}
J. P. Raymond and M. Vanninathan, A fluid-structure model coupling the
  {N}avier-{S}tokes equations and the {L}am\'{e} system.
\newblock {\em J. Math. Pures Appl.}, (9) {102}(3), 546--596 (2014) 

\bibitem{RuSi} T. Runst and W. Sickel, {\em Sobolev Spaces of Fractional Order, Nemytskij Operators, and Nonlinear Partial Differential Equations}, De Gruyter Series in Nonlinear Analysis and Applications, vol.3, Walter de Gruyter \& Co., Berlin, New York (1996)


\bibitem{SaaSch21}
O. Saari and S. Schwarzacher, Construction of a right inverse for the divergence in non-cylindrical time dependent domains.
\newblock {\em Annals of PDE}, 9(1), 1--52 (2023)

\bibitem{SchSheTum23}
S. Schwarzacher, B. She, and K. Tuma.
\newblock Stability and error estimates of a linear numerical scheme
  approximating nonlinear fluid-structure interactions.
\newblock {\em arXiv:2301.05014}, (2023)

\bibitem{schwarzacher2022weak}
S. Schwarzacher and M. Sroczinsk, Weak-strong uniqueness for an elastic plate interacting with the {N}avier-{S}tokes equation.
{\em SIAM J. Math. Anal.}, {54}(4), 4104--4138 (2022)





\bibitem{serrin1962} 
J. Serrin, On the interior regularity of weak solutions of the Navier-Stokes equations. {\em Arch. Rational Mech. Anal.},
9:187–195 (1962)

\bibitem{serrin1963} 
J. Serrin, The initial value problem for the Navier-Stokes equations. {\em In Nonlinear Problems Proc. Sympos.,
Madison, Wis.}, 1962, pages 69–98. Univ. of Wisconsin Press, Madison, Wis. (1963)
%\bibitem{SchSro20}
%S. Schwarzacher, M. Sroczinski:
%\newblock Weak-strong uniqueness for an elastic plate interacting with the Navier Stokes equation.
%\newblock {\em arXiv preprint:} arXiv:2003.04049, (2020).
%\bibitem{Saal} J. Saal: The Stokes operator with Robin boundary conditions in solenoidal subspaces of
%$L^1(\bfR^n_+)$ and $L^\infty(\bfR^n_+)$. Commun. Partial Differ. Equ. 32(3), 343--373. (2007)
%\bibitem{SaSc} O. Saari, S. Schwarzacher: Construction of a right inverse for the divergence in non-cylindrical time dependent domains, arXiv:2107.09573v2
%\bibitem{Soa} V. A. Solonnikov: Estimates for solutions of a non-stationary linearized system of Navier--Stokes equations. (Russian) Trudy Mat. Inst. Steklov. 70, 213--317. (1964) 
%\bibitem{So} V. A. Solonnikov: Estimates for Solutions of Nonstationary Navier-Stokes Equations.
%Zap. Nauchn. Semin.
%POMI, 38, 153--231. (1972)





\bibitem{solonnikov1977} V. A. Solonnikov, Estimates for solutions of nonstationary {N}avier-{S}tokes equations.
\newblock {\em L. Soviet Math.}, 8(4), 467-528 (1977)



\bibitem{Tr} H. Triebel, \emph{Theory of Function Spaces}, Modern Birkh\"auser Classics, Springer, Basel (1983)

\bibitem{Tr2} H. Triebel, \emph{Theory of Function Spaces II}, Modern Birkh\"auser Classics, Springer, Basel (1992)
%
%\bibitem{Uk} S. Ukai: A solution formula for the Stokes equation in $\R^n_+$.
%Commun. Pure Appl.Math. 40(5),
%611--621. (1987)
\end{thebibliography}








































%


%

%


\end{document}
