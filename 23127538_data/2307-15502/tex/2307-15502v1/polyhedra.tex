\documentclass[fleqn, eqnumis, pdf]{hjms}
\usepackage{footnote}
\usepackage{multicol}
\usepackage{booktabs}
\usepackage[flushleft]{threeparttable}
\usepackage[font=small,labelfont=bf]{caption}
\begin{document}
\title{A Positive Answer to a Question of K. Borsuk on the Capacity of Polyhedra with Finite-by-Cyclic Fundamental Group}
\author{ Mojtaba~Mohareri\footnote{Department of Pure Mathematics, Ferdowsi University of Mashhad, P.O.Box 1159-91775, Mashhad, Iran,  \ Email: {\tt m.mohareri@stu.um.ac.ir }} and  
Behrooz~Mashayekhy\footnote{Department of Pure Mathematics, Center of Excellence in Analysis on Algebraic Structures, Ferdowsi University of Mashhad, P.O.Box 1159-91775, Mashhad, Iran,  \ Email: {\tt bmashf@um.ac.ir }} \footnote{Corresponding Author.}} 
\begin{abstract}
 Karol Borsuk in 1968 asked: Is it true that every finite polyhedron dominates only finitely many
different shapes? Danuta Ko\l{}odziejczyk showed that generally an answer to the Borsuk question is negative and also presented a positive answer by proving that every polyhedron with finite fundamental group dominates only finitely many different homotopy types (hence shapes). In this paper, we show that polyhedra with finite-by-cyclic fundamental group dominate only finitely many different homotopy types. As a consequence, we give a partial positive answer to this question of  Ko\l{}odziejczyk:  Does every polyhedron with abelian fundamental group dominate only finitely many different homotopy types? In fact, we show that every polyhedron with abelian fundamental group of rank 1 dominates only finitely many different homotopy types.  Finally, we prove  that every polyhedron dominates only finitely many homotopy types of simply connected CW-complexes.
\end{abstract}
\begin{keyword}
Homotopy domination, Homotopy type,  Polyhedron, CW-complex.
\end{keyword}
\begin{AMS}
55P15, 55P10, 55P55, 55Q20.
\end{AMS}
\section{Introduction and Motivation}
Recall that a \textit{domination} in a given category $\mathcal{C}$ is a morphism $f : X \to Y$ , where $X, Y \in Obj \mathcal{C}$, for which there exists a morphism $g : Y \to X$ of $\mathcal{C}$ such that $f\circ g = \mathrm{id}_Y$. Then we say that $Y$ is dominated by $X$, denoted by $Y \leqslant X$. 

In 1979 at the International Topology Conference in Moscow, K. Borsuk
defined the capacity of a compactum (i.e. a compact metric space) $A$, denoted by $C(A)$, as the cardinality of the class
consisting of the shapes of all compacta $X$, for which $Sh(X) \leqslant Sh(A)$. He also posed a
question: Is the capacity of each polyhedron finite? (see \cite{So}).

In the above problem  the notions shape and shape domination can be replaced by the notions homotopy type and homotopy domination, respectively. Indeed, by the known results in shape theory (see \cite{Ed,Ha}, \cite[Theorems 2.2.6, and 2.1.6]{Dy}) we get that, for each polyhedron $P$, there is a 1-1 functorial correspondence between the shapes of compacta shape dominated by $P$ and the homotopy types of CW-complexes (not necessarily finite) homotopy dominated by $P$ (in both pointed and unpointed cases).

Recall that each space homotopy dominated by a polyhedron has the homotopy type of a CW-complex, not necessarily finite (see \cite{Wall}). Thus the Borsuk problem is equivalent to: Does every polyhedron homotopy dominate only finitely many different
homotopy types?
In this paper we consider dominations of a polyhedron in the category of CW-complexes and homotopy classes of cellular maps between them. 

By the result of S. Mather \cite{Mather} (see also a simple paper of Holsztynski \cite{Hol}), every polyhedron dominates only a countable number of different homotopy types. In \cite{1}  D. Ko\l{}odziejczyk  showed that generally the answer to the Borsuk question is negative: there exists a polyhedron (even of dimension 2), which homotopy dominates infinitely many polyhedra of different homotopy types. Moreover, she proved that such examples are not rare: for every non-abelian poly-$\mathbb{Z}$-group $G$ and an integer $n\geq 3$ there exists a polyhedron $P$ with $\pi_1 (P)\cong G$ and $\dim P = n$ dominating infinitely many polyhedra of different homotopy types (see \cite{3}). Thus, there exist polyhedra with nilpotent fundamental groups with infinite capacity.

On the other hand, Ko\l{}odziejczyk in \cite{2} obtained, using the results of localization theory in the homotopy category of CW-complexes, that every simply-connected polyhedron dominates only finitely many different homotopy types. In \cite{4} she proved, in another way, that polyhedra with finite fundamental groups dominate only finitely many different homotopy types. In \cite{5}, by the fact that there exist polyhedra with nilpotent fundamental groups homotopy dominating infinitely many different homotopy types, Ko\l{}odziejczyk asked: Does every polyhedron $P$ with the abelian fundamental group $\pi_1 (P)$ dominate only finitely many different homotopy types? Extending the methods of \cite{4}, she proved that for some classes of polyhedra with abelian fundamental group, the answer to this question is positive: Every polyhedron $P$ with abelian fundamental group $\pi_1 (P)$ and finitely generated homology groups $H_i (\tilde{P};\mathbb{Z})$, for $i\geq 2$, has finite capacity, where $\tilde{P}$ is the universal covering of $P$ (see \cite[Theorem 2]{5}).

In this paper, we give a positive answer to the Borsuk question  for polyhedra with finite-by-cyclic fundamental group. This fact, not only extends the results of \cite{4}, but also gives a partial positive answer to the mentioned question of Ko\l{}odziejczyk:  Every polyhedron  with abelian fundamental group of rank 1 dominates only finitely many different homotopy types. 
Furthermore, we show by a similar argument that  polyhedra with splitting-simple fundamental group have finite capacity. 

Finally, following the question of Borsuk, it seems interesting to ask which subclass of all CW-complexes homotopy dominated by a polyhedron is finite. We prove that every polyhedron dominates only finitely many homotopy types of simply connected CW-complexes.
\section{Preliminaries}
Let us recall some definitions.
A group $G$ is \textit{finite-by-cyclic}  if it is an extension of a finite group by a
cyclic group, i.e., there exists a finite normal subgroup $H\unlhd G$ such that $G/H$ is cyclic.

Let $H$ be a subgroup of $G$. Then a  homomorphism $r:G\to H$ is said to be a \textit{retraction} if the inclusion  homomorphism $i:H\hookrightarrow G$ is a right inverse of $r$, i.e. $r(x)=x$ for all elements $x\in H$. Then $H$ is called  a \textit{retract} of $G$. 

Given a group $G$ with identity element $e$, a subgroup $H$, and a normal subgroup $N\unlhd G$. We say $G$ is the \textit{semidirect product} of $N$ and $H$, written by $G=N\rtimes H$ or $G=H\ltimes N$, if $G=NH$ and $N\cap H=1$. 
The lemma below can be easily  deduced from the definition. 
\begin{lemma}\label{Semi}
Let $G$ be a group. Then a subgroup $H$ of $G$ is a retract of $G$ if and only if $G=H\ltimes N$ for a normal subgroup $N$ of $G$.
\end{lemma}
Recall that a group $G$ is \textit{Hopfian} if every epimorphism $f :G\to G$ is an automorphism (equivalently, $N = 1$ is the only normal subgroup for which $G/N\cong G$). 

A homomorphism $f:G\to H$ of groups is an \textit{r-homomorphism} if there exists a homomorphism $g:H\to G$ such that $f\circ g=\textrm{id}_{H}$. Then $H$ is an \textit{r-image} of $G$.

A group $G$ is \textit{weakly Hopfian} if every r-homomorphism $f:G\to H\cong G$ is an isomorphism. In other words, $G=N\rtimes H$ and $H\cong G$ imply $N=1$.

For the definitions of a polyhedron and a CW-complex, see for example \cite[Chapters 7 and 8]{R}. In this paper, every polyhedron and  CW-complex is assumed to be finite and connected.  Also, every map between two CW-complexes is assumed to be cellular. A map $f :X \to Y$ between two CW-complexes $X$ and $Y$ is said to be a \textit{homology equivalence} if it induces an isomorphism of the integral homology groups $H_i (f):H_i (X;\mathbb{Z})\to H_i (Y ;\mathbb{Z})$, for all $i\in \mathbb{N}$ (see \cite[pp. 181-182]{Whi}). Since every finite polyhedron is homotopy equivalent to a finite CW-complex of the same dimension, and conversely, we use the terms ``polyhedron'' and ``finite CW-complex'' interchangeably. We assume that the reader is familiar with the basic notions and facts of homotopy theory.
\section{Polyhedra with finite-by-cyclic fundamental group}
D. Ko\l{}odziejczyk in \cite{4} presented a positive answer to the Borsuk question by proving that polyhedra with finite fundamental group dominate only finitely many different homotopy types. In this section, we intend to extend this result for polyhedra with finite-by-cyclic fundamental group.

The following lemma is an algebraic technical result which is a keynote to prove the main result of the section. 
\begin{lemma}\label{2}
Every finite-by-cyclic group have only finitely many  retracts.
\end{lemma}
\begin{proof}
If $G$ is finite, then clearly the result holds. So we can suppose that $G$ is finite-by-infinite cyclic. Hence $G$ has a finite normal subgroup $N$ such that $G/N$ is infinite cyclic. First, we show that $G$ contains only finitely many finite subgroups.  Let $K$ be a finite subgroup of $G$. Then $KN$ is also a finite subgroup of $G$ and therefore $KN/N$ is a finite subgroup of $G/N \cong \mathbb{Z}$. We conclude that $KN/N = \left\{0\right\}$, hence $KN = N$ and so $K\leq N$.  Hence, the number of finite retracts of a finite-by-cyclic group is finite.

Since $G$ is finite-by-infinite cyclic, it is well-known that $G$ is a virtually cyclic group.  Therefore $G$ has a finite index infinite cyclic normal subgroup $H$. We claim that $|G/H|$ is a finite upper bound on the order of all finite subgroups of $G$. By contrary, suppose that $C$ is a subgroup of $G$  whose order is larger than the order of the finite quotient group $G/H$. Therefore the homomorphism $C \to G/H$ defined by $g\mapsto gH$ has a nonempty kernel. Clearly $C \cap H$ is the kernel of the homomorphism. Since $H$ is infinite cyclic, any nontrivial element of the kernel has infinite order, implying that the subgroup $C$ is infinite which is a contradiction. 

 Now assume that $K$ is an infinite retract of $G$. Then by Lemma \ref{Semi},  we have  $G = K \ltimes L$ for a normal subgroup $L$ of $G$.  Then by the notation of the above paragraph we have $|L:L \cap H|=|LH:H| \le |G:H|$ is finite. We claim that $L$ is finite. Suppose by contrary that $L$ is infinite, then so is $L \cap H$, and hence since $H$ is infinite cyclic, $|H:L \cap H|$ is finite. Therefore, since $|G/H|$ is finite we have $|G:L \cap H|$ is finite, and so $|G:L|$ is finite. But $G/L \cong K$, so $K$ is finite which is a contradiction. Hence, if $K$ is infinite, then $L$ must be finite. Now by the result of the above paragraph we have $|G:K|=|L|\leq |G/H|$. Thus, infinite retracts of $G$ have finite index bounded by $|G/H|$. Since a finitely generated group has only finitely many subgroups of a given finite index, we deduce that there are only finitely many infinite retracts of $G$. 
\end{proof}
\begin{remark}
It is well-known that every virtually cyclic group is either a finite-by-cyclic group or a finite-by-infinite dihedral group. It should be emphasized that Lemma \ref{2} does not hold for virtually cyclic groups generally. In fact in the proof of Lemma \ref{2}, we showed that the number of infinite retracts of a virtually cyclic group  is finite, but this is not true for its finite retracts. For example, consider the infinite dihedral group $D_\infty = \langle a,b \mid a^2 = b^2 = 1\rangle $. It is a virtually cyclic group with infinitely many cyclic subgroups of order 2 each of which is a retract of $D_\infty$.
\end{remark}
For a polyhedron $P$, let $X\leqslant P$ denote that $X$ is homotopy dominated by $P$. Writing this we will have in mind a fixed domination $d_X : P \to X$ of $P$ over $X$, and a fixed inverse map $u_X : X \to P$ (i.e. $d_X u_X \simeq  \textrm{id}_X$). It is easily   seen that the map $k_X = u_X d_X : P \to P$ is an idempotent in the homotopy category of CW-complexes and homotopy classes of maps between them.  From now on ``dominated'' will always mean ``homotopy dominated''.

Using the above notation we state the following useful lemma.
\begin{lemma}\cite[Lemma 4]{4}\label{1}
Let $P$ be a polyhedron. Any subclass $\mathcal{W}$ of the class of all CW-complexes dominated by $P$ can be partitioned into finitely many classes such that if $X$ and $Y$ belong to the same class, then there exists a finite sequence $\{ X_j \}_{j=0}^{m}$,  where $X_0 = X$, $X_m = Y$ , $X_j \in \mathcal{W}$, such that the map $g_{XY}=d_Y k_{X_{m-1}}k_{X_{m-2}}\cdots k_{X_2}k_{X_1}u_X :X\to Y$ is a homology equivalence.
\end{lemma}
In what follows, $\tilde{X}$ will denote, as usual notation, the universal covering space of $X$.  Let $f:X\to Y$ be a cellular map of CW-complexes such that $f(x)=y$, for vertices $x\in X$, $y\in Y$. Choose $\tilde{x}\in p^{-1}(x)$, $\tilde{y}\in q^{-1}(y)$, where $p:\tilde{X}\to X$ and $q:\tilde{Y}\to Y$ denote the covering maps. Then it is well-known that there exists a unique map $\tilde{f}:\tilde{X}\to \tilde{Y}$ such that $q\tilde{f}=fp$ and $f(\tilde{x})=\tilde{y}$. Using this notation we have the following relationship between dominations of a polyhedron  and its universal covering space.
\begin{lemma}\label{lem1}
Let $X \leqslant P$, where $P$ is a polyhedron. Then $\tilde{X}\leqslant \tilde{P}$. 
\end{lemma}
\begin{proof}
Assume that $X\leqslant P$ and $P$ is a finite polyhedron. Let $d_X:P \to X$ and $u_X :X\to P$ be the domination of $P$ over $X$ and the inverse map, i.e., $d_X u_X\simeq \textrm{id}_{X}$. By the lifting criterion,  there exist continuous maps $\tilde{d}_X:\tilde{P}\to \tilde{X}$ and $\tilde{u}_X:\tilde{X}\to \tilde{P}$ such that $q\tilde{d}_X=d_X p$ and $p\tilde{u}_X=u_X q$, where $p:\tilde{P}\to P$ and $q:\tilde{X}\to X$ are covering maps.  Then we have $q\tilde{d}_X\tilde{u}_X\simeq q$. Suppose that the homotopy between $q \tilde{d}_X\tilde{u}_X$ and $q$ is given by some $F :\tilde{X} \times I \to X$. The restriction of $F$ to $\tilde{X} \times \{0\}$ lifts to some map $\tilde{X} \times \{0\} \to \tilde{X}$. By uniqueness of lifts, we deduce that this map must be $\tilde{d}_X\tilde{u}_X$. Then we know that there exists a unique homotopy $\tilde{F} : \tilde{X} \times I \to \tilde{X}$ such that $\tilde F|_{\tilde{X} \times \{0\}} = \tilde{d}_X\tilde{u}_X$ and $q \tilde{F} = F$. Now we have that $\tilde{F}(x,0) = \tilde{d}_X\tilde{u}_X$, $q \tilde{F}(x,1) = F(x,1) = q$. However we have by uniqueness of lifts that $\tilde{F}(x,1) = \textrm{id}_{\tilde{X}}$, and so $\tilde{F}$ is a homotopy between $\tilde{d}_X \tilde{u}_X$ and $\textrm{id}_{\tilde{X}}$. This ends the proof of the lemma.
\end{proof}
Now we are in a position to state and prove the main result of the paper.
\begin{theorem}\label{main}
Let $P$ be a polyhedron with finite-by-cyclic fundamental group $\pi_1 (P)$.   Then $P$ dominates only finitely many different homotopy types. 
\end{theorem}
\begin{proof}
Consider all the CW-complexes $X\leqslant P$, each with a fixed domination $d_X :P\to X$ and inverse map $u_X :X\to P$, where $P$ is a polyhedron with finite-by-cyclic fundamental group $\pi_1 (P)$. Note that the universal covering space $\tilde{P}$ of $P$ is a polyhedron. 

Let $\tilde{d}_X:\tilde{P}\to \tilde{X}$ and $\tilde{u}_X:\tilde{X}\to \tilde{P}$ be liftings of the maps $d_X$ and $u_X$, respectively, to the universal coverings. Then $\tilde{d}_X:\tilde{P}\to \tilde{X}$ is a domination of $\tilde{P}$ over $\tilde{X}$ with inverse map $\tilde{u}_X$ (by Lemma \ref{lem1}), $\tilde{k}_X=\tilde{u}_X \tilde{d}_X:\tilde{P}\to \tilde{P}$ is a homotopy idempotent. From now on, writing $\tilde{X}\leqslant \tilde{P}$, we will have in mind the domination $\tilde{d}_X$ of $\tilde{P}$ over $\tilde{X}$. 

By Lemma \ref{2}, $\pi_1 (P)$ has only finitely many retracts . Accordingly, by the fact that $\textrm{im} \pi_1 (u_X)$ is a retract of $\pi_1 (P)$ when $X\leqslant P$,  all the CW-complexes dominated by $P$ can be partitioned into finitely many classes, $\mathcal{W}_1 ,\ldots ,\mathcal{W}_n$, such that $X$ and $Y$ belong to the same class if and only if $\textrm{im} \pi_1 (u_X) = \textrm{im} \pi_1 (u_Y)$, where $\pi_1 (u_X) : \pi_1(X) \to \pi_1(P)$ and $\pi_1(u_Y ) : \pi_1(Y ) \to \pi_1(P)$ are the homomorphisms induced by $u_X$ and $u_Y$, respectively.

Let $\mathcal{W}_i (1\leq i\leq n)$ be a fixed class of this partition. Let  $\tilde{\mathcal{W}}_i (1\leq i\leq n)$ be the class of all CW-complexes $\tilde{X}\leqslant \tilde{P}$ for $X\in \mathcal{W}_i$. Then  $\tilde{\mathcal{W}}_i$ is a subclass of the class of all CW-complexes dominated by $\tilde{P}$.  Now  by Lemma \ref{1},  we can divide $\tilde{\mathcal{W}}_i$ into finitely many classes, $\tilde{\mathcal{W}}_{i_1},\ldots ,\tilde{\mathcal{W}}_{i_{l_i}}$, such that if $\tilde{X}$ and $\tilde{Y}$ belong to the same class, then there exists a finite sequence $\{ \tilde{X}_j \}_{j=0}^{m}$, $m\geq 1$,  where $\tilde{X}_0 = \tilde{X}$, $\tilde{X}_m = \tilde{Y}$ , $\tilde{X}_j \in \tilde{\mathcal{W}}_{i_{k_i}} (1\leq k_i\leq l_i)$, such that the map $\tilde{g}_{XY}=\tilde{d}_{Y} \tilde{k}_{X_{m-1}}\tilde{k}_{X_{m-2}}\cdots \tilde{k}_{X_2}\tilde{k}_{X_1}\tilde{u}_{X} :\tilde{X}\to \tilde{Y}$ is a homology equivalence.

Now let $\mathcal{W}_{i_{k_i}} (1\leq k_i\leq l_i)$  be the class of all CW-complexes $X\leqslant P$ for $\tilde{X}\in \tilde{\mathcal{W}}_{i_{k_i}}$,  and $X,Y\in \mathcal{W}_{i_{k_i}}$. Then by the fact that  $\tilde{X},\tilde{Y}\in \tilde{\mathcal{W}}_{i_{k_i}}$, the map $\tilde{g}_{XY}=\tilde{d}_{Y} \tilde{k}_{X_{m-1}}\tilde{k}_{X_{m-2}}\cdots \tilde{k}_{X_2}\tilde{k}_{X_1}\tilde{u}_{X} :\tilde{X}\to \tilde{Y}$ is a homology equivalence. Since $X_j \in \mathcal{W}_j$ $(j=0,1,\ldots ,m)$,  $\textrm{im}\pi_1 (u_{X_j})$ in $\pi_1 (P)$ are the same for all the $X_j$ and using this fact that $\pi_1 (P)=\textrm{im}\pi_1 (u_{X_j})\ltimes \mathrm{ker}\pi_1(d_{X_j})$ for $j=0,1,\ldots ,m$ one can conclude that   the map $g_{XY}=d_Y k_{X_{m-1}}k_{X_{m-2}}\cdots k_{X_2}k_{X_1}u_X :X\to Y$ induces an epimorphism $\pi_1 (g_{XY}):\pi_1 (X)\to \pi_1 (Y)$. Since $\pi_1 (X)\cong \textrm{im}\pi_1 (u_{X})=\textrm{im}\pi_1 (u_{Y})\cong \pi_1 (Y)$ and  that an epimorphism between isomorphic Hopfian groups is an isomorphism, we  conclude that $\pi_1 (g_{XY}):\pi_1 (X)\to \pi_1 (Y)$ is an isomorphism (note that finite-by-cyclic groups are Hopfian). Hence, $X$ and $Y$ have the same homotopy type by the Whitehead Theorem ( see \cite{Whi}, if $X$ and $Y$ are CW-complexes and there exists a map $f : X \to Y$ such that $f$ induces an isomorphism $\pi_1 (f) : \pi_1 (X) \to \pi_1(Y )$ and isomorphisms $H_i (\tilde{f}):H_i (\tilde{X};\mathbb{Z})\to H_i (\tilde{Y};\mathbb{Z})$ for all $i\in \mathbb{N}$, then $f$ is a homotopy equivalence). This implies that each $\mathcal{W}_{i_{k_i}} (1\leq k_i\leq l_i)$ contains only one homotopy type of a CW-complex dominated by $P$. Now by the fact that $\mathcal{W}_i =\bigcup_{k_i=1}^{l_i} \mathcal{W}_{i_{k_i}} (1\leq i\leq n)$, we conclude that each $\mathcal{W}_i (1\leq i\leq n)$ contains only finitely many different homotopy types of CW-complexes dominated by $P$, hence $P$ dominates only finitely many different homotopy types. 
\end{proof}
\begin{example}
A simple example of a polyhedron $P$ with finite-by-cyclic fundamental group is   $P=S^1\vee S^n$ $(n\geq 2)$, which has finite capacity by Theorem \ref{main}. It should be emphasized  that this result can not be concluded from \cite{4,5} since $\pi_1 (P)$ is an infinite abelian group and $H_n (\tilde{P};\mathbb{Z})$ is not finitely generated. 
\end{example}
Since there exist polyhedra with nilpotent fundamental groups homotopy dominating infinitely many different homotopy types \cite[Corollary 5]{3}, Ko\l{}odziejczyk asked the following question (see \cite{5}): Does every polyhedron $P$ with the abelian fundamental group $\pi_1 (P)$ dominate only finitely many different homotopy types? In \cite{5} she proved that every polyhedron $P$ with abelian fundamental group $\pi_1 (P)$ and finitely generated homology groups $H_i (\tilde{P};\mathbb{Z})$ for $i\geq 2$, where $\tilde{P}$ is the universal covering of $P$,  dominates only finitely many different homotopy types.  In the next corollary,  we give a partial positive answer to the above question. 
\begin{corollary}\label{coro}
Every polyhedron $P$ with abelian fundamental group $\pi_1 (P)$ of rank 1 dominates only finitely many different homotopy types.
\end{corollary}
\begin{proof}
Since $\pi_1 (P)$ is an abelian group of rank one, it is of the form  $\pi_1 (P)\cong G\oplus \mathbb{Z}$, where $G$ is a finite abelian group. Then $\pi_1 (P)$ is a finite-by-cyclic group, and so $P$ has finite capacity by  Theorem \ref{main}.
\end{proof}
\begin{remark}
It should be noted that one can not prove the same result as Corollary \ref{coro} for polyhedra with  abelian fundamental group of rank  greater than 1 by the same  technique applied in Theorem \ref{main}.  To illustrate this point, consider the finite polyhedron $P=S^1 \times S^1$ with  $\pi_1 (P)\cong \mathbb{Z}\oplus \mathbb{Z}$. Take $H_n=\langle (1,n)\rangle$ for any integer $n$ (with $K=\langle (0,1)\rangle$). Then we have  $\mathbb{Z}\oplus \mathbb{Z}=H_n \oplus K$ which shows that $\pi_1 (P)$ has infinitely many different retracts. Thus we are interested to know the answer to the following question: 

``Does every polyhedron $P$ with the abelian fundamental group $\pi_1 (P)$ of rank 2 dominate only finitely many different homotopy types?"
\end{remark}
 A nontrivial group  is said to be \textit{splitting-simple} if it has no proper nontrivial retracts. Clearly every simple group is splitting-simple. However, the infinite cyclic group $\mathbb{Z}$ is a splitting-simple group which is not simple.
 \begin{lemma}\label{lem2} ${\rm (i)}$ Every splitting-simple group is weakly Hopfian. 

${\rm (ii)}$ Let $r:G \to H$ be an r-homomorphism such that $G$ is weakly Hopfian. If $G$ and $H$ are isomorphic, then $r$ is an isomorphism. 
\end{lemma}
\begin{proof}
${\rm (i)}$ Assume that $G$ is a nontrivial splitting-simple group,  $G=N\rtimes H$ for some $N\lhd G$, and $H\cong G$. Since $G$ is  splitting-simple, $H=1$ or $H=G$. If $H=1$, then $G=1$ which is a contradiction. So $H=G$ which implies that $N=1$. Thus $G$ is weakly Hopfian.

${\rm (ii)}$ By the hypothesis, there exists an isomorphism $f:H \to G$.  Then $f\circ r:G \to G$ is an r-homomorphism. Since $G$ is weakly Hopfian, $f\circ  r$ is an isomorphism which follows that $r$ is an isomorphism.
\end{proof}
\begin{remark}
Note that a splitting-simple group is not Hopfian generally. As an illustration, the Pr\"ufer $p$-group is a non-Hopfian group, but it has no proper nontrivial retracts,  since each of its subgroups is finite and any homomorphic image is a divisible group (for the definition of a Pr\"ufer $p$-group, see \cite[p. 94]{Rob}).
\end{remark}
\begin{lemma}\label{lem3}
Let $P$ be a polyhedron with weakly Hopfian $\pi_1 (P)$ and $X\leqslant P$ such that $\pi_1 (P)\cong \pi_1 (X)$. Then $d_X:P \to X$ and $u_X :X\to P$ induce isomorphisms on fundamental groups. 
\end{lemma}
\begin{proof}
Since $d_X u_X \simeq  \textrm{id}_X$, we have $\pi_1 (d_X )\pi_1 (u_X)=\textrm{id}_{\pi_1 (X)}$. Then $\pi_1 (d_X ):\pi_1 (P)\to \pi_1 (X)$ is an r-homomorphism. So $\pi_1 (d_X ):\pi_1 (P)\to \pi_1 (X)$ is an isomorphism by the hypothesis and Lemma \ref{lem2}. Moreover, by the fact that  $\pi_1 (d_X )\pi_1 (u_X)=\textrm{id}_{\pi_1 (X)}$, $\pi_1 (u_X ):\pi_1 (X)\to \pi_1 (P)$ is also an isomorphism
\end{proof}
  Using the main idea of the proof of Theorem \ref{main} we have the following result.
\begin{theorem}\label{splitting}
Let $P$ be a  polyhedron with splitting-simple fundamental group. Then $P$ dominates only finitely many different homotopy types. 
\end{theorem}
\begin{proof}
Consider all the CW-complexes $X\leqslant P$, each with a fixed domination $d_X : P \to X$ and the inverse map $u_X : X \to P$, where $P$ is a polyhedron with   splitting-simple fundamental group $\pi_1 (P)$.  

By the fact that $\pi_1 (P)$ is splitting-simple,  all the CW-complexes dominated by $P$ can be partitioned into two classes, $\mathcal{W}_1$ the class of the CW-complexes $X$  with $\textrm{im} \pi_1 (u_X) =1$ and $\mathcal{W}_2$ the class of the CW-complexes $X$  with $\textrm{im} \pi_1 (u_X) = \pi_1 (P)$. Now by a similar argument with the proof of Theorem \ref{main} and the fact that the map $g_{XY}=d_Y k_{X_{m-1}}k_{X_{m-2}}\cdots k_{X_2}k_{X_1}u_X :X\to Y$ induces an isomorphism $\pi_1 (g_{XY}):\pi_1 (X)\to \pi_1 (Y)$ (by Lemmas \ref{lem2} and \ref{lem3}), we can prove that each of $\mathcal{W}_1$ and $\mathcal{W}_2$ can divide into finitely many classes such that if $X$ and $Y$ belong to the same class, they have the same homotopy types. Thus the proof is completed. 
\end{proof}
\begin{example}
Note that Higman groups $G_{n,r} $ are finitely presented infinite groups which are simple  when $n$ is even (see \cite{Hig}). Then  polyhedra $P$ with fundamental group $G_{n,r} $  are examples of finite polyhedra with splitting-simple fundamental group whose capacities are finite by the above theorem. Note that this result can not be followed from \cite{4,5}.
\end{example}
\section{The class of simply connected CW-complexes homotopy dominated by a polyhedron}
The question of Borsuk, as we mentioned earlier, was about whether the class of all CW-complexes homotopy dominated by a polyhedron is finite,  which was answered negatively in general by Ko\l{}odziejczyk in \cite{1}.  It seems interesting to ask which subclass of all CW-complexes homotopy dominated by a polyhedron is finite.
In this section, we intend to show that  the subclass consist of simply connected CW-complexes homotopy dominated by a polyhedron is finite. 

C.T.C. Wall \cite[the proof of Theorem A, implication (ii) $\Rightarrow$ (i), p. 60]{Wall} proved that if $X\leqslant P$, where $P$ is a finite CW-complex, then there exists a finite CW-complex $\tilde{P}$ such that $X\leqslant \tilde{P}$ and $\pi_1 (\tilde{P})\cong \pi_1 (X)$ (see also \cite[Lemma 2]{6}). In the next proposition, by presenting a shorter proof, we  show that $\tilde{P}$ can be a covering space of $P$. 
\begin{proposition}\label{pro}
Let $X\leqslant P$, where $P$ is a finite CW-complex, then there exists a finite CW-complex $\tilde{P}$ such that $X\leqslant \tilde{P}$ and $\pi_1 (\tilde{P})\cong \pi_1 (X)$. Moreover, $\tilde{P}$ is a covering space of $P$.
\end{proposition}
\begin{proof}
Assume that  $u_X:X \to P$ is the inverse map.   Since $P$  is path connected, locally path connected, and semilocally simply connected, for the subgroup $\textrm{im} \pi_1 (u_X) \subset \pi_1 (P)$ there is a covering space  $(\tilde{P}  ,p )$ such that $\textrm{im} \pi_1 (p) =\textrm{im} \pi_1 (u_X) $ (see \cite[Proposition 1.36]{Hat}). Now by the lifting criterion,    there exists a unique map $\tilde{u}_X:X\to \tilde{P}$ such that $p\tilde{u}_X=u_X$. Put $\tilde{d}_X:=d_X p:\tilde{P} \to X$, where $d_X :P\to X$ is the domination map. Then we have $\tilde{d}_X\tilde{u}_X=d_X p\tilde{u}_X=d_X u_X \simeq \textrm{id}_{X}$. This implies that $X\leqslant \tilde{P}$. Note that since $\pi_1 (u_X) $ and $ \pi_1 (p) $ are monomorphisms, we have $\pi_1 (\tilde{P})\cong  \textrm{im} \pi_1 (p) =\textrm{im} \pi_1 (u_X)  \cong \pi_1 (X)$.
\end{proof}
Now, we can state and prove the main result of this section.
\begin{theorem}\label{thm1}
Every polyhedron dominates only finitely many homotopy types of simply connected CW-complexes.
\end{theorem}
\begin{proof}
Assume that $X\leqslant P$ where $\pi_1 (X)=1$ and $P$ is a  polyhedron. Then by Proposition \ref{pro}, there is  a finite simply connected CW-complex $\tilde{P}$ such that $X\leqslant \tilde{P}$.  Therefore, the class of all  simply connected CW-complexes dominated by $P$ is a subclass of all  CW-complexes  dominated by  $\tilde{P}$.  Now since $\tilde{P}$ is  simply connected, the capacity of   $\tilde{P}$ is finite (see \cite{2}), and so the result holds.  
\end{proof}
\begin{thebibliography}{99}
\bibitem{So}   Borsuk, K. \textit{Some problems in the theory of shape of compacta}, Russian Math. Surveys \textbf{34} (6), 24-26, 1979.
\bibitem{Dy}   Dydak, J. and Segal, J. \textit{Shape Theory: An Introduction}, Lecture Notes in Math., vol. 688, Springer, Berlin, 1978.
\bibitem{Ed}  Edwards, D.A. and  Geoghegan, R. \textit{Shapes of complexes, ends of manifolds, homotopy limits and the Wall
obstruction}, Ann. of Math. \textbf{101} (2), 521-535, 1975.
\bibitem{Ha}  Hastings, H.M. and  Heller, A. \textit{Homotopy idempotents on finite-dimensional complexes split}, Proc. Amer. Math.
Soc.,  \textbf{85} (4),  619-622, 1982. 
\bibitem{Hat}   Hatcher, A. \textit{Algebraic Topology}, Cambridge University Press, 2002.
\bibitem{Hig} Higman, G. \textit{Finitely presented infinite simple groups}, Vol. 8. Department of Pure Mathematics, Department of Mathematics, IAS, Australian National University, 1974.
\bibitem{Hol}   Holsztynski, W.  \textit{A remark on homotopy and category domination}, Michigan Math. J., \textbf{18},  409, 1971.
\bibitem{1}    Ko\l{}odziejczyk, D. \textit{There exists a polyhedron dominating infinitely many different homotopy types}, Fund. Math. \textbf{151},  39-46, 1996.
\bibitem{2}  Ko\l{}odziejczyk, D. \textit{Simply-connected polyhedra dominate only finitely many different shapes}, Topology Appl. \textbf{112}, 289-295, 2001. 
\bibitem{3}    Ko\l{}odziejczyk, D.  \textit{Homotopy dominations within polyhedra}, Fund. Math. \textbf{178}, 189-202, 2003.
\bibitem{4}   Ko\l{}odziejczyk, D. \textit{Polyhedra with finite fundamental group dominate only finitely many different homotopy types}, Fund. Math. \textbf{180},  1-9, 2003.
\bibitem{5}  Ko\l{}odziejczyk, D. \textit{Polyhedra dominating finitely many different homotopy types}, Topology Appl., \textbf{146-147},  583-592, 2005.
\bibitem{6}    Ko\l{}odziejczyk, D. \textit{Homotopy dominations by polyhedra with polycyclic-by-finite fundamental groups}, Topology Appl. \textbf{153}, 294-302, 2005. 
\bibitem{Mather}  Mather, M. \textit{Counting homotopy types of manifolds}, Topology \textbf{4}, 93-94, 1965.
\bibitem{Rob}   Robinson, D.J.S. \textit{A Course in the Theory of Groups}, Springer, Berlin, 1982.
\bibitem{R}  Rotman, J.J. \textit{An Introduction to Algebraic Topology}, G.T.M. 119, Springer-Verlag, New York, 1988.
\bibitem{Wall}  Wall, C.T.C. \textit{Finiteness conditions for CW-complexes}, Ann. of Math. \textbf{81},  56-69, 1965.
\bibitem{Whi}  Whitehead, G. \textit{Elements of Homotopy Theory}, Springer, New York, 1978.
\end{thebibliography}
\end{document}
\typeout{get arXiv to do 4 passes: Label(s) may have changed. Rerun} 