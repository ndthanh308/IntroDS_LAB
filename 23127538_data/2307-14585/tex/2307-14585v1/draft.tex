\documentclass[aps,prd,twocolumn,showpacs,amsmath,amssymb]{revtex4-1}
\usepackage{amsmath}
\usepackage{graphicx}
\usepackage{subfigure}
\usepackage{epstopdf}
\usepackage{color}
\usepackage{multirow}
\usepackage{setspace}
\usepackage{overpic}
\usepackage{amssymb}
\usepackage[bookmarksnumbered, pdfstartview=FitH,colorlinks,urlcolor=blue, citecolor=blue,linkcolor=blue] {hyperref}
\usepackage{lineno}
\usepackage{bm}
\usepackage{rotating}
\usepackage{siunitx}
\usepackage[utf8]{inputenc}

\hyphenpenalty=5000 \tolerance=100

\setlength{\oddsidemargin}{-0.5cm}
\setlength{\oddsidemargin}{-0.5cm} \addtolength{\topmargin}{10mm}
\hoffset -0.2 in

\renewcommand\figurename{\rm Fig.}
\renewcommand\tablename{\rm Table}
\renewcommand{\thetable}{\arabic{table}}

\let\oldequation\equation
\let\oldendequation\endequation

\renewenvironment{equation}
  {\linenomathNonumbers\oldequation}
  {\oldendequation\endlinenomath}

\oddsidemargin  -0.2cm
\evensidemargin -0.2cm

\begin{document}

%\linenumbers

\title{\bf \boldmath{
Improved measurement of the branching fraction of $D_s^+\to\mu^+\nu_\mu$}{}
}

%\input{Source/authorlist_2023-01-19}

\author{
M.~Ablikim$^{1}$, M.~N.~Achasov$^{13,b}$, P.~Adlarson$^{75}$, X.~C.~Ai$^{81}$, R.~Aliberti$^{36}$, A.~Amoroso$^{74A,74C}$, M.~R.~An$^{40}$, Q.~An$^{71,58}$, Y.~Bai$^{57}$, O.~Bakina$^{37}$, I.~Balossino$^{30A}$, Y.~Ban$^{47,g}$, V.~Batozskaya$^{1,45}$, K.~Begzsuren$^{33}$, N.~Berger$^{36}$, M.~Berlowski$^{45}$, M.~Bertani$^{29A}$, D.~Bettoni$^{30A}$, F.~Bianchi$^{74A,74C}$, E.~Bianco$^{74A,74C}$, J.~Bloms$^{68}$, A.~Bortone$^{74A,74C}$, I.~Boyko$^{37}$, R.~A.~Briere$^{5}$, A.~Brueggemann$^{68}$, H.~Cai$^{76}$, X.~Cai$^{1,58}$, A.~Calcaterra$^{29A}$, G.~F.~Cao$^{1,63}$, N.~Cao$^{1,63}$, S.~A.~Cetin$^{62A}$, J.~F.~Chang$^{1,58}$, T.~T.~Chang$^{77}$, W.~L.~Chang$^{1,63}$, G.~R.~Che$^{44}$, G.~Chelkov$^{37,a}$, C.~Chen$^{44}$, Chao~Chen$^{55}$, G.~Chen$^{1}$, H.~S.~Chen$^{1,63}$, M.~L.~Chen$^{1,58,63}$, S.~J.~Chen$^{43}$, S.~M.~Chen$^{61}$, T.~Chen$^{1,63}$, X.~R.~Chen$^{32,63}$, X.~T.~Chen$^{1,63}$, Y.~B.~Chen$^{1,58}$, Y.~Q.~Chen$^{35}$, Z.~J.~Chen$^{26,h}$, W.~S.~Cheng$^{74C}$, S.~K.~Choi$^{10A}$, X.~Chu$^{44}$, G.~Cibinetto$^{30A}$, S.~C.~Coen$^{4}$, F.~Cossio$^{74C}$, J.~J.~Cui$^{50}$, H.~L.~Dai$^{1,58}$, J.~P.~Dai$^{79}$, A.~Dbeyssi$^{19}$, R.~ E.~de Boer$^{4}$, D.~Dedovich$^{37}$, Z.~Y.~Deng$^{1}$, A.~Denig$^{36}$, I.~Denysenko$^{37}$, M.~Destefanis$^{74A,74C}$, F.~De~Mori$^{74A,74C}$, B.~Ding$^{66,1}$, X.~X.~Ding$^{47,g}$, Y.~Ding$^{41}$, Y.~Ding$^{35}$, J.~Dong$^{1,58}$, L.~Y.~Dong$^{1,63}$, M.~Y.~Dong$^{1,58,63}$, X.~Dong$^{76}$, S.~X.~Du$^{81}$, Z.~H.~Duan$^{43}$, P.~Egorov$^{37,a}$, Y.~L.~Fan$^{76}$, J.~Fang$^{1,58}$, S.~S.~Fang$^{1,63}$, W.~X.~Fang$^{1}$, Y.~Fang$^{1}$, R.~Farinelli$^{30A}$, L.~Fava$^{74B,74C}$, F.~Feldbauer$^{4}$, G.~Felici$^{29A}$, C.~Q.~Feng$^{71,58}$, J.~H.~Feng$^{59}$, K~Fischer$^{69}$, M.~Fritsch$^{4}$, C.~Fritzsch$^{68}$, C.~D.~Fu$^{1}$, J.~L.~Fu$^{63}$, Y.~W.~Fu$^{1}$, H.~Gao$^{63}$, Y.~N.~Gao$^{47,g}$, Yang~Gao$^{71,58}$, S.~Garbolino$^{74C}$, I.~Garzia$^{30A,30B}$, P.~T.~Ge$^{76}$, Z.~W.~Ge$^{43}$, C.~Geng$^{59}$, E.~M.~Gersabeck$^{67}$, A~Gilman$^{69}$, K.~Goetzen$^{14}$, L.~Gong$^{41}$, W.~X.~Gong$^{1,58}$, W.~Gradl$^{36}$, S.~Gramigna$^{30A,30B}$, M.~Greco$^{74A,74C}$, M.~H.~Gu$^{1,58}$, Y.~T.~Gu$^{16}$, C.~Y~Guan$^{1,63}$, Z.~L.~Guan$^{23}$, A.~Q.~Guo$^{32,63}$, L.~B.~Guo$^{42}$, M.~J.~Guo$^{50}$, R.~P.~Guo$^{49}$, Y.~P.~Guo$^{12,f}$, A.~Guskov$^{37,a}$, T.~T.~Han$^{50}$, W.~Y.~Han$^{40}$, X.~Q.~Hao$^{20}$, F.~A.~Harris$^{65}$, K.~K.~He$^{55}$, K.~L.~He$^{1,63}$, F.~H~H..~Heinsius$^{4}$, C.~H.~Heinz$^{36}$, Y.~K.~Heng$^{1,58,63}$, C.~Herold$^{60}$, T.~Holtmann$^{4}$, P.~C.~Hong$^{12,f}$, G.~Y.~Hou$^{1,63}$, X.~T.~Hou$^{1,63}$, Y.~R.~Hou$^{63}$, Z.~L.~Hou$^{1}$, H.~M.~Hu$^{1,63}$, J.~F.~Hu$^{56,i}$, T.~Hu$^{1,58,63}$, Y.~Hu$^{1}$, G.~S.~Huang$^{71,58}$, K.~X.~Huang$^{59}$, L.~Q.~Huang$^{32,63}$, X.~T.~Huang$^{50}$, Y.~P.~Huang$^{1}$, T.~Hussain$^{73}$, N~H\"usken$^{28,36}$, W.~Imoehl$^{28}$, M.~Irshad$^{71,58}$, J.~Jackson$^{28}$, S.~Jaeger$^{4}$, S.~Janchiv$^{33}$, J.~H.~Jeong$^{10A}$, Q.~Ji$^{1}$, Q.~P.~Ji$^{20}$, X.~B.~Ji$^{1,63}$, X.~L.~Ji$^{1,58}$, Y.~Y.~Ji$^{50}$, X.~Q.~Jia$^{50}$, Z.~K.~Jia$^{71,58}$, P.~C.~Jiang$^{47,g}$, S.~S.~Jiang$^{40}$, T.~J.~Jiang$^{17}$, X.~S.~Jiang$^{1,58,63}$, Y.~Jiang$^{63}$, J.~B.~Jiao$^{50}$, Z.~Jiao$^{24}$, S.~Jin$^{43}$, Y.~Jin$^{66}$, M.~Q.~Jing$^{1,63}$, T.~Johansson$^{75}$, X.~K.$^{1}$, S.~Kabana$^{34}$, N.~Kalantar-Nayestanaki$^{64}$, X.~L.~Kang$^{9}$, X.~S.~Kang$^{41}$, R.~Kappert$^{64}$, M.~Kavatsyuk$^{64}$, B.~C.~Ke$^{81}$, A.~Khoukaz$^{68}$, R.~Kiuchi$^{1}$, R.~Kliemt$^{14}$, O.~B.~Kolcu$^{62A}$, B.~Kopf$^{4}$, M.~K.~Kuessner$^{4}$, A.~Kupsc$^{45,75}$, W.~K\"uhn$^{38}$, J.~J.~Lane$^{67}$, P. ~Larin$^{19}$, A.~Lavania$^{27}$, L.~Lavezzi$^{74A,74C}$, T.~T.~Lei$^{71,k}$, Z.~H.~Lei$^{71,58}$, H.~Leithoff$^{36}$, M.~Lellmann$^{36}$, T.~Lenz$^{36}$, C.~Li$^{48}$, C.~Li$^{44}$, C.~H.~Li$^{40}$, Cheng~Li$^{71,58}$, D.~M.~Li$^{81}$, F.~Li$^{1,58}$, G.~Li$^{1}$, H.~Li$^{71,58}$, H.~B.~Li$^{1,63}$, H.~J.~Li$^{20}$, H.~N.~Li$^{56,i}$, Hui~Li$^{44}$, J.~R.~Li$^{61}$, J.~S.~Li$^{59}$, J.~W.~Li$^{50}$, K.~L.~Li$^{20}$, Ke~Li$^{1}$, L.~J~Li$^{1,63}$, L.~K.~Li$^{1}$, Lei~Li$^{3}$, M.~H.~Li$^{44}$, P.~R.~Li$^{39,j,k}$, Q.~X.~Li$^{50}$, S.~X.~Li$^{12}$, T. ~Li$^{50}$, W.~D.~Li$^{1,63}$, W.~G.~Li$^{1}$, X.~H.~Li$^{71,58}$, X.~L.~Li$^{50}$, Xiaoyu~Li$^{1,63}$, Y.~G.~Li$^{47,g}$, Z.~J.~Li$^{59}$, Z.~X.~Li$^{16}$, C.~Liang$^{43}$, H.~Liang$^{1,63}$, H.~Liang$^{71,58}$, H.~Liang$^{35}$, Y.~F.~Liang$^{54}$, Y.~T.~Liang$^{32,63}$, G.~R.~Liao$^{15}$, L.~Z.~Liao$^{50}$, J.~Libby$^{27}$, A. ~Limphirat$^{60}$, D.~X.~Lin$^{32,63}$, T.~Lin$^{1}$, B.~J.~Liu$^{1}$, B.~X.~Liu$^{76}$, C.~Liu$^{35}$, C.~X.~Liu$^{1}$, F.~H.~Liu$^{53}$, Fang~Liu$^{1}$, Feng~Liu$^{6}$, G.~M.~Liu$^{56,i}$, H.~Liu$^{39,j,k}$, H.~B.~Liu$^{16}$, H.~M.~Liu$^{1,63}$, Huanhuan~Liu$^{1}$, Huihui~Liu$^{22}$, J.~B.~Liu$^{71,58}$, J.~L.~Liu$^{72}$, J.~Y.~Liu$^{1,63}$, K.~Liu$^{1}$, K.~Y.~Liu$^{41}$, Ke~Liu$^{23}$, L.~Liu$^{71,58}$, L.~C.~Liu$^{44}$, Lu~Liu$^{44}$, M.~H.~Liu$^{12,f}$, P.~L.~Liu$^{1}$, Q.~Liu$^{63}$, S.~B.~Liu$^{71,58}$, T.~Liu$^{12,f}$, W.~K.~Liu$^{44}$, W.~M.~Liu$^{71,58}$, X.~Liu$^{39,j,k}$, Y.~Liu$^{39,j,k}$, Y.~Liu$^{81}$, Y.~B.~Liu$^{44}$, Z.~A.~Liu$^{1,58,63}$, Z.~Q.~Liu$^{50}$, X.~C.~Lou$^{1,58,63}$, F.~X.~Lu$^{59}$, H.~J.~Lu$^{24}$, J.~G.~Lu$^{1,58}$, X.~L.~Lu$^{1}$, Y.~Lu$^{7}$, Y.~P.~Lu$^{1,58}$, Z.~H.~Lu$^{1,63}$, C.~L.~Luo$^{42}$, M.~X.~Luo$^{80}$, T.~Luo$^{12,f}$, X.~L.~Luo$^{1,58}$, X.~R.~Lyu$^{63}$, Y.~F.~Lyu$^{44}$, F.~C.~Ma$^{41}$, H.~L.~Ma$^{1}$, J.~L.~Ma$^{1,63}$, L.~L.~Ma$^{50}$, M.~M.~Ma$^{1,63}$, Q.~M.~Ma$^{1}$, R.~Q.~Ma$^{1,63}$, R.~T.~Ma$^{63}$, X.~Y.~Ma$^{1,58}$, Y.~Ma$^{47,g}$, Y.~M.~Ma$^{32}$, F.~E.~Maas$^{19}$, M.~Maggiora$^{74A,74C}$, S.~Malde$^{69}$, A.~Mangoni$^{29B}$, Y.~J.~Mao$^{47,g}$, Z.~P.~Mao$^{1}$, S.~Marcello$^{74A,74C}$, Z.~X.~Meng$^{66}$, J.~G.~Messchendorp$^{14,64}$, G.~Mezzadri$^{30A}$, H.~Miao$^{1,63}$, T.~J.~Min$^{43}$, R.~E.~Mitchell$^{28}$, X.~H.~Mo$^{1,58,63}$, N.~Yu.~Muchnoi$^{13,b}$, Y.~Nefedov$^{37}$, F.~Nerling$^{19,d}$, I.~B.~Nikolaev$^{13,b}$, Z.~Ning$^{1,58}$, S.~Nisar$^{11,l}$, Y.~Niu $^{50}$, S.~L.~Olsen$^{63}$, Q.~Ouyang$^{1,58,63}$, S.~Pacetti$^{29B,29C}$, X.~Pan$^{55}$, Y.~Pan$^{57}$, A.~~Pathak$^{35}$, P.~Patteri$^{29A}$, Y.~P.~Pei$^{71,58}$, M.~Pelizaeus$^{4}$, H.~P.~Peng$^{71,58}$, K.~Peters$^{14,d}$, J.~L.~Ping$^{42}$, R.~G.~Ping$^{1,63}$, S.~Plura$^{36}$, S.~Pogodin$^{37}$, V.~Prasad$^{34}$, F.~Z.~Qi$^{1}$, H.~Qi$^{71,58}$, H.~R.~Qi$^{61}$, M.~Qi$^{43}$, T.~Y.~Qi$^{12,f}$, S.~Qian$^{1,58}$, W.~B.~Qian$^{63}$, C.~F.~Qiao$^{63}$, J.~J.~Qin$^{72}$, L.~Q.~Qin$^{15}$, X.~P.~Qin$^{12,f}$, X.~S.~Qin$^{50}$, Z.~H.~Qin$^{1,58}$, J.~F.~Qiu$^{1}$, S.~Q.~Qu$^{61}$, C.~F.~Redmer$^{36}$, K.~J.~Ren$^{40}$, A.~Rivetti$^{74C}$, V.~Rodin$^{64}$, M.~Rolo$^{74C}$, G.~Rong$^{1,63}$, Ch.~Rosner$^{19}$, S.~N.~Ruan$^{44}$, N.~Salone$^{45}$, A.~Sarantsev$^{37,c}$, Y.~Schelhaas$^{36}$, K.~Schoenning$^{75}$, M.~Scodeggio$^{30A,30B}$, K.~Y.~Shan$^{12,f}$, W.~Shan$^{25}$, X.~Y.~Shan$^{71,58}$, J.~F.~Shangguan$^{55}$, L.~G.~Shao$^{1,63}$, M.~Shao$^{71,58}$, C.~P.~Shen$^{12,f}$, H.~F.~Shen$^{1,63}$, W.~H.~Shen$^{63}$, X.~Y.~Shen$^{1,63}$, B.~A.~Shi$^{63}$, H.~C.~Shi$^{71,58}$, J.~L.~Shi$^{12}$, J.~Y.~Shi$^{1}$, Q.~Q.~Shi$^{55}$, R.~S.~Shi$^{1,63}$, X.~Shi$^{1,58}$, J.~J.~Song$^{20}$, T.~Z.~Song$^{59}$, W.~M.~Song$^{35,1}$, Y. ~J.~Song$^{12}$, Y.~X.~Song$^{47,g}$, S.~Sosio$^{74A,74C}$, S.~Spataro$^{74A,74C}$, F.~Stieler$^{36}$, Y.~J.~Su$^{63}$, G.~B.~Sun$^{76}$, G.~X.~Sun$^{1}$, H.~Sun$^{63}$, H.~K.~Sun$^{1}$, J.~F.~Sun$^{20}$, K.~Sun$^{61}$, L.~Sun$^{76}$, S.~S.~Sun$^{1,63}$, T.~Sun$^{1,63}$, W.~Y.~Sun$^{35}$, Y.~Sun$^{9}$, Y.~J.~Sun$^{71,58}$, Y.~Z.~Sun$^{1}$, Z.~T.~Sun$^{50}$, Y.~X.~Tan$^{71,58}$, C.~J.~Tang$^{54}$, G.~Y.~Tang$^{1}$, J.~Tang$^{59}$, Y.~A.~Tang$^{76}$, L.~Y~Tao$^{72}$, Q.~T.~Tao$^{26,h}$, M.~Tat$^{69}$, J.~X.~Teng$^{71,58}$, V.~Thoren$^{75}$, W.~H.~Tian$^{59}$, W.~H.~Tian$^{52}$, Y.~Tian$^{32,63}$, Z.~F.~Tian$^{76}$, I.~Uman$^{62B}$, S.~J.~Wang $^{50}$, B.~Wang$^{1}$, B.~L.~Wang$^{63}$, Bo~Wang$^{71,58}$, C.~W.~Wang$^{43}$, D.~Y.~Wang$^{47,g}$, F.~Wang$^{72}$, H.~J.~Wang$^{39,j,k}$, H.~P.~Wang$^{1,63}$, J.~P.~Wang $^{50}$, K.~Wang$^{1,58}$, L.~L.~Wang$^{1}$, M.~Wang$^{50}$, Meng~Wang$^{1,63}$, S.~Wang$^{39,j,k}$, S.~Wang$^{12,f}$, T. ~Wang$^{12,f}$, T.~J.~Wang$^{44}$, W.~Wang$^{59}$, W. ~Wang$^{72}$, W.~P.~Wang$^{71,58}$, X.~Wang$^{47,g}$, X.~F.~Wang$^{39,j,k}$, X.~J.~Wang$^{40}$, X.~L.~Wang$^{12,f}$, Y.~Wang$^{61}$, Y.~D.~Wang$^{46}$, Y.~F.~Wang$^{1,58,63}$, Y.~H.~Wang$^{48}$, Y.~N.~Wang$^{46}$, Y.~Q.~Wang$^{1}$, Yaqian~Wang$^{18,1}$, Yi~Wang$^{61}$, Z.~Wang$^{1,58}$, Z.~L. ~Wang$^{72}$, Z.~Y.~Wang$^{1,63}$, Ziyi~Wang$^{63}$, D.~Wei$^{70}$, D.~H.~Wei$^{15}$, F.~Weidner$^{68}$, S.~P.~Wen$^{1}$, C.~W.~Wenzel$^{4}$, U.~W.~Wiedner$^{4}$, G.~Wilkinson$^{69}$, M.~Wolke$^{75}$, L.~Wollenberg$^{4}$, C.~Wu$^{40}$, J.~F.~Wu$^{1,63}$, L.~H.~Wu$^{1}$, L.~J.~Wu$^{1,63}$, X.~Wu$^{12,f}$, X.~H.~Wu$^{35}$, Y.~Wu$^{71}$, Y.~J.~Wu$^{32}$, Z.~Wu$^{1,58}$, L.~Xia$^{71,58}$, X.~M.~Xian$^{40}$, T.~Xiang$^{47,g}$, D.~Xiao$^{39,j,k}$, G.~Y.~Xiao$^{43}$, H.~Xiao$^{12,f}$, S.~Y.~Xiao$^{1}$, Y. ~L.~Xiao$^{12,f}$, Z.~J.~Xiao$^{42}$, C.~Xie$^{43}$, X.~H.~Xie$^{47,g}$, Y.~Xie$^{50}$, Y.~G.~Xie$^{1,58}$, Y.~H.~Xie$^{6}$, Z.~P.~Xie$^{71,58}$, T.~Y.~Xing$^{1,63}$, C.~F.~Xu$^{1,63}$, C.~J.~Xu$^{59}$, G.~F.~Xu$^{1}$, H.~Y.~Xu$^{66}$, Q.~J.~Xu$^{17}$, Q.~N.~Xu$^{31}$, W.~Xu$^{1,63}$, W.~L.~Xu$^{66}$, X.~P.~Xu$^{55}$, Y.~C.~Xu$^{78}$, Z.~P.~Xu$^{43}$, Z.~S.~Xu$^{63}$, F.~Yan$^{12,f}$, L.~Yan$^{12,f}$, W.~B.~Yan$^{71,58}$, W.~C.~Yan$^{81}$, X.~Q.~Yan$^{1}$, H.~J.~Yang$^{51,e}$, H.~L.~Yang$^{35}$, H.~X.~Yang$^{1}$, Tao~Yang$^{1}$, Y.~Yang$^{12,f}$, Y.~F.~Yang$^{44}$, Y.~X.~Yang$^{1,63}$, Yifan~Yang$^{1,63}$, Z.~W.~Yang$^{39,j,k}$, Z.~P.~Yao$^{50}$, M.~Ye$^{1,58}$, M.~H.~Ye$^{8}$, J.~H.~Yin$^{1}$, Z.~Y.~You$^{59}$, B.~X.~Yu$^{1,58,63}$, C.~X.~Yu$^{44}$, G.~Yu$^{1,63}$, J.~S.~Yu$^{26,h}$, T.~Yu$^{72}$, X.~D.~Yu$^{47,g}$, C.~Z.~Yuan$^{1,63}$, L.~Yuan$^{2}$, S.~C.~Yuan$^{1}$, X.~Q.~Yuan$^{1}$, Y.~Yuan$^{1,63}$, Z.~Y.~Yuan$^{59}$, C.~X.~Yue$^{40}$, A.~A.~Zafar$^{73}$, F.~R.~Zeng$^{50}$, X.~Zeng$^{12,f}$, Y.~Zeng$^{26,h}$, Y.~J.~Zeng$^{1,63}$, X.~Y.~Zhai$^{35}$, Y.~C.~Zhai$^{50}$, Y.~H.~Zhan$^{59}$, A.~Q.~Zhang$^{1,63}$, B.~L.~Zhang$^{1,63}$, B.~X.~Zhang$^{1}$, D.~H.~Zhang$^{44}$, G.~Y.~Zhang$^{20}$, H.~Zhang$^{71}$, H.~H.~Zhang$^{59}$, H.~H.~Zhang$^{35}$, H.~Q.~Zhang$^{1,58,63}$, H.~Y.~Zhang$^{1,58}$, J.~J.~Zhang$^{52}$, J.~L.~Zhang$^{21}$, J.~Q.~Zhang$^{42}$, J.~W.~Zhang$^{1,58,63}$, J.~X.~Zhang$^{39,j,k}$, J.~Y.~Zhang$^{1}$, J.~Z.~Zhang$^{1,63}$, Jianyu~Zhang$^{63}$, Jiawei~Zhang$^{1,63}$, L.~M.~Zhang$^{61}$, L.~Q.~Zhang$^{59}$, Lei~Zhang$^{43}$, P.~Zhang$^{1}$, Q.~Y.~~Zhang$^{40,81}$, Shuihan~Zhang$^{1,63}$, Shulei~Zhang$^{26,h}$, X.~D.~Zhang$^{46}$, X.~M.~Zhang$^{1}$, X.~Y.~Zhang$^{50}$, X.~Y.~Zhang$^{55}$, Y.~Zhang$^{69}$, Y. ~Zhang$^{72}$, Y. ~T.~Zhang$^{81}$, Y.~H.~Zhang$^{1,58}$, Yan~Zhang$^{71,58}$, Yao~Zhang$^{1}$, Z.~H.~Zhang$^{1}$, Z.~L.~Zhang$^{35}$, Z.~Y.~Zhang$^{44}$, Z.~Y.~Zhang$^{76}$, G.~Zhao$^{1}$, J.~Zhao$^{40}$, J.~Y.~Zhao$^{1,63}$, J.~Z.~Zhao$^{1,58}$, Lei~Zhao$^{71,58}$, Ling~Zhao$^{1}$, M.~G.~Zhao$^{44}$, S.~J.~Zhao$^{81}$, Y.~B.~Zhao$^{1,58}$, Y.~X.~Zhao$^{32,63}$, Z.~G.~Zhao$^{71,58}$, A.~Zhemchugov$^{37,a}$, B.~Zheng$^{72}$, J.~P.~Zheng$^{1,58}$, W.~J.~Zheng$^{1,63}$, Y.~H.~Zheng$^{63}$, B.~Zhong$^{42}$, X.~Zhong$^{59}$, H. ~Zhou$^{50}$, L.~P.~Zhou$^{1,63}$, X.~Zhou$^{76}$, X.~K.~Zhou$^{6}$, X.~R.~Zhou$^{71,58}$, X.~Y.~Zhou$^{40}$, Y.~Z.~Zhou$^{12,f}$, J.~Zhu$^{44}$, K.~Zhu$^{1}$, K.~J.~Zhu$^{1,58,63}$, L.~Zhu$^{35}$, L.~X.~Zhu$^{63}$, S.~H.~Zhu$^{70}$, S.~Q.~Zhu$^{43}$, T.~J.~Zhu$^{12,f}$, W.~J.~Zhu$^{12,f}$, Y.~C.~Zhu$^{71,58}$, Z.~A.~Zhu$^{1,63}$, J.~H.~Zou$^{1}$, J.~Zu$^{71,58}$
\\
\vspace{0.2cm}
(BESIII Collaboration)\\
\vspace{0.2cm} {\it
$^{1}$ Institute of High Energy Physics, Beijing 100049, People's Republic of China\\
$^{2}$ Beihang University, Beijing 100191, People's Republic of China\\
$^{3}$ Beijing Institute of Petrochemical Technology, Beijing 102617, People's Republic of China\\
$^{4}$ Bochum Ruhr-University, D-44780 Bochum, Germany\\
$^{5}$ Carnegie Mellon University, Pittsburgh, Pennsylvania 15213, USA\\
$^{6}$ Central China Normal University, Wuhan 430079, People's Republic of China\\
$^{7}$ Central South University, Changsha 410083, People's Republic of China\\
$^{8}$ China Center of Advanced Science and Technology, Beijing 100190, People's Republic of China\\
$^{9}$ China University of Geosciences, Wuhan 430074, People's Republic of China\\
$^{10}$ Chung-Ang University, Seoul, 06974, Republic of Korea\\
$^{11}$ COMSATS University Islamabad, Lahore Campus, Defence Road, Off Raiwind Road, 54000 Lahore, Pakistan\\
$^{12}$ Fudan University, Shanghai 200433, People's Republic of China\\
$^{13}$ G.I. Budker Institute of Nuclear Physics SB RAS (BINP), Novosibirsk 630090, Russia\\
$^{14}$ GSI Helmholtzcentre for Heavy Ion Research GmbH, D-64291 Darmstadt, Germany\\
$^{15}$ Guangxi Normal University, Guilin 541004, People's Republic of China\\
$^{16}$ Guangxi University, Nanning 530004, People's Republic of China\\
$^{17}$ Hangzhou Normal University, Hangzhou 310036, People's Republic of China\\
$^{18}$ Hebei University, Baoding 071002, People's Republic of China\\
$^{19}$ Helmholtz Institute Mainz, Staudinger Weg 18, D-55099 Mainz, Germany\\
$^{20}$ Henan Normal University, Xinxiang 453007, People's Republic of China\\
$^{21}$ Henan University, Kaifeng 475004, People's Republic of China\\
$^{22}$ Henan University of Science and Technology, Luoyang 471003, People's Republic of China\\
$^{23}$ Henan University of Technology, Zhengzhou 450001, People's Republic of China\\
$^{24}$ Huangshan College, Huangshan 245000, People's Republic of China\\
$^{25}$ Hunan Normal University, Changsha 410081, People's Republic of China\\
$^{26}$ Hunan University, Changsha 410082, People's Republic of China\\
$^{27}$ Indian Institute of Technology Madras, Chennai 600036, India\\
$^{28}$ Indiana University, Bloomington, Indiana 47405, USA\\
$^{29}$ INFN Laboratori Nazionali di Frascati , (A)INFN Laboratori Nazionali di Frascati, I-00044, Frascati, Italy; (B)INFN Sezione di Perugia, I-06100, Perugia, Italy; (C)University of Perugia, I-06100, Perugia, Italy\\
$^{30}$ INFN Sezione di Ferrara, (A)INFN Sezione di Ferrara, I-44122, Ferrara, Italy; (B)University of Ferrara, I-44122, Ferrara, Italy\\
$^{31}$ Inner Mongolia University, Hohhot 010021, People's Republic of China\\
$^{32}$ Institute of Modern Physics, Lanzhou 730000, People's Republic of China\\
$^{33}$ Institute of Physics and Technology, Peace Avenue 54B, Ulaanbaatar 13330, Mongolia\\
$^{34}$ Instituto de Alta Investigaci\'on, Universidad de Tarapac\'a, Casilla 7D, Arica, Chile\\
$^{35}$ Jilin University, Changchun 130012, People's Republic of China\\
$^{36}$ Johannes Gutenberg University of Mainz, Johann-Joachim-Becher-Weg 45, D-55099 Mainz, Germany\\
$^{37}$ Joint Institute for Nuclear Research, 141980 Dubna, Moscow region, Russia\\
$^{38}$ Justus-Liebig-Universitaet Giessen, II. Physikalisches Institut, Heinrich-Buff-Ring 16, D-35392 Giessen, Germany\\
$^{39}$ Lanzhou University, Lanzhou 730000, People's Republic of China\\
$^{40}$ Liaoning Normal University, Dalian 116029, People's Republic of China\\
$^{41}$ Liaoning University, Shenyang 110036, People's Republic of China\\
$^{42}$ Nanjing Normal University, Nanjing 210023, People's Republic of China\\
$^{43}$ Nanjing University, Nanjing 210093, People's Republic of China\\
$^{44}$ Nankai University, Tianjin 300071, People's Republic of China\\
$^{45}$ National Centre for Nuclear Research, Warsaw 02-093, Poland\\
$^{46}$ North China Electric Power University, Beijing 102206, People's Republic of China\\
$^{47}$ Peking University, Beijing 100871, People's Republic of China\\
$^{48}$ Qufu Normal University, Qufu 273165, People's Republic of China\\
$^{49}$ Shandong Normal University, Jinan 250014, People's Republic of China\\
$^{50}$ Shandong University, Jinan 250100, People's Republic of China\\
$^{51}$ Shanghai Jiao Tong University, Shanghai 200240, People's Republic of China\\
$^{52}$ Shanxi Normal University, Linfen 041004, People's Republic of China\\
$^{53}$ Shanxi University, Taiyuan 030006, People's Republic of China\\
$^{54}$ Sichuan University, Chengdu 610064, People's Republic of China\\
$^{55}$ Soochow University, Suzhou 215006, People's Republic of China\\
$^{56}$ South China Normal University, Guangzhou 510006, People's Republic of China\\
$^{57}$ Southeast University, Nanjing 211100, People's Republic of China\\
$^{58}$ State Key Laboratory of Particle Detection and Electronics, Beijing 100049, Hefei 230026, People's Republic of China\\
$^{59}$ Sun Yat-Sen University, Guangzhou 510275, People's Republic of China\\
$^{60}$ Suranaree University of Technology, University Avenue 111, Nakhon Ratchasima 30000, Thailand\\
$^{61}$ Tsinghua University, Beijing 100084, People's Republic of China\\
$^{62}$ Turkish Accelerator Center Particle Factory Group, (A)Istinye University, 34010, Istanbul, Turkey; (B)Near East University, Nicosia, North Cyprus, 99138, Mersin 10, Turkey\\
$^{63}$ University of Chinese Academy of Sciences, Beijing 100049, People's Republic of China\\
$^{64}$ University of Groningen, NL-9747 AA Groningen, The Netherlands\\
$^{65}$ University of Hawaii, Honolulu, Hawaii 96822, USA\\
$^{66}$ University of Jinan, Jinan 250022, People's Republic of China\\
$^{67}$ University of Manchester, Oxford Road, Manchester, M13 9PL, United Kingdom\\
$^{68}$ University of Muenster, Wilhelm-Klemm-Strasse 9, 48149 Muenster, Germany\\
$^{69}$ University of Oxford, Keble Road, Oxford OX13RH, United Kingdom\\
$^{70}$ University of Science and Technology Liaoning, Anshan 114051, People's Republic of China\\
$^{71}$ University of Science and Technology of China, Hefei 230026, People's Republic of China\\
$^{72}$ University of South China, Hengyang 421001, People's Republic of China\\
$^{73}$ University of the Punjab, Lahore-54590, Pakistan\\
$^{74}$ University of Turin and INFN, (A)University of Turin, I-10125, Turin, Italy; (B)University of Eastern Piedmont, I-15121, Alessandria, Italy; (C)INFN, I-10125, Turin, Italy\\
$^{75}$ Uppsala University, Box 516, SE-75120 Uppsala, Sweden\\
$^{76}$ Wuhan University, Wuhan 430072, People's Republic of China\\
$^{77}$ Xinyang Normal University, Xinyang 464000, People's Republic of China\\
$^{78}$ Yantai University, Yantai 264005, People's Republic of China\\
$^{79}$ Yunnan University, Kunming 650500, People's Republic of China\\
$^{80}$ Zhejiang University, Hangzhou 310027, People's Republic of China\\
$^{81}$ Zhengzhou University, Zhengzhou 450001, People's Republic of China\\
\vspace{0.2cm}
$^{a}$ Also at the Moscow Institute of Physics and Technology, Moscow 141700, Russia\\
$^{b}$ Also at the Novosibirsk State University, Novosibirsk, 630090, Russia\\
$^{c}$ Also at the NRC "Kurchatov Institute", PNPI, 188300, Gatchina, Russia\\
$^{d}$ Also at Goethe University Frankfurt, 60323 Frankfurt am Main, Germany\\
$^{e}$ Also at Key Laboratory for Particle Physics, Astrophysics and Cosmology, Ministry of Education; Shanghai Key Laboratory for Particle Physics and Cosmology; Institute of Nuclear and Particle Physics, Shanghai 200240, People's Republic of China\\
$^{f}$ Also at Key Laboratory of Nuclear Physics and Ion-beam Application (MOE) and Institute of Modern Physics, Fudan University, Shanghai 200443, People's Republic of China\\
$^{g}$ Also at State Key Laboratory of Nuclear Physics and Technology, Peking University, Beijing 100871, People's Republic of China\\
$^{h}$ Also at School of Physics and Electronics, Hunan University, Changsha 410082, China\\
$^{i}$ Also at Guangdong Provincial Key Laboratory of Nuclear Science, Institute of Quantum Matter, South China Normal University, Guangzhou 510006, China\\
$^{j}$ Also at Frontiers Science Center for Rare Isotopes, Lanzhou University, Lanzhou 730000, People's Republic of China\\
$^{k}$ Also at Lanzhou Center for Theoretical Physics, Lanzhou University, Lanzhou 730000, People's Republic of China\\
$^{l}$ Also at the Department of Mathematical Sciences, IBA, Karachi 75270, Pakistan\\
}
%% ends here %%
}




\begin{abstract}
Using $e^+e^-$ collision data with an integrated luminosity of $7.33~\mathrm{fb}^{-1}$ collected at center-of-mass energies between 4.128 and 4.226 GeV with the BESIII detector operating at the BEPCII collider, the branching fraction of the leptonic decay $D_s^+\to\mu^+\nu_\mu$ is measured
to be $(0.5294\pm0.0108_{\rm stat}\pm0.0085_{\rm syst})$\%.
Based on this, the product of the $D_s^+$ decay constant $f_{D_s^+}$ and the magnitude of the $c\to s$ quark mixing matrix element $|V_{cs}|$ is determined to be $f_{D_s^+}|V_{cs}|=241.8\pm2.5_{\rm stat}\pm2.2_{\rm syst}~\mathrm{MeV}$.
Using the value of $|V_{cs}|$ given by the global standard model fit, $f_{D_s^+}$ is found to be $248.4\pm2.5_{\rm stat}\pm2.2_{\rm syst}$\,MeV. Alternatively, using the value of $f_{D_s^+}$ from a recent lattice quantum chromodynamics calculation, $|V_{cs}|$ is determined to be $0.968\pm0.010_{\rm stat}\pm0.009_{\rm syst}$.
\end{abstract}

\maketitle
\section{Introduction}
Experimental studies of the leptonic decay $D^+_s\to \ell^+\nu_\ell$~($\ell=e$, $\mu$ or $\tau$) are important to explore both the strong and weak interactions in the charm quark sector. In the standard model~(SM), the $D^+_s\to \ell^+\nu_\ell$ decay partial width is given by~\cite{decayrate}
\begin{equation}
\Gamma_{D^+_{s}\to\ell^+\nu_\ell}=\frac{G_F^2}{8\pi}|V_{cs}|^2
f^2_{D^+_{s}}
m_\ell^2 m_{D^+_{s}} \left (1-\frac{m_\ell^2}{m_{D^+_{s}}^2} \right )^2,
\end{equation}
where
$G_F$ is the Fermi coupling constant,
$m_\ell$ is the lepton mass,
$m_{D^+_{s}}$ is the $D^+_{s}$ mass,
$f_{D^+_{s}}$ is the $D^+_{s}$ decay constant, and
$|V_{cs}|$ is the magnitude of the $c\to s$
Cabibbo-Kobayashi-Maskawa~(CKM) matrix element.

In recent years, many studies of $D^+_s\to \ell^+\nu_\ell$ have been performed by the CLEO~\cite{cleo2009,cleo2009a,cleo2009b}, BaBar~\cite{babar2010}, Belle~\cite{belle2013}, and
BESIII~\cite{bes2016,besiii2,Hajime,bes32,bes33,bes34,bes35} experiments.
The experimental precision of $f_{D^+_s}$, however, is still worse than the
lattice quantum chromodynamics~(LQCD)
calculation~\cite{FLab2018}.
Intensive experimental studies of $D^+_s\to \ell^+\nu_\ell$ can determine $f_{D^+_{s}}$ and $|V_{cs}|$ with improved precision, and can thereby calibrate various theoretical calculations of $f_{D^+_{s}}$~\cite{FLab2018,LQCD,etm2015,ukqcd2017,milc2012,hpqcd2010,hpqcd2012,etm2012,chen2014,pacs2011,becirevic2013}
and test the unitarity of the CKM matrix.

In addition, scrutinizing lepton flavor universality~(LFU) violation in $D^+_s\to\ell^+\nu_\ell$ decays offers an important test of the SM~\cite{babar_1,babar_2,lhcb_1,lhcb_2,lhcb_3,belle_RD1,belle_RD2,belle_RD3,HFLAV}. In the SM, the ratio of the branching fraction (BF) of $D^+_s\to \tau^+\nu_\tau$
over that of $D^+_s\to \mu^+\nu_\mu$ is predicted to be 9.75 with a
negligible uncertainty, hence any observed deviation
from this value would means a sign of new physics beyond the SM.
Some hints of LFU violation in semileptonic $B$ decays were reported at BaBar, LHCb, and Belle~\cite{babar_1,babar_2,lhcb_1,lhcb_2,lhcb_3,belle_RD1,belle_RD2,belle_RD3}.
The measured BF ratios $\mathcal{R}_{D^{(*)}}^{\tau/\ell}=\mathcal{B}_{B\to \bar D^{(*)}\tau^+\nu_{\tau}}/\mathcal{B}_{B\to \bar D^{(*)}\ell^+\nu_{\ell}}$ deviate from the SM predictions by about $3.3\sigma$~\cite{HFLAV}.
A test of LFU with $D^+_s\to \ell^+\nu_\ell$ decays may shed light on this tension.

In this paper, we report an improved measurement of
the BF of the  $D_s^+\to\mu^+\nu_\mu$ decay
by analyzing 7.33\,fb$^{-1}$ of $e^+e^-$
collision data collected with the BESIII detector
at center-of-mass energies of $E_{\rm cm} = 4.128, 4.156, 4.178, 4.189, 4.199, 4.209, 4219,$ and $4.226$~GeV. Charge-conjugate ($c.c.$) modes are always implied in the text. Benefiting from a larger data sample, more tag modes, and measurement from muon identifier modules, the results obtained in this work supersede the previous BESIII measurement with the muon identifier modules using data at $E_{\rm cm}=4.178$~GeV~\cite{besiii2} and the measurement without the muon identifier modules using data at $E_{\rm cm}$ between 4.178 and 4.226~GeV~\cite{Hajime}.

\section{BESIII detector and Monte Carlo simulation}

The BESIII detector is a magnetic
spectrometer~\cite{BESIII} located at the Beijing Electron
Positron Collider (BEPCII)~\cite{Yu:IPAC2016-TUYA01}. The
cylindrical core of the BESIII detector consists of a helium-based
 multilayer drift chamber (MDC), a plastic scintillator time-of-flight
system (TOF), and a CsI (Tl) electromagnetic calorimeter (EMC),
which are all enclosed in a superconducting solenoidal magnet
providing a 1.0~T magnetic field. The solenoid is supported by an
octagonal flux-return yoke with resistive plate counter muon
identifier modules interleaved with steel. The solid angle coverage for detecting charged particles is 93\% over $4\pi$. The
charged-particle momentum resolution at $1~{\rm GeV}/c$ is
$0.5\%$, and the resolution of the specific ionization energy loss~(d$E$/d$x$) is $6\%$ for the electrons
from Bhabha scattering. The EMC measures photon energies with a
resolution of $2.5\%$ ($5\%$) at $1$~GeV in the barrel (end cap)
region. The time resolution of the TOF barrel part is 68~ps, while
that of the end cap part is 110~ps. The end cap TOF
system was upgraded in 2015 using multi-gap resistive plate chamber technology, providing
a time resolution of 60 ps~\cite{60ps1,60ps2}.
Approximately 83\%
of the data used here was collected after this upgrade, and the corresponding luminosities~\cite{luminosities,luminosities2} at each energy are given in Table~\ref{tab:mbc}.
More details about the design and performance of the BESIII detector are given in Ref.~\cite{BESIII}.

Simulated samples produced with {\sc geant4}-based~\cite{geant4}
Monte Carlo (MC) software, which includes the geometric description of
the BESIII detector and the detector response, are used to determine
the detection efficiency and to estimate background contributions. The
simulations include the beam energy spread and initial state radiation
in the $e^+e^-$ annihilations modeled with the generator {\sc
  kkmc}~\cite{kkmc,kkmc2}. An inclusive MC sample with an equivalent luminosity of 40 times that of the data is produced at  center-of-mass energies between 4.128 and 4.226 GeV. It includes open-charm processes, initial state radiation (ISR) production of $\psi(3770)$, $\psi(3686)$ and $J/\psi$, $q\bar q$ $(q=u, d, s)$ continuum processes, along with Bhabha scattering, $\mu^+\mu^-$, $\tau^+\tau^-$, and $\gamma\gamma$ events. In the simulation, the production of open-charm processes directly produced via $e^+e^-$ annihilations are modeled with the generator {\sc conexc}~\cite{conexc}, and their subsequent decays are modeled by {\sc evtgen}~\cite{evtgen,evtgen2} with known BFs from the Particle Data Group (PDG)~\cite{PDG2022}. The input cross
section of $e^+e^-\to D_s^{\pm}D_s^{*\mp}$ is based on the cross section measurement in the energy range from threshold to 4.7 GeV. The ISR production of vector charmonium(-like) states and the continuum processes are incorporated in  {\sc
  kkmc}~\cite{kkmc,kkmc2}. The remaining unknown decays of the
charmonium states are modeled by {\sc
  lundcharm}~\cite{lundcharm,lundcharm2}. Final state radiation is
incorporated using {\sc photos}~\cite{photos}.

\section{Method}
\label{sec:method}

At $E_{\rm cm}$ from 4.128 to 4.226 GeV, the $D^+_s$ mesons are produced mainly through the process $e^+e^-\to D^+_sD_s^{*-}+c.c$.
A single-tag (ST) sample is selected by fully reconstructing a $D_s^-$ meson in one of several `tag' modes, and
a double-tag (DT) sample is reconstructed by further
requiring a $\gamma/\pi^0$ and a $D_s^+$ reconstructed using the remaining neutral clusters and tracks.
Then the BF of $D^+_s\to \mu^+\nu_\mu$ is determined by
\begin{equation}
\mathcal B_{D_s^+\to\mu^+\nu_\mu}=\frac{N_{\rm DT}}{N_{\rm ST} \cdot \epsilon_{\gamma(\pi^0)\mu^+\nu_\mu}},
\label{eq1}
\end{equation}
where $N_{\rm DT}$ is the DT yield in data, $N_{\rm ST}=\Sigma_i N_{\rm ST}^{i}$ is the total ST yield in data summing over tag mode $i$, $\epsilon_{\gamma(\pi^0)\mu^+\nu_\mu}=\Sigma_i(N_{\rm ST}^{i}/N_{\rm ST}) \cdot (\epsilon^{i}_{\rm DT}/\epsilon^{i}_{\rm ST})$ is the effective signal efficiency of detecting $D^+_s\to \mu^+\nu_\mu$ in the presence of the ST $D^-_s$ candidate, averaged by the ST yields in data, and $\epsilon^{i}_{\rm DT}$ and $\epsilon^{i}_{\rm ST}$ are the detection efficiencies of the DT and ST candidates, respectively.

\section{Single Tag Selection}

The ST $D_s^-$ mesons are reconstructed using 16 hadronic $D_s^-$ decay modes:
$K^+K^-\pi^-$,
$K^+K^-\pi^-\pi^0$,
$\pi^+\pi^-\pi^-$,
$K_S^0K^-$,
$K_S^0K^-\pi^0$,
$K^-\pi^+\pi^-$,
$K_S^0K_S^0\pi^-$,
$K_S^0K^+\pi^-\pi^-$,
$K_S^0K^-\pi^+\pi^-$,
$\eta_{\gamma\gamma}\pi^-$,
$\eta_{\pi^+\pi^-\pi^0}\pi^-$,
$\eta^\prime_{\pi^+\pi^-\eta_{\gamma\gamma}}\pi^-$,
$\eta^\prime_{\gamma\rho^0}\pi^-$,
$\eta_{\gamma\gamma}\rho^-$,
$\eta_{\pi^+\pi^-\pi^0}\rho^-$, and
$\eta_{\gamma\gamma}\pi^+\pi^-\pi^-$,
where the subscripts on the $\eta(\eta^{\prime})$ represent the decay modes used to reconstruct the $\eta(\eta^{\prime})$.

All charged tracks except for those from $K_S^0$ decays must originate from the interaction
point~(IP) with a distance of closest approach less than 1 cm in the transverse plane
and less than 10 cm along the $z$ axis. The polar angle ($\theta$) is required to be within the MDC
acceptance $|\rm{cos\theta}|<0.93$, where $\theta$ is defined with
respect to the symmetry axis of the MDC taken as the $z$ axis. Measurements of $\text{d}E/\text{d}x$ in the MDC and the flight time in the TOF are combined for particle identification~(PID) by calculating
confidence levels for the pion and kaon hypotheses ($CL_\pi$,~$CL_K$).
Kaon~(pion) candidates are required to satisfy $CL_{K(\pi)}>CL_{\pi(K)}$.

To select $K_S^0$ candidates, pairs of oppositely-charged tracks with distance
of closest approach to the IP less than 20 cm along the $z$ axis are assigned as
$\pi^+\pi^-$ without PID requirements.
These $\pi^+\pi^-$ combinations are required to
have an invariant mass within $\pm12$\,MeV of the nominal $K_S^0$ mass~\cite{PDG2022} and
a decay length greater than twice the vertex resolution away from the IP.

The $\pi^0$ and $\eta$ mesons are reconstructed via their decays to $\gamma\gamma$. Here, the $\gamma$ candidates are identified using isolated showers in the EMC. The deposited energy of each shower must be greater than 25 (50) MeV in the EMC barrel
(end-cap) region~\cite{BESIII}.
To suppress electronic noise and showers
unrelated to the event, the difference between the EMC time and
the event start time is required to be within $[0, 700]$ ns~\cite{BESIII}.
The opening angle between a shower and
the nearest charged track has to be greater than $10^{\circ}$.
The $\gamma\gamma$ combinations with invariant masses $M_{\gamma\gamma}\in(0.115,\,0.150)$
and $(0.500,\,0.570)$\,GeV$/c^{2}$ are regarded as $\pi^0$ and $\eta$ mesons, respectively.
A kinematic fit is further performed to
constrain $M_{\gamma\gamma}$ to the $\pi^{0}$ or $\eta$
nominal mass~\cite{PDG2022}.

The $\eta$ candidates for the $\eta\pi^-$
ST channel are also reconstructed
via $\pi^0\pi^+\pi^-$ final states with an invariant mass
within $(0.53,\,0.57)~\mathrm{GeV}/c^2$.
The $\eta^\prime$ mesons are reconstructed via two decay modes, $\eta\pi^+\pi^-$ and $\gamma\rho^0$,
whose invariant masses are required to be within
$(0.946,\,0.970)$ and $(0.940,\,0.976)~\mathrm{GeV}/c^2$, respectively.
In addition, the minimum energy
of the $\gamma$ from $\eta^\prime\to\gamma\rho^0$ decays must be greater than 0.1\,GeV.
The $\rho^0$ and $\rho^+$ mesons are reconstructed from $\pi^+\pi^-$ and $\pi^+\pi^0$
combinations whose invariant masses are required to be within $(0.67,\,0.87)~\mathrm{GeV}/c^2$.

The momentum of any pion not originating from a $K_S^0$, $\eta$, or $\eta^\prime$ decay is required to be greater than 0.1\,GeV/$c$
to reject transition pions from $D^*$ decays.
When selecting $\pi^+\pi^-\pi^-$ and $K^-\pi^+\pi^-$ combinations, peaking backgrounds from
$K^0_S\pi^-$ and $K_S^0K^-$ components are rejected by requiring that the invariant mass of any $\pi^+\pi^-$ combination
satisfy $|M_{\pi^+\pi^-}-M_{K_S^0}|>0.030~$~GeV/$c^2$~\cite{PDG2022}.

The beam-constrained mass of the ST $D_s^-$
candidate
\begin{equation}
M_{\rm BC}\equiv\sqrt{(E_{\rm cm}/2)^2/c^4-|\vec{p}_{D_s^-}|^2/c^2}
\end{equation}
is used to suppress the non-$D_s^+D^{*-}_s$ events,
where
$\vec{p}_{D_s^-}$ is the momentum of the ST $D_s^-$ candidate.
The requirements on $M_{\rm BC}$ for the eight center-of-mass energies are listed in Table~\ref{tab:mbc}, which retain the $D_s^-$ mesons
from both the $e^+e^-$ annihilation
and the $D_s^{*-}$ decay. In each event, we only keep
the candidate with the $D_s^-$ recoil mass
\begin{equation}
M_{\rm rec} \equiv \sqrt{ \left (E_{\rm cm} - \sqrt{|\vec p_{D^-_s}|^2c^2+m^2_{D^-_s}c^4} \right )^2/c^4
-|\vec p_{D^-_s}|^2/c^2}
\end{equation}
closest to the nominal $D_s^{*+}$ mass~\cite{PDG2022} per tag mode per charge.
Figure~\ref{fig:stfit} shows the invariant mass ($M_{\rm tag}$) spectra of the accepted ST
candidates with all datasets.

\begin{table}[htbp]
\centering
\caption{The integrated luminosities and requirements on $M_{\rm BC}$ for the ST candidates at various energy points.}
\begin{tabular}{cSc}
\hline
\hline
$E_{\rm cm}$ (GeV)  &{Luminosity (pb$^{-1}$)}& $M_{\rm BC}$ (GeV/$c^2$) \\\hline
4.128               &401.5    & $(2.010,2.061)$                        \\
4.157               &408.7    & $(2.010,2.070)$                        \\
4.178               &3189.0   & $(2.010,2.073)$                        \\
4.189               &569.8    & $(2.010,2.076)$                        \\
4.199               &526.0    & $(2.010,2.079)$                        \\
4.209               &571.7    & $(2.010,2.082)$                       \\
4.219               &568.7    & $(2.010,2.085)$                        \\
4.226               &1091.7   & $(2.010,2.088)$                        \\
\hline
\hline
\end{tabular}
\label{tab:mbc}
\end{table}

% Figure environment removed

At each energy point, the ST yield for each tag mode is obtained by a fit to the corresponding $M_{\rm tag}$ spectrum.
The signal is described by the MC-simulated shape convolved with a Gaussian function
representing the resolution difference between data and MC simulation.
For the tag mode $D^-_s\to K_S^0K^-$,
the peaking background from $D^-\to K^0_S\pi^-$ is
described by the MC-simulated shape smeared with the same Gaussian function
as used in the signal shape with the magnitude free in the fit.
The non-peaking background is modeled by a second- or third-order Chebychev polynomial
function, which is validated with the inclusive MC sample.
The fit results of $M_{\rm tag}$ summed over
all energy points are shown in Fig.~\ref{fig:stfit}.
The events in the signal regions
are kept for further analysis.
As an example, the ST yields in data
and the ST efficiencies at 4.178 GeV are shown in Table~\ref{tab:styields}.
The total ST yields at the different energy points are summarized in Table~\ref{tab:sigeff}.

\begin{table*}[htbp]
\caption{The $M_{\rm tag}$ requirements, ST yields in data ($N_{\rm ST}^{i}$), ST efficiencies ($\epsilon_{\rm ST}^{i}$), and effective signal efficiencies with transition $\gamma(\pi^0)$ matched ($\epsilon^{i,\rm matched}_{\gamma(\pi^0) \mu^+\nu_\mu}$) of various tag modes obtained from data at $E_{\rm cm}=4.178$ GeV.
The uncertainties are statistical only.}
\label{tab:styields}
\centering
\begin{tabular}{lc r@{\,}c@{\,}l r@{\,}c@{\,}l r@{\,}c@{\,}l}
\hline\hline
Tag mode & $M_{\rm tag}$ (GeV/$c^2$) &\multicolumn{3}{c}{$N_{\rm ST}^{i}$} &\multicolumn{3}{c}{$\epsilon_{\rm ST}^{i}$ (\%)} &\multicolumn{3}{c}{$\epsilon^{i,\rm matched}_{\gamma(\pi^0) \mu^+\nu_\mu}$ (\%)}\\
\hline
$K^+K^-\pi^-$                           &$(1.950,1.986)$& $13717$&$\pm$&$608$  &$40.92$&$\pm$&$0.02$&$49.25$&$\pm$&$ 0.21$\\
$K^+K^-\pi^-\pi^0$                      &$(1.947,1.982)$& $42119$&$\pm$&$851$  &$11.77$&$\pm$&$0.01$&$58.33$&$\pm$&$ 0.74$\\
$\pi^-\pi^+\pi^-$                       &$(1.952,1.984)$& $36497$&$\pm$&$873$  &$52.13$&$\pm$&$0.05$&$51.16$&$\pm$&$ 0.19$\\
$K_S^0 K^-$                             &$(1.948,1.991)$& $30956$&$\pm$&$261$  &$47.63$&$\pm$&$0.05$&$50.32$&$\pm$&$ 0.23$\\
$K_S^0 K^-\pi^0$                        &$(1.946,1.987)$& $11182$&$\pm$&$449$  &$17.01$&$\pm$&$0.04$&$58.42$&$\pm$&$ 0.74$\\
$K^-\pi^+\pi^-$                         &$(1.953,1.983)$& $16514$&$\pm$&$632$  &$45.42$&$\pm$&$0.07$&$51.44$&$\pm$&$ 0.41$\\
$K_S^0 K_S^0 \pi^-$                     &$(1.951,1.986)$& $5088$&$\pm$&$149$   &$22.82$&$\pm$&$0.07$&$51.97$&$\pm$&$ 0.91$\\
$K_S^0 K^+\pi^-\pi^-$                   &$(1.953,1.983)$& $14855$&$\pm$&$235$  &$21.12$&$\pm$&$0.04$&$50.72$&$\pm$&$ 0.79$\\
$K_S^0 K^-\pi^+\pi^-$                   &$(1.958,1.980)$& $7621$&$\pm$&$270$   &$18.51$&$\pm$&$0.05$&$52.15$&$\pm$&$ 0.43$\\
$\eta\pi^-$                             &$(1.930,2.000)$& $19239$&$\pm$&$468$  &$48.79$&$\pm$&$0.06$&$54.12$&$\pm$&$ 0.20$\\
$\eta_{\pi^+\pi^-\pi^0}\pi^+$           &$(1.941,1.990)$& $5693$&$\pm$&$201$   &$23.49$&$\pm$&$0.07$&$55.51$&$\pm$&$ 0.34$\\
$\eta^\prime_{\eta\pi^+\pi^-}\pi^-$     &$(1.940,1.996)$& $9730$&$\pm$&$140$   &$25.26$&$\pm$&$0.05$&$55.12$&$\pm$&$ 0.31$\\
$\eta^\prime_{\gamma\rho}\pi^-$         &$(1.938,1.992)$& $24698$&$\pm$&$656$  &$32.53$&$\pm$&$0.04$&$54.10$&$\pm$&$ 0.26$\\
$\eta\rho^-$                            &$(1.920,2.006)$& $39670$&$\pm$&$1673$ &$19.88$&$\pm$&$0.02$&$66.92$&$\pm$&$ 0.37$\\
$\eta_{\pi^+\pi^-\pi^0}\rho^-$          &$(1.927,1.997)$& $10504$&$\pm$&$928$  &$ 9.23$&$\pm$&$0.02$&$67.48$&$\pm$&$ 0.57$\\
$\eta_{\gamma\gamma}\pi^+\pi^-\pi^-$    &$(1.946,1.990)$& $23417$&$\pm$&$1232$ &$24.84$&$\pm$&$0.04$&$63.74$&$\pm$&$ 0.32$\\
\hline
\hline
\end{tabular}
\end{table*}

\begin{table}[htbp]\centering
  \caption{The ST yields in data ($N_{\rm ST}$), the weighted signal efficiencies with transition $\gamma(\pi^0)$ matched and unmatched within $20^\circ$ ($\epsilon^{\rm matched}_{\gamma(\pi^0) \mu^+\nu_\mu}$ and $\epsilon^{\rm unmatched}_{\gamma(\pi^0) \mu^+\nu_\mu}$) for various energy points. The uncertainties are statistical only.}
  \label{tab:sigeff}
  \begin{tabular}{cr@{\,}c@{\,}l r@{\,}c@{\,}l r@{\,}c@{\,}l}
  \hline
  \hline
  $E_{\rm cm}$ (GeV) & \multicolumn{3}{c}{$N_{\rm ST}$}  &\multicolumn{3}{c}{$\epsilon^{\rm matched}_{\gamma(\pi^0) \mu^+\nu_\mu}$} (\%)  & \multicolumn{3}{c}{$\epsilon^{\rm unmatched}_{\gamma(\pi^0) \mu^+\nu_\mu}$ (\%)}\\ \hline
  4.128 &$38203$&$\pm$&$1794$   &$57.48$&$\pm$&$0.14$  &$5.91$&$\pm$&$0.05$\\
  4.157 &$55280$&$\pm$&$971$    &$56.26$&$\pm$&$0.13$  &$5.96$&$\pm$&$0.05$\\
  4.178 &$435100$&$\pm$&$2926$  &$54.35$&$\pm$&$0.12$  &$5.64$&$\pm$&$0.05$\\
  4.189 &$72852$&$\pm$&$1030$   &$52.76$&$\pm$&$0.12$  &$5.67$&$\pm$&$0.05$\\
  4.199 &$68054$&$\pm$&$1136$   &$53.31$&$\pm$&$0.12$  &$5.94$&$\pm$&$0.05$\\
  4.209 &$68608$&$\pm$&$1281$   &$51.12$&$\pm$&$0.12$  &$5.70$&$\pm$&$0.05$\\
  4.219 &$58151$&$\pm$&$1159$   &$49.47$&$\pm$&$0.12$  &$5.41$&$\pm$&$0.04$\\
  4.226 &$91218$&$\pm$&$1739$   &$49.66$&$\pm$&$0.12$  &$5.62$&$\pm$&$0.05$\\
  \hline\hline
  \end{tabular}
\end{table}

\section{Double Tag Selection}

After reconstructing the ST $D_s^-$ mesons,
the $D_s^+\to\mu^+\nu_\mu$ candidates are selected using the
remaining neutral and charged tracks.
To select the soft $\gamma(\pi^0)$ from the $D_s^{*}$ decay and
to separate signals from combinatorial backgrounds,
we define two kinematic variables
\begin{equation}
\Delta E \equiv E_{\rm cm}-E_{\rm tag}-E_{\rm miss}-E_{\gamma(\pi^0)}
\end{equation}
and
\begin{align}
M_{\rm miss}^2&\equiv\left (E_{\rm cm}-E_{\rm tag}-E_{\gamma(\pi^0)}-E_{\mu}\right )^2/c^4\nonumber\\
&-|-\vec{p}_{\rm tag}-\vec{p}_{\gamma(\pi^0)}-\vec{p}_{\mu}|^2/c^2.
\end{align}
Here
$E_{\rm miss} \equiv \sqrt{|\vec{p}_{\rm miss}|^2/c^4+m_{D_s^+}^2}$ and
$\vec{p}_{\rm miss} \equiv -\vec{p}_{\rm tag}-\vec{p}_{\gamma(\pi^0)}$ are
the missing energy and momentum of the recoiling system of the soft $\gamma(\pi^0)$ and the ST $D_s^-$,
where $E_i$ and $\vec p_i$ ($i=\mu^+,\gamma(\pi^0)$ or tag) denote
the energy and momentum of the muon, $\gamma(\pi^0)$ or ST $D^-_s$, respectively.
$M_{\rm miss}^2$ is the missing mass square of the undetected neutrino.
We loop over all remaining $\gamma$ or $\pi^0$ candidates
and choose the one giving a minimum $|\Delta E|$.
The events with $\Delta E\in(-0.05,\,0.10)$\,GeV are accepted.
The muon candidate is required to have an opposite charge
to the ST $D^-_s$ meson and
deposited energy in the EMC within $(0.0,\,0.3)$\,GeV.
It must also satisfy the requirement on the
hit depth ($d_{\mu^+}$) in
the muon identifier modules dependent on the
$|\cos\theta_{\mu^+}|$ and $p_{\mu^+}$ as shown in
Table~\ref{tab:muonid}.

\begin{table}[hbtp]
  \begin{center}
  \caption{The $\cos(\theta_{\mu^+})$ and $p_{\mu^+}$ dependent requirements
    of $d_{\mu^+}$ for muon candidates. To compute the $d_{\mu^+}$ requirements, the momentum
  $p_{\mu^+}$ is taken in units of GeV/$c$.}
  \begin{tabular}{c|c|c} \hline\hline
  $|\cos\theta_{\mu^+}|$&$p_{\mu^+}$ (GeV/$c$)& $d_{\mu^+}$ (cm) \\ \hline
           &$p_{\mu^+}\le0.88$      &$>17.0$        \\
  $$(0.00,0.20)$$&$0.88<p_{\mu^+}<1.04$&$>100.0\times p_{\mu^+}-71.0$ \\
           &$p_{\mu^+}\ge1.04$      &$>33.0$ \\ \hline
           &$p_{\mu^+}\le0.91$      &$>17.0$        \\
  $$(0.20,0.40)$$&$0.91<p_{\mu^+}<1.07$&$>100.0\times p_{\mu^+}-74.0$ \\
           &$p_{\mu^+}\ge1.07$      &$>33.0$ \\ \hline
           &$p_{\mu^+}\le0.94$      &$>17.0$        \\
  $$(0.40,0.60)$$&$0.94<p_{\mu^+}<1.10$&$>100.0\times p_{\mu^+}-77.0$ \\
           &$p_{\mu^+}\ge1.10$      &$>33.0$ \\ \hline
  $$(0.60,0.80)$$&                  &$>17.0$ \\ \hline
  $$(0.80,0.93)$$&                  &$>17.0$ \\ \hline\hline
  \end{tabular}
  \label{tab:muonid}
  \end{center}
\end{table}

To suppress backgrounds with extra photon(s),
the maximum energy of the unused showers in the DT
selection ($E^{\mathrm{extra}~\gamma}_{\rm max}$) is required to be less than 0.3\,GeV.
No additional charged tracks are allowed in the event.
To improve the $M_{\rm miss}^2$ resolution,
the candidate tracks plus the missing neutrino
are subjected to a 4-constraint kinematic fit requiring energy and momentum conservation.
In addition, the invariant masses of the two $D_s$ mesons are constrained to the known $D_s$ mass,
and the invariant mass of the $D_s^-\gamma(\pi^0)$ or $D_s^+\gamma(\pi^0)$ combination is
constrained to the nominal $D_s^*$ mass.
The combination with the minimum $\chi^2$ is kept.
Figure~\ref{fig:mm2ft} shows the $M_{\rm miss}^2$
distribution for the accepted DT candidate events in data.

The efficiencies of the DT reconstruction are determined with the signal MC samples.
Dividing them by the ST efficiencies determined with the inclusive MC sample yields the corresponding
efficiencies of the $\gamma(\pi^0)\mu^+\nu_\mu$ reconstruction.
The efficiency averaged over all tag modes
is then determined from
\begin{equation}
\epsilon_{\gamma(\pi^0)\mu^+\nu_\mu}=f_{\mu\,\rm PID}^{\rm cor}f_{\rm tag~bias}^{\rm cor}\sum_{i}(N_{\rm ST}^i\epsilon_{\rm DT}^i)/(N_{\rm ST}^{\rm tot}\epsilon_{\rm ST}^i).
\end{equation}
In the above equation, the correction factor $f_{\mu\,\rm PID}^{\rm cor}$ accounts for the differences of $\mu^+$ identification efficiencies between data and MC simulation. It is non-negligible mainly due to the imperfect simulation of $d_{\mu^+}$~\cite{besiii2}. The energy point dependent correction factors for $\mu^+$ identification efficiencies are $(87.6{\text -}93.9)\%$ with uncertainties of $(0.4{\text -}1.1)\%$, depending on the data taking status. These efficiencies are estimated using $e^+e^-\to\gamma\mu^+\mu^-$ samples and reweighted by
the $\mu^+$ two dimensional distribution in $|\cos\theta_{\mu^+}|$ and $p_{\mu^+}$ of $D_s^+\to\mu^+\nu_\mu$.

The correction factor $f_{\rm tag~bias}^{\rm cor}$ takes into account the differences of the ST efficiencies in the inclusive and signal MC samples due to different track multiplicities. This may cause incomplete cancellation of the uncertainties of the ST efficiencies. It is estimated to be about 99.57\% after considering the differences of the efficiencies of tracking or PID of $K^\pm$ and $\pi^\pm$, as well as the selections of neutral particles between data and MC simulation.

In this analysis, the shapes of signal candidates are divided into two types: one describes the signal candidates with transition $\gamma(\pi^0)$ matched and another describes the signal candidates unmatched.
The matched type requires the angle between the flight
direction of the reconstructed $\gamma(\pi^0)$ and that of MC truth
to be less than $20^\circ$ in an event, otherwise the event is classified as the unmatched type.
Both event types have real $D^+_s\to \mu^+\nu_\mu$, but the matched type forms the peak and the unmatched type forms the combinatorial shape in the $M_{\rm miss}^2$ distribution.
The average signal efficiencies for finding $\gamma(\pi^0)\mu^+\nu_\mu$ at various energy points are shown in Table~\ref{tab:sigeff}.

The background includes two components. One is the events with wrongly tagged $D_s^-$ decays (``non-$D_s^-$ background''), and another is with correctly tagged $D_s^-$ decays but incorporating particle mis-identifications (``real-$D_s^-$ background'')
which is mainly from $D_s^+\to \tau^+(\to \pi^+\bar\nu_\tau)\nu_\tau$ decay.
Studies on the inclusive MC samples show
that the two components make
comparable contributions and do not peak in the signal region.
The total background fraction is about $8.2\%$.

\section{Branching Fraction Measurement}

To obtain the BF of $D_s^+\to\mu^+\nu_\mu$,
we perform a simultaneous fit to the $M_{\rm miss}^2$ distributions at
the eight energy points. The BF of $D_s^+\to\mu^+\nu_\mu$ at different energy points
is the common parameter in the fit.
The signal shapes with transition $\gamma(\pi^0)$ matched and unmatched are described
by individual simulated shapes derived from the signal MC sample, where the former ones
are further convolved with a Gaussian function with free parameters to
consider the resolution difference
between data and MC simulation.
The shapes of other backgrounds are derived from the
individual MC simulated shapes.
In the fit, the BF, the ratios between the signal
events with transition $\gamma(\pi^0)$ matched and unmatched,
and the background yields are all free.

Figure \ref{fig:mm2ft} shows the fit result with all datasets.
The BF of $D_s^+\to\mu^+\nu_\mu$ is obtained to be
\begin{equation}
{\mathcal{B}}_{D_s^+\to\mu^+\nu_\mu}=(0.5294\pm0.0108)\%,
\nonumber
\end{equation}
which corresponds to a signal yield of $2514.5\pm51.6$ events.
Here, the uncertainty is statistical only.

% Figure environment removed

\section{Systematic uncertainties}

Sources of the systematic uncertainties in the branching fraction measurement are summarized in Table~\ref{tab:sys_crs}.
Each of them is estimated relative to the measured BF and described below.

\begin{itemize}
\item
ST yield:
The systematic uncertainty is estimated by varying the signal and background shapes in the fit.
The alternative signal shapes are obtained by varying the nominal matched angle by $\pm 5^\circ$. The relative difference of the ST yields between data and the inclusive MC sample is assigned as the systematic uncertainty. In addition, the uncertainty due to the background fluctuation in the ST yield is also considered as a systematic uncertainty. Adding these three systematic effects quadratically gives a total systematic uncertainty of 0.44\%.
\item
$\mu^+$ tracking and PID:
The $\mu^+$ tracking and PID efficiencies are studied with the control sample $e^+e^-\to\gamma\mu^+\mu^-$.
After correcting the detection efficiency by $f^{\rm cor}_{\mu\,\rm PID}$, we assign 0.24\% and 0.19\%
as the uncertainties in $\mu^+$ tracking and PID efficiencies, respectively.
\item
Transition $\gamma(\pi^0)$ reconstruction: The selection efficiencies of $\gamma$ and $\pi^0$ were studied with $J/\psi\to\pi^+\pi^-\pi^0$ decays~\cite{sys_gamma_llc}. The systematic uncertainty  is assigned to be 1\%.
\item
The least $|\Delta E|$ selection: The systematic uncertainty of selecting the transition $\gamma(\pi^0)$with the least $|\Delta E|$ method is estimated by using the control samples of $D^+_s\to K^+K^-\pi^+$ and $D^+_s\to \eta\pi^0\pi^+$. The difference of the efficiencies of selecting the transition $\gamma\,(\pi^0)$ candidates between data and MC simulation, 0.70\%, is taken as the corresponding systematic uncertainty.
\item
$E_{\rm max}^{{\rm extra~}\gamma}$ \& $N_{\rm ncharged}^{\rm extra}$ requirements: The efficiency for the requirements of $E^{\mathrm{extra}~\gamma}_{\rm max}$ and
no extra good charged track is studied with the control samples of $D^+_s\to K^+K^-\pi^+$ and $D^+_s\to K_S^0K^+$.
The systematic uncertainty is taken to
be 0.29\% considering the efficiency differences between data and MC simulation.
\item
$M_{\rm miss}^2$ fit:
The systematic uncertainty due to the signal shape with transition $\gamma(\pi^0)$ matched is estimated with an alternative signal shape of a double Gaussian function.  The systematic uncertainty due to the signal shape with transition $\gamma(\pi^0)$ un-matched is estimated by replacing the nominal shape with a second order Chebychev function. The systematic uncertainty due to the real-$D^-_s$ background is estimated by varying the weights of various background sources within $\pm 1\sigma$ of individual BFs. The systematic uncertainty due to the non-$D^-_s$ background is estimated by varying the shape smoothness parameter from 3 to 2.
For different sources, the changes of the fitted signal yield, 0.64\%, 0.18\%, 0.24\% and 0.10\%, are taken as individual uncertainties.
The total systematic uncertainty due to the $M_{\rm miss}^2$ fit is obtained to be 0.72\% by adding all four uncertainties in quadrature.

\item
Quoted BFs: The uncertainty due to the quoted BFs of $D_s^{*-}$ subdecays from the PDG~\cite{PDG2022} is examined by varying each subdecay BF by $\pm 1\sigma$. The change of the signal efficiency, 0.34\%, is taken as the associated uncertainty.
\item
Contribution from $D^+_s\to\gamma\mu^+\nu_\mu$:
The systematic uncertainty due to the contribution from the background of $D^+_s\to\gamma\mu^+\nu_\mu$ is estimated with the known upper limit on the BF of $D^+_s\to\gamma e^+\nu_e$~\cite{PDG2022}. After fixing this background yield in the $M_{\rm miss}^2$ fit, the change of the measured BF to the nominal one, 0.30\%, is taken as the corresponding systematic uncertainty.
\end{itemize}

Assuming all systematic uncertainties are independent, the total systematic uncertainty in the measurement of the BF of $D_s^+\to\mu^+\nu_\mu$ is
1.61\% by adding them in quadrature.
Here, 1.61\% corresponds to the absolute systematic uncertainty of 0.0085\% for the measured BF.

\begin{table}[htbp]\centering
  \caption{Relative systematic uncertainties in the measurement of the BF of $D_s^+\to\mu^+\nu_\mu$.}
  \label{tab:sys_crs}
  \begin{tabular}{lc}
  \hline\hline
  Source & Uncertainty (\%)\\ \hline
  ST yield        & 0.44\\
  $\mu^+$ tracking & 0.24\\
  $\mu^+$ PID      & 0.19\\
  Transition $\gamma(\pi^0)$ reconstruction & 1.00\\
  Least $|\Delta E|$ selection & 0.70\\
  $E_{\rm max}^{{\rm extra~}\gamma}$ \& $N_{\rm ncharged}^{\rm extra}$ requirements & 0.29\\
  $M_{\rm miss}^{2}$ fit & 0.72\\
  Quoted BFs & 0.34\\
  Contribution from $D^+_s\to\gamma\mu^+\nu_\mu$ & 0.30\\
  \hline
  Total & 1.61\\
  \hline\hline
  \end{tabular}
\end{table}


\section{Results of $f_{D_s^+}$ and $|V_{cs}|$}

Combining the measured BF with
the world average values of $G_F$, $m_\mu$, $m_{D^+_s}$ and
the $D_s^+$ lifetime ~\cite{PDG2022} in Eq. (1) yields
\begin{equation}
f_{D_s^+}|V_{cs}|=241.8\pm2.5_{\rm stat}\pm2.2_{\rm syst}~\mathrm{MeV}.
\nonumber
\end{equation}
Here the systematic uncertainty arises mainly from the uncertainties in the measured
BF (0.8\%) and the lifetime of the $D^+_s$ (0.4\%).
Taking the CKM matrix element $|V_{cs}|=0.97349\pm0.00016$ from the global SM fit
~\cite{PDG2022} or the averaged decay constant
$f_{D_s^+}=249.9\pm0.5~\mathrm{MeV}$ from recent LQCD calculations~\cite{FLab2018,etm2015}
as input, we determine
\begin{equation}f_{D_s^+}=248.4\pm2.5_{\rm stat}\pm2.2_{\rm syst}~\mathrm{MeV}
\nonumber
\end{equation}
and
\begin{equation}
|V_{cs}|=0.968\pm0.010_{\rm stat}\pm0.009_{\rm syst}.
\nonumber
\end{equation}
The additional systematic uncertainties according to the input parameters are
negligible for $|V_{cs}|$ and 0.2\% for $f_{D_s^+}$.

Using the BESIII combined result of ${\mathcal{B}}_{D_s^+\to\tau^+\nu_\tau}=(5.32\pm0.07\pm0.07)\%$ ~\cite{bes35}, we obtain $\mathcal{B}_{D_s^+\to\tau^+\nu_\tau}/\mathcal{B}_{D_s^+\to\mu^+\nu_\mu}=10.05\pm0.35$, where the statistical and systematic uncertainties have been combined in quadrature.
This ratio agrees with the SM predicted value of 9.75 within uncertainty.

\begin{table*}[htbp]\centering
  \caption{Comparison between ${\mathcal B} (D^+_s \to \mu^+\nu_\mu)$ and  $f_{D^+_s}|V_{cs}|$ measured by various experiments. The first uncertainties are statistical, the second are systematic, and the third are from the input lifetime of $D^+_s$. The superscripts $a$ and $b$ denote the measurements are made with and without using the $d_{\mu^+}$ requirements.}
  \label{tab:bf_muv}
  \scalebox{0.8}{
  \begin{tabular}{lc r@{\,}c@{\,}l c r@{\,}c@{\,}c@{\,}c@{\,}c@{\,}c@{\,}l}
  \hline
  \hline
  Experiment &$E_{\rm cm}$ (GeV) &\multicolumn{3}{c}{Reaction chain} &\multicolumn{1}{c}{$\mathcal B$ (\%)}&\multicolumn{7}{c}{$f_{D^+_s}|V_{cs}|$ (MeV)}\\ \hline
  CLEO~\cite{cleo2009}        &4.170       &$e^+e^-$&$\to$&$D_s^{\pm}D_s^{*\mp}$ &$0.565\pm0.045\pm0.017$  &$249.8$&$\pm$&$10.0$&$\pm$&$3.8$&$\pm$&$1.0$\\
  BaBar~\cite{babar2010}      &10.56       &$e^+e^-$&$\to$&$DKX\gamma D_s^-$     &$0.602\pm0.038\pm0.034$  &$257.8$&$\pm$&$8.2$&$\pm$&$7.3$&$\pm$&$1.0$\\
  Belle~\cite{belle2013}      &10.56       &$e^+e^-$&$\to$&$DKX\gamma D_s^-$     &$0.531\pm0.028\pm0.020$  &$242.2$&$\pm$&$6.4$&$\pm$&$4.7$&$\pm$&$1.0$\\
  BESIII$^b$~\cite{bes2016}       &4.009       &$e^+e^-$&$\to$&$D_s^{+}D_s^{-}$ &$0.517\pm0.075\pm0.021$  &$238.9$&$\pm$&$17.5$&$\pm$&$4.9$&$\pm$&$0.9$\\
  BESIII$^a$~\cite{besiii2}   &4.178       &$e^+e^-$&$\to$&$D_s^{\pm}D_s^{*\mp}$ &$0.549\pm0.016\pm0.015$  &$246.2$&$\pm$&$3.6$&$\pm$&$3.4$&$\pm$&$1.0$\\
  BESIII$^b$~\cite{Hajime}    &4.178-4.226 &$e^+e^-$&$\to$&$D_s^{\pm}D_s^{*\mp}$ &$0.535\pm0.013\pm0.016$  &$243.1$&$\pm$&$3.0$&$\pm$&$3.6$&$\pm$&$1.0$\\
{\bf This work$^a$}    &{\bf 4.128-4.226}  &{\boldmath$e^+e^-$}&{\boldmath$\to$}&{\boldmath$D_s^{\pm}D_s^{*\mp}$} &{\boldmath$0.5294\pm0.0108\pm0.0085$}
&{\boldmath$241.8$}&{\boldmath$\pm$}&{\boldmath$2.5$}&{\boldmath$\pm$}&{\boldmath$2.2$}&{\boldmath$\pm$}&{\boldmath$1.0$}\\
   \hline
  \hline
  \end{tabular}
  }
\end{table*}

\section{Summary}
In summary, by analyzing 7.33~fb$^{-1}$ of $e^+e^-$ collision data
collected at $E_{\rm cm}$ from 4.128 to 4.226\,GeV with the BESIII detector,
we measure the $\mathcal{B}(D^+_s\to\mu^+\nu_\mu)$, the product of $f_{D^+_s}|V_{cs}|$, the decay constant $f_{D_s^+}$, and the CKM matrix element $|V_{cs}|$. Comparison between the $\mathcal{B}(D^+_s\to\mu^+\nu_\mu)$ and $f_{D^+_s}|V_{cs}|$ obtained in this work and the previous measurements are shown in Table~\ref{tab:bf_muv}. These results are important to calibrate various theoretical calculations of $f_{D_s^+}$ and test the unitarity of the CKM matrix with better accuracy. We also test LFU with $D_s^+\to\ell^+\nu_\ell$ decays, and no violation is found.

\section{Acknowledgement}
The BESIII Collaboration thanks the staff of BEPCII and the IHEP computing center for their strong support. This work is supported in part by National Key R\&D Program of China under Contracts Nos. 2020YFA0406400, 2020YFA0406300; National Natural Science Foundation of China (NSFC) under Contracts Nos. 11875170, 12035009, 11635010, 11735014, 11835012, 11935015, 11935016, 11935018, 11961141012, 12022510, 12025502, 12035013, 12061131003, 12192260, 12192261, 12192262, 12192263, 12192264, 12192265, 12221005, 12225509, 12235017; the Chinese Academy of Sciences (CAS) Large-Scale Scientific Facility Program; the CAS Center for Excellence in Particle Physics (CCEPP); CAS Key Research Program of Frontier Sciences under Contracts Nos. QYZDJ-SSW-SLH003, QYZDJ-SSW-SLH040; 100 Talents Program of CAS; The Institute of Nuclear and Particle Physics (INPAC) and Shanghai Key Laboratory for Particle Physics and Cosmology; ERC under Contract No. 758462; European Union's Horizon 2020 research and innovation programme under Marie Sklodowska-Curie grant agreement under Contract No. 894790; German Research Foundation DFG under Contracts Nos. 443159800, 455635585, Collaborative Research Center CRC 1044, FOR5327, GRK 2149; Istituto Nazionale di Fisica Nucleare, Italy; Ministry of Development of Turkey under Contract No. DPT2006K-120470; National Research Foundation of Korea under Contract No. NRF-2022R1A2C1092335; National Science and Technology fund of Mongolia; National Science Research and Innovation Fund (NSRF) via the Program Management Unit for Human Resources \& Institutional Development, Research and Innovation of Thailand under Contract No. B16F640076; Polish National Science Centre under Contract No. 2019/35/O/ST2/02907; The Swedish Research Council; U. S. Department of Energy under Contract No. DE-FG02-05ER41374.

\bibliography{bibliography}

\end{document}
