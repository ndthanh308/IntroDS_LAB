\documentclass[11pt]{article}
\usepackage{enumerate}
\usepackage{amsthm,amsmath,amssymb}
\usepackage{graphicx}
\usepackage{lineno}
\usepackage[colorlinks=true,citecolor=black,linkcolor=black,urlcolor=blue]{hyperref}
\usepackage[english]{babel}
\usepackage{amsfonts}
%\usepackage{amsthm}
%\usepackage{latexsym}
\usepackage{epsfig, subfigure}
\usepackage{amscd,latexsym}
%\usepackage[pdftex]{graphicx}
%\usepackage{graphicx}
\usepackage{url}
\usepackage{color}
\usepackage{authblk}
\usepackage{subfigure}

\usepackage{multicol}  

\usepackage{enumitem}


\textwidth=15.5cm
\textheight=21.9cm
\voffset=-1cm
\hoffset=-1.8cm
%\setlength{\oddsidemargin}{1mm} \setlength{\evensidemargin}{1mm}
%\setlength{\marginparwidth}{0mm} \setlength{\marginparsep}{0mm}



\theoremstyle{plain}
\newtheorem{theorem}{Theorem}
\newtheorem{lemma}[theorem]{Lemma}
\newtheorem{cor}[theorem]{Corollary}
\newtheorem{prop}[theorem]{Proposition}
\newtheorem{conj}[theorem]{Conjecture}
\newtheorem{fact}[theorem]{Fact}
\newtheorem{observation}[theorem]{Observation}
\newtheorem{claim}[theorem]{Claim}
\newtheorem{clm}{Claim}{\itshape}{\rmfamily}
\newtheorem{remark}[theorem]{Remark}
\newtheorem{note}[theorem]{Note}
\newtheorem{res}[theorem]{Result}
\newtheorem{defi}[theorem]{Definition}



\newcommand{\gp}{\gamma_{\stackrel{}{P}}}
\newcommand{\ra}{\rm rank \it}
\newcommand{\nul}{\rm null \it}
\newcommand{\pt}{\rm pt \it}
\newcommand{\mr}{\rm mr \it}
\newcommand{\M}{\rm M \it}
\newcommand{\Z}{\rm Z \it}
\newcommand{\C}{\rm Comp \it}
\newcommand{\ter}{\rm ter \it}
\newcommand{\Mc}{\cal M \it}

\newcommand{\ds}{\displaystyle }


\newcommand{\Ct}{\rm Ct} \newcommand{\Ext}{\rm Ext}
\newcommand{\Per}{\rm Per} \newcommand{\Ecc}{\rm Ecc}
\newcommand{\ecc}{\rm ecc} \newcommand{\gin}{\rm gin}
\newcommand{\edim}{\rm edim}%edge metric dimension
\newcommand{\sdim}{\rm sdim}%strong metric dimension
\newcommand{\ldim}{\rm ldim}%local metric dimension
\newcommand{\adim}{\rm adim}%metric antidimension
\newcommand{\mdim}{\rm mdim}%mixed  metric dimension
\newcommand{\omdim}{\rm omdim}%outer multiset dimension
\newcommand{\tmdim}{\rm tmdim}%(total) multiset dimension
\newcommand{\fdim}{\rm fdim}%fractional metric dimension
\newcommand{\ddim}{\rm ddim}%Dominating metric dimension%metric-location-domination number
\newcommand{\pdim}{\rm pdim}%partition dimension
\newcommand{\tdim}{\rm tdim}%threshold dimension
\newcommand{\smd}{\rm smd}%solid-metric dimension
\newcommand{\tmd}{\rm tmd}%truncated metric dimension
\newcommand{\cdim}{\rm cdim}%Connected metric dimension
\newcommand{\idim}{\rm idim}%Independent metric dimension
\newcommand{\spdim}{\rm spdim}%Strong partition  dimension
\newcommand{\twodim}{\rm dim\textsubscript{2}} %Fault-tolerant metric dimension
\newcommand{\dmd}{\rm dmd}%doubly metric dimension

\newcommand{\ex}{\rm \lambda}%number of exterior major vertices


\newcommand{\bit}{\begin{itemize}}
\newcommand{\eit}{\end{itemize}}
\newcommand{\ben}{\begin{enumerate}}
\newcommand{\een}{\end{enumerate}}


\begin{document}

\title{Metric Location in Pseudotrees: A survey and new results}



\author[1]{Jos\'e C\'aceres\thanks{jcaceres@ual.es}}
\author[2]{Ignacio M. Pelayo\thanks{ignacio.m.pelayo@upc.edu}}



\affil[1]{Departmento de Matem\'aticas, Universidad de Almer\'{\i}a, Spain}
\affil[2]{Departament de Matem\`atiques, Universitat Polit\`ecnica de Catalunya, Spain}




%\date{}


\maketitle

\begin{abstract}
The aim of this paper is to revise the literature on different metric locations in the families of paths, cycles, trees and unicyclic graphs, as well as, provide several new results on that matter.
\end{abstract}


%\linenumbers

%\tableofcontents

%\printindex



%%%%%%%%%%%%%%%%%%%%%%%%%%%%%%
%%%%%%%%%%%%%%%%%%%%%%%%%%%%%%

%%%%%%%%%%%%%%%%%%%%%%%%%%%%%%
%%%%%%%%%%%%%%%%%%%%%%%%%%%%%%




\vspace{+.1cm}\noindent \textbf{Keywords:} Metric dimension.

\vspace{+.1cm}\noindent \textbf{AMS subject classification:} 05C50, 15A18, 05C69.


%%%%%%%%%%%%%%%%%%%%%%%%%Introduction%%%%%%%%%%%%%%%%%%%%%%%%%%%%%%%%%%%
%      introduccion
%%%%%%%%%%%%%%%%%%%%%%%%%Introduction%%%%%%%%%%%%%%%%%%%%%%%%%%%%%%%%%%%



%\end{document}


%%%%%%%%%%%%%%%%%%%%%%%%%%%%%%%%%%%%%%%%%%%%%%%%%%%%%%%%%%%%%%%%%%%%%%%%%%%%%%%%%%%%%%%%%%%%%%%%%%
%%%%%%%%%%%%%%%%%%%%%%%%%%%%%%%%%%%%%%%%%%%%%%%%%%%%%%%%%%%%%%%%%%%%%%%%%%%%%%%%%%%%%%%%%%%%%%%%%%
%%%%%%%%%%%%%%%%%%%%%%%%%%%%%%%%%%%%%%%%%%%%%%%%%%%%%%%%%%%%%%%%%%%%%%%%%%%%%%%%%%%%%%%%%%%%%%%%%%
%%%%%%%%%%%%%%%%%%%%%%%%%%%%%%%%%%%%%%%%%%%%%%%%%%%%%%%%%%%%%%%%%%%%%%%%%%%%%%%%%%%%%%%%%%%%%%%%%%
\section{Introduction}\label{sec1:intro}

Location problems consist of determining a reference set in a graph such that every vertex is unequivocally associated to a set of ``coordinates'' that localize it. 
Since the vertex set of a graph is enough for this task, the question is not to prove its existence, but to find the minimum one. 

Metric location in graphs were officially come into being with the publication of two seminal papers by P. Slater~\cite{s75} and F. Harary and R. A. Melter~\cite{hm76}. 
However the field can be traced back to a paper by R. Silverman~\cite{s60} about the problem of detecting a certain subset of a given set by comparing it with some other subsets. 
The approach to solve that problem was to define a distance between the different subsets of a set and to choose carefully the comparisons needed. 
To the best of our knowledge, in this work the concept of metric basis is named for the first time.

The adjective ``metric'' is used through the paper to refer those locations in which the usual distance between graph vertices plays a crucial role, in the spirit of the original definition given in~\cite{s75,hm76}. By contrast, there exist other types of locations in which the role of the graph distance is substituted by the relation of adjacency of vertices, given rise to ``neighbor'' locations which are not treated here.

Since the publication of the works by P. Slater~\cite{s75} and F. Harary and R. A. Melter~\cite{hm76}, metric location theory has evolved at great velocity giving rise to a hundred of different variations, developing many new methods and deepening into several directions, although the first of those variations had to wait 20 years to appear in a paper published in 2003 by R. C. Brighham, G. Chartrand, R. D. Dutton and P. Zhang~\cite{bcdz03} who imposed the condition of being dominating to the location sets. 
Almost immediately, in 2004, A. Seb\H{o} and E. Tannier~\cite{st04} published their definition of strong metric dimension by using a certain notion of betweeness in geodetic paths. Just a year later, J.  Cáceres, C. Hernando, M. Mora, I. M. Pelayo, M. L. Puertas, C. Seara and D. R. Wood~\cite{chmppsw07} defined what is  currently  known as the  doubly metric dimension in which vertices have not only different coordinates but also, the coordinates of two vertices have not a fixed difference. 
C. Hernando, M. Mora, P. Slater and D. R.  Wood~\cite{hmsw08} asked themselves in 2008 what if an element of the metric location set ``fails" or give incorrect lectures, and gave rise to the fault-tolerant metric dimension. In 2010, F. Okamoto, B. Phinezy and P. Zhang showed in~\cite{opz10} a local metric dimension to distinguish only adjacent vertices instead of using a coloring. In 2015, A. Estrada-Moreno, I.G. Yero and J.A. Rodríguez-Velázquez in~\cite{eyr14} generalized the fault-tolerant dimension into the k-metric dimension by considering the case in which more than one element of the location set might fail. In 2017, A. Kelenc, D. Kuziak, A. Taranenko and I.G. Yero~\cite{kkty17} distinguished vertices and edges with the mixed metric dimension, and a year later, A. Kelenc, N. Tratnik and I.G. Yero~\cite{kty18} introduced the edge metric dimension for locating only edges.

One of the crucial facts of the field is that computing a minimum metric location set is typically NP-hard. 
Therefore, it becomes essential to identify graph families for which this computation can be accomplished efficiently in polynomial time. 
Among these families, trees are often considered, and various algorithms have been developed specifically for this purpose.

Since trees do not contain cycles, it is natural to explore the graph family characterized by a single cycle: unicyclic graphs. 
There exists an extensive volume of research papers that makes it exceedingly difficult to follow the advancements in this particular area. 
Consequently, the authors recognized the necessity of conducting a comprehensive survey on the topic of metric location within this graph family. Although there are many location parameters, including variations of variations, we opted to choose nine of them having in common that the corresponding dimensions for paths, cycles and trees are well-known (see Table~\ref{megatable1}). However, not in all cases a complete characterization of locations in unicyclic graphs has been found and we were able to provide some new results in four of the dimensions studied. 

The paper is organized as follows: after concluding this section by a description of the notation that will be followed in the rest of the work, Section~\ref{dmd} is dedicated to doubly metric locating sets where the work has been completed. This metric location parameter is chosen to be the first  because of the application it has for the rest of metric locations. 
Section~\ref{dim} is devoted to the original and first metric location that gave rise to the entire field. Here we provided significant new results as well as for the  strong metric dimension which is studied in Section~\ref{sdim}. We finish with the dominating metric dimension in Section~\ref{ddim} where again the goal has been achieved.

In two of the remaining five parameters (mixed and local metric dimensions) other authors have proved bounds and exact values (see Table~\ref{megatable2}), however for the edge, fault-tolerant and k-metric dimensions (see Table~\ref{megatable3}) there exist partial results but not a complete characterization of those dimensions for unicyclic graphs. In Section~\ref{omlp}, we offer a glimpse of the actual landscape for those cases and the paper is finished with a section of conclusions and further work.



%%%%%%%%%%%%%%%%%%%%%%%%%%%%%%%%%%%%%%%%%%%%%%%%%%%%%%%%%%%%%%%%%%%%%%%%%%%%%%%%%%%%%%%%%%%%%%%%%%
%%%%%%%%%%%%%%%%%%%%%%%%%%%%%%%%%%%%%%%%%%%%%%%%%%%%%%%%%%%%%%%%%%%%%%%%%%%%%%%%%%%%%%%%%%%%%%%%%%
\subsection{Basic terminology}

All the graphs considered are undirected, simple, finite and (unless otherwise stated) connected.
Unless otherwise specified,  $G=(V,E)$ stands for a graph of order $n$ and size $m=1$, being $|V|=n$, $V=[n]=\{1,\ldots,n\}$ and $|E|=m$.
Let $v$ be a vertex of $G$.
The \emph{open neighborhood} of $v$ is $\displaystyle N_G(v)=\{w \in V(G) :vw \in E\}$, and the \emph{closed neighborhood} of $v$ is $N_G[v]=N_G(v)\cup \{v\}$ (we will write $N(v)$ and $N[v]$ if the graph $G$ is clear from the context).
The \emph{degree} of $v$ is $\deg(v)=|N(v)|$.
The minimum degree  (resp. maximum degree) of $G$ is $\delta(G)=\min\{\deg(u):u \in V(G)\}$ (resp. $\Delta(G)=\max\{\deg(u):u \in V(G)\}$).


For any two vertices $u,v\in V(G)$ of a connected graph $G$, a $u-v$ \emph{geodesic} is a  $u-v$ shortest  path, i.e., a $u-v$ path of minimum order. 
The length of a $u-v$ geodesic is called the \emph{distance} $d_G(u,v)$ between $u$ and $v$, or simply  $d(u,v)$, when the graph $G$ is clear from the context. 
The diameter of $G$ is ${\rm diam}(G) = \max\{d(v,w) : v,w \in V(G)\}$.
The distance between a vertex $v\in V(G)$ and a set of vertices $S\subseteq V(G)$, denoted by  $d(v,S)$, is the minimum of the distances between $v$ and the vertices of $S$, that is, $d(v,S)=\min\{d(v,w):w\in S\}$.
Given a vertex $v \in V(G)$ and an edge $e = xy \in E(G)$, the distance\ between $v$ and $e$ is $d_G(v, e) =\min\{d_G(v,x), d_G(v,y)\}$.


Let $u,v \in V(G)$ be  a pair of vertices such that  $d(u,w)=d(v,w)$ for all $w\in V(G)\setminus\{u,v\}$, i.e.,  such that  either $N(u)=N(v)$ or $N[u]=N[v]$.
In both cases, $u$ and $v$ are said to be \emph{twins}.
Let $W$ be a set of vertices of $G$.
If the vertices of $W$ are pairwise twins, then $W$ is called a \emph{twin set} of $G$.

Let $W\subseteq V(G)$ be a subset of vertices of  $G$.
The  \emph{closed neighborhood} of $W$ is $N[W]=\cup_{v\in W} N[v]$.
The subgraph of $G$ induced by $W$, denoted by $G[W]$, has $W$ as vertex set and $E(G[W]) = \{vw \in E(G) : v \in W,w \in W\}$.

For a graph $G$ of minimum degree $\delta(G)=1$, end-vertices are called \emph{leaves}.
The set and the number of leaves of $G$ are denoted by ${\cal L}(G)$ and $\ell(G)$, respectively.
A \emph{support vertex} (resp. \emph{strong support vertex}) is a vertex adjacent to a leaf (resp., to at least two leaves).
A \emph{major vertex} of $G$ is any vertex of degree at least 3.
A \emph{terminal vertex} of a major vertex $v$ of $G$ is a leaf $u$ such that 
$d(u,v)<d(u,w)$ for every other major vertex $w$ of $G$.
An \emph{exterior major vertex}  
(resp., \emph{strong exterior major vertex})
is a major vertex  with at least one terminal vertex (resp., at least two terminal vertices)
(see Figure \ref{niceunic}).


%%%%%%%%%%%%%%%%%%%%%%%%%%%%%%%%%%
% Figure environment removed
%%%%%%%%%%%%%%%%%%%%%%%%%%%%%%%%%%


The number of exterior major vertices (resp., strong exterior major vertices) of $G$ is denoted by $\lambda(G)$ (resp. $\lambda_s(G)$).
A leaf of $G$ is called \emph{strong} if its exterior major vertex is strong.
The number of strong leaves of $G$ is denoted  by $\ell_s(G)$.
Clearly, $\ell_s(G)-\lambda_s(G)=\ell(G)-\lambda(G)$.

A graph $G$ of order $n$ and size $m$ is called \emph{unicyclic} if $n=m$.
Let  $C_g$ be its unique cycle, being $g$ the girth of $G$.
A unicyclic graph $G$ of order $n$ and girth $g$ is said to be \emph{proper} if $n>g$, i.e. if it contains at least a leaf.
If $g$ is odd (rep., even), then $G$ is said to be an \emph{odd (rep., even) unicyclic graph}.
The connected component of $G-E(C_g)$ containing a vertex $v\in V(C_g)$  is denoted by $T_v$ and  is called the \emph{branching tree} of $v$.
The tree $T_v$ is said to be \emph{trivial} if $V(T_v)=\{v\}$.
A branching tree is called a \emph{thread} if both $T_v$ is a path and $\deg(v)=3$.
A vertex $v\in V(G)$ is a \emph{branching vertex} if  either $v \not\in V (C_g)$ and $\deg (v) \ge 3$ or $v \in V (C_g)$ and $\deg (v ) \ge 4$. 
A vertex $v \in V (C_g)$ is \emph{branch-active} if $T_v$ contains a branching vertex.
The number of branch-active  vertices  of $C_g$ is denoted by $\rho(G)$.



A triple of vertices $u,v,w \in V(C_g)$ is called a \emph{geodesic triple} of $G$ if
$d(u,v)+d(v,w)+d(w,u)=g$.
A vertex $v$ from $C_g$ is called a \emph{root vertex} on $C_g$ if $T_v$ is non-trivial and otherwise,  $v$ is said to be a \emph{trivial vertex} of $C_g$. 
The set and number of all root vertices of $C_g$  is denoted by $C_3(C_g)$ and $c_3(C_g)$, while the set and the number of all trivial vertices of $C_g$  are denoted by $C_2(C_g)$ and $c_2(C_g)$, respectively. 
Certainly, every vertex of $C_g$ of  degree at least 3 is a root vertex, while all vertices of $C_g$ that are of degree 2 are trivial.






%The \emph{complement} of $G$,  denoted by $\overline{G}$, is the  graph on the same vertices as $G$ such that two vertices are adjacent in $\overline{G}$ if and only if they are not adjacent in $G$.
%Let $G_1$, $G_2$ be two graphs having disjoint vertex sets.
%The (\emph{disjoint}) \emph{union} $G=G_1+G_2$ is the graph such that  $V(G)=V(G_1)\cup V(G_2)$ and $E(G)=E(G_1)\cup E(G_2)$.
%The \emph{join} $G=G_1\vee G_2$ is the graph such that
%$V(G)=V(G_1)\cup V(G_2)$ and $E(G)=E(G_1)\cup E(G_2)\cup \{uv:u\in V(G_1),v\in V(G_2)\} $.}
%
%
%The line graph of a graph $G = (V,E)$, denoted $L(G)$, is the graph having vertex set $E$, with
%two vertices in $L(G)$ adjacent if and only if the corresponding edges share an endpoint in $G$.


%Let $K_n$, $P_n$, $W_n$ and $C_n$ denote, respectively, the complete graph, path, wheel and cycle of order $n$; 
%$K_{1,n}$ denotes the star with $n + 1$ vertices such that $n$ of them are end-vertices.


A set $U$ of vertices of a graph $G$ is called \emph{independent} if no two vertices in $U$ are adjacent. 
The \emph{independence number} of $G$, denoted by $\alpha(G)$, is the cardinality of a maximum independent set of $G$.



A vertex $v$ of a graph $G$ is said to be a \emph{boundary vertex} 
of a vertex $u$ if no neighbor of $v$ is further away from $u$ than $v$, i.e., if for every vertex $w\in N(v)$, 
$d(u,w) \le d(u,v)$. 
The set of boundary vertices of a vertex $u$ is denoted by $\partial(u)$.
The \emph{boundary} of $G$, denoted by $\partial(G)$, is the set of all of its boundary vertices, i.e., 
$\partial(G)=\cup_{u \in V(G)}\partial(u)$.
Given a  pair of vertices $u,v\in V(G)$ if $v\in \partial(u)$, then $v$ is also said to be \emph{maximally distant} from $u$.
Moreover, a pair of  vertices $u,v\in V(G)$ are said to be \emph{mutually maximally distant}, or simply $MMD$, if both $v\in \partial(u)$ and $u\in \partial(v)$.
Notice that, as was pointed out in \cite{ryko14}, the boundary of $G$ can also be defined as the set of $MMD$ vertices of $G$, i.e., 
$\partial(G)=\{v\in V(G): {\rm there \, exists} \,  u\in V(G) \, {\rm such \, that} \, u,v \,{\, \rm are} \,  MMD \}$.
A pair of $MMD$  vertices of a cycle are also called \emph{antipodal}.



For additional details and information on basic graph theory, we refer the reader
to \cite{clz16}.





%%%%%%%%%%%%%%%%%%%%%%%%%%%%%%%%%%%%%%%%%%%%%%%%%%%%%%%%%%%%%%%%%%%%%%%%%%%%%%%%%%%%%%%%%%%%%%%%%%
%%%%%%%%%%%%%%%%%%%%%%%%%%%%%%%%%%%%%%%%%%%%%%%%%%%%%%%%%%%%%%%%%%%%%%%%%%%%%%%%%%%%%%%%%%%%%%%%%%
%%%%%%%%%%%%%%%%%%%%%%%%%%%%%%%%%%%%%%%%%%%%%%%%%%%%%%%%%%%%%%%%%%%%%%%%%%%%%%%%%%%%%%%%%%%%%%%%%%
%%%%%%%%%%%%%%%%%%%%%%%%%%%%%%%%%%%%%%%%%%%%%%%%%%%%%%%%%%%%%%%%%%%%%%%%%%%%%%%%%%%%%%%%%%%%%%%%%%
\section{Doubly metric dimension}\label{dmd}

This parameter was formally introduced by J. C\'aceres et al. in \cite{chmppsw07} as a means to study the metric dimension in grids, cylinders and tori. 

In this paper, it will also be  a powerful tool to study unicyclic graphs, so we commence with this section.  

Let $G=(V,E)$ be a graph.
A pair of vertices $u, v\in V$ is said to \emph{doubly resolve} two vertices $x,y \in V$ if
$$d_G(x,u)-d_G(x,v) \neq d_G(y,u)-d_G(y,v).$$



%%%%%%%%%%%%%%%%%%
\begin{defi} \label{dls} 
{\rm A  set of vertices $S$ of  $G$ is called \emph{doubly locating} if every pair of  vertices $x,y \in V$ is doubly resolved by some pair  of vertices of $S$.}
\end{defi}
%%%%%%%%%%%%%%%%%%


A doubly locating set of  minimum cardinality is called  a \emph{doubly metric basis} of $G$. 
The \emph{doubly metric dimension} of $G$, denoted by $\dmd(G)$,  is the  cardinality of a doubly metric basis. 
To know more about this parameter see mainly  \cite{chmppsw07,j22.2} and also \cite{kkcs12.1,kkcs12.2,mkkc12}.

The problem of computing the doubly metric dimension of trees was approached and completely solved in \cite{chmppsw07}.


%%%%%%%%%%%%%%%%%
%%%%%%%%%%%%%%%%%%
%%%%%%%%%%%%%%%%%%
\begin{theorem}
{\rm\cite{chmppsw07}}
Let  $T$ be  a tree  having $\ell(T)$ leaves. 
Then, $\dmd(T)  = \ell(T)$.
\label{dmd.trees}
\end{theorem}
%%%%%%%%%%%%%%%%%
%%%%%%%%%%%%%%%%%%
%%%%%%%%%%%%%%%%%%


In particular, $\dmd(P_n)  = 2$.
The problem of characterizing all minimum doubly locating sets of a cycle $C_g$ was  implicitly approached and  partially proved in  \cite{chmppsw07}.


%%%%%%%%%%%%%%%%%
%%%%%%%%%%%%%%%%%%
%%%%%%%%%%%%%%%%%%
\begin{theorem}
%{\rm\cite{chmppsw07}}
Consider the cycle $C_g$ and assume that $V(C_g)=[g]$.
Then,

\begin{enumerate}[label=\rm \bf(\arabic*)]

\item If $g$ is odd, then $\dmd(C_g)=2$.
\newline
Moreover, a pair of vertices is a doubly metric basis if and only if they are antipodal.

\item
If $g$ is even, then $\dmd(C_g)=3$.
\newline
Moreover, a triple of vertices is a doubly metric basis if and only if it is a geodesic triple.

\end{enumerate}
\label{dmd.cycles}
\end{theorem}
%%%%
\begin{proof}

\begin{enumerate}[label=\rm \bf(\arabic*)]

\item
Suppose that $g=2k+1$ and take the antipodal pair $\{1,k+1\}$.
Notice that for every $i \in [g]$:

$$ d(i,k+1)-d(i,1)=   
\left \{ \begin{array}{cl}
k-2i & {\rm if}\, 1 \le i \le k+1, \\
2i-3k-3 & {\rm if}\, k+2 \le i \le 2k+1.  \\
\end{array}\right . $$

Let $\{i,j\}$ such that $1 \le i < j \le 2k+1$.
We distinguish cases:

{\bf Case (a)}:
If $1 \le i < j \le k+1$, then $d(i,k+1)-d(i,1) = k-2i \neq k-2j = d(j,k+1)-d(j,1)$.


{\bf Case (b)}:
If $k+2 \le i < j \le 2k+1$, then $d(i,k+1)-d(i,1) = 2i-3k-3 \neq 2j-3k-3 = d(j,k+1)-d(j,1)$.


{\bf Case (c)}:
If $1 \le i \le k+1$ and $k+2 \le  j \le 2k+1$, then $d(i,k+1)-d(i,1) = k-2i \neq 2j-3k-3 = d(j,k+1)-d(j,1)$, since $k-2i = 2j-3k-3$ if and only if $2j-2i = -4k-3$.

To end the proof of this item, consider a pair of non-antipodal vertices $\{1,h\}$.
W.l.o.g., we can assume that $1 < h < k+1$.
Notice that the pair $\{h,h+1\}$ is not doubly resolved by $\{1,h\}$, since $d(h,h)-d(h,1)=1-h=d(h+1,h)-d(h+1,1)$.


\item
Suppose that $g=2k$ and take a pair of vertices $\{1,h\}$.
W.l.o.g., we can assume that $1 < h \le k$.
If $h<k$, then the pair $\{k,k+1\}$ is not doubly resolved by $\{1,h\}$, 
since $d(k,h)-d(k,1)=1-h=d(k+1,h)-d(k+1,1)$.
If $h=k$, then the pair $\{k-1,k+1\}$ is not doubly resolved by $\{1,k\}$, 
since $d(k-1,k)-d(k-1,1)=3-k=d(k+1,k)-d(k+1,1)$.
Hence, $\dmd(C_g)\ge3$.


Take a geodesic triple of vertices $S$.
W.l.o.g., we can assume that $S=\{1,\alpha, \beta\}$ and $1 < \alpha < k \le \beta  \le 2k+1$.
Take a pair pf vertices $\{i,j\}$.
W.l.o.g., we can assume that $1 < i < \alpha$.
We distinguish cases:

{\bf Case (a)}:
If $1 < i < j < \alpha$, then $\{1,\alpha\}$ doubly resolves $\{i,j\}$, since
$d(i,\alpha)-d(i,1) = \alpha-2i+1 \neq \alpha-2j+1 = d(j,\alpha)-d(j,1)$.


{\bf Case (b)}:
If $\alpha < j \le k$, then 
$\{1,\alpha\}$ doubly resolves $\{i,j\}$, since
$d(i,\alpha)-d(i,1) = \alpha-2i+1 \neq -\alpha +1 = d(j,\alpha)-d(j,1)$.


{\bf Case (c)}:
If $k < j$, then 
$\{1,\beta\}$ doubly resolves $\{i,j\}$.
To prove this we distinguish cases.

{\bf Case (c.1)}:
If $i+k-1 \le \beta $ and $j < \beta$, then
$d(i,\beta)-d(i,1)=(i-1+2k-\beta)-(i-1)=2k-\beta$ and $d(j,\beta)-d(j,1)=(\beta-j)-(2k-j)=\beta-2k$.


{\bf Case (c.2)}:
If $i+k-1 \le \beta $ and $\beta < j$, then
$d(i,\beta)-d(i,1)=(i-1+2k-\beta)-(i-1)=2k-\beta$ and $d(j,\beta)-d(j,1)=(j-\beta)-(2k-j)=2j-\beta-2k$.
Notice that if  $2k-\beta=2j-\beta-2k$, then $j=2k$, a contradiction.


{\bf Case (c.3)}:
If $\beta < i+k-1$ and $j<\beta$, then
$d(i,\beta)-d(i,1)=(\beta-i)-(i-1)=\beta-2i+1$ and $d(j,\beta)-d(j,1)=(\beta-j)-(2k-j)=\beta-2k$.
Notice that if $\beta-2i+1=\beta-2k$, then $2i-1=2k$, a contradiction.



{\bf Case (c.4)}:
If $\beta < i+k-1$ and $ \beta < j$, then
$d(i,\beta)-d(i,1)=(\beta-i)-(i-1)=\beta-2i+1$ and $d(j,\beta)-d(j,1)=(j-\beta)-(2k-j)=2j-\beta-2k$.
Notice that if $\beta - 2i + 1 = 2j-\beta - 2k$, then $ 2i + 2j = 2\beta + 2k + 1$, a contradiction.

Hence, we have proved both that $\dmd(C_g)=3$ and that every geodesic triple is a doubly metric basis.
To end the proof of this item, consider a non-geodesic triple of vertices $S=\{ 1, \alpha, \beta \}$.
W.l.o.g., we can assume that $1 < \alpha < \beta < k$.
Notice that the pair $\{\beta,\beta+1\}$ is not doubly resolved neither by $\{1,\alpha\}$, nor by $\{1,\beta\}$, nor by $\{\alpha, \beta\}$, since for every $x\in S$, $d(\beta+1,x)=d(\beta,x)+1$.
\end{enumerate}
\end{proof}
%%%%%%%%%%%%%%%%%
%%%%%%%%%%%%%%%%%%
%%%%%%%%%%%%%%%%%%





In \cite{j22.2}, after noticing that every doubly locating set of a unicyclic graph $G$ must contain all its leaves, it was proved that if $G$ is a proper uniclyclic graph of girth $g$  with $\ell$ leaves,  then  $\ell \le \dmd(G) \le \ell+1$ if $g$ is odd, and $\ell \le \dmd(G) \le \ell+2$ if $g$ is even.
Starting both from this result and from Theorem \ref{dmd.cycles}, we have characterized the different families of unicyclic  graphs achieving each of the possible values for its doubly metric dimension.




%%%%%%%%%%%%%%%%%
%%%%%%%%%%%%%%%%%%
%%%%%%%%%%%%%%%%%%
\begin{theorem}
%{\rm\cite{chmppsw07}}
Let  $G$ be  a proper unicyclic graph with $\ell$ leaves. 
Then, 

\begin{enumerate}[label=\rm \bf(\arabic*)]

\item If $g$ is odd, then 
$ \dmd(G)  =  \left \{ \begin{array}{cl}
\ell  & {\rm if}\, C_3(C_g)\, {\rm contains \, an\, antipodal\, pair}, \\
\ell + 1 & {\rm otherwise}. \\  
\end{array}\right . $


\item
If $g$ is even, then
$ \dmd(G)  =  \left \{ \begin{array}{cl}
\ell  & {\rm if}\, C_3(C_g)\, {\rm contains \, a\, geodesic\, triple}, \\
\ell + 2 & {\rm if}\, c_3(C_g)=1, \\
\ell + 1 & {\rm otherwise}. \\  
\end{array}\right . $

\end{enumerate}
\label{dmd.unic1}
\end{theorem}
%%%%
\begin{proof}


\begin{enumerate}[label=\rm \bf(\arabic*)]


\item
If $C_3(C_g)$ contains an antipodal pair then, according to  Theorem \ref{dmd.cycles} {\bf(1)}, the set ${\cal L}(G)$ of leaves of $G$ is a doubly  locating set of $G$.

Conversely, if $C_3(C_g)$ contains no antipodal pairs, then to obtain a doubly  locating set of $G$, in addition to ${\cal L}(G)$, a trivial vertex $h$  of $C_g$ must be added, to be more precise, a vertex $h$ of $C_g$ antipodal to one of the vertices of $C_3(C_g)$.
Then, according to  Theorem \ref{dmd.cycles} {\bf(1)}, the set ${\cal L}(G)\cup \{h\}$  is a doubly locating set of $G$.

\item
If $C_3(C_g)$ contains a geodesic triple then, according to  Theorem \ref{dmd.cycles} {\bf(2)}, the set ${\cal L}(G)$ of leaves of $G$ is a doubly  locating set of $G$.

Conversely, suppose that $C_3(C_g)$ contains no geodesic triples. 
If $c_3(C_g)=1$ and $C_3(C_g)=\{h_1\}$, take a pair of trivial vertices $h_2,h_3 \in [g]$ such that that $\{h_1,h_2,h_3\}$ be a geodesic triple of $C_g$.
Then, according to  Theorem \ref{dmd.cycles} {\bf(2)}, the set ${\cal L}(G)\cup \{h_2,h_3\}$  is a minimum doubly locating set of $G$.
If $c_3(C_g)\ge 2$ and $\{h_1, h_2\}\subseteq C_3(C_g)$, take a trivial vertex $h_3 \in [g]$ such that that $\{h_1,h_2,h_3\}$ be a geodesic triple of $C_g$.
Then, according to  Theorem \ref{dmd.cycles} {\bf(2)}, the set ${\cal L}(G)\cup \{h_3\}$  is a minimum doubly locating set of $G$.

\end{enumerate}
\end{proof}
%%%%%%%%%%%%%%%%%
%%%%%%%%%%%%%%%%%%
%%%%%%%%%%%%%%%%%%


Hence, in this section we have completed previous works to obtain a total characterization of the doubly metric dimension of unicyclic graphs.













%%%%%%%%%%%%%%%%%%%%%%%%%%%%%%%%%%%%%%%%%%%%%%%%%%%%%%%%%%%%%%%%%%%%%%%%%%%%%%%%%%%%%%%%%%%%%%%%%%
%%%%%%%%%%%%%%%%%%%%%%%%%%%%%%%%%%%%%%%%%%%%%%%%%%%%%%%%%%%%%%%%%%%%%%%%%%%%%%%%%%%%%%%%%%%%%%%%%%
%%%%%%%%%%%%%%%%%%%%%%%%%%%%%%%%%%%%%%%%%%%%%%%%%%%%%%%%%%%%%%%%%%%%%%%%%%%%%%%%%%%%%%%%%%%%%%%%%%
%%%%%%%%%%%%%%%%%%%%%%%%%%%%%%%%%%%%%%%%%%%%%%%%%%%%%%%%%%%%%%%%%%%%%%%%%%%%%%%%%%%%%%%%%%%%%%%%%%
\section{Metric dimension}\label{dim}

Being introduced primarily in~\cite{hm76,s75}, a large part of the literature regarding with graph locating have to do with metric dimension, the first and more important way of locating vertices in a graph. 

Let $G=(V,E)$ be a graph.
A vertex $v\in V$ is said to \emph{resolve} two vertices $x,y \in V$ if $d_G(x,v)\neq d_G(y,v)$.


%%%%%%%%%%%%%%%%%%
\begin{defi} \label{ls} 
{\rm A  set of vertices $S$ of  $G$ is called \emph{metric-locating} if  every pair of distinct  vertices  $x,y \in V$ is resolved by a vertex of $S$.}
\end{defi}
%%%%%%%%%%%%%%%%%%

Metric-locating sets are also known as locating sets and resolving sets.
Moreover, if $S$ is a metric-locating set of a graph $G$, then it is usually said that $S$ resolves the set of vertices $V(G)$.


A metric-locating set of  minimum cardinality is called  a \emph{metric basis} of $G$. 
The \emph{metric dimension} of $G$, denoted by $\dim(G)$,  is the  cardinality of a metric basis. 
To know more about this parameter see mainly  \cite{chmppsw07, cejo00, hm76,s75} and also \cite{bcggmmp13, bc11, bm11, bdfhmp18, bcpz03,  cgh08, eky17,hmpscp05, hmpsw10, krr96, pz02, st04, ss21.5,ss22,ss22.2,ss22.1}.

It is straightforward to check, first,  that $\dim(P_n)=1$, being a vertex $v$  of $P_n$ a metric basis if and only if $v$ is a leaf, and second, that  $\dim(C_g)=2$, being any pair of vertices (resp., of non-antipodal vertices) of $C_g$ a metric basis if $g=2k+1$ is odd (resp., $g=2k$ is even).


%%%%%%%%%%%%%%%%%%
%%%%%%%%%%%%%%%%%%
\begin{lemma}\cite{cejo00}
\label{lem:cejo00}
If $G$ is a graph, then $\dim(G)\geq \ell(G)-\ex(G)$.
\end{lemma}
%%%%%%%%%%%%%%%%%%
%%%%%%%%%%%%%%%%%%

Moreover, it was also proved in \cite{cejo00} that any locating set of a graph $G$ must contain, at least, all but one of the terminal vertices of  every  exterior major vertex of $G$.



%%%%%%%%%%%%%%%%%
%%%%%%%%%%%%%%%%%%
%%%%%%%%%%%%%%%%%%
\begin{theorem}
{\rm\cite{cejo00,hm76,s75}}
Let  $T$ be  a tree  having $\ell(T)$ leaves and $\lambda(T)\geq 1$ exterior major vertices. 
Then, $$\dim(T)  = \ell(T)-\lambda(T).$$
\label{thmc3.f1}
\end{theorem}
%%%%%%%%%%%%%%%%%
%%%%%%%%%%%%%%%%%%
%%%%%%%%%%%%%%%%%%




In~\cite{ss21.5,ss22}, the authors gave a characterization of unicyclic graphs $G$ based on forbidden configurations. However this characterization is a bit cryptic and difficult to use for practical purposes.

%%%%%%%%%%%%%%%%%
%%%%%%%%%%%%%%%%%
%%%%%%%%%%%%%%%%%
\begin{theorem}
{\rm \cite{ss21.5}}
Let  $G$ is be unicyclic graph  with $\ell$ leaves, $\lambda$ exterior major vertices and $\rho$ branch-active vertices.
If $\hat{\rho}=max\{2-\rho,0\}$, then  
$$\ell - \lambda + \hat{\rho} \le \	dim(G ) \le  \ell - \lambda + \hat{\rho} +1.$$
\end{theorem}
%%%%
%\begin{proof} 
%to be done. Rather easy.
%\end{proof}
%%%%%%%%%%%%%%%%
%%%%%%%%%%%%%%%%
%%%%%%%%%%%%%%%%



In the previous section, it was talked about the use of doubly locating sets for obtaining results in metric dimension. In the case of unicyclic graphs, the relationship is given by the next result.

%%%%%%%%%%%%%%%%%%%%%
%%%%%%%%%%%%%%%%%%%%%
\begin{lemma}
\label{lem:dr4md}
Let  $G$ be  a proper  unicyclic graph being $C_g$ its unique cycle. 
If $S$ is a doubly locating set of $V(C_g)$, then $S$ resolves $V(C_g)$ together with the vertices of all threads in $G$.
\end{lemma}
%%%
\begin{proof}
Let $v,v'\in S$. 
Assume that there exists one thread beginning in a certain vertex $u_i\in V(C_g)$. Let $\{x_0,x_1,...,x_r\}$ the vertices of the thread in consecutive order where $x_0=u_i$. 
Clearly, $r(x_k|S)=r(u_i|S)+(k,k,\ldots,k)$ and so, two vertices of the same thread cannot have the same representation and thus, they are resolved by $S$. 

However, it is possible that $r(u_j|S)=r(x_k|S)$ for $u_j\in V(C_g)$ and $x_k$ in the thread. In this case $r(u_j|S)=r(x_k|S)=r(u_i|S)+(k,k,\ldots,k)$ and that means that $d(v,u_j)-d(v,u_i)=d(v',u_j)-d(v',u_i)=k$ contradicting the fact that $S$ is a doubly locating set for the vertices of the cycle. 

Finally, suppose that there are at least two threads $\{x_0,x_1,...,x_r\}$ and $\{y_0,y_1,...,y_s\}$ where $u_i=x_0$ and $u_j=y_0$ and such that $r(x_k|S)=r(y_l|S)$ (w.l.o.g we can assume that $k\geq l$). 
Thus, $r(u_i|S)+(k,k,\ldots,k)=r(x|S)=r(y_l|S)=r(u_j|S)+(l,l,\ldots,l)$ and this implies that 
$d(v,u_i)-d(v,u_j)=d(v',u_i)-d(v',u_j)=k-l$ and again $S$ would not doubly resolve $u_i$ and $u_j$. So, $S$ resolves the vertices of the cycle together with the vertices of any thread.
\end{proof}
%%%%%%%%%%%%%%%%%%%%%
%%%%%%%%%%%%%%%%%%%%%




In the rest of the section, we divide the study into two parts depending on the parity of the unique cycle. 

%%%%%%%%%%%%%%%%%%%%%%%%%%%%%%%%%%%%%%%%%%%%%%%%%%%%%%%%%%%%%%%%%%%%%%%%%%%%%%%%%%%%%%%%%%%%%%%%%%
%%%%%%%%%%%%%%%%%%%%%%%%%%%%%%%%%%%%%%%%%%%%%%%%%%%%%%%%%%%%%%%%%%%%%%%%%%%%%%%%%%%%%%%%%%%%%%%%%%
%%%%%%%%%%%%%%%%%%%%%%%%%%%%%%%%%%%%%%%%%%%%%%%%%%%%%%%%%%%%%%%%%%%%%%%%%%%%%%%%%%%%%%%%%%%%%%%%%%
\subsection{Odd unicyclic graphs}\label{dim.odduni}

Let us start with the easiest case, although a complete explicit characterization seems to be  very complicated and obscure. 
The cases  in which $\rho(G)=0,1$ have been solved with the following two results.


%%%%%%%%%%%%%%%%%%%
%%%%%%%%%%%%%%%%%%%
\begin{lemma} \label{lem:rho=0.odd}
Let  $G$ be  an odd proper unicyclic graph, being $C_g$ its unique cycle.
If $\rho(G)=0$, then $\dim(G)=2$.
\end{lemma}
%%%%
\begin{proof}
Let $v$ and $v'$ be a pair of antipodal vertices of $C_g$. 
By Theorem~\ref{dmd.cycles}, they doubly resolve all the vertices of the cycle $C_g$, and hence, according to Lemma~\ref{lem:dr4md}, since $\rho(G)=0$, they  resolve all the vertices of $G$.
\end{proof}
%%%%%%%%%%%%%%%%%%%
%%%%%%%%%%%%%%%%%%%




%%%%%%%%%%%%%%%%%%%%%
%%%%%%%%%%%%%%%%%%%%%
\begin{lemma} \label{lem:rho=1.odd}
Let  $G$ be  an odd proper unicyclic graph, being $C_g$ its unique cycle. 
If $\rho(G)=1$, then $\dim(G)=\ell(G)-\lambda(G)+1$.
\end{lemma}
%%%
\begin{proof}
Under the condition of the hypothesis, there is exactly  one branch-active vertex $v$ in the cycle $C_g$, being $T_v$  its branching tree. 

Given a metric basis $S$ of $G$ and according to Lemma~\ref{lem:cejo00}, all the terminal vertices except one of each exterior major vertex with positive terminal degree of $T_v$ should belong to $S$. However, since $v$ is a cut-point of $G$, not all the vertices of $S$ lie in $T_v$ otherwise $S$ will not resolve two vertices of $C_g$ at the same distance from $v$. 
Thus, a lower bound for a metric dimension is $\dim(G)\geq \ell(G)-\lambda(G)+1$.

Now, let $S$ be a set with $\ell(G)-\lambda(G)$ leaves together with a vertex $v'$ antipodal of $v$. If $u\in S$ is a leaf, then it is clear that $\{u,v'\}$ doubly resolves all the vertices of the cycle, and hence by Lemma~\ref{lem:dr4md}, $S$ resolves all the vertices of the cycle and the threads. 

Moreover, $S$ resolves all the vertices in $T_v$ except perhaps $v$ and other vertex $u'$ with the same exterior major vertex as $v$. 
But in this case, they are resolved by $v'$ since $d(v,v')<d(u',v')$. 
Finally, a vertex in $T_v$ and any other vertex in $V(G)\setminus V(T_v)$ is resolved by a leaf in $S$.

Thus, $S$ is a metric-locating set achieving the lower bound $\ell(G)-\lambda(G)+1$, i.e., $S$  is a metric basis of $G$.
\end{proof}
%%%%%%%%%%%%%%%%%%%%%
%%%%%%%%%%%%%%%%%%%%%



For the rest of the cases, i.e., when $\rho(G)\geq 2$,  we can give a partial result of the value of the metric dimension.

%%%%%%%%%%%%%%%%%%%%%
%%%%%%%%%%%%%%%%%%%%%
\begin{lemma}\label{lem:rho>=2.odd}
Let  $G$ be  an odd proper unicyclic graph, being $C_g$ its unique cycle. 
If $g\ge 3$, $\rho(G)\ge 2$ and there are, at least, two antipodal branch-active vertices, 
then $\dim(G)=\ell(G)-\lambda(G)$. Particularly, if $g=3$ and $\rho(G)\geq 2$ then $\dim(G)=\ell(G)-\lambda(G)$.
\end{lemma}
%%%
\begin{proof}
Let $S$ be a vertex subset formed by $\ell-\lambda$ leaves for each branching tree. 
By Lemma~\ref{lem:cejo00},  $\dim(G)\geq |S|$, and thus it remains to prove that $S$ is a metric-locating set. 
We first claim  that $S$ doubly resolves the vertices of $V(C_g)$.

Let $T_1,T_2$ be the branching trees whose branch-active vertices $u_1,u_2$ are antipodal, and let $v_1,v_2$ be two leaves such that $v_i\in V(T_i)\cap S$ for $i=\{1,2\}$. By Lemma~\ref{dmd.cycles}, $\{u_1,u_2\}$ doubly resolves the vertices of cycle. However for any $x\in V(C_g)$, we have that $r(x|\{v_1,v_2\})=r(x|\{u_1,u_2\})+(d(u_1,v_1),d(u_2,v_2))$, and hence $\{v_1,v_2\}$ also doubly resolves the vertices of the cycle, and so do $S$ since $\{v_1,v_2\}\subseteq S$.

By Lemma~\ref{lem:dr4md}, $S$ resolves all the vertices of the cycle and the threads of $G$. 
A similar reasoning can be used to prove that $S$ distinguishes between a vertex in a branching-tree, and a vertex either in a thread or in the cycle $C_g$.

Two vertices in the same branching-tree $T$ different from its branch-active vertex are resolved by the leaves in $V(T)\cap S$. 
A vertex in $T$ and its branch-active vertex $u$ are resolved by a leaf of other branching-tree in $S$. So $S$ distinguishes two vertices of the same branching-tree.

Finally, let us consider two vertices $x$ and $y$ in different branching-trees. 
Its braching-active vertices are joined by a shortest path in the cycle. 
If we delete the rest of the cycle and all the vertices hanging out, we build a subtree $T$ of $G$ which is also an isometric subgraph. 
By Theorem~\ref{thmc3.f1}, $x$ and $y$ are resolved in $T$ by the vertices in $V(T)\cap S$. 
However, since $T$ is an isometric subgraph, $x$ and $y$ are also resolved in $G$ by the same vertices in $S$. And this completes the proof.

For the particular case of $g=3$, note that if $\rho(G)\geq 2$ there are always two branching-active vertices that are antipodal.
\end{proof}
%%%%%%%%%%%%%%%%%%%%%
%%%%%%%%%%%%%%%%%%%%%




The metric dimension of an odd unicyclic graph $G$ with with $\ell$ leaves, $\lambda$ exterior major vertices and 
$\rho\ge2$ branch-active vertices without antipodal branch-active vertices may be either 
$\dim(G)=\ell  $ or $\dim(G)=\ell - \lambda +1 $. 

%As an example, the unicyclic graph $G$ of Figure~\ref{fig:odd.rho>=2} has $\dim(G)= \ell(G)-\lambda(G)+1$ (a metric basis is $\{v_1,v_2,w\}$). Nevertheless, if  vertex $w$ is removed, then $\dim(G-w)=\ell(G-w)-\lambda(G-w)$, being  a metric basis the pair $\{v_1,v_2\}$.


As an example, the unicyclic graph $G$ displayed in  Figure~\ref{fig:odd.rho>=2} (left) has $\dim(G)= \ell(G)-\lambda(G)$ (a metric basis is $\{v_1,v_2\}$). Nevertheless, if  $G'$ is the graph obtained from $G$ by adding  a vertex $w$ as shown in Figure~\ref{fig:odd.rho>=2} (right), then  $\dim(G')=\ell(G')-\lambda(G')+1$, being  a metric basis  the set $\{v_1,v_2,w\}$.




%%%%%%%%%%%%%%%%%%%%%
%%%%%%%%%%%%%%%%%%%%%
% Figure environment removed
%%%%%%%%%%%%%%%%%%%%%
%%%%%%%%%%%%%%%%%%%%%



As an immediate consequence of the previous 3 lemmas, the following result is obtained.

%%%%%%%%%%%%%%%%%%%%%
%%%%%%%%%%%%%%%%%%%%%
\begin{cor}\label{cor:xulo.odd}
Let  $G$ be  an odd unicyclic graph,  being $C_g$ its unique cycle,  with $\ell$ leaves, $\lambda$ exterior major vertices and $\rho$ branch-active vertices
If $g\ge 3$, then $\dim(G)=\ell - \lambda + \max\{2-\rho,0\} $ whenever any the following conditions holds.

\begin{enumerate}

\item $0 \le \rho \le 1$.


\item $\rho\ge 2$ and there are, at least, two antipodal branch-active vertices.


\item $g=3$.

\end{enumerate}
\end{cor}
%%%%%%%%%%%%%%%%%%%%%
%%%%%%%%%%%%%%%%%%%%%











%%%%%%%%%%%%%%%%%%%%%%%%%%%%%%%%%%%%%%%%%%%%%%%%%%%%%%%%%%%%%%%%%%%%%%%%%%%%%%%%%%%%%%%%%%%%%%%%%%
%%%%%%%%%%%%%%%%%%%%%%%%%%%%%%%%%%%%%%%%%%%%%%%%%%%%%%%%%%%%%%%%%%%%%%%%%%%%%%%%%%%%%%%%%%%%%%%%%%
%%%%%%%%%%%%%%%%%%%%%%%%%%%%%%%%%%%%%%%%%%%%%%%%%%%%%%%%%%%%%%%%%%%%%%%%%%%%%%%%%%%%%%%%%%%%%%%%%%
\subsection{Even unicyclic graphs}\label{dim.evenuni}

The case in which the unicyclic graph has a even cycle turns out to be much more difficult than the previous case. Hence, the results are not so satisfactory.


%%%%%%%%%%%%%%%%%%%%%
%%%%%%%%%%%%%%%%%%%%%
\begin{lemma}\label{lem:evencycle.rho=0}
Let  $G$ be  an even proper unicyclic graph, being $C_g$ its unique cycle.
If $g\ge8$, $\rho(G)=0$ and $c_2(C_g)\le1$, then $\dim(G)=3$.
\end{lemma}
%%%
\begin{proof}
On the contrary, suppose that $\{u,v\}$ is a metric basis of $G$ in the conditions of the hypothesis. 
Since $u$, $v$ or both may be a vertex in a thread or in the cycle, we will use the following notation for treating all the cases at once: If $u$ is a vertex in the interior of a thread, $u'$ will be the vertex of that thread that also belongs to the cycle, otherwise $u=u'$. Analogously, we  define $v'$.

Note that $u'$ and $v'$ cannot be antipodal in the cycle since, otherwise $\{u,v\}$ will not be resolving. 
Thus, there is a unique shortest path between $u'$ and $v'$ and $d(u',v')\leq k-1$. 
We  organize the proof by studying different distances between $u'$ and $v'$. 

Firstly, suppose that $d(u',v')=1$ (see Figure~\ref{fig:cycleeven} left). Since $g\geq 8$, we can add $w_1,w_2,w_3\in V(C_g)$ to obtain the shortest path $u-v-w_1-w_2-w_3$. The minimum number of threads is $2k-1$, so $w_1$ or $w_2$ is the beginning of a thread and if we denote $w_i'$ ($i=1,2$) as the adjacent to $w_i$ in the thread then we have that $r(w_i'|\{u,v\})=r(w_{i+1}|\{u,v\})$ and hence $\{u,v\}$ is not a locating set. 

Consider now the case $2\leq d(u',v')<k-1$ (see Figure~\ref{fig:cycleeven} center). Again since $g\geq 8$, we can take $u_1,u_2,v_1,v_2\in V(C_g)$ to construct the path $u_2-u_1-u'-\ldots -v'-v_1-v_2$. 
This path might not be a shortest path but the subpaths $u'-v_2$ and $u_2-v'$ are shortest paths and that is enough for our reasoning. 
Then, either $u_1$ or $v_1$ is the beginning of a thread. 
Without loss of generality, we can assume that it is $v_1$ and its adjacent in the thread is $v_1'$. 
Then, $r(v_1'|\{u,v\})=r(v_2|\{u,v\})$ and again $\{u,v\}$ is not a locating set.

The final case occurs when $d(u',v')=k-1$ (see Figure~\ref{fig:cycleeven} right). Let $u_1$ and $v_1$ be the adjacent vertices to $u'$ and $v'$ respectively in the cycle which are not in the shortest path $u'-v'$ (see Figure~\ref{fig:cycleeven}). Then the shortest path $u_1-v_1$ has the same length as $u'-v'$, and for every interior vertex $x$ of $u'-v'$, corresponds with $x_1$ in $u_1-v_1$ such that $d(u',x)=d(u_1,x_1)$. 
Moreover, $r(x_1|\{u,v\})=r(x|\{u,v\})+(1,1)$. 
Since there are $k-2$ internal vertices in the path $u'-v'$, at least one of them, say $x$, has a thread and its adjacent vertex in that thread $x'$ and $x_1$ has the same coordinates with respect to $\{u,v\}$. 
So again $\{u,v\}$ cannot be a locating set and $\dim(G)=3$.
\end{proof}
%%%%%%%%%%%%%%%%%%%%%
%%%%%%%%%%%%%%%%%%%%%




%%%%%%%%%%%%%%%%%%%%%
%%%%%%%%%%%%%%%%%%%%%
% Figure environment removed
%%%%%%%%%%%%%%%%%%%%%
%%%%%%%%%%%%%%%%%%%%%




The above result is the best bound we can obtain, since there are unicyclic graphs such that the number of threads is $2k-2$ and the metric dimension equals two. 
Consider for example a unicyclic graph $G$ with girth $g=2k$ with $k\geq 3$ and such that every vertex has a thread of length one except two antipodal vertices. 
In Figure~\ref{fig:example}, it is shown the case $k=7$,  where the pair $\{u,v\}$ of square vertices is a metric basis. 
The construction can be repeated for any integer $k\geq 3$.


%%%%%%%%%%%%%%%%%%%%%
%%%%%%%%%%%%%%%%%%%%%
% Figure environment removed
%%%%%%%%%%%%%%%%%%%%%
%%%%%%%%%%%%%%%%%%%%%
Now let us check the cases we left out int he previous result.

\begin{lemma}
Let $G$ be an uniciclic graph where $C_g$ is the unique cycle and $\rho(G)=0$. Then: 
\begin{enumerate}
\item If $g=4$ then $\dim(G)=2$.
\item Let $g=6$. Then $c_2(C_g)\neq 0$ if and only if $\dim(G)=2$.
\end{enumerate}
\end{lemma}
\begin{proof}
Consider first a unicyclic graph $G$ with $g=4$ and $\rho(G)=0$. Whenever you have two threads whose initial vertices are adjacent, say $u_1,u_2$, in the cycle then pick the leaves of those two threads as metric basis, say $v_1,v_2$. 
If the only two threads of $G$ have as initial vertices $u_1$ and $u_3$, two non-adjacent vertices of the cycle, then pick $\{v_1,u_2\}$ as a metric basis. 

If $G$ contains only one thread in, say $u_1$, then $\{v_1,u_2\}$ will be a metric basis. Finally, if $G$ does not have threads then $G=C_4$ and two consecutive vertices as $u_1,u_2$ are enough to resolve the graph.

Suppose now that $g=6$. If $c_2(C_g)=0$ then there exists a leaf for every vertex in the cycle. The reader can easily prove that $\dim(K_1\odot C_g)=3$ and so $\dim(G)=3$. If $c_2(G)\neq 0$ then there exists at least one vertex in the cycle without a thread. If there is only one $u_1$, pick the leaves hanging from the adjacent vertices in the cycle to $u_1$. It is not difficult to check that those pair of vertices form a metric basis. The same applies if there is a vertex in the cycle adjacent to two vertices with threads. 

Otherwise, we have at  least two adjacent vertices $u_1,u_2\in V(C_g)$ such that other adjacent to one of them, say $u_3$, has a thread. Let $v_3$ be the leaf in that thread. Again it is not difficult to prove that $\{u_1,v_3\}$ is a basis for $G$ and therefore $\dim(G)=2$.
\end{proof}

The next result is very similar to Lemma~\ref{lem:rho>=2.odd} but with the special features that the even cycle produces in the problem.
%%%%%%%%%%%%%%%%%%%%%
%%%%%%%%%%%%%%%%%%%%%
\begin{lemma}\label{lem:evencycle.rho>=3}
Let  $G$ be  an even unicyclic graph, being $C_g$ its unique cycle.
If $\rho(G)\geq 3$ and there are three branching-active vertices forming a geodesic triple, then 
$\dim(G)=\ell(G)-\lambda(G)$.
\end{lemma}
%%%
\begin{proof}
Let $S$ be a vertex subset with $\ell(G)-\lambda(G)$ leaves. 
By Lemma~\ref{lem:cejo00}, we know that any locating set must contain $S$, hence $\dim(G)\geq |S|$. 
We next show that $S$ is a locating set.

First, we claim that a subset of $S$, and consequently $S$, doubly resolves the vertices of the cycle. Let $T_1,T_2,T_3$ be the trees whose branching-active vertices $u_1,u_2,u_3$ is a geodesic triple. We construct $S'=\{v_1,v_2,v_3\}$ where $v_i\in V(T_i)\cap S$, so $S'\subseteq S$. As $\{u_1,u_2,u_3\}$ is a geodesic triple, then they doubly resolves the vertices of the cycle, by Theorem~\ref{dmd.cycles}. That means:
\begin{multline}
\forall x,y\in V(C_g), \exists i,j\in \{1,2,3\}: d(x,u_i)-d(x,u_j)\neq d(y,u_i)-d(y,u_j)\Rightarrow\\
\Rightarrow d(x,u_i)-d(x,u_j)+(d(u_i,v_i)-d(u_j,v_j))\neq d(y,u_i)-d(y,u_j)+(d(u_i,v_i)-d(u_j,v_j))\Rightarrow\\
\Rightarrow d(x,u_i)+d(u_i,v_i)-(d(x,u_j)+d(u_j,v_j))\neq d(y,u_i)+d(u_i,v_i)-(d(y,u_j)+d(u_j,v_j))\Rightarrow\\
\Rightarrow d(x,v_i)-d(x,v_j)\neq d(y,v_i)-d(y,v_j)
\end{multline}
And consequently, $S'$ also doubly resolves the vertices of the cycle. The rest of the proof is similar to the one of Lemma~\ref{lem:rho>=2.odd}.
\end{proof}
%%%%%%%%%%%%%%%%%%%%%
%%%%%%%%%%%%%%%%%%%%%




%%%%%%%%%%%%%%%%%%%%%
%%%%%%%%%%%%%%%%%%%%%
% Figure environment removed
%%%%%%%%%%%%%%%%%%%%%
%%%%%%%%%%%%%%%%%%%%%






As an immediate consequence of the previous 3 lemmas, the following result is obtained.

%%%%%%%%%%%%%%%%%%%%%
%%%%%%%%%%%%%%%%%%%%%
\begin{cor}\label{cor:xulo.even}
Let  $G$ be  an even unicyclic graph,  being $C_g$ its unique cycle,  with $\ell$ leaves, $\lambda$ exterior major vertices and $\rho$ branch-active vertices. If $g\ge 3$, then $\dim(G)=\ell - \lambda + \max\{2-\rho,0\} $ whenever any the following conditions holds.

\begin{enumerate}
\item $\rho=0, g\geq 8$ and $c_2\geq 2$.
\item $\rho=0$ and $g=4$.
\item $\rho=0, g=6$ and $c_2\neq 0$.
\item $\rho\geq 3$ with three branching-active vertices forming a geodesic triple.
\end{enumerate}
\end{cor}
%%%%%%%%%%%%%%%%%%%%%
%%%%%%%%%%%%%%%%%%%%%







%%%%%%%%%%%%%%%%%%%%%%%%%%%%%%%%%%%%%%%%%%%%%%%%%%%%%%%%%%%%%%%%%%%%%%%%%%%%%%%%%%%%%%%%%%%%%%%%%%
%%%%%%%%%%%%%%%%%%%%%%%%%%%%%%%%%%%%%%%%%%%%%%%%%%%%%%%%%%%%%%%%%%%%%%%%%%%%%%%%%%%%%%%%%%%%%%%%%%
\section{Strong metric dimension}\label{sdim}
The strong metric dimension is the most atypical one among the metric dimensions. To start with, it does not rely in coordinates to distinguish the vertices of the graph but in a certain notion of betweenness in shortest paths. Let $G=(V,E)$ be a graph, then a vertex $v\in V$ is said to \emph{strong resolve} two vertices $x,y \in V(G)$ if there exists either  a $v-x$ geodesic that contains $y$, or a $v-y$ geodesic  that contains $x$. 

%%%%%%%%%%%%%%%%%%
\begin{defi} \label{sls} 
{\rm A  set of vertices $S$ of $G$ is called \emph{strong locating} if every pair of distinct vertices $x,y \in V(G)$ is strong resolved by a vertex $v \in S$.}
\end{defi}
%%%%%%%%%%%%%%%%%%

In other words, $S$ is a strong locating set if, for every distinct vertices $x,y \in V(G)$ , there is a vertex $v \in S$, such that either 
$d_G(v,x)=d_G(v,y)+d_G(y,x)$ or $d_G(v,y)=d_G(v,x)+d_G(x,y)$.


A strong locating set of  minimum cardinality is called  a \emph{strong metric basis} of $G$. The \emph{strong metric dimension} of $G$, denoted by $\sdim(G)$,  is the  cardinality of a strong metric basis. 
This parameter was formally introduced  by A. S\H{e}bo and E. Tannier in \cite{st04}. 
To know more about this parameter see mainly  \cite{k20,op07,ryko14,st04} and also \cite{gsw14,kkcm14,kkcs12.1,kkcs12.2,kyr13,mo11,y13}.

The next result is mentioned in many papers, and mainly by this reason, we include here the proof for the sake of completeness.



%%%%%%%%%%%%%%%%%
%%%%%%%%%%%%%%%%%%
%%%%%%%%%%%%%%%%%%
\begin{theorem}
Let $S$  be a proper subset of vertices  of a graph $G=(V,E)$.
Then, the following statements are equivalent.



\begin{enumerate}

\item $S$ is a strong locating set.

\item $G$ is uniquely determined by the matrix of distances between the vertices of $S$ and the vertices of $V$.

%\item The distance matrix ${\cal D}_V$  is uniquely determined by the submatrix ${\cal D}_S$.

\end{enumerate}
\label{sdim.dmatrix}
\end{theorem}
%%%%%%%
\begin{proof}
Assume that $S$ is a strong locating set of $G$.
Take $u,v \in V \setminus S$.
Let $w \in S$ such that $d(w,v)=d(w,u)+d(u,v)$.
We distinguish cases.


{\bf Case (a)}:
If $d(w,v) \ge d(w,u)+2$, then $d(u,v) \ge 2$, i.e., $uv \not \in E)$.



{\bf Case (b)}:
If $d(w,v) = d(w,u)+1$, then $d(u,v) = 1$, i.e., $uv  \in E)$.



Conversely, suppose that $S$ is not a strong locating set of $G$.
Take $u,v \in V \setminus S$ such that for every vertex $w \in S$, 
$d(w,v) < d(w,u) + d(u,v)$ and $d(w,u) < d(w,v) + d(v,u)$.

Take $w \in S$ and $z \in  V \setminus S$.
If  $\rho$ is a $w-z$ geodesic, then either $u$ or $v$ does not belong to $\rho$, since otherwise either
$d(w,v) = d(w,u) + d(u,v)$ or  $d(w,u) = d(w,v) + d(v,u)$.
Hence, both if $uv  \in E$ and $uv \not \in E$, the matrix of distances between the vertices of $S$ and the vertices of $V$ is the same.
\end{proof}
%%%%%%%%%%%%%%%%%
%%%%%%%%%%%%%%%%%%
%%%%%%%%%%%%%%%%%%

As it was mentioned above, strong locating sets do not rely on coordinates so it would be desirable to find a characterization of some of the vertices that belong to a strong locating set. 
The next pair of results gives us precisely that characterization.




%%%%%%%%%%%%%%%%%%%%%%%%%%%%%%
%%%%%%%%%%%%%%%%%%%%%%%%%%%%%%
\begin{lemma}\label{sdimlema1}
Let $S$ be a strong locating set of a graph $G$. 
If $u,v \in V(G)$ are \emph{MMD}, then $\{u,v\}\cap S \neq \emptyset$.
\end{lemma}
%%%%%%%
\begin{proof}
On the contrary, assume $u,v\notin S$.
Let $w\in S$ such that the pair $u,v$ is strongly resolved by $w$. 
Suppose that $u$ lies  in a $w-v$ geodesic. 
Thus, there exists a neighbor of $u$ in this path that is farther away from $v$ than $u$, contradicting the fact that $u,v$ are \emph{MMD}.
\end{proof}
%%%%%%%%%%%%%%%%%%%%%%%%%%%%%%
%%%%%%%%%%%%%%%%%%%%%%%%%%%%%%


As an immediate consequence of this lemma, the following one holds.




%%%%%%%%%%%%%%%%%%%%%%%%%%%%%%
%%%%%%%%%%%%%%%%%%%%%%%%%%%%%%
\begin{lemma}\label{sdimlema2}
Let $S$ be a strong locating set of a graph $G$ with $\ell$ leaves. 
Then,
\begin{enumerate}

\item Every pair of leaves are MMD.

\item $S$ contains at least $\ell -1$ leaves.

\end{enumerate}
\end{lemma}
%%%%%%%%%%%%%%%%%%%%%%%%%%%%%%
%%%%%%%%%%%%%%%%%%%%%%%%%%%%%%


Next, the known results on strong metric dimension on cycles and trees are shown, the starting point to unicyclic graphs.



%%%%%%%%%%%%%%%%%
%%%%%%%%%%%%%%%%%%
%%%%%%%%%%%%%%%%%%
\begin{theorem}
%{\rm\cite{chmppsw07}}
Consider the cycle $C_g$.
Then, $\sdim(C_g)=\lceil \frac{g}{2} \rceil$.
Moreover,

\begin{enumerate}

\item If $g=2k+1$ is odd, then every strong metric basis consists of $k+1$ consecutive vertices of $C_g$.

\item
If $g=2k$ is even, then  a $k$-set $S$ is a  strong metric basis of $C_g$ if and only if
the set $V(G)\setminus S$ contains no  antipodal pair.

\end{enumerate}
\label{sdim.cycles}
\end{theorem}
%%%%%%%
\begin{proof}
Let $S$ be a strong locating set of $C_g$.
According to Lemma~\ref{sdimlema1}, the set $V(C_g)\setminus S$ contains no  antipodal pair, which means that the cardinality of $V(G)\setminus S$ is at most $\lfloor \frac{g}{2} \rfloor$.
Hence, $\sdim(C_g) = g -\lfloor \frac{g}{2} \rfloor = \lceil \frac{g}{2} \rceil$.

Both items {\rm (1)} and {\rm (2)}, are immediately derived from Lemma~\ref{sdimlema1}, by noticing that 
for every pair of non-antipodal vertices $u,v \in V(C_g)\setminus S$, $d(u,v)<\lfloor \frac{g}{2}\rfloor$.
\end{proof}
%%%%%%%%%%%%%%%%%
%%%%%%%%%%%%%%%%%%
%%%%%%%%%%%%%%%%%%




%%%%%%%%%%%%%%%%%
%%%%%%%%%%%%%%%%%
%%%%%%%%%%%%%%%%%
\begin{theorem}{\rm \cite{st04}}~
Let $T$ be a tree with $\ell$ leaves.
Then, any set consisting of all the leaves except one is a minimum strong locating set of $T$.
Hence, $\sdim(T)=\ell-1$.
\end{theorem}
\label{sdim.trees}
%%%%%%%
\begin{proof}
According to Lemma~\ref{sdimlema1}, for any pair of vertices mutually maximally distant, at least one of them must belong to a given  strong locating code $S$. 
In a tree, the vertices which are  mutually maximally distant are the leaves, so $S$ must contain all the leaves except one. 

Conversely, let $S$ be a set containing all the leaves of $T$ except one. 
It is clear that the shortest path between any pair of vertices $u,v$  can be extended until both of its extremes are leaves. 
Hence, at least one of them is in $S$ and strongly resolves the pair $u,v$.
\end{proof}
%%%%%%%%%%%%%%%%%
%%%%%%%%%%%%%%%%%
%%%%%%%%%%%%%%%%%







%%%%%%%%%%%%%%%%%%%%%%%%
% Figure environment removed
%%%%%%%%%%%%%%%%%%%%%%%%




To compute the strong metric dimension, it is necessary, in most cases, to use the strong resolving graph introduced by O. R. Oellermann and J. Peters-Fransen in~\cite{op07}.
The \emph{strong resolving graph} $G^{'}_{SR}$ of a connected graph $G$ is the graph with vertex set $V(G_{SR})=V(G)$ where two vertices are adjacent if they are mutually maximally distant in $G$.
Another definition, given in~\cite{ryko14}, reduces the vertex set to be $V(G_{SR})=\partial(G)$. 
Both graphs, $G^{'}_{SR}$ and $G_{SR}$, only differ that the former may contain isolated vertices which do not appear in the later (see Figure \ref{fig:sdim3}). 
Although we will stick formally with the second definition, this does not affect to the proof of any of the results given.


%%%%%%%%%%%%%%%%%
%%%%%%%%%%%%%%%%%
%%%%%%%%%%%%%%%%%
\begin{theorem}{\rm \cite{op07,ryko14}}
Let $G$ be a connected graph. 
Then,
$$ \sdim(G)=|\partial(G)|-\alpha(G_{SR}).$$
\label{sdim.partalpha}
\end{theorem}
%%%%%%%%%%%%%%%%%
%%%%%%%%%%%%%%%%%
%%%%%%%%%%%%%%%%%



Observe that, $[P_n]_{SR}=K_2$, $[C_n]_{SR}=C_n$ and $[T_{\ell}]_{SR}=K_{\ell}$, where $T_{\ell}$ denotes any tree with $\ell$ leaves.
Notice also that $\alpha(K_r)=1$ and  $\alpha(C_n)=\left \lfloor \frac{g}{2} \right \rfloor$.
Hence,  as a direct consequence of the previous theorem, it is derived that $\sdim(P_n)=1$, 
$\sdim(C_n)=\left \lceil \frac{n}{2} \right \rceil$ and $\sdim(T_{\ell})=\ell-1$.


The so-called \emph{closed necklace} $\widehat{G}$  associated to a unicyclic graph $G$ of girth $g$, is the unicyclic graph where, for every vertex $i \in [g]$, the branching tree of $i$ is replaced by the star $K_{1,\ell_i}$, where $\ell_i$ is the number of leaves of $T_i$, other than $i$.
It is a rutinary exercise to check, first, that $\partial(\widehat{G})=\partial(G)$, and second, that 
$\widehat{G}_{SR}=G_{SR}$ (see Figure \ref{fig:sdim3} for an example).

Let $G=(V,E)$ be a closed necklace of order $n$, girth $g$ with $\ell$ leaves.
Clearly, if $C_g$ is this cycle and $c_2(C_g)=h$, then $|\partial(G)|=\ell+h$.
Moreover, the vertex set $V(G_{SR})=\partial(G)$ of its strong resolving graph $G_{SR}$,
can be partitioned into two sets $U$ and $W$, 
in such a way  that $G_{SR}[U]=K_{\ell}$, and $G_{SR}[W]$ is a graph of order $h$ and maximum degree at most 2.


%%%%%%%%%%%%
%%%%%%%%%%%%
\begin{remark}
If $g$ is odd, then  $G_{SR}$ is a Hamiltonian graph  and $G_{SR}[W]$ is a disjoint union of paths (see Figure \ref{fig:sdim5}).
Hence, $\displaystyle \left \lceil \frac{h}{2} \right \rceil \le \alpha(G_{SR}) \le h+1$.
\label{sdimremark1}
\end{remark}
%%%%%%%%%%%%
%%%%%%%%%%%%



%%%%%%%%%%%%
%%%%%%%%%%%%
\begin{remark}
If  $g$ is even, then $G_{SR}[W]=rK_2+sK_1$, where $r$ is the number of antipodal pairs of $C_2(C_g)$, $t$ is the number of  antipodal pairs of $C_3(C_g)$ and $2s=g-2r-2t$ 
(see Figure \ref{fig:sdim4}).
Hence, $h=2r+s$ and 
$\displaystyle \left \lceil \frac{h}{2} \right \rceil = r + \left  \lceil \frac{s}{2} \right \rceil  \le r+s  \le  \alpha(G_{SR}) \le r+s+1 $.
\label{sdimremark2}
\end{remark}
%%%%%%%%%%%%
%%%%%%%%%%%%




%%%%%%%%%%%%%%%%%%%%%%%%
% Figure environment removed
%%%%%%%%%%%%%%%%%%%%%%%%


%%%%%%%%%%%%%%%%%%%%%%%%
% Figure environment removed
%%%%%%%%%%%%%%%%%%%%%%%%


The next result provides the range of the value of the strong metric dimension for unicyclic graphs.


%%%%%%%%%%%%%%%%%
%%%%%%%%%%%%%%%%%
%%%%%%%%%%%%%%%%%
\begin{theorem}{\rm \cite{k20}}\label{thm:k20}
Let  $G$ be  a proper  unicyclic graph of girth $g$,  being $C_g$ its unique cycle and  having $\ell$ leaves. 
If $c_2(C_g)=h$,  then 

$$
\max \left \{ \left \lceil \frac{g}{2} \right \rceil,\ell-1\right \}
\leq 
\sdim(G)
\leq 
\ell + \left \lfloor \frac{h}{2} \right \rfloor.$$
%where $\ell$ is the number of leaves of $G$, and $c_2(C_g)$ is the set of vertices of  $C_g$ having degree two.
\end{theorem}
\label{sdim.bounds}
%%%%%%%
\begin{proof}
If $S$ is an strong locating set of $G$, then for any pair $i,j\in [g] = V(C_g)$ of antipodal vertices of $C_g$, 
$(T_i\cup T_j)\cap S\neq \emptyset$.
Hence, from  Lemma \ref{sdimlema1} and Lemma \ref{sdimlema2}, 
the inequality $\max \left \{ \left \lceil \frac{g}{2} \right \rceil,\ell-1\right \}  \leq  \sdim(G)$ is immediately derived.

According to Remarks \ref{sdimremark1} and \ref{sdimremark2}, $\alpha(G_{SR})\ge \alpha(P_h) = \left \lceil \frac{h}{2} \right \rceil$.
Hence, 
$$\sdim(G)=|\partial(G)|-\alpha(G_{SR}) \le \ell + h - \left  \lceil \frac{h}{2} \right \rceil = 
\ell + 
\left  \lfloor \frac{h}{2} \right \rfloor.$$
\end{proof}
%%%%%%%%%%%%%%%%%
%%%%%%%%%%%%%%%%%
%%%%%%%%%%%%%%%%%




Now, we address the problem not only of proving that the  bounds displayed in Theorem~\ref{thm:k20} are tight but also of characterizing the families of graphs achieving each of them.


%%%%%%%%%%%%%%%%%
%%%%%%%%%%%%%%%%%
%%%%%%%%%%%%%%%%%
\begin{prop}\label{prop.st1}
Let $G$ be a proper unicyclic graph of girth $g$,   being $C_g$ its unique cycle and  having $\ell$ leaves. 
If $c_2(C_g)=h$,  then 
$\sdim(G)=\ell-1$ if and only if either $g=3$ and $h=0$ or $g\ge4$ and any of the following conditions hold:

\begin{enumerate}

\item $ 0 \le h \le 1$,

\item $g=2k$ even, $\displaystyle  2 \le h \le k-1$ and $C_2(C_g)$ contains no pair of antipodal vertices.

\item $g=2k+1$ odd, $\displaystyle  2 \le h \le k-1$, $C_2(C_g)$ contains no pair of antipodal vertices and $C_3(C_g)$ contains at least one antipodal triple.


\end{enumerate}
\label{sdim.ell-1}
\end{prop}
%%%%%
\begin{proof} 
According to Theorem  \ref{sdim.partalpha}, 
$\sdim(G) = |\partial(G)|-\alpha(G_{SR}) = \ell + h - \alpha(G_{SR})$.
Hence, $\sdim(G)=\ell-1$ if  and only if $\alpha(G_{SR})=h+1$.

If $g=3$ then, in all cases, $\alpha(G_{SR})=1$.
Hence, $\sdim(G)=\ell+h-1=\ell-1$ if and only if $h=0$.

Suppose that $g\ge4$.
We distinguish cases:

{\bf Case (a)}:
If $h=0$, then $G_{SR}\cong K_{\ell}$.
Thus, $\alpha(G_{SR})=1=h+1$.


{\bf Case (b)}:
If $h=1$, then $G_{SR}$  is a graph of order $\ell+1$ containing the clique $K_{\ell}$ and minimum degree either 1 or 2, depending of the parity of $g$.
Thus, $\alpha(G_{SR})=2=h+1$.

{\bf Case (c)}:
Suppose that $h\ge2$ and  $g=2k$ is even.
According to Remark \ref{sdimremark2}, $h=2r+s$ and $r+s \le \alpha (G_{SR}) \le  r+s +1$.
Hence, $r=0$, i.e., $C_2(C_g)$ contains no pair of antipodal vertices, if and only if both $h=s$ and 
$s \le \alpha (G_{SR}) \le  s +1$.
Moreover, $\alpha (G_{SR}) =  s $ if and only if $h=k$, which means that 
$\alpha (G_{SR}) =  s +1 =h+1$ if and only if $r=0$ and $h \le k-1$.



{\bf Case (d)}:
Suppose that $h\ge2$ and  $g=2k+1$ is odd.
According to Remark \ref{sdimremark1}, $\left \lceil \frac{h}{2} \right \rceil \le \alpha(G_{SR}) \le h+1$.
In addition, $\alpha (G_{SR}) = h+1$ if and only if both $[W]G_{SR}=hK_1$ and $C_3(C_g)$ contains at least one antipodal triple, which also includes the fact that $h \le k-1$.
\end{proof}
%%%%%%%%%%%%%%%%%
%%%%%%%%%%%%%%%%%
%%%%%%%%%%%%%%%%%



%%%%%%%%%%%%%%%%%
%%%%%%%%%%%%%%%%%
%%%%%%%%%%%%%%%%%
\begin{prop}\label{prop.st2}
Let $G$ be a proper unicyclic graph of grith $g$,     being $C_g$ its unique cycle and  having $\ell$ leaves.
If $c_2(C_g)=h$, then 
$\displaystyle \sdim(G)=\ell + \left \lfloor \frac{h}{2} \right \rfloor$
if and only if any of the following conditions hold:

\begin{enumerate}

\item  $g$ is even and $h=g-1$.

\item $g$ is odd and $g-2 \le h \le g-1$. 

\end{enumerate}
\label{sdim.ell+h/2}
\end{prop}
%%%%%
\begin{proof} 
According to Theorem \ref{sdim.bounds} and Remarks \ref{sdimremark1} and \ref{sdimremark2}, 
$\sdim(G)=\ell + \left \lfloor \frac{h}{2} \right \rfloor$ if and only if 
$\alpha (G_{SR}) = \left \lceil \frac{h}{2} \right \rceil$.
We distinguish cases:

{\bf Case (a)}:
Suppose that $g$ is even.
According to Remark \ref{sdimremark2}, $h=2r+s$ and $r+s \le \alpha (G_{SR}) \le  r+s +1$.
Assume that $h=g-1$. 
Then, $2r+s=2r+2s+2t-1$, which means that $t=0$, $s=1$ and $h=2r+1$.
So, $\alpha (G_{SR}) = r+1= \left \lceil  \frac{h}{2} \right \rceil $.

Conversely, suppose that $\alpha (G_{SR}) = \left \lceil  \frac{h}{2} \right \rceil  = 
r + \left \lceil  \frac{s}{2} \right \rceil $.
Thus, $r+s \le r + \left \lceil  \frac{s}{2} \right \rceil \le  r+s +1$, i.e., 
$s \le  \left \lceil  \frac{s}{2} \right \rceil \le s +1$, which means that 
$s =  \left \lceil  \frac{s}{2} \right \rceil$
since $\left \lceil  \frac{s}{2} \right \rceil < s +1$, for every positive integer $s$.
After noticing that $t=0$ and $g=2r+2s+2t$,
we distinguish cases:

{\bf Case (a.1)}:
If $s=0$, then $G=C_g$, a contradiction.

{\bf Case (a.2)}:
If $s=1$, then $h=2r+1$ and $g=2r+2$, i.e., $h=g-1$.

{\bf Case (b)}:
Suppose that $g$ is odd.
If $g-2 \le h \le g-1$,  then according to Remark \ref{sdimremark1}, in all possible cases,  
$\alpha (G_{SR})= \left \lceil  \frac{h}{2} \right \rceil$.

Conversely, suppose both that $\alpha (G_{SR}) = \left \lceil  \frac{h}{2} \right \rceil $ and $h\le g-3$.
If $\displaystyle G[W]=\sum_{i=1}^{\rho} P_{h_i}$, then
$$\displaystyle \sum_{i=1}^{\rho} \alpha(P_{h_i})   \le   \alpha (G_{SR}) \le \sum_{i=1}^{\rho} \alpha(P_{h_i}) +1.$$
Hence, $\alpha (G_{SR}) = \left \lceil  \frac{h}{2} \right \rceil $ if and only if 
$\displaystyle \alpha (G_{SR}) =  \sum_{i=1}^{\rho} \alpha(P_{h_i}) = \left \lceil  \frac{h}{2} \right \rceil = \alpha (P_{h})$.
After noticing that  $\displaystyle \alpha (P_{h}) =  \sum_{i=1}^{\rho} \alpha(P_{h_i})$ if and only if there is at most an index $j \in [\rho]$ such that $P_j$  is an odd path, we distinguish cases.

{\bf Case (c)}:
There is a unique index $j \in [\rho]$ such that $P_j$  is an odd path.
We distinguish cases.

{\bf Case (c.1)}:
If $\rho=2$, then we can suppose w.l.o.g. that $P_{h_1}$ is the even path and that between $P_{h_1}$ and $P_{h_2}$ there are in $G_{SR}$ at least two vertices (see Figure \ref{fig:sdim6}, left).
Certainly, it is possible to select a maximum independent set in $P_1$  not containing its last vertex. 
Hence, 
$\displaystyle \alpha (G_{SR}) =  \sum_{i=1}^{\rho} \alpha(P_{h_i}) + 1 > \left \lceil  \frac{h}{2} \right \rceil$, a contradiction.


{\bf Case (c.2)}:
Assume that $\rho\ge 3$.
Take $P_{h_1}$ and $P_{h_2}$ and suppose w.l.o.g. that they are both even paths and  consecutive in $G_{SR}$ (see Figure \ref{fig:sdim6}, right).
Clearly, it is possible to select a maximum independent set in $P_{h_1}$ (resp. in $P_{h_2}$) not containing its last vertex (resp. its first vertex). 
Hence, 
$\displaystyle \alpha (G_{SR}) =  \sum_{i=1}^{\rho} \alpha(P_{h_i}) + 1 > \left \lceil  \frac{h}{2} \right \rceil$, a contradiction.


%%%%%%%%%%%%%%%%%%%%%%%%
% Figure environment removed
%%%%%%%%%%%%%%%%%%%%%%%%


{\bf Case (d)}:
For every index $j \in [\rho]$,  $P_{h_1}$  is an even path.
Take $P_{h_1}$ and $P_{h_2}$ and suppose w.l.o.g. that they are consecutive in $G_{SR}$ (see Figure \ref{fig:sdim6}, right).
Clearly, it is possible to select a maximum independent set in $P_{h_1}$ (resp. in $P_{h_2}$) not containing its last vertex (resp. its first vertex). 
Hence, 
$\displaystyle \alpha (G_{SR}) =  \sum_{i=1}^{\rho} \alpha(P_{h_i}) + 1 > \left \lceil  \frac{h}{2} \right \rceil$, a contradiction.
\end{proof}
%%%%%%%%%%%%%%%%%
%%%%%%%%%%%%%%%%%
%%%%%%%%%%%%%%%%%



As for the problem of characterizing the  family of proper unicyclic  graphs $G$ of girth $g$ with at most $\lceil \frac{g}{2} \rceil$  leaves such that $\sdim(G)=\lceil \frac{g}{2} \rceil$, it was solved when $g$ is even in \cite{k20},  while the odd case still remains open.



%%%%%%%%%%%%%%%%%
%%%%%%%%%%%%%%%%%
%%%%%%%%%%%%%%%%%
\begin{prop}
Let $G$ be an even  proper unicyclic graph of grith $g$,  being $C_g$ its unique cycle and  having $\ell$ leaves.
Then, $\sdim(G)=\frac{g}{2}$ if and only if the following conditions hold:

\begin{enumerate}

\item  $\rho(G)=0$

\item There is at most one pair of antipodal vertices in $C_g$, each one of degree at least 3.

\end{enumerate}
\label{sdim.g/2}
\end{prop}
%%%%%%%%%%%%%%%%%
%%%%%%%%%%%%%%%%%
%%%%%%%%%%%%%%%%%



%We finalize this section by presenting  two results, the first one 



Next, we present a result that clearly shows how easy is to compute the strong metric dimension of an even unicyclic graph.


%%%%%%%%%%%%%%%%%
%%%%%%%%%%%%%%%%%
%%%%%%%%%%%%%%%%%
\begin{theorem}\label{th:sd}
Let $G$ be a unicyclic graph of even grith $g=2k$, being $C_g$ its unique cycle and having $\ell$ leaves.
If $C_2(C_g)$ contains $r$ antipodal pairs and $C_3(C_g)$ contains $t$ antipodal pairs, then

$$
\sdim(G)=
\left \{ \begin{array}{cl}
\ell + r -1 & {\rm if}\, t \ge 1, \\
\ell + r & {\rm if}\, t=0.  \\
\end{array}\right . 
$$
\label{sdim.even}
\end{theorem}
%%%%%
\begin{proof} 
Let $s$ such that $2s=g-2r-2t$. 
According to Theorem \ref{sdim.partalpha},
 
$$\sdim(G)=|\partial(G)|-\alpha(G_{SR}) = \ell + c_2(C_g) - \alpha(G_{SR}) = \ell +2r + s - \alpha(G_{SR}).$$

As shown in Figure \ref{fig:sdim4} (when $g=8$, $r=2$, $s=1$ and $t=1$), it is easy to check that  
$\alpha(G_{SR})=r+s$ (resp., $\alpha(G_{SR}=r+s+1$) if $t=0$ (resp., $t\ge1$).
\\ 
Hence, $\sdim(G)= \ell +2r + s - r-s = \ell + r$ (resp., $\sdim(G)= \ell +2r + s - r-s -1 = \ell + r -1$)  if $t=0$ (resp., $t\ge1$).
\end{proof}
%%%%%%%%%%%%%%%%%
%%%%%%%%%%%%%%%%%
%%%%%%%%%%%%%%%%%



We finalize this section by showing a result that  allows us to affirm that the problem of computing the strong metric dimension of an odd unicyclic  graph is  certainly quite more complicated than in the  even case.



%%%%%%%%%%%%%%%%%
%%%%%%%%%%%%%%%%%
%%%%%%%%%%%%%%%%%
\begin{prop}
Let $G$ be a unicyclic graph of odd grith $5 \le g=2k+1$ with $\ell$ leaves. If $C_g$ is its cycle and   $2 \le c_2(C_g)=h \le 2k-2$, then

$$
\max\{\ell-1, \ell +h -k\} \le \sdim(G) \le \ell + \Big\lfloor \frac{h}{2}  \Big\rfloor -1 .
$$

Moreover, both bounds are tight.
\label{sdim.odd3}
\end{prop}
%%%%%
\begin{proof} 
The inequality $\ell-1\le \sdim(G)$ and the upper bound are a direct consequence of Theorem \ref{sdim.bounds} and Proposition \ref{sdim.ell+h/2}, respectively.
Hence, to prove the lower bound  it is enough to show that if $ k-1 \le h \le 2k-2$ , then $\alpha(G_{SR}) \le k$, since $\sdim(G) = \ell + h - \alpha(G_{SR})$.

According to Remark \ref{sdimremark1}, suppose that $G_{SR}[W]$ is the disjoint union of $\rho$ paths, in such a way that $\rho=\mu+\eta$, there are $\mu$ odd paths and $\eta$ even paths.
If $\displaystyle G_{SR}[W] = \sum_{\kappa=1}^{\mu} P_{i_{\kappa}} + \sum_{\kappa=1}^{\eta} P_{j_{\kappa}}$, then  %$\alpha(P_{j_{\kappa}}),
$\displaystyle \sum_{\kappa=1}^{\mu} \alpha(P_{i_{\kappa}}) + \sum_{\kappa=1}^{\eta} \alpha(P_{j_{\kappa}})
\le
\alpha(G_{SR}) 
\le 
\sum_{\kappa=1}^{\mu} \alpha(P_{i_{\kappa}}) + \sum_{\kappa=1}^{\eta} \alpha(P_{j_{\kappa}}) +1$.
If for every $\kappa \in [\mu]$ (resp., $\kappa \in [\eta]$), 
$i_{\kappa}=2r_{\kappa}+1$ (resp., $j_{\kappa}=2s_{\kappa}$ ), then 

$$\displaystyle \sum_{\kappa=1}^{\mu} r_{\kappa} + \sum_{\kappa=1}^{\eta} s_{\kappa} + \mu
\le
\alpha(G_{SR}) 
\le 
\sum_{\kappa=1}^{\mu} r_{\kappa} + \sum_{\kappa=1}^{\eta} s_{\kappa} + \mu +1.$$

and

$$
\displaystyle
g
\ge
\sum_{\kappa=1}^{\mu} 2r_{\kappa} + \mu + \sum_{\kappa=1}^{\eta} 2s_{\kappa} + \rho
= 2\left[ 
\sum_{\kappa=1}^{\mu} r_{\kappa} + \sum_{\kappa=1}^{\eta} 2s_{\kappa} + \mu 
\right] + \eta.
$$

We distinguish cases.


{\bf Case (1)}:
$\eta \ge 2$.
In this case,  
$k \ge \sum_{\kappa=1}^{\mu} r_{\kappa} + \sum_{\kappa=1}^{\eta} 2s_{\kappa} + \mu +1  \ge \alpha(G_{SR})$, since $g$ is odd.


{\bf Case (2)}:
$g-h \ge \rho +2$.
In this case,
$$\displaystyle g \ge h + \rho +2 = 
h + \mu + \eta +2 =
2\left[ 
\sum_{\kappa=1}^{\mu} r_{\kappa} + \sum_{\kappa=1}^{\eta} s_{\kappa} + \mu +1
\right] + \eta.$$
Hence,
$k \ge \sum_{\kappa=1}^{\mu} r_{\kappa} + \sum_{\kappa=1}^{\eta} s_{\kappa} + \mu +1 \ge \alpha(G_{SR})$, since $g$ is odd.


{\bf Case (3)}:
$\rho \le g-h \le \rho +1$ and $0 \le \eta \le 1$.
In this case, 
$\displaystyle \alpha(G_{SR})= \sum_{\kappa=1}^{\mu} r_{\kappa} + \sum_{\kappa=1}^{\eta} s_{\kappa} +\mu$.

Hence, $k \ge \sum_{\kappa=1}^{\mu} r_{\kappa} + \sum_{\kappa=1}^{\eta} s_{\kappa} + \mu  = \alpha(G_{SR})$, since $g$ is odd.
%We distinguish subcases.
%
%{\bf Case (3.1)}:
%$\eta =0$.
%
%
%{\bf Case (3.2)}:
%$\eta =1$.


%\noindent
To show the tightness of the upper bound, consider the case , $h=g-3=2k-2$ and $\rho=\eta=1$, and notice that
$\alpha(G_{SR}=\alpha(P_{2k-2}+1=k$.
Hence, $\sdim(G)=\ell + h -\alpha(G_{SR}) = \ell+  2k-2 - k = \ell + k -2 = \ell + \Big\lfloor \frac{h}{2}  \Big\rfloor -1$.

The tightness of the lower bound when  $0 \le h \le k-1$ is a direct consequence of Proposition \ref{sdim.ell-1}.

If $h=k$, consider the case $\displaystyle G_{SR}[W] = kK_{1}$ and notice that $\alpha(G_{SR})=k$.
If $k+1 \le h \le 2k-2$, then consider the case
$\displaystyle G_{SR}[W] = (g-h-1)K_{1} + P_{2h-2k}$.
Then, $\alpha(G_{SR})= g-h-1 + \alpha(P_{2h-2k}) = 2k+1-h-1+h-k=k$.
Thus if $k\le h \le 2k-2$, then $\sdim(G)=\ell + h -\alpha(G_{SR})= \ell + h -k$.
\end{proof}
%%%%%%%%%%%%%%%%%
%%%%%%%%%%%%%%%%%
%%%%%%%%%%%%%%%%%







%%%%%%%%%%%%%%%%%%%%%%%%%%%%%%%%%%%%%%%%%%%%%%%%%%%%%%%%%%%%%%%%%%%%%%%%%%%%%%%%%%%%%%%%%%%%%%%%%%
%%%%%%%%%%%%%%%%%%%%%%%%%%%%%%%%%%%%%%%%%%%%%%%%%%%%%%%%%%%%%%%%%%%%%%%%%%%%%%%%%%%%%%%%%%%%%%%%%%
\section{Dominating metric dimension}\label{ddim}

After the introduction of metric dimension, it seems natural to study when a metric-locating set verifies additional properties, and among all the vertex-set properties, one of the most important is domination. 
Thus, R. C. Brigham et al. in \cite{bcdz03}, and also and independently M. A. Henning and O. R.  Oellermann in \cite{ho04}, defined the following parameter.

\begin{defi}
{\rm A set of vertices of a graph $G$ is called \emph{metric-locating-dominating} (an \emph{MLD-set} for short) if it is both locating and dominating.}
\end{defi}


An MLD-set of  minimum cardinality is called  an \emph{MLD-basis} of $G$. 
The \emph{dominating metric dimension} of $G$, denoted by $\ddim(G)$,  is the  cardinality of a MLD-basis. 
To know more about this parameter see mainly  \cite{bcdz03,ho04,chmpp13} and also \cite{ghm18,hmp14,zab22}.


As usual, the dominating metric dimension of trees is well-known.


%%%%%%%%%%%%%%%%%
%%%%%%%%%%%%%%%%%
%%%%%%%%%%%%%%%%%
\begin{theorem} 
{\rm\cite{ho04}}           
Let $T$ be a tree with $\ell(T)$ leaves and $s(T)$ support vertices.\\
Then,  $\ddim(T)=\gamma(T)+ \ell(T) - s(T)$.
\label{thmho04.1}
\end{theorem}
%%%%%%%%%%%%%%%%%
%%%%%%%%%%%%%%%%%
%%%%%%%%%%%%%%%%%



%%%%%%%%%%%%%%%%%
%%%%%%%%%%%%%%%%%
\begin{cor} 
{\rm\cite{ho04}}           
Let $T$ be a tree.
Then,  $\ddim(T)=\gamma(T)$ if and only if $T$ contains no strong support vertex.
\label{corho04.1}
\end{cor}
%%%%%%%%%%%%%%%%%
%%%%%%%%%%%%%%%%%

In particular, $\ddim(P_n)=\gamma(P_n)=\lceil \frac{n}{3}  \rceil$. The same reason as in trees can be applied in general graphs to obtain a lower bound.




%%%%%%%%%%%%%%%%%
%%%%%%%%%%%%%%%%%
\begin{prop}    
Let $G$ be a graph with $\ell(G)$ leaves and $s(G)$ support vertices.\\
Then,  $\ddim(G) \ge \gamma(G)+ \ell(G) - s(G)$.
\label{lmddim3}
\end{prop}
%%%%%
\begin{proof}
Consider the set $S(G)$ of strong support vertices of $G$.
Let $D$ be a minimum MLD-set of $G$, containing as few leaves as possible.
This means  that, for every vertex $v \in S(G)$, $D$ contains all except one leaf $v'$ adjacent to $v$, as well as vertex $v$.
So, the set $\displaystyle D'= D -\Big( \cup_{v\in S(G)} (L_v-v')  \Big)$
is a dominating set of $G$.
Hence,
$$\gamma(G) \le |D'| \le  |D| - \sum_{v\in S(G)} (|L_v|-1) = \ddim(G) -\ell(G) +s(G),$$
and the inequality $\ddim(G) \ge \gamma(G)+ \ell(G) - s(G)$ follows.
\end{proof}
%%%%%%%%%%%%%%%%%
%%%%%%%%%%%%%%%%%

As an immediate consequence of this result the following one holds.



%%%%%%%%%%%%%%%%%
%%%%%%%%%%%%%%%%%
\begin{lemma}    
For every graph  $G$, $\gamma(G) \le \ddim(G)$.
Moreover, if  $\ddim(G) = \gamma(G)$, then $G$ has no strong support vertex.
\label{lmddim2}
\end{lemma}
%%%%%
%\begin{proof}
%Assume on the contrary that $G$ contains a strong support vertex $v$.
%Take an MLD-set $D$ of $G$ of minimum cardinality, containing as few leaves as possible.
%Certainly, $D$ contains all except one leaf $v'$ adjacent to $v$, as well as vertex $v$.
%Deleting from $D$ of the leaves adjacent to vertex $v$, we obtain  a dominating set of $G$ whose cardinality is less than $|D|$.
%Hence, $\gamma(T) < |D| = \ddim(G)$, a contradiction.
%\end{proof}
%%%%%%%%%%%%%%%%%
%%%%%%%%%%%%%%%%%



As a consequence, there exists several situations in which a dominating set is also an MLD-set.


%%%%%%%%%%%%%%%%%
%%%%%%%%%%%%%%%%%
\begin{lemma}    
Let $G$ be a unicyclic graph of girth $g$, being $C_g$ its unique cycle, with no strong support vertex.
If $D$ is a dominating set of $G$, then $D$ is an MLD-set of $G$ whenever any of the following conditions holds.

\begin{enumerate}


\item $g \not\in \{3,4,6\}$.


\item $g=3$ and $|D\cap V(C_3)|\ge2$.


\item $g \in \{4,6\}$ and $|D\cap V(C_g)|\ge3$.


\end{enumerate}

\label{lmddim1}
\end{lemma}
%%%%%
\begin{proof}
Let $u,v \in V(G)-D$.
Suppose that $N(u)\cap D = N(v) \cap D$.
Notice that there is a unique vertex $z$ such that $N(u)\cap D = N(v) \cap D =\{z\}$, 
since either $g\neq 4$ or $g=4$ and $|D\cap V(C_4)|\ge3$.
As $z$ is not an strong support vertex, then we can suppose w.l.o.g. that $deg(v)\ge 2$.
Take a vertex $w \in N(v)-z$.
Notice that $w \not\in D$, as $N(v) \cap D =\{z\}$.
Let $y\in N(w) \cap D$.
Observe that $y\neq z$, as either $g\neq 3$ or $g=3$ and $|D\cap V(C_3)|\ge2$.
Certainly, $d(v,y)=2$ and it is also clear that $d(u,y)>2$, unless $g=6$ (see Figure \ref{uvzwyx} (a)).
If $g=6$ and $d(u,y)=2$, take the  vertex $x\in N(u)\cap N(y)$ and notice first that necessarily it belongs to $D$ since $|D\cap V(C_6)|\ge3$, and second that $d(u,x)=1$ and $d(v,x)=3$ (see Figure \ref{uvzwyx} (b)).
%{\color{red} acabar esto bien (el caso g=4 aun no sale)}
%
%Moreover, , as either $g \not\in \{4,6\}$ or $g \in \{4,6\}$ and $|D\cap V(C_g)|\ge3$.
%Hence, $D$ is an MLD-set of $G$.
%%Suppose that $d(u,y)=2$.
%%Let $x \in N(u)\cap N(y)$.
%%Observe that $x\neq w$ since either $g\neq 4$ or $g=4$ and $|D\cap V(C_4)|\ge3$.
%%Notice also that $
\end{proof}
%%%%%%%%%%%%%%%%%
%%%%%%%%%%%%%%%%%



%%%%%%%%%%%%%%%%%%%%%%%%%%%%%%%%%%
% Figure environment removed
%%%%%%%%%%%%%%%%%%%%%%%%%%%%%%%%%%



From the preceding lemmas, the following result  immediately follows.


%%%%%%%%%%%%%%%%%
%%%%%%%%%%%%%%%%%
\begin{cor}    
Let $G$ be a unicyclic graph of girth $g$.
If $g \not\in \{3,4,6\}$,
then  $\ddim(G)=\gamma(G)$ if and only if $G$ contains no strong support vertex.
\label{corddim1}
\end{cor}
%%%%%%%%%%%%%%%%%
%%%%%%%%%%%%%%%%%





In particular, if $g \not\in \{3,4,6\}$, then  $\ddim(C_g)= \gamma(C_g)=\lceil \frac{g}{3} \rceil$. The previous results lead us to obtain the same characterization as trees for unicyclic graphs except for the cases $g=3,4,6$.






%%%%%%%%%%%%%%%%%
%%%%%%%%%%%%%%%%%
%%%%%%%%%%%%%%%%%
\begin{theorem}    
Let $G$ be a unicyclic graph of girth $g$ with $\ell(G)$ leaves and $s(G)$ support vertices.\\
If $g \not\in \{3,4,6\}$, then  $\ddim(G)=\gamma(G)+ \ell(G) - s(G)$.
\label{thmddim1}
\end{theorem}
%%%%%%
\begin{proof}
According to Proposition \ref{lmddim3}, $\ddim(G) \ge \gamma(G)+ \ell(G) - s(G)$.
To prove the other inequality,
consider the set  $S(G)$ of strong support vertices of $G$.
If $|S(G)|$=0, then $\ell(G) = s(G)$, and according to Corollary \ref{corddim1}, the equality holds.
Suppose thus that $|S(G)|\ge1$.
For each vertex $v \in S(G)$, take a vertex $v' \in L_v = N(v)\cap \ell(G)$.
Consider the unicyclic graph $G'$ obtained from $G$ by deleting, for every vertex $v \in S(G)$, the set $L_v -v'$.
Hence, $G'$ has no strong support vertex.
Let $D'$ be a minimum dominating set of $G'$ containing all the support vertices of $G'$.
According to Lemma \ref{lmddim1}, $D'$ is an MLD-set of $G'$.
Thus, 
$\displaystyle  D' \cup \Big( \cup_{v\in S(G)} (L_v-v') \Big)$ 
is an MLD-set of $G$, and so
$$\displaystyle \ddim(G) \le |D'| + \sum_{v\in S(G)} (|L_v|-1) = \gamma(G) +\ell(G) - s(G).$$
\end{proof}
%%%%%%%%%%%%%%%%%
%%%%%%%%%%%%%%%%%
%%%%%%%%%%%%%%%%%


The cases $g=3,4$ and $6$ are studied in the following two results.



%%%%%%%%%%%%%%%%%
%%%%%%%%%%%%%%%%%
\begin{lemma}    
Let $G$ be a unicyclic graph of girth $g=4$, being $C_4$ its unique cycle, with no strong support vertex.
If $D$ is a minimum dominating set of $G$ and $|D\cap V(C_g)|=1$, then $\gamma(G) \le \ddim(G) \le \gamma(G)+1$.
\label{lmddim4}
\end{lemma}
%%%
\begin{proof}
Let $u,v \in V(G)-D$.
Suppose that $N(u)\cap D = N(v) \cap D$.
Notice that there is a unique vertex $z$ such that $N(u)\cap D = N(v) \cap D =\{z\}$,  since $|D\cap V(C_4)|=1$.
As $z$ is not an strong support vertex, then we can suppose w.l.o.g. that $deg(v)\ge 2$.
Take a vertex $w \in N(v)-z$ and notice that $w \not\in D$, as $N(v) \cap D =\{z\}$.
Let $y\in N(w) \cap D$.
Observe that $y\neq z$, as $g\neq 3$.
Certainly, $d(v,y)=2$ 
Suppose that $d(u,y)=2$, as otherwise we are done.
This means that $w\in N(u)$.
We distinguish cases.

{\bf Case (a)}:
If $deg(v)\ge3$, take $x \in N(v)\setminus \{w,z\}$.
Let $\eta \in D \cap N(x)$ and notice that $d(v,\eta)=2$ and $d(u,\eta)=4$ (see Figure \ref{abcdeyeta} (a)).

{\bf Case (b)}:
If $deg(u)=deg(v)=2$, consider the set $D'=D+v$.
In order to prove that $D'$  is an MLD-set of $G$, take a pair of vertices $a,b \in V(G)-D'$.
If $N(a)\cap D = N(b) \cap D$, then there is a unique vertex $c$ such that $N(a)\cap D = N(b) \cap D =\{c\}$.
As $c$ is not an strong support vertex, then we can suppose w.l.o.g. that $deg(b)\ge 2$.
Take a vertex $d \in N(b)-c$ and notice that $d\not\in D$, as $N(b) \cap D =\{c\}$.
Let $e\in N(d) \cap D$ and check that $e\neq e$.
Certainly, $d(b,e)=2$.
Suppose that $d(a,e)=2$, as otherwise we are done.
This means that $e\in N(c)$ and also that $c=z$, $e=v$, $b=u$ and $d=w$.
Hence, $d(b,y)=d(u,y)=2$ and $d(a,y)=4$ (see Figure \ref{abcdeyeta} (b)).
\end{proof}
%%%%%%%%%%%%%%%%%
%%%%%%%%%%%%%%%%%
%%%%%%%%%%%%%%%%%



%%%%%%%%%%%%%%%%%%%%%%%%%%%%%%%%%%
% Figure environment removed
%%%%%%%%%%%%%%%%%%%%%%%%%%%%%%%%%%


By considering the previous Lemma along with  items (2) and (3) of Lemma \ref{lmddim1}, and reasoning in  a similar way as in the proof Theorem \ref{thmddim1}, the following result holds.


%%%%%%%%%%%%%%%%%
%%%%%%%%%%%%%%%%%
%%%%%%%%%%%%%%%%%
\begin{prop}    
Let $G$ be a unicyclic graph of girth $g$ with $\ell(T)$ leaves and $s(T)$ support vertices.
If $g \in \{3,4,6\}$, then  $\gamma(G)+ \ell(T) - s(T) \le \ddim(G) \le \gamma(G)+ \ell(T) - s(T)+1$.
\label{thmddim2}
\end{prop}
%%%%
%\begin{proof}
%According to Proposition \ref{lmddim3}, $\gamma(G) + \ell(G) - s(G) \le  \ddim(G)$.
%To prove the second inequality, consider .... {\Large \color{red}\fbox{\color{blue} to be done}}
%\end{proof}
%%%%%%%%%%%%%%%%%, 
%%%%%%%%%%%%%%%%%
%%%%%%%%%%%%%%%%%



As for tightness of the  above upper bound, in some cases it is needed to add  an additional vertex to have an MLD-set starting from a minimum dominating set and the set of leaves hanging on the strong support vertices (see Figure \ref{g346}, for some examples).

%%%%%%%%%%%%%%%%%%%%%%%%%%%%%%%%%%
% Figure environment removed
%%%%%%%%%%%%%%%%%%%%%%%%%%%%%%%%%%















%%%%%%%%%%%%%%%%%%%%%%%%%%%%%%%%%%%%%%%%%%%%%%%%%%%%%%%%%%%%%%%%%%%%%%%%%%%%%%%%%%%%%%%%%%%%%%%%%%
%%%%%%%%%%%%%%%%%%%%%%%%%%%%%%%%%%%%%%%%%%%%%%%%%%%%%%%%%%%%%%%%%%%%%%%%%%%%%%%%%%%%%%%%%%%%%%%%%%
%%%%%%%%%%%%%%%%%%%%%%%%%%%%%%%%%%%%%%%%%%%%%%%%%%%%%%%%%%%%%%%%%%%%%%%%%%%%%%%%%%%%%%%%%%%%%%%%%%
%%%%%%%%%%%%%%%%%%%%%%%%%%%%%%%%%%%%%%%%%%%%%%%%%%%%%%%%%%%%%%%%%%%%%%%%%%%%%%%%%%%%%%%%%%%%%%%%%%
\section{Other metric locating parameters}\label{omlp}

In this section, we briefly revise other metric locating parameters that deserve attention but they have not included previously.



%%%%%%%%%%%%%%%%%%%%%%%%%%%%%%%%%%%%%%%%%%%%%%%%%%%%%%%%%%%%%%%%%%%%%%%%%%%%%%%%%%%%%%%%%%%%%%%%%%
%%%%%%%%%%%%%%%%%%%%%%%%%%%%%%%%%%%%%%%%%%%%%%%%%%%%%%%%%%%%%%%%%%%%%%%%%%%%%%%%%%%%%%%%%%%%%%%%%%
\subsection{Fault-tolerant metric dimension}\label{dim2}

 
A  set of vertices $S$ of a graph  $G$ is called \emph{$2$-locating} if  every pair of distinct  vertices  $x,y \in V (G)$ is resolved by at least $2$ elements of $S$.

A $2$-locating set of  minimum cardinality is called  a \emph{$2$-metric basis} of $G$. 
The \emph{$2$-metric dimension} of $G$, denoted by $\dim_2(G)$,  is the  cardinality of a $2$-metric basis. 



This parameter was formally introduced by C. Hernando et al. in \cite{hmsw08}. 
In this work, 2-metric locating sets were called \emph{fault-tolerant sets}, and the 2-metric dimension was  named the \emph{fault-tolerant metric dimension} of $G$.
To know more about this parameter see also  \cite{cjs10,jscs09}.

It is a routine exercise to check that  $\dim_2(P_n)=2$ and $\dim_2(C_n)=3$.


%%%%%%%%%%%%%%%%%
%%%%%%%%%%%%%%%%%%
%%%%%%%%%%%%%%%%%%
\begin{theorem}
{\rm\cite{hmsw08}}
Let  $T$ be  a tree  having $\ell_s$ strong leaves. 
Then, $\dim_2(T)  = \ell_s$.
%\label{thmc3.f1}
\end{theorem}
%%%%%%%%%%%%%%%%%
%%%%%%%%%%%%%%%%%%
%%%%%%%%%%%%%%%%%%


%%%%%%%%%%%%%%%%%%%%%%%%%%%%%%%%%%%%%%%%%%%%%%%%%%%%%%%%%%%%%%%%%%%%%%%%%%%%%%%%%%%%%%%%%%%%%%%%%%
%%%%%%%%%%%%%%%%%%%%%%%%%%%%%%%%%%%%%%%%%%%%%%%%%%%%%%%%%%%%%%%%%%%%%%%%%%%%%%%%%%%%%%%%%%%%%%%%%%
\subsection{$k$-metric dimension}\label{dimk}


Let $k,n$ be a pair of vertices such that $2 \le k \le n-1$.
A  set of vertices $S$ of a graph  $G$ of order $n$ is called \emph{$k$-locating} if  every pair of distinct  vertices  $x,y \in V (G)$ is resolved by at least $k$ elements of $S$.

A $k$-locating set of  minimum cardinality is called  a \emph{$k$-metric basis} of $G$. 
The \emph{$k$-metric dimension} of $G$, denoted by $\dim_k(G)$,  is the  cardinality of a $k$-metric basis. 

This parameter was formally introduced by A. Estrada-Moreno in \cite{ery15}. 
The case $k=2$ was previously introduced by C. Hernando et al. in \cite{hmsw08} (see  previous subsection). 
To know more about this parameter see also \cite{eyr14,eyr16,eyr16.2,mst22}.


A  graph  $G$ is called 
\emph{$k$-metric dimensional}, if $k$ is the largest integer such
that $G$ contains a $k$-locating set.









%%%%%%%%%%%%%%%%%
%%%%%%%%%%%%%%%%%
%%%%%%%%%%%%%%%%%
\begin{prop}
{\rm \cite{e21, ery15}}
Let  $k,n\ge3$ be an integers such that $k+1 \le n$.
Then, 
\begin{itemize}

\item $P_n$ is $(n-1)$-dimensional.

\item $\dim_k(P_n)=k+1$.

\item If $n$ is odd (resp., even) , then $C_n$ is $(n-1)$-dimensional (resp., $(n-2)$-dimensional).

\item $\dim_k(C_n)=
\left \{ 
\begin{array}{cl}
k+2 & {\rm if}\, n  \, {\rm is}\, {\rm even}\, {\rm and}\, \frac{n}{2} \le k \le n-2, \\
k+1 & {\rm otherwise}.  \\
\end{array}
\right .$.

\end{itemize}

\label{propc4.f1}
\end{prop}
%%%%%%%%%%%%%%%%%
%%%%%%%%%%%%%%%%%
%%%%%%%%%%%%%%%%%



%%%%%%%%%%%%%%%%%
%%%%%%%%%%%%%%%%%
%%%%%%%%%%%%%%%%%
\begin{theorem}
{\rm \cite{ery15}}
Let  $T$ be a $k$-metric dimensional of order $n\ge3$.
Then,

\begin{itemize}

\item $2\leq k\leq n-1$.

\item $k=2$ if and only if $G$ has  twin vertices.

\end{itemize}

\end{theorem}
%%%%%%%%%%%%%%%%%
%%%%%%%%%%%%%%%%%
%%%%%%%%%%%%%%%%%


Let ${\cal M}(T)$ denote the set of strong exterior major vertices of a tree $T$.
For every vertex $w \in  {\cal M}(T)$,
let $Ter(w)$ and $ter(w)$ denote the set and  number of terminal vertices of $w$, respectively.

If $Ter(w)=\{u_1,\ldots,u_h\}$, then 
$\displaystyle l(w)=\min_{i\in[h]}d(u_i,w)$
and
$\displaystyle \zeta(w)=\min_{\stackrel{i,j\in[h]}{i\neq  j}}[d(u_i,w)+d(w,w_j)]$.

Finally, $\displaystyle  \zeta(T)=\min_{w \in {\cal M}(T)} \zeta(w)$ and
$I_r(w)=
\left \{ 
\begin{array}{cl}

[ter(w)-1][r-l(w)]+l(w) & {\rm if}\,   l(w) \le \, \lfloor \frac{r}{2} \rfloor\\

[ter(w)-1]\lceil \frac{r}{2} \rceil +  \lfloor \frac{r}{2} \rfloor & {\rm otherwise}.  \\

\end{array}
\right .
$.



%%%%%%%%%%%%%%%%%
%%%%%%%%%%%%%%%%%
%%%%%%%%%%%%%%%%%
\begin{theorem}
{\rm \cite{ery15}}
Let  $T$ be a $k$-metric dimensional tree, other than a path.
Then, 
\begin{itemize}

\item $k=\zeta(T)$.

\item For every $r \in [\zeta(T)]$,
$\ds \dim_r (T) = \sum_{w \in {{\cal M}(T)}} I_r(w) $.

\end{itemize}

\end{theorem}
%%%%%%%%%%%%%%%%%
%%%%%%%%%%%%%%%%%
%%%%%%%%%%%%%%%%%




In \cite{e21}, some partial results for unicyclic graphs are presented. 
For example, it was given a closed formula for the $k$-metric dimension of a unicyclic graph $G$ of girth $g$, whenever $\rho(G)=0$ and $c_2(C_g)=g-1$, where $C_g$ is the cycle of $G$.




%%%%%%%%%%%%%%%%%%%%%%%%%%%%%%%%%%%%%%%%%%%%%%%%%%%%%%%%%%%%%%%%%%%%%%%%%%%%%%%%%%%%%%%%%%%%%%%%%%
%%%%%%%%%%%%%%%%%%%%%%%%%%%%%%%%%%%%%%%%%%%%%%%%%%%%%%%%%%%%%%%%%%%%%%%%%%%%%%%%%%%%%%%%%%%%%%%%%%
%%%%%%%%%%%%%%%%%%%%%%%%%%%%%%%%%%%%%%%%%%%%%%%%%%%%%%%%%%%%%%%%%%%%%%%%%%%%%%%%%%%%%%%%%%%%%%%%%%
%%%%%%%%%%%%%%%%%%%%%%%%%%%%%%%%%%%%%%%%%%%%%%%%%%%%%%%%%%%%%%%%%%%%%%%%%%%%%%%%%%%%%%%%%%%%%%%%%%
\subsection{Edge metric dimension}\label{edim}

Let $G=(V,E)$ be a graph.
A vertex $v\in V$ is said to resolve two edges $e,f \in E$ if $d_G(v,e)\neq d_G(v,f)$.
A  set of vertices $S$ of a graph $G$ is called \emph{edge locating} if  every pair of distinct  edges  $e,f\in E$ is resolved by a vertex of $S$.


An edge locating set of  minimum cardinality is called  a \emph{edge metric basis} of $G$. 
The \emph{edge metric dimension} of $G$, denoted by $\edim(G)$,  is the  cardinality of an edge metric basis. 
This parameter was formally introduced by A. Kelenc et al. in \cite{kty18}. 
To know more about this parameter see also \cite{azszas22,g20,kmmsy21,py20,ss22.2,ss22.1}.


It is a routine exercise to check that $\edim(P_n)=1$ and $\edim(C_g)=2$.



%%%%%%%%%%%%%%%%%
%%%%%%%%%%%%%%%%%%
%%%%%%%%%%%%%%%%%%
\begin{theorem}
{\rm\cite{kty18}}
Let  $T$ be  a tree  having $\ell(T)$ leaves and $\lambda(T)\geq 1$ exterior major vertices. 
Then, $$\edim(T)  = \ell(T)-\lambda(T).$$
%\label{thmc3.f1}
\end{theorem}
%%%%%%%%%%%%%%%%%
%%%%%%%%%%%%%%%%%%
%%%%%%%%%%%%%%%%%%




%%%%%%%%%%%%%%%%%
%%%%%%%%%%%%%%%%%
%%%%%%%%%%%%%%%%%
\begin{theorem}
{\rm \cite{ss22}}
If  $G$ is a unicyclic graph  with $\ell$ leaves, $\lambda$ exterior major vertices and $\rho$ branch-active vertices,
then  $$\ell - \lambda + \max\{2-\rho,0\} \le \edim(G ) \le  \ell - \lambda + \max\{2-\rho,0\} +1.$$
\end{theorem}
%%%%
%\begin{proof} 
%to be done. Rather easy.
%\end{proof}
%%%%%%%%%%%%%%%%
%%%%%%%%%%%%%%%%
%%%%%%%%%%%%%%%%




%%%%%%%%%%%%%%%%%
%%%%%%%%%%%%%%%%%
%%%%%%%%%%%%%%%%%
\begin{theorem}
{\rm \cite{wyc22}}
Let $G$ be a unicyclic graph.
If  $e\in E(G)$ and $G-e$ is connected, then $\edim(G) \le \edim(G-e) +1$.
\end{theorem}
%%%%
%\begin{proof} 
%to be done. Rather easy.
%\end{proof}
%%%%%%%%%%%%%%%%
%%%%%%%%%%%%%%%%
%%%%%%%%%%%%%%%%





%%%%%%%%%%%%%%%%%
%%%%%%%%%%%%%%%%%
%%%%%%%%%%%%%%%%%
\begin{theorem}
{\rm \cite{ss22}}
If  $G$ is a unicyclic graph.
Then,   $$|\dim(G)-\edim(G)| \le 1.$$
\end{theorem}
%%%%
%\begin{proof} 
%to be done. Rather easy.
%\end{proof}
%%%%%%%%%%%%%%%%
%%%%%%%%%%%%%%%%
%%%%%%%%%%%%%%%%

Moreover, a  characterization of the sets of unicyclic graphs  attaining each of the three possible equalities  was also given in \cite{ss22}.
In particular, it was proved that if $G$ is an odd (resp., even) unicyclic graph, then $\dim(G)\le\edim(G)$ (resp.,  $\dim(G)\ge\edim(G)$).






%%%%%%%%%%%%%%%%%%%%%%%%%%%%%%%%%%%%%%%%%%%%%%%%%%%%%%%%%%%%%%%%%%%%%%%%%%%%%%%%%%%%%%%%%%%%%%%%%%
%%%%%%%%%%%%%%%%%%%%%%%%%%%%%%%%%%%%%%%%%%%%%%%%%%%%%%%%%%%%%%%%%%%%%%%%%%%%%%%%%%%%%%%%%%%%%%%%%%
\subsection{Mixed metric dimension}\label{mdim}


A  set of vertices $S$ of a graph $G$ is called \emph{mixed locating} if  every pair of distinct  elements (vertices or edges)   is resolved by a vertex  of $S$.

A mixed locating set of  minimum cardinality is called  a \emph{mixed metric basis} of $G$. 
The \emph{mixed metric dimension} of $G$, denoted by $\mdim(G)$,  is the  cardinality of mixed  metric basis. 
This parameter was formally introduced by A. Kelenc et al. in \cite{kkty17}. 
To know more about this parameter see  \cite{kkty17,ss21,ss21.2}.



It is straightforward to check that, for every integer $n\ge3$,  $\mdim(P_n)=2$ and $\mdim(C_n)=3$.



%%%%%%%%%%%%%%%%%%%%%%%%
%%%%%%%%%%%%%%%%%%%%%%%%
%%%%%%%%%%%%%%%%%%%%%%%%
\begin{theorem}{\rm \cite{kkty17}}
Let $G$ be a graph of order $n\ge2$.
Then, $\mdim(G)=2$ if and only if $G$ is the path $P_n$.
\end{theorem}
%%%%%%%%%%%%%%%%%%%%%%%%
%%%%%%%%%%%%%%%%%%%%%%%%
%%%%%%%%%%%%%%%%%%%%%%%%


%%%%%%%%%%%%%%%%%%%%%%%%
%%%%%%%%%%%%%%%%%%%%%%%%
%%%%%%%%%%%%%%%%%%%%%%%%
\begin{theorem}{\rm \cite{kkty17}}
For any tree $T$ with $\ell$ leaves, $\mdim(T)=\ell$.
\end{theorem}
%%%%%%%%%%%%%%%%%%%%%%%%
%%%%%%%%%%%%%%%%%%%%%%%%
%%%%%%%%%%%%%%%%%%%%%%%%


%%%%%%%%%%%%%%%%%%%%%%%%
%%%%%%%%%%%%%%%%%%%%%%%%
%%%%%%%%%%%%%%%%%%%%%%%%
\begin{theorem}{\rm \cite{ss21.2}}
Let $G$ be a unicyclic graph whose cycle $C_g$ has $t$ root vertices.
Then,
$$\mdim(G)=\ell(G)+\max\{3-t,0\}+\epsilon$$
where $\epsilon=1$ if $t\ge3$ and  there is no geodesic triple of root vertices on $C_g$, while $\epsilon=0$ otherwise.
\label{thm1mdim}
\end{theorem}
%%%%%%%%%%%%%%%%%%%%%%%%
%%%%%%%%%%%%%%%%%%%%%%%%
%%%%%%%%%%%%%%%%%%%%%%%%





%%%%%%%%%%%%%%%%%%%%%%%%%%%%%%%%%%%%%%%%%%%%%%%%%%%%%%%%%%%%%%%%%%%%%%%%%%%%%%%%%%%%%%%%%%%%%%%%%%
%%%%%%%%%%%%%%%%%%%%%%%%%%%%%%%%%%%%%%%%%%%%%%%%%%%%%%%%%%%%%%%%%%%%%%%%%%%%%%%%%%%%%%%%%%%%%%%%%%
\subsection{Local metric dimension}\label{ldim}


%%%%%%%%%%%%%%%%%%
\begin{def} \label{dls} 
A  set of vertices $S$ of a graph $G$ is called \emph{local locating} if,  for every pair of adjacent vertices $x,y \in V(G)$, there is a vertex $v \in S$, such that 
$d_G(x,v) \neq d_G(y,x)$.
\end{def}
%%%%%%%%%%%%%%%%%%


In other words,
$S$ is a local locating set of $G$ if, for every pair of adjacent  vertices  $x,y \in E (G)$, $r(x|S) \neq r(y|S)$.

A local locating set of  minimum cardinality is called  a \emph{local metric basis} of $G$. 
The \emph{local metric dimension} of $G$, denoted by $\ldim(G)$,  is the  cardinality of a local metric basis. 
This parameter was formally introduced by F. Okamoto et al. in \cite{opz10}.
To know more about this parameter see also  \cite{berr19,rbg16,rgb15}.





%%%%%%%%%%%%%%%%%
%%%%%%%%%%%%%%%%%
%%%%%%%%%%%%%%%%%
\begin{theorem}
{\rm \cite{opz10}}
 $\ldim(G) =1 $ if and only if $G$ is bipartite.
 \label{th:ldimbipartite}
\end{theorem}
%%%%%%%%%%%%%%%%%
%%%%%%%%%%%%%%%%%
%%%%%%%%%%%%%%%%%





%%%%%%%%%%%%%%%%%
%%%%%%%%%%%%%%%%%
%%%%%%%%%%%%%%%%%
\begin{theorem}{\rm \cite{frsb22}}
Let $G$ be a unicyclic graph, being $C_g$ its unique cycle. Then,

$$ \ldim(G)=\left\{\begin{array}{cr}
1 & g\textrm{ is even}\\
2 & g\textrm{ is odd}
\end{array}\right.$$
\label{thmldim1}
\end{theorem}
%%%%
\begin{proof}
Clearly, if $g$ is even then $G$ is bipartite, and Theorem~\ref{th:ldimbipartite} applies.

On the other hand, let $G$ be a unicyclic graph with odd cycle $C_g$, and let $V(C_g)=\{v_1,v_2,\ldots ,v_g\}$. We will prove that $S=\{v_1,v_g\}$ is a local metric basis of $G$. 

It is a well-known result that any two vertices in an odd cycle is a metric basis for that cycle. 
So, $S$ is also a local locating set for it. 
Take now a maximal subtree $T_i$ of $G$ rooted in the vertex $v_i$ of the cycle. 
Any vertex $u\in V(T_i)$ has $r(v_i|S)+d(v_i,u)$ as representation with respect to $S$. 
Since two adjacent vertices in $T_i$ cannot be at the same distance of $v_i$, $S$ resolves all the vertices in $T_i$. Similarly, two adjacent vertices in the same thread $H_j$ cannot have the same coordinates with respect to $S$. Hence $\ldim(G)\leq 2$.

By Theorem~\ref{th:ldimbipartite}, $2\leq \ldim(G)$ since $G$ is not bipartite, thus the result holds.
\end{proof}
%%%%%%%%%%%%%%%%%
%%%%%%%%%%%%%%%%%
%%%%%%%%%%%%%%%%%







%%%%%%%%%%%%%%%%%%%%%%%%%%%%%%%%%%%%%%%%%%%%%%%%%%%%%%%%%%%%%%%%%%%%%%%%%%%%%%%%%%%%%%%%%%%%%%%%%%
%%%%%%%%%%%%%%%%%%%%%%%%%%%%%%%%%%%%%%%%%%%%%%%%%%%%%%%%%%%%%%%%%%%%%%%%%%%%%%%%%%%%%%%%%%%%%%%%%%
%%%%%%%%%%%%%%%%%%%%%%%%%%%%%%%%%%%%%%%%%%%%%%%%%%%%%%%%%%%%%%%%%%%%%%%%%%%%%%%%%%%%%%%%%%%%%%%%%%
%%%%%%%%%%%%%%%%%%%%%%%%%%%%%%%%%%%%%%%%%%%%%%%%%%%%%%%%%%%%%%%%%%%%%%%%%%%%%%%%%%%%%%%%%%%%%%%%%%
\section{Conclusions and Further work}\label{cfw}


The primary purpose of this paper has been to survey the state of art of the main  metric locating parameters for the family of pseudotrees, i.e., for paths, cycles, trees and unicyclic graphs.




During the process of writing this survey, the authors noticed that regarding paths, cycles and trees, everything is known about the nine parameters analyzed here (see Table \ref{megatable1}). However, a complete characterization have not been obtained for unicyclic graphs in all the metric locations that we have surveyed, so we set out to contribute with a number of new results.
As a consequence, we have been able to make significant progress in four of the nine parameters considered in this work.

%%%%%%%%%%%%%%%%%%%%%%%%%
%%%%%%%%%%%%%%%%%%%%%%%%%
%%%%%%%%%%%%%%%%%%%%%%%%%
%%%%%%%%%%%%%%%%%%%%%%%%%
\begin{table}[h]
  \begin{center}
   
$
\begin{array}{c|cccc|ccccc}

\hline

G & {\dmd}  & {\dim} & {\sdim} & {\ddim} & {\dim_2} & {\dim_k} & {\edim} & {\mdim}   & {\ldim} 
\\


\hline 

P_n & 2 &  1  &  1  &  \lceil \frac{n}{3}  \rceil &  2  &  k+1   &   1  &  2   & 1 
\\ 
 &&&&&  &&&& \\

C_n & 2,3 &  2 &  \lceil \frac{n}{2}  \rceil & \lceil \frac{n}{3}  \rceil    &  3  &  \begin{array}{c} k+1,\\k+2 \end{array}   &   2  &  3  &  1,2  
\\ 
&&&&&  &&&& \\


T_n  & \ell &  \ell-\lambda & \ell-1 &  \gamma+\ell-s  &  \ell_s &  \cite{ery15}   &   \ell-\lambda  &  \ell  &  1   
\\ 

\hline



\end{array}
$

\end{center}
\caption{ \label{megatable1} Metric locating parameters of paths, cycles and trees of order $n\ge4$.}
\end{table}
%%%%%%%%%%%%%%%%%%%%%%%%%
%%%%%%%%%%%%%%%%%%%%%%%%%
%%%%%%%%%%%%%%%%%%%%%%%%%
%%%%%%%%%%%%%%%%%%%%%%%%%


To be more precise, we have completed the work for the doubly metric dimension and the dominating metric dimension and we also have contributed with significant new results for the metric dimension and the strong metric dimension.

As for the remaining five parameters, for two of them (the mixed metric dimension and the local metric dimension) the values have already be computed, while for the remaining three parameters (the edge metric dimension, the fault-tolerant metric dimension and the  $k$-metric dimension), there are plenty of work to be done to get bounds and exact values.

For further details, see Table \ref{megatable2}  and Table \ref{megatable3}, 
having in mind that  $G$ denotes a proper unicyclic graph  with girth $g$,  $\ell$ leaves, $\ell_s$ strong leaves, $s$ support vertices, $\lambda$ exterior major vertices, $\rho$ branch-active vertices, $c_2$ trivial vertices and $t$ root vertices. Moreover, $\hat{\rho}=max\{2-\rho,0\}$, $\hat{t}=max\{3-t,0\}$ and $\mu(g,\ell)=\max\{\lceil \frac{g}{2} \rceil, \ell-1\} $.










%%%%%%%%%%%%%%%%%%%%%%%%%
%%%%%%%%%%%%%%%%%%%%%%%%%
%%%%%%%%%%%%%%%%%%%%%%%%%
%%%%%%%%%%%%%%%%%%%%%%%%%
\begin{table}[h]
  \begin{center}
    
$
\begin{array}{c|cc|cc}

\hline

{\rm  \, Bounds} & {\dmd}  &   {\ddim} &  {\mdim}   & {\ldim} 
\\


\hline 

{\rm Lower} & \ell &    \gamma + \ell - s &  \ell + \hat{t}   & 1 
\\ 
 &&&  & \\

{\rm Upper} & \ell +2 &    \gamma + \ell - s +1   &   \ell + \hat{t}+1  &  2  
\\ 
&&&  & \\



\hline 

{\rm Exact \, values}  & {\rm Theorem} \, \ref{dmd.unic1} &   {\rm Theorem} \, \ref{thmddim1}  &   {\rm Theorem} \,  \ref{thm1mdim}  &  {\rm Theorem} \, \ref{thmldim1}   
\\ 

\hline


\end{array}
$

 \end{center}
  \caption{\label{megatable2} Metric locating parameters of proper unicyclic graphs (no open problems).}
\end{table}
%%%%%%%%%%%%%%%%%%%%%%%%%
%%%%%%%%%%%%%%%%%%%%%%%%%
%%%%%%%%%%%%%%%%%%%%%%%%%
%%%%%%%%%%%%%%%%%%%%%%%%%




%%%%%%%%%%%%%%%%%%%%%%%%%
%%%%%%%%%%%%%%%%%%%%%%%%%
%%%%%%%%%%%%%%%%%%%%%%%%%
%%%%%%%%%%%%%%%%%%%%%%%%%
\begin{table}[h]
  \begin{center}
$
\begin{array}{c|cc|ccc}

\hline

{\rm  \, Bounds} &  {\dim} & {\sdim} &  {\dim_2}  &  {\dim_k} &  {\edim} 
\\


\hline 

{\rm Lower} &  \ell -\lambda +\hat{\rho} &  \mu(g,\ell)  &  -- &   --
&        \ell -\lambda +\hat{\rho}  
\\ 
 &&  &&& \\

{\rm Upper} &  \ell -\lambda +\hat{\rho}+1 &  \ell + \lfloor \frac{c_2}{2}  \rfloor & --  &--
&  \ell -\lambda +\hat{\rho}+1  
\\ 
&&  &&& \\

\hline


{\rm Exact \, values}  &   \begin{array}{c} Cor. \ref{cor:xulo.odd}, \ref{cor:xulo.even}\\and~\cite{ss22}\end{array} & \begin{array}{c}Prop.~\ref{prop.st1},\ref{prop.st2}\\ and~ Th.~\ref{th:sd}\end{array}  & -- & --  &    \cite{ss22}  
\\ 

\hline


\end{array}
$

 \end{center}
  \caption{\label{megatable3} Metric locating parameters of proper unicyclic graphs (with open problems).}
\end{table}
%%%%%%%%%%%%%%%%%%%%%%%%%
%%%%%%%%%%%%%%%%%%%%%%%%%
%%%%%%%%%%%%%%%%%%%%%%%%%
%%%%%%%%%%%%%%%%%%%%%%%%%


We conclude with a list  of  open problems. 



%%%%%%%%%%%%%%
{\rm \bf  Open Problem 1}:
Starting mainly both  from \cite{ss22} and from Section \ref{dim} of this paper, characterizing the family of proper odd unicyclic graphs $G$ such that $\dim(G)=\ell - \lambda + \max\{2-\rho,0\} $.
%%%%%%%%%%%%%%



%%%%%%%%%%%%%%
\vspace{.6cm}
{\rm \bf  Open Problem 2}:
Starting mainly both  from \cite{ss22} and from Section \ref{dim} of this paper, characterizing the family of proper even unicyclic graphs $G$ such that $\dim(G)=\ell - \lambda + \max\{2-\rho,0\}$.
%%%%%%%%%%%%%%



%%%%%%%%%%%%%%
\vspace{.6cm}
{\rm \bf  Open Problem 3}:
Starting mainly both  from \cite{k20} and from Section \ref{sdim} of this paper, characterizing the  family of proper odd unicyclic  graphs $G$ of girth $g$ with at most $\lceil \frac{g}{2} \rceil$  leaves such that $\sdim(G)=\lceil \frac{g}{2} \rceil$.
%%%%%%%%%%%%%%



%%%%%%%%%%%%%%
\vspace{.6cm}
{\rm \bf  Open Problem 4}:
Starting mainly both  from \cite{k20} and from Section \ref{sdim} of this paper, either obtaining a result 
similar to the one displayed in Theorem \ref{sdim.even} when $g$ is odd, or at least characterizing the families of graphs achieving the bounds shown in Proposition \ref{sdim.odd3}.
%%%%%%%%%%%%%%



%%%%%%%%%%%%%%
\vspace{.6cm}
{\rm \bf  Open Problem 5}:
Starting mainly both  from \cite{ss22} and from Section \ref{dim} of this paper,
characterizing the family of proper unicyclic graphs $G$ such that $\edim(G)=\ell - \lambda + \max\{2-\rho,0\} $, or at least obtaining similar results to those appearing in subsections \ref{dim.odduni} and \ref{dim.evenuni}.
%%%%%%%%%%%%%%




%%%%%%%%%%%%%%
\vspace{.6cm}
{\rm \bf  Open Problem 6}:
Starting mainly from \cite{e21}, \cite{ery15} and \cite{hmsw08}, obtain tight, both lower and upper bounds, of the fault-tolerant metric dimension of proper unicyclic graphs. 
%%%%%%%%%%%%%%




%%%%%%%%%%%%%%
\vspace{.6cm}
{\rm \bf  Open Problem 7}:
Starting mainly from \cite{e21} and \cite{ery15}, obtain tight, both lower and upper bounds, of the $k$-metric dimension of proper unicyclic graphs without strong support vertices, for every integer $k\ge3$. 
%%%%%%%%%%%%%%










%%%%%%%%%%%%%%%%%%%%%%%%%%%%%%%%%%%%%%%%%%%%%%%%%%%%%%%%%%%%%%%%%%%%%%%%%%%%%%%%%%%%%%%%%%%%%%%%%%
%%%%%%%%%%%%%%%%%%%%%%%%%%%%%%%%%%%%%%%%%%%%%%%%%%%%%%%%%%%%%%%%%%%%%%%%%%%%%%%%%%%%%%%%%%%%%%%%%%
%%%%%%%%%%%%%%%%%%%%%%%%%%%%%%%%%%%%%%%%%%%%%%%%%%%%%%%%%%%%%%%%%%%%%%%%%%%%%%%%%%%%%%%%%%%%%%%%%%
%%%%%%%%%%%%%%%%%%%%%%%%%%%%%%%%%%%%%%%%%%%%%%%%%%%%%%%%%%%%%%%%%%%%%%%%%%%%%%%%%%%%%%%%%%%%%%%%%%
\begin{thebibliography}{}


%
%
\bibitem{azszas22} {\sc Alrowaili, D., Zahid, Z.,; Siddique, I., Zafar, S., Ahsan, M., Sindhu, M. S.}:
On the constant edge resolvability of some unicyclic and multicyclic graphs.
{J. Math.}, Art. ID 6738129, 9 pp. (2022). DOI: 10.1155/2022/6738129



%
%
\bibitem{bcggmmp13} {\sc Bailey, R. F., Caceres, J., Garijo, D., Gonzalez, A., Marquez, A., Meagher, K.,  Puertas, M. L.}:
Resolving sets for Johnson and Kneser graph.
{European J. Combin.}, \textbf{34} (4), 736--751 (2013). DOI: 10.1016/j.ejc.2012.10.008


%
%
\bibitem{bc11} {\sc Bailey, R. F., Cameron, P. J.}:
Base size, metric dimension and other invariants of groups and graphs.
{Bull. Lond. Math. Soc.}, \textbf{43} (2), 209--242 (2011).  DOI: 10.1112/blms/bdq096


%
%
\bibitem{bcdz03} {\sc Brigham, R. C., Chartrand, G., Dutton, R. D., Zhang, P.}:
Resolving domination in graphs. 
{Math. Bohem.}, \textbf{128} (1), 25--36 (2003). DOI: 10.21136/MB.2003.133935
%{\scriptsize \fbox{\color{red}$\ddim$}} {\scriptsize \fbox{\color{blue}24c}}
%\vspace{.2cm}


%
%
\bibitem{bm11} {\sc Bailey, R. F., Meagher, K.}:
On the metric dimension of Grassmann graphs.
{Discrete Math. Theor. Comput. Sci.}, \textbf{13} (4), 97--104 (2011). 

%
%
\bibitem{berr19} {\sc Barragan-Ramirez, G. A., Estrada-Moreno, A., Ramirez-Cruz, Y., Rodriguez-Velazquez, J. A.}:
The local metric dimension of the lexicographic product of graphs.
{Bull. Malays. Math. Sci. Soc.}, \textbf{42} (5), 2481--2496 (2019). DOI: 10.1007/s40840-018-0611-3


%
%
\bibitem{bdfhmp18} {\sc Beaudou, L., Dankelmann, P., Foucaud, F., Henning, M., Mary, A., Parreau, A.}:
Bounding the order of a graph using its diameter and metric dimension: a study through tree decompositions and VC dimension.
{SIAM J. Discrete Math.}, \textbf{32} (2), 902--918 (2018).  DOI: 10.1137/16M1097833



%
%
\bibitem{bcpz03} {\sc Buczkowski, P. S., Chartrand, G., Poisson, C., Zhang, P.}:
On k-dimensional graphs and their bases. 
{Period. Math. Hungar.}, \textbf{46} (1), 9--15 (2003). DOI: 10.1023/A:1025745406160

%
%
\bibitem{chmppsw07} {\sc Caceres, J., Hernando, C.,  Mora, M.,  Pelayo, I. M.,  Puertas, M. L., Seara, C.,  Wood, D. R.}:
On the metric dimension of Cartesian products of graphs.
{SIAM J. Discrete Math.}, \textbf{21} (2), 423--441  (2007).  DOI: 10.1137/050641867



%
%
\bibitem{chmpp13} {\sc Caceres, J., Hernando, C.,  Mora, M.,  Pelayo, I. M.,  Puertas, M. L.}:
Locating–dominating codes: Bounds and extremal cardinalities.
{Appl. Math. Comput.}, \textbf{220}, 38--45  (2013). DOI: 10.1016/j.amc.2013.05.060


%
%
\bibitem{cgh08} {\sc Chappell, G. G., Gimbel, J.,  Hartman, C.}:
Bounds on the metric and partition dimensions of a graph.
{Ars Combin.}, \textbf{88}, 349--366  (2008). 





%
%
%%%%%%%%%%%%%%%%%%cejo00
\bibitem{cejo00} {\sc Chartrand, G., Eroh, L.,  Johnson, M. A.,  Oellermann, O. R.}:
Resolvability in graphs and the metric dimension of a graph.
{Discrete Appl. Math.}, \textbf{105} (1-3), 99--113  (2000).  DOI: 10.1016/S0166-218X(00)00198-0


%
%
\bibitem{clz16} {\sc Chartrand, G.,  Lesniak, L., Zhang, P.}:
Graphs and digraphs. Sixth edition.
CRC Press, Boca Raton, FL, xii+628 pp. (2016). DOI: doi.org/10.1201/b19731


%
%
\bibitem{cjs10} {\sc Chaudhry, M. A., Javaid, I., Salman, M.}:
Fault-tolerant metric and partition dimension of graphs.
{Util. Math.}, \textbf{83}, 187--199  (2010). 




%
%
\bibitem{eky17} {\sc Eroh, L., Kang, C. X.,  Yi, E.}:
A comparison between the metric dimension and zero forcing number of trees and unicyclic graphs.
{Acta Math. Sin. (Engl. Ser.)}, \textbf{33} (6), 731--747  (2017). DOI: 10.1007/s10114-017-4699-4
%{\scriptsize \fbox{\color{red}$\dim$, $ZF$}} {\scriptsize \fbox{\color{blue}4c}}
%\vspace{.2cm}


%
%
\bibitem{e21} {\sc Estrada-Moreno, A.}:
The $k$-Metric Dimension of a Unicyclic Graph.
{Mathematics}, \textbf{9} (21), 2789 14pp (2021). DOI: 10.3390/math9212789
%{\scriptsize \fbox{\color{red}$\dim_k$ }} {\scriptsize \fbox{\color{blue}0c}}
%\vspace{.2cm}


%
%
\bibitem{ery15} {\sc Estrada-Moreno, A.,  Rodriguez-Velazquez, J. A.,  A., Yero, I. G.}:
The $k$-metric dimension of a graph.
{Appl. Math. Inf. Sci.}, \textbf{9} (6), 2829--2840 (2015). DOI: 10.12785/amis/090609
%{\scriptsize \fbox{\color{red}$\dim_k$}} {\scriptsize \fbox{\color{blue}19c}}
%\vspace{.2cm}


%
%
\bibitem{eyr14} {\sc Estrada-Moreno, A., Yero, I. G., Rodriguez-Velazquez, J. A.}:
$k$-metric resolvability in graphs.
{Electron. Notes Discrete Math.}, \textbf{46}, 121--128 (2014). DOI: 10.12785/amis/090609


%
%
\bibitem{eyr16} {\sc Estrada-Moreno, A., Yero, I. G., Rodriguez-Velazquez, J. A.}:
The $k$-metric dimension of the lexicographic product of graphs.
{Discrete Math.}, \textbf{339} (7), 1924--1934 (2016). DOI: 10.1016/j.disc.2015.12.024


%
%
\bibitem{eyr16.2} {\sc Estrada-Moreno, A., Yero, I. G., Rodriguez-Velazquez, J. A.}:
The $k$-metric dimension of Corona product graphs.
{Bull. Malays. Math. Sci. Soc.}, \textbf{39} (1), S135--S156 (2016). DOI: 10.1007/s40840-015-0282-2



%
%
\bibitem{g20} {\sc Geneson, J.}:
Metric dimension and pattern avoidance in graphs.
{Discrete Appl. Math.}, \textbf{284}, 1--7  (2020). DOI: 10.1016/j.amc.2018.03.053


%
%
\bibitem{frsb22} {\sc Fitriani, D.,  Rarasati, A.,  Saputro, S. W., Baskoro, E. T.}:
The local metric dimension of split and unicyclic graphs.
{Indonesian J. of Comb.}, \textbf{6} (1), 50--57 (2022). DOI: 10.19184/ijc.2022.6.1.3




%
%
\bibitem{ghm18} {\sc Gonzalez, A.,  Hernando, C., Mora, M.}:
Metric-locating-dominating sets of graphs for constructing related subsets of vertices.
{Appl. Math. Comput.}, \textbf{332}, 449--456  (2018). DOI: 10.1016/j.amc.2018.03.053


%
%
\bibitem{gsw14} {\sc Grigorious, C., Stephen, S., William, A.}:
On the strong dimension of diametrically vertex uniform graphs.
{Int. J. Comput. Algorithm}, \textbf{03}, 625--628 (2014). DOI: 10.26493/1855-3974.813.903



%
%
\bibitem{hm76} {\sc Harary F., Melter, R. A.}:
On the metric dimension of a graph.
{Ars Combin}, \textbf{2}, 191--195  (1976). DOI: 10.1137/050641867
%{\scriptsize \fbox{\color{red}$\dim$}} {\scriptsize \fbox{\color{blue}262c}}




%
%
\bibitem{ho04} {\sc Henning, M. A., Oellermann, O. R.}:
Metric-locating-dominating sets in graphs.
{Ars Combin.}, \textbf{73} (8), 129--141   (2004). 


%
%
\bibitem{hmp14} {\sc Hernando, C.,  Mora, M.,  Pelayo, I. M.}:
Nordhaus-Gaddum bounds for locating domination.
{European J. Combin.}, \textbf{36}, 1--6.  (2014). DOI: 10.1016/j.ejc.2013.04.009


%
%
\bibitem{hmpscp05} {\sc Hernando, C.,  Mora, M.,  Pelayo,  I. M., Seara, C., Caceres, J., Puertas, M. L.}:
On the metric dimension of some families of graphs.
{Electron. Notes Discrete Math.}, \textbf{22}, 129--133 (2005). DOI: 10.1016/j.endm.2005.06.023


%
%
\bibitem{hmpsw10} {\sc Hernando, C.,  Mora, M.,  Pelayo, I. M.,  Seara, C.,  Wood, D. R.}:
Extremal graph theory for metric dimension and diameter.
{Electron. J. Combin.}, \textbf{17} (1), 28 pp. (2010). DOI: 10.37236/302


%
%
\bibitem{hmsw08} {\sc Hernando, C.,  Mora, M.,  Slater, P. J.,  Wood, D. R.}:
{\it 	Fault-tolerant metric dimension of graphs}.
{Ramanujan Math. Soc. Lect. Notes Ser.} \textbf{5}, 81--85 (2008).




%
%
\bibitem{j22.2} {\sc Jannesari, M.}:
On doubly resolving sets in graphs.
{Bull. Malays. Math. Sci. Soc.}, \textbf{45} (5), 2041--2052 (2022). DOI: 10.1007/s40840-022-01366-1


%
%
\bibitem{jscs09} {\sc Javaid, I., Salman, M., Chaudhry, M. A., Shokat, S.}:
Fault-tolerance in resolvability.
{Util. Math.}, \textbf{312} (80), 263--275 (2009).  


%
%
\bibitem{kty18} {\sc Kelenc, A., Tratnik, N., Yero, I. G.}:
Uniquely identifying the edges of a graph: the edge metric dimension.
{Discrete Appl. Math.}, \textbf{251}, 204--220 (2018). DOI: 10.1016/j.dam.2018.05.052




%
%
\bibitem{kkty17} {\sc Kelenc, A., Kuziak, D., Taranenko, A., Yero, I. G.}:
Mixed metric dimension of graphs.
{Appl. Math. Comput.}, \textbf{314}, 429--438 (2017). DOI: 10.1016/j.amc.2017.07.027


%
%
\bibitem{krr96} {\sc Khuller, S., Raghavachari, B., Rosenfeld, A.}:
Landmarks in graphs.
{Discrete Appl. Math.}, \textbf{70} (3), 217--229 (1996). DOI: 10.1016/0166-218X(95)00106-2




%
%
\bibitem{kmmsy21} {\sc Knor, M., Majstorovic, S., Masa Toshi, A. T., Skrekovski, R., Yero, I. G.}:
Graphs with the edge metric dimension smaller than the metric dimension.
{Appl. Math. Comput.}, \textbf{401}, Paper No. 126076, 7 pp (2021). DOI: 10.1016/j.amc.2021.126076


%
%
\bibitem{kkcm14} {\sc Kratica, J., Kovacevic-Vujcic, V., Cangalovic, M., Mladenovic, N.}:
Strong metric dimension: a survey.
{Yugosl. J. Oper. Res.}, \textbf{24} (2), 187--198 (2014). DOI: 10.2298/YJOR130520042K


%
%
\bibitem{kkcs12.1} {\sc Kratica, J., Kovacevic-Vujcic, V., Cangalovic, M., Stojanovic, M.}:
Minimal doubly resolving sets and the strong metric dimension of Hamming graphs.
{Appl. Anal. Discrete Math.}, \textbf{6} (1), 63--71 (2012). DOI: 10.2298/AADM111116023K



%
%
\bibitem{kkcs12.2} {\sc Kratica, J., Kovacevic-Vujcic, V., Cangalovic, M., Stojanovic, M.}:
Minimal doubly resolving sets and the strong metric dimension of some convex polytopes.
{Appl. Math. Comput.}, \textbf{218} (19), 9790--9801 (2012). DOI: 10.1016/j.amc.2012.03.047



%
%
\bibitem{k20} {\sc Kuziak, D.}:
{ 	 The strong resolving graph and the strong metric dimension of cactus graphs}.
{Mathematics}, \textbf{8}, 1266 (2020). DOI: 10.3390/math8081266


%
%
\bibitem{kyr13} {\sc Kuziak, D.,   Yero, I. G., Rodriguez-Velazquez, J. A.}:
On the strong metric dimension of corona product graphs and join graphs.
{Discrete Appl. Math.}, \textbf{161} (7-8), 1022--1027 (2013).  DOI: 10.1016/j.disc.2012.07.025


%
%
\bibitem{mo11} {\sc May, T. R.,   Oellermann, O. R.}:
The strong dimension of distance-hereditary graphs.
{J. Combin. Math. Combin. Comput.}, \textbf{76}, 59--73  (2011). 


%
%
\bibitem{mkkc12} {\sc Mladenovic, N., Kratica, J., Kovacevic-Vujcic,V., Cangalovic, M.}:
Variable neighborhood search for metric dimension and minimal doubly resolving set problems.
{European J. Oper. Res.}, \textbf{220} (2), 328--337 (2012). DOI: 10.1016/j.ejor.2012.02.019


%
%
\bibitem{mst22} {\sc Mora, M., Souto-Salorio, M. J., Tarrio-Tobar, A. D.}:
Resolving sets tolerant to failures in three-dimensional grids.
{Mediterr. J. Math.}, \textbf{19} (4), Paper No. 188, 19 pp.(2022). DOI: 10.1007/s00009-022-02096-1





%
%
\bibitem{op07} {\sc Oellermann, O. R.,   Peters-Fransen, J.}:
The strong metric dimension of graphs and digraphs.
{Discrete Appl. Math.}, \textbf{155} (3), 356--364 (2007). DOI: 10.1016/j.dam.2006.06.009





%
%
\bibitem{opz10} {\sc Okamoto, F.,   Phinezy, B., Zhang, P.}:
The local metric dimension of a graph.
{Math. Bohem.}, \textbf{135} (3), 239--255 (2010). DOI: 10.21136/MB.2010.140702


%
%
\bibitem{py20} {\sc Peterin, I.,   Yero, I. G.}:
Edge metric dimension of some graph operations.
{Bull. Malays. Math. Sci. Soc.}, \textbf{43} (3), 2465--2477 (2020). DOI: 10.1007/s40840-019-00816-7




%
%
\bibitem{pz02} {\sc Poisson, C.,  Zhang, P.}:
The metric dimension of unicyclic graphs.
{J. Combin. Math. Combin. Comput.}, \textbf{40}, 17--32 (2002). 



%
%
\bibitem{rbg16} {\sc Rodriguez-Velazquez, Barragan-Ramirez, G. A., Garcia Gomez, C.}:
On the local metric dimension of Corona product graphs..
{Bull. Malays. Math. Sci. Soc.}, \textbf{39} (1), S157--S173 (2016). DOI: 10.1007/s40840-015-0283-1



%
%
\bibitem{rgb15} {\sc Rodriguez-Velazquez, Garcia Gomez, C.,  Barragan-Ramirez, G. A.}:
Computing the local metric dimension of a graph from the local metric dimension of primary subgraphs.
{Int. J. Comput. Math.}, \textbf{92} (4), 686--693 (2015). DOI: 10.1080/00207160.2014.918608


%
%
\bibitem{ryko14} {\sc Rodriguez-Velazquez, J. A., Yero, I. G., Kuziak, D., Oellermann, O. R.}:
On the strong metric dimension of Cartesian and direct products of graphs.
{Discrete Math.}, \textbf{335}, 8--19 (2014). DOI: 10.1016/j.disc.2014.06.023



%
%
\bibitem{st04} {\sc S\H{e}bo, A., Tannier, E.}:
On metric generators of graphs.
{Math. Oper. Res.}, \textbf{29} (2), 383--393 (2004). DOI: 10.1287/moor.1030.0070





%
%
\bibitem{ss21.5} {\sc Sedlar, J., Skrekovski, R.}:
Bounds on metric dimensions of graphs with edge disjoint cycles.
{Appl. Math. Comput.}, \textbf{396}, Paper No. 125908, 10 pp. (2021). DOI: 10.1016/j.amc.2020.125908



%
%
\bibitem{ss21} {\sc Sedlar, J., Skrekovski, R.}:
Extremal mixed metric dimension with respect to the cyclomatic number.
{Appl. Math. Comput.}, \textbf{404}, Paper No. 126238, 8 pp. (2021). DOI: 10.1016/j.amc.2021.126238


%
%
\bibitem{ss21.2} {\sc Sedlar, J., Skrekovski, R.}:
Mixed metric dimension of graphs with edge disjoint cycles.
{Discrete Appl. Math.}, \textbf{300}, 1--8 (2021). DOI: 10.1016/j.dam.2021.05.004




%
%
\bibitem{ss22.2} {\sc Sedlar, J., Skrekovski, R.}:
Metric dimensions vs. cyclomatic number of graphs with minimum degree at least two.
{Appl. Math. Comput.}, \textbf{427}, Paper No. 127147, 12 pp. (2022). DOI: 10.1016/j.amc.2022.127147





%
%
\bibitem{ss22.1} {\sc Sedlar, J., Skrekovski, R.}:
Vertex and edge metric dimensions of cacti.
{Discrete Appl. Math.}, \textbf{320}, 126--139 (2022). DOI: 10.1016/j.dam.2022.05.008




%
%
\bibitem{ss22} {\sc Sedlar, J., Skerovski, R.}:
Vertex and edge metric dimensions of unicyclic graphs.
{Discret. Appli. Math.} \textbf{314}, 81--92 (2022). DOI: 10.1016/j.dam.2022.02.022





%
%
\bibitem{s75} {\sc Slater, P. J.}:
Leaves of trees.
{Congr. Numer.}, \textbf{14}, 549--559 (1975). 



%
%
\bibitem{s60} {\sc Silverman, R.}:
A metrization for power-sets with applications to combinatorial analysis.
{Canadian J. Math.}, \textbf{12}, 158--176 (1960). 

%
%
\bibitem{y13} {\sc Yi, E.}:
On strong metric dimension of graphs and their complements.
{Acta Math. Sin.}, \textbf{29} (8), 1479--1492 (2013). DOI: 10.1007/s10114-013-2365-z


%
%
\bibitem{wyc22} {\sc Wei, M., Yue, J., Chen, L.}:
The effect of vertex and edge deletion on the edge metric dimension of graphs.
{J. Comb. Optim.}, \textbf{44} (1), 331--342 (2022). DOI: 10.1007/s10878-021-00838-7


%
%
\bibitem{zab22} {\sc Zulfaneti, Assiyatun, H., Baskoro, E. T.}:
On metric-location-domination number of graphs.
{Int. J. Math. Comput. Sci.}, \textbf{17} (4), 1721--1733 (2022). 

\end{thebibliography}


%\include{Symbols}


%\printindex


\end{document}
%%%%%%%%%%%%%%%%%%%%%%%%%%%%%%%%%%%%%%%%%%%%%%%%%%%%%%%%%%%%%%%%%%%%%%%%%%%%%%%%%%%%%%%%%%%%%%%%%%%%%%%%%%%%%%%%%%%%%%%%%%%%%%%%%%%%%%
%%%%%%%%%%%%%%%%%%%%%%%%%%%%%%%%%%%%%%%%%%%%%%%%%%%%%%%%%%%%%%%%%%%%%%%%%%%%%%%%%%%%%%%%%%%%%%%%%%%%%%%%%%%%%%%%%%%%%%%%%%%%%%%%%%%%%%
%%%%%%%%%%%%%%%%%%%%%%%%%%%%%%%%%%%%%%%%%%%%%%%%%%%%%%%%%%%%%%%%%%%%%%%%%%%%%%%%%%%%%%%%%%%%%%%%%%%%%%%%%%%%%%%%%%%%%%%%%%%%%%%%%%%%%%
%%%%%%%%%%%%%%%%%%%%%%%%%%%%%%%%%%%%%%%%%%%%%%%%%%%%%%%%%%%%%%%%%%%%%%%%%%%%%%%%%%%%%%%%%%%%%%%%%%%%%%%%%%%%%%%%%%%





