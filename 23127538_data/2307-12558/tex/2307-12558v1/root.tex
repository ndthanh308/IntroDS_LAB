%%%%%%%%%%%%%%%%%%%%%%%%%%%%%%%%%%%%%%%%%%%%%%%%%%%%%%%%%%%%%%%%%%%%%%%%%%%%%%%%
%2345678901234567890123456789012345678901234567890123456789012345678901234567890
%        1         2         3         4         5         6         7         8

\documentclass[letterpaper, 10 pt, conference]{ieeeconf}  % Comment this line out if you need a4paper

%\documentclass[a4paper, 10pt, conference]{ieeeconf}      % Use this line for a4 paper

\IEEEoverridecommandlockouts                              % This command is only needed if 
                                                          % you want to use the \thanks command

\overrideIEEEmargins                                      % Needed to meet printer requirements.

%In case you encounter the following error:
%Error 1010 The PDF file may be corrupt (unable to open PDF file) OR
%Error 1000 An error occurred while parsing a contents stream. Unable to analyze the PDF file.
%This is a known problem with pdfLaTeX conversion filter. The file cannot be opened with acrobat reader
%Please use one of the alternatives below to circumvent this error by uncommenting one or the other
%\pdfobjcompresslevel=0
%\pdfminorversion=4

% See the \addtolength command later in the file to balance the column lengths
% on the last page of the document

% The following packages can be found on http:\\www.ctan.org
%\usepackage{graphics} % for pdf, bitmapped graphics files
%\usepackage{epsfig} % for postscript graphics files
%\usepackage{mathptmx} % assumes new font selection scheme installed
%\usepackage{times} % assumes new font selection scheme installed
%\usepackage{amsmath} % assumes amsmath package installed
%\usepackage{amssymb}  % assumes amsmath package installed

\usepackage{graphicx}
\usepackage{amsmath}
\usepackage{amssymb}
\usepackage{booktabs}
\usepackage{float}
\usepackage{color}
\usepackage{subcaption}
% \usepackage[pagebackref,breaklinks,colorlinks]{hyperref}

\title{\LARGE \bf
Revisiting Event-based Video Frame Interpolation
}


% \author{Jiaben Chen$^{1}$, $^{2}$% <-this % stops a space
\author{Jiaben Chen$^{1}$, Yichen Zhu$^{2}$, Dongze Lian$^{3}$, Jiaqi Yang$^{2}$, Yifu Wang$^{2}$, Renrui Zhang$^{4}$, Xinhang Liu$^{2}$, \\ Shenhan Qian$^{5}$, Laurent Kneip$^{2}$ and Shenghua Gao$^{2}$
\thanks{This work was supported by NSFC \#61932020, \#62172279, \#62250610225, Program of Shanghai Academic Research Leader, projects 22dz1201900 and 22ZR1441300 funded by the Shanghai Science Foundation and “Shuguang Program” supported by Shanghai Education Development Foundation and Shanghai Municipal Education Commission.}
\thanks{$^{1}$UC San Diego; $^{2}$ShanghaiTech University; $^{3}$National University of Singapore; $^{4}$Shanghai AI Laboratory; $^{5}$Technical University of Munich}
% \thanks{$^{2}$ShanghaiTech University}
% \thanks{$^{3}$National University of Singapore}
% \thanks{$^{4}$Shanghai AI Laboratory}
% \thanks{$^{5}$Hong Kong University of Science and Technology}
% \thanks{*This work was not supported by any organization}% <-this % stops a space
% \thanks{$^{1}$Albert Author is with Faculty of Electrical Engineering, Mathematics and Computer Science,
%         University of Twente, 7500 AE Enschede, The Netherlands
%         {\tt\small albert.author@papercept.net}}%
% \thanks{$^{2}$Bernard D. Researcheris with the Department of Electrical Engineering, Wright State University,
%         Dayton, OH 45435, USA
%         {\tt\small b.d.researcher@ieee.org}}%
}


\begin{document}



\maketitle
\thispagestyle{empty}
\pagestyle{empty}


%%%%%%%%%%%%%%%%%%%%%%%%%%%%%%%%%%%%%%%%%%%%%%%%%%%%%%%%%%%%%%%%%%%%%%%%%%%%%%%%
\begin{abstract}

Dynamic vision sensors or event cameras provide rich complementary information for video frame interpolation. Existing state-of-the-art methods follow the paradigm of combining both synthesis-based and warping networks. However, few of those methods fully respect the intrinsic characteristics of events streams. Given that event cameras only encode intensity changes and polarity rather than color intensities, estimating optical flow from events is arguably more difficult than from RGB information. We therefore propose to incorporate RGB information in an event-guided optical flow refinement strategy. Moreover, in light of the quasi-continuous nature of the time signals provided by event cameras, we propose a divide-and-conquer strategy in which event-based intermediate frame synthesis happens incrementally in multiple simplified stages rather than in a single, long stage. Extensive experiments on both synthetic and real-world datasets show that these modifications lead to more reliable and realistic intermediate frame results than previous video frame interpolation methods. Our findings underline that a careful consideration of event characteristics such as high temporal density and elevated noise benefits interpolation accuracy.
   

\end{abstract}


%%%%%%%%%%%%%%%%%%%%%%%%%%%%%%%%%%%%%%%%%%%%%%%%%%%%%%%%%%%%%%%%%%%%%%%%%%%%%%%%
\section{INTRODUCTION}

% \textcolor{blue}{Swinging golf clubs, racing sports cars, and diving roller coasters all are ubiquitous, daily events passing by within short time intervals, and people often use their smartphones to record such memorable moments. Modern portable devices, however, capture videos at a fixed, low frame rate. Though professional high-speed cameras are capable of capturing videos at a high temporal resolution, deploying such high-speed recording capabilities on consumer-level devices remains impractical due to large memory requirements and high power consumption.} 
Video Frame Interpolation (VFI), the task of converting low frame rate videos into high frame rate sequences by interpolating new frames between two given consecutive frames \cite{niklaus2017video,niklaus2018context}, has promising applications in the field of robotics. By providing additional visual information, it can enhance robot motion planning and control. Although dedicated hardware for capturing high frame-rate videos exists, it would be costly to deploy such devices.

% Figure environment removed
% In order to overcome the above limitations, the community has proposed Video Frame Interpolation (VFI) methods that convert low frame rate videos into high frame rate sequences by interpolating new frames between two given consecutive frames \cite{niklaus2017video,niklaus2018context}. However, 
In RGB image-based VFI, the leveraged information remains limited by the frame rate of the original videos. Recently, event cameras have proven their ability to complement RGB imagery with temporally dense sensing, thus providing additional constraints to improve traditional VFI \cite{lin2020learning,han2021evintsr,yu2021training}. This paper aims at event-supported VFI. 

Existing work such as Time Lens \cite{tulyakov2021time} already utilizes an event camera for event-based VFI. It achieves remarkable performance by combining a synthesis and a warping network in a two-stream architecture. The synthesis network directly synthesizes intermediate frames, while the warping network utilizes events to generate optical flow for frame warping. However, Time Lens \cite{tulyakov2021time} solely employs events to estimate optical flow in the warping network. We argue that directly estimating optical flow from events is a very challenging task, and inferior results can compromise the performance of VFI. Furthermore, Time Lens \cite{tulyakov2021time} utilizes only the outer image timestamps within the synthesis network and does not consider the fact that the event camera captures the image change in a continuous, high-frequent way. Consequently,  the full potential of including events has not yet been unlocked. To better leverage event traits within VFI, we propose the following two improvements.
%event cameras focus more on sensitive motion capture compared with the RGB camera due to its high-frequency sampling, 

First, event cameras only encode changes of intensity and polarity rather than absolute color intensities \cite{gallego2020event}. Estimating optical flow from events only is therefore arguably more difficult than from RGB cameras. RGB-based optical flow methods evaluate pixel value similarities between consecutive frames. In contrast, in event sequences, corresponding pixels present similar numbers of accumulated events within relatively small time intervals. The high sensitivity of event cameras leads to an abundance of events within short time intervals and non-negligible noise accumulations. On the other hand, spatial event distribution is often sparse and lacking clear information in low-textured regions. The strength of events is therefore considered to lie in their ability to sense change that occurs over very short periods of time. Over longer time periods, however, the accumulation of noise will blur information and complicate the identification of accurate correspondences. The calculation of optical flow solely from event sequences therefore appears to be more challenging than traditional image-matching-based approaches.
 % \textcolor{blue}{Similar to regular RGB-based methods, purely event-based optical flow estimation \cite{tulyakov2021time} still relies on the brightness constancy assumption.} \textcolor{blue}{State-of-the-art methods have proven their ability to recover complete and accurate optical flow fields.}

We argue that \textbf{event sequences are utilized better to calibrate optical flow estimated from RGB images rather than to estimate optical flow solely from an event camera}. To this end, we propose an \emph{event-guided recurrent warping} strategy, which more closely respects the intrinsic properties and advantages of event cameras.

Second, different from RGB cameras that capture intensities at a low frame rate, event cameras \cite{lichtsteiner2008128} record per-pixel intensity changes asynchronously at high, quasi-continuous temporal resolution. In the synthesis network, Time Lens \cite{tulyakov2021time} directly synthesizes the middle frame in one step. The rich continuous-time nature of events is not fully utilized thus leading to sub-optimal interpolation results. Motivated by global and local temporal feature fusion strategies in other video processing problems \cite{feichtenhofer2019slowfast}, we propose to replace the one-step method to obtain a single intermediate frame by a \emph{proxy-guided synthesis} strategy that synthesizes multiple intermediate frames in a step-by-step manner. Specifically, by slicing an event sequence along time, we can synthesize intermediate frames (\emph{proxies}) from thinner event segments. % By doing so, the final intermediate frames can be synthesized step-by-step, which we denote as \emph{proxy-guided synthesis} strategy.
Our main rationale behind this design is \textit{divide and conquer}. \textbf{By splitting the events into multiple slices and synthesizing in stages, each stage is demanded to regress from a simpler function with reduced non-linearity using data that is less affected by temporal error accumulations}. Rather than solving a large difficult problem over a global time interval and implicitly understanding complicated non-linear motion over time, we propose to solve the problem sequentially over multiple local intervals containing simpler motion. The proposed \emph{proxy-guided synthesis} intuitively makes better use of the rich continuous-time nature of events. 

The overall architecture of our proposed framework for event-based VFI is shown in Fig. \ref{fig:overview}, and our main contributions are summarized as follows:
\begin{itemize}
    \item We propose an \emph{Event-guided Recurrent Warping} strategy, which uses events to calibrate optical flow estimates from RGB images. Our solution better grasps the advantages of event sequences.
    % We propose a reasonable data fusion strategy for event camera based VFI. We consider the fact that event cameras only encode changes of intensity and polarity rather than color intensities. Rather than estimating optical flow solely from events,
    \item Exploiting the rich quasi-continuous nature of the temporal information provided by events, we design a \emph{Proxy-guided Synthesis} strategy to incrementally synthesize intermediate frames and effectively combine both local and global temporal information. 
    \item Extensive experiments on both synthetic and real-world benchmarks demonstrate VFI performance improvement by our approach.
    %The temporal consistency constraint is also introduced to ensure the consistency of the synthesis results. 
\end{itemize}

\section{Related Work}
\subsection{Frame-based Interpolation}
% State-of-the-art video interpolation methods estimate the intermediate frame by inferring motion from low-frame-rate adjacent keyframes commonly via optical flow approximation.% 
% Optical flow Methods:
% Optical flow-agnostic Methods:
Most existing frame-based VFI methods estimate intermediate frames with sparse input RGB frames at a fixed low rate, categorized into kernel-based and flow-based methods.

\noindent \textbf{Kernel-based approaches} model VFI as a local convolution of input key frames by predicting a spatially-adaptive convolutional kernel instead of estimating explicit intermediate motion \cite{niklaus2017video}. However, CNN-based fail at handling large displacements due to limited kernel size, or otherwise induce exploding computation times and memory consumption. \\
\noindent \textbf{Flow-based approaches} proceed by estimating optical flow between input frames and generate intermediate frames using image warping \cite{jaderberg2015spatial}. Linear \cite{jiang2018super,li2020video}, quadratic \cite{xu2019quadratic} and cubic \cite{chi2020all} trajectory assumptions have been made to approximate intermediate motion. Techniques like softmax splatting \cite{niklaus2020softmax}, motion fields \cite{park2020bmbc,park2021asymmetric}, Transformers \cite{lu2022video}, privileged distillation \cite{huang2022real}, and extra information like contextual maps \cite{niklaus2018context}, temporal sliding windows of frames \cite{kalluri2020flavr} and depth maps \cite{bao2019depth} have all been adopted to improve flow estimation and interpolation accuracy. Although these methods present great variety in their designs and approaches to handle complex motion, they are all limited by low-order assumptions within their inner flow estimator, which can be attributed to the lack of intermediate motion signals.

In a nutshell, frame-based approaches rely on the brightness constancy assumption and use limited visual information, thus leading to limited performance in scenarios with complex non-linear motion and illumination changes.
\subsection{Event-based interpolation}
In contrast to conventional cameras, event cameras \cite{lichtsteiner2008128} are relatively new sensors that capture temporally dense motion signals as a sequence of asynchronous per-pixel brightness changes. Owing to the rapid development of industrial event cameras, recent years have witnessed a surge of event-based methods addressing the VFI problem. Purely event-based methods learn video reconstruction from data and train deep neural networks like CNNs \cite{paredes2021back} and RNNs \cite{rebecq2019events}. More recently, the community has started to investigate the highly interesting sensor fusion variant of complementary event-plus-frame methods to solve the VFI problem\cite{wang2020event,pan2019bringing,lin2020learning,han2021evintsr,zhang2022unifying}. By capturing dense motion information and compressed true visual signals \cite{tulyakov2021time}, event cameras can serve as an ideal supplement for frame-based VFI in challenging scenarios. Following work \cite{tulyakov2022time} further explores using events to estimate non-linear motion between frames with a motion spline estimator. \cite{wu2022video} proposes to use events to blend optical flow from VFI flow estimator \cite{huang2022real}. Works like \cite{yu2021training,he2022timereplayer} explore potential of event-based VFI in weakly and unsupervised learning setting, respectively. \cite{tulyakov2021time} introduces a pipeline combining synthesis and warping-based approaches, showing promising results in high dynamic settings. However, their synthesis-based methods feed the whole event sequence directly into a synthesis network, thus leading to a loss of the rich temporal information provided by events and blurry intermediate frames. Furthermore, the method estimates optical flow solely from events which yields messy predictions when events are noisy, sparse or not correctly registered.

% \section{PROCEDURE FOR PAPER SUBMISSION}

% \subsection{Selecting a Template (Heading 2)}

% First, confirm that you have the correct template for your paper size. This template has been tailored for output on the US-letter paper size. 
% It may be used for A4 paper size if the paper size setting is suitably modified.

% \subsection{Maintaining the Integrity of the Specifications}

% The template is used to format your paper and style the text. All margins, column widths, line spaces, and text fonts are prescribed; please do not alter them. You may note peculiarities. For example, the head margin in this template measures proportionately more than is customary. This measurement and others are deliberate, using specifications that anticipate your paper as one part of the entire proceedings, and not as an independent document. Please do not revise any of the current designations

\section{Method}

\subsection{Preliminaries}

\noindent \textbf{Problem Definition.} 
Given consecutive video frames $I_0, I_1\in\mathbb{R}^{W \times H \times 3}$ as inputs, where $W$ and $H$ are the width and height of the frames, respectively, VFI aims to predict an intermediate new frame $\hat{I}_t$ at time $t \in (0, 1)$. For event-based VFI, event sequences $\mathbb{E}_{0 \rightarrow t}$ and $\mathbb{E}_{t \rightarrow 1}$ are also included as inputs, which encompass all triggered events between the time of the input RGB frame $I_0$ and the time $t$ of the intermediate frame, and the events between the latter and the input RGB frame $I_1$.

\noindent \textbf{Event Representation.} 
The events triggered in a given time interval form a sequence $\{e_i = (x_i, y_i, t_i, p_i)\}_{i \in [1,M]}$, where $M$ indicates the total number of events. Each event is comprised of a pixel position $x_i, y_i$ indicating the location of a brightness change, a timestamp $t_i$ indicating when the change occured, and a polarity $p_i$ indicating whether the perceived logarithmic brightness at that pixel increased or decreased by a certain threshold amount. Owing to their unique asynchronous nature, raw events cannot be directly taken as an input to a neural network. Following \cite{zhu2019unsupervised}, we adopt a discretized event volume representation by dividing the temporal dimension into $B$ bins. In this way, event sequences are encoded as a $B$-channel tensor permitting the application of 2D convolutions in spatial dimensions. All modules in our work use converted, voxel grid-based event representations.

%This event representation can reconstruct the events without much information loss in spatial and temporal domain.
% Additionally, since events can be viewed as a compressed form of real visual signal and events could bridge the "blind time" gap of traditional cameras, they turned to be useful for our task \cite{tulyakov2021time} in challenging scenarios with highly dynamic range or high-speed non-linear motion.

\subsection{Overview of the architecture}
%In this section, we briefly introduce our proposed video frame interpolation method which supports arbitrary time frame (one or more) interpolation. 
Drawing inspiration from Time Lens \cite{tulyakov2021time}, we propose a VFI framework that relies on two complementary modules: a synthesis module and a warping module. The overall architecture is shown in Fig. \ref{fig:overview}. It is mainly composed of three components: a \emph{Proxy-guided Synthesis} Module, an \emph{Event-guided Recurrent Warping} Module, and an \emph{Attention-based Averaging} Module. In the \emph{Proxy-guided Synthesis} Module, we slice the two event voxel grids between the boundary RGB frames ($I_0$ and $I_1$) and the desired intermediate frame $\hat{I}_t$ into two sub-slices along the temporal dimension, and we utilize these slices to incrementally synthesize further intermediate frames $\hat{I}^{\text{syn}}_t$, i.e., proxies (Sec. \ref{3.3}). In the \emph{Event-guided Recurrent Warping} Module, an optical flow estimate is first predicted from RGB frames and then refined by the event sequences. The output frame $\hat{I}^{\text{warp}}_t$ is generated by backward warping (Sec. \ref{3.4}). Finally, we output the interpolated frame $\hat{I}_t$ via an \emph{Attention-based Averaging} Module that blends the results of the previous two modules, thus aiming at overcoming individual deficiencies and combining the advantages of the synthesis-based and warping-based schemes (Sec. \ref{3.5}). A detailed exposition of each module is presented in the following sections.

%It is worth noting that each module is bidirectional, we would only elaborate our designed process from direction $0$ to $t$ for simplicity (and vice versa for $1$ to $t$).
% We propose a learning-based system which could be seen as two stages: separate interpolation modules and attention-averaging module. And our overall pipeline draws inspiration from \cite{tulyakov2021time}, which demonstrates the merits of combining different modules with attention-based averaging. As is illustrated in , our pipeline consists mainly of five complementary different modules, integrating each module's advantage to generate better interpolation results. Specifically, the first stage is composed of (1) the \emph{Fusion Module} is a synthesis-based method inspired by \cite{tulyakov2021time} which fuse the input (both event sequences and keyframes) to directly interpolate the intermediate frame; (2) the \emph{Warping Module} is a warping-based method which firstly compute the bidirectional optical flow from events and frames, secondly generate the interpolated frame by utilizing the estimated flow to warp the keyframes; (3) the \emph{Flow-based Module} is also a warping-based method from \cite{huang2021rife} using only the keyframe information, however directly estimating the intermediate flow in a coarse-to-fine fashion instead of estimating the bidirectional flow as (2); (4) the \emph{Flow-agnostic Module} from \cite{kalluri2020flavr} uses 3D temporal-spatial convolution to estimate a new frame without optical flow, aiming to model the temporal dynamics from input keyframes. Finally, our second stage: (5) the \emph{Attention-based Averaging Module} learns weights to optimally combine the four above modules, blending the results to overcome the demerits of different methods and to marry the advantages of synthesis-based, flow-based and flow-agnostic schemes.   
%介绍Pipeline是什么,整体图
% Figure environment removed
\subsection{Proxy-guided Synthesis Module \label{3.3}}
Aiming to take advantage of the rich continuous time signals provided by event cameras, we divide the event sequences along the time axis to obtain multiple event segments. More specifically, we split the synthesis module into two branches: the \emph{Direct Synthesis Branch}, and the \emph{Transitional Synthesis Branch}. The complete structure is shown in Fig. \ref{fig:synthesis}. The event segmentation strategy is illustrated in Fig. \ref{fig:Segment}.
\noindent \textbf{Direct Synthesis.} In the \emph{Direct Synthesis Branch}, we directly regress the intermediate frame from input RGB frames $I_0$ and $I_1$ as well as the two event sequences $\mathbb{E}_{0 \rightarrow t}$ and $\mathbb{E}_{1 \rightarrow t}$. Note that we utilize the event sequence reversal strategy from \cite{tulyakov2021time} to compute $\mathbb{E}_{1 \rightarrow t}$ from input sequence $\mathbb{E}_{t \rightarrow 1}$.

Taking $I_0$ and $\mathbb{E}_{0 \rightarrow t}$ as inputs, a neural network is employed to directly synthesise the desired intermediate frame $\hat{I}_{0 \rightarrow t}^{\text{direct}}$. Meanwhile, the intermediate frame is also generated in reverse direction by considering $I_1$ and $\mathbb{E}_{1 \rightarrow t}$ as inputs, denoted $\hat{I}_{1 \rightarrow t}^{\text{direct}}$:
\begin{equation}
\setlength{\abovedisplayskip}{5pt}
\setlength{\belowdisplayskip}{5pt}
    \left\{
    \begin{array}{cc}
         \hat{I}_{0 \rightarrow t}^{\text{direct}} &= f_1(I_0, \mathbb{E}_{0 \rightarrow t}) \\
         \hat{I}_{1 \rightarrow t}^{\text{direct}} &= f_1(I_1, \mathbb{E}_{1 \rightarrow t}),
    \end{array}
    \right.
\end{equation}
where $f_1$ is a neural network function. The \emph{Direct Synthesis Branches} generate intermediate frames using global time signals, which is insufficient to fully exploit the rich continuous-time nature of the events. To enhance the interpolation, we further consider more local time signals by our \emph{Transitional Synthesis Branch}.  
% % Figure environment removed


\noindent \textbf{Transitional Synthesis.} In the \emph{Transitional Synthesis Branch}, we learn from local time intervals as shown in Fig. \ref{fig:Segment}. Instead of the straightforward approach of direct synthesis from a concatenation of all input frames and events as introduced in \cite{tulyakov2021time}, we propose to generate the final intermediate frame step-by-step. The global time interval is divided into $T$ time segments.
%
%In stead of the straightforward approach which directly concatenate input frames and whole event sequences to regress desired frame $I_t$, we propose to generate the final result step by step. Given input frame $I_0$ and events $\mathbb{E}_{0.5t \rightarrow t}$, we first synthesize a transitional frame $I_{0.5t}$ using a skip-connected Hourglass \cite{newell2016stacked} U-Net architecture \cite{ronneberger2015u,jiang2018super}, \textit{i}.\textit{e}.
%
Given an input RGB frame $I_0$ and an event sequence \{$\mathbb{E}_{0 \rightarrow t}$\}, we obtain the sliced sequences \{$\mathbb{E}_{0 \rightarrow \frac{1}{T} t}$, $\mathbb{E}_{\frac{1}{T} t \rightarrow \frac{2}{T} t},$ $\ldots$, $\mathbb{E}_{\frac{i - 1}{T} t \rightarrow \frac{i}{T} t}$, $\ldots$, $\mathbb{E}_{\frac{T-1}{T} t \rightarrow t}$\}. We first synthesize a transitional frame $\hat{I}^{\text{proxy}}_{0 \rightarrow \frac{1}{T} t}$ using $I_0$ and $\mathbb{E}_{0 \rightarrow \frac{1}{T} t}$. We then continue and repeat the above process gradually to synthesize the desired intermediate frame $\hat{I}_{0 \rightarrow t}^{\text{proxy}}$ from this direction:
\begin{equation}
    \left\{
    \begin{array}{cl}
        \hat{I}^{\text{proxy}}_{0 \rightarrow \frac{1}{T} t} &= f_2(I_0, \mathbb{E}_{0 \rightarrow \frac{1}{T} t}), \\
         & \vdots  \\
        \hat{I}^{\text{proxy}}_{0 \rightarrow \frac{i}{T} t} &= f_2(\hat{I}^{\text{proxy}}_{0 \rightarrow \frac{i - 1}{T} t}, \mathbb{E}_{\frac{i - 1}{T} t \rightarrow \frac{i}{T} t}), \\
         & \vdots  \\
        \hat{I}_{0 \rightarrow t}^{\text{proxy}} &= f_2(\hat{I}^{\text{proxy}}_{0 \rightarrow \frac{T-1}{T} t}, \mathbb{E}_{\frac{T-1}{T} t \rightarrow t}),
    \end{array}
    \right.
\end{equation}
where $f_2$ is a neural network function.
    
%motivated by the temporal consistency of event camera, 
%we segment each event sequences between RGB keyframe and the desired intermediate frame into two sub-voxels, which is illustrated in Fig. \ref{fig:Segment} as $\mathbb{E}_{0 \rightarrow 0.5t}, \mathbb{E}_{0.5t \rightarrow t}, \mathbb{E}_{1.5t \rightarrow t}, \mathbb{E}_{2 \rightarrow 1.5t}$. 

%In stead of the straightforward approach which directly concatenate input frames and whole event sequences to regress desired frame $I_t$, we propose to generate the final result step by step. Given input frame $I_0$ and events $\mathbb{E}_{0.5t \rightarrow t}$, we first synthesize a transitional frame $I_{0.5t}$ using a skip-connected Hourglass \cite{newell2016stacked} U-Net architecture \cite{ronneberger2015u,jiang2018super}, \textit{i}.\textit{e}.,

The reverse direction adopts a similar process to generate the intermediate frame $\hat{I}_{1 \rightarrow t}^{\text{proxy}}$. 
%It is worth noting that we set $T = 2$, as shown in Fig. \ref{fig:synthesis} and Fig. \ref{fig:Segment}.

Event cameras provide quasi-continuous recordings of true visual signals between two RGB frames. If we only conduct direct synthesis from the complete event sequence, the quality of the interpolated frame will suffer in challenging situations in which substantial noise may have occurred. 
%We conduct the above transitional synthesizing approach to add additional consistency constraint to the intermediate interpolation process to guarantee the consistency of results of frame interpolation. 
The proposed \emph{Transitional Synthesis Branch} enables the model to learn from local time intervals within which the motion is simpler and noise accumulations are less apparent. A different problem is hence divided into multiple similar, simpler problems. The network implicitly focuses on the quality of intermediate results to ensure that the final cumulative result is more accurate. The \emph{Transitional Synthesis Branch} takes advantage of the continuous-time nature of the event sequence to break a hard learning problem into a sequence of similar, simpler learning problems, and thus provides an interesting novel direction for event-based learning algorithms. Note that in this particular work, we use $T=2$.
% Having the result of the transitional synthesis branch, we can demand that it is consistent with the result of the direct synthesis branch. That is to say, the generated video frames using the global time interval and the local time interval should be same. 

% Events are continuous recordings of the true visual signal between two RGB frames, hence we add a consistency restraint to the intermediate process to guarantee the consistency of results of frame interpolation. Also, since we generate the final result step by step, we could add more interpretability to the process instead of only focusing on the final output of the network

% Such a branch takes full advantage of the continuous attribute of the event sequence through learning from local time intervals and the consistency constraint, which produces non-neglectable impact on algorithms using events.

%We interpolate a transitional frame $I_{1.5t}$ from RGB frame $I_2$ and event sequence $\mathbb{E}_{2 \rightarrow 1.5t}$, and then generate one desired intermediate frame $\hat{I}_{t}^{tran2}$ with our interpolated middle result $I_{1.5t}$ and event sequence $\mathbb{E}_{1.5t \rightarrow t}$.

%Our intuition behind this design is the idea of \emph{Divide and Conquer}. By splitting the time interval into multiple slices and synthesizing in stages, each synthesizing stage has to regress from a less non-linear intensity development during its own stage. Rather than solving the large difficult problem with large time interval and implicit understanding of a complicated non-linear motion over time, we propose to solve it as a sequence of multiple much simpler problems with small intervals and simpler motions. Different from \cite{tulyakov2021time} which use the whole event sequence to synthesize, we not only supervise the final output, but also pay more attention to the intermediate results based on the assessment that better intermediate results would lead to better final results. Events are continuous recordings of the true visual signal between two RGB frames, if we only conduct the process from the whole event sequence, there is no guarantee that the intermediate moments are continuous, hence we add a consistency restraint to the intermediate process to guarantee the consistency of results of frame interpolation. Also, since we generate the final result step by step, we could add more interpretability to the process instead of only focusing on the final output of the network. Additionally, events record complicated and high frequency intensity changes during a time interval. .\\

%Including the whole sequence into consideration is also of vital importance. If we only focus on small time intervals, the information utilized in each period is local and limited. Therefore, we adopt a branch for direct synthesizing to learn from a more global temporal interval and again add consistency to the predictions from each sub stages in the transitional branch since we build connections from the isolated predictions from separate small intervals.\\
% Figure environment removed

\noindent \textbf{Synthesis Fusion.} After acquiring four interpolation candidates ($\hat{I}_{0 \rightarrow t}^{\text{direct}}$, $\hat{I}_{1 \rightarrow t}^{\text{direct}}$, $\hat{I}_{0 \rightarrow t}^{\text{proxy}}$ and $\hat{I}_{1 \rightarrow t}^{\text{proxy}}$) from the \emph{Direct Synthesis} and \emph{Transitional Synthesis} branches, we apply the \emph{Synthesis Fusion} module to generate the final prediction $\hat{I}_{t}^{\text{syn}}$ by fusing the results through an additional neural network (see Fig. \ref{fig:synthesis}). 
%
The \emph{Synthesis Fusion} block combines the results of two branches to cover both global and local time intervals. Taking the \emph{Direct Synthesis Module} into consideration is of vital importance because the information utilized within the \emph{Transitional Synthesis} branches is local and limited. The \emph{Direct Synthesis Module} learns from the complete temporal interval and helps to build connections between the isolated predictions from separate small intervals, thereby helping to ensure consistency within each sub-stage's prediction in the transitional branch. Owing to the combined design, our temporal pyramid-like architecture yields robust and consistent interpolation results.

% \begin{equation}
%     \begin{aligned}
%         \hat{I}^{s}_t &= f_3(\hat{I}_{t}^{dl}, \hat{I}_{t}^{dr}, \hat{I}_{t}^{tl}, \hat{I}_{t}^{tr}).
%     \end{aligned}
% \end{equation}
%Benefited from this synthesizing fusion, we combine the results of two branches to cover the global and local time intervals.

%with better use of event camera's time-consistency property. With global motion consideration in the synthesizing branch and continuous sub-interval-predictions from the transitional branch, our temporal-pyramid-like architecture yield results with integrity and consistency.\\
%Since this is a flow-free module without the assumption of brightness constancy, it could naturally avoid the pre-mentioned drawbacks of warping-based methods and could better handle challenging situations like changes of luminance and new object occurance on account of the intrinsic merits of event cameras and the auxiliary visual information provided by asynchronous events. However, the fusion module has defects of distortion of visual edges caused by noise of events and insufficient triggered events in low-texture regions, which we aim to utilize below modules to compensate.
\noindent \textbf{Loss Function.} We employ the \emph{$\ell_1$} loss as the reconstruction loss and the \emph{perceptual loss} \cite{zhang2018unreasonable} to supervise the results from both branches and the final prediction. The reconstruction loss $L_r$ is defined as
\begin{equation}
    \begin{aligned}
            L_r &= ||\hat{I}_{t}^{\text{syn}} - I_t||_1 + ||\hat{I}_{0 \rightarrow t}^{\text{direct}} - I_t||_1 + ||\hat{I}_{1 \rightarrow t}^{\text{direct}} - I_t||_1  \\
            &+ ||\hat{I}_{0 \rightarrow t}^{\text{proxy}} - I_t||_1 + ||\hat{I}_{1 \rightarrow t}^{\text{proxy}} - I_t||_1,
            % t_i^{*} &= (B-1)\frac{t_i-t_1}{t_N-t_1} \\
            % V(x,y,t) &= \sum_{i}p_ik_b(x-x_i)k_b(y-y_i)k_b(t-t_i^{*})\\
    \end{aligned}
\end{equation}
where $I_t$ is ground truth, and the \emph{perceptual loss} $L_{\text{perceptual}}$ follows \cite{niklaus2017video,tulyakov2021time} to constrain the similarity between prediction and ground truth from features outputted from the pretrained VGG-16 model.
% \begin{align}
%         L_{perceptual} &= \frac{1}{N} \sum_{i}^{N} ||\phi(I'_{t}) - \phi(I_1)||_2 + ||\phi(\hat{I}_{t}^{tran1}) - \phi(I_1)||_2  + \notag\\
%         & ||\phi(\hat{I}_{t}^{tran2}) - \phi(I_1)||_2 +  ||\phi(\hat{I}_{t}^{dire1}) - \phi(I_1)||_2 + ||\phi(\hat{I}_{t}^{dire2}) - \phi(I_1)||_2 
% \end{align}
%where $\phi$ denote \textmd{conv4\_3} features of ImageNet pre-trained VGG16 model \cite{simonyan2014very,jiang2018super}.
% Apart from the $L_1$ loss and the $L_{perceptual}$ loss, we perform the temporal consistency constraint between the outputs from direct synthesis branch and the transitional synthesis branch. The $L_{consistency}$ is defined as:
% \begin{equation}
%     \begin{aligned}
%             L_{consistency} = ||\hat{I}_{0 \rightarrow t}^{direct} - \hat{I}_{0 \rightarrow t}^{proxy}||_1 + ||\hat{I}_{1 \rightarrow t}^{direct} - \hat{I}_{1 \rightarrow t}^{proxy}||_1.
%     \end{aligned}
% \end{equation}
Combined, the synthesis loss $L_s$ is given by
\begin{equation}
\label{lambda}%
    \begin{aligned}
        L_s &= L_r + \lambda_1 L_{\text{perceptual}},
    \end{aligned}
\end{equation}
where $\lambda_1$ is a weight factor to balance both loss functions. 

% Figure environment removed


\subsection{Event-guided Recurrent Warping Module \label{3.4}}

In light of an event camera's ability to sense densely in time and their high sensitivity with respect to image motion, we propose to utilize event sequences to guide and refine the optical flow estimated from two boundary RGB frames $I_0$ and $I_1$ instead of computing optical flow solely from event sequences (cf. Time Lens \cite{tulyakov2021time}). As demonstrated in Fig. \ref{fig:warping}, we adopt a three-stage architecture to generate a warping-based intermediate video frame. In the first stage, we initialize bi-directional optical flow through an RGB-based optical flow prediction network. Subsequently, the second stage utilizes event sequences to update the previously estimated initial optical flow. Finally, the third stage again adds the two boundary RGB frames to re-refine the estimated bi-directional optical flow. Given the refined optical flow and the two boundary frames, we then utilize backward warping and fusion to obtain the interpolated intermediate frame.

\noindent \textbf{Initial RGB-based Estimation:} Given input frames $I_0$ and $I_1$, we first employ a pre-trained RGB-based optical flow estimation network (GMFlow \cite{xu2022gmflow}) to predict bi-directional optical flow $F_{0 \rightarrow 1}$ and $F_{1 \rightarrow 0}$. Following \cite{jiang2018super}, we then approximate initial time-parametrized optical flow fields $\hat{F}_{t \rightarrow 0}^{\text{init}}$ and $\hat{F}_{t \rightarrow 1}^{\text{init}}$ through an interpolation inspired by linear fusion: 
\begin{equation}
    \begin{aligned}
        \hat{F}_{t \rightarrow 0}^{\text{init}} &= -(1-t)t{F}_{0 \rightarrow 1} + t^2F_{1 \rightarrow 0} \\
        \hat{F}_{t \rightarrow 1}^{\text{init}} &= (1-t)^2{F}_{0 \rightarrow 1} - t(1-t)F_{1 \rightarrow 0}.
    \end{aligned}
\end{equation}
Given that RGB cameras sample sparsely in time, this simple linear interpolation based temporal parametrization of the optical flow is approximate and leads to obvious limitations when modeling complicated non-linear motion. Hence, we further propose to utilize event sequences for refinement.
%and demonstrates messy artifacts with small motion. Hence, we further propose to utilize event sequences to update the coarse optical flow.\\

\noindent \textbf{Event-based Update:} Given initial optical flow results $\hat{F}_{t \rightarrow 0}^{\text{init}}$ and $\hat{F}_{t \rightarrow 1}^{\text{init}}$, we utilize events $\mathbb{E}_{t \rightarrow 0}$ and $\mathbb{E}_{1 \rightarrow t}$ (computed using the reversing strategy mentioned in Sec. \ref{3.3}) to learn event-guided residual optical flows through a neural network $g_1$:
\begin{equation}
    \begin{aligned}
        \Delta \hat{F}_{t \rightarrow 0}^{\text{event}} &= g_1(\hat{F}_{t \rightarrow 0}^{\text{init}}, \mathbb{E}_{t \rightarrow 0})\\
        \Delta \hat{F}_{t \rightarrow 1}^{\text{event}} &= g_1(\hat{F}_{t \rightarrow 1}^{\text{init}}, \mathbb{E}_{t \rightarrow 1}).
    \end{aligned}
\end{equation}
Updated optical flows are then given by:
\begin{equation}
    \begin{aligned}
        \hat{F}_{t \rightarrow 0}^{\text{update}} &= \hat{F}_{t \rightarrow 0}^{\text{init}} + \Delta \hat{F}_{t \rightarrow 0}^{\text{event}}\\
        \hat{F}_{t \rightarrow 1}^{\text{update}} &= \hat{F}_{t \rightarrow 1}^{\text{init}} + \Delta \hat{F}_{t \rightarrow 1}^{\text{event}}.
    \end{aligned}
\end{equation}
Our approach improves the use of event sequences to estimate optical flow when compared against \cite{tulyakov2021time}, which directly yields optical flow solely from events. The high sensitivity of event cameras leads to an abundance of events within short time intervals. Accumulation of substantial noise accompanied by the spatially sparse distribution of events complicates the estimation of optical flow purely based on event sequences. By recording the high-frequent brightness changes in the image, event cameras inherently serve better to sense instantaneous motion rather than to reveal pixel-level invariant correspondences based on accumulated images. We therefore propose the alternative of utilizing the rich motion information of event sequences to guide and constrain the optical flow estimated from two RGB frames.
%Although directly computes optical flow from pure events is much infeasible, we could use 
%Recording high-frequency lightness changes of the scene, event cameras intrinsically serves better as a reflection of the dynamic motion instead of being used as two accumulated images to find invariant correspondence.
%We here utilize the rich motion information from high-frequency event camera to calibrate the misestimated optical flow from RGB images. Therefore, our approach stays more close to event camera's intrinsic properties and better grasp the advantages of event sequences.
%  Similar to the VIO (IMU+RGB) system in SLAM research, IMU provides motion signals with high frequency but also high signal-to-noise ratio, while RGB cameras record visual information steadily but with low frequency. They adopt a tightly-coupled strategy to compensate the high-frequency information and low-frequency information with each other.

\noindent \textbf{Refinement and Backward Warping.} The updated optical flow fields still perform poorly along motion boundaries where non-smoothness in the optical flow fields occurs \cite{jiang2018super}. Motivated by \cite{li2020video}, we concatenate flow fields $\hat{F}_{t \rightarrow 0}^{\text{update}}$ and $\hat{F}_{t \rightarrow 1}^{\text{update}}$ with boundary RGB frames $I_0$ and $I_1$ and feed them into a neural network $g_2$ to compute residual optical flows $\Delta \hat{F}_{t \rightarrow 0}$ and $\Delta \hat{F}_{t \rightarrow 1}$:
\begin{equation} 
    \begin{aligned}
        \Delta \hat{F}_{t \rightarrow 0}, \Delta \hat{F}_{t \rightarrow 1} &= g_2(\hat{F}_{t \rightarrow 0}^{\text{update}}, \hat{F}_{t \rightarrow 1}^{\text{update}}, I_0, I_1).\\
    \end{aligned}
\end{equation}
Next, we compute the re-refined optical flow field using
\begin{equation}
    \begin{aligned}
        \hat{F}_{t \rightarrow 0}^{\text{refine}}  &= \hat{F}_{t \rightarrow 0}^{\text{update}} + \Delta \hat{F}_{t \rightarrow 0} \\
        \hat{F}_{t \rightarrow 1}^{\text{refine}}  &= \hat{F}_{t \rightarrow 1}^{\text{update}} + \Delta \hat{F}_{t \rightarrow 1}.
    \end{aligned}
\end{equation}
To conclude, we backward warp \cite{jiang2018super,bao2019depth,park2020bmbc} RGB frames $I_0$ and $I_1$ using the refined optical flow fields $\hat{F}_{t \rightarrow 0}^{\text{refine}}$ and $\hat{F}_{t \rightarrow 1}^{\text{refine}}$ through differentiable interpolation \cite{jaderberg2015spatial}, respectively. This generates the two frames $\hat{I}_{t \rightarrow 0}^{\text{warp}}$ and $\hat{I}_{t \rightarrow 1}^{\text{warp}}$ at timestamp $t$, which we feed into a warp fusion network to yield the intermediate frame $\hat{I}^{\text{warp}}_t$.

\noindent \textbf{Loss Function.} The \emph{$\ell_1$} loss is employed to supervise the interpolated results both before and after the fusion module. More specifically, the warping loss $L_w$ is defined as:
\begin{equation}
    \begin{aligned}
        L_w &= ||\hat{I}^{\text{warp}}_t - I_t||_1 + ||\hat{I}_{t \rightarrow 0}^{\text{warp}} - I_t||_1 + ||\hat{I}_{t \rightarrow 1}^{\text{warp}} - I_t||_1,
    \end{aligned}
\end{equation}
where $I_t$ is ground truth.

\subsection{Attention-based Averaging Module\label{3.5}}
% - 介绍下Attention/Fusion Strategy
% - 用到timelens思想 引用一下
% - 该板块loss
%Strategies to combine intermediate candidates from different modules through a synthesis neural network has been successful in recent video interpolation algorithms \cite{niklaus2020softmax,jiang2018super,park2020bmbc}, and \cite{tulyakov2021time} also takes a similar attention-based averaging module to blend results from warping and synthesis module. 

Our \emph{Proxy-guided Synthesis} Module is flow-free. It shows an improved ability in handling challenging situations such as fast non-linear motion, illumination changes or new object occurrences. The reason is attributed to the auxiliary visual information provided by the dynamic information in the form of events \cite{tulyakov2021time}. However, given that the synthesis module synthesizes new frames directly from events, it has defects along edges caused by noise in the events and insufficient sensitivity in low-texture regions. The \emph{Event-guided Recurrent Warping} module complements the behavior in these regions, but relies on the brightness constancy assumption, which has limited performance in the before mentioned challenging situations. We add the \emph{Attention-based Averaging} module to compensate for individual weaknesses and combine the advantages of the synthesis and warping modules \cite{tulyakov2021time}. Similar to \cite{niklaus2020softmax,jiang2018super,park2020bmbc,tulyakov2021time}, the \emph{Attention-based Averaging} module takes the output frames of the \emph{Proxy-guided Synthesis} module $\hat{I}^{\text{syn}}_t$ and the \emph{Event-guided Recurrent Warping} module $\hat{I}^{\text{warp}}_t$, and learns weights to blend the results in a pixel-wise fashion and yield the final interpolated result $\hat{I}_t$ with reduced distortions and motion blur. We adopt the same loss functions as in Sec. \ref{3.3} to supervise the final result $\hat{I}_t$.

% the \emph{Attention-based Averaging Module} learns weights to optimally combine the four above modules, blending the results to overcome the demerits of different methods and to marry the advantages of synthesis-based, flow-based and flow-agnostic schemes.   

% Since this is a flow-free module without the assumption of brightness constancy, it could naturally avoid the pre-mentioned drawbacks of warping-based methods and could better handle challenging situations like changes of luminance and new object occurance on account of the intrinsic merits of event cameras and the auxiliary visual information provided by asynchronous events. However, the fusion module has defects of distortion of visual edges caused by noise of events and insufficient triggered events in low-texture regions, which we aim to utilize below modules to compensate.

% Figure environment removed 

\section{Experiments}

% - 我们在真实、虚拟的数据集均进行测试
% - 衡量的metric
% Note that, we segment each event sequences $\mathbb{E}_{\tau_1 \rightarrow \tau_2}$ to 6 time bins, as we empirically discovered that too few or many bins would either generate sparse representations or lose rich temporal information.
\subsection{Experimental Setup}

\noindent \textbf{Datasets.} We use the Vimeo90K Septuplet dataset \cite{xue2019video} for training. Following the same setting as \cite{tulyakov2021time}, events sequences are synthetically generated using an event simulator-based \cite{rebecq2018esim} video-to-event method \cite{gehrig2020video}. The resolution of all frames and events is $448 \times 256$. We perform the quantitative evaluation on standard VFI datasets to verify our method's robustness and generalization ability. We do not conduct fine-tuning for each dataset. Specifically, we test on datasets with frames and synthetic events constructed as in \cite{tulyakov2021time}, which includes the Vimeo90K Triplet testing set \cite{xue2019video} and Middlebury \cite{baker2011database}. Moreover, again adhering to the evaluation method used in \cite{tulyakov2021time}, we perform experiments on datasets with frames and real events, including the High Quality Frames (HQF) dataset \cite{stoffregen2020reducing} and the High Speed Event-RGB (HS-ERGB) dataset \cite{tulyakov2021time}.

\noindent \textbf{Baselines.} We compare our method against state-of-the-art video frame interpolation methods, which can be divided into two categories. The frame-based category contains DAIN \cite{bao2019depth}, RRIN \cite{li2020video}, SuperSlomo \cite{jiang2018super}, BMBC \cite{park2020bmbc}, ABME \cite{park2021asymmetric} and FLAVR \cite{kalluri2020flavr}. And the event-based category contains Time Lens \cite{tulyakov2021time} and Time Lens++ \cite{tulyakov2022time}. 
% Since currently the training set of the HS-ERGB dataset \cite{tulyakov2021time} is not publicly available, both our algorithm and the baselines are not fine-tuned on real events.
%For all the baselines, we use the pre-trained models provided by the authors without fine-tuning for each dataset.\\

\noindent \textbf{Evaluation Metrics.} We adopt signal-to-noise ratio (PSNR), structural similarity (SSIM) and LPIPS to evaluate the interpolation quality.
% We furthermore indicate the average PSNR and SSIM scores of all predicted frames for multi-frame interpolation.

\noindent \textbf{Training Details.} We train the Proxy-guided Synthesis Module, Event-guided Recurrent Warping Module, and Attention-based Averaging Module for 40, 40 and 10 epochs, respectively. We train the first two modules separately and freeze their weights when training the third module. We use the Sintel checkpoint of GMFlow \cite{xu2022gmflow}, and is freezed during training. We use Adam optimizer with $\beta_1=0.9$ and $\beta_2=0.99$, and start with an initial learning rate of $1 \times 10^{-4}$. Cosine Annealing is employed to decay the learning rate. %No data augmentation is adopted. 
Regarding hyper-parameters, we empirically set $\lambda_1 = 1$ in Eq. (\ref{lambda}) in Sec. \ref{3.3}. We choose neural networks $f_1$, $f_2$ in Sec. \ref{3.3} and $g_1$, $g_2$ in Sec. \ref{3.4} as skip-connected Hourglass U-Net architectures \cite{jiang2018super,tulyakov2021time}.
% Experiments are conducted on 8 NVIDIA TITAN V GPUs.



%pure image-based warping module versus event-guided warping module

\begin{table*}[ht!]
\setlength{\belowcaptionskip}{-0.5cm}
\scriptsize
\centering
\setlength{\tabcolsep}{1.73mm}
\begin{tabular}{lccccrcccccccc}
\hline
           & \multicolumn{7}{c}{(a) Synthetic Dataset}                                 & \multicolumn{1}{l}{} & \multicolumn{5}{c}{(b) Real-world Dataset}                \\  \cline{2-8} \cline{10-14} 
           & Triplet \cite{xue2019video}     &  & \multicolumn{2}{c}{Septuplet \cite{xue2019video}} &  & \multicolumn{2}{c}{Middlebury \cite{baker2011database}} &                      & \multicolumn{2}{c}{HQF \cite{stoffregen2020reducing}}   &  & \multicolumn{2}{c}{HS-ERGB (close) \cite{tulyakov2021time}} \\ \cline{2-2} \cline{4-5} \cline{7-8} \cline{10-11} \cline{13-14} 
           & x2          &  & x2            & x4            &  & x2             & x4            &                      & x2          & x4          &  & x6            & x8            \\ \cline{1-14}  
SuperSlomo \cite{jiang2018super} & 33.44/0.951 &  & 33.96/0.943   & 29.44/0.888   &  & 29.68/0.876    & 26.42/0.819   &                      & 28.76/0.861 & 25.54/0.761 &  & 28.35/0.788   & 27.27/0.755   \\
RRIN \cite{li2020video}       & 34.68/0.962 &  & 35.56/0.954   & 29.41/0.891   &  & 31.17/0.894    & 27.28/0.841   &                      & 29.76/0.874 & 26.11/0.778 &  & 28.70/0.813   & 27.44/0.800   \\
BMBC \cite{park2020bmbc}       & 35.09/0.963 &  & 35.48/0.949   & 30.33/0.897   &  & 30.71/0.889    & 26.45/0.821   &                      & \textcolor{blue}{30.74}/0.875 & 27.01/0.781 &  & 29.32/0.821   & 27.89/0.808   \\
ABME \cite{park2021asymmetric}       & 36.22/\textcolor{red}{0.969} &  & 36.53/0.955   & -             &  & 31.66/0.900    & -             &                      & 30.58/0.880 & -           &  & -   & -             \\
DAIN \cite{bao2019depth}       & 34.70/0.964 &  & 35.29/0.954   & 29.87/0.900   &  & 30.90/0.896    & 26.65/0.831   &                      & 29.82/0.875 & 26.10/0.782 &  & 29.03/0.807   & 28.50/0.801   \\
Time Lens \cite{tulyakov2021time}   & \textcolor{blue}{36.31}/0.962 &  & \textcolor{blue}{36.87}/\textcolor{blue}{0.960}   & \textcolor{blue}{35.58}/\textcolor{blue}{0.949}   &  & \textcolor{red}{33.27}/\textcolor{red}{0.929}    & \textcolor{red}{32.13}/\textcolor{red}{0.908}   &                      & 30.57/\textcolor{blue}{0.903} & \textcolor{red}{28.98}/\textcolor{red}{0.873} &  & \textcolor{blue}{32.19}/\textcolor{blue}{0.839}   & \textcolor{blue}{31.68}/\textcolor{blue}{0.835}    \\
\textbf{Ours}       & \textcolor{red}{36.56}/\textcolor{blue}{0.965} &  & \textcolor{red}{38.14}/\textcolor{red}{0.968}   & \textcolor{red}{36.34}/\textcolor{red}{0.960}   &  & \textcolor{blue}{32.51}/\textcolor{blue}{0.909}    & \textcolor{blue}{31.01}/\textcolor{blue}{0.886}   &                      & \textcolor{red}{31.75}/\textcolor{red}{0.910} & \textcolor{blue}{28.56}/\textcolor{blue}{0.850} &  & \textcolor{red}{33.21}/\textcolor{red}{0.847}   & \textcolor{red}{32.95}/\textcolor{red}{0.844}   \\ \hline
\end{tabular}



\caption{Quantitative results on (a) synthetic datasets: Vimeo 90k Triplet \cite{xue2019video}, Vimeo 90k Setuplet \cite{xue2019video} and Middlebury \cite{baker2011database}; (b) real datasets: HQF dataset \cite{stoffregen2020reducing}, and the close sequences of the HS-ERGB dataset \cite{tulyakov2021time} ($\times 2$, $\times 4$, $\times 6$, $\times 8$ denote skip 1, 3, 5, 7 frames, respectively). \textcolor{red}{Red} and \textcolor{blue}{blue} indicate the best and second best performance, respectively.}
\label{tab:quanti}%
\end{table*}

\begin{table}
\setlength{\belowcaptionskip}{-0.7cm}
\tiny
        \centering
        \renewcommand{\tabcolsep}{2pt}
        \begin{tabular}{c|cc}
        \toprule
        \textbf{Method}        & \textbf{PSNR$\uparrow$} & \textbf{LPIPS$\downarrow$} \\ 
        \midrule
        FLAVR \cite{kalluri2020flavr} & 27.42 &0.031 \\
        SuperSlomo \cite{jiang2018super}      & 30.05         & 0.103        \\
        Timelens \cite{tulyakov2021time}      & 33.48       & 0.017         \\ 
        Timelens++ \cite{tulyakov2022time} &33.09 &0.016 \\
        \midrule
        \textbf{Ours} &\textbf{33.53} &\textbf{0.015} \\
        \bottomrule
        \end{tabular}
        \caption{ Quantitative results on HS-ERGB dataset \cite{tulyakov2021time}. We conduct $\times$8 interpolation, and average the score over close and far subset.}
        %For this dataset, we average the score from close and far subset.
        \label{ablat:hsergb}
\end{table}

    %proving that the improved performance of our warping-based module comes from the proposed usage of event information, instead of the initial results from an image-based optical flow network. 
    % Also in Fig. \ref{fig:flow ablation}, our estimated flow (d) reserves more detailed information like motion details of face than flow w/o events.
% \end{itemize}
% \begin{table*}[ht!]
% \scriptsize
%   \centering
%   \caption{Quantitative results on Vimeo 90k Triplet \cite{xue2019video}, Vimeo 90k Setuplet \cite{xue2019video} and Middlebury \cite{baker2011database} datasets ($\times 2$, $\times 4$ denote skip 1 frame and 3 frames separately). \textcolor{red}{Red} and \textcolor{blue}{Blue} indicate the best and second best performance respectively.}
%     \begin{tabular}{lccccrcc}
%     \toprule
%     \multirow{2}[4]{*}{} & Triplet  &       & \multicolumn{2}{c}{Septuplet} &       & \multicolumn{2}{c}{Middlebury} \\
% \cmidrule{2-2}\cmidrule{4-5}\cmidrule{7-8}          & x2    &       & x2    & x4    &       & x2    & x4 \\
%     \midrule
%     SuperSlomo \cite{jiang2018super}& 33.44/0.951 &       & 33.96/0.943 & 29.44/0.888 &       & 29.68/0.876 & 26.42/0.819 \\
%     RRIN \cite{li2020video}  & 34.68/0.962 &       & 35.56/0.954 & 29.41/0.891 &       & 31.17/0.894 & 27.28/0.841 \\
%     BMBC \cite{park2020bmbc} & 35.09/0.963 &       & 35.48/0.949 & 30.33/0.897 &       & 30.71/0.889 & 26.45/0.821 \\
%     ABME \cite{park2021asymmetric} & \textcolor{blue}{36.22}/\textcolor{red}{0.969} &       & 36.53/0.955 & - &       & 31.66/0.900 & - \\
%     DAIN \cite{bao2019depth} & 34.70/\textcolor{blue}{0.964} &       & 35.29/0.954 & 29.87/0.900 &       & 30.90/0.896 & 26.65/0.831 \\
%     TimeLens\cite{tulyakov2021time} & \textcolor{red}{36.31}/0.962 &       & \textcolor{blue}{36.87}/\textcolor{blue}{0.960} & \textcolor{blue}{35.58}/\textcolor{blue}{0.949} &       & \textcolor{red}{33.27}/\textcolor{red}{0.929} & \textcolor{red}{32.13}/\textcolor{red}{0.908} \\
%     \textbf{Ours} & 36.06/0.962 &       & \textcolor{red}{38.14}/\textcolor{red}{0.968} & \textcolor{red}{36.34}/\textcolor{red}{0.960} &       & \textcolor{blue}{32.51}/\textcolor{blue}{0.909} & \textcolor{blue}{31.01}/\textcolor{blue}{0.886} \\
%     \bottomrule
%     \end{tabular}%
%   \label{tab:syn}%
% \end{table*}%
% 衡量各部分组成,以及具体设置的不同表现
% - 三个部分各自的表现
% - Synthesis:切一刀、两刀、不切;
% - Warping:没有event 只用raft效果
\subsection{Comparisons Against State-of-the-art Methods}
%\noindent \textbf{Single Frame Interpolation.} 

\noindent \textbf{Evaluations on datasets with synthetic events.}
Table \ref{tab:quanti} shows that our method outperforms both frame-based and event-based state-of-the-art algorithms on the Vimeo90K Septuplet dataset \cite{xue2019video} in terms of both single-frame interpolation and multi-frame interpolation by a significant margin. 
%DUET sets new \emph{state-of-the-art} performance on this dataset with a PSNR of 38.14 and SSIM of 0.967 for single frame interpolation. 
For frame-based methods, our approach exceeds SuperSlomo\cite{jiang2018super} and RRIN\cite{li2020video} which approximate bi-directional optical flow, and ABME \cite{park2021asymmetric} which estimates bilateral motion fields. This demonstrates the advantage of the auxiliary visual information introduced by the highly dynamic event cameras. Our model also outperforms the event-based method Time Lens \cite{tulyakov2021time}, which justifies the superiority the design choices we made in our synthesis and warping modules. To verify the generalization ability of our method, we also evaluate our method on Middlebury \cite{baker2011database} and Vimeo90K (triplet) \cite{xue2019video}, where our method also demonstrates favorable performance.
%Middlebury is a relatively challenging testset with complex motions and dynamic scenes. Whereas, our result of PSNR 32.51 shows comparable performance against baseline models. This demonstrates that our propose method is robust to non-linear motion and complex scenes. 
Fig. \ref{fig:quali} illustrates qualitative comparisons on the Middlebury dataset \cite{baker2011database}. As can be observed, our method maintains good performance while baselines produce blurry results or noticeable artifacts.\\
%DUET also has better performance than \cite{kalluri2020flavr}, which is a flow-agnositic method and also use extra information provided by a temporal window of other frames, demonstrating the superiority of our usage of events within two frames without additional inputs for other frames.
% \begin{table*}[ht!]
% \scriptsize
%   \centering
%   \caption{Quantitative results on High Quality Frames (HQF) DAVIS240 dataset \cite{stoffregen2020reducing} and the close planar sequences of High speed Event and RGB camera (HS-ERGB) dataset \cite{tulyakov2021time} ($\times 2$, $\times 4$, $\times 6$, $\times 8$ denote skip 1, 3, 5, 7 frames separately). \textcolor{red}{Red} and \textcolor{blue}{Blue} indicate the best and second best performance respectively.}
%     \begin{tabular}{lccccc}
%     \toprule
%     \multirow{2}[4]{*}{} & \multicolumn{2}{c}{HQF} &       & \multicolumn{2}{c}{HS-ERGB(close)} \\
% \cmidrule{2-3}\cmidrule{5-6}          & x2    & x4    &       & x6    & x8 \\
%     \midrule
%     SuperSlomo \cite{jiang2018super} & 28.76/0.861 & 25.54/0.761 &       & 28.35/0.788 & 27.27/0.755 \\
%     RRIN \cite{li2020video} & 29.76/0.874 & 26.11/0.778 &       & 28.70/0.813 & 27.44/0.800 \\
%     BMBC \cite{park2020bmbc} & \textcolor{blue}{30.74}/0.875 & 27.01/0.781 &       & 29.32/0.821 & 27.89/0.808 \\
%     ABME \cite{park2021asymmetric} & 30.58/0.880 & - &       & - & - \\
%     DAIN \cite{bao2019depth} & 29.82/0.875 & 26.10/0.782 &       & 29.03/0.807 & 28.50/0.801 \\
%     TimeLens \cite{tulyakov2021time} & 30.57/\textcolor{blue}{0.903} & \textcolor{red}{28.98}/\textcolor{red}{0.873} &       & \textcolor{blue}{32.19}/\textcolor{blue}{0.839} & \textcolor{blue}{31.68}/\textcolor{blue}{0.835} \\
%     \textbf{Ours} & \textcolor{red}{31.75}/\textcolor{red}{0.910} & \textcolor{blue}{28.56}/\textcolor{blue}{0.850} &       & \textcolor{red}{32.69}/\textcolor{red}{0.847} & \textcolor{red}{32.15}/\textcolor{red}{0.844} \\
%     \bottomrule
%     \end{tabular}%
%   \label{tab:real}%
% \end{table*}%
\noindent \textbf{Evaluations on datasets with real events.}
 To evaluate the adaptability of our method to real events, we assess performance of both single and multi-frame interpolation on the HQF \cite{stoffregen2020reducing} and close sequences of HS-ERGB datasets \cite{tulyakov2021time}. Results are summarized in Table \ref{tab:quanti}. The results on datasets with real events are consistent with the results obtain for synthetic events, although the synthetic-to-real gap necessarily causes a slight drop in overall performance with respect to the results we obtained on synthetic datasets. Furthermore, we compare with recently proposed Timelens++ \cite{tulyakov2022time} on HS-ERGB dataset \cite{tulyakov2021time} in Table \ref{ablat:hsergb}, where we conduct $\times8$ interpolation and average the score over close and far subsets. The results are consistent with those obtained for single-frame interpolation, and our method keeps outperforming the state-of-the-art. Finally, we captured demo videos of daily scenes with complex motion using a DAVIS 346 event camera, please refer to the supplemental video.
 
\begin{table*}[htbp]
\tiny
\vspace{0.25cm}
    \centering
    \begin{subtable}{0.19\linewidth}
        \centering
        \begin{tabular}{c|cc}
        \toprule
        \textbf{Module}                & \textbf{PSNR$\uparrow$} & \textbf{SSIM$\uparrow$} \\ \hline
        Warping      & 35.77         & 0.964         \\
        Synthesis & 37.71         & 0.959         \\
        Averaging      & 38.14         & 0.968         \\ \hline
        \end{tabular}
        \caption{}
        \label{ablat:a}
    \end{subtable}
    %
    \begin{subtable}{0.19\linewidth}
        \centering
        \begin{tabular}{c|cc}
        \toprule
        \textbf{\# Proxies}        & \textbf{PSNR$\uparrow$} & \textbf{SSIM$\uparrow$} \\ \hline
        1       & 35.87         & 0.954         \\
        2  & 35.95         & 0.959         \\
        4        & 34.61         & 0.943         \\ \hline
        \end{tabular}
        \caption{}
        \label{ablat:b}
    \end{subtable}
    %
    \begin{subtable}{0.19\linewidth}
        \centering
        \begin{tabular}{c|cc}
        \toprule
        \textbf{Case}        & \textbf{PSNR$\uparrow$} & \textbf{SSIM$\uparrow$} \\ \hline
        only 1      & 35.87         & 0.954         \\
        1 \& 2     & 37.71         & 0.959         \\
        1 \&2 \& 4       & 37.78         & 0.956         \\ \hline
        \end{tabular}
        \caption{}
        \label{ablat:c}
    \end{subtable}
    %
    \begin{subtable}{0.19\linewidth}
        \centering
        \begin{tabular}{c|cc}
        \toprule
        \textbf{Case}        & \textbf{PSNR$\uparrow$} & \textbf{SSIM$\uparrow$} \\ \hline
        RGB-guided      & 33.74         & 0.938         \\
        event-guided       & 35.77         & 0.964        \\ \hline
        \end{tabular}
        \caption{}
        \label{ablat:d}
    \end{subtable}
    %
    \begin{subtable}{0.19\linewidth}
        \centering
        \begin{tabular}{c|cc}
        \toprule
        \textbf{Case}        & \textbf{PSNR$\uparrow$} & \textbf{SSIM$\uparrow$} \\ \hline
        w/o event      & 32.24         & 0.941         \\
        w event       & 35.77         & 0.964         \\ \hline
        \end{tabular}
        \caption{}
        \label{ablat:e}
    \end{subtable}
    \caption{\textbf{Ablation results.} Ablation studies of our method to evaluate (a) the performance of each module, (b) different event segmentation strategies (division into 1,2, or 4 proxies, respectively), (c) different fusion strategies, (d) RGB-guided warping vs event-guided warping , (e) the importance of event-based updating.}
    \label{tab:ablation}
    % Average results (PSNR/SSIM) are reported on Vimeo90K (Septuplet) test set \cite{xue2019video}.}
\end{table*}
% \begin{table*}[h!]
%  % \scriptsize
%     \centering
%     \begin{subtable}{0.31\linewidth}
%         \centering
%         \begin{tabular}{c|cc}
%         \toprule
%         \textbf{Module}                & \textbf{PSNR} & \textbf{SSIM} \\ \hline
%         Warping      & 35.77         & 0.964         \\
%         Synthesis & 37.71         & 0.959         \\
%         Averaging      & 38.14         & 0.968         \\ \hline
%         \end{tabular}
%         \caption{}
%         \label{ablat:a}
%     \end{subtable}
%     %
%     \begin{subtable}{0.31\linewidth}
%         \centering
%         \begin{tabular}{c|cc}
%         \toprule
%         \textbf{\# Proxies}        & \textbf{PSNR} & \textbf{SSIM} \\ \hline
%         1       & 35.87         & 0.954         \\
%         2  & 35.95         & 0.959         \\
%         4        & 34.61         & 0.943         \\ \hline
%         \end{tabular}
%         \caption{}
%         \label{ablat:b}
%     \end{subtable}
%     %
%     \begin{subtable}{0.31\linewidth}
%         \centering
%         \begin{tabular}{c|cc}
%         \toprule
%         \textbf{Case}        & \textbf{PSNR} & \textbf{SSIM} \\ \hline
%         only 1      & 35.87         & 0.954         \\
%         1 \& 2     & 37.71         & 0.959         \\
%         1 \&2 \& 4       & 37.78         & 0.956         \\ \hline
%         \end{tabular}
%         \caption{}
%         \label{ablat:c}
%     \end{subtable}
%     %
%     \begin{subtable}{0.48\linewidth}
%         \centering
%         \begin{tabular}{c|cc}
%         \toprule
%         \textbf{Case}        & \textbf{PSNR} & \textbf{SSIM} \\ \hline
%         RGB-guided      & 33.74         & 0.938         \\
%         event-guided       & 35.77         & 0.964        \\ \hline
%         \end{tabular}
%         \caption{}
%         \label{ablat:d}
%     \end{subtable}
%     %
%     \begin{subtable}{0.48\linewidth}
%         \centering
%         \begin{tabular}{c|cc}
%         \toprule
%         \textbf{Case}        & \textbf{PSNR} & \textbf{SSIM} \\ \hline
%         w/o event      & 32.24         & 0.941         \\
%         w event       & 35.77         & 0.964         \\ \hline
%         \end{tabular}
%         \caption{}
%         \label{ablat:e}
%     \end{subtable}
%     %
%     \caption{\textbf{Ablation results.} Ablation experiments for our method on (a) performance of each component, (b) different event segmentation strategies (divide into 1,2,4 proxies respectively), (c) different fusion strategies, (d) RGB-guided warping \vs event-guided warping , (e) importance of event update. Average results (PSNR/SSIM) are reported on Vimeo90K (Septuplet) test set \cite{xue2019video}.}
%     \label{tab:ablation}
% \end{table*}
% Figure environment removed
% Figure environment removed
 \subsection{Ablation Experiments}
We present a detailed ablation study of our method in Table. \ref{tab:ablation}. We analyze the contributions of the three key modules to the final interpolation and compare several architecture design choices to assess their effects. Ablations are conducted on the Vimeo90K Septuplet dataset \cite{xue2019video}.\\
\noindent \textbf{The impact of the different modules.} To analyze the contribution of our three modules, we show the interpolation quality of each module in Table \ref{ablat:a}. The result demonstrates that both our synthesis and warping module achieve good interpolation results. Besides, it shows that our attention-based averaging module successfully combines the advantages of the two previous components, resulting in an increase in PSNR and SSIM to 38.14 and 0.968, respectively. Fig. \ref{fig:component} visualizes the impacts of the different modules. The first row of Fig. \ref{fig:component} shows an example where our warping module yields blurry results due to insufficient ability to model overlapping motion. In contrast, our attention module predicts correct results using the better results of the synthesis module. The second row of Fig. \ref{fig:component} shows an example where our synthesis module generates blurry results due to event noise. However, the attention module again interpolates clear results by giving preference to the good predictions of the warping module. Furthermore, we present visualization results on the HS-ERGB dataset \cite{tulyakov2021time} in Fig. \ref{fig:zurich}. The interpolation error maps demonstrate that our attention-based averaging module effectively leverages the synthesis module's better results for nearby objects and the warping module's superior results for distant objects. Consequently, our approach achieves significantly fewer errors in the final output.\\

\noindent \textbf{Ablations of the Proxy-guided Synthesis Module}

    \noindent \textbf{\textit{The impact of different event segmentation strategies.}} 
    Table. \ref{ablat:b} reports results for different numbers of proxies dividing the original event sequence. As shown, we obtain best performance when the number of proxies is set to 2, followed by 1 and then 4. When the number of proxies is set to 1, we directly feed the whole event sequence into the neural network without any segmentation. The accumulated grid-based representation of the events results in a loss of the quasi-continuous, high-frequent temporal information provided by the original events, thus resulting in worse performance. When setting the number of proxies to four ($T=4$), the performance drops as the event representation becomes very sparse, thereby providing insufficient information for detailed frame synthesis. The ideal proxy number remains a function of the original video framerate. %As discussed in Sec. \ref{3.3}, we propose to segment events sequences into N time intervals to gradually synthesize the final frame in our synthesis-based module, and 
    %We settle our final choice as N=1 considering the discussed demerits and worse performance for either fewer or more divisions.
    % \\
    % \item 
    
    \noindent \textbf{\textit{The impact of different fusion strategies.}} Table. \ref{ablat:c} shows results of different fusion choices in the synthesis module. As discussed in Sec. \ref{3.3}, we propose to segment events into $T$ proxies. We compare the choices of: 1) only $T$ = 1, 2) fuse $T$ = 1 and $T$ = 2 from two individual branches, and 3) fuse $T$ = 1, $T$ = 2 and $T$ = 4 from three individual branches. Table \ref{ablat:c} shows that another improvement can be achieved by combining an additional branch of $T$ = 4, which however comes at an unjustifiable increase in computational cost during both training and testing.\\
    %but with the sacrifice of bigger model and longer training time.
% \end{itemize}

\noindent \textbf{Ablations of the Event-guided Warping Module}\\
\noindent \textbf{\textit{The impact of event-guided optical flow.}} To justify the core design of our \emph{Event-guided Recurrent Warping Module}, we check the alternative of using images to refine a purely event-based initial flow, and show the results in Table \ref{ablat:d}. 
    We first replace our image-based optical flow network with the pre-trained warping network of \cite{tulyakov2021time} (trained on the same dataset as ours), which estimates optical flow solely from events. 
    %The following procedures are similar to our original pipeline with the only difference that this time 
    We then use images to refine the estimated flow instead of the original \emph{Event Update} block. For the sake of a fair comparison, this alternative is trained for the same number of epochs than the original network. As shown in Table. \ref{ablat:d}, our method performs better than the image-guided event-based flow estimation, which confirms our design choice. 
    \noindent \textbf{\textit{The importance of the Event Update block.}} We evaluate the effectiveness of the \emph{Event Update} block in Sec. \ref{3.4} by training a warping-based module without event information. It directly uses the pre-trained GMFlow optical flow network \cite{xu2022gmflow} and skips the \emph{Event Update} block, while leaving the remaining architecture unchanged. Table. \ref{ablat:e} shows the results obtained by this variant, which are much worse than the original ones (3.53 dB drop in terms of PSNR). This confirms the importance of the \emph{Event Update} block.

\section{Conclusion}
The present work shows that a careful consideration of the inherent properties of event cameras in the design of neural architectures can help to improve prediction results, in this case for video frame interpolation. Event cameras only encode changes of intensity and polarity rather than absolute color intensities, and thus prove to be a better choice for RGB based optical flow prediction refinement rather than for direct optical flow prediction. Furthermore, in light of the quasi-continuous nature of event streams, we propose an incremental synthesis strategy that breaks down the global prediction into multiple simpler and equivalent short-term prediction steps. Extensive experiments on both synthetic and real-world datasets show that our method demonstrates favorable VFI performance. We believe that the ideas communicated in this work can be replicated towards alternative event-based tasks, thereby making a strong contribution towards the combined use of regular and dynamic vision sensors.

% \noindent\textbf{Limitation and discussion.} With the lack of training data for HS-ERGB dataset, we did not finetune our model on data with real events, which might limit our model's applicability to real data. Additionally, our event-based updated flow is initialized by pre-trained RAFT. An unsatisfactory initial estimation might lead to suboptimal final inter-frames flow estimation. We plan to improve  
{\small
\bibliographystyle{IEEEtran}
\bibliography{egbib}
}
% \section*{APPENDIX}

% Appendixes should appear before the acknowledgment.

% \section*{ACKNOWLEDGMENT}

% The preferred spelling of the word ÒacknowledgmentÓ in America is without an ÒeÓ after the ÒgÓ. Avoid the stilted expression, ÒOne of us (R. B. G.) thanks . . .Ó  Instead, try ÒR. B. G. thanksÓ. Put sponsor acknowledgments in the unnumbered footnote on the first page.
% \addtolength{\textheight}{-12cm}   % This command serves to balance the column lengths
%                                   % on the last page of the document manually. It shortens
%                                   % the textheight of the last page by a suitable amount.
%                                   % This command does not take effect until the next page
%                                   % so it should come on the page before the last. Make
%                                   % sure that you do not shorten the textheight too much.

% %%%%%%%%%%%%%%%%%%%%%%%%%%%%%%%%%%%%%%%%%%%%%%%%%%%%%%%%%%%%%%%%%%%%%%%%%%%%%%%%



% %%%%%%%%%%%%%%%%%%%%%%%%%%%%%%%%%%%%%%%%%%%%%%%%%%%%%%%%%%%%%%%%%%%%%%%%%%%%%%%%



% %%%%%%%%%%%%%%%%%%%%%%%%%%%%%%%%%%%%%%%%%%%%%%%%%%%%%%%%%%%%%%%%%%%%%%%%%%%%%%%%









\end{document}
