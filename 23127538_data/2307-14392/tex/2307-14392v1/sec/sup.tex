% \documentclass[10pt,twocolumn,letterpaper]{article}

% \usepackage{iccv}
% \usepackage{times}
% \usepackage{epsfig}
% \usepackage{graphicx}
% \usepackage{amsmath}
% \usepackage{amssymb}


% \usepackage{color}
% \usepackage{times}
% \usepackage{overpic}
% \usepackage{bm}
% \usepackage{tabu}
% \usepackage{bbding}
% \usepackage{multicol}
% \usepackage{multirow}
% \usepackage[table]{xcolor}

% % Include other packages here, before hyperref.

% % If you comment hyperref and then uncomment it, you should delete
% % egpaper.aux before re-running latex.  (Or just hit 'q' on the first latex
% % run, let it finish, and you should be clear).
% \usepackage[pagebackref=true,breaklinks=true,letterpaper=true,colorlinks,bookmarks=false]{hyperref}

% % \iccvfinalcopy % *** Uncomment this line for the final submission

% \def\iccvPaperID{3694} % *** Enter the ICCV Paper ID here
% \def\httilde{\mbox{\tt\raisebox{-.5ex}{\symbol{126}}}}

% % Pages are numbered in submission mode, and unnumbered in camera-ready
% \ificcvfinal\pagestyle{empty}\fi

\begin{appendix}

% %%%%%%%%% TITLE
% \title{Supplement of Human-centric Scene Understanding for 3D Large-scale Scenarios}

% \author{First Author\\
% Institution1\\
% Institution1 address\\
% {\tt\small firstauthor@i1.org}
% % For a paper whose authors are all at the same institution,
% % omit the following lines up until the closing ``}''.
% % Additional authors and addresses can be added with ``\and'',
% % just like the second author.
% % To save space, use either the email address or home page, not both
% \and
% Second Author\\
% Institution2\\
% First line of institution2 address\\
% {\tt\small secondauthor@i2.org}
% }


% \maketitle
% % Remove page # from the first page of camera-ready.
% \ificcvfinal\thispagestyle{empty}\fi


%%%%%%%%% ABSTRACT
\section{Implement details}


\subsection{Human-centric Instance Segmentation}
In HHOI module, the threshold  $\tau$ for sampling high confidence features is set to 0.8 and the number of sampled points $M=256$.
In Point-wise Prediction and Refinement process,
the loss can be formulated as following: 
$\mathcal{L} = \mathcal{L}_{\text {semantic}}+\mathcal{L}_{\text {offset}}+\mathcal{L}_{\text {class}}+\mathcal{L}_{\text {mask}}+\mathcal{L}_{\text{mask score}}.$
$$
L_{\text {semantic }}=\frac{1}{N} \sum_{i=1}^N \operatorname{CE}\left(\boldsymbol{s}_i, s_i^*\right),$$
$$
L_{\text {offset }}=\frac{1}{\sum_{i=1}^N \mathbb{I}_{\left\{\boldsymbol{p}_i\right\}}} \sum_{i=1}^N \mathbb{I}_{\left\{\boldsymbol{p}_i\right\}}\left\|\boldsymbol{o}_i-\boldsymbol{o}_i^*\right\|_1,
$$
$$
L_{\text {class }}=\frac{1}{K} \sum_{k=1}^K \mathrm{CE}\left(\boldsymbol{c}_k, c_k^*\right), 
$$
$$
L_{\text {mask }}=\frac{1}{\sum_{k=1}^K \mathbb{I}_{\left\{\boldsymbol{m}_k\right\}}} \sum_{k=1}^K \mathbb{I}_{\left\{\boldsymbol{m}_k\right\}} \mathrm{BCE}\left(\boldsymbol{m}_k, \boldsymbol{m}_k^*\right), 
$$
$$
\mathcal{L}_{\text{mask score}}=\frac{1}{\sum_{k=1}^{N_{g t}} \mathbb{I}_{\left\{iou_k\right\}}} \sum_{k=1}^{N_{g t}} \mathbb{I}_{\left\{iou_k\right\}}\left\|iou_k-iou_k^*\right\|_2
$$
where $*$ denotes the ground truth.



\subsection{Human-centric Action Recognition}
% % Figure environment removed
The input for action recognition is frames of large scene point cloud $P\in R^{N\times 4}$ with the 3D location and reflection intensity (x, y, z, r). We extend the length and width of bounding box obtained from human detector by $\Delta h$ and $\Delta w$ respectively, where $\Delta h$ and $\Delta w$ are both set to 0.2 meters. After cropping point clouds with bounding boxes, we use clustering algorithm to find k(k=3) nearest neighbors of the ego point cloud with their relative distances. Next, the point cloud of every single person will be normalized, and sampled by farthest point sample algorithm to n points(n=512). The features of k neighbours and ego will be extracted by HPFE simultaneously to get features of dimension $(k+1) \times c$, which will be input to ENFI afterwards.

In HPFE, we use set abstractions(SA) to down-sample R times on origin point clouds to fork R branches with different resolutions. R is set to 5 by default.
$$P_i \in R^{(n/2^r) \times (32*2^r)} r\in [1,...,R],i\in [1,...,L]$$
where $P_i$ is the feature dimension of R branches. Then we use different sampling radius for the R resolution branches, which are $0.05*(r+1),r\in [1,...,R]$, so that the receptive field of SA will expand with the improving of resolution. After that, we apply equal sampling for L times(L is set to 2) for all branches simultaneously. Finally, we down sample the features of the low-resolution channels to get five features of the same size, which will be fused together to get hierarchical fusion feature.



\section{Dataset details}
% Figure environment removed


\subsection{Object category for segmentation and detection}
We merge several categories which have low frequency of occurrence and similar geometry shapes in our dataset into a new class, and we also drop some category which only appear in training or testing set with low frequency. The merging list is shown in Table ~\ref{tab:merge}. The categories of objects after merging is 17 and the number of objects in each category is illustrated in Figure. \ref{fig:segcount}.
% Please add the following required packages to your document preamble:
% \usepackage{multirow}

\begin{table}[ht]
\centering
\caption{Object merging list. We merge the categories on the left into the category on the right.}
    \label{tab:merge}
    \setlength{\tabcolsep}{0.8mm}
\begin{tabular}{c|c}
\hline
banner,plank,paper,door,dog,megaphone,guitar  & \multicolumn{1}{c}{\multirow{2}{*}{other}}       \\
toy car,merry go round,car,tricycle,umbrella, & \multicolumn{1}{c}{}                             \\\hline
printer,podium                                &
cabinet                                          \\\hline
bicycle                                       & motorbike                                        \\\hline
{(}two-wheeled{)} {(} self-{)}balancing car   & scooter                                          \\\hline
flat car,stroller,perambulator &cart                                        \\\hline
rockery                                       & slide                                            \\\hline
stool                                         & chair                                            \\\hline
suitcase                                      & box                                              \\\hline
eraser,phone,cup,food,cellphone,red flag,     & \multicolumn{1}{c}{\multirow{4}{*}{obj in hand}} \\
cap,camera,sponge,projector,balloon,          & \multicolumn{1}{c}{}                             \\
plush toy,toy wings,clothes,flower,           & \multicolumn{1}{c}{}                             \\
badminton rocket,handbag,plastic bag,         &
\\\hline\multicolumn{1}{c}{}                            
\end{tabular}
\end{table}


\subsection{Action category for  recognition and detection}
 It is common for a person to perform multiple actions simultaneously. To prioritize these actions, we assign each action to a numerical priority value. We then merge these prioritized actions into 12 categories based on their similarity and frequency of occurrence. Actions with low frequency are dropped to ensure a manageable number of categories. To illustrate this process, we provide a merging Table ~\ref{tab:action_merge} that maps each prioritized action to its corresponding category. The number of each action after the merging process is shown in Figure.~\ref{fig:action_count}.



% Figure environment removed




% Please add the following required packages to your document preamble:
% \usepackage{multirow}
\begin{table}[ht]
\caption{Detailed action priority and merge information.}
\centering
\label{tab:action_merge}
\resizebox{0.85\linewidth}{!}{   
\begin{tabular}{l|c|l}
\hline
Merged   Action                      & priority            & Original Action                    \\ \hline
\multirow{7}{*}{Lift}                & \multirow{7}{*}{0}  & taking clothes                     \\ \cline{3-3} 
                                     &                     & lifting a   plastic bag            \\ \cline{3-3} 
                                     &                     & lifting a bag                      \\ \cline{3-3} 
                                     &                     & taking   things/exchanging items   \\ \cline{3-3} 
                                     &                     & lifting   things                   \\ \cline{3-3} 
                                     &                     & lifting something                  \\ \cline{3-3} 
                                     &                     & moving planks                      \\ \hline
\multirow{3}{*}{Carry}               & \multirow{3}{*}{1}  & carrying other things              \\ \cline{3-3} 
                                     &                     & carrying a bag                     \\ \cline{3-3} 
                                     &                     & carrying bags                      \\ \hline
Move                                 & 2                   & moving boxes                       \\ \hline
\multirow{10}{*}{Pull\_Push}          & \multirow{10}{*}{3} & pulling a suitcase                 \\ \cline{3-3} 
                                     &                     & pulling a chair                    \\ \cline{3-3} 
                                     &                     & pulling a flatcar                  \\ \cline{3-3} 
                                     &                     & pushing a cart                     \\ \cline{3-3} 
                                     &                     & pushing a stroller                 \\ \cline{3-3} 
                                     &                     & pushing a flatcar                  \\ \cline{3-3} 
                                     &                     & pushing a table                    \\ \cline{3-3} 
                                     &                     & holding a spring car               \\ \cline{3-3} 
                                     &                     & pushing something                  \\ \cline{3-3} 
                                     &                     & pushing something                  \\ \hline
\multirow{14}{*}{Sit}                & \multirow{5}{*}{4}  & riding a bicycle                   \\ \cline{3-3} 
                                     &                     & riding an electric bicycle         \\ \cline{3-3} 
                                     &                     & riding a tricycle                   \\ \cline{3-3} 
                                     &                     & riding on the carousel             \\ \cline{3-3} 
                                     &                     & sitting in a spring car            \\ \cline{2-3} 
                                     & \multirow{9}{*}{13} & crouching                          \\ \cline{3-3} 
                                     &                     & sitting on the ground              \\ \cline{3-3} 
                                     &                     & crouching or sitting on the ground \\ \cline{3-3} 
                                     &                     & sitting on the ground              \\ \cline{3-3} 
                                     &                     & sitting                            \\ \cline{3-3} 
                                     &                     & sitting on a trunk                 \\ \cline{3-3} 
                                     &                     & sitting in a chair                 \\ \cline{3-3} 
                                     &                     & sitting on the stool               \\ \cline{3-3} 
                                     &                     & squatting                          \\ \hline
\multirow{5}{*}{Scooter-BalanceBike} & \multirow{5}{*}{5}  & riding a two-wheel balance car     \\ \cline{3-3} 
                                     &                     & riding a balance car               \\ \cline{3-3} 
                                     &                     & riding an electric skateboard      \\ \cline{3-3} 
                                     &                     & riding a skateboard                \\ \cline{3-3} 
                                     &                     & standing on a trolley              \\ \hline
\multirow{8}{*}{Hum-Inter}           & \multirow{8}{*}{6}  & hugging                            \\ \cline{3-3} 
                                     &                     & pulling a baby                     \\ \cline{3-3} 
                                     &                     & being hold by someone else         \\ \cline{3-3} 
                                     &                     & taking a baby                      \\ \cline{3-3} 
                                     &                     & holding the baby                   \\ \cline{3-3} 
                                     &                     & Being held by someone else         \\ \cline{3-3} 
                                     &                     & carrying a baby                    \\ \cline{3-3} 
                                     &                     & being carry by someone else        \\ \hline
\multirow{3}{*}{Fitness}             & \multirow{3}{*}{7}  & fitness with a twister             \\ \cline{3-3} 
                                     &                     & fitness with a elliptical trainer  \\ \cline{3-3} 
                                     &                     & fitness with a stepper             \\ \hline
\multirow{6}{*}{Entertain}           & \multirow{6}{*}{8}  & climbing the swing                 \\ \cline{3-3} 
                                     &                     & climbing slide                     \\ \cline{3-3} 
                                     &                     & holding the slide                  \\ \cline{3-3} 
                                     &                     & sliding                            \\ \cline{3-3} 
                                     &                     & playing seesaw                     \\ \cline{3-3} 
                                     &                     & sitting in a cavern                \\ \hline
\multirow{2}{*}{Sports}              & 9                   & playing basketball                 \\ \cline{2-3} 
                                     & 10                  & playing badminton                  \\ \hline
\multirow{5}{*}{Standing}            & 11                  & taking the escalator               \\ \cline{2-3} 
                                     & 14                  & running                            \\ \cline{2-3} 
                                     & \multirow{3}{*}{15} & walking                            \\ \cline{3-3} 
                                     &                     & standing                           \\ \cline{3-3} 
                                     &                     & leaning                            \\ \hline
Bending$\_$Over                         & 12                  & bending over                       \\ \hline
\multirow{8}{*}{Other}               & \multirow{8}{*}{16} & cabinet interaction                \\ \cline{3-3} 
                                     &                     & standing on the stool              \\ \cline{3-3} 
                                     &                     & getting in the car                 \\ \cline{3-3} 
                                     &                     & getting out of the car             \\ \cline{3-3} 
                                     &                     & driving a toy car                  \\ \cline{3-3} 
                                     &                     & lying                              \\ \cline{3-3} 
                                     &                     & writing on the blackboard          \\ \cline{3-3} 
                                     &                     & …                                  \\ \hline
\end{tabular}
}
\end{table}




\section{More Experiment}
We take pre-trained CenterPoint as the 3D Detector and add a feature extractor for cropped individual point cloud for the second-stage action recognition comparison, the detailed comparison result is shown in Table ~\ref{tab:action_exp1}. Our method outperforms others in most of categories. The comparison result which uses 3D bounding boxes from ground truth is shown in Table ~\ref{tab:action_abcde_ground}. We further provide
action visualization in Figure ~\ref{fig:vis-action}.
% \begin{wrapfigure}{r}{3.5cm}
% Figure environment removed

\begin{table*}[ht]
\caption{Detailed comparison results of action recognition on HuCenLife. All methods are based on the same 3D detector (centerpoint) for fair evaluation.}
\label{tab:action_exp1}

\setlength{\tabcolsep}{1.3mm}
\resizebox{\linewidth}{!}{     
\begin{tabular}{c|c|c|c|c|c|c|c|c|c|c|c|c|c|c|c}
\hline
Method           & Lift         & Carry         & Move         & Pull\_Push     & Sit           & Scooter-BalanceBike & Hum-Inter    & Fitness       & Entertain     & Sports        & Bend-Over    & Standing      & mAP         & mRecall     & mPrec         \\ \hline
Baseline         & 0.5          & 1.6           & 0.2           & 13.8          & 2.2           & 21.8                & 0            & 0             & 2.4           & 6.9           & 0.1          & \textbf{38.3} & 7.3         & 14.6        & 19.9          \\ \hline
ViT              & 4.1          & 1.6           & 5.1           & 8.2           & 0.6           & 4.7                 & 0.1          & \textbf{27.3} & 6             & \textbf{46.6} & 0.1          & 8.3           & 9.4         & 23.1        & 19.9          \\ \hline
PVT              & 1.4          & 10.5          & 8.9           & 21            & 16.8          & 56.8                & 5.9          & 1.7           & 1             & 25.1          & 4.3          & 5.2           & 13.2        & 30.5        & 19.8          \\ \hline
PointNet         & 1.6          & 3.1           & 4.6           & 20.1          & 24.4          & 22.3                & 0.7          & 0.6           & 0.6           & 17.1          & 1.5          & 4.2           & 8.4         & 26.3        & 15.5          \\ \hline
PointNet++       & 3.6          & 25.3          & 10.6          & 21            & 25.5          & 51                  & 3.5          & 2.7           & 3.3           & 30.3          & 4.1          & 6.5           & 15.6        & 34.2        & 22.7          \\ \hline
PointMLP         & 2.9          & 4.1           & 7.6           & 24.6          & 23.6          & 34.4                & 2.8          & 1.8           & 2.7           & 25.4          & 1.6          & 3.9           & 11.3        & 28          & 19.4          \\ \hline
PointNeXt        & 2            & 13.3          & 15.2          & 26.1          & 12.8          & 61.1                & 5.4          & 4.7           & 1.7           & 26.6          & 3.2          & 8.4           & 15          & 33          & 21.2          \\ \hline
Ours             & 5            & \textbf{26.5} & \textbf{20.1} & \textbf{35.8} & \textbf{26.5} & \textbf{68.5}       & 6.8          & 6.2           & \textbf{11.2} & 30.4          & 4.5          & 10.8          & \textbf{21} & \textbf{40} & \textbf{26.9} \\ \hline
Ours(w/o   ENFI) & \textbf{6.1} & 16.7          & 16.8          & 31            & 18.4          & 55.8                & \textbf{7.8} & 3.9           & 1.3           & 11.7          & \textbf{4.6} & 10.9          & 15.4        & 37.1        & 24.7          \\ \hline
\end{tabular}
}
\end{table*}


\begin{table*}[ht]
\caption{Detailed comparison results of action recognition on HuCenLife. All methods are based on the ground truth bounding boxes. mAcc stands for mean accuracy.}
\label{tab:action_abcde_ground}

\setlength{\tabcolsep}{1.3mm}
\resizebox{\linewidth}{!}{     
\begin{tabular}{c|c|c|c|c|c|c|c|c|c|c|c|c|c}
\hline
Method         & Lift & Carry & Move & Pull\_Push & Sit  & Scooter-BalanceBike & Hum-Inter & Fitness & Entertain & Sports & Bend-Over & Standing & mAcc \\ \hline
ViT            & 9.1  & 10.7  & 26.2 & 36.3       & 25.3 & 15.2               & 1.9      & 51.6    & 50.9      & 65.5   & 13.5      & 16.0     & 26.9 \\ \hline
PVT            & 4.5  & 42.8  & 31.2 & 35.6       & 40.0 & 74.7               & 7.2      & 36.4    & 0.4       & 16.2   & 54.4      & 31.6     & 31.3 \\ \hline
PointNet       & 7.8  & 29.1  & 32.8 & 33.2       & 47.2 & 53.1               & 7.5      & 46.9    & 19.1      & 20.1   & 57.4      & 20.9     & 31.3 \\ \hline
PointNet++     & 11.1 & 41.1  & 37.7 & 23.5       & 66.7 & 80.3               & 15.5     & 39.3    & 55.4      & 11.4   & 30.3      & 8.6      & 35.1 \\ \hline
PointMLP       & 25.6 & 46.4  & 35.4 & 57.2       & 55.2 & 79.7               & 4.9      & 54.5    & 27.8      & 15.3   & 29.1      & 32.8     & 38.7 \\ \hline
PointNext      & 11.8 & 46.7  & 24.0 & 49.4       & 50.1 & 76.1               & 21.6     & 46.9    & 36.5      & 10.2   & 36.2      & 53.0     & 38.5 \\ \hline
Ours           & 19.8 & 38.9  & 30.0 & 59.8       & 62.5 & 86.6               & 62.5     & 61.8    & 32.4      & 18.2   & 35.0      & 24.8     & \textbf{44.4} \\ \hline
Ours(w/o ENFI) & 18.9 & 49.5  & 47.6 & 57.2       & 53.3 & 83.1               & 28.8     & 31.5    & 31.2      & 19.2   & 53.6      & 33.8     & 42.3 \\ \hline
               
\end{tabular}
}
\end{table*}

%-------------------------------------------------------------------------

% {\small
% \bibliographystyle{ieee_fullname}
% \bibliography{egbib}
% }

\end{appendix}