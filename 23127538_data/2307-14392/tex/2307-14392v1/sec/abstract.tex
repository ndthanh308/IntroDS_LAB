% The research of scene understanding based on image and video has developed for a long time. However, the existing data sets mainly focus on narrow and single indoor scenes. For government security or automatic driving scenarios, the existing work cannot accurately locate and analyze multiple people in a large scene at the same time, Image-based methods often fail to work at night, and due to the lack of depth information in the image, it is difficult to distinguish the people who block each other in a scene. However, LiDAR can capture accurate depth information in a long distance and a wide range and LiDAR-based methods is able to accurately locate human position and provide 3D surface point cloud reflecting human posture. We propose HuCenLife, a Human-centric Scene Understanding dataset for 3D Large-scale Scenarios collected by one LiDAR and six cameras, and provides rich annotations for multiple tasks. Based on HuCenLife, we provide three baselines for instance segmentation, 3D human detection and human action recognition. In addition, we also explored the potential application of HuCenLife in other tasks.
Human-centric scene understanding is significant for real-world applications, but it is extremely challenging due to the existence of diverse human poses and actions, complex human-environment interactions, severe occlusions in crowds, etc. In this paper, we present a large-scale multi-modal dataset for human-centric scene understanding, dubbed HuCenLife, which is collected in diverse daily-life scenarios with rich and fine-grained annotations. Our HuCenLife can benefit many 3D perception tasks, such as segmentation, detection, action recognition, etc., and we also provide benchmarks for these tasks to facilitate related research. In addition, we design novel modules for LiDAR-based segmentation and action recognition, which are more applicable for large-scale human-centric scenarios and achieve state-of-the-art performance. The dataset and code can be found at
\url{https://github.com/4DVLab/HuCenLife.git}.
