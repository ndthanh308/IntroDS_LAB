\documentclass[aps,prl,nobibnotes,twocolumn,superscriptaddress,citeautoscript,showkeys]{revtex4-2}

\pdfoutput=1
\usepackage{graphicx}
\usepackage{amsmath, amsfonts}
\usepackage{amssymb}
\usepackage[colorlinks=true,linkcolor=blue,citecolor=blue]{hyperref}
\usepackage[utf8]{inputenc}
\usepackage{float}

\begin{document}

\title[Autonomous neural information processing by a dynamical memristor circuit]{Autonomous neural information processing by a dynamical memristor circuit}


\author{D\'{a}niel Moln\'{a}r}
\affiliation{Department of Physics, Institute of Physics, Budapest University of Technology and Economics, M\H{u}egyetem rkp. 3, H-1111 Budapest, Hungary}
\affiliation{ELKH-BME Condensed Matter Research Group, M\H{u}egyetem rkp. 3, H-1111 Budapest, Hungary}

\author{T\'{i}mea N\'{o}ra T\"{o}r\"{o}k}
\affiliation{Department of Physics, Institute of Physics, Budapest University of Technology and Economics, M\H{u}egyetem rkp. 3, H-1111 Budapest, Hungary}
\affiliation{Institute of Technical Physics and Materials Science, Centre for Energy Research, Konkoly-Thege M. \'{u}t 29-33, 1121 Budapest, Hungary}

\author{Roland K\"{o}vecs}
\affiliation{Department of Physics, Institute of Physics, Budapest University of Technology and Economics, M\H{u}egyetem rkp. 3, H-1111 Budapest, Hungary}

\author{L\'{a}szl\'{o} P\'{o}sa}
\affiliation{Department of Physics, Institute of Physics, Budapest University of Technology and Economics, M\H{u}egyetem rkp. 3, H-1111 Budapest, Hungary}
\affiliation{Institute of Technical Physics and Materials Science, Centre for Energy Research, Konkoly-Thege M. \'{u}t 29-33, 1121 Budapest, Hungary}

\author{P\'{e}ter Bal\'{a}zs}
\affiliation{Department of Physics, Institute of Physics, Budapest University of Technology and Economics, M\H{u}egyetem rkp. 3, H-1111 Budapest, Hungary}

\author{Gy\"{o}rgy Moln\'{a}r}
\affiliation{Institute of Technical Physics and Materials Science, Centre for Energy Research, Konkoly-Thege M. \'{u}t 29-33, 1121 Budapest, Hungary}

\author{Nadia Jimenez Olalla}
\affiliation{Institute of Electromagnetic Fields, ETH Zurich, Gloriastrasse 35, 8092 Zurich, Switzerland}

\author{Juerg Leuthold}
\affiliation{Institute of Electromagnetic Fields, ETH Zurich, Gloriastrasse 35, 8092 Zurich, Switzerland}

\author{J\'{a}nos Volk}
\affiliation{Institute of Technical Physics and Materials Science, Centre for Energy Research, Konkoly-Thege M. \'{u}t 29-33, 1121 Budapest, Hungary}

\author{Mikl\'{o}s Csontos}
\affiliation{Institute of Electromagnetic Fields, ETH Zurich, Gloriastrasse 35, 8092 Zurich, Switzerland}

\author{Andr\'{a}s Halbritter}\email{halbritter.andras@ttk.bme.hu}
\affiliation{Department of Physics, Institute of Physics, Budapest University of Technology and Economics, M\H{u}egyetem rkp. 3, H-1111 Budapest, Hungary}
\affiliation{ELKH-BME Condensed Matter Research Group, M\H{u}egyetem rkp. 3, H-1111 Budapest, Hungary}


\begin{abstract}
Analog tunable memristors are widely utilized as artificial synapses in various neural network applications. However, exploiting the dynamical aspects of their conductance change to implement active neurons is still in its infancy, awaiting the realization of efficient neural signal recognition functionalities. Here we experimentally demonstrate an artificial neural information processing unit that can detect a temporal pattern in a very noisy environment, fire a single output spike upon successful detection and reset itself in a fully unsupervised, autonomous manner. This circuit relies on the dynamical operation of only two memristive blocks: a non-volatile Ta$_2$O$_5$ device and a volatile VO$_2$ unit. A fading functionality with exponentially tunable memory time constant enables adaptive operation dynamics, which can be tailored for the targeted temporal pattern recognition task. In the trained circuit false input patterns only induce short-term variations. In contrast, the desired signal activates long-term memory operation of the non-volatile component, which triggers a firing output of the volatile block.
\end{abstract}

\keywords{memristor, resistive switching, artificial neuron, edge computing, tantalum pentoxide, vanadium dioxide}

\date{\today}
\maketitle

\section{Introduction}

Due to their compact size and unsophisticated structure \cite{Pi2019}, low-energy operation \cite{Pickett2012}, high speed \cite{Csontos2023,Witzleben2020,Choi2016a} and CMOS compatibility \cite{Jo2008,Cai2019a,Rao2023}, the applications of memristor synapses in artificial neural networks (ANNs) are booming. Crossbar arrays \cite{Xia2019} of nonvolatile memristors exhibiting multiple resistive states with linear current-voltage [$I(V)$] characteristics \cite{Rao2023,Li2018a} play a key role: The single devices represent the analog synaptic weights, whereas the crossbar array grants the full connectivity between the neighboring neural layers which are typically implemented by standard CMOS circuits \cite{Xia2009,Sheng2019}. Such hybrid CMOS-memristor architectures have been utilized in various neuromorphic computing tasks from data classification \cite{Bayat2018,Li2018b,Wang2018c} to feature extraction \cite{Li2018a,Lin2020} and as field-programmable analog arrays \cite{Li2022}.

Memristor-based synaptic layers efficiently accelerate ANN operation at low energy costs. However, the recursive tuning of the synaptic weights during training still generates extensive software overheads. Alternative approaches exploit the dynamical properties of resistive switching phenomena such as the resistance relaxation of volatile memristors \cite{Du2017b,Moon2019} or the exponential dependence of the switching time on the bias voltage \cite{Waser2009,Gubicza2015a} which facilitates short-term/long-term learning and forgetting abilities \cite{Hasegawa2010,Serb2016,Wang2019a}. As a common figure of merit, ANNs relying on dynamical memristor operation require considerably less network nodes and trained synapses to achieve the same computational efficiency as the more traditional deep neural networks (DNNs) where merely the static properties of the memristors are exploited.

The role of dynamical complexity in the nonlinear, and often adaptive current response of memristors and their small circuits have been revealed recently \cite{Kumar2022}. In a pioneering work, W.~Yi et al. \cite{Yi2018} have faithfully reproduced 23 known neural spiking patterns by utilizing the fourth-order dynamical complexity facilitated by only two coupled VO$_{2}$ memristors. As a next step toward fully memristor-based, bio-realistic spiking neural networks (SNNs) exhibiting unrivaled footprint, energy efficiency and computational power, the exploration of such dynamical building blocks and the corresponding training algorithms is imperative.

% Figure environment removed

Here we demonstrate that a small dynamical circuit based on a single non-volatile Ta$_2$O$_5$ memristor is sufficient to recognize specific temporal patterns, such as neural voltage spikes, in a complex input stream. By the application of a series resistor element and a negative offset voltage, a memory time constant is introduced to the circuit. The exponentially broad voltage-tunability of this time constant is the key enabling feature of temporal data classification. Extending the circuit with a volatile memristor, neural firing and reset functionalities are added. Our circuit has been shown to operate reliably in very noisy environments, satisfying bio-realistic and/or edge computing requirements. The crucial condition of operation is the presetting of the memory time constant of the nonvolatile memristor. The dynamical properties can be optimized to the specific input signal stream by adjusting only one resistor value and one constant voltage offset. Thereby, very power-efficient and robust, autonomous neuromorphic edge transceivers can be realized and operated at negligible training costs.


\section{Results and discussion}

Our report is organized as follows. First, the basic building blocks of our autonomous signal detector circuit are introduced. Next, we demonstrate through simulation and experiment, that by applying a negative offset voltage, adjustable long-term/short-term memory can be implemented in a simple circuit of a nonvolatile memristor and a series resistor. This variable memory effect is utilized to distinguish between positive-signed sub-threshold / super-threshold voltage pulses which induce a transition to a short-lived / long-standing low resistance state (LRS) of the memristor, before the initial high resistance state (HRS) is restored by the permanent negative offset voltage. We show that the same concept can be applied also to reliably discriminate Gaussian voltage pulses from random voltage noise of comparable amplitude. Finally, we extend the above detector unit with (i) an oscillator circuit based on a volatile Mott memristor and (ii) a feedback loop. Thereby we realize a complete artificial neuron which emits a neural voltage spike upon receiving a pre-defined temporal signal pattern.


\subparagraph{The architecture of the neuron circuit.}

In our autonomous signal detection scheme we utilize a Ta/Ta$_2$O$_5$/Pt nonvolatile memristor symbolized by the red icon in Fig.~\ref{fig1}(a) and a VO$_2$-based volatile memristor denoted by the blue icon in Fig.~\ref{fig1}(a). The former / latter device was fabricated in a vertical crosspoint stack / planar electrode arrangement, as illustrated in Fig.~\ref{fig1}(b) and (c), respectively. Further details on the device preparation are provided in the Methods section. A representative current-voltage [$I(V)$] characteristics of the Ta$_2$O$_5$ device is shown in Fig.~\ref{fig1}(d). It was measured in the setup highlighted in red in Fig.~\ref{fig1}(g). The $V_{\rm bias}^{\rm T}$ bias voltage corresponding to the voltage drop on the memristor is calculated according to $V_{\rm bias}^{\rm T}=V_{\rm drive}-I_{\rm T}\cdot R_{\rm s,T}$, where $R_{\rm s,T}$ is a series resistor and $V_{\rm drive}$ is the triangular output of the voltage source. Positive voltage refers to a higher potential on the Ta top electrode with respect to the Pt bottom electrode. The hysteretic $I(V)$ trace corresponds to filamentary type resistive switching \cite{Csontos2023,Miao2011,Lee2011a} between metallic states in the analog tunable 1~k$\Omega$\,--\,10~k$\Omega$ regime \cite{Rao2023,Li2018a,Torok2023}.

Figure~\ref{fig1}(e) displays a typical $I(V)$ curve of the Au/VO$_{2}$/Au memristor, measured in series with the $R_{\rm s,V}$ resistor. This corresponds to unipolar, volatile resistive switching due to a temperature and electric field controlled Mott transition \cite{Valle2019,Posa2023}. In our work, the asymmetric layout and the only 30~nm wide lateral gap of the Au electrodes grant a very small and, thus, homogeneous active volume. A detailed experimental analysis and finite element simulations of the device performance and resistive switching mechanism can be found in Ref.~\cite{Posa2023}. In the experiments discussed later, the circuit environment of the Au/VO$_{2}$/Au memristor, highlighted in blue in Fig.~\ref{fig1}(g), is further extended with the capacitor $C$ and the additional resistor $R_{\rm L}$. These, together with the $R_{\rm s,V}$ resistor form a conventional VO$_{2}$ oscillation circuit \cite{Yi2018}. Figure~\ref{fig1}(f) illustrates a single output voltage spike of the oscillator circuit. 

Figure~\ref{fig1}(g) summarizes the architecture of the complete neuron circuit where the lower and upper dashed arrows symbolize the coupling of the input detector and the output firing modules as well as the feedback of the output to the input, respectively. The detailed circuit schematics, including the instrumental implementation of the coupling and the feedback is provided in the Methods section.


\subparagraph{The memory time constant of the detector circuit.}


First, we provide an insight into the operation principle of our neural spike detection scheme via numerical simulations. The key ingredient is the exponential slowdown of the resistive switching as the $V_{\rm bias}^{\rm T}$ bias voltage is linearly decreased on the Ta/Ta$_2$O$_5$/Pt memristor. The exponential voltage dependence of the $\tau$ resistive switching time constant on the bias voltage is commonly known as the `voltage-time dilemma' \cite{Waser2009,Gubicza2015a}. It is a general figure of merit observed in a broad range of resistive switching systems \cite{Chen2017} which describes the dependence of the $I(V)$ characteristics or the pulsed switching response on the speed and voltage amplitude of the driving signal.

In our simulations, we assume $\tau=10^{-\left(\left|V_{\rm bias}^{\rm T}\right|-b\right)/a}$, where the $a$ and $b$ adjustable parameters describe the voltage-dependent dynamical properties. When a time-dependent $V_{\rm drive}(t)$ driving voltage is applied to the memristor and the $R_{\rm s,T}$ serial resistor, the voltage drop on the memristor is calculated as $V_{\rm bias}^{\rm T}=V_{\rm drive}-I_{\rm T}\cdot R_{\rm s,T}$. The dynamics of the system is determined by a minimal model according to the ${\rm d}x/{\rm d}t=-x/\tau$ and ${\rm d}x/{\rm d}t=(1-x)/\tau$ differential equations for positive/negative bias voltages, respectively. The $0\le x\le 1$ state variable is related to the memristor resistance according to $R_{\rm T}(t)=V_{\rm bias}^{\rm T}/I_{\rm T}=R_{\rm LRS}+\left(R_{\rm HRS}-R_{\rm LRS}\right)\cdot x(t)$. 

In a highly simplified physical picture, the variation of $x$ can be associated with the voltage-controlled manipulation of the oxygen ion concentration in the filamentary region of the Ta/Ta$_2$O$_5$/Pt memristor \cite{Kumar2022}. In the initial HRS state ($R_{\rm HRS}$) the poorly conducting filamentary region is saturated with oxygen ions ($x=1$). Due to the applied positive voltage the mobile oxygen ions are removed from the active region at a rate which is proportional both to the voltage-controlled time constant $\tau$ and the $\sim x$ amount of oxygen ions in the filamentary region. Such a process results in the decrease of the filament resistance and is self-terminated when all the mobile oxygen ions are removed ($x=0$ and $R_{\rm T}=R_{\rm LRS}$). In practice, however, the set process is rather limited by the serial resistance, i.e., when $R_{\rm T}(t)$ becomes smaller than $R_{\rm s,T}$ the bias voltage on the memristor decreases, meanwhile the voltage-dependent time constant exponentially increases, and the set transition stops. At opposite voltage polarity, the oxygen ions are actuated from the surrounding oxide matrix towards the filamentary region, yielding an increase of the resistance. In this case, the rate is again proportional to $\tau$, as well as the $\sim(1-x)$ amount of the available mobile oxygen ions in the oxide.


\subparagraph{Simulated temporal pattern detection.}

Using the above dynamical equations the response of the system to a time-dependent driving voltage signal can be simulated. First, we calculate the $I(V)$ curves at different $f_{\rm drive}$ frequencies and $V_{\rm drive}^{0}$ amplitudes of a triangular $V_{\rm drive}$. Next, we compare these to the measured frequency-dependent $I(V)$ characteristics of our Ta$_{2}$O$_{5}$ memristor. This comparison allows us to extract the actual, device-specific values of the $a$ and $b$ model parameters (see Methods section for further details). Adapting the latter into the simulation, the realistic voltage-dependent dynamical properties of the Ta$_{2}$O$_{5}$ devices can be modeled.

In the simulations shown in Fig.~\ref{fig3} we apply positive-signed voltage pulses superimposed at a constant negative voltage offset to the Ta$_{2}$O$_{5}$ memristor \,--\, $R_{\rm s,T}$ circuit. Such a voltage driving scheme is displayed in Fig.~\ref{fig3}(a). We demonstrate that thereby an adjustable long-term/short-term memory can be introduced to the system. The Ta$_{2}$O$_{5}$ memristor is initialized in its HRS and the $V_{\rm pulse}^{0}$ peak value and $T_{\rm pulse}$ duration of the voltage pulses are selected such that they induce a resistive switching to the LRS. In the LRS between the positive pulses the memristor is exposed to a negative voltage bias which can be calculated as $V_{\rm bias}^{\rm T}=V_{\rm offset}\cdot R_{\rm LRS}/\left( R_{\rm LRS}+R_{\rm s,T}\right)$. Note, that here $V_{\rm offset}$ and $R_{\rm s,T}$ are adjustable variables, whereas $R_{\rm LRS}$ can be fine-tuned by the appropriate combination of $V_{\rm pulse}^{0}$ and $R_{\rm s,T}$, as explained earlier in this Section.

% Figure environment removed

Figure~\ref{fig3}(b) shows a scenario, when the magnitude of the $V_{\rm offset}$=-1.65~V is sufficient to facilitate a complete reset transition before the arrival of the next set pulse, realizing a short-term memory effect. The calculated time-dependence of $V_{\rm bias}^{\rm T}$ and the $R_{\rm T}$ memristor resistance are displayed in gray and red, respectively. The corresponding $\tau$ is exhibited in Fig.~\ref{fig3}(c). Its time dependence exponentially magnifies the effect of the linear-scale time-evolution of $V_{\rm bias}^{\rm T}$: The resistive switching transition is drastically slowed down in both directions, as lower $R_{\rm T}$ values and, thus, lower $V_{\rm bias}^{\rm T}$ levels are approached. This behavior mimics the short-term potentiation (STP) known from the early studies of the human nervous system \cite{Hodgkin1952}.

In contrast, Fig.~\ref{fig3}(d) demonstrates that by reducing the magnitude of the negative voltage offset by $\approx$20\% to $V_{\rm offset}$=-1.3~V, a long-term memory effect can be induced. In this case $\tau$, shown in Fig.~\ref{fig3}(e), exceeds the separation of the programming voltage pulses and the Ta$_{2}$O$_{5}$ memristor cannot return to its HRS. Instead, the subsequent positive pulses further decrease the resistance of the LRS, as seen in the red trace in Fig.~\ref{fig3}(d). Such a long-term memory effect is reminiscent of the long-term potentiation (LTP) of the biological synapses \cite{Hodgkin1952}.

The previous simulations demonstrate that the learning/forgetting ability of the simple analog circuit containing one nonvolatile memristor can be tuned over orders of magnitude in the time domain by the convenient, linear-scale adjustment of a negative voltage offset. Consequently, this feature can also be exploited to distinguish between different temporal input signals. Figure~\ref{fig3}(f) shows a simulation where $R_{\rm s,T}$ and $V_{\rm offset}$ were optimized such that the resistance of the memristor exhibits only a short-term excursion into a moderately low LRS upon receiving occasional, high-amplitude noise spikes embedded in a white noise input stream. However, when a Gaussian `neural spike' with a comparable amplitude but longer duration arrives, a transition into a virtually permanent lower resistance LRS is triggered. This demonstrates that the dynamics of the memristive signal detection circuit is sensitive to the `strength' of the input signal both in the voltage and time domain: A transition from STP to LTP can be induced not only by increasing the voltage amplitude at fixed length but also by increasing the length of the signal at a certain voltage amplitude. Thereby the memristor circuit is capable of reliable pattern recognition also in situations where traditional voltage threshold-based trigger circuits are challenged by potential noise spikes of short duration but high amplitude.

\subparagraph{Pulse sequence and neural spike detection experiments.}

The simulation results of the signal detection scheme were experimentally verified using a single Ta$_2$O$_5$ memristor and a $R_{\rm s,T}$=1~k$\Omega$ series resistor, as shown in Fig.~\ref{fig4}. Long- and short-term memory operation due to uniform, positive voltage pulses over a constant negative background are demonstrated in Fig.~\ref{fig4}(a) and (b), respectively. Here the peak values of the programming pulses as well as the initial conditions of the circuit are identical, except for the negative voltage offset which was changed from $V_{\rm offset}$=-1.6~V (long-term memory) to $V_{\rm offset}$=-2~V (short-term memory). The selectivity of the STP/LTP response of the circuit to the programming voltage amplitude was tested by exposing it to the input pattern consisting of uniform programming pulses except for one which was 0.5~V higher in its peak value, as shown in Fig.~\ref{fig4}(c). Here, the negative offset is optimized to erase the long-term effect of sub-threshold pulses, but the third, higher pulse drives the system to a long-term memory state. This observation implies that the circuit can function as a detector module which recognizes super-threshold input signals by the transition into an enduring LRS of the Ta$_2$O$_5$ memristor whereas sub-threshold signals only induce short-term, fully recovering resistance changes, if any.

% Figure environment removed

Neural spike detection in a noisy environment, as proposed by the simulations shown in Fig.~\ref{fig3}(f), has been tested experimentally by applying independent time series of identical white noise characteristics, a negative voltage offset and occasional neural spikes, i.e., Gaussian voltage pulses to the detector circuit. Example time traces of such input streams are illustrated in Fig.~\ref{fig4}(d), where every second sequence contains a Gaussian spike, as labeled by the arrows. The measured resistance response of the memristor, shown in Fig.~\ref{fig4}(e) verifies that the neural spikes always trigger the transition into an enduring LRS whereas random noise only results in quickly fading, smaller resistance changes. This contrast is better exemplified in Fig.~\ref{fig4}(f), where a selected single time trace contains both a higher-amplitude, short noise spike and a comparable amplitude, but longer lasting neural spike. The former induces a smaller magnitude, quickly fading resistance drop while the latter facilitates the transition to an enduring LRS.

For a statistical analysis 50-50 noisy time series including/lacking a neural spike were experimentally evaluated by the detector circuit at nominally identical conditions, assuming an optimized $V_{\rm offset}$. As successful detection events only those time series are considered which (i) contain a neural spike and only this spike is detected at its true position or (ii) there is no neural spike in the stream and none is detected, accordingly. The detection accuracy, defined as the ratio of the successfully classified streams was found to be 99\% at 0.4~V pulse amplitude and 0.2~V rms noise. According to the rms to peak-to-peak (pp) conversion of white noise, this rms value yields $<$1.6V peak-to-peak fluctuations in 98\% of the time. In comparison, an alternative approach based solely on the threshold switching property of metal-oxide memristors yielded in $\approx$60\% detection accuracy at $\approx$1.7~V spike amplitude and 0.21~V rms / 1.1~V pp noise levels \cite{Gupta2016,Gupta2017,Gupta2019}.


\subparagraph{Neural transceiver operation.}

% Figure environment removed

In order to add the transceiver functionality and automated reset to our Ta$_{2}$O$_{5}$-based neural spike detection circuit, we extend it by a VO$_{2}$-based neural spike firing module and a global feedback loop, according to the blue shaded layout parts and the gray dashed arrow in Fig.~\ref{fig1}(g), respectively. The detailed circuit implementation is outlined in the Methods section. First, we discuss the behavior of the zero-feedback circuit which actuates a firing output upon spike detection. Next, the global feedback is activated, completing the circuit operation with output optimization and automated reset.

Figures~\ref{fig5}(a-d) exemplify an experiment utilizing the input stream displayed in Fig.~\ref{fig5}(a), consisting of white noise, a single neural spike and a constant negative voltage offset. The strength of the neural spike and the negative voltage offset are adjusted such that the memory time constant of the detector circuit is in the order of $\approx$100~ms, as shown in Fig.~\ref{fig5}(b). While the Ta$_{2}$O$_{5}$ memristor resides in its LRS, the voltage drop on the $R_{\rm s,T}$ series resistor, arising from the constant negative $V_{\rm offset}$, is increased in magnitude due to the voltage divider effect of $R_{\rm T}$ and $R_{\rm s,T}$. We utilize this voltage drop as the input signal of the VO$_{2}$ oscillator module. The coupling between the detector and oscillator modules is realized through a differential amplifier which sets the appropriate input level for the VO$_{2}$ circuit and also acts as a low-pass filter. This filtering implements an integration time constant that averages out the rapidly changing signal components, and practically measures the effect of the negative offset voltage on the $R_{\rm s,T}$ series resistor. The latter exhibits an order of magnitude increase upon the set transition of the Ta$_{2}$O$_{5}$ memristor. While the Ta$_{2}$O$_{5}$ memristor is in its HRS, the oscillation module is in its steady state, maintaining a constant, low-level $I_{\rm V}$ output current corresponding to the HRS of the VO$_{2}$ memristor. However, when the Ta$_{2}$O$_{5}$ circuit detects a neural spike, the increased voltage input of the VO$_{2}$ circuit results in the firing of periodic current pulses, as shown in Fig.~\ref{fig5}(c). This so-called tonic spiking pattern, magnified in Fig.~\ref{fig5}(d) is explained in terms of the periodic charging and discharging of the parallel capacitor $C$ due to the current supplied through $R_{\rm L}$ and the voltage-induced current instability of the VO$_{2}$ memristor, respectively \cite{Yi2018}. The conditions of the periodic firing are maintained as long as the fading LRS of the Ta$_{2}$O$_{5}$ resides below a threshold, determined by $V_{\rm offset}$, $R_{\rm s,T}$, the voltage gain of the coupling amplifier and the $I(V)$ characteristics of the VO$_{2}$ memristor. The rise and fall times as well as the periodicity of the individual $I_{\rm V}$ output current pulses are tunable via the $R_{\rm L}$, $C$ and $R_{\rm s,V}$ component values.

Finally, the neuro-transceiver circuit is completed by introducing the global feedback from its spiking output to the detector input. This is realized via a second differential amplifier such that the positive valued output spikes of the VO$_{2}$ module appear with a negative sign in the $V_{\rm drive}$ detector input stream. At the same time, they are amplified to a level where a single negative spike is sufficient to trigger the reset transition of the Ta$_2$O$_5$ memristor in a short time compared to the repetition rate of the output spikes. As a result, the oscillator module is brought back to its steady state before the execution of its second output spike and the neuro-transceiver circuit is ready to receive, detect and flag the next neural input spike.

The reliable operation of the above scheme is experimentally demonstrated in Figs.~\ref{fig5}(e-g) where the circuit is exposed to 10 independent input streams consisting of white noise and a constant negative offset. The latter now incorporates a portion of the default output through the global feedback loop. Every second input sequence also contains a Gaussian neural spike, as shown by the black traces in Fig.~\ref{fig5}(e). The resistance response of the Ta$_{2}$O$_{5}$ memristor is displayed in Fig.~\ref{fig5}(f) in red. During the input sequences lacking a neural signal $R_{\rm T}$ stays in the HRS and the corresponding $I_{\rm V}$ current output of the VO$_{2}$ module, shown in Fig.~\ref{fig5}(e), also remains at a constant low level. However, whenever a neural spike arrives at the input, $R_{\rm T}$ drops to the LRS, an output spike is triggered and the HRS of $R_{\rm T}$ is restored. A magnified view of an independent time series, exhibited in Figs.~\ref{fig5}(h-j), allows a better insight into the dynamics of the detection. The set transition of $R_{\rm T}$ (red) occurs instantaneously during the $\approx$5~ms rise time of the neural spike (black). The output spike (blue) is fired 13~ms later. This delay corresponds to the elongated rise time of the detector module's output at the firing module's input due to the low-pass filter implemented in their coupling. The output spike simultaneously appears with a negative sign at the input (black) through the feedback amplifier and facilitates the reset of the Ta$_{2}$O$_{5}$ memristor and, thus, the whole circuit within 1~ms.

Note, that the time constant of the low-pass filter exceeds not only the one of the random fluctuations but also the duration of the neural spikes. This grants that neither the noise nor the neural spikes can directly trigger the firing of the VO$_{2}$ circuit. The latter is only facilitated by the set transition of the Ta$_{2}$O$_{5}$ memristor, whose latching operation improves the reliability of the neuron circuit. Moreover, the filtering allows a longer `integration' of the input stream by the detector circuit, resulting in higher detection accuracy. The cut-off frequency of the coupling and, thus, the integration time of the detector circuit are instrumentally optimized to suit the specific input pattern.

The statistical evaluation of the detection accuracy of the complete neuro-transceiver was carried out as described for the detector circuit in Section~2.3. In comparison to the latter, the experimentally verified detection accuracy was slightly decreased from 99\% to 98\% at comparable signal-to-noise ratio characterized by 0.5~V pulse amplitude and 0.22~V rms / 1.76~V pp noise levels. The additional detection failure was identified to arise from the partial set transition of the Ta$_{2}$O$_{5}$ memristor at a true pulse detection event which was insufficient to access the oscillatory operation regime of the VO$_{2}$ memristor.


\section{Conclusion}

In conclusion, we experimentally demonstrated an artificial neuro-transceiver unit realized by a coupled dynamical memristor circuit consisting of one nonvolatile and one volatile memristor. The broad, voltage-tunable dynamics of nonvolatile resistive switching was utilized to discriminate between random noise and neural voltage spikes. As a key ingredient, a variable fading memory was introduced to the LRS of the memristor by applying a negative voltage offset. A close to 100\% spike detection accuracy was achieved even at high noise levels, which is attributed to the combined dynamical impact of the amplitude and the duration of the neural spikes, in contrast to more noise-sensitive voltage threshold detection schemes. In order to extend the neural spike detection with conditional firing, the nonvolatile detector module was coupled to a volatile Mott memristor oscillator circuit. The dynamical complexity of the latter was exploited by alternating its operation point between the linear HRS response and tonic spiking regimes according to the actual output status of the spike detector. Finally, the artificial neuron circuit was completed by adding a global feedback coupling. Thereby an improved control over the firing output and an automated reset functionality were facilitated.

As a basic operation flow, the artificial neuron was shown to detect neural spikes in a noisy environment, fire a single output spike upon a successful detection event and reset in a fully unsupervised, autonomous manner. Moreover, thank to its simple architecture, the detector module or, alternatively, a small number of parallel units can be conveniently sensitized to detect different features in the input data stream at negligible training costs, solely by adjusting the negative offset and $R_{\rm s,T}$ values. As the applied Ta$_{2}$O$_{5}$ memristors exhibit voltage-tunable resistive switching times spanning from bio-compatible time-scales down to picoseconds \cite{Csontos2023}, a broad variety of input signals can be considered. We propose that by taking advantage of the rich dynamical complexity of the VO$_{2}$ oscillator module \cite{Yi2018}, such a versatility at the front-end can be complemented by a similar diversity in the output pattern, giving rise to the realization of more general-purpose neural signal processing units. 

During standby conditions the involved memristors idle in their HRS keeping the overall power consumption of the circuit extremely low. Meanwhile, the requirements on the standard peripheral circuits facilitating the detector-oscillator coupling and the global feedback are undemanding, which enables their low-cost CMOS integration and packaging. Fulfilling these key criteria allows the seamless integration of our neuron circuit into a plethora of edge computing applications.


\section{Methods}

\noindent \textbf{Sample fabrication.} The 10~nm thick Ti adhesive layer and the 40~nm thick Pt bottom electrode of the Ta/Ta$_{2}$O$_{5}$/Pt memristor were subsequently deposited on a 280~nm thick SiO$_{2}$ substrate by electron beam evaporation at a base pressure of 10$^{-7}$~mbar at a rate of 0.1~nm/s. The 5~nm thick Ta$_{2}$O$_{5}$ layers were sputtered by reactive high-power impulse magnetron sputtering (HiPIMS) from a Ta target at 6~mTorr pressure, 45~sccm Ar and 5~sccm O$_{2}$ flow rates and 250~W RF power. The thickness and stoichiometric composition of the Ta$_{2}$O$_{5}$ layer were confirmed by XPS spectroscopy. The Ta top electrode and its Pt cap were sputtered on top of the Ta$_{2}$O$_{5}$ film at 4~mTorr pressure, 45~sccm Ar flow and 250~W RF power / 125~W dc power for Ta / Pt, preventing the formation of a native oxide layer at the Ta$_{2}$O$_{5}$/Ta interface. The 2.5~$\mu$m wide bottom and top electrodes were patterned by standard optical lithography and lift-off.

The VO$_{2}$ films were created by the post-deposition heat treatment of a Si/SiO$_{2}$/V structure, where the SiO$_{2}$ / V thickness was 1~$\mu$m / 100~nm. The heat treatment was carried out in air at 400~$^{\circ}$C temperature and 0.1~mbar pressure over 4.5~hours, resulting in a 180~nm thick V$_{2}$O$_{5}$ bottom layer and a 40~nm thick VO$_{2}$ top layer as verified by cross-sectional TEM and EELS analyses \cite{Posa2023}. The metal electrodes consisting of a 10~nm thick Ti adhesion layer and a 50~nm thick Au film were patterned by standard electron-beam lithography and deposited by electron-beam evaporation at 10$^{-7}$~mbar base pressure at rates of 0.1~nm/s and 0.4~nm/s, respectively, followed by lift-off.\\

% Figure environment removed


\noindent \textbf{Switching dynamics simulation of the Ta$_2$O$_5$ detector circuit.} Figure~\ref{fig2} shows the results of the simulations in comparison with selected experimental $I(V)$ traces of the Ta$_{2}$O$_{5}$ memristor in series with $R_{\rm s,T}$=1~k$\Omega$. The individual simulated traces in Figs.~\ref{fig2}(a) and (b) illustrate the onset of nonvolatile resistive switching as the $V_{\rm drive}^{0}$ is linearly increased at a constant $f_{\rm drive}$ or, alternatively, as $f_{\rm drive}$ is exponentially decreased at a constant $V_{\rm drive}^{0}$. The color-scale plot in Fig.~\ref{fig2}(c) displays the $R_{\rm HRS}/R_{\rm LRS}$ resistance ratio deduced from the zero-bias slopes of the simulated $I(V)$ traces for the experimentally relevant 1.2~V$<V_{\rm drive}^{0}<$1.7~V voltage interval. Note, that while the latter covers a $\approx$40\% variation the $f_{\rm drive}$ driving frequency spans over 5 orders of magnitude. The colored empty circles in Fig.~\ref{fig2}(c) label the driving parameters of the measured $I(V)$ traces shown in the corresponding colors in Fig.~\ref{fig2}(d). The measured $I(V)$ characteristics in Fig.~\ref{fig2}(d) unambiguously confirm that achieving an identical $R_{\rm HRS}/R_{\rm LRS}$ resistance ratio at linearly increasing voltage requires an exponentially increasing frequency. In the experiment, a constant $R_{\rm HRS}/R_{\rm LRS}\approx$1.5 ratio was chosen, representing the onset of resistive switching. We explored those $V_{\rm drive}^{0}$ and $f_{\rm drive}$ settings which reproduced this ratio. Finally, the simulation parameters were optimized at $a$=0.088~V and $b$=1.06~V by finding the best match between the simulated and measured $I(V)$ traces throughout the investigated $V_{\rm drive}^{0}$\,--\,$f_{\rm drive}$ plane.\\



% Figure environment removed

\noindent \textbf{The implementation of the neuron circuit.} The artificial neuron circuit consists of two main parts. Its detailed schematic including peripheral instruments is shown in Fig.~\ref{fig6}. The Ta$_2$O$_5$ module (red) is responsible for the detection of the neural spikes in the input signal. The VO$_2$ oscillator module (blue) generates the output spikes. The $V_{\rm drive}$ input voltage pattern of the neuron circuit is provided by adding the off-set spiking output of a Rigol DG5252 arbitrary waveform generator (AWG) and the noise output of a Siglent SDG1050 AWG via a LeCroy DA1850A differential amplifier. $V_{\rm drive}$ is independently recorded at channel B of a Picoscope 6424E digital storage oscilloscope (DSO). The output of the detector module, i.e., the voltage on the $R_{\rm s,T}$ series resistor, is measured via a Femto DLPCA-200 current amplifier (red) at channel C of the DSO. The operation point of the oscillator module is set by the constant voltage output of an Agilent 33220A AWG unit. This constant voltage and the voltage output of the detector module are added and amplified via a Tektronix AM502 differential amplifier (red-blue). Additionally, this unit also incorporates a 100~Hz low-pass filter to remove the noise from the input signal of the oscillator module, greatly improving the reliability of the latter. The output of the oscillator module, i.e., the voltage on the $R_{\rm s,V}$ series resistor, is measured via a second Femto DLPCA-200 current amplifier (blue) at channel A of the DSO. The feedback coupling (black) is incorporated to reset the neuron circuit after a detection event, enabling continuous operation. It is implemented through an SRS SR-235 amplifier located at the input of the circuit (red). This unit amplifies the output of the oscillator module and adds it to the input signal of the detector module with a negative polarity. The gain of the feedback must be optimized to match the device characteristics of the Ta$_{2}$O$_{5}$ memristor. A too strong reset pulse increases the risk that the detector circuit will idle in a too high $R_{\rm HRS}$ or exhibit only a smaller, short-lived resistance change upon the arrival of the next neural spike. In contrast, when the gain of the coupling is too low, a single reset pulse may not be sufficient to restore the optimal HRS. Consequently, a single neural spike detection event triggers multiple output spikes.



\section*{Acknowledgements}

This research was supported by the Ministry of Culture and Innovation and the National Research, Development and Innovation Office within the Quantum Information National Laboratory of Hungary (Grant No. 2022-2.1.1-NL-2022-00004), and the NKFI K143169, K143282 and TKP2021-NVA-03 grants. Project no. 963575 has been implemented with the support provided by the Ministry of Culture and Innovation of Hungary from the National Research, Development and Innovation Fund, financed under the KDP-2020 funding scheme. L.P. and Z.B. acknowledge the the support of the Bolyai J\'{a}nos Research Scholarship of the Hungarian Academy of Sciences. J.L., M.C. and N.J.O. acknowledge the financial support of the Werner Siemens Stiftung.


\section*{Author contributions}

The Ta$_{2}$O$_{5}$-based temporal pattern recognition experiments were developed and performed by D.M. The experiments with the combined Ta$_{2}$O$_{5}$ and VO$_{2}$ circuit were carried out by D.M., T.N.T. and R.K. The simulations were initiated by the work of P.B. and were extended and finalized by D.M. The Ta$_{2}$O$_{5}$ memristors were developed and fabricated by M.C. and N.J.O. in the group of J.L. The VO$_{2}$ devices were developed and manufactured by L.P., T.N.T. and G.M. in the group of J.V. The project was conceived and supervised by A.H. The manuscript was written by M.C, D.M., T.N.T. and A.H. All authors contributed to the discussion of the results.

%\bibliographystyle{naturemag}
\bibliography{References}

\end{document}