\section{Introduction}
%\textcolor{red}{Paragraph1: Importance of recommender systems (and citing related works)}\\
Recommender systems achieve much more attention today due to the need for personalization in a wide range of applications. They aim to consider the user preferences to suggest the most suitable item for the user.

One common classification of the recommender systems is based on their way of extracting users' preferences. It classifies the recommender systems as collaborative
filter-based, content-based, or hybrid. Through collaborative filtering, numerous users' preferences are collected to estimate the user's response. To predict and recommend new items to a user, the similarity of the user's history is compared with existing users, and the predictions are made based on their existing ratings. This approach's effectiveness heavily depends on the presence of substantial prior interactions between users and items. Therefore, in the cold-start situation where users and items have a sparse history of interactions, they could not perform well.

To solve the cold starting problem, content-based systems were introduced \cite{mooney2000content,narducci2016concept}. They use the side information of the users and items to make recommendations. They rely on the similarity of a user with items. Therefore, they cannot consider the user history of items to calculate user preferences. Also, they cannot be efficient if the side information is unavailable for various reasons, including privacy. Although hybrid systems use collaboration and content information simultaneously, they do not adequately solve the mentioned challenges.
%\textcolor{red}{Paragraph2: Mentioning sequential recommendation as the important challenge of the recommender system (and citing related works)} \\

Sequential recommendation also plays a crucial role in a real-world application. The goal is to extract user preferences based on the sequence of user's interactions to predict more possible items that the user would have an interaction with. A significant challenge can be long-tailed interaction data in some real-world scenarios due to limited interactions by new users \cite{yin2020learning}. Several sequential recommendation problems can benefit from transformer-based approaches, including SASRec \cite{kang2018self} and BERT4Rec \cite{sun2019bert4rec}, which consistently capture dynamic behaviors. However, in real-world scenarios, the sequence length of the cold users is not long enough for these methods to be effective \cite{liu2021augmenting}.

%\textcolor{red}{Paraghraph3: Mentioning Cold start problem and its challenges (needing side information, not considering sequences, different sizes of interaction for users (both for cold and warm users), capturing dynamic features, needing graph patterns that are not explicitly defined)}\\
Cold-start can deteriorate recommender system performance and. The majority of existing cold-start methods rely on auxiliary information or knowledge from other domains. In a cold-start sequential recommendation, side information is absent, and the goal is to model dynamic user behavior even in short sequences. Sharing knowledge among users could also improve performance, especially for cold-users and cold-items. In response, meta-learning-based research has generated promising results against the cold-start problem in recent years \cite{lee2019melu}. Meta-learning algorithms could improve user recommendations by utilizing the meta-knowledge extracted from the users to enhance new user recommendations. Although meta-learning has many advantages, it has one fundamental disadvantage when used in recommender systems. It is rooted in the generalization meta-learning algorithms aim for. Since users have different preferences, there is a possibility that each user might have a similar preference, like most users, or a preference that is less popular. When designing a recommender system, it is important to take into account the needs of both minor and major users. In contrast, meta-learning approaches bias the model parameters by the major users and get stuck in local minima.

% \textcolor{red}{Paragraph4: Introduction of the proposed model} \\
In this paper, we propose a meta-learning-based sequential recommender system named ClusterReq that addresses the preceding problems. It utilizes the Model-Agnostic Meta-Learning algorithm (MAML) \cite {finn2017model}, to share knowledge among users and adapt quickly to new users with limited transactions. Hence, a personalized recommendation is provided for each user based on their item transaction history. Then, a robustly designed clustering approach modulates the network parameters to prevent local minima made by the major user group. As a result, the proposed model can take into account transitional dynamic preferences and changes in preferences even in short sequences. The contributions of this work can be summarized as follows: 

% \textcolor{red}{Paraghraph5: Summarizing the contribution of the paper}\\

\begin{itemize}
    % \item Formulate the cold start problem as a few-shot problem to take advantage of meta-learning.  

\item  Utilizing a meta-learning approach to design an architecture to address sequential recommendation without side information and allow fast adaptation for users who start cold.

\item Solving the challenge of meta-learning bias towards major users in recommendation systems by clustering users and modulating their parameters to avoid local minimums.

\item Achieving 16-39\% improvement in MRR, on three real-world benchmarks compared with the state-of-the-art.
\end{itemize}

% The {\it IJCAI--22 Proceedings} will be printed from electronic
% manuscripts submitted by the authors. These must be PDF ({\em Portable
%         Document Format}) files formatted for 8-1/2$''$ $\times$ 11$''$ paper.

% \subsection{Length of Papers}


% All paper {\em submissions} to the main track must have a maximum of six pages, plus at most one for references/acknowledgements/ethics statement. The seventh page cannot contain anything else (or you will be charged for it).

% The length rules may change for final camera-ready versions of accepted papers and differ between tracks. Some tracks may include only references in the last page, whereas others allow for any content in all pages. Similarly, some tracks allow you to buy a few extra pages should you want to, whereas others don't.

% If your paper is accepted, please carefully read the notifications you receive, and check the proceedings submission information website\footnote{\url{https://proceedings.ijcai.org/info}} to know how many pages you can finally use. That website holds the most up-to-date information regarding paper length limits at all times.


% \subsection{Word Processing Software}

% As detailed below, IJCAI has prepared and made available a set of
% \LaTeX{} macros and a Microsoft Word template for use in formatting
% your paper. If you are using some other word processing software, please follow the format instructions given below and ensure that your final paper looks as much like this sample as possible.
