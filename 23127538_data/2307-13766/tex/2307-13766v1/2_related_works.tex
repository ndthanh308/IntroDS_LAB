\section{Related Works}

%\noindent\textbf{\textcolor{red}{Sequential Recommendation}}\\
\textbf{Sequential Recommendation.}
The sequential recommender systems model user preferences based on the sequence provided for each individual. Users' preferences have been modeled in these systems using a variety of deep learning architectures in recent years. Users were embedded in a transition space in TransRec \cite{he2017translation} so that each user could be considered as a vector connecting sequence items. More recently, convolutional neural networks \cite{yuan2019simple,tang2018personalized} and recurrent neural networks \cite{hidasi2015session,hidasi2018recurrent} were used to find appropriate user embedding that takes into account the order of sequence items. Moreover, transformers solve the sequential recommendation problem more efficiently by prioritizing the entire sequence without forgetting earlier items in the sequence \cite{kang2018self,sun2019bert4rec}. Finally, graph neural networks were adopted to solve the problem by utilizing graphical information \cite{wang2021session,wu2019session}. The above-mentioned works have contributed to the effective extraction of more information from an interaction sequence. In contrast, they could not be effective if the user behavior sequence is short in a cold-start scenario.

% \noindent\textbf{\textcolor{red}{Meta-learning}}\\

% \noindent\textbf{\textcolor{red}{Meta-learning for Recommendation}}\\
\textbf{Meta-learning for Recommendation.}
It is possible to use meta-learning in few-shot learning due to its ability to adapt rapidly with only a small number of samples. Previous research has used MAML-based models to solve the user-cold start problem \cite{wei2020fast,lee2019melu}. To capture meta-path semantics, MetaHIN \cite{lu2020meta} integrates heterogeneous information networks (HINs) into MAML. To adapt well to different tasks and avoid local optimum, MAMO proposes different task-specific adaptation strategies \cite{dong2020mamo}. It introduces two memory arrays: a feature-specific memory for initializing shared parameters and a task-specific memory for guiding the model. Recently, an attempt was made by TaNP \cite{lin2021task} to solve MAML problems such as model sensitivity and local optima. It customizes global knowledge to task-related decoder parameters for estimating user preferences. To capture task dependency more effectively, PAML \cite{wang2021preference} considers user relations with defined palindrome paths among them. Therefore, model parameters can be more precisely tailored to capture diverse paths and interactions. More recently, \cite{pang2022pnmta} was developed to alleviate the problem of limitations of representing users imposed by the particular task setting of MAML-based recommender systems. Firstly, a pre-trained model was obtained with a non-meta-learning method. After that, an encoder modulator corrects the prior parameters for the meta-learning task. The authors in \cite{wang2021sequential} examined the problem of sequential recommendation and extracting transitional patterns among users' sequences without needing side information of users and items.

Several previous works \cite{dong2020mamo,lin2021task,wang2021preference,pang2022pnmta,song2021cbml} have attempted to address MAML's potential problems in recommendation systems. The problem relates specifically to the elimination of preferences belonging to minor users. These users are less prevalent in the training data than the majority of users. However, \cite{dong2020mamo,lin2021task,song2021cbml} primarily relies on k-means-based clustering, which is inappropriate for many scenarios where users could be divided into primary and minor groups. To cluster users, they also require auxiliary information from users and items. In many scenarios, such data does not exist. Lastly, clustering users based on considering interacting items independently could not be effective, especially in cold-start scenarios in which a limited number of items are available for a user.

The method in \cite{wang2021preference,pang2022pnmta} is based on modulating prior knowledge based on user embeddings. However, the modulation would be biased toward major users such that majors would collapse modulation for other minor clusters. In addition, they need side information like explicit friends in \cite{wang2021preference} to embed users properly. If limited data is available about explicit friends, joint inference methods could estimate the missing information \cite{ramezani2023joint}, but without that information modulation will not benefit learning.  Also, embedding users without considering the sequence of items or users' similarity would not be practical for sequential recommendations. It is worth noting that CBML \cite{song2021cbml} relies on computing gradients for users to cluster them. This can result in higher computational costs than other models. Additionally, CBML may require multiple support adaptation steps in more complex scenarios to ensure effective performance.


% \noindent\textbf{\textcolor{red}{Modulation in cold-start recommendation}}\\

% \LaTeX{} and Word style files that implement these instructions
% can be retrieved electronically. (See Appendix~\ref{stylefiles} for
% instructions on how to obtain these files.)

% \subsection{Layout}

% Print manuscripts two columns to a page, in the manner in which these
% instructions are printed. The exact dimensions for pages are:
% \begin{itemize}
%     \item left and right margins: .75$''$
%     \item column width: 3.375$''$
%     \item gap between columns: .25$''$
%     \item top margin---first page: 1.375$''$
%     \item top margin---other pages: .75$''$
%     \item bottom margin: 1.25$''$
%     \item column height---first page: 6.625$''$
%     \item column height---other pages: 9$''$
% \end{itemize}

% All measurements assume an 8-1/2$''$ $\times$ 11$''$ page size. For
% A4-size paper, use the given top and left margins, column width,
% height, and gap, and modify the bottom and right margins as necessary.

% \subsection{Format of Electronic Manuscript}

% For the production of the electronic manuscript, you must use Adobe's
% {\em Portable Document Format} (PDF). A PDF file can be generated, for
% instance, on Unix systems using {\tt ps2pdf} or on Windows systems
% using Adobe's Distiller. There is also a website with free software
% and conversion services: \url{http://www.ps2pdf.com}. For reasons of
% uniformity, use of Adobe's {\em Times Roman} font is strongly suggested.
% In \LaTeX2e{} this is accomplished by writing
% \begin{quote}
%     \mbox{\tt $\backslash$usepackage\{times\}}
% \end{quote}
% in the preamble.\footnote{You may want also to use the package {\tt
%             latexsym}, which defines all symbols known from the old \LaTeX{}
%     version.}

% Additionally, it is of utmost importance to specify the {\bf
%         letter} format (corresponding to 8-1/2$''$ $\times$ 11$''$) when
% formatting the paper. When working with {\tt dvips}, for instance, one
% should specify {\tt -t letter}.

% \subsection{Title and Author Information}

% Center the title on the entire width of the page in a 14-point bold
% font. The title must be capitalized using Title Case. Below it, center author name(s) in 12-point bold font. On the following line(s) place the affiliations.

% \subsubsection{Author Names}

% Each author name must be followed by:
% \begin{itemize}
%     \item A newline {\tt \textbackslash{}\textbackslash{}} command for the last author.
%     \item An {\tt \textbackslash{}And} command for the second to last author.
%     \item An {\tt \textbackslash{}and} command for the other authors.
% \end{itemize}

% \subsubsection{Affiliations}

% After all authors, start the affiliations section by using the {\tt \textbackslash{}affiliations} command.
% Each affiliation must be terminated by a newline {\tt \textbackslash{}\textbackslash{}} command. Make sure that you include the newline on the last affiliation too.

% \subsubsection{Mapping Authors to Affiliations}

% If some scenarios, the affiliation of each author is clear without any further indication (\emph{e.g.}, all authors share the same affiliation, all authors have a single and different affiliation). In these situations you don't need to do anything special.

% In more complex scenarios you will have to clearly indicate the affiliation(s) for each author. This is done by using numeric math superscripts {\tt \$\{\^{}$i,j, \ldots$\}\$}. You must use numbers, not symbols, because those are reserved for footnotes in this section (should you need them). Check the authors definition in this example for reference.

% \subsubsection{Emails}

% This section is optional, and can be omitted entirely if you prefer. If you want to include e-mails, you should either include all authors' e-mails or just the contact author(s)' ones.

% Start the e-mails section with the {\tt \textbackslash{}emails} command. After that, write all emails you want to include separated by a comma and a space, following the same order used for the authors (\emph{i.e.}, the first e-mail should correspond to the first author, the second e-mail to the second author and so on).

% You may ``contract" consecutive e-mails on the same domain as shown in this example (write the users' part within curly brackets, followed by the domain name). Only e-mails of the exact same domain may be contracted. For instance, you cannot contract ``person@example.com" and ``other@test.example.com" because the domains are different.


% \subsubsection{Blind Review}

% In order to make blind reviewing possible, authors must omit their
% names and affiliations when submitting the paper for review. In place
% of names and affiliations, provide a list of content areas. When
% referring to one's own work, use the third person rather than the
% first person. For example, say, ``Previously,
% Gottlob~\shortcite{gottlob:nonmon} has shown that\ldots'', rather
% than, ``In our previous work~\cite{gottlob:nonmon}, we have shown
% that\ldots'' Try to avoid including any information in the body of the
% paper or references that would identify the authors or their
% institutions. Such information can be added to the final camera-ready
% version for publication.

% \subsection{Abstract}

% Place the abstract at the beginning of the first column 3$''$ from the
% top of the page, unless that does not leave enough room for the title
% and author information. Use a slightly smaller width than in the body
% of the paper. Head the abstract with ``Abstract'' centered above the
% body of the abstract in a 12-point bold font. The body of the abstract
% should be in the same font as the body of the paper.

% The abstract should be a concise, one-paragraph summary describing the
% general thesis and conclusion of your paper. A reader should be able
% to learn the purpose of the paper and the reason for its importance
% from the abstract. The abstract should be no more than 200 words long.

% \subsection{Text}

% The main body of the text immediately follows the abstract. Use
% 10-point type in a clear, readable font with 1-point leading (10 on
% 11).

% Indent when starting a new paragraph, except after major headings.

% \subsection{Headings and Sections}

% When necessary, headings should be used to separate major sections of
% your paper. (These instructions use many headings to demonstrate their
% appearance; your paper should have fewer headings.). All headings should be capitalized using Title Case.

% \subsubsection{Section Headings}

% Print section headings in 12-point bold type in the style shown in
% these instructions. Leave a blank space of approximately 10 points
% above and 4 points below section headings.  Number sections with
% arabic numerals.

% \subsubsection{Subsection Headings}

% Print subsection headings in 11-point bold type. Leave a blank space
% of approximately 8 points above and 3 points below subsection
% headings. Number subsections with the section number and the
% subsection number (in arabic numerals) separated by a
% period.

% \subsubsection{Subsubsection Headings}

% Print subsubsection headings in 10-point bold type. Leave a blank
% space of approximately 6 points above subsubsection headings. Do not
% number subsubsections.

% \paragraph{Titled paragraphs.} You should use titled paragraphs if and
% only if the title covers exactly one paragraph. Such paragraphs should be
% separated from the preceding content by at least 3pt, and no more than
% 6pt. The title should be in 10pt bold font and ended with a period.
% After that, a 1em horizontal space should follow the title before
% the paragraph's text.

% In \LaTeX{} titled paragraphs should be typeset using
% \begin{quote}
%     {\tt \textbackslash{}paragraph\{Title.\} text} .
% \end{quote}

% \subsection{Special Sections}

% \subsubsection{Appendices}

% Appendices are optional. Appendices must appear after the main
% content. Appendix sections must use letters instead of arabic numerals. In \LaTeX,  you can use the {\tt \textbackslash{}appendix} command to achieve this followed by  {\tt \textbackslash section\{Appendix\}} for your appendix sections.

% \subsubsection{Ethical Statement}

% Ethical Statement is optional. You may include an Ethical Statement to discuss  the ethical aspects and implications of your research. The section should be titled \emph{Ethical Statement} and be typeset like any regular section but without being numbered. This section may be placed on the References page.

% Use
% \begin{quote}
%     {\tt \textbackslash{}section*\{Ethical Statement\}}
% \end{quote}

% \subsubsection{Acknowledgements}

% Acknowledgements are optional. In the camera-ready version you may include an unnumbered acknowledgments section, including acknowledgments of help from colleagues, financial support, and permission to publish. This is not allowed in the version submitted for blind review. If present, acknowledgements must be in a dedicated, unnumbered section appearing after all regular sections but before references.  This section may be placed on the References page.

% Use
% \begin{quote}
%     {\tt \textbackslash{}section*\{Acknowledgements\}}
% \end{quote}
% to typeset the acknowledgements section in \LaTeX{}.



% \subsubsection{References}

% The references section is headed ``References'', printed in the same
% style as a section heading but without a number. A sample list of
% references is given at the end of these instructions. Use a consistent
% format for references. The reference list should not include publicly unavailable work.

% \subsubsection{Order of Sections}
% Sections should be arranged in the following order:
% \begin{enumerate}
%     \item Main content sections (numbered)
%     \item Appendices (optional, numbered using capital letters)
%     \item Ethical statement (optional, unnumbered)
%     \item Acknowledgements (optional, unnumbered)
%     \item References (required, unnumbered)
% \end{enumerate}



% \subsection{Citations}

% Citations within the text should include the author's last name and
% the year of publication, for example~\cite{gottlob:nonmon}.  Append
% lowercase letters to the year in cases of ambiguity.  Treat multiple
% authors as in the following examples:~\cite{abelson-et-al:scheme}
% or~\cite{bgf:Lixto} (for more than two authors) and
% \cite{brachman-schmolze:kl-one} (for two authors).  If the author
% portion of a citation is obvious, omit it, e.g.,
% Nebel~\shortcite{nebel:jair-2000}.  Collapse multiple citations as
% follows:~\cite{gls:hypertrees,levesque:functional-foundations}.
% \nocite{abelson-et-al:scheme}
% \nocite{bgf:Lixto}
% \nocite{brachman-schmolze:kl-one}
% \nocite{gottlob:nonmon}
% \nocite{gls:hypertrees}
% \nocite{levesque:functional-foundations}
% \nocite{levesque:belief}
% \nocite{nebel:jair-2000}

% \subsection{Footnotes}

% Place footnotes at the bottom of the page in a 9-point font.  Refer to
% them with superscript numbers.\footnote{This is how your footnotes
%     should appear.} Separate them from the text by a short
% line.\footnote{Note the line separating these footnotes from the
%     text.} Avoid footnotes as much as possible; they interrupt the flow of
% the text.
