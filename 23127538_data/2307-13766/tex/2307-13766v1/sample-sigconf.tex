%%
%% This is file `sample-authordraft.tex',
%% generated with the docstrip utility.
%%
%% The original source files were:
%%
%% samples.dtx  (with options: `authordraft')
%% 
%% IMPORTANT NOTICE:
%% 
%% For the copyright see the source file.
%% 
%% Any modified versions of this file must be renamed
%% with new filenames distinct from sample-authordraft.tex.
%% 
%% For distribution of the original source see the terms
%% for copying and modification in the file samples.dtx.
%% 
%% This generated file may be distributed as long as the
%% original source files, as listed above, are part of the
%% same distribution. (The sources need not necessarily be
%% in the same archive or directory.)
%%
%% Commands for TeXCount
%TC:macro \cite [option:text,text]
%TC:macro \citep [option:text,text]
%TC:macro \citet [option:text,text]
%TC:envir table 0 1
%TC:envir table* 0 1
%TC:envir tabular [ignore] word
%TC:envir displaymath 0 word
%TC:envir math 0 word
%TC:envir comment 0 0
%%
%%
%% The first command in your LaTeX source must be the \documentclass command.
%\documentclass[sigconf,authordraft]{acmart}
\documentclass[sigconf, nonacm]{acmart}
% \usepackage[table]{xcolor}
\usepackage{tabularray}
\usepackage{soul}
\UseTblrLibrary{siunitx, varwidth}
\usepackage{etoolbox}
\newrobustcmd\B{\DeclareFontSeriesDefault[rm]{bf}{b}\bfseries} 
%% NOTE that a single column version may required for 
%% submission and peer review. This can be done by changing
%% the \doucmentclass[...]{acmart} in this template to 
%% \documentclass[manuscript,screen]{acmart}
%% 
%% To ensure 100% compatibility, please check the white list of
%% approved LaTeX packages to be used with the Master Article Template at
%% https://www.acm.org/publications/taps/whitelist-of-latex-packages 
%% before creating your document. The white list page provides 
%% information on how to submit additional LaTeX packages for 
%% review and adoption.
%% Fonts used in the template cannot be substituted; margin 
%% adjustments are not allowed.

%%
%% \BibTeX command to typeset BibTeX logo in the docs
\AtBeginDocument{%
  \providecommand\BibTeX{{%
    \normalfont B\kern-0.5em{\scshape i\kern-0.25em b}\kern-0.8em\TeX}}}

%% Rights management information.  This information is sent to you
%% when you complete the rights form.  These commands have SAMPLE
%% values in them; it is your responsibility as an author to replace
%% the commands and values with those provided to you when you
%% complete the rights form.
\setcopyright{acmcopyright}
\copyrightyear{2023}
\acmYear{2023}
\acmDOI{XXXXXXX.XXXXXXX}

%% These commands are for a PROCEEDINGS abstract or paper.
\acmConference[CIKM '23]{The 32nd ACM International
Conference on Information and Knowledge Management}{June 03--05,
  2018}{Woodstock, NY}
%
%  Uncomment \acmBooktitle if th title of the proceedings is different
%  from ``Proceedings of ...''!
%
%\acmBooktitle{Woodstock '18: ACM Symposium on Neural Gaze Detection,
%  June 03--05, 2018, Woodstock, NY} 
\acmPrice{15.00}
\acmISBN{978-1-4503-XXXX-X/18/06}


%%
%% Submission ID.
%% Use this when submitting an article to a sponsored event. You'll
%% receive a unique submission ID from the organizers
%% of the event, and this ID should be used as the parameter to this command.
%%\acmSubmissionID{123-A56-BU3}

%%
%% For managing citations, it is recommended to use bibliography
%% files in BibTeX format.
%%
%% You can then either use BibTeX with the ACM-Reference-Format style,
%% or BibLaTeX with the acmnumeric or acmauthoryear sytles, that include
%% support for advanced citation of software artefact from the
%% biblatex-software package, also separately available on CTAN.
%%
%% Look at the sample-*-biblatex.tex files for templates showcasing
%% the biblatex styles.
%%

%%
%% For managing citations, it is recommended to use bibliography
%% files in BibTeX format.
%%
%% You can then either use BibTeX with the ACM-Reference-Format style,
%% or BibLaTeX with the acmnumeric or acmauthoryear sytles, that include
%% support for advanced citation of software artefact from the
%% biblatex-software package, also separately available on CTAN.
%%
%% Look at the sample-*-biblatex.tex files for templates showcasing
%% the biblatex styles.
%%

%%
%% The majority of ACM publications use numbered citations and
%% references.  The command \citestyle{authoryear} switches to the
%% "author year" style.
%%
%% If you are preparing content for an event
%% sponsored by ACM SIGGRAPH, you must use the "author year" style of
%% citations and references.
%% Uncommenting
%% the next command will enable that style.
%%\citestyle{acmauthoryear}



%%
%% end of the preamble, start of the body of the document source.
\begin{document}

%%
%% The "title" command has an optional parameter,
%% allowing the author to define a "short title" to be used in page headers.
% \title{Sequential User Cold-Start Recommendation with Modulation based Meta-Learning }

\title{ClusterSeq: Enhancing Sequential Recommender Systems with Clustering based Meta-Learning}

%%
%% The "author" command and its associated commands are used to define
%% the authors and their affiliations.
%% Of note is the shared affiliation of the first two authors, and the
%% "authornote" and "authornotemark" commands
%% used to denote shared contribution to the research.
% \author{Ben Trovato}
% \authornote{Both authors contributed equally to this research.}
% \email{trovato@corporation.com}
% \orcid{1234-5678-9012}
% \author{G.K.M. Tobin}
% \authornotemark[1]
% \email{webmaster@marysville-ohio.com}
% \affiliation{%
%   \institution{Institute for Clarity in Documentation}
%   \streetaddress{P.O. Box 1212}
%   \city{Dublin}
%   \state{Ohio}
%   \country{USA}
%   \postcode{43017-6221}
% }

\author{Mohammmadmahdi Maheri}
\affiliation{%
  \institution{Imperial College London}
  \streetaddress{}
  \city{London}
  \country{United Kingdom}}
\email{m.maheri23@imperial.ac.uk}

\author{Reza Abdollahzadeh}
\affiliation{%
  \institution{Sharif University of Technology}
  \city{Tehran}
  \country{Iran}
}
\email{re.abd@student.sharif.edu}

\author{Bardia Mohammadi}
\affiliation{%
  \institution{Sharif University of Technology}
  \city{Tehran}
  \country{Iran}
}
\email{bardia.mohammadi@sharif.edu}

\author{Mina Rafiei}
\affiliation{%
  \institution{Sharif University of Technology}
  \city{Tehran}
  \country{Iran}
}
\email{m.rafiei@sharif.edu}

\author{Jafar Habibi}
\affiliation{%
  \institution{Sharif University of Technology}
  \city{Tehran}
  \country{Iran}
}
\email{jhabibi@sharif.edu}

\author{Hamid R. Rabiee}
\affiliation{%
  \institution{Sharif University of Technology}
  \city{Tehran}
  \country{Iran}
}
\email{rabiee@sharif.edu}



%%
%% By default, the full list of authors will be used in the page
%% headers. Often, this list is too long, and will overlap
%% other information printed in the page headers. This command allows
%% the author to define a more concise list
%% of authors' names for this purpose.
\renewcommand{\shortauthors}{Maheri, et al.}
\newcommand{\rafiei}[1]{\textcolor{red}{rafiei: #1}}

%%
%% The abstract is a short summary of the work to be presented in the
%% article.
% \begin{abstract}
% In real-world applications, sequential recommendation systems suffer from user cold-starts because determining user preferences is difficult based on a few interactions. To address the cold-start problem, several previous studies have combined meta-learning with user and item-side information. However, despite promising results, they have some fundamental issues with modeling user preferences dynamics, especially for "minor users" who have different preferences than more common or "major users." In this paper, we propose a Meta-Learning Clustering-Based Sequential Recommender System, named ClusterSeq, that incorporates dynamic information in the sequence of users to predict the following items more accurately without side information. In this model, minor users' preferences would be considered without being collapsed by major users. In addition, users will be clustered to take advantage of other users' knowledge in the same cluster. In different experimental settings, ClusterSeq has been validated against multiple benchmark datasets. Based on empirical results, ClusterSeq consistently outperforms several state-of-the-art meta-learning recommenders. Compared to other existing meta-learning approaches, the proposed method exhibits a significant improvement of 16-39\% in Mean Reciprocal Rank (MRR).

% \end{abstract}

\begin{abstract}
In practical scenarios, the effectiveness of sequential recommendation systems is hindered by the user cold-start problem, which arises due to limited interactions for accurately determining user preferences. Previous studies have attempted to address this issue by combining meta-learning with user and item-side information. However, these approaches face inherent challenges in modeling user preference dynamics, particularly for "minor users" who exhibit distinct preferences compared to more common or "major users." To overcome these limitations, we present a novel approach called ClusterSeq, a Meta-Learning Clustering-Based Sequential Recommender System. ClusterSeq leverages dynamic information in the user sequence to enhance item prediction accuracy, even in the absence of side information. This model preserves the preferences of minor users without being overshadowed by major users, and it capitalizes on the collective knowledge of users within the same cluster. Extensive experiments conducted on various benchmark datasets validate the effectiveness of ClusterSeq. Empirical results consistently demonstrate that ClusterSeq outperforms several state-of-the-art meta-learning recommenders. Notably, compared to existing meta-learning methods, our proposed approach achieves a substantial improvement of 16-39\% in Mean Reciprocal Rank (MRR).

\end{abstract}

%%
%% The code below is generated by the tool at http://dl.acm.org/ccs.cfm.
%% Please copy and paste the code instead of the example below.
%%
% \begin{CCSXML}
% <ccs2012>
%  <concept>
%   <concept_id>10010520.10010553.10010562</concept_id>
%   <concept_desc>Information systems~Recommender systems</concept_desc>
%   <concept_significance>500</concept_significance>
%  </concept>
%  <concept>
%   <concept_id>10010520.10010575.10010755</concept_id>
%   <concept_desc>Computer systems organization~Redundancy</concept_desc>
%   <concept_significance>300</concept_significance>
%  </concept>
%  <concept>
%   <concept_id>10010520.10010553.10010554</concept_id>
%   <concept_desc>Computer systems organization~Robotics</concept_desc>
%   <concept_significance>100</concept_significance>
%  </concept>
%  <concept>
%   <concept_id>10003033.10003083.10003095</concept_id>
%   <concept_desc>Networks~Network reliability</concept_desc>
%   <concept_significance>100</concept_significance>
%  </concept>
% </ccs2012>
% \end{CCSXML}

% \ccsdesc[500]{Information systems~Recommender systems}
% \ccsdesc[300]{Computer systems organization~Redundancy}
% \ccsdesc{Computer systems organization~Robotics}
% \ccsdesc[100]{Networks~Network reliability}

%%
%% Keywords. The author(s) should pick words that accurately describe
%% the work being presented. Separate the keywords with commas.
\keywords{recommender systems, cold-start, meta-learning}

%% A "teaser" image appears between the author and affiliation
%% information and the body of the document, and typically spans the
%% page.
% \begin{teaserfigure}
%   % Figure removed
%   \caption{Seattle Mariners at Spring Training, 2010.}
%   \Description{Enjoying the baseball game from the third-base
%   seats. Ichiro Suzuki preparing to bat.}
%   \label{fig:teaser}
% \end{teaserfigure}

% \received{20 February 2007}
% \received[revised]{12 March 2009}
% \received[accepted]{5 June 2009}

%%
%% This command processes the author and affiliation and title
%% information and builds the first part of the formatted document.
\maketitle


\section{Introduction}
Deep learning models have been widely used in many applications.
For example, BERT~\citep{devlin_bert_2019}, GPT-3~\citep{brown_language_2020}, and T5~\citep{raffel_exploring_2020} achieved state-of-the-art~(SOTA) results on different natural language processing~(NLP) tasks. 
For computer vision~(CV), Transformer-like models such as ViT~\citep{dosovitskiy_image_2021} and Swin Transformer~\citep{liu_swin_2021} deliver excellent accuracy performance upon multiple tasks. 


At the same time, training deep learning models has been a critical problem troubling the community due to the long training time, especially for those large models with billions of parameters~\citep{brown_language_2020}. 
In order to enhance the training efficiency, researchers propose some manually designed parallel training strategies~\citep{narayanan_efficient_2021,shazeer_mesh-tensorflow_2018,xu_gspmd_2021}. 
However, selecting, tuning, and combining these strategies require extensive domain knowledge in deep learning models and hardware environments. With the increasing diversity of modern hardware architectures~\cite{flynn_very_1966,flynn_computer_1972} and the rapid development of deep learning models, these manually designed approaches are bringing heavier burdens to developers. 
Hence, \emph{automatic parallelism} is introduced to automate the parallel strategy searching for training models.


There are two main categories of parallelism in deep learning models: inter-layer parallelism~\citep{huang_gpipe_2019,narayanan_pipedream_2019,narayanan_memory-efficient_2021,fan_dapple_2021,li_chimera_2021,lepikhin_gshard_2021,du_glam_2022,fedus_switch_2022} and intra-layer parallelism~\citep{li_pytorch_2020,narayanan_efficient_2021,rasley_deepspeed_2020,fairscale_authors_fairscale_2021}. 
Inter-layer parallelism partitions the model into disjoint sets on different devices without slicing tensors. 
Alternatively, intra-layer parallelism partitions tensors in a layer along one or more axes and distributes them across different devices.


Current automatic parallelism techniques focus on optimizing strategies within these two categories. However, they treat these two categories separately. 
Some methods~\citep{zhao_vpipe_2022,jia_exploring_2018,cai_tensoropt_2022,wang_supporting_2019,jia_beyond_2019,schaarschmidt_automap_2021,liu_colossal-auto_2023} overlook potential opportunities for inter- or intra-layer parallelism, the others optimize inter- and intra-layer parallelism hierarchically and sequentially~\citep{narayanan_pipedream_2019,fan_dapple_2021,he_pipetransformer_2021,tarnawski_efficient_2020,tarnawski_piper_2021,zheng_alpa_2022}. 
As a result, current automatic parallelism techniques often fail to achieve the global optima and instead become trapped in local optima. 
Therefore, a unified inter- and intra-layer approach is needed to enhance the effectiveness of automatic parallelism.


This paper aims to find the optimal parallelism strategy while simultaneously considering inter- and intra-layer parallelism. 
It enables us to search in a more extensive strategy space where the globally optimal solution lurk. 
However, unifying inter- and intra-layer parallelism in automatic parallelism brings us two challenges. 
Firstly, to adopt a unified perspective on the inter- and intra-layer automatic parallelism, we should not formalize them with separate formulations as prior works. Therefore, how can we express these parallelism strategies in a unified formulation? 
Secondly, previous methods take a long time to obtain the solution with a limited strategy space. Therefore, how can we ensure that the best solution can be obtained in a reasonable time while expanding the strategy space?


To solve the above challenges, we propose UniAP. For the first challenge, UniAP adopts the mixed integer quadratic programming~(MIQP)~\citep{lazimy_mixed_1982} to search for the globally optimal parallel strategy automatically. 
It unifies the inter- and intra-layer automatic parallelism in a single MIQP formulation. 
For the second challenge, our complexity analysis and experimental results show that UniAP can obtain the globally optimal solution in a significantly shorter time.


The contributions of this paper are summarized as follows: 
\begin{itemize}
    \item We propose UniAP, the first framework to unify inter- and intra-layer automatic parallelism in model training.
    \item The optimal parallel strategies discovered by UniAP exhibit scalability on training throughput and strategy searching time.
    \item The experimental results show that UniAP speeds up model training on four Transformer-like models by up to 1.70$\times$ and reduces the strategy searching time by up to 16$\times$, compared with the SOTA method.
\end{itemize}

% Figure environment removed

\section{Related Works}
\subsection{Density-aware Dehazing Methods}

In recent years, several methods \cite{zhang2021hierarchical,deng2020hardgan,guo2022image,yang2022self,wang2021haze,yi2022two,yeperceiving} have attempted to improve the dehazing performance by enabling the network to perceive haze density.

\subsubsection{Density-awareness via estimating T-map} Haze density is influenced by several factors and is inversely proportional to T-map, so some methods learn density information by estimating T-map. Lou et al. \cite{lou2020integrating} predict a T-map first for nighttime image dehazing. Zhang et al. \cite{zhang2021hierarchical} estimate a low-resolution T-map and then jointly input the feature map and the estimated T-map to a Laplacian pyramid decoder to achieve a restored image. Yang et al. \cite{yang2022self} propose a semi-supervised method that does not require paired data. The method estimates T-map, scattering coefficient, and depth to reconstruction hazy images and restores clear images. However, these methods require additional labeled data and might be inaccurate due to the complexity of practical scenes \cite{li2017aod}.
% The Haze-Aware Feature Distillation (HARD) module is designed in \cite{deng2020hardgan}, which introduces two dual-channel maps to describe the atmospheric brightness and pixel-wise spatial information of each feature channel respectively, and calculates a haze aware map through an InstanceNorm followed by a sigmoid layer. Finally it fuses the above three factors.

% Figure environment removed

\subsubsection{Density-awareness via extracting density features directly} Research works \cite{deng2020hardgan,chen2020unsupervised,yeperceiving} directly learn haze density information without estimating a T-map. Deng et al. \cite{deng2020hardgan} design a Haze-Aware Representation Distillation (HARD) module to extract global brightness and a haze-aware map. Chen et al. \cite{chen2020unsupervised} propose an attention mechanism based on dark channel prior to describe haze concentration. However, not estimating the T-map would result in a lack of a comparator to measure density. Generating intermediate results and using the information contained therein can address this issue.

\subsection{Dehazing Methods with Intermediate Results}
Considering the difficulty of recovering images directly from the haze input, dehazing methods \cite{bai2022self,chen2021desmokenet,yeperceiving,hong2022uncertainty} which generate intermediate results (or one result) inside the network to facilitate the dehazing process are proposed. Bai et al. \cite{bai2022self} first generate a reference image by a deep pre-dehazer, and then develop a progressive feature fusion module to fuse the hazy and reference features, which achieves high metrics on several datasets. Chen et al. \cite{chen2021desmokenet} first remove light and thick smoke by a Smoke Remove Network (SRN) to gain a coarse output, which is concatenated with the original input and fed to a Pixel Compensation Network (PCN) to recover the missing pixels in the thick smoke. Hong et al. \cite{hong2022uncertainty} propose an Uncertainty-Driven Dehazing Network. In this method, intermediate results are together generated with uncertainty maps for uncertainty features extraction. Ye et al. \cite{yeperceiving} also pre-generate a pseudo-haze-free image. The hazy input and the pseudo-haze-free image are concatenated to estimate a Density Encoding Matrix describing the relationship between haze density and absolute position and mixed up to the following deep layers.

Despite the above methods extracting feature from intermediate results, they do not fully consider the differences between these results and the haze inputs, especially the differences in haze density. Simple concatenation \cite{chen2021desmokenet,bai2022self} or linear summation \cite{yeperceiving} might lead the networks to rely on the uncertain learning process and lose the capture of information about the differences between the two images. In addition, the lack of a targeted design that addresses the relationship between the intermediate results and the original input leads the extracted features not fine enough and limits the dehazing performance.

Our DFR-Net improves on the aforementioned methods by exploring and refining density features through the utilization of density differences between a generated proposal image and the hazy input, thereby achieving an awareness of haze density and superior dehazing performance.


\section{The proposed model}
In this section, the proposed model will be discussed in detail by describing its ability to address the previously mentioned challenges and their solutions. Specifically, the model is designed to satisfy the following goals: 1) Sharing knowledge among different users to improve performance on cold-start users, 2) Capturing short-range and long-range dynamics in users' preferences, 3) Optimizing the model to perform accurately on cold-start users, and 4) Avoiding local minima in shared parameters and considering major and minor preferences, simultaneously.

\subsection{Problem Setup}
We assume there exists a set of users $U$ divided between training and testing ($U_{train} \cap U_{test} = \emptyset$), and items $I$. The model's goal is to predict the next-item preference score of all items and recommend top-N items by inputting a short chronological order sequence of a user's preference $Seq_u = (i_{u,1},i_{u,2},...,i_{u,k-1})$. Note that the items and the users have no auxiliary information, and embedding is calculated only by their ID. 
%\rafiei{ as a result is not necessary} as a result.

To mimic the cold-start scenario, the users with a limited number of transactions $(K)$ will exist in the $U_{test}$. Based on the proposed optimization strategy and architecture, the model's parameters will adapt quickly to the user's short available transactions. The model's performance for predicting the next item of test users will be reported.  

\subsection{Proposed Architecture}
The following sections provide detailed information about each module's components and its significance. We will first discuss the meta-learning approach to overcoming the few-shot problem in section \ref{subsection:few-shot}. In section \ref{subsection:model_overview}, we will provide an overview of the proposed architecture. The details of the sub-modules will be described in sections \ref{subsection:dynamic_transition} and \ref{subsection:clustering_module}. Finally, we describe how to optimize the model's parameters in the meta-learning setting in section \ref{subsection:optimization}.

\subsubsection{Few-shot Recommendation}
\label{subsection:few-shot}
Meta-learning could transfer knowledge from data-rich users to cold users. By considering each user as a separate task, meta-learning extracts common knowledge among users. Therefore, a cold-start user needs much less data to converge to its optimal parameters, and user preferences could be detected with a few-shot approach. When using meta-learning, you must define support and query sets. 

The support set of a user is used to adapt shared knowledge (parameters) to the user. Thus, the user's personalized parameter will be adapted to the support set. Afterward, the user's query set is used to evaluate the adaptation. We assume users with a constant (K) sequence length as:  

\[
I_{1}
\to
I_{2}
\to
...
\to
I_{k-1}
\to
I_{k}
\]

We define each user $u$ as a task $\tau_{u}$, and its sequence is divided into support and query sets. First, $k-1$ items in a sequence are considered support-set, and the last item of the sequence is in query-set. Meta-learning tries to predict the query set by adapting to the previous items in the support set. 

The approach to sampling training data also needs to be designed. Users usually have a wide range of history lengths that should be used efficiently to extract sequential patterns. Users' history of transactions should also be appropriately utilized, even in a short user sequence, to improve performance for cold-start users. Sampling the training data should mimic cold-start users. Therefore, it only samples a short sequence of transactions for each user. First, a user is sampled from $U_{train}$ to construct a meta-training task. All user interactions will then be limited to a sequence of length $K$. In this way, the training is more like a test scenario in which test users will be considered.

\subsubsection{Model Overview}
\label{subsection:model_overview}

% Figure environment removed

In terms of capturing both short-range and long-range dynamics of users' preferences, and considering all major and minor groups of users, we designed a dynamic transition modeling and clustering module. This will be explained in the following sections \ref{subsection:dynamic_transition} and \ref{subsection:clustering_module} respectively.

Figure \ref{fig:overallarchitecture} shows the overall view of our proposed model. First, the Dynamic Transition model converts item transitions to user embeddings. Then, using a clustering module, we calculate two soft clustering assignments. These assignments are used to calculate loss values and condition user embeddings for the next item prediction. Finally, we score positive and negative item predictions and propagate it backwards on the marginal ranking loss.


\subsubsection{Dynamic Transition Modeling}
%\rafiei{I will add comments after completion}
\label{subsection:dynamic_transition}
Considering sequential patterns and extracting dynamic patterns needs a particular architecture in few-shot settings. Also, the model needs to detect temporal and long-term user preference changes. As shown in Figure \ref{fig:modelarchitecture}, attention-based recurrent neural network architecture is proposed to satisfy these goals.

% Figure environment removed

This architecture consists of an encoder and a decoder incorporating an attention mechanism. In the first stage of the process, the user transition sequence $u_i$ is fed to the encoder, which is essentially a Gated Recurrent Unit (GRU) gate \cite{chung2014empirical}:
\begin{equation}
\begin{aligned}
    \label{eq:dyn_trans_enc}
    o_{i, enc}^{(k)} &= f_{enc}(u_i^{(k-1)}, h_{i, enc}^{(k-1)}) \\
    h_{i, enc}^{(k)} &= h_{enc}(u_i^{(k-1)}, h_{i, enc}^{(k-1)})
\end{aligned}
\end{equation}
$f_{enc}$ and $h_{enc}$ are GRU functions to generate its output and hidden vector, respectively. We consider $O_{i, enc}\in \mathbb{R}^{K\times D} = \{o_{i, enc}^{(k)}\}$ the primary context vector of this sequence. $h_{i, enc}^{(0)}$ is initialized by a zero vector. \\
As for the decoder, each sequence item passes through the network in different iterations. We use the decoder output vector $o_{i, dec}^{(k-1)}$ and hidden value $h_{i, dec}^{(k-1)}$ of the previous iteration to apply attention to the context vector for the current one:
\begin{equation}
\begin{aligned}
    \label{eq:dyn_trans_dec_attn}
    attn\_{w_i}^{(k)} &= softmax(fc_1(o_{i, dec}^{(k-1)};h_{i, dec}^{(k-1)})) \\
    attn_i^{(k)} &= attn_{w_i}^{(k)} \cdot O_{i, enc}
\end{aligned}
\end{equation}

Here, the semicolon sign $;$ shows the concatenation of vectors, $fc_1 \colon \mathbb{R}^{2D} \mapsto \mathbb{R}^{K}$ is a one-layer fully connected network and $attn\_w_i^{(k)} \in \mathbb{R}^{K}$ indicates attentional weights to be applied to the context vector concerning the previous $k-1$ items. The following will calculate the decoder's output and hidden vectors:

\begin{equation}
\begin{aligned}
    \label{eq:dyn_trans_dec_gru}
    X_i^{k} &= relu(fc_2(o_{i, dec}^{(k-1)};attn_i^{(k)})) \\ 
    o_{i, dec}^{(k)} &= softmax(fc_3(f_{dec}(X_i^{k}, h_{i, dec}^{(k-1)}))) \\
    h_{i, dec}^{(k)} &= h_{dec}(h_{i, dec}^{(k-1)})
\end{aligned}
\end{equation}

We consider the last encoder hidden vector $h_{i, enc}^{(K)}$ as $h_{i, dec}^{(0)}$ and a zero vector as $o_{i, dec}^{(0)}$. Also, $fc_2 \colon \mathbb{R}^{2D} \mapsto \mathbb{R}^{D}$ and $fc_3 \colon \mathbb{R}^{D} \mapsto \mathbb{R}^{D}$ are one-layer fully connected networks and $f_{dec}$ and $h_{dec}$ are the output and hidden functions of the GRU gate.
% \textcolor{red}{It seems better to explain that the final output vector and hidden vector will produce user embedding, which will be used later in the clustering module,...}

%\noindent\textbf{\textcolor{red}{@reze: Please add figure of the model here}}\\

%\noindent\textbf{\textcolor{red}{@reze: Please write about the architecture of the model (NLP or Transformer) - formulation}}\\

\subsubsection{Clustering Module}
\label{subsection:clustering_module}
%\rafiei{I will add comments after completion}
Our goal should be to consider both major and minor preferences sufficiently, and avoid major users distorting the model parameters as shown in Figure \ref{fig:cluster_fig}. In order to accomplish this, a clustering model is developed. Based on the fact that only implicit interactions between users are available, the clustering must be based solely on these interactions. Clustering, however, would be beneficial only if the major cluster did not disrupt clustering by attracting all the points to its center.

% Figure environment removed

To address the problem of collapsing by major users, the proposed module is inspired by \cite{opochinsky2020k}. More specifically, it consists of a Graph Convolutional Network (GCN) to capture topological information from users' interaction data, combined with a K Auto Encoder (KAE) clustering module. The GCN network determines the topological clusters, and the auto-encoder chooses the encoding cluster based on the auto-encoded number. In order to accomplish this, we must first reconstruct user embeddings using $M$ randomly initialized autoencoders. Each autoencoder reconstructs the user embedding with the corresponding reconstruction loss. Therefore, each user's encoding cluster assignment is based on its reconstruction loss. The encoding clustering assignments are calculated according to equation \ref{eq:kae_cluster_assignment}.
% \textcolor{red}{\st{reze: This most definitely will change in the future}}\\
% \textcolor{red}{\st{mohammad: number of autoencoders that corresponds to the assignment is forgotten}}\\
% \textcolor{red}{\st{mohammad: d must be defined}}\\
\begin{equation}  
    \label{eq:kae_cluster_assignment}
    % c_u = arg\,min_{i=1}^M d(e_u,\hat{e}_u (i))
    c_{enc}^{u} = softmax([-d(e_u,\hat{e}_u (i))])
\end{equation}
$c_u$ is the assignment of user $u$. In addition, $e_u$ and $ \hat{e}_u (i)$ are embeddings of the user and their reconstructions by the $i$th autoencoder, respectively, and $d$ is the $L2$ distance norm. The reconstruction loss of the autoencoders should also be calculated. The loss is calculated based on \ref{eq:kae_loss}:

\begin{equation}
    \label{eq:kae_loss}
    \mathcal{L}_{rec} = L(\theta_1, ... , \theta_M) = min_i d(e_u,\hat{e}_u (i))
\end{equation}
Where $\theta_j$ is the parameters of the $j$th autoencoder.

Then, we construct a users' relation graph $G$ using user embedding cosine similarity:

\begin{equation}
\begin{aligned}
    \label{eq:user_relation_graph}
    \forall{u_i \in{U_{train}}}, \forall{(u_i, u_j)\in{G}}; \\ \{u_j\} = argmax_{A'\subset{U_{train}-{u_i}}, |A'|=n_{adj}} \\ {\frac{u_i.u_j}{|u_i||u_j|} + \sigma \sum_{i_1, i_2 \in shared(u_i,u_j)}}{\frac{i_1.i_2}{|i_1||i_2|})}
\end{aligned}
\end{equation}

% \textcolor{red}{\st{New way of reconstruction of the graph should be explained here}}

Here, for each user embedding $u_i$ we select $n_{adj}$ other user embeddings with the maximum user and item similarity values with $u_i$ to construct a graph $G$. As for the user similarity value, we compute the cosine similarity between user embeddings. As for item similarities, we calculate the sum of the cosine similarity values of all shared items between $u_i$ and $u_j$ ($shared(u_i,u_j)$). Here, $\sigma$ is a user-item balancing hyperparameter.

We assume the autoencoder with minimum reconstruction loss for user $u$ has $L = L_{enc} + L_{dec}$ layers and $H_u^{(l)}$ is the representation learned by layer $l$ of this autoencoder.
Here, $H_u^{(0)}$ is the user dynamic transition embedding. 
Corresponding to each encoder layer, there is a GCN layer ($L_{enc}$ layers in total) which we show by $Z^{(l)}$ and is calculated sequentially for user $u$ as follows:

\begin{equation}
\begin{aligned}
    \label{eq:gcn_update}
    Z_u'(l) &= (1-\epsilon)Z_u^{(l-1)} + \epsilon H_u^{(l-1)} \\
    Z_u^{(l)} &= \phi{(AZ_u'^{(l)}W_l)}
\end{aligned}
\end{equation}
Taking $B$ as the batch size, $D_{l, in}$ and $D_{l, out}$ as the in and out dimensions of $l$th layer of autoencoders,
 $A \in \mathbb{R}^{B\times B}$ is the normalized adjacency matrix of graph $G$ and $W_l \in \mathbb{R}^{D_{l, in}\times D_{l, out}}$ 
 is the weight matrix of $l$th layer. Using the GCN output, we calculate the topological clustering assignment of $u$:

 \begin{equation}
\begin{aligned}
    \label{eq:top_clustering}
    c_{top}^u = softmax(Z_u^{(L_{enc})}) \\
\end{aligned}
\end{equation} 

%As suggested in (Kim et. al. 2020) \noindent\textbf{\textcolor{red}{(fix it)}}, 
We optimize clustering assignment $c_{enc}^u = [c_{i,j}]$ by closing representations to cluster centers and making harder assignments using KL divergence loss: 
% \textcolor{red}{\st{mohammad:closing representation to cluster centers? or making the cluster assignment harder?}}\\
% \textcolor{red}{\st{mohammad:Loss mode seems wrong} -> Re: In reality, we calc log of $c^u_top_mod$ to calc KL value, but for the sake of simplicity, I didn't include it.}

\begin{equation}
\begin{aligned}
    \label{eq:top_clustering_modified}
    c_{enc\_mod}^u = [c'_{i,j}]; c'_{i,j} = \frac{c_{i,j}^2 / f_{i,j}}{\sum_{j'}{c_{i,j'}/f_{i,j'}}} \\
    \mathcal{L}_{mod} = KL(c_{top}^u || c_{top_{mod}}^u)
\end{aligned}
\end{equation} 
where $f_{i,j} = \sum_{i}{c_{i,j}}$. This clustering assignment is used to supervise $c_{enc}^u$ using KL divergence loss:

\begin{equation}
\begin{aligned}
    \label{eq:kld_top_enc_clustering}
    \mathcal{L}_{combo} = KL(c_{enc\_mod}^u || c_{top}^u)
\end{aligned}
\end{equation} 

The overall clustering loss is calculated as the sum of all previously mentioned losses:

\begin{equation}
\begin{aligned}
    \label{eq:clustering_loss}
    %\mathcal{L}_{clustering\_module} =
    \mathcal{L}_{CM} =
    \mathcal{L}_{rec} + \mathcal{L}_{mod} + \mathcal{L}_{combo}
\end{aligned}
\end{equation} 

The user embedding $u$ is then conditioned on the encoding clustering assignment $c_{enc}^u$ as shown in 
\ref{eq:clustering_conditioning}:

\begin{equation}
\begin{aligned}
    \label{eq:clustering_conditioning}
    u' = f_{gamma}(c_{enc_{mod}}^u).u + f_{beta}(c_{enc_{mod}}^u)
\end{aligned}
\end{equation} 
where $f_{gamma}$ and $f_{beta}$ are one-layer fully connected networks and $.$ is element-wise multiplication.
% \textcolor{red}{Mohammad: final prediction of the model (sum of the modified user embed and last item) should be considered. In this section or a new section.}

%\noindent\textbf{\textcolor{red}{@reze and @mohammad: Please write a paragraph about the architecture of the clustering module - why we use GNN for clustering - Emphasis on implicit transition and no side information to use in the clustering process - formulation. }}\\

%\noindent\textbf{\textcolor{red}{@reze : Please add figure of the clustering module}}\\


\subsubsection{Optimization and Fast Adaptation}
\label{subsection:optimization}
The proposed model needs proper optimization to transfer extracted knowledge to new users and adapt personalized parameters based on their few transactions. Specifically, some initialized parameters will be adjusted to support a user's set, to produce personalized parameters. In contrast, other parameters are not personalized for each user and shared. We denote the parameters that will not adapt to support set and are shared among tasks (users) with $\omega$. Other parameters are denoted by $\Phi$.   
%\rafiei{how do you choose whether parameters are shared or meta?}
It would be better to share the clustering module parameters among all tasks since this module detects similar tasks, as shown in Figure \ref{fig:overallarchitecture}.
%\rafiei{why?} 
 In contrast, the user's preference extractor will be adapted by each user, so its parameters are in the personalized parameter set $\Phi$. In addition, all model parameters are denoted by $\theta$ $(\theta = \omega \cup \Phi)$. Therefore, user-specific parameters could be adapted to the support set by equation \ref{eq:user_specific_update}:

\begin{equation}
    \label{eq:user_specific_update}
    \omega'_{n} = \omega - \alpha \nabla_{\omega} \mathcal{L}_{S_n}
\end{equation}

\noindent In which $\alpha$ is the task learning rate (local learning rate) and $\mathcal{L}_{S_n}$ are the losses calculated on the support set data and $\nabla_{\omega} \mathcal{L}_{S_n}$ is its corresponding gradient. More specifically, the support loss is based on ranking margin loss. This is the difference between the score of the positive and negative items' scores. The positive sample is the item in the actual sequence of a user, and the negative samples are items that are not a member of the user's actual sequence and are sampled randomly. The support loss $\mathcal{L}_{S_n}$ is calculated based on the equation \ref{eq:support_loss}:
%\rafiei{define $\nabla_{\omega} \mathcal{L}_{S_n}$ } \\

\begin{equation}
\begin{aligned}
    \label{eq:support_loss}
    \mathcal{L}_{S_n} = \sum_{i=3}^{K-1} max(0,\lambda + s(I_{i-1} \xrightarrow[]{} I_i) - s(I_{i-1} \xrightarrow[]{} I_i'))\\
    s(I_{i-1} \xrightarrow[]{} I_i) = \lVert 
    f( \{I_0,...,I_{i-1} \} , \theta)  - I_i
    \rVert _{2}
\end{aligned}
\end{equation}

$\lambda$ represents the margin value, and $I_k'$ represents the negative sample, an item that has never interacted with the user $u_n$. Also, $s(.)$ is the function that calculates the score of the predicted item compared with the ground truth ($I_i$) on the given parameters of the neural network $\theta$. The function $f(.)$ symbolizes the model's output based on the user's history sequence.
%\rafiei{explain more about the formula.}\\

The query loss will be calculated based on the $\omega'_{n}$ on the query set $Q_n$. Specifically, the query loss is calculated by \ref{eq:query_loss}:

\begin{equation}
    \label{eq:query_loss}
    \mathcal{L}_{Q_n} = 
    %\sum_{(I_{k-1} \xrightarrow[]{} I_k) %\in Q_n} 
    max(0,\lambda + s(I_{K-1} \xrightarrow[]{} I_K) - s(I_{K-1} \xrightarrow[]{} I_K'))
\end{equation}

Based on the meta-learning goals, the learning process tries to find the initial parameters for $\theta$, which reduces the overall loss of the query sets of all tasks. More specifically, the overall loss is calculated by \ref{eq:overal_loss}:

%\noindent\textbf{\textcolor{red}{reze : We'd better include SDCN loss here}}\\

\begin{equation}
    \label{eq:overal_loss}
    \theta = \min_{\theta} \sum_{\tau_n \in p(\tau)}\mathcal{L}_{Q_n}(\omega'_n,\phi) + 
    \mathcal{L}_{CM_n}
\end{equation}

$\mathcal{L}_{CM_n}$ is the loss corresponding to the clustering module for $u_n$, which was defined in \ref{eq:clustering_loss}.
As the equation defines, meta-training tasks come from a distribution, and we only have a random sample set in the training phase. So by minimizing their query loss on the adopted parameters, the algorithm will converge to a proper initialization parameter for the tasks. The query loss is calculated based on the parameters adopted on each user's training support set. To solve the equation, the Stochastic Gradient Descent (SGD) is used in \ref{eq:sgd}:


\begin{equation}
    \label{eq:sgd}
    \theta \xleftarrow{} \theta - \beta \nabla_{\theta}  \sum_{\tau_n \in p(\tau)} \mathcal{L}_{Q_n}(\omega'_n,\phi)
\end{equation}

\noindent in which $\beta$ is the meta-learning rate.

The test phase is almost similar to the training phase. For each cold-user not seen in training, the task-specific parameters are adapted to its support set as in \ref{eq:user_specific_update}. Secondly, the score of the target item in the query set will be compared with 100 random negative samples to calculate the evaluation metrics.



% Place all illustrations (figures, drawings, tables, and photographs)
% throughout the paper at the places where they are first discussed,
% rather than at the end of the paper.

% They should be floated to the top (preferred) or bottom of the page,
% unless they are an integral part
% of your narrative flow. When placed at the bottom or top of
% a page, illustrations may run across both columns, but not when they
% appear inline.

% Illustrations must be rendered electronically or scanned and placed
% directly in your document. They should be cropped outside \LaTeX{},
% otherwise portions of the image could reappear during the post-processing of your paper.
% When possible, generate your illustrations in a vector format.
% When using bitmaps, please use 300dpi resolution at least.
% All illustrations should be understandable when printed in black and
% white, albeit you can use colors to enhance them. Line weights should
% be 1/2-point or thicker. Avoid screens and superimposing type on
% patterns, as these effects may not reproduce well.

% Number illustrations sequentially. Use references of the following
% form: Figure 1, Table 2, etc. Place illustration numbers and captions
% under illustrations. Leave a margin of 1/4-inch around the area
% covered by the illustration and caption.  Use 9-point type for
% captions, labels, and other text in illustrations. Captions should always appear below the illustration.

\section{Experiment}

\subsection{Experimental Setup}

% \begin{table}[]
% \caption{  Dataset Statistics }
% \begin{tabular}{cccc}
% \hline
% Dataset     & Users & Items  & Avg. Length of Sequence \\ \hline
% Electronics & 22685 & 20712  & 15.26                   \\
% Movies      & 26933 & 18855  & 28.97                   \\
% Beauty      & 82659 & 119365 & 28.97                   \\ \hline
% \end{tabular}
% \end{table}

\begin{table}[]
\caption{  Dataset Statistics }
\label{table:statistics}
\begin{tabular}{cccc}
\hline
Dataset     & Users & Items  & Avg. Length of Sequence \\ \hline
Electronics & 29710 & 20712  & 13.51                   \\
Movies      & 199435 & 155527  & 10.87                   \\
Beauty      & 82659 & 124859 & 6.96                   \\ \hline
\end{tabular}
\end{table}

\subsubsection{Datasets}
%\textbf{\textcolor{red}{@Bardia: please write about each dataset and its preprocessing that we used + statistics table}}

From Amazon, we have adopted three widely used real-world datasets, which are shown in Table \ref{table:statistics}. The Electronics dataset is derived from the public Amazon review dataset. This includes reviews of Amazon products belonging to the "Electronics" category from May 1996 to July 2014. Both The Movies and The Beauty are drawn from the same "Movie" and "Beauty" Amazon review categories. User reviews are treated as an interaction between them. These interactions are treated equally on all items. The $K$ parameter specifies the minimum number of transactions a user must keep. In addition, we delete users with fewer interactions in the system. We sort the data in order of the first transaction time, user ID, and transaction time. We then assign a new label to the items and users according to their appearance time, so that the first user is one, the second user is two, etc. Lastly, we will separate the test data from the test items. We will only keep the test data that contains interactions that use items from the test items in the test data. For each user node in the test and validation sets, we take each observed edge as a positive sample of the user. We then randomly select 100 items that did not interact with the current user as negative samples. Then based on the rank of the positive sample's score among negative samples, evaluation metrics will be calculated as in 


\cite{wang2021sequential,wei2020fast,kang2018self}.


\subsubsection{Baselines}


% Please add the following required packages to your document preamble:
% \usepackage{booktabs}
% \usepackage{multirow}
% Please add the following required packages to your document preamble:
% \usepackage{multirow}
% \begin{table}[]
% \caption{ Comparison of Different Models }
% \begin{tabular}{|c|cc|cc|cc|}
% \hline
% \multirow{}{}{Methods} & \multicolumn{2}{c|}{Electronics}   & \multicolumn{2}{c|}{Movie}         & \multicolumn{2}{c|}{Beauty}        \\ \cline{2-7} 
%                          & \multicolumn{1}{c|}{Hit@1} & MRR   & \multicolumn{1}{c|}{Hit@1} & MRR   & \multicolumn{1}{c|}{Hit@1} & MRR   \\ \hline
% BERT4Rec                 & \multicolumn{1}{c|}{0.200} & 0.323 & \multicolumn{1}{c|}{0.220} & 0.351 & \multicolumn{1}{c|}{0.214} & 0.341 \\ \hline
% MeLU                     & \multicolumn{1}{c|}{0.136} & 0.243 & \multicolumn{1}{c|}{0.168} & 0.289 & \multicolumn{1}{c|}{0.160} & 0.279 \\ \hline
% MAMO                     & \multicolumn{1}{c|}{0.127} & 0.296 & \multicolumn{1}{c|}{0.194} & 0.320 & \multicolumn{1}{c|}{0.195} & 0.310 \\ \hline
% MetaTL                   & \multicolumn{1}{c|}{0.241} & 0.320 & \multicolumn{1}{c|}{0.267} & 0.337 & \multicolumn{1}{c|}{0.231} & 0.328 \\ \hline
% MetaCF                   & \multicolumn{1}{c|}{0.210} & 0.330 & \multicolumn{1}{c|}{0.234} & 0.365 & \multicolumn{1}{c|}{0.220} & 0.340 \\ \hline
% \textbf{Our Model}                & \multicolumn{1}{c|}{}      &       & \multicolumn{1}{c|}{}      &       & \multicolumn{1}{c|}{}      &       \\ \hline
% \end{tabular}
% \end{table}


% \usepackage{tabularray}
% \begin{table}
% \centering
% \caption{Baselines}
% \label{table:evaltable}
% \begin{tblr}{
%   cells = {c},
%   cell{1}{1} = {r=2}{},
%   cell{1}{2} = {c=2}{},
%   cell{1}{4} = {c=2}{},
%   cell{1}{6} = {c=2}{},
%   % vline{2-3,5} = {1}{},
%   % vline{4,6} = {2}{},
%   % vline{2,4,6} = {3-8}{},
%   hline{1,3-9} = {-}{},
%   hline{2} = {2-7}{},
% }
% Methods            & Electronics &         & Movie   &         & Beauty  &         \\
%                    & $Hit@1$     & $MRR$   & $Hit@1$ & $MRR$   & $Hit@1$ & $MRR$   \\
% BERT4Rec           & $0.200$     & $0.323$ & $0.220$ & $0.351$ & $0.214$ & $0.341$ \\
% MeLU               & $0.136$     & $0.243$ & $0.168$ & $0.289$ & $0.160$ & $0.279$ \\
% MAMO               & $0.127$     & $0.296$ & $0.194$ & $0.320$ & $0.195$ & $0.310$ \\
% MetaTL             & $0.241$     & $0.320$ & $0.267$ & $0.337$ & $0.231$ & $0.328$ \\
% MetaCF             & $0.210$     & $0.330$ & $0.234$ & $0.365$ & $0.220$ & $0.340$ \\
% \textbf{Our Model} & \textbf{0.271} & \textbf{0.383} & \textbf{0.251} & \textbf{0.371} & \textbf{0.251} & \textbf{0.360} 
% \end{tblr}
% \end{table}





%\textbf{\textcolor{red}{@Bardia: please write about baselines and their corresponding adaptations}}

We compare the proposed model with the following methods: 

(i) Sequential recommendation baselines utilize different methods to capture the sequential patterns in the interaction sequences of users:

\begin{itemize}
    % \item SASRec:  Rely on Gated Recurrent Units, the simple convolutional generative network, and the self-attention layers to learn sequential user behaviors, respectively.
    \item SASRec: presents a self-attentive sequential recommendation model that utilizes Gated Recurrent Units, a simple convolutional generative network, and a self-attention mechanism to capture sequential patterns in user behavior and improve recommendation accuracy. The model is trained using a modified BPR loss function. 

    % \item BERT4Rec: adopts the bi-directional transformer to extract the sequential patterns, which is state-of-the-art for the sequential recommendation.
    \item BERT4Rec: proposes a novel recommendation model that uses the BERT architecture to capture sequential patterns in user behavior and improve recommendation accuracy. The model is trained using the bi-directional transformer to extract sequential patterns, outperforming other state-of-the-art models in accuracy and robustness to cold-start and long-tail item problems. The paper acknowledges some BERT4Rec limitations, such as its computational complexity and data requirements. However, it argues that the model's benefits justify the additional computational resources. 

\end{itemize}

(ii) Cold-start baselines include methods that provide accurate recommendations for customers with limited information. We modify these cold-start baselines to fit the case without auxiliary information. To deal with this issue, in the no side-information setting, for the datasets, we convert them into implicit recommendations by setting rated items to 1 and others to 0, and we utilize the Marginal Ranking loss function, which is the same as in our model, as we make implicit recommendations for binary signals. We just use the ID embedding of users and items as a feature (some methods like NGCF and LightGCN use this kind of embedding). In the training phase of recommendations, we sample data from the user and corresponding positive and negative items to calculate the loss at each step.

\begin{itemize}
    \item MeLU: Resolve the cold-start problem faced by existing recommender systems. The MeLU method uses meta-learning to estimate new users' preferences based on items they have consumed in the past. Moreover, the system provides a strategy to select evidence candidates to estimate customized preferences. It is shown that MeLU has a lower mean absolute error than two comparative models when tested on two benchmark datasets. In addition, the evidence selection strategy is tested in a user study. It aims to overcome the limitations of previous recommendation studies. These studies provided poor recommendations for users who consumed few items and inadequate evidence for candidates to identify user preferences.

    \item MetaTL: For cold-start users with minimal logged interactions, capturing sequential patterns of users for sequential recommenders is challenging. Models with limited interactions lose their predictive power due to difficulties in learning sequential patterns. Using meta-learning, the method proposes an innovative MetaTL framework that models users' transition patterns. A translation-based architecture extracts dynamic transition patterns from sequential recommendations in MetaTL, and meta-transitional learning facilitates fast learning for cold-start users with limited interaction. Meta-learning can improve sequential recommendations for cold-start users by inferring accurate sequential interactions.

    \item MAMO: Two memory matrices are used to store task-specific and feature-specific memories to support personalized parameter initialization and fast user preference prediction. 
    
    \item MetaCF: Discusses the cold-start problem in Collaborative Filtering (CF), where limited data is available for new users in the system. Previous approaches use user profiles, but these are not always available due to privacy concerns. MetaCF is a novel learning paradigm that leverages meta-learning to enable fast adaptation for new users. MetaCF learns a suitable initialization model for rapidly adapting to a new user. Adaptation rates are optimized in a fine-grained manner using Dynamic Subgraph Sampling to account for the dynamic arrival of new users. The proposed framework outperforms state-of-the-art baselines by a large margin in the cold-start scenario with limited user-item interactions.

\end{itemize}

\subsubsection{Evaluation Metrics}
%\textbf{\textcolor{red}{@Bardia: please write about them like metaTL with different sentences.}}

% In the experiment, each user only has one positive and true item for testing. With the predicted scores, we take each observed edge as a positive sample of the user and then randomly select 100 items that did not interact with the current user as the negative samples. This method has been widely used in many other works. Then we rank the list of positive and 100 negative items. We use Hit Ratio at rank 10 (HR@10)as the evaluation metric to measure the ranking performance. Mean Reciprocal Rank ($MRR$) indicates the rankings of the positive items. We also evaluate the Hit Rate ($Hit$) for the top-1 prediction. $Hit$@1 = 1 if the positive item is ranked top-1, otherwise $HR$@1 = 0. Also, note that $HR$@1 equals the recall or NDCG for top-1 prediction.

Each user was tested on only one positive and true item during the experiment. Based on the predicted scores, observed edges were taken as positive samples for the user. 100 items without interaction with the user were randomly selected as negative samples. This method is commonly used in other works. The list of 100 negative and positive items was ranked, and Hit Ratio at rank 10 (HR@10) was applied as the evaluation metric to measure ranking performance. Mean Reciprocal Rank (MRR) was used to indicate the ranking of positive items, and Hit Rate (Hit) was evaluated for the top-1 prediction. If the positive item was ranked top-1, Hit@1 was equal to 1; otherwise, it was 0. It should be noted that HR@1 is equivalent to recall or NDCG for top-1 prediction.

\subsection{Overall Performance}

\begin{table*}[!t]
\centering
\caption{Experimental results of different methods under K=3 on three data sets}
\label{table:evaltable}
\begin{tblr}{
  cells = {c},
  cell{1}{1} = {r=2}{},
  cell{1}{2} = {c=3}{},
  cell{1}{5} = {c=3}{},
  cell{1}{8} = {c=3}{},
  hline{1,3-13} = {-}{},
  hline{2} = {2-11}{},
}
Methods            & Electronics    &                &                & Movies         &                &                & Beauty         &                &                \\
                   & $MRR$          & $Hit@1$        & $NDCG@5$       & $MRR$          & $Hit@1$        & $NDCG@5$       & $MRR$          & $Hit@1$        & $NDCG@5$       \\
BERT4Rec           & $0.323$        & $0.200$        & $0.319$        & $0.421$        & $0.220$        & $0.357$        & $0.341$        & $0.214$        & $0.338$        \\
MeLU               & $0.243$        & $0.136$        & $0.265$        & $0.336$        & $0.168$        & $0.302$        & $0.279$        & $0.160$        & $0.292$        \\
MAMO               & $0.296$        & $0.127$        & $0.313$        & $0.384$        & $0.194$        & $0.345$        & $0.310$        & $0.195$        & $0.336$        \\
MetaTL             & $0.320$        & $0.241$        & $0.324$        & $0.438$        & $0.319$        & $0.412$        & $0.328$        & $0.231$        & $0.335$        \\
MetaCF             & $0.330$        & $0.210$        & $0.313$        & $0.474$        & $0.276$        & $0.397$        & $0.340$        & $0.220$        & $0.322$        \\
\textbf{ClusterSeq} & $\mathbf{0.383}$ & $\mathbf{0.262}$ & $\mathbf{0.391}$ & $\mathbf{0.660}$ & $\mathbf{0.542}$ & $\mathbf{0.685}$ & $\mathbf{0.443}$ & $\mathbf{0.254}$ & $\mathbf{0.341}$ \\


\end{tblr}
\end{table*}

In this study, we evaluated the performance of ClusterSeq and state-of-the-art models under K = 3 on several datasets. The results are presented in Table \ref{table:evaltable}. The best-performing method in each column is highlighted in bold. The findings indicate that ClusterSeq outperforms the competing models in all datasets, demonstrating its effectiveness in providing accurate recommendations for cold-start users with limited interactions.

We started with basic neural models for sequential recommendations. We discovered that BERT4Rec performed poorly due to its inability to capture patterns in user interaction sequences and learn effective embeddings for cold-start users. However, utilizing transformers to extract sequential patterns proved more effective as they aggregate items with attention scores. This leads to more informative representations for users with limited interactions.

MeLU, MAMO, MetaCF, and MetaTL are meta-learning-based methods that provide cold-start recommendations. As MeLU and MAMO require side information about users and items, we used their historical interactions as side information. However, MeLU and MAMO failed to produce satisfying results, as they are designed for scenarios with abundant auxiliary user/item information, which is not the case here. On the other hand, MetaCF and MetaTL performed well in the sequential recommendation, highlighting the importance of fast adaptation in cold-start scenarios. Nevertheless, they still fell short of ClusterSeq's proposed clustering patterns for a cold-start sequential recommendation.

\subsection{Ablation Study}


We compare the proposed model with its variants and some baselines under different K values (i.e., how many interactions are initially present) to evaluate its effectiveness. Our original experiment demonstrated that BERT4Rec is the state-of-the-art sequential recommendation method, and MetaTL is one of the strongest cold-start baselines (and illustrates meta-transitional learning). Despite its high prediction power, BERT4Rec performs poorly on cold-start sequential recommendation tasks with a limited number of items. In sequential and cold-start user recommendations with different numbers of initial interactions, the proposed model can outperform state-of-the-art methods due to the well-designed optimization steps and clustering of users within the graph.

% Figure environment removed

We compare the performance of our entire model (with the clustering module) to several model variants that do not include the clustering module. We evaluate these models on a standard benchmark dataset and report the results regarding our evaluation metrics.
Figure \ref{fig:ablation_results} shows the results of the ablation study. Clearly, the entire model achieves the highest evaluation metrics, indicating that the clustering module is necessary for the model to achieve its best performance. Performance is significantly reduced when the clustering module is removed.
% \textcolor{red}{k=3 with clustering module: mrr:35.2 with clustering module:38.3}


\subsection{Parameter Analysis}
In this section, we investigate the impact of model parameters on the recommendation performance of our proposed model under cold-start scenarios. We examined how cluster number affects performance, followed by the impact of dimensions of user representations and learning rates.

\subsubsection{Number of clusters}
To study the effect of the number of clusters, we vary the number of clusters and plot the performance of the proposed method in terms of MRR in Figure \ref{fig:num_clusters}. We observed that the performance of the proposed method is generally stable for different number of clusters. In particular, we find that the proposed model is robust against this hyperparameter which is hard to estimate.

% Figure environment removed

\subsubsection{Impact of Embedding Dimensions}
Next, we explore the influence of the dimension of user embeddings on the recommendation performance of the proposed model. We vary the embedding sizes from 32 to 512 and plot the resulting performance in terms of MRR in Figure \ref{fig:embedding_dim}. Our model achieves optimal performance when the embedding dimension is set to 256. Our model is not generally stable around the optimal setting, indicating that it is important to set the embedding dimensions carefully.

% Figure environment removed


\subsubsection{Impact of Batch size}
Lastly, we analyzed the impact of batch size on the performance of our model. We conducted experiments by varying the batch size from 64 to 4096 and evaluated the resulting performance in terms of MRR, as presented in Figure \ref{fig:batch_size}. Our observations show that the optimal performance of our model is achieved when the batch size is set to 1024. Additionally, we note that the training process can converge even with smaller batch sizes like 512 or 256. Overall, this analysis highlights the importance of selecting an appropriate batch size to achieve high performance in recommendation tasks.

% Figure environment removed

To sum up, our analysis of the model parameters indicates that the recommendation performance of our model is stable for a reasonable range of hyperparameter values. However, some parameters are found to be critical for achieving optimal performance. Therefore, our findings emphasize the significance of meticulously tuning the model parameters to attain high recommendation performance in cold-start scenarios.

% Tables are considered illustrations containing data. Therefore, they should also appear floated to the top (preferably) or bottom of the page, and with the captions below them.

% \begin{table}
%     \centering
%     \begin{tabular}{lll}
%         \hline
%         Scenario  & $\delta$ & Runtime \\
%         \hline
%         Paris     & 0.1s     & 13.65ms \\
%         Paris     & 0.2s     & 0.01ms  \\
%         New York  & 0.1s     & 92.50ms \\
%         Singapore & 0.1s     & 33.33ms \\
%         Singapore & 0.2s     & 23.01ms \\
%         \hline
%     \end{tabular}
%     \caption{Latex default table}
%     \label{tab:plain}
% \end{table}

% \begin{table}
%     \centering
%     \begin{tabular}{lrr}
%         \toprule
%         Scenario  & $\delta$ (s) & Runtime (ms) \\
%         \midrule
%         Paris     & 0.1          & 13.65        \\
%                   & 0.2          & 0.01         \\
%         New York  & 0.1          & 92.50        \\
%         Singapore & 0.1          & 33.33        \\
%                   & 0.2          & 23.01        \\
%         \bottomrule
%     \end{tabular}
%     \caption{Booktabs table}
%     \label{tab:booktabs}
% \end{table}

% If you are using \LaTeX, you should use the {\tt booktabs} package, because it produces better tables than the standard ones. Compare Tables \ref{tab:plain} and~\ref{tab:booktabs}. The latter is clearly more readable for three reasons:

% \begin{enumerate}
%     \item The styling is better thanks to using the {\tt booktabs} rulers instead of the default ones.
%     \item Numeric columns are right-aligned, making it easier to compare the numbers. Make sure to also right-align the corresponding headers, and to use the same precision for all numbers.
%     \item We avoid unnecessary repetition, both between lines (no need to repeat the scenario name in this case) as well as in the content (units can be shown in the column header).
% \end{enumerate}



%The proposed method is trained on a set of public datasets available in the ultrasound toolbox [10]. The proposed approach has been accepted for presentation during the Challenge on Ultrasound Beamforming with Deep Learning (CUBDL) at the 2020 IEEE International Ultrasonics Symposium (IUS) [11], [12]. synthetic / PICMUS difference

%ALSO DISCUSS ABOUT TYPE OF PRETRAINED VS TRAINED
Regarding the computing time, \yz{our approaches need 3-4 minutes to form one image, which is slower than DAS1, PCF~\cite{PCF} and MNV2~\cite{MNV2}, but faster \ji{than EMV ~\cite{asl_eigenspace-based_2010} and RED~\cite{RED_USIPB}, which need 8 and 20 minutes, respectively.} RED is slow because each iteration contains an inner iteration while \ji{EMV spends time on covariance matrix evaluation and decomposition.} Our iteration restoration approaches require multiple multiplication operations with the singular vector matrix, which currently hinders real-time imaging. }Accelerating this process is one of our key focuses for future work.

\begin{comment}
the computationally expensive SVD for DDRM actually does not affect imaging time since the SVD results can be precomputed, but the multiplication operation with the singular vector matrix during the image reconstruction process currently hinders real-time imaging. On our machine equipped with the GPU NVIDIA Quadro RTX 3000, each iteration takes approximately 4.5 seconds.
\end{comment}

\YZ{In conclusion, for the first time, we achieve the reconstruction of ultrasound images with} 
\DM{ two adapted diffusion models, DRUS and WDRUS. }
\yz{Different from previous model-based deep learning methods which are task-specific and require a large amount of data pairs for supervised training, our approach requires none or just a small fine-tuning dataset composed of high-quality (e.g., DAS101) images only (there is no need for paired data). Furthermore, the fine-tuned diffusion model can be used}
%applied to} 
\dm{for other US related inverse problems.}
%diverse inverse problems, e.g., DRUS and WDRUS, as long as the same prior knowledge.}
\YZ{Finally, our method demonstrated competitive performance compared to DAS75, and other state-of-the-art approaches on the PICMUS dataset.}



\begin{comment}
\YZ{Our approach has demonstrated superior performance compared to DAS, both in terms of visual quality and evaluation metrics, on both synthetic and PICMUS datasets}, \DM{even though we fine-tuned the \texttt{f-number} to fit DDRM while kept the default values from the open-source code for DAS, as 
%While we used slightly different parameters, e.g. \texttt{f-number} for DAS and our methods when testing on the PICMUS dataset, such as using default values from the PICMUS open-source code for DAS while fine-tuning these parameters for our methods to fit DDRM, 
fundamentally, the number of plane waves affects the image quality of DAS.} 
\YZ{Our method is able to compete with 75 plane waves, which is sufficient evidence of its effectiveness. As for the distortion observed in the WDRUS results, it may be due to the amplification of errors in the ultrasound model by the whitening operator, but the specific reason requires further investigation.
}

\YZ{
Our method provides an important insight for the medical imaging field by addressing the challenge of training} \DM{model-based deep learning methods
%neural networks 
}\YZ{when access to datasets is restricted due to privacy concerns. The model we used was trained on ImageNet only, without any ultrasound data.
}
\DM{However, it} 
\YZ{
%It 
should be noted that  %images in the 
ImageNet data %dataset 
are significantly different from ultrasound images. For example, pixel values in natural images are always positive, while} 
%in ultrasound image reconstruction, 
 \DM{the reconstructed ultrasound 
$\xv$ contains both positive and negative values, and 
%ultrasound images 
are typically displayed after log compression.}
\YZ{Therefore, fine-tuning existing models with a small amount of ultrasound data may lead to better results.
}

\YZ{
Although DDRM relies on the computationally expensive SVD, it does not affect imaging time since} 
%its results can be saved and repeatedly used. 
\DM{ the SVD results can be precomputed.}
\YZ{However, the multiplication operation} \DM{ with the singular vector matrix 
%between the singular matrix and other vectors 
}
\YZ{during the image reconstruction process currently hinders real-time imaging. On our machine equipped with the GPU NVIDIA Quadro RTX 3000, each iteration takes approximately 4.5 seconds. Accelerating this process is one of our key focuses for future work.}

\YZ{
In conclusion, for the first time, we achieve the reconstruction of ultrasound images with} 
%a diffusion model and test two ultrasound models, DRUS and WDRUS. 
\DM{ two adapted diffusion models, DRUS and WDRUS. }
\YZ{Our method with single plane wave is even comparable to DAS with 75 plane waves,}
%which are often used to produce target images, in the case where the generative model has never been trained on ultrasound data.
\DM{which is often used as reference to train generative models, whereas our diffusion model was never trained on ultrasound data}
\end{comment}

\bibliographystyle{ACM-Reference-Format}
\bibliography{sample-base}

\end{document}
\endinput
%%
%% End of file `sample-authordraft.tex'.
