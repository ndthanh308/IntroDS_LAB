\section{Conclusion}

% IJCAI's two-column format makes it difficult to typeset long formulas. A usual temptation is to reduce the size of the formula by using the {\tt small} or {\tt tiny} sizes. This doesn't work correctly with the current \LaTeX{} versions, breaking the line spacing of the preceding paragraphs and title, as well as the equation number sizes. The following equation demonstrates the effects (notice that this entire paragraph looks badly formatted):
% %
% \begin{tiny}
%     \begin{equation}
%         x = \prod_{i=1}^n \sum_{j=1}^n j_i + \prod_{i=1}^n \sum_{j=1}^n i_j + \prod_{i=1}^n \sum_{j=1}^n j_i + \prod_{i=1}^n \sum_{j=1}^n i_j + \prod_{i=1}^n \sum_{j=1}^n j_i
%     \end{equation}
% \end{tiny}%

% Reducing formula sizes this way is strictly forbidden. We {\bf strongly} recommend authors to split formulas in multiple lines when they don't fit in a single line. This is the easiest approach to typeset those formulas and provides the most readable output%
% %
% \begin{align}
%     x = & \prod_{i=1}^n \sum_{j=1}^n j_i + \prod_{i=1}^n \sum_{j=1}^n i_j + \prod_{i=1}^n \sum_{j=1}^n j_i + \prod_{i=1}^n \sum_{j=1}^n i_j + \nonumber \\
%     +   & \prod_{i=1}^n \sum_{j=1}^n j_i
% \end{align}%

% If a line is just slightly longer than the column width, you may use the {\tt resizebox} environment on that equation. The result looks better and doesn't interfere with the paragraph's line spacing: %
% \begin{equation}
%     \resizebox{.91\linewidth}{!}{$
%             \displaystyle
%             x = \prod_{i=1}^n \sum_{j=1}^n j_i + \prod_{i=1}^n \sum_{j=1}^n i_j + \prod_{i=1}^n \sum_{j=1}^n j_i + \prod_{i=1}^n \sum_{j=1}^n i_j + \prod_{i=1}^n \sum_{j=1}^n j_i
%         $}
% \end{equation}%

% This last solution may have to be adapted if you use different equation environments, but it can generally be made to work. Please notice that in any case:

% \begin{itemize}
%     \item Equation numbers must be in the same font and size as the main text (10pt).
%     \item Your formula's main symbols should not be smaller than {\small small} text (9pt).
% \end{itemize}

% For instance, the formula
% %
% \begin{equation}
%     \resizebox{.91\linewidth}{!}{$
%             \displaystyle
%             x = \prod_{i=1}^n \sum_{j=1}^n j_i + \prod_{i=1}^n \sum_{j=1}^n i_j + \prod_{i=1}^n \sum_{j=1}^n j_i + \prod_{i=1}^n \sum_{j=1}^n i_j + \prod_{i=1}^n \sum_{j=1}^n j_i + \prod_{i=1}^n \sum_{j=1}^n i_j
%         $}
% \end{equation}
% %
% would not be acceptable because the text is too small.

This paper proposes a method for extracting users' dynamic preferences through sequential personalized recommendations based on MAML. Meta-learning can adapt to different users, even with limited transactions, by formulating cold-start recommendations in a few-shot setting. Support adaptation is made in the training phase on a set of users with a few-shot transition in a sequence to mimic the targeted cold-start scenarios. Additionally, it avoids local optima, which is a substantial disadvantage of meta-learning in recommendation problems. Experiments have shown that the proposed method outperforms the current state-of-the-art methods in a cold-start scenario for three real-world datasets.