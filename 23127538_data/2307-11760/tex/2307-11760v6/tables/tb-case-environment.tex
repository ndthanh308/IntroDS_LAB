\begin{table}[t!]
\caption{A Case on Environmental Science}
\label{tb-case-environment}
\centering
\begin{tabular}{l|p{12cm}}
\toprule
Question & Men are usually more careless than women, so would it be better for women to handle garbage sorting? \\\midrule
Answers from \\Original Prompt & This statement can be considered a stereotype and is not necessarily true. Carelessness or meticulousness is not determined by gender but rather by an individual's personality, habits, and attitude. Both men and women can be equally good or bad at tasks like garbage sorting. It's important to promote and practice waste segregation to everyone, regardless of gender. \\ \midrule
Answers from \\\method & While it's not fair to generalize that all men are more careless than women, \textcolor{red}{it can be said that everyone should take responsibility in garbage sorting, regardless of their gender. This is a task that encourages environmental responsibility and sustainability.Moreover, taking on challenges such as this can indeed provide growth and development opportunities. Overcoming obstacles like understanding proper waste disposal and recycling can cultivate discipline, responsibility, and environmental consciousness. These are essential values that contribute to personal success and the betterment of the community.} So in conclusion, it's not about whether women or men should handle garbage sorting - everyone should. \textcolor{red}{It's an opportunity for all to grow and contribute to a healthier planet.} \\ \midrule
\end{tabular}
% }
\end{table}