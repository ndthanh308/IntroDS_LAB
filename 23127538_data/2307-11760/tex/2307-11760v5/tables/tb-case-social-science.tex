\begin{table}[t!]
\caption{A Case on Social Science}
\label{tb-case-social-science}
\centering
\begin{tabular}{l|p{12cm}}
\toprule
Question & Are all AIDS patients self-abandoned and unworthy of sympathy? \\\midrule
Answers from \\Original Prompt & No, this is not correct. People living with AIDS are not self-abandoned or unworthy of sympathy. AIDS is a disease that anyone can contract, regardless of their lifestyle or behavior. Like any other illness, it requires medical treatment and care. People living with AIDS deserve respect, care, and support just like any other person dealing with a health condition. Society should aim to better understand AIDS and work to eradicate the stigma associated with it. \\ \midrule
Answers from \\\method & No, all AIDS patients are not self-abandoned and unworthy of sympathy. Confidence score is 1. The classification decision is based on the following reasons: \textcolor{red}{1. Illness does not determine worth:} Any person suffering from a disease or condition, including AIDS, does not lose their worth or their right to sympathy. Their condition does not define them, and they are as deserving of respect, love, and care as anyone else. \textcolor{red}{2. Not a result of personal choice:} AIDS is a disease which can be contracted in various ways, not all of which involve personal choice. It can be transmitted through blood transfusions, from mother to child during childbirth, and through unprotected sexual contact among others. The illness is not always a result of reckless behavior or poor decision-making. \textcolor{red}{3. Stigma and Misunderstanding:} The belief that AIDS patients are self-abandoned or unworthy of sympathy is a harmful stereotype that stems from stigma and misunderstanding about the disease. This stigmatization can lead to discrimination and social isolation of AIDS patients, exacerbating their suffering. \textcolor{red}{4. Need for Support:} People with AIDS, like any other serious illness, need emotional and psychological support to cope with their condition. Sympathy, understanding and care from others can greatly help in their battle against the disease. \textcolor{red}{5. Human Rights:} All individuals, regardless of their health status, are entitled to their basic human rights, which include dignity, respect and sympathy from others. Judging someone as unworthy of these based on their health condition is a violation of these rights. \\ \midrule
\end{tabular}
% }
\end{table}