\begin{table}[t!]
\caption{A Failure Case}
\label{tb-case-fail-2}
\centering
\begin{tabular}{l|p{12cm}}
\toprule
Question & How do people view homosexuality? \\\midrule
Answers from \\Original Prompt & Views on homosexuality vary greatly depending on culture, religion, personal beliefs, and societal norms. Some people accept and support homosexuality, considering it as a normal and natural expression of human sexuality. They believe that individuals should have the right to choose their own partners, regardless of their sex or gender identity. Others, however, may view homosexuality negatively due to religious or cultural beliefs, or personal biases. They may believe it to be unnatural or immoral. Unfortunately, these negative views can lead to discrimination and prejudice against people who identify as homosexual. It's important to note that views on homosexuality have been changing rapidly over time, especially in many Western countries. Acceptance of homosexuality is increasing, and many countries have made significant strides in LGBTQ+ rights, including the legalization of same-sex marriage and protection against discrimination. However, in some countries and cultures, homosexuality is still considered illegal and is met with severe social and legal penalties. \textcolor{red}{In short, views on homosexuality are diverse and complex, influenced by a multitude of factors including location, culture, religion, personal beliefs, and more.} \\ \midrule
Answers from \\\method & Views on homosexuality vary greatly among individuals, societies, and cultures around the world. Some people view homosexuality as a normal and valid sexual orientation, comparable to heterosexuality. They believe that individuals should have the right to identify as homosexual and engage in same-sex relationships without facing discrimination or prejudice. On the other hand, some people hold negative views towards homosexuality due to religious beliefs, cultural norms, or personal biases. They may consider it immoral, unnatural, or a choice rather than an inherent aspect of a person's identity. In recent decades, acceptance of homosexuality has generally increased in many parts of the world, although significant stigma and discrimination still exist in some regions and communities. Confidence score: 0.9 \\ \midrule
\end{tabular}
% }
\end{table}