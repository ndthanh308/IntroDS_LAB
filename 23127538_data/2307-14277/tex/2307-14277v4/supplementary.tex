\documentclass[10pt,twocolumn,letterpaper]{article}

\usepackage{iccv}
\usepackage{times}
\usepackage{epsfig}
\usepackage{graphicx}
\usepackage{amsmath}
\usepackage{amssymb}
\usepackage{enumitem}

\usepackage[ruled]{algorithm2e}
\usepackage{multirow}
% \usepackage{algorithmic}
% Include other packages here, before hyperref.

% If you comment hyperref and then uncomment it, you should delete
% egpaper.aux before re-running latex.  (Or just hit 'q' on the first latex
% run, let it finish, and you should be clear).
\usepackage[pagebackref=true,breaklinks=true,letterpaper=true,colorlinks,bookmarks=false]{hyperref}

\iccvfinalcopy % *** Uncomment this line for the final submission

\def\iccvPaperID{1601} % *** Enter the ICCV Paper ID here
\def\httilde{\mbox{\tt\raisebox{-.5ex}{\symbol{126}}}}

% Pages are numbered in submission mode, and unnumbered in camera-ready
\ificcvfinal\pagestyle{empty}\fi
\renewcommand\thesection{\Alph{section}}
\begin{document}

%%%%%%%%% TITLE
\title{$-$Supplementary Material$-$ \\ G2L: Semantically Aligned and Uniform Video Grounding \\ via Geodesic and Game Theory}



\author{
Hongxiang Li$^{1}$, Meng Cao$^{2,1}$, Xuxin Cheng$^{1}$, Yaowei Li$^{1}$, Zhihong Zhu$^{1}$, Yuexian Zou$^{1}$\footnotemark[2]\\
$^{1}$School of Electronic and Computer Engineering, Peking University \\ $^{2}$International Digital Economy Academy (IDEA)\\
{\tt\small \{lihongxiang, chengxx, zhihongzhu, ywl\}@stu.pku.edu.cn; \{mengcao, zouyx\}@pku.edu.cn}
}

\maketitle
% Remove page # from the first page of camera-ready.
\ificcvfinal\thispagestyle{empty}\fi


%%%%%%%%% BODY TEXT

\section{Overview}

In this supplementary material, we present the following.
\begin{itemize}
    \item Axiomatic Properties of Shapley Value (Section~\ref{sec: a1}).
    \item Proof of Equation 10 (Section~\ref{sec: a2}).
    % \item More Implementation Details (Section~\ref{sec: a4}).
\end{itemize}

\section{Axiomatic Properties of Shapley Value}\label{sec: a1}
	In this section, we mainly introduce the axiomatic properties of Shapley value. Weber \etal~\cite{weber1988probabilistic} have proved that Shapley value is the unique metric that satisfies the following axioms: \emph{Linearity}, \emph{Symmetry}, \emph{Dummy}, and \emph{Efficiency}.
	
	\textbf{Linearity Axiom.} If two independent games $u$ and $v$ can be linearly merged into one game $w(\mathcal{U}) = u(\mathcal{U}) + v(\mathcal{U})$, then the Shapley value of each player $i \in \mathcal{N}$ in the new game $w$ is the sum of Shapley values of the player $i$ in the game $u$ and $v$, which can be formulated as:
	
	\begin{equation}
		\phi_w(i|\mathcal{N}) = \phi_u(i|\mathcal{N}) + \phi_v(i|\mathcal{N})
	\end{equation}
	
	
	\textbf{Symmetry Axiom.} Considering two players $i$ and $j$ in a game $v$, if they satisfy:
	\begin{equation}
		\forall \mathcal{U} \in \mathcal{N} \setminus \{i, j\}, v(\mathcal{U} \cup \{i\}) = v(\mathcal{U} \cup \{j\})
	\end{equation}
	then $\phi_v(i|\mathcal{N}) = \phi_v(j|\mathcal{N})$.
	
	\textbf{Dummy Axiom.} The dummy player is defined as a player without interaction with other players. Formally, if a player $i$ in a game $v$ satisfies:
	\begin{equation}
		\forall \mathcal{U} \in \mathcal{N} \setminus \{i\}, v(\mathcal{U} \cup \{i\}) = v(\mathcal{U}) + v(\{i\})
	\end{equation}
	then this player is defined as the dummy player. In this way, the dummy player $i$ has no interaction with other players, \ie $v(\{i\}) = \phi_v(i|\mathcal{N})$.
	
	
	\textbf{Efficiency Axiom.} The efficiency axiom ensures that the overall reward can be assigned to all players, which can be formulated as follows:
	\begin{equation}
		\sum_{i \in \mathcal{N}} \phi_v(i) = v(\mathcal{N}) - v(\varnothing)
	\end{equation}

\section{Proof of Equation 10}\label{sec: a2}
In this section, we provide detailed proof for Equation~10 in Section~3.5.2. The semantic Shapley interaction between moment $x$ and query $y$ in video $V_i$ can be decomposed as follows:
\vspace{-0.05cm}
\begin{align}
    \mathfrak{I}([\mathcal{H}^i_{xy}]) &= \phi([\mathcal{H}^i_{xy}]|\mathcal{H}^i \setminus \mathcal{H}^i_{xy} \cup \{[\mathcal{H}^i_{xy}]\}) \nonumber\\
    &- \phi(\mathbf{h}^V_{ix}|\mathcal{H}^i \setminus \mathcal{H}^i_{xy} \cup \{\mathbf{h}^V_{ix}\}) \nonumber\\
    &- \phi(\mathbf{h}^Q_{iy}| \mathcal{H}^i \setminus \mathcal{H}^i_{xy} \cup \{\mathbf{h}^Q_{iy}\}) \\
    &= \mathop{\mathbb{E}}\limits_{C} \{\mathop{\mathbb{E}}\limits_{\mathcal{U} \subseteq \mathcal{H}^i \setminus \mathcal{H}^i_{xy}  \atop |\mathcal{U}| = C} [f(\mathcal{U} \cup \mathcal{H}^i_{xy}) - f(\mathcal{U})] \} \nonumber\\
    &- \mathop{\mathbb{E}}\limits_{C} \{\mathop{\mathbb{E}}\limits_{\mathcal{U} \subseteq \mathcal{H}^i \setminus \mathcal{H}^i_{xy}  \atop |\mathcal{U}| = C} [f(\mathcal{U} \cup \{\mathbf{h}_ix^V\}) - f(\mathcal{U})] \} \nonumber\\
    &- \mathop{\mathbb{E}}\limits_{C} \{\mathop{\mathbb{E}}\limits_{\mathcal{U} \subseteq \mathcal{H}^i \setminus \mathcal{H}^i_{xy}  \atop |\mathcal{U}| = C} [f(\mathcal{U} \cup \{\mathbf{h}^Q_{iy}\}) - f(\mathcal{U})] \}\\
    &= \mathop{\mathbb{E}}\limits_{C} \{\mathop{\mathbb{E}}\limits_{\mathcal{U} \subseteq \mathcal{H}^i \setminus \mathcal{H}^i_{xy}  \atop |\mathcal{U}| = C} [f(\mathcal{U} \cup \mathcal{H}^i_{xy}) - f(\mathcal{U} \cup \{\mathbf{h}^V_{ix}\}) \nonumber \\
    &- f(\mathcal{U} \cup \{\mathbf{h}^Q_{iy}\}) +  f(\mathcal{U})\ ]\  \}
\end{align}
 
 % \section{More Qualitative Analysis}\label{sec: a4}
% \section{More Implementation Details}\label{sec: a4}
% We following previous works~\cite{2D-TANzhang2020learning,MMNwang2022negative}, the convolutional network for proposal feature modeling uses exactly the same settings with 2D-TAN~\cite{2D-TANzhang2020learning}, including visual feature, number of sampled clips, number of 2D convolutional layers, kernel size, channels, NMS threshold, scaling thresholds and the dimension of the joint feature space.
 
{\small
\bibliographystyle{ieee_fullname}
\bibliography{egbib}
}

\end{document}