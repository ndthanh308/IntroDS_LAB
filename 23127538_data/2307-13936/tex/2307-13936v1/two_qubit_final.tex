\RequirePackage{lineno}
\documentclass[a4,nofootinbib,notitlepage,superscriptaddress]{revtex4-1}
\linespread{1.3} % one-and-a-half spacing
\usepackage{bm}
\usepackage{lineno}
\usepackage[pdftex]{graphicx}

\newcommand{\I}{\mathrm{i}}

\usepackage{amsmath,amsfonts,amssymb,mathtools}
\usepackage{amsthm}

\usepackage{bbm,bm}
\usepackage{float}
\usepackage{mathrsfs}

\usepackage[normalem]{ulem}

\usepackage{physics}
\usepackage{xcolor}
\newcommand{\comment}[1]{\textcolor{red}{#1}}
\usepackage[utf8]{inputenc}
\usepackage{pgfplots}
\usepgfplotslibrary{groupplots}
\usepgfplotslibrary{fillbetween}
\raggedbottom
\usepackage{apptools}
\AtAppendix{\counterwithin{theorem}{section}}

\DeclarePairedDelimiterXPP{\sfTr}[1]{\mathsf{Tr}}{[}{]}{}{#1}
\DeclarePairedDelimiterXPP{\sfTrAbs}[1]{\mathsf{TrAbs}}{[}{]}{}{#1}
\DeclarePairedDelimiterXPP{\opTr}[1]{\mathrm{Tr}}{[}{]}{}{#1}
\DeclarePairedDelimiterXPP{\bbTr}[1]{\mathbb{T}\mathrm{r}}{[}{]}{}{#1}




\newtheorem{theorem}{Theorem}
\newtheorem{lemma}[theorem]{Lemma}
\newtheorem{proposition}[theorem]{Proposition}
\newtheorem{corollary}[theorem]{Corollary}

\def\ANU{Centre for Quantum Computation and Communication Technology, Department of Quantum Science, Australian National University, Canberra, ACT 2601, Australia.}
\def\NTU{School of Physical and Mathematical Sciences, Nanyang Technological University, Singapore 639673, Republic of Singapore}
\def\UEC{Graduate School of Informatics and Engineering, The
  University of Electro-Communications, Tokyo 182-8585, Japan}
  \def\Astar{Institute of Materials Research and Engineering, Agency for Science Technology and Research (A*STAR), 2 Fusionopolis Way, 08-03 Innovis 138634, Singapore}
\begin{document}
\title{Multiparameter estimation with two qubit probes in noisy channels}
%
\author{Lorc\'{a}n O. Conlon}
\email{lorcanconlon@gmail.com}
\affiliation{\ANU}
\affiliation{\Astar}
%\author{Jun Suzuki}
%\email{junsuzuki@uec.ac.jp}
%\affiliation{\UEC}
\author{Ping Koy Lam}
\affiliation{\ANU}
\affiliation{\Astar}
\author{Syed M. Assad}
\email{cqtsma@gmail.com}
\affiliation{\ANU}
\affiliation{\Astar}
%\affiliation{\NTU}
%\affiliation{Centre for Quantum Computation and Communication Technology, Department of Quantum Science,\\ Research School of Physics and Engineering, Australian National University, Canberra ACT 2601, Australia.}%
\date{\today}

\begin{abstract}
This work compares the performance of single and two qubit probes for estimating several phase rotations simultaneously under the action of different noisy channels. We compute the quantum limits for this simultaneous estimation using collective and individual measurements  by evaluating the Holevo and Nagaoka--Hayashi Cram\'{e}r-Rao bounds respectively. Several quantum noise channels are considered, namely the decohering channel, the amplitude damping channel and the phase damping channel. For each channel we find the optimal single and two qubit probes. Where possible we demonstrate an explicit measurement strategy which saturates the appropriate bound and we investigate how closely the Holevo bound can be approached through collective measurements on multiple copies of the same probe. We find that under the action of the considered channels, two qubit probes show enhanced parameter estimation capabilities over single qubit probes for almost all non-identity channels, i.e. the achievable precision with a single qubit probe degrades faster with increasing exposure to the noisy environment than that of the two qubit probe. However, in sufficiently noisy channels, we show that it is possible for single qubit probes to outperform maximally entangled two qubit probes. This work shows that, in order to reach the ultimate precision limits allowed by quantum mechanics, entanglement is required in both the state preparation and state measurement stages. It is hoped the tutorial-esque nature of this paper will make it easily accessible.
%Throughout this study we found persistent results which we show are true for a broad class of channels; a two qubit probe is able to estimate a second parameter with no decrease in the achievable precision, for a single qubit probe estimating a second parameter causes the achievable precision to at best decrease by a factor of 2 in the absence of any noise and for a single qubit probe it is impossible to construct an unbiased estimator for estimating a rotation about all three axes of the Bloch sphere. 
\end{abstract}
\maketitle
%\tableofcontents
\section{Introduction}


% Figure environment removed

Quantum parameter estimation involves preparing a quantum probe, allowing this
probe to interact with the system we wish to learn about and then
examining the probe at the output. The maximum precision with which certain
parameters can be estimated is dictated by the laws of quantum mechanics~\cite{robertson1929uncertainty, arthurs1965bstj,heisenberg1985quantentheoretische}. %By preparing $N$ identical, independent probes and averaging the results obtained it is possible to enhance the attainable precision. The minimum variance which can be attained using probes which are not entangled scales as ${1}/{N}$, however through the use of entangled probes this scaling can be improved to ${1}/{N^2}$~\cite{giovannetti2004quantum,giovannetti2006quantum, giovannetti2011advances}. 
Quantum resources have been proposed as a way to improve measurement sensitivity in optical interferometry~\cite{caves1981quantum,barnett2003ultimate,dorner2009optimal,demkowicz2009quantum,zhuang2018distributed,ge2018distributed,conlon2022enhancing}, quantum superresolution
\cite{tsang2016quantum, tsang2019resolving}, quantum
positioning~\cite{giovannetti2001quantum, lamine2008quantum}, and tests of fundamental physics~\cite{brady2022entangled,marchese2023optomechanics,shi2023ultimate}. Several experiments have demonstrated enhanced precision estimation through the use of quantum resources~\cite{higgins2007entanglement, kacprowicz2010experimental, yonezawa2012quantum,girolami2014quantum,strobel2014fisher,slussarenko2017unconditional, zhang2019quantum,mccormick2019quantum,wang2019heisenberg,aasi2013enhanced,guo2020distributed,liu2021distributed,backes2021quantum,casacio2021quantum,marciniak2022optimal,malia2022distributed,PhysRevLett.130.123603}. 

%Parameter estimation and quantum metrology are two of the areas of research that best showcase the advantage quantum mechanical resources can offer over their classical counterparts. 
 
Arguably, the full range of quantum mechanical effects is only revealed through multi-parameter metrology owing to the possible incompatibility of conjugate observables. There are many physically motivated reasons for studying multiparameter estimation. The simultaneous estimation of several parameters can enhance our ability to measure the different components of a magnetic field~\cite{baumgratz2016quantum,hou2020minimal,montenegro2022sequential,kaubruegger2023optimal}, multiple phase shifts~\cite{spagnolo2012quantum,humphreys2013quantum,yue2014quantum,gagatsos2016gaussian,ciampini2016quantum, pezze2017optimal,zhang2017quantum}, a phase shift and loss or phase diffusion~\cite{crowley2014tradeoff, szczykulska2017reaching}, and can improve the tracking of chemical processes~\cite{cimini2019quantum}. Additionally, quantum super-resolution for resolving two incoherent point sources of light~\cite{tsang2016quantum,chrostowski2017super,vrehavcek2017multiparameter} and many estimation problems concerning Gaussian quantum states~\cite{chiribella2006joint,monras2011measurement,Genoni2013,gao2014bounds,bradshaw2017tight,bradshaw2018ultimate,assad2020accessible,park2022optimal} can be cast as multiparameter estimation problems. Given this motivation, there has been significant experimental~\cite{steinlechner2013quantum, vidrighin2014joint,hou2016achieving,liu2018loss,li2023optimal} and theoretical~\cite{vaneph2013quantum,suzuki2015parameter,suzuki2016explicit,szczykulska2016multi,proctor2018multiparameter,gessner2018sensitivity,tsang2019quantum,carollo2019quantumness,tsang2020quantum,demkowicz2020multi,razavian2020quantumness,gessner2020multiparameter,lu2021incorporating,gebhart2021bayesian,albarelli2022probe,huang2021quantum,gianani2021kramers,hanamura2021estimation,di2022multiparameter,hosseiny2022estimating,fadel2022multiparameter,len2022multiparameter,xie2022quantum} interest in quantum multiparameter estimation. See Refs.~\cite{liu2019quantum,albarelli2020perspective,sidhu2020geometric,polino2020photonic} for recent reviews on the subject. 


The physical process of quantum metrology can be described as a quantum channel with several different variations possible depending on the degree to which we wish to exploit quantum mechanical effects \cite{bennett1998quantum}. In this work, we consider four different schemes; Classical-Classical (CC), Classical-Quantum(CQ), Classical with ancilla-Classical with ancilla($\text{C}_\text{a}$$\text{C}_\text{a}$) and Classical with ancilla-Quantum with ancilla($\text{C}_\text{a}$$\text{Q}_\text{a}$). These four schemes are distinguished by how the states are prepared, either using entangled states or non-entangled states, and by how they are measured at the output, as shown in Fig.~\ref{fig:scheme}. Note that, on the state preparation side, we only consider either single-qubit or two-qubit states. This is distinct from previous studies that have considered the case where entanglement is generated across all input states before the channel~\cite{giovannetti2006quantum}. In our model, as the quantum states pass through the channel they experience a small rotation about all three axes by angles $\theta=(\theta_x,\theta_y,\theta_z)$ before experiencing some form of decoherence. The quantum states are then measured, either using collective measurements or individual measurements. Individual measurements here means that the probe states are measured one by one, in contrast to the most general measurement type, a collective measurement, which involves measuring multiple probe states simultaneously in an entangling basis. The quantum measurement strategies in Fig.~\ref{fig:scheme}, thus refer to performing a collective measurement on asymptotically many copies of the probe state. However, we can also consider performing a collective measurement on a finite number of copies of the probe state. It is known that collective measurements offer no advantage over individual measurements (i.e. the CC and CQ strategies are equivalent, as are $\text{C}_\text{a}$$\text{C}_\text{a}$ and $\text{C}_\text{a}$$\text{Q}_\text{a}$) for estimating only a single parameter or for estimating multiple parameters with pure states~\cite{Matsumoto2002}. However, for the most general metrology problem, multiparameter estimation with impure probe states, quantum resources can offer an advantage at both the state preparation and state measurement stages. The advantage of quantum resources at either of these stages can be thought of as quantum enhanced metrology. 

%Despite the myriad physical motivations behind quantum metrology many questions remain unanswered and specific results are known for only a few examples. 
%Another bound is the Gill-Massar bound~\cite{gill2005state}. 
There are several known methods to determine fundamental limits on how accurately the parameters of interest can be measured. One common approach is to use the quantum Fisher information (QFI). There are several variants of the QFI, including the QFI based on the symmetric logarithmic derivative (SLD bound), introduced by Helstrom~\cite{helstrom1967minimum,helstrom1968minimum} and the QFI based on the right logarithmic derivative (RLD bound)~\cite{yuen1973}. It is known that for estimating a single parameter, the SLD bound can always be saturated, making it a particularly important bound. However, for estimating multiple parameters the SLD bound may not be attainable if the optimal measurements for estimating each parameter individually do not commute. An attainable bound on the ultimate achievable precision in quantum parameter estimation theory was formulated by Holevo~\cite{holevo2011probabilistic,holevo1973statistical}, the Holevo Cram{\'{e}}r-Rao bound. Throughout this paper we shall simply refer to this as the Holevo bound. The Holevo bound is important as it known to be asymptotically attainable when one uses a collective measurement on infinitely many copies of the probe state~\cite{kahn2009local,yamagata2013quantum,yang2019attaining}. As such, the Holevo bound offers a way to investigate the precision attainable with the CQ and $\text{C}_\text{a}$$\text{Q}_\text{a}$ strategies in the asymptotic limit. In several different scenarios, utilising pure states and/or estimating a single parameter, the SLD bound and Holevo bound have been saturated experimentally~\cite{yu2022quantum,li2022geometric} or there exist theoretical proposals to saturate these bounds~\cite{bradshaw2018ultimate,helstrom1968minimum}. However, it has recently been proven that if the SLD bound or Holevo bound cannot be saturated with individual measurements, then they cannot be saturated with any physical measurement~\cite{conlon2022gap}. This property is known as gap persistence and is an important caveat to the statement that the Holevo bound is asymptotically attainable.


In light of the practical difficulties associated with the SLD bound and Holevo bound, another bound of particular interest is the Nagaoka Cram{\'{e}}r-Rao bound (Nagaoka bound)~\cite{nagaoka2005new} which applies when one is restricted to measuring the probe states individually, i.e. the CC and $\text{C}_\text{a}$$\text{C}_\text{a}$ strategies in Fig.~\ref{fig:scheme}. As it was originally introduced, the Nagaoka bound applies only to two parameter estimation. This bound was later generalised to the $n$-parameter case by Hayashi~\cite{hayashi1997linear,conlon2021efficient}, which we shall refer to as the Nagaoka--Hayashi bound (NHB)\footnote{For estimating one or two parameters the NHB reduces to the Nagaoka bound and so in these scenarios we shall use the two names interchangeably.}. The Nagaoka bound is known to be a tight bound for probes existing in a two-dimensional Hilbert space~\cite{nagaoka2005generalization}, i.e. there always exists a measurement which can achieve the same variance as the Nagaoka bound. However, it has since been shown that the NHB is not always a tight bound~\cite{hayashi2022tight}. Importantly, the NHB will always give a variance smaller than or equal to that of the Holevo bound as individual measurements are a subset of collective measurements. Experimentally, the collective measurements required to surpass the NHB and approach the Holevo bound are extremely challenging to implement, hence there have been a very limited number of demonstrations of such measurements~\cite{roccia2017entangling,hou2018deterministic,parniak2018beating,wu2019experimentally,wu2020minimizing,yuan2020direct,conlon2023approaching,conlon2023discriminating}. \footnote{Also note that the techniques demonstrated in Ref.~\cite{martinez2023certification} could in principle be used to implement collective measurements.} Recently, there has been a great deal of work developing new computational techniques for calculating the Holevo bound~\cite{bradshaw2018ultimate,albarelli2019evaluating,sidhu2021tight} and NHB~\cite{conlon2021efficient}.

%It has recently been shown that both the Holevo bound and NHB can be solved using a semidefinite program (SDP)~\cite{albarelli2019evaluating,conlon2021efficient}. The recasting of the Holevo bound as a SDP was originally performed to compute the ultimate limits for estimating a displacement in both quadratures of a Gaussian probe~\cite{bradshaw2018ultimate}.  Additionally, an analytic approach which provides upper and lower bounds to the Holevo bound for estimating two parameters has recently been developed~\cite{sidhu2021tight}. These techniques greatly simplify the task of calculating the ultimate bounds on the achievable precision.


This work aims to demonstrate the efficacy of these new techniques by finding the single and two qubit probes which optimise the Holevo and Nagaoka bounds for several different noise channels. There have been several previous experimental and
theoretical considerations as to how noisy channels affect the
achievable precision \cite{dorner2009optimal,kacprowicz2010experimental,genoni2011optical,datta2011quantum}. In this work we study specific examples of these noisy channels for
qubit probes. We consider the problem of simultaneous estimation of
three independent rotations around the $x$, $y$ and $z$ axes of the
Bloch sphere. We find a hierarchy between the four
different schemes in Fig.~\ref{fig:scheme} in terms of the
achievable precision subject to a noisy channel; $\text{C}_\text{a}$$\text{Q}_\text{a}$$\geq \{\text{C}_\text{a}\text{C}_\text{a}, \text{CQ}\}
\geq \text{CC}$, with no general ordering between $\text{C}_\text{a}$$\text{C}_\text{a}$ and $\text{CQ}$. Although one might expect that more entangled probes offer a quantum advantage over unentangled probes, this is not necessarily
true. In very noisy channels probes with more entanglement can offer a disadvantage. As expected, it was also found that, for a noisy channel, collective measurements
offer an advantage over individual measurements. Thus, we can consider
these quantum mechanical effects as offering an increased robustness
to noise. Typically, when estimating multiple parameters there is a trade-off between the number of parameters we wish to measure and the accuracy with which we can measure them~\cite{ragy2016compatibility,kull2020uncertainty}, however we show for multiple-qubit probes this is not necessarily true. Some of the other, more surprising, features of quantum metrology are evident in the examples considered, for example we find scenarios where states which experience decoherence outperform those which do not and we find discontinuities in the Holevo bound. We note that investigating these distinct schemes is different to many other works which have studied the scaling of the variance with the number of input probe states $N$, i.e. Heisenberg scaling (1/$N^2$) or scaling at the standard quantum limit (1/$N$)~\cite{ballester2005optimal,imai2007geometry,wang2019heisenberg,napolitano2011interaction,cimini2019quantum,hayashi2022global}.

%we find decohered states which are able to achieve a better measurement precision than states with less decoherence and

%
%We will begin by introducing the relevant bounds used in this paper in section \ref{Prelim}. In sections \ref{firstres} to \ref{lastres} we discuss the results obtained for specific channels. Finally in section \ref{generalisation} we generalise some of the results obtained to arbitrary channels and probe states. In section \ref{generalisation} we show that it is impossible, with a single qubit probe, to construct an unbiased estimator for estimating a rotation about all three axes. However,this does not eliminate the possibility of using a biased estimator to estimate these three parameters. It is known that biased estimators can outperform unbiased estimators in some situations~\cite{liu2016valid}. In section \ref{generalisation} we also consider how the estimation performance of different probes is affected by the estimation of a second and third parameter. Typically when estimating multiple parameters there is a trade-off between how many parameters we wish to measure and the accuracy with which we can measure them~\cite{ragy2016compatibility, kull2020uncertainty}., however we show for multiple-qubit probes this is not necessarily true
We begin by introducing the relevant bounds used in this paper in section \ref{Prelim}. In sections \ref{firstres} to \ref{lastres} we discuss the results obtained for specific channels. For each channel we consider single qubit probes, two qubit probes and the attainability of the Holevo bound. In the appendices we construct explicit measurement schemes which saturate the Nagaoka bound and NHB for many of the examples considered.
\section{Preliminaries}
\label{Prelim}


\subsection{Parameter estimation and quantum Fisher information}
%We consider a finite $d$-dimensional Hilbert space. 
The family of states being investigated,
$\rho_{\theta}$, are parameterised by
${\theta}=(\theta_{1},\dots, \theta_{n})$, where the $\theta_{j}$
are the unknown parameters we wish to estimate. In this paper the
$\theta_{j}$ are qubit rotations on the Bloch sphere. We can measure
the parameters we wish to estimate using a positive operator-valued
measure (POVM). A POVM is described by a set of positive linear
operators, $\Pi_{k}$, that sum up to the identity
\begin{equation}
\label{eq:POVMcondition}
\sum_{k}\Pi_{k}=\mathbbm{1}\;.
\end{equation}
The $k$-th outcome is realized with probability
$\text{Tr}\{\rho_{\theta}\Pi_{k}\}$. Based on these measurement outcomes we can
construct an unbiased estimator for the parameters of interest,
$\hat{\theta}$. Our estimated value is constructed from the estimator coefficients $\xi$
\begin{equation}
\hat{\theta}_j=\sum_{k}\xi_{j,k}p_{\theta}(k)\;,
\end{equation}
where $p_{\theta}(k)$ is the probability of obtaining the
measurement outcome denoted $k$ and the sum is over all possible
measurement outcomes. The aim of parameter estimation is to minimise the sum
of the mean squared error between our unbiased estimate and the actual
values we wish to measure, $\theta$. For estimating several parameters
simultaneously the mean-square error matrix,
$V_{\theta}$, has elements given by
\begin{equation}
\begin{split}
[V_{\theta}]_{jk}&=\sum_{x}\left(\xi_{j,x}-\theta_{j}\right)\left(\xi_{k,x}-\theta_{k}\right)p_{\theta}(x)\;.
%&=\left(\hat{\theta}_{j}-\theta_{j}\right)\left(\hat{\theta}_{k}-\theta_{k}\right)\;.
\end{split}
\end{equation}

If we have $N$ independent and identical copies of
the quantum state, $\rho_{\theta}$, the sum of the 
variance of our estimators is bounded by the quantum Cram{\'e}r--Rao bound
\begin{equation}
\text{Tr}\left\{ V_{\theta}\right\} \geq \frac{1}{N} \text{Tr}\left\{J(\rho_{\theta})^{-1}\right\}\;,
\label{QCRB}
\end{equation}
where $J(\rho_{\theta})$ is the QFI matrix with
elements,% \comment{The equation for the QCR bound is not
%true for a general QFI matrix. The LHS above corresponds to the usual
%variance only when we use the SLD QFI.}
\begin{equation}
  \label{eq:QFI}
  \left[J(\rho_{\theta})\right]_{jk}=\text{Tr}\left\{\frac{\partial\rho_{\theta}}{\partial
    \theta_j} \mathcal{L}_k 
\right\}\;,
\end{equation}
and $\mathcal{L}$ is the quantum analogue for the classical
logarithmic derivative. There is no
unique way to define the quantum logarithmic derivative and each
definition gives rise to a different QFI. %Although the different QFI's are not tight bounds in general, they have become popular owing to their ease of computation~\cite{pinel2013quantum}. 
Two of the most prominent QFI's are those based on the symmetric logarithmic derivative (SLD) and the right logarithmic derivative (RLD). The SLD and RLD operators combined with Eqs.~\eqref{QCRB} and \eqref{eq:QFI} give rise to the SLD bound, $C^\text{SLD}$, and the RLD bound, $C^\text{RLD}$ respectively. Although neither the SLD bound nor the RLD bound are attainable in general, both bounds are useful in certain scenarios. For this work, the SLD bound shall be used when we consider estimating a single parameter, as in this scenario it is known that the SLD bound is attainable. The SLD operators, $\mathcal{L}$, can be computed as
\begin{equation}
\label{eq:sld:comp}
\mathcal{L}_k =2\sum_{m,p}\ket{e_m}\frac{\bra{e_m}\frac{\partial\rho_{\theta}}{\partial
    \theta_k}\ket{e_p}}{\lambda_m+\lambda_p}\bra{e_p}\;,
\end{equation}
where $\ket{e_m}, \lambda_m$ are the eigenvectors and eigenvalues of the density matrix, $\rho=\sum_{m}\lambda_m\ket{e_m}\bra{e_m}$, and the sum is over all $\lambda_m+\lambda_p\neq0$~\cite{pinel2013quantum}. Thus, for any given problem the SLD QFI is generated in a completely deterministic manner meaning no optimisation is required. As we shall only use the SLD bound for single parameter estimation, the corresponding bound on the variance in estimating the parameter $\theta_k$ is given by
\begin{align}
\label{eq:SLDbound}
 v_k\geq C^\text{SLD}=\frac{1}{J(\rho_{\theta})_{kk}}\;,
\end{align}
with $J$ defined using the SLD operator, Eq.~\eqref{eq:sld:comp}.


%However, in this work we shall focus on computing alternative bounds which require a non-trivial minimisation, as these bounds are tighter and hence more physically meaningful.
%For
%completeness, we mention one particularly important
%version of the logarithmic derivatives which was introduced by Helstrom: the SLD, which gives rise to the SLD Fisher information matrix, $J_\text{SLD}(\theta)$ \cite{helstrom1976quantum}. To be precise the variance described in Eq~\ref{QCRB} corresponds to the SLD QFI. The SLD operators,
%$\mathcal{L}_{ i}$, are formally defined as the solution to
%\begin{equation}
%\frac{\partial \rho_{\theta}}{\partial \theta_j}=\frac{1}{2}(\rho_{\theta}\mathcal{L}_{ j}+\mathcal{L}_{ j}\rho_{\theta})\;.
%\end{equation}
%%The RLD operators, $\mathcal{\tilde{L}}_{j}$, are formally defined as the solution to
%%\begin{equation}
%%\frac{\partial {\rho}_\theta}{\partial \theta_j}=\rho_{\theta}\mathcal{\tilde{L}}_{j}\;.
%%\end{equation}
%For a given probe, the SLD operator can be computed as
%\begin{equation}
%\label{SLD1}
%\mathcal{L}_j(\theta)= \sum_{mn} \frac{\ket{e_m}\bra{e_m} \dot\rho_{\theta_j} \ket{e_n}\bra{e_n}  }{m_\text{SLD}(\lambda_m,\lambda_n)}\;,
%\end{equation}
%where $\dot\rho_{\theta_j}=\frac{\partial \rho}{\partial \theta_j}$, $\lambda_{m}$ represents the $m$-th eigenvalue of the probe, $\rho_{\theta}$,
%with corresponding eigenvector $\ket{e_m}$ and the sum is over all $n,m$ where $m_\text{SLD}(\lambda_m,\lambda_n)\neq0$. $m_\text{SLD}$ is the
%arithmetic mean function given by
%\begin{equation}
%m_\text{SLD}(\lambda_{m},\lambda_{n})=\frac{\lambda_{m}+\lambda_{n}}{2}\;.
%\end{equation}
%%Similarly we can calculate the RLD bound with
%%\begin{equation}
%%\tilde{\mathcal{L}}_j(\theta)= \sum_{mn} \frac{\ket{e_m}\bra{e_m} \dot\rho_{\theta_j} \ket{e_n}\bra{e_n}  }{m_\text{RLD}(\lambda_m,\lambda_n)}\;,
%%\end{equation}
%%where the function $m_\text{RLD}$ here is the harmonic mean function
%%\begin{equation}
%%m_\text{RLD}(\lambda_{i},\lambda_{j})=\frac{2}{\frac{1}{\lambda_{i}}+\frac{1}{\lambda_{j}}}\;.
%%\end{equation}
%
%
%Having the logarithmic derivative operator, we can obtain the SLD
%Fisher information matrix, $J_\text{SLD}(\theta)$ %and the RLD Fisher
%%information, $J_\text{RLD}(\theta)$
% from Eq.~(\ref{eq:QFI}). The individual
%elements are given by
%\begin{align}
%\left[J_\text{SLD}(\rho_{\theta})\right]_{jk}&= \frac{1}{2}\text{tr}[\rho_{\theta}(\mathcal{L}_{\theta, j}\mathcal{L}_{\theta, k}+\mathcal{L}_{\theta, k}\mathcal{L}_{\theta, j})]\;.
%\end{align}
%From the QCR bound we can lower bound the mean-square error matrix using this definition of the Fisher information. The SLD bound is given by
%\begin{equation}
%\text{Tr}\left\{V_{\theta}(\hat{\theta})\right\} \geq \frac{1}{N}\text{Tr}\left\{J_\text{SLD}(\rho_{\theta})^{-1}\right\}\eqqcolon \frac{1}{N}C^\text{SLD}.
%\end{equation}
%%and the RLD bound is given by
%%\begin{equation}
%%\text{Tr}\left\{V_{\theta}(\hat{\theta})\right\} \geq\frac{1}{N}
%%\text{Tr}\left\{J_\text{RLD}(\theta)^{-1}\right\} \eqqcolon
%%\frac{1}{N} C^\text{RLD}\;.
%%\end{equation}
%%\comment{When computed this way, the LHS is not the usual
%%  variance. Check with jun Suzuki's paper on the expression for the
%%  RLD bound. We can condense this derivation, just state the results
%%  or put it in an appendix if we're not using the SLD and RLD bounds
%%  in the results section.}


\subsection{Holevo and Nagaoka--Hayashi bounds}
Holevo unified the SLD bound and the RLD bound through the Holevo bound, which we denote $\mathcal{H}$. The Holevo bound is achieved asymptotically and is assured to be at
least as informative as  $C^\text{SLD}$ or  $C^\text{RLD}$, i.e. $\mathcal{H}\geq C^\text{SLD},C^\text{RLD}$. The Holevo bound
involves a minimisation over $X=(X_1,X_2,\ldots,X_n)$, where $X_j$ are
Hermitian operators that satisfy the unbiased conditions
\begin{align}
\label{eq_xcon1}
\text{Tr}\left\{\rho_{\theta} X_j\right\}&=0 \;, \\
\label{eq_xcon2}
\text{Tr}\left\{\frac{\partial \rho_\theta}{\partial\theta_j} X_k\right\}&=\delta_{jk}\;.
\end{align}
The Holevo bound is
\begin{align}
\label{eq_hol2}
\mathcal{H} \coloneqq  \min_{X} \text{Tr}\left\{  Z_\theta[X]\right\} +\text{TrAbs} \left\{\Im Z_\theta[X]\right\}\;,
\end{align}
where
\begin{align}
\label{eq_zmat}
Z_\theta[X]_{jk} \coloneqq  \text{Tr} \left\{ \rho X_j X_k\right\}\;,
\end{align}
and  $\text{TrAbs}\{\text{Im}Z_{\theta}[X]\}$ is
the sum of the absolute values of the eigenvalues of the matrix $\text{Im}Z_{\theta}[X]$.
$Z$ takes the role of the inverse of the Fisher information matrix. The Holevo bound sets a limit to the sum of the variance of an unbiased estimate
\begin{align}
  \text{Tr}\{V_\theta\}\geq \mathcal{H}\;.
\end{align}
Holevo derived this bound in his original work~\cite{holevo1976noncommutative}, but the bound in the form shown above was introduced by
Nagaoka~\cite{nagaoka2005new}.  A major obstacle preventing the more
widespread use of the Holevo bound is that, unlike the RLD and SLD
bounds which can be calculated directly, the Holevo bound involves a
non-trivial optimisation problem. However, as mentioned in the introduction, these difficulties have been somewhat alleviated in recent years~\cite{bradshaw2018ultimate,albarelli2019evaluating,sidhu2021tight}.

%The Holevo bound is an upper bound on the accessible information obtained using collective measurements.

 % For two-qubits, we perform this
% optimisation numerically. For three-qubit probes in order to calculate
% the optimal X matrices we make certain simplifying assumptions about
% the symmetry of these matrices.

For estimating two parameters, Nagaoka derived the bound
\begin{align}
\label{eq_hol2N}
  \text{Tr}\{V_\theta\}\geq \mathcal{N} \coloneqq  \min_{X} \text{Tr}\left\{  Z_\theta[X]\right\}
  +\text{TrAbs} \left\{ \rho [X_1,X_2]\right\}\;,
\end{align}
valid if we are restricted to individual measurements. This bound is always
more informative than or equal to the Holevo bound. As mentioned in the introduction, for two-dimensional
systems, it is known to be attainable~\cite{nagaoka2005new} and was conjectured by Nagaoka to be
attainable in higher dimensional systems as well~\cite{nagaoka2005generalization}. For estimating more than two parameters with individual measurements we shall use the NHB~\cite{hayashi1997linear,conlon2021efficient}. The NHB can be computed from the following optimisation problem~\cite{conlon2021efficient}
 \begin{align}
 \text{Tr}\{V_\theta\}\geq\mathcal{N}\coloneqq   \min_{\mathbb{L},\,X}\left\{
                          \bbTr{\mathbb{S}_\theta
                          \mathbb{L}}\,\big|\, \mathbb{L}_{jk}=\mathbb{L}_{kj}\, \mathrm{
       Hermitian, }\, \mathbb{L}\geq {X} X^\intercal
    \right\} \;,\label{eq:NHB}
\end{align}
where $\mathbb{S}_\theta= {1}_n\otimes \rho_\theta$, ${1}_n$ is the $n\times n$ identity matrix\footnote{Note that a different notation is used for identity matrices which exist in the classical vector space, compared to those in the quantum Hilbert space as in Eq.~\eqref{eq:POVMcondition}.} and $X=(X_1,X_2,...,X_n)^\intercal$ is a vector of Hermitian estimator observables $X_j$ that satisfy the locally unbiased conditions, Eqs.~\eqref{eq_xcon1} and \eqref{eq_xcon2}. The matrix $\mathbb{S}_\theta$, exists on an extended quantum-classical Hilbert space. We use the symbol $\bbTr{\cdot}$ to denote trace over both classical and quantum systems. We shall use the symbol $\mathcal{N}$ to denote both the Nagaoka bound and NHB, and it will be obvious from the number of parameters being estimated which we are referring to. We can define the most informative bound as the precision achievable through individual measurements, $C^\text{MI}$, such that for 2 dimensional systems $C^\text{MI}=\mathcal{N}$~\cite{nagaoka2005new}. As mentioned in the introduction, the NHB is not always a tight bound~\cite{hayashi2022tight}. An alternative bound for individual measurements was introduced by Gill and Massar~\cite{gill2005state}. However, as this bound is, in general, less tight than the NHB we shall not consider it in this work.



There is a hierarchy between the bounds described above, $C^\text{MI}\geq\mathcal{N}\geq\mathcal{H}\geq\max(C^\text{RLD},C^\text{SLD})$. However, it is known that for
estimating a single parameter the SLD bound, the Holevo
bound, the NHB and the most informative bound coincide, $C^\text{MI}=\mathcal{N}=\mathcal{H}=C^\text{SLD}$~\cite{suzuki2016explicit,suzuki2019information}. This gives a simple method of
explicitly computing the achievable precision for estimating one
parameter.  Additionally, when estimating any number of parameters using pure states, the Holevo bound and the NHB are equal, $\mathcal{N}=\mathcal{H}$, i.e. the CC and CQ strategies are equivalent, as are the $\text{C}_\text{a}$$\text{C}_\text{a}$ and $\text{C}_\text{a}$$\text{Q}_\text{a}$ strategies. Finally, note that although~the Holevo bound is tighter than the SLD bound, it has recently been proven that the difference between the two is at most a factor of 2~\cite{carollo2019quantumness,tsang2020quantum}. 

\subsection{Quantum Channels}
\label{channel}
In this work, we consider estimating qubit rotations in a noisy channel. There are three rotation parameters we wish to estimate: $\theta_x$, $\theta_y$ and $\theta_z$. These three
parameters describe the rotation amplitudes about the three
axes of the Bloch sphere:
\begin{align}
  \label{eq:three_rotations}
  U(\theta_x,\theta_y,\theta_z) = \exp(\I \frac{\theta_z}{2}\sigma_z) \exp(\I \frac{\theta_y}{2}\sigma_y) \exp(\I \frac{\theta_x}{2}\sigma_x)\;,
\end{align}
where $\sigma_x$, $\sigma_y$ and $\sigma_z$ are the Pauli matrices. We insist that the rotation amplitudes are small so that the order of the rotations does not matter. Note that this assumption of small rotations is relevant provided it is possible to perform some prior characterisation of the quantity to be estimated, such as in Ref.~\cite{li2023optimal}. We are interested in finding the
optimal single-qubit and two-qubit probes for estimating
these parameters. We shall consider the three cases with: (i) a single
rotation when $\theta_y=\theta_z=0$, (ii) two rotations when
$\theta_z=0$ and (iii) all three rotations present. Depending on the number of parameters, the
optimal probe and measurement strategy will be different.
For certain channels, when estimating a single parameter $\theta_x$, as we shall show later, any pure state in the
$y{-}z$ plane will be optimal. 

The rotations transform an input quantum state $\rho$ to the state $ \rho_\theta=U(\theta_x,\theta_y,\theta_z)\rho U(\theta_x,\theta_y,\theta_z)^\dagger$. In our model, this rotated state is then subject to some noise. A noisy quantum channel can be described in the operator-sum representation as a linear map acting on the density matrix of the state subjected to the channel,
\begin{equation}
\label{krauss}
\mathcal{E}_\theta(\rho)=\sum_{a=1}^{K}M_{a}\rho_\theta M_{a}^{\dagger}\;,
\end{equation}
where the operators, $M_{a}$, obey the completeness relation, $\sum_{a=1}^{K}M_{a}^{\dagger}M_{a}=\mathbbm{1}$. In what follows, we shall drop the explicit dependence of the quantum channel on $\theta$. The operators are known as operational elements of the channel, or Kraus
operators~\cite{kraus1983states}. Note that, as is evident from Eq.~\eqref{krauss}, the overall quantum channel we consider involves rotating the quantum state before subjecting it to the noise. $K$ is known as the
Kraus number and satisfies $K\leq d^2$, where $d$ is the dimensions of
the system in question. For the map
$\mathcal{E}(\rho)$ to represent a deterministic, physical channel it
must be linear, trace-preserving and completely
positive~\cite{serafini2017quantum}. A deterministic channel here means
that we do not allow postselection on certain measurement
results. Quantum channels such as this are also called
`trace-preserving completely positive' maps. There are two distinct scenarios when considering estimation in the presence of noise. One is to perform the estimation using noisy probe states, $\mathcal{E}(\rho)$, i.e. the decoherence happens before the rotations. The other is to use pure probe states, where the decoherence occurs after the rotation. In this work we consider the second option. In all cases we assume that the noise parameter is known, i.e. we do not need to treat it as a nuisance parameter as has been considered before~\cite{suzuki2020nuisance,suzuki2020quantum}.
% Figure environment removed


\section{Results}
We shall now present our results on the optimal estimation variances
for three channels: 1) decoherence channel, 2) amplitude damping channel and 3) phase damping
channel. Fig.~\ref{fig:BS_effect} shows the effect each channel has on the Bloch sphere. These three channels have been considered before in the context of the QFI~\cite{ozaydin2014phase, ma2011quantum}, and their physical motivation includes energy dissipation and decoherence in trapped ions~\cite{huelga1997improvement, myatt2000decoherence, turchette2000decoherence, nielsen2002quantum}. For each channel, we present
the optimal achievable precisions for estimating one and two qubit rotations with single qubit probes and one, two and three
qubit rotations with two qubit probes. For single parameter estimation with single qubit probes, the optimal probe state is determined from the SLD bound, Eq.~\eqref{eq:SLDbound}. In all other scenarios we find the optimal probe numerically. We compare the performance of using an entangled
two-qubit probe, where only one qubit passes through the channel, with that of having only a single qubit probe. Thus, in our work, computing the NHB for the single qubit and two-qubit probe states, corresponds to the CC and $\text{C}_\text{a}$$\text{C}_\text{a}$ strategies respectively. Computing the Holevo bound for the single qubit and two-qubit probe states, corresponds to the CQ and $\text{C}_\text{a}$$\text{Q}_\text{a}$ strategies respectively. When we wish to compute the precision attainable with collective measurements on a finite number of copies, $M$, of the probe state, we shall evaluate the NHB for the state $\rho^{\otimes M}$, scaled by a factor of $M$ to ensure a fair comparison in terms of resources used. Note that the Holevo bound for the state $\rho^{\otimes M}$, scaled by a factor of $M$, also provides a bound on the precision attainable with collective measurements on $\rho^{\otimes M}$. However, this bound is only guaranteed to be tight as $M\to\infty$. In many of the problems we consider we are able to obtain analytic solutions for the matrices which optimise the Holevo and Nagaoka bounds based on the symmetries of the system. However, an analytic solution was not always possible and in this case a combination of numerics and guesswork was used to obtain results. %All of the results presented in the main text 
%We present numerical simulations which support our analytic results.% {\color{red}The probe space is convex}, and both the NHB and Holevo-bound are convex functions, so we are guaranteed that there exist no local minima, so finding the optimal probe is easy numerically.


\subsection{Decoherence channel}
\label{firstres}
We first explicitly compute the achievable precisions for individual and
collective measurements with one and two-qubit probes under
the action of a decoherence channel.  This channel is represented
by the following Kraus operators
\begin{align}
\begin{array}{rclcrcl}
    M_{0}&=&\sqrt{1-\frac{3\epsilon}{4}}\mathbbm{1}&,&M_{1}&=&\sqrt{\frac{\epsilon}{4}}\sigma_{x}\;,\\
M_{2}&=&\sqrt{\frac{\epsilon}{4}}\sigma_{y}&, &M_{3}&=&\sqrt{\frac{\epsilon}{4}}\sigma_{z}\;,
\end{array}
\end{align}
where $0\leq \epsilon \leq 1$ parametrises the decoherence strength.
These Kraus operators act only on the probe qubit as we wish to consider the situation where only a single
qubit passes through the channel \cite{preskillnotes}, as shown in Fig.~\ref{fig:scheme}.


\subsubsection{Single qubit probe}
\label{singqub}
% Figure environment removed

With a single qubit, the optimal probe for sensing a rotation about
the $x$-axis is any pure state in the Y-Z plane of the Bloch sphere. This is easily verified as, for estimating a single parameter, the Holevo bound coincides with the SLD bound, Eq.~\eqref{eq:SLDbound}. For the probe $\ket{\psi}=\text{cos}(\theta/2)\ket{0}+e^{\I\phi}\text{sin}(\theta/2)\ket{1}$ and a particular choice of the decoherence parameter, $\epsilon=0.05$, we show the Holevo bound as a function of the Bloch sphere angles, $\theta$ and $\phi$, in Fig.~\ref{BS_DC} (i). This figure verifies our claim that any probe in the Y-Z plane ($\phi=\pi/2,3\pi/2$) is optimal. The computation of the SLD bound is described in appendix~\ref{apennewSLD}. We now consider the state $\ket{0}$, one of the many possible optimal states. After the decoherence channel this probe is left in the the state
\begin{equation}
\rho=\ket{0}\bra{0}(1-\epsilon)+\frac{\epsilon}{2}\;.
\end{equation}
%When estimating a single parameter we use the SLD bound to compute the optimal precision. 
For a single qubit the optimal $X_{x}$ matrix for estimating a rotation about the
$x$-axis is
\begin{equation}
\label{xone}
X_{x}=\frac{\I}{1-\epsilon}\left(\ket{0}\bra{1}-\ket{1}\bra{0}\right)\;.
\end{equation}
We show how $X_{x}$ is computed in appendix \ref{apenxmat}. Using this matrix, the Holevo bound is given by
\begin{equation}
\label{Hol1}
v_x\geq \mathcal{H}^\text{d}_{1 \text q,1} =\frac{1}{(1-\epsilon)^{2}}\;.
\end{equation}
We use the superscript `d' to denote the decoherence channel, the first subscript `1q'
to denote a one-qubit probe and the second subscript `1' to denote single parameter estimation.
%In this case the SLD bound is tight. %but the RLD bound is not,
%$v_{1,\text{RLD}}(\theta_{x})=\frac{2\epsilon
%  (1-\frac{\epsilon}{2})}{(1-\epsilon)^{2}}$. 
For single parameter estimation, an individual measurement
  provides as much information as a collective measurement~\cite{ragy2016compatibility}.  Hence
  \begin{align}
    \label{NagD1}
    v_x \geq \mathcal{N}^\text{d}_{1\text q,1} = \frac{1}{(1-\epsilon)^2}\;,
  \end{align}
where we use $\mathcal{N}$ to indicate the Nagaoka bound. In appendix~\ref{apenDC1q} we show an explicit measurement reaching this bound. 
We now consider estimating two parameters, rotations about the $x$ and $y$ axes. The $\ket{0}$ probe is
still optimal as shown in Fig.~\ref{BS_DC} (ii). The
optimal precision with individual measurements can be obtained using
the Nagaoka bound and is given by
\begin{equation}
\label{Nag12}
v_x+v_y \geq \mathcal{N}^\text{d}_{1\text q,2}=\frac{4}{(1-\epsilon)^{2}}\;,
\end{equation} 
which is exactly four times the single parameter Nagaoka bound. This
implies that an optimal strategy for estimating both parameters is to
estimate each parameter separately: use $N/2$ probes to estimate $\theta_x$
and the remaining $N/2$ probes to estimate $\theta_y$. We show a measurement strategy reaching this bound in appendix~\ref{apenDC1q}. The  $X$-matrix which is required to achieve this Nagaoka bound is given by
\begin{equation}
X_{y}=\frac{1}{1-\epsilon}\left(\ket{0}\bra{1}+\ket{1}\bra{0}\right)\;.
\end{equation}
The same $X$-matrix also optimises the Holevo bound, giving
\begin{equation}
\label{Hol12}
v_x+v_y \geq \mathcal{H}^\text{d}_{1\text q,2}=\frac{4-2\epsilon}{(1-\epsilon)^{2}}\;,
\end{equation}
which is slightly smaller than the corresponding Nagaoka bound when $\epsilon$ is
not equal to 0 or 1. Thus, in this case, we can achieve better
precision by performing a collective measurement. The optimal strategy
gives
\begin{align}
v^*_x=v^*_y = \frac{1}{2}\mathcal{H}^\text{d}_{1\text q,2}=\frac{2-\epsilon}{(1-\epsilon)^2}\;,
\end{align}
which is always greater than $\mathcal{H}^\text{d}_{1\text q,1}$. This
indicates that by estimating the second parameter we lose some precision in our
estimate of the first parameter as can be expected.% As the Holevo bound does not equal the Nagaoka bound for any $0<\epsilon<1$, the gap persistence theorem shows that the Holevo bound cannot be saturated by any physical measurement in this scenario~\cite{conlon2022gap}.

%
%If we try to estimate a third parameter using the single qubit probe
%the performance deteriorates further, We can show that for a general
%one qubit probe $\ket{\psi}=\ket{0}a+\ket{1}b$, it is impossible to
%construct an unbiased estimator for estimating a rotation about all
%three axis, if $a$ and $b$ are both real numbers. This is
%because $\partial \rho /\partial \theta_z$ is a multiple of
%$\partial \rho /\partial \theta_x$
%\begin{equation}
%\frac{\partial \rho}{\partial \theta_x}= \frac{\partial \rho}{\partial \theta_z}\left(\frac{b^{2}-a^{2}}{2ab}\right).
%\end{equation}
%Thus,it is impossible to construct an unbiased estimator to
%simultaneously estimate $\theta_{x}$ and $\theta_{z}$ with a single
%qubit with real coefficients---every estimator is biased. We extend this result in the appendix. %~\comment{(What about
%  %a probe with complex coefficients, pointing to (1,1,1)?)} \change{I think even with complex coefficients we still can't construct an unbiased estimator}


\subsubsection{Two-qubit probe}


%% Figure environment removed
For estimating one, two or three rotations with a two-qubit probe, we numerically verify that any maximally
entangled two-qubit probe is equally optimal. Note that, in a slightly different setting, the maximally entangled state was proven to be optimal in the noiseless case~\cite{fujiwara2001estimation}. We consider the following probe
\begin{align}
\ket{\psi_{0}} &=\frac{1}{\sqrt{2}} \left(\ket{01}+\ket{10}\right)\;.
\label{2qstart}
\end{align}
After passing through the channel the probe becomes
\begin{equation}
\rho=(1-\epsilon)\ket{\psi_{0}}\bra{\psi_{0}}+\frac{\epsilon}{4}\mathbbm{1}\;.
\end{equation}
For estimating one parameter with this probe the optimal $X$ matrix is
\begin{equation}
\label{xtwo}
X_{x}=\frac{\I}{(1-\epsilon)}\left(\ket{\psi_{0}}\bra{\psi_{2}}-\ket{\psi_{2}}\bra{\psi_{0}}\right)\;,
\end{equation}
where
\begin{align}
\ket{\psi_{2}} &=\frac{1}{\sqrt{2}} \left(\ket{00}+\ket{11}\right)\;.
\end{align}
The bounds for both variances, using individual and collective measurements, are given by 
\begin{equation}
\label{Hol2qDC}
v_{x} \geq \mathcal{H}^\text{d}_{2\text q,1}=\mathcal{N}^\text{d}_{2\text q,1}=\frac{2-\epsilon}{2(1-\epsilon)^{2}}\;,
\end{equation}
which coincides with the SLD bound. In appendix~\ref{app2q2cDC} we show that by performing a measurement on both qubits we can achieve this precision. We note that $\mathcal{H}^\text{d}_{2\text
  q,1}\leq\mathcal{H}^\text{d}_{1\text q,1}$ which indicates that using a
two-qubit probe gives a better estimation precision even though the
second probe does not go through the channel. We quantify the
estimation precision as
\begin{align}
  \label{eq:4}
  \text{precision} = \frac{1}{v^*_x}\;,
\end{align}
where $v^*_x$ is the variance obtained from the optimal strategy. We compare single-qubit and two-qubit probes for estimating a single parameter in Fig.~\ref{Indv_singl}.

% Figure environment removed


We now proceed to estimating two parameters using an entangled
two-qubit probe. Similar to the single parameter case, we can write
%\begin{align}
%\frac{\partial \rho}{\partial \theta_y}&=\frac{(1-\epsilon)}{2}[\ket{\psi_{0}}\bra{\psi_{3}}+\ket{\psi_{3}}\bra{\psi_{0}}]\\
%\frac{\partial \rho}{\partial \theta_z}&=\frac{i(1-\epsilon)}{2}[\ket{\psi_{1}}\bra{\psi_{0}}-\ket{\psi_{0}}\bra{\psi_{1}}]\;,
%\end{align}
\begin{align}
X_{y}=\frac{-1}{(1-\epsilon)}\left(\ket{\psi_{0}}\bra{\psi_{3}}+\ket{\psi_{3}}\bra{\psi_{0}}\right)\;,\\
X_{z}=\frac{-\I}{(1-\epsilon)}\left(\ket{\psi_{1}}\bra{\psi_{0}}-\ket{\psi_{0}}\bra{\psi_{1}}\right)\;,
\label{xzone}
\end{align}
where
\begin{align}
  \ket{\psi_{1}} &=\frac{1}{\sqrt{2}} \left(\ket{01}-\ket{10}\right) \;, \\
\ket{\psi_{3}} &=\frac{1}{\sqrt{2}} \left(\ket{00}-\ket{11}\right)\;,
\label{2qend}
\end{align}
to arrive at
\begin{align}
\label{Hol22DCjoint}
v_x+v_y \geq  \mathcal{H}^\text{d}_{2\text{q},2} =  \frac{2-\epsilon}{(1-\epsilon)^2}\;.
\end{align}
The optimal variance of each parameter is
\begin{align}
\label{Hol22DCindividual}
  v^*_x=v^*_y= \frac{1}{2}   \mathcal{H}^\text{d}_{2\text{q},2} =  \frac{2-\epsilon}{2(1-\epsilon)^2}\;,
\end{align}
which is exactly equal to the single parameter result
$\mathcal{H}^\text{d}_{2\text q,1}$ in Eq.~\eqref{Hol2qDC}. Hence, we
find that with a two-qubit probe, we can estimate a second parameter
without any degradation in the precision of the first. For the two-qubit probe we are also able to compute the Nagaoka bound,
\begin{equation}
\label{Nag22DC}
v_x+v_y \geq \mathcal{N}^\text{d}_{2\text{q},2}=\frac{4-\epsilon}{2(1-\epsilon)^{2}}\;.
\end{equation}
For a general $\epsilon$, this Nagaoka bound is larger than the corresponding Holevo
bound indicating that an individual measurement is inferior to a
collective measurement. In appendix \ref{app2q2cDC} we construct a measurement scheme which saturates this bound.

% Figure environment removed


 Finally, for estimating three parameters, with the optimal $X_{z}$ matrix given in Eq.~\eqref{xzone}, we find that
\begin{align}
\label{Hgen3DC}
v_x+v_y+v_z \geq  \mathcal{H}^\text{d}_{2\text{q},3} =  \frac{6-3\epsilon}{2(1-\epsilon)^2}\;,
\end{align}
and the optimal variance of each parameter is
\begin{align}
  v^*_x=v^*_y=v^*_z =\frac{1}{3}   \mathcal{H}^\text{d}_{2\text{q},3} =  \frac{2-\epsilon}{2(1-\epsilon)^2}\;.
\end{align}
This is once again equal to the single parameter result
$\mathcal{H}^\text{d}_{2\text q,1}$ in Eq.~(\ref{Hol2qDC}). Hence, we
find that with a two-qubit probe, we can estimate all three parameters
simultaneously just as well as we can estimate just one parameter. The ability of entangled probe states to avoid trade-offs in multiparameter estimation has been observed before~\cite{baumgratz2016quantum,bradshaw2017tight,bradshaw2018ultimate}.
However, when restricted to individual measurements the NHB is given by
\begin{align}
v_x+v_y+v_z \geq  \mathcal{N}^\text{d}_{2\text{q},3} =  \frac{3}{(1-\epsilon)^2}\;,
\label{Ngen3DC}
\end{align}
and the optimal variance of each parameter is
\begin{align}
  v^*_x=v^*_y=v^*_z =\frac{1}{3}   \mathcal{N}^\text{d}_{2\text{q},3} =  \frac{1}{(1-\epsilon)^2}\;.
\end{align}
We see that when we are restricted to individual measurements the estimation of another parameter further degrades measurement precision. The same $X$ matrices which optimise the Holevo bound, optimise the NHB. In appendix~\ref{decohapen3} we show that there exists a measurement strategy which saturates the NHB in this case. The differences between individual and collective measurement precisions
are highlighted in Fig.~\ref{Indv_mult}. This figure shows the hierarchy
between the different schemes mentioned earlier in the text:
$\text{C}_\text{a}$$\text{Q}_\text{a}$$\geq \{\text{C}_\text{a}\text{C}_\text{a}\text{,CQ}\} \geq$ CC. For this particular example we find that $\text{C}_\text{a}$$\text{C}_\text{a}$$\geq$CQ.


% Figure environment removed

\subsubsection{Decoherence of both qubits}
Thus far we have considered the case where the second qubit experiences no decoherence. This lack of decoherence is equivalent to storing the second qubit in a perfect quantum memory, something which is not feasible with current technology.
Thus, we now consider the channel where both qubits in the two qubit probe experience some decoherence. We expose the two qubit probe to the channel where the first and second qubit experience decoherence amplitudes of $\epsilon_{1}$ and $\epsilon_{2}$ respectively. Under the action of this channel for estimating either one or two parameters the maximally entangled two qubit probe achieves a Holevo bound of 
\begin{equation}
\label{2qdecoh}
\mathcal{H}^\text{2d}_{2\text q,1}=\frac{1}{2}\mathcal{H}^\text{2d}_{2\text q,2}=\frac{1-\frac{1}{2}(\epsilon_{1}+\epsilon_{2})+\frac{1}{2}\epsilon_{1}\epsilon_{2}}{(1-\epsilon_{1})^{2}(1-\epsilon_{2})^{2}}\;.
\end{equation}
Although we have shown the advantages offered by two qubit probes over single qubit probes, this expression highlights the dangers associated with using highly entangled probes for estimation. Eq~(\ref{2qdecoh}) is symmetric in $\epsilon_{1}$ and $\epsilon_{2}$, which is somewhat surprising given that the rotation we are trying to estimate acts on the first qubit only. In spite of this, decoherence of the second qubit is equally damaging to our estimation ability. We see that when the second qubit is fully decohered we are unable to estimate with any precision at all, regardless of the decoherence of the first qubit. This is shown in Fig.~\ref{Indv_decoh_DC}

Perhaps the most physically relevant scenario is the one in which both qubits experience the same decoherence, i.e. $\epsilon_{1}=\epsilon_{2}$. In this situation for estimating a single parameter the single qubit probe described in section \ref{singqub} always outperforms the two qubit probe considered. For estimating two parameters with the two qubit probe the Holevo bound, Eq~(\ref{2qdecoh}), remains unchanged. Even though, when using the maximally entangled two qubit probe, we can estimate a second parameter for free, when the noise in the system is sufficiently high the single qubit probe can still outperform the two qubit probe for estimating two parameters. Thus, we can see that the optimal probe to use in this instance depends on the decoherence amplitudes experienced by both qubits. It is worth noting that in this high noise regime a different two qubit probe will be optimal and the optimised two qubit probe will always perform better than or equal to the optimised single qubit probe. The fact that the maximally entangled two qubit probe is no longer optimal is a reflection of the fact that highly entangled states are very susceptible to loss and noise, see Refs.~\cite{dorner2009optimal,demkowicz2012elusive}.


% Figure environment removed

\subsubsection{Collective measurements on multiple copies of the state}
Before concluding discussions on the decoherence channel, we consider what happens when we can perform collective measurements on \textit{M} copies of the single qubit probe, i.e. we have the state $\rho^{\otimes \textit{M}}$ available to measure. As $\textit{M}\rightarrow\infty$, we expect to find $\mathcal{N}_{1q^{\otimes \textit{M}}, 2}^{d}\rightarrow\mathcal{H}_{1q^{\otimes \textit{M}},2}^{d}=\frac{1}{\textit{M}}\mathcal{H}_{1q,2}^{d}$, where we have used the additivity of the Holevo bound~\cite{hayashi2008asymptotic}. Let us first consider $\textit{M}=2$. With two copies the probe becomes $\rho^{\otimes 2}=\text{diag}\{\left(1-\frac{\epsilon}{2}\right)^2, \frac{\epsilon}{2}\left(1-\frac{\epsilon}{2}\right), \frac{\epsilon}{2}\left(1-\frac{\epsilon}{2}\right), \frac{\epsilon^2}{4}\}$. The derivatives of this matrix with respect to the parameters of interest are given by:
\begin{equation}
  \begin{gathered}
  \frac{\partial \rho_\theta}{\partial \theta_x}=
  \begin{pmatrix}0&\I A&\I A&0\\
   -\I A&0&0&\I B\\
    -\I A&0&0&\I B\\
    0&-\I B&-\I B&0
  \end{pmatrix}\;\text{and}\qquad
  \frac{\partial \rho_\theta}{\partial \theta_y}=\frac{1}{4}
  \begin{pmatrix}0&A&A&0\\
   A&0&0& B\\
   A&0&0& B\\
    0& B& B&0
  \end{pmatrix}\;,
  \end{gathered}
\end{equation}
where $A=\frac{1}{2}(1-\epsilon)\left(1-\frac{\epsilon}{2}\right)$ and $B=\frac{1}{4}(1-\epsilon)\epsilon$. For this state the optimal $X_x$ and $X_y$ matrices are given by
\begin{equation}
  \begin{gathered}
  X_x=
  \frac{\I}{2(1-\epsilon)}
  \begin{pmatrix}0&1& 1&0\\
   -1&0&0&1\\
    -1&0&0&1\\
    0&-1&-1&0
  \end{pmatrix}\;\text{and}\qquad
  X_y=\frac{1}{2(1-\epsilon)}
  \begin{pmatrix}0&1&1&0\\
   1&0&0&1\\
   1&0&0&1\\
    0&1&1&0
  \end{pmatrix}\;,
  \end{gathered}
\end{equation}
which give a Nagaoka bound of $\mathcal{N}_{1q^{\otimes 2}, 2}^{d}=(2-\epsilon+\frac{\epsilon^2}{2})/(1-\epsilon)^2$. So it is clear that with just two copies of the probe the Nagaoka bound is close to the Holevo bound. A similar result has been observed recently for optical magnetometry systems~\cite{friel2020attainability} and a measurement saturating the two-copy Nagaoka bound has been implemented experimentally~\cite{conlon2023approaching}. In appendix~\ref{app:1q2cDC} we show a measurement scheme which attains this bound. With the development of recent techniques, it is now possible to compute the Nagaoka bound efficiently~\cite{conlon2021efficient}. This allows us to compute the Nagaoka bound for many copies of the single qubit probe. In Fig.~\ref{decohmcop} we compare the Holevo bound to the Nagaoka bound for an increasing number of copies of the single qubit probe. We plot the difference in achievable precisions as a measure of how close the two bounds are. As expected, with an increasing number of copies of the probe state, the Nagaoka bound tends to the Holevo bound. However, these results are purely theoretical and in an experimental implementation with current capabilities such precisions can only be reached for measurements on a limited number of copies of the probe state~\cite{conlon2023approaching}. Note that if we are restricted to performing separable measurements (non-entangling POVMs) on the state $\rho^{\otimes M}$, it is known that there is no advantage compared to individual measurements~\cite{hayashi2016quantum}.
%




\newpage
\subsection{Amplitude Damping Channel}
We now consider an amplitude damping channel. This channel models
the decay of a two level atom from the excited state to the ground
state. The Kraus operators for this channel are:
\begin{equation}
\label{ampeq}
M_{0}=
\begin{pmatrix}
1&0\\
0&\sqrt{1-p}\\
\end{pmatrix}\;,\quad
M_{1}=
\begin{pmatrix}
0&\sqrt{p}\\
0&0\\
\end{pmatrix}\;.
\end{equation}
These operators model an atom which if it is in the excited state will
decay to the ground state with probability $p$, and if it is in
the ground state will remain unaffected. Thus, when we consider this
channel we get significantly different variances depending on the
probe we chose for the single qubit case. For example, the
$\ket{0}$ and $\ket{1}$ states, corresponding to the ground and
excited states respectively, will behave differently
depending on the decoherence amplitude~\cite{preskillnotes}. If we
wish to consider the time evolution of an atomic system we can
imagine applying these operators to our quantum state once per time
interval. In each time interval the atomic system has a certain
probability of decaying and as $t\rightarrow\infty$ all of the atoms
end up in the ground state.

\subsubsection{Single qubit probe}
Similar to the decoherence channel in the single qubit case, the optimal probe for sensing a rotation about
the $x$-axis is any pure state which lies in the Y-Z plane of the Bloch sphere, shown in Fig.~\ref{fig:AD_xy_cont} (i).
These probes are able to estimate a
single parameter with a Holevo bound of
\begin{equation}
v_x\geq \mathcal{H}^\text{am}_{1\text{q},1}=\frac{1}{1-p}\;.
\label{adfirst}
\end{equation}
At first sight, it might seem strange that the state $\ket{0}$ which
is unaffected by the channel will perform just as well as $\ket{1}$
which decays through the channel. The reason for this is that the
probe undergoes the rotation before the decoherence. After the information
has already been encoded onto the probe, the amplitude damping channel
destroys the information at the same rate for both probes. This can be
seen from the derivatives
\begin{align}
  \frac{\partial }{\partial \theta_x}\mathcal{E}\left(\ket{0}\bra{0}\right)=
  -\frac{\partial }{\partial \theta_x}\mathcal{E}\left(\ket{1}\bra{1}\right)=
   -\frac{\sqrt{1-p}}{2} \sigma_y\;,
\end{align}
which only differ in sign for the two probes. Here $\mathcal{E}$
represents the amplitude damping channel. The $X$ matrix which saturates this Holevo bound is 
\begin{equation}
\label{xxampd}
X_{x}=\pm \frac{1}{\sqrt{1-p}}\sigma_y\;,
\end{equation}
where the $+$ is for the $\ket{1}$ state and the $-$ is for the $\ket{0}$ state. For estimating a single parameter using these probes the Holevo bound is equal to the SLD bound.

For two parameter estimation using individual measurements, the two probes $\ket{0}$ and $\ket{1}$
remain optimal, as shown in Fig~\ref{fig:AD_xy_cont} (ii). We find that for either probe, the Nagaoka bound is given by
\begin{equation}
v_x+v_y \geq \mathcal{N}^\text{am}_{1\text q,2}=\frac{4}{1-p}\;,
\label{amp1nagjoint}
\end{equation}
which is exactly four times the optimal single parameter estimate. In appendix~\ref{apenAD1q} we describe the measurement strategy required to reach this bound. The optimal
strategy is to use half of the probes to estimate $\theta_x$ and the
remaining half to estimate $\theta_y$. The variance in each estimate is
\begin{equation}
v_x^{*}=v_y^{*} =\frac{1}{2} \mathcal{N}^\text{am}_{1\text q,2}=\frac{2}{1-p}\;.
\label{amp1nag}
\end{equation}

Interestingly, when allowing for collective measurements, the optimal probe for
estimating a rotation about the $x$ and $y$ axes is the state
$\ket{1}$, which now outperforms the state $\ket{0}$. This is
very surprising since the $\ket{1}$ probe is affected by the
channel, while the $\ket{0}$ probe isn't. This can be viewed as \textit{decoherence assisted metrology}. We can understand this phenomenon by observing that although the state $\ket{0}$ does not experience any decoherence, the rotated state $\ket{0}+(\theta_y-\mathrm{i}\theta_x)/2\ket{1}$ does experience decoherence. Fig.~\ref{fig:AD_xy_cont} (iii) depicts the difference between the two probes $\ket{0}$ and $\ket{1}$.The difference in the probe
performance can be attributed to the difference in the partial
derivatives of the probe after the channel. The Holevo bounds for these two probes apply in the asymptotic limit, but this limiting behaviour is already present if we consider a collective measurement on two probes. We can see that
\begin{align}
  \label{eq:3}
  \frac{\partial}{\partial \theta_x}( \rho \otimes \rho) =
  \frac{\partial \rho}{\partial \theta_x} \otimes \rho+
  \rho \otimes \frac{\partial \rho}{\partial \theta_x}\;,
\end{align}
will be different for the probes $\ket{0}$ and $\ket{1}$ even if they
have the same $\partial \rho/\partial \theta_x$. We will return to this in section 4. The probe $\ket{1}$ achieves a Holevo bound for estimating two parameters of\footnote{The Holevo bound obtained from the suboptimal probe $\ket{0}$ is
$4/(1-p)$ which coincides with the result for an individual
measurement. Thus, collective measurements do not provide any advantage
when using the probe $\ket{0}$.}
\begin{equation}
  v_x+v_y \geq \mathcal{H}^\text{am}_{1\text q,2}=
  \bigg\{\begin{array}{lll}
           4&\text{for}&p \leq 1/2\;,\\
           \frac{4p}{1-p}&\text{for}&p > 1/2\;.
                                      \end{array}
                                     \;.
\label{amphol2joint}
\end{equation}
For every $0<p<1$, a collective measurement will give greater
precision compared to an individual measurement.
The minimum variance attained by the optimal probe $\ket{1}$ is given by
\begin{equation}
  v_x^{*}=v_y ^{*} = \frac{1}{2} \mathcal{H}^\text{am}_{1\text q,2}=
    \bigg\{\begin{array}{lll}
           2&\text{for}&p \leq 1/2\;,\\
           \frac{2p}{1-p}&\text{for}&p > 1/2\;.
                                      \end{array}
                                      \;.
\label{amphol2}
\end{equation}
This is plotted in Fig.~\ref{Ampdamp}. The $X_y$ matrix required to saturate this bound is
\begin{equation}
\label{ytwo}
X_{y}=\frac{-1}{\sqrt{1-p}}\sigma_x\;.
\end{equation}

% Figure environment removed


\subsubsection{Two-qubit probe}
% Figure environment removed

We now consider the two qubit case where the first qubit is exposed to
the amplitude damping channel and the second qubit is passed through
the identity channel. We consider the probe $\frac{1}{\sqrt{2}}(\ket{0, 1}+\ket{1, 0})$ which exhibits some of the advantages offered by two qubit probes. Using this probe we are able to estimate a rotation around the $x$-axis with a Holevo bound given by
\begin{equation}
v_x \geq \mathcal{H}^\text{am}_{2\text{q},1}=\mathcal{N}^\text{am}_{2\text{q},1}=\frac{1}{1-p}\;.
\label{adhol2}
\end{equation}
Again this coincides with the SLD bound. Unlike the decoherence
channel, for estimating a single parameter under the action of this
channel the entangled two qubit probe doesn't actually offer any
advantage. For estimating two parameters using individual measurements, the Bell state is optimal and the minimum variance that can be obtained is
\begin{equation}
  v_x+v_y \geq \mathcal{N}^\text{am}_{2\text q,2}=
  \left\{\begin{array}{lll}
           \frac{16\left(3-\sqrt{1-p}\right)\left(1-\sqrt{1-p}\right)}{p(8+p)(1-p)}&\text{for}&0<p \leq 2\sqrt{2}-2\;,\\
           \frac{4p}{1-p}&\text{for}& 2\sqrt{2}-2 < p <1\;.
                                      \end{array}
                                      \right.
\label{ampnag2}
\end{equation}
For all $0<p<1$, this probe is always better than the optimal single qubit probe. The $X$ matrices required to saturate this bound when $p<2\sqrt{2}-2$ are
\begin{equation}
  \begin{gathered}
  X_x=
  \begin{pmatrix}0&-a\I& -b\I&(-1+\I)w\\
   a\I&0&0&a\I\\
    b\I&0&0&b\I\\
    (-1-\I)w&-a\I&-b\I&0
  \end{pmatrix}\;\text{and}\qquad
  X_y=
  \begin{pmatrix}0&-a& -b&(-1-\I)w\\
   -a&0&0&a\\
    -b&0&0&b\\
    (-1+\I)w&a&b&0
  \end{pmatrix}\,,
  \end{gathered}
\end{equation}
where $a=2/(4(1-p)+(4-p)\sqrt{1-p})$, $b=2(3+p)/((4+p)(1-p)+(4+2p)\sqrt{1-p})$ and $w=\sqrt{(a^2+b^2)/2}$. When $p>2\sqrt{2}-2$ the required matrices are
\begin{equation}
  \begin{gathered}
  X_x=
  \begin{pmatrix}0&-\I& \frac{-\I}{\sqrt{1-p}}&c_{14+}\\
   \I&c_{22}&-c_{23}\I&\frac{\I p}{2\sqrt{1-p}}-\sqrt{1-p}c_{34}\\
    \frac{\I}{\sqrt{1-p}}&c_{23}\I&\frac{-c_{22}}{1-p}&\frac{\I p}{2(1-p)}+c_{34}\\
    c_{14-}&\frac{-\I p}{2\sqrt{1-p}}-\sqrt{1-p}c_{34}&\frac{-\I p}{2(1-p)}+c_{34}&-c_{44}
  \end{pmatrix}\;,
    \end{gathered}
\end{equation}
and
\begin{equation}
  \begin{gathered}
  X_y=
  \begin{pmatrix}0&-1& \frac{-1}{\sqrt{1-p}}&c_{14-}\\
   -1&-c_{22}&-\I c_{23}&\frac{p}{2\sqrt{1-p}}-\I\sqrt{1-p}c_{34}\\
    \frac{-1}{\sqrt{1-p}}&\I c_{23}&\frac{c_{22}}{1-p}&\frac{p}{2(1-p)}+\I c_{34}\\
    c_{14+}&\frac{p}{2\sqrt{1-p}}+\I\sqrt{1-p}c_{34}&\frac{p}{2(1-p)}-\I c_{34}&c_{44}
  \end{pmatrix}\,,
    \end{gathered}
  \end{equation}
where $c_{14\pm}=\frac{-1\pm\I}{\sqrt{2}}\sqrt{\frac{2-p}{1-p}}$, $c_{22}=\sqrt{(p^2+4p-4)/(2(2-p))}$, $c_{23}=\sqrt{(p^2+4p-4)/(2(1-p)(2-p))}$, \newline $c_{34}=\sqrt{p^2+4p-4}/(2(1-p))$ and $c_{44}=p\sqrt{p^2+4p-4}/(\sqrt{2(2-p)}(1-p))$. These matrices are constructed such that $X_x$ and $X_y$ commute. A POVM can then be formed from the simultaneous eigenvectors of the two matrices which saturates the Nagaoka bound. 
% Although we cannot find an analytic expression for the optimal POVM required to reach the NHB, as we can with the other channels, we can numerically verify that there always exists a 4-outcome projective POVM that can saturate this bound for all p.

When allowing for collective measurements, the maximally entangled
probe gives
\begin{equation}
  v_x+v_y \geq \mathcal{H}^\text{am}_{2\text q,2}(\text{bell})=
  \left\{\begin{array}{lll}
           \frac{(2-p)^2}{2(1-p)^2}&\text{for}& p \leq 2/3\;,\\
           \frac{4p}{1-p}&\text{for}& 2/3 < p <1\;.
                                      \end{array}
                                      \right.
\label{amphol2ME}
\end{equation}
Plotting both $\mathcal{H}^\text{am}_{2\text q,2}$ and
$\mathcal{H}^\text{am}_{1\text q,2}(\text{bell})$ in Fig.~\ref{Ampdamp}, we see that at some
values of $p$, the single qubit probe $\ket{1}$ outperforms the
maximally entangled probe. This means that the maximally entangled
probe is not always the optimal two qubit probe for sensing the
channel. The optimal two qubit probe depends on the noise in the channel. The optimal two qubit probe can be written as
  $\ket{\psi}=a\ket{0,1}+b\ket{1,0}$, where $a$ and $b$ depend
  on $p$. With an optimised probe, we get
\begin{equation}
  v_x+v_y \geq \mathcal{H}^\text{am}_{2\text q,2}=
  \bigg\{\begin{array}{lll}
           \frac{2}{1-p}&\text{for}& p \leq 1/2\;,\\
           \frac{4p}{1-p}&\text{for}& 1/2 < p <1\;.
                                      \end{array}
                                      %\right
                                      \label{opholad}
\end{equation}
The maximally entangled probe is only optimal when $p=0$ and when
$p \geq 2/3$. For $p \geq 1/2$ the probe $\ket{1,0}$, which performs
identically to the single qubit probe $\ket{1}$, is optimal.  When
$p<1/2$, the optimal two-qubit probe is
\begin{align}
  \label{eq:3}
  \ket{\psi} = \sqrt{\frac{1-2p}{2-2p}}\ket{0,1}+\frac{1}{\sqrt{2-2p}}\ket{1,0}\;.
\end{align}
The optimal $X_x$ and $X_y$ matrices corresponding to this probe are
\begin{align}
  \label{eq:8}
  X_x&=\frac{-\I}{\sqrt{1-p}}\left(\ket{0}\bra{1}-\ket{1}\bra{0} \right)
      \otimes \ket{0}\bra{0}-
      \frac{\I \sqrt{1-2p}}{1-p} \ket{1}\bra{1} \otimes \left(\ket{1}\bra{0}-\ket{0}\bra{1} \right) \;,\\
  X_y&=-\frac{1}{\sqrt{1-p}}\left(\ket{0}\bra{1}+\ket{1}\bra{0} \right)
      \otimes \ket{0}\bra{0}+
      \frac{\sqrt{1-2p}}{1-p} \ket{1}\bra{1} \otimes \left(\ket{1}\bra{0}+\ket{0}\bra{1} \right)\;.
\end{align}
By direct substitution, one can check that these matrices satisfy
the unbiased conditions. By direct substitution, we get the $Z$ matrix
as
\begin{align}
  Z=\begin{pmatrix}
 \frac{1}{1-p}&0\\
 0&\frac{1}{1-p}
 \end{pmatrix}\;,
\end{align}
which gives $\mathcal{H}=2/(1-p)$. The attainable precisions for estimating two parameters with both single and two qubit probes are plotted in Fig.~\ref{Ampdamp}. Notably this example shows that in different regimes, it is possible to have either $\text{C}_\text{a}$$\text{C}_\text{a}$$>$CQ $\left(1/\mathcal{N}^{\text{am}}_{2q,2}>1/\mathcal{H}^{\text{am}}_{1q,2}\right)$ or CQ$>$$\text{C}_\text{a}$$\text{C}_\text{a}$ $\left(1/\mathcal{H}^{\text{am}}_{1q,2}>1/\mathcal{N}^{\text{am}}_{2q,2}\right)$.

%For measuring a rotation about the $x$ and $y$ axis using this
%probe the Holevo bound coincides with the Nagaoka bound and both are
%given by\comment{The $X$ and $Y$ matrices didn't give this result?}
%\begin{equation}
%v_x+v_y \geq \mathcal{H}^\text{am}_{2\text q,2}= \mathcal{N}^\text{am}_{2\text q,2} =\frac{2}{1-p}.
%\end{equation}
%Thus,in this instance collective measurements offer no advantage over individual measurements. The optimal strategy is to have the same estimation ability in both directions
%\begin{equation}
%v_x^{*}=v_y ^{*} = \frac{1}{2} \mathcal{H}^\text{am}_{2\text q,2}=\frac{1}{1-p}\;.
%\label{amphol2}
%\end{equation}
%Thus,using this probe we are able to estimate the second parameter for
%free. We see that the two qubit probe offers an advantage over the
%single qubit probes for estimating multiple parameters.


Using the maximally entangled state as our probe, we find that the Holevo bound for estimating three parameters is given by
\begin{equation}
v_{x}+v_{y}+v_{z}\geq\mathcal{H}_{2q,3}^\text{am}= 
 \bigg\{\begin{array}{lll}
           \frac{(3-2p)(2-p)}{2(1-p)^{2}}&\text{for}& p \leq 2/3\;,\\
           \frac{(2+7p)}{2-2p}&\text{for}& 2/3 < p <1\;.
                                      \end{array}
                                      \label{adall3mark2}
\end{equation}
The optimal $X_x$, $X_y$ and $X_z$ matrices for this probe are 
%\begin{align}
%X_x&=i\,c_1(\ket{0,0}\bra{0,1}-\ket{0,1}\bra{0,0})+i\,c_2(\ket{0,0}\bra{1,0}-\ket{1,0}\bra{0,0})+ic_3(\ket{1,1}\bra{1,0}-\ket{1,0}\bra{1,1}),\\
%X_y&=c_1(\ket{0,0}\bra{0,1}+\ket{0,1}\bra{0,0})+c_2(\ket{0,0}\bra{1,0}+\ket{1,0}\bra{0,0})-c_3(\ket{1,1}\bra{1,0}+\ket{1,0}\bra{11}),\\
%X_z&=i\,c_2(-\ket{0,1}\bra{1,0}+\ket{1,0}\bra{0,1}),
%\end{align}
\begin{equation}
  \begin{gathered}
  X_x=-
\I
  \begin{pmatrix}0&c_1& c_2&0\\
   -c_1&0&0&0\\
    -c_2&0&0&-c_3\\
    0&0&c_3&0
  \end{pmatrix}\;\text{,}\qquad
  X_y=-
  \begin{pmatrix}0&c_1& c_2&0\\
   c_1&0&0&0\\
    c_2&0&0&-c_3\\
    0&0&-c_3&0
  \end{pmatrix}\,\text{and}\qquad
  X_z=\I 
  \begin{pmatrix}0&0& 0&0\\
   0&0&-c_2&0\\
    0&c_2&0&0\\
    0&0&0&0
  \end{pmatrix}\,,
  \end{gathered}
\end{equation}
where 
\begin{align}
c_1&= \bigg\{\begin{array}{lll}
           \frac{p}{2-2p}&\text{for}& 0\leq p \leq 2/3 \;,\\
           1&\text{for}&  2/3< p <1\;,
                                      \end{array}
                                     \;,\\
c_2&=\frac{1}{\sqrt{1-p}}\;,\\
c_3&=\frac{1}{1-p}-c_1\;.\\
\end{align}
We note that the maximally entangled probe is not the optimal
probe for estimating three parameters under the action of this
channel. In Fig.~\ref{ad3opt} however, it can be seen that the maximally entangled
probe achieves a variance which is very close to the numerically
optimised Holevo bound.


% Figure environment removed

\subsubsection{Decoherence of both qubits}
If we consider the more realistic channel where each qubit is individually subject to the amplitude damping channel then the estimation performance can only at best match the performance of the channel where the second qubit is not affected by the channel. For estimating $\theta_{z}$ using a $\frac{1}{\sqrt{2}}\left(\ket{0,1}+\ket{1,0}\right)$ probe with both qubits exposed to separate amplitude damping channels, with probabilities $p_{1}$ and $p_{2}$ respectively, the Holevo bound is given by
\begin{equation}
\label{addual}
v_z \geq \mathcal{H}^\text{2am}_{2\text q,1}=\frac{1}{2-2p_{1}}+\frac{1}{2-2p_{2}}\;.
\end{equation}
%When $p_{2}=0$ this reduces to the Holevo bound for estimating only $\theta_{z}$ using the two qubit probe considered. 
We can compare this to the Holevo bound for estimating $\theta_{z}$  using a single qubit probe, which is given by Eq.~\eqref{adfirst}, if we orient the probe such that it is optimal for sensing a rotation about the $z$-axis. We again see that, although we are trying to estimate a rotation on the first qubit only, the Holevo bound is symmetric in $p_{1}$ and $p_{2}$. If the second qubit is maximally exposed to the amplitude damping channel the first qubit can no longer be used for parameter estimation. When both qubits are exposed to the same decoherence amplitude, $p_{1}=p_{2}$, the Holevo bound for the two qubit probe is equal to the Holevo bound for the single qubit probe. Thus, although more entangled probes offer some advantages under certain conditions, they also have weaknesses that single qubit probes do not have, shown in Fig.~\ref{ad_both_decoh}.  

% Figure environment removed

\subsubsection{Collective measurements on multiple copies of the state}
As with the decoherence channel, we now consider performing collective measurements on \textit{M} copies of the single qubit state. The
probes $\ket{0}$ and $\ket{1}$ perform equally well for simultaneous
estimation of two rotations about the $x$ and $y$ axes when we are restricted
to individual measurements. Both probes give a Nagaoka bound of $\mathcal{N}^{\text{am}}_{1\text{q},2}\left(\ket{0}\right)=\mathcal{N}^{\text{am}}_{1\text{q},2}\left(\ket{1}\right) = 4/(1-p)$. The Nagaoka bound is known to be attainable for a two-dimensional state. An optimal measurement is to perform a measurement of $\sigma_x$ half of
the time and a measurement of $\sigma_y$ the other half of the time, as shown in appendix \ref{apenAD1q}. However, we know from the Holevo bound, that when allowing for collective measurements on infinitely many probes, the probe $\ket{1}$ actually outperforms
$\ket{0}$. For
the probe $\ket{0}$, collective measurements do not provide any
advantage over individual measurements, but for the probe $\ket{1}$, collective measurements
improve the precision. We show how this is possible by constructing an explicit
estimator to approach the Holevo bound. We first consider doing a
collective measurement on the probe $\ket{1,1}$.

After the amplitude damping channel, the probe $\ket{1}$ transforms to
the mixed state $p \ket{0}\bra{0}+(1-p)\ket{1}\bra{1}$. Two copies of this
will be in the state $\rho_1=p^2 \ket{0,0}\bra{0,0}+p(1-p)\ket{0,1}\bra{0,1}+p(1-p) \ket{1,0}\bra{1,0}+(1-p)^2\ket{1,1}\bra{1,1}$. As this is a mixed state, collective measurements may offer an advantage. By performing the optimisation over the matrices $X$ and $Y$, we find that the
Nagaoka bound for the state $\ket{0,0}$ and $\ket{1,1}$ is
\begin{align}
  \label{nagad1q2c}
  \mathcal{N}^{\text{am}}_{1\text{q}^{\otimes2},2}\left(\ket{0,0}\right) &= \frac{2}{1-p}\;,\\
  \mathcal{N}^{\text{am}}_{1\text{q}^{\otimes2},2}\left(\ket{1,1}\right) &= \frac{2}{1-p}-2p\;.
    \label{nagad1q2c22}
\end{align}
The result for $\ket{0,0}$ is not surprising. As $\ket{0,0}$ is unaffected by this channel, it remains as a pure state and hence collective measurements cannot offer any benefit. Therefore, the minimum variance is exactly half of
the individual measurement case, meaning that the optimal
estimation is to measure each probe individually.  But the result for
$\ket{1,1}$ is smaller than half of the individual measurement
case. This is now a four dimensional state and it is not certain that
the Nagaoka bound can still be attained\footnote{Nagaoka conjectured
  it is always attainable, but only proved for certain specific
  cases.}.  We construct an explicit measurement which
saturates the bound in appendix \ref{11est}, showing that the bound is attainable. 

The optimal $X$ and $Y$ matrices that give the Nagaoka bound for
$\ket{1,1}$ probe are\footnote{For the probe $\ket{0,0}$, the optimal
  matrices have the same form except for a sign change.}
%\begin{align}
%  \label{eq:1}
%  X&=\frac{1}{\sqrt{1-p}} \big(\ket{y+,y+}\bra{y+,y+}-\ket{y-,y-}\bra{y-,y-} \big)\;,\\
%  Y&=-\frac{1}{\sqrt{1-p}} \big(\ket{x+,x+}\bra{x+,x+}-\ket{x-,x-}\bra{x-,x-} \big)\;.
%\end{align}
\begin{align}
  \label{eq:1}
 X_x&=\frac{1}{\sqrt{1-p}} \big(\ket{y+,y+}\bra{y+,y+}-\ket{y-,y-}\bra{y-,y-} \big)\;,\\
  X_y&=-\frac{1}{\sqrt{1-p}} \big(\ket{x+,x+}\bra{x+,x+}-\ket{x-,x-}\bra{x-,x-} \big)\;,
\end{align}
where $\ket{y\pm}=\left(\ket{0}\pm \mathrm{i}\ket{1}\right)/\sqrt{2}$ and $\ket{x\pm}=\left(\ket{0}\pm \ket{1}\right)/\sqrt{2}$ are the eigenvectors of $\sigma_y$ and $\sigma_x$ respectively. In Fig.~\ref{AD_collective} we show how the Nagaoka bound tends towards the Holevo bound for the probe $\ket{1}$, with an increasing number of copies of the probe. This shows the same effect as Fig.~\ref{decohmcop}, but in a slightly different representation. 


% Figure environment removed

\newpage
\subsection{Phase Damping Channel}
\label{lastres}
Finally, we consider the phase damping channel. This channel describes the loss of information without the loss of energy and is unique to quantum systems. It is represented by the Kraus operators
\begin{equation}
M_{0}=\sqrt{1-\frac{\epsilon}{2}}\mathbbm{1}\;,\quad M_{1}=\sqrt{\frac{\epsilon}{2}}\sigma_{z}\;.
\end{equation}
This channel can be thought of as applying $\sigma_{z}$ with probability $\frac{\epsilon}{2}$. We can see why this channel is uniquely quantum mechanical by considering repeated applications of the channel to an arbitrary density matrix. If the initial quantum state is
\begin{equation}
\rho_{0}=\begin{pmatrix}
a&b\\
b^{*}&c\\
\end{pmatrix}\;,
\end{equation}
then after the action of this channel the state becomes
\begin{equation}
\rho_{\epsilon}=\begin{pmatrix}
a&(1-\epsilon)b\\
(1-\epsilon)b^{*}&c\\
\end{pmatrix}\;.
\end{equation}
Repeated application of this channel is equivalent to allowing $\epsilon\rightarrow 1$. The density matrix loses its off-diagonal elements and ends up as a completely diagonal matrix, i.e. the quantum state has become a completely classical mixture of $\ket{0}$'s and $\ket{1}$'s. In this way the quantum mechanical properties of the system are lost. Thus, this channel does not have a direct classical analogue.


\subsubsection{Single Qubit Probe}
Geometrically this channel represents a loss of information about
the $x$ and $y$ components of the Bloch vector. When restricted to real probes, for estimating a rotation about the $x$-axis the probes
$\ket{0}$ and $\ket{1}$ perform identically under the action of this
channel. Both probes are able to estimate a rotation about either the
$x$ or the $y$-axis individually with a Holevo bound given by Eq.
\eqref{Hol1}. This coincides with the SLD bound for these probes as is expected. The $X_{x}$ matrix required to saturate the Holevo bound is $\sigma_y/(1-\epsilon)$. However, by allowing complex probes the Holevo bound can be improved. The reason for this is that the phase damping channel affects the Bloch sphere in an asymmetric way. One optimal probe is $\ket{\psi}=\frac{1}{2}\left((1+\I)\ket{0}+(1-\I)\ket{1}\right)$ which can achieve a Holevo bound of 1, no matter what the decoherence amplitude. The $X_x$ matrix required to saturate this is $X_x=-\ket{0}\bra{0}+\ket{1}\bra{1}$. However, as is evident from Fig.~\ref{fig:PD_xy_comp} (i), almost any probe in the Y-Z plane of the Bloch sphere is optimal. For probes of the form $\ket{\psi}=\text{cos}(\theta/2)\ket{0}+e^{\I\phi}\text{sin}(\theta/2)$, any state with $\phi=\pi/2$ or $3\pi/2$ will be optimal, except for states where $\theta$ is exactly equal to $0$ or $\pi$. Similarly if we want to estimate a rotation about the $y$-axis only any state with $\phi=0$ or $\pi$ will be optimal, except when $\theta$ is exactly equal to $0$ or $\pi$. The discontinuity in the Holevo bound exactly at these extreme points has been observed before and corresponds to a point where the rank of the state changes~\cite{vsafranek2017discontinuities,vsafranek2018simple,seveso2019discontinuity,rezakhani2019continuity,goldberg2021taming,ye2022quantum}. The small rotation which we are trying to estimate changes the states $\ket{0}$ or $\ket{1}$ which are are rank 1 states and are not decohered to states which have rank 2 and are decohered. 


For estimating a rotation about the $x$ and $y$ axes simultaneously, consider the probe state $\ket{\psi}=a\ket{0}+\sqrt{1-a^2}\ket{1}$, with $a=1-\delta$, where $\delta$ is small. With this probe as $\delta\rightarrow 0$, the Holevo bound is given by
\begin{equation}
\label{hol12pd}
v_x+v_y \geq \mathcal{H}^\text{pd}_{1\text q,2}=    \bigg\{\begin{array}{lll}
           4&\text{for}&\epsilon \leq 1-1/\sqrt{2}\;,\\
           \frac{1}{(2-\epsilon)(1-\epsilon)^2\epsilon}&\text{for}&\epsilon > 1-1/\sqrt{2}\;.
                                      \end{array}
\end{equation}
It might be expected that one could always estimate $\theta_x$ and $\theta_y$ with a minimum variance of 4, given that it is possible to estimate both individually with a variance of 1, simply by using half the probes to estimate $\theta_x$ and the other half to estimate $\theta_y$. However,  Fig.~\ref{fig:PD_xy_comp} shows why this is not possible. The optimal probes for estimating each angle are different and so the multiparameter estimation is degraded. The $X_{y}$ matrix required to saturate this Holevo bound is given by
%\begin{equation}
%X_{y}=\begin{bmatrix}
%0&\frac{-1}{1-\epsilon}\\
%\frac{-1}{1-\epsilon}&\frac{2}{a\sqrt{1-a^2}} 
%\end{bmatrix},
%\end{equation}
%when $\epsilon\leq 1-\frac{1}{\sqrt{2}}$ and 
%\begin{equation}
%X_{y}=\frac{1}{\epsilon(2-\epsilon)}\begin{bmatrix}
%0&\frac{-1}{1-\epsilon}\\
%\frac{-1}{1-\epsilon}&\frac{1}{a\sqrt{1-a^2}} 
%\end{bmatrix},
%\end{equation}
%
\begin{equation}
X_{y}= \left\{\begin{array}{lll}
           \begin{pmatrix}
0&\frac{-1}{1-\epsilon}\\
\frac{-1}{1-\epsilon}&\frac{2}{a\sqrt{1-a^2}} 
\end{pmatrix}&\text{for}&\epsilon \leq 1-1/\sqrt{2}\;,\\
           \frac{1}{\epsilon(2-\epsilon)}\begin{pmatrix}
0&-1+\epsilon\\
-1+\epsilon&\frac{1}{a\sqrt{1-a^2}} 
\end{pmatrix}&\text{for}&\epsilon > 1-1/\sqrt{2}\;.
                                      \end{array}\right.
\end{equation}

We see when $a=0$ or $a=1$ these matrices contain infinities, but this is not reflected in the QFI which is optimised as $a\rightarrow1$. In fact the matrix $X_y$ only satisfies the unbiased conditions in Eqs.~(\ref{eq_xcon1}) and (\ref{eq_xcon2}), when $\delta\rightarrow0$. Interestingly exactly at $\delta=0$, so that $a=1$, the Holevo bound is only $\frac{4}{(1-\epsilon)^2}$. This discontinuity is again explained by the changing rank of the state at this point and is shown in Fig~\ref{fig:PD_xy_comp} (iii). 

For estimating the same two parameters with individual measurements we find that the Nagaoka bound is given by
\begin{equation}
\label{nag12pd}
v_x+v_y \geq \mathcal{N}^\text{pd}_{1\text q,2}= \frac{(2-\epsilon)^{2}}{(1-\epsilon)^{2}}\;.
\end{equation}
Thus, collective
measurements offer an advantage over individual measurements in this
particular instance. In appendix \ref{nagexp} we present a measurement saturating the Nagaoka bound, and show that it requires an unequal number of measurements of the two parameters. This is due to the fact that when estimating these two parameters individually we do not obtain the same minimum variance. To saturate the Nagaoka bound we require the following $X_y$ matrix
\begin{equation}
X_y=\frac{1}{\sqrt{1-a^2}}\left(\left(a-\frac{1}{a}\right)\ket{0}\bra{0}+a\ket{1}\bra{1}\right)\;.
\end{equation}
Again there is a discontinuity exactly at $a=1$; in this case the Nagaoka bound is $\frac{4}{(1-\epsilon)^2}$. 

% Figure environment removed

\subsubsection{Two-qubit probe}
We note that some of the results presented for two-qubit probes under the action of this channel have already been presented in Ref.~\cite{conlon2021efficient}, however we include them here for completeness. We consider the scenario where the first qubit is subject to the phase damping channel and the second auxiliary qubit experiences only the identity channel. The probes $\ket{0, 0}$ and $\ket{1, 1}$ perform identically to their single qubit counterparts. Thus, these probes are still unable to estimate a rotation about all three axes. This is to be expected since the probes are separable, hence the idler qubit has no effect.


% Figure environment removed


For estimating a rotation about the $x$-axis the probe $\frac{1}{\sqrt{2}}\left(\ket{0, 1}+\ket{1, 0}\right)$ is optimal. When we estimate a rotation about either the $x$-axis or the $y$-axis individually using this probe the Holevo bound is independent of the channel parameter $\epsilon$. The Holevo bound remains unaffected when we estimate a rotation about both the $x$ and $y$ axes simultaneously. 
\begin{equation}
\label{hol22pd}
v_x+v_y \geq \mathcal{H}^\text{pd}_{2\text q,2}=2\;.
\end{equation}
This is optimised when $v_{x}^{*}=v_{y}^{*}=1$. Thus, the entangled probe is able to estimate a second parameter at no additional cost. The $X_{x}$ and $X_{y}$ matrices which saturate the Holevo bound are given by
\begin{align}
X_{x}&=\I\left(\ket{\psi_{0}}\bra{\psi_{2}}-\ket{\psi_{2}}\bra{\psi_{0}}+\ket{\psi_{1}}\bra{\psi_{3}}-\ket{\psi_{3}}\bra{\psi_{1}}\right)\;,\\
X_{y}&=-\left(\ket{\psi_{2}}\bra{\psi_{1}}+\ket{\psi_{1}}\bra{\psi_{2}}+\ket{\psi_{0}}\bra{\psi_{3}}+\ket{\psi_{3}}\bra{\psi_{0}}\right)\;.
\end{align}
For estimating these two parameters simultaneously the Bell state is still optimal and the Nagaoka bound is given by:
\begin{equation}
\label{nagpd}
v_x+v_y \geq \mathcal{N}^\text{pd}_{2\text q,2}=\frac{4}{2-\epsilon}\;.
\end{equation}
The $X$ matrices required to achieve the minimum possible variance using individual measurements are given by
\begin{align}
X_{x}=\frac{2\I}{2-\epsilon}\left(\ket{\psi_{0}}\bra{\psi_{2}}-\ket{\psi_{2}}\bra{\psi_{0}}\right) \;,\\
X_{y}=\frac{-2}{2-\epsilon}\left(\ket{\psi_{0}}\bra{\psi_{3}}+\ket{\psi_{3}}\bra{\psi_{0}}\right) \;.
\end{align}
Again owing to the symmetry in the channel the total variance is minimised when $v_{x}^{*}=v_{y}^{*}=\frac{1}{2}\mathcal{N}^\text{pd}_{2\text q,2}$. Thus, for estimating these two parameters individual measurements are not as powerful as collective measurements. For estimating a rotation about all three axes simultaneously this probe is able to achieve a Holevo bound of
\begin{equation}
\label{holpd32q}
v_x+v_y+v_z \geq \mathcal{H}^\text{pd}_{2\text q,3}=2+\frac{1}{(1-\epsilon)^{2}}\;.
\end{equation}
The $X_{z}$ matrix required to saturate this Holevo bound is
\begin{equation}
X_{z}=\frac{\I}{(1-\epsilon)}\left(\ket{\psi_{0}}\bra{\psi_{1}}-\ket{\psi_{1}}\bra{\psi_{0}}\right)\;.
\label{pd2q3}
\end{equation}
The same $X_z$ matrix optimises the NHB for estimating all three rotations simultaneously and we have
\begin{equation}
v_x+v_y+v_z \geq \mathcal{N}^\text{pd}_{2\text q,3}=\frac{4}{2-\epsilon}+\frac{1}{(1-\epsilon)^{2}}\;.
\label{nagpd3}
\end{equation}

In appendix \ref{phasedampapen} we construct a measurement scheme which saturates this bound. This channel showcases many of the benefits of using quantum resources in parameter estimation as summarised in Fig.~\ref{fig:PDexample}.\\
% Figure environment removed
\subsubsection{Decoherence of both qubits}
As with previous channels, if we consider the channel where both qubits are exposed to the phase damping channel individually, the performance can only worsen compared to the original channel. However, as mentioned above when using the probe $\frac{1}{\sqrt{2}}\left(\ket{0,1}+\ket{1,0}\right)$ to estimate $\theta_{x}$ or $\theta_{y}$ we can achieve a Holevo variance of 1 regardless of the noise in the channel. This is still true when we consider decohering both qubits. Nevertheless, the effects of exposing the second qubit to the phase damping channel can be examined by considering a rotation about the $z$-axis. This is shown in Fig.~\ref{PhaseDamp2chvv} where we plot the Holevo bound for the $\frac{1}{\sqrt{2}}\left(\ket{0,1}+\ket{1,0}\right)$ probe with both qubits exposed to separate phase damping channels. Using this probe to estimate a rotation about the $z$-axis the Holevo bound is given by
\begin{equation}
v_z \geq \mathcal{H}^\text{2pd}_{2\text q,1}=\frac{1}{(1-\epsilon_{1})^{2}(1-\epsilon_{2})^{2}}\;.
\label{pddualdecoh}
\end{equation}
When $\epsilon_{2}=0$ we recover the case where the second qubit is exposed to the identity channel, but for all other values of $\epsilon_{2}$ the performance of this probe worsens. For this particular example, when the auxillarly qubit experiences no decoherence the single qubit and two qubit probes perform equally well. Thus, when the secondary qubit experiences any decoherence the two qubit probe performs worse than the single qubit probe. In this situation, which is more realistic, using an entangled two qubit probe can bring a disadvantage as the additional ancillary qubit introduces extra noise into the system which lowers the achievable precision.  


\subsubsection{Collective measurements on many copies of the state}
Finally, we consider what happens when we can perform measurements on more than one copy of the single qubit probe. Owing to the singularity when the single qubit probe, $\ket{\psi}=\text{a}\ket{0}+\sqrt{1-\text{a}^2}\ket{1}$, has $\text{a}=1$, it is difficult to find analytic results for many copies of this probe. However, it is easy to numerically verify that with an increasing number of copies of the state the Nagaoka bound tends to the Holevo bound as expected. This is plotted in Fig.\ref{PDnagtendhol}. 



% Figure environment removed
\newpage
%
%\section{Discussions}
%\label{generalisation}
%In this paper we have described several results that are true for all
%of the individual channels considered, namely that for an entangled
%two qubit probe estimating a second parameter does not affect the
%achievable precision, for a single qubit probe estimating a second
%parameter will lower the achievable precision and for a single qubit
%probe it is impossible to estimate three parameters simultaneously, In
%this section we show that these three results are true for a broad
%class of channels.
%
%\subsection{Entangled probe can estimate two and three parameters as well as
%  estimating just a single parameter. }
%  We will begin by showing that estimating a second parameter using an entangled
%two qubit probe does not affect the achievable precision. Our
%initial probe $\ket{\psi_{0}}\bra{\psi_{0}}$, as described below, is subject to a channel described by some Krauss
%operators as in Eq.~(\ref{krauss}). 
%   \begin{equation}
%\ket{\psi_{0}}=
%\begin{bmatrix}
%a \\
% b \\
%  c\\
%  d\\
% \end{bmatrix}\;,\label{eq:psi0}
% \end{equation}
% For simplicity we will assume
%that the channel only affects the first qubit in the probe. To further
%simplify proceedings we will only consider channels whose Krauss
%operators can be written in terms of the Pauli operators. Although
%this may seem restrictive it covers most of the channels considered in
%this work and many channels can be described in this way. Under these
%assumptions the effect of the channel is to transform our probe into
%\begin{equation}
%\epsilon(\rho)=\sum_{a=0,3}C_{a}\ket{\psi_{a}}\bra{\psi_{a}}\;,
%\end{equation}
%where $C_{a}$ are different constants which depend on the particular channel being considered and the $\psi_{a}$ are the matrices, defined as
%\begin{equation}
% \label{psii}
%\ket{\psi_{1}}=
%\begin{bmatrix}
%a \\
% b \\
% -c\\
%  -d\\
% \end{bmatrix}\
%\ket{\psi_{2}}=
%\begin{bmatrix}
%c\\
% d\\
%a\\
%b\\
% \end{bmatrix}\
%\ket{\psi_{3} }=
%\begin{bmatrix}
%c\\
%d\\
%-a\\
%-b\\
% \end{bmatrix}\
%\end{equation}
% We note that these matrices form an orthonormal basis when the probe $\ket{\psi_{0}}$ is symmetric, as is the case for nearly all the probes considered in this work.
%We can now consider the effect of a small rotation around the $x$-axis of this probe after is has been influenced by the channel.
%\begin{equation}
%\dot{\rho}_{\theta_{x}}=\frac{i(C_{0}-C_{2})}{2}[\ket{\psi_{2}}\bra{\psi_{0}}-\ket{\psi_{0}}\bra{\psi_{2}}]  +\frac{i(C_{1}-C_{3})}{2}[\ket{\psi_{1}}\bra{\psi_{3}}-\ket{\psi_{3}}\bra{\psi_{1}}]\;.
%\end{equation}
%And similarly for a rotation about the y-axis
%\begin{equation}
%\dot{\rho}_{\theta_{y}}=\frac{(C_{3}-C_{0})}{2}[\ket{\psi_{3}}\bra{\psi_{0}}+\ket{\psi_{0}}\bra{\psi_{3}}] +\frac{(C_{1}-C_{2})}{2}[\ket{\psi_{1}}\bra{\psi_{2}}+\ket{\psi_{2}}\bra{\psi_{1}}]\;.
%\end{equation}
%We cannot write down an explicit expression for the $X_{x}$ and $X_{y}$ matrices that provide the optimal measurement because the normalisation changes depending on the specific channel. The general form of the $X_{x}$ matrix is
%\begin{equation}
%X_{x}=iA[\ket{\psi_{2}}\bra{\psi_{0}}-\ket{\psi_{0}}\bra{\psi_{2}}]  +iB[\ket{\psi_{1}}\bra{\psi_{3}}-\ket{\psi_{3}}\bra{\psi_{1}}]\;.
%\end{equation}
%With the normalisation given by
%\begin{equation}
%1=A(C_{0}-C_{2})+B(C_{1}-C_{3})\;.
%\end{equation}
%However,we can still proceed to calculate the Holevo bound we can achieve through the SLD bound. It is easy to calculate from $\dot{\rho}_{\theta_{x}}$ that the Holevo bound (1/SLD Fisher information) is given by
%\begin{equation}
%v_x \geq \mathcal{H}^\text{arb}_{2\text q,1}=\frac{1}{\frac{(C_{0}-C_{2})^{2}}{C_{0}+C_{2}}+\frac{(C_{1}-C_{3})^{2}}{C_{1}+C_{3}}}\;,
%\end{equation}
%which we can see agrees with the variances calculated explicitly in the text. Similarly if we wish to estimate $\theta_{y}$ only, we can show the achievable variance is given by 
%\begin{equation}
%v_y \geq \mathcal{H}^\text{arb}_{2\text q,1}=\frac{1}{\frac{(C_{0}-C_{3})^{2}}{C_{0}+C_{3}}+\frac{(C_{2}-C_{1})^{2}}{C_{2}+C_{1}}}\;,
%\end{equation}
%also in agreement with the explicit calculations. It is also easy to
%see from these SLD calculations that the second parameter we have
%estimated comes at no cost. When we write $\dot{\rho}_{\theta_{x}}$ in
%the basis described by our orthonormal vectors we see that it only has
%non-zero terms at coordinates $(0, 2)$, $(2, 0)$, $(1, 3)$ and
%$(3, 1)$. $\dot{\rho}_{\theta_{y}}$ in the same basis only has
%non-zero terms at coordinates $(0, 3)$, $(3, 0)$, $(1, 2)$ and $(2, 1)$. Thus,
%when we calculate
%$J_\text{SLD}=\tr[\dot{\rho}_{\theta_{x(y)}}\mathcal{J}^{-1}_{\rho}(\dot{\rho}_{\theta_{y(x)}})]$
%we see this must always be zero. Thus, estimating this additional
%parameter does not come at any cost. It is trivial to show that this
%result extends to estimating a third parameter. After a small rotation
%around the z-axis the probe is transformed to
%\begin{equation}
%\dot{\rho}_{\theta_{z}}=\frac{i(C_{1}-C_{0})}{2}[\ket{\psi_{1}}\bra{\psi_{0}}-\ket{\psi_{0}}\bra{\psi_{1}}]  +\frac{i(C_{2}-C_{3})}{2}[\ket{\psi_{2}}\bra{\psi_{3}}-\ket{\psi_{3}}\bra{\psi_{2}}]\;.
%\end{equation}
%From this, following similar arguments we can easily see that estimating this parameter does not hinder our original estimation accuracies. 
%
%\subsection{With single qubit probe, estimating a second parameter
%  causes the precision of the first parameter to degrade.}
%However,the above analysis does not hold when we consider a single
%qubit probe. This is because the single qubit probe exists in a two
%dimensional vector space and therefore only needs two vectors to span
%its subspace. However,we have been using four orthonormal vectors
%above to describe our probes. Thus, these four vectors form an
%overcomplete basis when we only consider a singe probe. This means
%that when we calculate
%$J_\text{SLD}=\tr[\dot{\rho}_{\theta_{x(y)}}\mathcal{J}^{-1}_{\rho}(\dot{\rho}_{\theta_{y(x)}})]$
%for a single qubit we do not obtain zero because coordinates which
%previously were separate from each other now overlap. Thus, using a
%single qubit estimating a second parameter lowers the achievable
%precision. Although this is not a complete proof for the action of any
%channel, it does describe many quantum channels.
%
% In this paper we have found that  when we go from estimating a single parameter to estimating two parameters, using a single qubit probe, the Holevo bound doubles when there is no noise present. We now show this to be true for the probe $\ket{\psi}=a\ket{0}+\sqrt{1-a^{2}}\ket{1}$, assuming $a$ is real. Considering small rotations about the $x$ and $y$-axis' we find
%\begin{align}
%\dot{\rho}_{\theta_{x}}&=\frac{i(2a^{2}-1)}{2}
%\begin{bmatrix}
%0&1\\
%-1&0\\
%\end{bmatrix}\;,\\
%\dot{\rho}_{\theta_{y}}&=
%\begin{bmatrix}
%-a\sqrt{1-a^{2}}&\frac{2a^{2}-1}{2}\\
%\frac{2a^{2}-1}{2}&a\sqrt{1-a^{2}}\\
%\end{bmatrix}\;.
%\end{align}
%From this we can find the optimal unbiased X matrices which saturate the SLD bound for estimating a single parameter
%\begin{align}
%X_{x}&=\frac{i}{(2a^{2}-1)}
%\begin{bmatrix}
%0&1\\
%-1&0\\
%\end{bmatrix}\;,\\
%X_{y}&=
%\begin{bmatrix}
%\frac{-2\sqrt{1-a^{2}}}{a}&1\\
%1&0\\
%\end{bmatrix}\;.
%\end{align}
%From this the Holevo bound for estimating a rotation about the
%$x$-axis only is given by $\frac{0.25}{(0.5-a^{2})^{2}}$. For
%estimating a rotation about the $y$-axis only the Holevo bound is
%always 1. We see that for estimating a rotation about the $x$-axis the
%Holevo bound is minimised when $a=0, 1$. At best then for
%estimating a single parameter we can achieve a Holevo bound of 1. However
%if we use the same $X$ matrices to estimate both parameters
%simultaneously we see the Holevo bound increases to
%\begin{equation}
%\frac{1}{2}[v_x+v_y] \geq \frac{1}{2}\mathcal{H}^\text{arb}_{2\text q,1}=0.5\left(1+\frac{0.25}{(0.5-a^{2})^{2}}+\abs{\frac{1}{0.5-a^{2}}}\right)\;.
%\end{equation}
%Unsurprisingly this is also minimised when $a=0$ or $1$. We see that the minimum Holevo bound has now doubled to 2. However,we note that this is not in general true under the influence of a noisy channel as illustrated by some of the examples in the text.
%
%\subsection{Single qubit probes are hopeless for estimating three
%rotation parameters simultaneously.}
%Finally we wish to show that it is impossible to construct an unbiased estimator capable of simultaneously estimating three rotations using a single qubit probe.
%We can describe an arbitrary single qubit state after the action of some channel as
%\begin{equation}
% \rho=\mathcal(\ket{\psi_{0}}\bra{\psi_{0}})=
%\begin{bmatrix}
% a&b\\
% c&d\\
% \end{bmatrix}\;,
% \end{equation}
% where $a$, $b$, $c$ and $d$ are numbers. After a small rotation about the three axis' we find:
% \begin{align}
%\dot{ \rho}_{\theta_{x}}&=\frac{i}{2}
%\begin{bmatrix}
% c-b&d-a\\
% a-d&b-c\\
% \end{bmatrix}\;,\\
%\dot{ \rho}_{\theta_{x}}&=\frac{1}{2}
%\begin{bmatrix}
% -b-c&a-d\\
% a-d&b+c\\
% \end{bmatrix}\;,\\
%\dot{ \rho}_{\theta_{z}}&=
%\begin{bmatrix}
% 0&ib\\
% -ic&0\\
% \end{bmatrix}\;.
% \end{align}
% We now wish to define an matrix $X_{x}$ capable of estimating the rotation about the $x$-axis. This matrix must satisfy the conditions (\ref{eq_xcon1}) and (\ref{eq_xcon2}) to be unbiased. We define our estimation matrix as
%  \begin{equation}
%X_{x}=
%\begin{bmatrix}
% e&f\\
% g&h\\
% \end{bmatrix}\;.
% \end{equation}
% We now have four equations from the conditions and four unknowns,
% $e$, $f$, $g$ and $h$ to solve for. If we fix $e$, $f$ and $g$ using
% conditions $\tr[\rho X_{x}]=0$ and $\tr[\dot{\rho}_{\theta_{y,z}}X_{x}]=0$
% then we find that $\tr[\dot{\rho}_{\theta_{x}}X_{x}]=0$
% identically. Thus, it is impossible to construct an estimator, $X_{x}$
% which satisfies the unbiased conditions for estimating simultaneous rotations about all three axis of the Bloch sphere. As we have considered the
% most general single qubit probe we can conclude that a single qubit
% probe can never be an unbiased estimator for a rotation in all three
% directions. One possible explanation for this is that when we try to
% estimate three independent parameters with a single qubit probe our
% estimates will always be correlated in some way.

\section{Conclusion}
This case study has examined the performance of single and two qubit probes for multiparameter estimation under the action of several quantum noise channels. The two qubit probes considered offer an increased robustness to environmental noise compared to single qubit probes. It is shown that entanglement is a necessary resource to reach the ultimate limits in quantum metrology, required in both the state preparation and state measurement stages. Having the option of employing entanglement at both of these stages always results in a better or equivalent measurement precision than using entanglement at only one of these stages. However, there is no general hierarchy between using entanglement in the state preparation stage only and using entanglement in the measurement stage only. Indeed, different scenarios have been presented where each of these settings outperforms the other. We have also found that while certain channels allow two qubit probes to outperform single qubit probes, these situations are somewhat unrealistic. In particular, under realistic channels, for a two qubit probe to obtain the maximum advantage over single qubit probes we require that the second qubit is stored in a perfect quantum memory. If the quantum memory we store the second qubit in adds too much noise, while the first qubit passes through the channel, then the single qubit probe can outperform certain two qubit probes. Thus, we have shown that although entanglement in principle improves parameter estimation in qubit states, it also offers potential drawbacks under certain conditions. For each channel we considered the attainability of the Holevo bound with a finite number of copies of the probe state. It was found that a collective measurement on a limited number of copies of the probe state was sufficient to attain a precision close to the Holevo bound. This suggests that the ultimate limits in quantum metrology may be approximately attainable in the near future. 
%It should be noted however, that in any scenario where the Holevo bound does not equal the Nagaoka bound, it is known that no physical measurement exists which saturates the Holevo bound~\cite{conlon2022gap}.

There are several possible ways to extend the results in this paper. We have focused on estimating small fixed rotations, but in a more realistic scenario, the quantum channel may be dynamic, meaning that the rotations we wish to sense are continually changing. In this situation, quantum resources can still offer an advantage~\cite{yonezawa2012quantum}. However, the assumption made in this paper that the angles are well known, i.e. we are performing local estimation, may no longer hold. Our results could be extended to this case through Bayesian estimation~\cite{suzuki2023bayesian,rubio2020bayesian}. Finally, it would be interesting to investigate whether these results can be applied to quantum key distribution, where it is important to estimate the parameters of noisy channels~\cite{thearle2016estimation,wang2017measurement,wang2022twin}.


\section*{Acknowledgements}

This research was funded by the Australian Research Council Centre of Excellence CE170100012, Laureate Fellowship FL150100019 and the Australian Government Research Training Program Scholarship.

%{\color{red}%
%
%
%Bayesian Thermometry :
%
%1) Global Quantum Thermometry
%
%2) Fundamental Limits in Bayesian Thermometry and Attainability via Adaptive Strategies
%}

%Finally we have generalised some of the results obtained. We have shown that it is impossible to construct an unbiased estimator for estimating a rotation about all three axis of the Bloch sphere simultaneously using a single qubit probe. We have also shown that for a broad class of channels, namely those whose Krauss operators can be written in terms of Pauli matrices, a two qubit probe will always be able to estimate a second and third parameter with no loss of precision in the first estimation. Finally we have shown that for a single qubit probe estimating a second parameter causes the achievable precision to degrade by a factor 2. 

\section{Appendix}
\appendix
%\section{Symmetry}
%\label{apena}
%Some of the channels discussed are symmetric about all three axis of the Bloch sphere. In these situations, 
%due to the symmetry of the problem, the optimal probe and measurement must have
%$v_x=v_y$ or $v_x=v_y=v_z$. Let's consider two parameters and assume
%there exists an optimal scheme with $v_x  \neq v_y$. We will show that this leads
%to a contradiction by constructing a scheme which gives a lower
%combined variance $\tilde{v}_x+\tilde{v}_{y} < v_{x}+v_{y}$. The scheme is as follows. If scheme A provides a variance of $\text{Var}\hat\theta_{xA}=v_{1}$ and
%$\text{Var}\hat \theta_{yA}=v_2$, then we can construct scheme B with  $\text{Var}\hat\theta_{xB}=v_{2}$ and
%$\text{Var}\hat \theta_{yB}=v_1$. Measuring the qubits with scheme A half of the
%time and scheme B half of the time will give us two estimates for
%$\theta_x$ and two estimates for $\theta_y$. Consider the estimator
%\begin{align}
%  \label{eq:15}
%\hat \theta_x&=\frac{v_2}{v_1+v_2}\hat\theta_{xA} +
%\frac{v_1}{v_1+v_2}\hat \theta_{xB}\;,\\
%\hat \theta_y&=\frac{v_1}{v_1+v_2}\hat\theta_{yA} +
%\frac{v_2}{v_1+v_2}\hat \theta_{yB}\;.
%\end{align}
%This scheme will have a combined variance of
%\begin{equation}
%  \label{eq:17}
%\text{Var}\hat\theta_{x}  + \text{Var}{\hat\theta_{y}} = \frac{4 v_{1} v_{2}}{v_{1}+v_{2}}
%  < v_{1}+v_{2}\;,
%\end{equation}
%when $v_{1} \neq v_{2}$. Therefore the optimal scheme must have $v_{1}=v_{2}$.
%
%

\section{Computing the SLD bound for single parameter estimation in the decoherence channel}
\label{apennewSLD}
In Fig.~\ref{BS_DC} (i) we present results for single parameter estimation using single qubit probes, obtained through the SLD bound. In this appendix we provide all the necessary information to compute the SLD bound. We consider an input probe state of the form $\ket{\psi}=\text{cos}(\theta/2)\ket{0}+e^{\I\phi}\text{sin}(\theta/2)\ket{1}$. Assuming $\theta_x$ is small, the derivative of the probe state with respect to $\theta_x$, after passing through the decoherence channel is given by
\begin{equation}
\frac{\partial\rho}{\partial\theta_x}=\frac{1-\epsilon}{2}\begin{pmatrix}
\text{sin}(\theta)\text{sin}(\phi)&\mathrm{i}\text{cos}(\theta)\\
-\mathrm{i}\text{cos}(\theta)&\text{sin}(\theta)\text{sin}(\phi)
\end{pmatrix}\;.
\end{equation}
The unnormalised eigenvectors of the probe state after the channel are given by
\begin{equation}
\ket{e_1}=\begin{pmatrix}
\text{e}^{-\mathrm{i}\phi}\left(\frac{1}{\text{tan}(\theta)}-\frac{1}{\text{sin}(\theta)}\right)\\
1
\end{pmatrix}\qquad\text{and}\qquad\ket{e_2}=\begin{pmatrix}
\text{e}^{-\mathrm{i}\phi}\left(\frac{1}{\text{tan}(\theta)}+\frac{1}{\text{sin}(\theta)}\right)\\
1
\end{pmatrix}\;,
\end{equation}
with corresponding eigenvalues of $\lambda_1=\epsilon/2$ and $\lambda_2=1-\epsilon/2$. This allows the SLD operators to be computed from Eq.~\eqref{eq:sld:comp} as
\begin{equation}
\mathcal{L}_x=(1-\epsilon)\begin{pmatrix}
\text{sin}(\theta)\text{sin}(\phi)&\mathrm{i}\text{cos}(\theta)\\
-\mathrm{i}\text{cos}(\theta)&-\text{sin}(\theta)\text{sin}(\phi)
\end{pmatrix}\;.
\end{equation} 
Using this operator and Eq.~\eqref{eq:SLDbound}, the SLD bound can be computed as 
\begin{equation}
\label{eq:SLDappendix}
v_x\geq C^\text{SLD}=\frac{1}{(1-\epsilon)^2(\text{cos}(\theta)^2+\text{sin}(\theta)^2\text{sin}(\phi)^2)}\;.
\end{equation}
It is clear that the variance is minimised when $\phi=\pi/2,3\pi/2$ or when $\theta=0,\pi$. For these values of $\theta$ and $\phi$ it is easily verified that the SLD bound coincides with the Holevo bound, Eq.~\eqref{Hol1}. For the amplitude damping and phase damping channels the SLD bound can be computed in a similar manner.

\section{Explicit construction of $X_{x}$ matrix for single parameter estimation in the decoherence channel}
\label{apenxmat}
Here we demonstrate how the optimal $X_{x}$ matrix for estimating a single parameter using a single qubit probe subject to the decoherence channel is constructed. For this we consider the real single qubit probe, $\ket{\psi}=\ket{0}$. After subjecting this probe to a small rotation about the $x$-axis and some decoherence in the channel we can write
\begin{gather}
\frac{\partial \rho}{\partial \theta_x}=\frac{\mathrm{i}}{2}(1-\epsilon)\left(\ket{0}\bra{1}-\ket{1}\bra{0}\right)\;,\\
\rho=\left(1-\frac{\epsilon}{2}\right)\ket{0}\bra{0}+\frac{\epsilon}{2}\ket{1}\bra{1}\;,
\end{gather}
where $\rho$ is the probe after it has passed through the channel and we have assumed the rotation $\theta_{x}$ is small. We now construct a matrix $X_{x}$
\begin{equation}
X_{x}=A\ket{0}\bra{0}+B\ket{1}\bra{1}+C\left(\ket{0}\bra{1}-\ket{1}\bra{0}\right)\;,
\end{equation}
which must be Hermitian and satisfy the conditions given by Eqs.~\eqref{eq_xcon1} and \eqref{eq_xcon2}. As $\frac{\partial \rho}{\partial \theta_x}$ is purely imaginary this constrains the form of $X_{x}$. From Eqs.~\eqref{eq_xcon1} and~\eqref{eq_xcon2} we get the following conditions
\begin{equation}
C=\frac{\mathrm{i}}{(1-\epsilon)} 
\qquad\text{and}\qquad
A=\frac{-B\epsilon}{2-\epsilon}\;.
\end{equation}
We can then evaluate the Holevo bound using Eq.~\eqref{eq_hol2}.
\begin{equation}
\text{Tr}\{\rho.X_{x}.X_{x}\}=\frac{1}{(1-\epsilon)^2}+\frac{B^2\epsilon}{2-\epsilon}\;.
\end{equation}
This is minimised by taking $A=B=0$, which leaves
\begin{equation}
\text{Tr}\{\rho.X_{x}.X_{x}\}=\frac{1}{(1-\epsilon)^{2}}\;.
\end{equation}
This gives the Holevo bound in Eq.~\eqref{Hol1} and the $X_x$ matrix given by Eq.~\eqref{xone}.

\section{Measurement scheme attaining the Nagaoka bound for single qubit probes subject to the decoherence channel}
\label{apenDC1q}
In this appendix we show a measurement scheme which saturates the Nagaoka bound for single qubit probes estimating one and two parameters in the decoherence channel. From the eigenvectors of the $X_{x}$ and $X_{y}$ matrices for this probe we can construct the optimal POVM for saturating the Nagaoka bound. The POVMs necessary are
\begin{align}
\label{POVM1qDC}
\Pi_{1}&=\frac{1}{4}\left(-\I\ket{0}+\ket{1}\right)\left(\I\bra{0}+\bra{1}\right)\;,\\
\Pi_{2}&=\frac{1}{4}\left(\I\ket{0}+\ket{1}\right)\left(-\I\bra{0}+\bra{1}\right)\;,\\
\Pi_{3}&=\frac{1}{4}\left(\ket{0}+\ket{1}\right)\left(\bra{0}+\bra{1}\right)\;,\\
\Pi_{4}&=\frac{1}{4}\left(-\ket{0}+\ket{1}\right)\left(-\bra{0}+\bra{1}\right)\;.
\label{POVM1qDCend}
\end{align}
The probability of each of the four outcomes is 1/4. We can construct unbiased estimators from these POVMs, using the estimator coefficients $\xi_{x,2}=\xi_{y,3}=-\xi_{x,1}=-\xi_{y,4}=\frac{2}{1-\epsilon}$ and $\xi_{i,j}=0$ for all other $i$ and $j$. This measurement strategy gives a variance of 
\begin{align}
v_x=\frac{1}{4}\left(\xi_{x,1}^2+\xi_{x,2}^2\right) \;,\\
v_y=\frac{1}{4}\left(\xi_{y,3}^2+\xi_{y,4}^2\right) \;,
\end{align}
which gives a total variance of $v_x+v_y=\frac{4}{(1-\epsilon)^2}$ which coincides with Eq.~\eqref{Nag12}. We can also see that by ignoring the POVM's corresponding to estimating $\theta_y$, $\Pi_3$ and $\Pi_4$, the remaining POVMs multiplied by 2 attain the Nagaoka bound for estimating a single parameter from Eq~\eqref{NagD1}.

\section{Measurement saturating the Nagaoka bound for estimating two parameters with a two qubit probe subject to the decoherence channel}
\label{app2q2cDC}
With a two qubit probe, we claim that individual measurements can achieve a variance of $\frac{4-\epsilon}{2(1-\epsilon)^{2}}$, Eq.\eqref{Nag22DC}. We define the four sub-normalised projectors
\begin{align}
  \begin{drcases}\ket{\phi_1}\\
    \ket{\phi_2}\end{drcases}=\frac{1}{2}
  \begin{pmatrix}1\\\pm a\I\\\pm a\I\\1\end{pmatrix}\qquad \text{and}\qquad
  \begin{drcases}\ket{\phi_3}\\
    \ket{\phi_4}\end{drcases}=\frac{1}{2}
  \begin{pmatrix}1\\\mp b\\\mp b\\-1\end{pmatrix}\;,\;
\end{align}
where $a$ and $b$ are two non-zero real parameters satisfying
$a^{2}+b^{2}\leq 1$. An optimal strategy that saturates the Nagaoka bound for estimating $\theta_x$ and $\theta_y$ consists of measuring
the five-outcome POVM with $\Pi_j = \ket{\phi_j}\bra{\phi_j}$ for $j=1,2,3,4$
and $\Pi_{5}=1-(\Pi_1 + \Pi_2 + \Pi_3 +\Pi_4)$.  The probability for
each POVM outcome is
\begin{equation}
\begin{aligned}
  \begin{drcases}p_1\\
    p_2\end{drcases}&=a^2\left(\frac{1}{2}-\frac{3\epsilon}{8}\right)+\frac{\epsilon}{8}\;,\\
  \begin{drcases}p_3\\
    p_4\end{drcases}&=b^2\left(\frac{1}{2}-\frac{3\epsilon}{8}\right)+\frac{\epsilon}{8}\;,\\
  p_5&=1-0.5\epsilon-\left(a^2+b^2\right)\left(1-\frac{3\epsilon}{4}\right)\;.
\end{aligned}
\end{equation}
From this POVM, we can construct unbiased estimators for $\theta_{x}$ and $\theta_{y}$ with
\begin{equation}
  \begin{gathered}
\xi_{x,1} = -\xi_{x,2} =\frac{1}{(1-\epsilon)a} \;,\qquad\xi_{x,3}=\xi_{x,4}=\xi_{x,5}=0\;,\\
\xi_{y,3}=-\xi_{y,4}=\frac{1}{(1-\epsilon)b}\;,\qquad \xi_{y,1}=\xi_{y,2}=\xi_{y,5}=0\;.
\end{gathered}
\end{equation}
In the asymptotic limit, the variances in our estimate of $\theta_{x}$ and
$\theta_{y}$ are
\begin{equation}
  \begin{aligned}
    v_{x}&=\xi_{x,1}^2 \,p_1 + \xi_{x,2}^2\, p_2
    =\frac{a^2\left(1-\frac{3\epsilon}{4}\right)+\frac{\epsilon}{4}}{a^2(1-\epsilon)^2}\;,\\
    v_{y}&=\xi_{y,3}^2 \,p_3 + \xi_{y,4}^2\,
    p_4=\frac{b^2\left(1-\frac{3\epsilon}{4}\right)+\frac{\epsilon}{4}}{b^2(1-\epsilon)^2} \;,
  \end{aligned}
\end{equation}
which are optimised when $a=b=\frac{1}{\sqrt{2}}$. The sum $v_x+v_y=\frac{4-\epsilon}{2(1-\epsilon)^2}$
saturates the Nagaoka bound as claimed. We can also use the same POVM's to saturate Eq.~\eqref{Hol2qDC}, the Holevo (and Nagaoka) bound for estimating a single parameter in the decoherence channel. For this we require $a=1, b=0$ and we see $v_x=\frac{2-\epsilon}{2(1-\epsilon)^2}$. The cost of this is however, that we learn nothing about $v_y$. This measurement strategy gives rise to a tradeoff curve, between how much we can learn about $\theta_x$ and how much we can learn about $\theta_y$ as shown in Fig.\ref{DCH2N}. 

% Figure environment removed

\section{Measurement scheme required to saturate the NHB for estimating three parameters with the optimal two qubit probe in the decoherence channel}
\label{decohapen3}
We now show that the NHB can also be saturated when we estimate all three parameters simultaneously using a two qubit probe. However, we now require a new set of POVMs:

\begin{align}
  \begin{drcases}\ket{\psi_{x,1}}\\
    \ket{\psi_{x,2}}\end{drcases}=
  \frac{1}{2}\begin{pmatrix}1\\\pm \frac{\I}{\sqrt{3}}\\\pm \frac{\I}{\sqrt{3}} \\1\end{pmatrix}\qquad \text{and}\qquad
  \begin{drcases}\ket{\psi_{y,1}}\\
    \ket{\psi_{y,2}}\end{drcases}=
 \frac{1}{2} \begin{pmatrix}1\\\pm \frac{1}{\sqrt{3}}\\\pm \frac{1}{\sqrt{3}} \\-1\end{pmatrix}\;.
\end{align}
The POVM's which we will use to estimate $\theta_x$ and $\theta_y$ are then given by 
\begin{align}
\Pi_{x,j}&=\ket{\psi_{x,j}}\bra{\psi_{x,j}}\;,\\
\Pi_{y,j}&=\ket{\psi_{y,j}}\bra{\psi_{y,j}}\;.
\end{align}
This gives the first four POVMs. The POVMs required to estimate $\theta_z$ are given by 
\begin{equation}
\begin{drcases}\Pi_{z,1}\\
    \Pi_{z,2}\end{drcases}=\begin{pmatrix}
0&0&0&0\\
0&\frac{1}{3}&-\frac{1}{6}\mp\frac{\I}{\sqrt{12}}&0\\
0&-\frac{1}{6}\pm\frac{\I}{\sqrt{12}}&\frac{1}{3}&0\\
0&0&0&0\\
\end{pmatrix}\;.
\end{equation}
These POVM's already sum to the identity and so no extra POVMs are required. We see that from these we can construct unbiased estimators for $\theta_x$, $\theta_y$ and $\theta_z$ as
\begin{align}
\hat{\theta}_x&=\frac{\sqrt{3}}{1-\epsilon}\left(\Pi_{x,1}-\Pi_{x,2}\right)\;,\\
\hat{\theta}_y&=\frac{\sqrt{3}}{1-\epsilon}\left(\Pi_{y,2}-\Pi_{y,1}\right)\;,\\
\hat{\theta}_z&=\frac{\sqrt{3}}{1-\epsilon}\left(\Pi_{z,1}-\Pi_{z,2}\right)\;,\\
\end{align}
which give variances of
\begin{align}
v_x=\left(\frac{\sqrt{3}}{1-\epsilon}\right)^2\left(\text{Tr}\{\rho\Pi_{x,1}\}+\text{Tr}\{\rho\Pi_{x,2}\}\right)\;,\\
v_y=\left(\frac{\sqrt{3}}{1-\epsilon}\right)^2\left(\text{Tr}\{\rho\Pi_{y,1}\}+\text{Tr}\{\rho\Pi_{y,2}\}\right)\;,\\
v_z=\left(\frac{\sqrt{3}}{1-\epsilon}\right)^2\left(\text{Tr}\{\rho\Pi_{z,1}\}+\text{Tr}\{\rho\Pi_{z,2}\}\right)\;.\\
\end{align}
This gives a total variance of 
\begin{equation}
v_{\text{tot}}=v_x+v_y+v_z=\frac{3}{(1-\epsilon)^2}\;,
\end{equation}
which coincides with the variance from the NHB given in Eq.~\eqref{Ngen3DC}. 

\section{Measurement saturating the Nagaoka bound for estimating two parameters with two copies of a single qubit probe subject to the decoherence channel}
\label{app:1q2cDC}
Using single qubit probes in the decoherence channel the Holevo bound is given by Eq.~\eqref{Hol12}, $\mathcal{H}^\text{d}_{1\text q,2}=\frac{4-2\epsilon}{(1-\epsilon)^{2}}$, and the Nagaoka bound is given by Eq.~\eqref{Nag12}, $\mathcal{N}^\text{d}_{1\text q,2}=\frac{4}{(1-\epsilon)^{2}}$. We also claim that with two copies of the single qubit probe the Nagaoka bound becomes $\mathcal{N}^\text{d}_{1\text q^{\otimes 2},2}=(2-\epsilon+\frac{\epsilon^2}{2})/(1-\epsilon)^2$. The measurement strategy which saturates this bound is very similar to that in appendix~\ref{app2q2cDC}. We use the following four sub-normalised projectors
\begin{align}
  \begin{drcases}\ket{\phi_1}\\
    \ket{\phi_2}\end{drcases}=\frac{1}{2}
  \begin{pmatrix}1\\\pm a\I\\\pm a\I\\-1\end{pmatrix}\qquad \text{and}\qquad
  \begin{drcases}\ket{\phi_3}\\
    \ket{\phi_4}\end{drcases}=\frac{1}{2}
  \begin{pmatrix}1\\\mp b\\\mp b\\1\end{pmatrix}\;,\;
\end{align}
where $a$ and $b$ are two non-zero real parameters satisfying
$a^{2}+b^{2}\leq 1$. An optimal strategy that saturates the Nagaoka bound for estimating $\theta_x$ and $\theta_y$ consists of measuring
the five-outcome POVM with $\Pi_j = \ketbra{\phi_j}$ for $j=1,2,3,4$
and $\Pi_{5}=1-(\Pi_1 + \Pi_2 + \Pi_3 +\Pi_4)$.  The probability for
each POVM outcome is
\begin{equation}
\begin{aligned}
p_1=p_2&=\frac{1}{4}\left(1+(a^2-1)\epsilon+\frac{1}{2}(1-a^2)\epsilon^2\right)\;,\\
p_3=p_4&=\frac{1}{4}\left(1+(b^2-1)\epsilon+\frac{1}{2}(1-b^2)\epsilon^2\right)\;,\\
  p_5&=\frac{1}{4}(2-a^2-b^2)(2-\epsilon)\epsilon\;.
\end{aligned}
\end{equation}
From this POVM, we can construct unbiased estimators for $\theta_{x}$ and $\theta_{y}$ with
\begin{equation}
  \begin{gathered}
\xi_{x,2} = -\xi_{x,1} =\frac{1}{(1-\epsilon)a} \;,\qquad\xi_{x,3}=\xi_{x,4}=\xi_{x,5}=0\;,\\
\xi_{y,3}=-\xi_{y,4}=\frac{1}{(1-\epsilon)b}\;,\qquad \xi_{y,1}=\xi_{y,2}=\xi_{y,5}=0\;.
\end{gathered}
\end{equation}
In the asymptotic limit, the variances in our estimate of $\theta_{x}$ and
$\theta_{y}$ are
\begin{equation}
  \begin{aligned}
    v_{x}&=\xi_{x,1}^2 \,p_1 + \xi_{x,2}^2\, p_2
    =\frac{1+(a^2-1)\epsilon+\frac{1}{2}(1-a^2)\epsilon^2}{2a^2(1-\epsilon)^2}\;,\\
    v_{y}&=\xi_{y,3}^2 \,p_3 + \xi_{y,4}^2\,
    p_4=\frac{1+(b^2-1)\epsilon+\frac{1}{2}(1-b^2)\epsilon^2}{2b^2(1-\epsilon)^2}\;,
  \end{aligned}
\end{equation}
which are optimised when $a=b=\frac{1}{\sqrt{2}}$. The sum $v_x+v_y=(2-\epsilon+\frac{\epsilon^2}{2})/(1-\epsilon)^2$
saturates the Nagaoka bound as claimed.

\section{POVM's attaining the Nagaoka bound for single qubit probes subject to the amplitude damping channel}
\label{apenAD1q}
Owing to the similarities between the optimal $X$ matrices required for the amplitude damping channel and the decoherence channel, the POVMs required to saturate the Nagaoka bound in the amplitude damping channel are the same as those given in Eqs.~\eqref{POVM1qDC} to \eqref{POVM1qDCend}. In order to construct unbiased estimators we require the estimator coefficients $\xi_{x,1}=\xi_{y,3}=-\xi_{x,2}=-\xi_{y,4}=\frac{2}{\sqrt{1-p}}$ and $\xi_{i,j}=0$ for all other $i$ and $j$. This measurement gives $v_x+v_y=\frac{4}{1-p}$, in agreement with Eq.~\eqref{amp1nag}. As before we can consider only the first two POVMs, scaled by a factor of two and we arrive at the Holevo bound (and Nagaoka bound) for estimating a single parameter, $v_x=\frac{1}{1-p}$.

\section{Optimal estimator for estimating two parameters
  in the amplitude damping channel with two copies of the single-qubit probe}
  \label{11est}
We use the same POVMs as described in appendix \ref{app:1q2cDC} to saturate the Nagaoka bound in Eq.~\eqref{nagad1q2c22}. The probability for
each POVM outcome is now
\begin{equation}
\begin{aligned}
p_1=p_2&=\frac{1}{4}\left(1+2(a^2-1)p+2(1-a^2)p^2\right)\;,\\
p_3=p_4&=\frac{1}{4}\left(1+2(b^2-1)p+2(1-b^2)p^2\right)\;,\\
  p_5&=\left(2-a^2-b^2\right)(1-p)p\;.
\end{aligned}
\end{equation}
We require the following estimator coefficients
\begin{equation}
  \begin{gathered}
\xi_{x,1} = -\xi_{x,2} =\frac{1}{a\sqrt{1-p}} \;,\qquad\xi_{x,3}=\xi_{x,4}=\xi_{x,5}=0\;,\\
\xi_{y,3}=-\xi_{y,4}=\frac{1}{b\sqrt{1-p}}\;,\qquad \xi_{y,1}=\xi_{y,2}=\xi_{y,5}=0\;.
\end{gathered}
\end{equation}
In the asymptotic limit, the variances in our estimate of $\theta_{x}$ and
$\theta_{y}$ become
\begin{equation}
  \begin{aligned}
    v_{x}&=\xi_{x,1}^2 \,p_1 + \xi_{x,2}^2\, p_2
    =\frac{0.5+(a^2-1)p+(1-a^2)p^2}{a^2(1-p)^2}\;,\\
    v_{y}&=\xi_{y,3}^2 \,p_3 + \xi_{y,4}^2\,
    p_4=\frac{0.5+(b^2-1)p+(1-b^2)p^2}{b^2(1-p)^2}\;,
  \end{aligned}
\end{equation}
which are optimised when $a=b=\frac{1}{\sqrt{2}}$. The sum $v_x+v_y=\frac{2-2p+2p^2}{1-p}$
saturates the Nagaoka bound in Eq.~\eqref{nagad1q2c22} as claimed.
  
 \section{Explicit construction of the optimal estimator for estimating a single parameter
  in the phase damping channel with a single qubit probe}
   From Fig.~\ref{fig:PD_xy_comp} we know that for estimating a small rotation about the $x$-axis, the probe $\ket{\psi}=\text{cos}(\theta/2)\ket{0}+\I\text{sin}(\theta/2)$ is optimal for all $\theta$ except $\theta=0$ or 1. By constructing the optimal estimator for this channel we show why this is true. After the channel the probe and its derivative become
   \begin{equation}
   \rho=\begin{pmatrix}
   \text{cos}(\frac{\theta}{2})^2&-\frac{\I}{2}(1-\epsilon)\text{sin}(\theta)\\
   \frac{\I}{2}(1-\epsilon)\text{sin}(\theta)&\text{sin}(\frac{\theta}{2})^2
   \end{pmatrix}\qquad\text{and}\qquad
   \frac{\partial\rho}{\partial\theta_x}=\begin{pmatrix}
   \frac{\text{sin}(\theta)}{2}&\frac{\I}{2}(1-\epsilon)\text{cos}(\theta)\\
  -\frac{\I}{2}(1-\epsilon)\text{cos}(\theta)&-\frac{\text{sin}(\theta)}{2}
   \end{pmatrix}\;,
   \end{equation}
  where we have taken the limit $\theta_x\rightarrow0$. For this derivative the optimal $X_x$ matrix is
  \begin{equation}
  X_x=\begin{pmatrix}
  \text{tan}(\frac{\theta}{2})&0\\
  0&-\text{cot}(\frac{\theta}{2})
  \end{pmatrix}\;,
  \end{equation}
  which gives a Holevo bound equal to 1 irrespective of the value of $\epsilon$. However, the optimal matrix $X_x$ contains infinities for either $\ket{0}$ and $\ket{1}$ and so a different sub-optimal solution is found exactly at these points. 
  
 
\section{Explicit construction of the optimal estimator for estimating two parameters
  in the phase damping channel with a single qubit probe}
  \label{nagexp}
Consider the probe $\ket{\psi}=a\ket{0}+b\ket{1}$ where $a=\cos\theta$ and
$b=\sin \theta$ are real numbers. This probe passes through a channel
rotating it by $\theta_x$ around the $x$-axis and $\theta_y$
around the $y$-axis. It then goes through a phase damping channel
parametrised by $\epsilon$. The probe after the channel becomes

\begin{align}
  \label{eq:1}
  \rho(\theta_x,\theta_y) =
  \begin{pmatrix}
    a^2 + a b \theta_y & (1-\epsilon)\left(a b+\frac{a^2-b^2}{2}\left(-\mathrm{i} \theta_x-\theta_y \right)\right)\\
 (1-\epsilon)\left(a b+\frac{a^2-b^2}{2}\left(\mathrm{i} \theta_x-\theta_y \right)\right)&    b^2 - a b \theta_y 
    \end{pmatrix}\;
    \end{align}
where we have assumed that both $\theta_x$ and $\theta_y$ are
small. Suppose we have a total of $n$ copies. We choose to measure
$n_y$ copies in the $\sigma_z$ basis, and the remaining $n_x$ copies
in the $\sigma_y$ basis. When measuring in the $\sigma_z$ basis, the
two outcomes occur with probabilities
\begin{align}
  \label{eq:2}
  p(\pi_1|\theta_x,\theta_y) &= a(a+b\theta_y)\;,\\
  p(\pi_2|\theta_x,\theta_y) &= b(b-a\theta_y)\;,
\end{align}
which is independent of $\theta_x$. When $a=0$ or $a=1$, the outcomes
(up to first order) are also independent of $\theta_y$. When measuring in the $\sigma_y$ basis, the
two outcomes occur with probabilities
\begin{align}
  p(\pi_3|\theta_x,\theta_y) &= \frac{1}{2} + \frac{1-\epsilon}{2}\left(a^2-b^2\right)\theta_x\;,\\
  p(\pi_4|\theta_x,\theta_y) &= \frac{1}{2} - \frac{1-\epsilon}{2}\left(a^2-b^2\right)\theta_x\;.
\end{align}
The outcomes of these measurements are independent of $\theta_y$. From
these four outcomes, we construct the two independent unbiased estimators for
$\theta_x$ and $\theta_y$ as
\begin{align}
  \hat{\theta}_y &= \frac{n_1-n_2}{2 a\, b\, n_y}-\frac{a^2-b^2}{2 a\, b}\;,\\
  \hat{\theta}_x &= \frac{n_3-n_4}{n_x(1-\epsilon)\left(a^2-b^2\right)}\;,
\end{align}
where the different $n_j$ are the number of times we get the $j$-th outcome. Note that
$\hat\theta_y$ is not well defined when $a=1$ or $0$.

The variance of $\hat{\theta}_y$ is given by
\begin{align}
  \label{eq:4}
  \text{var}\left(\hat\theta_y\right) &=
                             \text{var}\left(\frac{n_1-n_2}{2a\,b\,n_y}\right)\\
 % &=
 %   \text{var}\left(\frac{2n_1-n_y}{2a\,b\,n_y}\right)\\
    %                       &=\frac{\text{var}(n_1)}{a^2 b^2 n_y^2}\\
                    %       &=\frac{n_y p_1 p_2}{ a^2 b^2 n_y^2}\\
  &=\frac{1}{n_y}\;.
\end{align}
We have used the fact that $n_1$ follows a binomial distribution and
has variance given by $\text{var}(n_1)=n_y p_1 p_2$ and that $p_1=a^2$
and $p_2=b^2$ when $\theta_y$ is small. Notice that the
variance of $\hat\theta_y$ does not depend on $a$ or $\epsilon$. Next, the variance of $\hat\theta_x$ is
\begin{align}
  \label{eq:5}
  \text{var}\left(\hat\theta_x\right) &= \text{var}\left(
                             \frac{n_3-n_4}{n_x(1-\epsilon)(a^2-b^2)}\right)\\
% &= \text{var}\left(
                             %\frac{2 n_3-n_x}{n_x(1-\epsilon)(a^2-b^2)}\right)\\
                          % &=\frac{4\text{var}(n_3)}{n_x^2(1-\epsilon)^2(a^2-b^2)^2}\\
                       % &=\frac{4n_x p_3 p_4}{n_x^2(1-\epsilon)^2(a^2-b^2)^2}\\
                        &=\frac{1}{n_x(1-\epsilon)^2(a^2-b^2)^2}\;.
\end{align}
We have used that fact that when $\theta_{x}$ is small, $p_{3}=p_{4}=\frac{1}{2}$. This is smallest when $a$ approaches $1$ or $0$ where
\begin{align}
  \label{eq:6}
  \text{var}\left(\hat\theta_x\right) = \frac{1}{n_x(1-\epsilon)^2}\;.
\end{align}
Putting everything together, we get
\begin{align}
  \label{eq:7}
\text{var}\left(\hat\theta_y\right)+\text{var}\left(\hat\theta_x\right)  = \frac{1}{n_y} + \frac{1}{n_x(1-\epsilon)^2}\;.
\end{align}
All that remains is to find $n_y$ and $n_x$ that minimises the above
expression subject to a fixed total number of probes, $n_y+n_x=n$. This optimal
fraction is given by
\begin{align}
  \label{eq:9}
  n_y=\frac{1-\epsilon}{2-\epsilon}n,\\
  n_x=\frac{1}{2-\epsilon}n\;,\\
\end{align}
which gives
\begin{align}
\text{var}\left(\hat\theta_y\right)+\text{var}\left(\hat\theta_x\right)  = \left(\frac{2-\epsilon}{1-\epsilon}\right)^2\frac{1}{n}\;.
\end{align}
This expression coincides with the Nagaoka bound given in Eq.~\eqref{nag12pd}.


\section{Measurement scheme which saturates the NHB for estimating three parameters in the phase damping channel
  with a two-qubit probe}
\label{phasedampapen}
The measurements presented in this appendix have already been presented in Ref.~\cite{conlon2021efficient}, however we include them here for completeness. We first describe the measurement strategy required to saturate the Nagaoka bound for estimating $\theta_x$ and $\theta_y$ simultaneously using a two qubit probe subject to the phase damping channel. We require the same POVMs as used in appendix \ref{app2q2cDC}. The probability for
each POVM outcome is
\begin{equation}
\begin{aligned}
  \begin{drcases}p_1\\
    p_2\end{drcases}&=\frac{1}{4}a(2-\epsilon)(a\pm \theta_x)\;,\\
  \begin{drcases}p_3\\
    p_4\end{drcases}&=\frac{1}{4}b(2-\epsilon)(b\pm \theta_y)\;,\\
  p_5&=1- \frac{1}{2}(2-\epsilon)\left(a^2+b^2\right)\;.
\end{aligned}
\end{equation}
The unbiased estimator coefficients for $\theta_{x}$ and $\theta_{y}$ are given by
\begin{equation}
  \begin{gathered}
\xi_{x,1} = -\xi_{x,2} =\frac{2}{(2-\epsilon)a} \;,\qquad\xi_{x,3}=\xi_{x,4}=\xi_{x,5}=0\;,\\
\xi_{y,3}=-\xi_{y,4}=\frac{2}{(2-\epsilon)b}\;,\qquad \xi_{y,1}=\xi_{y,2}=\xi_{y,5}=0\;.
\end{gathered}
\end{equation}
The fifth outcome $\Pi_5$ is only included to ensure the POVM elements sum to the identity, it does not improve the measurement precision. For a finite sample, to have a better estimate of $\theta_x$ and
$\theta_y$ it is beneficial to have both $a$ and $b$ large so
the outcomes $\Pi_1$ to $\Pi_4$ occur more often. However, in the
asymptotic limit, the variances in our estimate of $\theta_{x}$ and
$\theta_{y}$ are
\begin{equation}
  \begin{aligned}
    v_{x}&=\xi_{x,1}^2 \,p_1 + \xi_{x,2}^2\, p_2
    =\frac{4(p_1+p_2)}{(2-\epsilon)^2 a^2}=\frac{2}{2-\epsilon}\;,\\
    v_{y}&=\xi_{y,3}^2 \,p_3 + \xi_{y,4}^2\,
    p_4=\frac{4(p_3+p_4)}{(2-\epsilon)^2 b^2}=\frac{2}{2-\epsilon}\;,
  \end{aligned}
\end{equation}
which do not depend on $a$ or $b$. The sum $v_x+v_y=4/(2-\epsilon)$
saturates the Nagaoka bound as claimed.

We now consider estimating all three parameters $\theta_x$, $\theta_y$ and
$\theta_z$. One measurement strategy is to
use the same POVM outcomes for estimating
$\theta_x$ and $\theta_y$ but splitting $\Pi_5$ to get some
information on $\theta_z$. Ideally, we would like to set $a=b=0$ and use the following four
projectors
\begin{align}
  \label{nag_2pp}
  \Pi_1=\Pi_2&=\frac{1}{4}\begin{pmatrix}1&0&0&1\\
    0&0&0&0\\
    0&0&0&0\\
    1&0&0&1
  \end{pmatrix},\qquad
  \Pi_3=\Pi_4=\frac{1}{4}\begin{pmatrix}1&0&0&-1\\
    0&0&0&0\\
    0&0&0&0\\
    -1&0&0&1
  \end{pmatrix},\qquad
\Pi_5= \frac{1}{2}\begin{pmatrix}0&0&0&0\\
    0&1&\I&0\\
    0&-\I&1&0\\
    0&0&0&0
  \end{pmatrix}+
    \frac{1}{2}\begin{pmatrix}0&0&0&0\\
    0&1&-\I&0\\
    0&\I&1&0\\
    0&0&0&0
  \end{pmatrix}\;,
\end{align}
to obtain the most information on $\theta_z$ without affecting the estimate
of $\theta_x$ and $\theta_y$. But the problem is that at this
singular point, the first
four outcomes $\Pi_1$, $\Pi_2$, $\Pi_3$ and $\Pi_4$ do not give any
information on $\theta_x$ and $\theta_y$. To fix this, we need both $a$
and $b$ to be close to but not exactly zero. Writing
$\delta=(a^2+b^2)/2$, we can split $\Pi_5$ as
\begin{align}
  \Pi_{5}&=\begin{pmatrix}0&0&0&0\\
    0&1-\delta&-\delta&0\\
    0&-\delta&1-\delta&0\\
    0&0&0&0
  \end{pmatrix}\\
  &=\underbrace{\delta\begin{pmatrix}0&0&0&0\\
    0&1&-1&0\\
    0&-1&1&0\\
    0&0&0&0
  \end{pmatrix}}_{\Pi_5^{(3)}}+
\underbrace{          \frac{1-2\delta}{2}\begin{pmatrix}
             0&0&0&0\\
             0&1&-\I&0\\
             0&\I&1&0\\
             0&0&0&0
  \end{pmatrix}}_{\Pi_6^{(3)}}+
\underbrace{           \frac{1-2\delta}{2}\begin{pmatrix}0&0&0&0\\
    0&1&\I&0\\
    0&-\I&1&0\\
    0&0&0&0
  \end{pmatrix}}_{\Pi_7^{(3)}}\;,
\end{align}
which has outcome probabilities
\begin{equation}
\begin{aligned}
  p_5&=\delta \,\epsilon\;,\\
  \begin{drcases}p_6\\
    p_7\end{drcases}&=\frac{1}{2}(1-2\delta)\left(1\pm (1-\epsilon
    )\theta_z \right)\;.
\end{aligned}
\end{equation}
This together with
\begin{equation}
  \xi_{z,1}=\xi_{z,2}=\xi_{z,3}=\xi_{z,4}=\xi_{z,5}=0\;,\qquad
  \text{and}\qquad \xi_{z,6}=-\xi_{z,7}=\frac{1}{(1-\epsilon)(1-2\delta)}\;,
\end{equation}
give a variance for estimating $\theta_z$ as
$  v_z=\dfrac{1}{(1-\epsilon)^2(1-2\delta)} $
which approaches $v_z=\dfrac{1}{(1-\epsilon)^2}$ as $\delta$ tends to zero.


\bibliography{ref}
\bibliographystyle{unsrtnat}
%\bibliographystyle{apa}
%\bibliographystyle{plain}



\end{document}
%%% Local Variables:
%%% mode: latex
%%% TeX-master: t
%%% End:
