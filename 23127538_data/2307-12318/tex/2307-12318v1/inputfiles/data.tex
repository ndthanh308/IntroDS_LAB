
\section{Literature data \label{sec:corotdata}}
%We selected three interesting stars among CoRoT tarjets: ID 102314644, ID 102358531 and ID 102298126. All of them were observed during the third long run, LRa03, which targeted the Anti-Galactic center.   

%$342\,598$


\subsection{Known stellar quantities from the literature}
\cid\   ($V \sim 12.2$, $\alpha = 6 \rm h 10 \rm m 26.73 \rm s$ and $\delta = +4^{\circ}18' 12.19" $) was observed during the third \corot\ long run, LRa03, which targeted the Anti-Galactic centre (see Fig~\ref{fig:fieldmapcorot}).  
The observations lasted 148 days from 2009, October 10th to 2010, March 1st. 
The EXODAT database \citep{2009AJ....138..649D} indicates the star has an A5V spectral type and 2MASS photometry of $J$=11.394, $H$=11.18, $K$= 11.131. 
It also indicates a star with reddening of $E(B-V) = 0.4$ mag, however, more recently, \citet{lallement2019} estimated $E(B-V) = 0.248 \pm 0.079$ mag based on the distance of the star \footnote{\url{https://stilism.obspm.fr/reddening?frame=galactic&vlong=204.3733&ulong=deg&vlat=-7.104248&ulat=deg&valid=}}. 
The sky map given by the \corot\  database is shown in Fig.~\ref{fig:fieldmapcorot} upper panel which clearly identifies the target.  We also give a wider angle sky map showing our target at the centre and the positions of Gaia Data Release (GDR2/GDR3) identified sources  \citep{gaiadr2, refId0, 2022arXiv220800211G}.

The photometry and various identifications of the star are given in Table~\ref{tab:literaturedata}.




% Figure environment removed


\begin{table}[]
    \centering
        \caption{Identification and literature data for \object{CoRoT~102314644}. }
    \begin{tabular}{llllllll}
    \hline\hline
    Parameter  & Value & Ref.\\
    \hline
Id &         \cid \\
          & GDR2 3317411131453435008\\
          & GEDR3 3317411131453435008\\
          & USNO-A2 	0900-02423283\\
        &  2MASS	06102674+0418122\\
$\alpha$  [deg] & 	92.611376 & 1\\
$\delta$  [deg] & 	+4.303372 & 1\\
$\alpha$ [hr mn ss]	 &   6h 10m 26.73 s \\ 
$\delta$ [hr mn ss] & +4h 18m 12.19s  \\
$l$ [deg] & 204.373325 \\
$b$  [deg] & --7.104326 \\
%$l$ \\
%$b$ \\
Spectral Type  & A5V & 2\\
%E(B-V) [mag] & 0.40 & 2\\
E(B-V) [mag]& 0.248 $\pm$ 0.079 & 3\\
$C$ [mag] & 12.3779\\
         $R$  [mag]	 & 12.3779 \\
$J$   [mag]& 11.394 $\pm$ 0.023  \\ 
$H$   [mag]&  11.18 $\pm$ 0.023 \\ 
$K$	  [mag]& 11.131$\pm$ 0.023 \\ 
$G$ [mag] & 12.451 & 1\\ % SNR flux/error = 1391 
	$G_{BP}$  [mag]& 12.7584  & 1\\
	$G_{RP}$  [mag]& 11.977113  & 1\\
	$G_{BP} - G_{RP}$  [mag]& 0.781295  & \\
	$v_{\rm rad}$  [km~s$^{-1}$] & 32.9 $\pm$ 10.2 & 4\\
	$\pi_{GEDR}$  [mas] & 0.988 $\pm$ 0.013 & 1\\
	$\pi_{sys}$  [mas] & --0.271  & 5\\

 %       \teff  [K] & 7065 $\pm 460$ & this work\\%[5793, 5908, 6449] \\%2.7099545\\ %2.2743871   - 2.8190932
%	$L_\star$  [$L_{\odot}$] & 13.6 $\pm$2.9& this work\\% 7.6 - 8.5
%	$R_\star$  [$R_{\odot}$] & 2.27 $\pm$ 0.07 & this work\\
%	$\log g$ [cgs, dex] & 3.9 $\pm$ 0.1 & this work\\
	%$^{+0.08}_{-0.03}$ & %	BC [mag] & 0.07 $\pm$ 0.02\\
\hline\hline
    \end{tabular}
    \label{tab:literaturedata}
      References: $^1$\cite{refId0}, $^2$\cite{2009AJ....138..649D},    $^3$\cite{lallement2019},
      $^4$\cite{gaiadr2}, $^5$\cite{lindegren2021}
    \end{table}

\subsection{Fundamental stellar parameters \label{sec:extinction}}

Gaia eDR3 also provides additional properties of the star: its parallax 
$\pi$, its radial velocity $v_{\rm rad}$ and photometry $G$, $G_{\rm BP}$ and 
 $G_{\rm RP}$, given in Table~\ref{tab:literaturedata}. 
 For $\pi$ we applied the recommended parallax zero-point correction of -0.027 mas based on the magnitude, colour and sky position of the star \citep{lindegren2021}.
Using the extinction, we dereddened the photometry and used the colour-\teff\ relations 
from 
\citet{2020arXiv201102517C} to derive \teff. To convert the extinction from E(B-V) to other bands, we assumed a reddening law R = 3.1 and we used the coefficients from  \cite{2018A&A...614A..19D}.
The colour-\teff\  relations require \logg\ and [Fe/H] as input, and so we 
used $\log g = 3.9$ (see below) and assumed solar metallicity in the absence of literature values.  
Then, using $G$, extinction $A_G$, the parallax and a bolometric correction, we calculated 
the luminosity, $L$.   
Using the Stefan-Boltzmann law with these values we estimated the stellar radius. 
Finally, using an estimate of mass between 1.7 and 2.1 $M_{\sun}$ we calculated a surface gravity of 3.9 $\pm$ 0.1 using the derived radius.


\teff\ and  $L$ are highly correlated because
they both depend on the extinction value. To calculate the uncertainties and correlations in the \teff\- $L$ plane, we performed
simulations where we perturbed the input values ($E(B-V)$, $\pi$, $G$, $G_{\rm BP}$, and $G_{\rm BP}$) by their errors.  Then we propagated these perturbed values to the \teff, $L$, radius, and \logg. The values obtained for $L$ and \teff are in agreement with the assumption of the star being a hybrid $\gamma$ Dor-$\delta$ Scuti.  
 The derived values and their 1-D uncertainties are: $L_\star =  13.6$ $\pm$ $2.9$ $L_{\odot}$; $ \teff = 7065 $ $\pm$ $460$ and $R_\star =  2.27 \pm 0.07$ $R_{\odot}$. In our interpretation of the models in Sect.~\ref{sec:interpretation} we used these values as a first approximation to constrain the models\footnote{Since the finalisation of the work, Gaia DR3 proposes $L/L_{\sun}=11.9 \pm 0.4$ and $T_{eff}=6842^{+300}_{-200}$ K which are in good agreement with ours, and the slight differences have little impact on the results.}.


\subsection{\corot \ Light curve}

We followed a similar analysis of this \corot\ light curve to that performed in \citet{2012A&A...540A.117C} and \citet{2013A&A...556A..87C}. We used the reduced N2 light curves from \citet{2009A&A...506..411A}. 
The light curve consists of a total of $386\,381$ measurements obtained with a temporal resolution of 32 s.  
We retained only $342\,598$ points, those flagged as "0" by the \corot\ pipeline that were not affected by instrumental effects such as stray-light or cosmic rays. 
We then corrected the measurements by long-term trends (systematic trends).  Individual measurements considered outliers (primarily high-flux data points caused by cosmic ray impacts) were removed by an iterative procedure. 
We retained a total of $340\,257$ measurements in total, which gives an 
approximate frequency resolution of 0.008 c/d.

The resulting light curve is represented at different timescales in Fig. \ref{lightcurve}. 
The amplitude has been calculated by converting from flux to magnitudes and subtracting the mean. 
%applying a constant of $26.6423$.  
%The precision of the constant is not relevant to the subsequent analysis presented here.
The timescale is labelled in units of the \corot\ Julian day (JD), where the starting \corot\ JD corresponds to HJD 2445545.0 (2000, January 1st at UT 12:00:00). 
On the top panel, we show the full corrected light curve spanning 148 days.  In the middle and 
lower panels, we show 20 and 5 days time spans, respectively.  Here we can distinguish two kinds of periodic time scales: one corresponding to low frequencies, characteristic of $\gamma$ Dor stars (middle panel), and one due to higher frequencies, which are characteristic of the $\delta$ Sct star (lower panel). 


% Figure environment removed
