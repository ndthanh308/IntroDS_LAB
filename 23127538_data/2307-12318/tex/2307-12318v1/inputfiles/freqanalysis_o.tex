\section{Light curve analysis \label{sec:frequencyanalisis}}

We analysed the frequency content of the light curve using the package Period04 \citep{2005CoAst.146...53L}. We searched frequencies in the interval [0;100] c/d. For each detected frequency, the amplitude and the phase were calculated by a least squares sine fit. The data were then cleaned of this signal (this is known as pre-whitening) and a new analysis was performed on the residuals.  This iterative procedure was continued until we reached the signal to noise (S/N) equal to 5.2 as it is recommended \citep{2021AcA....71..113B}.  The first Fourier transform in the range 0 -- 30 c/d is depicted in Fig.\ref{fourierTF1}, with the y-axis showing amplitude.

We eliminated frequencies lower than 0.25 c/d. These correspond to trends in the \corot\ data \citep{2012A&A...540A.117C}, and the satellite orbital frequency ($f_{\rm sat}=13.97213$ c/d) along with its harmonics. In addition, small-amplitude frequencies 
with a separation 
from large-amplitude frequencies less than the  frequency resolution were ignored.   
These smaller amplitude frequencies are not real and are due to the spectral window or to amplitude or frequency variability of the pulsations during the observations \citep{2016MNRAS.460.1970B}. 

% In addition, small-amplitude frequencies separated by $1/\Delta T $ from a large-amplitude frequency were ignored, , where 32 s is observation frequency.  These are due to the the spectral window, which is related to the length of the data set.
%add reference to this, and even put in appendix if its useful for the reader})
As a result, we obtained a total of 68 stellar frequencies. The first 10 frequencies with the highest amplitude are shown in Table \ref{10freq} and the complete list with uncertainties is given in Tables \ref{listatotal1} and \ref{listatotal2}.

We also included in Tables \ref{listatotal1} and \ref{listatotal2} an identity for each frequency (see next Section). Briefly, we identified two ranges of frequencies: $\delta$ Sct and $\gamma$ Dor frequency ranges, which are labelled with ``p'' and ``g'', respectively; and the frequency with the highest amplitude in each range has the sub-index ``1'' and subsequent frequencies with lower amplitudes are labelled with increasing sub-index.





% Figure environment removed





The uncertainties in the frequencies were calculated by performing Monte-Carlo-like simulations on the light curve and recalculating the frequency content of each simulated light curve.  More concretely, 
we created a fake signal $s_j$ by adding background noise to the original signal.   We calculated the periodogram and then fit the individual frequencies of the simulated periodogram.  The fit to each frequency $f_{j,i}$, where $i$ runs over the list of independent frequencies, was retained for each $j$ = 1, ... $N$ simulation.   We used $N$ = 500 as this provided a good balance between computation time and enough sampling.  We then analysed the resulting distributions of each $f_i$, by calculating the 68\%, 95\% and 99.7\% confidence intervals.  We checked first that 
these values scaled roughly as we expect them to.  We report the 99.7\% interval ($\sim\pm3\sigma$) in the second column in Tables~ \ref{listatotal1} and \ref{listatotal2} .  



\begin{table}
  \centering
  \caption{List of the first ten frequencies with the highest amplitudes.} 
  \begin{tabular}{ccccc}
    \hline\hline\noalign{\smallskip}
 & Frequency & Amplitude   & Phase & Ident   \\
 &   [c/d]   & [mmag] &   $\Phi$[rad]  &    \\

 
\hline 
$F_{1}$   & 11.39107 & 8.680 & 0.991701 & $p_{1}$\\
$F_{2}$   & 0.65259  & 4.470 & 0.819798 & $2f_{rot}$\\
$F_{3}$   & 11.89972   & 3.726& 0.572764 & $p_{2}$\\ 
$F_{4}$   & 1.00595   & 2.002 & 0.601673 & $g_{1}$ \\ 
$F_{5}$   & 0.87286  & 1.881 & 0.772675 & $g_{2}$ \\
$F_{6}$   & 0.90251  & 1.522 & 0.581354  & $g_{3}$\\
$F_{7}$   & 0.93445   & 1.496 & 0.237861 & $g_{4}$\\ 
$F_{8}$   & 0.32629    &1.374 & 0.489074 & $f_{rot}$\\ 
$F_{9}$   & 0.88683  & 1.238  & 0.682482& $g_{5}$\\
$F_{10}$  & 11.25403   & 1.165 & 0.117229 & $p_{3}$\\ 


\hline\hline
\end{tabular}
\label{10freq}
\end{table}




