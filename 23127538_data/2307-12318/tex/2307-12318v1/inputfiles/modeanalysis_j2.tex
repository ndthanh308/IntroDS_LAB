

\section{Analysis of extracted frequencies\label{sec:mode-analysis}}

%The frequencies identified in Sect.~\ref{sec:frequencyanalisis} can be discussed 
%in terms of different mode identifications. 
We analyze the frequencies derived in Sect. ~\ref{sec:frequencyanalisis} and we distinguish four main regimes to 
discuss: \dst\- type frequencies, \gds\- type periods, a regime with a coupling of ``p'' and ``g'' modes, and frequencies whose nature we discussed in terms of surface activity or gravitational effects provoked by a companion. One of the tools we used for the analysis of the frequencies is the phase diagram. The construction of these diagrams consists in taking all the observations and folding the light curve modulo a single standardized period (in time).  Each time point is then assigned a phase with respect to this chosen period, and it takes a value of between 0 and 1, ($0< \phi < 1$).  All measurements are then plotted with phase as the independent variable. 
%a duration of one unit is assigned

\subsection{Spots or binarity?}
\label{surfaceactivity}

%\textbf{CHANGE: to mention here Rossby modes and the paper in which Vichi collaborated.}
We noted that the first low frequency $F_2=0.65259$ c/d with $A=4.47$ mmag has a half frequency harmonic $F_8=0.32630$ c/d with $A=1.37$ mmag. Such a combination of a frequency and a lower amplitude half frequency corresponds to a double wave curve typical for spotted or eclipsing stars (see e.g \citealt{2017MNRAS.468.2017P}). Figure  \ref{diagramafaseprot} shows the phase diagram corresponding to $F_8=0.32630$ c/d after removing all frequencies corresponding to pulsation modes (see Sect. \ref{subsect:gammaDor} and \ref{ssec:deltascutidomain}). It clearly shows a double wave curve which can be explained in terms of spots or a companion of an ellipsoidal variable, assuming that $F_8=0.32630$ c/d is the orbital frequency.
In the case of spots, the star appears slightly fainter when a large dark spot is on the visible side, and slightly brighter when it is not. Note that the phase diagram corresponding to the rotation frequency in a regular single star without pulsation frequencies or surface activity, should be flat. 
A similar effect would be produced by a companion in an ellipsoidal variable system. These systems are non-eclipsing close binaries whose components are distorted by their mutual gravitation and the variations observed in the light curve are due to the changing variations are therefore due to the changing cross-sectional
areas and surface luminosities that the distorted stars present
to the observer at different phases \citep{1985ApJ...295..143M}.

We explored the possibility of being in the presence of one of these systems. We followed the equations in \citet{1985ApJ...295..143M} assuming $P=3.06 d$, $R_1=2.27 R_{\odot}$ as derived in Sec. 2.2, $M_1=1.75 M_{\odot}$, $\tau =0.2$ and $\mu=0.4$ from \citet{2011A&A...529A..75C} and $\Delta m=0$. We found possible solutions for a mass companion, resulting impossible to dismiss this hypothesis, for example, $M_2=0.7 M_{\odot}, 1.4M_{\odot}$ for $A=12R_{\odot}, 13R_{\odot}$ respectively, being $A$ the semimajor axis.

With the aim to explore the existence of spots, we examined the behaviour of the star over several rotational periods, assuming a rotational frequency equal to $F_8=0.32630$ c/d ($\sim$ 3.06466 d). We binned the data of the light curve in groups of ten measurements by assigning the average in time and magnitude to each group, and then we pre-whitened the data with all the pulsational frequencies. The result is presented in Fig. \ref{curvaluz2} for the duration of  3 rotational periods, each of them separated with horizontal lines. Two phenomena are present: amplitude variations from one orbit to another and moving bumps. The moving bumps might be explained by spots located at different latitudes. Additionally, the changes shown in Fig. \ref{curvaluz2} can be due to spots with a short lifetime. In the Sun, for example, the lifetime of the spots can vary between hours to months and it is known that they usually migrate \citep{2003A&ARv..11..153S}. Besides, it has been shown that for hot stars the lifetime tends to decrease, especially for those stars with short rotational periods  \citep{2017MNRAS.472.1618G} as the case of \cid. This suggests that \cid\ can be a spotted star with a rotation period of $P_{rot}=3.0647$ d.  Nevertheless, we found frequencies ($F_{49}$, $F_{55}$ and $F_{64}$) that are linear combination of $f_{rot}$ and this strongly suggest that the origin of $f_{rot}$ is not surface activity \citep{2015MNRAS.450.3015K} but, possibly, the beating of undetected pulsation frequencies. In order to determine properly the origin of these variabilities, spectroscopic measurements are required.
 

% Figure environment removed


 
% Figure environment removed


\subsection{$\gamma$ Doradus domain}
\label{subsect:gammaDor}
%lista de frecuencias y figuras

We found a total of 29 frequencies in the range of 0.3262 -- 3.6631 c/d. From these frequencies, those we consider g-modes oscillations are labelled as ``g'' modes in Tables \ref{listatotal1} and \ref{listatotal2}. The frequency with the highest amplitude in this domain, after $F_2=2f_{rot}$, is $F_4=1.0059$ c/d with $A=2.0$ mmag.
%Data phased with this frequency are represented in Fig \ref{diagramafasegammador} \re{wich clearly shows an sinusoidal behaviour indicating that this frequency is an oscilation eigenmode}.

%% Figure environment removed


% Figure environment removed


Light variabilities from orbital or rotational variation are typically non-sinusoidal, thus, in order to distinguish between possible real $g$-modes and the frequencies corresponding to the spots in this domain, we analyse the phase diagram for each frequency. The phase diagrams for typical $g$ and $p$ modes frequencies have a sinusoidal behaviour. For instance, in Fig. \ref{diagramafasedeltasct} we have folded the light curve at the period corresponding to $F_1$, and here we can clearly observe sinusoidal behaviour.  This suggests that $F_1$ is an oscillation eigenmode. On the other hand, for $F_{18}=0.4638$ c/d,  a non-sinusoidal can be  spotted. In Fig. \ref{diagramafasespot} the phase diagram for $F_{18}$ for different amplitude scales is depicted. It seems that there is a maximum around $0.1$ and a minimum between 0.3 and 0.4. This suggests that $F_{18}$ may corresponds to periods related to spots. Nevertheless, we note that this test provides only hints about the origin of the frequency and is not conclusive. In fact, if $F_{18}$ were originated by spots, it would imply over 40$\%$ in differential rotation, which is a value slightly high for A-F stars \citep{2013A&A...560A...4R}.


Considering $F_{18}$ as originated from spots and dismissing the rotational frequency and its harmonics, we retain a total of 26 frequencies in the $\gamma$ Doradus domain, possibly $g$-modes, depicted in black in Fig. \ref{amplitudefregammador}. In addition, we searched for frequency combinations in this range, but no frequency couplings or splittings were found among these $g$-modes. We labelled the frequencies '$F_k$' as a combination of frequencies after finding a fit of at least two significant digits among all the possible combinations of type '$mF_{i} \pm nF_{j}$', for the given frequency  '$F_k$'.
%F16 was removed to do this figure


% Figure environment removed


%\red{why is the equidistant property important?  introduce this better and explain HOW you did this analysis.}//
Hybrid $\delta$ Sct-$\gamma$ Dor stars, as well as $\gamma$ Dor stars,  are characterised by having high-order $g$ modes. For these modes, with high radial order ($k$) and long periods, the separation of consecutive periods ($|\Delta k|=1$) becomes nearly constant and it depends on the harmonic degree ($\ell$), given the asymptotic theory of non-radial stellar pulsation \citep{1980ApJS...43..469T} in which the asymptotic period spacing is:

\begin{equation}
  \Delta \Pi_l=\frac{\Pi_0}{\sqrt{\ell(\ell+1)}},
\label{dpasymp}
\end{equation}

with

\begin{equation}
  \Pi_0=2\pi^2 \left(\int_{r_1}^{r_2}N\frac{dr}{r}\right)^{-1},
\label{pasymp}
\end{equation}

where r is the distance from the stellar centre, N is the Brunt--V\"ais\"al\"a frequency and $r_1$ and $r_2$ are the boundaries of the propagation region.

Motivated by this fact, we searched for equidistant $\gamma$ Dor periods, by analysing the differences between all the periods found in the $\gamma$ Dor domain. We found a series of 6 equidistant periods with a mean separation of $\Delta \Pi= 1621$ sec (see Table \ref{asimp}). These periods correspond to $g$-modes of the same harmonic degree $\ell$ and consecutive radial orders $k$. The asymptotic series is depicted in Fig \ref{asymp}. In the top panel of this figure, we show the periods ($\Pi$) versus an arbitrary radial order ($k$). We can see that these periods are almost equally spaced forming a line. In the bottom panel of this figure, we show the forward period spacing ($\Delta \Pi=\Pi_{k+1}-\Pi_{k}$) versus $k$, and we 
denote the corresponding average period spacing with the red horizontal continuous line. According to \citet{2016A&A...593A.120V}, the value we found is more likely to correspond to an asymptotic series with $\ell=2$. In this paper the authors determine values of about 3100 s and 1800 s for the asymptotic period spacing calculated with $\ell=1$ and $\ell=2$ respectively, employing Eq. \ref{dpasymp} and \ref{pasymp}. In fact, our models predict a harmonic value $\ell=2$ for this series.





\begin{table}
  \centering
  \caption{List of the six periods of the asymptotic series.} 
  \begin{tabular}{cccc}
    \hline\hline\noalign{\smallskip}
 & Period & A    & Ident   \\
 &   [sec]   & [mmag] &   \\
\hline 
$F_{14}$  & 90878.5  & 0.841&  $g_{8}$ \\
$F_{7}$   & 92460.8  & 1.496 & $g_{4}$\\ 
$F_{32}$ & 94061.3   &  0.387   &$g_{19}$\\
$F_{6}$   & 95733.0  & 1.522 & $g_{3}$\\
$F_{9}$   & 97425.6  & 1.238  & $g_{5}$\\
$F_{5}$   & 98984.9 & 1.881 & $g_{2}$ \\

\hline\hline
\end{tabular}
\label{asimp}
\end{table}


% Figure environment removed

%


\begin{table}[h]
    \centering
    \begin{tabular}{lllll}
    \hline\hline
    Parameter & Value \\
    \hline
    $\Delta \Pi$ & 1621 s\\
    $P_{\rm rot}$ & 3.064 d\\
    p-mode & labelled as '$p_i$' in Tables~\ref{listatotal1} and \ref{listatotal2}\\
    g-mode & labelled as '$g_i$' in Tables~\ref{listatotal1} and \ref{listatotal2}\\
    p-g-modes & see Table~\ref{combpg}\\
    quintuplet & see Table~\ref{quintuplete}\\
\hline\hline
    \end{tabular}
    \caption{Summary of the variable content of the star.
    %, see Sect.~\ref{sec:frequencyanalisis}.
    }
    \label{tab:freqmode_summary}
\end{table}



\subsection{$\delta$ Scuti domain\label{ssec:deltascutidomain}}
In the $\delta$ Scuti domain, we found a total of 38 frequencies in the range 8.6 -- 24.73 c/d. The highest amplitude frequency in this range is $F_1=11.3910$ c/d with $A=0.008$ mag. 
A phase diagram folded with this frequency shows sinusoidal behaviour (Fig.~ \ref{diagramafasedeltasct}), indicating thus that $F_1$ is an eigenmode.

%Data phased with this frequency are represented in Fig \ref{diagramafasedeltasct}, , which clearly shows a sinusoidal behaviour indicating, as in the case of $F_4$, that this frequency is an oscillation eigenmode.
%As can be seen in thiswhy do you do this diagram? is it to show that it is due to sinusoidal behaviour, i.e. mode frequencies?}.

% Figure environment removed


Stellar rotation induces rotational splitting of the frequencies in the pulsation spectra. Considering rigid rotation and the first-order perturbation theory, the components of the rotational multiplets are:

\begin{equation}
\nu_{nlm}= \nu_{nl}+m(1-C_{nl})\frac{\Omega}{2\pi}
\end{equation}

where $\nu_{ln}$ is the central mode of the multiplet and $\Omega/2\pi$ is the rotational frequency. We found a quintuplet centred on $p_1=F_1$ (see Table \ref{quintuplete}), which clearly indicates that this frequency is a non-radial mode with $\ell=2$. The differences between the central mode and the components of the quintuplets are given in the last column of Table \ref{quintuplete}. Considering $C_{nl} \approx 0$ for $p$ modes, we find a very good agreement with the value for $f_{rot}=0.32629$ c/d derived in Sec. \ref{surfaceactivity}. However, this match does not dismiss the possibility of \cid\ being an ellipsoidal variable. In fact, an alternative interpretation of this splitting would be tidally deformed oscillation modes that have variable amplitude over the orbit, in case 0.32629 c/d is indeed a binary orbital period.



We also found 4 combinations between $p$ modes exclusively, and the harmonics for $p_1$ and $p_2$ (see Table \ref{combp}). The linear combination between two frequencies, yields a third frequency whose amplitude is smaller than those that form it. It is important to distinguish between mode-coupled frequencies from "pure" frequencies because when developing asteroseismic modelling, only frequencies that come from pulsation, i.e. "pure" frequencies can be accurately calculated and thus used. 
%those that When an asteroseismic modelling is performed, \red{what is coupling and why do we care?}

\begin{table}
  \centering
  \caption{List of frequencies of the quintuplet.} 
  \begin{tabular}{ccccc}
    \hline\hline\noalign{\smallskip}
 & Frequency & A    & Ident &$p_1-F_i$  \\
 &   [c/d]   & [mmag] & & [c/d]  \\
\hline 
$F_{19}$  & 10.73844 & 0.667& $p_{1}-2f_{rot}$&0.65263\\
$F_{63}$  &11.06506   &  0.081& $p_1-f_{rot}$&0.32601\\
$F_{1}$   & 11.39107 & 8.680 & $p_{1}$&--\\

$F_{58}$	& 11.71775    &  0.083&	 $p_1+f_{rot}$&-0.32668\\ 

$F_{47}$ &12.04353   &0.133   &  $p_{1}+2f_{rot}$&-0.65246\\ 

\hline\hline
\end{tabular}
\label{quintuplete}
\end{table}

%lista de frecuencias


\begin{table}
  \centering
  \caption{List of combinations between $p$ modes and harmonics.} 
  \begin{tabular}{ccccc}
    \hline\hline\noalign{\smallskip}
 & Frequency & A   &Phase & Ident   \\
 &   [c/d]   & [mmag] &[rad]&         \\
\hline 
$F_{38}$ & 23.29078   & 0.306& 0.938  & $p_{1}+p_{2}$\\
$F_{46}$  & 22.64486   &0.143& 0.362  & $p_{1}+p_3$\\
$F_{48}$ & 22.80735 & 0.125& 0.498   & $p_{1}+p_{4}$\\
$F_{66}$  & 24.73414  & 0.052&0.559  & $p_{1}+p_{5}$\\ 
$F_{23}$  & 22.78214  & 0.539& 0.406   & $2p_{1}$\\ 
$F_{52}$	 & 23.79931  & 0.102& 0.865  & $2p_{2}$\\ 



\hline\hline
\end{tabular}
\label{combp}
\end{table}



Removing the couplings, the harmonics and the splitting corresponding to $p_1$, we retain a total of 15 independent frequencies in the range of 10.9 -- 21.4 c/d, depicted in black in Fig. \ref{amplitudefredsc}.%[10.9249;21.3941] c/d. 
%These \texttbf{frequencies} are depicted in the amplitude versus frequency diagram by the black frequencies, Fig. \ref{amplitudefredsc}.
%by the black frequencies shows the logarithm of the amplitude versus the frequency of the $p$ modes frequencies founded in the $\delta$ Scuti domain. 

% Figure environment removed

%\red{remove + from figure, change scale to mmag, and then plot "log" without mentioning the word log.  }

%In addition, we found $p$ and $g$ modes coupling in the $\delta$ Sct domain
\subsection{$P$ and $g$ modes combinations}


The coupling between $p$ and $g$ modes was originally proposed as a way to explore $g$ modes in the Sun, see \citet{1993ASPC...42..273K} and more recently \citet{fossat17}. According to these studies, internal solar $g$-modes produce frequency modulation of $p$-modes which results in a pair of side-lobes symmetrically placed about each $p$-mode frequency.
We explored this feature of $g$-modes in $p$-modes by searching combinations of frequencies in the $\delta$ Sct domain. We found these combinations in the form of $p_1 \pm g_i$, with $i=1,2,3$ and $p_1-g_4$ and $p_1-g_7$. The list of coupled $p$ and $g$ modes is given in Table \ref{combpg}. This same interaction has also been found in two other hybrid stars, namely, CoRoT-100866999 and CoRoT-105733033 studied in detail in \citet{2013A&A...556A..87C} and \citet{2012A&A...540A.117C}, respectively. This indicates that the coupling mechanism  first proposed by \citet{1993ASPC...42..273K} also operates in hybrid $\delta$ Sct and $\gamma$ Dor stars.

Is important to notice that the detection of a combination between $p$ and $g$-modes, i.e. $p_i \pm g_j$, implies that $p_i$ and $g_j$ originated in the same star.

Additionally, we found one frequency between the $\delta$ Sct and $\gamma$ Dor domains, $i_1=5.038$ c/d in Table \ref{listatotal1}, whose position in the frequency spectrum did not allow us to safely classify them.


\begin{table}
  \centering
  \caption{List of $p$ and $g$ mode coupling for the highest amplitude frequency.} 
  \begin{tabular}{cccc}
    \hline\hline\noalign{\smallskip}
 & Frequency & A    & Ident   \\
 &   [c/d]   & [mmag]   &    \\
\hline 
$F_{50}$  & 10.38536& 0.113   & $p_1-g_1
$\\ 
%$F_{40}$ & 10.89836 & 0.0002863 & 0.563443 & $p_{1}-g_{1}$\\
$F_{55}$  & 12.39788 & 0.0920  & $p_1+g_1
$\\ 
%$F_{59}$ &12.39788     &  0.000093 & 0.711660 &$p_1+g_1$\\
$F_{49}$  & 10.51816 & 0.115  & $p_1-g_2
$\\ 
%$F_{53}$	&10.51816    &  0.000114 &	 0.095373 &$p_1-g_2$\\
$F_{54}$  & 12.26440 & 0.096   & $p_1+g_2
$\\ 
%$F_{58}$ & 12.26440    &  0.000095 & 0.287726 &$p_1+g_2$\\
$F_{62}$  & 10.48902 & 0.0808  & $p_1-g_3
$\\
%$F_{67}$ &10.48902&	  0.000080 &  0.452563  &$p_1-g_3$\\
$F_{60}$  & 12.29424 & 0.0836   & $p_1+g_3
$\\ 
%$F_{65}$ &12.29424    &  0.000075 & 0.418908 &$p_1+g_3$\\
$F_{57}$  & 10.45718 & 0.0881   & $p_1-g_4
$\\ 
%$F_{61}$	&10.45718     &  0.000086 &  0.441093 &$p_1-g_4$\\
%$F_{63}$  & 12.32561 & 0.0851   & $p_1+g_4
%$\\ 
%$F_{63}$ &12.32561  &  0.000071 & 0.147445 &$p_1+g_4$\\
%$F_{74}$  & 10.50459 & 0.0661   & $p_1-g_5
%$\\
%$F_{74}$	&10.50459      &  0.000068 &  0.726095 &$p_1-g_5$\\
%$F_{73}$  & 10.23606 & 0.0667 & $p_1-g_6
%$\\
%$F_{81}$  & 24.73414  & 0.000052&0.559623  & $p_{1}+p_{5}$\\ 
%$F_{73}$	&10.23606     &  0.000066 &  0.275057 &$p_1-g_6$\\ 
%$F_{82}$  & 12.54761 & 0.052 & $p_1+g_6
%$\\ 
%$F_{82}$ &12.54761     &  0.000050 & 0.024806 &$p_1+g_6$\\
$F_{59}$  & 8.62954 &  0.0848 & $p_1-g_7
$\\ 
%$F_{64}$ &8.62954    &  0.000083 &  0.993462 &$p_1-g_7$\\
%$F_{78}$  & 14.15072 & 0.0592  & $p_1+g_7
%$\\
%$F_{78}$ &14.15072       & 0.000059 & 0.690401 &$p_1+g_7$\\
\hline

%\hline\hlinehttps://github.com/wmwolf/py_mesa_reader
\end{tabular}
\label{combpg}
\end{table}
