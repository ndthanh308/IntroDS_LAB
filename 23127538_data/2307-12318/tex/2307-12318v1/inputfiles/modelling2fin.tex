




%                                                One column figure
%----------------------------------------------------------- S_vib
\section{Interpretation of frequency data\label{sec:interpretation}}

\subsection{Rotational period and critical velocity}

The analysis of low frequencies in A-F stars is a tricky task. It requires several considerations, especially when analyzing hybrid pulsators and this problem arises not only with CoRoT observations but also with TESS data. Many phenomena can mimic stellar oscillations and additional data than photometry is required to disentangle the possible phenomena \citep{2022A&A...666A.142S}.

In Sec. \ref{surfaceactivity} we interpreted the period found  $P_{\rm rot} = 3.064$ d, or $f_{\rm rot}= 0.326$ c/d in two different ways: the rotational period of the star or the orbital period of a binary system. Given that the splitting found can also be interpreted as tidally deformed oscillation modes that have variable amplitude over the orbit of a binary system, we could not rule out the possibility of \cid\  being a binary system.


With the aim to test further the case of a single star, we calculated the rotational and critical velocities for the values obtained in Sec. \ref{sec:corotdata}. By considering the estimated radius, $R_{\star} \sim 2.27  R_{\sun}$, we obtain a linear rotational velocity ($v=2\pi R/P_{rot}$) of $\sim$ 37 km~s$^{-1}$. In this case, the corresponding rotational critical velocity ($v_{crit}=\sqrt{GM_{*}/R_{*}}$) for a mass of 1.75$M_{\star}$ would be $\sim$ 383 km~s$^{-1}$, meaning that the linear velocity is less than 10\% of the critical velocity. 

The effect of rotation in main sequence stars varies parameters involved in the modelling of stars such as the mean period spacing and the splitting of $p$-modes even at linear velocities which are a low percentage of the critical velocity. Nevertheless, in this work, we present a preliminary model of \cid\ without considering rotation, as a first approximation. 

\subsection{Use of stellar models to constrain the mass and age}

With the aim to perform a preliminary modelling of \cid\, we first explore the position of this star in the HR diagram for masses and overshooting parameters.
%, without considering stellar rotation.

The stellar structure and evolution models were calculated with Cesam2k code \citep{2008Ap&SS.316...61M}\footnote{The following physics were considered: The opacities are those from \citet{1996ApJ...464..943I} and \citet{1994ApJ...437..879A}, we used the equation of state of OPAL project \citep{1996ApJ...456..902R} and a nuclear network with the following elements:  $^{1}H$, $^{2}H$, $^{3}He$, $^{4}He$, $^{7}Li$, $^{7}Be$, $^{12}C$, $^{13}C$, $^{14}N$ to describe the H (proton-proton chain and CNObi-cycle), and He burning and C ignition with reaction rates extracted from \citep{1999NuPhA.656....3A}. In addition, we adopted the classical mixing length theory (MLT) \citep{1958ZA.....46..108B} for convection with a free parameter $\alpha = 1.85$.  The occurrence of diffusion and mass loss during the evolution was dismissed and the solar metallicity distribution considered \citet{1998SSRv...85..161G}. We used MARCS atmosphere models \citep{2008A&A...486..951G}. All of our models have an initial H and He abundances per mass unit of $0.72$ and $0.26$ with an initial value $Z/X=0.0028$.}. We considered masses between $1.5$ and $1.8 
 M_{\odot}$ with a mass step of $0.05M_{\odot}$  and overshoot parameters of $\alpha=0.0$, $0.1$ and $0.3$. Overshooting phenomena were considered as an extent of the chemical mixing region around the convective core through the expression for the overshooting distance:
\begin{equation}
  d_{OV}=\alpha_{OV} \times min(H_P,r_S)
\end{equation}
where $H_P$ is the local pressure scale height and $r_S$ is the Schwarzschild limit of the core.


% Figure environment removed

Fig. \ref{HRexplore} shows the HR diagram with the evolutionary sequences for different masses and overshooting parameters from the pre-main sequences up to an abundance of H of 10$^{-6}$ in the core, along with the error boxes centred on the values of $Log (L/L_{\sun})$ and $Log (T_{\rm eff})$ derived in Sec. \ref{sec:corotdata}. 


In order to find a representative model for \cid, we selected different models indicated with circles inside the box shown in Fig. \ref{HRexplore}, and then we calculated their oscillation modes with GYRE code \citep{2013MNRAS.435.3406T}. We computed adiabatic radial and non-radial ($\ell=0,1$ and $2$) $p$- and $g$-modes in the frequencies range [0.3, 23] c/d, thus encompassing the range of observed frequencies.



\subsection{Asteroseismic analysis}

The presence of a series of equidistant periods in \cid\ (see Sect. \ref{subsect:gammaDor}) provides us with a useful tool for the search of a representative model: $ \overline{\Delta \Pi}$, the mean period spacing of high order $g$-modes.

As stars evolve in the main sequence and consume H in the core, the Brunt-V\"ais\"al\"a (B-V) frequency, which governs the behaviour of $g$ modes, is affected by the change of the convective core. For masses greater than $\sim 1.5 M_{\odot}$, the core shrinks and its edge moves inward as the star evolves. The period can be expressed as:

\begin{equation}
  \Pi_n \approx \frac{2 \pi^2 |n|}{\sqrt{l(l+1)}} \left[\int_a^b\frac{N}{r}dr\right]^{-1}
\end{equation}

where N is the Brunt-V\"ais\"al\"a frequency, and a and b are the lower and upper boundary of the propagation zone of the $g$-mode. Thus, during the evolution, the integral increases since it expands toward inner regions resulting in a decreasing period and therefore a decreasing period spacing of $g$ \citep[see][for example]{2008MNRAS.386.1487M}.



We used this parameter as an indicator of the evolutionary status of stars at the main sequence \citep{2015MNRAS.447.3264S, Kurtz2014, 2017A&A...597A..29S} which allowed us to place constraints in the search for a representative model. For each model inside the box in Fig. \ref{HRexplore}, we calculate the mean period spacing of $g$-modes for $\ell=1, 2$, as follows:

\begin{equation}
\overline{\Delta \Pi}_{\ell}= \frac{P_j-P_i}{n-1}
\end{equation}

where $P_j$ and $P_i$ are the closest periods to the extremes inside the observed interval [90878.5:98984.9] s where the asymptotic series lie; and $n$ is the number of periods found in this range.



Table \ref{tab:modelsdp} summarizes the mass, the overshooting parameter, the age, $\overline{\Delta \Pi}_{\ell}$ and the difference between $\overline{\Delta \Pi}_{\ell}$ and the value found in Sect. \ref{subsect:gammaDor} for modes with $\ell=1$ and 2 for \cid. 

Another parameter we employed to select our best model is the ratio between the period spacing for $\ell=1$ and $\ell=2$, which should be equal to $\sqrt 3$ in the asymptotic regime. We also included this value in Table \ref{tab:modelsdp} for the selected models. We decided to use this criterion due to the possible deviation from the asymptotic regime with the adopted search. Our model was selected by the one with the lowest $D_{l=2}$ among those ones closest to $\frac{\Delta \Pi_{l=1}}{\Delta \Pi_{l=2}}= \sqrt{3}$. This model has $1.75 M_{\odot}$, no core overshooting, $1241.24 \times 10^{6}$ yrs and its luminosity and radius are $11.36 \rm L_{\odot}$ and $2.48 \rm R_{\odot}$. We notice that mode-trapping or other internal mode-selection mechanisms might prevent us from detecting more periods belonging to the observed asymptotic series resulting in a mean period spacing of $g$-modes apart from the asymptotic value.




\begin{table}[h]
    \centering
    \begin{tabular}{llllllll}
    \hline\hline
Mass & Age&   $\Delta \Pi_{1}$ & $D_{1}$ & $\Delta \Pi_{2}$&  $D_{2}$& $\frac{\Delta \Pi_{l=1}}{\Delta \Pi_{l=2}}$ \\
    
    [$M_{\odot}$] & [10$^{6}$ yr]&  [s] & [s] &[s]& [s] & \\
    \hline
%    1.30 & 2043.8819 &1460.3399 & 160.6601\\
%     & 2931.3676 & 1542.8499& 78.1501\\
%     & 3393.7598 & 929.5375& 691.4625\\
%    \hline
%     1.40 & 1547.2234  & 1658.5750&  37.575\\
%     & 2432.1614&1390.2 & 230.8\\
%     &  2678.3151& 982.0333& 638.9667 \\
%    \hline
%    1.50 &  2175.9& 1388& 233& 1285 & 336 \\
%    \hline
%    1.55 &  1616.7 & 3338& 1717& 1979 & 358\\
%     &1962.5 & 2252& 631& 1033& 588\\
%    \hline
%    1.55  & 1761.22 & 3505& 1882& 1368 & 253\\
%    $\alpha_{ov}$=0.1 &   2087.5 &2521& 900& 1524 &97 \\
%         &2238.3 & 1578& 43 & 1592&29 \\
%    \hline
%     1.60 &1583.3 & 1766&145 & 2114& 493\\
%     & 1804.9 & 1513& 108& 775& 846\\
%      \hline
%     1.60 &1742.9 & 2311& 690& 1709 & 88 \\
%    $\alpha_{ov}$=0.1 & 2022.3& 2313& 1390 & 1259 &362\\
%      \hline
     1.65 & 1628.81& 1662& 41& 1494 &127 &1.112 \\
    
       \hline
%     1.65 & 2293.2& 1305& 316& 986 & 635 \\
%    $\alpha_{ov}$=0.3 & & & & & \\

 %    \hline

 %   1.70& 1063.0067 &1588.4750 & 32.525\\
 %    & 1188.6825  & 2170.5& 549.5\\
 %    &1401.1247 & 1376.4399&244.56 \\
 %  \hline
    1.70 & 1480 &3723 & 2102& 1179 & 442&3.157\\
    & 1489.46  & 1334& 287& 792& 829&1.684\\
 %   $\alpha_{ov}$=0.3 & & & & & \\
    \hline
     1.70 & 1318.24  & 4062 &2440 & 2085& 464&1.948\\
    $\alpha_{ov}$=0.3 &1631.84 &3858 &2237 &2333 & 712&1.653 \\
    \hline
    1.75 &   955.35 & 2999& 1378& 2148 & 527&1.396\\
      & 1241.24  & 2313&692 & 1491& 130&1.551 \\
       & 1352.82  & 2425& 804& 1233& 388&1.966\\
        & 1367.39  & 2040& 419& 1061& 560&1.922\\
%    $\alpha_{ov}$=0.3 & & & & & \\
%     &  &  1440 (m=-1)  &\\

    \hline
      1.75 & 1052.77  & 3815 & 2194 & 1624& 3&2.349\\
    $\alpha_{ov}$=0.1 &1315.38 &2117 &496 &2161 &540&0.979 \\
%     1.80 & 707.6874 & 2154.6666& 533.66\\
%     &1075.8561  &1690.1749 & 69.1749 \\
%     & 1256.3255 &1043.86 &577.14 \\
      \hline
      1.75 & 1301.64  & 4177 & 2556& 2428& 807&1.720\\
    $\alpha_{ov}$=0.3 &1582.64 &2799 &1178 &2086 &465&1.341 \\ 
      \hline
    1.80 &  869.37 & 2596& 975& 2201 &580&1.179\\
       &  1134.88 & 4022&2401 & 1297 & 324&3.101 \\
      &  1250.19 & 2330 & 709&1335 & 286&1.745\\
      & 1261.07 & 1595&26 & 1004 & 617&1.588\\
      \hline
  1.80 & 1041.9 & 3398& 1777& 1717 & 96 &1.979\\
   $\alpha_{ov}$=0.1 &  1266.84 & 2644&1023 & 2060 & 439&1.283\\
    \hline
    1.80             & 1175.14& 2859& 1238& 2085 &464&1.371 \\
   $\alpha_{ov}$=0.3 & 1429.6 & 3264&1643 & 2193 &572&1.488 \\ 
                     & 1517.43  & 2814 &1193 &2562 &941&1.098 \\ 
 %   $\alpha_{ov}$=0.1 & & & & & \\
%     &  &  1392(m=-1)  &\\
\hline\hline
    \end{tabular}
    \caption{Mass, age, mean period spacing of $g$ modes, the difference ($D_{l}=|\Delta \Pi - \Delta \Pi _ o|$) with the observed one for $\ell=1$ and 2 modes and the ratio $\frac{\Delta \Pi_{l=1}}{\Delta \Pi_{l=2}}$ for the selected models in our preliminary exploration indicated in Fig. \ref{HRexplore}.  The uncertainty in $\Delta\Pi$ is on the order of 20s.}
    \label{tab:modelsdp}
\end{table}
