
  \abstract 
  % context heading (optional)
  % {} leave it empty if necessary  
   {Observations from space missions have allowed significant progress in many scientific domains due to the absence of atmospheric noise contributions and having uninterrupted data sets.   In the context of asteroseismology, this has been extremely 
   beneficial because many oscillation frequencies with small amplitudes, not observable from the ground, can be detected.   One example of this success is the large number of hybrid $\delta$ Sct-$\gamma$ Dor stars discovered. These stars have radial and non-radial $p$- and $g$-modes simultaneously excited to an observable level allowing us to probe both the external and near-to-core layers of the star.}
  % aims heading (mandatory)
   {We analyse the light curve of hybrid $\delta$ Sct-$\gamma$ Dor star CoRoT ID 102314644 and characterise its frequency spectrum. Using the detected frequencies, we perform an initial interpretation developing stellar models.}
   % an initial interpretation developusing stellar models
  % methods heading (mandatory)
   {The frequency analysis is obtained with a classical Fourier analysis through the Period04 package after removing residual instrumental effects from the CoRoT light curve.  
   Detailed analysis on the individual frequencies is performed by using phase diagrams and other light curve characteristics. 
   An initial stellar modelling is then performed using the Cesam2k stellar evolution code and the GYRE pulsation code, considering adiabatic pulsations.}
   %$p$ and $g$ modes periods were considered in the pulsation models.}
  % results heading (mandatory)
   {We detected 29 $\gamma$ Dor type frequencies in the range $[0.32-3.66]$ cycles per day (c/d) and a series of 6 equidistant periods with a mean period spacing of $\Delta \Pi=1612$ s. In the $\delta$ Sct domain we found 38 frequencies in the range $[8.63-24.73]$ c/d and a quintuplet centred on the frequency $p_1=11.39$ c/d and derived a possible rotational period of 3.06 d. The frequency analysis of this object suggests the presence of spots at the stellar surface, nevertheless we could not dismiss the possibility of a binary system. The initial modelling of the frequency data along with external constraints has allowed us to refine its astrophysical parameters giving a  mass of approximately 1.75 \msol, a radius of 2.48 \rsol\ and an age of 1241 Myr.}     
   {The observed period spacing, a $p$-mode quintuplet, the possible rotation period and the analysis of the individual frequencies provide important input constraints for the understanding of different phenomena such as the transport of angular momentum, differential rotation and magnetic fields operating in A-F-type stars. Nevertheless, is fundamental to accompany photometric data with spectroscopic measurements in order to distinguish variations between surface activity from a companion.}