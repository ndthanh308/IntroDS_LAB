\section{Introduction \label{sec:intro}}
%Dominio y analisis
%Resultados
%tabla de frecuencias con combinaciones

In the last decade, several space missions such as the COnvection ROtation and planetary Transits (\corot) satellite \citep{2009A&A...506..411A} and NASA's Kepler space telescope \citep{2016RPPh...79c6901B}, have revolutionised asteroseismology,
thanks to their high-precision allowing the detection of very small amplitude modes that
are not detectable from ground-based instruments.  Indeed \dst\ stars have been known
for many decades now due to the high amplitude of some of their oscillation modes
which reach up to tenths of a magnitude, while \gds\ stars are known only since 1999 \citep{Kaye_1999} and thanks to 
uninterrupted data from space it was possible the detection of their low amplitude periodicities near one day \citep{2010aste.book.....A}.   
%These stars are A--F type stars, found both on the main sequence, as well as pre- and post- main sequence, on the lower end of the so-called instability strip \red{add citations}.
The existence of hybrid \dst-\gds\ stars has been known since 2002 \citep{2002MNRAS.333..262H}.  Their unique character of exhibiting both radial and non-radial pressure ($p$) oscillation modes typical of $\delta$ Sct variable stars, and gravity ($g$) pulsation modes characteristic of $\gamma$ Dor variable stars simultaneously allows one to probe their stellar structure from the core to the envelope. 

The $\delta$ Sct stars lie on and above the main sequence with masses of $1.5-2.5 M_{\odot}$ approximately and spectral types between A2 and F5. They exhibit radial and non-radial $p$- and $g$- modes driven by the $\kappa$ mechanism operating in the He II partial ionisation zone \citep{1962ZA.....54..114B} and the turbulent pressure acting in the hydrogen ionisation zone \citep{2014ApJ...796..118A}.

The $\gamma$ Dor variables are generally cooler than $\delta$ Sct
stars, with $T_{\rm eff}$ centred between 6700 K and 7400 K (spectral types
between A7 and F5) and masses in the range 1.5 to 1.8 $M_{\odot}$ approximately \citep{2015pust.book.....C}. They pulsate in low-degree, high-order $g$ modes apparently driven by a flux modulation mechanism called convective blocking and induced by the outer convective
zone \citep{2000ApJ...542L..57G, 2004A&A...414L..17D, 2005A&A...434.1055G}. The high-order g modes ($n \gg 1$) excited in these stars, allow the use of 
the asymptotic theory \citep{1980ApJS...43..469T} and the departures from
uniform period spacing to explore the possible
chemical inhomogeneities in the structure of the convective
cores \citep{2008MNRAS.386.1487M}. 

The aforementioned distinction between $\delta$ Sct and $\gamma$ Dor stars is a topic of debate. Diverse studies on samples of $\delta$ Sct and $\gamma$ Dor stars suggest that the hybrid behaviour on these stars is very common \citep{2010ApJ...713L.192G, 2011A&A...534A.125U, 2015AJ....149...68B, 2015MNRAS.452.3073B}. Moreover, in 2016, \cite{2016MNRAS.457.3163X} calculated a  theoretical instability strip using a non-local and time-dependent convection theory and concluded that the $\kappa$ mechanism operates significantly in warm $\delta$ Sct and $\gamma$ Dor stars while the coupling between convection and oscillations is responsible for excitation in cool stars. Furthermore, the instability strips of $\delta$ Sct and $\gamma$ Dor stars partially overlap
in the Hertzprung-Russell (HR) diagram \citep[see, for instance, Fig. 1 of][]{2010ApJ...713L.192G}, explaining the existence of hybrid $\delta$ Sct-$\gamma$ Dor stars. As we mentioned, the simultaneous presence of both $g$ and $p$ non-radial, along with radial excited modes, allows one to place strong constraints on the whole interior structure.  In addition, some of these objects show rapid rotation, making these objects excellent targets for modelling stellar structure and to test different physical phenomena such as the effect of angular transport induced by rotation \citep{2019ARA&A..57...35A, 2019A&A...626A.121O}.


Although a significant number of hybrid $\delta$ Sct-$\gamma$ Dor stars is currently known \citep{2010ApJ...713L.192G, 2014MNRAS.437.1476B}, the analysis of low frequencies in A-F stars still represents a challenge due to the different origins that these frequencies can have, e.g. spots, field stars contaminating the light apertures of the main target, a companion forming a non-eclipsing binary system, Rossby modes usually present in moderate to rapid rotating stars and more \citep{ 2019MNRAS.482.1757L, 2018Ap&SS.363..260C, 2018MNRAS.474.2774S}. Our aim in this paper is to present for the first time a complete observational analysis of the light curve and the frequencies of the hybrid $\delta$ Sct-$\gamma$ Dor  \cid\ along with the corresponding interpretation.

%\footnote{modes whose restoring force is the Coriolis force and are usually present in moderate to rapid rotating stars}
The paper is laid out as follows: both literature and \corot\ data are presented in Sect.~\ref{sec:corotdata}, followed by the description of the frequency analysis in Sec.~\ref{sec:frequencyanalisis}.  Detailed analysis of the frequencies including their mode identification is then presented and discussed in Sect.~\ref{sec:mode-analysis}.  An initial interpretation of the oscillation modes with stellar models is presented in Sect.~\ref{sec:interpretation}, and we then conclude in Sect.~\ref{conclusion}.

