\section{Summary and Conclusions}
\label{conclusion}
 
% It is important to mention that we also found two frequencies between these ranges, i.e. $F_{47}=5.0388$ c/d and $F_{76}=7.7282$ c/d indicating no clear distinction between both domains.
% Write about the relation between the low temperature (for a gamma Dor star and an A2 star) and the existence of spots.

In this work, we have presented a detailed analysis of the light curve of \cid\ and its frequencies.
This star exhibits a rich frequency spectrum, with characteristics typical of hybrid $\delta$ Sct-$\gamma$ Dor stars. Such objects offer a great opportunity to explore both the outer regions as well as their deep interior, due to the simultaneous presence of $p$ and $g$ modes. We performed an in-depth analysis of the frequency and variable content of the time series: 
%This star also appears to have spots, which allowed us to derive the precise value of the rotational frequency in the surface. 

-- We detected two separate frequency domains, corresponding to $\gamma$ Dor domain and $\delta$ Sct type oscillations.  We detected 26 pure frequencies in the $\gamma$-Dor range of [0.32,3.66] c/d, and 15 pure frequencies in the $\delta$-Sct range [9.38, 21.39] c/d (Fig. \ref{fourierTF1} and Tables~\ref{listatotal1} and \ref{listatotal2}).

-- In the $\gamma$ Dor domain, we found an asymptotic series of 6 equidistant periods with a mean separation of 1621s $\pm$ 20s (Fig.~\ref{asymp} and Table~\ref{asimp}) which most likely corresponds to $\ell=2$. 

-- In the $\delta$ Sct domain, we found a quintuplet centred in the highest amplitude frequency of this domain, $p_1$. The splitting in the frequencies of this quintuplet suggests that  $f_{rot}=0.32629$ c/d is a rotational frequency (Table~\ref{quintuplete}).
%allowed us to extract the star's rotational frequency, $f_{rot}=0.32629$ c/d, see Ta ble~\ref{quintuplete}.

-- The phase diagram corresponding to $f_{rot}$  %=0.32629$ 
(Fig \ref{diagramafaseprot}) along with the moving bumps and the amplitude variation from one orbit to another in Fig. \ref{curvaluz2} suggest the presence of spots in this hybrid star, in the case of $f_{rot}$ being a rotational frequency.

-- Another remarkable characteristic of this hybrid star is the presence of coupling between $p$ and $g$ modes in the $\delta$ Sct domain (Table~\ref{combpg}). This phenomenon, probably common among hybrid $\delta$ Sct-$\gamma$ Dor stars, should provide information about their internal structure and the resonant cavities in these kinds of stars. 
%A detailed modelling of this stars should be developed in order to address the study of this phenomena.

-- We developed a preliminary modelling for \cid\ by employing our frequency analysis along with the parameters derived in Sec. \ref{sec:extinction}, corrected for extinction. We obtained a mass and age of $1.75 M_{\odot}$ and $1241 \times 10^6$ yrs, without overshooting.   The model parameters are $L=11.36L_{\odot}$, $T_{\rm eff}=6726$ K, $R=2.48 R_{\odot}$ and mean period spacing $\Delta\Pi$ = 1624 s, which of course reproduce the derived parameters in Sec. \ref{sec:extinction} within their uncertainties.


Finally, we highlight the need to follow up this star with spectroscopic measurements in order to detect orbital radial velocities deviations from a possible companion or width-line variations over a rotational period from a line corresponding to surface activity in the case of \cid\ being a spotted star.
