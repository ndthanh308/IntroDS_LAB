\documentclass[12pt,usenames,dvipsnames]{amsart}

\usepackage{amsmath,amssymb,amsthm,array,bm,calrsfs,comment,enumitem,graphicx,hyperref,latexsym,multirow,tikz-cd}
%\newcommand\hmmax{0}
%\newcommand\bmmax{0}
\usepackage{mathpazo}
\DeclareMathAlphabet{\pazocal}{OMS}{zplm}{m}{n}
\usepackage{float}
%\usepackage{mathabx}

\setcounter{MaxMatrixCols}{12}

\usepackage[margin=1in]{geometry}
%\usepackage[backend=bibtex]{biblatex}
%\addbibresource{bibliographie.bib}

\usepackage[cmtip, all]{xy}

\newcommand{\Aut}{\mathsf{Aut}}
\newcommand{\codim}{\mathsf{codim}}
\newcommand{\Conv}{\mathsf{conv}}
\newcommand{\face}{\mathsf{face}}
\newcommand{\GL}{\mathsf{GL}}
\newcommand{\Hom}{\mathsf{Hom}}
\newcommand{\id}{\mathsf{id}}
\newcommand{\init}{\mathsf{in}}
\newcommand{\PGL}{\mathsf{PGL}}
\newcommand{\Proj}{\mathsf{Proj}}
\newcommand{\sat}{\mathsf{sat}}
\newcommand{\sch}{\mathsf{sch}}
\newcommand{\Set}{\mathsf{Set}}
\newcommand{\Sets}{\mathsf{Sets}}
\newcommand{\sgn}{\mathsf{sgn}}
\newcommand{\SL}{\mathsf{SL}}
\newcommand{\smgp}{\mathsf{smgp}}
\newcommand{\Span}{\mathsf{span}}
\newcommand{\Spec}{\mathsf{Spec}}
\newcommand{\Star}{\mathsf{Star}}
\newcommand{\Sym}{\mathsf{Sym}}
\newcommand{\Top}{\mathsf{top}}
\newcommand{\trop}{\mathsf{trop}}
\newcommand{\Trop}{\mathsf{Trop}}
\newcommand{\val}{\mathsf{val}}
\newcommand{\Verts}{\mathsf{vert}}
\newcommand{\red}{\mathsf{red}}

\newcommand{\Gr}{\mathsf{Gr}}
\newcommand{\TGr}{\mathsf{TGr}}

\newcommand{\field}[1]{\mathbb{#1}}
\newcommand{\A}{\mathbb{A}}
\newcommand{\C}{\mathbb{C}}
\newcommand{\F}{\mathbb{F}}
\newcommand{\G}{\mathbb{G}}
\newcommand{\N}{\mathbb{N}}
\renewcommand{\P}{\mathbb{P}}
\newcommand{\Q}{\mathbb{Q}}
\newcommand{\R}{\mathbb{R}}
\newcommand{\Z}{\mathbb{Z}}

\newcommand{\bd}{\mathbf{d}}

\newcommand{\bB}{\mathbf{B}}
\newcommand{\bG}{\mathbf{G}}
\newcommand{\bM}{\mathbf{M}}


\newcommand{\TS}{\text{TS}}

\newcommand{\cB}{\mathcal{B}}
\newcommand{\cC}{\mathcal{C}}
\newcommand{\cD}{\mathcal{D}}
\newcommand{\cE}{\mathcal{E}}
\newcommand{\cG}{\mathcal{G}}
\newcommand{\cH}{\mathcal{H}}
\newcommand{\cI}{\mathcal{I}}
\newcommand{\cK}{\mathcal{K}}
\newcommand{\cL}{\mathcal{L}}
\newcommand{\cM}{\mathcal{M}}
\newcommand{\cO}{\mathcal{O}}
\newcommand{\cP}{\mathcal{P}}
\newcommand{\cS}{\mathcal{S}}
\newcommand{\cX}{\mathcal{X}}
\newcommand{\cY}{\mathcal{Y}}

\newcommand{\fp}{\mathfrak{p}}
\newcommand{\fq}{\mathfrak{q}}
\newcommand{\fX}{\mathfrak{X}}
\newcommand{\fY}{\mathfrak{Y}}


\newcommand{\pB}{\pazocal{B}}
\newcommand{\pC}{\pazocal{C}}
\newcommand{\pL}{\pazocal{L}}
\newcommand{\pO}{\pazocal{O}}
\newcommand{\pQ}{\pazocal{Q}}
\newcommand{\pR}{\pazocal{R}}
\newcommand{\pU}{\pazocal{U}}
\newcommand{\pX}{\pazocal{X}}
\newcommand{\pY}{\pazocal{Y}}

\newcommand{\kk}{\mathbf{k}}

\newcommand{\sQ}{\mathsf{Q}}
\newcommand{\sU}{\mathsf{U}}

\newcommand{\sfp}{\mathsf{p}}
\newcommand{\sq}{\mathsf{q}}

\newcommand{\sa}{\mathsf{a}}
\newcommand{\su}{\mathsf{u}}
\newcommand{\sv}{\mathsf{v}}
\newcommand{\sw}{\mathsf{w}}

\newcommand{\set}[1]{\{#1\}}
\newcommand{\bk}[2]{\langle #1, #2 \rangle}

\newcommand{\bases}{\mathcal{B}}
\newcommand{\ETrop}{\mathfrak{Trop}}
\newcommand{\spMat}{\sQ_\mathsf{mk}}
\newcommand{\spCone}{\pC_\mathsf{mk}}
\newcommand{\spw}{\sw_\mathsf{mk}}
\newcommand{\sing}{\mathsf{sing}}
\newcommand{\gr}{}
%\newcommand{\gr}{\mathsf{gr}}
\newcommand{\pr}{\mathsf{pr}}
\newcommand{\Zone}{\langle \mathbf{1}\rangle}
\newcommand{\CCt}{\C(\!(t)\!)}
\newcommand{\conv}{\mathsf{conv}}
\newcommand{\symdiff}{\triangle}
\newcommand{\cl}{\mathsf{cl}}

\newcommand{\Sn}[1]{\mathfrak{S}_{#1}}
\newcommand{\chow}[2]{#1/\!\!\!/#2}

\newcommand{\pe}[3]{#1\, +_{#2} \, #3}

 
\DeclareMathOperator{\diag}{diag}
\DeclareMathOperator{\rank}{rank}
\DeclareMathOperator{\relint}{relint}



\newtheorem{theorem}{Theorem}[section]
\newtheorem{lemma}[theorem]{Lemma}
\newtheorem{proposition}[theorem]{Proposition}
\newtheorem{corollary}[theorem]{Corollary}
\newtheorem{conjecture}[theorem]{Conjecture}

\newtheorem*{theorem*}{Theorem}

\theoremstyle{definition}
\newtheorem{question}[theorem]{Question}
\newtheorem{observation}[theorem]{Observation}
\newtheorem{example}[theorem]{Example}



\theoremstyle{remark}
\newtheorem{remark}[theorem]{Remark}
\newtheorem{construction}[theorem]{Construction}


\newtheorem{definition}[theorem]{Definition}  

\numberwithin{equation}{section}
\numberwithin{table}{section}
\numberwithin{figure}{section}

\newcommand\blfootnote[1]{%
	\begingroup
	\renewcommand\thefootnote{}\footnote{#1}%
	\addtocounter{footnote}{-1}%
	\endgroup
}

\newcommand{\dan}[1]{{\color{Purple} \sf Dan: [#1]}}

\newcommand{\dante}[1]{{\color{Green} \sf Dante: [#1]}}



\title{Singular matroid realization spaces}



\author{Daniel Corey, Dante Luber}
%\author{Dante Luber}


\begin{document}

	\maketitle

 

\begin{abstract}
    We study smoothness of realization spaces of matroids for small rank and ground set. For $\C$-realizable matroids, when the rank is  $3$, we prove that the realization spaces are all smooth when the ground set has  $11$ or fewer elements, and there are singular realization spaces for $12$ and greater elements.  For rank $4$ and $9$ or fewer elements, we prove that these realization spaces are smooth. As an application, we prove that $\Gr^{\circ}(3,n;\C)$---the locus of the Grassmannian where all Pl\"ucker coordinates are nonzero---is not sch\"on for $n\geq 12$. 
    \medskip
    
    \noindent \textbf{Keywords}: Grassmannian, matroid, singularities.
    
    \medskip
    
    \noindent \textbf{2020 MSC}: 14B05 (primary) 05E14, 14T90, 52B40 (secondary)
\end{abstract}



\section{Introduction}\label{sec:intro}

Matroid realization spaces---algebraic varieties, or more generally schemes, that parameterize hyperplane arrangements realizing a fixed matroid---provide a rich source of singular spaces in algebraic geometry, at least at an abstract level. By Mn\"ev's universality theorem \cite{Mnev85, Mnev88}, for each singularity type (a notion made precise by Vakil in \cite{Vakil}) there is a rank $3$ matroid $\sQ$ such that this singularity type appears in the realization space of $\sQ$. This theorem serves as a template proving that many moduli spaces satisfy \textit{Murphy's law}, i.e., that every singularity type appears on that moduli space.

While the proofs of Mn\"{e}v's universality theorem are constructive (see \cite[Theorem~5.3]{Cartwright},  \cite[I.14]{Lafforgue2003}, or \cite{VakilLee}) the size of the ground set becomes large even for the simplest singularities. Thus, the primary goal of this paper is to find the smallest $n$ such that there is a rank $3$ matroid on $n$ elements whose realization space is singular over $\C$. 

In earlier work \cite{CoreyGrassmannians,CoreyLuber}, the authors prove that the realization spaces of $\C$--realizable rank $3$ matroids on $8$ or fewer elements are all smooth (which is closely related to earlier connectivity results in \cite{NazirYoshinaga}).  Furthermore, they are irreducible except for the realization space of the M\"{o}bius-Kantor matroid.  This is the matroid associated to the unique arrangement of 8 points and 8 lines in $\P^{2}$ such that each point lies on exactly 3 lines, and each line contains exactly 3 points. Our first two results show that the answer to the problem posed in the previous paragraph is $n=12$. 

\begin{theorem}
\label{thm:intro-3-leq11-smooth}
    The realization spaces of $\C$-realizable rank $3$ matroids on 11 or fewer elements are smooth over $\C$. 
\end{theorem}


\begin{theorem}
\label{thm:intro-3-geq12-singular}
    For $n\geq 12$, there exists a rank $3$-matroid on $n$ elements whose realization space has nodal singularities over $\C$.     
\end{theorem}

\noindent We also initiate a systematic study of smoothness and irreducibility for rank $4$ matroids. 

\begin{theorem}\label{thm:intro-4-n-matroids-smooth}
    For $n\leq 9$, the realization spaces of $\C$-realizable rank $4$ matroids on $n$ elements are smooth.
\end{theorem}

To prove Theorems \ref{thm:intro-3-leq11-smooth} and \ref{thm:intro-4-n-matroids-smooth}, we use a combination of theoretical results and computer computations in \texttt{OSCAR}. We prove structural results that allow us to deduce smoothness of realization spaces for some matroids by knowing smoothness of realization spaces of matroids on smaller ground sets. For example, the singularity types in a realization space are preserved under \textit{principal extension}, i.e., the act of freely adding an element to a flat of a matroid. See Proposition \ref{prop:principalExtension}, as well as  Proposition \ref{prop:2planes} for a stronger criterion. We use in a crucial way the database of matroids developed by \cite{MatsumotoMoriyamaImaiBremner}. There are $500\, 957$ (possibly nonrealizable)  matroids covered by these two theorems, and using our structural results, we need only verify smoothness directly for $50\,945$ of them, about $10\%$ of the total.


Our main application of Theorem \ref{thm:intro-3-geq12-singular} is to find singular initial degenerations of the Grassmannian. Recall that the Grassmannian $\Gr(d,n;\C)$ is the algebraic variety that parameterizes $d$-dimensional linear subspaces of $\C^{n}$. Via the Pl\"ucker embedding, $\Gr(d,n;\C)$ is realized as a closed subvariety of $\P^{\binom{[n]}{d}-1}$, and its homogeneous coordinates are called \textit{Pl\"ucker coordinates}. The open subvariety $\Gr^{\circ}(d,n;\C)$ defined by the nonvanishing of all Pl\"ucker coordinates is a widely studied object in tropical geometry and moduli theory. Its tropicalization $\TGr^{\circ}(d,n;\C)$ is closely related to phylogenetic trees (for $n=2$), valuated matroids, and toric degenerations \cite{SpeyerSturmfels2004a}.  The space $\Gr^{\circ}(d,n;\C)$ is also connected, via the Gelfand-MacPherson correspondence, to the moduli space $\pR(d,n)$ of $n$ hyperplanes in $\P_{\C}^{d-1}$ in linear general position up to the action of $\PGL_{d}(\C)$ (alternately, $\pR(d,n)$ is the realization space of the uniform matroid).  

In \cite{CoreyGrassmannians,CoreyLuber}, the authors prove that the normalization of the Chow quotient of $\Gr(3,n;\C)$ by the diagonal torus of $\PGL_{n}(\C)$ is the log canonical model of $\pR(3,n)$ for $n=6,7,8$ (the case $n=6$ was proved in \cite{Luxton}), which fully resolves \cite[Conjecture~1.6]{KeelTevelev2006}. The key step in the proof is to show that $\Gr^{\circ}(3,n;\C)$ is sch\"on for $n\leq 8$. (A closed subvariety of an algebraic torus is \textit{sch\"on} if its initial degenerations are all smooth; we review these notions in \S\ref{sec:initial_degenerations}).  Proving that a variety is sch\"on is, in general, a challenging task. The Grassmannian case relies heavily on a result of the first author \cite[Theorem~1.1]{CoreyGrassmannians} which relates the initial degenerations of $\Gr^{\circ}(3,n;\C)$ to matroid strata of the Grassmannian. Because of this connection and Mn\"ev's universality theorem, one may suspect that $\Gr^{\circ}(3,n;\C)$ is not sch\"on for large $n$. Using Theorem \ref{thm:intro-3-geq12-singular}, we confirm this suspicion in the following theorem. 

\begin{theorem}
\label{thm:not-shon-geq-12}
    The open Grassmannian $\Gr^{\circ}(3,n;\C)$ and moduli space $\pR(3,n)$ are not sch\"on for $n\geq 12$. 
\end{theorem}

Since the initial degenerations of $\Gr^{\circ}(d,n;\C)$ and $\pR(d,n)$ differ by a torus factor, sch\"oness of $\Gr^{\circ}(d,n;\C)$ is equivalent to that of $\pR(d,n)$. This leaves open a small range for $d=3$ where we do not know if $\Gr^{\circ}(d,n;\C)$ or $\pR(d,n)$ is sch\"on.


\begin{question}\label{question:3-n range}
    Is $\Gr^{\circ}(3,n;\C)$ sch\"on for $n=9$, $10$, or $11$?
\end{question}

\noindent Resolving the above cases in the affirmative could be challenging, as the techniques used in prior work and this paper are already known to fail for $n=9$, see \cite[Example 8.2]{CoreyGrassmannians}.



\subsection*{Code}\label{code} We use \texttt{OSCAR} \cite{OSCAR-book,OSCAR} which runs using \texttt{julia} \cite{BezansonEdelmanKarpinskiShah}. Specifically, \texttt{OSCAR} is essential to the proofs of Propositions \ref{prop:3-9smooth}, \ref{prop:3-10-11smooth}, \ref{prop:4-8-smooth}, and \ref{prop:4-9-smooth}. We include supplementary computations useful, although not strictly necessary, for the proofs of Theorem \ref{thm:singular3-12} and Proposition \ref{prop:singularIsom}. The code can be found at the following github repository:

\begin{center}
    \url{https://github.com/dcorey2814/matroidRealizationSpaces}
\end{center}


\subsection*{Notation and conventions}
Given a set $E$, denote by $\binom{E}{d}$ the set of size $d$ subsets of $E$.  We write $[n] =\{1,\dots,n\}$. When $n<10$, we express subsets subsets of $[n]$ by juxtaposing their elements, e.g., we write $ijk$ in place of $\{i,j,k\}$,  $A\cup x$ in place of $A\cup \{x\}$, and $A\setminus x$ in place of $A\setminus \{x\}$. Finite tuples appear frequently when forming minors of a matrix. Let  $\lambda$ and $\mu$ be two tuples. We write $\lambda \setminus \mu$ for the tuple obtained by removing the elements of $\mu\cap \lambda$ from $\lambda$ while preserving the order and $\lambda\cup \mu$ for the tuple obtained by appending $\mu$ to the end of $\lambda$.   

\subsection*{Acknowledgements}
We thank Antony Della Vecchia, Michael Joswig, Lars Kastner, Lukas K\"{u}hne, Benjamin Lorenz, and Marcel Wack for helpful discussions. DC is supported by the SFB-TRR project ``Symbolic Tools in Mathematics and their Application'' (project-ID 286237555), and DL is supported by "Facets of Complexity” (GRK 2434, project-ID 385256563). 



\section{Background}

In this section we review preliminary material on the Grassmannian, matroid strata, and realization spaces. In particular, for a given matroid, we demonstrate how to compute the coordinate rings of the associated strata in the Grassmannian, as well as its realization space.

\subsection{Grassmannian}\label{sec:grassmannian}
Let $\F$ be a field. The Grassmannian $\Gr(d,n;\F)$ parameterizes the $d$-dimensional subspaces of an $n$-dimensional $\F$-vector space.  We may represent $\Gr(d,n;\F)$ as a $d(n-d)$--dimensional projective variety via the Pl\"ucker embedding, defined as follows. Let $V\subset \F^n$ be a $d$-dimensional subspace and $A$ a full-rank $d\times n$ matrix whose row span is $V$. Given a sequence $\lambda = (i_1,\ldots,i_d)$, denote by $A_{\lambda}$ the determinant of the $d\times d$ matrix formed by the columns of $A$ indexed by $i_1,\ldots,i_d$ in that order. The $\lambda$-th Pl\"ucker coordinate of $V$ is $p_{\lambda}(V) = A_{\lambda}$; the collection of Pl\"ucker coordinates are well defined up to simultaneous scaling by an element of $\F^{*}$. The \emph{Pl\"ucker embedding} is given by 
\begin{equation*}
    \Gr(d,n;\F)\hookrightarrow \P^{{[n]\choose d}-1}_{\F} \hspace{15pt} V \mapsto [p_{\lambda}(V):\lambda \in \textstyle{\binom{[n]}{d}}].
\end{equation*}



\subsection{Matroids and their strata}
Fix nonnegative integers $d\leq n$. A \textit{matroid} $\sQ$ is a pair $(E,\pB)$ where $E$ is a $n$-element set and $\pB$ is a nonempty subset of $\binom{E}{d}$ satisfying the \textit{basis exchange axiom}: For each pair of distinct elements $\lambda_1,\lambda_2$ of $\pB$ and $b_1\in \lambda_1\setminus \lambda_2$, there exists $b_2\in \lambda_2\setminus \lambda_1$ such that $(\lambda_1 \setminus b_1)\cup b_2$ lies in $\pB$. An element of $\pB$ is called a \textit{basis} of $\sQ$; for emphasis, we write $\pB(\sQ)$ for the set of bases of a matroid. The integer $d$ is called the \textit{rank} of $\sQ$ and the set $E$ is called the \textit{ground set} of $\sQ$. A \textit{$(d,E)$--matroid} is a rank $d$ matroid with ground set $E$; when $E=[n]:=\{1,\ldots,n\}$, we simply call this a  \textit{$(d,n)$--matroid}.

Let $\sQ$ be a $(d,E)$--matroid. An element $\ell\in E$ is called a \textit{loop} if $\ell\notin \lambda$ for all $\lambda \in \pB(\sQ)$. Two distinct elements $a,b\in E$ are \textit{parallel} if $a$ and $b$ are not loops and $ab\not\subset \lambda$ for all $\lambda \in \pB(\sQ)$. A matroid is \textit{simple} if it has no loops or parallel elements. 

Let $V\subset \F^{n}$ be a $d$-dimensional linear subspace. Its matroid is
\begin{equation*}
    \sQ(V) = \{\lambda \in \textstyle{\binom{[n]}{d}} \, : \, Codep_{\lambda}(V) \neq 0\}.
\end{equation*}
A $(d,E)$--matroid is $\F$-realizable if there exists a $d$-dimensional linear subspace $V\subset \F^{n}$ such that $\sQ \cong \sQ(V)$. Such a $V$ is called a \textit{realization} of $\sQ$.  The \textit{matroid stratum} of a realizable $(d,n)$--matroid is, as a set,
\begin{equation*}
\Gr(\sQ; \F) := \{V\in\Gr(d,n;\F) \, | \,  p_{\lambda}(V) \neq 0 \text{ if and only if } \lambda\in \bases(\sQ)\}.   
\end{equation*}
Therefore, the Grassmannian $\Gr(d,n;\F)$ admits a decomposition into subvarieties indexed by $(d,n)$--matroids (although this is not a \textit{Whitney stratification}, see \cite[\S 5]{GelfandGoreskyMacPhersonSerganova}). 

\subsection{The coordinate ring of a matroid stratum}\label{sec:coordrings_tsc}
In this section, we describe $\Gr(\sQ;\F)$ as an affine scheme, more precisely, we review the presentation for the coordinate ring of $\Gr(\sQ;\F)$ in terms of matrix coordinates. Let $\sQ$ be a $\F$-realizable $(d,n)$--matroid, and fix a \textit{reference basis} $\lambda_{0} \in \pB(\sQ)$.  To streamline the exposition, we assume that $\lambda_{0} = \{1,\ldots,d\}$. Define the polynomial ring 
\begin{equation*}
    B^{\gr} =  \F[x_{ij} \, : \, 1\leq i \leq d,\, 1\leq j \leq n-d].
\end{equation*}
and the matrix of variables
\begin{equation*}
A^{\gr} = \begin{bmatrix}
    1 &0 &\dots &0 &x_{11} &\dots &x_{1,n-d}\\
    0 &1 &\dots &0 &x_{21} &\dots &x_{2,n-d}\\
    \vdots &\vdots &\dots &\vdots &\vdots &\dots &\vdots\\
    0 &0 &\dots &1 &x_{d1} &\dots &x_{d,n-d}\\
\end{bmatrix}
\end{equation*}
Recall from \S\ref{sec:grassmannian} that, given a $d\times n$ matrix $A$ and a tuple $\lambda = (i_1,\ldots,i_d)$ of distinct elements of $[n]$, we write $A_{\lambda}$ for the determinant of the $d\times d$ submatrix formed by the columns of $A$ indexed by $\lambda$ in order. If $\lambda$ is given as a set, $A_{\lambda}$ is computed by taking the columns in increasing order. If $|\lambda| \neq d$, or if $A_{\lambda}$ has a repeated entry, then $A_{\lambda} = 0$. The variable $x_{ij}$ can be expressed as a maximal minor:
\begin{equation*}
    x_{ij} = (-1)^{d+i} A_{[d]\setminus i \cup (d+j)}
\end{equation*}
Define the polynomial ring
\begin{equation*}
B_{\sQ} = \F[x_{ij}\, : \, ([d]\setminus i)\cup (d+j) \in \bases(\sQ)]
\end{equation*}
and the surjective ring homomorphism
\begin{equation}
\label{eq:piQ}
    \pi_{\sQ}^{\gr}:B \to B_{\sQ}, \hspace{10pt} \pi_{\sQ}(x_{ij}) = 
    \begin{cases}
    x_{ij} & \text{ if } ([d]\setminus i)\cup (d+j) \in \bases(\sQ), \\
    0 & \text{ otherwise.} 
    \end{cases}   
%    \mapsto \to B/\langle x_{ij} : [d]\triangle \{i,d+j\}\notin\bases(\sQ)\rangle \cong B_{\sQ}.
\end{equation}
Define the ideal $I_{\sQ}^{\gr}$, multiplicative semigroup $U_{\sQ}^{\gr}$, and ring $S_{\sQ}$
\begin{equation*}
    I_{\sQ}^{\gr} = \langle \pi_{\sQ}^{\gr}(A_\lambda^{\gr}): \lambda \notin\bases(\sQ)\rangle, \hspace{10pt} 
    U_{\sQ}^{\gr} = \langle \pi_{\sQ}^{\gr}(A_{\lambda}^{\gr}):\lambda\in \bases(\sQ)\rangle_{\smgp}, \hspace{10pt} S_{\sQ} = U_{\sQ}^{-1} B_{\sQ} / I_{\sQ}. 
\end{equation*}
Thus, $\Gr(\sQ;\F)$ is the affine scheme $\Spec(S_{\sQ})$. When $\F$ is algebraically closed, the closed points of $\Gr(\sQ;\F)$ correspond to the realizations of $\sQ$ as defined in the previous section. 


\subsection{Matroid realization spaces and their coordinate rings}
%\subsection{Coordinate rings of realization spaces}
\label{sec:coordrings_realization_spaces}
A \textit{hyperplane arrangement} is a sequence of $n$ hyperplanes in $\P^{d-1}$. Given a hyperplane arrangement  $(h_1,\ldots,h_n)$, its matroid $\sQ$ has ground set $[n]$ and bases
\begin{equation*}
\pB(\sQ) = \{ \lambda \in \textstyle{\binom{[n]}{d}} \, : \, \bigcap_{i\in \lambda} h_i = \emptyset \}.
\end{equation*}
A loopless matroid is the matroid of a hyperplane arrangement in $\P^{d-1}_{\F}$ if and only if it is $\F$-realizable. 

Let $\sQ$ be a $\F$-realizable, loopless, $(d,n)$--matroid. The \textit{realization space}  of $\sQ$ is a locally-closed subscheme 
\begin{equation*}
    \pR(\sQ;\F) \subset \left(\prod_{i=1}^{n}(\P_{\F}^{d-1})^{\vee} \right)/ \PGL_{d}(\F) 
\end{equation*}
that parameterizes equivalence classes of hyperplane arrangements; here, two hyperplane arrangements $(g_1,\ldots,g_n)$ and $(h_1,\ldots,h_n)$ are equivalent if there is a $\phi \in \PGL_{d}(\F)$ such that $\phi(g_i) = h_i$ for $i=1,\ldots,n$.

Denote by  $T\subset \PGL_{n}(\F)$ the diagonal subtorus. The torus $T \cong (\F^*)^{n}/\F^*$ acts on linear subspaces of $\F^{n}$ by scaling coordinates, and this induces an action of $T$ on $\Gr(d,n;\F)$, and on each matroid stratum $\Gr(\sQ;\F)$. If $\sQ$ is \textit{connected}, i.e., $\sQ$ cannot be expressed as a nontrivial direct sum, then the action $T \curvearrowright \Gr(\sQ;\F)$ is free. By the \textit{Gelfand-MacPherson correspondence}, we have an isomorphism
\begin{equation*}
    \pR(\sQ;\F) \cong \Gr(\sQ;\F)/T.
\end{equation*}
In particular, $\Gr(\sQ;\F)$ and $\pR(\sQ;\F)$ have the same singularity types and connected components. 
The scheme $ \pR(\sQ;\F)$ is affine, and the general procedure to obtain its coordinate ring is similar to that of a matroid stratum with a few subtle differences that we describe now.

We assume that $\sQ$ is a simple, connected, $\F$-realizable $(d,n)$--matroid that has a circuit of size $d+1$. Note that such a circuit need not exist, see \S \ref{sec:4n} for some examples. Nevertheless, this hypothesis greatly simplifies the following construction. So as to not overburden the notation with extra decorations, we use symbols similar to those for the coordinate ring of a matroid stratum. In later sections, we hope that the context clarifies the usage. 


Let $\gamma_{0}$ be a $d+1$-element circuit of $\sQ$ which we call our \textit{reference circuit}. After applying a suitable permutation, we may assume that $\gamma_0 = \{1,2,\ldots,d+1\}$. Since $\gamma_0$ is a circuit, the subset $\lambda_0 = \{1,2,\ldots,d\}$ is a basis of $\sQ$. Define 
\begin{equation*}
    B = \F[x_{ij} \, : \, 1 \leq i \leq d, \, 1\leq j \leq n-d-1]
\end{equation*}
and the matrix of variables
\begin{equation*}
A = \begin{bmatrix}
    1 &0 &\dots &0 & 1 &x_{11} &\dots &x_{1,n-d-1}\\
    0 &1 &\dots &0 & 1 &x_{21} &\dots &x_{2,n-d-1}\\
    \vdots &\vdots & \vdots &\dots &\vdots &\vdots &\dots &\vdots\\
    0 &0 &\dots &1 &1 &x_{d1} &\dots &x_{d,n-d-1}\\
\end{bmatrix}
\end{equation*}
Define the function $\mu:[n-d-1] \to [d]$, which depends on $\sQ$ and $\lambda_0$, by
\begin{equation*}
    \mu(j) = \max(i\in [d] \, : \, ([d] \setminus i)\cup (d+j+1) \in \pB(\sQ)).
\end{equation*}
Define the polynomial ring
\begin{equation*}
B_{\sQ} = \F[x_{ij}\, : \, ([d] \setminus i)\cup (d+j+1) \in \bases(\sQ) \text{ and } i\neq \mu(j)]
\end{equation*}
and the surjective ring homomorphism $\pi_{\sQ}:B \to B_{\sQ}$ by
\begin{equation*}
    \pi_{\sQ}(x_{ij}) = 
    \begin{cases}
    x_{ij} & \text{ if } ([d]\setminus i) \cup (d+j+1)\in \bases(\sQ) \text{ and } i\neq \mu(j), \\
    1 & \text{ if } i = \mu(j), \\
    0 & \text{ otherwise.} 
    \end{cases}   
\end{equation*}
Define the ideal $I_{\sQ}$ and multiplicative semigroup $S_{\sQ}$ by
\begin{equation*}
    I_{\sQ} = \langle \pi_{\sQ}( A_\lambda): \lambda \notin\bases(\sQ)\rangle, \hspace{20pt} 
    U_{\sQ} = \langle \pi_{\sQ}(A_{\lambda}):\lambda\in \bases(\sQ)\rangle_{\smgp}
\end{equation*}
The coordinate ring of the affine scheme $\pR(\sQ;\F)$ is
\begin{equation*}
\F\left[\pR(\sQ;\F)\right]\cong U_{\sQ}^{-1}B_{\sQ}/I_{\sQ}.
\end{equation*}

\begin{example}
\label{ex:3-9}
We demonstrate the procedure in \ref{sec:coordrings_realization_spaces} to compute the coordinate ring of the $(3,9)$-matroid $\sQ$, where the nonbases of $\sQ$ are given by the colinearities in Figure \ref{fig:3_9_example}. We then show that $\pR(\sQ,\C)$ is smooth.

% Figure environment removed
The matrix of indeterminants below realizes $\sQ$, provided that the minors indexed by nonbases are zero, and those indexed by bases are nonzero. 

\begin{equation*}
A = \begin{bmatrix}
    1   &0   &0   &1   &x_1   &x_2    &0   &x_5   &x_7\\
    0   &1   &0   &1    &1   &x_3   &x_4   &x_6    &0\\
    0   &0   &1   &1   & 0    &1    &1    &1    &1
\end{bmatrix}
\end{equation*}

Hence, we have $\pR(\sQ,\C)\cong \Spec\left(U^{-1}_{\sQ}B_{\sQ}/I_{\sQ}\right)$ where $B_{\sQ} = \C[x_1,\ldots,x_7]$,  $I_{\sQ}$ is the ideal
   \begin{equation*}
   I_{\sQ} = \left\langle \begin{array}{c}
       x_1 - 1, \, 
       x_2 - 1, \,
       x_3 - x_7 - 1, \, 
       x_4 - 1, \,
       x_5 - x_7, \,
       x_6 - x_7 - 1, \, 
       x^2_{7} + 1
   \end{array}\right\rangle
   \end{equation*}
   and $U_{\sQ}$ is the multiplicative semigroup generated by 
   {\footnotesize
   \begin{gather*}
x_1,
x_2,
x_3,
x_4,
x_5,
x_6,
x_7,
x_3 - 1,
x_5 - 1,
x_6 - 1,
x_7 - 1, 
x_2 - x_3,
x_2 - x_5,
x_2 - x_7, 
x_3 - x_4, 
x_4 - x_6, 
x_5 - x_6, \\
x_1 + x_7 - 1,
x_1x_3 - x_1 - x_2 + 1,
x_1x_3 - x_1x_6 - x_2 + x_5,
x_1x_3 - x_2, 
x_1x_3 - x_2 + x_7,
x_1x_3 - x_1x_4 - x_2, \\
x_1x_4 - x_1 + 1,
x_1x_4 + x_7,
x_1x_6 - x_1 - x_5 + 1,
x_1x_6 - x_5, 
x_1x_6 - x_5 + x_7,
x_2 + x_3x_7 - x_3 - x_7, \\
x_2x_4 - x_2 + x_3 - x_4,
x_2x_4 - x_2x_6 + x_3x_5 - x_4x_5, 
x_2x_6 - x_2 - x_3x_5 + x_3 + x_5 - x_6,
x_2x_6 - x_3x_5, \\
x_2x_6 - x_3x_5 + x_3x_7 - x_6x_7, 
x_4x_5 - x_4 - x_5 + x_6, 
x_4x_5 - x_4x_7 + x_6x_7,
x_4x_7 - x_4 - x_7,
x_5 + x_6x_7 - x_6 - x_7.
   \end{gather*}
   }
Next we apply the ring homomorphism $\phi:B_{\sQ}\to \C[x]$ given by
\begin{equation*}
\begin{array}{ccccccc}
     x_1\mapsto 1 \, &x_2\mapsto 1  \, &x_3\mapsto x+1 &x_4\mapsto 1 \, &x_5\mapsto x \, &x_6\mapsto x+1  &x_7\mapsto x
\end{array}
\end{equation*}
With relations from $I_{\sQ}$ and $U_{\sQ}$, the map $\phi$ induces an isomorphism 
\begin{equation*}
U^{-1}_{\sQ}B_{\sQ}/I_{\sQ} \cong U^{-1}\C[x]/\langle x^2+1\rangle \text{ where } U = \langle x+1,x-1,x \rangle_{\smgp}.
\end{equation*} 
Since $x+1,x-1,x$ are units in $\C[x]/\langle x^2+1\rangle$, we have $\pR(\sQ,\C)\cong \Spec(\C[x]/\langle x^2+1\rangle)$. The coordinate ring of $\pR(\sQ;\C)$ is isomorphic to $\C \times \C$, and so $\pR(\sQ;\C)$ is a reduced 0-dimensional scheme consisting of 2 closed points. Thus $\pR(\sQ;\C)$ is smooth. 

The homomorphism $\phi$ and induced isomorphism mirror \cite[Algorithm~6.12]{CoreyLuber}, a technique we use in later examples and throughout the computational aspects of our proofs. 
\end{example}


\section{Matroid strata and basic operations}

We are primarily interested in determining which matroids have smooth realization spaces. In this section, we describe how matroid strata behave under basic matroid operations, including direct sum, duality, and principal extension.  Realization spaces for matroids constructed via these operations can be studied by applying knowledge about matroids of smaller rank, or on a smaller ground set. The first two propositions are well-established, see \cite[Proposition~9.4]{KatzMatroid}.

Given two matroids $\sQ_1$ and $\sQ_2$ on the ground sets $E_1$ and $E_2$ respectively, the \textit{direct sum} of $\sQ_1$ and $\sQ_{2}$, denoted $\sQ_{1}\oplus \sQ_{2}$, is the matroid on $E_1\sqcup E_2$ with bases
\begin{equation*}
    \pB(\sQ_{1} \oplus \sQ_{2}) = \{ \lambda_1\cup \lambda_2 \, : \, \lambda_1\in \pB(\sQ_1), \lambda_2\in \pB(\sQ_2) \}.
\end{equation*}
If $V_1$ and $V_2$ are realizations of $\sQ_1$ and $\sQ_2$ respectively, then $V_1\oplus V_2$ is a realization of $\sQ_1\oplus \sQ_2$. More generally, we have the following proposition. 
\begin{proposition}
\label{prop:directsum}
    If $\sQ_1$ and $\sQ_2$ are $\F$-realizable, then $\sQ_1 \oplus \sQ_2$ is $\F$-realizable and
    \begin{equation*}
        \Gr(\sQ_1 \oplus \sQ_2; \F) \cong \Gr(\sQ_1;\F) \oplus \Gr(\sQ_2;\F).
    \end{equation*}
\end{proposition}

\noindent Given a $(d,E)$--matriod $\sQ$, its \textit{dual} is the $(|E|-d,E)$ matroid $\sQ^{\vee}$ whose bases are
\begin{equation*}
    \pB(\sQ^{\vee}) = \{E\setminus \lambda \, : \, \lambda \in \pB(\sQ)\}.
\end{equation*}
If $V\subset \C^{n}$ is a realization of $\sQ$, then the orthogonal complement $V^{\perp}\subset (\C^{n})^{\vee} \cong \C^{n}$ is a realization of $\sQ^{\vee}$, where $V^{\perp} = \{\sv\in (\C^{n})^{\vee} \, : \, \bk{\su}{\sv} = 0 \text{ for all } \su\in V \}$. More generally, we have the following. 
\begin{proposition}
\label{prop:duality}
    If $\sQ$ is $\F$-realizable, then so is $\sQ^{\vee}$ and
    \begin{equation*}
        \Gr(\sQ^{\vee};\F) \cong \Gr(\sQ;\F).
    \end{equation*}
\end{proposition}

If $\sQ$ is a $\F$-realizable matroid and $\sQ' = \sQ\setminus a$, then the inclusion $B_{\sQ'}\subset B_{\sQ}$ induces a morphism $\varphi_{\sQ,\sQ'}:\Gr(\sQ;\F) \to \Gr(\sQ';\F)$. This is a special case of the face morphism in \cite[Proposition I.6]{Lafforgue2003} and \cite[Proposition~3.1]{CoreyGrassmannians}, which we recall in \S \ref{sec:inverse_limits_tight_spans} (more precisely, this corresponds to a face contained in the boundary of the hypersimplex defined by $x_a=0$). This morphism need not be smooth, see \cite[Example~8.4]{CoreyGrassmannians} and Example \ref{ex:morphismNotSmooth}. A key goal is to isolate conditions where this morphism is smooth and surjective with connected fibers. One broad class of examples comes from principal extensions of matroids, as we now describe. 


Let $\sQ'$ be a $(d,E')$--matroid, and let $\eta'$ be a flat of $\sQ'$. The \textit{principal extension} of $\sQ'$ by $a$ into $\eta'$ is the $(d,E)$--matroid $\sQ = \pe{\sQ'}{\eta'}{a}$ where $E = E'\sqcup a$ and 
%\dante{notation of Q is kind of confusing}
\begin{equation*}
    \pB(\sQ) = \pB(\sQ') \cup \{(\lambda' \setminus b )\cup a \, : \, \lambda'\in \pB(\sQ') \text{ and } b\in \lambda'\cap \eta'\}.
\end{equation*}
If $\F$ is an infinite field and $\sQ'$ is $\F$-realizable, then $\sQ$ is also $\F$-realizable \cite[Lemma~2.1]{MayhewNewmanWhittle}. In the next proposition, we prove that principal extension preserves smoothness. Here is the intuitive idea. All projective realizations of $\sQ$  are obtained by freely placing, in a projective realization of $\sQ'$, a point on the linear subspace spanned by the flat $\eta'$ outside of finitely many hypersurfaces, and hence $\Gr(\sQ;\C)$ is an open subscheme of $\Gr(\sQ';\C) \times (\G_{m})^k$. 

\begin{proposition}
\label{prop:principalExtension}
%Fix an algebraically closed field $\F$.
Let $\sQ, \sQ'$ be rank-$d$, $\F$-realizable matroids such that $\sQ = \pe{\sQ'}{\eta'}{a}$ where $\eta'$ is a rank-$k$ flat of $\sQ'$. Then we have a commutative diagram
\begin{equation*}
    \begin{tikzcd}
 \Gr(\sQ;\C)\arrow[dr, "\varphi_{\sQ,\sQ'}"'] \arrow[r, hook, "\iota"] &\Gr(\sQ';\C) \times (\G_{m})^k \arrow[d, "\pr_1"] 
\\
& \Gr(\sQ';\C) 
\end{tikzcd}
\end{equation*}
such that $\iota$ is an open affine immersion, $\pr_1$ is the projection onto the first factor, and $\varphi_{\sQ,\sQ'}$ is smooth and surjective (on closed points) with connected fibers. 
\end{proposition}



\begin{lemma}
\label{lem:expand}
Define the $d\times n$ matrix of variables
    \begin{equation}
    \label{eq:A}
A = \begin{bmatrix}
    1 &0 &\cdots &0 &x_{11} &\cdots &x_{1,n-d-1} & y_1\\
    0 &1 &\cdots &0 &x_{21} &\cdots &x_{2,n-d-1} & y_2 \\
    \cdots &\cdots &\cdots &\cdots &\cdots &\cdots &\cdots & \cdots\\
    0 &0 &\cdots &1 &x_{d1} &\cdots &x_{d,n-d-1} & y_d
    \end{bmatrix}
    \end{equation}
Given a strictly increasing sequence $\lambda' = (i_1,\ldots,i_{d-1})$ with $i_{d-1}<n$, we have 
\begin{equation*}
    A_{\lambda'\cup n} =  (-1)^{d+1}\sum_{b=1}^{d} A_{b\cup \lambda'}\, y_{b}
\end{equation*}
\end{lemma}
The proof of this lemma follows from computing $A_{\lambda\cup n}$ by a cofactor expansion along the column indexed by $n$. 

\begin{proof}[Proof of Proposition \ref{prop:principalExtension}]
After applying a suitable isomorphism, we may assume that $E=[n]$, $a=n$, $[d]$ is a basis of $\sQ'$, and $[k] = \eta \cap [d]$.  Let $A$ be the matrix from Formula \eqref{eq:A}. Throughout, let $A_{\lambda}(\sQ) = \pi_{\sQ}(A_{\lambda})$ where $\pi_{\sQ}$ is the ring map from Formula \eqref{eq:piQ}.


\medskip


\noindent \textbf{Claim:} If $\lambda' \in  \binom{[n-1]}{d}$ such that $\lambda' \cup n \notin \pB(\sQ)$, then $A_{\lambda' \cup n}(\sQ) \in I_{\sQ'} \cdot B_{\sQ}$. 

\medskip

\noindent Suppose $\lambda'$ and $n$ are as in the claim.  By Lemma \ref{lem:expand},
\begin{equation*}
    A_{\lambda' \cup n}(\sQ) = (-1)^{d+1}\sum_{b\in [k]\setminus \lambda'}  A_{b \cup \lambda'}(\sQ) \, y_{b}.
\end{equation*}
Because $\lambda' \cup n$ is not a basis of $\sQ$, the set $b \cup \lambda'$ (for $b \in [k]\setminus \lambda'$) is not a basis of $\sQ'$, and therefore it is not a basis of $\sQ$. So $A_{\lambda'\cup n}(\sQ) \in I_{\sQ'} \cdot B_{\sQ}$, as required. 

As a consequence of this claim, the ideal $I_{\sQ}$ is the extension of $I_{\sQ'} \subset B_{\sQ'}$ to $B_{\sQ}$. So  $ U_{\sQ}^{-1}(S_{\sQ'}\otimes \C[y_1^{\pm}, \ldots, y_{k}^{\pm}]) \to S_{\sQ}$ is an isomorphism, and hence we have an open immersion $\iota:\Gr(\sQ;\C) \hookrightarrow \Gr(\sQ';\C)\times (\G_m)^{k}$. This implies that $\varphi_{\sQ,\sQ'}$ is smooth and its nonempty fibers are connected. 

Finally, we must show that $\varphi_{\sQ,\sQ'}$ is surjective on closed points. Let $V'\subset \C^{n-1}$ be a linear subspace with $\sQ(V') = \sQ'$ and $A'$ be a $\C$-valued, $d\times (n-1)$--matrix whose row span is $V'$. Choose $y_1,\ldots,y_k$ algebraically independent over $\Q$. Because the polynomials $f\in S_{\sQ}$ have $\Z$-coefficients, we have that $f(x_{ij},y_{\ell})\neq 0$ for all $f\in U_{\sQ}$. Thus, with $y = [y_1,\ldots,y_k,0,\ldots,0]^T \in \C^{d}$, the row span $V$ of $[A'|y]$ is a $\C$-realization of $\sQ$ and $\varphi_{\sQ,\sQ'}(V) = V'$. 
\end{proof}


Recall that a hyperplane of a rank $d$ matroid is a flat of rank $d-1$. For a simple rank $d$ matroid $\sQ$, a \emph{proper hyperplane} of $\sQ$ is a hyperplane $H$ with at least $d$ elements. We denote the set of proper hyperplanes by $\pL(\sQ)$. 

\begin{proposition}
\label{prop:2planes}
    Let $\sQ$ be a $\C$-realizable, simple, connected, rank $d\geq 3$ matroid on $E$ with rank function $\rho$, and let $a\in E$.   If
    \begin{equation*}
        \pL_{d-1}(\sQ,a) = \{ H \in \pL(\sQ) \, : \, a\in H,\text{ and } \rho(H\setminus a) = d-1 \}
    \end{equation*}
    has at most 2 elements, then the morphism $\Gr(\sQ;\C) \to \Gr(\sQ\setminus a;\C)$ is smooth and surjective with connected fibers. 
\end{proposition}

\noindent Alternatively, the set $\pL_{d-1}(\sQ,a)$ consists of those hyperplanes of $\sQ$ containing $a$ such that $H\setminus a$ is a hyperplane of $\sQ\setminus a$. When $|\pL_{d-1}(\sQ,a)|\leq 1$, Proposition \ref{prop:principalExtension} applies, see Lemma \ref{lem:01planes} below. If $\pL_{d-1}(\sQ,a)=\{H_1,H_2\}$ but $|H_1\cap H_2|$ has too few elements, then Proposition \ref{prop:principalExtension} does not apply. Nevertheless, the idea is similar. The projective realizations of $\sQ$ are obtained by freely placing, in a projective realization of $\sQ'$, a point on the codimension-2 subspace lying at the intersection of the hyperplanes spanned by $H_1$ and $H_2$ (outside of finitely many hypersurfaces). So $\Gr(\sQ;\C)$ is isomorphic to an open subscheme of $\Gr(\sQ;\C)\times (\G_{m})^{d-2}$. The bound $|\pL_{d-1}(\sQ,a)|\leq 2$ is essential, see Example \ref{ex:morphismNotSmooth}. %Many situations in this proposition are covered by Proposition \ref{prop:principalExtension}, as we see in the following Lemma. 

\begin{lemma}
\label{lem:01planes}
    Let $\sQ$ be a $\C$-realizable, simple, connected, rank $d\geq 3$ matroid on $E$ with rank function $\rho$, and let $a\in E$.
    \begin{enumerate}
        \item  If $\pL_{d-1}(\sQ,a) = \emptyset$, then $\sQ = \pe{\sQ'}{E'}{a}$ where $E' = E\setminus a$.  
        \item If $\pL_{d-1}(\sQ,a) = \{H_1,\ldots,H_r\}$ and 
        \begin{equation*}
            \rho(H_1\cap \cdots \cap H_r\setminus a) = \rho(H_1\cap \cdots \cap H_r) = d-r,
        \end{equation*}
        then $\sQ = \pe{\sQ'}{\eta'}{a}$ where $\eta' = H_1\cap \cdots \cap H_r\setminus a$. 
    \end{enumerate}
\end{lemma}

\noindent In the proof, we make use of the closure operator $\cl_{\sQ}$ matroid $\sQ$. Given a subset $A\subset E$, its closure $\cl_{\sQ}(A)$ is the smallest flat containing $A$.  See \cite{Oxley} for details. 

\begin{proof}
Suppose $\pL_{d-1}(\sQ,a) = \emptyset$. The hyperplanes of $\sQ$ containing $a$ are of the form $\eta'\cup a$ for flats $\eta'$ of $\sQ'$ of rank $d-2$. These are exactly the hyperplanes of $\pe{\sQ'}{E'}{a}$ that contain $a$ by \cite[Proposition~7.3.3]{Brylawski}, and so statement (1) follows. 


Consider statement (2) and suppose $\sQ \neq \pe{\sQ'}{\eta'}{a}$. If $\lambda \in \pB(\sQ) \setminus \pB(\pe{\sQ'}{\eta'}{a})$, then for all $b\in \eta'\setminus \lambda$ the set $(\lambda\setminus a) \cup b$ is a not a basis of $\sQ'$. So $\eta' \subset \cl_{\sQ'}(\lambda\setminus a)$, which is a hyperplane $H'$ of $\sQ'$.  Since $\rho(H'\cup a) = d$, the set $H'$ is a hyperplane of $\sQ$ not containing $a$. Furthermore, we have that $a\notin \cl_{\sQ}(\eta')$, i.e., $\rho(\eta') < \rho(\eta)$, because $\cl_{\sQ}(\eta') \subset H'$. Conversely, if $\lambda'\in \pB(\sQ')$ and $b\in \lambda'\cap \eta'$ such that $\lambda = (\lambda'\setminus b) \cup a \in \pB(\pe{\sQ'}{\eta'}{a})\setminus \pB(\sQ)$, then $H = \cl_{\sQ}(\lambda)$ is a proper hyperplane of $\sQ$. Since $\rho(H\cup b)=d$, we have that $b\in \eta\setminus H$, and so $H \neq H_1,\ldots,H_r$. Furthermore, $\rho(H\setminus a) \geq \rho(\lambda'\setminus b) = d-1$, and hence $\{H,H_{1},\ldots,H_{r}\}\subset  \pL_{d-1}(\sQ,a)$. 
\end{proof}


\begin{proof}[Proof of Proposition \ref{prop:2planes}]
By Lemma \ref{lem:01planes}, if $|\pL_{d-1}(\sQ,a)|\leq 1$, or if $\pL_{d-1}(\sQ,a) = \{H_1,H_2\}$ such that $\rho(H_1\cap H_2\setminus a) = d-2$, then $\sQ$ is a principal extension, and hence Proposition \ref{prop:principalExtension} applies. 
So we may assume that $\pL_{d-1}(\sQ,a) = \{H_1, H_2\}$ such  $\rho(H_1\cap H_2) < d-2$. Since $\sQ$ is connected, it has no coloops, and therefore there is a basis of $\sQ$ that does not contain $a$.  After applying a suitable isomorphism, we may assume that $E=[n]$, $a=n$,  $[d]$ is a basis of $\sQ$, $[d-1]\subset H_1$, and $d-1\notin H_2$. 
    
Let $A$ be the matrix from Formula\eqref{eq:A}. As in the proof of Proposition \ref{prop:principalExtension}, let $A_{\lambda}(\sQ) = \pi_{\sQ}(A_{\lambda})$ where $\pi_{\sQ}$ is the ring map from Formula \eqref{eq:piQ}. Because $[d-1]\subset H_1$, we have $\pi_{\sQ}(y_d) = 0$, and since $|H_1\cap H_2|\leq d-3$, we have $\pi(y_k) = y_k$ for $k\leq d-1$. Fix a $(d-1)$--element independent set $\mu\subset H_2\setminus n$. By Lemma \ref{lem:expand} we have
\begin{equation*}
A_{\mu \cup n}(\sQ) = (-1)^{d+1}\sum_{k=1}^{d-1}  A_{k \cup \mu}(\sQ)\, y_k \equiv 0 \mod I_{\sQ}.
\end{equation*}
Since $d-1\notin H_2$, the set $(d-1)\cup \mu$ is a basis of $\sQ$.
Solving for $y_{d-1}$ yields the following rational function 
\begin{equation*}
    g = \frac{-1}{A_{(d-1)\cup \mu}(\sQ)} \left( 
    \sum_{k=1}^{d-2} A_{k\cup \mu}(\sQ) \, y_{k} 
    \right).
\end{equation*}
\noindent Let $U$ be the multiplicative semigroup of $S_{\sQ'} \otimes \C[y_1^{\pm},\ldots, y_{d-1}^{\pm}]$ defined by 
\begin{equation*}
    U = \langle f(x_{ij},y_1,\ldots,y_{d-2},g)\, : \, f\in U_{\sQ} \rangle_{\smgp}.
\end{equation*}    
Because $\sQ_{|[n-1]} = \sQ'$, the semigroup $U_{\sQ'}$ is contained in $U$. Define a ring homomorphism
\begin{gather*}
    \phi: U_{\sQ}^{-1}B_{\sQ} \to U^{-1}(B_{\sQ'} \otimes \C[y_{1}^{\pm}, \ldots, y_{d-2}^{\pm}]) \\
    \phi(x_{ij}) = x_{ij}, \hspace{10pt} \phi(y_1) = y_1, \ldots,  \phi(y_{d-2}) = y_{d-2},  \hspace{10pt} \phi(y_{d-1}) = g.
\end{gather*}
\noindent Now let $\lambda$ be any $(d-1)$--element subset of $H_2\setminus n$. Using Lemma \ref{lem:expand}, we have
\begin{equation*}
\phi(A_{\lambda\cup n}(\sQ)) =\frac{-1}{A_{(d-1)\cup\mu}(\sQ)} \sum_{k=1}^{d-2} (A_{(d-1) \cup \mu}(\sQ)A_{k\cup \lambda}(\sQ) - A_{k \cup \mu}(\sQ)A_{(d-1)\cup \lambda}(\sQ) )y_{k}
\end{equation*}
By the quadratic Pl\"ucker relations (see, e.g., \cite[Chapter~9.1]{FultonYT}) we have
\begin{equation*}
    A_{(d-1)\cup \mu}(\sQ)A_{k\cup \lambda}(\sQ) - A_{k\cup \mu}(\sQ)A_{(d-1)\cup\lambda}(\sQ) \equiv \sum_{m\in \mu} \pm A_{(k,d-1)\cup\mu\setminus m}(\sQ) A_{m \cup \lambda}(\sQ) \mod I_{\sQ}
\end{equation*}
The expression on the right lies in $I_{\sQ'}$ because $m \cup \lambda \subset H_{2}\setminus n$.  We conclude that $\phi(A_{\mu}(\sQ)) \in I_{\sQ'}$. Therefore, the morphism $\phi$ descends to a morphism on ring quotients
    \begin{equation*}
        \phi:S_{\sQ} \to U^{-1}(S_{\sQ'} \otimes \C[y_1^{\pm},\ldots,y_{d-2}^{\pm}])
    \end{equation*}
The map $\phi$ is a partial inverse to the inclusion morphism $S_{\sQ'}\otimes \C[y_1^{\pm},\ldots,y_{d-2}^{\pm}] \to S_{\sQ}$. This proves that $\Gr(\sQ;\C) \to \Gr(\sQ';\C)$ factors as an open immersion $\Gr(\sQ;\C)\hookrightarrow \Gr(\sQ';\C)\times (\G_m)^{d-2}$ followed by the projection $\Gr(\sQ';\C)\times (\G_m)^{d-2} \to \Gr(\sQ';\C)$, which is smooth.  
\end{proof}

\begin{corollary}
Let $\sQ$ and $\sQ'$ be as in Proposition \ref{prop:principalExtension} or \ref{prop:2planes}. Then  $\Gr(\sQ;\C)$ is smooth (resp. irreducible) if and only if $\Gr(\sQ';\C)$ is smooth (resp. irreducible).
\end{corollary}

% Figure environment removed

\begin{example}
\label{ex:morphismNotSmooth}
    Let $\sQ$ be the $(4,10)$--matroid whose bases are $\binom{[10]}{4}$ except for $H_1 = \{1,2,3,10\}$, $H_2 = \{4,5,6,10\}$, and $H_3 = \{7,8,9,10\}$, and let $\sQ' = \sQ\setminus \{10\}$. We show that the morphism $\Gr(\sQ;\C) \to \Gr(\sQ';\C)$ is not smooth. Define the matrix of variables
    \begin{equation*}
    A = \begin{bmatrix}
            1 & 0 & 0 & 0 & x_{11} & x_{12} & x_{13} & x_{14} & x_{15} & y_{1} \\ 
            0 & 1 & 0 & 0 & x_{21} & x_{22} & x_{23} & x_{24} & x_{25} & y_{2} \\ 
            0 & 0 & 1 & 0 & x_{31} & x_{32} & x_{33} & x_{34} & x_{35} & y_{3} \\ 
            0 & 0 & 0 & 1 & x_{41} & x_{42} & x_{43} & x_{44} & x_{45} & 0
        \end{bmatrix}
    \end{equation*}
    Using Lemma \ref{lem:expand}, we may express the ideal $I_{\sQ}$ as 
    \begin{align*}
    I_{\sQ} = \langle 
        A_{1456} \, y_1 - A_{2456} \, y_2 + A_{3456} \, y_3, \;
        A_{1789} \, y_1 - A_{2789} \, y_2 + A_{3789} \, y_3\rangle
    \end{align*}
    Solving for $y_{3}$ using the first generator and performing this substitution into the second generator yields
    \begin{equation*}
        (A_{1789}A_{3456} -A_{1456}A_{3789}) \, \frac{y_1}{A_{3456}} + (-A_{2789}A_{3456} +  A_{3789}A_{2456}) \,\frac{y_2}{A_{3456}} \in I_{\sQ}.
    \end{equation*}
    Over the open set $\{-A_{2789}A_{3456} +  A_{3789}A_{2456} \neq 0\}$ of $\Gr(\sQ';\C)$, the morphism $\varphi_{\sQ,\sQ'}:\Gr(\sQ;\C) \to \Gr(\sQ';\C)$ has 1-dimensional fibers (since we may then eliminate $y_2$). However, over the closed set $\{-A_{2789}A_{3456} +  A_{3789}A_{2456} = 0\}$ of $\Gr(\sQ';\C)$, the morphism $\varphi_{\sQ,\sQ'}$ has 2-dimensional fibers. Therefore, $\Gr(\sQ;\C) \to \Gr(\sQ';\C)$ is not flat, in particular, it is not smooth.  We can understand this geometrically in the following way. For a general point configuration in $\P^{3}$ realizing $\sQ$, the hyperplanes spanned by $H_1,H_2,$ and $H_3$ intersect at a single point. This is the right-side illustrated in Figure \ref{fig:notSmoothMorphism}. However, the hyperplanes are allowed to intersect along a line, this is the left-side illustration in this figure. 
\end{example}

\section{Rank 3 matroids}\label{sec:3n}

Let $\sQ$ be a simple $(3,n)$--matroid. A proper hyperplane of $\sQ$ is called a \textit{line}, and we denote the set of lines  by $\pL(\sQ)$. As the non-proper hyperplanes are exactly the 2-element subsets not contained in a proper hyperplane, $\sQ$ is uniquely determined by $\pL(\sQ)$.  The matroid $\sQ$ satisfies the \textit{3 lines property} if every element in the ground set is contained in at least 3 lines. The following Proposition is \cite[Lemma~3.2]{NazirYoshinaga} and \cite[Lemma~4.1]{CoreyGrassmannians}; it is a direct consequence of Proposition \ref{prop:2planes}.  


\begin{proposition}\label{prop:3lines}
Suppose $\Gr(\sQ;\C)$ is smooth for all $\C$-realizable rank $3$ matroids $\sQ$ on $<n$ elements. 
Then the matroid strata for all $\C$-realizable, $(3,n)$--matroids that are either not simple, not connected, or do not satisfy the 3 lines property are smooth. 
\end{proposition}

\noindent Matroid strata and realization spaces for rank $3$ matroids on $8$ or fewer elements have been studied in \cite{CoreyGrassmannians,CoreyLuber}. We summarize these finding in the following Proposition. 
\begin{proposition}
    For $n\leq 8$, the realization space $\pR(\sQ;\C)$ is smooth for all $(3,n)$--matroids realizable over $\C$. They are also irreducible except for the case where $\sQ$ is the M\"{o}bius-Kantor matroid. 
\end{proposition}


\subsection{Rank 3 matroids on 11 or fewer elements}
\label{sec:3leq11}

Here is an outline of our general strategy to show that the realization spaces for $(3,n)$--matroids (for $9\leq n\leq 11$) are smooth. See Examples \ref{ex:3-9} and \ref{ex:3-10} for illustrations.

\begin{itemize}
    \item[Step 1.] Find all simple $(3,n)$--matroids that satisfy the 3 lines property (we modify this step for $n=9$). We use the catalog of small matroids from \texttt{polyDB} \cite{polyDB}, which originates in \cite{MatsumotoMoriyamaImaiBremner}.  
    \item[Step 2.] Remove all matroids not realizable over $\C$. 
    \item[Step 3.]  Using Algorithm \cite[Algorithm~6.12]{CoreyLuber}, systematically eliminate variables to produce a simpler presentation for the coordinate ring of  $\pR(\sQ;\C)$.  Call this new ring $S$. 
    \item[Step 4.] Determine the singular locus, e.g.,  by applying the Jacobian criterion \cite[Corollary~16.20]{Eisenbud} to $S$.
\end{itemize}


\begin{proposition}
\label{prop:3-9smooth}
    All realization spaces for $(3,9)$--matroids are smooth and, except for those listed in Table \ref{tab:39}, irreducible. 
\end{proposition}

\begin{proof}
    Up to $\Sn{9}$--symmetry, there are $383$ simple $(3,9)$--matroids, and $374$ are $\C$-realizable.  Since there are not too many such matroids, we do not isolate those that satisfy the 3 lines property; this allows us to classify those realization spaces that are also irreducible. 
    
    For all matroids except those in Table \ref{tab:39}, we find a reference circuit so that, after applying Step 3 from above, the ideal reduces to $\langle 0 \rangle$. Therefore, these realization spaces are smooth and irreducible. For the matroids in Table \ref{tab:39}, the presentation of the coordinate ring of $\pR(\sQ;\C)$ is obtained using the reference circuit $\{1,2,3,4\}$ and applying Step 3. As the ideals are each generated by a single degree 2 univariate polynomial, these realization spaces are smooth with 2 irreducible components. 
\end{proof}

In Table \ref{tab:39}, the 3rd matroid is the Hesse matroid,  the 4th matroid is studied in Example \ref{ex:3-9}, and the 6th one is the Perles matroid. The remaining matroids are obtained from the M\"{o}bius-Kantor matroid by principal extension. 


\begin{table*}[h]
\centering
	\begin{tabular}{ |c|c|c|c| }
	\hline
	 \multirow{2}{5cm}{\centering \textbf{Lines} $\pL(\sQ)$} & \textbf{Ambient} & \multirow{2}{1.5cm}{\centering \textbf{Ideal}} & \multirow{2}{5cm}{\centering \textbf{Semigroup}} \\
   & \textbf{ring} & &\\
\hline 
   \multirow{4}{5cm}{\centering 127, 138, 145, 246, 258, 347, 356, 678} & \multirow{4}{1.5cm}{\centering $\C[x,y,z]$} & \multirow{4}{2cm}{\centering $x^2-x+1$} & \multirow{4}{5.5cm}{\centering {\footnotesize $x, y, z, x - 1, y - 1, z - 1, x - y, xz - x + y - z,  x - y + z, y - z, x - y + z - 1, x + z - 1, xy - y + z, xy - xz + z,  xz - y$}} \\
    & & & \\
    & & & \\
    & & & \\
	\hline 
    \multirow{2}{5cm}{\centering 128, 135, 147, 239, 245, 267, 346, 378, 568} & \multirow{2}{1.5cm}{\centering $\C[x,y]$} & \multirow{2}{2cm}{\centering $x^2-x+1$} & \multirow{2}{5.6cm}{\centering {\footnotesize $x, y, x - 1, y - 1, x - y, x - y - 1, xy + 1, xy - x + 1, xy - y + 1, xy + x - y$}} \\
    & & & \\
	\hline 
   \multirow{2}{5cm}{\centering 127, 138, 145, 169, 239, 246, 258,  347, 356, 489, 579, 678} & \multirow{2}{1.5cm}{\centering $\C[x]$} & \multirow{2}{2cm}{\centering $x^2-x+1$} & \multirow{2}{5cm}{\centering $x, x-1$} \\
     & & & \\
	\hline 
   \multirow{2}{5cm}{\centering 125, 139, 147, 168, 237, 246, 289, 345, 578, 679} & \multirow{2}{1.5cm}{\centering $\C[x]$} & \multirow{2}{2.4cm}{\centering $x^2+1$} & \multirow{2}{5.7cm}{\centering $x,x-1,x+1$} \\
    & & & \\
			\hline 
   \multirow{2}{5cm}{\centering   1258, 136, 149, 237, 269, 345, 467, 579} & \multirow{2}{1.5cm}{\centering $\C[x,y]$} & \multirow{2}{2cm}{\centering $x^2-x+1$} & \multirow{2}{5.4cm}{\centering  $x, y, x-1, y-1, x - y, xy - 1, xy - x - y, x + y - 1$} \\
    & & & \\
			\hline 
   \multirow{2}{5cm}{\centering 1258, 136, 179, 237, 249, 345, 389, 468, 567} & \multirow{2}{1.5cm}{\centering $\C[x]$} & \multirow{2}{2cm}{\centering $x^2+x-1$} & \multirow{2}{4cm}{\centering $x, x+1, x-1$} \\
     & & & \\
			\hline 
   \multirow{2}{5cm}{\centering 1258, 136, 237, 269, 345, 389, 468, 479, 567} & \multirow{2}{1.5cm}{\centering $ \C[x]$} & \multirow{2}{2cm}{\centering $x^2+x+1$} & \multirow{2}{4cm}{\centering $x, x+1, x-1$} \\
     & & & \\
			\hline 
   \multirow{2}{5cm}{\centering 1259, 1367, 238, 247, 345, 469, 568, 789} & \multirow{2}{1.5cm}{\centering $\C[x]$} & \multirow{2}{2cm}{\centering $x^2-x+1$} & \multirow{2}{4cm}{\centering $x, x-1, x-2$} \\
     & & & \\
			\hline 
		\end{tabular}
		\caption{$\C$-realizable $(3,9)$--matroids with reducible realization spaces}
		\label{tab:39}
	\end{table*}

\begin{proposition}
\label{prop:3-10-11smooth}
    All realization spaces for $(3,10)$ and $(3,11)$--matroids are smooth. 
\end{proposition}

\begin{proof}
    Up to $\Sn{10}$--symmetry, there are $5249$ simple $(3,10)$--matroids, $151$ satisfy the 3 lines property, and 107 of these are $\C$-realizable. Similarly, up to $\Sn{11}$--symmetry, there are $232\,928$ simple $(3,11)$--matroids, $16\,234$ satisfy the 3 lines property, and $11\,516$ of them are $\C$-realizable.  For all of these matroids (for both $n=10,11$), we find a reference circuit so that, after applying Step 3, the ideal $I_{\sQ}$ reduces to a principal ideal. Given a ring of the form $S = U^{-1}\C[x_1,\ldots,x_m]/\langle f\rangle$, the ideal of the singular locus of $\Spec(S)$ is 
    \begin{equation*}
        J = \langle f, \frac{\partial f}{\partial x_1}, \ldots, \frac{\partial f}{\partial x_m}\rangle.
    \end{equation*}
    Thus $\Spec(S)$ is smooth if and only if $J = \langle 1 \rangle$ in $U^{-1}\C[x_1,\ldots,x_m]$, equivalently, the saturation of $J$ by $U$ is the unit ideal in $\C[x_1,\ldots,x_m]$. We use this to verify that these realization spaces are smooth. 
\end{proof}

\noindent Theorem \ref{thm:intro-3-leq11-smooth} now follows from Propositions \ref{prop:3-9smooth} and \ref{prop:3-10-11smooth}. 


\begin{example}
\label{ex:3-10}
    Let $\sQ$ be the simple $(3,10)$--matroid whose lines are
    \begin{equation*}
        \mathcal{L}(\mathsf{Q}) = \left\{
        \begin{array}{c}
        \{1,2,5\}, \{1,3,6\}, \{1,4,8\}, \{2,3,7\}, \{2,4,9\}, \{2,6,10\}, \\ 
        \{3,4,5\}, \{4,6,7\}, \{5,9,10\},\{6,8,9\}, \{7,8,10\}
        \end{array}
         \right\}.
    \end{equation*}
    Let $B=\C[x,y]$ and define the $B$-valued matrix $A$ by
    \begin{equation*}
    A = 
\begin{bmatrix}
    1 & 0 & 0 & 1 & 1 & x & 0 & x^2-xy-1 & 1 & x \\
    0 & 1 & 0 & 1 & 1 & 0 & x & x-y-1 & -x+y+1 & y\\
    0 & 0 & 1 & 1 & 0 & 1 & x-1 & x-y-1 & 1 & 1
\end{bmatrix}        
    \end{equation*}
    Define the ideal $I\subset \C[x,y]$ and multiplicative semigroup $U \subset \C[x,y]$ by
\begin{align*}
    I &= \langle x^2y - x^2 - xy^2 + xy - y \rangle \\
    U &= \left\langle 
    \begin{array}{c}
x, y, x - 1, y - 1, x - y,  x - y - 1, xy - x + 1, xy - x - y, x^2 - xy - 1, \\
x^2 - xy + y, x^2 - xy - x + y + 1, x^3 - 2x^2 - xy^2 + 2xy - 2y 
    \end{array}
    \right\rangle_{\smgp}
\end{align*}
Denote by $f$ the unique generator of $I$. 
The $\C$-realizations of $\sQ$ are exactly the row spans of $A$ for any $(x,y)$ such that $f(x,y)=0$ and $g(x,y) \neq 0$ for all $g\in U$. The singular locus of $I$ is the ideal  $J\subset U^{-1}\C[x,y]$ defined by
\begin{align*}
    J = \langle f,\, \frac{\partial f}{\partial x},\, \frac{\partial f}{\partial y}  \rangle 
    = \langle x^2y-x^2-xy^2+xy-y,\, 2x-y,\, x^2-2xy+x-1\rangle.
\end{align*}
From the second generator, $y\equiv 2x \mod J$. Performing this substitution in the remaining two generators yields $-2x^3+x^2-2x$ and $-3x^2+x-1$. Because these univariate polynomials are relatively prime, the ideal $J$ is the unit ideal, and therefore $\pR(\sQ;\C)$ is smooth. 
\end{example}



\subsection{(3,12)--matroids and beyond}
\label{sec:3-12}
In this section we exhibit a rank $3$ matroid on $12$ elements, denoted $\sQ_{\sing}$, whose realization space is singular. Using this, we show that singular realization spaces exist for $(3,n)$-matroids with $n>12$.
\begin{remark}
We discovered the matroid $\sQ_{\sing}$ by experiments using \texttt{OSCAR}. While our examination of $(3,12)$-matroids is not exhaustive, our search yields $76$ singular realization spaces. This data is available in the github repository.
\end{remark}

Define
\begin{align}
\label{eq:singularMatroid}
    \mathcal{L} = 
    \left\{ \begin{array}{l} 
\{1,2,6,8\}, \{1,3,5,7\}, \{1,9,12\}, \{2,4,5,9\}, \{2,7,11\}, \{3,4,6\}, \\ 
\{3,8,9\}, \{3,10,12\}, \{4,7,8\},  \{4,10,11\}, \{5,6,10\}, \{8,11,12\}
\end{array} \right\}
\end{align}
and let $\sQ_{\sing}$ be the simple $(3,12)$--matroid with $\pL(\sQ_{\sing}) = \pL$. Let $B = \C[x,y]$ and define the $B$-valued matrix 
\begin{equation}
\label{eq:representationMatrix312}
    A = \begin{bmatrix}
    1 & 0 & 0 & 1 & 1 & 1 & y-1 & 1 & 1 & x & y-1 & x \\
    0 & 1 & 0 & 1 & 0 & 1 & 0 & y & y & y & -xy^2+2y^2-y & y \\
    0 & 0 & 1 & 1 & 1 & 0 & y & 0 & 1 & x-y & y & 1
    \end{bmatrix}
\end{equation}
The coordinate ring of $\pR(\sQ_{\sing};\C)$ is isomorphic to $U^{-1}B/I$ where 
\begin{equation*}
    I = \langle(xy+x-2y)(y^2-y+1)\rangle
\end{equation*}
and $U$ is generated by
\begin{gather*}
x,
y,
x - 1,
x - 2,
y - 1,
y + 1,
x - y,
x - 2y,
x - y - 1,
xy - y + 1,
xy - 2y + 1, \\
x + y^2 - y,
xy - 2y + 2,
x + y^2 - 2y,
x + y^2 - y - 1,
xy - y^2 - y + 1, \\
xy^2 - y^2 + y - 1,
xy^2 - 2y^2 + y - 1,
xy^2 - 2y^2 + 2y - 1,
x^2y - xy^2 - 2xy + x + 2y^2.
\end{gather*}
\begin{theorem}
\label{thm:singular3-12}
    The realization space $\pR(\sQ_{\sing};\C)$ has 2 irreducible components and 2 nodal singularities. 
\end{theorem}

\begin{proof} 
The description above realizes $\pR(\sQ_{\sing};\C)$ as a dense open subvariety of an affine plane curve with 2 irreducible components: $X_1$ corresponding to $xy + x - 2y = 0$ and $X_2$ corresponding to $y^2-y+1=0$. Individually, these irreducible components are smooth, so the singularities of $\pR(\sQ_{\sing};\C)$ are the intersection points of these curves. Solving the system of equations $ xy+x-2y = y^2-y+1 = 0$ yields the following two points
    \begin{equation*}
    \sq_1 = \left( \frac{3-\sqrt{-3}}{3},\frac{1-\sqrt{-3}}{2}\right) \hspace{20pt}
    \sq_2 = \left(\frac{3+\sqrt{-3}}{3},\frac{1+\sqrt{-3}}{2}\right)
    \end{equation*}
  It remains to show that $f(\sq_1)$ and $f(\sq_2)$ are nonzero for all $f$ in the generating set of $U$ recorded above, which is a routine verification. 
  Since $X_1$ and $X_2$ intersect transversely at these points, these singularities are nodes.
\end{proof}

For $n\geq 12$, denote by $\sQ_{n,\sing}$ the principal extension of the matroid $\sQ_{\sing}$ from Theorem \ref{thm:singular3-12} by $n-12$ elements into the unique maximal flat (i.e., the ground set). The following theorem is a direct consequence of Theorem \ref{thm:singular3-12} and Proposition \ref{prop:principalExtension}.

\begin{theorem}
\label{thm:singular3-n}
For $n\geq 12$ the realization space $\pR(\sQ_{n,\sing};\C)$ has nodal singularities.  
\end{theorem}


\section{Rank 4 matroids}\label{sec:4n}

For a simple and connected $(4,n)$-matroid $\sQ$, a proper hyperplane is called a \textit{plane}, and we denote the set of planes by $\pL(\sQ)$.  We say that a simple and connected $(4,n)$--matroid $\sQ$ satisfies the \textit{3 planes property} if every $a\in [n]$ is contained in at least $3$ proper hyperplanes of $\sQ$. Similarly, $\sQ$ satisfies the \textit{4 planes property} if every $a\in [n]$ is contained in at least $4$ proper hyperplanes of $\sQ$. The following is a direct consequence of Propositions \ref{prop:directsum}, \ref{prop:principalExtension}, and \ref{prop:2planes}.

\begin{proposition}
\label{prop:4planes}
Let $\sQ$ be a connected and $\C$-realizable  $(4,n)$--matroid such that
\begin{enumerate}
    \item $\sQ$ is not simple,
    \item $\sQ$  does not satisfy the 3 planes property, or
    \item $n\leq 9$ and $\sQ$ does not satisfy the 4 planes property. 
\end{enumerate}
    If the matroid strata are smooth for all $\C$-realizable rank $4$ matroids on $<n$ elements, then $\Gr(\sQ;\C)$ is smooth. Similarly, if the matroid strata are smooth and irreducible for all $\C$-realizable rank $4$ matroids on $<n$ elements, then $\Gr(\sQ;\C)$ is smooth and irreducible. 
\end{proposition}



\begin{proposition}
    If $\sQ$ is a $\C$--realizable $(4,n)$--matroid for $n\leq 7$, then $\Gr(\sQ;\C)$ is smooth and irreducible.
\end{proposition}

\begin{proof}
    This follows from duality (Proposition \ref{prop:duality}) and the fact that the matroid strata for matroids of rank at most $3$ on at most  $7$ elements are smooth and irreducible.  
\end{proof}



\begin{proposition}
\label{prop:4-8-smooth}
    All realization spaces for $\C$-realizable $(4,8)$-matroids are smooth and, except for those listed in Table \ref{tab:48}, irreducible. 
\end{proposition}

\begin{proof} 
We use \texttt{OSCAR} to prove this proposition. 
Up to $\Sn{8}$--symmetry, there are $592$ simple and connected $(4,8)$--matroids. Of those, $92$ satisfy the $4$ planes property and $66$ are $\C$-realizable. (Those matroids that are not simple, not connected, or do not satisfy the $4$ planes property have smooth and irreducible realization spaces by Proposition \ref{prop:4planes}).
For $63$ of these, we find a reference circuit such that $I_{\sQ}$ reduces to $\langle 0 \rangle$ in Step 3 from \S \ref{sec:3leq11}. So these realization spaces are smooth and irreducible.  The remaining $3$ are listed in Table \ref{tab:48}. The presentations listed in this table are obtained by using $\{1,2,3,4,5\}$ as a reference circuit and applying Step 3 from \S\ref{sec:3leq11}. As these ideals are each generated by a single degree 2 univariate polynomial, these realization spaces are smooth with 2 irreducible components. 
%Of the remaining $26$, all that admit a reference circuit are not realizable over $\C$. That is, there are $25$ non $\C$-realizable simple connected matroids satisfying the $4$-lines property. 
\end{proof}

\begin{table}[h]\label{table:4-8}
\centering
	\begin{tabular}{ |c|c|c|c| }
	\hline
	 \multirow{2}{5cm}{\centering \textbf{Planes} $\pL(\sQ)$} & \textbf{Ambient} & \multirow{2}{1.5cm}{\centering \textbf{Ideal}} & \multirow{2}{3cm}{\centering \textbf{Semigroup}} \\
   & \textbf{ring} & &\\
\hline 
   \multirow{2}{6cm}{\centering 3467, 2567, 2458, 2378, 1568, 1357, 1348, 1247, 1236} & \multirow{2}{1.5cm}{\centering $\C[x]$} & \multirow{2}{3cm}{\centering $x^2-3x+1$} & \multirow{2}{3cm}{\centering { $x, x-2,$ \\ $ x-1, x-3$}} \\
    & & & \\
	\hline
  \multirow{2}{6cm}{\centering 4568, 3467, 2567, 2378, 1357,
  1348, 1258, 1247, 1236} & \multirow{2}{1.5cm}{\centering $\C[x]$} & \multirow{2}{3cm}{\centering $3x^2-3x+1$} & \multirow{2}{4cm}{\centering { $x,x-1,3x-1,$ \\ $3x-2,2x-1, x-3$}} \\
    & & & \\
	\hline
 \multirow{2}{6cm}{\centering 12367, 5678, 3456, 2478, 2358, 1457, 1248, 1268, 1256, 1246} & \multirow{2}{1.5cm}{\centering $\C[x]$} & \multirow{2}{3cm}{\centering $3
x^2-x+1$} & \multirow{2}{3cm}{\centering {$x,x-1$}} \\
    & & & \\
	\hline 
	\end{tabular}
	\caption{$\C$-realizable $(4,8)$--matroids with disconnected realization spaces}
	\label{tab:48}
\end{table}

\noindent Interestingly, there is exactly one simple and connected $(4,8)$--matroid satisfying the $4$ planes property that does \emph{not} have a circuit of size $5$. 

\begin{example}\label{example:weird 4-8}
Let $\sQ$ be the matroid of the 8 points in $(\F_2)^{3}$, i.e., the matroid of the $\F_{2}$--valued matrix
\begin{equation*}
    A = \begin{bmatrix}
       0 &1 & 0 &0 &1 &1 &0 &1\\
       0 &0 &1 &0 &1 &0 &1 &1\\
       0 &0 &0 &1 &0 &1 &1 &1\\
        1 &1 &1 &1 &1 &1 &1 &1
     \end{bmatrix}
\end{equation*}
All circuits of this matroid have size at most $4$. While realizable over $\F_2$, this matroid is not realizable over $\C$. 
\end{example}

%\subsection{$(4,9)$-matroids}

\begin{proposition}
\label{prop:4-9-smooth}
    All realization spaces for $\C$-realizable $(4,9)$--matroids are smooth.
\end{proposition}
\begin{proof}
    Up to $\Sn{9}$--symmetry, there are $185\,911$ simple and connected matroids $(4,9)$--matroids, $61\, 228$  satisfy the $4$ planes property, and $39\,246$ of these are $\C$-realizable. 
    For each of these matroids, we find a reference circuit so that, after applying Step 3 from \S\ref{sec:3leq11}, the ideal $I_{\sQ}$ reduces to a principal ideal. We verify that these spaces are smooth using the Jacobian criterion as outlined in the proof of Proposition \ref{prop:3-10-11smooth}. 
\end{proof}

\section{Singular initial degenerations}

We begin this section with a brief overview of initial degenerations of Grassmannians and their relation to (valuated) matroids as described in \cite{CoreyGrassmannians}. In \S\ref{sec:singularInitialDeg312}, we develop refined techniques to study initial degenerations corresponding to corank vectors of simple rank-3 matroids on the way to proving Theorem \ref{thm:not-shon-geq-12}.


\subsection{Initial degenerations and tropicalization}
\label{sec:initial_degenerations}
Let $S = \C[x^{\pm}_{1},\dots,x^{\pm}_{a}]$, and given $\su = (u_1,\ldots,u_a) \in \Z^{a}$ set $x^{\su} = x_{1}^{u_1}\cdots x_{a}^{u_a}$. Given $f\in S$ and $\sw\in (\R^{a})^{\vee}$, the $\sw$-\emph{initial from} of $f$ is 
\begin{equation*}
\init_{\sw}f = \sum\limits_{\substack{\langle \su,\sw\rangle\\\text{ is minimal}}}c_{\su}x^{\su}\hspace{.5 cm} 
\text{where} 
\hspace{.5 cm} f = \sum c_{\su}x^{\su}.
\end{equation*}
Similarly, for a variety $X = V(I)$ defined by an ideal $I\subset S$, we obtain 
\begin{equation*}
\init_{\sw} I =\langle \init_{\sw}f\, : \, f\in I\rangle\hspace{.5 cm}\text{ and }\hspace{.5 cm}\init_{\sw}X = V(\init_{\sw}I),    
\end{equation*}
the $\sw$-\emph{initial ideal} of $I$ and $\sw$-\emph{initial degeneration} of $X$ respectively. The \emph{tropicalization of $X$} is 
\begin{equation*}
\Trop\, X = \{\sw\in (\R^{a})^{\vee} \, :\, \init_{\sw}I \neq\langle 1 \rangle\}.
\end{equation*}
When $X = \Gr^{\circ}(d,n)$, we write $\Trop\, \Gr^{\circ}(d,n) = \TGr^{\circ}(d,n)$.  The set $\TGr^{\circ}(d,n)$ is invariant under translation by $\mathbf{1}$, so we view $\TGr^{\circ}(d,n) \subset (\R^{\binom{[n]}{d}})^{\vee}/\Zone$. 

\subsection{Matroidal subdivisions of hypersimplices}\label{sec:matroid_subdivisions}
Denote by $\epsilon_1,\ldots,\epsilon_n$ the standard basis of $\Z^{n}$ and $\epsilon_1^*,\ldots,\epsilon_n^{*}$ its dual basis in $(\Z^{n})^{\vee}$. Given $\lambda = \{i_1,\ldots,i_d\} \in \binom{[n]}{d}$, set  $\epsilon_{\lambda} = \epsilon_{i_1} + \cdots + \epsilon_{i_d}$ and $\epsilon_{\lambda}^{*} = \epsilon_{i_1}^{*} + \cdots + \epsilon_{i_d}^{*}$. For a $(d,n)$-matroid $\sQ$, its  \emph{matroid polytope} $\Delta(\sQ)\subset \R^{n}$ is
\begin{equation*}
\Delta(\sQ) = \conv( \epsilon_{\lambda} \, : \, \lambda\in \pB(\sQ)).
\end{equation*}

Given a finite set $E$ and a nonnegative integer $d\leq |E|$, the rank-$d$ \textit{uniform matroid} on $E$, denoted $\sU_{d,E}$, is the matroid whose bases are the $r$-element subsets of $E$. When $E=[n]$, we simply write $\sU_{d,n}$. The matroid polytope of $\sU_{d,n}$ is called the \textit{hypersimplex}, and is denoted by $\Delta(d,n)$. The vertices of $\Delta(d,n)$ are exactly the vectors $\epsilon_{\lambda}$ for $\lambda \in \binom{[n]}{d}$.  

Given $\sw\in (\R^{\binom{[n]}{d}})^{\vee}/\Zone$, the \textit{regular subdivision} of $\Delta(d,n)$ induced by $\sw$, denoted by $\pQ(\sw)$, is obtained by lifting the vertex $\epsilon_{\lambda}$ to height $\sw_{\lambda}$ in $\R^{n}\times \R$, and projecting the lower faces of the resulting point configuration back to $\Delta(d,n)$; for a comprehensive treatment of regular subdivisions, see \cite[Chapter~2]{DeLoeraRambauSantos} or  \cite[Chapter~7]{GelfandKapranovZelevinsky} (where they are called \textit{coherent subdivisions}). By \cite[Proposition~2.2]{Speyer2008}, if $\sw \in \TGr^{\circ}(d,n)$, then $\pQ(\sw)$ is \textit{matroidal}, i.e., the cells of the subdivision are polytopes of $(d,n)$-matroids. We denote these matroids by $\sU_{\sw}^{\sv}$, and their bases are given by  
\begin{equation*}
    \pB(\sU_{\sw}^{\sv}) = \{ \lambda \in \textstyle{\binom{[n]}{d}} \, : \, \bk{\epsilon_{\lambda}}{\sv} + \sw_{\lambda} \text{ is minimal}   \}
\end{equation*}
for $\sv \in N_{\R}$. 





\subsection{Inverse limits of matroid strata}\label{sec:inverse_limits_tight_spans}
Let $\sw \in (\R^{[n]\choose d})^{\vee}/\Zone$ such that the corresponding subdivision $\pQ(\sw)$ is matroidal. We view $\pQ(\sw)$ as a poset via the face relation. If $\Delta(\sQ')$ is a face of $\Delta(\sQ)$, then the inclusion of polynomial rings $B_{\sQ'} \subset B_{\sQ}$ induces a morphism of matroid strata 
\begin{equation*}
    \varphi_{\sQ,\sQ'}:\Gr(\sQ;\F) \to \Gr(\sQ';\F)
\end{equation*}  
(provided these matroids are $\F$-realizable), see \cite[Proposition~I.6]{Lafforgue2003} or \cite[Proposition~3.1]{CoreyGrassmannians}. This yields a diagram of matroid strata over the poset $\pQ(\sw)$, denote the corresponding (finite) inverse limit by $\Gr(\sw)$. The following result in \cite{CoreyGrassmannians} is essential in our construction of singular initial degenerations.

\begin{theorem}\label{thm:closed_immersion}
Let $\sw\in\TGr^{\circ}(d,n)$. The inclusions 
\begin{equation*}
    \C[p_{\lambda} \, : \, \lambda \in \pB(\sQ)] \subset \C[p_{\lambda} \, : \, \lambda \in \textstyle{\binom{[n]}{d}}]
\end{equation*}
for $\Delta(\sQ) \in \pQ(\sw)$ induce a closed immersion 
\begin{equation*}
\psi_{\sw}: \init_{\sw}\Gr^{\circ}(d,n)\hookrightarrow\Gr(\sw).
\end{equation*} 
\end{theorem}

\noindent The dual graph of $\pQ(\sw)$, denoted $\Gamma(\sw)$, is the graph that has a vertex $v_{\sQ}$ for each maximal cell $\Delta(\sQ)$ in $\pQ(\sw)$, and $v_{\sQ_1}$ is connected to $v_{\sQ_2}$ by an edge if the polytopes $\Delta(\sQ_1)$ and $\Delta(\sQ_2)$ share a common facet. The graph $\Gamma(\sw)$ is a subposet of $\pQ(\sw)$ with the relation  $e\prec v$ if $v$ is a vertex of the edge $e$. By \cite[Proposition~C.12]{CoreyGrassmannians} (Appendix by Cueto), the inverse limit over this smaller poset is isomorphic to $\Gr(\sw)$. 

\subsection{Singular initial degenerations of the Grassmannian}
\label{sec:singularInitialDeg312}
Given a $(d,n)$--matroid $\sQ$, its \textit{corank} vector $\sw(\sQ) \in (\R^{\binom{[n]}{d}})^{\vee}/\langle\mathbf{1}\rangle$ is 
\begin{equation*}
    \sw(\sQ)_{\lambda} = d - \rho(\lambda)
\end{equation*}
where $\rho$ is the rank function of $\sQ$. 


\begin{proposition}
    Let $\sQ$ be a simple $(3,n)$--matroid with rank function $\rho$. The corank vector of $\sQ$ is
    \begin{align*}
        \sw(\sQ)_{\lambda} = 
        \begin{cases}
            0 & \text{ if } \lambda \in \pB(\sQ) \\
            1 & \text{ if } \lambda \in \binom{[n]}{3} \setminus \pB(\sQ)
        \end{cases}
    \end{align*}
\end{proposition}

\begin{proof}
    If $\lambda \in \pB(\sQ)$, then $\rho(\lambda) = 3$, and if $\lambda \in \binom{[n]}{r}\setminus \pB(\sQ)$, then $\rho(\lambda) \leq 2$.  
    Because $\sQ$ is simple, $\rho_(\lambda) \geq 2$ for any subset $\lambda$ with at least 2 elements, from which the proposition follows. 
\end{proof}

In terms of inequalities, the hypersimplex $\Delta(r,n)$ is
\begin{equation*}
    \Delta(r,n) = \{x\in \R^{n} \, : \, 0\leq x_i \leq 1, \; x_1 + \cdots + x_n = r\}.
\end{equation*}
If $\sQ$ is a simple and connected $(3,n)$--matroid, then by \cite[Theorem~2]{GelfandSerganova} and \cite[Proposition~5.1]{CoreyGrassmannians}.
\begin{equation}
\label{eq:DeltaQFacets}
    \Delta(\sQ) = \{x \in \Delta(3,n) \, : \, \sum_{i\in \eta} x_i \leq 2 \text{ for all } \eta \in \pL(\sQ) \}. 
\end{equation}
Given a subset $\eta \subset [n]$ of size at least 3,  denote by $\sQ(\eta)$ the $(3,n)$--matroid whose bases are
\begin{equation*}
    \pB(\sQ(\eta)) = \{ \lambda \in \textstyle{\binom{[n]}{3}} \, : \, |\lambda \cap \eta| \geq 2  \}. 
\end{equation*}
Its polytope is described in terms of inequalities as
\begin{equation*}
    \Delta(\sQ(\eta)) = \{x \in \Delta(3,n) \, : \, \sum_{i\in [n]\setminus \eta} x_i \leq 1 \}.
\end{equation*}


\begin{proposition}
\label{prop:subdStar}
    Let $\sQ$ be a simple and connected $(3,n)$--matroid and $\sw$ its corank vector. Then $\Gamma(\sw)$ is a star graph where the central node corresponds to $\sQ$ and its leaves correspond to $\sQ(\eta)$ for $\eta \in \pL(\sQ)$. 
\end{proposition}


\begin{proof}
First, we claim that the matroids of the maximal cells of $\pQ(\sw)$ are $\sQ$ and $\sQ(\eta)$ for $\eta \in \pL(\sQ)$.  Because
    \begin{equation*}
        \sQ = \sU_{\sw}^{0} 
        \hspace{20pt} \text{and} \hspace{20pt} \sQ(\eta) = \sU_{\sw}^{\sv} \hspace{10pt}\text{(where $\sv = -\epsilon_{\eta}$ for $\eta \in \pL(\sQ)$),}
    \end{equation*}
    we see that  $\sQ$ and $\sQ(\eta)$ for $\eta \in \pL(\sQ)$ are matroids of maximal cells of $\pQ(\sw)$ (they are maximal since these matroids are connected). Suppose $x \in \Delta(3,n) \setminus \Delta(\sQ)$. By the characterization of $\Delta(\sQ)$ in Formula \ref{eq:DeltaQFacets}, there is some $\eta \in \pL(\sQ)$ such that 
    \begin{equation*}
        \sum_{i\in \eta} x_i > 2, \hspace{20pt} \text{equivalently, } \hspace{20pt} \sum_{i\in [n]\setminus \eta} x_i < 1. 
    \end{equation*}
    From this, we see that $x\in \Delta(\sQ(\eta))$, and so there are no other maximal cells of $\pQ(\sw)$.    

    Next, we show that the only edges of $\Gamma(\sw)$ are between $v_{\sQ}$ and $v_{\sQ(\eta)}$. Indeed, $\Delta(\sQ)$ and $\Delta(\sQ(\eta))$ (for $\eta \in \pL(\sQ)$) intersect along a common facet whose matroid is
    \begin{equation*}
        \sQ(\eta)' = \{\lambda \in \textstyle{\binom{[n]}{3}} \, : \, |\lambda \cap \eta| = 2\}. 
    \end{equation*}
    
    We claim that, if  $\eta_1, \eta_2\in \pL(\sQ)$ are distinct, then  $\Delta(Q(\eta_1))$ and $\Delta(Q(\eta_2))$ either do not intersect, or intersect along a codimension 2 face. Since $\eta_1$ and $\eta_2$ are lines of the simple matroid $\sQ$, we have $|\eta_1 \cap \eta_2| \leq 1$. If $\eta_1 \cap \eta_2 = \emptyset$, then $\pB(\sQ(\eta_1)) \cap \pB(\sQ(\eta_2)) = \emptyset$, and so $\Delta(Q(\eta_1))$ and $\Delta(Q(\eta_2))$ do not intersect. If $\eta_1 \cap \eta_2 = \{a\}$, then the matroid $\sQ''$ of the intersection of $\Delta(Q(\eta_1))$ and $\Delta(Q(\eta_2))$ is
    \begin{equation*}
        \sQ'' = \sU_{1,\eta_1} \oplus \sU_{1,\eta_2} \oplus \sU_{1,\{a\}} \oplus \sU_{0, [n] \setminus (\eta_1\cup \eta_2)}.
    \end{equation*}
 In particular, $\Delta(\sQ'')$ has dimension $\leq n-3$, as required.     
\end{proof}

\begin{proposition}
\label{prop:smoothSurjectiveQeta}
    For any $\eta\subset [n]$ with $3 \leq |\eta| \leq n-2$, the face morphism  $\varphi_{\eta}:= \varphi_{\sQ(\lambda), \sQ(\lambda)'}:\Gr(\sQ(\eta);\C) \to \Gr(\sQ(\eta)';\C)$ is smooth and surjective with connected fibers. 
\end{proposition}

\begin{proof}
The morphism $\varphi_{\eta}:\Gr(\sQ(\eta);\C) \to \Gr(\sQ(\eta)';\C)$ is smooth and dominant with connected fibers by \cite[Proposition~4.5]{CoreyLuber}, but a more careful argument is needed to establish surjectivity. For brevity, let $\sQ = \sQ(\eta)$ and $\sQ' = \sQ(\eta)'$.  Without loss of generality, suppose $\eta = [k-3]\setminus \{3\}$ (where $4\leq k \leq n-1$). Define index sets $\Lambda'$ and $\Lambda$ by
\begin{align*}
    &\Lambda' = \{(i,j) \, : \, i=1,2;\, j=1,2,\ldots,k \} \cup \{(3,j) \, : \, j=k+1\ldots,n-3\}, \\
    &\Lambda = \Lambda' \cup \{(3,j) \, : \, j=1,\ldots,k\}.
\end{align*}

Let $A(x)$ be the matrix of variables
\begin{equation*}
        A(x) = \begin{bmatrix}
            1 & 0 & 0 & x_{11} & \cdots & x_{1k} & 0 & \cdots & 0 \\
            0 & 1 & 0 & x_{21} & \cdots & x_{2k} & 0 & \cdots & 0 \\
            0 & 0 & 1 & x_{31} & \cdots & x_{3k} & x_{3,k+1} & \cdots & x_{3, n-3} \\
        \end{bmatrix}
    \end{equation*}
Then  $B_{\sQ'} = \C[x_{ij}\, : \, ij\in \Lambda']$ and $B_{\sQ} = \C[x_{ij}\, : \, ij\in \Lambda]$. The coordinate rings of $\Gr(\sQ';\C)$ and $\Gr(\sQ; \C)$ are $S_{\sQ'} = U_{\sQ'}^{-1}B_{\sQ'}$ and $S_{\sQ} = U_{\sQ}^{-1}B_{\sQ}$, respectively, and the morphism $\Gr(\sQ;\C) \to \Gr(\sQ';\C)$ is induced by the inclusion $(U_{\sQ'})^{-1}B_{\sQ}' \subset U_{\sQ}^{-1}B_{\sQ}$. In particular, these are integral affine schemes of finite-type over $\C$. To show that $\varphi_{\eta}: \Gr(\sQ;\C) \to \Gr(\sQ';\C)$ is surjective, it suffices to show that it is surjective at the level of closed points \cite[8.4.F]{VakilFOAG}. 

Let $V'$ be a $\C$-realization of $\sQ'$, i.e., $V'$ is the row span of the matrix $A(y)$ where  $y_{ij} = 0$ for $(i,j)\in \Lambda\setminus \Lambda'$ and $y_{ij}$ for $(i,j)\in \Lambda'$ are nonzero complex numbers such that $f(y_{ij}) \neq 0$ for $f\in U'$. Choose complex numbers $z_{ij}$ such that $z_{ij}$ for $(i,j)\in \Lambda\setminus \Lambda'$ are algebraically independent over $\Q$ and $z_{ij} = y_{ij}$ for $(i,j)\in\Lambda'$. Let $V$ be the row span of $A(z)$. Since the semigroup $U_{\sQ}$ are generated by finitely many polynomials with \textit{integer} coefficients, we see that $V$ is a realization of $\sQ$ and $\varphi(V) = V'$, as required. 
\end{proof}

\begin{proposition}
\label{prop:compareLimitToCenter}
    Let $\sQ$ be a simple and connected $(3,n)$--matroid and $\sw$ its corank vector. Then the morphism $\varphi_{\sQ}:\Gr(\sw) \to \Gr(\sQ;\C)$ is smooth and surjective with connected fibers. Furthermore,
    \begin{equation*}
        \dim \Gr(\sw) = \dim \Gr(\sQ;\C) + \sum_{\eta\in \pL(\sQ)} (|\eta|-2).
    \end{equation*}
\end{proposition}


\begin{proof}
    By Proposition \ref{prop:smoothSurjectiveQeta}, the morphism
    \begin{equation*}
        \prod_{\eta\in \pL(\sQ)}\varphi_{\eta}: \prod_{\eta\in \pL(\sQ)} \Gr(\sQ(\eta);\C) \to \prod_{\eta\in \pL(\sQ)} \Gr(\sQ(\eta)';\C),
    \end{equation*}
    and therefore its base-change $\varphi_{\sQ}:\Gr(\sw) \to \Gr(\sQ)$, is smooth and surjective with connected fibers. Since
    \begin{equation*}
        \dim \Gr(\sQ(\eta);\C) = 2|\eta| + n  - 7  
        \hspace{20pt} \text{and} \hspace{20pt} \dim \Gr(\sQ(\eta)';\C) = |\eta| + n - 5, 
    \end{equation*}
    the proposition follows from \cite[Proposition~A.6]{CoreyGrassmannians}. 
\end{proof}


Given a closed subset $X$ of a scheme $Y$, denote by $X_{\red}$ the reduced induced scheme structure on $X$.

\begin{proposition}
\label{prop:codim0}
    Suppose $Y$ is an $n$-dimensional finite-type reduced affine scheme, and $X\subset Y$ is a closed subscheme of dimension $n$. Then $X_{\red}$ is a union of irreducible components of $Y$.
\end{proposition}

\begin{proof}
    Suppose $Y = \Spec(B)$ and $X = \Spec(B/I)$. Let $I'$ be a minimal prime of the radical of $I$. Since $\dim X = n$,  there is a length-$n$ chain of prime ideals $I' \subsetneq P_1 \subsetneq \cdots \subsetneq P_n$ in $B$. Since $\dim Y = n$, this means that $I'$ is a minimal prime of $B$, i.e., $\Spec(B/I')$ is an irreducible component of $Y$. 
\end{proof}


For the remainder of this section, we denote by $\sQ$ the simple $(3,n)$--matroid with $\pL(\sQ) = \pL$ from Formula \eqref{eq:singularMatroid}. This is obtained from the matroid considered in \S\ref{sec:3-12} by a free extension by $n-12$ elements. We also denote by $\sw \in (\R^{\binom{n}{3}})^{\vee}$ its corank vector. 

\begin{proposition}
\label{prop:singularIsom}
    The corank vector  $\sw$ lies in $\TGr^{\circ}(3,n;\C)$ and the closed immersion $\init_{\sw}\Gr^{\circ}(3,n;\C) \to \Gr(\sw)$ is an isomorphism.  
\end{proposition}

Suppose $X\subset \G_m^{a}$ is a closed subvariety and $K=\CCt$. Given  $\sfp \in X(K)$, recall that the \textit{tropicalization} $\trop(\sfp)$ is the coordinatewise valuation of $\sfp$. The \textit{exploded tropicalization} $\ETrop(\sfp)$ is the vector whose coordinates are the coefficients of the leading terms. By \cite[Lemma~3.2]{Payne}, $\ETrop(\sfp) \in \init_{\sw} X$ where $\sw=\trop(\sfp)$.  

\begin{proof}[Proof of Proposition \ref{prop:singularIsom}]
As in the proof of Theorem \ref{thm:singular3-12}, denote by $X_1 = V(xy+x-2y)$ and $X_{2} = V(y^2-y+1)$ the two irreducible components of $\pR(\sQ,\C)$. Let $A_{1}$ and $A_{2}$ be the matrices obtained by evaluating the matrix $A$ from \eqref{eq:representationMatrix312} at  $(x,y)=(3,-3)$ and $(x,y)=(3, (1-\sqrt{-3})/2)$, respectively. The row spans of these matrices are points of $X_1\setminus X_2$ and $X_2\setminus X_1$ respectively.  Define
    \begin{equation*}
    B_t =\begin{bmatrix}
    t & 0 & 2t & -t & t & 0 & -t & t & 0 & t & t & t \\
    -t & 0 & t & t & t & 0 & 3t & 0 & t & t & t & -t \\
    0 & 0 & -t & t & 0 & t & -t & -t & t & t & t & 0
    \end{bmatrix}.
    \end{equation*}
Set $C_{1} = A_1+B_t$ and $C_2 = A_{2} + B_{t}$. Let $D$ be a $3\times (n-12)$ matrix whose entries are complex numbers algebraically independent over $\Q$. Finally, set
\begin{equation*}
    E_{1} = [C_{1} | D], \hspace{15pt} \text{ and } \hspace{15pt} E_{2} = [C_{2} | D]. 
\end{equation*}
Denote by $\sfp_1(t)$ and $\sfp_2(t)$ the Pl\"ucker vectors of $E_1$ and $E_2$, respectively. Since the elements in $D$ are algebraically independent, we have that $\val_t(\sfp_i(t)_{\lambda}) = 0$ for all $\lambda\in \binom{[n]}{3}$ such that $\lambda \cap \{13,14,\ldots,n\} \neq \emptyset$. Thus, it is a routine verification to show that $\trop(\sfp_1(t))$ and $\trop(\sfp_2(t))$ equal $\sw$; in particular, this implies $\sw \in \TGr^{\circ}(3,n;\C)$.  

Denote by $\varphi_{\sQ}:\Gr(\sw) \to \Gr(\sQ;\C)$ the structural morphism from the inverse limit and  $\pi:\Gr(\sw) \to \pR(\sQ_{\sing};\C)$ the composition
\begin{equation*}
    \Gr(\sw) \xrightarrow{\varphi_{\sQ}} \Gr(\sQ;\C) \to \Gr(\sQ_{\sing};\C) \to \pR(\sQ_{\sing};\C).
\end{equation*}
The morphism $\pi$ is smooth and surjective with connected fibers by Propositions \ref{prop:principalExtension} and \ref{prop:compareLimitToCenter}. It fits into the commutative diagram
\begin{equation*}
\begin{tikzcd}
 \Gr(\sw) \arrow[r, hook, "\iota"] \arrow[dr, "\pi"'] &\pR(\sQ_{\sing}; \C) \times (\G_{m})^{3n-10} \arrow[d] \\
 & \pR(\sQ_{\sing};\C) 
\end{tikzcd}    
\end{equation*}
where $\iota$ is a dominant open immersion and the vertical arrow is the projection onto the first factor.  So $\Gr(\sw)$ is a reduced affine scheme with two irreducible components of dimension $3(n-3)$.

Set  $\sa_1 = \ETrop(\sfp_1)$ and $\sa_2 = \ETrop(\sfp_2)$; these points are in $\init_{\sw}\Gr^{\circ}(3,n)$ as discussed above. Since $(\sa_1)_{\lambda} = \sfp_{1}(0)_{\lambda}$ and $(\sa_2)_{\lambda} = \sfp_{2}(0)_{\lambda}$ for $\lambda \in \pB(\sQ_{\sing})$, we have  $(\pi\circ\psi_{\sw})(\sa_1) \in X_1\setminus X_2$ and $(\pi\circ\psi_{\sw})(\sa_2) \in X_2 \setminus X_1$ where $X_1$ and $X_{2}$ are the 2 irreducible components of $\pR(\sQ_{\sing};\C)$ described above. So the image of $\psi_{\sw}$ meets the 2 irreducible components of $\Gr(\sw)$ outside of their intersection. By Proposition \ref{prop:codim0}, we conclude that $\psi_{\sw}$ is an isomorphism. 
\end{proof}



\begin{proof}[Proof of Theorem \ref{thm:not-shon-geq-12}]
    This is a direct consequence of Theorem \ref{thm:singular3-n} and Proposition \ref{prop:singularIsom}. 
\end{proof}


\section{Concluding remarks}

We conclude with some potential directions for future work. Our examination of realization spaces of matroids did not completely exhaust available data. That is, there are still $(3,12)$-matroids that have gone unstudied, and the data set of $(4,10)$-matroids has not been examined at all. Of course, as more data becomes available, it would be natural to apply our methods to matroids of rank $d>4$. 


It is also natural to ask what conditions are sufficient or necessary to conclude that the realization space of a matroid is singular. Furthermore, for a given pair $(d,n)$, what singularity types can appear?  While Mn\"ev universality guarantees all singularity types appear on \emph{some} realization space, are there obstructions that prevent certain singularity types from appearing for a particular $(d,n)$ pair? 

Finally, we revisit Question \ref{question:3-n range} from the introduction. Theorem \ref{thm:not-shon-geq-12} leaves the problem of determining whether $\Gr^{\circ}(3,n)$ is sch\"on open only for $n=9,10,11$.  Some challenges to resolving this are immediately apparent. For this range of $n$, all matroid strata are smooth by \ref{prop:3-9smooth} and \ref{prop:3-10-11smooth}, so the technique used to construct the singular initial degeneration in Proposition \ref{prop:singularIsom} fails. By \cite[Example 8.2]{CoreyGrassmannians}, for $n=9$ there exist initial degenerations such that the closed immersion of Theorem \ref{thm:closed_immersion} is not an isomorphism. Thus an exhaustive study of inverse limits over dual graphs for matroidal subdivisions of hypersimplices (as in \cite{CoreyGrassmannians} and \cite{CoreyLuber}) is complicated.
\bibliographystyle{abbrv}
\bibliography{bibliographie}
\label{sec:biblio}

\end{document}



