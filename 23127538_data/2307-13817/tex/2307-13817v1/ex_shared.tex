% SIAM Shared Information Template
% This is information that is shared between the main document and any
% supplement. If no supplement is required, then this information can
% be included directly in the main document.


% Packages and macros go here
\usepackage{lipsum}
\usepackage{amsfonts}
\usepackage{graphicx}
\usepackage{epstopdf}
\usepackage{algorithmic}
\ifpdf
  \DeclareGraphicsExtensions{.eps,.pdf,.png,.jpg}
\else
  \DeclareGraphicsExtensions{.eps}
\fi

% Prevent itemized lists from running into the left margin inside theorems and proofs
\usepackage{enumitem}
\setlist[enumerate]{leftmargin=.5in}
\setlist[itemize]{leftmargin=.5in}

% Add a serial/Oxford comma by default.
\newcommand{\creflastconjunction}{, and~}

% Used for creating new theorem and remark environments
\newsiamremark{remark}{Remark}
\newsiamremark{hypothesis}{Hypothesis}
\crefname{hypothesis}{Hypothesis}{Hypotheses}
\newsiamthm{claim}{Claim}

% Sets running headers as well as PDF title and authors

% Title. If the supplement option is on, then "Supplementary Material"
% is automatically inserted before the title.
% Sets running headers as well as PDF title and authors
\headers{Dynamical Fractal: Theory and Case Study}{Junze Yin}

% Title. If the supplement option is on, then "Supplementary Material"
% is automatically inserted before the title.
\title{Dynamical Fractal: Theory and Case Study\footnote{This work was funded by Student Research Award (SRA) from Boston University Undergraduate Research Opportunity Program}}

% Authors: full names plus addresses.
\author{Junze Yin\thanks{Department of Mathematics \& Statistics, Boston University, Boston, MA 02215 USA 
  (\email{junze@bu.edu}).}}

\usepackage{amsopn}
\DeclareMathOperator{\diag}{diag}


%%% Local Variables: 
%%% mode:latex
%%% TeX-master: "ex_article"
%%% End: 
