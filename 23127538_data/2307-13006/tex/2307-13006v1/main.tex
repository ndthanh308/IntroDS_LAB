\documentclass[twocolumn]{revtex4}
\usepackage{amssymb}
\usepackage{amsthm}
\usepackage{amsmath}
\usepackage{latexsym}
\usepackage{xcolor}
\usepackage{epsfig}
\usepackage{float}
\usepackage{soul}
%\usepackage{orcidlink}
%\orcidlink{0000-0003-0941-8422}\orcidlink{0000-0003-4854-2960}
\begin{document}
\title{The shadows of quantum gravity on Bell's inequality}
\author{H. Moradpour$^1$\footnote{h.moradpour@riaam.ac.ir}, S. Jalalzadeh$^2$\footnote{shahram.jalalzadeh@ufpe.br}}
\address{$^1$ Research Institute for Astronomy and Astrophysics of Maragha (RIAAM), University of Maragheh, P.O. Box 55136-553, Maragheh, Iran\\
$^2$ Departamento de F\'{i}sica, Universidade Federal de Pernambuco, Recife-PE, 50670-901, Brazil}

\begin{abstract}
The validity of quantum mechanical operators in the presence of quantum aspects of gravity is a subject of investigation since they may not hold true and require generalization. One of the key objectives of the present study is to examine the impact of such generalizations on the non-locality that is inherent in quantum mechanics and manifests itself in Bell's inequality. Another aspect of the study is to analyze the consequences of introducing a non-zero minimal length for the well-established Bell's inequality. The findings of this research are expected to contribute to the theoretical understanding of the interplay between quantum mechanics and gravity.
\end{abstract}

\maketitle
\section{Introduction}

The quantum realm is governed by the Heisenberg uncertainty principle (HUP), which mandates that the Hamiltonian be written as the starting point, leading to the Schrodinger equation and, eventually, the eigenvalues and wave function of the quantum system under consideration. In Heisenberg's formulation of quantum mechanics (QM) in the Hilbert space, we encounter states rather than wave functions (although they are connected). In general, QM fails to produce satisfactory solutions for systems featuring the Newtonian gravitational potential in their Hamiltonian. Therefore, in conventional and widely accepted quantum mechanics, gravity is not accounted for in terms of its operators or corresponding Hilbert space (quantum states) carrying gravitational information.

The incompatibility of gravity and quantum mechanics is not limited to Newtonian gravity and persists even when general relativity is considered. On the other hand, the existence of gravity, even in a purely Newtonian regime, leads to a non-zero minimum (of the order of $10^{-35}\textmd{m}$ (Planck length) \cite{Mead:1964zz}) for the uncertainty in position measurement \cite{Mead:1966zz, Mead:1964zz, Kempf:1994su, Hossenfelder:2012jw}. Consistently, various scenarios of quantum gravity (QG), like String theory, also propose a non-zero minimal for length \cite{Kempf:1994su, Hossenfelder:2012jw}. The non-zero minimal length existence may affect the operators, and it leads to the generalization of HUP, called generalized uncertainty principle (GUP), \cite{Kempf:1994su, Hossenfelder:2012jw}.

Operators and system states in Quantum Gravity (QG) may differ from those in QM. They are, in fact, functions of ordinary operators that appear in QM \cite{Hossenfelder:2012jw}. For instance, when considering the first order of the Generalized Uncertainty Principle (GUP) parameter ($\beta$), we find that the momentum operator $\hat{P}$ can be expressed as $\hat{p}(1+\beta \hat{p}^2)$, where $\hat{P}$ and $\hat{p}$ represent momentum operators in QG and QM, respectively. In this representation, $\beta$ is positive, and the position operator remains unchanged \cite{Hossenfelder:2012jw}. It follows that gravity could impact our understanding of classical physics-based operator sets that have been established by QM \cite{sur, sur1}. Consequently, it is generally possible to write $\hat{O}=\hat{o}+\beta\hat{o}_p$, where $\hat{O}$ and $\hat{o}$ are operators in QG and QM, respectively, and $\hat{o}_p$ is the first-order correction obtained using perturbation theory \cite{Aghababaei:2022jxd}.

Motivated by the correlation between HUP and quantum non-locality (which is easily demonstrated in the square of Bell's inequality) \cite{Lan, Hei, Einstein:1935rr}, as well as the impact of GUP on operators, particularly angular momentum \cite{Bosso:2016frs}, recent studies have revealed that minimal length can alter the square of Bell's operator \cite{Aghababaei:2021yzx}. Furthermore, GUP can affect the entanglement between energy and time, as evidenced by the results of a Franson experiment (which serves as a testing setup for time-energy entanglement) \cite{Aghababaei:2022rqi}. Table \ref{tab1} clearly displays the generally expected modifications to operators and states resulting from minimal length. The term $|\psi\rangle_{p}$ indicates an increase in a quantum superposition, which is a probabilistic signal for entanglement enhancement \cite{sur, sur1} and, therefore, non-locality beyond quantum mechanics \cite{Popescu:2014wva}. It is apparent that gravity impacts the information bound \cite{Aghababaei:2022jxd}.


\begin{table}
\centering
\begin{tabular}{l|r}
QM & QG \\\hline
$\Delta\hat{x}\Delta\hat{p}\geq\frac{\hbar}{2}$ (HUP) & $\Delta\hat{x}\Delta\hat{P}\geq\frac{\hbar}{2}[1+\beta(\Delta\hat{P})^2]$ (GUP) \\
$\hat{o}$ & $\hat{O}=\hat{o}+\beta\hat{o}_p$ \\
$|\psi\rangle$ & $|\psi_{GUP}\rangle=|\psi\rangle+\beta|\psi\rangle_{p}$\\
%$|o\rangle^i$ & $|O\rangle_{GUP}^i=|o\rangle^i+\beta|o\rangle_{p}^i$
\end{tabular}
\caption{\label{tab1}A comparison between QM and QG (up to the first order of $\beta$). Here, $|\psi\rangle$ and $|\psi_{GUP}\rangle$ denote the quantum states in QM and QG, respectively, and $|\psi\rangle_{p}$ is also calculable using the perturbation theory.}
\end{table}

The inquiry into the influence of special and general relativity (SR and GR) on Bell's inequality (quantum non-locality) has been extensively studied over the years \cite{vonBorzeszkowski:2000my, Gingrich:2002ota, Peres:2002ip, Peres:2002wx}. The existing research on the effects of SR on Bell's inequality can be classified into three general categories, depending on the method of applying Lorentz transformations: (i) the operators change while the states remain unchanged, (ii) only the states undergo the Lorentz transformation while the operators remain unaltered (the reverse of the previous one), and (iii) both the operators and states are affected by the Lorentz transformation \cite{Ter0, Ter1, Ter2, Ter3, Ter4, Ter5, Kim:2004px, DV, Friis:2009va, DV1, DV2, MMM}. Furthermore, certain implications of GR and non-inertial observers have also been addressed in Refs.~\cite{Terashima:2003rjs, Fuentes-Schuller:2004iaz, Alsing:2006cj, Torres-Arenas:2018vei}. Given the ongoing effort to bridge QG with QM \cite{Ashtekar:2002sn}, exploring the effects of QG on quantum non-locality is deemed inevitable and advantageous.

In this study, our aim is to elucidate the influence of QG on Bell's inequality by examining the consequences of the minimal length on Bell's inequality (up to the first order of $\beta$). To achieve this objective, we follow the same approach as the three possibilities mentioned earlier, which investigated the impacts of SR on quantum non-locality. To pave the way toward our objective, we categorize the existing cases into three groups, which are discussed in detail in the subsequent section. Finally, the paper concludes by providing a summary of the findings.

\section{Bell's inequality and the implications of QG}

In the framework of QM, assume two particles and four operators $\hat{A}, \hat{A}^{\prime}, \hat{B}, \hat{B}^{\prime}$ with eigenvalues $\lambda^J$ ($J\in\{\hat{A}, \hat{A}^{\prime}, \hat{B}, \hat{B}^{\prime}\}$), while the first (second) two operators act on the first (second) particle. Now, operators $\hat{j}=\frac{\hat{J}}{|\lambda^J|}\in\{\hat{a}, \hat{a}^{\prime}, \hat{b}, \hat{b}^{\prime}\}$ have eigenvalues $\pm1$, and Bell's inequality is defined as
\begin{eqnarray}\label{1}
    \big\langle\hat{B}\big\rangle\equiv\big\langle\hat{a}(\hat{b}+\hat{b^{\prime}})+\hat{a^{\prime}}(\hat{b}-\hat{b^{\prime}})\big\rangle\leq2.
\end{eqnarray}

Taking into account the effects of QG (up to the first order), the operators are corrected as $\hat{J}_{GUP}=\hat{J}+\beta\hat{J}_p$ and $\hat{j}_{GUP}=\frac{\hat{J}+\beta\hat{J}_p}{|\lambda^J_{GUP}|}$ where $\lambda^J_{GUP}$ represents the eigenvalue of $\hat{J}_{GUP}$. Since QM should be recovered at the limit $\beta\rightarrow0$, one may expect $\lambda^J_{GUP}\simeq\lambda^J+\beta\lambda^J_p$. Moreover, as the $\beta\lambda^J_p$ term is perturbative, it is reasonable to expect $|\beta\frac{\lambda^J_p}{\lambda^J}|<<1$ leading to $|\lambda^J+\beta\lambda^J_p|=|\lambda^J|(1+\beta\frac{\lambda^J_p}{\lambda^J})$. Applying modifications to the states, operators, or both in quantum gravity can result in three distinct situations. Similar studies conducted on the effects of SR on Bell's inequality have also revealed three cases \cite{Ter0, Ter1, Ter2, Ter3, Ter4, Ter5, Kim:2004px, MMM}. Therefore, it is necessary to consider the possibilities arising from these situations to understand the implications of quantum gravitational modifications fully. In the following paragraphs, we will examine these possibilities in depth.

\subsubsection{Purely quantum mechanical entangled states in the presence of operators modified by QG}

Firstly, let us contemplate the scenario in which an entangled state ($|\xi\rangle$) has been prepared away from the QG influences. This implies that the objective has been accomplished using purely quantum mechanical procedures. Furthermore, it is assumed that an observer utilizes Bell measurements that are constructed through the incorporation of operators containing the QG corrections ($\hat{j}_{GUP}$). 
In the framework of QM, the violation amount of inequality~(\ref{1}) depends on the directions of Bell's measurements. Here, we have $\hat{j}=\hat{j}_{GUP}+\beta(\frac{\lambda^J_p}{\lambda^J}\hat{j}_{GUP}-\frac{\hat{J}_p}{|\lambda^J|})$ inserted into Eq.~(\ref{1}) to reach
\begin{eqnarray}\label{2}
&&\!\!\!\!\!\!\!\!\!\big\langle\hat{B}_{GUP}\big\rangle\equiv\\&&\!\!\!\!\!\!\!\!\! \big\langle\hat{a}_{GUP}\big(\hat{b}_{GUP}+\hat{b}^{\prime}_{GUP}\big)+\hat{a}^{\prime}_{GUP}\big(\hat{b}_{GUP}-\hat{b}^{\prime}_{GUP}\big)\big\rangle\leq2\nonumber\\&&\!\!\!\!\!\!\!\!\!-\big\langle\beta^{\prime}_{a}\hat{a}_{GUP}\big(\hat{b}_{GUP}+\hat{b}^{\prime}_{GUP})+\beta^{\prime}_{a^{\prime}}\hat{a}^{\prime}_{GUP}(\hat{b}_{GUP}-\hat{b}^{\prime}_{GUP}\big)\big\rangle-\nonumber\\&&\!\!\!\!\!\!\!\!\! \big\langle\hat{a}_{GUP}\big(\beta^{\prime}_{b}\hat{b}_{GUP}+\beta^{\prime}_{b^{\prime}}\hat{b}^{\prime}_{GUP}\big)+\hat{a}^{\prime}_{GUP}\big(\beta^{\prime}_{b}\hat{b}_{GUP}-\beta^{\prime}_{b^{\prime}}\hat{b}^{\prime}_{GUP}\big)\big\rangle\nonumber\\&&\!\!\!\!\!\!\!\!\! + \beta^{\prime\prime}_{a}\big\langle\hat{A}_{GUP}\big(\hat{b}_{GUP}+\hat{b}^{\prime}_{GUP})+\hat{A}^{\prime}_{GUP}(\hat{b}_{GUP}-\hat{b}^{\prime}_{GUP}\big)\big\rangle+\nonumber\\&&\!\!\!\!\!\!\! \beta^{\prime\prime}_{b}\big\langle\hat{a}_{GUP}\big(\hat{B}_{GUP}+\hat{B}^{\prime}_{GUP}\big)+\hat{a}^{\prime}_{GUP}\big(\hat{B}_{GUP}-\hat{B}^{\prime}_{GUP}\big)\big\rangle,\nonumber
\end{eqnarray}
 where $\beta^{\prime}_{j}=\beta\frac{\lambda^J_p}{\lambda_J}$, $\beta^{\prime\prime}_{j}=\beta|\lambda_J|^{-1}$ and the last two expressions have been written using $\beta^{\prime\prime}_{a}=\beta^{\prime\prime}_{a^\prime}$ and $\beta^{\prime\prime}_{b}=\beta^{\prime\prime}_{b^\prime}$. In this manner, it is clearly seen that although the state is unchanged, in general, $\big\langle\hat{B}_{GUP}\big\rangle\neq\big\langle\hat{B}\big\rangle$. In studying the effects of SR on Bell's inequality, whenever the states remain unchanged and Lorentz transformations only affect Bell's operator, a similar situation is also obtained \cite{Ter0, Ter1, Ter2, Ter3, Ter4, Ter5, Kim:2004px, MMM}.

\subsubsection{Purely quantum mechanical measurements and quantum gravitational states}

Now, let us consider the situation in which the Bell apparatus is built using purely quantum mechanical operators $j$, and the primary entangled state carries the Planck scale information, i.e., the quantum features of gravity. It means that the entangled state is made using the $j_{GUP}$ operators. A similar case in studies related to the effects of SR on Bell's inequality is the case where the Bell measurement does not go under the Lorentz transformation while the system state undergoes the Lorentz transformation \cite{Ter0, Ter1, Ter2, Ter3, Ter4, Ter5, Kim:2004px, MMM}. In this setup, we have $|\xi_{GUP}\rangle=|\xi\rangle+\beta|\xi\rangle_{p}$ and thus
\begin{eqnarray}\label{3}
&&\!\!\!\!\!\!\!\!\!\!\!\! \big\langle\xi_{GUP}\big|\hat{B}\big|\xi_{GUP}\big\rangle\equiv\big\langle\hat{B}\big\rangle_{GUP}=\big\langle\hat{B}\big\rangle+2\beta\big\langle\xi\big|\hat{B}\big|\xi\big\rangle_{p}\nonumber\\ && \!\!\!\!\!\!\!\!\!\!\!\!\Rightarrow \big\langle\hat{B}\big\rangle_{GUP}\leq 2\big(1+\beta\big\langle\xi\big|\hat{B}\big|\xi\big\rangle_{p}\big).
\end{eqnarray}

\subsubsection{Bell's inequality in a purely quantum gravitational regime}

In deriving Bell's inequality, it is a significant step to ensure that the operators' eigenvalues are only either $\pm1$, regardless of their origin, whether it be from QM or QG. If both the Bell measurement and the entangled state were prepared using the quantum gravitational operators, then it is evident that $\big\langle\xi_{GUP}\big|\hat{B}_{GUP}\big|\xi_{GUP}\big\rangle\leq2$. This result indicates that, when considering the effects of QG on both the state and the operators, Bell's inequality and the classical regime's limit (which is $2$ in the inequality) remain unchanged compared to the previous setups. The same outcome is also achieved when it comes to the relationship between SR and Bell's inequality, provided that both the system state and Bell's measurement undergo a Lorentz transformation \cite{Kim:2004px}.


\section{conclusion}

The study can be summarized by its two main components: $i$) the origin of entangled states and $ii$) Bell's measurement. Furthermore, the study has introduced the possibility of three outcomes depending on which cornerstone carries the quantum gravitational modifications. The first two scenarios suggest that if only one of the foundations stores the effects of QG, then a precise Bell measurement (depending on the value of $\beta$) could detect the effects of QG. This is due to the differences between $\big\langle\hat{B}\big\rangle$, $\big\langle\hat{B}_{GUP}\big\rangle$, and $\big\langle\hat{B}\big\rangle_{GUP}$. In the third case, Bell's inequality remains invariant if we consider the quantum aspects of gravity on both the states and the operators.

\subsection*{Data availability}
Data sharing is not applicable to
this article as no datasets were generated or analyzed during the
current study.

\subsection*{Declarations Conflict of interest}
The authors declare no conflict
of interest.

%%%%%%%%%%%%%%%%%%%%%%%%%%%%%%%%%%%%%%%%%%%%%%%%%%%%%%%
\begin{thebibliography}{99}

\bibitem{Mead:1964zz}
C.~A.~Mead,
``Possible Connection Between Gravitation and Fundamental Length,''
Phys. Rev. \textbf{135}, B849-B862 (1964)
doi:10.1103/PhysRev.135.B849

\bibitem{Mead:1966zz}
C.~A.~Mead,
``Observable Consequences of Fundamental-Length Hypotheses,''
Phys. Rev. \textbf{143}, 990-1005 (1966)
doi:10.1103/PhysRev.143.990

\bibitem{Kempf:1994su}
A.~Kempf, G.~Mangano and R.~B.~Mann,
``Hilbert space representation of the minimal length uncertainty relation,''
Phys. Rev. D \textbf{52}, 1108-1118 (1995)
doi:10.1103/PhysRevD.52.1108
[arXiv:hep-th/9412167 [hep-th]].

\bibitem{Hossenfelder:2012jw}
S.~Hossenfelder,
``Minimal Length Scale Scenarios for Quantum Gravity,''
Living Rev. Rel. \textbf{16}, 2 (2013)
doi:10.12942/lrr-2013-2
[arXiv:1203.6191 [gr-qc]].

\bibitem{sur}
``Survey the foundations,'' Nat. Phys. 18, 961 (2022). https://doi.org/10.1038/s41567-022-01766-x

\bibitem{sur1}
J. R. Hance, S. Hossenfelder,
``Bell’s theorem allows local theories of quantum mechanics,'' Nat. Phys. 18, 1382 (2022). https://doi.org/10.1038/s41567-022-01831-5

\bibitem{Aghababaei:2022jxd}
S.~Aghababaei, H.~Moradpour, S.~S.~Wani, F.~Marino, N.~A.~Shah and M.~Faizal.
``Ultimate information bounds beyond the quantum,''
[arXiv:2211.09227 [quant-ph]].

\bibitem{Lan}
L.~Landau, ``Experimental tests of general quantum theories,`` Letters in Mathematical Physics 14, 33-40 (1987) doi:10.1007/BF00403467

\bibitem{Hei}
W.~Heisenberg, ``Uber den anschaulichen Inhalt der quantentheoretischen Kinematik und Mechanik,'' Z. Physik 43, 172–198 (1927). doi:10.1007/BF01397280

\bibitem{Einstein:1935rr}
A.~Einstein, B.~Podolsky and N.~Rosen,
``Can quantum mechanical description of physical reality be considered complete?,''
Phys. Rev. \textbf{47}, 777-780 (1935)
doi:10.1103/PhysRev.47.777

\bibitem{Bosso:2016frs}
P.~Bosso and S.~Das,
``Generalized Uncertainty Principle and Angular Momentum,''
Annals Phys. \textbf{383}, 416-438 (2017)
doi:10.1016/j.aop.2017.06.003
[arXiv:1607.01083 [gr-qc]].

\bibitem{Aghababaei:2021yzx}
S.~Aghababaei, H.~Moradpour and H.~Shabani,
``Quantum gravity and the square of Bell operators,''
Quant. Inf. Proc. \textbf{21}, no.2, 57 (2022)
doi:10.1007/s11128-021-03397-2
[arXiv:2106.14400 [quant-ph]].

\bibitem{Aghababaei:2022rqi}
S.~Aghababaei, H.~Moradpour,
``Generalized uncertainty principle and quantum non-locality,''
Quant. Inf. Proc. \textbf{22}, no.4, 173 (2023)
doi:10.1007/s11128-023-03920-7
[arXiv:2202.07489 [quant-ph]].

\bibitem{Popescu:2014wva}
S.~Popescu,
``Nonlocality beyond quantum mechanics,''
Nature Phys. \textbf{10}, no.4, 264-270 (2014)
doi:10.1038/nphys2916

\bibitem{vonBorzeszkowski:2000my}
H.~von Borzeszkowski and M.~B.~Mensky,
``EPR effect in gravitational field: Nature of nonlocality,''
Phys. Lett. A \textbf{269}, 197-203 (2000)
doi:10.1016/S0375-9601(00)00230-9
[arXiv:quant-ph/0007085 [quant-ph]].

\bibitem{Gingrich:2002ota}
R.~M.~Gingrich and C.~Adami,
``Quantum Entanglement of Moving Bodies,''
Phys. Rev. Lett. \textbf{89}, 270402 (2002)
doi:10.1103/PhysRevLett.89.270402
[arXiv:quant-ph/0205179 [quant-ph]].

\bibitem{Peres:2002ip}
A.~Peres, P.~F.~Scudo and D.~R.~Terno,
``Quantum entropy and special relativity,''
Phys. Rev. Lett. \textbf{88}, 230402 (2002)
doi:10.1103/PhysRevLett.88.230402
[arXiv:quant-ph/0203033 [quant-ph]].

\bibitem{Peres:2002wx}
A.~Peres and D.~R.~Terno,
``Quantum information and relativity theory,''
Rev. Mod. Phys. \textbf{76}, 93-123 (2004)
doi:10.1103/RevModPhys.76.93
[arXiv:quant-ph/0212023 [quant-ph]].

\bibitem{Ter0} P. M. Alsing and G. J. Milburn, ``On entanglement and Lorentz transformations,'' Quant. Inf. Comput. {\bf2}, 487 (2002).

\bibitem{Ter1}
H. Terashima and M. Ueda, ``Einstein–Podolsky–Rosen correlation seen from moving observers,'' Quantum Inf. Comput. 3, 224–228 (2003)
doi:10.48550/arXiv.quant-ph/0204138

\bibitem{Ter2} H. Terashima and M. Ueda, ``Relativistic Einstein–Podolsky-Rosen correlation and Bell’s inequality,'' Int. J. Quant. Inf. {\bf1}, 93 (2003).
doi:10.1142/S0219749903000061

\bibitem{Ter3} D. Ahn, H-J, Lee, Y. H. Moon and S. W. Hwang, ``Relativistic entanglement and Bells inequality,'' Phys. Rev. A {\bf67}, 012103 (2003).

\bibitem{Ter4} 
D. Lee and E. Chang-Young, ``Quantum entanglement under Lorentz boost''. New J. Phys. {\bf6}, 67 (2004).

\bibitem{Kim:2004px}
W.~T.~Kim and E.~J.~Son,
``Lorentz invariant Bell's inequality,''
Phys. Rev. A \textbf{71}, 014102 (2005)
doi:10.1103/PhysRevA.71.014102
[arXiv:quant-ph/0408127 [quant-ph]].

\bibitem{Ter5}
T.~F.~Jordan, A.~Shaji and E.~C.~G.~Sudarshan,
``Lorentz transformations that entangle spins and entangle momenta,''
Phys. Rev. A \textbf{75}, 022101 (2007)
doi:10.1103/PhysRevA.75.022101
[arXiv:quant-ph/0608061 [quant-ph]].

\bibitem{DV}
J. Dunningham and V. Vedral, ``Entanglement and nonlocality of a single relativistic particle,'' Phys. Rev. A 80,
044302 (2009).

\bibitem{Friis:2009va}
N.~Friis, R.~A.~Bertlmann, M.~Huber and B.~C.~Hiesmayr,
``Relativistic entanglement of two massive particles,''
Phys. Rev. A \textbf{81}, 042114 (2010)
doi:10.1103/PhysRevA.81.042114
[arXiv:0912.4863 [quant-ph]].

\bibitem{DV1}
 P. L. Saldanha and V. Vedral, Phys. Rev. A 85, 062101
(2012).

\bibitem{DV2}
P. L. Saldanha and V. Vedral, New J. Phys. 14, 023041
(2012).

\bibitem{MMM}
H. Moradpour, S. Maghool, and S. A. Moosavi, 
``Three-particle Bell-like inequalities under Lorentz transformations,''
Quantum Inf. Process. 14 (2015) 3913
doi:10.1007/s11128-015-1064-3

\bibitem{Terashima:2003rjs}
H.~Terashima and M.~Ueda,
``Einstein-Podolsky-Rosen correlation in a gravitational field,''
Phys. Rev. A \textbf{69}, 032113 (2004)
doi:10.1103/PhysRevA.69.032113
[arXiv:quant-ph/0307114 [quant-ph]].

\bibitem{Fuentes-Schuller:2004iaz}
I.~Fuentes-Schuller and R.~B.~Mann,
``Alice falls into a black hole: Entanglement in non-inertial frames,''
Phys. Rev. Lett. \textbf{95}, 120404 (2005)
doi:10.1103/PhysRevLett.95.120404
[arXiv:quant-ph/0410172 [quant-ph]].

\bibitem{Alsing:2006cj}
P.~M.~Alsing, I.~Fuentes-Schuller, R.~B.~Mann and T.~E.~Tessier,
``Entanglement of Dirac fields in non-inertial frames,''
Phys. Rev. A \textbf{74}, 032326 (2006)
doi:10.1103/PhysRevA.74.032326
[arXiv:quant-ph/0603269 [quant-ph]].

\bibitem{Torres-Arenas:2018vei}
A.~J.~Torres-Arenas, E.~O.~Lopez-Zuniga, J.~A.~Saldana-Herrera, Q.~Dong, G.~H.~Sun and S.~H.~Dong,
``Entanglement measures of W-state in noninertial frames,''
Phys. Lett. B \textbf{789}, 93-105 (2019)
doi:10.1016/j.physletb.2018.12.010
[arXiv:1810.03951 [quant-ph]].

\bibitem{Ashtekar:2002sn}
A.~Ashtekar, S.~Fairhurst and J.~L.~Willis,
``Quantum gravity, shadow states, and quantum mechanics,''
Class. Quant. Grav. \textbf{20}, 1031-1062 (2003)
doi:10.1088/0264-9381/20/6/302
[arXiv:gr-qc/0207106 [gr-qc]].

\end{thebibliography}
\end{document}


%\begin{eqnarray}\label{2}
%&&\!\!\!\!\!\!\!\!\!\big\langle\hat{B}_{GUP}\big\rangle\equiv\\&&\!\!\!\!\!\!\!\!\! \big\langle\hat{a}_{GUP}\big(\hat{b}_{GUP}+\hat{b}^{\prime}_{GUP}\big)+\hat{a}^{\prime}_{GUP}\big(\hat{b}_{GUP}-\hat{b}^{\prime}_{GUP}\big)\big\rangle\leq2\nonumber\\&&\!\!\!\!\!\!\!\!\!-\big\langle\beta^{\prime}_{a}\hat{a}_{GUP}\big(\hat{b}_{GUP}+\hat{b}^{\prime}_{GUP})+\beta^{\prime}_{a^{\prime}}\hat{a}^{\prime}_{GUP}(\hat{b}_{GUP}-\hat{b}^{\prime}_{GUP}\big)\big\rangle-\nonumber\\&&\!\!\!\!\!\!\!\!\! \big\langle\hat{a}_{GUP}\big(\beta^{\prime}_{b}\hat{b}_{GUP}+\beta^{\prime}_{b^{\prime}}\hat{b}^{\prime}_{GUP}\big)+\hat{a}^{\prime}_{GUP}\big(\beta^{\prime}_{b}\hat{b}_{GUP}-\beta^{\prime}_{b^{\prime}}\hat{b}^{\prime}_{GUP}\big)\big\rangle\nonumber\\&&\!\!\!\!\!\!\!\!\! + \big\langle\beta^{\prime\prime}_{a}\hat{A}_{GUP}\big(\hat{b}_{GUP}+\hat{b}^{\prime}_{GUP})+\beta^{\prime\prime}_{a^{\prime}}\hat{A}^{\prime}_{GUP}(\hat{b}_{GUP}-\hat{b}^{\prime}_{GUP}\big)\big\rangle+\nonumber\\&&\!\!\!\!\!\!\!\!\!\!\!\!\! \big\langle\hat{a}_{GUP}\big(\beta^{\prime\prime}_{b}\hat{B}_{GUP}+\beta^{\prime}_{b^{\prime}}\hat{B}^{\prime}_{GUP}\big)+\hat{a}^{\prime}_{GUP}\big(\beta^{\prime\prime}_{b}\hat{B}_{GUP}-\beta^{\prime\prime}_{b^{\prime}}\hat{B}^{\prime}_{GUP}\big)\big\rangle,\nonumber
%\end{eqnarray}