\documentclass[12pt,a4paper]{eletter2}

\signature{Keisuke Fujii\\
Graduate School of Informatics,
Nagoya University\\
Furocho 1, Nagoya, 464-8603, JAPAN \\
Phone No: +81-52-789-4209 \\ 
Email Address: fujii@i.nagoya-u.ac.jp\\
}
% \signature{Calvin C. K. Yeung\\
% Graduate School of Informatics,
% Nagoya University\\
% Furocho 1, Nagoya, 464-8603, JAPAN \\
% Phone No: +81-080-8358-5368 \\ 
% Email Address: yeung.chikwong@g.sp.m.is.nagoya-u.ac.jp\\
% }

\date{\today}
\date{\today}

%\makelabels

\begin{document}
\begin{letter}
{%[EDITOR FULL NAME]\\
MLJ Contribution Information Sheet\\
}

 \opening{Title: A Strategic Framework for Optimal Decisions in Football 1-vs-1 Shot-Taking Situations: An Integrated Approach of Machine Learning, Theory-Based Modeling, and Game Theory}

 \medskip

\begin{enumerate}

\item What is the main claim of the paper? Why is this an important contribution to the machine learning literature?\\

The main claim of our paper is that the proposed framework, which combines game theory, machine learning models, and a theory-based shot block model to analyze agent interactions in football shot-taking scenarios, is an important and novel contribution to the machine learning literature. The research addresses the lack of effective data-driven and theory-based approaches to understanding the strategies employed by players in critical football situations. By leveraging game theory as a foundation and employing machine learning models to estimate the values of players' actions, the framework offers a comprehensive and data-driven analysis of shot-taking interactions. The introduction of the Expected Probability of Shot On Target (xSOT) metric further enhances the evaluation of players' decisions, accounting for both successful shots and shots that do not result in goals. This research not only advances the understanding of agent interactions in sports but also showcases the applicability of integrating game theory and machine learning for decision-making analysis in various domains. As such, our paper's contributions hold significant importance in the machine learning literature by providing a robust approach to analyzing complex agent interactions, which can be relevant and valuable across different fields beyond sports.\\

\item What is the evidence you provide to support your claim? Be precise.\\

We provide several key pieces of evidence to support its claim regarding the effectiveness of the proposed framework for analyzing agent interactions in football shot-taking scenarios. \\

Firstly, the research validates the framework through experiments, comparing its performance against baseline and ablated models. This comparative analysis demonstrates that the framework outperforms alternative approaches in comprehensively understanding the strategies employed by players. \\

Secondly, the paper establishes a high correlation between the newly introduced metric, the Expected Probability of Shot On Target (xSOT), and existing metrics xG and average goal scored. This alignment of information indicates that xSOT provides valuable insights and accurate evaluations of players' actions, further affirming the effectiveness of the proposed approach. \\

Additionally, real-world examples from the World Cup 2022 and EURO 2020 are included to illustrate the practical applicability of the framework in analyzing shot-taking situations in professional football games. These examples showcase how the proposed approach can offer valuable insights for coaches, players, and analysts in real-game scenarios. \\

\item What papers by other authors make the most closely related contributions, and how is your paper related to them?\\

The deployment of game theory in our framework is influenced by two significant contributions in the analysis of penalty kick strategy. Palacios-Huerta (2003) demonstrated the application of game theory to analyze the interaction strategy between penalty-takers and goalkeepers. Building on this, Tuyls et al. (2022) extended the study by incorporating machine learning to cluster players and analyzing interaction strategies within each cluster. Although the methodologies differ, both studies have shown the feasibility of analyzing interactions between opposing agents in football.\\

Additionally, the rule-based shot block model and the concept of xOSOT are built upon the work of Spearman (2018) OBSO and Teranishi et al. (2023) C-OBSO. These studies focused on modeling off-ball player scoring opportunities. By enhancing and adapting these components, we can now model the off-ball player's probability of controlling the football in a hypothetical pass for xOSOT, and build the rule-based shot block model. These advancements allow for a more comprehensive and counterfactual analysis of shot-taking interactions in football, providing valuable insights into players' actions and decision-making processes.\\

\item Have you published parts of your paper before, for instance in a conference? If so, give details of your previous paper(s) and a precise statement detailing how your paper provides a significant contribution beyond the previous paper(s).\\

This manuscript has not been published elsewhere and is not under consideration by another journal.\\
\end{enumerate}
 \end{letter}


\end{document}