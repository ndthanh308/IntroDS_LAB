\section{Manipulators}\label{manipulators}

As a proof of concept, three different origami manipulators constructed from combinations of multiple Self-Lock Origami units are presented.
These manipulators can be divided into three categories: Rotational, Translation, and Modular manipulators.
These manipulator concepts can cover a wide range of motions while benefiting from the self-lock structure's properties, such as compactability, light weight, conserving energy, high-speed rotational motion, and large moment.

Each manipulator is discussed through the results of a kinematic simulation in which the angles of the involved origami joints are swept through a range of angles in order to obtain a desired motion.
Dynamics, including gravity, are outside the scope of this section, as are the details of any particular actuation scheme.
The rotational and translational manipulators were simulated for different values of the reduced central angle $\alpha$, and the trade-offs involved in choosing a particular value are discussed.


\subsection{Rotational Manipulator}

The rotational manipulator shown in \cref{rotationalManipulator} consists of two origami units, connected by a welded joint between one of the output plates of the first origami and one of the input plates of the second origami.
Connecting them on more than one plate could jeopardise their mobility; the connection method described here leaves each origami unit with a single degree of freedom.
There is a possibility of a collision between the first origami's output plates and the second origami's input plates while the manipulators move.
To avoid this, the plates are cut at a slight angle in those problematic areas (\cref{rotationalManipulator}Ab).
Cutting them does not affect the manipulator's movements because the plates' central angles are unaffected.
The motion of the manipulator is given in terms of the position of a designated end effector on one corner of the second origami's output plate, indicated in \cref{rotationalManipulator}A with the red dot labelled ``marker''.

% Figure environment removed

Both of the origami making up the manipulator are constructed in the downwards configuration.
\Cref{rotationalManipulator}A gives the coordinate system of manipulators and shows the curling motion in 3D.
\Cref{rotationalManipulator}B shows the planar projection of this movement onto the $x$-$z$, $y$-$z$, and $x$-$y$ planes.
This manipulator is designed to produce a simple rotation about the $x$ axis, so the end effector does not move in this direction.
However, due to the change in the geometry, the end effector moves further from the $x = 0mm$ line with decreasing $\alpha$.
The greater the reduction of the central angles (i.e. the further $\alpha$ is from $90\degree$), the shorter the range of motion on the y-axis.
The 89\degree\ and 85\degree\ degrees manipulators have similar and very close rotation around $x = 0mm$ due to their very small central angle cuts.
Due to defining the common MPF state at output angle $\gamma \coloneqq 36.5\degree$ for all manipulator models, their final states and positions in all plots are the same.

The five manipulators with different values of $\alpha$ follow the overall trajectory but with a different initial state, so they all have the same range of motion on the $z$-axis.
The larger their central angle cuts are, the bigger the inaccessible configuration area near their flat state.
Therefore, their semi-flat configuration (initial state) starts farther away in the range of motion.
The movements of different rotational manipulator models in the 3D space are demonstrated in figure S8 in the supplementary material.


\subsection{Translational Manipulator}

The translational manipulator shown in \cref{TranslationalManipulator} is constructed from four separate origami units alternating downward and upward configurations.
This is necessary in order to enable the ``zigzag'' state depicted in \cref{TranslationalManipulator}Aa.
To obtain the closest possible motion to a linear translational movement, the manipulator's geometry is required to be symmetrical.
\Cref{TranslationalManipulator}Ac shows the location of these connections and cuts.

% Figure environment removed

Similar to the rotational manipulator, the origamis are connected on only one of their plates to retain the full number of degrees of freedom, and some perimeter cuts are made to avoid overlapping between the origamis' movements.
This manipulator is the only case where plate length deviates from the previously fixed 25mm, as indicated in \cref{TranslationalManipulator}Ab.
The first and last plates have length 25mm, but the lengths of other plates were derived geometrically.
The MPF angle $\gamma \coloneqq 36.5\degree$ and the desired height $d = 25mm$ of the origami in the MPF state are used to find the plate lengths $f = d \cot \gamma$ and $q = d \sec \gamma$.
 
\Cref{TranslationalManipulator}A shows the initial and final states of the 89\degree\ manipulator's translational movement in different views.
All origami units start in a semi-flat state and must be actuated simultaneously in order to achieve the depicted straight-line motion.
The first and the last joint are folded until reaching a 90\degree\ output angle.
The second and third joints' final configurations are MPF.
Therefore, as for the rotational manipulator, varying $\alpha$ affects the initial but not the final position.

The two-dimensional plots of the manipulators' movements using a marker and defined reference frame are presented in \cref{TranslationalManipulator}C.
The main translational movement occurs along the $y$ axis, but the end effector does also move somewhat in the $z$ axis.
In the $y$-$z$ plane depicted in \cref{TranslationalManipulator}Ba, the 89\degree\ manipulator stays closest to the $z = 0mm$ line.
The farther the origami models get from $\alpha = 90\degree$, the further their semi-flat state gets from flat. As a result, the range of motion in the $z$ axis increases and the $y$ axis decreases, and their movements changes from a translational to a somewhat more rotational motion.
Among all the translational manipulators, the 89\degree\ model produces the closest to ideal translational movement.


\subsection{Modular Manipulator}

Both manipulators could be combined or their constituent origami units could be actuated in other permutations in order to obtain a combination of translational and rotational motions which could be used for complex tasks.
Using the origami design in a modular structure could provide the opportunity to create and mimic traditional robots' complex movements.
This could be useful in exploration, where various motions are required in a limited space.
This section explores the possibility of a modular manipulator which combines Self-Lock Origami units in an arbitrary way.

A modular manipulator can be built up from Self-Lock Origami units using two different basic construction cells, depicted in \cref{modularMani}A.
The first consists of two origami connected directly to each other with welded joints and small cutouts to avoid self-collision as discussed previously.
The second connects two origami via a rigid ``bounding plate'' interposed between them. 
This enables connecting the origami plates at different angles to change or increase the manipulator's workspace.
Bounding plates could be designed in different shapes to help the origami robots to explore different locations, but in the present work only considers a square plate equal in size to the origami units.
In the binding phase, constraints are defined for the movements in 3 directions or welded joints to create configurations 1 and 2.
Then, the last origami joint is connected to a base plate which serves as the ground link.

% Figure environment removed

An example of such a modular manipulator made of 89\degree\ degrees origami joints is presented in \cref{modularMani}.
By defining different rotational motions on the origami joints, the manipulator could take various shapes in the simulation.
\Cref{modularMani}B shows the manipulator's compact and semi-flat states with two 89\degree\ origami connected at a right angle using a square bounding plate.
In the semi-flat state, none of the origami have been activated, and the manipulator is in the closest possible configuration to the flat state.
In the compact state, all joints have been activated and folded to the MPF configuration to minimise the overall size of the manipulator.
\Cref{modularMani}Ba additionally indicates the location of the end effector and the coordinate system.

The plots in \cref{modularMani}C present various possible trajectories which could be taken by this modular manipulator in each of the two-dimensional projections.
Each trajectory corresponds to a different sequence of joint activation, indicated in the legend as a sequence of joint numbers.
Each joint in the sequence folds from the semi-flat to MPF state, with each fold stopping early if a collision occurs.
For instance, activated joints 1-2 refers to folding of joint 1 and then folding joint 2 without overlapping joint 1.
The joint activation orders 1-2-3 and 1-2-3-4 have similar motion since after joints 1, 2, and 3 are activated, joint 4 does not have much space for folding.

Across the possible joint folding orders shown, the marker exhibits a large range of motion and a variety of different directions of motion in all three planes.
If only the joints on one side of the bounding plate are activated, the end effector trajectory is a a straight line.
The joint combination 1-2 generates straight-line movement in both the $x$-$y$ and $x$-$z$ planes, whereas the joint combination 3-4 produced straight-line motion in the $y$-$z$ plane.

One of the advantages of using origami in modular manipulators is the ability to convert them into structures with small volumes due to the joints' geometries.
Manipulators can achieve this by moving their origami into MPF or semi-flat states in different directions.
Also, depending on the application, some of the manipulator's joints can be excluded in order to produce different manipulators.
This can even be enforced temporarily, without modifying the manipulator, by simply folding the joints using their actuators and holding them in that state.