
\begin{thenomenclature} 
\nomgroup{A}
  \item [{$\alpha$}]\begingroup Reduced value of central angles $\alpha_{12}$ and $\alpha_{23}$.\nomeqref {7}\nompageref{10}
  \item [{$\alpha_{12}$}]\begingroup Central angle of plate 1 of the origami, here set to $\alpha$.\nomeqref {7}\nompageref{10}
  \item [{$\alpha_{23}$}]\begingroup Central angle of plate 2 of the origami, here set to $\alpha$.\nomeqref {7}\nompageref{10}
  \item [{$\alpha_{34}$}]\begingroup Central angle of plate 3 of the origami, here always equal to $\frac\pi2$.\nomeqref {7}\nompageref{10}
  \item [{$\alpha_{41}$}]\begingroup Central angle of plate 4 of the origami, here always equal to $\frac\pi2$.\nomeqref {7}\nompageref{10}
  \item [{$\gamma$}]\begingroup Value of $\theta_4$ in the MPF state.\nomeqref {7}\nompageref{10}
  \item [{$\theta_1$}]\begingroup Input angle, the angular deviation of plates 1 and 2 from coplanar.\nomeqref {7}\nompageref{10}
  \item [{$\theta_2$}]\begingroup Angular deviation of plates 2 and 3 from coplanar.\nomeqref {7}\nompageref{10}
  \item [{$\theta_3$}]\begingroup Angular deviation of plates 3 and 4 from coplanar.\nomeqref {7}\nompageref{10}
  \item [{$\theta_4$}]\begingroup Output angle, the angular deviation of plates 1 and 4 from coplanar.\nomeqref {7}\nompageref{10}
  \item [{$D$}]\begingroup Width of the pouch actuator when flat.\nomeqref {7}\nompageref{10}
  \item [{$d$}]\begingroup Height of an origami cell in its MPF state in the translational manipulator.\nomeqref {7}\nompageref{10}
  \item [{$f$}]\begingroup Width of the horizontal components of the ``zigzag'' configuration of the translational manipulator.\nomeqref {7}\nompageref{10}
  \item [{$L$}]\begingroup Length of the inflated pouch actuator.\nomeqref {7}\nompageref{10}
  \item [{$L_0$}]\begingroup Length of the pouch actuator when flat, equal to $2L_p$.\nomeqref {7}\nompageref{10}
  \item [{$L_1$}]\begingroup Length removed from the far side of plates 1 and 2 by cutting.\nomeqref {7}\nompageref{10}
  \item [{$L_p$}]\begingroup Half-length of the pouch actuator when flat, equal.\nomeqref {7}\nompageref{10}
  \item [{$m$}]\begingroup Length of the sides of the input plate before cutting.\nomeqref {7}\nompageref{10}
  \item [{$M_\text{input}$}]\begingroup Input moment created directly by the pouch motor.\nomeqref {7}\nompageref{10}
  \item [{$M_\text{output}$}]\begingroup Output moment created by the pouch motor through the mechanical advantage of the origami.\nomeqref {7}\nompageref{10}
  \item [{$n$}]\begingroup Distance from the corner of the plate where the corner of the pouch actuator meets the edge of the plate.\nomeqref {7}\nompageref{10}
  \item [{$P$}]\begingroup Fixed pressure within the pouch actuator.\nomeqref {7}\nompageref{10}
  \item [{$q$}]\begingroup Length of the two plates making up the diagonal in the ``zigzag'' configuration of the translational manipulator.\nomeqref {7}\nompageref{10}
  \item [{$S$}]\begingroup Angle subtended by one arc-shaped side of the inflated pouch actuator.\nomeqref {7}\nompageref{10}
  \item [{MA}]\begingroup Mechanical advantage of the origami structure, defined as $M_\text{output} / M_\text{input}$.\nomeqref {7}\nompageref{10}
  \item [{MPF}]\begingroup Maximum Possible Fold, the state where $\theta_4 = \gamma$.\nomeqref {7}\nompageref{10}

\end{thenomenclature}
