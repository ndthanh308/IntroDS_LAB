\section{Conclusions}\label{conclusions}

Origami structures could be a great substitute for traditional joints where space limitation is the main issue. They could adapt their size and shape based on the available space.
These computer simulations and mathematical models illustrate the origami structure's high performance. In different origami states, their motion's speed and output moment are more significant compared to a simple fold with the same actuator.
Increasing the thickness of the origami plates would increase the mechanical strength of the structure, but decrease its flexibility. On the other hand, increasing the soft joint length between the plates increases flexibility while decreasing strength. The trade-off between these characteristics is a potential direction of future research.

As proof of concept, different types of manipulators have been developed and simulated using the origami joints. They could produce a variety of rotational and translational motions depending on their origami's central angles and configurations.
The proposed joint could potentially have many industrial applications, especially in space where the weight and volume of the tools matter the most. It could generate various rotational and translational motions depending on the central angle while maintaining a very lightweight and low fabrication cost due to the flat-foldability properties (simplicity of sheet-like structures for mass production).

One limitation of this work is the assumption of a revolute joint in the design's analytical modelling. This could result in discrepancies between the experimental and the modelling results. Future work will focus on the development of methods and designs for soft joint production to mitigate the unpredictability of soft materials.
This paper has made the assumption that joint angles can be set to a desired value; for physical prototypes, a control scheme is required which combines the kinematic model with sensing and feedback based on the desired end-effector position of the manipulator.
Finally, future work will develop methods and experiments to take advantage of this origami joint for real-life applications.