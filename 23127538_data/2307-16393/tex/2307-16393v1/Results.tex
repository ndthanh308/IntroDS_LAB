\section{Model Results}\label{results}

This section demonstrates the kinematics of the Self-Lock Origami as well as its input and output moment, including how these quantities and their relationships change with $\alpha$.
\Cref{motion} shows the kinematics of the two configurations of the origami.
\Cref{motion}b displays $\theta_4$ as calculated by solving \cref{theta4}, \cref{motion}c shows $\theta_2$, which is equal to $\theta_4$ by symmetry, and \cref{motion}d shows $\theta_3$ by solving \cref{theta3}.

% Figure environment removed

The closer $\alpha$ is to $90\degree$, the steeper the slope of the output angle $\theta_4$ in the regime where $\theta_1$ is near zero.
The total observed range in the output angle $\theta_4$ is about $180\degree$ regardless of the value of $\alpha$.
The range of $\theta_3$, on the other hand, changes significantly as $\alpha$ decreases.
This is shown in \cref{motion}d, where the curves for both configurations for various values of $\alpha$ are symmetrical about local extrema at $\theta_1 = 0$.
As the angular deficit increases, this extreme value moves further from zero, whereas the overall range and the slope of $\theta_3$ decrease.

 %%% analyze Moment (input output together)

\Cref{moment} illustrates the theoretical curves \cref{T1,MechanicalAdvantage} of input moment and mechanical advantage as a function of $\theta_1$ for different values of the reduced central angle $\alpha$.
Also depicted is the output moment $M_\text{output} = \text{M\!A}\,M_\text{input}$.
For the moment study, only one of the origami pouches is considered, since in the typical use case only one pouch will be inflated at a time.
The top origami pouch is activated for the downward movement, and for the upward movement, the bottom pouch is.
Also, the direction of input and output moments are opposite (\cref{moment}ABa, Downward and upward movements).

% Figure environment removed

The figure displays curves corresponding to a variety of different values of $\alpha$.
Due to the symmetry between the downward and upward configurations of the Self-Lock origami, both the input and output moment differ only by sign between the two configurations.
As the input angle changes from $-90\degree$ to 0\degree\ degrees, the absolute value of the input moment changes from zero to its maximum value at $\theta_1 = 0\degree$ degrees since the pouch motors work has not been converted into any motion in this state.
As the origami rotates and the input angle gets farther from 0\degree\ degrees, the absolute value of the input moment decreases.
The shape of this decrease is symmetric about the $\theta_1 = 0\degree$ line.
These changes are due to observing the maximum rotational motion near the 0\degree\ degrees input angle and the small input moment values near $\theta_1 = 90\degree$ or $-90\degree$.

Note that the figure does not consider the effect of varying the pouch pressure $P$.
The input moment given by \cref{T1} depends only linearly on $P$, so all analysis is performed at a constant value $P = 10kPa$.
The higher the pressure, the bigger the input and output moments are.
However, large pressure values could jeopardise the inextensibility assumption for the pouch motor's material and result in model failure.
Thus the limits of this model's applicability are determined by the material properties of the pouch motor, which must be constructed from a material with a high Young's modulus while preserving flexibility properties.

Both downward and upward movements have the same mechanical advantage $\text{M\!A}$ since they are only different in their moment's directions.
As the input angle increases from negative to positive the mechanical advantage decreases and then increases in a concave shape.
The closer plates 1 and 2 get to the flat state ($\theta_1 = 0\degree$ degrees), the closer the mechanical advantage is to zero since the output moment goes to zero.
Crucially, the mechanical advantage is smallest near $\theta_1 = 0\degree$, exactly where the slope of $\theta_4$ with respect to $\theta_1$ is greatest.
Therefore, there is a trade-off between the speed and the moment of the origami's movement.
The closer $\alpha$ is to 90\degree, the more this trade-off is taken, with the speed of the origami near zero and the output moment far from zero both increasing dramatically.
However, the limiting value is not physically reasonable: at $\alpha = 90\degree$ the origami does not function.
Similarly, the slope and the mechanical advantage both approach infinity in this limit due to division by zero.
Therefore, in the remainder of the paper, the value of $\alpha$ where not otherwise stated shall be taken to be 89\degree, maximising these quantities as well as the flat-foldability of the structure while remaining within a practically realisable regime.