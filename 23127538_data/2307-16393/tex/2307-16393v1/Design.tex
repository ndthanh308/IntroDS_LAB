\section{Design}
\label{design}

The Self-Lock Origami is a rigid origami whose central angles sum to less than 360\degree.
The design consists of four solid adaptable plates connected by four joints (\cref{probFabricMov}C), three of which are passive and one of which will be actuated.
\Cref{probFabricMov}B shows the design and modelling process.
First, square-shaped origami plates of side length 25mm are defined. For modelling purposes, these are treated as having zero thickness.
Then, the central angles of two of the plates are reduced by drawing a line from one corner of the square (\cref{probFabricMov}B) with a specified angle $\alpha$ and extending the line until it intersects with the other edge of the square.
Revolute joints were then defined at the edges of adjacent plates, indicated with edge colour in \cref{probFabricMov}B.
These revolute joints are a first-order approximation to the kinematics of a flexible fold line.


\subsection{Motion Simulation}\label{simulation}

For origami assembly, joint constraints are necessary.
Various shapes are created by cutting angles in different origami plates' positions:
any of the origami's four central angles can be reduced in order to achieve the self-locking property.
Although \cref{probFabricMov} demonstrates only one of these, 16 different configurations were considered: four where the central angle was subtracted from both sides of a single fold line (\cref{L11}), and twelve other configurations shown in the supplemental material.
Note that only cuts reducing the central angles are considered, as an origami could have various shapes around its edges without compromising its motion \cite{zare2021design}.

% Figure environment removed

The structure of the grounded plate 1, connected to plate 2, is equivalent to a simple fold (\cref{motion}a). Therefore, the origami's input angle, $\theta_1$, and input moment are the same as the rotational angle and the output moment of a simple fold structure with the same actuator. Different actuators could be used to increase the input angle limits. However, it is not necessary since the output angle changes most for $\theta_1$ near zero (\cref{motion}).
Both the semi-flat and the MPF state occur at values of $\theta_1$ near $90\degree$.
Thus, in order to obtain the full range of motion, there must be pouch motors on both sides of plates 1 and 2.

Various motion simulations were implemented using the contact solver of Autodesk Inventor 2019 to find an origami that could maintain a flat configuration between its input plates at its initial state and obtain the maximum rotational movement (optimal performance).
Achieving a flat state by the origami's input plates could offer a more simplified model and better control by the pouch motor's actuator.
The pouch motor will be able to maintain a zero pressure state (no inflation) and provide a known shape and dimensions at the initial state for the modelling.

In all simulations, plate 1 is a grounded link at a fixed position with its outer corner at the origin. Gravity is neglected for simplicity and consistency with the analytical model.
The kinematics of the mechanism are simulated by sweeping the input angle between what the ``semi-flat'' and ``maximum possible fold'' (MPF) state.
The semi-flat state is defined as the state where the absolute sum of all the angles between origami plates is at its maximum.
Since the origami models to have zero thickness, they are kinematically capable of achieving fold angles more extreme than a real origami, so the maximum possible fold state is chosen to set the angle $\theta_4$ between the input and output plate to the constant MPF angle $\gamma \coloneqq 36.5\degree$.

\Cref{L11} shows origami types defined by different cutting positions. The green and yellow plates are input and output plates respectively. The ``initial state'' pictured in \cref{L11} shows the closest angle that origami's input plates could get to 180\degree (flat form), and the ``final state'' is the MPF state.
The motion of the outer corners of the input and output plate is shown using markers 1 and 2.

In order to obtain an origami which is symmetrical and also capable of reaching a state with fully flat input plates, origami type a is used in the remainder of the paper.
If the flat state of the input plates is not considered, all the origami types provide almost the same amount of rotational motions in slightly different directions.
This is discussed in the supplemental material (together with an expanded version of \cref{L11}) using an algorithm described previously \cite{zare2021design}.
Additional fold types not pictured were also considered. Certain of these types could also reach a state with flat input plates, but type a is used due to its symmetry, which simplifies modelling.