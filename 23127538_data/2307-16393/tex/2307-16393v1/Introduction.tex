\section{Introduction}

%%%%%%%%%%%%problem%%%%%%%%%%%%%

Origami structures have the potential to enhance robotics in various ways. They can help save space \cite{jasim2018origami,tang2014origami,arya2017crease}, reduce energy consumption \cite{zhai2020situ,quaglia2014balancing,ye2022novel}, decrease production time and cost \cite{dai2010origami,onal2011towards,zhakypov2015design,yang2017smartphone,kimionis20153d} by offering a compact and lightweight structure, and having a simple flat structure that is easy to produce. Hence, by replacing conventional robotic structures with origami structures, robots could take advantage of these benefits. This paper focuses on an origami rotational joint structure that could substitute the traditional revolute joint \cite{norton2008design} while offering the advantages of origami.

A variety of origami robots, ranging from legged walkers \cite{rus2018design,mehta2014cogeneration,mehta2014end,kohut2013precise,haldane2013animal,zhakypov2018design} to grippers and more \cite{rossiter2014kirigami, firouzeh2017grasp, geckeler2022bistable, chan2017design, suzuki2020origami}, make use of the simple fold as a rotational element in order to build up more sophisticated robots.
The simple fold is ideal for these purposes because it is simple and compact.
Tendon-driven robogami \cite{firouzeh2017under} joints with adjustable stiffness using SMP (shape memory polymer) layer could rotate along multiple axes. The joints are flat-foldable and modular. However, the rotational motion has limited range which depends on the solid panels' thickness and the flexible joint between the panels.

Many rotational joints have been developed beyond the simple fold; some can create limited rotation in multiple different directions \cite{koh2012omega,boyvat2017addressable,salerno2016novel}.
Pneumatic origami rotational actuators can produce significant force, but are large and cannot be flat-folded for deployability \cite{yi2018customizable}.
Foldable designs for hinge and pivot joints have been proposed \cite{sung2015foldable}.
These joints can be combined to generate desired kinematics, but they are relatively complicated and cannot be flat-folded.
Curved patterns can add stiffness to an origami structure capable of rotational motion \cite{zhai2020situ,taylor2019mr,baek2020ladybird,saito2017investigation}.
A curved kirigami joint has been shown to achieve rotational motion up to almost 17\degree\ in multiple directions \cite{qiu2021design}. In general, curved surfaces generate a low range of rotational motion compared to simple folds. Also, the fold action is impossible due to their geometrical constraints (no fold line).

A potential candidate to substitute the traditional rotational linkage in an origami robot is the Four-Vertex Origami, which generates 8.4 times more rotational angle than a simple fold for a given actuator displacement \cite{zare2021design}.
This is crucial when using pouch actuators, which pair naturally with flat-foldable or semi-flat-foldable origami due to their flat shape when deflated, but have a limited range of angular movement \cite{niiyama2015pouch}.
It can be manufactured in different shapes and sizes all the way down to the thickness of a sheet of paper, allowing its weight and size to be adjusted based on the application.
Furthermore, it can be assembled with other origami joints to create complex movements \cite{brown2022approaches}.
However, such an origami is underactuated when folded from flat, and in fact has a degenerate state equivalent to a simple fold (\cref{probFabricMov}A).
The uncertainty of which fold line will be activated when the pouch motor is inflated increases the chance of failure in the joint's rotation.

% Figure environment removed

Faber et al.\cite{faber2018bioinspired} designed an earwig-inspired spring origami joint similar to the four-vertex origami but with central angles around the common vertex summing to less than 360\degree.
The flexibility and extensibility of this joint provides a self-lock mechanism during the insect's flight.
However, there is an inherent trade-off in the use of non-rigid origami:
the joint spring must be matched precisely to the task,
and the stretchability of the fold lines declines over time and with every use.
Furthermore, depending on the number of folds, the width of flexible fold lines could be large, resulting in a weaker structure.

This paper proposes a new rigid origami design, the Self-Lock Joint, that combines the Four-Vertex origami \cite{zare2021design} and the earwig-inspired joint \cite{faber2018bioinspired}.
Reducing the central angles of the origami eliminates the degenerate state of the Four-Vertex origami.
Combining with the assumption of a fold line with non-extensible material and zero gaps, the joint could have a deployable structure, save energy, provide a state-lock mechanism, great moment, and downward and upward rotational movements of the structure.
% This paper is organized as follows: \nameref{design} explains the CAD modeling of origami plates, assembly, and joint constraints, while \nameref{modelings} develops mathematical models of motion, actuator's pressure, and the moment relations for the joint. \nameref{simulation} and \nameref{results} demonstrate and discuss the motion simulations and the mathematical models' results, followed by the conclusions in \nameref{conclusions}.

