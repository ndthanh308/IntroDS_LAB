\begin{abstract}
Origami structures have been widely explored in robotics due to their many potential advantages. Origami robots can be very compact, as well as cheap and efficient to produce. In particular, they can be constructed in a flat format using modern manufacturing techniques. Rotational motion is essential for robotics, and a variety of origami rotational joints have been proposed in the literature. However, few of these are even approximately flat-foldable. One potential enabler of flat origami rotational joints is the inclusion of lightweight pneumatic pouches which actuate the origami’s folds; however, pouch actuators only enable a relatively small amount of rotational displacement. The previously proposed Four-Vertex Origami is a flat-foldable structure which provides an angular multiplier for a pouch actuator, but suffers from a degenerate state. This paper presents a novel rigid origami, the Self-Lock Origami, which eliminates this degeneracy by slightly relaxing the assumption of flat-foldability. This joint is analysed in terms of a trade-off between the angular multiplier and the mechanical advantage. Furthermore, the Self-Lock Origami is a modular joint which can be connected to similar or different joints to produce complex movements for various applications; three different manipulator designs are introduced as a proof of concept.
\end{abstract}

\keywords{
Origami, Rotational joint, Deployable structure, Modular, Crease pattern, Earwig wing, Miura-ori, Spherical mechanisms, Pouch motors, Manipulator
}
