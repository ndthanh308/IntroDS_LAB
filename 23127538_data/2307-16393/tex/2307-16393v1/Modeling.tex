\section{Modelling}\label{modelings}

The Self-Lock Origami is a single-vertex origami with four fold lines, meaning that it has a single degree of freedom \cite{hull2002modelling, hull2002combinatorics, song2016microscale}.
As shown in \cref{probFabricMov}C, the same components can be assembled into two distinct configurations whose motions are vertical mirrors of each other.
These configurations can be used in applications to achieve motion in two different directions.

In this section, the relations between the origami's angles, output moment, input moment, and the pressure in a pouch actuator are developed.
Pouch actuators are a practical choice for the Self-Lock Origami because they are planar and compact.
They can be treated as a layer of the origami during layer-by-layer construction of an origami robot by programmable machines \cite{niiyama2015pouch}.
The mechanical work of the pouch motor inflation converts into the deformation of its shape and changes in its curvature length. These cause angular motions in the simple fold (a hinged structure) that it has been attached to \cite{niiyama2015pouch}.


\subsection{Origami Structure as a Spherical Mechanism}

The kinematics and dynamics of single-vertex rigid origami can be studied by noting that they are mechanically identical to spherical mechanisms \cite{bowen2014position, zare2021design}, for which a detailed theory exists \cite{chiang1988kinematics}.
To do this, the origami folds between the plates are treated as revolute joints.
Then, since each plate of a single-vertex origami is assumed to be inflexible, the single vertex acts as a fixed centre about which the four plates move as the bars of a spherical four-bar mechanism.

The first quantity of interest is the kinematic relationship between the input angle $\theta_1$ and output angle $\theta_4$.
The kinematics of a spherical four-bar mechanism are determined entirely by the angular lengths $\alpha_{12}$, $\alpha_{23}$, $\alpha_{34}$, and $\alpha_{41}$ of its links.
Given these quantities, the input and output angles are related by the following equation \cite{chiang1988kinematics}:

\begin{multline}
    \cos\alpha_{23} \cos\alpha_{41} \cos\alpha_{12}
    - (\sin\alpha_{23} \cos\alpha_{41} \cos\theta_4
    \\
    + \cos\alpha_{23} \sin\alpha_{41} \cos\theta_1) \sin\alpha_{12}
    \\
    + \sin\alpha_{23} \sin\alpha_{41} (\sin\theta_1\sin\theta_4
    - \cos\theta_1\cos\theta_4\cos\alpha_{12})
    \\
    = \cos\alpha_{34}
\end{multline}

However, the structure of the Self-Lock Origami allows this equation to be simplified significantly.
First, the angles $\alpha_{41}$ and $\alpha_{34}$ are both equal to 90\degree, which zeros several terms and simplifies others.
The symmetry of the system allows the parameter $\alpha \coloneqq \alpha_{12} (= \alpha_{23})$ to be defined, so several more terms can be simplified and combined, resulting in a simple relation between $\theta_1$ and $\theta_4$.

\begin{equation}\label{theta4}
\tan\theta_4 = \cos\alpha \cot\left(\frac{\theta_1}{2}\right)
\end{equation}

Besides the relation between the input and output angles, the kinematics of the other angles $\theta_2$ and $\theta_3$ may also be of interest.
Due to the symmetry of the origami, $\theta_2 = \theta_4$, but $\theta_3 \not= \theta_1$.
Instead, $\theta_3$ is calculated using the following relation \cite{yang1964application}:

\begin{equation}\label{theta3}
\cos \theta_{3} = \sin^2 \alpha \cos\theta_1 - \cos^2\alpha
\end{equation}

Up to angular equivalence, the kinematic \cref{theta4,theta3} each have two distinct solution curves, corresponding to the two configurations of the Self-Lock Origami.
In the up configuration, $\theta_4$ ranges over positive angles from near zero to almost $180\degree$, whereas in the down configuration $\theta_4$ ranges over positive angles from near zero to almost $-180\degree$.
In computations, the atan2 function is used to ensure that the returned value of $\theta_4$ corresponds to the correct configuration, as the single-argument $\tan^{-1}$ would result in a discontinuous curve.
Using $\cos^{-1}$ to calculate $\theta_3$ returns a single continuous positive curve, so the result is simply multiplied by $-1$ in the down configuration.


\subsection{Input and Output Moments}

We model the pouch motor as non-extensible with zero bending stiffness.
If fully inflated while detached from the origami, it would take a cylindrical form with height $D$, but it is affixed to the origami in its flat state as shown in \cref{pouch}, resulting in a rectangular shape with width $L_0$ and height $D$.
This length is split between plates 1 and 2, so we also define $L_p \coloneqq L_0 / 2$.
The dimensions are chosen to fit the available space on the cut input plates, as shown in \cref{pouch}.

% Figure environment removed

The moments that have been applied to (input moment) or generated by (output moment) the origami are modelled under the assumption of constant air pressures on the system's pouch motor. The moments could be calculated using the origami's input angle. A range of $-90\degree$ to 90\degree\ degrees has been selected for the input angle, $\theta_1$, due to the physical limitation of the pouch motor to produce even near 90\degree\ degrees rotational angle. Both upward and downward movements can be created by adding two pouch motors on the origami's input plates:
% (\cref{pressure}: Drawings - pouch1 and pouch2):
one on top for $\theta_1:(0\degree,90\degree)$ range of motion, and another one on the bottom for $\theta_1:(-90\degree, 0\degree)$ range of motion.

%%%%%%finding input moment
To estimate the input moment applied to a simple fold (a hinged structure), Niiyama et al. developed \cref{T1} \cite{niiyama2015pouch}. The Self-lock origami's input plates have a structure similar to a simple fold, so this equation can be used to calculate its input moment. Based on the proposed origami structure, the equation relies on the origami's input angle $\theta_1$ and $\alpha_{12}$. Given the fixed pressure $P$ as well as the width $D$, half-length $L_p$, and central angle $S$ of the pouch, the input moment can be calculated as follows:

\begin{multline}\label{T1}
        M_\text{input} = \frac{L_p^2 D P}{2S^2}\bigg(-1 + S^2 + \cos 2S \\
                       -\sqrt{2} \cos S \sqrt{-1 + 2S^2 + \cos 2S}\bigg)
\end{multline}

Here $S$ is derived from the assumption that the surface of the pouch acts as a section of a cylinder with constant curvature. The central angle is the angle subtended by the arc of this surface, which depends on $\theta_1$ and can be approximated well for $\theta_1 \in [-\frac\pi2, \frac\pi2]$ as follows \cite{sun2015self}:

\begin{equation}\label{phi}
    \begin{aligned}
        &S(\theta_1) = \sqrt{6(1 - L(\theta_1)/L_0)} \\
        &\text{where }L(\theta_1) = L_0 \sqrt{2(1 + \cos\theta_1)}
    \end{aligned}
\end{equation}

$M_\text{input}$ and $S$ also depend implicitly on the dimensions of the rectangular uninflated pouch.
All four plates of the origami were originally squares of side length $m = 25mm$, but the reduction of central angles to $\alpha$ on the input plates reduces the space available for the pouch, as illustrated in \cref{pouch}.
The length of the top side is reduced by a length $L_1 = m \cot\alpha$, and the misalignment of the right angles of the plates and the pouch means that the top right corner of the pouch touches the side of the plate at one point a distance $n = (m - L_1)\cot\alpha$ from its corner. The pouch length $L_p = L_0 / 2$ and width $D$ are then computed geometrically as follows:

\begin{equation}
    \begin{aligned}
        L_p &= (m - L_1) \csc\alpha = m(1 - \cot\alpha) \csc\alpha \\
        D &= (m - n) \csc\alpha \\
          &= m \csc\alpha (1 - \cot\alpha(1 - \cot\alpha))
    \end{aligned}
\end{equation}

Finally, the mechanical advantage of a generic spherical four-bar mechanism has been shown to be calculated as follows (adapted to a different naming convention for central angles) \cite{yang1965static}:

\begin{equation}\label{MechanicalAdvantage}
    \text{M\!A} = \frac{M_\text{output}}{M_\text{input}}
    = \frac{\sin\alpha_{41} \sin\theta_3}{\sin\alpha_{23} \sin\theta_2}
    = \frac{\sin\theta_3}{\sin\alpha \sin\theta_2}
\end{equation}
