\section{Total domination and $cd$-coloring in triangle-free $d$-regular Graphs}
\label{sec:hardness}
In this section, we prove Theorem~\ref{thm:regular}, i.e. we derive the hardness result for \CDC\ in triangle-free $d$-regular graphs for each fixed integer $d\geq 3$. As a corollary of this result, we obtain the hardness result for \TD\ in triangle-free $d$-regular graphs for each fixed integer $d\geq 3$. 
Hence, first, we propose a simple linear reduction for \CDC\ on triangle-free cubic graphs. Then, we generalize this reduction for triangle-free $d$-regular graphs for each fixed integer $d\geq 3$, by using two linear reductions: one for odd values of $d$ and the other for even values.

The Exponential-Time Hypothesis (ETH) along with the Sparsification   Lemma implies that \TSAT\ cannot be solved in subexponential-time, i.e.
in time $2^{o(n+m)}$, where $n$ is 
the number of variables and $m$ is the number of clauses in the input formula.
 To prove that a problem does not admit a subexponential-time algorithm, %(algorithm runs in $2^{o(n)}$ algorithm, where $n$ is the size of input instance), 
 it is sufficient to obtain a linear reduction from a problem 
which does not admit a subexponential time algorithm where a linear reduction is a polynomial-time reduction
in which the size of the resultant instance is linear in the size of the input instance.
We refer to Chapter 14 of the book~\cite{CyganOthParamAlgo} for more details about these concepts.
All reductions mentioned in this section are trivially linear.

To prove Theorem~\ref{thm:regular}, we use three constructions. 
Construction~\ref{cons:cubic} is used for a reduction from \textsc{Total Domination}  on bipartite graphs with bounded degree 3 to \CDC\ on triangle-free cubic graphs. Construction~\ref{cons:d-regular odd} is used for a reduction from \CDC\ on triangle-free $(d-2)$-regular graphs to \CDC\ on triangle-free $d$-regular graphs, for each odd integer $d\geq 5$, whereas the Construction~\ref{cons:d-regular even} is used for a reduction from \CDC\ on triangle-free $(d-1)$-regular graphs to \CDC\ on triangle-free $d$-regular graphs, for each even integer $d\geq 4$. 

% Figure environment removed
\begin{construction}
\label{cons:cubic}
    Let $G$ be a bipartite graph with bounded degree 3. We construct a gadget $W$ in the following way: Introduce a star graph $K_{1,2}$, which we denote by $S$, having $a$ as the root vertex, and $b,c$ as the leaves. Note that we also call $`a$' as the root vertex of $W$. Then we introduce two copies of the complete bipartite graph $K_{2,2}$, denoted by $CB_1$ and $CB_2$, respectively. Now the leaf $b$ (resp. $c$) of $S$ is adjacent to one of the partition $A_1$ (resp. $A_2$) of $CB_1$ (resp. $CB_2$).
    Furthermore,  each vertex of partition $B_1$ of $CB_1$ is adjacent to exactly one vertex of the other partition $B_{2}$ of $CB_{2}$  as shown in Figure~\ref{fig:cubic_gadget}. 
    Thereafter we construct a graph $G_c$ from $G$ using the gadget $W$ as follows: for every vertex $u$ of $G$ with degree 2, introduce a gadget $W$ such that the root vertex $a$ of $W$ is adjacent to $u$ in $G$. Similarly, for each  vertex $v$ in $G$ of degree 1, we introduce  two copies of  $W$ such that the root vertices of both the gadgets  are adjacent to $v$ in $G$.   
    The graph $G_c\setminus G$ contains $11x+22y$ vertices, where $x$ and $y$ denote the number of vertices in $G$ having degree 2 and degree 1, respectively. 
\end{construction}

 % Figure environment removed  

An example of the Construction~\ref{cons:cubic} corresponding to the bipartite graph $G$ with bounded degree 3 given in Figure~\ref{fig:bipartite} is as shown in Figure~\ref{fig:Cubic construction}.

 % Figure environment removed

Let $W$ be a gadget constructed in Construction~\ref{cons:cubic}. Let $H$ be a subgraph of $W$ induced by the vertices in two complete bipartite graphs $CB_1$ and $CB_2$, and the two leaves $b$ and $c$ of the star graph $S$.  Let $H'=H-\{b,c\}$. The following observation is true for the gadget $W$ in Construction~\ref{cons:cubic}.
\begin{observation}
    \label{obs:Hcoloring_cubic} 
     Let $W$ be the gadget defined in Construction~\ref{cons:cubic}.
     Let $H$ be a subgraph of $W$ induced by the vertices in two complete bipartite graphs $CB_1$ and $CB_{2}$, and the two leaves $b$  and $c$ of the star graph $S$. Let 
     $H'=H- \{b,c\}$. In any CD-coloring of $H$, the vertices in $H'$ require at least 4 colors. (Refer Figure~\ref{fig:cubic_gadget} for $H$ and $H'$). Moreover, there exists a $cd$-coloring of $W$ using exactly 4 colors.
\end{observation}

\begin{proof}
    The graph $H'$ contains an induced $C_6$ (for instance, the graph induced by the vertices $d,i,j,f,k,h$ as shown in Figure~\ref{fig:cubic_gadget}). It is easy to see that that the \XCD($C_6$)=4. Now consider the subgraph $H'$. It is obtained by 
     introducing new vertices adjacent to some vertices of the $C_6$.  Since none of the $3K_1$s present in the $C_6$ is dominated by any of the remaining vertices in $H'$, by Observation~\ref{obs:cdcolor_reduce}, it is clear that the number of colors needed for any $cd$-coloring of $H'$ is at least 4.  Further, note that all the independent sets in $H'$, which are dominated by $b$ or $c$, are already dominated by some vertex in $H'$. Therefore, again by Observation~\ref{obs:cdcolor_reduce}, it is clear that the number of colors needed for $V(H')$ in any $cd$-coloring of $H$ is at least 4. 
     Since the root vertex $a$ of the gadget is not adjacent to any of the vertices in $H'$, the presence of $a$ will not reduce the number of colors needed for $cd$-coloring of the subgraph $H'$ of the gadget $W$. Hence the number of colors needed for $cd$-coloring of $W$ is at least $4$. 
    Now it is easy to see that a 4-$cd$-coloring of any $C_6$ in $H'$ can be  extended to a 4-$cd$-coloring of the gadget $W$ (by using the leaves of star graph $S$, and one of the vertices from each $CB_i$, for $i=\{1,2\}$ as the dominating vertices of the color classes).
\end{proof}



The following lemma is based on the Construction~\ref{cons:cubic}, and as a consequence of which we have Theorem~\ref{thm:cubic}.

\begin{lemma}
    \label{lem:cubic}
    Let $G$ be a bipartite graph with bounded degree 3. Then $\gamma_t(G)=k$ if and only if $\XCD(G_c)=k+4x+8y$, where  $x$ and $y$  denote the number of vertices in $G$ having degree 2 and degree 1, respectively. 
\end{lemma}

\begin{proof}
    Let $\gamma_t(G)=k$. Then by Proposition~\ref{pro:triangle-free}, since $G$ is triangle-free, we have $\XCD(G)=k$. We claim that $\XCD(G_c)=k+4x+8y$.  Recall that the graph $G_c$ contains $x+2y$ gadgets, as there are $x$ number of degree two vertices, and $y$ number of degree one vertices in $G$. By Observation~\ref{obs:Hcoloring_cubic}, it is clear that these gadgets require exactly $4(x+2y)$ new colors, which are not used in $G$ (as none of these colors can be reused for coloring the vertices in $V(G_c)\setminus V(W)$).  Since \XCD($G$)=$k$, we then have \XCD($G_c$)=$k+4x+8y$.
    
\smallskip

    Now for the reverse direction, let $\XCD(G_c)=m$. Then, we claim that $\gamma_t(G)=k$, where $k=m-(4x+8y)$. 
    Note that for each gadget $W$, the subgraph $W- \{a\}$ is disjoint from $G$, as only the vertex $a$ in $W$ is adjacent to exactly one vertex  $u$ in $G$. Hence the color class dominated by $a$ can include only one vertex $u$ in $G$.  By Observation~\ref{obs:Hcoloring_cubic}, in any $cd$-coloring of $G_c$, the vertex set $H'$ of any gadget $W$ needs at least 4 colors, and none of these colors can be reused for coloring the vertices in $V(G_c)\setminus V(W)$. Again by Observation~\ref{obs:Hcoloring_cubic}, the same 4 colors used to color the vertices in $H'$ can be reused to color the entire vertices of $W$. Therefore, we now have a $cd$-coloring of $G_c$ using $m$ colors such that no colors used for the vertices in gadgets are used to color any vertex in $G$. Thus, $\XCD(G)=m-4(x+2y)=k$. Then by Proposition~\ref{pro:triangle-free}, $\gamma_t(G)=k$.
\end{proof}





\begin{theorem}
    \label{thm:cubic}
    \CDC\ is \NPC\ on triangle free cubic graphs. \TETHS.
\end{theorem}

\begin{proof}
    Notice that $G_c$ has $|V(G)|+ 11(x+2y)$ vertices, where $x$ and $y$ are the number of vertices in $G$ having degree 2 and degree 1, respectively. We know that there is a linear reduction from \textsc{Total Domination} in bipartite graph with bounded degree 3 to \CDC\ on triangle free cubic graphs due to Lemma~\ref{lem:cubic}. Thus, by using Proposition~\ref{pro:bipartite}, we are done.
\end{proof}




Now, we generalize  Construction~\ref{cons:cubic} to prove the hardness of \CDC\ on triangle-free $d$-regular graphs, for any constant $d\geq 4$. The following construction is used to prove the hardness of \CDC\ on triangle-free $d$-regular graphs, for any odd integer $d\geq 5$. 
 \begin{construction}
    \label{cons:d-regular odd}
     Let $G$ be a triangle-free $(d-2)$-regular graph, for any odd integer $d\geq 5$. 
     We construct a graph $G_c$ from $G$ %using a gadget $W$ 
    in the following way: 
%    \vspace{-0.25cm}
    \begin{itemize}
        \item First we construct a gadget $W$ as follows:  introduce a star graph $K_{1,d-1}$, which we denote by $S$, having the vertex $a$ as the root vertex. Note that we also call `$a$' as the root vertex of $W$.
        Further introduce $2(d-1)^2$ vertices which induces $d-1$ disjoint copies of complete bipartite graphs $K_{d-1,d-1}$, namely $C_{1},  C_{2}\ldots,C_{d-1}$, respectively.      
        The adjacency between these complete bipartite graphs $C_i=(A_i,B_i)$, for $1\leq i\leq (d-1)$, and the star graph $S$ is in such a way that the vertices of $A_i$ of the complete bipartite graph $C_i$ is adjacent to the leaf vertex, say, $v_i$  of $S$.
        Furthermore, for each odd integer $i\leq (d-2)$, each vertex in the partite set $B_i$ of $C_i$ is adjacent to exactly one vertex of the partite set $B_{i+1}$ of $C_{i+1}$ (for instance refer  Figure~\ref{fig:d-regular_gadget odd}, when $d=5$). Now, by $H_i'$, we denote the subgraph of $W$ induced by the vertices in the two complete bipartite graphs $C_i$ and $C_{i+1}$, for an odd integer $i\leq d-2$.  Then, by $H_i$, we denote the subgraph of $W$ induced by the vertices in $H_i'$ and the two leaves $v_i$ and $v_{i+1}$ of the star graph $S$ which are adjacent to the partitions $A_i$ and  $A_{i+1}$ of $C_i$ and $C_{i+1}$, respectively. Note that $H_i'=H_i-\{v_i,v_{i+1}\}$. The gadget $W$ contains $(2d^2-3d+2)$ vertices.  
     
     \item Now the graph $G_c$ is constructed from $G$ using $W$ in such a way that for each vertex $u$ in $G$, introduce two copies of $W$ such that root vertices of both the copies of $W$ is attached to $u$. 
     \end{itemize}
     Note that  $G_c-G$ contains $2n(2d^2-3d+2)$ vertices, where $n$ is the number of vertices in $G$.
\end{construction}

An example of the gadget $W$ in Construction~\ref{cons:d-regular odd} is shown in Figure~\ref{fig:d-regular_gadget odd}. We have the following observation for the gadget $W$ of Construction~\ref{cons:d-regular odd}. 

% Figure environment removed

\begin{observation}
    \label{obs:Hcoloring_regular_odd} 
     Let $W$ be a gadget, and  
    $H_i$ as well as $H_i'$ be the subgraphs of $W$ as defined in Construction~\ref{cons:d-regular odd}. 
     In any $cd$-coloring of $H_i$, the vertices in $H_i'$ require at least 4 colors. 
     Moreover, there exists a $cd$-coloring of $W$ using exactly $2(d-1)$ colors.
\end{observation}

\begin{proof}
    The graph induced by $H_i'$ contains a $C_6$ (for instance: induced by the vertices $d1,e1,g4,f4,g3,$ $e2$ as shown in Figure~\ref{fig:d-regular_gadget odd}). 
    It is clear that \XCD($C_6$)=4. 
    Now consider the subgraph  $H_i'$. It is obtained by 
    the introduction of new vertices adjacent to some vertices of the $C_6$.   Since none of the $3K_1$s present in the $C_6$ is dominated by any of the remaining vertices in $H_i'$, by Observation~\ref{obs:cdcolor_reduce}, it is clear that the number of colors needed for any $cd$-coloring of $H_i'$ is at least 4. Further, note that all the independent sets in $H_i'$, which are dominated by the leaf vertices $v_i$ or $v_{i+1}$ of the star graph $S$ (for instance the vertices $b$ and $c$ shown in Figure~\ref{fig:d-regular_gadget odd}), are already dominated by some vertex in $H_i'$. Therefore, again, by Observation~\ref{obs:cdcolor_reduce}, it is clear that the number of colors needed for $V(H_i')$ in any $cd$-coloring of $H_i$ is at least 4. 
    Since the root vertex $a$ of the gadget is not adjacent to any of the vertices in $H_i'$, the presence of $a$ will not reduce the number of colors needed for $cd$-coloring of the subgraph $H_i'$ of the gadget $W$. Hence, the number of colors needed for $cd$-coloring of $H_i$ is at least $4$. Therefore, the number of colors needed for $cd$-coloring of $W$ is at least $2(d-1)$, as there are $(d-1)/2$ copies of $H_i$ is present in $W$. 
    Now it is easy to see that a 4-$cd$-coloring of a $C_6$ in each $H_i'$ can be extended to a $2(d-1)$-$cd$-coloring of the gadget $W$ (by using the leaves of star graph $S$, and one of the vertices from the partition $A_i$ of each $C_i$, for $1\leq i\leq d-1$, as the dominating vertices of the color classes). 
\end{proof}

The following lemma is based on the Construction~\ref{cons:d-regular odd}.


\begin{lemma}
    \label{lem:d-regular odd}
    For each odd integer $d\geq 5$, let $G$ be a triangle-free ($d-2$)-regular graph, having $n$ vertices. Then $\XCD(G)=k$ if and only if $\XCD(G_c)=k+4n(d-1)$. 
\end{lemma}

\begin{proof}
    Let $\XCD(G)=k$. We claim that $\XCD(G_c)=k+4n(d-1)$.  Recall that the graph $G_c$ contains $2n$ gadgets as there are $n$ vertices in $G$ each having degree $d-2$. By Observation~\ref{obs:Hcoloring_regular_odd}, it is clear that any $cd$-coloring of these gadgets needs a total of at least $4n(d-1)$ colors. Since \XCD($G$)=$k$, we then have \XCD($G_c$)=$k+4n(d-1)$. 

    Now, for the reverse direction, let $\XCD(G_c)=m$. Then we claim that $\XCD(G)=k$, where $k=m-4n(d-1)$. 
    Note that for each gadget $W$, the subgraph $W- \{a\}$ is disjoint from $G$, as only $a$ in $W$ is adjacent to exactly one vertex  $u$ in $G$. Hence, the color class dominated by $a$ can include only one vertex $u$ in $G$. 
     By Observation~\ref{obs:Hcoloring_regular_odd}, in any $cd$-coloring of $G_c$, the vertex set of each copy of $H'$ of any gadget $W$ needs at least 4 colors, and none of these colors can be reused for coloring the vertices in $V(G_c)\setminus V(W)$. Again, by Observation~\ref{obs:Hcoloring_regular_odd}, the same 4 colors used to color the vertices in $H_i'$ can be reused to color the entire vertices of $H_i$. Hence, the 2($d-1$) colors used for $(d-1)/2$ copies of $H_i$ in $W$ can be reused to color the entire vertices of $W$. Therefore, we now have a $cd$-coloring of $G_c$ using $m$ colors such that no colors used for the vertices in gadgets are used to color any vertex in $G$. Thus, $\XCD(G)=m-4n(d-1)=k$, as there are $2n$ copies of $W$s in $G_c$.  
\end{proof}

The following construction is used to prove the hardness of \CDC\ on triangle-free $d$-regular graphs, for any even integer $d\geq 4$.  
 

\begin{construction}
    \label{cons:d-regular even}
     Let $G$ be a triangle-free $(d-1)$-regular graph, for any even integer $d\geq 4$.  Let $W$ be a gadget which is constructed in the following way. 
     The gadget $W$ contains $4d^2-10d+6$ vertices with two subgraphs $W_1$ and $W_2$ each having $2d^2-5d+3$ vertices. The adjacency among the vertices in $W_1$ (resp. $W_2$) is in such  a way that the $d-1$ vertices of $W_1$ (resp. $W_2$) induces a star graph $K_{1,d-2}$,  which we denote by $S_1$ (resp. $S_2$), having the vertex $a$ in $W_1$ (resp. the vertex $b$ in $W_2$) as the root vertex. Note that the vertices $a$ and $b$ are adjacent. Further in each set $W_i$, for $i\in \{1,2\}$, introduce $(2d^2-6d+4)$ vertices which induces $d-2$ disjoint copies of complete bipartite graphs $K_{d-1,d-1}$, namely $CB_{i,1},  CB_{i,2}\ldots,CB_{(i,d-2)}$ respectively.      
     The adjacency between these complete bipartite graphs $CB_{i,j}$, for $i\in \{1,2\}$ and $1\leq j\leq d-2$ and star graph $S_i$,  is in such a way that the vertices of one of the partition $A_{ij}$  of a complete bipartite graph $CB_{i,j}$ is connected to the leaf $j$  of  $S_i$.       
     Furthermore, for each odd integer $j\leq d-3$, each vertex of other partition $B_{i,j}$ of $CB_{i,j}$ is adjacent to exactly one vertex of the partition $B_{i,(j+1)}$ of $CB_{i,(j+1)}$  (for instance, refer Figure~\ref{fig:d-regular_gadget even}, when $d=4$). 
     Now the graph $G_c$ is constructed from $G$ using $W$ in such a way that consider an arbitrary pair-wise ordering $(v_1,v_2), (v_3,v_4)\ldots,(v_{n-1}, v_n)$ of vertices, in $G$.  Note that  such a pairing is possible, since $n$ is even (as $d-1$ is odd). 
     For a pair of vertices $(v_j,v_{j+1})$, for odd integer $j\leq n-1$ in this ordering, introduce a gadget $W$ such that the vertex $a$ (resp. $b$) of $W$ is adjacent to $v_j$ (resp. $v_{(j+1)}$) of $G$.
     The graph $G_c$ contains $n(2d^2-5d+3)$ vertices, where $n$ is the number of vertices in $G$.
\end{construction}



An example of the gadget $W$ in Construction~\ref{cons:d-regular even} is shown in Figure~\ref{fig:d-regular_gadget even}

% Figure environment removed

    Let $W$ be a gadget constructed in Construction~\ref{cons:d-regular even}.
     Let $H$ be an induced subgraph obtained by the union of vertices of two complete bipartite graphs $CB_{ij}$ and $CB_{i(j+1)}$, for odd integer $j$, and the two leaves $c$ and $d$ of the star graph $S_i$ which is adjacent to the partition $A_j$ and $A_{j+1}$ of $CB_{ij}$ and $CB_{i(j+1)}$, respectively. Let 
     $H'=H\setminus \{c,d\}$. 

The following observation is true for the gadget $W$ of Construction~\ref{cons:d-regular even}. 
\begin{observation}
    \label{obs:Hcoloring_regular_even}
     Let $W$ be a gadget constructed in Construction~\ref{cons:d-regular even}.
     Let $H$ be an induced subgraph obtained by the union of vertices of two complete bipartite graphs $CB_{ij}$ and $CB_{i(j+1)}$, for odd integer $j$, and the two leaves $c$ and $d$ of the star graph $S_i$ which is adjacent to the partition $A_j$ and $A_{j+1}$ of $CB_{ij}$ and $CB_{i(j+1)}$, respectively. Let 
     $H'=H\setminus \{c,d\}$. In any CD-coloring of $H$, the vertices in $H'$ requires at least 4 colors. Moreover there exists a $cd$-coloring of $W$ using exactly  $4(d-2)$ colors.
\end{observation}

\begin{proof}
    Let $H$ be a subgraph of $W_1$ in $W$. The graph induced by $H'$ contains a $C_6$ (for example: induced by the vertices $d1,e1,g3,f3,g2,e2$ as shown in Figure~\ref{fig:d-regular_gadget even}). 
    It is clear that the \XCD($C_6$)=4. 
    Now consider the subgraph $H'$. It is obtained by 
    the introduction of new vertices adjacent to some vertices of the $C_6$.  Since none of the $3K_1$s present in the $C_6$ is dominated by any of the remaining vertices in $H'$, by Observation~\ref{obs:cdcolor_reduce}, it is clear that the number of colors needed for any $cd$-coloring of $H'$ is at least 4. 
    Further, note that all the independent sets in $H'$, which are dominated by $c$ or $d$, are already dominated by some vertex in $H'$. Therefore, again by Observation~\ref{obs:cdcolor_reduce}, it is clear that the number of colors needed for $V(H')$ in any $cd$-coloring of $H$ is at least 4. 
    Since the  vertex $a$ of the  $W_1$ is not adjacent to any of the vertices in $H'$, the presence of $a$ will not reduce the number of colors needed for $cd$-coloring of the subgraph $H'$ of the gadget $W$. Hence the number of colors needed for $cd$-coloring of $H$ is at least $4$. Therefore, the number of colors needed for $cd$-coloring of $W_1$ is at least $2(d-2)$, as there are $(d-2)/2$ copies of $H$ is present in $W_1$. Similarly,  the number of colors needed for $cd$-coloring of $W_2$ is at least $2(d-2)$, as there are $(d-2)/2$ copies of $H$ is present in $W_2$. Hence, it is clear that the number of colors needed for $cd$-coloring of $W$ is at least $4(d-2)$.
    Now it is easy to see that a 4-$cd$-coloring of a $C_6$ in each $H'$ can be  extended to a 2$(d-2)-cd$-coloring of the subgraph $W_i$  for $i\in \{1,2\}$, (by using the leaves of star graph $S_i$, and one of the vertices from the partition $A_{ij}$ of each $CB_{ij}$, for $1\leq j\leq (d-2)/2$, as the dominating vertices of the color classes). Now it is obvious that, these 2$(d-2)$-$cd$-coloring of the subgraph $W_i$ and $W_2$ can be extended to a 4$(d-2)$-$cd$-coloring of $W$. 
\end{proof}

The following lemma is based on the Construction~\ref{cons:d-regular even}.


\begin{lemma}
    \label{lem:d-regular even}
    For even integer $d\geq 4$, let $G$ be a triangle-free $(d-1)$-regular graph having $n$ vertices. Then $\XCD(G)=k$ if and only if $\XCD(G_c)=k+2n(d-2)$. 
\end{lemma}

\begin{proof}
    Let $\XCD(G)=k$. We claim that $\XCD(G_c)=k+2n(d-2)$. Recall that the graph $G$ contains even number of vertices as it is a $(d-1)$-regular graph, where $d-1$ is odd. Hence $G_c$ contains $\frac{n}{2}$ gadgets as there are $n$ vertices in $G$ each having degree $d-1$. By Observation~\ref{obs:Hcoloring_regular_even}, it is clear that these gadgets need at least $2n(d-2)$ colors. Since \XCD($G$)=$k$, we then have \XCD($G_c$)=$k+2n(d-2)$. 

    Now for the reverse direction, let $\XCD(G_c)=m$. Then we claim that $\XCD(G)=k$, where $k=m-2n(d-2)$.  Note that only the vertices $a$ and $b$ of each gadget $W$ is adjacent to some vertices  $u$ and $v$ respectively in $G$. Hence the color class dominated by $a$ (resp. $b$) can include only one vertex $u$ (resp. $v$) in $G$.   
    By Observation~\ref{obs:Hcoloring_regular_even}, in any $cd$-coloring of $G_c$, the vertex set of each copy of $H'$ of any gadget $W$ needs at least 4 colors, and none of these colors can be reused for coloring the vertices in $V(G_c)\setminus V(W)$. Again by Observation~\ref{obs:Hcoloring_regular_even}, the same 4 colors used to color the vertices in $H'$ can be reused to color the entire vertices of $H$. Hence, the 4($d-2$) colors used for $(d-2)$ copies of $H$ in $W$ can be reused to color the entire vertices of $W$. Therefore, we now have a $cd$-coloring of $G_c$ using $m$ colors such that no colors used for the vertices in gadgets are used to color any vertex in $G$. Thus, $\XCD(G)=m-2n(d-2)=k$, as there are $\frac{n}{2}$ copies of $W$s in $G_c$.  
\end{proof}



\noindent\textit{Proof of Theorem~\ref{thm:regular}:} Note that $\mid V(G_{c1})\mid =2n(2d^2-3d+2)$, where $G_{c1}$ is obtained as per construction~\ref{cons:d-regular odd}. Similarly, $\mid V(G_{c2})\mid =n(2d^2-5d+3)$, where $G_{c2}$ is obtained as per construction~\ref{cons:d-regular even}. Hence, both constructions are linear with respect to their size of
inputs, and hence, their associated reductions are also linear. 
Thus, we know that there is a linear reduction from triangle-free $(d-2)$-regular graphs to triangle-free $d$-regular graphs for each fixed odd integer $d\geq 5$ due to Lemma \ref{lem:d-regular odd}. 
We also know that there is a linear reduction from triangle-free $(d-1)$-regular graphs to triangle-free $d$-regular graphs for each fixed even integer $d\geq 4$ due to Lemma \ref{lem:d-regular even}. Therefore, we are done using Proposition~\ref{pro:bipartite} and Theorem~\ref{thm:cubic}. \qed