\section{Introduction}
\label{sec:intro}
Graph theorists are always fascinated to study the relationship among correlated graph parameters as well as their structural features and explore their algorithmic consequences on the graphs for which these parameters coincide.
For instance, the \textit{chromatic number} \raisebox{1.5pt}{$\chi$} and the \textit{clique number} $\omega$. It is a well-known fact that for any graph $G$, \raisebox{1.5pt}{$\chi$}$(G)\geq \omega(G)$. 
Perfect graphs are the graphs $G$ having the property that for any induced subgraph $H$ of $G$, \raisebox{1.5pt}{$\chi$}$(H)= \omega(H)$. The notion of perfectness unifies the results concerning \textit{colorings} and \textit{cliques} for many important graph classes. The celebrated \textit{`Strong Perfect Graph Theorem'} ~\cite{ChudnovskyStrong06} gives a different perspective on perfect graphs by characterizing them by their structure instead of parameters.
Along these lines of research, here we explore the interconnections between a few correlated graph parameters. Domination and coloring are two important and well-motivated problems in graph theory. The central problem, `\textit{cd-coloring},' in this paper incorporates the flavors of both domination and coloring. Even though the other two problems studied in this paper, `\textit{total domination}' and \textit{`separated-cluster'}  have their own significance and are of independent interest, they share an interesting relationship with the `\textit{cd-coloring}' problem. In this paper, we explore these relationships in detail and obtain several exciting algorithmic consequences. 

 Let $G$ be an undirected graph without isolated vertices and $n=\mid V(G)\mid$. A \textit{proper vertex coloring} $c:V(G)\rightarrow \{1,2,\ldots,k\}$ of $G$ is the partitioning of the vertex set into $k$ color classes, say $C_1, C_2,\ldots, C_k$ such that for each $i\in \{1,2,\ldots,k\}$, $C_i$ is an independent set in $G$. Then $c$ is said to be a \textit{cd-coloring} of $G$ if, for each $j\in \{1,2,\ldots,k\}$, there exists a vertex $v_j\in V(G)$ such that $C_j\subseteq N(v_j)$. i.e., each class $C_j$ in $G$ has to be \textit{dominated} by a vertex $v_j\in V(G)\setminus C_j$. Hence the name \textit{class-domination} coloring, which we shortly call $cd$-coloring. 
 It is easy to see that if each vertex in $G$ is assigned a distinct color, then it is a $cd$-coloring of $G$ using $n$ colors. The minimum integer $k$ for which there exists a \textit{cd-coloring} of $G$ using $k$ colors is called the  \textit{cd-chromatic number} of $G$, denoted as $\XCD(G)$. Given an input graph $G$, the problem \CDC\ seeks to find the \textit{cd-chromatic number} of $G$.
\CDC\  is known to be \NPC\ for several special classes of graphs, including bipartite graphs~\cite{MerouaneHCK15} and chordal graphs~\cite{ShaluVSCDComplex20}. It is polynomial-time solvable for graph classes like trees~\cite{ShaluKiruCDTreeCobipar21}, co-bipartite graphs~\cite{ShaluKiruCDTreeCobipar21}, split graphs~\cite{MerouaneHCK15}, and claw-free graphs~\cite{ShaluVSCDComplex20}. Shalu et al.~\cite{ShaluVSCDComplex20} obtained a complexity dichotomy for \CDC\ for $H$-free graphs. \CDC\  is also studied in the paradigm of parameterized complexity~\cite{BanKasRamDomClusGr23,KritRaiST21}, and approximation complexity~\cite{ChenTheDC14}. In addition to its theoretical significance, \CDC\ has a wide range of practical applications in social networks~\cite{ChenTheDC14} and genetic networks~\cite{KlavTavaDomHerProd21}. 

A set $S\subseteq V(G)$ is said to be a \textit{total dominating set} if any vertex in $G$ has a neighbor in $S$. Note that the induced subgraph $G[S]$ does not contain an isolated vertex.
The \textit{total domination number} of $G$, denoted as \raisebox{1.5pt}{$\gamma$}$_t(G)$, is defined to be the minimum integer $k$ such that $G$ has a total dominating set of size $k$. Given an input graph $G$, in the problem \TD, we intend to find the total domination number of $G$. Total domination is one of the most popular variants of domination, and there are hundreds of research papers dedicated to this notion in the literature~\cite{MerouaneHCK15,HoppManTotDom19,HennYeoTotDom13,ZhuApproxMinTotDom09}.

A set $S\subseteq V(G)$ is said to be a \textit{separated-cluster} if no two vertices in $S$ lie at a distance of exactly 2 in $G$. The authors of~\cite{ShaluKiruCDTreeCobipar21,ShaluSandhyaLowBound17} call it a \textit{subclique}.   Since we have decided to call it a separated cluster instead of subclique, we would like to explain the reason for this deviation. We say that $S\subseteq V(G)$ induces a \textit{cluster} in $G$, if $G[S]$ is a disjoint union of cliques.
Now, if $S\subseteq V(G)$ has no two vertices lying at a distance exactly 2 in $G$, then we can see that $S$ would induce a cluster in $G$ with the following additional property; no two vertices belonging to two distinct cliques in the cluster $S$ have a common neighbor in $G$. Using this perspective, we can infer that \SCP\ is a variant of the well-known problem, {\sc Cluster Vertex Deletion} where we intend to find the minimum number of vertices whose deletion results in a disjoint union of cliques~\cite{ShamSharCluGrMod04} (or, alternatively, to find the maximum number of vertices that can be partitioned into disjoint union of cliques). The \textit{separated-cluster number} of $G$, denoted as $\SC(G)$, is defined to be the maximum integer $k$ such that $G$ has a separated-cluster of size $k$. Given an input graph $G$, in the problem \SCP,\ our goal is to find the \textit{separated-cluster number} of $G$. \SCP\ is known to be \NPC\ for graph classes like bipartite graphs, chordal graphs, $3K_1$-free graphs~\cite{ShaluSandhyaLowBound17}, and polynomial-time solvable for trees, co-bipartite graphs, cographs, split graphs~\cite{ShaluSandhyaLowBound17} etc.

\vspace{-0.25cm}
\subsection{\textbf{Total domination and $cd$-coloring}}
\vspace{-0.1cm}
Graph coloring variants that also incorporate the concepts of domination are well studied~\cite{ShaluKiruCDTreeCobipar21, ShaluSandhyaLowBound17, ChellaliVolkDomChr04,ChellaliMaffDom12}. A close connection between \textsc{CD-Coloring} and \textsc{Total Domination} was established by Merouane et al.~\cite{MerouaneHCK15}.  In particular, it is known that for any triangle-free graph $G$, \raisebox{1.5pt}{$\gamma$}$_t(G)=\XCD(G)$.  As a consequence, the complexity results of \TD\ on the subclasses of triangle-free graphs are also applicable to \CDC. For instance, the result that \TD\ is \NPC\ on bipartite graphs with bounded degree 3~\cite{ZhuApproxMinTotDom09} implies that \CDC\ is also \NPC\ on this graph class. On the other hand, even though  \TD\ is known to be \NPC\ on cubic graphs~\cite{GarJohIntract79}, the algorithmic complexity of \TD\ on triangle-free cubic graphs, in fact, triangle-free $d$-regular graphs for each $d\geq 1$ is not known. Since regular graphs may not be triangle-free in general, the algorithmic complexity of \CDC\  for cubic graphs can not be inferred from the results on \TD.    

In this paper, we study the \CDC\ on regular graphs and prove the following theorem.

\begin{theorem}
    \label{thm:regular}
    \CDC\ is \NPC\ on triangle-free $d$-regular graphs, for each fixed integer $d\geq 3$. \TETHS.
\end{theorem}
   It is fascinating to observe how \textit{cd-coloring},  a relatively new graph problem, sheds light on a classic problem of total domination. For instance, as an additional advantage, we can use Theorem~\ref{thm:regular} to obtain the following corollary which gives results on \TD\ by recalling the fact that for any triangle-free graph $G$, \raisebox{1.5pt}{$\gamma$}$_t(G)=\XCD(G)$. It is remarkable that the following corollary is stronger than the existing results of \TD\ on regular graphs.
 
 \begin{corollary}
\label{thm:total_domination_regular}
\TD\ on triangle-free $d$-regular graphs is \NPC, for any constant $d\geq 3$. \TETHS.    
\end{corollary}

%\vspace{-0.5cm}
\subsection{\textbf{Separated-Cluster and $cd$-coloring}}
\vspace{-0.05cm}
Interestingly, we can see that the  \textit{cd-chromatic number} $\XCD$ and \textit{separated-cluster number} $\SC$  follow a similar relationship as their classic counterparts of \textit{chromatic number} \raisebox{1.5pt}{$\chi$} and \textit{clique number} $\omega$. Since any two vertices in a separated-cluster of a graph $G$ cannot lie at a distance exactly 2 in $G$, we have $\XCD(G)\geq \SC(G)$ (as each vertex in $\SC(G)$ has to be in different color class). It is not difficult to see that, using the same idea of \textit{`Mycielskian'} construction~\cite{MycielColoriage55} (in classic coloring), we can have a family of graphs for which $\omega_s=2$ but $\XCD$ is arbitrarily large.

The following auxiliary graph construction is very useful in the study of separated-clusters and $cd$-coloring.

%\vspace{-0.5cm}
\begin{definition}[Auxiliary graph $G^*$~\cite{ShaluSandhyaLowBound17}]\label{def:aux}
    Given a graph $G$, the auxiliary graph $G^*$ is the graph having $V(G^*)=V(G)$ and $E(G^*)= \{uv:u,v\in V(G)$ and $d_G(u,v)=~2\}$, where  $d_G(u,v)$ denote the distance between the vertices $u$ and $v$ in $G$.
\end{definition}
%\vspace{-0.05cm}
%\vspace{-0.5cm}
From the definitions of the auxiliary graph $G^*$ of a graph $G$ and independence number  $\alpha(G^*)$  (the size of a maximum cardinality independent set in $G^*$), it is straightforward to observe that $\SC(G)=\alpha(G^*)$. Further, by the definition of the clique cover number $ k(G^*)$  (the minimum number of cliques needed to partition the vertex set of $G^*$), it is not difficult to infer that $\XCD(G)\geq k(G^*)$ (see Observation~\ref{obs:cliquecover} for details).
Note that the parameters $\alpha$ and $k$ are well-studied in the literature, particularly in connection with \textit{perfect graphs}.  \textit{A graph $G$ is perfect if, for any induced subgraph $H$ of $G$,  $\alpha(H)=k(H)$}.\\

\noindent\textbf{Our Results:} Motivated by the notion of perfect graphs in classic coloring theory, we introduce $cd$-perfectness.  
We say that a graph $G$ is \textit{$cd$-perfect} if for any induced subgraph $H$ of $G$, $\XCD(H)= \SC(H)$. Though some earlier researchers~\cite{ShaluKiruCDTreeCobipar21, ShaluSandhyaLowBound17} observed that each graph $G$ belonging to certain graph classes like co-bipartite graphs, trees etc. satisfy the condition that $\XCD(G)= \SC(G)$, it is in this paper, the notion of $cd$-perfectness is introduced for the first time. This general notion of \textit{cd-perfectness} allows us to unify several (existing) results concerning \textit{$cd$-coloring} and \textit{separated-clusters} for various graph classes.

As a first step to prove the structure of $cd$-perfect graphs, we prove a sufficient condition for a graph $G$ to satisfy the equality, $\XCD(G)= k(G^*)$  (see Theorem~\ref{thm:suff_cdclique}). 
This leads us to Theorem~\ref{thm:suff_cdsubclique}, where the collection of graphs $\mathcal{H}$ is as shown in Figure~\ref{fig:c6-free}. 

 % Figure environment removed

\begin{theorem} \label{thm:suff_cdsubclique}
     Let $G$ be an $\mathcal{H}$-free graph and $G^*$ its  auxiliary graph. If $k(G^*)=\alpha(G^*)$ then $\XCD(G)=\SC(G)$. Consequently, if $G$ is $\mathcal{H}$-free and $G^*$ is perfect, then $\XCD(G)=\SC(G)$.
 \end{theorem}

As an immediate corollary of Theorem~\ref{thm:suff_cdsubclique}, we obtain a sufficient condition for a graph to be $cd$-perfect (see Corollary~\ref{corr:suffcdperfect}). 
 We then use this result to bring several graph classes, like \textit{co-bipartite graphs}, \textit{chordal bipartite graphs}, etc., under the common umbrella of \textit{$cd$-perfect} graphs. Even though these results are structural in nature, they have several exciting algorithmic consequences. It is interesting to note that the same framework can be used as a tool to derive both positive and negative results concerning the algorithmic complexity of \CDC\ and \SCP. In particular, we have the following results.

\begin{enumerate}[label=(\roman*)]
   
    \item A beautiful interplay between the problems, {\sc Total Domination}, \CDC, and \SCP\ can be witnessed in Theorem~\ref{thm:chordalbip}, where we use Corollary~\ref{corr:suffcdperfect}, Theorem~\ref{thm:totdom}, and the fact that \raisebox{1.5pt}{$\gamma$}$_t(G)=\XCD(G)$ for triangle free-graphs $G$, to prove that the three problems mentioned above are equivalent and solvable in $O(n^2)$ time for chordal bipartite graphs. Consequently, this result improves and generalizes the existing $O(n^3)$-time algorithm for \SCP\ on the class of $P_6$-free chordal bipartite graphs~\cite{ShaluKiruP5Free22}. 
%\vspace{-0.5cm}
\begin{theorem}\label{thm:chordalbip}
    Let $G=(A,B,E)$ be a chordal bipartite graph. Then $G$ is $cd$-perfect. Consequently, {\sc Total Domination}, \CDC, and \SCP\  are all equivalent problems for chordal bipartite graphs and can be solved in $O(n^2)$ time.
\end{theorem}
 %  \vspace{-0.5cm}

     \item  We provide a unified approach (most of the time with an improvement) for finding polynomial-time algorithms for \CDC\  on certain graph classes. For instance, we prove the following theorem. 
%\vspace{-0.5cm}
    \begin{theorem} \label{thm:properalpha}
The \CDC\ problem can be solved in $O(n^{2.5})$ time for proper interval graphs and $3K_1$-free graphs.
\end{theorem}
     
    \item We also derive the following hardness results for $C_6$-free bipartite graphs.
%\vspace{-0.5cm}
    \begin{theorem}\label{thm:C_6free hard}
    The problems  \CDC\ and \SCP\ are NP-Complete for $C_6$-free bipartite graphs.
\end{theorem}

 % \vspace{-0.5cm} 
    \item Further, as an additional result, we prove the following theorem concerning separated-cluster on interval graphs which settles an open problem in~\cite{ShaluSandhyaLowBound17}.
%\vspace{-0.5cm}
    \begin{theorem}
    \label{thm:sepiterval}
    The \SCP\ problem in interval graphs can be solved in polynomial time.
\end{theorem}

\end{enumerate}