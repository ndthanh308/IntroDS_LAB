\begin{construction}
    \label{cons:d-regular even}
     Let $G$ be a triangle-free $(d-1)$-regular graph, for any even integer $d\geq 4$.  We construct a graph $G_c$ from $G$ %using a gadget $W$ 
    in the following way: 
   % \vspace{-0.25cm}
    \begin{itemize}
        \item First we construct a gadget $W$ as follows: introduce two sets of $2d^2-5d+3$ vertices which induce two subgraphs $W_1$ and $W_2$ in $W$. 
        The adjacency among the vertices in $W_1$ (respectively $W_2$) is in such  a way that the $d-1$ vertices of $W_1$ (respectively $W_2$) induces a star graph $K_{1,d-2}$,  which we denote by $S_1$ (respectively $S_2$), having the vertex $a$ in $W_1$ (respectively the vertex $b$ in $W_2$) as the root vertex. Note that the vertices $a$ and $b$ are adjacent. 
         
         \item Further in each set $W_i$, for $i\in \{1,2\}$, the remaining $(2d^2-6d+4)$ vertices  induces $d-2$ disjoint copies of complete bipartite graphs $K_{d-1,d-1}$, namely $C_{i1},  C_{i2}\ldots,$ $C_{i(d-2)}$ respectively.      
     The adjacency between these complete bipartite graphs $C_{ij}$, for $i\in\{1,2\}$ and $1\leq j\leq d-2$ and star graph $S_i$,  is in such a way that the vertices of one of the partition $A_{ij}$  of a complete bipartite graph $C_{ij}$ is adjacent to the leaf $v_j$  of  $S_i$.       
     Furthermore, for each odd integer $j\leq d-3$, each vertex of the partite set $B_{ij}$ of $C_{ij}$ is adjacent to exactly one vertex of  $B_{i(j+1)}$ of $C_{i(j+1)}$  (for instance, refer Figure~\ref{fig:d-regular_gadget even}, when $d=4$). 
     Now, by $H_{ij}'$ we denote the subgraph of $W_i$ induced by the vertices in the two complete bipartite graphs $C_{ij}$ and $C_{i(j+1)}$, for an odd integer $j\leq d-3$.  Then, by $H_{ij}$ we denote the subgraph of $W_i$ induced by the vertices in $H_{ij}'$ and  the two leaves $v_j$ and $v_{j+1}$ of the star graph $S_i$ which are adjacent to the partitions $A_{ij}$ and  $A_{i(j+1)}$ of $C_{ij}$ and $C_{i(j+1)}$, respectively. Note that $H_{ij}'=H_{ij}-\{v_j,v_{j+1}\}$.
     
     \item Now the graph $G_c$ is constructed from $G$ using $W$ in the following way: consider an arbitrary pair-wise ordering $(v_1,v_2), (v_3,v_4)\ldots,(v_{n-1}, v_n)$ of vertices, in $G$.  Note that such a pairing is possible, since $n$ is even (as $d-1$ is odd). 
     For a pair of vertices $(v_j,v_{j+1})$, for odd integer $j\leq n-1$ in this ordering, introduce a gadget $W$ such that the vertex $a$ (respectively $b$) of $W$ is adjacent to $v_j$ (respectively $v_{j+1}$) of $G$.
     \end{itemize}
     The graph $G_c-G$ contains $n(2d^2-5d+3)$ vertices, where $n$ is the number of vertices in $G$.
     
\end{construction}

An example of the gadget $W$ in Construction~\ref{cons:d-regular even} is shown in Figure~\ref{fig:d-regular_gadget even}. The following observation is true for the gadget $W$ of Construction~\ref{cons:d-regular even}. 

% Figure environment removed

\begin{observation}
    \label{obs:Hcoloring_regular_even}
    Let $W$ be a gadget, and  
    $H_{ij}$ as well as $H_{ij}'$, for $i\in\{1,2\}$ and  an odd integer $j\leq d-3$,  be the subgraphs of $W$ as defined in Construction~\ref{cons:d-regular even}. 
     In any $cd$-coloring of $H_{ij}$, the vertices in $H_{ij}'$ require at least 4 colors. 
     Moreover there exists a $cd$-coloring of $W$ using exactly  $4(d-2)$ colors.
\end{observation}
\begin{proof}
    %Let $H_{ij}$ be a subgraph of $W_{i}$, for $i\in \{1,2\}$ in $W$. 
    The graph induced by $H_{ij}'$ contains a $C_6$ (for instance: induced by the vertices $d1,e1,g3,f3,g2,$ $e2$ as shown in Figure~\ref{fig:d-regular_gadget even}). 
    It is clear that \XCD($C_6$)=4. 
    Now consider the subgraph $H_{ij}'$. It is obtained by 
    the introduction of new vertices adjacent to some vertices of the $C_6$.  Since none of the $3K_1$s present in the $C_6$ is dominated by any of the remaining vertices in $H_{ij}'$, by Observation~\ref{obs:cdcolor_reduce}, it is clear that the number of colors needed for any $cd$-coloring of $H_{ij}'$ is at least 4. 
    Further, note that all the independent sets in $H_{ij}'$, which are dominated by $v_j$ or $v_{j+1}$ of $S_i$, for $i\in\{1,2\}$, are already dominated by some vertex in $H_{ij}'$. Therefore, again by Observation~\ref{obs:cdcolor_reduce}, it is clear that the number of colors needed for $V(H_{ij}')$ in any $cd$-coloring of $H_{ij}$ is at least 4. 
    Since the vertex $a$ (respectively $b$) of $W_1$ (respectively $W_2$) is not adjacent to any of the vertices in $H_{1j}'$ (respectively $H_{2j}'$), the presence of $a$ (respectively $b$) will not reduce the number of colors needed for $cd$-coloring of the subgraph  $H_{1j}'$ (respectively $H_{2j}'$) of the gadget $W$. Hence, the number of colors needed for a $cd$-coloring of  $H_{1j}$ (respectively $H_{2j}$) is at least $4$. Therefore, the number of colors needed for a $cd$-coloring of $W_i$, for $i\in \{1,2\}$, is at least $2(d-2)$, as there are $(d-2)/2$ copies of $H_{ij}$ is present in $W_i$. %Similarly,  the number of colors needed for $cd$-coloring of $W_2$ is at least $2(d-2)$, as there are $(d-2)/2$ copies of $H$ is present in $W_2$. 
    Hence, it is clear that the number of colors needed for a $cd$-coloring of $W$ is at least $4(d-2)$.
    Now it is easy to see that a 4-$cd$-coloring of a $C_6$ in each $H_{ij}'$ can be extended to a 2$(d-2)-cd$-coloring of the subgraph $W_i$  for $i\in \{1,2\}$, (by using the leaves of star graph $S_i$, and one of the vertices from the partition $A_{ij}$ of each $C_{ij}$, for $1\leq j\leq (d-2)/2$, as the dominating vertices of the color classes). Now it is obvious that, these 2$(d-2)$-$cd$-coloring of the subgraph $W_1$ and $W_2$ can be extended to a 4$(d-2)$-$cd$-coloring of $W$. 
\end{proof}

The following lemma is based on the Construction~\ref{cons:d-regular even}.


\begin{lemma}
    \label{lem:d-regular even}
    For an even integer $d\geq 4$, let $G$ be a triangle-free $(d-1)$-regular graph having $n$ vertices. Then $\XCD(G)=k$ if and only if $\XCD(G_c)=k+2n(d-2)$. 
\end{lemma}

\begin{proof}
    Let $\XCD(G)=k$. We claim that $\XCD(G_c)=k+2n(d-2)$. Recall that the graph $G$ contains an even number of vertices as it is a $(d-1)$-regular graph, where $d-1$ is odd. Hence $G_c$ contains ${n}/{2}$ gadgets as there are $n$ vertices in $G$ each having degree $d-1$. By Observation~\ref{obs:Hcoloring_regular_even}, it is clear that these gadgets need at least $2n(d-2)$ colors. Since \XCD($G$)=$k$, we then have \XCD($G_c$)=$k+2n(d-2)$. 


    Now, for the reverse direction, let $\XCD(G_c)=m$. Then we claim that $\XCD(G)=k$, where $k=m-2n(d-2)$.  Note that only the vertices $a$ and $b$ of each gadget $W$ are adjacent to some vertices  $u$ and $v$, respectively, in $G$. Hence, the color class dominated by $a$ (respectively $b$) can include only one vertex $u$ (respectively $v$) in $G$.   
    By Observation~\ref{obs:Hcoloring_regular_even}, in any $cd$-coloring of $G_c$, the vertex set of each copy of $H_{ij}'$, for $i\in \{1,2\}$ and an odd integer $j\leq {(d-2)}/{2}$, of any gadget $W$ needs at least 4 colors, and none of these colors can be reused for coloring the vertices in $V(G_c)\setminus V(W)$. Again, by Observation~\ref{obs:Hcoloring_regular_even}, the same 4 colors used to color the vertices in $H_{ij}'$ can be reused to color the entire vertices of $H$. Hence, the 4($d-2$) colors used for $(d-2)$ copies of $H_{ij}$ in $W$ can be reused to color the entire vertices of $W$. Therefore, we now have a $cd$-coloring of $G_c$ using $m$ colors such that no colors used for the vertices in gadgets are used to color any vertex in $G$. Thus, $\XCD(G)=m-2n(d-2)=k$, as there are ${n}/{2}$ copies of $W$s in $G_c$.  
\end{proof}