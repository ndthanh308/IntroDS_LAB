%% 
%% Copyright 2007-2024 Elsevier Ltd
%% 
%% This file is part of the 'Elsarticle Bundle'.
%% ---------------------------------------------
%% 
%% It may be distributed under the conditions of the LaTeX Project Public
%% License, either version 1.3 of this license or (at your option) any
%% later version.  The latest version of this license is in
%%    http://www.latex-project.org/lppl.txt
%% and version 1.3 or later is part of all distributions of LaTeX
%% version 1999/12/01 or later.
%% 
%% The list of all files belonging to the 'Elsarticle Bundle' is
%% given in the file `manifest.txt'.
%% 
%% Template article for Elsevier's document class `elsarticle'
%% with numbered style bibliographic references
%% SP 2008/03/01
%% $Id: elsarticle-template-num.tex 249 2024-04-06 10:51:24Z rishi $
%%
%\documentclass[preprint,12pt]{elsarticle}
\documentclass{elsarticle}

\let\today\relax
\makeatletter
\def\ps@pprintTitle{%
    \let\@oddhead\@empty
    \let\@evenhead\@empty
    \def\@oddfoot{\footnotesize\itshape
         {} \hfill\today}%
    \let\@evenfoot\@oddfoot
    }
\makeatother

\usepackage{amssymb}
%% The amsmath package provides various useful equation environments.
\usepackage{amsmath}

\journal{}
\usepackage{natbib}
\newenvironment{changemargin}[3]{%
\begin{list}{}{%
\setlength{\leftmargin}{#1}%
\setlength{\rightmargin}{#2}%
\setlength{\topmargin}{#3}%
}%
\item[]}
{\end{list}}
\usepackage{adjustbox}


\usepackage[skipabove=8pt,skipbelow=4pt,innertopmargin=6pt,innerbottommargin=6pt]{mdframed}
\usepackage[T1]{fontenc}
\usepackage[utf8]{inputenc}
\usepackage{scrextend}
\usepackage{hyperref} 
\usepackage{ragged2e}

\raggedbottom

%\usepackage[subtle]{savetrees}
%\usepackage[margin=2cm]{geometry}
\usepackage{tikz,amsmath, amssymb,bm,color, amsthm,amsfonts}
\usetikzlibrary{positioning, calc,chains,fit,shapes}
%\usetikzlibrary{circuits.logic.US,circuits.logic.IEC,fit}
\usepackage{enumerate}
\usepackage{comment}
\usepackage{tikz}
\usepackage{graphics}
%\usepackage[cm]{fullpage}
\usepackage{longtable}
\usepackage{mdframed}
\usepackage{caption}
\usepackage{subcaption}
\usepackage{slashbox}
\usepackage{url}
\usepackage{framed}
\usepackage{array}
\usepackage{tabu}
\usepackage{lscape}
\usepackage{multirow}
\usepackage{ulem}
\usepackage{multicol}
\usepackage{placeins}
\usepackage{cite}
\usepackage{enumitem}
\usepackage{mathtools}
%\usepackage[numbers]{natbib}
%\usepackage{mathtools}
%\usepackage{authblk}

\mdfsetup{skipabove=2pt,skipbelow=2pt}
%\setlenght {\marginparwidth }{2cm}
%\usepackage{todonotes}

%\usepackage{floatrow}
%\usepackage{adjustbox}
%\setlength{\extrarowheight}{.05ex}
%\renewcommand\thesubfigure{\roman{subfigure}}


%\newtheorem{theorem}{Theorem}[section]
%\newtheorem{lemma}[theorem]{Lemma}
%\newtheorem{observation}[theorem]{Observation}
%\newtheorem{corollary}[theorem]{Corollary}
%\newtheorem{proposition}[theorem]{Proposition}
%\newtheorem{definition}[theorem]{Definition}
\newtheorem{construction}{Construction}
%\newtheorem{conjecture}{Conjecture}
%\newtheorem{remark}[theorem]{Remark}

\newcommand{\pname}[1]{\textnormal{\textsc{#1}}}
\newcommand{\cclass}[1]{\textnormal{\textsf{#1}}}
\newcommand{\nog}{nine} % no of members in the gang!
\newcommand{\nogd}{nineteen} % no of members in the gang - for deletion/completion
\newcommand{\nogl}{eighteen} % no of members in the larger gang - for editing
\newcommand{\nogld}{thirty eight} % no of members in the larger gang - for deletion/completion
\newcommand{\diffnog}{ten} %
%\newcommand{\dominatedby}{dominated by} %
%\newcommand{\dominatingset}{dominating set} %
%\newcommand{\dominates}{dominates} %
\newcommand{\simulates}{simulates} %
\newcommand{\baseset}{base} %
\newcommand{\issimulatedby}{is simulated by} %

\newcommand{\StarSAT}{\pname{8-SAT$_{\geq 6}$}}
\newcommand{\FSAT}{\pname{4-SAT$_{\geq 2}$}}
\newcommand{\FISAT}{\pname{5-SAT$_{\geq 3}$}}
\newcommand{\SIXSAT}{\pname{6-SAT$_{\geq 4}$}}
\newcommand{\ESAT}{\pname{8-SAT$_{\geq 6}$}}
\newcommand{\KSAT}{\pname{$k$-SAT$_{\geq {k-2}}$}}
\newcommand{\KSATO}{\pname{$k$-SAT}}
\newcommand{\ESATO}{\pname{8-SAT}}
\newcommand{\FSATO}{\pname{4-SAT}}
\newcommand{\FISATO}{\pname{5-SAT}}
\newcommand{\TSAT}{\pname{3-SAT}}
\newcommand{\HED}{\pname{${H}$-free Edge Deletion}}
\newcommand{\AEE}{\pname{${A}$-free Edge Editing}}
\newcommand{\AED}{\pname{${A}$-free Edge Deletion}}
\newcommand{\TSED}{\pname{$t$-star-free Edge Deletion}}
\newcommand{\ATSED}{\pname{Annotated $t$-star-free Edge Deletion}}
\newcommand{\AFSED}{\pname{Annotated $4$-star-free Edge Deletion}}
\newcommand{\FSED}{\pname{$4$-star-free Edge Deletion}}
\newcommand{\FVSED}{\pname{$5$-star-free Edge Deletion}}
\newcommand{\HEE}{\pname{${H}$-free Edge Editing}}
\newcommand{\HEC}{\pname{${H}$-free Edge Completion}}
\newcommand{\HDEE}{\pname{${H'}$-free Edge Editing}}
\newcommand{\HDDEE}{\pname{${H''}$-free Edge Editing}}
\newcommand{\HDED}{\pname{${H'}$-free Edge Deletion}}
\newcommand{\HDEC}{\pname{${H'}$-free Edge Completion}}
\newcommand{\HBEE}{\pname{${\overline{H}}$-free Edge Editing}}
\newcommand{\HBED}{\pname{${\overline{H}}$-free Edge Deletion}}
\newcommand{\HBEC}{\pname{${\overline{H}}$-free Edge Completion}}
\newcommand{\HOEDCE}{\pname{${H_1}$-free Edge Deletion(Completion/Editing)}}
\newcommand{\HEDCE}{\pname{${H}$-free Edge Deletion(Completion/Editing)}}
\newcommand{\HEEDC}{\pname{${H}$-free Edge Editing(Deletion/Completion)}}
\newcommand{\HDEEDC}{\pname{${H'}$-free Edge Editing(Deletion/Completion)}}
\newcommand{\BFED}{\pname{Bow-free Edge Deletion}}
\newcommand{\ABFED}{\pname{Annotated Bow-free Edge Deletion}}
\newcommand{\DTIS}{\pname{Distance-3 Independent Set}}
\newcommand{\SVC}{\pname{Strong Vertex Cover}}
\newcommand{\CLIQUE}{\pname{Clique}}
\newcommand{\IS}{\pname{Independent Set}}
\newcommand{\PFS}{\pname{Propagational-$f$ Satisfiability}}
\newcommand{\RHED}{\pname{Restricted ${H}$-free Edge Deletion}}
\newcommand{\RHEC}{\pname{Restricted ${H}$-free Edge Completion}}
\newcommand{\RHDED}{\pname{Restricted ${H'}$-free Edge Deletion}}
\newcommand{\RHDEC}{\pname{Restricted ${H'}$-free Edge Completion}}
\newcommand{\RHEE}{\pname{Restricted ${H}$-free Edge Editing}}
\newcommand{\PH}{$\cclass{NP} \subseteq \cclass{coNP/poly}$}
\newcommand{\NOPH}{$\cclass{NP} \not\subseteq \cclass{coNP/poly}$}
\newcommand{\LG}{\mathcal{W}}
\newcommand{\LGD}{\mathcal{W}'}
\newcommand{\LGDD}{\mathcal{W}''}


%\let\oldvee\vee
\renewcommand\vee{\boxtimes}

\newcommand\addvmargin[1]{
  \node[fit=(current bounding box),inner ysep=#1,inner xsep=0]{};
}
\setlength{\fboxrule}{0pt}

\newcommand{\defstage}[2]{% PGD Version
  \hfill\\\smallskip\noindent%
  \begin{tabularx}{\textwidth}{|l X|}%
    \hline%
    \multicolumn{2}{|l|}{\textbf{#1}}\\%
    &#2\\\hline%
  \end{tabularx}%
%  \smallskip%
}
\setlength\extrarowheight{15pt}

\newcounter{rowcntr}[table]
\renewcommand{\therowcntr}{\thetable.\arabic{rowcntr}}

% A new columntype to apply automatic stepping
\newcolumntype{N}{>{\refstepcounter{rowcntr}\therowcntr}c}

% Reset the rowcntr counter at each new tabular
\AtBeginEnvironment{longtabu}{\setcounter{rowcntr}{0}}

\newcounter{rowcntra}[table]
\renewcommand{\therowcntra}{\arabic{rowcntra}}

% A new columntype to apply automatic stepping
\newcolumntype{M}{>{\refstepcounter{rowcntra}\therowcntra}c}

% Reset the rowcntr counter at each new tabular
\AtBeginEnvironment{tabular}{\setcounter{rowcntra}{0}}

\newcommand{\NPC}{NP-Complete}


\newcommand{\highlight}[1]{\textcolor{blue}{#1}}
\newcommand{\dhanya}[1]{\textcolor{blue}{dhanya: #1}}


%\newcommand{\XCD1}[1]{\pname{$\chi_{cd}$\ensuremath{(#1)}}}
\newcommand{\XCD}{\pname{$\chi_{cd}$}}
\newcommand{\SC}{\pname{$\omega_{s}$}}

\newcommand{\CDC}{\textsc{CD-coloring}}
\newcommand{\SCP}{\textsc{Separated-Cluster}}
\newcommand{\TD}{\textsc{Total Domination}}
\newcommand{\ISP}{\textsc{Independent Set}}
\newcommand{\CC}{\textsc{Clique Cover}}
\newcommand{\TETHS}{Further, the problem cannot be solved in time \ensuremath{2^{o(|V(G)|)}}, unless the ETH fails}
%\usetikzlibrary{positioning,chains,shapes,calc}
\usetikzlibrary{fit}
\thispagestyle{empty}
\usetikzlibrary{
  graphs,
  graphs.standard
}

\usepackage{xcolor}
\usepackage{textcomp}
%\nocopyright
\renewcommand\footnoterule{\rule{0.25\linewidth}{0.5pt}}

\begin{document}

\begin{frontmatter}

%% Title, authors and addresses

\title{Total Domination, Separated-Cluster, CD-Coloring: Algorithms and Hardness}

\renewcommand\rightmark{\textit{Total Domination, Separated-Cluster, CD-Coloring: Algorithms and Hardness}}
\renewcommand\leftmark{\textit{Total Domination, Separated-Cluster, CD-Coloring: Algorithms and Hardness}}

\author[1]{Dhanyamol Antony\corref{cor1}}\ead{dhanyamolantony@iisertvm.ac.in}

\author[2] {L. Sunil Chandran\corref{cor1}}\ead{sunil@iisc.ac.in}
%\equalcont{These authors contributed equally to this work.}

\author[3]{Ankit Gayen\corref{cor1}}\ead{ankit.gayen@ens-lyon.fr}

\author[2]{Shirish Gosavi\corref{cor1}}\ead{shirishgp@iisc.ac.in}
%\equalcont{These authors contributed equally to this work.}

\author[4]{Dalu Jacob\corref{cor1}}\ead{dalujacob@maths.iitd.ac.in}


\cortext[cor1]{Corresponding author}

\affiliation[1]{organization={School of Data Science, Indian Institute of Science Education and Research Thiruvananthapuram},%Department and Organization
            %addressline={}, 
           % city={},
            postcode={695551}, 
            state={Kerala},
            country={India}}
\affiliation[2]{organization={Computer Science and Automation department, Indian Institute of Science},%Department and Organization
           % addressline={}, 
            city={Bengaluru},
            postcode={560012}, 
            state={Karnataka},
            country={India}}
\affiliation[3]{organization={D\'{e}partment d'informatique, \'{E}cole normale sup\'{e}rieure de Lyon},%Department and Organization
         %   addressline={}, 
           % city={},
         %   postcode={}, 
       %     state={},
            country={France}}
\affiliation[4]{organization={Department of Mathematics, Indian Institute of Technology Delhi},%Department and Organization
           % addressline={}, 
            city={New Delhi},
            postcode={110016}, 
           % state={},
            country={India}}

            
%% Abstract
\begin{abstract}
%% Text of abstract
% shirish Abstract text.
Domination and coloring are two classic problems in graph theory. In this paper, our major focus is on the  \CDC\ problem, which incorporates the flavors of both domination and coloring in it. Let $G$ be an undirected graph without isolated vertices.  A proper vertex coloring~$c:V(G)\rightarrow\{1,2,\ldots, k\}$ of $G$ is said to be a \textit{cd-coloring}, if for each color class $C_j$, where $j\in \{1,2,\ldots,k\}$, with respect to $c$, there exists a vertex $v_j$ in $G$ such that $C_j\subseteq N(v_j)$. The minimum integer $k$ for which there exists a \textit{cd-coloring} of $G$ using $k$ colors is called the  \textit{cd-chromatic number} of $G$, denoted as $\XCD(G)$. A set $S\subseteq V(G)$ is said to be a \textit{total dominating set}, if any vertex in $G$ has a neighbor in $S$. The \textit{total domination number} of $G$, denoted as \raisebox{1.5pt}{$\gamma$}$_t(G)$, is defined to be the minimum integer $k$ such that $G$ has a total dominating set of size $k$. A set $S\subseteq V(G)$ is said to be a \textit{separated-cluster} (also known as \textit{sub-clique}) if no two vertices in $S$ lie at a distance exactly 2 in $G$. The \textit{separated-cluster number} of $G$, denoted as $\SC(G)$, is defined to be the maximum integer $k$ such that $G$ has a separated-cluster of size $k$.

In this paper, we contribute to the literature connecting \CDC\ with \TD\ and \SCP. 
%\begin{addmargin}[2.5em]{2.5em}
\begin{enumerate}[label=(\roman*)]
    \item It is known that \TD\ is \NPC\ for \textit{cubic graphs and triangle-free subcubic graphs}. We strengthen this result by proving that both the problems {\sc CD-Coloring} and \TD\ are \NPC, and do not admit any subexponential-time algorithms on \textit{triangle-free $d$-regular graphs, for each fixed integer $d\geq 3$}, assuming the Exponential Time Hypothesis.
    
    \item For any graph $G$, it is easy to see that $\XCD(G)\geq \SC(G)$. Analogous to the well-known notion of \textit{`perfectness'}, we introduce the notion of \textit{`cd-perfectness'}. We prove a sufficient condition for a graph $G$ to be \textit{cd-perfect} (i.e. $\XCD(H)= \SC(H)$, for any induced subgraph $H$ of $G$). Our sufficient condition turns out to be necessary for certain graph classes (like \textit{triangle-free} graphs).  
    
    \item The notion of `$cd$-perfectness' provides a framework to study the algorithmic complexity of \CDC\ and \SCP. By using this framework we provide both positive and negative results concerning the algorithmic complexity of \CDC\ and \SCP.
    
    \item We also settle an open question by proving that the \SCP\ problem is solvable in polynomial time for the class of \textit{interval graphs}. 
\end{enumerate}
%\end{addmargin}

\end{abstract}

%% Keywords
\begin{keyword}
Total Domination \sep CD-coloring \sep Separated-Cluster \sep CD-perfectness
\end{keyword}
\end{frontmatter}

%% Add \usepackage{lineno} before \begin{document} and uncomment 
%% following line to enable line numbers
%% \linenumbers

%% main text
%%

%% Use \section commands to start a section
\section{Introduction}
\label{sec:intro}
Graph theorists are always fascinated to study the relationship among correlated graph parameters as well as their structural features and explore their algorithmic consequences on the graphs for which these parameters coincide.
For instance, the \textit{chromatic number} \raisebox{1.5pt}{$\chi$} and the \textit{clique number} $\omega$. It is a well-known fact that for any graph $G$, \raisebox{1.5pt}{$\chi$}$(G)\geq \omega(G)$. 
Perfect graphs are the graphs $G$ having the property that for any induced subgraph $H$ of $G$, \raisebox{1.5pt}{$\chi$}$(H)= \omega(H)$. The notion of perfectness unifies the results concerning \textit{colorings} and \textit{cliques} for many important graph classes. The celebrated \textit{`Strong Perfect Graph Theorem'} ~\cite{ChudnovskyStrong06} gives a different perspective on perfect graphs by characterizing them by their structure instead of parameters.
Along these lines of research, here we explore the interconnections between a few correlated graph parameters. Domination and coloring are two important and well-motivated problems in graph theory. The central problem, `\textit{cd-coloring},' in this paper incorporates the flavors of both domination and coloring. Even though the other two problems studied in this paper, `\textit{total domination}' and \textit{`separated-cluster'}  have their own significance and are of independent interest, they share an interesting relationship with the `\textit{cd-coloring}' problem. In this paper, we explore these relationships in detail and obtain several exciting algorithmic consequences. 

 Let $G$ be an undirected graph without isolated vertices and $n=\mid V(G)\mid$. A \textit{proper vertex coloring} $c:V(G)\rightarrow \{1,2,\ldots,k\}$ of $G$ is the partitioning of the vertex set into $k$ color classes, say $C_1, C_2,\ldots, C_k$ such that for each $i\in \{1,2,\ldots,k\}$, $C_i$ is an independent set in $G$. Then $c$ is said to be a \textit{cd-coloring} of $G$ if, for each $j\in \{1,2,\ldots,k\}$, there exists a vertex $v_j\in V(G)$ such that $C_j\subseteq N(v_j)$. i.e., each class $C_j$ in $G$ has to be \textit{dominated} by a vertex $v_j\in V(G)\setminus C_j$. Hence the name \textit{class-domination} coloring, which we shortly call $cd$-coloring. 
 It is easy to see that if each vertex in $G$ is assigned a distinct color, then it is a $cd$-coloring of $G$ using $n$ colors. The minimum integer $k$ for which there exists a \textit{cd-coloring} of $G$ using $k$ colors is called the  \textit{cd-chromatic number} of $G$, denoted as $\XCD(G)$. Given an input graph $G$, the problem \CDC\ seeks to find the \textit{cd-chromatic number} of $G$.
\CDC\  is known to be \NPC\ for several special classes of graphs, including bipartite graphs~\cite{MerouaneHCK15} and chordal graphs~\cite{ShaluVSCDComplex20}. It is polynomial-time solvable for graph classes like trees~\cite{ShaluKiruCDTreeCobipar21}, co-bipartite graphs~\cite{ShaluKiruCDTreeCobipar21}, split graphs~\cite{MerouaneHCK15}, and claw-free graphs~\cite{ShaluVSCDComplex20}. Shalu et al.~\cite{ShaluVSCDComplex20} obtained a complexity dichotomy for \CDC\ for $H$-free graphs. \CDC\  is also studied in the paradigm of parameterized complexity~\cite{BanKasRamDomClusGr23,KritRaiST21}, and approximation complexity~\cite{ChenTheDC14}. In addition to its theoretical significance, \CDC\ has a wide range of practical applications in social networks~\cite{ChenTheDC14} and genetic networks~\cite{KlavTavaDomHerProd21}. 

A set $S\subseteq V(G)$ is said to be a \textit{total dominating set} if any vertex in $G$ has a neighbor in $S$. Note that the induced subgraph $G[S]$ does not contain an isolated vertex.
The \textit{total domination number} of $G$, denoted as \raisebox{1.5pt}{$\gamma$}$_t(G)$, is defined to be the minimum integer $k$ such that $G$ has a total dominating set of size $k$. Given an input graph $G$, in the problem \TD, we intend to find the total domination number of $G$. Total domination is one of the most popular variants of domination, and there are hundreds of research papers dedicated to this notion in the literature~\cite{MerouaneHCK15,HoppManTotDom19,HennYeoTotDom13,ZhuApproxMinTotDom09}.

A set $S\subseteq V(G)$ is said to be a \textit{separated-cluster} if no two vertices in $S$ lie at a distance of exactly 2 in $G$. The authors of~\cite{ShaluKiruCDTreeCobipar21,ShaluSandhyaLowBound17} call it a \textit{subclique}.   Since we have decided to call it a separated cluster instead of subclique, we would like to explain the reason for this deviation. We say that $S\subseteq V(G)$ induces a \textit{cluster} in $G$, if $G[S]$ is a disjoint union of cliques.
Now, if $S\subseteq V(G)$ has no two vertices lying at a distance exactly 2 in $G$, then we can see that $S$ would induce a cluster in $G$ with the following additional property; no two vertices belonging to two distinct cliques in the cluster $S$ have a common neighbor in $G$. Using this perspective, we can infer that \SCP\ is a variant of the well-known problem, {\sc Cluster Vertex Deletion} where we intend to find the minimum number of vertices whose deletion results in a disjoint union of cliques~\cite{ShamSharCluGrMod04} (or, alternatively, to find the maximum number of vertices that can be partitioned into disjoint union of cliques). The \textit{separated-cluster number} of $G$, denoted as $\SC(G)$, is defined to be the maximum integer $k$ such that $G$ has a separated-cluster of size $k$. Given an input graph $G$, in the problem \SCP,\ our goal is to find the \textit{separated-cluster number} of $G$. \SCP\ is known to be \NPC\ for graph classes like bipartite graphs, chordal graphs, $3K_1$-free graphs~\cite{ShaluSandhyaLowBound17}, and polynomial-time solvable for trees, co-bipartite graphs, cographs, split graphs~\cite{ShaluSandhyaLowBound17} etc.

\vspace{-0.25cm}
\subsection{\textbf{Total domination and $cd$-coloring}}
\vspace{-0.1cm}
Graph coloring variants that also incorporate the concepts of domination are well studied~\cite{ShaluKiruCDTreeCobipar21, ShaluSandhyaLowBound17, ChellaliVolkDomChr04,ChellaliMaffDom12}. A close connection between \textsc{CD-Coloring} and \textsc{Total Domination} was established by Merouane et al.~\cite{MerouaneHCK15}.  In particular, it is known that for any triangle-free graph $G$, \raisebox{1.5pt}{$\gamma$}$_t(G)=\XCD(G)$.  As a consequence, the complexity results of \TD\ on the subclasses of triangle-free graphs are also applicable to \CDC. For instance, the result that \TD\ is \NPC\ on bipartite graphs with bounded degree 3~\cite{ZhuApproxMinTotDom09} implies that \CDC\ is also \NPC\ on this graph class. On the other hand, even though  \TD\ is known to be \NPC\ on cubic graphs~\cite{GarJohIntract79}, the algorithmic complexity of \TD\ on triangle-free cubic graphs, in fact, triangle-free $d$-regular graphs for each $d\geq 1$ is not known. Since regular graphs may not be triangle-free in general, the algorithmic complexity of \CDC\  for cubic graphs can not be inferred from the results on \TD.    

In this paper, we study the \CDC\ on regular graphs and prove the following theorem.

\begin{theorem}
    \label{thm:regular}
    \CDC\ is \NPC\ on triangle-free $d$-regular graphs, for each fixed integer $d\geq 3$. \TETHS.
\end{theorem}
   It is fascinating to observe how \textit{cd-coloring},  a relatively new graph problem, sheds light on a classic problem of total domination. For instance, as an additional advantage, we can use Theorem~\ref{thm:regular} to obtain the following corollary which gives results on \TD\ by recalling the fact that for any triangle-free graph $G$, \raisebox{1.5pt}{$\gamma$}$_t(G)=\XCD(G)$. It is remarkable that the following corollary is stronger than the existing results of \TD\ on regular graphs.
 
 \begin{corollary}
\label{thm:total_domination_regular}
\TD\ on triangle-free $d$-regular graphs is \NPC, for any constant $d\geq 3$. \TETHS.    
\end{corollary}

%\vspace{-0.5cm}
\subsection{\textbf{Separated-Cluster and $cd$-coloring}}
\vspace{-0.05cm}
Interestingly, we can see that the  \textit{cd-chromatic number} $\XCD$ and \textit{separated-cluster number} $\SC$  follow a similar relationship as their classic counterparts of \textit{chromatic number} \raisebox{1.5pt}{$\chi$} and \textit{clique number} $\omega$. Since any two vertices in a separated-cluster of a graph $G$ cannot lie at a distance exactly 2 in $G$, we have $\XCD(G)\geq \SC(G)$ (as each vertex in $\SC(G)$ has to be in different color class). It is not difficult to see that, using the same idea of \textit{`Mycielskian'} construction~\cite{MycielColoriage55} (in classic coloring), we can have a family of graphs for which $\omega_s=2$ but $\XCD$ is arbitrarily large.

The following auxiliary graph construction is very useful in the study of separated-clusters and $cd$-coloring.

%\vspace{-0.5cm}
\begin{definition}[Auxiliary graph $G^*$~\cite{ShaluSandhyaLowBound17}]\label{def:aux}
    Given a graph $G$, the auxiliary graph $G^*$ is the graph having $V(G^*)=V(G)$ and $E(G^*)= \{uv:u,v\in V(G)$ and $d_G(u,v)=~2\}$, where  $d_G(u,v)$ denote the distance between the vertices $u$ and $v$ in $G$.
\end{definition}
%\vspace{-0.05cm}
%\vspace{-0.5cm}
From the definitions of the auxiliary graph $G^*$ of a graph $G$ and independence number  $\alpha(G^*)$  (the size of a maximum cardinality independent set in $G^*$), it is straightforward to observe that $\SC(G)=\alpha(G^*)$. Further, by the definition of the clique cover number $ k(G^*)$  (the minimum number of cliques needed to partition the vertex set of $G^*$), it is not difficult to infer that $\XCD(G)\geq k(G^*)$ (see Observation~\ref{obs:cliquecover} for details).
Note that the parameters $\alpha$ and $k$ are well-studied in the literature, particularly in connection with \textit{perfect graphs}.  \textit{A graph $G$ is perfect if, for any induced subgraph $H$ of $G$,  $\alpha(H)=k(H)$}.\\

\noindent\textbf{Our Results:} Motivated by the notion of perfect graphs in classic coloring theory, we introduce $cd$-perfectness.  
We say that a graph $G$ is \textit{$cd$-perfect} if for any induced subgraph $H$ of $G$, $\XCD(H)= \SC(H)$. Though some earlier researchers~\cite{ShaluKiruCDTreeCobipar21, ShaluSandhyaLowBound17} observed that each graph $G$ belonging to certain graph classes like co-bipartite graphs, trees etc. satisfy the condition that $\XCD(G)= \SC(G)$, it is in this paper, the notion of $cd$-perfectness is introduced for the first time. This general notion of \textit{cd-perfectness} allows us to unify several (existing) results concerning \textit{$cd$-coloring} and \textit{separated-clusters} for various graph classes.

As a first step to prove the structure of $cd$-perfect graphs, we prove a sufficient condition for a graph $G$ to satisfy the equality, $\XCD(G)= k(G^*)$  (see Theorem~\ref{thm:suff_cdclique}). 
This leads us to Theorem~\ref{thm:suff_cdsubclique}, where the collection of graphs $\mathcal{H}$ is as shown in Figure~\ref{fig:c6-free}. 

 % Figure environment removed

\begin{theorem} \label{thm:suff_cdsubclique}
     Let $G$ be an $\mathcal{H}$-free graph and $G^*$ its  auxiliary graph. If $k(G^*)=\alpha(G^*)$ then $\XCD(G)=\SC(G)$. Consequently, if $G$ is $\mathcal{H}$-free and $G^*$ is perfect, then $\XCD(G)=\SC(G)$.
 \end{theorem}

As an immediate corollary of Theorem~\ref{thm:suff_cdsubclique}, we obtain a sufficient condition for a graph to be $cd$-perfect (see Corollary~\ref{corr:suffcdperfect}). 
 We then use this result to bring several graph classes, like \textit{co-bipartite graphs}, \textit{chordal bipartite graphs}, etc., under the common umbrella of \textit{$cd$-perfect} graphs. Even though these results are structural in nature, they have several exciting algorithmic consequences. It is interesting to note that the same framework can be used as a tool to derive both positive and negative results concerning the algorithmic complexity of \CDC\ and \SCP. In particular, we have the following results.

\begin{enumerate}[label=(\roman*)]
   
    \item A beautiful interplay between the problems, {\sc Total Domination}, \CDC, and \SCP\ can be witnessed in Theorem~\ref{thm:chordalbip}, where we use Corollary~\ref{corr:suffcdperfect}, Theorem~\ref{thm:totdom}, and the fact that \raisebox{1.5pt}{$\gamma$}$_t(G)=\XCD(G)$ for triangle free-graphs $G$, to prove that the three problems mentioned above are equivalent and solvable in $O(n^2)$ time for chordal bipartite graphs. Consequently, this result improves and generalizes the existing $O(n^3)$-time algorithm for \SCP\ on the class of $P_6$-free chordal bipartite graphs~\cite{ShaluKiruP5Free22}. 
%\vspace{-0.5cm}
\begin{theorem}\label{thm:chordalbip}
    Let $G=(A,B,E)$ be a chordal bipartite graph. Then $G$ is $cd$-perfect. Consequently, {\sc Total Domination}, \CDC, and \SCP\  are all equivalent problems for chordal bipartite graphs and can be solved in $O(n^2)$ time.
\end{theorem}
 %  \vspace{-0.5cm}

     \item  We provide a unified approach (most of the time with an improvement) for finding polynomial-time algorithms for \CDC\  on certain graph classes. For instance, we prove the following theorem. 
%\vspace{-0.5cm}
    \begin{theorem} \label{thm:properalpha}
The \CDC\ problem can be solved in $O(n^{2.5})$ time for proper interval graphs and $3K_1$-free graphs.
\end{theorem}
     
    \item We also derive the following hardness results for $C_6$-free bipartite graphs.
%\vspace{-0.5cm}
    \begin{theorem}\label{thm:C_6free hard}
    The problems  \CDC\ and \SCP\ are NP-Complete for $C_6$-free bipartite graphs.
\end{theorem}

 % \vspace{-0.5cm} 
    \item Further, as an additional result, we prove the following theorem concerning separated-cluster on interval graphs which settles an open problem in~\cite{ShaluSandhyaLowBound17}.
%\vspace{-0.5cm}
    \begin{theorem}
    \label{thm:sepiterval}
    The \SCP\ problem in interval graphs can be solved in polynomial time.
\end{theorem}

\end{enumerate}
\section{Preliminaries}
In this section, we describe the necessary background for automated planning and the significance of the International Planning Competition. 

% \subsection{Ontology}
% A formal ontology is typically represented as a set of concepts, relations, and axioms. A concept represents a set of objects or entities that share common properties, while a relation represents a connection or association between two or more concepts. Axioms are statements that define the relationships between concepts and relations. It is a formal representation of knowledge that is designed to facilitate automated reasoning and information processing. It acts as a structured vocabulary that describes a domain and promotes interoperability, data integration, and communication between humans and machines. Formally, an ontology $O$ can be represented as a tuple $(C, R, A)$, where $C$ is the set of concepts, $R$ is the set of relations, and $A$ is the set of axioms. Each concept \textit{c} $\in$ $C$ can be represented as a set of attributes, denoted as $Att(c)$. Similarly, each relation \textit{r} $\in$ $R$ can be represented as a set of attributes, denoted as $Att(r)$.

% Ontology is a branch of philosophy that deals with the nature of existence and being. In the field of computer science, however, ontology refers to a formal representation of knowledge that is designed to facilitate automated reasoning and information processing. It is a structured vocabulary that describes a domain and promotes interoperability, data integration, and communication between humans and machines. Various tools and methodologies, including Protege and ontology editors, are available for ontology creation. Ontologies are increasingly important in artificial intelligence, knowledge engineering, and the semantic web, and researchers are exploring their potential in diverse domains and applications.

% Figure environment removed

\subsection{Automated Planning}

Automated planning, also known as AI planning, is the process of finding a sequence of actions that will transform an initial state of the world into a desired goal state \cite{ghallab2004automated}. It involves constructing a plan or a sequence of actions that will achieve a specified objective while respecting any constraints or limitations that may be present. Formally, automated planning can be defined as a tuple $(S, A, T, I, G)$, where:
\begin{itemize}
    \item $S$ is the set of possible states of the world
    \item $A$ is the set of possible actions that can be taken
    \item $T$ is the transition function that describes the effects of taking an action on the current state of the world
    \item $I$ is the initial state of the world
    \item $G$ is the desired goal state
\end{itemize}
Using this notation, the problem of automated planning can be framed as finding a sequence of actions $\prec a_1, a_2, ..., a_k\succ$ that will transform the initial state $I$ into the goal state $G$, while respecting any constraints or limitations on the actions. 
 % In automated planning, 
 A problem is defined in terms of a domain and a problem instance. The domain defines the possible actions that can be taken and the effects of each action, while the problem instance specifies the initial state of the world and the desired goal state. 
Various techniques can be used to solve the planning problem, such as search algorithms, constraint-based reasoning, and optimization methods. These techniques involve exploring the space of possible plans and selecting the one that satisfies the objective and any constraints. Figure \ref{fig:planning_bw} illustrates an automated planning scenario for the blocksworld domain, where an initial state can be transformed into a goal state by executing a sequence of actions.

% \noindent \textbf{Attributes modeled about a domain.}
%   %\noindent \textbf{Attributes modeled in a domain file}
%  \begin{enumerate}
%      \item \textbf{Requirements:} A list of requirements that the planner must satisfy in order to solve the domain. Requirements include durative actions, conditional effects, or negative preconditions. For example, in blocksworld domain with types involved, one of the requirements is \emph{typing}.
%     \item \textbf{Predicates:} Predicates are fundamental elements in the planning domain that define the properties of the world. They are used to describe the initial and goal states, as well as the preconditions and effects of actions. Predicates are usually defined as logical expressions over a set of variables, where each variable can take on a finite number of values. In the context of planning, predicates are typically used to represent facts about the world that can be true or false, such as the location of an object or the status of a machine. For example, in blocksworld domain, the predicate \verb|(on b1 b2)| could indicate that block 'b2' is on top of block 'b1'.
%      \item \textbf{Actions:} Actions are the basic units of change in the planning domain. They represent atomic operations that can be performed to transform the world from one state to another. Each action has a name, a set of parameters, preconditions that must be satisfied before the action can be executed, and effects that describe the changes that the action makes to the world. Actions can be used to model a wide variety of operations, ranging from simple movements or transformations to complex processes such as planning or decision-making. For example, in blocksworld domain, the action \verb|unstack b2 b1| can be used to unstack block 'b2' from block 'b1'. 
     
%      \item \textbf{Preconditions:} Preconditions are the conditions that must be true before an action can be executed. They are usually defined using predicates and can involve multiple variables. Preconditions can also be negative, which means that a certain condition must not be true for an action to be executed. In planning, preconditions ensure that actions are only executed when the necessary conditions have been met, such as ensuring that a machine is turned off before it is serviced. For example, in blocksworld domain, the action \verb|unstack b2 b1| has a precondition of \verb|(on b1 b2)|, meaning that for the action to be valid, the block 'b2' should be on top of block 'b1'.
     
%      \item \textbf{Effects:} Effects describe the changes that an action makes to the world. They are usually defined using predicates and can involve multiple variables. Effects can be positive, which means that a certain condition becomes true after the action is executed, or negative, which means that a certain condition becomes false after the action is executed. In the context of planning, effects are used to model the changes that result from executing an action, such as moving an object from one location to another or turning a machine on. For example, in blocksworld domain, when the action \verb|unstack b2 b1| is executed, one of its effect is \verb|(not (on b1 b2))|, indicating that block 'b2' is no longer on top of block 'b1'.
     
%      \item \textbf{Constants:} Constants are values that are fixed and do not change during the execution of the planning problem. They are used to represent objects or entities in the world that have a fixed value, such as the speed limit on a road. Constants can be used to simplify the planning problem by reducing the number of variables that need to be considered and by providing a fixed set of values that can be used in predicates and actions. For example, in blocksworld domain, the constant \emph{table} could represent the surface on which the blocks are initially placed.
     
%      \item \textbf{Types:} Types are used to classify objects or entities in the world based on their attributes or properties. They are used to define the domain of values that a variable can take on and can be used to constrain the values that are assigned to variables. In the context of planning, types are typically used to group related objects or entities together, such as cars or bicycles, and to specify the properties that are common to all members of a type, such as their color or size. For example, in blocksworld domain with types involved, one can represent the predicate as \verb|(on ?x - block ?y - block)| stating that the parameters in the predicate are of type \emph{block}.

%  \end{enumerate}


% ######### Shorter version for AI Planning preliminaries
% \subsection{Automated Planning}

% Automated planning, also known as AI planning, finds actions transforming an initial world state into a goal state \cite{ghallab2004automated}. It involves creating a plan, respecting constraints, defined as $(S, A, T, I, G)$ where $S$ is the world states set, $A$ is the actions set, $T$ is the state transition function, $I$ is the initial state, and $G$ is the goal state. The challenge is to find actions $\prec a_1, a_2, ..., a_k\succ$ converting $I$ to $G$ under constraints. 

% A problem has a domain (defining actions and effects) and an instance (specifying initial and goal states). Various techniques can be used to solve the planning problem, such as search algorithms, constraint-based reasoning, and optimization methods. These techniques involve exploring the space of possible plans and selecting the one that satisfies the objective and any constraints. Figure \ref{fig:planning_bw} illustrates an automated planning scenario for the blocksworld domain, where an initial state can be transformed into a goal state by executing a sequence of actions.

\noindent \textbf{Attributes modeled about a domain.}
 \begin{enumerate}
     \item \textbf{Requirements:} A list of requirements that the planner must satisfy to solve the given domain, e.g., \emph{typing} in blocksworld with types.
     \item \textbf{Predicates:} Define world properties, e.g., \verb|(on b1 b2)| in blocksworld.
     \item \textbf{Actions:} Units of change with preconditions and effects, e.g., \verb|unstack b2 b1| in blocksworld.
     \item \textbf{Preconditions:} Conditions for action execution, e.g., \verb|(on b1 b2)| for \\ \verb|unstack b2 b1|.
     \item \textbf{Effects:} Post-action world changes, e.g., \verb|(not (on b1 b2))| after \\ \verb|unstack b2 b1|.
     \item \textbf{Constants:} Fixed values, e.g., \emph{table} in blocksworld.
     \item \textbf{Types:} Classifications based on attributes, e.g., \\ \verb|(on ?x - block ?y - block)| in typed blocksworld.
 \end{enumerate}

\noindent \textbf{Attributes modeled about a problem instance from a domain.}
\begin{enumerate}
    \item \textbf{Name:} The name of the planning problem.
    \item \textbf{Domain:} The name of the planning domain that the problem belongs to.
    \item \textbf{Objects:} A list of objects that are present in the planning problem. Objects are typically defined in terms of their type and name. In the example shown in Figure \ref{fig:planning_bw}, objects are b1, b2, and b3.
    \item \textbf{Initial State:} A description of the initial state of the world, including the values of all relevant predicates. Figure \ref{fig:planning_bw} represents an example initial state.
    \item \textbf{Goal State:} A description of the desired goal state of the world, including the values of all relevant predicates. Figure \ref{fig:planning_bw} represents an example goal state.
\end{enumerate}

% \vspace{2cm}
\subsection{International Planning Competition (IPC)}

% IPC serves as a significant means of assessing and comparing various planning systems. By presenting new planners and benchmark problems each year, the competitions aim to stimulate the advancement of new planning methodologies and reflect current trends and challenges in the field. The competition comprises multiple tracks, each covering various planning problems such as classical, temporal, and probabilistic planning. These tracks include benchmark problems that evaluate the performance of planners concerning parameters such as plan quality, plan length, and run time. The results of these competitions provide insights into the current state-of-the-art in planning and help identify the strengths and weaknesses of different planning systems. IPC can serve as an excellent starting point for building a planning-related ontology as the benchmark problems used in these competitions can provide a comprehensive overview of the domain and the types of problems that planners need to solve. 

IPC is pivotal for evaluating and contrasting planning systems. Introducing new planners and benchmarks, it promotes innovative planning methodologies and reflects the field's evolving challenges. The competition has multiple tracks, such as classical and probabilistic planning, with benchmarks assessing plan quality, length, and run time. IPC results offer a glimpse into the latest in planning, highlighting system pros and cons. The benchmarks from IPC are ideal for crafting a planning-related ontology, encapsulating the domain's breadth and planners' challenges.

\section{Total domination and $cd$-coloring in triangle-free $d$-regular Graphs}
\label{sec:hardness}
In this section, we derive the hardness results for \CDC\ and \TD\ in triangle-free $d$-regular graphs for each fixed $d\geq 3$, and also prove that for each fixed integer $d\geq 3$, there exists a family of $d$-regular graphs for which the difference between $\XCD$ and $\gamma_t$ is arbitrarily high.
\subsection{Hardness results}
Here, we obtain the hardness results for \CDC\ in triangle-free $d$-regular graphs for each fixed $d\geq 3$.  First, we propose a simple linear reduction for triangle-free cubic graphs. Then, we generalize this result for  triangle-free $d$-regular graphs, for each fixed $d\geq 3$, by using  two linear reductions; one for odd values of $d$ and the other for even values.
We begin by stating the following theorem, which is one of the main results of this section.  


\begin{theorem}
    \label{thm:regular}
    \CDC\ is \NPC\ on triangle-free $d$-regular graphs, for each fixed integer $d\geq 3$. \TETHS.
\end{theorem}

To prove that a problem does not admit a subexponential-time algorithm, it is sufficient to obtain a linear reduction from a problem known not to admit
a subexponential-time algorithm, where a linear reduction is a polynomial-time reduction
in which the size of the resultant instance is linear in the size of the input instance. 
All reductions mentioned in this section are trivially linear.
We refer to the book~\cite{DBLP:books/sp/CyganFKLMPPS15} for more deatils about these concepts.
To prove the above theorem, we use three constructions. 
Construction~\ref{cons:cubic} is used for a reduction from \textsc{Total Domination}  on bipartite graphs  with bounded degree 3 to \CDC\ on triangle-free cubic graphs. Construction~\ref{cons:d-regular odd} is used for a reduction from \CDC\ on triangle-free $(d-2)$-regular graphs to \CDC\ on triangle-free $d$-regular graphs, for each odd integer $d\geq 5$, whereas the Construction~\ref{cons:d-regular even} is used for a reduction from \CDC\ on triangle-free $(d-1)$-regular graphs to \CDC\ on triangle-free $d$-regular graphs, for each even integer $d\geq 4$. 

% Figure environment removed
\begin{construction}
\label{cons:cubic}
    Let $G$ be a bipartite graph with bounded degree 3. We construct a gadget $W$ in the following way: Introduce a star graph $K_{1,2}$, which we denote by $S$, having $a$ as the root vertex, and $b,c$ as the leaves. Note that we also call $`a$' as the root vertex of $W$. Then we introduce two copies of the complete bipartite graph $K_{2,2}$, denoted by $CB_1$ and $CB_2$, respectively. Now the leaf $b$ (resp. $c$) of $S$ is adjacent to one of the partition $A_1$ (resp. $A_2$) of $CB_1$ (resp. $CB_2$).
    Furthermore,  each vertex of partition $B_1$ of $CB_1$ is adjacent to exactly one vertex of the other partition $B_{2}$ of $CB_{2}$  as shown in Figure~\ref{fig:cubic_gadget}. 
    Thereafter we construct a graph $G_c$ from $G$ using the gadget $W$ as follows: for every vertex $u$ of $G$ with degree 2, introduce a gadget $W$ such that the root vertex $a$ of $W$ is adjacent to $u$ in $G$. Similarly, for each  vertex $v$ in $G$ of degree 1, we introduce  two copies of  $W$ such that the root vertices of both the gadgets  are adjacent to $v$ in $G$.   
    The graph $G_c\setminus G$ contains $11x+22y$ vertices, where $x$ and $y$ denote the number of vertices in $G$ having degree 2 and degree 1, respectively. 
\end{construction}

 % Figure environment removed  

An example of the Construction~\ref{cons:cubic} corresponding to the bipartite graph $G$ with bounded degree 3 given in Figure~\ref{fig:bipartite} is as shown in Figure~\ref{fig:Cubic construction}.

 % Figure environment removed

Let $W$ be a gadget constructed in Construction~\ref{cons:cubic}. Let $H$ be a subgraph of $W$ induced by the vertices in two complete bipartite graphs $CB_1$ and $CB_2$, and the two leaves $b$ and $c$ of the star graph $S$.  Let $H'=H-\{b,c\}$. The following observation is true for the gadget $W$ in Construction~\ref{cons:cubic}.
\begin{observation}
    \label{obs:Hcoloring_cubic} 
     Let $W$ be the gadget defined in Construction~\ref{cons:cubic}.
     Let $H$ be a subgraph of $W$ induced by the vertices in two complete bipartite graphs $CB_1$ and $CB_{2}$, and the two leaves $b$  and $c$ of the star graph $S$. Let 
     $H'=H- \{b,c\}$. In any CD-coloring of $H$, the vertices in $H'$ require at least 4 colors. (Refer Figure~\ref{fig:cubic_gadget} for $H$ and $H'$). Moreover, there exists a $cd$-coloring of $W$ using exactly 4 colors.
\end{observation}

\begin{proof}
    The graph $H'$ contains an induced $C_6$ (for instance, the graph induced by the vertices $d,i,j,f,k,h$ as shown in Figure~\ref{fig:cubic_gadget}). It is easy to see that that the \XCD($C_6$)=4. Now consider the subgraph $H'$. It is obtained by 
     introducing new vertices adjacent to some vertices of the $C_6$.  Since none of the $3K_1$s present in the $C_6$ is dominated by any of the remaining vertices in $H'$, by Observation~\ref{obs:cdcolor_reduce}, it is clear that the number of colors needed for any $cd$-coloring of $H'$ is at least 4.  Further, note that all the independent sets in $H'$, which are dominated by $b$ or $c$, are already dominated by some vertex in $H'$. Therefore, again by Observation~\ref{obs:cdcolor_reduce}, it is clear that the number of colors needed for $V(H')$ in any $cd$-coloring of $H$ is at least 4. 
     Since the root vertex $a$ of the gadget is not adjacent to any of the vertices in $H'$, the presence of $a$ will not reduce the number of colors needed for $cd$-coloring of the subgraph $H'$ of the gadget $W$. Hence the number of colors needed for $cd$-coloring of $W$ is at least $4$. 
    Now it is easy to see that a 4-$cd$-coloring of any $C_6$ in $H'$ can be  extended to a 4-$cd$-coloring of the gadget $W$ (by using the leaves of star graph $S$, and one of the vertices from each $CB_i$, for $i=\{1,2\}$ as the dominating vertices of the color classes).
\end{proof}



The following lemma is based on the Construction~\ref{cons:cubic}, and as a consequence of which we have Theorem~\ref{thm:cubic}.

\begin{lemma}
    \label{lem:cubic}
    Let $G$ be a bipartite graph with bounded degree 3. Then $\gamma_t(G)=k$ if and only if $\XCD(G_c)=k+4x+8y$, where  $x$ and $y$  denote the number of vertices in $G$ having degree 2 and degree 1, respectively. 
\end{lemma}

\begin{proof}
    Let $\gamma_t(G)=k$. Then by Proposition~\ref{pro:triangle-free}, since $G$ is triangle-free, we have $\XCD(G)=k$. We claim that $\XCD(G_c)=k+4x+8y$.  Recall that the graph $G_c$ contains $x+2y$ gadgets, as there are $x$ number of degree two vertices, and $y$ number of degree one vertices in $G$. By Observation~\ref{obs:Hcoloring_cubic}, it is clear that these gadgets require exactly $4(x+2y)$ new colors, which are not used in $G$ (as none of these colors can be reused for coloring the vertices in $V(G_c)\setminus V(W)$).  Since \XCD($G$)=$k$, we then have \XCD($G_c$)=$k+4x+8y$.
    
\smallskip

    Now for the reverse direction, let $\XCD(G_c)=m$. Then, we claim that $\gamma_t(G)=k$, where $k=m-(4x+8y)$. 
    Note that for each gadget $W$, the subgraph $W- \{a\}$ is disjoint from $G$, as only the vertex $a$ in $W$ is adjacent to exactly one vertex  $u$ in $G$. Hence the color class dominated by $a$ can include only one vertex $u$ in $G$.  By Observation~\ref{obs:Hcoloring_cubic}, in any $cd$-coloring of $G_c$, the vertex set $H'$ of any gadget $W$ needs at least 4 colors, and none of these colors can be reused for coloring the vertices in $V(G_c)\setminus V(W)$. Again by Observation~\ref{obs:Hcoloring_cubic}, the same 4 colors used to color the vertices in $H'$ can be reused to color the entire vertices of $W$. Therefore, we now have a $cd$-coloring of $G_c$ using $m$ colors such that no colors used for the vertices in gadgets are used to color any vertex in $G$. Thus, $\XCD(G)=m-4(x+2y)=k$. Then by Proposition~\ref{pro:triangle-free}, $\gamma_t(G)=k$.
\end{proof}





\begin{theorem}
    \label{thm:cubic}
    \CDC\ is \NPC\ on triangle free cubic graphs. \TETHS.
\end{theorem}

\begin{proof}
    Notice that $G_c$ has $|V(G)|+ 11(x+2y)$ vertices, where $x$ and $y$ are the number of vertices in $G$ having degree 2 and degree 1, respectively. We know that there is a linear reduction from \textsc{Total Domination} in bipartite graph with bounded degree 3 to \CDC\ on triangle free cubic graphs due to Lemma~\ref{lem:cubic}. Thus, by using Proposition~\ref{pro:bipartite}, we are done.
\end{proof}




Now, we generalize  Construction~\ref{cons:cubic} to prove the hardness of \CDC\ on triangle-free $d$-regular graphs, for any constant $d\geq 4$. The following construction is used to prove the hardness of \CDC\ on triangle-free $d$-regular graphs, for any odd integer $d\geq 5$. 
 \begin{construction}
    \label{cons:d-regular odd}
     Let $G$ be a triangle-free $(d-2)$-regular graph, for any odd integer $d\geq 5$. 
     We construct a graph $G_c$ from $G$ %using a gadget $W$ 
    in the following way: 
%    \vspace{-0.25cm}
    \begin{itemize}
        \item First we construct a gadget $W$ as follows:  introduce a star graph $K_{1,d-1}$, which we denote by $S$, having the vertex $a$ as the root vertex. Note that we also call `$a$' as the root vertex of $W$.
        Further introduce $2(d-1)^2$ vertices which induces $d-1$ disjoint copies of complete bipartite graphs $K_{d-1,d-1}$, namely $C_{1},  C_{2}\ldots,C_{d-1}$, respectively.      
        The adjacency between these complete bipartite graphs $C_i=(A_i,B_i)$, for $1\leq i\leq (d-1)$, and the star graph $S$ is in such a way that the vertices of $A_i$ of the complete bipartite graph $C_i$ is adjacent to the leaf vertex, say, $v_i$  of $S$.
        Furthermore, for each odd integer $i\leq (d-2)$, each vertex in the partite set $B_i$ of $C_i$ is adjacent to exactly one vertex of the partite set $B_{i+1}$ of $C_{i+1}$ (for instance refer  Figure~\ref{fig:d-regular_gadget odd}, when $d=5$). Now, by $H_i'$, we denote the subgraph of $W$ induced by the vertices in the two complete bipartite graphs $C_i$ and $C_{i+1}$, for an odd integer $i\leq d-2$.  Then, by $H_i$, we denote the subgraph of $W$ induced by the vertices in $H_i'$ and the two leaves $v_i$ and $v_{i+1}$ of the star graph $S$ which are adjacent to the partitions $A_i$ and  $A_{i+1}$ of $C_i$ and $C_{i+1}$, respectively. Note that $H_i'=H_i-\{v_i,v_{i+1}\}$. The gadget $W$ contains $(2d^2-3d+2)$ vertices.  
     
     \item Now the graph $G_c$ is constructed from $G$ using $W$ in such a way that for each vertex $u$ in $G$, introduce two copies of $W$ such that root vertices of both the copies of $W$ is attached to $u$. 
     \end{itemize}
     Note that  $G_c-G$ contains $2n(2d^2-3d+2)$ vertices, where $n$ is the number of vertices in $G$.
\end{construction}

An example of the gadget $W$ in Construction~\ref{cons:d-regular odd} is shown in Figure~\ref{fig:d-regular_gadget odd}. We have the following observation for the gadget $W$ of Construction~\ref{cons:d-regular odd}. 

% Figure environment removed

\begin{observation}
    \label{obs:Hcoloring_regular_odd} 
     Let $W$ be a gadget, and  
    $H_i$ as well as $H_i'$ be the subgraphs of $W$ as defined in Construction~\ref{cons:d-regular odd}. 
     In any $cd$-coloring of $H_i$, the vertices in $H_i'$ require at least 4 colors. 
     Moreover, there exists a $cd$-coloring of $W$ using exactly $2(d-1)$ colors.
\end{observation}

\begin{proof}
    The graph induced by $H_i'$ contains a $C_6$ (for instance: induced by the vertices $d1,e1,g4,f4,g3,$ $e2$ as shown in Figure~\ref{fig:d-regular_gadget odd}). 
    It is clear that \XCD($C_6$)=4. 
    Now consider the subgraph  $H_i'$. It is obtained by 
    the introduction of new vertices adjacent to some vertices of the $C_6$.   Since none of the $3K_1$s present in the $C_6$ is dominated by any of the remaining vertices in $H_i'$, by Observation~\ref{obs:cdcolor_reduce}, it is clear that the number of colors needed for any $cd$-coloring of $H_i'$ is at least 4. Further, note that all the independent sets in $H_i'$, which are dominated by the leaf vertices $v_i$ or $v_{i+1}$ of the star graph $S$ (for instance the vertices $b$ and $c$ shown in Figure~\ref{fig:d-regular_gadget odd}), are already dominated by some vertex in $H_i'$. Therefore, again, by Observation~\ref{obs:cdcolor_reduce}, it is clear that the number of colors needed for $V(H_i')$ in any $cd$-coloring of $H_i$ is at least 4. 
    Since the root vertex $a$ of the gadget is not adjacent to any of the vertices in $H_i'$, the presence of $a$ will not reduce the number of colors needed for $cd$-coloring of the subgraph $H_i'$ of the gadget $W$. Hence, the number of colors needed for $cd$-coloring of $H_i$ is at least $4$. Therefore, the number of colors needed for $cd$-coloring of $W$ is at least $2(d-1)$, as there are $(d-1)/2$ copies of $H_i$ is present in $W$. 
    Now it is easy to see that a 4-$cd$-coloring of a $C_6$ in each $H_i'$ can be extended to a $2(d-1)$-$cd$-coloring of the gadget $W$ (by using the leaves of star graph $S$, and one of the vertices from the partition $A_i$ of each $C_i$, for $1\leq i\leq d-1$, as the dominating vertices of the color classes). 
\end{proof}

The following lemma is based on the Construction~\ref{cons:d-regular odd}.


\begin{lemma}
    \label{lem:d-regular odd}
    For each odd integer $d\geq 5$, let $G$ be a triangle-free ($d-2$)-regular graph, having $n$ vertices. Then $\XCD(G)=k$ if and only if $\XCD(G_c)=k+4n(d-1)$. 
\end{lemma}

\begin{proof}
    Let $\XCD(G)=k$. We claim that $\XCD(G_c)=k+4n(d-1)$.  Recall that the graph $G_c$ contains $2n$ gadgets as there are $n$ vertices in $G$ each having degree $d-2$. By Observation~\ref{obs:Hcoloring_regular_odd}, it is clear that any $cd$-coloring of these gadgets needs a total of at least $4n(d-1)$ colors. Since \XCD($G$)=$k$, we then have \XCD($G_c$)=$k+4n(d-1)$. 

    Now, for the reverse direction, let $\XCD(G_c)=m$. Then we claim that $\XCD(G)=k$, where $k=m-4n(d-1)$. 
    Note that for each gadget $W$, the subgraph $W- \{a\}$ is disjoint from $G$, as only $a$ in $W$ is adjacent to exactly one vertex  $u$ in $G$. Hence, the color class dominated by $a$ can include only one vertex $u$ in $G$. 
     By Observation~\ref{obs:Hcoloring_regular_odd}, in any $cd$-coloring of $G_c$, the vertex set of each copy of $H'$ of any gadget $W$ needs at least 4 colors, and none of these colors can be reused for coloring the vertices in $V(G_c)\setminus V(W)$. Again, by Observation~\ref{obs:Hcoloring_regular_odd}, the same 4 colors used to color the vertices in $H_i'$ can be reused to color the entire vertices of $H_i$. Hence, the 2($d-1$) colors used for $(d-1)/2$ copies of $H_i$ in $W$ can be reused to color the entire vertices of $W$. Therefore, we now have a $cd$-coloring of $G_c$ using $m$ colors such that no colors used for the vertices in gadgets are used to color any vertex in $G$. Thus, $\XCD(G)=m-4n(d-1)=k$, as there are $2n$ copies of $W$s in $G_c$.  
\end{proof}

The following construction is used to prove the hardness of \CDC\ on triangle-free $d$-regular graphs, for any even integer $d\geq 4$.  
 

\begin{construction}
    \label{cons:d-regular even}
     Let $G$ be a triangle-free $(d-1)$-regular graph, for any even integer $d\geq 4$.  Let $W$ be a gadget which is constructed in the following way. 
     The gadget $W$ contains $4d^2-10d+6$ vertices with two subgraphs $W_1$ and $W_2$ each having $2d^2-5d+3$ vertices. The adjacency among the vertices in $W_1$ (resp. $W_2$) is in such  a way that the $d-1$ vertices of $W_1$ (resp. $W_2$) induces a star graph $K_{1,d-2}$,  which we denote by $S_1$ (resp. $S_2$), having the vertex $a$ in $W_1$ (resp. the vertex $b$ in $W_2$) as the root vertex. Note that the vertices $a$ and $b$ are adjacent. Further in each set $W_i$, for $i\in \{1,2\}$, introduce $(2d^2-6d+4)$ vertices which induces $d-2$ disjoint copies of complete bipartite graphs $K_{d-1,d-1}$, namely $CB_{i,1},  CB_{i,2}\ldots,CB_{(i,d-2)}$ respectively.      
     The adjacency between these complete bipartite graphs $CB_{i,j}$, for $i\in \{1,2\}$ and $1\leq j\leq d-2$ and star graph $S_i$,  is in such a way that the vertices of one of the partition $A_{ij}$  of a complete bipartite graph $CB_{i,j}$ is connected to the leaf $j$  of  $S_i$.       
     Furthermore, for each odd integer $j\leq d-3$, each vertex of other partition $B_{i,j}$ of $CB_{i,j}$ is adjacent to exactly one vertex of the partition $B_{i,(j+1)}$ of $CB_{i,(j+1)}$  (for instance, refer Figure~\ref{fig:d-regular_gadget even}, when $d=4$). 
     Now the graph $G_c$ is constructed from $G$ using $W$ in such a way that consider an arbitrary pair-wise ordering $(v_1,v_2), (v_3,v_4)\ldots,(v_{n-1}, v_n)$ of vertices, in $G$.  Note that  such a pairing is possible, since $n$ is even (as $d-1$ is odd). 
     For a pair of vertices $(v_j,v_{j+1})$, for odd integer $j\leq n-1$ in this ordering, introduce a gadget $W$ such that the vertex $a$ (resp. $b$) of $W$ is adjacent to $v_j$ (resp. $v_{(j+1)}$) of $G$.
     The graph $G_c$ contains $n(2d^2-5d+3)$ vertices, where $n$ is the number of vertices in $G$.
\end{construction}



An example of the gadget $W$ in Construction~\ref{cons:d-regular even} is shown in Figure~\ref{fig:d-regular_gadget even}

% Figure environment removed

    Let $W$ be a gadget constructed in Construction~\ref{cons:d-regular even}.
     Let $H$ be an induced subgraph obtained by the union of vertices of two complete bipartite graphs $CB_{ij}$ and $CB_{i(j+1)}$, for odd integer $j$, and the two leaves $c$ and $d$ of the star graph $S_i$ which is adjacent to the partition $A_j$ and $A_{j+1}$ of $CB_{ij}$ and $CB_{i(j+1)}$, respectively. Let 
     $H'=H\setminus \{c,d\}$. 

The following observation is true for the gadget $W$ of Construction~\ref{cons:d-regular even}. 
\begin{observation}
    \label{obs:Hcoloring_regular_even}
     Let $W$ be a gadget constructed in Construction~\ref{cons:d-regular even}.
     Let $H$ be an induced subgraph obtained by the union of vertices of two complete bipartite graphs $CB_{ij}$ and $CB_{i(j+1)}$, for odd integer $j$, and the two leaves $c$ and $d$ of the star graph $S_i$ which is adjacent to the partition $A_j$ and $A_{j+1}$ of $CB_{ij}$ and $CB_{i(j+1)}$, respectively. Let 
     $H'=H\setminus \{c,d\}$. In any CD-coloring of $H$, the vertices in $H'$ requires at least 4 colors. Moreover there exists a $cd$-coloring of $W$ using exactly  $4(d-2)$ colors.
\end{observation}

\begin{proof}
    Let $H$ be a subgraph of $W_1$ in $W$. The graph induced by $H'$ contains a $C_6$ (for example: induced by the vertices $d1,e1,g3,f3,g2,e2$ as shown in Figure~\ref{fig:d-regular_gadget even}). 
    It is clear that the \XCD($C_6$)=4. 
    Now consider the subgraph $H'$. It is obtained by 
    the introduction of new vertices adjacent to some vertices of the $C_6$.  Since none of the $3K_1$s present in the $C_6$ is dominated by any of the remaining vertices in $H'$, by Observation~\ref{obs:cdcolor_reduce}, it is clear that the number of colors needed for any $cd$-coloring of $H'$ is at least 4. 
    Further, note that all the independent sets in $H'$, which are dominated by $c$ or $d$, are already dominated by some vertex in $H'$. Therefore, again by Observation~\ref{obs:cdcolor_reduce}, it is clear that the number of colors needed for $V(H')$ in any $cd$-coloring of $H$ is at least 4. 
    Since the  vertex $a$ of the  $W_1$ is not adjacent to any of the vertices in $H'$, the presence of $a$ will not reduce the number of colors needed for $cd$-coloring of the subgraph $H'$ of the gadget $W$. Hence the number of colors needed for $cd$-coloring of $H$ is at least $4$. Therefore, the number of colors needed for $cd$-coloring of $W_1$ is at least $2(d-2)$, as there are $(d-2)/2$ copies of $H$ is present in $W_1$. Similarly,  the number of colors needed for $cd$-coloring of $W_2$ is at least $2(d-2)$, as there are $(d-2)/2$ copies of $H$ is present in $W_2$. Hence, it is clear that the number of colors needed for $cd$-coloring of $W$ is at least $4(d-2)$.
    Now it is easy to see that a 4-$cd$-coloring of a $C_6$ in each $H'$ can be  extended to a 2$(d-2)-cd$-coloring of the subgraph $W_i$  for $i\in \{1,2\}$, (by using the leaves of star graph $S_i$, and one of the vertices from the partition $A_{ij}$ of each $CB_{ij}$, for $1\leq j\leq (d-2)/2$, as the dominating vertices of the color classes). Now it is obvious that, these 2$(d-2)$-$cd$-coloring of the subgraph $W_i$ and $W_2$ can be extended to a 4$(d-2)$-$cd$-coloring of $W$. 
\end{proof}

The following lemma is based on the Construction~\ref{cons:d-regular even}.


\begin{lemma}
    \label{lem:d-regular even}
    For even integer $d\geq 4$, let $G$ be a triangle-free $(d-1)$-regular graph having $n$ vertices. Then $\XCD(G)=k$ if and only if $\XCD(G_c)=k+2n(d-2)$. 
\end{lemma}

\begin{proof}
    Let $\XCD(G)=k$. We claim that $\XCD(G_c)=k+2n(d-2)$. Recall that the graph $G$ contains even number of vertices as it is a $(d-1)$-regular graph, where $d-1$ is odd. Hence $G_c$ contains $\frac{n}{2}$ gadgets as there are $n$ vertices in $G$ each having degree $d-1$. By Observation~\ref{obs:Hcoloring_regular_even}, it is clear that these gadgets need at least $2n(d-2)$ colors. Since \XCD($G$)=$k$, we then have \XCD($G_c$)=$k+2n(d-2)$. 

    Now for the reverse direction, let $\XCD(G_c)=m$. Then we claim that $\XCD(G)=k$, where $k=m-2n(d-2)$.  Note that only the vertices $a$ and $b$ of each gadget $W$ is adjacent to some vertices  $u$ and $v$ respectively in $G$. Hence the color class dominated by $a$ (resp. $b$) can include only one vertex $u$ (resp. $v$) in $G$.   
    By Observation~\ref{obs:Hcoloring_regular_even}, in any $cd$-coloring of $G_c$, the vertex set of each copy of $H'$ of any gadget $W$ needs at least 4 colors, and none of these colors can be reused for coloring the vertices in $V(G_c)\setminus V(W)$. Again by Observation~\ref{obs:Hcoloring_regular_even}, the same 4 colors used to color the vertices in $H'$ can be reused to color the entire vertices of $H$. Hence, the 4($d-2$) colors used for $(d-2)$ copies of $H$ in $W$ can be reused to color the entire vertices of $W$. Therefore, we now have a $cd$-coloring of $G_c$ using $m$ colors such that no colors used for the vertices in gadgets are used to color any vertex in $G$. Thus, $\XCD(G)=m-2n(d-2)=k$, as there are $\frac{n}{2}$ copies of $W$s in $G_c$.  
\end{proof}



\noindent\textit{Proof of Theorem~\ref{thm:regular}:} Note that $|V(G_{c1})|=2n(2d^2-3d+2)$, where $G_{c1}$ is obtained as per construction~\ref{cons:d-regular odd}. Similarly, $|V(G_{c2})|=n(2d^2-5d+3)$, where $G_{c2}$ is obtained as per construction~\ref{cons:d-regular even}. Hence, both constructions are linear with respect to their size of
inputs and hence their associated reductions are also linear reductions. 
Thus, we know that there is a linear reduction from triangle-free $(d-2)$-regular graphs to triangle-free $d$-regular graphs for odd integer $d\geq 5$ due to Lemma \ref{lem:d-regular odd}. 
We also know that there is a linear reduction from triangle-free $(d-1)$-regular graphs to triangle-free $d$-regular graphs  for even integer $d\geq 4$ due to Lemma \ref{lem:d-regular even}. Thus, we are done using Proposition~\ref{pro:bipartite} and Theorem~\ref{thm:cubic}.

The following corollary is obtained from Theorem~\ref{thm:regular} and Proposition~\ref{pro:triangle-free}.

\begin{theorem}
\label{thm:total_domination_regular}
Total domination in triangle-free $d$-regular graphs  is \NPC, for any constant $d\geq 3$. \TETHS.
\end{theorem}

\input{stuctural/bounds_regular_td}

\section{Separated-Cluster and cd-coloring: cd-perfectness}
\label{sec:structural}


\label{subsec:cdprfect}
Here we introduce the notion of \textit{cd-perfect graphs}, which is defined below.
\begin{definition}[\textit{cd}-perfect graphs]
    An undirected graph $G$ is said to be $cd$-perfect if, for any induced subgraph $H$ of $G$, we have $\XCD(H)=\SC(H)$.
\end{definition}

In this section, we  study $\XCD(G)$ and $\SC(G)$ of a graph $G$, 
by relating them to some well-known graph parameters of the auxiliary graph $G^*$. This simple reduction has various interesting consequences. Recall from Definition~\ref{def:aux} that given a graph $G$, the auxiliary graph $G^*$ is the graph having $V(G^*)=V(G)$ and $E(G^*)=E(G^2)\setminus E(G)$. i.e. $E(G^*)= \{uv:u,v\in V(G)$ and $d_G(u,v)=2\}$. Note that throughout this section, we use the definition that a set $S\subseteq V(G)$ is a \textit{seperated-cluster}, if for any pair of vertices $u,v\in V(G)$, we have $d_G(u,v)\neq 2$.


\medskip

 The following proposition trivially follows from the definition of $G^*$.
\begin{proposition} \label{pro:omega_s}
    For any graph $G$, we have $\SC (G)=\alpha(G^*)$.
\end{proposition}

We then note the observation below.
\begin{observation} \label{obs:cliquecover}
    For any graph $G$, we have $\XCD (G)\geq k(G^*)$
\end{observation}
\begin{proof}
    Let $\XCD(G)=l$, where $I_1,I_2,\ldots, I_l$ denote the corresponding color classes of $G$. Let $j\in \{1,2,\ldots,l\}$. Then by the definition of $cd$- coloring, we have that $I_j$ is an independent set in $G$ and there exists a vertex $v_j\in V(G)$ such that $I_j\subseteq N_G(v_j)$. This implies that for any pair of vertices $u,v\in V(I_j)$, we have $d_G(u,v)=2$. This further implies that $I_j$ is a clique in $G^*$, and therefore $\{I_1,I_2,\ldots, I_l\}$ is a clique cover of $G^*$. Thus we have $\XCD (G)\geq k(G^*)$.
\end{proof}

 % Figure environment removed
Note that the reverse inequality of Observation~\ref{obs:cliquecover} is not necessarily true. For instance, consider the graph $C_6$ (an induced cycle on 6 vertices). It is not difficult to verify that $\XCD(C_6)=4$. But as $C_6^*$ is a disjoint union of two triangles, we have that $k(C_6^*)=2<\XCD(C_6)$. In Theorem~\ref{thm:suff_cdclique}, we prove a sufficient condition for a graph $G$ to satisfy the equality in Observation~\ref{obs:cliquecover}. First, we define certain graphs: Consider $C_6$. Since $C_6$ is a bipartite graph, observe that $V(C_6)$ is a disjoint union of two independent sets, say $A$ and $B$, where $|A|=|B|=3$. We denote by $C_6^1$, $C_6^2$, and $C_6^3$, the graphs obtained by respectively adding \textit{1 edge, 2 edges, and 3 edges} to exactly one of the partite sets $A$ or $B$ of $C_6$ (see Figure~\ref{fig:c6-free}). Let $\mathcal{H}=\{C_6, C_6^1$, $C_6^2$, and $C_6^3\}$. First, we note a useful structural observation for graphs in $\mathcal{H}$.
\begin{observation}\label{obs:structureH}
    For each $H\in \mathcal{H}$, we have a set $S\subseteq V(H)$ such that $S$ is independent in $H$ with $|S|=3$. Moreover, the vertices in $S$ form an asteroidal-triple in $H$.
\end{observation}

Now, in the following lemma, we prove a property satisfied by $G^*$, when $G$ is restricted to be an $\mathcal{H}$-free graph. 
\begin{lemma}\label{lem:neighborhood}
Let $G$ be an $\mathcal{H}$-free graph and $G^*$ its corresponding auxiliary graph. Let $K$ be any clique in $G^*$. Then there exists a vertex $v_k\in V(G)$ such that $K\subseteq N_G(v_k)$.
\end{lemma}
\begin{proof}
    We prove this by induction on $|K|$. Consider the base case where $|K|=2$. Let $K=\{x,y\}$. Note that $K$ is a clique in $G^*$. By the definition of $G^*$, $xy\in E(G^*)$ implies that $d_G(x,y)=2$. i.e. there exists a vertex $v_k$ such that $x,y\in N_G(v_k)$. As this proves the base case, we can therefore assume that $|K|\geq 3$. By the induction hypothesis, we can also assume that the condition in the lemma holds for every clique, say $K'$ in $G^*$ with $|K'|<|K|$. Let $x_1,x_2,x_3\in K$ (this is possible, since $|K|\geq 3)$. For $i\in \{1,2,3\}$, let $K_i$ denote the clique $K\setminus \{x_i\}$. By the induction hypothesis, there exists a vertex, say $v_i\in V(G)$, such that $K_i\subseteq N_G(v_i)$. Note that $K$ is an independent set in $G$ (as $K$ is a clique in $G^*$), and therefore $v_i\notin K$ for any $i\in \{1,2,3\}$. Also, in particular, we have $v_ix_j\in E(G)$ for each $i,j\in \{1,2,3\}$, where $i\neq j$. Suppose that $v_ix_i\notin E(G)$ for each $i\in \{1,2,3\}$.  This implies that $v_1$, $v_2$, and $v_3$ are all distinct vertices in $G$. Consider the induced subgraph, $H=G[\{x_1,x_2,x_3,v_1,v_2,v_3\}]$. Since $\{x_1,x_2,x_3\}$ is an independent set in $G$, we then have, 

  $$H=\begin{cases} 
      C_6, & |E(G[\{v_1,v_2,v_3\}]|=0 \\
      C_6^1, & |E(G[\{v_1,v_2,v_3\}]|=1 \\
      C_6^2, & |E(G[\{v_1,v_2,v_3\}]|=2 \\
      C_6^3, & |E(G[\{v_1,v_2,v_3\}]|=3
   \end{cases}
$$
This contradicts the fact that $G$ is $\mathcal{H}$-free. Therefore, we have that $v_ix_i\in E(G)$ for some $i\in \{1,2,3\}$. Then, as $K=K_i\cup\{x_i\}$, the vertex $v_i$ in $G$ has the property that $K\subseteq N_G(v_i)$. This proves the lemma.
\end{proof}
\begin{theorem}\label{thm:suff_cdclique}
Let $G$ be an $\mathcal{H}$-free graph and $G^*$ its corresponding auxiliary graph. Then $\XCD (G)= k(G^*)$
\end{theorem}

\begin{proof}
By Observation~\ref{obs:cliquecover}, here it is enough to show that $\XCD (G)\leq k(G^*)$. Let $k(G^*)=t$, where $K_1,K_2,\ldots, K_t$ denote the corresponding cliques in a minimum clique cover of $G^*$. Since $G$ is $\mathcal{H}$-free, for each $j\in \{1,2,\ldots,t\}$, by Lemma~\ref{lem:neighborhood}, we have that there exists a vertex $v_j\in V(G)$ such that $K_j\subseteq N_G(v_j)$. As each set $K_j$ is an independent set in $G$ (since $K_j$ is a clique in $G^*$), we can now infer that each set $K_j$ is a valid color class for the $cd$-coloring of $G$. Since $V(G)$ is the disjoint union of the sets, $K_1,K_2,\ldots, K_t$, we can therefore conclude that $\XCD (G)\leq k(G^*)$. This proves the theorem.
 \end{proof}
 The following theorem is then immediate from Proposition~\ref{pro:omega_s} and Theorem~\ref{thm:suff_cdclique}. The theorem provides us with a sufficient condition for graphs $G$ to have $\XCD(G)=\SC(G)$.
 \begin{theorem} \label{thm:suff_cdsubclique}
     Let $G$ be an $\mathcal{H}$-free graph and $G^*$ its corresponding auxiliary graph. If $k(G^*)=\alpha(G^*)$ then $\XCD(G)=\SC(G)$. Consequently, if $G$ is $\mathcal{H}$-free and $G^*$ is perfect, then $\XCD(G)=\SC(G)$.
 \end{theorem}
 We then have the following corollary.
 \begin{corollary}\label{corr:suffcdperfect}
     Let $G$ be an $\mathcal{H}$-free graph. If for any induced subgraph $H$ of $G$, we have $k(H^*)=\alpha(H^*)$ then $G$ is $cd$-perfect. Consequently, if $G$ is $\mathcal{H}$-free and $H^*$ is perfect for any induced subgraph $H$ of $G$, then $G$ is $cd$-perfect.
 \end{corollary}
 It is known in the literature that if $G$ is a co-bipartite graph, then $\XCD(G)=\SC(G)$~\cite{DBLP:conf/caldam/ShaluK21}. As a consequence of Theorem~\ref{thm:suff_cdsubclique}, in the following corollary, we now have a simple and shorter proof for the same. 
 \begin{corollary}\label{corr:cobipartite}
    Let $G$ be a co-bipartite graph. Then $G$ is $cd$-perfect. 
\end{corollary}
\begin{proof}
Let $G=(A,B,E)$ be a co-bipartite graph. Since the class of co-bipartite graphs is closed under taking induced subgraphs, to prove the theorem, it is enough to show that  $\XCD(G)=\SC(G)$. Note that $\alpha(G)\leq 2$ (as any independent set in $G$ can have only at most one vertex from each of the sets $A$ and $B$). But for each graph $H\in \mathcal{H}$, we have an independent set of size 3 (by Observation~\ref{obs:structureH}). This implies that $G$ is $\mathcal{H}$-free. Since $G^*$ is a bipartite graph (as $G^*$ is a subgraph of complement of $G$), and therefore perfect, we then have $\XCD(G)=k(G^*)=\alpha(G^*)=\SC(G)$, by Theorem~\ref{thm:suff_cdsubclique}. 
\end{proof}

 \noindent\textbf{Note:}
Consider a graph $H\in \mathcal{H}$. It is not difficult to see that $\XCD(H)=3=\SC(H)$, if $H\neq C_6$, and $\XCD(H)=4>2=\SC(H)$, if $H=C_6$. This is why the sufficient condition for $cd$-perfectness given in Corollary~\ref{corr:suffcdperfect} is not a necessary condition.  To obtain a necessary and sufficient condition, we consider the set $\mathcal{H'}=\{C_6^1,C_6^2,C_6^3\}$. Then, in the following theorem, we have a characterization for  $\mathcal{H'}$-free graphs (a superclass of triangle-free graphs, $3K_1$-free graphs, etc.) to be $cd$-perfect.
 \begin{theorem}
     Let $G$ be an $\mathcal{H'}$-free graph. Then $G$ is $cd$-perfect if and only if $G$ is $C_6$-free and $k(H^*)=\alpha(H^*)$, for each induced subgraph $H$ of $G$. 
 \end{theorem}
 \begin{proof}
Suppose that $G$ is $C_6$ free and $k(H^*)=\alpha(H^*)$, for each induced subgraph $H$ of $G$. Let $H$ be any induced subgraph of $G$. Since $G$ is $\mathcal{H'}$-free and $C_6$-free, we then have $H$ is $\mathcal{H}$-free. Now, as $k(H^*)=\alpha(H^*)$, by Corollary~\ref{corr:suffcdperfect}, we can conclude that $G$ is $cd$-perfect.

\smallskip
On the other hand, assume that $G$ is $cd$-perfect. Clearly, $G$ is $C_6$-free, as otherwise, $C_6$ would be an induced subgraph of $G$ with $\XCD(C_6)=4>2=\SC(C_6)$. Thus we have that $G$ is $\mathcal{H}$-free, and therefore any induced subgraph $H$ of $G$ is also $\mathcal{H}$-free. Then, by Theorem~\ref{thm:suff_cdclique}, we have that $\XCD(H)=k(H^*)$. By Proposition~\ref{pro:omega_s}, we  have $\SC(H)=\alpha(H^*)$. Since $\XCD(H)=\SC(H)$, we then have $k(H^*)=\alpha(H^*)$.
 \end{proof}

Let $C_n$ denote an induced cycle of length $n$. The following observation is noted in~\cite{DBLP:conf/caldam/ShaluVS17}.
\begin{observation}[\cite{DBLP:conf/caldam/ShaluVS17}]\label{obs:holes}
 For $n\geq 4$, we have  $\XCD (C_n)=\SC(C_n)$ if and only if $n=4k$ for some integer $k\geq 1$.
\end{observation}
We also note the observation below.
\begin{observation}\label{obs:antiholes}
For $n\geq 5$, we have  $\XCD (\bar{C_n})=\SC(\bar{C_n})$ if and only if $n=2k$ for some integer $k\geq 2$.
\end{observation}
\begin{proof}
First, note that for each $n\geq 5$,  $\bar{C_n}$ is $\mathcal{H}$-free (since $\alpha(\bar{C_n})\leq 2$, for each $n\geq 5$, but $\alpha(H)=3$ for each $H\in \mathcal{H}$). Also, for each $n\geq 5$, it is easy to see that $\bar{C_n}^*=C_n$ (since $\bar{C_n}^2$ is a clique on $n$ vertices). Therefore, by Theorem~\ref{thm:suff_cdclique} and Proposition~\ref{pro:omega_s}, we have, $\XCD(\bar{C_n})=k(\bar{C_n}^*)=k(C_n)=\lceil \frac{n}{2}\rceil$ and $\SC(\bar{C_n})=\alpha(\bar{C_n}^*)=\alpha(C_n)=\lfloor \frac{n}{2}\rfloor$. Therefore, we can conclude that $\XCD (\bar{C_n})=\SC(\bar{C_n})$ if and only if $n=2k$ for some integer $k\geq 2$.
\end{proof}
In the following theorem, we then have some necessary conditions for a graph $G$ to be $cd$-perfect. Observe that these conditions are \textit{almost} consistent with the necessary conditions for a graph to be perfect. The proof of the theorem is immediate from Observations~\ref{obs:holes} and~\ref{obs:antiholes}.

\begin{theorem}\label{thm:necessary}
If a graph $G$ is $cd$-perfect, then $G$ is $C_n$-free for each $n\geq 4$ with $n\neq 4k$, and $\bar{C_n}$-free for each $n\geq 5$ with $n\neq 2k$, where $k$ is a positive integer.
\end{theorem}


%\documentclass{article}
\documentclass[a4paper,fleqn]{cas-dc2}
\usepackage{cas-dfrws} %overwrite some of the original (cas-dc.cls)
\renewcommand{\dfrwsconference}{Digital Forensics Research Conference USA
(DFRWS USA), 2023}

%font that is somewhat close to Gullivier (font elsevier uses)
\usepackage{stix}

\usepackage{graphicx}
\usepackage{array, multirow}
\usepackage{enumitem}
%\usepackage{babel}
\usepackage[utf8]{inputenc}
%\usepackage{hyphenat} useful should there be problems with overflowing hboxes
%because of words with hythens (e.g. quasi\hyph instantaneous)
\usepackage[dvipsnames]{xcolor}
\usepackage[colorlinks]{hyperref}
\hypersetup{
    linkcolor=CornflowerBlue,
    citecolor=CornflowerBlue,
    urlcolor=cyan
}
\usepackage{algorithm}
\usepackage{algpseudocode}
\usepackage{amsmath}
\usepackage{amsthm}
\usepackage{amsfonts}
\usepackage{mathtools}

\newtheorem{definition}{Definition}
\newtheorem{proposition}[definition]{Proposition}
\newtheorem{theorem}[definition]{Theorem}
\newcommand*{\todo}[1]{\textbf{({\color{red}TODO:} #1)}}

\usepackage[authoryear,longnamesfirst]{natbib}
\bibliographystyle{cas-model2-names}


 \def\hyph{-\penalty0\hskip0pt\relax}
\hyphenation{in-stan-ta-neous}
%\hyphenation{qua-si-in-stan-ta-neous}
\hyphenation{mem-o-ry}
\hyphenation{con-sis-ten-cy}
\hyphenation{in-cre-ment-ed}
\hyphenation{ad-di-tion-al-ly}
\hyphenation{in-con-sis-ten-cies}
\hyphenation{con-sid-er-ing}


\begin{document}

\shortauthors{Ottmann et~al.}

\author[1]{Jenny Ottmann}[orcid=0000-0003-1090-0566]
\cormark[1]
\ead{jenny.ottmann@fau.de}
\credit{Conceptualization, Methodology, Investigation, Software, Writing - Original Draft, Writing - Review and Editing}

\author[1]{{\"U}same Cengiz}[orcid=0009-0004-4092-7668]
\ead{uesame.cengiz@fau.de}
\credit{Methodology, Investigation, Writing - Review and Editing}

\author[2]{Frank Breitinger}[orcid=0000-0001-5261-4600]
\ead{frank.breitinger@unil.ch}
\ead[url]{https://FBreitinger.de}
\credit{Conceptualization, Writing - Original Draft, Writing - Review and Editing, Supervision}

\author[1]{Felix Freiling}[orcid=0000-0002-8279-8401]
\cormark[1]
\ead{felix.freiling@fau.de}
\credit{Conceptualization, Writing - Original Draft, Writing - Review and Editing, Supervision}


\address[1]{Department of Computer Science,
  Friedrich-Alexander-Universit\"at Erlangen-N\"urnberg (FAU),
  Erlangen, Germany}

\address[2]{School of Criminal Justice,
  University of Lausanne, 1015 Lausanne, Switzerland}

\cortext[1]{Corresponding authors.}


\title[mode=title]{As if Time Had Stopped -- Checking Memory Dumps for Quasi-Instantaneous Consistency}
\shorttitle{As if Time Had Stopped}
\begin{abstract}
  % 
	Memory dumps that are acquired while the system is running often contain
	inconsistencies like page smearing which hamper the analysis.  One
	possibility to avoid inconsistencies is to pause the system during the
	acquisition and take an instantaneous memory dump. While this is
	possible for virtual machines, most systems cannot be frozen and thus
	the ideal dump can only be quasi\hyph instantaneous, i.e., consistent despite
	the system running.
	In this article, we introduce a method allowing us
	to measure quasi\hyph instantaneous consistency and show both, theoretically,
	and practically, that our method is valid but that in reality, dumps can be but usually are not quasi\hyph instantaneously consistent. For the assessment, we run a pivot
	program enabling the evaluation of quasi\hyph instantaneous consistency for
	its heap and allowing us to pinpoint where exactly inconsistencies
	occurred.
  % 
\end{abstract}

% Each keyword is seperated by \sep
\begin{keywords}
	Memory acquisition \sep Consistency \sep Quasi-instantaneous consistency\sep Instantaneous snapshot \sep Experiment \sep Live system memory capture
\end{keywords}


\maketitle

%\onecolumn
\section{Introduction}

%Concurrency and memory snapshotting, inconsistencies like page smear
%\citep{pathforw}, not very amusing, a real hassle, hampers understanding and analysis

The acquisition and analysis of main memory are common tasks for
forensic investigators, e.g., to find encryption keys for storage or
analyze malware that only runs in memory. A common acquisition
procedure is to perform \emph{live memory acquisition}, i.e., to
utilize the software on the system under investigation to access and
dump memory. However, as the system is running and memory contents are
continuously updated by concurrent processes, the quality of such
snapshots is (at best) unclear.  A symptom of bad memory snapshots,
that is commonly observed, is \emph{page smearing} which is defined as
``an inconsistency that occurs in memory captures when the acquired
page tables reference physical pages whose contents changed during the
acquisition process'' \citep{pathforw}.  It is well-known that
established tools like Volatility have difficulties parsing low-quality
memory snapshots, resulting in cases where snapshots
cannot be analyzed at all. But what, actually, is a ``good'' memory
snapshot?

%Commonly accepted: freezing system,
%then snapshot $\Rightarrow$ instantaneous snapshot, gold standard, no
%concurrency $\Rightarrow$ no problems. 

In practice, it is commonly accepted that freezing a system, i.e.,
stopping concurrent system activity before taking a memory snapshot,
produces the highest quality. Such snapshots are often referred to as
\emph{instantaneous} snapshots. Methods to create instantaneous
snapshots either have strong assumptions, e.g., assume that the
analyzed system runs as a virtual machine
\citep{martignoni2010,yu2012,kiperberg2019}, or are cumbersome to
execute, like cold boot attacks
\citep{DBLP:journals/cacm/HaldermanSHCPCFAF09,DBLP:journals/di/0004GF16}.
Therefore, in many practical situations memory acquisition is
necessarily performed live and the resulting snapshots are not
instantaneous. But in what sense can non-instantaneous snapshots be
compared regarding quality?

It has been observed \citep{introducing,defining} that certain
snapshots acquired live cannot be distinguished from instantaneous
snapshots. Such snapshots are called \emph{time-consistent}
\citep{introducing} or \emph{quasi-instantaneous}
\citep{defining}. By definition quasi-instantaneous snapshots avoid the many
hassles associated with live memory acquisition, but unless the memory
acquisition method itself provides consistency guarantees,
it was not known how memory snapshots can be tested for
quasi-instantaneous consistency. Clearly, such methods must rely on
some form of consistency indicators within the image. How these may
look like to precisely determine the consistency of a snapshot was so
far unclear. In this article, we describe a method to measure
quasi\hyph instantaneous consistency of memory snapshots based on
well-defined consistency indicators.  

% This method is first described theoretically and its applicability is
% then demonstrated through an experiment.


% From a practical point of view it is clear that a good memory
% snapshot allows the analyst to extract the needed information. But for testing
% or when we try to understand better under which circumstances inconsistencies
% occur and if there are ways to avoid them, more formal criteria are needed.
% %There have been attempts to define and formalize quality criteria, such as
% %``consistency''
% %that (1)   allows better arguments in court (forensiscally sound), and (2) should be
% %verifiable to allow testing in practice.
% %but so far no one has measured the quality of snapshots. 
% %\todo{Wuerde ich nicht ganz so stark ausdruecken, in der related work sind ja
% %Beispiele fuer Versuche.}
% There have been suggestions for formal quality criteria \citep{correctness} but
% some of them can be difficult to observe exactly as shown by the evaluations that
% applied them \citep{evalplat, evaluatingat}.
% %Sophisticated consistency criteria allow to identify where exactly mismatches
% %between contents occur \todo{reference pagani for the mismatch idea, also
% %reference page smearing as example from Jan's paper for contamination
% %that is hard to classify?} 
% %and also allow gathering data on where  exactly mismatches occur can help to identify improvement
% %possibilities (decide where a non-sequential dump starts, point to
% %alternatives to acquisition of complete memory)


\subsection{Related work}

%Quality criteria: first V+F, the Pagani: back to the whiteboard (this paper
%might fit better into introduction ``Our work can help mitigating this issue by
%assessing how existing techniques are affected by non-atomic acquisitions, and
%help design new heuristics which are more robust against the presence of
%inconsistent information.'') and time-based approach. Collect indications of when a memory block is saved, can be used to prioritize. 

After multiple works about the quality of
memory dumps \citep{Inoue:2011:VIT,Lempereur:2012:PAP,Campbell:2013:VMA},
three formal criteria for the assessment of a memory
dump's quality were defined by \citet{correctness}: \emph{correctness},
\emph{atomicity}, and \emph{integrity}.
Correctness is fulfilled if the contents of the memory dump are an exact copy of
the memory contents at the time of their acquisition. Atomicity addresses the
causal consistency of the memory dump. It depends on the cause-effect
relationships between memory accesses by different processes. The last criterion,
integrity, is assessed in relation to a point in time shortly before the memory
acquisition is started. Memory contents that change after this point in time
and before they were copied by the memory acquisition program lower the degree
of integrity of the memory dump. Two applications of the criteria for practical
evaluations of memory acquisition methods followed: one with a white-box testing
method \citep{evalplat}, and one with a black-box testing method
\citep{evaluatingat}.

In contrast to abstract measures such as atomicity, \citet{introducing} took a content-based approach to
assess the consistency of a memory dump. A memory dump is \emph{time-consistent}
if there ``exists a hypothetical atomic acquisition process that could have
returned the same result''. One method they applied in their evaluation to
assess the consistency of a memory dump is the number of virtual memory areas
(VMAs) that are attributed to a task by different sources. If the numbers differ
an inconsistency in a memory dump has been spotted.

Based on the idea of time consistency, \citet{defining} introduced two
formal criteria, \emph{instantaneous consistency}, and
\emph{quasi\hyph instantaneous} consistency. While the former
criterion portrays the ideal case for memory acquisition, pausing the
system's execution and copying all memory contents at the same time,
the latter can be fulfilled even if the system cannot be paused.  It
requires that the contents of the memory dump could have also been
acquired with a hypothetical instantaneous snapshot. Or in other
words, there was a time at which the dump's contents were
coexistent in memory. Therefore, a memory dump that fulfills the
latter criterion is as consistent as an instantaneous snapshot.  So
while quasi\hyph instantaneous consistency is as good as instantaneous
consistency, \citet{defining} fail to give a method to check or
observe it. Such a method would allow testing snapshots of benchmark
acquisition methods to gain trust in data and methods.


%Testing digital forensic tools (white box vs black box, mal wieder G+F)?
\subsection{Contributions}

In this paper, we devise a method with which (under certain
assumptions) it is possible to find out whether a portion of a snapshot is
quasi\hyph instantaneously consistent. Assumptions are the existence of
consistency indicators in memory. These represent information on the
last event that happened in a particular memory region and that potentially 
changed the content of that region. This extends the content-based approach of \citet{introducing}. 
Given such indicators, we show how it is
possible to test whether a snapshot is quasi\hyph instantaneously consistent.
Furthermore, if a memory dump is not quasi\hyph instantaneously consistent, we can
use the output of the method to assess the degree of inconsistency.

We present a formalization of the method and prove its correctness. We
also show how the necessary data structure for storing consistency
indicators can be implemented with increasingly efficient storage
requirements.
%Give an algorithm, prove the correctness of the algorithm under increasingly efficient storage requirements.
In a practical evaluation, we apply the method to frozen and live snapshots. As
expected, snapshots of frozen systems are always quasi\hyph instantaneously
consistent, those taken of live systems not necessarily.
%Practical evaluation: apply method to frozen and live snapshots. Frozen snapshots are always q-i consistent, live snapshots not always
In summary, the contributions of this paper are threefold: We provide
%
\begin{itemize}
\item a method to observe quasi\hyph instantaneous consistency,
\item a proof that it works theoretically, and
\item a proof-of-concept implementation that allows measuring
  consistency indicators in practice.
\end{itemize}

While we focus on main memory, our approach can naturally be applied
to situations in which other forms of storage (like persistent disk
storage) are acquired in a live fashion.


\subsection{Outline}

We first revisit the system model and previous consistency definitions
in Section~\ref{sec:definitions}. Our new method to observe and check
quasi\hyph instantaneous snapshots is presented in
Sections~\ref{sec:observing} and \ref{sec:checking}. Ways to improve
the memory efficiency of our method are discussed in
Section~\ref{sec:efficiency}. We provide the results of our practical
evaluation in Section~\ref{sec:evaluation} and discuss the results in
Section~\ref{sec:discussion}. We conclude in
Section~\ref{sec:conclusion}.


\section{Consistency of Snapshots}
\label{sec:definitions}

The consistency of a snapshot can be assessed from different
perspectives.  One is the causal perspective which takes into account
the active processes in the system and their causal relationships
\citep{correctness}. The basic idea of \emph{causal} consistency is
that the snapshot contains the cause for every effect. If the actions
of malware can be observed in the snapshot, all causally preceding
events must also be contained in the snapshot (e.g. the malware
infection). This definition is very generic and does not reference any
notion of real-time. As long as cause-effect relations are respected,
the system does not need to be frozen to acquire a snapshot that is
causally consistent.

The perspective we take in this article is more restrictive. We accept snapshots
 as consistent only if their contents were coexistent in memory at a previous
point in time. This consistency criterion is called \emph{quasi\hyph instantaneous
consistency} \citep{defining}.
%While in an ideal scenario every snapshot is taken instantaneously, i.e. the
%system can be frozen, often times, this is not possible and the system continues
%its execution parallely to the memory acquisition. This can lead to a snapshot
%which contains data in a state in which it was never present in the actual
%memory. %Such a snapshot would be classified as inconsistent when applying the
%consistency criterion \emph{quasi-instantaneous consistency},
%explained subsequently.
To approach the formal definition of quasi\hyph instantaneous
consistency, we need to introduce some basic aspects of the system
model we assume.


%Apart from the ideal case, \emph{instantaneous consistency}, we differentiate
%between \emph{causal consistency} and \emph{quasi\hyph instantaneous consistency}

\subsection{Model}\label{sec:model}
Based on \citet{correctness}, we define memory, events (modifying operations
on memory), and snapshots.

\paragraph{Memory}
We observe accesses to the set $R=\{r_1,\ldots,r_n\}$ of $n$ memory
regions. Intuitively, a memory region can be regarded as that part of
memory that can be acquired in one atomic action. Depending on the
real system, memory regions can consist of a single byte or a full
memory page.  Memory regions have values $v$ at specific points in
time. The sets of all possible values and points in time are denoted
$V$ and $T$, respectively. Memory can therefore be expressed by the
function $m: R \times T \rightarrow V$.


\paragraph{Events}
When a process performs an operation on a memory region this results
in an event $e$. We denote by $E$ the set of all such events. For any
event $e \in E$, $e.r$ denotes the memory region on which $e$
happened. An execution of the system is defined by a sequence of
events $\eta := [e_1,\ldots]$. As time between two events is of no
concern to our model, we define $T$ to be the set of natural numbers
$\mathbb{N}$. We assume that events generally \emph{change} memory
contents. So if no event happens on a region $r$ between times $t$ and
$t+n$, then the corresponding values in the memory are identical.
Formally:
$\forall r \in R, \forall t,n \in \mathbb{N}: m(r,t) = m(r, t+n)
\Leftrightarrow \forall k, t < k \leq t+n: e_k.r \neq r$.


\paragraph{Snapshot}
%We formalize a snapshot as a function $s: R \rightarrow V \times T$,
%i.e., for every memory region we store the value and the time this
%value was copied from memory. We denote by $s(r).v$ the value stored
%for region $r$ in snapshot $s$ and by $s(r).t$ the corresponding
%time.

We formalize a snapshot as a function $s: R \rightarrow V \times T$,
i.e., for every memory region we store the value and the time at which it was
copied. We denote by $s(r).v$ the value stored
for region $r$ in snapshot $s$ and by $s(r).t$ the corresponding
time. Note, as established above, the time $t$ advances whenever an event is
executed. The vector containing all values in all regions of the snapshot is denoted $V_s :=
[s(r_1).v, \ldots, s(r_n).v]$, the vector containing all times $T_s
:= [s(r_1).t, \ldots, s(r_n).t]$.

The model can be visualized using space/time diagrams \citep{virtual}. An
example for a system with three memory regions, $r_1$, $r_2$, and $r_3$ is shown
in Fig.~\ref{fig:model}. The arrows represent the memory regions over time,
events, $e_1$, $e_2$, $e_3$, and $e_4$ in the example, are denoted by black
dots.  The time at which a memory region is copied in a snapshot is denoted with
a rectangle. The rectangles belonging to one snapshot are connected to each
other. In the example two snapshots, $s_1$ and $s_2$, can be seen.

% Figure environment removed



\subsection{Quasi-instantaneous consistency}

\citet{defining} defined the following notions of consistency.
The ideal case for a snapshot is that it is taken \emph{instantaneously}. In a
snapshot that satisfies instantaneous consistency every memory region was copied
at the same time. 

\begin{definition}[instantaneous consistency]
  % 
  A snapshot $s$ satisfies \emph{instantaneous consistency} iff all
  memory regions in $s$ were acquired at the same point in
  time. Formally:
  % 
  $$ \forall r, r'\in R: s(r).t = s(r').t$$
  % 
  If s satisfies instantaneous consistency we call s instantaneous.
\end{definition}

When a system cannot be frozen it might still be possible to acquire a snapshot
with the same contents as if it had been taken instantaneously. In this case the 
content is identical to an instantaneous snapshot (although not taken instantaneously) 
and we call such a snapshot quasi\hyph instantane\-ously consistent.


\begin{definition}[quasi-instantaneous consistency]
	\label{def:quasi}
  %
  A snapshot $s$ satisfies \emph{quasi\hyph instantaneous consistency} iff
  the values in the snapshot could have also been acquired with an
  instantaneous snapshot $s'$. Formally:
  %
  % \begin{multline*}
    %\exists s': s' $ is instantaneous $ \mathrel{\land} 
	  %(\forall r \in R: s'(r).v = s(r).v)
  $$
	  \exists s': s' \textrm{is instantaneous} \mathrel{\land} (V_{s'} = V_s)
  $$
  % \end{multline*}
  %
  If s satisfies quasi\hyph instantaneous consistency we call s quasi\hyph instantaneous.
\end{definition}

Two example snapshots are shown in Fig.~\ref{fig:model}. Snapshot
$s_1$ is quasi\hyph instantaneous since an instantaneous snapshot can
be found that would have had the same contents. Such an instantaneous
snapshot could have been taken right after event $e_2$ took place
and is indicated by a dashed vertical line. For the second snapshot,
$s_2$, on the other hand, it is not possible to construct an
instantaneous snapshot with the same contents.  The reason is that
event $e_3$ happened before $e_4$ but in the snapshot the changes made
by $e_3$ cannot be seen while those made by $e_4$ are included.  Thus,
snapshot $s_2$ is \emph{not} quasi\hyph instantaneous.



\section{Observing Quasi-Instantaneous Consistency}
\label{sec:observing}

Our approach to observe quasi\hyph instantaneous consistency is based
on the observation of \emph{consistency indicators} within the
snapshot. Oftentimes, such indicators already exist as part of the
running system. For example, kernel data structures that save
redundant information can serve as indicators \citep{introducing}.
However, artificial consistency indicators can also be deployed as
part of general forensic readiness procedures or within the memory
manage\-ment of individual processes.



\subsection{Current time and time of last event}\label{sec:idea}


% Figure environment removed

If we want to know exactly in which memory regions inconsistent
contents are located, knowledge about previous states of the memory
contents is necessary. An example is shown in Fig.~\ref{fig:idea} where 
events happen in real-time and the timestamps of events are recorded in a 
table shown below the space/time diagram. Note, such a data structure of all event
timestamps enumerates all possible instantaneous snapshots since
memory contents only change through events. For example, the
instantaneous snapshot taken right after the event at 13:08, $s_3$ in the
figure, would contain data as changed by the events at
13:05 (on region $r_1$), 13:08 (on region $r_2$) and 13:06 (on region $r_3$).

To identify if a snapshot is quasi\hyph instantaneous, we need to find
the ``matching'' instantaneous snapshot in the list of all
instantaneous snapshots described above. To do this, we can either
``scan'' the data structure from beginning to end, or search in the
vicinity of the timestamps that are stored in the snapshot. For a more
specific search, the ability to determine the time of the last event
on each memory region relative to the time at which the snapshot was
taken on that region is helpful. For example, for snapshot $s_1$, the
time of the most recent event on $r_1$ is 13:05. If the vector of
these time stamps matches one of the possible instantaneous snapshots
listed in the data structure, the snapshot is quasi\hyph
instantaneously consistent.

To illustrate the idea, Fig.~\ref{fig:idea} depicts two snapshots
$s_1$ and $s_2$; $s_1$ is quasi\hyph instantaneous since the vector of
the last events matches the known state added at time 13:05. For $s_2$
the searched vector is \{13:09, 13:01, 13:06\}.  Since this state is
not contained in the known state array, the snapshot is not quasi\hyph
instantaneous.

%We call the state in
%the data structure after inserting a new time stamp $x$ the \emph{current time}
%of x.


%it would have been possible to acquire the same contents
%with a snapshot at the highest time stamp in the current time vector.

%Roughly speaking: if snapshot time corresponds to some current time, then the snapshot is instantaneous.

\subsection{Two-dimensional global counter array}\label{sec:twodim}

We formalize this idea based on state information stored in unique
counters saved at each memory region access in a global structure, the
\emph{global counter array}, and the region itself. The global counter array can
be implemented in different variations. We introduce the general idea first,
followed by a variant of the global counter array that carries redundant
information helpful for visualization.

\paragraph{Global counter array}
The global counter array $G$ is a two-dimensional array $R \times T$. As
defined in section~\ref{sec:model} $T = \mathbb{N}$. It
contains a row for each $r \in R$. Its rows and columns are initialized with
zero. Since $T$ is infinite, theoretically, $G$ is also an infinite data
structure. However, at any finite point in time $G$ is also finite.

The current column to write to in $G$ is identified using the current logical
time $t \in \mathbb{N}$. It is initialized with zero. Algorithm~\ref{alg:update}
shows the sequence of actions triggered by an event $e$ on $r_i$.
When a memory region $r_i$ is accessed the time $t$ is incremented by one and a
value $x$ written to $G$ at the index $t$: $G[r_i][t] := x$. The value $x$ is
dependent on the implementation variant chosen for the global counter array as
we will see later. The time $t$ is saved in the memory region $r_i$ on which the
event occurred.

\algblockdefx[Event]{Event}{EndEvent}
{\textbf{Upon} }
{\textbf{Update finished}}
\algtext*{EndEvent}
%\algblock[Event]{Upon}{Updated}
\begin{algorithm}
\caption{Sequence of actions triggered by an event $e$ on memory region $r_i$}
	\label{alg:update}
\begin{algorithmic}
	\Event Event $e$ on $r_i$
	\State $t := t+1$
	\State $G[r_i][t] := x$
	\State Save $t$ in $r_i$
	\EndEvent
	\end{algorithmic}
\end{algorithm}

\paragraph{Current time}
We denote the vector $G_t$ which contains the index of the last visible status
update for each $r$ in the global counter array $G$ at a logical point in time
$t$ the \emph{current time} of $t$. The value in the vector at index $i$ is
returned by $G_t[i]$.


%\paragraph{Global time of snapshot}
%The memory regions of a snapshot contain the last logical time at which an event
%was executed on them before they were copied. Since the global time shows
%us the state of the memory regions in the snapshot, we can compare it to the
%states saved in the global counter array to find out if the memory contents were
%coexistent at some point in time in the memory as well.



% Figure environment removed


\subsection{Carry along global counter array}

One possibility for keeping track of coexistent states is to save the value of
$t$ for each event on a region $r$ in $G$ and carrying along the last visible counter
updates for all other regions. Obviously, this representation is also a direct representation of all possible instantaneous snapshots with logical time.

When an event $e$ occurs on memory region $r_i$, the sequence of actions shown
in Algorithm~\ref{alg:update} is followed: First, the time $t$ is incremented by
one. The value $x$ is written to $G$ at the index $t$, i.e., $t$: $G[r_i][t] := t$.
Then the time $t$ is saved in $r_i$ as well. Additionally, for all other $r$, the
value at index $t-1$ is written to $G$, thereby carrying along the values of
previous updates: $G[r_i][t] := G[r_i][t-1], \forall r \in R: r \neq e.r$.
An example of how $G$ is updated for each event is shown in
Fig.~\ref{fig:carry}.



\paragraph{Current time}
The current time for a logical point in time $t$ is reconstructed from
the values for each row in $G$ at index $t$:
$G_t := {G[r_1][t],\ldots, G[r_n][t] }$.  For example, the current
time in Fig.~\ref{fig:carry} at time $t=5$ is $G_t=(1,5,4)$. This time
can be used to check for quasi-instantaneous consistency, as we now
explain.



\section{Checking Quasi-Instantaneous Consistency}
\label{sec:checking}

The question is how to verify if a snapshot is quasi\hyph instantaneously 
consistent. For this purpose, it needs to be determined if a
point in time exists at which the same contents were coexistent in memory as in
the snapshot. Comparing the last point in time saved in each memory region to
the states saved in the global counter array allows us to do this.

Given that the value of each memory region $r$ in the snapshot $s$ is defined by the last
event that occurred on the region, each $s$ is equivalent in its values
to the normalized snapshot $N(s)$, which is taken right after the occurrence of the last
event for each memory region $r$ from the point of view of $s$.
Fig.~\ref{fig:norm} shows an example of a snapshot and its normalized form.

% Figure environment removed

\begin{definition}[Normalized snapshot $N(s)$]
For $r \in R$, we define $t'_r$ as the point in time at which the last event
relative to a snapshot $s$ was executed on $r$: $t'_r := \max(\{0\} \cup \{ i \leq s(r).t
\;|\; e_i.r = r\})$.
For each memory region $r$ the snapshot $N(s)$ contains the appropriate point in
time $t'_r$ and the value saved in the memory region at point in time $t'_r$:
$N(s)(r) := (t'_r, m(r, t'_r))$.
\end{definition}


\begin{proposition}\label{prop:normal}
%
	The values of $N(s)$ and $s$ are equivalent: $V_{N(s)} = V_s$
%
\end{proposition}
\begin{proof}

	We want to show that $\forall r \in R: N(s)(r).v = s(r).v$.
	%If $t := s(r).t$, then: $m(r, t'_r) = m(r, t)$.

	Fix an $r \in R$ and let $t := s(r).t$. The case $t'_r = t$ is trivial,
	since then $m(r, t'_r) = m(r, t)$ by
	assumption. As such, $V_{N(s)} = V_s$.

	Consider $t'_r < t$. If we assume $m(r, t'_r) \neq m(r, t)$,
	an event $e_z$ occurs after $t'_r$ at time $z$: $e_z.r =
	r$ and $t'_r < z \leq t$. It follows that $z \in \{ i \leq s(r).t \;|\; e_i.r =
	r\}$, but then $t'_r$ is not the time at which the last event
	happened on region $r$, since $z > t'_r$. This contradicts the
	definition of $N(s)$. Hence, $m(r, t'_r) = m(r, t)$ and $V_{N(s)} = V_s$
	accordingly.

\end{proof}


Thus, when looking at a snapshot $s$, it is equivalent to look at
$N(s)$ instead. When comparing two snapshots, $s_1$ and $s_2$, they can be
substituted with $N(s_1)$ and $N(s_2)$, respectively. This makes comparisons
easier, since the values stored in the normalized snapshots are equal iff the
times are equal.
%This comparison is easier to make than the one between $s_1$ and $s_2$ because
%of the following:


\begin{proposition}\label{prop:nvaluetime}
%
For two snapshots $s_1$ and $s_2$: $T_{N(s_1)} = T_{N(s_2)} \Leftrightarrow
	V_{N(s_1)} = V_{N(s_2)}$
%
\end{proposition}

\begin{proof}

\noindent ($\Rightarrow$) Let $T_{N(s_1)}$ be equal to $T_{N(s_2)}$:
		If for both $N(s_1)$ and $N(s_2)$ the time at which the last event
		which occurred on a region $r \in R$ is equal, $t :=
		N(s_1)(r).t = N(s_2)(r).t$, then we know that $m(r,
		N(s_1)(r).t) = m(r, t) = m(r, N(s_2)(r).t)$. As such,
		$V_{N(s_1)}
			= V_{N(s_2)}$.

	\noindent ($\Leftarrow$) Let $V_{N(s_1)}$ be equal to $V_{N(s_2)}$: According to Proposition
		\ref{prop:normal}, given an $r \in R$, we have
		$N(s_1)(r).v = s_1(r).v = s_2(r).v = N(s_2)(r).v$.	
		We need to show $N(s_1)(r).t = N(s_2)(r).t$.
		
		If the values of $s_1$ and $s_2$ are equal, $s_1(r).v =
		s_2(r).v$, this means $m(r,s_1(r).t) =
		m(r,s_2(r).t)$.
		Let $t_1 := s_1(r).t$, $t_2 := s_2(r).t$.
		W.l.o.g. $t_1 < t_2$.
		Then an $n \in \mathbb{N}$ exists for which $t_2 = t_1 + n$.
		Since $m(r,t_1) = m(r,t_2)$, there has been no $e_k$ where $t_1
		< k \leq t_1 + n$ with $e_k.r = r$.

		Then the sets $\{i \leq t_1\ |\ e_i.r = r\}$ and $\{i \leq t_1 +
		n\ |\ e_i.r = r\}$ are equal, the latter of course being $\{i
		\leq t_2\ |\ e_i.r = r\}$. Now $N(s_1)(r).t = max \{\{0\} \cup
		\{i \leq t_1\ |\ e_i.r = r\}\} = max \{\{0\} \cup \{i \leq t_2\
		|\ e_i.r = r\}\} = N(s_2)(r).t$, which completes the proof.

\end{proof}


Now that we know that for two normalized snapshots, their values are only equal iff
their times are equal, the question remains how we can use this to check
snapshot $s$ for quasi\hyph instantaneous consistency. The missing piece to perform the
check is the associated instantaneous snapshot of $s$, denoted $\hat{s}$.
Fig.~\ref{fig:asso} shows a snapshot $s$ and its associated instantaneous snapshot $\hat{s}$
taken at the highest time found in the normalized snapshot $N(s)$.

\begin{definition}[Associated instantaneous snapshot of $s$]
For a snapshot $s$, we denote by $\hat{s}$ the instantaneous snapshot at
$\hat{t}_s$ which we call the \emph{associated instantaneous snapshot}
of $s$, where $\hat{t}_s := \mathsf{max}(\{N(s)(r).t \;| \; r \in R\})$.
Note that $T_{N(\hat{s})} = G_{\hat{t}_s}$.
\end{definition}

% Figure environment removed

If a snapshot is equal in its values to its associated instantaneous snapshot it is
quasi\hyph instantaneously consistent. As we have established that two normalized
snapshots will be equal in their values iff their times are equal, we can perform
the comparison based solely on the times of the normalized snapshot of $s$,
$N(s)$, and the normalized snapshot of its associated instantaneous snapshot, $N(\hat{s})$.

\begin{theorem}
	A snapshot $s$ is quasi\hyph instantaneously consistent iff $T_{N(s)} =
	T_{N(\hat{s})}$.
\end{theorem}

\begin{proof}
%	\begin{enumerate}
\noindent ($\Rightarrow$) Given an instantaneous snapshot $s'$ for
			which $V_{s'} = V_s$ , we show
			that $T_{N(s)} = T_{N(\hat{s})}$.

			Since $V_{s'}= V_s$, according to Proposition
			\ref{prop:normal} we can use the normalized snapshots
			instead: $V_{N(s')} = V_{N(s)}$. It follows that
			$T_{N(s')} = T_{N(s)}$ according to Proposition
			\ref{prop:nvaluetime}. Therefore $\hat{t}_{s'} =
			\hat{t}_s$. Since there is only one instantaneous
			snapshot at any given time $t$, $\hat{s'} = s' =
			\hat{s}$. Thus, when we substitute $s'$ by $\hat{s}$,
			$T_{N(s)} = T_{N(\hat{s})}$. 

\noindent ($\Leftarrow$) Given $T_{N(s)} = T_{N(\hat{s})}$ we show
			that a snapshot $s'$ exists which is instantaneous and
			for which $V_{s'} = V_s$.

			Let $s' := \hat{s}$. By definition $\hat{s}$ is
			instantaneous. According to Proposition \ref{prop:nvaluetime},
			$T_{N(s)} = T_{N(\hat{s})} \Rightarrow V_{N(s)} =
			V_{N(\hat{s})}$. Therefore, according to Proposition \ref{prop:normal},
			$V_s = V_{\hat{s}}$.

\end{proof}

With this theorem, we have shown that we can use the normalized snapshot instead
of the original snapshot to evaluate if the snapshot is quasi\hyph instantaneously consistent
or not. It also becomes apparent that comparing the time is sufficient to
establish if the values of the snapshot were coexistent in memory at some point
in time. Since $T_{N(\hat{s})} = G_{\hat{t}_s}$ it also shows how the states saved
in the global counter array are used to determine existent states. Algorithm
\ref{alg:quasi} summarizes how the check is performed.


\begin{algorithm}
\caption{Checking for quasi\hyph instantaneous consistency}\label{alg:quasi}
\begin{algorithmic}
	\State Compute $N(s)$ \Comment Extract time $t$ saved in each region
	\State $\hat{t}_s := \mathsf{max}(\{N(s)(r).t \;| \; r \in R\})$
	\State Compute $G_{\hat{t}_s}$	\Comment See Algorithm \ref{alg:spec}
	\State $T_{N_{\hat{s}}} := G_{\hat{t}}$
	\If{$T_{N_{\hat{s}}} = T_{N(s)}$}
		\State $s$ is quasi\hyph instantaneously consistent
	\Else
		\State $s$ is not quasi\hyph instantaneously consistent
	\EndIf
	\end{algorithmic}
\end{algorithm}


\section{Improving Memory Efficiency}
\label{sec:efficiency}

%Aim is to reduce memory overhead. First step, simplify two dimensional
%representation. Second step, go from two dimensions to one. Final
%representation is a very efficient encoding of the information we
%need.

The implementation of the global counter array as shown in
Section~\ref{sec:twodim} is inefficient both computationally and
regarding memory usage. In the following, we first show a more efficient two-dimensional
implementation. As it becomes apparent that one dimension is enough to carry the
necessary information, we then present a one-dimensional variant we used for
implementing the global counter array.

\subsection{Simplified global counter array}

Looking at Fig.~\ref{fig:carry} it becomes apparent that a lot of redundant
information is saved in the global counter array $G$ because the index at which $t$ is written and its
value are identical. Additionally, from a practical perspective, it is more
efficient to not carry along previous values of $t$. Instead, when an update of
$t$ occurs a $1$ is written at index $t$ for the appropriate $r$. For all other
$r$ the initial value, $0$, is not changed.

For the simplified version, an event on memory region $r_i$ triggers the sequence of actions shown
in Algorithm~\ref{alg:update}: First, the time $t$ is incremented by one.
Then, a value $x$ is written to $G$ at the index $t$, for the simplified global
counter array $x$ is always $1$: $G[r_i][t] := 1$. Lastly,
the value of $t$ is saved in $r_i$. Here, no additional steps are necessary.
Fig.~\ref{fig:simpler} shows the same sequence of events as in
Fig.~\ref{fig:carry} but with the adapted implementation of $G$. 

% Figure environment removed



\paragraph{Current time}
Because the last updates are not carried along, reconstructing $G_t$ requires to
find the last update for all memory regions $r$ except the one at which a $1$
can be found in $G$ for time $t$. This can be done as shown in Algorithm~\ref{alg:spec}.

\begin{algorithm}
	\caption{Computing the current time $G_t$ for the logical time
	$t$}\label{alg:spec}
\begin{algorithmic}
	\State Initialize vector $G_t$ with $-1$
	\State $t_i := t$
	\While{$\exists i: G_t[i] = -1 $}
		\State Find row $r_h$ where ($G[r_h][t_i] = 1) \wedge
		(G_t[r_h] = -1)$
		\State $G_t[r_h] := t_i$
		\State $t_i := t_i - 1$
		\If{$t_i = 0$}
			\ForAll{ $G_t[i]$ for which $G_t[i] = -1$}
				\State $G_t[i] := 0$
			\EndFor
		\EndIf
	\EndWhile
	\end{algorithmic}
\end{algorithm}

%\paragraph{Reconstruction of all current times} All $T_c(t_l), 0 \leq t_l \leq g $
%can be reconstructed as follows. Starting with the most right values: For each
%line $l$ the index in $G$ at which the last $1$ was noted is saved in the vector
%$H$ of size $n$ (using the algorithm \ref{alg:spec}). The highest value in $H$,
%$c_h$ is the logical time for which the global time is
%reconstructed. The index at which $c_h$ is found in $H$, $l_h$, identifies the
%corresponding line. Because $g$ is always incremented by one, the indices of the
%highest counter state for each $l$ are equal to the counter value. Therefore
%$T_c(c_h)$ is contained in $H$: $T_c(c_h)\gets {[H[l_1],\ldots, H[l_i] }, i\in
%n$. 
%
%Then the next smaller index for $G[l_h][c_h]$ is determined
%by searching for the next counter to the left of $c_h$ that is smaller than the
%current one but higher than zero. If no such smaller value can be found, $H[l_h]
%\gets 0$. When the next index has been found the current
%time for the logical time $T_c(c_h - 1)$ can be constructed with the help of $H$
%as described above. This is repaeated until all states have been found, which is
%the case when $\forall i \in 0,\ldots , n: H[i] = 0$. 
%
%
%\begin{algorithm}
%	\caption{Reconstructing $T_c$ for all observed $t_l$}\label{alg:rec}
%\begin{algorithmic}
%	\State $ H \gets \forall l$ highest index at which $1$ can be found
%	\While{$\exists i, i \in 0, \ldots, n-1: H[i] > 0$}
%		\State Determine $c_h$    \Comment Index of highest value in $H$
%		\State Determine $l_h$    \Comment Line in which $c_h$ is found
%		\State Compute $T_c(c_h)$ \Comment See algorithm \ref{alg:spec}
%
%		\State $T_c(c_h) \gets [H[l_1],\ldots, H[l_i] ], i\in n $ 
%		\State $H[l_h] \gets c$
%	\EndWhile
%\end{algorithmic}
%\end{algorithm}
%


\subsection{One-dimensional global counter array}

The implementation variant of $G$ shown in
Fig.~\ref{fig:simpler} uses more memory than necessary to carry
the needed information. Although at each logical point in time $t$ only one
memory region is updated, for all other regions memory is reserved with only
zeros entered. Since we only need to save information for exactly one region per
logical point in time, we can save the known states in a list instead of a
two-dimensional array. Since no second dimension exists to indicate the region
at which the event occurred, this information needs to be saved in the list.

The index to write to in the one-dimensional global counter array $G$ remains
the time $t$. When a memory region $r_i$ is accessed, $t$ is incremented and $i$
saved in $G$ at index $t$: $G[t] := i$. The value of $t$ is saved in $r_i$.
An example of the adapted global counter array with the same sequence of events as in the
previous two examples is shown in Fig.~\ref{fig:list}.

% Figure environment removed


\paragraph{Reconstructing a specific current time}
To reconstruct the coexistent values at a logical point in time $t$,
entries for each memory region in $G$ at or before $t$ need to be searched. If no
entry for a memory region can be found no events occurred yet which means the
saved value in the region equals zero. Algorithm~\ref{alg:list} shows the
detailed procedure.

\begin{algorithm}
	\caption{Computing the current time $G_t$ for the logical time
	$t$ based on the one-dimensional global counter array}\label{alg:list}
\begin{algorithmic}
	\State Initialize vector $G_t$ with $-1$
	\State $t_i := t$
	\While{$\exists i: G_t[i] = -1 $}
		\State $r := G[t_i]$
		\If{$G_t[r] = -1$}
			\State $G_t[r] := t_i$
		\EndIf
		\State $t_i := t_i - 1$
		\If{$t_i = 0$}
			\ForAll{ $G_t[i]$ for which $G_t[i] = -1$}
				\State $G_t[i] := 0$
			\EndFor
		\EndIf
	\EndWhile
	\end{algorithmic}
\end{algorithm}


\section{Evaluation}
\label{sec:evaluation}

%Ziel: zeigen, dass der Check funktioniert.
Now that we have shown that, given proper consistency indicators, theoretically quasi\hyph instantaneous
consistency can be observed, we present a practical
proof-of-concept application of the method. It allows observing the
quasi\hyph instantaneous consistency of memory regions in one process. We consider
two main system states for the evaluation, frozen and running. We expect that
memory dumps taken of frozen systems satisfy quasi\hyph instantaneous consistency,
while those taken concurrently to the running system are expected to not 
(necessarily) be consistent. The evaluation is performed with a semi-automated procedure,
described subsequently, which mainly differs in the method chosen to create the
memory dump depending on the system state.

\subsection{Procedure}

For the evaluation of memory dumps taken in both system states, we use a virtual
machine running Ubuntu 18.04 with 4\,GB of RAM and 4 CPUs. Quasi\hyph instantaneous
consistency is observed in a specifically crafted \emph{pivot program}.
It meets the requirements to apply the method practically: 
\begin{enumerate}
	\item The ability
to observe accesses to memory regions 
	\item The ability to write counter values
to memory regions
	\item Enough memory for the global counter array.
\end{enumerate}
%(1) the ability
%to observe accesses to memory regions, (2) the ability to write counter values
%to memory regions, (3) enough memory for the global counter array.
In the pivot program memory regions are represented by list elements and changes on them are
tracked in a one\hyph dimensional global counter array. The array is allocated
with a fixed size that is sufficient to capture the events taking place during
the intended runtime of the program. The changes on the list elements
are performed by one or more threads. The threads randomly choose a list element to
remove from the synchronized list and after a short wait reinsert the element at
the beginning of the list. Each update (insertion/removal) of a list element causes an
update of the time of the last event in the list element and the global
counter array. The number of list elements and threads is set at the program start.

Memory dumps of the live system are taken for two different levels of activity,
\emph{low} and \emph{high}. When creating memory dumps for the low activity
level only the pivot program is executed. In comparison, the high activity level 
executes several other programs in parallel. This level of
activity is also generated for the frozen system snapshots.
Memory dumps taken in the frozen and the live system state with high activity
can be summarized as follows (manually performed actions are labeled
with numbers, automated actions with letters):
 
\begin{enumerate}
	\item Start VM
		\begin{enumerate}
			\item Start pivot program
			\item Mount shared folder
			\item Start grep: \texttt{timeout 2m grep -r "libc" /  \&}
			\item Retrieve meta info of pivot program (pid, heap
				range)
			\item Move meta info to shared folder
		\end{enumerate}
	\item Open Firefox
	\item Open YouTube, click on video
	\item Open LibreOffice Writer, continuously write text
		\begin{enumerate}[resume]
			\item Take memory dump
			\item Dump pivot program's heap contents
		\end{enumerate}
\end{enumerate}

The memory dump is taken approximately one minute after the \texttt{grep}
command was executed. For all memory dumps taken without freezing the system,
as a last step the heap contents of the pivot program are dumped using
\texttt{gdb}.

Using this procedure, 30 memory dumps were created in total. Ten for the frozen
system state with high activity, ten for the live system state with high
activity, and ten for the live system state with low activity (details of how
these memory dumps were acquired are given below). All memory dumps and
analysis results, as well as the scripts used for their analyis and the source
code of the pivot program are publicly
available\footnote{\url{https://zenodo.org/record/8089517}}. The number of active
threads in the pivot program is set to eight for high activity. The memory dumps
of a live system with low activity are taken with the same timing as those with
high activity but without steps (c), (2), (3), and (4), and the number of active
threads in the pivot program is set to one instead of eight.


\subsection{Analysis}
In the analysis two types of inconsistencies are evaluated, quasi\hyph instantaneous
inconsistencies in the pivot program's heap, and inconsistencies between numbers
of virtual memory areas (VMAs) saved for each process by the kernel.

\paragraph{Quasi-instantaneous inconsistencies} When searching for quasi\hyph instantaneous
inconsistencies in the pivot program's process address space, first its heap is extracted from the
memory dump using the Volatility plugin \texttt{linux\_dump\_map}. Next, the
list elements and the time of the last event on them as well as the global
counter array are retrieved from the heap pages using a python script. In the
case of the memory dumps taken without freezing the system, the global counter
array is instead retrieved from the heap dump taken with \texttt{gdb}. This is
necessary as, while in the virtual process memory the global counter array is
located after the list elements and their counters, in the physical memory they
might be jumbled. If the global counter array is acquired before the list
elements, its contents could be not up to date with the last changes made on the
list elements.

To check for violations of quasi\hyph instantaneous consistency, we follow the
steps of Algorithm~\ref{alg:quasi}: The time of the last
event for each region is saved in a vector which is equivalent to the normalized
snapshot $N(s)$. Then, the maximal time stamp in this vector $\hat{t}$ is
identified, upon which the current time $G_{\hat{t}}$ is computed from the global counter array.
Lastly, the normalized snapshot and the current time, i.e., the snapshot's
associated instantaneous snapshot, are compared. Should they differ in one or
more values, a violation of quasi\hyph instantaneous consistency has been
identified.

\paragraph{VMA inconsistencies}
To gain insight into inconsistencies in kernel data structures, we use a method
suggested by \citet{introducing}. The number of VMAs assigned to each process
can be retrieved from different sources, a linked list of VMAs managed for each
process, a red-black tree of the VMAs, and the counter of assigned VMAs saved
for each process in its \texttt{task\_struct} structure. The Volatility plugin
\texttt{linux\_validate\_vmas}\footnote{Published by the authors at
\url{https://github.com/pagabuc/atomicity_tops}.} retrieves the number of VMAs
from the three sources and compares them. If a mismatch is detected, the
name of the corresponding process and the different values are returned. The
total number of processes with inconsistent VMA numbers is gathered for each
dump.



\subsection{Frozen system}

To take a memory dump of the frozen system, we use \texttt{virsh dump} with
option \texttt{-{}-memory-only}. As this command has to be performed by the
host, a script is started on the host once the shared folder has been
mounted that executes the command after one minute.
In the ten created memory dumps, as expected, no quasi\hyph instantaneous
or VMA inconsistencies were found. All ten snapshots were quasi\hyph instantaneously consistent.

\subsection{Running system}

\begin{table*}[pos=ht]
	\centering
	\begin{tabular}{lllcccc}
                \toprule
                %\cmidrule[0.5pt]{3-9}
		System state & Inconsistency type & Activity & Min & Max & Average & Affected
		dumps\\
                \midrule
		\multirow{2}{*}{Frozen} & Quasi\hyph instantaneous &
		\multirow{2}{*}{High}  & 0 &0 &0 & 0/10 \\
		& VMA & & 0 & 0 & 0 & 0/10 \\ \midrule
		\multirow{4}{*}{Live} & \multirow{2}{*}{Quasi-instantaneous} & Low & 0 & 3 & 0.8 & 5/10 \\ 
		& & High & 0 & 37 & 13.8 & 7/10 \\ 
		&\multirow{2}{*}{VMA} & Low & 0 & 1 & 0.1 & 1/10\\
		&& High & 3 & 7 & 4.9 & 9/9 \\
                \bottomrule
        \end{tabular}
        \caption{The table shows the minimum, maximum and average number of
	quasi\hyph instantaneous and VMA inconsistencies found in the 30 memory dumps
	created for the evaluation. Out of the memory dumps taken with
	high system load one could not be analyzed regarding VMA
	inconsistencies. Therefore the average number of inconsistencies is
	calculated for nine instead of ten dumps.}
        \label{tab:incons}
\end{table*}

We use LiME with option \texttt{format=lime} to take memory dumps of
running systems. This is done from within the VM and integrated into the same
script that performs the other automated tasks.
For each activity (low and high), ten memory dumps were taken. The
observed inconsistencies are summarized in Table~\ref{tab:incons}.

For low system activity, fewer inconsistencies occurred than for
high system activity. The number of memory dumps affected by quasi\hyph instantaneous
inconsistencies is higher than the number of dumps in which VMA inconsistencies
were found.

With higher activity, the number of inconsistencies rises distinctly. Seven out
of ten memory dumps contain quasi\hyph instantaneous inconsistencies. Out of
them one only contained two inconsistencies, the others 15 or more.
The three memory dumps that contain no quasi\hyph instantaneous
inconsistencies, and the one with only two are noteworthy compared to the
average number of found inconsistencies. For VMA inconsistencies a similar
observation can be made. The nine memory dumps that could be analyzed
regarding VMA inconsistencies, contained them. They are attributed to processes
related to the web browser, audio, and gnome-shell.
One memory dump had to be excluded from the VMA inconsistency check as 
during the check for VMA inconsistencies in this dump, the function
used in the \texttt{linux\_validate\_vmas} Volatility plugin to traverse the red
black tree did not return and the plugin's
execution had to be stopped. Memory dumps for which the plugin did
not terminate were also observed by \citet{introducing}. While this is
probably one symptom of inconsistencies in the memory dump, no statement about
the number of VMA inconsistencies for this memory dump can be made.



\section{Discussion}
\label{sec:discussion}

As expected most of the memory dumps taken on the live system with high activity
are not quasi\hyph instantaneously consistent. But the numbers vary
noticeably and three memory dumps did not have inconsistencies. Using the
information available through the check for quasi\hyph instantaneous consistency, we
can perform some further examinations. Given the small number of memory dumps,
we do not claim that these results can be generalized to any memory acquisition
but they show the advantages of using our method when investigating the reasons
for inconsistencies in a memory dump.

\paragraph{Reconstruction of physical addresses:} While checking the heap of the pivot
program for inconsistencies, the virtual addresses of the list elements and the
global counter array were gathered. They are ordered sequentially on adjacent
pages in the virtual memory but their mappings to physical pages do not have to
be in the same order or address range.
%As we found quasi\hyph instantaneous inconsistencies in every
%memory dump taken of live systems with high activity, like the VMA
%inconsistencies, we further examine the memory dumps. 
Therefore, we reconstructed to which physical pages they were mapped using
Volatility's \texttt{linux\_memmap} plugin. From these mappings, we could
reconstruct the range of physical addresses in which the list elements and the
global counter array were located. This allows us to calculate the distance of each address
to the nearest next address (i.e, the nearest list element).
Table~\ref{tab:dist} summarizes the findings per memory dump ordered by the
number of found quasi\hyph instantaneous inconsistencies in them.
\emph{Range (in pages)} is the size of the physical address range in which the
list elements and global counter array are located, displayed as number of pages.
It is calculated by subtracting the lowest found address from the highest. The
\emph{distances} columns include the number of list elements that were within a 10
pages radius or directly neighbors, respectively. The largest found distance
between two elements is given by \emph{Max distance}.
All distances are given as the number of pages (the page size is 4096 bytes).

%\paragraph{Activity causes `spread':} 
\paragraph{Spread is bad:}
The table supports the intuition that a longer
range in which the addresses are distributed will likely lead to more
inconsistencies. %The longest maximal distances are also observed for these
%memory dumps. 
Or vice-versa, in memory dumps with fewer inconsistencies, more
contents of interest are located on adjacent pages than in those with more
inconsistencies. 

\paragraph{Details - dump \#1:} 
It has the highest number of inconsistencies but a relatively high number of
adjacent physical addresses and not the largest range of addresses. The maximal
distance between two list elements is however the largest one in the evaluated
memory dumps. Taking a look at the location of the list element for which
the current time was calculated reveals that it is located towards the end of
the memory range and is separated by the observed largest distance from the
previous 93 list elements. Thus, it is likely that there was a longer time frame
during the acquisition between copying the previous elements and the last ones.
Combining this with the earlier acquisition of most of the list elements, it
becomes likely that updates on them are missed. 

\paragraph{Details - dump \#6:}
Here, the range is the third
smallest, and 85 of 101 addresses have a distance between one and ten pages but
still 15 inconsistencies occurred. A closer look at the list of elements
for which inconsistencies occurred in this memory dump provides a possible
explanation. They are located more in the beginning of the address range with
mostly smaller distances between them while the list element with the highest
time stamp, i.e. the one for which the current time was identified, is located
more towards the end. A big gap, 72\,745 pages (the largest distance plus some
smaller gaps afterward), lies between the last list
element with an inconsistency and the one with the highest time stamp.
Therefore, similarly to dump \#1, it becomes more likely that changes on the earlier list elements
occur before this list element is acquired. From the difference between the
counters of the list elements for which inconsistencies were detected and their
values in the current time we can also see that many updates occurred on them,
the smallest number of missed updates is 42, the largest 367.
%\todo{Ist das irgendwie hilfreich/nachvollziehbar? FF: ja. ``missed updates'' sind ein interessanter Parameter, den man messen koennte.}

\begin{table*}[pos=ht]
	\centering
	\begin{tabular}{cccccc}
                %\cline{2-8}
                \toprule
                %\cmidrule[0.5pt]{3-9}
		\# &Inconsistencies & Range (in pages) &
		\multicolumn{1}{c@{\hspace*{\tabcolsep}\makebox[0pt]{$\supset$}}}{Distances $<=10$ pages}&
		Distances $=1$ page & Max distance\\
                %\midrule
		\midrule
		1&37 & 224\,575 & 61 & 43 & 103\,122\\
		2&30 & 423\,245 & 47 & 26 & 79\,613\\
		3&21 & 141\,591 & 20 & 5 & 54\,774\\
		4&17 & 150\,635 & 33 & 5 & 53\,319\\
		5&16 & 267\,028 & 44 & 23 & 82\,596 \\
		6&15 & 79\,296 & 85 & 42 & 71\,215\\
		7&2 & 99\,921 & 81 & 45 & 55\,761 \\
		8&0 & 82\,526 & 76 & 40 & 62\,653\\
		9&0 & 12\,132 & 75 & 57 & 3\,170\\
		10&0 & 4\,431 & 97 & 26 & 2\,665\\
                \bottomrule
        \end{tabular}
        \caption{The table shows the number of quasi\hyph instantaneous
	inconsistencies found in the ten memory dumps taken of the live system
	with high activity. For each dump the range in which the physical
	addresses of the 100 list elements and the global counter array are
	contained is given in pages. Additionally, the number of distances between the list
	elements and global counter array that are smaller than 11 pages and the
	subset of these distances that is equal to one page are shown. The last
	column shows the maximal observed distance between two list elements in
	the memory dump.}
        \label{tab:dist}
\end{table*}

%\textrightarrow q-i allows to identify where outdated information is located.
%This ins turn helps to formulate wellfounded hypotheses for their reason and might
%help to identify mitigation strategies.

\paragraph{Pivot program - pros and cons:}
	The examples show how checking for quasi\hyph instantaneous consistency allows
gaining more insights into where content mismatches occur and how many changes
on the memory regions were missed. This is currently limited to the pivot
program. But the usage of the pivot program also has benefits, since its size can be
chosen (for example by changing the number of list elements or their
size) and the degree of activity can be manipulated by the number of threads and
the frequency at which they access the list elements. As the pivot program is
started as a user process, it also allows us to gather insights into the influence
memory allocation strategies and fragmentation have on the process address space
layout at the physical level. The different ranges and distances shown in
Table~\ref{tab:dist} show that even when starting the program at approximately
the same time after booting the layout varies distinctly.
Should it be the goal
to observe the consistency when the physical pages are located in specific page
ranges, it would also be possible to remap the process's heap pages to different
physical addresses using, for example, a kernel module.

VMA inconsistencies could also be analyzed more thoroughly. For example, it would
be possible to determine the addresses of the elements of the red-black tree,
the elements of the VMA list, and the counter for VMAs, and judging from their
relative locations it would be possible to estimate where the inconsistency
could stem from, e.g., is the counter outdated or the number of elements in the
list. But identifying where exactly information is missing would require more
effort and may be impossible. It would also be more difficult or impossible to
find out how many updates were missed exactly. The observed memory range is also
limited when looking only at VMA inconsistencies. 

To cover a bigger memory range
having multiple indicators for content mismatches at hand would be convenient,
searching for them might be eased by understanding the structures used by the
operating system and their connections with each other better
\citep{pagani_back_2019}.

%Are there any advantages over using indicators like VMA inconsistencies?
%Yes: 
%1. With a q-i measuring method we can identify which part of the information
%is outdated. This is missing when indicators like VMA inconsistencies are used.
%But it should be possible to determine the virtual addresses of the used
%structures and the physical pages they are mapped to \textrightarrow it seems
%reasonable to assume that information located in pages that are acquired earlier
%will contain the outdated information. So even for VMA inconsistencies we could
%formulate informed guesses. Conclusion: It is still a useful aspect of the
%method but it could be rebuilt using structures that are already a part of the
%system.
%2. Maybe useful: We know how many updates are missing (Example: counter value of
%list element is 10, in the current time it is 50, we missed 40 accesses to the
%list element). Gives us a feeling for
%how ``busy'' the memory region was during the acquisition. More complicated with
%VMA inconsistencies (but maybe possible)


\section{Conclusions and Future Work}
\label{sec:conclusion}

So, finally, how can we obtain a good memory snapshot?
While this is trivial for systems that can be paused (instantaneous snapshot), the 
situation is more complex for running ones. 
We, therefore, looked into the notion of quasi\hyph instantaneous consistency which is a similar property
but also works for active systems.

In this paper, we showcased a method to observe quasi\hyph instantaneous consistency, 
validated it theoretically, and also demonstrated it in a case study.
Our method allows assessing a portion of a memory dump for quasi\hyph instantaneous
inconsistencies based on a \emph{single} memory dump. This
includes locating the memory regions
with inconsistencies and evaluating how many events on them were missed.
A property that is useful when searching for improvements in existing memory
acquisition methods. For example, identifying alternative orders of memory
acquisition, comparable to the adaptations to LiME suggested by \citet{introducing}.
%
Our tests then confirmed that instantaneous snapshots (frozen systems) are indeed perfect
and are the method of first resort. Furthermore, for live systems, we were able to show that a high system load 
results in more inconsistencies. More tests are needed here, but this could potentially mean
that it may be wise to close (non-relevant) applications before obtaining a snapshot 
on a running system.
%
%% Frank:Habe mir erlaubt das auszukommentieren, mich bring es durcheinander .. 
%
%But it is only suited for testing environments. 
%When a memory dump taken during
%a live investigation should be assessed, indicators like VMA inconsistencies
%are helpful. To cover a bigger memory range having multiple indicators
%for content mismatches at hand would be convenient, searching for them might be
%eased by understanding the structures used by the operating system and their
%connections with each other better \citep{pagani_back_2019}.
%
%
%The q-i observation method is flexible in the sense that it can be applied to
%differently defined memory regions and different ranges of memory. By
%integrating the method on the kernel or hypervisor level we could monitor
%arbitratry address ranges or processes. Obviously this requires a lot of additional
%implementation work compared to the current scope of the implementation. The
%scope of the insights gained by VMA inconsistencies varies according to the
%number of running processes but is always limited to the structures that are
%used to manage the VMAs. If insights about other memory parts would be desired
%different indicators would need to be identified.


%\todo{Add: qui-consistency ist ein erwuenschter Zustand, waere eine wichtige research
%Richtung. Forensic soundness? Als open problem oder so formulieren}
%\subsection{Future work}

Moving forward, the method could be used to evaluate different memory
acquisition tools. Thereby, a broader data base could be built to investigate our
preliminary observations further. The method for observing quasi\hyph instantaneous consistency
could also be moved to different memory ranges than the pivot program. Concerning
the main memory, it would be possible to integrate it at the hypervisor level. From
there the address ranges that contain contents of interest could be identified
and observed. For example, the memory ranges containing structures that are
necessary for the analysis, like page tables and process structures. A second consideration
is alienating this method to live acquisition of hard disk contents where it
could be possible to integrate it into the file system or block device drivers.
%Move away from pivot program and integrate observation of qui-c into system. For
%main memory setups with a hypervisor are feasible. When we think about file
%systems, we could integrate the global counter array and update method into a
%file system driver. Interesting in scenarios where file system acquisition takes
%place or hard disks are copied during operation.

Our implementation also suffers some shortages which require attention. For instance, the 
pivot program utilizes a fixed-size global counter array which is not practical for larger
memory regions or longer observation times.  One possibility would be to implement
it as a ring buffer, i.e., 
%When moving away from the pivot program, the method likely has to be adapted to
%allow the observation of a dynamically changing sets of memory regions and longer
%observation times. As there is no infinite space the global counter array will
%be full at some point in time, even when considering dynamically allocating more
%space for it. It could be implemented like a ring buffer, 
restarting at the beginning once the last entry has been written. This would require an overflow
detection in the implementation, and in the analysis, with the latter being the
more difficult task.

\subsection*{Acknowledgments}

Work was supported by Deutsche
Forschungsgemeinschaft (DFG, German Research Foundation) as part of
the Research and Training Group 2475 ``Cybercrime and Forensic
Computing'' (grant number 393541319/GRK2475/1-2019).

\printcredits

\bibliography{../quellen}


\end{document}

\subsubsection{Hardness implications}
In contrast to the previous section, here, we see an application of Proposition~\ref{pro:omega_s} and Theorem~\ref{thm:suff_cdclique} in deriving hardness results for \CDC\ and \SCP. 

\section*{$C_6$-free bipartite graphs}

%\label{subsec:c6-free}
We prove the hardness by proposing a reduction from the problems \CC\ and \ISP\ for diamond-free graphs to the problems \CDC\ and \SCP\ for $C_6$-free bipartite graphs, respectively. Note that both the problems, {\sc Clique Cover} and {\sc Independent Set}, are NP-hard for diamond-free graphs~\cite{kral2001complexity,Poljak1974}. Recall that, a \textit{a diamond} is a graph obtained by deleting an edge from $K_4$ (complete graph on 4 vertices)

First, we note the following observation for diamond-free graphs~\cite{chiarelli2021strong}.

\begin{observation}\label{obs:diamondfree}
Any diamond-free graph with $m$ edges can have at most $m$ maximal cliques.
\end{observation}

Let $G$ be a diamond-free graph and let $\mathcal{K}=\{K_1,K_2,\ldots,K_l\}$ denote the collection of all maximal cliques in $G$. By Observation~\ref{obs:diamondfree}, we have that $l\leq |E(G)|$. From $G$, we now construct a bipartite graph $B_G=(A,B,E)$ as follows: we define $A=V(G)\cup\{u\}$ and $B=\mathcal{K}=\{K_1,K_2,\ldots,K_l\}$ (each vertex in the partite set $B$ represents a maximal clique in $G$). For a pair of vertices, $a\in A\setminus \{u\}$ and $K_j\in B$ (where $j\in \{1,2,\ldots,l\}$), we make $a$ and $K_j$ adjacent in $B_G$ if and only if $a\in K_j\subseteq V(G)$. In addition, we make the vertex $u$ adjacent to all the vertices in $B$. i.e. $E(B_G)=\{ab:a\in A\setminus \{u\}, b=K_j\in B$, for some $j\in \{1,2,\ldots,l\}$, and $a\in K_j\subseteq V(G)\}\cup \{ub:b\in B\}$. We then have the following lemmas.

\begin{lemma}\label{lem:isomprphic}
Let $G$ be a diamond-free graph and $B_G=(A,B,E)$ be the corresponding bipartite graph (as defined above). We have $B_G^2[A\setminus \{u\}]\cong G$ and $B_G^2[B]$ is a clique.
\end{lemma}
\begin{proof}
Let $H=B_G^2[A\setminus \{u\}]$. Clearly, $V(H) = V(G)$. Let $x,y\in V(H)$. Then, as $B_G$ is a bipartite graph, and $x,y\in V(H)\subseteq A\setminus \{u\}$, we have $xy\in E(H)\iff \exists z=K_j\in B$ for some $j\in \{1,2,\ldots,l\}$ such that $x,y\in N_{B_G}(z)$ (i.e. both the vertices $x$ and $y$ belong to a same maximal clique in $G$) $\iff xy\in E(G)$. This proves that $B_G^2[A\setminus \{u\}]\cong G$. Further, it is easy to see that $B_G^2[B]$ is a clique, since $B$ is an independent set and $B\subseteq N_{B_G}(u)$.
\end{proof}

\begin{lemma}\label{lem:diamondC_6}
    Let $G$ be a diamond-free graph and $B_G=(A,B,E)$ be the corresponding bipartite graph. Then $B_G$ is $C_6$-free.
\end{lemma}
\begin{proof}
    Suppose not. Let $S=\{a_1,b_1,a_2,b_2,a_3,b_3,a_1\}$ be a set of vertices in $B_G$ that induces a $C_6$ in $B_G$, where $\{a_1,a_2,a_3\}\subseteq A$ and $\{b_1,b_2,b_3\}\subseteq B$.  Clearly, $\{a_1,a_2,a_3\}$ induces a triangle in $B_G^2[A]$. Since each of the vertex $a_i$ has a non-neighbor in $B$, we have $u\neq a_i$ for any $i\in \{1,2,\ldots,3\}$. This implies that $\{a_1,a_2,a_3\}$ induces a triangle in $B_G^2[A\setminus \{u\}]$, and thus a triangle in $G$ by Lemma~\ref{lem:isomprphic}. Now, consider the vertex $b_1$. We have $N_{B_G}(b_1)\cap S =\{a_1,a_2\}$. Clearly, $\{a_1,a_2\}$ is not a maximal clique in $G$, since $\{a_1,a_2,a_3\}$ induces a triangle in $G$. Note that $b_1=K$, where $K$ is some maximal clique in $G$. As $a_3\notin N_{B_G}(b_1)$, we have by the definition of $B_G$ that $a_3\notin K$. Since $K$ is a maximal clique in $G$ containing the set $\{a_1,a_2\}$, but not the vertex $a_3$, there exists a vertex $a_4\in V(G)\subseteq A$ such that $a_4\in K$ and $a_3a_4\notin E(G)$. This implies that $\{a_1,a_2,a_3,a_4\}$ induces a diamond in $G$. This contradicts the fact that $G$ is diamond-free.
\end{proof}
\begin{theorem}\label{thm:C_6free hard}
    The problems  \CDC\ and \SCP\ are NP-Compete for $C_6$-free bipartite graphs.
\end{theorem}
\begin{proof}
    Let $G$ be a diamond-free graph, and let $B_G$ be the corresponding bipartite graph. By Proposition~\ref{pro:omega_s} applied to the bipartite graph $B_G$, we have that $\SC(B_G)=\alpha(B_G^*)$, where $B_G^*=B_G^2- B_G$ = $B_G^2[A]\uplus B_G^2[B]$. Therefore,  we have $\SC(B_G)=\alpha(B_G^*) = \alpha(B_G^2[A])+\alpha(B_G^2[B])$ = $\alpha(B_G^2[A\setminus \{u\}])+\alpha(B_G^2[B])$ (the last equality is due to the fact that, in the graph $B_G^2[A]$, the vertex $u$ is adjacent to every vertex in $A\setminus \{u\}$). This implies that $\SC(B_G)=\alpha(B_G^*) = \alpha(G)+1$ (by Lemma~\ref{lem:isomprphic}). Note that $G$ is diamond-free and $B_G$ is a $C_6$-free bipartite graph (by Lemma~\ref{lem:diamondC_6}). Therefore, the fact that \ISP\ problem is NP-hard for diamond-free graphs implies that \SCP\ is NP-hard for $C_6$-free bipartite graphs. 

    \medskip

    By Lemma~\ref{lem:diamondC_6}, we have that $B_G$ is $C_6$-free. Since $B_G$ is also triangle-free, we can therefore conclude that $B_G$ is $\mathcal{H}$-free. Therefore, by Theorem~\ref{thm:suff_cdclique}, we have that $\XCD(B_G)=k(B_G^*)$. As before, since $B_G^*= B_G^2[A]\uplus B_G^2[B]$, we then have $\XCD(B_G)=k(B_G^*) = k(B_G^2[A])+k(B_G^2[B])$ = $k(B_G^2[A\setminus \{u\}])+k(B_G^2[B])$ (again, the last equality is due to the fact that, in the graph $B_G^2[A]$, the vertex $u$ is adjacent to very vertex in $A\setminus \{u\}$). This implies that $\XCD(B_G)=k(B_G^*) = k(G)+1$ (by Lemma~\ref{lem:isomprphic}). As noted before, $G$ is diamond-free and $B_G$ is a $C_6$-free bipartite graph. Therefore,  the fact that \CC\ problem is NP-hard for diamond-free graphs implies that \CDC\ is NP-hard for $C_6$-free bipartite graphs. Hence the theorem.
\end{proof}


%\input{stuctural/bounds}

\section{Separted-cluster problem in Interval graphs}
\label{sec:interval}

In this section, we settle an open problem proposed by Shalu et.al.~\cite{DBLP:conf/caldam/ShaluVS17}, by proving that \SCP\ is polynomial-time solvable for the class of interval graphs. We achieve this, by showing a polynomial-time reduction of \SCP\ problem on interval graphs to the maximum weighted independent set problem on cocomparability graphs (which can be solved in time linear to the size of the input graph). Note that throughout the section, we make use of the alternative definition of \textit{separated-cluster}. i.e. for a graph $G$, we say that a family of disjoint cliques, say $\mathcal{S}=\{C_1,C_2,\ldots,C_k\}$ is a \textit{separated cluster} in $G$ if there does not exist a pair of vertices $x\in C_i$, $y\in C_j$, where $i,j\in \{1,2,\ldots,k\}$, $i\neq j$ and $z\in V(G)\setminus (C_i\cup C_j)$ such that $x,y\in N_G(z)$.  

\medskip

Let $G$ be an interval graph with an interval representation, $\{I_v\}_{v\in V(G)}$, where $|V(G)|=n$. 
Without loss of generality, we can assume that \textit{all the end-points of the intervals in $\{I_v\}_{v\in V(G)}$ are distinct}. 
For a vertex $v\in V(G)$, for the sake of convenience, we denote by $l(v)$ and $r(v)$, the \textit{left} and \textit{right end-points} of the interval $I_v$, respectively.  


\medskip

\noindent\textbf{Definition of $\hat{\mathcal{C}}$:} Let  $\{C_1,C_2,\ldots, C_l\}$ be the collection of all \textit{maximal cliques} in $G$.  For $i\in \{1,2,\ldots,l\}$, let $|C_i|=t$. We enumerate the vertices belonging to $C_i$ in two different ways (with respect to the left and right end-points of their intervals). i.e. $C_i^l=(v_{l1},v_{l2},\ldots, v_{lt})$ is such that $l(v_{l1})< l(v_{l2})<\cdots < l(v_{lt})$, and let $C_i^r=(v_{r1},v_{r2},\ldots, v_{rt})$ is such that $r(v_{r1})< r(v_{r2})<\cdots <r(v_{rt})$. We then define the following.  Let $i\in \{1,2,\ldots,l\}$ and $p,q\in \{0,1,2,\ldots,|C_i|-1\}$. For $p,q>0$, let $C_i^{v_{lp}}=\{u\in C_i: l(u)>l(v_{lp})\}$ and $C_i^{v_{rq}}=\{u\in C_i:  r(u)<r(v_{rq})\}$. And, let $C_i^{v_{l0}}=C_i^{v_{r0}}=C_i$ (note that for each $i$, $v_{l0}$ and $v_{r0}$ represent the left and right counterparts of a dummy vertex for the sake of convenience in the notation).  We then define the set $C_i^{v_{lp}v_{rq}}=C_i^{v_{lp}}\cap C_i^{v_{rq}}$. In other words, $C_i^{v_{lp}v_{rq}}$ is either the maximal clique $C_i$ (when $p=q=0$), or a clique obtained from $C_i$, \textit{by dropping the vertices} whose left end-point is less than or equal to that of $l(v_{lp})$ (if $p>0$) or their right end-point is  greater than or equal to that of $r(v_{rq})$ (if $q>0$) or both (if both $p,q>0$). 
Let $\hat{\mathcal{C}_i}=\{C_i^{v_{lp}v_{rq}}:p,q\in \{0,1,2,\ldots,|C_i|-1\}\}$. We then let $\hat{\mathcal{C}}=\{\hat{\mathcal{C}_i}:i\in \{1,2,\ldots,l\}\}$. Note that as $G$ is an interval graph, we have that $l\leq n$~\cite{golumbic2004algorithmic}. Since the number of ordered pairs of the form $(p,q)$ is at most $n^2$ (as $p,q\in \{0,1,2,\ldots,|C_i|-1\}$, $|C_i|\leq n$), we then have that $|\hat{\mathcal{C}}|\leq n^3$.  

\medskip

Let $\mathcal{S}$ be a separated cluster in $G$. Note that by definition, any pair of cliques belonging to $\mathcal{S}$ are disjoint. Since $G$ is an interval graph, the cliques in $\mathcal{S}$ have disjoint \textit{Helly regions} (the common intersection region of intervals belonging to the clique) in the interval representation of $G$. Let $S_1,S_2,\ldots,S_k$ be an ordering of the cliques in $S$ having the following property: \textit{ for any pair $r,s\in \{1,2,\ldots,k\}$, we have $S_r<S_s$ if and only if the Helly region of the clique $S_r$ lies entirely left to the Helly region of the clique $S_s$}. For a clique $S_i\in \mathcal{S}$, where $i\in \{1,2,\ldots,k\}$, let $C_i$ be a maximal clique in $G$ such that $S_i\subseteq C_i$ (possibly, $C_i=S_i$). Suppose that $C_i\setminus S_i\neq \emptyset$. Clearly, by the definition of $\mathcal{S}$, no vertex in $C_i\setminus S_i$ can be adjacent to any vertex in $S_j$, for $j\neq i$. We say that a vertex $x\in C_i\setminus S_i$ has a \textit{left-conflict} (resp. \textit{right-conflict}) with respect to $\mathcal{S}$, if there exists a clique $S_j$ with $S_j<S_i$ (resp. $S_j>S_i$) and vertices $y\in S_j$, $z\in N(y)\cap (V(G)\setminus S_j)$ such that $xz\in E(G)$.  

 The following observation is due to the definitions of \textit{left-conflict} (resp. \textit{right-conflict}), and $C_i^{v_{lp}}$ (resp. $C_i^{v_{rq}}$).
\begin{observation}\label{obs:conflict}
    Let $\mathcal{S}=\{S_1,S_2,\ldots,S_k\}$ be a separated cluster in $G$. For $i\in \{1,2,\ldots,k\}$, let $C_i$ be a maximal clique in $G$, such that $S_i\subseteq C_i$. Let a vertex $v\in C_i\setminus S_i$ has a \textit{left-conflict} (resp. \textit{right-conflict}) with respect to $\mathcal{S}$. Let $v=v_{lp}$ (resp. $v=v_{rq}$) in the ordering $C_i^l$ (resp. $C_i^r$). Then every vertex in $C_i\setminus C_i^{v_{lp}}$ (resp. $C_i\setminus C_i^{v_{rq}}$) has a \textit{left-conflict} (resp. \textit{right-conflict}). In other words, we then have, $S_i\subseteq C_i^{v_{lp}}$ (resp. $S_i\subseteq C_i^{v_{rq}}$).
\end{observation}
\begin{observation} \label{obs:noconflict}
Let $\mathcal{S}=\{S_1,S_2,\ldots,S_k\}$ be a separated cluster in $G$. For $i\in \{1,2,\ldots,k\}$, let $C_i$ be a maximal clique in $G$, such that $S_i\subseteq C_i$. If there exists a vertex $x\in C_i\setminus S_i$ such that $x$ has neither a left-conflict or a right-conflict with respect to $\mathcal{S}$, then $\mathcal{S}=\{S_1,S_2,\ldots, S_i',\ldots,S_k\}$ is also a separated cluster in $G$, where $S_i'=S_i\cup\{x\}$.
\end{observation}
\begin{proof}
    Since any pair of cliques in $\mathcal{S}$ are mutually non adjacent, we have $C_i\cap S_j=\emptyset$ for each $j\in \{1,2,\ldots,k\}$, with $i\neq j$. Further, no vertex in $C_i$ can be adjacent to any vertex in $S_j$, with $i\neq j$, as otherwise, there exists a vertex $u\in S_j$, which is lying exactly at a distance 2 from each vertex in $S_i$, contradicting the fact that $\mathcal{S}$ is a separated cluster. Now, consider a vertex $x\in C_i\setminus S_i$,  which has neither a \textit{left-conflict} or a \textit{right-conflict} with respect to $\mathcal{S}$. For $j\neq i$, let $y\in S_j$. Suppose that $S_j<S_i$ (resp. $S_j>S_i$). If $d_G(x,y)=2$, then by a previous observation, there exists a vertex $z\in N(y)\cap (V(G)\setminus S_j)$ such that $xz\in E(G)$. This implies that $x$ has a \textit{left-conflict} (resp. \textit{right-conflict}) with respect to $\mathcal{S}$. Since we have a contradiction, we can therefore conclude that $d_G(x,y)\neq 2$ for any vertex $y\in S_j$, where $j\neq i$. This implies that $\mathcal{S}=\{S_1,S_2,\ldots, S_i',\ldots,S_k\}$ is also a separated cluster in $G$, where $S_i'=S_i\cup\{x\}$. 
\end{proof}
\begin{lemma} \label{lem:maxsep}
    Let $\mathcal{S}=\{S_1,S_2,\ldots,S_k\}$ be a maximum cardinality separated cluster in $G$. Then for each $i\in \{1,2,\ldots,k\}$, we have $S_i\in \hat{\mathcal{C}}$.
\end{lemma}
\begin{proof}
For $i\in \{1,2,\ldots,k\}$, let $S_i$ be a clique in $\mathcal{S}$. If $S_i$ is a maximal clique in $G$, then $S_i\in \hat{\mathcal{C}}$, as $\hat{\mathcal{C}}$ contains all maximal cliques in $G$. Suppose that $S_i$ is not a maximal clique in $G$. Then, there exists a maximal clique, say $C_i$ in $G$, such that $S_i\subseteq C_i$. Clearly, $C_i\setminus S_i \neq \emptyset$. Since $\mathcal{S}$ is a maximum cardinality separated cluster in $G$, by Observation~\ref{obs:noconflict}, we have that every vertex in $C_i\setminus S_i$ has either a \textit{left-conflict} or a \textit{right-conflict} or both. 

Define $L=\{u\in C_i\setminus S_i: u$ has a \textit{left-conflict} with respect to $\mathcal{S}$\}, and $R=\{u\in C_i\setminus S_i: u$ has a \textit{right-conflict} with respect to $\mathcal{S}$\}. Therefore, we have $C_i\setminus S_i=L\cup R$. Since $L,R\subseteq C_i$, we may consider $L\subseteq C_i^l$ and $R\subseteq C_i^r$. We then define the following:
$$v_{lp}=\begin{cases}
     \text{ the vertex in } L \text{ having the property that } l(v_{lp})=\max\{l(v_{li}):v_{li}\in L\subseteq C_i^l\}, & \text{ if } L\neq \emptyset \\
     v_{l0}, & \text{ otherwise}
\end{cases}$$
 $$v_{rq}=\begin{cases}
     \text{ the vertex in } R \text{ having the property that } r(v_{rq})=\min\{r(v_{ri}):v_{ri}\in R\subseteq C_i^r\}, & \text{ if } R\neq \emptyset \\
     v_{r0}, & \text{ otherwise}
\end{cases}$$
We then have the following claim.

\medskip

\noindent\textbf{Claim: $S_i=C_i^{v_{lp}v_{rq}}$}

\smallskip
Let $u\in S_i\subseteq C_i$. If $v_{lp}\neq v_{l0}$, then as $v_{lp}\in L\subseteq C_i$ has a \textit{left-conflict}, by Observation~\ref{obs:conflict}, we have that $S_i\subseteq C_i^{v_{lp}}$. Similarly, if $v_{rq}\neq v_{r0}$, then again, by Observation~\ref{obs:conflict}, we have that $S_i\subseteq C_i^{v_{rq}}$. Also, if $v_{lp}= v_{l0}$ then $S_i\subseteq C_i= C_i^{v_{lp}}$, and if $v_{rq}=v_{r0}$ then $S_i\subseteq C_i=C_i^{v_{rq}}$. Therefore, in any case, we have, $S_i \subseteq C_i^{v_{lp}}\cap C_i^{v_{rq}}\subseteq C_i^{v_{lp}v_{rq}}$. On the other hand, by the choices of $v_{lp}$ and $v_{rq}$, we have that no vertex in $C_i^{v_{lp}v_{rq}}$ can have a \textit{left-conflict} or a \textit{right-conflict}. Then by Observation~\ref{obs:noconflict} and the fact that $\mathcal{S}$ is a maximum cardinality separated cluster, we can conclude that $C_i^{v_{lp}v_{rq}} \subseteq S_i$ as well. This proves our claim, and also completes the proof of the lemma, since $C_i^{v_{lp}v_{rq}}\in \hat{\mathcal{C}}$.
\end{proof}
For the reduction purpose, we now define the following:
\begin{definition}[Weighted conflict graph $G_c$] \label{def:conflict}
    Given an interval graph, $G$, having maximal cliques, say $\{C_1,C_2,\ldots,C_l\}$, the weighted conflict graph $G_c$ is defined to be the graph with $V(G_c)=\hat{\mathcal{C}}=\{C_i^{v_{lp}v_{rq}}: i\in \{1,2,\ldots,l\}, p,q\in \{0,1,2,\ldots,|C_i|-1\}\}$, and weight function $w(X)=|X|$, for each $X\in V(G_c)$. For any pair of vertices, say $X,Y\in V(G_c)$, we make $X$ and $Y$ adjacent in $G_c$, if and only if at least one of the following conditions is true.
    \begin{enumerate}[label=(\alph*)]
        \item \label{cond:1} $X\cap Y\neq \emptyset$
        \item \label{cond:2} there exist vertices, $x\in X$ and $y\in Y$ such that $xy\in E(G)$
        \item \label{cond:3} there exist vertices, $x\in X$, $y\in Y$, and $z\in V(G)\setminus (X\cup Y)$ such that $x,y\in N_G(z)$.
    \end{enumerate}
\end{definition}
Our aim is to prove that $G_c$ is a cocomparability graph. For this, we need the following vertex ordering characterization of cocomparability graphs due to Damaschke~\cite{damaschke1990forbidden}.
\begin{theorem}[~\cite{damaschke1990forbidden}]\label{thm:cocomp}
    An undirected graph $H$ is a cocomparability graph if and only if there is an ordering $<$ of $V(H)$ such that for any three vertices $i<j<k$, if $ik\in E(H)$, then either $ij\in E(H)$ or $jk\in E(H)$.
\end{theorem}
The vertex ordering specified in Theorem~\ref{thm:cocomp} is called an \textit{umbrella-free} ordering~\cite{damaschke1990forbidden}. Theorem~\ref{thm:cocomp} essentially says that cocomparability graphs are exactly those graphs whose vertex set admits an \textit{umbrella-free} ordering.

\medskip

Let $G$ be an interval graph with an interval representation, $\{I_v\}_{v\in V(G)}$, where $|V(G)|=n$. For the sake of convenience, let $V(G_c)=\hat{\mathcal{C}}=\{K_1,K_2,\ldots,K_t\}$, where $t=|\hat{\mathcal{C}}|\leq n^3$. Since each of the sets $K_j$, where $j\in \{1,2,\ldots,t\}$ is a clique in $G$, they have a \textit{Helly region} (the common intersection region of intervals  belonging to the clique), say $H_j$ in the interval representation, $\{I_v\}_{v\in V(G)}$. Let $j\in \{1,2,\ldots,t\}$. Let $r(H_j)$ denote the right end-point of the \textit{Helly region} corresponding to the clique $K_j$. We then define an ordering $<$ of the vertices in $G_C$ having the following property: \textit{for any two vertices $K_i,K_j\in V(G)$, where $i,j\in \{1,2,\ldots,t\}$, we have $K_i<K_j$ if and only if $r(H_i)\leq r(H_j)$} (break the ties arbitrarily, i.e. for any pair $i,j\in \{1,2,\ldots,t\}$,  if $r(H_i)= r(H_j)$, then we can have either $K_i<K_j$ or $K_j<K_i$). 

We then have the following lemma.

\begin{lemma}
    For any interval graph $G$, the weighted conflict graph $G_c$ is a cocomparability graph.
\end{lemma}
\begin{proof}
    To prove the lemma, it is enough to show that the ordering $<$ of $V(G_c)$ (defined in the paragraph above) is an umbrella-free ordering. Suppose not. Then there exist cliques, say, $K_i,K_j,K_l$ in $V(G_c)$ such that $K_i<K_j<K_l$, $K_iK_l\in E(G_c)$, but $K_iK_j\notin E(G_c)$ and $K_jK_l\notin E(G_c)$. Since $K_i<K_j<K_l$, we have by the definition of $<$ that $r(H_i)\leq r(H_j)\leq r(H_l)$ (where $H_i$, $H_j$, and $H_l$ denote the \textit{Helly regions} of the cliques, $K_i$, $K_j$, and $K_l$ respectively). Note that $K_iK_l\in E(G_c)$. Therefore, by the definition of $G_c$, at least one of the conditions in Definition~\ref{def:conflict} has to be true. Suppose that the edge $K_iK_l\in E(G_c)$ is due to Condition~\ref{cond:1}. This implies that $K_i\cap K_l\neq \emptyset$. Let $x\in K_i\cap L_l$. Since, $r(H_i)\leq r(H_j)\leq r(H_l)$, we then have that the interval representing the vertex $x$ intersects with the \textit{Helly region}, $H_j$ of the clique $K_j$. Therefore, by Condition~\ref{cond:2} in Definition~\ref{def:conflict}, we have that both the edges, $K_iK_j, K_jK_l\in E(G_c)$. This is a contradiction. Suppose that the edge $K_iK_l\in E(G_c)$ is due to Condition~\ref{cond:2}. This implies that there exist vertices $x\in K_i$ and $y\in K_l$ such that $xy\in E(G)$. Note that $r(u)< r(H_j)$ for each vertex $u\in K_i$ (since $r(H_i)\leq r(H_j)$, $K_iK_j\notin E(G_c)$, and by Condition~\ref{cond:2} of Definition~\ref{def:conflict}). Similarly, we have $l(v)> r(H_j)$ for each vertex $v\in K_l$ (as $r(H_j)\leq r(H_l)$, $K_jK_l\notin E(G_c)$, and by Condition~\ref{cond:2} of Definition~\ref{def:conflict}). Therefore, as $x\in K_i$ and $y\in K_l$, in particular, we have $r(x)<r(H_j)<l(y)$. This implies that $I_x\cap I_y=\emptyset$. Therefore, $xy\notin E(G)$. This is a contradiction to our assumption. Suppose that the edge $K_iK_l\in E(G_c)$ is due to Condition~\ref{cond:3}. This implies that there exist vertices $x\in K_i$, $y\in K_l$, and $z\in V(G)\setminus (K_i\cup K_l)$ such that $x,y\in N_G(z)$. This implies that the interval representing the vertex $z$ intersects with both the \textit{Helly regions} $H_i$ and $H_l$ of the cliques $K_i$ and $K_j$, respectively. Since the point $r(H_j)$ lies in between the \textit{Helly regions} $H_i$ and $H_l$, we can infer that the interval representing the vertex $z$ intersects with the \textit{Helly region} $H_j$ of clique $K_j$ as well. Therefore, by Condition~\ref{cond:2} or Condition~\ref{cond:3} in Definition~\ref{def:conflict} (depending on whether the vertex $z$ belongs to $K_j$ or not), we have that both the edges, $K_iK_j, K_jK_l\in E(G_c)$. This is again a contradiction. Since we obtain a contradiction in all the cases, we can, therefore, conclude that the ordering $<$ of $V(G_c)$ (defined in the paragraph above) is an umbrella-free ordering. This implies that $G_c$ is a cocomparability graph by Theorem~\ref{thm:cocomp}. Hence the lemma.
        \end{proof}

\noindent\textbf{Reduction:} Here, we show a polynomial-time reduction of \SCP\ problem on interval graphs to the maximum weighted independent set problem on cocomparability graphs. We first note the following theorem due to K\"ohler and Mouatadid~\cite{DBLP:journals/ipl/KohlerM16}.
\begin{theorem}[\cite{DBLP:journals/ipl/KohlerM16}]\label{thm:indptcocomp}
    Let $H$ be a cocomparability graph. Then an independent set of maximum possible weight in $H$ can be computed  in $O(|V(H)+|E(H)|)$ time.
\end{theorem}
We then have the following main theorem.
\begin{theorem}\label{thm:reduction}
    Let $G$ be an interval graph, and $G_c$, its conflict-graph. Then $\mathcal{S}=\{S_1,S_2,\ldots,S_k\}$ is a maximum cardinality separated cluster in $G$ if and only if $\mathcal{S}=\{S_1,S_2,\ldots,$\\$S_k\}$ is a maximum weighted independent set in $G_c$.
\end{theorem}
\begin{proof}
Let $\mathcal{T}\subseteq V(G_c)=\hat{\mathcal{C}}$. Then by the definition of the conflict-graph $G_c$ (see Definition~\ref{def:conflict}), we have that $\mathcal{T}$ is an independent set in $G_c$ if and only if $\mathcal{T}$ is a separated cluster in $G$. Since weights of the vertices in the conflict-graph $G_c$ are exactly the cardinality of their corresponding sets,  this implies that $\mathcal{T}$ is a maximum weighted independent set in $G_c$ if and only if $\mathcal{T}$ is a maximum cardinality separated cluster among all the separated cluster in $G$ having their cliques in $\hat{\mathcal{C}}=V(G_c)$. Further, as $\mathcal{S}=\{S_1,S_2,\ldots,S_k\}$ is a maximum cardinality separated cluster in $G$, we have by Lemma~\ref{lem:maxsep} that  $\mathcal{S}=\{S_1,S_2,\ldots,S_k\}\subseteq \hat{\mathcal{C}}=V(G_c)$. This proves the theorem.
\end{proof}
We then have the following corollary due to Theorem~\ref{thm:indptcocomp} and Theorem~\ref{thm:reduction}.
\begin{corollary}
    The \SCP\ problem in interval graphs can be solved  in polynomial time.
\end{corollary}

%\begin{comment}
\section{System Architecture}
\label{appendix:architecture}
\system has a novel modularized system architecture with three key components: 
\emph{StreamManager}, 
\emph{TxnManager} and \emph{TxnScheduler}. 
These components are instantiated in each thread locally.
The execution outline of \system is presented in Algorithm~\ref{alg:algo}.
Transactional stream processing is continuous and potentially never ends (Line 1$\sim$8).
The dependency resolution and execution of state transactions are separated into two non-overlapping phases by punctuations~\cite{Tucker:2003:EPS:776752.776780} (Line 2 and 5), which guarantees that no subsequent input event will have a smaller timestamp. 
Effectively, a batch of state transactions is collected during the first phase, and processed during the second phase.

In the first phase (i.e., stream processing phase), 
the \emph{StreamManager} conducts preprocessing for every input event ($e$). Similar to some prior works~\cite{tstream}, state transactions may be issued but not immediately processed during preprocessing (Line 3).
The \emph{pre\_processing} and \emph{post\_processing} functions are exposed as APIs to users.
The \emph{TxnManager} handles dependency resolution (Line 4) among state transactions and insert decomposed operations to construct a \tpg. We discuss the detailed two-phase \tpg construction process in Section~\ref{subsec:construction}.

In the second phase  (i.e., transaction processing phase), 
the \emph{TxnManager} is first involved again to refine (Line 6) the constructed \tpg with further dependency resolution.
The \emph{TxnScheduler} 
schedules operations for concurrent execution based on the constructed \tpg according to the three dimensions of scheduling decisions (Line 7). 
In particular, a scheduling decision model $M$ is instantiated based on the constructed \tpg (Line 14).
\textbf{\circled{1}} Guided by $M$, execution threads adopt an exploration strategy (Section~\ref{subsec:explore}) to explore the constructed \tpg for operations available to be scheduled constrained by dependencies. 
\textbf{\circled{2}} 
During exploration, one or multiple operations may be treated as the 
% basic 
unit of scheduling (Section~\ref{subsec:granularity}). 
Subsequently, \textbf{\circled{3}} every thread executes operation(s) in the unit of scheduling with various abort handling mechanisms (Section~\ref{subsec:abort_handling}).
Only when state transactions are processed (i.e., committed or aborted) can the associated input events be postprocessed (Line 8) by the \emph{StreamManager} based on transaction processing results.
\end{comment}

\begin{comment}
\begin{algorithm}
\footnotesize
    \KwData{$e$ \tcp{Input event}}
    \KwData{$txn_{ts}$ \tcp{State transaction}}
    \KwData{$G$ \tcp{The currently constructed TPG}}
    \While{!finish processing of input streams}{
        \eIf(\tcp*[h]{Phase 1}){\text{$e$ is not a $punctuation$}}{
                $txn_{ts}$ $\gets$ PRE\_Processing($e$)\;
                \textbf{TPG\_Construction}($G$, $txn_{ts}$)\; 
          }(\tcp*[h]{Phase 2}){
                \textbf{TPG\_Refinement}($G$)\; 
                \textbf{TXN\_Scheduling}($G$)\; 
                POST\_Processing()\;
          }
    }
    
    \SetKwFunction{FMain}{TPG\_Construction}
    \SetKwProg{Fn}{Function}{:}{}
    \Fn{\FMain{$G$, $txn_{ts}$}}{
        $O_{1..k}$ $\gets$ \textbf{Partition} $txn_{ts}$\;
        \ForEach{\text{operation $O_{i}$ $\in$ $O_{1..k}$}}{
            \textbf{Identify} its \ld\;
            $G$ $\gets$ $G$ + $O_{i}$ \;
        }
    }
    \SetKwFunction{FMain}{TPG\_Refinement}
    \SetKwProg{Fn}{Function}{:}{}
    \Fn{\FMain{$G$}}{
        \ForEach{\text{vertex $e_{i}$ $\in$ $G$}}{
            \textbf{Identify} its \td, \pd\;
        }
    }
    
    \SetKwFunction{FMain}{TXN\_Scheduling}
    \SetKwProg{Fn}{Function}{:}{}
    \Fn{\FMain{$G$}}{
        $M$ $\gets$ Instantiated with $G$;\tcp{A decision model}
        \While{!finish scheduling of $G$
        }{
          \textbf{\circled{2}} $Scheduling Unit$ $\gets$ \textbf{\circled{1}} \emph{Explore}($G$, $M$)\; 
            \textbf{\circled{3}} \emph{Execute with Abort Handling} ($Scheduling Unit$)\; 
        }
    }
  \caption{Execution Outline of \system}
  \label{alg:algo}
\end{algorithm}
\end{comment}
\section{Conclusion and Future Work}
In this work, I design corruption-robust algorithms for the Lipschitz contextual search problem. I present the \emph{agnostic checking} technique and demonstrate its effectiveness in designing corruption-robust algorithms. There are several open problems for future research. First, in the algorithm I propose for pricing loss, the schedule for agnostic checks is fixed upfront. Can the learner design an adaptive checking schedule for the pricing loss? Second, this work assumes the learner has knowledge of the Lipschitz constant $L$. Can the learner design efficient no-regret algorithms without knowledge of $L$? 
\bibliographystyle{elsarticle-num} 


    
\RaggedRight%\setlength{\emergencystretch}{.5em}
\bibliography{main}

\end{document}