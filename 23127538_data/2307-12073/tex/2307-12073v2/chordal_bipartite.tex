\subsubsection*{\textbf{Chordal bipartite graphs}} 
The following theorem is proved in~\cite{ShaluKiruP5Free22}.
\begin{theorem}[\cite{ShaluKiruP5Free22}] \label{thm:shalu}
    Let $G$ be a $P_6$-free chordal bipartite graph. Then $\XCD(G)=\SC(G)$, and the problem \SCP\ can be solved in $O(n^3)$ time for $P_6$-free chordal bipartite graphs.
\end{theorem}

We improve and generalize the result above in Theorem~\ref{thm:chordalbip}.
First, note the following theorem proved in~\cite{DamDomChordBipGr90}.
\begin{theorem}[\cite{DamDomChordBipGr90}] \label{thm:totdom}
    The problem {\sc Total Domination} can be solved in $O(n^2)$ time for chordal bipartite graphs.  
\end{theorem}

We now have the following observations concerning the auxiliary graph of bipartite graphs and chordal bipartite graphs.
 
\begin{observation}\label{obs:bipsquare}
Let $G=(A,B,E)$ be a bipartite graph then $G^*=G^2[A]\cup G^2[B]$. 
\end{observation}
\begin{proof}
    The observation follows from the fact that for any pair of vertices $u$ and $v$ in $G$, we have $uv\in E(G^*)$ (or $d_G(u,v)=2$) if and only if either $u,v\in A$ and $\exists$ $w\in B$ such that $u,v\in N_G(w)$  or alternatively, $u,v\in B$ and $\exists$ $w\in A$ such that $u,v\in N_G(w)$. 
\end{proof}
\begin{observation} \label{obs:chordalbip}
    Let $G=(A,B,E)$ be a chordal bipartite graph. Then, $G^*$ is chordal. 
\end{observation}
\begin{proof}
    Suppose not. Since $G^*=G^2[A]\cup G^2[B]$ (by Observation~\ref{obs:bipsquare}), we have that at least one of the subgraphs $G^2[A]$ or $G^2[B]$ is not chordal. Without loss of generality, we can assume that $G^2[A]$ is not chordal (the other case is symmetric).  Let $C_k=(a_0,a_1\ldots,a_k,a_0)$ be an induced cycle in $G^2[A]$, where $k\geq 3$. Since $G$ is bipartite, and $C_k$ is an induced cycle, this is possible only if for each $i\in \{0,1,\ldots,k\}$, there exists a vertex $b_i$ in $G$ such that $N_G(b_i)\cap V(C_k) = \{a_i,a_{i+1}\}$  (mod $k+1$). This further implies that $(a_0,b_0,a_1,b_1,a_2,\ldots,a_k,b_k,a_0)$ is an induced cycle of length at least $2k$ in $G$. Since $k\geq 3$, this contradicts the fact that $G$ is a chordal bipartite graph. Hence, the observation.
\end{proof}

Now we are ready to prove Theorem~\ref{thm:chordalbip}.\\ %We then have the following theorem.} % \dhanya{due to Theorem~\ref{thm:totdom}}

\noindent\textit{Proof of Theorem~\ref{thm:chordalbip}}
    Clearly, $G$ is $\mathcal{H}$-free. By Observation~\ref{obs:chordalbip}, we have that $G^*$ is chordal, and therefore, perfect. Since any induced subgraph $H$ of $G$ is also chordal bipartite, it follows from Corollary~\ref{corr:suffcdperfect} that $G$ is  $cd$-perfect. Further, as $G$ is triangle-free, by Proposition~\ref{pro:triangle-free}, we have that $\XCD(G)=$\raisebox{2pt}{$\gamma$}$_t(G)$. Since 
 $\XCD(G)=\SC(G)$, the latter statement of the theorem is now immediate from Theorem~\ref{thm:totdom}. \qed


\begin{remark}
    Let $\mathcal{C}=\{G=(A,B,E): G$ is an $\mathcal{H}$-free bipartite graph where $G^2[A]$ and $G^2[B]$ are perfect\}. As in the proof of the above theorem, by Observation~\ref{obs:bipsquare} and Corollary~\ref{corr:suffcdperfect}, we can observe that {\sc Total Domination}, \CDC, and \SCP\  are all equivalent problems on $\mathcal{C}$. Moreover, since the problem \CC\ is polynomial-time 
 solvable for perfect graphs~\cite{GrotSchrEllipsoid81}, by Theorem~\ref{thm:suff_cdclique}, we then have that all the problems {\sc Total Domination}, \CDC, and \SCP\ can be solved in polynomial time for graphs in $\mathcal{C}$.
\end{remark}