 \begin{construction}
    \label{cons:d-regular odd}
     Let $G$ be a triangle-free $(d-2)$-regular graph, for any odd integer $d\geq 5$. 
     We construct a graph $G_c$ from $G$ %using a gadget $W$ 
    in the following way: 
%    \vspace{-0.25cm}
    \begin{itemize}
        \item First we construct a gadget $W$ as follows:  introduce a star graph $K_{1,d-1}$, which we denote by $S$, having the vertex $a$ as the root vertex. Note that we also call `$a$' as the root vertex of $W$.
        Further introduce $2(d-1)^2$ vertices which induces $d-1$ disjoint copies of complete bipartite graphs $K_{d-1,d-1}$, namely $C_{1},  C_{2}\ldots,C_{d-1}$, respectively.      
        The adjacency between these complete bipartite graphs $C_i=(A_i,B_i)$, for $1\leq i\leq (d-1)$, and the star graph $S$ is in such a way that the vertices of $A_i$ of the complete bipartite graph $C_i$ is adjacent to the leaf vertex, say, $v_i$  of $S$.
        Furthermore, for each odd integer $i\leq (d-2)$, each vertex in the partite set $B_i$ of $C_i$ is adjacent to exactly one vertex of the partite set $B_{i+1}$ of $C_{i+1}$ (for instance refer  Figure~\ref{fig:d-regular_gadget odd}, when $d=5$). Now, by $H_i'$, we denote the subgraph of $W$ induced by the vertices in the two complete bipartite graphs $C_i$ and $C_{i+1}$, for an odd integer $i\leq d-2$.  Then, by $H_i$, we denote the subgraph of $W$ induced by the vertices in $H_i'$ and the two leaves $v_i$ and $v_{i+1}$ of the star graph $S$ which are adjacent to the partitions $A_i$ and  $A_{i+1}$ of $C_i$ and $C_{i+1}$, respectively. Note that $H_i'=H_i-\{v_i,v_{i+1}\}$. The gadget $W$ contains $(2d^2-3d+2)$ vertices.  
     
     \item Now the graph $G_c$ is constructed from $G$ using $W$ in such a way that for each vertex $u$ in $G$, introduce two copies of $W$ such that root vertices of both the copies of $W$ is attached to $u$. 
     \end{itemize}
     Note that  $G_c-G$ contains $2n(2d^2-3d+2)$ vertices, where $n$ is the number of vertices in $G$.
\end{construction}

An example of the gadget $W$ in Construction~\ref{cons:d-regular odd} is shown in Figure~\ref{fig:d-regular_gadget odd}. We have the following observation for the gadget $W$ of Construction~\ref{cons:d-regular odd}. 

% Figure environment removed

\begin{observation}
    \label{obs:Hcoloring_regular_odd} 
     Let $W$ be a gadget, and  
    $H_i$ as well as $H_i'$ be the subgraphs of $W$ as defined in Construction~\ref{cons:d-regular odd}. 
     In any $cd$-coloring of $H_i$, the vertices in $H_i'$ require at least 4 colors. 
     Moreover, there exists a $cd$-coloring of $W$ using exactly $2(d-1)$ colors.
\end{observation}

\begin{proof}
    The graph induced by $H_i'$ contains a $C_6$ (for instance: induced by the vertices $d1,e1,g4,f4,g3,$ $e2$ as shown in Figure~\ref{fig:d-regular_gadget odd}). 
    It is clear that \XCD($C_6$)=4. 
    Now consider the subgraph  $H_i'$. It is obtained by 
    the introduction of new vertices adjacent to some vertices of the $C_6$.   Since none of the $3K_1$s present in the $C_6$ is dominated by any of the remaining vertices in $H_i'$, by Observation~\ref{obs:cdcolor_reduce}, it is clear that the number of colors needed for any $cd$-coloring of $H_i'$ is at least 4. Further, note that all the independent sets in $H_i'$, which are dominated by the leaf vertices $v_i$ or $v_{i+1}$ of the star graph $S$ (for instance the vertices $b$ and $c$ shown in Figure~\ref{fig:d-regular_gadget odd}), are already dominated by some vertex in $H_i'$. Therefore, again, by Observation~\ref{obs:cdcolor_reduce}, it is clear that the number of colors needed for $V(H_i')$ in any $cd$-coloring of $H_i$ is at least 4. 
    Since the root vertex $a$ of the gadget is not adjacent to any of the vertices in $H_i'$, the presence of $a$ will not reduce the number of colors needed for $cd$-coloring of the subgraph $H_i'$ of the gadget $W$. Hence, the number of colors needed for $cd$-coloring of $H_i$ is at least $4$. Therefore, the number of colors needed for $cd$-coloring of $W$ is at least $2(d-1)$, as there are $(d-1)/2$ copies of $H_i$ is present in $W$. 
    Now it is easy to see that a 4-$cd$-coloring of a $C_6$ in each $H_i'$ can be extended to a $2(d-1)$-$cd$-coloring of the gadget $W$ (by using the leaves of star graph $S$, and one of the vertices from the partition $A_i$ of each $C_i$, for $1\leq i\leq d-1$, as the dominating vertices of the color classes). 
\end{proof}

The following lemma is based on the Construction~\ref{cons:d-regular odd}.


\begin{lemma}
    \label{lem:d-regular odd}
    For each odd integer $d\geq 5$, let $G$ be a triangle-free ($d-2$)-regular graph, having $n$ vertices. Then $\XCD(G)=k$ if and only if $\XCD(G_c)=k+4n(d-1)$. 
\end{lemma}

\begin{proof}
    Let $\XCD(G)=k$. We claim that $\XCD(G_c)=k+4n(d-1)$.  Recall that the graph $G_c$ contains $2n$ gadgets as there are $n$ vertices in $G$ each having degree $d-2$. By Observation~\ref{obs:Hcoloring_regular_odd}, it is clear that any $cd$-coloring of these gadgets needs a total of at least $4n(d-1)$ colors. Since \XCD($G$)=$k$, we then have \XCD($G_c$)=$k+4n(d-1)$. 

    Now, for the reverse direction, let $\XCD(G_c)=m$. Then we claim that $\XCD(G)=k$, where $k=m-4n(d-1)$. 
    Note that for each gadget $W$, the subgraph $W- \{a\}$ is disjoint from $G$, as only $a$ in $W$ is adjacent to exactly one vertex  $u$ in $G$. Hence, the color class dominated by $a$ can include only one vertex $u$ in $G$. 
     By Observation~\ref{obs:Hcoloring_regular_odd}, in any $cd$-coloring of $G_c$, the vertex set of each copy of $H'$ of any gadget $W$ needs at least 4 colors, and none of these colors can be reused for coloring the vertices in $V(G_c)\setminus V(W)$. Again, by Observation~\ref{obs:Hcoloring_regular_odd}, the same 4 colors used to color the vertices in $H_i'$ can be reused to color the entire vertices of $H_i$. Hence, the 2($d-1$) colors used for $(d-1)/2$ copies of $H_i$ in $W$ can be reused to color the entire vertices of $W$. Therefore, we now have a $cd$-coloring of $G_c$ using $m$ colors such that no colors used for the vertices in gadgets are used to color any vertex in $G$. Thus, $\XCD(G)=m-4n(d-1)=k$, as there are $2n$ copies of $W$s in $G_c$.  
\end{proof}