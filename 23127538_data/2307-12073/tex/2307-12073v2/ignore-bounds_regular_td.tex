\subsection{Relationship between $\gamma_t(G)$ and \XCD($G$)}
\label{subsec:bound_regular_td}
We now focus on the parameters, $\XCD$ and $\gamma_t$.
Note that the difference between $\XCD$ and $\gamma_t$ can be arbitrarily high even for regular graphs in general.
For example, for any integer $r\geq 1$, consider the complete graph $K_r$ on $r$ vertices. We have, $\XCD(K_r)=r$, where as $\gamma_t(K_r)=2$. In the following theorem, we prove a stronger statement: 

\begin{theorem}\label{thm:xcd_tds_bound}
Let $d\geq 3$ be any fixed integer. Then there exist an integer $n_0(d)$ such that for all even integer $n\geq n_0(d)$ there exists a family of connected $d$-regular graphs $\{G_{n,d}\}$ such that $\XCD(G_{n,d})-\gamma_t(G_{n,d})\geq n$.
\end{theorem}

 To prove this, we make use of the following construction. 
 \begin{construction}
 \label{cons:bound_regular}
    Let $n_0(d)=4(d-3)$, for $d\geq 3$. Given an even integer $n\geq n_0(d)$, first, we construct an intermediate graph $G_n$ as follows: $V(G_n)=V(C_n)$, where  $C_n$ denote an induced cycle of length $n$. We repeat the following procedure $d-3$ times. For $i\in \{1,2,\ldots, d-3\}$, in the $i^{th}$ iteration, consider the pair of vertices, $u_i$ and $u_{\frac{n}{2}+i-1}$ (that lies diagonally opposite in the cycle $C_n$), and make them adjacent in $G_n$. Let $\{v_1,v_2,\ldots,v_{\frac{n-2}{2}}\}$ denote the vertices in $C_n$ between $u_i$ and $u_{\frac{n}{2}+i-1}$ listed in clockwise direction, and let $\{v_1',v_2',\ldots,v_{\frac{n-2}{2}}'\}$ denote the vertices in $C_n$ between $u_1$ and $u_{\frac{n}{2}+i-1}$ listed in anti-clockwise direction. For each $j\in \{1,2,\ldots, \frac{n-2}{2}\}$, make the pair of vertices $v_i$ and $v_i'$ adjacent in $G_n$.  
It is not difficult to see that the graph $G_n$ obtained after applying the above procedure $d-3$ times to $C_n$, is a $(d-1)$-regular graph. Note that when $d=3$, we let $G_n=C_n$, as $C_n$ itself is a $d-1=2$-regular graph. Now, we construct a $d$-regular graph $G_{n,d}$ from $G_n$ by considering the following two cases: 

\medskip

  \noindent\textbf{Case 1: $d$ is odd.}
  Here, we define a gadget $W$ as follows: Introduce a  vertex $r$, which we call as the root vertex of the gadget $W$. 
  Now introduce $\lfloor\frac{d}{2}\rfloor$ copies of $K_{d+1}$, denoted as $X_1,X_2,\ldots, X_{\lfloor\frac{d}{2}\rfloor}$, and delete one edge, say $(u_i,v_i)$ from each copy $X_i$. Further, make the vertices $u_i$ and $v_i$ in each $X_i$, adjacent to the root vertex $r$ of $W$. Clearly, each vertex other than the root vertex in the gadget $W$ now has degree $d$.
  Now, for each vertex $v$ in $G_n$, introduce a gadget $W$ such that $v$ is adjacent to the vertex $r$ of $W$.
  The resultant graph is called $G_{n,d}$, where $G_{n,d} - G_n$ contains $n(\lfloor\frac{d}{2}\rfloor (d+1) +1)$ vertices. Clearly,  $G_{n,d}$ is a $d$-regular graph.
  
\smallskip

  See Figure~\ref{fig:bound_regular_odd} for an illustration of the graph $G_{n,d}$, when $n=10$ and $d=5$.

  % Figure environment removed

\medskip

  \noindent\textbf{Case 2: $d$ is even.}
  Here, we define a gadget $W$ as follows: Introduce two vertices $r_1$ and $r_2$ such that $r_1$ is adjacent to $r_2$.  Now for each vertex $r_j$, for $j\in \{1,2\}$, introduce $\frac{d-2}{2}$ copies of $K_{d+1}$, denoted as $X_{j,1},X_{j,2},\ldots, X_{j,\frac{d-2}{2}}$, and delete one edge $(u_i,v_i)$ from each $X_{j,i}$.  Further, make the vertices $u_{j,i}$ and $v_{j,i}$ from each copy $X_{j,i}$, adjacent to $r_i$. Now consider an arbitrary pair-wise ordering $(v_1,v_2), (v_3,v_4)\ldots,(v_{n-1}, v_n)$ of vertices in $G_n$. Note that such a pair-wise ordering exists, since $|V(G_n)|=n$ is even.
  For each odd $k\in \{1,2,\ldots,n-1\}$, corresponding to each pair of vertices $(v_k,v_{k+1})$ in this ordering, a gadget $W$ is introduced in such a way that the vertex $r_1$ of $W$
  is adjacent to $v_k$ and $r_2$ of $W$ is adjacent to $v_{k+1}$. 
  The resultant graph is called $G_{n,d}$, where $G_{n,d} - G_n$ contains {$\frac{n}{2}((d-2)(d+1) +1))$ vertices}. It is not difficult to see that  $G_{n,d}$ is a $d$-regular graph. 
  
 \end{construction}
 
\begin{proof}
     As observed before, the resultant graph $G_{n,d}$ is $d$-regular in both cases. Note that during the construction of $G_{n,d}$ in both cases, the graph $C_n$ contributes degree 2 to each vertex. To construct an intermediate graph $G_n$, with degree $d-1$, from $C_n$, we need to perform the operation $d-3$ times (finding the diagonals and adding edges between symmetrical pairs), we need at least $4(d-3)$ vertices. 
    This justifies the choice of $n_0(d)$.
    
\smallskip

    \noindent\textit{Case 1: $d$ is odd.} Let $G_{n,d}$ be the graph obtained from Construction~\ref{cons:bound_regular} for case 1. For each set $X_i$ in  a gadget $W$ of $G_{n,d}$, the vertices in $X_i\setminus \{u_i,v_i\}$ require $d-1$ colors for $cd$-coloring (as it contains $K_{d-1}$ as an induced subgraph). Note that we can give one of these colors for $r$ also. Similarly, for every $i$, the vertices $u_i,v_i$ together induce a star graph in $G_{n,d}$ with $r$ as the root of the star. Hence, we can give the same color for them as these vertices are the leaves of a star. Note that this color can be given to the vertex $u$, which is adjacent to $r$ in $G_n$. It is easy to see that this is an optimal $cd$-coloring of $G_{n,d}$. Thus, \XCD($G_{n,d}$)=$n(\lfloor(\frac{d}{2})\rfloor(d-1)+1)$.

    It is easy to see that each gadget $W$ of $G_{n,d}$ can be totally dominated by the root $r$ of the star graph in $W$, and one of the vertices $u_i$ or $v_i$ from each set $X_i$. Thus $\gamma_t(G_{n,d})=n(\lfloor(\frac{d}{2})\rfloor+1)$.
   Thus we have, \XCD($G_{n,d}$)$=n(\lfloor(\frac{d}{2
    })\rfloor(d-1)+1)$, and $\gamma_t(G_{n,d})=n(\lfloor(\frac{d}{2})\rfloor+1)$.
    This implies that $\XCD(G_{n,d})-\gamma_t(G_{n,d})\geq n$.\\

     \noindent\textit{Case 2: $d$ is even.} Let $G_{n,d}$ be the graph obtained from Construction~\ref{cons:bound_regular} for case 2. For each set $X_i$ in  the subgraph $W_i$, for $i\in \{1,2\}$ of $W$ in $G_{n,d}$, the set of vertices $X_{i,j}\setminus \{u_{i,j},v_{i,j}\}$ require $d-1$ colors for $cd$-coloring. Note that we can give one of these colors used in $X_{i,j}$, which is attached to $r_i$  for coloring  $r_i$  also. Similarly, for every $j$ , the vertices $u_{ij},v_{ij}$  induces a star graph $S_i$ in $W$ with $r_i$  as the root of $S_i$. Hence we need two colors $c_1$ and $c_2$, for coloring the leaf vertices of the stars $S_1$ and $S_2$, respectively. Note that these colors can be given to the vertices $v_i$ and $v_j$ in $G_n$, which is adjacent to $r_1$ and $r_2$, respectively, in $G_{n,k}$. It is easy to see that this is an optimal $cd$-coloring of $G_{n,d}$. Thus, \XCD($G_{n,d}$)=$n(\frac{d-2}{2} (d-1)+2)$.

     Note that each gadget $W$ of $G_{n,d}$ can be totally dominated by the root vertices $r_1$, $r_2$ of star graph $S_1$ and $S_2$, and one of the vertices $u_{i,j}$ or $v_{i,j}$ from each set $X_{i,j}$. Thus $\gamma_t(G_{n,d})=n(\frac{d-2}{2}+2)$. Thus we have, \XCD($G_{n,d}$)$=n(\frac{d-2}{2}(d-1)+2)$, and $\gamma_t(G_{n,d})=n(\frac{d-2}{2}+2)$.
    This implies that $\XCD(G_{n,d})-\gamma_t(G_{n,d})\geq n$.
\end{proof}