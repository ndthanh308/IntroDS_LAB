\subsubsection{\textbf{Hardness implications:}}
In contrast to the previous section, here, we see an application of Proposition~\ref{pro:omega_s} and Theorem~\ref{thm:suff_cdclique} in deriving hardness results for \CDC\ and \SCP. 

\subsubsection*{\textbf{$C_6$-free bipartite graphs}}

%\label{subsec:c6-free}
Here, we prove the hardness of the problems \CDC\ and \SCP\ for $C_6$-free bipartite graphs by proposing a polynomial-time reduction from the problems \CC\  and \ISP\ for diamond-free graphs. We begin by stating one of the main results in this section. 
%The proof of this result is the main content of this subsection.

Note that both {\sc Clique Cover} and {\sc Independent Set} are known to be NP-hard for diamond-free graphs~\cite{KralKraComplexColGr01,PoljakStabSet74}. Recall that, a \textit{diamond} is a graph obtained by deleting an edge from $K_4$ (complete graph on 4 vertices). First, we note the following observation for diamond-free graphs~\cite{ChiarStrongCli21}.

\begin{observation}[\cite{ChiarStrongCli21}]\label{obs:diamondfree}
Any diamond-free graph with $m$ edges can have at most $m$ maximal cliques.
\end{observation}
Let $G$ be a diamond-free graph and let $\mathcal{C}=\{C_1,C_2,\ldots,C_l\}$ denote the collection of all maximal cliques in $G$. By Observation~\ref{obs:diamondfree}, we have that $l\leq~ \mid~E(G)\mid$.  From $G$, we now construct a bipartite graph $B_G=(A,B,E)$ in polynomial-time as follows:
\begin{construction}  \label{cons:c6-freebipartite}
     Define $A=V(G)\cup\{u\}$ and $B=\mathcal{C}=\{C_1,C_2,\ldots,C_l\}$ (each vertex in the partite set $B$ represents a maximal clique in $G$). For a pair of vertices, $a\in A\setminus \{u\}$ and $C_j\in B$ (where $j\in \{1,2,\ldots,l\}$), we make the vertices $a$ and $C_j$ adjacent in $B_G$ if and only if $a\in C_j\subseteq V(G)$. In addition, we make the vertex $u\in A$ adjacent to all the vertices in $B$. i.e. $E(B_G)=\{aC_j:a\in A\setminus \{u\}, C_j\in \mathcal{C}=B$ with $a\in C_j\subseteq V(G)\}\cup \{uC_j:C_j\in \mathcal{C}=B\}$. Clearly, $B_G$ is a bipartite graph.
\end{construction}

% Example Shirish 


 % Figure environment removed


  An example of  Construction~\ref{cons:c6-freebipartite} is shown in Figure~\ref{fig:diamond-free to bipartite}. We then have the following lemmas, which are crucial for the reduction. 

\begin{lemma}\label{lem:isomprphic}
Let $G$ be a diamond-free graph and $B_G=(A,B,E)$ the corresponding bipartite graph (as in Construction~\ref{cons:c6-freebipartite}). Then, $B_G^2[A\setminus \{u\}]\cong G$ and $B_G^2[B]$ is a clique.
\end{lemma}
\begin{proof}
Let $H=B_G^2[A\setminus \{u\}]$. Clearly, by the definition of $B_G$, every vertex in $V(H)$ corresponds to a vertex in $V(G)$. Let $x,y\in V(H)$. Then, as $B_G$ is a bipartite graph, and $x,y\in V(H)\subseteq A\setminus \{u\}$, we have $xy\in E(H)\iff \exists C_j\in B$ for some $j\in \{1,2,\ldots,l\}$ such that $x,y\in N_{B_G}(C_j)$ (i.e. both the vertices $x$ and $y$ belong to a same maximal clique $C_j$ in $G$) $\iff xy\in E(G)$. This proves that $B_G^2[A\setminus \{u\}]\cong G$. Further, it is easy to see that $B_G^2[B]$ is a clique, since $B\subseteq N_{B_G}(u)$.
\end{proof}

\begin{lemma}\label{lem:diamondC_6}
    Let $G$ be a diamond-free graph and $B_G=(A,B,E)$ be the corresponding bipartite graph obtained from Construction~\ref{cons:c6-freebipartite}. Then, $B_G$ is a $C_6$-free bipartite graph.
\end{lemma}
\begin{proof}
    Suppose not. Let $S=(a_1,C_1,a_2,C_2,a_3,C_3,a_1)$ be an induced $C_6$ in $B_G$, where $\{a_1,a_2,a_3\}\subseteq A$ and $\{C_1,C_2,C_3\}\subseteq B$.  Clearly, $\{a_1,a_2,a_3\}$ induces a triangle in $B_G^2[A]$. Since each of the vertex $a_i$ has a non-neighbor in $V(S)\cap B$, we have $u\neq a_i$ for any $i\in \{1,2,3\}$. This implies that $\{a_1,a_2,a_3\}$ induces a triangle in $B_G^2[A\setminus \{u\}]$, and thus a triangle in $G$ by Lemma~\ref{lem:isomprphic}. Now, consider the vertex $C_1$ in the cycle $S$. We have $N_{B_G}(C_1)\cap V(S) =\{a_1,a_2\}$. Clearly, $\{a_1,a_2\}$ is not a maximal clique in $G$, since $\{a_1,a_2,a_3\}$ induces a triangle in $G$. As $a_3\notin N_{B_G}(C_1)$, we have by the definition of $B_G$ that $a_3\notin C_1$ in $G$. Since $C_1$ is a maximal clique in $G$ containing the set $\{a_1,a_2\}$, but not the vertex $a_3$, there exists a vertex $a_4\in V(G)\subseteq A$ such that $a_4\in C_1$ in $G$ and $a_3a_4\notin E(G)$. This implies that $\{a_1,a_2,a_3,a_4\}$ induces a diamond in $G$, a contradiction to the fact that $G$ is diamond-free.
\end{proof}
We are now ready to prove Theorem~\ref{thm:C_6free hard}.\\


\noindent\textit{Proof of Theorem~\ref{thm:C_6free hard}}.
    Let $G$ be a diamond-free graph, and let $B_G$ be the corresponding bipartite graph obtained from Construction~\ref{cons:c6-freebipartite}. Since $B_G$ is $C_6$-free (by Lemma~\ref{lem:diamondC_6}) and triangle-free, we can therefore conclude that $B_G$ is $\mathcal{H}$-free. Therefore, by Theorem~\ref{thm:suff_cdclique}, we have that $\XCD(B_G)=k(B_G^*)$. Since $B_G^*= B_G^2[A]\uplus B_G^2[B]$, we then have $\XCD(B_G)=k(B_G^*) = k(B_G^2[A])+k(B_G^2[B])$ = $k(B_G^2[A\setminus \{u\}])+k(B_G^2[B])$ (again, the last equality is due to the fact that, in the graph $B_G^2[A]$, the vertex $u$ is adjacent to very vertex in $A\setminus \{u\}$). This implies that $\XCD(B_G)=k(B_G^*) = k(G)+1$ (by Lemma~\ref{lem:isomprphic}). Recall that \CC\ is NP-hard for diamond-free graphs. This implies that \CDC\ is NP-hard for $C_6$-free bipartite graphs.

    
    By Proposition~\ref{pro:omega_s} applied to the $C_6$-bipartite graph $B_G$, we have that $\SC(B_G)=\alpha(B_G^*)$, where $B_G^*=B_G^2- B_G$ = $B_G^2[A]\uplus B_G^2[B]$. Therefore,  we have $\SC(B_G)=\alpha(B_G^*) = \alpha(B_G^2[A])+\alpha(B_G^2[B])$ = $\alpha(B_G^2[A\setminus \{u\}])+\alpha(B_G^2[B])$ (the last equality is due to the fact that, in the graph $B_G^2[A]$, the vertex $u$ is adjacent to every vertex in $A\setminus \{u\}$). This implies that $\SC(B_G)=\alpha(B_G^*) = \alpha(G)+1$ (by Lemma~\ref{lem:isomprphic}). %Note that $B_G$ is a $C_6$-free bipartite graph (by Lemma~\ref{lem:diamondC_6}).
    Recall that \ISP\ is NP-hard for diamond-free graphs. This implies that \SCP\ is NP-hard for $C_6$-free bipartite graphs. 
    Hence, the theorem.\qed

\subsection{\large{\textbf{Some necessary conditions for $cd$-perfectness}}}

Let $C_n$ denote an induced cycle of length $n$ and $\bar{C_n}$ its complement. By the Strong Perfect Graph theorem, we have that $C_{n}$ or $\bar{C_{n}}$ is perfect if and only if $n=2k$ for some $k\geq 2$. In an upcoming theorem, we give an \textit{almost similar} necessary condition for $cd$-perfect graphs. The following observation for the cycles is noted in~\cite{ShaluSandhyaLowBound17}.
\begin{observation}[\cite{ShaluSandhyaLowBound17}]\label{obs:holes}
 For $n\geq 4$, we have  $\XCD (C_n)=\SC(C_n)$ if and only if $n=4k$ for some integer $k\geq 1$.
\end{observation}
We now have a similar observation for the complements of cycles.
%As in perfect graphs, we note the following necessary condition for $cd$-perfect graphs.
\begin{observation}\label{obs:antiholes}
For $n\geq 5$, we have  $\XCD (\bar{C_n})=\SC(\bar{C_n})$ if and only if $n=2k$ for some integer $k\geq 2$.
\end{observation}
\begin{proof}
First, note that for each $n\geq 5$,  $\bar{C_n}$ is $\mathcal{H}$-free (since $\alpha(\bar{C_n})\leq 2$, for each $n\geq 5$, but $\alpha(H)=3$ for each $H\in \mathcal{H}$). Also, for each $n\geq 5$, it is easy to see that $\bar{C_n}^*=C_n$ (since $\bar{C_n}^2$ is a clique on $n$ vertices). Therefore, by Theorem~\ref{thm:suff_cdclique} and Proposition~\ref{pro:omega_s}, we have, $\XCD(\bar{C_n})=k(\bar{C_n}^*)=k(C_n)=\lceil \frac{n}{2}\rceil$ and $\SC(\bar{C_n})=\alpha(\bar{C_n}^*)=\alpha(C_n)=\lfloor \frac{n}{2}\rfloor$. Therefore, we can conclude that $\XCD (\bar{C_n})=\SC(\bar{C_n})$ if and only if $n=2k$ for some integer $k\geq 2$.
\end{proof}
As noted earlier, in the following theorem, we summarize some necessary conditions for a graph $G$ to be $cd$-perfect. 
%Observe that these conditions are \textit{almost} consistent with the necessary conditions for a graph to be perfect. 
The proof of the theorem is immediate from Observations~\ref{obs:holes} and~\ref{obs:antiholes}.

\begin{theorem}\label{thm:necessary}
If a graph $G$ is $cd$-perfect, then $G$ is $C_n$-free for each $n\geq 4$ with $n\neq 4k$, and $\bar{C_n}$-free for each $n\geq 5$ with $n\neq 2k$, where $k$ is a positive integer.
\end{theorem}