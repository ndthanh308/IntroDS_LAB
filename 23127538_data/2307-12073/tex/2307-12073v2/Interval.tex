\section{Separated-cluster problem in Interval graphs}
\label{sec:interval}In this section, we settle an open problem proposed by Shalu et al.~\cite{ShaluSandhyaLowBound17}, by proving that \SCP\ is polynomial-time solvable for the class of interval graphs (Theorem~\ref{thm:sepiterval}). We achieve this by showing a polynomial-time reduction of \SCP\ problem on interval graphs to the maximum weighted independent set problem on cocomparability graphs (which can be solved in time linear to the size of the input graph~\cite{KohLalMaxWtIndSetAlg16}). An undirected graph is called \textit{comparability graph}, if it is transitively orientable, i.e., its edges can be directed such that if $a\rightarrow b$ and $b\rightarrow c$ are directed edges, then $a\rightarrow c$ is a directed edge. \textit{Cocomparability graphs} are complements of comparability graphs. Throughout the section, we use an alternative definition of \textit{separated-cluster}. i.e. for a graph $G$, we say that a family of disjoint cliques in $G$, forms a \textit{separated-cluster}  $\mathcal{S}=\{S_1,S_2,\ldots,S_k\}$, if any pair of cliques in $\mathcal{S}$ are mutually non-adjacent and no two vertices belonging to distinct cliques in $\mathcal{S}$ have a common neighbor in $G$. Formally, for $i,j\in \{1,2,\ldots,k\}$ with $i\neq j$, if $x\in S_i$ and $y\in S_j$, then $xy\notin E(G)$, and there does not exist a vertex $z\in V(G)\setminus (S_i\cup S_j)$ such that $x,y\in N_G(z)$.  \\

\noindent\textbf{Sketch of the reduction: }
Let $G$ be an interval graph with an interval representation, $\{I_v\}_{v\in V(G)}$, where $n=~ \mid V(G) \mid $. 
Without loss of generality, we can assume that \textit{all the end-points of the intervals in $\{I_v\}_{v\in V(G)}$ are distinct}. 
For a vertex $v\in V(G)$, for the sake of convenience, we denote by $l(v)$ and $r(v)$, the \textit{left} and \textit{right end-points} of the interval $I_v$, respectively. Our basic idea for the reduction is as follows. Recall that in \SCP\ problem, we need to find a collection of cliques in $G$ satisfying certain conditions. First, note that the cliques belonging to a separated-cluster are not necessarily maximal in $G$. Therefore, our primary goal is to find a polynomial (in $n$) sized collection of cliques in $G$ that contains all the cliques that may possibly belong to {some} maximum cardinality separated-cluster in $G$. To achieve this, we heavily use the interval representation of $G$, and extend the collection of maximal cliques in $G$ to a \textit{``special"} collection of cliques denoted as $\hat{\mathcal{C}}$. Each member of $\hat{\mathcal{C}}$ is obtained by \textit{``pruning"} a maximal clique in $G$ with respect to a fixed interval representation of $G$. Then, from the input graph $G$, we construct a weighted conflict graph $G_c$, having the collection of cliques $\hat{\mathcal{C}}$ as the vertex set, and the vertices are assigned a weight same as the cardinalities of the corresponding sets (when they are considered as cliques in $G$). The graph $G_c$ is referred to as a conflict graph because, each vertex in $G_c$ represents a clique in $G$, and the edges between two vertices are added to $G_c$ if the corresponding cliques can not appear together in any separated-cluster.  We then prove that the weighted conflict graph $G_c$ is a \textit{cocomparability} graph and \textit{a maximum cardinality separated-cluster in the interval graph $G$ is equivalent to the maximum cardinality independent set in $G_c$}. 

% Example
% Figure environment removed
% Shirish end Example
% Figure environment removed

Note that most of the definitions in this section are based on a fixed interval representation, $\{I_v\}_{v\in V(G)}$ of $G$. Let  $\{C_1,C_2,\ldots, C_x\}$ be the collection of all \textit{maximal cliques} in $G$ (refer Figure~\ref{fig:separated_cluster_interval_graph}) and let $t_i=~ \mid C_i \mid $, for $i\in \{1,2,\ldots,x\}$. Note that,  as $G$ is an interval graph, we have  $x\leq n$~\cite{GolumAlgoGrTh04}.  Now to define the desired collection of cliques, we first introduce the following. We enumerate the vertices belonging to $C_i$ in two different ways (namely, in the increasing order of the left and right end-points of their intervals). i.e. $C_i^{l\_list}=(v^1,v^2,\ldots, v^{t_i})$ is such that $l(v^1)< l(v^2)<\cdots < l(v^{t_i})$, and $C_i^{r\_list}=(u^1,u^2,\ldots, u^{t_i})$ is such that $r(u^1)< r(u^2)<\cdots < r(u^{t_i})$. For an example, refer Table~\ref{tab:lists} that corresponds to the graph in Figure~\ref{fig:separated_cluster_interval_graph} and \ref{fig:Interval representation of an interval graph1}. 

%\vspace{-3.5cm}
\begin{table}[ht]
    \centering
    \tiny{
    \begin{adjustbox}{width=0.9\textwidth}
\begin {tabular} {|l|l|l|}
\hline
  \multicolumn{1}{|c|}{$C_i$} & \multicolumn{1}{c|}{$C_i^{l\_list}$} & \multicolumn{1}{c|}{$C_i^{r\_list}$} \\
\hline
 \makecell[tl]{$C_1 = \{ v_1, v_3\} $\\ $\lvert C_1 \rvert = 2$} &  $C_1^{l\_list} = (v_1, v_3) $ &$C_1^{r\_list} = (v_1, v_3) $ \\
\hline
 \makecell[tl]{$C_2 = \{ v_2, v_3\} $ \\$\lvert C_2 \rvert = 2$} &  $C_2^{l\_list} = (v_3, v_2)$ & $C_2^{r\_list} = (v_2, v_3)$ \\
\hline
 \makecell[tl]{ $C_3 = \{ v_3, v_4, v_5\} $ \\$\lvert C_3 \rvert = 3$}     &   $C_3^{l\_list} = (v_3, v_5, v_4)$ & $C_3^{r\_list} = (v_3, v_4, v_5)$ \\
\hline
 \makecell[tl]{ $C_4 = \{ v_5, v_6\} $ \\$\lvert C_4 \rvert = 2$} & $C_4^{l\_list} = ( v_5, v_6)$ &$C_4^{r\_list} = (v_5, v_6)$ \\
\hline
 \makecell[tl]{ $C_5 = \{ v_6, v_7, v_8, v_9\} $ \\ $\lvert C_5 \rvert = 4$}   &   $C_5^{l\_list} = (v_6, v_9, v_7, v_8)$ & $C_5^{r\_list} = (v_6, v_8, v_7, v_9)$ \\
\hline
 \makecell[tl]{ $C_6 = \{ v_9, v_{10}\} $ \\$\lvert C_6 \rvert = 2$} &  $C_6^{l\_list} = (v_9, v_{10})$&$C_6^{r\_list} = (v_9, v_{10})$ \\
\hline
\end {tabular}

%\bigskip
\end{adjustbox}}
 \caption{The enumerations $C_i^{l\_list}$ and $C_i^{r\_list}$ of the maximal cliques $C_i$ of $G$ shown in Figure~\ref{fig:separated_cluster_interval_graph}}
    \label{tab:lists}
\end{table}
    
    Let $i\in \{1,2,\ldots,x\}$ and $p,q\in \{1,2,\ldots, \mid C_i \mid \}$. 
    We now define $C_i^l(v^p)$ (respectively $C_i^r(u^q)$) as the set of vertices in $C_i$ whose left (respectively, right) end-points of the intervals are greater (respectively less) than or equal to the left (respectively right) end-points interval of  $v^p$ (respectively $u^q$). i.e., $C_i^l(v^p)=\{u\in C_i: l(u) \geq l(v^p)\}$ and $C_i^r(u^q)=\{u\in C_i:  r(u) \leq r(u^q)\}$ (refer Table~\ref{tab:collection}). Clearly, $C_i=C_i^l(v^1)\supseteq C_i^l(v^2)\supseteq \ldots \supseteq C_i^l(v^{t_i})=\{v^1\}$ and $\{u^1\}=C_i^r(u^1)\subseteq C_i^r(u^2)\subseteq \ldots \subseteq C_i^r(u^{t_i})=C_i$. We then define the cliques $C_i^{lr}(v^p,u^q)$ as the set of vertices of $C_i$ whose corresponding intervals are completely contained in the interval $[l(v^p), r(u^q)]$, i.e., $C_i^{lr}(v^p,u^q) = C_i^l(v^p)\cap C_i^r(u^q)$. 
    
   
\begin{definition}[The collection of cliques $\hat{\mathcal{C}}$]
   For $\{i\in \{1,2,\ldots,x\}\}$,  let $\hat{\mathcal{C}_i}=\{C_i^{lr}(v^p,u^q):p,q\in \{1,2,\ldots, \mid C_i \mid \}\}$.  Then the collection of cliques $\hat{\mathcal{C}}=\bigcup_{i\in[x]}\hat{\mathcal{C}_i}$, for $i\in \{1,2,\ldots,x\}$. 
\end{definition}
\begin{table*}[t]
\caption{Coverage improvement compared to \thehuzz{}~\cite{kande2022thehuzz}.}
\resizebox{\textwidth}{!}{%
\begin{tabular}{|c|c|ccc|ccc|ccc|}
\hline
\multirow{2}{*}{Core} & \thehuzz{}~\cite{kande2022thehuzz} & \multicolumn{3}{c|}{PSO} & \multicolumn{3}{c|}{PSO+Reset} & \multicolumn{3}{c|}{PSO+Reset+Seed (\ourtool)} \\ \cline{2-11} 
 & Total & \multicolumn{1}{l|}{Total} & \multicolumn{1}{l|}{Increment} & Speedup & \multicolumn{1}{c|}{Total} & \multicolumn{1}{c|}{Increment} & \multicolumn{1}{l|}{Speedup} & \multicolumn{1}{c|}{Total} & \multicolumn{1}{c|}{Increment} & \multicolumn{1}{l|}{Speedup} \\ \hline
\cva{}~\cite{cva6} & $6214$ & \multicolumn{1}{c|}{$6074$} & \multicolumn{1}{c|}{\cvaCovIncVani $\%$} & \cvaSpdupVani $\times$ & \multicolumn{1}{c|}{$6334$} & \multicolumn{1}{c|}{\cvaCovIncRst $\%$} & \cvaSpdupRst $\times$ & \multicolumn{1}{c|}{$6277$} & \multicolumn{1}{c|}{\cvaCovIncSeed $\%$} & $\mathbf{2.22}\times$ \\ \hline
\boom{}~\cite{boom} & $23086$ & \multicolumn{1}{c|}{$23052$} & \multicolumn{1}{c|}{\boomCovIncVani $\%$} & \boomSpdupVani $\times$ & \multicolumn{1}{c|}{$23088$} & \multicolumn{1}{c|}{\boomCovIncRst $\%$} & \boomSpdupRst $\times$ & \multicolumn{1}{c|}{$23092$} & \multicolumn{1}{c|}{\boomCovIncSeed $\%$} & \boomSpdupSeed $\times$ \\ \hline
\rc{}~\cite{rocket_chip_generator} & $11947$ & \multicolumn{1}{c|}{$11846$} & \multicolumn{1}{c|}{\rcCovIncVani $\%$} & \multicolumn{1}{c|}{\rcSpdupVani $\times$} & \multicolumn{1}{c|}{$11864$} & \multicolumn{1}{c|}{\rcCovIncRst $\%$} & \rcSpdupRst $\times$ & \multicolumn{1}{c|}{$11958$} & \multicolumn{1}{c|}{\rcCovIncSeed $\%$} & \rcSpdupSeed $\times$ \\ \hline
\end{tabular}%
}
\label{tab:cov}
\end{table*}


%%%% old version %%%%
% % Please add the following required packages to your document preamble:
% % \usepackage{multirow}
% % \usepackage{graphicx}
% \begin{table*}[]
% \caption{Coverage improvement compared to \thehuzz{}~\cite{kande2022thehuzz}.}
% \resizebox{\textwidth}{!}{%
% \begin{tabular}{|c|c|ccc|ccc|}
% \hline
% \multirow{2}{*}{Core} & \thehuzz~\cite{kande2022thehuzz} & \multicolumn{3}{c|}{PSOReset} & \multicolumn{3}{c|}{PSORstSeed} \\ \cline{2-8} 
%  & Total & \multicolumn{1}{c|}{Total} & \multicolumn{1}{c|}{Increment} & Speedup & \multicolumn{1}{c|}{Total} & \multicolumn{1}{c|}{Increment} & Speedup \\ \hline
% \cva{}~\cite{cva6} & $6214$ & \multicolumn{1}{c|}{$6334$} & \multicolumn{1}{c|}{\cvaCovIncRst{}\%} & \cvaSpdupRst{} & \multicolumn{1}{c|}{$6277$} & \multicolumn{1}{c|}{\cvaCovIncSeed{}\%} & \cvaSpdupSeed{} \\ \hline
% \rc{}~\cite{rocket_chip_generator} & $11947$ & \multicolumn{1}{c|}{$11864$} & \multicolumn{1}{c|}{\rcCovIncRst{}\%} & \rcSpdupRst & \multicolumn{1}{c|}{$11958$} & \multicolumn{1}{c|}{\rcCovIncSeed{}\%} & \rcSpdupSeed \\ \hline
% \boom{}~\cite{boom} & $23086$ & \multicolumn{1}{c|}{$23088$} & \multicolumn{1}{c|}{\boomCovIncRst{}\%} & \boomSpdupReset & \multicolumn{1}{c|}{$23092$} & \multicolumn{1}{c|}{\boomCovIncSeed{}\%} & \boomSpdupSeed \\ \hline
% \end{tabular}%
% }
% \label{tab:cov}
% \end{table*} 

To illustrate the above definitions, consider the interval graph $G$ in Figure~\ref{fig:separated_cluster_interval_graph} and its interval representation given in Figure~\ref{fig:Interval representation of an interval graph1}. It is easy to see that the collection of maximal cliques in $G$ is $\{C_1, C_2, \ldots, C_6\}$. For each $C_i$, the enumerations of $C_i$, namely, $C_i^{l\_list}$ and $C_i^{r\_list}$ are listed in Table~\ref{tab:lists}. Further, for each $C_i$, where $i\in \{1,2,\ldots,6\}$, in Table~\ref{tab:collection}, we summarize the collection of cliques $C_i^l(v_p)$, $C_i^r(u^q)$, and $C_i^{lr}(v^p,u^q)$. Note that, in the third column of Table~\ref{tab:collection}, we did not include entries corresponding to those pairs $p,q\in \{1,2,\ldots,\lvert C_i \rvert \}$ for which $C_i^{lr}(v^p,u^q)=\emptyset$. Now, for each $i\in \{1,2,\ldots,6\}$, we have $\hat{\mathcal{C}_i}=\{C_i^{lr}(v^p,u^q):p,q\in \{1,2,\ldots, \mid C_i \mid \}\}$; i.e., $\hat{\mathcal{C}}=\bigcup_{i\in[x]} \hat{\mathcal{C}_i}:i\in \{1,2,\ldots,6\}$.\\

We note the following.

\begin{remark} \label{rem:clique_complexity}
Since $C_i \in \hat{\mathcal{C}_i}$ for each $i \in \{1, 2, \ldots, x \}$ (as $C_i = C_i^{lr}(v^1, u^{\lvert C_i \rvert}))$, the collection of cliques  $\hat{\mathcal{C}}$ contains all the maximal cliques in $G$. Also, as the number of ordered pairs of the form $(p,q)$ is at most $n^2$ (as $p,q\in \{1,2,\ldots, \mid C_i \mid \}$, $ \mid C_i \mid \leq n$), we then have that $ \mid \hat{\mathcal{C}} \mid \leq n^3$. 

\end{remark}

\noindent\textbf{Properties of separated-cluster in interval graphs:}
Let $\mathcal{F}$ be a family of disjoint cliques in an interval graph $G$. With respect to an interval representation $\{I_v\}_{v\in V(G)}$ of $G$, we can define a natural ordering $(S_1,S_2,\ldots,S_k)$ of the cliques in $\mathcal{F}$ as follows: \textit{ for any pair $i,j\in \{1,2,\ldots,k\}$, we define $S_i<S_j$ if and only if the Helly region  (the common intersection region of intervals belonging to the same clique) of the clique $S_i$, denoted as $H_{S_i}$, lies entirely to the left of the Helly region of the clique $S_j$}. We will assume this ordering throughout this section. For a clique $S_i\in \mathcal{S}$, where $i\in \{1,2,\ldots,k\}$, let $C_i$ be a maximal clique in $G$ such that $S_i\subseteq C_i$ (possibly, $C_i=S_i$). 

Let $\mathcal{S}=\{S_1,S_2,\ldots,S_k\}$ be a separated-cluster in $G$. Let $C_i$ be a maximal clique in $G$ such that $S_i\subseteq C_i$. Let $j\in \{1,2,\ldots,k\}$, with $j\neq i$. 
We then have the following observations.

\begin{observation}\label{obs:propmaxclique}

(i) $S_j\cap C_i=\emptyset$\\ (ii) no vertex in $C_i$ is adjacent to any vertex in $S_j$.
\end{observation}
%\vspace{-0.75cm}
\begin{proof}
    To prove (i), for the sake of contradiction assume that  $S_j\cap C_i\neq \emptyset$, and let $z\in S_j\cap C_i$. %Since $S_i\cap S_j=\emptyset$ (by the definition of separated-cluster), we have $z\in S_j$ and $z\in C_i\setminus S_i$. 
    Then $z$ is adjacent to all the vertices in $S_i$, as $C_i$ is a clique and $S_i\subseteq C_i$, which contradicts the fact that $\mathcal{S}$ is a separated cluster. 
     (ii) Suppose that there exists a vertex, say $z\in C_i$, such that $z$ is adjacent to some vertex in $S_j$. Clearly, by the definition of $\mathcal{S}$, we have $z\notin S_i$ and by (i), $S_j\cap C_i=\emptyset$. %Thus $Z\notin S_i$. If either $S_i$ and $S_j$ have adjacent vertices, or there exists a vertex $z\in C_i\setminus S_i$ such that 
    This implies that $z$ is a common neighbor of some vertex in $S_i$ and some vertex in $S_j$, where $i\neq j$. This again contradicts the definition of the separated cluster $\mathcal{S}$.
\end{proof}

A vertex $x\in C_i\setminus S_i$ is said to have a \textit{left-conflict} with respect to $\mathcal{S}$, 
if there exists $S_j\in \mathcal{S}$, with $j<i$, and $\exists z\in V(G)\setminus(S_i\cup S_j)$ such that $z$ is a common neighbor of both $x$ and some vertex $y\in S_j$, i.e., $z\in N(x)\cap N(y)$. Similarly, a vertex $x\in C_i\setminus S_i$ is said to have a \textit{right-conflict} with respect to $\mathcal{S}$, 
if there exists $S_j \in \mathcal{S}$ with $j>i$ and $\exists z\in V(G)\setminus(S_i\cup S_j)$ such that $z$ is a common neighbor of both $x$ and some vertex $y\in S_j$, i.e., $z\in N(x)\cap N(y)$. 

 The following observation is due to the definitions of \textit{left-conflict} (respectively \textit{right-conflict}) and $C_i^l(v^p)$ (respectively $C_i^r(u^q)$).
\begin{observation}\label{obs:conflict}
     Let $v$ be a vertex in $C_i\setminus S_i$ which has a \textit{left-conflict}  with respect to $\mathcal{S}$, where $v=v^p$, $p\in \{1,2,\ldots, \mid C_i \mid \}$ with respect to the ordering $C_i^{l\_list}$ of $C_i$. Then every vertex in $C_i\setminus C_i^l(v^p)$  has a \textit{left-conflict} (i.e. $S_i\subseteq C_i^{l}(v^{p+1})$). Similarly, let $v\in C_i\setminus S_i$ be a vertex which has a  \textit{right-conflict} with respect to $\mathcal{S}$, where $v=u^q$, $q\in \{1,2,\ldots, \mid C_i \mid \}$ with respect to the ordering  $C_i^{r\_list}$ of $C_i$. Then every vertex in  $C_i\setminus C_i^r(u^q)$ has a \textit{right-conflict} (i.e. $S_i\subseteq C_i^{r}(u^{q-1})$).
     
\end{observation}

We also note the following.
\begin{observation} \label{obs:noconflict}
%Let $\mathcal{S}=\{S_1,S_2,\ldots,S_k\}$ be a separated-cluster in $G$. For $i\in \{1,2,\ldots,k\}$, let $C_i$ be a maximal clique in $G$ such that $S_i\subseteq C_i$.
If there exists a vertex $x\in C_i\setminus S_i$ such that $x$ has neither a left-conflict nor a right-conflict with respect to $\mathcal{S}$, then $\mathcal{S'}=\{S_1,S_2,\ldots, S_i',\ldots,S_k\}$ is also a separated-cluster in $G$, where $S_i'=S_i\cup\{x\}$.
\end{observation}
\vspace{-0.5cm}
\begin{proof}
    To prove this, it is enough to show that $\mathcal{S'}$ satisfies the conditions in the definition of separated-clusters. Let $j\in \{1,2,\ldots,k\}$ with $j\neq i$. Since $\mathcal{S}$ is a separated-cluster in $G$, by Observation~\ref{obs:propmaxclique}, we have that $C_i\cap S_j=\emptyset$ and no vertex in $C_i$ can be adjacent to any vertex in $S_j$. This implies that $S_i'\cap S_j=\emptyset$  and $S_i'$ is non-adjacent to $S_j$. Suppose that there exists a vertex $y\in S_j$ and $z\in V(G)\setminus (S_i\cup S_j)$ such that $x,y\in N_G(z)$. Since $x\in C_i$, we then have that $x$ has a \textit{left-conflict} with respect to $\mathcal{S}$ if $S_j<S_i$, and $x$ has a \textit{right-conflict} with respect to $\mathcal{S}$ if $S_i<S_j$. In either case, we have a contradiction to the choice of $x$. Now since $\mathcal{S}$ is a separated-cluster, we can therefore conclude that $\mathcal{S'}=\{S_1,S_2,\ldots, S_i',\ldots,S_k\}$ is also a separated-cluster in $G$. 
\end{proof}

In the following lemma, we observe a crucial property of a maximum cardinality separated-cluster.

\begin{lemma} \label{lem:maxsep}
    Let $\mathcal{S}=\{S_1,S_2,\ldots,S_k\}$ be a maximum cardinality
    separated-cluster 
    in $G$. Then for each $S_i$, for $i\in \{1,2,\ldots,k\}$, we have %$C_i\setminus S_i$ has either left-conflict or right-conflict. 
    $S_i\in \hat{\mathcal{C}}$.
\end{lemma}
\vspace{-0.5cm}

\begin{proof}

    For $i\in \{1,2,\ldots,k\}$, let $S_i\in \mathcal{S}$. If $S_i$ is a maximal clique in $G$, then $S_i\in \hat{\mathcal{C}}$ (since $\hat{\mathcal{C}}$ contains all the maximal cliques in $G$). Suppose that $S_i$ is not a maximal clique in $G$. Then, there exists a maximal clique, say $C_i$ in $G$, such that $S_i\subseteq C_i$. Clearly, $C_i\setminus S_i \neq \emptyset$. Since $\mathcal{S}$ is a maximum cardinality separated-cluster in $G$, by Observation~\ref{obs:noconflict}, we have that every vertex in $C_i\setminus S_i$ has either a \textit{left-conflict} or a \textit{right-conflict} or both. 

   If $C_i\setminus S_i$ has at least one vertex with a left-conflict (respectively,  right-conflict), then let $v^p\in C_i\setminus S_i$ (respectively, $u^q \in C_i\setminus S_i$) be the vertex having maximum left-end point (respectively, minimum right-end point) with respect to $C_i^{l\_list}$ (respectively, $C_i^{r\_list}$)  such that $v^p$ has a left-conflict (respectively, $u^q$ has a right-conflict). Then, by Observation~\ref{obs:conflict}, we have $S_i \subseteq C_i^l(v^{p+1})$ (respectively, $S_i \subseteq C_i^r(u^{q-1})$). 
   %and $u^q \in C_i\setminus S_i$ be the vertex having the minimum right-end point with respect to $C_i^{r\_list}$ such that $u^q$ (if exists) has a right-conflict. 
   And, if  $C_i\setminus S_i$ does not have a vertex having a left-conflict (respectively, right-conflict), then we consider $v^{p+1}$ as $v^1$ (respectively, $u^{q-1}$ as $u^{ \mid C_i \mid }$).  %any case, by Observation~\ref{obs:conflict}, we have $S_i \subseteq C_i^l(v^{p+1})\cap C_i^r(u^{q-1}) = C_i^{lr}(v^{p+1},u^{q-1})$. 
   Therefore, in any case, we have, $S_i\subseteq  C_i^l(v^{p+1})\cap C_i^r(u^{q-1})=C_i^{lr}(v^{p+1},u^{q-1})$, for some $p,q\in \{0,1,\ldots, \mid C_i \mid +1\}$.

    
    Now to prove the reverse inequality, first note that by the definitions of $v^p$, $u^q$, and $C_i^{lr}$, no vertex in $C_i^{lr}(v^{p+1},u^{q-1})\subseteq C_i$ can have a \textit{left-conflict} or a \textit{right-conflict} with respect to  $\mathcal{S}$. Therefore, by Observation~\ref{obs:noconflict}, and the fact that $\mathcal{S}$ is a maximum cardinality separated-cluster, we can conclude that $C_i^{lr}(v^{p+1},u^{q-1}) \subseteq S_i$. Thus,  $S_i=C_i^{lr}(v^{p+1},u^{q-1})$ for some $p,q\in \{0,1,\ldots, \mid C_i \mid +1\}$.
Since $S_i = C_i^{lr}(v^{p+1},u^{q-1})\in \hat{\mathcal{C}}$, the statement of the lemma follows.
\end{proof}

We use the following construction to reduce the \SCP\ problem on interval graphs to the maximum weighted independent set problem on cocomparability graphs.

\begin{construction}[Weighted conflict graph $G_c$] \label{def:conflict}
    Given an interval graph $G$ having maximal cliques, say $\{C_1,C_2,\ldots,C_l\}$, the weighted conflict graph $G_c$ is defined as follows: $V(G_c)=\hat{\mathcal{C}}=\{C_i^{lr}(v^p, u^q): i\in \{1,2,\ldots,l\}, p,q\in \{1,2,\ldots, \mid C_i \mid \}\}$ (each vertex in $G_c$ represents a clique in $G$, which is also a member of $\hat{\mathcal{C}}$).  For the sake of convenience, let $V(G_c)=\hat{\mathcal{C}}=\{K_1,K_2,\ldots,K_t\}$, where $t= \mid \hat{\mathcal{C}} \mid \leq n^3$. Now, for each vertex, say $K_j\in V(G_c)$, we define the weight function $w(K_j)= \mid K_j \mid $ (i.e., the cardinality of the corresponding clique $K_j$ in $G$). For any pair of vertices, say $K_i,K_j\in V(G_c)$ with $i\neq j$, we make $K_i$ and $K_j$ adjacent in $G_c$ if and only if at least one of the following conditions is true in $G$.
    \begin{enumerate}[label=(\alph*)]
        \item \label{cond:1} $K_i\cap K_j\neq \emptyset$, or
        \item \label{cond:2} there exist vertices, $x\in K_i$ and $y\in K_j$ such that $xy\in E(G)$, or
        \item \label{cond:3} there exist vertices, $x\in K_i$, $y\in K_j$, and $z\in V(G)\setminus (K_i\cup K_j)$ such that $x,y\in N_G(z)$.
    \end{enumerate}
\end{construction}

We claim that  $G_c$ is a cocomparability graph. To prove this,  we use the following vertex ordering characterization of cocomparability graphs  due to Damaschke~\cite{DamForbidOS90}.

\begin{theorem}[~\cite{DamForbidOS90}]\label{thm:cocomp}
    An undirected graph $H$ is a cocomparability graph if and only if there is an ordering $<$ of $V(H)$ such that for any three vertices $i<j<k$, if $ik\in E(H)$, then either $ij\in E(H)$ or $jk\in E(H)$.
\end{theorem}
The vertex ordering specified in Theorem~\ref{thm:cocomp} is called \textit{umbrella-free} ordering~\cite{DamForbidOS90} which essentially says that \textit{cocomparability graphs are exactly those graphs whose vertex set admits an umbrella-free} ordering.

Let $G$ be an interval graph with an interval representation, $\{I_v\}_{v\in V(G)}$, where $n= \mid V(G) \mid $ and $G_c$ be the corresponding weighted conflict graph obtained by Construction~\ref{def:conflict}.  Recall that $V(G_c)=\{K_1,K_2,\ldots,K_t\}$, where $t=~\mid\hat{\mathcal{C}} \mid~ \leq ~n^3$. Let $j\in \{1,2,\ldots,t\}$. Since each of the vertices $K_j$ in $V(G_c)$ represent a clique in $G$, they have a \textit{Helly region} (the common intersection region of intervals belonging to the same clique), say $H_{K_j}$ in the interval representation, $\{I_v\}_{v\in V(G)}$.  Let $r(H_{K_j})$ denote the right end-point of the \textit{Helly region} to the clique corresponding to $K_j$.  We then define an ordering $<$ of the vertices in $G_c$ as follows: \textit{for any two vertices $K_i,K_j\in V(G_c)$, where $i,j\in \{1,2,\ldots,t\}$, we say that $K_i<K_j$ if and only if $r(H_{K_i})\leq r(H_{K_j})$} (break the ties arbitrarily, i.e. for any pair $i,j\in \{1,2,\ldots,t\}$,  if $r(H_{K_i})= r(H_{K_j})$ then we can have either $K_i<K_j$ or $K_j<K_i$). \\

We then have the following lemma. 

\begin{lemma}
    For any interval graph $G$, the weighted conflict graph $G_c$ is a cocomparability graph.
\end{lemma}
\begin{proof}
    To prove the lemma, it is enough to show that the ordering $<$ of $V(G_c)$ (defined in the paragraph above) is an umbrella-free ordering. Suppose not. Then there exist cliques, say, $K_i,K_j,K_l$ in $V(G_c)$ such that $K_i<K_j<K_l$, $K_iK_l\in E(G_c)$, but $K_iK_j\notin E(G_c)$ and $K_jK_l\notin E(G_c)$. Since $K_i<K_j<K_l$, we have by the definition of $<$ that $r(H_{K_i})\leq r(H_{K_j})\leq r(H_{K_l})$ (where $H_{K_i}$, $H_{K_j}$, and $H_{K_l}$ denote the \textit{Helly regions} of the cliques, $K_i$, $K_j$, and $K_l$ respectively). Note that $K_iK_l\in E(G_c)$. Therefore, by the definition of $G_c$, at least one of the conditions in Construction~\ref{def:conflict} has to be true. 

    \noindent\textbf{Case-1:} \textit{The edge $K_iK_l\in E(G_c)$ is due to Condition~\ref{cond:1}}. 

    \noindent This implies that $K_i\cap K_l\neq \emptyset$. Let $x\in K_i\cap K_l$. Since $r(H_i)\leq r(H_j)\leq r(H_l)$, we then have that the interval representing the vertex $x$ intersects with the \textit{Helly region}, $H_j$ of the clique $K_j$. Therefore, by Condition~\ref{cond:1} or~\ref{cond:2} in Construction~\ref{def:conflict} (depending on the fact that the vertex $x$ belongs to $K_j$ or not), we have that both the edges, $K_iK_j, K_jK_l\in E(G_c)$. This is a contradiction.

    \noindent\textbf{Case-2:} \textit{The edge $K_iK_l\in E(G_c)$ is due to Condition~\ref{cond:2}}.

    \noindent This implies that there exist vertices $x\in K_i$ and $y\in K_l$ such that $xy\in E(G)$. Note that for each vertex $u\in K_i$, we have $r(u)< l(H_j)\leq r(H_j)$ (Since by assumption, $K_iK_j\notin E(G_c)$ which will be violated due to  Conditions~\ref{cond:1} and~\ref{cond:2} of Construction~\ref{def:conflict}). %(since $r(H_i)\leq r(H_j)$, $K_iK_j\notin E(G_c)$, and by Conditions~\ref{cond:1} and~\ref{cond:2} of Construction~\ref{def:conflict}). 
    Similarly, we have $r(H_j)<l(v)$ for each vertex $v\in K_l$ (as $r(H_j)\leq r(H_l)$, $K_jK_l\notin E(G_c)$, and by Conditions~\ref{cond:1} and~\ref{cond:2} of Construction~\ref{def:conflict}). Therefore, as $x\in K_i$ and $y\in K_l$, in particular, we have $r(x)<r(H_j)<l(y)$. This implies that $I_x\cap I_y=\emptyset$. Therefore, $xy\notin E(G)$, a contradiction.

    \noindent\textbf{Case-3:} \textit{The edge $K_iK_l\in E(G_c)$ is due to Condition~\ref{cond:3}}.
    
   \noindent This implies that there exist vertices $x\in K_i$, $y\in K_l$, and $z\in V(G)\setminus (K_i\cup K_l)$ such that $x,y\in N_G(z)$. Note that $xy\notin E(G)$, as we already have a contradiction for this in Case-2. Now since $r(H_i)\leq r(H_j)\leq r(H_l)$ and $xz,yz\in E(G)$, %$K_iK_j\notin E(G_c)$, and $K_jK_l\notin E(G_c)$, 
   this further implies that the $l(z)\leq r(x) < l(y)\leq r(z)$. Since the interval of $x$ starts before $r(H_i)$ and interval of $y$ ends after $r(H_l)$, the interval from $r(H_i)$ to $r(H_l)$ is covered by the intervals of $x,y$ and $z$.
 
 In either cases, by Condition~\ref{cond:1} or~\ref{cond:2} or~\ref{cond:3} in Construction~\ref{def:conflict} we have at least one of the edges, $K_iK_j, K_jK_l\in E(G_c)$. This is again a contradiction.

   Since we obtain a contradiction in all the possible cases, we can therefore conclude that the ordering $<$ of $V(G_c)$ (defined in the paragraph above) is an umbrella-free ordering. This implies that $G_c$ is a cocomparability graph by Theorem~\ref{thm:cocomp}. Hence the lemma.
\end{proof}

\noindent\textbf{Reduction:} Here, we show a polynomial-time reduction of \SCP\ problem on interval graphs to the maximum weighted independent set problem on cocomparability graphs. We first note the following theorem due to K\"ohler and Mouatadid~\cite{KohLalMaxWtIndSetAlg16}.
\begin{theorem}[\cite{KohLalMaxWtIndSetAlg16}] \label{thm:indptcocomp}
    Let $H$ be a cocomparability graph with weight function\\ $w:~V(G)~\rightarrow~R^+$. Then an independent set of maximum possible weight in $H$ can be computed in $O( \mid V(H)+ \mid E(H) \mid )$ time.
\end{theorem}

%\vspace{-0.5cm}

We then have the following main theorem.

\begin{theorem}\label{thm:reduction}
    Let $G$ be an interval graph and $G_c$ be its weighted conflict graph. Then $\mathcal{S}=\{S_1,S_2,\ldots,S_k\}$ is a maximum cardinality separated-cluster in $G$ if and only if $\mathcal{S}=\{S_1,S_2,\ldots,S_k\}$ is a maximum weighted independent set in $G_c$.
\end{theorem}
\vspace{-0.5cm}
\begin{proof}
First, note that if $\mathcal{S}=\{S_1,S_2,\ldots,S_k\}$ is a maximum cardinality separated-cluster in $G$, then by Lemma~\ref{lem:maxsep}, we have $\mathcal{S}=\{S_1,S_2,\ldots,S_k\}\subseteq \hat{\mathcal{C}}=V(G_c)$. Moreover by the definition of separated-clusters, the conditions for $S_i,S_j\in \mathcal{S}$ is consistent with the conditions~\ref{cond:1}, \ref{cond:2}, and~\ref{cond:3} of the weighted conflict graph $G_c$. Therefore, $\mathcal{S}\subseteq V(G_c)$ is an independent set in $G_c$.
On the other hand, if  $\mathcal{S}=\{S_1,S_2,\ldots,S_k\}$ is a independent set in $G_c$ then clearly,  for $S_i,S_j\in \mathcal{S}$ the conditions~\ref{cond:1}, \ref{cond:2} and~\ref{cond:3}  are satisfied. Therefore, $\mathcal{S}$ is a separated-cluster in $G$. Since weights of the vertices in $G_c$ (i.e. cliques in $G$) are exactly the cardinality of their corresponding sets, we can therefore conclude that $\mathcal{S}\subseteq V(G_c)=\hat{\mathcal{C}}$ is a maximum weighted independent set in $G_c$ if and only if $\mathcal{S}$ is a maximum cardinality separated-cluster in $G$. This proves the theorem.
\end{proof}

Now Theorem~\ref{thm:sepiterval} follows from Theorems~\ref{thm:indptcocomp} and~\ref{thm:reduction}.