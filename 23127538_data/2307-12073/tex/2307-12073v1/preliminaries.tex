\section{Preliminaries}
\label{sec:prelim}
All graphs considered in this paper are undirected, simple, and finite. 
Let $G$ be a graph.
The vertex set and edge set of $G$ are denoted by $V(G)$ and $E(G)$, respectively.  
 A \textit{subgraph} $G'$ of $G$ is a graph such that $V(G')\subseteq V(G)$ and $E(G')\subseteq E(G)$. 
Let $S\subseteq V(G)$, then \textit{induced subgraph}, denoted as $G[S]$, is a graph whose vertex set is $S$ and whose edge set contains all the edges in $E(G)$ that have both their endpoints in $S$. Sometimes, we also call $G[S]$ to be the \textit{graph induced by the vertices} in $S$.
By $\bar{G}$, we denote the \textit{complement} of a graph $G$, which is the graph having the same set of vertices as $G$ such that two vertices $u,v$ in $\bar{G}$ are adjacent if and only if they are non-adjacent in $G$. 
For a vertex $v\in G$, we define the \textit{neighborhood} of $v$ in $G$, denoted as $N_G(v)$, to be the set of all the vertices  adjacent to $v$ in $G$ (sometimes, we omit the subscript, if the graph $G$ is clear from the context). The degree of a vertex $v$ in $G$, denoted by $d_G(v)$, is the number of edges incident on $v$ in $G$. The maximum degree of a graph $G$ is denoted by $\Delta(G)$.
We denote by $d_G(u,v)$, the \textit{distance} between two vertices $u$ and $v$ in $G$, which is the number of edges in the shortest path between $u$ and $v$ in $G$. The \textit{square} of a graph $G$, denoted by $G^2$, is the graph having $V(G^2)=V(G)$, and $E(G^2)=\{uv:d_G(u,v)\leq 2\}$.
 The \textit{disjoint union} of two graphs $G_1$ and $G_2$, denoted by $G_1\uplus G_2$, is the graph having
    $V(G_1\uplus G_2) = V(G_1)\cup V(G_2)$ and $E(G_1\uplus G_2) = E(G_1) \cup E(G_2)$. The disjoint union of $t$ copies of a graph $G$ is denoted by $tG$. Let $H$ be a subgraph of $G$. We denote by $G-H$, the graph obtained from $G$ by removing the vertices in $H$, i.e. $G-H = G[V(G)\setminus V(H)]$. 



A set $M\subseteq E(G)$ is said to be a \textit{matching} in $G$ if no two edges in $M$ share a common vertex.
A set $S\subseteq V(G)$ is said to be a \textit{clique} (resp. an \textit{independent set}) if every pair of vertices in $S$ are adjacent (resp. non-adjacent) in $G$. 
The number of vertices in the largest clique (resp. independent set) of a graph $G$ is called the \textit{clique number} (resp. \textit{independence number}) of $G$, denoted as $\omega(G)$ (resp. $\alpha(G)$). Note that if $G$ is a weighted graph (vertices in $G$ have weights on it), then \textit{weight of a set} $S\subseteq V(G)$ is defined to be the sum of weights of the vertices in $S$, and a  \textit{maximum weighted independent set} problem seeks to find an independent set in $G$ having maximum weight. 
A \textit{clique cover} of a graph $G$ is the partition of the vertices of $G$ into cliques. A clique $K$ in $G$ is said to be \textit{maximal} if $K\cup \{v\}$ does not form a clique for any vertex $v\in V(G)$. 
The \textit{clique cover number}, denoted by $k(G)$, is the smallest $k$ for which $G$ has a clique cover of size $k$.
 A set $S\subseteq V(G)$ is said to be a \textit{dominating set} in $G$ if every vertex in $V(G) \setminus S$ has a neighbor in $S$. The \textit{domination number} of $G$, denoted as $\gamma(G)$, is defined to be the minimum integer $k$ such that $G$ has a dominating set of size $k$. A set $S\subseteq V(G)$ is said to be a \textit{total dominating set} if any vertex in $G$ has a neighbor in $S$. The \textit{total domination number} of $G$, denoted as $\gamma_t(G)$, is defined to be the minimum integer $k$ such that $G$ has a total dominating set of size $k$.
The \textit{proper coloring} of a graph $G$ is the coloring of the vertices of $G$ such that no two adjacent vertices of $G$ have the same color. A proper coloring of $G$ using at most $k$ colors is called a  $k$-coloring of $G$. The smallest integer $k$ for which there exists a proper coloring  of $G$ using $k$ colors is called the \textit{chromatic number} and is denoted as $\chi(G)$. A \textit{cd-coloring} of graph $G$ is a proper coloring of $G$ having an additional property that each color class has a dominating vertex (a vertex adjacent to every vertex in the color class) in $G$. The smallest integer $k$ for which there exists a $cd$-coloring  of $G$ using $k$ colors is called \textit{cd-chromatic number} of $G$, and is  denoted as $\XCD(G)$. A set $S\subseteq V(G)$ is said to be a \textit{separated-cluster} (also known as \textit{sub-clique}) if no two vertices in $S$ lie at a distance 2 in $G$. Alternatively, a \textit{separated-cluster} of $G$ can also be viewed as a family of subsets $\mathcal{S}$, where $G[\mathcal{S}]$ is a disjoint union of cliques in $G$, say $\mathcal{S}=\{C_1,C_2,\ldots,C_k\}$ having the following property: there does not exist a pair of vertices $x\in C_i$, $y\in C_j$, where $i,j\in \{1,2,\ldots,k\}$, $i\neq j$ and $z\in V(G)$ such that $x,y\in N_G(z)$. The \textit{separated-cluster number} of $G$, denoted as $\SC(G)$, is defined to be the maximum integer $k$ such that $G$ has a separated-cluster of size $k$.  


 A graph $G$ is said to be \textit{perfect}, if $\chi(H)=\omega(H)$, for every induced subgraph $H$ of $G$. The celebrated \textit{Strong Perfect Graph Theorem} proved by Chudnovsky et.al~\cite{chudnovsky2006strong} says that  \textit{perfect graphs are exactly the graphs that do not contain odd holes (odd-length induced cycles of length at least 5) or odd antiholes (complements of odd holes) as induced subgraphs}.


A graph $G$ is said to be \textit{$H$-free}, if it does not contain $H$ as an induced subgraph. A \textit{diamond} is a graph on 4 vertices and 5 edges. i.e. it is the graph obtained by deleting an edge from $K_4$ (complete graph on 4 vertices). A graph is said to be \textit{d-regular} if every vertex of $G$ has degree $d$. A \textit{bipartite} graph, denoted as $G=(A,B,E)$, is a graph whose vertices can be partitioned into two independent sets $A$ and $B$ such that every edge contains its one end-point in $A$ and the other end-point in $B$. A \textit{complete bipartite} graph, denoted as $K_{m,n}$, is a bipartite graph $G=(A,B,E)$ with $|A|=m$, $|B|=n$ such that each vertex in $A$ is adjacent to every vertex in $B$. A \textit{star graph} is the complete bipartite graph, $K_{1,t}$ on $t+1$ vertices for some integer $t\geq 1$. A \textit{claw} $K_{1,3}$ is a star on 4 vertices. The complement of a bipartite graph is called \textit{co-bipartite} graph. A \textit{chordal bipartite graph} is a bipartite graph that does not contain any induced cycle of length $k$, where $k\geq 6$. A collection $\{I_v\}_{v\in V(G)}$, of intervals on a real line is said to be an \textit{interval representation} of a graph $G$ if for any pair of vertices $u,v\in V(G)$, we have $uv\in E(G)$ if and only if $I_u\cap I_v\neq \emptyset$. A graph is said to be an \textit{interval graph} if it has a corresponding interval representation. A \textit{proper interval} graph is an interval graph having an interval representation in which no interval properly contains another. 
A graph is said to be \textit{chordal} if it does not contain any induced cycle of length $k$, where $k\geq 4$. An independent set of three vertices of a graph is said to form an \textit{asteroidal triple (AT)}, if every two of them are connected by a path that does not contain any neighbor of the third one. It is known that \textit{interval graphs are exactly the graphs which are chordal and AT-free}.  

\medskip

Even though the parameters $\XCD$ and $\SC$ have a strong similarity to their classic counterparts, $\chi$, and $\omega$, here we note an important property of $\XCD$ and $\SC$, which is fundamentally different from $\chi$ and $\omega$. 

\medskip

\noindent\textbf{Note:}
For any induced subgraph $H$ of a graph $G$, we have $\chi(H)\leq \chi(G)$ and $\omega(H)\leq \omega(G)$. But it can be seen that this is \textit{not true} in general for the parameters $\XCD$ and $\SC$. i.e. it is possible for  $G$ to have an induced subgraph $H$ such that $\XCD(H)>\XCD(G)$ or $\SC(H)>\SC(G)$. For instance, let $G=K_{1,3}$, the star graph, with central vertex, say, $u$, and leaf vertices, say, $v_1,v_2,v_3$. It is easy to verify that $\XCD(G)=\SC(G)=2$. On the other hand, consider the induced subgraph of leaf vertices, say, $H=G[\{v_1,v_2,v_3\}]$. It can be seen that, $\XCD(H)=\SC(H)=3$.

\medskip

In the following observation, we have a natural sufficient condition for an induced subgraph, $H$ of a graph $G$ to have \XCD($G$) $\geq$ \XCD($H$). We use this observation later.


\begin{observation}
    \label{obs:cdcolor_reduce}
     Let $H$ be an induced subgraph of the graph $G$.  Suppose  there does not  exist a set of independent vertices $I\subseteq V(H)$ such that $I$ has a dominating vertex in $G$ but not in $H$, then  \XCD($G$) $\geq$ \XCD($H$).
\end{observation}

\begin{proof} 
   Let \XCD($G$) =$k$, and $\{I_1,I_2,\ldots, I_k\}$ be the color classes in $G$ with respect to a $k-cd$-coloring of $G$.  Then the independent sets $\{I_{h1},I_{h2},\ldots, I_{hk}\}$ is a partition of $V(H)$, where $I_{hi}= I_{i}\cap V(H)$. Note that each of the sets in $\{I_1, I_2,\ldots, I_k\}$ is dominated by a vertex in $G$. From the statement of the observation, it is then clear that each of the sets in $\{I_{h1},I_{h2},\ldots, I_{hk}\}$ is also dominated by a vertex in $H$. This implies that $\XCD(H)\leq k=\XCD(G)$. 
\end{proof}

We also make use of the following known results.

\begin{proposition}[\cite{DBLP:conf/interaction/Zhu09}]
    \label{pro:bipartite}
   Total domination in bipartite graphs with bounded degree 3 is  \NPC. \TETHS.
\end{proposition}


\begin{proposition}[\cite{DBLP:journals/gc/MerouaneHCK15}]
    \label{pro:triangle-free}
   Let $G$ be a triangle-free graph. Then \XCD ($G$) = $\gamma_t (G)$.
\end{proposition}

The major problems that we consider in this paper are: \\


\begin{mdframed}
  \textbf{\sc{\TD\ } }\\ \textbf{Input:} A graph $G$ and an integer $k$\\ \textbf{Question:} Does there exist a \textit{total domination }set $S\subseteq V(G)$ of size at most $k$ in $G$?\\
  
\end{mdframed}

\begin{mdframed}
  \textbf{\sc{\CDC\ } }\\ \textbf{Input:} A graph $G$ and an integer $k$\\ \textbf{Question:} Does there exist a $cd$-coloring of $G$ with at most $k$ colors?
\end{mdframed}

\begin{mdframed}
  \textbf{\sc{\SCP\ } }\\
  \textbf{Input:} A graph $G$ and an integer $k$\\ \textbf{Question:} Does there exist a \textit{seprarated-cluster} of size at least $k$ in $G$?  
\end{mdframed}


