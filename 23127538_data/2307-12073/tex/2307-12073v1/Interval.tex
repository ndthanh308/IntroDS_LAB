\section{Separted-cluster problem in Interval graphs}
\label{sec:interval}

In this section, we settle an open problem proposed by Shalu et.al.~\cite{DBLP:conf/caldam/ShaluVS17}, by proving that \SCP\ is polynomial-time solvable for the class of interval graphs. We achieve this, by showing a polynomial-time reduction of \SCP\ problem on interval graphs to the maximum weighted independent set problem on cocomparability graphs (which can be solved in time linear to the size of the input graph). Note that throughout the section, we make use of the alternative definition of \textit{separated-cluster}. i.e. for a graph $G$, we say that a family of disjoint cliques, say $\mathcal{S}=\{C_1,C_2,\ldots,C_k\}$ is a \textit{separated cluster} in $G$ if there does not exist a pair of vertices $x\in C_i$, $y\in C_j$, where $i,j\in \{1,2,\ldots,k\}$, $i\neq j$ and $z\in V(G)\setminus (C_i\cup C_j)$ such that $x,y\in N_G(z)$.  

\medskip

Let $G$ be an interval graph with an interval representation, $\{I_v\}_{v\in V(G)}$, where $|V(G)|=n$. 
Without loss of generality, we can assume that \textit{all the end-points of the intervals in $\{I_v\}_{v\in V(G)}$ are distinct}. 
For a vertex $v\in V(G)$, for the sake of convenience, we denote by $l(v)$ and $r(v)$, the \textit{left} and \textit{right end-points} of the interval $I_v$, respectively.  


\medskip

\noindent\textbf{Definition of $\hat{\mathcal{C}}$:} Let  $\{C_1,C_2,\ldots, C_l\}$ be the collection of all \textit{maximal cliques} in $G$.  For $i\in \{1,2,\ldots,l\}$, let $|C_i|=t$. We enumerate the vertices belonging to $C_i$ in two different ways (with respect to the left and right end-points of their intervals). i.e. $C_i^l=(v_{l1},v_{l2},\ldots, v_{lt})$ is such that $l(v_{l1})< l(v_{l2})<\cdots < l(v_{lt})$, and let $C_i^r=(v_{r1},v_{r2},\ldots, v_{rt})$ is such that $r(v_{r1})< r(v_{r2})<\cdots <r(v_{rt})$. We then define the following.  Let $i\in \{1,2,\ldots,l\}$ and $p,q\in \{0,1,2,\ldots,|C_i|-1\}$. For $p,q>0$, let $C_i^{v_{lp}}=\{u\in C_i: l(u)>l(v_{lp})\}$ and $C_i^{v_{rq}}=\{u\in C_i:  r(u)<r(v_{rq})\}$. And, let $C_i^{v_{l0}}=C_i^{v_{r0}}=C_i$ (note that for each $i$, $v_{l0}$ and $v_{r0}$ represent the left and right counterparts of a dummy vertex for the sake of convenience in the notation).  We then define the set $C_i^{v_{lp}v_{rq}}=C_i^{v_{lp}}\cap C_i^{v_{rq}}$. In other words, $C_i^{v_{lp}v_{rq}}$ is either the maximal clique $C_i$ (when $p=q=0$), or a clique obtained from $C_i$, \textit{by dropping the vertices} whose left end-point is less than or equal to that of $l(v_{lp})$ (if $p>0$) or their right end-point is  greater than or equal to that of $r(v_{rq})$ (if $q>0$) or both (if both $p,q>0$). 
Let $\hat{\mathcal{C}_i}=\{C_i^{v_{lp}v_{rq}}:p,q\in \{0,1,2,\ldots,|C_i|-1\}\}$. We then let $\hat{\mathcal{C}}=\{\hat{\mathcal{C}_i}:i\in \{1,2,\ldots,l\}\}$. Note that as $G$ is an interval graph, we have that $l\leq n$~\cite{golumbic2004algorithmic}. Since the number of ordered pairs of the form $(p,q)$ is at most $n^2$ (as $p,q\in \{0,1,2,\ldots,|C_i|-1\}$, $|C_i|\leq n$), we then have that $|\hat{\mathcal{C}}|\leq n^3$.  

\medskip

Let $\mathcal{S}$ be a separated cluster in $G$. Note that by definition, any pair of cliques belonging to $\mathcal{S}$ are disjoint. Since $G$ is an interval graph, the cliques in $\mathcal{S}$ have disjoint \textit{Helly regions} (the common intersection region of intervals belonging to the clique) in the interval representation of $G$. Let $S_1,S_2,\ldots,S_k$ be an ordering of the cliques in $S$ having the following property: \textit{ for any pair $r,s\in \{1,2,\ldots,k\}$, we have $S_r<S_s$ if and only if the Helly region of the clique $S_r$ lies entirely left to the Helly region of the clique $S_s$}. For a clique $S_i\in \mathcal{S}$, where $i\in \{1,2,\ldots,k\}$, let $C_i$ be a maximal clique in $G$ such that $S_i\subseteq C_i$ (possibly, $C_i=S_i$). Suppose that $C_i\setminus S_i\neq \emptyset$. Clearly, by the definition of $\mathcal{S}$, no vertex in $C_i\setminus S_i$ can be adjacent to any vertex in $S_j$, for $j\neq i$. We say that a vertex $x\in C_i\setminus S_i$ has a \textit{left-conflict} (resp. \textit{right-conflict}) with respect to $\mathcal{S}$, if there exists a clique $S_j$ with $S_j<S_i$ (resp. $S_j>S_i$) and vertices $y\in S_j$, $z\in N(y)\cap (V(G)\setminus S_j)$ such that $xz\in E(G)$.  

 The following observation is due to the definitions of \textit{left-conflict} (resp. \textit{right-conflict}), and $C_i^{v_{lp}}$ (resp. $C_i^{v_{rq}}$).
\begin{observation}\label{obs:conflict}
    Let $\mathcal{S}=\{S_1,S_2,\ldots,S_k\}$ be a separated cluster in $G$. For $i\in \{1,2,\ldots,k\}$, let $C_i$ be a maximal clique in $G$, such that $S_i\subseteq C_i$. Let a vertex $v\in C_i\setminus S_i$ has a \textit{left-conflict} (resp. \textit{right-conflict}) with respect to $\mathcal{S}$. Let $v=v_{lp}$ (resp. $v=v_{rq}$) in the ordering $C_i^l$ (resp. $C_i^r$). Then every vertex in $C_i\setminus C_i^{v_{lp}}$ (resp. $C_i\setminus C_i^{v_{rq}}$) has a \textit{left-conflict} (resp. \textit{right-conflict}). In other words, we then have, $S_i\subseteq C_i^{v_{lp}}$ (resp. $S_i\subseteq C_i^{v_{rq}}$).
\end{observation}
\begin{observation} \label{obs:noconflict}
Let $\mathcal{S}=\{S_1,S_2,\ldots,S_k\}$ be a separated cluster in $G$. For $i\in \{1,2,\ldots,k\}$, let $C_i$ be a maximal clique in $G$, such that $S_i\subseteq C_i$. If there exists a vertex $x\in C_i\setminus S_i$ such that $x$ has neither a left-conflict or a right-conflict with respect to $\mathcal{S}$, then $\mathcal{S}=\{S_1,S_2,\ldots, S_i',\ldots,S_k\}$ is also a separated cluster in $G$, where $S_i'=S_i\cup\{x\}$.
\end{observation}
\begin{proof}
    Since any pair of cliques in $\mathcal{S}$ are mutually non adjacent, we have $C_i\cap S_j=\emptyset$ for each $j\in \{1,2,\ldots,k\}$, with $i\neq j$. Further, no vertex in $C_i$ can be adjacent to any vertex in $S_j$, with $i\neq j$, as otherwise, there exists a vertex $u\in S_j$, which is lying exactly at a distance 2 from each vertex in $S_i$, contradicting the fact that $\mathcal{S}$ is a separated cluster. Now, consider a vertex $x\in C_i\setminus S_i$,  which has neither a \textit{left-conflict} or a \textit{right-conflict} with respect to $\mathcal{S}$. For $j\neq i$, let $y\in S_j$. Suppose that $S_j<S_i$ (resp. $S_j>S_i$). If $d_G(x,y)=2$, then by a previous observation, there exists a vertex $z\in N(y)\cap (V(G)\setminus S_j)$ such that $xz\in E(G)$. This implies that $x$ has a \textit{left-conflict} (resp. \textit{right-conflict}) with respect to $\mathcal{S}$. Since we have a contradiction, we can therefore conclude that $d_G(x,y)\neq 2$ for any vertex $y\in S_j$, where $j\neq i$. This implies that $\mathcal{S}=\{S_1,S_2,\ldots, S_i',\ldots,S_k\}$ is also a separated cluster in $G$, where $S_i'=S_i\cup\{x\}$. 
\end{proof}
\begin{lemma} \label{lem:maxsep}
    Let $\mathcal{S}=\{S_1,S_2,\ldots,S_k\}$ be a maximum cardinality separated cluster in $G$. Then for each $i\in \{1,2,\ldots,k\}$, we have $S_i\in \hat{\mathcal{C}}$.
\end{lemma}
\begin{proof}
For $i\in \{1,2,\ldots,k\}$, let $S_i$ be a clique in $\mathcal{S}$. If $S_i$ is a maximal clique in $G$, then $S_i\in \hat{\mathcal{C}}$, as $\hat{\mathcal{C}}$ contains all maximal cliques in $G$. Suppose that $S_i$ is not a maximal clique in $G$. Then, there exists a maximal clique, say $C_i$ in $G$, such that $S_i\subseteq C_i$. Clearly, $C_i\setminus S_i \neq \emptyset$. Since $\mathcal{S}$ is a maximum cardinality separated cluster in $G$, by Observation~\ref{obs:noconflict}, we have that every vertex in $C_i\setminus S_i$ has either a \textit{left-conflict} or a \textit{right-conflict} or both. 

Define $L=\{u\in C_i\setminus S_i: u$ has a \textit{left-conflict} with respect to $\mathcal{S}$\}, and $R=\{u\in C_i\setminus S_i: u$ has a \textit{right-conflict} with respect to $\mathcal{S}$\}. Therefore, we have $C_i\setminus S_i=L\cup R$. Since $L,R\subseteq C_i$, we may consider $L\subseteq C_i^l$ and $R\subseteq C_i^r$. We then define the following:
$$v_{lp}=\begin{cases}
     \text{ the vertex in } L \text{ having the property that } l(v_{lp})=\max\{l(v_{li}):v_{li}\in L\subseteq C_i^l\}, & \text{ if } L\neq \emptyset \\
     v_{l0}, & \text{ otherwise}
\end{cases}$$
 $$v_{rq}=\begin{cases}
     \text{ the vertex in } R \text{ having the property that } r(v_{rq})=\min\{r(v_{ri}):v_{ri}\in R\subseteq C_i^r\}, & \text{ if } R\neq \emptyset \\
     v_{r0}, & \text{ otherwise}
\end{cases}$$
We then have the following claim.

\medskip

\noindent\textbf{Claim: $S_i=C_i^{v_{lp}v_{rq}}$}

\smallskip
Let $u\in S_i\subseteq C_i$. If $v_{lp}\neq v_{l0}$, then as $v_{lp}\in L\subseteq C_i$ has a \textit{left-conflict}, by Observation~\ref{obs:conflict}, we have that $S_i\subseteq C_i^{v_{lp}}$. Similarly, if $v_{rq}\neq v_{r0}$, then again, by Observation~\ref{obs:conflict}, we have that $S_i\subseteq C_i^{v_{rq}}$. Also, if $v_{lp}= v_{l0}$ then $S_i\subseteq C_i= C_i^{v_{lp}}$, and if $v_{rq}=v_{r0}$ then $S_i\subseteq C_i=C_i^{v_{rq}}$. Therefore, in any case, we have, $S_i \subseteq C_i^{v_{lp}}\cap C_i^{v_{rq}}\subseteq C_i^{v_{lp}v_{rq}}$. On the other hand, by the choices of $v_{lp}$ and $v_{rq}$, we have that no vertex in $C_i^{v_{lp}v_{rq}}$ can have a \textit{left-conflict} or a \textit{right-conflict}. Then by Observation~\ref{obs:noconflict} and the fact that $\mathcal{S}$ is a maximum cardinality separated cluster, we can conclude that $C_i^{v_{lp}v_{rq}} \subseteq S_i$ as well. This proves our claim, and also completes the proof of the lemma, since $C_i^{v_{lp}v_{rq}}\in \hat{\mathcal{C}}$.
\end{proof}
For the reduction purpose, we now define the following:
\begin{definition}[Weighted conflict graph $G_c$] \label{def:conflict}
    Given an interval graph, $G$, having maximal cliques, say $\{C_1,C_2,\ldots,C_l\}$, the weighted conflict graph $G_c$ is defined to be the graph with $V(G_c)=\hat{\mathcal{C}}=\{C_i^{v_{lp}v_{rq}}: i\in \{1,2,\ldots,l\}, p,q\in \{0,1,2,\ldots,|C_i|-1\}\}$, and weight function $w(X)=|X|$, for each $X\in V(G_c)$. For any pair of vertices, say $X,Y\in V(G_c)$, we make $X$ and $Y$ adjacent in $G_c$, if and only if at least one of the following conditions is true.
    \begin{enumerate}[label=(\alph*)]
        \item \label{cond:1} $X\cap Y\neq \emptyset$
        \item \label{cond:2} there exist vertices, $x\in X$ and $y\in Y$ such that $xy\in E(G)$
        \item \label{cond:3} there exist vertices, $x\in X$, $y\in Y$, and $z\in V(G)\setminus (X\cup Y)$ such that $x,y\in N_G(z)$.
    \end{enumerate}
\end{definition}
Our aim is to prove that $G_c$ is a cocomparability graph. For this, we need the following vertex ordering characterization of cocomparability graphs due to Damaschke~\cite{damaschke1990forbidden}.
\begin{theorem}[~\cite{damaschke1990forbidden}]\label{thm:cocomp}
    An undirected graph $H$ is a cocomparability graph if and only if there is an ordering $<$ of $V(H)$ such that for any three vertices $i<j<k$, if $ik\in E(H)$, then either $ij\in E(H)$ or $jk\in E(H)$.
\end{theorem}
The vertex ordering specified in Theorem~\ref{thm:cocomp} is called an \textit{umbrella-free} ordering~\cite{damaschke1990forbidden}. Theorem~\ref{thm:cocomp} essentially says that cocomparability graphs are exactly those graphs whose vertex set admits an \textit{umbrella-free} ordering.

\medskip

Let $G$ be an interval graph with an interval representation, $\{I_v\}_{v\in V(G)}$, where $|V(G)|=n$. For the sake of convenience, let $V(G_c)=\hat{\mathcal{C}}=\{K_1,K_2,\ldots,K_t\}$, where $t=|\hat{\mathcal{C}}|\leq n^3$. Since each of the sets $K_j$, where $j\in \{1,2,\ldots,t\}$ is a clique in $G$, they have a \textit{Helly region} (the common intersection region of intervals  belonging to the clique), say $H_j$ in the interval representation, $\{I_v\}_{v\in V(G)}$. Let $j\in \{1,2,\ldots,t\}$. Let $r(H_j)$ denote the right end-point of the \textit{Helly region} corresponding to the clique $K_j$. We then define an ordering $<$ of the vertices in $G_C$ having the following property: \textit{for any two vertices $K_i,K_j\in V(G)$, where $i,j\in \{1,2,\ldots,t\}$, we have $K_i<K_j$ if and only if $r(H_i)\leq r(H_j)$} (break the ties arbitrarily, i.e. for any pair $i,j\in \{1,2,\ldots,t\}$,  if $r(H_i)= r(H_j)$, then we can have either $K_i<K_j$ or $K_j<K_i$). 

We then have the following lemma.

\begin{lemma}
    For any interval graph $G$, the weighted conflict graph $G_c$ is a cocomparability graph.
\end{lemma}
\begin{proof}
    To prove the lemma, it is enough to show that the ordering $<$ of $V(G_c)$ (defined in the paragraph above) is an umbrella-free ordering. Suppose not. Then there exist cliques, say, $K_i,K_j,K_l$ in $V(G_c)$ such that $K_i<K_j<K_l$, $K_iK_l\in E(G_c)$, but $K_iK_j\notin E(G_c)$ and $K_jK_l\notin E(G_c)$. Since $K_i<K_j<K_l$, we have by the definition of $<$ that $r(H_i)\leq r(H_j)\leq r(H_l)$ (where $H_i$, $H_j$, and $H_l$ denote the \textit{Helly regions} of the cliques, $K_i$, $K_j$, and $K_l$ respectively). Note that $K_iK_l\in E(G_c)$. Therefore, by the definition of $G_c$, at least one of the conditions in Definition~\ref{def:conflict} has to be true. Suppose that the edge $K_iK_l\in E(G_c)$ is due to Condition~\ref{cond:1}. This implies that $K_i\cap K_l\neq \emptyset$. Let $x\in K_i\cap L_l$. Since, $r(H_i)\leq r(H_j)\leq r(H_l)$, we then have that the interval representing the vertex $x$ intersects with the \textit{Helly region}, $H_j$ of the clique $K_j$. Therefore, by Condition~\ref{cond:2} in Definition~\ref{def:conflict}, we have that both the edges, $K_iK_j, K_jK_l\in E(G_c)$. This is a contradiction. Suppose that the edge $K_iK_l\in E(G_c)$ is due to Condition~\ref{cond:2}. This implies that there exist vertices $x\in K_i$ and $y\in K_l$ such that $xy\in E(G)$. Note that $r(u)< r(H_j)$ for each vertex $u\in K_i$ (since $r(H_i)\leq r(H_j)$, $K_iK_j\notin E(G_c)$, and by Condition~\ref{cond:2} of Definition~\ref{def:conflict}). Similarly, we have $l(v)> r(H_j)$ for each vertex $v\in K_l$ (as $r(H_j)\leq r(H_l)$, $K_jK_l\notin E(G_c)$, and by Condition~\ref{cond:2} of Definition~\ref{def:conflict}). Therefore, as $x\in K_i$ and $y\in K_l$, in particular, we have $r(x)<r(H_j)<l(y)$. This implies that $I_x\cap I_y=\emptyset$. Therefore, $xy\notin E(G)$. This is a contradiction to our assumption. Suppose that the edge $K_iK_l\in E(G_c)$ is due to Condition~\ref{cond:3}. This implies that there exist vertices $x\in K_i$, $y\in K_l$, and $z\in V(G)\setminus (K_i\cup K_l)$ such that $x,y\in N_G(z)$. This implies that the interval representing the vertex $z$ intersects with both the \textit{Helly regions} $H_i$ and $H_l$ of the cliques $K_i$ and $K_j$, respectively. Since the point $r(H_j)$ lies in between the \textit{Helly regions} $H_i$ and $H_l$, we can infer that the interval representing the vertex $z$ intersects with the \textit{Helly region} $H_j$ of clique $K_j$ as well. Therefore, by Condition~\ref{cond:2} or Condition~\ref{cond:3} in Definition~\ref{def:conflict} (depending on whether the vertex $z$ belongs to $K_j$ or not), we have that both the edges, $K_iK_j, K_jK_l\in E(G_c)$. This is again a contradiction. Since we obtain a contradiction in all the cases, we can, therefore, conclude that the ordering $<$ of $V(G_c)$ (defined in the paragraph above) is an umbrella-free ordering. This implies that $G_c$ is a cocomparability graph by Theorem~\ref{thm:cocomp}. Hence the lemma.
        \end{proof}

\noindent\textbf{Reduction:} Here, we show a polynomial-time reduction of \SCP\ problem on interval graphs to the maximum weighted independent set problem on cocomparability graphs. We first note the following theorem due to K\"ohler and Mouatadid~\cite{DBLP:journals/ipl/KohlerM16}.
\begin{theorem}[\cite{DBLP:journals/ipl/KohlerM16}]\label{thm:indptcocomp}
    Let $H$ be a cocomparability graph. Then an independent set of maximum possible weight in $H$ can be computed  in $O(|V(H)+|E(H)|)$ time.
\end{theorem}
We then have the following main theorem.
\begin{theorem}\label{thm:reduction}
    Let $G$ be an interval graph, and $G_c$, its conflict-graph. Then $\mathcal{S}=\{S_1,S_2,\ldots,S_k\}$ is a maximum cardinality separated cluster in $G$ if and only if $\mathcal{S}=\{S_1,S_2,\ldots,$\\$S_k\}$ is a maximum weighted independent set in $G_c$.
\end{theorem}
\begin{proof}
Let $\mathcal{T}\subseteq V(G_c)=\hat{\mathcal{C}}$. Then by the definition of the conflict-graph $G_c$ (see Definition~\ref{def:conflict}), we have that $\mathcal{T}$ is an independent set in $G_c$ if and only if $\mathcal{T}$ is a separated cluster in $G$. Since weights of the vertices in the conflict-graph $G_c$ are exactly the cardinality of their corresponding sets,  this implies that $\mathcal{T}$ is a maximum weighted independent set in $G_c$ if and only if $\mathcal{T}$ is a maximum cardinality separated cluster among all the separated cluster in $G$ having their cliques in $\hat{\mathcal{C}}=V(G_c)$. Further, as $\mathcal{S}=\{S_1,S_2,\ldots,S_k\}$ is a maximum cardinality separated cluster in $G$, we have by Lemma~\ref{lem:maxsep} that  $\mathcal{S}=\{S_1,S_2,\ldots,S_k\}\subseteq \hat{\mathcal{C}}=V(G_c)$. This proves the theorem.
\end{proof}
We then have the following corollary due to Theorem~\ref{thm:indptcocomp} and Theorem~\ref{thm:reduction}.
\begin{corollary}
    The \SCP\ problem in interval graphs can be solved  in polynomial time.
\end{corollary}
