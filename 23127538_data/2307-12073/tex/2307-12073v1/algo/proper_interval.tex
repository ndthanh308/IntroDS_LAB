\section*{Proper interval graphs and $3K_1$-free graphs}



%\label{subsec:Proper-interval graphs}
It is known that for the classes of proper interval graphs and $3K_1$-free graphs, the \CDC\ problem can be solved in $O(n^3)$ time~\cite{DBLP:journals/dam/ShaluVS20} (as they are sub-classes of claw-free graphs). Using the framework of auxiliary graph $G^*$ and Theorem~\ref{thm:suff_cdclique}, we show that the \CDC\ problem can be solved in $O(n^{2.5})$ time for the above classes of graphs. 

\begin{observation} \label{obs:propertriangle}
Let $G$ be a proper interval graph or a $3K_1$-free graph. Then $G^*$ is triangle-free.
\end{observation}

\begin{proof}
Let $G$ be a proper interval graph or a $3K_1$-free graph. For the sake of contradiction, assume that $K=\{a,b,c\}\subseteq V(G)=V(G^*)$ is a triangle in $G^*$. Note that for each graph $H\in \mathcal{H}$, we have an independent set of size 3, and therefore, a $3K_1$, and an asteroidal-triple (by Observation~\ref{obs:structureH}). Since $G$ is either a proper interval graph (which is AT-free), or a $3K_1$-free graph, we can therefore conclude that $G$ is an $\mathcal{H}$-free graph. Now, as $K$ induces a triangle in $G^*$, we have that $K$ is an independent set in $G$ (by the definition of $G^*$). Also, by Lemma~\ref{lem:neighborhood}, we have that there exists a vertex $v_c\in V(G)$ such that $K\subseteq N_G(v_c)$. This implies that the graph induced by the set $K\cup \{v_c\}$ is a claw in $G$. This contradicts the fact that $G$ is a proper interval graph or a $3K_1$-free graph (as they are sub-classes of claw-free graphs).   
\end{proof}

Now, before evaluating the complexity of the \CDC\ problem for proper interval graphs and $3K_1$-free graphs, we note down a few observations and a lemma. Note that for any graph $G$, as $G^*=G^2-G$, if $G^2$ can be computed in $O(n^2)$ time, then $G^*$ can also be computed in $O(n^2)$ time (as the subtraction of adjacency matrices of $G^2$ and $G$ takes only $O(n^2)$ time). The first observation is due to the result that squares of proper interval graphs can be computed in $O(n^2)$ time, which is proved in~\cite{DBLP:journals/akcej/PaulPP16}.
\begin{observation}\label{obs:proper}
    For a  proper interval graph $G$, the auxiliary graph $G^*$ can be computed in $O(n^2)$ time. 
\end{observation}
We now have some additional observations.
\begin{observation}\label{obs:cobipartite}
    For a  co-bipartite graph $G$, the auxiliary graph $G^*$ can be computed in $O(n^2)$ time.
\end{observation}
\begin{proof}
    Let $G$ be a co-bipartite graph. Note that for any pair of vertices $u,v\in V(G)$, we have $uv\in E(G^2)$ if and only if $u\in A$, $v\in B$, and either $N_B(u)\neq \emptyset$ or $N_A(v)\neq \emptyset$ or both. This implies that $G^2$ can be constructed in $O(n^2)$ time, and so does $G^*$.
\end{proof}
In the following lemma, we observe an important property of $3K_1$-free graphs that are not co-bipartite, which is crucial in proving Observation~\ref{obs:alpha}.
\begin{lemma}\label{lem:nocobipatite}
    Let $G$ be a $3K_1$-free graph. If $G$ is not co-bipartite, then $G^2$ is a clique.
\end{lemma}

    Let $u$ and $v$ be any two non-adjacent vertices in $G$. Let $d_G(u)$ and $d_G(v)$ denote the degrees of the vertices $u$ and $v$ in $G$, respectively.  We then claim the following:
    
    \smallskip
    
    \noindent\textbf{Claim:} $d_G(u)+d_G(v)\geq n-2$

    Suppose not. Let $d_G(u)+d_G(v) < n-2$. This implies that $|N_G(u)\cup N_G(v)\cup\{u,v\}|<n$. Therefore,  there exists a vertex $w\in V(G)$ such that $w$ is non-adjacent to both $u$ and $v$. Thus $\{u,v,w\}$ forms an independent set in $G$, which contradicts the fact that $G$ is $3K_1$-free. Hence the claim. 

    \smallskip
    Therefore, we can assume that $d_G(u)+d_G(v)\geq n-2$. Suppose that $d_G(u)+d_G(v)> n-2$. Then, as $|V(G)\setminus \{u,v\}|\leq n-2$, it should be the case that $N_G(u)\cap N_G(v)\neq \emptyset$. i.e. there exists a vertex $w\in V(G)$ such that $uw,wv\in E(G)$.  This implies that $uv\in E(G^2)$, as required. Now, consider the case that $d_G(u)+d_G(v)= n-2$. Suppose that $N_G(u)\cap N_G(v)=\emptyset$. Then, as $|N_G(u)\cup N_G(v)\cup \{u,v\}|=n$, we can consider $V(G)$ as the disjoint union of two sets $A=N_G(u)\cup \{u\}$ and $B=N_G(v)\cup \{v\}$. Since $G$ is not a co-bipartite graph, we can assume that at least one of the sets $N_G(u)$ or $N_G(v)$ contains at least two vertices in it.  Suppose that $|N_G(u)|\geq 2$ (resp. $|N_G(v)|\geq 2$). Let $x,y\in N_G(u)$ (resp. $N_G(v)$) be such that $xy\notin E(G)$. Then, as $x,y\notin N_G(v)$ (resp. $N_G(u)$), we then have that $\{x,y,v\}$ (resp. $\{x,y,u\}$) is an independent set in $G$. As this contradicts the fact that  $G$ is $3K_1$-free, we have that both the sets $N_G(u)$ and $N_G(v)$ form a clique. But then, this contradicts the fact that $G$ is not a co-bipartite graph (as $A$ and $B$ now form a two-clique partition of $V(G)$). Thus we can conclude that $N_G(u)\cap N_G(v)\neq \emptyset$, and therefore, $uv\in E(G^2)$, as required. Since $u,v$ is a pair of arbitrary non-adjacent vertices in $G$, we then have that $G^2$ is a clique.
    \begin{observation} \label{obs:alpha}
        For a $3K_1$-free graph $G$, the auxiliary graph $G^*$, can be computed in $O(n^2)$ time.
    \end{observation}
  
    \begin{proof}
        Let $G$ be a $3K_1$-free graph. We can check whether $G$ is co-bipartite or not in $O(n^2)$ time (as it is enough to check whether $\overline{G}$ is bipartite or not). If $G$ is co-bipartite, then by Observation~\ref{obs:cobipartite}, we have that $G^*$ can be computed in $O(n^2)$ time. If $G$ is not co-bipartite, by  Lemma~\ref{lem:nocobipatite}, we have that $G^2$ is a clique, and therefore, the auxiliary graph $G^*$, can be computed in $O(n^2)$ time.
    \end{proof}
    We then have the following theorem.
    \begin{theorem} \label{thm:properalpha}
The \CDC\ problem can be solved in $O(n^{2.5})$ time for proper interval graphs and $3K_1$-free graphs.
\end{theorem}
\begin{proof}
Let $G$ be a proper interval graph or a $3K_1$-free graph. As noted in the proof of Observation~\ref{obs:propertriangle}, we have that $G$ is $\mathcal{H}$-free. Therefore, by Theorem~\ref{thm:suff_cdclique}, we have that $\XCD(G)=k(G^*)$, where $G^*$ is the corresponding auxiliary graph. By Observation~\ref{obs:propertriangle}, we have that $G^*$ is triangle-free. This implies that each clique in any clique cover of $G^*$ is either an edge or a single vertex. This further implies that $k(G^*)=n-|M|$, where $M$ denotes the size of the maximum cardinality matching in $G^*$. Note that if $G$ is a proper interval graph, then the auxiliary graph $G^*$ can be constructed in $O(n^2)$ time by Observation~\ref{obs:proper}, and if $G$ is a  $3K_1$-free graph, then the auxiliary graph $G^*$ can be constructed in $O(n^2)$ time by Observation~\ref{obs:alpha}. Also, the size of the maximum 
 cardinality matching in $G^*$ can be computed in $O(\sqrt{n}m')$ time~\cite{DBLP:journals/siamcomp/HopcroftK73}, where $m'=|E(G^*)|\leq n^2$. Since the overall complexity is $O(n^{2.5})$, this proves the theorem.
\end{proof}

\begin{remark}
    As noted before, the time complexity of the \CDC\ problem for the class of proper interval graphs and $3K_1$-free graphs provided in Theorem~\ref{thm:properalpha} is an improvement over the existing algorithms~\cite{DBLP:journals/dam/ShaluVS20} for the problem in the same classes. Clearly, the class of $3K_1$-free graphs is not $cd$-perfect, as \SCP\ is \NPC\ for $3K_1$-free graphs~\cite{DBLP:conf/caldam/ShaluVS17}. In Section~\ref{sec:interval}, we give a polynomial-time algorithm for \SCP\ problem for the class of interval graphs. However, we note that the parameters $\XCD(G)$ and $\SC(G)$ are not necessarily equal for a proper interval graph $G$ (see an example in Figure~\ref{fig:proper_interval}). Therefore, the class of proper interval graphs, and in general interval graphs are not $cd$-perfect.  
\end{remark}


% Figure environment removed