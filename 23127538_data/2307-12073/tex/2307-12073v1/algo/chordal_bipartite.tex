\section*{ Chordal bipartite graphs} 



 First, we note some easy observations. 
\begin{observation}\label{obs:bipsquare}
Let $G=(A,B,E)$ be a bipartite graph then $G^*=G^2[A]\cup G^2[B]$. 
\end{observation}
\begin{proof}
    The observation follows from the fact that for any pair of vertices $u$ and $v$ in $G$, we have $uv\in E(G^*)$ (or $d_G(u,v)=2$) if and only if either $u,v\in A$ and $\exists$ $w\in B$ such that $u,v\in N_G(w)$  or alternatively, $u,v\in B$ and $\exists$ $w\in A$ such that $u,v\in N_G(w)$. 
\end{proof}
\begin{observation} \label{obs:chordalbip}
    Let $G=(A,B,E)$ be  a chordal bipartite graph. Then $G^*$ is chordal. 
\end{observation}
\begin{proof}
    Suppose not. Since $G^*=G^2[A]\cup G^2[B]$ (by Observation~\ref{obs:bipsquare}), we have that at least one of the subgraphs $G^2[A]$ or $G^2[B]$ is not chordal. Without loss of generality, we can assume that $G^2[A]$ is not chordal (the other case is symmetric).  Let $C_k=a_0,a_1\ldots,a_k,a_0$ be an induced cycle in $G^2[A]$, where $k\geq 3$. Since $G$ is bipartite, and $C_k$ is an induced cycle, this is possible only if for each $i\in \{0,1,\ldots,k\}$, there exists a vertex $b_i$ in $G$ such that $N_G(b_i)\cap V(C_k) = \{a_i,a_{i+1}\}$  (mod $k+1$). This further implies that $a_0,b_0,a_1,b_1,a_2,\ldots,a_k,b_k,a_0$ is an induced cycle of length at least $2k$ in $G$. Since $k\geq 3$, this contradicts the fact that $G$ is a chordal bipartite graph. Hence the observation.
\end{proof}
We then have the following theorem.
 
\begin{theorem}\label{thm:chordalbip}
    Let $G=(A,B,E)$ be  a chordal bipartite graph. Then $G$ is $cd$-perfect.
\end{theorem}
\begin{proof}
    Clearly, $G$ is $\mathcal{H}$-free. By Observation~\ref{obs:chordalbip}, we have that $G^*$ is chordal, and therefore, perfect. Since any induced subgraph $H$ of $G$ is also chordal bipartite, the proof of the theorem now follows from Corollary~\ref{corr:suffcdperfect}.
\end{proof}
 We note the following theorem proved in~\cite{damaschke1990domination}
\begin{theorem}[~\cite{damaschke1990domination}] \label{thm:totdom}
    The problem {\sc Total Dominating set} can be solved in $O(n^2)$ time for chordal bipartite graphs.  
\end{theorem}
Let $G$ be a chordal bipartite graph. As $G$ is triangle-free, by Proposition~\ref{pro:triangle-free}, we have that $\XCD(G)=\gamma_t(G)$. Further, by Theorem~\ref{thm:chordalbip}, we have that $\XCD(G)=\SC(G)$. Thus we have the following theorem by Theorem~\ref{thm:totdom}.
\begin{theorem}
    The problem \CDC\ and \SCP\  can be solved in $O(n^2)$ time for chordal bipartite graphs.
\end{theorem}
