\begin{comment}
    

\section{Appendix}
\label{sec:appendix}

\begin {proposition} 
If $G$ with  $\lvert V(G) \rvert = n$ has maximum independent set size less than or equal to two (that is, $\alpha(G) \leq 2$), then $G^2$ can be computed in $O(n^2)$ time.
\end {proposition}

\begin {proof}
Let $G = (V, E)$ be a simple, undirected graph. It is known that if $G$ is co-bipartite, then $G^2$ can be found in $O(n^2)$ time. If $G^2$ is co-bipartite or not can be decided in $O(n^2)$ time as follows:
\begin {enumerate}
\item Take complement of $G$.  Let it be $G^c$. This can be done in $O(n^2)$ time.
\item Check if $G^c$ is bipartite. Checking if a graph is bipartite or not can be done in $O(n^2)$ time. If $G^c$ is bipartite, then G is co-bipartite; otherwise not.
\end {enumerate}
Thus, if $G$ is co-bipartite then its square can be be found in $O(n^2)$ time.\\
Suppose $G$ is not co-bipartite. Then we show that $G^2$ is clique.
\begin {proof} 
Let $\delta$ be the minimum degree of $G$. Let $d_i$ stand for the degree of $v_i$ in $G$ for $1 \leq i \leq n$. Every pair $(v_i, v_j)$ of non-adjacent vertices will fall in one of the following three cases:   
\begin {enumerate}
\item $d_i + d_j < n-2$
\item $d_i + d_j > n-2$
\item $d_i + d_j = n-2$
\end {enumerate}
We show that $v_iv_j \in E(G^2)$ in each of the above case.
\begin {enumerate}
\item If $ d_i + d_j < n-2 $ then $v_iv_j \in E(G)$ and therefore $v_iv_j \in E(G^2)$. 
\begin {proof} Let $v_iv_j \notin E(G)$. Since $d_i + d_j < n-2$, the number of neighbors of $v_i$ and $v_j$ taken together will be less than $(n-2)$. There are $(n-2)$ vertices in $V - \{v_i, v_j \}$. Therefore, by the pigeonhole principle,  $\exists v_k \in V(G)$ s.t. $v_iv_k, v_jv_k \notin E(G)$. Thus, $\{v_i, v_j, v_k\}$ will be an independent set in $G$. This is a contradiction. Therefore, $v_iv_j \in E(G)$, and in turn $v_iv_j \in E(G^2)$. Hence proved.
\end {proof}

\item If $d_i + d_j > n-2$ then $v_iv_j \in E(G^2)$. 
\begin {proof} 
Since $d_i + d_j > n-2$, the number of neighbors of $v_i$ and $v_j$ taken together will be more than $(n-2)$. There are $(n-2)$ vertices in $V - \{v_i, v_j \}$. Therefore, by the pigeonhole principle,  $\exists v_k \in V(G)$ s.t. $v_k$ is a neighbor of both, $v_i$ and $v_j$. Thus, $v_iv_k, v_jv_k \in E(G)$. Therefore, $v_iv_j \in E(G^2)$. Hence proved. 
\end {proof}

\item If $d_i + d_j = n-2$ then $v_iv_j \in E(G^2)$. 
\begin {proof} 
Suppose $v_iv_j \notin E(G^2)$. It means $N(v_i) \cap N(v_j) = \phi$ in $G$. Since $\lvert V(G) \rvert = n$ and $d_i + d_j = n-2$, ${v_i} \cup N(v_i)$ and ${v_j}  \cup N(v_j)$ are disjoint partitions of $V(G)$. Further, ${v_i} \cup N(v_i)$ and ${v_j} \cup N(v_j)$ are cliques. If ${v_i} \cup N(v_i)$ is not a clique, then there will be two vertices $x, y \in N(v_i)$ s.t. $xy \notin E(G)$. But then the triplet $\{x, y, v_j \}$ will be an independent set of size $3$. This will be a contradiction to the fact that $\alpha(G) \leq 2$. For the same reason, $\{v_j\} \cup N(v_j)$ is a clique. That is, $G$ is a union of two cliques, with some edges connecting the vertices from the two cliques. It means $G$ is a co-bipartite graph. But it contradicts the assumption that $G$ is not co-bipartite. Therefore, $v_iv_j \in E(G^2)$. Hence proved.
\end {proof}
\end {enumerate}
Thus, if $G$ is not co-bipartite, its square can be found in $O(n^2)$ time.
\end {proof}
As $G^2$ can be found in $O(n^2)$ time in both the cases, namely, $G$ is co-bipartite and $G$ is not co-bipartite, we conclude that whenever $G$ has maximum independent size less than or equal to two (that is, $\alpha(G) \leq 2$, the square of $G$ can be found in $O(n^2)$ time. Hence proved.
\end {proof}
\end{comment}