\begin{abstract}

Domination and coloring are two classic problems in graph theory. In this paper, our major focus is on the  \CDC\ problem, which incorporates the flavors of both domination and coloring in it. Let $G$ be an undirected graph.  A proper vertex coloring of $G$ is said to be a \textit{cd-coloring}, if each color class has a dominating vertex in $G$. The minimum integer $k$ for which there exists a \textit{cd-coloring} of $G$ using $k$ colors is called the  \textit{cd-chromatic number} of $G$, denoted as $\XCD(G)$. A set $S\subseteq V(G)$ is said to be a \textit{total dominating set}, if any vertex in $G$ has a neighbor in $S$. The \textit{total domination number} of $G$, denoted as $\gamma_t(G)$, is defined to be the minimum integer $k$ such that $G$ has a total dominating set of size $k$. A set $S\subseteq V(G)$ is said to be a \textit{separated-cluster} (also known as \textit{sub-clique}) if no two vertices in $S$ lie at a distance 2 in $G$. The \textit{separated-cluster number} of $G$, denoted as $\SC(G)$, is defined to be the maximum integer $k$ such that $G$ has a separated-cluster of size $k$.

\smallskip

In this paper, we contribute to the literature connecting \CDC\ problem with the problems, \TD\ and \SCP. For any graph $G$, we have $\XCD(G)\geq \gamma_t(G)$ and $\XCD(G)\geq \SC(G)$. 
First, we explore the connection of the {\sc CD-Coloring} problem to the well-known problem {\TD}. Note that several bounds for $\gamma_t$ for regular graphs are studied in the literature. It is also known that \TD\ is \NPC\ for \textit{triangle-free $3$-regular graphs}. 
We generalize this result by proving that both the problems {\sc CD-Coloring} and \TD\ are \NPC, and do not admit any subexponential-time algorithms on \textit{triangle-free $d$-regular graphs, for each fixed integer $d\geq 3$}, assuming the Exponential Time Hypothesis. We also study the relationship between the parameters $\XCD(G)$ and $\SC(G)$. As in classic coloring, here also we can see that there exists a family of graphs $G$ with $\SC(G)=2$ and  $\XCD(G)$ to be arbitrarily large. Therefore, analogous to the well-known notion of \textit{`perfectness'}, here we introduce the notion of \textit{`cd-perfectness'}. We prove a sufficient condition for a graph $G$ to be \textit{cd-perfect} (i.e. $\XCD(H)= \SC(H)$, for any induced subgraph $H$ of $G$). For certain graph classes (like, \textit{triangle-free} graphs), our sufficient condition is also necessary.  
Here, we propose a generalized framework, via which we obtain several exciting consequences:  (1) in providing a unified approach for finding polynomial-time algorithms for various special graph classes, and (2) in proving hardness results for certain classes of graphs. In addition to this, we settle an open question by proving that the \SCP\ problem is \textit{solvable in polynomial time for the class of interval graphs}.
\end{abstract}