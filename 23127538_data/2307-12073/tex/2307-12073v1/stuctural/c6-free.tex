\subsubsection{Hardness implications}
In contrast to the previous section, here, we see an application of Proposition~\ref{pro:omega_s} and Theorem~\ref{thm:suff_cdclique} in deriving hardness results for \CDC\ and \SCP. 

\section*{$C_6$-free bipartite graphs}

%\label{subsec:c6-free}
We prove the hardness by proposing a reduction from the problems \CC\ and \ISP\ for diamond-free graphs to the problems \CDC\ and \SCP\ for $C_6$-free bipartite graphs, respectively. Note that both the problems, {\sc Clique Cover} and {\sc Independent Set}, are NP-hard for diamond-free graphs~\cite{kral2001complexity,Poljak1974}. Recall that, a \textit{a diamond} is a graph obtained by deleting an edge from $K_4$ (complete graph on 4 vertices)

First, we note the following observation for diamond-free graphs~\cite{chiarelli2021strong}.

\begin{observation}\label{obs:diamondfree}
Any diamond-free graph with $m$ edges can have at most $m$ maximal cliques.
\end{observation}

Let $G$ be a diamond-free graph and let $\mathcal{K}=\{K_1,K_2,\ldots,K_l\}$ denote the collection of all maximal cliques in $G$. By Observation~\ref{obs:diamondfree}, we have that $l\leq |E(G)|$. From $G$, we now construct a bipartite graph $B_G=(A,B,E)$ as follows: we define $A=V(G)\cup\{u\}$ and $B=\mathcal{K}=\{K_1,K_2,\ldots,K_l\}$ (each vertex in the partite set $B$ represents a maximal clique in $G$). For a pair of vertices, $a\in A\setminus \{u\}$ and $K_j\in B$ (where $j\in \{1,2,\ldots,l\}$), we make $a$ and $K_j$ adjacent in $B_G$ if and only if $a\in K_j\subseteq V(G)$. In addition, we make the vertex $u$ adjacent to all the vertices in $B$. i.e. $E(B_G)=\{ab:a\in A\setminus \{u\}, b=K_j\in B$, for some $j\in \{1,2,\ldots,l\}$, and $a\in K_j\subseteq V(G)\}\cup \{ub:b\in B\}$. We then have the following lemmas.

\begin{lemma}\label{lem:isomprphic}
Let $G$ be a diamond-free graph and $B_G=(A,B,E)$ be the corresponding bipartite graph (as defined above). We have $B_G^2[A\setminus \{u\}]\cong G$ and $B_G^2[B]$ is a clique.
\end{lemma}
\begin{proof}
Let $H=B_G^2[A\setminus \{u\}]$. Clearly, $V(H) = V(G)$. Let $x,y\in V(H)$. Then, as $B_G$ is a bipartite graph, and $x,y\in V(H)\subseteq A\setminus \{u\}$, we have $xy\in E(H)\iff \exists z=K_j\in B$ for some $j\in \{1,2,\ldots,l\}$ such that $x,y\in N_{B_G}(z)$ (i.e. both the vertices $x$ and $y$ belong to a same maximal clique in $G$) $\iff xy\in E(G)$. This proves that $B_G^2[A\setminus \{u\}]\cong G$. Further, it is easy to see that $B_G^2[B]$ is a clique, since $B$ is an independent set and $B\subseteq N_{B_G}(u)$.
\end{proof}

\begin{lemma}\label{lem:diamondC_6}
    Let $G$ be a diamond-free graph and $B_G=(A,B,E)$ be the corresponding bipartite graph. Then $B_G$ is $C_6$-free.
\end{lemma}
\begin{proof}
    Suppose not. Let $S=\{a_1,b_1,a_2,b_2,a_3,b_3,a_1\}$ be a set of vertices in $B_G$ that induces a $C_6$ in $B_G$, where $\{a_1,a_2,a_3\}\subseteq A$ and $\{b_1,b_2,b_3\}\subseteq B$.  Clearly, $\{a_1,a_2,a_3\}$ induces a triangle in $B_G^2[A]$. Since each of the vertex $a_i$ has a non-neighbor in $B$, we have $u\neq a_i$ for any $i\in \{1,2,\ldots,3\}$. This implies that $\{a_1,a_2,a_3\}$ induces a triangle in $B_G^2[A\setminus \{u\}]$, and thus a triangle in $G$ by Lemma~\ref{lem:isomprphic}. Now, consider the vertex $b_1$. We have $N_{B_G}(b_1)\cap S =\{a_1,a_2\}$. Clearly, $\{a_1,a_2\}$ is not a maximal clique in $G$, since $\{a_1,a_2,a_3\}$ induces a triangle in $G$. Note that $b_1=K$, where $K$ is some maximal clique in $G$. As $a_3\notin N_{B_G}(b_1)$, we have by the definition of $B_G$ that $a_3\notin K$. Since $K$ is a maximal clique in $G$ containing the set $\{a_1,a_2\}$, but not the vertex $a_3$, there exists a vertex $a_4\in V(G)\subseteq A$ such that $a_4\in K$ and $a_3a_4\notin E(G)$. This implies that $\{a_1,a_2,a_3,a_4\}$ induces a diamond in $G$. This contradicts the fact that $G$ is diamond-free.
\end{proof}
\begin{theorem}\label{thm:C_6free hard}
    The problems  \CDC\ and \SCP\ are NP-Compete for $C_6$-free bipartite graphs.
\end{theorem}
\begin{proof}
    Let $G$ be a diamond-free graph, and let $B_G$ be the corresponding bipartite graph. By Proposition~\ref{pro:omega_s} applied to the bipartite graph $B_G$, we have that $\SC(B_G)=\alpha(B_G^*)$, where $B_G^*=B_G^2- B_G$ = $B_G^2[A]\uplus B_G^2[B]$. Therefore,  we have $\SC(B_G)=\alpha(B_G^*) = \alpha(B_G^2[A])+\alpha(B_G^2[B])$ = $\alpha(B_G^2[A\setminus \{u\}])+\alpha(B_G^2[B])$ (the last equality is due to the fact that, in the graph $B_G^2[A]$, the vertex $u$ is adjacent to every vertex in $A\setminus \{u\}$). This implies that $\SC(B_G)=\alpha(B_G^*) = \alpha(G)+1$ (by Lemma~\ref{lem:isomprphic}). Note that $G$ is diamond-free and $B_G$ is a $C_6$-free bipartite graph (by Lemma~\ref{lem:diamondC_6}). Therefore, the fact that \ISP\ problem is NP-hard for diamond-free graphs implies that \SCP\ is NP-hard for $C_6$-free bipartite graphs. 

    \medskip

    By Lemma~\ref{lem:diamondC_6}, we have that $B_G$ is $C_6$-free. Since $B_G$ is also triangle-free, we can therefore conclude that $B_G$ is $\mathcal{H}$-free. Therefore, by Theorem~\ref{thm:suff_cdclique}, we have that $\XCD(B_G)=k(B_G^*)$. As before, since $B_G^*= B_G^2[A]\uplus B_G^2[B]$, we then have $\XCD(B_G)=k(B_G^*) = k(B_G^2[A])+k(B_G^2[B])$ = $k(B_G^2[A\setminus \{u\}])+k(B_G^2[B])$ (again, the last equality is due to the fact that, in the graph $B_G^2[A]$, the vertex $u$ is adjacent to very vertex in $A\setminus \{u\}$). This implies that $\XCD(B_G)=k(B_G^*) = k(G)+1$ (by Lemma~\ref{lem:isomprphic}). As noted before, $G$ is diamond-free and $B_G$ is a $C_6$-free bipartite graph. Therefore,  the fact that \CC\ problem is NP-hard for diamond-free graphs implies that \CDC\ is NP-hard for $C_6$-free bipartite graphs. Hence the theorem.
\end{proof}