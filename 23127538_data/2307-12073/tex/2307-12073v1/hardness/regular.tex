 

\begin{construction}
    \label{cons:d-regular odd}
     Let $G$ be a triangle-free $(d-2)$-regular graph, for any odd integer $d\geq 5$. 
     Let $W$ be a gadget which is constructed in the following way. Introduce a star graph $K_{1,d-1}$, which we denote by $S$, having the vertex $a$ as the root vertex. Note that we also call $`a$' as the root vertex of $W$. Further introduce $2(d-1)^2$ vertices which induces $d-1$ disjoint copies of complete bipartite graphs $K_{d-1,d-1}$, namely $CB_{1},  CB_{2}\ldots,CB_{(d-1)}$, respectively.      
     The adjacency between these complete bipartite graphs $CB_i$, for $1\leq i\leq (d-1)$, and the star graph $S$ is in such a way that the vertices of one of the partitions $A_i$ of the complete bipartite graph $CB_i$ is connected to the leaf vertex $i$  of $S$.
     Furthermore, for each odd integer $i\leq (d-2)$, each vertex of the partition $B_i$ of $CB_i$ is adjacent to exactly one vertex of the partition $B_{i+1}$ of $CB_{i+1}$ ( for instance refer  Figure~\ref{fig:d-regular_gadget odd}, when $d=5$). The gadget $W$ contains $(2d^2-3d+2)$ of vertices.  
     Now the graph $G_c$ is constructed from $G$ using $W$ in such a way that for each vertex $u$ in $G$, introduce two copies of $W$ such that the root (the vertex $a$) of each copy of $W$ is attached to $u$. Note that  $G_c$ contains $2n(2d^2-3d+2)$ vertices, where $n$ is the number of vertices in $G$.
\end{construction}


An example of the gadget $W$ in Construction~\ref{cons:d-regular odd} is shown in Figure~\ref{fig:d-regular_gadget odd}

% Figure environment removed

Let $W$ be a gadget constructed in Construction~\ref{cons:d-regular odd}. Let $H$ be a subgraph of $W$ induced by the vertices in two complete bipartite graphs $CB_i$ and $CB_{i+1}$, for an odd integer $i\leq d-3$, and the two leaves $b$ and $c$ of the star graph $S$ which are adjacent to the partition $A_i$ and  $A_{i+1}$ of $CB_i$ and $CB_{i+1}$, respectively. Let $H'=H-\{b,c\}$. Then, we have the following observation which is true for the gadget $W$ of Construction~\ref{cons:d-regular odd}. 
\begin{observation}
    \label{obs:Hcoloring_regular_odd} 
     Let $W$ be a gadget constructed in Construction~\ref{cons:d-regular odd}.
     Let $H$ be a subgraph of $W$ induced by the vertices in two complete bipartite graphs $CB_i$ and $CB_{i+1}$, for an odd integer $i\leq d-3$, and the two leaves $b$ and $c$ of the star graph $S$ which are adjacent to the partition $A_i$ and  $A_{i+1}$ of $CB_i$ and $CB_{i+1}$, respectively. 
     Let 
     $H'=H-\{b,c\}$. 
     In any CD-coloring of $H$, the vertices in $H'$ require at least 4 colors (for instance, refer Figure~\ref{fig:d-regular_gadget odd} for $H$ and $H'$, when $d=5$). Moreover, there exists a $cd$-coloring of $W$ using exactly $2(d-1)$ colors.
\end{observation}

\begin{proof}
    The graph induced by $H'$ contains a $C_6$ (for instance: induced by the vertices $d1,e1,g4,f4,g3,e2$ as shown in Figure~\ref{fig:d-regular_gadget odd}). 
    It is clear that the \XCD($C_6$)=4. 
    Now consider the subgraph  $H'$. It is obtained by 
    the introduction of new vertices adjacent to some vertices of the $C_6$.   Since none of the $3K_1$s present in the $C_6$ is dominated by any of the remaining vertices in $H'$, by Observation~\ref{obs:cdcolor_reduce}, it is clear that the number of colors needed for any $cd$-coloring of $H'$ is at least 4. Further, note that all the independent sets in $H'$, which are dominated by $b$ or $c$, are already dominated by some vertex in $H'$. Therefore, again by Observation~\ref{obs:cdcolor_reduce}, it is clear that the number of colors needed for $V(H')$ in any $cd$-coloring of $H$ is at least 4. 
    Since the root vertex $a$ of the gadget is not adjacent to any of the vertices in $H'$, the presence of $a$ will not reduce the number of colors needed for $cd$-coloring of the subgraph $H'$ of the gadget $W$. Hence the number of colors needed for $cd$-coloring of $H$ is at least $4$. Therefore, the number of colors needed for $cd$-coloring of $W$ is at least $2(d-1)$, as there are $(d-1)/2$ copies of $H$ is present in $W$. 
    Now it is easy to see that a 4-$cd$-coloring of a $C_6$ in each $H'$ can be  extended to a $2(d-1)-cd$-coloring of the gadget $W$ (by using the leaves of star graph $S$, and one of the vertices from the partition $A_i$ of each $CB_i$, for $1\leq i\leq d-1$, as the dominating vertices of the color classes). 
\end{proof}

The following lemma is based on the Construction~\ref{cons:d-regular odd}.


\begin{lemma}
    \label{lem:d-regular odd}
    For odd integer $d\geq 5$, let $G$ be a triangle-free ($d-2$)-regular graph, having $n$ vertices. Then $\XCD(G)=k$ if and only if $\XCD(G_c)=k+4n(d-1)$. 
\end{lemma}

\begin{proof}
    Let $\XCD(G)=k$. We claim that $\XCD(G_c)=k+4n(d-1)$.  Recall that the graph $G_c$ contains $2n$ gadgets as there are $n$ vertices in $G$ each having degree $d-2$. By Observation~\ref{obs:Hcoloring_regular_odd}, it is clear that these gadgets need a total of at least $4n(d-1)$ colors. Since \XCD($G$)=$k$, we then have \XCD($G_c$)=$k+4n(d-1)$. 

\smallskip

    Now for the reverse direction, let $\XCD(G_c)=m$. Then we claim that $\XCD(G)=k$, where $k=m-4n(d-1)$. 
    Note that for each gadget $W$, the subgraph $W- \{a\}$ is disjoint from $G$, as only $a$ in $W$ is adjacent to exactly one  vertex  $u$ in $G$. Hence, the color class dominated by $a$ can include only one vertex $u$ in $G$. 
     By Observation~\ref{obs:Hcoloring_regular_odd}, in any $cd$-coloring of $G_c$, the vertex set of each copy of $H'$ of any gadget $W$ needs at least 4 colors, and none of these colors can be reused for coloring the vertices in $V(G_c)\setminus V(W)$. Again by Observation~\ref{obs:Hcoloring_regular_odd}, the same 4 colors used to color the vertices in $H'$ can be reused to color the entire vertices of $H$. Hence, the 2($d-1$) colors used for $(d-1)/2$ copies of $H$ in $W$ can be reused to color the entire vertices of $W$. Therefore, we now have a $cd$-coloring of $G_c$ using $m$ colors such that no colors used for the vertices in gadgets are used to color any vertex in $G$. Thus, $\XCD(G)=m-4n(d-1)=k$, as there are $2n$ copies of $W$s in $G_c$.  
\end{proof}

