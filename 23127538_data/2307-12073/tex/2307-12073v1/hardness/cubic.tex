
% Figure environment removed
\begin{construction}
\label{cons:cubic}
    Let $G$ be a bipartite graph with bounded degree 3. We construct a gadget $W$ in the following way: Introduce a star graph $K_{1,2}$, which we denote by $S$, having $a$ as the root vertex, and $b,c$ as the leaves. Note that we also call $`a$' as the root vertex of $W$. Then we introduce two copies of the complete bipartite graph $K_{2,2}$, denoted by $CB_1$ and $CB_2$, respectively. Now the leaf $b$ (resp. $c$) of $S$ is adjacent to one of the partition $A_1$ (resp. $A_2$) of $CB_1$ (resp. $CB_2$).
    Furthermore,  each vertex of partition $B_1$ of $CB_1$ is adjacent to exactly one vertex of the other partition $B_{2}$ of $CB_{2}$  as shown in Figure~\ref{fig:cubic_gadget}. 
    Thereafter we construct a graph $G_c$ from $G$ using the gadget $W$ as follows: for every vertex $u$ of $G$ with degree 2, introduce a gadget $W$ such that the root vertex $a$ of $W$ is adjacent to $u$ in $G$. Similarly, for each  vertex $v$ in $G$ of degree 1, we introduce  two copies of  $W$ such that the root vertices of both the gadgets  are adjacent to $v$ in $G$.   
    The graph $G_c\setminus G$ contains $11x+22y$ vertices, where $x$ and $y$ denote the number of vertices in $G$ having degree 2 and degree 1, respectively. 
\end{construction}

 % Figure environment removed  

An example of the Construction~\ref{cons:cubic} corresponding to the bipartite graph $G$ with bounded degree 3 given in Figure~\ref{fig:bipartite} is as shown in Figure~\ref{fig:Cubic construction}.

 % Figure environment removed

Let $W$ be a gadget constructed in Construction~\ref{cons:cubic}. Let $H$ be a subgraph of $W$ induced by the vertices in two complete bipartite graphs $CB_1$ and $CB_2$, and the two leaves $b$ and $c$ of the star graph $S$.  Let $H'=H-\{b,c\}$. The following observation is true for the gadget $W$ in Construction~\ref{cons:cubic}.
\begin{observation}
    \label{obs:Hcoloring_cubic} 
     Let $W$ be the gadget defined in Construction~\ref{cons:cubic}.
     Let $H$ be a subgraph of $W$ induced by the vertices in two complete bipartite graphs $CB_1$ and $CB_{2}$, and the two leaves $b$  and $c$ of the star graph $S$. Let 
     $H'=H- \{b,c\}$. In any CD-coloring of $H$, the vertices in $H'$ require at least 4 colors. (Refer Figure~\ref{fig:cubic_gadget} for $H$ and $H'$). Moreover, there exists a $cd$-coloring of $W$ using exactly 4 colors.
\end{observation}

\begin{proof}
    The graph $H'$ contains an induced $C_6$ (for instance, the graph induced by the vertices $d,i,j,f,k,h$ as shown in Figure~\ref{fig:cubic_gadget}). It is easy to see that that the \XCD($C_6$)=4. Now consider the subgraph $H'$. It is obtained by 
     introducing new vertices adjacent to some vertices of the $C_6$.  Since none of the $3K_1$s present in the $C_6$ is dominated by any of the remaining vertices in $H'$, by Observation~\ref{obs:cdcolor_reduce}, it is clear that the number of colors needed for any $cd$-coloring of $H'$ is at least 4.  Further, note that all the independent sets in $H'$, which are dominated by $b$ or $c$, are already dominated by some vertex in $H'$. Therefore, again by Observation~\ref{obs:cdcolor_reduce}, it is clear that the number of colors needed for $V(H')$ in any $cd$-coloring of $H$ is at least 4. 
     Since the root vertex $a$ of the gadget is not adjacent to any of the vertices in $H'$, the presence of $a$ will not reduce the number of colors needed for $cd$-coloring of the subgraph $H'$ of the gadget $W$. Hence the number of colors needed for $cd$-coloring of $W$ is at least $4$. 
    Now it is easy to see that a 4-$cd$-coloring of any $C_6$ in $H'$ can be  extended to a 4-$cd$-coloring of the gadget $W$ (by using the leaves of star graph $S$, and one of the vertices from each $CB_i$, for $i=\{1,2\}$ as the dominating vertices of the color classes).
\end{proof}



The following lemma is based on the Construction~\ref{cons:cubic}, and as a consequence of which we have Theorem~\ref{thm:cubic}.

\begin{lemma}
    \label{lem:cubic}
    Let $G$ be a bipartite graph with bounded degree 3. Then $\gamma_t(G)=k$ if and only if $\XCD(G_c)=k+4x+8y$, where  $x$ and $y$  denote the number of vertices in $G$ having degree 2 and degree 1, respectively. 
\end{lemma}

\begin{proof}
    Let $\gamma_t(G)=k$. Then by Proposition~\ref{pro:triangle-free}, since $G$ is triangle-free, we have $\XCD(G)=k$. We claim that $\XCD(G_c)=k+4x+8y$.  Recall that the graph $G_c$ contains $x+2y$ gadgets, as there are $x$ number of degree two vertices, and $y$ number of degree one vertices in $G$. By Observation~\ref{obs:Hcoloring_cubic}, it is clear that these gadgets require exactly $4(x+2y)$ new colors, which are not used in $G$ (as none of these colors can be reused for coloring the vertices in $V(G_c)\setminus V(W)$).  Since \XCD($G$)=$k$, we then have \XCD($G_c$)=$k+4x+8y$.
    
\smallskip

    Now for the reverse direction, let $\XCD(G_c)=m$. Then, we claim that $\gamma_t(G)=k$, where $k=m-(4x+8y)$. 
    Note that for each gadget $W$, the subgraph $W- \{a\}$ is disjoint from $G$, as only the vertex $a$ in $W$ is adjacent to exactly one vertex  $u$ in $G$. Hence the color class dominated by $a$ can include only one vertex $u$ in $G$.  By Observation~\ref{obs:Hcoloring_cubic}, in any $cd$-coloring of $G_c$, the vertex set $H'$ of any gadget $W$ needs at least 4 colors, and none of these colors can be reused for coloring the vertices in $V(G_c)\setminus V(W)$. Again by Observation~\ref{obs:Hcoloring_cubic}, the same 4 colors used to color the vertices in $H'$ can be reused to color the entire vertices of $W$. Therefore, we now have a $cd$-coloring of $G_c$ using $m$ colors such that no colors used for the vertices in gadgets are used to color any vertex in $G$. Thus, $\XCD(G)=m-4(x+2y)=k$. Then by Proposition~\ref{pro:triangle-free}, $\gamma_t(G)=k$.
\end{proof}





\begin{theorem}
    \label{thm:cubic}
    \CDC\ is \NPC\ on triangle free cubic graphs. \TETHS.
\end{theorem}

\begin{proof}
    Notice that $G_c$ has $|V(G)|+ 11(x+2y)$ vertices, where $x$ and $y$ are the number of vertices in $G$ having degree 2 and degree 1, respectively. We know that there is a linear reduction from \textsc{Total Domination} in bipartite graph with bounded degree 3 to \CDC\ on triangle free cubic graphs due to Lemma~\ref{lem:cubic}. Thus, by using Proposition~\ref{pro:bipartite}, we are done.
\end{proof}


