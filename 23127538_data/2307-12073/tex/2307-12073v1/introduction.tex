\section{Introduction}
\label{sec:intro}
In graph theory, it is usual to study the relationship (if exists) between any two graph parameters, and further investigate  structural conditions for the graphs for which these two parameters coincide. For instance, for a given graph, the \textit{chromatic number}, $\chi$, and the \textit{clique number}, $\omega$, are two such classic parameters in the literature. It is a well-known fact that for any graph $G$, $\chi(G)\geq \omega(G)$. This inequality inspires us to look at many related questions: 
How big the difference between these two parameters can be? Or even restrictively, whether there exists a family of graphs $G$ having $\omega(G)$ to be low, but arbitrarily high values for $\chi(G)$. The famous \textit{`Mycielskian'} construction answers this question by providing us with an elegant construction~\cite{mycielski1955coloriage} for a family of triangle-free graphs $G$ (i.e. $\omega(G)$=2), having arbitrarily large chromatic number. 
Therefore, it is a valid and interesting question to ask about the structural properties of the graphs for which $\chi(G)=\omega(G)$. The celebrated \textit{`Strong Perfect Graph Theorem'}~\cite{chudnovsky2006strong} completely characterizes the graphs $G$ having the property that for any induced subgraph $H$ of $G$, $\chi(H)= \omega(H)$. Moreover, these interrelationships between different graph parameters may have interesting consequences in the algorithmic complexity of the corresponding problems on special graph classes.

\smallskip

Along this line of research, here we explore the interconnection between a few graph parameters. 
Domination and coloring are two classic problems in graph theory. 
We focus on some graph-theoretic problems, which incorporate the flavors of both domination and coloring in it. Let $G$ be an undirected graph.  A proper vertex coloring of $G$ is said to be a \textit{cd-coloring} if each color class has a dominating vertex in $G$. The minimum integer $k$ for which there exists a \textit{cd-coloring} of $G$ using $k$ colors is called the  \textit{cd-chromatic number} of $G$, denoted as $\XCD(G)$. Given an input graph $G$, the problem \CDC\ seeks to find the \textit{cd-chromatic number} of $G$. The \CDC\  is known to be \NPC\ for several special classes of graphs, including bipartite graphs, chordal graphs, etc.~\cite{DBLP:journals/gc/MerouaneHCK15,DBLP:journals/dam/ShaluVS20} and polynomial-time solvable for the classes of trees, co-bipartite graphs, split graphs, claw-free graphs, etc.~\cite{DBLP:conf/caldam/ShaluVS17,DBLP:journals/dam/ShaluVS20,DBLP:conf/caldam/ShaluK21}. Shalu.et.al~\cite{DBLP:journals/dam/ShaluVS20} obtained a complexity dichotomy for \CDC\ for $H$-free graphs. \CDC\  is also studied in the paradigm of parameterized complexity~\cite{DBLP:conf/mfcs/AbhinavBBKNOS22,DBLP:journals/dam/KrithikaRST21}. In addition to its theoretical significance, \CDC\ has a wide range of practical applications in social networks~\cite{chen2014dominated} and genetic networks~\cite{klavvzar2021dominated}. 


A set $S\subseteq V(G)$ is said to be a \textit{total dominating set}, if any vertex in $G$ has a neighbor in $S$. 
The \textit{total domination number} of $G$, denoted as $\gamma_t(G)$, is defined to be the minimum integer $k$ such that $G$ has a total dominating set of size $k$. Given an input graph $G$, in problem \TD, we intend to find the total domination number of $G$. Total domination is one of the most popular variants of domination, and there are hundreds of research papers dedicated to this notion in the literature~\cite{DBLP:journals/entcs/HoppenM19,DBLP:journals/gc/MerouaneHCK15,henning2013total,DBLP:conf/interaction/Zhu09}. In one part of this paper, we will get to see, how a relatively new graph parameter, like \textit{cd-coloring}, sheds light on a classic problem of total domination. 

A set $S\subseteq V(G)$ is said to be a \textit{separated-cluster} (also known as \textit{sub-clique}) if no two vertices in $S$ lie at a distance 2 in $G$. The \textit{separated-cluster number} of $G$, denoted as $\SC(G)$, is defined to be the maximum integer $k$ such that $G$ has a separated-cluster of size $k$. Given an input graph $G$, in \SCP,\ our goal is to find the \textit{separated-cluster number} of $G$.  We introduce this name, \textit{separated-cluster} because, if no two vertices in $S$ lie at a distance 2 in $G$, then $S$ is, in fact, a disjoint union of cliques, where any pair of cliques are separated by a distance greater than 2 (i.e. no two vertices belonging to two distinct cliques in $S$ can be either adjacent or have  a common neighbor). Using this perspective, we can infer that \SCP\ problem is a restricted version of the well-known problem, {\sc Cluster Vertex Deletion}, where we intend to find  minimum number of vertices whose deletion results in a disjoint union of cliques~\cite{DBLP:journals/algorithmica/GrammGHN04} (or, alternatively, to find maximum number of vertices that can be partitioned into disjoint cliques). The problem \SCP\ is known to be \NPC\ for bipartite graphs, chordal graphs, $3K_1$-free graphs etc.~\cite{DBLP:conf/caldam/ShaluVS17} and polynomial-time solvable for trees, co-bipartite graphs, cographs, split graphs etc.~\cite{DBLP:conf/caldam/ShaluVS17,DBLP:conf/caldam/ShaluK21}.
 
\subsubsection*{Total domination and $cd$-coloring}

Domination and graph coloring problems are often in relation. There exist variants of graph coloring problems in the literature that concern the domination problem~\cite{DBLP:journals/dm/ChellaliV04,DBLP:journals/gc/ChellaliM12,DBLP:conf/caldam/ShaluVS17,DBLP:conf/caldam/ShaluK21}. The close relationship between \CDC\ and \TD\ was established by Merouane et al.~\cite{DBLP:journals/gc/MerouaneHCK15}. They proved that for any triangle-free graph $G$, we have $\gamma_t(G)=\XCD(G)$. 
It is also known that \TD\ is \NPC\ for bipartite graphs with bounded degree 3 and triangle-free cubic graphs~\cite{DBLP:conf/interaction/Zhu09}.  Even though several bounds for $\gamma_t$ are known for regular graphs~\cite{DBLP:journals/entcs/HoppenM19},
to the best of our knowledge, the complexity of \TD\ remains unknown for triangle-free $d$-regular graphs for each fixed integer $d\geq 4$. (Note that the reduction for the hardness given in~\cite{DBLP:conf/interaction/Zhu09} for triangle-free cubic graphs does not seem to have an easy generalization).


\medskip

\noindent\textbf{Our results:}
Here, using the fact that $\gamma_t(G)=\XCD(G)$ for triangle-free graphs $G$~\cite{DBLP:journals/gc/MerouaneHCK15}, we prove that both the problems \CDC\ and \TD\ do not admit any subexponential-time algorithms on \textit{triangle-free $d$-regular graphs, for each fixed integer $d\geq 3$}, assuming the Exponential Time Hypothesis. In particular, we first provide a linear reduction from \TD\ in bipartite graphs with bounded degree 3 to \CDC\ in triangle-free cubic graphs. Further, we generalize this result by showing that \CDC\ is \NPC\ for triangle-free $d$-regular graphs for any fixed integer $d\geq 3$. This implies that \TD\ on triangle-free $d$-regular graphs is \NPC\ for any fixed integer $d\geq 3$. It is easy to see that both the problems can be solved in polynomial time for triangle-free $d$-regular graphs, when $d\leq 2$. Also, note that the difference between $\XCD$ and $\gamma_t$ can be arbitrarily high even for regular graphs in general (for example, for complete graphs, $K_r$ on $r$ vertices, we have, $\XCD(K_r)=r$, whereas $\gamma_t(K_r)=2$). Here, we infer a stronger observation by proving that for each fixed integer $d\geq 3$, there exists a family of connected $d$-regular graphs for which the difference between $\XCD$ and $\gamma_t$ is arbitrarily high.


\subsubsection*{Separated-Cluster and $cd$-coloring}

It is interesting to note that the notions of \textit{cd-chromatic number}, $\XCD$ and \textit{separated cluster number}, $\SC$, also follow a similar relationship as their classic counterparts of \textit{chromatic number}, $\chi$ and \textit{clique number}, $\omega$. Since any two vertices in a separated cluster cannot lie at a distance of 2 in the graph, for any graph $G$, we have $\XCD(G)\geq \SC(G)$. Using the same idea of \textit{`Mycielskian'} construction~\cite{mycielski1955coloriage} (as in classic coloring), we can have graphs for which $\omega_s=2$ but $\XCD$ to be arbitrarily high. 
Here we introduce the concept of \textit{`cd-perfectness'} and see how it relates to the classic notion of perfect graphs. Recall that, given a graph $G$, in \SCP, our goal is to find a subset of vertices in $G$ such that no two vertices in the set lie at a distance 2 in $G$. A natural approach to this problem is by considering the following auxiliary graph (the graph $G^*$ defined below is the same as in~\cite{DBLP:conf/caldam/ShaluVS17}). 

\begin{definition}[Auxiliary graph $G^*$]\label{def:aux}
    Given a graph $G$, the auxiliary graph $G^*$ is the graph having $V(G^*)=V(G)$ and $E(G^*)= \{uv:u,v\in V(G)$ and $d_G(u,v)=2\}$.
\end{definition}
Then, it is easy to observe that $\SC(G)=\alpha(G^*)$, where $\alpha(G^*)$ is the independence number (the size of maximum cardinality independent set) of $G^*$.

\medskip

\noindent\textbf{Our results:} Given a graph $G$, here, we note that $\XCD(G)\geq k(G^*)$, where $ k(G^*)$ is the clique cover number (the size of minimum number of cliques needed to partition the vertex set) of $G^*$. Note that the parameters $\alpha$ and $k$ are well-studied in the literature, particularly in connection with \textit{perfect graphs}. This motivated us to initiate the study of \textit{`cd-perfectness'}. i.e. analogous to the well-known notion of perfect graphs in the literature, here, we study the structural properties of graphs $G$, having the property that for any induced subgraph $H$ of $G$, $\XCD(H)= \SC(H)$. In some of the earlier works in $cd$-coloring, the researchers have observed that for certain classes of graphs, say $\mathcal{C}$, we have $\XCD(G)=\SC(G)$, for each graph, $G\in \mathcal{C}$~\cite{DBLP:conf/caldam/ShaluVS17,DBLP:conf/caldam/ShaluK21}. But to the best of our knowledge, no one has tried to study the structural properties of the graphs for which these two parameters coincide.  In this paper, we introduce a new approach that helps us to provide a sufficient condition for the graphs to be \textit{`cd-perfect'} (a graph $G$ is \textit{cd-perfect}, if for any induced subgraph $H$ of $G$ we have $\XCD(H)=\SC(H)$). For certain classes of graphs, like triangle-free graphs, we prove that this sufficient condition is also necessary. Interestingly, this framework of relating $\XCD(G)$ and $\SC(G)$  to $k(G^*)$ and $\alpha(G^*)$, respectively, together with the notion of $cd$-perfectness has consequences in deriving both positive and negative results concerning the algorithmic complexity of the problems \CDC\ and \SCP.

\begin{itemize}
    \item This technique helps us to derive a unified approach (mostly with an improvement) for guaranteeing polynomial-time algorithms to solve \CDC\  for various graph classes, including chordal bipartite graphs, proper interval graphs, $3K_1$-free graphs, etc.
    
    \smallskip
    
    \item The same tool can be used to derive hardness results for both the problems, \CDC\ and \SCP\, for certain graph classes like, $C_6$-free bipartite graphs. 
\end{itemize}
Here, we also settle an open problem proposed by Shalu et.al.~\cite{DBLP:conf/caldam/ShaluVS17}, by proving that \SCP\ can be \textit{solved in polynomial time for the class of interval graphs} (\textit{AT-free $\cap$ Chordal}). (Note that \SCP\ is NP-complete for both the classes, chordal graphs and AT-free graphs~\cite{DBLP:conf/caldam/ShaluVS17}.)
