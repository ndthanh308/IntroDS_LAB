\section{Introduction}
    Multiagent Systems are employed in a wide range of settings, where autonomous agents are expected to face dynamic situations and to be able to adapt in order to reach a given goal. In these contexts, it is crucial for agents to be able to reason on their physical environment as well as on the \emph{knowledge} that they have about other agents and the knowledge those possess. 
    
    Bolander and Andersen \cite{journals/jancl/Bolander2011} introduced \emph{epistemic planning} as a planning framework based on Dynamic Epistemic Logic (DEL), where \emph{epistemic states} are represented as Kripke models, \emph{event models} are used for representing epistemic actions, and \emph{product updates} define the application of said actions on states. On the one hand, the resulting framework is very expressive, and it allows one to naturally represent non-deterministic actions, partial observability of agents, higher-order knowledge and both factual and epistemic changes. 
    %
    On the other hand, decidability of epistemic planning is not guaranteed in general~\cite{journals/jancl/Bolander2011}. This has led to a considerable body of research adopting the DEL framework to obtain (un)decidability results for fragments of the epistemic planning problem (see \cite{journals/ai/Bolander2020} for a detailed exposition), typically by constraining the event models of actions. Nonetheless, even when such restriction are in place, epistemic planners directly employing the Kripke-based semantics of possible worlds face high complexities, hence considerable efforts have been put in studying action languages that are more amenable computationally \cite{journals/corr/Baral2015,conf/aips/Kominis2015,conf/icaps/Fabiano2020}.
    
    In contrast with the traditional approach in the literature, in this paper we depart from the Kripke-based semantics for DEL and adopt an alternative representation called \emph{possibilities}, first introduced by Gerbrandy and Groeneveld \cite{journals/jolli/Gerbrandy1997}. As we are going to show experimentally, this choice is motivated primarily by practical considerations. In fact, as we expand in Section \ref{sec:delphic}, possibilities support a concise representation of factual and epistemic information and yield a light update operator that promises to achieve better performances compared to the traditional Kripke-based semantics. This is due to the fact that possibilities are \emph{non-well-founded objects}, namely objects that have a \emph{circular} representation (see Aczel \cite{books/csli/Aczel1988} for an exhaustive introduction on non-well-founded set theory). In fact, due to their non-well-founded nature, possibilities naturally reuse previously calculated information, thus drastically reducing the computational overhead deriving from redundant information. Conceptually, whenever an agent does not update his knowledge upon an action, then the possibilities representing its knowledge are directly reused (see Examples \ref{ex:product-update} and \ref{ex:union-update}).
    
    This paper presents a novel formalization of epistemic planning based on possibilities. Although these objects have been previously used in place of Kripke models to represent epistemic states \cite{conf/icaps/Fabiano2020}, previous semantics lacked a general characterization of actions. In this paper, we complement the possibility-based representation of states by formalizing two novel concepts: \emph{eventualities}, representing epistemic actions, and \emph{union update}, providing an update operator based on possibilities and eventualities.
    The resulting planning framework, called \textsc{delphic} (\emph{DEL-planning with a Possibility-based Homogeneous Information Characterisation}), benefits from the compactness of possibilities and promises to positively impact the performance of planning.
    This suggests that \textsc{delphic} is a viable but also convenient alternative to Kripke-based representations. 
    We support this claim by implementing both frameworks in ASP and by setting up an experimental evaluation of the two implementations aimed at comparing the traditional Kripke semantics for DEL and \textsc{delphic}. The comparison confirms that \textsc{delphic} outperforms the traditional approach in terms of both space and time. We point out that time and space gains are obtained in the \emph{average case}, as there exist extreme (\emph{worst}) cases where the two semantics produce epistemic states with the same structure. This follows by the fact that the \textsc{delphic} framework is semantically equivalent to the Kripke-based one (Theorem \ref{th:delphic_eq}). As a result, the plan existence problems of both frameworks have the same complexity.
    
    %
    Partial evidences of the advantages of adopting possibilities were already experimentally witnessed in \cite{conf/icaps/Fabiano2020}. However, the planning framework therein corresponds only to a fragment of the DEL framework. 
    Indeed, as mentioned above, an actual possibility-based formalization of actions is there absent, in favour of a direct, ad-hoc encoding of the transition functions of three prototypical types of actions described in the action language $\mal$ \cite{journals/corr/Baral2015}, namely \emph{ontic}, \emph{sensing} and \emph{announcements} actions.
    %
    As already mentioned, we overcome this limitation by equipping \textsc{delphic} with eventualities, which we relate to DEL event models. 
    %
        
    In conclusion, we provide a threefold contribution:
    \begin{inparaenum}[\itshape (i)]
    \item we introduce \textsc{delphic} as a general DEL planning framework based on possibilities; 
    \item we formally show that \textsc{delphic} constitutes an alternative but semantically equivalent framework for epistemic planning, compared to the Kripke-based framework;
    \item we experimentally show that the underlying model employed by \textsc{delphic} indeed offers promising advantages in performance, in terms of both time and space.
    \end{inparaenum}
    
    The paper is organised as follows. In Section~\ref{sec:del}, we recall the necessary preliminaries on DEL; in Section~\ref{sec:delphic}, we formally define \textsc{delphic} and we show its equivalence with the Kripke-based framework %; in Section~\ref{sec:asp} we illustrate the two ASP encodings 
    and in Section~\ref{sec:eval} we show our experimental evaluation.

    
    %Our main contribution is the identification of a decidable fragment of the epistemic planning problem that is characterised by imposing a structural restriction on states and actions, hence on possibilities and eventualities, that prevents the planning states to grow unboundedly. 
    %
    %Nonetheless, the resulting epistemic planning framework maintains its crucial features, as we are able to capture factual change, epistemic pre- and postconditions, non-deterministic actions and partial observability. We also show that the planning frameworks of the action languages $ \mal $~\cite{journals/corr/Baral2015} and $ \mar $~\cite{conf/cilc/Fabiano2019} fall within such fragment. This result is novel in its own, since the decidability of such frameworks has not been addressed yet. As second contribution, we complement previous possibility-based formalisms for epistemic planning (specifically, $ \mar $ \cite{conf/cilc/Fabiano2019}) by providing a formalisation of the action language in terms of possibility-based event models and product updates. %, which are analogous to the well-understood notions of updates in DEL. 
    %
    %In other words, a possibility encodes both a possible \emph{world} and the information about which \emph{worlds} are considered possible by each agent. As a consequence, 
    %since in a particular world $ w $ an agent might consider $ w $ itself to be possible, representing worlds as possibilities leads to \emph{circular} objects. In fact, possibilities are grounded in \emph{non-well-founded set theory} and we refer to~\citeauthor{books/csli/Aczel1988}~\shortcite{books/csli/Aczel1988} for an exhaustive introduction to this field. 
    
    %The paper is organised as follows. In Section~\ref{sec:del} we recall some preliminary notions of DEL, we expand on the notion of possibilities, and we show their relation to Kripke models. In Section~\ref{sec:delphic}, a novel characterisation of \emph{update} is introduced, based on possibilities. In Section~\ref{sec:mep}, we lay the background on epistemic planning and epistemic action languages. In Section~\ref{sec:decidability}, we provide the main contribution, namely we show a decidable fragment of epistemic planning. We also show that the action languages discussed in Section~\ref{sec:mep} are within this fragment. Finally, in Section~\ref{sec:rel_works}, we discuss related work.
