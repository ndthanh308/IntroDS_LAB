\section{Conclusions}
    We have introduced a novel epistemic planning framework, called \textsc{delphic}, based on the formal notion of possibility, in place of the more traditional Kripke-based DEL representation. We have formally shown that these two frameworks are semantically equivalent. Possibilities provide a more compact representation of epistemic states, in particular by reusing common information across states. To show the benefits of possibilities, we have implemented \textsc{delphic} and the Kripke-based approach in ASP, performing a comparative experimental evaluation with known benchmark domains. The results show that \textsc{delphic} indeed outperforms the Kripke-based approach both in terms of space and time performances, and is thus a good candidate for practical DEL planning.

    In the future, we plan to exploit the performance gains provided by the \textsc{delphic} semantics in more competitive implementations based on C++. An interesting avenue of work is to deepen our analysis of possibility-based succinctness on fragments of DEL, where only a set of specific types of actions are allowed (\eg the language $m\mathcal{A}^*$ \cite{journals/corr/Baral2015} and the framework by Kominis and Geffner \cite{conf/aips/Kominis2015}).
    
    %of representation in the case where epistemic actions are specified relying on observability groups. In this case, it is in principle possible to more precisely measure the succinctness gain that possibilities provide w.r.t. the Kripke representation.
