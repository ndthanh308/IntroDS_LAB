\section{\textsc{delphic}}\label{sec:delphic}
    We introduce  the \textsc{delphic} framework for epistemic planning. \textsc{delphic} is built around the concept of \emph{possibility} (Definition \ref{def:possibility}), first introduced by Gerbrandy and Groeneveld to represent epistemic states. We develop a novel representation for epistemic actions inspired by possibilities, which we term \emph{eventualities} (Definition \ref{def:eventuality}). Then, we present a novel characterisation of update, called \emph{union update} (Definition \ref{def:update_pem}), based on possibilities and eventualities.

    \subsection{Possibilities}\label{sec:poss}
        %We now describe an alternative representation of epistemic states based on the notion of \emph{possibility}~\cite{journals/jolli/Gerbrandy1997}. 
        Possibilities are tightly related to \emph{non-well-founded sets}, \ie sets that may give rise to infinite \emph{descents} $X_1 \in X_2 \in \dots$ (\eg $\Omega = \{\Omega\}$ is a n.w.f. set). We refer the reader to Aczel \cite{books/csli/Aczel1988} for a detailed account on non-well-founded set theory.

        \begin{definition}[Possibility]\label{def:possibility}
            A \emph{possibility} $ \poss{u} $ for $ \Lang{} $ is a function that assigns to each atom $ \atom{p} \in \atomSet $ a truth value $ \poss{u}(\atom{p}) \in \{0, 1\} $ and to each agent $ i \in \agentSet $ a \emph{set of possibilities} $ \poss{u}(i) $, called \emph{information state}.
        \end{definition}

        \begin{definition}[Possibility Spectrum]\label{def:p-spectrum}
            A \emph{possibility spectrum} is a finite set of possibilities $ \poss{U} = \{\poss{u}_1, \dots \poss{u}_k\} $ that we call \emph{designated possibilities}.
        \end{definition}

        \noindent Possibility spectrums represent epistemic states in \textsc{delphic} and are able to represent the same information as MPKMs. Intuitively, each possibility $\poss{u}$ represent a possible world and the components $ \poss{u}(\atom{p})$ and $ \poss{u}(i)$ correspond to the valuation function and the accessibility relations of the world, respectively. Finally, the possibilities in a possibility spectrum represent the designated worlds. We formalize this intuition in Proposition \ref{prop:p-comparison}.
        
        % Notice that, on the one hand, worlds, accessibility relations and the valuation function of Kripke models are given as separated components. On the other hand, each possibility contains both a valuation $ \poss{u}(\atom{p}) $ for each atom in $ \atomSet $, and a set of worlds $ \poss{u}(i) $ that agent $ i $ considers possible, for each agent in $ \agentSet $.

        \begin{definition}[Truth in Possibilities]\label{def:p-truth}
            Let $ \poss{u} $ be a possibility, $ i \in \agentSet $, $ \atom{p} \in \atomSet $ and $\varphi,\psi \in \Lang{C}$ be two formulae. Then,
            
            {\centering
            $
                \begin{array}{@{}lll}
                    \poss{u} \models \atom{p}            & \text{ iff } & \poss{u}(\atom{p}) = 1 \\
                    \poss{u} \models \neg \varphi        & \text{ iff } & \poss{u} \not\models \varphi \\
                    \poss{u} \models \varphi \wedge \psi & \text{ iff } & \poss{u} \models \varphi \text{ and } \poss{u} \models \psi \\
                    \poss{u} \models \Box_i \varphi      & \text{ iff } & \forall \poss{v} \text{ if } \poss{v} \in \poss{u}(i) \text{ then } \poss{v} \models \varphi
                \end{array}
            $\par}
            \noindent Moreover, $ \poss{U} \models \varphi $ iff $ \poss{v} \models \varphi $, for all $ \poss{v} \in \poss{U} $.
        \end{definition}

        \paragraph{Comparing Possibilities and Kripke Models.}
        Gerbrandy and Groeneveld \cite{journals/jolli/Gerbrandy1997} show how possibilities and Kripke models correspond to each other. In what follows, we extend this result by analyzing the relation between possibility spectrums and MPKMs. First, following \cite{journals/jolli/Gerbrandy1997}, we give some definitions.

        \begin{definition}[Decoration of Kripke Model]\label{def:dec_km}
            The \emph{decoration} of a Kripke model $ M = (W, R, V) $ is a function $ \delta $ that assigns to each world $w \in W$ a possibility $ \poss{w} = \delta(w) $, such that:
            \begin{compactitem}
                \item $ \poss{w}(\atom{p}) = 1 \text{ iff } w \in V(\atom{p}) $, for each $ \atom{p} \in \atomSet $;
                \item $ \poss{w}(i)  = \{\delta(w') \mid w R_i w'\} $, for each $i \in \agentSet $.
            \end{compactitem}
        \end{definition}

        \noindent Intuitively, decorations provide a link between Kripke-based representations and their equivalent possibility-based ones: given $w$ in $M$, the decoration of $M$ returns the possibility that encodes $w$ (its valuation and accessibility relation). 

        \begin{definition}[Picture and Solution]\label{def:pic_km}
            If $ \delta $ is the decoration of a Kripke model $ M = (W, R, V) $ and $ W_d \subseteq W $, then $ (M, W_d) $ is the \emph{picture} of the possibility spectrum $ \poss{W} = \{\delta(w) \mid w \in W_d\} $. $ \poss{W} $ is called \emph{solution} of $ (M, W_d) $.
        \end{definition}

        \noindent Namely, the solution of a MPKM $ (M, W_d) $ is the possibility spectrum $\poss{W}$ that contains the possibilities calculated by the decoration function, one for each designated world in $W_d$. Finally, $ (M, W_d) $ is the picture of $\poss{W}$. Notice that, in general, \emph{different} MPKMs may share the \emph{same} solution. This observation will be formally stated in Proposition \ref{prop:p-comparison}. We now give an example (see also Figure~\ref{fig:ex4}).
        
        \begin{example}\label{ex:dec_km}
            The decoration $ \delta $ of the MPKM of Example~\ref{ex:k_model} assigns the possibilities $ \poss{w}_1 = \delta(w_1) $, $ \poss{w}_2 = \delta(w_2) $. Since $W_d = \{w_1\}$, we have that $ \poss{W} = \{\poss{w}_1\} $ is the solution of $(M,W_d)$, where:
            \begin{compactitem}
                \item $ \poss{w}_1(\atom{h}) = 1 $ and $\poss{w}_1(a) = \poss{w}_1(b) = \{\poss{w}_1, \poss{w}_2\} $;
                \item $ \poss{w}_2(\atom{h}) = 0 $ and $\poss{w}_2(a) = \poss{w}_2(b) = \{\poss{w}_1, \poss{w}_2\} $.
            \end{compactitem}
            %\noindent A graphical representation is given in Figure~\ref{fig:ex4}.
        \end{example}

        % Figure environment removed

        Notice that, in Example~\ref{ex:dec_km}, although the possibility $\poss{w_2}$ is not explicitly part of $\poss{W}$, it is ``stored'' \emph{within} $\poss{w_1}$. That is, we do not lose the information about $\poss{w_2}$.

        Given the above definitions, we are now ready to formally compare possibility spectrums with MPKMs. The following result generalize the one by Gerbrandy and Groeneveld \cite[Proposition 3.4]{journals/jolli/Gerbrandy1997}:

        \begin{proposition}\label{prop:p-comparison}\hfill
            \begin{compactenum}
                \item Each MPKM has a unique decoration;
                \item Each possibility spectrum has a MPKM as its picture;
                \item\label{compact-p} Two MPKMs have the same solution iff they are bisimilar.
                % \item If $ (M, w) $ is a picture of $ \poss{w} $, then $ (M, w) \models \varphi \text{ iff } \poss{w} \models \varphi $.
            \end{compactenum}
        \end{proposition}

        From item \ref{compact-p} of the above Proposition, we obtain the following remark:

        \begin{remark}\label{rem:compactness}
            Let $s = (M, W_d)$ be a MPKM and let $s'$ be its bisimulation contraction (\ie the smallest MPKM that is bisimilar to $s$). Since $s$ and $s'$ share the same solution $\poss{W}$, it follows that possibility spectrums naturally provide a more compact representation w.r.t. MPKMs.
        \end{remark}
        
        %Therefore: %From this consideration, we obtain the following remark:
		%
        %\begin{remark}\label{rem:p-compact}
        %    Since the class $\mathcal{S}_\poss{W}$ always contains a model $s$ that is minimal w.r.t. bisimulation contraction, it follows that possibility spectrums provide us with a inherently \emph{compact} way of representing epistemic states.
        %\end{remark}\todo{fix: possibilities are equivalent to bisimulation contractions}

        Finally, we show that the solution of a MPKM preserves the truth of formulae.

        \begin{proposition}\label{prop:truth}
            Let $(M, W_d)$ be a MPKM and let $\poss{W}$ be its solution. Then, for every $\varphi \in \Lang{}$, $(M,W_d) \models \varphi$ iff $\poss{W} \models \varphi$.

            \begin{proof}
                Let $\delta$ be the decoration of $(M, W_d)$. We denote with $eq(\psi)$ the fact that $(M,w) \models \psi$ iff $\delta(w) \models \psi$, for all $w {\in} W$.
                
                Consider now $w \in W$ and let $\poss{w} = \delta(w)$. We only need to show that $eq(\varphi)$ holds for any $\varphi \in \Lang{}$. The proof is by induction of the structure of $\varphi$. For the base case, let $\varphi=\atom{p}$. By Definition \ref{def:dec_km}, we immediately have that, for any $\atom{p} \in \atomSet$ and $w \in W$, $(M, w) \models \atom{p}$ iff $\poss{w} \models \atom{p}$ (\ie $eq(\atom{p})$).
                %
                For the inductive step, we have:
                \begin{compactitem}
                    \item Let $\varphi {=} \neg \psi$. From $eq(\psi)$ we get $eq(\neg\psi)$;% and assume $eq(\psi)$. Then, we have $eq(\neg\psi)$.
                    \item Let $\varphi {=} \psi_1 {\wedge} \psi_2$. From $eq(\psi_1)$, $eq(\psi_2)$ we get $eq(\psi_1 {\wedge} \psi_2)$;%and assume $eq(\psi_1)$ and $eq(\psi_2)$. Then, we have $eq(\psi_1 {\wedge} \psi_2)$.
                    %\item Let $\varphi {=} \Box_i\psi$ and assume $eq(\psi)$. Then, we have:
                    %{\centering
                    %    \begin{tabular}{lll}
                    %        $(M, w) {\models} \Box_i\psi$ & $\textover{\Leftrightarrow}{Def. \ref{def:k-truth}}$           & $\forall v \text{ if } w R_i v, \text{ then } (M, v) {\models} \psi$ \\
                    %                                      & $\textover{\Leftrightarrow}{Def. \ref{def:dec_km}, $eq(\psi)$}~$ & $\forall \poss{v} \text{ if } \poss{v} \in \poss{u}_i, \text{ then } \poss{v} {\models} \psi$ \\
                    %                                      & $\textover{\Leftrightarrow}{Def. \ref{def:p-truth}}$             & $\poss{w} \models \Box_i\psi$
                    %    \end{tabular}
                    %\par}
                    \item Let $\varphi {=} \Box_i\psi$ and assume $eq(\psi)$. Then we have: 
                    
                     %
                    {\centering
                       %	
                                 $(M, w) {\models} \Box_i\psi$
                                 %
                                 $\;\;\textover{\Leftrightarrow}{Def. \ref{def:k-truth}}\;\;$
                                 %
                                 $\forall v \text{ if } w R_i v, \text{ then } (M, v) {\models} \psi$ \\
                                 %
                                 $\textover{\Leftrightarrow}{Def. \ref{def:dec_km}, $eq(\psi)$}\quad\;\; $
                                 %
                                 $\forall \poss{v} \text{ if } \poss{v} \in \poss{u}_i, \text{ then } \poss{v} {\models} \psi$ 
                                 %
                                 $\;\;\textover{\Leftrightarrow}{Def. \ref{def:p-truth}}\;\;$
                                 %
                                 $\poss{w} \models \Box_i\psi$
          
                    \par}
                \end{compactitem}
            \end{proof}
        \end{proposition}

    \subsection{Eventualities}
        In \textsc{delphic}, we introduce the novel concept of \emph{eventuality} to model epistemic actions that is compatible with possibilities. In the remainder of the paper, we fix a fresh propositional atom $ \atom{pre} \notin \atomSet $ and let $ \atomSet' = \atomSet \cup \{\atom{pre}\} $. In the following definition, $ \atom{pre} $ encodes the precondition of an event, while the remaining atoms in $ \atomSet $ encode postconditions.

        \begin{definition}[Eventuality]\label{def:eventuality}
            An \emph{eventuality} $ \pem{e} $ for $ \Lang{} $ is a function that assigns to each atom $ \atom{p'} \in \atomSet' $ a formula $ \pem{e}(\atom{p'}) \in \Lang{} $ and to each agent $ i \in \agentSet $ a \emph{set of eventualities} $ \pem{e}(i) $, called \emph{information state}.
        \end{definition}

        \noindent Note that an eventuality is essentially a possibility that associates to each atom a formula (instead of a truth value).

        \begin{definition}[Eventuality Spectrum]\label{def:e-spectrum}
            An \emph{eventuality spectrum} is a finite set of eventualities $ \poss{E} {=} \{\poss{e}_1, \dots \poss{e}_k\} $ that we call \emph{designated eventualities}.
        \end{definition}

        \noindent Eventuality spectrums represent epistemic actions in \textsc{delphic}. 
        Moreover, we can easily show that they are able to represent the same information as MPEMs. 
        Intuitively, each eventuality $\pem{e}$ represents an event and the components $\pem{e}(\atom{pre})$ and $\pem{e}(\atom{p})$ represent the precondition and the postconditions of the event, respectively. Finally, the eventualities in an eventuality spectrum represent the designated events. We formalize this intuition in Proposition \ref{prop:e-comparison}.
        

    \paragraph{Comparing Eventualities and Event Models.}
        We now analyze the relationship between eventuality spectrums and MPEMs. We introduce the notions of decoration, picture and solution for event models.

        \begin{definition}[Decoration of an Event Model]\label{def:dec_em}
            The \emph{decoration} of an event model $ \E = (E, Q, pre, post) $ is a function $ \delta $ that assigns to each $ e \in E $ an eventuality $ \pem{e} = \delta(e) $, where:
            \begin{compactitem}
                \item $ \pem{e}(\atom{pre}) = pre(e)$ and $ \pem{e}(\atom{p}) = post(e)(\atom{p}) $, for each $\atom{p}\in \atomSet$;
                \item $ \pem{e}(i)          = \{\delta(e') \mid e Q_i e'\} $, for each $i \in \agentSet $.
            \end{compactitem}
        \end{definition}

        \begin{definition}[Picture and Solution]\label{def:pic_em}
            If $ \delta $ is the decoration of an event model $ \E = (E, Q, pre, post) $ and $ E_d \subseteq E $, then $ (\E, E_d) $ is the \emph{picture} of the eventuality spectrum $ \pem{E} = \{\delta(e) \mid e \in E_d\} $ and $ \pem{E} $ is the \emph{solution} of $ (\E, E_d) $.
        \end{definition}

        The above definitions are the counterparts of the notions of decoration and picture given in Definitions~\ref{def:dec_km} and \ref{def:pic_km}. %We now give a small example.

        \begin{example}\label{ex:dec_em}
            The decoration $ \delta $ of the MPEM of Example~\ref{ex:e_model} assigns the eventualities $ \pem{e}_1 = \delta(e_1) $ and $ \pem{e}_2 = \delta(e_2) $. Since $E_d = \{e_1\}$, we have that $ \pem{E} = \{\pem{e}_1\} $ is the solution of $(\E,E_d)$, where:
            \begin{compactitem}
                \item $ \pem{e}_1(\atom{pre}) = \atom{h} $; $\pem{e}_1(\atom{h}) = \atom{h}$; $\pem{e}_1(a) {=} \{\pem{e}_1\}$ and $\pem{w}_1(b) {=} \{\pem{w}_2\} $;
                \item $ \pem{e}_2(\atom{pre}) = \top $;     $\pem{e}_2(\atom{h}) = \atom{h}$; $\pem{e}_2(a) = \pem{w}_2(b) = \{\pem{e}_2\} $.
            \end{compactitem}
        \end{example}

        The following results formally compare eventuality spectrums with MPEMs.

        \begin{proposition}\label{prop:e-comparison}\hfill
            \begin{compactitem}
                \item Each MPEM has a unique decoration;
                \item Each eventuality spectrum has a MPEM as its picture;
                \item Two MPEMs have the same solution iff they are bisimilar.
            \end{compactitem}
        \end{proposition}

        Thus, analogously to the case of possibility spectrums, we can see that eventuality spectrums provide us with a compact representation of epistemic actions.
    
    \subsection{Union Update}
        We are now ready to present the novel formulation of update of \textsc{delphic}.  
        We say that an eventuality $\pem{e}$ is \emph{applicable} in a possibility $\poss{u}$ iff $ \poss{u} \models \pem{e}(\atom{pre}) $. Then, an eventuality spectrum $ \pem{E} $ is \emph{applicable} in a possibility spectrums $ \poss{U} $ iff for each $ \poss{u} \in \poss{U} $ there exists an applicable eventuality $ \pem{e} \in \pem{E} $. 

        \begin{definition}[Union Update]\label{def:update_pem}
            The \emph{union update} of a possibility $ \poss{u} $ with an applicable eventuality $ \pem{e} $ is the possibility $ \poss{u'} = \poss{u} \utimes \pem{e} $, where:
            
            {\centering
                $
            % \begin{equation*}
                \begin{array}{ll}
                    \poss{u'}(\atom{p}) & = 1 \text{ iff } \poss{u} \models \pem{e}(\atom{p}) \\
                    \poss{u'}(i)        & = \{\poss{v} \utimes \pem{f} \mid \poss{v} \in \poss{u}(i), \pem{f} \in \pem{e}(i) \text{ and } \poss{v} \models \pem{f}(\atom{pre})\}
                \end{array}
            % \end{equation*}
            $
            \par}
            
            \noindent The \emph{union update} of a possibility spectrum $ \poss{U} $ with an applicable eventuality spectrum $ \pem{E} $ is the possibility spectrum
            % \begin{equation*}

            {\centering
                $
            % \begin{equation*}
                \poss{U} \utimes \pem{E} = \{\poss{u} \utimes \pem{e} \mid \poss{u} \in \poss{U}, \pem{e} \in \pem{E} \text{ and } \poss{u} \models \pem{e}(\atom{pre})\}.
            % \end{equation*}%
                 $
            \par}
        \end{definition}

        \begin{example}\label{ex:union-update}
            The union update of the possibility spectrum $ \poss{W} $ of Example~\ref{ex:dec_km} with the eventuality spectrum of Example~\ref{ex:dec_em} is $\poss{W}\utimes\pem{E}=\{\poss{w_1} \utimes \pem{e_1}\}=\{\poss{v_3}\}$, where $\poss{v_3}(\atom{h})=1$, $\poss{v_3}(a)=\{\poss{v_3}\}$ and $\poss{v_3}(b)=\{\poss{w_1} \utimes \pem{e_2},\poss{w_2} \utimes \pem{e_2}\}=\{\poss{w_1},\poss{w_2}\}$.
            
            Notice that, since $\poss{w_1} \utimes \pem{e_2} {=} \poss{w_1}$ and $\poss{w_2} \utimes \pem{e_2} {=} \poss{w_2}$ the union update allows to reuse previously calculated information.
        \end{example}

    \paragraph{Comparing Union Update and Product Update.}
        Intuitively, it is easy to see that the possibility spectrum of Example \ref{ex:union-update} represents the same information of the MPKM of Example \ref{ex:product-update}. We formalize this intuition with the following lemma, witnessing the equivalence between product and union updates (full proof in the arXiv Appendix).

        \begin{lemma}\label{lem:updates_eq}
            Let $ (\E, E_d) $ be a MPEM applicable in a MPKM $ (M, W_d) $, with solutions $ \pem{E} $ and $ \poss{W} $, respectively. Then the possibility spectrum $ \poss{W'} = \poss{W} \utimes \pem{E} $ is the solution of $ (M', W'_d) = (M, W_d) \otimes (\E, E_d) $.
            
            % \begin{proof}
            %     Let $ M = (W, R, V) $, $ \E = (E, Q, pre, post) $ and $ M' = (W', R', V') $. Let $ \delta_M $ and $ \delta_{\E} $ be the decorations for $ M $ and $ \E $, respectively. Let then $ (\hat{M}, \hat{W}_d) $ be the picture of $ \poss{W'} $ via the decoration $ \delta $, where $\hat{M} = (\hat{W}, \hat{R}, \hat{V})$. By Proposition \ref{prop:p-comparison}, to prove that $ \poss{W'} $ is the solution of $ (M', W'_d) $, we need to show that $ (M', W'_d) \bisim (\hat{M}, \hat{W}_d) $. Let $ B \subseteq W' \times \hat{W} $ be a relation such that:
                
            %     {\centering
            %         $(w', \hat{w}) \in B \Leftrightarrow w' = (w, e) \wedge \delta(\hat{w}) = \delta_M(w) \utimes \delta_{\E}(e)$.
            %     \par}

            %     \noindent We now show that $ B $ is a bisimulation between $ M' $ and $ \hat{M} $. Let $ (w', \hat{w}) \in B $, with $ w' = (w, e) $ and let $ v' = (v, f) \in W' $. Let $ \poss{w} = \delta_M(w) $, $ \pem{e} = \delta_{\E}(e) $, $ \poss{v} = \delta_M(v) $ and $ \pem{f} = \delta_{\E}(f) $. Finally, let $ \poss{w'} = \poss{w} \utimes \pem{e} = \delta(\hat{w}) $ and $ \poss{v'} = \poss{v} \utimes \pem{f} $.
                % \begin{compactitem}
                %     \item (Atom) Let $ \atom{p} \in \atomSet $ be a propositional atom. Then:
                    
                %     {\centering
                %         \begin{tabular}{lll}
                %             $w' \in V'(\atom{p})$ & $\textover{\Leftrightarrow}{Def. \ref{def:update_em}}$                     & $(M, w) \models post(e)(\atom{p})$ \\
                %                                   & $\textover{\Leftrightarrow}{Pr.  \ref{prop:truth}, Def. \ref{def:dec_em}}$ & $ \poss{w} \models \pem{e}(\atom{p})$
                %                                     $\textover{\Leftrightarrow}{Def. \ref{def:update_pem}}$                      $\poss{w'}(\atom{p}) = 1 $ \\
                %                                   & $\textover{\Leftrightarrow}{Def. \ref{def:pic_km}}$                        & $\hat{w} \in \hat{V}(\atom{p})$
                %         \end{tabular}
                %     \par}

                %     \item (Forth/Back) Let $ i \in \agentSet $ be an agent. Then:
                        
                %     {\centering
                %         \begin{tabular}{lll}
                %             $w' R'_i v'$ & $\textover{\Leftrightarrow}{Def. \ref{def:update_em}}$                     & $w R_i v, e Q_i f, (M, w) \models pre(e) \text{ and }$ \\
                %                          &                                                                            & $(M, v) \models pre(f)$ \\
                %                          & $\textover{\Leftrightarrow}{Pr.  \ref{prop:truth}, Def. \ref{def:dec_em}}$ & $\poss{v} \in \poss{w}(i), \pem{f} \in \pem{e}(i), \poss{w} \models \pem{e}(\atom{pre}) \text{ and }$ \\
                %                          &                                                                            & $\poss{v} \models \pem{f}(\atom{pre})$ \\
                %                          & $\textover{\Leftrightarrow}{Def. \ref{def:update_pem}}$                    & $\poss{v'} \in \poss{w}'(i)$
                %                            $\textover{\Leftrightarrow}{Def. \ref{def:pic_km}}$                          $\hat{w} \hat{R}_i \hat{v}$
                %         \end{tabular}
                %     \par}

                %     \item (Designated) Let $ (w'_d, \hat{w}_d) \in B $, with $ w'_d = (w, e) $. Then:
                    
                %     {\centering
                %         \begin{tabular}{lll}
                %             $w'_d {\in} W'_d$ & $\textover{\Leftrightarrow}{Def. \ref{def:update_em}}$                     & $w {\in} W_d, e {\in} E_d \text{ and } (M, w) {\models} pre(e)$ \\
                %                               & $\textover{\Leftrightarrow}{Pr.  \ref{prop:truth}, Def. \ref{def:dec_em}}$ & $\poss{w} \in \poss{W}, \pem{e} \in \pem{E} \text{ and } \poss{w} \models \pem{e}(\atom{pre})$ \\
                %                               & $\textover{\Leftrightarrow}{Def. \ref{def:update_pem}}$                    & $\poss{w'} \in \poss{W'}$
                %                                 $\textover{\Leftrightarrow}{Def. \ref{def:pic_km}}$                          $\hat{w}_d \in \hat{W}_d$
                %         \end{tabular}
                %     \par}
                % \end{compactitem}
            % \end{proof}
        \end{lemma}

    \subsection{Plan Existence Problem in \textsc{delphic}}\label{sec:plan_ex_delphic}
        We conclude this section by giving the definitions of planning task and plan existence problem in \textsc{delphic}.
    
        \begin{definition}[\textsc{delphic}-Planning Task]
        \label{def:planning_task_delphic}
            A \emph{\textsc{delphic}-planning task} is a triple $ T = (\poss{W_0}, \Sigma,$ $ \varphi_g) $, where:
            \begin{inparaenum}[\itshape (i)]
            \item $ \poss{W_0} $ is an initial possibility spectrum; 
            \item $ \Sigma $ is a finite set of eventuality spectrums; 
            \item $ \varphi_g \in \Lang{C} $ is a \emph{goal formula}.
            \end{inparaenum}
        \end{definition}

        \begin{definition}\label{def:solution_delphic}
            A \emph{solution} (or \emph{plan}) to a \textsc{delphic}-planning task $(\poss{W_0}, \Sigma,$ $ \varphi_g)$ is a finite sequence $ \pem{E_1}, \dots, \pem{E}_\ell $ of actions of $\Sigma$ such that:
            \begin{compactenum}
                \item $ \poss{W_0} \utimes \pem{E_1} \utimes \dots \utimes \pem{E}_\ell \models \varphi_g $, and
                \item For each $ 1 {\leq} k {\leq} \ell $, $ \pem{E_k} $ is applicable in $ \poss{W_0} \utimes \pem{E_1} \utimes \dots \utimes \pem{E_{k-1}} $.
            \end{compactenum}
        \end{definition}

        \begin{definition}[Plan Existence Problem]\label{def:plan_ex_problem_delphic}
            Let $n {\geq} 1$ and $\mathcal{T}$ be a class of \textsc{delphic}-planning tasks. \planex{$\mathcal{T}$}{$n$} is the following decision problem: ``Given a \textsc{delphic}-planning task $ \poss{T} {\in} \mathcal{T} $, where $|\agentSet|{=}n$, does $ \poss{T} $ have a solution?''
        \end{definition}

        From Lemma \ref{lem:updates_eq}, we immediately get the following result:

        \begin{theorem}\label{th:delphic_eq}
            Let $T = (s_0, \actionSet, \varphi_g)$ be a DEL-planning task and let $\poss{T} = (\poss{W_0}, \Sigma, \varphi_g)$ be a \textsc{delphic}-planning task such that $\poss{W_0}$ is the solution of $s_0$ and $\Sigma$ is the set of solutions of $\actionSet$. Then, $\alpha_1, \dots, \alpha_\ell$ is a plan for $\planex{T}{n}$ iff $\pem{E_1}, \dots, \pem{E}_\ell$ is a plan for $\planex{\poss{T}}{n}$, where $\pem{E_i}$ is the solution of $\alpha_i$, for each $1 \leq i \leq \ell$.
        \end{theorem}
