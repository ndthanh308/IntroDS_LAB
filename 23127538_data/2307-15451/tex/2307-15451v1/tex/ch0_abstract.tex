\begin{abstract}
    Dynamic Epistemic Logic (DEL) provides a framework for epistemic planning that is capable of representing non-deterministic actions, partial observability, higher-order knowledge and both factual and epistemic change. The high expressivity of DEL challenges existing epistemic planners, which typically can handle only restricted fragments of the whole framework. 
    The goal of this work is to push the envelop of practical DEL planning, ultimately aiming for epistemic planners to be able to deal with the full range of features offered by DEL. Towards this goal, we question the traditional semantics of DEL, defined in terms on Kripke models. In particular, we propose an equivalent semantics defined using, as main building block, so-called \emph{possibilities}: non well-founded objects representing both factual properties of the world, and what agents consider to be possible. We call the resulting framework \textsc{delphic}. We argue that \textsc{delphic} indeed provides a more compact representation of epistemic states. To substantiate this claim, we implement both approaches in ASP and we set up an experimental evaluation to compare \textsc{delphic} with the traditional, Kripke-based approach. The evaluation confirms that \textsc{delphic} outperforms the traditional approach in space and time.
\end{abstract}
