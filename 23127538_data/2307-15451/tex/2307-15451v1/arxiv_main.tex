% This is samplepaper.tex, a sample chapter demonstrating the
% LLNCS macro package for Springer Computer Science proceedings;
% Version 2.21 of 2022/01/12
%
\documentclass[runningheads]{llncs}
%
\usepackage[T1]{fontenc}
% T1 fonts will be used to generate the final print and online PDFs,
% so please use T1 fonts in your manuscript whenever possible.
% Other font encondings may result in incorrect characters.
%
\usepackage{graphicx}
% Used for displaying a sample figure. If possible, figure files should
% be included in EPS format.
%
% If you use the hyperref package, please uncomment the following two lines
% to display URLs in blue roman font according to Springer's eBook style:
%\usepackage{color}
%\renewcommand\UrlFont{\color{blue}\rmfamily}
%



%%% Added packages %%%
\usepackage{amssymb}
\usepackage[english]{babel}
\usepackage[all]{foreign}
\usepackage{multirow}
\usepackage{paralist}
\usepackage[inline]{enumitem} % must be after paralist
\usepackage{subfig}
\usepackage{tikz}


%% Tikz settings %%
\tikzset{every picture/.style={line width=0.75pt}}
\usetikzlibrary{arrows,positioning,calc}


\newcommand\calF{\mathcal{F}}
\newcommand\calG{\mathcal{G}}
\newcommand\calM{\mathcal{M}}
\newcommand\calV{\mathcal{V}}
\newcommand\calU{\mathcal{U}}
\newcommand\calW{\mathcal{W}}
\newcommand\calP{\mathcal{P}}
\newcommand\calD{\mathbb{D}}
%%%%%%%%%%%%%%%%%
%% macros introduced by Luke 
\newcommand\mydef[1]{{\bf\em #1}}
%%%%%%%%%%%%%%%%%

\newcommand{\numviparams}{{| \lambda |}}
\newcommand{\scoreaccvars}[1]{s_1^{#1}, \ldots, s_{\numviparams}^{#1}}
\newcommand{\scoreaccvar}[2]{s_{#1}^{#2}}
\newcommand{\isdeterm}[1]{\text{Deterministic}({#1})}


\newcommand{\expect}[1]{\mathbb{E}\left[{#1}\right]}
\newcommand{\var}[1]{\mathbb{V}\left[ {#1} \right]}
\newcommand{\expectdist}[2]{\mathbb{E}_{#1}\left[ {#2} \right]}
\newcommand{\vardist}[2]{\mathbb{V}_{#1}\left[ {#2} \right]}
\newcommand{\cov}[2]{\mathbb{C}\text{ov}[{#1}][{#2}]}
\newcommand{\covv}[1]{\mathbb{C}\text{ov}[{#1}]}
\newcommand{\corr}[1]{\mathbb{C}\text{orr}[{#1}]}

\newcommand{\fix}[1]{\mathit{fix}\left({#1}\right)}
\newcommand{\sbr}[1]{\left\llbracket {#1} \right\rrbracket}
\newcommand{\ctxtype}[3]{{#1} \cong_\text{ctx} {#2} : {#3}}
\newcommand{\bigstep}[3]{{#1} \Downarrow_{#2} {#3}}


% PCF types
\newcommand{\bool}{\mathit{bool}}
\newcommand{\nat}{\mathit{nat}}

\newcommand{\ctx}[1]{\mathcal{C}\left[ {#1}\right] }
\newcommand{\pcft}[1]{\text{PCF}_{#1}}

\newcommand{\nfl}{\mathbb{N}_\bot}
\newcommand{\bfl}{\mathbb{B}_\bot}

% PCF constructs
\newcommand{\succc}[1]{\mathbf{succ}({#1})}
\newcommand{\succcn}[2]{\mathbf{succ}^{#1}({#2})}
\newcommand{\zero}{\mathbf{0}}
\newcommand{\zerotest}[1]{\mathbf{zero}\left({#1}\right)}
\newcommand{\pred}[1]{\mathbf{pred}\left( {#1} \right)}
\newcommand{\predn}[2]{\mathbf{pred}^{#1}\left( {#2} \right)}
\def\solvable{\#}

\newcommand{\true}{\mathbf{true}}
\newcommand{\false}{\mathbf{false}}
\newcommand{\pcffix}[1]{\mathbf{fix}\left({#1}\right)}
\newcommand{\pcffn}[3]{\mathbf{fn}~{#1}:{#2}\mathpunct{.}{#3}}
\newcommand{\pairtype}[2]{{#1} * {#2}}
\newcommand{\pairexp}[2]{\mathbf{pair}({#1}, {#2})}
\newcommand{\leftexp}[1]{\mathbf{left}({#1})}
\newcommand{\rightexp}[1]{\mathbf{right}({#1})}

\newcommand{\RationalPos}{\mathbb{Q}^{+}}

\newcommand{\meas}[1]{\mathbb{M}\left( {#1} \right) }
\newcommand{\integ}[1]{\sbr{#1}_I}

\newcommand{\notbigstep}[2]{{#1}~\cancel{\Downarrow}_{#2}}
\newcommand{\subtrace}[3]{{#1}^{{#2} \ldots {#3}}}
\newcommand{\supp}[1]{\textsf{supp}\left({#1}\right)}
\newcommand{\dom}[1]{\textsf{Dom}\left({#1}\right)}
\newcommand{\suppk}[2]{\textsf{Supp}^{#1}\left({#2}\right)}
\newcommand{\tracespace}{\bigcup_{n \in \mathbb{N}}[0, 1]^n}
\newcommand{\generictracespace}{\mathbb{T}}
\newcommand{\nnreals}{\mathbb{R}_{\geq 0}}
\newcommand{\posreals}{\mathbb{R}_{> 0}}
\newcommand{\reals}{\mathbb{R}}

\newcommand{\unrollkM}[2]{\textsf{unroll}_{#1}\left({#2}\right)}
\newcommand{\nphmcint}[5]{\Psi_\textsf{NP}\left({#1}, {#2}, {#3}, {#4}, {#5}\right)}

%SPCF constructs
\newcommand{\spcfvalues}{\Lambda^0_v}

\newcommand{\prevalueM}[1]{\textsf{value}^{-1}_{#1}(\spcfvalues{})}
\newcommand{\num}[1]{\underline{#1}}

% \theoremstyle{definition}
% \newtheorem{thm}{Theorem}
% \newtheorem{lem}{Lemma}
% \newtheorem{defn}{Definition}
% \newtheorem{conj}{Conjecture}
% \newtheorem{prop}{Proposition}

%\theoremstyle{definition}
%\newtheorem{defn}{Definition}[section]
%\newtheorem{example}[defn]{Example}
%
%
%\theoremstyle{plain}
%\newtheorem{thm}{Theorem}[section]
%\newtheorem{lem}[thm]{Lemma}
%\newtheorem{cor}[thm]{Corollary}
%\newtheorem{conj}[thm]{Conjecture}
%\newtheorem{prop}[thm]{Proposition}
%\newtheorem{remark}[thm]{Remark}

%% Proofs
%\let\oldproof\proof
%\renewcommand{\proof}{\color{blue}\oldproof}


\definecolor{codegreen}{rgb}{0,0.6,0}
\definecolor{codegray}{rgb}{0.5,0.5,0.5}
\definecolor{codepurple}{rgb}{0.58,0,0.82}
\definecolor{backcolour}{rgb}{0.95,0.95,0.92}

\lstdefinestyle{myStyle}{
    belowcaptionskip=1\baselineskip,
    breaklines=true,
    frame=none,
    basicstyle=\footnotesize\ttfamily,
    keywordstyle=\bfseries\color{green!40!black},
    commentstyle=\itshape\color{purple!40!black},
    identifierstyle=\color{blue},
    backgroundcolor=\color{gray!10!white},
    %backgroundcolor=\color{backcolour}, 
    numberstyle=\tiny\color{codegray},
    stringstyle=\color{codepurple},
    breakatwhitespace=false,                          
    keepspaces=true,                 
    numbers=left,       
    numbersep=5pt,                  
    showspaces=false,                
    showstringspaces=false,
    showtabs=false,                  
    tabsize=2,
}

% argmin/argmax
\DeclareMathOperator*{\argmax}{arg\,max}
\DeclareMathOperator*{\argmin}{arg\,min}

% Concatenation of lists
\newcommand\doubleplus{+\kern-1.3ex+\kern0.8ex}

% Program configurations
\newcommand{\tuple}[1]{\ensuremath{\langle #1 \rangle}}
% Rule based definitions
\newcommand{\Rule}[4][]{\ensuremath{\inferrule*[lab={\hypertarget{#2}{(\TirName{#2})}},#1]{#3}{#4}}}

% Calligraphic symbols
\newcommand{\calI}{{\mathcal I}} 
\newcommand{\calT}{{\mathcal T}}

%  Macro for new Y operator.
\newcommand{\yBounded}[3]{\mu^{#1}_{#2}\rvert_{#3}}

%%%%%%%%%%%%%%%%%
 
%%%%%%%%%%%%%%%%%

\newcommand{\expv}{\mathbb{E}}

\newcommand{\combTr}[2]{\left[\begin{matrix}
		#1\\
		#2
	\end{matrix} \right]}

\newcommand{\exType}[2]{\left\{\begin{matrix}
		#1\\
		#2
	\end{matrix} \right\}}
\newcommand{\myint}[1]{ [#1]}
\newcommand{\Uniform}{\ensuremath{\mathrm{Uniform}}}
\newcommand{\Normal}{\ensuremath{\mathrm{normal}}}
\DeclareMathOperator{\abs}{abs}
\DeclareMathOperator{\pdf}{pdf}

\newcommand{\intConf}[1]{\lceil#1\rceil}
\newcommand{\tr}{\boldsymbol{t}}

\newcommand{\sample}{\tt{sample}}
%\newcommand{\fix}{\texttt{fix}}
%\newcommand{\num}[1]{\underline{#1}}
\newcommand{\myif}{\texttt{if}}
\newcommand{\mylet}{\texttt{let} \, }
\newcommand{\myin}{\, \texttt{in} \,}
\newcommand{\mythen}{\, \texttt{then} \,}
\newcommand{\myelse}{\, \texttt{else} \,}
\newcommand{\score}{\tt{score}}
\newcommand{\tick}{\tt{tick}}

\newcommand{\term}{\tt{term}}
\newcommand{\pv}{\mathbf{v}}
\newcommand{\rv}{\mathbf{r}}

\newcommand{\interval}{\mathfrak{I}}

\newcommand{\typeReal}{\textbf{\textsf{R}}}

\newcommand{\symbolInt}{\myint{\cdot}}

\newcommand{\LambdaInterval}{\Lambda_{\interval}}
\newcommand{\LambdaSymbolic}{\Lambda_{\text{sym}}}

\newcommand{\toIntervalTerm}[1]{#1^{2\interval}}

%Others
\newcommand{\Sset}{\mathbb{S}}
\newcommand{\Iset}{\mathbb{I}}
\newcommand{\Rset}{\mathbb{R}}
\newcommand{\Nset}{\mathbb{N}}
\newcommand{\Zset}{\mathbb{Z}}

\newcommand{\Term}{\mathbb{T}}
\newcommand{\prob}{\mathbb{P}}
\newcommand{\expt}{\mathbb{E}}


\newcommand{\Leb}{\tt{Leb}}
\newcommand{\Red}{\tt{Red}}
\newcommand{\cost}{\text{cost}}

%\newcommand{\intervalab}[2]{\underline{[#1,#2]}}
\newcommand{\intervalab}{\underline{[a,b]}}
\newcommand{\interI}{\mathcal{I}}
\newcommand{\trans}{\mathcal{T}}

\newcommand{\iv}{\mathbb{I}}

% Programming language constructs
\newcommand{\lit}[1]{\underline{#1}}
\newcommand{\letIn}[1]{\mathsf{let}\,{#1}\,\mathsf{in}\,}
\newcommand{\fixLam}[2]{\mu {#1} {#2}.}
\newcommand{\ifElse}[3]{\mathsf{if} (#1 \le \num{0}) \, {#2} \,\mathsf{else}\, {#3}}

%%Basic notions
\newcommand{\pspace}{(\Omega,\mathcal{F},\probm)}
\newcommand{\probm}{\mathbb{P}}
\newcommand{\condexpv}[2]{{\expt}{\left[{#1} \mid {#2}\right]}}

\newcommand{\stdConf}[1]{(#1)}
%\newcommand{\intConf}[1]{\lceil#1\rceil}
%\newcommand{\intConf}[1]{(#1)}
%\newcommand{\symConf}[1]{\langle\!\langle  #1 \rangle\!\rangle}
%\newcommand\symPath[1]{(#1)}
\newcommand{\symPath}[1]{\langle\!\langle  #1 \rangle\!\rangle}
\newcommand\symConf[1]{(#1)}

\newcommand{\ifSimple}[3]{\mathsf{if}(#1, #2, #3)}
%\newcommand{\ifElse}[3]{\mathsf{if} (#1 \le 0) \, \allowbreak {#2} \, \allowbreak \mathsf{else}\, {#3}}
%\newcommand{\ifElse}[3]{\ifSimple{#1}{#2}{#3}}

%\newcommand{\trace}{\mathsf{s}}
%
%\newcommand\defn[1]{{\bf \em #1}}
\newcommand{\traces}{\mathbb{T}}
%
%\newcommand{\stdConf}[1]{(#1)}
%%\newcommand{\intConf}[1]{\lceil#1\rceil}
%\newcommand{\intConf}[1]{(#1)}
%%\newcommand{\symConf}[1]{\langle\!\langle  #1 \rangle\!\rangle}
%%\newcommand\symPath[1]{(#1)}
%\newcommand{\symPath}[1]{\langle\!\langle  #1 \rangle\!\rangle}
%\newcommand\symConf[1]{(#1)}

\newcommand{\valueSem}[1]{\mathsf{val}_{#1}} % value (semantics)
\newcommand{\weightSem}[1]{\mathsf{wt}_{#1}} % weight (semantics)
\newcommand{\measureSem}[1]{\llbracket #1 \rrbracket}
\newcommand{\posterior}{\mathsf{posterior}}


%%%%%%%%%
% 
%%%%%%%%
\newcommand{\loc}{\ell}
\newcommand{\locs}{\mathit{L}}
\newcommand{\blocs}{\mathit{L}_{\mathrm{b}}}

\newcommand{\iflocs}{\mathit{L}_{\mathrm{if}}}
\newcommand{\looplocs}{\mathit{L}_{\mathrm{while}}}

\newcommand{\alocs}{\mathit{L}_{\mathrm{a}}}
\newcommand{\wlocs}{\mathit{L}_{\mathrm{w}}}
\newcommand{\rlocs}{\mathit{L}_{\mathrm{r}}}
\newcommand{\Alocs}[1]{\mathit{L}_{\mathrm{A}}^{\mathsf{#1}}}
\newcommand{\Dlocs}{\mathit{L}_{\mathrm{nd}}}
\newcommand{\transitions}{{\rightarrow}}

%%% 
\newcommand{\plocs}{\mathit{L}_{\mathrm{p}}}
\newcommand{\tlocs}{\mathit{L}_{\mathrm{t}}}

\newcommand{\lin}{\loc_\mathrm{init}}
\newcommand{\lout}{\loc_\mathrm{out}}
\newcommand{\val}[1]{\mbox{\sl Val}_{#1}}

\newcommand{\pvars}{V_\mathrm{p}}
\newcommand{\rvars}{V_{\mathrm{r}}}
\newcommand{\pre}{\mathrm{pre}}

\newcommand{\sle}{\sqsubseteq}
\newcommand{\sge}{\sqsupseteq}

\newcommand{\lfp}{\mathrm{lfp}}
\newcommand{\gfp}{\mathrm{gfp}}

\newcommand{\rdvarjdis}{\mathcal D}
\newcommand{\sampset}{\textit{supp}}

\newcommand{\upd}{\mbox{\sl upd}}
\newcommand{\wet}{\mbox{\sl wt}}
\newcommand{\transset}{\mathfrak T}
\newcommand{\valin}{\pv_{\mathrm{init}}}
\newcommand{\ret}{\mbox{\sl ret}}

\newcommand{\win}{w_{\mathrm{init}}}

\newcommand{\sampdpd}{\overline{\Upsilon}}

\newcommand{\outmap}{\text{O}}
\newcommand{\sat}[1]{\langle #1 \rangle}
\newcommand{\monoid}{\mbox{\sl Monoid}}
\newcommand{\handelmanformat}{(\dagger)}

\newcommand{\trunc}{\mathcal{B}}

\newcommand{\ewt}{\mbox{\sl ewt}}
\newcommand{\statemap}{\text{St}}

\newcommand{\valrd}{{\mathbf{r}}}
\newcommand{\frmloc}{\ell^{\mathrm{src}}}
\newcommand{\toloc}{\ell^{\mathrm{dst}}}

\newcommand{\monomials}{\mathbf{M}}

\begin{document}
    \title{\textsc{delphic}: Practical DEL Planning via Possibilities (Extended Version)}
    
    %\titlerunning{Abbreviated paper title}
    % If the paper title is too long for the running head, you can set
    % an abbreviated paper title here
    
    \author{
        Alessandro Burigana\inst{1}\orcidID{0000-0002-9977-6735} \and
        Paolo Felli\inst{2}\orcidID{0000-0001-9561-8775} \and
        Marco Montali\inst{1}\orcidID{0000-0002-8021-3430}
    }
    
    \authorrunning{A. Burigana et al.}
    % First names are abbreviated in the running head.
    % If there are more than two authors, 'et al.' is used.
    
    \institute{
        Free University of Bozen-Bolzano, Italy
        \email{\{burigana, montali\}@inf.unibz.it}
        \and
        University of Bologna, Italy
        \email{paolo.felli@unibo.it}
    }
    
    \maketitle              % typeset the header of the contribution
    
    \begin{abstract}
    Dynamic Epistemic Logic (DEL) provides a framework for epistemic planning that is capable of representing non-deterministic actions, partial observability, higher-order knowledge and both factual and epistemic change. The high expressivity of DEL challenges existing epistemic planners, which typically can handle only restricted fragments of the whole framework. 
    The goal of this work is to push the envelop of practical DEL planning, ultimately aiming for epistemic planners to be able to deal with the full range of features offered by DEL. Towards this goal, we question the traditional semantics of DEL, defined in terms on Kripke models. In particular, we propose an equivalent semantics defined using, as main building block, so-called \emph{possibilities}: non well-founded objects representing both factual properties of the world, and what agents consider to be possible. We call the resulting framework \textsc{delphic}. We argue that \textsc{delphic} indeed provides a more compact representation of epistemic states. To substantiate this claim, we implement both approaches in ASP and we set up an experimental evaluation to compare \textsc{delphic} with the traditional, Kripke-based approach. The evaluation confirms that \textsc{delphic} outperforms the traditional approach in space and time.
\end{abstract}

    \section{Introduction} 
    Multi-agent systems find applications in a wide range of settings where the agents need to be able to reason about both the physical world and the \emph{knowledge} that other agents possess---that is, their \emph{epistemic state}.  
    %
    \emph{Epistemic planning} \cite{journals/jancl/Bolander2011} employs the theoretical framework of Dynamic Epistemic Logic (DEL) \cite{book/springer/vanDitmarsch2007} in the context of automated planning. The resulting formalism is able to represent nondeterminism, partial observability and arbitrary knowledge nesting. That is, agents have the power to reason about higher-order knowledge of other agents with no limitations.
    
    Due to the high expressive power of the DEL framework, the \emph{plan existence problem} (see Definition \ref{def:plan_ex_problem}), that asks whether there exists a plan to achieve a goal of interest, is undecidable in general \cite{journals/jancl/Bolander2011}. 
    As a consequence, in the past decade, DEL has been widely studied to obtain (un)decidability and complexity results for fragments of the planning problem.
    A common approach (see~Section~\ref{sec:rel_works}) consists in syntactically restricting the action theory, for instance by limiting the modal depth of the preconditions and postconditions of actions to a certain bound $d$ \cite{conf/ijcai/Bolander2015,conf/ijcai/Charrier2016,journals/ai/Bolander2020}. Nonetheless, the problem remains undecidable even with $d{=}2$ when only \emph{purely epistemic actions} are allowed, and with $d{=}1$ when factual change is involved. 
    This suggests that such syntactic restrictions are too strong in many practical cases, where reasoning about the knowledge of others is required.
    
    For this reason, in this paper we pursue a different strategy that we call \emph{semantic approach}. Namely, rather than imposing syntactical constraints, the semantic approach focuses on the axioms of the logic for epistemic planning. Specifically, we consider the multi-agent logic for knowledge S5$_n$ (where $n$ denotes the number of agents) and we augment it with an interaction axiom, called the \emph{(knowledge) commutativity} axiom (where, as customary, $\B{i} \varphi$ indicates that agent $i$ \emph{knows} that $\varphi$ holds):

    \begin{equation*}
        \begin{array}{lll}
            \axiom{C} & \B{i} \B{j} \varphi \rightarrow \B{j} \B{i} \varphi & \textnormal{(Commutativity)}
        \end{array}
    \end{equation*}
    
    \noindent This axiom imposes a principle of commutativity in the higher-order knowledge across agents. In the resulting logic, which we call C-S5$_n$, while agents have their own distinct individual knowledge, higher-order levels of perspectives of agents \emph{commute}. This assumption is well suited in cooperative planning domains \cite{journals/csur/Torreno2017}, where it is required that agents act and communicate in an observable way, thus making knowledge of agents accessible to others.
    
    We provide a threefold contribution. First, we show that the epistemic plan existence problem in the resulting framework becomes decidable. We do so by proving that the commutativity axiom ensures that the states in the logic C-S5$_n$ are bounded in size, which entails that the search space of the plan existence problem is finite. In doing so, we show that the logic C-S5$_n$ admits a \emph{finitary non-fixpoint} characterization of common knowledge, which is often regarded as a possible solution to paradoxes involving common knowledge (see \cite{journals/synthese/Paternotte11} for an overview).
    
    Second, we investigate the plan existence problem with different generalized principles of commutativity.
    Indeed, although the commutativity axiom is better fitting for tight-knit groups of agents, it may be less suited for representing more loosely organized groups.
    We define suitable generalizations parametrized by fixed integer constants $b{>}1$ and $1 {<} \ell {\leq} n$. The resulting axioms are the following (where $\pi$ is a permutation of the sequence $\langle i_1, \dots i_\ell\rangle$ of agents, as explained in more detail in Section \ref{sec:general-comm}):
    
    \begin{equation*}
        \begin{array}{lll}
            \axiom{C$^b$} & (\B{i} \B{j})^b \varphi \rightarrow (\B{j} \B{i})^b \varphi & \textnormal{($b$-Comm.)} \\
            \axiom{wC$_\ell$} & \B{i_1} \dots \B{i_\ell}\varphi \rightarrow \B{\pi_{i_1}} \dots \B{\pi_{i_\ell}}\varphi & \textnormal{(Weak comm.)}
        \end{array}
    \end{equation*}

    \noindent Concerning axiom \axiom{wC$_\ell$}, we show that the plan existence problem remains decidable for any $1 {<} \ell {\leq} n$. Relating axiom \axiom{C$^b$}, we show that the plan existence problem remains decidable in the presence of two agents ($n{=}2$), for any $b{>}1$. We also show that for any $n{>}2$ and any $b{>}1$ the problem becomes undecidable.

    Finally, we show that the knowledge (\ie S5$_n$) fragment of the well known planning system $m\mathcal{A}^*$ \cite{journals/corr/Baral2015} and the system by Kominis and Geffner \cite{conf/aips/Kominis2015} are captured by our formalism. Thus, we prove the decidability of a fragment $m\mathcal{A}^*$, which was still an open problem, and of the action formalism in \cite{conf/aips/Kominis2015}, confirming their previous results.

    Since the axioms of the logic for epistemic planning lie at the core of the semantic approach, we consider such axioms to define \emph{meaningful states}. In other words, when a certain principle is introduced to be an \emph{axiom} of the logic, an epistemic state is considered to be \emph{meaningful} if and only if such principle is satisfied. Thus, when planning under a logic $L$, we consider a plan to be meaningful and, in turn, valid only if all the states that it visits satisfy the axioms of $L$.
    At the same time, the semantics of the product update of DEL (see Definition \ref{def:update_em}) does not guarantee that the application of an action of a generic logic $L$ to an epistemic state of the same logic necessarily results in an epistemic state that satisfies all axioms of $L$. A well-known example of this phenomenon is found in the widely studied doxastic logic KD45$_n$ \cite{books/mit/Fagin2004}, where the \emph{consistency axiom} \axiom{D} is not guaranteed to be preserved by the product update. Addressing the problem of preservation of axioms after action updates is not trivial. Indeed, in the literature, considerable effort has been spent in developing different techniques to handle the preservation of axiom \axiom{D} \cite{workshop/nrac/Herzig2005,conf/aaai/Son2015}. Analogously to the case of \axiom{D} in KD45$_n$, in our framework axioms \axiom{C}, \axiom{C$^b$} and \axiom{wC$_\ell$} are not guaranteed to be preserved by the product update. As a result, as explained above, we consider a plan to be valid if it only visits meaningful epistemic states, \ie those satisfying the axioms of the considered logic. Importantly, our decidability results continue to hold even when one adopts more sophisticated revision techniques that handle the preservation problem by accepting and suitably curating non-preserving states. The development of such non-trivial techniques for our logics is independent of the analysis of decidability of the plan existence problem under the same logics, and is left as an important, follow-up work.
    
    The paper is organised as follows.
    In Section~\ref{sec:del}, we recall some preliminaries and define epistemic planning tasks. In Section~\ref{sec:new_logic}, we discuss in more detail the semantic approach, we introduce our new logic for epistemic planning and we discuss the commutativity axiom. In Section~\ref{sec:decidability}, we analyze decidability of epistemic planning under commutativity and its generalizations. In Section~\ref{sec:systems}, we apply our decidability results to existing epistemic planning systems.
    Finally, in Section~\ref{sec:rel_works}, we discuss related work.

    \section{Dynamic Epistemic Logic}\label{sec:del}
    This section is organized as follows. The syntax and semantics of DEL~\cite{book/springer/vanDitmarsch2007} are introduced in Section~\ref{sec:epistemic_models}, event models and the product update in Section~\ref{sec:event_models}.
    In Section \ref{sec:s5}, we recall the axioms of the logic S5$_n$.
    In Section~\ref{sec:plan_ex} we define the plan existence problem.

\subsection{Epistemic Models}\label{sec:epistemic_models}
    Let $\atomSet$ be a finite set of atomic propositions and $\agentSet = \{1, \dots, n\} $ a finite set of agents. The language $\Lang{C}$ of \emph{multi-agent epistemic logic on $ \atomSet $ and $ \agentSet $ with common knowledge} is defined by the following BNF:
    \begin{equation*}
        \varphi ::= \atom{p} \mid \neg \varphi \mid \varphi \wedge \varphi \mid \B{i} \varphi \mid \CK{G} \varphi,
    \end{equation*}

    \noindent where $ \atom{p} \in \atomSet $, $ i \in \agentSet $, and $ \varnothing \neq G \subseteq \agentSet $. Formulae $ \B{i} \varphi $ and $ \CK{G} \varphi $ are respectively read as ``agent $ i $ knows that $ \varphi $'' and ``group $G$ has common knowledge that $ \varphi $''.
    We define $\top$, $\bot$, $\vee$, $\rightarrow$ and $\D{i}$ as usual.

    \begin{definition}[Epistemic Model and State]
        An \emph{epistemic model} of $ \Lang{C} $ is a triple $ M = (W, R, V) $ where:
        \begin{compactitem}
            \item $ W \neq \varnothing $ is a finite set of possible worlds; 
            \item $ R: \agentSet \rightarrow 2^{W \times W} $ assigns to each agent $ i $ an accessibility relation $R(i)$ (abbreviated as $R_i$);
            \item $ V: \atomSet \rightarrow 2^W $ assigns to each atom a set of worlds.
        \end{compactitem}
        \noindent An \emph{epistemic state} is a pair $ (M, W_d) $, where $ W_d \subseteq W $ is a non-empty set of designated worlds.
    \end{definition}

    \noindent Intuitively, a designated world in $W_d$ is considered the current ``real" world from the perspective of an external observer (the planner) rather than of agents in $\agentSet$. Thus, $|W_d|>1$ represents the uncertainty of the observer about the real world. 
    
    The pair $(W, R)$ is called the \emph{frame} of $M$. We use the infix notation $ w R_i v $ in place of $ (w, v) \in R_i $.
    We also define $ R_G \doteq \cup_{i \in G} R_i $, where $ G \subseteq \agentSet $. The reflexive and transitive closure of $ R $ is denoted by $ R^* $. 
    Relations $R_i$ capture what agents consider to be possible: $ w R_i v $ denotes the fact that, in $w$, agent $i$ considers $v$ to be possible.
    %Formulae in $\Lang{C}$ are evaluated in worlds of a model. 		
    Throughout the paper, to support our exposition, we consider the example of the \emph{coordinated attack problem} \cite{book/springer/Gray1978,journals/jacm/Halpern1990}. It is a well-known problem that is often analyzed in the distributed systems literature. In what follows, we appeal to the DEL representation of this problem provided in \cite{journals/ai/Bolander2020}.

    \begin{example}[The Coordinated Attack Problem]\label{ex:k_model}
        Two generals, $\mathbf{a}$ and $\mathbf{b}$, are camped with their armies on two hilltops overlooking a common valley, where the enemy is stationed. The only way for them to defeat the enemy is to attack simultaneously. They can only communicate by means of a messenger, who may be captured at any time when crossing the valley. Neither general will attack until he is sure that the other will attack as well.
        
        General $\mathbf{a}$ and the messenger are initially together, and general $\mathbf{a}$ decides to attack at dawn. We use the atomic propositions $d$ to denote that `\emph{general $\mathbf{a}$ will attack at dawn}' and $m_i$, for $i = \mathbf{a}, \mathbf{b}$, to denote that the messenger is currently at the camp of general $i$. In this way, $\neg m_a \land \neg m_b$ expresses the fact that the messenger has been captured.
        
        The initial situation can be described by the epistemic state $s_0$ shown below\footnote{In all figures, the reflexive, symmetric and transitive closures of the relations are left implicit.}. Each bullet represents a world and the designated world is denoted by a circled bullet. There are two possible worlds, denoting the possibility that general $\mathbf{a}$ will attack at dawn ($w_1$), or not ($w_2$). Both generals know that the messenger is camped with general $\mathbf{a}$. The fact that general $\mathbf{b}$ does not know whether general $\mathbf{a}$ has decided to attack is represented by the indistinguishability relations between worlds $w_1$ and $w_2$. In fact, initially, general $\mathbf{b}$ has not enough information to know whether his ally has decided to attack.

        {\centering
            \begin{tikzpicture}[-,>=stealth',shorten >=1pt,auto,semithick]
    \node (A0) []                 {$s_0$} ;
    \node (A1) [right=.1cm of A0] {$=$} ;
    \node (W1) [pointedworld, right=.3cm of A1, label=below:{$w_1 : d, m_a$}] {} ;
    \node (W2) [world,        right=2cm of W1,  label=below:{$w_2 : m_a$}] {} ;

    \path (W1)
        edge node [above] {$b$} (W2);
\end{tikzpicture}

        \par}
    \end{example}

    \begin{definition}[Truth in epistemic states]
        Let $ M = (W, R, V) $ be an epistemic model, $ w \in W $, $i \in \agentSet$, $\varnothing \neq G \subseteq \agentSet$, $p \in \atomSet$ and $\varphi,\psi \in \Lang{C}$ be two formulae. Then,
        
        {\centering
        $
            \begin{array}{@{}lll}                    
                (M, w) \models \atom{p}            & \text{ iff } & w \in V(\atom{p}) \\
                (M, w) \models \neg \varphi        & \text{ iff } & (M, w) \not\models \varphi \\
                (M, w) \models \varphi \wedge \psi & \text{ iff } & (M, w) \models \varphi \text{ and } (M, w) \models \psi \\
                (M, w) \models \B{i} \varphi       & \text{ iff } & \forall v \text{ if } w R_i v \text{ then } (M, v) \models \varphi \\
                (M, w) \models \CK{G} \varphi       & \text{ iff } & \forall v \text{ if } w R_G^* v \text{ then } (M, v) \models \varphi
            \end{array}
        $
        \par}

        \noindent Let $(M, W_d)$ be an epistemic state. Then,

        {\centering
        $
            \begin{array}{@{}lll}                    
                (M, W_d) \models \varphi           & \text{ iff } & (M, w) \models \varphi \textnormal{ for all } w \in W_d
            \end{array}
        $
        \par}
    \end{definition}
	
	\noindent For instance, $(M,w)\models p$ means that $p$ is true in $w$; $(M,w)\models \D{i} p$ means that in $w$ the agent $i$ admits the possibility of $\varphi$ being true, i.e., there exists a world $v$ that $i$ considers possible (i.e., with $w R_i v$) such that $(M,v)\models \varphi$; $(M,w)\models \B{i} \varphi$ means that in $w$ the agent $i$ knows $\varphi$, as $\varphi$ holds in all worlds that $i$ considers possible.
	
    We recall the notion of bisimulation for epistemic states \cite{conf/kr/Bolander2021}. 
                
    \begin{definition}[Bisimulation]
        Let $s=((W, R, V), W_d)$ and $s'=((W', R', V'), W'_d)$ be two epistemic states. We say that $s$ and $s'$ are \emph{bisimilar}, denoted by $s \bisim s'$, if there exists non-empty binary relation $ Z \subseteq W \times W' $ satisfying:

        \begin{compactitem}
            \item \emph{Atoms}: if $(w, w') \in Z$, then for all $ \atom{p} \in \atomSet $, $ w \in V(\atom{p}) $ iff $ w' \in V'(\atom{p}) $.
            \item \emph{Forth}: if $(w, w') \in Z$ and $ w R_i v $, then there exists $ v' \in W' $ such that $ w' R'_i v' $ and $ (v, v') \in Z $.
            \item \emph{Back}: if $(w, w') \in Z$ and $ w' R'_i v' $, then there exists $ v \in W $ such that $ w R_i v $ and $ (v, v') \in Z $.
            \item \emph{Designated}: if $ w \in W_d $, then there exists $w' \in W'_d$ such that $(w, w') \in Z$, and vice versa.
        \end{compactitem}

        \noindent We say that $Z$ is a \emph{bisimulation} between $s$ and $s'$.
    \end{definition}

    \noindent Throughout the rest of the paper, we assume that \emph{each epistemic state is minimal w.r.t. bisimulation}.
    We denote the fact that $(w, w') \in Z$ by $w \bisim w'$.
    Finally, we introduce a notion of $k$-bisimulation for epistemic states. The following definition follows the one in \cite{books/cup/Blackburn2001} and generalizes that of \cite{conf/ijcai/Yu2013} by considering epistemic states with (possibly) multiple designated worlds.

    \begin{definition}[$k$-bisimulation]\label{def:k-bisim}
        Let $k \geq 0$ and let $s=((W, R, V),$ $W_d)$ and $s'=((W', R', V'), W'_d)$ be two epistemic states. We say that $s$ and $s'$ are $k$-bisimilar, denoted by $s \bisim_k s'$, if there exists a sequence of non-empty binary relations $Z_k \subseteq \ldots \subseteq Z_0$ (with $Z_0 \subseteq W \times W'$) satisfying (for any $i < k$):
        \begin{compactitem}
            \item \emph{Atoms}: if $(v, v') \in Z_0$, then for all $ \atom{p} \in \atomSet $, $ v {\in} V(\atom{p}) $ iff $ v' {\in} V'(\atom{p}) $.
            \item \emph{Forth}: if $(v, v') \in Z_{i+1}$ and $ v R_i u $, then there exists $ u' \in W' $ such that $ v' R'_i u' $ and $ (u, u') \in Z_i $.
            \item \emph{Back}: if $(v, v') \in Z_{i+1}$ and $ v' R'_i u' $, then there exists $ u \in W $ such that $ v R_i u $ and $ (u, u') \in Z_i $.
            \item \emph{Designated}: if $ v \in W_d $, then there exists $v' \in W'_d$ such that $(v, v') \in Z_k$, and vice versa.
        \end{compactitem}

        \noindent We say that $Z_k$ is a \emph{$k$-bisimulation} between $s$ and $s'$.
    \end{definition}

    \subsection{Event Models and Product Update}\label{sec:event_models}
        Information change is captured by \emph{product updates} of the current epistemic state with the \emph{event model} of actions. 
        
        \begin{definition}[Event Model and Action]
            An \emph{event model} for $ \Lang{C} $ is a tuple $ \E = (E, Q, pre, post) $ where:
            \begin{compactitem}
                \item $ E \neq \varnothing $ is a finite set of events; %, called \emph{domain}; 
                \item $ Q: \agentSet \rightarrow 2^{E \times E} $ assigns to each agent $ i $ an accessibility relation $ Q(i) $ (abbreviated as $Q_i$); 
                \item $ pre: E \rightarrow \Lang{C} $ assigns to each event a \emph{precondition}; 
                \item $ post: E \rightarrow (\atomSet \rightarrow \Lang{C}) $ assigns to each event and atom a \emph{postcondition}.
            \end{compactitem}
            \noindent An \emph{action} is a pair $ (\E, E_d) $ where $ E_d \subseteq E $ is a non-empty set of designated events.
        \end{definition}

        \noindent Similarly to epistemic states, the designated events in $E_d$ represent the ``real'' events that are taking place from the perspective of an external observer.

        The pair $(E, Q)$ is called the \emph{frame} of $\E$.
 		We use the infix notation $ e Q_i f $ in place of $ (e, f) \in Q_i$. 
 		These relations are analogous to the accessibility relations of epistemic models: they are used to specify how the knowledge of each agent is affected by an action, depending on which events each agent considers possible. 
		Intuitively, the precondition of an event $e$ specify whether $e$ \emph{could happen} in a certain world $w$, whereas the postconditions of $e$ describe how such event might change the factual properties of $w$ (see Definition \ref{def:update_em}). Formally, we say that an event $e$ is \emph{applicable} in a world $w$ of $M$ if $(M,w)\models pre(e)$.
        
        \begin{example}\label{ex:e_model}
            Imagine that general $\mathbf{a}$ decides to send the messenger to general $\mathbf{b}$ (action send$_{ab}$). While doing  so, the general considers two possible outcomes:
            \begin{enumerate*}[label=\arabic*)]
                \item the messenger safely arrives to the other side of the valley, or
                \item the messenger is captured by the enemy. 
            \end{enumerate*}
            %
            In the figure below, these eventualities are represented by events $e^a_1$ and $e^a_2$, respectively. The precondition of $e^a_1$ is $pre(e^a_1) = d \land m_a$, namely the message can only arrive to general $\mathbf{b}$ if general $\mathbf{a}$ has indeed decided to attack at dawn and if the messenger is at $\mathbf{a}$'s camp. The precondition of $e^a_2$ is simply $pre(e^a_2) = \top$, since the messenger could always be captured. We represent the fact that the messenger travels from one hilltop to the other\footnote{For simplicity, we assume that the truth value of each atomic proposition remains unchanged unless explicitly specified.} by having $post(e^a_1)(m_a) = \bot$ and $post(e^a_1)(m_b) = \top$. Finally, we denote the fact that the messenger is captured by having $post(e^a_2)(m_a) = \bot$ and $post(e^a_2)(m_b) = \bot$.

            {\centering
                \begin{tikzpicture}[-,>=stealth',shorten >=1pt,auto,semithick]
    \node (A0) []                 {send$_{ab}$} ;
    \node (A1) [right=.1cm of A0] {$=$} ;
    \node (E1) [pointedevent, right=.3cm of A1, label=below:{$e^a_1$}] {} ;
    \node (E2) [event,        right=2cm of E1, label=below:{$e^a_2$}] {} ;

    \path (E1)
        edge node [above] {$a$} (E2);
\end{tikzpicture}

            \par}

            Action send$_{ba}$ is defined similarly, by having $pre(e^b_1) = d \land m_b$, $pre(e^b_2) = \top$, $post(e^b_1)(m_a) = \top$, $post(e^b_1)(m_b) = \bot$, $post(e^b_2)(m_a) = \bot$ and $post(e^b_2)(m_b) = \bot$.

            {\centering
                \begin{tikzpicture}[-,>=stealth',shorten >=1pt,auto,semithick]
    \node (A0) []                 {send$_{ba}$} ;
    \node (A1) [right=.1cm of A0] {$=$} ;
    \node (E1) [pointedevent, right=.3cm of A1, label=below:{$e^b_1$}] {} ;
    \node (E2) [event,        right=2cm of E1, label=below:{$e^b_2$}] {} ;

    \path (E1)
        edge node [above] {$b$} (E2);
\end{tikzpicture}

            \par}
        \end{example}
        
        The \emph{product update} formalizes the execution of an action $(\E, E_d)$ on the current epistemic state $(M, W_d)$. Intuitively, the resulting epistemic state $(M', W_d')$ is computed by a cross product between the worlds in $M$ and the events in $\E$. A pair $(w,e)$ represents the world of $M'$ that results from applying the event $e$ on the world $w$.
        %
        We say that $ (\E, E_d) $ is \emph{applicable} in $ (M, W_d) $ iff for each world $ w_d \in W_d $ there exists an event $e_d\in E_d$ that is applicable in $w_d$. 
        
        \begin{definition}[Product Update]\label{def:update_em}
            Let $ (\E, E_d) $ be an action applicable in an epistemic state $ (M, W_d) $, where $ M = (W, R, V) $ and $ \E = (E, Q, pre, post) $. The \emph{product update} of $ (M, W_d) $ with $ (\E, E_d) $ is the epistemic state $ (M, W_d) \otimes (\E, E_d) = ((W', R', V'), W'_d) $, where:

            {\centering
            $
                \begin{array}{l@{\;}ll}
                    W'	         & = \{(w, e) \in W {\times} E \mid (M, w) \models pre(e)\} \\
                    R'_i	     & = \{((w, e), (v, f)) \in W' {\times} W' \mid w R_i v \text{ and } e Q_i f\} \\
                    V'(\atom{p}) & = \{(w, e) \in W' \mid (M, w) \models post(e)(\atom{p})\} \\
                    W'_d	     & = \{(w, e) \in W' \mid w \in W_d \text{ and } e \in E_d\}
                \end{array}
            $
            \par}
        \end{definition}

        
        \begin{example}\label{ex:update}
            Suppose that general $\mathbf{a}$ sends the messenger to general $\mathbf{b}$ (action send$_{ab}$) and that the message is successfully delivered. The situation is represented by epistemic state $s_1$, where $w'_1 = (w_1, e^a_1)$, $w'_2 = (w_1, e^a_2)$ and $w'_3 = (w_2, e^a_2)$ (recall that the reflexive, symmetric and transitive closures of the relations are left implicit).

            {\centering
                \section{Introduction}
The fundamental dichotomy in contact topology separates manifolds into the collection of overtwisted contact manifolds, which are flexible in the sense that an $h$-principle holds by the seminal work of Eliashberg \cite{zbMATH04121041} and Borman-Eliashberg-Murphy \cite{zbMATH06567662}, and the collection of tight contact manifolds, where some forms of symplectic rigidity are expected. Understanding the boundary between these two phenomena in various forms is at the heart of contact topology. 
 
One way to study contact manifolds is from a surgical perspective. Weinstein \cite{zbMATH00011093} showed that one can modify a contact manifold by attaching a symplectic handle along a neighborhood of an isotropic sphere, which is now referred to as a Weinstein handle \cite{zbMATH06054083}. Such a procedure is called an isotropic surgery by Conway and Etnyre \cite{zbMATH07206659}. One can reverse the procedure by attaching a symplectic handle along a neighborhood of a coisotropic sphere,  this leads to the concept of coisotropic surgeries \cite{zbMATH07206659}. Among them, arguably, the most interesting case is when the sphere is both isotropic and coisotropic, i.e.\ Legendrian. An isotropic surgery along a Legendrian sphere is often called a contact $(-1)$ surgery, while the coisotropic surgery along the Legendrian sphere is called a contact $(+1)$-surgery. Ding and Geiges \cite{zbMATH02103046} showed that every closed\footnote{All contact manifolds are assumed to be closed in this paper.} contact $3$-manifold can be obtained by contact $(\pm 1)$-surgery along a Legendrian link in the standard contact $3$-sphere. In higher dimensions,  Conway and Etnyre \cite{zbMATH07206659} showed that any contact manifold can be obtained from the standard contact sphere from a sequence of isotropic and coisotropic surgeries. Therefore, to determine whether a contact manifold is overtwisted or tight, one needs to understand if tightness is preserved in surgeries. In dimension $3$, by the work of Colin \cite{MR1447038} and Wand \cite{zbMATH06487151}, isotropic surgeries preserve tightness. On the other hand, the contact $(+1)$ surgery along the standard unknot in the standard contact $3$-sphere yields a tight manifold, while we have an overtwisted manifold if we stabilize the unknot and apply the surgery. Hence the devil in the question is coisotropic surgeries, in particular, contact $(+1)$-surgeries. 

Invariants from pseudo-holomorphic curves, e.g.\ symplectic field theory (SFT) by Eliashberg, Givental, and Hofer \cite{zbMATH01643843} and Heegaard Floer theory by Ozsv\'ath and Szab\'o  \cite{zbMATH02144173}, provide necessary conditions for a contact manifold to be overtwisted, namely the contact homology must vanish \cite{zbMATH05709738} from the SFT side and the contact Ozsv\'ath-Szab\'o invariant must vanish \cite{zbMATH02207895} from the Heegaard Floer theory side. Bourgeois and Niederkr{\"u}ger \cite{zbMATH05658836} introduced the notion of algebraically overtwisted manifolds for those with vanishing contact homology. However, neither conditions are sufficient by Avdek \cite{avdek2020combinatorial} and Ghiggini, Honda, and Van Horn-Morris \cite{ghiggini2007vanishing}, hence the combination of the two vanishing properties does not imply overtwistedness either by a contact connected sum. From the surgical perspective, the non-vanishing of contact homology and the non-vanishing of the contact Ozsv\'ath-Szab\'o invariant (both hold for the standard contact sphere when they can apply) are preserved in isotropic surgeries by the functoriality of those invariants. While their behaviors under coisotropic surgeries are more complicated as illustrated by the same example above. Even though both conditions are not sufficient to determine overtwistedness, understanding them in $(+1)$-surgeries can be viewed as the first step towards the geometric question of overtwistedness through coisotropic surgeries. On the Heegaard Floer theory side, a complete answer for the vanishing of the contact Ozsv\'ath-Szab\'o invariant in $(+1)$-surgery along a Legendrian knot was obtained by Golla \cite{zbMATH06413573}. In \cite{DLW}, Ding, Li, and Wu studied the vanishing of the contact Ozsv\'ath-Szab\'o invariant for $(+1)$-surgeries on two-component links. On the SFT side, the vanishing of contact homology through $(+1)$-surgeries was first studied by Avdek \cite{avdek2020combinatorial} in the standard contact $3$-sphere. In this paper, we study the same question but for general dimensions. In particular, our main theorem below can be viewed as an SFT analog of Ding-Li-Wu's result. 
\begin{theorem}\label{thm:main}
    Let $Y^{2n-1}$ be the contact boundary of a Liouville domain $W$, where $W$ is one of the following:
    \begin{enumerate}
        \item $W=V\times \D$ for a Liouville domain $V$ and $\D\subset \C$ is the unit disk, in particular, any subcritical Weinstein domain.
        \item $W$ is a flexible Weinstein domain \cite{zbMATH06054083} with $c_1(W)\in H^2(W;\Z)$ torsion.
    \end{enumerate}
    Let  $\Lambda$ be a Legendrian sphere in $Y$, such that $[\Lambda]\in H_{n-1}(\partial W;\Q)$ is nontrivial in $H_{n-1}(W;\Q)$.   Then the contact manifold $Y_{\Lambda}$ from a $(+1)$-surgery\footnote{The $+1$ surgery depends on a parametrization $L\simeq S^n$.} along $\Lambda$ is algebraically overtwisted, i.e.\ the contact homology vanishes. 
\end{theorem}
\begin{remark}\label{rmk:twist}
    Moreover, the contact homology over the twisted coefficient $\Q[H_2(Y_{\Lambda};\R)]$\footnote{It corresponds to $\cR=0$, i.e.\ the fully twisted theory in \cite{zbMATH06000009}.}  also vanishes for all contact manifolds from $(+1)$-surgeries in this paper. This implies all such contact manifolds have no weak fillings by \cite[Theorem 5]{zbMATH06000009}.
\end{remark}

An instant corollary of \Cref{thm:main} is the following.
\begin{corollary}\label{cor:OT}
    Let $V$ be a Liouville domain and $L\subset V$ be a Lagrangian sphere such that $[L]\ne 0 \in H_*(V;\Q)$, then for any Dehn-Seidel twist $\tau_L$\footnote{As a Dehn-Seidel twist also depends on a parametrization $L\simeq S^n$.}, the open book $\mathrm{OB}(V,\tau^{-1}_{L})$ with page $V$ and monodromy $\tau^{-1}_L$ has vanishing contact homology.
\end{corollary}
\begin{proof}
    The open book $\mathrm{OB}(V,\tau^{-1}_{L})$ is obtained from $(+1)$-surgery on the Legendrian lift of $L$ in the open book $\mathrm{OB}(V,\Id)=\partial(V\times \D)$, then \Cref{thm:main} applies as $H_*(V;\Q)\to H_*(\partial(V\times \D);\Q) \to H_*(V\times \D;\Q)$ is injective. 
\end{proof}

In particular, homotopically standard overtwisted $S^{2n+1}= \mathrm{OB}(T^*S^n,\tau^{-1})$ has vanishing contact homology, this was established by Bourgeois and van Koert \cite{zbMATH05709738} by direct computation. The assumption on the fundamental class of $L$ is likely to be redundant in view of the regular Lagrangian conjecture of Eliashberg, Ganatra, and Lazarev \cite[Problem 2.5]{zbMATH07195660}. Although many of the open books in \Cref{cor:OT} are negative stabilization, hence overtwisted \cite{zbMATH07010365}, it is unclear whether \Cref{cor:OT} always yield overtwisted manifolds.

\begin{corollary}\label{cor:3D}
    Let $\Lambda \cup U$ be a two-component link in $(S^3,\xi_{\std})$ with a nontrivial linking number and $U$ is the standard unknot, then the $(+1)$ surgeries along $\Lambda \cup U$ yield a contact manifold with vanishing contact homology.
\end{corollary}
\begin{proof}
    We first apply $(+1)$ surgery along $U$ to get $Y=\partial(T^*S^1\times \D)=S^1\times S^2$, then $\Lambda$ becomes a Legendrian knot $\Lambda'$ on $Y$ representing a nontrivial homology class in the $S^1$ factor as the linking number is nontrivial. Then we apply \Cref{thm:main} to $\Lambda'$. 
\end{proof}

Ding, Li, and Wu \cite[Theorem 1.1]{DLW} showed that the contact Ozsv\'ath-Szab\'o invariant also vanishes for contact manifolds in \Cref{cor:3D}. In fact, they established the vanishing result for other types of $U$, which are ``unknots" in the Heegaard Floer theory sense. On the other hand, the nontrivial linking number is a crucial requirement, and so is the homology requirement in our formulation. Moreover, our construction enjoys a local property as follows.

\begin{theorem}\label{thm:main'}
    In the following two cases:
    \begin{enumerate}
        \item\label{thm1} $Y_1$ is flexibly fillable or $Y_1=\partial(V\times \D)$ such that $c_1(Y)$ is torsion, $Y_2$ is a contact manifold of the same dimension with $c_1(Y_2)$ torsion.
        \item\label{thm2} $Y_1=\partial(V\times \D)$ for a Weinstein domain $V$, $Y_2$ is a contact manifold of the same dimension.
    \end{enumerate}
    If $\Lambda$ is a Legendrian sphere in $Y=Y_1\# Y_2$, such that $[\Lambda]$ has non-trivial image through $H_{n-1}(Y;\Q)\simeq H_{n-1}(Y_1;\Q)\oplus H_{n-1}(Y_2;\Q)\to H_{n-1}(Y_1;\Q) \to H_{n-1}(W;\Q)$, where $W$ is the natural filling in \Cref{thm:main}, then $Y_{\Lambda}$ is algebraically overtwisted.
\end{theorem}
Then we can upgrade \Cref{cor:3D} to the following for the special case of $Y_1=\partial(T^*S^{n-1}\times \D)$.
\begin{corollary}
    Let $\Lambda,U$ be two Legendrian spheres in $Y$, where $Y$ is a $2n-1$ dimensional contact manifold, and $U$ is a standard unknot in a Darboux chart. If the linking number is nontrivial\footnote{Here the linking number is defined to the intersection number of $\Lambda$ with a bounding ball of $U$ in the Darboux chart.},  the $(+1)$ surgeries along $\Lambda \cup U$ yield a contact manifold with vanishing contact homology.
\end{corollary}
As any overtwisted contact manifold $Y\# (S^{2n-1},\xi_{\mathrm{ot}})$ can be written as $(+1)$ surgeries from such links, this yields another proof of overtwisted contact manifold having vanishing contact homology, which was first proved by Bourgeois and van Koert \cite{zbMATH05709738}. Combined with \Cref{rmk:twist}, this gives another proof of the following:
\begin{corollary}[\cite{zbMATH06182635,schmaltz2020non}]
    Overtwisted contact manifolds have no weak filling.
\end{corollary}

A $(+1)$-surgery gives rise to a Weinstein cobordism whose concave boundary is $Y_{\Lambda}$, while the convex boundary is $Y$ and $\Lambda$ is filled by a Lagrangian disk in the cobordism. On the other hand, contact manifold $Y$ in \Cref{thm:main} enjoys strong uniqueness property for symplectic fillings by \cite{zbMATH07367119,zbMATH07673358}, in particular, the homology class $[\Lambda]$ should survive in the filling. Indeed, one can apply \cite[Theorem 4.4]{bowden2022making} to prove that the contact manifold from $(+1)$-surgery has no strong fillings. The proof of \Cref{thm:main,thm:main'} follows from singling out the pseudo-holomorphic curves obstructing fillings, whose degeneration in the surgery cobordism then yields the vanishing of contact homology of the concave boundary. Contact homology of $(+1)$ surgeries was studied by Avdek \cite{Av,avdek2020combinatorial}, where a much deeper picture between the relative SFT of the convex boundary and the absolute SFT of the concave boundary was studied. We point out here that our results remain in the realm of absolute SFT, i.e.\ we only use the topology of the surgery cobordism but not holomorphic curves with Lagrangian boundary conditions. 

Our proof has a functorial explanation as follows. Let $Y$ be a contact manifold, one tries to define the positive symplectic cohomology, where the underlying cochain complex $C_+(Y)$ is generated by two generators from each Reeb orbit. $C_+(Y)$ does not always form a cochain complex, but $C_+(Y)\otimes \CC(Y)$ is a $\CC(Y)$ DGA-module, where $\CC(Y)$ is the contact homology algebra (chain level) of $Y$ and the differential counts Floer cylinders with negative punctures asymptotic to Reeb orbits. The cochain complex for positive symplectic cohomology of an exact filling $W$ is then the tensor product with the ground field using the augmentation from $W$. Now let $W$ be an exact cobordism (e.g.\ the surgery cobordism) from concave boundary $Y_-$ to convex boundary $Y_+$, then we have the following diagram, which is commutative on homology,
$$
\xymatrix{
C_+(Y_+)\otimes \CC(Y_+)\ar[d] \ar[r] & C(Y_+)\otimes \CC(Y_+)\ar[d]\\
C(W,Y_-) \otimes \CC(Y_-)\ar[r] & C(Y_+)\otimes \CC(Y_-)}
$$
where $C(Y_{\pm}),C(W,Y_-)$ are Morse cochain complexes. When phrased using such a structure, the core of the proofs is finding a closed class in $C_+(Y_+)\otimes \CC(Y_+)$ that is mapped to $\alpha \otimes 1 \in H^*(Y_+)\otimes \CH(Y_+)$ through the top map, such that $\alpha$ is not in the image of $H^*(W,Y_-)\to H^*(Y_+)$. Then we must have $1=0$ in $\CH(Y_-)$. However, such an element is easy to find for $Y_+=Y$ in \Cref{thm:main,thm:main'}.

It is quite a challenge to determine whether contact manifolds in \Cref{thm:main,thm:main'} are overtwisted. In dimension $3$, there are sufficient conditions for the $(+1)$-surgeries to yield overtwisted manifolds by Ozbagci \cite{zbMATH02147034} and Lisca-Stipsicz \cite{zbMATH05190395} for knots and Ding-Li-Wu \cite{DLW} for links. In higher dimensions, Casals, Murphy, and Presas \cite{zbMATH07010365} showed that $(+1)$-surgeries along loose Legendrian spheres give overtwisted manifolds. Indeed, some cases of \Cref{thm:main} give overtwisted manifolds, for example, $W=T^*{S^{n-1}}\times \D$ and $\Lambda$ is the Legendrian lift of the Lagrangian zero section in $T^*S^{n-1}$, as this Legendrian is loose/stabilized. On the other hand, there are Legendrian knots with the homology property in \Cref{thm:main} that are not stabilized found by Ekholm and Ng \cite[Corollary 2.22, Proposition 3.9]{zbMATH06471194}. In higher dimensions, we have many such Legendrians from exotic Weinstein balls constructed in \cite{abouzaid2010altering,zbMATH05553983,zbMATH02242665,zbMATH07367119} using the work of Lazarev \cite{zbMATH07305775}.
\begin{proposition}\label{prop:exotic}
    For $Y=\partial(T^*S^{n-1}\times \D)\simeq S^{n-1}\times S^n$ with $n\ge 3$, there are infinitely many different non-loose Legendrian spheres in $Y$ that are smoothly isotopic to the standard loose $S^{n-1}$ above. When $n$ is odd\footnote{We expect this condition to be redundant.}, those Legendrians are formally Legendrian isotopic to the standard loose $S^{n-1}$.
\end{proposition}

One of the motivations of this project is to study the differences between overtwisted manifolds and algebraically overtwisted manifolds. 
\begin{question}[Folklore]\label{question:AO}
For $n\ge 2$, are there algebraically overtwisted but tight $2n-1$ dimensional contact manifolds?
\end{question}
To put it in a broader perspective, this question is one of the fundamental questions to understand the boundary between flexibility and rigidity phenomena in symplectic and contact topology. The first example of an algebraically overtwisted tight manifold was found by Avdek \cite{avdek2020combinatorial} in dimension $3$. The example follows from a $(+1)$-surgery on $(S^3,\xi_{\std})$ along a trefoil knot, and the tightness follows from the non-vanishing of the contact Ozsv\'ath-Szab\'o invariant. In view of \cite[Theorem 1.1]{DLW}, although \Cref{thm:main} may give new examples in dimension $3$, the tight criterion from contact Ozsv\'ath-Szab\'o invariant can not help, i.e.\ we need other criteria of tightness. In fact, the lack of tight criteria beyond contact homology is the major difficulty in answering \Cref{question:AO} in higher dimensions. Indeed, the existence of fillings, hypertight property, and properties on Conley-Zehnder indices, used as tight criteria in general dimensions, are all manifestations of the non-vanishing of contact homology. Nevertheless, \Cref{thm:main} potentially solves the easy half of \Cref{question:AO} by providing a flexible enough list of algebraically overtwisted manifolds. More precisely, we ask the following question. 

\begin{question}
    If we apply a $(+1)$-surgery along Legendrian spheres in \Cref{prop:exotic}, do we get (different) tight contact manifolds? 
\end{question}
We suspect the answer to be affirmative, for otherwise, the $(+1)$-surgeries would yield the homotopically standard overtwisted sphere, i.e.\ we get infinitely many different ways to get the homotopically standard overtwisted sphere but with the same formal data (at least for $n$ odd). 
\subsection*{Acknowledgments}
The author is supported by the National Natural Science Foundation of China under Grant No.\ 12288201 and 12231010. The author is grateful to Russell Avdek for enlightening discussions which lead to the functorial perspective in \S \ref{ss:43}, and Otto van Koert for pointing out \cite{zbMATH06562001} which leads to the proof of \Cref{prop:corb'} and their feedback on a preliminary version of the paper. The author would like to thank Youlin Li and Zhongtao Wu for helpful discussions and interest in the project. 
            \par}
            
            Now general $\mathbf{b}$ knows about the intentions of his ally ($\B{b}d$), but general $\mathbf{a}$ does not know that general $\mathbf{b}$ knows. So, $\mathbf{b}$ decides to send the messenger back to acknowledge that the message was received (action send$_{ba}$). Here, $d$ assumes the meaning of ``general $\mathbf{b}$ has received the message''. Assume again that the messenger succeeds. We obtain the epistemic state $s_2$, where $w''_1 = (w'_1, e^b_1)$, $w''_2 = (w'_1, e^b_2)$, $w''_3 = (w'_2, e^b_2)$ and $w''_4 = (w'_3, e^b_2)$.
            
            {\centering
                \begin{tikzpicture}[-,>=stealth',shorten >=1pt,auto,semithick]
    \node (A0) []                 {$s_2$} ;
    \node (A1) [right=.1cm of A0] {$=$} ;
    \node (W1) [pointedworld, right=.3cm of A1,  label=below:{$w''_1 : d, m_a$}] {} ;
    \node (W2) [world,        right=1.5cm of W1, label=below:{$w''_2 : d$}] {} ;
    \node (W3) [world,        right=1.5cm of W2, label=below:{$w''_3 : d$}] {} ;
    \node (W4) [world,        right=1.5cm of W3, label=below:{$w''_4 $}] {} ;

    \path (W1)
        edge node [above] {$b$} (W2) ;
    
    \path (W2)
        edge node [above] {$a$} (W3) ;

    \path (W3)
        edge node [above] {$b$} (W4) ;
    
\end{tikzpicture}

            \par}

            Now it holds that $\B{a}\B{b}d$, but it does not hold that $\B{b}\B{a}\B{b}d$. So, general $\mathbf{a}$ would need to send the messenger once again to general $\mathbf{b}$. However, it can be intuitively seen that, regardless of how many messages the generals exchange, they will never be sure that the other will attack at dawn. This will be stated formally in Section \ref{sec:plan_ex}.
        \end{example}

    \subsection{The logic S5$_n$}\label{sec:s5}
        In the epistemic logic literature there exist many different axiomatizations of the concept of \emph{knowledge}. In this paper, we adopt the multimodal logic S5$_n$. Its axioms are\footnote{Even though \axiom{K}, \axiom{T} and \axiom{5} are sufficient to characterize S5$_n$, we include axiom \axiom{4} as it constitutes an important epistemic principle.}:
        
        {\centering
        $
            \begin{array}{l@{}l@{}r@{}}
                \axiom{K}\;\; & \B{i} (\varphi \rightarrow \psi) \rightarrow
                                (\B{i} \varphi \rightarrow \B{i} \psi)  			& \textnormal{(Distribution)}           \\
                \axiom{T} & \B{i} \varphi \rightarrow \varphi                       & \textnormal{(Knowledge)}              \\
                \axiom{4} & \B{i} \varphi \rightarrow \B{i} \B{i} \varphi           & \textnormal{(Positive introspection)} \\
                \axiom{5} & \neg \B{i} \varphi \rightarrow \B{i} \neg \B{i} \varphi & \textnormal{(Negative introspection)}
            \end{array}
        $
        \par}

        \noindent Axioms \axiom{T}, \axiom{4} and \axiom{5} correspond, to the following \emph{frame properties}: reflexivity ($ \forall u (u R_i u) $), transitivity ($ \forall u, v, w (u R_i v \wedge v R_i w \rightarrow u R_i w) $) and Euclidicity ($ \forall u, v, w (u R_i v \wedge u R_i w \rightarrow v R_i w) $). 
        Moreover, axioms \axiom{T} and \axiom{5} together entail symmetry ($ \forall u, v (u R_i v \rightarrow v R_i u) $). 
        Thus, accessibility relations in S5$_n$ are equivalence relations.
        We refer to epistemic states (resp., epistemic models, actions, frames) satisfying the axioms of a logic $L$ as $L$-states (resp., $L$-models, $L$-actions, $L$-frames). In the rest of the paper, we assume that the accessibility relations of epistemic states and actions are equivalence relations.

\subsection{Plan Existence Problem}\label{sec:plan_ex}
	We now define our problem, adapting the formulation in~\cite{conf/ijcai/Aucher2013}. 

    \begin{definition}[Planning Task]
    \label{def:planning_task}
        An \emph{(epistemic) planning task} is a triple $ T = (s_0, \actionSet,$ $ \varphi_g) $, where $ s_0 $ is an initial epistemic state; $ \actionSet $ is a finite set of actions; $ \varphi_g \in \Lang{C} $ is a \emph{goal formula}.
    \end{definition}

    Given a logic $ L $, an \emph{$L$-planning task} $ (s_0, \actionSet, \varphi_g) $ is a planning task where $s_0$ is an $L$-state and each action in $\actionSet$ is an $L$-action. We denote the class of $L$-planning tasks with $\mathcal{T}_L$.
    We remark that, given a generic logic $L$, the product update of an L-state with an L-action, in general, is not necessarily an L-state. This is not a desired outcome in general, since axioms model some principles of knowledge/belief that always need to be satisfied.
    For instance, this is the case of the logic KD45$_n$, that captures the concept of \emph{belief}. In the literature, there exist different approaches to guarantee the preservation of the KD45$_n$ frame properties after the product update. For instance, some techniques involve belief revision techniques \cite{workshop/nrac/Herzig2005}, whereas others focus on defining some additional conditions to impose to both states and actions \cite{conf/aaai/Son2015}.

    The case of our logic C-S5$_n$ is similar to that of KD45$_n$. In fact, the frame property corresponding to axiom \axiom{C} (see Equation \ref{eq:A-frame} in Section \ref{sec:new_logic}) is not guaranteed to hold after the application of an action. In this paper, rather than devising some technique to guarantee the preservation of frame property \ref{eq:A-frame}, we instead opt for a rollback-style approach: an action is not to be applied in a state if it would lead to violate the axioms of C-S5$_n$.
    This leads to the following definition.

    \begin{definition}[Solution]\label{def:solution}
        A \emph{solution} to an $L$-planning task $(s_0, \actionSet,$ $ \varphi_g)$ is a finite sequence $ \alpha_1, \dots, \alpha_m $ of actions of $\actionSet$ such that:
        \begin{compactenum}
            \item $ s_0 \otimes \alpha_1 \otimes \dots \otimes \alpha_m \models \varphi_g $, and
            \item For each $ 1 \leq k \leq m $, $ \alpha_k $ is applicable in $ s_0 \otimes \alpha_1 \otimes \dots \otimes \alpha_{k-1} $ and $s_0 \otimes \alpha_1 \otimes \dots \otimes \alpha_k$ is an $L$-state.
        \end{compactenum}
    \end{definition}
    
    \begin{definition}[Plan Existence Problem]\label{def:plan_ex_problem}
        Let $n \geq 1$ and $\mathcal{T}_L$ be a class of epistemic planning tasks for a logic $L$. \planex{$\mathcal{T}_L$}{$n$} is the following decision problem: ``Given an $L$-planning task $ T = (s_0, \actionSet, \varphi_g) \in \mathcal{T}_L $, where $|\agentSet|=n$, does $ T $ have a solution?''
    \end{definition}

    
    \begin{example}\label{ex:task}
        In Example~\ref{ex:update} we have seen that, intuitively speaking, the two generals can not coordinate a winning attack. We now state this formally. Let $T_\textnormal{coord} = (s_0, \actionSet, \varphi_g) $ be an S5$_n$-planning task, where $\actionSet = \{\textnormal{send}_{ab}, \textnormal{send}_{ba}\}$ and $\varphi = \CK{\{a,b\}}d$. Then, $T_\textnormal{coord}$ has no solution. In fact, for any number $k \geq 0$ of delivered messages, one can show by induction the following (where $h\geq 0$):
        \begin{compactitem}
            \item $k{=}2h$: $(\B{a}\B{b})^h\B{a} d $ holds in $s_k$, but not $(\B{b}\B{a})^{h+1}d$;
            \item $k{=}2h{+}1$: $(\B{b}\B{a})^{h+1}d $ holds in $s_k$, but not $(\B{a}\B{b})^{h+1}\B{a}d$.
        \end{compactitem}
        \noindent Thus, common knowledge can not be achieved between the two generals in a finite number of steps. However, any search algorithm would never terminate, since at each step there is exactly one applicable action that, when applied, results in a new S5$_n$-state.
    \end{example}

    \section{\textsc{delphic}}\label{sec:delphic}
    We introduce  the \textsc{delphic} framework for epistemic planning. \textsc{delphic} is built around the concept of \emph{possibility} (Definition \ref{def:possibility}), first introduced by Gerbrandy and Groeneveld to represent epistemic states. We develop a novel representation for epistemic actions inspired by possibilities, which we term \emph{eventualities} (Definition \ref{def:eventuality}). Then, we present a novel characterisation of update, called \emph{union update} (Definition \ref{def:update_pem}), based on possibilities and eventualities.

    \subsection{Possibilities}\label{sec:poss}
        %We now describe an alternative representation of epistemic states based on the notion of \emph{possibility}~\cite{journals/jolli/Gerbrandy1997}. 
        Possibilities are tightly related to \emph{non-well-founded sets}, \ie sets that may give rise to infinite \emph{descents} $X_1 \in X_2 \in \dots$ (\eg $\Omega = \{\Omega\}$ is a n.w.f. set). We refer the reader to Aczel \cite{books/csli/Aczel1988} for a detailed account on non-well-founded set theory.

        \begin{definition}[Possibility]\label{def:possibility}
            A \emph{possibility} $ \poss{u} $ for $ \Lang{} $ is a function that assigns to each atom $ \atom{p} \in \atomSet $ a truth value $ \poss{u}(\atom{p}) \in \{0, 1\} $ and to each agent $ i \in \agentSet $ a \emph{set of possibilities} $ \poss{u}(i) $, called \emph{information state}.
        \end{definition}

        \begin{definition}[Possibility Spectrum]\label{def:p-spectrum}
            A \emph{possibility spectrum} is a finite set of possibilities $ \poss{U} = \{\poss{u}_1, \dots \poss{u}_k\} $ that we call \emph{designated possibilities}.
        \end{definition}

        \noindent Possibility spectrums represent epistemic states in \textsc{delphic} and are able to represent the same information as MPKMs. Intuitively, each possibility $\poss{u}$ represent a possible world and the components $ \poss{u}(\atom{p})$ and $ \poss{u}(i)$ correspond to the valuation function and the accessibility relations of the world, respectively. Finally, the possibilities in a possibility spectrum represent the designated worlds. We formalize this intuition in Proposition \ref{prop:p-comparison}.
        
        % Notice that, on the one hand, worlds, accessibility relations and the valuation function of Kripke models are given as separated components. On the other hand, each possibility contains both a valuation $ \poss{u}(\atom{p}) $ for each atom in $ \atomSet $, and a set of worlds $ \poss{u}(i) $ that agent $ i $ considers possible, for each agent in $ \agentSet $.

        \begin{definition}[Truth in Possibilities]\label{def:p-truth}
            Let $ \poss{u} $ be a possibility, $ i \in \agentSet $, $ \atom{p} \in \atomSet $ and $\varphi,\psi \in \Lang{C}$ be two formulae. Then,
            
            {\centering
            $
                \begin{array}{@{}lll}
                    \poss{u} \models \atom{p}            & \text{ iff } & \poss{u}(\atom{p}) = 1 \\
                    \poss{u} \models \neg \varphi        & \text{ iff } & \poss{u} \not\models \varphi \\
                    \poss{u} \models \varphi \wedge \psi & \text{ iff } & \poss{u} \models \varphi \text{ and } \poss{u} \models \psi \\
                    \poss{u} \models \Box_i \varphi      & \text{ iff } & \forall \poss{v} \text{ if } \poss{v} \in \poss{u}(i) \text{ then } \poss{v} \models \varphi
                \end{array}
            $\par}
            \noindent Moreover, $ \poss{U} \models \varphi $ iff $ \poss{v} \models \varphi $, for all $ \poss{v} \in \poss{U} $.
        \end{definition}

        \paragraph{Comparing Possibilities and Kripke Models.}
        Gerbrandy and Groeneveld \cite{journals/jolli/Gerbrandy1997} show how possibilities and Kripke models correspond to each other. In what follows, we extend this result by analyzing the relation between possibility spectrums and MPKMs. First, following \cite{journals/jolli/Gerbrandy1997}, we give some definitions.

        \begin{definition}[Decoration of Kripke Model]\label{def:dec_km}
            The \emph{decoration} of a Kripke model $ M = (W, R, V) $ is a function $ \delta $ that assigns to each world $w \in W$ a possibility $ \poss{w} = \delta(w) $, such that:
            \begin{compactitem}
                \item $ \poss{w}(\atom{p}) = 1 \text{ iff } w \in V(\atom{p}) $, for each $ \atom{p} \in \atomSet $;
                \item $ \poss{w}(i)  = \{\delta(w') \mid w R_i w'\} $, for each $i \in \agentSet $.
            \end{compactitem}
        \end{definition}

        \noindent Intuitively, decorations provide a link between Kripke-based representations and their equivalent possibility-based ones: given $w$ in $M$, the decoration of $M$ returns the possibility that encodes $w$ (its valuation and accessibility relation). 

        \begin{definition}[Picture and Solution]\label{def:pic_km}
            If $ \delta $ is the decoration of a Kripke model $ M = (W, R, V) $ and $ W_d \subseteq W $, then $ (M, W_d) $ is the \emph{picture} of the possibility spectrum $ \poss{W} = \{\delta(w) \mid w \in W_d\} $. $ \poss{W} $ is called \emph{solution} of $ (M, W_d) $.
        \end{definition}

        \noindent Namely, the solution of a MPKM $ (M, W_d) $ is the possibility spectrum $\poss{W}$ that contains the possibilities calculated by the decoration function, one for each designated world in $W_d$. Finally, $ (M, W_d) $ is the picture of $\poss{W}$. Notice that, in general, \emph{different} MPKMs may share the \emph{same} solution. This observation will be formally stated in Proposition \ref{prop:p-comparison}. We now give an example (see also Figure~\ref{fig:ex4}).
        
        \begin{example}\label{ex:dec_km}
            The decoration $ \delta $ of the MPKM of Example~\ref{ex:k_model} assigns the possibilities $ \poss{w}_1 = \delta(w_1) $, $ \poss{w}_2 = \delta(w_2) $. Since $W_d = \{w_1\}$, we have that $ \poss{W} = \{\poss{w}_1\} $ is the solution of $(M,W_d)$, where:
            \begin{compactitem}
                \item $ \poss{w}_1(\atom{h}) = 1 $ and $\poss{w}_1(a) = \poss{w}_1(b) = \{\poss{w}_1, \poss{w}_2\} $;
                \item $ \poss{w}_2(\atom{h}) = 0 $ and $\poss{w}_2(a) = \poss{w}_2(b) = \{\poss{w}_1, \poss{w}_2\} $.
            \end{compactitem}
            %\noindent A graphical representation is given in Figure~\ref{fig:ex4}.
        \end{example}

        % Figure environment removed

        Notice that, in Example~\ref{ex:dec_km}, although the possibility $\poss{w_2}$ is not explicitly part of $\poss{W}$, it is ``stored'' \emph{within} $\poss{w_1}$. That is, we do not lose the information about $\poss{w_2}$.

        Given the above definitions, we are now ready to formally compare possibility spectrums with MPKMs. The following result generalize the one by Gerbrandy and Groeneveld \cite[Proposition 3.4]{journals/jolli/Gerbrandy1997}:

        \begin{proposition}\label{prop:p-comparison}\hfill
            \begin{compactenum}
                \item Each MPKM has a unique decoration;
                \item Each possibility spectrum has a MPKM as its picture;
                \item\label{compact-p} Two MPKMs have the same solution iff they are bisimilar.
                % \item If $ (M, w) $ is a picture of $ \poss{w} $, then $ (M, w) \models \varphi \text{ iff } \poss{w} \models \varphi $.
            \end{compactenum}
        \end{proposition}

        From item \ref{compact-p} of the above Proposition, we obtain the following remark:

        \begin{remark}\label{rem:compactness}
            Let $s = (M, W_d)$ be a MPKM and let $s'$ be its bisimulation contraction (\ie the smallest MPKM that is bisimilar to $s$). Since $s$ and $s'$ share the same solution $\poss{W}$, it follows that possibility spectrums naturally provide a more compact representation w.r.t. MPKMs.
        \end{remark}
        
        %Therefore: %From this consideration, we obtain the following remark:
		%
        %\begin{remark}\label{rem:p-compact}
        %    Since the class $\mathcal{S}_\poss{W}$ always contains a model $s$ that is minimal w.r.t. bisimulation contraction, it follows that possibility spectrums provide us with a inherently \emph{compact} way of representing epistemic states.
        %\end{remark}\todo{fix: possibilities are equivalent to bisimulation contractions}

        Finally, we show that the solution of a MPKM preserves the truth of formulae.

        \begin{proposition}\label{prop:truth}
            Let $(M, W_d)$ be a MPKM and let $\poss{W}$ be its solution. Then, for every $\varphi \in \Lang{}$, $(M,W_d) \models \varphi$ iff $\poss{W} \models \varphi$.

            \begin{proof}
                Let $\delta$ be the decoration of $(M, W_d)$. We denote with $eq(\psi)$ the fact that $(M,w) \models \psi$ iff $\delta(w) \models \psi$, for all $w {\in} W$.
                
                Consider now $w \in W$ and let $\poss{w} = \delta(w)$. We only need to show that $eq(\varphi)$ holds for any $\varphi \in \Lang{}$. The proof is by induction of the structure of $\varphi$. For the base case, let $\varphi=\atom{p}$. By Definition \ref{def:dec_km}, we immediately have that, for any $\atom{p} \in \atomSet$ and $w \in W$, $(M, w) \models \atom{p}$ iff $\poss{w} \models \atom{p}$ (\ie $eq(\atom{p})$).
                %
                For the inductive step, we have:
                \begin{compactitem}
                    \item Let $\varphi {=} \neg \psi$. From $eq(\psi)$ we get $eq(\neg\psi)$;% and assume $eq(\psi)$. Then, we have $eq(\neg\psi)$.
                    \item Let $\varphi {=} \psi_1 {\wedge} \psi_2$. From $eq(\psi_1)$, $eq(\psi_2)$ we get $eq(\psi_1 {\wedge} \psi_2)$;%and assume $eq(\psi_1)$ and $eq(\psi_2)$. Then, we have $eq(\psi_1 {\wedge} \psi_2)$.
                    %\item Let $\varphi {=} \Box_i\psi$ and assume $eq(\psi)$. Then, we have:
                    %{\centering
                    %    \begin{tabular}{lll}
                    %        $(M, w) {\models} \Box_i\psi$ & $\textover{\Leftrightarrow}{Def. \ref{def:k-truth}}$           & $\forall v \text{ if } w R_i v, \text{ then } (M, v) {\models} \psi$ \\
                    %                                      & $\textover{\Leftrightarrow}{Def. \ref{def:dec_km}, $eq(\psi)$}~$ & $\forall \poss{v} \text{ if } \poss{v} \in \poss{u}_i, \text{ then } \poss{v} {\models} \psi$ \\
                    %                                      & $\textover{\Leftrightarrow}{Def. \ref{def:p-truth}}$             & $\poss{w} \models \Box_i\psi$
                    %    \end{tabular}
                    %\par}
                    \item Let $\varphi {=} \Box_i\psi$ and assume $eq(\psi)$. Then we have: 
                    
                     %
                    {\centering
                       %	
                                 $(M, w) {\models} \Box_i\psi$
                                 %
                                 $\;\;\textover{\Leftrightarrow}{Def. \ref{def:k-truth}}\;\;$
                                 %
                                 $\forall v \text{ if } w R_i v, \text{ then } (M, v) {\models} \psi$ \\
                                 %
                                 $\textover{\Leftrightarrow}{Def. \ref{def:dec_km}, $eq(\psi)$}\quad\;\; $
                                 %
                                 $\forall \poss{v} \text{ if } \poss{v} \in \poss{u}_i, \text{ then } \poss{v} {\models} \psi$ 
                                 %
                                 $\;\;\textover{\Leftrightarrow}{Def. \ref{def:p-truth}}\;\;$
                                 %
                                 $\poss{w} \models \Box_i\psi$
          
                    \par}
                \end{compactitem}
            \end{proof}
        \end{proposition}

    \subsection{Eventualities}
        In \textsc{delphic}, we introduce the novel concept of \emph{eventuality} to model epistemic actions that is compatible with possibilities. In the remainder of the paper, we fix a fresh propositional atom $ \atom{pre} \notin \atomSet $ and let $ \atomSet' = \atomSet \cup \{\atom{pre}\} $. In the following definition, $ \atom{pre} $ encodes the precondition of an event, while the remaining atoms in $ \atomSet $ encode postconditions.

        \begin{definition}[Eventuality]\label{def:eventuality}
            An \emph{eventuality} $ \pem{e} $ for $ \Lang{} $ is a function that assigns to each atom $ \atom{p'} \in \atomSet' $ a formula $ \pem{e}(\atom{p'}) \in \Lang{} $ and to each agent $ i \in \agentSet $ a \emph{set of eventualities} $ \pem{e}(i) $, called \emph{information state}.
        \end{definition}

        \noindent Note that an eventuality is essentially a possibility that associates to each atom a formula (instead of a truth value).

        \begin{definition}[Eventuality Spectrum]\label{def:e-spectrum}
            An \emph{eventuality spectrum} is a finite set of eventualities $ \poss{E} {=} \{\poss{e}_1, \dots \poss{e}_k\} $ that we call \emph{designated eventualities}.
        \end{definition}

        \noindent Eventuality spectrums represent epistemic actions in \textsc{delphic}. 
        Moreover, we can easily show that they are able to represent the same information as MPEMs. 
        Intuitively, each eventuality $\pem{e}$ represents an event and the components $\pem{e}(\atom{pre})$ and $\pem{e}(\atom{p})$ represent the precondition and the postconditions of the event, respectively. Finally, the eventualities in an eventuality spectrum represent the designated events. We formalize this intuition in Proposition \ref{prop:e-comparison}.
        

    \paragraph{Comparing Eventualities and Event Models.}
        We now analyze the relationship between eventuality spectrums and MPEMs. We introduce the notions of decoration, picture and solution for event models.

        \begin{definition}[Decoration of an Event Model]\label{def:dec_em}
            The \emph{decoration} of an event model $ \E = (E, Q, pre, post) $ is a function $ \delta $ that assigns to each $ e \in E $ an eventuality $ \pem{e} = \delta(e) $, where:
            \begin{compactitem}
                \item $ \pem{e}(\atom{pre}) = pre(e)$ and $ \pem{e}(\atom{p}) = post(e)(\atom{p}) $, for each $\atom{p}\in \atomSet$;
                \item $ \pem{e}(i)          = \{\delta(e') \mid e Q_i e'\} $, for each $i \in \agentSet $.
            \end{compactitem}
        \end{definition}

        \begin{definition}[Picture and Solution]\label{def:pic_em}
            If $ \delta $ is the decoration of an event model $ \E = (E, Q, pre, post) $ and $ E_d \subseteq E $, then $ (\E, E_d) $ is the \emph{picture} of the eventuality spectrum $ \pem{E} = \{\delta(e) \mid e \in E_d\} $ and $ \pem{E} $ is the \emph{solution} of $ (\E, E_d) $.
        \end{definition}

        The above definitions are the counterparts of the notions of decoration and picture given in Definitions~\ref{def:dec_km} and \ref{def:pic_km}. %We now give a small example.

        \begin{example}\label{ex:dec_em}
            The decoration $ \delta $ of the MPEM of Example~\ref{ex:e_model} assigns the eventualities $ \pem{e}_1 = \delta(e_1) $ and $ \pem{e}_2 = \delta(e_2) $. Since $E_d = \{e_1\}$, we have that $ \pem{E} = \{\pem{e}_1\} $ is the solution of $(\E,E_d)$, where:
            \begin{compactitem}
                \item $ \pem{e}_1(\atom{pre}) = \atom{h} $; $\pem{e}_1(\atom{h}) = \atom{h}$; $\pem{e}_1(a) {=} \{\pem{e}_1\}$ and $\pem{w}_1(b) {=} \{\pem{w}_2\} $;
                \item $ \pem{e}_2(\atom{pre}) = \top $;     $\pem{e}_2(\atom{h}) = \atom{h}$; $\pem{e}_2(a) = \pem{w}_2(b) = \{\pem{e}_2\} $.
            \end{compactitem}
        \end{example}

        The following results formally compare eventuality spectrums with MPEMs.

        \begin{proposition}\label{prop:e-comparison}\hfill
            \begin{compactitem}
                \item Each MPEM has a unique decoration;
                \item Each eventuality spectrum has a MPEM as its picture;
                \item Two MPEMs have the same solution iff they are bisimilar.
            \end{compactitem}
        \end{proposition}

        Thus, analogously to the case of possibility spectrums, we can see that eventuality spectrums provide us with a compact representation of epistemic actions.
    
    \subsection{Union Update}
        We are now ready to present the novel formulation of update of \textsc{delphic}.  
        We say that an eventuality $\pem{e}$ is \emph{applicable} in a possibility $\poss{u}$ iff $ \poss{u} \models \pem{e}(\atom{pre}) $. Then, an eventuality spectrum $ \pem{E} $ is \emph{applicable} in a possibility spectrums $ \poss{U} $ iff for each $ \poss{u} \in \poss{U} $ there exists an applicable eventuality $ \pem{e} \in \pem{E} $. 

        \begin{definition}[Union Update]\label{def:update_pem}
            The \emph{union update} of a possibility $ \poss{u} $ with an applicable eventuality $ \pem{e} $ is the possibility $ \poss{u'} = \poss{u} \utimes \pem{e} $, where:
            
            {\centering
                $
            % \begin{equation*}
                \begin{array}{ll}
                    \poss{u'}(\atom{p}) & = 1 \text{ iff } \poss{u} \models \pem{e}(\atom{p}) \\
                    \poss{u'}(i)        & = \{\poss{v} \utimes \pem{f} \mid \poss{v} \in \poss{u}(i), \pem{f} \in \pem{e}(i) \text{ and } \poss{v} \models \pem{f}(\atom{pre})\}
                \end{array}
            % \end{equation*}
            $
            \par}
            
            \noindent The \emph{union update} of a possibility spectrum $ \poss{U} $ with an applicable eventuality spectrum $ \pem{E} $ is the possibility spectrum
            % \begin{equation*}

            {\centering
                $
            % \begin{equation*}
                \poss{U} \utimes \pem{E} = \{\poss{u} \utimes \pem{e} \mid \poss{u} \in \poss{U}, \pem{e} \in \pem{E} \text{ and } \poss{u} \models \pem{e}(\atom{pre})\}.
            % \end{equation*}%
                 $
            \par}
        \end{definition}

        \begin{example}\label{ex:union-update}
            The union update of the possibility spectrum $ \poss{W} $ of Example~\ref{ex:dec_km} with the eventuality spectrum of Example~\ref{ex:dec_em} is $\poss{W}\utimes\pem{E}=\{\poss{w_1} \utimes \pem{e_1}\}=\{\poss{v_3}\}$, where $\poss{v_3}(\atom{h})=1$, $\poss{v_3}(a)=\{\poss{v_3}\}$ and $\poss{v_3}(b)=\{\poss{w_1} \utimes \pem{e_2},\poss{w_2} \utimes \pem{e_2}\}=\{\poss{w_1},\poss{w_2}\}$.
            
            Notice that, since $\poss{w_1} \utimes \pem{e_2} {=} \poss{w_1}$ and $\poss{w_2} \utimes \pem{e_2} {=} \poss{w_2}$ the union update allows to reuse previously calculated information.
        \end{example}

    \paragraph{Comparing Union Update and Product Update.}
        Intuitively, it is easy to see that the possibility spectrum of Example \ref{ex:union-update} represents the same information of the MPKM of Example \ref{ex:product-update}. We formalize this intuition with the following lemma, witnessing the equivalence between product and union updates (full proof in the arXiv Appendix).

        \begin{lemma}\label{lem:updates_eq}
            Let $ (\E, E_d) $ be a MPEM applicable in a MPKM $ (M, W_d) $, with solutions $ \pem{E} $ and $ \poss{W} $, respectively. Then the possibility spectrum $ \poss{W'} = \poss{W} \utimes \pem{E} $ is the solution of $ (M', W'_d) = (M, W_d) \otimes (\E, E_d) $.
            
            % \begin{proof}
            %     Let $ M = (W, R, V) $, $ \E = (E, Q, pre, post) $ and $ M' = (W', R', V') $. Let $ \delta_M $ and $ \delta_{\E} $ be the decorations for $ M $ and $ \E $, respectively. Let then $ (\hat{M}, \hat{W}_d) $ be the picture of $ \poss{W'} $ via the decoration $ \delta $, where $\hat{M} = (\hat{W}, \hat{R}, \hat{V})$. By Proposition \ref{prop:p-comparison}, to prove that $ \poss{W'} $ is the solution of $ (M', W'_d) $, we need to show that $ (M', W'_d) \bisim (\hat{M}, \hat{W}_d) $. Let $ B \subseteq W' \times \hat{W} $ be a relation such that:
                
            %     {\centering
            %         $(w', \hat{w}) \in B \Leftrightarrow w' = (w, e) \wedge \delta(\hat{w}) = \delta_M(w) \utimes \delta_{\E}(e)$.
            %     \par}

            %     \noindent We now show that $ B $ is a bisimulation between $ M' $ and $ \hat{M} $. Let $ (w', \hat{w}) \in B $, with $ w' = (w, e) $ and let $ v' = (v, f) \in W' $. Let $ \poss{w} = \delta_M(w) $, $ \pem{e} = \delta_{\E}(e) $, $ \poss{v} = \delta_M(v) $ and $ \pem{f} = \delta_{\E}(f) $. Finally, let $ \poss{w'} = \poss{w} \utimes \pem{e} = \delta(\hat{w}) $ and $ \poss{v'} = \poss{v} \utimes \pem{f} $.
                % \begin{compactitem}
                %     \item (Atom) Let $ \atom{p} \in \atomSet $ be a propositional atom. Then:
                    
                %     {\centering
                %         \begin{tabular}{lll}
                %             $w' \in V'(\atom{p})$ & $\textover{\Leftrightarrow}{Def. \ref{def:update_em}}$                     & $(M, w) \models post(e)(\atom{p})$ \\
                %                                   & $\textover{\Leftrightarrow}{Pr.  \ref{prop:truth}, Def. \ref{def:dec_em}}$ & $ \poss{w} \models \pem{e}(\atom{p})$
                %                                     $\textover{\Leftrightarrow}{Def. \ref{def:update_pem}}$                      $\poss{w'}(\atom{p}) = 1 $ \\
                %                                   & $\textover{\Leftrightarrow}{Def. \ref{def:pic_km}}$                        & $\hat{w} \in \hat{V}(\atom{p})$
                %         \end{tabular}
                %     \par}

                %     \item (Forth/Back) Let $ i \in \agentSet $ be an agent. Then:
                        
                %     {\centering
                %         \begin{tabular}{lll}
                %             $w' R'_i v'$ & $\textover{\Leftrightarrow}{Def. \ref{def:update_em}}$                     & $w R_i v, e Q_i f, (M, w) \models pre(e) \text{ and }$ \\
                %                          &                                                                            & $(M, v) \models pre(f)$ \\
                %                          & $\textover{\Leftrightarrow}{Pr.  \ref{prop:truth}, Def. \ref{def:dec_em}}$ & $\poss{v} \in \poss{w}(i), \pem{f} \in \pem{e}(i), \poss{w} \models \pem{e}(\atom{pre}) \text{ and }$ \\
                %                          &                                                                            & $\poss{v} \models \pem{f}(\atom{pre})$ \\
                %                          & $\textover{\Leftrightarrow}{Def. \ref{def:update_pem}}$                    & $\poss{v'} \in \poss{w}'(i)$
                %                            $\textover{\Leftrightarrow}{Def. \ref{def:pic_km}}$                          $\hat{w} \hat{R}_i \hat{v}$
                %         \end{tabular}
                %     \par}

                %     \item (Designated) Let $ (w'_d, \hat{w}_d) \in B $, with $ w'_d = (w, e) $. Then:
                    
                %     {\centering
                %         \begin{tabular}{lll}
                %             $w'_d {\in} W'_d$ & $\textover{\Leftrightarrow}{Def. \ref{def:update_em}}$                     & $w {\in} W_d, e {\in} E_d \text{ and } (M, w) {\models} pre(e)$ \\
                %                               & $\textover{\Leftrightarrow}{Pr.  \ref{prop:truth}, Def. \ref{def:dec_em}}$ & $\poss{w} \in \poss{W}, \pem{e} \in \pem{E} \text{ and } \poss{w} \models \pem{e}(\atom{pre})$ \\
                %                               & $\textover{\Leftrightarrow}{Def. \ref{def:update_pem}}$                    & $\poss{w'} \in \poss{W'}$
                %                                 $\textover{\Leftrightarrow}{Def. \ref{def:pic_km}}$                          $\hat{w}_d \in \hat{W}_d$
                %         \end{tabular}
                %     \par}
                % \end{compactitem}
            % \end{proof}
        \end{lemma}

    \subsection{Plan Existence Problem in \textsc{delphic}}\label{sec:plan_ex_delphic}
        We conclude this section by giving the definitions of planning task and plan existence problem in \textsc{delphic}.
    
        \begin{definition}[\textsc{delphic}-Planning Task]
        \label{def:planning_task_delphic}
            A \emph{\textsc{delphic}-planning task} is a triple $ T = (\poss{W_0}, \Sigma,$ $ \varphi_g) $, where:
            \begin{inparaenum}[\itshape (i)]
            \item $ \poss{W_0} $ is an initial possibility spectrum; 
            \item $ \Sigma $ is a finite set of eventuality spectrums; 
            \item $ \varphi_g \in \Lang{C} $ is a \emph{goal formula}.
            \end{inparaenum}
        \end{definition}

        \begin{definition}\label{def:solution_delphic}
            A \emph{solution} (or \emph{plan}) to a \textsc{delphic}-planning task $(\poss{W_0}, \Sigma,$ $ \varphi_g)$ is a finite sequence $ \pem{E_1}, \dots, \pem{E}_\ell $ of actions of $\Sigma$ such that:
            \begin{compactenum}
                \item $ \poss{W_0} \utimes \pem{E_1} \utimes \dots \utimes \pem{E}_\ell \models \varphi_g $, and
                \item For each $ 1 {\leq} k {\leq} \ell $, $ \pem{E_k} $ is applicable in $ \poss{W_0} \utimes \pem{E_1} \utimes \dots \utimes \pem{E_{k-1}} $.
            \end{compactenum}
        \end{definition}

        \begin{definition}[Plan Existence Problem]\label{def:plan_ex_problem_delphic}
            Let $n {\geq} 1$ and $\mathcal{T}$ be a class of \textsc{delphic}-planning tasks. \planex{$\mathcal{T}$}{$n$} is the following decision problem: ``Given a \textsc{delphic}-planning task $ \poss{T} {\in} \mathcal{T} $, where $|\agentSet|{=}n$, does $ \poss{T} $ have a solution?''
        \end{definition}

        From Lemma \ref{lem:updates_eq}, we immediately get the following result:

        \begin{theorem}\label{th:delphic_eq}
            Let $T = (s_0, \actionSet, \varphi_g)$ be a DEL-planning task and let $\poss{T} = (\poss{W_0}, \Sigma, \varphi_g)$ be a \textsc{delphic}-planning task such that $\poss{W_0}$ is the solution of $s_0$ and $\Sigma$ is the set of solutions of $\actionSet$. Then, $\alpha_1, \dots, \alpha_\ell$ is a plan for $\planex{T}{n}$ iff $\pem{E_1}, \dots, \pem{E}_\ell$ is a plan for $\planex{\poss{T}}{n}$, where $\pem{E_i}$ is the solution of $\alpha_i$, for each $1 \leq i \leq \ell$.
        \end{theorem}

    % \input{tex/ch4_asp.tex}
    \section{Experimental Evaluation}\label{sec:eval}
    In this section, we describe our experimental evaluation of the Answer Set Programming (ASP) encodings of \textsc{delphic} and of the traditional Kripke semantics for DEL. Due to space constraints, we provide a brief overview of the encodings\footnote{The full code and documentation of the ASP encodings are available at \url{github.com/a-burigana/delphic\_asp}.} (the full presentation can be found in the arXiv Appendix).

    The aim of the evaluation is to compare the semantics of \textsc{delphic} and the traditional Kripke-based one in terms of both time and space. We do so by testing the encodings on epistemic planning benchmarks collected from the literature\footnote{Due to space limits, the description of the benchmarks is delegated to the arXiv Appendix. All benchmarks are available at \url{github.com/a-burigana/delphic\_asp}.} (\eg \emph{Collaboration and Communication}, \emph{Grapevine} and \emph{Selective Communication}). Time and space performances are respectively evaluated on the total solving time (given in seconds) and the grounding size (\ie the number of ground ASP atoms) provided by the ASP-solver \emph{clingo} output statistics. We now describe the encodings (Section \ref{sec:encodings}) and discuss the obtained results (Section \ref{sec:results}).

    \subsection{ASP Encodings}\label{sec:encodings}
        Since our goal is to achieve a fair comparison the two semantics, we implemented a baseline ASP encoding for both of them. Although optimizations for both encoding are possible, the baseline implementations are sufficient to show our claim. Towards the goal of a fair and transparent comparison, we opted for a declarative language such as ASP (notice that, as our goal is simply to compare the two baselines, the choice of an alternative declarative language would make little difference). In fact, while imperative approaches would render the comparison less clear, as one would need to delve into opaque implementation details, ASP allows to write the code that is transparent and easy to analyze. In fact, the two ASP encodings are very similar, since the representation of \textsc{delphic} objects (possibility/eventuality spectrums) and DEL objects (MPKMs/MPEMs) closely mirror each other. The only difference is in the two update operators (\ie union update and product update). This homogeneity is instrumental to obtain a fair experimental comparison of the two encodings.
        
        We now briefly describe our encodings, assuming that the reader is familiar with the basics concepts of ASP. The two encodings were developed by following the formal definitions of \textsc{delphic} and DEL objects (possibility/eventuality spectrums and MPKMs/MPEMs) and update operators (union and product update) introduced in the previous sections. To increase the efficiency of the solving and grounding phases, the two encodings make use of the \emph{multi-shot} solving approach provided by the ASP-solver \emph{clingo}, which allows for a fine-grained control over grounding and solving of ASP programs. Specifically, this approach allows one to divide an ASP encoding into sub-programs, then handling grounding and solving of these sub-programs separately. In particular, this technique is useful to implement \emph{incremental solving}, which, at each time step, allows to extend the ASP program in order to look for solutions of increasing size.
        Intuitively, every step mimics a Breadth-First Search over the planning state space: at each time step $\asp{t}$, if a solution is not found (\ie there is no plan of length $\asp{t}$ that satisfies the goal), the ASP program is expanded to look for a longer plan.
        For a detailed introduction on multi-shot ASP, we refer the reader to \cite{journals/tplp/Gebser2019,journals/tplp/Kaminski2023}.

        Finally, to visually witness the compactness that possibility spectrums provide w.r.t. MPKMs (see Remark \ref{rem:compactness}), we exploited the Python API offered by \emph{clingo} to implement a graphical representation of the epistemic states visited by the planner. This provides an immediate way of concretely compare the size of output of the two encodings on a given domain instance. Due to space reasons, we report an example of graphical comparison in the arXiv Appendix.

    \subsection{Results}\label{sec:results}
        We ran our test on a 1.4GHz Quad-Core Intel Core i5 machine with 8GB of memory and with a macOS 12.6 operating system and using \emph{clingo} version 5.6.2 with timeout (t.o.) of 10 minutes. The results are shown in Figure \ref{fig:results}. Space and time results are expressed in number of ASP atoms and in seconds, respectively.
        The comparison clearly shows that the \textsc{delphic} encoding outperforms the one based on the traditional Kripke semantics both in terms of space and time. As shown in Figure \ref{fig:results}.a, the number of ASP atoms produced by the \textsc{delphic} encoding is smaller than the ones produced by the Kripke-based ones. The ``spikes'' witnessed in the latter case are found in presence of instances with longer solutions. This indicates that \textsc{delphic} scales much better in terms of plan length than the traditional Kripke-semantics. In turn, this is positively reflected by the time results graph. In fact, observing space and time results together, we can see how the growth of the size of the epistemic states negatively affects the planning process in terms of time performances. This concretely shows that possibilities can be exploited to achieve more efficient planning tools, thus allowing epistemic planners to be able to deal with the full range of features offered by DEL.
        
        % Figure environment removed

        We now analyze the results in detail. The central factor that contributes to the performance gains of \textsc{delphic} is the fact that possibilities allow for a more efficient use of space during the computation of a solution. Specifically, this efficiency results from two key aspects. First, as shown in Remark \ref{rem:compactness}, possibility spectrums are able to represent epistemic information in a more compact way. Working with compact objects contributes significantly to reducing the size of epistemic states after sequences of updates. Second, as shown in Example \ref{ex:union-update}, possibilities naturally allow to reuse previously calculated information (\ie other possibilities that were calculated in previous states). We give a more concrete example of this property in Figure \ref{fig:delphic-reuse}, that shows a sequence of epistemic states (surrounded by rectangles) from a generalization of the Coin in the Box domain of Example \ref{ex:k_model}. We clearly see how the possibilities $\poss{w_0}$ and $\poss{w_1}$ are \emph{reused} in the epistemic states $s_1$, $s_2$, $s_3$ and $s_4$.
        %
        The space efficiency provided by \textsc{delphic} is clearly witnessed in Figure \ref{fig:results}.a. In presence of instances with longer solutions, \textsc{delphic} outperforms the Kripke-based representation, as the latter requires a considerable amount of space to compute a solution (\ie the spikes of the graph).

        % Figure environment removed

        The space efficiency of \textsc{delphic} is directly reflected on time performances. Indeed, in Figure \ref{fig:results}.b are shown the same peaks in correspondence of instances with longer solutions. As a result, we can conclude that the \textsc{delphic} framework allows for a more scalable implementation both in terms of space and time performances. Finally, we point out that the analyzed performance gains are obtained in the \emph{average case}, as there exist extreme (\emph{worst}) cases where the two semantics produce epistemic states with the same structure. In fact, we recall that the \textsc{delphic} framework is semantically equivalent to the Kripke-based one (Theorem \ref{th:delphic_eq}). Thus, we can conclude that \textsc{delphic} provides a practical and usable framework for DEL planning that can be exploited to tackle a wide range of concrete epistemic planning scenarios.
        
        We close this section by noting that a similar, but less general result, was obtained by Fabiano et al. \cite{conf/icaps/Fabiano2020}, where a possibility-based semantics is compared to the traditional Kripke-based one on a fragment of DEL called $\mal$ \cite{journals/corr/Baral2015}, that allows three kinds of actions, \ie \emph{ontic}, \emph{sensing} and \emph{announcement} actions. Since \textsc{delphic} is equivalent to the full DEL framework (see Theorem \ref{th:delphic_eq}), our comparison indeed provides a generalization of the claim made by Fabiano et al.
        % We point out that the timings of our ASP encoding of the Kripke semantics are comparable to those obtained by the solver EFP 2.0 \cite{conf/icaps/Fabiano2020}, that also implements the Kripke semantics. In particular:
        % \begin{enumerate*}
        %     \item the results for the \textbf{Gr} and \textbf{AL} domains of our encoding outperform those of EFP 2.0 (Table 1, column 9 and Table 4, column 4, respectively);
        %     \item our encoding is less performing than, but nonetheless comparable to, EFP 2.0 on the \textbf{CB} and \textbf{CC} domains.
        % \end{enumerate*}
        
    \section{Conclusions}
    We have introduced a novel epistemic planning framework, called \textsc{delphic}, based on the formal notion of possibility, in place of the more traditional Kripke-based DEL representation. We have formally shown that these two frameworks are semantically equivalent. Possibilities provide a more compact representation of epistemic states, in particular by reusing common information across states. To show the benefits of possibilities, we have implemented \textsc{delphic} and the Kripke-based approach in ASP, performing a comparative experimental evaluation with known benchmark domains. The results show that \textsc{delphic} indeed outperforms the Kripke-based approach both in terms of space and time performances, and is thus a good candidate for practical DEL planning.

    In the future, we plan to exploit the performance gains provided by the \textsc{delphic} semantics in more competitive implementations based on C++. An interesting avenue of work is to deepen our analysis of possibility-based succinctness on fragments of DEL, where only a set of specific types of actions are allowed (\eg the language $m\mathcal{A}^*$ \cite{journals/corr/Baral2015} and the framework by Kominis and Geffner \cite{conf/aips/Kominis2015}).
    
    %of representation in the case where epistemic actions are specified relying on observability groups. In this case, it is in principle possible to more precisely measure the succinctness gain that possibilities provide w.r.t. the Kripke representation.


    \subsubsection*{Acknowledgements.} This research has been partially supported by the Italian Ministry of University and Research (MUR) under PRIN project PINPOINT Prot. 2020FNEB27, and by the Free University of Bozen-Bolzano with the ADAPTERS project.

    %
    % ---- Bibliography ----
    %
    % BibTeX users should specify bibliography style 'splncs04'.
    % References will then be sorted and formatted in the correct style.
    %
    % \bibliographystyle{splncs04}
    % \bibliography{bibliography}

    \begin{thebibliography}{10}
        \providecommand{\url}[1]{\texttt{#1}}
        \providecommand{\urlprefix}{URL }
        \providecommand{\doi}[1]{https://doi.org/#1}
        
        \bibitem{books/csli/Aczel1988}
        Aczel, P.: Non-well-founded sets, {CSLI} lecture notes series, vol.~14. {CSLI}
          (1988)
        
        \bibitem{conf/ijcai/Aucher2013}
        Aucher, G., Bolander, T.: Undecidability in epistemic planning. In: Rossi, F.
          (ed.) {IJCAI} 2013, Proceedings of the 23rd International Joint Conference on
          Artificial Intelligence, Beijing, China, August 3-9, 2013. pp. 27--33.
          {IJCAI/AAAI} (2013)
        
        \bibitem{conf/tark/Batlag1998}
        Baltag, A., Moss, L.S., Solecki, S.: The logic of public announcements and
          common knowledge and private suspicions. In: Gilboa, I. (ed.) Proceedings of
          the 7th Conference on Theoretical Aspects of Rationality and Knowledge
          (TARK-98), Evanston, IL, USA, July 22-24, 1998. pp. 43--56. Morgan Kaufmann
          (1998)
        
        \bibitem{journals/corr/Baral2015}
        Baral, C., Gelfond, G., Pontelli, E., Son, T.C.: An action language for
          multi-agent domains: Foundations. CoRR  \textbf{abs/1511.01960} (2015)
        
        \bibitem{journals/jancl/Bolander2011}
        Bolander, T., Andersen, M.B.: Epistemic planning for single and multi-agent
          systems. J. Appl. Non Class. Logics  \textbf{21}(1),  9--34 (2011)
        
        \bibitem{journals/ai/Bolander2020}
        Bolander, T., Charrier, T., Pinchinat, S., Schwarzentruber, F.: Del-based
          epistemic planning: Decidability and complexity. Artif. Intell.
          \textbf{287},  103304 (2020)
        
        \bibitem{conf/kr/Bolander2021}
        Bolander, T., Dissing, L., Herrmann, N.: {DEL-based Epistemic Planning for
          Human-Robot Collaboration: Theory and Implementation}. In: {Proceedings of
          the 18th International Conference on Principles of Knowledge Representation
          and Reasoning}. pp. 120--129 (11 2021)
        
        \bibitem{journals/tplp/Burigana2020}
        Burigana, A., Fabiano, F., Dovier, A., Pontelli, E.: Modelling multi-agent
          epistemic planning in {ASP}. Theory Pract. Log. Program.  \textbf{20}(5),
          593--608 (2020)
        
        \bibitem{book/aup/vanDitmarsch2008}
        van Ditmarsch, H., Kooi, B.: Semantic results for ontic and epistemic change,
          pp. 87--117. Texts in Logic and Games 3, Amsterdam University Press (2008)
        
        \bibitem{book/springer/vanDitmarsch2007}
        van Ditmarsch, H.P., van~der Hoek, W., Kooi, B.P.: Dynamic Epistemic Logic,
          vol.~337. Springer Netherlands (2007)
        
        \bibitem{conf/icaps/Fabiano2020}
        Fabiano, F., Burigana, A., Dovier, A., Pontelli, E.: {EFP} 2.0: {A} multi-agent
          epistemic solver with multiple e-state representations. In: Beck, J.C.,
          Buffet, O., Hoffmann, J., Karpas, E., Sohrabi, S. (eds.) Proceedings of the
          Thirtieth International Conference on Automated Planning and Scheduling,
          Nancy, France, October 26-30, 2020. pp. 101--109. {AAAI} Press (2020)
        
        \bibitem{journals/tplp/Gebser2019}
        Gebser, M., Kaminski, R., Kaufmann, B., Schaub, T.: Multi-shot {ASP} solving
          with clingo. Theory Pract. Log. Program.  \textbf{19}(1),  27--82 (2019)
        
        \bibitem{journals/jolli/Gerbrandy1997}
        Gerbrandy, J., Groeneveld, W.: Reasoning about information change. J. Log.
          Lang. Inf.  \textbf{6}(2),  147--169 (1997)
        
        \bibitem{journals/tplp/Kaminski2023}
        Kaminski, R., Romero, J., Schaub, T., Wanko, P.: How to build your own
          asp-based system?! Theory Pract. Log. Program.  \textbf{23}(1),  299--361
          (2023)
        
        \bibitem{conf/aips/Kominis2015}
        Kominis, F., Geffner, H.: Beliefs in multiagent planning: From one agent to
          many. In: Brafman, R.I., Domshlak, C., Haslum, P., Zilberstein, S. (eds.)
          Proceedings of the Twenty-Fifth International Conference on Automated
          Planning and Scheduling, {ICAPS} 2015, Jerusalem, Israel, June 7-11, 2015.
          pp. 147--155. {AAAI} Press (2015)
        
        \bibitem{journals/apf/Kripke1963}
        Kripke, S.A.: Semantical considerations on modal logic. Acta Philosophica
          Fennica  \textbf{16}(1963),  83--94 (1963)
        
    \end{thebibliography}
        
    
    \newpage
    \appendix

    \section{Full Proofs}
    \begin{lemma}\label{lem:updates_eq}
        Let $ (\E, E_d) $ be an MPEM applicable in the MPKM $ (M, W_d) $, with solutions $ \pem{E} $ and $ \poss{W} $, respectively. Then the possibility spectrum $ \poss{W'} = \poss{W} \utimes \pem{E} $ is the solution of $ (M', W'_d) = (M, W_d) \otimes (\E, E_d) $.
        
        \begin{proof}
            Let $ M = (W, R, V) $, $ \E = (E, Q, pre, post) $ and $ M' = (W', R', V') $. Let $ \delta_M $ and $ \delta_{\E} $ be the decorations for $ M $ and $ \E $, respectively. Let then $ (\hat{M}, \hat{W}_d) $ be the picture of $ \poss{W'} $ via the decoration $ \delta $, where $\hat{M} = (\hat{W}, \hat{R}, \hat{V})$. By Proposition 1, to prove that $ \poss{W'} $ is the solution of $ (M', W'_d) $, we need to show that $ (M', W'_d) \bisim (\hat{M}, \hat{W}_d) $. Let $ B \subseteq W' \times \hat{W} $ be a relation such that:
            
            \begin{equation*}
                (w', \hat{w}) \in B \Leftrightarrow w' = (w, e) \wedge \delta(\hat{w}) = \delta_M(w) \utimes \delta_{\E}(e).
            \end{equation*}

            \noindent We now show that $ B $ is a bisimulation between $ M' $ and $ \hat{M} $. Let $ (w', \hat{w}) \in B $, with $ w' = (w, e) $ and let $ v' = (v, f) \in W' $. Let $ \poss{w} = \delta_M(w) $, $ \pem{e} = \delta_{\E}(e) $, $ \poss{v} = \delta_M(v) $ and $ \pem{f} = \delta_{\E}(f) $. Finally, let $ \poss{w'} = \poss{w} \utimes \pem{e} = \delta(\hat{w}) $ and $ \poss{v'} = \poss{v} \utimes \pem{f} $.
            \begin{compactitem}
                \item (Atom) Let $ \atom{p} \in \atomSet $ be a propositional atom. Then:
                
                {\centering
                    \begin{tabular}{lll}
                        $w' \in V'(\atom{p})$ & $\textover{\Leftrightarrow}{Def. 6}$                     & $(M, w) \models post(e)(\atom{p})$ \\
                                              & $\textover{\Leftrightarrow}{Pr.  2, Def. 17}$ & $ \poss{w} \models \pem{e}(\atom{p})$ \\
                                              & $\textover{\Leftrightarrow}{Def. 19}$                    & $\poss{w'}(\atom{p}) = 1 $ \\
                                              & $\textover{\Leftrightarrow}{Def. 14}$                        & $\hat{w} \in \hat{V}(\atom{p})$
                    \end{tabular}
                \par}

                \item (Forth/Back) Let $ i \in \agentSet $ be an agent. Then:
                    
                {\centering
                    \begin{tabular}{lll}
                        $w' R'_i v'$ & $\textover{\Leftrightarrow}{Def. 6}$                     & $w R_i v, e Q_i f, (M, w) \models pre(e) \text{ and }$ \\
                                     &                                                                            & $(M, v) \models pre(f)$ \\
                                     & $\textover{\Leftrightarrow}{Pr.  2, Def. 17}$ & $\poss{v} \in \poss{w}(i), \pem{f} \in \pem{e}(i), \poss{w} \models \pem{e}(\atom{pre}) \text{ and }$ \\
                                     &                                                                            & $\poss{v} \models \pem{f}(\atom{pre})$ \\
                                     & $\textover{\Leftrightarrow}{Def. 19}$                    & $\poss{v'} \in \poss{w}'(i)$ \\
                                     & $\textover{\Leftrightarrow}{Def. 14}$                        & $\hat{w} \hat{R}_i \hat{v}$
                    \end{tabular}
                \par}

                \item (Designated) Let $ (w'_d, \hat{w}_d) \in B $, with $ w'_d = (w, e) $. Then:
                
                {\centering
                    \begin{tabular}{lll}
                        $w'_d {\in} W'_d$ & $\textover{\Leftrightarrow}{Def. 6}$                     & $w {\in} W_d, e {\in} E_d \text{ and } (M, w) {\models} pre(e)$ \\
                                          & $\textover{\Leftrightarrow}{Pr.  2, Def. 17}$ & $\poss{w} \in \poss{W}, \pem{e} \in \pem{E} \text{ and } \poss{w} \models \pem{e}(\atom{pre})$ \\
                                          & $\textover{\Leftrightarrow}{Def. 19}$                    & $\poss{w'} \in \poss{W'}$ \\
                                          & $\textover{\Leftrightarrow}{Def. 14}$                        & $\hat{w}_d \in \hat{W}_d$
                    \end{tabular}
                \par}
            \end{compactitem}
        \end{proof}
    \end{lemma}

    \section{ASP Encodings}\label{sec:asp}
    In this section, we describes the Answer Set Programming (ASP) encodings of \textsc{delphic} and of the traditional Kripke semantics for DEL. We assume that the reader is familiar with the basics concepts of ASP. 
    %
    Notably, our encodings make use of the \emph{multi-shot} solving strategy provided by the ASP-solver \emph{clingo}, which provides fine-grained control over grounding and solving of ASP programs, and is instrumental to implementing an incremental strategy for solving. For a detailed introduction on multi-shot ASP solving, we refer the reader to \cite{journals/tplp/Gebser2019,journals/tplp/Kaminski2023}.
    
    We developed our ASP encodings in such a way that \textsc{delphic} objects (\eg possibility/eventuality spectrums) and DEL objects (\eg MPKMs/MPEMs) are encoded by closely mirroring each other, and only differing in the two update operators (\ie union update and product update). This homogeneity is instrumental to obtain a fair experimental comparison of the two encodings, presented in Section 4 of the main paper. At the same time, as detailed later, we stress that the timings exhibited by our ASP encoding of the Kripke semantics are comparable to those shown by the state-of-the-art solver EFP 2.0 \cite{conf/icaps/Fabiano2020}, which also implements on the Kripke semantics.
    %
    Due to the similarity of the two encodings, in the following we use generic terms such as ``epistemic state/action'' and ``planning task'' to abstract away from which underlying semantics is actually chosen.

    The \textsc{delphic} encoding presented in this section constitutes a contribution of this paper of independent interest. In fact, it generalizes the ASP-based epistemic planner \textsf{PLATO} \cite{journals/tplp/Burigana2020}, which implements a possibility-based semantics for a fragment of DEL \cite{conf/icaps/Fabiano2020}. 
    Moreover, having an ASP-based solver for epistemic planning has several benefits. 
    First, the declarative encoding of the \textsc{delphic} semantics allows for an implementation that is transparent and easier to inspect. 
    Second, we exploited the Python API offered by \emph{clingo} to implement a graphical representation that shows the epistemic states visited by the planner. We thus obtain a practical and useful tool that allows to visualize the evolution of the system. This feature is instrumental in different tasks, such as designing new epistemic planning domains and debugging the correctness of implementations\footnote{The full code and documentation of the ASP encodings are available at 
    %the following 
    %\href{https://github.com/a-burigana/delphic_asp}{\emph{link}}
    \url{https://github.com/a-burigana/delphic\_asp}.}. We concretely show a graphical comparison of the output of the two encodings in the appendix on a concrete domain instance (we do not directly report it here due to space reasons).
    
    The remainder of the section is as follows. First, we briefly describe how \emph{incremental solving} is achieved by the multi-shot solving technique (Section \ref{sec:multi-shot}). Then, we illustrate both encodings component by component: 
    \begin{inparaenum}[\itshape (i)] 
    \item formulae in Section \ref{sec:formuale-enc}, 
    \item planning tasks in Section \ref{sec:task-enc}, 
    \item epistemic states in Section \ref{sec:state-enc}, 
    \item truth conditions in Section \ref{sec:truth-enc}, and 
    \item update operators in Section \ref{sec:update-enc}).
    \end{inparaenum}

    \subsection{Multi-shot Encoding}\label{sec:multi-shot}
        The multi-shot approach allows one to divide an ASP encoding into sub-programs, then handling grounding and solving of these sub-programs separately. In particular, this technique is useful to implement \emph{incremental solving}, which, at each time step, allows to extend the ASP program in order to look for solutions of increasing size. Intuitively, every step mimics a Breadth-First Search over the planning state space: at each time step $\asp{t}$, if a solution is not found (\ie there is no plan of length $\asp{t}$ that satisfies the goal), the ASP program is expanded to look for a longer plan.

        To achieve incremental solving, we build on the approach by Gebser et al. \cite{journals/tplp/Gebser2019}, splitting our encodings in three subprograms:
        \begin{enumerate*}
            \item program $\asp{base}$, which contains all the static information (\ie input information on the planning task);
            \item program $\asp{step(t)}$, where $\asp{t}{>}0$, which describes the evolution of the system (\ie updates semantics); and
            \item program $\asp{check(t)}$, where $\asp{t}{\geq}0$, which verify the truth of formulae of the domain and, in particular, of the goal formula.
        \end{enumerate*}
        %
        Here, $\asp{t}$ represents the current time step that is being considered. In the reminder of this section, when we describe components of the encodings pertaining to the sub-programs $\asp{step(t)}$ and $\asp{check(t)}$, we assume $\asp{t}$ fixed.

    \subsection{Formulae}\label{sec:formuale-enc}
        We represent epistemic formulae through nested ASP predicates. To enhance the performance of grounding and solving, we assume that all input formulae are given in a normal form where occurrences of the $\Box_i$ operators are replaced by the dual representation $\neg\Diamond_i\neg$, and where double negation is simplified.
        %
        Agents and atoms are represented by the ASP predicates $\asp{agent(AG)}$ and $\asp{atom(P)}$, respectively. 
        A formula $\varphi$ is encoded inductively on its structure:
        \begin{enumerate*}
            \item $\varphi=\atom{p}$ is encoded by $\asp{p}$;
            \item $\varphi=\neg\psi$ is encoded by $\asp{neg(PSI)}$;
            \item $\varphi=\psi_1 \wedge \psi_2$ is encoded by $\asp{and(PSI_1, PSI_2)}$; and
            \item $\varphi=\Diamond_i\psi$ is encoded by $\asp{dia(i, PSI)}$.
        \end{enumerate*}

    \subsection{Planning Tasks}\label{sec:task-enc}    
        We now describe our ASP encoding of a planning task.

        \mypar{Initial State.}
        The initial state is given by the following ASP predicates:
        \begin{enumerate*}
            \item $\asp{w\_init(W)}$: $\asp{W}$ is an initial possibility/possible world;
            \item $\asp{r\_init(W_1, W_2, AG)}$: in $\asp{W_1}$, agent $\asp{AG}$ considers $\asp{W_2}$ to be possible;
            \item $\asp{v\_init(W, P)}$: atom $\asp{P}$ is true in $\asp{W}$;
            \item $\asp{dw\_init(W)}$: $\asp{W}$ is a designated possibility/world.
        \end{enumerate*}

        \mypar{Actions.}
        To obtain efficient encodings of the two semantics, and directly support existing benchmarks from the literature, we introduce two features in the definition of actions, namely \emph{(global) action preconditions} and \emph{observability conditions}.

        An action precondition is specified with the ASP predicate $\asp{action\_pre(ACT, PRE)}$ and represents the applicability of the action as a whole. This is syntactic sugar: action preconditions do not modify the expressiveness of epistemic actions, and one can always get an equivalent epistemic action that does not employ them.

        Observability conditions provide a useful way to compactly represent epistemic actions. Namely, agents are split into \emph{observability groups} classifying different perspectives of agents w.r.t.~an action. For instance, in Example 2, %\ref{ex:e_model}
        agent $a$ is \emph{fully observant}, since it is the one that performs the action, whereas agent $b$ is oblivious, since it does not know that the action is taking place. Then, as we show below, the information regarding which eventuality/event are considered to be possible is lifted to observability groups. Each action must then specify the observability conditions for each agent, assigning each agent to an observability group. This strategy yields an much more succinct representation since each action needs to be substantiated for each group combination, rather than for each agent combination.
        
        We are now ready to describe the ASP encoding of an action $\asp{ACT}$:
        \begin{enumerate*}
            \item $\asp{e(ACT, E)}$: $\asp{E}$ is an eventuality/event of $\asp{ACT}$;
            \item $\asp{q(ACT, E_1, E_2, GR)}$: in $\asp{E_1}$, the observability group $\asp{GR}$ considers $\asp{E_2}$ to be possible;
            \item $\asp{obs(ACT, AG, GR, COND)}$: agent $\asp{AG}$ is in the observability group $\asp{GR}$ if the condition $\asp{COND}$ is satisfied by the current epistemic state;
            \item $\asp{pre(ACT, E, PRE)}$: the precondition of $\asp{E}$ is $\asp{PRE}$;
            \item $\asp{post(ACT, E, P, POST)}$: the postcondition of $\asp{P}$ in $\asp{E}$ is $\asp{POST}$;
            \item $\asp{de(ACT, E)}$: $\asp{E}$ is a designated eventuality/event of $\asp{ACT}$.
        \end{enumerate*}

        We also define some auxiliary ASP predicates that will be used in the update encodings (Section \ref{sec:update-enc}), namely $\asp{idle(ACT, E)}$ and $\asp{inertia(ACT, E, P)}$, which are calculated at the beginning of the ASP computation. The former predicate states that $\asp{E}$ does not affect the worlds in any way (\eg $e_2$ in Example 2 is idle). %\ref{ex:e_model}
        The latter predicate states that atom $\asp{P}$ is not changed by the postconditions of $\asp{E}$.

        \mypar{Goal.}
        The goal of a planning task is represented by the ASP $\asp{goal(F)}$ predicates. It is possible to declare multiple goal formulae, so that the goal condition of the planning task is the conjunction of all these $\asp{goal}$ predicates.

    \subsection{Epistemic States}\label{sec:state-enc}
        The components of an epistemic state that must be represented (in \textsc{delphic} as well as in the traditional DEL semantics) are four:
        \begin{inparaenum}[\itshape (i)]
            \item possibilities/possible worlds;
            \item information states/accessibility relations;
            \item valuation of propositional atoms;
            \item designated possibilities/worlds.
        \end{inparaenum}
        %We now describe each component.
        

        \mypar{Possibilities and Possible Worlds.}
        To describe possibilities and possible worlds, we make use of the ASP predicate $\asp{w(t, W, E)}$. A possibility/world needs three variable to be univocally identified:
        \begin{compactitem}
            \item $\asp{t}$: represents the time instant when the possibility/world was created. In the initial state, the time is set to $\asp{0}$;
            \item $\asp{W}$: represents the possibility/world that is being updated;
            \item $\asp{E}$: represents the eventuality/event that is updating $\asp{W}$.
        \end{compactitem}

        As at planning time new possibilities/worlds are created dynamically, we are faced with the challenge of finding a suitable ASP representation that correctly and univocally encodes the worlds that are being updated during each action. This is best explained with an example. Suppose we update the possibility/world $\asp{w(t{-}1, W, E)}$ with the eventuality/event $\asp{e(ACT, F)}$ and let $\asp{w(t, X, F)}$ be the result of the update. Intuitively, we can see $\asp{X}$ as representing the world being updated (\ie $\asp{w(t{-}1, W, E)}$). Thus, the challenge we are facing here is to find a suitable representation for $\asp{X}$.

        On the one hand, when encoding possibilities, we have to bear in mind that at each time step, we might end up updating possibilities that were previously calculated at any time in the past. As a result, to correctly and univocally encode $\asp{X}$, we need to keep track of the following information:
        \begin{inparaenum}[\itshape (i)]
            \item the time when a possibility was created;
            \item the identifier of the possibility; and
            \item the eventuality that created it.
        \end{inparaenum}
        As a result, we represent $\asp{X}$ as the ASP tuple $\asp{(t{-}1, W, E)}$.

        On the other hand, when encoding possible worlds, we can always be sure that at each time step we are updating worlds that were created at precisely that time. Thus, the ASP encoding of a possible world is slightly simplified and we represent $\asp{X}$ with the ASP tuple $\asp{(W, E)}$.

        Finally, the initial possibilities/worlds are calculated from the initial state representation with the ASP rule $\asp{w(0, W, null)}$ \texttt{:-} $\asp{w\_init(W).}$, where $\asp{null}$ is a placeholder that indicates that no action occurred before time $\asp{t}$.

        \begin{example}\label{ex:asp-worlds}
            The possibilities of Example 6 are represented in ASP as follows:%\ref{ex:union-update}
            \begin{inparaenum}[\itshape (i)]
                \item $\poss{w_1}$: $\asp{w(0, w_1, null)}$;
                \item $\poss{w_2}$: $\asp{w(0, w_2, null)}$; and
                \item $\poss{v_3}$: $\asp{w(1, (0, w_1, null), e_1)}$.
            \end{inparaenum}
            Again, notice that, in the ASP encoding of possibilities, we are able to reuse previously calculated information.

            Similarly, the possible worlds of Example 3 are represented in ASP as:%\ref{ex:product-update}
            \begin{inparaenum}[\itshape (i)]
                \item $v_1$: $\asp{w(1, (w_1, null), e_2)}$;
                \item $v_2$: $\asp{w(1, (w_2, null), e_2)}$; and
                \item $v_3$: $\asp{w(1, (w_1, null), e_1)}$.
            \end{inparaenum}
        \end{example}

        \mypar{Information States and Accessibility Relations.}
        Let $\asp{w(Tw, W, Ew)}$ and $\asp{w(Tv, V, Ev)}$ be the ASP representations of two possibilities $\poss{w}$ and $\poss{v}$. Since the possibilities contained in the information states of $\poss{w}$ might have been calculated at any time step in the past, in order to encode the fact that $\poss{v} \in \poss{w}(i)$ (where $i \in \agentSet$), we need to keep track of the time when $\poss{v}$ was created. As a result, the resulting encoding is given by the ASP predicate $\asp{r(Tw, W, Ew, Tv, V, Ev, I)}$.

        Let now $\asp{w(Tw, W, Ew)}$ and $\asp{w(Tv, V, Ev)}$ refer to two possible worlds $w$ and $v$. When representing accessibility relations, we can always be sure that when $w R_i v$, the ASP representation of the worlds $w$ and $v$ share the same time value (\ie $\asp{Tw = Tv}$). Thus, we can simplify the encoding as follows: $\asp{r(Tw, W, Ew, V, Ev, I)}$.

        \begin{example}\label{ex:asp-rels}
            The information states of Example 6 %\ref{ex:union-update}
            are represented in ASP as follows (we do not further expand the information states of $\poss{w_1}$ and $\poss{w_2}$ as they refer to previously calculated information):
            \begin{compactitem}
                \item $\poss{v_3} {\in} \poss{v_3}(a)$: $\asp{r(1, (0, w_1, null), e_1, 1, (0, w_1, null), e_1, a)}$;
                \item $\poss{w_1} {\in} \poss{v_3}(b)$: $\asp{r(1, (0, w_1, null), e_1, 0, w_1, null, b)}$; and
                \item $\poss{w_2} {\in} \poss{v_3}(b)$: $\asp{r(1, (0, w_1, null), e_1, 0, w_2, null, b)}$.
            \end{compactitem}

            The accessibility relations of Example 3 %\ref{ex:product-update}
            are represented in ASP as follows:
            \begin{compactitem}
                \item $v_3 R_a v_3$: $\asp{r(1, (w_1, null), e_1, (w_1, null), e_1, a)}$;
                \item $v_3 R_b v_1$: $\asp{r(1, (w_1, null), e_1, (w_1, null), e_2, b)}$;
                \item $v_3 R_b v_2$: $\asp{r(1, (w_1, null), e_1, (w_2, null), e_2, b)}$; and
                \item $v_x R_i v_y$: $\asp{r(1, (w_x, null), e_2, (w_y, null), e_2, i)}$, for each $x,y \in \{1, 2\}$ and $i \in \{a,b\}$.
            \end{compactitem}
        \end{example}

        \mypar{Valuations.}
        For each atom $\asp{P}$, we encode the fact that $\asp{P}$ is true in the possibility/world $\asp{w(t, W, E)}$ with the ASP predicate $\asp{v(t, W, E, P)}$. We only represent \emph{true} atoms. The initial valuation is calculated from the initial state representation with the ASP rule $\asp{v(0, W, null, P)}$ \texttt{:-} $\asp{v\_init(W, P)}$.

        \mypar{Designated Possibilities and Worlds.}
        A designated possibility/world is represented by the ASP predicate $\asp{dw(t, W, E)}$, where variables $\asp{t}$, $\asp{W}$ and $\asp{E}$ have the same meaning as in $\asp{w(t, W, E)}$. The initial designated possibilities/worlds are calculated from the initial state representation with the ASP rule $\asp{dw(0, W, null)}$ \texttt{:-} $\asp{dw\_init(W)}$.

    \subsection{Truth Conditions}\label{sec:truth-enc}
 For space constraints, we abbreviate the representation $\asp{Tx, X, Ex}$ of a possibility/world as $\asp{\bar{X}}$.
        Truth conditions of formulae are encoded by predicate $\asp{holds(t, W, E, F)}$, defined by induction on the structure of formulae as follows:

        {\centering
            \begin{tabular}{l@{}l}
                $\asp{holds(\bar{W}, P)}$             & \texttt{:-~} $\asp{v(\bar{W}, P), atom(P).}$ \\
                $\asp{holds(\bar{W}, neg(F))}$        & \texttt{:-~} $\asp{not\ holds(\bar{W}, F).}$ \\
                $\asp{holds(\bar{W}, and(F_1, F_2))}$ & \texttt{:-~} $\asp{holds(\bar{W}, F_1), holds(\bar{W}, F_2).}$ \\
                $\asp{holds(\bar{W}, dia(AG, F))}$    & \texttt{:-~} $\asp{r(\bar{W}, \bar{V}, AG), holds(\bar{V}, F).}$
            \end{tabular}
        \par}

    \subsection{Update Operators}\label{sec:update-enc}
        We now describe the ASP encodings of the update operators. As the encodings differ, we present them individually. 
        
        \mypar{Union Update.}
        Let the ASP representations of a possibility spectrum $\poss{W}$ at time $\asp{t}$ and of an eventuality spectrum $\pem{ACT}$ be given. We now show how the encoding of the update $\poss{W'} = \poss{W} \utimes \pem{E}$ is obtained. We adopt again the short representation for possibilities/worlds ($\asp{\bar{X}}$). 
        We point out that the following ASP rules have a one-to-one correspondence with Definition 19. %\ref{def:update_pem}
        First, the designated possibilities of $\poss{W'}$ are determined by the following ASP rules:

        {\centering
            \begin{tabular}{@{}l@{}l@{}l}
                $\asp{dw(t, \bar{W}, E)}$ & \texttt{:-~} & $\asp{dw(\bar{W}), de(ACT, E), pre(ACT, E, PRE),}$ \\
                                          &              & $\asp{holds(\bar{W}, PRE).}$
            \end{tabular}
        \par}

        To describe how possibilities are updated, we need the following ASP predicate:

        {\centering
            \begin{tabular}{@{}l@{}l@{}l}
                $\asp{qt(t, E, F, I)}$   & \texttt{:-~} & $\asp{q(ACT, E, F, GR), obs(t, I, GR).}$
            \end{tabular}
        \par}
        
        \noindent That is, we evaluate the observability conditions of each agent to determine the information states of the action. From this, we can obtain all the updated possibilities recursively:

        {\centering
            \begin{tabular}{@{}l@{}l@{}l}
                $\asp{w(t, \bar{W}, E)}$ & \texttt{:-~} & $\asp{dw(t, \bar{W}, E), de(ACT, E).}$ \\
                $\asp{w(t, \bar{W}, E)}$ & \texttt{:-~} & $\asp{w(t, \bar{V}, F), pre(ACT, E, PRE), \textnormal{-}idle(ACT, E),}$ \\
                                         &              & $\asp{r(\bar{V}, \bar{W}, I), qt(t, F, E, I), holds(\bar{W}, PRE).}$
            \end{tabular}
        \par}

        \noindent The first rule states that a designated possibility is a possibility. Let now $\poss{v'} = \poss{v} \utimes \pem{f}$. Then, the second rule states that for any $\poss{w}$ and $\pem{e}$ such that $\poss{w} \in \poss{v}(i)$, $\pem{e} \in \pem{f}(i)$ and $\poss{w} \models \pem{e}(\atom{pre})$, we create $\poss{w'} = \poss{w} \utimes \pem{e}$. Notice that we require that the eventuality $\pem{e}$ is not \emph{idle}. An eventuality/event is idle when its precondition is $\top$ and when its postconditions are the identity function. In this way, we do not copy redundant information.

        Let now $\asp{\bar{W}'}$ and $\asp{\bar{V}'}$ stand for $\asp{t, \bar{W}, E}$ and $\asp{t, \bar{V}, F}$, respectively, and let $\asp{\bar{U}}$ stand for $\asp{Tu, \bar{U}, Eu}$. We encode information states as follows:

        {\centering
            \begin{tabular}{@{}l@{}l@{}l}
                $\asp{r(\bar{W}', \bar{V}', I)}$ & \texttt{:-~} & $\asp{r(\bar{W}, \bar{V}, I), qt(t, E, F, I).}$ \\
                $\asp{r(\bar{W}', \bar{U}, ~I)}$ & \texttt{:-~} & $\asp{r(\bar{W}, \bar{U}, I), qt(t, E, F, I), Tu{\leq}t, idle(ACT, F).}$
            \end{tabular}
        \par}

        \noindent The first rule states that if $\poss{w'} = \poss{w} \utimes \pem{e}$ and $\poss{v'} = \poss{v} \utimes \pem{f}$ are both created at time $\asp{t}$ and it holds that $\poss{v} \in \poss{w}(i)$ and $\pem{f} \in \pem{e}(i)$, then $\poss{v'} \in \poss{w'}(i)$. The second rule states that if $\poss{w'} = \poss{w} \utimes \pem{e}$ is created at time $\asp{t}$ and it holds that $\poss{u} \in \poss{w}(i)$ and there exists an eventuality $\pem{f}$ such that $\pem{f} \in \pem{e}(i)$ and $\poss{u} = \poss{u} \utimes \pem{f}$, then $\poss{u} \in \poss{w'}(i)$. In this way, we are able to reuse previously calculated information when encoding information states of possibilities.
		
        Finally, %having $\asp{\bar{W}'}$ as above, 
we encode the valuation of atoms as follows:

        {\centering
            \begin{tabular}{@{}l@{}l@{}l}
                $\asp{v(\bar{W}', P)}$ & \texttt{:-~} & $\asp{post(ACT, E, P, POST), holds(\bar{W}, POST).}$ \\
                $\asp{v(\bar{W}', P)}$ & \texttt{:-~} & $\asp{inertia(ACT, E, P), v(\bar{W}, P).}$
            \end{tabular}
        \par}

        \noindent The first rule states that if $\poss{w} \models \pem{e}(\atom{p})$, then $\poss{w'}(p) {=} 1$. The second rule states that if there are no postconditions associated to an atom (represented by the ASP predicate $\asp{inertia}$), then in $\poss{w'}$ we keep the truth value assigned to $p$ in $\poss{w}$.
        
        \mypar{Product Update.}
            Let the ASP representations of a MPKM $(M, W_d)$ at time $\asp{t}$ and of a MPEM $(\E, E_d)$ be given. The following ASP rules have a one-to-one correspondence with Definition 6. %\ref{def:update_em}
            Designated worlds and valuation of atoms are defined in the same way as in the previous case. As above, let $\asp{\bar{W}'}$ and $\asp{\bar{V}'}$ stand for $\asp{t, \bar{W}, E}$ and $\asp{t, \bar{V}, F}$, respectively. Then, updated possible worlds and accessibility relations are encoded as:% follows:

            {\centering
                \begin{tabular}{@{}l@{}l@{}l}
                    $\asp{w(t, \bar{W}, E)}$         & \texttt{:-~} & $\asp{w(\bar{W}), e(ACT, E), holds(\bar{W}, PRE).}$ \\
                    $\asp{r(\bar{W}', \bar{V}', I)}$ & \texttt{:-~} & $\asp{r(\bar{W}, \bar{V}, I), qt(t, E, F, I).}$
                \end{tabular}
            \par}

    \subsection{Epistemic Planning Domains}\label{sec:domains}
    We overview the planning domains used for the evaluation.
    
    
%\begin{compactitem}
    %\item 
    \mypar{Assemble Line (AL):} there are two agents, each responsible for processing a different part of a product. Each agent can fail in processing its part and can inform the other agent of the status of her task. The agents can decide to \emph{assemble} the product or to \emph{restart} the process, depending on their knowledge about the product status. This domain is parametrized on the maximum \emph{modal depth} $d$ of formulae that appear in the action descriptions. The goal in this domain is fixed, \ie the agents must assemble the product. The aim of this domain is to analyze the impact of the modal depth both in terms of grounding and solving performances.
    
    %\item
    \mypar{Coin in the Box (CB):} this is a generalization of the domain presented in Example 1. %\ref{ex:k_model}
    Three agents are in a room where a closed box contains a coin. None of them knows whether the coin lies heads or tails up. To look at the coin, the agents need to first open the box. Agents can be attentive or distracted: only attentive agents are able to see what actions are executed. Moreover, agents can \emph{signal} others to make them attentive and can also distract them. The goal is for one or more agents to learn the coin position.
    
    %\item 
    \mypar{Collaboration and Communication (CC):} $n {\geq} 2$ agents move along a corridor with $k {\geq} 2$ rooms in which $m {\geq} 1$ boxes are located. Whenever an agent enters a room, it can determine whether a box is there. Agents can then communicate information about the position of the boxes to each other. The goal is to have some agent learn the position of one or more boxes and to know what other agents have learnt.
    
    %\item 
    \mypar{Grapevine (Gr):} $n {\geq} 2$ agents can move along a corridor with $2$ rooms and share their own ``secret'' to agents in the same room. The goal requires agents to know the secret of one or more agents and to hide their secret to others.
    
    %\item 
    \mypar{Selective Communication (SC):} $n {\geq} 2$ agents can move along a corridor with $k {\geq} 2$ rooms. The agents might share some information, represented by an atom $q$. In each room, only a certain subset of agents is able to listen to what others share. The goal requires for some specific subset of agents to know the information and to hide it to others.
%\end{compactitem}
    
% \begin{table}[t]
%     \centering
%     \begin{table*}[t]
    \centering
    \begin{tabular}{|c|c|c|c|c|c|c|c|c|c|}                                                                                                                                                                 \hline
        \multicolumn{10}{|c|}{Assemble}                                                                                                                                                                 \\ \hline
        \multirow{2}{*}{$|\agentSet|$} & \multirow{2}{*}{$|\atomSet|$}  & \multirow{2}{*}{$|W|$}  & \multirow{2}{*}{$|\actionSet|$}  & \multirow{2}{*}{$L$}  & \multirow{2}{*}{$d$} & \multicolumn{2}{|c|}{Delphic} & \multicolumn{2}{|c|}{Kripke} \\ \cline{7-10}
                            &                     &                     &                     &                     &                    & Time          & Atoms         & Time          & Atoms        \\ \hline
        \multirow{10}{*}{2} & \multirow{10}{*}{4} & \multirow{10}{*}{4} & \multirow{10}{*}{6} & \multirow{10}{*}{5} & 2                  & 2.560         & 59332         & 3.153         & 123780       \\
                            &                     &                     &                     &                     & 3                  & 2.621         & 60390         & 3.606         & 121194       \\
                            &                     &                     &                     &                     & 4                  & 2.913         & 61422         & 4.396         & 128644       \\
                            &                     &                     &                     &                     & 5                  & 3.117         & 62478         & 4.148         & 125708       \\
                            &                     &                     &                     &                     & 6                  & 3.304         & 63564         & 4.774         & 128328       \\
                            &                     &                     &                     &                     & 7                  & 3.410         & 64622         & 4.917         & 130464       \\
                            &                     &                     &                     &                     & 8                  & 3.372         & 65680         & 5.556         & 138372       \\
                            &                     &                     &                     &                     & 9                  & 3.566         & 66738         & 6.161         & 140804       \\
                            &                     &                     &                     &                     & 10                 & 3.739         & 67796         & 6.888         & 136576       \\
                            &                     &                     &                     &                     & 24                 & 13.103        & 108000        & 25.264        & 218362       \\ \hline
    \end{tabular}
    \caption{Results for \textbf{AL}.}
    \label{tab:al}
\end{table*}

%     \vspace*{6pt}
%     \begin{table*}[t]
    \centering
    \begin{tabular}{|c|c|c|c|c|c|c|c|c|c|}                                                                                                                                                              \hline
        \multicolumn{10}{|c|}{Coin in the Box}                                                                                                                                                       \\ \hline
        \multirow{2}{*}{$|\agentSet|$} & \multirow{2}{*}{$|\atomSet|$} & \multirow{2}{*}{$|W|$} & \multirow{2}{*}{$|\actionSet|$}  & \multirow{2}{*}{$L$} & \multirow{2}{*}{$d$} & \multicolumn{2}{|c|}{\textsc{delphic}} & \multicolumn{2}{|c|}{Kripke} \\ \cline{7-10}
                            &                    &                    &                     &                    &                    & Time          & Atoms         & Time          & Atoms        \\ \hline
        \multirow{5}{*}{3}  & \multirow{5}{*}{5} & \multirow{5}{*}{2} & \multirow{5}{*}{21} & 2                  & 1                  & 0.077         & 2459          & 0.098         & 3094         \\
                            &                    &                    &                     & 3                  & 1                  & 0.215         & 5828          & 0.231         & 8394         \\
                            &                    &                    &                     & 5                  & 3                  & 5.137         & 77310         & 7.014         & 122265       \\
                            &                    &                    &                     & 6                  & 3                  & 27.428        & 316840        & 54.091        & 586037       \\
                            &                    &                    &                     & 7                  & 3                  & t.o.          & -             & t.o.          & -            \\ \hline
    \end{tabular}
    \caption{Results for \textbf{CB}.}
    \label{tab:cb}
\end{table*}

%     \caption{Results for \textbf{AL} and \textbf{CB}.}
%     \label{tab:res1}
% \end{table}

% \begin{table}[t]
%     \centering
%     \begin{table*}[t]
    \centering
    \begin{tabular}{|c|c|c|c|c|c|c|c|c|c|}                                                                                                                                                                \hline
        \multicolumn{10}{|c|}{Collaboration and Communication}                                                                                                                                     \\ \hline
        \multirow{2}{*}{$|\agentSet|$} & \multirow{2}{*}{$|\atomSet|$}  & \multirow{2}{*}{$|W|$}  & \multirow{2}{*}{$|\actionSet|$}  & \multirow{2}{*}{$L$} & \multirow{2}{*}{$d$} & \multicolumn{2}{|c|}{Delphic} & \multicolumn{2}{|c|}{Kripke} \\ \cline{7-10}
                            &                     &                     &                     &                    &                    & Time           & Atoms        & Time          & Atoms        \\ \hline
        \multirow{6}{*}{2}  & \multirow{6}{*}{10} & \multirow{6}{*}{4}  & \multirow{6}{*}{20} & 3                  & 2                  & 0.129          & 4579         & 0.186         & 8859         \\
                            &                     &                     &                     & 4                  & 1                  & 0.455          & 10900        & 0.532         & 25916        \\
                            &                     &                     &                     & 5                  & 2                  & 2.467          & 37882        & 3.599         & 98226        \\
                            &                     &                     &                     & 6                  & 2                  & 9.435          & 147183       & 25.365        & 385343       \\
                            &                     &                     &                     & 7                  & 2                  & 66.278         & 636799       & 254.544       & 1652783      \\
                            &                     &                     &                     & 8                  & 2                  & t.o.           & -            & t.o.          & -            \\ \hline
        \multirow{6}{*}{3}  & \multirow{6}{*}{13} & \multirow{6}{*}{4}  & \multirow{6}{*}{30} & 3                  & 2                  & 0.167          & 7112         & 0.328         & 13532        \\
                            &                     &                     &                     & 4                  & 1                  & 0.774          & 15873        & 0.958         & 37125        \\
                            &                     &                     &                     & 5                  & 2                  & 5.580          & 57033        & 9.321         & 147549       \\
                            &                     &                     &                     & 6                  & 2                  & 18.550         & 214086       & 78.667        & 557746       \\
                            &                     &                     &                     & 7                  & 2                  & 143.178        & 934859       & t.o.          & -            \\
                            &                     &                     &                     & 8                  & 2                  & t.o.           & -            & t.o.          & -            \\ \hline
        \multirow{6}{*}{3}  & \multirow{6}{*}{15} & \multirow{6}{*}{8}  & \multirow{6}{*}{42} & 3                  & 2                  & 0.905          & 17288        & 1.153         & 36040        \\
                            &                     &                     &                     & 4                  & 1                  & 3.939          & 46741        & 4.392         & 114453       \\
                            &                     &                     &                     & 5                  & 2                  & 14.207         & 184241       & 85.423        & 477169       \\
                            &                     &                     &                     & 6                  & 2                  & 114.014        & 760234       & 465.422       & 1963514      \\
                            &                     &                     &                     & 7                  & 2                  & t.o.           & -            & t.o.          & -            \\
                            &                     &                     &                     & 8                  & 2                  & t.o.           & -            & t.o.          & -            \\ \hline
        \multirow{5}{*}{2}  & \multirow{5}{*}{14} & \multirow{5}{*}{9}  & \multirow{5}{*}{28} & 3                  & 2                  & 0.541          & 13456        & 0.723         & 28514        \\
                            &                     &                     &                     & 4                  & 1                  & 2.778          & 39077        & 3.020         & 96399        \\
                            &                     &                     &                     & 5                  & 2                  & 12.418         & 152271       & 38.747        & 393337       \\
                            &                     &                     &                     & 6                  & 2                  & 57.073         & 650494       & 146.581       & 1681652      \\
                            &                     &                     &                     & 7                  & 2                  & t.o.           & -            & t.o.          & -            \\ \hline
        \multirow{5}{*}{2}  & \multirow{5}{*}{17} & \multirow{5}{*}{18} & \multirow{5}{*}{40} & 3                  & 2                  & 2.688          & 39310        & 3.324         & 87272        \\
                            &                     &                     &                     & 4                  & 1                  & 7.849          & 128993       & 14.633        & 322395       \\
                            &                     &                     &                     & 5                  & 2                  & 52.642         & 535713       & 115.531       & 1376539      \\
                            &                     &                     &                     & 6                  & 2                  & t.o.           & -            & t.o.          & -            \\
                            &                     &                     &                     & 7                  & 2                  & t.o.           & -            & t.o.          & -            \\ \hline
    \end{tabular}
    \caption{Results for \textbf{CC}.}
    \label{tab:cc}
\end{table*}

%     \vspace*{6pt}
%     \begin{table*}[t]
    \centering
    \begin{tabular}{|c|c|c|c|c|c|c|c|c|c|}                                                                                                                                                                \hline
        \multicolumn{10}{|c|}{Grapevine}                                                                                                                                                               \\ \hline
        \multirow{2}{*}{$|\agentSet|$} & \multirow{2}{*}{$|\atomSet|$}  & \multirow{2}{*}{$|W|$}  & \multirow{2}{*}{$|\actionSet|$}  & \multirow{2}{*}{$L$} & \multirow{2}{*}{$d$} & \multicolumn{2}{|c|}{Delphic} & \multicolumn{2}{|c|}{Kripke} \\ \cline{7-10}
                            &                     &                     &                     &                    &                    & Time          & Atoms         & Time          & Atoms        \\ \hline
        \multirow{6}{*}{3}  & \multirow{6}{*}{9}  & \multirow{6}{*}{8}  & \multirow{6}{*}{24} & 2                  & 1                  & 0.104         & 3644          & 0.165         & 9256         \\
                            &                     &                     &                     & 3                  & 1                  & 0.197         & 5385          & 0.892         & 28491        \\
                            &                     &                     &                     & 4                  & 1                  & 0.500         & 9294          & 4.142         & 98558        \\
                            &                     &                     &                     & 5                  & 1                  & 1.934         & 18691         & 44.456        & 372271       \\
                            &                     &                     &                     & 6                  & 2                  & 15.943        & 43315         & t.o.          & -            \\
                            &                     &                     &                     & 7                  & 2                  & 52.829        & 107146        & t.o.          & -            \\ \hline
        \multirow{6}{*}{4}  & \multirow{6}{*}{12} & \multirow{6}{*}{16} & \multirow{6}{*}{40} & 2                  & 1                  & 0.360         & 10684         & 0.810         & 32104        \\
                            &                     &                     &                     & 3                  & 1                  & 1.096         & 17182         & 6.613         & 109032       \\
                            &                     &                     &                     & 4                  & 1                  & 3.694         & 33024         & 41.454        & 412426       \\
                            &                     &                     &                     & 5                  & 1                  & 15.049        & 73698         & t.o.          & -            \\
                            &                     &                     &                     & 6                  & 2                  & 87.727        & 184055        & t.o.          & -            \\
                            &                     &                     &                     & 7                  & 2                  & t.o.          & -             & t.o.          & -            \\ \hline
        \multirow{6}{*}{5}  & \multirow{6}{*}{15} & \multirow{6}{*}{32} & \multirow{6}{*}{60} & 2                  & 1                  & 1.153         & 33600         & 4.593         & 113362       \\
                            &                     &                     &                     & 3                  & 1                  & 3.695         & 58527         & 48.582        & 417771       \\
                            &                     &                     &                     & 4                  & 1                  & 9.469         & 123422        & t.o.          & -            \\
                            &                     &                     &                     & 5                  & 1                  & 77.333        & 298973        & t.o.          & -            \\
                            &                     &                     &                     & 6                  & 2                  & t.o.          & -             & t.o.          & -            \\
                            &                     &                     &                     & 7                  & 2                  & t.o.          & -             & t.o.          & -            \\ \hline
    \end{tabular}
    \caption{Results for \textbf{Gr}.}
    \label{tab:gr}
\end{table*}

%     \vspace*{6pt}
%     \begin{table*}[t]
    \centering
    \begin{tabular}{|c|c|c|c|c|c|c|c|c|c|}                                                                                                                                                               \hline
        \multicolumn{10}{|c|}{Selective Communication}                                                                                                                                                \\ \hline
        \multirow{2}{*}{$|\agentSet|$} & \multirow{2}{*}{$|\atomSet|$}  & \multirow{2}{*}{$|W|$} & \multirow{2}{*}{$|\actionSet|$}  & \multirow{2}{*}{$L$} & \multirow{2}{*}{$d$} & \multicolumn{2}{|c|}{Delphic} & \multicolumn{2}{|c|}{Kripke} \\ \cline{7-10}
                            &                     &                    &                     &                    &                    & Time          & Atoms         & Time          & Atoms        \\ \hline
        \multirow{4}{*}{3}  & \multirow{4}{*}{5}  & \multirow{4}{*}{2} & \multirow{4}{*}{7}  & 3                  & 2                  & 0.026         & 943           & 0.029         & 1848         \\
                            &                     &                    &                     & 5                  & 1                  & 0.111         & 5451          & 0.354         & 18231        \\
                            &                     &                    &                     & 6                  & 3                  & 0.190         & 7775          & 1.948         & 74916        \\
                            &                     &                    &                     & 8                  & 3                  & 2.063         & 62367         & 81.041        & 1318062      \\ \hline
        \multirow{3}{*}{7}  & \multirow{3}{*}{5}  & \multirow{3}{*}{2} & \multirow{3}{*}{7}  & 5                  & 1                  & 0.188         & 11342         & 0.545         & 30615        \\
                            &                     &                    &                     & 7                  & 2                  & 1.778         & 67906         & 20.629        & 579934       \\
                            &                     &                    &                     & 8                  & 2                  & 4.192         & 140081        & 179.118       & 2617071      \\ \hline
        \multirow{4}{*}{8}  & \multirow{4}{*}{11} & \multirow{4}{*}{2} & \multirow{4}{*}{13} & 9                  & 1                  & 0.236         & 10156         & 52.347        & 976904       \\
                            &                     &                    &                     & 10                 & 2                  & 0.339         & 14519         & 178.486       & 1036341      \\
                            &                     &                    &                     & 14                 & 2                  & 17.766        & 494657        & t.o.          & -            \\
                            &                     &                    &                     & 15                 & 2                  & 16.430        & 481155        & t.o.          & -            \\ \hline
        \multirow{4}{*}{9}  & \multirow{4}{*}{11} & \multirow{4}{*}{2} & \multirow{4}{*}{13} & 6                  & 2                  & 4.542         & 167342        & 9.196         & 414842       \\
                            &                     &                    &                     & 8                  & 2                  & 27.503        & 653357        & t.o.          & -            \\
                            &                     &                    &                     & 9                  & 2                  & 94.207        & 1693676       & t.o.          & -            \\
                            &                     &                    &                     & 12                 & 2                  & t.o.          & -             & t.o.          & -            \\ \hline
        \multirow{4}{*}{9}  & \multirow{4}{*}{11} & \multirow{4}{*}{2} & \multirow{4}{*}{13} & 9                  & 2                  & 0.373         & 21893         & 29.649        & 494173       \\
                            &                     &                    &                     & 10                 & 2                  & 0.723         & 41257         & 195.693       & 1135358      \\
                            &                     &                    &                     & 13                 & 2                  & 1.989         & 95485         & t.o.          & -            \\
                            &                     &                    &                     & 17                 & 2                  & 288.827       & 4023556       & t.o.          & -            \\ \hline
        \multirow{8}{*}{9}  & \multirow{8}{*}{12} & \multirow{8}{*}{2} & \multirow{8}{*}{14} & 4                  & 1                  & 0.084         & 4712          & 0.143         & 12602        \\
                            &                     &                    &                     & 5                  & 1                  & 0.234         & 15034         & 0.848         & 49580        \\
                            &                     &                    &                     & 6                  & 1                  & 0.678         & 43058         & 3.560         & 212899       \\
                            &                     &                    &                     & 7                  & 1                  & 3.165         & 132163        & 27.126        & 1030477      \\
                            &                     &                    &                     & 8                  & 1                  & 15.983        & 393797        & t.o.          & -            \\
                            &                     &                    &                     & 9                  & 1                  & 32.373        & 782700        & t.o.          & -            \\
                            &                     &                    &                     & 10                 & 1                  & 129.964       & 2458577       & t.o.          & -            \\
                            &                     &                    &                     & 11                 & 1                  & 289.298       & 4209828       & t.o.          & -            \\ \hline
    \end{tabular}
    \caption{Results for \textbf{SC}.}
    \label{tab:sc}
\end{table*}

%     \caption{Results for \textbf{CC}, \textbf{Gr} and \textbf{SC}.}
%     \label{tab:res2}
% \end{table}

    \section{Graphical Comparison}
    We briefly report a concrete example on the succinctness of the possibility-based representation, compared to the Kripke-based one (Figures \ref{fig:delphic} and \ref{fig:kripke}). The figures were generated by our tool. As you can see, even for small plans of length 5, the \textsc{delphic} semantics provide a much more succinct representation. Specifically, in Figure \ref{fig:delphic} we obtain a total of 15 possibilities, while in Figure \ref{fig:kripke} we obtain 126 possible worlds. This clearly confirms the benefits of the \textsc{delphic} semantics.

    % Figure environment removed

    % Figure environment removed

    \section{Full Experimental Results}
    We here report the complete tables with the results of our experimental evaluations (Tables \ref{tab:al}-\ref{tab:sc}).

    \begin{table*}[t]
    \centering
    \begin{tabular}{|c|c|c|c|c|c|c|c|c|c|}                                                                                                                                                                 \hline
        \multicolumn{10}{|c|}{Assemble}                                                                                                                                                                 \\ \hline
        \multirow{2}{*}{$|\agentSet|$} & \multirow{2}{*}{$|\atomSet|$}  & \multirow{2}{*}{$|W|$}  & \multirow{2}{*}{$|\actionSet|$}  & \multirow{2}{*}{$L$}  & \multirow{2}{*}{$d$} & \multicolumn{2}{|c|}{Delphic} & \multicolumn{2}{|c|}{Kripke} \\ \cline{7-10}
                            &                     &                     &                     &                     &                    & Time          & Atoms         & Time          & Atoms        \\ \hline
        \multirow{10}{*}{2} & \multirow{10}{*}{4} & \multirow{10}{*}{4} & \multirow{10}{*}{6} & \multirow{10}{*}{5} & 2                  & 2.560         & 59332         & 3.153         & 123780       \\
                            &                     &                     &                     &                     & 3                  & 2.621         & 60390         & 3.606         & 121194       \\
                            &                     &                     &                     &                     & 4                  & 2.913         & 61422         & 4.396         & 128644       \\
                            &                     &                     &                     &                     & 5                  & 3.117         & 62478         & 4.148         & 125708       \\
                            &                     &                     &                     &                     & 6                  & 3.304         & 63564         & 4.774         & 128328       \\
                            &                     &                     &                     &                     & 7                  & 3.410         & 64622         & 4.917         & 130464       \\
                            &                     &                     &                     &                     & 8                  & 3.372         & 65680         & 5.556         & 138372       \\
                            &                     &                     &                     &                     & 9                  & 3.566         & 66738         & 6.161         & 140804       \\
                            &                     &                     &                     &                     & 10                 & 3.739         & 67796         & 6.888         & 136576       \\
                            &                     &                     &                     &                     & 24                 & 13.103        & 108000        & 25.264        & 218362       \\ \hline
    \end{tabular}
    \caption{Results for \textbf{AL}.}
    \label{tab:al}
\end{table*}

    \begin{table*}[t]
    \centering
    \begin{tabular}{|c|c|c|c|c|c|c|c|c|c|}                                                                                                                                                              \hline
        \multicolumn{10}{|c|}{Coin in the Box}                                                                                                                                                       \\ \hline
        \multirow{2}{*}{$|\agentSet|$} & \multirow{2}{*}{$|\atomSet|$} & \multirow{2}{*}{$|W|$} & \multirow{2}{*}{$|\actionSet|$}  & \multirow{2}{*}{$L$} & \multirow{2}{*}{$d$} & \multicolumn{2}{|c|}{\textsc{delphic}} & \multicolumn{2}{|c|}{Kripke} \\ \cline{7-10}
                            &                    &                    &                     &                    &                    & Time          & Atoms         & Time          & Atoms        \\ \hline
        \multirow{5}{*}{3}  & \multirow{5}{*}{5} & \multirow{5}{*}{2} & \multirow{5}{*}{21} & 2                  & 1                  & 0.077         & 2459          & 0.098         & 3094         \\
                            &                    &                    &                     & 3                  & 1                  & 0.215         & 5828          & 0.231         & 8394         \\
                            &                    &                    &                     & 5                  & 3                  & 5.137         & 77310         & 7.014         & 122265       \\
                            &                    &                    &                     & 6                  & 3                  & 27.428        & 316840        & 54.091        & 586037       \\
                            &                    &                    &                     & 7                  & 3                  & t.o.          & -             & t.o.          & -            \\ \hline
    \end{tabular}
    \caption{Results for \textbf{CB}.}
    \label{tab:cb}
\end{table*}

    \begin{table*}[t]
    \centering
    \begin{tabular}{|c|c|c|c|c|c|c|c|c|c|}                                                                                                                                                                \hline
        \multicolumn{10}{|c|}{Collaboration and Communication}                                                                                                                                     \\ \hline
        \multirow{2}{*}{$|\agentSet|$} & \multirow{2}{*}{$|\atomSet|$}  & \multirow{2}{*}{$|W|$}  & \multirow{2}{*}{$|\actionSet|$}  & \multirow{2}{*}{$L$} & \multirow{2}{*}{$d$} & \multicolumn{2}{|c|}{Delphic} & \multicolumn{2}{|c|}{Kripke} \\ \cline{7-10}
                            &                     &                     &                     &                    &                    & Time           & Atoms        & Time          & Atoms        \\ \hline
        \multirow{6}{*}{2}  & \multirow{6}{*}{10} & \multirow{6}{*}{4}  & \multirow{6}{*}{20} & 3                  & 2                  & 0.129          & 4579         & 0.186         & 8859         \\
                            &                     &                     &                     & 4                  & 1                  & 0.455          & 10900        & 0.532         & 25916        \\
                            &                     &                     &                     & 5                  & 2                  & 2.467          & 37882        & 3.599         & 98226        \\
                            &                     &                     &                     & 6                  & 2                  & 9.435          & 147183       & 25.365        & 385343       \\
                            &                     &                     &                     & 7                  & 2                  & 66.278         & 636799       & 254.544       & 1652783      \\
                            &                     &                     &                     & 8                  & 2                  & t.o.           & -            & t.o.          & -            \\ \hline
        \multirow{6}{*}{3}  & \multirow{6}{*}{13} & \multirow{6}{*}{4}  & \multirow{6}{*}{30} & 3                  & 2                  & 0.167          & 7112         & 0.328         & 13532        \\
                            &                     &                     &                     & 4                  & 1                  & 0.774          & 15873        & 0.958         & 37125        \\
                            &                     &                     &                     & 5                  & 2                  & 5.580          & 57033        & 9.321         & 147549       \\
                            &                     &                     &                     & 6                  & 2                  & 18.550         & 214086       & 78.667        & 557746       \\
                            &                     &                     &                     & 7                  & 2                  & 143.178        & 934859       & t.o.          & -            \\
                            &                     &                     &                     & 8                  & 2                  & t.o.           & -            & t.o.          & -            \\ \hline
        \multirow{6}{*}{3}  & \multirow{6}{*}{15} & \multirow{6}{*}{8}  & \multirow{6}{*}{42} & 3                  & 2                  & 0.905          & 17288        & 1.153         & 36040        \\
                            &                     &                     &                     & 4                  & 1                  & 3.939          & 46741        & 4.392         & 114453       \\
                            &                     &                     &                     & 5                  & 2                  & 14.207         & 184241       & 85.423        & 477169       \\
                            &                     &                     &                     & 6                  & 2                  & 114.014        & 760234       & 465.422       & 1963514      \\
                            &                     &                     &                     & 7                  & 2                  & t.o.           & -            & t.o.          & -            \\
                            &                     &                     &                     & 8                  & 2                  & t.o.           & -            & t.o.          & -            \\ \hline
        \multirow{5}{*}{2}  & \multirow{5}{*}{14} & \multirow{5}{*}{9}  & \multirow{5}{*}{28} & 3                  & 2                  & 0.541          & 13456        & 0.723         & 28514        \\
                            &                     &                     &                     & 4                  & 1                  & 2.778          & 39077        & 3.020         & 96399        \\
                            &                     &                     &                     & 5                  & 2                  & 12.418         & 152271       & 38.747        & 393337       \\
                            &                     &                     &                     & 6                  & 2                  & 57.073         & 650494       & 146.581       & 1681652      \\
                            &                     &                     &                     & 7                  & 2                  & t.o.           & -            & t.o.          & -            \\ \hline
        \multirow{5}{*}{2}  & \multirow{5}{*}{17} & \multirow{5}{*}{18} & \multirow{5}{*}{40} & 3                  & 2                  & 2.688          & 39310        & 3.324         & 87272        \\
                            &                     &                     &                     & 4                  & 1                  & 7.849          & 128993       & 14.633        & 322395       \\
                            &                     &                     &                     & 5                  & 2                  & 52.642         & 535713       & 115.531       & 1376539      \\
                            &                     &                     &                     & 6                  & 2                  & t.o.           & -            & t.o.          & -            \\
                            &                     &                     &                     & 7                  & 2                  & t.o.           & -            & t.o.          & -            \\ \hline
    \end{tabular}
    \caption{Results for \textbf{CC}.}
    \label{tab:cc}
\end{table*}

    \begin{table*}[t]
    \centering
    \begin{tabular}{|c|c|c|c|c|c|c|c|c|c|}                                                                                                                                                                \hline
        \multicolumn{10}{|c|}{Grapevine}                                                                                                                                                               \\ \hline
        \multirow{2}{*}{$|\agentSet|$} & \multirow{2}{*}{$|\atomSet|$}  & \multirow{2}{*}{$|W|$}  & \multirow{2}{*}{$|\actionSet|$}  & \multirow{2}{*}{$L$} & \multirow{2}{*}{$d$} & \multicolumn{2}{|c|}{Delphic} & \multicolumn{2}{|c|}{Kripke} \\ \cline{7-10}
                            &                     &                     &                     &                    &                    & Time          & Atoms         & Time          & Atoms        \\ \hline
        \multirow{6}{*}{3}  & \multirow{6}{*}{9}  & \multirow{6}{*}{8}  & \multirow{6}{*}{24} & 2                  & 1                  & 0.104         & 3644          & 0.165         & 9256         \\
                            &                     &                     &                     & 3                  & 1                  & 0.197         & 5385          & 0.892         & 28491        \\
                            &                     &                     &                     & 4                  & 1                  & 0.500         & 9294          & 4.142         & 98558        \\
                            &                     &                     &                     & 5                  & 1                  & 1.934         & 18691         & 44.456        & 372271       \\
                            &                     &                     &                     & 6                  & 2                  & 15.943        & 43315         & t.o.          & -            \\
                            &                     &                     &                     & 7                  & 2                  & 52.829        & 107146        & t.o.          & -            \\ \hline
        \multirow{6}{*}{4}  & \multirow{6}{*}{12} & \multirow{6}{*}{16} & \multirow{6}{*}{40} & 2                  & 1                  & 0.360         & 10684         & 0.810         & 32104        \\
                            &                     &                     &                     & 3                  & 1                  & 1.096         & 17182         & 6.613         & 109032       \\
                            &                     &                     &                     & 4                  & 1                  & 3.694         & 33024         & 41.454        & 412426       \\
                            &                     &                     &                     & 5                  & 1                  & 15.049        & 73698         & t.o.          & -            \\
                            &                     &                     &                     & 6                  & 2                  & 87.727        & 184055        & t.o.          & -            \\
                            &                     &                     &                     & 7                  & 2                  & t.o.          & -             & t.o.          & -            \\ \hline
        \multirow{6}{*}{5}  & \multirow{6}{*}{15} & \multirow{6}{*}{32} & \multirow{6}{*}{60} & 2                  & 1                  & 1.153         & 33600         & 4.593         & 113362       \\
                            &                     &                     &                     & 3                  & 1                  & 3.695         & 58527         & 48.582        & 417771       \\
                            &                     &                     &                     & 4                  & 1                  & 9.469         & 123422        & t.o.          & -            \\
                            &                     &                     &                     & 5                  & 1                  & 77.333        & 298973        & t.o.          & -            \\
                            &                     &                     &                     & 6                  & 2                  & t.o.          & -             & t.o.          & -            \\
                            &                     &                     &                     & 7                  & 2                  & t.o.          & -             & t.o.          & -            \\ \hline
    \end{tabular}
    \caption{Results for \textbf{Gr}.}
    \label{tab:gr}
\end{table*}

    \begin{table*}[t]
    \centering
    \begin{tabular}{|c|c|c|c|c|c|c|c|c|c|}                                                                                                                                                               \hline
        \multicolumn{10}{|c|}{Selective Communication}                                                                                                                                                \\ \hline
        \multirow{2}{*}{$|\agentSet|$} & \multirow{2}{*}{$|\atomSet|$}  & \multirow{2}{*}{$|W|$} & \multirow{2}{*}{$|\actionSet|$}  & \multirow{2}{*}{$L$} & \multirow{2}{*}{$d$} & \multicolumn{2}{|c|}{Delphic} & \multicolumn{2}{|c|}{Kripke} \\ \cline{7-10}
                            &                     &                    &                     &                    &                    & Time          & Atoms         & Time          & Atoms        \\ \hline
        \multirow{4}{*}{3}  & \multirow{4}{*}{5}  & \multirow{4}{*}{2} & \multirow{4}{*}{7}  & 3                  & 2                  & 0.026         & 943           & 0.029         & 1848         \\
                            &                     &                    &                     & 5                  & 1                  & 0.111         & 5451          & 0.354         & 18231        \\
                            &                     &                    &                     & 6                  & 3                  & 0.190         & 7775          & 1.948         & 74916        \\
                            &                     &                    &                     & 8                  & 3                  & 2.063         & 62367         & 81.041        & 1318062      \\ \hline
        \multirow{3}{*}{7}  & \multirow{3}{*}{5}  & \multirow{3}{*}{2} & \multirow{3}{*}{7}  & 5                  & 1                  & 0.188         & 11342         & 0.545         & 30615        \\
                            &                     &                    &                     & 7                  & 2                  & 1.778         & 67906         & 20.629        & 579934       \\
                            &                     &                    &                     & 8                  & 2                  & 4.192         & 140081        & 179.118       & 2617071      \\ \hline
        \multirow{4}{*}{8}  & \multirow{4}{*}{11} & \multirow{4}{*}{2} & \multirow{4}{*}{13} & 9                  & 1                  & 0.236         & 10156         & 52.347        & 976904       \\
                            &                     &                    &                     & 10                 & 2                  & 0.339         & 14519         & 178.486       & 1036341      \\
                            &                     &                    &                     & 14                 & 2                  & 17.766        & 494657        & t.o.          & -            \\
                            &                     &                    &                     & 15                 & 2                  & 16.430        & 481155        & t.o.          & -            \\ \hline
        \multirow{4}{*}{9}  & \multirow{4}{*}{11} & \multirow{4}{*}{2} & \multirow{4}{*}{13} & 6                  & 2                  & 4.542         & 167342        & 9.196         & 414842       \\
                            &                     &                    &                     & 8                  & 2                  & 27.503        & 653357        & t.o.          & -            \\
                            &                     &                    &                     & 9                  & 2                  & 94.207        & 1693676       & t.o.          & -            \\
                            &                     &                    &                     & 12                 & 2                  & t.o.          & -             & t.o.          & -            \\ \hline
        \multirow{4}{*}{9}  & \multirow{4}{*}{11} & \multirow{4}{*}{2} & \multirow{4}{*}{13} & 9                  & 2                  & 0.373         & 21893         & 29.649        & 494173       \\
                            &                     &                    &                     & 10                 & 2                  & 0.723         & 41257         & 195.693       & 1135358      \\
                            &                     &                    &                     & 13                 & 2                  & 1.989         & 95485         & t.o.          & -            \\
                            &                     &                    &                     & 17                 & 2                  & 288.827       & 4023556       & t.o.          & -            \\ \hline
        \multirow{8}{*}{9}  & \multirow{8}{*}{12} & \multirow{8}{*}{2} & \multirow{8}{*}{14} & 4                  & 1                  & 0.084         & 4712          & 0.143         & 12602        \\
                            &                     &                    &                     & 5                  & 1                  & 0.234         & 15034         & 0.848         & 49580        \\
                            &                     &                    &                     & 6                  & 1                  & 0.678         & 43058         & 3.560         & 212899       \\
                            &                     &                    &                     & 7                  & 1                  & 3.165         & 132163        & 27.126        & 1030477      \\
                            &                     &                    &                     & 8                  & 1                  & 15.983        & 393797        & t.o.          & -            \\
                            &                     &                    &                     & 9                  & 1                  & 32.373        & 782700        & t.o.          & -            \\
                            &                     &                    &                     & 10                 & 1                  & 129.964       & 2458577       & t.o.          & -            \\
                            &                     &                    &                     & 11                 & 1                  & 289.298       & 4209828       & t.o.          & -            \\ \hline
    \end{tabular}
    \caption{Results for \textbf{SC}.}
    \label{tab:sc}
\end{table*}



\end{document}
