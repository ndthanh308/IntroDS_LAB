\subsection{Epistemic Planning Domains}\label{sec:domains}
    We overview the planning domains used for the evaluation.
    
    
%\begin{compactitem}
    %\item 
    \mypar{Assemble Line (AL):} there are two agents, each responsible for processing a different part of a product. Each agent can fail in processing its part and can inform the other agent of the status of her task. The agents can decide to \emph{assemble} the product or to \emph{restart} the process, depending on their knowledge about the product status. This domain is parametrized on the maximum \emph{modal depth} $d$ of formulae that appear in the action descriptions. The goal in this domain is fixed, \ie the agents must assemble the product. The aim of this domain is to analyze the impact of the modal depth both in terms of grounding and solving performances.
    
    %\item
    \mypar{Coin in the Box (CB):} this is a generalization of the domain presented in Example 1. %\ref{ex:k_model}
    Three agents are in a room where a closed box contains a coin. None of them knows whether the coin lies heads or tails up. To look at the coin, the agents need to first open the box. Agents can be attentive or distracted: only attentive agents are able to see what actions are executed. Moreover, agents can \emph{signal} others to make them attentive and can also distract them. The goal is for one or more agents to learn the coin position.
    
    %\item 
    \mypar{Collaboration and Communication (CC):} $n {\geq} 2$ agents move along a corridor with $k {\geq} 2$ rooms in which $m {\geq} 1$ boxes are located. Whenever an agent enters a room, it can determine whether a box is there. Agents can then communicate information about the position of the boxes to each other. The goal is to have some agent learn the position of one or more boxes and to know what other agents have learnt.
    
    %\item 
    \mypar{Grapevine (Gr):} $n {\geq} 2$ agents can move along a corridor with $2$ rooms and share their own ``secret'' to agents in the same room. The goal requires agents to know the secret of one or more agents and to hide their secret to others.
    
    %\item 
    \mypar{Selective Communication (SC):} $n {\geq} 2$ agents can move along a corridor with $k {\geq} 2$ rooms. The agents might share some information, represented by an atom $q$. In each room, only a certain subset of agents is able to listen to what others share. The goal requires for some specific subset of agents to know the information and to hide it to others.
%\end{compactitem}
    
% \begin{table}[t]
%     \centering
%     \begin{table*}[t]
    \centering
    \begin{tabular}{|c|c|c|c|c|c|c|c|c|c|}                                                                                                                                                                 \hline
        \multicolumn{10}{|c|}{Assemble}                                                                                                                                                                 \\ \hline
        \multirow{2}{*}{$|\agentSet|$} & \multirow{2}{*}{$|\atomSet|$}  & \multirow{2}{*}{$|W|$}  & \multirow{2}{*}{$|\actionSet|$}  & \multirow{2}{*}{$L$}  & \multirow{2}{*}{$d$} & \multicolumn{2}{|c|}{Delphic} & \multicolumn{2}{|c|}{Kripke} \\ \cline{7-10}
                            &                     &                     &                     &                     &                    & Time          & Atoms         & Time          & Atoms        \\ \hline
        \multirow{10}{*}{2} & \multirow{10}{*}{4} & \multirow{10}{*}{4} & \multirow{10}{*}{6} & \multirow{10}{*}{5} & 2                  & 2.560         & 59332         & 3.153         & 123780       \\
                            &                     &                     &                     &                     & 3                  & 2.621         & 60390         & 3.606         & 121194       \\
                            &                     &                     &                     &                     & 4                  & 2.913         & 61422         & 4.396         & 128644       \\
                            &                     &                     &                     &                     & 5                  & 3.117         & 62478         & 4.148         & 125708       \\
                            &                     &                     &                     &                     & 6                  & 3.304         & 63564         & 4.774         & 128328       \\
                            &                     &                     &                     &                     & 7                  & 3.410         & 64622         & 4.917         & 130464       \\
                            &                     &                     &                     &                     & 8                  & 3.372         & 65680         & 5.556         & 138372       \\
                            &                     &                     &                     &                     & 9                  & 3.566         & 66738         & 6.161         & 140804       \\
                            &                     &                     &                     &                     & 10                 & 3.739         & 67796         & 6.888         & 136576       \\
                            &                     &                     &                     &                     & 24                 & 13.103        & 108000        & 25.264        & 218362       \\ \hline
    \end{tabular}
    \caption{Results for \textbf{AL}.}
    \label{tab:al}
\end{table*}

%     \vspace*{6pt}
%     \begin{table*}[t]
    \centering
    \begin{tabular}{|c|c|c|c|c|c|c|c|c|c|}                                                                                                                                                              \hline
        \multicolumn{10}{|c|}{Coin in the Box}                                                                                                                                                       \\ \hline
        \multirow{2}{*}{$|\agentSet|$} & \multirow{2}{*}{$|\atomSet|$} & \multirow{2}{*}{$|W|$} & \multirow{2}{*}{$|\actionSet|$}  & \multirow{2}{*}{$L$} & \multirow{2}{*}{$d$} & \multicolumn{2}{|c|}{\textsc{delphic}} & \multicolumn{2}{|c|}{Kripke} \\ \cline{7-10}
                            &                    &                    &                     &                    &                    & Time          & Atoms         & Time          & Atoms        \\ \hline
        \multirow{5}{*}{3}  & \multirow{5}{*}{5} & \multirow{5}{*}{2} & \multirow{5}{*}{21} & 2                  & 1                  & 0.077         & 2459          & 0.098         & 3094         \\
                            &                    &                    &                     & 3                  & 1                  & 0.215         & 5828          & 0.231         & 8394         \\
                            &                    &                    &                     & 5                  & 3                  & 5.137         & 77310         & 7.014         & 122265       \\
                            &                    &                    &                     & 6                  & 3                  & 27.428        & 316840        & 54.091        & 586037       \\
                            &                    &                    &                     & 7                  & 3                  & t.o.          & -             & t.o.          & -            \\ \hline
    \end{tabular}
    \caption{Results for \textbf{CB}.}
    \label{tab:cb}
\end{table*}

%     \caption{Results for \textbf{AL} and \textbf{CB}.}
%     \label{tab:res1}
% \end{table}

% \begin{table}[t]
%     \centering
%     \begin{table*}[t]
    \centering
    \begin{tabular}{|c|c|c|c|c|c|c|c|c|c|}                                                                                                                                                                \hline
        \multicolumn{10}{|c|}{Collaboration and Communication}                                                                                                                                     \\ \hline
        \multirow{2}{*}{$|\agentSet|$} & \multirow{2}{*}{$|\atomSet|$}  & \multirow{2}{*}{$|W|$}  & \multirow{2}{*}{$|\actionSet|$}  & \multirow{2}{*}{$L$} & \multirow{2}{*}{$d$} & \multicolumn{2}{|c|}{Delphic} & \multicolumn{2}{|c|}{Kripke} \\ \cline{7-10}
                            &                     &                     &                     &                    &                    & Time           & Atoms        & Time          & Atoms        \\ \hline
        \multirow{6}{*}{2}  & \multirow{6}{*}{10} & \multirow{6}{*}{4}  & \multirow{6}{*}{20} & 3                  & 2                  & 0.129          & 4579         & 0.186         & 8859         \\
                            &                     &                     &                     & 4                  & 1                  & 0.455          & 10900        & 0.532         & 25916        \\
                            &                     &                     &                     & 5                  & 2                  & 2.467          & 37882        & 3.599         & 98226        \\
                            &                     &                     &                     & 6                  & 2                  & 9.435          & 147183       & 25.365        & 385343       \\
                            &                     &                     &                     & 7                  & 2                  & 66.278         & 636799       & 254.544       & 1652783      \\
                            &                     &                     &                     & 8                  & 2                  & t.o.           & -            & t.o.          & -            \\ \hline
        \multirow{6}{*}{3}  & \multirow{6}{*}{13} & \multirow{6}{*}{4}  & \multirow{6}{*}{30} & 3                  & 2                  & 0.167          & 7112         & 0.328         & 13532        \\
                            &                     &                     &                     & 4                  & 1                  & 0.774          & 15873        & 0.958         & 37125        \\
                            &                     &                     &                     & 5                  & 2                  & 5.580          & 57033        & 9.321         & 147549       \\
                            &                     &                     &                     & 6                  & 2                  & 18.550         & 214086       & 78.667        & 557746       \\
                            &                     &                     &                     & 7                  & 2                  & 143.178        & 934859       & t.o.          & -            \\
                            &                     &                     &                     & 8                  & 2                  & t.o.           & -            & t.o.          & -            \\ \hline
        \multirow{6}{*}{3}  & \multirow{6}{*}{15} & \multirow{6}{*}{8}  & \multirow{6}{*}{42} & 3                  & 2                  & 0.905          & 17288        & 1.153         & 36040        \\
                            &                     &                     &                     & 4                  & 1                  & 3.939          & 46741        & 4.392         & 114453       \\
                            &                     &                     &                     & 5                  & 2                  & 14.207         & 184241       & 85.423        & 477169       \\
                            &                     &                     &                     & 6                  & 2                  & 114.014        & 760234       & 465.422       & 1963514      \\
                            &                     &                     &                     & 7                  & 2                  & t.o.           & -            & t.o.          & -            \\
                            &                     &                     &                     & 8                  & 2                  & t.o.           & -            & t.o.          & -            \\ \hline
        \multirow{5}{*}{2}  & \multirow{5}{*}{14} & \multirow{5}{*}{9}  & \multirow{5}{*}{28} & 3                  & 2                  & 0.541          & 13456        & 0.723         & 28514        \\
                            &                     &                     &                     & 4                  & 1                  & 2.778          & 39077        & 3.020         & 96399        \\
                            &                     &                     &                     & 5                  & 2                  & 12.418         & 152271       & 38.747        & 393337       \\
                            &                     &                     &                     & 6                  & 2                  & 57.073         & 650494       & 146.581       & 1681652      \\
                            &                     &                     &                     & 7                  & 2                  & t.o.           & -            & t.o.          & -            \\ \hline
        \multirow{5}{*}{2}  & \multirow{5}{*}{17} & \multirow{5}{*}{18} & \multirow{5}{*}{40} & 3                  & 2                  & 2.688          & 39310        & 3.324         & 87272        \\
                            &                     &                     &                     & 4                  & 1                  & 7.849          & 128993       & 14.633        & 322395       \\
                            &                     &                     &                     & 5                  & 2                  & 52.642         & 535713       & 115.531       & 1376539      \\
                            &                     &                     &                     & 6                  & 2                  & t.o.           & -            & t.o.          & -            \\
                            &                     &                     &                     & 7                  & 2                  & t.o.           & -            & t.o.          & -            \\ \hline
    \end{tabular}
    \caption{Results for \textbf{CC}.}
    \label{tab:cc}
\end{table*}

%     \vspace*{6pt}
%     \begin{table*}[t]
    \centering
    \begin{tabular}{|c|c|c|c|c|c|c|c|c|c|}                                                                                                                                                                \hline
        \multicolumn{10}{|c|}{Grapevine}                                                                                                                                                               \\ \hline
        \multirow{2}{*}{$|\agentSet|$} & \multirow{2}{*}{$|\atomSet|$}  & \multirow{2}{*}{$|W|$}  & \multirow{2}{*}{$|\actionSet|$}  & \multirow{2}{*}{$L$} & \multirow{2}{*}{$d$} & \multicolumn{2}{|c|}{Delphic} & \multicolumn{2}{|c|}{Kripke} \\ \cline{7-10}
                            &                     &                     &                     &                    &                    & Time          & Atoms         & Time          & Atoms        \\ \hline
        \multirow{6}{*}{3}  & \multirow{6}{*}{9}  & \multirow{6}{*}{8}  & \multirow{6}{*}{24} & 2                  & 1                  & 0.104         & 3644          & 0.165         & 9256         \\
                            &                     &                     &                     & 3                  & 1                  & 0.197         & 5385          & 0.892         & 28491        \\
                            &                     &                     &                     & 4                  & 1                  & 0.500         & 9294          & 4.142         & 98558        \\
                            &                     &                     &                     & 5                  & 1                  & 1.934         & 18691         & 44.456        & 372271       \\
                            &                     &                     &                     & 6                  & 2                  & 15.943        & 43315         & t.o.          & -            \\
                            &                     &                     &                     & 7                  & 2                  & 52.829        & 107146        & t.o.          & -            \\ \hline
        \multirow{6}{*}{4}  & \multirow{6}{*}{12} & \multirow{6}{*}{16} & \multirow{6}{*}{40} & 2                  & 1                  & 0.360         & 10684         & 0.810         & 32104        \\
                            &                     &                     &                     & 3                  & 1                  & 1.096         & 17182         & 6.613         & 109032       \\
                            &                     &                     &                     & 4                  & 1                  & 3.694         & 33024         & 41.454        & 412426       \\
                            &                     &                     &                     & 5                  & 1                  & 15.049        & 73698         & t.o.          & -            \\
                            &                     &                     &                     & 6                  & 2                  & 87.727        & 184055        & t.o.          & -            \\
                            &                     &                     &                     & 7                  & 2                  & t.o.          & -             & t.o.          & -            \\ \hline
        \multirow{6}{*}{5}  & \multirow{6}{*}{15} & \multirow{6}{*}{32} & \multirow{6}{*}{60} & 2                  & 1                  & 1.153         & 33600         & 4.593         & 113362       \\
                            &                     &                     &                     & 3                  & 1                  & 3.695         & 58527         & 48.582        & 417771       \\
                            &                     &                     &                     & 4                  & 1                  & 9.469         & 123422        & t.o.          & -            \\
                            &                     &                     &                     & 5                  & 1                  & 77.333        & 298973        & t.o.          & -            \\
                            &                     &                     &                     & 6                  & 2                  & t.o.          & -             & t.o.          & -            \\
                            &                     &                     &                     & 7                  & 2                  & t.o.          & -             & t.o.          & -            \\ \hline
    \end{tabular}
    \caption{Results for \textbf{Gr}.}
    \label{tab:gr}
\end{table*}

%     \vspace*{6pt}
%     \begin{table*}[t]
    \centering
    \begin{tabular}{|c|c|c|c|c|c|c|c|c|c|}                                                                                                                                                               \hline
        \multicolumn{10}{|c|}{Selective Communication}                                                                                                                                                \\ \hline
        \multirow{2}{*}{$|\agentSet|$} & \multirow{2}{*}{$|\atomSet|$}  & \multirow{2}{*}{$|W|$} & \multirow{2}{*}{$|\actionSet|$}  & \multirow{2}{*}{$L$} & \multirow{2}{*}{$d$} & \multicolumn{2}{|c|}{Delphic} & \multicolumn{2}{|c|}{Kripke} \\ \cline{7-10}
                            &                     &                    &                     &                    &                    & Time          & Atoms         & Time          & Atoms        \\ \hline
        \multirow{4}{*}{3}  & \multirow{4}{*}{5}  & \multirow{4}{*}{2} & \multirow{4}{*}{7}  & 3                  & 2                  & 0.026         & 943           & 0.029         & 1848         \\
                            &                     &                    &                     & 5                  & 1                  & 0.111         & 5451          & 0.354         & 18231        \\
                            &                     &                    &                     & 6                  & 3                  & 0.190         & 7775          & 1.948         & 74916        \\
                            &                     &                    &                     & 8                  & 3                  & 2.063         & 62367         & 81.041        & 1318062      \\ \hline
        \multirow{3}{*}{7}  & \multirow{3}{*}{5}  & \multirow{3}{*}{2} & \multirow{3}{*}{7}  & 5                  & 1                  & 0.188         & 11342         & 0.545         & 30615        \\
                            &                     &                    &                     & 7                  & 2                  & 1.778         & 67906         & 20.629        & 579934       \\
                            &                     &                    &                     & 8                  & 2                  & 4.192         & 140081        & 179.118       & 2617071      \\ \hline
        \multirow{4}{*}{8}  & \multirow{4}{*}{11} & \multirow{4}{*}{2} & \multirow{4}{*}{13} & 9                  & 1                  & 0.236         & 10156         & 52.347        & 976904       \\
                            &                     &                    &                     & 10                 & 2                  & 0.339         & 14519         & 178.486       & 1036341      \\
                            &                     &                    &                     & 14                 & 2                  & 17.766        & 494657        & t.o.          & -            \\
                            &                     &                    &                     & 15                 & 2                  & 16.430        & 481155        & t.o.          & -            \\ \hline
        \multirow{4}{*}{9}  & \multirow{4}{*}{11} & \multirow{4}{*}{2} & \multirow{4}{*}{13} & 6                  & 2                  & 4.542         & 167342        & 9.196         & 414842       \\
                            &                     &                    &                     & 8                  & 2                  & 27.503        & 653357        & t.o.          & -            \\
                            &                     &                    &                     & 9                  & 2                  & 94.207        & 1693676       & t.o.          & -            \\
                            &                     &                    &                     & 12                 & 2                  & t.o.          & -             & t.o.          & -            \\ \hline
        \multirow{4}{*}{9}  & \multirow{4}{*}{11} & \multirow{4}{*}{2} & \multirow{4}{*}{13} & 9                  & 2                  & 0.373         & 21893         & 29.649        & 494173       \\
                            &                     &                    &                     & 10                 & 2                  & 0.723         & 41257         & 195.693       & 1135358      \\
                            &                     &                    &                     & 13                 & 2                  & 1.989         & 95485         & t.o.          & -            \\
                            &                     &                    &                     & 17                 & 2                  & 288.827       & 4023556       & t.o.          & -            \\ \hline
        \multirow{8}{*}{9}  & \multirow{8}{*}{12} & \multirow{8}{*}{2} & \multirow{8}{*}{14} & 4                  & 1                  & 0.084         & 4712          & 0.143         & 12602        \\
                            &                     &                    &                     & 5                  & 1                  & 0.234         & 15034         & 0.848         & 49580        \\
                            &                     &                    &                     & 6                  & 1                  & 0.678         & 43058         & 3.560         & 212899       \\
                            &                     &                    &                     & 7                  & 1                  & 3.165         & 132163        & 27.126        & 1030477      \\
                            &                     &                    &                     & 8                  & 1                  & 15.983        & 393797        & t.o.          & -            \\
                            &                     &                    &                     & 9                  & 1                  & 32.373        & 782700        & t.o.          & -            \\
                            &                     &                    &                     & 10                 & 1                  & 129.964       & 2458577       & t.o.          & -            \\
                            &                     &                    &                     & 11                 & 1                  & 289.298       & 4209828       & t.o.          & -            \\ \hline
    \end{tabular}
    \caption{Results for \textbf{SC}.}
    \label{tab:sc}
\end{table*}

%     \caption{Results for \textbf{CC}, \textbf{Gr} and \textbf{SC}.}
%     \label{tab:res2}
% \end{table}
