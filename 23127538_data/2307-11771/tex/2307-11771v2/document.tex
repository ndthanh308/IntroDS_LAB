\documentclass[conference, letterpaper]{IEEEtran}

\ifCLASSINFOpdf

\else

\fi


% *** GRAPHICS RELATED PACKAGES ***
%
\ifCLASSINFOpdf
   \usepackage[pdftex]{graphicx}
\else
\fi

% *** MATH PACKAGES ***
%
\usepackage[cmex10]{amsmath}
\usepackage{color}
\usepackage{cite}
\usepackage{fancyhdr}
\usepackage[caption=false,font=footnotesize]{subfig}
\usepackage{enumitem}
\usepackage{hyperref}
\renewcommand{\thispagestyle}[2]{} 


\fancypagestyle{plain}{
        \fancyhead{}
        \fancyhead[C]{first page center header}
        \fancyfoot{}
        \fancyfoot[C]{first page center footer}
}
\pagestyle{fancy}


\headheight 20pt
\footskip 20pt

\rhead{}

%Enter the first page number of your paper below
%\setcounter{page}{1}

%Header
%\fancyhead[R]{\textit{(IJACSA) International Journal of Advanced Computer Science and Applications, \\ Vol. XXX, No. XXX, 2014}}
\renewcommand{\headrulewidth}{0pt}

%Footer
%\fancyfoot[C]{www.ijacsa.thesai.org}
\renewcommand{\footrulewidth}{0.5pt}
\fancyfoot[R]{\thepage \  $|$ P a g e }


\begin{document}

%
% paper title
% can use linebreaks \\ within to get better formatting as desired
\title{AN INTEGRATED NPL APPROACH TO SENTIMENT ANALYSIS IN SATISFACTION SURVEYS}


% author names and affiliations
% use a multiple column layout for up to three different
% affiliations
\author{\IEEEauthorblockN{\textbf{Pinto Luque Edson Bladimir}}
\IEEEauthorblockA{Faculty of Statistic and Computer Engineering\\Universidad Nacional del Altiplano de Puno, P.O. Box 291\\
Puno - Perú \\
Email: epintol@est.unap.edu.pe}}

% conference papers do not typically use \thanks and this command
% is locked out in conference mode. If really needed, such as for
% the acknowledgment of grants, issue a \IEEEoverridecommandlockouts
% after \documentclass

% for over three affiliations, or if they all won't fit within the width
% of the page, use this alternative format:
% 
%\author{\IEEEauthorblockN{Michael Shell\IEEEauthorrefmark{1},
%Homer Simpson\IEEEauthorrefmark{2},
%James Kirk\IEEEauthorrefmark{3}, 
%Montgomery Scott\IEEEauthorrefmark{3} and
%Eldon Tyrell\IEEEauthorrefmark{4}}
%\IEEEauthorblockA{\IEEEauthorrefmark{1}School of Electrical and Computer Engineering\\
%Georgia Institute of Technology,
%Atlanta, Georgia 30332--0250\\ Email: see http://www.michaelshell.org/contact.html}
%\IEEEauthorblockA{\IEEEauthorrefmark{2}Twentieth Century Fox, Springfield, USA\\
%Email: homer@thesimpsons.com}
%\IEEEauthorblockA{\IEEEauthorrefmark{3}Starfleet Academy, San Francisco, California 96678-2391\\
%Telephone: (800) 555--1212, Fax: (888) 555--1212}
%\IEEEauthorblockA{\IEEEauthorrefmark{4}Tyrell Inc., 123 Replicant Street, Los Angeles, California 90210--4321}}




% use for special paper notices
%\IEEEspecialpapernotice{(Invited Paper)}




% make the title area
\maketitle


\begin{abstract}
%\boldmath
The research project aims to apply an integrated approach to natural language processing (NLP) to satisfaction surveys. It will focus on understanding and extracting relevant information from survey responses, analyzing feelings, and identifying recurring word patterns. NLP techniques will be used to determine emotional polarity, classify responses into positive, negative, or neutral categories, and use opinion mining to highlight participants' opinions. This approach will help identify the most relevant aspects for participants and understand their opinions in relation to those specific aspects.

A key component of the research project will be the analysis of word patterns in satisfaction survey responses using NPL. This analysis will provide a deeper understanding of feelings, opinions, and themes and trends present in respondents' responses. The results obtained from this approach can be used to identify areas for improvement, understand respondents' preferences, and make strategic decisions based on analysis to improve respondent satisfaction.
\end{abstract}
% IEEEtran.cls defaults to using nonbold math in the Abstract.
% This preserves the distinction between vectors and scalars. However,
% if the conference you are submitting to favors bold math in the abstract,
% then you can use LaTeX's standard command \boldmath at the very start
% of the abstract to achieve this. Many IEEE journals/conferences frown on
% math in the abstract anyway.

% no keywords


\begin{IEEEkeywords}
Sentimental Analysis; BERT; NPL
\end{IEEEkeywords}


% For peer review papers, you can put extra information on the cover
% page as needed:
% \ifCLASSOPTIONpeerreview
% \begin{center} \bfseries EDICS Category: 3-BBND \end{center}
% \fi
%
% For peerreview papers, this IEEEtran command inserts a page break and
% creates the second title. It will be ignored for other modes.
\IEEEpeerreviewmaketitle

\section{INTRODUCTION}
Our research focuses on analyzing satisfaction survey responses for success and sustainability, as understanding customer sentiment and opinions is crucial to identifying areas for improvement and making informed decisions.


Opinion mining, commonly known as sentiment analysis and emotion recognition\cite{Gandhi2023} \cite{Xu2021} is an increasingly popular NLP technique in this developing area of study is to enable intelligent systems to observe, infer and understand human emotions. \cite{Denecke2023} Computer science, psychology, social sciences and cognitive sciences are included in the interdisciplinary discipline.\cite{Gandhi2023}. These techniques have proven effective in fields such as product evaluation, social network analysis\cite{Aldinata2023} \cite{Carlos2018}, and business decision making.\cite{Liu2012} This technique is used to collect and examine popular sentiment and ideas.\cite{Gandhi2023} therefore by applying NLP techniques in satisfaction surveys, organizations can gain deeper insights into respondents' preferences, needs and concerns, giving them a competitive advantage and enabling informed decisions, learning, communication and situational awareness in human-centric environments..\cite{Bhattacharjee2022} \cite{Shaik2023} \cite{Gandhi2023} This valuable insight helps organizations adapt their products and services, improve their strategies, and adapt to different situations.


Sentiment analysis is a system that automatically detects the opinions expressed in the comments \cite{Benchimol2022}.Sentiment analysis focused primarily on sentiment at the text and sentence level. It is challenging to discover numerous sentiment features present in the text, as these sentiment analyzes often only consider the signal element of the sentiment. However, human emotions are complicated.\cite{Denecke2023}


Sentiment analysis research can be categorized into the following three subfields based on granularity: aspect-level sentiment analysis, sentence-level sentiment analysis, and document-level sentiment analysis.\cite{Jiang2023}
Determining the polarity of the sentiment of a sentence specifically in the suggestions and opinions that the user provides when answering a satisfaction survey is what we intend to do in this article. Polarity typically includes positive, negative, and neutral.\cite{Chen2022} All of this is done to enhance the first responder experience and provide an accurate and comprehensive view of perceptions and requirements, supporting informed decision making.\cite{Umair2022}


\section{METHODOLOGY}
Natural language processing (NLP) is a branch of linguistics and artificial intelligence that focuses on how human and machine language interact.\cite{Rakshitha2021} focuses on methods for managing or analyzing massive amounts of data that produce recommendations for the workshop course. as a result, the computer being able to "understand" the document's concept.\cite{Denecke2023}The goal of this research is to more precisely examine and categorize the user attitudes. The BERT model is employed in this essay to categorize emotions.\cite{Selvakumar2022}The data included in the papers may then be specifically abstracted using this technique, followed by polarity-based categorization. Technology-assisted polarity allows the review or essay to be divided into good, negative, or neutral categories.\cite{Shaik2023} \cite{Peng2022}The emotion of the text, the topic, the entities, and the category of the phrase or sentence are all examined using natural language processing and other methods. Speech recognition, natural language production, and natural language understanding are difficulties that NLP must overcome.\cite{Hu2004}


The procedures for categorizing sensations are as follows:
\begin{itemize}
    \item data gathering
    \item data preparation
    \item Using BERT to classify emotions
\end{itemize}


% Figure environment removed

\subsection{DATA COLLECTION}
\subsubsection{Data for training}
In this article, we will focus on the academic semester 2023-I. This semester's data set contains 87839 user suggestions regarding the teacher's teaching in the form of text, and these will be used to model the feelings classifier with the aid of BERT. The data set for the training comes from the Teacher Evaluations provided by the National University of the Altiplano each academic semester.

The reviews in both data sets are gathered in.csv file format and are provided as text documents.
\begin{table}[ht]
    \begin{tabular}{ p{1.2cm} p{7.2cm} }
         \hline
         Academic Department & Student response to teaching Performance\\
         \hline
         DERECHO& 	Tiene buena metodología de enseñanza\\
        TURISMO& 	Debe organizar su tiempo y planificar sus clases\\
        INGENIERIA CIVIL& 	Buen docente aunque considero que tiene mucho más conocimiento por impartir\\
        DERECHO& 	El docente es especialista en su área.\\
        INGENIERIA DE SISTEMAS	& El curso se ha desarrollado de manera VIRTUAL, sin embargo, considero que para un mayor aprendizaje y a la par de que haya una igualdad se debería realizar de manera PRESENCIAL.\\
        INGENIERIA QUIMICA	& El docente desarrolla las sesiones activamente y es especialista en su área.\\
         INGENIERIA CIVIL& Se desarrollan las clases de manera activa y el docente es especialista en su área.\\
        DERECHO	& Se desarrollan las clases de manera activa.\\
        INGENIRIA DE MINAS	& El docente es especialista en su área y desarrolla las clases de una manera amigable para el estudiante y a su vez profundizando los temas.\\
        \hline
    \end{tabular}
    \label{Tabla 1}
\end{table}

\subsubsection{Data for training}
The objective in this study is to be able to carry out a statistical analysis based on the feelings of the survey according to its polarity, the data in this study belong to the suggestions of the satisfaction survey carried out on teachers of a workshop course taught by the National University of the Altiplano.
\begin{table}[ht]
    \centering
    \begin{tabular}{ c l }
        \hline
         Level of satisfaction & Suggestions\\
         \hline
        Very satisfied & That the recorded sessions are permanently available to teachers\\
        Satisfied & Share the material with us\\
        little satisfied & Brevity and more interaction\\
        Very satisfied & Congratulations\\
        Satisfied & Facilitate PPTs\\
        Satisfied & Only punctuality has to be respected.\\
        Very satisfied & More courses because they are very interesting\\
        Satisfied & Specify the processes and responsible\\
        Satisfied & Deliver previous material\\
        Satisfied & Plan your activities for the whole year\\
        \hline
    \end{tabular}
    \label{Tabla 2}
\end{table}
The table shows 10 data, in general there is a total of 506 records of teachers

\subsection{DATA PROCESSING}

Records from the data set are given as input. The data is preprocessed using the following procedures for deep learning-based sentiment classification (BERT), in contrast to aspect-based sentiment classification.
\begin{itemize}
    \item \textit{Tokenization:} Creates a blob from reviews before turning it into a string of words.
    \item \textit{Numerization:} Each token in the corpus vocabulary must be assigned a distinct integer.
    \item \textit{Padding:} If a sentence is more than the prescribed length, omit the words; if it is shorter, add zeros.
    \item \textit{Embeddings:} To be used as model parameters, the words in the sentences are mapped to a 'n' dimensional vector.
\end{itemize}

\subsection{USING BERT TO CLASSIFY EMOTIONS}

BERT is utilized in this study to rate reviews according to sentiment. BERT receives the preprocessed data as input for sentiment categorization.\cite{Sun2017} The BERT model learns the correct connections between the complete string of words in a review using multiple self-attention based encoders with hidden layers. The bi-directional characteristic of BERT is employed to determine the viewpoint of a statement that is environment-focused.\cite{Shahade2023}

\subsection{TRANSFORMATION ENCODER}

The BERT transformer architecture is utilized in this work to do sentiment analysis on Spanish-language texts. Greater classification accuracy of feelings is achieved by fitting the pre-trained model to a Spanish corpus and supervised training using annotated data sets.\cite{Khaleghparast2023} \cite{Nur2023} The section provides information on how BERT was modified for this purpose, including the changes made to the output layer and the preprocessing methods that were used. This strategy marks a substantial advancement in Spanish natural language processing, offering up new opportunities for useful applications like opinion analysis and comprehending human language in this situation. The findings show promise in fields like emotion mining and the examination of opinions in Spanish-language texts.The pretrained bert model that was used is dccuchile/bert-base-spanish-wwm-uncased, in order to perform the sentiment analysis in Spanish \cite{CaneteCFP2020}

% Figure environment removed

\section{RESULTS}

The data set for this study's evaluation of teacher performance at the National University of the Altiplano includes 25,000 user comments for various teachers in various study programs, and 506 teacher suggestions from a workshop course were used to verify the model.

The training and test were conducted using the following ratios: 70:30, 80:20, and 90:10, in 10 ephocs having the following accuracys 93.5\% , 99.6\% and 94.3\% respectively. with the 80:20 ratio producing the best results of those. Initially, testing were conducted on the teacher evaluation data set itself. 

In addition, the proposed BERT model was compared to other models, and it was shown to perform better in classifying the presented data set's emotional content. When categorizing emotions from the teacher assessment data set, our system achieves an accuracy of about 99.6 \%.

Then, in order to better grasp the sentiments surrounding this course and to be able to improve in the upcoming events, this trained Bert model was used to categorize feelings in the workshop course that was made available to university teachers.

\subsection{Data Prediction}

Once we have the model already trained with our teacher evaluation data, we proceed to apply this model to our data from the workshop course provided to teachers, to later carry out a statistical analysis of the feelings (negative and positive) regarding this workshop course, likewise we show a cloud of words, in order to better understand and analyze the developed course.

% Figure environment removed

As can be seen, when carrying out the analysis of feelings for the workshop course offered to teachers, it is found that 79\% have a negative feeling regarding the developed course and we can see this in the word cloud Fig.\ref{fig:world}, texts how to improve, requesting slides and none that is being characterized as a negative feeling. and only 21 \% have a positive feeling regarding this course.

% Figure environment removed

% An example of a floating figure using the graphicx package.
% Note that \label must occur AFTER (or within) \caption.
% For figures, \caption should occur after the \includegraphics.
% Note that IEEEtran v1.7 and later has special internal code that
% is designed to preserve the operation of \label within \caption
% even when the captionsoff option is in effect. However, because
% of issues like this, it may be the safest practice to put all your
% \label just after \caption rather than within \caption{}.
%
% Reminder: the "draftcls" or "draftclsnofoot", not "draft", class
% option should be used if it is desired that the figures are to be
% displayed while in draft mode.
%
%% Figure environment removed

% Note that IEEE typically puts floats only at the top, even when this
% results in a large percentage of a column being occupied by floats.


% An example of a double column floating figure using two subfigures.
% (The subfig.sty package must be loaded for this to work.)
% The subfigure \label commands are set within each subfloat command, the
% \label for the overall figure must come after \caption.
% \hfil must be used as a separator to get equal spacing.
% The subfigure.sty package works much the same way, except \subfigure is
% used instead of \subfloat.
%
%% Figure environment removed
%
% Note that often IEEE papers with subfigures do not employ subfigure
% captions (using the optional argument to \subfloat), but instead will
% reference/describe all of them (a), (b), etc., within the main caption.


% An example of a floating table. Note that, for IEEE style tables, the 
% \caption command should come BEFORE the table. Table text will default to
% \footnotesize as IEEE normally uses this smaller font for tables.
% The \label must come after \caption as always.
%
%\begin{table}[!t]
%% increase table row spacing, adjust to taste
%\renewcommand{\arraystretch}{1.3}
% if using array.sty, it might be a good idea to tweak the value of
% \extrarowheight as needed to properly center the text within the cells
%\caption{An Example of a Table}
%\label{table_example}
%\centering
%% Some packages, such as MDW tools, offer better commands for making tables
%% than the plain LaTeX2e tabular which is used here.
%\begin{tabular}{|c||c|}
%\hline
%One & Two\\
%\hline
%Three & Four\\
%\hline
%\end{tabular}
%\end{table}


% Note that IEEE does not put floats in the very first column - or typically
% anywhere on the first page for that matter. Also, in-text middle ("here")
% positioning is not used. Most IEEE journals/conferences use top floats
% exclusively. Note that, LaTeX2e, unlike IEEE journals/conferences, places
% footnotes above bottom floats. This can be corrected via the \fnbelowfloat
% command of the stfloats package.



% conference papers do not normally have an appendix


% use section* for acknowledgement
\section{DISCUSSION AND CONCLUSIONS}

Additionally, the effectiveness of BERT is evaluated in Fig. 2 in comparison to recent work on the same data set from \cite{Thet2010}, \cite{Kumar2019}, \cite{Maulana2020}, and \cite{Shaukat2020} for sentiment categorization. It demonstrates that BERT performs better than current literature-based efforts.

% Figure environment removed

\begin{enumerate}[label=(\Alph*)]
    \item Sentiment Analysis With BERT
    \item Lexicon with Multilayer Perceptron
    \item SVM with Information Gain
    \item Lexicon with Multilayer Perceptron
    \item Aspect Based Sentiment Analysis
\end{enumerate}

The suggested BERT-based sentimental classification outperforms existing ML and DL models and is able to accurately identify emotion polarity.\cite{Razali2021} Additionally taken into account when determining the emotion of reviews are word negations and intensifications. By experimenting with alternative algorithms for emotive analysis of user evaluations, this work might be further enhanced in the future. It is possible to shorten the additional processing time for a single review and improve the performance of the suggested design.


% trigger a \newpage just before the given reference
% number - used to balance the columns on the last page
% adjust value as needed - may need to be readjusted if
% the document is modified later
%\IEEEtriggeratref{8}
% The "triggered" command can be changed if desired:
%\IEEEtriggercmd{\enlargethispage{-5in}}

% references section

% can use a bibliography generated by BibTeX as a .bbl file
% BibTeX documentation can be easily obtained at:
% http://www.ctan.org/tex-archive/biblio/bibtex/contrib/doc/
% The IEEEtran BibTeX style support page is at:
% http://www.michaelshell.org/tex/ieeetran/bibtex/
\newpage
\bibliographystyle{IEEEtran}
% argument is your BibTeX string definitions and bibliography database(s)
%\bibliography{IEEEabrv,../bib/paper}
%
% <OR> manually copy in the resultant .bbl file
% set second argument of \begin to the number of references
% (used to reserve space for the reference number labels box)
%\bibliographystyle{ieeetr}
\bibliography{document}

% that's all folks
\end{document}
