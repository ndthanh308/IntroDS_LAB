\documentclass{article}

\usepackage{arxiv}

\usepackage[utf8]{inputenc} % allow utf-8 input
\usepackage[T1]{fontenc}    % use 8-bit T1 fonts
\usepackage{hyperref}       % hyperlinks
\usepackage{url}            % simple URL typesetting
\usepackage{booktabs}       % professional-quality tables
\usepackage{amsfonts}       % blackboard math symbols
\usepackage{nicefrac}       % compact symbols for 1/2, etc.
\usepackage{microtype}      % microtypography
\usepackage{cleveref}       % smart cross-referencing
\usepackage{lipsum}         % Can be removed after putting your text content
\usepackage{graphicx}
\usepackage{natbib}
\usepackage{doi}

\title{eXplainable Artificial Intelligence (XAI) in age prediction: A systematic review}

% Here you can change the date presented in the paper title
%\date{September 9, 1985}
% Or remove it
%\date{}

\author{ \href{https://orcid.org/0000-0001-9277-502X}{% Figure removed\hspace{1mm}Alena Kalyakulina}\\
	\texttt{kalyakulina.alena@gmail.com} \\
	%% examples of more authors
	\And
	\href{https://orcid.org/0000-0002-0540-9281}{% Figure removed\hspace{1mm}Igor Yusipov} \\
	\texttt{yusipov.igor@gmail.com} \\
	%% \AND
	%% Coauthor \\
	%% Affiliation \\
	%% Address \\
	%% \texttt{email} \\
	%% \And
	%% Coauthor \\
	%% Affiliation \\
	%% Address \\
	%% \texttt{email} \\
	%% \And
	%% Coauthor \\
	%% Affiliation \\
	%% Address \\
	%% \texttt{email} \\
}

% Uncomment to override  the `A preprint' in the header
%\renewcommand{\headeright}{Technical Report}
%\renewcommand{\undertitle}{Technical Report}
\renewcommand{\shorttitle}{XAI in age prediction}

%%% Add PDF metadata to help others organize their library
%%% Once the PDF is generated, you can check the metadata with
%%% $ pdfinfo template.pdf
\hypersetup{
pdftitle={XAI in age prediction},
pdfsubject={cs.AI},
pdfauthor={Alena Kalyakulina, Igor Yusipov},
pdfkeywords={explainable artificial intelligence, aging, longevity, age-related diseases, machine learning},
}

\begin{document}
\maketitle

\begin{abstract}
	eXplainable Artificial Intelligence (XAI) is now an important and essential part of machine learning, allowing to explain the predictions of complex models. XAI is especially required in risky applications, particularly in health care, where human lives depend on the decisions of AI systems. One area of medical research is age prediction and identification of biomarkers of aging and age-related diseases. However, the role of XAI in the age prediction task has not previously been explored directly. 
	
	In this review, we discuss the application of XAI approaches to age prediction tasks. We give a systematic review of the works organized by body systems, and discuss the benefits of XAI in medical applications and, in particular, in the age prediction domain. 
\end{abstract}


% keywords can be removed
\keywords{explainable artificial intelligence \and aging \and longevity \and age-related diseases \and machine learning}


\section{Introduction}\label{sec:introduction}
Machine learning (ML), and deep learning (DL) in particular, is currently one of the most common approaches used in many tasks in different research areas. Deep models operate with a large amount of input data, training many layers, but in most cases, their work process is not transparent. That is why they are also called black boxes \citep{Saleem2022}. Decision-making process in such deep architectures is difficult to explain, therefore questions about the trustworthiness of such models and the security of their deployment arise. The problem of explainability of artificial intelligence (AI) models is being actively studied \citep{Baehrens2010, Lipton2018, Samek2017, Simonyan2014}, and eXplainable Artificial Intelligence (XAI) has become an important area of AI \citep{Nauta2023}. Major goals of XAI are to develop approaches capable of describing the reasons for model decision-making, and, more profoundly, to develop interpretable and logically explainable models. XAI explanations must be understandable, reliable, and yet the models to which they are applied must retain predictive accuracy \citep{Saleem2022}. There are various classifications of the explainability types, and one of the most used is the division into global and local explainability. Global interpretation allows one to ‘open’ the black box of AI models by explaining the predictions of the model as a whole. Local interpretation reveals the reasons for the model's decision-making for each particular sample. These two types of explainability represent two sides of the same coin: global explainability allows us to establish general patterns, while local explainability allows us to track these patterns at the individual level. The lack of explainability significantly limits the use and deployment of models, especially in applications involving risk, where human life and health may depend on the decision. One such area is healthcare.

An example of AI application in medical practice is clinical decision support systems that assist clinicians in diagnosing diseases and making treatment decisions \citep{Amann2020}. One of the most important requirements for such systems is clinical validation and the ability to verify model decisions. Medical data are subject to random errors (due to recording and processing errors, noise, and others), so AI systems will inevitably make errors as well. Another source of errors may be systemic problems caused, in particular, by limited train subset. In this case, AI models may make errors because an individual sample may significantly differ from the population on which the model was trained \citep{Amann2020}. In both cases, explainability is extremely important; it allows one to track different types of errors, identify their causes, and adjust the behavior of the system. Explainability can also offer a personalized approach based on individual patient characteristics and risk factors. XAI approaches can offer an explanation in natural language or visualize how different factors affected the final outcome (risk score, diagnosis, or proposed treatment), which can increase patient awareness with proper use \citep{Politi2013, Stacey2017}, and allow clinicians to make confident clinical decisions, adapting predictions and recommendations to individual circumstances if necessary \citep{Beil2019}.

One of the many areas of medical studies is the search for biomarkers of aging, which has become particularly active with the growth of ML and DL research. A separate task here is the prediction of biological age, which characterizes human health in various aspects and can differ from chronological age. Age is a universal attribute of all living organisms and a biologically meaningful characteristic associated with risk of mortality, disease, and general well-being. Even though each individual feature may not be explicitly related to age, a combination of features may have predictive power \citep{Zhavoronkov2019}. Therefore, various types of data may be used to predict age, such as laboratory tests, magnetic resonance (MRI) and X-ray images, electrocardiogram (ECG) and electroencephalogram (EEG) signals, and many other inputs. Development and analysis of age predictors can help, in particular, in the study of age-related diseases \citep{Zhavoronkov2011}, immunological aging, response to medications and vaccines \citep{Zhavoronkov2019} and in many other healthcare applications. Recent studies have been encouraging about the possibility of extending life expectancy \citep{Partridge2018}, but biological age and the aging rate need to be assessed before various life extension techniques can be applied \citep{Rutledge2022}. Age prediction models are usually called aging clocks. One of the first and fundamental works in this field is the epigenetic clock from Horvath, which estimates age using a linear model based on DNA methylation data \citep{Horvath2013}. Since then, many models have been proposed using different input data and modern techniques \citep{Mamoshina2019}. However, there are still many challenges in overcoming which XAI can be helpful. In particular, it is not always possible to separate the chronological aging component from the biological aging component in age predictor models \citep{Bell2019}. Here, the global aspect of XAI approaches can be used to identify hallmarks for particular age stages in healthy individuals that are part of the chronological aging component, as well as the effects of the environment, medical history, and many other factors on the human body to determine the biological component. The XAI can also help with identifying the set of biomarkers that are most representative of an individual's health status, as well as discarding redundant ones.

A summary of the discussed questions here is shown in Figure \ref{fig:age_prediction}. Age prediction is commonly a regression problem that can use various biological features as an input. Models seek to predict a person's chronological age with as small an error as possible, but their result is more about biological age. The difference between chronological age and biological age is called acceleration (if biologically the person is older than chronologically) or deceleration (if biologically the person is younger than chronologically), as shown in Figure \ref{fig:age_prediction}A. The problem of age prediction inevitably raises the question of choosing a suitable model. Among the many criteria that influence the final choice, there are two main ones: model performance in terms of predictive accuracy and interpretability (Figure \ref{fig:age_prediction}B). Classical, simple models (e.g., linear, treelike ones) are usually easy to interpret, but they may not take into account the complex relationships between the input features, thereby showing worse results. More complex state-of-the-art models (like deep neural network (DNN) architectures) often give better results, but they are black boxes, which do not allow to explain the principles by which models make this or that decision. This is where XAI approaches come in.

% Figure environment removed

The primary aim of this review is to provide the current state-of-the art of XAI methods applied to human biological age estimation. For a systematic analysis of existing works, a division into the main body systems is proposed, for each of which different types of laboratory tests and metrics are used as an input for constructing ‘aging clock’ models. For each work, the main biological results obtained with the XAI techniques are described, as well as those revealing previously unknown dependencies between various biomarkers and age acceleration, or confirming such dependencies found earlier. This paper has the following structure. Section \ref{sec:AI} describes the place of age prediction problem and XAI methods among all ML approaches, describes the main classifications of XAI methods and their application to different types of data. Section \ref{sec:XAI} presents a comprehensive review of studies that use XAI methods to explain the results of age predictor models, divided by the assessed body systems and the types of input data used. Section \ref{sec:conclusion} concludes the work, summarizing the achievements of XAI in the field of age-associated changes in the human body.

\section{Artificial Intelligence, data types and methods}\label{sec:AI}

AI is an extremely broad field of research and development, finding its application in almost all spheres of modern life and helping to solve many complex problems. AI makes it possible to create data-driven decision-making systems \citep{Ali2023}. ML models are a broad class of AI approaches, combining both classical models and more complex DNNs. The main types of ML include supervised learning, unsupervised learning, and reinforcement learning. Depending on the type of the predicted value in supervised learning, there are classification tasks (class is predicted) and regression tasks (continuous value is predicted). The place of biological age prediction is shown in Figure \ref{fig:AI}A, which is a chronological age regression task (age - continuous value - is predicted) on different biomedical data. The representation as an age range classification task is less common. Biological age can be predicted using both simple methods and complex neural network architectures. At the same time, an increasing amount of data requires more and more advanced methods to be able to detect nonlinear relationships between input parameters and their impact on the final result of the biological age prediction. Most neural network architectures are ‘black boxes’ that are models for which the reasons for their decisions and consequences to which changes in input data may lead are unknown. This is where XAI methods can help.

The chronology of publication of papers proposing the XAI methods considered in this review in biological age prediction tasks is shown in Figure \ref{fig:AI}B. The earliest methods were the permutation feature importance (PFI), variable importance measure (VIM) \citep{Breiman2001}, and partial dependence plot (PDP) \citep{Friedman2001} proposed in 2001. PFI, one of the types of VIMs, is used for tabular data and represents a reduction in the model estimate when a single feature value is randomly shuffled. These metrics were originally proposed for tree models, but can also be used for other methods. PDPs are used to analyze and visualize the interaction between the set of interesting input features and the final model estimate. One of the first approaches developed for input image data is saliency maps \citep{Simonyan2014}, which are still widely used today and have been adapted for sequence input data as well. Saliency maps show which regions of the input images were used by the AI model to make decisions and can provide a visual representation of how regions important to the model fit with human attitudes. Guided Backpropagation \citep{Springenberg2015} builds on the ideas of Deconvolution \citep{Zeiler2013} and Saliency \citep{Simonyan2014}, solving the problem of negative gradient flux and minimizing the noise they cause. This approach is also used for input image data. Another well-known method for images and sequences is Class Activation Mapping (CAM) \citep{Zhou2016}. CAM usually uses a global average pooling layer after the convolutional layers and before the final fully connected layer. Later, a Grad-CAM (Gradient-weighted Class Activation Mapping) modification was proposed, which uses a gradient approach to generate CAM \citep{Selvaraju2020}. Accumulated local effects (ALE) \citep{Apley2020} are similar to the concept of PDPs in that they both aim to describe how functions on average affect model predictions. ALE eliminates the bias that occurs in PDPs when the feature of interest is highly correlated with other features. LIME (Local Interpretable Model-agnostic Explanations) provides locally accurate explanations in the neighborhood of the explained instance \citep{Ribeiro2016}. After obtaining a surrogate dataset, it weighs each row according to how close they are to the original sample and uses a feature selection method, such as Lasso, to obtain the most important features. As the name implies, this approach can be applied to any model and provides only local explainability. DeepLIFT (Deep Learning Important FeaTures) is a method of decomposing the output prediction of a neural network to a particular input by back propagating the contributions of all neurons in the network to each input feature, comparing the activation of each neuron to its ‘reference activation’ \citep{Shrikumar2017}. It is applied only to neural networks. One of the best known and most widely used XAI methods is SHAP (Shapley Additive exPlanations), a game-theoretic approach to explaining the results of any ML model. It relates optimal credit allocation to local explanations using classical Shapley values from game theory and related extensions \citep{Lundberg2017}. SHAP is applicable to almost any model, input data type, and is used for both global and local explainability. SmoothGrad \citep{Smilkov2017} adds Gaussian noise to the input data and calculates the average of all samples to reduce the importance of less frequent features. This approach is also specific to neural networks and is typically used for image input data. One of the most recent approaches for explaining the results of convolutional neural networks (CNNs) with input image data is attention maps \citep{Jetley2018}. This approach generates intermediate representations of the input image at different stages of the CNN pipeline and outputs a two-dimensional matrix of scores for each map. As for saliency maps, CAM and Grad-CAM, this approach allows to compare the representation of the neural network and the human in terms of important regions of the image highlighted for decision making. DeepPINK (Deep feature selection using Paired-Input Nonlinear Knockoffs) \citep{Lu2018} offers a special pairwise-connected layer for the neural network to encourage competition between each original feature and its knockoff counterpart. DeepPINK offers an algorithm-independent measure of feature importance with more power than the naïve combination of the knockoffs idea with a vanilla multilayer perceptron.

ML models can be built on different types of data that they take as an input. Among the most common data types are images, tables, and sequences (like signals or texts). Images and sequences are examples of unstructured data, while for tabular data the features are already extracted. For different types of data, various types of methods dominate: for images and sequences, CNNs are the most common \citep{Hershey2017, Sultana2018}, for tabular data gradient-boosted decision trees (GBDTs) has long shown the best results, but now specialized neural network approaches that adapt techniques from other fields are also actively developing \citep{Borisov2022, Grinsztajn2022, Shwartz-Ziv2022}. XAI methods are also mostly specific to different types of data - some methods are applied to images and sequences, others are applied to tabular data (Figure \ref{fig:AI}C). But approaches like SHAP and LIME can be adapted to almost any type of input data and different models, and are very common in a wide range of applications. Different maps are usually used for images - saliency maps, CAM, attention maps and their modifications, as well as methods based on backpropagation. Classical approaches such as PFI, PDP and their modifications, as well as DeepPINK, are used for tabular data.

XAI methods are divided into model-agnostic, which can be applied to any type of ML models, and model-specific, which can be applied only to a certain class of models (usually to neural network architectures). Among model-agnostic methods there are such classical ones as PFI, VIM, PDP, ALE, as well as more modern SHAP and LIME. The other considered methods are specific for neural networks (usually convolutional ones). Another type of classification of XAI approaches includes global and local explainability. Global explainability attempts to interpret the behavior of the model as a whole, revealing general patterns. Local explainability attempts to obtain an interpretation for each individual sample and to identify the features that affect the decision in each case. The distribution of methods among the different groups is shown in Figure \ref{fig:AI}C.

% Figure environment removed


\section{XAI in age prediction studies}\label{sec:XAI}

There are a lot of different age prediction models. Only a subset of them use XAI approaches to identify the most important features that contribute most to the final prediction. Because the models use a variety of biomedical data as input, we order them by the main body systems (Figure \ref{fig:XAI}). Separately, we also consider aggregated information that combines data from multiple body systems.

% Figure environment removed

\subsection{Nervous System}\label{sec:XAI:subsec:nervous}

\subsubsection{Brain age prediction with MRI data}\label{sec:XAI:subsec:nervous:MRI}

Brain age prediction is one of the most common tasks due to the close relationship between brain function, aging process, and neurodegenerative age-related diseases \citep{Cole2017}. MRI is a common noninvasive test that is very informative in assessing the functional state of the brain. Higher brain age compared to chronological age is evidence of accelerated aging and requires increased attention to health status, including the progression of neurological, neurodegenerative, and psychiatric diseases \citep{BoscoloGalazzo2022}, while lower brain age compared to chronological age is evidence of decelerated aging. To predict brain age, various metrics and morphological characteristics of MRI scans or whole images can be used as input for ML models. The most advanced DL approaches require XAI techniques to understand which MRI parameters affect the final model solutions and to increase the confidence in the models to enable clinical applications.

Approaches to solving the brain age prediction problem were proposed even before XAI methods became widespread. However, the need to determine the importance of certain features and to explain model predictions was identified even then. In particular, \citep{Cherubini2016} made one of the first attempts to visualize the most important MRI regions based on the analysis of individual voxels (single elements of a 3D brain image). Age prediction was performed using a simple linear model, and the coefficient values before the individual voxels allowed to identify and to highlight the most important brain regions for age prediction on the 3D model.

The development of DL models led to the creation of architectures capable of taking images as an input, like CNNs. Such models have also become widespread in the age prediction task based on MRI images. However, the explanation of which features influenced the result is significantly complicated by the large dimensionality and complexity of the model. Comparatively simple methods for explaining the importance of individual brain regions in age prediction are rather straightforward and computationally costly, although they have been used quite successfully (e.g., PFI \citep{Kolbeinsson2020}). However, more advanced XAI methods have also been developed in this area, and the importance of individual brain regions in the age prediction task has been determined using various maps that highlight the most active regions in the image: attention maps, CAM, saliency maps \citep{Feng2020, Hepp2021, Hu2021, Lam2020, Mouches2022, Ren2022, Wang2019, Yin2023}, SmoothGrad \citep{Levakov2020, Wilms2021}. These approaches have also uncovered interesting relationships between aging, brain function, and neurodegenerative diseases. In particular, the highlighting of important features for age prediction allowed to identify brain regions associated with the development of dementia \citep{Wang2019}, as well as to find an overlap of brain regions associated with both age and Alzheimer's disease \citep{Lam2020}. In \citep{Levakov2020}, the aggregation of explanation maps for many samples allowed to identify the relationship of cerebrospinal fluid (CSF) volume with age. Changes in the activation patterns of different brain regions with age were shown in \citep{Feng2020}. For younger samples, a smaller error was observed in \citep{Hepp2021}, and higher estimation accuracy in central brain regions was also shown. Work \citep{Hu2021} found that the most age-associated brain regions in children and adolescents are associated with movements, language, and processing of sensory information like vision and sense of touch (precentral gyrus, postcentral gyrus, inferior parietal lobule, middle temporal gyrus, medio ventral occipital cortex). In contrast, in the elderly, the amygdala, hippocampus, and thalamus, which are responsible for the limbic system, behavioral and emotional responses, and consolidating memories, are the most involved in age prediction \citep{Ren2022}. Sex-specific aspects of brain aging were revealed in \citep{Yin2023}.

Less popular, but nevertheless also used, is the U-Net architecture-based importance map extraction approach and related modifications (U-Noise) \citep{Bintsi2021, Popescu2021}, as well as the guided-backpropagation approach \citep{Cho2022}.

Another representation of MRI data is tabular, which characterizes morphological features: the sizes and volumes of the main structural brain regions measured from MRI images, as well as the ratios between them. As the input data in this case are in the form of a table (with numeric, ordinal, or categorical values), and all features have already been extracted, the approaches to explaining age prediction models based on such data are different. The most common XAI approaches, such as SHAP and LIME, are widely used in this case \citep{Ball2021, Ballester2023, Cumplido-Mayoral2023, Gomez-Ramirez2022, Han2022, Lombardi2021, Ran2022, Salih2021, Scheda2022} and also allow to make conclusions about the influence of morphological features on brain age prediction. Interestingly, a direct comparison of these methods for local explainability (explaining the effect of individual features on age prediction for each individual) of brain age predictions in \citep{Lombardi2021} showed little agreement between the methods: SHAP method emphasized the importance of metrics related to the precentral gyrus, inferior and lateral occipital cortex, and statistical descriptors of CSF volume, whereas LIME method valued more white matter volumes of opercular and triangular part of inferior frontal gyrus and inferior temporal gyrus. Work \citep{Ball2021} focused on age prediction for a cohort of children and adolescents to use XAI for identifying typical changes during adulthood that appeared to reflect developmental remodeling in the cortex. Similar result was observed in \citep{Scheda2022}: cortical thickness and brain complexity showed maximum contribution to age prediction in both children and adults; like total intracranial volume and cortical thickness in \citep{Han2022} for different models. Brain-to-intracranial-volume ratio appeared to be the most important in predicting brain age in work \citep{Gomez-Ramirez2022}. The brain age vector, stroked with Shapley values, has been shown to be a useful tool for the early screening of mild cognitive impairment and even Parkinson's disease \citep{Ran2022}. In \citep{Ballester2023} the relationship between brain age and total gray matter volume in schizophrenia was revealed using SHAP. The use of different brain MRI metrics in \citep{Cumplido-Mayoral2023} showed that the most age-associated brain regions in both males and females were amygdala, nucleus accumbens, cerebellar white matter, lateral ventricles, and insula. Specific for females in the age prediction task were the thicknesses of the transverse temporal cortex, the pars triangularis, the inferior parietal cortex, and the left frontal pole, as well as the volume of the left entorhinal cortex. Specific for males were the thicknesses of the left isthmus cingulate, the right cuneus, and the cortical volume of the superior frontal and right rostral middle regions. An unusual application of SHAP was proposed in \citep{Sun2022}: instead of morphological characteristics, voxel MRI parameters were used as input data, and SHAP highlighted important voxels, which were then used to build dynamic connectivity networks. 

Original approach was proposed in \citep{Monti2020}: functional connectivity networks were constructed for different brain regions, which were then used to solve the age prediction problem. Self-explainable normalizing flows were proposed in \citep{Wilms2021}, demonstrating comparable results with CNN model and SmoothGrad XAI approach.

\subsubsection{Brain age prediction with EEG data}\label{sec:XAI:subsec:nervous:EEG}

Another common non-invasive test to evaluate brain function and activity is EEG. Changes in the brain caused by aging or age-associated diseases can affect the electrophysiological activity of the brain. The input for ML models in the age prediction task from EEG data can be the signals themselves or their frequency, spectral, amplitude, and/or other characteristics. 

XAI approaches in the age prediction task based on EEG data are not commonly used; more often, built-in mechanisms for determining the feature importance are considered \citep{AlZoubi2018, Sun2019, Vandenbosch2019}. Nevertheless, even these mechanisms are able to indicate important EEG details related to age. In particular, it was shown in \citep{Vandenbosch2019} that low-frequency power decreases sharply from childhood to adolescence, with peak alpha frequency increasing with age and peak alpha power decreasing with age. A study of EEG during sleep \citep{Sun2019} has shown that the duration of Non-REM (non-rapid eye movement) phases of sleep, as well as the total sleep time and the awakening time rate influence accelerated brain aging. In recent work \citep{Khayretdinova2022}, deep CNNs were used to predict brain age, and attribution maps were used as an XAI method to identify the EEG signal elements whose activity most affects age prediction. Interestingly, the state of the eyes (open or closed), as well as the activity from the frontal electrodes (which may reflect eye movement activity) were found to be significant features.

\subsubsection{Retinal age prediction}\label{sec:XAI:subsec:nervous:retina}

Health changes, symptoms of various diseases, age-associated conditions can be reflected in the eye and, in particular, the retina, so the prediction of retinal age seems to be a promising and interesting direction. In general, in ophthalmology, aging is a significant factor in the development of many eye diseases, leading to a significant decrease in visual acuity and even to vision loss. It is also worth noting that most tests that assess retinal age are non-invasive and can be used in clinical applications for example to track early signs of diseases (particularly cardiovascular diseases) or to estimate the rate of their development. The types of data used for the age prediction task include retinal fundus photographs \citep{Nusinovici2022, Poplin2018, Zhu2023}, anterior segment morphological features \citep{Ma2021}, and macular optical coherence tomography (OCT) \citep{Chueh2022, Shigueoka2021}.

If different types of images are used to predict retinal age, CNNs and corresponding XAI methods have traditionally been used to find the image details that most strongly influence the final result: attention maps \citep{Poplin2018, Zhu2023}, Grad-CAM \citep{Chueh2022, Shigueoka2021}, and saliency maps \citep{Nusinovici2022}. They allow to identify some significant effects. In particular, \citep{Poplin2018} showed that vascular regions in the retina are not only associated with age, but can also indicate cardiovascular risk, while perivascular surroundings reflect changes in Haemoglobin A1c levels. It is also interesting that sex differences in the ocular fundus were found to be concentrated in the optic disc, vessels and macula. Retinal vessels have also been associated with retinal age predictions in \citep{Zhu2023}. For age prediction by macular OCT, whole layers of retina were found to be the most important for the age groups considered \citep{Chueh2022, Shigueoka2021}. Retinal fundus photographs were used not only to predict age but also to estimate mortality, with macula, optic disc, and retinal vessels making the highest contribution \citep{Nusinovici2022}.

Tabular data are also used to predict retinal age, albeit not very often, and usually represent various morphological metrics measured from retinal fundus photographs. Such metrics can be measured manually, or modern medical image segmentation approaches can be used. In this case, not the same XAI approaches as for images are used to determine feature importance, but, for example, PFI \citep{Ma2021}. In \citep{Ma2021}, the most important metric for age prediction is anterior chamber volume, while negatively correlated with age, as well as absolute degree of anterior corneal astigmatism and corneal thickness parameters. 

\subsection{Cardiovascular System}\label{sec:XAI:subsec:cardiovascular}

\subsubsection{Age prediction with blood data}\label{sec:XAI:subsec:cardiovascular:blood}

Blood is an integral part of the cardiovascular system. It circulates within the human body, communicating with all systems, and it is an informative indicator of health status. The biochemical blood test is a common tool for assessing the human condition in clinical practice and is a rather sensitive indicator of many pathologies, including age-associated changes and diseases. Whole blood, its various products and indicators were among the first data types, which served as a basis for building models of human biological age estimation or the so-called ‘clocks’. 

Blood biochemistry data have not been avoided by such models. These data always have a tabular format, representing a set of numerical values for each individual. In this regard, classical methods \citep{Sagers2020}, gradient ensemble approaches \citep{Wood2019}, and neural networks \citep{Mamoshina2018, Mamoshina2019a, Putin2016} are used to predict age in this case, and XAI methods, such as PFI \citep{Mamoshina2018, Mamoshina2019a, Putin2016} and SHAP \citep{Wood2019}, are used to explain predictions. However, not all biochemical parameters and cell counts affect biological age equally. In \citep{Putin2016}, levels of albumin, glucose, alkaline phosphatase, urea, and erythrocytes were found to be important markers of age in the constructed hematological clocks. When studying the ethnic specificities of age prediction, 5 markers, namely albumin, hemoglobin, urea, and glucose, were the most predictive for all the considered populations: Canadian, South Korean, and Eastern European \citep{Mamoshina2018}. A higher rate of aging in smokers was shown by hematological clocks in \citep{Mamoshina2019a}; it was also found that such cardiovascular risk indicators as high cholesterol ratio and fasting glucose significantly influenced age predictions in smokers. Glucose was also at the top by importance among both men and women in \citep{Wood2019}, SHAP was also used to explain individual predictions, highlighting the contribution of levels of each individual feature to the resulting age prediction for each participant. In \citep{Sagers2020}, a built-in feature importance method for a random forest showed that the ranking of features for age prediction was highly dependent on age range and highly correlated with sex and race/ethnicity.

Less common, but nevertheless significantly informative, are immunological profile data obtained also from whole blood. It can characterize not only the status of the human immune system but also can be associated with the phenomenon of inflammation - an increase in circulating inflammatory mediators with age. Since the immunological profile is tabular data (numerical values of cytokine levels) like the biochemical profile data, ML methods for tabular data and appropriate XAI methods are applied to it. However, explanations of age prediction models from immunological data are not very common to date. Work \citep{Kalyakulina2023} using SHAP values shows that CXCL9 level plays a crucial role in age prediction of healthy participants and patients with end-stage renal chronic disease. 

One of the most well-known concepts among all age prediction models is the epigenetic clocks. It uses DNA methylation data, a modification that affects DNA activity without changing the primary sequence, to predict age. DNA methylation of any human tissue and/or organ can be considered, however, whole blood methylation is the most commonly used for age prediction due to its low invasiveness. There are many technologies for obtaining DNA methylation data of different resolution, low resolution data are more often used in forensic applications, and high resolution epigenome-wide data are more often used for scientific purposes. As well as other previously mentioned data obtained from whole blood, methylation data are tabular numerical data.

The best known classical works proposing epigenetic clocks for both research \citep{Hannum2013, Horvath2013, Levine2018, Lu2019} and forensic applications \citep{Park2016, Zbiec-Piekarska2015} used linear models, in particular ElasticNet, which are easily explainable. Also, classical models, which have a built-in functionality for determining the importance of all input features, have been used to construct epigenetic clocks \citep{Gao2020, Montesanto2020}. However, more recent approaches to epigenetic age estimation, variational autoencoders \citep{Levy2020}, DNNs \citep{deLimaCamillo2022}, tabular data-handling architectures like TabNet \citep{deLimaCamillo2022} are more interesting and XAI methods like SHAP \citep{Levy2020, deLimaCamillo2022} and DeepPINK \citep{deLimaCamillo2022} are used to interpret them. Such approaches helped to identify meaningful relationships between the methylation of different genome regions, aging, and age-associated diseases. In particular, work \citep{Levy2020} proposed MethylNet, one of the first models that allowed not only the use of modern methods to predict age or classify diseases, but also to explain the obtained results using XAI. It was also shown in this work that all participants can be conventionally divided into two groups according to the value of chronological age (younger than 44 years and older), in each of which similar CpG sites significantly affect age predictions, while these CpG sites between groups differ significantly. This suggests that DNA methylation patterns change with age and differ significantly in young and old age. One more noteworthy work \citep{deLimaCamillo2022} proposes an AltumAge model, a pan-tissue epigenetic clock with a mechanism to explain predictions based on SHAP and DeepPINK. Although methylation data from various organs and tissues were used, special attention was paid specifically to blood DNA methylation and comparison with existing models. Certain features (CpG sites) most associated with age were identified, different types of dependence (linear and nonlinear) were shown, and features associated with various diseases affecting age-related acceleration in blood methylation (HIV, Down Syndrome, autism, atherosclerosis) were selected.

The unusual data used to predict age are the results of toxicological blood tests. Numerical values of various metabolites can be used for age estimation, as well as many previously reviewed measures. In \citep{Lassen2023}, metabolic profiles of drivers suspected of driving under the influence of drugs were examined and age was predicted for them using ML techniques. Using SHAP, it was shown that the levels of such age-related, age-associated diseases and stress biomarkers as acylcarnitines, cortisol, and benzoic acid contribute the most to age acceleration (predicted age is higher than chronological age). At the same time, tryptophan pathway metabolites, serotonin, and kynurenate contribute to age deceleration (predicted age is lower than chronological age). 

Another interesting type of data used for age prediction is circular RNA (circRNA) levels, which can also be obtained from human blood. The application of such data is promising in forensic science, especially in the case of fragmentary available data after the discovery of crime evidence, when it is necessary to determine the age of the victim or perpetrator. Like all the previously discussed blood parameters, this one is numerical and the input data itself is tabular. But age prediction and model explanations based on circRNA data are not very common at the moment. Work \citep{Wang2022} considers whole blood circRNA levels and builds a model perspective for forensic applications. Not the most popular VIM approach to determine the contribution of different circRNAs to age prediction in terms of the effect on model accuracy variation and standard deviations.

\subsubsection{Heart and arterial age prediction with image data}\label{sec:XAI:subsec:cardiovascular:image}

Cardiovascular age can be assessed not only by various blood tests but also by MRI or ultrasound images. In this case, specialized neural network architectures (CNNs) are used for age prediction, and, accordingly, specialized XAI methods, as attention maps. Although imaging for cardiovascular age prediction is not frequently used, it is worth noting two preprints that use XAI approaches to determine the importance of parts of the heart \citep{Goallec2021} and carotid \citep{Goallec2021a} images. The work \citep{Goallec2021} uses frames from cardiac MRI videos and cardiac MRI images to predict heart age, and attention maps have shown that mitral and tricuspid valves, aorta, and interventricular septum have the greatest impact on age prediction. In \citep{Goallec2021a} carotid ultrasound images were used to predict arterial age, attention maps showed that the carotid artery itself, as well as surrounding tissue and jugular veins make the highest contribution. Vascular images obtained by magnetic resonance angiography are also used in the assessment of cardiovascular age. It was shown in \citep{Nam2020} that the vascular regions along the cerebral arteries are most contributing to the cerebral vascular aging estimation result.

\subsubsection{Cardiovascular age prediction with signal data}\label{sec:XAI:subsec:cardiovascular:signal}

Another type of data not previously discussed that can be used to predict age is signals (sequences). The most well-known signal in cardiovascular studies is the ECG, which is a recording of the electrical activity of the heart. This test is noninvasive, so it is extremely common in diagnostic applications. To use ECG data in predicting the age of the heart, it is not necessary to use modern approaches. One elegant and fairly simple option is to reduce the data to tabular. To do this, numerical metrics (usually durations and amplitudes) of different waves and complexes of ECG signal are calculated, which are used as an input. In this case, in particular, classical and linear models with a built-in functional for determining feature importance are used for heart age prediction \citep{Attia2021, Lindow2022, Starc2012}. In one of the first works \citep{Starc2012} that predicted heart age by ECG metrics, normalized RR-interval variability (which can be regarded as one of the indicators of arrhythmia) turned out to be the most important feature. A related result was obtained in \citep{Attia2021}, where the mean RR-interval duration, as well as the maximum amplitudes of the peaks R and S of the QRS complex were among the most important features for age prediction. However, a comparison between models based on human-derived features and neural network-derived features showed a preponderance in favor of the latter, suggesting the hypothesis that not all important ECG features for age prediction may be detectable by humans within traditional analysis. In \citep{Lindow2022}, heart age was calculated separately for men and women, and it was shown that the sets of important features in the two sexes are very similar and include, in particular, length of P wave, QT interval, and heart rate.

More modern and advanced methods allow the signals themselves to be used as input data for prediction models. In particular, image-based approaches can be adapted to unidimensional data (signals), in which case the approaches previously considered for images, such as saliency maps \citep{Lima2021}, can also be used to explain predictions. In \citep{Lima2021}, an extremely interesting experiment was performed: physicians were shown similar-looking ECG recordings, but the predicted age differed significantly. It was shown that physicians are not always able to identify differences in the ECG, leading to different predictions within the human-based approach. This, as in the aforementioned work \citep{Attia2021}, suggests the hypothesis that ML models can find ECG features that cannot be identified within the traditional analysis. At the same time, saliency maps have shown that the highest contribution to the prediction is made by the low-frequency components of the ECG, which include, in particular, P and T waves. Special models capable of handling biomedical signals, such as ECG12Net, are also being developed, and XAI methods, such as CAM \citep{Chang2022}, are also applied to them. In \citep{Chang2022} for patients with coronary artery disease, the effect of relative irregular baseline, as well as aVL leads in general, on the prediction of elevated cardiac age has been shown. 

Another type of signal widely used for rapid assessment of cardiovascular health is the photoplethysmogram (PPG), which characterizes the pulse wave velocity and the filling of small vessels with blood. As for ECG, one approach to age prediction by PPG is to convert the signal to tabular data. For this purpose, various temporal, amplitude, frequency metrics are calculated, which are used as input for ML models, in particular GBDTs \citep{Shin2022}. Although GBDT models have a built-in functionality for determining feature importance, a deeper analysis of global and local explainability can be performed using XAI approaches, like SHAP. In \citep{Shin2022}, the nasal PPG is studied, and the highest contribution to age prediction is the difference between incident wave peak and reflected wave peak amplitudes, which tends to decrease with age. As for ECG, modern methods like CNN, which take waveforms as input and use Grad-Cam method XAI to explain the predictions, are also used for PPG \citep{Shin2022a}. This approach evaluates the shape of individual signal elements and features it as an integral part of the data, taking part in the age prediction. The work \citep{Shin2022a} shows that the waveform near the systolic peak contributes most to the characterization of vascular aging and this result is consistent across age groups.

\subsection{Respiratory System}\label{sec:XAI:subsec:respiratory}

\subsubsection{Age prediction with chest X-ray data}\label{sec:XAI:subsec:respiratory:xray}

Chest X-ray is a common medical procedure that is often performed to assess lung function and detect signs of various respiratory diseases. Due to the relative safety and noninvasiveness of this test, it can be used to assess not only the respiratory age of an individual but also the age of the structures adjacent to the lungs. As modern neural network architectures allow the use of images as an input, special XAI methods, such as saliency maps \citep{Karargyris2019} and Grad-CAM \citep{Ieki2022, Raghu2021}, are used for identifying the regions of the X-ray image most contributing to age prediction. In \citep{Karargyris2019}, different projections of chest X-rays were investigated, and such regions as the neck, clavicles, mediastinum, ascending aortic arch and spine were highlighted as the most important for age prediction. Interestingly, the important regions appeared to be age-dependent: lungs and bone and joint regions (clavicles and spine) were more significant only in younger participants. In \citep{Raghu2021}, it was found that the mediastinum, heart silhouette, and aortic protrusion dilate and become tortuous with age. Age prediction is also affected by such parts of the images as the diaphragm silhouette, upper mediastinum and lower neck, associated with age-related degenerative changes in the lower cervical spine. Aortic tortuosity and calcification, also found in \citep{Ieki2022}, have been shown to be associated with aging and are signs of atherosclerotic diseases. It has also been shown that changes in the lungs, such as fibrosis, detected on X-rays, significantly increase the predicted age.

\subsection{Endocrine and Digestive Systems}\label{sec:XAI:subsec:endocrine}

\subsubsection{Abdominal age prediction with MRI data}\label{sec:XAI:subsec:endocrine:MRI}

Age-related changes affect all systems and organs of the human body, including abdominal organs. Since these organs are deep inside the body, invasive tests are rarely used, and MRI is a common tool for assessing the condition of these organs. As previously discussed, MRI images of various organs are often used to assess a patient's age, and special XAI approaches, like attention maps, are used to identify important regions. Unfortunately, there are few studies on age prediction from abdominal MRI scans. One such study \citep{LeGoallec2022} looks at MRI of the liver and pancreas for age prediction. Attention maps built separately on liver MRI images and on pancreatic MRI images highlighted common important areas of the abdomen, including the liver, stomach, spleen, as well as muscle, bone, and fatty tissue. This may reflect age-associated changes in the liver associated with inflammation, decreased blood flow, and decreased liver volume. Age-related changes in the pancreas include fatty degeneration and lobularity. Interestingly, for the examples of decelerated and normal aging, the important areas of MRI are more concentrated in the liver, whereas for the examples of accelerated aging they are more concentrated in the pancreas and stomach.

\subsubsection{Age prediction with gut microbiome data}\label{sec:XAI:subsec:endocrine:gut}

The gut microbiome is a huge community and represents a complex, constantly evolving system. It is responsible for many body functions, including digestive, immune, and metabolic ones, it changes over time and can also be an indicator of an individual's health status. Therefore, models for assessing age by the gut microbiome have been proposed, and in this case, XAI methods are of particular importance. They allow to estimate which kind of microorganisms most influence age prediction and which composition of microbiota leads to accelerated or decelerated aging. Gut microbiota data are tabular and represent quantitative estimates of various microorganisms; accordingly, both built-in explainability methods for classical models \citep{Gopu2020, Huang2020, Shen2022} and specialized XAI methods, like ALE \citep{Galkin2020} and PFI \citep{Chen2022}, are applied to them. 

In \citep{Huang2020} it was shown that there are regional differences in the gut microbiome that are important for age prediction. In particular, Bifidobacterium were found to be specific for the age predictor in the Chinese cohort, while Lachnospiraceae, Ruminococcaceae, and Clostridiaceae were important for all studied cohorts. In \citep{Gopu2020}, it was found that Haemophilus, Turicibacter, and Romboutsia groups are the most negatively correlated with age (i.e., an increase in their number is associated with a decrease in the predicted age), while Streptococcus and Propionibacterium are the most positively correlated (an increase in their number is associated with an increase in the predicted age). In \citep{Galkin2020}, among the important features appear both those that have a positive effect on intestinal function (Bifidobacterium spp., Akkermansia muciniphila, Bacteroides spp.) and those that have a negative effect (Escherichia coli, Campylobacter jejuni). Finegoldia magna, Bifidobacterium dentium, and Clostridium clostridioforme, as shown in \citep{Chen2022} have an abundance with age. Interestingly, Cellulosilyticum has an abundance in the long-lived group \citep{Shen2022}.

\subsection{Skeletal and Muscular Systems}\label{sec:XAI:subsec:skeletal}

\subsubsection{Bone age assessment}\label{sec:XAI:subsec:skeletal:bone}

In clinical practice, estimation of bone age by hand X-rays is widespread. They are most often used in two extreme age cases - in pediatrics and gerontology. In the process of growth and development, certain parts of the child's hand bones develop at a specific time, so age assessment by X-rays allows to trace the presence and development of genetic, endocrinological and other diseases. On the contrary, disorders in the bone structure are associated with many age-associated diseases, and bone fragility affects many elderly people, so even here bone X-rays are often used for diagnostics. Nevertheless, many developing methods for bone age assessment can be extended to all age ranges, rather than being limited to children and the elderly. Since images are used for bone age assessment, XAI methods, such as attention maps \citep{Lee2017, Wu2019}, CAM \citep{Bui2019, Zhao2018}, and detection of regions of interest (ROIs) corresponding to the most active neurons \citep{Spampinato2017}, are used to identify the regions of X-rays most important for the outcome.

In \citep{Spampinato2017} the authors propose a CNN-based BoNet model developed specifically for processing images of hand X-rays. The most active neurons of this network highlighted radius and ulna, as well as tiny parts of carpal zones (which differ from the classical approaches of clinicians based on the Tanner-Whitehouse method that highlights entire carpal zones). Interestingly, the regions most important for bone age prediction vary in different age groups \citep{Lee2017}. Attention maps for CNN showed that in prepuberty the model focuses on carpal bones and mid-distal phalanges, in early-mid and late-puberty phalanges play the greatest role, and in postpuberty the wrist (where radius and ulna are close to each other) comes first. At the same time, no significant differences between males and females were found. However, a more detailed analysis in \citep{Zhao2018} using CAM showed that the metacarpal bones are important for predicting bone age in males, while for females the method focuses on a large number of hand bones, including caudal phalanges, metacarpal bones and carpal bones. Carpals have been shown to be important for infants and toddlers, metacarpals and phalanges for older ages \citep{Wu2019}. Interestingly, the same work has shown how noise can introduce uncertainty into the outcome of attention maps and highlight incorrect (mostly background) regions of the image.

An interesting approach is proposed in \citep{Bui2019}. The authors do not use the whole X-ray image of the hand for bone age prediction, but limit themselves to six individual regions, selected by Tanner-Whitehouse (TW3) methods: dp3 (distal phalanx of the third finger), mp3 (middle phalanx of the third finger), pp3 (proximal phalanx of the third finger), mc1 (first metacarpal), ulna, radius. In this case, the maps are used for a more detailed analysis of the important regions for age assessment. Differences between age groups are also very clear in this case, as they show in detail the difference between the bone structures.

Another application of bone age prediction models is forensic science. Estimation of age-at-death (exact value or range) is an important step in the study of human remains, and pelvic bones are most commonly used for this purpose. Bone characteristics such as surface estimates, texture, and structure of different parts of the pelvis are usually considered as an input for ML models. Thus, the data have a tabular structure, and both classical methods and DNNs are used to estimate bone age. In \citep{Koterova2018}, the built-in XAI techniques for classical methods highlighted such regions of the pubic symphysis as posterior plate, ventral plate, dorsal lip. They helped to distinguish between samples under 30 years old, 30-40 years old, and over 40 years old. Infrequent rule-based XAI methods are also found in explaining age-at-death estimation in forensic applications \citep{Gamez-Granados2022}. Here, the articular face, dorsal plateau, and ventral margin showed their importance in age assessment for all age ranges. Upper symphysial extremity, bony nodule and lower symphysial extremity played a major role in young samples, while irregular porosity and ventral bevel played a major role in older samples.

\subsubsection{Dental age prediction}\label{sec:XAI:subsec:skeletal:dental}

One of the most common organs used for age estimation in forensic science is teeth. They are rather resistant to the influence of negative environmental factors, and at the same time, they explicitly reflect age changes. Age estimation using dental images can be used in archaeology, anthropology, forensic science and many other applications where it is necessary to verify age or provide evidence that a person is a child or an adult. Dental information can be presented in different forms, but orthopantomographs - panoramic radiological images of teeth and surrounding bone structures - are the most common representation. Modern DL approaches can handle them as input, and the common XAI methods for these models are used to highlight the most important parts of the image for the final decision: DeepLIFT \citep{deBack2019}, Grad-CAM \citep{Atas2022, Guo2021, Kim2021, Sathyavathi2023, Vila-Blanco2020, Wallraff2021}. Nevertheless, as for the images of other body structures, the panoramic images of teeth can also be presented in tabular form. In this case, various numeric, ordinal, and categorical features describing each individual image are used as input for ML models. To determine the most important features, both built-in feature importance methods \citep{Stepanovsky2017} and special methods, such as SHAP \citep{Lee2022, Patil2023}, are used here.

Tabular representation is not very common in dental age prediction tasks, but there are several interesting works that determine the importance of individual features. The work \citep{Stepanovsky2017} uses unusual input: for each tooth, the numerical value of each feature is the average age over some representative population, reflecting the same type and degree of dental development. Only samples from 3 to 20 years old were considered. It was found that for males, the most important teeth for age estimation are all mandibular (lower jaw) molars, as well as the 2nd premolars, 2nd and 3rd molars of the maxilla (upper jaw). For females, only the 1st premolars, 2nd molars on the mandible and central incisors, canines, 1st premolars, 1st and 2nd molars on the maxilla are the most important. The work \citep{Lee2022} uses more traditional metrics derived for panoramic radiographs, like sizes of certain teeth, interdental intervals, root and crown lengths. For the young participants, the most specific features are the distance between the mandibular canal and alveolar crest, tooth and pulp areas of the upper first molar, and pulp area of the lower first molar. The most specific features for the older participants are the number of teeth, the number of implants, the number of crown treatments, and the presence of periodontitis. It can be noted that the features specific to the young are mostly concentrated on the characteristics of the first molars, and the features specific to the elderly are concentrated on the general condition of the teeth. An interesting approach is considered in \citep{Patil2023}, where the authors are focused on the images of the second and third molars, with only the values of the mesial and distal roots being considered for each tooth. The length of the right side third molar mesial root showed to be the most important for the classification of all considered age ranges (12-25 years with division into 2, 3 and 5 equal groups).

More popular representation of the input data is the whole images. In \citep{deBack2019} the orthopantomograms of participants 5-25 years old were considered. It is interesting that for the youngest participants, the most informative regions are not only concentrated around the molars but also include the maxillary sinus. At older ages, the nasal septum is highlighted along with the molars. The mandibular molars also made the highest contribution to age prediction in \citep{Vila-Blanco2020} for samples younger than 25 years, in \citep{Wallraff2021} for samples 11-20 years old, in \citep{Sathyavathi2023} for samples 10-30 years old, and in \citep{Hou2021} for samples 0-93 years old. In \citep{Guo2021}, the deep CNN in age prediction focused not on tooth morphology (high-density region on X-ray images), but on low-density regions, like dental pulp cavity, periodontal membrane, area between adjacent teeth and area between deciduous and permanent teeth. In \citep{Atas2022}, also not only the teeth were found to be important for age estimation but also the gingival tissue and bone of the maxilla.

Work \cite{Kim2021} proposes to consider not the whole panoramic dental X-ray image, but to focus only on the images of the first molars for each individual (two first molars for each jaw - 4 images in total). However, a different task is considered: not age regression, but the classification of age ranges of different sizes. It was shown that even when considering a single tooth, the details important for age prediction are different in different age groups. The first molar pulp was the most important for all the considered decadal age groups under 50 years (0-9 years, 10-19 years, 20-29 years, 30-39 years, 40-49 years) and over 60 years. Age 0-9 years is also characterized by the eruption degree of the second molar, age 10-19 years is determined by the condition of the alveolar bone and maxillary sinus. At the age of 20-29 years, the periapical area of the first molar comes to the focus, at the age of 30-39 years - the interdental space between the first molar and the second molar, and at the age of 40-49 and 50-59 years - the interdental space and level of alveolar bone between the first molar and the second molar. For patients over the age of 60 years, the occlusal levels of the teeth are important.

An interesting approach is proposed in \citep{Vila-Blanco2022}. The authors present a special architecture of the CNN, which first performs segmentation of the panoramic image and highlights the regions corresponding to each individual tooth. Then, based on these regions, a separate network is constructed which generates estimated per-tooth age distributions for each subject and a final prediction based on certain aggregation policies. For young subjects, it has been shown that the distributions for canines and premolars are best centered on real age. 

\subsubsection{Muscular age prediction with gene expression data}\label{sec:XAI:subsec:skeletal:gene}

It is also possible to estimate an individual's age using more complex data, in particular gene expression profiles. They are numerical values of the expression levels of multiple genes, thus representing typical tabular data. Obtaining such data is usually expensive and time-consuming, so they are not often used in age prediction tasks. Gene profiles can be obtained for different body systems, in particular, for skeletal muscles, as in \citep{Mamoshina2018a}. In this work, the authors consider classical methods with built-in functionality for determining the importance of individual features; they also use the Borda count algorithm to combine the rankings of the features for different models. Among the most important genes were those known to be therapeutic targets for many drugs, as well as genes related to skeletal muscle relaxation. The authors suggest that this result may be important in the development of neuromuscular damage therapy.

\subsection{Integumentary System}\label{sec:XAI:subsec:integumenary}

\subsubsection{Age prediction with skin microbiome data}\label{sec:XAI:subsec:integumenary:skin}

The microbiome can be found not only in the gut, but also on the skin. It is a complex system under constant contact with the environment. The skin microbiome changes over time and can characterize a person's health and wellbeing. There are models for assessing age by skin microbiome composition, and in this case, XAI methods allow to estimate the influence of different groups of microorganisms on accelerated or decelerated aging. These skin microbiome data, like the gut microbiome one, are tabular, representing quantitative estimates of different microorganisms, and, accordingly, both built-in explainability methods for classical models \citep{Huang2020} and specialized XAI methods, like SHAP \citep{Carrieri2021}, are used for them. 

In \citep{Huang2020} it was found that the age-associated composition of the skin microbiome differs in males and females; there are also differences between the forehead and palm microbiome. Among the most important features negatively correlated with age were Mycoplasma, Enterobacteriaceae, and Pasteurellaceae groups, which are involved in age-dependent changes in physiological skin characteristics, such as sebum production and dryness. In \citep{Carrieri2021}, explanations of the age predictor showed Propionibacterium to be the most important feature for predicting young age, reflecting a decrease in relative Propionibacterium abundance with age and associated with a decrease in sebum secretion. The families Alloprevotella, Granulicatella, Gemella and Lactobacillus are also among the features specific for young age. Bacillus, on the contrary, is the most important feature for predicting old age and reflects its dominance in the skin microbiome of the elderly. The Bacteroides, Pseudomonas, and Bergeyella families are also among the elderly-specific features.

\subsubsection{Epidermal age prediction with gene expression data}\label{sec:XAI:subsec:integumenary:gene}

Gene expression profiles, as discussed earlier, are tabular data representing numerical values of gene expression levels. Transcriptome profiles can be obtained, for example, for epidermis, as in \citep{Holzscheck2021}. In this work, the authors obtain pathway ranking based on the correlation of activations of intermediate neurons with chronological age, which is an uncommon approach. The pathways responsible for p53 and TNFa/NFkB signaling, as well as responses to ultraviolet radiation and interferon gamma, were found to be most significantly associated with age.

\subsection{Aggregated Information}\label{sec:XAI:subsec:aggregated}

\subsubsection{Age prediction with whole-body MRI data}\label{sec:XAI:subsec:aggregated:MRI}

Not only biomarkers localized to a single body system can be used for age prediction. Aging is a complex process affecting the whole organism, so age-related changes can develop at different rates and intensities for different organism structures. In this case, more complex biomarkers affecting two or more systems of the human body can be considered. This potentially allows the identification of higher-level correlations between aging patterns between these systems. 

Some of these extensive examinations are whole-body MRIs, which allow analysis of the human body from the neck to the knees, or whole-body X-rays from head to feet, both of which take into account all internal organs as well as muscle and fat tissue distributions. They are less detailed than images of individual organs (e.g., the brain), but provide a more comprehensive assessment that includes the many body systems and interactions between them. As in the previously discussed cases, MRI/X-ray data are images, which means that CNNs are the most common for age prediction, and saliency maps \citep{Langner2020} and Grad-RAM \citep{Goallec2021b} are used to explain their predictions.

In \citep{Langner2020}, MRI image sets for each individual are combined into two types of images, based on the water signal and the fat signal in two projections. The most active regions on the saliency maps were the knee joint, the aortic arch, the area covering the heart, the surrounding tissues, and part of the lungs. In younger participants, the knees, including the contour of the tibia and the outer edge of the calf muscle, almost always appeared to be important, unlike the thoracic region. In \citep{Goallec2021b}, single projection X-ray images were used to construct the age predictor. The most important areas on the attention maps included the neck, upper body, hips, and knees. The maps were also constructed for selected body parts: the lumbar region was most often highlighted for X-ray images of the spine in the sagittal projection; for the hip joint, the greater trochanter of the femur and the joint itself were highlighted; for the knee, the thigh bone, the tibia, and the joint itself were highlighted.

\subsubsection{Face age prediction}\label{sec:XAI:subsec:aggregated:face}

Predicting age from facial images is part of the broad topic of automatic facial analysis. How old a person looks is influenced not only by chronological age but also by health status, environmental exposures, and many other factors. Aging signs can be reflected on the face unevenly, at different time periods and at different levels of manifestation in different people. Therefore, with the simplicity of obtaining data (an ordinary facial photograph), age prediction is a difficult task. XAI methods explain the predictions of models that can handle images. Face age prediction from images was one of the first fields where explanation methods for models working specifically with images as input (mainly CNNs and their modifications) were applied.

In \citep{Agustsson2017} a neural network architecture capable to analyze specifically facial images is proposed, which include pre-processing, aligning and predicting two types of age - real and apparent (perceived by human observers). Sensitivity maps for individual pixels in relation to the predicted age allowed authors to determine which parts of the image the models rely on when making a decision. The sensitivity areas are shifting with age: the forehead and the space between the eyes are important for young people, important areas are relatively uniformly distributed across the face for middle-aged people, the chin and the neck areas are important for seniors. In \citep{Gao2018} work, activation (score) maps for different age groups (children 0-3 years old, adults 20-35 years old, seniors 65-100 years old) also highlight different areas. The results for the first two groups are similar to previous work: for infants, the eyes are important, for adults the eyes, nose, and mouth are important. For the seniors, highlighted areas include the forehead, eyebrows, eyes, and nose. In \citep{Abdolrashidi2020}, edge patterns around facial parts as well as wrinkles were highlighted when predicting sex and age. The individual layer activation maps in \citep{Letzgus2022} showed the importance of the eyes for age prediction in infants, the central part of the face in adults (mainly eyes, nose, and mouth), and the whole face in seniors (with particular highlighting of wrinkles).

\subsubsection{Pan-tissue epigenetic and transcriptomic clocks}\label{sec:XAI:subsec:aggregated:clocks}

Age predictor models, or so-called clocks (in particular, epigenetic or transcriptomic ones), widely used for whole blood data, can be generalized to the data from different tissues. These models are called pan-tissue clocks. One of the most famous and widely used pan-tissue models is the epigenetic clock proposed by Horvath \citep{Horvath2013}. It is based on the ElasticNet linear model and is easily interpreted. The more recent AltumAge model presented in \citep{deLimaCamillo2022} is a DNN. SHAP and DeepPINK approaches are applied to its results to explain the predictions. Using them, higher age acceleration has been shown in brain samples from patients with autism and multiple sclerosis, in liver samples from patients with non-alcoholic fatty liver disease, and in pancreatic samples from patients with type 2 diabetes. Also, these XAI approaches helped to identify for many considered tissues the relationship between chromatin states and age predictions. 

Epigenetic modifications can influence gene expression, so transcriptomic models are related to epigenetic ones. In \citep{Shokhirev2021}, the tissue-specificity and sex-specificity of the proposed age predictor by transcriptomic data were revealed by constructing feature rankings for ensemble models. The authors show different aging signatures for retina, brain, blood, heart, and bone identified by constructing an age predictor for the transcriptomic data of the corresponding tissue. Genes specific only for males and for females were also identified.

The multimodal aging clock proposed in \citep{Urban2023} considers both methylation and transcriptomic data in a tissue-agnostic fashion. SHAP has been used to identify the most important genes that correspond to pathways associated with aging and age-associated diseases: tRNA processing in mitochondrion, amino acid transport across plasma membrane, suppression of apoptosis, vasopressin-like receptors, highly sodium permeable postsynaptic acetylcholine nicotinic receptors, cytosolic sulfonation of small molecules. The association of the obtained list of genes with drug targets for idiopathic pulmonary fibrosis, chronic obstructive pulmonary disease, Parkinson's disease, and heart failure was also shown.

\subsubsection{Age prediction with medical records data}\label{sec:XAI:subsec:aggregated:record}

Data discussed in this section for age prediction is called aggregated because it involves several body systems at once. Nevertheless, it represents one type of analysis in each case. However, there is a specific field that focuses on developing models for multiple heterogeneous tests, measures, and biomarkers that can be called really aggregated data. Typically, such data are electronic medical records that include medical history, anthropometric measures, laboratory tests (blood, urine, feces), physical examination results, and many other characteristics. Such data allows for a comprehensive assessment of a person's health status and takes many aspects into account when constructing an age estimate. Because these data include various measurements and medical history, they are usually presented in tabular form with categorical, ordinal, and continuous values. Feature rating construction for classical models \citep{Yang2022}, PFI \citep{Bae2021} and SHAP \citep{Bernard2023} are used to explain model predictions and to highlight the most important features among the overall set. 

In \citep{Bae2021} different types of models for biological age estimation are considered: linear and polynomial models, ensemble models and DNNs. PFI scores showed that levels of creatinine and aspartate aminotransferase, as well as waist circumference were among the most important features for all models. Sex, body mass index, lean body mass, lactate dehydrogenase and blood urea nitrogen levels were also among the important features for different models. It can be noted that not only blood parameters stand out among the important ones but also indexes characterizing the physique. In \citep{Yang2022} in the features ranking for an ensemble of classical models, the highest score was shown for the values of diastolic and systolic blood pressure, height, sex, and platelet content. It was also shown that a body shape index and waist-to-height ratio are associated with predicted age and represent health risk indicators. The physiological age model proposed in \citep{Bernard2023} using SHAP shows that parameters related to metabolism, nitrogen (uric metabolites and creatinine), carbon (glycohemoglobin, triglycerides, and glucose), and liver function (albumin, ALT, and GGT) contribute the most. PDPs show the effect of each variable on the predicted age by averaging the effect of all other variables. In particular, SHAP values are negative for low glycohemoglobin levels, with a sharp increase occurring for values in the 5-6\% window. The 5.4\% threshold characterizes the boundary for young subjects, progressing with age and increasing to 5.8\% for subjects over 50 years of age.

We should additionally mention models that are not age predictors per se, but are closely related to them. These are mortality predictor models. They predict the risk of mortality over a certain period of time, taking as an input many different tests as the age predictors described above. The most common XAI approach here is SHAP \citep{Qiu2022a, Qiu2022, Thorsen-Meyer2020}. In particular, in \cite{Qiu2022a}, red cell distribution width, serum albumin, arm circumference, platelet count, and serum chloride levels have the highest impact on 5-year mortality. Some of these parameters (red cell distribution width, serum albumin) are known markers of mortality risk, while others (platelet count, serum chloride levels) were discovered for the first time. The importance of red cell distribution width increases from 1-year to 10-year mortality, and the importance of serum albumin decreases. In \citep{Qiu2022} it was shown that cystatin C, smoking status, history for chronic and cancer diseases dominate to estimate all-cause mortality and neoplasm-cause mortality in women aged 65 years. An application of such mortality predictor models is described in \citep{Thorsen-Meyer2020}, which predicts 90-day mortality for patients in intensive care units. The highest influence in this case is age at admission, heart rate, surgical intervention, blood pressure, blood oxygen saturation, Glasgow Coma Scale, temperature, length of stay in hospital before admission to the intensive care unit. Such models are critically important for proper allocation of hospital workload and timely assessment of the risks of possible health deterioration.

\section{Conclusion}\label{sec:conclusion}

XAI approaches are a powerful tool for interpreting the results of complex models, which is especially important in risky applications, such as medical ones, as any wrong decision can not only harm human health, but even human life. Such models must not only solve problems effectively, but also allow experts to validate the results of their decisions. This allows to monitor the incorrect behavior of the model, to correct it and, as a result, to trust such systems. Generally speaking, interpretability is an aspect of reliability, a requirement for all modern AI systems, which are usually black boxes with a non-transparent decision-making process. In this review, we describe common XAI approaches and their classification, and we provide a detailed analysis of their application to age prediction models.

% Figure environment removed

The use of XAI approaches in the biological age prediction task yields various benefits, shown schematically in Figure \ref{fig:benefit}. One of the most significant and obvious benefits is highlighting the most important biomarkers of aging and age-associated diseases. The ability to detect the most important markers helps to identify age-associated parameters for all the considered body systems. Related to this, another achievement of XAI is the ability to detect ‘risk factors’ - biomarkers that require special attention during aging. Being able to spot ‘red flags’ before serious symptoms occur can help significantly improve people's quality of life. This leads to the next advantage of XAI - a personalized approach. Most methods provide explanations for each specific sample, identifying the individual signs that had the highest impact on the final prediction for that particular person. This will allow more precise identification of personalized schemes not only for the treatment of age-associated diseases but also for preventive measures to slow down the progression of aging signs. For some age-associated diseases, XAI methods allow us to identify targets for their early detection and treatment. Another important aspect of XAI for public health can be to support the organization of intensive care units. Determining the most important features in predicting mortality will allow the proper allocation of healthcare facility resources and provide the most efficient care. Among the important benefits of XAI in the age prediction tasks are the identification of the characteristics of human development at different stages, the identification of previously unknown dependencies between different body systems in the aging process, and the comparison of the sensitivity of different approaches to different aspects of aging.

Overall, we can note that XAI approaches in the age prediction task expand the horizons of our knowledge about human aging, develop a personalized approach in medicine, and necessarily should be the standard for all AI systems in healthcare.

\section*{Abbreviations}

AI - Artificial Intelligence; ALE - Accumulated Local Effects; ALT - ALanine Transaminase; CAM - Class Activation Mapping; CNN - Convolutional Neural Network; CSF - CerebroSpinal Fluid; DeepLIFT - Deep Learning Important FeaTures; DeepPINK - Deep feature selection using Paired-Input Nonlinear Knockoffs; DL - Deep Learning; DNA - DeoxyriboNucleic Acid; DNN - Deep Neural Network; ECG - Electrocardiogram; EEG - Electroencephalogram; GBDT - Gradient-Boosted Decision Tree; GGT - Gamma-Glutamyl Transferase; Grad-CAM - Gradient-weighted Class Activation Mapping; HIV - Human Immunodeficiency Virus; LIME - Local Interpretable Model-agnostic Explanations; ML - Machine Learning; MRI - Magnetic Resonance Imaging; OCT - Optical Coherence Tomography; PDP - Partial Dependence Plot; PFI - Permutation Feature Importance; PPG - PhotoPlethysmoGram; REM - Rapid Eye Movement; RNA - RiboNucleic Acid; ROI - Region Of Interest; SHAP - Shapley Additive exPlanations; VIM - Variable Importance Measure; XAI - eXplainable Artificial Intelligence.


\bibliographystyle{plainnat}
\bibliography{arxiv_preprint}  %%% Uncomment this line and comment out the ``thebibliography'' section below to use the external .bib file (using bibtex) .


%%% Uncomment this section and comment out the \bibliography{references} line above to use inline references.
% \begin{thebibliography}{1}

% 	\bibitem{kour2014real}
% 	George Kour and Raid Saabne.
% 	\newblock Real-time segmentation of on-line handwritten arabic script.
% 	\newblock In {\em Frontiers in Handwriting Recognition (ICFHR), 2014 14th
% 			International Conference on}, pages 417--422. IEEE, 2014.

% 	\bibitem{kour2014fast}
% 	George Kour and Raid Saabne.
% 	\newblock Fast classification of handwritten on-line arabic characters.
% 	\newblock In {\em Soft Computing and Pattern Recognition (SoCPaR), 2014 6th
% 			International Conference of}, pages 312--318. IEEE, 2014.

% 	\bibitem{keshet2016prediction}
% 	Keshet, Renato, Alina Maor, and George Kour.
% 	\newblock Prediction-Based, Prioritized Market-Share Insight Extraction.
% 	\newblock In {\em Advanced Data Mining and Applications (ADMA), 2016 12th International 
%                       Conference of}, pages 81--94,2016.

% \end{thebibliography}


\end{document}
