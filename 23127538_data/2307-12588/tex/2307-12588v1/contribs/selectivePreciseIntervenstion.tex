

\bbot\ is equipped with a novel weeding tool design enabling high precision plant-level field interventions. 
It consists of a set of replicated linear actuators and is controlled via the intervention controller unit consisting of several components which are elaborated in the following sections.
%The weeding implement is controlled via the intervention controller unit consisting of several components which are elaborated in the following sections.
We briefly explain the conceptual design of the weeding tool, its requirements and operation assumptions.
Then, we introduce our method for managing targets in the work-space of the weeding tools.
Finally, we elaborate on path planning strategies used for controlling intervention heads in action.

% To properly deploy proposed weeding implement one important challenges is how to efficiently plan paths for intervention heads, such that we minimize the number of not managed weeds in a weeding scenario.

% We define the weeding scenario as below:
% \RNum{1} weeds get detected, classified an tracked within the view-able area of down facing camera $C_{detect}$ via monitoring system explained in~\secref{subsec:cropWeedMonitoring}.
% \RNum{2} the track-lets of each detected instance is being passed to a proper weeding actuator weeding-tool manager node.
% \RNum{3} each intervention path planner plans an optimal path and action time relative to meta information of each track-let like: weed type, size and category as it is explained in ??? in more details. 
% \RNum{4} goal target commands gets published through ROS, sending nozzle on actuator $i$ to requested target $j$.
% \RNum{5} valve of nozzles get activated based on heads pose relative to target weed (stem or bbox?)


\subsection{Plant-level Treatment In Field}
\label{subsec:plantLevelTreatment}

%We assume the robot moves along a crop-row with constant speed $\gamma$ so the kinematics of weeds at time $t$ w.r.t the weeding implement on the robot could be visualized as~\figref{fig:kinematicsWeeding}.
%Consequently the intervention is time-critical and must respect the spatial ordering of the weeds.
%Similar to~\cite{bawden2017robot} we assume weeds are uniformly distributed in the field with density $\lambda$ weeds/$m^2$. 
%Hence, using a Poisson process we can explain the distance between the weeding implement and individual weeds by accounting for the arrival rate of $\eta =\lambda \times \Pi$.
%We use the motion along the $x$-axis of the robot frame $\mathcal{F}_R$, to explain the weeds interval distance ($\delta_x$).

We assume the robot moves along a crop-row with constant speed $\gamma$.
Consequently, intervention is time-critical and must respect the spatial ordering of the weeds.
There is a constant gap ($\Gamma$) between the tools and the area sensed by the camera ($C_{detect}$).
Similar to~\cite{bawden2017robot} we assume weeds are uniformly distributed in the field with density $\lambda$ weeds/$m^2$. 
Hence, using a Poisson process we can explain the distance between the weeding implement and individual weeds by accounting for the arrival rate of $\eta =\lambda \times \Pi$.
We use the motion along the $x$-axis of the robot frame $\mathcal{F}_R$ to explain the weeds interval distance ($\delta_x$), visualized in ~\figref{fig:kinematicsWeeding}.
This can be shown using the following probability density function,
% 
\begin{equation}
    \label{eq:weedsIntervaldistance}
    f(\delta_x) = \lambda \Pi e^{-\lambda \Pi \delta_x},
\end{equation}
% 
also the location of weeds on the $y$ axis can be represented via a uniformly distributed random variable $y$ as,
% 
\begin{equation}
    \label{eq:yPDF}
	f(y) =\left\{\begin{array}{cc}
	    \frac{1}{\Pi} & for  \ \ 0 \leq y \leq \Pi \\
	     0 & otherwise
	\end{array}\right..
\end{equation}
% 
To engage the $i$-th intervention head with the $j$-th weed it has to traverse,
% 
\begin{equation}
    \label{eq:yDistance}
    \delta^{ij}_{y} = | \textit{h}_{i} - \textit{n}_{j} |, 
    \text{ where }  0 \leq \delta^{ij}_{y} \leq \Pi ,
\end{equation}
%
where $\textit{h}_{i}$ is the current position of $i$-th intervention head and the $\textit{n}_{j}$ denotes to the position of the $j$-th weed.
Therefore, the probability of visiting the $j$-th weed with $i$-th intervention head can be calculated with,
% 
\begin{equation}
    \label{eq:weedingProb}
    \textit{P}_{ij} = \textit{P}\left( \dfrac{\gamma}{\vartheta}  <   \dfrac{\delta_{x}}{\delta_{y}} \right),
\end{equation}
%
where $\vartheta$ denotes to the maximum velocity of linear axes.
We assume all targets are detected by the time they reach the bottom edge of the camera's viewable area.
% \TODO{MAH confusing: We suppose targets get detected by the time passing the center of viewable area of camera $C_{detect}$.}
%As~\figref{fig:kinematicsWeeding} depicts there is a gap with a length of $\Gamma$ between the work-space of the weeding implement and the bottom edge of the viewable area of the front camera $C_{detect}$. % in front of robot.
%Finally, this distance provides extra time for the planning system, this is defined by $\tau_e=\Gamma/\gamma$ and depends on the linear velocity of robot and length of region~$\Gamma$.
%% \TODO{MAH confusing: Finally, this provides extra time for detection and planning which we refer to it by $\tau_e=\Gamma/\gamma$ that depends on the linear velocity of robot and length of region~$\Gamma$.}
% Let $\kappa = \tau_v + \tau_i$ represent the intervention head engagement time for our spot spray systems to be constant, 
% where $\tau_v$ shows the constant operation time of the nozzle valve (On-Off time) and $\tau_i$ denotes the engagement time with weed $i$ in milliseconds.
%
% Figure environment removed






\subsection{Target-Space Management}
\label{subsec:targetSpaceManagement}

Planning the motion of each intervention head must be done prior to targets entering the weeding tool work-space. % of weeding tool. 
In the proposed workflow, the intervention controller node receives the detected targets at time $t + \tau_d$ where $\tau_d$ is the time required for detection in the monitoring node.
The monitoring node provides plant specific information like: plant category, pixel-wise segmentation, estimated area, and the bounding box.
Furthermore, we estimate plant centers based on provided bounding box in the scene.  % around the detection.
%Then, the aim would be to find the best motion plans for all interventions heads to minimize the targets which are not being visited (sprayed).
%This information is then used in the target-space management step to assign targets between intervention heads.
This information is then used in the target-space management step to assign targets to the intervention heads.
Based on this the next step finds the best motion plan for each intervention head by maximizing the number of targets that are visited (sprayed).
%Based on this the next step finds the best motion plan for each intervention head that maximizes the number of targets that are visited (sprayed).

Let $\mathcal{H}$ denote the number of independent controllable intervention heads and $\mathcal{N}$ be the number of targets that appear under the robot. %to be the number of targets appeared in the underneath the robot. %, the main goal is to visit all the targets with at least one of the intervention heads as they pass the work-space of weeding tool without the need for stopping the robot.
%By considering the fact that robot is moving forward and, we refrain from moving backwards, the intervention is time-critical and must be in spacial order.
We use a uni-directional constrained node-graph to model the targets-space.

To obtain the global spatial order of targets in a segment we use the $\delta_x$ of each weed (see \figref{fig:kinematicsWeeding}).
In~\figref{fig:nodeGraph}(a), each node (circle) shows a weed along with the connecting path between nodes $j$ to $k$ represented with a uni-directional link (arrow) $l_{jk}$.
The link $l_{jk}$ exists if, node $j$ geometrically is located after node $k$ in the 3D world frame  $\mathcal{F}_w$ in the direction of motion.
Furthermore, the link $l_{jk}$ is associated with an inter-weed cost $\varrho_{jk}$ based on the distance of nodes $j$ and $k$ and a property denoting motion probability of $\textit{P}_{jk}$ based on~\eqref{eq:weedingProb}.
We calculate inter-weed costs using the top-right of the cost-matrix~$\mathcal{G}_{\mathcal{N} \times \mathcal{N}}$ (to respect the weeds spatial order).


% \begin{equation}
%     \label{eq:costMatrix} 
%     \varrho_{jk} = (n_j - n_k)^2
% \end{equation}
% 
There are $\mathcal{H}$ independent interventions heads and so multiple plans which can lead to the same number of targets being visited (sprayed).
%As the weeding tool contains $M$ independently controllable interventions heads, it provides this flexibility to plan a variety of engagement routes by considering different method of target assignments.
%Considering sets of weeds in any given work-space motivates a multi-query approach  to the problem.
To solve this problem, we consider the weeds as a sets of targets detected in one location, this motivates us to assign intervention targets to the $\mathcal{H}$ heads as either distance-based or work-space division-based assignments.
% To solve this problem, we consider weeds to be presented as a set which motivates us to consider assigning targets to the $\mathcal{H}$ intervention heads as either a distance-based assignment or a work-space division-based assignment approach.
%Prior to plan engagement routes, we use a high level planner to distribute targets between different intervention heads while, making sure that all the targets will be at least visited once. 
%we suggest to represent the work-space in two different ways:

\begin{enumerate}
    \item \textbf{\textit{Distance-based Target Assignment (D)}:}
        In this approach, target $j$ gets assigned to the laterally closest intervention head along the sliding direction ($y$-axis). 
        This means, selected intervention head $i$ has the least motion required to reach the weed $j$.
        The lateral distance between heads and weeds are defined based on 2D euclidean distance between projection of intervention head's position on ground plane and weed positions on same plane w.r.t the $\mathcal{F}_w$ frame.
        
    \item \textbf{\textit{Static Work-space Division-based Target Assignment (SD)}:}
        In this method, we divide the work-space of weeding tool to $\mathcal{H}$ sub-sections of width $\Pi/\mathcal{H}$ meters.
        Hence, each intervention head is only responsible for engaging with weeds laying within it's sub-work-space as shown in~\figref{fig:nodeGraph}(b)-top.
        
    \item \textbf{\textit{Dynamic Work-space Division-based Target Assignment (DD)}:}
        % \TODO{MAH let's talk about this, it still confuses me reading this and then looking at the figure.}
        In this model, for each new set of detected weeds we first determine the minimum region of intervention defined by $y_{min}$ and $y_{max}$ (see Fig.~\ref{fig:nodeGraph}~(b)). 
        The minimum region of intervention is then divided into $\mathcal{H}$ equal sub-regions.
        This process assists in optimizing the planning for weed engagement by potentially reducing the area any one tool has to cover.
        %In this model, for each new segment we define an intervention necessary region by finding the spared of weeds positions along $y$-axis underneath the robot. 
        %Then we divide this region into $\mathcal{H}$ equal sub-regions for performing weeding engagements as shown in~\figref{fig:nodeGraph}(b)-bottom.
        %This technique ensures more proper load balancing over interventions heads in case of facing portions of the fields with non-uniform weed distribution.
\end{enumerate}
% 
%We note that the above approaches have been derived to demonstrate the potential of our proposed system and we believe future work can derive more advanced approaches to provide even better coverage.

% Figure environment removed



\subsection{Intervention Heads Route Planning}
\label{subsec:routePlanning}

\begin{comment}
At this point we propose that a subset of targets at time $t + \tau_d + \tau_m$ are assigned to intervention head $i$, where $\tau_m$ is the time required for segment management.
The problem which must be addressed is how to plan $\mathcal{H}$ independent efficient routes using the the prior knowledge of an intervention head's position, robot linear speed as well as the limits of speed and acceleration of linear axes.
%The problem which must be addressed is: with the prior knowledge of an intervention head's position, robot linear speed and limits of speed and acceleration of linear axes, plan $\mathcal{H}$ independent efficient routes to guide intervention heads.
We refer to the time required for planning as $\tau_p$.
The planned routes must guide intervention heads through all their assigned targets while minimizing the chance of missing any target.
\end{comment}

%At this point we propose that a subset of targets at time $t + \tau_d + \tau_m$ are assigned to intervention head $i$, where $\tau_m$ is the time required for segment management.
Here we address how to plan $\mathcal{H}$ independent efficient routes.
The planned routes must guide intervention heads through all their assigned targets while minimizing the chance of missing any target.
This has to take into account the the prior knowledge of an intervention head's position, robot linear speed as well as the limits of speed and acceleration of linear axes.
%The problem which must be addressed is: with the prior knowledge of an intervention head's position, robot linear speed and limits of speed and acceleration of linear axes, plan $\mathcal{H}$ independent efficient routes to guide intervention heads.
%We refer to the time required for planning as $\tau_p$.

The planning approach generates~$m$ potential trajectories~$\Vec{\mathbf{T}}=[\Vec{T}_0,\dots,\Vec{T}_m]$ for each intervention head.
Each trajectory $\Vec{T}_i$ is an ordered list of length $q$ consisting of weed positions which can be visited.
%included in the trajectory with length $q$. }
%\TODO{CM to AA: You wrote $\Vec{T}_m$ but do you mean $\Vec{T}_{\mathcal{H}}$, No this is actually $\Vec{T}_m$ where $m$ is the number of predicted possible trajectories for each intervention head}.
%To obtain possible trajectories, we use the following approaches:
To obtain $\Vec{\mathbf{T}}$ we use the following approaches:
% 
\begin{enumerate}  
    \item \textbf{\textit{Brute-Force}}:
    In this case we compute all possible routes by finding the permutation of all nodes in the graph (without considering the direction of links).
    Then the routes with the lowest cost and maximum success rate will be selected from all predicted routes. % could be drawn out of the queue of all predicted routes.
    \item \textbf{\textit{Open Loop Traveling Salesman Planning}}:
    This approach, termed OTSP, is a variant of the classic travelling salesman problem where the agent must visit all nodes of a graph once without making a loop back (Hamilton loop) to the start node~\cite{chieng2014performance}.
    %The OTSP is a variant of classic TSP where the agent must visit all nodes of a graph once without making a loop back (Hamilton loop) to the start node~\cite{chieng2014performance}.
    %The problem which here is being solved is also 
    To solve this we use an approach similar to $n$OTSP where the agent only needs to visit $n$ nodes in the graph, however, in our problem setting we aim to maximize the number of visited nodes while considering other important criteria like cost and success rate.
    We use our constrained uni-directional node-graph representation as a base for solving $n$OTSP using dynamic programming.
\end{enumerate}

The optimal trajectory for each intervention head is obtained by considering two criteria: number of nodes successfully visited and the total movement of the predicted trajectory.
In every trajectory, we calculate the number of nodes that satisfy~\eqref{eq:weedingProb} to determine if a node can be successfully visited.
This gives us an updated set $\Vec{\mathbf{T}'}$ which only consists of nodes in the trajectories which are feasible.
From this updated set $\Vec{\mathbf{T}'}$ we then calculate the movement cost-matrix~$\mathcal{G}$,
% 
\begin{equation}
    \label{eq:trajectoryCost} 
    \mathcal{G}(\Vec{T}'_i) = \sum_{j=0}^{q-1} (n_j - n_{j+1})^2.
\end{equation}
%
After this process, the trajectory from $\Vec{\mathbf{T}'}$ with the maximum number of successfully vised nodes is passed to the intervention controller.
In the case multiple trajectories successfully visit the same number of nodes, the trajectory which also minimizes the movement cost will be passed to the intervention controller.







