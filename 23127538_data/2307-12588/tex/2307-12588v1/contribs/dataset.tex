
We used BonnBot-I to prepare two different specifically designed datasets suited fro precision agricultural application. The datasets are captured at sugar-beet and corn fields in campus Klein-Altendorf (CKA) of the University of Bonn. 
Both datasets consist of temporally sparse annotations, this means the annotation of one image does not overlap with another images.
We have introduced \textit{SB20} dataset in \cite{ahmadi2021virtual}, and to adapt it to precision agricultural setting we augment its annotation with adding stem location of plants.
% Figure environment removed 
Here, we introduce \textit{CN20} dataset collected in a corn field at CKA.
The data was captured using an Intel RealSense D435i sensor (used in initial BonnBot-I sensor configuration) with a nadir view of the ground in front of the robot and resolution of $1280\times720$ with a frame-rate of $15Hz$. 
The data covers six rows of crop providing a large distribution of crops sizes due to non-homogeneous growth stage and several weed types listed in table~\tabref{tab:cn20Instances}. 

The dataset comes with RGB-D frames of crops and nine different categories of weeds containing totally 2566 and 1261 instances of crop and weeds respectively. 
The annotations include instance based pixel-wise segmentation, bounding boxes and stem locations of each instance in Coco format.
Similar to \textit{SB20} the image data for each timestamp is augmented with position, orientation, velocity information.
The proposed data structures used to generate real temporal sequences where only the final frame in the sequence is labelled, as presented in \cite{ahmadi2021virtual}, reviving temporal relations between annotated and not-annotated frames.

Furthermore, with post-processing masks and using depth information the growth-stage of individual crops is calculable.
To train out DNN-based monitoring system we divided the images of dataset in to sets of training, validation and evaluation of size \TODO{we must add final size of sets!} $149$, $39$ and $44$, respectively.
\figref{fig:datasetSample} shows an example annotated image of \textit{CN20} dataset.



