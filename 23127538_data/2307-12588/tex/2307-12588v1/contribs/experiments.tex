


%In the following we demonstrate result of several experiments accomplished in simulation and real environments to show the versatility and robustness of \bbot\ with it's novel weeding implement and our monitoring and intervention approach. 
% For all experiments we set the robot linear speed to a constant $\gamma=0.5m/s$.
% We deploy intervention heads with their maximum acceleration ans speed which are  $10m/s^2$ and $\vartheta=5m/s$, respectively.
% The overall intervention accuracy is dependent on nozzle heads and nozzle height.


% and carries $\mathcal{H}=4$ interventions heads which are independently controlled and each can engage with only one weed at the-time.


\subsection{Field Monitoring} 
\label{subsec:cropWeedMonitoring}
% and Visualization}

We evaluate the performance improvements to field monitoring techniques when using a range of localization sensors.
%Here, we evaluate the performance of our field monitoring technique explained in \secref{subsec:cropWeedMonitoring}.
%We use Mask-RCNN as our based method for instance-based semantic segmentation with conjunction with reprojection mathod explained in~\secref{subsec:intraCameraTracking}.
Our previous work~\cite{halstead2021crop} is enhanced by fusing the wheel odometry and GNSS, which are available on the platform, using EKF. % incorporating multiple localization sensors, available of a weeding robot, using EKF. % fusion, yielding blow centimeter accuracy.
%This approach is explained in~\cite{halstead2021crop} which we are able increase it accuracy using output of EKF fusion, yielding blow centimeter accuracy.
With this information we are able to improve the average normalized absolute error (NAE) across the three CN20 evaluation rows using our dynamic radius (DR) tracker from $8.3\%$ to $3.5\%$, more than halving the NAE.
This is a considerable improvement as the absolute error of the field tracking system is less than $5\%$ NAE.
We attribute this performance improvement to better localization information supplied through multiple localization sensors and sensor fusion.
%This also highlights that while wheel odometry alone is accurate it also has limitations, such as wheel slippage.
To fully appreciate the impact of sensor fusion using EKF we also explore the improvements for a pixel-wise matching tracker (IoU).
%, intersection-over-union instead of the spatial matching DR.
This is much simpler tracking approach, although susceptible to leaf movement due to wind, is greatly improvement with the NAE improving from $71.7\%$ to $25.7\%$ with the incorporation of multiple localization sensors and sensor fusion.
%fused sensor information as the NAE  to approach when compared to DR and is less robust to weather interfering with the leafy structure of the corn plant, using this criteria we are able to improve CN20 tracking from $71.7\%$ to $25.7\%$.
%fused sensor information as the NAE  to approach when compared to DR and is less robust to weather interfering with the leafy structure of the corn plant, using this criteria we are able to improve CN20 tracking from $71.7\%$ to $25.7\%$.
%approach when compared to DR and is less robust to weather interfering with the leafy structure of the corn plant, using this criteria we are able to improve CN20 tracking from $71.7\%$ to $25.7\%$.
%This is a significant boost in performance for a difficult task just from this fusion of sensors.
Finally, in our experiments we were consistently able to achieve $R^2$ scores greater than $0.95$, showing that our tracker was linearly correlated with the ground truth.
These results demonstrate the importance of multiple localization sensors and the considerable positive impact they can have on existing field monitoring techniques.

% To fully appreciate the impact of incorporating extra sensors, the performance of an even simpler tracking system, the intersection-over-union tracker, is improved from $71.7\%$ to $25.7\%$.
% In all experiments we were able to achieve $R^2$ scores greater than $0.95$.

%a simplistic intersection over union tracker from an error of $71.5\%$ to $25.7\%$.
%This experiment is we demonstrate results of the approach on \textit{SB20} dataset and directly compare it to~\cite{halstead2021crop}, additionally we run the same algorithm on \textit{CB20} introduced dataset and compare the results on similar settings.
%\begin{itemize}
%    \item stem location estimation or growth-stage estimation can be added too.
%    \item Performance of Weeding with and without VO+EKF
%\end{itemize}
%+ \textbf{orthomosaic and 3D map outputs !!}

\begin{comment}
\begin{table}[!h]
    \caption{Results placeholder for tracking. \TODO{CM to ALL: I remove the $R^{2}$ as it doesn't say anything and consider removing IoU, however, IoU still says that it helps a lot so it might be worth keeping.}}
    \label{tab:trking}
    \centering
    \begin{tabular}{l | cc | cc}
    \toprule
    \multirow{2}{*}{\textbf{Test}} 
    & \multicolumn{2}{c}{\textbf{Wheel Odometry}} 
    & \multicolumn{2}{c}{\textbf{Wheel Odometry + GPS}}\\
    & $R^2$ & NAE & $R^2$ & NAE \\
    \hline
	\midrule
    IoU & 0.997 & 0.717 & 0.976 & 0.257\\
        DR  & 0.993 & 0.083 & 0.962 & 0.035\\
    \hline
    \end{tabular}
\end{table}

What do we see:
\begin{itemize}
    \item Rsquared values are similar showing mostly that three samples isn't really enough to give a robust score for this metric.
    \item DR does a significantly better job:
    \begin{itemize}
        \item Corn leaves are delicate in shape; and
        \item Corn leaves move easier due to wind; and
        \item IoU requires a highly accurate pixel-wise matching scheme to recognize the same object; and
        \item DR only matches on spatial similarities not pixel locations; and
        \item DR still has issues in that it matches in 360 degrees, particularly with larger skips as shown in ~\cite{HalsteadFrontiers}.
        \item I would find images but I don't think we have the space to be honest.
    \end{itemize}
    \item Talk about the different odometry techniques when the arrive.
\end{itemize}
\end{comment}

\subsection{Weeding Planning Performance} 
% Performance of weeding behaviour planner for various cases Real predictions and simulation with certain weed density

% \TODO{CM to ALL: I think this is trying to say that you get similar performance when comparing nOTSP to Brute Force but that nOTSP is heaps faster. I see the timing diagram but is there a set of results with the number of targets hit as well? I've try to put some suggestions for this paragraph below as update (blue) text.}
To evaluate the performance of the weeding tool planning explained in~\secref{subsec:selectivePreciseIntervenstion}, we use our native python simulator (\secref{subsec:simulation}) specifically designed for this purpose. % elaborated in~\secref{subsec:simulation}. 
We investigate the performance of our approach using two different types of crop-row models: simulated rows of crops/weeds and real plant distributions predicted from our monitoring systems for both SB20 and CN20 datasets evaluation rows.
We initially evaluated the performance of the Brute-Force planner against the $n$OTSP planner and found the results where similar in trivial cases with a limited number of targets ($\mathcal{N}\leq10$).
% \TODO{What is ``limited targets''? This is a weak statement without something to back it up. Can we direct them to Figure 8 and point to values less than 5? Or do we have some other metric that proves this statement?}
% Initially we evaluate the difference in performance of Brute-Force planner with $n$OTSP which turned our to be pretty similar for cases with limit number of targets in segments.
% % Figure environment removed


% The benefits of the $n$OTSP approach are further shown in~Fig.~\ref{fig:runTime}.
The computational expense of the Brute-Force method increases almost exponentially when there are more than four targets as it has a run-time complexity of $O(n!)$ in comparison with $O(n^22^n)$ of graph based $n$OTSP.
This is a significant problem for weeding applications where real-time performance is important.
In the case where we see only ten weeds in the planning region the Brute-Force approach requires $3.7$s compared to $n$OTSP which requires only $266\mu$s.
This is a prohibitive quality of the Brute-Force approach and for our two weeding experiments we employ the $n$OTSP technique.

% In our experiments we make use of an $n$OTSP algorithm.
% This is because in initial experiments it was found that the Brute-Force approach quickly became too computationally expensive to consider. 
% In Fig.~\ref{fig:runTime}, it can be seen that the computation time to plan actions using a Brute Force quickly become prohibitively expensive ($\sim100$mseconds) once there are 8 weeds in any one segment. Therefore, for the remaining of our experiments we employ an $n$OTSP algorithm and we then explore the potential of the three different target assignment methods.

\subsubsection{Planning on Simulated Crop-Rows}

In our first experiment we evaluate the performance of different weed densities and a different number of linear axes.
This evaluation uses the simulation environment introduced in Sec.~\ref{subsec:simulation}, allowing us to control the experimental parameters.
For our platform hyper-parameters we keep a constant robot speed of $\gamma=0.5$ and set the velocity of the linear actuators to $\vartheta=5$.
The field parameters are set to $3$ crops-rows in a single lane with a length of $20$m.
To fully analyse the performance of our approach we vary the weed density such that $\lambda\,=\left[3, 5, 10, 20, 40\right]$ representing the weeds per $m^2$.
Finally, to outline the benefit of having multiple linear axes we show the performance for $\mathcal{H}\,=\left[1,2,4,8\right]$.

% In the first set of experiments we use simulated fields assuming uniform weed distributions, see Sec.~\ref{subsec:simulation}.
% Additionally, we evaluate a single crop lane of width $1.3m$ with $3$ crop-rows and of length $20m$.
% We keep fixed the robot speed $\gamma=0.5$ and velocity of the linear actuators ($\vartheta=5$).
% The weed density is then varied by setting $\lambda\,=\left[3, 5, 10, 20, 40\right]$, which represents the number of weeds per $m^{2}$, and the number of actuators is also varied $\mathcal{H}\,=\left[1,2,4,8\right]$.

The results for this simulated experiment are summarized in Fig.~\ref{fig:simulationPerformace}, where we provide the comparison of weed density to the percentage of missed targets.
From this figure it is evident that increasing the number of linear axes has an obvious impact on results, with $8$ axes performing better than all others.
We see from this that even with a distribution of $40$ weeds (the hardest case) the worst performing $8$ axes system, Distance-Based, achieves a loss of $\sim{15\%}$.

% The results on uniform weed distributions in simulation are summarized in Fig.~\ref{fig:simulationPerformace}.
% \TODO{MAH: You need to describe this better, I have no idea what is what based on the figure.


Overall, the Distance-Based approach routinely performs worse as the weed density increases, this is particularly evident as the number of heads is increased.
This will be further evaluated in the next evaluation, but as the distribution grows the intervention heads need to travel further to meet the demands of the planner, this movement can negatively impact the capacity for the intervention head to reach the next weed.
Finally, the two division based methods appear to perform at a commensurate level across all weed densities in our simulations.
We attribute this to the wide distribution of weeds negating the impact of the dynamic approach, meaning, weeds generally appear across the entire lane rather than concentrated in a specific region.
Overall, this shows the validity of our planning methods for weed intervention in a uniformly distributed pattern.

% It can be seen that the worst performing system is the distance-based target assignment which is outperformed by the two other methods, especially as the number of intervention heads $\mathcal{H}$ increases. 
% Also, the difference between static work-space division and dynamic work-space division is negligible throughout all of the setups in simulation.

\begin{comment}
The critical parameter effecting the performance of intervention controller include: weed density ($\lambda$), number of interventions heads ($\mathcal{H}$), robot speed ($\gamma$), and velocity of linear actuators ($\vartheta$).
To evaluate weeding system regarding its performance in dealing with different weed densities of fields, we generate $5$ different row-crop field models with various weed densities ($\lambda\,=\,3, 5, 10, 20, 40$).
All models contain one lane of crop with width of $1.3m$ and $3$ crop-rows, with length of $20m$.
By running robot model on the these field models, we collected rate of missed weeds (`loss`) in each pass.
During all trials, we kept $\gamma$, $\vartheta$ static to ensure gaining comparably outlines.

\figref{fig:simulationPerformace} summarizes the performance of the weeding system in cases which~$\mathcal{H}=1, 2, 4, 8$ where outlines of $n$OTSP planner method in three different modes of target assignments are plotted.
In general, distance-based target assignment is outperformed by the two other methods, specially for larger number of target, and the difference seems to get larger even with higher number of used intervention heads.
Nevertheless, for segments with few targets all methods have small loss.
the difference between static work-space division and dynamic work-space division is negligible throughout the all setups.

\end{comment}

% 
% Figure environment removed
% 

\subsubsection{Planning on Real Crop-Rows}


% \TODO{We don't use a sliding window, we use manually crop images from the lane based on the tracking output~\secref{}.}

To further evaluate the performance of our three planning approaches, we use real weed distributions captured in the fields within weeding simulator.
% We perform further evaluations on our three approaches, however, in this case we simulate the weed data based on real distributions captured in the field.
%The data is abstracted in a cropped manner (i.e. non-overlapping images), we do not use a rolling window based approach over the lane. 
This information is obtained from the crop monitoring approach outlined in~\secref{subsec:fieldMonitoring} and aggregated into a simulated row.
% for weeding.
We perform this on the evaluation rows for both the SB20 and CN20 datasets where we have three different weed distributions: low (SB20-S2), medium (CN20), and high (SB20-S1).

% We further evaluated our approaches in simulation but using real-world weed distributions.
% We obtained these distribution maps using our field monitoring approach for both sugar-beet(\textit{SB20}) and corn(\textit{CN20}) fields.
% These fields contain three different weed distributions: low (SB20-S2), medium (CN20) and high (SB20-S1).


The results of our three planning approaches are displayed in~\tabref{tab:realPlanning}, in this experiment we only use four intervention heads as this accurately evaluates \bbot\ performance.
\tabref{tab:realPlanning} displays two metrics, first, the percentage of missed weeds, and second, the mean and standard deviation of the distance moved by the axes.

\begin{table}[!b]
    \vspace{-6mm}
	\centering
	\caption{The rate of Loss ($\%$) and average traveled distance ($m$) of interventions heads in real-world weeding scenarios.}
	\resizebox{\columnwidth}{!}{%
	\begin{tabular}{l | cc|cc|cc}
	\toprule
	 & \multicolumn{2}{c|}{\textbf{Dist.-Based (D)}}  & \multicolumn{2}{c|}{\textbf{
	 Sub-Div. (SD)}} & \multicolumn{2}{c}{\textbf{Dyn.-Div. (DD)}} \\
	 & (\%) & (m) & (\%) & (m) & (\%) & (m) \\\hline	 
	 \midrule
	CN20        & \textbf{0.0}  & 2.7$\pm$0.2    & 4.3    & 2.7$\pm$2.8   & 3.4          & 4.0$\pm$3.9  \\
    SB20-S1     & \textbf{11.9} & 10.1$\pm${0.9} & 19.8   & 4.7$\pm$3.5   & 13.5         & 5.0$\pm$1.8  \\
    SB20-S2     & \textbf{0.0}  & 1.4$\pm${0.2}  & 2.3    & 1.5$\pm${0.6} & \textbf{0.0} & 1.0$\pm${0.8}\\
    \bottomrule
    \end{tabular}}
    % \vspace{-4mm}
	\label{tab:realPlanning}
\end{table}


For the percentage of missed weeds we see that the high (SB20-S1) level of weeds is the most difficult to intervene on, this is somewhat expected due to the heavy distribution of weeds.
However, the simple Distance-Based approach outperforms the other two approaches for this distribution which is similar to that seen in~\figref{fig:simulationPerformace} for a density of $10$ per $m^2$.
The poorer performance of the two division based approaches can be attributed to the distribution of the weeds in a real crop-row.
%\TODO{I think we need to talk about this if you had time: MAH, why what do you think it we could add?:
While, we performed the Chi-squared test on our rows and found no evidence to suggest they were not uniform we still assert that some sections of the row do not conform to the uniform property, increasing complexity.
This performance is mirrored through the other rows, where the Distance-Based approach achieves higher scores.
However, once we reduce the weed density (SB20-S2, CN20) we are able to achieve a percentage of missed weeds close to zero for all approaches.


Our final evaluation is based on the movement requirements of the different planning approaches, where a value closer to zero is desired.
While, Distance-Based achieved the best performance for the percentage of missed weeds we see the negative aspect of this approach here.
Both the division based approaches are able achieve considerably better results, in fact for SB20-S1 both division based approaches move half as much as the Distance-Based.
Even with considerably more standard deviation these two approaches outline the benefit of sub-sampling the work-space for more efficient tool usage.
We believe that in the future this work can be used to provide baseline information about robotic path planning for weeding applications.
This includes potential improvements by jointly minimizing the total distance travelled (of the intervention heads) and the loss.

% The results in \tabref{tab:realPlanning} outline the rate of loss and average traveled distance of intervention heads $\mathcal{H}=4$ for real-world weed distributions.
% Interestingly, it can be seen that now the distance-based target assignment performs the best by achieving the lowest loss (\%).
% Additionally, the dynamic work-space division considerably outperforms the static work-space division.
% Furthermore, although the distance-based target assignment approach achieves the lowest loss the dynamic work-space division obtains a similar (but slightly higher loss) with considerably less distance travelled.

% The above results highlight two points for future work in this area.
% First, we believe this is because the assumption of uniform weed distribution does not hold in real-world fields and indicates that any further development should be conducted on realistic distribution of the weed population.
% Second, any future work should consider how to jointly minimize the total distance travelled (of the intervention heads) and the loss.

% (i) that these are results for initial systems that show great potential, (ii) that distance-based is clearly giving better performance for coverage but that (iii) any future work should consider how to minimize motion because a dynamic sub-division has similar performance to the distance-based approach but requires a lot less movement of the tool.

% \subsubsection{Ablation-Study on Different Weeding Scenarios}
% spray footage, size of spray drop-let and , various scenes and plants with various growth-stages (requiring adequate amount of sprayed herbicide), strategy of intervention with different weed types (bushy, branchy, vertically and horizontally shape weeds etc).




% \subsection{Real-World Intervention Performance} 
% The third experiment is designed to show the capability of the \bbot\ in real-world condition. 
% Here we tend to outline and capture the important parameters of robots operation like: run-time performance each component in weeding pipeline (detection and tracking, planning, actuation), accuracy of operations and ????.
% \begin{itemize}
%     \item 
%     \item footprint of sprayers (via nozzle with different footprint types like: line, circle, spot)
%     \item can be evaluated with RMSE (from plant center to center of spray cone!) 
% \end{itemize}



% \begin{table}[!h]
%     \vspace{-2mm}
% 	\centering
% 	\caption{\TODO{MAH Better caption}Only with best planning method (CD-BF)}
% 	\begin{tabular}{l | ccc}
% 	\toprule
% 	 & \textbf{Distance-Based}  & \textbf{
% 	 Dynamic-Division} & \textbf{Sub-Division} \\\hline
% 	 \midrule
%     %& \multicolumn{3}{c}{SB20} \\
%     %\midrule
%     %R1-S1     & \textbf{15.8}\% & 25.0\% & 17.7\%\\
%     %R3-S1     & \textbf{11.5}\% & 21.5\% & 15.2\%\\
%     SB20-S1     & \textbf{11.9\%} & 13.5\% & 19.8\% \\
%     SB20-S2     & \textbf{0.0}\%  & \textbf{0.0}\% & 2.3\%  \\
%     CN20        & \textbf{0.0\%}  & 3.4\% & 4.3\%  \\
%     %R11-S1    & 8.5\%  & 13.0\% & \textbf{7.66}\% \\
%     % \midrule
%     % & \multicolumn{3}{c}{CN20} \\
%     % \midrule
%     % R7        & \textbf{0.0}\% & 4.1\% & \textbf{0.0}\% \\
%     % R9        & \textbf{0.0}\% & 7.7\% & 7.7\% \\
%     % R11       & \textbf{0.0}\% & 1.2\% & 2.4\% \\
%     \bottomrule
%     \end{tabular}
%     % \vspace{-4mm}
% 	\label{tab:movement_params}
% \end{table}


% \begin{table}[!h]
%     \vspace{-2mm}
% 	\centering
% 	\caption{$\mu \pm \sigma$ of axes traveled distances (CD-BF)}
% 	\begin{tabular}{l | ccc}
% 	\toprule
% 	 & \textbf{Distance-Based}  & \textbf{Sub-Division} & \textbf{
% 	 Dynamic-Division}\\\hline
% 	 \midrule
%     & \multicolumn{3}{c}{SB20} \\
%     \midrule
%     R1-S1     & 11.1$\pm$1.7 & 5.1$\pm$4.1 & 5.4$\pm$2.4\\
%     R3-S1     & 11.2$\pm$1.3 & 4.8$\pm$3.1 & 5.3$\pm$1.5\\
%     R3-S2     & 1.4$\pm$0.2  & 1.0$\pm$0.8 & 1.5$\pm$0.6\\
%     R11-S1    & 7.9$\pm$1.2  & 4.2$\pm$3.3 & 4.2$\pm$1.6\\
%     \midrule
%     & \multicolumn{3}{c}{CN20} \\
%     \midrule
%     R7        & 1.1$\pm$0.2 & 1.7$\pm$1.7 & 2.7$\pm$2.9 \\
%     R9        & 2.8$\pm$0.8 & 2.5$\pm$3.2 & 4.0$\pm$0.7 \\
%     R11       & 4.0$\pm$0.7 & 3.6$\pm$3.7 & 4.9$\pm$4.7 \\
%     \bottomrule
%     \end{tabular}
%     % \vspace{-4mm}
% 	\label{tab:movement_params}
% \end{table}