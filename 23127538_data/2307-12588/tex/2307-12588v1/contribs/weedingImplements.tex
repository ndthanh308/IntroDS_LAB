
% Here I want to only describe the mechanic and electronics of the weeding tool 

%The necessity of this problem is getting bolder as we find out more about the harms that broadcast weeding reinforcements are causing to our eco-system.
Achieving flexible, and repeatable weeding implements that can deploy a variety of end-effectors is a key objective for BonnBot-I.
This enables the system to change tools given the current soil and weed populations.
The proposed design utilizes independently controlled high-resolution Igus linear actuators fixed at height of $0.72m$ above the ground and creating a working space of $1.3m\times0.36m$; the current design uses 4 such linear actuators.
%The proposed design utilizes four independently controlled high-resolution igus ZLW-1040S linear actuators fixed on the BonnBot-I at height of $0.6m$ above the ground and creating a working space of $1.3m\times0.36m$ as shown in \figref{fig:robotScheme}.
Each linear axis has a length of $1.3m$ and is controlled by an Igus dryve D1 motor control system via Modbus connection with a maximum resolution of $0.01cm$ and is capable of performing translations with maximum velocity and acceleration of $5m/s$ and $10m/s^2$, respectively.
All linear axes are equally-spaced and currently carry spot-spray nozzles, however, the system design permits many kinds of end-effectors, such as mechanical hoeing, providing flexibility. % for possible weeding tools like mechanical spot hoeing.
%all linear axes carry spray valves so can be used independently to engage with weeds.

To control the linear actuators and spray valves we use a ROS operated Raspberry-Pi 3B and to ensure minimal action delay the nozzles are accessed via high speed N-channel MOSFET-transistors.
Hence, ultimate operation time for each spray head in our system adds up to $10\sim12$ms including valves On-Off time. 
The spray system consists of a reservoir tank capable of carrying $5L$s of compressed liquid with a maximum $16bar$ pressure, as well as a compact $8bar$ portable compress which is fixed on the robot.
As the droplet size from the spray nozzles depends on the liquid pressure we use individually adjustable valves.
This allows us to control the spray footprints on the ground individually for each nozzle between $0.02m$ to $0.13m$; for this paper we assume a constant spray footprint of $0.05m$.



\begin{comment}
To control Linear actuator and nozzle valves we use a Raspberry-Pi 3-B single-board computer which all its outputs are interfaced via PiXtend v1.3 I/O board. 
Pixtend is a PLC (Programmable Logic Controller) compliant logic board with a wide range of isolated digital and analog inputs and outputs.
It provides standard serial interfaces RS232, RS485, Ethernet, CAN and etc, and is shown in ~\figref{fig:weeding_controler}.
We use High speed N-Channel MOSFET-Transistor (IRFZ44N 55V, 41A) of Pixtend board to control the spray nozzle valves, this ensures minimize action delay in electronics. 
% % Figure environment removed 
\subsubsection{Spray Infrastructure}
all linear axes carry spray implements so can be used independently to engage with weeds.
We use ASCO™ solenoid L172V03 spray valve with On-Off time equal to $\sim10ms$, which are lowered to a height of $0.22m$ using an aluminum level arm. 
The spray system consist of a reservoir tank capable of carrying $4L$ compressed liquid with maximum $80bar$ pressure.
And each nozzle is being fed with compressed herbicide via isolated pipes and control valve. 
This way we can control output pressure of each spray separately.
Also as the droplet size of spray nozzles depend on the liquid pressure we can control the spray footprints on the ground and droplet sizes individually for each nozzle.
\end{comment}


% \subsubsection{Hoeing Implement}
% On the two other linear axes we have fixed a high-speed pneumatic cylinders with length of $12cm$ used for turning soil and taking out the weeds from soil which in agriculture it is called hoeing.
% Hoeing aims to sever the top growth from the roots, just below the soil surface, then leaves the weed in the sun to wither to dry-out.
% While, this approach does not guarantee successful treatment to all types of weeds spatially deep-rooted or perennial weeds~\cite{larkcom2013grow}.
% But, by throwing the plant out of the soil we ensure to increase the success rate.
% Furthermore, by considering different types of hoeing axles 
% Long-handled hoes are easier on the back, whereas a short-handled ‘onion hoe’ is better for closely planted areas, where you don’t want to damage nearby plants. 
% We propose the following shape which ...
% The pneumatic cylinders control the positions of a lever-arm carrying a hoeing head used turning soil.
% As~\figref{fig:heingTool} shows the mechanism is fixed with a vertically fixed Aluminium lever-arm  

% % Figure environment removed