
%The main use case of BonnBot-I is to manage weeds in arable fields.
%To perform this task it is equipped with a novel weeding tool described in \secref{subsec:weedingImplements}.
%The actual treatment controlled based on an advanced DNN driven monitoring approach whcih is explained in \secref{subsec:cropWeedMonitoring}. 
%The monitoring approach estimates critical phenotypic information like crop/weed classification and segmentation, stem location and pixel-wise area of plants (or BBCH) of crops located underneath the robot in real-time while robot is moving through crop-row.

%The scarcity of proper agricultural datasets suited for precision weed management still is a challenge in agriculture.
%Hence we gathered and annotated two datasets which are being used for our plant monitoring system which is more elaborated in the following.




%\subsection{Dataset}
%\label{subsec:dataset}
%\section{Dataset Description}
\label{sec:dataset}


In this section, we describe our dataset. 
\dataset contains \num{1209} unique roots.
A root refers to the first element in a package name. 
For example, in the package name "com.example.mypackage", "com" is the root.

We also collected data on the number of fields in the package names of the apps in our dataset. 
A field refers to a dot-separated element in a package name.
For example, in the package name "com.example.mypackage", there are three fields: "com", "example", and "mypackage".
Results are visible in Table~\ref{table:num_of_fields}.
There are 733 package names with only one field, \num{14368} package names with two fields, etc.

\begin{table}
    \centering
    \caption{Number of package names per field in \dataset}
    \begin{adjustbox}{width=.7\columnwidth,center}
        \begin{tabular}{lr|lr}
            \hline
            Fields & Count & Fields & Count \\ \hline
            with 1 field & \num{733} & with 5 fields & \num{100}\\ 
            with 2 fields & \num{14368} & with 6 fields & \num{6}\\ 
            with 3 fields & \num{4231} & with 7 fields & \num{3}\\ 
            with 4 fields & \num{720} & with 8 fields & \num{1}\\
            \hline \hline
            \multicolumn{2}{l|}{Total} & \multicolumn{2}{r}{\libsAfterRefinement}
        \end{tabular}
    \end{adjustbox}
    \label{table:num_of_fields}
\end{table}



There are significantly fewer package names with four or more fields. 
The number of package names with one field is relatively low compared to the others.
This suggests that many package names in the dataset follow a standard naming convention with a domain name followed by one or more subpackages.
The presence of package names with four or more fields may indicate the use of more complex or specialized naming conventions\footnote{Examples of libraries are:  
\href{https://mvnrepository.com/artifact/riddley/riddley}{riddley}, 
\href{https://mvnrepository.com/artifact/jakarta.annotation}{jakarta.annotation}, 
\href{https://mvnrepository.com/artifact/com.vogle.sbpayment}{com.vogle.sbpayment}, 
\href{https://mvnrepository.com/artifact/pl.robakowski.jersey.bootstrap}{pl.robakowski.jersey.bootstrap}, 
\href{https://mvnrepository.com/search?q=de.tudresden.inf.lat.jsexp}{de.tudresden.inf.lat.jsexp}, 
\href{https://mvnrepository.com/artifact/de.hs_rm.cs.vs.tools.vocabularygenerator}{de.hs\_rm.cs.vs.tools.vocabularygenerator}, 
\href{https://mvnrepository.com/artifact/eu.adlogix.com.google.api.ads.dfp}{eu.adlogix.com.google.api.ads.dfp}, 
\href{https://mvnrepository.com/artifact/us.gov.dot.faa.ang.c55/huggs}{us.gov.dot.faa.ang.c55.gradle.huggs}.
}.


Table~\ref{table:top_ten} presents the top 10 most frequent roots and the top 10 most frequent fields found. 
In the first two columns, we can see that the root "com" is by far the most frequent, with more than \num{10000} occurrences. 
The second most frequent root is "net", with \num{1265} occurrences. 
In the second two columns, which represents the most frequent fields, including the roots, we can see that the field "com" is still the most frequent. 
The second two columns do not differ much from the first two columns, except for the two fields "gradle" and "android" that now appear.
This could indicate that Android libraries are prevalent in the dataset.
It is confirmed in the last two columns, which represent the most frequent fields without the roots. 
After "gradle" and "android", the third most frequent field is "sdk", with 138 occurrences.
We see a shift in the most prevalent fields. 
Instead of roots, we now see fields such as "sdk", "maven", "plugin(s)", "api", "tools", and "common".
This may be indicative of the types of libraries.
Overall, the results suggest that most package names are from the "com" domain and that Android libraries are well represented.


\begin{table}
    \centering
    \caption{Top 10 roots and fields present in \dataset}
    \begin{adjustbox}{width=.9\columnwidth,center}
        \begin{tabular}{c|c|c|c|c|c}
            \hline
            \multicolumn{2}{c|}{\textbf{Top 10 roots}} & \multicolumn{2}{c|}{\textbf{Top 10 most used fields}} & \multicolumn{2}{c}{\textbf{Top 10 most used fields w/o roots}}\\ \hline
            \textbf{Root} & \textbf{Count} & \textbf{Field} & \textbf{Count} & \textbf{Field} & \textbf{Count} \\ \hline \hline
            com & \num{10520} & com & \num{10551} & gradle & \num{250} \\ \hline
            net & \num{1265} & net & \num{1273} & android & \num{239} \\ \hline
            de & \num{917} & de & \num{918} & sdk & \num{138} \\ \hline
            cn & \num{663} & cn & \num{663} & plugin & \num{120} \\ \hline
            dev & \num{458} & dev & \num{465} & maven & \num{108} \\ \hline
            me & \num{411} & me & \num{413} & plugins & \num{93} \\ \hline
            eu & \num{223} & gradle & \num{254} & api & \num{60} \\ \hline
            ru & \num{202} & android & \num{240} & oss & \num{57} \\ \hline
            fr & \num{188} & eu & \num{224} & tools & \num{54} \\ \hline
            ch & \num{181} & ru & \num{203} & common & \num{39} \\ \hline
        \end{tabular}
    \end{adjustbox}
    \label{table:top_ten}
\end{table}

%\subsection{Crop/Weed Instance Monitoring and Tracking}
%\label{subsec:cropWeedMonitoring}
%

\TODO{To perform field monitoring we make use of Mask-RCNN in conjunction with re-projection as described in our prior work by Halstead et al.~\cite{halstead2021crop}. 
Mask-RCNN provides instance-based semantic segmentation which includes the class information (e.g. weed species) and viewable surface area from the RGB-D camera.
The strength of the approach lies in being able to estimate how far the camera has moved, for instance Halstead~et~al. used wheel odometry, to re-project information between frames and conduct better object tracking as well as counting.
Despite the success of the proposed approach with wheel odometry, the extra localization information (e.g. GPS and IMU) available on \bbot\ should further enhance this technique.}

\TODO{In this paper we enhance demonstrate that the extra sensors on \bbot enhance the localization information which also enhances field monitoring.
In particular, we fuse the available odomtery and GPS information with an Extended Kalman Filter(EKF)~\cite{wei2011intelligent}.
This algorithm recursively estimates the state of a non-linear system in an optimal way.
Furthermore, adding local source of motion estimation can considerably reduce the risk of outage due to lack of proper satellite observations~\cite{ahmadi2021towards}.}

\TODO{The classic EKF method consists of two steps: the prediction and the correction. 
In prediction step the algorithm tries to predict future state of the system based on motion model, while the correction step improves predicted state's accuracy using real measurements from sensors. 
Using this principle, the state of the system can be determined recursively in real time.
In our system, the wheel odometry is used as control data~$u_t=(d_t^{t+1}, w_t)$ in the prediction step and position and orientation of the embedded EKF solution for the SBG system provides corrections.
As the robot velocity is always controlled with a differential controller model within crop-rows we express its kinematics model in x-y plane as:}
% 
\begin{equation}
    \begin{array}{lr}
        x_{t+1} = x_t + d_t^{t+1} \cdot cos(\theta_{t+1}), \\
        y_{t+1} = y_t + d_t^{t+1} \cdot sin(\theta_{t+1}). \\
        \theta_{t+1} = \theta_t + w_t
    \end{array}
    \label{eq:ekf_motionModel}
\end{equation}
% 
where $d_t^{t+1}$ is circular arc traveled within $t$ to $t+1$ and $w_t^{t+1}$ is the rotation angle around z-axis.
So, let the vector $\Vec{\mathbf{X}}_t = [x_t, y_t, z_t, \phi_t, \psi_t, \theta_t]^T$ be the state vector of the \bbot\ at time $t$.
where $x_t, y_t, z_t$ denote the current position in \textit{ENU} coordinate frame, and $\phi_t, \psi_t, \theta_t$ specify roll, pitch and yaw angles of the robot in world coordinate frame $\mathcal{F}_w$.

\TODO{In our prior work~\cite{halstead2021crop} some of the most challenging plants were grasses and so we propose to evaluate improvements to field monitoring (via improved localization with more sensors) on a grass crop. 
In particular, our evaluations for field monitoring in Section~\ref{sec:exp} are conducted on corn fields which have similar weed species to our prior work but the crop (the dominant plant) has a thinner leaf structure which makes it more difficult to track.}

%The trained model is used to evaluate the tracking performance of the BonnBon-I platform using varying types of odometry information.
%Specifically to improve the accuracy of image-based tracking results we tend to use Extended Kalman Filter(EKF) to fuse odomtery and GPS information which is explained in the following.
%
%However, wheel odometry is a coarse
%methodand tracking is counting is achieving by performing tracking which uses the re-projection of masks.
%This re-projection can benefit from the 

%then performedas our based method for instance-based semantic segmentation with conjunction with reprojection mathod explained in~\secref{subsec:intraCameraTracking}. 

%Similar to~\cite{halstead2018fruit, halstead2021crop}, we augment a Mask-RCNN network to incorporate super-class and sub-class classification heads. 
%The super-class represents generalized object recognition as a binary background versus plant classifier, while the sub-class outputs species level classifications of the crop and weed types.
%The overall benefit of this approach is the ability to accurately segment objects (super-class) while providing fine-grained classification with minimal extra overhead.

\begin{comment}
The performance of this approach is outlined in~\cite{halstead2021crop}, where its benefit of being crop and environment agnostic was shown for both arable farmland and a glasshouse setting.
The strength of this approach is further outlined in~\cite{halstead2020} where the parallel network structure was compared to a standard $N$-class network.
The parallel structure was able to more accurately detect the presence of objects in the scene while still supplying relevant species level information.

This approach is used on the two datasets mentioned in~\secref{subsec:dataset}.
For SB20 we train a model using the same parameters outlined in~\cite{halstead2021crop} and store the fine-grained and bounding box location, sub-class label, and size information.
These characteristics are used for the intervention experiments.
%
% % Figure environment removed 
%
We once again show the flexibility of this approach by applying it to a novel crop - corn.
While the corn fields have similar weed species to the sugar beet dataset of~\cite{halstead2021crop} the thinner leaf structure of the early growth corn makes it a complex crop to classify and segment.
The trained model is used to evaluate the tracking performance of the BonnBon-I platform using varying types of odometry information.
Specifically to improve the accuracy of image-based tracking results we tend to use Extended Kalman Filter(EKF) to fuse odomtery and GPS information which is explained in the following.
\end{comment}

\begin{comment}
In~\cite{halstead2021crop} we showed that by augmenting Mask-RCNN architecture with super-class and sub-class heads we can avoid miss-classifications of less represented sub-classes. 
Specially for agriculture we can consider that their is a primary class (plant) with sub-classes (fine-grained classification) which in most of the cases the distribution of species appearance are not uniform.
The benefit of this approach is demonstrated in~\cite{halstead2021crop} for both glasshouses, and arable farmland and outlined the benefit of having crop, environment, and platform invariant approaches.
Figure~\ref{fig:subcls} outlines the basic sub-class structure inserted into the Mask-RCNN network.
This super-class (plant) and sub-class (species) alignment ensures that the super-class generalises to what a ``plant'' is while the sub-class provides the species level classification required for weeding~\cite{halstead2018fruit, halstead2020}.
While these less represented classes may be missed entirely in an $N$-class method the generalised super-class approach is still able to identify them as a ``plant''.


% Mask-RCNN is designed to be an $N$-class classifier with an bounding box regression and instance based segmentation within the bounded region.
% The strength of this is that multiple classes can be classified, regressed, and segmented within the same network.
% However, this approach is not always optimal.
% For agriculture we can consider that their is a primary class (plant) with sub-classes (fine-grained classification).

% This super-class (plant) and sub-class (species) alignment ensures that the super-class generalises to what a ``plant'' is while the sub-class provides the species level classification required for weeding.
% A key benefit of this technique~\cite{halstead2018fruit, halstead2020, halstead2021crop} is the ability to still recognise less represented classes.
% While these less represented classes may be missed entirely in an $N$-class method the generalised super-class approach is still able to identify them as a ``plant''.

% As a species level classifier~\cite{halstead2021crop} showed the benefit of this approach for both glasshouses, and arable farmland and outlined the benefit of having crop, environment, and platform invariant approaches.
% Figure~\ref{fig:subcls} outlines the basic sub-class structure inserted into the Mask-RCNN network.
% Generally, this network has a classification head, and regression head, and a mask head.
% In our approach we insert a sub-class classifier in the same network location as the classifier and regressor, which utilise the same embedding layer as these two.



% BonnBot-I is a arable farmland monitoring platform that works in row-crop fields.
% This allows it to capture data of any crop type that in these fields. 
Previously we have used sugar beet data captured on this platform in~\cite{halstead2021crop}, to show the flexibility of this approach we will utilise corn data captured on the same platform.

In this case the plants as a whole represent the super-class and the species represent the sub-class.

\begin{table}[!h]
    \vspace{-2mm}
	\centering
	\caption{\TODO{Dataset Sub Categories ... - MAH I don't think we need this table, we have already given the corn annotations previously.}}
	\begin{tabular}{l cc cc cc cc c}
	\toprule
	%&& \multicolumn{2}{c}{BUP} \\\hline
	  & \textbf{SB} & \textbf{CN} & \textbf{Bi} & \textbf{An} & \textbf{Chy} & \textbf{Pe} & \textbf{Th} & \textbf{Ch} & \textbf{Un} \\\hline
	 \midrule
    SB20 & 768 & - & 241 & 19 & 64 & 620 & 775 & 232 & 206 \\ 
    CN20 & - &  &  &  &  &  &  &  & \\
    \bottomrule
    \end{tabular}
    % \vspace{-4mm}
	\label{tab:dataset_cats}
\end{table}
\end{comment}
% Mask-RCNN with species level weed classification.

% \cite{halstead2018fruit, halstead2020, halstead2021crop}

% \begin{itemize}
%     \item Standard mask and faster pipeline. Talk about N Classes.
%     \item Refer to \cite{halstead2018fruit} and \cite{halstead2020} for an overview of why a super-class sub-class relationship does a better job when object counts are limited compared to an N-super-class classifier.
%     \item Describe the sub-class layer, maybe with a NEW figure.
%     \item refer to \cite{halstead2021crop} to show it working as a species-level classifier.
%     \item Talk about how the platform enables this stuff.
%     \item Corn seg
% \end{itemize}

% % Figure environment removed



\begin{comment}
\subsubsection{Precise in Field localization}
\label{sec:fieldLocalization}
The accuracy of in-field localization directly influences the accuracy of monitoring system and interventions. 
Hence to maximize weeding operations resolution a millimeter-level accurate position determination is crucial.
To increase the accuracy of robot localization from couple of centimeters (provided by SBG system) to level of a few millimeters we fuse SBG measurement with robot's wheel odometry using Extended Kalman Filter (EKF)~\cite{wei2011intelligent} which is an algorithm to recursively estimate the state of a non-linear system in an optimal way.
Furthermore, adding local source of motion estimation can considerably reduce the risk of outage due to lack of proper satellite observations~\cite{ahmadi2021towards}.

The classic EKF method consist of two steps:the prediction and the correction. 
In prediction step the algorithm tries to predict future state of the system based on motion model, while the correction step improves predicted state's accuracy using real measurements from sensors. 
Using this principle, the state of the system can be determined recursively in real time.
In our system, the wheel odometry is used as control data~ $u_t=(d_t^{t+1}, w_t)$ in prediction step and position and orientation of embedded EKF solution of SBG system provides corrections.
As the robot velocity is always controlled with a differential controller model within crop-rows we express its kinematics model in x-y plane as: 
% 
\begin{equation}
    \begin{array}{lr}
        x_{t+1} = x_t + d_t^{t+1} \cdot cos(\theta_{t+1}), \\
        y_{t+1} = y_t + d_t^{t+1} \cdot sin(\theta_{t+1}). \\
        \theta_{t+1} = \theta_t + w_t
    \end{array}
    \label{eq:ekf_motionModel}
\end{equation}
% 
where $d_t^{t+1}$ is circular arc traveled within $t$ to $t+1$ and $w_t^{t+1}$ is the rotation angle around z-axis.
So, let the vector $\Vec{\mathbf{X}}_t = [x_t, y_t, z_t, \phi_t, \psi_t, \theta_t]^T$ be the state vector of the \bbot\ at time $t$.
where $x_t, y_t, z_t$ denote the current position in \textit{ENU} coordinate frame, and $\phi_t, \psi_t, \theta_t$ specify roll, pitch and yaw angles of the robot in world coordinate frame $\mathcal{F}_w$.



% In the correction step, using SBG module output and wheel Odomtery readings we determine travelled distance based on odometry, GPS and O which are contained in measurement vector as $\mathbf{d}_{xy}^{o/s}$ in row. 
% This leads to the following measurement vector as time $t$:
% \begin{equation}
%   \label{eq:ekf_measurementVec}
%   \Vec{\mathbf{z}}_t = [x^s, y^s, z^s, \mathbf{d}_{xy}^{o/s}, a_z^i, \phi^i, \psi^i, \theta^i, \boldsymbol{\omega}_x^{i}, \boldsymbol{\omega}_y^{i}, \boldsymbol{\omega}_z^{i}]
% \end{equation}
% where, SBG ($s$) ensure the absolute positioning is obtained, 
% as , the IMU ($i$) provides absolute roll, pitch and yaw angles 
% where, $a_z^t$ represent the acceleration in z-direction and $\boldsymbol{\omega}_x^{i/v}, \boldsymbol{\omega}_y^{i/v}, \boldsymbol{\omega}_z^{i/v}$ indicate the angular velocities in x,y and z directions estimated from VO and IMU units.

% EKF explanation + VO version !! 
\begin{comment}
\subsubsection{Extended Kalman Filter (EKF)}: 
To fuse measurement of different systems we use the Extended Kalman filter (EKF)~\cite{??} which is an algorithm to recursively estimate the state of a non-linear system in an optimal way.
This classic method consist of two steps:the prediction and the correction. 
In prediction step the algorithm tries to predict future state of the system based on the system model, while the correction step improves this prediction using real measurements or additional information. 
By applying this principle, the state of the system can be determined recursively in real time.
Let $\Vec{\mathbf{X}}_t = [x_t, y_t, \phi_t, \psi_t, \theta_t]^T$ denote the state vector of the \bbot\ at time $t$.
where $x_t, y_t, z_t$ denote the current position in \textit{NEU} coordinate frame, and $\phi_t, \psi_t, \theta_t$ specifies the roll, pitch and yaw angles of the vehicle in world coordinate frame $\mathcal{F}_w$.
As the robot is always controlled with differential model within crop-rows we express its kinematic model in x-y plane as: 
\begin{equation}
    \begin{array}{lr}
        x_{t+1} = x_t + d_t^{t+1} \cdot cos(\theta_{t+1}), \\
        y_{t+1} = y_t + d_t^{t+1} \cdot sin(\theta_{t+1}). \\
        \theta_{t+1} = \theta_t + w_t^{t+1}
    \end{array}
    \label{eq:ekf_motionModel}
\end{equation}
Where $d_t^{t+1}$ is xy-distance traveled within $t$ to $t+1$ and $w_t^{t+1}$ is the rotation angle around z-axis.

In the correction step, using SBG module embedded EKF output and Visual-Odomtery we determine travelled distance based on odometry, GPS and VO which are contained in measurement vector as $\mathbf{d}_{xy}^{o/g/v}$ in row. 
This leads to the following measurement vector as time $t$:
\begin{equation}
  \label{eq:ekf_measurementVec}
  \Vec{\mathbf{z}}_t = [x^g, y^g, \mathbf{d}_{xy}^{o/g/v}, a_z^i, \phi^i, \psi^i, \theta^i, \boldsymbol{\omega}_x^{i/v}, \boldsymbol{\omega}_y^{i/v}, \boldsymbol{\omega}_z^{i/v}]
\end{equation}

as GPS ($g$) we ensure the absolute positioning is obtained, the IMU ($i$) provides absolute roll, pitch and yaw angles 
where, $a_z^t$ represent the acceleration in z-direction and $\boldsymbol{\omega}_x^{i/v}, \boldsymbol{\omega}_y^{i/v}, \boldsymbol{\omega}_z^{i/v}$ indicate the angular velocities in x,y and z directions estimated from VO and IMU units. 
\end{comment}

%  VO part asnd tracking part
\begin{comment}
\subsubsection{Intra-Camera Target Tracking}
Tracking-via-segmentation aims to exploit known properties of the agricultural scene as a robotic platform traverses a row.
As the scene remains relatively static between captures (from the camera on the platform) it can be assumed that the plants are both temporally and spatially static.
Assuming these properties we are able to conclude that an object captured at $t$ will be close in image location at $t+1$.
The platform movement and the frames-per-second of the camera allow us to make this assumption as we are not moving too rapidly that the scence changes dramatically.
Halstead et al.~\cite{halstead2021crop} provides a detailed outline of this approach for the original intersection-over-union based approach.

Early tracking-via-segmentation~\cite{halstead2018fruit} relies heavily on the  assumption of spatial and temporal consistancy. 
If the captured images are too far apart objects can not be matched which duplicates tracklets.
However, in~\cite{smitt2020pathobot} we showed that even with larger movements between captures can be aggregated together through reprojection using the wheel odometry of the platform and the depth information.
One problem with these early approaches was the matching criterion: IoU.
In~\cite{halstead2021crop} it was shown that for small discrepancies in image captures, even with reprojection, this criterion struggles to match objects, particularly small objects.

To allow for this~\cite{halstead2021crop} outlined a new matching criterion and compared it directly to the IoU.
This dynamic radius was able to match objects, even small objects, due to the underlying technique of using the Euclidean distance compare the center of mass between two objects, if this value is below a threshold, based on a radius value, the objects are aggregated into a single tracklet.
There is a single issue with this approach in that it can match 360$^o$ around the center of mass meaning incorrect matches do occur.
\TODO{total plant number od corn dataset is: row 7 & 545, row 9 & 802, row 11 & 713}
% \begin{table}
%     \centering
%     \caption{\TODO{Ground truth of the corn tracking data - MAH Put this in the text no reason for a table.}}
%     \label{tab:gttrack}
%     \begin{tabular}{|l|c|}
%         \hline
%         row id & total plants \\
%         \hline\hline
%         row 7 & 545 \\
%         row 9 & 802 \\
%         row 11 & 713 \\
%         \hline
%     \end{tabular}
% \end{table}

% \cite{smitt2020pathobot, halstead2021crop}
% \begin{itemize}
%     \item Describe tracking via segmentation with a figure.
%     \item Talk about \cite{smitt2020pathobot} as the initial trials to get this working.
%     \item Introduce \cite{halstead2021crop} and the reasons for DR over IoU and the benefits of it.
%     \item Talk about how the platform enables this type of thing.
%     \item Corn tracking?
% \end{itemize}
\begin{itemize}
    \item explain the necessity of VO and improvement of the tracking
    \item Describe VO approach briefly!
    \item explain relevance of VO output to prev. section (intra-camera tracking).
    \item explain ekf where VO gets fused with GPS output to take positioning accuracy from couple of centimeters to millimeter accuracy! (must be proved ... maybe be with simple map or orthomosaic!)
\end{itemize}
% Using Depth information of D455 sensor we are able to accurately localize and geo-reference each crops/weed on the ground, ??? increasing the intervention accuracy.
% Accessing consistent input modalities with such methods can reveal helpful representations of data and aid in estimating phenotypic information at without any extra costs for the system. 
% A like the depth modality which can be used for estimating growth-stage of crops and geometric based data association between consequent frames.
Visual Odometry~\cite{ahmadi2021registration} 
\end{comment}
\end{comment}


% \bbot\ performs not only weed management but also field monitoring.
One of the key benefits of \bbot\ as a platform is that it can monitor the state of the field while traversing it, this supplements its key function as a weeding platform.
We demonstrate the potential of \bbot\ for field monitoring by illustrating how the extra localization sensors can enhance the existing tracking algorithms in our prior work~\cite{halstead2021crop}.
This approach used Mask-RCNN to provide instance-based segmentation, species-level information (e.g. crop and weed species) and the viewable surface area. % from the RGB-D camera.
This approach included a tracking-via-segmentation technique that outlined the benefit of of spatial matching operator, coined dynamic radius (DR), over a pixel-wise version, referred to as intersection over union (IoU).
This technique also exploited re-projection between frames using wheel odometry and camera parameters.
This enabled more accurate tracking of objects in the scene, however, wheel odometry is often prone to errors.
Using the extra GPS sensors available on \bbot\ has the potential to increase performance by re-projecting more accurately between frames. %, especially over large frame skips.
% The strength of the approach lied in being able to estimate how far the camera had moved, for instance Halstead~et~al. used wheel odometry, to re-project information between frames and conduct better object tracking as well as counting.
% Despite the success of the proposed approach with wheel odometry, the extra localization information (e.g. GPS and IMU) available on \bbot\ should further enhance such an approach.

In this paper we demonstrate that the extra localization sensors on \bbot\ can be used to enhance the performance of field monitoring.
In particular, we fuse the available odometery and GPS information with an EKF~\cite{wei2011intelligent}.
This algorithm recursively estimates the state of a non-linear system in an optimal way.
Furthermore, adding local source of motion estimation can considerably reduce the risk of outage due to lack of proper satellite observations~\cite{ahmadi2021towards}.

\begin{comment}
The classic EKF method consists of two steps: the prediction and the correction. 
In prediction step the algorithm tries to predict future state of the system based on motion model, while the correction step improves predicted state's accuracy using real measurements from sensors. 
Using this principle, the state of the system can be determined recursively in real time.
In our system, the wheel odometry is used as control data~$u_t=(d_t^{t+1}, w_t)$ in the prediction step and position and orientation of the embedded EKF solution for the SBG system provides corrections.
As the robot velocity is always controlled with a differential controller model within crop-rows we express its kinematics model in x-y plane as:
% 
\begin{equation}
    \begin{array}{lr}
        x_{t+1} = x_t + d_t^{t+1} \cdot cos(\theta_{t+1}), \\
        y_{t+1} = y_t + d_t^{t+1} \cdot sin(\theta_{t+1}). \\
        \theta_{t+1} = \theta_t + w_t
    \end{array}
    \label{eq:ekf_motionModel}
\end{equation}
% 
where $d_t^{t+1}$ is circular arc traveled within $t$ to $t+1$ and $w_t^{t+1}$ is the rotation angle around z-axis.
So, let the vector $\Vec{\mathbf{X}}_t = [x_t, y_t, z_t, \phi_t, \psi_t, \theta_t]^T$ be the state vector of the \bbot\ at time $t$.
where $x_t, y_t, z_t$ denote the current position in \textit{ENU} coordinate frame, and $\phi_t, \psi_t, \theta_t$ specify roll, pitch and yaw angles of the robot in world coordinate frame $\mathcal{F}_w$.
\end{comment}

In our prior work~\cite{halstead2021crop} we concentrated on sweet pepper in a horticultural setting and sugar beet in arable farmland.
In both cases the objects we aim to detect are somewhat robust to external influences such as weather conditions.
However, some of the weed species witnessed in the arable farmland were grasses and their accurate localization proved difficult.
In this paper we further outline the ability for our approach to be crop agnostic by performing monitoring on a novel grass crop dataset consisting of corn.
Corn has a long leaf structure which makes it susceptible to weather conditions resulting in a difficult crop to localize, due to this, in~\secref{subsec:cropWeedMonitoring} we outline our performance using both the pixel-wise and spatial matching criteria from~\cite{halstead2021crop}.
% \TODO{MAH this needs some work, also missing reference:
% In our prior work~\cite{halstead2021crop}, some of the most challenging plants were grasses and so we propose to evaluate improvements to field monitoring (via improved localization with more sensors) on a grass crop. 
% In particular, our evaluations for field monitoring in Section~\ref{subsec:cropWeedMonitoring} are conducted on corn fields which is a grass crop that has a thin and long leaf structure which makes it more difficult to track.
% }

This corn data set (CN20) was acquired using \bbot\ from a phenotyping field at campus Klein-Altendorf (CKA) of the University of Bonn.
The data was captured using an Intel RealSense D435i sensor with a nadir view of the ground in front of the robot and resolution of $1280\times720$ with a frame-rate of $15Hz$. 
The non-overlapping training, validation, and evaluation data includes RGB-D frames for $170$, $43$ and $70$ images respectively.
This data comes from six different rows providing unique crop and weed distributions due to the non-homogeneous growth stage of the weeds.
% The data comes with $283$ RGB and depth frames covering six rows of crop providing a large distribution of crops sizes due to non-homogeneous growth stage and several weed types.
In total there are nine different categories of weeds containing a total of $2566$ and $1261$ instances of crop and weeds respectively. 
The data is annotated to include instance based pixel-wise segmentation, bounding boxes and stem locations of each instance in Coco format~\cite{lin2014microsoft}.
% To train out DNN-based monitoring system we divided the images of dataset in to sets of training, validation and evaluation of size $170$, $43$ and $70$, respectively.
% $149$, $39$ and $44$,
\figref{fig:datasetSample} shows an example annotated image of CN20 dataset.

% Figure environment removed 
