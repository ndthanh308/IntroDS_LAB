

The preference for consuming more natural and organic foods has rapidly increased in recent years~\cite{blasco2002ae}.
This has forced the agricultural industry to use fewer agri-chemicals when dealing with weeds while maintaining the quality and quantity of the crop.
Weed intervention is an important aspect of arable farming due to its competition with crops for nutrients in the soil~\cite{slaughter2008autonomous} which potentially reduces yield.
% Weeds affect crop yield due to competition with the crops to acquire plant nutrients and resources~\cite{slaughter2008autonomous}.
Currently, to alleviate this impact, the majority of farmers use uniform treatments for weed control for instance by treating the entire field with herbicide irrespective of weed presence. % if weeds are present or not. %the majority of farmers conduct uniform treatment in the fields for weed control.
This approach has led to an increasing number of herbicide-resistant weed species~\cite{heap22_website} as well as negatively impacting the environment by increasing soil erosion and water contamination~\cite{mia2020sustainable}.% due to overloading fields with agri-chemicals.
%Furthermore, overloading fields with agri-chemicals can cause negative environmental impacts like increasing soil erosion and water contamination~\cite{mia2020sustainable}.

Robotic weed intervention has the potential to revolutionize weeding paradigms through plant-level weed management.
For instance, by only treating a plant if it is present and using the most appropriate action for the particular plant species~\cite{Bawden17_1}. %, remodelling how weed management is performed.
To achieve plant and species specific treatments, robots are driven by advanced perception systems that can also provide rich crop monitoring information~\cite{halstead2021crop}.
Yet, robotic design has primarily explored how to design the platform to perform weeding operation.
%Furthermore, as these robots are driven by advanced perception systems they can also provide rich crop monitoring information.
% Hence, the current ideas on how weed management must be done need to be rethought.
% Robotics can be the enabling factor for plant-level precision crop/weed management in arable fields.

Several robotic weed control platforms have been introduced offering active and passive interventions in field.
A variety of weeding implements have been investigated including physical~\cite{chang2021mechanical}, chemical~\cite{wu2020robotic}, electrocuting~\cite{ascard200710}, laser-based~\cite{xiong2017development}.
Given the variety of tools, it is clear that there is no one best solution and robotic solutions should be able to cater to a variety of tools.
The multi-modal approach of Bawden et al.~\cite{Bawden17_1} provided a clear step in this direction, however, a downside of their approach was the need to densely replicate each tool as they were mounted statically.
We propose to overcome this limitation by considering replicated movable tools.
%Each approach comes with pros and cons, for instance the multi-modal approach in~\cite{Bawden17_1,chang2021mechanical} has a minimum tool width of ~11cm, which is imprecise if we consider that standard crop-row intervals can be as small as $35cm$.}



\begin{comment}
\TODO{CM to ALL: I am thinking that the introduction should deal with the big picture (previous paragraph) and then state that we need systems to do multiple tasks and two of these are field monitoring as well as weed management.}
Due to these influences agricultural robotics has, in recent years, experienced a revolution.
This has been driven, in part, by cultural expectations and resource allocation with the expectation that novel technologies result in more sustainable solutions~\cite{zambon2019revolution}. 
% In recent years, the agricultural robotics experienced a revolution in expectations from the community, the allocated resources for automatizing, and the efforts for adapted new technologies all towards providing more sustainable solutions~\cite{zambon2019revolution}.
Innovations due to these expectations has resulted in less labor-intensive solutions that also impact the surrounding environment less.
The impact to labor costs is an important factor in platform uptake by farmers as this was indicated as a key expenditure by farming businesses~\cite{ABARES2018}.
Improved automated monitoring techniques also enable more robust decision management schemes, including intervention, while significantly reducing environmental impacts such as soil compaction.
In all, these novel approaches are able to improve farming operations by reducing agro-chemical use and directly impacting crop yield~\cite{steward2002distance}.
% Recent advances in this field offer not only offer less labor-intensive solutions and limit the soil compaction but also use perception-based field monitoring techniques that are able to considerably reduce the amount of used agro-chemicals and directly affect yield~\cite{steward2002distance}.

\TODO{CM to ALL: if we keep this paragraph it needs to address the following points (in no particular order): other methods exist, but they usually suffer from issues such as requiring the robot to stop or that they still cover a large area.}
Several robotic weed control platforms have been introduced offering active and passive interactions in field.
A variety of weeding implements have been investigated including physical~\cite{Bawden17_1,chang2021mechanical}, chemical~\cite{wu2020robotic}, electrocuting~\cite{ascard200710}, and more recently laser-based~\cite{xiong2017development} implements.
An essential capability of any robotic weeding platform is precise intervention that ensures the crop is not hit while simultaneously reducing the use of inputs (e.g. agro-chemicals).
% An essential capability in weed management is to increase precision of operations to plant level treatments that directly reduces amount of used agro-chemicals in field.
% \todo{MAH: No idea what's happening after this point} 
While current robotic precision weeding systems offer more precise solutions in comparison to classic methods of broad-cast weed management significantly larger space is still untouched. 
\end{comment}

% 
% Figure environment removed

In this paper we introduce \bbot, a robot that performs both field monitoring and precision weed management.
%The novelty of the weed management is through the replication of linear systems which enable precise intervention, up to ??0.05m??.
It enhances the capabilities of a recently published field monitoring technique~\cite{halstead2021crop} by using multiple localization sensors (GPS and odometry).
The plant counting performance is improved, reducing the normalized absolute error by more than half from $8.3\%$ down to $3.5\%$.
In developing \bbot\ we also propose a novel arrangement of weeding tools to enable precise weed management by replicating linear systems.
%We introduce a novel weeding tools design which consists of replicated linear systems.
An advantage of this approach is that we can deploy both fewer and smaller (more precise) tools, rather than densely replicated tools~\cite{Bawden17_1}, while still allowing for good coverage.
We demonstrate that it is feasible to have a system with just 4 replicated tools of size $0.05m$ on linear systems to cover a width of $1.3m$ which would normally require at least $26$ non-overlapping tools.
%This is less than half the size described in~\cite{Bawden17_1}. % 9 tools over 1m is 11.11cm, but they also overlap. The overlap margin was not given...
%The weed management system is built upon a field monitoring system and so we show that \bbot\ is capable of both field monitoring and weed management demonstrating the flexibility of \bbot.
%To achieve this, \bbot\ incorporates the use of extra sensors such as GPS to improve upon the field monitoring approach previously proposed in~\cite{halstead2021crop}.
%The field monitoring enables both crop and weed monitoring in conjunction with size estimation which demonstrates the flexibility of this platform.
This leads to the following novel contributions:
%In this paper we present our contribution as below:
\begin{enumerate}
    \item We introduce BonnBot-I a fully autonomous precision weeding platform fully compatible with ROS.
    \item Propose a new concept for weeding tools that enables flexible high-precision weed management.
    \item Improve crop monitoring performance by exploiting the extra sensors available with \bbot. %existing field monitoring systems by} using EKF (odometry and GPS) output.
    \item Release a new dataset consisting of corn as the crop, CN20. 
    This is a challenging dataset for crop monitoring approaches as it is a grass crop. %dataset from crop-corn, a specifically designed dataset from weeding and crop monitoring purposes.
    \item Introduce a framework for testing different weeding intervention strategies using a simulation environment.
\end{enumerate}


%In this paper we present the details of design, implementation and evaluations of our weed monitoring and intervention mobile robot platform \bbot\, capable of conducting inter and intra-row precision interventions in fields. 
%In this paper we aim at introducing our novel weeding implement design consisting of several repeatable, high speed high resolution linear axes and extensively evaluates it performance in challenging weeding scenarios.
%In this paper we present our contribution as below:
%\begin{enumerate}
%    \item \TODO{We introduce BonnBot-I a fully autonomous precision weeding platform fully compatible with ROS...}
%    \item \TODO{releasing corn dataset \textbf{CN20}.}
%    \item \TODO{improving monitoring system (MaskRCNN with EKF output).}
%    \item \TODO{introducing a new weeding tool concept suitable for high-precision weed management (including the feasibility check) in simulation and real weed distributions.}
%    \item \TODO{introducing a framework for testing different weeding intervention strategies + simulator and etc.}
%    \item \TODO{Demonstration of our platform performance for performing autonomous weeding in field (or on markers).}
%\end{enumerate}


% This paper is organised in the following manner: \secref{sec:relatedworks} reviews the prior work in subject of autonomous navigation in row-crop field and agricultural robotics. In \secref{sec:platform} we introduce, our weeding platform and its configuration towards autonomous navigation in field. Our proposed navigation approach is explained in \secref{sec:autoNav}. The \secref{sec:exp} describes our experimental evaluations and implementation details and finally the conclusions of work is drawn in \secref{sec:conc}.        # self.reset()